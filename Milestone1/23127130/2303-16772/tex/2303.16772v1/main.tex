\documentclass[autheryear, times, 9pt]{elsarticle}

\usepackage{geometry}
\setlength{\baselineskip}{25.0pt} \setlength{\textwidth}{16.0cm} \setlength{\textheight}{22.5cm} \setlength{\headheight}{0.2cm} \setlength{\headsep}{0.2cm}
\setlength{\oddsidemargin}{0.3cm} \setlength{\topmargin}{0.3cm} \setlength{\topskip}{0.5cm}

\usepackage{self}
\usepackage{amsthm}
\usepackage{booktabs}
\usepackage{times}
\usepackage{url}
\usepackage{float}
\usepackage{algorithm,algorithmic}
\usepackage{bm}
\usepackage{mathrsfs}

\usepackage{amsmath}
\usepackage{longtable}
\usepackage{amsfonts}
\usepackage{amssymb}

\usepackage{graphicx}
\usepackage{epstopdf}

\usepackage{multirow}
\usepackage{subcaption}
\usepackage{verbatim}
%\usepackage{lineno}
\usepackage{enumitem}
\usepackage{color}
\usepackage{subcaption}
\usepackage{epsfig}
\usepackage{apxproof}

%\usepackage{lineno}
%\modulolinenumbers[5]


\usepackage{hyperref}
\usepackage{xcolor}
\hypersetup{
    colorlinks,
    linkcolor={red!50!black},
    citecolor={blue!50!black},
    urlcolor={blue!80!black}
}

\newtheorem{corollary}{Corollary}
\newtheorem{lemma}{Lemma}
\newtheorem{proposition}{Proposition}
\newtheorem{definition}{Definition}
\newtheorem{observation}{Observation}
\newtheorem{example}{Example}
\newtheorem{theorem}{Theorem}
\newtheorem{remark}{Remark}
\newtheorem{condition}{Condition}


\newcommand{\xx}{{\mathsf x}}
\newcommand{\XX}{{\mathsf X}}

\pagenumbering{arabic}

\usepackage{natbib}

\hypersetup{
    colorlinks,
    linkcolor={red!50!black},
    citecolor={blue!50!black},
    urlcolor={blue!80!black}
}

\makeatletter
\newcommand{\rmnum}[1]{\romannumeral #1}
\newcommand{\Rmnum}[1]{\expandafter\@slowromancap\romannumeral #1@}
\newenvironment{breakablealgorithm}
  {% \begin{breakablealgorithm}
   \begin{center}
     \refstepcounter{algorithm}% New algorithm
     \hrule height.8pt depth0pt \kern2pt% \@fs@pre for \@fs@ruled
     \renewcommand{\caption}[2][\relax]{% Make a new \caption
       {\raggedright\textbf{\ALG@name~\thealgorithm} ##2\par}%
       \ifx\relax##1\relax % #1 is \relax
         \addcontentsline{loa}{algorithm}{\protect\numberline{\thealgorithm}##2}%
       \else % #1 is not \relax
         \addcontentsline{loa}{algorithm}{\protect\numberline{\thealgorithm}##1}%
       \fi
       \kern2pt\hrule\kern2pt
     }
  }{% \end{breakablealgorithm}
     \kern2pt\hrule\relax% \@fs@post for \@fs@ruled
   \end{center}
  }
\makeatother

\bibliographystyle{agsm}
\biboptions{authoryear}

\graphicspath{{img/}}
\journal{Transportation Research Part B}


\begin{document}

\begin{frontmatter}

\title{Maximin Headway Control of Automated Vehicles for System Optimal Dynamic Traffic Assignment in General Networks}
\author[mymainaddress]{Jinxiao Du}
\ead{jinxiao.du@connect.polyu.hk}
\author[mymainaddress]{Wei Ma\corref{mycorrespondingauthor}}
\cortext[mycorrespondingauthor]{Corresponding author}
\ead{wei.w.ma@polyu.edu.hk}	
\address[mymainaddress]{Department of Civil and Environmental Engineering\\The Hong Kong Polytechnic University, Hong Kong SAR}

\begin{abstract}
This study develops the headway control framework in a fully automated road network, as we believe headway of Automated Vehicles (AVs) is another influencing factor to traffic dynamics in addition to conventional vehicle behaviors ({\em e.g.} route and departure time choices).
Specifically, we aim to search for the optimal time headway between AVs on each link that achieves the network-wide system optimal dynamic traffic assignment (SO-DTA). 
% We belive headway of AVs is a critical factor in addition to its which has been rarely explored in the existing literature
To this end, the headway-dependent fundamental diagram (HFD) and headway-dependent double queue model (HDQ) are developed to model the effect of dynamic headway on roads, and a dynamic network model is built. It is rigorously proved that the minimum headway could always achieve SO-DTA, yet the optimal headway is non-unique. Motivated by these two findings, this study defines a novel concept of maximin headway, which is the largest headway that still achieves SO-DTA in the network. Mathematical properties regarding maximin headway are analyzed and  an efficient solution algorithm is developed. 
% The maximin automated vehicle headway control problem aims at finding the maximin headway control on each road segment to minimize the total travel time and guarantee safety in a fully connected and automated traffic network.
% A novel aspect of the proposed framework is that we explore how controlling automated vehicle headway in each link would affect traffic states and derive the headway-dependent fundamental diagram (HFD), where the fundamental diagram is dynamic by controlling automated vehicle headway. 
% We propose a headway-dependent double queue model (HDQ) to represent the influence of headway control to the downstream queue and upstream queue by HFD. 
% For system optimal dynamic traffic assignment (SO-DTA) concern, SO-DTA problem may have multiple solutions. 
% We prove that SO-DTA could be achieved under the minimum headway setting, which provides us a direct way to achieve SO-DTA and corresponding SO-DTA solutions by solving mixed integer linear programming under minimum headway setting. 
% For safety concern, we define the maximin headway control as the ``largest'' headway setting to achieve SO-DTA with the objective of guaranteeing both SO-DTA and safety. 
% A maximin headway control solving algorithm is proposed to solve the ``largest'' headway setting under SO-DTA that could be achieved by minimum headway setting. 
% Sensitivity analysis is conducted to explore how the travel demand and minimum headway value would respectively affect the gap between the maximin headway control and minimum headway setting. 
Numerical experiments on both a small and large network verify the effectiveness of the maximin headway control framework as well as the properties of maximin headway.
This study sheds light on deriving the desired solution among the non-unique solutions in SO-DTA and provides implications regarding the safety margin of AVs under SO-DTA.
\end{abstract}

		\begin{keyword}
			Automated Vehicles (AVs) \sep System Optimal Dynamic Traffic Assignment (SO-DTA) \sep Headway Control \sep  Maximin Optimization \sep Double Queue 
		\end{keyword}
  
\end{frontmatter}
%\linenumbers
%\modulolinenumbers[5]

% \clearpage

\section{Introduction}

With the rapid development of advanced sensors, wireless communication, and artificial intelligence, it is envisioned that automated vehicles (AVs) will play a significant role in future intelligent transportation systems. Ultimately, the fully automated environment would be realized to significantly mitigate congestion and ensure safety in general urban networks \citep{wadud2016help}. Therefore, it is essential for public agencies ({\em e.g.}, Transport Department) and private sectors ({\em e.g.}, Transportation Network Companies) to understand the network-wide effects of AVs in the fully automated environment and to develop effective management and control schemes for AVs. However, recent studies mainly focus on the microscopic behaviors of AVs, while studies on the spatio-temporal effects of AVs in networked traffic dynamics are still lacking \citep{YU2021103101}.

Among various traffic network models, system optimal dynamic traffic assignment (SO-DTA) quantifies the best possible performance of a traffic network given the dynamic network demand and supply. 
%P2 SO-DTA is one of the most important problems in dynamic traffic assignment (DTA) with the goal to quantify the best possible performance of a traffic network. 
Both the optimal objective function ({\em i.e.}, the total travel time) and the optimal solution ({\em i.e.}, network dynamics under SO-DTA) could benchmark different traffic operation and management strategies and provide policy implications for decision-makers.
%important for traffic management and traffic management department is aware of the traffic assignment pattern to reduce congestion in whole traffic network and save total travel time although it is hard to assign the vehicles as this pattern in a human-driven vehicle (HV) traffic. 
Additionally, it is practical to control all the AVs to achieve SO-DTA under fully automated environments \citep{nguyen2021system}.
% Therefore, we aim to explicitly model the roles of AVs in SO-DTA and analytically derive the SO-DTA solution on general networks.

In this paper, we consider the headway control of AVs, in addition to the travelers' behaviors ({\em e.g.}, route and departure time choices). That is, AVs can decide the headway between each other according to the mobility and safety requirements of each road segment \citep{hatipoglu1996longitudinal}. In particular, headway refers to the time headway throughout the paper.
%P3 As is introduced in , headway control is an important control policy of AV. 
%Most of studies related to AV control is at microscopic level and there is few work focusing on AV control on general networks. 
Existing studies have shown that different headway settings of AVs could affect the fundamental diagrams of each road \citep{ZHOU2020102614, GONG2016314}, hence affecting the solutions of SO-DTA. However, most current studies are at a microscopic level, while how the AV headway control could affect the network-wide SO-DTA is still an open question. 
%with the development and wide application of AVs, the AVs headway control to achieve SO-DTA in general networks shows great potential in future intelligent transportation system and it gives traffic management department an insight to control AV headway and assign traffic to save total travel time. 
%However, network-wide effects of headway...

Intuitively, smaller AV headway yields higher road capacities and throughput, leading to shorter travel time, and hence SO-DTA can be achieved with the smallest AV headway. To verify this intuition, later this paper {\em analytically} proves that Observation~\ref{ob:1} holds in general networks. 


%To achieve SO-DTA, we analytically prove that if we 
%in this paper
%We first prove that SO-DTA could be achieved under minimum headway setting and it provides us a way to derive the SO-DTA first.
%P4 SO can be analytically proved
\begin{observation}
\label{ob:1}
Given a range of the AV headway, SO-DTA is always achieved when the AV headway is set as the smallest value in the range.
\end{observation}

Note that the range of AV headway depends on the minimum safety requirements of AVs, specifications of AVs, and the maximum capacity of the specific road \citep{li2017vehicle}.
Furthermore, similar to the existing SO-DTA models \citep{SHEN20141}, we have Observation~\ref{ob:nonunique} holds, indicating that SO-DTA may be achieved under different headway settings.

\begin{observation}
\label{ob:nonunique}
There can be multiple AV headway settings yielding the same SO-DTA solution in general networks.
\end{observation}

%Observation~\ref{ob:nonunique} inspires 
%In particular, 

%If we suppose the time lag gap that should be related with AVs manufacture is fixed, a smaller headway would correspond to a safer traffic condition. 
%And \cite{SHI2021103134} conducts experiments on car-following characteristics of commercial AVs with different headway settings and empirical result validates theoretical conclusions in \cite{tradeoff}. 

%Therefore, exploring the optimal AV headway control to guarantee both SO-DTA and safety has a great significance in future traffic management.  


%P6 Safety


%For our problem to explore maximin headway control in fully AVs environment, it is necessary to propose a proper AVs model to incorporate the features of our objective.  




%Besides, \cite{tradeoff} analytically shows that there exists a trade-off between the time lag gap and the safety buffer  that collectively constitutes the headway, where time lag gap represents mechanical response delay and safety buffer measures the safety and is to absorb AV overshoot in the worst case, such as a sharp full stop. 

Observation~\ref{ob:nonunique} inspires us to search for the second criterion in the SO-DTA problem. For AVs, safety is always an indispensable factor.  
%for the solution of SO-DTA problem, \cite{SHEN20141} shows the SO-DTA problem may have multiple solutions, which means SO-DTA may be achieved under different headway settings in our problem. For our problem, safety is also an important consideration. Therefore, it is necessary to explore the relationship between safety and headway first. 
For example, \cite{tradeoff} proved that there exists a trade-off between the time lag gap and the safety buffer that collectively constitutes the headway. Suppose the time lag gap is fixed because it is based on the vehicle specifications and mechanical properties, AVs could be safer when the headway becomes larger. Under the same traffic conditions, we expect to have larger headway between AVs.

%And \cite{SHI2021103134} conducts experiments on car-following characteristics of commercial AVs with different headway settings and empirical result validates theoretical conclusions in \cite{tradeoff}.

In view of this, this paper proposes a novel concept of the maximin headway of AVs for SO-DTA, as defined in Definition~\ref{def:maximin}.

\begin{definition}[Maximin Headway - Layman's version]
\label{def:maximin}
The largest headway setting for AVs that still achieves the SO-DTA is referred to as the maximin headway of AVs. 
\end{definition}
One can see that the maximin headway is the ``largest'' headway setting that achieves SO-DTA, meaning that such a headway setting is the ``safest'' under the SO-DTA. An illustration of the maximin headway is also presented in Figure~\ref{fig:maximin}. Existing literature mainly focuses on the solutions to SO-DTA \citep{zhang2020path, qian2012system}, while this paper discusses the ``best'' solution among the non-unique solutions of SO-DTA for AVs, which we believe is interesting and unique.  In particular, differences between minimum headway and maximin headway provide policy implications for road segment design and AV deployment.



\begin{figure}[h]
    \centering
    \includegraphics[width=0.7\linewidth]{minimax}
    \caption{Illustration of the maximin headway for SO-DTA.}
    \label{fig:maximin}
\end{figure}





%Therefore, we define the maximin headway control as the ``largest'' headway setting to achieve SO-DTA, which represents the headway setting makes the whole traffic network ``safest'' under the SO-DTA. \cite{COMO2016446} studies the design of optimal traffic flow controls for freeway networks in SO-DTA problem and discusses convexity and robustness of it. But we construct the maximin headway control in a different way in this problem. 


%The existence condition of maximin headway control is discussed and it inspires us a way to derive the maximin headway control under current headway setting that achieves SO-DTA. 

%Based on above two theoretical propositions, we propose an maximin headway control solving  algorithm to derive the maximin headway control. In addition to derive the maximin headway control, this algorithm also provides us an efficient and direct way to achieve SO-DTA and derive the actual minimum total travel time (TTT). We just need to solve a mixed integer linear programming under minimum headway setting rather than solve an nonlinear programming that is hard to find an effective way to solve. In order to demonstrate it more intuitively, we compare it with a sensitivity analysis-based algorithm to derive minimum TTT and find that minimum TTT solved by the first method is smaller. Then we conduct maximin headway control solving algorithm on a small network and Sioux Falls network. Finally, we conduct the sensitivity analysis of the gap between the maximin headway control and minimum headway setting and find a general but not strict relationship that the gap value would decrease when travel demand or minimum headway value increases.

% Figure~\ref{fig:maximin} illustrates 
% The maximin headway 




To summarize, this paper proposes a maximin headway control framework that achieves SO-DTA. Specifically,
% to minimize the total travel time and guarantee safety. 
% To address this problem, 
we first explore how the headway of AVs would affect the road fundamental diagram and develop a headway-dependent double queue model (HDQ), which is proved to be a generalized form of the double queue model \citep{MA201498}.
% to incorporate the effects of headway control on both downstream queue and upstream queue by HFD and dividing a link into flow area and buffer area. 
% And we show that HDQ is a generalized form of DQ. If we drop some additional constraints of HDQ compared with DQ, HDQ would be reduced to DQ.
Then we formulate the network-wide traffic dynamics with SO-DTA for AVs, followed by the rigorous proof that Observation~\ref{ob:1} and Observation \ref{ob:nonunique} hold. Then we formally define the maximin headway control for SO-DTA, propose a provable approach to solve it analytically, and show the uniqueness of the maximin headway. Lastly, numerical experiments are conducted on the small and large networks to verify our proofs.

% We first explore how AV headway control would affect the traffic states in a fully connected and automated traffic network and derive HFD. Then we propose the HDQ model to incorporate the influence of headway control to both downstream queue and upstream queue. We show that HDQ is a generalized form of DQ and could be reduced to DQ if we drop some additional constraints of HDQ. With the objective of guaranteeing both SO-DTA and safety, we define the maximin headway control. Then we prove that the SO-DTA could be achieved under minimum headway setting and discuss the existence condition of maximin headway control. 
% An maximin headway control solving algorithm is proposed to solve the maximin headway control and we conduct the sensitivity analysis to explore how the travel demand and minimum headway constraint would respectively affect the gap between the maximin headway control and minimum headway setting. 

The contributions of this paper are summarized as follows:
\begin{itemize}
    \item The headway-dependent fundamental diagram (HFD) and double queue model (HDQ) are developed to depict the effects of AV headway on network-wide traffic dynamics.
    \item It is analytically proved that the SO-DTA can be achieved under minimum headway.
    \item We first time propose the maximin headway control of AVs for SO-DTA in general dynamic networks; the maximin headway is proved to be unique, and it can be obtained analytically using Linear Programming (LP).
    
    % Headway-dependent fundamental diagram (HFD). We explore that headway could affect the fundamental diagram of pure AV in the following aspects: scope of free-flow region, upper bound of flow and shockwave travel speed.
    % \item Headway-dependent double queue model (HDQ). We propose a headway-dependent double queue model to  represent the influence of headway control to the downstream queue and upstream queue by HFD. Besides, we HDQ is a generalized form of double queue model (DQ) and could be reduced to DQ under the assumption of DQ if we consider headway to be fixed rather than dynamic.
    % \item A simple and direct way to derive minimum total travel time and values of solutions under SO-DTA.
    % We prove that SO-DTA could be achieved under minimum headway setting, which converts an nonlinear programming difficult to solve SO-DTA into a mixed integer linear programming.
    % \item Define and discuss the similarities and differences between optimal headway control and maximin headway control. 
    % \begin{itemize}
    %     \item optimal headway control. optimal headway control is the headway setting where SO-DTA could be achieved. There could be multiple system optimal controls to achieve SO-DTA because SO-DTA could have multiple solutions. And it is hard to derive all optimal headway controls, but we prove that minimum headway setting is always a system headway control. optimal headway control only guarantee SO-DTA. We still need to find the headway setting among optimal headway controls to make traffic network ``safest''.
    %     \item maximin headway control. maximin headway control is defined as the ``largest'' headway setting under SO-DTA for the objective of guaranteeing both SO-DTA and safety. We propose an maximin headway control solving algorithm by solving ``largest'' headway setting based on SO-DTA that is achieved under minimum headway setting. 
    % \end{itemize}
\end{itemize}


The remainder of this paper is organized as follows: related literature is reviewed in Section~\ref{sec:lit}. Section \ref{sec:model} presents headway-dependent flow dynamics using HFD and HDQ, Section \ref{sec_system_optimal_headway_formulation} develops the formulation of optimal headway control, Section \ref{sec_Optimal_maximin_Headway_Formulation} proposes the formulation of maximin headway control and the corresponding solution algorithm, and Section \ref{sec:Numerical_Results} shows the numerical results in the small and large networks. Lastly, conclusions are drawn in section \ref{sec:Conclusion}.


\section{Literature review}
\label{sec:lit}
In this section, we summarize the literature regarding the traffic network models for AVs. In particular, we compare with existing studies on dynamic link models, network models, and SO-DTA, to further highlight the contributions of this paper.

\subsection{Dynamic link models}

Dynamic link models are an essential component to model the flow dynamics on road segment, and 
%Dynamic network loading (DNL) model is an important part of DTA problem. For our problem that focuses on the maximin headway control for SO-DTA and safety on general networks, 
it is necessary to develop an appropriate link model to depict the effects of AV headway control. The cell transmission model (CTM) is a classical link model and could capture congestion spillbacks \citep{DAGANZO199579}. However, CTM requires the decomposition of link into cells cells both in space and time, making the dimension of a DTA problem significantly increased.
%is much larger than the DTA problems with other DNL models. 
The link transmission model (LTM) is derived from the kinematic wave theory but does not require the decomposition of link \citep{LTM}. LTM considers the constraints on the sending flow and receiving flow, which are limited in the free-flow state and congested state respectively. Directly derived from LTM, \citet{osorio2011dynamic} and \citet{MA201498} propose the double queue model (DQ) to define the specific formulation of the downstream queue and upstream queue in a link for single-destination traffic networks. DQ is based on a forward free-flow travel time delay in downstream queue and a backward shockwave travel time delay in upstream queue. %But we are unclear about the specific traffic states and how the flow propagates along the link, so DQ is not applicable in our problem where the effects of AV headway control on general networks are required. 
Besides, \cite{NGODUY202156} proposes a multi-class two regime transmission model (MTTM) model to capture the evolution of the queue lengths, where a link is divided into two areas: free-flow area and congested area. Overall, none of the above mentioned models consider the effects of different AV headway on the road segment, and hence we call for a headway-dependent dynamic link model to depict the flow dynamics of AVs with different headway.


\subsection{DTA with AVs}

Dynamic traffic assignment (DTA) models are designed to explore the time-dependent traffic states pattern given a time-varying travel demand on general traffic networks \citep{szeto2006dynamic}. There are two representative types of DTA: dynamic user equilibrium (DUE) and system optimal dynamic traffic assignment (SO-DTA). SO-DTA quantifies the best possible performance of a traffic network based on the dynamic extension of Wardrop’s second principle \citep{wardrop1952road}. For its advantage of exploring the spatial-temporal traffic dynamics to achieve the best performance of traffic networks, the research of the application of SO-DTA problem focuses on many different areas, such as emission reduction \citep{lu2016eco,ma2017emission,long2018link,tan2021emission}, congestion mitigation \citep{levin2017congestion,samaranayake2018discrete,liu2020integrated}, parking services \citep {levin2019linear,qian2014optimal}, traffic signal control \citep{lo2001cell,han2016robust}, emergency evacuation \citep{liu2006cell,chiu2007modeling} and network design \citep{waller2001stochastic,waller2006linear}. 

% Compared with DUE, SO-DTA is less attainable for its high requirement for travelers' cooperation and currently few direct control methods of human-driven vehicles to save social cost \citep{chakraborty2020dynamic}. Therefore, the traffic management strategies are widely developed to encourage the realization of SO-DTA, such as congestion pricing \citep{yang1998departure,carey2012dynamic}, travel incentives \citep{chakraborty2020dynamic} and network access control \citep{zhang2010access}.

%For the dynamic traffic assignment (DTA) problem, \citet{LEVIN2016103} propose a multi-class cell transmission model and proves this model is consistent with the hydrodynamic theory. \citet{NGODUY202156} explores the impact of in traffic network by multi-class SO-DTA approach and proposes a multi-class two regime transmission model (MTTM) to simulate the multi-class traffic dynamics, where a link is divided into two areas: free-flow and congested area. \cite{9408374} proposes a unified optimization strategy framework for shared AVs to minimize both the system-side and travelers-side costs. Besides, there are also many other works in other AV modeling topics rather than traffic assignment problem. \cite{GONG201825} develops a cooperative platoon control in mixed traffic to guarantee both the stability in system level and mobility in traveler level. \cite{GUO2019313} summarizes the traffic signal control policies in the presence of AVs. \cite{doi:10.1287/trsc.2021.1061} proposes a rhythmic control method for AVs in a general network to make them move in network as rhythm and avoid collisions at all intersections. \cite{GHIASI2019210} proposes an AV trajectory control method for the stability of mixed traffic. \cite{YH} represents a cooperative driving strategy of AVs in a mixed traffic at intersections. \cite{YU2021103101} summarizes that the research on AV modeling at a macroscopic level mainly focuses on these aspects: traffic capacity, stability and efficiency. 

Different from human-driven vehicles (HVs), AVs show great potential in improving control efficiency and may play an important role in achieving SO-DTA \citep{wang2018dynamic}. 
The current research on traffic assignment with AVs is summarized in Table \ref{literature}. For the static traffic assignment problem of AVs, \cite{CHEN2016143} proposes a mathematical model to derive an optimal AV lane control with mixed traffic, where AV lines are only permitted for AVs and it is necessary to specify the deployment plan. \citet{Bagloee} focuses on a static traffic assignment problem on a mixed traffic where AVs are seeking for system optimal (SO) and HVs are seeking for user equilibrium (UE), and considers the influence of many realistic features in the solution, such as road capacity and elastic demand. \cite{CHEN201744} presents an optimal AVs zones design framework to improve the performance of the whole traffic network, where the government may make a plan of AVs zones only allowing AVs to enter to promote the adoption of AVs in the future. \cite{ZHANG201875} proposes a route control scheme to balance the system efficiency and the control intensity by controlling a relatively small portion of all vehicles. \cite{WANG2019139} addresses the multi-class traffic assignment problem by assuming AVs follow SO and HVs follow cross-nested logit (CNL) stochastic user equilibrium (SUE) to capture HV drivers' uncertainty due to their limited knowledge of traffic conditions in the whole network. \cite{WANG2020227} develops a mixed behavioral equilibrium model in a mixed traffic considering different mode choices for HVs and AVs.% where travelers have three choices: choose their own HVs, take AV mobility service by a firm or government. 



\begin{table}[H]
    \centering
    \resizebox{1.02\textwidth}{!}{
    \begin{tabular}{lllll}
        \toprule
        Reference      & Type  &  HV & AV & Description \\
        \midrule
        \cite{LEVIN2016103} &  Dynamic  & DUE & DUE 
        & A multi-class CTM model for mixed traffic \\
        \cite{CHEN2016143}  & Static & UE & UE   
        & An optimal AV lanes control framework for mixed traffic\\
        \cite{Bagloee} & Static & UE & SO 
        & The mixed traffic network with elastic demand \\
        \cite{CHEN201744} & Static & UE & SO 
        & Optimal design of AV zones for mixed traffic \\
        \cite{ZHANG201875} & Static & UE & SO 
        & A route control scheme to balance efficiency and control \\
        \cite{WANG2019139} & Static & CNL & UE 
        & A multi-class traffic assignment model for mixed traffic \\
        \cite{WANG2020227} & Static & SUE  & SO
        & A mixed equilibrium model with mode choice \\
        \cite{NGODUY202156} & Dynamic & SO-DTA & SO-DTA 
        & Impact of AVs in traffic network by multi-class SO-DTA \\
        \cite{9408374} & Dynamic & \textbackslash & SO-DTA
        & A unified optimization strategy framework for shared AVs \\
        {\bf This paper} & Dynamic & \textbackslash & SO-DTA 
        & Maximin AV headway control for SO-DTA on general networks \\
        \bottomrule
    \end{tabular}
    }
    \caption{Literature summary of traffic assignment with AVs.}
    \label{literature}
\end{table}



In summary, the work related to the traffic assignment problem of AVs is mainly static and there is few work exploring how AV headway control would affect the traffic states in whole general networks as a SO-DTA problem. 
There is few work discussing the roles of headway in the control of  AVs. 
Therefore, this paper is designed to propose an optimal AV headway control framework for SO-DTA in general networks. 

%TODO: \cite{COMO2016446} studies the design of optimal traffic flow controls for freeway networks in SO-DTA problem and discusses convexity and robustness of it. But we construct the maximin headway control in a different way in this problem. 


\subsection{Uniqueness of SO-DTA}
%any paper that considers the non-unique of the solution?

The solution of SO-DTA provides valuable insights for decision-makers to develop traffic management strategies \citep{van2016user}. Most of the related works focus on the solution algorithm of SO-DTA \citep{chow2009properties,zhang2020path,qian2012system,long2019link}, and there is a few work discussing the non-uniqueness and properties of the solutions of SO-DTA. For example, \citet{SHEN20141} proves that SO-DTA may have multiple solutions with same objective but different queue locations in networks. To explore the optimal queue distribution for the solution of SO-DTA, \citet{MA201498} proposes the solution algorithm and discusses the existence of the free-flow optimal solution of SO-DTA that all vehicles are only allowed to wait in the origin nodes, and \citet{ngoduy2016optimal} optimizes the heterogeneity of the queue lengths and the total queue lengths in traffic networks for all solutions of SO-DTA. Though both of above work discuss the non-uniqueness of the solutions of SO-DTA and propose the criteria to ``pick up'' the desired solution of SO-DTA, the criteria of both works are mainly related to congestion patterns. In reality, decision-makers could develop traffic management strategies considering different perspectives.
% , for example, congestion could be minimized by both network access control and signal control, and network access control would be preferred with the consideration of energy saving and emission reduction. 
Therefore, this paper discusses the ``safest'' solution among the non-unique solutions of SO-DTA and proposes the maximin headway in this problem as the largest headway solution in SO-DTA.



\section{Modeling headway-dependent flow dynamics}
\label{sec:model}

In this section, we first propose the headway-dependent fundamental diagram (HFD) and develop the headway-dependent double queue (HDQ) model, which is proved to be a generalized form of DQ, and then the network-wide flow dynamics are modeled through a series of linear constraints. 
%we explore the effects of AV headway control to the whole traffic network and model headway-dependent flow dynamics. we propose the HFD and HDQ, then we prove that HDQ is a generalized form of DQ. Besides, other DTA related constraints are proposed in this section.

\subsection{Notations}
The road network is represented as a directed graph \textit{G}(\textit{V},\textit{E}), where \textit{V} is the set of all nodes including dummy nodes and \textit{E} is the set of all links including O-D connectors. We denote $\widetilde{R}$ as the set of dummy origins and $\widetilde{S}$ as the set of dummy destinations. Naturally, we have $\widetilde{R}\subset\textit{V}$ and $\widetilde{S}\subset\textit{V}$. The frequently used notations are stated in Table \ref{notation_con} in \ref{app:notations}.


\subsection{Headway-Dependent Fundamental Diagram}
\label{sec:HFD}

This section explores how different headway would affect the fundamental diagram.
%and derive the headway-dependent fundamental diagram for AVs. 
Without loss of generality, it is assumed that the headway and length of AV are homogeneous. 
% Before this, the assumptions regarding AV headway control are stated as following:
% \begin{itemize}
% \item All AVs in a link have same headway.
% \item All AVs have same length \textit{L}.
% \end{itemize}
Then we present the fundamental diagram for AVs based on the results by \citet{ZHOU2020102614}.%has theoretically derived the fundamental diagram of pure AVs. And we will briefly show this process. 
Considering link $(i,j)$, we denote $\Delta_{x_{i,j}}$ as the spacing of AVs in link $(i,j)$, $\textit{v}_{i,j}$ as the speed of AVs on link $(i,j)$, $\textit{h}_{i,j}$ as the headway of AVs in link $(i,j)$ and \textit{L} as the length of each AV. The relationship between the headway $\textit{h}_{i,j}$ and spacing $\Delta_{x_{i,j}}$ can be described as follows:

\begin{equation}
\Delta_{x_{i,j}}=\textit{h}_{i,j}\textit{v}_{i,j}+\textit{L}.
\nonumber
\end{equation}

Besides, the relationship between the spacing $\Delta_{x_{i,j}}$ and density $\rho_{i,j}$ is shown as follows:
\begin{equation}
\Delta_{x_{i,j}}=\frac{1}{\rho_{i,j}}.
\nonumber
\end{equation}

% The relationship among speed $\textit{v}_{i,j}$, density $\rho_{i,j}$ and flow $\textit{f}_{i,j}$ is shown as following:

% \begin{equation}
% \textit{v}_{i,j}=\frac{\textit{f}_{i,j}}{\rho_{i,j}} \nonumber
% \end{equation}

The speed $\textit{v}_{i,j}$ should be limited within free-flow speed $\textit{v}_{i,j}^f$ as follows:

\begin{equation}
0\leq\textit{v}_{i,j}\leq\textit{v}_{i,j}^f. \nonumber
\end{equation}

Combining the above equations, we have Equation \ref{eq:FD} to represent the fundamental diagram of AVs denoted as $\textit{f}^{~FD}_{i,j}(t)$ for $(i,j)\in E\setminus(L_R\cup L_S)$ and $t\in[0,T]$, where $L_R$ and $L_S$  represent the set of origin and destination connectors, respectively.

\begin{equation}
\textit{f}^{~FD}_{i,j}(t)=\left\{
\begin{array}{lcl}
\textit{v}_{i,j}^{f}{~} \rho_{i,j}(t),  & &   {0\leq\rho_{i,j}(t)<\frac{1}{\textit{h}_{i,j}(t)\textit{v}_{i,j}^f+\emph{L}}} \\
\frac{1-\rho_{i,j}(t)\emph{L}}{\textit{h}_{i,j}(t)}, &   &{\frac{1}{\textit{h}_{i,j}(t)\textit{v}_{i,j}^f+\emph{L}}   \leq \rho_{i,j}(t)\leq\frac{1}{\emph{L}}}
\end{array} \right.
\label{eq:FD}
\end{equation}
where $\rho_{i,j}(t)$ and $\textit{h}_{i,j}(t)$ are the density of the flow area and headway of link $(i,j)$ at time $t$, respectively. Figure \ref{fig:FD} shows the fundamental diagram of AVs with different headway based on Equation~\ref{eq:FD}. This figure is also validated by the experimental results from \citet{SHI2021279}, which empirically validates the fundamental diagram by collecting GPS trajectories of a platoon of AVs.

\begin{figure}[h]
    \centering
    \includegraphics[width=\linewidth]{FD}
    \caption{Fundamental diagram of AVs with different headway.}
    \label{fig:FD}
\end{figure}

As shown in Figure \ref{fig:FD}, headway could affect the fundamental diagram of AVs in the following aspects:

\begin{itemize}
    \item Scope of free-flow region: a smaller headway corresponds to a larger scope of free-flow region.
    \item Upper bound of flow: a smaller headway corresponds to a larger upper bound of flow.
    \item Shockwave travel speed: a smaller headway corresponds to a higher shockwave travel speed.
\end{itemize}

Besides, we suppose the first AV entering an empty link would drive at a  free-flow travel speed for link$(i,j)\in E\setminus(L_R\cup L_S)$ and $t\in[0,T]$.  Then the headway-dependent fundamental diagram (HFD) is presented in Equation \ref{eq:f_con}:

\begin{equation}
\textit{f}_{i,j}(t)=\left\{
\begin{array}{lcl}
0,  &  & \text{if}~\rho_{i,j}(t-\frac{L_{i,j}}{\textit{v}_{i,j}^f})=0 \\
\textit{f}^{~FD}_{i,j}(t), &  & \text{otherwise}
\end{array} \right.
\label{eq:f_con}
\end{equation}
% So we call the equation \ref{eq:f_con} as , where the fundamental diagram is dynamic if we could control the headway at any time. 
where $\textit{f}_{i,j}(t)$ is the inflow at the boundary of the flow area and buffer area of link $(i,j)$ at time $t$. The conservation law of traffic flow is shown as follows:
\begin{equation}
\rho_{t}+\emph{f}_{x}=0. \label{16} \nonumber %TODO
\end{equation}

Assuming the density is homogeneous in a link,  the change rate of density is presented in Equation \ref{eq:density_con} for $(i,j)\in E\setminus(L_R\cup L_S),~ s^{\prime}\in \widetilde{S}$ and  $t\in[0,T]$:

\begin{equation}
\dot{\rho}_{i,j}^{s^{\prime}}(t)=\frac{\textit{u}_{i,j}^{s^{\prime}}(t)-\textit{f}_{i,j}^{~s^{\prime}}(t)}{L_{i,j}},
\label{eq:density_con}
\end{equation}
where $\rho_{i,j}^{s^{\prime}}(t)$, $\textit{u}_{i,j}^{s^{\prime}}(t)$ and $\textit{f}_{i,j}^{~s^{\prime}}(t)$ are the density of the flow area, inflow of the flow area and inflow at the boundary of the flow area and buffer area with destination $s^{\prime}$ of link $(i,j)$ at time $t$; $L_{i,j}$ is the length of link $(i,j)$. Therefore, the dynamic link density $\rho_{i,j}(t)$ for $(i,j)\in E\setminus(L_R\cup L_S),~ s^{\prime}\in \widetilde{S}$ and  $t\in[0,T]$ is represented as follows:

\begin{equation}
\dot{\rho}_{i,j}(t)=\frac{\textit{u}_{i,j}(t)-\textit{f}_{i,j}(t)}{L_{i,j}}
\label{eq:density_total_con},
\end{equation}
where $\textit{u}_{i,j}(t)$ is the inflow of the flow area of link $(i,j)$ at time $t$.

Besides, $\tau_{i,j}^{w}(t)$ represents the shockwave travel time on link $(i,j)$ at time $\textit{t}$. If the fundamental diagram is fixed, we have a constant shockwave travel time $\tau_{i,j}^{w}=\frac{L_{i,j}}{\textit{v}_{i,j}^w}$, where $\textit{v}_{i,j}^w$ is the shockwave speed on link $(i,j)$. However, in HFD, the shockwave speed is also dynamic given different headway. Equation \ref{eq:shockwave_con} indicates that the shockwave travel time should satisfy Equation~\ref{eq:shockwave_con} for $(i,j)\in E\setminus(L_R\cup L_S),~ s^{\prime}\in \widetilde{S}$ and $t\in[0,T]$. 

\begin{equation}
\int_{\textit{t}-\tau_{i,j}^{w}(t)}^{\textit{t}}\frac{\textit{L}}{\textit{h}_{i,j}(t)} dt =L_{i,j}
\label{eq:shockwave_con}
\end{equation}

\subsection{Headway-dependent double queue model}
\label{sec:HDQ}

This section proposes a dynamic link model considering time-dependent AV headway, namely the headway-dependent double queue (HDQ) model. We further prove HDQ is a generalized form of the double queue (DQ) model.

%The link transmission model (LTM) is a classical DNL model derived from the kinematic wave theory to solve the DTA problem in \cite{LTM}. Eqs.(4.29) and (4.33) in \cite{LTM} show the constraints on the sending flow and receiving flow, which are limited in the free-flow state and congested state respectively. Besides, \cite{OSORIO20111410} proposed a stochastic differential DNL model, where a link is treated as consisting of two queues: downstream queue and upstream queue. 
%Therefore, \cite{MA201498} proposed the double queue model (DQ) to define the specific formulation of the downstream queue and upstream queue in a link and and it is proved that the double queue model could be derived directly from LTM. 


% The cumulative outflow of link $(i,j)$ at time $\textit{t}$ for $ (i,j)\in E$ and $t\in[0,T]$:

% \begin{equation}

% \nonumber
% \end{equation}
% We use  and $\textit{q}_{i,j}^{\mathcal{U}}(t)$ to represent the downstream and upstream queue of link $(i,j)$ at time $t$ in DQ, respectively.  




% DQ defines the formulation of downstream queue and upstream queue on each link, while flow between two queues are not specified. 
%the traffic states in each link and how the flow propagates along the link.
% In contrast, \cite{NGODUY202156} proposes a multi-class two regime transmission model (MTTM),  which divides each link into two areas: free-flow area and congested area. 
%Different with DQ, MTTM could capture the evolution of the queue lengths. 


% And it only considered the downstream queue. 
% But MTTM assumes the traffic queue could only happen at the downstream bottleneck and does not consider the upstream queue. 
% Similar with DQ model, we only know about the inflow and outflow in a link and also unclear about how the flow propagates along the link. The effect of headway control could not also be reflected in MTTM model.  


In HDQ, each link is divided into two areas: flow area and buffer area. The flow area depicts the traffic flow dynamics behavior and captures the upstream queue in the link, and we suppose the traffic flow would propagate based on HFD. The buffer area captures the downstream queue in the link and the maximum number of queued vehicle  is limited based on the holding capacity in the buffer area \citep{GUO202087}. An illustration of the HDQ is shown in Figure~\ref{fig:link}.  
One can see that the HDQ is inspired by the MTTM and DQ model, and it shares similarties with the Spatial Queue model \citep{gawron1998iterative}.
% In this paper, the buffer area acts like a Spatial Queue model (SQ) and captures the upstream queue in a link with a queue capacity. 

\begin{figure}[h]
    \centering
    \includegraphics[width=0.7\linewidth]{doublequeue}
    \caption{Flow area and buffer area in a link.}
    \label{fig:link}
\end{figure}


The HDQ has following characteristics that are different from the DQ and MTTM: 
\begin{itemize}
\item The MTTM overlooks the possible congestion in the flow area, while the HDQ consider both upstream and downstream queues.
% A link is divided into two areas: flow area and buffer area. Flow area is to model the upstream queue and buffer area is to model the downstream queue. Different with the free-flow area in MTTM, there could also be congested in flow area in HDQ and we consider both the upstream and downstream queue in HDQ rather than only considering downstream queue in MTTM.
% \item Rather than only knowing inflow and outflow in MTTM and DQ, we add a third variable $f$ and assume it propagates in flow area following HFD to represent the effect of headway control to the traffic states and whole network. $f$ is not only the outflow of the flow area, but also the inflow of the buffer area.
\item Comparing to the DQ, the HDQ makes use of the variable $f$ to represent the effect of headway control on the traffic states and network conditions.
\end{itemize}


The cumulative flow for $ (i,j)\in E$ and $t\in[0,T]$ is defined as follows:
\begin{equation}
\textit{U}_{i,j}(t) = \int_{0}^{\textit{t}}\textit{u}_{i,j}(t)~dt, \quad \textit{V}_{i,j}(t) =\int_{0}^{\textit{t}}\textit{v}_{i,j}(t)~dt, \quad
\textit{F}_{i,j}(t) ~= ~\int_{0}^{\textit{t}}\textit{f}_{i,j}(t)~dt,
\nonumber
\end{equation}
where $\textit{U}_{i,j}(t)$, $\textit{V}_{i,j}(t)$ and $\textit{F}_{i,j}(t)$ are the cumulative inflow of the flow area, cumulative outflow of the buffer area and cumulative inflow at the boundary of the flow area and  buffer area of link $(i,j)$ at time $t$, respectively. 

Then the downstream queue and upstream queue in HDQ for $ (i,j)\in E$ and $t\in[0,T]$ are presented as follows:

\begin{eqnarray}
\textit{q}_{i,j}^{\mathcal{D}}(t) &=& \int_{0}^{\textit{t}} \textit{f}_{i,j}(t) ~dt - \int_{0}^{\textit{t}} \textit{v}_{i,j}(t) ~dt + \textit{q}_{i,j}^{\mathcal{D}}(0) ~=~
\textit{F}_{i,j}(t)-\textit{V}_{i,j}(t),
\label{eq: down_HDQ} \\
\textit{q}_{i,j}^{\mathcal{U}}(t) &=& \int_{0}^{\textit{t}} \textit{u}_{i,j}(t) ~dt - \int_{0}^{\textit{t}-\tau_{i,j}^{w}} \textit{f}_{i,j}(t) ~dt + \textit{q}_{i,j}^{\mathcal{U}}(0) ~=~
\textit{U}_{i,j}(t)-\textit{F}_{i,j}(t-\tau_{i,j}^{w}(t)),
\label{eq: up_HDQ}
\end{eqnarray}
where $\textit{q}_{i,j}^{\mathcal{D}}(t)$ and $\textit{q}_{i,j}^{\mathcal{U}}(t)$ represent the downstream and upstream queue of link $(i,j)$ at time $t$ in HDQ, respectively.
% where $\textit{q}_{i,j}^{\mathcal{D}}(t)$ and $\textit{q}_{i,j}^{\mathcal{U}}(t)$ represent the cumulative number of vehicles entering and exiting link $(i,j)$ at time $t$ in HDQ, repsetively.
%, and Equation \ref{eq: up_HDQ} measures the possible number of vehicles at the entrance of link $(i,j)$ at time $t$ by HDQ if we already know the number of vehicle have left the flow area of link $(i,j)$ by time $t$. 
% Then we will show that the HDQ is a generalized form of DQ. 

Proposition \ref{prop:HDQ&DQ} presents the relationship between the DQ and HDQ, indicating that HDQ could be reduced to DQ under its assumption.

\begin{proposition}[Reduction of HDQ to DQ]
%TDOO: do not understand
%HDQ is a generalized form of DQ as both the upstream and downstream queue in DQ are made up of the congestion in flow area and buffer area in HDQ.
HDQ is a generalized form of DQ by incorporating the influence of headway; HDQ could be reduced to the DQ under the assumption of DQ.
\label{prop:HDQ&DQ}
\end{proposition}
\begin{proof}
See \ref{app:generality}.
\end{proof}

To consider general networks with multiple destinations, we further represent the downstream and upstream queue in HDQ by Equations \ref{eq:dq_con} and \ref{eq:uq_con} for $ (i,j)\in E,~ s^{\prime}\in \widetilde{S}$ and $t\in[0,T]$.

\begin{eqnarray}
\textit{q}_{i,j}^{\mathcal{D},s^{\prime}}(t) &=& \int_{0}^{\textit{t}} \textit{f}_{i,j}^{~ s^{\prime}}(t) ~dt - \int_{0}^{\textit{t}} \textit{v}_{i,j}^{s^{\prime}}(t) ~dt + \textit{q}_{i,j}^{\mathcal{D},s^{\prime}}(0), 
\label{eq:dq_con} \\
\textit{q}_{i,j}^{\mathcal{U},s^{\prime}}(t) &=& \int_{0}^{\textit{t}} \textit{u}_{i,j}^{~ s^{\prime}}(t) ~dt - \int_{0}^{\textit{t}-\tau_{i,j}^{w}(t)} \textit{f}_{i,j}^{~ s^{\prime}}(t) ~dt + \textit{q}_{i,j}^{\mathcal{U},s^{\prime}}(0), 
\label{eq:uq_con} 
\end{eqnarray}
where $\textit{q}_{i,j}^{\mathcal{D},s^{\prime}}(t)$, $\textit{q}_{i,j}^{\mathcal{U},s^{\prime}}(t)$ and $\textit{v}_{i,j}^{s^{\prime}}(t)$ are the downstream queue, upstream queue and outflow of the buffer area with destination $s^{\prime}$ of link $(i,j)$ at time $t$, respectively.

We further have Equations \ref{eq:sum_1_con} and \ref{eq:sum_2_con} hold, meaning that traffic flow dynamics variables are the summation of their destination-specified variables for $ (i,j)\in E,~ s^{\prime}\in \widetilde{S}$ and $t\in[0,T]$.

\begin{alignat}{4}
\textit{q}_{i,j}^{\mathcal{D}}(t) &=\sum_{s^{\prime}}\textit{q}_{i,j}^{\mathcal{D},s^{\prime}}(t);   &&\quad
\textit{q}_{i,j}^{\mathcal{U}}(t) = \sum_{s^{\prime}} \textit{q}_{i,j}^{\mathcal{U},s^{\prime}}(t); &&&\quad
\rho_{i,j}(t) = \sum_{s^{\prime}} \rho_{i,j}^{s^{\prime}}(t).   \label{eq:sum_1_con}\\
\textit{u}_{i,j}(t) &= \sum_{s^{\prime}} \textit{u}_{i,j}^{s^{\prime}}(t); &&\quad \textit{f}_{i,j}(t) = \sum_{s^{\prime}} \textit{f}_{i,j}^{~ s^{\prime}}(t); &&&\quad
\textit{v}_{i,j}(t) = \sum_{s^{\prime}} \textit{v}_{i,j}^{s^{\prime}}(t). 
\label{eq:sum_2_con}
\end{alignat}


Additionally, Equation \ref{eq:ineq_1_con} and \ref{eq:ineq_2_con} indicate that the downstream queue, upstream queue, inflow rate and outflow rate are bounded by their respective capacity for $ (i,j)\in E\setminus(L_{R}\cup L_{S}),~ s^{\prime}\in \widetilde{S}$ and $t\in[0,T]$.

\begin{alignat}{4}
\textit{q}_{i,j}^{\mathcal{D}}(t) &\leq \bar{Q}_{i,j}^{\mathcal{D}},   \quad
\textit{q}_{i,j}^{\mathcal{U}}(t) &&\leq \bar{Q}_{i,j}^{\mathcal{U}},
\label{eq:ineq_1_con}\\
\textit{u}_{i,j}(t) &\leq \bar{C}_{i,j}^{\mu}, \quad 
\textit{v}_{i,j}(t) &&\leq \bar{C}_{i,j}^{\nu},
\label{eq:ineq_2_con}
\end{alignat}
where $\bar{Q}_{i,j}^{\mathcal{D}}$ and $\bar{Q}_{i,j}^{\mathcal{U}}$ are the downstream queue capacity and upstream queue capacity, respectively; $\bar{C}_{i,j}^{\mu}$ and $\bar{C}_{i,j}^{\nu}$ are the inflow and outflow capacity of link $(i,j)$ at time $t$, respectively.
We further have Proposition \ref{prop:property} indicating that the impact to the maximal flow rate of the link is dependent on the HDQ.

\begin{proposition}[Headway-dependent flow rate bounds] The upper bounds of the inflow rate $\textit{u}_{i,j}(t)$ and outflow rate $\textit{v}_{i,j}(t)$ are dependent on the upstream queue $\textit{q}^{\mathcal{U}}_{i,j}(t)$ and downstream queue $\textit{q}^{\mathcal{D}}_{i,j}(t)$, respectively, and their relation is shown as follows.
\begin{equation}
\begin{array}{ccccc}
\qquad\qquad~\textit{u}_{i,j}(t) \leq \begin{cases}
\bar{C}_{i,j}^{\mu}, & \textit{q}_{i,j}^{\mathcal{U}}(t)<\bar{Q}_{i,j}^{u} \\
\min \left\{ \textit{f}_{i,j}(t-\tau_{i,j}^{w}(t)),~ \bar{C}_{i,j}^{\mu} \right\}, & \textit{q}_{i,j}^{\mathcal{U}}(t)=\bar{Q}_{i,j}^{u}
\end{cases} \\
\textit{v}_{i,j}(t) \leq \begin{cases}
\bar{C}_{i,j}^{\nu}, & \textit{q}_{i,j}^{\mathcal{D}}(t)>0 \\
\min \left\{ \textit{f}_{i,j}(t),~ \bar{C}_{i,j}^{\nu} \right\}, &
\textit{q}_{i,j}^{\mathcal{D}}(t)=0
\end{cases}
% \textit{u}_{i,j}(t) \leq \left\{ \begin{array}{lcl}
% \bar{C}_{i,j}^{\mu}  &  & ,~\textit{q}_{i,j}^{\mathcal{U}}(t)<\bar{Q}_{i,j}^{u} \\
% \min \left\{ \textit{f}_{i,j}(t-\tau_{i,j}^{w}(t)),~ \bar{C}_{i,j}^{\mu} \right\} &  & ,~\textit{q}_{i,j}^{\mathcal{U}}(t)=\bar{Q}_{i,j}^{u}
% \end{array} \right\} \\
% \textit{v}_{i,j}(t) \leq \left\{ \begin{array}{lcl}
% \bar{C}_{i,j}^{\nu}  &  & ,~\textit{q}_{i,j}^{\mathcal{D}}(t)>0 \\
% \min \left\{ \textit{f}_{i,j}(t),~ \bar{C}_{i,j}^{\nu} \right\} &  & ,~\textit{q}_{i,j}^{\mathcal{D}}(t)=0
% \end{array} \right\}
\end{array}  
\nonumber
\end{equation}
\label{prop:property}
\end{proposition}


\begin{proof}
See \ref{app:HDQ}.
\end{proof}

\subsection{Dynamic network traffic model}
\label{sec:Autonomous_vehicle_Headway_Control}
This section models the network-wide effects of AV headway control by developing a dynamic network model. 
% Before this, we have the following assumption:
% \begin{assumption}
% \item The possible conflicts at nodes when AVs pass the intersections are not considered. We suppose all AVs would pass the intersections and enter the designed link successfully. 
% \end{assumption}
The flow conservation at nodes is stated in Equation \ref{eq:flow_conser_con} for $i \in N\setminus(\widetilde{R}\cup \widetilde{S}),~ s^{\prime}\in \widetilde{S}$ and $t\in[0,T]$.

\begin{equation}
\sum_{p}\textit{v}_{p,i}^{s^{\prime}}(t)=\sum_{j}\textit{u}_{i,j}^{s^{\prime}}(t)
\label{eq:flow_conser_con}
\end{equation}

The link conservation is ensured through the HDQ model. Equation \ref{ineq:nonnega_con} presents the non-negative constraint of all the variables involved for $(i,j)\in E,~ s^{\prime}\in \widetilde{S}$ and $t\in[0,T]$.

\begin{equation}
\rho_{i,j}^{s^{\prime}}(t),~ \textit{u}_{i,j}^{s^{\prime}}(t), ~ \textit{f}_{i,j}^{~s^{\prime}}(t), ~ \textit{v}_{i,j}^{s^{\prime}}(t), ~ \textit{q}_{i,j}^{\mathcal{D},s^{\prime}}(t), ~
\textit{q}_{i,j}^{\mathcal{U},s^{\prime}}(t)\geq0
\label{ineq:nonnega_con}
\end{equation}

Besides, the number of AVs in link $(i,j)$  with destination $s^{\prime}$ at time $t$ could be represented as the sum of number of AVs in the flow area and buffer area, as shown in Equation~\ref{eq:sum}.

\begin{equation}
\label{eq:sum}
\begin{array}{clll}
 \textit{N}_{i,j}^{s^{\prime}}(t) &=&  \int_{\textit{0}}^{\textit{t}}\textit{u}_{i,j}^{s^{\prime}}(t) ~dt - \int_{\textit{0}}^{\textit{t}}\textit{v}_{i,j}^{s^{\prime}}(t) ~dt  \\
&=& \int_{\textit{0}}^{\textit{t}}\textit{u}_{i,j}^{s^{\prime}}(t) ~dt -
\int_{\textit{0}}^{\textit{t}}\textit{f}_{i,j}^{~s^{\prime}}(t) ~dt +
\int_{\textit{0}}^{\textit{t}}\textit{f}_{i,j}^{~s^{\prime}}(t) ~dt -
\int_{\textit{0}}^{\textit{t}}\textit{v}_{i,j}^{s^{\prime}}(t) ~dt \\ &=& L_{i,j} ~\rho_{i,j}^{s^{\prime}}(t)+\textit{q}_{i,j}^{\mathcal{D},s^{\prime}}(t)
\end{array} 
\end{equation}



The initial condition of the network is stated in Equation \ref{eq:init_con} for $(i,j)\in E, ~ s^{\prime}\in \widetilde{S}$ and $t\in[0,T]$, which guarantees the whole traffic network is empty before loading.

\begin{equation}
\rho_{i,j}(0), ~
\textit{q}_{i,j}^{\mathcal{D},s^{\prime}}(0)=0
\label{eq:init_con}
\end{equation}

The end condition constraint is presented in Equation \ref{ineq:end_con} for $(i,j)\in E, ~ s^{\prime}\in \widetilde{S}$ and $t\in[0,T]$, which guarantees the whole traffic network is empty in the end. 
\begin{equation}
\textit{q}_{i,j}^{\mathcal{D},s^{\prime}}(T)=0, ~
\rho_{i,j}(T)<\frac{1}{L_{i,j}}
\label{ineq:end_con}
\end{equation}

Importantly, for the change rate of link density can be represented by $\dot{\rho}_{i,j}(t)=-\frac{\textit{v}_{i,j}^{f}\,\rho_{i,j}(t)}{L_{i,j}}, t\geq t_{0}$ based on Equation~\ref{eq:density_total_con} when $\textit{u}_{i,j}(t) = 0$. This represents the scenario in which no vehicles will enter the link 
% if the density $\rho_{i,j}(t)$ would be in free-flow region and $u_{i,j}(t)=0$ after time $t_{0}$, then $f_{i,j}(t)=\textit{v}_{i,j}^{f}\,\rho_{i,j}(t)$ for $t\geq t_{0}$ and we have 
when $\rho_{i,j}(t_{0})>0$. $\rho_{i,j}(t)$ has the shape of the negative exponential function, and hence it will take infinite time to make $\rho_{i,j}(t)=0$. In order to avoid this situation, we set the end constraint of the flow area in link $(i,j)$ as $\rho_{i,j}(T)<\frac{1}{L_{i,j}}$, which means that we suppose the flow area is empty when the number of vehicles in flow area is smaller than 1 at the end time $T$. Furthermore, we could also set a large positive constant $c$ as $\rho_{i,j}(T)<\frac{1}{cL_{i,j}}$ to ensure the vehicles in total network is close to 0 at end time $T$.
% Furthermore, Equation \ref{eq:var_dummy_con} means that for variables related to OD connectors, we only consider the  $\textit{f},\,\textit{q}^{d},\, \textit{v}$ in dummy origin links and $\textit{u}$ in dummy destination links for $ r \in \widetilde{R},~ s \in \widetilde{S}$ and $t\in [0,T]$.

% \begin{equation}
% \textit{u}_{r^{\prime},r}^{s^{\prime}}(t), ~
% \rho_{r^{\prime},r}^{s^{\prime}}(t), ~
% \textit{q}_{r^{\prime},r}^{\mathcal{U},s^{\prime}}(t), ~
% \textit{f}_{s,s^{\prime}}^{~s^{\prime}}(t), ~
% \rho_{s,s^{\prime}}^{s^{\prime}}(t), ~
% \textit{v}_{s,s^{\prime}}^{s^{\prime}}(t), ~
% \textit{q}_{s,s^{\prime}}^{\mathcal{D},s^{\prime}}(t), ~
% \textit{q}_{s,s^{\prime}}^{\mathcal{U},s^{\prime}}(t) =0 
% \label{eq:var_dummy_con}
% \end{equation}

% Besides, at the dummy destination node $s$, only flow with destination $s^{\prime}$ would enter the dummy link $(s,s^{\prime})$. The equation \ref{eq:u_dummy_con} represents this constraint to ensure the flow would arrive in their own destinations for $ s^{\prime},j^{\prime}\in \widetilde{S},~ s^{\prime} \neq j^{\prime}$ and $t \in [0,T]$.

% \begin{equation}
% \textit{u}_{s, s^{\prime}}^{j^{\prime}}(t) =0 
% \label{eq:u_dummy_con}
% \end{equation}

Equation \ref{eq:demand_con} depicts the travel demand entering the origin connectors  for $r \in \widetilde{R},~ s^{\prime}\in \widetilde{S}$ and $t \in [0,T]$:

\begin{equation}
\textit{f}_{r^{\prime},r}^{~s^{\prime}}(t)=\left\{
\begin{array}{lcl}
\textit{d}_{r^{\prime},s^{\prime}}(t),  &  & 0\leq\textit{t}\leq\textit{T}_\textit{1} \\
0, &  & \textit{T}_\textit{1}<\textit{t}\leq\textit{T}
\end{array}  \right.
\label{eq:demand_con}
\end{equation}
where $\textit{d}_{r^{\prime},s^{\prime}}(t)$ represents the travel demand of O-D pair $(r^{\prime},s^{\prime})$ at time $t$.

Additionally, Equation \ref{ineq:headway_con} limits the minimum and maximum  headway for $(i,j)\in E\setminus(L_R\cup L_S),~ s^{\prime}\in \widetilde{S}$ and $t\in[0,T]$.

\begin{equation}
\textit{h}_{i,j}^{min}\leq\textit{h}_{i,j}(t)\leq
\textit{h}_{i,j}^{max}
\label{ineq:headway_con}
\end{equation}


\section{Optimal headway control}
\label{sec_system_optimal_headway_formulation}

% This section aims to formulate the optimal headway control for SO-DTA. 
% So this section introduces the optimal headway control framework in the aspect of SO-DTA and safety concerns would be considered in Section \ref{sec_Optimal_maximin_Headway_Formulation}. 
% We briefly introduce the optimal headway control and maximin headway control here. 
% Strict definition of optimal headway control and maximin headway control would be presented in Section \ref{subsec:Mixed_Integer_Linear_Programming_Problem_Formulation} and \ref{subsec:Optimal_maximin_Headway_Definition_and_Formulation}.

% \begin{itemize}
%     \item optimal headway control is the headway setting where SO-DTA could be achieved to minimize TTT and there could be multiple optimal headway controls. Safety is not considered for optimal headway control.
%     \item maximin headway control is the ``largest'' headway setting amone all optimal headway controls so that both SO-DTA and safety are considered. 
% \end{itemize}

In this section, we first formulate the continuous and discretized headway control for SO-DTA.
Then we prove that the minimum headway control could achieve SO-DTA, and it is not the unique solution. 
%Additionally, we further develop a mixed integer linear programming (MILP) formulation to model how the headway affects SO-DTA. 
% Then we convert it into . Finally, we define the optimal headway control and propose a sensitivity analysis-based optimal headway control solving algorithm to derive the optimal headway control and minimum TTT. 

\subsection{Continuous-time formulation}
\label{sec:contin_SO-DTA}

The objective of the SO-DTA aims to minimize total travel time (TTT), which equals to the sum of the travel time on road networks and the waiting time at origins, as represented by Equation~\ref{eq:obj_con}. The decision variables include traffic dynamics $\textbf{x}$ and headway $\textbf{h}$. 

% The objective is designed to minimize total travel time (TTT), which equals to the sum of total travel time in regular network and waiting time in dummy origin links. $\sum\limits_{r^{\prime}}\int_{\textit{0}}^{\textit{t}}\textit{v}_{r^{\prime},r}(t) ~dt - \sum\limits_{s^{\prime}}\int_{\textit{0}}^{\textit{t}}\textit{u}_{s,s^{\prime}}(t) ~dt$ is the number of AVs in regular traffic network at time $t$.
% $\sum\limits_{(r^{\prime},s^{\prime})}\textit{q}_{r^{\prime},\textit{r}}^{d,s^{\prime}}(t) ~dt$ is the number of AVs waiting in the dummy origin links at time $t$. So 

\begin{equation}
\min_{\textbf{x}, \textbf{h}} ~TTT= \int_{\textit{0}}^{\textit{T}}\left(\sum_{r^{\prime}}\int_{\textit{0}}^{\textit{t}}\textit{v}_{r^{\prime},r}(t) ~dt - \sum_{s^{\prime}}\int_{\textit{0}}^{\textit{t}}\textit{u}_{s,s^{\prime}}(t) ~dt\right) ~dt + \int_{0}^{T}\sum\limits_{(r^{\prime},s^{\prime})}\textit{q}_{r^{\prime},\textit{r}}^{\mathcal{D},s^{\prime}}(t) ~dt
\label{eq:obj_con}
\end{equation}
subject to:
\begin{itemize}
\item Headway-dependent fundamental diagram constraints: \textit{Eqs}. (\ref{eq:FD})-(\ref{eq:shockwave_con})
\item Headway-dependent double queue constraints: \textit{Eqs}.  (\ref{eq:dq_con})-(\ref{eq:ineq_2_con})
\item Flow conservation constraints: \textit{Eq}. (\ref{eq:flow_conser_con})
\item Non-negative constraints: \textit{Eq}. (\ref{ineq:nonnega_con})
\item Initial constraints: \textit{Eq}. (\ref{eq:init_con})
\item End constraints: \textit{Eq}. (\ref{ineq:end_con})
\item Other constraints: \textit{Eqs}. (\ref{eq:demand_con})-(\ref{ineq:headway_con})
\end{itemize}

% The first term in the objective function represents total travel time in regular network and the last term represents waiting time in dummy origin links. 

Proposition \ref{prop:feasibility} presents the feasibility of the proposed optimal headway control problem, where $\textit{T}_\textit{1}$, $n^{r^{\prime},s^{\prime}}$, $P^{r^{\prime},s^{\prime}}$, $T_{i,j}^{r^{\prime},s^{\prime}}$ and $\delta_{i,j}^{P^{r^{\prime},s^{\prime}}}$ are defined in \ref{app:feas}.

\begin{proposition}[Feasibility]
    %If $\textit{T}>\textit{T}_\textit{1}+\sum\limits_{(r^{\prime},s^{\prime})}n^{r^{\prime},s^{\prime}}\sum\limits_{(i,j)~in~P^{r^{\prime},s^{\prime}}}T_{i,j}^{r^{\prime},s^{\prime}}$, the optimal headway control problem is always feasible.
    If $\textit{T}>\textit{T}_\textit{1}+\sum\limits_{(r^{\prime},s^{\prime})}n^{r^{\prime},s^{\prime}}\sum\limits_{(i,j)}T_{i,j}^{r^{\prime},s^{\prime}}\delta_{i,j}^{P^{r^{\prime},s^{\prime}}}$, the optimal headway control problem in Equation~\ref{eq:obj_con} is always feasible.
    \label{prop:feasibility}
\end{proposition}
\begin{proofsketch}
It is obvious that if we choose a large enough time horizon $T$, the problem is always feasible, which means all AVs would arrive at their destinations before $T$. So the basic idea for the proof  is to find a large enough $T$ and construct a feasible solution under $T$. Details are presented in \ref{app:feas}.
\end{proofsketch}


\subsection{Discretized-time formulation}

This section discretizes the formulation proposed in Section~\ref{sec:contin_SO-DTA}. 
We set $\textit{T}=\textit{N}\Delta_{t}$, where $\textit{N}$  represents the number of time intervals and $\Delta_{t}$ represents the length of a time interval. We choose a small $\Delta_{t}$ such that $\Delta_{t}L<min\{L_{i,j}h_{i,j}^{min}\}$ to have a non-zero shockwave travel time.
% Besides, $\textit{n}_{\textit{1}}=\frac{\textit{T}}{\Delta_{t}}$ should be an integer. 
The continuous formulation is non-linear programming due to the headway-dependent fundamental diagram constraints. 
% In order to convert the proposed problem into a solvable mixed integer linear programming (MIP) when fixing the headway control variables, we do the following relaxation:

% For simplicity, we suppose 
% $\Delta_{t}>\textit{max}\{\frac{L_{i,j}}{\textit{v}_{i,j}^f}\}$, which represents the $\Delta_{t}$ is larger than free-flow travel speed in any link so the constraint \ref{eq:f_con} could be simplified. If there exists a significant variance among the free-flow travel time of all links, we just set free-flow travel time of all links are multiples of the length of a time interval $\Delta_{t}$, which would add if-else constraints to and require many other binary variables to represent whether the first AV is entering an empty flow area or not. This constraint only sets that the first AV entering an empty flow area of a link would drive in free-flow speed and doesn't involve the headway setting, so it is unrelated with different headway settings and doesn't effect that we convert this problem into a MILP.
% Therefore, in this paper we set $\Delta_{t}>\textit{max}\{\frac{L_{i,j}}{\textit{v}_{i,j}^f}\}$ to reduce computation cost and it wouldn't affect the conclusion in Proposition \ref{prop:minimum} and Proposition \ref{prop:optimal} in Section \ref{subsub:analytics}.

To identify the congestion and non-congestion region in HFD by Equation~\ref{eq:FD}, we introduce the binary variables $\delta_{i,j}(k)$ to indicate whether the flow area of link $(i,j)$ in the $k$th time interval is congested or not.
% If $\delta_{i,j}(t)=0$, the flow area of link $(i,j)$ is not congested in $kth$ time interval. Otherwise, the flow area of link $(i,j)$ is congested in $kth$ time interval. 
Hence constraint \ref{eq:f_con} could be equivalently converted to a binary constraint. Other variables are also discretized and a summary of the notations is shown in Table \ref{notation_dis}.

\begin{table}[H]
    \centering
    \resizebox{1.02\textwidth}{!}{
    \begin{tabular}{lll}
        \toprule
        Notation   & Description \\
        \midrule
         $\rho_{i,j}(k)$ & Density of the flow area of link $(i,j)$ in the  $\textit{k}$th time interval \\
         $\rho_{i,j}^{s^{\prime}}(k)$ & Density of the flow area of link $(i,j)$ with destination $s^{\prime}$ in the  $\textit{k}$th time interval \\
        $\textit{q}_{i,j}^{\mathcal{D}}(k)$ & Downstream queue of link $(i,j)$ in the  $\textit{k}$th time interval  \\
        $\textit{q}_{i,j}^{\mathcal{D},s^{\prime}}(k)$ & Downstream queue of link $(i,j)$ with destination $s^{\prime}$ in the  $\textit{k}$th time interval  \\
        $\textit{q}_{i,j}^{\mathcal{U}}(k)$ & Upstream queue of link $(i,j)$ in the  $\textit{k}$th time interval  \\
        $\textit{q}_{i,j}^{\mathcal{U},s^{\prime}}(k)$ & Upstream queue of link $(i,j)$ with destination $s^{\prime}$ in the  $\textit{k}$th time interval  \\
        $\textit{u}_{i,j}(k)$ & Inflow of the flow area of link $(i,j)$ in the  $\textit{k}$th time interval  \\
        $\textit{u}_{i,j}^{s^{\prime}}(k)$ & Inflow of the flow area of link $(i,j)$ with destination $s^{\prime}$ in the  $\textit{k}$th time interval  \\
        $\textit{f}_{i,j}(k)$ & Inflow at the boundary of the flow area and  buffer area of link $(i,j)$ in the  $\textit{k}$th time interval  \\
        $\textit{f}_{i,j}^{~s^{\prime}}(k)$ & Inflow at the boundary of the flow area and buffer area of link $(i,j)$ with destination $s^{\prime}$ in the $\textit{k}$th time interval \\
        $\textit{v}_{i,j}(k)$ & Outflow of the buffer area of link $(i,j)$ in the  $\textit{k}$th time interval  \\
        $\textit{v}_{i,j}^{s^{\prime}}(k)$ & Outflow of the buffer area of link $(i,j)$ with destination $s^{\prime}$ in the  $\textit{k}$th time interval  \\
        $\textit{h}_{i,j}(k)$ & Headway of automated vehicles in link $(i,j)$ and $\textit{k}$th time interval  \\
        $\delta_{i,j}(k)$ & Binary variable to indicate whether congested in the flow area of link $(i,j)$ in the  $\textit{k}$th time interval \\
        $n_{i,j}^{w}(k)$ & Shockwave travel time on link $(i,j)$ in the  $\textit{k}$th time interval \\
        \bottomrule
    \end{tabular}
    }
    \caption{Notations in the discretized formulation.}
    \label{notation_dis}
\end{table}

We are now ready to present the discretized version of the headway-dependent SO-DTA formulation as follows: 
% The objective function is stated as Equation \ref{eq:obj_dis} and represents the total travel time (TTT) in the whole traffic network.
\begin{equation}
\min_{\textbf{x},\textbf{h}} ~TTT=  \quad \sum_{r^{\prime}}\sum_{\textit{k}=1}^{\textit{N}}\Delta_{t}\left(\sum_{\textit{l}=1}^{\textit{k}}\Delta_{t}\textit{v}_{r^{\prime},r}(l)\right)-\sum_{s^{\prime}}\sum_{\textit{k}=1}^{\textit{N}}\Delta_{t}\left(\sum_{\textit{l}=1}^{\textit{k}}\Delta_{t}\textit{u}_{s,s^{\prime}}(l)\right)+\sum_{(r^{\prime},s^{\prime})}\sum_{\textit{k}=1}^{\textit{N}}\Delta_{t}\textit{q}_{r^{\prime},\textit{r}}^{\mathcal{D},s^{\prime}}(k)
\label{eq:obj_dis}
\end{equation}

subject to

1. Headway-dependent fundamental diagram constraints for $(i,j)\in E\setminus(L_R\cup L_S),~ s^{\prime}\in \widetilde{S}$ and $1\leq k \leq N$. We relax the assumption of Equation \ref{eq:f_con} as $f_{i,j}(k)=f_{i,j}^{FD}(k)$. 

\begin{equation}
\begin{aligned}
if ~ \delta_{i,j}(k) &=0:\\
 & 0\leq\rho_{i,j}(k) < \frac{1}{\textit{h}_{i,j}(k)\textit{v}_{i,j}^f+\textit{L}} \\
 & \textit{f}_{i,j}(k)=\textit{v}_{i,j}^{f}{~} \rho_{i,j}(k)
 \label{eq:free_dis}
\end{aligned}
\end{equation}

\begin{equation}
\begin{aligned}
if ~ \delta_{i,j}(k) &=1:\\
 & \frac{1}{\textit{h}_{i,j}(k)\textit{v}_{i,j}^f+\textit{L}} \leq \rho_{i,j}(k)\leq\frac{1}{\emph{L}} \\
 & \textit{f}_{i,j}(k)=\frac{1-\rho_{i,j}(k)\emph{L}}{\textit{h}_{i,j}(k)}
\label{eq:cong_dis}
\end{aligned}
\end{equation}

\begin{equation}
\rho_{i,j}^{s^{\prime}}(k)=\rho_{i,j}^{s^{\prime}}(k-1)+\Delta_{t}\frac{\textit{u}_{i,j}^{s^{\prime}}(k)-\textit{f}_{i,j}^{~s^{\prime}}(k)}{L_{i,j}} 
\label{eq:density_dis}
\end{equation}

The discretized form of Equation \ref{eq:shockwave_con} representing the shockwave travel time is shown as follows.

\begin{eqnarray}
\sum\limits_{l=k-n_{i,j}^{w}(k)}^{k} \Delta_{t} \frac{\textit{L}}{\textit{h}_{i,j}(l)}  &\leq& L_{i,j} 
\nonumber \\
\sum\limits_{l=k-n_{i,j}^{w}(k)-1}^{k} \Delta_{t} \frac{\textit{L}}{\textit{h}_{i,j}(l)}  &>& L_{i,j} 
\nonumber
\end{eqnarray}

We assume that shockwave travel time $n_{i,j}^{w}(k)$ is only determined by the headway $\textit{h}_{i,j}(k)$ in Equation \ref{eq:shockwave_dis1} and \ref{eq:shockwave_dis2}.
\begin{eqnarray}
n_{i,j}^{w}(k) \Delta_{t} \frac{\textit{L}}{\textit{h}_{i,j}(k)}  &\leq& L_{i,j} 
\label{eq:shockwave_dis1} \\
(n_{i,j}^{w}(k)+1) \Delta_{t} \frac{\textit{L}}{\textit{h}_{i,j}(k)}  &>& L_{i,j} 
\label{eq:shockwave_dis2}
\end{eqnarray}

2. Headway-dependent double queue constraints constraints for $s^{\prime}\in \widetilde{S}$ and $1\leq k \leq N$.  We limit $(i,j)\in E$ in Equations \ref{eq:dq_dis} to \ref{eq:sum_2_dis} and $(i,j)\in E\setminus(L_R\cup L_S)$ in  Equations \ref{eq:ineq_1_dis} to \ref{eq:ineq_2_dis}.

\begin{eqnarray}
\textit{q}_{i,j}^{\mathcal{D},s^{\prime}}(k) &=& \sum_{l=0}^{\textit{k}}\Delta_{t} \textit{f}_{i,j}^{~ s^{\prime}}(l) - \sum_{l=0}^{\textit{k}}\Delta_{t} \textit{v}_{i,j}^{s^{\prime}}(l)  + \textit{q}_{i,j}^{\mathcal{D},s^{\prime}}(0)
\label{eq:dq_dis} \\
\textit{q}_{i,j}^{\mathcal{U},s^{\prime}}(k) &=& \sum_{l=0}^{\textit{k}}\Delta_{t} \textit{u}_{i,j}^{~ s^{\prime}}(l) - \sum_{l=0}^{\textit{k}-n_{i,j}^{w}(k)}\Delta_{t} \textit{f}_{i,j}^{~ s^{\prime}}(t)  + \textit{q}_{i,j}^{\mathcal{U},s^{\prime}}(0)
\label{eq:uq_dis} 
\end{eqnarray}

\begin{alignat}{4}
\textit{q}_{i,j}^{\mathcal{D}}(k) &=\sum_{s^{\prime}}\textit{q}_{i,j}^{\mathcal{D},s^{\prime}}(k);   &&\quad
\textit{q}_{i,j}^{\mathcal{U}}(k) = \sum_{s^{\prime}} \textit{q}_{i,j}^{\mathcal{U},s^{\prime}}(k); &&&\quad
\rho_{i,j}(k) = \sum_{s^{\prime}} \rho_{i,j}^{s^{\prime}}(k)   \label{eq:sum_1_dis}\\
\textit{u}_{i,j}(k) &= \sum_{s^{\prime}} \textit{u}_{i,j}^{s^{\prime}}(k); &&\quad \textit{f}_{i,j}(k) = \sum_{s^{\prime}} \textit{f}_{i,j}^{~s^{\prime}}(k); &&&\quad
\textit{v}_{i,j}(k) = \sum_{s^{\prime}} \textit{v}_{i,j}^{s^{\prime}}(k) 
\label{eq:sum_2_dis}
\end{alignat}

\begin{alignat}{4}
\textit{q}_{i,j}^{\mathcal{D}}(k) &\leq \bar{Q}_{i,j}^{\mathcal{D}};   \quad
\textit{q}_{i,j}^{\mathcal{U}}(k) &&\leq \bar{Q}_{i,j}^{\mathcal{U}}
\label{eq:ineq_1_dis}\\
\textit{u}_{i,j}(k) &\leq \bar{C}_{i,j}^{\mu}; \quad 
\textit{v}_{i,j}(k) &&\leq \bar{C}_{i,j}^{\nu}
\label{eq:ineq_2_dis}
\end{alignat}

3. Flow conservation constraints for $i \in N\setminus(\widetilde{R}\cup \widetilde{S}),~ s^{\prime}\in \widetilde{S}$ and $1\leq k \leq N$.

\begin{equation}
\sum_{p}\textit{v}_{\emph{p},\emph{i}}^{s^{\prime}}(k)=\sum_{j}\textit{u}_{i,j}^{s^{\prime}}(k)
\label{eq:flow_conser_dis}
\end{equation}

4. Non-negative Constraint for $(i,j)\in E,~ s^{\prime}\in \widetilde{S}$ and $1\leq k \leq N $.

\begin{equation}
\rho_{i,j}^{s^{\prime}}(k), \textit{u}_{i,j}^{s^{\prime}}(k), ~ \textit{f}_{i,j}^{~s^{\prime}}(k), ~ \textit{v}_{i,j}^{s^{\prime}}(k), ~ \textit{q}_{i,j}^{\mathcal{D},s^{\prime}}(k), ~
\textit{q}_{i,j}^{\mathcal{U},s^{\prime}}(k)\geq0
\label{ineq:nonnega_dis}
\end{equation}

4. Initial condition constraint for $(i,j)\in E$ and $s^{\prime}\in \widetilde{S}$.

\begin{equation}
\rho_{i,j}(0), ~
\textit{q}_{i,j}^{\mathcal{D},s^{\prime}}(0)=0
\label{eq:init_dis}
\end{equation}

5. End condition constraint for $(i,j)\in E$ and $s^{\prime}\in \widetilde{S}$.

\begin{equation}
\textit{q}_{i,j}^{\mathcal{D},s^{\prime}}(N)=0, ~
\rho_{i,j}(N)<\frac{1}{L_{i,j}}
\label{ineq:end_dis}
\end{equation}

6. Other constraints for $(i,j)\in E\setminus(L_R\cup L_S),~ r^{\prime} \in \widetilde{R},~ s^{\prime} \in \widetilde{S}$ and $1\leq k \leq N $, where $\textit{d}_{r^{\prime},s^{\prime}}(k)$ represents the travel demand of O-D pair $(r^{\prime},s^{\prime})$ in the $k$th time interval and $n_{1}=\frac{T_{1}}{\Delta_{t}}$.


%\begin{equation}
%\textit{u}_{r^{\prime},r}^{s^{\prime}}(k), ~
%\rho_{r^{\prime},r}^{s^{\prime}}(k), ~
%\textit{q}_{r^{\prime},r}^{u,s^{\prime}}(k), ~
%\textit{f}_{s,s^{\prime}}^{~s^{\prime}}(k), ~
%\rho_{s,s^{\prime}}^{s^{\prime}}(k), ~
%\textit{v}_{s,s^{\prime}}^{s^{\prime}}(k), ~
%\textit{q}_{s,s^{\prime}}^{d,s^{\prime}}(k), ~
%\textit{q}_{s,s^{\prime}}^{u,s^{\prime}}(k) =0 
%\label{eq:var_dummy_dis}
%\end{equation}

%\begin{equation}
%\textit{u}_{s, s^{\prime}}^{j^{\prime}}(k) =0 
%\label{eq:u_dummy_dis}
%\end{equation}

\begin{equation}
\textit{f}_{r^{\prime},r}^{~s^{\prime}}(k)=\left\{
\begin{array}{lcl}
\textit{d}_{r^{\prime},s^{\prime}}(k),  &  & 1\leq\textit{k}\leq\textit{n}_\textit{1} \\
0, &  & \textit{n}_\textit{1}<\textit{k}\leq\textit{N}
\end{array} \right.
\label{eq:demand_dis}
\end{equation}

\begin{equation}
\textit{h}_{i,j}^{min}\leq\textit{h}_{i,j}(k)\leq
\textit{h}_{i,j}^{max} 
\label{ineq:headway_dis}
\end{equation}

We define the  minimum total travel time and optimal headway of the discretized form of SO-DTA for Equations \ref{eq:obj_dis} to \ref{ineq:headway_dis} in Definition \ref{def:optimal_headway}.

\begin{definition}
\label{def:optimal_headway}
$\forall~\textbf{h}$ satisfies $\textit{h}_{i,j}^{min}\leq\textit{h}_{i,j}(k)\leq
\textit{h}_{i,j}^{max}$ for each element, if $\exists~ \textbf{x}$ satisfies Equations \ref{eq:free_dis} to \ref{eq:demand_dis}, then $\textbf{h}$ and $\textbf{x}$ constitute a feasible solution of SO-DTA problem, denote the total travel time as $TTT(\textbf{x})$ and the set of feasible traffic dynamics under $\textbf{h}$ as $\Omega_{\textbf{h}}$. $TTT_{\textbf{h}}=min\{TTT(\textbf{x})|x \in \Omega_{\textbf{h}}\}$ represents the minimum TTT if we set the headway variable $\textbf{h}$ as an exogenous input. We define the minimum total travel time and optimal headway  as follows:
\begin{itemize}
    \item Minimum total travel time (minimum TTT). $TTT^{*}$ is the minimum total travel time if  $TTT^{*}=min\{TTT_{\textbf{h}}\,|\,\textit{h}_{i,j}^{min} 
    \leq \textit{h}_{i,j}(k) \leq
\textit{h}_{i,j}^{max}; (i,j) \in E\setminus(L_R\cup L_S); 1\leq k \leq N\}$.
    \item Optimal headway. $\textbf{h}$ is the optimal headway if $TTT_{\textbf{h}}=TTT^{*}$; $\textbf{h}$ and $\textbf{x}$ is the optimal solution of SO-DTA problem if  $\textbf{x} \in \{\textbf{x}\,|\,TTT(\textbf{x})=TTT_{\textbf{h}}=TTT^{*};x \in \Omega_{\textbf{h}}\}$.  
\end{itemize}
%\begin{itemize}
%\item Feasible solution, $\forall~\textbf{h}$ satisfies $\textit{h}_{i,j}^{min}\leq\textit{h}_{i,j}(k)\leq
%\textit{h}_{i,j}^{max}$ for each element, if $\exists~ \textbf{x}$ satisfies Equations \ref{eq:free_dis} to \ref{eq:demand_dis}, then $\textbf{h}$ and $\textbf{x}$ constitute a feasible solution of SO-DTA problem and we denote the set of feasible DTA variables under $\textbf{h}$ as $\Omega_{\textbf{h}}$.
%\item TTT Determined Variables. Equation \ref{eq:obj_dis} implies that $TTT$ is directly determined by {$\textit{u}_{i,j}$}, $\textbf{v}$ and $\textbf{q}^{}$
%\item $TTT_{\textbf{h}}=min\{TTT(\textbf{x})|x \in \Omega_{\textbf{h}}\}$. 
%\item 
%\end{itemize}
\end{definition}

The proof of feasibility for the discretized form is similar that for the continuous form, and hence the detailed proof is omitted. %so we will not show the proof of feasibility of discretized form of the proposed problem in this paper.
The proposed problem is a nonlinear integer programming due to Equation \ref{eq:free_dis} to \ref{eq:cong_dis} and Equation \ref{eq:shockwave_dis1} to \ref{eq:shockwave_dis2}. However, if we set the headway variable $\textbf{h}$ as an exogenous input, the problem would become a mixed integer linear programming (MILP) in \ref{app:MILP}. 
The corresponding sensitivity-based solution algorithm is presented in \ref{app:SA}. 

However, for the SO-DTA problem, sensitivity analysis approach has large computation cost and it is hard to guarantee the convergence to global  optimal solution. Proposition \ref{prop:minimum} states that the minimum TTT can be achieved under the minimum headway setting, providing us a direct way to reach the state of SO-DTA. On top of that, Proposition \ref{prop:nonuni} states that it is possible to have multiple AV headway settings that yield the same SO-DTA conditions. 


%The shockwave speed $\tau_{i,j}^{w}(k)$ is fixed to $\frac{L_{i,j}}{\frac{\textit{L}}{\textit{h}_{i,j}(k)}}$, which equals to $\frac{L_{i,j}\textit{h}_{i,j}(k)}{\textit{L}}$, so the constraint \ref{eq:shockwave_con} could be eliminated, and $n_{i,j}^{w}(k)$=$\left\lfloor\frac{L_{i,j}\textit{h}_{i,j}(k)}{\textit{L}\Delta_{t}}\right\rfloor$ represents the shockwave travel time of link $(i,j)$ in $kth$ time interval.



\begin{proposition}[Minimun headway achieves SO-DTA]
%SO-DTA could be achieved under the minimum headway setting. Mathematically,
Supposing that $\textbf{h}^{min}=\{h_{i,j}(k)=h_{i,j}^{min}~|~(i,j) ~\in E\setminus(L_R\cup L_S); 1\leq k \leq N\}$, for $\forall \textbf{h}$ satisfies Equation \ref{ineq:headway_dis}, we have 
%$h_{i,j}^{min} \leq h_{i,j}(k) \leq h_{i,j}^{max}$, 
\begin{equation}
TTT_{\textbf{h}^{min}}=TTT^{*} \leq TTT_{\textbf{h}}.
%Mathematically, $\forall~\textbf{h}$ satisfies $\textit{h}_{i,j}^{min}\leq\textit{h}_{i,j}(k)\leq
%\textit{h}_{i,j}^{max}$ for each element, we have $TTT(\textbf{h})\geq TTT(\textbf{h}^{min})$.
\nonumber
\end{equation}
\label{prop:minimum}
\end{proposition}

\begin{proofsketch}
For any feasible solution $\textbf{x}$ and $\textbf{h}$ of the proposed SO-DTA formulation, we only change a smaller $\textit{h}_{i,j}(k)$ in $\textbf{h}$ and keep all other headway in $\textbf{h}$ unchanged. The new headway is denoted as $\textbf{h}^{*}$. Then we prove that we could always find a feasible $\textbf{x}^{*}$ under $\textbf{h}^{*}$ and the total travel time of $\textbf{x}^{*}$ is less than or equal to the total travel time of $\textbf{x}$
under original headway $\textbf{h}$. When we traverse all links and time intervals, it is proved that SO-DTA could be achieved under the minimum headway.
Details are presented in \ref{app:minimum}.
\end{proofsketch}


\begin{proposition}[Non-uniqueness]
%SO-DTA may have multiple optimal headway controls. Specifically, there exists multiple optimal headway controls if free-flow state in HFD is achieved in optimal traffic states. Mathematically, 

\begin{eqnarray}
& \forall \textbf{h} \text{ s.t. } TTT_{\textbf{h}}=TTT^{*}, if~ \exists~ \textbf{x} \in \Omega_{\textbf{h}} \text{ s.t. } TTT(\textbf{x})=TTT_{\textbf{h}}=TTT^{*} ~and~ \exists ~\delta_{i,j}(k) \in \textbf{x} \text{ s.t. }  \delta_{i,j}(k)=0
\nonumber\\
& then~ \exists~ \textbf{h}^{1} \neq \textbf{h}  \text{ s.t. } \textbf{x} \in \Omega_{\textbf{h}^{1}} ~and~ TTT_{\textbf{h}^{1}}=TTT(\textbf{x})=TTT_{\textbf{h}}=TTT^{*}.
\nonumber
\end{eqnarray}

\label{prop:nonuni}
\end{proposition}
%\begin{proofsketch}
%Suppose for any optimal solution $\textbf{x}^{*}$ and $\textbf{h}^{*}$ of the proposed SO-DTA problem, we only change $\textit{h}_{i,j}(k)$ in $\textit{h}^{*}$ if the corresponding $\delta_{i,j}(k)=0$, and we keep all other headway in $\textbf{h}^{*}$ unchanged. We denote the new headway setting as $\textbf{h}$, and it can be proved that there exists a certain scope of $\textit{h}_{i,j}(k)$ in which $\textbf{x}^{*}$ could still be achieved under $\textbf{h}$, implying that $\textbf{x}^{*}$ and $\textbf{h}$ are the new optimal solution of the proposed SO-DTA problem. Details are presented in \ref{app:nonuniq}.
%\end{proofsketch}
\begin{proof}
See \ref{app:nonuniq}.
\end{proof}

Proposition~\ref{prop:minimum} aligns with Observation~\ref{ob:1}, and Proposition~\ref{prop:nonuni} proves Observation~\ref{ob:nonunique}. Then we have two natural questions arise:

\begin{itemize}
    \item If we decide to control AVs with minimum headway in reality, would it cause other concerns ({\em e.g.}. safety)?
    \item Is there a ``best'' headway among all the optimal headway given the non-uniqueness of SO-DTA solutions?
\end{itemize}

Both questions motivate us to search for a better headway control setting that still achieve the SO-DTA, which is presented in the following section.



\section{Maximin headway control}
\label{sec_Optimal_maximin_Headway_Formulation}

% Section \ref{sec_system_optimal_headway_formulation}  solves the SO-DTA with the minimum headway control, yet the following questions reamin:
% proposes the formulation of SO-DTA problem, proves that SO-DTA could be achieved under the minimum headway setting in Proposition \ref{prop:minimum} and may have multiple optimal solutions in Proposition \ref{prop:nonuni}. However, there are still some questions about it.



In this section, we define the maximin headway as the ``largest'' headway setting among all the optimal headway setting that achieve the SO-DTA.
The problem of maximin headway control is formulated and solved analytically. 
% and proposes a maximin headway control solving algorithm by getting the state of SO-DTA under minimum headway setting first and exploring the ``largest'' headway setting to keep SO-DTA in Section \ref{subsub:analytics}.

%Section \ref{sec_system_optimal_headway_formulation} introduces the optimal headway contrl formulation to achieve SO-DTA and proposes a sensitivity analysis-based algorithm to derive the optimal headway control and minimum TTT. However, safety is also an important concern in headway control. SO-DTA may be achieved among different headway control settings. Therefore, how to explore the ``safest'' headway control among all these optimal headway controls is an important problem in our work. Besides, sensitivity analysis-based algorithm in section \ref{sec_system_optimal_headway_formulation} is not necessary to guarantee the convergence to minimum TTT.

%Therefore, in this section, we define the maximin  Besides, we theoretically prove that SO-DTA could be achieved under minimum headway setting, which provides us a more effective and direct way to derive the minimum TTT compared with the sensitivity analysis-based algorithm. And we would show this in section \ref{sec:Numerical_Results}. Besides, we propose an maximin headway control solving algorithm based on the above proposition by achieving SO-DTA first under the minimum headway setting.

\subsection{Maximin headway control formulation}
\label{subsec:Optimal_maximin_Headway_Definition_and_Formulation}

% The minimum headway setting has been proved the ability to achieve SO-DTA in Proposition \ref{prop:minimum}. 
% However, in common sense, smaller headway corresponds to the less time for drivers or vehicles to adjust when happening emergencies front, which increases the risk of accident. Therefore, we first explore the relationship between headway and safety in general traffic networks.

% There are many theoretical and experimental research focusing on the affect of headway on safety in traffic networks. \cite{tradeoff} analytically shows that there exists a trade-off between the time lag gap and the safety buffer that collectively constitutes the headway. Time lag gap represents mechanical response delay and measures the time of AV to perform the instruction. Time lag gap is always related with the AV manufacture and mechanical loss in an travel is too tight so we ignore it. So the time lag gap is supposed to be fixed in this problem. Safety buffer measures the safety and is to absorb AV overshoot in the worst case, such as a sharp full stop. Therefore, a smaller headway would correspond to a safer traffic condition. What's more, \cite{SHI2021103134} validates  theoretical conclusions in \cite{tradeoff} by conducting experiments on car-following characteristics of commercial AVs with different headway settings. All above works have proved that smaller headway settings  would always create a less safety environment in traffic network.

% Therefore, although minimum headway setting has already been proved as the optimal headway control, minimum headway setting is not the desired headway control for the safety concern. Proposition \ref{prop:nonuni} has proved the non-uniqueness of optimal solutions of SO-DTA. Therefore, we define the maximin headway control as the ``largest'' headway setting to achieve SO-DTA, which represents the headway setting makes the whole traffic network ``safest'' under SO-DTA.
 

%We first give the definition of maximin headway control. The goal of our work is to explore the maximin headway control to guarantee both the safety and SO-DTA. Section\ref{sec_system_optimal_headway_formulation} proposes the optimal headway control formulation and a sensitivity analysis-based algorithm to solve the headway control to achieve minimum TTT, but it is hard to guarantee the convergence to the actual minimum TTT. Therefore, it is necessary to define the maximin headway first to specify the "safest" headway control among all optimal headway controls. 

%The maximin headway control should first be a optimal headway control. We show there may be multiple optimal headway controls. For the solution of SO-DTA problem, \cite{SHEN20141} shows the SO-DTA problem may have multiple solutions, which means SO-DTA could be achieved under different headway settings. Therefore, there may be multiple optimal headway controls and we should choose the ``safest'' headway among them. \cite{tradeoff} analytically shows that there exists a trade-off between the time lag gap and the safety buffer that collectively constitutes the headway. Time lag gap represents mechanical response delay and measures the time of AV to perform the instruction. Time lag gap is always related with the AV manufacture and mechanical loss in an travel is too tight so we ignore it. So the time lag gap is supposed to be fixed in this problem. Safety buffer measures the safety and is to absorb AV overshoot in the worst case, such as a sharp full stop. Therefore, a smaller headway would correspond to a safer traffic condition. What's more, \cite{SHI2021103134} validates  theoretical conclusions in \cite{tradeoff} by conducting experiments on car-following characteristics of commercial AVs with different headway settings. Therefore, we define the maximin headway control as the ``largest'' headway setting to achieve SO-DTA, which represents the headway setting makes the whole traffic network ``safest'' under SO-DTA.

We first rigorously define the maximin headway  in Definition~\ref{def:optimal}.

\begin{definition}[Maximin Headway]
$\textbf{h}^{*}$ is called the maximin headway if it satisfies the following two conditions:

\begin{itemize} 
\item SO-DTA: $TTT_{\textbf{h}^{*}} = TTT^{*}$, meaning that $\textbf{h}^{*}$ is the optimal headway for SO-DTA.
\item Largest headway:
$\forall~\textbf{h}$ \textit{s.t.} $TTT_{\textbf{h}} = TTT^{*}$, we have $||~ \textbf{h}~ ||_{1} \leq ||~ \textbf{h}^{*} ||_{1}$, where $||\cdot||_{1}$ is the 1-norm.
\end{itemize}
\label{def:optimal}
\end{definition}


Definition~\ref{def:optimal} aligns with Definition~\ref{def:maximin} with more rigor and mathematical details. %It is straightforward to see that the maximin headway setting is unique for SO-DTA.  
We formulate a bi-level optimization to search for the maximin headway, as shown in Equation~\ref{eq:maximin}.

\begin{equation}
\begin{aligned}
\max\limits_{\textbf{h}} \quad & \mathbf{1}^T~\textbf{h}\\
\textrm{s.t.} \quad & \textbf{h} \in \left\{\mathop{\arg\min}_{\textbf{x},\textbf{h}}  TTT\,(\ref{eq:obj_dis})\right\} \\
& \qquad \quad \textrm{s.t.} \quad  \textit{Headway-dependent fundamental diagram constraints}\, (\ref{eq:free_dis})-(\ref{eq:shockwave_dis2}) \\
& \qquad \quad \quad \quad \, \textit{Headway-dependent double queue constraints}\, (\ref{eq:dq_dis})-(\ref{eq:ineq_2_dis})   \\
& \qquad \quad \quad \quad \, \textit{Flow conservation constraints}\, 
(\ref{eq:flow_conser_dis}) \\
& \qquad \quad \quad \quad \, \textit{Non-negative Constraints}\, 
(\ref{ineq:nonnega_dis}) \\
& \qquad \quad \quad \quad \, \textit{Initial condition constraints}\, 
(\ref{eq:init_dis}) \\
& \qquad \quad \quad \quad \, \textit{Initial condition constraints}\, 
(\ref{ineq:end_dis}) \\
& \qquad \quad \quad \quad \, \textit{Other constraints}\,
(\ref{eq:demand_dis})-(\ref{ineq:headway_dis})
% &\Longleftrightarrow 
% \max\limits_{\textbf{h}\in \textit{H}_{SO}} \quad  \lvert\rvert~\textbf{h}~\lvert\rvert\\
\end{aligned}
\label{eq:maximin}
\end{equation}

% However, it is difficult to derive the actual minimum TTT by sensitivity analysis-based algorithm, not to mention the set of optimal headway control $\textit{H}_{SO}$. Therefore, 
We note it is a direct extension to replace $\mathbf{1}^T$ with a weight vector $\mathbf{w}^T$ to represent the importance of headway on different roads. Proposition \ref{prop:unique} discusses the uniqueness of maximin headway of Formulation \ref{eq:maximin}.


% Formulation~\ref{eq:maximin} can be eqvanlently converted to $\max\limits_{\textbf{h}\in \textit{H}_{SO}} ||\textbf{h}||$ based on the analysis in \ref{app:MILP}.

\begin{proposition}[Uniqueness of maximin headway]
Suppose $\Phi =  \mathop{\cup}\limits_{\textbf{h}}\Omega_{\textbf{h}}$ is the set of feasible traffic dynamics $\textbf{x}$,
\begin{eqnarray}
& If ~\exists ~\textbf{x}^{*} \in \Phi ~\textit{s.t.}~ TTT(\textbf{x}^{*})<TTT(\textbf{x}), ~for~\forall \textbf{x}\in \Phi ~and~ \textbf{x} \neq \textbf{x}^{*}, 
\nonumber \\
& then ~\exists ~\textbf{h}^{*} ~satisfying~
%~ Equation ~\ref{ineq:headway_dis}~ and~ 
TTT_{\textbf{h}^{*}}=TTT(\textbf{x}^{*}), ~we~ have ~
||~ \textbf{h}^{*}~ ||_{1} > ||~ \textbf{h} ||_{1}, ~\forall \textbf{h}~ satisfying~ %~ Equation ~\ref{ineq:headway_dis}~ and~ 
TTT_{\textbf{h}}=TTT(\textbf{x}^{*})~and~ \textbf{h}^{*}\neq\textbf{h}. 
\nonumber \\
& \Longleftrightarrow If ~\textbf{x}^{*} ~with~ TTT(\textbf{x}^{*})=TTT^{*} ~is~ unique, the~ optimal~ solution~ of~ Formulation~\ref{eq:maximin} ~is ~unique.
\nonumber
\end{eqnarray}
\label{prop:unique}
\end{proposition}

\begin{proof}
See \ref{app:unique}.
\end{proof}

% it is necessary to find a way to explore the SO-DTA and derive the minimum TTT. 

\subsection{Maximin headway control solution algorithm}
\label{subsub:analytics}

This section proposes a maximin headway control solving algorithm with the idea of exploring the maximum increase of headway to maintain the states of SO-DTA achieved by minimum headway setting.
%In this section, we first present an equivalent form of the objective function indicating that SO-DTA encourages to push flow to arrive their own destinations as early as possible. Inspired by this, we prove that SO-DTA could be achieved under minimum headway setting, which provides us a direct way to derive the SO-DTA and SO-DTA solution just by solving the Programming (\Rmnum{1}) under minimum headway setting. Based on the above proposition and the definition of maximin headway control, we propose an maximin headway control solving algorithm to derive the maximin headway control, which guarantees both the SO-DTA and safety.
Proposition \ref{prop:optimal} provides us a method to derive a ``larger'' optimal headway  given the current optimal headway.

%Proposition \ref{prop:minimum} shows us a way to achieve SO-DTA, we just need to solve the Programming (\Rmnum{1}) under minimum headway $\textbf{h}^{min}$, so we could derive the minimum TTT and corresponding solution variable values. Then according to the definition of maximin headway control, we still need to find the maximin headway that still achieves SO-DTA. Proposition \ref{prop:optimal} discusses the existence condition of maximin headway control and shows us a way to derive the maximin headway setting but keep SO-DTA achievable by making the solution variable values under SO-DTA unchanged in the proof of Proposition \ref{prop:optimal}. The detailed proof of proposition \ref{prop:optimal} is shown in Appendix.

\begin{proposition}[Larger headway for SO-DTA]
Suppose $\textbf{h}^{*}$ satisfies Equation \ref{ineq:headway_dis} and $ \textbf{x}^{*} \in \Omega_{\textbf{h}^{*}}$ \textit{s.t.} $TTT(\textbf{x}^{*})=TTT_{\textbf{h}^{*}}=TTT^{*}$. If~ $\textbf{h}$ satisfies following constraints:

\begin{equation}
\begin{aligned}
if ~ \delta_{i,j}^{*}(k) &=0:\\
 & \rho_{i,j}^{*}(k)\textit{v}_{i,j}^f\textit{h}_{i,j}(k) < 1-\rho_{i,j}^{*}(k)\textit{L} \\
% & \textit{f}_{i,j}^{*}(k)=\textit{v}_{i,j}^{f}{~} \rho_{i,j}^{*}(k) \\
 & L_{i,j}\textit{h}_{i,j}(k) \geq \Delta_{t}\textit{L}\textit{n}_{i,j}^{w,*}(k)\\
 & L_{i,j}\textit{h}_{i,j}(k) < \Delta_{t}\textit{L}\textit{n}_{i,j}^{w,*}(k)+\Delta_{t}\textit{L}\\
 & \textit{h}_{i,j}^{*}(k) < \textit{h}_{i,j}(k) \leq \textit{h}_{i,j}^{max} \\
 if ~ \delta_{i,j}^{*}(k) &=1:\\
 & \textit{h}_{i,j}(k) = \textit{h}_{i,j}^{*}(k),
 \nonumber
\end{aligned}
\end{equation}
where $\delta_{i,j}^{*}(k)$, $\rho_{i,j}^{*}(k)$ and $\textit{n}_{i,j}^{w,*}(k) \in \textbf{x}^{*}$, then we have $\textbf{x}^{*} \in \Omega_{\textbf{h}}$, $TTT_{\textbf{h}}=TTT(\textbf{x}^{*})=TTT^{*}$ and $||~ \textbf{h}^{*}~ ||_{1} < ||~ \textbf{h} ||_{1}$. 

%SO-DTA could be achieved under different headway settings if  set $\left\{\delta_{i,j}(k)\mid \delta_{i,j}(k)=0 ~for~\textit{(i,j)}~and~ \textit{k}\right\}$ in one SO-DTA solution is nonempty. Mathematically, $\forall~\textbf{h}$ satisfies $\textit{h}_{i,j}^{min}\leq\textit{h}_{i,j}(k)\leq
%\textit{h}_{i,j}^{max}$ for each $for~\textit{(i,j)}~and~ \textit{k}$, if the set $\left\{\delta_{i,j}(k)\mid \delta_{i,j}(k)=0 ~for~\textit{(i,j)}~and~ \textit{k}\right\}$ for the solution of Programming (\Rmnum{1}) under $\textbf{h}$ is nonempty, then we could always find such a $\textbf{h}^{*}$ satisfying $TTT(\textbf{h}^{*})=TTT(\textbf{h})$ and $\lvert\rvert ~\textbf{h}^{*} ~\lvert\rvert > \lvert\rvert ~\textbf{h} ~ \lvert\rvert$.
\label{prop:optimal}
\end{proposition}

\begin{proof}
See \ref{app:optimal}.
\end{proof}

Proposition \ref{prop:minimum} and \ref{prop:optimal} inspire us a way to derive the maximin headway control and it mainly consists of the following two steps: 1) derive the flow dynamics $\textbf{x}^{*}$ for SO-DTA under the minimum headway $\textbf{h}^{min}$; 2) explore the ``largest'' headway setting based on $\textbf{x}^{*}$ by remaining the same TTT.

%Therefore, maximin headway control solving algorithm is proposed to derive the maximin headway control based on Proposition \ref{prop:minimum} and Proposition \ref{prop:optimal} in Algorithm \ref{algo:optimal}, where Programming (\Rmnum{2}) is to derive the values of DTA solution variables for SO-DTA under minimum headway setting and Programming (\Rmnum{3}) is to solve the maximin headway control by exploring the ``largest'' headway control but keep SO-DTA unchanged.

To this end, we propose an algorithm to solve the maximin headway control in Algorithm \ref{algo:optimal} and prove the correctness of Algorithm \ref{algo:optimal} under the condition of the existence of unique optimal flow dynamics in Proposition \ref{prop:correctness}. 
One can see from Algorithm \ref{algo:optimal} that, once the traffic conditon $\textbf{x}^{*}$ is determined, we can run a linear programming (LP) to solve for the maximin headway, and hence the solution process is computationally efficient.

%\begin{algorithm}[H]
\begin{breakablealgorithm}%[H]
\caption{Maximin Headway Control Solving Algorithm}
{\textbf{Input:} parameters in Table \ref{notation_con}}.\qquad\qquad\qquad\qquad\qquad\qquad\qquad\qquad~
\begin{algorithmic}[1]
\STATE Solve the discretized SO-DTA formulation (Equations \ref{eq:obj_dis} to \ref{eq:demand_dis}) under minimum headway $\textbf{h}^{min}$ to derive the optimal flow dynamics $\textbf{x}^{*}$ with $TTT(\textbf{x}^{*})=TTT_{\textbf{h}^{min}}=TTT^{*}$. To be specific, we can solve Formulation~\ref{eq:milp} to obtain $\textbf{x}^{*}$, and details of the formulation is presented in \ref{app:MILP}.
\begin{eqnarray}
%\intertext{This text be on the far left}
\begin{aligned}
%{Programming ~ (\Rmnum{1})} \\
    \textbf{x}^{*} =& \arg\min \limits_{\mathbf{x}} ~~  TTT =\textbf{P}^{T}\textbf{x}\\
    & \, s.t.  \quad \textbf{A}(\textbf{h}^{min})\textbf{x}=\textbf{B}(\textbf{h}^{min}) \\
    & \quad \quad ~~ \textbf{C}(\textbf{h}^{min})\textbf{x}\leq\textbf{D}(\textbf{h}^{min}) \\
\end{aligned}
\label{eq:milp}
\end{eqnarray}
\STATE Solve Formulation~\ref{eq:swap} under the  flow dynamics $\textbf{x}^{*}$ and derive the maximin headway  $\textbf{h}^{*}$, where $\delta_{i,j}^{*}(k)$, $\rho_{i,j}^{*}(k)$ and $\textit{n}_{i,j}^{w,*}(k) \in \textbf{x}^{*}$.

\begin{eqnarray}
%\intertext{This text be on the far left}
\begin{aligned}
%{Programming ~ (\Rmnum{2})} \\
    &\max \limits_{\mathbf{h}} ~~  \mathbf{1}^T~\textbf{h}\\
    & \, s.t.  \quad if ~ \delta_{i,j}^{*}(k) =0:\\
 & \quad \quad \quad \quad \rho_{i,j}^{*}(k)\textit{v}_{i,j}^f\textit{h}_{i,j}(k) < 1-\rho_{i,j}^{*}(k)\textit{L} \\
 %& \quad \quad \quad \quad \textit{f}_{i,j}^{*}(k)=\textit{v}_{i,j}^{f}{~} \rho_{i,j}^{*}(k) \\
 & \quad \quad \quad \quad L_{i,j}\textit{h}_{i,j}(k) \geq \Delta_{t}\textit{L}\textit{n}_{i,j}^{w,*}(k)\\
 & \quad \quad \quad \quad L_{i,j}\textit{h}_{i,j}(k) < \Delta_{t}\textit{L}\textit{n}_{i,j}^{w,*}(k)+\Delta_{t}\textit{L}\\
 & \quad \quad \quad \quad \textit{h}_{i,j}^{min}(k) < \textit{h}_{i,j}(k) \leq \textit{h}_{i,j}^{max} \\
 & \quad \quad ~ if ~ \delta_{i,j}^{*}(k) =1:\\
 & \quad \quad \quad \quad \textit{h}_{i,j}(k) = \textit{h}_{i,j}^{min}(k)
\end{aligned}
\label{eq:swap}
\end{eqnarray}

%TODO: output
\end{algorithmic}
{\textbf{Output:} maximin headway $\textbf{h}^{*}$ and optimal flow dynamics $\textbf{x}^{*}$ with $\textbf{x}^{*} \in \Omega_{\textbf{h}^{*}}$ and $TTT_{\textbf{h}^{*}}=TTT(\textbf{x}^{*})=TTT^{*}$}.\quad~~
\label{algo:optimal}
%\end{algorithm}
\end{breakablealgorithm}

\begin{proposition}[Correctness of Algorithm~\ref{algo:optimal}]
The output $\textbf{h}^{*}$ of Algorithm~\ref{algo:optimal} is the optimal solution of Formulation~\ref{eq:maximin} if $~\exists ~\textbf{x}^{*} \in \Phi ~\textit{s.t.}~ TTT(\textbf{x}^{*})<TTT(\textbf{x}), \forall \textbf{x}\in \Phi ~and~ \forall  \textbf{x} \neq \textbf{x}^{*}$.
\label{prop:correctness}
\end{proposition}

\begin{proof}
Proposition \ref{prop:correctness} can be directly obtained by combining Proposition \ref{prop:minimum}, \ref{prop:unique}, and \ref{prop:optimal}.
\end{proof}

%As both Programming II and Programmign III are linear programming, 

\section{Numerical experiments}
\label{sec:Numerical_Results}

The proposed model and propositions are validated on both a small and a large-scale network in this section. 
% The first network is a small network consisting of two origin nodes and one destination node. 
%We will perform the maximin headway control and sensitivity analysis-based optimal headway control solving algotithm on this small network and compare the minimum TTT solved by these two algorithms respectively to demonstrate the correctness and effectiveness of the proposed maximin headway control solving algorithm in achieving SO-DTA.
%Besides, we derive the maximin headway control and explore how the gap between maximin headway control and minimum headway setting is sensitive to different parameters, such as the value of minimum headway and travel demand. The second network is the Sioux Falls network. And We will perform the maximin headway control solving algorithm on the Sioux Falls network to derive the maximin headway control.

\subsection{A small network}

In the small network, we first solve for the optimal headway control and maximin headway control to demonstrate that the maximin headway control achieves the SO-DTA with large headway settings. Additionally, the sensitivity analysis for the maximin headway control is also conducted.

% e will perform the maximin headway control and sensitivity analysis-based optimal headway control solving algotithm on this small network and compare the minimum TTT solved by these two algorithms respectively to demonstrate the correctness and effectiveness of the proposed maximin headway control solving algorithm in achieving SO-DTA.
% Besides, we derive the maximin headway control and explore how the gap between maximin headway control and minimum headway setting is sensitive to different parameters, such as the value of minimum headway and travel demand.


\subsubsection{Settings}

The small network consists of 5 nodes and 6 links, Node 1 and 2 are connected to the origins and Node 5 is connected with the destination. 
%We add the dummy node for each origin and destination node. 
The minimum and maximum headway in each link are set as 0.5 seconds and 2.5 seconds, respectively.  We further set the total time horizon $\textit{T}$ as 90 minutes and the length of each time interval $\Delta_{t}$ as 5 minutes. The travel demand for each O-D pair is 50 $\textit{vehicles}/\textit{mins}$ and the time horizon of demand $\textit{T}_{1}$ is 40 minutes. Other parameters are illustrated in \ref{app:parameter}.

\begin{figure}[h]
    \centering
    \includegraphics[width=0.6\linewidth]{small_network}
    \caption{The small network.}
    \label{fig:network}
\end{figure}


\subsubsection{Results for optimal and maximin headway control}

% This section displays the numerical result of proposed model in a small network (Figure~\ref{fig:network}), where we have two origin nodes and one destination node. We first perform the maximin headway control solving algorithm (Algorithm \ref{algo:optimal}) to solve the maximin headway control. 

The three different scenarios are defined for comparison as follows:
\begin{itemize}
\item \textbf{Min-HW}: the headway of each link is directly set as the minimum headway. 
\item \textbf{SO-HW}: we directly solve for the SO-DTA by searching for the optimal headway, and the solution algorithm is based on the sensitivity-based  algorithm (Algorithm \ref{algo:sen}) described in \ref{app:SA}.
\item \textbf{Maximin-HW}: we solve the maximin headway by Algorithm~\ref{algo:optimal}.
\end{itemize}

Based on Proposition~\ref{prop:minimum}, we already know that the minimum headway setting would yield the SO-DTA, and the total travel time $TTT(\textbf{h}^{min})$ = 27,740. Additionally, the dynamic inflow and outflow of each link in the SO-DTA are presented in Figure~\ref{fig:Flow_in_links}.


%Solve the Programming (\Rmnum{1}) and derive the optimal SO-DTA variables $\textbf{x}^{*}$ and $TTT(\textbf{h}_{min})$ = 27,740, which means the minimum total travel time is about 27,740 minutes.

% Solve the Programming (\Rmnum{2}) and derive the maximin headway control $\textbf{h}^{*}$. The gap between the sum of maximin headway control and sum of minimum headway setting is about 1.28 minutes. On average, for each headway in maximin headway control, it is about 2.42 times as large as the value of minimum headway. 



%\begin{figure}[h]
%    \centering
%    \includegraphics[width=0.6\linewidth]{small_network}
%    \caption{Small traffic network}
%    \label{fig:network}
%  \end{figure}

% And we show the change of inflow and outflow in each link based on the  optimal SO-DTA variables $\textbf{x}^{*}$ by solving the Programming (\Rmnum{1}) in Figure~\ref{fig:Flow_in_links}.

\begin{figure}[h]
    \centering
    \includegraphics[width=0.8\linewidth]{Flow_in_links}
    \caption{Inflow and outflow of each link under SO-DTA.}
    \label{fig:Flow_in_links}
\end{figure}

The convergence curve of Algorithm~\ref{algo:sen} in \textbf{SO-HW} is shown in Figure~\ref{fig:FTTT_algorithm2}, and one the see that the algorithm converges to about $TTT=33,500$ minutes, which is larger than 27,740 derived by Algorithm \ref{algo:optimal} in \textbf{Minimum-HW}. This verifies that the \textbf{Min-HW} has lower TTT than \textbf{SO-HW}, and the sensitivity-based method might not converge to the optimal solution of SO-DTA.


% Then we perform the sensitivity analysis-based optimal headway control solving algorithm (Algorithm \ref{algo:sen}) in \ref{app:SA} to solve the minimum TTT in another way. The Figure~\ref{fig:FTTT_algorithm2} shows the value of minimum TTT in each iteration and it converges to about 33,500 minutes, which is larger than 27,740 derived by Algorithm \ref{algo:optimal}. We have proved that minimum TTT of proposed SO-DTA problem from Equations \ref{eq:obj_dis} to \ref{def:optimal_headway} could be analytically derived under minimum headway setting in Proposition \ref{prop:minimum} and numerical results also validates this finding.

\begin{figure}[h]
    \centering
    \includegraphics[width=0.7\linewidth]{TTT_algorithm2}
    \caption{Convergence curve in terms of TTT by Algorithm \ref{algo:sen}.}
    \label{fig:FTTT_algorithm2}
\end{figure}

As for the \textbf{Maximin-HW}, the TTT remains 27,740 minutes, meaning that the maximin headway still achieves the SO-DTA. More importantly, the maximin headway is about 2.42 times as large as the minimum headway on average, demonstrating the merits of the maximin headway control. Table \ref{table2} presents the average maximin headway and minimum headway of each link. One can see that the gap between the maximin and minimum headway is significant, and the gap could also reflect the safety margin of each link.


%TODO: table


\begin{table}[htbp]
	\centering   
	\begin{tabular}{c c c}  
		\hline  
		& & \\[-6pt] 
		Link & Minimum Headway (seconds) & Maximin Headway (seconds) \\  
		\hline
		& &\\[-6pt]  
		1 $\to$ 3 & 0.500 & 0.994 \\
        \hline
        & &\\[-6pt] 
        1 $\to$ 4 & 0.500 & 1.050 \\
        \hline
        & &\\[-6pt] 
        2 $\to$ 3 & 0.500 & 2.500 \\
		\hline
        & &\\[-6pt] 
        2 $\to$ 4 & 0.500 & 0.962 \\
		\hline
        & &\\[-6pt]
        3 $\to$ 5 & 0.500 & 0.944\\
		\hline
        & &\\[-6pt]
        4 $\to$ 5 & 0.500 & 0.804 \\
		\hline
	\end{tabular}
 \caption{Maximin and minimum headway in the small network.}  
	\label{table2} 
\end{table}

\subsubsection{Sensitivity analysis for the maximin headway control}
This section explores how different network configurations would effect the average gap between the maximin headway control and minimum headway control and it could be formulated as $\frac{||\textbf{h}^{*}-\textbf{h}^{min}||_{1}}{n_{L}N}$, where $\textbf{h}^{*}$ is the maximin headway solved by Algorithm \ref{algo:optimal} and $n_{L}$ is the number of links in set $E\setminus(L_R\cup L_S)$. In particular, we study the sensitivity regarding the travel demand and minimum headway.

{\bf Travel demand.} We first conduct the sensitivity analysis regarding travel demand under different minimum headway requirements. We vary the demand of each O-D pair from 30 to 70 under different minimum headway requirements, then we solve  Algorithm \ref{algo:optimal} and derive the value of the average gap between the maximin headway and minimum headway, and the results are shown in  Figure \ref{fig:Gap_demand}. 

\begin{figure}[h]
    \centering
    \includegraphics[width=0.7\linewidth]{Gap_demand}
    \caption{The average gap between the maximin and minimum headway under different traffic demand.}
    \label{fig:Gap_demand}
\end{figure}

One can see that the average gap between maximin and minimum headway settings decreases when the demand increases, indicating that the safety margin for SO-DTA will become marginal when more people travel.

% As is shown in Figure \ref{fig:Gap_demand},  there also exists a general but not strict relationship that the average  gap value would decrease when minimum headway value increases.

%As is shown in Figure \ref{fig:Gap_demand}, there exists a general but not strict relationship that the average gap value would decrease when travel demand increases.


{\bf Minimum headway.} 
Then the sensitivity analysis for the minimum headway value is conducted. We vary the value of minimum headway from 0.20s to 1.10s under different O-D demand and keep other parameters unchanged. Then we solve the Algorithm \ref{algo:optimal} and derive the value of the average gap between the maximin headway control and minimum headway setting under different minimum headway values. Figure \ref{fig:Gap_minimum_headway} shows how the average gap between the maximin headway control and minimum headway setting would change corresponding to different minimum headway values. 

\begin{figure}[H]
    \centering
    \includegraphics[width=0.7\linewidth]{Gap_minimum_headway}
    \caption{The average gap between the maximin and minimum headway under different minimum headway settings.}
    \label{fig:Gap_minimum_headway}
\end{figure}

One can see that in general, the average headway gap between maximin and minimum headway is relatively stabilized under different minimum headway, indicating that there is always room to enlarge the headway after achieving SO-DTA under different safety requirements ({\em i.e}, minimum headway).




%\clearpage
\subsection{Sioux Falls network}
In this section, we solve for the maximin headway control on the Sioux Falls network.
\subsubsection{Settings}
The Sioux Falls network consists of 24 nodes and 76 links, and the detailed network settings, such as link length and capacity, can be referred to
%There are totally 16 O-D pairs and we add dummy node for each origin and destination node. We set that there are totally 100 vehicles for each O-D pair. 
% For other parameters setting in Sioux Falls Network, such as link length and capacity.
%one can refer to 
the website\footnote{\href{https://github.com/bstabler/TransportationNetworks/tree/master/SiouxFalls}{https://github.com/bstabler/TransportationNetworks/tree/master/SiouxFalls}}. 
Besides, we further define that the minimum and maximum headway in each link are set as 0.5 seconds and 2.5 seconds, respectively.
%Then we test the proposed model in the Sioux Falls Network and use the origin and destination nodes setting in \cite{SiouxFalls}. As is shown in Figure \ref{fig:Large_Network}, the Sioux Falls Network consists of 24 nodes and 76 links. There are totally 4 origin nodes and 4 destination nodes. We focus on the extended Sioux Falls with dummy links and nodes. For other parameters setting in Sioux Falls Network, such as link length and capacity, one can refer to the website \footnote{\href{https://github.com/bstabler/TransportationNetworks/tree/master/SiouxFalls}{https://github.com/bstabler/TransportationNetworks/tree/master/SiouxFalls}}.
%Besides, the minimum and maximin headway are respectively set as 0.5 seconds and 2.5 seconds, which are about 0.00833 and 0.04167 minutes. We perform the maximin headway control solving algorithm (Algorithm \ref{algo:optimal}) to solve the maximin headway.

\begin{figure}[h]
    \centering
    \includegraphics[width=0.4\linewidth]{Sioux_Falls.png}
    \caption{Overview of the Sioux Falls network.}
    \label{fig:Large_Network}
\end{figure}


\subsubsection{Experimental results}

%TODO: Running time
% The experiments were performed on a desktop with 32GB RAM and an AMD Ryzen 7 5700G processor, and the running time is about 2 hours 36 minutes. 
We consider \textbf{Min-HW} and \textbf{Maximin-HW} by setting the minimum headway directly and  running Algorithm \ref{algo:optimal}, respectively. \textbf{SO-HW} is omitted as we have already proved that  \textbf{Min-HW} is equivalent to \textbf{SO-HW}. Details of \textbf{Min-HW} and \textbf{Maximin-HW} are presented as follows:

\begin{itemize}
\item \textbf{Min-HW}: we directly set the minimum headway for each link, and we have  $TTT(\textbf{h}_{min})$ = 23,674 minutes.
\item \textbf{Maximin-HW}: The $TTT$ is the same as that of the \textbf{Min-HW}, and the average gap between the maximin headway and the minimum headway is 1.045 seconds. On average, each headway setting in maximin headway control is about 3.09 times as large as the minimum headway. 
%\item Solve the Programming (\Rmnum{1}) under the maximin headway control $\textbf{h}^{*}$ and find that $TTT(\textbf{h}^{*})=TTT(\textbf{h}_{min})$ = 23,674, which shows that SO-DTA could still be achieved under maximin headway control and safety could be guaranteed compared with the minimum headway setting.
\end{itemize}



%\clearpage



\section{Conclusion}
\label{sec:Conclusion}

In this paper, we propose a maximin headway control framework for SO-DTA on general networks in a fully automated environment. 
Both HFD and HDQ are developed to capture of effects of dynamic AV headway control. Particularly, the HDQ is proved to be a generalized version of the DQ. 
% We define the pure AV fundamental diagram with a dynamic headway control as headway-dependent fundamental diagram (HFD), where headway would affect the fundamental diagram of pure AV in these aspects: scope of free-flow region, upper bound of flow and shockwave speed. In order to capture effects of dynamic AV headway control on dynamic network loading (DNL), we propose the headway-dependent double queue model (HDQ) where headway control would effect both downstream and upstream queue by HFD. 
% Then we show HDQ is a generalized form of double queue model (DQ) and the properties of HDQ are discussed.
It is rigorously proved that the minimum headway could achieve SO-DTA. On top of this, the maximin headway of AVs is innovatively defined as the ``largest'' headway setting that still achieves SO-DTA. 
% We define optimal headway control as the headway setting where SO-DTA could be achieved and maximin headway control as the ``largest'' headway setting to achieve SO-DTA for both safety and SO-DTA concerns. We discuss the similarities and differences between them. For SO-DTA concern, SO-DTA problem may have multiple solutions, which means SO-DTA could be achieved under different headway settings. It is difficult to directly solve an nonlinear programming to derive the minimum total travel time (TTT) and optimal headway control.
% But we theoretically prove that SO-DTA could be achieved under minimum headway setting, which provides us an effective and direct way to achieve SO-DTA and derive actual minimum TTT. For safety concern, 
We further propose an efficient solution algorithm to derive the maximin headway. Numerical experiments demonstrate the correctness and effectiveness of the proposed framework. 
% We conduct the sensitivity analysis to explore  how the travel demand and minimum headway constraint would respectively affect the average gap between the maximin headway control and minimum headway setting. 
Sensitivity analysis demonstrates that the increase in travel demand  would normally correspond to a decrease in gap values between the maximin and minimum headway, which helps the policymakers understand the safety margin of AVs on different links. We believe this study provides novel angles and mathematical techniques to derive and analyze the ``desired solution'' among the non-unique solutions in SO-DTA.
% and the gap is relatively stabilized under different minimum headway.

% TODO safety margin

Future studies can be conducted to consider mixed traffic conditions, heterogeneous travel behaviors, and stochastic traffic demand and road supply. 
For example, this study is based on the HFD and HDQ in a fully automated environment so that we could control the headway of all AVs for SO-DTA. However, in mixed traffic conditions, the route choice of human-driving vehicles (HVs) may follow dynamic user equilibrium (DUE), making the problem more complex \citep{ZHANG201875}. It is also interesting to explore the headway control on a limited number of roads or with limited control capacities \citep{battifarano2023impact}. Additionally, quantifying the relationship between headway and safety is critical to the practical deployment of the developed headway control framework in the real world. 
% These AVs may act like ``coordinators'' and further studies could focus on the headway control strategy to minimize total travel time in a mixed traffic conditions, where HVs are supposed to follow DUE.


\section*{Acknowledgments}
The work described in this paper was supported by the National Natural Science Foundation of China (No. 52102385),  a grant from the Research Grants Council of the Hong Kong Special Administrative Region, China (Project No. PolyU/25209221), and grants from the Otto Poon Charitable Foundation Smart Cities Research Institute (SCRI) at the Hong Kong Polytechnic University (Project No. P0043552 \& P0036472).

\bibliography{ref}

\clearpage

\appendix
  % \renewcommand{\appendixname}{Appendix}

  \section{Notations}
  \label{app:notations}
  
  \begin{table}[H]
    \centering
    \resizebox{1.02\textwidth}{!}{
    \begin{tabular}{lll}
        \toprule
        Type      & Notation   & Description \\
        \midrule
        Sets      & \textit{V} & Set of nodes \\
                  & \textit{R} & Set of origins \\
                  & $\widetilde{R}$ & Set of dummy origins \\
                  & \textit{S} & Set of destinations\\
                  & $\widetilde{S}$ & Set of dummy destinations \\
                  & \textit{E} & Set of links \\
                  & $L_R$ & Set of origin connectors \\%dummy origin links \\    
                  & $L_S$ & Set of destination connectors \\%dummy destination links \\
        Parameters & $\textit{v}_{i,j}^f$ & Free-flow speed of link $(i,j)$ \\
                  & $L_{i,j}$ & Length of link $(i,j)$ \\
                  & \textit{L} & Length of an AV \\
                  & $\bar{Q}_{i,j}^{\mathcal{U}}$ & Upstream queue capacity of link $(i,j)$ \\
                  & $\bar{Q}_{i,j}^{\mathcal{D}}$ & Downstream queue capacity of link $(i,j)$ \\
                  & $\bar{C}_{i,j}^{\mu}$ & Inflow capacity of link $(i,j)$ \\ 
                  & $\bar{C}_{i,j}^{\nu}$ & Outflow capacity of link $(i,j)$ \\
                  & $\textit{T}_\textit{1}$ & Time horizon of demand \\
                  & \textit{T} & Total time horizon \\
                  & $\textit{h}_{i,j}^{min}$ & Lower bound of headway in link $(i,j)$ \\
                  & $\textit{h}_{i,j}^{max}$ & Upper bound of headway in link $(i,j)$ \\
        Variables & $\textit{u}_{i,j}(t)$ & Inflow of the flow area of link $(i,j)$ at time $t$ \\
                  & $\textit{u}_{i,j}^{s^{\prime}}(t)$ & Inflow of the flow area of link $(i,j)$ with destination $s^{\prime}$ at time $t$ \\
                  & $\textit{v}_{i,j}(t)$ & Outflow of the buffer area of link $(i,j)$ at time $t$ \\
                  & $\textit{v}_{i,j}^{s^{\prime}}(t)$ & Outflow of the buffer area of link $(i,j)$ with destination $s^{\prime}$ at time $t$ \\
                  & $\textit{f}_{i,j}(t)$ & Inflow at the boundary of the flow area and  buffer area of link $(i,j)$ at time $t$ \\
                  & $\textit{f}_{i,j}^{~ s^{\prime}}(t)$ & Inflow at the boundary of the flow area and  buffer area of link $(i,j)$ with destination $s^{\prime}$ at time $t$ \\
                  & $\rho_{i,j}(t)$ & Density of the flow area of link $(i,j)$ at time $t$ \\
                  & $\rho_{i,j}^{s^{\prime}}(t)$ & Density of the flow area of link $(i,j)$ with destination $s^{\prime}$ at time $t$ \\
                  & $\textit{q}_{i,j}^{\mathcal{D}}(t)$ & Downstream queue of link $(i,j)$ at time $t$ \\
                  & $\textit{q}_{i,j}^{\mathcal{D},s^{\prime}}(t)$ & Downstream queue of link $(i,j)$ with destination $s^{\prime}$ at time $t$ \\
                  & $\textit{q}_{i,j}^{\mathcal{U}}(t)$ & Upstream queue of link $(i,j)$ at time $t$ \\
                  & $\textit{q}_{i,j}^{\mathcal{U},s^{\prime}}(t)$ & Upstream queue of link $(i,j)$ with destination $s^{\prime}$ at time $t$ \\
                  & $\textit{h}_{i,j}(t)$ & Headway of automated vehicles in link $(i,j)$ at time $t$ \\
                  & $\tau_{i,j}^{w}(t)$ & Shockwave travel time on link $(i,j)$ at time $t$ \\
                  & $\textit{d}_{r^{\prime},s^{\prime}}(t)$ & Travel demand of O-D pair  $(r^{\prime},s^{\prime})$ at time $t$ \\
                  & $\textit{U}_{i,j}(t)$ & Cumulative inflow of the flow area of link $(i,j)$ at time $t$ \\
                  & $\textit{F}_{i,j}(t)$ & Cumulative inflow at the boundary of the flow area and buffer area of link $(i,j)$ at time $t$ \\
                  & $\textit{V}_{i,j}(t)$ & Cumulative outflow of the buffer area of link $(i,j)$ at time $t$ \\
        \bottomrule
    \end{tabular}
    }
    \caption{Notation table.}
    \label{notation_con}
\end{table}

  
  \section{Proof of Proposition~\ref{prop:HDQ&DQ}}
  \label{app:generality}
Proposition \ref{prop:HDQ&DQ} indicates that the headway-dependent double queue (HDQ) model is a generalized form of the double queue (DQ) model.
The formulation of the downstream queue $\textit{q}_{i,j}^{d}(t)$ and upstream queue $\textit{q}_{i,j}^{u}(t)$ in DQ for $ (i,j)\in E$ and $t\in[0,T]$ are shown as follows:
\begin{eqnarray}
\textit{q}_{i,j}^{d}(t) &=& \int_{0}^{\textit{t}-\tau_{i,j}^{f}} \textit{u}_{i,j}(t) ~dt - \int_{0}^{\textit{t}} \textit{v}_{i,j}(t) ~dt + \textit{q}_{i,j}^{d}(0) ~ = ~ \textit{U}_{i,j}(t-\tau_{i,j}^{f})-\textit{V}_{i,j}(t),
\label{eq:down_DQ} 
\nonumber\\
\textit{q}_{i,j}^{u}(t) &=& \int_{0}^{\textit{t}} \textit{u}_{i,j}(t) ~dt - \int_{0}^{\textit{t}-\tau_{i,j}^{w}} \textit{v}_{i,j}(t) ~dt + \textit{q}_{i,j}^{u}(0)  ~ = ~
\textit{U}_{i,j}(t) - \textit{V}_{i,j}(t-\tau_{i,j}^{w}),
\label{eq:up_DQ}
\nonumber
\end{eqnarray}
 where $\textit{q}_{i,j}^{d}(t)$ and $\textit{q}_{i,j}^{u}(t)$ represent the downstream and upstream queue in link $(i,j)$ at time $t$ in DQ, respectively, and the two queues are interpreted as follows:

\begin{itemize}
    \item Downstream queue $\textit{q}_{i,j}^{d}(t)$ measures the number of vehicles at the end of link $(i,j)$ at time $t$. DQ assumes the inflow $u_{i,j}$ would travel from the entrance to the end of link $(i,j)$ at the free-flow speed, and the outflow of link $(i,j)$ at time $t$ is determined by the outflow capacity link $(i,j)$ and upstream queue in next link.
    \item Upstream queue $\textit{q}_{i,j}^{u}(t)$ measures the number of vehicles at the entrance of link $(i,j)$ at time $t$. DQ assumes the space of outflow $v_{i,j}$ would travel from the end to the entrance of link $(i,j)$ at the shockwave speed. 
\end{itemize}

However, the assumptions of DQ model may have following limitations:

\begin{itemize}
    \item Congestion may occur in the flow area of link so that AVs may not travel from the entrance to the end of link at the free-flow speed. The influence of headway on the flow of link is also not considered in DQ.
    \item Space may not propagate to the entrance of link in time and retain in the downstream. Take an extreme example, if the front AV leave the link but other AVs behind are still, the space left by the front AV would still retain in the downstream and it is not accurate to use the outflow to calculate the propagation of space in shockwave speed. Besides, the influence of headway on the shockwave speed on link is also not considered in DQ. 
\end{itemize}

The HDQ addresses the above limitations by exploring the effects of headway. Equation \ref{eq: down_HDQ} and \ref{eq: up_HDQ} formulate the downstream and upstream queue in HDQ. The relationship between the HDQ and DQ are presented in Equation \ref{eq:down_DQ_HDQ} and \ref{eq:up_DQ_HDQ}.

  \begin{eqnarray}
\textit{q}_{i,j}^{d}(t) &=&  \textit{U}_{i,j}(t-\tau_{i,j}^{f})-\textit{V}_{i,j}(t) \nonumber \\
&=& \textit{U}_{i,j}(t-\tau_{i,j}^{f})-\textit{F}_{i,j}(t)+\textit{F}_{i,j}(t)
-\textit{V}_{i,j}(t) \nonumber \\
&=& \textit{q}_{i,j}^{\mathcal{D}}(t)+\textit{U}_{i,j}(t-\tau_{i,j}^{f})-\textit{F}_{i,j}(t) \label{eq:down_DQ_HDQ} \\
\textit{q}_{i,j}^{u}(t) &=& 
\textit{U}_{i,j}(t) - \textit{V}_{i,j}(t-\tau_{i,j}^{w}) \nonumber \\
&=& \textit{U}_{i,j}(t) - \textit{F}_{i,j}(t-\tau_{i,j}^{w}) + \textit{F}_{i,j}(t-\tau_{i,j}^{w}) - \textit{V}_{i,j}(t-\tau_{i,j}^{w}) \nonumber \\
&=& \textit{q}_{i,j}^{\mathcal{U}}(t)+\textit{q}_{i,j}^{\mathcal{D}}(t-\tau_{i,j}^{w})
\label{eq:up_DQ_HDQ}
\end{eqnarray}

In Equation \ref{eq:down_DQ_HDQ}, $\textit{U}_{i,j}(t-\tau_{i,j}^{f})-\textit{F}_{i,j}(t)$ represents the congestion in the flow area in HDQ. However, under the assumption of DQ, we have $\textit{u}_{i,j}(t-\tau_{i,j}^{f}) =\textit{f}_{i,j}(t)$, meaning that vehicles would travel from the entrance of link to the boundary of the flow area and buffer area at free-flow speed. Therefore, the downstream queue of HDQ is equivalent to the DQ as follows:

\begin{equation}
\textit{q}_{i,j}^{d}(t) = \textit{q}_{i,j}^{\mathcal{D}}(t)+\textit{U}_{i,j}(t-\tau_{i,j}^{f})-\textit{F}_{i,j}(t) = \textit{q}_{i,j}^{\mathcal{D}}(t).
    \nonumber
\end{equation}

In Equation \ref{eq:up_DQ_HDQ}, the upstream queue in DQ actually consists of the queue in both flow area and buffer area in HDQ. The upstream queue in HDQ actually consider the flow entering buffer area to fill the leaving space. If we ignore it, we would derive the upstream queue by outflow $v$ as DQ.

Different from DQ, the HDQ specifies the traffic state in a link and incorporates the effect of headway. To calculate the number of AVs at the entrance and end of a link, we just focus on the flow area and buffer area, respectively. Compared with the DQ, the HDQ provides a more precise way to calculate the downstream and upstream queue if we are clear about the state of boundary of the flow area and buffer area compared with the DQ. If under the assumption of DQ, the effect of headway is ignored and HDQ would be reduced to DQ.  

%Different from DQ, HDQ specifies the traffic state in a link and incorporates the effect of headway. If we know the flow at the boundary of the flow area and buffer area, we could use the buffer area to calculate the number of vehicles at the end of link as $\textit{q}_{i,j}^{\mathcal{D}}(t)$ and use flow area to calculate the number of vehicles at the entrance of link as $\textit{q}_{i,j}^{\mathcal{U}}(t)$. As Equations \ref{eq:down_DQ_HDQ} and \ref{eq:up_DQ_HDQ} show, downstream and upstream queues in DQ consist of the congestion in flow area and buffer area. HDQ provides a precise way to calculate the downstream and upstream queue if we are clear about the state of boundary of the flow area and buffer area. Therefore, HDQ is a generalized form of DQ as they both measure the number of vehicles at the entrance and end of vehicles as upstream and downstream queues.

%Generally, as Equation \ref{eq:down_DQ_HDQ} shows, downstream queue in DQ representing the number of vehicles at the end of link consists of vehicles traveling from the entrance of link to the boundary in flow area in HDQ and vehicles in buffer area in HDQ. If     

%In DQ, the upstream queue $\textit{q}_{i,j}^{d}(t)$ measures the number of vehicles at the entrance of link by assuming the space caused by outflow $v_{i,j}(t)$ would travel from the entrance of link to the end of link in shockwave speed. 

%Different with DQ, we divide a link into flow area and buffer area in HDQ. Equation \ref{eq:up_DQ_HDQ} presents that the upstream queue in DQ consists of the upstream queue in flow area and downstream queue in buffer area in HDQ.  

%In DQ, the upstream queue $\textit{q}_{i,j}^{i}(t)$ is always larger than 
%the downstream queue $\textit{q}_{i,j}^{d}(t)$ as follows.

%\begin{eqnarray}
%\textit{q}_{i,j}^{u}(t)-\textit{q}_{i,j}^{d}(t) &=& 
%\textit{U}_{i,j}(t) - \textit{V}_{i,j}(t-\tau_{i,j}^{w}) - %(\textit{U}_{i,j}(t-\tau_{i,j}^{f})-\textit{V}_{i,j}(t)) 
%\nonumber \\
%&=& \textit{U}_{i,j}(t)- \textit{U}_{i,j}(t-\tau_{i,j}^{f}) + \textit{V}_{i,j}(t)- \textit{V}_{i,j}(t-\tau_{i,j}^{w}) 
%\nonumber \\
%&\geq& 0
%\nonumber
%\end{eqnarray}

%Therefore, in DQ, we have $\textit{q}_{i,j}^{d}(t) \leq \bar{Q}_{i,j}^{d} \leq \textit{q}_{i,j}^{u}(t) \leq \bar{Q}_{i,j}^{u}$, where $\bar{Q}_{i,j}^{d}$ and  $\bar{Q}_{i,j}^{u}$ are downstream and upstream queue capacity in link $(i,j)$. For the upstream queue in HDQ, we mainly focus on the upstream queue in flow area $\textit{q}_{i,j}^{\mathcal{U}}(t)$ and 

%For downstream queue in DQ, we measure the number of vehicles at the end of link by assuming the flow would travel from the entrance of link to the end of link in free-flow state. Equation \ref{eq:down_DQ_HDQ} shows that the downstream queue in DQ is comprised of two parts: 
%$\textit{q}_{i,j}^{\mathcal{D}}(t)$ and $\textit{U}_{i,j}(t-\tau_{i,j}^{f})-\textit{F}_{i,j}(t)$. For the first part, $\textit{q}_{i,j}^{\mathcal{D}}(t)$ is the downstream queue of link $(i,j)$ in HDQ at time $t$. For the second part, if we still assume the flow would travel from the entrance of link to the end of link in free-flow state, we would have $\textit{U}_{i,j}(t-\tau_{i,j}^{f})-\textit{F}_{i,j}(t)=0$, then $\textit{q}_{i,j}^{\mathcal{D}}(t)=\textit{q}_{i,j}^{\mathcal{D}}(t)$. In this paper, HDQ makes it possible to consider a generalized case of how the flow propagates along the link: either free-flow state or congested state in a dynamic fundamental diagram. As we already know the flow entering the buffer area by HFD, so we only consider the vehicles in buffer area as downstream queue in HDQ. 


%For upstream queue in DQ, we measure the number of vehicles at the entrance of link by assuming the shockwave would travel from the end of link to the entrance of link in congested state. Equation \ref{eq:down_DQ_HDQ} shows that the downstream queue in DQ $\textit{q}_{i,j}^{\mathcal{U}}(t)$ is comprised of two parts:
%$\textit{q}_{i,j}^{\mathcal{U}}(t)$ and $\textit{q}_{i,j}^{\mathcal{D}}(t-%\tau_{i,j}^{w})$. For the first part, $\textit{q}_{i,j}^{\mathcal{U}}(t)$ is the upstream queue of HDQ at time $t$.
%For the second part, %$\textit{q}_{i,j}^{\mathcal{D}}(t-\tau_{i,j}^{w})$ is the downstream queue of link $(i,j)$ in HDQ at time $t-\tau_{i,j}^{w}$. If we don't add the buffer area in a link, we would have %$\textit{q}_{i,j}^{\mathcal{D}}(t-\tau_{i,j}^{w})=0$, then $\textit{q}_{i,j}^{\mathcal{U}}(t)=\textit{q}_{i,j}^{\mathcal{U}}(t)$. Similarly, as we know the flow leaving the flow area by HFD, so we only consider the vehicles in flow area as upstream queue in HDQ. 

  \section{Proof of Proposition~\ref{prop:property}}
  \label{app:HDQ}
  This appendix proves the properties of the headway-dependent double queue (HDQ) model, and we mainly show the constraints on the flow rates of the HDQ.
  
  With Eqs.(4.29) in \cite{LTM} and the formulation of the downstream queue in HDQ, Equation \ref{app:eq_send_HDQ} presents the constraint on the sending flow $\textit{S}_{i,j}(t)$ of HDQ.
  
  \begin{equation}
   \textit{S}_{i,j}(t)\leq \textit{V}_{i,j}(t+\Delta_{t})-\textit{V}_{i,j}(t) \leq  \textit{F}_{i,j}(t+\Delta_{t})-\textit{V}_{i,j}(t)
  \label{app:eq_send_HDQ}
  \end{equation}
  
  With Eqs.(4.33) in \cite{LTM} and the formulation of the upstream queue in HDQ, the constraint on the receiving flow $\textit{R}_{i,j}(t)$ of the HDQ is stated in Equation \ref{app:eq_receive_HDQ}.
  
  \begin{equation}
  \textit{R}_{i,j}(t) \leq \textit{U}_{i,j}(t+\Delta_{t})-\textit{U}_{i,j}(t) \leq  \bar{Q}_{i,j}^{u}+\textit{F}_{i,j}(t+\Delta_{t}-\tau_{i,j}^{w}(t))-\textit{U}_{i,j}(t)
  \label{app:eq_receive_HDQ}
  \end{equation}
  
  When the time step $\Delta_{t}$ approaches zero, the constraint of outflow $\textit{v}$ is shown in \ref{app:v_con} and the constraint of inflow $\textit{u}$ is shown in \ref{app:u_con}.
  
  \begin{eqnarray}
\textit{v}_{i,j}(t) ~\leq~ \lim_{\Delta_{t} \to 0} \frac{\textit{S}_{i,j}(t)} {\Delta_{t}} 
&\leq& \lim_{\Delta_{t} \to 0} \min \left\{ \frac{\textit{F}_{i,j}(t+\Delta_{t})-\textit{V}_{i,j}(t)}{\Delta_{t}},~ \bar{C}_{i,j}^{\nu} \right\} \nonumber \\ 
&=& \lim_{\Delta_{t} \to 0} \min \left\{ \frac{\textit{F}_{i,j}(t+\Delta_{t})-\textit{F}_{i,j}(t)}{\Delta_{t}}+  \frac{\textit{F}_{i,j}(t)-\textit{V}_{i,j}(t)}{\Delta_{t}}, ~ \bar{C}_{i,j}^{\nu} \right\} \nonumber \\
&=& \lim_{\Delta_{t} \to 0} \min \left\{ \frac{\textit{F}_{i,j}(t+\Delta_{t})-\textit{F}_{i,j}(t)}{\Delta_{t}}+  \frac{\textit{q}_{i,j}^{\mathcal{D}}(t)}{\Delta_{t}}, ~ \bar{C}_{i,j}^{\nu} \right\} \nonumber \\
&=& \left\{ \begin{array}{lcl}
\bar{C}_{i,j}^{\nu},  &  & \textit{q}_{i,j}^{\mathcal{D}}(t)>0 \\
\min \left\{ \textit{f}_{i,j}(t),~ \bar{C}_{i,j}^{\nu} \right\}, &  & \textit{q}_{i,j}^{\mathcal{D}}(t)=0
\end{array} \right. \label{app:v_con}\\
\textit{u}_{i,j}(t) ~\leq~ \lim_{\Delta_{t} \to 0} \frac{\textit{R}_{i,j}(t)} {\Delta_{t}} 
&\leq& \lim_{\Delta_{t} \to 0} \min \left\{ \frac{\bar{Q}_{i,j}^{u}+\textit{F}_{i,j}(t+\Delta_{t}-\tau_{i,j}^{w}(t))-\textit{V}_{i,j}(t)}{\Delta_{t}},~ \bar{C}_{i,j}^{\mu} \right\} \nonumber \\
&=& \lim_{\Delta_{t} \to 0} \min \left\{ \frac{\bar{Q}_{i,j}^{u}-\left(\textit{V}_{i,j}(t)-\textit{F}_{i,j}(t-\tau_{i,j}^{w}(t))\right)}{\Delta_{t}}+  \frac{\textit{F}_{i,j}(t+\Delta_{t}-\tau_{i,j}^{w}(t))-\textit{F}_{i,j}(t-\tau_{i,j}^{w}(t))}{\Delta_{t}}, ~ \bar{C}_{i,j}^{\mu} \right\} \nonumber \\
&=& \lim_{\Delta_{t} \to 0} \min \left\{ \frac{\bar{Q}_{i,j}^{u}-\textit{q}_{i,j}^{\mathcal{U}}(t)}{\Delta_{t}}+  \frac{\textit{F}_{i,j}(t+\Delta_{t}-\tau_{i,j}^{w}(t))-\textit{F}_{i,j}(t-\tau_{i,j}^{w}(t))}{\Delta_{t}}, ~ \bar{C}_{i,j}^{\mu} \right\} \nonumber \\
&=& \left\{ \begin{array}{lcl}
\bar{C}_{i,j}^{\mu},  &  & \textit{q}_{i,j}^{\mathcal{U}}(t)<\bar{Q}_{i,j}^{u} \\
\min \left\{ \textit{f}_{i,j}(t-\tau_{i,j}^{w}(t)),~ \bar{C}_{i,j}^{\mu} \right\}, &  & \textit{q}_{i,j}^{\mathcal{U}}(t)=\bar{Q}_{i,j}^{u}
\end{array} \right. \label{app:u_con}
\end{eqnarray}
  
  \section{Proof of Proposition~\ref{prop:feasibility}}
  \label{app:feas}
  In general, we could construct a feasible solution to Equation~\ref{eq:obj_con} using the following three steps:
  
\begin{itemize} 
\item Let all the demand of each O-D pair arrive at the downstream queue in the origin connectors while not releasing them. We define the time period for such a process as  $\textit{T}_\textit{1}$. We don't consider the capacity of number of AVs waiting at origins here. 
%TODOSuppose the demand does not exceed the downstream queue capacity, otherwise we would release the flow as following steps once reaching the downstream queue capacity. The proof of this case is similar so we won't discuss this case here. 

\item For each O-D pair $(r^{\prime},s^{\prime})$, we divide the total demand $D^{r^{\prime},s^{\prime}}$ into many equal batches $B^{r^{\prime},s^{\prime}}$, and $D^{r^{\prime},s^{\prime}}=n^{r^{\prime},s^{\prime}} B^{r^{\prime},s^{\prime}}$. We only use one path of each O-D pair $(r^{\prime},s^{\prime})$, which is denoted as $P^{r^{\prime},s^{\prime}}$. We only release one batch of an O-D pair only once. After all the AVs in this batch arrive at the destination, we will release another batch of this O-D pair. Repeat this process until finishing all batches of an O-D pair and continue another O-D pair.

%\item We only release one batch of an O-D pair only once. After all the AVs in this batch arrive at the destination, we will release another batch of this O-D pair. Repeat this process until finishing all batches of an O-D pair and continue another O-D pair.

\item In each batch, only when all the AVs in this batch arrive at the downstream queue in each link, we will release them into the next link in the selected path. The time of passing link $(i,j)$ for batch $B^{r^{\prime},s^{\prime}}$ is denoted by $T_{i,j}^{r^{\prime},s^{\prime}}$. %Therefore, the total  time consumed is $\textit{T}_\textit{1}+\sum\limits_{(r^{\prime},s^{\prime})}n^{r^{\prime},s^{\prime}}\sum\limits_{(i,j)~in~P^{r^{\prime},s^{\prime}}}T_{i,j}^{r^{\prime},s^{\prime}}$.
\end{itemize}

Therefore, the total time consumed by above three steps is $\textit{T}_\textit{1}+\sum\limits_{(r^{\prime},s^{\prime})}n^{r^{\prime},s^{\prime}}\sum\limits_{(i,j)~in~P^{r^{\prime},s^{\prime}}}T_{i,j}^{r^{\prime},s^{\prime}}$. If we choose the time horizon $T$ that is larger than $\textit{T}_\textit{1}+\sum\limits_{(r^{\prime},s^{\prime})}n^{r^{\prime},s^{\prime}}\sum\limits_{(i,j)~in~P^{r^{\prime},s^{\prime}}}T_{i,j}^{r^{\prime},s^{\prime}}$, we could construct a solution so that the problem is feasible. To this end, we present the details of how to construct the flow as follows.

%Therefore, if we choose the time horizon $T$ that is larger than $\textit{T}_\textit{1}+\sum\limits_{(r^{\prime},s^{\prime})}n^{r^{\prime},s^{\prime}}\sum\limits_{(i,j)~in~P^{r^{\prime},s^{\prime}}}T_{i,j}^{r^{\prime},s^{\prime}}$, we could construct a solution so that the problem is feasible. 
%To this end, we present the details of how to construct the flow as follows.

In order to satisfy the queue capacity constraint in  Equation \ref{eq:ineq_1_con}, the size of batch should be constrained as follows:

\begin{equation}
    B^{r^{\prime},s^{\prime}}<\mathop{min}\limits_{(i,j)~in~P^{r^{\prime},s^{\prime}}}\left\{\bar{Q}_{i,j}^{\mathcal{U}},\bar{Q}_{i,j}^{\mathcal{D}}\right\}.
    \nonumber
\end{equation}

%In order to satisfy the queue capacity, we should control the size of batch so that $B^{r^{\prime},s^{\prime}}<\mathop{min}\limits_{(i,j)~in~P^{r^{\prime},s^{\prime}}}\left\{\bar{Q}_{i,j}^{\mathcal{U}},\bar{Q}_{i,j}^{\mathcal{D}}\right\}$.
After determining the size of each batch, the next step is to determine the flow rate of each batch. Suppose we release the batch at a constant rate, so $B^{r^{\prime},s^{\prime}}=T_{B}^{r^{\prime},s^{\prime}}~F_{B}^{r^{\prime},s^{\prime}}$, where $T_{B}^{r^{\prime},s^{\prime}}$ is the time of releasing all batches once and $F_{B}^{r^{\prime},s^{\prime}}$ is the flow rate of releasing all batches once. In order to satisfy the inflow and outflow capacity constraints in Equation \ref{eq:ineq_2_con}, we keep free-flow state rather than congested state in the flow area in each link and we desire the flow rate $F_{B}^{r^{\prime},s^{\prime}}$ wouldn not be too small to make sure the density in this period is larger than $\frac{1}{L_{i,j}}$ , the flow rate $F_{B}^{r^{\prime},s^{\prime}}$ should be constrained as follows:

\begin{equation}
    \mathop{max}\limits_{(i,j)~in~P^{r^{\prime},s^{\prime}}}\left\{\frac{\textit{v}_{i,j}^f}{L_{i,j}}\right\}\leq F_{B}^{r^{\prime},s^{\prime}}<\mathop{min}\limits_{(i,j)~in~P^{r^{\prime},s^{\prime}}}\left\{\bar{C}_{i,j}^{\mu},\bar{C}_{i,j}^{\nu},\frac{\textit{v}_{i,j}^f}{\textit{h}^{max}_{i,j}(t)\textit{v}_{i,j}^f+\emph{L}}\right\}. 
    \nonumber
\end{equation}

Normally, the flow capacity is always larger than $\frac{\textit{v}_{i,j}^f}{L_{i,j}}$. 
It is meaningless to discuss the maximin headway control under the case that the flow capacity is extremely small that we only allow a small number of vehicle to enter or exit the link. Otherwise, when releasing all AVs in this batch, the density in link $(i,j)$ is already smaller than $\frac{1}{L_{i,j}}$, which already satisfies the end constraint. So in this case $T_{i,j}^{r^{\prime},s^{\prime}}=T_{B}^{r^{\prime},s^{\prime}}$. 

%In this case, there is no obvious difference of total travel time under different headway settings. Therefore, we won't discuss this case.

Then we calculate $T_{i,j}^{r^{\prime},s^{\prime}}$, which represents the time of passing link $(i,j)$ for batch $B^{r^{\prime},s^{\prime}}$ and $T_{i,j}^{r^{\prime},s^{\prime}}$ is comprised of three time periods:

\begin{itemize} 
\item Only $\textit{u}_{i,j}(t)$ exists. This time period means that the first vehicle entering the link $(i,j)$ will drive in a free-flow speed $\textit{v}_{i,j}^{f}$ and the time of this period is $\tau_{i,j}^{f}=\frac{L_{i,j}}{\textit{v}_{i,j}^f}$. And the density of link $(i,j)$ at time $\tau_{i,j}^{f}$ is $\rho_{i,j}(\tau_{i,j}^{f})=\frac{F_{B}^{r^{\prime},s^{\prime}}}{L_{i,j}}~\frac{L_{i,j}}{\textit{v}_{i,j}^{f}}=\frac{F_{B}^{r^{\prime},s^{\prime}}}{\textit{v}_{i,j}^{f}}$.
\item Both $\textit{u}_{i,j}(t)$ and $\textit{v}_{i,j}(t)$ exist. This time period means that we still release the batch from prior link in this batch to link $(i,j)$ and also exists flow $\textit{f}_{i,j}(t)$ which represents the inflow at the boundary between the flow area and buffer area of link $(i,j)$. Besides, $\dot{\rho}_{i,j}(\tau_{i,j}^{f})=\frac{\textit{u}_{i,j}(\tau_{i,j}^{f})-\textit{f}_{i,j}(\tau_{i,j}^{f})}{L_{i,j}}=\frac{F_{B}^{r^{\prime},s^{\prime}}-\textit{v}_{i,j}^{f}~\frac{F_{B}^{r^{\prime},s^{\prime}}}{\textit{v}_{i,j}^{f}}}{L_{i,j}}=0$. So when $\tau_{i,j}^{f}\leq t \leq T_{B}^{r^{\prime},s^{\prime}}$, we have $u_{i,j}(t)=f_{i,j}(t)=F_{B}^{r^{\prime},s^{\prime}}$ and $\rho_{i,j}(t)=\frac{F_{B}^{r^{\prime},s^{\prime}}}{L_{i,j}}$.
\item Only $\textit{v}_{i,j}(t)$ exists. This time period means that there is no releasing flow of batch from prior link and the end constraint is $\rho_{i,j}(T)<\frac{1}{L_{i,j}}$. So we solve the following equations and derive $T_{i,j}^{r^{\prime},s^{\prime}}=T_{B}^{r^{\prime},s^{\prime}}+\frac{L_{i,j}}{\textit{v}_{i,j}^{f}}~\ln{\frac{L_{i,j}~F_{B}^{r^{\prime},s^{\prime}}}{\textit{v}_{i,j}^{f}}}$.
\end{itemize}

\begin{equation}
\left\{
\begin{array}{lr}
\dot{\rho}_{i,j}(t)=\frac{\textit{u}_{i,j}(t)-\textit{f}_{i,j}(t}{L_{i,j}}= \frac{-\textit{v}_{i,j}^{f}~\rho_{i,j}(t)}{L_{i,j}} \\
\rho_{i,j}(T_{B}^{r^{\prime},s^{\prime}})=\frac{F_{B}^{r^{\prime},s^{\prime}}}{L_{i,j}} \\
\rho_{i,j}(T_{i,j}^{r^{\prime},s^{\prime}})=\frac{1}{L_{i,j}}
\end{array} \right.
\nonumber
\end{equation}

It is obvious that other constraints are satisfied by constructing the solution based on above procedures. Therefore, if we choose the time horizon $T$ so that $T$ is larger than $\textit{T}_\textit{1}+\sum\limits_{(r^{\prime},s^{\prime})}n^{r^{\prime},s^{\prime}}\sum\limits_{(i,j)}T_{i,j}^{r^{\prime},s^{\prime}}\delta_{i,j}^{P^{r^{\prime},s^{\prime}}}$, where $\delta_{i,j}^{P^{r^{\prime},s^{\prime}}}$ indicates whether link $(i,j)$ is in the path $P^{r^{\prime},s^{\prime}}$, then we could always find a feasible solution of proposed problem.

  \section{Sensitivity analysis of the headway control for SO-DTA}
  \label{app:Sensitivity_Analysis}
  
  This section proposes a sensitivity-based algorithm to solve the SO-DTA problem. We first transform the discretized version of the headway-dependent SO-DTA formulation into mixed integer linear programming (MILP)  given an exogenous headway variable $\textbf{h}$, as shown in \ref{app:MILP}, then the sensitivity-based algorithm is presented. 
  
 \subsection{MILP formulation}
 \label{app:MILP}

 This section transforms the discretized version of SO-DTA problem formulation from Equations \ref{eq:obj_dis} to \ref{ineq:headway_dis} into MILP. Although we transform it as MILP, it is easy to be solved by CPLEX and the $\delta^{*}$ in solution of MILP benefits the analysis of traffic dynamics and maximin headway. Besides, we are also able to transform the problem from Equations \ref{eq:obj_dis} to \ref{ineq:headway_dis} into linear programming (LP) if we regard the piecewise function in Equations \ref{eq:free_dis} and \ref{eq:cong_dis} as the minimum value of two strightlines. 
 
 The objective of section \ref{sec_system_optimal_headway_formulation} is to find the optimal headway $\left\{\textit{h}_{i,j}(k)\right\}$. In the proposed programming, the involved variables include $\left\{\rho_{i,j}(k)\right\}$, 
$\left\{\rho_{i,j}^{s^{\prime}}(k)\right\}$, 
$\left\{\textit{q}_{i,j}^{\mathcal{D}}(k)\right\}$,
$\left\{\textit{q}_{i,j}^{\mathcal{D},s^{\prime}}(k)\right\}$, $\left\{\textit{q}_{i,j}^{\mathcal{U}}(k)\right\}$, $\left\{\textit{q}_{i,j}^{\mathcal{U},s^{\prime}}(k)\right\}$,
$\left\{\textit{u}_{i,j}(k)\right\}$,
$\left\{\textit{u}_{i,j}^{s^{\prime}}(k)\right\}$,
$\left\{\textit{f}_{i,j}(k)\right\}$,
$\left\{\textit{f}_{i,j}^{~s^{\prime}}(k)\right\}$,
$\left\{\textit{v}_{i,j}(k)\right\}$,
$\left\{\textit{v}_{i,j}^{s^{\prime}}(k)\right\}$,
$\left\{\textit{n}_{i,j}(k)\right\}$,
$\left\{\textit{h}_{i,j}(k)\right\}$ and
$\left\{\delta_{i,j}(k)\right\}$, where binary variables $\left\{\delta_{i,j}(k)\right\}$ is used to indicate whether the flow area of link $(i,j)$ in the $k$th time interval  is congested or not.
However, the proposed problem is hard to solve due to the nonlinearity in Equations \ref{eq:free_dis} to \ref{eq:cong_dis} and \ref{eq:shockwave_dis1} to \ref{eq:shockwave_dis2}.

We note that if the headway $\left\{\textit{h}_{i,j}(k)\right\}$ is fixed, then the variables are  $\left\{\rho_{i,j}(k)\right\}$, 
$\left\{\rho_{i,j}^{s^{\prime}}(k)\right\}$, 
$\left\{\textit{q}_{i,j}^{\mathcal{D}}(k)\right\}$, $\left\{\textit{q}_{i,j}^{\mathcal{D},s^{\prime}}(k)\right\}$, $\left\{\textit{q}_{i,j}^{\mathcal{U}}(k)\right\}$, $\left\{\textit{q}_{i,j}^{\mathcal{U},s^{\prime}}(k)\right\}$,
$\left\{\textit{u}_{i,j}(k)\right\}$,
$\left\{\textit{u}_{i,j}^{s^{\prime}}(k)\right\}$,
$\left\{\textit{f}_{i,j}(k)\right\}$,
$\left\{\textit{f}_{i,j}^{~s^{\prime}}(k)\right\}$,
$\left\{\textit{v}_{i,j}(k)\right\}$,
$\left\{\textit{v}_{i,j}^{s^{\prime}}(k)\right\}$,
$\left\{\textit{n}_{i,j}(k)\right\}$ and
$\left\{\delta_{i,j}(k)\right\}$. Hence the proposed formulation becomes a mixed integer linear programming program (MILP).

For simplicity in notations, we use a generalized form of mixed integer linear programming to represent the formulation, as shown in Equation~\ref{eq:MILP_obj}.

%\begin{eqnarray}
%\intertext{This text be on the far left}
%\begin{aligned}
%{Programming ~ (\Rmnum{1})} \\
%     &\min \limits_{\mathbf{x}} ~~  TTT = \textbf{P}^{T}\textbf{y}\\
%    & \, s.t.  \quad \textbf{A}(\textbf{h})\textbf{X}  =  \textbf{B}(\textbf{h}) \\
%    & \quad \quad ~~ \textbf{C}(\textbf{h})\textbf{X} \leq \textbf{D}(\textbf{h})\\
%\end{aligned}
%\nonumber
%\end{eqnarray}

\begin{eqnarray}
     &\min \limits_{\mathbf{x}} ~~  TTT = \textbf{P}^{T}\textbf{x} 
     \label{eq:MILP_obj}\\
    & \, s.t.  \quad \textbf{A}(\textbf{h})\textbf{x}  =  \textbf{B}(\textbf{h}) \nonumber \\
    & \quad \quad ~~ \textbf{C}(\textbf{h})\textbf{x} \leq \textbf{D}(\textbf{h}) \nonumber
\end{eqnarray}

where we have:

\begin{itemize} 
\item $\textbf{x}=\left\{\rho_{i,j}(k),~
\rho_{i,j}^{s^{\prime}}(k),~
\textit{q}_{i,j}^{\mathcal{D}}(k),~ \textit{q}_{i,j}^{\mathcal{D},s^{\prime}}(k),~
\textit{q}_{i,j}^{\mathcal{U}}(k),~ \textit{q}_{i,j}^{\mathcal{U},s^{\prime}}(k),~
\textit{u}_{i,j}(k),~
\textit{u}_{i,j}^{s^{\prime}}(k),~
\textit{f}_{i,j}(k),~
\textit{f}_{i,j}^{~s^{\prime}}(k),~
\textit{v}_{i,j}(k),~
\textit{v}_{i,j}^{s^{\prime}}(k),~
\textit{n}_{i,j}(k),~
\delta_{i,j}(k) \right\}$ for $(\textit{i}.\textit{j})\in \textit{E}, ~ 1\leq{k}\leq\textit{N};
\textit{s}^{\prime} \in \widetilde{S}$, and the dimension of variables $\textbf{x}$ is $n$.

\item $\textbf{h}=\left\{\textit{h}_{\emph{i},~\emph{j}}(k) \mid (\textit{i},\textit{j})\in E\setminus(L_R\cup L_S);~1\leq{k}\leq\textit{N}\right\} \in\Re^{m}$, and the dimension of parameters $\textbf{h}$ is $m$.
\item $\textbf{A}(\textbf{h})=
\left[
\begin{matrix}
a_{1}(\textbf{h})  \\
\vdots \\
a_{l_{1}}(\textbf{h})\\
\end{matrix}
\right]=
\left[
\begin{matrix}
a_{11}(\textbf{h})  & \cdots & a_{1n}(\textbf{h}) \\
\vdots  & \ddots & \vdots \\
a_{l_{1}1}(\textbf{h})  & \cdots & a_{l_{1}n}(\textbf{h})\\
\end{matrix}
\right] \in\Re^{l_{1}\times n}; \quad
\textbf{B}(\textbf{h})=
\left[
\begin{matrix}
b_{1}(\textbf{h})  \\
\vdots \\
b_{l_{1}}(\textbf{h})\\
\end{matrix}
\right] \in\Re^{l_{1}}
$. $l_{1}$ represents the number of equality constraints. 
\item $\textbf{C}(\textbf{h})=
\left[
\begin{matrix}
c_{1}(\textbf{h})  \\
\vdots \\
c_{l_{2}}(\textbf{h})\\
\end{matrix}
\right]=
\left[
\begin{matrix}
c_{11}(\textbf{h})  & \cdots & a_{1n}(\textbf{h}) \\
\vdots  & \ddots & \vdots \\
c_{l_{2}1}(\textbf{h})  & \cdots & c_{l_{2}n}(\textbf{h})\\
\end{matrix}
\right] \in\Re^{l_{2}\times n}; \quad
\textbf{D}(\textbf{h})=
\left[
\begin{matrix}
d_{1}(\textbf{h})  \\
\vdots \\
d_{l_{2}}(\textbf{h})\\
\end{matrix}
\right] \in\Re^{l_{2}}
$. $l_{2}$ represents the number of inequality constraints. 
\end{itemize}


\subsection{Sensitivity-based optimal headway control solution algorithm}
\label{app:SA}
  We then present the sensitivity analysis for the developed MILP,
  %to derive the minimum TTT and optimal headway control. 
  and the focus is to derive the gradient of TTT with respect to the headway.
The Lagrange multiplier of Equation~\ref{eq:MILP_obj} is shown in Equation \ref{eq:KKT_obj} and the corresponding Karush–Kuhn–Tucker (KKT) conditions are presented from Equations \ref{eq:KKT_1} to \ref{eq:KKT_5}.


%TODO
\begin{equation}
\min_\mathbf{x} \quad\textit{\textbf{L}}=\textbf{P}^{T}\textbf{x}+\bm{\lambda}^{T}\left(\textbf{A}(\textbf{h})\textbf{x}-\textbf{B}(\textbf{h})\right)+\bm{\mu}^{T}\left(\textbf{C}(\textbf{h})\textbf{x}-\textbf{D}(\textbf{h})\right)
\label{eq:KKT_obj}
\end{equation}

\begin{equation}
\textbf{P}^{T}+{\bm{\lambda^{*}}}^{T}\textbf{A}(\textbf{h})+
{\bm{\mu^{*}}}^{T}\textbf{C}(\textbf{h})=\bm{0}
\label{eq:KKT_1}
\end{equation}

\begin{equation}
\textbf{A}(\textbf{h})\textbf{x}^{*}=\textbf{B}(\textbf{h})
\label{eq:KKT_2}
\end{equation}

\begin{equation}
\textbf{C}(\textbf{h})\textbf{x}^{*}\leq\textbf{D}(\textbf{h})
\label{eq:KKT_3}
\end{equation}

\begin{equation}
{\bm{\mu^{*}}_{i}}\left(\textbf{C}(\textbf{h})\textbf{x}^{*}-\textbf{D}(\textbf{h})\right)_{i}=\bm{0}
\label{eq:KKT_4}
\end{equation}

\begin{equation}
\bm{\mu^{*}}\geq\bm{0}
\label{eq:KKT_5}
\end{equation}

where,

\begin{itemize} 
\item $\bm{\lambda^{*}}\in\Re^{l_{1}}$ and $\bm{\mu^{*}}\in\Re^{l_{2}}$. 
\item  $\textit{K}=\left\{i \mid \mu_{\textit{i}}^{*}>0 ; 1\leq\textit{i}\leq{l_{2}}\right\}$ is the set of index of all positive 
$\mu_{\textit{i}}^{*}$. And the size of $\textit{K}$ is $\textit{k}$. If i $\in \textit{K}$, then $\mu_{\textit{i}}^{*}>0$.
\end{itemize} 

Then we do the differential to the Equation \ref{eq:MILP_obj}  as $\mathbf{z}=\textbf{P}^{T}\textbf{X}$  and Equations \ref{eq:KKT_1} to \ref{eq:KKT_4}.

\begin{equation}
\textbf{P}^{T}d\textbf{x}-d\textbf{z}=0
\label{eq:KKT_dif_1}
\end{equation}

\begin{equation}
\left(\sum_{i=1}^{l_{1}}\lambda_{\textit{i}}^{*}\nabla_{\textbf{h}}a_{i}(\textbf{h})+
\sum_{j=1}^{l_{2}}\mu_{\textit{j}}^{*}\nabla_{\textbf{h}}c_{j}(\textbf{h})\right)d\bm{\textbf{h}}+\left(\textbf{A}(\textbf{h})\right)^{T}d\bm{\lambda}+\left(\textbf{C}(\textbf{h})\right)^{T}d\bm{\mu}=\bm{0}
\label{eq:KKT_dif_2}
\end{equation}

\begin{equation}
\textbf{A}(\textbf{h})d\textbf{x}+\left[\nabla_{\textbf{h}}\left(\textbf{A}(\textbf{h})\textbf{x}^{*}\right)-\nabla_{\textbf{h}}\textbf{B}(\textbf{h})\right]d\bm{\textbf{h}}=\bm{0}
\label{eq:KKT_dif_3}
\end{equation}

\begin{equation}
\textbf{C}_{\textbf{k}}(\textbf{h})d\textbf{x}+\left[\nabla_{\textbf{h}}\left(\textbf{C}_{\textbf{k}}(\textbf{h})\textbf{x}^{*}\right)-\nabla_{\textbf{h}}\textbf{D}_{\textbf{k}}(\textbf{h})\right]d\bm{\textbf{h}}=\bm{0}
\label{eq:KKT_dif_4}
\end{equation}

where,

\begin{itemize} 
\item $\nabla_{\textbf{h}}a_{i}(\textbf{h})=\nabla_{\textbf{h}}\left[
\begin{matrix}
a_{i1}(\textbf{h}) & \cdots &  a_{in}(\textbf{h})\\
\end{matrix}
\right]=\left[
\begin{matrix}
\frac{\partial a_{i1}(\textbf{h})}{\partial \textbf{h}_{1}} & \cdots &  \frac{\partial a_{i1}(\textbf{h})}{\partial \textbf{h}_{m}}\\
\vdots  & \ddots & \vdots \\
\frac{\partial a_{in}(\textbf{h})}{\partial \textbf{h}_{1}} & \cdots &  \frac{\partial a_{in}(\textbf{h})}{\partial \textbf{h}_{m}}\\
\end{matrix}\right] \in\Re^{n\times m}$
\item $\nabla_{\textbf{h}}c_{j}(\textbf{h})=\nabla_{\textbf{h}}\left[
\begin{matrix}
c_{j1}(\textbf{h}) & \cdots &  c_{jn}(\textbf{h})\\
\end{matrix}
\right]=\left[
\begin{matrix}
\frac{\partial c_{j1}(\textbf{h})}{\partial \textbf{h}_{1}} & \cdots &  \frac{\partial c_{j1}(\textbf{h})}{\partial \textbf{h}_{m}}\\
\vdots  & \ddots & \vdots \\
\frac{\partial c_{jn}(\textbf{h})}{\partial \textbf{h}_{1}} & \cdots &  \frac{\partial c_{jn}(\textbf{h})}{\partial \textbf{h}_{m}}\\
\end{matrix}\right] \in\Re^{n\times m}$
\item $\textbf{A}^{x}=\textbf{A}(\textbf{h})\textbf{x}^{*}=\left[
\begin{matrix}
a^{x}_{1}(\textit{x}^{*},\textbf{h}) \\
\vdots \\
a^{x}_{l_{1}}(\textit{x}^{*},\textbf{h})\\
\end{matrix}\right]\in\Re^{l_{1}}$;
$\nabla_{\textbf{h}}\textbf{A}^{x}=\left[
\begin{matrix}
\frac{\partial a^{x}_{1}(\textit{x}^{*},\textbf{h})}{\partial \textbf{h}_{1}} & \cdots &  \frac{\partial a^{x}_{1}(\textit{x}^{*},\textbf{h})}{\partial \textbf{h}_{m}}\\
\vdots  & \ddots & \vdots \\
\frac{\partial a^{x}_{l_{1}}(\textit{x}^{*},\textbf{h})}{\partial \textbf{h}_{1}} & \cdots &  \frac{\partial a^{x}_{l_{1}}(\textit{x}^{*},\textbf{h})}{\partial \textbf{h}_{m}}\\
\end{matrix}\right]\in\Re^{l_{1}\times m}$
\item $\textbf{C}_{\textbf{k}}(\textbf{h})=\left[
\begin{matrix}
\vdots  \\
c_{j}(\textbf{h})  \\
\vdots\\
\end{matrix}
\right]_{j \in \textit{K}} \in\Re^{k\times n}$; ~
$\textbf{C}_{\textbf{k}}^{x}=\textbf{C}_{\textbf{k}}(\textbf{h})\textbf{x}^{*}= \left[ 
\begin{matrix}
\vdots  \\
c_{j}^{x}(\textit{x}^{*},\textbf{h})  \\
\vdots\\
\end{matrix}
\right]_{j \in \textit{K}} \in\Re^{k}$; ~
$\textbf{D}_{\textbf{k}}(\textbf{h})=\left[
\begin{matrix}
\vdots  \\
d_{j}(\textbf{h})  \\
\vdots\\
\end{matrix}
\right]_{j \in \textit{K}} \in\Re^{k}$
\item $\nabla_{\textbf{h}}\textbf{C}_{\textbf{k}}^{x}=\left[
\begin{matrix}
\vdots & \cdots & \vdots \\
\frac{\partial c_{j}^{x}(\textit{x}^{*},\textbf{h}))}{\partial \textbf{h}_{1}} & \cdots & \frac{\partial c_{j}^{x}(\textit{x}^{*},\textbf{h}))}{\partial \textbf{h}_{m}} \\
\vdots & \cdots & \vdots \\
\end{matrix}
\right]_{j \in \textit{K}} \in\Re^{k\times m}$
\end{itemize}

We cast Equations \ref{eq:KKT_dif_1} to \ref{eq:KKT_dif_4} into the form of block matrices.

\begin{equation}
\left[
\begin{matrix}
\textbf{P}^{T} & \bm{0} & \bm{0} & -1 \\ 
\bm{0} & \left(\textbf{A}(\textbf{h})\right)^{T} & \left(\textbf{C}(\textbf{h})\right)^{T} & \bm{0} \\ 
\textbf{A}(\textbf{h}) & \bm{0} & \bm{0} & \bm{0} \\
\textbf{C}_{\textbf{k}}(\textbf{h}) & \bm{0} & \bm{0} & \bm{0} \\
\end{matrix}
\right] ~
\left[
\begin{matrix}
d\textbf{x} \\ 
d\bm{\lambda} \\ 
d\bm{\mu} \\
d\textbf{z} \\
\end{matrix}
\right]\quad = \quad -
\left[
\begin{matrix}
\bm{0} \\ 
\sum_{i=1}^{l_{1}}\lambda_{\textit{i}}^{*}\nabla_{\textbf{h}}a_{i}(\textbf{h})+
\sum_{j=1}^{l_{2}}\mu_{\textit{j}}^{*}\nabla_{\textbf{h}}c_{j}(\textbf{h}) \\ 
\nabla_{\textbf{h}}\left(\textbf{A}(\textbf{h})\textbf{x}^{*}\right)-\nabla_{\textbf{h}}\textbf{B}(\textbf{h}) \\
\nabla_{\textbf{h}}\left(\textbf{C}_{\textbf{k}}(\textbf{h})\textbf{x}^{*}\right)-\nabla_{\textbf{h}}\textbf{D}_{\textbf{k}}(\textbf{h}) \\
\end{matrix}
\right] ~ d\bm{\textbf{h}}
\label{matrix:1}
\end{equation}

We further define 
\begin{equation} \textbf{Q}=
\left[
\begin{matrix}
\textbf{P}^{T} & \bm{0} & \bm{0} & -1 \\ 
\bm{0} & \left(\textbf{A}(\textbf{h})\right)^{T} & \left(\textbf{C}(\textbf{h})\right)^{T} & \bm{0} \\ 
\textbf{A}(\textbf{h}) & \bm{0} & \bm{0} & \bm{0} \\
\textbf{C}_{\textbf{k}}(\textbf{h}) & \bm{0} & \bm{0} & \bm{0} \\
\end{matrix}
\right].
\label{matrix:2}
\end{equation}

The dimension of matrix $\textbf{Q}$ is $\left(n+l_{1}+k+1\right)\times\left(n+l_{1}+l_{2}+1\right)$. If there exists $\mu_{\textit{i}}^{*}=0$, then the matrix $\textbf{Q}$ is not a square matrix and hence not invertible. So we calculate the generalized inverse matrix of $\textbf{Q}$ and denote it as $\textbf{Q}^{-1}$ \citep{penrose1955generalized}.

\begin{equation}
\left[
\begin{matrix}
d\textbf{x} \\ 
d\bm{\lambda} \\ 
d\bm{\mu} \\
d\textbf{z} \\
\end{matrix}
\right]\quad = \quad - \textbf{Q}^{-1}
\left[
\begin{matrix}
\bm{0} \\ 
\sum_{i=1}^{l_{1}}\lambda_{\textit{i}}^{*}\nabla_{\textbf{h}}a_{i}(\textbf{h})+
\sum_{j=1}^{l_{2}}\mu_{\textit{j}}^{*}\nabla_{\textbf{h}}c_{j}(\textbf{h}) \\ 
\nabla_{\textbf{h}}\left(\textbf{A}(\textbf{h})\textbf{x}^{*}\right)-\nabla_{\textbf{h}}\textbf{B}(\textbf{h}) \\
\nabla_{\textbf{h}}\left(\textbf{C}_{\textbf{k}}(\textbf{h})\textbf{x}^{*}\right)-\nabla_{\textbf{h}}\textbf{D}_{\textbf{k}}(\textbf{h}) \\
\end{matrix}
\right] ~ d\bm{\textbf{h}} 
\label{matrix:3}
\end{equation}

Then we replace $d\bm{\textbf{h}}$ with the identity matrix $\textit{I}_{m}$ and derive the gradient $\frac{\partial \textbf{z}}{\partial \textbf{h}}$.

\begin{equation}
\left[
\begin{matrix}
\frac{\partial \textbf{x}}{\partial \textbf{h}}\vspace{1ex} \\ 
\frac{\partial \bm{\lambda}}{\partial \textbf{h}} \vspace{1ex} \\ 
\frac{\partial \bm{\mu}}{\partial \textbf{h}} \vspace{1ex}\\
\frac{\partial \textbf{z}}{\partial \textbf{h}} \\
\end{matrix}
\right] \quad = \quad - \textbf{Q}^{-1}
\left[
\begin{matrix}
\bm{0} \\ 
\sum_{i=1}^{l_{1}}\lambda_{\textit{i}}^{*}\nabla_{\textbf{h}}a_{i}(\textbf{h})+
\sum_{j=1}^{l_{2}}\mu_{\textit{j}}^{*}\nabla_{\textbf{h}}c_{j}(\textbf{h}) \\ 
\nabla_{\textbf{h}}\left(\textbf{A}(\textbf{h})\textbf{x}^{*}\right)-\nabla_{\textbf{h}}\textbf{B}(\textbf{h}) \\
\nabla_{\textbf{h}}\left(\textbf{C}_{\textbf{k}}(\textbf{h})\textbf{x}^{*}\right)-\nabla_{\textbf{h}}\textbf{D}_{\textbf{k}}(\textbf{h}) \\
\end{matrix}
\right] 
\label{matrix:4}
\end{equation}


Therefore, a sensitivity-based optimal headway control algorithm is proposed based on exploring the gradient descent of TTT, as shown in Algorithm \ref{algo:sen}.

\begin{algorithm}[H]
\caption{Sensitivity-based Optimal Headway Solving Algorithm}
{\textbf{Input:}  learning rate $\eta$}.

{\textbf{Output:} time-dependent and link-specific headway $\left\{\textit{h}_{\emph{i},\,\emph{j}}(k) \right\}$}.

Initialize the headway $\textbf{h}$.
\begin{algorithmic}[1]
\FOR{$(\textit{i}=0; \textit{i}<\textit{I}; ++\textit{i})$}
\STATE Solve the Formulation~\ref{eq:MILP_obj} given the current headway setting $\textbf{h}$ and derive $\textbf{z}^{*}, {\lambda}^{*}, {\mu}^{*}$ and $\textbf{x}^{*}$.
\STATE Conduct the sensitivity analysis and derive $ \frac{\partial \textbf{z}^{*}}{\partial \textbf{h}} $.
\STATE Update the headway by $\textbf{h}=\textbf{h}-\eta\frac{\partial \textbf{z}^{*}}{\partial \textbf{h}} $.
\ENDFOR
\end{algorithmic}
\textbf{Return:} optimal headway $\textbf{h}^{*}$.
\label{algo:sen}
\end{algorithm}

  
  \section{Proof of Proposition \ref{prop:minimum}}
  \label{app:minimum}
  
  %We first present an equivalent form of the objective function. The objective function represents the TTT, which equals to the sum of travel time on road networks and waiting time at origins. 

%\begin{equation}
%\begin{aligned}
%\textit{TTT} &=  \sum_{r^{\prime}}\sum_{\textit{k}=1}^{\textit{N}}\Delta_{t}\left(\sum_{\textit{l}=1}^{\textit{k}}\Delta_{t}\textit{v}_{r^{\prime},r}(l)\right)-\sum_{s^{\prime}}\sum_{\textit{k}=1}^{\textit{N}}\Delta_{t}\left(\sum_{\textit{l}=1}^{\textit{k}}\Delta_{t}\textit{u}_{s,s^{\prime}}(l)\right)+\sum_{(r^{\prime},s^{\prime})}\sum_{\textit{k}=1}^{\textit{N}}\Delta_{t}\textit{q}_{r^{\prime},\textit{r}}^{d,s^{\prime}}(k) \\
%&= \sum_{r^{\prime}}\sum_{\textit{k}=1}^{\textit{N}}\Delta_{t}\left(\sum_{\textit{l}=1}^{\textit{k}}\Delta_{t}\textit{f}_{r^{\prime},r}(l)\right)-\sum_{s^{\prime}}\sum_{\textit{k}=1}^{\textit{N}}\Delta_{t}\left(\sum_{\textit{l}=1}^{\textit{k}}\Delta_{t}\textit{u}_{s,s^{\prime}}(l)\right) \\
%&= (\Delta_{t})^{2}\sum_{r^{\prime}}\sum_{\textit{k}=1}^{\textit{N}}(N+1-k)~\textit{f}_{r^{\prime},r}(k)-(\Delta_{t})^{2}\sum_{s^{\prime}}\sum_{\textit{k}=1}^{\textit{N}}(N+1-k)~\textit{u}_{s,s^{\prime}}(k) \\
%&= (\Delta_{t})^{2}~(N+1)\left[\sum_{r^{\prime}}\sum_{\textit{k}=1}^{\textit{N}}\textit{f}_{r^{\prime},r}(k)-\sum_{s^{\prime}}\sum_{\textit{k}=1}^{\textit{N}}\textit{u}_{s,s^{\prime}}(k)\right]-(\Delta_{t})^{2}~\sum_{r^{\prime}}\sum_{\textit{k}=1}^{\textit{N}}k~\textit{f}_{r^{\prime},r}(k)+(\Delta_{t})^{2}~\sum_{s^{\prime}}\sum_{\textit{k}=1}^{\textit{N}}k~\textit{u}_{s,s^{\prime}}(k)
%\end{aligned}
%\end{equation}

%At the end, by Equation \ref{ineq:end_dis}, we have the equation \ref{eq:fu}. Although the number of AVs in each link should be less than 1 due to limitation of ordinary differential equation as shown in \ref{sec:Autonomous_vehicle_Headway_Control}, we could approximately regard that all vehicles would arrive in their own destinations at the end.

%\begin{equation}
%\sum_{r^{\prime}}\sum_{\textit{k}=1}^{\textit{N}}\textit{f}_{r^{\prime},r}(k)=\sum_{s^{\prime}}\sum_{\textit{k}=1}^{\textit{N}}\textit{u}_{s,s^{\prime}}(k)
%\label{eq:fu}
%\end{equation}

%And the second term could be converted as following by equation \ref{eq:demand_dis}:

%\begin{equation}
%\sum_{r^{\prime}}\sum_{\textit{k}=1}^{\textit{N}}k~\textit{f}_{r^{\prime},r}(k) = \sum_{r^{\prime}}\sum_{s^{\prime}}\sum_{\textit{k}=1}^{\textit{N}}k~\textit{d}_{r^{\prime},s^{\prime}}(k)
%\end{equation}

%Besides, $\textit{d}_{r^{\prime},s^{\prime}}(k)$ equals to the given travel demand of O-D pair  $(r^{\prime},s^{\prime})$ in $\emph{k}_{th}$ time interval. Therefore minimizing TTT is equivalent to minimizing $\sum_{s^{\prime}}\sum_{\textit{k}=1}^{\textit{N}}k~\textit{u}_{s,s^{\prime}}(k)$.

%\begin{equation}
%min~ TTT \Longleftrightarrow min~ \sum_{s^{\prime}}\sum_{\textit{k}=1}^{\textit{N}}k~\textit{u}_{s,s^{\prime}}(k)\nonumber
%\end{equation}

%It means that SO-DTA encourages to push flow to arrive their own destinations as early as possible, which is equivalent to minimizing TTT. 



%Next, in order to prove the proposition that the SO-DTA could be achieved under the minimum headway setting, the idea is that we first prove that if we only change a smaller $h_{i,j}(k)$ but keep other headway unchanged, the minimum total travel time under new headway setting is less than or equal to the minimum total travel time under original headway setting. So it could be presented as follows and we denote is Statement 1. 

%To prove the Proposition \ref{prop:minimum}, we first propose the Condition \ref{cond:1} and Proposition \ref{prop:ycl} provides us a simple way to prove Proposition \ref{prop:minimum} by presenting that Condition \ref{cond:1} is always satisfied. 

%As in Definition \ref{def:optimal_headway}, for headway $\textbf{h}=\left\{\textit{h}_{\emph{i},~\emph{j}}(k) \mid (\textit{i}.\textit{j})\in \textit{R};~1\leq{k}\leq\textit{N}\right\}$, suppose 
%Let
%$\textbf{h}=\left\{\textit{h}_{\emph{i},~\emph{j}}(k) \mid (\textit{i}.\textit{j})\in \textit{R};~1\leq{k}\leq\textit{N}\right\} \in\Re^{m}$ denote the headway setting in all links and time intervals and $TTT(\textbf{h})$ denote the minimum total travel time under the headway setting $\textbf{h}$. So it could be presented as following:

%\begin{condition}
%For $\textbf{h}_{1}=\left\{\textit{h}_{i,j}^{(1)}(k) \right\}$ and $\textbf{h}_{2}=\left\{\textit{h}_{i,j}^{(2)}(k) \right\}$, if there exist only one $(i,j)$ and $k$ such that $\textit{h}_{i,j}^{(1)}(k) leq \textit{h}_{i,j}^{(2)}(k)$ and $\textit{h}_{m,n}^{(1)}(t) = \textit{h}_{m,n}^{(2)}(t)$ for other links $(m,n)\neq(i,j)$ or time intervals $t\neq k$, then we have $TTT_{\textbf{h}_{1}}\leq TTT_{\textbf{h}_{2}}$.
%\label{cond:1}
%\end{condition}

%\begin{proposition}
%If Condition \ref{cond:1} is satisfied, Proposition \ref{prop:minimum} holds.
%\label{prop:ycl}
%\end{proposition}

%\begin{proof}
%If Condition \ref{cond:1} is satisfied, we just need to traverse all links $(i,j)$ and time interval $k$, it is obvious that we have $TTT_{\textbf{h}_{min}}\leq TTT_{\textbf{h}}$  for $\forall \textbf{h}$ satisfies $h_{i,j}^{min}\leq h_{i,j}(k)\leq h_{i,j}^{max}$.
%\end{proof}

%\begin{itemize}
%\item 
%For $\textbf{h}_{1}=\left\{\textit{h}_{i,j}^{(1)}(k) \right\}$ and $\textbf{h}_{2}=\left\{\textit{h}_{i,j}^{(2)}(k) \right\}$, if there exist only one $(i,j)$ and $k$ such that $\textit{h}_{i,j}^{(1)}(k) < \textit{h}_{i,j}^{(2)}(k)$ and $\textit{h}_{m,n}^{(1)}(t) = \textit{h}_{m,n}^{(2)}(t)$ for other links $(m,n)\neq(i,j)$ or time intervals $t\neq k$, then we have $TTT_{\textbf{h}_{1}}<TTT_{\textbf{h}_{2}}$

%If the item $h_{i,k}(k)$ in $\textbf{h}_{1}$ is less than it in $\textbf{h}_{2}$ and all other items in $\textbf{h}_{1}$ are equal to those in $\textbf{h}_{2}$, then $TTT(\textbf{h}_{1})\leq TTT(\textbf{h}_{2})$.
%\end{itemize} 

%Under the headway setting $\textbf{h}_{2}$, we solve the MILP in \ref{app:MILP} and derive the optimal solution, then we first prove the following statement:

We first present Lemma \ref{lemma:1} to show that a smaller headway $h_{i,j}(k)$ could generate a larger flow $f_{i,j}(k)$. 

\begin{lemma}
Given $u_{i,j}(k)$ and $\rho_{i,j}(k-1)$, if $h_{i,j}^{(1)}(k)\leq h_{i,j}^{(2)}(k)$, then $f_{i,j}^{(1)}(k)\geq f_{i,j}^{(2)}(k)$.
\label{lemma:1}
\end{lemma}


%\begin{itemize}
%\item Given certain $u_{i,j}(k)$ and $\rho_{i,j}(k-1)$, if $h_{i,j}^{(1)}(k)\leq h_{i,j}^{(2)}(k)$, then $f_{i,j}^{(1)}(k)\geq f_{i,j}^{(2)}(k)$, where $f_{i,j}^{(1)}(k)$ and $f_{i,j}^{(2)}(k)$ represent the inflow of the boundary of the flow area and buffer area of link (i,j) in the $kth$ time interval under headway setting $h_{i,j}^{(1)}(k)$ and $h_{i,j}^{(2)}(k)$, respectively.
%\end{itemize} 

\begin{proof}
We solve  $f_{i,j}(k)$ and $\rho_{i,j}(k)$ under given $u_{i,j}(k)$ and $\rho_{i,j}(k-1)$ in both free-flow and congested states:

\begin{itemize}
\item Free-flow state:

\begin{equation}
\left\{
\begin{array}{lr}
\textit{f}_{i,j}(k)=\textit{v}_{i,j}^{f}{~} \rho_{i,j}(k) \\
\rho_{i,j}(k)=\rho_{i,j}(k-1)+\Delta_{t}\frac{\textit{u}_{i,j}(k)-\textit{f}_{i,j}(k)}{L_{i,j}}
\end{array} \right.
\nonumber
\end{equation}

Then we obtain the following equations:

\begin{equation}
\left\{
\begin{array}{lr}
\textit{f}_{i,j}(k)=\frac{L_{i,j}~\textit{v}_{i,j}^{f}~\rho_{i,j}(k-1)+\Delta_{t}~\textit{v}_{i,j}^{f}~u_{i,j}(k)}{L_{i,j}+\Delta_{t}~\textit{v}_{i,j}^{f}} \\
\rho_{i,j}(k)=\frac{L_{i,j}~\rho_{i,j}(k-1)+\Delta_{t}~u_{i,j}(k)}{L_{i,j}+\Delta_{t}~\textit{v}_{i,j}^{f}}
\end{array} \right.
\nonumber
\end{equation}

\item Congested state:

\begin{equation}
\left\{
\begin{array}{lr}
\textit{f}_{i,j}(k)=\frac{1-\rho_{i,j}(k)~L}{h_{i,j}(k)} \\
\rho_{i,j}(k)=\rho_{i,j}(k-1)+\Delta_{t}\frac{\textit{u}_{i,j}(k)-\textit{f}_{i,j}(k)}{L_{i,j}}
\end{array} \right.
\nonumber
\end{equation}

Then the two quantities can be derived as:

\begin{equation}
\left\{
\begin{array}{lr}
\textit{f}_{i,j}(k)=\frac{L_{i,j}-L~L_{i,j}~\rho_{i,j}(k-1)-L~\Delta_{t}~u_{i,j}(k)}{L_{i,j}~h_{i,j}(k)-L~\Delta_{t}} \\
\rho_{i,j}(k)=\frac{h_{i,j}(k)~L_{i,j}~\rho_{i,j}(k-1)+h_{i,j}(k)~\Delta_{t}~u_{i,j}(k)-\Delta_{t}}{L_{i,j}~h_{i,j}(k)-L~\Delta_{t}}
\end{array} \right.
\nonumber
\end{equation}

\end{itemize} 

%Then we want to construct a solution the headway setting $\textbf{h}_{2}$ so that the corresponding total travel time is less than or equal to $TTT(\textbf{h}_{2})$, then $TTT(\textbf{h}_{1})\leq TTT(\textbf{h}_{2})$. Suppose under the headway setting $\textbf{h}_{2}$, we have already solved the Programming (\Rmnum{1}) and derive the solution. 
Next, We will discuss different states of $\rho_{i,j}^{(2)}(k)$:

\begin{itemize}

\item If $\rho_{i,j}^{(2)}(k)$ is under free-flow state:
Because $h_{i,j}^{(1)}(k)\leq h_{i,j}^{(2)}(k)$,  as is shown in Figure \ref{fig:FD}, it is obvious that $f_{i,j}^{(1)}(k)=f_{i,j}^{(2)}(k)$.

\item If $\rho_{i,j}^{(2)}(k)$ is under congested state:

\begin{equation}
\left\{
\begin{array}{lr}
\rho_{i,j}^{(2)}(k)=\frac{L_{i,j}~\rho_{i,j}^{(2)}(k-1)+\Delta_{t}~u_{i,j}^{(2)}(k)}{L_{i,j}+\Delta_{t}~\textit{v}_{i,j}^{f}}\geq \frac{1}{h_{i,j}^{(2)}(k)~\textit{v}_{i,j}^{f}+L} & \text{The density in free-flow state  should be infeasible}\\
\rho_{i,j}^{(2)}(k)=\frac{h_{i,j}^{(2)}(k)~L_{i,j}~\rho_{i,j}^{
(2)}(k-1)+h_{i,j}^{(2)}(k)~\Delta_{t}~u_{i,j}^{(2)}(k)-\Delta_{t}}{L_{i,j}~h_{i,j}^{(2)}(k)-L~\Delta_{t}} \geq \frac{1}{h_{i,j}^{(2)}(k)~\textit{v}_{i,j}^{f}+L} & \text{The density in congested state should be feasible}
\end{array} \right.
\nonumber
\end{equation}

For the free-flow state, if $L_{i,j}~h_{i,j}^{(2)}(k)-L~\Delta_{t}<0$, we have

\begin{equation}
\begin{aligned}
\rho_{i,j}^{(2)}(k)&=\frac{h_{i,j}^{(2)}(k)~L_{i,j}~\rho_{i,j}^{(2)}(k-1)+h_{i,j}^{(2)}(k)~\Delta_{t}~u_{i,j}^{(2)}(k)-\Delta_{t}}{L_{i,j}~h_{i,j}^{(2)}(k)-L~\Delta_{t}}\\
&\leq \frac{\frac{L_{i,j}+\Delta_{t}~\textit{v}_{i,j}^{f}}{h_{i,j}^{(2)}(k)~\textit{v}_{i,j}^{f}+L}~h_{i,j}^{(2)}(k)-\Delta_{t}}{L_{i,j}~h_{i,j}^{(2)}(k)-L~\Delta_{t}} = \frac{1}{h_{i,j}^{(2)}(k)~\textit{v}_{i,j}^{f}+L}
\nonumber
\end{aligned}
\end{equation}

It is contradicted with $\rho_{i,j}^{(2)}(k) \geq \frac{1}{h_{i,j}^{(2)}(k)~\textit{v}_{i,j}^{f}+L}$. Therefore, $L_{i,j}~h_{i,j}^{(2)}(k)-L~\Delta_{t}>0$. It is also similar to prove that if $\rho_{i,j}(k)$ is congested under the headway $h_{i,j}(k)$, then $L_{i,j}~h_{i,j}(k)-L~\Delta_{t}>0$. 

Next, we discuss the states of $\rho_{i,j}^{(1)}(k)$.

\begin{itemize}
\item[*] $\rho_{i,j}^{(1)}(k)$ is in congested state, then we have

\begin{equation}
\rho_{i,j}^{(1)}(k)=\frac{L_{i,j}~\rho_{i,j}^{(2)}(k-1)+\Delta_{t}~u_{i,j}^{(2)}(k)}{L_{i,j}+\Delta_{t}~\textit{v}_{i,j}^{f}}\geq \frac{1}{h_{i,j}^{(1)}(k)~\textit{v}_{i,j}^{f}+L}\geq
\frac{1}{h_{i,j}^{(2)}(k)~\textit{v}_{i,j}^{f}+L}
\nonumber
\end{equation}

The congested state of $\rho_{i,j}^{(1)}(k)$ can be calculated as follows:

\begin{equation}
\begin{aligned}
\rho_{i,j}^{(1)}(k)&=\frac{h_{i,j}^{(1)}(k)~L_{i,j}~\rho_{i,j}^{(2)}(k-1)+h_{i,j}^{(2)}(k)~\Delta_{t}~u_{i,j}^{(2)}(k)-\Delta_{t}}{L_{i,j}~h_{i,j}^{(1)}(k)-L~\Delta_{t}}\\
&\geq \frac{\frac{L_{i,j}+\Delta_{t}~\textit{v}_{i,j}^{f}}{h_{i,j}^{(1)}(k)~\textit{v}_{i,j}^{f}+L}~h_{i,j}^{(1)}(k)-\Delta_{t}}{L_{i,j}~h_{i,j}^{(1)}(k)-L~\Delta_{t}} = \frac{1}{h_{i,j}^{(1)}(k)~\textit{v}_{i,j}^{f}+L}
\nonumber
\end{aligned}
\end{equation}

The flow  $\textit{f}_{i,j}(k)=\frac{L_{i,j}-L~L_{i,j}~\rho_{i,j}(k-1)-L~\Delta_{t}~u_{i,j}(k)}{L_{i,j}~h_{i,j}(k)-L~\Delta_{t}}$ and we have proved that both numerator and denominator are positive, so if $h_{i,j}^{(1)}(k)\leq h_{i,j}^{(2)}(k)$, we have $f_{i,j}^{(1)}(k)\geq f_{i,j}^{(2)}(k)$.

\item[*] $\rho_{i,j}^{(1)}(k)$ is in free-flow state, then the free-flow state of $\rho_{i,j}^{(1)}(k)$ should be feasible, as shown in the followign equation.

\begin{equation}
\frac{1}{h_{i,j}^{(2)}(k)~\textit{v}_{i,j}^{f}+L} \leq \rho_{i,j}^{(1)}(k)=\frac{L_{i,j}~\rho_{i,j}^{(2)}(k-1)+\Delta_{t}~u_{i,j}^{(2)}(k)}{L_{i,j}+\Delta_{t}~\textit{v}_{i,j}^{f}} \leq \frac{1}{h_{i,j}^{(1)}(k)~\textit{v}_{i,j}^{f}+L}
\nonumber
\end{equation}

Therefore, for the flow $f_{i,j}^{(1)}(k)$, we have:

\begin{equation}
f_{i,j}^{(1)}(k)=\frac{L_{i,j}~\textit{v}_{i,j}^{f}~\rho_{i,j}^{(2)}(k-1)+\Delta_{t}~\textit{v}_{i,j}^{f}~u_{i,j}^{(2)}(k)}{L_{i,j}+\Delta_{t}~\textit{v}_{i,j}^{f}}\geq \frac{\textit{v}_{i,j}^{f}}{h_{i,j}^{(2)}(k)~\textit{v}_{i,j}^{f}+L} \geq f_{i,j}^{(2)}(k)
\nonumber
\end{equation}

\end{itemize} 
\end{itemize}
Combining the above proofs for the two states, we have $f_{i,j}^{(1)}(k)\geq f_{i,j}^{(2)}(k)$ hold.
\end{proof}


Then we develop Lemma \ref{lemma:2} to prove that the assumption of Lemma \ref{lemma:1} is reachable, which means that when $\textit{h}_{i,j}^{1}(k)\leq\textit{h}_{i,j}^{2}(k)$, we could always find the feasible $u_{i,j}(k)$ and $\rho_{i,j}(k-1)$ for both $\textit{h}_{i,j}^{1}(k)$ and $\textit{h}_{i,j}^{2}(k)$.

\begin{lemma}
For $\textbf{h}_{1}=\left\{\textit{h}_{i,j}^{(1)}(k) \right\}$ and $\textbf{h}_{2}=\left\{\textit{h}_{i,j}^{(2)}(k) \right\}$, if there exist only one $(i,j)$ and $k$ such that $\textit{h}_{i,j}^{(1)}(k) \leq \textit{h}_{i,j}^{(2)}(k)$ and $\textit{h}_{m,n}^{(1)}(l) = \textit{h}_{m,n}^{(2)}(l)$ for other links $(m,n)\neq(i,j)$ or time intervals $l\neq k$, suppose $u_{i,j}^{~ s^{\prime},(2)}(k) \in \textbf{x}_{2}, \forall s^{\prime}$, where $\textbf{x}_{2} \in \Omega_{\textbf{h}_{2}}$, then there always exists  $\textbf{x}_{1} \in \Omega_{\textbf{h}_{1}}$, such that $u_{i,j}^{~ s^{\prime},(1)}(k) \in \textbf{x}_{1}$ and $ u_{i,j}^{~ s^{\prime},(1)}(k)=u_{i,j}^{~ s^{\prime},(2)}(k)$.
\label{lemma:2}
\end{lemma}

\begin{proof}
In addition to the flow $f_{i,j}(k)$, headway affects the shockwave travel time $n_{i,j}^{w}(k)$ as follows: 


\begin{eqnarray}
n_{i,j}^{w}(k) \Delta_{t} \frac{\textit{L}}{\textit{h}_{i,j}(k)}  &\leq& L_{i,j}, 
\nonumber \\
(n_{i,j}^{w}(k)+1) \Delta_{t} \frac{\textit{L}}{\textit{h}_{i,j}(k)}  &>& L_{i,j}. 
\nonumber
\end{eqnarray}

To show the existence of a feasible $\textbf{x}_{1}$ such that $u_{i,j}^{~ s^{\prime},(1)}(k)=u_{i,j}^{~ s^{\prime},(2)}(k)$, we need to show that $u_{i,j}^{~ s^{\prime},(2)}(k)$ satisfies the upstream queue capacity under $\textit{h}_{i,j}^{(1)}(k)$ and other related variables in $\textbf{x}_{2}$ if we set $\textit{u}_{i,j}^{~ s^{\prime},(1)}(l)=\textit{u}_{i,j}^{~ s^{\prime},(2)}(l), \forall 1\leq l \leq k-1$. Therefore, we could also derive $\textit{f}_{i,j}^{~ s^{\prime},(1)}(l)=\textit{f}_{i,j}^{~ s^{\prime},(2)}(l), \forall 1\leq l \leq k-1$ by both the proof of Lemma \ref{lemma:1} and the assumption that $\textit{h}_{i,j}^{(1)}(l)=\textit{h}_{i,j}^{(2)}(l), \forall 1\leq l \leq k-1$.

If $h_{i,j}^{(1)}(k)\leq h_{i,j}^{(2)}(k)$, then $n_{i,j}^{w,(1)}(k)\leq n_{i,j}^{w,(2)}(k)$, which implies that if we use a smaller the headway in link $(i,j)$ at the $k_{th}$ time interval, the shockwave travel time would be smaller. By Equation \ref{eq:uq_dis}, we have $\textit{q}_{i,j}^{\mathcal{U},s^{\prime},(1)}(k) \leq \textit{q}_{i,j}^{\mathcal{U},s^{\prime},(2)}(k)$. In order to keep $\textit{u}_{i,j}^{~ s^{\prime},(2)}(k)$ unchanged under $h_{i,j}^{(1)}(k)$, we next prove $\textit{q}_{i,j}^{\mathcal{U},s^{\prime},(1)}(k)\geq0$.

\begin{eqnarray}
    \textit{q}_{i,j}^{\mathcal{U},s^{\prime},(1)}(k) &=& \sum_{l=0}^{\textit{k}}\Delta_{t} \textit{u}_{i,j}^{~ s^{\prime},(1)}(l) - \sum_{l=0}^{\textit{k}-n_{i,j}^{w,(1)}(k)}\Delta_{t} \textit{f}_{i,j}^{~ s^{\prime},(1)}(l) 
    \nonumber \\
    &=& \sum_{l=0}^{\textit{k}}\Delta_{t} \textit{u}_{i,j}^{~ s^{\prime},(2)}(l) - \sum_{l=0}^{\textit{k}-n_{i,j}^{w,(1)}(k)}\Delta_{t} \textit{f}_{i,j}^{~ s^{\prime},(1)}(l) 
    \nonumber \\
    &\geq&
    \sum_{l=0}^{\textit{k}}\Delta_{t} \textit{u}_{i,j}^{~ s^{\prime},(2)}(l) - \sum_{l=0}^{\textit{k}-1}\Delta_{t} \textit{f}_{i,j}^{~ s^{\prime},(1)}(l) 
    \nonumber \\
    &=&
    \sum_{l=0}^{\textit{k}}\Delta_{t} \textit{u}_{i,j}^{~ s^{\prime},(2)}(l) - \sum_{l=0}^{\textit{k}-1}\Delta_{t} \textit{f}_{i,j}^{~ s^{\prime},(2)}(t) 
    \nonumber \\
    &=& L_{i,j}\rho_{i,j}^{s^{\prime},(2)}(k-1)+\Delta_{t}\textit{u}_{i,j}^{~ s^{\prime},(2)}(k) \geq 0 
    \nonumber
\end{eqnarray}
\end{proof}

%We have proposed an equivalent form of our objective in the begin of \ref{app:minimum} as follows:

%\begin{equation}
%min~ TTT \Longleftrightarrow min~ \sum_{s^{\prime}}\sum_{\textit{k}=1}^{\textit{N}}k~\textit{u}_{s,s^{\prime}}(k)\nonumber
%\end{equation}


%Let $\Psi_{s}$ denote the set of link whose end node is destination $s$. Therefore, the total travel time could be represented as the sum of total travel time of links in $\Psi=\cup_{s}\Psi_{s}$ and total travel time of other links and waiting time in dummy origin links.

By Lemma \ref{lemma:1} and \ref{lemma:2}, for link $(j,s)$, where $s \in S$,  if $h_{j,s}^{(1)}(k)\leq h_{j,s}^{(2)}(k), h_{j,s}^{(1)}(l) = h_{j,s}^{(2)}(l), \forall l \neq k$  
and we set $u_{j,s}^{(1)}(l)=u_{j,s}^{(2)}(l)$ for all $1 \leq l \leq N$, we have $f_{j,s}^{(1)}(k)\geq f_{j,s}^{(2)}(k)$.
For density, we have the following equation hold:

\begin{equation}
\rho_{j,s}(k)=\rho_{j,s}(k-1)+\Delta_{t}\frac{\textit{u}_{j,s}(k)-\textit{f}_{j,s}(k)}{L_{j,s}}. 
\label{eq:den_app}
\end{equation}

If $\rho_{j,s}^{(1)}(k-1)=\rho_{j,s}^{(2)}(k-1)$, then we have $\rho_{i,j}^{(1)}(k) \leq \rho_{i,j}^{(2)}(k)$. If $\rho_{i,j}^{(2)}(k+1)$ is in the free-flow state, the following equations hold in the proof of Lemma \ref{lemma:1} for $\rho_{j,s}^{(2)}(k+1)$ and $f_{j,s}^{(2)}(k+1)$:

\begin{equation}
\left\{
\begin{array}{lr}
\textit{f}_{j,s}(k)=\frac{L_{j,s}~\textit{v}_{j,s}^{f}~\rho_{j,s}(k-1)+\Delta_{t}~\textit{v}_{j,s}^{f}~u_{j,s}(k)}{L_{j,s}+\Delta_{t}~\textit{v}_{j,s}^{f}} \\
\rho_{j,s}(k)=\frac{L_{i,j}~\rho_{j,s}(k-1)+\Delta_{t}~u_{j,s}(k)}{L_{j,s}+\Delta_{t}~\textit{v}_{j,s}^{f}}
\end{array} \right.
\nonumber
\end{equation}

So we have $f_{i,j}^{(1)}(k+1) \leq f_{i,j}^{(2)}(k+1)$ and $\rho_{i,j}^{(1)}(k+1) \leq \rho_{i,j}^{(2)}(k+1)$ if $\rho_{i,j}^{(2)}(k+1)$ hold  in the free-flow state. 
Then if $\rho_{i,j}^{(2)}(k+1)$ is in the congested state, we have the following equations in the proof of Lemma \ref{lemma:1} hold for $\rho_{j,s}^{(2)}(k+1)$ and $f_{j,s}^{(2)}(k+1)$:

\begin{equation}
\left\{
\begin{array}{lr}
\textit{f}_{j,s}(k)=\frac{L_{j,s}-L~L_{j,s}~\rho_{j,s}(k-1)-L~\Delta_{t}~u_{j,s}(k)}{L_{j,s}~h_{j,s}(k)-L~\Delta_{t}} \\
\rho_{j,s}(k)=\frac{h_{j,s}(k)~L_{j,s}~\rho_{j,s}(k-1)+h_{j,s}(k)~\Delta_{t}~u_{j,s}(k)-\Delta_{t}}{L_{j,s}~h_{j,s}(k)-L~\Delta_{t}}
\end{array} \right.
\nonumber
\end{equation}

Similarly,   $f_{i,j}^{(1)}(k+1) \geq f_{i,j}^{(2)}(k+1)$ and $\rho_{i,j}^{(1)}(k+1) \leq \rho_{i,j}^{(2)}(k+1)$ hold for  $\rho_{i,j}^{(2)}(k+1)$ in the congested state. 

Therefore, we have $\rho_{i,j}^{(1)}(k+1) \leq \rho_{i,j}^{(2)}(k+1)$. We conduct the same procedure for $k+2,\cdots,N$ and $\rho_{i,j}^{(1)}(l) \leq \rho_{i,j}^{(2)}(l)$ is derived for $l \geq k$ and $\rho_{i,j}^{(1)}(l) = \rho_{i,j}^{(2)}(l)$ for $l < k$. Besides, by $F_{j,s}(k)=L_{j,s}\rho_{j,s}(k)+U_{j,s}(k)$, so we have $F_{j,s}^{(1)}(l)=F_{j,s}^{(2)}(l)$ for $l < k$ and $F_{j,s}^{(1)}(l) \geq F_{j,s}^{(2)}(l)$ for $l \geq k$. 

Similarly, we could also get the $\rho_{j,s}^{s_{0}^{\prime},(1)}(l) \leq \rho_{j,s}^{s_{0}^{\prime},(2)}(l)$ for $l \geq k$ and $\rho_{i,j}^{s_{0}^{\prime},(1)}(l) = \rho_{i,j}^{s_{0}^{\prime},(2)}(l)$ for $l < k$ by the same procedure for any destination $s_{0}$. Besides,  for any destination $s_{0}$, we also have $F_{j,s}^{s_{0}^{\prime},(1)}(k)=F_{j,s}^{s_{0}^{\prime},(2)}(k)$ for $l < k$ and $F_{j,s}^{s_{0}^{\prime},(1)}(k) \geq F_{j,s}^{s_{0}^{\prime},(2)}(k)$ for $l \geq k$. 

Therefore, for destination $s_{0} \neq s$, we could set $v_{j,s}^{s_{0}^{\prime},(1)}(l)=v_{j,s}^{s_{0}^{\prime},(2)}(l),\forall 1 \leq l \leq N$. For destination $s$,  we  have $V_{j,s}^{s^{\prime},(1)}(k)=V_{j,s}^{s^{\prime},(2)}(k)$ for $l < k$ and $V_{j,s}^{s^{\prime},(1)}(k) \geq V_{j,s}^{s^{\prime},(2)}(k)$ for $l \geq k$, where $V_{j,s}^{s^{\prime}}(k)=\sum_{l=1}^{k}\Delta_{t}v_{j,s}^{s^{\prime}}(l)$ represents the cumulative outflow with destination 
$s^{\prime}$ of link $(j,s)$ in the $k$th time interval.
Then the total travel time in link $(j,s)$ can be written  as follows:

\begin{eqnarray}
    TTT_{j,s} &=& \sum_{k=1}^{N} \Delta_{t} \sum_{l=1}^{k} \Delta_{t} u_{j,s}(l)-v_{j,s}(l)
    \nonumber \\
    &=& \sum_{k=1}^{N}\sum_{s_{0}} \Delta_{t}\Delta_{t} k\,u_{j,s}^{s_{0}^{\prime}}(k) - \Delta_{t}\Delta_{t} k\,v_{j,s}^{s_{0}^{\prime}}(k) 
    \nonumber \\
    &=& \sum_{k=1}^{N}\sum_{s_{0} \neq s} \left(\Delta_{t}\Delta_{t} k\,u_{j,s}^{s_{0}^{\prime}}(k) - \Delta_{t}\Delta_{t} k\,v_{j,s}^{s_{0}^{\prime}}(k) \right) + \sum_{k=1}^{N} \left(\Delta_{t}\Delta_{t} k\,u_{j,s}^{s^{\prime}}(k) - \Delta_{t}\Delta_{t} k\,v_{j,s}^{s^{\prime}}(k)\right) 
    \nonumber \\
    &=& \sum_{k=1}^{N}\sum_{s_{0} \neq s} \left(\Delta_{t}\Delta_{t} k\,u_{j,s}^{s_{0}^{\prime}}(k) - \Delta_{t}\Delta_{t} k\,v_{j,s}^{s_{0}^{\prime}}(k) \right) + \sum_{k=1}^{N} \Delta_{t}\Delta_{t} k\,u_{j,s}^{s^{\prime}}(k) - \sum_{k=1}^{N} \Delta_{t} V_{j,s}^{s^{\prime}}(k) 
    \nonumber
\end{eqnarray}

Therefore, we have proved that $u_{j,s}^{s_{0}^{\prime},(1)}(l)=u_{j,s}^{s_{0}^{\prime},(2)}(l)$, $v_{j,s}^{s_{0}^{\prime},(1)}(l)=v_{j,s}^{s_{0}^{\prime},(2)}(l)$ for $s_{0}\neq s$ and $1 \leq l \leq N$. For destination $s$, we have $u_{j,s}^{s^{\prime},(1)}(l)=u_{j,s}^{s^{\prime},(2)}(k)$ and $V_{j,s}^{s^{\prime},(1)}(l) \geq V_{j,s}^{s^{\prime},(2)}(l)$ for $1 \leq l \leq N$. Then we have

\begin{equation}
    TTT_{j,s}^{(1)} \leq TTT_{j,s}^{(2)}.
    \nonumber
\end{equation}

The total travel time consists of the total travel time in link $(j,s)$, total travel time in other links and waiting time at origins. The inflow and outflow variables in other links and origins are not affected by the change of headway from $h_{i,j}^{(1)}(k)$ to $h_{i,j}^{(2)}(k)$, so $TTT_{\textbf{h}_{1}} \leq TTT_{\textbf{h}_{2}}$ if $h_{j,s}^{(1)}(k) \leq h_{j,s}^{(2)}(k)$.
For link $(i,j)$, the node ${j}$ acts like the destination node, and we want to push more flow with destination $s$ to leave link $(i,j)$ and enter link $(j,s)$. The process is similar to what we have done for link $(j,s)$ by only changing one headway in link $(i,j)$ and keep flow variables in other links unchanged. By repeating the above procedure for all the links and time intervals, we can show that TTT could be minimized under the minimum headway. 

%Therefore, we have proved this statement. So we could have $v_{i,j}^{(1)}(k)\geq v_{i,j}^{(2)}(k)$. And we keep all variables in other links unchanged, so the route behavior of original flows unchanged. We only care about the increasing flow $v_{i,j}^{(1)}(k)-v_{i,j}^{(2)}(k)$. 

%Among all downstream links of link $(i,j)$, the flow is stated as following:

%\begin{equation}
%\textit{f}_{\emph{j},\emph{h}}(k)=\left\{
%\begin{array}{lr}
%\frac{L_{j,h}~\textit{v}_{j,h}^{f}~\rho_{j,h}(k-1)+\Delta_{t}~\textit{v}_{j,h}^{f}~u_{j,h}(k)}{L_{j,h}+\Delta_{t}~\textit{v}_{j,h}^{f}} &  \text{free-flow} \\ \frac{L_{j,h}-L~L_{j,h}~\rho_{j,h}(k-1)-L~\Delta_{t}~u_{j,h}(k)}{L_{j,h}~h_{j,h}(k)-L~\Delta_{t}} & \text{congested}
%\end{array} \right.
%\nonumber
%\end{equation}

%For increasing inflow $u_{j,h}(k)$, we desire a larger flow $f_{j,h}(k)$. As is shown, if $\rho_{j,h}$ is under free-flow status and we increase the inflow $u_{j,h}(k)$, the flow $f_{j,h}(k)$ will also increase. We know that the objective  is to minimize  $\sum_{s^{\prime}}\sum_{\textit{k}=1}^{\textit{N}}k~\textit{u}_{s,s^{\prime}}(k)$, which means that we encourage to push flow to arrive destinations as early as possible. So we could arrange the increasing inflow to the downstream links that are in free-flow status under the headway setting $\textbf{h}_{2}$. If there is no such downstream link, we still arrange the same outflow  and inflow as under the headway setting $\textbf{h}_{2}$. The idea is to push more flow into the downstream queue and there would possibly be a larger outflow from downstream queue, so the in-flow entering downstream links would be larger.
%If we do the strategy for all links "behind" link $(i,j)$, then we push more flow to arrive destinations as early as possible. So $TTT(\textbf{h}_{1})\leq TTT(\textbf{h}_{2})$ and we prove the following statement:

%\begin{itemize}
%\item If the item $h_{i,k}(k)$ in $\textbf{h}_{1}$ is less than it in $\textbf{h}_{2}$ and all other items in $\textbf{h}_{1}$ are equal to those in $\textbf{h}_{2}$, then $TTT(\textbf{h}_{1})\leq TTT(\textbf{h}_{2})$.
%\end{itemize} 

%So if we traverse each link and each time interval separately, it is obvious that under the minimum headway setting, the SO-DTA could be achieved. Therefore, the proposition \ref{prop:minimum} has been proved.

\section{Proof of Proposition \ref{prop:nonuni}}
\label{app:nonuniq}

Suppose $\textbf{x}^{*}$ and $\textbf{h}^{*}$ are the optimal solution of the proposed SO-DTA problem. If $\exists~ (i,j) \in E\setminus(L_R\cup L_S) $ and $1 \leq k \leq N$, $s.t. ~ \delta_{i,j}^{*}(k)=0$, we have following equations.

\begin{eqnarray}
    &0\leq\rho_{i,j}^{*}(k) < \frac{1}{\textit{h}_{i,j}^{*}(k)\textit{v}_{i,j}^f+\textit{L}} 
    \nonumber \\
    %&\textit{f}_{i,j}^{*}(k)=\textit{v}_{i,j}^{f}{~} \rho_{i,j}^{*}(k) 
    %\nonumber \\
    &n_{i,j}^{w}(k) \Delta_{t} \frac{\textit{L}}{\textit{h}_{i,j}(k)}  \leq L_{i,j} 
\nonumber \\
&(n_{i,j}^{w}(k)+1) \Delta_{t} \frac{\textit{L}}{\textit{h}_{i,j}(k)}  > L_{i,j} 
\nonumber 
\end{eqnarray}

For the following linear Equations \ref{app:nonuniue1} to \ref {app:nonuniue4} of $\textit{h}_{i,j}(k)$, it is obvious that $\textit{h}_{i,j}^{*}(k)$ satisfies all of these equations.

\begin{eqnarray}
    &\rho_{i,j}^{*}(k)\textit{v}_{i,j}^f\textit{h}_{i,j}(k) <1-\rho_{i,j}^{*}(k)\textit{L} 
    \label{app:nonuniue1} \\
    &\textit{h}_{i,j}(k) \geq \frac{\Delta_{t}Ln_{i,j}^{w,*}(k)}{L_{i,j}}
    \label{app:nonuniue2} \\
    &\textit{h}_{i,j}(k) < \frac{\Delta_{t}L(n_{i,j}^{w,*}(k)+1)}{L_{i,j}}
    \label{app:nonuniue3} \\
&h_{i,j}^{min} \leq \textit{h}_{i,j}(k) \leq \textit{h}_{i,j}^{max} 
    \label{app:nonuniue4}
\end{eqnarray}

If $h_{i,j}^{min} \leq \textit{h}_{i,j}^{*}(k) < \textit{h}_{i,j}^{max}$,
Equations \ref{app:nonuniue1} to \ref {app:nonuniue4} could be formulated as following Equations \ref{eq:ineq1} and \ref{eq:ineq2} in general.  

\begin{eqnarray}
    a_{m}x &\geq& b_{m}, \quad  a_{m}>0;b_{m}>0;m=1,\cdots,M. 
    \label{eq:ineq1} \\
    c_{n}x &<& d_{n}, \quad  c_{n}\geq0;c_{n}\geq0;n=1,\cdots,N. 
    \label{eq:ineq2}
\end{eqnarray}

If there exists $x$ satisfing Equations \ref{eq:ineq1} and \ref{eq:ineq2}, then for any constant $c$ with $0 < c < \min \limits_{m}{\{\frac{b_{m}}{a_{m}},1\}}$, we have $cx$ also satisfy Equations \ref{eq:ineq1} and \ref{eq:ineq2}. Therefore, in this problem, there must exist such positive constant $c$ that $\textit{h}_{i,j}(k)=c\textit{h}_{i,j}^{*}(k)$ satisfies Equations \ref{app:nonuniue1} to \ref {app:nonuniue4}.
If $\textit{h}_{i,j}^{*}(k)=\textit{h}_{i,j}^{max}$, by Proposition \ref{prop:minimum}, minimum headway setting achieves SO-DTA, so we show that SO-DTA can have at least two optimal headway that achieves the same TTT.

\section{Proof of Proposition \ref{prop:unique}}
\label{app:unique}

In Proposition \ref{prop:unique}, if $~\textbf{x}^{*}$ with $TTT(\textbf{x}^{*})=TTT^{*}$ is unique, then for $\delta_{i,j}^{*}(k)=1$,
$\textit{h}_{i,j}^{*}(k)$ should satisfy the following equation.

\begin{equation}
    \textit{f}_{i,j}^{*}(k)=\frac{1-\rho_{i,j}^{*}(k)\emph{L}}{\textit{h}_{i,j}^{*}(k)}
    \nonumber
\end{equation}

Therefore, if $\hat{\textbf{h}}$ and $\bar{\textbf{h}}$ are the solution of Formulation \ref{eq:maximin}, we have $\hat{h}_{i,j}(k) = \bar{h}_{i,j}(k)$ when $\delta_{i,j}^{*}(k)=1$. 
Suppose $\hat{\textbf{h}} \neq \bar{\textbf{h}}$, then there must exist $(i_{1},j_{1}), (i_{2},j_{2}) \in E $ and $ 1\leq k_{1}, k_{2} \leq N$ such that $\hat{h}_{i_{1},j_{1}}(k_{1}) < \bar{h}_{i_{1},j_{1}}(k_{1})$ and $\hat{h}_{i_{2},j_{2}}(k_{2}) > \bar{h}_{i_{2},j_{2}}(k_{2})$. Besides, we have $\delta_{i_{1},j_{1}}^{*}(k_{1})=0$ and $\delta_{i_{2},j_{2}}^{*}(k_{2})=0$. So $\hat{h}_{i_{1},j_{1}}(k_{1})$, $\bar{h}_{i_{1},j_{1}}(k_{1})$, $\hat{h}_{i_{2},j_{2}}(k_{2})$ and $\bar{h}_{i_{2},j_{2}}(k_{2})$ should satisfy the Equations \ref{app:nonuniue1} to \ref{app:nonuniue4}. 

Therefore, if we construct a new headway $\textbf{h}$ as follows:

\begin{eqnarray}
    \textit{h}_{i,j}(k) = \begin{cases}
\hat{h}_{i,j}(k), & (i,j)\neq (i_{1},j_{1})~or~k\neq k_{1}\\
\bar{h}_{i,j}(k), & (i,j) = (i_{1},j_{1})~and~k = k_{1}
\end{cases}
\end{eqnarray}

We note that $\textbf{h}$ is the optimal headway of the discretized SO-DTA formulation (Equations \ref{eq:obj_dis} to \ref{eq:demand_dis}) as $\textbf{x}^{*} \in \Omega_{\textbf{h}}$. However, we also have $||~ \textbf{h}~ ||_{1} > ||~ \hat{\textbf{h}}~ ||_{1}$, and this contradicts to the assumption that $\hat{\textbf{h}}$ is the solution of Formulation \ref{eq:maximin}, which completes the proof.


\section{Proof of Proposition \ref{prop:optimal}}
  \label{app:optimal}
  
Suppose $\textbf{x}^{*} \in \Omega_{\textbf{h}^{*}}$ and  $TTT(\textbf{x}^{*})=TTT_{\textbf{h}^{*}}=TTT^{*}$, if $\textbf{h}$ satisfies Equation \ref{eq:buxiangqimingre}:

\begin{equation}
\begin{aligned}
if ~ \delta_{i,j}^{*}(k) &=0:\\
 & \rho_{i,j}^{*}(k)\textit{v}_{i,j}^f\textit{h}_{i,j}(k) < 1-\rho_{i,j}^{*}(k)\textit{L} \\
 %& \textit{f}_{i,j}^{*}(k)=\textit{v}_{i,j}^{f}{~} \rho_{i,j}^{*}(k) \\
 & L_{i,j}\textit{h}_{i,j}(k) \geq \Delta_{t}\textit{L}\textit{n}_{i,j}^{w,*}(k)\\
 & L_{i,j}\textit{h}_{i,j}(k) < \Delta_{t}\textit{L}\textit{n}_{i,j}^{w,*}(k)+\Delta_{t}\textit{L}\\
 & \textit{h}_{i,j}^{*}(k) < \textit{h}_{i,j}(k) \leq \textit{h}_{i,j}^{max} \\
 if ~ \delta_{i,j}^{*}(k) &=1:\\
 & \textit{h}_{i,j}(k) = \textit{h}_{i,j}^{*}(k)
 \label{eq:buxiangqimingre}
\end{aligned},
\end{equation}
we have $||~ \textbf{h}~ ||_{1} > ||~ \textbf{h}^{*}~ ||_{1}$ holds. Besides, it is obvious that $\textbf{x}^{*}$ and $\textbf{h}$ satisfy the Equations  \ref{eq:free_dis} and \ref{ineq:headway_dis}, so we have $\textbf{x}^{*} \in \Omega_{\textbf{h}}$ and $TTT_{\textbf{h}}=TTT(\textbf{x}^{*})=TTT^{*}$, which implies that $\textbf{h}$ is the optimal headway for SO-DTA. Therefore, we can construct such a larger optimal headway  $\textbf{h}$ compared with the original optimal headway  $\textbf{h}^{*}$.

\section{Settings of the small network}
  \label{app:parameter}
  For simplicity, the inflow capacity is set equal to the outflow capacity and the downstream queue capacity is set equal to the upstream queue capacity in each link. 
  % We denote them as flow capacity and queue capacity, respectively. 
  The detailed configurations of each link are presented in Table~\ref{table1}.
\begin{table}[H]%[htbp]
	\centering
     \resizebox{1.02\textwidth}{!}{
	\begin{tabular}{c c c c c}  
		\hline  
		& & & &\\[-6pt] 
		Link & Flow capacity (vehicles/min) & Queue capacity (vehicles) & Free flow speed (km/min) & Link length (km)\\  
		\hline
		& & & &\\[-6pt]  
		1 $\to$ 3 & 50 & 600 & 1.2 & 3.6 \\
        \hline
        & & & &\\[-6pt] 
        1 $\to$ 4 & 45 & 600 & 1.1 & 3.3 \\
        \hline
        & & & &\\[-6pt] 
        2 $\to$ 3 & 45 & 600 & 1.2 & 3.6 \\
		\hline
        & & & &\\[-6pt] 
        2 $\to$ 4 & 50 & 600 & 1.1 & 3.3 \\
		\hline
        & & & &\\[-6pt]
        3 $\to$ 5 & 60 & 600 & 1.0 & 4.0 \\
		\hline
        & & & &\\[-6pt]
        4 $\to$ 5 & 60 & 600 & 1.0 & 3.0 \\
		\hline
	\end{tabular}
 }
 \caption{Parameters settings in the small network.}  
	\label{table1} 
\end{table}


\end{document}


