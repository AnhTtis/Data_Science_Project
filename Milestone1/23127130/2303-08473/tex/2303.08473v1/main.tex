%%%%%%%%%%%%%%%%%%%%%%%%%%%%%%%%%%%%%%%%%%%%%%%%%%%%%%%%%%%%%%%%%%%%%%%%%%%%%%%%
%2345678901234567890123456789012345678901234567890123456789012345678901234567890
%        1         2         3         4         5         6         7         8

\documentclass[letterpaper, 10 pt, conference]{ieeeconf}  % Comment this line out if you need a4paper

%\documentclass[a4paper, 10pt, conference]{ieeeconf}      % Use this line for a4 paper

\IEEEoverridecommandlockouts                              % This command is only needed if 
                                                          % you want to use the \thanks command

\overrideIEEEmargins                                      % Needed to meet printer requirements.

%In case you encounter the following error:
%Error 1010 The PDF file may be corrupt (unable to open PDF file) OR
%Error 1000 An error occurred while parsing a contents stream. Unable to analyze the PDF file.
%This is a known problem with pdfLaTeX conversion filter. The file cannot be opened with acrobat reader
%Please use one of the alternatives below to circumvent this error by uncommenting one or the other
%\pdfobjcompresslevel=0
%\pdfminorversion=4

% See the \addtolength command later in the file to balance the column lengths
% on the last page of the document

% The following packages can be found on http:\\www.ctan.org
%\usepackage{graphics} % for pdf, bitmapped graphics files
\usepackage{epsfig} % for postscript graphics files
%\usepackage{mathptmx} % assumes new font selection scheme installed
\usepackage{times} % assumes new font selection scheme installed
\usepackage{amsmath} % assumes amsmath package installed
\usepackage{amssymb}  % assumes amsmath package installed
\usepackage{graphicx}
\usepackage{caption}
\usepackage{hyperref}
\usepackage{tikz}
\usetikzlibrary{shapes.geometric, arrows}
\usetikzlibrary{3d,fit, positioning}

\usepackage{import}
\subimport{./plot_neural_net}{init}

\graphicspath{ {./images/}}
%\usepackage{endfloat}


\begin{document}

\title{\LARGE \bf Unsupervised Traffic Scene Generation with Synthetic 3D Scene Graphs}

\author{Artem Savkin$^{1,2}$\\TUM, BMW \and Rachid Ellouze$^{1,2}$\\TUM, BMW \and Nassir Navab$^{1,3}$\\TUM, JHU \and Federico Tombari$^{1,4}$\\TUM, Google% <-this % stops a space
%\thanks{*Equal Contribution}% <-this % stops a space
\thanks{$^{1}$TUM, Boltzmannstr. 3, 85748 Munich, Germany
{\tt\small artem.savkin@tum.de};
%{\tt\small rachid.ellouze@tum.de};
{\tt\small tombari@in.tum.de}}%
\thanks{$^{2}$BMW AG, 80809 Munich, Germany}
\thanks{$^{1}$JHU, Baltimore, MD 21218, United States}
\thanks{$^{4}$Google, 8002 Zurich, Switzerland}
}

%\author{Anonymous submission}


\newcommand\blfootnote[1]{%
	\begingroup
	\renewcommand\thefootnote{}\footnote{#1}%
	\addtocounter{footnote}{-1}%
	\endgroup
}

\newcommand{\norm}[1]{\left\lVert#1\right\rVert}
\newcommand{\TODO}[1] {\textbf{[TODO: #1]}}
\newcommand{\Loss}{\mathcal{L}}
\newcommand{\Exp}{\mathop{\mathbb{E}}}
\newcommand{\rarr}{\rightarrow}
\newcommand{\larr}{\leftarrow}

\maketitle

%\twocolumn[{%
%\renewcommand\twocolumn[1][]{#1}%
%\maketitle
%\begin{figure}
    \centering
    \includegraphics[width=\columnwidth]{figures/teaser.pdf}
    \caption{We present \OurMethodName{}, a novel feature pre-training technique aiming for  deformable shapes. By pre-training \emph{local} feature extractors on 3D scenes for rigid alignment, our approach enables transfer learning to downstream deformable shape analysis tasks, such as shape matching and semantic segmentation.}
    \label{fig:teaser}
    \vspace{\figmargin}
\end{figure}

%}]

\begin{abstract}
Image synthesis driven by computer graphics achieved recently a remarkable realism, yet synthetic image data generated this way reveals a significant domain gap with respect to real-world data. This is especially true in autonomous driving scenarios, which represent a critical aspect for overcoming utilizing synthetic data for training neural networks. We propose a method based on domain-invariant scene representation to directly synthesize traffic scene imagery without rendering. Specifically, we rely on synthetic scene graphs as our internal representation and introduce an unsupervised neural network architecture for realistic traffic scene synthesis. We enhance synthetic scene graphs with spatial information about the scene and demonstrate the effectiveness of our approach through scene manipulation.
\end{abstract}

%\blfootnote{%\thanks{*Equal Contribution}% <-this % stops a space
%$^{1}$TUM, 85748 Munich, Germany
%{\tt\small artem.savkin@tum.de};
%{\tt\small rachid.ellouze@tum.de};
%{\tt\small tombari@in.tum.de}
%}%
%\blfootnote{$^{2}$BMW AG, 80809 Munich (Germany)}
%\blfootnote{$^{3}$JHU, Baltimore, MD 21218, United States}
%\blfootnote{$^{4}$Google, 8002 Zurich, Switzerland}


%\addtolength{\textheight}{-12cm}
% This command serves to balance the column lengths
% on the last page of the document manually. It shortens
% the text height of the last page by a suitable amount.

\section{Introduction}

The ability to reason about plans is critical for performing long-horizon tasks \citep{erol1996hierarchical, sohn2018hierarchical, sharma-etal-2022-skill}, compositional generalization \citep{corona-etal-2021-modular} and generalization to unseen tasks and environments \citep{shridhar2020alfred}.
Consider a simple long-horizon planning scenario where a robot is tasked with preparing a meal and serving it on the table. 
This presents a non-trivial planning problem since the agent needs to understand the sequence of operations required to perform the task and search for the relevant objects in the unfamiliar environment by interacting with various objects. %



Large language models have been recently shown to possess commonsense knowledge about the world such as object affordances and physical dynamics \citep{ouyang2022training,chowdhery2022palm}.
Early approaches considered text based environments and fine-tuned PLMs to predict actions given the history of past observations and actions \citep{jansen-2020-visually,micheli-fleuret-2021-language,yao-etal-2020-keep}.
Recent work has used this ability to reason about plans from text instructions in simulated household environments with simplifying assumptions such as text-only environment observations or feedback \citep{huang2022language,ahn2022can,li2022pre,logeswaran-etal-2022-shot}.


We focus on \emph{visually grounded planning} with PLMs --- the ability to adapt plans based on interaction and visual feedback from the environment.
While PLMs have strong planning commonsense priors, predictions from a PLM may not be directly realizable in the environment since the observation and action spaces are unknown.
This requires \emph{grounding} the PLM in the environment and adapting it to observe visual feedback, which is highly non-trivial.
Some prior works assume the availability of a pre-trained affordance function \citep{ahn2022can} or a success detector \citep{mirchandani2021ella}.
Notably, SayCan \citep{ahn2022can} completely decouples the PLM from observation information by selecting actions that have both high affordability (through a pre-trained affordance model) and high PLM likelihood.
Although this partially addresses the grounding problem, the use of visual feedback for action affordance alone is limited.
Often an agent must choose one of many affordable actions using information from observations.
For example, a driving agent should re-navigate and possibly turn around when encountering a ``road closed'' sign, but both turning around and driving forward are indistinguishable to SayCan because they are both affordable and the PLM is blind to observations.

Another workaround explored in prior work is translating the information in the visual observations to text using a pre-trained captioning system \citep{shridhar2021alfworld,huang2022language}.
However, it can be difficult to faithfully describe an image in words and information is lost in this inherently noisy process, which limits the information available to the planner.



Recent work shows that PLMs can be adapted for various natural language tasks by inserting tunable embeddings or soft prompts at the input of the PLM (also called prompt tuning or prefix tuning)~\citep{li-liang-2021-prefix,lester-etal-2021-power}.
This approach also extends to multi-modal understanding tasks such as image captioning \citep{mokady2021clipcap} and VQA \citep{tsimpoukelli2021multimodal} where images are encoded as soft prompts and finetuned for the target task.
Transformer based architectures have also been successfully applied to offline Reinforcement Learning in recent work \citep{chen2021decision,janner2021offline,li2022pre,reid2022can}.

Taking inspiration from these works, we propose the simple approach of embedding visual observations (`visual prompts') and \textit{directly inserting them as PLM input embeddings}.
The visual encoder and PLM are jointly trained for the target task, an approach we call \textbf{\oursfull}~(\ours).
By teaching the PLM to use observations for planning in an end to end manner, we remove the dependency on external data such as captions and affordability information that was used in prior work.
We show that this simple approach performs better than prior PLM-based planning approaches on two embodied planning benchmarks based on ALFWorld~\citep{shridhar2021alfworld} and Virtualhome~\cite{puig2018virtualhome}.



\section{Related Work}

\subsection{Pattern discovery on systematic AI errors}

Systematic errors, sometimes coined as blind spots or unknown-unknowns \cite{BeatMachineChallengingHumans}, refer to model's failure over a group of instances that share similar semantics. There are various approaches for discovering such patterns, including algorithmic, human, or hybrid techniques.

A number of studies have shown that fully algorithmic techniques can help automatically discover unknown-unknowns \cite{lakkaraju2017identifying, coveragebasedutility}. Recently, several studies have also been proposed to advance the methods towards discovering automatic slices or subclasses that are semantically coherent \cite{DominoDiscoveringSystematicErrorsCrossModal, SpotlightGeneralMethodDiscoveringSystematic}, or to propose a framework for evaluating blindspot discovery methods in a unified manner \cite{EvaluatingSystemicErrorDetectionMethods}.

On the other hand, researchers have also explored how human intelligence can identify blind spots where automatic techniques alone do not work. Several studies \cite{BeatMachineChallengingHumans, ContradictMachineHybridApproach, HybridHumanAIWorkflowsUnknown, InvestigatingHumanMachineComplementarity} demonstrated that a well-designed crowdsourcing study can detect problematic instances. Hybrid workflows to leverage the abilities of both humans and machines \cite{HybridHumanAIWorkflowsUnknown, lakkaraju2017identifying, han2021iterative, chung2018unknownexamples} have also been explored throughout several studies in proposing collaborative human-AI workflow \cite{HybridHumanAIWorkflowsUnknown} or generating text descriptions \cite{han2021iterative} about spurious patterns.

While these studies demonstrate how human intelligence plays a significant role, tool support is still lacking to guide practitioners to inspect, identify, and mitigate systematic errors. In our study, we provide a workflow and systematic support for inspecting which systematic errors are attributed to interpretable concepts.

\subsection{Visual analytics for ML diagnostics}
Visual analytics tools in recent years have evolved to offer interactive ways for inspecting the machine learning process. In general, these tools aim to better visualize the predictive results in a model-agnostic manner or present the structure of the model in a model-specific way. Model-agnostic approaches propose to better visualize machine learning results regardless of model types. Many visualizations among them are largely designed on the grounds of confusion matrix as tree or flow diagram \cite{shen2020designing, VisualizingSurrogateDecisionTrees}, comparative visual design \cite{ManifoldModelAgnosticFrameworkInterpretation, ExplainExploreVisualExplorationMachine, olson2021contrastive, kaul2021improvingcounterfactuals, krause2017workflow}, radial \cite{VisualMethodsAnalyzingProbabilistic} or multi-axes based layout \cite{SquaresSupportingInteractivePerformance}. On the other hand, model-specific inspections also gained attention to support the inspection of a deep neural network inside its layers, neurons, or activations \cite{liu2017analyzingtraining, ShapeShopUnderstandingDeepLearning, TopoActVisuallyExploringShape, DeepVIDDeepVisualInterpretation}.

Visual analytic tools can also help inspect and explain the potential cause of systematic failures such as a shifted or skewed distribution of the training examples termed as out-of-distribution \cite{OoDAnalyzerInteractiveAnalysisOutofDistribution}, covariate or concept shift \cite{DiagnosingConceptDriftVisual} or machine biases \cite{FairVisVisualAnalyticsDiscovering, FairSightVisualAnalyticsFairness, WhatIfToolInteractiveProbing}. The OoD analyzer \cite{OoDAnalyzerInteractiveAnalysisOutofDistribution} presented a grid-based layout to visualize the distributional differences in training and test sets. The problem of concept drift was tackled and presented as visualizations in a 2D heatmap visualization \cite{DiagnosingConceptDriftVisual} or distribution-based scatterplot \cite{ConceptExplorerVisualAnalysisConcept}. Other interactive tools such as Deblinder \cite{DiscoveringValidatingAIErrorsCrowdsourced}, SEAL \cite{SEALInteractiveToolSystematicError}, or Error Analysis \cite{erroranalysis} have recently been proposed to mitigate systematic errors with subclass labeling or user-generated report. Compared to previous work, our study aims to promote a human-in-the-loop workflow consisting of tasks to identify biased patterns and their association/attribution aspects with the perspective of spurious associations.

% Recent visualization studies also proposed how to better explain them with counterfactuals [BF-1], or to present them in a form of report [BF-3]. 


\subsection{Understanding model with concept interpretability}

The XAI methods to explain the behavior of black box models \cite{InterpretabilityFeatureAttributionQuantitative, AutomaticConceptbasedExplanations, BayesianCaseModelGenerative, ConceptWhiteningInterpretable2020} have been recently expanded to a concept-level sensitivity. The method called TCAV (Testing Concept Activation Vector) \cite{InterpretabilityFeatureAttributionQuantitative} provides a post-hoc method to explain the global influence of a concept in a pre-trained model. ACE (Automatic Concept Extraction) \cite{AutomaticConceptbasedExplanations} was proposed to identify and filter interpretable concepts from the meaningful clusters of segments on the basis of TCAV. In \cite{ConceptWhiteningInterpretable2020}, Concept Whitening (CW) purposefully alters batch normalization layers to a concept whitening layer to learn an interpretable latent space. Especially, the whitening step in this method points out that the concept space needs to be preprocessed to better align concept vectors.

These concept-level interpretability methods, however, require the human ability to observe and extract semantically meaningful concepts \cite{AutomaticConceptbasedExplanations}. There are various ways to identify and extract concepts in collaboration with humans and systems \cite{AutomaticConceptbasedExplanations, NeuroCartographyScalableAutomaticVisualSummarization, zhao2021humanintheloopextraction, DASHVisualAnalyticsDebiasingImage,  ConceptExplorerVisualAnalysisConcept, ProtoSteerSteeringDeepSequence, AnchorVizFacilitatingSemanticData, ConceptVectorTextVisualAnalytics, VisualConceptProgrammingVisualAnalytics}. ConceptExtract \cite{zhao2021humanintheloopextraction} aimed to support concept extraction and classification in a human-in-the-loop workflow and visual tools. In DASH \cite{kwon2022dash}, problematic biases from irrelevant concepts can be identified through observations from users, which were proposed to be mitigated through random image generation using GAN techniques. ConceptExplainer \cite{ConceptExplorerVisualAnalysisConcept} was designed to explore the concept associations focusing on validating conceptual overlapping between classes, especially serving as a concept exploration tool for non-expert users. In \cite{VisualConceptProgrammingVisualAnalytics}, a self-supervised technique was proposed to automatically extract visual vocabulary to allow experts to refine the labeled data and understand the concepts.

Unlike existing work, our study proposes an interactive workflow of exploring concepts for the purpose of inspecting systematic errors and spurious concept associations behind them. Similar to \cite{WhatDidMyAILearn}, our human-in-the-loop workflow aims to promote the sensemaking of practitioners specifically in the problem of systematic errors where they can iteratively work on subsetting, contrasting patterns in instances, and hypothesizing spurious associations.


% All these methods including.. share the idea of defining a concept vector with a group of semantically coherent segments. While we take the approach of pre-processing steps on concept space in [] and sensitivity, we expand the utility of concept exploration towards inspecting model's false behaviors. In our study, we demonstrate that using concept interpretability can help not only interpreting the concept association towards misclassificaitons, and tracing back ... then removing the biases to further improve the quality of classification.






% \input{figure_scheme}
\begin{figure*}[ht]

\begin{minipage}{0.195\textwidth}
\centering
\includegraphics[height=1\textwidth]{images/pfd2cs/graphs/2.png}
\includegraphics[width=1\textwidth]{images/pfd2cs/images/2.png}
\includegraphics[width=1\textwidth]{images/pfd2cs/layouts/2.png}
\end{minipage}
\vspace{0.005\textwidth}
% \begin{minipage}{0.195\textwidth}
% \centering
% \includegraphics[height=1\textwidth]{images/pfd2cs/graphs/26.png}
% \includegraphics[width=1\textwidth]{images/pfd2cs/layouts/26.png}
% \includegraphics[width=1\textwidth]{images/pfd2cs/images/26.png}
% \end{minipage}
% \begin{minipage}{0.195\textwidth}
% \centering
% \includegraphics[height=1\textwidth]{images/pfd2cs/graphs/29.png}
% \includegraphics[width=1\textwidth]{images/pfd2cs/layouts/29.png}
% \includegraphics[width=1\textwidth]{images/pfd2cs/images/29.png}
% \end{minipage}
\vspace{0.005\textwidth}
\begin{minipage}{0.195\textwidth}
\centering
\includegraphics[height=1\textwidth]{images/pfd2cs/graphs/24.png}
\includegraphics[width=1\textwidth]{images/pfd2cs/images/24.png}
\includegraphics[width=1\textwidth]{images/pfd2cs/layouts/24.png}
\end{minipage}
\begin{minipage}{0.195\textwidth}
\centering
\includegraphics[height=1\textwidth]{images/pfd2cs/graphs/42.png}
\includegraphics[width=1\textwidth]{images/pfd2cs/images/42.png}
\includegraphics[width=1\textwidth]{images/pfd2cs/layouts/42.png}
\end{minipage}
\vspace{0.005\textwidth}
\begin{minipage}{0.195\textwidth}
\centering
\includegraphics[height=1\textwidth]{images/pfd2cs/graphs/37.png}
\includegraphics[width=1\textwidth]{images/pfd2cs/images/37.png}
\includegraphics[width=1\textwidth]{images/pfd2cs/layouts/37.png}
\end{minipage}
\vspace{0.005\textwidth}
% \begin{minipage}{0.195\textwidth}
% \centering
% \includegraphics[height=1\textwidth]{images/pfd2cs/graphs/67.png}
% \includegraphics[width=1\textwidth]{images/pfd2cs/layouts/67.png}
% \includegraphics[width=1\textwidth]{images/pfd2cs/images/67.png}
% \end{minipage}
% \begin{minipage}{0.195\textwidth}
% \centering
% \includegraphics[height=1\textwidth]{images/pfd2cs/graphs/1920.png}
% \includegraphics[width=1\textwidth]{images/pfd2cs/layouts/1920.png}
% \includegraphics[width=1\textwidth]{images/pfd2cs/images/1920.png}
% \end{minipage}
\begin{minipage}{0.195\textwidth}
\centering
\includegraphics[height=1\textwidth]{images/pfd2cs/graphs/1864.png}
\includegraphics[width=1\textwidth]{images/pfd2cs/images/1864.png}
\includegraphics[width=1\textwidth]{images/pfd2cs/layouts/1864.png}
\end{minipage}
% \begin{minipage}{0.195\textwidth}
% \centering
% \includegraphics[height=1\textwidth]{images/pfd2cs/graphs/1722.png}
% % \includegraphics[width=1\textwidth]{images/pfd2cs/layouts/1722.png}
% \includegraphics[width=1\textwidth]{images/pfd2cs/images/1722.png}
% \end{minipage}
% \begin{minipage}{0.195\textwidth}
% \centering
% \includegraphics[height=1\textwidth]{images/pfd2cs/graphs/1732.png}
% % \includegraphics[width=1\textwidth]{images/pfd2cs/layouts/1732.png}
% \includegraphics[width=1\textwidth]{images/pfd2cs/images/1732.png}
% \end{minipage}
% \begin{minipage}{0.195\textwidth}
% \centering
% \includegraphics[height=1\textwidth]{images/pfd2cs/graphs/1634.png}
% % \includegraphics[width=1\textwidth]{images/pfd2cs/layouts/1634.png}
% \includegraphics[width=1\textwidth]{images/pfd2cs/images/1634.png}
% \end{minipage}
% \begin{minipage}{0.195\textwidth}
% \centering
% \includegraphics[height=1\textwidth]{images/pfd2cs/graphs/1567.png}
% % \includegraphics[width=1\textwidth]{images/pfd2cs/layouts/1567.png}
% \includegraphics[width=1\textwidth]{images/pfd2cs/images/1567.png}
% \end{minipage}
% % \begin{minipage}{0.195\textwidth}
% % \centering
% % \includegraphics[height=1\textwidth]{images/pfd2cs/graphs/1163.png}
% % % \includegraphics[width=1\textwidth]{images/pfd2cs/layouts/1163.png}
% % \includegraphics[width=1\textwidth]{images/pfd2cs/images/1163.png}
% % \end{minipage}
% \begin{minipage}{0.195\textwidth}
% \centering
% \includegraphics[height=1\textwidth]{images/pfd2cs/graphs/1117.png}
% % \includegraphics[width=1\textwidth]{images/pfd2cs/layouts/1117.png}
% \includegraphics[width=1\textwidth]{images/pfd2cs/images/1117.png}
% \end{minipage}

\caption{Examples of synthetic scene graphs and corresponding generated traffic scenes from Cityscapes.}
\label{fig:results_cityscapes}
\end{figure*}
\begin{figure*}[ht]

% \begin{minipage}{0.195\textwidth}
% \centering
% \includegraphics[height=1\textwidth]{images/pfd2bdd/graphs/980.png}
% \includegraphics[width=1\textwidth]{images/pfd2bdd/images/980.png}
% \end{minipage}
% \vspace{0.005\textwidth}
\begin{minipage}{0.195\textwidth}
\centering
\includegraphics[height=1\textwidth]{images/pfd2bdd/graphs/54.png}
\includegraphics[width=1\textwidth]{images/pfd2bdd/images/54.png}
\includegraphics[width=1\textwidth]{images/pfd2bdd/layouts/54.png}
% \includegraphics[width=1\textwidth]{images/pfd2bdd/gt/54.png}
\end{minipage}
\vspace{0.005\textwidth}
\begin{minipage}{0.195\textwidth}
\centering
\includegraphics[height=1\textwidth]{images/pfd2bdd/graphs/1172.png}
\includegraphics[width=1\textwidth]{images/pfd2bdd/images/1172.png}
\includegraphics[width=1\textwidth]{images/pfd2bdd/layouts/1172.png}
% \includegraphics[width=1\textwidth]{images/pfd2bdd/gt/1172.png}
\end{minipage}
\vspace{0.005\textwidth}
\begin{minipage}{0.195\textwidth}
\centering
\includegraphics[height=1\textwidth]{images/pfd2bdd/graphs/1109.png}
\includegraphics[width=1\textwidth]{images/pfd2bdd/images/1109.png}
\includegraphics[width=1\textwidth]{images/pfd2bdd/layouts/1109.png}
% \includegraphics[width=1\textwidth]{images/pfd2bdd/gt/1109.png}
\end{minipage}
\vspace{0.005\textwidth}
\begin{minipage}{0.195\textwidth}
\centering
\includegraphics[height=1\textwidth]{images/pfd2bdd/graphs/1634.png}
\includegraphics[width=1\textwidth]{images/pfd2bdd/images/1634.png}
\includegraphics[width=1\textwidth]{images/pfd2bdd/layouts/1634.png}
% \includegraphics[width=1\textwidth]{images/pfd2bdd/gt/1634.png}
\end{minipage}
\vspace{0.005\textwidth}
\begin{minipage}{0.195\textwidth}
\centering
\includegraphics[height=1\textwidth]{images/pfd2bdd/graphs/772.png}
\includegraphics[width=1\textwidth]{images/pfd2bdd/images/772.png}
\includegraphics[width=1\textwidth]{images/pfd2bdd/layouts/772.png}
% \includegraphics[width=1\textwidth]{images/pfd2bdd/gt/772.png}
\end{minipage}
\vspace{0.005\textwidth}
% \begin{minipage}{0.195\textwidth}
% \centering
% \includegraphics[height=1\textwidth]{images/pfd2bdd/graphs/1001.png}
% \includegraphics[width=1\textwidth]{images/pfd2bdd/images/1001.png}
% \end{minipage}
% \vspace{0.005\textwidth}
% \begin{minipage}{0.195\textwidth}
% \centering
% \includegraphics[height=1\textwidth]{images/pfd2bdd/graphs/1045.png}
% \includegraphics[width=1\textwidth]{images/pfd2bdd/images/1045.png}
% \end{minipage}
% \vspace{0.005\textwidth}
% \begin{minipage}{0.195\textwidth}
% \centering
% \includegraphics[height=1\textwidth]{images/pfd2bdd/graphs/1069.png}
% \includegraphics[width=1\textwidth]{images/pfd2bdd/images/1069.png}
% \end{minipage}
% \vspace{0.005\textwidth}
% \begin{minipage}{0.195\textwidth}
% \centering
% \includegraphics[height=1\textwidth]{images/pfd2bdd/graphs/1078.png}
% \includegraphics[width=1\textwidth]{images/pfd2bdd/images/1078.png}
% \vspace{0.001\textwidth}
% \end{minipage}
% \begin{minipage}{0.195\textwidth}
% \centering
% \includegraphics[height=1\textwidth]{images/pfd2bdd/graphs/1147.png}
% \includegraphics[width=1\textwidth]{images/pfd2bdd/images/1147.png}
% \end{minipage}
% \vspace{0.005\textwidth}
% \begin{minipage}{0.195\textwidth}
% \centering
% \includegraphics[height=1\textwidth]{images/pfd2bdd/graphs/1168.png}
% \includegraphics[width=1\textwidth]{images/pfd2bdd/images/1168.png}
% \end{minipage}
% \vspace{0.005\textwidth}
% \begin{minipage}{0.195\textwidth}
% \centering
% \includegraphics[height=1\textwidth]{images/pfd2bdd/graphs/1198.png}
% \includegraphics[width=1\textwidth]{images/pfd2bdd/images/1198.png}
% \end{minipage}
% \vspace{0.005\textwidth}
% \begin{minipage}{0.195\textwidth}
% \centering
% \includegraphics[height=1\textwidth]{images/pfd2bdd/graphs/3.png}
% \includegraphics[width=1\textwidth]{images/pfd2bdd/images/3.png}
% \end{minipage}
% \vspace{0.005\textwidth}
% \begin{minipage}{0.195\textwidth}
% \centering
% \includegraphics[height=1\textwidth]{images/pfd2bdd/graphs/37.png}
% \includegraphics[width=1\textwidth]{images/pfd2bdd/images/37.png}
% \end{minipage}
% \vspace{0.005\textwidth}
% % \begin{minipage}{0.195\textwidth}
% % \centering
% % \includegraphics[height=1\textwidth]{images/pfd2bdd/graphs/67.png}
% % \includegraphics[width=1\textwidth]{images/pfd2bdd/images/67.png}
% % \end{minipage}
% % \vspace{0.005\textwidth}
% % \begin{minipage}{0.195\textwidth}
% % \centering
% % \includegraphics[height=1\textwidth]{images/pfd2bdd/graphs/74.png}
% % \includegraphics[width=1\textwidth]{images/pfd2bdd/images/74.png}
% % \end{minipage}
% % \vspace{0.005\textwidth}
% % \begin{minipage}{0.195\textwidth}
% % \centering
% % \includegraphics[height=1\textwidth]{images/pfd2bdd/graphs/81.png}
% % \includegraphics[width=1\textwidth]{images/pfd2bdd/images/81.png}
% % \vspace{0.001\textwidth}
% % \end{minipage}
% % \begin{minipage}{0.195\textwidth}
% % \centering
% % \includegraphics[height=1\textwidth]{images/pfd2bdd/graphs/112.png}
% % \includegraphics[width=1\textwidth]{images/pfd2bdd/images/112.png}
% % \end{minipage}
% % \vspace{0.005\textwidth}
% % \begin{minipage}{0.195\textwidth}
% % \centering
% % \includegraphics[height=1\textwidth]{images/pfd2bdd/graphs/127.png}
% % \includegraphics[width=1\textwidth]{images/pfd2bdd/images/127.png}
% % \end{minipage}
% % \vspace{0.005\textwidth}
% % \begin{minipage}{0.195\textwidth}
% % \centering
% % \includegraphics[height=1\textwidth]{images/pfd2bdd/graphs/131.png}
% % \includegraphics[width=1\textwidth]{images/pfd2bdd/images/131.png}
% % \end{minipage}
% % \vspace{0.005\textwidth}
% % \begin{minipage}{0.195\textwidth}
% % \centering
% % \includegraphics[height=1\textwidth]{images/pfd2bdd/graphs/694.png}
% % \includegraphics[width=1\textwidth]{images/pfd2bdd/images/694.png}
% % \end{minipage}
% % \vspace{0.005\textwidth}
% % \begin{minipage}{0.195\textwidth}
% % \centering
% % \includegraphics[height=1\textwidth]{images/pfd2bdd/graphs/138.png}
% % \includegraphics[width=1\textwidth]{images/pfd2bdd/images/138.png}
% % \end{minipage}
% % \vspace{0.005\textwidth}
% % \begin{minipage}{0.195\textwidth}
% % \centering
% % \includegraphics[height=1\textwidth]{images/pfd2bdd/graphs/151.png}
% % \includegraphics[width=1\textwidth]{images/pfd2bdd/images/156.png}
% % \end{minipage}
% % \vspace{0.005\textwidth}
% % \begin{minipage}{0.195\textwidth}
% % \centering
% % \includegraphics[height=1\textwidth]{images/pfd2bdd/graphs/158.png}
% % \includegraphics[width=1\textwidth]{images/pfd2bdd/images/158.png}
% % \end{minipage}
% % \vspace{0.005\textwidth}
% % \begin{minipage}{0.195\textwidth}
% % \centering
% % \includegraphics[height=1\textwidth]{images/pfd2bdd/graphs/217.png}
% % \includegraphics[width=1\textwidth]{images/pfd2bdd/images/217.png}
% % \end{minipage}
% % \vspace{0.005\textwidth}
% % \begin{minipage}{0.195\textwidth}
% % \centering
% % \includegraphics[height=1\textwidth]{images/pfd2bdd/graphs/244.png}
% % \includegraphics[width=1\textwidth]{images/pfd2bdd/images/244.png}
% % \vspace{0.001\textwidth}
% % \end{minipage}
% % \begin{minipage}{0.195\textwidth}
% % \centering
% % \includegraphics[height=1\textwidth]{images/pfd2bdd/graphs/330.png}
% % \includegraphics[width=1\textwidth]{images/pfd2bdd/images/330.png}
% % \end{minipage}
% % \vspace{0.005\textwidth}
% % \begin{minipage}{0.195\textwidth}
% % \centering
% % \includegraphics[height=1\textwidth]{images/pfd2bdd/graphs/422.png}
% % \includegraphics[width=1\textwidth]{images/pfd2bdd/images/422.png}
% % \end{minipage}
% % \vspace{0.005\textwidth}
% % \end{figure*}
% % \begin{figure*}[ht]
% % \begin{minipage}{0.195\textwidth}
% % \centering
% % \includegraphics[height=1\textwidth]{images/pfd2bdd/graphs/467.png}
% % \includegraphics[width=1\textwidth]{images/pfd2bdd/images/467.png}
% % \end{minipage}
% % \vspace{0.005\textwidth}
% % \begin{minipage}{0.195\textwidth}
% % \centering
% % \includegraphics[height=1\textwidth]{images/pfd2bdd/graphs/521.png}
% % \includegraphics[width=1\textwidth]{images/pfd2bdd/images/521.png}
% % \end{minipage}
% % \vspace{0.005\textwidth}
% % \begin{minipage}{0.195\textwidth}
% % \centering
% % \includegraphics[height=1\textwidth]{images/pfd2bdd/graphs/623.png}
% % \includegraphics[width=1\textwidth]{images/pfd2bdd/images/623.png}
% % \end{minipage}
% % \vspace{0.005\textwidth}
% \begin{minipage}{0.195\textwidth}
% \centering
% \includegraphics[height=1\textwidth]{images/pfd2bdd/graphs/627.png}
% \includegraphics[width=1\textwidth]{images/pfd2bdd/images/627.png}
% \end{minipage}
% \vspace{0.005\textwidth}
% % \begin{minipage}{0.195\textwidth}
% % \centering
% % \includegraphics[height=1\textwidth]{images/pfd2bdd/graphs/652.png}
% % \includegraphics[width=1\textwidth]{images/pfd2bdd/images/652.png}
% % \end{minipage}
% % \vspace{0.005\textwidth}
% % \begin{minipage}{0.195\textwidth}
% % \centering
% % \includegraphics[height=1\textwidth]{images/pfd2bdd/graphs/672.png}
% % \includegraphics[width=1\textwidth]{images/pfd2bdd/images/672.png}
% % \end{minipage}
% % \vspace{0.005\textwidth}
% % \begin{minipage}{0.195\textwidth}
% % \centering
% % \includegraphics[height=1\textwidth]{images/pfd2bdd/graphs/726.png}
% % \includegraphics[width=1\textwidth]{images/pfd2bdd/images/726.png}
% % \end{minipage}
% % \vspace{0.005\textwidth}
% % \begin{minipage}{0.195\textwidth}
% % \centering
% % \includegraphics[height=1\textwidth]{images/pfd2bdd/graphs/764.png}
% % \includegraphics[width=1\textwidth]{images/pfd2bdd/images/764.png}
% % \end{minipage}
% % \vspace{0.005\textwidth}
% % \begin{minipage}{0.195\textwidth}
% % \centering
% % \includegraphics[height=1\textwidth]{images/pfd2bdd/graphs/803.png}
% % \includegraphics[width=1\textwidth]{images/pfd2bdd/images/803.png}
% % \end{minipage}
% % \vspace{0.005\textwidth}
% % \begin{minipage}{0.195\textwidth}
% % \centering
% % \includegraphics[height=1\textwidth]{images/pfd2bdd/graphs/805.png}
% % \includegraphics[width=1\textwidth]{images/pfd2bdd/images/805.png}
% % \end{minipage}
% % \vspace{0.005\textwidth}
% % \begin{minipage}{0.195\textwidth}
% % \centering
% % \includegraphics[height=1\textwidth]{images/pfd2bdd/graphs/849.png}
% % \includegraphics[width=1\textwidth]{images/pfd2bdd/images/849.png}
% % \end{minipage}
% % \vspace{0.005\textwidth}
% % \begin{minipage}{0.195\textwidth}
% % \centering
% % \includegraphics[height=1\textwidth]{images/pfd2bdd/graphs/829.png}
% % \includegraphics[width=1\textwidth]{images/pfd2bdd/images/829.png}
% % \end{minipage}
% % \vspace{0.005\textwidth}
% % \begin{minipage}{0.195\textwidth}
% % \centering
% % \includegraphics[height=1\textwidth]{images/pfd2bdd/graphs/919.png}
% % \includegraphics[width=1\textwidth]{images/pfd2bdd/images/919.png}
% % \end{minipage}
% % \begin{minipage}{0.195\textwidth}
% % \centering
% % \includegraphics[height=1\textwidth]{images/pfd2bdd/graphs/927.png}
% % \includegraphics[width=1\textwidth]{images/pfd2bdd/images/927.png}
% % \end{minipage}
% % \vspace{0.005\textwidth}
% \begin{minipage}{0.195\textwidth}
% \centering
% \includegraphics[height=1\textwidth]{images/pfd2bdd/graphs/1142.png}
% \includegraphics[width=1\textwidth]{images/pfd2bdd/images/1142.png}
% \end{minipage}
% \vspace{0.005\textwidth}
% \begin{minipage}{0.195\textwidth}
% \centering
% \includegraphics[height=1\textwidth]{images/pfd2bdd/graphs/1876.png}
% \includegraphics[width=1\textwidth]{images/pfd2bdd/images/1876.png}
% \end{minipage}
% \vspace{0.005\textwidth}
% \begin{minipage}{0.195\textwidth}
% \centering
% \includegraphics[height=1\textwidth]{images/pfd2bdd/graphs/928.png}
% \includegraphics[width=1\textwidth]{images/pfd2bdd/images/928.png}
% \end{minipage}
% \vspace{0.005\textwidth}
% % \caption{Layouts (bottom) produced from the corresponding synthetic scene graphs (top) along with ground-truth layouts(middle). Colors follow Cityscapes protocol.}
% % \label{fig:results_bdd}
% % \end{figure*}
% % \begin{figure*}[ht]
% % \begin{minipage}{0.195\textwidth}
% % \centering
% % \includegraphics[height=1\textwidth]{images/pfd2bdd/graphs/1224.png}
% % \includegraphics[width=1\textwidth]{images/pfd2bdd/images/1224.png}
% % \end{minipage}
% % \vspace{0.005\textwidth}
% % \begin{minipage}{0.195\textwidth}
% % \centering
% % \includegraphics[height=1\textwidth]{images/pfd2bdd/graphs/1430.png}
% % \includegraphics[width=1\textwidth]{images/pfd2bdd/images/1430.png}
% % \end{minipage}
% % \vspace{0.001\textwidth}
% % \begin{minipage}{0.195\textwidth}
% % \centering
% % \includegraphics[height=1\textwidth]{images/pfd2bdd/graphs/1462.png}
% % \includegraphics[width=1\textwidth]{images/pfd2bdd/images/1462.png}
% % \end{minipage}
% % \begin{minipage}{0.195\textwidth}
% % \centering
% % \includegraphics[height=1\textwidth]{images/pfd2bdd/graphs/1509.png}
% % \includegraphics[width=1\textwidth]{images/pfd2bdd/images/1509.png}
% % \end{minipage}
% % \vspace{0.001\textwidth}
% % \begin{minipage}{0.195\textwidth}
% % \centering
% % \includegraphics[height=1\textwidth]{images/pfd2bdd/graphs/1549.png}
% % \includegraphics[width=1\textwidth]{images/pfd2bdd/images/1549.png}
% % \end{minipage}
% % \vspace{0.005\textwidth}
% % \begin{minipage}{0.195\textwidth}
% % \centering
% % \includegraphics[height=1\textwidth]{images/pfd2bdd/graphs/1566.png}
% % \includegraphics[width=1\textwidth]{images/pfd2bdd/images/1566.png}
% % \end{minipage}
% % \vspace{0.005\textwidth}
% % \begin{minipage}{0.195\textwidth}
% % \centering
% % \includegraphics[height=1\textwidth]{images/pfd2bdd/graphs/1592.png}
% % \includegraphics[width=1\textwidth]{images/pfd2bdd/images/1592.png}
% % \end{minipage}
% % \vspace{0.005\textwidth}
% % \begin{minipage}{0.195\textwidth}
% % \centering
% % \includegraphics[height=1\textwidth]{images/pfd2bdd/graphs/1578.png}
% % \includegraphics[width=1\textwidth]{images/pfd2bdd/images/1578.png}
% % \end{minipage}
% % \vspace{0.005\textwidth}
% % \begin{minipage}{0.195\textwidth}
% % \centering
% % \includegraphics[height=1\textwidth]{images/pfd2bdd/graphs/1716.png}
% % \includegraphics[width=1\textwidth]{images/pfd2bdd/images/1716.png}
% % \end{minipage}
% % \vspace{0.001\textwidth}
% % \begin{minipage}{0.195\textwidth}
% % \centering
% % \includegraphics[height=1\textwidth]{images/pfd2bdd/graphs/1722.png}
% % \includegraphics[width=1\textwidth]{images/pfd2bdd/images/1722.png}
% % \end{minipage}
% % \vspace{0.001\textwidth}
% % \begin{minipage}{0.195\textwidth}
% % \centering
% % \includegraphics[height=1\textwidth]{images/pfd2bdd/graphs/1792.png}
% % \includegraphics[width=1\textwidth]{images/pfd2bdd/images/1792.png}
% % \end{minipage}
% % \begin{minipage}{0.195\textwidth}
% % \centering
% % \includegraphics[height=1\textwidth]{images/pfd2bdd/graphs/1830.png}
% % \includegraphics[width=1\textwidth]{images/pfd2bdd/images/1830.png}
% % \end{minipage}
% % \vspace{0.005\textwidth}
% % \begin{minipage}{0.195\textwidth}
% % \centering
% % \includegraphics[height=1\textwidth]{images/pfd2bdd/graphs/1854.png}
% % \includegraphics[width=1\textwidth]{images/pfd2bdd/images/1854.png}
% % \end{minipage}
% % \vspace{0.005\textwidth}
% % \begin{minipage}{0.195\textwidth}
% % \centering
% % \includegraphics[height=1\textwidth]{images/pfd2bdd/graphs/1881.png}
% % \includegraphics[width=1\textwidth]{images/pfd2bdd/images/1881.png}
% % \end{minipage}
% % \vspace{0.005\textwidth}
% \begin{minipage}{0.195\textwidth}
% \centering
% \includegraphics[height=1\textwidth]{images/pfd2bdd/graphs/1920.png}
% \includegraphics[width=1\textwidth]{images/pfd2bdd/images/1920.png}
% \end{minipage}
% \vspace{0.005\textwidth}
% % \begin{minipage}{0.195\textwidth}
% % \centering
% % \includegraphics[height=1\textwidth]{images/pfd2bdd/graphs/1922.png}
% % \includegraphics[width=1\textwidth]{images/pfd2bdd/images/1922.png}
% % \end{minipage}
% % \vspace{0.001\textwidth}
\caption{Examples of synthetic scene graphs with corresponding BDD generated traffic scenes and semantic maps.}
\label{fig:results_bdd}
\end{figure*}
\section{Approach}

% In this section, we first define our video music pair in Section~\ref{pro_def}. 
In this section, we introduce the problem formulation of the video-music matching task in Sec.~\ref{pro_def}. Then, we detail our loss functions in Sec.~\ref{objective} for lifting cross-modality features to a shared space.
Finally, we introduce our proposed \frameworkname~framework for video-music matching in Sec.~\ref{v_e}.

\subsection{Problem Definition}\label{pro_def}
Given a music set $\mathcal{M}=\{m\}_{i=1}^{N_m}$ with $N_m$ music and a video set $\mathcal{V}=\{v\}_{i=1}^{N_v}$ with $N_v$ videos from training dataset, where $(m,v)$ denote \music~and video samples. Our video-music matching task is formulated as two transformation functions ${\rm f}(m) \rightarrow y_m$ and ${\rm g}(v) \rightarrow y_v$, and $(y_m,y_v) \in [1,N_m]$ denote the predicted matching \music{} indexes in the training dataset. For matching video and \music{} as a metric learning problem, we further adopt the  shared weight $\textbf{W}$ to lift video and music features to a shared embedding space: 
\begin{align}
    \begin{split}
        % &{\rm h}({\rm f}(m)) \rightarrow y_m~,\\
        &\mathbf{W}\cdot{\rm f}(m) \rightarrow y_m~,\\
        &\mathbf{W} \cdot g(v) \rightarrow y_v~.
    \end{split}
    \label{eq:f-g}
\end{align}
During the testing stage, video and \music{} can be matched by estimating the cosine similarity between their features, i.e., ${\rm cos}({\rm f}(m),{\rm g}(v))$. For \music{} already included in the training dataset, we can directly use $y_v$ as the matched one. In this paper, the transformation functions $\rm f$ and $\rm g$ are respectively referred to as a video and a music encoder of which output embedding dimensions are both chosen as $l$.


\subsection{Cross-Modality Training Objectives}\label{objective}
Softmax loss is commonly used in the classification problem for minimizing intra-class and maximizing inter-class distances, which is formulated as:
\begin{equation}
    \begin{aligned}
        L_S(x_i, \mathbf{W}) = -\log \frac{e^{\mathbf{W}_{y_i} \cdot x_i}}{\sum_{j=1}^N e^{\mathbf{W}_j \cdot x_i}}~,
    \end{aligned}
    \label{eq:softmax-loss}
\end{equation}
where $\mathbf{W}$ is the prototype of each class, i.e., the weight of the last layer in a network, $N$ is the total number of classes, $x_i$ is the feature, and $y_i$ is the ground truth class index of $x_i$. For further improving the decision boundary between different classes, CosFace~\cite{cosface} proposed to lift the features and prototypes to a hyper-sphere by introducing a scaling term $s$ and a margin $\mu$:

\begin{equation}
\begin{aligned}
    &\cos(\theta_{k, i}) = \frac{\mathbf{W}_{k} \cdot x_i}{\norm{\mathbf{W}_{k}} \cdot \norm{x_i}}~, \\
    L_{C}(x_i, \mathbf{W})=&-\log \frac{e^{s \cdot [\cos(\theta_{y_i,i})-\mu]}}{e^{s \cdot [\cos(\theta_{y_i,i})-\mu]} + \sum_{j \neq i}^N e^{s \cdot \cos(\theta_{j,i})}}~.
    \label{eq:cosface}
\end{aligned}
\end{equation}
Since Equation~\eqref{eq:cosface} is based on the angles between intra and inter classes (i.e., $\theta_{y_i,i}$ and $\theta_{j,i}$) in a normalized feature space, the features $x_i$ are thus optimized in a hyper-sphere.

\paragraph{Cross-Modality Lifting Loss.}
\label{loss_class}
\fuen{
To solve the video-music matching task as a metric learning problem, we aim at lifting video and music features to the same hyper-sphere. In this way, we can match the videos to their most appropriate music by calculating the cosine similarity between them. Hence, we propose ``Cross-Modality Lifting Loss'' by adopting a shared prototype $\mathbf{W}$ for both video and music features and considering modality-to-prototype distances:
\begin{equation}
\begin{aligned}
    L_{LL} ({\rm g}(v), {\rm f}(m), \mathbf{W}) = L_C({\rm g}(v), \mathbf{W}) + \alpha L_C({\rm f}(m), \mathbf{W})~,
    \label{eq:LL}
\end{aligned}
\end{equation}
where $(v,m)$ are input music and video, $(\rm g,f)$ are the transformation functions as described in Equation~\eqref{eq:f-g}, and $\alpha$ is a hyper-parameter. In practice, the transformation functions are implemented as two independent feature encoders for video and music.}


\paragraph{Cross-Modality Similarity Loss.}
\label{loss_sim}
\begin{table*}[ht]  
\begin{center}   
\resizebox{\textwidth}{!}{\begin{tabular}{l||r||cc|cc|cc|
cc}  
  
\toprule % Toprule applied here  
  
\textbf{Dataset}&\multicolumn{1}{c||}{} & \multicolumn{2}{c|}{\textbf{Training}} & \multicolumn{2}{c|}{\textbf{Validation}} & \multicolumn{2}{c|}{\textbf{\closesettable}} & \multicolumn{2}{c}{\textbf{\opensettable}}\\
% & \multicolumn{2}{c|}{} & \multicolumn{2}{c|}{} & \multicolumn{2}{c|}{\textbf{(Weak generalization)}} & \multicolumn{2}{c|}{\textbf{(Strong generalization)}} &\\
% 
&\#Video &\#Music & VpM  & \#Music & VpM  & \#Music & VpM  & \#Music & VpM \\  
  % 
\midrule % Midrule applied here  
   % 
 \ourdataset{} & 150000 &265 & 400 & 265 & 20 & 265 & 80 & 125 & 140\\ 
 \midrule % Midrule applied here 
 Youtube-8M~\cite{abu2016youtube} & 5678 & 4654 & 1 & NA & NA & NA & NA & 1024 & 1024\\   
  % 
\bottomrule % Bottomrule applied here  
  % 
\end{tabular}}%
\caption{\textbf{Details of datasets.} The music for training, validation and \closesetname{} set is the same. Because the video-music pair is one-to-one mapping in Youtube-8M~\cite{abu2016youtube} dataset, validation and \closesetname{} are not included. \textbf{VpM} indicates the number of matching videos for each music.}  
\label{tab:table_msvd} 
\end{center}  
\end{table*}  
\fuen{
Although our proposed \faceloss~can effectively minimize the intra and maximize the inter class distances, we found that only considering such a modality-to-prototype distance still leads to a sub-optimal performance since videos and music are eventually matched based on their features instead of their prototype during the testing stage. To this end, we follow \cite{suris2018cross} to adopt ``Cross-Modality Similarity Loss'' aiming at addressing the video-to-music feature distances for improving our downstream video-music matching performance:
\begin{equation}
\begin{aligned}
    \cos(x_i, x_j) &= \frac{x_i \cdot x_j}{\norm{x_i} \cdot \norm{x_j}}~, \\
    L_{SL} ({\rm g}(v), {\rm f}(m), {\rm f}(m^\prime)) &= max[\tau, \cos({\rm g}(v), {\rm f}(m^\prime))] \\
    &- \cos({\rm g}(v), {\rm f}(m))~,
    \label{eq:SL}
\end{aligned}
\end{equation}
where $(v, m)$ indicates a positive video-music pair queried according to the ground truth music index of $v$, $(v, m^\prime)$ indicates a negative one randomly sampled from the dataset, and $\tau$ is a selected margin value. With \simloss, we can apply direct constraints in-between the predicted video and music features, and consider modality-to-modality distances, which provides consistent video-music matching schemes under both training and testing stages.
}


\subsection{\frameworkname~Framework}
\paragraph{Video Encoder.}
\label{v_e}
\fuen{To extract video features, we adopt a R(2+1)D ResNet-18~\cite{tran2018closer}, pretrained on Kinetics-400~\cite{kay2017kinetics} and followed by a fully-connected layer to infer the video features with embedding size $l$, as our video encoder ($\rm g$ in Equation~\eqref{eq:f-g}).}

\paragraph{Music Encoder.}
\label{m_e}
\fuen{we firstly calculate the Mel spectrograms of the input \music{} and adopt a ResNet-18~\cite{he2016deep} pretrained on ImageNet~\cite{deng2009imagenet} to extract the high-level music features. In addition, inspired by \cite{yi2021cross}, we use openSMILE~\cite{eyben2010opensmile} to extract the low-level music features, including MFCC, voice intensity, pitch, etc. The low-level and high-level features are then fused by concatenation and we infer the final music features by passing the fused features into an additional fully-connected layer with embedding size $l$. We refer this music encoder as $\rm f$ in Equation~\eqref{eq:f-g}.}

\paragraph{Training and Testing.}
\label{matching}
\fuen{During each training iteration, we use a pair of video and music for calculating $L_{LL}$ (Equation~\eqref{eq:LL}). For $L_{SL}$ (Equation~\eqref{eq:SL}), we randomly sample a negative music for calculation. The final training objective is established as:
\begin{equation}
\begin{aligned}
L(v, m, m^\prime) = L_{LL}({\rm g}(v), f(m), \mathbf{W})& + \beta L_{SL}({\rm g}(v), {\rm f}(m),\\
{\rm f}(m^\prime))
% &+ L_{SL}({\rm g}(v), {\rm f}(m^\prime))]~,
\label{eq:L-final}
\end{aligned}
\end{equation}
where $m^\prime$ indicates a randomly-sampled negative music, and $\beta$ is a hyper-parameter.}

\fuen{During the testing stage, the \closesetname~and \opensetname~sets are evaluated with different schemes, as illustrated in the right part of Fig.~\ref{fig:framework}.
For \closesetname~set, the video features are inferred from our video encoder $\rm g$, while the music features are directly pulled from the trained prototype since the trained prototype can represent the feature center of each training music, which can significantly reduce the inference time of \opensetname~features. 
For \opensetname~set, the music features are extracted from our music encoder $\rm f$.}

\fuen{Finally, we match the videos and music by calculating the cosine similarity between their features and select the top 20 music clips with the highest similarities as the matched music list.}
\section{Experiments}
\label{sec:experiments}

\subsection{Setup}
\textbf{Datasets.} We evaluate RFFR with four challenging datasets specifically designed for deepfake detection. We adopt the high quality (HQ) version of Faceforensics++ (FF)~\cite{ff} for training our deepfake detector. Faceforensics++ includes videos of real faces as well as four subsets of fake faces, each manipulated with a different algorithm, namely Deepfakes (DF), Face2Face (F2F), FaceSwap (FSW) and NeuralTextures (NT). We also utilize the test set of Celeb-DF~\cite{celeb-df} and DFDC~\cite{dfdc} for evaluating the cross-dataset performance of our model. Finally, in addition to real faces of Faceforensics++, we adopt the real face images from ForgeryNet (FN)~\cite{forgerynet} for learning RFFR, which helps improve representation learning with additional data.

\textbf{Implementation Details.} We extract the frames from all video datasets and use RetinaFace~\cite{retinaface} to detect and align the faces. All images are scaled to the size of $224 \times 224$. For our RFFR model, we adopt a base version of Masked Autoencoder (MAE)~\cite{mae} and train it on real faces with a batch size of $128$. Following MAE, we set the learning rate at $7.5 \times 10^{-5}$ and adjust it with a schedule with warmup and cosine decay. By default, we train this model with the real faces from both FF~\cite{ff} and FN~\cite{forgerynet}. 

For training the deepfake detector, we divide each image with $k = 4$ (Refer to Appendix for the motivation of choosing $k$). Each block enters the classifier with a probability of $p = 0.25$, and the residual images are amplified by $\alpha=4$. No data augmentation is applied to the images. We initialize both branches of Vision Transformer with ImageNet-pretrained weights and train them with a learning rate of $2 \times 10^{-5}$. During testing, we iteratively mask and restore all blocks to obtain a full residual image for the detector to process. We evaluate the testing results with AUC (Area Under Curve). 

\subsection{Cross-domain performance evaluation}
In this section, we test the performance of our RFFR-based deepfake detector with cross-manipulation and cross-dataset evaluations. 

\textbf{Cross-manipulation evaluations.} We train our deepfake detector on each subset of Faceforensics++ and test on all four subsets to demonstrate our model's ability to identify different manipulations, including those not seen during training. \emph{We adopt the HQ version of FF for both training and testing, and only use one frame every video for testing.} We compare our results with state-of-the-art image-based methods Multi-Attention~\cite{multiatt}, DCL~\cite{dcl}, RECCE~\cite{recce} and UIA-ViT~\cite{uia}. We ran the public code of RECCE and UIA-ViT to produce results under the same setting.

In~\cref{tab:cross-manipulation}, we show that our method outperforms the state-of-the-art methods under most settings, with a maximum improvement of $10.25\%$ (F2F $\rightarrow$FSW). Meanwhile, our model remains effective under the four intra-domain settings, which are shown in gray. The method tends to slightly underperform when trained on NeuralTextures, likely because its manipulation patterns only exist in certain small regions, and may be neglected during our block sampling. Nevertheless, compared to existing methods, our deepfake detector yields much better overall performances. 

\begin{table}[t]
\setlength\tabcolsep{4.5pt} 
\caption{Cross-manipulation performances in terms of AUC(\%) compared with previous methods. Classifiers are trained on one subset of FF and tested on all four subsets. Intra-domain results are marked in gray. We ran the public code of methods marked with "*" to produce results under identical settings \emph{(HQ for training and single frames for testing).}}
\vspace{-1.5em}
\label{tab:cross-manipulation}
\begin{center}  
\scalebox{0.80}{
\begin{tabular}{c|l|cccc|c}
\toprule
Training &\multirow{2}*{Method} & \multicolumn{4}{c|}{Test data} & \multirow{2}*{Avg} \\
\cmidrule(lr){3-6}
     data  &            ~                   & DF    & F2F   & FSW   & NT    & ~   \\
     
\midrule
\multirow{5}*{DF}
& MultiAtt~\cite{multiatt} & \cellcolor{Gray}99.92 & 75.23 & 40.61 & 71.08 & 71.71                \\ 
& DCL~\cite{dcl}       & \cellcolor{Gray}\textbf{99.98} & \textbf{77.13} & 61.01 & 75.01 & 78.28              \\
& RECCE*~\cite{recce}     & \cellcolor{Gray}99.19 & 74.39 & 57.42 & \textbf{85.04} & 79.01                \\ 
& UIA-ViT*~\cite{uia}  & \cellcolor{Gray}99.39      &   74.44    &   53.89    &   70.92    & 74.66 \\ 
& Ours  & \cellcolor{Gray}99.19 & 76.61 & \textbf{68.96} & 74.83 & \textbf{79.90}            \\ 
       
\midrule
\multirow{5}*{F2F}
        & MultiAtt~\cite{multiatt}       & 86.15 & \cellcolor{Gray}99.13 & 60.14 & 64.59 & 77.50 \\
        & DCL~\cite{dcl}       & 91.91 & \cellcolor{Gray}99.21 & 59.58 & 66.67 & 79.34 \\
       & RECCE*~\cite{recce}       & 88.04 & \cellcolor{Gray}98.93 & 67.35 & 74.16 & 82.12 \\
       & UIA-ViT*~\cite{uia}       & 83.39 & \cellcolor{Gray}98.32 & 68.37 & 67.17 & 79.31 \\
       & Ours                                  & \textbf{93.75} & \cellcolor{Gray}\textbf{99.61} & \textbf{78.62} & \textbf{79.56} & \textbf{87.81} \\

\midrule
\multirow{5}*{FSW}
& MultiAtt~\cite{multiatt} & 64.13 & 66.39 & \cellcolor{Gray}99.67 & 50.10 & 70.07              \\
& DCL~\cite{dcl}           & 74.80 & 69.75 & \cellcolor{Gray}99.90 & 52.60 & 74.26              \\
& RECCE*~\cite{recce}       & 66.66 & 73.66 & \cellcolor{Gray}\textbf{99.76} & \textbf{57.46} & 74.39               \\

& UIA-ViT*~\cite{uia}       &   81.02    &   66.30    & \cellcolor{Gray}99.04      &   49.26    & 73.91 \\ 
& Ours                                           & \textbf{87.46} & \textbf{75.96} & \cellcolor{Gray}99.42 & 55.87 & \textbf{79.68}            \\ 

\midrule
\multirow{5}*{NT}
& MultiAtt~\cite{multiatt} & 87.23 & 75.33 & 48.22 & \cellcolor{Gray}98.66 & 77.36                \\
& DCL~\cite{dcl}      & 91.23 & 79.31 & 52.13 & \cellcolor{Gray}\textbf{98.97} & 80.41                \\
& RECCE*~\cite{recce}    & \textbf{90.20}  & 76.65 & \textbf{58.06} & \cellcolor{Gray}97.17 & \textbf{80.52}                \\
 & UIA-ViT*~\cite{uia}  &    79.37   &   67.98    &   45.94    &\cellcolor{Gray}94.59       & 71.97 \\
 & Ours     & 84.31 & \textbf{81.04} & 54.67 & \cellcolor{Gray}96.19 & 79.05          \\
       
\bottomrule
\end{tabular}}
\vspace{-2em}
\end{center}
\end{table}

\textbf{Cross-dataset evaluations.} We train our model on the Faceforensics++ dataset and evaluate its performance on the test sets of Celeb-DF\cite{celeb-df} and DFDC~\cite{dfdc}. Specifically, following the previous practice in~\cite{lip}, we validate the model on Celeb-DF and use the selected model to test on DFDC.  \emph{We adopt the HQ version of FF for training, and only use one frame every video for testing.} Under the same setting, we ran the public code of RECCE~\cite{recce}, UIA-ViT~\cite{uia} and SBI~\cite{sbi} to produce corresponding results. In Table~\ref{tab:cross-dataset}, we show a competitive performance with existing image-based methods, signaling satisfying adaptability of RFFR to different datasets, especially high quality datasets like Celeb-DF. 
  
SBI~\cite{sbi} is a recent powerful deepfake detection method. By utilizing a hand-crafted blending algorithm to produce diverse fake samples, it achieves highly competitive performances on datasets including Celeb-DF. We show that by training on fake samples generated by SBI, our approach can further improve upon their state-of-the-art result. 

\begin{table}[]
\setlength\tabcolsep{4.5pt} 
\caption{Cross-dataset performances in terms of AUC(\%) compared with previous methods. Classifiers are trained on FF and tested on Celeb-DF and DFDC. We ran the public code of methods marked with "*" to produce results under identical settings \emph{(HQ for training and single frames for testing).}}
\vspace{-1em}
\label{tab:cross-dataset}
\begin{center}  
\scalebox{0.90}{
\begin{tabular}{l|cc}
\toprule
\multirow{2}*{Method} & \multicolumn{2}{c}{Test data}\\
\cmidrule{2-3}
        ~                           &     Celeb-DF         &  DFDC \\
\midrule
      Xception~\cite{xception}  &     65.30       &    -  \\
      Face X-ray~\cite{xray}          &     74.20       &     70.00 \\
      MultiAtt~\cite{multiatt}        &     67.44       &     67.34 \\
      SPSL~\cite{SPSL}                &     76.88        &   -  \\
      SOLA~\cite{sola}                &       76.02         &  -    \\
      SLADD~\cite{sladd}              &    79.70       &  -  \\
      RECCE*~\cite{recce}             &     68.94       &   68.34   \\
      UIA-ViT*~\cite{uia}             &     80.31      &   67.93   \\
      SBI*~\cite{sbi}                       &       86.46     &   66.60     \\
\midrule
 	Ours                                      &   81.97  & \textbf{72.08}  \\
    Ours + SBI~\cite{sbi}                  &  \textbf{88.98}           &    67.84   \\
\bottomrule
\end{tabular}}
\vspace{-2.5em}
\end{center}
\end{table}

\subsection{Ablation Study}
\label{ablation}

In this section, we analyze the effect of our implementations for RFFR learning and deepfake detection. 

\textbf{Effect of the training data for RFFR.} The effectiveness of deepfake detection with RFFR depends on the quality of representation learning, where the real faces plays an important role. In this experiment, we examine the effect of scaling the real face dataset for representation learning. As a baseline, we learn RFFR with only real faces from Faceforensics++ (FF), the same data we use for the downstream classification tasks. Meanwhile, another model is supplemented with real faces from both FF and ForgeryNet (FN), a significantly larger and more diverse dataset. We train deepfake detectors on the F2F subset of FF with residual images produced by these two models. In Table~\ref{tab:data}, we demonstrate that including the extra dataset of ForgeryNet for learning RFFR consistently improves the performances of the deepfake detector in all tests, creating a maximum performance gain of $9.57\%$  in terms of AUC (F2F $\rightarrow$ NT).

We note that learning RFFR with FF already allows our deepfake detector to outperform the state-of-the-arts. Nevertheless, learning with extra data enhances the efficacy of our real face foundation representations, and further improves the downstream task of deepfake detection. Therefore, refining the representation learning of real faces, especially with large-scale datasets, could be a viable path for further improving generalized deepfake detection. 

In addition, we examine the scalability of RECCE under the same setting, considering that RECCE~\cite{recce} also involves learning to reconstruct real samples for deepfake detection. However, their performance gain is less significant than ours. Although the reconstruction branch of RECCE~\cite{recce} is able to highlight forgery cues with residual images, they tend to involve more background noise caused by imperfect reconstructions, as depicted in~\cref{fig:unet_comparison},. This undermines the ability of residual images to expose artifacts for deepfake detection. 

\begin{table}[t]
\setlength\tabcolsep{4.5pt} 
\caption{Deepfake detection performances of RECCE~\cite{recce} and our method with different real face dataset, namely the real faces from Faceforensics++ (FF) alone, and FF combined with ForgeryNet (FF + FN). Classifiers are trained on F2F and tested on four subsets of FF. We present the results in AUC (\%).  }
\vspace{-1.5em}
\label{tab:data}
\begin{center}  
\scalebox{0.90}{
\begin{tabular}{c|c|cccc|c}
\toprule
\multirow{2}*{Method} & Real face  & \multicolumn{4}{c|}{Test data} & \multirow{2}*{Avg} \\
\cmidrule(lr){3-6}
&dataset  &      DF    & F2F   & FSW   & NT    & ~   \\
    \midrule
\multirow{2}*{RECCE~\cite{recce}}&FF           & 88.04          & 98.93          & 67.35          & 74.16          & 82.12          \\
&FN + FF &  90.12       & 99.24       & 69.89    & 79.59     & 84.71		\\
    \midrule
\multirow{2}*{Ours}&FF           & 90.16          & 98.56          & 74.10          & 69.99          & 83.20          \\
&FN + FF & \textbf{93.44}       & \textbf{99.61}        & \textbf{78.62}       & \textbf{79.56}        & \textbf{87.81}		\\
\bottomrule
\end{tabular}}
\vspace{-1em}
\end{center}
\end{table}

\textbf{Effect of masked image modeling for RFFR.} We analyze the effect of using MIM-based residual images for deepfake detection. We train a UNet-based autoencoder (AE) to learn the reconstruction of real faces and obtain residual images. Our MIM-trained inpainting model and the AE are compared on the quality of reconstruction in~\cref{fig:unet_comparison}. Note that despite being trained with real faces, the AE "generalizes" well to fake images, preserving delicate details, including the artifacts caused by manipulations. Such generalization leaves the residual images empty with little information. 

\begin{figure}
\centering
  \includegraphics[width=0.9\columnwidth]{figs/compare_ICCV_Final.pdf}
  \vspace{-1em}
   \caption{Reconstruction results and residual images of the autoencoder (AE), RECCE~\cite{recce} and our inpainting model. AE reconstructs both images perfectly, leaving no information in residual images. RECCE~\cite{recce} suffers from insufficient training. Our model successfully highlights potential artifacts in the residual image of only the fake face, and therefore can best facilitate deepfake detection. }
\vspace{-1em}
\label{fig:unet_comparison}
\end{figure}

Masked image modeling enables our model to learn better real face representations and inpaint fake faces with real textures instead of artifacts. In the downstream task of deepfake detection,  our classifier generalizes significantly better than the AE-based classifier, which performs only marginally better than learning with no residuals (detailed in Appendix). Both the reconstruction results and the downstream performance confirm the validity of our choice to learn RFFR with MIM instead of direct reconstruction. 


\textbf{Effect of classifier backbone.} In Table~\ref{tab:backbone}, we present the deepfake detection results of vanilla Xception~\cite{xception} and Vision Transformer (ViT)~\cite{vit}, both trained with full original images. The models are trained with the F2F subset of FF and tested on all four subsets. While a larger backbone increases a deepfake detector's generalization performance in some cases, it is not the primary factor of our performance improvement. Instead, it is the residual input aided by RFFR that leads the performance gain.

\begin{table}[t]
\setlength\tabcolsep{4.5pt} 
\caption{Comparing ours results with vanilla backbones. We present the results in AUC (\%).  }
\label{tab:backbone}
\vspace{-1.5em}
\begin{center}  
\scalebox{0.90}{
\begin{tabular}{c|c|cccc|c}
\toprule
Training  &  \multirow{2}*{Method}    &   \multicolumn{4}{c|}{Test Data} & \multirow{2}*{Avg} \\
\cmidrule(lr){3-6}
 data  &   ~  &   DF    & F2F   & FSW   & NT    & ~   \\
    \midrule
\multirow{3}*{F2F} & Xception~\cite{xception} & 84.94          & 99.26          & 58.82          & 71.19          & 78.55          \\
                                   & ViT~\cite{vit}      & 84.25          & 97.89          & 65.53          & 65.18          & 78.21          \\
                                   & Ours     & \textbf{93.44} & \textbf{99.61} & \textbf{78.62} & \textbf{79.56} & \textbf{87.81} \\
\bottomrule
\end{tabular}}
\vspace{-1.5em}
\end{center}
\end{table}

\textbf{Effect of classifier design.} We compare different variants of our classifier design. Specifically, we analyze the performance gains brought by the introduction of two branches and the random input mechanism. We test six variants of our classifier by training them with the F2F subset of FF and testing with the FSW subset. The settings of these variants are specified by the input data they accept, as shown in~\cref{tab:classifier}. 

\begin{table}[t]
\caption{Deepfake detection performances with classifiers of different inputs in terms of AUC (\%). We train the classifiers on F2F and test on FSW.}
\label{tab:classifier}
\vspace{-1.5em}
\begin{center}
\begin{tabular}{c|c|c|c|c}
\toprule
\multicolumn{2}{c|}{Original Image} & \multicolumn{2}{c|}{Residual Image} & \multirow{2}*{AUC (\%)} \\
\cline{1-4}
               Full        &             Random           &          Full          &          Random          &   ~\\
 \hline
\checkmark        &                                       &                            &                                   &  65.53\\
% \hline
                              &                                      &   \checkmark    &                                   &  66.30  \\
 %\hline
\checkmark        &                                      &   \checkmark    &                                   &  71.48  \\
 %\hline
                             &       \checkmark          &                             &                                   &  70.76  \\
%\hline
                             &                                       &                             &      \checkmark       &  68.10  \\
 %\hline
                             &        \checkmark         &                             &      \checkmark       &  \textbf{78.62}  \\
\bottomrule
\end{tabular}
\vspace{-2em}
\end{center}
\end{table}

\begin{table*}[t]
\setlength\tabcolsep{4.5pt} 
\caption{Deepfake detection performances of validated and non-validated models. Classifiers are trained on F2F and tested on four subsets of FF. We present the results and the performance gaps in AUC (\%). Second best results are underlined. }
\label{tab:validation}
\vspace{-1em}
\begin{center}  
\scalebox{0.90}{
\begin{tabular}{c|c|llll|l}
\toprule
\multirow{2}*{Method}  & \multirow{2}*{Validated} & \multicolumn{4}{c|}{Test Data} & \multirow{2}*{Avg} \\
\cmidrule(lr){3-6}
~                   &                      ~                   &      DF               & F2F                    & FSW                 & NT                    & ~   \\
    \midrule
\multirow{2}*{Xception\cite{xception}} &   \checkmark    & 84.94                 & 99.26                & 58.82                 & 71.19                & 78.55            \\
~ &                                             -                              & 83.08   (- 1.86) & 99.12   (- 0.14) & 46.63   (- 12.19) & 64.93   (- 6.26)  & 73.44   (- 5.11)  \\
 \hline
 \multirow{2}*{RECCE\cite{recce}} &\checkmark               & 88.04                & 98.93                 & 67.35                & 74.16                & 82.12            \\
 ~&                                                -                  & 74.51   (- 8.57) & 99.22   (+ 0.29)  & 50.17   (- 17.18) & 59.46   (- 14.70)  & 70.84   (- 11.28) \\
 \hline
\multirow{2}*{Ours} &    \checkmark  & \textbf{93.44}            & \textbf{99.61}            & \textbf{78.62}            & \textbf{79.56}            & \textbf{87.81}            \\
 ~&  - & \underline{91.56} (- 1.88) & \underline{99.39}   (- 0.22) & \underline{76.00}   (- 2.62)  & \underline{76.41} (- 3.15) & \underline{85.84}   ( - 1.97)    \\
\hline
\end{tabular}}
\vspace{-2em}
\end{center}
\end{table*}

We treat the vanilla ViT with full original image input as a baseline, which achieves an AUC of $65.53\%$. By switching to accept the full residual images, we obtain a $0.77\%$ performance gain. Combining the two modalities to form a dual-branch classifier further increases our result to $71.48\%$. This demonstrates that the artifacts are better exploited when both the original and the residual images enter the classifier, and are used as references to each other. Therefore, both modalities should be considered for classification. 

In addition, we improve on the test by merely modifying the baseline ViT to accept randomly selected original image blocks. This results in a $5.23\%$ increase in performance. Similarly, changing full residual input to random residual blocks also results in a $1.8\%$ improvement. These observations confirm our hypothesis in \cref{sec:method_deepfake_detection} that models benefit from learning with random inputs, which prevents the model from only focusing on the most prominent features in an image, and forces it to learn from subtle artifacts. 

Finally, bringing in the random input mechanism for the dual-branch classifier completes our full implementation, which maximally exploits the artifacts exposed by RFFR and achieves the best performance of $78.62\%$. 



\subsection{Validation-free Model Selection}
\label{sec:validation-free}

\begin{figure}
\centering
  \includegraphics[width=0.5\textwidth]{figs/validation-free_ICCV_Final.png}
  \vspace{-1.5em}
   \caption{Comparing the validation curves of RFFR-based deepfake detector and previous methods. Detectors are trained on the F2F subset of FF for $15k$ iterations and validated on four different subsets. (a) to (d) correspond to experiments on DF, F2F, FSW and NT.  Results are reported in AUC (\%). All three methods perform well when validated on F2F. However, under cross-manipulation settings, only our method avoids overfitting during training. The curves are smoothed for better visibility.}
\label{fig:validation-free}
\vspace{-1em}
\end{figure}

Models expected to generalize to other domains benefit from target domain validations~\cite{domainbed}. By frequently performing model validation, we can select the model  that best suits the detection of target manipulation, resulting in high performance on the test set. While using such an \textit{oracle} could be acceptable for the early development of cross-domain algorithms~\cite{domainbed}, it is not ideal for applications, as labeled data of unseen manipulation is usually not available. 

In this section, we demonstrate the potential of our deepfake detector to circumvent this practice and therefore avoid the need for extra validation data. As shown in \cref{tab:validation}, we train our classifier on F2F for 15k iterations and directly use the final model for testing. Simultaneously, we employ four validation sets to select the models with the best validation performances on target data. All validated and non-validated models are tested under the same conditions. We report all results on the target test sets in Table~\ref{tab:validation}. The performance gaps between validated and non-validated models are reported along with the test results. Although our non-validated models are not performing as well as those selected with a validation set, we show that our model remains effective on target data, with a maximum performance drop of $3.15\%$ and an average drop of $1.97\%$. However, previous methods~\cite{xception, recce} suffer from significantly larger performance drops when evaluated under the same procedure. 

To take a closer look at how the cross-manipulation performances vary during training, we train the deepfake detectors again with F2F. We test the AUC performances on all target subsets every 50 iterations to produce validation curves in \cref{fig:validation-free}. Our RFFR-based deepfake detector consistently maintains a high performance long after its peaks without serious overfitting. On the contrary, both previous methods compared here overfit quickly after reaching their highest target domain performances. In addition, compared methods exhibit large fluctuations across different evaluations, while our model remains stable. This suggests that with RFFR, our model focuses exclusively on generalizable features which fall outside the distribution of RFFR. Such resistance to overfitting guarantees our model a satisfying performance even when labeled validation sets are not available, which is generally expected in practice. We present more results on validation-free evaluations in Appendix.

\section{Conclusion}
In this work, we propose a method to ease data generation for realistic traffic scenes from domain agnostic scene representation called scene graphs instead of using photo-realistic rendering. We utilize synthetic scene graphs enhanced by spatial attributes (\textit{z}) and spatial relations (e.g., \textit{behind}). Furthermore, we introduce an unsupervised approach for realistic image generation from synthetic scene graphs. The approach shows convincing generation results as demonstrated in the proposed qualitative evaluation. We also show the effectiveness of our method through traffic scene manipulation and validation on a downstream task.

\section{Acknowledgment}
The research leading to these results is funded by the German Federal Ministry for Economic Affairs and Energy within the project “KI Absicherung – Safe AI for Automated Driving". The authors would like to thank the consortium for the successful cooperation.
% \newpage
{
\bibliographystyle{ieee}
\bibliography{literature}
}

% \section{Appendix}
% \begin{figure*}[t!]
\begin{minipage}{0.325\textwidth}
\centering
\includegraphics[height=0.5\textwidth]{images/pfd2bdd/graphs/67.png}
\includegraphics[height=0.5\textwidth]{images/pfd2bdd/graphs/88.png}
\includegraphics[height=0.5\textwidth]{images/pfd2bdd/graphs/127.png}
\includegraphics[height=0.5\textwidth]{images/pfd2bdd/graphs/138.png}
\includegraphics[height=0.5\textwidth]{images/pfd2bdd/graphs/156.png}
\includegraphics[height=0.5\textwidth]{images/pfd2bdd/graphs/158.png}
% \includegraphics[height=0.5\textwidth]{images/pfd2bdd/graphs/171.png}
\includegraphics[height=0.5\textwidth]{images/pfd2bdd/graphs/217.png}
\caption*{SG}
\end{minipage}
\begin{minipage}{0.325\textwidth}
\includegraphics[width=1\textwidth]{images/pfd2bdd/images/67.png}
\includegraphics[width=1\textwidth]{images/pfd2bdd/images/88.png}
\includegraphics[width=1\textwidth]{images/pfd2bdd/images/127.png}
\includegraphics[width=1\textwidth]{images/pfd2bdd/images/138.png}
\includegraphics[width=1\textwidth]{images/pfd2bdd/images/156.png}
\includegraphics[width=1\textwidth]{images/pfd2bdd/images/158.png}
% \includegraphics[width=1\textwidth]{images/pfd2bdd/images/171.png}
\includegraphics[width=1\textwidth]{images/pfd2bdd/images/217.png}
\caption*{\textit{BDD}}
\end{minipage}
\begin{minipage}{0.325\textwidth}
\includegraphics[width=1\textwidth]{images/pfd2cs/images/67.png}
\includegraphics[width=1\textwidth]{images/pfd2cs/images/88.png}
\includegraphics[width=1\textwidth]{images/pfd2cs/images/127.png}
\includegraphics[width=1\textwidth]{images/pfd2cs/images/138.png}
\includegraphics[width=1\textwidth]{images/pfd2cs/images/156.png}
\includegraphics[width=1\textwidth]{images/pfd2cs/images/158.png}
% \includegraphics[width=1\textwidth]{images/pfd2cs/images/171.png}
\includegraphics[width=1\textwidth]{images/pfd2cs/images/217.png}
\caption*{\textit{CS}}
\end{minipage}
\caption{An example of traffic scene manipulation by changing spatial relation of 2 car }
\label{fig:joint_horizontal}
\end{figure*}

\begin{figure*}[t!]
\begin{minipage}{0.325\textwidth}
\centering
% \includegraphics[height=0.5\textwidth]{images/pfd2bdd/graphs/265.png}
% \includegraphics[height=0.5\textwidth]{images/pfd2bdd/graphs/276.png}
% \includegraphics[height=0.5\textwidth]{images/pfd2bdd/graphs/74.png} too wide
% \includegraphics[height=0.5\textwidth]{images/pfd2bdd/graphs/421.png}
% \includegraphics[height=0.5\textwidth]{images/pfd2bdd/graphs/422.png}
\includegraphics[height=0.5\textwidth]{images/pfd2bdd/graphs/444.png}
\includegraphics[height=0.5\textwidth]{images/pfd2bdd/graphs/465.png}
% \includegraphics[height=0.5\textwidth]{images/pfd2bdd/graphs/522.png}
% \includegraphics[height=0.5\textwidth]{images/pfd2bdd/graphs/562.png}
\includegraphics[height=0.5\textwidth]{images/pfd2bdd/graphs/574.png}
\includegraphics[height=0.5\textwidth]{images/pfd2bdd/graphs/623.png}
% \includegraphics[height=0.5\textwidth]{images/pfd2bdd/graphs/1660.png}
\includegraphics[height=0.5\textwidth]{images/pfd2bdd/graphs/1716.png}

\caption*{SG}
\end{minipage}
\begin{minipage}{0.325\textwidth}
% \includegraphics[width=1\textwidth]{images/pfd2bdd/images/265.png}
% \includegraphics[width=1\textwidth]{images/pfd2bdd/images/276.png}
% \includegraphics[width=1\textwidth]{images/pfd2bdd/images/74.png}
% \includegraphics[width=1\textwidth]{images/pfd2bdd/images/421.png}
% \includegraphics[width=1\textwidth]{images/pfd2bdd/images/422.png}
\includegraphics[width=1\textwidth]{images/pfd2bdd/images/444.png}
\includegraphics[width=1\textwidth]{images/pfd2bdd/images/465.png}
% \includegraphics[width=1\textwidth]{images/pfd2bdd/images/522.png}
% \includegraphics[width=1\textwidth]{images/pfd2bdd/images/562.png}
\includegraphics[width=1\textwidth]{images/pfd2bdd/images/574.png}
\includegraphics[width=1\textwidth]{images/pfd2bdd/images/623.png}
% \includegraphics[width=1\textwidth]{images/pfd2bdd/images/1660.png}
\includegraphics[width=1\textwidth]{images/pfd2bdd/images/1716.png}
\caption*{\textit{BDD}}
\end{minipage}
\begin{minipage}{0.325\textwidth}
% \includegraphics[width=1\textwidth]{images/pfd2cs/images/265.png}
% \includegraphics[width=1\textwidth]{images/pfd2cs/images/276.png}
% \includegraphics[width=1\textwidth]{images/pfd2cs/images/74.png}
% \includegraphics[width=1\textwidth]{images/pfd2cs/images/421.png}
% \includegraphics[width=1\textwidth]{images/pfd2cs/images/422.png}
\includegraphics[width=1\textwidth]{images/pfd2cs/images/444.png}
\includegraphics[width=1\textwidth]{images/pfd2cs/images/465.png}
% \includegraphics[width=1\textwidth]{images/pfd2cs/images/522.png}
% \includegraphics[width=1\textwidth]{images/pfd2cs/images/562.png}
\includegraphics[width=1\textwidth]{images/pfd2cs/images/574.png}
\includegraphics[width=1\textwidth]{images/pfd2cs/images/623.png}
% \includegraphics[width=1\textwidth]{images/pfd2cs/images/1660.png}
\includegraphics[width=1\textwidth]{images/pfd2cs/images/1716.png}
\caption*{\textit{CS}}
\end{minipage}
\caption{An example of traffic scene manipulation by changing spatial relation of 2 car }
\label{fig:joint_horizontal}
\end{figure*}

\begin{figure*}[t!]
\begin{minipage}{0.325\textwidth}
\centering
\includegraphics[height=0.5\textwidth]{images/pfd2bdd/graphs/726.png}
% \includegraphics[height=0.5\textwidth]{images/pfd2bdd/graphs/764.png}
\includegraphics[height=0.5\textwidth]{images/pfd2bdd/graphs/772.png}
\includegraphics[height=0.5\textwidth]{images/pfd2bdd/graphs/928.png}
\includegraphics[height=0.5\textwidth]{images/pfd2bdd/graphs/980.png}
\includegraphics[height=0.5\textwidth]{images/pfd2bdd/graphs/1001.png}
% \includegraphics[height=0.5\textwidth]{images/pfd2bdd/graphs/1074.png} too wide
% \includegraphics[height=0.5\textwidth]{images/pfd2bdd/graphs/1078.png} too wide
\includegraphics[height=0.5\textwidth]{images/pfd2bdd/graphs/1549.png}
\includegraphics[height=0.5\textwidth]{images/pfd2bdd/graphs/1566.png}
\includegraphics[height=0.5\textwidth]{images/pfd2bdd/graphs/1922.png}
\caption*{SG}
\end{minipage}
\begin{minipage}{0.325\textwidth}
\includegraphics[width=1\textwidth]{images/pfd2bdd/images/726.png}
% \includegraphics[width=1\textwidth]{images/pfd2bdd/images/764.png}
\includegraphics[width=1\textwidth]{images/pfd2bdd/images/772.png}
\includegraphics[width=1\textwidth]{images/pfd2bdd/images/928.png}
\includegraphics[width=1\textwidth]{images/pfd2bdd/images/980.png}
\includegraphics[width=1\textwidth]{images/pfd2bdd/images/1001.png}
% \includegraphics[width=1\textwidth]{images/pfd2bdd/images/1074.png}
% \includegraphics[width=1\textwidth]{images/pfd2bdd/images/1078.png}
\includegraphics[width=1\textwidth]{images/pfd2bdd/images/1549.png}
\includegraphics[width=1\textwidth]{images/pfd2bdd/images/1566.png}
\includegraphics[width=1\textwidth]{images/pfd2bdd/images/1922.png}
\caption*{\textit{BDD}}
\end{minipage}
\begin{minipage}{0.325\textwidth}
\includegraphics[width=1\textwidth]{images/pfd2cs/images/726.png}
% \includegraphics[width=1\textwidth]{images/pfd2cs/images/764.png}
\includegraphics[width=1\textwidth]{images/pfd2cs/images/772.png}
\includegraphics[width=1\textwidth]{images/pfd2cs/images/928.png}
\includegraphics[width=1\textwidth]{images/pfd2cs/images/980.png}
\includegraphics[width=1\textwidth]{images/pfd2cs/images/1001.png}
% \includegraphics[width=1\textwidth]{images/pfd2cs/images/1074.png}
% \includegraphics[width=1\textwidth]{images/pfd2cs/images/1078.png}
\includegraphics[width=1\textwidth]{images/pfd2cs/images/1549.png}
\includegraphics[width=1\textwidth]{images/pfd2cs/images/1566.png}
\includegraphics[width=1\textwidth]{images/pfd2cs/images/1922.png}
\caption*{\textit{CS}}
\end{minipage}
\caption{An example of traffic scene manipulation by changing spatial relation of 2 car }
\label{fig:joint_horizontal}
\end{figure*}

% \begin{figure*}[t!]
% \begin{minipage}{0.325\textwidth}
% \centering
% % \includegraphics[height=0.5\textwidth]{images/pfd2bdd/graphs/1414.png} too wide
% % \includegraphics[height=0.5\textwidth]{images/pfd2bdd/graphs/1592.png}
% % \includegraphics[height=0.5\textwidth]{images/pfd2bdd/graphs/1792.png} too wide
% \caption*{SG}
% \end{minipage}
% \begin{minipage}{0.325\textwidth}
% % \includegraphics[width=1\textwidth]{images/pfd2bdd/images/1414.png}
% % \includegraphics[width=1\textwidth]{images/pfd2bdd/images/1592.png}
% % \includegraphics[width=1\textwidth]{images/pfd2bdd/images/1792.png}
% \caption*{\textit{BDD}}
% \end{minipage}
% \begin{minipage}{0.325\textwidth}
% % \includegraphics[width=1\textwidth]{images/pfd2cs/images/1414.png}
% % \includegraphics[width=1\textwidth]{images/pfd2cs/images/1592.png}
% % \includegraphics[width=1\textwidth]{images/pfd2cs/images/1792.png} 
% \caption*{\textit{CS}}
% \end{minipage}
% \caption{An example of traffic scene manipulation by changing spatial relation of 2 car }
% \label{fig:joint_horizontal}
% \end{figure*}

\end{document}
