\pdfoutput=1
\documentclass[aps,prl,superscriptaddress,twocolumn,]{revtex4-2}
%\documentclass[aps,prx,superscriptaddress,onecolumn]{revtex4-2}
\bibliographystyle{apsrev4-2}
\usepackage{amssymb}
\usepackage{braket}
\usepackage{graphicx}
\usepackage{mathrsfs}
\usepackage{bm}
\usepackage{tabularx}
\usepackage{amsmath}
\usepackage[colorlinks=true,linkcolor=blue,citecolor=blue,urlcolor=blue]{hyperref}
\usepackage{tikz} %textcolor
\DeclareMathOperator{\Tr}{Tr}
\newcommand{\vect}{\mathbf}
\newcommand{\iu}{{i\mkern1mu}}
\newcommand*\diff{\mathop{}\!\mathrm{d}}
\newcommand*\Diff[1]{\mathop{}\!\mathrm{d^#1}}

\newcommand{\RN}[1]{%
	\textup{\uppercase\expandafter{\romannumeral#1}}%
}
\newcommand{\Rn}[1]{%
	\textup{\lowercase\expandafter{\romannumeral#1}}%
}

\newcommand{\TGZ}[1]{\textcolor{orange}{[TGZ: #1]}}
\newcommand{\ZYN}[1]{\textcolor{cyan}{[ZYN: #1]}}

\begin{document}
	
	
	\title{Universal Properties of the Spectral Form Factor in Open Quantum Systems}
	\author{Yi-Neng Zhou}
	\affiliation{Institute for Advanced Study, Tsinghua University, Beijing,100084, China}
	
	
	\author{Tian-Gang Zhou}
	\affiliation{Institute for Advanced Study, Tsinghua University, Beijing,100084, China}

	%
	%
	\author{Pengfei Zhang}
 	\thanks{PengfeiZhang.physics@gmail.com}
	\affiliation{Department of Physics, Fudan University, Shanghai, 200438, China}
	\date{\today}		
	
	
\begin{abstract}
The spectral form factor (SFF) can probe the eigenvalue statistic at different energy scales as its time variable varies. In closed quantum chaotic systems, the SFF exhibits a universal dip-ramp-plateau behavior, which reflects the spectrum rigidity of the Hamiltonian. In this work, we explore the universal properties of SFF in open quantum systems. We find that in open systems the SFF first decays exponentially, followed by a linear increase at some intermediate time scale, and finally decreases to a saturated plateau value. We derive universal relations between (1) the early-time decay exponent and Lindblad operators; (2) the long-time plateau value and the number of steady states. We also explain the effective field theory perspective of universal behaviors. We verify our theoretical predictions by numerically simulating the Sachdev-Ye-Kitaev (SYK) model, random matrix theory (RMT), and the Bose-Hubbard model. 
	
\end{abstract}
	
\maketitle
	
\emph{Introduction.} The spectral form factor (SFF) has attracted much attention in recent years for its direct relation to the eigenvalue statistics at different energy scales and its utility as a robust diagnosis of quantum chaos\cite{brezinUniversalityCorrelationsEigenvalues1993,brezinCorrelationsNearbyLevels1996,liuSpectralFormFactors2018,mullerPeriodicorbitTheoryUniversality2005,cotlerBlackHolesRandom2017,kosManyBodyQuantumChaos2018,bertiniExactSpectralForm2018a,chanSpectralStatisticsSpatially2018,kudler-flamConformalFieldTheory2020,winerHydrodynamicTheoryConnected2022a,barneySpectralStatisticsMinimal2023}. The structure of SFF is a direct indicator of the energy spectrum correlation in quantum systems. As its time variable increases, it reveals the eigenvalue statistics at a smaller energy scale. The SFF among different models reveals the symmetry that those models preserve. It exhibits several universal properties including the initial decay, the increase at intermediate time scales which shows a linear ramp in models with spectrum rigidity, and finally the saturation to a plateau value. The "dip-ramp-plateau" structure is ubiquitous in quantum chaos systems\cite{saadSemiclassicalRampSYK2019a,gharibyanOnsetRandomMatrix2018,winerExponentialRampQuadratic2020}. 
 
However, the interaction and exchange between the system and environment are inevitable, and therefore it is natural to focus on the corresponding problem in open systems. Generalizing familiar concepts in closed systems to open systems has helped people discover much more interesting and novel physics. Recently, the development of entropy dynamics\cite{zhouEnyiEntropyDynamics2021a}, entanglement phase transition\cite{mazzucchiQuantumMeasurementinducedDynamics2016,liQuantumZenoEffect2018,skinnerMeasurementInducedPhaseTransitions2019,liMeasurementdrivenEntanglementTransition2019,szyniszewskiEntanglementTransitionVariablestrength2019,chanUnitaryprojectiveEntanglementDynamics2019,vasseurEntanglementTransitionsHolographic2019,zhouEmergentStatisticalMechanics2019,gullansScalableProbesMeasurementInduced2020,jianMeasurementinducedCriticalityRandom2020,fujiMeasurementinducedQuantumCriticality2020,zabaloCriticalPropertiesMeasurementinduced2020,gullansDynamicalPurificationPhase2020,choiQuantumErrorCorrection2020,baoTheoryPhaseTransition2020,nahumMeasurementEntanglementPhase2021,PhysRevB.103.174309,sangMeasurementprotectedQuantumPhases2021,albertonEntanglementTransitionMonitored2021,lavasaniMeasurementinducedTopologicalEntanglement2021,PhysRevB.103.224210,LeGal:2022rwf,jianMeasurementInducedPhaseTransition2021a,zhangUniversalEntanglementTransitions2022,liuNonunitaryDynamicsSachdevYeKitaev2021,zhang2021emergent,zhang2022quantum,PhysRevB.106.224305}, operator complexity\cite{liuKrylovComplexityOpen2022,PathakOperatorGrowthOpen2023} in open quantum many-body systems have aroused much interest in condensed matter physicists. As a versatile probe tool for many-body systems, the generalization of the SFF in open systems and its universal properties are still an open question\cite{liSpectralStatisticsNonHermitian2021c,kawabataDynamicalQuantumPhase2022}. 
	
In this paper,  we study the SFF in open quantum systems driven by the Lindblad master equation. The definition excludes possible exponential growth over time in general non-Hermitain systems. For concreteness, we define the normalized SFF as the ratio between SFF with dissipation and that without dissipation. We find some universal properties of the normalized SFF according to its early-time and late-time dynamics. More specifically, we find this normalized SFF has an early-time exponential decay behavior related to the Lindblad operators and late-time plateau behavior related to the number of steady states. We demonstrate the universality of these properties in open systems by studying three different models with dissipation: the random matrix model, the SYK model, and the Bose-Hubbard model. In the random matrix model and the Bose-Hubbard model, the numerical results agree well with our conjecture. Furthermore, using the path-integral method, we give a candidate semi-classical explanation of the SFF in systems with dissipation, and this is a novel perspective for understanding the universal properties of the normalized SFF.
\begin{figure}[t] 
	\centering 
	\includegraphics[width=0.5\textwidth]{Fig_cartoon} 
	\caption{The universal properties of the normalized SFF in open systems. Its has an early-time exponential decay and a long-time plateau behavior.}
	\label{cartoon}
\end{figure}

\emph{The definition of SFF in open systems.} In the closed system, the SFF can be defined as the size fluctuation of the analytic continuation of the thermal partition function of the quantum system
\begin{equation} 
	\begin{split}
		&F(t,\beta=0)=\frac{|\mathcal{Z}(it)|^2}{[\mathcal{Z}(0)]^2}=\frac{1}{[\mathcal{Z}(0)]^2}\sum_{m,n} e^{-i(E_m-E_n)t}.
		\label{close_SFF_definition}
	\end{split}
\end{equation}
with $\mathcal{Z}(it)=\operatorname{Tr}(e^{-itH})$.
From this expression, we see that SFF captures the energy level correlations of the full spectrum of the system, and the energy scale that it probes decreases as its time variable increases. At early time, this SFF captures the energy level correlations at an energy scale much larger that the mean energy level spacing of the system, and it usually has a decay behavior which is often called \textit{slope}. This slope region is non-universal in different models for it sees the details of the energy spectrum of the system. At the intermediate time scale, SFF measures the energy level correlation in the same order as the mean energy level spacing, and in some models that have level repulsion, we see a linear ramp of SFF as time increases. Therefore, SFF can be used to diagnose spectral rigidity. Over a long-time, the SFF often saturates to a constant plateau value which is determined by each single energy level.  

In the open system, we consider the time evolution of the system driven by the Lindblad Master equation 
	\begin{equation} 
		\frac{\partial{\rho}}{\partial t}=-i[H,\rho] +2\gamma\sum_m L_m \rho L^{\dagger}_m-\gamma\sum_m\lbrace L^{\dagger}_m L_m, \rho\rbrace.
	\end{equation}
Here, $\gamma$ is the dissipation strength, and $L_{\alpha}$ is the Lindblad jump operator. 

If we use the Choi-Jamiolkwski isomorphism\cite{TysonOperatorSchmidtDecompositionsFourier2003,VidalMixedStateDynamicsOneDimensional2004} to map the density matrix $\rho = \sum_{m,n}|m\rangle \langle n|$ to a wave function defined on a double space as $| \psi_{\rho}^D(t)\rangle=\sum_{mn}\rho_{mn}|m\rangle\otimes |n\rangle$, then after this mapping the wave function $\psi_{\rho}^D$ in the double system satisfies a Schrodinger-like equation $	i \partial_t{\psi_{\rho}^D(t)}=H^D\psi_{\rho}^D(t)$.
Here, $H^D=H_s-iH_d$ is defined on the double space with 
\begin{equation*} 
		H_s = H_L\otimes \mathcal{I}_R - \mathcal{I}_L\otimes H^T_R
\end{equation*}
and
\begin{equation} 
\begin{split}
	H_d =&\gamma \sum_m [-2\hat{L}_{m,L}\otimes \hat{L}_{m,R}^*\\&+(\hat{L}^\dagger_m\hat{L}_m)_L\otimes \mathcal{I}_R +\mathcal{I}_L\otimes (\hat{L}^\dagger_m\hat{L}_m)_{R}^*].
 \end{split}
    \label{Hd}
\end{equation}
Operators with subscript $L$ and $R$ stand for operators acting on the left and the right systems respectively, and $T$
stands for the transpose, and $\mathcal{I}$ represents the identity operator.
 
Similar to the SFF defined in the closed system the Eq.~\eqref{close_SFF_definition}, we can define the  SFF in the open system as 
	\begin{equation} 	
	F_\gamma(t) =\frac{1}{[\mathcal{Z}(0)]^2} \operatorname{Tr}(e^{-iH^D t})=\frac{1}{[\mathcal{Z}(0)]^2}\sum_l   e^{(-i\alpha_l-\beta_l)t} .
	\label{definition_2_beta0}
	\end{equation}
Here, we use the subscript $\gamma $ to denote the SFF in open systems (that is the dissipation strength $\gamma$ is non-zero).  When we set the dissipation strength in the Lindblad evolution as zero, we find that this definition is the same as that in the closed system Eq.~\eqref{close_SFF_definition}.

Since the imaginary part of the Lindblad spectrum is always non-negative, this SFF defined in Eq.~\eqref{definition_2_beta0} will not grow exponentially. Thus, although the Lindblad spectrum is complex, the SFF defined in Eq.~\eqref{definition_2_beta0} will decay exponentially in time till it reaches the steady state value. In addition, there is an alternative approach to defining the SFF in open systems that has a close relation to the definition the Eq.~\eqref{definition_2_beta0}, and the details of this discussion are included in the supplementary material\cite{SM}.


\emph{The universal function of the normalized SFF.} Let us now consider the behavior of the normalized SFF in open systems defined as
\begin{equation} 
	g(t,\gamma)  \equiv\frac{	F_\gamma(t) }{	F(t,\beta=0) }.
	\label{normalized_SFF}
\end{equation}
The motivation here is to find some universal properties of this normalized SFF.
We summarize some universal properties of this normalized SFF including the early-time exponential decay behavior related to the Lindblad operators and the late-time plateau behavior related to the number of the steady state, and it is illustrated in Fig.~\ref{cartoon}. We summarize these universal properties below:
 

1.At the early time $\gamma t \ll 1$, the normalized SFF has an exponential decay behavior 
\begin{equation} 
	g(t,\gamma)  =e^{-\alpha \gamma t}.\ \ \ \text{with}\ \alpha = 2\sum_m \langle L_m^{\dagger}L_m\rangle.
	\label{alpha_def}
\end{equation}
Here $\langle L_m^{\dagger}L_m\rangle\equiv \operatorname{Tr}\left[ L_m^{\dagger}L_m\right]/d$. $d$ is the Hilbert space dimension of the Hamiltonian $H$. 


2. The long-time behavior of this normalized SFF is a constant plateau whose value is given by 
\begin{equation} \label{longtime_SFF}
	g(t\to \infty,\gamma \neq 0)  = \frac{1}{d}.
\end{equation}


Below, we give some simple arguments for these universal properties. At small $\gamma t$, the SFF becomes
\begin{equation} 
	\begin{split}
		&F_\gamma(t)
		\simeq\frac{1}{[\mathcal{Z}_0(0)]^4}\operatorname{Tr}[  e^{-iH_st}]\operatorname{Tr}[e^{-H_dt}]\\
		&=F(t,\beta=0)\frac{1}{[\mathcal{Z}_0(0)]^2}\operatorname{Tr}[e^{-H_dt}].
	\end{split}
	\label{initial_alpha}
\end{equation}
In the early-time regime, it is known that the correlation between the left and right contour is much smaller than the correlation within the same contour \cite{saadSemiclassicalRampSYK2019a}, then we ignore the first term of the $H_d$ in Eq.~\eqref{Hd} when evaluating the last line of Eq.~\eqref{initial_alpha}. Using the fact that the second and the third terms of the $H_d$ commute with each other and the trace can be moved on the exponential at small $\gamma t$, we further obtain 
\begin{equation}
	F_\gamma(t)\simeq F(t,\beta=0)e^{- 2\sum_m \langle L_m^{\dagger}L_m\rangle\gamma t}.
\end{equation}
This leads to the expression of $\alpha$ in the Eq.~\eqref{alpha_def}, and its detailed derivation is in the supplementary material\cite{SM}. As time increases, the correlation between the left and right contour generally increases, thus the assumption above is not valid at the intermediate time. Therefore, the normalized SFF generally does not have this exponential decay behavior at the intermediate time scales $\gamma t \sim 1$.

The final plateau value of the normalized SFF can be understood by investigating Eq.~\eqref{definition_2_beta0}. Only the steady state with zero-imaginary eigenvalue will give a non-vanishing contribution to the long-time plateau value of SFF, and this gives the expression Eq.~\eqref{longtime_SFF}. In addition, if there are more than one steady state, then Eq.~\eqref{longtime_SFF} should be changed to 
$g(t\to \infty,\gamma \neq 0)  = \frac{\theta}{d}$. Here, $\theta$ is the total number of steady states.

Moreover, we can analyze the late-time regime using the effective field theory approach\cite{saadSemiclassicalRampSYK2019a,winerHydrodynamicTheoryConnected2022a}. Without any dissipation, the linear ramp can be understood as an integration over the zero mode $\Delta$ and its conjugate variable $E_{\text{aux}}$. $\Delta$ describes the relative time shift between forward and backward evolution branches and $E_{\text{aux}}$ can be understood as the energy of the system. In closed systems, there is no coupling between two branches, and The effective action $S^0_\text{eff}(\Delta,E_{\text{aux}})$ does not depend on $\Delta$. Consequently, the integral over $\Delta$ from $0$ to $t$ leads to a linear slope. When the dissipation strength becomes small but finite, we find perturbatively:
\begin{equation}
    \delta S_\text{eff}=-\gamma t \sum_i G_{\text{W},i}(\Delta,E_{\text{aux}}).
\end{equation}
Here $G_{\text{W},i}(t,E_{\text{aux}})$ is the Wightmann Green's function of operator $L_i$with energy $E_{\text{aux}}$ \cite{saadSemiclassicalRampSYK2019a}. This leads to a finite mass for $\Delta$, which increases linearly as time increases. In particular, as $t\rightarrow \infty$, the mode will be pinned at $\Delta =0$, which terminates the presence of the linear ramp.


\emph{Examples.} In the following, we use the SYK model, the random matrix model, and the Bose-Hubbard model as examples to illustrate these universal properties of the normalized SFF in the open system. 

We comment here that the SYK model and the random matrix model are both good examples to analytically calculate the SFF since they both involve random averages over different realizations that rattle the energy eigenvalues.  The random average then smooths out the fluctuations that come from the oscillating terms in the SFF, thus making it a smooth function of time. In comparison, the SFF has extensive spikes in the Bose-Hubbard model that come from the zeros of the SFF, and we need to do the time slice average to get a smooth  SFF curve. 


\emph{A. SYK Model.} 
We consider the SFF of the SYK model whose Hamiltonian is of the form
\begin{equation} 
	H = i^{\frac{q}{2}}\sum_{a_1<...<a_q}^N J_{a_1,...,a_q}\psi_{a_1}...\psi_{a_q}.
\end{equation}
Here, $J_{a_1,...,a_q}$ is a random variable that satisfies the Gaussian
distribution with mean zero and variance 
\begin{equation*} 
	\langle J_{a_1,...,a_q}(t) J_{a_1^{'},...,a_q^{'}}(t') \rangle=\delta_{a_1,a_1^{'}}...\delta_{a_q,a_q^{'}}\delta(t-t')\frac{J(q-1)!}{N^q},
\end{equation*}
and $\psi$ is the Majorana fermion operator.



\begin{figure}[t] 
	\centering 
	\includegraphics[width=0.5\textwidth]{Fig_SYK} 
	\caption{The normalized SFF dynamics for SYK model as a function of $\gamma t$. $\gamma$ is the dissipation strength. The Lindblad jump operators are chosen as the single Majorana Fermion operators. The dashed line is a theoretical prediction of the initial slope based the Eq.~\eqref{alpha_def}. The left inset shows the early-time behavior of the normalized SFF. The right inset is a log-log plot of the SFF at different dissipation strengths, and the purple line is SFF without dissipation for comparison. The total number of Majorana Fermion is $N=10$, and the random sample sizes is 200.}
	\label{SYK_all}
\end{figure} 
We numerically compute the SFF in Fig.~\ref{SYK_all}, and there are several noteworthy features of this figure. First, we find curves with different dissipation $\gamma$ collapse well into a single line when they are plotted in terms of $\gamma t$. Second, the early-time exponential decay in the SYK model is visible in the figure, and it agrees well with our analytical result $e^{-N\gamma t}$ at early time region $\gamma t<0.2$. Third, the long-time value of the SFF curve is a non-vanishing plateau whose value is $1/2^N$.

Furthermore, we can then write the SFF of the SYK model as a path-integral with the Lindblad operator chosen as the single Majorana fermion operator $L_i = \psi_i$. Also, the dissipation strength is chosen as the constant $\gamma$. We can then solve the early-time saddle-point solutions of the effective action, and to the first-order of dissipation strength $\gamma$, the effective action at the saddle point is $I[G,\Sigma]=I_0[G,\Sigma]+N\gamma T$. Thus, we obtain the normalized SFF as $g(t,\gamma) =  \exp[-N\gamma T]$. It has an exponential decay behavior at the early time. The details of the derivation of the SFF in the SYK model are included in the supplementary. A similar analysis of the SFF in the Brownian SYK is also included, in which the normalized SFF also has an early-time exponential decay behavior\cite{SM}.


\emph{B. The Random Matrix Theory.}  Consider the SFF in GUE. The SFF we defined in Eq.~\eqref{definition_2_beta0} can be written in RMT as 
\begin{equation} 
	F_\gamma(t)=\frac{1}{N^2}\langle \operatorname{Tr} \delta(\lambda -H^D) \rangle
\end{equation}
with $H^D=H_s -i H_d$ being a random matrix defined on double space. Here, $H_s = H_L\otimes \mathcal{I}_R - \mathcal{I}_L\otimes H^T_R$, and $H_d =\gamma (-2\hat{L}_L\otimes \hat{L}_R^*+(\hat{L}^\dagger\hat{L})_L\otimes \mathcal{I}_R +\mathcal{I}_L\otimes (\hat{L}^\dagger\hat{L})_R^*$ with $H$ and $L$ both are $N \times N $ random Hermitian matrix. The bracket means an averaging with respect to the Gaussian distribution:
\begin{equation} 
	P(H)=\frac{1}{\mathcal{Z}}e^{-\frac{N}{2}\operatorname{Tr}(H^2)},\ \ \ P(L)=\frac{1}{\mathcal{Z}}e^{-\frac{N}{2}\operatorname{Tr}(L^2)}.
\end{equation}
Then we consider the SFF in open systems in RMT. The SFF of open systems defined in Eq.\ref{definition_2_beta0} can be written as 
\begin{equation} 
	F_\gamma(t) =\frac{\int dH dLe^{-\frac{N}{2}\Tr H^2}e^{-\frac{N}{2}\Tr L^2}\Tr e^{-itH^D}}{\int dH dL e^{-\frac{N}{2}\Tr H^2}e^{-\frac{N}{2}\Tr L^2}}.
\end{equation}
\begin{figure}[t] 
	\centering 
	\includegraphics[width=0.53\textwidth]{Fig_RMT} 
	\caption{The normalized SFF dynamics for GUE random matrices with dimension $N_{dim}=20$ as a function of $\gamma t$. $\gamma$ is the dissipation strength. The Lindblad jump operators are chosen as the random hermitian matrix of GUE. The dashed line is a theoretical prediction of the initial slope based the Eq.~\eqref{alpha_def}. The left inset shows the early-time behavior of the normalized SFF. The right inset is a log-log plot of the SFF as a function of $tJ$, and the purple line is SFF without dissipation for comparison. Here, the random realization of $H$ and $L$ is independent, and we randomize them each for 100 realizations.}
	\label{RMT_compare}
\end{figure}

In Fig.~\ref{RMT_compare}, we present $F_\gamma(t)$ for the GUE ensemble of matrices with dimension $N=20$. We find that without dissipation the SFF first dips below its plateau value and then climb back up in a linear fashion (this region is also called the \textit{ramp}), joining onto the plateau as depicted in the right inset of the Fig.\ref{RMT_compare}. Also, when we add a small dissipation, we find a similar dip-ramp behavior of the SFF, whereas it then decays to a plateau value that is lower than the case without dissipation. Moreover, the height of the plateau is of order $1/N$ without dissipation which is the mean level spacing, and the height of the plateau is of order $1/N^2$ with non-zero dissipation. 
	
To understand this behavior of SFF with dissipation, we can directly calculate the normalized SFF, and the derivation details are included in the supplementary material\cite{SM}. We obtain the normalized SFF at early times 
\begin{equation} 
	g(t,\gamma)\simeq e^{-2\gamma t}, \gamma t \ll 1,
\end{equation}
and this is an exponential decay behavior which is also visible in the numerical results in Fig.~\ref{RMT_compare}, and it is in good agreement with $e^{-2\gamma t}$ at $\gamma t<0.5$.
On the other hand, in the long time limit $t \to \infty$, we find $g(t\to \infty,\gamma \neq 0)=\frac{1}{N}$, and $g(t\to \infty,\gamma =0)=1$. This explains the difference between the final plateau value in the case with and without dissipation as depicted in Fig.~\ref{RMT_compare}.



\emph{C. Bose-Hubbard Model.} We now consider the SFF in the Bose-Hubbard model with dissipation. The Hamiltonian of the Bose Hubbard model is
\begin{equation}
	\hat{H}=-J\sum_{\langle i,j\rangle}\hat{b}_{i}^{\dagger}\hat{b}_{j}+\frac{U}{2}\sum_{\langle i,j\rangle}\hat{n}_{i}(\hat{n}_{i}-1)
\end{equation}
Here, $J$ is the strength of the nearest neighbor hopping, and $U$ is the strength of the on-site interaction. In an open system, we set $\gamma$ as a time-independent dissipation strength. Also, we set the Lindblad jump operators as $\hat{L}_{m}=\hat{n}_{m}$. Here $m=1,2,..., N_s$, and $N_s$ is the total number of sites. 
\begin{figure}[t] 
	\centering 
	\includegraphics[width=0.52\textwidth]{Fig_BHM}
	\caption{The normalized SFF dynamics for 1D the Bose-Hubbard model as a function of $\gamma t$. $\gamma$ is the dissipation strength. The dashed line is a fitting of the initial slope based the Eq.~\eqref{alpha_def}. The left inset shows the short-time behavior of the normalized SFF. The right inset is a log-log plot of the SFF as a function of $tJ$, and the purple line is SFF without dissipation for comparison. Here, $U/J = 3.128$ and the number of sites $N_s = 4$, and the number of bosons $N_b = 4$.}
	\label{BHM_compare}
\end{figure}

The normalized SFF of the Bose-Hubbard model is illustrated in Fig.~\ref{BHM_compare}, and SFF is illustrated in the right inset. In our numerical simulation, we set $U/J=3.128$ which is in the quantum critical region of 1D BHM, and the model itself is the most chaotic\cite{IppeiBHMOneDim2011}. Meanwhile, since the SFF has extensive spikes in the Bose-Hubbard model that come from the zeros of the SFF, we perform the time slice average to get a smooth SFF curve in Fig.~\ref{BHM_compare}. The number of time points that we average over is $N_{average}=10$. The details of this average are added in the supplementary\cite{SM}.
The initial exponential decay curve obtained by Eq.~\eqref{alpha_def} is also included for comparison. The early-time exponential decay of the normalized SFF is visible in the left inset of Fig.~\ref{BHM_compare}, and it agrees well with the theoretical curve at $\gamma t<0.15$. 



\emph{Conclusion.} In this letter, we have generalized the SFF to open quantum systems driven by the Lindblad master equation. We show that the normalized SFF of open systems generally has a dip-ramp structure and then decays to the plateau behavior at small dissipation strength. In particular, we unveil two universal properties of the normalized SFF including the early-time exponential decay behavior determined by the Lindblad operators and the late-time plateau behavior that relates to the number of the steady state. Our main tools are the SYK model, the random matrix model, and the Bose-Hubbard model. Using numerical techniques, we have obtained the behavior of SFF in these three models at all times. Then we are able to extract the universal early time and late time behavior of the normalized SFF, and we find good agreement between the numerics and analytical results.

Our work potentially opens up many interesting directions: firstly, the dynamics of the SFF of open systems have a close relationship with the Lindblad spectrum\cite{ZyczkowskiUniversalSpectraRandom2019}, and therefore the SFF can be used as a diagnosis of the structure of the Lindblad spectrum. Secondly, it will be interesting to study the intermediate time scales behavior of the SFF of the open system which might go through a phase transition and have some critical behaviors\cite{kawabataDynamicalQuantumPhase2022}. Thirdly, the SFF in open systems that we discussed here can be similarly measured in experiments \cite{vasilyevMonitoringQuantumSimulators2020b,ZollerProbingManyBodyQuantum2022} via generalization to the double space, and the detail is left to the supplementary\cite{SM}.

\textit{Acknowledgements.} We thank Hui Zhai for the invaluable discussions and for carefully reading the manuscript. We thank Yingfei Gu, Haifeng Tang, and Hanteng Wang for the helpful discussions.

\bibliography{ref.bib}

	



\end{document}
