\pdfoutput=1
\documentclass[aps,prx,superscriptaddress,one ocolumn,]{revtex4-2}
%\documentclass[aps,prx,superscriptaddress,onecolumn]{revtex4-2}
\usepackage{amssymb}
\usepackage{braket}
\usepackage{graphicx}
\usepackage{mathrsfs}
\usepackage{bm}
\usepackage{tabularx}
\usepackage{amsmath}
\usepackage[colorlinks=true,linkcolor=blue,citecolor=blue,urlcolor=blue]{hyperref}
\usepackage{tikz} %textcolor
\DeclareMathOperator{\Tr}{Tr}
\newcommand{\vect}{\mathbf}
\newcommand{\iu}{{i\mkern1mu}}
\newcommand*\diff{\mathop{}\!\mathrm{d}}
\newcommand*\Diff[1]{\mathop{}\!\mathrm{d^#1}}

\newcommand{\RN}[1]{%
	\textup{\uppercase\expandafter{\romannumeral#1}}%
}
\newcommand{\Rn}[1]{%
	\textup{\lowercase\expandafter{\romannumeral#1}}%
}

\newcommand{\TGZ}[1]{\textcolor{orange}{[TGZ: #1]}}
\newcommand{\ZYN}[1]{\textcolor{cyan}{[ZYN: #1]}}

\begin{document}
	
	
	\title{Supplementary material: The Universal Properties of the Spectral Form Factor in Open Quantum Systems}
	\author{Yi-Neng Zhou}
	\affiliation{Institute for Advanced Study, Tsinghua University, Beijing,100084, China}
	
	
	\author{Tian-Gang Zhou}
	\affiliation{Institute for Advanced Study, Tsinghua University, Beijing,100084, China}

	%
	%
	\author{Pengfei Zhang}
	\affiliation{Department of Physics, Fudan University, Shanghai, 200438, China}
	\thanks{PengfeiZhang.physics@gmail.com}
	\date{\today}		
	

\begin{abstract}
    In this supplementary, we show (A) alternative definitions of SFF; (B) the derivation of the pre-factor $\alpha$ in early decay region; (C, D, E) detailed calculation of SFF in three examples; (F) possible experimental realization of SFF.
\end{abstract}
\appendix
\maketitle


\section{An alternative approach to getting the definition of the spectral form factor in open system}
In this section, we provide another definition of the spectral form factor(SFF) in open quantum systems whose dynamics are driven by the Lindblad master equation. Also, we compare this new definition of SFF with that we have used in the main text.
In the closed system, SFF can also be defined through the fidelity between the system's density matrix and the coherent Gibbs state
\begin{equation} 
	F(t,\beta)=\langle \psi_\beta | \rho(t) 	|\psi_\beta\rangle.
	\label{close_SFF_definition_appendix}
\end{equation}
Here, the coherent Gibbs state of inverse temperature $\beta$ is defined as
\begin{equation} \label{coherent}
	|\psi_\beta\rangle=\frac{1}{\sqrt{\mathcal{Z}(\beta)}}\sum_n e^{-\frac{\beta E_n}{2}} |n\rangle
\end{equation}
with $\mathcal{Z}(\beta)=\sum_n e^{-\beta E_n}$.
In open systems, we consider the time evolution of the system driven by the Lindblad Master equation 
\begin{equation} 
	\frac{\partial{\rho}}{\partial t}=-i[H,\rho] +\sum_{\alpha}\gamma_{\alpha}L_{\alpha} \rho L^{\dagger}_{\alpha}-\frac{1}{2} \sum_{\alpha}\gamma_{\alpha} \lbrace L^{\dagger}_{\alpha} L_{\alpha}, \rho\rbrace,
\end{equation}
and we assume that the initial state is the coherent Gibbs state, then the initial density matrix is $	\rho_{\beta}=	|\psi_\beta\rangle \langle \psi_\beta |$.
If we use the Choi-Jamiolkwski isomorphism to map the density matrix to a wave function defined on a double space
\begin{equation} 
	| \psi_{\rho}^D(t)\rangle=\sum_{mn}\rho_{mn}|m\rangle\otimes |n\rangle,
\end{equation}
then, after this mapping the wave function $\psi_{\rho}^D$ in the double system satisfies a Schrodinger-like equation
\begin{equation} 
	i\hbar \partial_t{\psi_{\rho}^D(t)}=H^D\psi_{\rho}^D(t).
\end{equation}
Here, $H^D=H_s-iH_d$ with $H_s = H_L\otimes \mathcal{I}_R - \mathcal{I}_L\otimes H^T_R$, and $H_d =\gamma (-2\hat{L}_L\otimes \hat{L}_R^*+(\hat{L}^\dagger\hat{L})_L\otimes \mathcal{I}_R +\mathcal{I}_L\otimes (\hat{L}^\dagger\hat{L})_R^*$, and operators with subscript $L$ and $R$ stand for operators acting on the left and the right systems respectively, and $T$
stands for the transpose, and $\mathcal{I}$ represents the identity operator.
Using this mapping, we can rewrite the SFF defined in Eq.~\eqref{close_SFF_definition_appendix} as
\begin{equation} 
	\tilde{F}_\gamma(t,\beta)=\langle \psi_\beta^D | \psi_{\rho}^D(t)\rangle.
\end{equation}
Here, we use the subscript $\gamma $ to denote the SFF in the open system (that is the dissipation strength is non-zero). Then, SFF can be viewed as the overlap between the double space wave function at time $t$ and the double space initial wave function, which is the double space coherent Gibbs state defined as $|\psi_\beta^D \rangle= 	|\psi_\beta\rangle \otimes 	|\psi_\beta\rangle$.

We consider that this non-Hermitian Hamiltonian $H^D$ can be diagonalized, and gives a set of eigenstates that satisfy $H^D|\psi_{\rho}^l(t)\rangle=\epsilon_l |\psi_{\rho}^{D,l}(t)\rangle$ with $\epsilon_l=\alpha_l+i\beta_l$ is the eigenvalue of the eigenstate $l$.
The initial state in the double space can be expanded as 
\begin{equation} 
	| \psi_{\rho}^D(0)\rangle=	|\psi_\beta^D \rangle= \sum_l c_{l} | \psi_{\rho}^{D,l}\rangle,
\end{equation}
and the time evolution of this double space wave function is given by
\begin{equation} 
	| \psi_{\rho}^D(t)\rangle=\sum_l c_{l} e^{(-i\alpha_l-\beta_l)t}| \psi_{\rho}^{D,l}\rangle.
\end{equation}
Here, $c_{l} =\langle \psi_{\rho}^{D,l}|\psi_\beta^D \rangle$. Therefore, the SFF can be further written as
\begin{equation} 
	\tilde{F}_{\gamma}(t,\beta)=\sum_l c_{l} e^{(-i\alpha_l-\beta_l)t}\langle \psi_\beta^D | \psi_{\rho}^{D,l}\rangle=\sum_l |c_{l}|^2 e^{(-i\alpha_l-\beta_l)t}.
	\label{sff_closed}
\end{equation}
Thus, we find Lindblad spectrum and the initial state distribution on the Lindblad spectrum fully determine this quantity. Since $\beta_l$ is always non-negative, SFF will not grow exponentially. Consequently, although the Lindblad spectrum is complex, the SFF calculated from it will decay exponentially in time till it reaches its steady-state value. For the steady-state, the plateau value of the SFF is given by
\begin{equation} 
	\lim_{t\to \infty}	\tilde{F}_{\gamma}(t,\beta)=\sum_{\beta_0=0,\alpha_0}|c_{0}|^2 e^{-i\alpha_0t}.
\end{equation}
Thus, the plateau value of the SFF depends on the overlap of the initial state and the steady state.

In addition, since the second Renyi entropy is defined as $	e^{-s^{(2)}}=\langle \psi_{\rho}^D(t)| \psi_{\rho}^D(t)\rangle$, there is a direct connection between the dynamics of the SFF and entropy dynamics. The SFF defined above can be expressed in the energy basis as
\begin{equation} 
	\tilde{F}_{\gamma}(t,\beta)  =\frac{1}{[\mathcal{Z}(\beta)]^2}\sum_{m,n,m',n'}e^{-\frac{\beta}{2} (E_m+E_n+E_{m'}+E_{n'})} 
	\langle m|\otimes \langle n|e^{-iH^Dt} |m'\rangle \otimes |n'\rangle.
	\label{definition_1_appendix} 
\end{equation}
Here, $m,n,m',n'$ are the eigenstates of $H$. We can further define a new type of SFF by preserving only the diagonal elements in this eigenstate basis of the Eq.~\eqref{definition_1_appendix}. This new definition of SFF can be written explicitly as 
\begin{equation} 
	F_{\gamma}(t,\beta)=\frac{1}{[\mathcal{Z}(\beta)]^2}\sum_{m,n} e^{-\beta (E_n+E_m)} \langle m|\otimes \langle n|e^{-iH^Dt} |m\rangle \otimes |n\rangle.
	\label{definition_2_appendix}
\end{equation}
And this is the definition of the SFF in the open system that we have used in the main text if we set $\beta=0$. We can then numerically compare these two different definitions of SFF in the open system,  as shown in  Fig.~\ref{2def}. We find that these two definitions are almost the same in the SYK model as an example.
\begin{figure}[h] 
	\centering 
	\includegraphics[width=0.5\textwidth]{two_def} 
	\caption{The comparison of two different definitions of SFF in the open system. The red line is the SFF defined in Eq.~\eqref{definition_1_appendix}, and the blue line is the SFF defined in Eq.~\eqref{definition_2_appendix}. The green line is the difference between these two definitions.}
	\label{2def}
\end{figure} 

If we consider the case of the infinite temperature $\beta=0$, we have 
\begin{equation} 	
	F_\gamma(t,\beta=0) =\frac{1}{[\mathcal{Z}(0)]^2} \Tr(e^{-iH^D t})=\frac{1}{[\mathcal{Z}(0)]^2}\sum_l   e^{(-i\alpha_l-\beta_l)t}  
	\label{definition_2_beta0_appendix}
\end{equation}
Let us now compare these definitions of SFF of an open system in Eq.~\eqref{definition_2_beta0_appendix} and that in Eq.~\eqref{sff_closed}. We find that this new definition in the Eq.~\eqref{definition_2_beta0_appendix} simply takes all $|c_{l}|^2=\frac{1}{[\mathcal{Z}(0)]^2}$ in the Eq.~\eqref{sff_closed}.


\section{The derivation of the pre-factor $\alpha$ in early decay}
In this section, we give the detailed derivation of the expression of pre-factor $\alpha$ in the early decay region.
At small $\gamma t$, the SFF becomes
\begin{equation} 
	\begin{split}
		F_\gamma(t)&= \frac{1}{[\mathcal{Z}(0)]^4}\operatorname{Tr}[  e^{-iH^Dt}]\\
		&=\frac{1}{[\mathcal{Z}(0)]^4}\operatorname{Tr}[e^{-i(H_s-iH_d)t}]\\
		&\simeq \frac{1}{[\mathcal{Z}(0)]^4}\operatorname{Tr}[e^{-iH_st}]\operatorname{Tr}[e^{-H_dt}].
	\end{split}
\end{equation}
Using the definition of $H_s$
\begin{equation} 
	H_s = H_L\otimes \mathcal{I}_R - \mathcal{I}_L\otimes H^T_R,
\end{equation}
we obtain $\frac{1}{[\mathcal{Z}(0)]^2}\operatorname{Tr}[e^{-iH_st}]=|\operatorname{Tr}[e^{-iHt}]|^2=F(t,\beta=0)$.
Thus, we have
\begin{equation} 
	\begin{split}
		F_\gamma(t)\simeq \frac{1}{[\mathcal{Z}(0)]^2} F(t,\beta=0)\operatorname{Tr}[e^{-H_dt}].
	\end{split}
	\label{initial_alpha_appendix}
\end{equation}
Now, recall the definition of $H_d$
\begin{equation} 
		H_d =\gamma \sum_m [-2\hat{L}_{m,L}\otimes \hat{L}_{m,R}^*+(\hat{L}^\dagger_m\hat{L}_m)_L\otimes \mathcal{I}_R +\mathcal{I}_L\otimes (\hat{L}^\dagger_m\hat{L}_m)_{R}^*].
	\label{Hd_appendix}
\end{equation}
Since in the early time, we assume that the correlation between the left and right contour is much smaller than the correlation within the same contour, then we ignore the first term of the $H_d$ in Eq.~\eqref{Hd_appendix} when evaluating the Eq.~\eqref{initial_alpha_appendix}. This leads to
\begin{equation} 
	\begin{split}
		F_\gamma(t)\simeq \frac{1}{[\mathcal{Z}(0)]^2} F(t,\beta=0)\operatorname{Tr}[e^{-\gamma t\sum_m\left[(\hat{L}^\dagger_m\hat{L}_m)_L\otimes \mathcal{I}_R +\mathcal{I}_L\otimes (\hat{L}^\dagger_m\hat{L}_m)_{R}^*\right]}].
	\end{split}
\end{equation}
Using the fact that the second and the third term of the $H_d$ commute with each other, we further obtain 
\begin{equation} 
	\begin{split}
		F_\gamma(t)&\simeq \frac{1}{[\mathcal{Z}(0)]^4} F(t,\beta=0)\operatorname{Tr}[e^{-\gamma t\sum_m(\hat{L}^\dagger_m\hat{L}_m)_L\otimes \mathcal{I}_R }]\operatorname{Tr}[e^{-\gamma t\sum_m\mathcal{I}_L\otimes(\hat{L}^\dagger_m\hat{L}_m)_{R}^*}]\\
		&= \frac{1}{[\mathcal{Z}(0)]^2}F(t,\beta=0)\operatorname{Tr}_L[e^{-\gamma t\sum_m(\hat{L}^\dagger_m\hat{L}_m)_L }]\operatorname{Tr}_R[e^{-\gamma t\sum_m (\hat{L}^\dagger_m\hat{L}_m)_{R}^*}].
	\end{split}
\end{equation}
Also, since the trace can be moved on the exponential at small $\gamma t$, we further obtain
\begin{equation} 
	\begin{split}
		F_\gamma(t)&\simeq  F(t,\beta=0)e^{-\gamma t\operatorname{Tr}_L\left[\sum_m(\hat{L}^\dagger_m\hat{L}_m)_L\right] }e^{-\gamma t\operatorname{Tr}_R\left[\sum_m\mathcal{I}_L\otimes (\hat{L}^\dagger_m\hat{L}_m)_{R}^*\right]}\\
		&=  F(t,\beta=0)e^{-2\gamma t\operatorname{Tr}(\sum_m\hat{L}^\dagger_m\hat{L}_m) }.
	\end{split}
\end{equation}
Finally, at small $\gamma t$, we have
\begin{equation} 
	g(t,\gamma)  \equiv\frac{	F_\gamma(t) }{F(t,\beta=0) }=e^{-\alpha \gamma t}.
\end{equation}
with 
\begin{equation} 
	\alpha = 2\operatorname{Tr}\left[\sum_m L_m^{\dagger}L_m \right].
\end{equation}


\section{SFF of the random matrix theory}

\subsection{Random matrix theory in closed system}
In this section, we review some basics of calculating SFF in the random matrix theory (RMT). Let us first review how to calculate the SFF in closed systems in RMT. The $N \times N $ random Hermitian matrix $H$ of the Gaussian Unitary Ensemble (GUE) is defined to be averaged for the following Gaussian distribution:
\begin{equation} 
	P(H)=\frac{1}{\mathcal{Z}}\exp[-\frac{N}{2}\Tr(H^2)].
\end{equation}
One can write this Gaussian distribution in the eigenvalue basis, where the distribution over the set of matrices could reduce to the distribution of eigenvalues with the following joint distribution
\begin{equation} 
	P(\lambda_1,\lambda_2,...,\lambda_N)=\exp[-\frac{N}{2}\sum_{i=1}^N \lambda_i^2]\prod_{i<j}^{N}(\lambda_i - \lambda_j)^2.
\end{equation}
We could further compute the $n$-point correlation function ($n<N$) as
\begin{equation} 
	\rho^{(n)}(\lambda_1,\lambda_2,...,\lambda_n)= \int d\lambda_{n+1}...\lambda_{N} P(\lambda_1,\lambda_2,...,\lambda_N).
\end{equation}
People find that the correlation function could be determined by a kernel $K$ in the large N limit\cite{Wigner,RMT_book,RMT_book2}:
\begin{equation} 
	\rho^{(n)}(\lambda_1,\lambda_2,...,\lambda_n)= \frac{(N-n)!}{N!}\det(K(\lambda_i,\lambda_j))_{i,j=1}^n
\end{equation}
where the kernel $K$, in the large $N$ limit, behaves as
\begin{equation} 
	K(\lambda_i,\lambda_j)= \begin{cases}
		\frac{N}{2\pi} \sqrt{4-\lambda_i^2}, & i=j  \\
		\frac{N}{\pi} \frac{\sin[L(\lambda_i-\lambda_j)]}{L(\lambda_i-\lambda_j)}, & i \neq j.\\
	\end{cases}
	\label{kernal_RMT}
\end{equation} 
In the colliding case $i = j$, this kernel is the familiar Wigner's semicircle law. While in the case where $i \neq j$, this kernel is called the sine kernel in RMT. Then, we can calculate the simple one-point form factor as 
\begin{equation} 
	\begin{split}
		g^{(1)}(\beta,t)
		=&\frac{1}{N}\int d\lambda_{1}\lambda_2...\lambda_{N} P(\lambda_1,\lambda_2,...,\lambda_N)\exp[-(\beta+it)\lambda_{1}]\\
		=&\frac{1}{N}\int d\lambda_{1} 	\rho^{(1)}(\lambda_1)\exp[-(\beta+it)\lambda_{1}].
	\end{split}
\end{equation}
Similarly, we can calculate the two-point form factor, which is what we called SFF in closed systems: 
\begin{equation} 
	\begin{split}
		F(t,\beta)&=g^{(2)}(\beta,t)\\
		&=\frac{1}{N^2}\int d\lambda_{1}\lambda_2...\lambda_{N} P(\lambda_1,\lambda_2,...,\lambda_N)\exp[-(\beta+it)\lambda_{1}]\exp[-(\beta-it)\lambda_{2}]\\
		&=\frac{1}{N^2}\int d\lambda_{1} d\lambda_{2} 	\rho^{(2)}(\lambda_1,\lambda_2)\exp[-(\beta+it)\lambda_{1}]\exp[-(\beta-it)\lambda_{2}].
		\label{SFF_RMT_closed}
	\end{split}
\end{equation}
For simplicity, we consider infinite temperature $\beta=0$,
\begin{equation} 
	\label{SFF_closed}
	F(t,\beta=0)=\frac{1}{N^2}\int d\lambda_{1} d\lambda_{2} 	\rho^{(2)}(\lambda_1,\lambda_2)\exp[-it(\lambda_{1}-\lambda_{2})].
\end{equation}
We find here that the SFF in the closed system is determined by the two-point correlation function. In the case of the infinite temperature, SFF is simply the Fourier transform of the two-point correlation function.


\subsection{Random matrix theory in open system} 
In this section, we calculate the normalized SFF of the open system in the GUE. The SFF we defined in Eq.~\eqref{definition_2_beta0_appendix} can be written in the RMT as 
\begin{equation} 
	F_\gamma(t)=\frac{1}{N^2}\langle \Tr \delta(\lambda -H^D) \rangle.
\end{equation}
Here, $H^D=H_s -i H_d$ is the random matrix defined in the double space. Here, $H_s = H_L\otimes \mathcal{I}_R - \mathcal{I}_L\otimes H^T_R$, and $H_d =\gamma (-2\hat{L}_L\otimes \hat{L}_R^*+(\hat{L}^\dagger\hat{L})_L\otimes \mathcal{I}_R +\mathcal{I}_L\otimes (\hat{L}^\dagger\hat{L})_R^*$ with $H$ and $L$ both are $N \times N $ random Hermitian matrix. The bracket means an averaging for the Gaussian distribution:
\begin{equation} 
	P(H)=\frac{1}{\mathcal{Z}}\exp[-\frac{N}{2}\Tr(H^2)]
\end{equation}
and 
\begin{equation} 
	P(L)=\frac{1}{\mathcal{Z}}\exp[-\frac{N}{2}\Tr(L^2)].
\end{equation}
To understand this behavior of SFF with dissipation, we first use an approximation
\begin{equation} 
	\Tr [e^{-itH^D}] \simeq \Tr[ e^{-itH_s}e^{-tH_d}] \simeq \frac{1}{N^2}\Tr [e^{-itH_s}] \Tr[ e^{-tH_d} ].
\end{equation}
This approximation is good in the early-time limit. Then, the SFF can be formulated as
\begin{equation} 
	\begin{split}
		F_\gamma(t)\simeq&\frac{\int dH dLe^{-\frac{N}{2}\Tr H^2}e^{-\frac{N}{2}\Tr L^2}\frac{1}{N^2}\Tr e^{-itH_s}\Tr e^{-tH_d}}{\int dH dL e^{-\frac{N^2}{2}\Tr H^2}e^{-\frac{N}{2}\Tr L^2}}\\
		=&\frac{1}{ N^4}\sum_{i,j,s,r=1}^N\int d\lambda_{i}\lambda_j dl_s dl_r \rho_H^{(2)}(\lambda_i,\lambda_j)\rho_L^{(2)}(l_s,l_r)e^{-it(\lambda_{i}-\lambda_{j})}e^{-\gamma t(l_{s}-l_{r})^2}
	\end{split}
\end{equation}
Then, we further obtain
\begin{equation} 
	\begin{split}
		g(t,\gamma)=&\frac{1}{N^2}\sum_{s,r=1}^N\int dl_s dl_r\rho_L^{(2)}(l_s,l_r)e^{-\gamma t(l_{s}-l_{r})^2}\\
		=&\frac{1}{N^2}\left[N\int dl_1 \rho_L^{(2)}(l_1,l_1)
		+N(N-1)\int dl_1 dl_2\rho_L^{(2)}(l_1,l_2)e^{-\gamma t(l_{1}-l_{2})^2}\right]\\
		=&\frac{1}{N}+\frac{(N-1)}{N}\int dl_1 dl_2\rho_L^{(2)}(l_1,l_2)e^{-\gamma t(l_{1}-l_{2})^2}.
	\end{split}
\end{equation}
Then, we calculate the normalized SFF below. Using 
\begin{equation} 
	\rho^{(2)}(l_1,l_2)=\frac{N}{(N-1)}\rho(l_1) \rho(l_2)-\frac{N}{(N-1)}\frac{\sin^2[N(l_1-l_2)]}{[N(l_1-l_2)]^2},
\end{equation}
we obtain
\begin{equation} 
g(t,\gamma)=g(t,\gamma)^{\text{disc}}+g(t,\gamma)^{\text{conn}}.
\end{equation}
Here, the connected part of $g(t,\gamma)$ is
\begin{equation} 
g(t,\gamma)^{\text{conn}}=\frac{1}{N}- \int dl_1 dl_2\frac{\sin^2[N(l_1-l_2)]}{[N(l_1-l_2)]^2}e^{-\gamma t(l_{1}-l_{2})^2}.
\end{equation}
We further define $u_1= l_1-l_2$ and $u_2=l_{2}$, and we use the box approximation to deal with this divergent integral. Then, we have
\begin{equation} 
	\begin{split}
		g(t,\gamma)^{\text{conn}}=&\frac{1}{N}- \int du_1 du_2\frac{\sin^2(Nu_1)}{(Nu_1)^2}e^{-\gamma tu_1^2}\\
		=&\frac{1}{N}-2u_{\text{cut}} \int du_1 \frac{\sin^2(Nu_1)}{(Nu_1)^2}e^{-\gamma tu_1^2}\\
		=&\frac{1}{N}-\frac{2u_{\text{cut}}}{N} \int dx \frac{\sin^2(x)}{(\pi x)^2}e^{-\gamma \frac{t}{L^2}x^2}
        \label{g_conn}
	\end{split}
\end{equation}
$u_{\text{cut}}$ is the constant introduced with box approximation. Besides, the disconnected part of $g(t,\gamma)$ is
\begin{equation} 
	\begin{split}
		g(t,\gamma)^{\text{disc}}=&\int dl_1 dl_2\rho(l_1) \rho(l_2)e^{-\gamma t(l_{1}-l_{2})^2}\\
		=&\int_{-2}^2 dl_1\int_{-2}^2  dl_2\frac{1}{(2\pi)^2}\sqrt{4-l_1^2}\sqrt{4-l_2^2}e^{-\gamma t(l_{1}-l_{2})^2}.
	\end{split}
\end{equation}


\begin{figure}[h] 
	\centering 
	\includegraphics[width=0.5 \textwidth]{G2} 
	\caption{The numerical results for $g(t,\gamma)$ dynamics in the Eq.~\eqref{g2_integral} at the limit $N \to \infty$.} 
	\label{g_disconnect}
\end{figure}

One can try to solve the $u_{\text{cut}}$ in the Eq.~\eqref{g_conn} by checking the consistency of the result at $t = 0$. We notice that normalization condition gives 	$g(0,\gamma)=1$, and direct calculation shows $g(0,\gamma)^{\text{disc}}=1$. These two facts lead to $g(t,\gamma)^{\text{conn}}=0$. As a result, the connected part can be determined as
\begin{equation} 
	g(t,\gamma)^{\text{conn}}=\frac{1}{N}- \frac{1}{N\sqrt{\gamma \frac{t}{N^2}+\sqrt{1+(\gamma \frac{t}{N^2})}^2}}\simeq \frac{\gamma t}{2N^3}.
\end{equation}

Final form of normalized SFF is
\begin{equation} 
	\label{g2_integral}
	\begin{split}
		g(t,\gamma)&=\frac{1}{N}(1- \frac{1}{\sqrt{\gamma \frac{t}{N^2}+\sqrt{1+(\gamma \frac{t}{N^2})}^2}}) + \int_{-2}^2 dl_1\int_{-2}^2  dl_2\frac{1}{(2\pi)^2}\sqrt{4-l_1^2}\sqrt{4-l_2^2}e^{-\gamma t(l_{1}-l_{2})^2}\\
		&\simeq \frac{\gamma t}{2N^3} + \int_{-2}^2 dl_1\int_{-2}^2  dl_2\frac{1}{(2\pi)^2}\sqrt{4-l_1^2}\sqrt{4-l_2^2}e^{-\gamma t(l_{1}-l_{2})^2}.
	\end{split}
\end{equation}

\section{SFF of SYK model}
\subsection{The Brownian SYK with dissipation}
We consider the SFF in the Brownian SYK model with dissipation. Its Hamiltonian is of the form
\begin{equation} 
	H(t) = i^{\frac{q}{2}}\sum_{a_1<...<a_q}^N J_{a_1,...,a_q}(t)\psi_{a_1}...\psi_{a_q}.
\end{equation}
Here, $J_{a_1,...,a_q}(t)$ is a random variable that satisfies the Gaussian
distribution with zero mean value and variance  
\begin{equation} 
	\langle J_{a_1,...,a_q}(t) J_{a_1^{'},...,a_q^{'}}(t') \rangle=\delta_{a_1,a_1^{'}}...\delta_{a_q,a_q^{'}}\delta(t-t')\frac{J(q-1)!}{N^q}.
\end{equation}
We first write the SFF of the Brownian SYK model as a path integral with the Lindblad operator chosen as the single Majorana operator $L_i = \psi_i$. Also, the dissipation strength is chosen as the constant $\gamma$. Then, the SFF in the open system can be written as 
\begin{equation} 
	\begin{split}
		&F_\gamma(T) =\frac{1}{2^N}\langle \Tr e^{-itH^D}\rangle\\
		&=\frac{1}{2^N}\int \mathcal{D}\psi^L_a \mathcal{D}\psi^R_a \exp\left\{ i \left[ \int_0^T dt \frac{i}{2}\psi_a^{(j)}\partial_t\psi_a^{(j)}-J_{a_1 a_2 ... a_q}(t) 
		(i^{\frac{q}{2}}\psi^{L}_{a_1 a_2 ... a_q}-(-i)^{\frac{q}{2}}\psi^{R}_{a_1 a_2 ... a_q} )   +2\gamma \psi_a^L  \psi_a^R-iN\gamma     \right]\right\}
	\end{split}
\end{equation}
After integrating out $J_{a_1 a_2 ... a_q}(t)$ variables, we obtain
\begin{equation} 
	\begin{split}
		&F_\gamma(T) =\frac{1}{2^N}\int \mathcal{D}\psi^L_a \mathcal{D}\psi^R_a \\
		&\exp\left\{ -\int_0^T dt \left[ \frac{1}{2}\psi_a^{(j)}\partial_t\psi_a^{(j)} +\frac{J^2(q-1)!}{N^{q-1}}\sum_{a_1<a_2<...<a_q}(\frac{1}{2^q}-\psi^{L}_{a_1 a_2 ... a_q} \psi^{R}_{a_1 a_2 ... a_q} )+2i\gamma \psi_a^L  \psi_a^R+N\gamma  \right] \right\}.
	\end{split}
\end{equation}
Furthermore, we can represent Majorana fermions in terms of spin variables as $\psi_a^L  \psi_a^R=\frac{i}{2}\sigma_a^z$, then the above expression can be understood as a normal thermal partition function of a spin system $F_\gamma(t,\beta=0) =\frac{1}{2^N}\exp[-TH_{spin}]$ with 
\begin{equation} 
	H_{spin}=\frac{J^2(q-1)!}{2^qN^{q-1}}\sum_{a_1<a_2<...<a_q}(1-\sigma_1^z...\sigma_q^z)-\gamma\sum_a\sigma_a^z+N\gamma.
\end{equation}

In the long time limit, this factor can be viewed as a projection operator onto the ground states. The two ground states are when all spins are up or down without dissipation. When we add small dissipation as a perturbation to the original degenerate ground stats, the energy of the two perturbed ground states is $ E_g = 0$ and $2N\gamma$. Following the argument, the SFF reads 
\begin{equation} 
	F_\gamma(t\to \infty,\beta=0) =\frac{1}{2^N}\left\{\exp[-T\times0]+\exp[-T\times2N\gamma]\right\}=\frac{1}{2^N}\left\{1+\exp[-2N\gamma T]\right\}.
\end{equation}

At long time limit $\gamma T \gg 1$, this SFF is $1/2^N$ which is half of that with no dissipation. This explains the late-time plateau behavior of the SFF in the SYK model with dissipation, and the plateau value is $1/2$ of that with zero dissipation. This is because the dissipation breaks the degeneracy of the ground states, and gives a positive energy correction to one of the original ground states, thus this state decays as time increases, and only one state with zero energy survives.  

Next, we look for the saddle points in the semi-classical analysis of this SFF. There is an exact familiar rewrite of the SYK model in terms of variables $G$ and $\Sigma$. Without dissipation, we know an obvious saddle point which is simply $G_{LR}=\Sigma_{LR}=0$. Then, we assume that there is a saddle point near 0 when $\gamma t \ll 1$.
The total $G_{LR},\Sigma_{LR}$  integrand becomes
\begin{equation} 
	\exp\left\{ N\left[\log(2\cos(\frac{T\Sigma_{LR}}{4}))-\frac{JT}{q2^q}+i^q\frac{JT}{q}G_{LR}^q-\frac{T}{2}\Sigma_{LR}G_{LR}-2i\gamma TG_{LR}-\gamma T\right]\right\}
\end{equation}
We assume $G_{LR},\Sigma_{LR}$ to be very small, thus $\tan(\frac{T\Sigma_{LR}}{4})\simeq \frac{T\Sigma_{LR}}{4}$. We here simply take $q=2$, then we arrive at the solution 
\begin{equation} 
	\Sigma_{LR}=-4i\gamma, G_{LR}=\frac{i\gamma T}{2}.
\end{equation}
Then, the saddle point action according to this saddle point is $	2^N\exp[-\frac{JNT}{8}]\exp[-N\gamma T]$, and this explains the early-time exponential decay behavior of the SFF in Brownian SYK model. In addition, there is another non-trivial saddle points solution. 
We can obtain one set of saddle points in the long-time limit
\begin{equation} 
	G_{LR}=\pm \frac{i}{2}, \Sigma_{LR}=\mp \frac{iJ}{2^{q-2}}-4i\gamma.
\end{equation}
Then, the saddle point action is $0, -2N\gamma T$. Therefore, the SFF is $	\exp[0]+\exp[-2N\gamma T]=1+\exp[-2N\gamma T]$, and this saddle point explains the late-time plateau behavior of SFF. This is exactly what we obtain through the spin Hamiltonian analysis. 

\subsection{The regular SYK with dissipation}
We further consider the normal SYK model. We can write a path-integral expression for the SFF of the regular SYK model as
\begin{equation} 
	F_\gamma(t) =\int  \mathcal{D}G \mathcal{D}\Sigma e^{-NI[G,\Sigma]}  
\end{equation}
with	
\begin{equation} 
	\begin{split}
		&I[G,\Sigma]=-\log Pf(\delta_{ab}\partial_t-\Sigma)+\frac{1}{2}\int_0^T\int_0^Tdt_1 dt_2\lbrack\Sigma_{ab}G_{ab}-\frac{J^2}{q}s_{ab}G^q_{ab}+2i\gamma G_{ab}\delta_{ab^{'}}\delta(t_1-t_2)\rbrack+\gamma T.
	\end{split}
\end{equation}
The saddle point equations can then be written as
\begin{equation} 
	\binom{G_{LL}\ G_{LR}}{G_{RL}\ G_{RR}}=-\dbinom{iw+\Sigma_{LL} \ \ \ \Sigma_{LR}}{\Sigma_{RL} \ \ \  iw+\Sigma_{RR}}^{-1}
\end{equation}
where $	\Sigma_{ab} = s_{ab}J^2G_{ab}^{q-1}-2i\gamma \delta_{ab^{'}}\delta(t_1-t_2)$. We assume there is a saddle point solution near 0 that is proportional to $\gamma$. Thus, to the first order of $\gamma$, we have  
\begin{equation} 
	\binom{G_{LL}\ G_{LR}}{G_{RL}\ G_{RR}}\simeq-\frac{1}{(i\omega+\Sigma_{LL})(i\omega+\Sigma_{RR})}\dbinom{iw+\Sigma_{RR} \ \ \  -\Sigma_{LR}}{-\Sigma_{RL} \ \ \  iw+\Sigma_{LL}}
\end{equation}
We take $q=2$ for simplicity, and we have
\begin{equation} 
	\begin{split}
		&G_{LL}=G_{RR}=\frac{i\omega\pm\sqrt{4J^2-\omega^2}}{2J^2}\\ &G_{LR}=G_{RL}=\frac{-2i\gamma G_{LL} G_{RR} }{1+J^2G_{LL} G_{RR}}.
	\end{split}
\end{equation}
The saddle point action is $	I[G,\Sigma]=I_0[G,\Sigma]+\gamma T$. Thus, we obtain the normalized SFF as $g(t,\gamma) =\exp[-N\gamma T]$. It has an exponential decay behavior at early time.
 
 

\section{SFF of Bose-Hubbard model}
The SFF of the Bose-Hubbard model has extensive spikes that come from the zeros of the SFF. In order to transform the curve of SFF to a smooth function, we perform a time average to the numerical results of the SFF for the Bose-Hubbard model, as we treat in Fig.~(4) of the main text. From time $tJ$ between $10^{-1}$ to $10^3$, we equally divide the total time into $N_t=1000$ pieces in the log scale. And we get the SFF at each time point by averaging the value of SFF between $N_{average}=10$ neighbor points. Also, the larger the $N_{average}$ becomes, the smoother the SFF curves will be. Below, we show the numerical results of SFF at different $N_{average}=1,5,20$ in Fig.~\eqref{BHM}.
\begin{figure}[h] 
	\centering \includegraphics[width=1.0\textwidth]{randomBHM}
	\caption{The log-log plot of the SFF as a function of $tJ$, and the purple line is SFF without dissipation for comparison. Here, the time slice average $N_{average}=1,5,20$ for (a),(b),(c) respectively.}
	\label{BHM}
\end{figure}




\section{The experimental realization of the SFF in open systems.}
In this section, we give a possible experiment realization proposal of the SFF in open systems. We first prepare initial the double space wave function as
\begin{equation} 
	|\psi^{D,0}\rangle = \frac{1}{\sqrt{N}}\sum_n|n\rangle_L\otimes|n\rangle_R,
\end{equation}
then we perform the evolution for a time $t$ with the
quantum non-demolition (QND) Hamiltonian in the double space 
\begin{equation} 
	\mathcal{U}^D_{QND}(t)=\exp\left[-iH^D_{QND}(t)\right]
\end{equation}
with 
\begin{equation} 
	H^D_{QND}=H^D\otimes|0\rangle_{c} \langle 0|.
\end{equation}
Here, $H^D$ is the mapping of the Lindblad master equation onto the double space, and 'c' denotes the ancilla qubit which is also called the control qubit. 
Finally, we measure the expectation
values of $\sigma_x$ and $\sigma_y$ for the ancilla qubit, as shown in Fig.~\eqref{experiment}.
\begin{figure}[h] 
	\centering 
	\includegraphics[width=0.53\textwidth]{experiment_FIG}
	\caption{A quantum circuit employing a QND coupling of the quantum simulator in double space to an ancilla qubit to measure the SFF.} 
	\label{experiment}
\end{figure}
After direct calculation, we obtain 
\begin{equation} 
	\langle \sigma_x(t)\rangle=\sum_l \cos(\epsilon_lt),
\end{equation}
and
\begin{equation} 
	\langle \sigma_y(t)\rangle=\sum_l \sin(\epsilon_lt).
\end{equation}
Here, $\{\epsilon_l\}$ is the Lindblad spectrum. Therefore, SFF can be obtained by 
\begin{equation} 
	F_{\gamma}(t)=\langle \sigma_y(t)\rangle-i\langle \sigma_x(t)\rangle.
\end{equation}


\begin{thebibliography}{50}%
\bibitem{Wigner} Wigner, E. P. On the Distribution of the Roots of Certain Symmetric Matrices. Annals of Mathematics 1958, 67 (2), 325–327.

\bibitem{RMT_book} Guhr, T.; Müller–Groeling, A.; Weidenmüller, H. A. Random-Matrix Theories in Quantum Physics: Common Concepts. Physics Reports 1998, 299 (4–6), 189–425. 

\bibitem{RMT_book2} M. Mehta, Random Matrices, Volume 142 - 3rd Edition.




 \end{thebibliography}





\end{document}
