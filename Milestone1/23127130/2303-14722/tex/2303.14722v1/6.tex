
\section[Dancing with a tambourine]{Dancing with a tambourine\footnote{Traditional ritual when the shaman tries to make it rain, but, oddly enough, it doesn't rain.}}

The special status of the cases $k=10$ and 11 (the problem of \textit{redundant} colors, adding of which does not increase $d$), as well as $k=9$ (a noticeable discrepancy between predictions from graphs and tilings), prompted us to conduct additional research. All these cases are connected by the mysterious number $\sqrt3$.

\paragraph{Obscured by clouds.}
Perhaps, we thought, if we take more points, predictions by linear extrapolation will be more reliable and revealing.

For e-graphs with $k=9$ and 10, we limited the range of $a$ values under study (to reduce the computation time), but for each $a\in L$ in this range, we determined the minimum value of $b$ that led to a $k$-UNSAT solution. As a result, we got a cloud of points (upper graphs in Fig.~\ref{fcloud}). As expected, the dependence $d(r)$ is far from being monotonic. There are both successful pairs $(a, b)$ and pairs that give weak estimates of $d$.

In the previous section, to build a prediction line, we used the optimal values of $r$ (sliding along the bottom edge of the cloud). However, it turned out that if we use all available points to build a straight line (shown in gray in Fig.~\ref{fcloud}), minimizing the standard deviation, the predictions at $r\rightarrow 0$ do not decrease. Perhaps there are not enough points available, or another way to draw a line should be used.

\paragraph{Bi-chromatic vertices.}
Maybe optimizing the position of bi-chromatic vertex will improve the estimates and reduce the prediction bias?

In the same range of $a$ for $k=9$ and 10, we tested all possible distances $s\in[\sqrt{a}, \sqrt{b}]$ between tri- and bi-chromatic vertices. Only some $s$ provide a minimum value of $b$. For example, for $k=9$ we obtain the following optimal sets $(a, b, s^2)$: $(31, 111, 37)$, $(49, 169, 64)$, $(91, 304, 100)$, $(156, 516, 169)$. For some $(a, b)$ there are several such sets. If we do not take into account the position of the bi-chromatic vertex and use an arbitrary $s$ (usually, we took $s=\sqrt{a}$), then this gives weaker estimates: $(31, 121)$, $(49, 171)$, $(91, 309)$, $(156, 523)$.

Thus, in a significant number of cases, the influence of the relative position of the bi-chromatic vertices was confirmed (see Fig.~\ref{fcloud}). But oddly enough, with a general downward shift and a decrease in the spread of $d$ estimates, %the configuration of the point cloud changes so that the predictions at $r\rightarrow 0$ increase, and the straight lines drawn along the center and edges of the cloud diverge (do not shrink to a single point, as expected).
the cloud of points changes so that the predictions at $r\rightarrow 0$ obtained by straight lines drawn along the center and edges of the cloud increase and diverge (do not shrink to a single point, as expected).

Note that the search was not exhaustive: we did not consider all non-isomorphic options of $s$ for placing a bi-chromatic vertex in an e-graph. But this remark concerns only some of $s^2\in L$ %L\"{o}schian numbers 
(for example, 49 and 91).



\begin{figure}[!t]
\centering
{
\centering
\begin{tabular}{ccc}
    \includegraphics[scale=0.35]{f9a.png} & & \includegraphics[scale=0.35]{f10a.png} \\ [1mm]
    \includegraphics[scale=0.35]{f9b.png} & & \includegraphics[scale=0.35]{f10b.png} \\ [1mm]
    \includegraphics[scale=0.35]{f9c.png} & & \includegraphics[scale=0.35]{f10c.png} \\
%    \textit{a}) 
    $k=9$,\;\; $27\le a\le 156$ & & $k=10$,\;\; $27\le a\le 211$ 
\end{tabular} \par
}
\caption{Clouds of e-graph estimates $d(r)$ for $k=9$ and 10. From top to bottom, %in the indicated range $a\in L$, 
all minimal $k$-UNSAT solutions with a random (black) and optimal (blue) position of the bi-chromatic vertex are shown, as well as $k$-SAT solutions (green) corresponding to the latter case. %The values $\sqrt{1/a}$ and $\sqrt{b/a}$ are shown along the horizontal and vertical axes, respectively.
}
\label{fcloud}
\end{figure}



\paragraph{Impregnable fortress.}
We also tried to go the other way and analyze the obtained $k$-SAT solutions for possible clues in the tiling construction, which would break through the $d=\sqrt3$ threshold for $k=10$ and 11. Fig.~\ref{fsat} shows examples of these SAT colorings. It can be seen that the colors of the vertices are grouped and form pseudo-tiles.

\begin{figure}[!t]
\centering
{
\centering
\begin{tabular}{@{}c>{\!}c@{}}
%    \includegraphics[scale=0.17]{k8.png} & \includegraphics[scale=0.17]{k9.png} \\
    \includegraphics[scale=0.17]{k10.png} & \includegraphics[scale=0.17]{k11.png} \\
%    \textit{a}) 
    $k=10$,\; $(m,a,b)=(60,144,508)$ & $k=11$,\; $(m,a,b)=(45,73,279)$ 
\end{tabular} \par
}
\caption{SAT-colorings.}
\label{fsat}
\end{figure}


But we did not notice significant clues here. In the case of 10 colors, a hexagonal tiling with 9 primary colors and one redundant color is clearly visible. And Fig.~\ref{fatt} shows an adaptation of the considered SAT solution using 11 colors, which also does not allow to beat $d=\sqrt3$. Most of these attempts are broken on the "Mercedes logo": three tiles with a common vertex and unit diagonals, the ends of which form an equilateral triangle with side $\sqrt3$. In Fig.~\ref{fatt}, these are triplets of tiles $\{1, 5, 6\}$ and $\{1, 8, 9\}$.


\begin{figure}[!b]
\centering
\includegraphics[scale=0.37]{p11}
\caption{An example of an unsuccessful attempt to overcome the magical threshold $d=\sqrt3$ when tiling a plane with 11 colors.
%Unsuccessful attempt ($d\le\sqrt3$) to tile the plane with 11 colors. 
}
\label{fatt}
\end{figure}


Let us explain how this constraint works in this case. Denote the common vertices of the triples of tiles $\{1, 5, 6\}$, $\{1, 8, 9\}$, $\{4, 5, 8\}$, $\{4, 6, 9\}$ as $O, A, B, C$. Each of the pairs of tiles $\{1, 5\}$, $\{1, 6\}$, $\{5, 6\}$ in a row is bounded by tiles of the same color, that is, sequences of colors 8158, 9169, 4564 are formed. To obtain the maximum of $d$, points $A, B, C$ must be spaced apart in pairs for the maximum distance. But since the distance from these points to the point $O$ cannot be greater than one, then the smallest of the distances $AB$, $AC$ and $BC$ cannot be greater than $\sqrt3$.
We failed to come up with a tiling for $k=11$ (and even more so for $k=10$) in which the described constraint does not occur, or others do not appear that lead to the same result.

