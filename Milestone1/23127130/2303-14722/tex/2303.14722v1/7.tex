
\section{Additions}

\paragraph{Asymptotic estimates}
%Asymptotic estimates 
of $\chi(d)$ for $d\rightarrow\infty$ can be obtained on the basis of the most efficient tiling of the plane and the most dense packing of points.

The densest packing of points on a disk of diameter $d$ with an unlimited increase in the number of vertices $q$ is ensured by their placement at the nodes of a unit hexagonal lattice. The number of points that fit on the disk and give an upper bound of $\chi$ is estimated from the ratio of the areas of the disk and the unit equilateral triangle: $\chi_{ub}\approx\frac{\pi}{\sqrt3}d^2$.

The most efficient tile shape is a regular hexagon with a side of $1/2$, and the best placement of the tile centers is a hexagonal lattice with a step $d+\sqrt3/2$. The required number of colors $\chi$ is obtained by the ratio of the area of the rhombus with side $d$ formed by the lattice to the area of the tile: $\chi_{lb}\approx\frac43d^2$.

Thus, %we get 
$\chi/d^2\in (4/3, \pi/\sqrt3)+o(1)\approx(1.3333, 1.8138)$.
In other words, for large $d$, %in the limit, 
the bounds on $d$ for a fixed $\chi$ differ by a factor of about $7/6$.

\paragraph{Hardware.}
Two computers worked non-stop for a year, checking more than 10\,000 graphs (approximately equal numbers of e- and w-graphs). Some of them required CNF files around 2 GB in size, some took more than a month (unsuccessful with one exception). 

Parameters of computers: i) Intel Core i5-9400F, 2.9 GHz, 6 cores/6 threads, 16 GB RAM; ii) AMD Ryzen5 4600H, 3 GHz, 6 cores/12 threads, 24 GB RAM. Their total performance turned out to be the same. % (it was compared on several graphs). 
Each graph was checked in a separate thread. The computation time is given in terms of the first computer (2 times faster per thread) without taking into account the assembly time of the CNF file.

\paragraph{Observations.}
The \texttt{glucose} and \texttt{kissat} solvers did not differ much in computation time, only on complex graphs requiring many hours, \texttt{kissat} was noticeably more efficient.

In our studies, e-graphs performed better than w-graphs. Apparently, the vertices outside the circle of radius $d$, which we call the \textit{border}, play an important role here. The width of the border (for a fixed $d$) definitely affects both the $k$-colorability and the computational rate, and some optimum is observed.

Only in two cases did the w-graphs prove to be better: i) when estimating $d_{ub}$ for $\chi=9$; ii) when searching for a minimal graph inside the island of stability. In \cite{pmag}, the 18-vertex w-graph allowed us to give a human verifiable proof for $\chi(d)=7$.

The most efficient tiling giving the maximum ratios $d^2/k$ is observed for $k=u^2$, $u\in\mathbb{Z}_{>0}$. In this case, hexagonal tiles of the same color are oriented towards each other with sides (see Fig. 3). A side effect is the inefficiency of subsequent $k$: in the range $k\in[1, 200]$ all $k= u^2+1$ and almost all $k= u^2+2$ give smaller values of $d(k)$, except for $k=6$ and 27. It can be predicted that on all $k= u^2+1$ there will be problems with obtaining tilings that provide $d(k)>d(k-1)$.

\paragraph{Open questions.}
The most obvious (and difficult) goal of subsequent attacks is to break the threshold $d_{lb}=\sqrt3$ for $\chi=10$ and 11, or to prove that this is impossible.
Here are some other questions:

Is it possible to get more efficient tilings by increasing the size of the repeating pattern, for example by using several different tile shapes for each color?
Is it possible to beat the radial coloring of the annulus for $k=6$?
How to explain the frequent repetition of the same values of $a$ and $b$ (for example, 91) in the table of record e-graphs?
What is the main reason for the noticeable difference in the slopes of extrapolation lines for different $k$? (Maybe it's the relative orientation of pseudo-tiles that are assembled from vertices of the same color?)
Is there a bias in the estimates used, and how can it be eliminated?

\paragraph{Conjectures.} 
We have made some progress in confirming Conjecture~\ref{cexoo} by reducing %the lower bound of the island of certainty 
$d_{min}$ from 1.285 to 1.085 for $\chi=7$.
Consistent with Conjecture~\ref{cwes}, we have discovered several new islands of certainty, and have optimistic forecasts for several others.
However, we propose the opposite
\begin{conj}
\label{cpar}
%For some $\chi\ge 7$ the island of certainty does not exist.
For some integers $k\ge 7$, there is no $d$ such that $\chi(d)=k$.
\end{conj}

It means that, for some $\chi\ge 7$, the island of certainty does not exist. In other words, as $d$ grows, the value of $\chi$ can change in non-unit steps. %This conjecture can also be reformulated in terms of redundant colors: some colors are redundant in the sense that their addition does not lead to an increase in $d$. 
Perhaps $k=10$ and 11 are the closest examples of redundant colors for which there is no island of certainty.

\paragraph{Thanks} to Aubrey de Grey for the corrections. Special thanks to Tom Sirgedas for his wonderful program.

In the endless ocean of chromatic numbers it is difficult to find an island of exact knowledge. But if you see several islands at once, this is an occasion to wonder if there is a mainland nearby.
