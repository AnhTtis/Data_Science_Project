%\newpage
\section{Divination on coffee grounds}

\paragraph{Extrapolation.}

Now we will delve into the realm of assumptions, predictions and speculations. The parameters of e- and w-graphs will act as coffee grounds. As a modern divination tool, we will use a ruler.

Let us try to place the estimates $d=\sqrt{b/a}$ given in the Table~\ref{texoo} as points on the coordinate plane depending on the parameter $r=\sqrt{1/a}$, which has the meaning of a resolution element (see Fig.~\ref{fpred}). The dots line up. It can be assumed that with further growth of $a$, new estimates of $d$ %will 
fall approximately on the same line. This, in turn, allows one to predict the asymptotic value of $d$ (column “pred.” in the Table~\ref{tmain}) for $r\rightarrow 0$. 
%that the estimates will tend to as $a$ increases indefinitely (that is, $r\rightarrow 0$).

In Fig.~\ref{fpred}, $k$-UNSAT solutions are shown in black, and the nearest $k$-SAT solutions for the same $a\in L$ are shown in green. Triangles mark especially hard cases, where there are intermediate values of $b\in L$ for which the solver did not give a solution in an acceptable time.

It is clear that various straight lines can be drawn through a small number of reference points, and the degree of scatter of such predictions will be significant. But the general trend can be caught. Linear extrapolation shows what the upper bound of $d(\chi)$ can supposedly be with unlimited growth of the graph. If we focus on this hypothetical bound, we will not only expand the existing islands of certainty, but also get several new ones. The exceptions in the range $6\le k\le 15$ are %the cases 
$k=10$ and 11. %, for which the existence of an island of certainty remains in question.



\begin{figure}[H]
\centering
{
\centering
\begin{tabular}{@{}c@{\;}cc@{\;}c@{}}
 6 & \includegraphics[scale=0.35]{c6}  &  7 & \includegraphics[scale=0.35]{c7}  \\ [1mm]
 8 & \includegraphics[scale=0.35]{c8}  &  9 & \includegraphics[scale=0.35]{c9}  \\ [1mm]
10 & \includegraphics[scale=0.35]{c10} & 11 & \includegraphics[scale=0.35]{c11} \\ [1mm]
12 & \includegraphics[scale=0.35]{c12} & 13 & \includegraphics[scale=0.35]{c13} \\ [1mm]
14 & \includegraphics[scale=0.35]{c14} & 15 & \includegraphics[scale=0.35]{c15} 
\end{tabular} \par
}
\caption{Prediction of asymptotic upper bound of $d$ using estimates $d(r)$ for $6\le k\le 15$. %The horizontal axis shows the value $\sqrt{1/a}$, the vertical axis shows the value $\sqrt{b/a}$. 
Black and green marks correspond to $k$-UNSAT and $k$-SAT solutions for the same $r$.
%obtained for the same optimal $a$. If these solutions don't correspond to the nearest $b\in L$, the points are replaced by triangles. 
Inclined lines are drawn along the black marks.}
\label{fpred}
\end{figure}



For w-graphs, we obtained similar predictions (see Table~\ref{tann}) by assuming $r=2\pi/p$. For $k=5$ and 8, the predicted values of $d$ are even smaller than the lower bounds obtained by annulus tiling. This gives an order of %an idea of the 
magnitude of the prediction error. For $k=3$, a w-graph with the smallest possible $d$ has already been found \cite{pmag}.


\paragraph{Confirmations.}
The above scheme of reasoning may seem unreliable, like spring ice. But here are some indirect confirmations that it works: i) the new points do fit quite well on the lines obtained earlier; ii) for small $k$, the predictions practically coincide with the bounds obtained using tilings; iii) for $k=8$, the bound $d=1.444$ was actually predicted a month before we found the corresponding tiling. At that time, the best tiling gave $d=7/5$ \cite{pgray}, and the extrapolation predicted $d\rightarrow 1.44$ (the new points moved this estimate up). Such a good match not only increases the confidence in such predictions, but also allows us to make the following
\begin{conj}
\label{ck8}
The tiling for $k=8$, shown in Fig.~\ref{fper}, is optimal in terms of forbidden distance interval and cannot be improved.
\end{conj}
If we discard caution, then the same can be conjectured for all other $k\le 9$, as well as for $k=12$, meaning the tilings shown in Fig.~\ref{fhex}. By the way, the prediction $d\rightarrow 1$ for $k=6$ can be considered an argument in favor of Exoo's conjecture~\ref{cexoo}. In tilings for $k\ge 13$, we have less confidence.


\paragraph{Inconsistencies.}

As $k$ increases, there are increasing discrepancies between predictions and estimates based on tilings. So, for $k=9$, instead of the expected $d=\sqrt3$ (the existence of a tiling with $d>\sqrt3$ seems incredible), we get $d\rightarrow 1.755$, and for $k=12$, $d\rightarrow 2.05$ is predicted instead of 2. Even more mysterious are the predictions $d\rightarrow 1.81$ for $k=10$ and 11, despite the fact that all our attempts to get a tiling with $d>\sqrt3$ failed. %were unsuccessful.

%\paragraph{Possible explanations.}
Deviations from expectations require explanation. We tend to believe that there are several sources of bias that shift the estimates predicted by the linear extrapolation method towards larger values.
Here are some of them:
i) non-optimal choice of the $q$-clique, as well as the position of the bi-chromatic vertex;
ii) insufficient graph order;
iii) limited possibilities for choosing better graph parameters with increasing $a$ and/or $k$ due to computational difficulties.
Also, we are not sure that the hexagonal lattice used in e-graphs is optimal and leads to unbiased estimates.

Probably, there also exist some as yet unknown constraints, similar to the ($q+3$)-argument using poly-chromatic vertices, which start working at, say, $d>\sqrt3$, reducing prediction bias.



