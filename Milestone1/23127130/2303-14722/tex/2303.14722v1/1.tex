
%\section{Conjectures}
\section{Introduction}

In the endless ocean of chromatic numbers, it is very difficult to find an island of exact knowledge (except in trivial cases). Usually, instead, we know the bounds on possible values, and improving these bounds is rather slow. Even the sensational breakthrough \cite{grey} of Aubrey de Grey, who narrowed down the set of possible values of the chromatic number of the plane to $\chi\in\{5, 6, 7\}$, can already be considered a page of history\footnote{For more on the history of the coloring problem, see \cite{soi}, and a number of recent developments in this area are outlined in \cite{wes}.}. For other spaces and types of chromatic numbers, the situation is usually even worse.

But even here occasionally there are places where you can feel the earth under your feet. One such place was discovered by Geoffrey Exoo, considering chromatic numbers with an interval of forbidden distances \cite{exoo}.
Namely, for some of these intervals, the exact value of $\chi$ can be found.

\paragraph{Definitions.}
The \textit{chromatic number} of a graph is the minimum number of colors required to color all its vertices so that adjacent vertices take on different colors. Such a coloring is called \textit{proper}. The \textit{chromatic number of the plane} $\chi$ is the chromatic number of an infinite graph, for which the set of vertices coincides with the set of all points of the plane, and the set of edges consists of all pairs of vertices at a unit \textit{distance} from each other. Such a distance is called \textit{forbidden}.

Here we consider $\chi$ as a function of the interval of forbidden distances $[\,1, d\,]$, where $d\ge 1$. By the principle of continuity, $\chi(d)$ is a non-decreasing step function.
As in the case of the traditional chromatic number $\chi(d=1)$, instead of the exact value, only the \textit{bounds} of possible values of $\chi(d>1)$ are usually known. The same is true for the inverse function $d(\chi)$: for each $\chi$, one can specify lower bound $d_{lb}$ and upper bound $d_{ub}$. As a rule, $d_{lb}$ is given by a proper \textit{tiling} of the plane using $\chi$ colors, and $d_{ub}$ can be found using a $\chi$-colorable \textit{finite graph}.

An \textit{island of certainty} is a set %of values 
$d_{min}<d\le d_{max}$ for which the value of $\chi(d)$ is known exactly. The island of certainty arises if $d_{min}<d_{max}$, where $d_{min}(\chi)=d_{ub}(\chi-1)$, and $d_{max}(\chi)=d_{lb}( \chi+1)$.

Note that hereinafter, the letter $d$ denotes the ratio of the upper and lower values of the forbidden distance interval, and not any specific distance. We will also omit subscripts when the meaning is clear from the context.

\paragraph{Background.}

Previously, two islands of certainty were reported. The first such island (with $\chi=7$) was found by Exoo \cite{exoo}. He also proposed%the following
\footnote{
Actually, %in the original 
this conjecture was formulated somewhat differently, but in a private correspondence, Exoo confirmed that there is an inaccuracy in the original formulation.}
\begin{conj}[Exoo]
\label{cexoo}
$\chi(d)=7$ for all $d\in(1, \sqrt7/2\,]$.
\end{conj}

Recently, Joanna Chybowska-Sokół, Konstanty Junosza-Szaniawski, and Krzysztof W\k{e}sek\footnote{
For a number of reasons, we refer to this group of authors as W\k{e}sek et al. Firstly, we want to restore some justice: it's a shame to be at the end of the alphabetical list and always hide behind these "et al." (Although we know a good workaround: articles should be written without co-authors.) Secondly, a male surname is usually more stable than a female one. Finally, so shorter.}
discovered  another island of certainty (with $\chi=9$) and put forward \cite{wes} the following %made a conjecture \cite{wes}, which in our terms looks like this:

\begin{conj}[W\k{e}sek et al.]
\label{cwes}
%Each $\chi\ge 7$ has its own island of certainty.
For any integer $k\ge 7$, there exists $d>1$ such that $\chi(d)=k$.
\end{conj}

Or, in our terms, each $\chi\ge 7$ has its own island of certainty.

The purpose of this work is to expand the known islands of certainty and discover new ones. For the most part, we have limited ourselves to studying small $\chi\le 16$. Since the tradition is already visible here, at the end we will propose our conjecture too.


\begin{figure}[!t]
\centering
\includegraphics[scale=0.35]{chi1.png}
\caption{Chromatic number $\chi(d)$ for small $d$ values. Black bold horizontal lines mark the islands of certainty. The triangles mark the upper bounds of $d$ obtained by linear extrapolation, as explained in the text.}
\label{fchi}
\end{figure}


\paragraph{Main results.}

Our main results are presented in Fig.~\ref{fchi} and in Table~\ref{tmain}.
%(there is no need to read further, since there will be nothing new).

For $d<1$ the value $\chi$ is undefined, but can be extended from the side of the lower bounds $d_{lb}$ (obtained by tilings).


\begin{table}[!b]
{
\caption{Main results. Bounds on distance interval $d(\chi)$.
%Islands of certainty.
}
\label{tmain}
\smallskip
{
\centering
\footnotesize
\begin{tabular}{@{\;}c|*{7}{>{\!}c<{\!}|}>{\!}c@{\;} } 
\hline
\T\small{$\chi$} & \small{island} & \small{lower}   & \multicolumn{2}{>{\!}c<{\!}|}{\small{island of certainty}} & \small{upper} & \small{clique} & \multicolumn{2}{c}{\small{line}}  \\
\cline{4-5} \cline{8-9}
\B              & \small{status} & \small{bound} & \small{min} & \small{max} & \small{bound} & \small{packing} & \small{pred.} & \small{slope}  \\
\hline
\hline \T
 7 & old  & 0.992076 & 1.085134 & 1.322876 & 1.387777 & 1.414214 & 1.00  & 0.96 \\
 8 & new  & 1.322876 & 1.387777 & 1.444157 & 1.526316 & 1.618034 & 1.323 & 1.72 \\
 9 & old  & 1.444157 & 1.526316 & 1.732051 & 1.764000 & 1.902113 & 1.45  & 1.24 \\
10 & ?   & 1.732051 & 1.764000 & 1.732051 & 1.858032 & 2.000000 & 1.755 & 0.75 \\
11 & ?   & 1.732051 & 1.858032 & 1.732051 & 1.941451 & 2.246979 & 1.81  & 0.8  \\
12 & new  & 1.732051 & 1.941451 & 2.000000 & 2.149463 & 2.569237 & 1.81  & 1.3  \\
13 & new  & 2.000000 & 2.149463 & 2.179449 & 2.331924 & 2.777311 & 2.06  & 1.3  \\
14 & pred & 2.179449 & 2.331924 & 2.260808 & 2.456210 & 2.867455 & 2.23  & 1.3  \\
15 & pred & 2.260808 & 2.456210 & 2.346969 & 2.618615 & 2.909313 & 2.31  & 1.4  \\
\B 
16 & pred & 2.346969 & 2.618615 & 2.598076 &          & 3.151196 & 2.52  & 0.8  \\
\hline
\end{tabular}

}
}
\end{table}

In Table~\ref{tmain} we distinguish four (discovery) states of the island of certainty for each $\chi\in[\,7, 16\,]$: “old” and “new” for confirmed islands (previously known and newly discovered), “pred” and “?” for unconfirmed ones (predicted by linear extrapolation and in doubt). Columns 3 to 7 show the bounds in sequence: $d_{lb}$, $d_{min}$, $d_{max}$, $d_{ub}$, $d_{ub}^*$.% (the latter is obtained by point packing, discussed below). 

Pairs of estimates $(d_{lb}, d_{max})$ and $(d_{min}, d_{ub})$ repeat each other with a shift by one row, by definition. %, in accordance with the above definitions of these quantities. 
The record estimates of $d_{lb}$ correspond to the tilings shown in Fig.~\ref{fhex} $(k=6, 7, 9, 12, 15)$ and Fig.~\ref{fper} $(k=8, 14, 15)$, with $\chi =k+1$. The record estimates of $d_{ub}$ obtained by graphs (a typical view of which is shown in Fig.~\ref{fgr}) are taken from Table~\ref{tann} ($k=6$) and Table~\ref{texoo} (bottom rows for each $k\neq 9$), with $\chi=k+3$ and $\chi=k$ respectively. The estimates $d_{ub}^{*}$ obtained by point packing are taken from Table~\ref{tcli}, with $\chi=q+3$.
The last two columns of Table~\ref{tmain} show the linear extrapolation parameters obtained from Fig.~\ref{fpred}: the predicted value of $d_{min}$ and the line slope.

The rest of this paper is organized as follows. In Section 2, we consider the basic tools and constructions for obtaining estimates $d(\chi)$. In Sections 3 and 4, we present lower and upper bounds for $d$ obtained from tilings and graphs. In Section 5, we try to predict further progress in refining these bounds. In Section 6, we discuss some difficult cases on which we are hopelessly stuck. Finally, in Section 7 we make some concluding remarks and formulate our conjecture.% promised earlier.



