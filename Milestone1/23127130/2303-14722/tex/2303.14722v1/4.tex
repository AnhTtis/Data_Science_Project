\section{Graphs}
\paragraph{Point packing.}

The simplest estimate of the upper bound of $d(\chi)$ can be obtained %on the basis of a complete graph with $q$ vertices ($q$-clique), the minimum distance between vertices of which is equal to one. 
from a set of $q$ points such that the distance between any two points falls within the forbidden interval $[\,1,d\,]$, giving a $q$-clique.
Thus, the problem is reduced to finding the most dense packing of points. %vertices, which minimizes the maximum distance between vertices.

Although such an estimate has almost no practical use, the point packing problem is interesting in itself. It is similar to the ball-packing problem, but has its own peculiarities: the width of the finite region covering the centers of the balls (and not the entire balls) is minimized.

The optimization results for some small $q$ are shown in Fig.~\ref{fcli} and Table~\ref{tcli}. It is noteworthy that, as a rule, the point configurations that first come to mind are far from optimal. We used a simple algorithm, but even that proved to be much more successful than trying to find the best solution by hand (with the exception of some symmetrical cases).


\begin{table}[!b]
{
\caption{Minimal width $d$ of $q$-clique for $q\le 40$.
}
\label{tcli}
\smallskip
\scriptsize
{
\centering
\begin{tabular}{@{\;}c@{\;\;}|@{\;\;\;}*{9}{>{\!\!}c<{\!\!\!\!}}*{1}{>{\!\!}c}@{\;}}
\hline
\T\B\
\footnotesize{$q$}  & +1 & +2 & +3 & +4 & +5 & +6 & +7 & +8 & +9 & +10 \\
\hline \T
    +0 & 0.00000 & 1.00000 & 1.00000 & 1.41421 & 1.61803 & 1.90211 & 2.00000 & 2.24698 & 2.56924 & 2.77731 \\
   +10 & 2.86745 & 2.90931 & 3.15120 & 3.31583 & 3.46965 & 3.59016 & 3.75373 & 3.83735 & 3.86370 & 4.09079 \\ 
   +20 & 4.21604 & 4.36509 & 4.47656 & 4.57823 & 4.70821 & 4.77604 & 4.80451 & 5.02067 & 5.09162 & 5.18583 \\
\B +30 & 5.32165 & 5.41447 & 5.49343 & 5.56799 & 5.69445 & 5.73008 & 5.75877 & 5.96806 & 6.02541 & 6.15823 \\
\hline
\end{tabular}

}
}
\end{table}



The calculations were carried out using the \texttt{Mathematica} program. The algorithm is based on the optimization procedure, where the variables are the coordinates of the points, the parameter to be minimized is the largest distance between them, and the constraints are the minimum distance (greater than or equal to one).

The optimization procedure was repeated many times with different initial values of the coordinates, not necessarily satisfying the constraints. At the first stage, we took random initial coordinates in some area. At the next stage, we made small changes to the coordinates that corresponded to the best current solution. Two strategies for changing coordinates were used alternately. The first strategy was to introduce a large ($\pm 2$) random correction to the coordinates of one or several randomly selected vertices. With the second strategy, the coordinates of all vertices were changed by a small random value ($\pm 0.05$). In each case, several hundred iterations were performed. The essence of applying different strategies is to simultaneously solve the problems of approaching the optimum and jumping out of local minima.


\begin{figure}[H]
\centering
{
\centering
\includegraphics[scale=0.28]{q3}\includegraphics[scale=0.28]{q4}\includegraphics[scale=0.28]{q5}\includegraphics[scale=0.28]{q6}\includegraphics[scale=0.28]{q7}\includegraphics[scale=0.28]{q8} \includegraphics[scale=0.28]{q9} \includegraphics[scale=0.28]{q10} \\[2pt]
\includegraphics[scale=0.28]{q11}\includegraphics[scale=0.28]{q12}\includegraphics[scale=0.28]{q13}\includegraphics[scale=0.28]{q14} \includegraphics[scale=0.28]{q15} \includegraphics[scale=0.28]{q16}
\includegraphics[scale=0.28]{q17}\includegraphics[scale=0.28]{q18}\includegraphics[scale=0.28]{q19}\includegraphics[scale=0.28]{q20}\includegraphics[scale=0.28]{q21}
\includegraphics[scale=0.28]{q22}\;\includegraphics[scale=0.28]{q23}\;\includegraphics[scale=0.28]{q24}\;\includegraphics[scale=0.28]{q25}
\includegraphics[scale=0.28]{q26}\;\includegraphics[scale=0.28]{q27}\;\includegraphics[scale=0.28]{q28}\;\includegraphics[scale=0.28]{q29}
\includegraphics[scale=0.28]{q30}\;\includegraphics[scale=0.28]{q31}\;\includegraphics[scale=0.28]{q32}\;\includegraphics[scale=0.28]{q33} \\[3pt]
\includegraphics[scale=0.28]{q34}\;\includegraphics[scale=0.28]{q35}\;\includegraphics[scale=0.28]{q36}\;\includegraphics[scale=0.28]{q37}
}
\vspace{-4mm}
\caption{Packing of $q$-cliques for $3\le q\le 37$. The minimal %(1) 
and maximal %($d$) 
distances between vertices are shown in black and gray, respectively.}
\label{fcli}
\end{figure}


The considered algorithm reveals weak convergence, which can be explained by the presence of many local minima in the search area; only a small fraction of iterations leads to an improvement in the result. %Finding better algorithm is an open problem.
However, the algorithm is quite efficient. The results show that i) as $q$ increases, the symmetry of the packings is usually broken; ii) a fairly common pattern is concentric circles of points; iii) in contrast to tilings, there is a monotonic increase in the estimate of $d$ with the number of colors $q$.

\begin{figure}[!t]
\centering
\includegraphics[scale=0.35]{chi2.png}
\caption{The lower bounds on $\chi(d)$ obtained by packing the points of the $q$-clique (lower black curve) and by e- and w-graphs (upper green curve). The dotted line shows an estimate based on a $q$-clique on a hexagonal lattice with step $1/\sqrt{91}$, which turns out to be efficient for many e-graphs.}
\label{flow}
\end{figure}

Fig.~\ref{flow} shows lower bounds on $\chi$ based on point packing given the $(q+3)$-argument. For comparison, estimates based on e- and w-graphs are also shown: it can be seen that with increasing $d$ they diverge more and more.


%\newpage
\paragraph{W\k{e}sek graphs.}

We tried to play with the w-graph parameters, resulting in a slight improvement in numerical estimates compared to the results \cite{wes} of W\k{e}sek et al. (see Table~\ref{tann}, where the latter are listed in the “previous best” column). In the original, W\k{e}sek et al. gave "exact" values of the bounds using trigonometric expressions, but this does not make much sense, since there are no fundamental difficulties in further improving these results. % (apparently, W\k{e}sek et al. chose the parameters of the graph arbitrarily). 
We limited ourselves to approximate values. When searching by $d$, we used a step of 0.001 or 0.01 depending on $k$.

We noticed that the dependence of the estimate $d$ on the number of circles $c$ and the number of vertices on the circle $p$ is extremely non-monotonic with narrow local minima. For example, for $k=4$ and $d=1.457$, the local optimum is observed at the following values of $p$: 716, 824, 932, 974, 1040, $\dots$ (we don't see patterns here). For intermediate values of $p$, the estimate of $d$ increases.

A uniform arrangement of the radii of the circles (inside the annulus) is also generally not optimal. For example, for $k=6$ with three circles and $d=1.786$ the optimal radius of the middle circle is about 1.54 ($p=953, 1007,\dots$). With a uniform arrangement of circles, some of them can be discarded. So, for the graph with parameters $(k, d, p, c, q)=(6, 1.764, 294, 12, 4)$, it is possible to remove 6 out of 12 circles without loss of $k$, with numbers 2, 3, 4, 5, 9, 11 (if we count from the center).


\begin{table}[!t]
{
\caption{Estimates of the distance interval $d$ based on coloring of annuli. %Annulus based results.
}
\label{tann}
\smallskip
\smallskip
{
\centering
\footnotesize
\begin{tabular}{%@{\;}c|@{\;}c@{\;}||
@{\;}c@{\;}|
*{3}{>{\!\!}c<{\!\!}|}*{1}{>{\!\!\!}c<{\!\!\!}|}*{1}{>{\!}c<{\!}|}*{1}{>{\!\!\!}c<{\!\!\!}|}*{2}{>{\!\!}c<{\!\!}|}
*{1}{>{\!}c<{\!}|}@{\;}r@{\;}|
*{1}{>{\!}c<{\!}|}>{\!}c@{\;}
}
\hline
\T
%\small{$\chi$} & \small{plane} & 
\small{$k$}   & \multicolumn{2}{c|}{\small{annulus tiling}}   & \multicolumn{3}{@{}c@{}|}{\small{previous best}} & \multicolumn{5}{@{}c@{}|}{\small{our best w-graph}} & \multicolumn{2}{c}{\small{line}} \\
%\cline{4-5}
\cline{2-13}
\B
%& \small{tiling \cite{wes}} &
& \small{radial} & \small{\!arbitrary\!%special
} & \small{$d$} & \small{$p$} & \small{$c$} & \small{$d$} & \small{$p$} & \small{$c$} & \small{$q$} & \small{time, s} & \small{pred.} & \small{slope} \\
\hline
\hline \T
  3 & 1.285575 & 1.285575 & 1.28599 & 1300 & 2 & 1.2856&   18 &  1 & 2 & 0      &  &  \\
  4 & 1.414214 & 1.414214 & 1.47145 &  180 & 2 & 1.457 &  716 &  2 & 3 & 1.9    & 1.452 & 0.52 \\
  5 & 1.618034 & 1.691392 & 1.71433 &  180 & 3 & 1.696 &  388 &  3 & 4 & 2.2    & 1.685 & 0.54 \\
  6 & 1.732051 & 1.732051 & 1.82843 &  120 & 3 & 1.764 &  294 & 12 & 4 & 23     & 1.749 & 0.72 \\
  7 & 1.801938 & 1.847759 & 2.01176 &  120 & 3 & 1.893 &  363 &  4 & 4 & 182    & 1.869 & 1.23 \\
  8 & 1.847759 & 1.940393 &         &      &   & 1.975 &  232 & 12 & 5 & 15813  & 1.927 & 1.91 \\
  9 & 1.879385 & 2.175091 &         &      &   & 2.224 &  296 & 12 & 6 & 235    & 2.200 & 1.17 \\
 10 & 1.902113 & 2.23649* &         &      &   & 2.380 &  259 & 12 & 7 & 1835   & 2.357 & 1.27 \\
 11 & 1.918986 & 2.33483* &         &      &   & 2.560 &  136 &  6 & 7 & 137504 & 2.498 & 1.18 \\
 12 & 1.931852 & 2.532089 &         &      &   & 2.687 &  150 &  6 & 8 & 139626 & 2.615 & 1.71 \\
\hline
\end{tabular}

}
}
\end{table}

% $k=5$, $d=1.691391691...$ & $k=6$, $d=1.732050807...$ \\ [15pt]
% $k=7$, $d=1.847759065...$ & $k=8$, $d=1.940392662...$ \\ [15pt]
% $k=9$, $d=2.175091460...$ & $k=12$, $d=2.53208888624...$

Initially, we tried to apply a brute force method, reducing $d$ only due to a significant increase in $p$ and $c$. But, as practice has shown, an exhaustive search by these parameters turned out to be more effective. Of course, in this case, it was necessary to check a much larger number of graphs, but the total time for checking them still turned out to be much less compared to what was spent on large graphs in the brute force method.
Table~\ref{tann} shows the time of checking single graph with the found optimal parameters using the SAT solver \texttt{glucose}.



\paragraph{Exoo graphs.}

In \cite{exoo}, Exoo used the pair $(a, b)=(25, 43)$ on a 203-vertex graph with $k=6$, which roughly corresponds to our $\oplus^8 H$ graph and gives a $k$-UNSAT solution on $d=\sqrt {43/25}\approx 1.311488$. Checking it with the \texttt{glucose} solver takes about 1000 seconds. Using the pair $(36, 52)$ on $\oplus^8 H$, we get $d=\sqrt{52/36}\approx 1.201850$, but this takes about %$5\cdot 10^5$ 
500\,000 seconds, or almost a week. But if we use a modified %cation of this 
graph with tri- and bi-chromatic vertices, the verification time is reduced to about 10 seconds.
%In further calculations, we used a modified version of the e-graphs with tri- and bi-chromatic vertices. For comparison: on the last graph considered, this reduced the check time to about 10 seconds.

We continued the hunt for records, successively achieving lower and lower estimates of $d$ for several values of $k$.
Simplistically, we adhered to the following scheme for finding the minimum values of $d=\sqrt{b/a}$. We start with a pair of small initial values $(a, b)$, forcing $k$-UNSAT on some suitable graph $\oplus^m H$. With a fixed $a$, the value of $b$ is gradually decreased until a $k$-SAT solution is reached (thus, the preceding $k$-UNSAT solution gives the record value of $d$). Further, the resulting estimate $d$ acts as a threshold, and at approximately constant $d$, the values of $a$ and $b$ are gradually increased until a $k$-UNSAT solution is found. Then the value of $a$ is fixed again and the cycle repeats.

The results are shown in Table~\ref{texoo}. Here, for each $k$, the parameters of several graphs that give a $k$-UNSAT solution are shown, including the number of %$l\in L$
L\"{o}schian distances $l$ in the range $[\,a, b\,]$, the order of the maximum clique $q$, and the check time in the \texttt{kissat} solver (with the \texttt{--forcephase} key that Marijn Heule opened us in secret). %The order of the graph $\oplus^m H$ is $3m^2+3m+1$. 
For most graphs (except for the largest ones), the minimum value of $m$ is given with a step of 5. An asterisk marks the cases for which the dependence of $\chi$ on the position of the bi-chromatic vertex was observed. For $k=9$ and 10, such a study was carried out purposefully, for others it was a by-product of repeated calculations on different solvers. % or using different pre-colored $q$-cliques.

%In total, we tested several thousand e-graphs. The time of checking one graph ranged from fractions of a second to several days and even months. 
In most cases, an obstacle to a further decrease in the value of $d(\chi)$ was the increasing computation time. For $k=7$ and 9, the limiting factor was the growth of $m$ and, as a result, the size %(gigabytes) 
and the assembly time of CNF files for SAT solver. Suitable values of $m$ were chosen empirically to obtain a $k$-UNSAT solution. On average, the ratio $m/\sqrt{b}$ was about 1.3, but sometimes more than 2 was required. %In other words, an effective graph must have a sufficient proportion of vertices (a \textit{border}) beyond the outer annulus radius $\sqrt{b}$.


\begin{table}[!t]
{
\caption{Exoo type graphs.}
\label{texoo}
\smallskip
\footnotesize

{
\centering

\begin{tabular}{@{}cc@{}}
%\begin{tabular}{@{}c*{6}{|>{\!\!}c<{\!\!}}|@{\,}r@{}}
\begin{tabular}{@{}c@{\;}|@{\;}c@{\;}|@{\;}c@{\;}|@{\:}c@{\:}|@{\:}c@{\:}|@{\;}c@{\;}|@{\;\:}c@{\;\:}|@{\,}r@{}}
\hline
\T\B\
\small{$k$}  & \small{$a$} & \small{$b$} & \small{$d$} & \small{$l$}& \small{$m$} & \small{$q$} & \small{time, s} \\
\hline
\hline \T
 6 &  13 &  21 & 1.27098 &   4 &  5 & 3 & 0.1 \\
   &  19 &  28 & 1.21395 &   5 & 10 & 3 & 1.9 \\
   &  28 &  39 & 1.18019 &   5 & 10 & 3 & 37 \\
   &  49 &  64 & 1.14286 &   6 & 10 & 3 & 1394 \\
   &  73 &  91 & 1.11650 &   7 & 15 & 3 & 41939 \\
   &  91 & 111 & 1.10444 &   8 & 15 & 3 & 169147 \\
   & 169 & 199 & 1.08513 &  11 & 30 & 3 & 4176344 \\
\hline \T
 8 &  27 &  76 & 1.67774 &  18 & 10 & 4 & 0.9 \\
   &  48 & 127 & 1.62660 &  27 & 15 & 4 & 41 \\
   &  57 & 148 & 1.61136 &  30 & 20 & 4 & 89 \\
   &  79 & 201 & 1.59509 &  38 & 20 & 4 & 403 \\
   &  91 & 225 & 1.57243 &  41 & 20 & 4 & 1277 \\
   & 169 & 403 & 1.54422 &  68 & 30 & 4 & 32728 \\
   & 361 & 841 & 1.52632 & 127 & 65 & 4 & 866306 \\
\hline \T
10 &  25 &  97 & 1.96977 &  25 & 15 & 6 &*1.6 \\
   &  36 & 133 & 1.92209 &  33 & 20 & 5 & 47 \\
   &  91 & 331 & 1.90719 &  71 & 20 & 5 & 433 \\
   & 124 & 441 & 1.88586 &  90 & 35 & 5 & 7347 \\
   & 169 & 589 & 1.86687 & 116 & 40 & 5 & 44824 \\
   & 208 & 724 & 1.86568 & 139 & 45 & 5 & 189431 \\
   & 241 & 832 & 1.85803 & 158 & 45 & 5 & 290312 \\
\hline \T
12 &  49 & 247 & 2.24518 &  61 & 20 & 7 & 47 \\
   &  91 & 439 & 2.19640 & 100 & 25 & 7 & 2694 \\
   & 124 & 589 & 2.17945 & 128 & 30 & 7 & 16705 \\
   & 163 & 763 & 2.16356 & 162 & 30 & 7 & 69195 \\
   & 189 & 877 & 2.15412 & 186 & 40 & 7 & 90155 \\
   & 208 & 961 & 2.14946 & 198 & 40 & 7 & 330248 \\
\hline \T
14 &  13 &  93 & 2.67467 &  28 & 10 & 9 & 0.4 \\
   &  25 & 171 & 2.61534 &  47 & 15 & 8 & 833 \\
   &  39 & 256 & 2.56205 &  66 & 20 & 8 & 22768 \\
   &  49 & 304 & 2.49080 &  76 & 30 & 8 & 212177 \\
   &  91 & 549 & 2.45621 & 129 & 30 & 8 & 2263048 \\
\hline
\end{tabular}

&
\begin{tabular}{@{}c@{\;}|@{\;}c@{\;}|@{\,}c@{\,}|@{\:}c@{\:}|@{\:}c@{\:}|@{\;}c@{\;}|@{\;\:}c@{\;\:}|@{}r@{}}
\hline
\T\B\
\small{$k$}  & \small{$a$} & \small{$b$} & \small{$d$} & \small{$l$}& \small{$m$} & \small{$q$} & \small{\,time, s} \\
\hline
\hline \T
 7 & 121 & 268 & 1.48825 &  42 & 25 & 4 & 2.5 \\
   & 183 & 388 & 1.45610 &  59 & 25 & 4 &*7 \\
   & 217 & 444 & 1.43041 &  65 & 30 & 4 & 12 \\
   & 300 & 604 & 1.41892 &  84 & 35 & 3 & 627 \\
   & 507 & 988 & 1.39596 & 125 & 45 & 3 & 6199 \\
   & 567 &1101 & 1.39348 & 139 & 50 & 3 & 7743 \\
   & 675 &1300 & 1.38778 & 159 & 60 & 3 & 15341 \\
\hline \T
 9 &  31 & 111 & 1.89226 &  27 & 25 & 5 &*17 \\
   &  49 & 169 & 1.85714 &  37 & 20 & 5 &*10 \\
   &  91 & 304 & 1.82775 &  63 & 25 & 5 &*40 \\
   & 156 & 516 & 1.81871 & 101 & 40 & 5 &*13 \\
   & 268 & 876 & 1.80794 & 163 & 35 & 5 & 75 \\
   & 361 &1161 & 1.79334 & 209 & 50 & 5 & 100 \\
   & 541 &1731 & 1.78875 & 297 & 55 & 5 & 174 \\
\hline \T
11 &  21 &  91 & 2.08167 &  24 & 15 & 7 & 429 \\
   &  36 & 151 & 2.04803 &  38 & 15 & 7 & 13 \\
   &  49 & 193 & 1.98463 &  45 & 25 & 6 & 23027 \\
   &  63 & 247 & 1.98006 &  57 & 20 & 6 & 12223 \\
   &  73 & 283 & 1.96894 &  63 & 25 & 6 & 125967 \\
   &  91 & 343 & 1.94145 &  75 & 25 & 6 & 307255 \\
   &&&&&&\\
\hline \T
13 &  21 & 133 & 2.51661 &  38 & 15 & 8 & 5 \\
   &  36 & 217 & 2.45515 &  56 & 15 & 8 & 35 \\
   &  49 & 283 & 2.40323 &  70 & 20 & 8 &*1034 \\
   &  91 & 511 & 2.36968 & 119 & 25 & 8 & 2252 \\
   & 144 & 793 & 2.34669 & 176 & 35 & 8 & 105923 \\
   & 169 & 919 & 2.33192 & 201 & 40 & 8 & 143781 \\
\hline \T
15 &  13 &  97 & 2.73158 &  29 & 15 & 9 & 62 \\
   &  28 & 201 & 2.67929 &  55 & 15 & 9 & 850 \\
   &  36 & 252 & 2.64575 &  67 & 20 & 9 & 6769 \\
   &  49 & 337 & 2.62251 &  87 & 25 & 9 & 132533 \\
   &  91 & 624 & 2.61861 & 148 & 30 & 9 & 477925 \\
\hline
\end{tabular}
\end{tabular}

}
}
\end{table}


