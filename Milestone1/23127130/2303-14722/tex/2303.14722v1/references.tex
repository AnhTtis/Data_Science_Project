

\begin{thebibliography}{100}

\bibitem{wes}
J. Chybowska-Sokół, K. Junosza-Szaniawski, and K. W\k{e}sek.
Coloring distance graphs on the plane. 
\textit{arXiv}:2201.04499v1, 12 Jan 2022.
%\small \texttt{
%https://arxiv.org/abs/2201.04499}
%\normalsize

\bibitem{cou}
D. Coulson. On the chromatic number of plane tilings.
\textit{Journal of the Australian Mathematical Society}, vol.~77, no.~2, 2004, pp.~191--196.

\bibitem{cur}
J. D. Currie and R. B. Eggleton. Chromatic properties of the Euclidean plane.
\textit{arXiv}:1509.03667, 11 Sep 2015.

\bibitem{exoo}
G. Exoo. $\varepsilon$-unit distance graphs.
\textit{Discrete} \& \textit{Computational Geometry}, vol.~33, %no.~2, 
2005, pp.~117--123.

\bibitem{grey}
A.D.N.J. de Grey.
The chromatic number of the plane is at least 5.
\textit{Geombinatorics}, vol.~28, no.~1, 2018, pp.~18--31.

\bibitem{pgray} % gray
A.D.N.J. de Grey, J. Parts.
Tiling the plane with hexagons: improved separations for $k$-colourings.
\textit{Geombinatorics}, vol.~32, no.~2, 2022, pp.~57--71.

\bibitem{heu}
M.J.H. Heule.
Computing small unit-distance graphs with chromatic number 5.
\textit{Geombinatorics}, vol.~28, no.~1, 2018, pp.~32–-50.

\bibitem{pmag} % magenta
J. Parts.
On the plane and its coloring.
\textit{Geombinatorics}, vol.~31, no.~4, 2022, pp.~189--195.

\bibitem{sir}
T. Sirgedas. 
Hadwiger-Nelson problem / Simulator.
\small \texttt{
https:// groups.google.com/g/hadwiger-nelson-problem/c/a701Kwnhp\_A}
\normalsize

\bibitem{soi}
A. Soifer. \textit{The mathematical coloring book. Mathematics of coloring and the colorful life of its creators}. Springer, New York, 2009. 


\comment{%-----------------------------------------------------
\bibitem{psma} % small
J. Parts.
A small 6-chromatic two-distance graph in the plane.
\textit{Geombinatorics}, vol.~29, no.~3, 2020, pp.~111--115.

\bibitem{plar} % large
J. Parts. 
Graph minimization, focusing on the example of 5-chromatic unit-distance graphs in the plane, \textit{Geombinatorics}, vol.~29, No.~4, 2020, pp.~137--166.

\bibitem{pgrn} % green
J. Parts. 
What percent of the plane can be properly 5- and 6-colored?
\textit{Geombinatorics}, vol.~30, No.~1, 2020, pp.~25--39.
\textit{arXiv}:2010.12668.
	
\bibitem{ppin} % pink
J. Parts. 
The chromatic number of the plane is at least 5 – a human-verifiable proof.
\textit{Geombinatorics}, vol.~30, No.~2, 2020, pp.~77--102.
\textit{arXiv}:2010.12661.

\bibitem{pblu} % blue
J. Parts.
On upper bounds for the multi-fold chromatic numbers of the plane.
\textit{Geombinatorics}, vol.~30, no.~4, 2021, pp.~177--189.

\bibitem{pbro} % brown
J. Parts.
A 6-chromatic odd-distance graph in the plane.
\textit{Geombinatorics}, vol.~31, no.~3, 2022, pp.~124--137.

\bibitem{pyel} % yellow
J.H. Conway, A.D.N.J. de Grey, J. Parts, A. Soifer. 
Is there a better visualization of Coulson’s "15-colouring of 3-space omitting [monochromatic] distance one"?
\textit{Geombinatorics}, vol.~31, no.~4, 2022, p.~156.

\bibitem{pmag} % magenta
J. Parts.
On the plane and its coloring.
\textit{Geombinatorics}, vol.~31, no.~4, 2022, pp.~189--195.

\bibitem{pgry} % gray
A.D.N.J. de Grey, J. Parts.
Tiling the plane with hexagons: improved separations for $k$-colourings.
\textit{Geombinatorics}, vol.~32, no.~2, 2022, pp.~57--71.

\bibitem{peme} % emerald
A.D.N.J. de Grey, J. Parts.
On lower bounds of the order of $k$-chromatic unit distance graphs.
\textit{Geombinatorics}, vol.~32, no.~2, 2022, pp.~72--74.



\bibitem{ard}
H. Ardal, J. Ma\v{n}uch, M. Rosenfeld, S. Shelah, and L. Stacho.
The odd-distance plane graph.
\textit{Discrete} \& \textit{Computational Geometry}, vol.~42, no.~2, 2009, pp.~132--141.

\bibitem{cro}
H.T. Croft.
Incidence incidents.
Eureka (Cambridge), No.~30, 1967, pp.~22-26.

\bibitem{exoo}
G. Exoo and D. Ismailescu.
A 6-chromatic two-distance graph in the plane.
\textit{Geombinatorics}, vol.~29, no.~3, 2020, pp.~97--103.

\bibitem{heu2}
M.J.H. Heule.
Odd-distance virtual edges in unit-distance graphs.
\textit{Geombinatorics}, vol.~31, no.~2, 2021, pp.~68--76.

\bibitem{pri}
D. Pritikin.
All unit-distance graphs of order 6197 are 6-colorable.
\textit{Journal of Combinatorial Theory}, Series B 73, 1998, pp.~159--163 

\bibitem{pol}
D.H.J. Polymath.
Comments in Polymath16, thread 17, Dec 2021 -- Feb 2022,
%https://dustingmixon.wordpress.com/2021/02/01/polymath16-seventeenth-thread-declaring-victory/\#comment-33033
\small \texttt{
https://asone.ai/polymath/index.php?title= Hadwiger-Nelson\_problem
}\normalsize

}%end of comment----------------------------------------------

\end{thebibliography}

