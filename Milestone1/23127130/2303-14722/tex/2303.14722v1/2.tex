%\section{Constructions}
\section{Preliminaries}
\paragraph{Poly-chromatic vertices.}

The transition from a single forbidden distance ($d=1$) to a non-zero interval of forbidden distances ($d>1$) greatly simplifies the proof of some facts and changes the numerical estimates. For example, the straight line $\mathbb{R}^1$ can no longer be properly colored in two colors, and a third one is needed. And in the plane $\mathbb{R}^2$, this immediately raises the trivial estimate based on an equilateral triangle from $\chi(d=1)\ge 3$ to $\chi(d>1)\ge 6$.

Indeed, in the case of $d>1$, we can always place the unit triangle on an arbitrarily colored plane in such a way that the $\varepsilon$-neighborhood of one of its vertices will contain at least three colors (will be tri-chromatic), and the $\varepsilon$-neighborhood of another vertex will contain at least two colors (will be bi-chromatic). We call this observation the $(q+3)$-argument, which reflects the increase in $\chi$ compared to the estimate based on a $q$-vertex \textit{clique}. Corresponding but stronger theorems are also available, for example, in Coulson \cite{cou}, Currie-Eggleton \cite{cur}, and W\k{e}sek et al. \cite{wes}. In fact, the latter repeats Coulson's argument: any coloring of the plane in two colors leads to the formation of monochrome stripes or annuli, the length or diameter of which exceeds one.

%\newpage 

\paragraph{Graph constructions.}
We will use, with some modifications, the constructions proposed by Exoo \cite{exoo} and by W\k{e}sek et al. \cite{wes}, which we call e-graph and w-graph after their authors, respectively.

By an e-graph we mean a finite graph $r\! \oplus^m\! H$ with $3m^2+3m+1$ vertices located on a hexagonal lattice with step $r=\sqrt{1/a}$ and edges formed by all pairs of vertices at a distance from 1 to $d=\sqrt{b/a}$. Here $a, b\in L=\{u^2+uv+v^2;\; u, v\in\mathbb{Z}_{\ge 0}\}$, the so-called \textit{L\"{o}schian numbers}; $H$ is a 7-vertex \textit{wheel graph} with edges of unit length; and $\oplus^m$ is the Minkowski sum applied to $m$ identical copies of the graph (e.g. $\oplus^3 H=H\oplus H\oplus H$). In our case, the e-graph is bounded by a hexagon with side $r\cdot m$, unlike the original \cite{exoo}, where a rectangle was used. %The \textit{border} is the region of the e-graph that extends beyond the disk of radius $d$. 
It is convenient to omit the normalizing factor $r$ and use the interval of forbidden distances $[\sqrt{a}, \sqrt{b}]$.

In a w-graph, all vertices are located inside the \textit{annulus}, that is, in the area between two concentric circles with radii 1 and $d$. More precisely, $c$ circles are used, each of which has $p$ vertices evenly spaced. Since all $p\cdot c$ vertices are at a forbidden distance from the center, then by placing a tri-chromatic vertex there, we immediately increase the estimate $\chi$ by 3.

%To calculate the chromatic number of graphs, we use SAT solvers. To speed up the calculations, we first select the maximum clique with $q$ vertices and color them in different colors. 
Unlike the original constructions of e- and w-graphs, we use both tri- and bi-chromatic vertices, which gives a noticeable improvement in estimates.

Fig.~\ref{fgr} shows examples of e- and w-graphs. The edges are not shown (there are too many of them). Instead, selected vertices adjacent to tri- and bi-chromatic vertices are highlighted. %The vertices of the $q$-clique are also highlighted and the 
The main parameters of the graphs are also listed. Note that the graphs also have other hidden parameters (such as %the configuration of the $q$-clique, 
the position of the bi-chromatic vertex). %, but we do not discuss them here.

\begin{figure}[!t]
\centering
{
\centering
\begin{tabular}{@{}c@{\:}c@{}}
    \includegraphics[scale=0.18]{e_graph.png} & \includegraphics[scale=0.18]{w_graph.png} \\
%    \textit{a}) 
    e-graph, $(m,a,b,q)=(12,27,76,4)$ & %\textit{b}) 
    w-graph, $(p,c,q)=(40,4,5)$ 
\end{tabular} \par
}
\caption{Examples of e- and w-graphs with $d\approx 1.678$. The vertices of the pre-colored $q$-clique are enlarged. The tri-chromatic vertex is placed in the center of the graph, the bi-chromatic one is above it. Black and dark gray highlight vertices adjacent to tri- and bi-chromatic vertices, respectively.}
\label{fgr}
\end{figure}

It can be seen that for the same $d$, the w-graph occupies a smaller area and may have a larger $q$-clique and a smaller number of vertices (especially for small $\chi$). In addition, the $(q + 3)$-argument can significantly reduce the complexity of the computational problem. % (the SAT solver works more and more slowly with an increase in the number of colors).

%However, studies have shown that e-graphs give better estimates of $d(\chi)$ in comparable time. So, only in the case of $\chi=9$ the w-graph showed better results. Also, e-graphs have better potentialities: the minimum achievable values of $d(\chi)$ for a w-graph are noticeably limited by annulus tiling (which we discuss in the next section).


\paragraph{Colors.}
Along with $\chi$, we use the notation $k$ for the number of colors. But one should be careful with this parameter, because in the considered cases, depending on the context, the difference $\chi-k$ can be from 0 to 5. In particular, tiling the plane with $k$ colors gives $d_{lb}$ for $\chi$ of $k$, and tiling the annulus with $k$ colors gives a lower bound on $d_{min}$ for $\chi$ of $k+4$, which can be obtained using a w-graph together with a tri-chromatic vertex.

\paragraph{Tools.}
To calculate $\chi$ of graphs, we use so-called \textit{SAT solvers}. The input information of the solver is a \textit{propositional formula} in CNF format that describes the structure of the graph assuming its proper $k$-coloring (see \cite{heu} for details). To speed up calculations, the vertices of one of the maximum $q$-cliques of the graph are preliminarily colored, which is taken into account in the formula. If the graph is $k$-colorable, then the formula is \textit{satisfiable} ($k$-SAT solution), otherwise it is \textit{unsatisfiable} ($k$-UNSAT solution). The solver may also give no solution in some reasonable time. To determine $\chi$, one need to check several values of $k$.


