%\newpage
\section{Tilings}
Our main task is to color all points of the plane with a given number of colors $k$ so as to maximize the interval of forbidden distances $[\,1, d\,]$ for which any two points take on a different color.

In the case of tilings, this problem can be reformulated as follows: for a given $k$, cover the plane with tiles, each of which receives one of $k$ colors and has a \textit{width} (maximum size) not exceeding one, so as to maximize the distance $d$ between the nearest tiles of the same color. This formulation allows us to include in consideration the cases $k<7$ as well.

\paragraph{Identical tiles.}

We start with the case when all tiles have the same shape and are obtained from each other by translation. Such tilings, discussed in detail in \cite{pgray}, demonstrate high efficiency in tiling the plane. Here we briefly repeat some of the results.


\begin{figure}[!t]
\centering
\includegraphics[scale=0.37]{ploe_}
\caption{Trivial tiling of the plane with regular hexagonal tiles. The base tile is highlighted in gray. Numbered tiles are those closest to the base tile for the given number of colors $k$. Mirrored copies of such tiles are omitted for clarity.}
\label{floe}
\end{figure}

The simplest, and usually the most efficient, are regular hexagons with side length $1/2$. But here tiles of the same color must form a regular hexagonal sublattice, which is possible only if $k\in L$ %is the so-called \textit{L\"{o}schian number} ($L$) of the form $k=u^2+uv+v^2$ for some integers $u>0, u\ge v\ge 0$ 
(see Fig.~\ref{floe}). 
In other cases, irregular hexagons give larger value of $d$. Table~\ref{thex} shows the optimal values of $d$ for all $k\in[1,\, 200]$. The cases $k\in L$ are highlighted in gray. It can be seen that an increase in $k$ does not always lead to an increase in $d$. Moreover, for some $k$, a decrease in $d$ is observed. However, this is due to the implicit requirement for tiles of the same color to form a lattice. Relaxing this requirement yields a nondecreasing function $d(k)$, but leaves the question of whether this function is strictly increasing in the case of arbitrary tilings.



\begin{table}[!t]
{
\caption{Minimum distances for lattice-sublattice coloring.}
\label{thex}
\smallskip
%\linespread{0.9}
%\small
%\footnotesize
\scriptsize

{
\centering
\scriptsize
\begin{tabular}{@{\;}c@{\;\;}|@{\;\;\;}*{9}{>{\!\!}c<{\!\!\!\!}}*{1}{>{\!\!}c}@{\;}}

\hline
\T\B\
\footnotesize{$k$}  & +1 & +2 & +3 & +4 & +5 & +6 & +7 & +8 & +9 & +10 \\
\hline \T
   +0 &\gr 0.00000 & 0.00000 &\gr 0.50000 &\gr 0.86603 & 0.83333 & 0.99208 &\gr 1.32288 & 1.40000 &\gr 1.73205 & 1.65831 \\
  +10 & 1.67416 &\gr 2.00000 &\gr 2.17945 & 2.17945 & 2.18661 &\gr 2.59808 & 2.51979 & 2.47863 &\gr 2.78388 & 2.86282 \\
  +20 &\gr 3.04138 & 3.01579 & 2.94299 & 3.17878 &\gr 3.46410 & 3.39077 &\gr 3.50000 &\gr 3.60555 & 3.42423 & 3.76229 \\
  +30 &\gr 3.90512 & 3.87199 & 3.91379 & 4.00412 & 3.98392 &\gr 4.33013 &\gr 4.27200 & 4.19539 &\gr 4.44410 & 4.41935 \\
  +40 & 4.50035 & 4.65107 &\gr 4.76970 & 4.73559 & 4.60794 & 4.84520 & 4.78005 &\gr 5.00000 &\gr 5.19615 & 5.13667 \\
  +50 & 5.06837 &\gr 5.29150 & 5.12431 & 5.33379 & 5.25115 & 5.53345 &\gr 5.63471 & 5.60196 & 5.47726 & 5.69491 \\
  +60 &\gr 5.76628 & 5.69604 &\gr 5.89491 &\gr 6.06218 & 6.00871 & 5.99567 &\gr 6.14410 & 5.98639 & 6.17729 & 6.20261 \\
  +70 & 6.16402 & 6.41174 &\gr 6.50000 & 6.46932 &\gr 6.50000 &\gr 6.55744 & 6.16188 & 6.54545 &\gr 6.72681 & 6.70343 \\
  +80 &\gr 6.92820 & 6.87989 & 6.81757 &\gr 7.00000 & 6.94304 & 7.02700 & 6.82668 & 6.97285 & 7.15312 & 7.28729 \\
  +90 &\gr 7.36546 & 7.33696 &\gr 7.36546 & 7.40759 & 7.28947 & 7.44444 &\gr 7.56637 & 7.35575 & 7.59056 &\gr 7.79423 \\
 +100 & 7.75030 & 7.70955 &\gr 7.85812 & 7.80525 & 7.88072 & 7.69172 & 7.84502 &\gr 8.00000 &\gr 8.04674 & 8.16091 \\
 +110 &\gr 8.23104 &\gr 8.20462 & 8.10131 & 8.26755 & 8.15518 & 8.30005 &\gr 8.41130 & 8.18005 & 8.33073 & 8.47751 \\
 +120 &\gr 8.66025 & 8.62006 & 8.56481 &\gr 8.71780 & 8.58790 & 8.73717 &\gr 8.76071 & 8.70315 &\gr 8.84590 & 8.70760 \\
 +130 & 8.85303 & 9.03313 &\gr 9.09670 & 9.07217 & 8.97583 & 9.12889 & 9.11716 & 9.11822 &\gr 9.26013 & 9.21474 \\
 +140 & 9.18245 & 9.31939 & 9.35781 &\gr 9.52628 & 9.48928 & 9.45424 &\gr 9.57862 &\gr 9.53939 & 9.59554 & 9.52353 \\
 +150 &\gr 9.65660 & 9.68793 & 9.58933 & 9.71491 & 9.73937 &\gr 9.90430 &\gr 9.96243 & 9.93960 & 9.84962 & 9.99120 \\
 +160 & 9.95752 & 9.99490 &\gr 10.1119 & 9.97235 & 10.1062 & 10.1655 & 10.0539 & 10.2373 &\gr 10.3923 & 10.3581 \\
 +170 &\gr 10.3320 &\gr 10.4403 & 10.3258 & 10.4553 &\gr 10.4762 & 10.4581 & 10.5400 & 10.3539 & 10.4812 & 10.5999 \\
 +180 &\gr 10.6888 & 10.7747 &\gr 10.8282 & 10.8069 & 10.7747 & 10.8542 & 10.3829 & 10.8426 &\gr 10.9659 & 10.9258 \\
\B +190 &10.8944 &\gr 11.0148 &\gr 11.0340 & 10.9440 & 11.1148 &\gr 11.2583 & 11.2265 & 11.1803 &\gr 11.3027 & 11.2188 \\
\hline
\end{tabular}

}
}
\end{table}



\begin{table}[!t]
{
\caption{Classes of coloring for lattice-sublattice scheme.}
\label{tclas}
\smallskip

{
\centering
\footnotesize
\begin{tabular}{@{\;}c@{\;\;}|l@{}}
\hline
\T\B\small{class} & \small{number of colors $k$} \\
\hline
\hline \T
$L+$ & 3, 4, 7, 9, 12, 13, 16, 19, 21, 25, 27, 28, 31, 36, 39, 43, 48, 49, 52, 57, 61, 63, \\
     & 64, 67, 73, 76, 79, 81, 84, 91, 97, 100, 103, 108, 109, 111, 117, 121, 124, 127, \\
     & 129, 133, 139, 144, 147, 151, 156*, 157, 163, 169, 172, 175, 181, 183, 189, \\
     & 192, 193, 196, 199 \\
\hline \T $L-$ & 37, 75, 93, 112, 148, 171 \\
\hline \T
$\overline{L}+$ & 6, 8, 20, 24, 30, 33, 34, 41, 42, 46, 54, 56, 60, 69, 70, 72, 86, 89, 90, 94, 96, \\
     & 99, 105, 110, 114, 116, 120, 126, 131, 132, 136, 142, 143, 149, 152, 154, 155, \\
     & 156*, 160, 162, 166, 168, 174, 177, 180, 182, 186, 195 \\
\hline \T
$\overline{L}-$ & 1, 2, 5, 10, 11, 14, 15, 17, 18, 22, 23, 26, 29, 32, 35, 38, 40, 44, 45, 47, 50, 51, \\
     & 53, 55, 58, 59, 62, 65, 66, 68, 71, 74, 77, 78, 80, 82, 83, 85, 87, 88, 92, 95, 98, \\
     & 101, 102, 104, 106, 107, 113, 115, 118, 119, 122, 123, 125, 128, 130, 134, 135, \\
     & 137, 138, 140, 141, 145, 146, 150, 153, 158, 159, 161, 164, 165, 167, 170, 173, \\
     & 176, 178, 179, 184, 185, 187, 188, 190, 191, 194, 197, 198, 200 \\
\hline
\end{tabular}

}
}
\end{table}

%\newpage



In Table~\ref{tclas}, values of $k$ are divided into four classes $\{L+, L-, \overline{L}+, \overline{L}-\}$ depending on whether they belong to L\"{o}schian numbers %$L$-numbers 
and whether they lead to an increase (+) in $d$ compared to all previous values\footnote{
The case $k=156$ falls into two classes, $L+$ and $\overline{L}+$, since both variants of tilings exist for it, and both ensure the growth of $d$. And irregular hexagons are more efficient here.}.

Fig.~\ref{fhex} shows the best hexagonal tilings identified in [6] for some $k$. Only one of the $k$ colors is shown, the rest are obtained by translation.


%\newpage
\paragraph{Different tiles.}

We can expand the search for tilings by allowing tiles of different shapes. (To keep the search practical, we however constrain the search by giving each color its own fixed tile shape.) Does this help to increase $d$? Yes, for some $k$ we  found superior tilings, as shown in Fig.~\ref{fper}.



\begin{figure}[H]
\centering
{
\centering
\begin{tabular}{@{}c@{\;}cc@{\;}c@{}}
 6 & \includegraphics[scale=0.25]{t6}  & 7 & \includegraphics[scale=0.25]{t7}  \\ [2mm]
 8 & \includegraphics[scale=0.25]{t8}  & 9 & \includegraphics[scale=0.25]{t9}  \\ [2mm]
 10& \includegraphics[scale=0.25]{t10} & 11& \includegraphics[scale=0.25]{t11} \\ [2mm]
 12& \includegraphics[scale=0.25]{t12} & 13& \includegraphics[scale=0.25]{t13} \\ [2mm]
 14& \includegraphics[scale=0.25]{t14} & 15& \includegraphics[scale=0.25]{t15} \\
\end{tabular} \par
}
\caption{Optimal hexagonal tilings found in \cite{pgray} for $6\le k\le 15$.}
\label{fhex}
\end{figure}



\begin{figure}[H]
\centering
{
\centering
\begin{tabular}{@{}c@{\;\;}c@{}}
8  & \includegraphics[scale=0.37]{p8}  \\ [4mm]
14 & \includegraphics[scale=0.37]{p14} \\ [4mm]
15 & \includegraphics[scale=0.37]{p15} \\
\end{tabular} \par
}
\caption{Non-trivial periodic tilings of the plane with 8, 14, and 15 colors.}
\label{fper}
\end{figure}



When searching for better tilings, we used the \texttt{HNT} program \cite{sir}, specially developed for solving such problems by Tom Sirgedas. The program allows you to set the tiling area, restrictions on the width and distance between tiles, as well as specify the initial location of the tiles, after which it %automatically 
tries to optimize the shape of the tiles. We checked the resulting tilings using the \texttt{FindMaximum} function of the \texttt{Mathematica} software package.

At first glance at the tilings for $k=8$ and 14, one might get the impression that this is a heap of heterogeneous tiles (in the case of $k=15$, the periodicity of the pattern is more obvious). But, if you look closely, you will see the symmetry.

For $k=8$ we got the estimate\footnote{
We give a lot of digits for verification purposes.} $d\approx1.444157346767732$. The tiling includes tiles of three different shapes (considering reflections as equivalent): heptagons $\{1, 2, 8\}$, pentagons $\{3, 5, 7\}$, and hexagons $\{4, 6\}$. If the coloring is not taken into account, the tiling has the symmetry of a regular triangle with respect to the common vertex of the heptagons. All diagonals of pentagons have unit length. Replacing the edge between tiles 2 and 8 with tiles of the ninth color results in a 9-coloring with hexagons.

For $k=14$ we got $d\approx2.260808070967297$. The tiling has third-order rotational symmetry about the center of tiles 1 and 12, and includes tiles of six shapes: $\{1\}$, $\{2, 4, 6\}$, $\{3, 5, 7\}$, $\{8, 11, 13\}$, $\{9, 10 , 14\}$, $\{12\}$.

For $k=15$ we got $d\approx2.346969102448257$. The tiling has horizontal and vertical axes of symmetry passing through the centers of tiles 1 and includes tiles of five shapes: $\{1\}$, $\{2, 4, 6, 8\}$, $\{3, 5, 7, 9\}$, $\{10, 11\}$, $\{12, 13, 14, 15\}$.


None of our attempts to get a more efficient tiling for $k=10$ and 11 were successful (we will return to this problem below).

\paragraph{What's next?}

Is it possible to obtain even more efficient tilings? This question remains open. Note that so far we made some progress whenever we made the tilings more complex: we started with the same shape of all tiles, going from regular hexagons to less symmetrical ones, then we used a different tile shape for each color. One can try to expand the search even more: for example, to abandon the use of a fixed shape of tiles for each color, or to use a different average number of tiles for different colors. However, so far we do not know of any such tiling that would be more efficient in our task.



\newpage
\paragraph{Tiling of annuli}
%Annulus tiling 
helps to determine the potential of the w-graph approach. Moreover, the task of annulus coloring can be interesting in itself.

The coloring is constructed in such a way that the inner and outer radii of the annulus numerically coincide with the boundaries of the  forbidden distance interval $[\,1, d\,]$. If the annulus can be tiled using $k$ colors, then any w-graph on this annulus is $k$-colorable. %then this gives a lower bound on $d$ for a $(k+4)$-chromatic w-graph.

W\k{e}sek et al. limited themselves to considering so-called \textit{radial} colorings, in which the annulus is partitioned into sectors bordered by straight lines from its center (and arcs of radii 1 and $d$). It is clear that such a simple construction cannot give estimates $d>2$, that is, it is not applicable for large $k$.

Fig.~\ref{fann} shows the tilings of the annuli with the largest values of $d$ that we were able to obtain. All annuli are drawn to the same scale and have the same inner radius (although optical illusion convinces us otherwise).

As in the case of the plane, to find the optimal annulus tiling, we used the \texttt{HNT} program and the optimization function of the \texttt{Mathematica} software package.
%To find the optimal tiling, we used the \texttt{HNT} program \cite{sir}, specially developed for solving such problems by Tom Sirgedas. The program allows you to set the tiling area, restrictions on the size of the tiles and the distance between them, as well as specify the initial location of the tiles, after which it automatically tries to optimize the shape of the tiles. We checked the resulting tilings using the \texttt{FindMaximum} function of the \texttt{Mathematica} software package. 
The refined values of $d$ are shown in the column "arbitrary" of Table~\ref{tann}. For comparison, the "radial" column shows the estimates obtained using the radial coloring.

A radial coloring turned out to be the best for $k=3, 4$, and 6. In other cases, a tiling with multiple "floors" (distances from the center) gives better results. In most cases, we have obtained symmetrical colorings. For $k=10$ and 11, we got an asymmetric coloring by taking the initially symmetrical tiling with $k=9$ and $d\approx 2.175091$, and manually adding tiles of new colors. For these two tilings, the values of $d$ found in the \texttt{HNT} program were not checked (marked with an asterisk in the Table~\ref{tann}).

As tilings show, compared to e-graphs, the w-graph approach has fundamental limitations. For example, for $k=3$ (or $\chi=7$) it is impossible to get $d<2\sin(2\pi/9)\approx 1.285575$, while with the help of e-graphs this barrier can be easily overcome. However, our estimates do not unambiguously indicate the inefficiency of W\k{e}sek's approach for larger $k$. 

We expect that some our estimates of $d$ can be improved.
It is noteworthy that for $k=6$ we failed to improve the radial coloring with $d=\sqrt3$. As you see, %will see later (and already seen), 
this value of $d$ is a tough nut to crack. The same difficulties arise when tiling the plane.


\begin{figure}[H]
\centering
{
\centering
\includegraphics[scale=0.2]{a3}\;\;\;\includegraphics[scale=0.2]{a4}\;\;\;\;\;\,\includegraphics[scale=0.2]{a5}\;\;\, \\[2pt]
\includegraphics[scale=0.2]{a6}\,\includegraphics[scale=0.2]{a7}\,\includegraphics[scale=0.2]{a8} \\[0pt]
\raggedright
\includegraphics[scale=0.2]{a9}\;\;\; \includegraphics[scale=0.2]{a10} \\[0pt]
\raggedleft
\includegraphics[scale=0.2]{a11}\;\; \includegraphics[scale=0.2]{a12}\; \\[0pt]
}
\caption{Coloring of annuli for $3\le k\le 12$.}
\label{fann}
\end{figure}

