
\newcommand{\head}[1]{#1}


 \begin{center}
 \footnotesize
     \resizebox{1.0\columnwidth}{!}{
	\begin{tabularx}{\textwidth}{p{.135\textwidth}X}
		\toprule
		\head{Transformation} & \head{Description}
		% % % % % % % % % % % % % % % % % %	% % % % % % % % %		
		\\
		\midrule
		\addlinespace[5pt]
	    Reference addition &
		Given a database of bibtex records of real papers, this transformation adds papers to the bibliography.
		%
		It has two options:
		\begin{itemize}
		    \item \emph{Add unmodified papers.} This option treats the insertion as optimization problem. It tries to find $k$ bibtex records such that the number of added words is maximized. This allows us to maximize the impact of the added papers in the bibliography.
		    \item \emph{Add adapted papers.} This option adds $r$ words into a randomly selected bibtex record, which is then added to the bibliography. This transformation allows us to add very specific words which are difficult to add in the normal text in a meaningful way. In the experiments, $r$ is set to 3, \ie, each added bibliography entry has only 3 additional words to avoid suspicion.
		\end{itemize}
		\\ %
		Synonym &
		This transformation replaces words by synonyms using a security domain-specific word embedding~\cite{mikolov-13-distributed}. To this end, the word embeddings are computed on a collection of 11,770 security papers (Section~\ref{sec:discussion} presents the dataset). Two options are implemented:
		\begin{itemize}
		    \item \emph{Add.} Allows adding a word. For each word in the text, it obtains its synonyms. If one of the synonyms is in the list of words that should be added, the synonym is used as replacement for the text word.
		    \item \emph{Delete.} Allows removing a word by replacing it with one of its synonyms.
		\end{itemize}
		The transformation iterates over possible synonyms and only uses a synonym if it has the same part-of-speech (POS) tag as the original word. From the remaining set of synonyms, the transformation randomly chooses a candidate. 
		%
		\\ \tabspac
		Spelling mistake &
		Inserts a spelling mistake into a word that should be deleted. 
		\begin{itemize}
		    \item \emph{Most common misspelling.} This option tries to find a misspelling from a list of 78 rules for most common misspellings, such as \emph{appearence} instead of \emph{appearance} (rule: ends with -ance), or \emph{basicly} instead of \emph{basically} (rule: ends with -ally).
		    \item \emph{Swap or delete.} Swap two adjacent letters or delete a letter in the word. Chooses between both ways randomly.
		\end{itemize}
		The transformation first tries to find a common misspelling, and if not possible, it applies the swap-or-delete strategy.
		\\ \tabspac
		Language model & 
        Uses a language model, here \mbox{OPT}~\cite{zhang-22-opt}, to create sentences with the requested words. To create more security-related sentences, we use the corpus from Section~\ref{sec:discussion} consisting of 11,770 security papers to finetune the OPT-350m model. Equipped with this model, the transformer appends new text at the end of the related work or discussion section. To this end, we extract some text before the insertion position and ask the model to complete the text while choosing suitable words from the set of requested words. 
		\\
		\midrule
		\addlinespace[5pt]
		Homoglyph &
		Replaces a single character in a word by a visually identical or similar homoglyph. 
		For instance, we can replace the Latin letter \emph{A} by its Cyrillic equivalent.
		\\ \tabspac
		\midrule
		\addlinespace[5pt]
		Hidden box &
		Uses the accessibility support with the latex package \emph{accsupp} that allows defining an alternative text over an input text. Only the input text is visible, while the feature extractor processes the alternative text. This allows adding an arbitrary number of words as alternative text. As the input text is not processed, we can also delete words or text in this way. Two options are implemented:
		\begin{itemize}
			\item \emph{Add.} Allows adding an arbitrary number of 
			words in the alternative text. This step requires defining
			the alternative text at least over a visible word that is,
			however, not extracted as feature afterwards anymore. To reduce side effects,
			the transformation first checks if the attack requests a word to be reduced.
			If so, it lays the alternative text over this word. Otherwise, a stop word is 
			chosen that would be ignored in the preprocessing stage 
			anyway. The step thus reduces possible side effects. 
            % 
		   \item \emph{Delete.} Adds an empty alternative text over the input word that needs to be removed, so that the word is not extracted anymore.\vspace{-0.8em}
		   % 
		\end{itemize}\\
		\bottomrule
	\end{tabularx}}
 \end{center}
	
%