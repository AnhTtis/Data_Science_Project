\section{Training Corpus}
\label{app:corpus}
\vspace{-0.5em}

In the following, we describe how the corpus for the simulated paper-reviewer assignment process is generated. 
%
The PC of the \emph{43rd IEEE Symposium on Security and Privacy} conference consists of 165 persons. For each PC member, we construct an archive of papers representative for the person's expertise and interests by crawling their \emph{Google Scholar} profile. In rare cases, this profile is not available and we use the profile from \emph{DBLP computer science bibliography} instead. We sort all papers first by year and then by number of citations to obtain an approximation of the recent research interests. From this list, we remove all papers with no citation and for which we cannot obtain a PDF file (e.g., paywalled files we cannot access). Furthermore, we remove papers that are already used as a target submission. From the remaining list, we select the first 40 papers (if available). 

To construct reviewer archives $\archive_\reviewer$, we randomly sample 20 paper for each reviewer and compile the corpus as the union of these archives. The remaining 20 papers are used to simulate the black-box scenario. Here, we consider different levels of overlaps between 0\% (i.e., no overlap between the training data of the surrogates and target system) and 100\% (i.e., complete overlap).

\section{Hyperparameter Selection}
\label{app:hyperparameters}
\vspace{-0.5em}
In the following, we describe how we determine the hyperparameters of our attack in the two scenarios.

\paragraph{White-box scenario.}
We perform a grid search over 100 randomly sampled targets from all three objectives and optimize parameters as a trade-off between attack efficacy and efficiency. To not overfit to a specific model, we train 8 AutoBid systems on different random seeds and randomly select one system per target. Note that an attacker with full-knowledge could also choose parameters that perform best for a specific target. We set the beam width $\beamwidth = 8 \in \{2^0, \dots, 2^3\}$, step size $\stepsize = 128 \in \{2^5, \dots, 2^8\}$, number of successors $\nosuccessors = 512 \in \{2^7,\dots, 2^{9}\}$, and reviewer window to $\reviewerwindow = 6 \in \{2^1, \dots, 2^3\}$ with offset $\revieweroffset = 3 \in \{0, \dots, 3\}$. The target rank for rejected reviewer is set to $rank^{rej}_{\submission} = 10$ and we consider $\reviewerwordsmax = 5000$ words per reviewer. We run the attack for at most $\maxitr = 1000$ iterations with at most $\switches = 8$ transitions between spaces and a target margin of $\margin = -0.02$. 

\paragraph{Black-box scenario.}
We repeat the grid search and train 8 systems on a surrogate corpus at 70\% overlap. We randomly sample 100 targets from all three objectives and assign each a random surrogate system. We set the beam width $\beamwidth = 4 \in \{2^0, \dots, 2^3\}$, step size $\stepsize = 256 \in \{2^5, \dots, 2^8\}$, number of successors $\nosuccessors = 128 \in \{2^7,\dots, 2^{9}\}$, and reviewer window to $\reviewerwindow = 2 \in \{2^1, \dots, 2^3\}$ with offset $\revieweroffset = 1 \in \{0, \dots, 3\}$. Finally, to increase the robustness of our attack, we set the margin as $\margin = -0.16$. All other parameters are the same as before.

\section{Feature-Space Search}
\label{app:boxplots}
\vspace{-0.5em}

We report the $L_1$ norms of individual attacks exemplary for the selection objective. We consider 8 different assignment system and sample 100 random targets per system (i.\,e., 800 attacks in total).\smallskip

\vspace{-1em}
\begin{center}
\includegraphics[trim=10 10 10 10, clip, width=\columnwidth]{figures/boxplots_appendix_small.pdf}  
\end{center}

\section{Generalization of Attack}
\label{app:generalizaton}
\vspace{-0.5em}

We empirically evaluate our attack on two conferences with differently sized committees: (a) the \emph{29th USENIX Security Symposium} with 120 reviewers and (b) the \emph{43rd IEEE Symposium on Security and Privacy} conference with a larger committee consisting of 165 reviewers. We simulate the attack for all three objectives and report the aggregated success rate, the median running time, and the median $L_1$ and $L_\infty$ norm.

\vspace{-1em}
\begin{center}
\resizebox{1\columnwidth}{!}{
\begin{tabular}{@{}lcccc@{}}
\toprule
& Success Rate & Running Time & \phantom{aaa}$L_1$\phantom{aaa} & \phantom{aaa}$L_\infty$\phantom{aaa} \\
\midrule
USENIX '20 & 99.62\,\% & 7m~38s & $1033$ & $30$ \\
\rule{0pt}{2ex}%<--- do not remove
IEEE S\&P '22 & 99.67\,\% & 7m~12s & $1115$ & $35$ \\
\bottomrule
\end{tabular}}
\end{center}

\section{Scaling of Target Reviewer}
\label{app:scaling}
\vspace{-0.5em}

We run the attack for different combinations of the number of selected and rejected target reviewers. For each combination, we report the median $L_1$ norm as well as the success rate over 100 targets.

\begin{center}
    \includegraphics[trim=10 10 10 22, clip, width=0.95\columnwidth]{figures/scaling.pdf}
\end{center}

\newpage
\section{Surrogate Ensembles}
\label{app:surrogate-boxplots}
\vspace{-0.5em}

We report the $L_1$ norms for the black-box scenario with varying sizes of surrogate ensembles. We report the $L_1$ over 100 targets for all three objectives.

\begin{center}
\includegraphics[trim=10 10 10 10, clip, width=\columnwidth]{figures/surrogates_boxplots_appendix_small.pdf}  
\end{center}

\section{Overlap}
\label{app:cp-overlap}
\vspace{-0.5em}

We compare the cross-entropy of reviewer-to-words distributions across models trained on a corpus with different overlaps. We randomly select 10 reviewers and report the mean cross-entropy and standard deviation between 8 models each (i.e., 64 pairs per overlap and reviewer).

\begin{center}
\scriptsize
 \resizebox{0.9\columnwidth}{!}{
        \begin{tabular}{@{}ccccc@{}} 
            \toprule
             \multirow{2}{*}{\#}   & \multicolumn{4}{@{}c}{Overlap} \\
             & 0\% & 30\% & 70\% & 100\%  \\
            \midrule
1 &  $13.19\pm0.46$ &        $13.13\pm0.47$ &        $13.12\pm0.37$ &        $13.20\pm0.44$ \\
% \rule{0pt}{1ex}%<--- do not remove
2 &  $12.56\pm0.29$ &        $12.55\pm0.37$ &        $12.64\pm0.34$ &        $12.50\pm0.29$ \\
% \rule{0pt}{1ex}%<--- do not remove
3 &  $13.58\pm0.63$ &        $13.56\pm0.56$ &        $13.47\pm0.62$ &        $13.52\pm0.63$ \\
% \rule{0pt}{1ex}%<--- do not remove
4 &  $12.43\pm0.50$ &        $12.29\pm0.48$ &        $12.35\pm0.54$ &        $12.32\pm0.50$ \\ 
% \rule{0pt}{1ex}%<--- do not remove
5 &  $13.41\pm0.51$ &        $13.41\pm0.61$ &        $13.50\pm0.56$ &        $13.31\pm0.66$ \\
% \rule{0pt}{1ex}%<--- do not remove
6 &  $12.84\pm0.23$ &        $12.81\pm0.21$ &        $12.93\pm0.25$ &        $12.90\pm0.23$ \\
% \rule{0pt}{1ex}%<--- do not remove
7 &  $14.20\pm0.42$ &        $14.28\pm0.44$ &        $14.39\pm0.48$ &        $14.08\pm0.41$ \\ 
% \rule{0pt}{1ex}%<--- do not remove
8 &  $13.57\pm0.46$ &        $13.59\pm0.46$ &        $13.55\pm0.40$ &        $13.66\pm0.42$ \\ 
% \rule{0pt}{1ex}%<--- do not remove
9 &  $13.44\pm0.72$ &        $13.33\pm0.68$ &        $13.54\pm0.67$ &        $13.44\pm0.76$ \\ 
% \rule{0pt}{1ex}%<--- do not remove
10 &  $15.24\pm0.59$ &        $15.08\pm0.59$ &        $15.31\pm0.66$ &        $14.88\pm0.61$ \\ 
\bottomrule
        \end{tabular}
    }
\end{center}

\newpage
\onecolumn
\enlargethispage{1em}

\begin{multicols}{2}

\section{Committee Size}
\label{app:committee-size}
\vspace{-0.5em}

We simulate the attack with varying sizes of the program committee. For each size, we report the mean number of required modifications over 8 target systems each sampled with a random committee. For each objective, we compute 280 adversarial papers per target system.
\vspace{-0.5em}

\begin{center}
\includegraphics[trim=10 10 10 0, clip, width=\columnwidth]{figures/committees.pdf}  
\end{center}

\section{Load balancing}
\label{app:load-balancing}
\vspace{-0.5em}

We simulate the attack with varying numbers of concurring submissions between 200 and 1,000. We report the mean success rate over 8 target systems each sampled with a random committee. For each objective, we compute 280 adversarial papers per target system.
\vspace{-0.5em}

\begin{center}
\includegraphics[trim=10 10 10 0, clip, width=\columnwidth]{figures/load_balancing.pdf}  
\end{center}
\end{multicols}

\vspace{-1.25em}
\section{Problem-Space Transformations}
\label{app:problem-space-transformations}
\vspace{-0.5em}
Detailed description of problem-space transformations. For the transformations' categorization, see Table~\ref{table:problem-space-overview-transformations}.
\vspace{-0.25em}

\newcommand{\head}[1]{#1}


 \begin{center}
 \footnotesize
     \resizebox{1.0\columnwidth}{!}{
	\begin{tabularx}{\textwidth}{p{.135\textwidth}X}
		\toprule
		\head{Transformation} & \head{Description}
		% % % % % % % % % % % % % % % % % %	% % % % % % % % %		
		\\
		\midrule
		\addlinespace[5pt]
	    Reference addition &
		Given a database of bibtex records of real papers, this transformation adds papers to the bibliography.
		%
		It has two options:
		\begin{itemize}
		    \item \emph{Add unmodified papers.} This option treats the insertion as optimization problem. It tries to find $k$ bibtex records such that the number of added words is maximized. This allows us to maximize the impact of the added papers in the bibliography.
		    \item \emph{Add adapted papers.} This option adds $r$ words into a randomly selected bibtex record, which is then added to the bibliography. This transformation allows us to add very specific words which are difficult to add in the normal text in a meaningful way. In the experiments, $r$ is set to 3, \ie, each added bibliography entry has only 3 additional words to avoid suspicion.
		\end{itemize}
		\\ %
		Synonym &
		This transformation replaces words by synonyms using a security domain-specific word embedding~\cite{mikolov-13-distributed}. To this end, the word embeddings are computed on a collection of 11,770 security papers (Section~\ref{sec:discussion} presents the dataset). Two options are implemented:
		\begin{itemize}
		    \item \emph{Add.} Allows adding a word. For each word in the text, it obtains its synonyms. If one of the synonyms is in the list of words that should be added, the synonym is used as replacement for the text word.
		    \item \emph{Delete.} Allows removing a word by replacing it with one of its synonyms.
		\end{itemize}
		The transformation iterates over possible synonyms and only uses a synonym if it has the same part-of-speech (POS) tag as the original word. From the remaining set of synonyms, the transformation randomly chooses a candidate. 
		%
		\\ \tabspac
		Spelling mistake &
		Inserts a spelling mistake into a word that should be deleted. 
		\begin{itemize}
		    \item \emph{Most common misspelling.} This option tries to find a misspelling from a list of 78 rules for most common misspellings, such as \emph{appearence} instead of \emph{appearance} (rule: ends with -ance), or \emph{basicly} instead of \emph{basically} (rule: ends with -ally).
		    \item \emph{Swap or delete.} Swap two adjacent letters or delete a letter in the word. Chooses between both ways randomly.
		\end{itemize}
		The transformation first tries to find a common misspelling, and if not possible, it applies the swap-or-delete strategy.
		\\ \tabspac
		Language model & 
        Uses a language model, here \mbox{OPT}~\cite{zhang-22-opt}, to create sentences with the requested words. To create more security-related sentences, we use the corpus from Section~\ref{sec:discussion} consisting of 11,770 security papers to finetune the OPT-350m model. Equipped with this model, the transformer appends new text at the end of the related work or discussion section. To this end, we extract some text before the insertion position and ask the model to complete the text while choosing suitable words from the set of requested words. 
		\\
		\midrule
		\addlinespace[5pt]
		Homoglyph &
		Replaces a single character in a word by a visually identical or similar homoglyph. 
		For instance, we can replace the Latin letter \emph{A} by its Cyrillic equivalent.
		\\ \tabspac
		\midrule
		\addlinespace[5pt]
		Hidden box &
		Uses the accessibility support with the latex package \emph{accsupp} that allows defining an alternative text over an input text. Only the input text is visible, while the feature extractor processes the alternative text. This allows adding an arbitrary number of words as alternative text. As the input text is not processed, we can also delete words or text in this way. Two options are implemented:
		\begin{itemize}
			\item \emph{Add.} Allows adding an arbitrary number of 
			words in the alternative text. This step requires defining
			the alternative text at least over a visible word that is,
			however, not extracted as feature afterwards anymore. To reduce side effects,
			the transformation first checks if the attack requests a word to be reduced.
			If so, it lays the alternative text over this word. Otherwise, a stop word is 
			chosen that would be ignored in the preprocessing stage 
			anyway. The step thus reduces possible side effects. 
            % 
		   \item \emph{Delete.} Adds an empty alternative text over the input word that needs to be removed, so that the word is not extracted anymore.\vspace{-0.8em}
		   % 
		\end{itemize}\\
		\bottomrule
	\end{tabularx}}
 \end{center}
	
%
