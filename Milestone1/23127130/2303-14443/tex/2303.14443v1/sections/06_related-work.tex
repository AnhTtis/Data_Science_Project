\section{Related Work}
\label{sec:related}

Our attack touches different areas of security research. In the following, we examine related concepts and methods.

\paragraph{Adversarial learning.} A large body of work has focused on methods for creating adversarial examples that mislead learning-based systems~\cite{biggio-18-wild}. However, most of this work considers attacks in the image domain and assumes a one-to-one mapping between pixels and features. This assumption does not hold in discrete domains, leading to the notion of \emph{problem-space attacks}~\cite{pierazzi-20-intriguing, quiring-19-misleading}. Our work follows this research strand and introduces a new hybrid attack strategy for operating in both the feature space and problem space.
Furthermore, we examine weak spots in \emph{text preprocessing}, which extend the attack surface for adversarial papers. 
These findings complement prior work advocating that the security of preprocessing in machine learning needs be considered in general~\cite{quiring-20-adversarial}. 

Table~\ref{table:related-work-overview} summarizes prior work on misleading text classifiers. While we build on some insights developed in these works, text classification and paper assignment differ in substantial aspects: First, the majority of prior work focuses on untargeted attacks that aim at removing individual features. In our case, however, we have to consider a targeted attack where an adversary needs to specifically change the assignment of reviewers. Second, prior attacks often directly exploit the gradient of neural networks or compute a gradient by using word importance scores. Such gradient-style attacks are not applicable to probabilistic topic models. 

In view of these differences, our work is more related to the attack from \citet{zhou-20-evalda} which studies the manipulation of \ac{LDA}. The authors show that an evasion is \emph{NP-hard} and present an attack to promote and demote individual \ac{LDA} topics. For our manipulation, however, we need to adjust not only individual topics but the complete topic distribution as well as consider side effects with concurring reviewers. 

\begin{table}[t]
\centering
\footnotesize
\caption{Overview of related attacks against text classifiers.}
\vspace{-0.1cm}
\resizebox{\columnwidth}{!}{
\begin{tabularx}{\columnwidth}{p{.125\textwidth}C{0.125cm}C{0.125cm}C{0.125cm}C{0.125cm}C{0.125cm}C{0.15cm}C{0.125cm}C{0.125cm}X}
\toprule 
& \multicolumn{4}{c}{Perturbation} & \multicolumn{2}{c}{Constr.} & \multicolumn{2}{c}{Attack} & \\
\cmidrule(lr){2-5} \cmidrule(lr){6-7} \cmidrule(lr){8-9} 
Paper & \rotatebox{90}{Char} & \rotatebox{90}{Word} & \rotatebox{90}{Sentence}  & \rotatebox{90}{Format} & \rotatebox{90}{Semantics} & \rotatebox{90}{Plausibility} & \rotatebox{90}{Untargeted} & \rotatebox{90}{Targeted} & Classifier \\
\midrule
\emph{This work} & \CIRCLE & \CIRCLE & \CIRCLE & \CIRCLE & \cmark & \cmark & \cmark & \cmark & Assign. \\
\citet{alzantot-18-generating} & \CIRCLE & \CIRCLE & &  & \cmark & \xmark & \cmark & \xmark & NN \\
\citet{ebrahimi-18-hotflip} & \CIRCLE & \CIRCLE & & & \cmark & \xmark & \cmark & \xmark & NN  \\
\citet{eger-19-text} & \CIRCLE & & &  & \cmark & \xmark & \cmark & \xmark  &  NN   \\
\citet{gao-18-blackbox} & \CIRCLE & & &  & \cmark & \xmark & \cmark & \xmark  & NN  \\
\citet{iyyer-18-adversarial} & & & \CIRCLE & & \cmark & & \cmark & \xmark & NN \\
\citet{jin-20-bert} & & \CIRCLE & &  & \cmark & \xmark & \cmark & \xmark  & NN \\
\citet{li-19-textbugger} & \CIRCLE & \CIRCLE & & & \cmark & \xmark & \cmark & \xmark  & NN,LR \\
\citet{liu-20-joint} & \CIRCLE & \CIRCLE & &  & \cmark & \xmark & \cmark & \xmark  & NN \\
\citet{papernot-16-crafting} & & \CIRCLE & & & \xmark & \xmark & \cmark & \xmark  & NN \\
\citet{ren-19-generating} & & \CIRCLE & & & \cmark & \xmark & \cmark & \xmark  & NN \\
\bottomrule
\end{tabularx}}
\label{table:related-work-overview}
% \vspace{-1.5em}
\end{table}
\paragraph{Attacks on assignment systems.}
Finally, another strain of research has explored the robustness of paper-reviewer assignment systems. 
Most of these works are based on \emph{content-masking attacks}~\cite{markwood-17-pdf, tran-19-pdfphantom}, which use format-level transformation to exploit the discrepancy between displayed and extracted text.
More specifically, \citet{markwood-17-pdf} and \citet{tran-19-pdfphantom}, similar to our work, target the paper-reviewer assignment task.
Their attack is evaluated against Latent Semantic Indexing \cite{deerwester-90-indexing}---that is not used in real-world systems like TPMS. 
Although \citet{tran-19-pdfphantom} recognize the shortcomings of format-level transformations, they do not explore text-level transformations or the interplay between the problem space and feature space of topic models. 

Complementary to our work, a further line of research focuses on the collusion of reviewers. These works have analyzed semi-automatic paper matching systems under the assumption that malicious researchers can manipulate the paper assignment by carefully adjusting their paper biddings.
\citet{jecmen-20-mitigating} propose a probabilistic matching to decrease the probability of a malicious reviewer to be assigned to a target submission, while \citet{wu-21-making} tries to limit the disproportional influence of malicious biddings.
