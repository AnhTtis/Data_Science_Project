
\pdfoutput=1
\documentclass[11pt]{article}

\usepackage[]{acl}

\usepackage{times}
\usepackage{latexsym}
\usepackage{graphicx}
\usepackage{arydshln}
\usepackage{subfigure}
\usepackage{makecell}
\usepackage{colortbl}
\usepackage{pifont}
\usepackage{amssymb}
\usepackage{algorithm}
\usepackage{algorithmic}
\usepackage{enumerate}

\usepackage[T1]{fontenc}
\usepackage{xcolor}
\usepackage[utf8]{inputenc}

\usepackage{microtype}

\usepackage{array}
\usepackage{pifont}
\usepackage{tabularx}
\usepackage{adjustbox}
\usepackage{multirow}
\usepackage{enumitem}
\usepackage{xspace}
\usepackage{tcolorbox}
\usepackage{xparse}
\usepackage{soul}
\usepackage{booktabs,amsfonts,dcolumn}
\usepackage{hyperref}
\usepackage{url}
\usepackage{amsmath,amsthm,amsfonts,amssymb,bm,stmaryrd}
\usepackage[noorphans,vskip=0.75ex,leftmargin=2ex]{quoting}
\usepackage{breqn}
\usepackage{makecell}

% \usepackage{fnpos}
% \usepackage{algorithm}


\definecolor{OliveGreen}{rgb}{0,0.4,0}


\usepackage{cleveref}

\crefname{section}{§}{§§}
\Crefname{section}{§}{§§}


% for note
\newcommand{\ccmark}{\ding{51}}%
\newcommand{\xxmark}{\ding{55}}%
\definecolor{color_m}{RGB}{72,117,170}
\definecolor{color_f}{RGB}{201,89,72}
\definecolor{color_c}{RGB}{230,230,230}
\definecolor{color_e}{RGB}{100,155,74}
\definecolor{ccon}{HTML}{fee9d4}
\definecolor{cood}{HTML}{d8f0d3}
\definecolor{cid}{HTML}{dae8f5}
\definecolor{gg}{HTML}{e2f0cb}


\newcommand{\CC}{\cellcolor{gray!15}}

\title{Edge-free but Structure-aware: Prototype-Guided \\ Knowledge Distillation from GNNs to MLPs}

% \author{
% Taiqiang Wu \\
% Tsinghua Shenzhen International School, Tsinghua University \\
% wtq20@mails.tsinghua.edu.cn
% }

\author{Taiqiang Wu$^{1,2}$, Zhe Zhao $^{2}$, Jiahao Wang$^{3}$, \\ \textbf{Xingyu Bai$^{1}$, Lei Wang$^{4}$, Ngai Wong$^{3}$, Yujiu Yang$^{1}$}\Thanks{ Corresponding author} \\
$^{1}$ Shenzhen International Graduate School, Tsinghua University \ $^{2}$ Tencent \\
$^{3}$ The University of Hong Kong \ $^{4}$ Ping An Technology (Shenzhen) \\
wtq20@mails.tsinghua.edu.cn, yang.yujiu@sz.tsinghua.edu.cn}

\begin{document}

\maketitle


\begin{abstract}
% While Graph Neural Networks~(GNNs) achieve success for graph machine learning, they are difficult to be applied in real applications due to the high neighborhood-fetching latency.
%neglect
% The high neighborhood-fetching latency in Graph Neural Networks~(GNNs) has led to increased attention to Knowledge Distillation~(KD) methods from GNNs to MLPs.
% and overlook the information from GNN teachers. 
Distilling high-accuracy Graph Neural Networks~(GNNs) to low-latency multilayer perceptrons~(MLPs) on graph tasks has become a hot research topic.
However, MLPs rely exclusively on the node features and fail to capture the graph structural information.
Previous methods address this issue by processing graph edges into extra inputs for MLPs, but such graph structures may be unavailable for various scenarios.
To this end, we propose a Prototype-Guided Knowledge Distillation~(PGKD) method, which does not require graph edges~(edge-free) yet learns structure-aware MLPs.
Specifically, we analyze the graph structural information in GNN teachers, and distill such information from GNNs to MLPs via prototypes in an edge-free setting.
Experimental results on popular graph benchmarks demonstrate the effectiveness and robustness of the proposed PGKD.
\end{abstract}

The advance of Pre-trained Language Models (PLMs) like GPT-3 \cite{brown2020language} and LLaMA \cite{DBLP:journals/corr/abs-2302-13971} has substantially improved the performance of deep neural networks across a variety of Natural Language Processing (NLP) tasks. Various language models, based on the Transformer \cite{vaswani2017attention} architecture,  have been proposed, leading to state-of-the-art (SOTA) performance on the fundamental discrimination tasks. These models are first trained with self-supervised training objectives (e.g., predicting masked tokens according to surrounding tokens) on massive unlabeled text data, then fine-tuned on annotated data to adapt to downstream tasks of interest.  However, annotated data is usually limited to a wide range of downstream tasks, which results in overfitting and a lack of generalization to unseen data.

One straightforward way to deal with this data scarcity problem is data augmentation , and incorporating generative models to perform data augmentation has been widely adopted recently . Despite its popularity, the generated text can easily deviate from the real data distribution without exploiting any of the signals passed back from the discrimination task. In previous studies, generative data augmentation and discrimination have been well studied as separate problems, but it is less clear how these two can be leveraged in one framework and how their performances can be improved simultaneously. \looseness=-1

Generative Adversarial Networks (GANs) \cite{https://doi.org/10.48550/arxiv.1406.2661} are good attempts to couple generative and discriminative models in an adversarial manner, where a two-player minimax game between learners is carefully crafted. GANs have achieved tremendous success in domains such as image generation , and related studies have also shown their effectiveness in semi-supervised learning. However,  in the text field, GANs are difficult to train, most training objectives work well for only one model, either the discriminator or the generator, so rarely both learners can be optimal at the same time. This essentially arises from the adversarial nature of GANs, that during the process, optimizing one learner can easily destroy the learning ability of the other, making GANs fail to converge.

Another limitation of simultaneously optimizing the generator and the discriminator comes from the discrete nature of text in NLP, as no gradient propagation can be done from discriminators to generators. One theoretically sound attempt is to use reinforcement learning (RL), but the sparsity and the high variance of the rewards in NLP make the training particularly unstable \cite{caccia2019language}. 

To address these shortcomings, we novelly introduce a self-consistent learning framework based on one generator and one discriminator: the generator and the discriminator are alternately trained by way of cooperation instead of competition, and the selected samples are used as the medium to pass the feedback signal from the discriminator. Specifically, in each round of training, the samples generated by the generator are synthetically labeled by the discriminator, and then only part of them would be selected based on dynamic thresholds and used for the training of the discriminator and the generator in the next round. Several benefits can be discovered from this cooperative training process. First, a closed-loop form of cooperation can be established so that we can get the optimal generator and discriminator at the same time. Second, this framework helps improve the generation quality while ensuring the domain specificity of generator, which in turn contributes to training. Third, a steady stream of diverse synthetic samples can be added to the training in each round and lead to continuous improvement of the performance of all learners. Finally, we can start the training with only domain-related corpus and obtain strong results, while these data can be easily sampled with little cost or supervision. Also, the performance on labeled datasets can be further boosted based on the strong baselines. As an example to demonstrate the effectiveness of our framework in the text field, we examine it on four downstream text generation benchmarks, including AFQMC, CHIP-STS, QQP, and MRPC. The experiments show that our method significantly improves over standalone state-of-the-art discriminative models on zero-shot and full-data settings.

Our contributions are summarized as follows,

$\bullet$ We propose a self-consistent learning framework in the text field that incorporates the generator and the discriminator, in which both achieve remarkable performance gains simultaneously.

$\bullet$ We propose a dynamic selection mechanism such that cooperation between the generator and the discriminator drives the convergence to reach their scoring consensus.

$\bullet$ Experimental results show that the generator in our framework can continuously adjust its generation samples based on the performance of downstream tasks, while the discriminator can outperform the strong baselines.


% 
\section{Related Work}\label{sec:related}

Grasping is a fundamental problem for robotic manipulation and has been extensively studied. Most work focuses on parallel-jaw grippers \cite{DBLP:conf/cvpr/FangWGL20,jiang2011efficient,DBLP:conf/iccv/MousavianEF19,DBLP:conf/icra/MuraliMEPF20,DBLP:conf/icra/SundermeyerMTF21}  due to their simplicity, low DoFs, and computational efficiency. However, parallel-jaw grippers are less efficient and less reliable for manipulating arbitrary-shaped objects. To achieve user-friendly interaction, multi-finger robotic hands and dexterous grasping remain a hot research topic in the field of robotic manipulation~\cite{rimon2019mechanics}. This research can be briefly divided into two categories: the traditional analytical sampling-based method and the data-driven method.

\textbf{Traditional analytical sampling-based methods}\cite{ciocarlie2007dexterous,DBLP:conf/icra/GoldfederALP07,DBLP:conf/iros/HangSK14,DBLP:conf/icra/MillerKCA03,DBLP:conf/icra/PelossofMAJ04} sampled various grasp candidates and evaluated them based on certain metrics considering the physical properties of objects such as wrench space~\cite{DBLP:conf/icra/BorstFH04}. In general, both the object model and environment are assumed to be known in advance~\cite{DBLP:journals/ram/MillerA04}. Eigengrasp~\cite{ciocarlie2007dexterous} reduced the dimensions of grasp search space by performing principal component analysis (PCA) on grasping pose and configuration data. Although the reduction increases the efficiency of generating grasps, the search space of the random sampling process for grasps is still very huge. As a result, these sampling-based methods are less efficient in practical use.

\textbf{Data-driven methods} fall into one of two primary types.
The one is an extension of the traditional sampling-based method~\cite{DBLP:conf/iros/VarleyWWA15,DBLP:conf/icra/BorstFH04}. Instead of computing physical metrics, this method directly estimates grasp quality metrics from trained deep models. The grasp success rate can be greatly improved since traditional metrics cannot be computed accurately from an incomplete view of a novel object without any contact feedback. However, they are still dependent on known object models and exhibit the problem of huge sampling and search space.
%
The other data-driven method is performed in an end-to-end manner~\cite{DBLP:conf/iros/HangSK14,DBLP:conf/rss/LiuP0GM20,DBLP:journals/corr/abs-1908-04293,DBLP:conf/iros/LiuP0GM19,DBLP:conf/icra/KapplerBS15,DBLP:conf/iros/VarleyWWA15,mahler2017dex}. Specifically, this method takes the image or point cloud data of a grasped object as input and outputs a high-quality grasp. These approaches are able to effectively generate grasps and are robust to unknown objects. However, many can only handle a single object. Grasping may often fail due to the potential collision between the gripper and the environment.
%
Some recent work~\cite{DBLP:conf/icra/LiWL0LZ22,DBLP:conf/icra/LundellCLVWRMK21,DBLP:journals/corr/abs-2103-04783} predicts
collision-free \mbox{6-DoF} grasping in clutter using multi-finger grippers. They only classify the grasp types and do not take into account of the properties of multi-finger grasps. Our approach considers the gripper's physical structure and does not rely on the grasp types. Using a novel grasping representation and an end-to-end deep neural network based on contacts, our approach significantly reduces the search space for grasping and can generate reliable grasp poses.

\section{Preliminaries}

% \subsection{Task Formulation}
\paragraph{Notations.} Let $\mathcal{G}=(\mathcal{V},\mathcal{E})$ denote a graph, where $\mathcal{V}$ stands for all $N$ nodes with features $\mathbf{X} \in \mathbb{R}^{N \times D}$ and $\mathcal{E}$ stands for all edges.
We represent edges with an adjacency matrix $\mathbf{A}$, and $A_{u,v}=1$ if edge $(u,v) \in \mathcal{E}$ or be 0 otherwise.
For node classification task, the target is $\mathbf{Y} \in \mathbb{R}^{N \times K}$, where row $y_{v} \in R^{K}$ denotes the $K$-dim one-hot label for node $v$.
We adopt superscript $^L$ for labeled nodes~(i.e. $\mathcal{V}^{L}$, $\mathbf{X}^{L}$, and $\mathbf{Y}^{L}$) and superscript $^U$ for the rest unlabeled nodes~(i.e. $\mathcal{V}^{U}$, $\mathbf{X}^{U}$, and $\mathbf{Y}^{U}$).

\paragraph{Graph Neural Network.} Most GNNs follow the message-passing framework, where the representation $\mathbf{h}_v$ of node $v$ is updated by aggregating messages from its neighbors $\mathcal{N}_v$.
For the $l$-th layer, $\mathbf{h}^{l}_v$ is obtained from the previous layer's representations of its neighbors as follows:
\begin{equation}
    % \mathbf{h}^{l}_v = \text{UP}(\text{AG}(\{ h^{l-1}_{u}: u \in  \mathcal{N}_v \}), \mathbf{h}^{l-1}_v) ,
    h^{(l)}_{N(v)}=\text{AGGR}(\{ h^{l-1}_{u}: u \in  \mathcal{N}_v \})
\end{equation}
\begin{equation}
    h^{(l)}_{v} = \text{UPDATE}(h^{(l)}_{N(v)}, h^{l-1}_{v}),
\end{equation}
where AGGR and UPDATE denote the aggregate and update operations, respectively.

\paragraph{Transductive vs Inductive.} 
\label{setting_intro}
There are two setting for graph learning: transductive and inductive.
For transductive setting, models can utilize all node features and graph edges.
For inductive setting, we split the unlabeled data into disjoint inductive subset and observed subset~(i.e. $\mathcal{V}^U=\mathcal{V}^U_{obs} \cup \mathcal{V}^U_{ind}$ and $\mathcal{V}^U_{obs} \cap \mathcal{V}^U_{ind} = \emptyset$).
The edges between $\mathcal{V}^U_{obs}$ and $\mathcal{V}^U_{ind}$ are preserved~(cf. Table \ref{tab:trans_ind}).

\begin{table}[t]

\centering
%\tableindent 
\renewcommand\arraystretch{1.4}


\resizebox{\columnwidth}{!}
{%
\begin{tabular}{lllll}
\hline
\multicolumn{2}{c}{\textbf{Model Setting}}                & \textbf{Train} & \textbf{Test} & \textbf{KD} \\ \hline
\multicolumn{1}{l}{\multirow{2}{*}{GNN}} & \textit{tran} & $(\mathbf{X}, \mathcal{G}, \mathbf{Y}^{L})$     & $(\mathbf{X}^{U}, \mathcal{G}, \mathbf{Y}^{U})$    & $\mathbf{H}$  \\ 
\multicolumn{1}{l}{}                     & \textit{ind}   & $(\mathbf{X}^{L}, \mathcal{G}_{obs}, \mathbf{X}^{U}_{obs}, \mathbf{Y}^{L})$     & $(\mathbf{X}^{U}_{ind}, \mathbf{Y}^{U}_{ind})$    & $\mathbf{H}^{L} \cup \mathbf{H}^{U}_{obs}$ \\ \hline
\multicolumn{2}{c}{MLP}                          & $(\mathbf{X}^{L}, \mathbf{Y}^{L})$    &  $(\mathbf{X}^{U}, \mathbf{Y}^{U})$   & -  \\ \hline
\end{tabular}
}
\caption{
The inputs for GNNs and MLPs in different settings: 
transductive (\textit{tran}) and inductive (\textit{ind}).
KD denotes the employed features for knowledge distillation.
$\mathbf{H}$ denotes the graph nodes representations from GNN teachers.
}
\label{tab:trans_ind}

\end{table}


\begin{figure*}[t]
    \centering
    \includegraphics[width=\linewidth]{fig/ModelStructure.pdf}
    \caption{Overall architecture of \name. The left part represents different modality-specific encoders to extract latent features and the multimodal fusion module to integrate multimodal representations. The right part represents the contextual relational model decoders to get the similarity score and the decision fusion module to make the final prediction on all modalities.}
    \label{fig:model}
\end{figure*}

\section{Methodology}

Formally, a knowledge graph is defined as $\mathcal{G} = \langle \mathcal{E}, \mathcal{R}, \mathcal{T} \rangle$, where $\mathcal{E}$ and $\mathcal{R}$ indicate sets of entities and relations, respectively. 
$\mathcal{T} = \{(h, r, t) | h, t \in \mathcal{E}, r \in \mathcal{R}\}$ represents relational triples of the KG.
In multimodal KGs, each entity in KGs is represented by multiple features from different modalities.
Here, we define the set of modalities $\mathcal{K} = \{s, v, t, m\}$ where $s, v, t, m$ denote structural, visual, textual and multimodal modality, respectively.
Due to the complexity of real-world knowledge, it is almost impossible to take all the triples into account.
Therefore, given a well-formulated KG, the \emph{Link Prediction} task aims at predicting missing links between entities.
Specifically, link prediction models expect to learn a score function of relational triples to estimate the likelihood of a triple, which is always formulated as $\psi : \mathcal{E} \times \mathcal{R} \times \mathcal{E} \to \mathbb{R}$.


\subsection{Overall Architecture}

In order to fully exploit the complicated interaction between different modalities, we propose a two-stage fusion model instead of simply considering the multimodal information separately in a unified vector space.
As shown in Figure~\ref{fig:model}, \name consists of four key components:
\begin{itemize}[leftmargin=*]
	\item[1] The Modality-Specific Encoders are used for extracting structural, visual and textual features as the input of multimodal fusion stage.
	\item[2] The Multimodal Fusion Module, which is the first fusion stage, effectively models bilinear interactions between different modalities based on \textit{Tucker} decomposition and contrastive learning.
	\item[3] The Contextual Relational Model calculates the similarity of contextual entity representations to formulate triple scores as modality-specific predictions for decision fusion stage.
	\item[4]  The Decision Fusion Module, which is the second fusion stage, takes all the similarity scores from structural, visual, textual and multimodal models into account to make the final prediction.
\end{itemize}

\subsection{Modality-Specific Encoders}
In this subsection, we first introduce the pre-trained encoders used for different modalities.
These encoders are not fine-tuned during training and we treat them as fixed feature extractors to obtain the modality-specific entity representations.
Note that \name is a general framework and it is straightforward to replace them with other up-to-date encoders or add ones for new modalities into \name.

\subsubsection{Structural Encoder}

From the most basic view, the structural information of KG, we employ a Graph Attention Network (GAT)\footnote{https://github.com/Diego999/pyGAT}~\cite{DBLP:conf/iclr/VelickovicCCRLB18} with TransE loss.

Specifically, our GAT encoder takes L1 distance of neighbor aggregated representations as energy function of triples, which is $E(h, r, t) = ||\mathbf{h}+\mathbf{r}-\mathbf{t}||$.
In the training process, we minimize the following Hinge loss~\eqref{eq-gat-loss}:
\begin{equation}\label{eq-gat-loss}
    \begin{split}
        \mathcal{L}_{GAT} = & \sum_{(h,r,t) \in \mathcal{T}}\sum_{(h', r, t') \in \mathcal{T'}} \mathrm{max} \{0,  \\
        &\gamma + E(h,r,t) - E(h',r,t')\}
    \end{split}
\end{equation}
where $\gamma$ is margin hyper-parameter and $\mathcal{T'}$ denotes set of negative triples derived from $\mathcal{T}$. 
$\mathcal{T'}$ is created by randomly replacing head or tail entities of triples in $\mathcal{T}$, which is~\eqref{eq-gat-neg}:
\begin{equation}\label{eq-gat-neg}
    \mathcal{T'} = \{(h',r,t)|h' \in \mathcal{E} \backslash h\} \cup \{(h,r,t')|t' \in \mathcal{E} \backslash t\}
\end{equation}

\subsubsection{Visual Encoder} 
Visual features are greatly expressive while providing different views of knowledge from traditional KGs. 
To effectively extract visual features, we utilize VGG16\footnote{https://github.com/machrisaa/tensorflow-vgg} pre-trained on \textit{ImageNet}\footnote{https://image-net.org/} to get image embeddings of corresponding entities following~\cite{DBLP:conf/esws/LiuLGNOR19}.
Specifically, we take outputs of the last hidden layer before softmax operation as visual features, which are 4096-dimensional vectors.

\subsubsection{Textual Encoder} 
Entity descriptions contain much richer but more complex knowledge than pure KGs.
To fully extract the complex knowledge, we employ BERT~\cite{DBLP:conf/naacl/DevlinCLT19} as the textual encoder, which is very expressive to get description embeddings of corresponding entities.
The textual features are 768-dimensional vectors, i.e., pooled outputs of pre-trained BERT-Base model\footnote{https://github.com/huggingface/transformers}.

\subsection{Multimodal Fusion}
The multimodal fusion stage aims to effectively get multimodal representations, which fully capture the complex interactions between different modalities.
Many existing multimodal fusion methods have achieved promising results in many tasks like VQA (Visual Question Answering).
However, most of them aim at finding the commonality to get more precise representations by modality projecting~\cite{DBLP:conf/nips/FromeCSBDRM13,DBLP:conf/aaai/CollellZM17} or cross-modal attention~\cite{DBLP:conf/aaai/PerezSVDC18}.
These types of methods will suffer from the loss of unique information in different modalities and can not achieve sufficient interaction between modalities.
To this end, we propose to employ the bilinear models, which have a strong ability to realize full parameters interaction as the cornerstone to perform the fusion of multimodal information.
Specifically, we extend the \textit{Tucker} decomposition, which decomposes the tensor into a core tensor transformed by a matrix along with each mode to 4-mode factors as expressed in Equation~\eqref{eq-tucker}:
\begin{equation}\label{eq-tucker}
    \mathcal{P} = (((\mathcal{P}_c \times \mathbf{M}_s) \times \mathbf{M}_v) \times \mathbf{M}_t) \times \mathbf{M}_d
\end{equation}
where $\mathbf{M}_s \in \mathbb{R}^{d_s \times t_s}$, $\mathbf{M}_v \in \mathbb{R}^{d_v \times t_v}$, $\mathbf{M}_t \in \mathbb{R}^{d_t \times t_t}$,  $\mathbf{M}_d \in \mathbb{R}^{\mathcal{D} \times t_d}$ denotes transformation matrix and $\mathcal{P}_c \in \mathbb{R}^{t_s \times t_v \times t_t \times t_d}$ denotes a smaller core tensor.

In such a situation, entity embeddings are first projected into a low-dimensional space and then fused with the core tensor $\mathcal{P}_c$.
Following~\cite{DBLP:conf/iccv/Ben-younesCCT17}, we further reduce the computation complexity by decomposing the core tensor $\mathcal{P}_c$ to merge representations of all modalities into a unified space with element-wise product.
The detailed calculation process is expressed as Equation~\eqref{eq-fusion}:
\begin{equation}\label{eq-fusion}
    \mathbf{e}_m = \tilde{\mathbf{e}}_s^\mathsf{T} \mathbf{M}_d^s * \tilde{\mathbf{e}}_v^\mathsf{T} \mathbf{M}_d^v * \tilde{\mathbf{e}}_t^\mathsf{T} \mathbf{M}_d^t
\end{equation}
where $\tilde{\mathbf{e}}_k = \mathrm{ReLU}(\mathbf{e}_k\mathbf{M}_k) \in \mathbb{R}^{t_k}$ denotes latent representations and $\mathbf{e}_k \in \mathbb{R}^{d_k}$ is the original embedding representations and $\mathbf{M}_d^k \in \mathbb{R}^{t_k \times t_d}$ is decomposed transformation matrix for each modality $k \in \{s, v, t\}$.

However, the multimodal bilinear fusion has no bound limitation while the gradient produced by the final prediction result can only implicitly guide parameter learning.
To alleviate this problem, we add constraints to limit the correlation between different modality representations of the same entity to be stronger.
Therefore, we further leverage contrastive learning~\cite{DBLP:conf/icml/ChenK0H20,DBLP:conf/nips/LiSGJXH21,DBLP:conf/cvpr/Yuan0K0WMKF21} between different entities and modalities as an additional learning objective for regularization.
In the settings of contrastive learning, we take the pairs of representations of the same entity of different modalities as positive samples and the pairs of representations of different entities as negative samples.
As shown in Figure~\ref{fig:cl}, we aim at limiting the distance of negative samples to be larger than positive samples to enhance multimodal fusion, which is:
\begin{equation}
    d(f(x), f(x^+)) << d(f(x), f(x^-))
\end{equation}
where $d(\cdot, \cdot)$ denotes the distance measure and $f(\cdot)$ denotes the embedding function. The superscript $+, -$ represent the positive and negative samples, respectively.

\begin{figure}
    \centering
    \includegraphics[width=\linewidth]{fig/ContrastiveLearning.pdf}
    \caption{Example of multimodal contrastive learning. The distance between the representations of the same entity in different modalities is minimized, while the distance between the representations of different entities is maximized.}
    \label{fig:cl}
\end{figure}

Specifically, we randomly sample $N$ entities from the entity set as a minibatch and define contrastive learning loss upon it.
The positive pairs are naturally obtained with the same entities while the negative pairs are constructed by negative sharing~\cite{DBLP:conf/kdd/ChenSSH17} of all other entities.
We take the latent representations $\tilde{\mathbf{e}}_k = \mathrm{ReLU}(\mathbf{e}_k\mathbf{M}_k) \in \mathbb{R}^{t_k}$ and leverage cosine similarity $d(u, v) = - \mathbf{u}^\mathsf{T}\mathbf{v}/||\mathbf{u}||\mathbf{v}||$ as distance measure.
Then we have the following contrastive loss function for each entity $i$:
\begin{equation}\label{eq-cl}
    \mathcal{L}_{CLi} = \frac{1}{3N} \sum_{p,q \in \mathcal{M}} \sum_{j=1}^N  d(e_i^{p}, e_i^{q}) - d(e_i^{p}, e_j^{q}) + 2
\end{equation}
where $\mathcal{M} = \{(s, v), (s, t), (v, t)\}$ is set of modality pairs.

\subsection{Contextual Relational Model}
After obtaining representations of each modality and multimodal, we then design a contextual relational model, which takes relations in triples as contextual information for scoring, to get the predictions.
Note that this relational model can be easily replaced by any scoring function like TransE.

Due to the variety and complexity of relations in KGs, we argue that improving the degree of parameter interaction~\cite{DBLP:conf/aaai/VashishthSNAT20} is crucial for better modeling the relational triples.
The degree of parameter interaction means the calculation ratio of each parameter to all other parameters. 
For example, dot product could achieve $1/d$ degree while cross product could achieve $(d-1)/d$ degree.
Based on this assumption, we propose to use bilinear outer product between entity and relation embeddings to incorporate contextual information into entity representations.
Instead of taking relations as input as in previous studies, our contextual relational model utilizes relations to provide context in the transformation matrix of entity embeddings.
Then, entity embeddings are projected using the contextual transformation matrix to get \emph{contextual embeddings}, which are used for calculating similarity with all candidate entities.
The learning objective is to minimize the binary cross-entropy loss.
For each modality $k \in \mathcal{K}$, the computation details are shown as Equation~\eqref{eq-crm} to Equation~\eqref{eq-loss}:
\begin{gather}
    \hat{\mathbf{e}}_k = \mathbf{e}_k^\mathsf{T}\mathbf{W}_k^r  + \mathbf{b} = \mathbf{e}_k^\mathsf{T}\mathbf{W}_k\mathbf{r} + \mathbf{b}_k \label{eq-crm} \\
    \mathbf{y}_k = \sigma(\mathrm{cosine}(\mathbf{e}_k, \hat{\mathbf{e}}_k)) = \sigma (\frac{\mathbf{e}_k \cdot \hat{\mathbf{e}}_k}   
    {|\mathbf{e}_k| |\hat{\mathbf{e}}_k|}) \label{eq-sim} \\
    \mathcal{L}_k = -\frac{1}{N} \sum_{i=1}^N (t_i \cdot \mathrm{log}(y_{i,k})+(1-t_i) \cdot \mathrm{log}(1-y_{i,k})) \label{eq-loss}
\end{gather}
where $\mathbf{e}_k$ and $\hat{\mathbf{e}}_k$ are original and contextual entity embeddings respectively;
$\mathbf{W}_k^r = \mathbf{W}_k \mathbf{r}$ denotes contextual transformation matrix which is obtained by matrix multiplication of weight matrix $\mathbf{W}_k$ and relation vectors $\mathbf{r}$ while $\mathbf{b}_k$ is a bias vector;
$\sigma$ is sigmoid function and $\mathbf{y}_k = [y_{1,k},y_{2,k},...,y_{N,k}]$ is final prediction of modality $k$.

\subsection{Decision Fusion}
Existing multimodal approaches mainly focus on projecting different modality representations into a unified space and predicting with commonality between modalities, which will fail to preserve the modality-specific knowledge.
We alleviate this problem in the decision fusion stage by joint learning and combining predictions of different modalities to further leverage the complementarity.

Under the multimodal settings, we assign different contextual relational models for each modality and utilize their own results for training in different views.
Recall the contrastive learning loss in Equation~\eqref{eq-cl}, the overall training objective is to minimize the joint loss shown in Equation~\eqref{eq-mmloss}:
\begin{equation}\label{eq-mmloss}
    \mathcal{L}_{Joint} = \gamma_s \mathcal{L}_s + \gamma_v \mathcal{L}_v + \gamma_t \mathcal{L}_t + \gamma_m \mathcal{L}_{m} + \mathcal{L}_{CL}
\end{equation}
where $\mathcal{L}_k$ denotes binary cross entropy loss for modality $k$ as Equation~\eqref{eq-loss} and $\gamma_k$ is a learned weight parameter.

\begin{algorithm}[t]
\caption{Optimization Algorithm.}\label{alg:optim}
\begin{algorithmic}[1]
\STATE \textbf{Input:} Multimodal Knowledge Graph $\mathcal{G}$
\STATE \textbf{Output:} Trained Model $\mathcal{M}$
\STATE Pre-train structural encoder GAT on $\mathcal{G}$ with the loss in Equation(1)
\STATE Obtain pre-trained visual encoder VGG16 and textual encoder BERT-base
\STATE Initialize the entity embeddings $\mathbf{E}_s, \mathbf{E}_v, \mathbf{E}_t$ in $\mathcal{M}$ with the outputs of pre-trained encoders
\WHILE{not converge}
    \STATE Sample a batch of entities from $\mathcal{G}$
    \FOR{Entity $e$ in batch}
    \STATE Obtain the structural, visual, textual embeddings $\mathbf{e}_s, \mathbf{e}_v, \mathbf{e}_t$ of entity $e$
    \STATE Compute the multimodal fused embeddings $\mathbf{e}_m$ of entity $e$ with Equation (4)
    \STATE Compute the contrastive learning loss $\mathcal{L}_{CL}$ with Equation (6)
    \STATE Compute the loss $\mathcal{L}_s, \mathcal{L}_v, \mathcal{L}_t, \mathcal{L}_m$ with modality-specific scorers via Equation (7) - Equation (9)
    \STATE Compute the joint loss $\mathcal{L}_{Joint}$ with the above losses $\mathcal{L}_s, \mathcal{L}_v, \mathcal{L}_t, \mathcal{L}_m, \mathcal{L}_{CL}$ via Equation (10)
    \STATE Update model parameters of $\mathcal{M}$ by minimizing $\mathcal{L}_{Joint}$
    \ENDFOR
\ENDWHILE
\RETURN $\mathcal{M}$
\end{algorithmic}
\end{algorithm}

To better illustrate the whole training process of \name, we describe it via the pseudo-code of the optimization algorithm.
As shown in Algorithm~\ref{alg:optim}, we first obtain the pre-trained encoders of structural, visual and textual and utilize them for entity embeddings (line 3-5).
Since the pre-trained models are much larger and more complex than \name, they are not fine-tuned and their outputs are directly used as inputs of \name.
The multimodal embeddings are obtained by multimodal fusion while contrastive learning is applied to further enhance the fusion stage (line 9-11).
During training, each modality delivers its own prediction and loss via the modality-specific scorers (line 12), and then the joint prediction and loss are computed based on all modalities including multimodal ones (line 14).

For inference, we propose to jointly consider the predictions of each modality as well as multimodal ones.
Specifically, the overall predictions are shown in Equation~\eqref{eq-df}:
\begin{equation}\label{eq-df}
    \mathbf{y}_{Joint} = \frac{\gamma_s \mathbf{y}_s + \gamma_v \mathbf{y}_v + \gamma_t \mathbf{y}_t + \gamma_m \mathbf{y}_m} {\gamma_s + \gamma_v + \gamma_t + \gamma_m}
\end{equation}
where $\gamma_k$ denotes weight for modality $k$ as same as Equation~\eqref{eq-mmloss} while the values in $\mathbf{y}$ are in [0, 1].



% \clearpage
\section{Experimental Results}
In this section, we validate the effectiveness of our proposal. We first introduce datasets, metrics and implementation details involved in our evaluation. Then, we compare \netname{} with state-of-the-art methods, conduct an ablation study on our model and, finally, discuss its limitations.


\begin{table*}[htbp] \scriptsize
	\renewcommand\tabcolsep{2.3pt} 
	\centering
	\scalebox{0.85}{
	\begin{tabular}{@{}ccccccccccccccccc@{}}
		\toprule
		 Dataset & Scale & Metrics & GF~\cite{he2010guided} & SD~\cite{ham2017robust}  & GSRPT~\cite{lutio2019guided} & MSG~\cite{hui2016depth} & DKN~\cite{kim2021deformable} & FDKN~\cite{kim2021deformable} & PMBANet~\cite{ye2020pmbanet} & FDSR~\cite{he2021towards} & JIIF~\cite{tang2021joint} & DCTNet~\cite{zhao2022discrete} & LGR~\cite{de2022learning} & DADA~\cite{metzger2022guided} & DSR-EI & DSR-EI$^+$ \\ \midrule
		\multirow{6}{*}{\rotatebox[origin=l]{90}{\scriptsize \textbf{Middlebury}}} & \multirow{2}{*}{$4\times$} 
		& MSE & 33.3 & 24.9 & 39.8 & 4.13 & 4.29 & 3.60 & 4.72 & 7.72 & 2.70 & 5.00 & 3.04 & \bronze{2.58} & \gold{2.46} & \silver{2.56} \\
		& & MAE & 1.27 & 0.46 & 0.79 & 0.22 & 0.18 & 0.16 & 0.25 & 0.35 & \bronze{0.11} & 0.24 & 0.13 & \bronze{0.11} & \silver{0.08} & \gold{0.07} \\ \cline{2-17}
		& \multirow{2}{*}{$8\times$} 
		& MSE & 40.5 & 82.5 & 32.7 & 10.5 & 11.2 & 10.4 & 9.48 & 23.2 & 8.01 & 15.1 & 7.26 & \silver{5.68} & \bronze{6.20} & \gold{5.13} \\
		& & MAE & 1.49 & 0.86 & 0.82 & 0.43 & 0.38 & 0.37 & 0.38 & 0.69 & 0.27 & 0.57 & 0.24 & \bronze{0.20} & \gold{0.18} & \gold{0.18} \\ \cline{2-17}
		& \multirow{2}{*}{$16\times$} 
		& MSE & 67.4 & 511 & 41.5 & 34.2 & 47.6 & 38.5 & 30.6 & 55.4 & 37.5 & 52.3 & 24.7 & \silver{16.3} & \gold{15.8} & \bronze{16.6}  \\
		& & MAE & 2.21 & 1.73 & 1.24 & 1.06 & 1.42 & 1.18 & 0.89 & 1.51 & 0.98 & 1.50 & 0.67 & \bronze{0.48} & \silver{0.47} & \gold{0.40} \\ \hline\hline
	    % middlebury end
		\multirow{6}{*}{\rotatebox[origin=l]{90}{\scriptsize \textbf{NYUv2}}} & \multirow{2}{*}{$4\times$}
		& MSE & 114 & 36.0 & 112 & 6.85 & 11.4 & 9.07 & 10.8 & 10.1 & \bronze{3.28} & 3.63 & 6.45 & 4.83 & \silver{2.82} & \gold{2.75}\\
		& & MAE & 3.91 & 1.31 & 3.61 & 0.81 & 1.03 & 0.85 & 0.93 & 0.94 & \bronze{0.52} & 0.68 & 0.73 & 0.64 & \silver{0.49} & \gold{0.47}\\ \cline{2-17}
		& \multirow{2}{*}{$8\times$} 
		& MSE & 142 & 105 & 122 & 24.1 & 29.8 & 29.9 & 17.2 & 19.5 & \bronze{15.2} & 20.9 & 19.6 & 16.6 & \gold{11.8} & \gold{11.8}\\
		& & MAE & 4.47 & 2.57 & 3.86 & 1.66 & 1.82 & 1.80 & 1.38 & 1.38 & \bronze{1.29} & 1.79 & 1.42 & 1.30 & \silver{1.12} & \gold{1.09}\\ \cline{2-17}
		& \multirow{2}{*}{$16\times$} 
		& MSE & 249 & 533 & 219 & 84.5 & 115 & 113 & 84.9 & 86.4 & 59.9 & 77.0 & 67.5 & \bronze{59.0} & \silver{47.8} & \gold{47.1} \\
		& & MAE & 6.34 & 5.07 & 5.40 & 3.35 & 4.01 & 3.95 & 3.26 & 3.35 & 2.81 & 3.61 & 2.90 & \bronze{2.64} & \silver{2.48} & \gold{2.40}\\ \hline\hline
		% NYU end
		\multirow{6}{*}{\rotatebox[origin=l]{90}{\scriptsize \textbf{DIML}}} & \multirow{2}{*}{$4\times$}
		& MSE & 25.6 & 10.5 & 20.7 & 1.73 & 3.47 & 2.20 & 3.05 & 2.75 & \bronze{1.19} & 2.09 & 1.68 & 1.33 & \silver{0.70} & \gold{0.65} \\
		& & MAE & 1.45 & 0.40 & 1.15 & 0.22 & 0.33 & 0.23 & 0.31 & 0.29 & \bronze{0.16} & 0.31 & 0.20 & 0.17 & \silver{0.13} & \gold{0.12} \\ \cline{2-17}
		& \multirow{2}{*}{$8\times$} 
		& MSE & 34.1 & 44.9 & 23.0 & 4.13 & 5.47 & 5.95 & 5.87 & 8.40 & 3.65 & 7.08 & 3.51 & \bronze{2.93} & \silver{2.12} & \gold{2.09} \\
		& & MAE & 1.77 & 0.83 & 1.26 & 0.40 & 0.45 & 0.47 & 0.47 & 0.66 & 0.32 & 0.65 & 0.31 & \bronze{0.28} & \gold{0.22} & \gold{0.22} \\ \cline{2-17}
		& \multirow{2}{*}{$16\times$} 
		& MSE & 66.3 & 41.1 & 39.3 & 13.0 & 19.3 & 20.8 & 13.8 & 32.9 & 11.7 & 23.4 & 9.45 & \bronze{7.61} & \gold{6.29} & \silver{6.31} \\
		& & MAE & 2.74 & 1.91 & 1.78 & 0.93 & 1.20 & 1.24 & 0.87 & 1.66 & 0.81 & 1.75 & 0.68 & \bronze{0.59} & \silver{0.52} & \gold{0.50} \\
		% DIML end
    \bottomrule
	\end{tabular}}
    \vspace{-0.3cm}
	\caption{\textbf{Results on Middlebury, NYUv2 and DIML datasets.} The lower the MSE and MAE, the better.}
	\label{sota_comparison_mid_nyu_diml}
\end{table*}



\begin{table*}[t] \footnotesize
	\renewcommand\tabcolsep{1.5pt} 
	\centering
	\scalebox{0.85}{
	\begin{tabular}{@{}ccccccccccccccccc@{}}
		\toprule
		 Scale & SDF~\cite{li2016deep} & SVLRM~\cite{pan2019spatially} & DJF~\cite{li2016deep} & DJFR~\cite{li2019joint} & PAC~\cite{su2019pixel} & CUNet~\cite{deng2020deep} & FDKN~\cite{kim2021deformable} & DKN~\cite{kim2021deformable} & FDSR~\cite{he2021towards} & DCTNet~\cite{zhao2022discrete} & RSAG~\cite{yuan2023recurrent} & DSR-EI & DSR-EI$^+$ \\ \midrule
		$4\times$ & 2.00 & 3.39 & 3.41 & 3.35 & 1.25 & 1.18 & 1.18 & 1.30 & 1.16 & \bronze{1.07} & 1.14 & \gold{0.91} & \gold{0.91} \\
		$8\times$ & 3.23 & 5.59 & 5.57 & 5.57 & 1.98 & 1.95 & 1.91 & 1.96 & 1.82 & 1.78 & \bronze{1.75} & \gold{1.37} & \silver{1.38} \\
		$16\times$ & 5.16 & 8.28 & 8.15 & 7.99 & 3.49 & 3.45 & 3.41 & 3.42 & 3.06 & 3.18 & \bronze{2.96} & \gold{2.10} & \gold{2.10}  \\
    \bottomrule
	\end{tabular}}
	\vspace{-0.3cm}
	\caption{\textbf{Results on the RGBDD dataset.} We report RMSE, the lower the better.}
	\label{sota_comparison_rgbdd}
\end{table*}



\subsection{Datasets and Metrics}
We evaluate \netname{} on four datasets, compared with existing methods when super-solving depth maps by three different upsampling factors: $4\times,\ 8\times$, and $16\times$. 

\textbf{Middlebury}\cite{scharstein2003high,scharstein2007learning,hirschmuller2007evaluation,scharstein2014high}. We train all learning-based methods using 50 RGB-D images with ground truth from Middlebury 2005, 2006 and 2014 datasets. As in~\cite{de2022learning}, we retain 5 for validation and 5 for testing. 

\textbf{NYUv2}\cite{silberman2012indoor}. It contains 1449 RGB-D images in total. Following \cite{de2022learning}, we randomly split it into 849 RGB-D images for the training set, 300 for the validation set and 300 for the test set. Compared to \cite{ye2020pmbanet,liu2022pdr}, it comes with a validation set to make the comparison fairer.

\textbf{DIML}\cite{kim2016structure,kim2017deep,kim2018deep,cho2021deep} consists of 2 million color images and corresponding depth maps from indoor and outdoor scenes. We adopt the same strategy outlined in \cite{de2022learning}, i.e., considering only the indoor data subset, and use 1440 for training, 169 for validation, and 503 for testing.

\textbf{RGBDD}\cite{he2021towards} is a new real-world dataset for GDSR, which consists of 4811 image pairs. For evaluation, we follow the protocol described in \cite{he2021towards}, using 2215 images (1586 portraits, 380 plants, 249 models) as the training set and 405 images (297 portraits, 68 plants, 40 models) as the test set. 

\textbf{Metrics.} Following \cite{de2022learning}, we compute mean square error (MSE / $cm^2$) and mean absolute error (MAE / $cm$) as metrics on Middlebury, NYUv2 and DIML. For RGBDD, we use root mean square error (RMSE / $cm$) as in \cite{he2021towards}. 

\subsection{Implementation Details}
During training, the HR depth maps and the color images are randomly cropped into $256\times 256$ patches. LR depth patches are generated by bicubic interpolation at $64\times 64$, $32\times 32$, $16\times 16$ resolution for $4\times$, $8\times$ and $16\times$ factors, respectively. We randomly extract about 75K, 168K, 223K and 232K patches from Middlebury, NYUv2, DIML and RGBDD for training. Before being fed to the network, depth maps and images are normalized in the [0, 1] range.

We use Pytorch \cite{paszke2019pytorch} to implement and train \netname{}, on a single Nvidia RTX 3090 GPU. The batch size is set to 4, using Adam as the optimizer. The learning rate is initialized to $1\times 10^{-4}$, then performing a 5-epoch warm-up and cosine annealing. We use random rotation, horizontal/vertical flipping as data augmentation. According to the size of the four datasets, we train our network for 1505, 198, 155 and 109 epochs on Middlebury, NYUv2, DIML and RGBDD, respectively. 
When evaluating results on a specific dataset, we do not perform any pre-training on the others. Following \cite{de2022learning}, testing is performed by processing $256\times256$ patches at a time on Middlebury, NYUv2 and DIML for fairness, while full-resolution images are processed for RGBDD.

\begin{figure*}[t] 
	\centering
	\renewcommand\tabcolsep{1.5pt} 
	\begin{tabular}{cccccccccccc}
	\vspace{-0.1cm}
    \rotatebox[origin=l]{90}{\scriptsize \quad \textbf{Middlebury}} & \includegraphics[height=0.6in]{./figs/sota_comp_middlebury/389/Middlebury_389_img.pdf}
        \hspace{-1.8mm} & \includegraphics[height=0.6in]{./figs/sota_comp_middlebury/389/Middlebury_389_source.pdf}
	\hspace{-1.8mm} &  \includegraphics[height=0.6in]{./figs/sota_comp_middlebury/389/Middlebury_389_GT.pdf}
	\hspace{-1.8mm} & \includegraphics[height=0.6in]{./figs/sota_comp_middlebury/389/Middlebury_389_PMBA.pdf}
	\hspace{-1.8mm} & \includegraphics[height=0.6in]{./figs/sota_comp_middlebury/389/Middlebury_389_FDSR.pdf}
	\hspace{-1.8mm} & \includegraphics[height=0.6in]{./figs/sota_comp_middlebury/389/Middlebury_389_JIIF.pdf}
	\hspace{-1.8mm} & \includegraphics[height=0.6in]{./figs/sota_comp_middlebury/389/Middlebury_389_DCTnet.pdf}
	\hspace{-1.8mm} & \includegraphics[height=0.6in]{./figs/sota_comp_middlebury/389/Middlebury_389_LGR.pdf}
	\hspace{-1.8mm} & \includegraphics[height=0.6in]{./figs/sota_comp_middlebury/389/Middlebury_389_MSS.pdf}
        
        \hspace{-1.8mm} & \includegraphics[height=0.6in]{./figs/sota_comp_middlebury/389/Middlebury_389_ours.pdf}
    \\ \vspace{-0.1cm}
    
    \rotatebox[origin=l]{90}{\scriptsize \quad \textbf{NYUv2}} & \includegraphics[height=0.6in]{./figs/sota_comp_nyu/357/NYU_357_img.pdf}
	\hspace{-1.8mm} & \includegraphics[height=0.6in]{./figs/sota_comp_nyu/357/NYU_357_source.pdf}
	\hspace{-1.8mm} & \includegraphics[height=0.6in]{./figs/sota_comp_nyu/357/NYU_357_GT.pdf}
	\hspace{-1.8mm} & \includegraphics[height=0.6in]{./figs/sota_comp_nyu/357/NYU_357_PMBA.pdf}
	\hspace{-1.8mm} & \includegraphics[height=0.6in]{./figs/sota_comp_nyu/357/NYU_357_FDSR.pdf}
	\hspace{-1.8mm} & \includegraphics[height=0.6in]{./figs/sota_comp_nyu/357/NYU_357_JIIF.pdf}
	\hspace{-1.8mm} & \includegraphics[height=0.6in]{./figs/sota_comp_nyu/357/NYU_357_DCTnet.pdf}
	\hspace{-1.8mm} & \includegraphics[height=0.6in]{./figs/sota_comp_nyu/357/NYU_357_LGR.pdf}
	\hspace{-1.8mm} & \includegraphics[height=0.6in]{./figs/sota_comp_nyu/357/NYU_357_MSS.pdf}
 
	\hspace{-1.8mm} & \includegraphics[height=0.6in]{./figs/sota_comp_nyu/357/NYU_357_ours.pdf}
	\\ 
	
    \rotatebox[origin=l]{90}{\scriptsize \quad \textbf{DIML}} & \includegraphics[height=0.6in]{./figs/sota_comp_diml/856/DIML_856_img.pdf}
	\hspace{-1.8mm} & \includegraphics[height=0.6in]{./figs/sota_comp_diml/856/DIML_856_source.pdf}
	\hspace{-1.8mm} & \includegraphics[height=0.6in]{./figs/sota_comp_diml/856/DIML_856_GT.pdf}
	\hspace{-1.8mm} & \includegraphics[height=0.6in]{./figs/sota_comp_diml/856/DIML_856_PMBA.pdf}
	\hspace{-1.8mm} & \includegraphics[height=0.6in]{./figs/sota_comp_diml/856/DIML_856_FDSR.pdf}
	\hspace{-1.8mm} & \includegraphics[height=0.6in]{./figs/sota_comp_diml/856/DIML_856_JIIF.pdf}
	\hspace{-1.8mm} & \includegraphics[height=0.6in]{./figs/sota_comp_diml/856/DIML_856_DCTnet.pdf}
	\hspace{-1.8mm} & \includegraphics[height=0.6in]{./figs/sota_comp_diml/856/DIML_856_LGR.pdf}
	\hspace{-1.8mm} & \includegraphics[height=0.6in]{./figs/sota_comp_diml/856/DIML_856_MSS.pdf}
 
	\hspace{-1.8mm} & \includegraphics[height=0.6in]{./figs/sota_comp_diml/856/DIML_856_ours.pdf}
 \\
	& \scriptsize \textbf{(a)} RGB & \scriptsize \textbf{(b)} Bicubic & \scriptsize \textbf{(c)} GT & \scriptsize \textbf{(d)} PMBA & \scriptsize \textbf{(e)} FDSR & \scriptsize \textbf{(f)} JIIF & \scriptsize \textbf{(g)} DCTNet & \scriptsize \textbf{(h)} LGR & \scriptsize \textbf{(i)} \netname{} & \scriptsize \textbf{(j)} \netname{} (depth)
	\end{tabular}
    \vspace{-0.3cm}
	\caption{\textbf{Qualitative comparison on Middlebury, NYUv2 and DIML datasets (scaling factor $8\times$).} From left to right: (a) RGB image, (b) Bicubic upsampled depth map, (c) GT; then, error maps achieved by selected methods: (d) PMBA~\cite{ye2020pmbanet}, (e) FDSR~\cite{he2021towards}, (f) JIIF~\cite{tang2021joint}, (g) DCTNet~\cite{zhao2022discrete}, (h) LGR~\cite{de2022learning}; finally, (i) error maps and (j) predictions by \netname.} 
	\label{qualitative}
\end{figure*}


\begin{table*}[htbp] \footnotesize
	\renewcommand\tabcolsep{1.5pt} 
	\centering
	\scalebox{0.85}{
	\begin{tabular}{@{}ccccccccccccccccc@{}}
		\toprule
		 Testing Dataset & Metric & GF\cite{he2010guided} & SD~\cite{ham2017robust}  & GSRPT~\cite{lutio2019guided} & MSG~\cite{hui2016depth} & FDKN~\cite{kim2021deformable} & PMBANet~\cite{ye2020pmbanet} & FDSR~\cite{he2021towards} & JIIF~\cite{tang2021joint} & DCTNet~\cite{zhao2022discrete} & LGR~\cite{de2022learning} & \netname$^+$ \\ \midrule
		\multirow{2}{*}{DIML}
		& MSE & 34.1 & 44.9 & 23.0 & 5.76 & 6.74 & 7.35 & 7.73 & \silver{4.10} & 5.64 & \bronze{4.95} & \gold{3.72} \\
		& MAE & 1.77 & 0.83 & 1.26 & 0.51 & 0.53 & 0.59 & 0.74 & \silver{0.38} & 0.77 & \bronze{0.40} & \gold{0.36} \\ \hline
		\multirow{2}{*}{Middlebury\textit{-HR}}
		& MSE & 40.5 & 82.5 & 32.7 & 11.0 & \bronze{10.0} & \silver{9.62} & 18.4 & 19.3 & 17.5 & \gold{8.25} & 14.6 \\
		& MAE & 1.49 & 0.86 & 0.82 & 0.54 & \silver{0.43} & \bronze{0.46} & 0.73 & 0.74 & 0.77 & \gold{0.35} & 0.54  \\ \hline
		\multirow{2}{*}{Middlebury\textit{-LR}}
		& MSE & 25.6 & 28.8 & 15.8 & 8.89 & 5.54 & 4.16 & 6.92 & 4.40 & 6.96 & 5.94 & \gold{3.44} \\
		& MAE & 2.31 & 2.07 & 1.73 & 1.62 & 0.99 & \silver{0.91} & 1.09 & \bronze{0.92} & 1.15 & 1.11 & \gold{0.87}  \\
        \bottomrule
	\end{tabular}}
	\vspace{-0.3cm}
	\caption{\textbf{Cross-dataset generalization.} All methods are trained on NYUv2 and tested on DIML/Middlebury with factor $8\times$. Middlebury\textit{-HR} is the test set defined in \cite{de2022learning}, Middlebury\textit{-LR} is the one from \cite{tang2021joint}. The lower MSE and MAE, the better. }
	\label{cross-data_comparison}
\end{table*}

\subsection{Comparison with State-of-the-Art}
We compare \netname{} to GF \cite{he2010guided}, SD \cite{ham2017robust}, GSRPT \cite{lutio2019guided}, MSG \cite{hui2016depth}, DKN and its fast implementation FDKN \cite{kim2021deformable}, PMBANet \cite{ye2020pmbanet}, FDSR \cite{he2021towards}, JIIF \cite{tang2021joint}, DCTNet \cite{zhao2022discrete}, LGR \cite{de2022learning}, and finally to DADA~\cite{metzger2022guided} on Middlebury, NYUv2 and DIML datasets. We could not compare with PDRNet \cite{liu2022pdr} under the same setting because the source code is unavailable at the time of writing. For the other methods, we use the results from \cite{de2022learning} or the officially published codes, and results from \cite{yuan2023recurrent,metzger2022guided} for concurrent works. On the RGBDD dataset, the proposed network is compared to SDF~\cite{li2016deep}, SVLRM \cite{pan2019spatially}, DJF~\cite{li2016deep}, DJFR~\cite{li2019joint}, PAC~\cite{su2019pixel}, CUNet~\cite{deng2020deep}, FDKN~\cite{kim2021deformable}, DKN~\cite{kim2021deformable}, FDSR~\cite{he2021towards}, DCTNet~\cite{zhao2022discrete} and RASG~\cite{yuan2023recurrent}. To be fair with DCTNet~\cite{zhao2022discrete}, we downsample depth maps as the LR input.  
When reporting results, we highlight \gold{absolute}, \silver{second} and \bronze{third} best methods for each metric on each dataset.

\textbf{Quantitative Comparison.} Tabs. \ref{sota_comparison_mid_nyu_diml} and \ref{sota_comparison_rgbdd} report the accuracy of super-solved depth maps at factors $4\times$, $8\times$ and $16\times$ on the four datasets. As expected, learning-based methods show a significant improvement over traditional methods \cite{he2010guided,ham2017robust,lutio2019guided}. \netname{} vastly outperforms any existing network, with larger gaps in accuracy with the increasing of the upsampling factor. This can be attributed to the limitations affecting existing methods, i.e., 1) the guidance of either explicit or implicit RGB features alone being insufficient; 2) multi-modal information fusion on a single scale being not flexible enough to deal with complex scenes. Both limitations are fully addressed by \netname, which consistently outperforms concurrent works \cite{metzger2022guided,yuan2023recurrent}. 


The margin is consistent both on perfect (Middlebury) and noisy datasets (NYUv2, DIML, RGBDD), with the latter being a more challenging, realistic benchmark. Although \netname$^+$ is definitely the absolute best, its margin over \netname{} is negligible, with tiny gains yielded by NLSPN with respect to our main modules. Indeed, \netname{} alone consistently outperforms any other approach already.

       
\textbf{Qualitative Comparison.}
Fig. \ref{qualitative} shows qualitative comparisons of $8\times$ super-solved depth maps on Middlebury, NYUv2 and DIML datasets, respectively. From left to right, we show, the RGB image and LR depth map, followed by the ground truth HR depth and error maps obtained by several state-of-the-art frameworks, concluding with ours in the second-to-last columns. In each of the three examples, the lower error magnitude produced by \netname{}$^+$ further demonstrates its superior accuracy. 

\textbf{Cross-dataset Generalization.}
We conclude the comparison with existing methods by conducting cross-dataset experiments with $8\times$ factor. All methods are trained on the NYUv2 dataset and directly evaluated on DIML and Middlebury. Table \ref{cross-data_comparison} collects quantitative results for the 11 selected methods. Again, CNN-based methods attain better performance than traditional approaches, despite the domain gap playing a significant role in performance -- as evident by comparing results with Table \ref{cross-data_comparison}. Nonetheless, \netname{} outperforms any other framework on DIML. 


\begin{figure}	
	\centering	
	\captionsetup[subfigure]{font=footnotesize,textfont=footnotesize}
	\subfloat[RGB]{	
		\centering	
		\label{cross_dataset} 
		\includegraphics[height=0.8in]{./figs/ablation_figure/cross_dataset/receptive_field/cross_dataset.pdf}}	
	\hspace{-2mm}
	\subfloat[$D_{hr}$]{	
		\centering	
		\label{HR}
		\includegraphics[width=0.8in]{./figs/ablation_figure/cross_dataset/receptive_field/HR.pdf}}
	\hspace{-2mm}
	\subfloat[$D_{lr}$]{	
		\centering	
		\label{LR}
		\includegraphics[width=0.8in]{./figs/ablation_figure/cross_dataset/receptive_field/LR.pdf}}
		\vspace{-0.3cm}
	\caption{\textbf{Image context processed on Middlebury -- HR vs LR.} (a) RGB image and depth patches $D$ processed when testing on (b) Middlebury\textit{-HR} and (c) Middlebury\textit{-LR}. }	
	\label{hr-lr} 
\end{figure}

When considering the Middlebury dataset, we evaluate using the setting proposed in \cite{de2022learning} -- Middlebury\textit{-HR} in the table. In this case, our results are slightly less accurate compared to a few existing methods. However, given the very high resolution of Middlebury images, we argue that this testing protocol -- i.e., consisting of processing $256\times 256$ crops at a time -- penalizes our network's ability to leverage the global context in the input that results irremediably reduced to a very local area in these images. Therefore, we also evaluate on Middlebury test set defined by~\cite{tang2021joint} -- Middlebury-\textit{LR} in the table. Note that different subsets of images are used in Middlebury\textit{-HR} and Middlebury-\textit{LR} splits. Besides, Middlebury-\textit{LR} images are resized and processed without cropping, i.e., used at full-size after resizing, allowing to fully exploit global context, while this is not feasible with Middlebury-\textit{HR} due to memory constraints. In this case, \netname{} attains the best performance again, confirming our previous analysis, as shown in Tab. \ref{cross-data_comparison}. Such a difference in terms of context is highlighted in Fig. \ref{hr-lr}.

\begin{table}[t]
    \centering
	\renewcommand\tabcolsep{3pt} 
    \scalebox{0.5}{
    \begin{tabular}{ccc}

    \begin{tabular}{@{}ccccccc@{}} %\label{hf_infomation}
		\toprule
		\textbf{No.} & \textbf{Gradient} & \tabincell{c}{\textbf{Shallow} \\ \textbf{Feature}} & \textbf{LCF} & \textbf{ResBlock} & \textbf{MSE} & \textbf{MAE}\\
		\midrule
		(\uppercase\expandafter{\romannumeral1}) & \XSolidBrush &  \Checkmark     &  \Checkmark &  & 13.1 & 1.19 \\
		(\uppercase\expandafter{\romannumeral2}) & \Checkmark &    \XSolidBrush   &   &  & 12.4 & 1.14 \\
		(\uppercase\expandafter{\romannumeral3}) & \Checkmark &    \Checkmark     &   & \Checkmark & 12.3 & 1.15 \\
		\rowcolor{LightYellow}
		(\uppercase\expandafter{\romannumeral4}) & \Checkmark &    \Checkmark     & \Checkmark  &  & \gold{11.8} & \gold{1.12} \\
		\bottomrule
	\end{tabular}
	
	& \quad &
	
	\begin{tabular}{@{}clcc@{}} %\label{edge_types}
		\toprule
		\specialrule{0em}{3pt}{3pt}
		\multicolumn{1}{c}{\textbf{No.}} & 
		\tabincell{l}{\textbf{HF Information} \textbf{ \quad\quad\quad\quad}} & \textbf{MSE} & \textbf{MAE}\\
		\specialrule{0em}{3pt}{2pt}
		\midrule
		(\uppercase\expandafter{\romannumeral1}) & 
		{Canny Edge} & 12.0 & 1.13 \\
		(\uppercase\expandafter{\romannumeral2}) & 
		{Gaussian Edge} & 12.1 & 1.16 \\
		(\uppercase\expandafter{\romannumeral3}) & 
		{DCT} & 12.1 & 1.15 \\
		(\uppercase\expandafter{\romannumeral4}) & 
		{Wavelet Transform} & 12.1 & 1.15  \\
		\rowcolor{LightYellow}
		(\uppercase\expandafter{\romannumeral5}) & 
		{Gradient Map} & \gold{11.8} & \gold{1.12} \\
		\bottomrule
	\end{tabular}
	
	\\
	\textbf{(a)} & \quad & \textbf{(b)} 
	\\
	\\
	
	
	\begin{tabular}{@{}clcccc@{}} %\label{dsp_ablation}
		\toprule
		\textbf{No.} & \textbf{Config.} & \textbf{Params (M)} & \textbf{Flops (G)} & \textbf{MSE} & \textbf{MAE}\\
		\midrule
		(\uppercase\expandafter{\romannumeral1}) & EdgeNet \cite{liu2021multi}    & 5.78 &  95.6  & 12.0 & \gold{1.12} \\
		(\uppercase\expandafter{\romannumeral2}) & SCPA \cite{zhao2020efficient}  & 0.29 &  13.1  & 12.5 & 1.16 \\
		\rowcolor{LightYellow}
		(\uppercase\expandafter{\romannumeral3}) & HFEB       & \gold{0.27} & \gold{11.6}  & \gold{1.18} & \gold{1.12} \\
		\rowcolor{white}
		\bottomrule
		\multicolumn{4}{c}{\quad\quad\textbf{(c)}} \\
		\\
		\toprule
		\textbf{No.} & \textbf{Config.} & \textbf{Params (M)} & \textbf{MSE} & \textbf{MAE}\\
		\midrule
		(\uppercase\expandafter{\romannumeral1}) & 
		w/o AFFM        & -   & 12.7 & 1.16 \\
		(\uppercase\expandafter{\romannumeral2}) & 
		w/o att         & 1.3 & 12.2 & 1.13 \\
		(\uppercase\expandafter{\romannumeral3}) & 
		Concat.  & 4.5 & 12.2 & 1.13 \\
		\rowcolor{LightYellow}
		(\uppercase\expandafter{\romannumeral4}) & 
		AFFM & 3.0 & \gold{11.8} & \gold{1.12} & \\
		\rowcolor{white}
		\bottomrule
		\multicolumn{4}{c}{\quad\quad\textbf{(e)}} \\
	\end{tabular}
	
	
	& \quad &
	
	
	\begin{tabular}{@{}clccc@{}} %\label{affm_setting}
		\toprule
		\textbf{No.} & \textbf{Scales} & \textbf{Params (M)} & \textbf{MSE} & \textbf{MAE}\\
		\midrule
		(\uppercase\expandafter{\romannumeral1}) & 
		H1              & 1.5 & 12.3 & 1.14 \\
		\rowcolor{LightYellow}
		(\uppercase\expandafter{\romannumeral2}) & 
		H1, H2       & 3.0 & \gold{11.8} & \gold{1.12} \\
		\rowcolor{white}
		(\uppercase\expandafter{\romannumeral3}) & 
		H1, H2, H3              & 4.5 & \gold{11.8} & \gold{1.12} \\
		\bottomrule
		\multicolumn{4}{c}{\textbf{(d)}} \\
% 		\\
% 		\\
        \specialrule{0em}{5.4pt}{5.4pt} %
		\toprule
		\specialrule{0em}{1.7pt}{1.7pt} %
		\textbf{No.} & \textbf{Stages} & \textbf{Params (M)} & \textbf{MSE} & \textbf{MAE}\\
		\specialrule{0em}{1.7pt}{1.7pt} %
		\midrule
		\specialrule{0em}{1.8pt}{1.8pt} %
		(\uppercase\expandafter{\romannumeral1}) & 
		$1$   & 14.2 & 13.3 & 1.19 \\
		\specialrule{0em}{1.8pt}{1.8pt} %
		\rowcolor{LightYellow}
		(\uppercase\expandafter{\romannumeral2}) & 
		$2$   & 25.0 & 11.8 & 1.12 \\
		\specialrule{0em}{1.8pt}{1.8pt} %
		\rowcolor{white}
		(\uppercase\expandafter{\romannumeral3}) & 
		$3$   & 37.5 & \gold{11.6} & \gold{1.10} \\
		\specialrule{0em}{1.8pt}{1.8pt} %
		\bottomrule
		\multicolumn{4}{c}{\quad\quad\textbf{(f)}} \\
	\end{tabular}
	
    \end{tabular}}
    \vspace{-0.3cm}
    \caption{\textbf{Ablation study (NYUv2 test set, $8\times$ factor).} We measure the impact of (a) explicit vs implicit HR features, (b) different kinds of HF supervision, (c) different sub-networks for explicit HF features extraction, (d) scales at which AFFM is applied, (e) modules building AFFM, (f) number of stages in GDRB. In yellow, configurations corresponding to our final model without NLSPN.}
    \label{tab:ablations}
\end{table}


\subsection{Ablation Study}
We now perform a series of ablation experiments to measure the impact of key components and parameters in \netname. Tab. \ref{tab:ablations} collects the outcome of these studies, conducted on NYUv2 test set with $8\times$ factor. Without loss of fairness, NLSPN is never used here -- to fully focus on the impact of single components. 

\textbf{(a) Implicit vs Explicit High-Frequency Features.}
To measure the impact of both implicit and explicit HR features, we compare the performance of the proposed network and its variants when extracting either only one of the two. The quantitative results are collected in Tab.~\ref{tab:ablations}(a). Without the help of gradient maps (I), the performance of the network significantly degrades. We believe this is caused by the difficulty in effectively extracting fine structures or salient edges required for LR depth maps from implicit HF features alone. Moreover, explicit features highlight regions in the image that need to be focused on, avoiding \netname{} to learn to localize them and easing its task. 


Nonetheless, explicit HF features alone as guidance (II) are insufficient as well. We argue that the explicit information might neglect some RGB features, whereas implicit HF feature extraction can recover them. Furthermore, to verify the effectiveness of LCF, we replace it with ResBlock~\cite{he2016deep} (III) to extract shallow features from RGB images, highlighting a negative impact on implicit features extraction -- i.e., it results less accurate than (II). 

\textbf{(b) Ablation on Explicit High-Frequency Features.}
We now investigate which kind of HF information is more effective for our framework. Purposely, we train HFEB with supervision coming from five different HF features used as ground truth edge maps $E_{gt}$. Tab.~\ref{tab:ablations}(b) collects results from this experiment, highlighting that Canny edges (I) and Gradient maps (V) lead to slightly better results. 


\textbf{(c) Impact of HFEB.}
To verify the effectiveness of HFEB, we replace it with EdgeNet~\cite{liu2021multi} -- based on the widely-used U-net structure -- and SCPA~\cite{zhao2020efficient}, which inspires our scaling strategy. As shown in Tab.~\ref{tab:ablations}(c), EdgeNet (I) achieves lower MSE and MAE than SCPA (II), yet needs more parameters -- 5.78M vs. 0.29M. HFEB (III) yields the same accuracy as EdgeNet, with fewer parameters than SCPA, thus being both more accurate and efficient. 



\textbf{(d -- e) Impact of AFFM.}
We now measure the effectiveness of AFFM. Tab.~\ref{tab:ablations}(d) shows results obtained by deploying AFFM at different scales, respectively the highest (I), the first two (II) and all of the three scales. We can notice how performing fusion at the highest scale alone results insufficient, whereas using multi-scale features for fusion yields improvements, despite saturating already when using two scales, with the lowest one not providing additional, meaningful details to be taken into account.

Furthermore, we ablate AFFM in its single components. Tab.~\ref{tab:ablations}(e) resumes the outcome of this evaluation. 
We first test the performance of \netname{} without AFFM (I), highlighting a large drop in accuracy. By adding dynamic fusion, yet without using attention (II) vastly improves the results already, while replacing the weighted sum in the upper of Fig.~\ref{affm} with concatenation and a ResBlock~\cite{he2016deep} (III) yields worse results compared to our full AFFM (IV). 

\textbf{(f) Impact of Stages Number.}
To conclude, we evaluate the impact of the multi-stage design.
As shown in Tab.~\ref{tab:ablations}(f), a single-stage architecture (I) is vastly outperformed by deploying two stages (II), yet at the expense of doubling the number of parameters. Furthermore, while the three-stage architecture (III) still yields some improvement, the benefit is minor in comparison to the significant increase in parameters. Hence, we choose two stages as the default configuration to balance accuracy and efficiency.


\begin{table}[t] \footnotesize
	\renewcommand\tabcolsep{1.5pt} 
	\centering
	\scalebox{0.8}{
	\begin{tabular}{@{}lcccccc@{}}
		\toprule
		 & PMBANet~\cite{ye2020pmbanet} & FDSR~\cite{he2021towards} & JIIF~\cite{tang2021joint} & DCTNet~\cite{zhao2022discrete} & LGR~\cite{de2022learning} & Ours \\ 
		 \midrule
		 Runtime (ms)
		 & 26.9 & 1.03 & 89.8 & 9.03 & 26.4 & 51.5\\
		 Memory Peak (GB)
		 & 3.07 & 2.05 & 2.36 & 0.26 & 0.19 & 18.6 \\ 
		\bottomrule
	\end{tabular}}
	\vspace{-0.3cm}
	\caption{\textbf{Computational requirements}. Experiments on Nvidia RTX 3090 GPU, with $256\times256$ input and $8\times$ factor.}
	\label{runtime_memory}
    
\end{table}

\subsection{Limitations}
We conclude by listing a few limitations of \netname. As previously pointed out, global context is crucial for it to achieve the best performance. When this is unavailable, some accuracy is lost when generalizing across datasets. Moreover, the significant improvements over existing methods are paid for in terms of time/memory requirements. Tab. \ref{runtime_memory} highlights the higher runtime and, more evidently, peak memory usage. Future work will aim at reducing the overhead, while minimizing the drop in accuracy.


% noise
\begin{figure*}[!t]
	\centering
	\includegraphics[width=\linewidth]{figures/noise.pdf}
	\caption{
    The performance of GNN teacher, distilled MLP students via GLNN and PKGD when adding different noise to the initial node features.
	% Comparisons among GNN teacher, distilled MLP students via GLNN and PKGD.
 %    We add noise to the initial node features.
    For GNN teachers, we select SAGE, GAT, GCN and APPNP, respectively.
    \textbf{Upper}: \textbf{Cora} dataset and \textit{transductive} setting.
    \textbf{Lower}: \textbf{Pubmed} dataset and \textit{inductive} setting.
    }
	\label{noise}
	% \vspace{-0.5em}
\end{figure*}

\section{Analysis and Discussion}
We further explore the ability to capture graph structural information as well as the robustness of the proposed PGKD.
We also visualize the distributions of node representations for deeper insights.

\subsection{Can PGKD distill the Impact of Graph Edges?}

As mentioned in Section \ref{impact}, the intra-class edges guarantee the homophily for nodes from the same class, while the inter-class edges determine the pattern of distances among class prototypes.

% For Intra-class edges, we calculate the average L2 distance for the features of connected nodes in Graph.
We adopt SAGE as the GNN teacher and perform experiments under a transductive setting, and then calculate the average L2 distance for the features of connected nodes in the graph.
Table \ref{tab:intr_dis} shows the average distance of initial node features and node features from GNN teacher~(SAGE), GLNN, and PGKD.
The distance of the GNN teacher is the shortest due to the information aggregation operations along graph edges.
Meanwhile, the distance for GLNN is much longer due to the weak awareness of such graph structural information.
PGKD gets shorter distances than GLNN, showing a great ability to capture intra-class graph structural information.
In particular, PGKD gets a L2 distance of 0.82 on Citeseer, which is shorter than 1.40 from the GNN teacher.

\begin{table}[!t]
\centering
%\tableindent 
\renewcommand\arraystretch{1.2}
\resizebox{0.9\columnwidth}{!}
{%
\begin{tabular}{lcccc}
\hline
\textbf{Dataset}    & \textbf{Input}   & \textbf{GNN}  & \textbf{GLNN} & \textbf{PGKD} \\ \hline
Cora       & 4.40  & 1.95 & 3.02 & 2.47 \\
Citeseer   & 5.66  & 1.40 & 3.10 & 0.82 \\
A-computer & 17.63 & 2.35 & 7.14 & 4.76 \\ \hline
\end{tabular}
}
\caption{
Average L2 distance for the features of connected nodes on different datasets.
% We adapt SAGE for GNN teacher.
% GLNN learns MLP by vanilla logit-base KD.
}
\label{tab:intr_dis}
\end{table}


% \begin{table}[!t]
% \centering
% %\tableindent 
% \renewcommand\arraystretch{1.2}
% % \resizebox{0.8\columnwidth}{!}
% % {%
% \begin{tabular}{lccc}
% \hline
% \textbf{Dataset} & \textbf{GNN}   & \textbf{GLNN}  & \textbf{PGKD}  \\ \hline
% Cora                         & -0.94 & -0.88 & -0.92 \\
% Citeseer                     & -0.71 & -0.62 & -0.67     \\
% A-computer                   & -0.75 & -0.60 & -0.77    \\
% \hline
% \end{tabular}
% % }
% \caption{
% Spearman correlation $\rho$ between class distances and inter-class edges quantity.
% $\rho \to -1$ indicates more negatively correlated for two variables.
% }
% \label{tab:inter}
% \end{table}

\begin{table}[!t]
% \begin{wraptable}{r}{0.5\textwidth} 
\centering
%\tableindent 
% \renewcommand\arraystretch{1.2}
% \vspace{-0.6cm}

% \vspace{0.3cm}
\resizebox{0.35\textwidth}{!}
{%
\begin{tabular}{lccc}
\hline
\textbf{Dataset} & \textbf{GNN}   & \textbf{GLNN}  & \textbf{PGKD}  \\ \hline
Cora                         & -0.94 & -0.88 & -0.92 \\
Citeseer                     & -0.71 & -0.62 & -0.67     \\
A-computer                   & -0.75 & -0.60 & -0.77    \\
\hline
\end{tabular}
}
\caption{
Spearman correlation $\rho$ between class distances and inter-class edges quantity.
$\rho \to -1$ indicates more negatively correlated.
}
% \end{wraptable}
\label{tab:inter}
% \vspace{-0.3cm}
\end{table}

The inter-class edges determine the pattern of distances among class prototypes.
Specifically, the prototypes of two classes would be closer with more inter-edges connecting them in GNNs.
We take statistics on the class distances~(defined as L2 distances among class prototypes) and quantity of corresponding inter-class edges.
For qualitative analysis, we calculate the Spearman correlation.
From Table \ref{tab:inter}, the GNN teacher has a low Spearman correlation, whereas GLNN shows a relatively high value.
Meanwhile, the proposed PGKD, thanks to the intra-class loss, can better capture the intra-class graph structural information and exhibits a much lower correlation.

\subsection{Is PGKD Robust to Noisy Node Features?}
To analyze the robustness of PGKD on node noise, we further evaluate the performance after adding Gaussian noise of different levels to initial node features $X$.
Specifically, we replace $X$ with $(1-\alpha)X+\alpha \epsilon$, where $\epsilon$ denotes the isotropic Gaussian noise independent from $X$, and $\alpha \in [0,1]$ controls the noise level.
A larger $\alpha$ means a stronger noise.
Figure \ref{noise} shows the performance of GNN, GLNN, and PGKD under different noise levels.
On both Cora and Citeseer, PGKD outperforms GLNN consistently as the noise level ranges from 0.1 to 0.9.
Particularly, PGKD could get better results than GAT and APPNP on Pumbed with $\alpha=0.9$. 
These show that PGKD is more robust than GLNN with respect to noisy input node features due to its ability to capture graph structural information.



% split
\begin{figure}[!t]
	\centering
	\includegraphics[width=\linewidth]{figures/ratio.pdf}
	\caption{
	The performance of GNN teacher, distilled MLP students via GLNN and PKGD under \textit{inductive} setting with different split ratio.
    % \textbf{Upper}: \textbf{Citeseer} dataset and SAGE as GNN teacher.
    % \textbf{Lower}: \textbf{Pubmed} dataset and GCN as GNN teacher.
    We select GCN as GNN teacher and perform experiments on \textbf{Pubmed} dataset.
    }
	\label{ratio}
	% \vspace{-0.5em}
\end{figure}

\subsection{Impact of Inductive Split Ratio}
To evaluate the ability for less observed data under inductive setting, we conduct the experiments under different split ratios, defined as the ratio $|\mathcal{V}^{U}_{ind}|/|\mathcal{V}^{U}|$.
A larger split ratio means less observed unlabeled data during training and more inductive unlabeled data for test~(cf. Section \ref{setting_intro}).
% Please refer to Section \ref{setting_intro} for more details about inductive setting.
As shown in Figure \ref{ratio}, the performance of the GNN teacher is not monotonically decreasing since the way to split graph~(i.e. the edges to remove) is also vital as the number of nodes for training.
PGKD outperforms GLNN and GNN under all split ratios.
Also, the performance of PGKD is more stable than GLNN.
This proves that PKGD, explicitly capturing the graph structural information, is robust and effective under different inductive split ratios.

\subsection{Impact of MLP Setting}

\begin{table}[t]
\centering
%\tableindent 
% \renewcommand\arraystretch{1.2}

\resizebox{0.5\textwidth}{!}
{%
\begin{tabular}{lllcccc}
\hline
\textbf{\#L} & \textbf{\#H} & \textbf{Params} & \textbf{MLP} & \textbf{GLNN} & \textbf{PGKD} &  $\Delta$\textbf{GLNN} \\ \hline
2 & 64 & 0.09M                & 53.40        & 73.30         & 74.00  & \textbf{$\uparrow$0.70}       \\
2 & 128 & 0.18M              & 59.48        & 71.66         & 74.71 & \textbf{$\uparrow$3.05}        \\
3 & 128 & 0.20M               & 54.33        & 73.07         & 74.24 & \textbf{$\uparrow$1.17}        \\
2 & 512 & 0.73M               & 56.21        & 73.54         & 74.47  & \textbf{$\uparrow$0.92}       \\
3 & 512 & 1.00M               & 54.57        & 72.83         & 74.00   & \textbf{$\uparrow$1.17}      \\ \hline
\end{tabular}
}

\caption{
Comparisons for vanilla MLP, distilled MLP students via GLNN and PGKD with different MLP settings on \textbf{Cora} under \textit{inductive} setting.
We report the average test accuracy (\%).
\textbf{\#L} denotes the layers and \textbf{\#H} denotes dimension of hidden state.
}
\label{tab:mlp_impact}
\end{table}

We further conduct experiments using different MLP settings.
The GNN teacher is a two-layer GCN with 0.18M parameters and gets an accuracy of 83.37\% on Cora dataset.
As shown in Table \ref{tab:mlp_impact}, the vanilla MLP shows an overfitting trend when the number of parameters increases, while the PGKD does not.
Meanwhile, PGKD gets the highest results under all settings and shows consistent improvement over GLNN.
In particular, GLNN gets a score of 74.71\%~(\#L=2, \#H=128), which is 3.05\% higher than GLNN.
Such findings indicate that PGKD is more robust and effective in different MLP settings.

\subsection{Node Representation Distribution}

We visualize the distribution of node representations from GNNs and MLPs~(vanilla MLPs without KD, MLPs from GLNN, and MLPs from PGKD) via t-SNE \cite{JMLR:v9:vandermaaten08a}.
We select the GAT as the GNN teachers.
Figure \ref{node_visualize} shows the results on Cora and Citeseer under transductive setting.
Due to the message passing architecture, the node representations in the same class from GNNs are much more gathered than vanilla MLPs.
PGKD captures such graph information via intra-class loss, while vanilla MLPs and MLPs from GLNN lack such capability. 
The same-class features from both GLNN and vanilla MLP are slightly dispersed, while the features from PGKD are more clustered inside a class and separable between classes.
Moreover, PGKD can learn better class prototype distributions.
Specifically, in the GNN representations on Cora, the dark green and purple classes are far from each other. 
PGKD captures such a behavior well, where GLNN fails.

\begin{figure*}[!t]
	\centering
	\includegraphics[width=\linewidth]{figures/visualize_feats_new.pdf}
	\caption{
	The distribution of node representations for GNN teacher, vanilla MLP, and distilled MLPs from GLNN and PKGD.
    % under \textit{transductive} setting.
    % For GNN teacher, we select GAT.
    \textbf{Upper}: \textbf{Cora} dataset.
    \textbf{Lower}: \textbf{Citeseer} dataset.
    }
	\label{node_visualize}
	% \vspace{-0.5em}
\end{figure*}
% \input{sections/6-Related}

\section{Related Work}\label{sec:related}

Grasping is a fundamental problem for robotic manipulation and has been extensively studied. Most work focuses on parallel-jaw grippers \cite{DBLP:conf/cvpr/FangWGL20,jiang2011efficient,DBLP:conf/iccv/MousavianEF19,DBLP:conf/icra/MuraliMEPF20,DBLP:conf/icra/SundermeyerMTF21}  due to their simplicity, low DoFs, and computational efficiency. However, parallel-jaw grippers are less efficient and less reliable for manipulating arbitrary-shaped objects. To achieve user-friendly interaction, multi-finger robotic hands and dexterous grasping remain a hot research topic in the field of robotic manipulation~\cite{rimon2019mechanics}. This research can be briefly divided into two categories: the traditional analytical sampling-based method and the data-driven method.

\textbf{Traditional analytical sampling-based methods}\cite{ciocarlie2007dexterous,DBLP:conf/icra/GoldfederALP07,DBLP:conf/iros/HangSK14,DBLP:conf/icra/MillerKCA03,DBLP:conf/icra/PelossofMAJ04} sampled various grasp candidates and evaluated them based on certain metrics considering the physical properties of objects such as wrench space~\cite{DBLP:conf/icra/BorstFH04}. In general, both the object model and environment are assumed to be known in advance~\cite{DBLP:journals/ram/MillerA04}. Eigengrasp~\cite{ciocarlie2007dexterous} reduced the dimensions of grasp search space by performing principal component analysis (PCA) on grasping pose and configuration data. Although the reduction increases the efficiency of generating grasps, the search space of the random sampling process for grasps is still very huge. As a result, these sampling-based methods are less efficient in practical use.

\textbf{Data-driven methods} fall into one of two primary types.
The one is an extension of the traditional sampling-based method~\cite{DBLP:conf/iros/VarleyWWA15,DBLP:conf/icra/BorstFH04}. Instead of computing physical metrics, this method directly estimates grasp quality metrics from trained deep models. The grasp success rate can be greatly improved since traditional metrics cannot be computed accurately from an incomplete view of a novel object without any contact feedback. However, they are still dependent on known object models and exhibit the problem of huge sampling and search space.
%
The other data-driven method is performed in an end-to-end manner~\cite{DBLP:conf/iros/HangSK14,DBLP:conf/rss/LiuP0GM20,DBLP:journals/corr/abs-1908-04293,DBLP:conf/iros/LiuP0GM19,DBLP:conf/icra/KapplerBS15,DBLP:conf/iros/VarleyWWA15,mahler2017dex}. Specifically, this method takes the image or point cloud data of a grasped object as input and outputs a high-quality grasp. These approaches are able to effectively generate grasps and are robust to unknown objects. However, many can only handle a single object. Grasping may often fail due to the potential collision between the gripper and the environment.
%
Some recent work~\cite{DBLP:conf/icra/LiWL0LZ22,DBLP:conf/icra/LundellCLVWRMK21,DBLP:journals/corr/abs-2103-04783} predicts
collision-free \mbox{6-DoF} grasping in clutter using multi-finger grippers. They only classify the grasp types and do not take into account of the properties of multi-finger grasps. Our approach considers the gripper's physical structure and does not rely on the grasp types. Using a novel grasping representation and an end-to-end deep neural network based on contacts, our approach significantly reduces the search space for grasping and can generate reliable grasp poses.

\section{Conclusion}

% In this work, we propose PGKD to distill the knowledge from high-accuracy GNNs to low-latency MLPs.
% The distillation process is edge-free and the learned MLP students are structure-aware.
% Firstly, we analyze the impact of graph structure~(graph edges) on GNNs.
% Specifically, we categorize the graph edges into Intra-class edges and Inter-class edges and study their impact, respectively.
% Based on the analysis, we design two corresponding losses via class prototypes to transfer the graph structural knowledge from GNNs to MLPs.
% Experiments on popular benchmarks demonstrate the effectiveness of our proposed PGKD.
% Further analysis indicate that PGKD is robust to noisy node features and performs well in different training settings.

% For future work, we would consider to apply PGKD to other graph tasks other than node classification.
% Moreover, generating the prototypes basing on the node representations rather than the class labels would be another interesting topic.

A novel PGKD scheme has been proposed to distill the knowledge from high-accuracy GNNs to low-latency MLPs, wherein the distillation process is edge-free and the learned MLP students are structure-aware. 
Specifically, we analyze the impact of graph structure~(graph edges) on GNNs and categorize them into intra-class and inter-class edges. 
Two corresponding losses via class prototypes are designed to transfer the graph structural knowledge from GNNs to MLPs.
Experiments on popular benchmarks demonstrate the effectiveness of PGKD.
Additionally, we show PGKD is robust to noisy node features, and performs well under different training settings.

For our future work, PGKD will be generalized to other graph tasks beyond node classification. 
Another interesting direction will be to generate prototypes utilizing node representations rather than class labels.


\clearpage
\section*{Limitations}
In PGKD, we adopt the class prototypes to capture graph structural information for MLPs in an edge-free setting.
Subsequently, PGKD requires slightly more computing cost compared to the baseline GLNN.
Meanwhile, the gap between the MLP learned by PGKD and its teacher GNN under the inductive setting is larger than that under the transductive setting, especially on Cora and Penn94 datasets.
More effort to improve the performance under the inductive setting is required underway.


% \clearpage
% \balance
\bibliography{main}
\bibliographystyle{acl_natbib}



% \clearpage
% \section{Appendix for Proofs}

\paragraph{Proof of Theorem \ref{thm:main}.}

\begin{proof}
\label{proof:main}
Our proof has two steps. In Step 1, we will show that SimCLR is equivalent to minimizing the cross entropy loss defined in Eqn.~(\ref{eqn:cross-entropy}). 
In Step 2, we will show  that minimizing the cross-entropy loss 
is equivalent to spectral clustering on $\bfpi$. 
Combining the two steps together, we have proved our theorem. 

\textbf{Step 1: } SimCLR is equivalent to minimizing the cross entropy loss.

The cross-entropy loss takes expectation over 
$\bfW_\bfX\sim \mathbb{P}(\cdot ; \bfpi)$, 
which means $\bfW_\bfX$ has exactly one non-zero entry in each row $i$. By Lemma~\ref{lem:multinomial}, we know every row $i$ of $\bfW_\bfX$ is independent of other rows. Moreover, 
$\bfW_{\bfX,i}\sim \mathcal{M}(1, \bfpi_i/\sum_j \bfpi_{i,j})=\mathcal{M}(1, \bfpi_i)$, because $\bfpi_i$ itself is a probability distribution.
Similarly, we know $\bfW_\bfZ$ also has the row-independent property by sampling over $\mathbb{P}(\cdot;\bfK_\bfZ)$.
Therefore, by Lemma~\ref{lem:cross_split}, we know Eqn.~(\ref{eqn:cross-entropy}) is equivalent to:
\[
 -\sum_{i=1}^n \mathbb{E}_{\bfW_{\bfX,i}}[\log \mathbb{P}(\bfW_{\bfZ,i}=\bfW_{\bfX,i};\bfK_\bfZ)],
\]

This expression takes expectation over $\bfW_{\bfX,i}$ for the given row $i$. Notice that 
$\bfW_{\bfX,i}$ has exactly one non-zero entry, which equals $1$ (same for $\bfW_{\bfZ,i}$). 
As a result
we expand the above expression to be:
\begin{equation}
 -\sum_{i=1}^n \sum_{j\neq i} \Pr(\bfW_{\bfX,i,j}=1)\log \Pr(\bfW_{\bfZ,i,j}=1).
\label{eqn:detailed-expansion}    
\end{equation}


By Lemma~\ref{lem:multinomial}, $\Pr(\bfW_{\bfZ,i,j}=1)=\bfK_{\bfZ,i,j}/\|\bfK_{\bfZ,i}\|_1$ for $j\neq i$. Recall that $\bfK_\bfZ=(k(\bfZ_i-\bfZ_j))_{(i,j)\in[n]^2}$, which means 
$\bfK_{\bfZ,i,j}/\|\bfK_{\bfZ,i}\|_1=\frac{\exp(-\|\bfZ_i-\bfZ_j\|^2/{2\tau})}{\sum_{k\neq i}
\exp(-\|\bfZ_i-\bfZ_k\|^2/{2\tau})
}$ for $j\neq i$, when $k$ is the Gaussian kernel with variance $\tau$. 

Notice that $\bfZ_i=f(\bfX_i)$, so we know
\begin{equation}
-\log \Pr(\bfW_{\bfZ,i,j}=1)=
-\log \frac{\exp(-\|f(\bfX_i)-f(\bfX_j)\|^2/{2\tau})}{\sum_{k\neq i}
\exp(-\|f(\bfX_i)-f(\bfX_k)\|^2/{2\tau}),
}
\label{eqn:infonce-equivalence}    
\end{equation}


The right hand side is exactly the InfoNCE loss defined in Eqn.~(\ref{eqn:infonce}).
Inserting Eqn.~(\ref{eqn:infonce-equivalence}) into Eqn.~(\ref{eqn:detailed-expansion}), we get the SimCLR algorithm, which first samples augmentation pairs $(i,j)$ with $\Pr(\bfW_{\bfX,i,j}=1)$ for each row $i$, and then optimize the InfoNCE loss. 

\textbf{Step 2: } minimizing the cross entropy loss 
is equivalent to spectral clustering on $\bfpi$.


By Lemma~\ref{lem:convert_to_spectral}, we may further convert the loss to 
\begin{equation}
\label{eqn:main-theorem-repul-attr}
\min_{\bfZ}
-\sum_{(i,j)\in [n]^2} \mathbf{P}_{i,j}
\log k (\bfZ_i-\bfZ_j)+\log \mathbf{R}(\bfZ).
\end{equation}
Since $k$ is the Gaussian kernel, this reduces to \[
\min_\bfZ \mathrm{tr}(\bfZ^\top \mathbf{L}(\bfpi) \bfZ)
+\log \mathbf{R}(\bfZ),
\]

where we use the fact that $\mathbb{E}_{\bfW_\bfX\sim \mathbb{P}(\cdot; \bfpi)}[\mathbf{L}(\bfW_\bfX)]
=\mathbf{L}(\bfpi)
$, because the Laplacian operator is linear and $
\mathbb{E}_{\bfW_\bfX\sim \mathbb{P}(\cdot; \bfpi)}(\bfW_\bfX)=\bfpi
$.
\end{proof}

\paragraph{Proof of Theorem \ref{thm:clip}.}
\begin{proof}
Since $\bfW_\bfX\sim \mathbb{P}(\cdot;\bfpi_{\mathbf{A}, \mathbf{B}})$, we know 
$\bfW_\bfX$ has exactly one non-zero entry in each row, denoting the pair that got sampled. 
A notable difference compared to the previous proof is we now have $n_\mathcal{A}+n_\mathcal{B}$ objects in our graph. CLIP deals with this by taking a mini-batch of size $2N$, 
such that $n_\mathcal{A}=n_\mathcal{B}=N$, and adding the $2N$ InfoNCE losses together. We label the objects in $\mathcal{A}$ as $[n_\mathcal{A}]$, and the objects in $\mathcal{B}$ as $\{n_\mathcal{A}+1, \cdots, n_\mathcal{A}+n_\mathcal{B}\}$. 

Notice that $\bfpi_{\mathbf{A}, \mathbf{B}}$ is a bipartite graph, so the edges of objects in $\mathcal{A}$ will only connect to object in $\mathcal{B}$ and vice versa. We can define the similarity matrix in $\cZ$ as $\bfK_\bfZ$, 
where $\bfK_\bfZ(i, j+n_\mathcal{A})=\bfK_\bfZ(j+n_\mathcal{A},i)= k(\bfZ_i-\bfZ_j)$ for $i\in [n_\mathcal{A}], j\in [n_\mathcal{B}]$, and otherwise we set $\bfK_\bfZ(i,j)=0$. 
The rest is same as the previous proof. 
\end{proof}

\paragraph{Proof of Theorem \ref{thm:exponential}.}

\begin{proof}
\label{proof:exponential}
Since the objective function consists of a linear term combined with an entropy regularization, which is a strongly concave function, the maximization problem is a convex optimization problem. Owing to the implicit constraints provided by the entropy function, the problem is equivalent to having only the equality constraint. We then introduce the Lagrangian multiplier $\lambda$ and obtain the following relaxed problem:

$$
\widetilde{E}(\boldsymbol{\alpha})=\psi_{1}-\sum_{i=1}^n \alpha_{i} \psi_{i}+\tau \sum_{i=1}^n \alpha_{i}\log \alpha_{i}+\lambda\left(\boldsymbol{\alpha}^{\top} \mathbf{1}_n-1\right).
$$

As the relaxed problem is unconstrained, taking the derivative with respect to $\alpha_{i}$ yields

$$
\frac{\partial \widetilde{E}(\boldsymbol{\alpha})}{\partial \alpha_{i}}=-\psi_{i}+\tau\left(\log \alpha_{i}+\alpha_{i} \frac{1}{\alpha_{i}}\right)+\lambda=0.
$$

Solving the above equation implies that $\alpha_{i}$ takes the form
$
\alpha_{i}=\exp \left(\frac{1}{\tau} \psi_{i}\right) \exp \left(\frac{-\lambda}{\tau}-1\right).
$ Since $\alpha_{i}$ lies on the probability simplex, the optimal $\alpha_{i}$ is explicitly given by
$
\alpha^{*}_{i}=\frac{\exp \left(\frac{1}{\tau} \psi_{i}\right)}{\sum_{i^{\prime}=1}^n \exp \left(\frac{1}{\tau} \psi_{i^{\prime}}\right)} .
$ Substituting the optimal point into the objective function, we obtain
$$
\begin{aligned}
E\left(\boldsymbol{\alpha}^*\right)  &=\psi_1-\sum_{i=1}^n \frac{\exp \left(\frac{1}{\tau} \psi_{i}\right)}{\sum_{i^{\prime}=1}^n \exp \left(\frac{1}{\tau} \psi_{i^{\prime}}\right)} \psi_{i}+\tau \sum_{i=1}^n \frac{\exp \left(\frac{1}{\tau} \psi_{i}\right)}{\sum_{i^{\prime}=1}^n \exp \left(\frac{1}{\tau} \psi_{i^{\prime}}\right)}\log \frac{\exp \left(\frac{1}{\tau} \psi_{i}\right)}{\sum_{i^{\prime}=1}^n \exp \left(\frac{1}{\tau} \psi_{i^{\prime}}\right)} \\
& =\psi_1 - \tau \log \left(\sum_{i=1}^n \exp \left(\frac{1}{\tau} \psi_{i}\right)\right).
\end{aligned}
$$
Thus, the Lagrangian dual function is given by
\begin{equation*}
-E\left(\boldsymbol{\alpha}^*\right)= -\tau \log \frac{\exp \left(\frac{1}{\tau} \psi_{1}\right)}{\sum_{i=1}^n \exp \left(\frac{1}{\tau} \psi_{i}\right)}.\qedhere
\end{equation*}
\end{proof}



\section{More on Experiments} \label{section: experiment_details}

\paragraph{CIFAR-10 and CIFAR-100} CIFAR-10 ~\citep{krizhevsky2009learning} and CIFAR-100 ~\citep{krizhevsky2009learning} are well-known classic image classification datasets. Both CIFAR-10 and CIFAR-100 contain a total of 60k $32 \times 32$ labeled images of different classes, with 50k for training and 10k for testing. CIFAR-10 is similar to CIFAR-100, except there are 10 different classes in CIFAR-10 and 100 classes in CIFAR-100.

\paragraph{TinyImageNet} TinyImageNet ~\citep{le2015tiny} is a subset of ImageNet ~\citep{deng2009imagenet}. There are 200 different object classes in TinyImageNet, with 500 training images, 50 validation images, and 50 test images for each class. All the images in TinyImageNet are colored and labeled with a size of $64 \times 64$.

\textbf{Pseudo-code.} Algorithm \ref{alg:Training Procedure} presents the pseudo-code for our empirical training procedure.

\begin{algorithm}[!htbp]
\caption{Training Procedure}
\label{alg:Training Procedure}
\begin{algorithmic}[1]
\REQUIRE trainable encoder network $f$, batch size $N$, augmentation strategy \textit{aug}, loss function $L$ with hyperparameters \textit{args}
\FOR {sampled minibatch ${x_i}_{i=1}^N$}
\FORALL{$i \in { 1, ..., N }$}
\STATE draw two augmentations $t_i = \textit{aug}\left(x_i\right) $, $t_i' = \textit{aug}\left(x_i\right) $
\STATE $z_i = f\left(t_i\right)$, $z_i' = f\left(t_i'\right)$
\ENDFOR
\STATE compute loss $\mathcal{L} = L(N, z, z', \textit{args})$
\STATE update encoder network $f$ to minimize $\mathcal{L}$
\ENDFOR
\STATE \textbf{Return} encoder network $f$
\end{algorithmic}
\end{algorithm}

We also provide the pseudo-code for our core loss function used in the training procedure in Algorithm \ref{alg:Core loss}. The pseudo-code is almost identical to SimCLR's loss function, with the exception of an extra parameter $\gamma$.

\begin{algorithm}[!htbp]
\caption{Core loss function $\mathcal{C}$}
\label{alg:Core loss}
\begin{algorithmic}[1]
\REQUIRE batch size $N$, two encoded minibatches $z_1, z_2$, $\gamma$, temperature $\tau$
\STATE $z = \textit{concat}\left(z_1, z_2\right)$
\FOR {$i \in {1, ..., 2N }, j \in {1, ..., 2N}$ }
\STATE $s_{i,j} = \Vert z_i - z_j \Vert_2^{\gamma}$
\ENDFOR
\STATE \textbf{define} $l(i, j)$ \textbf{as} $l(i, j) = - \log \frac{exp\left(s_{i,j}/\tau \right)}{\sum_{k=1}^{2N} \mathbf{1}{[k \ne i]} exp\left(s{i, j} / \tau \right)} $
\STATE \textbf{Return} $\frac{1}{2N} \sum_{k=1}^N\left[l(i, i+N) + l(i+N, i)\right]$
\end{algorithmic}
\end{algorithm}

Utilizing the core loss function $\mathcal{C}$, we can define all kernel loss functions used in our experiments in Table \ref{table: loss definition}. For all $z_i \in z$ with even dimensions $n$, we define $z_{L_i} = z_i\left[0:n/2\right]$ and $z_{R_i} = z_i\left[n/2:n\right]$.

\begin{table}[ht]
\centering
\begin{tabular}{{@{}l|l@{}}}
Kernel  &  Loss function \\ \midrule
Laplacian & $\mathcal{C}\left(N, z, z', \gamma=1, \tau\right)$\\ \midrule
Sum       & $\lambda * \mathcal{C}\left(N, z, z', \gamma=1, \tau_1\right) + (1-\lambda) * \mathcal{C}\left(N, z, z', \gamma=2, \tau_2\right)$  \\ \midrule
Concatenation Sum&$\lambda * \mathcal{C}\left(N, z_L, z'_L, \gamma=1, \tau_1\right) + (1-\lambda) * \mathcal{C}\left(N, z_R, z'_R, \gamma=2, \tau_2\right)$\\ \midrule
$\gamma = 0.5$ & $\mathcal{C}\left(N, z, z', \gamma=0.5, \tau\right)$          \\ 

\end{tabular}

\caption{Definition of kernel loss functions in our experiments}
\label {table: loss definition}
\end{table}

\textbf{Baselines.} We reproduce the SimCLR algorithm using PyTorch Lightning~\citep{PytorchLightning}.

\textbf{Encoder details.}
The encoder $f$ consists of a backbone network and a projection network. We employ ResNet50~\citep{ResNet} as the backbone and a 2-layer MLP (connected by a batch normalization~\citep{ioffe2015batch} layer and a ReLU \cite{nair2010rectified} layer) with hidden dimensions 2048 and output dimensions 128 (or 256 in the concatenation kernel case).

\textbf{Encoder hyperparameter tuning.}
For each encoder training case, we randomly sample 500 hyperparameter groups (sample details are shown in Table \ref{table: Hyperparameter sample}) and train these samples simultaneously using Ray Tune ~\citep{RayTune}, with the ASHA scheduler~\citep{li2018massively}. Ultimately, the hyperparameter group that maximizes the online validation accuracy (integrated in PyTorch Lightning) within 5000 validation steps is chosen for the given encoder training case.

\begin{table}[ht]
\centering

\begin{tabular}{@{}l|l|l@{}}
\midrule
Hyperparameter  & Sample Range & Sample Strategy \\ \midrule
start learning rate & $\left[10^{-2}, 10\right]$ & log uniform \\ \midrule
$\lambda$       & $\left[0, 1\right]$ & uniform \\ \midrule
$\tau$, $\tau_1$, $\tau_2$ & $\left[0, 1\right]$ & log uniform \\ \midrule
\end{tabular}

\caption{Hyperparameters sample strategy}
\label {table: Hyperparameter sample}
\end{table}

\textbf{Encoder training.} 
We train each encoder using the LARS optimizer~\citep{LARSOptimizer}, LambdaLR Scheduler in PyTorch, momentum 0.9, weight decay $10^{-6}$, batch size 256, and the aforementioned hyperparameters for 400 epochs on a single A-100 GPU.

\textbf{Image transformation.} The image transformation strategy, including augmentation, is identical to the default transformation strategy provided by PyTorch Lightning.

\textbf{Linear evaluation.}
The linear head is trained using the SGD optimizer with a cosine learning rate scheduler, batch size 64, and weight decay $10^{-6}$ for 100 epochs. The learning rate starts at $0.3$ and ends at $0$.

\textbf{Moco Experiments.} We also tested our method based on MoCo~\citep{he2019moco}. The results are summarized in Table \ref{tab:results-moco}. Here we choose ResNet18~\citep{ResNet} as the backbone and set a temperature of $0.1$ as default. For our simple sum kernel, we set $\lambda=0.8$. The results show that our method outperforms the original MoCo method.

\begin{table}[thb]
\centering
\caption{MoCo Experiment Results on CIFAR-10 and CIFAR-100.}
\label{tab:results-moco}
\resizebox{\textwidth}{!}{%
\begin{tabular}{@{}c|ccc|ccc@{}}
\toprule
\multirow{3}{*}{Method} & \multicolumn{3}{c|}{CIFAR-10} & \multicolumn{3}{c}{CIFAR-100} \\ \cmidrule(lr){2-4} \cmidrule(lr){5-7} 
                        & 200 epochs & 400 epochs    & 1000 epochs   & 200 epochs & 400 epochs & 1000 epochs         \\ \midrule
MoCo (repro.)         & $76.41 \pm 0.12$    & $80.01 \pm 0.15$          & $84.45 \pm 0.08$    & $\mathbf{47.02 \pm 0.11}$ & $52.50 \pm 0.07$ & $57.62 \pm 0.15$            \\
\midrule
Laplacian Kernel        & ${78.09 \pm 0.10}$    & $\mathbf{83.85 \pm 0.09}$          & $\mathbf{88.34 \pm 0.16}$    & $46.12 \pm 0.22$   & $53.44 \pm 0.17$ & $59.10 \pm 0.14$        \\
Simple Sum Kernel & $\mathbf{78.12 \pm 0.15}$   & $83.23 \pm 0.18$ & $87.50 \pm 0.20$ & $46.65 \pm 0.06$ & $\mathbf{53.62 \pm 0.19}$ & $\mathbf{59.83 \pm 0.12}$\\
\bottomrule
\end{tabular}
}
\end{table}



\section{More Experiments on Synthetic Data}


Consider a scenario with $n$ clusters, each containing $k$ vertices. Let the probability of vertices $u$ and $v$ from the same cluster belonging to $\bfpi$ be $p$. Conversely, for vertices $u$ and $v$ from different clusters, let the probability of belonging to $\pi$ be $q$. We generate the graph $\bfpi$ randomly, based on $p$ and $q$. We experiment with values of $k=100$ and $n=6$ for ease of visualization, embedding all points in a two-dimensional space. Each vertex's initial position originates from a normal distribution. In each iteration, we sample a subgraph of $\bfpi$ uniformly, ensuring each vertex has an out-degree of $1$. We then optimize the corresponding vectors using InfoNCE loss with an SGD optimizer and iterate until convergence. Our experimental setup consists of an SGD learning rate of $1$, an InfoNCE loss temperature of $0.5$, and a batch size of $50$. We evaluate two scenarios with different $p$ and $q$ values: $p=1$, $q=0$, and $p=0.75$, $q=0.2$. The results of these experiments are visualized in Figure \ref{fig:vis-spectral-cluster}. The obtained embeddings exhibit the hallmark pattern of spectral clustering of graph $\bfpi$.

\begin{figure}[!tb]
\centering
\subfigure{
\includegraphics[width=1\textwidth]{Figures/cluster_pi.png}
\label{fig:vis-cluster}
}
\subfigure{
\includegraphics[width=1\textwidth]{Figures/noised_cluster_pi.png}
\label{fig:vis-noised-cluster}
}
\caption{Visualizations of the optimization process using InfoNCE Loss on the vectors corresponding to $\bfpi$. Points of identical color belong to the same cluster within $\bfpi$. To showcase the internal structure of $\bfpi$, we randomly select 10 vertices from each cluster to display the edge distribution of $\bfpi$.}
\label{fig:vis-spectral-cluster}
\end{figure}



\end{document}
