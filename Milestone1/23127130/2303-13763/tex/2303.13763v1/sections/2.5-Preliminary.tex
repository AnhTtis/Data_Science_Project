\section{Preliminaries}

% \subsection{Task Formulation}
\paragraph{Notations.} Let $\mathcal{G}=(\mathcal{V},\mathcal{E})$ denote a graph, where $\mathcal{V}$ stands for all $N$ nodes with features $\mathbf{X} \in \mathbb{R}^{N \times D}$ and $\mathcal{E}$ stands for all edges.
We represent edges with an adjacency matrix $\mathbf{A}$, and $A_{u,v}=1$ if edge $(u,v) \in \mathcal{E}$ or be 0 otherwise.
For node classification task, the target is $\mathbf{Y} \in \mathbb{R}^{N \times K}$, where row $y_{v} \in R^{K}$ denotes the $K$-dim one-hot label for node $v$.
We adopt superscript $^L$ for labeled nodes~(i.e. $\mathcal{V}^{L}$, $\mathbf{X}^{L}$, and $\mathbf{Y}^{L}$) and superscript $^U$ for the rest unlabeled nodes~(i.e. $\mathcal{V}^{U}$, $\mathbf{X}^{U}$, and $\mathbf{Y}^{U}$).

\paragraph{Graph Neural Network.} Most GNNs follow the message-passing framework, where the representation $\mathbf{h}_v$ of node $v$ is updated by aggregating messages from its neighbors $\mathcal{N}_v$.
For the $l$-th layer, $\mathbf{h}^{l}_v$ is obtained from the previous layer's representations of its neighbors as follows:
\begin{equation}
    % \mathbf{h}^{l}_v = \text{UP}(\text{AG}(\{ h^{l-1}_{u}: u \in  \mathcal{N}_v \}), \mathbf{h}^{l-1}_v) ,
    h^{(l)}_{N(v)}=\text{AGGR}(\{ h^{l-1}_{u}: u \in  \mathcal{N}_v \})
\end{equation}
\begin{equation}
    h^{(l)}_{v} = \text{UPDATE}(h^{(l)}_{N(v)}, h^{l-1}_{v}),
\end{equation}
where AGGR and UPDATE denote the aggregate and update operations, respectively.

\paragraph{Transductive vs Inductive.} 
\label{setting_intro}
There are two setting for graph learning: transductive and inductive.
For transductive setting, models can utilize all node features and graph edges.
For inductive setting, we split the unlabeled data into disjoint inductive subset and observed subset~(i.e. $\mathcal{V}^U=\mathcal{V}^U_{obs} \cup \mathcal{V}^U_{ind}$ and $\mathcal{V}^U_{obs} \cap \mathcal{V}^U_{ind} = \emptyset$).
The edges between $\mathcal{V}^U_{obs}$ and $\mathcal{V}^U_{ind}$ are preserved~(cf. Table \ref{tab:trans_ind}).

\begin{table}[t]

\centering
%\tableindent 
\renewcommand\arraystretch{1.4}


\resizebox{\columnwidth}{!}
{%
\begin{tabular}{lllll}
\hline
\multicolumn{2}{c}{\textbf{Model Setting}}                & \textbf{Train} & \textbf{Test} & \textbf{KD} \\ \hline
\multicolumn{1}{l}{\multirow{2}{*}{GNN}} & \textit{tran} & $(\mathbf{X}, \mathcal{G}, \mathbf{Y}^{L})$     & $(\mathbf{X}^{U}, \mathcal{G}, \mathbf{Y}^{U})$    & $\mathbf{H}$  \\ 
\multicolumn{1}{l}{}                     & \textit{ind}   & $(\mathbf{X}^{L}, \mathcal{G}_{obs}, \mathbf{X}^{U}_{obs}, \mathbf{Y}^{L})$     & $(\mathbf{X}^{U}_{ind}, \mathbf{Y}^{U}_{ind})$    & $\mathbf{H}^{L} \cup \mathbf{H}^{U}_{obs}$ \\ \hline
\multicolumn{2}{c}{MLP}                          & $(\mathbf{X}^{L}, \mathbf{Y}^{L})$    &  $(\mathbf{X}^{U}, \mathbf{Y}^{U})$   & -  \\ \hline
\end{tabular}
}
\caption{
The inputs for GNNs and MLPs in different settings: 
transductive (\textit{tran}) and inductive (\textit{ind}).
KD denotes the employed features for knowledge distillation.
$\mathbf{H}$ denotes the graph nodes representations from GNN teachers.
}
\label{tab:trans_ind}

\end{table}

