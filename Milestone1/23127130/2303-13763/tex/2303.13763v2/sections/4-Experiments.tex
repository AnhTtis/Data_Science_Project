\section{Experiments}

This section presents an empirical study of PGKD on several popular graph benchmarks.
% This section, we perform an empirical study of PGKD on several popular graph benchmarks.
% We further do the ablation studies.

\subsection{Datasets}
To evaluate the performance of PGKD, we consider six popular benchmarks, including four homophilous graph datasets, namely, Cora \cite{DBLP:journals/aim/SenNBGGE08}, Citeseer \cite{DBLP:journals/aim/SenNBGGE08}, Pubmed \cite{namata2012query}, and A-computer \cite{DBLP:journals/corr/abs-1811-05868}, and two heterophilous graph datasets, namely, Penn94 \cite{DBLP:conf/nips/LimHLHGBL21} and Twitch-gamer \cite{DBLP:conf/nips/LimHLHGBL21}.
Table \ref{tab: dataset} shows the dataset details.
We split these datasets for train/validation/test following GLNN \cite{DBLP:conf/iclr/ZhangLSS22} for fair comparison.
For the metric, we report the average accuracy on test data over five runs with different random seeds.

The ARMBench dataset presents: 1) a collection of sensor data acquired by a robotic manipulation workcell performing pick-and-place operation, 2) metadata and reference images for objects in containers, 3) a set of annotations acquired either automatically, by virtue of the system design, or via manual labeling, and 4) tasks and metrics to benchmark perception algorithms for robotic manipulation. Fig.\ \ref{fig:contributions} illustrates the benchmark tasks and variety of objects captured in the dataset. The dataset captures diversity in objects with respect to Amazon product categories as well as physical characteristics such as size, shape, material, deformability, appearance, fragility, etc. 

The data collection platform is a robotic manipulation workcell performing pick-and-place operation in a warehouse \cite{Sparrow2022}. The workcell contains a robotic arm mounted with a vacuum-based end-effector. It is presented with a heterogeneous collection of objects placed in unstructured configurations within a container (storage tote). The robotic arm is tasked with picking one object at a time (singulation) and place it on moving trays until the container is empty. The empty container ejects the workcell and is replaced by a new container. While the operation is completely autonomous, it includes a human-in-the-loop to monitor the status of each pick-and-place activity, annotate, and resolve any defects during manipulation. Multiple imaging sensors are placed in the workcell to facilitate and validate the pick-and-place operation. Following is a list of sensor data (Fig.\ \ref{fig:intro}) associated with each pick activity:
\begin{itemize}
\item Pick-image: A 5\,MP camera is used to capture a top-down image of the container.
% \item Pick-3D: Two Ensenso sensors capture the 3D point cloud of the source container.
\item Transfer-images: Multiple 5\,MP cameras are placed on different sides in the workcell to capture the moving object from different viewpoints.
% \item Transfer-Barcode: Multiple Cognex barcode sensors are used to scan the barcode of the object during transfer.
\item Place-image: A top-down view of the object is captured once it is placed on the tray.
\item Video: A camera is mounted to capture 720p videos of pick-and-place manipulation processes at 30\,FPS
\end{itemize}
Additionally, the following metadata (Fig.\ \ref{fig:contributions} (b)) is available by virtue of a warehouse tracking system:
\begin{itemize}
\item Container-manifest: A list of objects present in the container along with data such as product description, coarse dimensions, and weight.
\item Reference images: One or more images of objects from previous operations within the warehouse.
\end{itemize}
The sensor data and metadata were consumed by perception algorithms required to autonomously operate the robotic workcell. Benchmarking against these algorithms would not only optimize a manipulation task such as the one used for data collection but also enable more complex and intentional manipulation. This work considers a subset of such perception tasks namely object segmentation, object identification, and defect detection. These are critical not only to make informed grasping and motion decisions but also to track the state of the objects and containers within the warehouse. The following sections will describe these tasks and present the challenges using annotations, baseline algorithms, and evaluation metrics.

\subsection{Implementation}
\paragraph{GNN Teacher.} To evaluate the ability on different backbones, we select four popular GNNs as the teacher model: GraphSAGE \cite{DBLP:conf/nips/HamiltonYL17}, GAT \cite{DBLP:journals/corr/abs-1710-10903}, GCN \cite{DBLP:conf/iclr/KipfW17} and APPNP \cite{DBLP:conf/iclr/KlicperaBG19}, and perform experiments under both transductive and inductive settings.



\paragraph{Baselines.} For baselines, we do not compare with the regularization methods since these methods utilize the graph edges as extra inputs.
In real-world applications, these graph edges may be unavailable such as in federated graph learning.
Therefore, we conduct all experiments in the edge-free setting.
For fairness, we select the edge-free GLNN \cite{DBLP:conf/iclr/ZhangLSS22} as our baseline, which adapts vanilla logit-base KD from GNNs to MLPs.

\paragraph{Hyper-parameters.} We distill the two-layer GNN teacher to MLP student with two layers~(on Cora, Citeseer, and A-computer) or three layers~(on Pumbed, Penn94, and Twitch-gamer).
For PGKD, we employ grid search to train the MLPs, where $\lambda_1$ is searched in $\{0.1, 0.2, 0.4\}$ and $\lambda_2$ in $\{0.05, 0.1\}$.
We set $\tau_1$ and $\tau_2$ as 1 and 10, respectively.
The hidden state dimension is 128 for both GNNs and MLPs.
On all the datasets, the MLPs are trained for 500 epochs with early stopping.

\subsection{Main Results}


 \section{Main results}\label{sec:Main}
 \subsection{Continuous-time flow}\label{subsec:cont_time}
 In this section, we analyze the ordinary differential equation $\dot{\sx}(t) = \OF(\sx(t))$.  In all the remainder, we fix $r_1 >0$ and $K \subset \bbR^{n}$ with $ K = \{ x \in \bbR^n : \norm{h(x)} \leq r_1\}$.
Consider the following assumption:
 \begin{assumption}\label{hyp:cont_model}
   \begin{enumerate}[label=\roman*), nosep,leftmargin=15pt]
     \item\label{hyp:k_comp} The set $K$ is compact and $\nabla h$ is of full rank on $K$.
     \item\label{hyp:k_fullr}  It holds that $\nabla h^{\top} \nabla h A \in \bbR^{n_h \times n_h}$ is symmetric positive definite on $K$.
     \item\label{hyp:A_loclip} The function $A : K \rightarrow \bbR^{n_h \times n_h}$ can be extended to a locally Lipschitz continuous function on some neighborhood of $K$.
     \item\label{hyp:hA_eigenv}  There is $\alpha_m >0$ such that  $\inf_{x \in K} \lambda_m(x) > \alpha_m$, where $\lambda_m(x)$ is the minimal eigenvalue of $\nabla h^{\top}(x) \nabla h(x) A(x)$
   \end{enumerate}
 \end{assumption}
Note that as soon as $\cM$ is compact, there is always some $r_1 > 0$ such that \Cref{hyp:cont_model}-\ref{hyp:k_comp} holds. Moreover, \Cref{hyp:cont_model}-\ref{hyp:k_fullr}--\ref{hyp:A_loclip} are satisfied for the matrices $A$ given in~\Cref{ex:vanil,ex:mj_flow}.   As is often the case, to analyze the trajectory of an ordinary differential equation we need to find an energy (or Lyapunov) function. For $M > 0$, we define $\Lambda_M :\bbR^{n} \rightarrow \bbR$ as:
\begin{equation}\label{eq:def_LambdaM} \Lambda_M = f + M \norm{h} \, .
\end{equation}
The following theorem is our first main result, it shows that for $M$ large enough, $\Lambda_M$ decreases along any trajectory. This observation immediately implies the convergence of any bounded trajectory to the set of critical points. 
\begin{theorem}\label{th:cont_time}
  Assume \Cref{hyp:cont_model}.
  For any $x_0$ such that $\norm{h(x_0)} \leq r_1$ there is $\sx:\bbR_{+} \rightarrow \bbR^n$ a unique solution to 
  \begin{equation}
      \label{eq:orth_flow}
      \dot{\sx}(t) = \OF(\sx(t))
  \end{equation}
  starting at $x_0$. In addition, it holds that:
  \begin{enumerate}[nosep]
    \item For any $t \geq 0$, $\norm{h(\sx(t))} \leq \rme^{-\alpha_m t} \norm{h(x_0)}$,  where $\alpha_m$ is defined in \Cref{hyp:cont_model}-\ref{hyp:hA_eigenv}.
  \item For all $M \geq \overline{M}= M_1/\alpha_m$, with $M_1 = \sup_{x \in K} \norm{A^{\top} \nabla h^{\top}(\nabla f - \nabla h A h)}$, we get
  \begin{equation*}
    \inf_{0 \leq t \leq T} \norm{\OF(\sx(t))}^2= \inf_{0 \leq t \leq T} \norm{\dot{\sx}(t)}^2 \leq \frac{1}{T} \int_{0}^{T} \norm{\dot{\sx}(t)}^2 \rmd t \leq \frac{\Lambda_{M}(\sx(0)) - \Lambda_{M}(\sx(T))}{T} \, .
  \end{equation*}
  \item Let $x^*$ be in the limit set of $\sx$, i.e. there is $t_n \rightarrow + \infty$ such that $\sx(t_n) \rightarrow x^*$. Then $x^*$ is a critical point of \eqref{eq:main_opt_prob}.
  \end{enumerate}
\end{theorem}

\begin{proof}
 The existence and uniqueness of a local solution of \eqref{eq:orth_flow} follows from the fact that $\OF$ is locally Lipschitz continuous. As we shall see, such a solution must lie in $K$, which is compact by \Cref{hyp:cont_model}. This implies that the domain of a local solution can be extended to $\bbR_{+}$. Indeed, let $\sx$ be such a solution. Since for all $v \in V$, it holds that $\nabla h^{\top} v = 0$, we get using \Cref{hyp:cont_model}-\ref{hyp:hA_eigenv}:
\begin{equation}\label{eq:h_decrease_intm}
\frac{\dif}{ \dif t} \norm{h(\sx)}^2 = - 2 h^{\top}(\sx) \nabla h^{\top}(\sx) \nabla h(\sx) A(\sx) h(\sx)  \leq -2 \alpha_m \norm{h(\sx)}^2 \, ,
\end{equation}
 and Grönwall's lemma implies that $\norm{h(\sx(t))} \leq \rme^{-\alpha_m t} \norm{h(\sx(0))} $, for $t \geq 0$.
 Therefore, any local solution stays away from the boundary of $K$ and can be extended to a global solution for which the first claim holds. We now prove the second claim. Denote $D_h = (\nabla h^{\top} \nabla h)^{-1}$. In order to simplify the notations we omit the dependence on $x$ (see Lemma~\ref{lm:aff_proj}), and get
\begin{equation}\label{eq:interm_OFA}
\OF= - \nabla f +  \nabla h \left(  D_h \nabla h^{\top} \nabla f - A h \right)  \, ,
\end{equation}
where $D_h := (\nabla h^{\top} \nabla h)^{-1}$. This implies $\nabla h^{\top} \OF = - \nabla h^{\top} \nabla hA h$. Therefore, we have
\begin{align}\label{eq:err_f}
    \begin{split}
      \norm{(\OF + \nabla f)^{\top} \OF} &= \norm{\left(  D_h\nabla h^{\top} \nabla f - A h \right)^{\top} \nabla h^{\top} \OF}
      \\
      &\leq \norm{ h^{\top} A^{\top} \nabla h^{\top} \nabla h Ah - \nabla f^{\top} \nabla h A h}  \leq M_1 \norm{h} \, .
    \end{split}
\end{align}
Finally, if $\sx \not \in \cM$, we have
\begin{equation}\label{eq:f_decrease}
  \frac{\dif}{\dif t}f(\sx) = \nabla f(\sx)^{\top} \dot{\sx} = - \norm{\dot{\sx}}^2 + (\dot{\sx} + \nabla f(\sx))^{\top}\dot{\sx}(t) \leq - \norm{\dot{\sx}}^2 + M_1 \norm{h(\sx)} \, .
\end{equation}
Therefore, using~\eqref{eq:h_decrease_intm} and \eqref{eq:f_decrease} we obtain
\begin{equation}\label{eq:strict_lyap}
  \frac{\dif}{\dif t} \Lambda_M(\sx) \leq - \norm{\dot{\sx}}^2  \leq - \norm{\nabla_{V}f(\sx)}^2\, ,
\end{equation}
where the last inequality comes from the fact that the projection of $\dot{\sx}(t)$ onto $V$ is  $\nabla_V f$.
Integrating the last inequality we obtain the second claim for $\sx$.

To establish the third claim, we notice that $\OF \neq 0$ as soon as $x \notin \cM$ or $x \in \cM$ and $\Grad (f) \neq 0$. Equation~\eqref{eq:strict_lyap} then shows that $\Lambda_M$ is a strict Lyapunov function for the ODE~\eqref{eq:orth_flow} and the set of critical points of \eqref{eq:main_opt_prob}. In particular, LaSalle's invariance principle (see e.g. \cite[Theorem 2.17]{har_dynsyst91}) then implies that any limit point of $\sx$ must be contained in the set of critical points of \eqref{eq:main_opt_prob}.
\end{proof}
\vspace{-10pt}
\subsection{Algorithm}\label{sec:det_alg}
In this section we analyze the algorithms provided by the discretization of ODE~\eqref{eq:orth_flow} both in the deterministic and stochastic settings.
  Consider a filtered probability space $(\Omega, \mcF, \{\mcF_k, k >0\},  \bbP)$. Fix $x_0 \in K$ and let $(\eta_{k})_{k \geq 1}$ be a sequence of random variables adapted to $(\mcF_k)$. Our method, \algo, produces iterates as follows:
 \begin{equation}\label{eq:orth_alg}
     x_{k+1} = x_k + \gamma_k v_k + \gamma_k \eta_{k+1} , \quad{} \textrm{ with } v_k = \OF(x_k)
 \end{equation} 
 and with $(\gamma_k)$ a sequence of positive step sizes. The perturbation $(\eta_k)$ allows to capture the case where $\nabla f(x)$ (and hence $\nabla_V f(x)$) is unknown. This covers both streaming data and finite-sum problems in machine learning; see \citep{lan2020first}.
Recall that $\bbE_k$ denotes the conditional expectation given $\mcF_k$ and consider the following assumptions.
\begin{assumption}\label{hyp:disc_model}
\begin{enumerate}[label=\roman*), nosep]
    \item\label{hyp:fh_Lipgrad}The function $f$ (respectively $h$) has $L_f$ (respectively $L_h$) Lipschitz gradients on $K$.
    \item\label{hyp:iter_bound}The iterates $(x_k)$ remain in $K$, $\bbP$-almost surely.
    \item\label{hyp:zer_mean} For every $k \in \bbN$, it holds that $\eta_{k+1} \in V(x_k)$ and $\bbE_k[\eta_{k+1}] = 0$.
    \item\label{hyp:var_bound} There is a constant $\sigma \geq 0$ such that for all $k \in \bbN$, $\bbE_k[\norm{\eta_{k+1}}^2] \leq \sigma^2$.
\end{enumerate}
\end{assumption}
 \begin{example}\label{ex:SA}
 In the stochastic approximation framework, it is assumed that there is a probability space $(\Xi, \mcT, \mu)$ and a $\mu$-integrable function $g: \bbR^n \times \Xi \rightarrow \bbR^n$ such that for each $x \in \bbR^n$, $\int g(x, s) \mu(\rmd s)  = \nabla f(x)$. Let $(\xi_k)_{k \geq 1}$ be a sequence of i.i.d random variables defined on $(\Omega, \mcF, \bbP)$, taking values in $\Xi$ and such that the distribution of $\xi_k$ is $\mu$. We consider the following recursion
 \begin{equation*} 
 x_{k+1} = x_k - \gamma_k \nabla h(x_k) A(x_k) h(x_k) - \gamma_k g_V(x_k, \xi_{k+1}) \, , \end{equation*}
where $g_V(x, \xi)$ denotes the orthogonal projection of $g(x, \xi)$ onto $V(x)$. Thus, if we denote $\eta_{k+1} :=\nabla_V f(x_k) - g_V(x_k, \xi_{k+1})$ and $\mcF_k:= \sigma(\xi_1, \dots, \xi_k)$, we obtain \eqref{eq:orth_alg}. Note also that in this case $\eta_{k+1} \in V(x_k)$, $\bbE_k[\eta_{k+1}] = 0$, and if for some $\sigma >0$, it holds that $\sup_{x \in \bbR^n} \bbE[\norm{g(x,\xi) - \nabla f(x)}^2] \leq \sigma^2$, then $\bbE_k[\norm{\eta_{k+1}}^2] \leq \sigma^2$. 
 \end{example}
The deterministic setting is recovered by setting $\sigma = 0$. If $A$ is defined only on $K$ (see  \Cref{ex:mj_flow}), then \Cref{hyp:disc_model}-\ref{hyp:iter_bound} is required for the recursions to be properly defined. However, for $A$ as in~\Cref{ex:vanil}, this assumption is not needed. Nevertheless, it is necessary for our convergence analysis, and we show in \Cref{proof:safe_step}, that, under mild assumptions, if the step-sizes are small enough \Cref{hyp:disc_model}-\ref{hyp:iter_bound} is automatically satisfied.

The following theorem is the discrete counterpart of \Cref{th:cont_time}. It shows that \algo\ converges to the set of the critical points essentially at the same rate than (unconstrained) gradient descent. 
%Convergence is measured through $\OF$ which is meaningful thanks to Lemma~\ref{lm:OF_crit}.


\begin{theorem}\label{th:gen_rates}
  Assume \Cref{hyp:cont_model}--\ref{hyp:disc_model}. For any $M \geq \overline{M}$, where $\overline{M}$ is defined in \Cref{th:cont_time}, denote $D_M := \Lambda_M(x_0) - \inf_{x \in K} \Lambda_M(x)$ and let $\gamma \leq \gamma_{\max}= \min\left(\alpha_m^{-1}, (L_f + M L_h)^{-1} \right)$. Then, the following holds.
  \begin{enumerate}[nosep,leftmargin=15pt]
      \item If $\sigma = 0$, and for all $k$, $\gamma_k \equiv \gamma$, then:
      \begin{equation}\label{eq:det_rates}
  \inf_{ 0 \leq k \leq N-1}  \norm{\OF(x_k)}^2= \inf_{0\leq k \leq N-1} \norm{v_k}^2 \leq \frac{2 D_M}{N\gamma} \, .
  \end{equation}
   Furthermore, it holds that $\OF(x_k) \rightarrow 0$ and any accumulation point $x^*$ of $(x_k)$ is a critical point of Problem~\eqref{eq:main_opt_prob}.
  \item Otherwise, fix some constant $\bar D >0$, $N >0$ and $\gamma := \min(\gamma_{\max}, \bar D(\sigma \sqrt{N})^{-1})$. If $\gamma_k \equiv \gamma$, and $\hat k$ is uniformly sampled in $\{0, \dots, N-1\}$, then:
  \begin{equation}\label{eq:sto_rate}
  \bbE\left[\norm{\OF(x_{\hat k})}^2\right] \leq \frac{2D_M(L_f + M L_h+ \alpha_m)}{N} + \frac{\sigma}{\sqrt{N}}\left( \bar D (L_f + M L_h) + \frac{2 D_M}{\bar D}\right) \, .
  \end{equation}
  \end{enumerate}
\end{theorem}
\begin{proof}
Using a Taylor expansion of $\Lambda_M$ and using the upper-bound on $\gamma_k$, we obtain
    \begin{equation}\label{eq:rem_decr}
    2\left(\bbE_k[\Lambda_M(x_{k+1})] - \Lambda_M(x_k)\right) \leq - \gamma\norm{v_k}^2 + L_f + M L_h \sigma^2 \gamma^2 \, .
  \end{equation}
  Our claims then follow by telescoping this inequality and applying a standard proof technique (see e.g. \citet[Chapter~6]{lan2020first}) both in the deterministic and stochastic framework. Further details are given in \Cref{proof:gen_rates}.
\end{proof}


The preceding theorem shows that the rate of convergence of our algorithm, measured through $\OF$, is identical to the one obtained by gradient descent in a non-convex framework: $\cO(\varepsilon^{-2})$ in the deterministic setting and $\cO(\varepsilon^{-4})$ in the stochastic setting. As recently shown in \cite{CarmonLowerBF, CarmonLowerBF_sto}, these rates are tight, which makes our algorithm near-optimal in both cases.
  
   The term $(L_f + M L_h)$ in the definition of $\gamma_{\max}$ is  the Lipschitz constant of $\nabla f + M \nabla h$, hence our bound on the step sizes is reminiscent of the $L_f^{-1}$ bound required for convergence of standard gradient descent.
Note also that only an upper bound on $\overline{M}$ is required to achieve such rates. Indeed, in the deterministic setting, we can combine our method with line search; see \Cref{rm:safe_step}.
%below, we can estimate this threshold on the way: We start with one candidate and double it, e.g., if equation~\eqref{eq:rem_decr} is not satisfied (i.e., $\Lambda_M$ does not decrease). In this way, our candidate for the upper bound of $\overline{M}$ is modified only a finite number of times, so our convergence rate is preserved.
In the stochastic framework, performing line search is not an option, but we note that the discussion of \citet[Corollary~2.2.]{gha_lan13} applies here as well. In particular, we can make an error of the order of $\sqrt{N}$ in estimating $(L_f + M L_h)$ while maintaining our rate of convergence of $\cO(\varepsilon^{-4})$. If all constants are known, then the optimal $\overline{D}$ in equation~\eqref{eq:sto_rate} is $\sqrt{2D_M/(L_f + M L_h)}$. Finally, a nonconstant choice of step sizes is possible without affecting the final results; see \cite[Chapter~6]{lan2020first}. The choice of step size is further discussed in \Cref{proof:safe_step}.
%Thanks to the last proposition and remark, we can now made the following assumption.
%\begin{assumption}\label{hyp:iterates_bounded}
 % The iterates $(x_k)$ produced by Algorithm~\eqref{eq:orth_alg} remain in $K$.
%\end{assumption}


We conduct the experiments on six benchmarks and select SAGE as GNN teachers for small datasets~(Cora, Citeseer and A-computer) and GCN for large datasets~(Penn94, Pubmed and Twitch-gamer).
Meanwhile, we reproduce GLNN from its official codes.
Table \ref{tab: main_res} reports the accuracy results.
Some observations are in place:
\begin{enumerate}
    \item PKGD outperforms GLNN on all six benchmarks with higher average scores under both transductive and inductive settings, thus demonstrating the effectiveness of PGKD in capturing graph structural information for the MLPs.
    In particular, PGKD achieves 76.35\% on Pubmed under inductive setting, which is 1.86\% higher than GLNN.
    PKGD can even outperform GNN teachers on some datasets~(Citeseer and Pubmed). 
    \item The standard deviations of PGKD are smaller than GLNN for almost all datasets, showing the stability and robustness of PGKD.
    For instance, PGKD gets 0.39\% on A-computer under transductive setting, which is approximately 3$\times$ smaller than the 1.04\% of GLNN.
\end{enumerate}


\subsection{Ablation Studies}
To better understand PGKD, we conduct ablation experiments on intra-class loss and inter-class loss.
Without loss of generality, we select SAGE, GAT, GCN, APPNP as GNN teachers and compare the performance on Citeseer under inductive setting and Cora under transductive setting.


Table \ref{tab:abaltion}\&\ref{tab:abaltion2} show the experiment results.
From the tables, we find that the performance drops when either intra-class loss or inter-class loss is removed, indicating that both intra-class information and inter-class information are vital.
In general, removing the intra-class loss would lead to a larger drop than the inter-class loss under the transductive setting, but a smaller drop under the inductive setting.
Moreover, it is quite interesting to find that PGKD with one loss exclusively would perform worse than GLNN, but is better than GLNN with two losses together.
For example, PGKD gets 68.62\% and 68.23\%~(APPNP as GNN teacher) on Citeseer with one loss exclusively, which are lower than 69.23\% of GLNN.
However, PGKD would get a higher 69.78\% than GLNN with two losses.
Adopting SAGE as the GNN teacher on Citeseer also leads to a similar situation.
Such phenomenon indicates that simultaneously capturing both intra-class information and inter-class information is crucial for the MLP training.

\begin{table}[!t]
% \begin{wraptable}{l}{0.5\textwidth} 
\centering
\resizebox{0.5\textwidth}{!}
{%
\begin{tabular}{lcccc}
\hline
\textbf{Model} & \textbf{SAGE} & \textbf{GAT} & \textbf{GCN} & \textbf{APPNP} \\ \hline
GNN            & 69.89$\pm$2.83    & 71.69$\pm$2.90    & 70.83$\pm$3.12   & 72.93$\pm$2.11     \\
GLNN           & 69.34$\pm$2.10    & 69.12$\pm$3.83   & 69.01$\pm$2.92   & 69.23$\pm$1.74     \\ \hline
PGKD           & 69.94$\pm$2.03    & 69.89$\pm$4.51   & 70.00$\pm$2.00   & 69.78$\pm$1.59     \\
\quad -$\mathcal{L}_{intra}$           & 69.01$\pm$1.76    & 68.51$\pm$5.03   & 69.28$\pm$2.28   & 68.62$\pm$1.77     \\
\quad -$\mathcal{L}_{inter}$           & 68.62$\pm$2.87    & 69.17$\pm$3.62   & 69.12$\pm$1.98   & 68.23$\pm$2.01     \\ \hline
\end{tabular}
}

%\tableindent 
% \renewcommand\arraystretch{1.2}
\caption{
The ablation accuracy~(\%) on \textbf{Citeseer} dataset for several GNN teachers under \textit{inductive} setting.
Results are averaged for five runs.
}

\label{tab:abaltion}
% \vspace{-0.1cm}
\end{table}

\begin{table}[!t]
\centering
%\tableindent 
\renewcommand\arraystretch{1.4}
\resizebox{\columnwidth}{!}
{%
\begin{tabular}{lcccc}
\hline
\textbf{Model} & \textbf{SAGE} & \textbf{GAT} & \textbf{GCN} & \textbf{APPNP} \\ \hline
GNN            & 81.11$\pm$2.05    & 81.81$\pm$1.28    & 82.24$\pm$0.59   & 83.26$\pm$0.87     \\
GLNN           & 80.22$\pm$1.81    & 79.92$\pm$1.00   & 81.43$\pm$0.18   & 79.40$\pm$1.34     \\ \hline
PGKD           & 82.15$\pm$0.19    & 81.65$\pm$1.47   & 82.39$\pm$0.64   & 82.59$\pm$1.11     \\
\quad -$\mathcal{L}_{intra}$           & 80.83$\pm$1.26    & 80.21$\pm$1.10   & 81.12$\pm$0.92   & 79.55$\pm$1.04     \\
\quad -$\mathcal{L}_{inter}$           & 81.17$\pm$1.62    & 81.98$\pm$1.04   & 81.91$\pm$0.54   & 82.79$\pm$0.86     \\ \hline
\end{tabular}
}
\caption{
The ablation results~(\%) on \textbf{Cora} dataset for several GNN teachers under \textit{transductive} setting.
We report the results for five runs.
}
\label{tab:abaltion2}
\vspace{-1em}
\end{table}