\pdfoutput=1


\documentclass[11pt]{article}


\usepackage[]{coling}

% Standard package includes
\usepackage{times}
\usepackage{latexsym}


\usepackage[T1]{fontenc}

\usepackage[utf8]{inputenc}

\usepackage{microtype}

\usepackage{inconsolata}

\usepackage{graphicx}

\usepackage[T1]{fontenc}
\usepackage{graphicx}

\usepackage{booktabs} 
\usepackage{amsfonts} 
\usepackage{amsmath} 
\usepackage{multirow}

% table
\usepackage{tcolorbox}
\usepackage{colortbl}
\usepackage{xcolor}
\usepackage{array} 
\usepackage{makecell}
\usepackage{pifont}
\newcommand{\ccmark}{\ding{51}}%
\newcommand{\xxmark}{\ding{55}}%
\definecolor{OliveGreen}{rgb}{0,0.4,0}
\definecolor{gg}{HTML}{e2f0cb}


\title{Edge-free but Structure-aware: Prototype-Guided \\ Knowledge Distillation from GNNs to MLPs}


\newcommand*{\affmark}[1][*]{\textsuperscript{#1}}
\usepackage{fdsymbol}

\author{
Taiqiang Wu\affmark[$\diamondsuit$]\thanks{\ \ This work was done when Taiqiang was interning at Tencent. Corresponding authors: Yujiu Yang (yang.yujiu@sz.tsinghua.edu.cn) and Ngai Wong (nwong@eee.hku.hk).} \
Zhe Zhao \affmark[$\spadesuit$] \ 
Jiahao Wang\affmark[$\diamondsuit$] \\
\textbf{Xingyu Bai}\affmark[$\clubsuit$] \ \textbf{Lei Wang}\affmark[$\heartsuit$] \
\textbf{Ngai Wong}\affmark[$\diamondsuit$] \
\textbf{Yujiu Yang}\affmark[$\clubsuit$]
\\
\affmark[$\diamondsuit$]The University of Hong Kong \ \affmark[$\clubsuit$]Tsinghua University \\
\affmark[$\heartsuit$]Ping An Technology \
\affmark[$\spadesuit$]Tencent AI Lab
\\
{\tt takiwu@connect.hku.hk} 
% \ {\tt yang.yujiu@sz.tsinghua.edu.cn} \ {\tt nwong@eee.hku.hk}
}



\begin{document}
\maketitle

\begin{abstract}
% 15--250 words.
Distilling high-accuracy Graph Neural Networks~(GNNs) to low-latency multilayer perceptrons~(MLPs) on graph tasks has become a hot research topic. 
However, conventional MLP learning relies almost exclusively on graph nodes and fails to effectively capture the graph structural information. 
Previous methods address this issue by processing graph edges into extra inputs for MLPs, but such graph structures may be unavailable for various scenarios. 
To this end, we propose Prototype-Guided Knowledge Distillation~(PGKD), which does not require graph edges~(edge-free setting) yet learns structure-aware MLPs. 
Our insight is to distill graph structural information from GNNs. 
Specifically, we first employ the class prototypes to analyze the impact of graph structures on GNN teachers, and then design two losses to distill such information from GNNs to MLPs. 
Experimental results on popular graph benchmarks demonstrate the effectiveness and robustness of the proposed PGKD.
\end{abstract}

The advance of Pre-trained Language Models (PLMs) like GPT-3 \cite{brown2020language} and LLaMA \cite{DBLP:journals/corr/abs-2302-13971} has substantially improved the performance of deep neural networks across a variety of Natural Language Processing (NLP) tasks. Various language models, based on the Transformer \cite{vaswani2017attention} architecture,  have been proposed, leading to state-of-the-art (SOTA) performance on the fundamental discrimination tasks. These models are first trained with self-supervised training objectives (e.g., predicting masked tokens according to surrounding tokens) on massive unlabeled text data, then fine-tuned on annotated data to adapt to downstream tasks of interest.  However, annotated data is usually limited to a wide range of downstream tasks, which results in overfitting and a lack of generalization to unseen data.

One straightforward way to deal with this data scarcity problem is data augmentation , and incorporating generative models to perform data augmentation has been widely adopted recently . Despite its popularity, the generated text can easily deviate from the real data distribution without exploiting any of the signals passed back from the discrimination task. In previous studies, generative data augmentation and discrimination have been well studied as separate problems, but it is less clear how these two can be leveraged in one framework and how their performances can be improved simultaneously. \looseness=-1

Generative Adversarial Networks (GANs) \cite{https://doi.org/10.48550/arxiv.1406.2661} are good attempts to couple generative and discriminative models in an adversarial manner, where a two-player minimax game between learners is carefully crafted. GANs have achieved tremendous success in domains such as image generation , and related studies have also shown their effectiveness in semi-supervised learning. However,  in the text field, GANs are difficult to train, most training objectives work well for only one model, either the discriminator or the generator, so rarely both learners can be optimal at the same time. This essentially arises from the adversarial nature of GANs, that during the process, optimizing one learner can easily destroy the learning ability of the other, making GANs fail to converge.

Another limitation of simultaneously optimizing the generator and the discriminator comes from the discrete nature of text in NLP, as no gradient propagation can be done from discriminators to generators. One theoretically sound attempt is to use reinforcement learning (RL), but the sparsity and the high variance of the rewards in NLP make the training particularly unstable \cite{caccia2019language}. 

To address these shortcomings, we novelly introduce a self-consistent learning framework based on one generator and one discriminator: the generator and the discriminator are alternately trained by way of cooperation instead of competition, and the selected samples are used as the medium to pass the feedback signal from the discriminator. Specifically, in each round of training, the samples generated by the generator are synthetically labeled by the discriminator, and then only part of them would be selected based on dynamic thresholds and used for the training of the discriminator and the generator in the next round. Several benefits can be discovered from this cooperative training process. First, a closed-loop form of cooperation can be established so that we can get the optimal generator and discriminator at the same time. Second, this framework helps improve the generation quality while ensuring the domain specificity of generator, which in turn contributes to training. Third, a steady stream of diverse synthetic samples can be added to the training in each round and lead to continuous improvement of the performance of all learners. Finally, we can start the training with only domain-related corpus and obtain strong results, while these data can be easily sampled with little cost or supervision. Also, the performance on labeled datasets can be further boosted based on the strong baselines. As an example to demonstrate the effectiveness of our framework in the text field, we examine it on four downstream text generation benchmarks, including AFQMC, CHIP-STS, QQP, and MRPC. The experiments show that our method significantly improves over standalone state-of-the-art discriminative models on zero-shot and full-data settings.

Our contributions are summarized as follows,

$\bullet$ We propose a self-consistent learning framework in the text field that incorporates the generator and the discriminator, in which both achieve remarkable performance gains simultaneously.

$\bullet$ We propose a dynamic selection mechanism such that cooperation between the generator and the discriminator drives the convergence to reach their scoring consensus.

$\bullet$ Experimental results show that the generator in our framework can continuously adjust its generation samples based on the performance of downstream tasks, while the discriminator can outperform the strong baselines.



\section{Related Work}\label{sec:related}

Grasping is a fundamental problem for robotic manipulation and has been extensively studied. Most work focuses on parallel-jaw grippers \cite{DBLP:conf/cvpr/FangWGL20,jiang2011efficient,DBLP:conf/iccv/MousavianEF19,DBLP:conf/icra/MuraliMEPF20,DBLP:conf/icra/SundermeyerMTF21}  due to their simplicity, low DoFs, and computational efficiency. However, parallel-jaw grippers are less efficient and less reliable for manipulating arbitrary-shaped objects. To achieve user-friendly interaction, multi-finger robotic hands and dexterous grasping remain a hot research topic in the field of robotic manipulation~\cite{rimon2019mechanics}. This research can be briefly divided into two categories: the traditional analytical sampling-based method and the data-driven method.

\textbf{Traditional analytical sampling-based methods}\cite{ciocarlie2007dexterous,DBLP:conf/icra/GoldfederALP07,DBLP:conf/iros/HangSK14,DBLP:conf/icra/MillerKCA03,DBLP:conf/icra/PelossofMAJ04} sampled various grasp candidates and evaluated them based on certain metrics considering the physical properties of objects such as wrench space~\cite{DBLP:conf/icra/BorstFH04}. In general, both the object model and environment are assumed to be known in advance~\cite{DBLP:journals/ram/MillerA04}. Eigengrasp~\cite{ciocarlie2007dexterous} reduced the dimensions of grasp search space by performing principal component analysis (PCA) on grasping pose and configuration data. Although the reduction increases the efficiency of generating grasps, the search space of the random sampling process for grasps is still very huge. As a result, these sampling-based methods are less efficient in practical use.

\textbf{Data-driven methods} fall into one of two primary types.
The one is an extension of the traditional sampling-based method~\cite{DBLP:conf/iros/VarleyWWA15,DBLP:conf/icra/BorstFH04}. Instead of computing physical metrics, this method directly estimates grasp quality metrics from trained deep models. The grasp success rate can be greatly improved since traditional metrics cannot be computed accurately from an incomplete view of a novel object without any contact feedback. However, they are still dependent on known object models and exhibit the problem of huge sampling and search space.
%
The other data-driven method is performed in an end-to-end manner~\cite{DBLP:conf/iros/HangSK14,DBLP:conf/rss/LiuP0GM20,DBLP:journals/corr/abs-1908-04293,DBLP:conf/iros/LiuP0GM19,DBLP:conf/icra/KapplerBS15,DBLP:conf/iros/VarleyWWA15,mahler2017dex}. Specifically, this method takes the image or point cloud data of a grasped object as input and outputs a high-quality grasp. These approaches are able to effectively generate grasps and are robust to unknown objects. However, many can only handle a single object. Grasping may often fail due to the potential collision between the gripper and the environment.
%
Some recent work~\cite{DBLP:conf/icra/LiWL0LZ22,DBLP:conf/icra/LundellCLVWRMK21,DBLP:journals/corr/abs-2103-04783} predicts
collision-free \mbox{6-DoF} grasping in clutter using multi-finger grippers. They only classify the grasp types and do not take into account of the properties of multi-finger grasps. Our approach considers the gripper's physical structure and does not rely on the grasp types. Using a novel grasping representation and an end-to-end deep neural network based on contacts, our approach significantly reduces the search space for grasping and can generate reliable grasp poses.

\section{Preliminaries}

% \subsection{Task Formulation}
\paragraph{Notations.} Let $\mathcal{G}=(\mathcal{V},\mathcal{E})$ denote a graph, where $\mathcal{V}$ stands for all $N$ nodes with features $\mathbf{X} \in \mathbb{R}^{N \times D}$ and $\mathcal{E}$ stands for all edges.
We represent edges with an adjacency matrix $\mathbf{A}$, and $A_{u,v}=1$ if edge $(u,v) \in \mathcal{E}$ or be 0 otherwise.
For node classification task, the target is $\mathbf{Y} \in \mathbb{R}^{N \times K}$, where row $y_{v} \in R^{K}$ denotes the $K$-dim one-hot label for node $v$.
We adopt superscript $^L$ for labeled nodes~(i.e. $\mathcal{V}^{L}$, $\mathbf{X}^{L}$, and $\mathbf{Y}^{L}$) and superscript $^U$ for the rest unlabeled nodes~(i.e. $\mathcal{V}^{U}$, $\mathbf{X}^{U}$, and $\mathbf{Y}^{U}$).

\paragraph{Graph Neural Network.} Most GNNs follow the message-passing framework, where the representation $\mathbf{h}_v$ of node $v$ is updated by aggregating messages from its neighbors $\mathcal{N}_v$.
For the $l$-th layer, $\mathbf{h}^{l}_v$ is obtained from the previous layer's representations of its neighbors as follows:
\begin{equation}
    % \mathbf{h}^{l}_v = \text{UP}(\text{AG}(\{ h^{l-1}_{u}: u \in  \mathcal{N}_v \}), \mathbf{h}^{l-1}_v) ,
    h^{(l)}_{N(v)}=\text{AGGR}(\{ h^{l-1}_{u}: u \in  \mathcal{N}_v \})
\end{equation}
\begin{equation}
    h^{(l)}_{v} = \text{UPDATE}(h^{(l)}_{N(v)}, h^{l-1}_{v}),
\end{equation}
where AGGR and UPDATE denote the aggregate and update operations, respectively.

\paragraph{Transductive vs Inductive.} 
\label{setting_intro}
There are two setting for graph learning: transductive and inductive.
For transductive setting, models can utilize all node features and graph edges.
For inductive setting, we split the unlabeled data into disjoint inductive subset and observed subset~(i.e. $\mathcal{V}^U=\mathcal{V}^U_{obs} \cup \mathcal{V}^U_{ind}$ and $\mathcal{V}^U_{obs} \cap \mathcal{V}^U_{ind} = \emptyset$).
The edges between $\mathcal{V}^U_{obs}$ and $\mathcal{V}^U_{ind}$ are preserved~(cf. Table \ref{tab:trans_ind}).

\begin{table}[t]

\centering
%\tableindent 
\renewcommand\arraystretch{1.4}


\resizebox{\columnwidth}{!}
{%
\begin{tabular}{lllll}
\hline
\multicolumn{2}{c}{\textbf{Model Setting}}                & \textbf{Train} & \textbf{Test} & \textbf{KD} \\ \hline
\multicolumn{1}{l}{\multirow{2}{*}{GNN}} & \textit{tran} & $(\mathbf{X}, \mathcal{G}, \mathbf{Y}^{L})$     & $(\mathbf{X}^{U}, \mathcal{G}, \mathbf{Y}^{U})$    & $\mathbf{H}$  \\ 
\multicolumn{1}{l}{}                     & \textit{ind}   & $(\mathbf{X}^{L}, \mathcal{G}_{obs}, \mathbf{X}^{U}_{obs}, \mathbf{Y}^{L})$     & $(\mathbf{X}^{U}_{ind}, \mathbf{Y}^{U}_{ind})$    & $\mathbf{H}^{L} \cup \mathbf{H}^{U}_{obs}$ \\ \hline
\multicolumn{2}{c}{MLP}                          & $(\mathbf{X}^{L}, \mathbf{Y}^{L})$    &  $(\mathbf{X}^{U}, \mathbf{Y}^{U})$   & -  \\ \hline
\end{tabular}
}
\caption{
The inputs for GNNs and MLPs in different settings: 
transductive (\textit{tran}) and inductive (\textit{ind}).
KD denotes the employed features for knowledge distillation.
$\mathbf{H}$ denotes the graph nodes representations from GNN teachers.
}
\label{tab:trans_ind}

\end{table}


\begin{figure*}[t]
    \centering
    \includegraphics[width=\linewidth]{fig/ModelStructure.pdf}
    \caption{Overall architecture of \name. The left part represents different modality-specific encoders to extract latent features and the multimodal fusion module to integrate multimodal representations. The right part represents the contextual relational model decoders to get the similarity score and the decision fusion module to make the final prediction on all modalities.}
    \label{fig:model}
\end{figure*}

\section{Methodology}

Formally, a knowledge graph is defined as $\mathcal{G} = \langle \mathcal{E}, \mathcal{R}, \mathcal{T} \rangle$, where $\mathcal{E}$ and $\mathcal{R}$ indicate sets of entities and relations, respectively. 
$\mathcal{T} = \{(h, r, t) | h, t \in \mathcal{E}, r \in \mathcal{R}\}$ represents relational triples of the KG.
In multimodal KGs, each entity in KGs is represented by multiple features from different modalities.
Here, we define the set of modalities $\mathcal{K} = \{s, v, t, m\}$ where $s, v, t, m$ denote structural, visual, textual and multimodal modality, respectively.
Due to the complexity of real-world knowledge, it is almost impossible to take all the triples into account.
Therefore, given a well-formulated KG, the \emph{Link Prediction} task aims at predicting missing links between entities.
Specifically, link prediction models expect to learn a score function of relational triples to estimate the likelihood of a triple, which is always formulated as $\psi : \mathcal{E} \times \mathcal{R} \times \mathcal{E} \to \mathbb{R}$.


\subsection{Overall Architecture}

In order to fully exploit the complicated interaction between different modalities, we propose a two-stage fusion model instead of simply considering the multimodal information separately in a unified vector space.
As shown in Figure~\ref{fig:model}, \name consists of four key components:
\begin{itemize}[leftmargin=*]
	\item[1] The Modality-Specific Encoders are used for extracting structural, visual and textual features as the input of multimodal fusion stage.
	\item[2] The Multimodal Fusion Module, which is the first fusion stage, effectively models bilinear interactions between different modalities based on \textit{Tucker} decomposition and contrastive learning.
	\item[3] The Contextual Relational Model calculates the similarity of contextual entity representations to formulate triple scores as modality-specific predictions for decision fusion stage.
	\item[4]  The Decision Fusion Module, which is the second fusion stage, takes all the similarity scores from structural, visual, textual and multimodal models into account to make the final prediction.
\end{itemize}

\subsection{Modality-Specific Encoders}
In this subsection, we first introduce the pre-trained encoders used for different modalities.
These encoders are not fine-tuned during training and we treat them as fixed feature extractors to obtain the modality-specific entity representations.
Note that \name is a general framework and it is straightforward to replace them with other up-to-date encoders or add ones for new modalities into \name.

\subsubsection{Structural Encoder}

From the most basic view, the structural information of KG, we employ a Graph Attention Network (GAT)\footnote{https://github.com/Diego999/pyGAT}~\cite{DBLP:conf/iclr/VelickovicCCRLB18} with TransE loss.

Specifically, our GAT encoder takes L1 distance of neighbor aggregated representations as energy function of triples, which is $E(h, r, t) = ||\mathbf{h}+\mathbf{r}-\mathbf{t}||$.
In the training process, we minimize the following Hinge loss~\eqref{eq-gat-loss}:
\begin{equation}\label{eq-gat-loss}
    \begin{split}
        \mathcal{L}_{GAT} = & \sum_{(h,r,t) \in \mathcal{T}}\sum_{(h', r, t') \in \mathcal{T'}} \mathrm{max} \{0,  \\
        &\gamma + E(h,r,t) - E(h',r,t')\}
    \end{split}
\end{equation}
where $\gamma$ is margin hyper-parameter and $\mathcal{T'}$ denotes set of negative triples derived from $\mathcal{T}$. 
$\mathcal{T'}$ is created by randomly replacing head or tail entities of triples in $\mathcal{T}$, which is~\eqref{eq-gat-neg}:
\begin{equation}\label{eq-gat-neg}
    \mathcal{T'} = \{(h',r,t)|h' \in \mathcal{E} \backslash h\} \cup \{(h,r,t')|t' \in \mathcal{E} \backslash t\}
\end{equation}

\subsubsection{Visual Encoder} 
Visual features are greatly expressive while providing different views of knowledge from traditional KGs. 
To effectively extract visual features, we utilize VGG16\footnote{https://github.com/machrisaa/tensorflow-vgg} pre-trained on \textit{ImageNet}\footnote{https://image-net.org/} to get image embeddings of corresponding entities following~\cite{DBLP:conf/esws/LiuLGNOR19}.
Specifically, we take outputs of the last hidden layer before softmax operation as visual features, which are 4096-dimensional vectors.

\subsubsection{Textual Encoder} 
Entity descriptions contain much richer but more complex knowledge than pure KGs.
To fully extract the complex knowledge, we employ BERT~\cite{DBLP:conf/naacl/DevlinCLT19} as the textual encoder, which is very expressive to get description embeddings of corresponding entities.
The textual features are 768-dimensional vectors, i.e., pooled outputs of pre-trained BERT-Base model\footnote{https://github.com/huggingface/transformers}.

\subsection{Multimodal Fusion}
The multimodal fusion stage aims to effectively get multimodal representations, which fully capture the complex interactions between different modalities.
Many existing multimodal fusion methods have achieved promising results in many tasks like VQA (Visual Question Answering).
However, most of them aim at finding the commonality to get more precise representations by modality projecting~\cite{DBLP:conf/nips/FromeCSBDRM13,DBLP:conf/aaai/CollellZM17} or cross-modal attention~\cite{DBLP:conf/aaai/PerezSVDC18}.
These types of methods will suffer from the loss of unique information in different modalities and can not achieve sufficient interaction between modalities.
To this end, we propose to employ the bilinear models, which have a strong ability to realize full parameters interaction as the cornerstone to perform the fusion of multimodal information.
Specifically, we extend the \textit{Tucker} decomposition, which decomposes the tensor into a core tensor transformed by a matrix along with each mode to 4-mode factors as expressed in Equation~\eqref{eq-tucker}:
\begin{equation}\label{eq-tucker}
    \mathcal{P} = (((\mathcal{P}_c \times \mathbf{M}_s) \times \mathbf{M}_v) \times \mathbf{M}_t) \times \mathbf{M}_d
\end{equation}
where $\mathbf{M}_s \in \mathbb{R}^{d_s \times t_s}$, $\mathbf{M}_v \in \mathbb{R}^{d_v \times t_v}$, $\mathbf{M}_t \in \mathbb{R}^{d_t \times t_t}$,  $\mathbf{M}_d \in \mathbb{R}^{\mathcal{D} \times t_d}$ denotes transformation matrix and $\mathcal{P}_c \in \mathbb{R}^{t_s \times t_v \times t_t \times t_d}$ denotes a smaller core tensor.

In such a situation, entity embeddings are first projected into a low-dimensional space and then fused with the core tensor $\mathcal{P}_c$.
Following~\cite{DBLP:conf/iccv/Ben-younesCCT17}, we further reduce the computation complexity by decomposing the core tensor $\mathcal{P}_c$ to merge representations of all modalities into a unified space with element-wise product.
The detailed calculation process is expressed as Equation~\eqref{eq-fusion}:
\begin{equation}\label{eq-fusion}
    \mathbf{e}_m = \tilde{\mathbf{e}}_s^\mathsf{T} \mathbf{M}_d^s * \tilde{\mathbf{e}}_v^\mathsf{T} \mathbf{M}_d^v * \tilde{\mathbf{e}}_t^\mathsf{T} \mathbf{M}_d^t
\end{equation}
where $\tilde{\mathbf{e}}_k = \mathrm{ReLU}(\mathbf{e}_k\mathbf{M}_k) \in \mathbb{R}^{t_k}$ denotes latent representations and $\mathbf{e}_k \in \mathbb{R}^{d_k}$ is the original embedding representations and $\mathbf{M}_d^k \in \mathbb{R}^{t_k \times t_d}$ is decomposed transformation matrix for each modality $k \in \{s, v, t\}$.

However, the multimodal bilinear fusion has no bound limitation while the gradient produced by the final prediction result can only implicitly guide parameter learning.
To alleviate this problem, we add constraints to limit the correlation between different modality representations of the same entity to be stronger.
Therefore, we further leverage contrastive learning~\cite{DBLP:conf/icml/ChenK0H20,DBLP:conf/nips/LiSGJXH21,DBLP:conf/cvpr/Yuan0K0WMKF21} between different entities and modalities as an additional learning objective for regularization.
In the settings of contrastive learning, we take the pairs of representations of the same entity of different modalities as positive samples and the pairs of representations of different entities as negative samples.
As shown in Figure~\ref{fig:cl}, we aim at limiting the distance of negative samples to be larger than positive samples to enhance multimodal fusion, which is:
\begin{equation}
    d(f(x), f(x^+)) << d(f(x), f(x^-))
\end{equation}
where $d(\cdot, \cdot)$ denotes the distance measure and $f(\cdot)$ denotes the embedding function. The superscript $+, -$ represent the positive and negative samples, respectively.

\begin{figure}
    \centering
    \includegraphics[width=\linewidth]{fig/ContrastiveLearning.pdf}
    \caption{Example of multimodal contrastive learning. The distance between the representations of the same entity in different modalities is minimized, while the distance between the representations of different entities is maximized.}
    \label{fig:cl}
\end{figure}

Specifically, we randomly sample $N$ entities from the entity set as a minibatch and define contrastive learning loss upon it.
The positive pairs are naturally obtained with the same entities while the negative pairs are constructed by negative sharing~\cite{DBLP:conf/kdd/ChenSSH17} of all other entities.
We take the latent representations $\tilde{\mathbf{e}}_k = \mathrm{ReLU}(\mathbf{e}_k\mathbf{M}_k) \in \mathbb{R}^{t_k}$ and leverage cosine similarity $d(u, v) = - \mathbf{u}^\mathsf{T}\mathbf{v}/||\mathbf{u}||\mathbf{v}||$ as distance measure.
Then we have the following contrastive loss function for each entity $i$:
\begin{equation}\label{eq-cl}
    \mathcal{L}_{CLi} = \frac{1}{3N} \sum_{p,q \in \mathcal{M}} \sum_{j=1}^N  d(e_i^{p}, e_i^{q}) - d(e_i^{p}, e_j^{q}) + 2
\end{equation}
where $\mathcal{M} = \{(s, v), (s, t), (v, t)\}$ is set of modality pairs.

\subsection{Contextual Relational Model}
After obtaining representations of each modality and multimodal, we then design a contextual relational model, which takes relations in triples as contextual information for scoring, to get the predictions.
Note that this relational model can be easily replaced by any scoring function like TransE.

Due to the variety and complexity of relations in KGs, we argue that improving the degree of parameter interaction~\cite{DBLP:conf/aaai/VashishthSNAT20} is crucial for better modeling the relational triples.
The degree of parameter interaction means the calculation ratio of each parameter to all other parameters. 
For example, dot product could achieve $1/d$ degree while cross product could achieve $(d-1)/d$ degree.
Based on this assumption, we propose to use bilinear outer product between entity and relation embeddings to incorporate contextual information into entity representations.
Instead of taking relations as input as in previous studies, our contextual relational model utilizes relations to provide context in the transformation matrix of entity embeddings.
Then, entity embeddings are projected using the contextual transformation matrix to get \emph{contextual embeddings}, which are used for calculating similarity with all candidate entities.
The learning objective is to minimize the binary cross-entropy loss.
For each modality $k \in \mathcal{K}$, the computation details are shown as Equation~\eqref{eq-crm} to Equation~\eqref{eq-loss}:
\begin{gather}
    \hat{\mathbf{e}}_k = \mathbf{e}_k^\mathsf{T}\mathbf{W}_k^r  + \mathbf{b} = \mathbf{e}_k^\mathsf{T}\mathbf{W}_k\mathbf{r} + \mathbf{b}_k \label{eq-crm} \\
    \mathbf{y}_k = \sigma(\mathrm{cosine}(\mathbf{e}_k, \hat{\mathbf{e}}_k)) = \sigma (\frac{\mathbf{e}_k \cdot \hat{\mathbf{e}}_k}   
    {|\mathbf{e}_k| |\hat{\mathbf{e}}_k|}) \label{eq-sim} \\
    \mathcal{L}_k = -\frac{1}{N} \sum_{i=1}^N (t_i \cdot \mathrm{log}(y_{i,k})+(1-t_i) \cdot \mathrm{log}(1-y_{i,k})) \label{eq-loss}
\end{gather}
where $\mathbf{e}_k$ and $\hat{\mathbf{e}}_k$ are original and contextual entity embeddings respectively;
$\mathbf{W}_k^r = \mathbf{W}_k \mathbf{r}$ denotes contextual transformation matrix which is obtained by matrix multiplication of weight matrix $\mathbf{W}_k$ and relation vectors $\mathbf{r}$ while $\mathbf{b}_k$ is a bias vector;
$\sigma$ is sigmoid function and $\mathbf{y}_k = [y_{1,k},y_{2,k},...,y_{N,k}]$ is final prediction of modality $k$.

\subsection{Decision Fusion}
Existing multimodal approaches mainly focus on projecting different modality representations into a unified space and predicting with commonality between modalities, which will fail to preserve the modality-specific knowledge.
We alleviate this problem in the decision fusion stage by joint learning and combining predictions of different modalities to further leverage the complementarity.

Under the multimodal settings, we assign different contextual relational models for each modality and utilize their own results for training in different views.
Recall the contrastive learning loss in Equation~\eqref{eq-cl}, the overall training objective is to minimize the joint loss shown in Equation~\eqref{eq-mmloss}:
\begin{equation}\label{eq-mmloss}
    \mathcal{L}_{Joint} = \gamma_s \mathcal{L}_s + \gamma_v \mathcal{L}_v + \gamma_t \mathcal{L}_t + \gamma_m \mathcal{L}_{m} + \mathcal{L}_{CL}
\end{equation}
where $\mathcal{L}_k$ denotes binary cross entropy loss for modality $k$ as Equation~\eqref{eq-loss} and $\gamma_k$ is a learned weight parameter.

\begin{algorithm}[t]
\caption{Optimization Algorithm.}\label{alg:optim}
\begin{algorithmic}[1]
\STATE \textbf{Input:} Multimodal Knowledge Graph $\mathcal{G}$
\STATE \textbf{Output:} Trained Model $\mathcal{M}$
\STATE Pre-train structural encoder GAT on $\mathcal{G}$ with the loss in Equation(1)
\STATE Obtain pre-trained visual encoder VGG16 and textual encoder BERT-base
\STATE Initialize the entity embeddings $\mathbf{E}_s, \mathbf{E}_v, \mathbf{E}_t$ in $\mathcal{M}$ with the outputs of pre-trained encoders
\WHILE{not converge}
    \STATE Sample a batch of entities from $\mathcal{G}$
    \FOR{Entity $e$ in batch}
    \STATE Obtain the structural, visual, textual embeddings $\mathbf{e}_s, \mathbf{e}_v, \mathbf{e}_t$ of entity $e$
    \STATE Compute the multimodal fused embeddings $\mathbf{e}_m$ of entity $e$ with Equation (4)
    \STATE Compute the contrastive learning loss $\mathcal{L}_{CL}$ with Equation (6)
    \STATE Compute the loss $\mathcal{L}_s, \mathcal{L}_v, \mathcal{L}_t, \mathcal{L}_m$ with modality-specific scorers via Equation (7) - Equation (9)
    \STATE Compute the joint loss $\mathcal{L}_{Joint}$ with the above losses $\mathcal{L}_s, \mathcal{L}_v, \mathcal{L}_t, \mathcal{L}_m, \mathcal{L}_{CL}$ via Equation (10)
    \STATE Update model parameters of $\mathcal{M}$ by minimizing $\mathcal{L}_{Joint}$
    \ENDFOR
\ENDWHILE
\RETURN $\mathcal{M}$
\end{algorithmic}
\end{algorithm}

To better illustrate the whole training process of \name, we describe it via the pseudo-code of the optimization algorithm.
As shown in Algorithm~\ref{alg:optim}, we first obtain the pre-trained encoders of structural, visual and textual and utilize them for entity embeddings (line 3-5).
Since the pre-trained models are much larger and more complex than \name, they are not fine-tuned and their outputs are directly used as inputs of \name.
The multimodal embeddings are obtained by multimodal fusion while contrastive learning is applied to further enhance the fusion stage (line 9-11).
During training, each modality delivers its own prediction and loss via the modality-specific scorers (line 12), and then the joint prediction and loss are computed based on all modalities including multimodal ones (line 14).

For inference, we propose to jointly consider the predictions of each modality as well as multimodal ones.
Specifically, the overall predictions are shown in Equation~\eqref{eq-df}:
\begin{equation}\label{eq-df}
    \mathbf{y}_{Joint} = \frac{\gamma_s \mathbf{y}_s + \gamma_v \mathbf{y}_v + \gamma_t \mathbf{y}_t + \gamma_m \mathbf{y}_m} {\gamma_s + \gamma_v + \gamma_t + \gamma_m}
\end{equation}
where $\gamma_k$ denotes weight for modality $k$ as same as Equation~\eqref{eq-mmloss} while the values in $\mathbf{y}$ are in [0, 1].



% \clearpage
\section{Experimental Results}
In this section, we validate the effectiveness of our proposal. We first introduce datasets, metrics and implementation details involved in our evaluation. Then, we compare \netname{} with state-of-the-art methods, conduct an ablation study on our model and, finally, discuss its limitations.


\begin{table*}[htbp] \scriptsize
	\renewcommand\tabcolsep{2.3pt} 
	\centering
	\scalebox{0.85}{
	\begin{tabular}{@{}ccccccccccccccccc@{}}
		\toprule
		 Dataset & Scale & Metrics & GF~\cite{he2010guided} & SD~\cite{ham2017robust}  & GSRPT~\cite{lutio2019guided} & MSG~\cite{hui2016depth} & DKN~\cite{kim2021deformable} & FDKN~\cite{kim2021deformable} & PMBANet~\cite{ye2020pmbanet} & FDSR~\cite{he2021towards} & JIIF~\cite{tang2021joint} & DCTNet~\cite{zhao2022discrete} & LGR~\cite{de2022learning} & DADA~\cite{metzger2022guided} & DSR-EI & DSR-EI$^+$ \\ \midrule
		\multirow{6}{*}{\rotatebox[origin=l]{90}{\scriptsize \textbf{Middlebury}}} & \multirow{2}{*}{$4\times$} 
		& MSE & 33.3 & 24.9 & 39.8 & 4.13 & 4.29 & 3.60 & 4.72 & 7.72 & 2.70 & 5.00 & 3.04 & \bronze{2.58} & \gold{2.46} & \silver{2.56} \\
		& & MAE & 1.27 & 0.46 & 0.79 & 0.22 & 0.18 & 0.16 & 0.25 & 0.35 & \bronze{0.11} & 0.24 & 0.13 & \bronze{0.11} & \silver{0.08} & \gold{0.07} \\ \cline{2-17}
		& \multirow{2}{*}{$8\times$} 
		& MSE & 40.5 & 82.5 & 32.7 & 10.5 & 11.2 & 10.4 & 9.48 & 23.2 & 8.01 & 15.1 & 7.26 & \silver{5.68} & \bronze{6.20} & \gold{5.13} \\
		& & MAE & 1.49 & 0.86 & 0.82 & 0.43 & 0.38 & 0.37 & 0.38 & 0.69 & 0.27 & 0.57 & 0.24 & \bronze{0.20} & \gold{0.18} & \gold{0.18} \\ \cline{2-17}
		& \multirow{2}{*}{$16\times$} 
		& MSE & 67.4 & 511 & 41.5 & 34.2 & 47.6 & 38.5 & 30.6 & 55.4 & 37.5 & 52.3 & 24.7 & \silver{16.3} & \gold{15.8} & \bronze{16.6}  \\
		& & MAE & 2.21 & 1.73 & 1.24 & 1.06 & 1.42 & 1.18 & 0.89 & 1.51 & 0.98 & 1.50 & 0.67 & \bronze{0.48} & \silver{0.47} & \gold{0.40} \\ \hline\hline
	    % middlebury end
		\multirow{6}{*}{\rotatebox[origin=l]{90}{\scriptsize \textbf{NYUv2}}} & \multirow{2}{*}{$4\times$}
		& MSE & 114 & 36.0 & 112 & 6.85 & 11.4 & 9.07 & 10.8 & 10.1 & \bronze{3.28} & 3.63 & 6.45 & 4.83 & \silver{2.82} & \gold{2.75}\\
		& & MAE & 3.91 & 1.31 & 3.61 & 0.81 & 1.03 & 0.85 & 0.93 & 0.94 & \bronze{0.52} & 0.68 & 0.73 & 0.64 & \silver{0.49} & \gold{0.47}\\ \cline{2-17}
		& \multirow{2}{*}{$8\times$} 
		& MSE & 142 & 105 & 122 & 24.1 & 29.8 & 29.9 & 17.2 & 19.5 & \bronze{15.2} & 20.9 & 19.6 & 16.6 & \gold{11.8} & \gold{11.8}\\
		& & MAE & 4.47 & 2.57 & 3.86 & 1.66 & 1.82 & 1.80 & 1.38 & 1.38 & \bronze{1.29} & 1.79 & 1.42 & 1.30 & \silver{1.12} & \gold{1.09}\\ \cline{2-17}
		& \multirow{2}{*}{$16\times$} 
		& MSE & 249 & 533 & 219 & 84.5 & 115 & 113 & 84.9 & 86.4 & 59.9 & 77.0 & 67.5 & \bronze{59.0} & \silver{47.8} & \gold{47.1} \\
		& & MAE & 6.34 & 5.07 & 5.40 & 3.35 & 4.01 & 3.95 & 3.26 & 3.35 & 2.81 & 3.61 & 2.90 & \bronze{2.64} & \silver{2.48} & \gold{2.40}\\ \hline\hline
		% NYU end
		\multirow{6}{*}{\rotatebox[origin=l]{90}{\scriptsize \textbf{DIML}}} & \multirow{2}{*}{$4\times$}
		& MSE & 25.6 & 10.5 & 20.7 & 1.73 & 3.47 & 2.20 & 3.05 & 2.75 & \bronze{1.19} & 2.09 & 1.68 & 1.33 & \silver{0.70} & \gold{0.65} \\
		& & MAE & 1.45 & 0.40 & 1.15 & 0.22 & 0.33 & 0.23 & 0.31 & 0.29 & \bronze{0.16} & 0.31 & 0.20 & 0.17 & \silver{0.13} & \gold{0.12} \\ \cline{2-17}
		& \multirow{2}{*}{$8\times$} 
		& MSE & 34.1 & 44.9 & 23.0 & 4.13 & 5.47 & 5.95 & 5.87 & 8.40 & 3.65 & 7.08 & 3.51 & \bronze{2.93} & \silver{2.12} & \gold{2.09} \\
		& & MAE & 1.77 & 0.83 & 1.26 & 0.40 & 0.45 & 0.47 & 0.47 & 0.66 & 0.32 & 0.65 & 0.31 & \bronze{0.28} & \gold{0.22} & \gold{0.22} \\ \cline{2-17}
		& \multirow{2}{*}{$16\times$} 
		& MSE & 66.3 & 41.1 & 39.3 & 13.0 & 19.3 & 20.8 & 13.8 & 32.9 & 11.7 & 23.4 & 9.45 & \bronze{7.61} & \gold{6.29} & \silver{6.31} \\
		& & MAE & 2.74 & 1.91 & 1.78 & 0.93 & 1.20 & 1.24 & 0.87 & 1.66 & 0.81 & 1.75 & 0.68 & \bronze{0.59} & \silver{0.52} & \gold{0.50} \\
		% DIML end
    \bottomrule
	\end{tabular}}
    \vspace{-0.3cm}
	\caption{\textbf{Results on Middlebury, NYUv2 and DIML datasets.} The lower the MSE and MAE, the better.}
	\label{sota_comparison_mid_nyu_diml}
\end{table*}



\begin{table*}[t] \footnotesize
	\renewcommand\tabcolsep{1.5pt} 
	\centering
	\scalebox{0.85}{
	\begin{tabular}{@{}ccccccccccccccccc@{}}
		\toprule
		 Scale & SDF~\cite{li2016deep} & SVLRM~\cite{pan2019spatially} & DJF~\cite{li2016deep} & DJFR~\cite{li2019joint} & PAC~\cite{su2019pixel} & CUNet~\cite{deng2020deep} & FDKN~\cite{kim2021deformable} & DKN~\cite{kim2021deformable} & FDSR~\cite{he2021towards} & DCTNet~\cite{zhao2022discrete} & RSAG~\cite{yuan2023recurrent} & DSR-EI & DSR-EI$^+$ \\ \midrule
		$4\times$ & 2.00 & 3.39 & 3.41 & 3.35 & 1.25 & 1.18 & 1.18 & 1.30 & 1.16 & \bronze{1.07} & 1.14 & \gold{0.91} & \gold{0.91} \\
		$8\times$ & 3.23 & 5.59 & 5.57 & 5.57 & 1.98 & 1.95 & 1.91 & 1.96 & 1.82 & 1.78 & \bronze{1.75} & \gold{1.37} & \silver{1.38} \\
		$16\times$ & 5.16 & 8.28 & 8.15 & 7.99 & 3.49 & 3.45 & 3.41 & 3.42 & 3.06 & 3.18 & \bronze{2.96} & \gold{2.10} & \gold{2.10}  \\
    \bottomrule
	\end{tabular}}
	\vspace{-0.3cm}
	\caption{\textbf{Results on the RGBDD dataset.} We report RMSE, the lower the better.}
	\label{sota_comparison_rgbdd}
\end{table*}



\subsection{Datasets and Metrics}
We evaluate \netname{} on four datasets, compared with existing methods when super-solving depth maps by three different upsampling factors: $4\times,\ 8\times$, and $16\times$. 

\textbf{Middlebury}\cite{scharstein2003high,scharstein2007learning,hirschmuller2007evaluation,scharstein2014high}. We train all learning-based methods using 50 RGB-D images with ground truth from Middlebury 2005, 2006 and 2014 datasets. As in~\cite{de2022learning}, we retain 5 for validation and 5 for testing. 

\textbf{NYUv2}\cite{silberman2012indoor}. It contains 1449 RGB-D images in total. Following \cite{de2022learning}, we randomly split it into 849 RGB-D images for the training set, 300 for the validation set and 300 for the test set. Compared to \cite{ye2020pmbanet,liu2022pdr}, it comes with a validation set to make the comparison fairer.

\textbf{DIML}\cite{kim2016structure,kim2017deep,kim2018deep,cho2021deep} consists of 2 million color images and corresponding depth maps from indoor and outdoor scenes. We adopt the same strategy outlined in \cite{de2022learning}, i.e., considering only the indoor data subset, and use 1440 for training, 169 for validation, and 503 for testing.

\textbf{RGBDD}\cite{he2021towards} is a new real-world dataset for GDSR, which consists of 4811 image pairs. For evaluation, we follow the protocol described in \cite{he2021towards}, using 2215 images (1586 portraits, 380 plants, 249 models) as the training set and 405 images (297 portraits, 68 plants, 40 models) as the test set. 

\textbf{Metrics.} Following \cite{de2022learning}, we compute mean square error (MSE / $cm^2$) and mean absolute error (MAE / $cm$) as metrics on Middlebury, NYUv2 and DIML. For RGBDD, we use root mean square error (RMSE / $cm$) as in \cite{he2021towards}. 

\subsection{Implementation Details}
During training, the HR depth maps and the color images are randomly cropped into $256\times 256$ patches. LR depth patches are generated by bicubic interpolation at $64\times 64$, $32\times 32$, $16\times 16$ resolution for $4\times$, $8\times$ and $16\times$ factors, respectively. We randomly extract about 75K, 168K, 223K and 232K patches from Middlebury, NYUv2, DIML and RGBDD for training. Before being fed to the network, depth maps and images are normalized in the [0, 1] range.

We use Pytorch \cite{paszke2019pytorch} to implement and train \netname{}, on a single Nvidia RTX 3090 GPU. The batch size is set to 4, using Adam as the optimizer. The learning rate is initialized to $1\times 10^{-4}$, then performing a 5-epoch warm-up and cosine annealing. We use random rotation, horizontal/vertical flipping as data augmentation. According to the size of the four datasets, we train our network for 1505, 198, 155 and 109 epochs on Middlebury, NYUv2, DIML and RGBDD, respectively. 
When evaluating results on a specific dataset, we do not perform any pre-training on the others. Following \cite{de2022learning}, testing is performed by processing $256\times256$ patches at a time on Middlebury, NYUv2 and DIML for fairness, while full-resolution images are processed for RGBDD.

\begin{figure*}[t] 
	\centering
	\renewcommand\tabcolsep{1.5pt} 
	\begin{tabular}{cccccccccccc}
	\vspace{-0.1cm}
    \rotatebox[origin=l]{90}{\scriptsize \quad \textbf{Middlebury}} & \includegraphics[height=0.6in]{./figs/sota_comp_middlebury/389/Middlebury_389_img.pdf}
        \hspace{-1.8mm} & \includegraphics[height=0.6in]{./figs/sota_comp_middlebury/389/Middlebury_389_source.pdf}
	\hspace{-1.8mm} &  \includegraphics[height=0.6in]{./figs/sota_comp_middlebury/389/Middlebury_389_GT.pdf}
	\hspace{-1.8mm} & \includegraphics[height=0.6in]{./figs/sota_comp_middlebury/389/Middlebury_389_PMBA.pdf}
	\hspace{-1.8mm} & \includegraphics[height=0.6in]{./figs/sota_comp_middlebury/389/Middlebury_389_FDSR.pdf}
	\hspace{-1.8mm} & \includegraphics[height=0.6in]{./figs/sota_comp_middlebury/389/Middlebury_389_JIIF.pdf}
	\hspace{-1.8mm} & \includegraphics[height=0.6in]{./figs/sota_comp_middlebury/389/Middlebury_389_DCTnet.pdf}
	\hspace{-1.8mm} & \includegraphics[height=0.6in]{./figs/sota_comp_middlebury/389/Middlebury_389_LGR.pdf}
	\hspace{-1.8mm} & \includegraphics[height=0.6in]{./figs/sota_comp_middlebury/389/Middlebury_389_MSS.pdf}
        
        \hspace{-1.8mm} & \includegraphics[height=0.6in]{./figs/sota_comp_middlebury/389/Middlebury_389_ours.pdf}
    \\ \vspace{-0.1cm}
    
    \rotatebox[origin=l]{90}{\scriptsize \quad \textbf{NYUv2}} & \includegraphics[height=0.6in]{./figs/sota_comp_nyu/357/NYU_357_img.pdf}
	\hspace{-1.8mm} & \includegraphics[height=0.6in]{./figs/sota_comp_nyu/357/NYU_357_source.pdf}
	\hspace{-1.8mm} & \includegraphics[height=0.6in]{./figs/sota_comp_nyu/357/NYU_357_GT.pdf}
	\hspace{-1.8mm} & \includegraphics[height=0.6in]{./figs/sota_comp_nyu/357/NYU_357_PMBA.pdf}
	\hspace{-1.8mm} & \includegraphics[height=0.6in]{./figs/sota_comp_nyu/357/NYU_357_FDSR.pdf}
	\hspace{-1.8mm} & \includegraphics[height=0.6in]{./figs/sota_comp_nyu/357/NYU_357_JIIF.pdf}
	\hspace{-1.8mm} & \includegraphics[height=0.6in]{./figs/sota_comp_nyu/357/NYU_357_DCTnet.pdf}
	\hspace{-1.8mm} & \includegraphics[height=0.6in]{./figs/sota_comp_nyu/357/NYU_357_LGR.pdf}
	\hspace{-1.8mm} & \includegraphics[height=0.6in]{./figs/sota_comp_nyu/357/NYU_357_MSS.pdf}
 
	\hspace{-1.8mm} & \includegraphics[height=0.6in]{./figs/sota_comp_nyu/357/NYU_357_ours.pdf}
	\\ 
	
    \rotatebox[origin=l]{90}{\scriptsize \quad \textbf{DIML}} & \includegraphics[height=0.6in]{./figs/sota_comp_diml/856/DIML_856_img.pdf}
	\hspace{-1.8mm} & \includegraphics[height=0.6in]{./figs/sota_comp_diml/856/DIML_856_source.pdf}
	\hspace{-1.8mm} & \includegraphics[height=0.6in]{./figs/sota_comp_diml/856/DIML_856_GT.pdf}
	\hspace{-1.8mm} & \includegraphics[height=0.6in]{./figs/sota_comp_diml/856/DIML_856_PMBA.pdf}
	\hspace{-1.8mm} & \includegraphics[height=0.6in]{./figs/sota_comp_diml/856/DIML_856_FDSR.pdf}
	\hspace{-1.8mm} & \includegraphics[height=0.6in]{./figs/sota_comp_diml/856/DIML_856_JIIF.pdf}
	\hspace{-1.8mm} & \includegraphics[height=0.6in]{./figs/sota_comp_diml/856/DIML_856_DCTnet.pdf}
	\hspace{-1.8mm} & \includegraphics[height=0.6in]{./figs/sota_comp_diml/856/DIML_856_LGR.pdf}
	\hspace{-1.8mm} & \includegraphics[height=0.6in]{./figs/sota_comp_diml/856/DIML_856_MSS.pdf}
 
	\hspace{-1.8mm} & \includegraphics[height=0.6in]{./figs/sota_comp_diml/856/DIML_856_ours.pdf}
 \\
	& \scriptsize \textbf{(a)} RGB & \scriptsize \textbf{(b)} Bicubic & \scriptsize \textbf{(c)} GT & \scriptsize \textbf{(d)} PMBA & \scriptsize \textbf{(e)} FDSR & \scriptsize \textbf{(f)} JIIF & \scriptsize \textbf{(g)} DCTNet & \scriptsize \textbf{(h)} LGR & \scriptsize \textbf{(i)} \netname{} & \scriptsize \textbf{(j)} \netname{} (depth)
	\end{tabular}
    \vspace{-0.3cm}
	\caption{\textbf{Qualitative comparison on Middlebury, NYUv2 and DIML datasets (scaling factor $8\times$).} From left to right: (a) RGB image, (b) Bicubic upsampled depth map, (c) GT; then, error maps achieved by selected methods: (d) PMBA~\cite{ye2020pmbanet}, (e) FDSR~\cite{he2021towards}, (f) JIIF~\cite{tang2021joint}, (g) DCTNet~\cite{zhao2022discrete}, (h) LGR~\cite{de2022learning}; finally, (i) error maps and (j) predictions by \netname.} 
	\label{qualitative}
\end{figure*}


\begin{table*}[htbp] \footnotesize
	\renewcommand\tabcolsep{1.5pt} 
	\centering
	\scalebox{0.85}{
	\begin{tabular}{@{}ccccccccccccccccc@{}}
		\toprule
		 Testing Dataset & Metric & GF\cite{he2010guided} & SD~\cite{ham2017robust}  & GSRPT~\cite{lutio2019guided} & MSG~\cite{hui2016depth} & FDKN~\cite{kim2021deformable} & PMBANet~\cite{ye2020pmbanet} & FDSR~\cite{he2021towards} & JIIF~\cite{tang2021joint} & DCTNet~\cite{zhao2022discrete} & LGR~\cite{de2022learning} & \netname$^+$ \\ \midrule
		\multirow{2}{*}{DIML}
		& MSE & 34.1 & 44.9 & 23.0 & 5.76 & 6.74 & 7.35 & 7.73 & \silver{4.10} & 5.64 & \bronze{4.95} & \gold{3.72} \\
		& MAE & 1.77 & 0.83 & 1.26 & 0.51 & 0.53 & 0.59 & 0.74 & \silver{0.38} & 0.77 & \bronze{0.40} & \gold{0.36} \\ \hline
		\multirow{2}{*}{Middlebury\textit{-HR}}
		& MSE & 40.5 & 82.5 & 32.7 & 11.0 & \bronze{10.0} & \silver{9.62} & 18.4 & 19.3 & 17.5 & \gold{8.25} & 14.6 \\
		& MAE & 1.49 & 0.86 & 0.82 & 0.54 & \silver{0.43} & \bronze{0.46} & 0.73 & 0.74 & 0.77 & \gold{0.35} & 0.54  \\ \hline
		\multirow{2}{*}{Middlebury\textit{-LR}}
		& MSE & 25.6 & 28.8 & 15.8 & 8.89 & 5.54 & 4.16 & 6.92 & 4.40 & 6.96 & 5.94 & \gold{3.44} \\
		& MAE & 2.31 & 2.07 & 1.73 & 1.62 & 0.99 & \silver{0.91} & 1.09 & \bronze{0.92} & 1.15 & 1.11 & \gold{0.87}  \\
        \bottomrule
	\end{tabular}}
	\vspace{-0.3cm}
	\caption{\textbf{Cross-dataset generalization.} All methods are trained on NYUv2 and tested on DIML/Middlebury with factor $8\times$. Middlebury\textit{-HR} is the test set defined in \cite{de2022learning}, Middlebury\textit{-LR} is the one from \cite{tang2021joint}. The lower MSE and MAE, the better. }
	\label{cross-data_comparison}
\end{table*}

\subsection{Comparison with State-of-the-Art}
We compare \netname{} to GF \cite{he2010guided}, SD \cite{ham2017robust}, GSRPT \cite{lutio2019guided}, MSG \cite{hui2016depth}, DKN and its fast implementation FDKN \cite{kim2021deformable}, PMBANet \cite{ye2020pmbanet}, FDSR \cite{he2021towards}, JIIF \cite{tang2021joint}, DCTNet \cite{zhao2022discrete}, LGR \cite{de2022learning}, and finally to DADA~\cite{metzger2022guided} on Middlebury, NYUv2 and DIML datasets. We could not compare with PDRNet \cite{liu2022pdr} under the same setting because the source code is unavailable at the time of writing. For the other methods, we use the results from \cite{de2022learning} or the officially published codes, and results from \cite{yuan2023recurrent,metzger2022guided} for concurrent works. On the RGBDD dataset, the proposed network is compared to SDF~\cite{li2016deep}, SVLRM \cite{pan2019spatially}, DJF~\cite{li2016deep}, DJFR~\cite{li2019joint}, PAC~\cite{su2019pixel}, CUNet~\cite{deng2020deep}, FDKN~\cite{kim2021deformable}, DKN~\cite{kim2021deformable}, FDSR~\cite{he2021towards}, DCTNet~\cite{zhao2022discrete} and RASG~\cite{yuan2023recurrent}. To be fair with DCTNet~\cite{zhao2022discrete}, we downsample depth maps as the LR input.  
When reporting results, we highlight \gold{absolute}, \silver{second} and \bronze{third} best methods for each metric on each dataset.

\textbf{Quantitative Comparison.} Tabs. \ref{sota_comparison_mid_nyu_diml} and \ref{sota_comparison_rgbdd} report the accuracy of super-solved depth maps at factors $4\times$, $8\times$ and $16\times$ on the four datasets. As expected, learning-based methods show a significant improvement over traditional methods \cite{he2010guided,ham2017robust,lutio2019guided}. \netname{} vastly outperforms any existing network, with larger gaps in accuracy with the increasing of the upsampling factor. This can be attributed to the limitations affecting existing methods, i.e., 1) the guidance of either explicit or implicit RGB features alone being insufficient; 2) multi-modal information fusion on a single scale being not flexible enough to deal with complex scenes. Both limitations are fully addressed by \netname, which consistently outperforms concurrent works \cite{metzger2022guided,yuan2023recurrent}. 


The margin is consistent both on perfect (Middlebury) and noisy datasets (NYUv2, DIML, RGBDD), with the latter being a more challenging, realistic benchmark. Although \netname$^+$ is definitely the absolute best, its margin over \netname{} is negligible, with tiny gains yielded by NLSPN with respect to our main modules. Indeed, \netname{} alone consistently outperforms any other approach already.

       
\textbf{Qualitative Comparison.}
Fig. \ref{qualitative} shows qualitative comparisons of $8\times$ super-solved depth maps on Middlebury, NYUv2 and DIML datasets, respectively. From left to right, we show, the RGB image and LR depth map, followed by the ground truth HR depth and error maps obtained by several state-of-the-art frameworks, concluding with ours in the second-to-last columns. In each of the three examples, the lower error magnitude produced by \netname{}$^+$ further demonstrates its superior accuracy. 

\textbf{Cross-dataset Generalization.}
We conclude the comparison with existing methods by conducting cross-dataset experiments with $8\times$ factor. All methods are trained on the NYUv2 dataset and directly evaluated on DIML and Middlebury. Table \ref{cross-data_comparison} collects quantitative results for the 11 selected methods. Again, CNN-based methods attain better performance than traditional approaches, despite the domain gap playing a significant role in performance -- as evident by comparing results with Table \ref{cross-data_comparison}. Nonetheless, \netname{} outperforms any other framework on DIML. 


\begin{figure}	
	\centering	
	\captionsetup[subfigure]{font=footnotesize,textfont=footnotesize}
	\subfloat[RGB]{	
		\centering	
		\label{cross_dataset} 
		\includegraphics[height=0.8in]{./figs/ablation_figure/cross_dataset/receptive_field/cross_dataset.pdf}}	
	\hspace{-2mm}
	\subfloat[$D_{hr}$]{	
		\centering	
		\label{HR}
		\includegraphics[width=0.8in]{./figs/ablation_figure/cross_dataset/receptive_field/HR.pdf}}
	\hspace{-2mm}
	\subfloat[$D_{lr}$]{	
		\centering	
		\label{LR}
		\includegraphics[width=0.8in]{./figs/ablation_figure/cross_dataset/receptive_field/LR.pdf}}
		\vspace{-0.3cm}
	\caption{\textbf{Image context processed on Middlebury -- HR vs LR.} (a) RGB image and depth patches $D$ processed when testing on (b) Middlebury\textit{-HR} and (c) Middlebury\textit{-LR}. }	
	\label{hr-lr} 
\end{figure}

When considering the Middlebury dataset, we evaluate using the setting proposed in \cite{de2022learning} -- Middlebury\textit{-HR} in the table. In this case, our results are slightly less accurate compared to a few existing methods. However, given the very high resolution of Middlebury images, we argue that this testing protocol -- i.e., consisting of processing $256\times 256$ crops at a time -- penalizes our network's ability to leverage the global context in the input that results irremediably reduced to a very local area in these images. Therefore, we also evaluate on Middlebury test set defined by~\cite{tang2021joint} -- Middlebury-\textit{LR} in the table. Note that different subsets of images are used in Middlebury\textit{-HR} and Middlebury-\textit{LR} splits. Besides, Middlebury-\textit{LR} images are resized and processed without cropping, i.e., used at full-size after resizing, allowing to fully exploit global context, while this is not feasible with Middlebury-\textit{HR} due to memory constraints. In this case, \netname{} attains the best performance again, confirming our previous analysis, as shown in Tab. \ref{cross-data_comparison}. Such a difference in terms of context is highlighted in Fig. \ref{hr-lr}.

\begin{table}[t]
    \centering
	\renewcommand\tabcolsep{3pt} 
    \scalebox{0.5}{
    \begin{tabular}{ccc}

    \begin{tabular}{@{}ccccccc@{}} %\label{hf_infomation}
		\toprule
		\textbf{No.} & \textbf{Gradient} & \tabincell{c}{\textbf{Shallow} \\ \textbf{Feature}} & \textbf{LCF} & \textbf{ResBlock} & \textbf{MSE} & \textbf{MAE}\\
		\midrule
		(\uppercase\expandafter{\romannumeral1}) & \XSolidBrush &  \Checkmark     &  \Checkmark &  & 13.1 & 1.19 \\
		(\uppercase\expandafter{\romannumeral2}) & \Checkmark &    \XSolidBrush   &   &  & 12.4 & 1.14 \\
		(\uppercase\expandafter{\romannumeral3}) & \Checkmark &    \Checkmark     &   & \Checkmark & 12.3 & 1.15 \\
		\rowcolor{LightYellow}
		(\uppercase\expandafter{\romannumeral4}) & \Checkmark &    \Checkmark     & \Checkmark  &  & \gold{11.8} & \gold{1.12} \\
		\bottomrule
	\end{tabular}
	
	& \quad &
	
	\begin{tabular}{@{}clcc@{}} %\label{edge_types}
		\toprule
		\specialrule{0em}{3pt}{3pt}
		\multicolumn{1}{c}{\textbf{No.}} & 
		\tabincell{l}{\textbf{HF Information} \textbf{ \quad\quad\quad\quad}} & \textbf{MSE} & \textbf{MAE}\\
		\specialrule{0em}{3pt}{2pt}
		\midrule
		(\uppercase\expandafter{\romannumeral1}) & 
		{Canny Edge} & 12.0 & 1.13 \\
		(\uppercase\expandafter{\romannumeral2}) & 
		{Gaussian Edge} & 12.1 & 1.16 \\
		(\uppercase\expandafter{\romannumeral3}) & 
		{DCT} & 12.1 & 1.15 \\
		(\uppercase\expandafter{\romannumeral4}) & 
		{Wavelet Transform} & 12.1 & 1.15  \\
		\rowcolor{LightYellow}
		(\uppercase\expandafter{\romannumeral5}) & 
		{Gradient Map} & \gold{11.8} & \gold{1.12} \\
		\bottomrule
	\end{tabular}
	
	\\
	\textbf{(a)} & \quad & \textbf{(b)} 
	\\
	\\
	
	
	\begin{tabular}{@{}clcccc@{}} %\label{dsp_ablation}
		\toprule
		\textbf{No.} & \textbf{Config.} & \textbf{Params (M)} & \textbf{Flops (G)} & \textbf{MSE} & \textbf{MAE}\\
		\midrule
		(\uppercase\expandafter{\romannumeral1}) & EdgeNet \cite{liu2021multi}    & 5.78 &  95.6  & 12.0 & \gold{1.12} \\
		(\uppercase\expandafter{\romannumeral2}) & SCPA \cite{zhao2020efficient}  & 0.29 &  13.1  & 12.5 & 1.16 \\
		\rowcolor{LightYellow}
		(\uppercase\expandafter{\romannumeral3}) & HFEB       & \gold{0.27} & \gold{11.6}  & \gold{1.18} & \gold{1.12} \\
		\rowcolor{white}
		\bottomrule
		\multicolumn{4}{c}{\quad\quad\textbf{(c)}} \\
		\\
		\toprule
		\textbf{No.} & \textbf{Config.} & \textbf{Params (M)} & \textbf{MSE} & \textbf{MAE}\\
		\midrule
		(\uppercase\expandafter{\romannumeral1}) & 
		w/o AFFM        & -   & 12.7 & 1.16 \\
		(\uppercase\expandafter{\romannumeral2}) & 
		w/o att         & 1.3 & 12.2 & 1.13 \\
		(\uppercase\expandafter{\romannumeral3}) & 
		Concat.  & 4.5 & 12.2 & 1.13 \\
		\rowcolor{LightYellow}
		(\uppercase\expandafter{\romannumeral4}) & 
		AFFM & 3.0 & \gold{11.8} & \gold{1.12} & \\
		\rowcolor{white}
		\bottomrule
		\multicolumn{4}{c}{\quad\quad\textbf{(e)}} \\
	\end{tabular}
	
	
	& \quad &
	
	
	\begin{tabular}{@{}clccc@{}} %\label{affm_setting}
		\toprule
		\textbf{No.} & \textbf{Scales} & \textbf{Params (M)} & \textbf{MSE} & \textbf{MAE}\\
		\midrule
		(\uppercase\expandafter{\romannumeral1}) & 
		H1              & 1.5 & 12.3 & 1.14 \\
		\rowcolor{LightYellow}
		(\uppercase\expandafter{\romannumeral2}) & 
		H1, H2       & 3.0 & \gold{11.8} & \gold{1.12} \\
		\rowcolor{white}
		(\uppercase\expandafter{\romannumeral3}) & 
		H1, H2, H3              & 4.5 & \gold{11.8} & \gold{1.12} \\
		\bottomrule
		\multicolumn{4}{c}{\textbf{(d)}} \\
% 		\\
% 		\\
        \specialrule{0em}{5.4pt}{5.4pt} %
		\toprule
		\specialrule{0em}{1.7pt}{1.7pt} %
		\textbf{No.} & \textbf{Stages} & \textbf{Params (M)} & \textbf{MSE} & \textbf{MAE}\\
		\specialrule{0em}{1.7pt}{1.7pt} %
		\midrule
		\specialrule{0em}{1.8pt}{1.8pt} %
		(\uppercase\expandafter{\romannumeral1}) & 
		$1$   & 14.2 & 13.3 & 1.19 \\
		\specialrule{0em}{1.8pt}{1.8pt} %
		\rowcolor{LightYellow}
		(\uppercase\expandafter{\romannumeral2}) & 
		$2$   & 25.0 & 11.8 & 1.12 \\
		\specialrule{0em}{1.8pt}{1.8pt} %
		\rowcolor{white}
		(\uppercase\expandafter{\romannumeral3}) & 
		$3$   & 37.5 & \gold{11.6} & \gold{1.10} \\
		\specialrule{0em}{1.8pt}{1.8pt} %
		\bottomrule
		\multicolumn{4}{c}{\quad\quad\textbf{(f)}} \\
	\end{tabular}
	
    \end{tabular}}
    \vspace{-0.3cm}
    \caption{\textbf{Ablation study (NYUv2 test set, $8\times$ factor).} We measure the impact of (a) explicit vs implicit HR features, (b) different kinds of HF supervision, (c) different sub-networks for explicit HF features extraction, (d) scales at which AFFM is applied, (e) modules building AFFM, (f) number of stages in GDRB. In yellow, configurations corresponding to our final model without NLSPN.}
    \label{tab:ablations}
\end{table}


\subsection{Ablation Study}
We now perform a series of ablation experiments to measure the impact of key components and parameters in \netname. Tab. \ref{tab:ablations} collects the outcome of these studies, conducted on NYUv2 test set with $8\times$ factor. Without loss of fairness, NLSPN is never used here -- to fully focus on the impact of single components. 

\textbf{(a) Implicit vs Explicit High-Frequency Features.}
To measure the impact of both implicit and explicit HR features, we compare the performance of the proposed network and its variants when extracting either only one of the two. The quantitative results are collected in Tab.~\ref{tab:ablations}(a). Without the help of gradient maps (I), the performance of the network significantly degrades. We believe this is caused by the difficulty in effectively extracting fine structures or salient edges required for LR depth maps from implicit HF features alone. Moreover, explicit features highlight regions in the image that need to be focused on, avoiding \netname{} to learn to localize them and easing its task. 


Nonetheless, explicit HF features alone as guidance (II) are insufficient as well. We argue that the explicit information might neglect some RGB features, whereas implicit HF feature extraction can recover them. Furthermore, to verify the effectiveness of LCF, we replace it with ResBlock~\cite{he2016deep} (III) to extract shallow features from RGB images, highlighting a negative impact on implicit features extraction -- i.e., it results less accurate than (II). 

\textbf{(b) Ablation on Explicit High-Frequency Features.}
We now investigate which kind of HF information is more effective for our framework. Purposely, we train HFEB with supervision coming from five different HF features used as ground truth edge maps $E_{gt}$. Tab.~\ref{tab:ablations}(b) collects results from this experiment, highlighting that Canny edges (I) and Gradient maps (V) lead to slightly better results. 


\textbf{(c) Impact of HFEB.}
To verify the effectiveness of HFEB, we replace it with EdgeNet~\cite{liu2021multi} -- based on the widely-used U-net structure -- and SCPA~\cite{zhao2020efficient}, which inspires our scaling strategy. As shown in Tab.~\ref{tab:ablations}(c), EdgeNet (I) achieves lower MSE and MAE than SCPA (II), yet needs more parameters -- 5.78M vs. 0.29M. HFEB (III) yields the same accuracy as EdgeNet, with fewer parameters than SCPA, thus being both more accurate and efficient. 



\textbf{(d -- e) Impact of AFFM.}
We now measure the effectiveness of AFFM. Tab.~\ref{tab:ablations}(d) shows results obtained by deploying AFFM at different scales, respectively the highest (I), the first two (II) and all of the three scales. We can notice how performing fusion at the highest scale alone results insufficient, whereas using multi-scale features for fusion yields improvements, despite saturating already when using two scales, with the lowest one not providing additional, meaningful details to be taken into account.

Furthermore, we ablate AFFM in its single components. Tab.~\ref{tab:ablations}(e) resumes the outcome of this evaluation. 
We first test the performance of \netname{} without AFFM (I), highlighting a large drop in accuracy. By adding dynamic fusion, yet without using attention (II) vastly improves the results already, while replacing the weighted sum in the upper of Fig.~\ref{affm} with concatenation and a ResBlock~\cite{he2016deep} (III) yields worse results compared to our full AFFM (IV). 

\textbf{(f) Impact of Stages Number.}
To conclude, we evaluate the impact of the multi-stage design.
As shown in Tab.~\ref{tab:ablations}(f), a single-stage architecture (I) is vastly outperformed by deploying two stages (II), yet at the expense of doubling the number of parameters. Furthermore, while the three-stage architecture (III) still yields some improvement, the benefit is minor in comparison to the significant increase in parameters. Hence, we choose two stages as the default configuration to balance accuracy and efficiency.


\begin{table}[t] \footnotesize
	\renewcommand\tabcolsep{1.5pt} 
	\centering
	\scalebox{0.8}{
	\begin{tabular}{@{}lcccccc@{}}
		\toprule
		 & PMBANet~\cite{ye2020pmbanet} & FDSR~\cite{he2021towards} & JIIF~\cite{tang2021joint} & DCTNet~\cite{zhao2022discrete} & LGR~\cite{de2022learning} & Ours \\ 
		 \midrule
		 Runtime (ms)
		 & 26.9 & 1.03 & 89.8 & 9.03 & 26.4 & 51.5\\
		 Memory Peak (GB)
		 & 3.07 & 2.05 & 2.36 & 0.26 & 0.19 & 18.6 \\ 
		\bottomrule
	\end{tabular}}
	\vspace{-0.3cm}
	\caption{\textbf{Computational requirements}. Experiments on Nvidia RTX 3090 GPU, with $256\times256$ input and $8\times$ factor.}
	\label{runtime_memory}
    
\end{table}

\subsection{Limitations}
We conclude by listing a few limitations of \netname. As previously pointed out, global context is crucial for it to achieve the best performance. When this is unavailable, some accuracy is lost when generalizing across datasets. Moreover, the significant improvements over existing methods are paid for in terms of time/memory requirements. Tab. \ref{runtime_memory} highlights the higher runtime and, more evidently, peak memory usage. Future work will aim at reducing the overhead, while minimizing the drop in accuracy.



% noise
\begin{figure*}[!t]
	\centering
	\includegraphics[width=\linewidth]{figures/noise.pdf}
	\caption{
    The performance of GNN teacher, distilled MLP students via GLNN and PKGD when adding different noise to the initial node features.
	% Comparisons among GNN teacher, distilled MLP students via GLNN and PKGD.
 %    We add noise to the initial node features.
    For GNN teachers, we select SAGE, GAT, GCN and APPNP, respectively.
    \textbf{Upper}: \textbf{Cora} dataset and \textit{transductive} setting.
    \textbf{Lower}: \textbf{Pubmed} dataset and \textit{inductive} setting.
    }
	\label{noise}
	% \vspace{-0.5em}
\end{figure*}

\section{Analysis and Discussion}
We further explore the ability to capture graph structural information as well as the robustness of the proposed PGKD.
We also visualize the distributions of node representations for deeper insights.

\subsection{Can PGKD distill the Impact of Graph Edges?}

As mentioned in Section \ref{impact}, the intra-class edges guarantee the homophily for nodes from the same class, while the inter-class edges determine the pattern of distances among class prototypes.

% For Intra-class edges, we calculate the average L2 distance for the features of connected nodes in Graph.
We adopt SAGE as the GNN teacher and perform experiments under a transductive setting, and then calculate the average L2 distance for the features of connected nodes in the graph.
Table \ref{tab:intr_dis} shows the average distance of initial node features and node features from GNN teacher~(SAGE), GLNN, and PGKD.
The distance of the GNN teacher is the shortest due to the information aggregation operations along graph edges.
Meanwhile, the distance for GLNN is much longer due to the weak awareness of such graph structural information.
PGKD gets shorter distances than GLNN, showing a great ability to capture intra-class graph structural information.
In particular, PGKD gets a L2 distance of 0.82 on Citeseer, which is shorter than 1.40 from the GNN teacher.

\begin{table}[!t]
\centering
%\tableindent 
\renewcommand\arraystretch{1.2}
\resizebox{0.9\columnwidth}{!}
{%
\begin{tabular}{lcccc}
\hline
\textbf{Dataset}    & \textbf{Input}   & \textbf{GNN}  & \textbf{GLNN} & \textbf{PGKD} \\ \hline
Cora       & 4.40  & 1.95 & 3.02 & 2.47 \\
Citeseer   & 5.66  & 1.40 & 3.10 & 0.82 \\
A-computer & 17.63 & 2.35 & 7.14 & 4.76 \\ \hline
\end{tabular}
}
\caption{
Average L2 distance for the features of connected nodes on different datasets.
% We adapt SAGE for GNN teacher.
% GLNN learns MLP by vanilla logit-base KD.
}
\label{tab:intr_dis}
\end{table}


% \begin{table}[!t]
% \centering
% %\tableindent 
% \renewcommand\arraystretch{1.2}
% % \resizebox{0.8\columnwidth}{!}
% % {%
% \begin{tabular}{lccc}
% \hline
% \textbf{Dataset} & \textbf{GNN}   & \textbf{GLNN}  & \textbf{PGKD}  \\ \hline
% Cora                         & -0.94 & -0.88 & -0.92 \\
% Citeseer                     & -0.71 & -0.62 & -0.67     \\
% A-computer                   & -0.75 & -0.60 & -0.77    \\
% \hline
% \end{tabular}
% % }
% \caption{
% Spearman correlation $\rho$ between class distances and inter-class edges quantity.
% $\rho \to -1$ indicates more negatively correlated for two variables.
% }
% \label{tab:inter}
% \end{table}

\begin{table}[!t]
% \begin{wraptable}{r}{0.5\textwidth} 
\centering
%\tableindent 
% \renewcommand\arraystretch{1.2}
% \vspace{-0.6cm}

% \vspace{0.3cm}
\resizebox{0.35\textwidth}{!}
{%
\begin{tabular}{lccc}
\hline
\textbf{Dataset} & \textbf{GNN}   & \textbf{GLNN}  & \textbf{PGKD}  \\ \hline
Cora                         & -0.94 & -0.88 & -0.92 \\
Citeseer                     & -0.71 & -0.62 & -0.67     \\
A-computer                   & -0.75 & -0.60 & -0.77    \\
\hline
\end{tabular}
}
\caption{
Spearman correlation $\rho$ between class distances and inter-class edges quantity.
$\rho \to -1$ indicates more negatively correlated.
}
% \end{wraptable}
\label{tab:inter}
% \vspace{-0.3cm}
\end{table}

The inter-class edges determine the pattern of distances among class prototypes.
Specifically, the prototypes of two classes would be closer with more inter-edges connecting them in GNNs.
We take statistics on the class distances~(defined as L2 distances among class prototypes) and quantity of corresponding inter-class edges.
For qualitative analysis, we calculate the Spearman correlation.
From Table \ref{tab:inter}, the GNN teacher has a low Spearman correlation, whereas GLNN shows a relatively high value.
Meanwhile, the proposed PGKD, thanks to the intra-class loss, can better capture the intra-class graph structural information and exhibits a much lower correlation.

\subsection{Is PGKD Robust to Noisy Node Features?}
To analyze the robustness of PGKD on node noise, we further evaluate the performance after adding Gaussian noise of different levels to initial node features $X$.
Specifically, we replace $X$ with $(1-\alpha)X+\alpha \epsilon$, where $\epsilon$ denotes the isotropic Gaussian noise independent from $X$, and $\alpha \in [0,1]$ controls the noise level.
A larger $\alpha$ means a stronger noise.
Figure \ref{noise} shows the performance of GNN, GLNN, and PGKD under different noise levels.
On both Cora and Citeseer, PGKD outperforms GLNN consistently as the noise level ranges from 0.1 to 0.9.
Particularly, PGKD could get better results than GAT and APPNP on Pumbed with $\alpha=0.9$. 
These show that PGKD is more robust than GLNN with respect to noisy input node features due to its ability to capture graph structural information.



% split
\begin{figure}[!t]
	\centering
	\includegraphics[width=\linewidth]{figures/ratio.pdf}
	\caption{
	The performance of GNN teacher, distilled MLP students via GLNN and PKGD under \textit{inductive} setting with different split ratio.
    % \textbf{Upper}: \textbf{Citeseer} dataset and SAGE as GNN teacher.
    % \textbf{Lower}: \textbf{Pubmed} dataset and GCN as GNN teacher.
    We select GCN as GNN teacher and perform experiments on \textbf{Pubmed} dataset.
    }
	\label{ratio}
	% \vspace{-0.5em}
\end{figure}

\subsection{Impact of Inductive Split Ratio}
To evaluate the ability for less observed data under inductive setting, we conduct the experiments under different split ratios, defined as the ratio $|\mathcal{V}^{U}_{ind}|/|\mathcal{V}^{U}|$.
A larger split ratio means less observed unlabeled data during training and more inductive unlabeled data for test~(cf. Section \ref{setting_intro}).
% Please refer to Section \ref{setting_intro} for more details about inductive setting.
As shown in Figure \ref{ratio}, the performance of the GNN teacher is not monotonically decreasing since the way to split graph~(i.e. the edges to remove) is also vital as the number of nodes for training.
PGKD outperforms GLNN and GNN under all split ratios.
Also, the performance of PGKD is more stable than GLNN.
This proves that PKGD, explicitly capturing the graph structural information, is robust and effective under different inductive split ratios.

\subsection{Impact of MLP Setting}

\begin{table}[t]
\centering
%\tableindent 
% \renewcommand\arraystretch{1.2}

\resizebox{0.5\textwidth}{!}
{%
\begin{tabular}{lllcccc}
\hline
\textbf{\#L} & \textbf{\#H} & \textbf{Params} & \textbf{MLP} & \textbf{GLNN} & \textbf{PGKD} &  $\Delta$\textbf{GLNN} \\ \hline
2 & 64 & 0.09M                & 53.40        & 73.30         & 74.00  & \textbf{$\uparrow$0.70}       \\
2 & 128 & 0.18M              & 59.48        & 71.66         & 74.71 & \textbf{$\uparrow$3.05}        \\
3 & 128 & 0.20M               & 54.33        & 73.07         & 74.24 & \textbf{$\uparrow$1.17}        \\
2 & 512 & 0.73M               & 56.21        & 73.54         & 74.47  & \textbf{$\uparrow$0.92}       \\
3 & 512 & 1.00M               & 54.57        & 72.83         & 74.00   & \textbf{$\uparrow$1.17}      \\ \hline
\end{tabular}
}

\caption{
Comparisons for vanilla MLP, distilled MLP students via GLNN and PGKD with different MLP settings on \textbf{Cora} under \textit{inductive} setting.
We report the average test accuracy (\%).
\textbf{\#L} denotes the layers and \textbf{\#H} denotes dimension of hidden state.
}
\label{tab:mlp_impact}
\end{table}

We further conduct experiments using different MLP settings.
The GNN teacher is a two-layer GCN with 0.18M parameters and gets an accuracy of 83.37\% on Cora dataset.
As shown in Table \ref{tab:mlp_impact}, the vanilla MLP shows an overfitting trend when the number of parameters increases, while the PGKD does not.
Meanwhile, PGKD gets the highest results under all settings and shows consistent improvement over GLNN.
In particular, GLNN gets a score of 74.71\%~(\#L=2, \#H=128), which is 3.05\% higher than GLNN.
Such findings indicate that PGKD is more robust and effective in different MLP settings.

\subsection{Node Representation Distribution}

We visualize the distribution of node representations from GNNs and MLPs~(vanilla MLPs without KD, MLPs from GLNN, and MLPs from PGKD) via t-SNE \cite{JMLR:v9:vandermaaten08a}.
We select the GAT as the GNN teachers.
Figure \ref{node_visualize} shows the results on Cora and Citeseer under transductive setting.
Due to the message passing architecture, the node representations in the same class from GNNs are much more gathered than vanilla MLPs.
PGKD captures such graph information via intra-class loss, while vanilla MLPs and MLPs from GLNN lack such capability. 
The same-class features from both GLNN and vanilla MLP are slightly dispersed, while the features from PGKD are more clustered inside a class and separable between classes.
Moreover, PGKD can learn better class prototype distributions.
Specifically, in the GNN representations on Cora, the dark green and purple classes are far from each other. 
PGKD captures such a behavior well, where GLNN fails.

\begin{figure*}[!t]
	\centering
	\includegraphics[width=\linewidth]{figures/visualize_feats_new.pdf}
	\caption{
	The distribution of node representations for GNN teacher, vanilla MLP, and distilled MLPs from GLNN and PKGD.
    % under \textit{transductive} setting.
    % For GNN teacher, we select GAT.
    \textbf{Upper}: \textbf{Cora} dataset.
    \textbf{Lower}: \textbf{Citeseer} dataset.
    }
	\label{node_visualize}
	% \vspace{-0.5em}
\end{figure*}
% \input{sections/6-Related}
% 
\section{Related Work}\label{sec:related}

Grasping is a fundamental problem for robotic manipulation and has been extensively studied. Most work focuses on parallel-jaw grippers \cite{DBLP:conf/cvpr/FangWGL20,jiang2011efficient,DBLP:conf/iccv/MousavianEF19,DBLP:conf/icra/MuraliMEPF20,DBLP:conf/icra/SundermeyerMTF21}  due to their simplicity, low DoFs, and computational efficiency. However, parallel-jaw grippers are less efficient and less reliable for manipulating arbitrary-shaped objects. To achieve user-friendly interaction, multi-finger robotic hands and dexterous grasping remain a hot research topic in the field of robotic manipulation~\cite{rimon2019mechanics}. This research can be briefly divided into two categories: the traditional analytical sampling-based method and the data-driven method.

\textbf{Traditional analytical sampling-based methods}\cite{ciocarlie2007dexterous,DBLP:conf/icra/GoldfederALP07,DBLP:conf/iros/HangSK14,DBLP:conf/icra/MillerKCA03,DBLP:conf/icra/PelossofMAJ04} sampled various grasp candidates and evaluated them based on certain metrics considering the physical properties of objects such as wrench space~\cite{DBLP:conf/icra/BorstFH04}. In general, both the object model and environment are assumed to be known in advance~\cite{DBLP:journals/ram/MillerA04}. Eigengrasp~\cite{ciocarlie2007dexterous} reduced the dimensions of grasp search space by performing principal component analysis (PCA) on grasping pose and configuration data. Although the reduction increases the efficiency of generating grasps, the search space of the random sampling process for grasps is still very huge. As a result, these sampling-based methods are less efficient in practical use.

\textbf{Data-driven methods} fall into one of two primary types.
The one is an extension of the traditional sampling-based method~\cite{DBLP:conf/iros/VarleyWWA15,DBLP:conf/icra/BorstFH04}. Instead of computing physical metrics, this method directly estimates grasp quality metrics from trained deep models. The grasp success rate can be greatly improved since traditional metrics cannot be computed accurately from an incomplete view of a novel object without any contact feedback. However, they are still dependent on known object models and exhibit the problem of huge sampling and search space.
%
The other data-driven method is performed in an end-to-end manner~\cite{DBLP:conf/iros/HangSK14,DBLP:conf/rss/LiuP0GM20,DBLP:journals/corr/abs-1908-04293,DBLP:conf/iros/LiuP0GM19,DBLP:conf/icra/KapplerBS15,DBLP:conf/iros/VarleyWWA15,mahler2017dex}. Specifically, this method takes the image or point cloud data of a grasped object as input and outputs a high-quality grasp. These approaches are able to effectively generate grasps and are robust to unknown objects. However, many can only handle a single object. Grasping may often fail due to the potential collision between the gripper and the environment.
%
Some recent work~\cite{DBLP:conf/icra/LiWL0LZ22,DBLP:conf/icra/LundellCLVWRMK21,DBLP:journals/corr/abs-2103-04783} predicts
collision-free \mbox{6-DoF} grasping in clutter using multi-finger grippers. They only classify the grasp types and do not take into account of the properties of multi-finger grasps. Our approach considers the gripper's physical structure and does not rely on the grasp types. Using a novel grasping representation and an end-to-end deep neural network based on contacts, our approach significantly reduces the search space for grasping and can generate reliable grasp poses.

\section{Conclusion}

% In this work, we propose PGKD to distill the knowledge from high-accuracy GNNs to low-latency MLPs.
% The distillation process is edge-free and the learned MLP students are structure-aware.
% Firstly, we analyze the impact of graph structure~(graph edges) on GNNs.
% Specifically, we categorize the graph edges into Intra-class edges and Inter-class edges and study their impact, respectively.
% Based on the analysis, we design two corresponding losses via class prototypes to transfer the graph structural knowledge from GNNs to MLPs.
% Experiments on popular benchmarks demonstrate the effectiveness of our proposed PGKD.
% Further analysis indicate that PGKD is robust to noisy node features and performs well in different training settings.

% For future work, we would consider to apply PGKD to other graph tasks other than node classification.
% Moreover, generating the prototypes basing on the node representations rather than the class labels would be another interesting topic.

A novel PGKD scheme has been proposed to distill the knowledge from high-accuracy GNNs to low-latency MLPs, wherein the distillation process is edge-free and the learned MLP students are structure-aware. 
Specifically, we analyze the impact of graph structure~(graph edges) on GNNs and categorize them into intra-class and inter-class edges. 
Two corresponding losses via class prototypes are designed to transfer the graph structural knowledge from GNNs to MLPs.
Experiments on popular benchmarks demonstrate the effectiveness of PGKD.
Additionally, we show PGKD is robust to noisy node features, and performs well under different training settings.

For our future work, PGKD will be generalized to other graph tasks beyond node classification. 
Another interesting direction will be to generate prototypes utilizing node representations rather than class labels.


\clearpage
\section*{Limitations}
In PGKD, we adopt the class prototypes to capture graph structural information for MLPs in an edge-free setting.
Subsequently, PGKD requires slightly more computing cost compared to the baseline GLNN.
Meanwhile, the gap between the MLP learned by PGKD and its teacher GNN under the inductive setting is larger than that under the transductive setting, especially on Cora and Penn94 datasets.
More effort to improve the performance under the inductive setting is required underway.


% This must be in the first 5 lines to tell arXiv to use pdfLaTeX, which is strongly recommended.
\pdfoutput=1
% In particular, the hyperref package requires pdfLaTeX in order to break URLs across lines.

\documentclass[11pt]{article}

% Remove the "review" option to generate the final version.
%\usepackage[review]{ACL2023}
\usepackage{ACL2023}

% Standard package includes
\usepackage{times}
\usepackage{latexsym}

% For proper rendering and hyphenation of words containing Latin characters (including in bib files)
\usepackage[T1]{fontenc}
% For Vietnamese characters
% \usepackage[T5]{fontenc}
% See https://www.latex-project.org/help/documentation/encguide.pdf for other character sets

% This assumes your files are encoded as UTF8
\usepackage[utf8]{inputenc}

% This is not strictly necessary, and may be commented out.
% However, it will improve the layout of the manuscript,
% and will typically save some space.
\usepackage{microtype}

% This is also not strictly necessary, and may be commented out.
% However, it will improve the aesthetics of text in
% the typewriter font.
\usepackage{inconsolata}


% If the title and author information does not fit in the area allocated, uncomment the following
%
%\setlength\titlebox{10cm}
%
% and set <dim> to something 5cm or larger.

%%%%%%%%%%%%%%%%%%%%%%%%%%%%%%%%%%
\usepackage{graphicx}
\usepackage{amsfonts}
\usepackage{amsmath}
\usepackage{bigdelim}
\usepackage{diagbox}
\usepackage{amsthm}
\usepackage{makecell}
\usepackage{mathtools}
\usepackage{booktabs}
\usepackage[shortlabels]{enumitem}
\graphicspath{ {figs/} }

\theoremstyle{remark}
\newtheorem*{question}{Question}

\newcommand{\tk}[1]{\textcolor{blue}{{#1}}}
\newcommand{\sy}[1]{\textcolor{red}{{#1}}}
\newcommand{\mg}[1]{\textcolor{purple}{{#1}}}
\newcommand{\lh}[1]{\textcolor{green}{{#1}}}
\newcommand{\lc}[1]{\textcolor{green}{{#1}}}

% Rounded color box
\definecolor{light_blue}{HTML}{cfdfff}
\usepackage[most]{tcolorbox}
\tcbset{on line, 
        boxsep=1pt, left=0pt,right=0pt,top=0pt,bottom=0pt,
        colframe=white,colback=light_blue,  
        highlight math style={enhanced}
        }

\newcommand{\quash}[1]{}  %Anything in \quash is ignored
\newcommand{\gpt}{\textsc{GPT-2}}
\newcommand{\bert}{\textsc{BERT}}
\newcommand{\bertlarge}{\textsc{BERT-large}}
\newcommand{\mask}{\texttt{[MASK]}}
\newcommand{\cls}{\texttt{[CLS]}}
\newcommand{\sep}{\texttt{[SEP]}}
\newcommand{\mat}{\texttt{mat}}
\newcommand{\id}{\texttt{id}}
\newcommand{\matl}{\texttt{mat}_{\ell \rightarrow \ell'}}
\newcommand{\matattnl}{\texttt{mat\_attn}_{\ell \rightarrow \ell'}}
\newcommand{\matffl}{\texttt{mat\_ffn}_{\ell \rightarrow \ell'}}
\newcommand{\matlnl}{\texttt{mat\_ln1\_ln2}_{\ell \rightarrow \ell'}}
\newcommand{\idl}{\texttt{id}_{\ell \rightarrow \ell'}}
\newcommand{\matlL}{\texttt{mat}_{\ell \rightarrow L}}
\newcommand{\matattnlL}{\texttt{mat\_attn}_{\ell \rightarrow L}}
\newcommand{\matfflL}{\texttt{mat\_ffn}_{\ell \rightarrow L}}
\newcommand{\matlnlL}{\texttt{mat\_ln1\_ln2}_{\ell \rightarrow L}}
\newcommand{\idlL}{\texttt{id}_{\ell \rightarrow L}}

\definecolor{blue(munsell)}{rgb}{0.0, 0.5, 0.69}
%%%%%%%%%%%%%%%%%%%%%%%%%%%%%%%%%%

\title{Jump to Conclusions: Short-Cutting Transformers\\With Linear Transformations}

% Author information can be set in various styles:
% For several authors from the same institution:
% \author{Author 1 \and ... \and Author n \\
%         Address line \\ ... \\ Address line}
% if the names do not fit well on one line use
%         Author 1 \\ {\bf Author 2} \\ ... \\ {\bf Author n} \\
% For authors from different institutions:
% \author{Author 1 \\ Address line \\  ... \\ Address line
%         \And  ... \And
%         Author n \\ Address line \\ ... \\ Address line}
% To start a seperate ``row'' of authors use \AND, as in
% \author{Author 1 \\ Address line \\  ... \\ Address line
%         \AND
%         Author 2 \\ Address line \\ ... \\ Address line \And
%         Author 3 \\ Address line \\ ... \\ Address line}

\author{Alexander Yom Din$^{1}$ ~~~~~ Taelin Karidi$^{1}$ ~~~~~ Leshem Choshen$^{1}$ ~~~~~
Mor Geva$^{2}$ 
\vspace{0.2cm} \\
$^1$Hebrew University of Jerusalem ~~~ $^2$Google Research \\
\small{\texttt{\{alexander.yomdin, taelin.karidi, leshem.choshen\}@mail.huji.ac.il}}, \small{\texttt{pipek@google.com}}}

\quash{
\author{Alexander Yom Din \\
  Hebrew University of Jerusalem \\ \texttt{alexander.yomdin@mail.huji.ac.il} \\\And
  Taelin Karidi \\
  Hebrew University of Jerusalem \\
  \texttt{taelin.karidi@mail.huji.ac.il} \\\And
  Leshem Choshen \\
  Hebrew University of Jerusalem \\ \texttt{leshem.choshen@mail.huji.ac.il} \\\And
  Mor Geva \\
  Google Research \\
  \texttt{pipek@google.com} \\}
}

\begin{document}
\maketitle



\begin{abstract}
% \vspace{-1em}
The diffusion-based generative models have achieved remarkable success in text-based image generation. However, since it contains enormous randomness in generation progress, it is still challenging to apply such models for real-world visual content editing, especially in videos. 
In this paper, we propose \texttt{FateZero}, a zero-shot text-based editing method on real-world videos without per-prompt training or use-specific mask. 
\RM{Specifically, different from a pipeline of two independent inversion and then generation stages, we find the intermediate attention maps during inversions store better structure and motion information. We thus reform them to temporally casual attention and replace them in the generation progress. To further reduce the unnecessary semantic leakage of source video and enhance the editing quality, we then remix the temporally casual attentions via the cross-attention features of the source prompt as the mask.}
To edit videos consistently, we propose several techniques based on the pre-trained models. Firstly, in contrast to the straightforward DDIM inversion technique, our approach captures intermediate attention maps during inversion, which effectively retain both structural and motion information. These maps are directly fused in the editing process rather than generated during denoising. To further minimize semantic leakage of the source video, we then fuse self-attentions with a blending mask obtained by cross-attention features from the source prompt. Furthermore, we have implemented a reform of the self-attention mechanism in denoising UNet by introducing spatial-temporal attention to ensure frame consistency.
Yet succinct, our method is the first one to show the ability of zero-shot text-driven video style and local attribute editing from the trained text-to-image model. We also have a better zero-shot shape-aware editing ability based on the text-to-video model~\cite{tuneavideo}. \RM{Besides video, our unified method also achieves state-of-the-art performance in zero-shot image editing.\chenyang{Need exp or remove the zero-shot image}} Extensive experiments demonstrate our superior temporal consistency and editing capability than previous works.
% The code will be released.
% \chenyang{emphasize: our observation at inversion time} \xiaodong{replacing the bold part to the actual pipeline: \textbf{Specifically, we work on replacing and mixing the attention maps between the inversion and generation since the self-attention map keeps the structure of the original natural image and the cross-attention is semantic-related, after remixing, we replace them in the corresponding generation steps for denoising.}}
% \footnote{Since there is no general video diffusion model is publicly available, we use one-shot video generation method~(Tune-A-Video~\cite{tuneavideo}) as the pretrained video diffusion model for zero-shot video editing\xiaodong{can be removed if we actually zero-shot on video}.}.
\end{abstract}
\section{Introduction}

The ability to reason about plans is critical for performing long-horizon tasks \citep{erol1996hierarchical, sohn2018hierarchical, sharma-etal-2022-skill}, compositional generalization \citep{corona-etal-2021-modular} and generalization to unseen tasks and environments \citep{shridhar2020alfred}.
Consider a simple long-horizon planning scenario where a robot is tasked with preparing a meal and serving it on the table. 
This presents a non-trivial planning problem since the agent needs to understand the sequence of operations required to perform the task and search for the relevant objects in the unfamiliar environment by interacting with various objects. %



Large language models have been recently shown to possess commonsense knowledge about the world such as object affordances and physical dynamics \citep{ouyang2022training,chowdhery2022palm}.
Early approaches considered text based environments and fine-tuned PLMs to predict actions given the history of past observations and actions \citep{jansen-2020-visually,micheli-fleuret-2021-language,yao-etal-2020-keep}.
Recent work has used this ability to reason about plans from text instructions in simulated household environments with simplifying assumptions such as text-only environment observations or feedback \citep{huang2022language,ahn2022can,li2022pre,logeswaran-etal-2022-shot}.


We focus on \emph{visually grounded planning} with PLMs --- the ability to adapt plans based on interaction and visual feedback from the environment.
While PLMs have strong planning commonsense priors, predictions from a PLM may not be directly realizable in the environment since the observation and action spaces are unknown.
This requires \emph{grounding} the PLM in the environment and adapting it to observe visual feedback, which is highly non-trivial.
Some prior works assume the availability of a pre-trained affordance function \citep{ahn2022can} or a success detector \citep{mirchandani2021ella}.
Notably, SayCan \citep{ahn2022can} completely decouples the PLM from observation information by selecting actions that have both high affordability (through a pre-trained affordance model) and high PLM likelihood.
Although this partially addresses the grounding problem, the use of visual feedback for action affordance alone is limited.
Often an agent must choose one of many affordable actions using information from observations.
For example, a driving agent should re-navigate and possibly turn around when encountering a ``road closed'' sign, but both turning around and driving forward are indistinguishable to SayCan because they are both affordable and the PLM is blind to observations.

Another workaround explored in prior work is translating the information in the visual observations to text using a pre-trained captioning system \citep{shridhar2021alfworld,huang2022language}.
However, it can be difficult to faithfully describe an image in words and information is lost in this inherently noisy process, which limits the information available to the planner.



Recent work shows that PLMs can be adapted for various natural language tasks by inserting tunable embeddings or soft prompts at the input of the PLM (also called prompt tuning or prefix tuning)~\citep{li-liang-2021-prefix,lester-etal-2021-power}.
This approach also extends to multi-modal understanding tasks such as image captioning \citep{mokady2021clipcap} and VQA \citep{tsimpoukelli2021multimodal} where images are encoded as soft prompts and finetuned for the target task.
Transformer based architectures have also been successfully applied to offline Reinforcement Learning in recent work \citep{chen2021decision,janner2021offline,li2022pre,reid2022can}.

Taking inspiration from these works, we propose the simple approach of embedding visual observations (`visual prompts') and \textit{directly inserting them as PLM input embeddings}.
The visual encoder and PLM are jointly trained for the target task, an approach we call \textbf{\oursfull}~(\ours).
By teaching the PLM to use observations for planning in an end to end manner, we remove the dependency on external data such as captions and affordability information that was used in prior work.
We show that this simple approach performs better than prior PLM-based planning approaches on two embodied planning benchmarks based on ALFWorld~\citep{shridhar2021alfworld} and Virtualhome~\cite{puig2018virtualhome}.



\section{Related Work}

%Here we summarize prior work on transfer learning and property inference.

%\shortsection{Transfer Learning}
%%Transfer learning reuses features learned by pre-trained models for new tasks, with the pretext that inherent similarities in the generic features will be useful for the downstream tasks and hence reducing their cost of downstream training. Specifically, the downstream model trainer will use a pre-trained upstream model as the starting point for the downstream training, with inclusion of (or replacement with) the task-specific classification layer/module. The downstream model is then trained by either updating all layers of the model (including ones reused from upstream model) or freezing some earlier layers of the reused parts as the ``feature extractor'' and only updating the rest. The latter approach is more popular as the reused feature extractors can already learn useful feature representations and the training cost is also much lower and affordable for individuals with limited computational resources. We study the vulnerability of the latter transfer learning approach in this paper. 


%\shortsection{Transfer Learning} 
Several works have demonstrated risks associated with transfer learning across a variety of attack goals. Wang et al.~\cite{wang2018great} and Yao et al.~\cite{yao2019latent} consider manipulating the upstream model such that the fine-tuned downstream models contain backdoors, misclassifying test inputs that contain predefined backdoor triggers. These transfer manipulations are tailored to their particular attack goals and cannot be applied for the property inference goal considered in this paper. Zou et al.~\cite{zou2020privacy} study the threat of membership inference attacks on transfer learning, but with normally trained upstream models.  
%\dnote{its clear that the goals are different for these attacks, but how similar are the methods?} \ynote{similarity of the methods? more details about the methods? do not know what is expected here}
%In contrast, we investigate the possibility of boosting the effectiveness of property inference by manipulating the upstream model training. % Schuster et al.~\cite{schuster2020humpty} show that the attacker can modify the corpus on which the word embedding is trained such that the downstream NLP models which use that embedding will behave abnormally.

%\shortsection{Property Inference}
The risk of property inference was introduced by Ateniese et al.~\cite{ateniese2015hacking}, % introduces the threat of inferring properties of the training data from pre-trained models, 
and several subsequent works have developed property inference (also known as distribution inference) attacks~\cite{Wang2022GroupPI, suri2022formalizing, Jurez2022BlackBoxAF, Hartmann2022DistributionIR}.
% Ganju et al.~\cite{ganju2018property} and Suri and Evans~\cite{suri2022formalizing} 
These works study property inference against normally trained models, and they launch attacks using a variety of black-box and white-box attacks. All the white-box attacks use meta-classifiers, which take the permutation-invariant representation~\cite{ganju2018property} of the model parameters as the features. We use the state-of-the-art white-box attack~\cite{suri2022formalizing} in our experiments.
%We will use the state-of-the-art white-box method proposed by Ganju et al.~\cite{ganju2018property} and later extended by suri et al.~\cite{suri2022formalizing} in this paper.
%\dnote{do we use these attacks?} 
Melis et al.~\cite{melis2019exploiting} and Zhang et al.~\cite{zhang2021leakage} focus on property inference in distributed training scenarios. In their settings, the attacker is a participant in the global model training and conducts property inference using meta-classifiers that are trained on model outputs or gradients. Similarly, Suri et al.~\cite{suri2022subject} focus on federated learning settings where the attacker is a participant (or the central server) that utilizes black-box attacks for inferring membership of data from particular subjects. %\dnote{if we use black-box attacks, explain which ones, or how ours are related to previous ones} 
For our experiments, We improve the black-box meta-classifier proposed by Zhang et al.~\cite{zhang2021leakage} using the ``query tuning'' technique in Xu et al.~\cite{xu2019detecting}. 

The closest works to ours are Chase et al.~\cite{saeed} and Chaudhari et al.~\cite{Chaudhari2022SNAPEE}, which both consider a scenario where the attacker can manipulate some of the training data of the model to induce a model that significantly increases property inference risk.
% \dnote{it enables precise property inference attacks?}.
These works assume an adversary with the ability to poison the victim's training data, while the adversary in our scenario has no access to the victim's training data, and therefore, their methods are not applicable.
% \dnote{example how different from ours, and why the methods are not applicable}
%Thus, their methods are not applicable to our transfer learning scenario.
%Their methods rely on inducing certain behavior correlated with the properties to be inferred, and thus are not applicable to our transfer learning scenario. \anote{Still a bit unclear why that is the case.}
%
There are also works similar to ours that leverage ``adversarial initializations'' for attack purposes.
% \cite{grosse2019adversarial, boenisch2021curious, wen2022fishing, fowl2021robbing}.
Grosse et al.~\cite{grosse2019adversarial} focus on scenarios where the attacker can control the parameter initialization of a model, and demonstrate that the attacker can use special initializations to damage the performance of the trained model. %This attack is orthogonal to ours.
Other works \cite{boenisch2021curious, wen2022fishing, fowl2021robbing} show that the malicious central server in a federated learning protocol can reconstruct some training samples via falsifying the global model in some training rounds and then analyzing the submitted gradients. These kinds of attacks do not apply to our transfer-learning scenario since the attacker cannot access the downstream gradients, and can only manipulate the upstream training.

\iffalse %%%%%%%%%%%%%%%%%%%%%%%%%%%%%%%%

In this section, we provide the background and also the summary of prior attacks on transfer learning (Section~\ref{sec:transfer_learning}) and property inference (Section~\ref{sec:property_inference}). Then, we introduce the closely related manipulation attacks against machine learning models to boost different privacy risks in Section~\ref{sec:active_inference_attacks}.

%\anote{Do we really need a dedicated section for this? It's barely 2 paragraphs right now.}

%\dnote{the most closely related work to ours are works that attempt to amplify inference attacks by poisoning models, the two most relevant I know of are \url{https://www.computer.org/csdl/proceedings-article/sp/2022/131600b569/1CIO8nmuota} and \url{https://arxiv.org/abs/2204.00032}, but need to look thoroughly for others. We should definitely be describing this and relating it to our work, probably in the introduction. Most of what is here is Background, but should be clear what this section is for (not muddling background and related work)}

\subsection{Transfer Learning} \label{sec:transfer_learning}
Transfer learning reuses features learned by pre-trained models for new tasks, with the pretext that inherent similarities in generic features can be useful for downstream tasks, thus reducing the cost of downstream training. Specifically, the downstream model trainer uses a pre-trained upstream model as the starting point for downstream training, with the inclusion (or replacement) of task-specific classification layers/modules. The downstream model is then trained by either updating all layers of the model (including ones reused from the upstream model) or freezing some earlier layers of the reused parts as the ``feature extractor'' and only updating the rest. The latter approach is more popular as the reused feature extractors can already learn useful feature representations and the training cost is also much lower and affordable for individuals with limited computational resources. We study the vulnerability of the latter transfer learning approach in this paper. 
%mainly in two ways:  1) all the layers (including ones reused from ) and tune the full model; the other one is to freeze some earlier layers of the model as the feature extractor and only tune the rest later layers. The second update strategy could achieve better efficiency since the frozen layers can already produce meaningful feature representations~\cite{wang2018great,yao2019latent}, and we will study the transfer learning using this strategy. 

Recently, various attacks have been proposed for the transfer learning setting, but with different attack goals from ours. Wang et al.~\cite{wang2018great} generate adversarial examples against black-box student models that transfer knowledge from publicly available teacher models without repeated queries. Yao et al.~\cite{yao2019latent} propose to manipulate the upstream model such that the downstream models derived from the upstream model contain backdoors, which would misclassify test inputs that contain some predefined backdoor triggers. Zou et al.~\cite{zou2020privacy} study the threat of membership inference attacks on transfer learning and the upstream models are trained normally. In contrast, we investigate the possibility of boosting the effectiveness of property inference by manipulating the upstream model training. Schuster et al.~\cite{schuster2020humpty} show that the attacker can modify the corpus on which the word embedding is trained such that the downstream NLP models which use that embedding will behave abnormally.

%This additionally allows model trainers to achieve satisfactory performance with limited training samples, leading to reduced computational costs. The most common approach reuses parameters in the earlier layers of the pre-trained model, either by fixing them as the feature extractor or just using them for initialization, to conduct downstream training.

\subsection{Property Inference} \label{sec:property_inference}

\shortsection{Property Inference Attacks} In property inference attacks, the adversary aims to infer some sensitive properties of some data, given a model trained on it. For example, the adversary may be interested in sensitive properties like the presence of people of a specific race in the dataset~\cite{ateniese2015hacking, melis2019exploiting}), or even be curious about the 
the statistics of the training set (e.g, the ratio of people with a specific gender~\cite{saeed, ganju2018property, suri2022formalizing, zhang2021leakage}).


Ateniese et al.~\cite{ateniese2015hacking} were the first to identify the threat of inferring properties of the training data from pre-trained models. Ganju et al.~\cite{ganju2018property} and Suri and Evans~\cite{suri2022formalizing} 
study property inference against normally trained models, and they launch attacks using white-box meta-classifiers, which utilize the permutation-invariance representation~\cite{ganju2018property} of the model parameters, while other works focus on distributed training~\cite{zhang2021leakage} where the attacker is a participant in the global model training and conducts property inference using meta-classifiers trained on model outputs. Similarly, Suri et al.~\cite{suri2022subject} focus on federated learning, where the attacker is a participant (or the central server) that utilizes black-box attacks for inferring membership of data from particular subjects. Chase et al.~\cite{saeed} propose an active property inference attack for data poisoning scenarios, which we will cover and compare to in Section~\ref{sec:active_inference_attacks}.

%The closest work to ours are by Chase et al.~\cite{saeed} and Tramer et al.~\cite{tramer2022truth}. In their work, the attacker can manipulate some of the training data of the model such that a model trained (from scratch) on the poisoned data has an increased inference risk. However, their methods are not applicable to the transfer learning scenario. 
%In this work, we will focus on the property inference in transfer learning scenarios in which the attacker releases the upstream model and infer sensitive properties of the downstream models tuned from that upstream model.
% 

\shortsection{Defenses}
Defending against property inference attacks is an open problem. There are no studies in the current literature on active adversaries, and only a couple on passive ones. Ma et. al.~\cite{ma2021nosnoop} propose a defense against property inference attacks on data batches in the  collaborative learning setting. However, adversaries in the transfer-learning setting do not have access to batch-wise gradients of the downstream trainer. Chen and Ohrimenko~\cite{chen2022protecting} utilize mechanisms that add carefully-crafted noise to features to provide theoretical guarantees against inference adversaries, but focus on query-based access to the underlying dataset, not a machine learning model trained on it. These existing defenses thus do not apply to our threat model.

%propose a framework that reduces property inference to Boolean functions of individual members, posing the ratio of members satisfying the given function in a dataset as the property. These property inference attacks have since then been proposed as distribution inference attacks~\cite{suri2022formalizing}, presenting such attacks as inferring properties of the distributions used to sample datasets, differentiating them from exact inference attacks like dataset inference~\cite{maini2021dataset}. Nearly all property inference attacks use meta-classifiers to perform inference: training models on versions of datasets with and without the target property, followed by training a meta-classifier on top of these classifiers's model representations. These representations can take several forms: using model weights themselves with permutation-invariance~\cite{ganju2018property}, or model activations or logits for a generated set of query points~\cite{xu2019detecting}. However, the capability of such approaches is limited: the most that these attacks have been shown to work is medium-sized convolutional networks on the CelebA dataset~\cite{suri2022formalizing}.


\subsection{Active Privacy Attacks} \label{sec:active_inference_attacks}
% Perhaps the closely related works to ours as ones that proactively enhance the effectiveness of privacy attacks by manipulating the model training process in certain ways~\cite{saeed, melis2019exploiting, nasr2019comprehensive, tramer2022truth}. 
%shown that the adversary can, by using proactive ways, achieve stronger attacks that infer private information from deep learning systems~\cite{nasr2019comprehensive, melis2019exploiting, tramer2022truth, saeed}. In this section, we introduce the ones that are close to ours.

In the decentralized federated learning training, by submitting specially crafted gradients to the central server, malicious agents can increase membership inference risk~\cite{nasr2019comprehensive} and property inference risks~\cite{melis2019exploiting} of other benign agents' training data. However, these attacks do not apply to transfer learning scenario, as the attacker cannot control model gradients of downstream training. In the centralized setting, researchers propose attacks to poison the victim's training data such that the impacts of attribute inference and membership inference~\cite{tramer2022truth} and property inference~\cite{saeed} attacks are amplified on the poisoned model.
The ability to poison the victim's data is a threat model orthogonal to ours, since we have no access to the victim's downstream data. While there is scope to combine such approaches for stronger attacks (albeit with stronger access assumptions), we choose to focus on the scenario with no read/write access to the victim's data.

\fi %%%%%%%%%%%%%%%%%%%%%%%%%%%%%%%%

\section{Linear Shortcut Across Blocks}
\label{sec:layer_jump}

To use a hidden representation from layer $\ell<L$ as a final representation, we propose to cast it using linear regression, while skipping the computation in-between these layers. More generally, this approach can be applied to cast any $\ell$-th hidden representation to any subsequent layer $\ell'>\ell$.


\subsection{Method}
\label{subsec:methodology_linear_shortcut}

Given a source layer $\ell$ and a target layer $\ell'$ such that $0 \leq \ell < \ell' \leq L$, our goal is to learn a mapping
%$A_{\ell', \ell} \in \mathbb{R}^{d_h \times d_h}$
from hidden representations at layer $\ell$ to those at layer $\ell'$. To this end, we first collect a set of corresponding hidden representation pairs $(h^\ell, h^{\ell'})$. Concretely, we run a set $\mathcal{T}$ of input sequences through the model, and for each input $s$, we extract the hidden representations $h_{i_s}^{\ell}, h_{i_s}^{\ell'}$, where $i_s$ is a random position in $s$.
Next, we learn a matrix $A_{\ell', \ell} \in \mathbb{R}^{d_h \times d_h}$ by fitting linear regression over $\mathcal{T}$, i.e., $A_{\ell', \ell}$ is a numerical minimizer for:
$$ A \mapsto \sum_{s \in \mathcal{T}} || A \cdot h_{i_s}^\ell - h_{i_s}^{\ell'} ||^2,$$ 
and define the mapping of a representation $h$ from layer $\ell$ to layer $\ell'$ as:
\begin{equation}
\label{eq:linear_jump}
    \matl{} (h) \coloneqq A_{\ell', \ell} \cdot h.
\end{equation}


\subsection{Baseline}
\label{subsec:baseline}

We evaluate 
% our method against 
the prevalent approach of ``reading'' hidden representations directly, without any transformation. 
Namely, the propagation of a hidden representation from layer $\ell$ to layer $\ell'$ is given by the identity function, dubbed \id{}:

$$ \idl{} (h) \coloneqq h.$$

% Notably, 
This baseline 
assumes that representations at different layers operate in the same linear space.

\subsection{Quality of Fit}
\label{subsec:experiments_r2}

We first evaluate our method by measuring how well the learned linear mappings approximate the representations at the target layer. To this end, we calculate the (coordinate-averaged) $r^2$-score of our mapping's outputs with respect to the representations obtained from a full inference pass, and compare to the same for the \id{} baseline.


\paragraph{Models.}

We use \gpt{} \cite{radford2019language}, a decoder-only auto-regressive LM, with $L = 48$, $d_h = 1600$, and \bert{} \cite{devlin-etal-2019-bert}, an encoder-only model trained with masked language modeling, with $L=24$, $d_h=1024$.
% \footnote{\label{footnote:hf}We use models and data from Huggingface \cite{wolf-etal-2020-transformers,lhoest-etal-2021-datasets}.}
%For masked token prediction, we use a masked LM head pre-trained for our \bert{} model.

% \footnote{Specifically, we use the Huggingface Transformers \cite{wolf-etal-2020-transformers} implementations of all these models.}

%\sy{We use \gpt{} \cite{radford2019language}, a decoder-only auto-regressive LM, coming in four scales; $\texttt{gpt2}$ ($L = 12$, $d_h = 768$), $\texttt{gpt2-medium}$ ($L = 24$, $d_h = 1024$), $\texttt{gpt2-large}$ ($L = 36$, $d_h = 1280$) and $\texttt{gpt2-xl}$ ($L = 48$, $d_h = 1600$). Also, we use \bert{} \cite{devlin-etal-2019-bert}, an encoder-only model trained with masked language modeling, coming in two scales;  \texttt{bert-base-uncased} ($L=12$, $d_h=768$) and \texttt{bert-large-uncased} ($L=24$, $d_h=1024$). For masked token prediction, we use masked LM heads pre-trained for our models. Specifically, we use the Huggingface Transformers \cite{wolf-etal-2020-transformers} implementations of all these models. The plots presented in this section are for $48$-layered \gpt{} and $24$-layered \bert{}.}

%\sy{We use \gpt{} \cite{radford2019language}, a decoder-only auto-regressive LM, in the Huggingface \cite{wolf-etal-2020-transformers} implementation\footnote{\url{https://huggingface.co/gpt2}}, coming in four scales; $\texttt{gpt2}$ ($L = 12$, $d_h = 768$), $\texttt{gpt2-medium}$ ($L = 24$, $d_h = 1024$), $\texttt{gpt2-large}$ ($L = 36$, $d_h = 1280$) and $\texttt{gpt2-xl}$ ($L = 48$, $d_h = 1600$). Also, we use \bert{} \cite{devlin-etal-2019-bert}, an encoder-only model trained with masked language modeling, in the Hugginface implementation, coming in two scales;  \texttt{bert-base-uncased}\footnote{\url{https://huggingface.co/bert-base-uncased}} ($L=12$, $d_h=768$) and \texttt{bert-large-uncased}\footnote{\url{https://huggingface.co/bert-large-uncased}} ($L=24$, $d_h=1024$). For masked token prediction, we use the \texttt{BertForMaskedLM} heads from Huggingface, pretrained for these models. The plots presented in this section are for $48$-layered \gpt{} and $24$-layered \bert{}.}

\paragraph{Data.}
We sample random sentences from Wikipedia,
% \footref{footnote:hf} 
collecting 9,000 (resp. 3,000) sentences for the training set $\mathcal{T}$ (resp. validation set $\mathcal{V}$).\footnote{We use sentences rather than full documents to simplify the analysis.}
%\sy{We use two data sources to evaluate our method. One is Wikiepdia \cite{lhoest-etal-2021-datasets}\footnote{\url{https://huggingface.co/datasets/wikipedia}}; we use \texttt{spaCy}\footnote{\url{https://spacy.io/}} to divide documents into sentences\footnote{We use sentences rather than full documents to simplify the analysis.}\footnote{We pick randomly a Wikipedia document and then pick randomly a sentence ending in a newline character in it. \sy{[maybe this footnote is not needed?]}}, collecting 9,000 (resp. 3,000) random sentences for the training set $\mathcal{T}$ (resp. validation set $\mathcal{V}$). The second is a news article sentences dataset, the 10K English 2020 news sentences corpus
% \footnote{\url{https://downloads.wortschatz-leipzig.de/corpora/eng_news_2020_10K.tar.gz}} from the Leipzig Corpora Collection \cite{goldhahn-etal-2012-building}, which we randomly divide into a training set $\mathcal{T}$ consisting of 9,000 examples and a validation set $\mathcal{V}$ consisting of 1,000 examples.
% We truncate sentences to the maximal token length allowed by the model \mg{do we ever need to truncate? a sentence has about 10 words and the max. input len is thousands} \sy{[I surely did not need to in Leipzig, but discovered (via a transformers runtime warning) that I do need to for some (probably a minority) of the Wikipedia sentences. This probably has to do with that it is not really ``sentences" necessarily, for example, I noticed that it has some listings or something like that (bulleted items)... So some minority might get very long I guess...]}.
For each example $s$, we select a random position $i_s$ and extract the hidden representations $h_{i_s}^{\ell}$ at that position from all the layers.
For \bert{}, we first replace the input token at position $i_s$ with a \mask{} token, as our motivation is interpreting predictions, which are obtained via masked tokens in \bert{} (see \S\ref{subsec:BERT}).
Thus, in this case, the hidden representations we consider
%in the case of \bert{}
are of \mask{} tokens only.
%As we observed highly similar results for the two data sources across all our experiments, throughout the paper we will mainly report results for Wikipedia (except for \S\ref{sec:robustness}, where we cross-validate).


\begin{figure}[t]
\includegraphics[scale=0.2]{figs/r2_scores_48.pdf}
% \includegraphics[width=\columnwidth]{figs/r2_scores_48.pdf}
\caption{The coordinate-averaged $r^2$-score of $\matl{}$ (left) and $\idl{}$ (right) (\gpt{}).}
\label{fig:r2_scores}
\end{figure}


\begin{figure}[t]
\setlength{\belowcaptionskip}{-10pt}
\includegraphics[scale=0.2]{figs/bertmask_r2_scores_24.pdf}
% \includegraphics[width=\columnwidth]{figs/bertmask_r2_scores_24.pdf}
\caption{The coordinate-averaged $r^2$-score of $\matl{}$ (left) and $\idl{}$ (right) (\bert{}).}
\label{fig:bertmask_r2_scores}
\end{figure}



\paragraph{Evaluation.}
For every pair of layers $\ell, \ell'$, such that $0 \leq \ell < \ell' \leq L$, we use the training set $\mathcal{T}$ to fit linear regression as described in \S\ref{subsec:methodology_linear_shortcut}, and obtain a mapping $\matl{}$. 
Next, we evaluate the quality of $\matl{}$ as well as of $\idl{}$ using the $r^2$-coefficient, uniformly averaged over all coordinates. Concretely, we compute the $r^2$-coefficient of each of the predicted representations $\matl{} (h_{i_s}^{\ell})$ and $\idl{} (h_{i_s}^{\ell})$ versus the true representations $h_{i_s}^{\ell'}$
over all $s \in \mathcal{V}$.
%as we vary $s \in \mathcal{V}$.
%for every $s \in \mathcal{V}$.



\paragraph{Results.}
Results for \gpt{} and \bert{} are presented in Figs.~\ref{fig:r2_scores} and~\ref{fig:bertmask_r2_scores}, respectively.
In both models, \mat{} consistently yields better approximations than \id{}, as it obtains higher $r^2$-scores (in blue) across the network. 
This gap between \mat{} and \id{} is especially evident in \bert{}, where \id{} completely fails to map the representations between most layers, suggesting that hidden representations are modified  substantially by every transformer block.
Overall, this highlights the shortcoming of existing practices to inspect representations in the same linear space, and the gains from using our method to approximate future layers.
% in the network.
\section{Linear Shortcut for Language Modeling}
\label{sec:prediction}

We saw that our method approximates future hidden representations substantially better than a naive propagation. 
In this section, we will show that this improvement also translates to better predictive abilities from earlier layers. Specifically, we will use our method to estimate how often intermediate representations encode the final prediction, in the context of two fundamental LM tasks; next token prediction and masked token prediction.

\paragraph{Evaluation Metrics.}
Let $h, h' \in \mathbb{R}^{d_h}$ be a final representation and a substitute final representation obtained by some mapping, and denote by $\delta (h), \delta (h') \in \mathbb{R}^{d_v}$ their corresponding output probability distributions (obtained through projection to the output vocabulary -- see details below). 
We measure the prediction quality of $h'$ with respect to $h$ using two metrics:
\begin{itemize}
[leftmargin=*,topsep=1pt,parsep=1pt]
    \item \textbf{Precision@$k$} ($\uparrow$ is better): This checks whether the token with the highest probability according to $\delta(h')$ appears in the top-$k$ tokens according to $\delta(h)$. Namely, we sort $\delta(h)$ and assign a score of $1$ if $\arg\max(\delta(h'))$ appears in the top-$k$ tokens by $\delta(h)$, and $0$ otherwise.
    
    \item \textbf{Surprisal} ($\downarrow$ is better): We measure the minus log-probability according to $\delta(h)$, of the highest-probability token according to $\delta(h')$. Intuitively, low values mean that the model sees the substitute result as probable and hence not surprising.
\end{itemize}

\noindent We report the average Precision@$k$ and Surprisal over the validation set $\mathcal{V}$.



\subsection{Next Token Prediction}
\label{subsec:next_token_prediction_task}

Auto-regressive LMs output for every position a probability distribution over the vocabulary for the next token. Specifically, the output distribution for every position $i$ is given by $\delta (h_i^L)$, where:
\begin{equation}\label{eq:output_distribution}
    \delta (h) = \texttt{softmax} ( E^\top \cdot h) \in \mathbb{R}^{d_v}
\end{equation}
For some LMs, including \gpt{}, a layer normalization $\texttt{ln\_f}$ is applied to the final layer representation before this conversion (i.e., computing $\delta (\texttt{ln\_f}(h))$ rather than $\delta (h)$).

Recall that our goal is to measure how well this distribution can be estimated from intermediate representations, i.e. estimating $\delta (h_i^L)$ from $\delta (h_i^\ell)$ where $\ell<L$. To this end, we first run examples from the validation set through the model, while extracting for each example $s$ the hidden representation of a random position $i_s$ at every layer. Next, we apply our mappings $\matlL{}$ and the $\idlL{}$ baseline to cast the hidden representations of every layer $\ell$ to final layer substitutes (see \S\ref{sec:layer_jump}). Last, for each layer, we convert its corresponding final-layer substitute to an output distribution (Eq.~\ref{eq:output_distribution}) and compute the average Precision@$k$ (for $k=1,5,10$) and Surprisal scores with respect to the final output distribution, over the validation set.

\paragraph{Results.}
Figs.~\ref{fig:pre} and~\ref{fig:surp} show the average Precision@$k$ and Surprisal scores per layer in $48$-layered \gpt{}, respectively (the plots for the other \gpt{} models are presented in \S\ref{sec:app_scale}). Across all layers, \mat{} outperforms \id{} in terms of both scores, often by a large margin (e.g. till layer $44$ the Precision@$1$ achieved by \mat{} is bigger than that of $\id{}$ by more than $0.2$). 
This shows that linear mappings enable not just better estimation of final layer representations, but also of the predictions they induce. Moreover, the relatively high Precision@$k$ scores of \mat{} in early layers ($0.62$-$0.82$ for $k=10$, $0.52$-$0.74$ for $k=5$, and $0.28$-$0.45$ for $k=1$) suggest that early representations already encode a good estimation of the final prediction. Also, the substantially lower Surprisal scores of \mat{} compared to \id{} imply that our method allows for a more representative reading into the layer-wise prediction-formation of the model than allowed through direct projection to the vocabulary.

\begin{figure}[t]
\centering
\includegraphics[scale=0.4]{figs/pre_48.pdf}
\caption{Precision@$k$ ($k = 1,5, 10$) of $\matlL{}$ and $\idlL{}$ for next token prediction in $48$-layered \gpt{}.}
\label{fig:pre}
\end{figure}

\begin{figure}[t]
\centering
\includegraphics[scale=0.35]{figs/surp_48.pdf}
\caption{Surprisal for $\matlL$ and the baseline $\idlL{}$ ($48$-layered \gpt{} next token prediction task). A 95\% confidence interval surrounds the lines.}
\label{fig:surp}
\end{figure}

\subsection{Masked Token Prediction}
\label{subsec:BERT}

We now conduct the same experiment for the task of masked language modeling, where the model predicts a probability distribution of a masked token in the input rather than the token that follows the input. Unlike next token prediction, where the output distribution is computed from representations of varying input tokens, in masked token prediction the output is always obtained from representations of the same input token (i.e. \texttt{[MASK]}).

For this experiment, we use \bert{}, on top of which we use a pretrained masked language model head $\delta$; given a token sequence $s$, a \mask{} token inside it and its final representation $h$, $\delta (h) \in \mathbb{R}^{d_v}$
 is a probability distribution over tokens giving the model's assessment
 of the likelihood of tokens to be fitting in place of the \mask{} token in $s$.


\begin{figure}[t]
\centering
\includegraphics[scale=0.4]{figs/bertmask_pre_24.pdf}
\caption{Precision@$k$ ($k = 1,5, 10$) for  $\matlL{}$ and the baseline $\idlL{}$ ($24$-layered \bert{} masked token prediction task).}
\label{fig:bertmask_pre}
\end{figure}

\begin{figure}[t]
\centering
\includegraphics[scale=0.35]{figs/bertmask_surp_24.pdf}
\caption{Surprisal for $\matlL{}$ and the baseline $\idlL{}$ ($24$-layered \bert{} masked token prediction task). A 95\% confidence interval surrounds the lines.}
\label{fig:bertmask_surp}
\end{figure}

\paragraph{Results.}
Figs.~\ref{fig:bertmask_pre} and~\ref{fig:bertmask_surp} present the average Precision@$k$ and Surprisal scores per layer in $24$-layered \bert{} (the plots for the $12$-layered \bert{} model are presented in \S\ref{sec:app_scale}), overall showing trends similar to those observed for next token prediction in \gpt{} (\S\ref{subsec:next_token_prediction_task}). This is despite the differences between the two tasks and the considerable architectural differences between \bert{} and \gpt{}.
Notably, the superiority of \mat{} over \id{} in this setting is even more prominent; 
while \mat{}'s precision is between $0.2-0.6$ in the first ten layers (Fig.~\ref{fig:bertmask_pre}), \id{}'s precision for all values of $k$ is close to zero, again strongly indicating that our method allows for better reading into early layer hidden representations. 
More generally, \mat{} improves the Precision@$1$ of \id{} by more than $17\%$ at most layers, and unveils that a substantial amount of predictions ($>25\%$ starting from layer $3$) appear already in the very first layers.
Interestingly, the (rough) divide between the first half of layers and last half of layers for $\id{}$ in Figs.~\ref{fig:bertmask_pre},~\ref{fig:bertmask_surp} seems to align with the two-hump shape of the blue region for $\mat{}$ in Fig.~\ref{fig:bertmask_r2_scores}.

\paragraph{Analysis.}
We manually compare the predictions of our mapping $\matlL{}$ with $\idlL{}$, for a $24$-layered \bert{} model.  Concretely, we select 50 random sentences from the Leipzig dataset. Next, for each layer $\ell$, we manually analyze how many of the top-$5$ tokens according to $\matlL{}$ and $\idlL{}$ fit into context. We consider a token to fit into context if it is grammatically plausible within the sentence (see Tab.~\ref{tab:manual} for concrete examples).
In the resulting $1250$ instances (i.e. $50$ sentences $\times$ $25$ representations), we observe a substantially higher plausibility rate of $85.36\%$ for \mat{} compared to $52.8\%$ for \id{}. In fact, only in less than $4.3\%$ of the instances there are more plausible tokens among the top-$5$ tokens according to \id{} than among the top-$5$ tokens according to \mat{}, further supporting the Surprisal results above.

\begin{table*}
\footnotesize
\setlength{\belowcaptionskip}{-15pt}
\begin{tabular}{p{0.3\linewidth}ccccc}
& $\texttt{id}_{4 \rightarrow 24}$ & $\texttt{mat}_{4 \rightarrow 24}$ & $\texttt{id}_{12 \rightarrow 24}$ & $\texttt{mat}_{12 \rightarrow 24}$ & $\texttt{id}_{24 \rightarrow 24}$ \\ \midrule
\multirow{5}{=}{aldridge had shoulder surgery in \mask{}.} & fellowship & \tcbox{time} & cyclist & \tcbox{2009} & \tcbox{september} \\
& employment & \tcbox{it} & emergencies & \tcbox{2008} & \tcbox{november} \\
& agreement & her & seniors & \tcbox{2010} & \tcbox{december} \\
& \#\#ostal & them & cycling & \tcbox{2006} & \tcbox{august} \\
& \#\#com & work & \tcbox{pennsylvania} & \tcbox{2007} & \tcbox{july} \\ \midrule
\multirow{5}{=}{on your next view you will be asked to \mask{} continue reading.} & \#\#com & be & be & be & \tcbox{please} \\
& accreditation & get & undergo & \tcbox{please} & \tcbox{simply} \\ 
& $	\copyright$ & go & spartans & help & \tcbox{also} \\ 
& fellowship & \tcbox{help} & seniors & \tcbox{simply} & \tcbox{again} \\ 
& summer & have & * & say & \tcbox{immediately} \\ \bottomrule
\end{tabular}
\caption{Examples of top-$5$ predictions at layers $4$, $12$ and $24$, under the mappings $\matlL{}$ and $\idlL{}$, for a $24$-layered \bert{} model. Grammatically plausible predictions (according to a human annotator) are marked in \tcbox{blue}. Note that at layer $24$ the predictions of $\matlL{}$ and $\idlL{}$ are the same (by definition).} 
\label{tab:manual}
\end{table*}

\section{Implication to Early Exiting}
\label{sec:applications}

%The fact that it is often possible to approximate
The possibility of approximating
the final prediction already in the early layers has important implications for efficiency; applying our linear mapping instead of executing transformer blocks of quadratic time complexity, could save a substantial portion of the computation. In this section, we demonstrate this in the context of early exiting.

When 
% performing transformer model inference under 
using an early exit strategy \cite{schwartz-etal-2020-right, xin-etal-2020-deebert, schuster2022confident}, one aims at deciding dynamically at which layer to stop the computation and ``read'' the prediction from the hidden representation of that layer.
More precisely, under a confidence measure paradigm, one decides to stop the computation for a position $i$ at layer $\ell$ based on a confidence criterion, that is derived from casting the hidden representation $h_i^\ell$ as a final-layer representation and converting it to an output probability distribution. Specifically, following \citet{schuster2022confident}, a decision to exit is made if the difference between the highest and the second highest probabilities is bigger than $$ 0.9 \cdot \lambda + 0.1 \cdot {\rm exp} (-4 i / N),$$
where $N$ is the average length of the input until position $i_s$ for $s \in \mathcal{V}$, and $\lambda$ is a hyper-parameter.

\begin{figure}[t]
\setlength{\belowcaptionskip}{-10pt}
\centering
\includegraphics[width=\columnwidth]{figs/ee_gpt2bert.pdf}
\caption{Precision@$1$ with early exit and ``fixed exit'', applied to the $24$-layer \gpt{} for next token prediction (left) and the $24$-layer \bert{} for masked token prediction (right). Varying the confidence parameter $\lambda$, the $x$-coordinate is the average number of layers processed before an early exit decision is reached.}
\label{fig:ee_gpt2bert}
\end{figure}

\quash{
\begin{figure}[t]
\setlength{\belowcaptionskip}{-10pt}
\centering
\includegraphics[scale=0.35]{figs/ee_pre1_24.pdf}
\caption{Precision@$1$ for the various early exit methods, and previous ``fixed exit'' methods for comparison ($24$-layer \gpt{} next token prediction task). Varying the confidence parameter $\lambda$, the $x$-coordinate is the average number of layers processed before an early exit decision is reached.}
\label{fig:ee_pre1}
\end{figure}
}

\paragraph{Experiment.}
We assess the utility of our mapping $\matlL{}$ for early exit as a plug-and-play replacement for $\idlL{}$, through which intermediate representations are cast into final-layer representations.
We use \gpt{} for the next token prediction and \bert{} for masked token prediction (both with 24 layers).
We run each of the models over the validation set examples, while varying the confidence parameter $\lambda$ and using either $\idlL{}$ or $\matlL{}$ for casting intermediate representations.
Furthermore, we compare these early exit variants to the ``fixed exit'' strategy from \S\ref{sec:prediction}, where the computation is stopped after a pre-defined number of layers rather than relying on a dynamic decision.
We evaluate each variant in terms of both prediction's accuracy, using the Precision@$1$ metric (see \S\ref{sec:prediction}), and efficiency, measured as the average number of transformer layers processed during inference.


\paragraph{Results.}
%Figs.~\ref{fig:ee_pre1} and~\ref{fig:bertmask_ee_pre1}
Fig.~\ref{fig:ee_gpt2bert}
plots the average Precision@$1$ score against the average number of layers processed, for $24$-layer \gpt{} and $24$-layer \bert{}. For both models, under an early exit strategy our mapping \mat{} again provides a substantial improvement over \id{}.
For example, aiming at $95\%$ average precision, \mat{} saves $\sim3.3$ ($13.8$\%) layers in \gpt{} compared to only $\sim1.4$ ($5.9$\%) layers by \id{}, and $\sim4.8$ ($20$\%) layers in \bert{} versus $\sim3.5$ ($14.6$\%) layers by \id{}.
These results highlight the potential gains prominent early exit methods can obtain by using our method.
Notably, in both models and for each of the mapping methods, early exit obtains better results than fixed layer exit, as expected. 

\quash{
\begin{figure}[t]
\setlength{\belowcaptionskip}{-10pt}
\centering
\includegraphics[scale=0.35]{figs/bertmask_ee_pre1_24.pdf}
\caption{Precision@$1$ for the various early exit methods, and previous ``fixed exit'' methods for comparison ($24$-layer \bert{} masked token prediction task). Varying the confidence parameter $\lambda$, the $x$-coordinate is the average number of layers processed before an early exit decision is reached.}
\label{fig:bertmask_ee_pre1}
\end{figure}
}
\section{Linear Shortcut Across Sub-Modules}
\label{sec:submodules}

% Our experiments show that
% , despite the commonly-applied simplification by interpretability works, transformer layers do not operate in the same linear space and 
% there is a major gap in approximating future representations using an identity mapping (\S\ref{sec:layer_jump}, \S\ref{sec:prediction}).
% Here, 
In this section, we investigate whether discrepancies across layers result from specific sub-modules or are a general behaviour of all sub-modules in the network.  
This is done by extending our approach to test how well particular components in transformer blocks can be linearly approximated. 


\paragraph{Method.}

Consider \gpt{} for definiteness, then:
% we have 
$$ \texttt{b}_{\ell} = \texttt{b}_{\ell}^{\texttt{ffn}} \circ \texttt{b}_{\ell}^{\texttt{attn}}$$ 
% with
\begin{equation}\label{eq:attn} \texttt{b}^{\texttt{attn}}_{\ell} (H) = \texttt{attn}_{\ell} (\texttt{ln1}_{\ell} (H)) + H,\end{equation} 
where $\texttt{attn}_{\ell}$ is
%a multi-head self-attention
a MHSA
layer and \texttt{ln1} is a layer normalization (LN), and 
$$ \texttt{b}^{\texttt{ffn}}_{\ell} (H) = \texttt{ffn}_{\ell} (\texttt{ln2}_{\ell} (H)) + H,$$  
where $\texttt{ffn}_{\ell}$ is
%a feed-forward network
an FFN
layer and $\texttt{ln2}$ is a LN.
\quash{
Given a block $\texttt{b}_\ell$ and one of its sub-modules $\texttt{ln1}_\ell, \ \texttt{attn}_\ell, \ \texttt{ln2}_\ell$, or $\texttt{ffn}_\ell$, we fit linear regression approximating the output of the sub-module given its input and then use it in order to define mappings, as we now describe.
}
Given a block $\texttt{b}_\ell$ and one of its sub-modules $\texttt{ln1}_\ell, \ \texttt{attn}_\ell, \ \texttt{ln2}_\ell$, or $\texttt{ffn}_\ell$, we fit linear regression approximating the output of the sub-module given its input, and then use it to define mappings $\matattnl{}$, $\matlnl{}$ and $\matffl{}$.
%We provide the definition of $\matattnl{}$ below, and that of the other two in App. \ref{sec:app_submodule_skip_description}.
We provide the formal definitions of these mappings in App. \ref{sec:app_submodule_skip_description}.
\iffalse
\paragraph{$\matattnl{}$.}
%Illustrating this on $\texttt{attn}_\ell$ for definiteness,
For an input $s$, let $v^\ell_{i_s}$ be the vector at position $i_s$ in the output of $\texttt{attn}_\ell (\texttt{ln1}_\ell (H^{\ell - 1}))$. We denote by $A_\ell^{\texttt{attn}} \in \mathbb{R}^{d_h \times d_h}$ the matrix numerically minimizing 
$$ A \mapsto \sum_{s \in \mathcal{T}} || A \cdot \texttt{ln1}_\ell (h^{\ell-1}_{i_s}) - v^\ell_{i_s}||^2,$$
and define an attention sub-module replacement (Eq.~\ref{eq:attn}) by $$
\texttt{b}^{\overline{\texttt{attn}}}_\ell (h) \coloneqq A_{\ell}^{\texttt{attn}} \cdot \texttt{ln1}_\ell (h) + h. $$
We then define a mapping between two layers ${\ell \rightarrow \ell'}$ by:
$$ \matattnl{} (h) \coloneqq $$
$$ \texttt{b}^{\texttt{ffn}}_{\ell'} ( \texttt{b}^{\overline{\texttt{attn}}}_{\ell'} ( \ldots (\texttt{b}^{\texttt{ffn}}_{\ell+1} ( \texttt{b}^{\overline{\texttt{attn}}}_{\ell+1} (h)))\ldots)).$$ 
Namely, when applying each $\ell''$-th block, $\ell < \ell'' \leq \ell'$, we replace its attention sub-module $\texttt{attn}_{\ell''}$ by its linear approximation.
%In an analogous way, we consider the mappings $\matffl{}$ and $\matlnl{}$, where in the latter we perform the linear shortcut both for \texttt{ln1} and for \texttt{ln2} (see~\S\ref{sec:app_submodule_skip_description} for precise descriptions).
Importantly, unlike the original attention module, the approximation $\texttt{b}^{\overline{\texttt{attn}}}_\ell$ operates on each position independently, and therefore applying $\matattnl{}$ disables any contextualization between the layers $\ell$ and $\ell'$. Note that this is not the case for $\matffl{}$ and $\matlnl{}$, which retain the self-attention sub-modules and operate contextually.
\fi

\paragraph{Evaluation.}


We analyze the $24$-layered \gpt{}, and proceed completely analogously to \S\ref{subsec:next_token_prediction_task}, evaluating the Precision@$1$ and Surprisal metrics for the mappings $\matattnlL{}$, $\matfflL{}$ and $\matlnlL{}$.

\begin{figure}[t]
\setlength{\belowcaptionskip}{-0pt}
\centering
%\includegraphics[scale=0.2]
\includegraphics[width=\columnwidth]{figs/parts_presurp_24.pdf}
\caption{Precision@$1$ and Surprisal for the various sub-module linear mappings, and $\matlL{}$ for comparison ($24$-layer \gpt{} next token prediction task). A 95\% confidence interval surrounds the Surprisal lines.}
\label{fig:parts_presurp}
\end{figure}

\quash{
\begin{figure}[t]
\centering
\includegraphics[scale=0.4]{figs/parts_pre1_24.pdf}
\caption{Precision@$1$ for the various sub-module linear shortcut mappings, and the mapping $\matlL{}$ for comparison (\gpt{} next token prediction task).}
\label{fig:parts_pre1}
\end{figure}

\begin{figure}[t]
\centering
\includegraphics[scale=0.35]{figs/parts_surp_24.pdf}
\caption{Surprisal for the various sub-module linear shortcut mappings, and the mapping $\matlL{}$ for comparison (\gpt{} next token prediction task). A 95\% confidence interval surrounds the lines.}
\label{fig:parts_surp}
\end{figure}
}

\paragraph{Results.}
Fig.~\ref{fig:parts_presurp} shows the average Precision@$1$ and Surprisal scores per layer.
From a certain layer (\textasciitilde$7$), all sub-module mappings achieve better results than the full-block mapping $\matlL{}$. Thus, it is not just the cumulative effect of all the sub-modules in the transformer block that is amenable to linear approximation, but also individual sub-modules can be linearly approximated. 
Furthermore, the linear approximation of attention sub-modules is less harmful than that of the FFN or LN sub-modules. 
% Hypothetically, 
A possible reason is that the linear replacement of FFN or LN ``erodes'' the self-attention computation after a few layers. 
Moreover, the good performance of $\matattnlL{}$ suggests that contextualization often exhausts itself in early layers; speculatively, it is only in more delicate cases that the self-attention of late layers adds important information. Last, remark the sharp ascent of the scores for layer normalization in layers $5$-$8$, for which we do not currently see a particular reason. To conclude, we see that the possibility of linear approximation permeates
%the various
transformer components.


\section{Related Work}

Recently, there was a lot of interest in utilizing intermediate representations in transformer-based LMs, both for interpretability and for efficiency.

In the direction of interpretability, one seeks to understand the prediction construction process of the model \cite{tenney-etal-2019-bert, voita-etal-2019-bottom}.

More recent works use mechanistic interpretability and view the inference pass as a residual stream of information \cite{dar2022analyzing,geva-etal-2022-transformer}. Additionally, there are works on probing, attempting to understand what features are stored in the hidden representations \cite{adi2017finegrained, conneau-etal-2018-cram,liu-etal-2019-linguistic}. Our work is different in that it attempts to convert intermediate representations into a final-layer form, which is interpretable by design.

In the direction of efficiency, there is the thread of work on early exit, where computation is cut at a dynamically-decided earlier stage \cite{schwartz-etal-2020-right,xin-etal-2020-deebert,schuster2022confident}. Other works utilize a fixed early stage network to parallelize inference \citep{leviathan2022fast, chen2023accelerating}. However, intermediate representations are directly propagated in these works, which we show is substantially worse than our approach. Moreover, our method requires training considerably less parameters than methods such as \citet{schuster-etal-2021-consistent}, that learn a different output softmax for each intermediate layer.  

More broadly, skipping transformer layers and analyzing the linearity properties of transformer components have been discussed in prior works \cite{Zhao2021of,mickus-etal-2022-dissect,wang-etal-2022-skipbert,lamparth2023analyzing}.


\section{Conclusion and Future Work}

We present a simple and effective method for enhancing utilization of hidden representations in transformer-based LMs, that uses 
pre-fitted context-free and token-uniform linear mappings.
Through a series of experiments on different data sources, model architectures and scales, we show that our method consistently outperforms the prevalent practice of interpreting representations in the final-layer space of the model, yielding better approximations of succeeding representations and the predictions they induce, thus allowing a more faithful interpretation of the model's prediction-formation.
We demonstrate the practicality of our method for improving computation efficiency, saving a substantial amount of compute on top of prominent early exiting approaches. 
Also, by extending our method to sub-modules, 
% more specifically the attention sub-modules, 
we observe that replacing a part of the transformer inference by a non-contextual linear computation often results in a small deterioration of the prediction.
This opens new research directions for improving model efficiency,
% and parallelizability.
% including breaking the computation into several parallelizable tasks.
including breaking the computation into parallel tasks.

\section*{Limitations}

Although we see in this work that there is more linear structure to transformer inference than could be explained solely by the residual connection, we do not elucidate a reason for that. We also do not try to formulate formal criteria according to which to judge, in principle, the quality of ways of short-cutting transformer inference in-between layers. In addition, our experiments cover only English data.


%\section*{Ethics Statement}
%Scientific work published at ACL 2023 must comply with the ACL Ethics Policy.\footnote{\url{https://www.aclweb.org/portal/content/acl-code-ethics}} We encourage all authors to include an explicit ethics statement on the broader impact of the work, or other ethical considerations after the conclusion but before the references. The ethics statement will not count toward the page limit (8 pages for long, 4 pages for short papers).

\section*{Acknowledgements}

We thank Tal Schuster for constructive comments.

% Entries for the entire Anthology, followed by custom entries
\bibliography{anthology,custom}
\bibliographystyle{acl_natbib}

\appendix

\section{Descriptions of $\matattn{}$, $\matff{}$ and $\matln{}$}
\label{sec:app_submodule_skip_description}

Here we detail the definitions of the mappings $\matattnl{}$, $\matffl{}$ and $\matlnl{}$ utilized in \S\ref{sec:submodules}.

\paragraph{Description of $\matattnl{}$.}
%Illustrating this on $\texttt{attn}_\ell$ for definiteness,
For an input $s$, let $v^\ell_{i_s}$ be the vector at position $i_s$ in the output of $\texttt{attn}_\ell (\texttt{ln1}_\ell (H^{\ell - 1}))$. We denote by $A_\ell^{\texttt{attn}} \in \mathbb{R}^{d_h \times d_h}$ the matrix numerically minimizing 
$$ A \mapsto \sum_{s \in \mathcal{T}} || A \cdot \texttt{ln1}_\ell (h^{\ell-1}_{i_s}) - v^\ell_{i_s}||^2,$$
and define an attention sub-module replacement (Eq.~\ref{eq:attn}) by $$
\texttt{b}^{\overline{\texttt{attn}}}_\ell (h) \coloneqq A_{\ell}^{\texttt{attn}} \cdot \texttt{ln1}_\ell (h) + h. $$
We then define a mapping between two layers ${\ell \rightarrow \ell'}$ by:
$$ \matattnl{} (h) \coloneqq $$
$$ \texttt{b}^{\texttt{ffn}}_{\ell'} ( \texttt{b}^{\overline{\texttt{attn}}}_{\ell'} ( \ldots (\texttt{b}^{\texttt{ffn}}_{\ell+1} ( \texttt{b}^{\overline{\texttt{attn}}}_{\ell+1} (h)))\ldots)).$$ 
Namely, when applying each $\ell''$-th block, $\ell < \ell'' \leq \ell'$, we replace its attention sub-module $\texttt{attn}_{\ell''}$ by its linear approximation.
%In an analogous way, we consider the mappings $\matffl{}$ and $\matlnl{}$, where in the latter we perform the linear shortcut both for \texttt{ln1} and for \texttt{ln2} (see~\S\ref{sec:app_submodule_skip_description} for precise descriptions).
Importantly, unlike the original attention module, the approximation $\texttt{b}^{\overline{\texttt{attn}}}_\ell$ operates on each position independently, and therefore applying $\matattnl{}$ disables any contextualization between the layers $\ell$ and $\ell'$. Note that this is not the case for $\matffl{}$ and $\matlnl{}$, which retain the self-attention sub-modules and operate contextually.

\paragraph{Description of $\matffl{}$.}
Let $v^\ell_{i_s}$ be the vector at position $i_s$ in the output of $\texttt{ln2}_{\ell} (\texttt{b}_\ell^{\texttt{attn}} (H^{\ell - 1}))$, for a given input $s$. We denote by $A_\ell^{\texttt{ffn}} \in \mathbb{R}^{d_h \times d_h}$ the matrix numerically minimizing 
$$ A \mapsto \sum_{s \in \mathcal{T}} || A \cdot v^{\ell}_{i_s} - \texttt{ffn}_{\ell} (v^\ell_{i_s})||^2,$$
and define a replacement of the feed-forward sub-module $\texttt{b}_{\ell}^{\texttt{ffn}}$ by $$ \texttt{b}^{\overline{\texttt{ffn}}}_\ell (H) \coloneqq A_{\ell}^{\texttt{ffn}} \cdot \texttt{ln2}_\ell (H) + H.$$
We then define a mapping between two layers ${\ell \rightarrow \ell'}$ by:
$$ \matffl{} (H) \coloneqq $$
$$ \texttt{b}^{\overline{\texttt{ffn}}}_{\ell'} ( \texttt{b}^{\texttt{attn}}_{\ell'} ( \ldots (\texttt{b}^{\overline{\texttt{ffn}}}_{\ell+1} ( \texttt{b}^{\texttt{attn}}_{\ell+1} (H))\ldots)).$$

\paragraph{Description of $\matlnl{}$.}
Let $v^\ell_{i_s}$ be the vector at position $i_s$ in the output of $\texttt{b}^{\texttt{attn}}_{\ell} (H^{\ell - 1})$, for a given input $s$. We denote by $A_\ell^{\texttt{ln1}} \in \mathbb{R}^{d_h \times d_h}$ the matrix numerically minimizing 
$$ A \mapsto \sum_{s \in \mathcal{T}} || A \cdot h^{\ell}_{i_s} - \texttt{ln1}_{\ell} (h^\ell_{i_s})||^2$$ and we denote by $A_\ell^{\texttt{ln2}} \in \mathbb{R}^{d_h \times d_h}$ the matrix numerically minimizing $$ A \mapsto \sum_{s \in \mathcal{T}} || A \cdot v^{\ell}_{i_s} - \texttt{ln2}_{\ell} (v^\ell_{i_s})||^2.$$ We define a replacement of the block $\texttt{b}^{\texttt{attn}}_{\ell}$ by \begin{equation} \texttt{b}^{\overline{\texttt{ln1}}}_\ell (H) \coloneqq \texttt{attn}_{\ell} (A_{\ell}^{\texttt{ln1}} \cdot H) + H\end{equation} and we define a replacement of the block $\texttt{b}^{\texttt{ffn}}_{\ell}$ by \begin{equation} \texttt{b}^{\overline{\texttt{ln2}}}_\ell (H) \coloneqq \texttt{ffn}_{\ell} (A_{\ell}^{\texttt{ln2}} \cdot H) + H.\end{equation}
We then define a mapping between two layers ${\ell \rightarrow \ell'}$ by:
$$ \matlnl{} (H) \coloneqq $$
$$ \texttt{b}^{\overline{\texttt{ln2}}}_{\ell'} ( \texttt{b}^{\overline{\texttt{ln1}}}_{\ell'} ( \ldots (\texttt{b}^{\overline{\texttt{ln2}}}_{\ell+1} ( \texttt{b}^{\overline{\texttt{ln1}}}_{\ell+1} (H))\ldots)).$$


\end{document}

% \bibliography{ref}

\section{Available operations and hyperparameters}
\label{anx:op_hp}

\begin{table}[htbp]
\caption{Operations available in our search space and used for the Monash time series archive dataset and their hyperparameters that can be optimized.}
\begin{center}
\begin{tabularx}{\linewidth}{|X|X X|}
        \hline
        \textbf{Operation} & \multicolumn{2}{c|}{\textbf{Optimized hyperparameters}} \\
        \hline
        Identity & \multicolumn{2}{c|}{-} \\
        \hline
        Fully-Connected (MLP) & Output shape & Integer\\
        \hline
        \multirow{2}{*}{Attention}
                       & Initialization type & [convolution, random] \\
                       & Heads number & Integer \\
        \hline
        1D Convolution & Kernel size & Integer \\
        \hline
        \multirow{2}{*}{Recurrence}
                        & Output shape & Integer\\
                       & Recurrence type & [LSTM, GRU, RNN] \\
        \hline
        \multirow{2}{*}{Pooling}
                        & Pooling size & Integer\\
                       & Pooling type & [Max, Average] \\
        \hline
        Dropout & Dropout Rate & Float\\      
        \hline                       
    \end{tabularx}
\label{tab1}
\end{center}
\end{table}

Activation functions, $\forall x \in \mathbb{R^D}$
\begin{itemize}
    \item Id: $\mathrm{id}(x) = x$
    \item Sigmoid: $\mathrm{sigmoid}(x)={\frac {1}{1+{\rm {e}}^{-x}}} $
    \item Swish: $\mathrm{swish}(x) = x \times \mathrm{sigmoid}(\beta x) ={\frac {x}{1+e^{-\beta x}}}$
    \item Relu: $\mathrm{relu}(x) = \max(0,x)$
    \item Leaky-relu: $\mathrm{leakyRelu}(x) = \mathrm{relu}(x) + \alpha \times \min(0,x)$, in our case: $\alpha = 10^{-2}$
    \item Elu: $\mathrm{elu}(x) = \mathrm{relu}(x) + \alpha \times \min(0, e^x - 1)$
    \item Gelu: $\mathrm{gelu}(x) = x\mathbb{P}(X\leq x) \approx 0.5x(1+\tanh[\sqrt{2/\pi}(x+0.044715x^3)])$
    \item Softmax: $\sigma(\mathbf{x})_j=\frac{\mathrm e^{x_j}}{\sum_{d=1}^D\mathrm e^{x_d}}$ $\forall j\in\left\{1,\ldots,D \right\}$
\end{itemize}

\newpage

\section{Monash datasets presentation} \label{part info monash}

\begin{table}[htbp]
\scriptsize
\begin{center}
\caption{Information about the Monash datasets \citep{godahewa2021monash}.}
\label{tab:info}
\begin{tabularx}{\linewidth}{|c|X X X X X|}
\hline

\textbf{Dataset} & \textbf{Domain} & \textbf{Nb of series} & \textbf{Multivariate} & \textbf{Lag} & \textbf{Horizon} \\
\hline
\hline
Aus. elec & Energy & 5 & No & 420 & 336 \\
\hline
Births & Nature & 1 & No & 9 & 30 \\
\hline
Bitcoin & Economic & 18 & No & 9 & 30 \\
\hline
Carparts & Sales & 2674 & Yes & 15 & 12 \\
\hline
Dominick & Sales & 115704 & No & 10 & 8 \\
\hline
Elec. hourly & Energy & 321 & Yes & 30 & 168 \\
\hline
Elec. weekly & Energy & 321 & Yes & 65 & 8 \\
\hline
Fred MD & Economic & 107 & Yes & 15 & 12 \\
\hline
Hospital & Health & 767 & Yes & 15 & 12 \\
\hline
KDD & Nature & 270 & No & 210 & 168 \\
\hline
\makecell{Kaggle\\weekly} & Web & 145063 & Yes & 10 & 8 \\
\hline
\makecell{M1\\monthly} & Multiple & 1001 & No & 15 & - \\
\hline
M1 quart.  & Multiple & 1001 & No & 5 & - \\
\hline
M1 yearly & Multiple & 1001 & No & 2 & - \\
\hline
\makecell{M3\\monthly}  & Multiple & 3003 & No & 15 & - \\
\hline
M3 other & Multiple & 3003 & No & 2 & - \\
\hline
M3 quart. & Multiple & 3003 & No & 5 & - \\
\hline
M3 yearly & Multiple & 3003 & No & 2 & - \\
\hline
M4 daily & Multiple & 100000 & No & 9 & - \\
\hline
M4 hourly & Multiple & 100000 & No & 210 & - \\
\hline
\makecell{M4\\monthly}& Multiple & 100000 & No & 15 & - \\
\hline
M4 quart. & Multiple & 100000 & No & 5 & - \\
\hline
M4 weekly & Multiple & 100000 & No & 65 & - \\
\hline
NN5 daily & Banking & 111 & Yes & 9 & - \\
\hline
NN5 weekly & Banking & 111 & Yes & 65 & 8 \\
\hline
Pedestrians & Transport & 66 & No & 210 & 24\\
\hline
Rideshare & Transport & 2304 & Yes & 210 & 168 \\
\hline
Saugeen & Nature & 1 & No & 9 & 30 \\
\hline
Solar 10mn & Energy & 137 & Yes & 50 & 1008 \\
\hline
\makecell{Solar\\weekly} & Energy & 137 & Yes & 6 & 5 \\
\hline
Sunspot & Nature & 1 & No & 9 & 30 \\
\hline
Temp. rain & Nature & 32072 & Yes & 9 & 30 \\
\hline
\makecell{Tourism\\monthly} & Tourism & 1311 & No & 2 & - \\
\hline
\makecell{Tourism\\quart.} & Tourism & 1311 & No & 5 & - \\
\hline
\makecell{Tourism\\yearly }& Tourism & 1311 & No & 2 & - \\
\hline
\makecell{Traffic\\hourly} & Transport & 862 & Yes & 30 & 168 \\
\hline
\makecell{Traffic\\weekly} & Transport & 862 & Yes & 65 & 8 \\
\hline
\makecell{Vehicle\\trips} & Transport & 329 & No & 9 & 30 \\
\hline
Weather & Nature & 3010 & No & 9 & 30 \\
\hline
\end{tabularx}
\end{center}
\end{table}




\end{document}
