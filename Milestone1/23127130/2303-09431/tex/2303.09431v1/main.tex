\documentclass[10pt,twocolumn,letterpaper]{article}

\usepackage{iccv}
\usepackage{times}
\usepackage{epsfig}
\usepackage{graphicx}
\usepackage{amsmath}
\usepackage{amssymb}
\usepackage{booktabs}
\usepackage{multirow}
\usepackage{multicol}
\usepackage{cite}
\usepackage{comment}
\usepackage[export]{adjustbox}
% Include other packages here, before hyperref.

% If you comment hyperref and then uncomment it, you should delete
% egpaper.aux before re-running latex.  (Or just hit 'q' on the first latex
% run, let it finish, and you should be clear).
\usepackage[pagebackref=true,breaklinks=true,letterpaper=true,colorlinks,bookmarks=false]{hyperref}

\iccvfinalcopy % *** Uncomment this line for the final submission


\newcommand{\mj}[1]{\color{blue} #1 \color{black}}
\newcommand{\todo}[1]{\color{red} #1 TODO: \color{black}}
\newcommand{\diego}[1]{\color{green} Diego: #1 \color{black}}
\newcommand{\fede}[1]{\color{purple} Fede: #1 \color{black}}
\newcommand{\fabian}[1]{\color{blue} #1 \color{black}}
\newcommand{\abhijit}[1]{\color{pink} Abhijit: #1 \color{black}}
\newcommand{\michael}[1]{\color{red} Michael: #1 \color{black}}

\newcommand{\reals}{\mathbb{R}}
\newcommand{\tp}{^\text{T}}
\newcommand{\cmark}{\ding{51}}%
\newcommand{\xmark}{\ding{55}}%

% \newcolumntype{P}[1]{>{\centering\arraybackslash}p{#1}}


% Support for easy cross-referencing
\usepackage[capitalize]{cleveref}
\crefname{section}{Sec.}{Secs.}
\Crefname{section}{Section}{Sections}
\Crefname{table}{Table}{Tables}
\crefname{table}{Tab.}{Tabs.}

\def\iccvPaperID{6373} % *** Enter the ICCV Paper ID here
\def\httilde{\mbox{\tt\raisebox{-.5ex}{\symbol{126}}}}

% Pages are numbered in submission mode, and unnumbered in camera-ready
\ificcvfinal\pagestyle{empty}\fi

\begin{document}

%%%%%%%%% TITLE

\title{NeRFMeshing: Distilling Neural Radiance Fields\\into Geometrically-Accurate 3D Meshes}


% \author{Marie-Julie Rakotosaona\\
% Google\\
% {\tt\small  mrakotosaona@google.com}
% % For a paper whose authors are all at the same institution,
% % omit the following lines up until the closing ``}''.
% % Additional authors and addresses can be added with ``\and'',
% % just like the second author.
% % To save space, use either the email address or home page, not both
% \and
% Fabian Manhardt \\
% Google \\
% % First line of institution2 address\\
% {\tt\small fabianmanhardt@google.com}

% \and
% Diego Martin Arroyo \\
% Google \\
% % First line of institution2 address\\
% {\tt\small martinarroyo@google.com}
% \and
% Michael Niemeyer  \\
% Google \\
% % First line of institution2 address\\
% {\tt\small mniemeyer@google.com}

% \and
% Abhijit Kundu  \\
% Google \\
% % First line of institution2 address\\
% {\tt\small abhijitkundu@google.com}\and
% Federico Tombari   \\
% Google, TU Munich \\
% % First line of institution2 address\\
% {\tt\small  tombari@in.tum.de}
% }

\author{
Marie-Julie Rakotosaona~\textsuperscript{1} \quad
Fabian Manhardt~\textsuperscript{1} \quad
Diego Martin Arroyo~\textsuperscript{1} \quad \\
Michael Niemeyer~\textsuperscript{1} \quad
Abhijit Kundu~\textsuperscript{1} \quad
Federico Tombari~\textsuperscript{1,2} \quad
\\[1em]
\textsuperscript{1} Google
\quad
\textsuperscript{2}{TU Munich}
\vspace*{-5mm}
}

\maketitle
% Remove page # from the first page of camera-ready.
\ificcvfinal\thispagestyle{empty}\fi


%%%%%%%%% ABSTRACT
\begin{abstract}
With the introduction of Neural Radiance Fields (NeRFs), novel view synthesis has recently made a big leap forward. 
At the core, NeRF proposes that each 3D point can emit radiance, allowing to conduct view synthesis using differentiable volumetric rendering. While neural radiance fields can accurately represent 3D scenes for computing the image rendering, 3D meshes are still the main scene representation supported by most computer graphics and simulation pipelines, enabling tasks such as real time rendering and physics-based simulations.  
Obtaining 3D meshes from neural radiance fields still remains an open challenge since NeRFs are optimized for view synthesis, not enforcing an accurate underlying geometry on the radiance field.
We thus propose a novel compact and flexible architecture that enables easy 3D surface reconstruction from any NeRF-driven approach.
Upon having trained the radiance field, we distill the volumetric 3D representation into a Signed Surface Approximation Network, allowing easy extraction of the 3D mesh and appearance.
Our final 3D mesh is physically accurate and can be rendered in real time on an array of devices.
\end{abstract}

%%%%%%%%% BODY TEXT

\section{Introduction}

The increasing complexity of source code poses a key challenge to the reliability of large-scale software systems. Software bugs in these systems can lead to safety issues~\cite{bug_safety} for users around the world as well as cause non-negligible financial losses~\cite{bug_loss}. As such, developers have to spend a large amount of time and effort on bug fixing. Consequently, \aprfull (\apr), designed to automatically generate patches to fix software bugs, has attracted wide attention from both academia and industry~\cite{long2016prophet, legoues2012genprog, long2015spr, lou2020can, tufano2018empstudy}. 


To achieve \apr, one popular approach is known as Generate-and-Validate (G\&V)~\cite{qi2015gv, ghanbari2019prapr, lou2020can, le2016hdrepair, legoues2012genprog, wen2018capgen, hua2018sketchfix, martinez2016astor, koyuncu2020fixminder, liu2019tbar, liu2019avatar}, which is typically based on the following pipeline: First, fault localization techniques~\cite{wong2016fl, abreu2007ochiai, zhang2013injecting, papadakis2015metallaxis, li2019deepfl, li2017transforming} are applied to determine the suspicious locations in programs where bugs are likely to exist. Then, the buggy locations are used by the \apr tools to generate a list of patches that replace buggy lines with correct lines. Afterward, each patch is validated against the original test suite to identify any \emph{plausible patches} (i.e., passing all tests in the test suite). Finally, to determine the \emph{correct patches}, developers examine the list of plausible patches to see if any of them can correctly fix the bug. 

Traditional \apr tools can mainly be categorized into heuristic-based~\cite{legoues2012genprog, le2016hdrepair, wen2018capgen}, constraint-based~\cite{mechtaev2016angelix, le2017s3, demacro2014nopol, long2015spr} and \template~\cite{ghanbari2019prapr, hua2018sketchfix, martinez2016astor, liu2019tbar, liu2019avatar}. Among these traditional tools, \template \apr tools~\cite{ghanbari2019prapr, liu2019tbar, benton2020effectiveness} have been able to achieve state-of-the-art results. \Template \apr tools typically leverage pre-defined templates (e.g., adding a nullness check) for bug fixing. However, since these fix templates are typically handcrafted, the number and types of bugs they are able to fix can be limited. 



To address the limitations of traditional \apr, researchers have proposed various \learning \apr tools~\cite{li2020dlfix, chen2018sequencer, jiang2021cure, lutellier2020coconut, zhu2021recoder, ye2022rewardrepair} based on the \nmtfull (\nmt) architecture~\cite{sutskever2014mt} where the input is the buggy code snippets and the goal is to translate the buggy code snippets into a fixed version. To accomplish this, \learning \apr tools require supervised training datasets with pairs of both buggy and fixed code snippets in order to learn how to perform this translation step. These training data are usually obtained by mining historical bug fixes using heuristics/keywords~\cite{dallmeier2007benchmark}, which can be imprecise for identifying bug-fixing commits; even the actual bug-fixing commits can include irrelevant code changes, leading to further pollution in the dataset~\cite{xia2022alpharepair}.
% 
Moreover, it can be hard for such \apr tools to generalize and fix bug types unseen during training. 



To better leverage recent advances in \plmfull{s} (\plm{s}), researchers~\cite{xia2022alpharepair, xia2023repairstudy, kolak2022patch, prenner2021codexws} have directly applied \plm{s} to generate patches without bug-fixing datasets. These \llm-based \apr tools work by either directly generating a complete code function~\cite{prenner2021codexws, xia2023repairstudy} or predict/infill the correct code snippet given its surrounding context~\cite{xia2022alpharepair, xia2023repairstudy}. By directly using \llm{s} that are pre-trained on billions of open-source code snippets, \llm-based \apr tools can achieve state-of-the-art performance on many repair datasets~\cite{xia2022alpharepair}. 


% 
%
%

Traditional \apr tools have long used the insight of the \emph{plastic surgery hypothesis}~\cite{barr2014plastic} where it states that the code ingredients to fix a bug already exist within the same project. Traditional \apr tools have manually designed pattern-~\cite{ghanbari2019prapr, saha2017elixir} or heuristic-based~\cite{jiang2018simfix, legoues2012genprog} approaches to finding and using such relevant code ingredients to generate fixes for bugs. However, the plastic surgery hypothesis has been largely ignored in \llm-based \apr. In fact, \llm provides a unique opportunity to fully automate the plastic surgery hypothesis idea via fine-tuning (learning project-specific information via model updates from the buggy project) and prompting (directly providing relevant code ingredients to the model), and make it directly applicable to different languages (since the \llm{s} are typically multi-lingual).%
Moreover, despite the intensive manual efforts involved, traditional \apr tools still cannot fully leverage project-specific information due to large search space for leveraging/composing existing code ingredients. In contrast, the project-specific information can effectively leveraged by \llm{s} due to their power in code understanding/vectorization, e.g., even partial/imprecise information may still guide \llm{s} in correct patch generation!
 To this end, we ask the question: \emph{How useful is the plastic surgery hypothesis in the era of \plm{s}}?








\mypara{Our Work.} To answer the question, we present \ourtech{\xspace} -- a \llm-based approach that automatically utilizes the plastic surgery hypothesis by systematically combining multiple fine-tuning and prompting strategies for \apr. \ourtech fine-tunes \plm{s} using two novel domain-specific training strategies: \textbf{\epfinetune} -- we fine-tune using the original buggy project by aggressively masking out a high percentage of tokens, which allows \plm to learn project-specific code tokens and programming styles; and \textbf{\rofinetune} -- which only masks out a single continuous code sequence per training sample, allowing the model to get used to the final \csapr task of predicting a single continuous code sequence. Furthermore, we directly leverage the ability for \plm{s} to understand natural language instructions and introduce a novel prompting strategy, \textbf{\idprompting}, which uses information retrieval and static analysis to obtain a list of relevant identifiers for the buggy lines. While such relevant identifiers are critical for fixing some difficult bugs, they may not be seen by the \llm during inference due to limited context window size. Through the use of prompting, we directly tell the model to use these extracted identifiers (relevant code ingredients) to generate the correct code. Finally, to perform repair, we combine all four model variants (including the base model, both fine-tuned models and the base model with prompting) for the final repair.





While our insight of leveraging the plastic surgery hypothesis for \llm-based \apr is generalizable across different types of \plm{s}, to implement \ourtech, we choose a recent \plm{\xspace}, \ctfive~\cite{wang2021codet5}, which is pre-trained on millions of open-source code snippets. \ctfive is an encoder-decoder model trained using \mspfull (\msp) objective where a percentage of tokens are masked out and each continuous masked token sequence is referred to as a masked span. Also, although we only extract relevant identifiers from the current buggy project (since this paper focuses on the plastic surgery hypothesis), our work can be easily extended to obtain other code information (such as relevant statements or functions) from other sources, such as  the massive pre-training corpora~\cite{husain2020codesearchnet} or historical bug-fixing datasets~\cite{jiang2019infer}, which can provide more coding knowledge for \llm{s}. Besides, although we mainly focus on using traditional string comparison algorithms for information retrieval in this paper, these techniques can be easily replaced by other frequency-based retrieval~\cite{robertson2009probabilistic} and neural search (or embedding-based search)~\cite{reimers2019sentence}.
  In summary, this paper makes the following contributions:


%


\begin{itemize}[noitemsep, leftmargin=*, topsep=0pt]
    \item \textbf{Dimension.} This paper is the first to revisit the important plastic surgery hypothesis in the era of \llm{s}. It opens up a new dimension for \llm-based \apr to incorporate previously neglected information from the buggy project itself to boost \apr performance. Furthermore, it demonstrates the promising future of retrieval-based prompting for modern \llm-based \apr.
    \item \textbf{Implementation.} We implement \ourtech based on the recent \ctfive model. We augment the model using two novel fine-tuning strategies: \epfinetune and \rofinetune, along with a novel prompting strategy based on information retrieval and static analysis: \idprompting. We combine the patches generated by all four models together and perform patch ranking to speed up \apr.% 
    \item \textbf{Evaluation Study.} We conduct an extensive evaluation against state-of-the-art \apr tools. On the widely studied \dfj 1.2 and 2.0 datasets~\cite{just2014dfj}, \ourtech is able to achieve the new state-of-the-art results of 89 and 44 correct bug fixes (15 and 8 more than best baseline) respectively.  Furthermore, we perform a broad ablation study to justify our design. \ourtech demonstrates for the first time that the plastic surgery hypothesis can substantially boost \llm-based \apr and advance state-of-the-art \apr, while being fully automated and general. Moreover, even partial/imprecise code ingredients may still effectively guide \llm{s} for \apr!
\end{itemize}


\section{Related work}
\noindent \textbf{Video foundation models.}
With sufficient computational power and an abundant source of data, there have been attempts to build a single large-scale foundation model that can be adapted to diverse downstream tasks.
Along with the success of foundations models in the natural language processing domain~\cite{brown2020language,chen2021evaluating,devlin2019bert} and in computer vision~\cite{bertasius2021space,jia2021scaling,radford2021learning}, video data has become another data type of interest, as it has grown in scale due to numerous internet video-sharing platforms.
Accordingly, several methods to train a video foundation model have been proposed.
Due to the innate multi-modality of video data, \textit{i.e.}, a combination of visual $\cdot$ vocal $\cdot$ textual context, most works have centered around the variations of the cross-modal attention mechanism \cite{akbari2021vatt,bertasius2021space,gabeur2020multi,luo2020univl,neimark2021video,tan2021look,wei2020multi,yang2021taco}.
In addition, as most video data lack proper labels or descriptions, contrastive learning methods were studied to learn meaningful feature representations or enhance video-text alignment in a self-supervised manner \cite{akbari2021vatt,kuang2021video,luo2020univl,yang2021taco}.

More specifically, MERLOT \cite{zellers2021merlot} proposed a multi-modal representation learning method for visual commonsense reasoning, which also performed well in twelve video reasoning tasks.
VATT \cite{akbari2021vatt} introduced a multi-modal learning method via contrastive learning. 
The pre-trained model performed well in a variety of vision tasks from image classification to video action recognition and zero-shot video retrieval.
Another representative work, UniVL \cite{luo2020univl} proposed a straightforward pre-training method with auxiliary loss functions. 
After fine-tuning on a specific task, the pre-trained model performed outstandingly in a wide range of tasks of text-to-video retrieval, action segmentation, action step localization, video sentiment analysis, and video captioning.
Other foundation models for multiple video tasks include \cite{li2020hero,sun2019learning,sun2019videobert,zhu2020actbert,fu2021violet,wang2022all}. 

\noindent \textbf{Auxiliary learning.}
In order to enhance the performance of one or a multitude of primary tasks, auxiliary learning methods can be incorporated.
\cite{ruder2017overview} introduced Multi-task learning (MTL) to the deep neural networks by training a single model with multiple task losses to assist learning on the main task.
Such a method is generally adapted to pre-train the foundation models in the self-supervised manner~\cite{li2020hero,sun2019learning,sun2019videobert,zhu2020actbert,fu2021violet,wang2022all}.
However, these various pretext task losses used in the pre-training phase are ignored in the fine-tuning phase, and only the primary task loss is minimized.

Recently, meta-learning methods have been introduced for auxiliary learning.
\cite{liu2019self,navon2020auxiliary,shu2019meta} proposed a meta-learning method in which the model learns auxiliary tasks to generalize well to unseen data. 
In these settings, a separate subset of data is held out as the primary task, while the others are used as auxiliary tasks that aid the primary task's performance.
Similar methods were adopted for computer vision tasks such as semantic segmentation \cite{xu2021leveraging}.
Other domain applications include navigation tasks with reinforcement learning \cite{ye2021auxiliary}, or self-supervised learning methods on graph data \cite{hwang2020self}.
\section{Method}
\label{sec:method}

% \ml{``Inconsistent'' to ``large variation''}

% In this section, we propose our methods based on the observations in Section \ref{sec:motivation}.
In this section, we propose two techniques to further enhance the strong baseline to capture the variation of activation distributions better.
We first introduce spatial re-scaling to adapt the network to pixel-to-pixel variation.
We then propose channel-wise shifting and re-scaling to better capture the channel-to-channel variation.
Meanwhile, as both of the two methods are image-dependent, the image-to-image variation can be captured naturally.
By combining the two methods with our strong baseline, we build our enhanced BNN for SR, named EBSR.

% Because the activation distributions among pixels, channels and images have large variations \red{**are highly inconsistent} in SR networks, we introduce spatial re-scaling to adapt to pixel-wise variations and channel shift and re-scaling to adapt to channel-wise variations. And both of them are image-dependent to adapt to image-wise variations, which means during inference our network re-scales and shifts the distributions of activations flexibly for different input images. Based on these methods, we build an enhanced binary neural network for image super-resolution (EBSR).

% According to [3], the difference of activation magnitudes indicates different scaling factors are needed for each pixel.

\subsection{Spatial Re-scaling}
% It is better to use different scaling factors for different pixels to reduce the quantization error and retain more detailed information for image super-resolution. 

% \ml{In the main method, we do not need to introduce the previous works but can focus on introducing our own method. Channel rescaling in Real-to-binary Net is not relevant in this context.}

% Re-scaling the output of binary convolutions was proposed at the birth of BNN in XNOR-Net \cite{rastegari2016xnor} to reduce quantization error and improve accuracy for image classification tasks.
% It is computed as below:
% \begin{equation}
% \mathcal{A} * \mathcal{W} \approx(\operatorname{sign}(\mathcal{A}) \circledast \operatorname{sign}(\mathcal{W})) \odot \mathcal{K} \alpha
% \label{eq:xnor-net rescale}
% \end{equation}
% where $\circledast$ denotes the binary convolution and $\odot$ denotes the element-wise multiplication.
% $\mathcal{A}$, $\mathcal{W}$, $\alpha$, and $\mathcal{K}$ denote the activation, weight, weight scaling factor, and activation scaling factor, respectively.
%  Later in XNOR-Net++ \cite{bulat2019xnor}, Bulat et al. fuse the activation and weight scaling factors into a single one that is learned end-to-end based on gradients and this improves the classification accuracy on ImageNet dataset.

% % It is computed as Eq.~\ref{eq:xnor-net rescale}, where $\circledast$ denotes 
% %  the binary convolution and $\odot$ denotes the element-wise multiplication. The binary convolution of $\mathcal{A}$ and $\mathcal{W}$ is rescaled by the weight scaling factor $\alpha$ and the activation scaling factor $\mathcal{K}$, both of which are calculated analytically.


% \zc{Similarly, you should explain the meaning of A, W and the operators $\circledast$ in the formula}
% Then in Real-to-binary Net \cite{martinez2020training}, Martinez et al. used a data-driven channel re-scaling module that takes the pre-convolution activations as input to predict the activation scaling factor. Unlike that in XNOR-Net++ \cite{bulat2019xnor}, these scaling factors are not fixed during inference but rather inferred from data. By doing this, they further improved the classification accuracy on ImageNet over XNOR-Net++. 
As is shown in Figure \ref{fig:pixel}, activation distributions have large pixel-to-pixel variation in SR networks
and the difference of activation magnitudes indicates different scaling factors are preferred for different pixels.
Inspired by \cite{martinez2020training}, we propose spatial re-scaling to better adapt the network to the spatial variation
of activation distributions in SR networks.
% fit the various pixel-wise distributions in SR networks.
We take the real-valued activations $A$ before convolution as input and predict pixel-wise scaling factors $S(A)$, which re-scale the binary convolution output. Spatial re-scaling process can be formulated as follows:
\begin{equation}
A * W \approx(\operatorname{sign}(A) \circledast \operatorname{sign}(W)) \odot \alpha \odot S(A)
\label{eq:spatial rescale}
\end{equation}
where $\circledast$ denotes 
the binary convolution and $\odot$ denotes the element-wise multiplication. $A$, $W$, $\alpha$, and $S\left(A\right)$ denote real-valued activations, weights, the scaling factor of weights, and the spatial-wise scaling factor of activations respectively. $S\left(A\right) \in \mathbb{R}^{1\times H\times W}$ can be calculated with a convolution and a sigmoid function.
% as $\sigma\left( CONV\left(A\right)\right)$. 
As shown in Figure \ref{fig:method}(a), real-valued activations first go through a convolution layer,
which has an input channel of $C$ and an output channel of 1, 
and then pass through a sigmoid function to produce the scaling factors $S(A)$ along the spatial dimension.
During inference, the scaling factor will change dynamically according to different input feature maps.
By re-scaling binary convolution output using $S(A)$, we can reduce the quantization error and the original pixel-wise information in FP activation
will be preserved much better.
Spatial re-scaling leads to a large PSNR improvement of 0.24 dB (from 30.30 dB to 31.54 dB) on Set5 and 0.22 dB (from 25.09 dB to 25.31 dB)
on Urban100 compared with our strong baseline. 

\subsection{Channel-wise Shifting and Re-scaling}

\begin{table}[!tb]
\centering
\caption{Comparison between whether to fuse channel-wise shifting and re-scaling or not based on our baseline with spatial re-scaling. }
\label{tab:fusing}

\scalebox{0.65}{
\begin{tabular}{c|cc|cc|cc}
\hline
\multirow{2}{*}{Method}     & \multirow{2}{*}{OPs} & \multirow{2}{*}{Params} & \multicolumn{2}{c|}{Set5} & \multicolumn{2}{c}{Urban100} \\ \cline{4-7} 
                            &                      &                         & PSNR        & SSIM        & PSNR          & SSIM         \\ \hline
Baseline + spatial re-scale & 2.16G                & 0.05M                   & 31.54       & 0.883       & 25.31         & 0.759        \\
+ channel-wise shift and re-scale             & 2.34G                & 0.09M                   & 31.61       & 0.885       & 25.35         & 0.761        \\
+ w/ fusing                   & 2.27G                & 0.08M                   & \textbf{31.64}       & \textbf{0.885}       & \textbf{25.36}         & \textbf{0.761}        \\ \hline
\end{tabular}
}
\end{table}

In SR networks, activation distributions exhibit larger channel-to-channel variation (Figure \ref{fig:chl}).
Both the mean and magnitude of the activation distributions vary significantly across channels.
% Thus we use channel-wise shifting and re-scaling to adapt to various channel-wise distributions. 
\cite{martinez2020training} has proposed the data-driven channel re-scaling, 
but our method differs from them in further introducing data-driven thresholds to handle the channel-wise variation of both mean and magnitude.
Since the blocks to generate the scaling factors and thresholds are very similar, we further propose to fuse them into one module.
% and fusing channel-wise shifting and re-scaling into one module.
We evaluate the effect of fusing the two blocks in Table \ref{tab:fusing}.
With channel-wise shifting and re-scaling fused, our models have fewer operations and parameters overhead and slightly higher performance.

For the specific process, we take the real-valued activations as input and predict different thresholds and scaling factors for each channel. They are also image dependent, e.g., $\beta_{i}$ in Eq.\ref{eq:act_binarize} is no longer fixed during inference but generated according to different input feature maps. Channel-wise shifting and re-scaling can be formulated as follows:
\begin{equation}
A * W \approx(\operatorname{sign}(A-C_s(A)) \circledast \operatorname{sign}(W)) \odot \alpha \odot C_r(A)
\label{eq:channel-wise_shift_and_rescale}
\end{equation}
where $\circledast$ denotes 
the binary convolution and $\odot$ denotes the element-wise multiplication. $C_s(A), C_r(A) \in \mathbb{R}^{C\times1\times1}$ denote the channel-wise threshold and scaling factor, respectively. 
We show the block diagram in Figure \ref{fig:method}(b).
The real-valued input feature map is first squeezed to a ${C\times1\times1}$ vector by a global average pooling (GAP) layer.
The subsequent fully connected layers and ReLU learn the channel-wise information and output a ${2C\times1\times1}$ vector.
Then the ${2C\times1\times1}$ vector is split into two ${C\times1\times1}$ vectors.
We use the first $C$ channels as the channel-wise bias and pass the last $C$ channels through a sigmoid layer 
as the channel-wise scaling factor, which are used to shift the real-valued activations and re-scale the binary convolution output, respectively. 


% \ml{We can mention previously, channel-wise re-scale has been proposed. We propose to fuse them. Add the comparison between fuse v.s. no fuse.}

\begin{figure}[!tbp]%
  \centering
    \includegraphics[width=0.4\textwidth]{fig/methods.png}
  
% \subfloat[channel-wise shifting\&re-scale]{
%     \label{subfig:channel-wise shifting and re-scale}
%     \includegraphics[width=0.2\textwidth]{fig/chl shift and rescale.png}
%   }

  \caption{Block diagram for spatial re-scaling, and channel-wise shifting and re-scaling.} 
  % Input A is the real-valued activation tensor and C, H, and W denote its dimension. GAP stands for global average pooling. The reduction ratio r is set to 16 for a better trade-off between the performance and the number of operations and parameters.}
  \label{fig:method}
\end{figure}


\subsection{Network Structure}

Combining the spatial re-scaling and the channel-wise shifting and re-scaling methods, we construct the enhanced convolution layer (E-Conv).
Then we build our EBSR model based on E-Conv.
In Figure \ref{fig:E-conv}, we compare the binary convolution layer used in the baseline network and our proposed E-Conv.
We use spatial and channel-wise scaling factors to re-scale the binary convolution output,
and use channel-wise shifting to learn appropriate thresholds for each channel before binarization.
The scaling factors and threshold used in E-Conv are learnable and depend on the real-valued input activations.
In this way, our proposed EBSR can adapt to pixel-to-pixel, channel-to-channel, and image-to-image variations
to reduce the large binarization error and preserve more details.
% In this way, our proposed E-Conv reduces the large quantization error caused by binarization and keeps the original information of input feature maps to a large extent.


\begin{figure}[!tb]%
  \centering

    \includegraphics[width=0.5\textwidth]{fig/E-conv.png}

  \caption{Comparison of (a) the binary convolution layer with a skip connection used in our baseline network and (b) the proposed E-Conv.}
  \label{fig:E-conv}
\end{figure}


Figure \ref{fig:network} shows the basic block based on the E-Conv and our EBSR composed of the basic blocks. Following existing works, the convolution layers in the head and tail modules are not binarized. We choose the lightweight EDSR which has 16 basic blocks and 64 channels, and EDSR which has 32 basic blocks and 256 channels as our backbones, which correspond to EBSR-light and EBSR, respectively.

\begin{figure}[!tb]%
  \centering
  {
    \includegraphics[width=0.35\textwidth]{fig/network.png}
  }
  
  \caption{The structure of our proposed EBSR.  Convolution layers in purple are real-valued vanilla 3x3 convolutions.}
  \label{fig:network}
\end{figure}
\section{Evaluation}
\label{sec:eval}

We conduct experiments from different aspects to validate the efficacy of the propose \model\ framework. The implementation details for our \model\ and the baseline methods are presented in~\ref{sec:implement}. Our experiments aim to answer the following research questions:
\begin{itemize}[leftmargin=*]
    \item \textbf{RQ1}: How does the proposed \model\ perform on different experimental datasets in comparison to state-of-the-art baselines?
    \item \textbf{RQ2}: How does different sub-modules of the proposed \model\ framework contribute to the overall performance?
    \item \textbf{RQ3}: How scalabile is \model\ in handling large-scale data?
    \item \textbf{RQ4}: How does the model performance vary when tuning important hyperparameters of the proposed \model\ model?
    \item \textbf{RQ5}: How can our \model\ model address the over-smoothing issue compared with GNN-based recommendation methods?
\end{itemize}

\subsection{Experimental Settings}
\subsubsection{\bf Experimental Datasets}
\begin{table}[t]
    \centering
    \caption{Statistics of the experimental datasets.}
    \label{tab:datasets}
    \small
    \vspace{-0.18in}
    \begin{tabular}{ccccc}
        \toprule
        Dataset & \# Users & \# Items & \# Interactions & Interaction Density \\
        \midrule
        Gowalla & 25,557 & 19,747 & 294,983 & $5.85\times 10^{-4}$\\
        Yelp & 42,712 & 26,822 & 182,357 & $1.59\times 10^{-4}$\\
        Amazon & 76,469 & 83,761 & 966,680 & $1.51\times 10^{-4}$\\
        % Tmall & 805,506 & 584,050 & 39,183,700 & $8.33\times 10^{-5}$\\
        \bottomrule
    \end{tabular}
    \vspace{-0.15in}
    \Description{A table showing the statistics of the Gowalla data (25557 users, 19747 items, 294983 interactions), the Yelp data (42712 users, 26822 items, 182357 interactions), and the Amazon data (76469 users, 83761 items, 966680 interactions).}
\end{table}


\begin{table*}[h]
\vspace{-0.1in}
\caption{Performance comparison on Gowalla, Yelp, and Amazon datasets in terms of \textit{Recall} and \textit{NDCG}.}
\vspace{-0.15in}
\centering
%\ssmall
% \scriptsize
\footnotesize
%\small
\setlength{\tabcolsep}{1.2mm}
\begin{tabular}{|c|c|c|c|c|c|c|c|c|c|c|c|c|c|c|c|c|c|l|}
\hline
Data & Metric & BiasMF & NCF & AutoR & PinSage & STGCN & GCMC & NGCF & GCCF & LightGCN & DGCF & SLRec & NCL & SGL & HCCF & \emph{\model} & p-val.\\
\hline
\multirow{4}{*}{Gowalla}
&Recall@20 & 0.0867 & 0.1019 & 0.1477 & 0.1235 & 0.1574 & 0.1863 & 0.1757 & 0.2012 & 0.2230 & 0.2055 & 0.2001 & 0.2283 & 0.2332 & 0.2293 & \textbf{0.2434} & $2.1e^{-8}$\\
&NDCG@20 & 0.0579 & 0.0674 & 0.0690 & 0.0809 & 0.1042 & 0.1151 & 0.1135 & 0.1282 & 0.1433 & 0.1312 & 0.1298 & 0.1478 & 0.1509 & 0.1482 & \textbf{0.1592} & $1.2e^{-9}$\\
\cline{2-18}
&Recall@40 & 0.1269 & 0.1563 & 0.2511 & 0.1882 & 0.2318 & 0.2627 & 0.2586 & 0.2903 & 0.3181 & 0.2929 & 0.2863 & 0.3232 & 0.3251 & 0.3258 & \textbf{0.3399} & $2.4e^{-8}$\\
&NDCG@40 & 0.0695 & 0.0833 & 0.0985 & 0.0994 & 0.1252 & 0.1390 & 0.1367 & 0.1532 & 0.1670 & 0.1555 & 0.1540 & 0.1745 & 0.1780 & 0.1751 & \textbf{0.1865} & $1.7e^{-9}$\\
\hline

\multirow{4}{*}{Yelp}
&Recall@20 & 0.0198 & 0.0304 & 0.0491 & 0.0510 & 0.0562 & 0.0584 & 0.0681 & 0.0742 & 0.0761 & 0.0700 & 0.0665 & 0.0806 & 0.0803 & 0.0789 & \textbf{0.0823} & $3.7e^{-4}$\\
&NDCG@20 & 0.0094 & 0.0143 & 0.0222 & 0.0245 & 0.0282 & 0.0280 & 0.0336 & 0.0365 & 0.0373 & 0.0347 & 0.0327 & 0.0402 & 0.0398 & 0.0391 & \textbf{0.0414} & $3.8e^{-5}$\\
\cline{2-18}
&Recall@40 & 0.0307 & 0.0487 & 0.0692 & 0.0743 & 0.0856 & 0.0891 & 0.1019 & 0.1151 & 0.1175 & 0.1072 & 0.1032 & 0.1230 & 0.1226 & 0.1210 & \textbf{0.1251} & $4.8e^{-3}$\\
&NDCG@40 & 0.0120 & 0.0187 & 0.0268 & 0.0315 & 0.0355 & 0.0360 & 0.0419 & 0.0466 & 0.0474 & 0.0437 & 0.0418 & 0.0505 & 0.0502 & 0.0492 & \textbf{0.0519} & $2.4e^{-4}$\\
\hline

\multirow{4}{*}{Amazon}
&Recall@20 & 0.0324 & 0.0367 & 0.0525 & 0.0486 & 0.0583 & 0.0837 & 0.0551 & 0.0772 & 0.0868 & 0.0617 & 0.0742 & 0.0955 & 0.0874 & 0.0885 & \textbf{0.1067} & $1.1e^{-10}$\\
&NDCG@20 & 0.0211 & 0.0234 & 0.0318 & 0.0317 & 0.0377 & 0.0579 & 0.0353 & 0.0501 & 0.0571 & 0.0372 & 0.0480 & 0.0623 & 0.5690 & 0.0578 & \textbf{0.0734} & $7.0e^{-12}$\\
\cline{2-18}
&Recall@40 & 0.0578 & 0.0600 & 0.0826 & 0.0773 & 0.0908 & 0.1196 & 0.0876 & 0.1175 & 0.1285 &0.0912 & 0.1123 & 0.1409 & 0.1312 & 0.1335 & \textbf{0.1535} & $6.6e^{-10}$\\
&NDCG@40 & 0.0293 & 0.0306 & 0.0415 & 0.0402 & 0.0478 & 0.0692 & 0.0454 & 0.0625 & 0.0697 & 0.0468 & 0.0598 & 0.0764 & 0.0704 & 0.0716 & \textbf{0.0879} & $2.0e^{-12}$\\
\hline
\end{tabular}
\vspace{-0.1in}
\label{tab:overall_performance}
\Description{A table presenting the evaluated performance of the proposed \model\ model and the baselines, in which \model\ significantly outperforms the baseline methods.}
\end{table*}

% atings are transformed into binary implicit feedback following~\cite{he2020lightgcn}. We filter users and items with less than 3 interactions, 

Three benchmark datasets collected from real-world online services are used to evaluate the performance of \model. Data statistics are shown in Table~\ref{tab:datasets}. We split the interaction data into training set, validation set and test set with 70\%:5\%:25\%. Details of the experimental datasets are:
\begin{itemize}[leftmargin=*]
    \item \textbf{Gowalla}: This dataset is collected from Gowalla, including user check-in records at geographical locations, from Jan to Jun, 2010.
    \item \textbf{Yelp}: This dataset contains users' ratings on venues, collected from Yelp platform. The time range is from Jan to Jun, 2018.
    \item \textbf{Amazon}: This dataset is composed of users' rating behaviors over books collected from Amazon platform, during 2013.
\end{itemize}

\vspace{-0.1in}
\subsubsection{\bf Evaluation Protocols}
Following previous works on CF recommenders~\cite{wang2019neural, xia2022self}, we conduct all-rank evaluation, in which positive items from test set are ranked with all un-interacted items for each user. The widely-used \emph{Recall@N} and \emph{NDCG@N} metrics~\cite{wu2021self,2021knowledge} are used adopted for evaluation, where $N=20$ by default.

\vspace{-0.05in}
\subsubsection{\bf Baseline Models}
We compare \model\ with the following 14 baselines from 4 research lines for comprehensive validation.
\\\noindent\textbf{Traditional Collaborative Filtering Technique:}
\begin{itemize}[leftmargin=*]
    \item \textbf{BiasMF}~\cite{koren2009matrix}: It is a classic matrix factorization approach that combines user/item biases with learnable embedding vectors.
\end{itemize}
\textbf{Non-GNN Neural Collaborative Filtering}:
\begin{itemize}[leftmargin=*]
    \item \textbf{NCF}~\cite{he2017neural}: It is an early study of deep learning CF model that enhances the user-item interaction modeling with MLP networks.
    \item \textbf{AutoR}~\cite{sedhain2015autorec}: This method applies a three-layer autoencoder with fully-connected layers to encode user interaction vectors.
\end{itemize}
\textbf{Graph Neural Architectures for Collaborative Filtering}:
\begin{itemize}[leftmargin=*]
    \item \textbf{PinSage}~\cite{ying2018graph}: This method combines random walk with graph convolutions for web-scale graph in recommendation.
    \item \textbf{STGCN}~\cite{zhang2019star}: This method augments GCN with autoencoding sub-networks on hidden features for better inductive inference.
    \item \textbf{GCMC}~\cite{berg2017graph}: This is a representative work to introduce graph convolutional operations into the matrix completion task.
    \item \textbf{NGCF}~\cite{wang2019neural}: It is a GNN-based CF method which conducts graph convolutions on the user-item interaction graph for embeddings.
    \item \textbf{GCCF}~\cite{chen2020revisiting} and \textbf{LightGCN}\cite{he2020lightgcn}: These two methods propose to simplify conventional GCN structures by removing transformations and activations for improving performance.
\end{itemize}
\textbf{Disentangled GNN-based Collaborative Filtering}:
\begin{itemize}[leftmargin=*]
    \item \textbf{DGCF}\cite{wang2020disentangled}: This method disentangles user-item interactions into multiple hidden factors in the graph message passing process.
\end{itemize}
\textbf{Self-Supervised Learning Approaches for Recommendation}:
\begin{itemize}[leftmargin=*]
    \item \textbf{SLRec}~\cite{yao2021self}: This method applies contrastive learning to recommendation models with feature-level data augmentations.
    \item \textbf{NCL}~\cite{lin2022improving}: This approach enhances self-supervised graph CF models with enriched neighbor-wise contrastive learning.
    \item \textbf{SGL}~\cite{wu2021self}: It conducts various types of graph augmentations and feature augmentations with graph contrastive learning for CF.
    \item \textbf{HCCF}~\cite{xia2022hypergraph}: This method augments GNN-based CF with a global hypergraph GNN and conducts cross-view contrastive learning.
\end{itemize}

\subsection{Overall Performance Comparison (RQ1)}


The overall performance of \model\ and the baselines are shown in Table~\ref{tab:overall_performance}. From the results we have the following observations: \vspace{-0.05in}
\begin{itemize}[leftmargin=*]
    \item Our \model\ consistently achieves best performance compared to baselines methods. Also, we re-train \model\ and the best-performed baselines (\ie, SGL and NCL) for 5 times to calculate $p$-values. The experimental results validate the significance of the improvement by \model. Compared to the state-of-the-art GNN methods, the MLP-based inference model of our graph-less \model\ generates more accurate recommendation results, due to its adaptive contrastive knowledge distillation. Specifically, the dual-level KD in \model\ enables enriched and adaptive high-order smoothing, which not only distills the accurate dark knowledge in the well-trained GNN teacher, but also avoids being affected by the over-smoothing signals. Furthermore, the adaptive contrastive regularization automatically alleviates the over-smoothing effects, which further boosts the performance. \\\vspace{-0.12in}
    
    \item While the self-supervised learning schema greatly improves the performance of GNN-based CF, our graph-less \model\ model still significantly outperforms the SSL-enhanced graph models. We attribute the performance deficiency to the inherent incapability of existing SSL frameworks in filtering over-smoothing signals. For example, SGL augments model training by introducing random noises, which may even aggravate the inaccuracy in node embeddings when the noises are magnified through high-order graph propagation. As for NCL and HCCF, they seek to connect nodes based on global semantic relatedness, which may even over-smooth nodes distant from each other in the original graph. In comparison, our graph-less \model\ model abandons GNN architectures in the inference model, which fundamentally minimizes the possibility of over-smoothed node embeddings. Furthermore, our KD paradigm avoids distilling over-smoothed embeddings via the adaptive contrastive regularization. \\\vspace{-0.12in}
    
    \item We observe that non-GNN CF models (\ie, NCF and AutoR) present very bad performance, event though they have similar MLP-based network architectures as the inference model in \model. This sheds light on the deficiency of MLPs in modeling high-order graph connectivity into user/item embeddings. While sharing similar MLP structures, our \model\ is additionally supervised by knowledge distilled from advanced GNN models. This not only improves the optimization for MLP networks, but also makes it possible to adaptively filter the over-smoothing signals in parameter learning. The huge performance gap between NCF/AutoR and our \model\ strongly shows the effectiveness of our contrastive knowledge distillation.
\end{itemize}


\begin{table}[t]
    %\vspace{-0.05in}
    \caption{Ablation study on key components of \model.}
    \vspace{-0.15in}
    \centering
    %\small
    %\scriptsize
    %\ssmall
    \footnotesize
    %\small
    % \setlength{\tabcolsep}{1.2mm}
    \begin{tabular}{c|c|cc|cc|cc}
        \hline
        % \multirow{2}{*}{Category} & \multirow{2}{*}{Variant} 
        \multicolumn{2}{c|}{Data}& \multicolumn{2}{c|}{Gowalla} & \multicolumn{2}{c|}{Yelp} & \multicolumn{2}{c}{Amazon}\\
        % \cline{3-8}
        \hline
        \multicolumn{2}{c|}{Variant} & Recall & NDCG & Recall & NDCG & Recall & NDCG\\
        \hline
        % \hline
        % \multicolumn{8}{c}{Top-20}\\
        \hline
        \multicolumn{2}{c|}{-$\mathcal{L}_1$} & 0.2180 & 0.1415 & 0.0756 & 0.0377 & 0.1012 & 0.0692\\
        \hline
        \multirow{3}{*}{-$\mathcal{L}_2$} & User & 0.2292 & 0.1493 & 0.0806 & 0.0405 & 0.0998 & 0.0667 \\
        & Item & 0.2266 & 0.1477 & 0.0808 & 0.0406 & 0.0974 & 0.0649 \\
        & Both & 0.2222 & 0.1451 & 0.0787 & 0.0399 & 0.0938 & 0.0626 \\
        \hline
        \multirow{4}{*}{-$\mathcal{L}_3$} & U-I & 0.2330 & 0.1496 & 0.0814 & 0.0410 & 0.0939 & 0.0607 \\
        & U-U & 0.2349 & 0.1512 & 0.0811 & 0.0407 & 0.0965 & 0.0634 \\
        & I-I & 0.2331 & 0.1514 & 0.0813 & 0.0409 & 0.1009 & 0.0674\\
        & All & 0.2282 & 0.1480 & 0.0810 & 0.0407 & 0.0933 & 0.0605\\
        \hline
        \hline
        \multicolumn{2}{c|}{\emph{\model}} & \textbf{0.2434} & \textbf{0.1592} & \textbf{0.0823} & \textbf{0.0414} & \textbf{0.1067} & \textbf{0.0734}\\
        \hline
    \end{tabular}
    \vspace{-0.1in}
    \label{tab:module_ablation}
    \Description{A table presenting the results of module ablation study. The results are divided into three parts: loss $\mathcal{L}_1$ for the prediction-level distillation, loss $\mathcal{L}_2$ for the embedding level distillation, and loss $\mathcal{L}_3$ for the contrastive regularization. All ablated variants performs worse than the proposed \model.}
\end{table}

\vspace{-0.1in}
\subsection{Model Ablation Study (RQ2)}
We validate the effectiveness of the applied sub-modules in \model\ by ablating each module separately. The evaluated performance is shown in Table~\ref{tab:module_ablation}. We also show the performance change \wrt, training epochs in Figure~\ref{fig:ablation_lines}. We have the following observations:
\begin{itemize}[leftmargin=*]
    \item \textbf{Effect of Prediction-Level Distillation}: Our prediction-level distillation (\ie, $\mathcal{L}_1$) excavates deep dark knowledge in the teacher using the pair-wise ranking task with enriched KD samples. The variant -$\mathcal{L}_1$ removes this module, which leads to performance degradation on Gowalla and Yelp data. The results validate the effectiveness of learning from the predictive outputs of teacher model using our distillation loss $\mathcal{L}_1$.\\\vspace{-0.12in}
    % The prediction-level KD is removed to produce variant \textbf{-$\mathcal{L}_1$}. From the results we can observe that, removing $\mathcal{L}_1$ causes the most significant performance degradation compared to other variants on Gowalla data and Yelp data. This evidently reflects the importance of learning from the predictive outputs of teacher model. And it validates the effectiveness of excavating deep dark knowledge in the teacher using the pair-wise ranking task with enriched KD samples.
    \item \textbf{Effect of Embedding-Level Distillation}: We then test the effect of embedding-level KD with the variant -$\mathcal{L}_2$ by removing $\mathcal{L}_2$ on user/item embeddings. In some cases the alignment between users and the alignment between items have different effect on the performance. What's more, the results reveal not only the contribution of $\mathcal{L}_2$ to the final performance, but also its prominent accelerating effect in model training shown in Fig~\ref{fig:ablation_lines}. \\\vspace{-0.12in}
    \item \textbf{Effect of Contrastive Regularization}: We ablate \model\ without the contrastive regularization in variant -$\mathcal{L}_3$. The regularization for user-item, user-user, and item-item relatedness are individually ablated. We observe the importance of $\mathcal{L}_3$ for the superior performance, especially on Amazon data. We ascribe this to the larger scale of Amazon data which makes it more likely to over-smooth with irrelevant high-order neighbors. The incorporation of $\mathcal{L}_3$ can cancel out over-smoothing signals.\\\vspace{-0.12in}
    \item \textbf{Comparison to Student and Teacher Models}: From the learning curves in Fig~\ref{fig:ablation_lines}, we can observe the great performance gap between simple MLP student and advanced GNN teacher. The three augmented tasks greatly minimizes this gap by effectively distilling useful knowledge. Additionally, the distillation tasks accelerate the training to surpass the original teacher model.
\end{itemize}

\begin{figure}[t]
    \centering
    \includegraphics[width=0.43\columnwidth]{figs/ablation_converge_Gowalla_Recall.pdf}\quad
    \includegraphics[width=0.43\columnwidth]{figs/ablation_converge_Amazon_Recall.pdf}
    \vspace{-0.12in}
    \caption{Test performance in each epoch for ablated models.}
    \vspace{-0.1in}
    \label{fig:ablation_lines}
    \Description{A line figure showing the performance with respect to epochs for \model\ and some representative baselines. The figure shows that \model\ converges faster while training.}
\end{figure}

\begin{table}[t]
    \centering
    %\small
    \footnotesize
    \setlength{\tabcolsep}{1.4mm}
    % \caption{Model efficiency study on per-epoch training time and inference time on Gowalla, Yelp, and Amazon data.}
    \caption{Model performance and per-epoch model inference time of representative methods on large-scale Tmall dataset.}
    \label{tab:scalability}
    \vspace{-0.1in}
    \begin{tabular}{ccccccc}
        \hline
        Metric & \# Edges & DGCF & SGL & HCCF & NCL & \emph{\model}\\
        \hline
        \hline
        \multirow{2}{*}{R@20} & 1.6M & 0.0221 & 0.0258 & 0.0272 & 0.0286 & \multirow{2}{*}{\textbf{0.0308}}\\
        & 2.9M & 0.0253 & 0.0278 & 0.0283 & 0.0294 & \\
        \hline
        \multirow{2}{*}{N@20} & 1.6M & 0.0258 & 0.0296 & 0.0309 & 0.0337 & \multirow{2}{*}{\textbf{0.0366}}\\
        & 2.9M & 0.0279 & 0.0311 & 0.0319 & 0.0334 & \\
        \hline
        \multirow{2}{*}{Time} & 1.6M & 7190.2s & 1331.8s & 1342.5s & 1392.2s & \multirow{2}{*}{\textbf{785.1s}}\\
        & 2.9M & 11431.8s & 1456.3s & 1530.8s & 1693.8s & \\
        \hline
    \end{tabular}
    \vspace{-0.12in}
    \Description{A table showing the performance and the inference time of \model\ and baselines on the large-scale Tmall dataset. \model\ outperforms the baselines and consumes the least time for inference.}
\end{table}

\vspace{-0.1in}
\subsection{Model Scalability Study (RQ3)}
To validate the efficiency of our \model\ in handling large-scale real-world data, we compare \model\ with the best performed baselines on a e-commerce data collected from Tmall platform. The dataset contains around 40 million records of user clicks. To successfully run on this dataset, GNN-based methods have to sample subgraphs for information propagation. In contrast, graph sampling is not required by the MLP-based inference model of our \model. The performance and the inference time are shown in Table~\ref{tab:scalability}, where we run the baselines using graph sampling strategy~\cite{hu2020heterogeneous} with two scales (\ie, subgraphs contain 1.6M edges and 2.9M edges, respectively). We have mainly two key observations shown as follows:
\begin{itemize}[leftmargin=*]
    \item \textbf{More Accurate Recommendations}: \model\ achieves better recommendation performance in terms of Recall and NDCG. This reflects the higher probability of over-smoothing on the large but sparse interaction graph. Our \model\ avoids this problem without explicit graph message passing. Instead, informative knowledge is distilled from GNNs for model compression.
    \item \textbf{Much Higher Efficiency}: \model\ greatly reduces the inference time on the large Tmall data. \textit{Firstly}, the embedding process of our MLP predictor is agnostic to the holistic interaction graph, thus the large-scale graph does not increase much overhead for embedding processing. No graph sampling is required in comparison to GNNs. \textit{Secondly}, \model\ infers user-item relations based on simple MLPs. The computational costs of fully-connected layers in MLPs are much lower than the cost of GNNs.
\end{itemize}

\begin{figure}[t]
    \centering
    \includegraphics[width=0.3\columnwidth]{figs/hyper_gowalla_soft_Recall_20.pdf}\quad
    \includegraphics[width=0.3\columnwidth]{figs/hyper_gowalla_cd_Recall_20.pdf}\quad
    \includegraphics[width=0.3\columnwidth]{figs/hyper_gowalla_sc_Recall_20.pdf}\\
    \includegraphics[width=0.3\columnwidth]{figs/hyper_gowalla_soft_NDCG_20.pdf}\quad
    \includegraphics[width=0.3\columnwidth]{figs/hyper_gowalla_cd_NDCG_20.pdf}\quad
    \includegraphics[width=0.3\columnwidth]{figs/hyper_gowalla_sc_NDCG_20.pdf}\\
    \vspace{-0.12in}
    \caption{Hyperparameter study for our \model\ model on Gowalla dataset, in terms of \emph{Recall@20} and \emph{NDCG@20}.}
    \vspace{-0.1in}
    \label{fig:hyper2d}
    \Description{A line figure showing the performance change with respect to the weight of the prediction-level distillation, the embedding-level distillation, and the contrastive regularization.}
\end{figure}

\begin{figure}[t]
    \centering
    \subfigure[Pred. Distillation]{
        \includegraphics[width=0.3\columnwidth]{figs/hyper_soft_Recall.pdf}
        \label{fig:hyper3d_pred}
    }
    \subfigure[Embed. Distillation]{
        \includegraphics[width=0.3\columnwidth]{figs/hyper_cd_Recall.pdf}
        \label{fig:hyper3d_embed}
    }
    \subfigure[Contrastive Reg.]{
        \includegraphics[width=0.3\columnwidth]{figs/hyper_sc_Recall.pdf}
    }
    \vspace{-0.17in}
    \caption{Impact of weights and temperature in different learning objectives on Yelp, in terms of \emph{Recall@20}.}
    \vspace{-0.2in}
    \label{fig:hyper3d}
    \Description{A three-D figure showing the composite effect of the weight and the temperature coefficient on the performance, for the prediction-level distillation, the embedding-level distillation, and the contrastive regularization.}
\end{figure}
\subsection{Hyperparameter Study (RQ4)}
In this section, we examine the influence of different hyperparameters on the performance of \model. The effect of loss weights $\lambda_1, \lambda_2, \lambda_3$ are shown in Figure~\ref{fig:hyper2d}. The composite effect of loss weights and corresponding temperatures $\tau_1, \tau_2, \tau_3$ are shown in Figure~\ref{fig:hyper3d}. The effect of the size $|\mathcal{T}_1|$ for the prediction-level distillation is shown in Table~\ref{tab:batch_hyper}. Our observations are as follows:
\begin{itemize}[leftmargin=*]
    \item \textbf{Strength of Prediction-Level Distillation}. $\lambda_1, \tau_1$: This weight $\lambda_1$ and temperature $\tau_1$ jointly control the strength of the prediction-level KD $\lambda_1$. We first study the influence of $\lambda_1$ in Figure~\ref{fig:hyper2d} with $\tau_1$ fixed. When $\lambda_1$ is small, not enough knowledge is distilled to the student model which results in deficient performance. When $\lambda_1$ is too large, $\mathcal{L}_1$ cover up the optimization of main loss and yield degraded performance. Additionally, Figure~\ref{fig:hyper3d_pred} shows the positive effect of applying smaller $\tau_1$ to produce larger gradients.
    
    \item \textbf{Strength of Embedding-Level Distillation}. $\lambda_2, \tau_2$: The parameters control the strength of \model\ in restricting the embeddings in MLP to be close to embeddings in GNN. From Figure~\ref{fig:hyper3d_embed} it can be observed that $\lambda_2$ and $\tau_2$ jointly adjust the strength of embedding KD to have modest influence on optimization, to prevent from insufficient knowledge distillation and too-strict embedding regularization. Either large weight with low temperature or small weight with high temperature causes performance decay.
    
    \item \textbf{Strength of Contrastive Regularization} $\lambda_3, \tau_3$: These parameters determine the strength of push-away regularization for preventing over-smoothing. The results show that either too small weight $\lambda_3$ or too high temperature $\tau_3$ causes insufficient regularization and produces over-smoothed embeddings. Meanwhile, strong regularization may damage the modeling of node-wise affinity, and also yields worse performance.
    
    \item \textbf{Per-Batch Number of Samples to Distill} $|\mathcal{T}_1|$: This hyperparameter determines how many instances are sampled to conduct the prediction-level distillation in each training step. According to the results in Table~\ref{tab:batch_hyper}, increasing batch size brings better KD performance until the performance saturates. We ascribe this to the effect that larger batch size filters low-frequency noise in predictions made by the teacher model in \model.
\end{itemize}


\begin{table}[t]
    %\vspace{-0.05in}
    \caption{Investigation on the impact of batch size in the prediction-oriented distillation of the proposed \model.}
    \vspace{-0.15in}
    \centering
    %\small
    %\scriptsize
    %\ssmall
    \footnotesize
    %\small
    % \setlength{\tabcolsep}{1.2mm}
    \begin{tabular}{c|c|cccccc}
        \hline
        \multirow{2}{*}{Data} & \multirow{2}{*}{Metric} & \multicolumn{6}{c}{Batch Size $|\mathcal{T}_1|$ in Prediction-Level Distillation}\\
        \cline{3-8}
        & & $1e3$ & $5e3$ & $1e4$ & $5e4$ & $1e5$ & $5e5$\\
        \hline
        \hline
        \multirow{2}{*}{Gowalla} & Recall & 0.2208 & 0.2361 & 0.2399 & 0.2420 & 0.2434 & 0.2448\\
        & NDCG & 0.1441 & 0.1530 & 0.1554 & 0.1577 & 0.1592 & 0.1597\\
        \hline
        \multirow{2}{*}{Yelp} & Recall & 0.0443 & 0.0730 & 0.0773 & 0.0802 & 0.0823 & 0.0822\\
        & NDCG & 0.0210 & 0.0372 & 0.0392 & 0.0407 & 0.0414 & 0.0414\\
        \hline
    \end{tabular}
    \vspace{-0.2in}
    \label{tab:batch_hyper}
    \Description{A table recording the performance change of \model\ with respect to the }
\end{table}

\vspace{-0.1in}
\subsection{Over-Smoothing Investigation (RQ5)}
To investigate whether our graph-less \model\ framework is able to mitigate the over-smoothing effect in graph-structured relation learning for CF, we compare representative baselines and our \model\ model on the Mean Average Distance (MAD) values~\cite{chen2020measuring} over embeddings for the most popular users and items. The evaluation results are shown in Table~\ref{tab:mad}. Our \model\ has higher MAD values on both user and item embeddings for Gowalla and Yelp data, in comparison to not only GCN model GCCF, but also state-of-the-art SSL frameworks. It can be concluded that our \model\ framework better addresses the over-smoothing issue, by learning more uniform-distributed embeddings for users and items, to better characterize their unique interaction patterns. This should be attributed to the MLP-based inference framework, and the contrastive regularization that adaptively alleviates over-smoothing signals.


\begin{table}[t]
    %\vspace{-0.05in}
    \caption{Investigation on the ability to address the over-smoothing effect on Gowalla and Yelp data in terms of MAD.}
    \vspace{-0.15in}
    \centering
    % \small
    %\scriptsize
    %\ssmall
    \footnotesize
    %\small
    % \setlength{\tabcolsep}{1.2mm}
    \begin{tabular}{c|c|cccccc}
        \hline
        \multicolumn{2}{c|}{Data} & GCCF & LightGCN & SGL & NCL & HCCF & \emph{\model}\\
        \hline
        \hline
        \multirow{2}{*}{Gowalla} & User & 0.8276 & 0.8203 & 0.8412 & 0.8088 & 0.8394 & \textbf{0.8576}\\
        & Item & 0.7579 & 0.7614 & 0.7702 & 0.8169 & 0.7905 & \textbf{0.8335}\\
        \hline
        \multirow{2}{*}{Yelp} & User & 0.9226 & 0.9610 & 0.9755 & 0.9640 & 0.9749 & \textbf{0.9819}\\
        & Item & 0.6288 & 0.7095 & 0.7191 & 0.6953 & 0.6246 & \textbf{0.7662}\\
        \hline
    \end{tabular}
    \vspace{-0.1in}
    \label{tab:mad}
    \Description{A table presenting the evaluated MAD value of \model\ and baselines. The MAD value of \model\ is higher.}
\end{table}
\section{Conclusion}\label{sec:conclusion}
In this work, we focus on addressing the fundamental challenge of OOD detection tasks, which is how to fully understand the semantic discrepancy between the ID/OOD samples. We reveal that the key to success in the realistic SCOOD task is to allocate as many ID samples in the unlabeled set correctly as possible. To this end, we propose a novel uncertainty-aware optimal transport scheme that introduces class-specific energy scores as guidance for effective label assignment. Experimental results show that our method achieves better performance than previous state-of-the-art methods on SCOOD benchmarks.

\textbf{Limitations.} In addition to temperature scaling, other techniques such as feature clipping applied in ReAct~\cite{sun2021react} also enhance the performance of energy score, so how to obtain an OOD score that best fits the SCOOD task can be further explored. Moreover, a setting highly related to SCOOD has been proposed in \cite{katz2022training} and formulated as a constrained optimization problem. We will also theoretically analyze these practical OOD settings in our feature work.

% \section*{Acknowledgments}
\textbf{Acknowledgments.} 
This work is supported by National Key R\&D Program of China under Grant 2020AAA0105701, National Natural Science Foundation of China (NSFC) under Grants 61872327, Major Special Science and Technology Project of Anhui, National Natural Science Foundation of China (62033012) and Ant Group through Ant Research Intern Program.


% \clearpage
% \documentclass[./main.tex]{subfiles}
\begin{document}

\title{Supplemental Material\\From Clean Room to Machine Room: Commissioning of the First-Generation BrainScaleS Wafer-Scale Neuromorphic System}

\DeclareRobustCommand{\enumauthorrefmark}[1]{\smash{\textsuperscript{\footnotesize #1}}}

\newcommand{\contributedSymbol}{\IEEEauthorrefmark{1}}
\newcommand{\uheiSymbol}{\enumauthorrefmark{1}}
\newcommand{\ugoeSymbol}{\enumauthorrefmark{2}}


\author{
	\IEEEauthorblockN{%
		Hartmut Schmidt\contributedSymbol,
		José Montes\contributedSymbol,
		Andreas Grübl,
		Maurice Güttler,
		Dan Husmann,
		Joscha Ilmberger,\\
		Jakob Kaiser,
		Christian Mauch,
		Eric Müller,
		Lars Sterzenbach,
		Johannes Schemmel,
		Sebastian Schmitt\\
	}

	\thanks{
		\IEEEauthorblockA{%
		\contributedSymbol%
		Contributed equally\\
		}
	}
}

\maketitle
Next, we present the Supplementary Materials for the paper ``Re-ReND: Real-time Rendering of NeRFs across Devices''.
Specifically, in addition to the results reported in the paper, we report results of \methodname w.r.t. Image Quality~(Section~\ref{sec:im_qual}) and (Section~\ref{sec:quali}), Rendering Speed~(Section~\ref{sec:fps}), Mesh Size~(Section~\ref{sec:mesh_size} and Section~\ref{sec:meshi}), Disk Space~(Section~\ref{sec:disk_space}), validation of view-dependent effects (Section~\ref{sec:val}),  sensitivity to geometry variations (Section~\ref{sec:geo}) and Photo-metric quality w.r.t. embedding dimensionality $D$ (Section~\ref{sec:dim}).
Furthermore, we encourage the reviewers to watch the \textbf{associated video}, \texttt{Re-ReND.mp4}, demonstrating \methodname's capabilities of real-time rendering across devices.
% In particular, please refer to .
This video demonstrates how \methodname can render, in real time, a scene composed of tens (\Figure{composit}) or even thousands (\Figure{many_objects}) of objects. % , respectively. %  , or even with thousands of . %  in an AR headset.
\Figure{composit} illustrates such a scene, composed of moving chairs, hotdogs, the drumset, and a microphone.


% Finally, we also provide the PyTorch~\cite{NEURIPS2019_9015} and GLSL implementations of our method inside the folders called \texttt{Re-ReND\_Pytorch\_code} and \texttt{Re-ReND\_GLSL\_code}.

% \thispagestyle{empty}
% \appendix

%%%%%%%%% BODY TEXT - ENTER YOUR RESPONSE BELOW
% \section{The PyTorch code and GLSL code}

%  \begin{itemize}
%     \item Clean and README.md
%     \item Should I upload only pur method or MipNeRF and NeRF++?
%     \item Should I upload the generated data and the meshes in a google drive? What happens with anonymity?
% \end{itemize}

% \section{A video showing how we were measuring the FPS}
% \section{A video showing real scenes in comparison with MobileNeRF and SNeRG}
% \section{Qualitative Results}

%  \begin{itemize}
%     \item all objects visualizations 
% \end{itemize}

%-------------------------------------------------------------------------


\begin{figure}
    \centering
    \includegraphics[width=\linewidth]{pics/quantitative.pdf}
    \caption{Box plots of quantitative benchmarks MIG, FactorVAE, Disentanglement, and reconstruction error on dSprites and Shapes3D.}\label{fig:quantitative}
\end{figure}


\bibliographystyle{style/IEEEtran}
\bibliography{bib/vision}

\end{document}


% Please follow the steps outlined below when submitting your manuscript to
% the IEEE Computer Society Press.  This style guide now has several
% important modifications (for example, you are no longer warned against the
% use of sticky tape to attach your artwork to the paper), so all authors
% should read this new version.

{\small
\bibliographystyle{ieee_fullname}
\bibliography{egbib}
}

\appendix
\documentclass[./main.tex]{subfiles}
\begin{document}

\title{Supplemental Material\\From Clean Room to Machine Room: Commissioning of the First-Generation BrainScaleS Wafer-Scale Neuromorphic System}

\DeclareRobustCommand{\enumauthorrefmark}[1]{\smash{\textsuperscript{\footnotesize #1}}}

\newcommand{\contributedSymbol}{\IEEEauthorrefmark{1}}
\newcommand{\uheiSymbol}{\enumauthorrefmark{1}}
\newcommand{\ugoeSymbol}{\enumauthorrefmark{2}}


\author{
	\IEEEauthorblockN{%
		Hartmut Schmidt\contributedSymbol,
		José Montes\contributedSymbol,
		Andreas Grübl,
		Maurice Güttler,
		Dan Husmann,
		Joscha Ilmberger,\\
		Jakob Kaiser,
		Christian Mauch,
		Eric Müller,
		Lars Sterzenbach,
		Johannes Schemmel,
		Sebastian Schmitt\\
	}

	\thanks{
		\IEEEauthorblockA{%
		\contributedSymbol%
		Contributed equally\\
		}
	}
}

\maketitle
Next, we present the Supplementary Materials for the paper ``Re-ReND: Real-time Rendering of NeRFs across Devices''.
Specifically, in addition to the results reported in the paper, we report results of \methodname w.r.t. Image Quality~(Section~\ref{sec:im_qual}) and (Section~\ref{sec:quali}), Rendering Speed~(Section~\ref{sec:fps}), Mesh Size~(Section~\ref{sec:mesh_size} and Section~\ref{sec:meshi}), Disk Space~(Section~\ref{sec:disk_space}), validation of view-dependent effects (Section~\ref{sec:val}),  sensitivity to geometry variations (Section~\ref{sec:geo}) and Photo-metric quality w.r.t. embedding dimensionality $D$ (Section~\ref{sec:dim}).
Furthermore, we encourage the reviewers to watch the \textbf{associated video}, \texttt{Re-ReND.mp4}, demonstrating \methodname's capabilities of real-time rendering across devices.
% In particular, please refer to .
This video demonstrates how \methodname can render, in real time, a scene composed of tens (\Figure{composit}) or even thousands (\Figure{many_objects}) of objects. % , respectively. %  , or even with thousands of . %  in an AR headset.
\Figure{composit} illustrates such a scene, composed of moving chairs, hotdogs, the drumset, and a microphone.


% Finally, we also provide the PyTorch~\cite{NEURIPS2019_9015} and GLSL implementations of our method inside the folders called \texttt{Re-ReND\_Pytorch\_code} and \texttt{Re-ReND\_GLSL\_code}.

% \thispagestyle{empty}
% \appendix

%%%%%%%%% BODY TEXT - ENTER YOUR RESPONSE BELOW
% \section{The PyTorch code and GLSL code}

%  \begin{itemize}
%     \item Clean and README.md
%     \item Should I upload only pur method or MipNeRF and NeRF++?
%     \item Should I upload the generated data and the meshes in a google drive? What happens with anonymity?
% \end{itemize}

% \section{A video showing how we were measuring the FPS}
% \section{A video showing real scenes in comparison with MobileNeRF and SNeRG}
% \section{Qualitative Results}

%  \begin{itemize}
%     \item all objects visualizations 
% \end{itemize}

%-------------------------------------------------------------------------


\begin{figure}
    \centering
    \includegraphics[width=\linewidth]{pics/quantitative.pdf}
    \caption{Box plots of quantitative benchmarks MIG, FactorVAE, Disentanglement, and reconstruction error on dSprites and Shapes3D.}\label{fig:quantitative}
\end{figure}


\bibliographystyle{style/IEEEtran}
\bibliography{bib/vision}

\end{document}



\end{document}