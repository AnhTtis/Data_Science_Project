\section{Autotuning framework}\label{sec:framework}
\begin{figure*}[]
    \centering
    \includegraphics[width=12cm]{content/method/experimental-pipeline.png}
    \caption{Experiment pipeline, including sample collection process and prediction stage.}
    \label{fig:experiment-pipeline}
\end{figure*}

In this section we will describe the implementation of our autotuning framework and techniques as illustrated in Fig.~\ref{fig:experiment-pipeline}.

\subsection{Measurement}
The total runtime of performing a computation on a GPU is heavily dependent on data transfer between the CPU's main memory and the GPUs memory via the PCIe-bus. Therefore, it is imperative that we first transfer the data and \emph{then} start the measurement timer when executing the kernel. Respectively, the timer needs to stop after the kernel has finished, but \emph{before} the data is transferred back to the host device. This ensures that we measure the actual execution time of the computation and not the additional time to transfer data between devices.

