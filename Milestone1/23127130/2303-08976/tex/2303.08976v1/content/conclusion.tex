\section{Conclusion and Future work} \label{chap:conclusion}
%*** Added motivation: -- gjerne forbedre!
HPC systems with GPU are becoming increasingly more complex and challenging to hand-tune codes for. Autotuning framworks provide means that parameterizes kernels for a range of system parameters.
%**In conclusion, t
BAT 2.0, the new benchmarking suite introduced in this study, provides a comprehensive framework for evaluating the performance of optimization algorithms in modern computing systems utilizing GPUs. The results of our analysis reveal that the optimization parameters have a significant impact on performance and the need for global optimization. The importance of autotuning is highlighted in the portability study, which shows that optimal performance can only be achieved by optimizing each application for a specific target architecture. The benchmarking suite facilitates the study of optimization algorithms and their effectiveness in achieving optimal performance, positioning it as a valuable tool in modern autotuning research.

%\subsection{Current and Future work}
A C++-based interface towards BAT is currently under development to support tuners like KTT~\cite{filipovic_autotuning_2017} and CLTune~\cite{nugteren_cltune:_2015}. 
%*** slight rewording
Future work should thus include a comparison between C++-based and Python-based kernel tuners.
%\input{content/future-work}