% This must be in the first 5 lines to tell arXiv to use pdfLaTeX, which is strongly recommended.
\pdfoutput=1
% In particular, the hyperref package requires pdfLaTeX in order to break URLs across lines.

\documentclass[11pt]{article}

% Remove the "review" option to generate the final version.
% \usepackage[review]{ACL2023}
\usepackage{ACL2023}

% Standard package includes
\usepackage{times}
\usepackage{latexsym}
\usepackage{hyperref}
\usepackage{url}
% Standard package includes
\usepackage{times}
\usepackage{latexsym}
% \usepackage[margin=1in]{geometry}
\usepackage{amsmath}
\usepackage{amssymb}
\usepackage{color}
\usepackage{colortbl}% http://ctan.org/pkg/colortbl
\usepackage{graphicx}
\usepackage{hyperref}
\usepackage{booktabs}
\usepackage[T1]{fontenc}
\usepackage{fancyvrb}
\usepackage{multirow}
\usepackage{array}
\usepackage{listings}
% \usepackage{arydshln}
% \usepackage[table, dvipsnames]{xcolor}
% \usepackage{ulem}
\usepackage{soul}
\usepackage{pifont}
\usepackage{stfloats}
\usepackage{tikz}
\usepackage{adjustbox}
\usepackage{enumitem}
\usetikzlibrary{calc}
\usepackage{tabularx}
\usepackage{arydshln}
% \usepackage{multirow}
% \usepackage{multicol}
\usepackage[]{algorithm2e}
\SetAlFnt{\scriptsize}
\SetAlCapFnt{\scriptsize}
\SetAlCapNameFnt{\scriptsize}
% For proper rendering and hyphenation of words containing Latin characters (including in bib files)
\usepackage[T1]{fontenc}
\usepackage{bbm}
\usepackage{makecell}
% For Vietnamese characters
% \usepackage[T5]{fontenc}
% See https://www.latex-project.org/help/documentation/encguide.pdf for other character sets

% This assumes your files are encoded as UTF8
\usepackage[utf8]{inputenc}
\usepackage{breqn}
\usepackage{amsfonts}

% This is not strictly necessary, and may be commented out,
% but it will improve the layout of the manuscript,
% and will typically save some space.
\usepackage{microtype}
\usepackage{booktabs} % For formal tables
\usepackage{xspace}
\usepackage[capitalise,noabbrev]{cleveref}
\usepackage{hyperref}
\usepackage{subcaption}

% For proper rendering and hyphenation of words containing Latin characters (including in bib files)
\usepackage[T1]{fontenc}
% For Vietnamese characters
% \usepackage[T5]{fontenc}
% See https://www.latex-project.org/help/documentation/encguide.pdf for other character sets

% This assumes your files are encoded as UTF8
\usepackage[utf8]{inputenc}

% This is not strictly necessary, and may be commented out.
% However, it will improve the layout of the manuscript,
% and will typically save some space.
\usepackage{microtype}

% This is also not strictly necessary, and may be commented out.
% However, it will improve the aesthetics of text in
% the typewriter font.
\usepackage{inconsolata}

\makeatletter
\def\adl@drawiv#1#2#3{%
        \hskip.5\tabcolsep
        \xleaders#3{#2.5\@tempdimb #1{1}#2.5\@tempdimb}%
                #2\z@ plus1fil minus1fil\relax
        \hskip.5\tabcolsep}
\newcommand{\cdashlinelr}[1]{%
  \noalign{\vskip\aboverulesep
           \global\let\@dashdrawstore\adl@draw
           \global\let\adl@draw\adl@drawiv}
  \cdashline{#1}
  \noalign{\global\let\adl@draw\@dashdrawstore
           \vskip\belowrulesep}}
\makeatother

\lstset{
  basicstyle=\ttfamily,
  columns=fullflexible,
  % frame=single,
  breaklines=true,
  breakindent=0em,
}


\newcommand{\sys}{ART\xspace} %$\checkmark$

\newcommand{\draftonly}[1]{#1} 
% Uncomment for submission
% \renewcommand{\draftonly}[1]{}
\newcommand{\draftcomment}[3]{\draftonly{{\textcolor{#3}{[\textbf{#1--\textsc{#2}}]}}}}
\newcommand{\bvp}[1]{\draftcomment{\small #1}{bvp}{pink}}
\newcommand{\todo}[1]{\draftcomment{\small #1}{todo}{red}}
\newcommand{\hanna}[1]{\draftcomment{\small #1}{hanna}{blue}}
\newcommand{\marco}[1]{\draftcomment{\small #1}{qq}{green}}
\newcommand{\sameer}[1]{\draftcomment{\small #1}{sameer}{teal}}


\usepackage{fontawesome}
\usepackage{soul}
\newcommand{\colbox}[2]{ \begingroup
  \setlength{\fboxsep}{0pt}%  
  \colorbox{#1}{#2\/}%
  \endgroup}
\definecolor{lightgrey}{rgb}{0.95, 0.95, 0.95}
\definecolor{darkgrey}{rgb}{0.55, 0.55, 0.55}
\definecolor{cexample}{rgb}{0.23, 0.30, 0.45}
\usepackage{xspace}
\newcommand{\fexinline}[1]{#1\xspace}
\newcommand{\test}[2]{\colbox{lightgrey}{\texttt{\fexinline{#1}}}}
\newcommand{\exinline}[1]{\textcolor{cexample}{``#1''\xspace}}
\usepackage{amsmath}
% \newcommand{\test}[2]{\colbox{lightgrey}{\texttt{{#1\textcolor{darkgrey}\xspace}}}}

\colorlet{tableheadcolor}{gray!25}
\colorlet{tablerowcolor}{gray!10}
\colorlet{search}{blue!10}
\colorlet{string}{red!10}
\colorlet{code}{yellow!10}
\colorlet{arithmetic}{green!10}
\colorlet{freeform}{pink!10}
% \colorlet{freeform}{pink!10}


% If the title and author information does not fit in the area allocated, uncomment the following
%
%\setlength\titlebox{<dim>}
%
% and set <dim> to something 5cm or larger.

\title{\sys: Automatic multi-step reasoning and tool-use for\\large language models}

% Author information can be set in various styles:
% For several authors from the same institution:
% \author{Author 1 \and ... \and Author n \\
%         Address line \\ ... \\ Address line}
% if the names do not fit well on one line use
%         Author 1 \\ {\bf Author 2} \\ ... \\ {\bf Author n} \\
% For authors from different institutions:
% \author{Author 1 \\ Address line \\  ... \\ Address line
%         \And  ... \And
%         Author n \\ Address line \\ ... \\ Address line}
% To start a seperate ``row'' of authors use \AND, as in
% \author{Author 1 \\ Address line \\  ... \\ Address line
%         \AND
%         Author 2 \\ Address line \\ ... \\ Address line \And
%         Author 3 \\ Address line \\ ... \\ Address line}

\author{Bhargavi Paranjape, Scott Lundberg, Sameer Singh, 
  Hanna Hajishirzi, Luke Zettlemoyer, Marco Tulio Ribeiro}

\author{Bhargavi Paranjape$^{1}$~ Scott Lundberg$^{2}$~ Sameer Singh$^{3}$~ Hannaneh Hajishirzi$^{1,4}$\\
\textbf{Luke Zettlemoyer}$^{1,5}$~ \textbf{Marco Tulio Ribeiro}$^{2}$ \\
$^{1}$University of Washington,~$^{2}$Microsoft Research, ~ $^{3}$University of California, Irvine,\\$^{4}$Allen Institute of Artificial Intelligence,~ $^{5}$Meta AI
}

% \begin{document}
% <<mytable, results='asis', echo=F>>=
%   library("xtable")
% print(xtable(matrix(rnorm(32), 8,4), align="l|cc|cc|c"), 
%       booktabs=T, 
%       hline.after = c(-1,0,8), 
%       add.to.row = list(pos=list(4), command="\\hdashline \n"))
% @
% \end{document}

\begin{document}
\maketitle
\begin{abstract}


Over the past few years, there has been a significant amount of research focused on studying the ReLU activation function, with the aim of achieving neural network convergence through over-parametrization. However, recent developments in the field of Large Language Models (LLMs) have sparked interest in the use of exponential activation functions, specifically in the attention mechanism.

Mathematically, we define the neural function $F: \R^{d \times m} \times  \mathbb{R}^d \rightarrow \mathbb{R}$ using an exponential activation function. Given a set of data points with labels $\{(x_1, y_1), (x_2, y_2), \dots, (x_n, y_n)\} \subset \mathbb{R}^d \times \mathbb{R}$ where $n$ denotes the number of the data. Here $F(W(t),x)$ can be expressed as $F(W(t),x) := \sum_{r=1}^m a_r \exp(\langle w_r, x \rangle)$, where $m$ represents the number of neurons, and $w_r(t)$ are weights at time $t$. It's standard in literature that $a_r$ are the fixed weights and it's never changed during the training. We initialize the weights $W(0) \in \mathbb{R}^{d \times m}$ with random Gaussian distributions, such that $w_r(0) \sim \mathcal{N}(0, I_d)$ and initialize $a_r$ from random sign distribution for each $r \in [m]$.

Using the gradient descent algorithm, we can find a weight $W(T)$ such that $\| F(W(T), X) - y \|_2 \leq \epsilon$ holds with probability $1-\delta$, where $\epsilon \in (0,0.1)$ and $m = \Omega(n^{2+o(1)}\log(n/\delta))$. To optimize the over-parametrization bound $m$, we employ several tight analysis techniques from previous studies [Song and Yang arXiv 2019, Munteanu, Omlor, Song and Woodruff ICML 2022]. 

 

\end{abstract}

\section{Introduction}
\documentclass[../main.tex]{subfiles}
\begin{document}

Magnetically actuated medical robots (MAMR) have seen significant focus and development in recent decades due to their potential for miniaturization~\cite{hu2018small}, tether-less actuation~\cite{popek2016six} and high number of controllable degrees-of-freedom (DOFs)~\cite{salmanipour2018eight,pittiglio2022collaborative}. In fact, magnetically guided catheters have been used to treat cardiac arrhythmias since 2003~\cite{nelson2022magnetically,carpi2009stereotaxis}.

A key aspect in their actuation is pose estimation~\cite{bianchi2019localization,barducci2019adaptive}, enabling closed loop control and delivery of functionality~\cite{norton2019intelligent}. Imaging techniques have long been used for this purpose but are generally tied to limited resolution, harmful radiation exposure and need for additional hospital equipment~\cite{aziz2020medical,pane2022ultrasound,daguerre2022localization}. As such, methods based on magnetic field measurements have received significant attention, with magnetic tracking systems being widely available on the market. These, however, are not compatible with magnetic actuation systems due to distortions on the localization magnetic fields.

To address this issue, significant research on magnetic localization coupled with magnetic actuation systems has been done~\cite{khalil2019magnetic,popek2016six,son2018simultaneous,shao2019novel,taddese2018enhanced}. Several works have been based on magnetic field sensing arrays external to MAMR~\cite{micheal20222d,son2018simultaneous}. While advantageous from a miniaturization and internal power consumption point of view, these systems require calibration of large sensor arrays and have limited localization workspace dimensions. Internal sensing to the MAMR, on the other hand, does not suffer from workspace dimension restrictions. It requires, however, on-board power and heterogeneous localization magnetic fields, with 6-DOF localization having been shown for systems with a single external permanent magnet (EPM). \textcolor{black}{Internal sensing methods have been shown for endoscopic capsules, as well as magnetically guided catheters}~\cite{popek2016six,sperry2022six,taddese2018enhanced,fischer2022using}.

Over recent years the need for enhanced control and manipulability of MAMRs has led to the advent of actuation platforms based on multiple magnetic field sources (MMFS) such as multiple electromagnetic coils and multiple permanent magnets~\cite{kummer2010octomag,hoang2019independent,hong2020magnetic,pittiglio2022collaborative,ryan2017magnetic,stereotaxis_patent}. Some of these platforms have been cleared for human use such as Stereotaxis Genesis RMN\textsuperscript{\tiny\textregistered} based on two permanent magnets, and Magnetecs and Aeon Scientific based on multiple electromagnetic coils. 

Despite this progress, magnetic localization for such systems is lagging behind, with fluoroscopic imaging being currently used~\cite{nelson2022magnetically}. Unlike single magnetic field source systems where the singularity regions and localization limitations have been thoroughly investigated and solved for~\cite{taddese2018enhanced}, magnetic localization for MMFS systems suffers from additional challenges due to the superposition of the magnetic fields leading to configuration-specific singularity regions. Only recently, a 3D position localization system with internal magnetic field sensing was demonstrated for a multi-coil system, \textcolor{black}{for a 3~mm catheter}~\cite{fischer2022using}.

Furthermore, a common conundrum in 6-DOF magnetic localization with internal sensing is finding the rotation around gravity, due to the absence of the Earth's magnetic field measurement~\cite{mahony2008nonlinear}. This has been solved in the past by accurately initializing this missing rotation angle and tracking it with a gyroscope~\cite{pittiglio2020observability, di2016jacobian}. However, this is prone to errors over time, especially for slow moving systems where gyroscope data is not as sensitive. Additionally, if communication to the MAMR is lost, a new accurate initialization is needed, proving impossible mid medical intervention. More recently, Taddese et al.~\cite{taddese2018enhanced} fitted an auxiliary coil around a single EPM providing a second set of magnetic field measurements. This solves the missing rotation angle and is also able to eliminate the localization singularity plane when it comes to localization with respect to a single EPM. However, when MMFS are present in the workspace, that singularity plane ceases to exist due to the superposition of magnetic fields, and instead singularity regions are present depending on the relative pose of each EPM.

This paper introduces, for the first time, a 6-DOF magnetic localization method for systems with multiple EPMs without the need for any prior pose information. The method relies on measurements from an accelerometer and a single 3D magnetic field Hall effect sensor (HE), both internal to the MAMR. We analyze the effect that the number of EPMs in the workspace has on the full pose estimation; and demonstrate its performance in a two EPM magnetic actuation platform. Since adding an orthogonal coil is not able to solve for the singularity regions, in this work we do not consider it and instead solve for the missing rotation angle by using multiple magnetic field measurements at different EPM configurations. This works for static or quasi-static systems, with maximum MAMR velocity highly dependent on the actuation system and the magnetic field generated. This is the case for non-actuated parts of a larger system, such as the deployment point at the tip of an endoscope, or for MAMRs while the generated magnetic fields are sufficiently weak to induce actuation. Additionally, unlike common works in literature which parameterize the rotation matrix, in this work the full 6-DOF pose is estimated directly in the special euclidean group $SE(3)$. This avoids any singularities or non-unique representations of the orientation when using Euler angles or quaternions~\cite{mathavaraj2021se,mayhew2011quaternion,taddese2018enhanced}.

\end{document}


\section{Related Work}
\section{Related work}
% There is extensive recent work on speeding up analytical queries due to the need for consistent execution times in the face of the explosive growth in the volume of available data.
% In this section, we divide existing work into two categories: maintaining data freshness in MVs (\Cref{sec:server_side}) and optimizations for minimizing ad-hoc query latency (\Cref{sec:client_side}).

% \subsection{Maintaining Data Freshness in MVs}
% \label{sec:server_side}
% There exists a variety of data warehousing applications aimed at supporting low-latency analytical queries on fresh data.
% In particular, these applications require efficiency in the propagation of newly ingested data into downstream MVs.
 
\mypara{Efficient MV Refresh}
Incremental view maintenance (IVM) aims to update MVs to reflect newly ingested data, taking advantage of already computed results to perform the update in a manner more efficient than computing from scratch (full refresh)
~\cite{ahmad2012dbtoaster,mcsherry2013differential,armbrust2013generalized,zeng2016iolap, palpanas2002incremental, griffin1995incremental, agiwal2021napa, braun2015analytics}. 
There is an abundance of work in IVM, including incremental updates on duplicate values~\cite{griffin1995incremental}, non-distributive aggregate functions~\cite{palpanas2002incremental}, higher-order views~\cite{ahmad2012dbtoaster}, and sliding windows~\cite{braun2015analytics}. 
More recent works also investigate the scalability aspect of IVM, proposing scale-independent updates~\cite{armbrust2013generalized} and sampled views~\cite{zeng2016iolap}. Since \system is applicable to arbitrary SQL statements, \system is orthogonal to and is fully compatible with existing IVM techniques.

\mypara{MV Refresh Scheduling}
There exist works on scheduling the refresh of a MV set focusing on resolving cyclic dependencies~\cite{folkert2005optimizing}, minimizing weighted average staleness~\cite{golab2009scheduling}, and prioritizing MVs with the highest speedups on predicted future queries~\cite{ahmed2020automated}.
\system's scheduling to speed up the end-to-end refresh of the MV set is not addressed in existing works.

\mypara{DAG Workflow Scheduling}
The execution of workloads consisting of individual jobs with acyclic dependencies is a well-studied topic~\cite{apacheoozie,sparkdag,marchal2018parallel,bathie2020revisiting,baruah2022ilp}; many of these techniques can be applied to MV refresh runs studied in this paper.
Existing workflow scheduling systems such as Apache Oozie~\cite{apacheoozie}, Apache Airflow~\cite{airflow}, and Spark DAG scheduler~\cite{sparkdag} automate the execution of user-defined workflows following a topological order.
There are recent works aimed at finding more optimal execution orders in terms of peak memory usage~\cite{marchal2018parallel, bathie2020revisiting} and execution time on parallel platforms~\cite{baruah2022ilp}.
While \system is designed for use with MV refresh runs/workloads, our technique on joint scheduling and optimization can be reasonably applied to general workloads as a possible future direction.

% \paragraph{Incremental MV indexing}
% Update-optimized indices such as the log-structured merge-trees (LSM)~\cite{o1996log} are used for indexing MVs due to frequent updates induced by data ingestion~\cite{gupta2016mesa,agiwal2021napa}.
% \system is orthogonal to indexing: \system is capable of efficiently performing MV refresh runs regardless of whether the individual MVs are indexed or not.

% \subsection{Ad-hoc Query Latency Reduction}
% \label{sec:client_side}

% The minimization of ad-hoc analytical query response times is a well-studied topic due to latency being negatively correlated with the productivity of a data analyst during a data analysis session~\cite{liu2014effects}.
% Sessions are commonly conducted within visualization systems that contain a variety of optimization techniques to ensure that query response times fall within a certain latency tolerance.

% \mypara{Data prefetching}
% Data is often loaded into memory on a by-need basis in visualization systems to minimize interference with user-issued query computations~\cite{mani2017effective,xin2021enhancing,galakatos2017revisiting, yan2020auto, battle2016dynamic, crotty2016case, jalaparti2018netco}. 
% Query-time data retrieval can be significantly expedited by anticipating the data usage of the user in future queries and pre-loading the data into memory during the downtime between user queries (`think time'). SMART~\cite{mani2017effective} prefetches data for modified versions of current user-issued queries with different filters and dimensions. A-WARE~\cite{crotty2016case} maintains a sample store constantly refined through ingesting data based on speculations of future plots.
% ForeCache~\cite{battle2016dynamic} uses an SVM to predict the user's current analysis phase and accordingly prefetches data tiles partitioned based on different numerical values. NetCo predicts future queries via log analysis, and solves an ILP formulation to prefetch data to maximize the number of SLO-meeting queries~\cite{jalaparti2018netco}.
% In the case of MV refresh workloads, `think time' is nonexistent as individual MVs are refreshed back-to-back, rendering data prefetching techniques non-applicable.

\mypara{Intermediate Data Caching}
Some existing data visualization systems cache user-defined variables to support the typical incremental construction of data visualizations~\cite{zgraggen2016progressive, eichmann2020idebench} during data analysis sessions~\cite{jupyter, rstudio, colab}. 
Recent work proposes a management scheme for these cached variables under a memory constraint that greedily keeps variables with the highest estimated time savings based on predicted future user behavior via neural networks~\cite{xin2021enhancing}.
While useful for data visualization, a greedy approach to memory management fails to achieve satisfactory results compared to \system.

\mypara{Intermediate Result Reuse}

There exist works on storing intermediate results from computations to speedup future computations~\cite{yang2018intermediate, dursun2017revisiting, nagel2013recycling, michiardi2019memory, galakatos2017revisiting}.
Studied topics include the identification of reuse opportunities by finding overlaps in computation graphs of successive jobs~\cite{yang2018intermediate, michiardi2019memory},
selective storage under a space constraint with heuristics such as reuse probability~\cite{dursun2017revisiting}, expected savings~\cite{yang2018intermediate}, and recompute-storage cost difference~\cite{nagel2013recycling},
and rewriting incoming jobs to take advantage of stored intermediates~\cite{galakatos2017revisiting}.
These works share similarity with \system in their selection of items to store under a memory constraint, however, \system's problem setting requires it to uniquely consider the joint (re)ordering of job executions along with the selection of items.

% work that considers both job execution (re)order as well as intermediate result caching with a bounded amount of memory. but notably lack the joint aspect of \system and cannot be used to achieve immediate speedup on an incoming MV refresh run if no intermediates are stored beforehand. 

\mypara{Incremental Query Processing} Incremental processing (IQP) is useful for cases where not all data required for a query is immediately available. Similar to online aggregation~\cite{hellerstein1997online}, initial results of a query are computed on a subset of required data and progressively refined as the rest of the required data arrives in a predictable pattern~\cite{tang2019intermittent,wangtempura}. Tang et al. propose a dynamic programming formulation to pick intermediate states to store in memory given a limited memory budget~\cite{tang2019intermittent}. Tempura rewrites the query plan for more efficient execution based on predicted data arrival patterns~\cite{wangtempura}. While similarities exist between the problem setting of IQP and \system, such as management of bounded memory, \system notably includes additional joint optimization for the order of MV updates.

% \paragraph{Sampling}
% Sampling has seen wide use in visualization systems for reducing the computation time of ad-hoc queries by computing an approximate result over a subset of data as exact results are not always required by the user~\cite{crotty2016case, mani2017effective, zgraggen2014panoramicdata, kraska2021northstar, galakatos2017revisiting, kandula2016quickr}. 
% Commonly studied topics in sampling for ad-hoc queries include complex query sampling~\cite{kandula2016quickr}, rare event aggregation~\cite{kraska2021northstar, galakatos2017revisiting}, and maintaining consistency between related sampled visualizations~\cite{zgraggen2014panoramicdata}.
% Sampling server-side at the MV level compromises the assumptions of downstream applications and is thus not considered in \system.

% \paragraph{Progressive visualization}
% The latency tolerance for time-consuming queries can be circumvented by presenting a partially-computed visualization to the user within the tolerance, which is then incrementally refined until it is fully accurate~\cite{rahman2017ve, zgraggen2016progressive, crotty2015vizdom, kraska2021northstar, kamat2017infiniviz}.
% Example plots which benefit from progressive visualization include bar charts~\cite{kamat2017infiniviz} and heatmaps~\cite{rahman2017ve}.
% Similar to sampling, study on this topic is orthogonal to \system as pushing out partially-updated MVs compromises downstream assumptions.

\section{\sys}
\section{Method}
\label{sec:method}

% \ml{``Inconsistent'' to ``large variation''}

% In this section, we propose our methods based on the observations in Section \ref{sec:motivation}.
In this section, we propose two techniques to further enhance the strong baseline to capture the variation of activation distributions better.
We first introduce spatial re-scaling to adapt the network to pixel-to-pixel variation.
We then propose channel-wise shifting and re-scaling to better capture the channel-to-channel variation.
Meanwhile, as both of the two methods are image-dependent, the image-to-image variation can be captured naturally.
By combining the two methods with our strong baseline, we build our enhanced BNN for SR, named EBSR.

% Because the activation distributions among pixels, channels and images have large variations \red{**are highly inconsistent} in SR networks, we introduce spatial re-scaling to adapt to pixel-wise variations and channel shift and re-scaling to adapt to channel-wise variations. And both of them are image-dependent to adapt to image-wise variations, which means during inference our network re-scales and shifts the distributions of activations flexibly for different input images. Based on these methods, we build an enhanced binary neural network for image super-resolution (EBSR).

% According to [3], the difference of activation magnitudes indicates different scaling factors are needed for each pixel.

\subsection{Spatial Re-scaling}
% It is better to use different scaling factors for different pixels to reduce the quantization error and retain more detailed information for image super-resolution. 

% \ml{In the main method, we do not need to introduce the previous works but can focus on introducing our own method. Channel rescaling in Real-to-binary Net is not relevant in this context.}

% Re-scaling the output of binary convolutions was proposed at the birth of BNN in XNOR-Net \cite{rastegari2016xnor} to reduce quantization error and improve accuracy for image classification tasks.
% It is computed as below:
% \begin{equation}
% \mathcal{A} * \mathcal{W} \approx(\operatorname{sign}(\mathcal{A}) \circledast \operatorname{sign}(\mathcal{W})) \odot \mathcal{K} \alpha
% \label{eq:xnor-net rescale}
% \end{equation}
% where $\circledast$ denotes the binary convolution and $\odot$ denotes the element-wise multiplication.
% $\mathcal{A}$, $\mathcal{W}$, $\alpha$, and $\mathcal{K}$ denote the activation, weight, weight scaling factor, and activation scaling factor, respectively.
%  Later in XNOR-Net++ \cite{bulat2019xnor}, Bulat et al. fuse the activation and weight scaling factors into a single one that is learned end-to-end based on gradients and this improves the classification accuracy on ImageNet dataset.

% % It is computed as Eq.~\ref{eq:xnor-net rescale}, where $\circledast$ denotes 
% %  the binary convolution and $\odot$ denotes the element-wise multiplication. The binary convolution of $\mathcal{A}$ and $\mathcal{W}$ is rescaled by the weight scaling factor $\alpha$ and the activation scaling factor $\mathcal{K}$, both of which are calculated analytically.


% \zc{Similarly, you should explain the meaning of A, W and the operators $\circledast$ in the formula}
% Then in Real-to-binary Net \cite{martinez2020training}, Martinez et al. used a data-driven channel re-scaling module that takes the pre-convolution activations as input to predict the activation scaling factor. Unlike that in XNOR-Net++ \cite{bulat2019xnor}, these scaling factors are not fixed during inference but rather inferred from data. By doing this, they further improved the classification accuracy on ImageNet over XNOR-Net++. 
As is shown in Figure \ref{fig:pixel}, activation distributions have large pixel-to-pixel variation in SR networks
and the difference of activation magnitudes indicates different scaling factors are preferred for different pixels.
Inspired by \cite{martinez2020training}, we propose spatial re-scaling to better adapt the network to the spatial variation
of activation distributions in SR networks.
% fit the various pixel-wise distributions in SR networks.
We take the real-valued activations $A$ before convolution as input and predict pixel-wise scaling factors $S(A)$, which re-scale the binary convolution output. Spatial re-scaling process can be formulated as follows:
\begin{equation}
A * W \approx(\operatorname{sign}(A) \circledast \operatorname{sign}(W)) \odot \alpha \odot S(A)
\label{eq:spatial rescale}
\end{equation}
where $\circledast$ denotes 
the binary convolution and $\odot$ denotes the element-wise multiplication. $A$, $W$, $\alpha$, and $S\left(A\right)$ denote real-valued activations, weights, the scaling factor of weights, and the spatial-wise scaling factor of activations respectively. $S\left(A\right) \in \mathbb{R}^{1\times H\times W}$ can be calculated with a convolution and a sigmoid function.
% as $\sigma\left( CONV\left(A\right)\right)$. 
As shown in Figure \ref{fig:method}(a), real-valued activations first go through a convolution layer,
which has an input channel of $C$ and an output channel of 1, 
and then pass through a sigmoid function to produce the scaling factors $S(A)$ along the spatial dimension.
During inference, the scaling factor will change dynamically according to different input feature maps.
By re-scaling binary convolution output using $S(A)$, we can reduce the quantization error and the original pixel-wise information in FP activation
will be preserved much better.
Spatial re-scaling leads to a large PSNR improvement of 0.24 dB (from 30.30 dB to 31.54 dB) on Set5 and 0.22 dB (from 25.09 dB to 25.31 dB)
on Urban100 compared with our strong baseline. 

\subsection{Channel-wise Shifting and Re-scaling}

\begin{table}[!tb]
\centering
\caption{Comparison between whether to fuse channel-wise shifting and re-scaling or not based on our baseline with spatial re-scaling. }
\label{tab:fusing}

\scalebox{0.65}{
\begin{tabular}{c|cc|cc|cc}
\hline
\multirow{2}{*}{Method}     & \multirow{2}{*}{OPs} & \multirow{2}{*}{Params} & \multicolumn{2}{c|}{Set5} & \multicolumn{2}{c}{Urban100} \\ \cline{4-7} 
                            &                      &                         & PSNR        & SSIM        & PSNR          & SSIM         \\ \hline
Baseline + spatial re-scale & 2.16G                & 0.05M                   & 31.54       & 0.883       & 25.31         & 0.759        \\
+ channel-wise shift and re-scale             & 2.34G                & 0.09M                   & 31.61       & 0.885       & 25.35         & 0.761        \\
+ w/ fusing                   & 2.27G                & 0.08M                   & \textbf{31.64}       & \textbf{0.885}       & \textbf{25.36}         & \textbf{0.761}        \\ \hline
\end{tabular}
}
\end{table}

In SR networks, activation distributions exhibit larger channel-to-channel variation (Figure \ref{fig:chl}).
Both the mean and magnitude of the activation distributions vary significantly across channels.
% Thus we use channel-wise shifting and re-scaling to adapt to various channel-wise distributions. 
\cite{martinez2020training} has proposed the data-driven channel re-scaling, 
but our method differs from them in further introducing data-driven thresholds to handle the channel-wise variation of both mean and magnitude.
Since the blocks to generate the scaling factors and thresholds are very similar, we further propose to fuse them into one module.
% and fusing channel-wise shifting and re-scaling into one module.
We evaluate the effect of fusing the two blocks in Table \ref{tab:fusing}.
With channel-wise shifting and re-scaling fused, our models have fewer operations and parameters overhead and slightly higher performance.

For the specific process, we take the real-valued activations as input and predict different thresholds and scaling factors for each channel. They are also image dependent, e.g., $\beta_{i}$ in Eq.\ref{eq:act_binarize} is no longer fixed during inference but generated according to different input feature maps. Channel-wise shifting and re-scaling can be formulated as follows:
\begin{equation}
A * W \approx(\operatorname{sign}(A-C_s(A)) \circledast \operatorname{sign}(W)) \odot \alpha \odot C_r(A)
\label{eq:channel-wise_shift_and_rescale}
\end{equation}
where $\circledast$ denotes 
the binary convolution and $\odot$ denotes the element-wise multiplication. $C_s(A), C_r(A) \in \mathbb{R}^{C\times1\times1}$ denote the channel-wise threshold and scaling factor, respectively. 
We show the block diagram in Figure \ref{fig:method}(b).
The real-valued input feature map is first squeezed to a ${C\times1\times1}$ vector by a global average pooling (GAP) layer.
The subsequent fully connected layers and ReLU learn the channel-wise information and output a ${2C\times1\times1}$ vector.
Then the ${2C\times1\times1}$ vector is split into two ${C\times1\times1}$ vectors.
We use the first $C$ channels as the channel-wise bias and pass the last $C$ channels through a sigmoid layer 
as the channel-wise scaling factor, which are used to shift the real-valued activations and re-scale the binary convolution output, respectively. 


% \ml{We can mention previously, channel-wise re-scale has been proposed. We propose to fuse them. Add the comparison between fuse v.s. no fuse.}

\begin{figure}[!tbp]%
  \centering
    \includegraphics[width=0.4\textwidth]{fig/methods.png}
  
% \subfloat[channel-wise shifting\&re-scale]{
%     \label{subfig:channel-wise shifting and re-scale}
%     \includegraphics[width=0.2\textwidth]{fig/chl shift and rescale.png}
%   }

  \caption{Block diagram for spatial re-scaling, and channel-wise shifting and re-scaling.} 
  % Input A is the real-valued activation tensor and C, H, and W denote its dimension. GAP stands for global average pooling. The reduction ratio r is set to 16 for a better trade-off between the performance and the number of operations and parameters.}
  \label{fig:method}
\end{figure}


\subsection{Network Structure}

Combining the spatial re-scaling and the channel-wise shifting and re-scaling methods, we construct the enhanced convolution layer (E-Conv).
Then we build our EBSR model based on E-Conv.
In Figure \ref{fig:E-conv}, we compare the binary convolution layer used in the baseline network and our proposed E-Conv.
We use spatial and channel-wise scaling factors to re-scale the binary convolution output,
and use channel-wise shifting to learn appropriate thresholds for each channel before binarization.
The scaling factors and threshold used in E-Conv are learnable and depend on the real-valued input activations.
In this way, our proposed EBSR can adapt to pixel-to-pixel, channel-to-channel, and image-to-image variations
to reduce the large binarization error and preserve more details.
% In this way, our proposed E-Conv reduces the large quantization error caused by binarization and keeps the original information of input feature maps to a large extent.


\begin{figure}[!tb]%
  \centering

    \includegraphics[width=0.5\textwidth]{fig/E-conv.png}

  \caption{Comparison of (a) the binary convolution layer with a skip connection used in our baseline network and (b) the proposed E-Conv.}
  \label{fig:E-conv}
\end{figure}


Figure \ref{fig:network} shows the basic block based on the E-Conv and our EBSR composed of the basic blocks. Following existing works, the convolution layers in the head and tail modules are not binarized. We choose the lightweight EDSR which has 16 basic blocks and 64 channels, and EDSR which has 32 basic blocks and 256 channels as our backbones, which correspond to EBSR-light and EBSR, respectively.

\begin{figure}[!tb]%
  \centering
  {
    \includegraphics[width=0.35\textwidth]{fig/network.png}
  }
  
  \caption{The structure of our proposed EBSR.  Convolution layers in purple are real-valued vanilla 3x3 convolutions.}
  \label{fig:network}
\end{figure}

\section{Experimental Setup}
%% Macro setup
\definecolor{purple}{rgb}{1, 0, 1}

\newcommand{\ie}{\emph{i.e.,}\xspace}
\newcommand{\eg}{\emph{e.g.,}\xspace}
\newcommand{\abr}{\emph{abbr.}\xspace}
\newcommand{\ea}{\emph{et al.}\xspace}
\newcommand{\gensync}{\emph{GenSync}\xspace}
\newcommand{\colosseum}{\emph{Colosseum}\xspace}
\newcommand{\srep}{\emph{SREP}\xspace} % Set Reconciliation Enhances
\newcommand{\srepsim}{\emph{SREPSim}\xspace}
% Propagation
\newcommand{\esrep}{\emph{E-SREP}\xspace}
\newcommand{\epsrep}{\emph{EP-SREP}\xspace}
\newcommand{\mesrep}{\emph{ME-SREP}\xspace}
\newcommand{\mempoolsync}{\emph{MempoolSync}}

\newcommand{\fref}[1]{Fig.~\ref{#1}}
\newcommand{\tref}[1]{Table~\ref{#1}}
\newcommand{\aref}[1]{Algorithm~\ref{#1}}
\newcommand{\procref}[1]{Procedure~\ref{#1}}
\newcommand{\sref}[1]{Section~\ref{#1}}
\newcommand{\lineref}[1]{line~\ref{#1}}
\newcommand{\appref}[1]{Appendix~\ref{#1}}

% Change \eqref
\LetLtxMacro{\originaleqref}{\eqref}
\renewcommand{\eqref}{Eq.~\originaleqref}

% Theorems and corollaries
\newcounter{theoremcount}
\setcounter{theoremcount}{0}
\DeclareRobustCommand{\theorem}[1]{%
  \refstepcounter{theoremcount}%
  \noindent\textit{\textbf{Theorem \thetheoremcount\label{theorem:#1}: }}%
}
\DeclareRobustCommand{\theoremref}[1]{Theorem~\ref{theorem:#1}}

\DeclareRobustCommand{\proof}{\emph{Proof:}\xspace}
\DeclareRobustCommand{\qqed}{\hfill$\blacksquare$}

\newcounter{corollcount}
\setcounter{corollcount}{0}
\DeclareRobustCommand{\coroll}[1]{%
  \refstepcounter{corollcount}%
  \noindent\textit{\textbf{Corollary \thecorollcount\label{coroll:#1}: }}%
}
\DeclareRobustCommand{\corollref}[1]{Corollary~\ref{coroll:#1}}

\newcounter{lemmacount}
\setcounter{lemmacount}{0}
\DeclareRobustCommand{\lemma}[1]{%
  \refstepcounter{lemmacount}%
  \noindent\textit{\textbf{Lemma \thelemmacount\label{lemma:#1}: }}%
}
\DeclareRobustCommand{\lemmaref}[1]{Lemma~\ref{lemma:#1}}

\newcounter{definitioncount}
\setcounter{definitioncount}{0}
\DeclareRobustCommand{\definition}[1]{%
  \refstepcounter{definitioncount}%
  \noindent\textit{\textbf{Definition \thedefinitioncount\label{definition:#1}: }}%
}
\DeclareRobustCommand{\defref}[1]{Definition~\ref{definition:#1}}

%notes of different authors
\newif\ifnotes
\notestrue
\notesfalse

\newif\ifdiff
\difftrue
\difffalse

\newcommand{\anote}[1]{\ifnotes $\ll$\textsf{\textcolor{purple}{Ari: {#1}}}$\gg$ \fi}
\newcommand{\nnote}[1]{\ifnotes $\ll$\textsf{\textcolor{orange}{Novak: {#1}}}$\gg$ \fi}
\newcommand{\diff}[1]{\ifdiff\textcolor{orange}{#1}\else#1\fi}

%%% Local Variables:
%%% mode: latex
%%% TeX-master: "main"
%%% End:


%%% Redefining theorem-like environments
\newcounter{environments}

\newcounter{theoremCounter}
\newcounter{lemmaCounter}
\newcounter{definitionCounter}
\newcounter{propositionCounter}
\newcounter{corollaryCounter}
\newcounter{exampleCounter}
\newcounter{remarkCounter}
\newcounter{propertyCounter}
\newcounter{assumptionCounter}
\newcounter{proofCounter}

%\theorempreskip{1pt}
%\theorempostskip{1pt}

\let\proposition\relax
\let\theorem\relax
\let\lemma\relax
\let\definition\relax
\let\corollary\relax
\theoremseparator{.}
\theorembodyfont{\itshape}
\theoremsymbol{$\triangleleft$}
\newtheorem{theorem}[theoremCounter]{Theorem}
\newtheorem{lemma}[lemmaCounter]{Lemma}
\newtheorem{definition}[definitionCounter]{Definition}
\newtheorem{proposition}[propositionCounter]{Proposition}
\newtheorem{corollary}[corollaryCounter]{Corollary}

\let\remark\relax
\let\example\relax
\let\assumption\relax

\theorembodyfont{\normalfont}
\newtheorem{example}[exampleCounter]{Example}
\newtheorem{remark}[remarkCounter]{Remark}

\theoremheaderfont{\itshape}
\theoremsymbol{}
\renewtheorem{property}[remarkCounter]{Property}

\theoremheaderfont{\bfseries}
\theorembodyfont{\itshape}
\newtheorem{assumption}[assumptionCounter]{Assumption}
\theoremheaderfont{\itshape}


\theoremstyle{plain}
\theoremheaderfont{\itshape}
\theorembodyfont{\normalfont}
\let\proof\relax
\theoremseparator{.}
\theoremsymbol{\qedfull}
\newtheorem*{proof}{Proof}
\qedsymbol{\qedfull}

% Reset equation counters for each property!
\makeatletter
\@addtoreset{equation}{property}
\makeatother

% Tikz stuff
\newcommand{\seqarr}
{\begin{tikzpicture}
		\draw[-{Triangle[scale=.7]}] (0,0) --  (.3,0); 
\end{tikzpicture}}

\newcommand{\looparr}
{\begin{tikzpicture}[scale=0.7,baseline=-1.55ex]
		\draw[arrows = {-Stealth[inset=0pt, length=2pt, angle'=60]}] (0,0) arc (102:437:.2cm);
\end{tikzpicture}}

\definecolor{blue-violet}{rgb}{0.54, 0.17, 0.89}
\definecolor{cadmiumorange}{rgb}{0.93, 0.53, 0.18}
\definecolor{yellow-green}{rgb}{0.6, 0.8, 0.2}
\definecolor{green1}{rgb}{0.12, 0.3, 0.17}
\definecolor{byzantium}{rgb}{0.44, 0.16, 0.39}


\section{Results}
\section{Results}
\label{results}

\begin{figure*}[ht]
    \centering
    \includegraphics[scale=0.15,trim={0 2.5cm 0 5cm},clip]{images/aoi-single_burst}
    \caption{The time average peak Age of Information with burst and \gls{soa} loss values against the dynamic reliability logic for different network topologies.}
    \label{fig:aoi_burst}\vspace{-0.4cm}
\end{figure*}


This paper focuses on both transport layer and application layer metrics to determine the feasibility of dynamic reliability. For this, we have selected the session packet volume, as transmitted, retransmitted, lost and backlogged packets as \glspl{kpi} for the transport layer; while focusing on the \gls{aoi} for the application layer. The \gls{aoi} was chosen as a crucial indicator for the freshness of packets in real-time applications. More specifically, this work adopts the time average peak \gls{aoi} equation \cite{aoi_equation} depicted in Eq. \ref{aoi}, where $\Delta(r_{i+1})$ is the $i$th update at the time it was received at the server, for a session time period of $\tau$.

\begin{equation}
    \label{aoi}
    \gls{aoi}_\tau = \frac{1}{n-1}\sum_{i=1}^{n-1} \Delta(r_{i+1})
\end{equation}

We include a comparison between the vanilla QUIC implementation which does not enjoy the dynamic reliability extension, with a number of dynamic reliability policies. The tests were run a number of times for statistical significance, with the mean value of vanilla implementation used as a baseline for comparison. The topology utilised both random loss and bursty loss to explore the bounds of dynamic reliability. The \gls{soa} loss in the figures correspond to the loss values presented in Table. \ref{tab:path_char}, for ease of comparison between bursty and random loss scenarios.

\subsection{Transport-Layer KPIs}

To analyse the performance gain at the transport layer due to dynamic reliability, the volume of transmitted and backlogged packets is examined. The figures are in the form of boxplots, which take the vanilla implementation as a benchmark, depicted as the red dashed line.

As seen in Fig. \ref{fig:sent_burst}, the loss plays a crucial role in the performance of the reliability policies. The policies under random loss did incredibly well for the networks with a larger capacity, namely \gls{mmwave} and Sub-6~GHz, whereas for burst loss, the lower network capacities had a larger packet reduction. With the increase in burst loss, the behaviour of the set split reliable policies became unpredictable, if a reliable assignment happened to coincide with a burst loss, the number of transmitted packets increases, and vice versa. On the other hand, in smarter policies, such as Loss-Aware, the performance lightly matched the vanilla baseline, as the reliable assignment dominated the session to compensate for a higher burst loss. Not only that but, the burst loss also impacted the variance of the transmitted packets for the policies.

Unsurprisingly, the unreliable focused policy, 80-20 split, outperformed other policies for all topologies in random and bursty loss scenarios, with an approximate reduction of 80\%. That being said, the majority of the policies reduced the transmitted packets on the link by approximately 70\% for random loss, while the reduction started at $\approx 15\%$ and decreased as the loss increased for the burst loss scenario.

The retransmitted and lost packets, not shown due to space limitations, followed the same trend as the transmitted packets for the random loss scenarios. However, for the burst loss scenarios, the larger capacity networks had a lower reduction in the retransmitted and lost packets. This can be seen as a favorable outcome since the lower capacity networks are scarce on resources. It is important to note that the Loss-Aware policy mimicked the vanilla approach as the burst loss increased, signifying the overwhelming appointment of reliable packets in adapting to the harsh burst loss conditions.
 
Alternatively, Fig. \ref{fig:backlog_burst} clearly shows a stark comparison between the policies and loss scenario in the reduction of the backlogged packets. The Loss-Aware policy for random loss scenario reduced the backlogged packets by up to 50\%, beating all other policies by approximately 30\%. Furthermore, it is clear that the unreliability focused policies resulted in the lowest backlog for the session. In comparison, we notice that the burst loss and the backlogged frequency have a positive correlation, where the maximum reduction of the backlogged packets for the policies is at most 20\%. Much like the transmitted packets, the probability of a burst loss occurrence plays a vital role in the number of retransmissions sent and by extension the number of backlogged packets. Thus, we can conclude that the stress placed on the buffer is a result of the reliable packets which is tightly coupled with the congestion on the session. Whereas, unreliable focused policies did not encounter such a phenomenon regardless if it was experiencing a burst loss.


\subsection{Application-Layer KPIs}

The feasibility of dynamic reliability for real-time applications can be determined by the \gls{aoi}, with comparison across different topologies and policies. If we take a strict approach and consider anything below $10$~ms is real-time \cite{real-time}, then all the reliability policies passed that requirement, which is attractive for real-time applications, as shown in Fig. \ref{fig:aoi_burst}. Utilising the median as an estimate of the runs, the policies in the WLAN and Sub-6~GHz topology with random loss floated around $4-5$~ms with negligible difference, while the \gls{aoi} for \gls{mmwave} was $\approx 2-3$~ms. It is clear that the \gls{aoi} and the network capacity have a negative correlation, as the network capacity decreases, the \gls{aoi} increases. The same correlation is extended to the bursty loss scenarios, where \gls{mmwave} dominated the other topologies. That being said, it is crucial to note that the \gls{aoi} for the reliability policies is often slightly better than or equal to the \gls{aoi} of the vanilla implementation, proving that dynamic reliability reduces the congestion of the session at no cost to the \gls{aoi}.



% \section{Improving \sys}
% \begin{table}[!t]\begin{center}
\caption{\textbf{Analysis of offset mechanisms in 360Attention and backbone variants} on 360BEV-Matterport dataset.}
\vskip -1ex
\label{tab:analysis}
\setlength{\tabcolsep}{1mm}
\renewcommand{\arraystretch}{1.2}
\resizebox{\columnwidth}{!}{
    \begin{tabular}{ l l | c | c | l}
    \toprule[1pt]
    \textbf{Methods} & \textbf{Backbone} & \textbf{\#Param} & \textbf{FLOPs} & \textbf{mIoU} \\ \midrule\midrule
    
    \circled{1} Ours (360Attention offset) & MiT-B0 & 04.60M  & 248.57G & 36.98     \\
    \circled{2} Ours (360Attention offset) & MiT-B2 & 26.30M & 283.94G & 44.32 \\ 
    \circled{3} Ours (360Attention offset) & MiT-B4  & 62.91M & 341.34G &  \textbf{45.53}    \\  \midrule
    \circled{4} Ours (Multi-scale offset) & MiT-B2  & 26.43M  &284.17G &43.65~\obf{-0.67}   \\
    \circled{5} Ours (Fixed-range offset) & MiT-B2  & 26.30M & 283.44G &  43.28~\obf{-1.04}\\
    \circled{6} Ours (Separate offset) & MiT-B2 & 26.19M & 279.18G &  42.82~\obf{-1.50}\\\midrule
    \circled{7} Ours (360Attention offset) & MSCA-B  & 27.69M &274.59G & \textbf{46.31}~\gbf{+1.99} \\ 

    \bottomrule
    \end{tabular}
}
\end{center}
\vskip -4ex
\end{table}

\section{Conclusion}
\section{Conclusion}\label{sec:conclusion}
In this work, we focus on addressing the fundamental challenge of OOD detection tasks, which is how to fully understand the semantic discrepancy between the ID/OOD samples. We reveal that the key to success in the realistic SCOOD task is to allocate as many ID samples in the unlabeled set correctly as possible. To this end, we propose a novel uncertainty-aware optimal transport scheme that introduces class-specific energy scores as guidance for effective label assignment. Experimental results show that our method achieves better performance than previous state-of-the-art methods on SCOOD benchmarks.

\textbf{Limitations.} In addition to temperature scaling, other techniques such as feature clipping applied in ReAct~\cite{sun2021react} also enhance the performance of energy score, so how to obtain an OOD score that best fits the SCOOD task can be further explored. Moreover, a setting highly related to SCOOD has been proposed in \cite{katz2022training} and formulated as a constrained optimization problem. We will also theoretically analyze these practical OOD settings in our feature work.

% \section*{Acknowledgments}
\textbf{Acknowledgments.} 
This work is supported by National Key R\&D Program of China under Grant 2020AAA0105701, National Natural Science Foundation of China (NSFC) under Grants 61872327, Major Special Science and Technology Project of Anhui, National Natural Science Foundation of China (62033012) and Ant Group through Ant Research Intern Program.



% \section*{Ethics Statement}

% \section*{Acknowledgements}
% To thanks Ofir, H2Lab and ZLab memebers (Tim, Ari, Sewon, Akari) 

% Entries for the entire Anthology, followed by custom entries
\bibliography{anthology,acl2023}
\bibliographystyle{acl_natbib}

\appendix

\section{Appendix}
\label{sec:appendix}
\section{Appendix for Proofs}

\paragraph{Proof of Theorem \ref{thm:main}.}

\begin{proof}
\label{proof:main}
Our proof has two steps. In Step 1, we will show that SimCLR is equivalent to minimizing the cross entropy loss defined in Eqn.~(\ref{eqn:cross-entropy}). 
In Step 2, we will show  that minimizing the cross-entropy loss 
is equivalent to spectral clustering on $\bfpi$. 
Combining the two steps together, we have proved our theorem. 

\textbf{Step 1: } SimCLR is equivalent to minimizing the cross entropy loss.

The cross-entropy loss takes expectation over 
$\bfW_\bfX\sim \mathbb{P}(\cdot ; \bfpi)$, 
which means $\bfW_\bfX$ has exactly one non-zero entry in each row $i$. By Lemma~\ref{lem:multinomial}, we know every row $i$ of $\bfW_\bfX$ is independent of other rows. Moreover, 
$\bfW_{\bfX,i}\sim \mathcal{M}(1, \bfpi_i/\sum_j \bfpi_{i,j})=\mathcal{M}(1, \bfpi_i)$, because $\bfpi_i$ itself is a probability distribution.
Similarly, we know $\bfW_\bfZ$ also has the row-independent property by sampling over $\mathbb{P}(\cdot;\bfK_\bfZ)$.
Therefore, by Lemma~\ref{lem:cross_split}, we know Eqn.~(\ref{eqn:cross-entropy}) is equivalent to:
\[
 -\sum_{i=1}^n \mathbb{E}_{\bfW_{\bfX,i}}[\log \mathbb{P}(\bfW_{\bfZ,i}=\bfW_{\bfX,i};\bfK_\bfZ)],
\]

This expression takes expectation over $\bfW_{\bfX,i}$ for the given row $i$. Notice that 
$\bfW_{\bfX,i}$ has exactly one non-zero entry, which equals $1$ (same for $\bfW_{\bfZ,i}$). 
As a result
we expand the above expression to be:
\begin{equation}
 -\sum_{i=1}^n \sum_{j\neq i} \Pr(\bfW_{\bfX,i,j}=1)\log \Pr(\bfW_{\bfZ,i,j}=1).
\label{eqn:detailed-expansion}    
\end{equation}


By Lemma~\ref{lem:multinomial}, $\Pr(\bfW_{\bfZ,i,j}=1)=\bfK_{\bfZ,i,j}/\|\bfK_{\bfZ,i}\|_1$ for $j\neq i$. Recall that $\bfK_\bfZ=(k(\bfZ_i-\bfZ_j))_{(i,j)\in[n]^2}$, which means 
$\bfK_{\bfZ,i,j}/\|\bfK_{\bfZ,i}\|_1=\frac{\exp(-\|\bfZ_i-\bfZ_j\|^2/{2\tau})}{\sum_{k\neq i}
\exp(-\|\bfZ_i-\bfZ_k\|^2/{2\tau})
}$ for $j\neq i$, when $k$ is the Gaussian kernel with variance $\tau$. 

Notice that $\bfZ_i=f(\bfX_i)$, so we know
\begin{equation}
-\log \Pr(\bfW_{\bfZ,i,j}=1)=
-\log \frac{\exp(-\|f(\bfX_i)-f(\bfX_j)\|^2/{2\tau})}{\sum_{k\neq i}
\exp(-\|f(\bfX_i)-f(\bfX_k)\|^2/{2\tau}),
}
\label{eqn:infonce-equivalence}    
\end{equation}


The right hand side is exactly the InfoNCE loss defined in Eqn.~(\ref{eqn:infonce}).
Inserting Eqn.~(\ref{eqn:infonce-equivalence}) into Eqn.~(\ref{eqn:detailed-expansion}), we get the SimCLR algorithm, which first samples augmentation pairs $(i,j)$ with $\Pr(\bfW_{\bfX,i,j}=1)$ for each row $i$, and then optimize the InfoNCE loss. 

\textbf{Step 2: } minimizing the cross entropy loss 
is equivalent to spectral clustering on $\bfpi$.


By Lemma~\ref{lem:convert_to_spectral}, we may further convert the loss to 
\begin{equation}
\label{eqn:main-theorem-repul-attr}
\min_{\bfZ}
-\sum_{(i,j)\in [n]^2} \mathbf{P}_{i,j}
\log k (\bfZ_i-\bfZ_j)+\log \mathbf{R}(\bfZ).
\end{equation}
Since $k$ is the Gaussian kernel, this reduces to \[
\min_\bfZ \mathrm{tr}(\bfZ^\top \mathbf{L}(\bfpi) \bfZ)
+\log \mathbf{R}(\bfZ),
\]

where we use the fact that $\mathbb{E}_{\bfW_\bfX\sim \mathbb{P}(\cdot; \bfpi)}[\mathbf{L}(\bfW_\bfX)]
=\mathbf{L}(\bfpi)
$, because the Laplacian operator is linear and $
\mathbb{E}_{\bfW_\bfX\sim \mathbb{P}(\cdot; \bfpi)}(\bfW_\bfX)=\bfpi
$.
\end{proof}

\paragraph{Proof of Theorem \ref{thm:clip}.}
\begin{proof}
Since $\bfW_\bfX\sim \mathbb{P}(\cdot;\bfpi_{\mathbf{A}, \mathbf{B}})$, we know 
$\bfW_\bfX$ has exactly one non-zero entry in each row, denoting the pair that got sampled. 
A notable difference compared to the previous proof is we now have $n_\mathcal{A}+n_\mathcal{B}$ objects in our graph. CLIP deals with this by taking a mini-batch of size $2N$, 
such that $n_\mathcal{A}=n_\mathcal{B}=N$, and adding the $2N$ InfoNCE losses together. We label the objects in $\mathcal{A}$ as $[n_\mathcal{A}]$, and the objects in $\mathcal{B}$ as $\{n_\mathcal{A}+1, \cdots, n_\mathcal{A}+n_\mathcal{B}\}$. 

Notice that $\bfpi_{\mathbf{A}, \mathbf{B}}$ is a bipartite graph, so the edges of objects in $\mathcal{A}$ will only connect to object in $\mathcal{B}$ and vice versa. We can define the similarity matrix in $\cZ$ as $\bfK_\bfZ$, 
where $\bfK_\bfZ(i, j+n_\mathcal{A})=\bfK_\bfZ(j+n_\mathcal{A},i)= k(\bfZ_i-\bfZ_j)$ for $i\in [n_\mathcal{A}], j\in [n_\mathcal{B}]$, and otherwise we set $\bfK_\bfZ(i,j)=0$. 
The rest is same as the previous proof. 
\end{proof}

\paragraph{Proof of Theorem \ref{thm:exponential}.}

\begin{proof}
\label{proof:exponential}
Since the objective function consists of a linear term combined with an entropy regularization, which is a strongly concave function, the maximization problem is a convex optimization problem. Owing to the implicit constraints provided by the entropy function, the problem is equivalent to having only the equality constraint. We then introduce the Lagrangian multiplier $\lambda$ and obtain the following relaxed problem:

$$
\widetilde{E}(\boldsymbol{\alpha})=\psi_{1}-\sum_{i=1}^n \alpha_{i} \psi_{i}+\tau \sum_{i=1}^n \alpha_{i}\log \alpha_{i}+\lambda\left(\boldsymbol{\alpha}^{\top} \mathbf{1}_n-1\right).
$$

As the relaxed problem is unconstrained, taking the derivative with respect to $\alpha_{i}$ yields

$$
\frac{\partial \widetilde{E}(\boldsymbol{\alpha})}{\partial \alpha_{i}}=-\psi_{i}+\tau\left(\log \alpha_{i}+\alpha_{i} \frac{1}{\alpha_{i}}\right)+\lambda=0.
$$

Solving the above equation implies that $\alpha_{i}$ takes the form
$
\alpha_{i}=\exp \left(\frac{1}{\tau} \psi_{i}\right) \exp \left(\frac{-\lambda}{\tau}-1\right).
$ Since $\alpha_{i}$ lies on the probability simplex, the optimal $\alpha_{i}$ is explicitly given by
$
\alpha^{*}_{i}=\frac{\exp \left(\frac{1}{\tau} \psi_{i}\right)}{\sum_{i^{\prime}=1}^n \exp \left(\frac{1}{\tau} \psi_{i^{\prime}}\right)} .
$ Substituting the optimal point into the objective function, we obtain
$$
\begin{aligned}
E\left(\boldsymbol{\alpha}^*\right)  &=\psi_1-\sum_{i=1}^n \frac{\exp \left(\frac{1}{\tau} \psi_{i}\right)}{\sum_{i^{\prime}=1}^n \exp \left(\frac{1}{\tau} \psi_{i^{\prime}}\right)} \psi_{i}+\tau \sum_{i=1}^n \frac{\exp \left(\frac{1}{\tau} \psi_{i}\right)}{\sum_{i^{\prime}=1}^n \exp \left(\frac{1}{\tau} \psi_{i^{\prime}}\right)}\log \frac{\exp \left(\frac{1}{\tau} \psi_{i}\right)}{\sum_{i^{\prime}=1}^n \exp \left(\frac{1}{\tau} \psi_{i^{\prime}}\right)} \\
& =\psi_1 - \tau \log \left(\sum_{i=1}^n \exp \left(\frac{1}{\tau} \psi_{i}\right)\right).
\end{aligned}
$$
Thus, the Lagrangian dual function is given by
\begin{equation*}
-E\left(\boldsymbol{\alpha}^*\right)= -\tau \log \frac{\exp \left(\frac{1}{\tau} \psi_{1}\right)}{\sum_{i=1}^n \exp \left(\frac{1}{\tau} \psi_{i}\right)}.\qedhere
\end{equation*}
\end{proof}



\section{More on Experiments} \label{section: experiment_details}

\paragraph{CIFAR-10 and CIFAR-100} CIFAR-10 ~\citep{krizhevsky2009learning} and CIFAR-100 ~\citep{krizhevsky2009learning} are well-known classic image classification datasets. Both CIFAR-10 and CIFAR-100 contain a total of 60k $32 \times 32$ labeled images of different classes, with 50k for training and 10k for testing. CIFAR-10 is similar to CIFAR-100, except there are 10 different classes in CIFAR-10 and 100 classes in CIFAR-100.

\paragraph{TinyImageNet} TinyImageNet ~\citep{le2015tiny} is a subset of ImageNet ~\citep{deng2009imagenet}. There are 200 different object classes in TinyImageNet, with 500 training images, 50 validation images, and 50 test images for each class. All the images in TinyImageNet are colored and labeled with a size of $64 \times 64$.

\textbf{Pseudo-code.} Algorithm \ref{alg:Training Procedure} presents the pseudo-code for our empirical training procedure.

\begin{algorithm}[!htbp]
\caption{Training Procedure}
\label{alg:Training Procedure}
\begin{algorithmic}[1]
\REQUIRE trainable encoder network $f$, batch size $N$, augmentation strategy \textit{aug}, loss function $L$ with hyperparameters \textit{args}
\FOR {sampled minibatch ${x_i}_{i=1}^N$}
\FORALL{$i \in { 1, ..., N }$}
\STATE draw two augmentations $t_i = \textit{aug}\left(x_i\right) $, $t_i' = \textit{aug}\left(x_i\right) $
\STATE $z_i = f\left(t_i\right)$, $z_i' = f\left(t_i'\right)$
\ENDFOR
\STATE compute loss $\mathcal{L} = L(N, z, z', \textit{args})$
\STATE update encoder network $f$ to minimize $\mathcal{L}$
\ENDFOR
\STATE \textbf{Return} encoder network $f$
\end{algorithmic}
\end{algorithm}

We also provide the pseudo-code for our core loss function used in the training procedure in Algorithm \ref{alg:Core loss}. The pseudo-code is almost identical to SimCLR's loss function, with the exception of an extra parameter $\gamma$.

\begin{algorithm}[!htbp]
\caption{Core loss function $\mathcal{C}$}
\label{alg:Core loss}
\begin{algorithmic}[1]
\REQUIRE batch size $N$, two encoded minibatches $z_1, z_2$, $\gamma$, temperature $\tau$
\STATE $z = \textit{concat}\left(z_1, z_2\right)$
\FOR {$i \in {1, ..., 2N }, j \in {1, ..., 2N}$ }
\STATE $s_{i,j} = \Vert z_i - z_j \Vert_2^{\gamma}$
\ENDFOR
\STATE \textbf{define} $l(i, j)$ \textbf{as} $l(i, j) = - \log \frac{exp\left(s_{i,j}/\tau \right)}{\sum_{k=1}^{2N} \mathbf{1}{[k \ne i]} exp\left(s{i, j} / \tau \right)} $
\STATE \textbf{Return} $\frac{1}{2N} \sum_{k=1}^N\left[l(i, i+N) + l(i+N, i)\right]$
\end{algorithmic}
\end{algorithm}

Utilizing the core loss function $\mathcal{C}$, we can define all kernel loss functions used in our experiments in Table \ref{table: loss definition}. For all $z_i \in z$ with even dimensions $n$, we define $z_{L_i} = z_i\left[0:n/2\right]$ and $z_{R_i} = z_i\left[n/2:n\right]$.

\begin{table}[ht]
\centering
\begin{tabular}{{@{}l|l@{}}}
Kernel  &  Loss function \\ \midrule
Laplacian & $\mathcal{C}\left(N, z, z', \gamma=1, \tau\right)$\\ \midrule
Sum       & $\lambda * \mathcal{C}\left(N, z, z', \gamma=1, \tau_1\right) + (1-\lambda) * \mathcal{C}\left(N, z, z', \gamma=2, \tau_2\right)$  \\ \midrule
Concatenation Sum&$\lambda * \mathcal{C}\left(N, z_L, z'_L, \gamma=1, \tau_1\right) + (1-\lambda) * \mathcal{C}\left(N, z_R, z'_R, \gamma=2, \tau_2\right)$\\ \midrule
$\gamma = 0.5$ & $\mathcal{C}\left(N, z, z', \gamma=0.5, \tau\right)$          \\ 

\end{tabular}

\caption{Definition of kernel loss functions in our experiments}
\label {table: loss definition}
\end{table}

\textbf{Baselines.} We reproduce the SimCLR algorithm using PyTorch Lightning~\citep{PytorchLightning}.

\textbf{Encoder details.}
The encoder $f$ consists of a backbone network and a projection network. We employ ResNet50~\citep{ResNet} as the backbone and a 2-layer MLP (connected by a batch normalization~\citep{ioffe2015batch} layer and a ReLU \cite{nair2010rectified} layer) with hidden dimensions 2048 and output dimensions 128 (or 256 in the concatenation kernel case).

\textbf{Encoder hyperparameter tuning.}
For each encoder training case, we randomly sample 500 hyperparameter groups (sample details are shown in Table \ref{table: Hyperparameter sample}) and train these samples simultaneously using Ray Tune ~\citep{RayTune}, with the ASHA scheduler~\citep{li2018massively}. Ultimately, the hyperparameter group that maximizes the online validation accuracy (integrated in PyTorch Lightning) within 5000 validation steps is chosen for the given encoder training case.

\begin{table}[ht]
\centering

\begin{tabular}{@{}l|l|l@{}}
\midrule
Hyperparameter  & Sample Range & Sample Strategy \\ \midrule
start learning rate & $\left[10^{-2}, 10\right]$ & log uniform \\ \midrule
$\lambda$       & $\left[0, 1\right]$ & uniform \\ \midrule
$\tau$, $\tau_1$, $\tau_2$ & $\left[0, 1\right]$ & log uniform \\ \midrule
\end{tabular}

\caption{Hyperparameters sample strategy}
\label {table: Hyperparameter sample}
\end{table}

\textbf{Encoder training.} 
We train each encoder using the LARS optimizer~\citep{LARSOptimizer}, LambdaLR Scheduler in PyTorch, momentum 0.9, weight decay $10^{-6}$, batch size 256, and the aforementioned hyperparameters for 400 epochs on a single A-100 GPU.

\textbf{Image transformation.} The image transformation strategy, including augmentation, is identical to the default transformation strategy provided by PyTorch Lightning.

\textbf{Linear evaluation.}
The linear head is trained using the SGD optimizer with a cosine learning rate scheduler, batch size 64, and weight decay $10^{-6}$ for 100 epochs. The learning rate starts at $0.3$ and ends at $0$.

\textbf{Moco Experiments.} We also tested our method based on MoCo~\citep{he2019moco}. The results are summarized in Table \ref{tab:results-moco}. Here we choose ResNet18~\citep{ResNet} as the backbone and set a temperature of $0.1$ as default. For our simple sum kernel, we set $\lambda=0.8$. The results show that our method outperforms the original MoCo method.

\begin{table}[thb]
\centering
\caption{MoCo Experiment Results on CIFAR-10 and CIFAR-100.}
\label{tab:results-moco}
\resizebox{\textwidth}{!}{%
\begin{tabular}{@{}c|ccc|ccc@{}}
\toprule
\multirow{3}{*}{Method} & \multicolumn{3}{c|}{CIFAR-10} & \multicolumn{3}{c}{CIFAR-100} \\ \cmidrule(lr){2-4} \cmidrule(lr){5-7} 
                        & 200 epochs & 400 epochs    & 1000 epochs   & 200 epochs & 400 epochs & 1000 epochs         \\ \midrule
MoCo (repro.)         & $76.41 \pm 0.12$    & $80.01 \pm 0.15$          & $84.45 \pm 0.08$    & $\mathbf{47.02 \pm 0.11}$ & $52.50 \pm 0.07$ & $57.62 \pm 0.15$            \\
\midrule
Laplacian Kernel        & ${78.09 \pm 0.10}$    & $\mathbf{83.85 \pm 0.09}$          & $\mathbf{88.34 \pm 0.16}$    & $46.12 \pm 0.22$   & $53.44 \pm 0.17$ & $59.10 \pm 0.14$        \\
Simple Sum Kernel & $\mathbf{78.12 \pm 0.15}$   & $83.23 \pm 0.18$ & $87.50 \pm 0.20$ & $46.65 \pm 0.06$ & $\mathbf{53.62 \pm 0.19}$ & $\mathbf{59.83 \pm 0.12}$\\
\bottomrule
\end{tabular}
}
\end{table}



\section{More Experiments on Synthetic Data}


Consider a scenario with $n$ clusters, each containing $k$ vertices. Let the probability of vertices $u$ and $v$ from the same cluster belonging to $\bfpi$ be $p$. Conversely, for vertices $u$ and $v$ from different clusters, let the probability of belonging to $\pi$ be $q$. We generate the graph $\bfpi$ randomly, based on $p$ and $q$. We experiment with values of $k=100$ and $n=6$ for ease of visualization, embedding all points in a two-dimensional space. Each vertex's initial position originates from a normal distribution. In each iteration, we sample a subgraph of $\bfpi$ uniformly, ensuring each vertex has an out-degree of $1$. We then optimize the corresponding vectors using InfoNCE loss with an SGD optimizer and iterate until convergence. Our experimental setup consists of an SGD learning rate of $1$, an InfoNCE loss temperature of $0.5$, and a batch size of $50$. We evaluate two scenarios with different $p$ and $q$ values: $p=1$, $q=0$, and $p=0.75$, $q=0.2$. The results of these experiments are visualized in Figure \ref{fig:vis-spectral-cluster}. The obtained embeddings exhibit the hallmark pattern of spectral clustering of graph $\bfpi$.

\begin{figure}[!tb]
\centering
\subfigure{
\includegraphics[width=1\textwidth]{Figures/cluster_pi.png}
\label{fig:vis-cluster}
}
\subfigure{
\includegraphics[width=1\textwidth]{Figures/noised_cluster_pi.png}
\label{fig:vis-noised-cluster}
}
\caption{Visualizations of the optimization process using InfoNCE Loss on the vectors corresponding to $\bfpi$. Points of identical color belong to the same cluster within $\bfpi$. To showcase the internal structure of $\bfpi$, we randomly select 10 vertices from each cluster to display the edge distribution of $\bfpi$.}
\label{fig:vis-spectral-cluster}
\end{figure}



\end{document}
