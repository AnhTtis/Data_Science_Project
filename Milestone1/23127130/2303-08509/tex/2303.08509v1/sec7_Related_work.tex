\section{Related Work}
\label{sec:Related}
% {Existing adversarial attacks against Android malware detection include syntax feature oriented attacks and semantic feature oriented attacks.} 

% The former take syntax feature (e.g., permission, intent action and API calls) based classifier as their attack target, while the latter takes semantic features (e.g., function call graph) based classifiers as their attack target.

%\textbf{Syntax feature oriented AE attacks.}
{ Recently, adversarial attacks have been widely used in various fields, i.e., image classification \cite{ijcai2022p554,xia2022enhancing}, traffic analysis \cite{DBLP:conf/uss/NasrBH21,DBLP:journals/tifs/RahmanIMW21}, autonomous driving \cite{DBLP:conf/uss/Jing0D0L0NW21,DBLP:journals/corr/abs-2201-06192} and object detection \cite{DBLP:conf/uss/LovisottoTSSM21}. As for Android malware detection,}
there have been many studies \cite{huang2018adversarial,grosse2017adversarial,hu2017generating,2019lh,DBLP:journals/tifs/LiL20} on syntax features oriented AE generation. Huang \emph{et al.} \cite{huang2018adversarial}  use the saddle-point optimization formulation to generate adversarial examples in the discrete (e.g., binary) domain for malware detection. Grosse  \emph{et al.} \cite{grosse2017adversarial}  expand existing AE generation algorithms to construct a highly effective attack against malware detection models. In \cite{hu2017generating,2019lh}, Hu \emph{et al.} utilize a GAN to generate adversarial examples in black-box mode for malware detection. Li  \emph{et al.} \cite{ DBLP:journals/tifs/LiL20} propose { an ensemble approach that allows} attackers to perturb a malware example via multiple attack methods and multiple manipulation sets.

%In practice, the syntax feature based detection methods are susceptible to evasion and may 
%However, these works are only applicable to the Android malware detectors that adopt syntactic features and are difficult to be applied to semantic features-based Android malware detection methods.
%\textbf{Semantic feature oriented AE attacks.}
To achieve higher detection accuracy, more and more {Android malware detection methods} \cite{DBLP:conf/ccs/ZhangZZDCZZY20,DBLP:conf/ndss/MaricontiOACRS17,DBLP:conf/kbse/WuLZYZ019} focus on semantic features. Chen  \emph{et al.} \cite{chen2020android} introduce two AE generations methods in image classification to detect Android malware, and propose a method applying optimal perturbations onto Android APKs. Their method directly perturbs features in feature space. Pierazzi  \emph{et al.} \cite{DBLP:conf/sp/PierazziPCC20}
extract slices of bytecode (i.e., gadgets) from benign APKs and inject them into a malicious APK to generate adversarial malware. { Zhang \emph{et al.} \cite{DBLP:journals/corr/abs-2009-05602} propose a reinforcement learning based attack to deceive graph feature based malware detection models.  Recently, Bostani \emph{et al.} \cite{DBLP:journals/corr/abs-2110-03301} propose an interesting black-box attack EvadeDroid without requiring the knowledge about feature space. Different from {\ourtool}, EvadeDroid employs random search to find the desired perturbation from the code of benign apps.}



% In summary, the study of semantic feature oriented AE attacks is just beginning. {The open problems remain yet to solve include (but are not limited to) 1) how to manipulate malware codes with functionality preservation and 2) how to find desired adversarial perturbation with limited information}. 




