


\documentclass[letterpaper,twocolumn,10pt]{article}

\usepackage[table]{xcolor}
\usepackage{float}
\usepackage{usenix2019_v3}
\usepackage{fancyhdr}
\usepackage{graphicx}
\usepackage{subfigure}
\usepackage{multirow}
\usepackage{amsmath}
% \usepackage{algorithmic}
\usepackage[ruled,vlined]{algorithm2e}
\usepackage{bm}
\usepackage{amsfonts}
\usepackage{url}
\usepackage{amsmath}
\usepackage{soul}
\usepackage{comment}
\usepackage{tabularx}
\usepackage{array}
%\usepackage{longtable}
\usepackage{rotating}
\usepackage{booktabs}
\usepackage{overpic}
% \setlength{\dbltextfloatsep}{4pt}
\usepackage{float}
\usepackage{setspace}
% \def\bibfont{ \small}
% \def\bibfont{\fontsize{4.5}{10}\selectfont}


	\newcommand{\ourtool}{BagAmmo}
%	\newcommand{\ourevolution}{MPCE}
	\newcommand{\ouralgorithm}{Apoem}
	% \title{Black-box Adversarial Example Attack towards {FCG} Based Android Malware Detection under Incomplete Feature Information}


\begin{document}
\date{}

\title{\Large \bf Black-box Adversarial Example Attack towards {FCG} Based Android Malware \\ Detection under Incomplete Feature Information}




\author{Heng Li$^1$,
	 Zhang Cheng$^1,3$,
	 Bang Wu$^1$,
	 Liheng Yuan$^1$,
	 Cuiying Gao$^1$,
	 Wei Yuan$^1$,
	 Xiapu Luo$^2$ \\
	 $^1$ Huazhong University of Science and Technology\\
	 $^2$ The Hong Kong Polytechnic University\\
	 $^3$ NSFOCUS Technologies Group Co., Ltd\\
 }


\maketitle

\begin{abstract}
The {function call graph (FCG) } based Android malware detection methods have recently attracted increasing attention due to their promising performance. However, these methods are susceptible to adversarial examples (AEs). In this paper, we design a novel \textit{black-box} AE attack towards the {FCG} based malware detection system, called {\ourtool}. {To mislead its target system, {\ourtool} purposefully perturbs the FCG feature of malware through inserting "never-executed" function calls into malware code. The main challenges are two-fold. First, the malware functionality should not be changed by adversarial perturbation. Second, the information of the target system (e.g., the graph feature granularity and the output probabilities) is absent.}

{To preserve malware functionality, {\ourtool} employs the \textit{try-catch trap} to insert function calls to perturb the FCG of malware. Without the knowledge about feature granularity and output probabilities, {\ourtool} adopts the architecture of generative adversarial network (GAN), and leverages a multi-population co-evolution algorithm (i.e., {\ouralgorithm}) to generate the desired perturbation. Every population in {\ouralgorithm} represents a possible feature granularity, and the real feature granularity can be achieved when {\ouralgorithm} converges. }
	
{Through extensive experiments on over 44k Android apps and 32 target models, we evaluate the effectiveness, efficiency and resilience of {\ourtool}.  {\ourtool} achieves an average attack success rate of over 99.9\% on MaMaDroid, APIGraph and GCN, and still performs well in the scenario of concept drift and data imbalance.} {Moreover, {\ourtool} outperforms the state-of-the-art attack SRL in attack success rate.} 
	
	\end{abstract}
	
	\section{Introduction}


Occupying about $85\%$ of the global mobile operating system market, Android has become the main target of mobile malware in the world. A recent security report shows that on average, about 10000 new mobile malware samples were captured per day \cite{gdata}. The rapidly increasing of malware poses severe threats to Android users \cite{DBLP:conf/uss/SunSLM21,DBLP:conf/icssa/KimL18,DBLP:journals/tmc/LiuLZWZ20}, e.g., privacy leakage and economic losses. To tackle this problem, a variety of machine learning based Android malware detection methods have been designed to identify malware based on their features \cite{DBLP:conf/ccs/ZhangZZDCZZY20,DBLP:conf/ndss/MaricontiOACRS17,DBLP:conf/kbse/WuLZYZ019,DBLP:journals/tsmc/YuanJLC21,DBLP:conf/ndss/ArpSHGR14,DBLP:journals/tii/LiSYLSY18,DBLP:conf/kdd/HouYSA17,DBLP:journals/tifs/WangWFLHZ14,DBLP:conf/ccs/ZhangDYZ14,DBLP:conf/icccn/HuTMZZH14}. As a common feature for Android malware detection, Function Call Graph (FCG)  \cite{DBLP:conf/ccs/ZhangZZDCZZY20,DBLP:conf/ndss/MaricontiOACRS17,DBLP:conf/kbse/WuLZYZ019,DBLP:conf/kdd/HouYSA17,DBLP:journals/tifs/WangWFLHZ14,DBLP:conf/ccs/ZhangDYZ14,DBLP:conf/icccn/HuTMZZH14} (e.g., frequent subgraph \cite{DBLP:journals/tifs/Fan0LCTZL18} and E-FCG \cite{DBLP:journals/ijon/CaiJGLY21}) provides important clues for understanding how Android apps work. {In an FCG}, every node represents a function or an \textit{abstracted} function (e.g., class, package or family), and every edge denotes the calling relationship between caller and callee. 
\begin{figure}[bhtp]   
	\centering   
	\includegraphics[width=1\linewidth]{framework.pdf}
	\caption{{FCG based Android malware detection framework.} }  
	\label{fig:motivation}   
\end{figure} 
As depicted in Fig. \ref{fig:motivation}, {the FCG} based Android malware detection usually consists of three steps. First, the {FCG} feature (e.g., frequent subgraph) is extracted from the Android Package (APK) file. Second, the {FCG} is transformed into a feature vector, i.e., graph embedding. Third, the feature vector is processed for malware prediction. Existing studies \cite{DBLP:conf/ccs/ZhangZZDCZZY20,DBLP:conf/ndss/MaricontiOACRS17,DBLP:conf/kbse/WuLZYZ019} demonstrate that the {FCG} based Android malware detection methods can achieve promising performance. 

Unfortunately, the {FCG} based malware detection is susceptible to {adversarial examples (AEs) \cite{DBLP:journals/corr/SzegedyZSBEGF13,DBLP:conf/uss/SuyaC0020,DBLP:conf/sp/Carlini017,DBLP:conf/ijcai/XuC0CWHL19,DBLP:conf/www/SunWTHH20,DBLP:conf/kdd/0001WDWT21}}, which are generated by imposing well-crafted adversarial perturbations on normal examples to induce misclassification. To evade detection, an adversary just needs to manipulate a malicious app  by elaborately modifying (e.g., inserting non-functional function calls) and repackaging its code. Although malware manipulation takes place in problem space (depicted by the {first} box in Fig. \ref{fig:motivation}), it changes the {FCG} (e.g., adding new edges) and perturbs the feature vector in feature space (described by the {second} box in Fig. \ref{fig:motivation}). Once the perturbation helps the feature vector stride over the target classifier's decision boundary, the repackaged malware will evade detection. {Up to now, a variety of AE attacks towards Android malware detection have been proposed to produce evasive Android malware}. Most of them \cite{huang2018adversarial,grosse2017adversarial,hu2017generating,2019lh,DBLP:journals/tifs/LiL20} direct at non-graph features (i.e., syntax features) based detection models that use binary feature vectors  for app classification. % For instance, \cite{hu2017generating,2019lh} apply generative adversarial net (GAN) to modify malware's syntax features and design adversarial examples. 
 {Recently, increasing attention has been paid to} the AE attacks towards graph feature  (i.e., semantic feature) based detection models \cite{DBLP:journals/corr/abs-2110-03301}\cite{chen2020android}. {For example,  Bostani \emph{et al.} \cite{DBLP:journals/corr/abs-2110-03301} leverage random search to find optimal perturbation for APK files in a black-box setting. Chen \emph{et al.} \cite{chen2020android} propose a method to exert optimal perturbations on Android APK files.}
% To the best of our knowledge, only one AE attack method, i.e., \textit{Android HIV} [], has been developed against the graph-based Android malware detection. This method focuses on the scenario where Markov chains constructed from FCGs are used as the feature vectors for classification. It first figures out the adversarial perturbations (e.g., enlarging the transition probability between two states) that help the feature vector deceive its target classifier, and then modifies the FCG (e.g., adding more function calls) to match the perturbed feature vector. Guided by the modified FCG, it manipulates the malware sample to evade detection. As a pioneering method, however, Android HIV has two limitations. First, it requires the knowledge about feature space, including 1) Markov chains are used for classification, and 2) how to obtain Markov chains from FCGs. Second, it also requires the knowledge about problem space, i.e., which kind of FCGs (e.g., frequent subgraph or XXX) is used as the input of target system. Only with this knowledge can Android HIV accurately locate the function calls that needs to be increased or reduced. Unfortunately, these requirements are hard to satisfy in practice, since the process of graph construction and feature vector generation are invisible to adversaries. 

{Up to now, how to produce Android malware to circumvent the FCG based  detection is still an open issue. This motivates us to investigate the generation of AEs to fight against the FCG based Android malware detection. In practice, building evasive malware needs to consider the following realistic problems that have not been well addressed.}

\noindent (1) \textit{Malware functionality preservation}. {The malware manipulation should be able to} mislead its target classifier {in the premise of malware functionality preservation}. 
% The functionality preservation requirement strictly restricts the manipulation on malware. 

\noindent (2) \textit{Problem-feature space gap}. 
{Since the feature vector in feature space cannot be directly perturbed, adversaries have to modify malware code in problem space} and expect their modification brings about the desired adversarial perturbation on feature vector. 

\noindent(3) \textit{Strict black-box setting}. For adversaries, the target classifier is a strict black box and its architecture, parameters and output probabilities are all unknown. 

\noindent(4) \textit{{Feature information absence}}. Adversaries cannot get the {feature used by their target classifier, i.e., the FCG and the feature vector obtained by graph embedding (denoted in the second box of Fig. \ref{fig:motivation}).} {Moreover, a detection system may use one of several possible feature granularities, e.g., class level, package level and family level (as discussed in Subsection \ref{subsec: features}). In practice, the feature granularity information is often unavailable to adversaries. }


% {In realistic applications, an FCG based Android malware detection system may use one of several possible feature granularities, e.g., class level, package level and family level, which will be discussed in Subsection \ref{subsec: features}. For practical consideration, we assume that adversaries have to generate real evasive malware under incomplete feature information. Here incomplete feature information means adversaries does not know the granularity of the FCG feature.} 

{To overcome the above challenges}, we design a \textbf{b}lack-box \textbf{a}ttacks towards {FC\textbf{G}} based \textbf{A}ndroid \textbf{m}alware detection with \textbf{m}ulti-population co-ev\textbf{o}lution, termed {\ourtool}. {{\ourtool} works under the \textit{incomplete feature information} condition, which means adversaries do not know the granularity of the FCG feature used by their target system. }% The manipulation technique is used to modify the codes of malware samples for detection evasion, according to the derived adversarial perturbation. 
Our main tasks include designing a malware manipulation technique used in problem space, and developing an algorithm to derive adversarial perturbation in feature space. {{\ourtool} constructs a dedicated Generative Adversarial Network (GAN) and employs its \textit{generator}} to generate candidate manipulations under the guidance of its \textit{discriminator}. The generator is implemented by our proposed \textbf{A}dversarial multi-\textbf{p}opulation c\textbf{o}-\textbf{e}volution algorith\textbf{m} ({\ouralgorithm}). {\ourtool} iteratively queries its target detection system with manipulated samples, and gradually learns the desired manipulation from a sequence of query-reply pairs. {{\ourtool} uses the following techniques to overcome the above challenges.}

\noindent(1) {\ourtool} leverages a novel malware manipulation method "\textit{try-catch trap}" to insert never-executed function calls into malware code {for functionality preservation}.

\noindent(2) {{\ourtool} maps the FCG into a feature vector, which transfers the impacts of malware manipulation into feature space and hence bridges the problem-feature space gap.}

\noindent(3) To overcome the challenge of strict black box, the discriminator substitutes the target {classifier} and guides the generator to figure out the desirable manipulation rapidly.

\noindent(4) {In {\ouralgorithm}, every population corresponds to a possible feature granularity.} Owning to the cooperative evolution among populations, {\ouralgorithm} converges to the real feature granularity under incomplete feature information.


Our main contributions are summarized as follows.

\noindent$\bullet$ We propose a novel black-box AE attack {\ourtool} towards the {FCG} based Android malware detection. {{\ourtool} does not require  complete information about feature space, and hence it is a broad-spectrum attack with strong generalizability. }

\noindent$\bullet$  We  theoretically analyze why {\ouralgorithm} can mitigate the prematurity problem {that often plagues the evolution algorithms}. 

\noindent$\bullet$ {We conduct extensive experiments on three state-of-the-art (SOTA) malware detection methods MaMaDroid \cite{DBLP:conf/ndss/MaricontiOACRS17},  APIGraph    \cite{DBLP:conf/ccs/ZhangZZDCZZY20} and GCN\cite{DBLP:journals/corr/abs-2009-05602} with five classifiers (e.g., RF and DNN) under three feature granularities. {\ourtool} surpasses 
{the SOTA attack (i.e., reinforcement learning based method SRL) in our experiments.}} It achieves an average attack success rate of over
\textbf{99.9\%} on all \textbf{32} target detection systems. Our experiments also confirm the {\ourtool}'s attack efficiency and resilience to concept drift and data imbalance.
%the SOTA attack SRL in our experiments.}

% \noindent$\bullet$ Based on {\ourtool}, we develop a practical tool\footnote{We publish part of the code on https://github.com/USENIX497/USENIX-497} to {lay a try-catch trap} and produce a real evasive Android malware. This tool unpacks, manipulates and rebuilds APK files in an automated way, making it easy to use in practice.

{\textbf{Roadmap.} The remainder of the paper is organized as follows: §\ref{subsec: Preliminaries} introduces preliminaries; §\ref{subsec: Problem} presents the problem formulation; §\ref{sec:Adversarial manipulation design} discusses how to manipulate  the malware; §\ref{sec:method} describes the algorithm of perturbation generation; §\ref{SEC:EXP} gives the performance evaluation; §\ref{sec:Related} reviews relevant work; §\ref{sec:Limitations} provides {the limitations and discussion.}}
%the  potential limitations; §\ref{sec:Conclusion} concludes our work.}

	\section{Preliminaries} \label{subsec: Preliminaries}
% In this section, we introduce the necessary knowledge about the features used in Android malware detection and the graph based Android malware detection methods. 
%用两个mapping

\subsection{Features for Android malware detection}\label{subsec: features}
% Existing Android malware detection systems classify apps based on dynamic or static features of apps. Extracting dynamic features requires monitoring the execution of apps, inevitably incurring significant overheads. Different from dynamic features, static features can be obtained prior to app execution, and hence have been widely used in Android malware detection. 

 {In this subsection, we focus on the static features that are obtained prior to app execution and widely used in Android malware detection.} Earlier studies devote more attention to syntax features, e.g., requested permissions \cite{DBLP:journals/tii/LiSYLSY18,DBLP:conf/ndss/ZhouWZJ12,DBLP:conf/ccs/EnckOM09} , intent actions \cite{DBLP:journals/tsmc/YuanJLC21,DBLP:conf/uss/OcteauMJBBKT13,DBLP:conf/ntms/FereidooniCYS16}, Inter-Component 
Communications (ICCs) \cite{DBLP:conf/ndss/FengBMDA17,DBLP:conf/icse/BaiX00M20} and API calls \cite{DBLP:conf/ndss/ArpSHGR14,DBLP:journals/jnca/Seo0SBY14}. Recently, semantic features \cite{DBLP:conf/ccs/ZhangZZDCZZY20,DBLP:conf/ndss/MaricontiOACRS17,DBLP:conf/kbse/WuLZYZ019} (e.g., FCGs) have attracted increasing attention. They can characterize the behavior and functionality of apps, and hence achieve promising performance.

{ As the most common semantic feature, FCGs are often constructed based on smali files}. { A function or an \textit{abstracted} function denoted by its function name (e.g., java.lang.StrictMath: max()), class name (e.g., java.lang.StrictMath), package name (e.g., java.lang), or family name (e.g., java) can be used to represent a node in an FCG.}  Therefore, there exist four feature granularities in FCGs , i.e., function level, class level, package level and family level, as shown in Fig. \ref{fig:granularities}. 
{The features with finer granularities (e.g., class level)  usually have a more complex graph structure, causing heavier computational overhead and requiring dimensionality reduction \cite{DBLP:conf/ndss/MaricontiOACRS17}.}
% The features with coarser granularities (e.g., family level) are usually more resilient to API changes, and cause smaller computation and storage overheads, although they suffer less accuracy in characterizing the behavior and functionality of apps.
% \textcolor{green}{Furthermore, it is worth noting that the FCGs with the same feature granularity may differ in their node set. That is, they may adopt different functions or abstracted functions. For example, the frequent subgraphs proposed in \cite{DBLP:journals/tifs/Fan0LCTZL18} only use the sensitive APIs as their nodes, but a classical FCG can contain all Android APIs and user-defined functions.} 
\begin{figure}[bhtp]   
	\centering   
	\includegraphics[width=0.9\linewidth]{granularity.pdf}
	\put(-150,66){${G}_{family}$}
    \put(-128,28){${G}_{package}$}
	\put(-43,75){${G}_{function}$}
    \put(-35,8){${G}_{class}$}
	\caption{Different granularities of the FCG.}   
	\label{fig:granularities}   
\end{figure} 
%怎么构建图特征

% The process of obtaining a predetermined-granularity FCG from an app consists of three steps. First, one gets the classes.dex file through unpacking the corresponding APK file. Second, the classes.dex file is further decompiled into a series of smali files, which reveal the information of function calls in the app. Thrid, one can build an FCG with a required granularity based on the function call information. 
 
{Clearly, the knowledge about the feature granularity of the target system is helpful for adversaries to generate AEs. However, this prior knowledge is hard to obtain in practice. Hence, we put forward the incomplete feature information assumption, assuming that adversaries do not know the feature granularity of the target system. }

\subsection{{FCG} based Android malware detection}

%这里是举例子
%It is only recently that researchers have turned their attention to the graph based Android malware detection. 

% The {FCG} based detection methods choose some types of FCGs as feature and translate them into feature vectors for processing, as depicted in the second box in Fig. \ref{fig:motivation}. 

Here we introduce three state-of-the-art {FCG} based detection methods, which will act as the target detection systems in our experiments.  

\noindent \textbf{Mamadroid}. Mamadroid \cite{DBLP:conf/ndss/MaricontiOACRS17} considers the package-level or family-level FCGs as its features. More specifically, it adopts $340$ packages and $11$ families. To extract a feature vector from an FCG, Mamadroid constructs a Markov chain with the transition probabilities among packages or families. The extracted feature vectors are then used to train a classifier (e.g., KNN and SVM) for app classification. 

% \textbf{Malscan}. Malscan [] is another sate-of-the-art graph based Android malware detection method. It chooses sensitive APIs as the nodes in its FCGs, and then uses their centrality (e.g., degree centrality, Katz centrality and closeness centrality) as the feature vectors for classification.

\noindent \textbf{APIGraph}. Different from Mamadroid, APIGraph \cite{DBLP:conf/ccs/ZhangZZDCZZY20} is a general framework {for} further enhancing the performance of the graph based Android malware detection methods. It employs a clustering algorithm (e.g., K-means) to aggregate the nodes (i.e., functions) of an FCG, based on the similarity among their semantics. It then uses a specific function to represent all functions in every cluster. Finally, APIGraph builds a new FCG with coarser granularity, in which every node denotes a cluster of functions and every edge indicates the call between two clusters. Experiments show that the new FCG can result in better classification performance. 

% \noindent {
% \textbf{SRL}. SRL is a Semantics-preserving (i.e. functionality-preserving) Reinforcement Learning adversarial example attack against black-box GNNs (Graph Neural Networks) for Windows malware detection \cite{DBLP:journals/corr/abs-2009-05602}. This method uses semantic Nops injection to manipulate malware, and leverages reinforcement learning to select the
% appropriate semantic Nops and their corresponding basic blocks. It is noted that SRL is extensible to Android malware.
% }

\noindent {
 \textbf{GCN}. 
Graph Convolutional Network (GCN) is a powerful graph embedding method, which can be utilized to detect malware. For instance, the GCN is used to convert the control flow graph into a feature vector for malware detection in \cite{DBLP:journals/corr/abs-2009-05602}\footnote{ {\cite{DBLP:journals/corr/abs-2009-05602} mainly studies how to attack malware detectors, although it proposes a GCN based malware detection method.} }. In Section \ref{SEC:EXP}, we will apply the GCN to the FCG based Android malware detection. } %In our experiments, the GAN acts as the target model of {\ourtool}.    }
 

 
% \textcolor{green}{ While these methods have achieved impressive results, their network architectures suffer from an machine learning inherent shortcoming: adversarial example attack. Although there have been many works \cite{DBLP:journals/corr/GoodfellowSS14,DBLP:journals/corr/TanayG16,DBLP:conf/icml/PangLYZY22} study the reason of the adversarial example, few of them focus on the Android malware detection. In our opinion, there are two more reasons for adversarial example in Android malware detection. One is that the dataset in Android malware detection are likely be to a gathered state(e.g., a same Android malware family may have the similar code structure) which may increase the blind spots \cite{DBLP:journals/corr/SzegedyZSBEGF13} of the dataset. Second, the existing Android malware detection methods (especially the static detection method) cannot precisely  model the malware behavior (e.g., cannot analyse various conditional statement statically). Due to the above reasons, the existing Android malicious detection system is not really secure\cite{277204}. }

% \textcolor{green}{ While these methods have achieved impressive results, their network architectures suffer from an machine learning inherent shortcoming: adversarial example attack. Although there have been many works \cite{DBLP:journals/corr/GoodfellowSS14,DBLP:journals/corr/TanayG16,DBLP:conf/icml/PangLYZY22} study the reason of the adversarial example, few of them focus on the Android malware detection. In our opinion, there are two more reasons for adversarial example in Android malware detection. One is that the dataset in Android malware detection are likely be to a gathered state(e.g., a same Android malware family may have the similar code structure) which may increase the blind spots \cite{DBLP:journals/corr/SzegedyZSBEGF13} of the dataset. Second, the existing Android malware detection methods (especially the static detection method) cannot precisely  model the malware behavior (e.g., cannot analyse various conditional statement statically). Due to the above reasons, the existing Android malicious detection system is not really secure\cite{277204}. }



% \textcolor{green}{ While these methods have achieved impressive results, their network architectures suffer from an machine learning inherent shortcoming: adversarial example attack.  Although there have been many works  attribute the exist of the adversarial example to the misalignment  of the decision boundary  between the  classifier and real dataset \cite{DBLP:journals/corr/TanayG16,HU2022108824}, few of them focus on the Android malware detection. In our opinion, different from the image domain, in Android malware detection,  existing Android malware detection methods (especially the static detection method) cannot precisely  model the malware behavior (e.g., cannot analyse various conditional statement statically) which aggravates this misalignment, hence the existing Android malicious detection system is not really secure\cite{277204}.}

{ While these methods have achieved impressive results, they are susceptible to adversarial examples. The existence of adversarial examples is attributed to the problem that the decision boundaries of classification models are non-ideal \cite{DBLP:journals/corr/TanayG16,HU2022108824}. This problem becomes more serious in Android malware detection since the static analysis methods cannot precisely model the malware behavior. Therefore, the existing Android malware detection systems are not really secure\cite{277204}.}
	
\section{Problem formulation} \label{subsec: Problem}
 {Here we first introduce the system and threats considered in our work, and then propose an attack formulation to guide the design of black-box AE attacks.}
\subsection{{System \& Threat}}

{Fig. \ref{fig:motivation} depicts the {FCG} based Android malware detection system considered in this work. Suppose an adversary launches a black-box AE attack towards this system to produce real evasive malware. To this end, the adversary first gets the classes.dex file from an APK file, and further decompiles it into a series of smali files, as shown in Fig. \ref{fig:manipulation overview}. The adversary manipulates the smali code according to its perturbation, and rebuilds the code to obtain a new APK file. The adversary then queries the detection system with the generated malware sample, utilizes the received binary decision (i.e., benign or malicious) to update its perturbation, and then rebuilds a new malware sample. The above procedure is repeated until a real evasive malware is obtained. }

The adversary only knows that the target system uses FCG feature for malware detection. However, the adversary does not know the feature granularity and the graph embedding method used by the target system. Moreover, the adversary has no information about the architecture, the parameters and the output probabilities of the target classifier. As for the defender, it can use static analysis and white list based defenses to resist evasive malware. In addition, the defender may raise alarms once the number of queries from a user is unusually large\footnote{ {
Our experiments indicate that our method only needs dozens of queries to generate the perturbations that can successfully attack the target model. Moreover, our method can further reduce the number of perturbations by conducting more queries (e.g., several hundreds of queries). To accelerate the attack process, we provide a substitute network to fit the target model. The related experiments can be found in  Section \ref{sec:RQ2}}}.

%如果还有camera ready的版本我建议直接删掉这句话。。。。因为查询次数确实不占优。特别是我们通过多粒度把问题复杂化了,需要的次数更多。

% To find adversarial perturbation, the adversary needs to query the target system with its currently generated malware sample, and utilize the reply (i.e., the binary decision on whether the received malware sample is malicious or not) to update its generated perturbation and then rebuild a new malware sample. The above procedure is repeated until a desired adversarial perturbation is found and a real evasive malware is obtained. 

\begin{figure}[htbp]
	\centering
	\includegraphics[scale=0.8]{manipulation_overview.pdf}
	\caption{Overview of the AEs generation.}
	\label{fig:manipulation overview}
\end{figure}



\subsection{Attack formulation}
% Now we propose our attack formulation, and point out how to launch AE attacks inspired by this formulation. 
For convenience, {we first use $s$ and $m$ to refer to the malware sample and the manipulation}, respectively. We then use two functions $\mathcal M_G(\cdot)$ and $\mathcal M_V(\cdot)$ to denote the code-to-graph mapping and the graph-to-vector mapping shown in Fig. \ref{fig:motivation}, respectively. Through manipulating the malware sample $s$ with $m$, the adversary changes the input graph from $G=\mathcal M_G(s)$ to $\widetilde{G}=\mathcal M_G(s+m)$, where $G$ and $\widetilde{G}$ represent the original input {FCG} and the perturbed input {FCG}, respectively. Suppose $L(\cdot)$ denotes the label (i.e., benign or malicious) predicted by the target classifier. Then, the desired adversarial manipulation $m^*$ can be derived by solving the following problem:
\begin{equation}
L(\mathcal M_V(\mathcal M_G(s))) \neq L(\mathcal M_V(\mathcal M_G(s+m^*))) \label{objective: misleading}
\end{equation}
% s.t.,
% \begin{equation}
% 	F(s)=F(s+m^*) \label{constraint:preserving}
% \end{equation}
%  {Here $F(\cdot)$ represents the functionality set of malware, and the constraint \eqref{constraint:preserving} is introduced to guarantee the malware functionality is not changed by the manipulation. }
 {under the constraint of malware functionality preservation. }


{The above formulation points out two tasks for us: 1) designing a manipulation technique to modify malware code while preserving malware functionality, and 2) developing an adversarial perturbation generation algorithm to realize $m^*$. Due to the challenges of problem-feature space gap and strict black-box setting, $\mathcal M_G(\cdot)$ and $\mathcal M_V(\cdot)$ are actually unknown to the adversary. Hence it is extremely hard to derive the desired adversarial perturbation in one shot.} This motivates us to develop an evolutionary algorithm (i.e., {\ouralgorithm}) to gradually find the desired perturbation. {We will discuss how to fulfill  the above two tasks in Sections \ref{sec:Adversarial manipulation design} and \ref{sec:method}, respectively. } 

{Furthermore, it is noted that a variety of graph adversarial attack models \cite{DBLP:conf/ijcai/XuC0CWHL19,DBLP:conf/icml/BojchevskiG19,DBLP:conf/ijcai/Wu0TDLZ19,DBLP:conf/iclr/ZugnerG19,DBLP:conf/icml/DaiLTHWZS18,DBLP:conf/kdd/0001WDWT21,DBLP:conf/www/SunWTHH20} have been proposed in the community of machine learning. 
Although these methods offer inspirations to us, they cannot be directly applied to our attack for two reasons. First, graph adversarial attack models launch attacks from feature space. However, the attack against Android malware detection cannot directly access feature space, and has to indirectly affect feature space through manipulating malware code in problem space. Second, our attack needs to meet practical requirements (i.e., \textbf{R1}-\textbf{R4} discussed in Subsection \ref{subsec: four R}), which are absent in existing graph adversarial attacks. Therefore, specialized study is needed for malware adversarial attack design. } 




% (1) Adversarial manipulation needs to satisfy the realistic requirements for real-world application. For example, the adversary needs to appropriately choose the type of manipulation operations to preserve the functionality of the malware and to avoid the static analysis.  In Section \ref{sec:Adversarial manipulation design}, we will first discuss the requirements for adversarial manipulation, then analyze existing manipulation methods (e.g., adding or removing function calls) and introduce our method.

% (2) In our settings, $\mathcal M_G(\cdot)$ and $\mathcal M_V(\cdot)$ are unknown to the adversary. Hence it is almost impossible to derive the desired adversarial manipulation $m^*$ in one shot. A more promising way is to iteratively improve or update a manipulation $m$ until the objective (i.e., Eq. \ref{objective: misleading}) and the constraint (i.e., Eq. \ref{constraint:preserving}) are both satisfied. This motivates us to develop an evolutionary algorithm (i.e., {\ouralgorithm}) to find the desired perturbation.  

% In the following two sections, we first introduce our malware manipulation technique, and then propose our black-box attack method (including perturbation generation algorithm), respectively. 





% \subsection{Overview of {\ourtool}}
% The main task of designing {\ourtool} can be divided into two parts: 1) manipulation technique design, and 2) perturbation generation method. The manipulation technique is used in problem space, and the perturbation generation method is used in 

% Now in this section, we  summarizes our approach to generate  adversarial perturbations to the Android malware detection method on problem space. Our method can be divided into two parts, one is design an adversarial manipulation to modify the Android APK file, and the other one is to  study how to generate  appropriate perturbation for the adversarial manipulation.

% For adversarial manipulation,  it needs to  satisfy the some requirements for real-world application . For example, the adversary needs to appropriately choose the type of manipulation operations to preserve the functionality of the malware and to avoid the static analysis. In Section \ref{sec:Adversarial manipulation design}, we will first discuss the requirements for the adversarial manipulation. Then we analyse existing manipulation methods (e.g., adding or removing function calls) and introduce our method.


% For adversarial manipulation, the adversary need to solve the inequality \eqref{objective: misleading}. However, the code-to-graph mapping $\mathcal M_G(\cdot)$ and the graph-to-vector mapping $\mathcal M_V(\cdot)$ usually are unknown to the adversary. Hence it is almost impossible to obtain the desired adversarial manipulations in one shot. A promising alternative method is to iteratively update (or improve) the manipulations $m$ until the inequality and the constraint are both satisfied. In the above iterative operations, the adversary needs to evaluate the effects of current manipulations. To this end, the adversary sends the manipulated malware to the target system for malware detection, and uses the returned detection outcome (i.e., the predicted label) to help improve manipulations. A main challenge is that the detection outcome only tells whether the current manipulations induce misclassification or not, but it provides little information for manipulation improvement. To overcome this challenge, we will propose to simulate the target detection with a GCN in Section \ref{sec:method}. 
	
\section{ Malware manipulation} \label{sec:Adversarial manipulation design}
{
In this section, we first introduce the common requirements and the existing techniques \cite{DBLP:conf/sp/PierazziPCC20,DBLP:journals/corr/abs-2009-05602,chen2020android} of malware manipulation, and then propose a new malware manipulation technique.}

\subsection{Background of malware manipulation} \label{subsec: four R}
{Although the manipulation on malware is intuitively simple, the challenges come from the following requirements.} \\
\textbf{R1: Functional Consistency.} The malware functionality should keep consistently before and after manipulation.\\
% Therefore, malware manipulation should conform to smali syntax rules, and cannot change the functionality of malware. 
\textbf{R2: All-granularities influence.} Since the feature granularity (e.g., family level and package level) of malware detection is unknown, malware manipulation should be able to affect the features of all granularities \cite{DBLP:conf/ndss/MaricontiOACRS17}. \\
\textbf{R3: Resilience to static analysis.}  Malware manipulation 
% should be able to resist static analysis 
 {should not be hindered by static analysis inspection}
\footnote{{ In this work, the static analysis mainly refers to the program analysis techniques that only examine the source code but do not execute the program.}} \cite{DBLP:journals/tdsc/DemontisMBMARCG19,DBLP:conf/acsac/MoserKK07}, and cannot completely rely on dead codes  {(i.e., unreachable instruction blocks)}.\\
\textbf{R4: Non-stationary perturbation.} Manipulation  should be non-stationary and cannot be restricted to a fixed set of operations (e.g., a pre-determined white list  \cite{DBLP:journals/corr/abs-2009-05602}), {to reduce the risk of being identified. } \\
%  {\st{\textbf{R5: No-additional functions introduced.} When we add a certain API into the smali fils, there are no expected that other functions are inserted concomitantly. Because concomitantly functions may cause decrease of the attack success rate.}}

{Existing manipulation methods are summarized below.}\\
\textbf{Inserting dead codes}: To maintain functional consistency, \cite{chen2020android} chooses to insert dead codes (e.g., no-op calls) into smali files. {Unfortunately, these codes can be easily detected and filtered, violating the requirement \textbf{R3}. For example, \cite{DBLP:journals/tifs/Fan0LCTZL18} proposes a weighted sensitive-API-call-based Android malware family classification method, which can resist the impact of no-op calls. } \\
\textbf{Adding valueless calls}: \cite{chen2020android} creates user-defined classes and adds valueless calls (i.e., invoking empty functions) into them. {However, these calls may be susceptible to static analysis and cannot attack the class-granularity FCGs, violating the requirements \textbf{R2} and \textbf{R3}.  For example, the Android malware detection method proposed in \cite{DBLP:journals/tsmc/YuanJLC21} does not use self-defined functions as feature. Hence, this method is not influenced by the valueless calls inserted by adversaries. }\\
\textbf{Adding functions from a white list}: To change FCGs, the authors of \cite{DBLP:journals/corr/abs-2009-05602} add a function coming from a predetermined white list. However, once adversarial examples are captured, the white list will be revealed and adversarial attacks may fail. Please refer to  requirement \textbf{R4}.\\
{\textbf{Opaque predicates}: \cite{DBLP:conf/sp/PierazziPCC20} leverages opaque predicates to insert new APIs for malware detection evasion. Specifically, this method constructs obfuscated conditions where the outcome is always known in design phase but the truth value is difficult or impossible to determine by static analysis. Hence this method can effectively resist static analysis. However, it may introduce some undesired functions (e.g., the \textit{random} function), which impose unexpected impacts on FCGs. }

% For example, it may use the \textit{random} function to construct a 'if' conditions  that are almost impossible to meet. However, although the inserted functions in 'if' block evade the static analysis, it introduce unexpected function 'random' too, which may affect the attack performance. Please refer to the requirement \textbf{R5}  }

\subsection{The proposed manipulation method} \label{sec:adversarial manipulation method}
Here we design a new malware manipulation method to modify smali code. Clearly, we cannot remove nodes or edges from FCGs, according to the requirement \textbf{R1}. Hence, we only consider adding (or inserting) nodes or edges. However, adding isolated nodes (i.e., the functions that are not invoked or do not invoke others) is not recommended for two reasons. First, the isolated nodes are easily detected by static analysis {(e.g., some program analysis techniques that perform redundant code elimination would remove unreachable code \cite{guardsquare})}. Second, adding nodes usually cannot impact feature space, since lots of malware detectors utilize edges (instead of nodes) for classification. As a result, we select adding edges (i.e., calls) in our manipulation method. Then the rest of the problem includes:  how to create \textit{candidate} edges,  how to select desirable edges from the candidate edges, and  how to insert the selected edges.  {
In this section, we only consider the first and the third problems. The second problem will be solved in Section \ref{sec:method}.}

\textbf{(1) How to create candidate edges?}
\begin{figure}
	\centering
	\includegraphics[scale=0.85]{example.pdf}
	\caption{Selecting callers and callees from an FCG.}
	\label{fig:example}
\end{figure}


\begin{figure}
	\centering
	\includegraphics[scale=0.55]{Pseudo.pdf}
	\caption{An example of try-catch trap.}
	\label{fig:manipulation}
\end{figure}

 {Up to now, how to impose all-granularities influence (required by \textbf{R2}) on FCG with incomplete feature information (i.e., the feature granularity is unknown) has not been thoroughly studied. To tackle this problem, we propose to create an edge between two nodes of any type by adding a function call between a caller and a callee. This method changes the FCG no matter what kind of feature granularity is used. } Then the problem becomes how to determine the caller and the callee for every candidate edge. Due to the requirement \textbf{R4}, we cannot utilize a white list to generate callers and callees. Instead, we propose to generate them from the functions used by malware itself. In this way, we can ensure that the candidate edges created for different malware are diverse, hence satisfying the requirement \textbf{R4}.  

{Now we study where to place the added edges.} An FCG consists of \textit{non-leaf} nodes and \textit{leaf} nodes, as depicted in Fig. \ref{fig:example}. The non-leaf nodes are user-defined functions, and the leaf nodes correspond to Android standard functions (e.g., $java/io/File;-\textgreater exists()$) or the user-defined functions that do not invoke others. In our method, non-leaf nodes (i.e., user-defined functions) are selected as callers, since they are easily inserted with new function calls. Leaf nodes are chosen as callees,   since invoking a function that does not invoke others will not trigger unintended calls. Here we avoid generating unintended calls because they may further impose perturbations on the FCG, which is beyond our expectation. Furthermore, we supply more discussion on callee selection in Appendix \ref{sec:callees' limitation}. Now we can use the above method to create candidate edges. In Section \ref{sec:method}, we will propose an algorithm to select the most desirable edges for manipulation.

\textbf{(2) How to insert selected edges?}

 We assume that the desirable edges have been selected, and study how to insert the corresponding function calls into smali files under the requirements of \textbf{R1} and \textbf{R3}. Our proposed method is called \textbf{try-catch trap}. It first inserts a try-catch block into the caller, and places the statement of invoking callee in its try block. It then adds several statements in front of this function call statement. These statements are used to trigger a pre-selected exception (e.g., an arithmetic exception). Now we analyze why this method works. First, it inserts a function call statement in smali files, hence changing the FCG by adding a new edge. Second, the function call statement is never executed,  {hence preserving malware functionality}. For illustration, Fig. \ref{fig:manipulation} gives an example of try-catch trap. Suppose the codes in the left box come from a malware sample. The function \textit{callerEX()} is selected as our caller. We place a try-catch block in this function, and invoke the function callee() after the blue statement is executed. In this way, we can add a new edge into the FCG, as shown in Fig. \ref{fig:manipulation}. When the try-catch block is executed, an exception of \textit{IndexOutOfBoundsException} will be thrown, and the statement of function call will be skipped over. 
{In summary, our method can be considered as a variant of opaque predicates. It carefully constructs obfuscated conditions that are difficult to determine during static analysis, hence possessing the ability to resist static analysis.}

 {The main steps of inserting function calls are briefly described in Appendix \ref{appendix:smali}.}

 % The main steps of inserting function calls are briefly described below. First, we find the smali file related to the selected caller, according to the latter's full name. Second, we insert statements into the smali file to implement a try-catch trap. We can use five invocation types, including invoke-direct, invoke-virtual, invoke-static, invoke-super and invoke-interface. These invocation types result in different smali manipulations, due to their various requirements for register usage. To facilitate understanding, we provide several examples of invocation types in Fig. \ref{fig:direct} of Appendix \ref{appendix: smali example}. We also show how to modify the smali codes in Appendix \ref{appendix:smali}. Third, we use \textit{Apktool} to rebuild the modified smali files to APK file. In our work, the above operations are automatically conducted by a Python script. 

 
% \textcolor{green}{In summary, The key idea of our method and opaque predicates is similar, both we want to carefully constructed obfuscated conditions which are difficult to determine during a static analysis, hence {\ourtool}  has the ability to resist static analysis.}
	
% 	First of all, for API call graph, we can only add nodes or edges, but can not delete the original existing graph, because any deletion may affect the original APK function (\textbf{subject to C1}).
% 	Secondly, in order to ensure the diversity of perturbation, we do not set a fixed whitelist, but construct a perturbation candidate set according to the function used by each APK. So that the candidate perturbation of each different Android software are different and the attacker will not take a risk of whitelist leaking (\textbf{subject to C4}). What's more, because we immediately add a method to the api
%      call graph, it can  affect different granularity feature (\textbf{subject to C3}).
% 	Third, in order to ensure the automation and simplicity of the modification process,  we will record the statement method of each function (function-call notation, whether to use registers, etc.) during feature extraction of an Android APK. So there is no need to introduce human experience to achieve automatic perturbation. (\textbf{subject to C5}). 
% 	Finally, we supply a  new method to add these functions in smali code which can  evade being deleted by static analysis (\textbf{subject to C2}). This method will be discussed in next subsection.
	

% 	Based on the above considerations, our perturbation strategy is  adding edges to the existing API call graph. Specifically, when extracting the function call graph of a malware, the algorithm will construct the candidate perturbation pool that can be applied according to the existing functions in the graph. These candidate edges to be added involve the calling function and the called function:
	
% 	\begin{itemize}
% 		\item \textbf{Caller:} In order to facilitate subsequent modifications, we choose non-leaf nodes in the API call diagram as the keynote nodes. These non-leaf nodes are self-defined functions and can be  easily modified and injected new functions.
		
% 		\item \textbf{Callee:} The callee functions are selected from the existing leaf nodes in the API call graph. These callee functions have been used in the original apk, so it is easy to statement the function by following the original smali code. However, the selection of leaf nodes also should follow the android programming rules. 1) Access modifiers   restriction: For leaf nodes' access modifiers, we  only  remain those functions whose access modifier is 'public', so that the leaf nodes can be called in all other classes. 2) Parameter limitation:  When selecting leaf nodes, we also need to filter function according to the  input parameters. We will leave those whose input parameters are empty or  basic types(e.g., 'int','double', etc.)\footnote{This limitation is to ensure the ease of manipulation, }.
		
	
% 	\end{itemize}



	
	
	

	
\newcommand{\tabincell}[2]{\begin{tabular}{@{}#1@{}}#2\end{tabular}}  


% \begin{table}[H]
%     \renewcommand\arraystretch{1.3}
% 	\caption{Common declaration examples.}
% 	\centering
% 	\scalebox{0.8}{
% 		\begin{tabular}{cc|cc}
% 			\hline
% 			Type   & Smali definition  & Type   & Smali definition            \\ \hline
% 			\cellcolor{cyan!60!gray!10}byte   &\cellcolor{cyan!60!gray!10} const/4 v1, 0x0    & \cellcolor{cyan!60!gray!10}long                & \cellcolor{cyan!60!gray!10}const-wide/1   v8, 0x1            \\ 
% 			char   & const/16 v1, 0x61  & float               & const v1, 0x3f8ccccd              \\ 
% 			\cellcolor{cyan!60!gray!10}short  & \cellcolor{cyan!60!gray!10}const/4 v1, 0x1    & \cellcolor{cyan!60!gray!10}  boolean            & \cellcolor{cyan!60!gray!10}   const/4 v1, 0x0         \\ 
% 			int    & const/4 v1, 0x1    &   double       &        \tabincell{c}{ const-wide v1,  \\ 0x4000000000000000L}           \\ 
% 			\cellcolor{cyan!60!gray!10} String   & \cellcolor{cyan!60!gray!10} const-string v1, "123"    & \cellcolor{cyan!60!gray!10}-             & \cellcolor{cyan!60!gray!10}-                   \\ \hline
% % 			\multicolumn{2}{c|}{double} & \multicolumn{2}{c}{const-wide v1,  0x4000000000000000L} \\ \hline
% 		\end{tabular}%
% 	}
% 	\label{tab:statement}
% \end{table}		

	
\section{Adversarial perturbation generation} \label{sec:method}
{ In Subsection \ref{sec:adversarial manipulation method}, we propose the question of how to
 select desirable edges from the candidate edges. To answer this question, we develop a novel GAN model and the algorithm {\ouralgorithm} to find the desired adversarial perturbation. } 

\subsection{Challenges \& Solutions}

{We first introduce the main procedure of {\ourtool} below. }

 (1) Given a pre-selected malware sample, {\ourtool} finds some callers and callees from the smali codes, and uses them to create a set of candidate edges, as discussed in Section \ref{sec:Adversarial manipulation design}. With candidate edges, {\ourtool} generates a variety of samples through manipulating malware, and sends them (i.e., queries) to its target system for malware detection.
 
 (2) The target model sends back a reply for a query. In our strict black-box setting \cite{DBLP:conf/iclr/ZhaoDS18}, every reply contains only the binary classification outcome (i.e., malicious or benign). 
 
 (3) Through learning from the query-reply pairs, {\ourtool} gradually recognizes the most desirable edges that can successfully induce misclassification.

 The main challenges in designing {\ourtool} include: 1) {the feature granularity} of the target model is unknown, 2) a large number of queries are usually required in the strict black-box attack scenario \footnote{{In this scenario, both $\mathcal M_G(\cdot)$ and $\mathcal M_V(\cdot)$ mentioned in Subsection 3.2 are unknown. Moreover, the reply of the   black-box model contains only the
binary classification outcome (e.g., many malware detection websites \cite{VirusTotal} only sends back a binary decision instead of class probabilities). }} \cite{DBLP:conf/icdm/LiYZ18,DBLP:conf/eccv/AndriushchenkoC20}.
%  The first challenge increases the difficulty in selecting edges to perturb FCG, since {\ourtool} does not accurately know how its added edges impact the FCG. The second challenge {requires reducing the number of queries} in order to reduce overheads and avoid being defended. 
Our countermeasures are briefly explained below. 

\textbf{Surmising feature granularity}. {Our adversarial multi-population co-evolution algorithm, i.e., {\ouralgorithm}, uses one population to represent a possible feature granularity}. {The multiple populations, corresponding to multiple possible feature granularities}, cooperatively evolve until the population corresponding to the real feature granularity keeps alive but the others fade away. In this way, {\ourtool} can accurately identify the feature granularity used by its target model, {as will be shown in Subsection \ref{subsection:evolution}}. 

\textbf{Reducing the number of queries}. {\ourtool} constructs a novel \textit{substitute} model to simulate its target model. The substitute model is trained with the samples generated by {\ouralgorithm} and labeled by the target model. {As will be shown in Subsection \ref{subsection: substitute}}, once the substitute model is well trained, {\ourtool} only needs to attack it instead of the target model, {hence greatly reducing} the number of queries. 

\subsection{The overview of {\ourtool}}
{Following the architecture of GANs, {\ourtool} adopts a generator and a discriminator that are cooperatively trained.} 

\textbf{Generator}: The generator is responsible for generating perturbations, i.e., the new edges added into the FCG. It is implemented with an adversarial multi-population co-evolution algorithm (i.e., Apoem).

\textbf{Discriminator}: {The discriminator is introduced to stimulate the generator to improve its perturbations. It is implemented with a GCN, acting as a substitute network \cite{chen2020android} to simulate the target model.}

% As shown in Fig. \ref{fig:bg1}, the generator {leverages a multi-population co-evolution algorithm \ouralgorithm}  to generate perturbations, i.e., the new edges added into the FCG. {The discriminator is introduced to stimulate the generator to improve its perturbations. It is implemented with a graph convolutional network (GCN), which acts as a substitute network \cite{chen2020android} to simulate the target model. }

\textbf{Training}: {In each round of model training}, the generator modifies the malware's code and sends the rebuilt malware to the target model or the substitute model for malware detection. {\ourtool} makes a choice between the target model and the substitute model with a variable probability $p$. After receiving the queries, the target model sends back its replies, i.e., the binary decisions. With the query-reply pairs, {\ourtool} trains the substitute model and guides its generator to {improve its generated perturbations}. {The probability $p$ keeps growing as the number of rounds increases, to decrease the number of queries sent to the target model}.
 
% In the following, we first introduce the multi-population co-evolution mechanism and the substitute model, and then propose the formal description of our algorithm.
 
%  The main challenges in designing {\ourtool} include: 1) the granularity of the features used by the target model is unknown, 2) lots of queries are usually required in the strict black-box attack scenario \cite{DBLP:conf/icdm/LiYZ18,DBLP:conf/eccv/AndriushchenkoC20}. The first challenge increases the difficulty in selecting edges, since {\ourtool} does not accurately know how its function calls impact the API call graph. The second challenge raises an urgent demand for reducing query amount in order to reduce overheads and avoid being defended. Our countermeasures are briefly explained below. 


% \textbf{Surmising feature granularity}. {\ourtool} introduces the mechanism of \textbf{A}dversarial \textbf{M}ulti-\textbf{P}opulation \textbf{C}o-\textbf{E}volution (A-mpce) to overcome the challenge of unknown feature granularity. This mechanism uses multiple populations to represent all possible feature granularities. These populations cooperatively evolve until the population corresponding to the real feature granularity still keeps alive but the others fade away. In this way, {\ourtool} can accurately identify the feature granularity used by its target model. 


% \textbf{Reducing the amount of queries}. {\ourtool} constructs a novel \textit{substitute} model to simulate its target model, which is trained with the samples generated by {\ourtool} and labeled by the target model. Since the substitute model resembles the target model in malware detection, it can be used to help {\ourtool} accelerate its black-box attack. In fact, once the substitute model has been well trained, {\ourtool} only needs to attack it instead of the target model. In this way, {\ourtool} greatly reduces the amount of required queries. Existing substitute models \cite{chen2020android} are often built with MLPs. This is because existing attacks occur in feature space and choose to directly perturb feature vectors. However, {\ourtool} considers a more practical scenario where attacks take place in problem space, and hence it is faced with API call graphs instead of vectors. Under this situation, {\ourtool} builds a graph convolutional network (GCN) based substitute model whose inputs are graph data. As we all know, GCNs \cite{DBLP:conf/iclr/KipfW17} can efficiently extract features from graph data through utilizing the properties of nodes and edges in graph data. And this raises another important problem: how to assign appropriate properties to the nodes in graph data. To our knowledge, this problem has not been thoroughly studied in  existing literature. For the first time, we propose to use degree information as node properties, which can speed up the training of the substitute model.
% \begin{figure} 
% 	\centering   
% 	\includegraphics[width=0.95\linewidth]{overview.pdf}
% 	\caption{The overview of {\ourtool}.}   
% 	\label{fig:motivation_al}   
% \end{figure} 
    
% Now we give an overview of {\ourtool} in Fig. \ref{fig:motivation_al}. This figure shows how to generate a real evasive malware within three steps. First, {\ourtool} obtains the classes.dex file through unpacking an APK file, and then decompiles it into a series of smali files. Second, {\ourtool} extracts an FCG from the smali files, and uses our proposed AE generation algorithm to find the appropriate edges to perturb the API call graph. Third, {\ourtool} adversarially manipulates the Smail files and rebuilds a new APK file. The key module of {\ourtool} is the proposed AE generation algorithm, called Adversarial multi-population co-evolution algorithm ({\ouralgorithm}), which will be discussed in the next subsection.

%As . The GAN model consists of a generator network and a discriminator network. The former is responsible for perturbation generation, and the latter is introduced to stimulate the generator to improve the quality of its generated perturbations. In the initial phase of model training, the generator considers multiple populations corresponding to various feature granularities. Stimulated by the discriminator, the populations cooperate to evolve, until the population representing the real feature granularity thrives and the others become feeble. At this moment, the algorithm succeeds in capturing the real feature granularity and figuring out how to manipulate the input graph.  

\begin{figure*}[htbp]
	\centering
	\includegraphics[scale=0.5]{method1.pdf}
	\caption{The model architecture of {\ourtool}.}
	\label{fig:bg1}
\end{figure*}
% \subsection{The basic idea of {\ouralgorithm} algorithm}



% Given a malware sample $\mathcal{S}$, {\ourtool} aims to derive an evasive malware sample $\mathcal{\bar{S}}$, through adversarially manipulating $\mathcal{S}$. We use $FCG_{S}$ and $FCG_{\mathcal{\bar{S}}}$ to refer to the FCG of $\mathcal{S}$ and $\mathcal{\bar{S}}$, respectively. According to {\ourtool}, we know that $\mathcal{S}$ and $\mathcal{\bar{S}}$ have the same node set but $\mathcal{\bar{S}}$ contains more edges. The main task of our {\ouralgorithm} algorithm is to find these edges and generate $\mathcal{\bar{S}}$. To do it, {\ouralgorithm} trains a specially designed GAN model to generate $FCG_{\mathcal{\bar{S}}}$  based on $FCG_{S}$. According to $\mathcal{\bar{S}}$, {\ourtool} can adversarially manipulate the smali codes of $\mathcal{S}$,  and rebuilds a new APK file, i.e., $\mathcal{\bar{S}}$. %Please find the details on adversarial manipulation and APK rebuilding in Section XXX.

% The GAN model trained by {\ouralgorithm} is shown in Fig. \ref{fig:bg1}. This model consists of two components, i.e., a generator and a discriminator. The generator is responsible for generating perturbations in black-box environment, i.e., the new edges added into API call graph. The discriminator acts as a substitute network that fits or simulates the target classification model. It is introduced to stimulate our generator to improve its perturbations. It is worth noting that our GAN model differs from existing GANs in two respects. First, its generator is a multi-population co-evolution algorithm, instead of a neural network. Second, its discriminator is a GCN, instead of an MLP or CNN.  


% The training procedure of this GAN model consists of multiple rounds. In each round, the generator modifies $\mathcal{S}$ and sends the manipulated malware samples to the target model or the substitute model for malware detection. {\ouralgorithm} makes a choice between the target model and the substitute model with a probability $p$. After receiving these queries, the target model sends back its replies, i.e., the binary decisions. With these query-reply pairs, {\ouralgorithm} trains the substitute model and guides its generator to produce improved malware samples. It is worth noting that the probability $p$ keeps growing as the number of rounds increases. In this way, {\ourtool} can reduce the number of queries to the target model.

% In the following, we first introduce the multi-population co-evolution and substitute model, and then propose the formal description of our {\ouralgorithm} algorithm. 

\subsection{Adversarial Multi-population co-evolution}
\label{subsection:evolution}
\begin{figure}[htbp]
	\centering
	\includegraphics[scale=0.6]{ga.pdf}
	\caption{{How multiple populations cooperatively evolve?}}
	\label{fig:GA}
\end{figure}
    { The main challenge faced by the generator is 
  {that the real feature granularity is unknown.} To facilitate the understanding, we consider the case where the target system uses the family-level feature but we perturb the class-level feature. In this case, we will fall into a huge search space, hence prolonging model training time and requiring more queries. To alleviate this problem, {\ourtool} uses the {\ouralgorithm} algorithm to surmise the real feature granularity.} {\ouralgorithm} follows the general framework of evolutionary algorithms, but it introduces  cooperation among multiple populations to speed up convergence. Along with the evolution, the population corresponding to the real feature granularity gradually stands out from the crowd. In the following, we first describe the main components of {\ouralgorithm} depicted in the red block of Fig. \ref{fig:bg1}, and then discuss how to use these components {to generate the desired perturbation}. 
 
\textbf{(1) Population \& Individual}. 
{A population represents  a collection of generated AEs under a certain feature granularity. For example, the family-level population consists of the AEs generated under the assumption that the target classifier uses a family-level FCG as its input. {\ouralgorithm} adopts multiple populations, each of which corresponds to one possible feature granularity (i.e., family, package and class)}.
Each individual in a population gives a perturbation that can be imposed on the original FCG\footnote{Strictly speaking, an individual refers to an adversarial example in a population. However, the difference between adversarial example and malicious example is perturbation. Hence we use the perturbation to represent an individual.}, i.e., the set of edges added into the FCG. As shown in Fig. \ref{fig:GA} (a), the above graph denotes the original FCG, and the below graph represents an adversarial example. Accordingly, the perturbation, i.e., the edge set $(A\rightarrow E, B \rightarrow D)$, is considered as an individual. {We use $x_r^{(i,j)}$ to refer to the $j$-th individual of the $i$-th population in the $r$-th generation of {\ouralgorithm}. We have $x_r^{(i,j)} = \{ e_1^{(i,j)},e_2^{(i,j)},...,e_{n}^{(i,j)}\}$, where $e_{k}^{(i,j)}$ ($1\leq k\leq n$) is the added edge.}
	In the initial phase, we need to collect sufficient individuals to build the populations. Therefore, we randomly perturb the original FCG, and get a set of individuals for each population.
	
 \textbf{(2) Fitness \& Selection}.
	{{\ouralgorithm} employs the metric fitness to select superior individuals and eliminate inferior individuals.}
	{This metric reflects the aggressivity and the invisibility of an AE. Its calculation takes into account two factors:} threat degree $T$ and perturbation amount $L$. The threat degree is measured according to the output of the target model $F(\cdot)$ or the substitute model $S(\cdot)$\footnote{{For the target or substitute model, the input is detected to be malicious when its output $F(x)$ or $S(x)$ equals to or approaches 1.  }}. For an individual $x$, the threat degree is defined as:
		{
	\begin{equation}
	 T =\left\{\begin{array}{l}
1-F(x) \quad\text{if target model is used} \\
1-S(x) \quad\text{if substitute model is used}
\end{array}\right. 	    
	\end{equation}}
The perturbation amount is calculated as the number of added edges.
%  {
% 	\begin{equation}	
% 	L= len(\{ e_1^{(i,j)},e_2^{(i,j)},...,e_{n}^{(i,j)}\}) = n
% 	\end{equation}}
	{Furthermore, {\ouralgorithm} introduces the \textit{elitist} selection strategy \cite{2012Genetic}} to pass on the good genes of individuals to the next generation, { through retaining the fittest individuals and eliminating the others.} %Therefore, the individuals whose adversarial examples are detected to be benign by the target model or have the highest benign probability outputted by the substitute model are selected and survive.
%	It is worth noting that with the increase of rounds, the substitute network will be more similar to the target black-box classifier. It is more likely to take  substitute  network's output as threat level $T$, thus reducing the number of queries of the black-box model.

\textbf{ (3) Immigration}. {In general, the individuals with high fitness have a greater chance of producing better offsprings. To produce more high-quality individuals, {\ouralgorithm} leverages the immigration operation to transfer individuals with high fitness within one population into other populations. Accordingly, the superior individuals immigrate to different populations, making all populations cooperatively evolve to generate better AEs.}
{There exist two kinds of immigration in {\ouralgorithm}: fine-to-coarse (e.g., from class level to family level) and coarse-to-fine (e.g., from family level to class level), as shown in Fig. \ref{fig:GA} (b). We first consider the fine-to-coarse case where one individual in the class-level population is immigrated into the package-level population. In this case, the name of the packages related to the perturbation (e.g., java.lang.StrictMath->java.lang) is retained and the individual containing only package names is then put into the package-level population. Now we consider the coarse-to-fine case where the individual from the package-level population is injected into the class-level population. Since a package may contain multiple classes, we randomly select one class used by malware code to replace the package and then put the individual containing class names into the class-level population.}


\textbf{(4) Crossover}. 
{{\ouralgorithm} leverages crossover to randomly swap genes from two parents to produce offsprings. More specifically, $K$ pairs of individuals are randomly chosen from a population as parents, and half of the perturbation in every pair is exchanged to produce two offsprings, as shown in Fig. \ref{fig:GA} (c).} { Suppose the parents are $	 x_r^{(i,j_1)} =\{e_1^{(i,j_1)},e_2^{(i,j_1)},e_3^{(i,j_1)},e_4^{(i,j_1)}\}$ and $ x_r^{(i,j_2)} = \{e_1^{(i,j_2)},e_2^{(i,j_2)},e_3^{(i,j_2)},e_4^{(i,j_2)}\}$, where  $e_{k}^{(i,j)}$ is an added edge (e.g., A->E)  in Fig. \ref{fig:GA} (c) . The offsprings derived by crossover are $x_{r+1}^{(i,j_1)} = \{e_1^{(i,j_1)},e_2^{(i,j_1)},e_3^{(i,j_2)},e_4^{(i,j_2)}\}$ and $x_{r+1}^{(i,j_2)} = \{e_1^{(i,j_2)},e_2^{(i,j_2)},,e_3^{(i,j_1)},e_4^{(i,j_1)}\}$, respectively. }

	 
 \textbf{(5) Mutation}. 
{{\ouralgorithm} employs mutation to bring new changes to a population.} As depicted in Fig.  \ref{fig:GA} (d), there are three possible mutation modes: 1) randomly adding function calls on the existing perturbation, 2) randomly reducing existing perturbation, and 3) randomly exchanging existing  perturbations. {They can be mathematically expressed as $x_{r+1}^{(i,j)}=\{e_1^{(i,j)},  ... ,e_{n}^{(i,j)},e_{n+1}^{(i,j)}\}$, $	x_{r+1}^{(i,j)}=\{e_1^{(i,j)},  ... ,e_{n-1}^{(i,j)}\}$, and $x_{r+1}^{(i,j)}=\{e_1^{(i,j)},  ... ,e_{n-1}^{(i,j)},e_{n+1}^{(i,j)}\}$, respectively.}


 
%  Then our immigration operation can be defined by
	 
% 	 \begin{align} 
% 	 	x_{r+1}^{(i+1,k)}  &=  Immigration(\{x_r^{(i,j)}\}) \\
% 	 	&=\{e_1^{(i+1,k)},  ... ,e_{n}^{(i+1,k)}\}
% 	 \end{align}
	 

\subsection{Substitute model}
\label{subsection: substitute}
{{\ouralgorithm} only knows the binary decision of its target model, making it hard to accurately evaluate individuals. To overcome this challenge, we design a novel substitute model to simulate the target model, and provide {\ouralgorithm} with approximate class probabilities.}

The inputs of our substitute model are \textit{function-level} FCGs generated according to the perturbation produced by the generator. 
% Here our algorithm does not choose FCGs as inputs because the type of FCGs used by the target model is actually unknown in the scenario of incomplete feature information. 
% To fasten its training, the substitute model should efficiently and sufficiently extract features from its inputs. 
{We use a GCN  {(i.e., Graph Convolutional Network)} to extract features from the substitute model}, as shown in the green block in  Fig. \ref{fig:bg1}. GCNs extend convolution to graph data, and they are good at utilizing structural information and node information to fulfill graph-related machine learning tasks. However, the main obstacle of applying GCNs to our task is the absence of node property. That is, FCGs do not provide property information for their nodes. To alleviate this problem, we propose to use \textbf{out degree} and \textbf{in degree} of a node as its features.
% Accordingly, we can build a substitute model that leverages a GCN to extract features from FCGs and uses a fully-connected neural network for classification.


Now we briefly explain {how to use a GCN to extract} features from the inputs. The GCN has multiple convolutional layers. Each layer aggregates node properties using a propagation rule, and the aggregated features are then processed by the next layer. {Accordingly, we can obtain a feature vector to represent the FCG using iterative computation. 
% The details can be found in Appendix \ref{appendix: gcn}}. 


% We use an adjacent matrix $\textbf{A}$ and a node feature matrix $\textbf{X}$ to characterize every FCG inputted to our substitute model. Here the adjacent matrix reflects the topology information of a FCG, and the node feature matrix provides the node property information. We use $\mathbf{H}^{l}$ ($0\leq l \leq L$) to refer to the features obtained at layer $l$. When $l=0$ holds, $\mathbf{H}^{l}$ reduces to the node feature matrix. 
% The features obtained at layer $l+1$ are derived from the features at layer $l$ by  

% \begin{equation}\label{eq:gcn_propgation_rule}
% 	\mathbf{H}^{(l+1)} = \sigma (\tilde{\mathbf{D}}^{-\frac{1}{2}} \tilde{\mathbf{A}} \tilde{\mathbf{D}}^{-\frac{1}{2}} \mathbf{H}^{l}\mathbf{W}^{l}),
% \end{equation}
% where $\sigma (\cdot)$ is the activation function, $\tilde{\mathbf{A}}=\mathbf{A}+\mathbf{I}$, $\mathbf{I}$ represents an Identity Matrix, $\tilde{\mathbf{D}}$ is the degree matrix of $\tilde{\mathbf{A}}$ and $\mathbf{W}^{l}$ denotes the weights of layer $l$. 

% To well train the substitute model, we design a loss function defined as

% \begin{equation}\label{eq:loss}
% 	\mathcal{L}=-\mathbf{E}(\sum_{i} y_{i}^{\prime} \log \left(y_{i}\right))+ \lambda \sum_{i}|w_i^2|,
% \end{equation}
% where  $y_i^{\prime}$ is the true label labeled by the target model, $y_i$  is the prediction output, $w_i$ is the weights need to learn. In this loss function, the first term is binary-classification cross-entropy, and the second term is introduced to reduce overfitting.

\subsection{Algorithm design} 
{{\ouralgorithm} aims to conglutinate multi-population co-evolution mechanism and substitute model to cooperatively generate adversarial perturbations. Its main procedure is given in Algorithm 1.} In this algorithm, $F$ is the target classifier, $N$ is the maximum number of individuals in a population, and $r_ {max}$ denotes the maximum number of generations. {In every iteration, {\ouralgorithm} first randomly selects the target or the substitute model (lines 3-6),  calculates fitness for every individual (lines 8-12), then retains high-rate individuals based on fitness (line 13), and finally conducts immigration, crossover and mutation (line 14). In addition, the substitute model should be trained when it is selected, as denoted by lines 15-17. The individual with the highest fitness is outputted when {\ouralgorithm} terminates. Below we summarize the important considerations for {\ouralgorithm}}.
% , and provide theoretical analyses in Appendix \ref{appendix: theoretical analyses}. 
\begin{algorithm}[h]
	    \LinesNumbered
	\caption{The {\ouralgorithm} Algorithm}
	\label{algorithm ga}
	 \KwIn{The FCG $G$ of a given APP}%输入参数

% 	\textbf{Stage I: Preparation}

% 		 Given  a black-box classifier $F$;
		 
		 
		 
% 		 Set the maximum number of a population $N$;
		 
% 		 Set the maximum generation of the GA $r_ {max}$;
		 
    Population initialization;

	
% 	\textbf{Stage II: Training}
	
	\For{$r$ in $r_ {max}$ }{
		\uIf {$ (r-3)/r_ {max} > random(0, 1)$ }{
			Is\_substitute = 1;	}  
		\Else{Is\_substitute = 0;}

		\For{each $P^i$ }{
			\uIf {Is\_substitute = 1 }{Get fitness $T(x_r^{(i,j)})$ from substitute model;	}
			\Else{Get fitness $T(x_r^{(i,j)})$ from target model;}	
							
		 	Get  $L(x_r^{(i,j)})$ for every individuals;
		 
	     	Select top $N$ individuals according to  $T(x_r^{(i,j)})$ and $L(x_r^{(i,j)})$ in turn ; 
	     	
	     	Immigration(); Crossover();	Mutation();
	}
	
	\If {Is\_substitute = 1 }{
		Get the result from the target model $F(x_r^{(i,j)})$;
		
		Train substitute model with $F(x_r^{(i,j)})$ and $x_r^{(i,j)}$;}	
	
        Determine whether algorithm should terminates;
		
}
% 	\textbf{Stage III: Output}
	

   % \textbf{Output:}  The individual with highest fitness;

\end{algorithm}

% (1) \textit{How to prepare data?} \\
% \hspace*{0.5cm}Before {\ouralgorithm} starts, we select a malicious app, obtain its FCG, and take out its functions for creating candidate function calls. After {\ouralgorithm} starts, it uses the generator to generate candidate adversarial examples, employs the detection outcomes of target model to label them, and employs the labeled data to train the substitute model.  

\noindent (1) \textit{{How to implement co-evolution in {\ouralgorithm}?}}\\
{ The co-evolution in {\ouralgorithm} is two-fold. On one hand, the generator and the discriminator cooperate with each other to improve the generated perturbation. On the other hand, multiple populations cooperatively evolve through immigration.}


% \footnote{{This is the first reason why {\ouralgorithm} has the advantage of co-evolution. Another reason is {\ouralgorithm} indeed benefits from the cooperation between generator and discriminator in the GAN model.} }

% \noindent (2) \textit{How to reduce the  {number} of  queries?} \\
% { The substitute model provides approximate class probabilities for adversarial perturbation generation, which helps reduce the number of queries sent to the target model.} 
% As shown in line $3$ of Algorithm 1, {\ouralgorithm} determines whether or not to use the substitute model based on a probability. This probability is small in the beginning, since the substitute model has not been sufficiently trained at the moment. 
% The probability then keeps arising from the fourth round of {\ouralgorithm}, making the substitute model frequently selected and hence significantly decreasing the number of queries. 

\noindent (2) \textit{How to avoid premature convergence?}\\
{ When the genes of some high-rate individuals quickly dominate the population \cite{1997Degree}, premature convergence occurs and evolutionary algorithms converge to a local optimum. {\ouralgorithm} can mitigate premature convergence owing to the cooperation among populations.} Through immigration, different populations share their good genes and further promote their evolution. Meanwhile, immigration also helps the populations jump out from local optimum traps. {Our  theoretical analyses are given in Appendix \ref{appendix: theoretical analyses}}.

\noindent (3) \textit{When to terminate our algorithm?}\\
There are three stopping criteria for {\ouralgorithm}. First, all the offsprings cannot induce misclassification on the target model anymore. Second, the perturbation amount does not decrease within several continuous rounds. Third, the maximum number of rounds is reached. 

\noindent (4)  {\textit{How to modify the APK according to the output?}\\
The output of our algorithm is the caller-callee function pairs. According to the output, we use the try-catch trap mentioned in Section \ref{sec:Adversarial manipulation design} to insert the callee function into the caller function, in order to implement adversarial perturbation. The  implementation details can be found in Appendix \ref{appendix:smali}.}

 

% \begin{algorithm}
% \caption{algorithm caption}%算法名字
% \LinesNumbered %要求显示行号
% \KwIn{input parameters A, B, C}%输入参数
% \KwOut{output result}%输出
% some description\; %\;用于换行
% \For{condition}{
%   only if\;
%   \If{condition}{
%     1\;
%   }
% }
% \While{not at end of this document}{
%   if and else\;
%   \eIf{condition}{
%     1\;
%   }{
%     2\;
%   }
% }
% \ForEach{condition}{
%   \If{condition}{
%     1\;
%   }
% }
% \end{algorithm}
	
\section{Experiments}\label{SEC:EXP}

In this section, we conduct extensive experiments to evaluate {\ourtool} by answering the following research questions:\\
\textbf{RQ1: Effectiveness.} { Does {\ourtool} successfully attack  the SOTA Android malware detection methods? }\\
%How effective is {\ourtool} in attacking the SOTA Android malware detection methods? \  
%How efficient is {\ourtool} in attacks?\\
\textbf{RQ2: Evolution.} How do multiple populations in {\ouralgorithm} evolve?\\
\textbf{RQ3: Efficiency.} { Does the substitute model help to decrease
queries and improve attack efficiency? }\\
\textbf{RQ4: Overhead.} Is there a trade-off between manipulation overhead and attack success rate?\\
\textbf{RQ5: Resilience.} {Is {\ourtool} still effective when there exists concept drift or data imbalance?}\\
\textbf{RQ6: Functionality.} {Does our adversarial perturbation change the functionality of malware?}

%\subsection{Setting}\label{SEC:Setting}
\textbf{Datasets}.
{ Our dataset contains 21399 benign samples and 22975 malicious samples, which come from Androzoo\footnote{https://androzoo.uni.lu/}, Faldroid dataset \cite{DBLP:journals/tifs/Fan0LCTZL18} and Drebin dataset \cite{DBLP:conf/ndss/ArpSHGR14}. Every sample collected from  Androzoo is detected by VirusTotal \cite{VirusTotal}. Only when a sample is detected to be malicious by more than four antivirus systems, we label it as malware. The details of our dataset  are provided in Appendix \ref{appendix: dataset}.} 

Furthermore, our experiments adopt two configurations to evaluate {\ourtool}. According to the first configuration,  we use 10-fold cross-validation to train the target models. To evaluate the attack methods, we randomly choose $100$ malicious examples (not included in the training data of target model) that can be correctly classified by target models for evasive malware generation. For the second, we divide the dataset according to the years that Android apps emerge as discussed in Section \ref{subsection: resilience}. The newly-emerged malware samples are used for test, while the old data are used in training.


% (1) \textit{Training target models}.   {The dataset is randomly divided into 10 fold, and 10-fold cross-validation is used to train and evaluate the existing FCG based Android malware detection methods. Every model's detection performance is shown in Appendix \ref{appendix:performance}.}

% % We randomly select 7000 benign examples and 7000 malicious examples to train the existing FCG based Android malware detection methods.

% (2) \textit{Generating adversarial examples}. In our experiments, we randomly choose $100$ malicious examples  {(not be selected as  training data by target model)} that can be correctly classified by the target model from our dataset, and use them for adversarial example generation.

 {Finally, we also consider the scenarios of concept drift and data imbalance in Subsection \ref{subsection: resilience}. In the scenario of concept drift, 17685 samples (8,017 benign examples and 9,668 malicious examples) from Androzoo are grouped by production year (from 2016 to 2020) and used to train target models. In the scenario of data imbalance, we randomly disarrange samples and set the benign-malicious ratio to 10:1, following the experimental setting in \cite{277204}.     }

%Our dataset contains 9,668 benign examples and 8,017 malicious examples. All the data come from Androzoo\footnote{https://androzoo.uni.lu/}, a growing collection of Android applications collected from several sources including the official Google Play app market \cite{Allix:2016:ACM:2901739.2903508}. At present, AndroZoo has been widely used in Android malware detection \cite{DBLP:conf/ccs/ZhangZZDCZZY20}\cite{DBLP:journals/tsmc/YuanJLC21} and Android adversarial malware generation \cite{DBLP:conf/www/LiZYLGC21}. The details of the dateset used in our experiments are provided in Appendix \ref{appendix: dataset}. In our experiments, the dataset is used to train the target model and to evaluate attack performance of {\ourtool}.

\begin{table*}[!ht]
	\caption{ {Effectiveness of {\ourtool} towards MaMaDroid, APIGraph and GCN.}}
	\centering
	\scalebox{1}{
		\begin{tabular}{c|c|ccc|ccc|ccc}
			\hline
			\multicolumn{2}{c|}{\multirow{2}{*}{Classifier\textbackslash{}Level}} & \multicolumn{3}{c|}{Family} & \multicolumn{3}{c|}{Package} & \multicolumn{3}{c}{Class} \\ \cline{3-11} 
			\multicolumn{2}{c|}{}                                                       & ASR     & APR     & IR      & ASR     & APR     & IR       & ASR    & APR    & IR      \\ \hline
			\multirow{5}{*}{MaMaDroid}                      
			& \cellcolor{cyan!60!gray!10}RF                        & \cellcolor{cyan!60!gray!10}1.000    & \cellcolor{cyan!60!gray!10}0.021    & \cellcolor{cyan!60!gray!10}8.670    & \cellcolor{cyan!60!gray!10}1.000    & \cellcolor{cyan!60!gray!10}0.049    & \cellcolor{cyan!60!gray!10}13.640    & \cellcolor{cyan!60!gray!10} 1.000  & \cellcolor{cyan!60!gray!10}	0.083  & \cellcolor{cyan!60!gray!10}	12.490    \\   

			& DNN                       & 0.990    & 0.149    & 11.130    & 1.000    & 0.134    & 16.730    & 1.000   & 	0.153  & 15.907    \\  	

			& \cellcolor{cyan!60!gray!10}AB                        & \cellcolor{cyan!60!gray!10}1.000   & \cellcolor{cyan!60!gray!10}0.066    & \cellcolor{cyan!60!gray!10}10.270    & \cellcolor{cyan!60!gray!10}1.000    & \cellcolor{cyan!60!gray!10}0.072    & \cellcolor{cyan!60!gray!10}14.300    & \cellcolor{cyan!60!gray!10}1.000  & \cellcolor{cyan!60!gray!10}0.118  & \cellcolor{cyan!60!gray!10}15.460   \\     		

			& 1-NN                      & 1.000    & 0.031    & 7.000    & 1.000    & 0.109    & 11.630     &1.000   & 0.060  & 10.960    \\  	 	

			& \cellcolor{cyan!60!gray!10}3-NN                      & \cellcolor{cyan!60!gray!10}1.000   & \cellcolor{cyan!60!gray!10}0.037    & \cellcolor{cyan!60!gray!10}9.390    & \cellcolor{cyan!60!gray!10}1.000    & \cellcolor{cyan!60!gray!10}0.142    & \cellcolor{cyan!60!gray!10}13.380     & \cellcolor{cyan!60!gray!10} 1.000  & \cellcolor{cyan!60!gray!10}	0.072  & \cellcolor{cyan!60!gray!10}10.770   \\ \hline  	

			\multirow{5}{*}{APIGraph}                       
			& RF                        & 1.000    & 0.039    & 11.260    & 1.000    & 0.098    & 14.930     & 1.000  &0.040  & 	9.530    \\  	 

			& \cellcolor{cyan!60!gray!10}DNN                       & \cellcolor{cyan!60!gray!10}1.000    & \cellcolor{cyan!60!gray!10}0.132    & \cellcolor{cyan!60!gray!10}14.370    & \cellcolor{cyan!60!gray!10}1.000    & \cellcolor{cyan!60!gray!10}0.096    & \cellcolor{cyan!60!gray!10}18.630   &  \cellcolor{cyan!60!gray!10}	1.000      &    \cellcolor{cyan!60!gray!10}0.168 
	     &  \cellcolor{cyan!60!gray!10}12.566 
      \\  	 

			& AB                        & 1.000    & 0.093    & 14.510    & 0.990    & 0.131    & 18.350     &  1.000  & 0.067   &	12.250   \\ 	 

			& \cellcolor{cyan!60!gray!10}1-NN                      & \cellcolor{cyan!60!gray!10}1.000   & \cellcolor{cyan!60!gray!10}0.058    & \cellcolor{cyan!60!gray!10}11.190    & \cellcolor{cyan!60!gray!10}1.000   & \cellcolor{cyan!60!gray!10}0.089    & \cellcolor{cyan!60!gray!10}14.040     & \cellcolor{cyan!60!gray!10}1.000  & \cellcolor{cyan!60!gray!10}0.012  & \cellcolor{cyan!60!gray!10}6.910    \\ 		

			& 3-NN                      & 1.000   & 0.085    & 11.570    & 1.000   & 0.105    & 13.770     & 1.000   & 	0.019 & 7.780   \\ \hline  	

			\multirow{1}{*}{GCN} & \cellcolor{cyan!60!gray!10}DNN                        & \cellcolor{cyan!60!gray!10}1.000   & \cellcolor{cyan!60!gray!10}0.205  & \cellcolor{cyan!60!gray!10}11.610   & \cellcolor{cyan!60!gray!10}1.000  & \cellcolor{cyan!60!gray!10}0.104   & \cellcolor{cyan!60!gray!10}17.320   & \cellcolor{cyan!60!gray!10}-  & \cellcolor{cyan!60!gray!10}-  & \cellcolor{cyan!60!gray!10}-   \\  \hline
			
		\end{tabular}%
	}
	\label{tab:effectiveness}
\end{table*}


\textbf{Target Model}.
We choose {three SOTA malware detection methods (i.e., MaMadroid \cite{DBLP:conf/ndss/MaricontiOACRS17}, APIGraph \cite{DBLP:conf/ccs/ZhangZZDCZZY20} and GCN \cite{DBLP:journals/corr/abs-2009-05602})} as our target system. 
% Note that they do not consider the function-level FCG in order to avoid overly high input dimensionality. Hence we use class-level, family-level and package-level FCGs as the features in our experiments, respectively.
% \textcolor{green}{Due to its extremely high dimensionality, the function-level FCG is not suitable for these methods \cite{}, and thus are not considered in our experiments.}
In MaMadroid and APIGraph, we employ Random Forest (RF) \cite{Breiman2001Random}, AdaBoost (AB) \cite{2002Logistic}, 1-Nearest Neighbor (1-NN) \cite{1952Discriminatory}, 3-Nearest Neighbor (3-NN) and Dense Neural Network (DNN) as the target classifier, respectively. Similar to  \cite{DBLP:journals/corr/abs-2009-05602}, we use a two-layer DNN as the target classifier in the GCN-based method.
% To ensure these target classifiers work well, we evaluate their detection performance and show the results in Appendix \ref{appendix:performance}.


\textbf{Metric}.
We  use attack success rate (\textbf{ASR}), average perturbation ratio (\textbf{APR}), and the number of interaction rounds (\textbf{IR}) to evaluate {\ourtool}.
{
ASR corresponds to the ratio of the number of successfully generated AEs (denoted by $N_{success}$) to the number of malicious examples used for AE generation (denoted by $N_{total}$ ), i.e., $ASR=N_{success} / N_{total}$.
% \begin{equation}
% 	ASR=N_{success} / N_{total}
% \end{equation}
APR is the ratio of the number of added edges (denoted by $E_{added}$) to the total number of edges (denoted by $E_{total}$), i.e., $APR = E_{added} / E_{total}$. 
% \begin{equation}
% 	APR = E_{added} / E_{total}
% \end{equation}
}
IR is defined as the number of interactions between our attack model and the target model.

\subsection{RQ1: Effectiveness}\label{sec:RQ1}

\textbf{Experimental Setup.} To verify the attack effectiveness of {\ourtool}, we use {\ourtool} to attack the {32} target models\footnote{{ Our experiments use 2 traditional 
FCG-based feature extraction methods (MaMaDroid and APIgraph), 3 feature levels (class, family and package) and 5 target classifiers (RF, AB, etc.). Furthermore, 1 GCN feature extraction method is considered with 2 feature levels (family and package). Hence, there are $32= 2\times 3\times5+1\times2 $ classifiers.}} mentioned above, and calculate ASR, APR and IR on every target model. 

% {For illustration, we supply an instance of AE generated by {\ourtool} }in Appendix \ref{appendix:instance}.

{Furthermore, we also compare {\ourtool} with three attack methods, i.e., SRL \cite{DBLP:journals/corr/abs-2009-05602}, SRL\_N and Random Insertion (RI). To our knowledge, SRL is the SOTA malware AE generation method \footnote{{Note that SRL works on control flow graph instead of FCG. To apply SRL to the FCG based Android malware detection, we design a non-functional API list (instead of non-functional instruction list), which contains 17 non-functional APIs.}}.  Since SRL requires knowing the class probabilities outputted by the target model, we modify its reward function and create a variant of SRL (i.e., SRL\_N) that only relies on binary outputs. The RI attack method is also introduced from \cite{DBLP:journals/corr/abs-2009-05602}, and it randomly inserts non-functional functions. }


% First, we consider a state-of-the-art black-box attack algorithm SRL \cite{DBLP:journals/corr/abs-2009-05602}. SRL is proposed to  generate adversarial example for control flow graph \footnote{ {To make this method work well in function call graph based Android malware detection system, we design a  non-functional API list instead of non-functional instruction list. This list contains 17 non-functional APIs.}} based Android malware detection method. It uses  deep reinforcement learning algorithm to choose nops instruction and insert them into the APK file. 
% However, SRL needs the output probability of the sample to calculate the reward in it's algorithm, so we modify it's reward function\footnote{ {Only when an adversarial example  attacks successfully, the algorithm give a positive reward.}} and introduce a new method called SRL\_N which only rely on the binary output result to get reward. 
% Moreover, we design a Random Insertion(RI) attack which randomly insert the non-functional function mentioned in SRL.}
% When we begin the attack, we use $N$ malicious Android APKs as the seed to generate adversarial examples. Then $M$ adversarial examples are classified as benign by the target classifeir. The attack success rate can be calculate as $ASR = M/N$. ASR shows the attack effectiveness.
% The APR is the ratio of added edges to the total edges. We discover that with the growth of an APK's API calls, it needs more perturbation  to generate an adversarial example. So APR can better describe the perturbation generated by an attack than average perturbation (\textbf{AP}). 
% IR is defined as the interaction rounds between the {\ourtool} and black-box classifier. 




\begin{figure}[h]
	\centering
	\includegraphics[scale=0.7]{sota.pdf}
	\caption{{Comparison with SOTA methods.}}
	\label{fig:SOTA}
\end{figure}



\textbf{Results \& Analyses}.
{Table \ref{tab:effectiveness} reflects the attack performance of {\ourtool} on MaMaDroid, APIGraph and GCN under various feature granularities.}
{First, {\ourtool} achieves an average ASR of 99.9\% over 32 target models, hence confirming the effectiveness of {\ourtool}.} { Second, when attacking the family-granularity classifier, {\ourtool} achieves the lowest APR and IR (i.e., 0.071 and 10.936).} {This indicates that although the family-granularity FCG speeds up malware detection through reducing input complexity, it still improves the efficiency of {\ourtool} by reducing search space.} 

% First, {\ourtool} achieves an average ASR of 97.6\% over all 30 target models, hence confirming the effectiveness of {\ourtool}. Second, the DNN based target models exhibit stronger robustness against our attacks (average ASR is 92.0\%) than the other target models (average ASR is 99.0\%).
% It is because deep learning classifiers generally have better generalization capability and possess better decision boundaries. Therefore, it is more difficult to generate AEs to mislead deep learning classifiers. 


Fig. \ref{fig:SOTA} compares {\ourtool}, SRL, SRL\_N and RI with respect to ASR under various APRs. Not surprisingly, RI performs worst in our experiments due to its poor search strategy. SRL performs better than SRL\_N, because SRL has access to class probabilities, which is more valuable than binary decisions. It is worth noting that {\ourtool} still outperforms SRL (e.g., its ASR is 4\% higher when APR is 0.2), although {\ourtool} cannot utilize class probabilities. 
% The above results confirm the advantage of {\ourtool} in attack effectiveness. 
 {The above results confirm that, under a certain number of perturbations, our method generates a better combination of the added edges that are more deceptive to detectors, as compared to the other methods.}
%这里是审稿人5的Q5,同理,我不也不太想放实验,而且,我感觉这个better combination of the added edges不好理解,但是不提这个关键词,又怕那个审稿人不高兴。老师看看要不要换个词,比如说 the edges selected for insertion are more aggressive to the target model.本质上就是想说我们的方法效果更好,但是绕来绕去变复杂了。  

% under different APR of {\ourtool} and several SOTA methods. It can be observed that, although SRL and our method can achieve similar ASR (e.g., with 0.2 APR, our method achieves 0.81 average attack success rate, whihe SRL achieves 0.77), SRL requires the output probability to train it's model. If without this information, SRL\_N only  achieves 0.50  average attack success rate with 0.2 APR. This phenomenon proves that  our model can still train well and generate threatening adversarial examples even without the information of output probability.

% Moreover, it is usually more costly to generate adversarial examples for DNN based target models, since these target models often result in higher APR and IR than the others.  

\subsection{RQ2: Evolution }
\textbf{Experimental Setup.}
In this subsection, we use experiments to analyze the effects of multi-population co-evolution mechanism.
First, we want to show that this mechanism can overcome the challenge of unknown feature granularity. To this end, we compare our method with the single-population methods in attacking MaMaDroid. The single-population methods rely on one single population corresponding to class, package and family  level, denoted by {\ourtool}-C, {\ourtool}-P  and {\ourtool}-F, respectively. {We also randomly select a malware sample and take a close look at these methods' attack processes.}


 Second, we want to know whether the correct feature granularity is found by our method.  We then record the survival number of each population and analyze how these populations evolve. { In this experiment, we choose MaMaDroid with an RF classifier as our target model, and use family-level feature granularity in malware detection. } 
  


% In this subsection, we use experiments to analyze the effects of multi-population co-evolution mechanism. First, we want to show that multi-population co-evolution can overcome the challenge of unknown feature granularity. To this end, we use our method and the single-population methods to attack MaMaDroid. The single-population methods rely on one single population corresponding to class, package and family  level, denoted by {\ourtool}-C, {\ourtool}-P  and {\ourtool}-F, respectively. Second, we want to know whether the correct feature granularity is found by our method. 
%   {For this purpose, we record the average ASE of aboves methods. What's more, to have a  detail look of the process of above methods,  }
%   we randomly select a malware sample and use {\ourtool},  {\ourtool}-C, {\ourtool}-P  and {\ourtool}-F to manipulate it. We then record the survival number of each population and analyze how these populations evolve.In this experiment, we choose MaMaDroid with RF classifier as our target model. Furthermore, target models use package-level features for malware detection in all the experiments of this subsection. 

\begin{figure}[h]

	\centering
	\includegraphics[scale=0.55]{multi-alldata.pdf}
	\caption{ {Performance comparison with single-population.}}
	\label{fig:multi}
\end{figure}


\begin{figure*}[h]
	\centering
	\includegraphics[scale=0.55]{mult-P4.pdf}
	\caption{{Multi-population vs. single-population. }}
	\label{fig:exp_3}
\end{figure*}

\begin{figure}[h]
	\centering
	\includegraphics[scale=0.55]{living_number_change.pdf}
	\caption{{The changing trend of survival proportion.}}
	\label{fig:living_number_change}
\end{figure}










% \begin{table}[]
% 	\caption{  {Attack performace of Multi-population vs. single-population.}}
% 	\centering
% 	\scalebox{1}{
% \begin{tabular}{c|ccc}
% \hline
%  & ASR & APR & IR \\ \hline
% \cellcolor{cyan!60!gray!10}BagAmmo-C & \cellcolor{cyan!60!gray!10}0.770 & \cellcolor{cyan!60!gray!10}0.190 & \cellcolor{cyan!60!gray!10}13.232 \\
% BagAmmo-P & 0.800 & 0.206 & 12.361 \\
% \cellcolor{cyan!60!gray!10}BagAmmo-F & \cellcolor{cyan!60!gray!10}0.970 & \cellcolor{cyan!60!gray!10}0.155 & \cellcolor{cyan!60!gray!10}\textbf{9.082} \\ \hline
% BagAmmo & \textbf{0.990} & \textbf{0.149} & 11.130 \\ \hline
% \end{tabular}%
% }
% 	\label{tab:multi}
% \end{table}

\textbf{Results \& Analyses.} 
{ For comparison, we choose family-level feature granularity and evaluate {\ourtool} and single-population methods on \textbf{all}  test samples. The results are shown in Fig. \ref{fig:multi}. It can be seen that {\ourtool} performs best and achieves the highest ASR with the lowest APR. {\ourtool}-C and {\ourtool}-P perform worst since they use a false feature granularity. Surprisingly, {\ourtool} performs better than {\ourtool}-F (i.e., 2\% higher in ASR and 0.06 lower in APR). This is because the introduction of multiple populations helps to avoid premature convergence and approach a global optimum. However, it may result in more interactions with the target model. This accounts for why {\ourtool} has a higher IR than {\ourtool}-F.} 

% this also accounts for why {\ourtool} has the highest ASR with lowest APR. This phenomenon will be discussed in next experiment too. Besides, if an attacker choose a false granularity(i.e., {\ourtool}-P and  {\ourtool}-C in  Fig.  \ref{fig:multi}) to attack the target model, it will lose a lot attack success rate(e.g., 22\% in {\ourtool}-C and 19\% in {\ourtool}-P) and  increase a lot perturbation ratio(e.g., 0.41 in {\ourtool}-C and 0.57 in {\ourtool}-P). }

{Now we randomly select a malware sample, and use it to generate an AE to attack 5 classifiers under family-level feature granularity. The attack processes of all methods are depicted in Fig. \ref{fig:exp_3}.} In this figure, the vertical axis represents the perturbation ratio of all methods, and the horizontal axis shows the IR values. {If a curve exhibits an evident decreasing trend and falls below a low threshold, we can conclude that the corresponding method succeeds in generating an AE and defeating the target model. As for those curves keeping horizontal (e.g., the green curve in the first subfigure), the corresponding methods fail to generate AEs. }
% Therefor, we can evaluate the effect of multi-population based on the decease trend of perturbation ratio.
% Moreover, an anomalous decease trend often represents attack failure. For example, as shown in  Fig.\ref{fig:exp_3}-(1), the perturbation of {\ourtool}-P always maintains a maximum value. Our experimental results verify that the corresponding attack is unsuccessful.  
Fig.\ref{fig:exp_3} shows that the perturbation ratio of multi-population always has a satisfactory decreasing trend, hence confirming the effects of multi-population co-evolution.
Furthermore, using a single population may cause {premature convergence to a local optimum}, as indicated by Fig. \ref{fig:exp_3}-(1). However, {\ourtool} effectively mitigates this problem using  multiple populations. { Theoretical analyses are given in Appendix \ref{appendix: theoretical analyses}.}

Finally, we verify the multi-population  co-evolution method converges to the real feature granularity from a different perspective. We show the survival proportion (i.e., the ratio between the number of alive individuals and the total number of individuals) of different populations in Fig. \ref{fig:living_number_change}. At the beginning, the perturbations are randomly added, and the survival proportion of different populations are irregular. However, as the number of queries increases, the family population and class population gradually fall to a low level. Contrarily, the survival proportion of the population corresponding to the correct feature granularity (i.e., {family} level) gradually rises to a high level. This phenomenon also confirms the effects of multi-population co-evolution.


\subsection{RQ3: Efficiency}\label{sec:RQ2}
\textbf{Experimental Setup}. 
We conduct \textit{ablation studies} to verify the effects of the substitute model in decreasing queries and improving attack efficiency. For comparison, we remove the substitute model and guide the multi-population co-evolution algorithm only using the target model. This method is called {\ourtool}-Without-S. Then we use {\ourtool} and {\ourtool}-Without-S to manipulate the same APK file, {and compare their performance.} 
%Due to space limit, we only show the  experiments on MaMaDroid with family-level and package-level features.

% To be specific, we design three attack strategies by reserving a single population in {\ourtool} (i.e., family, package and class population), which are termed as {\ourtool}-F, {\ourtool}-P and {\ourtool}-C, respectively. We conduct experiments on the MaMaDroid  with  package feature granularity. 
\begin{figure*}[h]
	\centering
	\includegraphics[scale=0.6]{with_and_without2.pdf}
	\caption{{With substitute model vs. without substitute model.}}
	\label{fig:with_and_without}
\end{figure*}



% we modify our algorithm in single-population setting, and through comparing the rate of convergence  of these  algorithms with the multi-population algorithm to analyze the multi-population's influence on efficiency. To be specific, we only reserve a single population in {\ourtool} (i.e., family, package and class population), which are termed as {\ourtool}-F, {\ourtool}-P and {\ourtool}-C, respectively. We conduct experiments on the MaMaDroid  with  package feature granularity. 



\begin{figure}[h]
	\centering
	\includegraphics[scale=0.55]{withvswithout.pdf}
	\caption{{Can substitute model reduce queries?}}
	\label{fig:exp_2_alldata}
\end{figure}




% As show in Fig. \ref{fig:exp_3}, we conduct experiments among  MaMaDroid with 5 classifier on package granularity API call graph. In each subfigure, we compare {\ourtool} with the 3 single-populations algorithm (i.e., {\ourtool}-F, {\ourtool}-P and {\ourtool}-C), the horizontal axis represents the interactive rounds, and the vertical axis is the perturbations on an APK file. Through the perturbations' decrease speed, we can evaluate the effect of the muti-population. From Fig.\ref{fig:exp_3}, it can be observed that no matter  in which case (i.e, the type of the classifier), muti-population can gain the fastest  perturbations' decrease speed. What's more, the wrong granularity population may  cause the  slow convergence speed of the algorithm and even cause failure of the attack. For example, in some cases (e.g., {\ourtool}-F in RF classifier), the perturbation maintain a   maximum value, which means the attack is unsuccessful.  Finally, as the Fig. \ref{fig:exp_3}.(3) shows that, single population may converges prematurely to local optimum which is a common  phenomenon in genetic algorithm. However, {\ourtool} effectively relieve this  phenomenon through multi-population. We supply the theoretical proof in Sec????.

\textbf{Results \& Analyses}.
Substitute model's effects are shown in  Fig. \ref{fig:with_and_without}, where the solid and the dotted lines represent {\ourtool} and {\ourtool}-Without-S, respectively. {The vertical axis reflects perturbation ratio, and the horizontal axis indicates the number of queries}. It can be seen that in all cases, {\ourtool} always has a higher convergence speed. Moreover, {\ourtool} always requires fewer queries before the perturbation ratio is kept below a certain threshold (e.g., 0.1). 
% For example, to keep perturbation ratio below $0.1$, the amount of queries required by {\ourtool} is about $79$\% of that required by {\ourtool}-Without-S.
Note that the difference between two methods in the initial phase is relatively small. It is because the substitute model has not been well trained in this phase. However, after the substitute model is well trained with sufficient data\footnote{ {In general,  the training accuracy of the substitute model arises as
the number of iteration rounds increase. However, the increasing trend of training accuracy is not strictly monotonic, because the training data used in the iterations are different. }}, {\ourtool} performs more efficiently and exhibits its advantage.
%%%%%%%老师帮忙看看这里,我感觉审稿人5的意见4和5不太好放到论文里,感觉比较占空间,而且加上去也没有意义,虽然说放附录里也可以,但是附录已经很多内容了,我不知道是否应该加进去。所以审稿人5的Q4这里提了一下,而且这里我不知道其他人看不看得明白。。。。或者不要however后面的那一句话。



{Fig. \ref{fig:exp_2_alldata} compares {\ourtool } and {\ourtool }-Without-S in terms of IR. Its top-half part gives the results on the family-level FCG based MaMadroid, while the bottom-half part shows the results on the package-level FCG based MaMadroid. The horizontal axis indicates various classifiers (e.g., AB, RF and 1-NN). We can draw two conclusions from this figure. First, the package-level classifier is more difficult to attack. This is because package-level FCGs contain much more nodes than family-level FCGs, resulting in a larger search space for {\ourtool }. Second, using the substitute model reduces the number of queries in almost all cases and helps enhance  the attack efficiency.}



\subsection{RQ4:  {Manipulation Overhead}}
\textbf{Experimental Setup.}
Here we study the number of code modifications (i.e., manipulation overhead) required to generate a real evasive malware. We use experimental results to reflect the relationship between ASR and {the allowed perturbation ratio}. 
%More specifically, we use the cumulative distribution function (CDF)  to  compare the attack performance on different classifier. 
% \st{What's more, we compare the attack success rate on APIGraph and MaMaDroid under the same amount of code modifications. }


\textbf{Result \& Analysis.}
Experimental results are shown in Fig. \ref{fig:perturbation}. In this figure, the horizontal axis represents the allowed perturbation ratio, and the vertical axis gives the cumulative distribution function (CDF) of ASR. It can be observed that the ASR keeps rising with the increase of the allowed perturbation ratio. In practice, a larger perturbation ratio results in larger computational overhead for adversaries. Therefore, there exists a trade-off between manipulation overhead and attack success ratio. Moreover, 
% {Fig. \ref{fig:perturbation} reveals that the deep learning method has stronger robustness} against AE attacks than the traditional machine learning method. Finally,
 the 3-NN classifier is more robust than the 1-NN classifier. {It is because} the 3-NN classifier considers more data than 1-NN classifier when classifying a sample, which makes it distinguish benign and malicious apps easier. 

% \st{Now we turn to Fig. , which provides the comparison results between APIGraph and MaMaDroid. It can be observed that APIGraph is more easily to be beaten than MaMaDroid under the same condition. Our explanation is given below. APIGraph is an excellent method that can improve the classification performance of Android malware detection system, especially when concept drift occurs. APIGraph groups similar APIs into one cluster, hence reducing the number of nodes in FCG. As a result, the difficulty of attacking the classifiers enhanced by APIGraph is also decreased. This accounts for why APIGraph is more susceptible to our AE attacks. And this indicates there exists an interesting trade-off between accuracy and robustness in APIGraph.}

\begin{figure}[h]
	\centering
	\includegraphics[scale=0.4]{perturbation.pdf}
	\caption{{CDF of ASR under different perturbation ratios.}}
	\label{fig:perturbation}
\end{figure}

% \begin{figure}[h]
% 	\centering
% 	\includegraphics[scale=0.6]{APIGraph-mama.pdf}
% 	\caption{APIGraph vs. MaMaDroid.}
% 	\label{fig:APIGraph-mama}
% \end{figure}

\subsection{RQ5: Resilience}
\label{subsection: resilience}
\textbf{Experimental Setup.}
Concept drift \cite{263854} is often observed in the realistic applications of Android malware detection. 
% When a classifier trained with old training samples is used to classify newly-emerged samples, concept drift may take place and detection performance may heavily degrade. 
 {Concept drift undermines existing AE generation methods that insert APIs selected from a pre-determined white list, if the white list is not updated accordingly. Hence we want to know whether {\ourtool} is also susceptible to concept drift}. To this end, we use newly-emerged malware samples to generate AEs and attack the classifiers trained over old data. We divide the dataset into training sets and testing sets according to the years that Android app emerge. We construct four new datasets to evaluate {\ourtool} under concept drift, as depicted in Table \ref{tab:setting}. {The first row of Table \ref{tab:setting} is the year of training samples used for training target classifiers. The second row is the year of testing samples used for generating AEs. The third row is the accuracy of the classifier.}

{Data imbalance is another practical problem worthy of consideration\cite{277204,DBLP:conf/uss/PendleburyPJKC19}. Since malicious samples are more difficult to collect than benign samples, malware detection models are usually trained over imbalanced data. We want to know whether data imbalance negatively impacts the attack performance of {\ourtool}. Hence we evaluate {\ourtool} on the target model trained with imbalanced data (the benign-malicious ratio is 10:1). }

\begin{table}
	\caption{{The attack performance under concept drift.}}
	\renewcommand\arraystretch{1.5}
	\centering
	\scalebox{0.7}{
		\begin{tabular}{ccccc}
			\hline		
			Training set (year) & \cellcolor{cyan!60!gray!10}2016    & \cellcolor{cyan!60!gray!10}2016-2017 & \cellcolor{cyan!60!gray!10}2016-2018 & \cellcolor{cyan!60!gray!10}2016-2019 \\ 
			Testing set (year)  & 2017    & 2018      & 2019      & 2020      \\ 
			ACC          & 92.92\% & 94.08\% & 94.58\% & 95.47\% \\  \hline
			ASR          & \cellcolor{cyan!60!gray!10}100\%   & \cellcolor{cyan!60!gray!10}100\%   & \cellcolor{cyan!60!gray!10}100\%   & \cellcolor{cyan!60!gray!10}100\%  \\ \hline
		\end{tabular}%
		\label{tab:setting}
	}
\end{table}






\begin{table}[]
	\caption{{The APR on balanced and imbalanced data.}}
	\renewcommand\arraystretch{1.5}
	\centering
	\scalebox{0.7}{
\begin{tabular}{ccccccc}
\hline
Level & Case & AB & RF & 1NN & 3NN & Average \\ \hline
\multirow{2}{*}{Family} & \cellcolor{cyan!60!gray!10}Balance & \cellcolor{cyan!60!gray!10}0.066 & \cellcolor{cyan!60!gray!10}0.021 & \cellcolor{cyan!60!gray!10}0.031 & \cellcolor{cyan!60!gray!10}0.037 & \cellcolor{cyan!60!gray!10}0.039 \\
 & Imbalance & 0.050 & 0.024 & 0.021 & 0.021 & 0.029 \\ \hline
\multirow{2}{*}{Package} & \cellcolor{cyan!60!gray!10}Balance & \cellcolor{cyan!60!gray!10}0.072 & \cellcolor{cyan!60!gray!10}0.049 & \cellcolor{cyan!60!gray!10}0.109 & \cellcolor{cyan!60!gray!10}0.142 & \cellcolor{cyan!60!gray!10}0.093 \\
 & Imbalance & 0.041 & 0.040 & 0.094 & 0.105 & 0.070 \\ \hline
\end{tabular}%
}
\label{fig:imbalance}
\end{table}





% \begin{table}[H] 
% 	\caption{The partition of dataset.}
% 	\renewcommand\arraystretch{1.5}
% 	\centering
% 	\scalebox{0.7}{
% 		\begin{tabular}{ccccc}
% 			\hline		
% 			Training set (year) & \cellcolor{cyan!60!gray!10}2016    & \cellcolor{cyan!60!gray!10}2016-2017 & \cellcolor{cyan!60!gray!10}2016-2018 & \cellcolor{cyan!60!gray!10}2016-2019 \\ 
% 			Testing set (year)  & 2017    & 2018      & 2019      & 2020      \\ 
% 			Setting name       & \cellcolor{cyan!60!gray!10}DSY2017 & \cellcolor{cyan!60!gray!10}DSY2018   & \cellcolor{cyan!60!gray!10}DSY2019   & \cellcolor{cyan!60!gray!10}DSY2020   \\
% 			ACC          & 92.92\% & 94.08\% & 94.58\% & 95.47\% \\  \hline
% 			ASR          & \cellcolor{cyan!60!gray!10}100\%   & \cellcolor{cyan!60!gray!10}100\%   & \cellcolor{cyan!60!gray!10}100\%   & \cellcolor{cyan!60!gray!10}100\%  \\ \hline
% 		\end{tabular}%
% 		\label{tab:setting}
% 	}
% \end{table}



\textbf{Result \& Analysis.}
%We use the old dataset for model training, and use the samples in the next year for model testing. 
{In the experiments on concept drift, we use {\ourtool} to manipulate the test samples, and the results are used to attack the MaMaDroid model with the family-level feature and the classifier of RF}. The ASR of {\ourtool} in every scenario is presented in the last row of Table \ref{tab:setting}. First, this table indicates that with more training samples, the accuracy of the target classifier becomes higher. No matter how high the accuracy is, however, {\ourtool} always achieves a perfect ASR of 100\%. %This phenomenon tells us that {\ourtool} performs well under concept drift. 
 {This shows that {\ourtool} performs well under concept drift and is still efficient when the malware detection models learn more with the new data.}
Note that {\ourtool} uses the functions coming from the malware itself (instead of a static function set).  {\ourtool} reduces the risk of using functions that become outdated due to concept drift. As a result, {\ourtool} poses a persistent threat to malware detectors. 
%这里要提对抗样本吗
     {Finally, we also discuss how {\ourtool} performs when the defender has the knowledge of adversarial example in Appendix \ref{app:ar}}


{Table \ref{fig:imbalance} shows the experimental results of {\ourtool} in the cases of balanced and imbalanced data. Our experiments demonstrate that the DNN model performs very poorly when trained with imbalanced data. Therefore, we do not choose DNN as our target model. 
%In our experiments, {\ourtool} achieves an attack success rate of 100\% in almost all cases. Hence we only show APR in Table \ref{fig:imbalance}.
 {In both cases (i.e., balanced dataset and imbalance dataset), {\ourtool} achieves an attack success rate (i.e., ASR) of 100\%. Here we only show the values of APR in Table \ref{fig:imbalance}.}
A higher APR means a more difficult attack task. It can be seen that in the vast majority of cases, {\ourtool} needs fewer perturbations (i.e., has a lower APR) to attack the target model trained with imbalanced data. That is, data imbalance does not bring troubles to {\ourtool}. This is because that training with imbalanced data makes the target model more likely to classify malware as benign apps. Accordingly, this reduces the degree of difficulty in generating AEs.
% modifying malware to mislead the classification model.


% In family granularity, balance scene less need  0.01 APR than imbalance scene, and  in package granularity, balance scene less need  0.017 APR than imbalance scene. The explanation is follow: 
% With an imbalanced dataset (i.e., malware is much less than benign sample), a classification model will be more likely tends to classify malware as benign apps. And this reduces the degree of difficulty in generating adversarial examples through
% modifying malware to mislead the classification model.

% \begin{table}[]
% 	\renewcommand\arraystretch{1.5}
% 	\caption{The partition of dataset.}
% 	\centering
% 	\label{tab:concept drift}
% 	\scalebox{1}{
% 		\begin{tabular}{ccccc}
% 			\hline			
% 			Setting name & \cellcolor{cyan!60!gray!10}DSY2017 & \cellcolor{cyan!60!gray!10}DSY2018 & \cellcolor{cyan!60!gray!10}DSY2019 & \cellcolor{cyan!60!gray!10}DSY2020 \\
% 			ACC          & 92.92\% & 94.08\% & 94.58\% & 95.47\% \\
% 			ASR          & \cellcolor{cyan!60!gray!10}100\%   & \cellcolor{cyan!60!gray!10}100\%   & \cellcolor{cyan!60!gray!10}100\%   & \cellcolor{cyan!60!gray!10}100\%  \\
% 			\hline		
% 		\end{tabular}%
% 	}
% \end{table}


\subsection{{RQ6: Functionality} }
{\textbf{Experimental Setup.}
In this section, we first use static analysis to verify whether the perturbations generated by {\ourtool} are successfully imposed on malware. We then employ dynamic analysis to check whether the perturbation changes the functionality of the malware.}

{\textbf{Result \& Analysis.}
To know whether our perturbations are injected, we add a unique log statement when a perturbation (i.e., a try-catch trap) is injected. This log statement helps us to find the perturbation in the smali file. We then check if the found function calls in the smali file coincide with the perturbations generated by {\ourtool}. In our experiments, we evaluate 50 APK files, and we realize that all the generated perturbations are correctly injected into the smali file.
}

{ In our experiments of dynamic analysis, we first install and run 50 pairs of original and perturbed malware samples in Android Virtual Device (AVD). It is observed that every malware pair performs the same and has the same run-time UI. For further analysis, we insert three log statements, denoted by LOG1, LOG2 and LOG3,  into every try-catch block to record execution information. LOG1 is in front of the runtime exception, LOG2 is in front of the inserted function, and LOG3 is at the beginning of the catch block. We analyze 50 APK files aided by  Android Studio's log analysis tool (i.e., LogCat). We realize that either LOG1 or LOG3 of every APK file is normally executed, but no LOG2 is executed. This phenomenon means that all manipulated malware samples run properly, and the inserted functions are not invoked, hence posing no impact on the malware functionality.
}

%  {
% What's more, we use dynamic analysis to show that the functionality of a malware is not be influenced. First, we install and run the pairs of original and perturbed malware samples in
% Android Virtual Device (AVD). We  evaluate 50 pairs of APPs and every malware pair performs same and has the same run-time UI. Second, we  insert  log statement  into every try block to record execution information and use it to further confirm that the perturbed malware works the same as
% the original one. To be specific, we insert three log statements for every try-catch block, \textbf{LOG1} is before the runtime exception, \textbf{LOG2} is behind the  runtime exception and \textbf{LOG3} is in the beginning of the catch block. With the help of the  android Studio's log analysis tool to track the performance of every log statements.  We aslo evaluate 50 APPs, and all the LOG1 and LOG3  are normally  executed, and LOG2 never be executed. It means that all apK runs properly,and the inserted callee functions are not executed and do not affect the functionality of the original APK.}





	\section{Related Work}
\label{sec:Related}
% {Existing adversarial attacks against Android malware detection include syntax feature oriented attacks and semantic feature oriented attacks.} 

% The former take syntax feature (e.g., permission, intent action and API calls) based classifier as their attack target, while the latter takes semantic features (e.g., function call graph) based classifiers as their attack target.

%\textbf{Syntax feature oriented AE attacks.}
{ Recently, adversarial attacks have been widely used in various fields, i.e., image classification \cite{ijcai2022p554,xia2022enhancing}, traffic analysis \cite{DBLP:conf/uss/NasrBH21,DBLP:journals/tifs/RahmanIMW21}, autonomous driving \cite{DBLP:conf/uss/Jing0D0L0NW21,DBLP:journals/corr/abs-2201-06192} and object detection \cite{DBLP:conf/uss/LovisottoTSSM21}. As for Android malware detection,}
there have been many studies \cite{huang2018adversarial,grosse2017adversarial,hu2017generating,2019lh,DBLP:journals/tifs/LiL20} on syntax features oriented AE generation. Huang \emph{et al.} \cite{huang2018adversarial}  use the saddle-point optimization formulation to generate adversarial examples in the discrete (e.g., binary) domain for malware detection. Grosse  \emph{et al.} \cite{grosse2017adversarial}  expand existing AE generation algorithms to construct a highly effective attack against malware detection models. In \cite{hu2017generating,2019lh}, Hu \emph{et al.} utilize a GAN to generate adversarial examples in black-box mode for malware detection. Li  \emph{et al.} \cite{ DBLP:journals/tifs/LiL20} propose { an ensemble approach that allows} attackers to perturb a malware example via multiple attack methods and multiple manipulation sets.

%In practice, the syntax feature based detection methods are susceptible to evasion and may 
%However, these works are only applicable to the Android malware detectors that adopt syntactic features and are difficult to be applied to semantic features-based Android malware detection methods.
%\textbf{Semantic feature oriented AE attacks.}
To achieve higher detection accuracy, more and more {Android malware detection methods} \cite{DBLP:conf/ccs/ZhangZZDCZZY20,DBLP:conf/ndss/MaricontiOACRS17,DBLP:conf/kbse/WuLZYZ019} focus on semantic features. Chen  \emph{et al.} \cite{chen2020android} introduce two AE generations methods in image classification to detect Android malware, and propose a method applying optimal perturbations onto Android APKs. Their method directly perturbs features in feature space. Pierazzi  \emph{et al.} \cite{DBLP:conf/sp/PierazziPCC20}
extract slices of bytecode (i.e., gadgets) from benign APKs and inject them into a malicious APK to generate adversarial malware. { Zhang \emph{et al.} \cite{DBLP:journals/corr/abs-2009-05602} propose a reinforcement learning based attack to deceive graph feature based malware detection models.  Recently, Bostani \emph{et al.} \cite{DBLP:journals/corr/abs-2110-03301} propose an interesting black-box attack EvadeDroid without requiring the knowledge about feature space. Different from {\ourtool}, EvadeDroid employs random search to find the desired perturbation from the code of benign apps.}



% In summary, the study of semantic feature oriented AE attacks is just beginning. {The open problems remain yet to solve include (but are not limited to) 1) how to manipulate malware codes with functionality preservation and 2) how to find desired adversarial perturbation with limited information}. 





	%\section{Limitations and Future Directions}
%\label{sec:Limitations}
% \subsection{Non-strict black box attack}
% There is a other attack scene  which allows the target black-box classifier give the classification probability instead of the discrete prediction label \cite{DBLP:conf/icml/DaiLTHWZS18,DBLP:journals/corr/abs-2007-02734}. 
% In this case, the classifier can offer more information for {\ourtool} to converge more quickly. To show the effectiveness of our algorithm in this case, we use BagAmmo to attack a same  APK in these two settings(i.e., output the label and output the probability). And we record the perturbation in each iteration to show the convergence speed. We conduct experiments on the MaMaDroid method with DNN classifier and family/package   granularity feature.
% 数据上的迁移性,模型上的迁移性。感觉不够直观。。本质上两者都是为了减少查询量。
%   \textbf{Near-universal perturbations}. % Reduce computational cost   AEs' transferability among data
%Given a large-size APK file, {\ourtool} may suffer relatively high overheads  {in searching desirable perturbations. To further increase efficiency and decrease overhead, we will study how to more quickly find the desired perturbations in the future}. We borrow ideas from existing works on universal perturbation in the domain of image classification such as \cite{DBLP:journals/corr/Moosavi-Dezfooli16,DBLP:journals/corr/abs-2005-08087}. A universal perturbation can be applied to various inputs and induce misclassification. {In practice, it may be extremely hard to find absolutely universal perturbations for all Android malware. As a result, we choose} to obtain near-universal perturbations for Android malware. Then we consider these near-universal perturbations as the starting point of our algorithm, avoid searching from scratch, and hence accelerate algorithm convergence.


% BagAmmo suffers from computational consumption when an APK has a huge FCG.  {\ourtool} will analyze  the FCG and extract the candidate caller  functions and callee functions. If the given APK has a huge function call graph, the candidate solution space will be huge and  consume a lot computational memory. So even BagAmmo offers a convenient to the automated adversarial generation, in exchange, it costs a lot computational consumption. In the feature, to increase the attack efficiency and decrease the  computational consumption. We will study how to cut the candidate  perturbations and Keep the more threatening perturbations. For example, there are some works \cite{DBLP:journals/corr/Moosavi-Dezfooli16,DBLP:journals/corr/abs-2005-08087} study how to generate the universe perturbation which can be applied to various input and lead the misclassification of the model. { So a possible solution is to explore  AEs' transferability among different data and to generate a universe perturbation which can suitable most of the data.}

% \noindent \textbf{Transferring to other targets}. % Further Reduce the number of queries AEs' transferability among classifiers
% % BagAmmo employs a substitute model (i.e., its discriminator) to reduce the number of required queries. 
% {{\ourtool} needs to interact with its target system during attacks. In practice, the transferable attacks are more attractive since they do not require queries}. Recently, some works in the domain of image classification \cite{DBLP:conf/iclr/LiuCLS17,DBLP:conf/cvpr/WuSLK21} study how to generate AEs without requiring queries. These works are motivated by a phenomenon that AEs crafted for one model can also attack other models. {In the future, we will investigate how to leverage the transferability of AEs to attack various target models.}

%\noindent {\textbf{Transfer to other platforms}. The idea and the framework of {\ourtool} are transferable to a certain degree, since malware detection on other platforms can also use semantic features (e.g., FCGs). {\ourtool} should be tailored to the different programming languages used on these platforms. } 

% However, there are still two steps need to be accomplished. One is to change the script of the  try-catch trap into another programming language. The other is to  redefine the individual in {\ourtool} according to the corresponding platform.  }

% it still need to query the target black-box classifier.; And frequency query may cause the alarm of the model owner. So how to generate a adversarial example without the iteration with the target classifier is an important problem in practical application.   Recently, many works \cite{DBLP:conf/iclr/LiuCLS17,DBLP:conf/cvpr/WuSLK21} study how to generate adversarial examples without queries. These works are motivated by a phenomenon that adversarial examples crafted for one model can also attack other models. We  can leverage the transferability of adversarial examples to launch attacks without requiring any interaction with target Android malware detection models.




% \section{ {Limitations and Discussion}}
% \label{sec:Limitations}
% %这里我把结论那一章节去掉了。
%  {In this paper, we propose an effective black-box AE attack {\ourtool} towards the FCG based Android malware detection. {\ourtool} may raise the concern for AE threats on malware detection, and can be used to evaluate the robustness of existing detection
% methods. The limitations and future works are given below.}

% %我不知道为什么罗老师在回复里去掉了这句话,这里需要老师把握下。
%  {\textbf{Potential defenses}. 
% {\ourtool} targets static analysis methods and relies on inserting function calls to change FCG. It does not change the information flow of malware, and hence does not negatively impact dynamic analysis \cite{2017Malton}. We will explore how to construct AEs against dynamic analysis in future work. Another concern is whether a defender can detect {\ourtool} by counting the number of try/catch blocks. Since the number of added try-catch blocks is relatively small, however, it is difficult to find an appropriate detection threshold (i.e., the maximum allowed number of try-catch blocks). More discussions can be found in Appendix \ref{app:trycatch}}. 
 
% \noindent {\textbf{Transfer to other platforms}. The idea and the framework of {\ourtool} are transferable to a certain degree, since malware detection on other platforms can also use semantic features (e.g., FCGs). {\ourtool} should be tailored to the different programming languages used on these platforms. } 

 

% Reduce computational cost   AEs' transferability among data
%Given a large-size APK file, {\ourtool} may suffer relatively high overheads  {in searching desirable perturbations. To further increase efficiency and decrease overhead, we will study how to more quickly find the desired perturbations in the future}. We borrow ideas from existing works on universal perturbation in the domain of image classification such as \cite{DBLP:journals/corr/Moosavi-Dezfooli16,DBLP:journals/corr/abs-2005-08087}. A universal perturbation can be applied to various inputs and induce misclassification. {In practice, it may be extremely hard to find absolutely universal perturbations for all Android malware. As a result, we choose} to obtain near-universal perturbations for Android malware. Then we consider these near-universal perturbations as the starting point of our algorithm, avoid searching from scratch, and hence accelerate algorithm convergence.\textbf{}


\section{ {Limitations and Discussion}}
\label{sec:Limitations}
%这里我把结论那一章节去掉了。
 {In this paper, we propose a black-box AE attack BagAmmo towards the FCG based Android malware detection. We hope that our work has reference value for the
study of Android malware detection, and raises the concern for the threats posed by AE attacks. Moreover, our method can be used to evaluate the robustness of existing Android malware detection
methods. Below we discuss some limitations and future works.}

%我不知道为什么罗老师在回复里去掉了这句话,这里需要老师把握下。
 {\textbf{Dynamic analysis based defense}. 
Our method targets  the static analyse methods.
 It relies on inserting function calls to change FCG. But it does not change the information flow of malware. Therefore, it does not negatively impact dynamic analysis \cite{2017Malton}. We will explore how to construct adversarial examples against dynamic analysis based Android malware detection methods in future work.}

 % {\textbf{Transfer to other platforms}. The idea and the framework of {\ourtool} are transferable to a certain degree, since malware detection on other platforms can also use semantic features (e.g., FCGs). {\ourtool} should be tailored to the different programming languages used on these platforms. } 
 
\textbf{Transfer to other domains}. The idea and the framework of {\ourtool} are transferable to a certain degree, since many domains  use semantic features and graph structured data (e.g., intrusion detection system \cite{DBLP:journals/iotj/ZhouLLYSW22} and trajectory prediction system \cite{wong2022view,xia2022cscnet}).
% {\ourtool} should be tailored to the different programming languages used on these platforms. }


  {\textbf{Try/catch detection based defense}. Another concern is that whether a defender can detect the AEs generated by {\ourtool} by counting the number of try/catch blocks. This defense method requires a detection threshold for the number of try-catch blocks. Through comparing the try-catch block number of original malicious APKs and that of adversarially perturbed APKs, however, we find that the number of try-catch blocks added by our method is relatively small. So it is difficult to find an appropriate threshold for all APKs. Without such a threshold, this defense method may cause a high false positive or false negative rate.\footnote{ {More experimental results can be found in Appendix \ref{app:trycatch}}}} 

  \section{ Acknowledgments
}
This work was supported partially by the  Hong Kong RGC Project  (No.  PolyU15219319), HKPolyU Grant No.ZVG0,   Fundamental Research Funds for the Central Universities (HUST: Grant No. YCJJ202202016 and  2022JYCXJJ035) .
		% \section{Conclusion}\label{sec:Conclusion}
% Under the realistic assumption on incomplete feature information, we design a novel black-box AE attack {\ourtool} against the graph based Android malware detection, called {\ourtool}. {\ourtool} uses a novel GAN model to automatically generate real evasive malware. Its generator is implemented with a multi-population co-evolution algorithm. Its discriminator acts as a substitute model and stimulates the generator to improve its generated malware. We provide theoretical analyses for the proposed algorithm. To evaluate {\ourtool}, we implement two SOTA graph based Android malware detection methods  MaMaDroid and APIGraph, use three feature granularities and five classifiers (e.g., RF and DNN), and build 30 different target detection systems. Extensive experiments on these target models confirm the advantages of {\ourtool} in attack effectiveness, attack efficiency and resilience to concept drift. 

% {In this paper, we propose a black-box AE attack {\ourtool} towards the FCG based Android malware detection. {\ourtool} works without knowing the feature granularity of its target detection system. {\ourtool} leverages a multi-population co-evolution algorithm to find the desired perturbation, and uses a new technique, i.e., try-catch trap, to manipulate malware. Extensive experiments confirm the effectiveness, efficiency and resilience of {\ourtool}, and demonstrate the advantage of {\ourtool} over the SOTA attack SRL. }

% {
% We hope that our work possesses reference value to the study of Android malware detection, and raises concern of malware detection research community for the threats posed by AE attacks. Moreover, our proposed method can be used to evaluate the robustness of  existing Android malware detection methods.
% }

%  {Recently, more and more works focus on the adversarial examples generation on graph data. Different from the image data consisting of continuous
% features, graph data  and the nodes features are often discrete. Therefore, gradient based approaches  for finding perturbations are not suited \cite{DBLP:conf/kdd/ZugnerAG18}. What's more,  it's easy to perturb image data by increasing or decreasing pixel values. However, it's difficult to perturb graph data( e.g.,  edge or node usually has  actual physical meaning and can't be modified  arbitrarily ). \cite{DBLP:conf/kdd/ZugnerAG18} propose the first work of adversarial attack on attributed graphs. By applying perturbations on  graph structure (add or subtract edges) or node attributes (add, subtract attributes), it can successfully attack node classification model.
% Develop up to now, adversarial example generation methods can be divided into two categories: gradient-based and non-gradient-based according to the  knowledge level  of the attacker. 
% The gradient-based method usually used in white-box attack, and adversarial examples generated   along the direction of gradient descent of loss function \cite{DBLP:conf/ijcai/XuC0CWHL19,DBLP:conf/icml/BojchevskiG19,DBLP:conf/ijcai/Wu0TDLZ19,DBLP:conf/iclr/ZugnerG19}. 
% The non-gradient-based attack usually applies to black box attack scenarios, and it attackers usually use heuristics method to solve the perturbation \cite{DBLP:conf/icml/DaiLTHWZS18,DBLP:conf/kdd/0001WDWT21,DBLP:conf/www/SunWTHH20} .} 

%  {
% Although these methods offer a lot  inspiration to adversarial examples generation in Android malware detection. But there's still a long way to go. 
% The main challenge of adversarial example attacks for Android malware detection stems from the fact that the problem space and the feature space of Android malware detection are different and separated. Attackers need to modify the normal samples (i.e., malicious apps) in problem space (instead of feature space) under some strict and practical constraints (e.g., R1-R4 proposed in our manuscript). The problem-feature separation and the consideration for practical constrains make our problem different from the problem of graph classification oriented adversarial example attacks. The existing adversarial example attacks developed for graph classification tasks cannot be directly transferred to Android malware detection, since they are launched in feature space and their perturbations may be impossible to realize in problem space.
% }


% \cite{DBLP:conf/icml/DaiLTHWZS18} proposes a reinforcement learning based attack method that learns the generalizable attack policy in  black-box attack (i.e., only requiring prediction labels from the target classifier.)  


\bibliographystyle{unsrt}

\bibliography{ref}{}

	
\section{Appendix}

\subsection{The limitations in callee selection} \label{sec:callees' limitation}
As shown in Section \ref{sec:adversarial manipulation method}, we choose the leaf nodes as candidate callees. However, not all leaf nodes can be chosen as callees. There exist two limitations:
\begin{itemize}
\item \textit{Access modifier}. Some leaf-node functions are not allowed to be invoked at all. Therefore, we only consider those leaf-node functions whose access modifier is \textit{public}. 
\item \textit{Parameter type}. The arguments of some leaf-node functions are the instances of classes. Under this situation, invoking these functions will incur instantiating a class, hence generating an unintended edge. To avoid this problem, we propose to choose the leaf-node functions whose arguments are void or belong to the category of primitive data types (e.g., int and short) and the \text{String} class. 
\end{itemize}
 
 
% \subsection{Examples of invocation types in smali} \label{appendix: smali example}
% As shown in Section \ref{sec:adversarial manipulation method}, different invocation types require different smali manipulation. There are 5 invocation types in smali code, i.e., \textit{invoke-direct}, \textit{invoke-virtual}, \textit{invoke-static}, \textit{invoke-super} and  \textit{invoke-interface}. Fig. \ref{fig:direct} shows an example for every invocation type.  
% 	\begin{figure}[htbp]
% 		\centering
% 		\includegraphics[scale=0.6]{direct.pdf}
% 		\caption{The examples of smali code with different invocation types.}
% 		\label{fig:direct}
% 	\end{figure}

% \subsection{{Extracting features from an FCG}}
% \label{appendix: gcn}
% {
% As mentioned in Subsection \ref{subsection: substitute}, we employ a GCN model to extract features from an input, i.e., an FCG. The details are given below. We use an adjacent matrix $\textbf{A}$ and a node feature matrix $\textbf{X}$ to characterize every FCG inputted to our substitute model. Here the adjacent matrix reflects the topology information of an FCG, and the node feature matrix provides the node property information. We use $\mathbf{H}^{l}$ ($0\leq l \leq L$) to refer to the features obtained at layer $l$. When $l=0$ holds, $\mathbf{H}^{l}$ reduces to the node feature matrix. 
% The features obtained at layer $l+1$ are derived from the features at layer $l$ by  
% \begin{equation}\label{eq:gcn_propgation_rule}
% 	\mathbf{H}^{(l+1)} = \sigma (\tilde{\mathbf{D}}^{-\frac{1}{2}} \tilde{\mathbf{A}} \tilde{\mathbf{D}}^{-\frac{1}{2}} \mathbf{H}^{l}\mathbf{W}^{l}),
% \end{equation}
% where $\sigma (\cdot)$ is the activation function, $\tilde{\mathbf{A}}=\mathbf{A}+\mathbf{I}$, $\mathbf{I}$ represents an Identity Matrix, $\tilde{\mathbf{D}}$ is the degree matrix of $\tilde{\mathbf{A}}$ and $\mathbf{W}^{l}$ denotes the weights of layer $l$. }

% {To well train the substitute model, we design a loss function defined as
% \begin{equation}\label{eq:loss}
% 	\mathcal{L}=-\mathbf{E}(\sum_{i} y_{i}^{\prime} \log \left(y_{i}\right))+ \lambda \sum_{i}|w_i^2|,
% \end{equation}
% where  $y_i^{\prime}$ is the true label labeled by the target model, $y_i$  is the prediction output, $w_i$ is the weights that need to learn. In this loss function, the first term is binary-classification cross-entropy, and the second term is introduced to reduce overfitting.}


\subsection{Theoretical Analyses for our method} \label{appendix: theoretical analyses}

% \textcolor{red}{
Our method {\ourtool} utilizes the algorithm {\ouralgorithm} to find the desired perturbation for a given malware sample. Since {\ouralgorithm} is an evolutionary algorithm, how to mitigate premature convergence is an important issue. Here, premature convergence or prematurity is a common phenomenon that leads an evolutionary algorithm to converge quickly to a local optimum. For evolutionary algorithms, prematurity is often caused by the lack of gene diversity.

% there exists one important issue for {\ouralgorithm} that is  the mitigation of premature convergence. }


%However, {\ouralgorithm} effectively mitigates this problem by introducing multiple populations and immigration among populations. 


% \subsubsection{The convergence of {\ouralgorithm}}

% We assume two individuals $x^i$ and $x^j$. Then we define $\delta(\cdot)$ as a metric function that defines the distance between two individuals, i.e., 

% \begin{equation}
% 	\resizebox{.8\hsize}{!}{
% 		$\delta\left(x^i, x^j\right)= \begin{cases}0 &x^i=x^j \\ \left|1+M-F\left(x^i\right)\right|+\left|1+M-F\left(x^j\right)\right| & x^i \neq x^j\end{cases}$
% 	}
% \end{equation}
% where   $F$ is the fitness function and $M$ is the upper bound  of the $F$.


% %	 Note that, the $F$ can be further more defined as the sum of the  fitness in different populations (because the immigration operate, the individual will interact between different populations).

% %	\begin{equation}
% %		F = \sum_{m=1}^l{F_m}
% %	\end{equation}
% %	where l is the number of populations.

% We define the generator as a function $g$ and use $r$ to represent the evolutionary generation in {\ouralgorithm}. According to the selection strategy of {\ouralgorithm},  we have $F(g(x^i_{r+1})) \geq F(g(x^i_{r}))$. It means the fitness of the surviving offsprings are always {equal to} or higher than the parents.

% So we have

% \begin{equation}
% 	\resizebox{.9\hsize}{!}{$
% 		\begin{aligned}
% 			\delta\left(g\left(x^i_r\right),g\left(x^j_r\right)\right)&=\left|1+M-F\left(g\left(x^i_r\right)\right)\right|+\left|1+M-F\left(g\left(x^j_r\right)\right)\right| \\
% 			&\leq\varepsilon \cdot\left(\left|1+M-F\left(x^i_r\right)\right|+\left|1+M-F\left(x^j_r\right)\right|\right) \\
% 			&=\varepsilon \cdot \delta\left(x^i_r, x^j_r\right)(0<\varepsilon<1)
% 		\end{aligned}$}
% \end{equation}


% According to the Banach compression mapping principle, {\ouralgorithm} converges on a single fixed point $x^{*}$, i.e., 
% \begin{equation}
% 	x^{*}=\lim _{r \rightarrow \infty} g^{r}(x_0)
% \end{equation}
% where $ r$ gives how many rounds of iteration have been taken by {\ouralgorithm}.


%According to above analyses, we can draw the conclusion that with sufficient iterations, {\ouralgorithm} will finally converge to a point.

% \subsubsection{ \textbf{Mitigating prematurity using multiple populations }}
 In the following, we analyze how multiple populations introduced in {\ouralgorithm} mitigate the problem of premature convergence.

Due to the introduction of multiple populations, there exist a local optimal solution in each {population}. We define this locally optimal solution as $x^{*}_{p}$, where $p=1,2,...,l$ is the index of the population.

Then, the individuals that can achieve  the local optimal solutions with the {\ouralgorithm} $G$ are termed as:
\begin{equation}
	A_{p}^{*}=\left\{x \in A: G(x)=x^{*}_{p}\right\}
\end{equation}
where $A$ is the solution space.

Then, the probability {that an} individual $x\in A$ {belongs} to set $A_{p}^{*}$ can be represented as $\theta_{p}=P\left(A_{p}^{*}\right)$. It is clear that $\theta_{p}>0$ for $p=1, \ldots, l$ and $\sum_{p=1}^{l} \theta_{p}=1$.


The size of the set  $A_{p}^{*}$ can be termed as $n_{p}$.
According to the definition, we have $n_{p} \geq 0\ (p=1, \ldots, l)$, the random vector $\left(N_{1}, \ldots, N_{l}\right)$ follows the multinomial distribution and $\sum_{p=1}^{l} N_{p}=N$.
\begin{equation}
	\operatorname{Pr}\left\{n_{1}=N_{1}, \ldots, n_{l}=N_{l}\right\}=\left(\begin{array}{c}
		N \\
		N_{1}, \ldots, N_{l}
	\end{array}\right) \theta_{1}^{N_{1}} \ldots \theta_{l}^{N_{l}}
\end{equation}
where
\begin{equation}
	\quad\left(\begin{array}{c}
		N \\
		N_{1}, \ldots, N_{l}
	\end{array}\right)=\frac{N !}{N_{1} ! \ldots N_{l} !}, \quad N_{p} \geq 0 \quad(p=1, \ldots, l)
\end{equation}

We define $W$ as the number of  locally  optimal solutions found by {\ouralgorithm}. Then the probability of   $l$ locally  optimal solutions being found can be termed  as %${Pr}\{W=l \mid \theta\}$:
\begin{equation}
	\operatorname{Pr}\{W=l \mid \theta\}=\sum_{N_{1}+\cdots+N_{l}=N} \left(\begin{array}{c}
		N \\
		N_{1}, \ldots, N_{l}
	\end{array}\right) \theta_{{1}}^{N_{1}} \ldots \theta_{{l}}^{N_{l}} 
\end{equation}
where
\begin{equation}
	\theta=\left(\theta_{1}, \ldots, \theta_{l}\right).
\end{equation}


For the sake of analyzing the limit, we define
\begin{equation}	
	\delta=\min \left\{\theta_{1}, \ldots, \theta_{l}\right\} \leq 1 / l
\end{equation}

Then we have
\begin{equation}
	\begin{aligned}
		\operatorname{Pr}\{W=l \mid \theta\} & \geq \sum_{N_{1}+\ldots+N_{l}=N}\left(\begin{array}{c}
			N \\
			N_{1}, \ldots, N_{l}
		\end{array}\right) \delta^{N} \\
		&=(\delta l)^{N} \operatorname{Pr}\left\{W=l \mid\left(\frac{1}{l}, \ldots, \frac{1}{l}\right)\right\}
	\end{aligned}
\end{equation}

For any $l$ and $\theta$, we can find the least evaluation number $n^*$ such that for any given $\gamma \in(0,1)$, we will have $\operatorname{Pr}\{W=l \mid \theta\} \geq \gamma$ 
for all $n \geq n^{*}$.
Finding $n^{*}=n^{*}(\gamma, \theta)$ is the problem of finding  the 
(minimal) number of points in $A$ such that the probability that
all local minimizers will be found is at least $\gamma$.

We analyze the extreme cases that $\theta^{*}=\left(l^{-1}, \ldots, l^{-1}\right)$. Hence the problem of finding $n^{*}(\gamma, \theta)$ is reduced to that of finding $n^{*}\left(\gamma, \theta^{*}\right)$. For a large $N$, $n^{*}(\gamma, \theta)$ can be approximated as 

\begin{equation}
	\begin{aligned}
		\operatorname{Pr}\left\{W=l \mid \theta^{*}\right\}&=l^{-N}  \sum_{N_{1}+\cdots+N_{l}=N}\left(\begin{array}{c}
			N \\
			N_{1}, \ldots, N_{l}
		\end{array}\right) \\
		&= \sum_{p=0}^{l}(-1)^{p}\left(\begin{array}{c}
			l \\
			p
		\end{array}\right)(1-p / l)^{N} \\
		& \sim \exp \{-l \exp \{-N / l\}\}, \quad N \rightarrow \infty
	\end{aligned}
\end{equation}

By solving the equation $\exp (-l \exp (-N / l))=\gamma$ with respect to $N$, we obtain
the approximation

\begin{equation}
	n^{*}\left(\gamma, \theta^{*}\right) \simeq l \ln l+l \ln (-\ln \gamma) \label{muti-p}
\end{equation}

% \textbf{Analysis 1:}
% It can be concluded that a larger $l$ will cause a larger query number (i.e., query number) of BagAmmo, which means decrease the convergence rate.

With Eq. \eqref{muti-p}, we analyze the relationship between the number of required queries and the population number as follows. We can see that multiple populations ( i.e., $l>1$ ) help to slow down the convergence rate of the algorithm. As we all know, prematurity is a common  phenomenon in which an evolutionary algorithm early converges to a poor local optimum. However, {\ouralgorithm} begins its search in multi start $l$ which  makes the algorithm can find a better solution with a higher probability. Our algorithm effectively relieves this problem by introducing multiple populations, and prevents the algorithm from wasting many efforts on repeatedly finding the same local optimum.

% , and  avoid the algorithm wasting much effort on repeated finding a same local optimum.
% However, our algorithm effectively relieve this  phenomenon through multi-population.


% \subsubsection{ \textbf{The queried count analysis}}
% The purpose of BagAmmo is to find the global optimal solution $x^*$. 
% We define a set $B$, and if a individual falls in this set then this  individual is a global optimal solution. The set $B$ can be defined as:
% \begin{equation}
% 	\label{eq:a}
% 	B = B(x^*,\epsilon,\rho)=\{ x\in A:\rho(x,x^*) \leq \epsilon \}
% \end{equation}
% where the $A$ is the solution space and $x$ is a individual,  $\epsilon \textgreater 0$ is a infinitesimal  constant. $\rho$ is a metric function that defines the distance between two points in $A$.


% So given a individual $x^j$, we define the probability of this individual being the optimal solution as:


% \begin{equation}
% 	Pr\{x^i \in B \}  =  g_B(F(x^i))             
% \end{equation}
% where $F$ is the fitness function, and $g_B$  is a  {function with positive correlation.} 

% %	For  convenience, we term the $g_B(f) $ as $P_B$


% What's more, the  probability of this individual  being not the optimal solution as:

% \begin{equation}
% 	Pr\{x^i \notin B \} = 1- g_B(F(x^i)) 
% \end{equation}	

% %jump out of partial optimization

% % \quad for \quad  all \quad j
% %	therefore,


% For  convenience, we assume all individuals are independent identically distributed (i.e., IID), so probability of any one individual   being  the optimal solution $g_B(F(x^j))$ can be written as $P_B$.
% The probability of all the individuals  being not the global optimal solution can be termed as:  

% \begin{equation}
% 	Pr\{x^1 \notin B,\ldots, x^n \notin B\}  = (1-P_B)^n
% \end{equation}

% Therefore we can define the probability of finding the global optimal solution:

% \begin{equation}
% 	\resizebox{.85\hsize}{!}{
% 		$	Pr\{x^j \in B\  for\  at\  least\  one\ j, 1 \leq j\leq n\}  = 1-(1-P_B)^n $
% 	}
% \end{equation}

% Let us assume that we are required to reach the set B with probability at least $1-\gamma$ for some $0<\gamma<1$.
% This gives the following inequality for $n$:
% \begin{equation}
% 	1-(1-P(B))^{n} \geq 1-\gamma
% \end{equation}

% Solving it we have:
% \begin{equation}
% 	n \geq n(\gamma)=\frac{\ln \gamma}{\ln (1-P(B))}
% \end{equation}

% We assume that $P(B)$ is small:
% \begin{equation} 
% 	\ln(1-P(B)) \cong-P(B)
% \end{equation}

% We have:
% \begin{equation} 
% 	n \geq-\frac{\ln \gamma}{P(B)}
% \end{equation}

% \textbf{Conclusion:} That is, we need to make at least $\lceil-\ln \gamma / P(B)\rceil$ queries in BagAmmo to reach the set $B$ with probability $1-\gamma$.






% \begin{table*}[]
% 	\caption{{Performance of target models.}}
% 	\centering
% 	\label{tab:performance of target}
% \resizebox{\textwidth}{!}{%
% \begin{tabular}{cc|cccccccc|cccc}
% \hline
% \multicolumn{2}{c|}{\multirow{2}{*}{Classifier\textbackslash{}Level}} & \multicolumn{4}{c|}{Family level} & \multicolumn{4}{c|}{Package level} & \multicolumn{4}{c}{Class level} \\ \cline{3-14} 
% \multicolumn{2}{c|}{} & \textbf{Precision} & \textbf{Recall} & \textbf{F1-score} & \multicolumn{1}{c|}{\textbf{Accuracy}} & \textbf{Precision} & \textbf{Recall} & \textbf{F1-score} & \textbf{Accuracy} & \textbf{Precision} & \textbf{Recall} & \textbf{F1-score} & \textbf{Accuracy} \\ \hline
% \multicolumn{1}{c|}{\multirow{5}{*}{MaMaDroid}} & \cellcolor{cyan!60!gray!10}RF & \cellcolor{cyan!60!gray!10}0.9400 & \cellcolor{cyan!60!gray!10}0.9397 & \cellcolor{cyan!60!gray!10}0.9398 & \multicolumn{1}{c|}{\cellcolor{cyan!60!gray!10}0.9397} & \cellcolor{cyan!60!gray!10}0.9514 & \cellcolor{cyan!60!gray!10}0.9507 & \cellcolor{cyan!60!gray!10}0.9511 & \cellcolor{cyan!60!gray!10}0.9507 & \cellcolor{cyan!60!gray!10}0.9409 & \cellcolor{cyan!60!gray!10}0.9407 & \cellcolor{cyan!60!gray!10}0.9408 & \cellcolor{cyan!60!gray!10}0.9407 \\

% \multicolumn{1}{c|}{} & DNN & 0.8935 & 0.8925 & 0.8930 & \multicolumn{1}{c|}{0.8925} & 0.9296 & 0.9294 & 0.9295 & 0.9294 & 0.8794 & 0.8766 & 0.8780 & 0.8765 \\

% \multicolumn{1}{c|}{} & \cellcolor{cyan!60!gray!10}AB & \cellcolor{cyan!60!gray!10}0.8855 & \cellcolor{cyan!60!gray!10}0.8854 & \cellcolor{cyan!60!gray!10}0.8854 & \multicolumn{1}{c|}{\cellcolor{cyan!60!gray!10}0.8854} & \cellcolor{cyan!60!gray!10}0.9088 & \cellcolor{cyan!60!gray!10}0.9087 & \cellcolor{cyan!60!gray!10}0.9088 & \cellcolor{cyan!60!gray!10}0.9087 & \cellcolor{cyan!60!gray!10}0.8823 & \cellcolor{cyan!60!gray!10}0.8823 & \cellcolor{cyan!60!gray!10}0.8823 & \cellcolor{cyan!60!gray!10}0.8823 \\

% \multicolumn{1}{c|}{} & 1-NN & 0.9112 & 0.9108 & 0.9117 & \multicolumn{1}{c|}{0.9108} & 0.9272 & 0.9266 & 0.9269 & 0.9266 & 0.9151 & 0.9146 & 0.9148 & 0.9146 \\

% \multicolumn{1}{c|}{} & \cellcolor{cyan!60!gray!10}3-NN & \cellcolor{cyan!60!gray!10}0.9059 & \cellcolor{cyan!60!gray!10}0.9057 & \cellcolor{cyan!60!gray!10}0.9058 & \multicolumn{1}{c|}{\cellcolor{cyan!60!gray!10}0.9057} & \cellcolor{cyan!60!gray!10}0.9193 & \cellcolor{cyan!60!gray!10}0.9191 & \cellcolor{cyan!60!gray!10}0.9192 & \cellcolor{cyan!60!gray!10}0.9190 & \cellcolor{cyan!60!gray!10}0.9054 & \cellcolor{cyan!60!gray!10}0.9052 & \cellcolor{cyan!60!gray!10}0.9053 & \cellcolor{cyan!60!gray!10}0.9052 \\ \hline

% \multicolumn{1}{c|}{\multirow{5}{*}{APIGraph}} & RF & 0.9406 & 0.9401 & 0.9404 & \multicolumn{1}{c|}{0.9401} & 0.9421 & 0.9416 & 0.9418 & 0.9416 & 0.9324 & 0.9323 & 0.9324 & 0.9323 \\

% \multicolumn{1}{c|}{} & \cellcolor{cyan!60!gray!10}DNN & \cellcolor{cyan!60!gray!10}0.8963 & \cellcolor{cyan!60!gray!10}0.8959 & \cellcolor{cyan!60!gray!10}0.8961 & \multicolumn{1}{c|}{\cellcolor{cyan!60!gray!10}0.8960} & \cellcolor{cyan!60!gray!10}0.8999 & \cellcolor{cyan!60!gray!10}0.8992 & \cellcolor{cyan!60!gray!10}0.8995 & \cellcolor{cyan!60!gray!10}0.8992 & \cellcolor{cyan!60!gray!10}0.8137 & \cellcolor{cyan!60!gray!10}0.8073 & \cellcolor{cyan!60!gray!10}0.8105 & \cellcolor{cyan!60!gray!10}0.8073 \\

% \multicolumn{1}{c|}{} & AB & 0.8842 & 0.8842 & 0.8842 & \multicolumn{1}{c|}{0.8842} & 0.8874 & 0.8873 & 0.8874 & 0.8873 & 0.8659 & 0.8659 & 0.8659 & 0.8659 \\

% \multicolumn{1}{c|}{} & \cellcolor{cyan!60!gray!10}1-NN & \cellcolor{cyan!60!gray!10}0.9110 & \cellcolor{cyan!60!gray!10}0.9104 & \cellcolor{cyan!60!gray!10}0.9107 & \multicolumn{1}{c|}{\cellcolor{cyan!60!gray!10}0.9104} & \cellcolor{cyan!60!gray!10}0.9184 & \cellcolor{cyan!60!gray!10}0.9179 & \cellcolor{cyan!60!gray!10}0.9181 & \cellcolor{cyan!60!gray!10}0.9179 & \cellcolor{cyan!60!gray!10}0.9090 & \cellcolor{cyan!60!gray!10}0.9086 & \cellcolor{cyan!60!gray!10}0.9088 & \cellcolor{cyan!60!gray!10}0.9086 \\

% \multicolumn{1}{c|}{} & 3-NN & 0.9033 & 0.9029 & 0.9031 & \multicolumn{1}{c|}{0.9029} & 0.9097 & 0.9094 & 0.9095 & 0.9094 & 0.8972 & 0.8970 & 0.8971 & 0.8970 \\ \hline

% \multicolumn{1}{c|}{GCN} & \cellcolor{cyan!60!gray!10}DNN & \cellcolor{cyan!60!gray!10}0.8362 & \cellcolor{cyan!60!gray!10}0.8324 & \cellcolor{cyan!60!gray!10}0.8339 & \multicolumn{1}{c|}{\cellcolor{cyan!60!gray!10}0.8351} & \cellcolor{cyan!60!gray!10}0.9283 & \cellcolor{cyan!60!gray!10}0.9433 & \cellcolor{cyan!60!gray!10}0.9356 & \cellcolor{cyan!60!gray!10}0.9363 & \cellcolor{cyan!60!gray!10}- & \cellcolor{cyan!60!gray!10}- & \cellcolor{cyan!60!gray!10}- & \cellcolor{cyan!60!gray!10}- \\ \hline
% \end{tabular}%
% }
% \end{table*}


% \subsection{An instance of adversarial example}
% \label{appendix:instance}
% \begin{figure}[h]
% 	\centering
% 	\includegraphics[scale=0.4]{sangji.pdf}
% 	\caption{Visualization of the perturbation.}
% 	\label{fig:sangji}
% \end{figure}

% {Here we visualize the perturbation derived by our method. To this end, we randomly select } a malware sample and then use {\ourtool} to modify this malware to attack MaMaDroid which chooses package feature granularity and uses a DNN as its classifier. After the attack succeeds, we collect the inserted caller and callee, which are shown in Fig. \ref{fig:sangji}.

% In this figure, the left is the callers and the right is the callees. The connection between a caller and the corresponding callee means the inserted function call between them. The thickness degree of the line between the caller and the callee reflects the number of times that the function call is inserted by {\ourtool}. In other words, the thicker the line is, the more calls between the caller and the callee are added. 


% {In this figure, the thick lines indicate that if we perturb the corresponding nodes, the influence on classification outcome will be large. It can be seen } that the thickest line is connected with Ljava/util package. Hence, adding more APIs in Ljava/util package helps induce misclassification. 




\subsection{ {Implementation details and an instance of the smali code}}
\label{appendix:smali}


	\begin{figure}[!h]
		\centering
		\includegraphics[scale=0.6]{direct.pdf}
		\caption{The examples of smali code with different invocation types.}
		\label{fig:direct}
	\end{figure}

 {
In this section, we first provide the implementation details of the transformation from the generator’s output to the perturbation on the malware samples. Then we give an instance of the samli code.}
 {The output of the generator is pairs of caller-callee functions. There are three steps to implement output-to-perturbation transformation. First, for every function pair, we find the smali file related to the selected caller, according to the latter's full name. }
 {Second, we insert statements into the smali file to implement a try-catch trap. Here we can use five types of function invocation, including invoke-direct, invoke-virtual, invoke-static, invoke-super and invoke-interface. Different invocation types require different smali manipulation. Fig. \ref{fig:direct} shows an example for every invocation type.  }
 {Third, we use \textit{Apktool} to rebuild the modified smali files to APK file. The above operations are automatically conducted by a Python script. }




\begin{figure}[!h]
	\centering
	\includegraphics[scale=0.45]{smali_code.pdf}
	\caption{An instance of the smali code.}
	\label{fig:smali}
\end{figure}

\begin{figure}[!h]
	\centering
	\includegraphics[scale=1]{retraining.pdf}
	\caption{{Attack success rate after retraining.}}
	\label{fig:retraining}
\end{figure}


 \begin{figure}[!h]
	\centering
	\includegraphics[scale=0.65]{ration_ori_adv.pdf}
	\caption{ {The detection success ratio on VirusTotal}.}
	\label{fig:ratio}
\end{figure}

 \begin{figure*}[!h]
	\centering
	\includegraphics[scale=0.9]{try_catch_count.pdf}
	\caption{ {The number of try-catch blocks before and after {\ourtool} attack.}}
	\label{fig:try_catch}
\end{figure*}

%  {The output-to-perturbation transformation consists of three steps. First, we find the smali file related to the selected caller, according to the latter's full name. 
% Second, we insert statements into the smali file to implement a try-catch trap. Here we can use five types of function invocation, including invoke-direct, invoke-virtual, invoke-static, invoke-super and invoke-interface. These invocation types result in different smali manipulations, due to their various requirements for register usage. To facilitate the understanding, we provide several examples of invocation types in Fig. 15 in Appendix 10.2. We also show how to modify the smali codes in Appendix 10.7.
% Third, we use \textit{Apktool} to rebuild the modified smali files to APK file. The above operations are automatically conducted by a Python script. }


To show how {\ourtool}  manipulates the smali code, we supply a  practical manipulation instance in Fig. \ref{fig:smali}. From line 6 to line 11, {we can find} a runtime exception. To be specific, we initialize an array with length 3 and employ an opaque method to visit the 4-th element of this array. Then it will throw an exception \textit{java.lang.ArrayIndexOutOfBoundsException} and skip the inserted callee functions. In this way, our method can effectively insert calls and preserve the malware's original functionality.





% \begin{figure}[h]
% 	\centering
% 	\includegraphics[scale=1.1]{retrain_recall.pdf}
% 	\caption{{Recall of the benign samples after retraining.}}
% 	\label{fig:retraining_recall}
% \end{figure}

{It is worth noting that the statements added into a try block are not fixed. Hence {\ourtool} can resist the whitelist-based defense. For example, suppose we want to trigger the exception of \textit{IndexOutOfBoundsException} by inducing array access violation. For this purpose, we access the array index that exceeds the array length. {\ourtool} can generate countless variable names and variable values for such an array index. Therefore, it is impossible to build a white list to rule out the statements added by {\ourtool}.}

% since our program can automatically produce unlimited number of variants to trigger
% any selected exception, 
% \subsection{{Detection performance of target models}}
% \label{appendix:performance}
% {
% We use the 10-fold cross-validation to evaluate the detection performance of target models (i.e., MaMaDroid, APIGraph, and GCN). The results are given in Table \ref{tab:performance of target}. This table shows that the target models perform well on our dataset. Consequently, they are qualified for evaluating {\ourtool}. Moreover, Table \ref{tab:performance of target} does not provide the results of GCN under the class-level feature granularity. This is because under the class-level feature granularity, the property vector of nodes in FCG is high-dimensional and extremely sparse, making the performance of GCN significantly degrade. Therefore, the GCN under the class-level feature granularity is not used as our target model.   }

% {For convenience, here we give the hyper-parameters of these baselines. The network structure of the DNN is shown as follows: one input layer, four hidden layers (all with 300 neurons) and one output layer (with 1 dimension). The hidden layers use the relu activation function. The output layer uses the sigmoid activation. The GCN's structure is described as follows: one input layer, a graph convolution layer with 256 neurons, a graph convolution layer with 512 neurons, an average pooling layer, a full connection layer with 256 neurons, a full connection layer with 128 neurons and one output layer. All the hidden layers adopt the relu activation function, and the output layer uses the sigmoid activation function. The base model of the Adaboost corresponds to a decision tree, and the maximum depth is 3. Finally, KNN-1, KNN-3 and RF all use their default parameters in the scikit-learn. }

% Second, it is noted that under the package-level feature granularity, the target models usually perform best. This is because that the package-level feature granularity is coarser and may lose some useful information and the class-level feature granularity is too 

% Note that in family granularity's FCG, many information  will lose and cause lower detection performance than  package granularity's FCG. However, in  class granularity's FCG, every function will be described in too much detail. It causes that the feature  be so sparse and  some  redundant information be introduced. So, the classifiers with package granularity's FCG  have a higher overall Android malware detection performance. }

% Please add the following required packages to your document preamble:
% \usepackage{multirow}
% \usepackage{graphicx}
% Please add the following required packages to your document preamble:
% \usepackage{multirow}
% \usepackage{graphicx}% Please add the following required packages to your document preamble:
% \usepackage{multirow}
% \usepackage{graphicx}



\subsection{\textbf{Dataset in our experiments}} \label{appendix: dataset}
{ Our dataset includes 44375 Android APKs released from 2010 to 2020, which are collected from AndroZoo, FalDroid,and Drebin. Table \ref{tab:dataset} gives the source, count,and years of APKs in our dataset. }

%Note that the APKs that cannot be successfully uncompressed are not included in our dataset. }

% Note that, the count of the sample is those APKs can be successfully decoded and  extracted feature, so it may be a little less than the count in original paper.

% \begin{table}[H] 
% 	\caption{Dataset used in our experiments.}
% 	\centering
% 	\label{tab:dataset}
% 	%\rowcolors{1}{white}{blue!5}
% 	\scalebox{0.9}{
% 		\begin{tabular}{cccc}
% 			\hline
% 			\multicolumn{1}{c}{Label} & Year & Count & Avg. Size (KB) \\ \hline
% 			\multirow{5}{*}{Malicious} & \cellcolor{cyan!60!gray!10}2016 & \cellcolor{cyan!60!gray!10}1530  & \cellcolor{cyan!60!gray!10}4259.48                    \\
% 			& 2017 & 1870  & 4120.43                \\
% 			& \cellcolor{cyan!60!gray!10}2018 & \cellcolor{cyan!60!gray!10}1893  & \cellcolor{cyan!60!gray!10}4112.36                \\
% 			& 2019 & 1604  & 4539.50                \\
% 			& \cellcolor{cyan!60!gray!10}2020 & \cellcolor{cyan!60!gray!10}1120  & \cellcolor{cyan!60!gray!10}4545.07             \\ \hline
% 			\multirow{5}{*}{Benign}    & 2016 & 1662  & 4479.19              \\
% 			& \cellcolor{cyan!60!gray!10}2017 & \cellcolor{cyan!60!gray!10}1920  & \cellcolor{cyan!60!gray!10}4949.94               \\
% 			& 2018 & 2043  & 5215.32                \\
% 			& \cellcolor{cyan!60!gray!10}2019 & \cellcolor{cyan!60!gray!10}2272  & \cellcolor{cyan!60!gray!10}6002.58               \\
% 			& 2020 & 1771  & 6528.78               \\ \hline
% 			Total & \cellcolor{cyan!60!gray!10}- & \cellcolor{cyan!60!gray!10}17685  & \cellcolor{cyan!60!gray!10}4929.72             \\ \hline			
% 			\hiderowcolors
% 		\end{tabular}%
% 	}
% \end{table}


% Please add the following required packages to your document preamble:
% \usepackage{multirow}
% \usepackage{graphicx}
\begin{table}[H]
	\caption{{Dataset used in our experiments.}}
	\centering
	\label{tab:dataset}
\scalebox{1}{%
\begin{tabular}{c|ccc}
\hline
\textbf{Source} & Label & Years & Count \\ \hline
\multirow{2}{*}{Androdzoo} & \cellcolor{cyan!60!gray!10}Benign & \cellcolor{cyan!60!gray!10}2010-2020 & \cellcolor{cyan!60!gray!10}21399 \\
 & Malicious & 2015-2020 & 9668 \\ \hline
FalDroid & \cellcolor{cyan!60!gray!10}Malicious & \cellcolor{cyan!60!gray!10}2013-2014 & \cellcolor{cyan!60!gray!10}8407 \\ \hline
Drebin & Malicious & 2010-2012 & 4900 \\ \hline
Total & \cellcolor{cyan!60!gray!10}- & \cellcolor{cyan!60!gray!10}2010-2020 & \cellcolor{cyan!60!gray!10}44374 \\ \hline
\end{tabular}%
}
\end{table}




\subsection{Resistance to adversarial retraining.}
\label{app:ar}
% Adversarial retraining is regarded as the most effective defense method against AE attacks. In this section, we test our {\ourtool} with adversarial retraining. We randomly select 100 adversarial examples that can deceive target systems and divide them into training and testing set. We use different ratios of training samples to retraining the target classifier, in order to evaluate attack success ratio (ASR) on the testing set \textcolor{green}{ and the recall of benign samples.}



% %为什么要增加recall的实验?感觉对我们不利
% %能不能从fig 18推出一些对我们有利的东西

% \textcolor{green}{Our results are given in Fig. \ref{fig:retraining} and Fig. \ref{fig:retraining_recall}, respectively. In Fig. \ref{fig:retraining}, the vertical axis is ASR of {\ourtool} and  the horizontal axis is the ratio of AEs used in adversarial retraining. Not surprisingly, the ASR decreases with the increase of the AEs adopted by adversarial retraining. When the ratio exceeds 40\%, adversarial retraining becomes effective in resisting {\ourtool}. In Fig. \ref{fig:retraining_recall}, the vertical axis reflects the recall of benign samples. This figure shows that adversarial retraining does not markedly decrease the recall of benign samples. This is because that the number of AEs is much smaller than that of clean samples (about 40K). In fact, if we add more AEs or the clones of the original AEs into adversarial retraining, the recall will decrease to a very low level. This indicates the model gets overfitted over the AEs. Finally, it should be pointed out that {\ourtool} can be used to improve the robustness of existing malware detection models. More specifically, one just needs to use {\ourtool} to generate an appropriate number of AEs, and employes the latter to adversarially training his model.}
 
%  What's more, it is extremely difficult to collect so many adversarial examples for adversarial retraining. 
 
%  Finally, model owners can improve their models' defense capability through adversarially retraining their models.   } 



Adversarial retraining is regarded as the most effective defense method against AE attacks. In this section, we test {\ourtool} with adversarial retraining. We randomly select 100 adversarial examples that are generated by {\ourtool} and can deceive target systems. We divide these adversarial examples into a training and a test set. {Under various training sample proportions, we retrain the target classifier in order to evaluate ASR on the test set.}



%为什么要增加recall的实验?感觉对我们不利
%能不能从fig 18推出一些对我们有利的东西

{Our results are given in Fig. \ref{fig:retraining}, whose vertical axis is the ASR of {\ourtool} and the horizontal axis is the proportion of AEs used in adversarial retraining. Not surprisingly, the ASR decreases with the increase of the AEs adopted by adversarial retraining. When the ratio exceeds 40\%, adversarial retraining becomes effective in resisting {\ourtool}. 
In practice, however, it is extremely difficult to collect sufficient adversarial examples for adversarial retraining. On the other hand, it is also noted that aided by {\ourtool}, model owners can improve their models’ defense capability with adversarial retraining.
}





\subsection{  {Attack performance on VirusTotal}}
\label{app:virustotal}

We  evaluate the  performance of {\ourtool} on  VirusTotal. 
 To be specific, we use {\ourtool} to generate AEs (adversarial examples) through querying the MaMadroid detector, and upload them to VirusTotal for malware detection.  
 VirusTotal uses about 60 malware detection methods unknown to us. We then record the ratio of the successful detection methods to all the methods, denoted by R\_adv. For comparison, we also conduct the same setting for the original sample, and the corresponding ratio is termed as R\_ori. 
 What's more, we calculate the difference between the R\_adv and R\_ori, which termed as R\_ori - R\_adv.
 The results are shown in Fig. \ref{fig:ratio}.  The horizontal axis of this figure shows different APKs, and the vertical axis gives the ratios R\_adv (denoted by the red line) and R\_ori (denoted by the blue line). The yellow line shows the decreasing ratio of the successful detection methods. It can be seen that {\ourtool} can effectively reduce the probability of malware being detected, owing to the transferability of AEs \cite{DBLP:conf/iclr/LiuCLS17}. It is worth noting that this attack effect is achieved under the scenario where no queries are conducted and no prior knowledge about detection methods can be obtained. 



\subsection{  {The number of added try-catch blocks}}\label{app:trycatch}

 {Since {\ourtool} inserts try-catch blocks into malware code, a defender may choose to detect it through judging whether the number of try-catch blocks exceeds a predetermined threshold. However, it is difficult to find an appropriate threshold for all APKs. Without such a threshold, this defense method may cause a high false positive or false negative rate.}

 {To verify it, we record the ratio of try-catch block number to the function-calls number in 50 malicious APKs and the corresponding adversarially perturbed APKs, termed as R\_ORI and R\_AE, respectively. The results are shown in Fig. \ref{fig:try_catch}.
The horizontal axis of this figure shows the IDs of these APKs, and the vertical axis gives the count ratio of try-catch blocks. The orange and green bars mean the original APK and the  corresponding modified APK, respectively. We can draw two conclusions from this figure. First, the number of try-catch blocks added by our method is relatively small compared to that of existing try-catch blocks. Therefore it is hard to find a threshold to 
 clearly distinguish the original APK and the perturbed APK. Second, the number of try-catch blocks drastically fluctuates among various APKs. Thus, it is also difficult to set a fixed threshold  for all APKs. }


 

\end{document}
\endinput



