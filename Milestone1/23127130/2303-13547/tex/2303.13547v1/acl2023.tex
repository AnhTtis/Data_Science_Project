% This must be in the first 5 lines to tell arXiv to use pdfLaTeX, which is strongly recommended.
\pdfoutput=1
% In particular, the hyperref package requires pdfLaTeX in order to break URLs across lines.

\documentclass[11pt]{article}

% Remove the "review" option to generate the final version.
\usepackage{ACL2023}

% Standard package includes
\usepackage{times}
\usepackage{latexsym}
\usepackage{multirow}
\usepackage{tabularx}

\usepackage{graphicx}
\usepackage{makecell}

\usepackage{algorithm}

\usepackage{microtype}
\usepackage{algorithmic}
\usepackage{makecell}
\usepackage{cjhebrew}
% \usepackage{algpseudocode}
% \usepackage[table]{xcolor}
\usepackage{amsmath}
\usepackage{graphicx}
\usepackage{url}
\usepackage{multirow}
\usepackage{comment}
\usepackage{amssymb}
% \usepackage{pifont}
\usepackage{balance}
\usepackage{mathtools}
% \usepackage{ulem}
\usepackage{xcolor}
\usepackage{colortbl}
\usepackage{bbm}
\usepackage{enumitem}
\usepackage{marvosym}
\usepackage{textcomp}
\usepackage{array}
\usepackage{booktabs}
\usepackage{tabularx}
% \usepackage{CJKutf8}

% For proper rendering and hyphenation of words containing Latin characters (including in bib files)
\usepackage[T1]{fontenc}
% For Vietnamese characters
% \usepackage[T5]{fontenc}
% See https://www.latex-project.org/help/documentation/encguide.pdf for other character sets

% This assumes your files are encoded as UTF8
\usepackage[utf8]{inputenc}

% This is not strictly necessary, and may be commented out.
% However, it will improve the layout of the manuscript,
% and will typically save some space.
\usepackage{microtype}

% This is also not strictly necessary, and may be commented out.
% However, it will improve the aesthetics of text in
% the typewriter font.
\usepackage{inconsolata}


% If the title and author information does not fit in the area allocated, uncomment the following
%
%\setlength\titlebox{<dim>}
%
% and set <dim> to something 5cm or larger.

\title{A comprehensive evaluation of ChatGPT's zero-shot Text-to-SQL capability}

% Author information can be set in various styles:
% For several authors from the same institution:
% \author{Author 1 \and ... \and Author n \\
%         Address line \\ ... \\ Address line}
% if the names do not fit well on one line use
%         Author 1 \\ {\bf Author 2} \\ ... \\ {\bf Author n} \\
% For authors from different institutions:
% \author{Author 1 \\ Address line \\  ... \\ Address line
%         \And  ... \And
%         Author n \\ Address line \\ ... \\ Address line}
% To start a seperate ``row'' of authors use \AND, as in
% \author{Author 1 \\ Address line \\  ... \\ Address line
%         \AND
%         Author 2 \\ Address line \\ ... \\ Address line \And
%         Author 3 \\ Address line \\ ... \\ Address line}

% \author{First Author \\
%   Affiliation / Address line 1 \\
%   Affiliation / Address line 2 \\
%   Affiliation / Address line 3 \\
%   \texttt{email@domain} \\\And
%   Second Author \\
%   Affiliation / Address line 1 \\
%   Affiliation / Address line 2 \\
%   Affiliation / Address line 3 \\
%   \texttt{email@domain} \\}

  \author{Aiwei Liu$^{1}$, Xuming Hu$^{1}$, Lijie Wen$^{1}$, Philip S. Yu$^{1,2}$\\
  $^1$Tsinghua University\\  $^2$University of Illinois at Chicago\\
  $^1$\texttt{\{liuaw20, hxm19\}@mails.tsinghua.edu.cn}\\
  $^1$\texttt{wenlj@tsinghua.edu.cn}
  $^2$\texttt{psyu@uic.edu}\\
  }

\begin{document}
\maketitle


\begin{abstract}
The current study investigated possible human-robot kinaesthetic interaction using a variational recurrent neural network model, called PV-RNN, which is based on the free energy principle.
Our prior robotic studies using PV-RNN showed that the nature of interactions between top-down expectation and bottom-up inference is strongly affected by a parameter, called the meta-prior, which regulates the complexity term in free energy.
% The current study examines how the behaviours of robots alter by changing the meta-prior $w$ in human-robot kinaesthetic interaction.
The current study examines how changing the meta-prior $w$ in the interaction phase affects the counter force generated when an experimenter attempts to induce movement pattern transitions familiar to the robot through its prior training.
The study also compares the counter force generated when trained transitions are induced by a human experimenter and when untrained transitions are induced.
Our experimental results indicated that (1) the human experimenter needs more/less force to induce trained transitions when $w$ is set with larger/smaller values, (2) the human experimenter needs more force to act on the robot when he attempts to induce untrained as opposed to trained movement pattern transitions.
Our analysis of time development of essential variables and values in PV-RNN during bodily interaction clarified the mechanism by which gaps in actional intentions between the human experimenter and the robot can be manifested as reaction forces between them.


%% Hiroki writing 2022-11-4
%Current study investigates the dynamics of the latent states during human-robot kinaesthetic interaction using PV-RNN.
%We have achieved to observe and analyse the internal state of an RNN model based on the free energy principle, during real-time human-robot interaction.
%Essential characteristics observed in the previous study of this variational recurrent neural network model, PV-RNN, is that by changing a meta prior $w$, the balance between the top-down intention and the bottom-up perceptual reality changes.
%In the current study, we examined how changing the weighting parameter $w$ between accuracy and complexity in free energy principle affects the humanoid robot's behaviour through human-robot interaction. We have conducted some human-robot kinaesthetic interaction experiments with various $w$ and quantitatively analysed the latent variable and the force applied to the humanoid robot. We have observed that the force required to change the robot's intention has increased, both when the top-down intention was strengthened by changing the $w$ and when corresponding switch of its primitive was against the experience of the RNN during its training. The study confirms through quantitative analysis that by increasing or decreasing the $w$ in PV-RNN, humanoid robot leads or follows the human counterpart during the human-robot kinaesthetic interaction.

\begin{comment}
Comment from Jun #2
・最後にQualitativeな結果(インパクト)が欲しい
・Current study investigates the problem on~と書き出すのが一般的
・最初の一文と最後の一文を対応させる
・最後の一文はもう少しAbstractかつ包括的に
\end{comment}

\begin{comment}
Comment from Jun #1
We investigated how the kinaesthetic human-robot interaction can affect the internal state of a model based on the free energy principle. 
=> how the internal state is affected is not the most important point in this study. This part should be rewritten.

The key function of this variational recurrent neural network model, PV-RNN, is that by changing a meta prior $w$, it takes a balance between the "complexity” term and the ”accuracy” term which corresponds to a top-down intention and a bottom-up perceptual reality in the free energy principle, respectively. 
=> This is not key function of PV-RNN. It is an essential characteristics observed in the previous study. The grammar after $w$ is something strange. Rewrite these.

This research has conducted a human-robot interaction experiment with a robotic agent in a kinaesthetic sense.
=> The sentence is not good. "in a kinaesthetic sense" is grammatically wrong.
MODIFIED => "In the current study human-robot interaction experiments using the kinaesthetic sense were conducted."

We investigated that when human forces the agent to switch primitives from one to another, larger force was required both when the human intention is conflictive against the top-down the intention of the agent and when the agent has a stronger top-down intention by modifying the $w$.
=> You should write the essential results of the experiments rather than what we investigated and also how these results could contribute to the studies on human-robot interaction.
\end{comment}

\end{abstract}

\section{Introduction}
\label{sec:intro}
\begin{figure}[t]
\begin{center}
    \includegraphics[width=1\linewidth]{figures/teaser.pdf}
\end{center}
\vspace{-0.1in}
\caption{\textbf{{\em Foggy} vs {\em Clear} NeRF.} Our \ournerf gets rid of reconstruction errors manifested as foggy ``floaters" in the density volume without additional input or significant computational overhead. 
%
Below are density profiles along a given ray before and after our geometry correction procedure, where we discard density peaks corresponding to floaters.
}
\label{fig:teaser}
\vspace{-0.2in}
\end{figure}



%The emergence of 
Neural Radiance Fields (NeRFs)~\cite{mildenhall2020nerf}  %and its variants 
have made revolutionary contributions in %photo-realistic 
novel view synthesis~\cite{barron2021mip,barron2022mip}, 
autonomous driving~\cite{rematas2022urban,tancik2022block}, digital human~\cite{hong2022headnerf,zhao2022humannerf}, and 3D content generation~\cite{eg3d,poole2022dreamfusion,lin2022magic3d}.
%by leveraging a multi-layer perceptron (MLP) to implicitly model the mapping from input 5D coordinates (i.e., 3D coordinates $\mathbf{x} = (x,y,z)$ and 2D viewing directions $\mathbf{d}=(\theta,\phi)$) to volume density $\sigma$ and view-dependent emitted radiance color $\mathbf{c} = (r,g,b)$. 
%
%They then use traditional volume rendering mechanisms on the obtained continuous 5D function (i.e., MLP) to generate novel views. 
To date, unfortunately, most NeRF-based methods encounter challenges when tackling large-scale cluttered scenes (e.g., Fig.~\ref{fig:teaser}):
\begin{enumerate}[leftmargin=0.16in, topsep=2pt,itemsep=-1ex,partopsep=1ex,parsep=1ex]
\item Input observations used for NeRF are often too sparse  compared to forward-facing or synthetic looking-inward scenes;
%\item Recovering fine-grained objects within a large volume is challenging for NeRF; %in capturing details accurately.
\item View-dependent visual effects give rise to ambiguity, resulting in a ``foggy" density field as shown in Fig.~\ref{fig:teaser}. 
%
Such artifacts are particularly pronounced in indoor scenes strewn with view-dependent appearances, such as specular highlights, glossy surface reflections from man-made objects. 
\end{enumerate}

Despite attempts to enhance NeRF's rendering quality given suboptimal input, such as using 3D conical frustums~\cite{barron2021mip,barron2022mip}, physically-grounded augmentations~\cite{chen2022aug}, and misalignment correction~\cite{jiang2022alignerf},  these challenges have yet to be fully resolved.
%
Depth supervision~\cite{deng2022depth, wei2021nerfingmvs} or proxy geometry~\cite{xu2021scalable,wu2022scalable} images can help alleviate the challenges in handling large-scale with sparse input, at the expense of %but they come at the cost of requiring 
expensive pre-processing or additional input.
%
Another line of work~\cite{wang2021neus, oechsle2021unisurf, wang2022neuris} achieves better reconstruction of surface geometry by using signed distances instead of volume density as scene representation. However, they sacrifice the ability to synthesize photo-realistic novel views.

%We observe that NeRF has been suffering from foggy ``floater" artifacts in large-scale cluttered scenes.
%
%Such artifacts are particularly pronounced in indoor scenes strewn with view-dependent appearances from man-made objects. 
%
To address the above issues, we propose an extension to NeRF, dubbed as {\bf \ournerf}, which enforces effective {\em appearance} and {\em geometry} constraints conducive to accurate colors and 3D densities estimation. We believe \ournerf can contribute beyond novel view synthesis, such as NeRF object detection~\cite{hu2022nerf}, NeRF object segmentation~\cite{zhi2021place, liu2022unsupervised, fan2022nerf,ren2022neural}, and NeRF registration~\cite{goli2022nerf2nerf}, where the rooms for improvement are substantial if more accurate color and density estimation are available.

Correspondingly, there are two steps in \ournerf. First, for appearance correction, the view-independent and view-dependent color components are predicted from the underlying 3D scene, which is combined to produce the final color estimation (Fig.~\ref{fig:toaster}).
%
The view-independent component (diffuse color and shading) captures the overall scene color, while the view-dependent component (highlights or reflections) captures color variations due to changes in viewing angle.
%
\ournerf then discards these view-dependent appearances in the training views to prevent them from interfering with the density estimation.
%
Second, a simple and effective geometry correction procedure will be performed to further eliminate the foggy ``floaters" or density errors. This geometry correction procedure is based on an assumption in line with traditional ray tracing in computer graphics.
\begin{comment}
% xh: basically copying method
On the other hand, ClearNeRF performs a geometric correction procedure performed on each traced ray during inference to refine the density estimation and better tackle the floater artifacts. 
%
The geometry correction procedure assumes that there should only be one salient peak along each traced ray during NeRF inference. 
Only the salient peak closest to the ray origin (the camera center) corresponds to  true geometry while the others will be manifested as foggy floaters hovering in the density volume. 
%
This assumption is in line with traditional ray tracing in computer graphics where in the absence of noise, only one intersection per ray should be returned to indicate the closest ray-object intersection.
%
\end{comment}
%%%%%%%%%%%
%As shown in Fig.~\ref{fig:teaser}, when reconstructing an indoor scene with sparse input and highly view-dependent objects, NeRF produces severe floating artifacts due to its attempt to explain view-dependent appearances.
%
Experiments verify that our proposed \ournerf can effectively get rid of floater artifacts without additional input.% or significant computational overhead. 


In summary, our contributions include the following:
\begin{itemize}[leftmargin=0.16in, topsep=2pt,itemsep=-1ex,partopsep=1ex,parsep=1ex]
    \item We propose a concise method for decomposing view-independent and view-dependent appearance during NeRF training and eliminate the interference of view-dependent appearance.
    \item We propose a geometric correction procedure performed on each traced ray during inference to refine the density estimation and better tackle the floater artifacts.
    \item Extensive experiments and ablations verify the effectiveness of our core designs and results in improvements over the vanilla NeRF and other state-of-the-art alternatives.
    %without additional computational resources or other inputs.
\end{itemize}





\section{Method}

\begin{figure}
  \includegraphics[width=0.48\textwidth]{figures/chatgpt.pdf}
  \caption{Example prompts for Text-to-SQL using ChatGPT. The prompt at the top is for a single-turn scenario, while the one below is for multi-turn scenarios where only new questions are added in each interaction.}
  \label{fig:intro}
\end{figure}

In order to enable ChatGPT to generate accurate SQL outputs, we utilized the prompt as shown in Figure \ref{fig:intro}. To ensure a fair demonstration of ChatGPT's Text-to-SQL capabilities,  we directly adopted the Text-to-SQL prompt used in the OpenAI demo webwite\footnote{https://platform.openai.com/examples/default-sql-translate} without conducting further prompt exploration.

The upper half of Figure \ref{fig:intro} represents the prompt in a single-turn Text-to-SQL scenario, where only the database and question information is required in the prompt. Meanwhile, in order to facilitate further evaluations, we emphasize in the prompt that the generated SQL statements can be executed in an SQLite database. The lower half of Figure \ref{fig:intro} represents the prompt in a multi-turn Text-to-SQL scenario, where the prompt for the first interaction is the same as that in the single-turn scenario, and for subsequent interactions, only the new questions are required.
\section{Experiments}
\label{sec:exp}
% logic:
% 1, Experiment Settings 

% 2, Benchmark results 
    % 1, VIP-Seg
    % 2, VSPW
    % 3, KITTI-STEP

% 3, Ablation studies and analysis. 
    % 1, improvements on baseline 
    % 2, design choices of temporal contarstive loss 
    % 3, design choices of label assigin stragety
    % 4, Effect of tube frames choices for CS loss 
    % 5, Effect of large window size / overlap inference. 
    % 6, Comparison with the different tracking choices. 
    % 7, increased GFLops/Parameters analysis. 
    % 8. FPS/Window Cruves.

% 4, visualization results. 
    % 1, comparison with strong baseline. 
    % 2, attention mask arcoss differnt tube. 
    

% Due to the unavailability of the test set, we report the results on the \textit{validation set}. 

% The former mainly focuses on mask proposal level as PQ~\cite{kirillov2019panoptic} with different window sizes, while the latter emphasizes pixel-level segmentation and tracking without any thresholds.
% KITTI-STEP has 21 and 29 sequences for training and testing, respectively. The training sequences are split into a training set (12 sequences) and a validation set (9 sequences).

\subsection{Experimental Settings}
\noindent
\textbf{Dataset.} We conduct experiments on five video datasets: VIPSeg~\cite{miao2022large}, VSPW~\cite{miao2021vspw}, KITTI-STEP~\cite{STEP}, and YouTube-VIS-19/21~\cite{vis_dataset}. We mainly conduct experiments on VIPSeg due to its scene diversity and long-length clips. The training, validation, and test sets of VIPSeg contain 2,806/343/387 videos with 66,767/8,255/9,728 frames, respectively. Although VSPW and VIPSeg share the same video clips, the training details are different since they are different tasks. Please refer to the \textit{supplementary material} for other datasets.


\noindent
\textbf{Evaluation Metrics.} For the VPS task, we adopt two metrics: $VPQ$~\cite{kim2020vps} and $STQ$~\cite{STEP}. The metric $STQ$ contains geometric mean of two items: Segmentation Quality ($SQ$) and Association Quality ($AQ$), where $ STQ = (SQ \times AQ)^{\frac{1}{2}}$. The former evaluates the pixel-level tracking, while the latter evaluates the pixel-level segmentation results in a video clip. For the VSS task, the Mean Intersection over Union (\textit{mIoU}) and mean Video Consistency ($mVC$)~\cite{miao2021vspw} are used for reference. For the VIS task, \textit{mAP} is adopted.


\noindent
\textbf{Implementation Details and Baselines.} We implement our models in PyTorch~\cite{pytorch_paper} with the MMDetection toolbox~\cite{chen2019mmdetection}. We use the distributed training framework with 16 V100 GPUs. Each mini-batch has one image per GPU. Following previous work, we use the image baseline pre-trained on COCO dataset~\cite{coco_dataset}. ResNet~\cite{resnet}, STDC~\cite{STDCNet}, and Swin Transformer~\cite{liu2021swin} are adopted as the backbone networks, which are pre-trained on ImageNet, and the remaining layers adopt the Xavier initialization~\cite{xavier_init}. 
For the detailed settings of other datasets, pretraining, and fine-tuning, please refer to the \textit{supplementary material}. To further verify the effectiveness of our approach, we build a stronger baseline by unifying Video K-Net with Mask2Former, where we replace the image encoder with Mask2Former. We term it Video K-Net+. We denote the extended Mask2Former-VIS for VPS as Mask2Former-VIS+.


%%%%%%%% VIP-SEG %%%%%%%%%%%
\begin{table}[!t]
	\centering
	\caption{\small \textbf{Results on VIPSeg-VPS~\cite{miao2022large} validation dataset.} We report VPQ and STQ for reference. Following Miao~\etal~\cite{miao2022large}, we report VPQ scores at different window sizes (1, 2, 4, 6). We report the results obtained from either an efficient or a strong backbone for comparison.}
	\label{tab:vipseg_results}
  \scalebox{0.65}{
    \begin{tabular}{ r|c|cccccc}
    \toprule[0.15em]
     Method& backbone & $VPQ^{1}$ & $VPQ^{2}$ & $VPQ^{4}$ & $VPQ^{6}$ & VPQ & STQ \\
    \toprule[0.15em]
    VIP-DeepLab~\cite{ViPDeepLab} & ResNet50 & 18.4 & 16.9 & 14.8 & 13.7 & 16.0 & 22.0 \\
    VPSNet~\cite{kim2020vps} & ResNet50 & 19.9 & 18.1 & 15.8 & 14.5 & 17.0 & 20.8 \\
    SiamTrack~\cite{woo2021learning_associate_vps} & ResNet50 & 20.0 & 18.3 & 16.0 & 14.7 & 17.2 & 21.1 \\
    Clip-PanoFCN~\cite{miao2022large} & ResNet50 & 24.3 & 23.5 & 22.4 & 21.6 & 22.9 & 31.5 \\
    Video K-Net~\cite{li2022videoknet} & ResNet50 & 29.5 & 26.5 & 24.5 & 23.7 & 26.1 & 33.1 \\
    Video K-Net+~\cite{cheng2021mask2former,li2022videoknet} & ResNet50 & 32.1 & 30.5 & 28.5 & 26.7 & 29.1 & 36.6  \\
    Video K-Net~\cite{li2022videoknet} & Swin-base & 43.3 & 40.5 & 38.3 & 37.2 & 39.8 & 46.3 \\
    \hline
    Tube-Link & STDCv1 & 32.1 & 31.3 & 30.1 & 29.1 & 30.6 & 32.0 \\
    Tube-Link & STDCv2 & 33.2  & 31.8 & 30.6 & 29.6  &  31.4 & 32.8 \\
    \hline
    Tube-Link & ResNet50 & 41.2 & 39.5  & 38.0 & 37.0 &  39.2 & 39.5 \\
    Tube-Link & Swin-base & 54.5 & 51.4 & 48.6 & 47.1 & 50.4 & 49.4 \\
    % Tube-Link & Swin-large &  \lxt{wait results} \\
    \bottomrule[0.2em]
    \end{tabular}
}
\end{table}


%%%%%% VIS-Youtube %%%%%%%%%
\begin{table}[t]
  \centering
   \caption{\small \textbf{Results on the YouTube-VIS datasets.} We report the mAP metric. \textdagger~adopt COCO video pseudo labels. Axial means using the extra Axial Attention~\cite{axialDeeplab}. Our method does not apply these techniques for simplicity.}
  \label{tab:ytvis}
  \scalebox{0.68}{
  \begin{tabular}{l c | c  | c }
    \toprule[0.2em]
    Method & Backbone  & YTVIS-2019 & YTVIS-2021 \\
    \toprule[0.2em]
VISTR~\cite{VIS_TR} & ResNet50 & 36.2 & -  \\
TubeFormer~\cite{kim2022tubeformer} & ResNet50 + Aixal & 47.5  & 41.2  \\
IFC~\cite{hwang2021video} & ResNet50 & 42.8 & 36.6 \\
SeqFormer~\cite{seqformer} & ResNet50 & 47.4 & 40.5  \\
Mask2Former-VIS~\cite{cheng2021mask2former_vis}& ResNet50 & 46.4 & 40.6 \\
IDOL~\cite{IDOL} & ResNet50 & 46.4 & 43.9\\
IDOL~\cite{IDOL} \textdagger & ResNet50 & 49.5 & -\\
VITA~\cite{heo2022vita} \textdagger & ResNet50 & 49.8 & 45.7  \\
Min-VIS~\cite{huang2022minvis} &ResNet50& 47.4 & 44.2 \\
% GenVIS~\cite{heo2022generalized} & ResNet50 & 51.3 & 46.3 \\
\hline
Tube-Link & ResNet50 & 52.8 & 47.9  \\% & - \\
\hline
SeqFormer~\cite{seqformer} & Swin-large  & 59.3 & 51.8 \\% & - \\
Mask2Former-VIS~\cite{cheng2021mask2former_vis} & Swin-large &  60.4 & 52.6 \\
IDOL~\cite{IDOL}  & Swin-large  & 61.5 & 56.1 \\ %& 42.6\\
IDOL~\cite{IDOL}  & Swin-large \textdagger  & 64.3 & -\\
VITA~\cite{heo2022vita} \textdagger & Swin-large & 63.0 & 57.5 \\ 
Min-VIS~\cite{huang2022minvis} & Swin-large & 61.6 & 55.3 \\
\hline
Tube-Link & Swin-large  & 64.6 & 58.4  \\
    \bottomrule[0.2em]
  \end{tabular}
}
\end{table}



%%%%%% VSPW and VIP-Seg VSS%%%%%%%%%
\begin{table}[t]
  \centering
    \caption{\small \textbf{Results on VSPW-VSS validation set}. $mVC_{c}$ means that a clip with $c$ frames is used.}
    \label{tab:vspw}
  \scalebox{0.68}{
  \begin{tabular}{l c c c c c }
    \toprule[0.2em]
    \textbf{VPSW} & Backbone & mIoU & $mVC_{8}$ &$mVC_{16}$  \\
    \toprule[0.2em]
    DeepLabv3+~\cite{deeplabv3plus} & ResNet101 & 35.7 & 83.5 & 78.4 \\
    TCB(PSPNet)~\cite{miao2021vspw,zhao2017pyramid} & ResNet101 & 37.5 & 86.9 & 82.1  \\
    Video K-Net (Deeplabv3+)~\cite{li2022videoknet,deeplabv3plus} & ResNet101  & 37.9 & 87.0 & 82.1 \\
    Video K-Net (PSPNet)~\cite{li2022videoknet,zhao2017pyramid} & ResNet101  & 38.0 & 87.2  & 82.3 \\
    MRCFA~\cite{sun2022mining} & MiT-B5 & 49.9 & 90.9  &  87.4  \\
    CFFM~\cite{sun2022vss} & MiT-B5 & 49.3 & 90.8 & 87.1 \\
    TubeFormer~\cite{kim2022tubeformer} & Axial-ResNet50x64  &  63.2 &  92.1 & 88.0 \\
    \hline
    Tube-Link & ResNet50 & 42.3 & 86.8 & 83.2 \\
    Tube-Link & Swin-large & 59.7 & 90.3 & 88.4 \\
    \bottomrule[0.2em]
  \end{tabular}
  }

\end{table}


\begin{table}[t]
  \centering
    \caption{\small \textbf{Results on VIP-Seg-VSS validation set}. $mVC_{c}$ means that a clip with $c$ frames is used.}
    \label{tab:vipseg_vss}
  \scalebox{0.68}{
  \begin{tabular}{l c c c c c }
    \toprule[0.2em]
    \textbf{VPSW} & Backbone & mIoU & $mVC_{8}$ &$mVC_{16}$  \\
    \toprule[0.2em]
    Video K-Net (Deeplabv3+)~\cite{li2022videoknet,deeplabv3plus} & ResNet101  & 38.3 & 88.0 & 83.1 \\
    Video K-Net (PSPNet)~\cite{li2022videoknet,zhao2017pyramid} & ResNet101  & 39.0 & 88.2  & 84.2 \\
    Mask2Former~\cite{cheng2021mask2former} &  ResNet50 & 38.4 & 87.5 & 82.5 \\
    Video K-Net+~\cite{cheng2021mask2former,li2022videoknet} &  Swin-base & 57.2 & 90.1 & 87.8  \\
    \hline
    Tube-Link & ResNet50 & 43.4 & 89.2 & 85.4 \\
    Tube-Link & Swin-base & 62.3 & 91.4 & 89.3 \\
    Tube-Link & Swin-large & 64.9 & 92.4 & 89.9 \\
    \bottomrule[0.2em]
  \end{tabular}
  }

\end{table}


\subsection{Benchmark Results}


%%%%%% KITTI-STEP %%%%%%
\begin{table}[t]
  \centering
   \caption{\small \textbf{Results on the KITTI val set.} OF refers to an optical flow network~\cite{teed2020raft}.}
  \label{tab:kitti_step}
  \scalebox{0.68}{
  \begin{tabular}{l c c || c c c c }
    \toprule[0.2em]
    \textbf{KITTI-STEP} & Backbone & OF & STQ & AQ & SQ & VPQ \\
    \toprule[0.2em]
    P + Mask Propagation & ResNet50 & \checkmark & 0.67 & 0.63 & 0.71 & 0.44 \\
    Motion-Deeplab~\cite{STEP}& ResNet50 &  & 0.58 & 0.51 & 0.67 & 0.40  \\
    VPSNet~\cite{kim2020vps}& ResNet50  & \checkmark & 0.56 & 0.52 & 0.61 & {0.43}  \\
    TubeFormer-DeepLab~\cite{kim2022tubeformer} & ResNet-50 + Axial &  & 0.70 & 0.64 &  0.76 & 0.51 \\
    Video K-Net~\cite{li2022videoknet} & ResNet50 &  & 0.71 & 0.70  & 0.71  &  0.46 \\
    Video K-Net~\cite{li2022videoknet} & Swin-base &  & 0.73 & 0.72 & 0.73 & 0.53 \\
    \hline
    Tube-Link & ResNet50 &  & 0.68 & 0.67 & 0.69 & 0.51 \\
    Tube-Link & Swin-base &  & 0.72 & 0.69 & 0.74 & 0.56 \\
    \bottomrule[0.2em]
  \end{tabular}
  }
  \vspace{-4mm}
\end{table}

% \lxt{will be changed by test set Figure Results Further. This figure will be merged into it as subfigure.}
\begin{figure}[t]
  \centering
   \includegraphics[width=0.80\linewidth]{./figs/teaser_trade_off.pdf}
   \caption{\small Tube-Link also achieves the best accuracy and speed trade-off on VIP-Seg dataset. FPS is measured on RTX GPU.}
   \label{fig:curve_trade_off_vipseg}
\end{figure}

\noindent
\textbf{[VPS] Results on VIPSeg.} 
We present the results of our Tube-Link method compared to previous works on the VIPSeg dataset in Tab.~\ref{tab:vipseg_results}. Our approach outperforms Video K-Net\cite{li2022videoknet} (under the same backbone) with 12\%-15\% VPQ and 7\%-10\% STQ improvements, respectively. Notably, our method with Swin-base~\cite{liu2021swin} backbone achieves new state-of-the-art results. 
%
We also evaluate our method using a lightweight backbone~\cite{STDCNet} for more efficient inference on video clips, and it achieves even better results than all previous methods with a larger ResNet50 backbone. 
%
These results demonstrate the effectiveness of our approach in exploiting temporal information.  Benefiting from the joint inference of subclips, our method achieves a much faster inference speed, as shown in Fig.~\ref{fig:curve_trade_off_vipseg}. 



\begin{table*}[h!]
    \footnotesize
	\centering
	\caption{\small \textbf{Ablation studies and comparative analysis on VIPSeg validation set with the ResNet50 backbone.} 
	}
    \subfloat[Ablation Study on Each Component.]{
    \label{tab:ablation_a}
	    \begin{tabularx}{0.43\textwidth}{c c c c c} 
		        				\toprule[0.15em]
    	baseline  & TCL & CTL & $\mathrm{VPQ_{th}}$ & VPQ \\
        \toprule[0.15em]
            Mask2Former-VIS+ (F) & - & - & 29.4 & 32.4 \\
            \hline
            Mask2Former-VIS+ (T) & - & - & 31.0 & 34.5\\
             & \checkmark & - & 34.6  & 36.8  \\  
          \rowcolor{gray!15}  & \checkmark & \checkmark & 35.1 & 37.5 \\  
        \bottomrule[0.1em]
	    \end{tabularx}
    } \hfill
    \subfloat[Design Choices of TCL.]{
    \label{tab:ablation_b}
		\begin{tabularx}{0.28\textwidth}{c c c} 
			\toprule[0.15em]
			Method & VPQ & STQ \\
			\midrule[0.15em]
            Dense Query~\cite{qdtrack} & 30.2  & 30.1  \\
            Sparse Query~\cite{li2022videoknet} & 34.5  & 35.1 \\
            \rowcolor{gray!15} Global Query(Ours) &  37.5  & 36.5 \\
			\bottomrule[0.1em]
		\end{tabularx}
    } \hfill
    \subfloat[Association Target Assign.]{
    \label{tab:ablation_c}
		\begin{tabularx}{0.24\textwidth}{c c c} 
			\toprule[0.15em]
			Method & VPQ & STQ  \\
			\midrule[0.15em]
			All-Masks~\cite{qdtrack} & 30.1 & 29.2 \\
			GT-Mask~\cite{li2022videoknet} & 35.6 & 35.9 \\
			\rowcolor{gray!15} Tube-Mask & 37.5 & 36.5 \\
			\bottomrule[0.1em]
		\end{tabularx}
    } \hfill
    \vspace{2mm}
    \subfloat[Input Sub-clip Size with Tube Window Size of 2 as Input.]{
     \label{tab:ablation_d}
	    \begin{tabularx}{0.30\textwidth}{c c c c} 
		        				\toprule[0.15em]
    		 Clip Size & STQ & VPQ & $\mathrm{VPQ_{th}}$  \\
    		\toprule[0.15em]
    	    T=1 & 34.5 & 35.6 & 30.2 \\
    	    \rowcolor{gray!15} T=2 & 36.5 & 37.5 & 35.1 \\
    	    T=2(ovl) & 35.9 & 37.3 & 35.0 \\
    	    T=3 &  36.4 & 37.0 & 35.3 \\
        	\bottomrule[0.1em]
	    \end{tabularx}
    } \hfill
    \subfloat[Tube-Window for Inference with Input Sub-clip Size 2 for Training.]{
     \label{tab:ablation_e}
	    \begin{tabularx}{0.30\textwidth}{c  c c c} 
		        				\toprule[0.15em]
    		 Window Size & STQ & VPQ  & $\mathrm{VPQ_{th}}$ \\
    		\toprule[0.15em]
    	    W=2 &  36.5 & 37.5 & 35.1 \\
    	    W=4 &  39.2 & 39.0 & 38.2 \\
    	   \rowcolor{gray!15} W=6 &  39.5 & 39.2 & 38.9 \\
    	    W=8 &  38.3 & 38.5 & 37.3 \\
        	\bottomrule[0.1em]
	    \end{tabularx}
    } \hfill
    \subfloat[Tracking Choices with the Default Setting of Tab.(d). ]{
     \label{tab:ablation_f}
	    \begin{tabularx}{0.35\textwidth}{c c c c} 
		        				\toprule[0.15em]
    		 Settings  &  STQ & VPQ & $\mathrm{VPQ_{th}}$ \\
    		 \toprule[0.15em]
    		  Extra Tracker~\cite{wangUnitrack,deepsort}& 33.9 & 36.6 & 34.1 \\
    		  RoI Features~\cite{qdtrack} & 34.5 & 35.9 & 34.5 \\
    		  Query Embedding~\cite{li2022videoknet}  & 33.1  & 36.0  & 33.0 \\
    	     \rowcolor{gray!15} Our Tube embedding & 36.5 & 37.5 & 35.1\\
        	\bottomrule[0.1em]
	    \end{tabularx}
    } \hfill
\end{table*}


\noindent
\textbf{[VIS] Results on YouTube-VIS-2019/2021.} In Tab.~\ref{tab:ytvis}, we compare our method with state-of-the-art VIS methods on the YouTube-VIS 2019 and 2021 datasets. Our method achieves a 3.0\% and 2.2\% mAP gain over VITA~\cite{heo2022vita} when using the ResNet50 backbone. Furthermore, compared with the Mask2Former-VIS baseline~\cite{cheng2021mask2former_vis}, our method achieves 4-5\% mAP gains on the two datasets with different backbones. Our method also outperforms the previous near-online method TubeFormer~\cite{kim2022tubeformer} by 5-6\% in terms of mAP on the two VIS datasets.


\noindent
\textbf{[VSS] Results on VSPW and VIP-Seg.} We further conduct experiments on VSPW dataset~\cite{miao2021vspw} for VSS to demonstrate the generalization of Tube-Link. As shown in Tab.~\ref{tab:vspw}, our method achieves over 4\% mIoU improvement compared to the Mask2Former baseline. Under the same ResNet101 backbone, our method achieves the best results. Using the Swin base backbone, our method achieves about 3.7\% mIoU gains over Video K-Net+ with consistent improvements on $mVC$. Our method with a lightweight backbone achieves comparable results to DeepLabv3+ with ResNet101, but with about four times faster inference speed (shown in Fig.~\ref{fig:curve}). Without using any additional techniques, our method also outperforms recent methods specifically designed for VSS~\cite{sun2022vss,sun2022mining}. In Tab.~\ref{tab:vipseg_vss}, we also compare the video semantic segmentation methods in recent VIPSeg datasets with higher-resolution images. Compared with previous state-of-the-art methods, our approaches also achieve state-of-the-art results.

% Moreover, compared with the previous state-of-the-art Tubeformer~\cite{kim2022tubeformer}, our method achieves a better 1.7\% mIoU.


\noindent
\textbf{[VPS] Results on KITTI STEP.} 
We further validate our method on KITTI STEP~\cite{STEP} and report the results in Tab.~\ref{tab:kitti_step}. Our method achieves 0.51 VPQ with the ResNet50 backbone, setting a new state-of-the-art result \textit{without} using temporal attention or optical flow warping. When using a strong Swin-base~\cite{liu2021swin} backbone, our method still achieves better results than Video K-Net~\cite{li2022videoknet} by 3\% VPQ and comparable results on STQ. It is worth noting that one can further improve the performance of Tube-Link by employing a better tracker design.

\subsection{Ablation Study and Visual Analysis}
\label{sec:ablation}
% 1, improvements on baseline 
% 2, design choices of temporal contarstive loss 
% 3, design choices of label assigin stragety
% 4, Effect of tube frames choices for CS loss 
% 5, Effect of large window size / overlap inference. 
% 6, Comparison with the different tracking choices. 
% 7, increased GFLops/Parameters analysis. 
% 8. FPS/Window Cruves.

% In this section, we present some \textit{\textbf{key}} ablations on component design and analysis using VIPSeg dataset with ResNet50 backbone. 
%The default setting used in our model is indicated in gray.
%More results are provided in the supplementary material. 

% \cavan{image part? or do you mean feature extractor or encoder}
% \cite{li2022videoknet} as the baseline by replacing its encoder with Mask2Former~\cite{cheng2021mask2former}
\noindent
\textbf{Improvements over Strong VPS Baseline.} 
In Tab.~\ref{tab:ablation_a}, we demonstrate the effectiveness of each component proposed in Sec.~\ref{sec:tb_framework}. 
The first row shows the results of the frame matching baseline. After adopting the tube matching, we obtain a gain of 1.6\% $\mathrm{VPQ_{th}}$ and 2.1\% on VPQ, even without any specific tracking design, which results in the same observation as shown in Tab.~\ref{tab:toy_exp}. Thus, we use Mask2Former-VIS+ (T, T=2) as our baseline by default, which achieves a strong starting point of 34.5 VPQ. $\mathrm{VPQ_{th}}$ refers to the VPQ for the thing class. This result shows the effectiveness of the na\"{i}ve framework. The addition of TCL further boosts performance, with a gain of 3.5\% on $\mathrm{VPQ_{th}}$ and 1.7\% on VPQ. Furthermore, adding CTL, which makes the association more consistent, improves $\mathrm{VPQ_{th}}$ by 1.5\%.


\noindent
\textbf{Ablation on Temporal Contrastive Loss.} We also compare our TCL design with previous works that use dense queries~\cite{qdtrack} or sparse queries~\cite{li2022videoknet} for matching. Both settings use only one frame, while our subclip size is two. As shown in Tab.~\ref{tab:ablation_b}, our method achieves the best results since tube matching encodes more temporal information. In particular, we observe 3.0\% VPQ improvements compared to the strong Video K-Net baseline.


\begin{figure}[t!]
	\centering
	\includegraphics[width=1.0\linewidth]{./figs/tube_link_vis_results_1st.pdf}
	\caption{\small Comparison results on VIP-Seg and YuoTube-VIS. Our method achieves consistent segmentation (shown in orange boxes) and better tracking results (shown in red boxes).}
	\label{fig:visulize}
\end{figure}

% \gl{R-50 and R50 should be consistent. The same as R-101 and R101.}
\begin{figure}[t]
  \centering
   \includegraphics[width=1.\linewidth]{./figs/both.pdf}
   \caption{\small Efficiency Analysis of Tube-Link. Left: Segmentation results (mIoU) of VSPW with different subclip sizes. Right: Inference speed (FPS) with different subclip sizes.}
   \label{fig:curve}
\end{figure}



\noindent
\textbf{Ablation on Association Target Assignment.} 
In Table \ref{tab:ablation_c}, we show the results of the ablation study on building association targets. We find that using a tube-level mask achieves the best results. Using the mask from one of the input subclips leads to inferior results. This is because the ground truth masks of a single frame are not aligned with the input global queries, where the global queries are learned from multiple frames using Equation \eqref{equ:sp_attention}.

\noindent
\textbf{Effect of Sub-clip Size for Training.} 
In Tab.~\ref{tab:ablation_d}, we investigate the impact of subclip size on training. Tube-Link becomes an online method when the subclip size is 1. As shown in the table, enlarging the subclip size improves the performance. We also examine overlapping during sampling, denoted as ovl, where two input subclips overlap at one frame. As shown in Tab.~\ref{tab:ablation_d}, enlarging the subclip size to 2 achieves significant improvement. However, we find that either frame overlapping or using a larger subclip size ($T=3$) does not bring extra gains. Adding more frames does not benefit temporal association learning, since most instances are similar within a subclip. Moreover, using more frames is not memory-friendly during training. Thus, the subclip size is set to 2. We can enlarge the size for inference, as shown in Tab.~\ref{tab:ablation_e}.
  
% \cavan{to what value and why? During training?}
% \cavan{you mean we can set the subcip size to 2 during training and expand it during inference? This point is not clearly articulated here.}
%  \cavan{where? In future work?}
% \cavan{not sure why we use `Moreover', it doesn't connect well to the previous sentence.}
% % Hence, we can enlarge the subclip size for more efficient inference and global consistency within each tube. 

\noindent
\textbf{Effect of Sub-clip Size for Inference.} 
\if 0
The global queries for each tube learn to perform temporal association via cross-attention within each subclip. Despite the subclip size is limited during the training due to the memory issues, we can expand it during the inference.
For example, the subclip size is 2 during training and is set to 6 for inference. As shown in Tab.~\ref{tab:ablation_e}, we prove that enlarging subclip size for inference improves the performance by a significant margin for all three metrics: STQ, VPQ and $\mathrm{VPQ_{th}}$. When the size is 8, the performance drops. This is because the global queries cannot handle larger subclips as the offline method. Besides the effectiveness, increasing subclip size can also lead to faster speed for each clip input due to full utilization of GPU memory, as shown in Fig.~\ref{fig:curve}.
\fi
%
During training, the subclip size is limited due to memory constraints, but we can expand it during inference to improve the performance. For instance, we use a subclip size of 2 during training and increase it to 6 during inference. Tab.~\ref{tab:ablation_e} shows that enlarging the subclip size for inference improves the performance considerably for all three metrics: STQ, VPQ, and $\mathrm{VPQ_{th}}$. However, when the subclip size is further increased to 8, the performance drops because the global queries are not designed to handle larger subclips. Increasing the subclip size can also speed up the inference process by utilizing the full GPU memory, as demonstrated in Fig.~\ref{fig:curve}.

% \cavan{`lead to a higher number of frames leads to faster speed'? Rephrase this sentence.}

\noindent
\textbf{Different Tracking Choices.} 
\if 0
In Tab.~\ref{tab:ablation_f}, we compare different tracking approaches that were used in previous studies~\cite{qdtrack,li2022videoknet,deepsort}. The default Tube Embedding works best in our framework. It does not require any association embedding head or the RoI crop operation on the VIPSeg dataset. Our Tube-Link only uses the learned tube-level embedding for the association.
\fi
In Tab.~\ref{tab:ablation_f}, we compare different tracking approaches used in previous studies~\cite{qdtrack,li2022videoknet,deepsort} with our Tube-Link. Our Tube-Link only uses the learned tube-level embedding for the association. We find that the default tube embedding works best in our framework, without requiring any association embedding head or RoI crop operation on the VIPSeg dataset.  

\subsection{Visualization and More Analysis}
\label{sec:vis_analysis}

\noindent
\textbf{GFLops and Parameter Analysis.} Compared with Mask2Former baseline, we only add one $\mathrm{Emb}$ head and one self-attention layer, introducing only 2.2\% GFLops and 1.4\% extra parameters with $720 \times 1280$ input. 

%\cavan{the font size is too small to be visible. You can use a common legend for both plots and place it underneath the plots}



% \subsection{Visualization and Analysis}
% \cavan{Do we really need this section? The `Speed and Accuracy with different Input Subclip Size' can be merged with `Effect of Sub-clip Size For Inference.' in the ablation study. `Visual Improvements on Baseline' can be merged with `Improvements over Strong VPS Baseline'.}

\noindent
\textbf{Speed and Accuracy with Different Input Subclip Size.} 
As shown in Table \ref{tab:ablation_e}, adding more frames improves the VPS results. To further analyze the speed-accuracy trade-off, we present a detailed comparison of different methods on the VSPW dataset in Fig.~\ref{fig:curve}. The left plot shows that enlarging the subclip size also improves the VSS results. The right plot illustrates that increasing the subclip size improves the single-frame baseline by 1.25-1.5\% for various backbones. Both performance and speed reach a plateau when the size increases to 6. The experiment justifies our choice of using an input subclip size of 6 for inference.

\noindent
\textbf{Visual Improvements on Baselines.} In Fig.~\ref{fig:visulize}, we present the visual comparison with several strong baselines (Video K-Net+ and Mask2Fomer-VIS) in VPS and VIS settings. The results are randomly sampled from a long clip. We achieve better results on both segmentation and tracking. More visual examples can be found in the supplementary material. 

\section{Related work}

In recent years, large language models have improved significantly in various NLP areas, especially in generative tasks.
A lot of new concepts were introduced, starting from attention mechanism~\cite{bahdanau2014neural}, transformers~\cite{vaswani2017attention} to multitask, learning from instructions~\cite{wang2022super} and human feedback~\cite{wang2021putting}.
The last becomes extremely popular in the generative context including machine translation. 
% new architectures were proposed~\cite{radford2019language,brown2020language}, and, 
Consequently, the usage of machine translation tools has become a necessary compound for understanding a foreign language. 
Unfortunately, like other neural network-based algorithms, these tools are vulnerable to adversarial examples~\cite{DBLP:journals/corr/GoodfellowSS14}. 
Starting from text classification \cite{li-etal-2020-bert-attack,DBLP:conf/acl/EbrahimiRLD18,Li2018TextBuggerGA}, vulnerability and robustness received a lot of attention in the NLP community. 
For MT systems one of the pioneering works was~\cite{ebrahimi2018adversarial}, where authors proposed a character-level approach to generate adversarial examples.
% that neural MT systems are vulnerable to character-level perturbations, where only a few symbols in an input query are subject to change. 
Inheriting HotFlip~\cite{ebrahimi-etal-2018-hotflip} there were considered white-box and black-box settings, where only a few symbols in an input query are subject to change imitating typos.

While white-box optimization may yield stronger adversarial perturbations it implies access to the model's architecture and weights which is impractical in the case of online MT tools. 
In~\cite{wallace} there was considered a white-box universal approach to a targeted attack on conditional text generation. 
The authors modeled perturbation as an insertion of a trigger, a token sequence of small length, that results in a generated sequence similar to the target set of sentences. 
While during experiments certain triggers cause a model to produce sensitive racist output, they are generally meaningless and similarly to character-level attacks are easy to detect. 
Authors of~\cite{guo-etal-2021-gradient,9747475} reported high attack transferability making this approach promising for black-box setup, however,  the research is limited only to the GPT-2 model for generation task. 
The above papers use greedy techniques to walk through the searching space during the optimization, on the other hand, attacks on NLP models could be found via projection onto embeddings~\cite{wallace}, and for MT task this was discovered in~\cite{Seq2Sick,Sadrizadeh2023TargetedAA,sadrizadeh2023transfool}. 
In~\cite{zhang2021crafting}, it was shown that black-box optimization may yield transferable word-level attack that fools online translation tools, for example Baidu and Bing translators. 
This work proposed to use the word saliency as the measure of uncertainty. 
Masking candidates the saliency was estimated via additional BERT model~\cite{devlin2018bert}  which lead to strong readable and imperceptible adversaries, however, neither human evaluation was performed nor quantities results for online tools were given. In~\cite{wan2022paeg}, a gradient-based approach to generate phrase-level adversarial examples for neural MT systems was proposed. Similarly to~\cite{zhang2021crafting}, it is proposed to estimate the vulnerable word positions are estimated in an input phrase with the use of gradient information and replace corresponding words by the candidates computed with an auxiliary model.

% \mynote{actually we may underline that we do not generate adversarial examples per se (we arent aimed at misclassification), but rather generate inputs that are been translated though they should not}

% \mynote{TODO: Maybe add more criticism of zhang2021crafting and point out the differences in our approach.}

% \todopa{}{}{
% https://www.semanticscholar.org/paper/AdvAug\%3A-Robust-Adversarial-Augmentation-for-Neural-Cheng-Jiang/1e7d3a9846da556bc7b84ae1410d257b89448c30
% }

%\todopa{}{}{
%https://www.semanticscholar.org/paper/A-Targeted-Attack-on-Black-Box-Neural-Machine-with-Xu-Wang/2a46eb47e8742be29b16a5b83dc1a38616b24ce6
%}

%\todopa{}{}{https://www.semanticscholar.org/paper/PAEG\%3A-Phrase-level-Adversarial-Example-Generation-Wan-Yang/a6dd2a8debb5d5324c4f2be7fb7bb52ce109cbaf}

% \todopa{}{}{
% https://download.huan-zhang.com/events/srml2022/accepted/bhandari22lost.pdf
% }

%\todopa{}{}{http://fan-yao.com/paper/2021_SEED_nmtstroke.pdf}

% \todopa{}{}{https://arxiv.org/pdf/2303.01068v1.pdf}

%\todopa{kosinski2023theory}
%    {Theory of mind may have spontaneously emerged in large language models}
%    {https://arxiv.org/pdf/2302.02083.pdf}
%    {We can say that large language models are very clever now, etc...}

% \todopa{ebrahimi2018adversarial}
%     {On adversarial examples for character-level neural machine translation}
%     {https://arxiv.org/pdf/1806.09030.pdf}
%     {Very related work (see beamsearch in the text also)...}

% \todopa{zhang2021crafting}
%     {Crafting adversarial examples for neural machine translation}
%     {https://github.com/JHL-HUST/AdvNMT-WSLS}
%     {Very related work. See: ``Besides, WSLS exhibits strong transferability on attacking Baidu and Bing online translators.''}

% \todopa{sadrizadeh2023transfool}
%     {TransFool: An Adversarial Attack against Neural Machine Translation Models}
%     {https://arxiv.org/pdf/2302.00944.pdf}
%     {Very related work!}

\section{Conclusions}
\label{sec:conclusions}

In this work, we have presented \mname{}, a novel framework for the design of distributed deep learning applications based on split computing. As a result, \mname{} suggests the proper configuration to match the quality of service requirements of the application and provide high performance in terms of accuracy and latency time. The experiments give an overview of specific features of \mname{} and show that the results obtained are very good, while 
%however, 
%given space constraints, 
extensive benchmarks will be the subject of future work.

%In future work, we will fine-tune the current implementation by adding functions for enabling hardware-in-the-loop simulation and exploiting further optimizations. In addition, we will investigate specific tensor reconstruction techniques to handle packet losses in UDP transmission. 

% \input{subfiles/999_others}
\bibliography{custom}
\bibliographystyle{acl_natbib}
\end{document}
