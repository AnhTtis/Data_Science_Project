%\widetext
\begin{center}
\textbf{\large Supplemental Materials}
\end{center}
\setcounter{equation}{0}
\setcounter{figure}{0}
\setcounter{table}{0}
\setcounter{page}{1}
\renewcommand{\theequation}{S\arabic{equation}}
\renewcommand{\thefigure}{S\arabic{figure}}
\renewcommand{\thetable}{S\arabic{table}}
\renewcommand{\thesection}{S\arabic{section}}

\section{$\sigma_{abcd}$ from Boltzmann-transport formalism}

\

The first-order non-equilibrium distribution function $\ff_1$ calculated from \eq{tau-f} is
%
\beq
\ff_1 = \frac{\tau}{D}\big[\E +(\E \cdot\B )\OO\big]\cdot\nabla_\kk\ff_0~.
\eql{ff1}
\eeq
%
The contribution from Lorentz force is neglected because of 
$(\nabla_\kk\ek \times \B)\cdot\nabla_\kk \ff_0 = 0$.
Then, the second non-equilibrium distribution function $\ff_2$ can be found 
by applying \eq{tau-f} again to $\ff_1$,
\begin{multline}
\ff_2= \frac{\tau^2}{D^2}\big[\E +\nabla_\kk \ek \times\B +(\E \cdot\B )\OO \big]_b \\
\cdot \partial_b \Big\{ \big[ \E +(\E \cdot\B) \OO \big]_c \cdot \partial_c  \ff_0 \Big\}~,
\eql{ff2}
\end{multline}
where the expansion of $1/D$ 
\beq
\frac{1}{D} = 1 - \B \cdot\OO + (\B \cdot\OO)^2 - \cdots
\eql{1D}
\eeq
are used in the non-equilibrium distribution functions.

The response current $\tilde j$ includes Zeeman effect, where $\ek = \epsilon_\kk - \bf{m} \cdot \B $
and $\ff_0$ is a function of $\ek$. 
The symmetric terms for $\E$ indexes,
which proportional to $\E^2\B$, are listed as
\begin{widetext}
\bea
\tilde j^{2,1}_a = - &\tau^2\int \dk\Big\{\big[-2(\B \cdot\OO)E_b E_c (\nabla_\kk\ek)_\alpha\nonumber 
+ 2E_b(\E \cdot\B )\Omega_c (\nabla_\kk\ek)_a \nonumber 
+ E_b E_c\left(\nabla_\kk \ek\cdot\OO\right) B_a \big] \partial_b\partial_c\ff_0 \\
&+\big[ (\E \cdot\B)\partial_b\Omega_c E_b(\nabla_\kk\ek)_a\nonumber 
-(\B \cdot\partial_b\OO)E_c E_b (\nabla_\kk\ek)_a \big]\partial_c \ff_0 \Big\}~.\nonumber
\eql{j21}
\eea
\end{widetext}
When ignoring the Zeeman effect, the Berry curvature related conductivity \eq{sigma-mc'} can be obtained after integration by parts.

Then, the Zeeman energy $\bf{m} \cdot \B $ under low external magnetic field limit contribute
the magnetic moment related conductivity \eq{sigma-Z} , which are also symmetric with $\E$ indexes.
However, the Zeemann energy will raise one order of $\B$. The homogeneous terms can be obtained in   
\beq
\tilde j^{2,0}_a = - \tau^2 \int \dk E_b E_c (\nabla_\kk\ek)_a \partial_b\partial_c\ff_0~.
\eql{j20}
\eeq
In \eq{sigma-Z}, the first term is from the Zeemann correction of distribution function $\ff_0$,
and the Zeemann part from the velocity $\nabla_\kk\ek$ contributes the second terms.


\section{All components of $\gamma$-tensor in Te}
%
\begin{figure}[t]
\centering
\includegraphics[width=0.95\columnwidth]{fig/All_gamma.png}
\caption{All non-zero components of eMChA tensor $\gamma^\prime$
    }
\figl{allgamma}
\end{figure}
%
All non-zero components of $\gamma^\prime_{abcd}$ in bulk Te are shown in \fref{allgamma}, 
which follows non-zaro elements in \tref{Te-sigma-emcha-sym}.
The largest and the second largest components are $\gamma^\prime_{zzzz}$ and $\gamma^\prime_{yzzy}$, where $b= c = z$ and $a = d$.
Ohmic conductivity has the relation
$\sigma_{zz} < \sigma_{yy} = \sigma_{xx}$ close to Fermi level, which
leads larger $\gamma^\prime$ components with the z-directional electric field 
in the energy window [-0.01eV, 0eV].
According to \eq{gamma-prime}, 
$\gamma^\prime$-tensor inherits the anisotropy from $\sigma_{abcd}$.


\section{All non-zero components of $\gamma^\prime$-tensor in slab Te}
\stm{I would remove the slab section, it is not really discussed in the main text}
The structure of slab Te breaks $C_{3z}$ rotation symmetry and changes the magnetic group from 
P3$_1$211$^\prime$ to C21$^\prime$, if $C_{2x}$ is preserved. The non-zero components of 
$\sigma_{abcd}$ or $\gamma^\prime_{abcd}$
is different from bulk Te, which is shown in \tref{Te-sigma-emcha-sym-slab}.
\begin{table}[h]
\caption{matrix elements of $\sigma_{abcd}$ in slab Te with magnetic group C21$^\prime$.} % title name of the table
\centering % centering table
\makebox[\columnwidth]{\begin{tabular}{|c|c|cccccc|}
  \hline
  \multicolumn{2}{|c|}{C21$^\prime$}& \multicolumn{6}{c|}{$bc$}\\
  \cline{3-8}
    \multicolumn{2}{|c|}{$\sigma_{abcd}$} &xx&yy&zz&yz/zy&xz/zx&xy/yx\\
  \hline
         &xx& $\sigma_{xxxx}$ & $\sigma_{xyyx}$& $\sigma_{xzzx}$& 0               & $\sigma_{xxzx}$ & 0                 \\
         &xy& 0               & 0              & 0              & $\sigma_{xyxy}$ & 0               & $\sigma_{xxyy}$   \\
         &xz& $\sigma_{xxxz}$ & $\sigma_{xyyz}$& $\sigma_{xzzz}$& 0               & $\sigma_{xxzz}$ & 0                 \\
         &yx& 0               & 0              & 0              & $\sigma_{yyzx}$ & 0               & $\sigma_{yxyx}$   \\
 $ad$    &yy& $\sigma_{yxxy}$ & $\sigma_{yyyy}$& $\sigma_{yzzy}$& 0               & $\sigma_{yxzy}$ & 0                 \\
         &yz& 0               & 0              & 0              & $\sigma_{yyzz}$ & 0               & $\sigma_{yxyz}$   \\
         &zx& $\sigma_{zxxx}$ & $\sigma_{zyyx}$& $\sigma_{zzzx}$& 0               & $\sigma_{zxzx}$ & 0                 \\
         &zy& 0               & 0              & 0              & $\sigma_{yyzx}$ & 0               & $\sigma_{yxyx}$   \\
         &zz& $\sigma_{zxxz}$ & $\sigma_{zyyz}$& $\sigma_{zzzz}$& 0               & $\sigma_{zxzz}$ & 0                 \\
  \hline
\end{tabular}}
\tabl{Te-sigma-emcha-sym-slab}
\end{table}
In the eMChA measurement of slab Te, there is a large signal from $\gamma^\prime_{zzzy}$, which is not exit in bulk Te.
However, the non-zero component in \tref{Te-sigma-emcha-sym-slab} can explain its origin.

\section{Two bands effective model}
In the absence of spin-orbit coupling (SOC), 
there is a Weyl point generated by the intersection of two top valence bands in Te. 
An annihilated Weyl point forms the weak ``camelback'' shape of the valence band top \cite{stepan18}.
A two-band model can be used to simulate the ``camelback'' shape of the valence band top ,
\begin{widetext}
\beas{Te-like}
   \Ham_{\rm FS} &=& \Ham_{\rm Weyl} + \Ham_{\rm SB} + \Ham_{\rm \Delta} ~,\\
   \Ham_{\rm Weyl} &=& \begin{pmatrix}
   m - \t_1 \cdot \cos(\k) &
   t_2\cdot[\sin(k_x) - i\cdot \sin(k_y)] \\
   t_2\cdot[\sin(k_x) + i\cdot \sin(k_y)] &
       -m + \t_1\cdot \cos(\k)
   \end{pmatrix}~,\\
   \Ham_{\rm SB} &=& \begin{pmatrix}
   0&
   t_3\cdot \sin(k_z) \\
   t_3\cdot \sin(k_z)&
   0
   \end{pmatrix}~,\\
   \Ham_{\rm \Delta} &=& \begin{pmatrix}
   -\t_4\cdot \cos(2\k) &
   0 \\
   0&
   -\t_4\cdot \cos(2\k)
   \end{pmatrix}~,
\eeas
\end{widetext}
where $\t_1 = (t_1,t_1,t_1)$ and $\t_4 = (t_4,t_4/2,t_4/2)$.


$\Ham_{\rm Weyl}$ represents the standard Weyl point, while $\Ham_{\rm \Delta}$ causes the bands to bend, resulting in Rashba-like bands. 
The gap is opened by the symmetry-breaking part $\Ham_{\rm SB}$.
The band along $k_z$ calculated from the effective model $\Ham_{\rm FS}$ in \eq{Te-like} has similar ``camelback'' shape with \fref{Te}(a),
when $m$=4, $t_1$=2, $t_2$=1, $t_3$=0.4 and $t_4$=2.
With $k_y = k_x = 0$, the energy eigenvalue of two bands reads
\begin{widetext}
\beq
\epsilon^{u/l} = -t_4 - t_4\cos{2k_z} 
    \pm \frac{ \sqrt{
    2m^2 - 8m t_1 + 9t_1^2 + t_3^2 - (4m t_1 - 8t_1^2)\cos{k_z} +
    (t_1^2 - t_3^2)\cos{2k_z}  }}{\sqrt{2}}~,
\eql{Te-like-e}
\eeq
\end{widetext}
which are the only terms and their derivatives dependent on $t_4$.
However, $\Omega_z$, $m^{orb}_z$ and their derivatives are independent with $t_4$.
By fixing $m$=4, $t_1$=2, $t_2$=1, $t_3$=0.4, all the quantities on $k_z$ axis of two bands reads
\beas{quantities-Te-like}
\epsilon^{u/l} &=& -t_4 (1+\cos{2k_z}) \nonumber\\
	&&\pm \sqrt{2.08 + 1.92\cos{2k_z} }\\
v^{u/l}_z &=& 2t_4 \sin{2k_z} \pm \frac{3.84\sin{2k_z}}{\epsilon^--\epsilon^+ }\\
\Omega^{u/l}_z &=& \pm \frac{8\cos{k_z}}{(\epsilon^- - \epsilon^+)^3 }\\
m^{orb,u/l}_z &=& -\frac{4\cos{k_z}}{(\epsilon^- - \epsilon^+)^2 }
\eeas

\fref{eff-velocity} illustrates that by adjusting $t_4$, the impact of $v_z$ on $\sigma_{zzzz}$ can be examined.
As shown in \fref{eff-velocity}(a), changing $t_4$ from 0 to 4 greatly alters the band structures while keeping $\Omega_z$
and $m^{orb}_z$ constant. The velocity of the upper band changes the sign, with $v_z$ approaching zero at $t_4 = 2$.
\fref{eff-velocity}(c-d) and (f-g) demonstrate the sensitivity of the quantities to $t_4$.
Specifically, $-v_z\partial_z\Omega_zv_z$,
$v_{zz}\Omega_z v_z$, and $-\partial_{zz}m^{orb}_zv_z$ are enhanced or diminished by $t_4$ due to its influence on velocity $v_z$.
However, the $t_4$ component has a low proportion in $v_{zzz}$, 

\begin{multline}
v_{zzz} = -8 t_4 \sin{2k_z} + \frac{22.1184 \cos{2k_z}\sin{2k_z}} {(\epsilon^l/2 -\epsilon^u/2)^3 }\\
         - \frac{7.68 \sin{2k_z}}{\epsilon^l/2 -\epsilon^u/2 }
         + \frac{21.2337 \sin^3{2k_z}}{(\epsilon^l/2-\epsilon^u/2)^5 }~,
\eql{wv}
\end{multline}
and retains large magnitude. 
At $t_4 = 2$, $v_{zzz}m^{orb}z$ dominates $\sigma_{zzzz}$, while the influence of the others is suppressed by $v_z$.

\begin{figure}[h!]
\centering
\includegraphics[width=1.0\columnwidth]{fig/eff-velocity.png}
    \caption{
    Band structure (a) and quantities (b-h) of the effective model $\Ham_{\rm FS}$ with various $t_4$.
    %Fixing energy $\epsilon^\pm - t_4$, to make the bands located in the same window.
    (b) Velocity of the upper band with various $t_4$. (c-h) Quantities in $\sigma_{zzzz}$ of upper band
    with the same magnitude window and various $t_4$.
    }
\figl{eff-velocity}
\end{figure}


%%% Local Variables:
%%% mode: latex
%%% TeX-master: "pap"
%%% End:
