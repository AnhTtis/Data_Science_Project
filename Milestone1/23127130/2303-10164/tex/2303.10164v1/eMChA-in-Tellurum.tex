%======================
%\section{eMChA in Tellurum}
\secl{Te}
%======================

% \tents{Above we assumed that fixed external fields are applied and the current was calculated. However, in the experiments
% a reverse problem is solved: fixed currents $j_z$ is passed and the longitudinal voltage is measured (the magnetic field is kept fixed as well). }
% %thorugh were activated in Te nanowires \cite{calavalle2022gate}.
% %In trigonal elemental Te, atoms are helically stacked along the z-axis with a helical
% %period of three atoms based on a hexagonal lattice \cite{doi:10.1063/1.1743647}. 
% %The chiral structure break the invertion symmetry and
% %generates colossal chiral dependent axial Berry curvature and magnetic moment \cite{stepan18}.
% %There are chiral-dependent response currents detected as the difference in voltage 
% %by reverse driving current $j_z$ with static parallel $+B_z$. 
% A measured eMChA parameter $\gamma_{zzzz}$ proposed by Rikken \cite{Rikken19,Rikken01}
% can be obtained by two measurements with opposite directions (but the same magnitude) of applied current $\pm j_z$
% as a ratio between voltage difference and average, 
% \bea
% \gamma_{zzzz} = \frac{V_z^+ - V_z^-}{V_z^+ + V_z^-} =  \frac{E_z^+ - E_z^-}{E_z^+ + E_z^-}~.
% \eql{gamma-z}
% \eea
% To relate $\gamma_{zzzz}$ with $\sigma_{zzzz}^{\rm eMChA}$ we need to revert the equation 
% \beq
% j_\alpha = \sigma_{\alpha\beta}E_\beta + \sigma^{\rm eMChA}_{\alpha\beta\gamma\mu}E_\beta E_\gamma B_\mu
% \eeq
% into 
% \beq
% E_\alpha = \rho_{\alpha\beta}\,j_\beta - \sigma^{\rm eMChA}_{\alpha\beta'\gamma'\mu}\,\rho_{\beta'\beta}\,\rho_{\gamma'\gamma}\,j_\beta j_\gamma B_\mu
% \eeq
% upto order $j^2 B$, where $\sigma\equiv\rho^{-1}$ is the linear Ohmic conductivity, 
% %It was used to represent the strength of eMChA.
% %The subscript $\pm$ shows the positive and negative of $\pm j_z$
% %Based on \eq{R-DL}, the response current with $\pm j_z$ can be written as, 
% %\bea
% %j_z =& \sigma_{zz}E_z^+ + \sigma^{\rm MR}_{zzzz}E_z^+ B_z^2 + \sigma^{\rm MC}_{zzzz}(E_z^+)^2 B_z\\
% %-j_z =& -\sigma_{zz}E_z^- - \sigma^{\rm MR}_{zzzz}E_z^- B_z^2 + \sigma^{\rm MC}_{zzzz}(E_z^-)^2 B_z~.
% %\eql{measure-current}
% %\eea
% %By solving the quadratic equation, and
% %$\sigma^{\rm MR}_{zzzz} B_z^2 \ll \sigma^{\rm O}_zz$ in Te with low $B_z$, the parameter is
% Thus, we get
% \beq
% \gamma_{zzzz} = - \sigma^{\rm eMChA}_{zzzz}/(\sigma_{zz})^2.
% \eql{gamma-sigma}
% \eeq

% %\Red{supplementary materials: calculation of $\gamma$}

% %With the increase of $B_z$, the weight of "V-shape" linear-magnetoresistance increases \cite{calavalle2022gate}. It is also neglected 
% %in our simulation of eMChA in Te, because this $\B$ even conductivity is not what we 
% %focus on in this paper. 
% Relaxation time $\tau$ in \eqs{sigma-mc'}{sigma-Z} is a difficult quantity to quantify.
% However $\sigma_{\alpha\beta}$ is proportional to $\tau$ \cite{wang1966solid} and can be evaluated as.
% \beq
% \sigma_{\alpha\beta} = \tau\int\dk v_\alpha v_\beta f'
% \eeq
% Thus the magnitude of $\tau$ is eliminated from the simulation results of $\gamma_{zzzz}$ \tents{while we should keep in mind that we are making an approximation that $\tau$ is independent of $\k$.}

\textit{Numerical results for $p$-Te.}
%
The unit cell of trigonal Te contains three atoms disposed along a
spiral chain, with the chains arranged on a hexagonal net.  The
fully-relativistic electronic structure of the pristine (undoped) left-handed crystal
(space group P3$_2$21) is evaluated via density-functional theory in
the pseudopotential framework, using the HSE06 hybrid
functional~\cite{paier-jcp06} implemented in the VASP code package
\cite{kresse1999ultrasoft,kresse1996efficiency,PhysRevB.54.11169}.



In order to evaluate the needed $k$-space quantities % (energy bands, Berry
% curvatures, and magnetic moments, and their $\kk$ derivatives)
accurately and efficiently,
% on an ultra-dense grid of
% k-points,
we employ the Wannier interpolation scheme~\cite{wang-prb06}.  The
Wannier functions are constrained in a post-processing step using the Wannier90 code
package~\cite{wannier90}, starting from atom-centered $s$ and $p$
trial orbitals.
% of
% Te as initial guess.
In order to avoid artifacts due to possible
numerical violations of crystal symmetries by the Wannier model, the
Wannier Hamiltonian and orbital-based matrix elements are symmetrized
to satisfy the point-group symmetries.
% according to the corresponding \tents{\st{magnetic}} point group.

We implemented all terms of \eqs{sigma-Z}{sigma-mc'} within the
WannierBerri code \cite{wannierberri}.  Wannier interpolation allows
us to evaluate band properties and their momentum-space derivatives
directly, without finite difference schemes (details will be published
elsewhere).
% ~\cite{gradients}).
We perform the Brillouin zone integrals
on a grid of $400\times400\times400$ points, using 300 adaptive
refinement iterations, and employing tetrahedron method to accurately
describe the Fermi surface.

% Naturally, the measurable current is described by the part of the
% conductivity tensor $\sigma^{eMChA}_{\alpha\beta\gamma\mu}$ that is
% symmetric in the second and third indices, while the anti-symmetric
% part amounts to ``gauge freedom'' \cite{Ohmic-Hall}.  After removing
% the anti-symmetric part from the gauge, the symmetry of the tensor
% $\sigma^{eMChA}_{\alpha\beta\gamma\mu}$ is shown in
% \tref{Te-sigma-emcha-sym}.  The symmetry properties for other magnetic
% point groups can be obtained via $\rm MTENSOR$
% \cite{mTensor,mTensor-link} routine of Bilbao Crystallographic Server
% with Jahn symbol eV[V2]V.  Note that the % measurable
% intrinsic eMChA
% % parameters
% response tensor $\gamma'_{abcd}$ has the same symmetry as
% $\sigma_{abcd}$.

The eMChA conductivity $\sigma_{abcd}$ is a nonmagnetic polar tensor
symmetric in $bc$ (Jahn symbol
eV[V2]V). Table~\ref{tab:Te-sigma-emcha-sym} shows the symmetry
constrains for the point group 32 of trigonal tellurium, obtained
using the $\rm MTENSOR$ \cite{mTensor,mTensor-link} routine provided
by the Bilbao Crystallographic Server.
% ; since the trigonal axis is along $z$
% making the Ohmic resistivity diagonal, the 

\begin{table}[t]
\caption{Symmetry constraints on the components of the
  $\sigma_{abcd}$ tensor in trigonal tellurium \tents{(crystal class 32).}
  % magnetic group P3$_1$211$^\prime$.
} % title name of the table
\centering % centering table
\makebox[\columnwidth]{\begin{tabular}{|c|c|cccccc|}
  \hline
  \multicolumn{2}{|c|}{\tents{32}}& \multicolumn{6}{c|}{$\tents{bc=cb}$}\\
  \cline{3-8}
    \multicolumn{2}{|c|}{$\sigma_{abcd}$} &$xx$&$yy$&$zz$&$yz$&$xz$&$xy$\\
  \hline
         &$xx$& $\sigma_{yyyy}$ & $\sigma_{yxxy}$& $\sigma_{yzzy}$& $-\sigma_{yyzy}$& 0               & 0                 \\
         &$xy$& 0               & 0              & 0              & 0               & $-\sigma_{yyzy}$& $(\sigma_{yyyy}-\sigma_{yxxy})/2$ \\
         &$xz$& 0               & 0              & 0              & 0               & $\sigma_{yyzz}$ & $-\sigma_{yyyz}$  \\
         &$yx$& 0               & 0              & 0              & 0               & $-\sigma_{yyzy}$& $(\sigma_{yyyy}-\sigma_{yxxy})/2$ \\
 $ad$&$yy$& $\sigma_{yxxy}$ & $\sigma_{yyyy}$& $\sigma_{yzzy}$& $\sigma_{yyzy}$ & 0               & 0                 \\
         &$yz$& $-\sigma_{zyyy}$& $\sigma_{yyyz}$& 0              & $\sigma_{yyzz}$ & 0               & 0                 \\
         &$zx$& 0               & 0              & 0              & 0               & $\sigma_{zyzy}$ & $-\sigma_{zyyy}$  \\
         &$zy$& $-\sigma_{zyyy}$& $\sigma_{zyyy}$& 0              & $\sigma_{zyzy}$ & 0               & 0                 \\
         &$zz$& $\sigma_{zyyz}$ & $\sigma_{zyyz}$& $\sigma_{zzzz}$& 0               & 0               & 0                 \\
  \hline
\end{tabular}}
\tabl{Te-sigma-emcha-sym}
\end{table}

%\begin{table}[t]
%\caption{Non-zero components of $\sigma^{MC}_{\alpha\beta\gamma\mu}=\sigma^{MC}_{\alpha\gamma\beta\mu}$ in Te. The components in the same line have the same value.} % title name of the table
%\centering % centering table  
%\begin{tabular}{c}
%\hline\hline
% Same Value Components \\
% \hline
%    $\sigma_{xxxx},\sigma_{yyyy}$\\
%    $\sigma_{xzzx},\sigma_{yzzy}$\\ 
%    $\sigma_{xyyx},\sigma_{yxxy}$\\
%    $-\sigma_{xyzx},-\sigma_{xzyx},-\sigma_{xxzy},-\sigma_{xzxy},-\sigma_{yxzx},-\sigma_{yzxx},\sigma_{yyzy},\sigma_{yzyy}$\\
%    $\sigma_{xxyy},\sigma_{xyxy},\sigma_{yxyx},\sigma_{yyxx}$\\
%    $-\sigma_{xxyz},-\sigma_{xyxz},-\sigma_{yxxz},\sigma_{yyyz}$\\
%    $\sigma_{xxzz},\sigma_{xzxz},\sigma_{yyzz},\sigma_{yzyz}$\\
%    $-\sigma_{zxyx},-\sigma_{zyxx},-\sigma_{zxxy},\sigma_{zyyy}$\\
%    $\sigma_{zxzx},\sigma_{zzxx},\sigma_{zyzy},\sigma_{zzyy}$\\
%    $\sigma_{zxxz},\sigma_{zyyz}$\\
%    $\sigma_{zzzz}$\\
% \hline\hline
%\end{tabular}
%\tabl{Te-nonzero}
%\end{table}
%

The effect of doping is treated in the rigid-band approximation.  In
$p$-Te, the Fermi energy cuts the top of the valence band, forming a
small hole-like Fermi pocket near the H point of the Brillouin zone,
as shown in \fref{Te}(a).  We find that when the Fermi level is close
to the valence band maximum, the largest component is $\gamma'_{zzzz}$
and the second largest components are
\tents{$\gamma'_{xzzx}=\gamma'_{yzzy}$}., while other components are
orders of magnitude smaller (details in Supplementary Material
\cite{SM}).  The fact that the dominant components are driven by
currents parallel to the trigonal axis $\bf z$, suggests that the
strong coupling to $\Omega_z$ and $m_z$ is crucial to the effect.


%\Red{supplementary materials: figure of all nonzero $\gamma$}

%
\begin{figure}[t]
\centering
\includegraphics[width=1\columnwidth]{fig/emcha_H.pdf}
\caption{(a) Valence band top of Te around H point, in a small range
  of K$^\prime$-H-K path.  (b) eMChA conductivity $\sigma_{zzzz}$ as
  function of energy. Color lines are components generated from Berry
  curvature $\Omega$, orbital moment $m^{\rm orb}$, and spin moment
  $m^{\rm spin}$.  (c)-(d) $k_z$ dependent quantities in the
  conductivities $\sigma^\Omega_{zzzz}$ and $\sigma^{\rm Z}_{zzzz}$ around H
  point alone principle axis $k_z$.  }
\figl{Te}
\end{figure}
%
As discussed earlier, the eMChA conductivity has three types of
contributions: orbital Zeeman, spin Zeeman, and Berry curvature.  The
$\sigma_{zzzz}$ response is strongly dominates by the orbital Zeeman
term, as shown in \fref{Te}(b).  To better understand this result,
% in the context of a longitudinal collinear response,
consider the $zzzz$ components of
\eqs{sigma-mc'}{sigma-Z},
%
\begin{align}
\sigma^{\rm \Omega}_{zzzz}=&  \tau^2\int \dk ( -v^2_z\partial_z\Omega_z +
v_z v_{zz}\Omega_z  \nonumber\\
&+ 3 v_{zx}\Omega_x v_z + 3 v_{zy}\Omega_y v_z ) f_0' \eql{z-mc'}\\
\sigma^{\rm Z}_{zzzz}=& \tau^2 \int \dk  (v_{zzz}m_z - v_z \partial_{zz} m_z) f_0'\,.
\eql{z-z}
\end{align}
%
Due to the $f'_0$ factor, only states lying on the Fermi surface
contribute at 0K.  The evolution with $k_z$ near the H point of each
term in the integrand of \eqs{z-mc'}{z-z} is shown in \fref{Te}(c-d).
Since $\Omega_x$ and $\Omega_y$ are negligible near the H point, the
terms $3 v_{zx}\Omega_x v_z$ and $3 v_{zy}\Omega_y v_z$ in \eq{z-mc'}
can be ignored.

In \fref{Te}(d), the magnitudes in $\sigma^{\rm Z}_{zzzz}$ are
computed for different origins of the magnetic moment (spin and
orbital).  The magnitude of the orbital magnetic moment
$m^{\rm orb}_z$ is much greater than that of the spin magnetic moment
$m^{\rm spin}_z$ in Te, albeit with a different sign~\cite{stepan18}.
That leads to the dominance of $m^{\rm orb}_z$-related terms in
$\sigma^{\rm Z}_{zzzz}$.  On the other hand, the valence band top
around H exhibits a weak ``camel-back'' shape, as shown in
\fref{Te}(a).  The velocity $v_z$ is very small in the range of the
weak ``camel-back'', but changes rapidly outside that region.
However, the magnitudes of higher-order derivatives of energy,
$v_{zz}$ and $v_{zzz}$, are not small due to the details of the
"camel-back" shape.  Using \eq{z-z}, $v_{zzz}m_z$ undergoes
significant changes with variations in $k_z$, while
$- \partial_{zz} m_z v_z$ is restrained by $v_z$ near the Fermi
energy.  The same applies to \eq{z-mc'}, where all the terms in
$\sigma^\Omega_{zzzz}$ are constrained by $v_z$.

To understand why
$-\sigma^\Omega_{zzzz} \ll \sigma^{\rm Z;orb}_{zzzz}$, we use a
two-band model that can fit the band structure. We present the details
of two-band model in the Supplementary Material \cite{SM}.  In the
model energy eigenvalue of the upper and lower bands are denoted as
$\epsilon^u$ and $\epsilon^l$ respectively, and there is an energy gap
$\epsilon_g = \epsilon^u - \epsilon^l$ between them.  For the upper
band, the relation between $\Omega^u$ and $m^{\rm orb,u}$ is
$m^{{\rm orb};u} = -\epsilon_g\Omega^u$, and the same with the lower
band.  By substituting this relation into \eqs{sigma-mc'}{sigma-Z}, we
find that the only quantity that impacts the competition between
$\sigma^{\rm Z;orb}_{zzzz}$ and $\sigma^\Omega_{zzzz}$ is the
dispersion of energy.  By adjusting the parameters, the $v_z$ of the
``camel-back'' shape can successfully suppress
$- \partial_{zz} m_z v_z$ and $\Omega_z$ related terms.  The
insensitivity to the parameter makes $v_{zzz}m_z$ dominating.

%
\begin{figure}[t]
\centering
\includegraphics[width=1.0\columnwidth]{fig/T-dependent.pdf}
\caption{Temperature dependence, with different acceptor
  concentrations, of (a) eMChA parameter $\gamma'_{zzzz}$, where the
  black circles denotes the experimental mesurment of the six samples
  in Ref.\cite{calavalle2022gate}.  (c) Ohmic conductivity
  $\sigma_{zz}$ and (d) eMChA conductivity $\sigma_{zzzz}$ starting
  from 10K to room temperature.  (b) $\sigma_{zzzz}$ and $\sigma_{zz}$
  as a function of energy at 10K. The gray dashed lines show different
  chemistry potentials with different acceptor concentrations.  }
\figl{T-dependent}
\end{figure}
%

The temperature-dependent $\gamma'_{zzzz}$, $\sigma_{zz}$ and
$\sigma_{zzzz}$ are shown in \fref{T-dependent}(a,c,d) respectively,
which are calculated using the temperature-dependent Fermi-Dirac
distribution function $f_0(T,u,\epsilon)$ at chemical potential $u$.
\fref{T-dependent}(b) displays $\sigma_{zzzz}$ and $\sigma_{zz}$ as
functions of energy.  As the hole carrier concentration increases in
$p$-Te, the Fermi level shifts downward, resulting in changes in
$\sigma_{zzzz}$ and $\sigma_{zz}$.  The specific Fermi level for
various carrier concentrations is shown in \fref{T-dependent}(b).

When the Fermi level is close to the energy gap, the carrier
concentration is low, and the sensitivity of $\gamma'_{zzzz}$ to
temperature becomes significant compared to higher carrier
concentrations.  In $\sigma_{zzzz}$, there is a high peak next to the
energy gap, leading to extreme changes due to temperature because of
the smearing of the peak.  The fact that $\sigma'_{zzzz}$ and
$1/\sigma_{zz}$ have slopes in a similar trend exacerbates the
sensitivity of $\gamma'_{zzzz}$ to temperature.  When
$N_{\rm a} = 7.4\cdot10^{18}$cm$^{-3}$, there is a curve with an
opposite slope in \fref{T-dependent}(d) because the chemical potential
is on the other side of the peak of $\sigma_{zzzz}$ (see
\fref{T-dependent}(b)), causing the change due to temperature to be
opposite to the other cases.


In the experimental measurement of $p$-Te nanowires in
Ref.\cite{calavalle2022gate}, eMChA was scanned over all possible
directions of the magnetic field, and it was confirmed that the
response is the largest when $\B$ is parallel to the applied current,
in agreement with our symmetry analysis.  The eMChA resistivity was
measured for six samples. However, to extract $\gamma'_{zzzz}$ the
exact geometrical sizes of the nanowires are needed, which were
determined only for one of them. Assuming that the size is the same
(at least in order of magnitude) for all nanowires, the
% measured
value of $\gamma'_{zzzz}$ for six nanowires at 10K may be estimated
within the range of
$[10^{-11},10^{-9}]~\text{m}^2/\text{T}/\text{A}$. The acceptor
concentration was estimated as $7.4\cdot10^{17} \text{cm}^{-3}$ using
Hall resistance $R_{\rm H}$,
%
\beq
N_{\rm a} = \frac{B}{ewR_{\rm H}}~,
\eql{Na-Hall}
\eeq
%
where $w$ is the width of nanowires.  As shown in
Fig\ref{fig:T-dependent}(a), our estimated value of $\gamma'_{zzzz}$ is
significantly lower for this carrier concentration.  However,
\eq{Na-Hall} is only strictly valid for a parabolic band
dispersion. However, we can see that the valence band near the H point
is strongly non-parabolic, which may lead to an inaccurate estimation
of the carrier concentration. Taking into account sensitivity of
$\gamma'_{zzzz}$ to the carrier concentration, one might get a better
agreement if $N_{\rm a}$ is determined more accurately.

In earlier measurements at room temperature \cite{Rikken19}, the
largest component of the $\gamma$ tensor was found to be in the range
of $[10^{-9},10^{-8}]~\text{m}^2/\text{T}/\text{A}$ with an acceptor
concentration of around $10^{16} \text{cm}^{-3}$. However it was
observed that the effect is maximized when the magnetic field is
perpendicular to the current.  This contradicts to the symmetry
analysis of Table \ref{tab:Te-sigma-emcha-sym}, and therefore cannot
be described as a bulk property of trigonal tellurium, regardless of
the microscopic mechanism implied.

\textit{Summary.}
%
In this letter, we derived equations for quantifying eMChA
in real materials using the Boltzmann-transport formalism withing the
constant relaxation time approximation.  We use $p$-Te as a research
platform and employ DFT and Wannier function interpolation to make
theoretical predictions for eMChA conductivity $\sigma_{abcd}$ and
measurable tensor $\gamma'_{abcd}$.  We find that with low acceptor
concentrations, the orbital Zeeman coupling is the primary origin of
of the eMChA response compared with spin Zeeman coupling and Berry
curvature contributions.  We cannot boast of an excellent agreement
with the experimental data. On the other hand we cannot conclude that
there is a well-established agreement between different experiments,
therefore we hope that our results will be a good reference and
motivation for further experiments.  The developed methodology is made
available via an open-source code in order to motivate further
investigations of other materials.


%%% Local Variables:
%%% mode: latex
%%% TeX-master: "pap"
%%% End:
