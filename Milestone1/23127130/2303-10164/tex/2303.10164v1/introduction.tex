%======================
%\section{Introduction}
\secl{intro}
%======================

\textit{Introduction.}
%
The electrical transport properties of nonmagnetic crystals change
under an applied magnetic field $\B$, leading to phenomena such as the
ordinary and planar Hall effects~\cite{hurd-book72}, and the ordinary
magnetoresistance~\cite{pippard1989magnetoresistance}.
  % PhysRevB.88.104412}
The ordinary Hall effect is linear in $\B$ while the planar Hall and
the ordinary magnetoresistance are quadratic, but all three are linear
in the electric field $\E$.  Electric responses that are quadratic in
$\E$ become allowed in noncentrosymmetric media, and have recently
attracted a great deal of attention. One example is the nonlinear
anomalous Hall effect, a transverse response that occurs at order
$E^2$~\cite{deyo-arxiv09,PhysRevLett.115.216806,kang2019nonlinear}.


At order $E^2B$, an effect known as electrical magnetochiral
anisotropy (eMChA)~\cite{Rikken01,atzori-chir21} appears in certain
acentric conductors.
%
\ism{Is it restricted to the gyrotropic crystal classes? We can
  figure it out from the Jahn symbol using TENSOR.}
%
In addition to the ordinary $B^2$ magnetoresistance, such materials
display an anomalous magnetoresistance that is bilinear in $B$ and in
the applied current $I$. Thus, their linear resistance has the form
%
\beq
R(B,I) = (1 + \beta B^2 + \gamma I B)R_0
\eql{R-B-I}
\eeq
%
(the tensorial character is ignored for now).  The eMChA coefficient
$\gamma$ vanishes in centrosymmetric media, and in chiral media it is
equal and opposite for right- and left-handed enantiomers. eMChA is
not restricted to chiral media, and is sometimes referred to by other
names such as current rectification effect, unidirectional
magnetoresistance, and nonreciprocal resistance.

The eMChA has been measured in several nonmagnetic chiral materials
including bismuth helices~\cite{Rikken01}, single-walled carbon
nanotubes~\cite{krstic-jcp02},
% chiral structured
molecular conductors~\cite{pop2014electrical}, and $p$-doped trigonal
tellurium~\cite{Rikken19},
% (hereafter referred to as $p$-Te),
as well as in the polar semiconductor BiTeBr~\cite{ideue-natphys17}.
A strong eMChA response has also been reported in two nominally
centrosymmetric materials --~the layered semimetal
ZrTe$_5$~\cite{wang-prl22} and the kagome metal
CsV$_3$Sb$_5$~\cite{guo2022switchable}~~- suggesting that they
actually assume broken-symmetry forms.



% Hall effect~\cite{PhysRev.95.1154}, % anomalous Hall
% % effect~\cite{RevModPhys.82.1539},
% and planar Hall
% effect~\cite{PhysRevLett.90.107201,PhysRevB.96.041110,PhysRevLett.119.176804}.
% While magnetotransport phenomena linear in the electric field $\E$
% have been thoroughly investigated, the breaking of inversion symmetry
% in the crystal structure opens the possibility of nonlinear
% (second-order in $\E$) effects. \tents{Already i}n the absence of
% $\B$, a transverse current can be observed \tents{in certain
%   noncentrosymmetric conductors} due to the 
% % associated with the so-called ``Berry curvature
% % dipole''~\cite{PhysRevLett.115.216806}.
% % \tents{If, in addition to broken spatial inversion,}
% % % In addition, when
% % time-reversal symmetry is also broken \tents{spontaneously} by % the
% % intrinsic magnetic order, a longitudinal nonlinear response named
% % unidirectional magnetoresistance~\cite{PhysRevLett.117.127202} can be
% % observed.
% %
% %
% Under an external magnetic field,
% %
% % \ism{Now you are back to the case where spatial inversion is broken
% %   but there is no spontaneously-broken time reversal, right?  Maybe
% %   remove the sentence about spontaneously-broken time reversal?}
% %
% the Lorenz force and the Zeeman coupling
% %
% \ism{Is that all? What about the $D$ factor?}
% %
% modify the motion of conduction electrons, leading to nonlinear
% magnetotransport effects.

% \color{magenta} One such effect is electrical magneto-chiral
% anisotropy (eMChA)~\cite{Rikken01}, which consists of a change in
% resistance that depends linearly on \st{$\E$ and} $\B$ and on the
% current density $\I$.  Quite generally, the linear resistance of a
% nonmagnetic crystal can be expressed as
% %
% \beq
% R(\B,\I) = R_0 (1 + \beta \B^2 + \gamma \B \cdot \I)~,
% \eql{R-B-I}
% \eeq
% %
% \color{black}



%   It has been observed in bismuth
%   helices~\cite{Rikken01}, chiral structured molecular
%   conductors~\cite{pop2014electrical}, a charge-ordered kagome
%   metal~\cite{guo2022switchable}, and chiral magnetic
%   systems~\cite{yokouchi2017electrical,PhysRevLett.122.057206}, and
%   recently in $p$-doped trigonal tellurium \tents{(Te)} as
%   well~\cite{Rikken19}.
% %
% \ism{Add Ref.~\cite{calavalle2022gate} (nanogune paper) here?}
% %
% The
% common feature of these systems is that they are all chiral,
% %
% \ism{Being chiral is sufficient but necessary. Essentially the same
%   effect has been observed in the nonchiral but polar semiconductor
%   BiTeBr \href{http://dx.doi.org/10.1038/nphys4056}{(see here)}. The
%   name eMChA has historical roots, but is inaccurate. Can we refer the
%   reader to a review paper at the end of this paragraph?}
% %
% and hence
% can exist in the form of either dexter (D) or laevus (L) handed
% enantiomers.  In particular, trigonal elemental Te, atoms are
% helically stacked along the z-axis with a helical period of three
% atoms based on a hexagonal lattice
% \cite{doi:10.1063/1.1743647}.\stm{maybe place this statement later\\
% \tents{IS: I agree.}}

%In 2019, Rikken and Avarvari \cite{Rikken19} observed a change in the resistance of trigonal $p$-doped tellurium, which %linearly depends on $\B$ and appiled current $\I$.  
%a \tents{change in the resistivity of }\st{ in second-order response to the electric field in} $p$-doped \st{chiral }
%trigonal Tellurium (Te) under room temperature, which \st{is} 
%linearly depends on $\E$ and $\B$. 
%\tents{This was the first observation of the electrical magneto-chiral anisotropy (eMChA) in bulk crystals, however the effect was previously known, for instance,  in bismuth helices\cite{Rikken01}.
%In trigonal elemental Te, atoms are helically stacked along the z-axis with a helical
%period of three atoms based on a hexagonal lattice \cite{doi:10.1063/1.1743647}.
%The chiral structure breaks the inversion symmetry and
%generates colossal chiral dependent axial Berry curvature and magnetic moment \cite{stepan18}.
%The total longitudinal electrical resistance of dexter (D) and 
%laevus (L) handed enantiomers are 
% The change in resistance is described by~\cite{Rikken01}
% %
% \ism{Added reference.  In Ref.~\cite{Rikken01} as well as other
%   references, they write $\beta$ instead of $\mu^2$. Maybe the $\mu^2$
%   in Ref.~\cite{Rikken19} was a typo?}
% %
% \beq
% R^\text{D/L} (\B,\I) = R_0 (1 + \mu^2 \B^2 + \gamma^\text{D/L} \B \cdot \I)~,
% \eql{R-DL}
% \eeq
% %
% where  $\gamma^\text{D} = - \gamma^\text{L}$ describes the eMChA of the two enantiomers, $\mu$ describes the magnetoresistance, and $R_0$ is the linear resistance in the absence of $\B$.
% %Ohmic resistance, magnetoresistance, and eMChA resistance are shown in \eq{R-DL} on the right-hand side. 
% %The measurable tensor $\gamma^{D/L}$ which is the ratio of eMChA resistance and Ohmic resistance, 
% %can be peeled off by reversing the magnetic field.
% %This chirality-dependent transport property is also found in chiral structured monoculars \cite{pop2014electrical}, 
% %charge-ordered kagome metal \cite{guo2022switchable}, and chiral magnetic systems \cite{yokouchi2017electrical,PhysRevLett.%122.057206}. 

Recently, Calavalle \textit{et al.}~\cite{calavalle2022gate} reported
the manipulation and detection of eMChA
% by measuring the longitudinal second-order response voltage
in tellurium nanowires at low
temperatures.
% They concluded that the chiral-induced spin polarization leads to
% the unique magnetotransport.  The longitudinal response is pure,
% because it avoids interference from Hall-like effects.
Motivated by these measurements and by those of Ref.~\cite{Rikken19},
we undertake in this work an \textit{ab initio} study of the eMChA
response in $p$-doped trigonal tellurium, hereafter referred to as
$p$-Te. As it crystallizes in the simplest-possible chiral structure
with only three atoms per cell, elemental tellurium is an ideal
prototype material for performing \textit{ab initio} calculations of
eMChA.

% \textit{Bulk theory of eMChA.}
\textit{Bulk tensorial description.} To formulate eMChA as an intrinsic
bulk response, we divide \eq{R-B-I} by the cross-section area $A$ of
the sample, converting from resistance $R$ and current $I$ to
resistivity $\rho$ and current density $j$. Dropping the $B^2$ term
and defining $\gamma'=\gamma A$, one obtains
%
\beq
\rho(B,j)=(1+\gamma' j B)\rho_0\,.
\eql{rho-B-j}
\eeq

While experimentally one applies currents and measures voltages,
microscopic theory deals with the current-density response to applied
fields. We therefore take as the staring point of our theoretical
treatment the relation
%
\beq
j_a =
\sigma_{ab}E_b +
\sigma_{abcd}E_b E_c B_d\,,
\eql{j-E-EEB}
\eeq
%
where $\sigma_{ab}$ the linear Ohmic conductivity and $\sigma_{abcd}$
is the eMChA conductivity.  Solving for $\E$ as a function of $\B$ and
$\j$ to order $Bj^2$, one arrives at the following tensorial
generalization of \eq{rho-B-j},
%
\beq
\rho_{ae}(\B,\j)=(\delta_{ab}+\gamma'_{abcd} j_c B_d)\rho_{be}\,,
\eeq
%
where $\rho_{ac}\sigma_{cb}=\delta_{ab}$ and
%
\beq
\gamma'_{abcd}=-\rho_{aa'}\sigma_{a'bb'd}\rho_{b'c}\,.
\eql{gamma-prime}
\eeq
%
Thus, the main task of a microscopic theory of eMChA is to evaluate
the tensors $\sigma_{ab}$ and $\sigma_{abcd}$ defined by \eq{j-E-EEB},
which go into \eq{gamma-prime} for the intrinsic eMChA tensor
$\gamma'$. Note that while $\sigma_{abcd}$ has intrinsic permutation
symmetry $\sigma_{abcd}=\sigma_{acbd}$, the relation
$\gamma'_{abcd}=\gamma'_{acbd}$ only holds if the Cartesian axes are
chosen as principal axes of $\rho_{ab}$.

% Introducing the linear Ohmic resistivity according to
% $\rho^0_{\alpha\gamma}\sigma^0_{\gamma\beta}=\delta_{\alpha\beta}$ and letting
% $j^0_\alpha=\sigma^0_{\alpha\beta}E_\beta$, we can recast \eq{j-E-EEB} as
% $j_\alpha=\sigma_{\alpha\beta}(B,j_0)E_\beta$, where

%%% Local Variables:
%%% mode: latex
%%% TeX-master: "pap"
%%% End:
