%======================
%\section{General Theory}
\secl{general}
%======================

\textit{Boltzmann-transport theory of eMChA.}
%
To develop a microscopic theory of eMChA in nonmagnetic conductors, we
resort to the
semiclassical Boltzmann formalism with Berry-curvature and
  spin/orbital-moment corrections, which are widely used to describe linear
and nonlinear (magneto)transport in
solids~\cite{RevModPhys.82.1959,gao-fp19}.  Within the constant
  relaxation time approximation adopted below, this formalism accounts
  for intrinsic band structure effects but leaves out extrinsic
  effects such as skew scattering and side jump.

In the Boltzmann formalism, the current density takes the form
\beq
{\bf j} = - \int \dk \, D \rdot f\,,
\eql{BB-j}
\eeq
%
where the integral is over the Brillouin zone,
$\dk\equiv d^3k/(2\pi)^3$, $\rdot$ is the electron velocity, $f$ is
the nonequilibrium distribution function, and $D\equiv 1+\OO  \cdot\B$,
with $\OO$ the Bloch-state Berry curvature. We use units where
$e=\hbar=m_e=1$, and omit band summations.

We determine $f$ by solving the Boltzmann transport equation in the
constant relaxation-time approximation,
%
\beq
\kdot\cdot \nabla_{\kk} f = \frac{f_0(\ek)-f}{\tau}\,,
\eql{boltzmann}
\eeq
%
where $f_0(\ek)$ is the equilibrium Fermi-Dirac distribution function
evaluated at the Zeeman-shifted band energy
$\ek=\epsilon_\kk -\bf{m}_\kk \cdot \B$, with $\m_\kk$ the total (spin
plus orbital) magnetic moment $\m_\kk$ of a Bloch
state~\cite{gao-prb17}.
%  
The velocities $\rdot$ and
$\kdot$ in real space and in reciprocal space
satisfy the coupled equations~\cite{RevModPhys.82.1959}
%
\begin{subequations}
\bea
\rdot &= \nabla_\kk \ek -\kdot \times \OO \eql{BB-r}\,, \\
\kdot &= -\E - \rdot \times \B  \eql{BB-k}\,,
\eea
\label{BB-rk}
\end{subequations}
%
which can be decoupled as
%
\begin{subequations}
\bea
D\rdot &= \nabla_\kk \ek +\E \times \OO+
\left( \nabla_\kk \ek \cdot\OO\right)\B  \eql{BB-Dr}\,, \\
-D\kdot &=  \E+ \nabla_\kk \ek \times\B + \left(\E\cdot\B\right)\OO~.
\eql{BB-Dk}
\eea
\label{BB-Drk}
\end{subequations}
%
% To  evaluate the conductivities $\sigma_{ab}$ and $\sigma_{abcd}$ in
% \eq{j-E-EEB},
Next, we expand $f$ in \eq{boltzmann} as
% powers of $\tau$ as
$f = \sum_{n=0}^\infty f_n \tau^n$ to obtain the recursive relation
%
\beq
f_{n+1} = -\tau\kdot\cdot\nabla_\kk f_n~,
\eql{tau-f}
\eeq
%
which is solved starting from $f_0=f_0(\ek)$ using \eq{BB-Dk} for
$\dot\kk$. Finally, the current density is obtained as a power series
in $\E$ and $\B$ by inserting the resulting expression for $f$ into
\eq{BB-j}, together with \eq{BB-Dr} for~$\dot\rr$.

At linear order in $\E$, the above procedure yields the familiar
result for the Ohmic conductivity,
%
\beq
\sigma_{ab} = \tau\int\dk \, v_a v_b f_0'\,,
\eql{sigma-ab}
\eeq
%
where
$f_0^\prime = d
f_0(\epsilon)/d\epsilon\vert_{\epsilon=\epsilon_\kk}$. The derivation
of the expression at order $\E^2\B$ for the eMChA conductivity
$\sigma_{abcd}$ is outlined in the Supplementary Material~\cite{SM}.
Here, it suffices to note that in nonmagnetic conductors the only
nonvanishing contribution comes from $f_2$, and so it is proportional
to $\tau^2$. In particular, there is no contribution coming from the
$\B$ dependence of the chemical potential, which is an even function
of $\B$ in crystals with time-reversal symmetry~\cite{gao-prb17}. The
resulting expression
%
\beq
\sigma_{abcd}=
\sigma^{\rm Z}_{abcd} + \sigma^{\rm \Omega}_{abcd}
\eql{sigma-MC}
\eeq
%
comprises the Zeeman (Z) and Berry curvature $(\Omega)$ terms
%
\beq
\sigma^{\rm Z}_{abcd}= \tau^2 \int \dk \,
\left( v_{abc}m_d - v_c \partial_{ab} m_d   \right)f_0^\prime
\eql{sigma-Z}
\eeq
%
and
%
\begin{align}
\sigma^{\rm \Omega}_{abcd} = -\tau^2 &\int \dk \,
\Big[ 2 v_{ab} v_c \Omega_d + v_a v_c\partial_b\Omega_d \nonumber\\
&-\delta_{cd} (2v_{ae} v_b \Omega_e  - v_a v_e\partial_b\Omega_e )\nonumber\\
&-\delta_{ad}(v_{be} v_c\Omega_e +  v_c v_e\partial_b\Omega_e)\nonumber\\
&-\varepsilon_{edf}\varepsilon_{abg} v_f \partial_e \Omega_g v_c  
\Big] f_0^\prime
\eql{sigma-mc'}
\,,
\end{align}
%
where $\partial_a\equiv\partial/\partial_{k_a}$,
$v_a \equiv \partial_a \epsilon_\kk$ is the band velocity,
$v_{ab} \equiv \partial_{ab} \epsilon_\kk$ is the inverse
effective-mass tensor, and
$v_{abc} \equiv \partial_{abc} \epsilon_\kk$. 
Note that only the part of $\sigma_{abcd}$ that is invariant under permutations 
of  indices $b$ and $c$ contributes to the physical effect, while 
in \eqs{sigma-Z}{sigma-mc'} the symmetry under $b\leftrightarrow c$ is not explicitly seen.
Therefore, we further use the combinations $(\sigma_{abcd}+\sigma_{acbd})/2$
for the 
component $\sigma_{abcd}$.
The Zeeman response can
be further separated into spin and orbital parts, yielding three types
of contributions in total. The spin-Zeeman response requires spin-orbit coupling, while the
orbital-Zeeman and Berry curvature responses are present even without
spin-orbit coupling.

In the above derivation, we did not consider the corrections to the
orbital moment and Berry curvature due to applied
fields~\cite{gao-prl14,gao-fp19}.
Such corrections only affect lower-order terms $\E^1\B^1$ (Hall effect)
and $\E^2\B^0$ (nonlinear Hall effect), which are purely transverse;
therefore, the longitudinal resistivity remains unchanged.
In this manuscript, we mainly focus on the longitudinal effect.

\Eqr{sigma-ab}{sigma-mc'} are the needed ingredients to evaluate the
intrinsic contributions to the bulk eMChA tensor
$\gamma'_{abcd}$ in \eq{gamma-prime}. Inversion symmetry implies
$\epsilon(\kk)=\epsilon(-\kk)$, $\m(\kk)=\m(-\kk)$, and
$\OO(\kk)=\OO(-\kk)$; as a result, every term in
\eqs{sigma-Z}{sigma-mc'} vanishes in centrosymmetric crystals,
yielding $\gamma'_{abcd}=0$ as expected. Since
$\sigma_{ab}$ is linear in $\tau$ and $\sigma_{abcd}$ is quadratic,
$\gamma'_{abcd}$ becomes independent of
$\tau$ in the constant relaxation time approximation.

In the following, we first outline the computational methods,
and then report and discuss the numerical results of our
\textit{ab initio} calculations for $p$-Te.



%%% Local Variables:
%%% mode: latex
%%% TeX-master: "pap"
%%% End:
