
%===========================%
% DOCUMENT SETTINGS         %
%===========================%
% draftversion = true makes %
%   single column           %
%===========================%
% draftversion = fal se makes%
%   double column           %
%===========================%
\def\draftversion{false}
%\def\draftversion{true}
%===========================%
\def\showall{true}  % Set to false to hide figures

\RequirePackage{ifthen}
\ifthenelse{\equal{\draftversion}{true}}{
  \documentclass[aps,prl,,10pt,galley,amsmath,amssymb,
                 longbibliography,superscriptaddress,nofootinbib]{revtex4-2}
  
% \usepackage{showlabels}
}{
  \documentclass[aps,prl,10pt,twocolumn,amsmath,amssymb,floatfix,
    longbibliography,superscriptaddress,nofootinbib]{revtex4-2}
}

% \usepackage[latin1]{inputenc}
\usepackage[british]{babel}
\usepackage[all]{xy}
\usepackage{amscd}
\usepackage{amssymb}
\usepackage{amsthm}
\usepackage{enumitem}
\usepackage{mathrsfs,bbm}
\usepackage{xcolor,graphicx}
\usepackage{graphics}
\usepackage{soul}
\usepackage{comment}
\usepackage[all]{xy}
\usepackage{amscd}
\usepackage{amssymb,amsmath,latexsym}
\usepackage{amsthm}
\usepackage{enumitem}
\usepackage{mathrsfs,bbm}
\usepackage{dsfont}
\usepackage{tikz-cd}
\usepackage[T1]{fontenc}
\usepackage[utf8]{inputenc}  
 %
%%%%%%%%%%%%%%%%%%%%%%%%%%%%%%%%%%
%pagestyle
%%%%%%%%%%%%%%%%%%%%%%%%%%%%%%%%%%
%\pagestyle{plain}
\textwidth=430pt
\headsep=.7cm
\evensidemargin=15pt
\oddsidemargin=15pt
\leftmargin=0cm
\rightmargin=0cm
%%
%%%%%%%%%%%%%%%%%%%%%%%
\newcommand*\fixitem {\item[]%
  \refstepcounter{enumi}\hskip-\leftmargin\labelenumi\hskip\labelsep}
\newtheorem*{mainthm}{Main Theorem}
\newtheorem*{mainthm1}{Theorem}
\newtheorem*{maincor}{Corollary}
\usepackage[colorlinks=true]{hyperref}
\DeclareMathOperator{\Forall}{\forall}
\DeclareMathOperator{\Exists}{\exists}
\DeclareMathOperator{\ord}{ord}
\newcommand{\phiD}{\varphi_D}
\newcommand{\phiDI}{\varphi_{\mathbf{D}_I}}
\newcommand{\phiDIj}{\varphi_{\mathbf{D}_I (j)}}
\newcommand{\phiH}{\varphi_H}
\newcommand{\phiTimes}{\phiD \otimes \phiH}
\newcommand{\phiTimesDI}{\varphi_{\mathbf{D}_I} \otimes \phiH}
\newcommand{\R}{\mathscr{A}}
\newcommand{\X}{\mathscr{X}}
\newcommand{\Xf}{\mathscr{X}_{(k_0 ,i)}[r_0]}
\newcommand{\Xfr}{\mathscr{X}_{(k_0,i)}[r]}
\newcommand{\hotimes}{\widehat{\otimes}}
\newcommand{\C}{\mathbb{C}_p}
\newcommand{\V}{\mathscr{V}}
\newcommand{\B}{\mathscr{B}}
\newcommand{\dualD}{\mathfrak{D}}
\newcommand{\Dg}{\mathbf{D}}
\newcommand{\DD}{\mathcal{D}^0}
\newcommand{\DDg}{\mathcal{D}}
\newcommand{\DV}{\mathcal{D}}
\newcommand{\W}{\mathscr{W}_N}
\newcommand{\Ao}{\mathbf{A}^\circ}
\newcommand{\AoK}{\mathbf{A}^\circ_{\K}}
\newcommand{\AK}{\mathbf{A}_{/\K}}
\newcommand{\OOO}{\mathscr{A}^\circ}
\newcommand{\K}{\mathcal{K}} 
\newcommand{\OK}{\mathcal{O}_{\K}}
\newcommand{\varprojlog}[1]{\underleftarrow{\log\!^{#1}}}
\newcommand{\T}{\mathscr{T}}
\newcommand{\TT}{\mathbf{T}}
\newcommand{\VV}{\mathbf{V}}
\newcommand{\HH}{\mathcal{H}}
\newcommand{\hh}{\mathcal{H}^+}
\newcommand{\HG}[2]{\mathcal{H}_{#1}(#2)}
\newcommand{\hhl}{\mathcal{H}^{+,[l]}}
\newcommand{\hhj}{\mathcal{H}^{+,[j]}}
\newcommand{\hhjj}{\mathcal{H}^{+,[l,l']}}
\newcommand{\GS}{G_{\mathbb{Q},S}}
\newcommand{\Rf}{R_{(k_0 ,i)}[r_0]}
\newcommand{\Rfr}{R_{(k_0 ,i)}[r]}
\newcommand{\parT}{\langle T\rangle}
\newcommand{\Zf}{Z_{(k_0 ,i)}[r_0]}
\newcommand{\Zfr}{\mathscr{Z}_{(k_0 ,i)}[r]}
\newcommand{\ZFf}{\mathscr{Z}_{(k_0 ,i)}[r_0]}
\newcommand{\ZFfr}{\mathscr{Z}_{(k_0 ,i)}[r]}
\newcommand{\ZF}{\mathscr{Z}}

\begin{document}

\title{
  Electrical magnetochiral anisotropy
  in trigonal tellurium from first principles}


\author{Xiaoxiong Liu} \affiliation{Physik-Institut, Universit\"at
  Z\"urich, Winterthurerstrasse 190, CH-8057 Z\"urich, Switzerland}
\affiliation{Shenzhen Institute for Quantum Science and Engineering and Department of Physics, Southern University of Science and Technology (SUSTech), Shenzhen 518055, China}
\affiliation{Quantum Science Center of Guangdong-Hong Kong-Macao Greater Bay Area (Guangdong), Shenzhen 518045, China}
\affiliation{Shenzhen Key Laboratory of Quantum Science and Engineering, Shenzhen 518055, China}
\affiliation{International Quantum Academy, Shenzhen 518048, China}  

  
\author{Ivo Souza} \affiliation{Centro de F{\'i}sica de Materiales,
  Universidad del Pa{\'i}s Vasco, 20018 San Sebasti{\'a}n,
  Spain} \affiliation{Ikerbasque Foundation, 48013 Bilbao, Spain}

\author{Stepan S. Tsirkin} \affiliation{Centro de F{\'i}sica de Materiales,
  Universidad del Pa{\'i}s Vasco, 20018 San Sebasti{\'a}n,
  Spain} \affiliation{Ikerbasque Foundation, 48013 Bilbao, Spain}

  \begin{abstract}
  Structural chirality gives rise to characteristic responses that
  change sign with the handedness of the crystal structure. One
  example is electrical magnetochiral anisotropy (eMChA), a change in
  resistivity that depends linearly on the applied current and on the
  magnetic field.  Motivated by recent measurements of a strong eMChA
  in $p$-doped trigonal tellurium, we carry out an \textit{ab initio}
  study of the eMChA response in this material as a function of
  temperature and doping concentration.  We use the semiclassical
  Boltzmann transport formalism within the constant relaxation-time
  approximation to express the bulk eMChA response tensor in terms of
  the energy dispersion, intrinsic magnetic moment, and Berry
  curvature of the conduction Bloch states.  We find that the orbital
  Zeeman coupling dominates the calculated response, with smaller
  contributions coming from the spin Zeeman coupling and from the
  Berry curvature, and that the effect is maximal when both the
  current and magnetic field are along the trigonal axis. 
  The calculated data shows a reasonable agreement with the
   experiments. We provide the open-source code
   to facilitate further \textit{ab initio} studies of eMChA in other
   materials.
%
%\tenti{[IS: Say some words about comparison with experiment?]}
%
% while Berry curvature and spin add a significantly smaller
% correction.  The developed methodology is made available via the
% open-source code WannierBerri.
\end{abstract}
\mydate
\pacs{}
\maketitle

\ifthenelse{\equal{\draftversion}{true}}{Color code: \tents{Stepan}, \tenti{Ivo}, \tenxl{Xiaoxiong}}{}

\section{Introduction}
\label{sec:introduction}
% \begin{itemize}
%     % Diffusion of FL
%     \item {\st{Diffusion of FL}}
%     % Security threats to FL
%     \item {\st{Security threats to FL with particular focus on model poisoning}}
%     % Limitations of existing countermeasures
%     \item {\st{Current countermeasures (e.g., KRUM) and their limitations}}
%     % Proposed method and its advantages
%     \item {\st{Intuitive description of the proposed method and its difference (i.e., advantages) w.r.t. state of the art}}
%     % Main contributions
%     \item {\st{Summary of the main contributions of this work}}
%     % Paper's structure and organization
%     \item {\st{Paper's structure and organization}}
% \end{itemize}

% Diffusion of FL
Recently, {\em federated learning} (FL) has emerged as the leading paradigm for training distributed, large-scale, and privacy-preserving machine learning (ML) systems~\cite{mcmahan2017googleai,mcmahan2017aistats}. 
The core idea of FL is to allow multiple edge clients to collaboratively train a shared, global model without disclosing their local private training data.
%Specifically, an FL system consists of a central server and many edge clients; 
A typical FL round involves the following steps: {\em(i)} the server randomly picks some clients and sends them the current, global model; {\em(ii)} each selected client locally trains its model with its own private data; then, it sends the resulting local model to the server;\footnote{Whenever we refer to global/local model, we mean global/local model {\em parameters}.} {\em(iii)} the server updates the global model by computing an \emph{aggregation function}, usually the average (FedAvg), on the local models received from clients.
% \begin{enumerate}
%     \item[{\em(i)}] the server sends the current, global model to the clients and appoints some of them for training;
%     \item[{\em(ii)}] each selected client locally trains its copy of the global model with its own private data; then, it sends the resulting local model back to the server;\footnote{Whenever we refer to global/local model, we mean global/local model {\em parameters}.}
%     \item[{\em(iii)}] the server updates the global model by computing an \emph{aggregation function} on the local models received from clients (by default, the average, also referred to as FedAvg~\cite{mcmahan2017aistats}).
% \end{enumerate}
This process goes on until the global model converges. %(e.g., after a certain number of rounds or other similar stopping criteria).
%\\
% The advantages of FL over the traditional, centralized learning paradigm are undoubtedly clear in terms of flexibility/scalability (clients can join/disconnect from the FL network dynamically), network communications (only model weights\footnote{We will use \textit{parameters} and \textit{weights} interchangeably.} are exchanged between clients and server), and privacy (each client's private training data is kept local at the client's end and not uploaded to the server).
\\
% Security threats to FL
%However, the growing adoption of FL also raises security concerns~\cite{costa2022covert}, particularly about its confidentiality, integrity, and availability.
Although its advantages over standard ML, FL also raises security concerns~\cite{costa2022covert}. %, particularly about its confidentiality, integrity, and availability~\cite{costa2022covert}.
% OLD, LONG VERSION
% Indeed, some work deals with privacy leakage that may expose the local data of some clients~\cite{melis2019sp}. 
% A large body of work, instead, investigates attacks that usually aim to detriment the predictive accuracy of the learned global model. For instance, \emph{data poisoning} attacks achieve this goal by letting an adversary pollute the training set of some corrupt FL clients with maliciously crafted examples~\cite{jagielski2018sp}.
% Similarly, in \emph{model poisoning} the attacker attempts to tweak the global model weights~\cite{bhagoji2019pmlr} by directly perturbing the local model's weights of some infected FL clients before these are sent to the central server for aggregation, usually via so-called Byzantine attacks. 
% It turns out that Byzantine model poisoning attacks severely impact standard FedAvg; therefore, more robust aggregation functions must be designed to make FL systems secure.
Here, we focus on \emph{untargeted model poisoning} attacks~\cite{bhagoji2019pmlr}, where an adversary attempts to tweak the global model weights %\footnote{We will use the terms \textit{parameters} and \textit{weights} interchangeably.} 
by directly perturbing the local model's parameters of some infected clients before these are sent to the central server for aggregation.
In doing so, the adversary aims to jeopardize the global model \textit{indiscriminately} at inference time.
Such model poisoning attacks severely impact standard FedAvg; therefore, more robust aggregation functions must be designed to secure FL systems.
\\
% In this paper, we focus on designing a novel robust aggregation scheme at the server's end to contrast the effect of Byzantine model poisoning attacks.
%
% Current countermeasures and their limitations
%Several countermeasures have been proposed in the literature to combat model poisoning attacks on FL systems.
% Some methods use simple statistics more robust than plain average to smooth the impact of malicious updates (e.g., Trimmed Mean and FedMedian~\cite{yin2018icml}). 
% Other defenses implement outlier detection techniques to discard malicious updates from the aggregation performed at the server's end. Those are either based on heuristics (e.g., Krum/Multi-Krum~\cite{blanchard2017nips} and Bulyan~\cite{mhamdi2018pmlr}) or data-driven approaches (e.g., K-means clustering~\cite{shen2016acm} or DnC via spectral analysis~\cite{shejwalkar2021ndss}). 
% Finally, some strategies rely on a centralized ``source of trust'' to spot potential malicious updates (e.g., FLTrust~\cite{cao2020fltrust}).
% Several countermeasures have been proposed in the literature to combat model poisoning attacks on FL systems, i.e., to discard possible malicious local updates from the aggregation performed at the server's end. 
% These techniques range from simple statistics more robust than plain average (e.g., Trimmed Mean and FedMedian~\cite{yin2018icml}) to outlier detection heuristics (e.g., Krum/Multi-Krum~\cite{blanchard2017nips} and Bulyan~\cite{mhamdi2018pmlr}) or data-driven approaches (e.g., spectral analysis via K-means clustering~\cite{shen2016acm} or spectral analysis), or methods based on ``source of trust'' (e.g., FLTrust~\cite{cao2020fltrust}).
% OLD, LONG VERSION
%Several countermeasures have been proposed in the literature to combat Byzantine model poisoning attacks on FL systems.
% Descriptive statistics
% For example, Trimmed Mean and FedMedian aggregate local model updates using more robust statistics than standard average~\cite{yin2018icml}.
%
% % Heuristics for outlier detection
% Many existing Byzantine-resilient strategies implement some outlier detection heuristics to discard the model updates sent by potentially malicious clients from the input of the aggregation function.
% One of the most popular heuristics is Krum~\cite{blanchard2017nips}.
% This strategy tries to mitigate the impact of Byzantine attacks by selecting as a global model the local model with the smallest sum of Euclidean distances to {\em all} the other local models.
% Although powerful, Krum requires the server to know (or, at least, estimate) the number of malicious FL clients upfront, which is generally impossible in a realistic attack scenario. %
% Moreover, Krum may become ineffective for complex, high-dimensional model parameter spaces due to the curse of dimensionality.
% Bulyan~\cite{mhamdi2018pmlr} tries to overcome this issue by combining Krum with a variant of Trimmed Mean.
% % Data-driven outlier detection
% Other strategies use data-driven outlier detection techniques -- e.g., via K-means clustering~\cite{shen2016acm} -- to spot potential malicious local model updates. 
% %For instance, Shen et al. propose to cluster local model updates with K-means and thus identify outliers.
%
% % Other techniques
% As far as the server is concerned, any local model received can be from a potential malicious client. 
% FLTrust~\cite{cao2020fltrust} assumes the server acts as a client, i.e., trains a local model on an additional {\em trustworthy} dataset at the server's end and compares it against all the local models from other clients. 
% This way, the server can rely on some ``source of trust'' when discarding potentially malicious clients.
%\\
% Limitations of existing Byzantine-resilient strategies
Unfortunately, existing defense mechanisms either rely on simple heuristics (e.g., Trimmed Mean and FedMedian by~\cite{yin2018icml}) or need strong and unrealistic assumptions to work effectively (e.g., foreknowledge or estimation of the number of malicious clients in the FL system, as for Krum/Multi-Krum~\cite{blanchard2017nips} and Bulyan~\cite{mhamdi2018pmlr}, which, however, cannot exceed a fixed threshold).
Furthermore, outlier detection methods using K-means clustering~\cite{shen2016acm} or spectral analysis like DnC~\cite{shejwalkar2021ndss} do not directly consider the temporal evolution of local model updates received.
Finally, strategies like FLTrust~\cite{cao2020fltrust} require the server to collect its own dataset and act as a proper client, thereby altering the standard FL protocol.
\\
% OLD, LONG VERSION
% Overall, existing Byzantine-resilient strategies are either simple heuristics (e.g., FedMedian) or, if they are more complex, they rely on strong and unrealistic assumptions to work effectively (e.g., knowing the number of malicious clients in the FL system in advance, as for Krum and alike).
% Furthermore, data-driven outlier detection methods do not consider the temporary evolution of local model updates received (e.g., K-means clustering). 
% Finally, strategies like FLTrust requires the server to collect its own dataset and act as a proper client, thereby altering the standard FL protocol.
%
% Description of the proposed method
This work introduces a novel pre-aggregation \textit{filter} robust to untargeted model poisoning attacks. Notably, this filter $(i)$ operates without requiring prior knowledge or constraints on the number of malicious clients and $(ii)$ inherently integrates temporal dependencies. 
The FL server can employ this filter as a preprocessing step before applying \textit{any} aggregation function, be it standard like FedAvg or robust like Krum or Bulyan.
Specifically, we formulate the problem of identifying corrupted updates as a multidimensional (i.e., matrix-valued) time series anomaly detection task. 
The key idea is that legitimate local updates, resulting from well-calibrated iterative procedures like stochastic gradient descent (SGD) with an appropriate learning rate, show \textit{higher predictability} compared to malicious updates. This hypothesis stems from the fact that the sequence of gradients (thus, model parameters) observed during legitimate training exhibit regular patterns, as validated in Section~\ref{subsec:intuition}. %until convergence. 
%This regularity may be more pronounced for smooth convex loss functions, but it can still be captured within an appropriate time window, even for more complex and convoluted loss surfaces. 
%We provide evidence of this claim in Appendix~B, where we show that the average mutual information (i.e., ``predictability''), calculated over pairs of legitimate model updates sent at different FL rounds, is significantly higher than the corresponding computation for a malicious client.
\\
Inspired by the matrix autoregressive (MAR) framework for multidimensional time series forecasting~\cite{chen2021je}, we propose the FLANDERS ({\em \textbf{F}ederated \textbf{L}earning meets \textbf{AN}omaly \textbf{DE}tection for a \textbf{R}obust and \textbf{S}ecure}) filter.
The main advantages of FLANDERS over existing strategies like FLDetector~\cite{zhao2020multivariate} are its resilience to large-scale attacks, where $50\%$ or more FL participants are hostile, and the capability of working under realistic non-iid scenarios.
We attribute such a capability to two key factors: $(i)$ FLANDERS works without knowing a priori the ratio of corrupted clients, and $(ii)$ it embodies temporal dependencies between intra- and inter-client updates, quickly recognizing local model drifts caused by evil players. Below, we summarize our main contributions:

\begin{itemize}
\item[{\em(i)}]
We provide empirical evidence that the sequence of models sent by legitimate clients is more predictable than those of malicious participants performing untargeted model poisoning attacks.
\\
\item[{\em(ii)}] 
We introduce FLANDERS, the first pre-aggregation filter for FL robust to untargeted model poisoning based on multidimensional time series anomaly detection.
\\
\item[{\em(iii)}] 
We integrate FLANDERS into Flower,\footnote{\scriptsize{\url{https://flower.dev/}}} a popular FL simulation framework for reproducibility.
\\
\item[{\em(iv)}] 
We show that FLANDERS improves the robustness of the existing aggregation methods under multiple settings: different datasets, client's data distribution (non-iid), models, and attack scenarios.
\\
\item[{\em(v)}] 
We publicly release all the implementation code of FLANDERS along with our experiments.\footnote{\scriptsize{\url{https://anonymous.4open.science/r/flanders_exp-7EEB}}}
\end{itemize}

% Paper's structure and organization
The remainder of the paper is structured as follows. %some related work and the current state-of-the-art solutions to security issues that FL entails. 
Section~\ref{sec:background} covers background and preliminaries. 
In Section~\ref{sec:related}, we discuss related work.
Section~\ref{sec:problem} and Section~\ref{sec:method} describe the problem formulation and the method proposed. % to tackle it. 
Section~\ref{sec:experiments} gathers experimental results. %, and Section~\ref{sec:limitations} discusses some limitations of this work.
Finally, we conclude in Section~\ref{sec:conclusion}.
 %discusses the limitations of this work and draws future research directions.
%reports conclusions and draws perspectives for future research directions.

%%%%%%% OLD %%%%%%%
%to overcome the resilience of Byzantine failures in distributed Stochastic Gradient Descent computations. 
% The strength of Krum is its time complexity, which is linear in the gradient dimension. 
% However, the robustness of the approach is guaranteed for gradient-based learning applications only when the majority of the clients are not compromised. 
% Besides, the aggregation mechanism of Krum, as well as that of similar methods, is robust from a coarse-grained perspective and does not provide solutions to errors and perturbations that may occur at inference time.
%A related approach to~\cite{blanchard2017nips} is the work of Su et al.~\cite{su2016dc}. Here, the authors propose an iterated approximate agreement to tackle a multi-layer scenario attacked by Byzantine agents. 
%However, the method works efficiently on the sole discrete context and it is inapplicable to continuous state environments.
%\gabri{Maybe, we should just talk about the main limitations of existing countermeasures without digging into their details (or, we can just mention Krum as this is the most popular one). I will move the description of all these methods to the Related Work section.}

%======================
%\section{General Theory}
\secl{general}
%======================

\textit{Boltzmann-transport theory of eMChA.}
%
To develop a microscopic theory of eMChA in nonmagnetic acentric
crystals, we resort to the Berry-curvature-corrected semiclassical and
Boltzmann formalism, which has been widely used to describe nonlinear
(magneto)transport effects in
solids~\cite{deyo-arxiv09,PhysRevLett.115.216806,gao-prb17}.

In the Boltzmann formalism, the current density is expressed as the
integral of the electron velocity $\rdot$ weighted by the
non-equilibrium distribution function $f$,\footnote{For brevity we
  use units where $e=\hbar=m_e=1$, and omit band summations.}
%
\beq
{\bf j} = - \int \dk D \rdot f ~ \eql{BB-j}
\eeq
%
where $D=1+\OO\cdot\B$~\cite{PhysRevLett.95.137204}. We determine
$f$ by solving the Boltzmann equation in the constant
relaxation time approximation,
%
\beq
\kdot\cdot \nabla_{\kk} f = \frac{f_0-f}{\tau}\,,
\eeq
%
where $f_0$ is the equilibrium Fermi-Dirac distribution, \tents{whose
  argument is the band energy $\ek=\epsilon_\kk -\bf{m}_\kk \cdot \B$
  modified by the Zeeman coupling to the total (spin plus orbital)
  intrinsic magnetic moment $\m_\kk$ of a Bloch
  state~\cite{gao-prb17}}.  The velocity $\rdot$ and the rate of
change $\kdot$ in crystal momentum satisfy the coupled
equations~\cite{RevModPhys.82.1959}
%
\begin{subequations}
\bea
\rdot &= \nabla_\kk \ek -\kdot \times \OO \eql{BB-r}\,, \\
\kdot &= -\E - \rdot \times \B  \eql{BB-k}\,,
\eea
\label{BB-rk}
\end{subequations}
%
where $\OO$ is the Berry curvature.
%
% the velocity includes the anomalous velocity from Berry curvature $\OO$ \tents{ and $\kdot$ is affected by the Lorentz force.} 
% %In order to study the generation of eMChA in Te,  we have to \st{clear}\tents{clarify} the motion of electrons.
% \tents{Note that} in the presence of an external magnetic field the energy of the states the electron energy eigenvalues $\epsilon_\kk$ are modified by the Zeeman coupling as $\ek=\epsilon_\kk -\bf{m}_\kk \cdot \B$, where $\bf{m}_\kk$ is the magnetic (spin and orbital) moment of Bloch electrons.
%In Berry-Boltzmann equations, the velocity includes $\nabla_\kk \ek$ and anomalous velocity from Berry curvature $\OO$ 
%$\rdot$ and $\kdot$ are coupled through the anomalous velocity and the Lorentz force. 
% \tents{The formalism described above is sometimes called ``Berry-Boltzmann equations''\cite{PhysRevLett.115.117403}}
%
The two equations can be decoupled as
%
\begin{subequations}
\bea
D\rdot &= \nabla_\kk \ek +\E \times \OO+
\left( \nabla_\kk \ek \cdot\OO\right)\B  \eql{BB-Dr}\,, \\
-D\kdot &=  \E+ \nabla_\kk \ek \times\B + \left(\E\cdot\B\right)\OO~.
\eql{BB-Dk}
\eea
\label{BB-Drk}
\end{subequations}
%
To evaluate the conductivities $\sigma_{ab}$ and $\sigma_{abcd}$ in
\eq{j-E-EEB}, we expand $f$ in powers of $\tau$ as
$f = \sum_n f_n \tau^n$ using the recursive relation
%
\beq
f_{n+1} = -\tau\kdot\cdot\nabla_\kk f_n~,
\eql{tau-f}
\eeq
%
\stm{should we write $\ff$, or can we live with just $f$ }
\xlm{$\ff$ means including Zeeman effect. It is necessary to keep it because there is a Zeeman term from $\ff$. Emmm, do we need to change all the above from $f$ to $\ff$? I agree it looks confused from \eq{BB-j} to \eq{tau-f}. \\
  \tents{IS: I think we can live just with $f$, provided we make the
    provise in the magenta text: see below Eq.~(4) in
    Ref.~\cite{gao-prb17}.}}
%
with $\dot\kk$ given by \eq{BB-Dk}. The current density is then
obtained by inserting the resulting expression for $f$ into \eq{BB-j},
together with \eq{BB-r} for $\dot\rr$.

The above procedure yields the standard result
%
\beq
\sigma_{ab} = \tau\int\dk v_a v_b f_0'
\eql{sigma-ab}
\eeq
%
for the Ohmic conductivity. The derivation of the expression for the
eMChA conductivity $\sigma_{abcd}$ is given in the Supplementary
Material~\cite{SM}.
% We present the detailed derivation in the Supplementary Material
% \cite{SM},
Here, it suffices to note that in systems preserving time-reversal
symmetry the only non-vanishing contribution % to $\sigma^{\rm eMChA}$
comes from $f_2$, and so it is proportional to $\tau^2$. (In
particular, there is no contribution from the dependence of the
chemical potential on $\B$, because when the unperturbed crystal has
time-reversal symmetry, the magnetic field modifies the chemical
potential at order $B^2$ only~\cite{gao-prb17}.) The resulting
expression comprises Zeeman (Z) and Berry curvature $(\Omega)$ terms,
%
\beq
\sigma_{abcd}=
\sigma^{\rm Z}_{abcd} + \sigma^{\rm \Omega}_{abcd}
\eql{sigma-MC}
\eeq
%
given by
%
\ism{For clarity, in every term I have put the derivatives at the end,
  and then I ordered the Cartesian indices.}
%
\beq
\sigma^{\rm Z}_{abcd}= \tau^2 \int \dk
\Big[ v_{abc}m_d - v_c \partial_{ab} m_d   \Big]f_0^\prime
\eql{sigma-Z}
\eeq
%
and
%
\begin{align}
\sigma^{\rm \Omega}_{abcd} = -\tau^2 &\int \dk
\Big[ 2( v_{ab} v_c \Omega_d + v_a v_c\partial_b\Omega_d) \nonumber\\
&-2\delta_{cd} v_{ae} v_b \Omega_e \nonumber\\
&-\delta_{ad}(v_{be} v_c\Omega_e +  v_c v_e\partial_b\Omega_e)
\eql{sigma-mc'}\\
&+\delta_{dc} v_a v_e\partial_b\Omega_e - v_a v_c\partial_b\Omega_d \nonumber
%&-\epsilon_{\xi\mu\rho}\epsilon_{\beta\theta\alpha}( v_{\xi\rho}\Omega_\theta v_\gamma + v_\xi\partial_\rho\Omega_\theta v_\gamma ) 
\Big] f_0^\prime \nonumber ~,
\end{align}
%
where
$f_0^\prime = d f_0(\epsilon)/d\epsilon\vert_{\epsilon=\epsilon_\kk}$,
$\partial_a$ denotes $\partial/\partial_{k_a}$,
$v_a = \partial_a \epsilon_\kk$ is the band velocity,
$v_{ab} = \partial_{ab} \epsilon_\kk$, and
$v_{abc} = \partial_{abc} \epsilon_\kk$.  The Zeeman term in
\eq{sigma-Z} can be further separated into spin and orbital parts,
resulting in a total of three types of contributions. Of these, only
the spin-Zeeman response requires spin-orbit coupling.

From the constraints $\epsilon(\kk)=\epsilon(-\kk)$,
$\m(\kk)=\m(-\kk)$, and $\OO(\kk)=\OO(-\kk)$ imposed by the presence
inversion symmetry, it follows that every term in
\eqs{sigma-Z}{sigma-mc'} vanishes in centrosymmetric crystals as
expected. Moreover, since the Ohmic conductivity is linear in $\tau$
and the eMChA conductivity is quadratic, the intrinsic eMChA
coefficient defined by \eq{gamma-prime} becomes independent of $\tau$
in the constant relaxation time approximation.


% %  only by  The longitudinal eMChA response current should be generated from a symmetric conductivity \cite{Ohmic-Hall}, with $\ff_2$.
% If we neglect the Zeemann coupling, the  $\sigma^{\rm eMChA}$ is proportional to the Berry curvature $\OO$ of the Bloch bands and given by 
% %Assuming $\ek = \epsilon_\kk$ first,  the Berry curvature related conductivity $\sigma^{\rm \Omega}_{\alpha\beta\gamma\mu}$ is aggregated into 
% %a function of $\OO, \epsilon_\kk,f_0^\prime = d f_0(\epsilon_\kk)/d\epsilon_\kk$,
% % \begin{align}
% % \sigma^{\rm \Omega}_{\alpha\beta\gamma\mu} = -\tau^2 &\int \dk \Big[ 2( v_{\alpha\beta}  \Omega_\mu v_\gamma + v_\alpha\partial_\beta\Omega_\mu v_\gamma) \nonumber\\
% % &-2\delta_{\gamma\mu} v_{\alpha\rho} \Omega_\rho v_\beta \nonumber\\
% % &-\delta_{\alpha\mu}(v_{\beta\rho}\Omega_\rho v_\gamma + v_\rho\partial_\beta\Omega_\rho v_\gamma )  \eql{sigma-mc'}\\
% % &+\delta_{\mu\gamma} v_\alpha\partial_\beta\Omega_\rho v_\rho - v_\alpha\partial_\beta\Omega_\mu v_\gamma \nonumber
% % %&-\epsilon_{\xi\mu\rho}\epsilon_{\beta\theta\alpha}( v_{\xi\rho}\Omega_\theta v_\gamma + v_\xi\partial_\rho\Omega_\theta v_\gamma ) 
% % \Big] f_0^\prime \nonumber ~,
% % \end{align}
% %
% \begin{align}
% \sigma^{\rm \Omega}_{abcd} = -\tau^2 &\int \dk
% \Big[ 2( v_{ab}  \Omega_d v_c + v_a\partial_b\Omega_d v_c) \nonumber\\
% &-2\delta_{cd} v_{ae} \Omega_e v_b \nonumber\\
% &-\delta_{ad}(v_{be}\Omega_e v_c + v_e\partial_b\Omega_e v_c)
% \eql{sigma-mc'}\\
% &+\delta_{dc} v_a\partial_b\Omega_e v_e - v_a\partial_b\Omega_d v_c \nonumber
% %&-\epsilon_{\xi\mu\rho}\epsilon_{\beta\theta\alpha}( v_{\xi\rho}\Omega_\theta v_\gamma + v_\xi\partial_\rho\Omega_\theta v_\gamma ) 
% \Big] f_0^\prime \nonumber ~,
% \end{align}
% %
% \xlm{removed the cross product term\\
%   \tents{IS: What's the deal with that term?
%     It should not be there? It vanishes?}}
% %
% where 
% .
% The Zeeman coupling to the external magnetic field will contribute a
% correction proportional to the magnetic moment of Bloch
% electrons
% % $\sigma^{\rm Z}_{\alpha\beta\gamma\mu}$, based on the symmetric
% % conductivity which is a second order of $\E$ without magnetic field,
% %
% %
% where 
% %The magnetic moment $\bf m$ consists of orbital moment ${\bf m}^{orb}$ and spin moment ${\bf m}^{spin}$.
% Due to time-reversal symmetry $\bf m(\kk) = \bf - m(-\kk)$, hence the
% Zeeman
% %splitting energy $\bf{m} \cdot \B$ have different signs on $\pm \kk$ as well.
% coupling to the magnetic field modifies the chemical potential only at
% order $\propto \B^2$ which does not contribute to eMChA.
% % So the Fermi energy does not change due to the external $\B$.
% % $\sigma^{\rm \Omega}_{\alpha\beta\gamma\mu}$ and
% % $\sigma^{\rm Z}_{\alpha\beta\gamma\mu}$
% Thus, the entire eMChA conductivity is given by the sum
% %

% Note that each term of \Eqs{sigma-mc'}{sigma-Z} shows that eMChA is 
% %ubiquitous without 
% %centrosymmetry in each term. It declares the nonzero eMChA conductivity only requests 
% generally non-vanishing in any system that breaks the inversion symmetry.
% %But the chiral structure generates colossal axial Berry curvature and magnetic moment to make it easier to observe. 

%In the experiments, constant currents $\pm j_z$ were actived in Te nanowires.
%There are charality dependent response current detected as the difference in voltage 
%by reverse dirving current $j_z$ and reversing the magnetic field.

%The difference in voltage can be shown by the difference in the electric field ($E_z^+ - E_z^-$), which is solved from a quadratic equation, 
%\bea
%j_z =& \sigma^{\rm O}_{zz}E_z^+ + \sigma^{\rm MR}_{zzz}E_z^+ B_z^2 + \sigma^{\rm MC}_{zzzz}(E_z^+)^2 B_z\\
%j_z =& \sigma^{\rm O}_{zz}E_z^- + \sigma^{\rm MR}_{zzz}E_z^- B_z^2 - \sigma^{\rm MC}_{zzzz}(E_z^-)^2 B_z~.
%\eql{measure-current}
%\eea
%Ohmic conductivity $\sigma^{\rm O}$, and magnetoresistance conductivity $\sigma^{\rm MR}$ are magnetic field even. Assuming
%$\sigma^{\rm MR} \ll \sigma^{\rm O}$, the parameter $\gamma_{zzzz} = \sigma^{\rm MC}_{zzzz}/(\sigma^{\rm O}_{zz})^2$.




%%% Local Variables:
%%% mode: latex
%%% TeX-master: "pap"
%%% End:


%======================
%\section{eMChA in Tellurum}
\secl{Te}
%======================

\textit{Computational methods.}
The unit cell of trigonal Te contains three atoms disposed along a
spiral chain, with the chains arranged on a 2D hexagonal lattice.
%
The fully-relativistic electronic structure of the pristine (undoped)
%right-handed crystal (space group P3$_1$21) is evaluated via
right-handed crystal (space group P3$_1$21) is evaluated via
density-functional theory (DFT) in the pseudopotential framework, using the
HSE06 hybrid functional~\cite{paier-jcp06} implemented in the VASP
code package
\cite{kresse1999ultrasoft,kresse1996efficiency,PhysRevB.54.11169}.


In order to evaluate the needed $k$-space quantities accurately and
efficiently, we employ the Wannier interpolation
scheme~\cite{wang-prb06}.  The Wannier functions are constructed using
the Wannier90 code package~\cite{wannier90}, starting from
atom-centered $s$ and $p$ trial orbitals.  In order to avoid artifacts
due to possible numerical violations of crystal symmetries by the
Wannier model, the Wannier Hamiltonian and orbital-based matrix
elements are
symmetrized so as to satisfy the point-group symmetries.

We have implemented all terms in \eqs{sigma-Z}{sigma-mc'} in the
WannierBerri code package~\cite{wannierberri}.  Wannier interpolation
allows us to evaluate band properties and their momentum-space
derivatives directly, without finite difference
schemes 
~\cite{gradients}.
We perform the Brillouin-zone integrals on a symmetry-irreducible part
of a regular grid of $300\times300\times300$ points, using 30 adaptive
refinement iterations, and employing the tetrahedron 
method~\cite{tetrahedronmethod,tetrahedronmethod2}
to accurately describe the Fermi occupation factors \tenxl{$-f_0'$}.


\textit{Numerical results.}
The eMChA conductivity $\sigma_{abcd}$ is a nonmagnetic polar tensor
symmetric in $bc$, with the Jahn symbol
eV[V$^2$]V. Table~\ref{tab:Te-sigma-emcha-sym} shows the constraints
imposed on that tensor by the point group symmetry of trigonal
tellurium. 

\begin{table}[t]
\caption{Symmetry constraints on the components of the eMChA
    conductivity tensor $\sigma_{abcd}$ in crystal class
    32. The three largest calculated components in trigonal tellurium
    are highlighted in bold. The table was generated using the
$\rm MTENSOR$  routine~\cite{mTensor} provided by the
Bilbao Crystallographic Server~\cite{mTensor-link}.
}
\centering % centering table
\makebox[\columnwidth]{\begin{tabular}{|c|c|cccccc|}
  \hline
  \multicolumn{2}{|c|}{32}& \multicolumn{6}{c|}{$bc=cb$}\\
  \cline{3-8}
    \multicolumn{2}{|c|}{$\sigma_{abcd}$} &$xx$&$yy$&$zz$&$yz$&$xz$&$xy$\\
  \hline
         &$xx$& $\sigma_{yyyy}$ & $\sigma_{yxxy}$& $\boldsymbol{\sigma_{yzzy}}$& $-\sigma_{yyzy}$& 0               & 0                 \\
         &$xy$& 0               & 0              & 0              & 0               & $-\sigma_{yyzy}$& $(\sigma_{yyyy}-\sigma_{yxxy})/2$ \\
         &$xz$& 0               & 0              & 0              & 0               & $\sigma_{yyzz}$ & $-\sigma_{yyyz}$  \\
         &$yx$& 0               & 0              & 0              & 0               & $-\sigma_{yyzy}$& $(\sigma_{yyyy}-\sigma_{yxxy})/2$ \\
 $ad$&$yy$& $\sigma_{yxxy}$ & $\sigma_{yyyy}$& $\boldsymbol{\sigma_{yzzy}}$& $\sigma_{yyzy}$ & 0               & 0                 \\
         &$yz$& $-\sigma_{zyyy}$& $\sigma_{yyyz}$& 0              & $\sigma_{yyzz}$ & 0               & 0                 \\
         &$zx$& 0               & 0              & 0              & 0               & $\sigma_{zyzy}$ & $-\sigma_{zyyy}$  \\
         &$zy$& $-\sigma_{zyyy}$& $\sigma_{zyyy}$& 0              & $\sigma_{zyzy}$ & 0               & 0                 \\
         &$zz$& $\sigma_{zyyz}$ & $\sigma_{zyyz}$& $\boldsymbol{\sigma_{zzzz}}$& 0               & 0               & 0                 \\
  \hline
\end{tabular}}
\tabl{Te-sigma-emcha-sym}
\end{table}

In $p$-Te the Fermi energy cuts the top of the upper valence
band shown in \fref{Te}(a), forming a small hole-like Fermi pocket
near the H point.  We find that when the Fermi level is placed near
the top of the upper valence band, the largest component of the
  eMChA tensor is
%
\beq
\gamma'_{zzzz}=-\rho^2_{zz}\sigma_{zzzz}
\eql{largest-gamma}
\eeq
%
and the second largest components are
%
\beq
\gamma'_{xzzx}=\gamma'_{yzzy}=-\rho_{yy}\rho_{zz}\sigma_{yzzy}
\eeq
%
(here $\rho_{xx}=\rho_{yy}$),
while other symmetry-allowed components are orders of
magnitude smaller (details in Supplementary Material \cite{SM}).  The
predominance of components driven by magnetic field 
%
along the trigonal $z$
axis suggests that the strong coupling to
$m_z$ is
crucial to the effect.
%
\begin{figure}[t]
\centering
\includegraphics[width=1\columnwidth]{fig/emcha_H.pdf}
\caption{(a) Upper valence band % top
    of trigonal Te in a small segment of the K$^\prime$--H--K path
    around the H point. 
        (b) $zzzz$ component of the eMChA
    conductivity tensor as a function of chemical potential.
The total conductivity (black line) is broken down into
     contributions from the Berry curvature
  ($\Omega$), from the orbital Zeeman coupling
  (Z,orb), and from the spin Zeeman coupling
  (Z,spin). 
    %\xlm{In the code, I forget there is a '-' sign in the prefactor, so (b) is conductivity for left handed, a easy way to change is change 'right-handed' to 'left handed' in the above text. (I will calculate a left handed again). What I want to show in (c) and (d) are \eqs{z-z}{z-mc'}(before the integral with 'delta function'). Unit added!}  
    (c)-(d) $k_z$
    dependent quantities in the conductivities $\sigma^{\rm Z}_{zzzz}$
  and $\sigma^\Omega_{zzzz}$ around H point alone principle axis
    $k_z$.  $\tau = 10^{-15}$s is used in (b),(c) and (d).
    }\figl{Te}
\end{figure}
%
As discussed earlier, the intrinsic eMChA conductivity has
three types of contributions:
%
orbital Zeeman, spin Zeeman, and Berry curvature.  For $p$-Te,
  the $zzzz$ component is strongly dominated by orbital Zeeman
contributions, as shown in \fref{Te}(b).  To understand this result,
consider the $zzzz$ components of
\eqs{sigma-Z}{sigma-mc'},
%
\begin{align}
\sigma^{\rm Z}_{zzzz}=&
    \tau^2 \int \dk \, (-v_{zzz}m_z + v_z \partial_{zz} m_z) (- f_0')\,,
\eql{z-z}\\
\sigma^{\rm \Omega}_{zzzz}=&  \tau^2\int \dk \, ( v^2_z\partial_z\Omega_z -
v_z v_{zz}\Omega_z  \nonumber\\
    &- 3 v_{zx}\Omega_x v_z - 3 v_{zy}\Omega_y v_z ) (-f_0') \eql{z-mc'}\,.
\end{align}
%
%\xlm{Moving the "-" sign together with $f_0'$}
Due to the $-f'_0$ factors, only states located at the Fermi surface
contribute to the response at zero temperature.  Panels (c) and
  (d) of \fref{Te} show the evolution with $k_z$, near the H point,
of each term in the integrand of \eqs{z-z}{z-mc'}
(since $\Omega_x$ and $\Omega_y$ are negligible near  H,
we drop the terms in the second line of \eq{z-mc'}).

In \fref{Te}(c), the
contribution from each term in \eq{z-z} is further separated
  into orbital and spin parts; the orbital contributions are dominant,
  because $|m^{\rm orb}_z| \gg |m^{\rm spin}_z|$ in the upper valence
  band of Te~\cite{stepan18}. In each channel (spin or orbital), the
  first term in \eq{z-z} dominates over the second; this is due to the
  weak ``camel-back'' shape of the upper valence band near H, which
  suppresses the band gradient $v_z$ but not the higher derivative
  $v_{zzz}$. That also explains why the Berry-curvature contributions
  in \fref{Te}(d) are much smaller than the orbital-Zeeman ones in
  \fref{Te}(c), as both terms in the first line of \eq{z-mc'} contain
  $v_z$ factors.


\begin{figure}[t]
\centering
\includegraphics[width=1.0\columnwidth]{fig/T-dependent.pdf}
\caption{Calculated temperature dependence, for different
    acceptor concentrations $N_{\rm a}$, of 
    (c) Ohmic conductivity $\sigma_{zz}$,
    (b) eMChA conductivity $\sigma_{zzzz}$, and
    (c) eMChA parameter $\gamma'_{zzzz}$. 
    $\tau = 10^{-15}$s is used in (a) and (b).
    The circles in panel~(c) denote the 
    values measured in six different samples in Ref.~\cite{calavalle2022gate};
    the red circle denotes the sample whose size was determined most
    precisely~\cite{calavalle2022gate}. (d) $\sigma_{zzzz}$ and
    $\sigma_{zz}$ as functions of chemical
    potential at 10K. The gray dashed lines indicate the chemical
  potentials corresponding to four different acceptor concentrations.
    }
\figl{T-dependent}
\end{figure}
%

The temperature-dependent $\sigma_{zz}$, 
$\sigma_{zzzz}$ and  $\gamma'_{zzzz}$ are shown in \fref{T-dependent}(a,b,c) respectively,
%
which are calculated using the temperature-dependent Fermi-Dirac
distribution function $f_0(T,\mu,\epsilon)$ at chemical potential $\mu$.
\fref{T-dependent}(d) displays $\sigma_{zzzz}$ and $\sigma_{zz}$ at four 
acceptor concentrations at 0K.
As the hole acceptor concentration increases in
$p$-Te, the Fermi level shifts downward, resulting in changes in
$\sigma_{zzzz}$ and $\sigma_{zz}$.  The specific Fermi level for
various acceptor concentrations is shown in \fref{T-dependent}(d).

When the Fermi level is close to the top of the valence band,
the acceptor
concentration is low, and the sensitivity of $\gamma'_{zzzz}$ to
temperature becomes significant compared to higher acceptor
concentrations.  In $\sigma_{zzzz}$, there is a high peak next to the
energy gap, leading to extreme changes due to temperature because of
the smearing of the peak.  The fact that $\sigma_{zzzz}$ and
$1/\sigma_{zz}$ have slopes in a similar trend that exacerbates 
the sensitivity of $\gamma'_{zzzz}$ to temperature when $T<50$ K.  When
$N_{\rm a} = 7.4\cdot10^{18}$cm$^{-3}$, 
the corresponding
 curve in \fref{T-dependent}(b) has an opposite slope
because the chemical potential
is on the other side of the peak of $\sigma_{zzzz}$ (see
\fref{T-dependent}(d)).
In the temperature dependent $\gamma'$ measurement in Ref.\cite{Hirobe22},
$\gamma' \approx 10^{-12}$m$^2$/T/A for gate voltage $V_g = 80$V at 20K.
From 20K to 160K, $\gamma'$ drops by one order of magnitude, see Fig.~4 and Fig.~2(a) therein. 
It matches our simulation when acceptor concentration is
between $7.4 \cdot 10^{16}$cm$^{-3}$ to $7.4 \cdot 10^{17}$cm$^{-3}$.

In the experimental measurement of $p$-Te nanowires in
Ref.\cite{calavalle2022gate}, eMChA was scanned over all possible
directions of the magnetic field, and it was confirmed that the
response is  largest when $\B$ is parallel to the applied current,
in agreement with our simulation.
%
The eMChA resistivity was
measured for six samples. However, to extract $\gamma'_{zzzz}$ the
exact dimensions
of the nanowires are needed, which were
determined only for one of them. Assuming that the size is the same
(at least in order of magnitude) for all nanowires, the
% measured
value of $\gamma'_{zzzz}$ for six nanowires at 10K may be estimated
within the range of
$[10^{-11},10^{-9}]~\text{m}^2/\text{T}/\text{A}$. The acceptor
concentration was estimated as $7.4\cdot10^{17} \text{cm}^{-3}$ using
Hall resistance $R_{\rm H}$,
\beq
N_{\rm a}  = \frac{B_c}{ewR_{\rm H}}= \frac{\sigma_{aa}\sigma_{bb}}{e\sigma^{H}_{abc}}  ~,
\eql{Na-Hall}
\eeq
%
where $w$ is the thickness
of a nanowire. It also can be evaluated from first 
principle by evaluating 
Ohmic and Hall conductivity $\sigma_{aa/bb}$ and $\sigma^{H}_{abc}$ 
within the Boltzman transport theory~\cite{hurd-book72}.
Strictly speaking, the first equality in \eq{Na-Hall} is only valid  for a parabolic band
dispersion. Even though the valence band of Te near the H point
is strongly non-parabolic, by means of accurate calculations we confirm that 
\eq{Na-Hall} is still valid close
to the top of the
valence band,
and the error only  increases slightly as the 
chemical potential moves away from the top of the valence band, 
see \fref{Na-gamma}(b). 
%
\begin{figure}[t]
\centering
\includegraphics[width=1.0\columnwidth]{fig/Na.pdf}
    \caption{(a) Calculated $\gamma'_{zzzz}$ as a function of acceptor consentration. 
    (b) Comparison between acceptor concentration calculated using 
    precise cumulative density of states (CDOS) 
    and Hall resistance measurement.
    }
    \figl{Na-gamma}
\end{figure}
%%
As shown in \fref{T-dependent}(a), our estimated value of
$\gamma'_{zzzz}$ is significantly lower for this acceptor
concentration. However, $\gamma'_{zzzz}$ is extremely
  sensitive to acceptor concentration.  A tiny change of $N_{\rm a}$
  changes orders of magnitude of $\gamma'_{zzzz}$, see
  \fref{Na-gamma}(a). Therefore, an excellent agreement with the experimental
value  is difficult to achieve.

In earlier measurements at room temperature \cite{Rikken19}, the
largest component of the $\gamma$ tensor was found to be in the range
of $[10^{-9},10^{-8}]~\text{m}^2/\text{T}/\text{A}$ with an acceptor
concentration of around $10^{16} \text{cm}^{-3}$. However, it was
observed that the effect is maximized when the magnetic field is
perpendicular to the current.  This contradicts the symmetry analysis
in Table \ref{tab:Te-sigma-emcha-sym}, and therefore cannot be
described as a bulk property of trigonal tellurium, regardless of the
microscopic mechanism implied.

  \textit{The sign of the effect.} Within constant relaxation time approximation we find that for the right-handed crystal, 
  the sign of the eMChA resistivity (the $\gamma'_{zzzz}$ parameter) is negative, or equivalently, the sign of the eMChA resistivity is positive.
  The opposite signs are expected for the left-handed crystal. This is consistent with the experimental results in Ref.\cite{calavalle2022gate}.
  However, after publication of the preprint of the present manuscript, another article appeared \cite{okumura2024chiralorbitaltexturenonlinear}, 
  that claims the opposite sign of the eMChA conductivity. The lack of information on the geometrical size of the samples in Ref.\cite{okumura2024chiralorbitaltexturenonlinear}
  does not allow us to make a direct comparison of the magnitude of the effect. It should be noted that the chirality of the crystal structure in the two experiments 
  was determined by different methods. In Ref.\cite{calavalle2022gate}, the chirality was determined by modern scanning transmission electron microscopy (STEM) technique\cite{dong2020atomic,ben2021chain}, 
  while in Ref.\cite{okumura2024chiralorbitaltexturenonlinear} it was determined by the more traditional method of the etch pits\cite{Koma1970etchpits}. 
  It is known that determination of chirality of trigonal tellurium is a non-trivial task, and the results of different methods are not always consistent (see appendix B of \cite{Furakawa-chirality} for a systematic analysis).
  On the theoretical part,  another article appeared recently~\cite{Golub2020semiclassical} showing that with a different choice of the collision integral
  (beyond the constant relaxation time approximation),
  the sign of the eMChA conductivity can be changed.

\textit{Summary.}
%
In this letter, we derived equations for quantifying eMChA
in real materials using the Boltzmann-transport formalism within the
constant relaxation time approximation.  We use $p$-Te as a research
platform and employ DFT and Wannier interpolation to make
theoretical predictions for eMChA conductivity $\sigma_{abcd}$ and
measurable tensor $\gamma'_{abcd}$.  We find that with low acceptor
concentrations, the orbital Zeeman coupling is the primary origin of
the eMChA response compared with spin Zeeman coupling and Berry
curvature contributions. 
We cannot boast of an excellent agreement with experimental data, 
but neither do we find consistency between different experiments.
Therefore, we hope that our results will be a good reference and motivation for further experimental and theoretical investigations.
The developed methodology is made
available via an open-source code in order to motivate further
first-principles
investigations of other materials.

%%% Local Variables:
%%% mode: latex
%%% TeX-master: "pap"
%%% End:


%\tents{Note added: During the preparation of the manuscript, another preprint appeared
%cite{okumura2024chiralorbitaltexturenonlinear},
%which also studies the eMChA in trigonal tellurium, experimentally and based on $\mathbf{k}\cdot\mathbf{p}$ model. 
%Their results are consistent with ours, and they also find that the orbital Zeeman coupling dominates the eMChA 
%response.
%}\xlm{Where should the "note" be located???? }

\chapter*{Acknowledgement}
\addcontentsline{toc}{chapter}{Acknowledgement}
The authors thank Andrzej Kupsc, Sergey Barsuk, Olivier Callot and Wolfgang K{\"u}hn for their contribution on the CDR draft.
%The authors thank the international review committee XXX for their great effort in reading the CDR draft and providing valuable suggestions. 
The STCF working group thanks all 
the colleagues in the world-wide community for many profitable discussions
and expresses gratitude to the Hefei Comprehensive National Science Center for their strong support.  This work is supported by: international 
partnership program of the Chinese Academy of Sciences Grant No. 211134KYSB20200057.
%\bibliographystyle{apsrev4-1-title}
\bibliography{bib}

%\newpage
%\appendix
%% CVPR 2023 Paper Template
% based on the CVPR template provided by Ming-Ming Cheng (https://github.com/MCG-NKU/CVPR_Template)
% modified and extended by Stefan Roth (stefan.roth@NOSPAMtu-darmstadt.de)

\documentclass[10pt,twocolumn,letterpaper]{article}

%%%%%%%%% PAPER TYPE  - PLEASE UPDATE FOR FINAL VERSION
\usepackage{cvpr}      % To produce the REVIEW version
%\usepackage{cvpr}              % To produce the CAMERA-READY version
%\usepackage[pagenumbers]{cvpr} % To force page numbers, e.g. for an arXiv version

% Include other packages here, before hyperref.
\usepackage{graphicx}
\usepackage{amsmath}
\usepackage{amssymb}
\usepackage{booktabs}
\usepackage[ruled,vlined]{algorithm2e}
\usepackage{color}
\usepackage{amsthm}
\newtheorem{theorem}{Theorem}
\newtheorem{proposition}[theorem]{Proposition}
\newtheorem{lmm}{Theorem}
\newtheorem{rmk}{Theorem}
\newtheorem{lemma}[lmm]{Lemma}
\newtheorem{remark}[rmk]{Remark}
\setlength\parskip{0pt}
\newcommand{\PyComment}[1]{\ttfamily\textcolor{commentcolor}{\# #1}}  % add a "#" before the input text "#1"
\newcommand{\PyCode}[1]{\ttfamily\textcolor{black}{#1}} % \ttfamily is the code font
\definecolor{commentcolor}{RGB}{110,154,155}
% It is strongly recommended to use hyperref, especially for the review version.
% hyperref with option pagebackref eases the reviewers' job.
% Please disable hyperref *only* if you encounter grave issues, e.g. with the
% file validation for the camera-ready version.
%
% If you comment hyperref and then uncomment it, you should delete
% ReviewTempalte.aux before re-running LaTeX.
% (Or just hit 'q' on the first LaTeX run, let it finish, and you
%  should be clear).
\usepackage[pagebackref,breaklinks,colorlinks]{hyperref}


% Support for easy cross-referencing
\usepackage[capitalize]{cleveref}
\crefname{section}{Sec.}{Secs.}
\Crefname{section}{Section}{Sections}
\Crefname{table}{Table}{Tables}
\crefname{table}{Tab.}{Tabs.}


%%%%%%%%% PAPER ID  - PLEASE UPDATE
\def\cvprPaperID{9799} % *** Enter the CVPR Paper ID here
\def\confName{CVPR}
\def\confYear{2023}


\begin{document}
	
	%%%%%%%%% TITLE - PLEASE UPDATE
	\title{Transforming  Radiance Field with Lipschitz Network for 
		
		Photorealistic 3D Scene Stylization
  
  \textemdash\textemdash CVPR 2023 Supplementary Material}
	
	 \author{%
 	Zicheng Zhang\textsuperscript{1}%\thanks{Work done during an internship at JD AI Research.}
	\quad
	Yinglu Liu\textsuperscript{2}
	\quad
	Congying Han\textsuperscript{1}
	\quad
        Yingwei Pan\textsuperscript{2}
        \quad
	Tiande Guo\textsuperscript{1}
	\quad
	Ting Yao\textsuperscript{2}
	\quad
	\\ 
	\textsuperscript{1}University of Chinese Academy of Sciences
	\quad
	\textsuperscript{2}JD AI Research
	\quad
	\vspace{.5em} 
	\\
	\tt\small zhangzicheng19@mails.ucas.ac.cn
	\quad
	liuyinglu1@jd.com
	\quad
	hancy@ucas.ac.cn
	\\
	\tt\small
     panyw.ustc@gmail.com
	\quad
	tdguo@ucas.ac.cn
	\quad
	tingyao.ustc@gmail.com
}
	\maketitle
	\appendix

	\section{Proofs}
    \begin{proposition}\label{proposition 1}
		Considering  $f(\boldsymbol{c}) = \boldsymbol{A}\boldsymbol{c}+\boldsymbol{b}$, $\boldsymbol{A}\in \mathbb{R}^{3\times3}$, $\boldsymbol{b}\in \mathbb{R}^{3\times1}$, if $\mathbf{F}_{app}' = f\circ\mathbf{F}_{app}$, $\sum_{i=1}^{T}w_{i} = 1$ and  $vrr(\mathbf{r}_1,\mathbf{r}_2; \mathbf{F})<\epsilon$, we have  $vrr(\mathbf{r}_1,\mathbf{r}_2; \mathbf{F}')<K\epsilon$, where  $K =\left\| \boldsymbol{A} \right\|_2$ is the Lipschitz constant of $f$. 
	\end{proposition}
	
	\begin{proof} 
	\begin{small}
	\begin{equation}\label{eq: vrr}
		\begin{aligned}
			vrr(\mathbf{r}_1,\mathbf{r}_2; \mathbf{F}') &= \left \| {C}(\mathbf{r}_1;\mathbf{F}') -{C}(\mathbf{r}_2;\mathbf{F}')  \right \| \\ 
			&=\left \| \sum\limits_{i=1}^{T} w^{\mathbf{r}_1}_{i}f(\boldsymbol{c}^{\mathbf{r}_1}_i) -\sum\limits_{i=1}^{T} w^{\mathbf{r}_2}_{i}f(\boldsymbol{c}^{\mathbf{r}_2}_i)  \right \|\\
			&= \left \| \sum\limits_{i=1}^{T} w^{\mathbf{r}_1}_{i}(\boldsymbol{A}\boldsymbol{c}^{\mathbf{r}_1}_i+\boldsymbol{b}) -\sum\limits_{i=1}^{T} w^{\mathbf{r}_2}_{i}(\boldsymbol{A}\boldsymbol{c}^{\mathbf{r}_2}_i+\boldsymbol{b})  \right \| \\
			&= \left \| \sum\limits_{i=1}^{T} w^{\mathbf{r}_1}_{i}\boldsymbol{A}\boldsymbol{c}^{\mathbf{r}_1}_i -\sum\limits_{i=1}^{T} w^{\mathbf{r}_2}_{i}\boldsymbol{A}\boldsymbol{c}^{\mathbf{r}_2}_i  \right \| \\
			&= \left \| \boldsymbol{A} \left( \sum\limits_{i=1}^{T} w^{\mathbf{r}_1}_{i}\boldsymbol{c}^{\mathbf{r}_1}_i -\sum\limits_{i=1}^{T} w^{\mathbf{r}_2}_{i}\boldsymbol{c}^{\mathbf{r}_2}_i \right) \right \| \\
			&\leq \left\| \boldsymbol{A} \right\| \left \|  \sum\limits_{i=1}^{T} w^{\mathbf{r}_1}_{i}\boldsymbol{c}^{\mathbf{r}_1}_i -\sum\limits_{i=1}^{T} w^{\mathbf{r}_2}_{i}\boldsymbol{c}^{\mathbf{r}_2}_i \right \|
			\\
			&=\left\| \boldsymbol{A} \right\| vrr(\mathbf{r}_1,\mathbf{r}_2; \mathbf{F}) \\
			&< K \epsilon \notag
		\end{aligned}
	\end{equation}	
	\end{small}
	\end{proof}

	
	
	\begin{lemma}\label{lemma 1}
	Given $f=f_l \circ\cdots \circ f_1$,  $f_j(x) = \boldsymbol{A}_jx+\boldsymbol{b}$ if $j = l$ and  $\sigma(\boldsymbol{A}_{j}x)$ otherwise, where $\sigma$ is a $1$-Lipschitz function. Then $K = \Pi^l_{j=1}\left\| \boldsymbol{A}_{j} \right\|_{2} $ is the Lipschitz constant of $f$.
	\end{lemma}
	\begin{proof}
		Suppose that inputs $x$, $y$ belong to the domain of $f_{j}$, 
		\begin{equation}
			\begin{aligned}
			\left\| f_j(x) - f_{j}(y) \right\| &\leq  \left\| \sigma(\boldsymbol{A}_{j}x) - \sigma(\boldsymbol{A}_{j}y) \right\|  \\	
			 &\leq \left\|  \boldsymbol{A}_jx  - \boldsymbol{A}_jy \right\| \\
				&\leq \left\| \boldsymbol{A}_j \right\| \left\|  x  -  y \right\|.
			\end{aligned}
		\end{equation}
		When $l = 2$, the claim is clearly valid. The remaining cases can be easily proved by induction.
%		\begin{equation}
%			\begin{aligned}
%			\left\| f(x) - f(y) \right\| &\leq \left\| \boldsymbol{A}_l \right\| \left\| f^{l-1}(x) - f^{l-1}(y) \right\| \\
%			 &\leq 	\left\| \boldsymbol{A}_l \right\| \left\| f_{i}(x) - f_{i}(y) \right\| \\ 
%			 &\leq 	\left\| \boldsymbol{A}_l \right\| \left\| f_{i}(x) - f_{i}(y) \right\| \\
%%				&\leq \left\| \boldsymbol{A}_2\sigma(\boldsymbol{A}_1\boldsymbol{x}) - 		\boldsymbol{A}_2\sigma(\boldsymbol{A}_1\boldsymbol{y}) \right\| \\
%%				&\leq \left\| \boldsymbol{A}_2 \right\| \left\| \sigma(\boldsymbol{A}_1\boldsymbol{x})  - 		\sigma(\boldsymbol{A}_1\boldsymbol{y}) \right\| \\
%%				&\leq \left\| \boldsymbol{A}_2 \right\| \left\|  \boldsymbol{A}_1\boldsymbol{x}  - 	\boldsymbol{A}_1\boldsymbol{y} \right\|
%			\end{aligned}
%		\end{equation}
	\end{proof}

	\begin{proposition}\label{proposition 2}
		Considering $f=f_l \circ\cdots \circ f_1$,  $f_j(x) = \boldsymbol{A}_jx+\boldsymbol{b}$ if $j = l$ and  $\sigma(\boldsymbol{A}_{j}x)$ otherwise, where $\sigma=\max(0,x)$. If $\mathbf{F}_{app}' = f\circ\mathbf{F}_{app}$, $\sum_{i=1}^{T}w_{i} = 1$ and $\max_{i=1,\dots,T}\left\|  w^{\mathbf{r}_1}_{i}\boldsymbol{c}^{\mathbf{r}_1}_i - w^{\mathbf{r}_2}_{i}\boldsymbol{c}^{\mathbf{r}_2}_i\right\| < \epsilon/T$,  we have $vrr(\mathbf{r}_1,\mathbf{r}_2; \mathbf{F}')<K\epsilon$, where $K = \Pi^l_{j=1}\left\| \boldsymbol{A}_{j} \right\|_{2} $ is the Lipschitz constant of $f$. 
	\end{proposition}
	\begin{proof}
    \begin{small}
    Note that $\forall a\in \mathbb{R}^{+}$ and $1\leq j < l$, $af_j(x)=a\sigma(\boldsymbol{A}_{j}x)=\sigma(a\boldsymbol{A}_{j}x)=f_j(ax)$. Denoting $f^{j}=f_j \circ\cdots \circ f_1$, we can get the following derivation: 
    \begin{equation}
        \begin{aligned}
            af^{j}(x)&=a\sigma(\boldsymbol{A}_{j}f^{j-1}(x)) = \sigma(a\boldsymbol{A}_{j}f^{j-1}(x)) \\
            &=  \sigma(a\boldsymbol{A}_{j}\sigma(\boldsymbol{A}_{j-1}f^{j-2}(x))) \\
            &=  \sigma(\boldsymbol{A}_{j}\sigma(a\boldsymbol{A}_{j-1}f^{j-2}(x))) \\
            &\cdots \\
            & = f^{j}(ax).
        \end{aligned}
    \end{equation}
    Because the weights are always non-negative in the volume rendering integral,
    we further have
		\begin{equation}\label{eq: vrr}
		\begin{aligned}
			vrr(\mathbf{r}_1,\mathbf{r}_2; \mathbf{F}') &= \left \| {C}(\mathbf{r}_1;\mathbf{F}') -{C}(\mathbf{r}_2;\mathbf{F}')  \right \| \\ 
			&=\left \| \sum\limits_{i=1}^{T} w^{\mathbf{r}_1}_{i}f(\boldsymbol{c}^{\mathbf{r}_1}_i) -\sum\limits_{i=1}^{T} w^{\mathbf{r}_2}_{i}f(\boldsymbol{c}^{\mathbf{r}_2}_i)  \right \|\\
			&=\left \| \sum\limits_{i=1}^{T} w^{\mathbf{r}_1}_{i}\boldsymbol{A}_{l}f^{l-1}(\boldsymbol{c}^{\mathbf{r}_1}_i) -\sum\limits_{i=1}^{T} w^{\mathbf{r}_2}_{i}\boldsymbol{A}_{l}f^{l-1}(\boldsymbol{c}^{\mathbf{r}_2}_i)  \right \|\\
    		&=\left \| \sum\limits_{i=1}^{T} \boldsymbol{A}_{l}f^{l-1}(w^{\mathbf{r}_1}_{i}\boldsymbol{c}^{\mathbf{r}_1}_i) -\sum\limits_{i=1}^{T} \boldsymbol{A}_{l}f^{l-1}(w^{\mathbf{r}_2}_{i}\boldsymbol{c}^{\mathbf{r}_2}_i)  \right \|\\
			&\leq \sum\limits_{i=1}^{T}\left \|  \boldsymbol{A}_{l}f^{l-1}(w^{\mathbf{r}_1}_{i}\boldsymbol{c}^{\mathbf{r}_1}_i) - \boldsymbol{A}_{l}f^{l-1}(w^{\mathbf{r}_2}_{i}\boldsymbol{c}^{\mathbf{r}_2}_i)  \right \|.
			\\
		\end{aligned}
	\end{equation}	
    Based on above inequality and Lemma \ref{lemma 1}, we have
\begin{equation}\label{eq: vrr}
		\begin{aligned}
			&\ \ \ \ \left\|  \boldsymbol{A}_{l}f^{l-1}(w^{\mathbf{r}_1}_{i}\boldsymbol{c}^{\mathbf{r}_1}_i) - \boldsymbol{A}_{l}f^{l-1}(w^{\mathbf{r}_2}_{i}\boldsymbol{c}^{\mathbf{r}_2}_i) \right\|  \\
		&\leq \left\| \boldsymbol{A}_{l} \right\| \left\|  f^{l-1}(w^{\mathbf{r}_1}_{i}\boldsymbol{c}^{\mathbf{r}_1}_i) - f^{l-1}(w^{\mathbf{r}_2}_{i}\boldsymbol{c}^{\mathbf{r}_2}_i) \right\| \\
			&\leq \prod^{l}_{j=1} \left\| \boldsymbol{A}_{i} \right\| \ \left\|  w^{\mathbf{r}_1}_{i}\boldsymbol{c}^{\mathbf{r}_1}_i - w^{\mathbf{r}_2}_{i}\boldsymbol{c}^{\mathbf{r}_2}_i \right\|\\
			&= K \ \left\|  w^{\mathbf{r}_1}_{i}\boldsymbol{c}^{\mathbf{r}_1}_i - w^{\mathbf{r}_2}_{i}\boldsymbol{c}^{\mathbf{r}_2}_i \right\|.
		\end{aligned} 
	\end{equation}
	Therefore, 
	\begin{equation}\label{eq: vrr}
		\begin{aligned}
			vrr(\mathbf{r}_1,\mathbf{r}_2; \mathbf{F}') &\leq \sum\limits_{i=1}^{T} K \left\|  w^{\mathbf{r}_1}_{i}\boldsymbol{c}^{\mathbf{r}_1}_i - w^{\mathbf{r}_2}_{i}\boldsymbol{c}^{\mathbf{r}_2}_i \right\| \\
			&< K \sum\limits_{i=1}^{T} \epsilon/T = K\epsilon \\
		\end{aligned} 
	\end{equation}	
    \end{small}
	\end{proof}

		\begin{lemma} \label{lemma 2}
		$\mathbf{F}_{app}({\boldsymbol{x}, \boldsymbol{d}})$ = $\mathbf{F}_{sh}(\boldsymbol{x})\Gamma(\boldsymbol{d})+\boldsymbol{v}$, where $\Gamma(\boldsymbol{d}):\mathbb{R}^{2\times1} \rightarrow \mathbb{R}^{\ell\times1}$ is the spherical harmonic basis function, $\mathbf{F}_{sh}(\boldsymbol{x}) :\mathbb{R}^{3\times1} \rightarrow \mathbb{R}^{3\times \ell}$ is the coefficient function, and $\boldsymbol{v} \in \mathbb{R}^{3\times1}$. Given $\boldsymbol{A} \in \mathbb{R}^{3\times3}$, $\boldsymbol{b} \in \mathbb{R}^{3\times1}$, then
		$\boldsymbol{A}\mathbf{F}_{app}({\boldsymbol{x}, \boldsymbol{d}})+\boldsymbol{b} \Leftrightarrow 
		\boldsymbol{A}\mathbf{F}_{sh}({\boldsymbol{x}})+2\sqrt{\pi}[\boldsymbol{A}\boldsymbol{v}+\boldsymbol{b}-\boldsymbol{v},\boldsymbol{0}]$.
	\end{lemma}
	\begin{proof}
		\begin{equation}\label{eq: sh}
			\begin{aligned}
				\boldsymbol{A}\mathbf{F}_{app}({\boldsymbol{x}, \boldsymbol{d}})+\boldsymbol{b} &= 
				\boldsymbol{A}(\mathbf{F}_{sh}(\boldsymbol{x})\Gamma(\boldsymbol{d})+\boldsymbol{v}) +\boldsymbol{b} \\
				&= \boldsymbol{A}\mathbf{F}_{sh}(\boldsymbol{x})\Gamma(\boldsymbol{d})+\boldsymbol{A}\boldsymbol{v} +\boldsymbol{b} \\
				&= \boldsymbol{A}\mathbf{F}_{sh}(\boldsymbol{x})\Gamma(\boldsymbol{d})+\boldsymbol{A}\boldsymbol{v} +\boldsymbol{b}. \\
			\end{aligned}
		\end{equation}
		Because the first component of the spherical harmonic basis function outputs a constant value $\frac{1}{2\sqrt{\pi}}$,  we have
		\begin{equation}
			\begin{aligned}
				&(\boldsymbol{A}\mathbf{F}_{sh}({\boldsymbol{x}})+2\sqrt{\pi}[\boldsymbol{A}\boldsymbol{v}+\boldsymbol{b}-\boldsymbol{v},\boldsymbol{0}])\Gamma(\boldsymbol{d})+\boldsymbol{v}\\
				& = \boldsymbol{A}\mathbf{F}_{sh}\Gamma(\boldsymbol{d}) + \boldsymbol{A}\boldsymbol{v}+\boldsymbol{b}-\boldsymbol{v} + \boldsymbol{v} \\
				& =  \boldsymbol{A}\mathbf{F}_{sh}\Gamma(\boldsymbol{d}) + \boldsymbol{A}\boldsymbol{v}+\boldsymbol{b}
			\end{aligned}	
		\end{equation}
	\end{proof}
	\noindent\textbf{Remark of Lemma \ref{lemma 2}}. Similarly, it  can prove $\boldsymbol{A}\mathbf{F}_{sh}({\boldsymbol{x}})+[\boldsymbol{b},\boldsymbol{0}] \Leftrightarrow 
	 \boldsymbol{A}\mathbf{F}_{app}({\boldsymbol{x}, \boldsymbol{d}})+\frac{\boldsymbol{b}}{2\sqrt{\pi}} + \boldsymbol{v}-\boldsymbol{A}\boldsymbol{v}$.
	 
	 \begin{proposition}\label{proposition 3}
	 	Considering  $f(x) = \boldsymbol{A}x+\boldsymbol{b}$, $\boldsymbol{A}\in \mathbb{R}^{3\times \ell}$, $\boldsymbol{b}\in \mathbb{R}^{3\times \ell}$, if $\mathbf{F}_{sh}' = f\circ\mathbf{F}_{sh}$, $\sum_{i=1}^{T}w_{i} = 1$, $vrr(\mathbf{r}_1,\mathbf{r}_2; \mathbf{F})<\epsilon_1$ and $\left\| \Gamma(\boldsymbol{d^{\textbf{r}_{1}}}) -\Gamma(\boldsymbol{d^{\textbf{r}_{2}}}) \right\|<\epsilon_2$, we have  $vrr(\mathbf{r}_1,\mathbf{r}_2; \mathbf{F}')<K_1\epsilon_1$ + $K_2\epsilon_2$, where  $K_1 =\left\| \boldsymbol{A} \right\|_2$ and $K_2 =\left\| \boldsymbol{b} \right\|_2$. Moreover, if $\boldsymbol{b}$ vanishes except for the first column (\textit{i.e.}, the form in above remark), $vrr(\mathbf{r}_1,\mathbf{r}_2; \mathbf{F}')<K_1\epsilon_1$.
	 \end{proposition}
	
		\begin{proof} 
		\begin{small}
			\begin{equation}\label{eq: vrr}
				\begin{aligned}
					&vrr(\mathbf{r}_1,\mathbf{r}_2; \mathbf{F}')\\
					 &= \left \| {C}(\mathbf{r}_1;\mathbf{F}') -{C}(\mathbf{r}_2;\mathbf{F}')  \right \| \\ 
					&= \left \| \sum\limits_{i=1}^{T} w^{\mathbf{r}_1}_{i}\textbf{F}'(\boldsymbol{x}^{\textbf{r}_{1}}_{i})\Gamma(\boldsymbol{d^{\textbf{r}_{1}}}) -\sum\limits_{i=1}^{T} w^{\mathbf{r}_2}_{i}\textbf{F}'(\boldsymbol{x}^{\textbf{r}_{2}}_{i})\Gamma(\boldsymbol{d^{\textbf{r}_{2}}})  \right \| \\
%					&\ \ \ \ + \left \| \sum\limits_{i=1}^{T} w^{\mathbf{r}_2}_{i}\textbf{F}'(\boldsymbol{x}^{\textbf{r}_{2}}_{i})(\Gamma(\boldsymbol{d^{\textbf{r}_{2}}})- \Gamma(\boldsymbol{d^{\textbf{r}_{1}}}))  \right \| \\
					&\leq \left \| \sum\limits_{i=1}^{T} w^{\mathbf{r}_1}_{i}\boldsymbol{A}\textbf{F}(\boldsymbol{x}^{\textbf{r}_{1}}_{i})\Gamma(\boldsymbol{d^{\textbf{r}_{1}}}) -\sum\limits_{i=1}^{T} w^{\mathbf{r}_2}_{i}\boldsymbol{A}\textbf{F}(\boldsymbol{x}^{\textbf{r}_{2}}_{i})\Gamma(\boldsymbol{d^{\textbf{r}_{2}}}) \right \|  \\
					&\ \ \ \ +  \left \| \boldsymbol{b}\Gamma(\boldsymbol{d^{\textbf{r}_{1}}}) -\boldsymbol{b}\Gamma(\boldsymbol{d^{\textbf{r}_{2}}}) \right \| \\
%					&\leq  \left\| \boldsymbol{A} \right\| vrr(\mathbf{r}_1,\mathbf{r}_2; \mathbf{F})  +  \left \| \boldsymbol{b}\Gamma(\boldsymbol{d^{\textbf{r}_{1}}}) -\boldsymbol{b}\Gamma(\boldsymbol{d^{\textbf{r}_{2}}}) \right \|\\
					&\leq  \left\| \boldsymbol{A} \right\| vrr(\mathbf{r}_1,\mathbf{r}_2; \mathbf{F})  +  \left \| \boldsymbol{b}\right \| \left\| \Gamma(\boldsymbol{d^{\textbf{r}_{1}}}) -\Gamma(\boldsymbol{d^{\textbf{r}_{2}}})  \right\|\\
					&<  K_1\epsilon_1 + K_2\epsilon_2.\\
				\end{aligned}
			\end{equation}	
		If $\boldsymbol{b}$ vanishes except for the first column, $ \left \| \boldsymbol{b}\Gamma(\boldsymbol{d^{\textbf{r}_{1}}}) -\boldsymbol{b}\Gamma(\boldsymbol{d^{\textbf{r}_{2}}}) \right \| =0$, thus   $vrr(\mathbf{r}_1,\mathbf{r}_2; \mathbf{F}')<K_1\epsilon_1$.
	\end{small}
	\end{proof}

\noindent\textbf{Remarks}. Prop.~\ref{proposition 3} extends Lipschitz-constrained linear mapping in Prop.~\ref{proposition 1} from appearance representation to spherical harmonics. To prove the bound of Lipschitz MLP applied to spherical harmonics, some fussy assumptions are further required, and the proof will be trivial to repeat the above proving processes. We believe the three propositions have exhibited the intuition and importance of Lipschitz transformations for this task.
\section{More results}
For comprehensive analysis and evaluation, we have supplied a video\footnote{\url{https://www.youtube.com/watch?v=1ft8Mev3RmE}} in the supplementary materials,  which contains the continuous novel views of multiple scenes stylized with various references.  It can be observed that, both $\text{WCT}^2$ and CCPL create noises and disharmony to affect the photorealism of video. In specific, $\text{WCT}^2$ is likely to sharpen the edges excessively that produces artificial boundaries around edges (\textit{e.g.}, the trex and room scenes). It also generates noticeable noises in some stylized scenes. The results of CCPL usually have richer colors that enhances the visual effects. However, the variegated colors acceptable in a still image may be harmful to 3D scenes. For example, in the trex and fortress scenes, the interframe variations of colors results in artifacts and unconsistency of videos. In the flower scene, due to the unconsistency, the colorful leaves and flowers seem to be unrealistic and flickering. In contrast, LipRF can alleviate these downsides to generate more consistent and photorealistic stylized novel views while transferring the color style. The videos of LipRF are more like camera shots to meet the requirement of photorealistic 3D scene stylization. 


%	{\small
%		\bibliographystyle{ieee_fullname}
%		\bibliography{citation}
%	}
	

	
\end{document}


\end{document}
