%======================
%\section{eMChA in Tellurum}
\secl{Te}
%======================

\textit{Computational methods.}
The unit cell of trigonal Te contains three atoms disposed along a
spiral chain, with the chains arranged on a 2D hexagonal lattice.
%
The fully-relativistic electronic structure of the pristine (undoped)
%right-handed crystal (space group P3$_1$21) is evaluated via
right-handed crystal (space group P3$_1$21) is evaluated via
density-functional theory (DFT) in the pseudopotential framework, using the
HSE06 hybrid functional~\cite{paier-jcp06} implemented in the VASP
code package
\cite{kresse1999ultrasoft,kresse1996efficiency,PhysRevB.54.11169}.


In order to evaluate the needed $k$-space quantities accurately and
efficiently, we employ the Wannier interpolation
scheme~\cite{wang-prb06}.  The Wannier functions are constructed using
the Wannier90 code package~\cite{wannier90}, starting from
atom-centered $s$ and $p$ trial orbitals.  In order to avoid artifacts
due to possible numerical violations of crystal symmetries by the
Wannier model, the Wannier Hamiltonian and orbital-based matrix
elements are
symmetrized so as to satisfy the point-group symmetries.

We have implemented all terms in \eqs{sigma-Z}{sigma-mc'} in the
WannierBerri code package~\cite{wannierberri}.  Wannier interpolation
allows us to evaluate band properties and their momentum-space
derivatives directly, without finite difference
schemes 
~\cite{gradients}.
We perform the Brillouin-zone integrals on a symmetry-irreducible part
of a regular grid of $300\times300\times300$ points, using 30 adaptive
refinement iterations, and employing the tetrahedron 
method~\cite{tetrahedronmethod,tetrahedronmethod2}
to accurately describe the Fermi occupation factors \tenxl{$-f_0'$}.


\textit{Numerical results.}
The eMChA conductivity $\sigma_{abcd}$ is a nonmagnetic polar tensor
symmetric in $bc$, with the Jahn symbol
eV[V$^2$]V. Table~\ref{tab:Te-sigma-emcha-sym} shows the constraints
imposed on that tensor by the point group symmetry of trigonal
tellurium. 

\begin{table}[t]
\caption{Symmetry constraints on the components of the eMChA
    conductivity tensor $\sigma_{abcd}$ in crystal class
    32. The three largest calculated components in trigonal tellurium
    are highlighted in bold. The table was generated using the
$\rm MTENSOR$  routine~\cite{mTensor} provided by the
Bilbao Crystallographic Server~\cite{mTensor-link}.
}
\centering % centering table
\makebox[\columnwidth]{\begin{tabular}{|c|c|cccccc|}
  \hline
  \multicolumn{2}{|c|}{32}& \multicolumn{6}{c|}{$bc=cb$}\\
  \cline{3-8}
    \multicolumn{2}{|c|}{$\sigma_{abcd}$} &$xx$&$yy$&$zz$&$yz$&$xz$&$xy$\\
  \hline
         &$xx$& $\sigma_{yyyy}$ & $\sigma_{yxxy}$& $\boldsymbol{\sigma_{yzzy}}$& $-\sigma_{yyzy}$& 0               & 0                 \\
         &$xy$& 0               & 0              & 0              & 0               & $-\sigma_{yyzy}$& $(\sigma_{yyyy}-\sigma_{yxxy})/2$ \\
         &$xz$& 0               & 0              & 0              & 0               & $\sigma_{yyzz}$ & $-\sigma_{yyyz}$  \\
         &$yx$& 0               & 0              & 0              & 0               & $-\sigma_{yyzy}$& $(\sigma_{yyyy}-\sigma_{yxxy})/2$ \\
 $ad$&$yy$& $\sigma_{yxxy}$ & $\sigma_{yyyy}$& $\boldsymbol{\sigma_{yzzy}}$& $\sigma_{yyzy}$ & 0               & 0                 \\
         &$yz$& $-\sigma_{zyyy}$& $\sigma_{yyyz}$& 0              & $\sigma_{yyzz}$ & 0               & 0                 \\
         &$zx$& 0               & 0              & 0              & 0               & $\sigma_{zyzy}$ & $-\sigma_{zyyy}$  \\
         &$zy$& $-\sigma_{zyyy}$& $\sigma_{zyyy}$& 0              & $\sigma_{zyzy}$ & 0               & 0                 \\
         &$zz$& $\sigma_{zyyz}$ & $\sigma_{zyyz}$& $\boldsymbol{\sigma_{zzzz}}$& 0               & 0               & 0                 \\
  \hline
\end{tabular}}
\tabl{Te-sigma-emcha-sym}
\end{table}

In $p$-Te the Fermi energy cuts the top of the upper valence
band shown in \fref{Te}(a), forming a small hole-like Fermi pocket
near the H point.  We find that when the Fermi level is placed near
the top of the upper valence band, the largest component of the
  eMChA tensor is
%
\beq
\gamma'_{zzzz}=-\rho^2_{zz}\sigma_{zzzz}
\eql{largest-gamma}
\eeq
%
and the second largest components are
%
\beq
\gamma'_{xzzx}=\gamma'_{yzzy}=-\rho_{yy}\rho_{zz}\sigma_{yzzy}
\eeq
%
(here $\rho_{xx}=\rho_{yy}$),
while other symmetry-allowed components are orders of
magnitude smaller (details in Supplementary Material \cite{SM}).  The
predominance of components driven by magnetic field 
%
along the trigonal $z$
axis suggests that the strong coupling to
$m_z$ is
crucial to the effect.
%
\begin{figure}[t]
\centering
\includegraphics[width=1\columnwidth]{fig/emcha_H.pdf}
\caption{(a) Upper valence band % top
    of trigonal Te in a small segment of the K$^\prime$--H--K path
    around the H point. 
        (b) $zzzz$ component of the eMChA
    conductivity tensor as a function of chemical potential.
The total conductivity (black line) is broken down into
     contributions from the Berry curvature
  ($\Omega$), from the orbital Zeeman coupling
  (Z,orb), and from the spin Zeeman coupling
  (Z,spin). 
    %\xlm{In the code, I forget there is a '-' sign in the prefactor, so (b) is conductivity for left handed, a easy way to change is change 'right-handed' to 'left handed' in the above text. (I will calculate a left handed again). What I want to show in (c) and (d) are \eqs{z-z}{z-mc'}(before the integral with 'delta function'). Unit added!}  
    (c)-(d) $k_z$
    dependent quantities in the conductivities $\sigma^{\rm Z}_{zzzz}$
  and $\sigma^\Omega_{zzzz}$ around H point alone principle axis
    $k_z$.  $\tau = 10^{-15}$s is used in (b),(c) and (d).
    }\figl{Te}
\end{figure}
%
As discussed earlier, the intrinsic eMChA conductivity has
three types of contributions:
%
orbital Zeeman, spin Zeeman, and Berry curvature.  For $p$-Te,
  the $zzzz$ component is strongly dominated by orbital Zeeman
contributions, as shown in \fref{Te}(b).  To understand this result,
consider the $zzzz$ components of
\eqs{sigma-Z}{sigma-mc'},
%
\begin{align}
\sigma^{\rm Z}_{zzzz}=&
    \tau^2 \int \dk \, (-v_{zzz}m_z + v_z \partial_{zz} m_z) (- f_0')\,,
\eql{z-z}\\
\sigma^{\rm \Omega}_{zzzz}=&  \tau^2\int \dk \, ( v^2_z\partial_z\Omega_z -
v_z v_{zz}\Omega_z  \nonumber\\
    &- 3 v_{zx}\Omega_x v_z - 3 v_{zy}\Omega_y v_z ) (-f_0') \eql{z-mc'}\,.
\end{align}
%
%\xlm{Moving the "-" sign together with $f_0'$}
Due to the $-f'_0$ factors, only states located at the Fermi surface
contribute to the response at zero temperature.  Panels (c) and
  (d) of \fref{Te} show the evolution with $k_z$, near the H point,
of each term in the integrand of \eqs{z-z}{z-mc'}
(since $\Omega_x$ and $\Omega_y$ are negligible near  H,
we drop the terms in the second line of \eq{z-mc'}).

In \fref{Te}(c), the
contribution from each term in \eq{z-z} is further separated
  into orbital and spin parts; the orbital contributions are dominant,
  because $|m^{\rm orb}_z| \gg |m^{\rm spin}_z|$ in the upper valence
  band of Te~\cite{stepan18}. In each channel (spin or orbital), the
  first term in \eq{z-z} dominates over the second; this is due to the
  weak ``camel-back'' shape of the upper valence band near H, which
  suppresses the band gradient $v_z$ but not the higher derivative
  $v_{zzz}$. That also explains why the Berry-curvature contributions
  in \fref{Te}(d) are much smaller than the orbital-Zeeman ones in
  \fref{Te}(c), as both terms in the first line of \eq{z-mc'} contain
  $v_z$ factors.


\begin{figure}[t]
\centering
\includegraphics[width=1.0\columnwidth]{fig/T-dependent.pdf}
\caption{Calculated temperature dependence, for different
    acceptor concentrations $N_{\rm a}$, of 
    (c) Ohmic conductivity $\sigma_{zz}$,
    (b) eMChA conductivity $\sigma_{zzzz}$, and
    (c) eMChA parameter $\gamma'_{zzzz}$. 
    $\tau = 10^{-15}$s is used in (a) and (b).
    The circles in panel~(c) denote the 
    values measured in six different samples in Ref.~\cite{calavalle2022gate};
    the red circle denotes the sample whose size was determined most
    precisely~\cite{calavalle2022gate}. (d) $\sigma_{zzzz}$ and
    $\sigma_{zz}$ as functions of chemical
    potential at 10K. The gray dashed lines indicate the chemical
  potentials corresponding to four different acceptor concentrations.
    }
\figl{T-dependent}
\end{figure}
%

The temperature-dependent $\sigma_{zz}$, 
$\sigma_{zzzz}$ and  $\gamma'_{zzzz}$ are shown in \fref{T-dependent}(a,b,c) respectively,
%
which are calculated using the temperature-dependent Fermi-Dirac
distribution function $f_0(T,\mu,\epsilon)$ at chemical potential $\mu$.
\fref{T-dependent}(d) displays $\sigma_{zzzz}$ and $\sigma_{zz}$ at four 
acceptor concentrations at 0K.
As the hole acceptor concentration increases in
$p$-Te, the Fermi level shifts downward, resulting in changes in
$\sigma_{zzzz}$ and $\sigma_{zz}$.  The specific Fermi level for
various acceptor concentrations is shown in \fref{T-dependent}(d).

When the Fermi level is close to the top of the valence band,
the acceptor
concentration is low, and the sensitivity of $\gamma'_{zzzz}$ to
temperature becomes significant compared to higher acceptor
concentrations.  In $\sigma_{zzzz}$, there is a high peak next to the
energy gap, leading to extreme changes due to temperature because of
the smearing of the peak.  The fact that $\sigma_{zzzz}$ and
$1/\sigma_{zz}$ have slopes in a similar trend that exacerbates 
the sensitivity of $\gamma'_{zzzz}$ to temperature when $T<50$ K.  When
$N_{\rm a} = 7.4\cdot10^{18}$cm$^{-3}$, 
the corresponding
 curve in \fref{T-dependent}(b) has an opposite slope
because the chemical potential
is on the other side of the peak of $\sigma_{zzzz}$ (see
\fref{T-dependent}(d)).
In the temperature dependent $\gamma'$ measurement in Ref.\cite{Hirobe22},
$\gamma' \approx 10^{-12}$m$^2$/T/A for gate voltage $V_g = 80$V at 20K.
From 20K to 160K, $\gamma'$ drops by one order of magnitude, see Fig.~4 and Fig.~2(a) therein. 
It matches our simulation when acceptor concentration is
between $7.4 \cdot 10^{16}$cm$^{-3}$ to $7.4 \cdot 10^{17}$cm$^{-3}$.

In the experimental measurement of $p$-Te nanowires in
Ref.\cite{calavalle2022gate}, eMChA was scanned over all possible
directions of the magnetic field, and it was confirmed that the
response is  largest when $\B$ is parallel to the applied current,
in agreement with our simulation.
%
The eMChA resistivity was
measured for six samples. However, to extract $\gamma'_{zzzz}$ the
exact dimensions
of the nanowires are needed, which were
determined only for one of them. Assuming that the size is the same
(at least in order of magnitude) for all nanowires, the
% measured
value of $\gamma'_{zzzz}$ for six nanowires at 10K may be estimated
within the range of
$[10^{-11},10^{-9}]~\text{m}^2/\text{T}/\text{A}$. The acceptor
concentration was estimated as $7.4\cdot10^{17} \text{cm}^{-3}$ using
Hall resistance $R_{\rm H}$,
\beq
N_{\rm a}  = \frac{B_c}{ewR_{\rm H}}= \frac{\sigma_{aa}\sigma_{bb}}{e\sigma^{H}_{abc}}  ~,
\eql{Na-Hall}
\eeq
%
where $w$ is the thickness
of a nanowire. It also can be evaluated from first 
principle by evaluating 
Ohmic and Hall conductivity $\sigma_{aa/bb}$ and $\sigma^{H}_{abc}$ 
within the Boltzman transport theory~\cite{hurd-book72}.
Strictly speaking, the first equality in \eq{Na-Hall} is only valid  for a parabolic band
dispersion. Even though the valence band of Te near the H point
is strongly non-parabolic, by means of accurate calculations we confirm that 
\eq{Na-Hall} is still valid close
to the top of the
valence band,
and the error only  increases slightly as the 
chemical potential moves away from the top of the valence band, 
see \fref{Na-gamma}(b). 
%
\begin{figure}[t]
\centering
\includegraphics[width=1.0\columnwidth]{fig/Na.pdf}
    \caption{(a) Calculated $\gamma'_{zzzz}$ as a function of acceptor consentration. 
    (b) Comparison between acceptor concentration calculated using 
    precise cumulative density of states (CDOS) 
    and Hall resistance measurement.
    }
    \figl{Na-gamma}
\end{figure}
%%
As shown in \fref{T-dependent}(a), our estimated value of
$\gamma'_{zzzz}$ is significantly lower for this acceptor
concentration. However, $\gamma'_{zzzz}$ is extremely
  sensitive to acceptor concentration.  A tiny change of $N_{\rm a}$
  changes orders of magnitude of $\gamma'_{zzzz}$, see
  \fref{Na-gamma}(a). Therefore, an excellent agreement with the experimental
value  is difficult to achieve.

In earlier measurements at room temperature \cite{Rikken19}, the
largest component of the $\gamma$ tensor was found to be in the range
of $[10^{-9},10^{-8}]~\text{m}^2/\text{T}/\text{A}$ with an acceptor
concentration of around $10^{16} \text{cm}^{-3}$. However, it was
observed that the effect is maximized when the magnetic field is
perpendicular to the current.  This contradicts the symmetry analysis
in Table \ref{tab:Te-sigma-emcha-sym}, and therefore cannot be
described as a bulk property of trigonal tellurium, regardless of the
microscopic mechanism implied.

  \textit{The sign of the effect.} Within constant relaxation time approximation we find that for the right-handed crystal, 
  the sign of the eMChA resistivity (the $\gamma'_{zzzz}$ parameter) is negative, or equivalently, the sign of the eMChA resistivity is positive.
  The opposite signs are expected for the left-handed crystal. This is consistent with the experimental results in Ref.\cite{calavalle2022gate}.
  However, after publication of the preprint of the present manuscript, another article appeared \cite{okumura2024chiralorbitaltexturenonlinear}, 
  that claims the opposite sign of the eMChA conductivity. The lack of information on the geometrical size of the samples in Ref.\cite{okumura2024chiralorbitaltexturenonlinear}
  does not allow us to make a direct comparison of the magnitude of the effect. It should be noted that the chirality of the crystal structure in the two experiments 
  was determined by different methods. In Ref.\cite{calavalle2022gate}, the chirality was determined by modern scanning transmission electron microscopy (STEM) technique\cite{dong2020atomic,ben2021chain}, 
  while in Ref.\cite{okumura2024chiralorbitaltexturenonlinear} it was determined by the more traditional method of the etch pits\cite{Koma1970etchpits}. 
  It is known that determination of chirality of trigonal tellurium is a non-trivial task, and the results of different methods are not always consistent (see appendix B of \cite{Furakawa-chirality} for a systematic analysis).
  On the theoretical part,  another article appeared recently~\cite{Golub2020semiclassical} showing that with a different choice of the collision integral
  (beyond the constant relaxation time approximation),
  the sign of the eMChA conductivity can be changed.

\textit{Summary.}
%
In this letter, we derived equations for quantifying eMChA
in real materials using the Boltzmann-transport formalism within the
constant relaxation time approximation.  We use $p$-Te as a research
platform and employ DFT and Wannier interpolation to make
theoretical predictions for eMChA conductivity $\sigma_{abcd}$ and
measurable tensor $\gamma'_{abcd}$.  We find that with low acceptor
concentrations, the orbital Zeeman coupling is the primary origin of
the eMChA response compared with spin Zeeman coupling and Berry
curvature contributions. 
We cannot boast of an excellent agreement with experimental data, 
but neither do we find consistency between different experiments.
Therefore, we hope that our results will be a good reference and motivation for further experimental and theoretical investigations.
The developed methodology is made
available via an open-source code in order to motivate further
first-principles
investigations of other materials.

%%% Local Variables:
%%% mode: latex
%%% TeX-master: "pap"
%%% End:
