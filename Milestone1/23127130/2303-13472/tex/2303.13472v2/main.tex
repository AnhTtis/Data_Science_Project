\documentclass[acmtog,table,prologue,dvipsnames]{acmart} 
%\acmSubmissionID{XXXX}

\makeatletter

\usepackage{booktabs} % For formal tables

% TOG prefers author-name bib system with square brackets
\citestyle{acmauthoryear}
%\setcitestyle{nosort,square} % nosort to allow for manual chronological ordering



\usepackage[ruled]{algorithm2e} % For algorithms
\renewcommand{\algorithmcfname}{ALGORITHM}
\SetAlFnt{\small}
\SetAlCapFnt{\small}
\SetAlCapNameFnt{\small}
\SetAlCapHSkip{0pt}

%Our packages and commands
\newcommand{\etal}{\textit{et al}. }
\newcommand{\ie}{\textit{i.e.} }

\usepackage{amsmath}
\usepackage{color}
\usepackage{booktabs}
\usepackage{caption}
\usepackage{subcaption}
\usepackage{enumitem}
\usepackage{multirow}
\definecolor{darkorange}{rgb}{1.0, 0.55, 0.0}

\newcommand{\change}[1]{#1}

\newcommand{\apref}[1]{Appx.~\ref*{#1}}

\newcommand{\tenniscell}[5]{\begin{tabular}{@{}c@{}}\textcolor{red}{#1}\\\textcolor{teal}{#2}\\\textcolor{blue}{#3}\\\textcolor{violet}{#4}\\#5\end{tabular}}
\newcommand{\minecraftcell}[4]{\begin{tabular}{@{}c@{}}\textcolor{red}{#1}\\\textcolor{teal}{#2}\\\textcolor{blue}{#3}\\#4\end{tabular}}

\newcommand{\cellfirst}{\cellcolor{Red!40}}
\newcommand{\cellsecond}{\cellcolor{Orange!25}}
\newcommand{\cellthird}{\cellcolor{Yellow!25}}

\newcommand{\prehrulespace}{2mm}
\newcommand{\posthrulespace}{4mm}


\usepackage{amsmath}

\hyphenation{pa-ram-e-tri-zat-ion}

% Metadata Information
\acmJournal{TOG}
%\acmVolume{38}
%\acmNumber{4}
%\acmArticle{39}
%\acmYear{2019}
%\acmMonth{7}

% Copyright

\setcopyright{acmcopyright}
%\setcopyright{acmlicensed}
%\setcopyright{rightsretained}
%\setcopyright{usgov}
%\setcopyright{usgovmixed}
%\setcopyright{cagov}
%\setcopyright{cagovmixed}

% DOI
%\acmDOI{0000001.0000001_2}

% Paper history
%\received{February 2007}
%\received{March 2009}
%\received[final version]{June 2009}
%\received[accepted]{July 2009}


% Document starts
\begin{document}
% Title portion
\title{Promptable Game Models: Text-Guided Game Simulation via Masked Diffusion Models}
% 

% DO NOT ENTER AUTHOR INFORMATION FOR ANONYMOUS TECHNICAL PAPER SUBMISSIONS TO SIGGRAPH 2019!
\author{Willi Menapace}
\authornote{Work performed while the author was an intern at Snap Inc.}
\affiliation{
  \institution{University of Trento}
  \country{Italy}
}
\email{willi.menapace@unitn.it}
\author{Aliaksandr Siarohin}
\affiliation{
  \institution{Snap Inc.}
  \country{USA}
}
\email{asiarohin@snapchat.com}
\author{St\'{e}phane Lathuili\`{e}re}
\affiliation{
  \institution{LTCI, T\'{e}l\'{e}com Paris, Institut Polytechnique de Paris}
  \country{France}
}
\email{stephane.lathuiliere@telecom-paris.fr}
\author{Panos Achlioptas}
\affiliation{
  \institution{Snap Inc.}
  \country{USA}
}
\email{pachlioptas@gmail.com}
\author{Vladislav Golyanik}
\affiliation{
  \institution{MPI for Informatics, SIC}
  \country{Germany}
}
\email{golyanik@mpi-inf.mpg.de}
\author{Sergey Tulyakov}
\affiliation{
  \institution{Snap Inc.}
  \country{USA}
}
\email{stulyakov@snapchat.com}
\author{Elisa Ricci}
\affiliation{
  \institution{University of Trento, Fondazione Bruno Kessler}
  \country{Italy}
}
\email{e.ricci@unitn.it}

\renewcommand\shortauthors{Menapace, W. et al}

\begin{abstract}

\change{Neural video game simulators emerged as powerful tools to generate and edit videos. Their idea is to represent games as the evolution of an environment's state driven by the actions of its agents. While such a paradigm enables users to \emph{play} a game action-by-action, its rigidity precludes more semantic forms of control. To overcome this limitation, we augment game models with \emph{prompts} specified as a set of \emph{natural language} actions and \emph{desired states}.} The result---a \change{Promptable Game Model (PGM)}---makes it possible for a user to \emph{play} the game by prompting it with high- and low-level action sequences. Most captivatingly, our \change{PGM} unlocks the \emph{director's mode}, where the game is played by  specifying goals for the agents in the form of \change{a prompt}.
This requires learning ``game AI'', encapsulated by our animation model, to navigate the scene using high-level constraints, play against an adversary, and devise a strategy to win a point. To render the resulting state, we use a compositional NeRF representation encapsulated in our synthesis model. To foster future research, we present newly collected, annotated and calibrated Tennis and Minecraft datasets. Our method significantly outperforms existing neural video game simulators in terms of rendering quality and unlocks applications beyond the capabilities of the current state of the art. Our framework, data, and models are available at \href{https://snap-research.github.io/promptable-game-models/}{\url{snap-research.github.io/promptable-game-models}}. 


\end{abstract}

\setcopyright{acmlicensed}
\acmJournal{TOG}
\acmYear{2023} \acmVolume{1} \acmNumber{1} \acmArticle{1} \acmMonth{1} \acmPrice{15.00}\acmDOI{10.1145/3635705}

%
% The code below should be generated by the tool at
% http://dl.acm.org/ccs.cfm
% Please copy and paste the code instead of the example below.
%
\begin{CCSXML}
<ccs2012>
<concept>
<concept_id>10010147.10010371.10010372</concept_id>
<concept_desc>Computing methodologies~Rendering</concept_desc>
<concept_significance>500</concept_significance>
</concept>
<concept>
<concept_id>10010147.10010371.10010352</concept_id>
<concept_desc>Computing methodologies~Animation</concept_desc>
<concept_significance>500</concept_significance>
</concept>

</ccs2012>
\end{CCSXML}

\ccsdesc[500]{Computing methodologies~Rendering}
\ccsdesc[500]{Computing methodologies~Animation}

%
% End generated code
%

\keywords{neural radiance fields, diffusion models, human motion generation,
language modeling}

\begin{teaserfigure}
\centering
\vspace{-2mm}
  \includegraphics[width=\textwidth]{resources/teaser.pdf}
  \caption{
  We propose \change{Promptable Game Models (PGMs), controllable models of games} that are learned from annotated videos. Our \change{PGM} enables the generation of videos using \change{prompts}, a wide spectrum of \textcolor{blue}{conditioning signals} such as player poses, object locations, and detailed textual actions (see \includegraphics[]{resources/action_icon.pdf}) indicating what each player should do. Our \emph{Animation Model} uses this information to generate future, past, or interpolated \textcolor{red}{environment states} according to the learned game dynamics. At this stage, the model is able to perform complex action reasoning such as generating a winning shot if the action ``the [other] player does not catch the ball'' is specified, as shown in the figure.
  To accomplish this goal, the model decides that the bottom player should hit the ball with a ``lob'' shot, sending the ball high above the opponent, who is unable to catch it.
  Our model renders the scene from a user-defined viewpoint (see \includegraphics[]{resources/camera_icon.pdf}) using a \emph{Synthesis Model} where the style of the scene (see \includegraphics[]{resources/style_icon.pdf}) can be controlled explicitly.
  }
  \label{fig:teaser}
\end{teaserfigure}


\maketitle
% \documentclass[a4paper]{amsart}%[a4paper]
% %%%%% GENERAL MATH COMMANDS
% Reals
\newcommand{\R}{{\mathbb R}}
% Integers
\newcommand{\Z}{{\mathbb Z}}
% Naturals
\newcommand{\N}{{\mathbb N}}
% Expectation
\DeclareMathOperator*{\E}{\mathbb{E}}
% ^th notation
\newcommand{\tth}{^{\text{th}}}
% Small dots for integer range [a .. b]
\newcommand{\sdots}{\,..\,}
% Vectorized version of matrix
\newcommand{\matvec}{\mbox{vec}}

% := sign
\newcommand{\defeq}{\vcentcolon=}
% Zero function
\newcommand{\zf}{\mathbf{0}}
% Vector of ones
\newcommand{\ones}{\mathbf{1}}

% Argmin and argmax definitions
\DeclareMathOperator*{\argmax}{arg\,max}
\DeclareMathOperator*{\argmin}{arg\,min}


%%%%% PROBLEM STATEMENT NOTATION 
% \newcommandtwoopt{\St}[2][t][]{{S_{#1}^{#2}}} % State
\newcommand{\task}[1][i]{{\mathcal{T}_{#1}}} % Task, optionally takes index
\newcommand{\tasks}{\{ \task \}_{i=1}^N}
\newcommand{\losst}[1][i]{{l_{#1}}}
\newcommand{\lossv}[1][i]{{l_{#1}^{\textrm{val}}}}
\newcommand{\tasktarget}{{\mathcal{T}_{\textrm{target}}}}
\newcommand{\lossttarget}{l_{\textrm{target}}}
\newcommand{\lossvtarget}{l_{\textrm{target}}^{\textrm{val}}}
\newcommand{\lossttargetit}{l_{\textrm{target}}^{(k)}}
\newcommand{\losstotal}{l^{\textrm{total}}}
\newcommand{\lossopt}{l^*}

\newcommand{\thetait}[2]{\theta_{#1}^{(#2)}}
\newcommand{\phit}[1]{\phi^{(#1)}}
\newcommand{\hist}[2]{S_{#1}^{(#2)}}
\newcommand{\grad}[2]{G_{#1}^{(#2)}}

\newcommand{\Alg}{\textup{\textbf{Opt}}}
\newcommand{\MetaAlg}{\textup{\textbf{MetaOpt}}}

%%%%% Theorems
\newtheoremstyle{mytheoremstyle} % name
    {\topsep}                    % Space above
    {\topsep}                    % Space below
    {\itshape}                   % Body font
    {}                           % Indent amount
    {\scshape}                   % Theorem head font
    {.}                          % Punctuation after theorem head
    {.5em}                       % Space after theorem head
    {}  % Theorem head spec (can be left empty, meaning ‘normal’)
\theoremstyle{mytheoremstyle}
\theoremstyle{plain}
\newtheorem{theorem}{Theorem}
\newtheorem{proposition}{Proposition}
\newtheorem{assumption}{Assumption}
\newtheorem{definition}{Definition}
\newtheorem{lemma}{Lemma}
\theoremstyle{remark}
\newtheorem{remark}{Remark}

%
% \begin{document}
% \section{notation}\label{sec:notation}
For a positive integer $d$, we define $[d]:=\{1,2,\ldots,d\}$. 
The set of non-negative integers is denoted by $\NN:=\{0,1,2,\ldots\}$.
The cardinality of a set $S$ is denoted by $|S|$.
%Operations on $[d]$ cyclically.

Our \emph{graphs} are finite and undirected. We allow multiple edges and loops. A \emph{simple graph} is a graph without multiple edges or loops. 


A \emph{plane map} is a connected planar graph drawn in the plane without edge crossing, considered up to continuous deformation. 
The \emph{faces} of a plane map are the connected components of the complement of the graph. The infinite face is called \emph{outer face}, and the finite faces are called \emph{inner faces}. The vertices and edges incident to the outer face are called \emph{outer} while the other are called \emph{inner}. 
The numbers $\vv$, $\ee$ and $\ff$ of vertices, edges and faces of a plane map are related by the \emph{Euler relation}  $\vv+\ff=\ee+2$. 


We now define the class of plane maps which will be relevant for this article.
\begin{definition}\label{def:d-adapted}
A \emph{$d$-map} is a plane map such that the inner faces have degree at most $d$, and the outer face has degree $d$ and is incident to $d$ distinct vertices (in other words, the contour of the outer face is a simple cycle). 
We will assume that the outer vertices of a $d$-map are labeled $v_1,v_2,\ldots, v_d$ in clockwise order along the boundary of the outer face. %, as in Figure \ref{???}.\\
A \emph{$d$-adapted map} is a $d$-map such that any simple cycle which is not the contour of a face has length at least $d$.\\
\end{definition}
We point out that $d$-adapted maps are necessarily 2-connected (because a cut point in a $d$-map $G$ implies the existence of a simple cycle of length strictly less than the degree of an inner face of $G$, which shows that $G$ is not $d$-adapted).


In a plane map, a \emph{corner} is the sector delimited by two consecutive (half-)edges around a vertex. It is called an \emph{inner corner} if it lies in an inner face, and an \emph{outer corner} otherwise.
The \emph{degree} of a vertex or face is its number of incident corners. A  \emph{$d$-angulation} is a plane map with all faces of degree $d$. A \emph{$d$-angulation of the $k$-gon} is a plane map with inner faces of degree $d$, and outer face of degree $k$. 
A graph is \emph{bipartite} if it admits a bicoloring of its vertices such that adjacent vertices have different colors. It is known that a plane map is bipartite if and only if all its faces have even degree. For $k\geq 2$, a graph is called \emph{$k$-connected} if it is connected and the deletion of any subset of $(k-1)$ vertices does not disconnect it (loops are forbidden for $k\geq 2$, multiple edges are forbidden for $k\geq 3$). 




Let $G$ be an undirected graph. An \emph{arc} of $G$ is an edge $e$ of $G$ together with a chosen orientation of $e$ (so each edge of $G$ correspond to two arcs). The arc \emph{opposite} to an arc $a$, denoted by $-a$, is the arc corresponding to the same edge as $a$ but with the opposite direction. 
The endpoints of an arc $a$ are called the \emph{initial} and \emph{terminal} vertices of $a$ (with $a$ oriented from the initial vertex to the terminal vertex).  If $v$ is the initial (resp. terminal) vertex of the arc $a$, then we say that $a$ is an \emph{outgoing arc} (resp. \emph{ingoing arc}) at $v$. 
\\

%In a graph, a \emph{walk} (of length $k$) is a sequence $v_1,e_1,v_2,\ldots,e_k,v_{k+1}$ that alternates vertices and edges, such that $e_i$ connects $v_i$ to $v_{i+1}$ for $i\in[k]$. It is called a \emph{closed walk} if $v_1=v_{k+1}$. 
%\OB{Made a change in the def of walk (talking about arcs instead). Should we call them ``paths'' rather than ``walks''?}
A \emph{path} in an undirected graph $G$ is a sequence of arcs $a_1,a_2,\ldots,a_k$ such that the terminal vertex of $a_i$ is the initial vertex of $a_{i+1}$ for all $i\in[k-1]$. It is called a \emph{closed path} if the terminal vertex of $a_k$ is the initial vertex of $a_1$. A \emph{cycle} is a (cyclically ordered) closed path. A path or cycle is called \emph{simple} if it does not pass twice by the same vertex. The \emph{girth} of a graph is the minimum length of its simple cycles.   In a plane map, a closed path formed by the arcs around a face is called \emph{contour} of that face. It is known that face contours are simple cycles if the plane map is 2-connected. 
A simple cycle on a plane map is called \emph{counterclockwise} (resp. \emph{clockwise}) if the direction of arcs is counterclockwise (resp. clockwise) around the cycle.

Let $G$ be a graph.  Given an orientation of $G$, a \emph{directed path} (resp. \emph{directed cycle}) is a path (resp. cycle) $a_1,a_2,\ldots,a_k$ such that every arc $a_i$ is oriented according to the orientation of $G$.
A \emph{weighted orientation} of $G$ is an assignment of a non-negative integer to each arc of $G$. Given a weighted orientation $\cW$ of $G$, we call \emph{weight} of an edge the sum of the weights of the two corresponding arcs. 
Weighted orientations are a generalization of the classical (unweighted) orientations of $G$. Indeed the ``unweighted'' orientations of $G$ can be identified to the weighted orientations of $G$ such that the weight of every edge is 1 (for each edge, the arc of weight 1 is taken as the orientation of the edge). The \emph{outgoing weight} (shortly, the \emph{weight}) of a vertex $v$ is the sum of the weights of the arcs going out of $v$. Given a weighted orientation, we call \emph{positive path} (resp. \emph{positive cycle}) a path (resp. cycle) $a_1,a_2,\ldots,a_k$ such that the weight of every arc is positive (this generalizes the notion of \emph{directed path} and \emph{directed cycle}).  




A \emph{tree} is a connected, acyclic graph. For a tree $T$ with a vertex $v$ distinguished as its \emph{root}, we apply the usual ``genealogy'' vocabulary about trees, where $v$ is an \emph{ancestor} of all the other vertices, and every non-root vertex incident to $T$ has a \emph{parent} in $T$, etc. 
We say that we \emph{orient the tree $T$ toward its root} by orienting every edge from child to parent. With this orientation, every non-root vertex of $T$ is incident to one outgoing edge in $T$ (the edge leading to its parent).
%\OB{changed: calling ``subtree'' instead of ``tree''}
A \emph{subtree} of a graph $G$ is a subset of edges of $G$ such that this set of edges together with the incident vertices forms a tree. A \emph{spanning tree} of $G$ is a subtree of $G$ incident to every vertex of $G$. 





%\end{document}

\section{Introduction}
\label{sec:introduction}
% \begin{itemize}
%     % Diffusion of FL
%     \item {\st{Diffusion of FL}}
%     % Security threats to FL
%     \item {\st{Security threats to FL with particular focus on model poisoning}}
%     % Limitations of existing countermeasures
%     \item {\st{Current countermeasures (e.g., KRUM) and their limitations}}
%     % Proposed method and its advantages
%     \item {\st{Intuitive description of the proposed method and its difference (i.e., advantages) w.r.t. state of the art}}
%     % Main contributions
%     \item {\st{Summary of the main contributions of this work}}
%     % Paper's structure and organization
%     \item {\st{Paper's structure and organization}}
% \end{itemize}

% Diffusion of FL
Recently, {\em federated learning} (FL) has emerged as the leading paradigm for training distributed, large-scale, and privacy-preserving machine learning (ML) systems~\cite{mcmahan2017googleai,mcmahan2017aistats}. 
The core idea of FL is to allow multiple edge clients to collaboratively train a shared, global model without disclosing their local private training data.
%Specifically, an FL system consists of a central server and many edge clients; 
A typical FL round involves the following steps: {\em(i)} the server randomly picks some clients and sends them the current, global model; {\em(ii)} each selected client locally trains its model with its own private data; then, it sends the resulting local model to the server;\footnote{Whenever we refer to global/local model, we mean global/local model {\em parameters}.} {\em(iii)} the server updates the global model by computing an \emph{aggregation function}, usually the average (FedAvg), on the local models received from clients.
% \begin{enumerate}
%     \item[{\em(i)}] the server sends the current, global model to the clients and appoints some of them for training;
%     \item[{\em(ii)}] each selected client locally trains its copy of the global model with its own private data; then, it sends the resulting local model back to the server;\footnote{Whenever we refer to global/local model, we mean global/local model {\em parameters}.}
%     \item[{\em(iii)}] the server updates the global model by computing an \emph{aggregation function} on the local models received from clients (by default, the average, also referred to as FedAvg~\cite{mcmahan2017aistats}).
% \end{enumerate}
This process goes on until the global model converges. %(e.g., after a certain number of rounds or other similar stopping criteria).
%\\
% The advantages of FL over the traditional, centralized learning paradigm are undoubtedly clear in terms of flexibility/scalability (clients can join/disconnect from the FL network dynamically), network communications (only model weights\footnote{We will use \textit{parameters} and \textit{weights} interchangeably.} are exchanged between clients and server), and privacy (each client's private training data is kept local at the client's end and not uploaded to the server).
\\
% Security threats to FL
%However, the growing adoption of FL also raises security concerns~\cite{costa2022covert}, particularly about its confidentiality, integrity, and availability.
Although its advantages over standard ML, FL also raises security concerns~\cite{costa2022covert}. %, particularly about its confidentiality, integrity, and availability~\cite{costa2022covert}.
% OLD, LONG VERSION
% Indeed, some work deals with privacy leakage that may expose the local data of some clients~\cite{melis2019sp}. 
% A large body of work, instead, investigates attacks that usually aim to detriment the predictive accuracy of the learned global model. For instance, \emph{data poisoning} attacks achieve this goal by letting an adversary pollute the training set of some corrupt FL clients with maliciously crafted examples~\cite{jagielski2018sp}.
% Similarly, in \emph{model poisoning} the attacker attempts to tweak the global model weights~\cite{bhagoji2019pmlr} by directly perturbing the local model's weights of some infected FL clients before these are sent to the central server for aggregation, usually via so-called Byzantine attacks. 
% It turns out that Byzantine model poisoning attacks severely impact standard FedAvg; therefore, more robust aggregation functions must be designed to make FL systems secure.
Here, we focus on \emph{untargeted model poisoning} attacks~\cite{bhagoji2019pmlr}, where an adversary attempts to tweak the global model weights %\footnote{We will use the terms \textit{parameters} and \textit{weights} interchangeably.} 
by directly perturbing the local model's parameters of some infected clients before these are sent to the central server for aggregation.
In doing so, the adversary aims to jeopardize the global model \textit{indiscriminately} at inference time.
Such model poisoning attacks severely impact standard FedAvg; therefore, more robust aggregation functions must be designed to secure FL systems.
\\
% In this paper, we focus on designing a novel robust aggregation scheme at the server's end to contrast the effect of Byzantine model poisoning attacks.
%
% Current countermeasures and their limitations
%Several countermeasures have been proposed in the literature to combat model poisoning attacks on FL systems.
% Some methods use simple statistics more robust than plain average to smooth the impact of malicious updates (e.g., Trimmed Mean and FedMedian~\cite{yin2018icml}). 
% Other defenses implement outlier detection techniques to discard malicious updates from the aggregation performed at the server's end. Those are either based on heuristics (e.g., Krum/Multi-Krum~\cite{blanchard2017nips} and Bulyan~\cite{mhamdi2018pmlr}) or data-driven approaches (e.g., K-means clustering~\cite{shen2016acm} or DnC via spectral analysis~\cite{shejwalkar2021ndss}). 
% Finally, some strategies rely on a centralized ``source of trust'' to spot potential malicious updates (e.g., FLTrust~\cite{cao2020fltrust}).
% Several countermeasures have been proposed in the literature to combat model poisoning attacks on FL systems, i.e., to discard possible malicious local updates from the aggregation performed at the server's end. 
% These techniques range from simple statistics more robust than plain average (e.g., Trimmed Mean and FedMedian~\cite{yin2018icml}) to outlier detection heuristics (e.g., Krum/Multi-Krum~\cite{blanchard2017nips} and Bulyan~\cite{mhamdi2018pmlr}) or data-driven approaches (e.g., spectral analysis via K-means clustering~\cite{shen2016acm} or spectral analysis), or methods based on ``source of trust'' (e.g., FLTrust~\cite{cao2020fltrust}).
% OLD, LONG VERSION
%Several countermeasures have been proposed in the literature to combat Byzantine model poisoning attacks on FL systems.
% Descriptive statistics
% For example, Trimmed Mean and FedMedian aggregate local model updates using more robust statistics than standard average~\cite{yin2018icml}.
%
% % Heuristics for outlier detection
% Many existing Byzantine-resilient strategies implement some outlier detection heuristics to discard the model updates sent by potentially malicious clients from the input of the aggregation function.
% One of the most popular heuristics is Krum~\cite{blanchard2017nips}.
% This strategy tries to mitigate the impact of Byzantine attacks by selecting as a global model the local model with the smallest sum of Euclidean distances to {\em all} the other local models.
% Although powerful, Krum requires the server to know (or, at least, estimate) the number of malicious FL clients upfront, which is generally impossible in a realistic attack scenario. %
% Moreover, Krum may become ineffective for complex, high-dimensional model parameter spaces due to the curse of dimensionality.
% Bulyan~\cite{mhamdi2018pmlr} tries to overcome this issue by combining Krum with a variant of Trimmed Mean.
% % Data-driven outlier detection
% Other strategies use data-driven outlier detection techniques -- e.g., via K-means clustering~\cite{shen2016acm} -- to spot potential malicious local model updates. 
% %For instance, Shen et al. propose to cluster local model updates with K-means and thus identify outliers.
%
% % Other techniques
% As far as the server is concerned, any local model received can be from a potential malicious client. 
% FLTrust~\cite{cao2020fltrust} assumes the server acts as a client, i.e., trains a local model on an additional {\em trustworthy} dataset at the server's end and compares it against all the local models from other clients. 
% This way, the server can rely on some ``source of trust'' when discarding potentially malicious clients.
%\\
% Limitations of existing Byzantine-resilient strategies
Unfortunately, existing defense mechanisms either rely on simple heuristics (e.g., Trimmed Mean and FedMedian by~\cite{yin2018icml}) or need strong and unrealistic assumptions to work effectively (e.g., foreknowledge or estimation of the number of malicious clients in the FL system, as for Krum/Multi-Krum~\cite{blanchard2017nips} and Bulyan~\cite{mhamdi2018pmlr}, which, however, cannot exceed a fixed threshold).
Furthermore, outlier detection methods using K-means clustering~\cite{shen2016acm} or spectral analysis like DnC~\cite{shejwalkar2021ndss} do not directly consider the temporal evolution of local model updates received.
Finally, strategies like FLTrust~\cite{cao2020fltrust} require the server to collect its own dataset and act as a proper client, thereby altering the standard FL protocol.
\\
% OLD, LONG VERSION
% Overall, existing Byzantine-resilient strategies are either simple heuristics (e.g., FedMedian) or, if they are more complex, they rely on strong and unrealistic assumptions to work effectively (e.g., knowing the number of malicious clients in the FL system in advance, as for Krum and alike).
% Furthermore, data-driven outlier detection methods do not consider the temporary evolution of local model updates received (e.g., K-means clustering). 
% Finally, strategies like FLTrust requires the server to collect its own dataset and act as a proper client, thereby altering the standard FL protocol.
%
% Description of the proposed method
This work introduces a novel pre-aggregation \textit{filter} robust to untargeted model poisoning attacks. Notably, this filter $(i)$ operates without requiring prior knowledge or constraints on the number of malicious clients and $(ii)$ inherently integrates temporal dependencies. 
The FL server can employ this filter as a preprocessing step before applying \textit{any} aggregation function, be it standard like FedAvg or robust like Krum or Bulyan.
Specifically, we formulate the problem of identifying corrupted updates as a multidimensional (i.e., matrix-valued) time series anomaly detection task. 
The key idea is that legitimate local updates, resulting from well-calibrated iterative procedures like stochastic gradient descent (SGD) with an appropriate learning rate, show \textit{higher predictability} compared to malicious updates. This hypothesis stems from the fact that the sequence of gradients (thus, model parameters) observed during legitimate training exhibit regular patterns, as validated in Section~\ref{subsec:intuition}. %until convergence. 
%This regularity may be more pronounced for smooth convex loss functions, but it can still be captured within an appropriate time window, even for more complex and convoluted loss surfaces. 
%We provide evidence of this claim in Appendix~B, where we show that the average mutual information (i.e., ``predictability''), calculated over pairs of legitimate model updates sent at different FL rounds, is significantly higher than the corresponding computation for a malicious client.
\\
Inspired by the matrix autoregressive (MAR) framework for multidimensional time series forecasting~\cite{chen2021je}, we propose the FLANDERS ({\em \textbf{F}ederated \textbf{L}earning meets \textbf{AN}omaly \textbf{DE}tection for a \textbf{R}obust and \textbf{S}ecure}) filter.
The main advantages of FLANDERS over existing strategies like FLDetector~\cite{zhao2020multivariate} are its resilience to large-scale attacks, where $50\%$ or more FL participants are hostile, and the capability of working under realistic non-iid scenarios.
We attribute such a capability to two key factors: $(i)$ FLANDERS works without knowing a priori the ratio of corrupted clients, and $(ii)$ it embodies temporal dependencies between intra- and inter-client updates, quickly recognizing local model drifts caused by evil players. Below, we summarize our main contributions:

\begin{itemize}
\item[{\em(i)}]
We provide empirical evidence that the sequence of models sent by legitimate clients is more predictable than those of malicious participants performing untargeted model poisoning attacks.
\\
\item[{\em(ii)}] 
We introduce FLANDERS, the first pre-aggregation filter for FL robust to untargeted model poisoning based on multidimensional time series anomaly detection.
\\
\item[{\em(iii)}] 
We integrate FLANDERS into Flower,\footnote{\scriptsize{\url{https://flower.dev/}}} a popular FL simulation framework for reproducibility.
\\
\item[{\em(iv)}] 
We show that FLANDERS improves the robustness of the existing aggregation methods under multiple settings: different datasets, client's data distribution (non-iid), models, and attack scenarios.
\\
\item[{\em(v)}] 
We publicly release all the implementation code of FLANDERS along with our experiments.\footnote{\scriptsize{\url{https://anonymous.4open.science/r/flanders_exp-7EEB}}}
\end{itemize}

% Paper's structure and organization
The remainder of the paper is structured as follows. %some related work and the current state-of-the-art solutions to security issues that FL entails. 
Section~\ref{sec:background} covers background and preliminaries. 
In Section~\ref{sec:related}, we discuss related work.
Section~\ref{sec:problem} and Section~\ref{sec:method} describe the problem formulation and the method proposed. % to tackle it. 
Section~\ref{sec:experiments} gathers experimental results. %, and Section~\ref{sec:limitations} discusses some limitations of this work.
Finally, we conclude in Section~\ref{sec:conclusion}.
 %discusses the limitations of this work and draws future research directions.
%reports conclusions and draws perspectives for future research directions.

%%%%%%% OLD %%%%%%%
%to overcome the resilience of Byzantine failures in distributed Stochastic Gradient Descent computations. 
% The strength of Krum is its time complexity, which is linear in the gradient dimension. 
% However, the robustness of the approach is guaranteed for gradient-based learning applications only when the majority of the clients are not compromised. 
% Besides, the aggregation mechanism of Krum, as well as that of similar methods, is robust from a coarse-grained perspective and does not provide solutions to errors and perturbations that may occur at inference time.
%A related approach to~\cite{blanchard2017nips} is the work of Su et al.~\cite{su2016dc}. Here, the authors propose an iterated approximate agreement to tackle a multi-layer scenario attacked by Byzantine agents. 
%However, the method works efficiently on the sole discrete context and it is inapplicable to continuous state environments.
%\gabri{Maybe, we should just talk about the main limitations of existing countermeasures without digging into their details (or, we can just mention Krum as this is the most popular one). I will move the description of all these methods to the Related Work section.}
\section{Related work}
\noindent \textbf{Video foundation models.}
With sufficient computational power and an abundant source of data, there have been attempts to build a single large-scale foundation model that can be adapted to diverse downstream tasks.
Along with the success of foundations models in the natural language processing domain~\cite{brown2020language,chen2021evaluating,devlin2019bert} and in computer vision~\cite{bertasius2021space,jia2021scaling,radford2021learning}, video data has become another data type of interest, as it has grown in scale due to numerous internet video-sharing platforms.
Accordingly, several methods to train a video foundation model have been proposed.
Due to the innate multi-modality of video data, \textit{i.e.}, a combination of visual $\cdot$ vocal $\cdot$ textual context, most works have centered around the variations of the cross-modal attention mechanism \cite{akbari2021vatt,bertasius2021space,gabeur2020multi,luo2020univl,neimark2021video,tan2021look,wei2020multi,yang2021taco}.
In addition, as most video data lack proper labels or descriptions, contrastive learning methods were studied to learn meaningful feature representations or enhance video-text alignment in a self-supervised manner \cite{akbari2021vatt,kuang2021video,luo2020univl,yang2021taco}.

More specifically, MERLOT \cite{zellers2021merlot} proposed a multi-modal representation learning method for visual commonsense reasoning, which also performed well in twelve video reasoning tasks.
VATT \cite{akbari2021vatt} introduced a multi-modal learning method via contrastive learning. 
The pre-trained model performed well in a variety of vision tasks from image classification to video action recognition and zero-shot video retrieval.
Another representative work, UniVL \cite{luo2020univl} proposed a straightforward pre-training method with auxiliary loss functions. 
After fine-tuning on a specific task, the pre-trained model performed outstandingly in a wide range of tasks of text-to-video retrieval, action segmentation, action step localization, video sentiment analysis, and video captioning.
Other foundation models for multiple video tasks include \cite{li2020hero,sun2019learning,sun2019videobert,zhu2020actbert,fu2021violet,wang2022all}. 

\noindent \textbf{Auxiliary learning.}
In order to enhance the performance of one or a multitude of primary tasks, auxiliary learning methods can be incorporated.
\cite{ruder2017overview} introduced Multi-task learning (MTL) to the deep neural networks by training a single model with multiple task losses to assist learning on the main task.
Such a method is generally adapted to pre-train the foundation models in the self-supervised manner~\cite{li2020hero,sun2019learning,sun2019videobert,zhu2020actbert,fu2021violet,wang2022all}.
However, these various pretext task losses used in the pre-training phase are ignored in the fine-tuning phase, and only the primary task loss is minimized.

Recently, meta-learning methods have been introduced for auxiliary learning.
\cite{liu2019self,navon2020auxiliary,shu2019meta} proposed a meta-learning method in which the model learns auxiliary tasks to generalize well to unseen data. 
In these settings, a separate subset of data is held out as the primary task, while the others are used as auxiliary tasks that aid the primary task's performance.
Similar methods were adopted for computer vision tasks such as semantic segmentation \cite{xu2021leveraging}.
Other domain applications include navigation tasks with reinforcement learning \cite{ye2021auxiliary}, or self-supervised learning methods on graph data \cite{hwang2020self}.
\section{Method}
\label{sec:method}

% \ml{``Inconsistent'' to ``large variation''}

% In this section, we propose our methods based on the observations in Section \ref{sec:motivation}.
In this section, we propose two techniques to further enhance the strong baseline to capture the variation of activation distributions better.
We first introduce spatial re-scaling to adapt the network to pixel-to-pixel variation.
We then propose channel-wise shifting and re-scaling to better capture the channel-to-channel variation.
Meanwhile, as both of the two methods are image-dependent, the image-to-image variation can be captured naturally.
By combining the two methods with our strong baseline, we build our enhanced BNN for SR, named EBSR.

% Because the activation distributions among pixels, channels and images have large variations \red{**are highly inconsistent} in SR networks, we introduce spatial re-scaling to adapt to pixel-wise variations and channel shift and re-scaling to adapt to channel-wise variations. And both of them are image-dependent to adapt to image-wise variations, which means during inference our network re-scales and shifts the distributions of activations flexibly for different input images. Based on these methods, we build an enhanced binary neural network for image super-resolution (EBSR).

% According to [3], the difference of activation magnitudes indicates different scaling factors are needed for each pixel.

\subsection{Spatial Re-scaling}
% It is better to use different scaling factors for different pixels to reduce the quantization error and retain more detailed information for image super-resolution. 

% \ml{In the main method, we do not need to introduce the previous works but can focus on introducing our own method. Channel rescaling in Real-to-binary Net is not relevant in this context.}

% Re-scaling the output of binary convolutions was proposed at the birth of BNN in XNOR-Net \cite{rastegari2016xnor} to reduce quantization error and improve accuracy for image classification tasks.
% It is computed as below:
% \begin{equation}
% \mathcal{A} * \mathcal{W} \approx(\operatorname{sign}(\mathcal{A}) \circledast \operatorname{sign}(\mathcal{W})) \odot \mathcal{K} \alpha
% \label{eq:xnor-net rescale}
% \end{equation}
% where $\circledast$ denotes the binary convolution and $\odot$ denotes the element-wise multiplication.
% $\mathcal{A}$, $\mathcal{W}$, $\alpha$, and $\mathcal{K}$ denote the activation, weight, weight scaling factor, and activation scaling factor, respectively.
%  Later in XNOR-Net++ \cite{bulat2019xnor}, Bulat et al. fuse the activation and weight scaling factors into a single one that is learned end-to-end based on gradients and this improves the classification accuracy on ImageNet dataset.

% % It is computed as Eq.~\ref{eq:xnor-net rescale}, where $\circledast$ denotes 
% %  the binary convolution and $\odot$ denotes the element-wise multiplication. The binary convolution of $\mathcal{A}$ and $\mathcal{W}$ is rescaled by the weight scaling factor $\alpha$ and the activation scaling factor $\mathcal{K}$, both of which are calculated analytically.


% \zc{Similarly, you should explain the meaning of A, W and the operators $\circledast$ in the formula}
% Then in Real-to-binary Net \cite{martinez2020training}, Martinez et al. used a data-driven channel re-scaling module that takes the pre-convolution activations as input to predict the activation scaling factor. Unlike that in XNOR-Net++ \cite{bulat2019xnor}, these scaling factors are not fixed during inference but rather inferred from data. By doing this, they further improved the classification accuracy on ImageNet over XNOR-Net++. 
As is shown in Figure \ref{fig:pixel}, activation distributions have large pixel-to-pixel variation in SR networks
and the difference of activation magnitudes indicates different scaling factors are preferred for different pixels.
Inspired by \cite{martinez2020training}, we propose spatial re-scaling to better adapt the network to the spatial variation
of activation distributions in SR networks.
% fit the various pixel-wise distributions in SR networks.
We take the real-valued activations $A$ before convolution as input and predict pixel-wise scaling factors $S(A)$, which re-scale the binary convolution output. Spatial re-scaling process can be formulated as follows:
\begin{equation}
A * W \approx(\operatorname{sign}(A) \circledast \operatorname{sign}(W)) \odot \alpha \odot S(A)
\label{eq:spatial rescale}
\end{equation}
where $\circledast$ denotes 
the binary convolution and $\odot$ denotes the element-wise multiplication. $A$, $W$, $\alpha$, and $S\left(A\right)$ denote real-valued activations, weights, the scaling factor of weights, and the spatial-wise scaling factor of activations respectively. $S\left(A\right) \in \mathbb{R}^{1\times H\times W}$ can be calculated with a convolution and a sigmoid function.
% as $\sigma\left( CONV\left(A\right)\right)$. 
As shown in Figure \ref{fig:method}(a), real-valued activations first go through a convolution layer,
which has an input channel of $C$ and an output channel of 1, 
and then pass through a sigmoid function to produce the scaling factors $S(A)$ along the spatial dimension.
During inference, the scaling factor will change dynamically according to different input feature maps.
By re-scaling binary convolution output using $S(A)$, we can reduce the quantization error and the original pixel-wise information in FP activation
will be preserved much better.
Spatial re-scaling leads to a large PSNR improvement of 0.24 dB (from 30.30 dB to 31.54 dB) on Set5 and 0.22 dB (from 25.09 dB to 25.31 dB)
on Urban100 compared with our strong baseline. 

\subsection{Channel-wise Shifting and Re-scaling}

\begin{table}[!tb]
\centering
\caption{Comparison between whether to fuse channel-wise shifting and re-scaling or not based on our baseline with spatial re-scaling. }
\label{tab:fusing}

\scalebox{0.65}{
\begin{tabular}{c|cc|cc|cc}
\hline
\multirow{2}{*}{Method}     & \multirow{2}{*}{OPs} & \multirow{2}{*}{Params} & \multicolumn{2}{c|}{Set5} & \multicolumn{2}{c}{Urban100} \\ \cline{4-7} 
                            &                      &                         & PSNR        & SSIM        & PSNR          & SSIM         \\ \hline
Baseline + spatial re-scale & 2.16G                & 0.05M                   & 31.54       & 0.883       & 25.31         & 0.759        \\
+ channel-wise shift and re-scale             & 2.34G                & 0.09M                   & 31.61       & 0.885       & 25.35         & 0.761        \\
+ w/ fusing                   & 2.27G                & 0.08M                   & \textbf{31.64}       & \textbf{0.885}       & \textbf{25.36}         & \textbf{0.761}        \\ \hline
\end{tabular}
}
\end{table}

In SR networks, activation distributions exhibit larger channel-to-channel variation (Figure \ref{fig:chl}).
Both the mean and magnitude of the activation distributions vary significantly across channels.
% Thus we use channel-wise shifting and re-scaling to adapt to various channel-wise distributions. 
\cite{martinez2020training} has proposed the data-driven channel re-scaling, 
but our method differs from them in further introducing data-driven thresholds to handle the channel-wise variation of both mean and magnitude.
Since the blocks to generate the scaling factors and thresholds are very similar, we further propose to fuse them into one module.
% and fusing channel-wise shifting and re-scaling into one module.
We evaluate the effect of fusing the two blocks in Table \ref{tab:fusing}.
With channel-wise shifting and re-scaling fused, our models have fewer operations and parameters overhead and slightly higher performance.

For the specific process, we take the real-valued activations as input and predict different thresholds and scaling factors for each channel. They are also image dependent, e.g., $\beta_{i}$ in Eq.\ref{eq:act_binarize} is no longer fixed during inference but generated according to different input feature maps. Channel-wise shifting and re-scaling can be formulated as follows:
\begin{equation}
A * W \approx(\operatorname{sign}(A-C_s(A)) \circledast \operatorname{sign}(W)) \odot \alpha \odot C_r(A)
\label{eq:channel-wise_shift_and_rescale}
\end{equation}
where $\circledast$ denotes 
the binary convolution and $\odot$ denotes the element-wise multiplication. $C_s(A), C_r(A) \in \mathbb{R}^{C\times1\times1}$ denote the channel-wise threshold and scaling factor, respectively. 
We show the block diagram in Figure \ref{fig:method}(b).
The real-valued input feature map is first squeezed to a ${C\times1\times1}$ vector by a global average pooling (GAP) layer.
The subsequent fully connected layers and ReLU learn the channel-wise information and output a ${2C\times1\times1}$ vector.
Then the ${2C\times1\times1}$ vector is split into two ${C\times1\times1}$ vectors.
We use the first $C$ channels as the channel-wise bias and pass the last $C$ channels through a sigmoid layer 
as the channel-wise scaling factor, which are used to shift the real-valued activations and re-scale the binary convolution output, respectively. 


% \ml{We can mention previously, channel-wise re-scale has been proposed. We propose to fuse them. Add the comparison between fuse v.s. no fuse.}

\begin{figure}[!tbp]%
  \centering
    \includegraphics[width=0.4\textwidth]{fig/methods.png}
  
% \subfloat[channel-wise shifting\&re-scale]{
%     \label{subfig:channel-wise shifting and re-scale}
%     \includegraphics[width=0.2\textwidth]{fig/chl shift and rescale.png}
%   }

  \caption{Block diagram for spatial re-scaling, and channel-wise shifting and re-scaling.} 
  % Input A is the real-valued activation tensor and C, H, and W denote its dimension. GAP stands for global average pooling. The reduction ratio r is set to 16 for a better trade-off between the performance and the number of operations and parameters.}
  \label{fig:method}
\end{figure}


\subsection{Network Structure}

Combining the spatial re-scaling and the channel-wise shifting and re-scaling methods, we construct the enhanced convolution layer (E-Conv).
Then we build our EBSR model based on E-Conv.
In Figure \ref{fig:E-conv}, we compare the binary convolution layer used in the baseline network and our proposed E-Conv.
We use spatial and channel-wise scaling factors to re-scale the binary convolution output,
and use channel-wise shifting to learn appropriate thresholds for each channel before binarization.
The scaling factors and threshold used in E-Conv are learnable and depend on the real-valued input activations.
In this way, our proposed EBSR can adapt to pixel-to-pixel, channel-to-channel, and image-to-image variations
to reduce the large binarization error and preserve more details.
% In this way, our proposed E-Conv reduces the large quantization error caused by binarization and keeps the original information of input feature maps to a large extent.


\begin{figure}[!tb]%
  \centering

    \includegraphics[width=0.5\textwidth]{fig/E-conv.png}

  \caption{Comparison of (a) the binary convolution layer with a skip connection used in our baseline network and (b) the proposed E-Conv.}
  \label{fig:E-conv}
\end{figure}


Figure \ref{fig:network} shows the basic block based on the E-Conv and our EBSR composed of the basic blocks. Following existing works, the convolution layers in the head and tail modules are not binarized. We choose the lightweight EDSR which has 16 basic blocks and 64 channels, and EDSR which has 32 basic blocks and 256 channels as our backbones, which correspond to EBSR-light and EBSR, respectively.

\begin{figure}[!tb]%
  \centering
  {
    \includegraphics[width=0.35\textwidth]{fig/network.png}
  }
  
  \caption{The structure of our proposed EBSR.  Convolution layers in purple are real-valued vanilla 3x3 convolutions.}
  \label{fig:network}
\end{figure}
For this chapter, fix a prime $p$. We first discuss deformations of coalgebras from $\F_{p}$
to the $p$-adic integers and further to the $p$-completed sphere $\S_{p}^{\wedge}$ which leads
us to the question of how coalgebras behave with respect to $p$-completion. We introduce the
notion of a $p$-complete coalgebra and show that this is well behaved with respect to the
deformation theory discussed in the previous chapter. We then use this to iterate
Proposition~\ref{witt} and prove our main results, namely the existence of Witt Vectors
and spherical Witt Vectors for formally \'etale coalgebras. Then we specialize to the case
of homology coalgebras, show that for a finite space $X$ the coalgebra $\F_{p}[X]$ is formally
\'etale, and answer our initial question about the relation between $\S[X]^{\wedge}_{p}$
and $\F_{p}[X]$

\subsection{Coalgebras and $p$-completion}

We have seen that the functors that interest us are all \textit{nilcomplete}. For a nilcomplete
functor $X:\rm{CAlg}^{\rm{cn}} \to \cl{S}$ and a connective $\bb{E}_{\infty}$-ring $R$, we can construct
lifts from $X(\pi_{0}R)$ to $X(R)$ inductively along the Postnikov tower
\[ \dots \to \tau_{\leq2}R \to \tau_{\tau\leq 1}R \to \tau_{\leq0} R =\pi_{0}R.\]
This is however not quite enough to obtain our goal of lifting from $\F_{p}$ to the
$p$-completed sphere, we first need to pass to $\Z_{p}= \pi_{0}\S_{p}^{\wedge}$.
Explicitly, this means constructing lifts against the tower
\[\dots \to \Z/p^{3}\to \Z/p^{2}\to \Z/p\to \F_{p}\]
which is clearly presents a different problem. With the machinery developed thus far, we can already
prove the following for a general deformation problem.

\begin{proposition}\label{liftpgen}
  Let $X: \rm{CAlg}^{\rm{cn}} \to \cl{S}$ be a cohesive functor and $A\in X(\F_{p})$
  such that $T_{X_{A}}\simeq 0$. Then there exists a unique lift of $A$ to a point in
  $\flim_{n}X(\Z/p^{n})$.
\end{proposition}
\begin{proof}
  Set $A_{0}= A$, we inductively construct lifts against the tower of square zero extensions
  \[\dots \to \Z/p^{3} \to \Z/p^{2}\to \F_{p}.\]
  Suppose we have already constructed lifts $A_{k}$ for $k\le n$ for some $n$.
  Applying Proposition~\ref{bc} inductively, we get that
  \[T_{X_{A_{n}}}^{\F_{p}} \simeq T^{\F_{p}}_{X_{A_{0}}} \simeq 0.\]
  Thus, since $\Z/p^{n+1}\to \Z/p^{n}$ is a square zero extension with fiber $\F_{p}$,
  Proposition~\ref{deformations} implies that the fiber
  \[X_{A_{n}}^{\Z/p^{n+1}}=\rm{fib}_{A_{n}}(X(\Z/p^{n+1})\to \Z/p^{n})\]
  is contractible and we find an essentially unique lift $A_{n+1}$. This proves the claim.
\end{proof}
 Of course, for an arbitrary functor $X:\rm{CAlg}^{\rm{cn}} \to \cl{S}$ the natural map
$X\to \flim_{n}X(\Z/p^{n})$ might not be an equivalence, meaning that in this generality
we can only construct pro-$p$ objects of $X$ using this inductive method.
In fact, we have that $\rm{cCAlg}_{\Z_{p}}\neq  \flim_{n} \rm{cCAlg}_{\Z/p^{n}}$. To remedy
this problem we show that this limit admits a description via \textit{$p$-complete} coalgebras.
To do this, we first recall some facts about $p$-complete modules.

\begin{definition}
Let $R$ be an $\bb{E}_{\infty}$-ring, then $M \in \rm{Mod}_{R}$ is called
$p$-\textit{complete} if the limit
\[ \lim \left(\dots \rar{\cdot p} M \rar{\cdot p}M \right)\]
vanishes. We denote the full subcategory spanned by the $p$-complete modules by $(\rm{Mod}_{R})_{p}^{\wedge}$.
\end{definition}

\begin{remark}
The inclusion $(\rm{Mod}_{R})_{p}^{\wedge} \rari{} \rm{Mod_{R}}$ admits a left adjoint which takes a module $M$
to its \textit{$p$-completion} given by the limit
\[ \lim \left( \dots \to M/p^{2} \to M/p \right).\]
In fact, $M$ is $p$-complete if and only if the natural map $M \to \lim M/p^{n}$ is an equivalence.
This inherits a natural $R^{\wedge}_{p}$-module structure, thus $p$-completion also gives
an equivalence of categories $(\rm{Mod}_{R})^{\wedge}_{p} \simeq (\rm{Mod}_{R^{\wedge}_{p}})^{\wedge}_{p}$ which
allows us to identify these in what follows.\\
The tensor product of $p$-complete modules is in general not $p$-complete. However, the
category $(\rm{Mod}_{R})_{p}^{\wedge}$ admits a symmetric monoidal structure given by the formula
 \[ M \otimes_{(\rm{Mod}_{R})_{p}^{\wedge}} N := ( M \otimes N )^{\wedge}_{p}.\]
 With this monoidal structure the $p$-completion functor $\rm{Mod}_{R}\to (\rm{Mod}_{R})_{p}^{\wedge}$
 is strong monoidal, while the inclusion is only lax monoidal.
\end{remark}

 \begin{definition}
   Let $R$ be an $\bb{E}_{\infty}$-ring. We define the $\infty$-category of $p$-complete
   $R$-coalgebras is given by.
   \[ {(\rm{cCAlg}_{R})}^{\wedge}_{p}:= \rm{cCAlg}({(\rm{Mod}_{R})}^{\wedge}_{p}).\]
 \end{definition}

 \begin{warning}
   Let $R$ be a $\bb{E}_{\infty}$-ring. Notice that by our definition a $p$-complete $R$-coalgebra
   is the same as a $p$-complete $R^{\wedge}_{p}$-coalgebra and so we do not differentiate between
   the two notions.
   However, this is \textit{not} the same as an $R^{\wedge}_{p}$-coalgebra whose underlying
   spectrum is $p$-complete. The process of $p$-completion does refine to a functor
   $\rm{cCAlg}_{R} \to (\rm{cCAlg}_{R^{\wedge}_{p}})^{\wedge}_{p}$,
   but it does not factor through the category $\rm{cCAlg}_{R^{\wedge}_{p}}$.
 \end{warning}

 We now show check that the assignment $R \mapsto \rm{cCAlg}_{R}^{\rm{cn}}$ is subject to the machinery
 of deformation theory.

 \begin{lemma}\label{conil2}
   The following statements hold:
   \begin{enumerate}
     \item   Suppose we have a pullback diagram of connective $\bb{E}_{\infty}$-rings
   \[\begin{tikzcd}
	R\p & S\p \\
	R & S
	\arrow[from=1-1, to=2-1]
	\arrow[from=2-1, to=2-2]
	\arrow[from=1-2, to=2-2]
	\arrow[from=1-1, to=1-2]
\end{tikzcd}\]
such that the map $\pi_{0}R \to \pi_{0}S$ is surjective. Then the natural map
\[ (\rm{cCAlg}_{R\p}^{\rm{cn}})^{\wedge}_{p} \to (\rm{cCAlg}_{R}^{\rm{cn}})^{\wedge}_{p}\times_{(\rm{cCAlg}_{S}^{\rm{cn}})^{\wedge}_{p}} (\rm{cCAlg}_{S\p}^{\rm{cn}})^{\wedge}_{p}\]
is an equivalence.
     \item For every connective $\bb{E}_{\infty}$-ring $R$, the natural map
           \[ (\rm{cCAlg}_{R}^{\rm{cn}})^{\wedge}_{p} \to\flim_{n} (\rm{cCAlg}_{\tau_{\le n}R}^{\rm{cn}})^{\wedge}_{p}\]
           is an equivalence.
   \end{enumerate}
 \end{lemma}
 \begin{proof}
   Ad 1.: Arguing as in the proof of Proposition~\ref{Mod}, it suffices to show that the
   strong monoidal functor
   \begin{align*}
    (\rm{Mod}_{R\p})^{\wedge}_{p} \to (\rm{Mod}_{R})^{\wedge}_{p}\times_{(\rm{Mod}_{S})^{\wedge}_{p}} (\rm{Mod}_{S\p})^{\wedge}_{p}
   \end{align*}
   is an equivalence. Indeed, given a point $(M,N,h)$ in the pullback, the $R\p$-module $M \times_{M \otimes_{R} S}N$
   is again $p$-complete since $p$-completion commutes with limits. Thus, the inverse functor of
   Proposition~\ref{Mod} also induces a functor on the categories of $p$-complete modules. Moreover,
   we have that
   \[ ((M\times_{M\otimes_{R}S}N)\otimes_{R\p} R)^{\wedge}_{p} \simeq M^{\wedge}_{p} \simeq M\]
   \[ ((M \times_{M\otimes_{R}}N)\otimes_{R\p}S\p)^{\wedge}_{p}\simeq N^{\wedge}_{p} \simeq N,\]
   where the first equivalences hold by Proposition~\ref{Mod}, and the latter since $M$ and $N$ are
   to be $p$-complete. Finally, for $M\in (\rm{Mod}_{R\p})^{\wedge}_{p}$, we compute that
   \[ (M \otimes_{R\p} R)^{\wedge}_{p}\times_{(M \otimes_{R\p} S)^{\wedge}_{p}}(M \otimes_{R\p}S\p)^{\wedge}_{p}
     \simeq \left( M \otimes_{R\p} R \times_{M\otimes_{R\p} S} M \otimes_{R\p} S\p\right)^{\wedge}_{p}
   \simeq M^{\wedge}_{p} \simeq M,\]
 where we have again used the result of Proposition~\ref{Mod} and the fact that $p$-completion commutes
 with limits.\\
 Ad 2: This uses the exact same arguments applied to the equivalence of Corollary~\ref{nilcomplete}.
 \end{proof}

 \begin{corollary}
   For any $n\in \bb{N}$, the functor
   \[ \rm{CAlg}^{\rm{cn}} \to \cl{S} \qquad R \mapsto [(\rm{cCAlg}_{R}^{\rm{cn}})^{\wedge}_{p}]^{\Delta^{n}}\]
   is coherent and nilcomplete.
 \end{corollary}

 We now prove the crucial $p$-completeness result for $\Z_{p}$-modules. As before
 this will enable us to deduce the same result for coalgebras and allow us to tackle the
 actual problem of comparing coalgebras over $\F_{p}$, $\Z_{p}$ and $\S_{p}^{\wedge}$.
\begin{proposition}\label{pcomp}
  Let $\rm{Mod}^{\wedge}_{\Z_p} \subseteq \rm{Mod}_{\Z_{p}}$ denote the full subcategory spanned by the
  $p$-complete $\Z_{p}$-module spectra. Then the natural map
  \[ \rm{Mod}_{\Z_{p}} \to \flim_{n} \rm{Mod}_{\Z/p^{n}} \quad N \mapsto (N\otimes_{\Z_{p}}\Z/p^{n})\]
  restricts to a strong monoidal equivalence
  \[(\rm{Mod}_{\Z_{p}})^{\wedge}_{p} \simeq \flim_{n}\rm{Mod}_{\Z/p^{n}}. \]
\end{proposition}
\begin{proof}
  The functor admits a right adjoint which takes $(M_{n})\in \flim_{n}\rm{Mod}_{\Z/p^{n}}$ to the limit
  $\lim_{n}M_{n}$ taken in the category of $\Z_{p}$-modules. Since $p$-complete modules are closed under
  limits, the essential image of this functor is contained in $\rm{Mod}_{\Z_{p}}^{\wedge}$. Moreover,
  if $M\in \rm{Mod}_{\Z_{p}}^{\wedge}$, then we have that
  \[ \flim_{n}(M \otimes_{\Z_{p}} \Z/p^{n}) \simeq \flim_{n} M/p^{n} \simeq M^{\wedge}_{p}\simeq M.\]
  Hence, the counit of the adjunction is an equivalence on $p$-complete modules.
  Conversely, given $(N_{k})\in \flim_{k}\rm{Mod}_{\Z/p^{k}}$ write $N= \lim_{k}N$. We want
  to show that, for every $n$ the natural map
  \[ N \otimes_{\Z_{p}} \Z/p^{n}\rar{\sim}N_{n}\]
  is an equivalence. Since $N \otimes_{\Z_{p}}Z/p^{n}\simeq N/p^{n}$ and limits are exact, we have an equivalence
  \[N \otimes_{\Z_{p}}\Z/p^{n}\simeq \lim_{k >n}(N_{k}\otimes_{\Z_{p}}\Z/p^{n}).\]
  Thus, the unit of the adjunction may be written as
  \[ \lim_{k>n}(N_{k} \otimes_{\Z_{p}}\Z/p^{n}) \to \lim_{k>n}(N_{k}\otimes_{\Z/p^{k}}\Z/p^{n})\simeq N_{n}\]
  and so has fiber given by
  \[ F_{n}:=\lim_{k>n}\left(N_{k}\otimes_{\Z/p^{k}}\rm{fib}(\Z/p^{k}\otimes_{\Z_{p}}\Z/p^{n}\to \Z/p^{n}) \right).\]
  Now we compute the fiber of $\Z/p^{k}\otimes_{\Z_{p}}\Z/p^{n}\to \Z/p^{n}$ as the module
  \[ \rm{Tor}^{\Z_{p}}(\Z/p^{k}, \Z/p^{n})[1]\simeq \Z/p^{n}[1].\]
  The reduction map $\Z/p^{k}\to \Z/p^{k-1}$ is induced by the map of projective resolutions
\[\begin{tikzcd}
	{\Z_p} & {\Z_p} \\
	{\Z_p} & {\Z_p}
	\arrow["{\cdot p^k}", from=1-1, to=1-2]
	\arrow["\id", from=1-2, to=2-2]
	\arrow["{\cdot p}"', from=1-1, to=2-1]
	\arrow["{\cdot p^{k-1}}"', from=2-1, to=2-2],
\end{tikzcd}\]
hence, on Tor it induces the multiplication by $p$ map
\[ \Z/p^{n}=\rm{Tor}^{\Z_{p}}(\Z/p^{k}, \Z/p^{n})\rar{\cdot p} \rm{Tor}^{\Z_{p}}(\Z/p^{k-1}, \Z/p^{n}) =\Z/p^{n}.\]
Thus, if we have $k\p > k > n$ such that $k\p -k > n$, the transition map
\[ F_{k\p}=N_{k\p} \otimes \rm{Tor}^{\Z_{p}}(\Z/p^{k}, \Z/p^{n})\to N_{k} \otimes \rm{Tor}^{\Z_{p}}(\Z/p^{k-1}, \Z/p^{n})= F_{k}\]
vanishes since the Tor-groups are $p^{n}$-torsion. Choosing a cofinal subset $S\subseteq \bb{N}_{>n}$ such that
$\abs{k\p -k}> n$ for any distinct $k\p,k\in S$, we see that
\[ \lim_{k>n} F_{k}\simeq \lim_{k\in S} F_{k} \simeq 0 \]
vanishes. Thus, since limits are exact, the map $N \otimes_{\Z_{p}} \Z/p^{n}\rar{\sim}N_{n}$ is an equivalence.\\
To see that the functor $\rm{Mod}_{\Z_{p}}^{\wedge} \to \flim_n \rm{Mod}_{\Z/p^{n}}$ is strong monoidal,
we observe that since cofibers and limits are exact, we have for each $n$ equivalences
\begin{align*}
  (M \otimes_{\Z_{p}} N)^{\wedge}_{p} \otimes_{\Z_{p}}\Z/p^{n} &\simeq \lim_{k}(M/p^{k} \otimes_{\Z_{p}}N/p^{k})/p^{n}\\
                                              &\simeq \lim_{k}\left((M/p^{n} \otimes_{\Z_{p}} N/p^{n})\otimes_{Z_{p}}\Z/p^{k}\right) \\
  &\simeq ((N\otimes_{\Z_{p}}\Z/p^{n}) \otimes_{\Z_{p}} (M \otimes_{\Z_{p}}\Z/p^{n}))^{\wedge}_{p}.
\end{align*}
This proves the claim.
\end{proof}

\begin{corollary}\label{pcomp1}
  We have an equivalence of categories
  \[ (\rm{cCAlg}_{\Z_{p}})_{p}^{\wedge} \rar{\sim} \flim_{n} \rm{cCAlg}_{\Z/p^{n}} \quad A \mapsto (A\otimes_{\Z_{p}}\Z/p^{n})\]
  with inverse taking a system of coalgebras $(B_{n})$ to the limit $\lim_{n}B_{n}$ taken in the
  category of ($p$-complete) $\Z_{p}$-modules, equipped with the induced $p$-complete
  $\Z_{p}$-coalgebra structure.
\end{corollary}
\begin{proof}
This follows from Proposition~\ref{pcomp}, arguing as in the proof of Proposition~\ref{Mod}.
\end{proof}

\begin{corollary}\label{obliftzp}
  Let $X(\blank)= (\rm{cCAlg}_{\blank}^{\rm{cn}})^{\Delta^{0}}$ and $A\in X(\F_{p})$ such that $T_{X_{A}}\simeq 0$.
  Then the space of lifts of $A$ to a $p$-complete $\Z_{p}$-coalgebra is contractible
\end{corollary}
 \begin{proof}
 Combine Proposition~\ref{liftpgen} and Corollary~\ref{pcomp1}.
 \end{proof}

\begin{corollary}\label{mapliftzp}
  Let $\varphi: B\to A$ be a map of connective, formally \'etale $\F_{p}$-coalgebras. Then the space of
  lifts of $\varphi$ to a map of $p$-complete $\Z_{p}$-coalgebras $B\p \to A\p$ is contractible.
\end{corollary}
\begin{proof}
    Let $ \cl{X}(\blank)=\rm{cCAlg}_{\blank}^{\rm{cn}}$. By Proposition~\ref{etalchar} the natural map
    \[ T_{\cl{X}^{\Delta^{1}}_{\varphi}} \to T_{\cl{X}^{\Delta^{0}}_{B}}\]
    is an equivalence, but since $B$ is formally \'etale we have $T_{\cl{X}^{\Delta^{0}}_{B}} \simeq 0$.
    Hence, the claim follows by applying Proposition~\ref{liftpgen} to the functor $\cl{X}^{\Delta^{1}}$
    and using Corollary~\ref{pcomp1}.
\end{proof}

Having shown this, we can now construct a functor which is analogous to the classical
Witt-Vectors, which allow us to pass from \'etale $\F_{p}$-algebras to $\Z_{p}$-algebras.

\begin{theorem}
  Let $\cl{C}\subseteq (\rm{cCAlg}_{\Z_{p}}^{\rm{cn}})^{\wedge}_{p}$ denote the full subcategory spanned by those
  coalgebras $A$ for which $A\otimes_{\Z_{p}} \F_{p}$ is formally \'etale. Then the base change functor
  \[ \cl{C} \to \rm{cCAlg}_{\F_{p}}^{\rm{cn}, \rm{f\acute{e}t}}  \qquad A \mapsto A\otimes_{\Z_{p}}\F_{p}\]
  is fully faithful and essentially surjective. In particular, the quasi inverse defines a functor
  \[ W_{p}: \rm{cCAlg}_{\F_{p}}^{\rm{cn,f\acute{e}t}} \to (\rm{cCAlg}_{\Z_{p}}^{\rm{cn}})^{\wedge}_{p}\]
  which is fully faithful and satisfies $W_{p}(A)\otimes_{\Z_{p}}\F_{p} \simeq A$ for every connective, formally
  \'etale $\F_{p}$-coalgebra $A$.
\end{theorem}

\begin{proof}
  Combine Corollary~\ref{obliftzp} and Corollary~\ref{mapliftzp}.
\end{proof}

We now turn our attention to the leap from $\Z_{p}$ to $\S_{p}^{\wedge}$. The following proposition shows that,
for an arbitrary cohesive and nilcomplete functor, a $\Z_{p}$-valued point which has vanishing $\F_{p}$-tangent
complex admits a unique lift to a $\S_{p}^{\wedge}$-valued point. This is surprising, as we do not
actually require any information about the $\Z_{p}$-tangent complex, everything is determined by
what happens modulo $p$.

\begin{proposition}\label{spherelift}
  Let $X: \rm{CAlg}^{\rm{cn}} \to \cl{S}$ be a cohesive and nilcomplete functor and let $A \in X(\Z_{p})$
  such that $T_{X_{A\otimes_{\Z_{p}}\F_{p}}}\simeq 0$. Then $A$ admits an essentially unique lift to $X(\S_{p}^{\wedge})$.
\end{proposition}

\begin{proof}
  We inductively construct lifts against the Postnikov Tower
  \[ \dots \to \tau_{\leq2} \S_{p}^{\wedge}  \to \tau_{\leq 1} \S_{p}^{\wedge} \to \tau_{\leq 0} \S_{p}^{\wedge} \simeq \Z_{p}. \]
  Write $A=A_{0},~S_{n}= \tau_{\leq n}\S_{p}^{\wedge},~ M_{n} = \pi_{n}S_{n}$ and assume we have already constructed
  a unique lift $A_{n}$ to $X(S_{n})$. Consider the square zero extension
  \[ M_{n+1}[n+1] \to S_{n+1}\to S_{n}.\]
  Since $M_{n+1} = \pi_{n+1}S_{n+1}$ is concentrated in a single degree, the $S_{n}$-action factors
  through $S_{0}=\Z_{p}$. Moreover, since $\pi_{n+1}S_{n+1}$ is of finite $p$-torsion, the action
  further factors through $\Z/p^{k}$ for some $k\geq 0$. Thus, Proposition~\ref{bc} implies that
  we have an equivalence
  \[ T_{X_{A_{n}}}^{M_{n+1}[n+1]} \simeq \Sigma^{n}T_{X_{A_{n}}}^{M_{n+1}} \simeq T_{X_{A_{n} \otimes_{S_{n}} \Z/p^{k}}}^{M_{n+1}}.\]
  Arguing as in Proposition~\ref{cofib} with respect to the square zero extension
  \[ \F_{p} \to \Z/p^{k}\to \Z/p^{k-1},\]
  we see that we have a cofiber sequence
  \[  T^{M_{n+1}\otimes_{\Z/p^{k}}\F_{p}}_{X_{A_{n} \otimes_{S_{n}} \Z/p^{k-1}}}
    \to T_{X_{A_{n} \otimes_{S_{n}} \Z/p^{k}}}^{M_{n+1}}
    \to T^{M_{n+1}\otimes_{\Z/p^{k}}\Z/p^{{k-1}}}_{X_{A_{n} \otimes_{S_{n}} \Z/p^{k-1}}}.\]
  For the left hand term, Proposition~\ref{bc} gives the equivalence
  \[ T_{X_{A_{n}\otimes_{S_{n}}\Z/p^{k-1}}}^{M_{n+1}\otimes_{\Z/p^{k}}\F_{p}}
    \simeq T_{X_{A \otimes_{\Z_{p}}\F_{p}}}^{{M_{n+1}\otimes_{\Z/p^{k}}\F_{p}}}
    \simeq T_{X_{A\otimes_{\Z_{p}}\F_{p}}}\otimes_{\F_{p}}( M_{n+1}\otimes_{\Z/p^{k}}\F_{p} ) \simeq 0,\]
  where we have used that, since $M_{n+1}$ is finitely generated, the $\F_{p}$-module
  $M_{n+1}\otimes_{\Z/p^{k}}\F_{p}$ is perfect. For the right hand term we
  replace $M_{n+1}$ with $M_{n+1} \otimes_{\Z/p^{k}}\Z/p^{k-1}$ and repeat the argument,
  inductively yielding equivalences
  \[ T^{M_{n+1}}_{X_{A_{n}\otimes_{S_{n}}\Z/p^{k}}}
    \simeq T^{M_{n+1}\otimes_{\Z/p^{k}}\Z/p^{{k-1}}}_{X_{A_{n-1} \otimes_{S_{n-1}} \Z/p^{k-1}}}
  \simeq \cdots \simeq T^{M_{n+1}\otimes_{\Z/p^{k}} \F_{p}}_{X_{A \otimes_{\Z_{p}}\F_{p}}} \simeq 0.\]
In total, this shows that $T_{X_{A_{n}}}^{M_{n+1}[n+1]} \simeq 0$, and hence $A_{n}$ admits an essentially
unique lift to $X(S_{n+1})$. Thus, the fiber over $A$ of the map
\[ X(\S_{p}^{\wedge})\simeq \flim_{n}X(S_{n})\to X( \Z_{p})\]
is contractible and we are done.
  \end{proof}

  \begin{lemma}\label{pcomparison}
    Write $\cl{X}(\blank)=\rm{cCAlg}^{\rm{cn}}_{\blank}$ and $\cl{Y}(\blank)=
    (\rm{cCAlg}^{\rm{cn}}_{\blank})^{\wedge}_{p}$. Then the $p$-completion map $f:\cl{X}\to \cl{X}\p$
    induces an equivalence
    \[ T^{M}_{(\cl{X}^{\Delta^{n}})_{\xi}} \to  T^{M}_{(\cl{Y}^{\Delta^{n}})_{f(\xi)}}\]
        for every $\F_{p}$-module $M$, $n\in \bb{N}$ and $\xi \in \cl{X}(\F_{p})^{\Delta^{n}}$.
  \end{lemma}
  \begin{proof}
    For any $\F_{p}$-algebra $R$ the $p$-completion map gives an equivalence
    $\rm{Mod}_{R}\rar{\sim} (\rm{Mod}_{R})^{\wedge}_{p}$, since multiplication by some power of $p$
    is nullhomotopic over $\F_{p}$. In particular, this applies to the split square zero
    extension $\F_{p}\oplus M$ for any $M \in \rm{Mod}_{\F_{p}}$ and so the natural map
    $\cl{X}(\F_{p}\oplus M) \to \cl{Y}(\F_{p}\oplus M)$ is an equivalence as well.
    Consequently, we also obtain natural equivalences between the fibers
    \[ (\cl{X}^{\Delta^{n}})_{\xi}^{\F_{p}\oplus M} \to  (\cl{Y}^{\Delta^{n}})_{f(\xi_)}^{\F_{p}\oplus M},\]
    which induces the equivalence of spectra
    \[ T^{M}_{(\cl{X}^{\Delta^{n}})_{\xi}} \to  T^{M}_{(\cl{Y}^{\Delta^{n}})_{f(\xi)}}\]
      as claimed.
  \end{proof}

  \begin{corollary}\label{obliftsp}
    Let $X(\blank)=(\rm{cCAlg}^{\rm{cn}}_{\blank})^{\Delta^{0}}$ and $A \in X(\F_{p})$ such that
    $T_{X_{A}}\simeq 0$, then the space of lifts of $A$ to a $p$-complete $\S_{p}^{\wedge}$-coalgebra
    is contractible.
  \end{corollary}

  \begin{proof}
    Write $Y(\blank)= ((\rm{cCAlg}^{\rm{cn}}_{\blank})^{\wedge}_{p})^{\Delta^{0}}$. Then by Lemma~\ref{pcomparison}
    we have an equivalence $T_{X_{A}}\simeq T_{Y_{A}} \simeq 0$. Hence, we can apply Proposition~\ref{obliftzp} to
    obtain an essentially unique lift $A\p\in Y(Z_{p})$. Further applying Proposition~\ref{spherelift}
    to $A\p$ yields our claim.
  \end{proof}
  Thus, we can pointwise lift $\F_{p}$-coalgebras with vanishing tangent complex to $\S_{p}^{\wedge}$. If
  we moreover consider \textit{formally \'etale coalgebras}, we can make this lifting functorial
  in a coalgebraic analogue of the \textit{Spherical Witt Vectors} construction for
  $\bb{E}_{\infty}$-algebras over $\F_{p}$.

\begin{corollary}\label{mapliftsp}
  Let $\varphi:B\to A$ be a map of $\F_{p}$-coalgebras such that $A$ and $B$ are formally \'etale.
  Then the space of lifts of $\varphi$ to a map $\varphi\p: B\p \to A\p$ of $p$-complete
  $\S_{p}^{\wedge}$-coalgebras is contractible.
\end{corollary}

\begin{proof}
  Let $ \cl{X}(\blank)=\rm{cCAlg}_{\blank}^{\rm{cn}}$ and $\cl{Y}(\blank) =
  (\rm{cCAlg}_{\blank}^{\rm{cn}})^{\wedge}_{p}$. By Proposition~\ref{mapliftzp} the map $\varphi$ admits
  an essentially unique lift to a point $\psi \in \cl{Y}(\Z_{p})^{\Delta^{1}}$. Moreover, Lemma~\ref{pcomparison}
  yields an equivalence $T_{\cl{X}^{\Delta^{1}}_{\varphi}}\simeq T_{\cl{Y}^{\Delta^{1}}_{\varphi}}$. Since both $A$ and $B$ are
  formally \'etale Proposition~\ref{etalchar} gives equivalences
  \[ T_{\cl{X}^{\Delta^{1}}_{\varphi}} \rar{\sim} T_{\cl{X}^{\Delta^{0}}_{B}} \simeq 0\]
  Hence, we can apply Proposition~\ref{spherelift} to the functor $\cl{Y}^{\Delta^{1}}$ and the point
  $\psi \in \cl{Y}^{\Delta^{1}}$, proving the claim.
\end{proof}

\begin{theorem}\label{wittsp}
  Denote by $\cl{C}\subseteq (\rm{cCAlg}_{\S_{p}^{\wedge}}^{\rm{cn}})^{\wedge}_{p} $ the full subcategory spanned by those
  coalgebras $A$ such that $A\otimes_{\S_{p}^{\wedge}}\F_{p}$ is formally \'etale. Then the base change functor
  \[ \cl{C} \to \rm{cCAlg}_{\F_{p}}^{\rm{cn}, \rm{f\acute{e}t}} \qquad A \mapsto A \otimes_{\S_{p}^{\wedge}} \F_{p}\]
  is fully faithful and essentially surjective.
\end{theorem}
\begin{proof}
  Combine Corollary~\ref{obliftsp} and Corollary~\ref{mapliftsp}.
\end{proof}

\begin{remark}
  In the setting of Theorem~\ref{wittsp} the quasi-inverse to $\blank \otimes_{\S^{\wedge}_{p}}\F_{p}$ defines
  a fully faithful functor
  \[ W_{\S_{p}^{\wedge}}: \rm{cCAlg}_{\F_{p}}^{\rm{cn}, \rm{f\acute{e}t}}
    \to (\rm{cCAlg}_{\S_{p}^{\wedge}}^{\rm{cn}})^{\wedge}_{p}\]
  which satisfies $W_{\S_{p}^{\wedge}}(A)\otimes_{\S^{\wedge}_{p}}\F_{p} \simeq A$ for every connective, formally \'etale
  $\F_{p}$-coalgebra $A$. We call $W_{\S_{p}^{\wedge}}(A)$ the \textit{spherical Witt vectors} of $A$.
\end{remark}


\subsection{Homology coalgebras}

As observed in Example~\ref{homology}, for every space $X$ and every $\bb{E}_{\infty}$-ring $R$, the
$R$-homology $R[X]$ carries a natural $R$-coalgebra structure, which is a stronger invariant than its
underlying $R$-module. We now want to apply our results and see what can be said about the deformation
theoretic behavior of homology coalgebras. To do this, we first need to compute the cotangent complex of the
$\F_{p}$-cohomology.

\begin{definition}
  A space $X\in \cl{S}$ is called $p$-finite if the following conditions hold:
  \begin{enumerate}
    \item The space $X$ is truncated.
    \item The set $\pi_{0}X$ is finite.
    \item For each $n\geq 1$ and $x\in X$, we have that $\pi_{n}(X,x)$ is a finite $p$-group.
  \end{enumerate}
  We denote the full subcategory of $\cl{S}$ spanned by the $p$-finite spaces as $\cl{S}_{p}$ and call
 $\cl{S}^{\vee}_{p} =: \rm{Pro}(\cl{S}_{p})$ the category of $p$-\textit{profinite} spaces.
\end{definition}

\begin{remark}
We can regard $\cl{S}_{p}^{\vee}$ as the category of ``formal limits'' of $p$-finite spaces $\varprojlim X_{\alpha}$.
As such there is a functor $\cl{S}^{\vee}_{p}\to \cl{S}$ which takes a formal limit to the actual limit in $\cl{S}$.
This functor admits a left adjoint given by $Y \mapsto \flim_{Y_{\alpha} \to Y} Y_{\alpha}$, where the limit runs over all maps
from a $p$-finite space $Y_{\alpha}$ to $Y$.
\end{remark}

\begin{lemma}
  Let $X$ be a space and $\flim X_{\alpha}$ be its $p$-profinite completion. Then the natural map
  of cohomology rings
  \[ \fcolim \F_{p}^{X_{\alpha}} \to \F_{p}^{X} \]
  is an equivalence.
\end{lemma}
\begin{proof}
  This is immediate since the Eilenberg-MacLane spaces $K(\F_{p},n)$ are $p$-finite.
\end{proof}

\begin{proposition}[Mandell, Lurie]\label{coetal}
  Let $X$ be a space, then the $\F_{p}$-cohomology $\F_{p}^{X}$ is a formally \'etale $\F_{p}$-algebra.
\end{proposition}
\begin{proof}
  Since the functor $R \mapsto L_{R/\F_{p}}$ commutes with colimits, the claim follows from the fact that
  $L_{\F_{p}^{X}/\F_{p}}\simeq 0$ for every $p$-finite space $X$ which is proven
  in~\cite[][Proposition 2.4.12]{dag8}.
\end{proof}

Thus we obtain the following result about the homology coalgebra of a finite space $X$
with coefficients in a connective $\F_{p}$-algebra $R$:

\begin{corollary}\label{goal}
  Let $X$ be a finite space and $R$ be an $\F_{p}$-algebra, then $R[X]$ is a formally
  \'etale $R$-coalgebra.
\end{corollary}
\begin{proof}
  From Proposition~\ref{coetal} we get that
  \[ L_{R^{X}/R}\simeq L_{\F_{p}^{X}/\F_{p}}\otimes_{\F_{p}}R \simeq 0.\]
  Since $X$ is finite, the coalgebra $R[X]$ is dualizable with dual given by $R^{X}$, so the claim
  follows from Proposition~\ref{dualetal}.
\end{proof}

Moreover, for the case $R=\F_{p}$, we can use Theorem~\ref{wittsp} to give a partial answer to our
initial question about lifts of the coalgebra $\F_{p}[X]$.

\begin{corollary}
  Let $X$ be a finite space, then $\F_{p}[X]$ admits a unique lift to a $p$-complete $\S_{p}^{\wedge}$-coalgebra
  given by $W_{\S_{p}^{\wedge}}(\F_{p}[X]) \simeq (\S[X])^{\wedge}_{p}$. Moreover, for any other finite space $Y$
  the natural map
  \[\rm{Map}_{(\rm{cCAlg}_{\S_{p}^{\wedge}})^{\wedge}_{p}}((\S[Y])^{\wedge}_{p}, (\S[X])^{\wedge}_{p})
    \to \rm{Map}_{\rm{cCAlg}_{\F_{p}}}(\F_{p}[Y], \F_{p}[X])\]
  is a homotopy equivalence.
\end{corollary}
\begin{proof}
 Combine Corollary~\ref{goal} and Theorem~\ref{wittsp}.
\end{proof}

\section{Where to go from here}

We finish our discussion by explaining some of the shortcomings of our results and sketch a possible
way to proceed towards a coalgebraic analogue of Mandell's Theorem. The first missing puzzle piece is
the cotangent complex of a coalgebra $A$, which we have been unable to give a solid definition of.
The second and more important one is the relation to the \textit{coalgebra Frobenius}. We conjecture
that the class of \textit{perfect} coalgebras defined via this map give examples of non-dualizable
formally \'etale coalgebras. In particular, this conjecture would imply that the $\F_{p}$-homology
of \textit{any} space $X$ is formally \'etale.

\subsection{The cotangent complex of a coalgebra}
One of the first questions that arose during this project turned out to be one of the most subtle and
tricky ones, namely:

\begin{question}
  What is the cotangent complex of a coalgebra $A$?
\end{question}

Clearly, the existence of a single spectrum controlling the deformation theory of $A$ would be immensely
useful. However, it is not immediately clear what the universal property of such a spectrum should be,
i.e.~which space of derivations it should (co)represent.
Some inspiration can be gleamed from Proposition~\ref{cotangentder}. There we had seen that, for
$\varphi: B \to A$ a map of $R$-coalgebras with $A$ dualizable and $M$ an $R$-module, we have an equivalence
\[ \rm{Der}_{\varphi}(B, C_{A}(M)) \simeq \rm{Map}_{A^{\vee}}(L_{A^{\vee}/R}, \varphi^{\vee}_{\pt}\rm{map}_{R}(B, M)).\]
To get rid of the dependence on the second coalgebra $B$ one is tempted to take $B=R$ such that
$\rm{map}_{R}(B,M)\simeq M$. However, not every coalgebra $A$ admits a map $R\to A$, much less a canonical
one. The only natural choice for a map that is not the initial map would yield the following:

\begin{definition}[Preliminary 1.]
  Let $R$ be an $\bb{E}_{\infty}$-ring and $A\in \rm{cCAlg}_{R}$. The cotangent complex of $A$, if it exists,
  is the $R$-module $L_{A}$ corepresenting the functor
  \[ \rm{Mod}_{R}\to \rm{Mod}_{R} \qquad M \mapsto \rm{der}_{\id}(A, C_{A}(M))\]
\end{definition}

There are however several problems with this. Firstly, it is entirely unclear from the definition
whether $L_{A}$ vanishing would actually imply $A$ being formally \'etale. Moreover, in the dualizable
case it would lead to the rather awkward formula
\[ L_{A} \simeq L_{A^{\vee}/R}\otimes_{A^{\vee}}A.\]
Although somewhat plausible, this again gives us little information about what can actually be
deduced in the case that $L_{A}\simeq 0$.
This leaves us with several options, lest we accept that there is no good notion of one singular
cotangent complex. For one we could work with \textit{coaugmented} coalgebras, namely coalgebras
together with a map $R \to A$. For the purpose of understanding homology coalgebras this would correspond
to considering pointed spaces instead of just spaces, an entirely acceptable compromise, but beyond the
scope of this paper. \\
A different  approach would be to give up on the idea of corepresentability
and instead hope for a colimit preserving functor. For example, the functor
\[ \rm{Mod}_{R}\to \rm{Mod}_{R} \qquad M \mapsto C_{A}(M):=\rm{cofib}( A \rar{\eps} \Omega^{\infty}_{A}M).\]
seems to have no chance of preserving limits, but since colimits of coalgebras are formed underlying,
colimits are not out of the race. This leads us to the following idea:

\begin{definition}[Preliminary 2]\label{dream}
  Let $R$ be an $\bb{E}_{\infty}$-ring and $A\in \rm{cCAlg}_{R}$. We say that $A$ admits a cotangent
  complex $L_{A}:= C_{A}(R)$ if the functor $C_{A}(\blank):\rm{Mod}_{R} \to \rm{Mod}_{R}$ commutes
  with colimits. In this case we have $C_{A}(M)\simeq L_{A}\otimes M$ for every $ M \in \rm{Mod}_{R}$
\end{definition}

This definition is highly speculative, as the only coalgebras we know to admit a cotangent complex
in this sense are the formally \'etale coalgebras, for which the functor $C_{\blank}(A)$ is constant.
Conversely, if $A$ admits a cotangent complex then $L_{A}$ vanishes if and only if $A$ is formally
\'etale. Hence, the spectrum $L_{A}$ is precisely the obstruction to $A$ being formally \'etale,
which is the kind of conceptual clarity we are looking for.
While we lose any direct comparison to the cotangent complex of $A^{\vee}$ this is not entirely surprising,
since the property of being formally \'etale is defined very differently for $A^{\vee}$.
This leaves us with the following:

\begin{question}\label{cotangentdream}
  Let $R$ be an $\bb{E}_{\infty}$-ring. Does every $A \in \rm{cCAlg}_{R}$ admit a cotangent complex in the sense
  of Definition~\ref{dream}?
\end{question}

Regardless of the answer, the takeaway should be that the modules
$C_{A}(M)$ are exactly the obstruction towards $A$ being formally \'etale. Moreover, while the functor
$A\mapsto C_{A}(M)$ is very complicated, the dependence on $M$ should be relatively tame. That is,
for fixed $A$ it should be possible to describe the functor $M \mapsto C_{A}(M)$ in terms of a
formula involving $C_{A}(R)$. However, because $C_{A}(M)$ no longer has a direct relation to any
space of derivations or tangent complex, we cannot leverage results like Proposition~\ref{structure}
to obtain such a formula. We understand this as an indication that for these questions, the formalism may
have reached its limit.

\subsection{The Frobenius}
The most lacking thing about our results is the class of coalgebras that we can currently apply them to.
As of now, we are unable to give examples of formally \'etale coalgebras which are not dualizable. In
particular, we cannot describe the deformation theory of $R[X]$ for spaces $X$ which are not finite.
Attempts to reduce to the dualizable case all seem to fail for the following reason: Even though
we may write $X= \fcolim_{i}X_{i}$ where each $X_{i}$ is finite, giving the formula
$R[X]= \fcolim_{i}R[X_{i}]$, there is no reason why the functor
$\Omega^{\infty}_{\blank}(M): \rm{cCAlg}_{R}\to \rm{cCAlg}_{R}$ should commute with colimits.
Indeed, write $f_{M}:R\to R\oplus M$ for inclusion, then by definition
$\Omega^{\infty}_{\blank}(M) = f_{M,!} f^{\pt}_{M}$. The functor $f^{\pt}_{M}$ commutes with colimits,
and from Proposition~\ref{present} and the converse of the adjoint functor theorem we can deduce
that $f_{M,!}$ commutes with $\kappa$-filtered colimits for some regular cardinal $\kappa$. Thus, the class
of formally \'etale coalgebras is closed under $\kappa$-filtered colimits, but $\kappa$ is, in general, not countable.
% Closely related is the fact the notion of compactness is strangely behaved for coalgebras. For example,
% one can show that $\bb{Q}$ is not a compact object of $\rm{cCAlg}_{\Q}$, see~\cite[][Warning 1.2.15.]{ellII}.
% In particular, this means that
% \[ \rm{cSpec}(\fcolim_{i}\S[X_{i}])(\bb{Q})\neq \fcolim_{i}\rm{cSpec}(\S[X_{i}])(\Q),\]
% so we cannot deduce things about the cospectrum of infinite spaces in this way either. \\
This goes to show that the deformation theory of non-dualizable coalgebras is richer and more
interesting than that of the Ind-completion of dualizable coalgebras and requires additional input.
One contender for this additional input is the \textit{Coalgebra Frobenius} constructed by
Nikolaus:

\begin{theorem}[Nikolaus]
  Let $\cl{C} = (\rm{cCAlg}^{\rm{cn}}_{\S^{\wedge}_{p}})^{\wedge}_{p}$, then there exists a natural transformation
  $\psi_{p}:\id_{\cl{C}}\to \id_{\cl{C}}$ which on an object $A\in \cl{C}$ is given by the composition
  \[ \psi_{p}: A \rar{\Delta_{A}^{\otimes p}} (A^{\otimes p})^{hC_{p}} \rar{\rm{can}} (A^{\otimes p})^{tC_{p}} \rar{\sim} A,\]
  where the final map is the inverse of the \textit{Tate Diagonal}, see~\cite[][Theorem III.1.7]{tch}.
\end{theorem}

Given this map, we are naturally led to define \textit{perfect} coalgebras as follows:

\begin{definition}
  We say that $A \in  (\rm{cCAlg}^{\rm{cn}}_{\S^{\wedge}_{p}})^{\wedge}_{p}$ is \textit{perfect} if the coalgebra
  Frobenius $\psi_{p}: A\to A$ is a homotopy equivalence. We denote the full subcategory spanned by
  the perfect coalgebras by $(\rm{cCAlg}^{\rm{cn}}_{\S^{\wedge}_{p}})^{\wedge ,\rm{perf}}_{p} \subseteq
  (\rm{cCAlg}^{\rm{cn}}_{\S^{\wedge}_{p}})^{\wedge}_{p}$.
\end{definition}

\begin{example}\label{frobchains}
  Let $X$ be any space. Then $(\S[X])^{\wedge}_{p}$ is a perfect coalgebra since we have that
  \[\S[X]^{\wedge}_{p} \simeq (\S_{p}^{\wedge}[\colim_{X}\pt])^{\wedge}_{p} \simeq (\colim_{X} \S_{p}^{\wedge})^{\wedge}_{p}.\]
  On $\S_{p}^{\wedge}$ the map $\psi_{p}$ is necessarily given by the identity, because $\S_{p}^{\wedge}$
  is the terminal $p$-complete $\S_{p}^{\wedge}$-coalgebra. Thus, by naturality $\psi_{p}$ is given
  by the identity on $(\S[X])^{\wedge}_{p}$ as well.
\end{example}

We conjecture that this Frobenius map is related to the deformation theory of coalgebras in a similar
way to the Algebra Frobenius, in that it provides a sufficient condition for a coalgebra to be formally
\'etale.

\begin{conjecture}\label{frobcof}
  Let $A \in (\rm{cCAlg}^{\rm{cn}}_{\S^{\wedge}_{p}})^{\wedge}_{p}$ and write $A\p= A\otimes_{\S^{\wedge}_{p}}\F_{p}$.
  Then for any $M \in \rm{Mod}_{\F_{p}}^{\rm{cn}}$, the coalgebra Frobenius $\psi_p:A\to A$ induces the zero map
  on the $R$-module  $C_{A\p}(M) = \rm{cofib}(A\p \rar{\eta_{A\p}} \Omega^{\infty}_{A}(M))$.
\end{conjecture}

\begin{corollary}
  If Conjecture~\ref{frobcof} holds, then the base change functor
  \[ (\rm{cCAlg}^{\rm{cn}}_{\S^{\wedge}_{p}})^{\wedge ,\rm{perf}}_{p} \to \rm{cCAlg}_{\F_{p}}^{\rm{cn}}
  \qquad A \mapsto A\otimes_{\S_{p}^{\wedge}}\F_{p}\]
is fully faithful and factors through the full subcategory
$\rm{cCAlg}_{\F_{p}}^{\rm{cn}, \rm{f\acute{e}t}}\subseteq \rm{cCAlg}_{\F_{p}}^{\rm{cn}}$.
\end{corollary}
\begin{proof}
  Since $\psi_{p}:A\rar{\sim} A$ is an equivalence it induces an equivalence on $A\otimes_{\S_{p}^{\wedge}}\F_{p}$ and
  thus on $C_{A\otimes_{\S_{p}^{\wedge}}\F_{p}}(M)$ as well. However, since it also induces the zero map on the latter
  we get that $C_{A\otimes_{\S_{p}^{\wedge}}\F_{p}}(M)\simeq 0$. Thus, $A\otimes_{\S_{p}^{\wedge}}\F_{p}$ is formally \'etale and the
  claim follows from Theorem~\ref{wittsp}.
\end{proof}

Combining this with Example~\ref{frobchains} would allow us to fully answer our initial question about
homology coalgebras.

\begin{corollary}\label{dream2}
  If Conjecture~\ref{frobcof} holds, then for any space $X$ the $\F_{p}$-chains $\F_{p}[X]$
  are formally \'etale. In particular $\F_{p}[X]$ admits a unique and functorial lift to a $p$-complete
  $\S_{p}^{\wedge}$-coalgebra given by $\S[X]^{\wedge}_{p}= W_{\S_{p}^{\wedge}}(\F_{p}[X])$.
\end{corollary}

The fact that Conjecture~\ref{frobcof} needs to be checked for every connective $\F_{p}$-module should
be understood as an extension of our failure to find a cotangent complex. Indeed, if $\F_{p}[X]$ admits
a cotangent complex in the sense of Definition~\ref{dream}, then to obtain Corollary~\ref{dream2} it
would suffice to show that $\psi_{p}$ induces the zero map on $C_{A\otimes_{\S_{p}^{\wedge}}\F_{p}}(\F_{p})
= L_{A\otimes_{\S_{p}^{\wedge}}\F_{p}}$. However, even for this specific module the conjecture is difficult
to attack from our present position. The problem is the tricky right adjoint
$\rm{cCAlg}_{\F_{p}\oplus \F_{p}}\to \rm{cCAlg}_{\F_{p}}$ appearing in the definition of
$C_{A\otimes_{\S^{\wedge}_{p}}\F_{p}}(\F_{p})$. Because there is no known formula for this functor, attempts to verify
the conjecture have thus far been unsuccessful in all non-trivial cases. This warrants further investigation
of the coalgebra Frobenius and Conjecture~\ref{dream2}.

 \section{Benchmarks and Evaluation}
\label{sec:eval}

We evaluate \krakenSpace to answer the following set of questions:
\begin{itemize}
\item How much improvement does partial evaluation and our implemented compiler optimizations give \kraken? %(\S \ref{sec:eval2})
\item How much faster is our purely functional f-expr language, \krakenSpace, compared to other implementations of fexprs? %(\S \ref{sec:eval1} - \ref{sec:eval2})
\item How does \kraken's performance, with its fexprs, compare to macros? %(\S \ref{sec:eval1}, \S \ref{sec:eval3})
\item How do the different partial evaluation mechanisms/optimizations in \krakenSpace contribute towards reduction in overall runtime?
%\item What does \krakenSpace do internally when we create a data structure and evaluate it for some function? (\S \ref{sec:casestudy})
\end{itemize}

\textbf{Experimental Setup}: 
We ran these experiments in a reproducible Nix environment on a NixOS install \cite{10.1145/1411203.1411255} (Kernel 6.0.0) on a laptop with 8 cores / 16 threads and 64 GB of RAM.
Our code contains the scripts and Nix Flakes needed to reproduce the exact set of dependencies to run our tests.
%The code can be found at \url{https://github.com/limvot/kraken}.

The Kraken benchmarks were run using both the Wasmtime and WAVM WebAssembly engines for most benchmarks.
The Wasmtime WebAssembly engine is one of the most popular, developed by the Bytecode Alliance itself, and uses the CraneLift code generation backend.
The WAVM WebAssembly engine is interesting for its use of LLVM, and it often produces the fastest code on benchmarks but has a higher startup time.
We eliminated the Cfold Wasmtime benchmark due to problems running out of stack space (a known property of the Cfold benchmark).

\textbf{Benchmarks}: 
To showcase the capability of Kraken, we created benchmarks that are commonly implemented in functional languages and have been used as benchmarks in other papers \cite{reinking2021perceus, 10.1145/3547646}.
The benchmarks are
\begin{itemize}
\item Fib - Calculating the nth Fibonacci number
\item RB-Tree - Inserting n items into a red-black tree, then traversing the tree to sum its values
\item Deriv - Computing a symbolic derivative of a large expression
\item Cfold - Constant-folding a large expression
\item NQueens - Placing n number of queens on the board such that no two queens are diagonal, vertical, or horizontal from each other
\end{itemize}
All benchmarks besides Fibonacci use the fexpr version of match for pattern matching in \kraken, which is equivalent to the macro version in NewLisp. We also RB-Tree using NewLisp's~\cite{mueller2018newlisp} version of fexpr match. We modified the sizes of the problems presented to the benchmark to account for the longer running times of some of the less-optimized implementations.
The code for Kraken and NewLisp is very similar, and we should note that it is very unidiomatic NewLisp.
Our goal was not to compare Kraken and NewLisp as implementation languages for Red-Black Trees, but to stress test a single reasonably complex fexpr/macro, namely pattern matching.
% \textbf{Comparison with other languages}: We evaluated \krakenSpace against a language that contains f-exprs, as well as against itself with various optimizations disabled. The only other language we could find which contains a real f-expr mechanism is NewLisp~\cite{mueller2018newlisp} and so we ported \kraken's benchmark implementation to NewLisp.

%The six state-of-the-art languages are Java 17.0.1, Swift 5.4.2, Koka 2.3.2, C++, Haskell 8.10.7, and OCaml 4.12.
%The language choices were taken directly from Perceus reference-counting paper \cite{reinking2021perceus}.
%The Fibonacci benchmark additionally tests Python 3.9.11 and Chez Scheme 9.5.4.
%Koka, Ocaml and Haskell are good comparison points as statically-typed, compiled, functional programming languages, while Chez Scheme is a good comparison point as a mature and industrial strength dynamically-typed Scheme implementation known for its performance. 
%\subsection{Basic Level Comparison}
\subsection{The Effect of Partial Evaluation on Eval Calls}

\begin{table}[h]
\caption{Number of eval calls with no partial evaluation for Fexprs}
	\begin{tabular}{||c | c c c c c ||} 
		\hline
		&Evals & Eval w1 Calls & Eval w0 Calls & Comp Dyn & Comp Dyn\\ 
        & & & & w1 Calls & w0 Calls\\ [0.5ex] 
		\hline\hline
		Cfold 5 & 10897376 & 2784275 & 879066  & 1 & 0 \\ 
		\hline
		  Deriv 2  & 11708558 & 2990090 & 946500 & 1 & 0 \\ 
        \hline
		  NQueens 7 & 13530241 & 3429161 & 1108393 & 1 & 0 \\ 
    \hline
		  Fib 30 & 119107888 & 30450112 & 10770217 & 1 & 0 \\ 
    \hline
		  RB-Tree 10 & 5032297 & 1291489 & 398104 & 1 & 0 \\ 
		\hline
	\end{tabular}
    \label{npe:calls}
 \end{table}

As mentioned before, using fexprs without partial evaluation will prelude optimization and cause a massive amount of repeated work. Table \ref{npe:calls} and Table \ref{pe:calls} show the number of calls to the \krakenSpace runtime's eval function, the number of times the runtime's eval function executed a call to an applicative with wrap\_level=1, the number of times the runtime's eval function executed a call to an operative with wrap\_level=0, the number of compiled dynamic calls to applicatives with wrap\_level=1, and the number of compiled dynamic calls to operatives with wrap\_level=0.
These are shown for \krakenSpace test cases with partial evaluation turned off and turned on. 
\begin{table}[h]
\caption{Number of eval calls in Partially Evaluated Fexprs}
	\begin{tabular}{||c | c c c c c ||} 
		\hline
		&Evals & Eval w1 Calls & Eval w0 Calls & Comp Dyn & Comp Dyn\\ 
        & & & & w1 Calls & w0 Calls\\ [0.5ex] 
		\hline\hline
		Cfold 5 & 0 & 0 & 0  & 0 & 0 \\ 
		\hline
		  Deriv 2  & 0 & 0 & 0 & 2 & 0 \\ 
        \hline
		  NQueens 7 & 0 & 0 & 0 & 0 & 0 \\ 
    \hline
		  Fib 30 & 0 & 0 & 0 & 0 & 0 \\ 
    \hline
		  RB-Tree 10 & 0 & 0 & 0 & 10 & 0 \\ 
		\hline
	\end{tabular}
    \label{pe:calls}
 \end{table}

\begin{table}[h]
\caption{Number of calls to the runtime's eval function for RB-Tree. The table shows the non-partial evaluation numbers -> partial evaluation numbers.}
	\begin{tabular}{||c | c c c c c ||} 
		\hline
		&Evals & Eval w1 Calls & Eval w0 Calls & Comp Dyn & Comp Dyn\\ 
        & & & & w1 Calls & w0 Calls\\ [0.5ex] 
		\hline\hline
		  RB-Tree 7 & 2952848 -> 0 & 757932 -> 0 & 233513 -> 0 & 1 -> 7 & 0 -> 0\\ 
        \hline
		  RB-Tree 8 & 3532131 -> 0 & 906548 -> 0 & 279379 -> 0 & 1 -> 8 & 0 -> 0\\ 
        \hline
		  RB-Tree 9 & 4278001 -> 0 & 1097965 -> 0 & 3383831 -> 0 & 1 -> 9 & 0 -> 0\\ 
		\hline
	\end{tabular}
    \label{pe:rb}
    \vspace{-4mm}
 \end{table}

Without partial evaluation, no compilation can be done because it is impossible to tell if arguments to calls will be evaluated. In all benchmarks, partial evaluation removed all calls to the runtime's eval function, resulting in a completely compiled program. Looking at RB-Tree, there are over a million calls to combiners with wrap level 1 (normal functions), and 398,000 calls to combiners with wrap level 0 (operatives replacing macros). This massive blowup in the number of calls is due to the repeated and exponential re-execution of macro-like-combiners in the definition of other macro-like-combiners, as discussed in the Introduction.

The non-partially-evaluated benchmarks show 1 compiled dynamic call to an applicative (its the first call into eval) and 0 compiled dynamic calls to operatives, because there is no compilation at all. For the partially evaluated benchmarks, there are a few compiled dynamic calls to applicatives due to higher-order function use in the benchmarks, and there are no compiled dynamic calls to operatives, as all operative use has been eliminated.
We also varied the inputs for RB-Tree shown in Table \ref{pe:rb} to give a sense for how the number scale with respect to input size.

The incredible slowdown implied by these tables comes to full fruition in our RB-Tree test in Fig.~\ref{fig:kraken_nqueens_rbtree}.
We kept this run shorter because Kraken's non-partial-evaluating interpreter takes an incredibly long time even for 100 insertions (40 minutes).
The compounding layers of repeated macro-like operative calls in the non-partially-evaluated Kraken version cause a ~70,000x slowdown relative to the partial evaluated, optimized, and compiled version.
For the remaining benchmarks, we remove the naive interpreted \krakenSpace version, as in each case its performance is so bad as to blow out the graph and make it impossible to do any comparison.
In our optimized Kraken, our partial evaluation algorithm is able to fully collapse these levels of inefficiency, evaluate and inline the results, and give the backend more specialized code to optimize, emitting a compiled version that handily beats not only the NewLisp-fexpr implementation but even the NewLisp-macro implementation, as can be seen in Fig.~\ref{fig:kraken_vs_world_fib}.
We kept the benchmark sizes small in this test because the stack limits of NewLisp prevent sizes larger then ~880, while the Tail Call Elimination performed by the \krakenSpace compiler allows us to run much larger benchmarks, including the run of 4,800,000 inserts to the RB-Tree.
This result shows the dramatic effect of partial evaluation and compiler optimizations on runtime for \kraken. Our technique takes the performance of a fully fexpr based language from being completely infeasible to being faster than a macro-based dynamic scripting language currently in use.
% \begin{center}
% \begin{table}[ht]
% \caption{Number of call to the runtime's eval function for Fib. The table shows the non-partial evaluation numbers -> partial evaluation numbers}
% 	\begin{tabular}{||c | c c c c c ||} 
% 		\hline
% 		&Evals & Eval w1 Calls & Eval w0 Calls & Comp Dyn w1 Calls & Comp Dyn w0 Calls\\ [0.5ex] 
% 		\hline\hline
% 		Fib 10 & 8468 -> 0 & 2167 -> 0  & 777 -> 0 & 1 -> 0 & 0 -> 0 \\ 
% 		\hline
% 		  Fib 15  & 87916 -> 0 & 22478 -> 0 & 7961 -> 0 & 1 -> 0 & 0 -> 0 \\ 
%         \hline
% 		  Fib 20 & 969010 -> 0 & 247731 -> 0 & 87633 -> 0 & 1 -> 0 & 0 -> 0 \\ 
%     \hline
% 		  Fib 25 & 10740492 -> 0 & 2745825 -> 0  & 971209 -> 0 & 1 -> 0 & 0 -> 0 \\ 
% 		\hline
% 	\end{tabular}
%     \label{pe:fib}
%  \end{table}
% \end{center}

\begin{figure}[h]
\caption{Constant Fold and Deriv}
\includegraphics[width=0.45\textwidth]{cfold_table.csv_}
\includegraphics[width=0.45\textwidth]{deriv_table.csv_}
\label{fig:kraken_const_deriv}
\vspace{-6mm}
\end{figure}
\subsection{Comparison between Kraken Versions}
Beyond the massive speedup from partial-evaluation, Fig. \ref{fig:kraken_const_deriv} and \ref{fig:kraken_nqueens_rbtree} show the effect of the various compiler optimizations we described by disabling them one by one.
 Our main four optimizations have a strong positive effect on runtime, with the exception of lazy environment instantiation. Lazy environment instantiation helps massively on fib, and some on Deriv, but generally hurts the rest slightly.


\begin{figure}[h]
\caption{N-Queens}
\includegraphics[width=0.45\textwidth]{nqueens_table.csv_}
\includegraphics[width=0.45\textwidth]{slow_rbtree_table.csv_}
\label{fig:kraken_nqueens_rbtree}
\vspace{-4mm}
\end{figure}


\subsection{Comparison against Others}


To give a general idea of our current performance, we also show a Fibonacci benchmark that mostly exercises pure function-call speed and inlining as seen in Fig. ~\ref{fig:kraken_vs_world_fib}.
We include Python and Chez Scheme to give a general idea for where an exemplar slow and an exemplar fast dynamic language would fall.
With the benefit of our partial evaluation, compilation, and leaning upon mature WebAssembly implementations, we beat both, but this should be taken with a grain of salt, as this is a very limited micro-benchmark only meant to give a general sense of the order of magnitude of our performance.



\label{sec:eval1}
\begin{figure}[h]
\caption{Kraken vs. Others. Ordered by fastest to slowest}
\includegraphics[width=0.45\textwidth]{fib_table.csv_}
\includegraphics[width=0.45\textwidth]{rbtree_table.csv_}
\label{fig:kraken_vs_world_fib}
\end{figure}

%\label{sec:eval_nqueens}
%\begin{figure}[h]
%\caption{N-Queens}
%\includegraphics[width=0.45\textwidth]{nqueens_table.csv_}
%\includegraphics[width=0.45\textwidth]{slow_nqueens_table.csv_}
%\label{fig:kraken_nqueens}
%\end{figure}

%\label{sec:eval_nqueens}
%\begin{figure}[h]
%\caption{Kraken, N-Queens, absolute value and log-scale}
%\includegraphics[width=0.45\textwidth]{nqueens_table.csv_}
%\includegraphics[width=0.45\textwidth]{nqueens_table.csv_log}
%\label{fig:kraken_nqueens}
%\end{figure}
%\label{sec:eval_nqueensp}
%\begin{figure}[h]
%\caption{Kraken, N-Queens, absolute value and log-scale}
%\includegraphics[width=0.45\textwidth]{slow_nqueens_table.csv_}
%\includegraphics[width=0.45\textwidth]{slow_nqueens_table.csv_log}
%\label{fig:kraken_nqueensp}
%\end{figure}

%\label{sec:eval_cfold}
%\begin{figure}[h]
%\caption{C-Fold}
%\includegraphics[width=0.45\textwidth]{cfold_table.csv_}
%\includegraphics[width=0.45\textwidth]{slow_cfold_table.csv_}
%\label{fig:kraken_cfold}
%\end{figure}
%\label{sec:eval_cfold}
%\begin{figure}[h]
%\caption{Kraken, C-Fold, absolute value and log-scale}
%\includegraphics[width=0.45\textwidth]{cfold_table.csv_}
%\includegraphics[width=0.45\textwidth]{cfold_table.csv_log}
%\label{fig:kraken_cfold}
%\end{figure}
%\label{sec:eval_cfoldp}
%\begin{figure}[h]
%\caption{Kraken, C-Fold, absolute value and log-scale}
%\includegraphics[width=0.45\textwidth]{slow_cfold_table.csv_}
%\includegraphics[width=0.45\textwidth]{slow_cfold_table.csv_log}
%\label{fig:kraken_cfoldp}
%\end{figure}

%\label{sec:eval_deriv}
%\begin{figure}[h]
%\caption{Deriv}
%\includegraphics[width=0.45\textwidth]{deriv_table.csv_}
%\includegraphics[width=0.45\textwidth]{slow_deriv_table.csv_}
%\label{fig:kraken_deriv}
%\end{figure}
%\label{sec:eval_deriv}
%\begin{figure}[h]
%\caption{Kraken, Deriv, absolute value and log-scale}
%\includegraphics[width=0.45\textwidth]{deriv_table.csv_}
%\includegraphics[width=0.45\textwidth]{deriv_table.csv_log}
%\label{fig:kraken_deriv}
%\end{figure}
%\label{sec:eval_derivp}
%\begin{figure}[h]
%\caption{Kraken, Deriv, absolute value and log-scale}
%\includegraphics[width=0.45\textwidth]{slow_deriv_table.csv_}
%\includegraphics[width=0.45\textwidth]{slow_deriv_table.csv_log}
%\label{fig:kraken_derivp}
%\end{figure}

%\subsection{Comparison against state-of-the-art languages}
%\label{sec:eval3}

%\begin{figure}[h]
%\caption{Kraken vs. S.o.t.A.}
%\includegraphics[width=0.45\textwidth]{cfold_table.csv_}
%\includegraphics[width=0.45\textwidth]{rbtree_table.csv_}
%\label{fig:kraken_vs_world1}
%\end{figure}

%\begin{figure}[h]
%\caption{Kraken vs. S.o.t.A.}
%\includegraphics[width=0.45\textwidth]{deriv_table.csv_}
%\includegraphics[width=0.45\textwidth]{nqueens_table.csv_}
%\label{fig:kraken_vs_world2}
%\end{figure}

% \begin{figure}[h]
% \caption{Kraken vs. S.o.t.A. (Log)}
% \includegraphics[width=0.45\textwidth]{cfold_table.csv_log}
% \includegraphics[width=0.45\textwidth]{rbtree_table.csv_log}
% \label{fig:kraken_vs_world_log_1}
% \end{figure}
% \begin{figure}[h]
% \caption{Kraken vs. S.o.t.A. (Log)}
% \includegraphics[width=0.45\textwidth]{deriv_table.csv_log}
% \includegraphics[width=0.45\textwidth]{nqueens_table.csv_log}
% \label{fig:kraken_vs_world_log_2}
% \end{figure}

%As we noted before with the Fib(30) microbenchmark in Section \ref{sec:eval1}, we remain significantly slower than state-of-the-art compiled languages.
%This is particularly true for memory-intensive benchmarks due to our naive reference-counting and malloc/free implementations.
%However, our results are of a similar order of magnitude to the difference between the state-of-the-art compiled languages and dynamic scripting languages, like Python's results in the Fib(30) microbenchmark.
%We assert that is not a fundamental limitation because the classic f-expr slowness is being eliminated, as shown by Fig. \ref{fig:kraken_vs_newlisp1} and Fig. \ref{fig:kraken_vs_newlisp2}.
%In future work, we plan to expand our compile-time analysis and optimization to implement a modified, dynamic-language version of Perceus reference counting.
%With this change, we belive \krakenSpace can be competitive with these state-of-the-art languages.

%\subsection{Case Study: Red-Black Tree}
%\label{sec:casestudy}

%\begin{figure}[h]
%\caption{Kraken vs. S.o.t.A. - RB-Tree Focus}
%\includegraphics[width=0.4\textwidth]{rbtree_table.csv_}
%\includegraphics[width=0.4\textwidth]{rbtree_table.csv_log}
%\label{fig:kraken_vs_world_rbtree}
%\end{figure}


%To evaluate our partial evaluation algorithm and compiler, we extracted the benchmarks used by the Koka language project from their code repository and added Kraken versions, as well as implementing a naive Fibonacci microbenchmark ourselves to evaluate pure function call speed.\\
%With partial evaluation and the compiler optimizations listed above, we get fairly strong performance on purely numerical computations, such as the naive Fibonacci microbenchmark.
%Unfortunately, the overhead of our unsophisticated reference counting, dynamic type checking, and bounds checking causes poor performance on benchmarks involving data structures relative to mainstream programming language implementations.
%This is not a fundamental limitation, and will be addressed in future work, as recounted in the next section.
%It should be noted, however, that while the performance relative to established language implementations is very poor for the memory-intensive benchmarks (600-900x slower), we still realize a massive speedup compared to an unoptimized and non-partial-evaluated f-expr implementation (100,000x faster)!

\section{Conclusions}
We consider the phase-extraction problem, and we showed that, given a unitary $U = e^{i\pi H}$ and its inverse $U^{\dag}$, we could implement a block-encoding of $\phi(H)$ for some smooth function $\phi(x)$. The word `smooth' here means existence and continuity of the derivatives: the higher the number of continuous derivatives that a function has, the faster its Fourier sum (and thus the Laurent polynomial on the eigenphases) uniformly converges to that function. We are confident this can have many more applications beyond what is shown in this work. It is also worth remarking that Jackson showed that the convergence rate of a Fourier series is almost-optimal, in the sense that no trigonometric (or, equivalently, complex exponential) series can approximate the desired function faster, up to that $\log d$ factor~\cite[p.\ 21]{jacksonTheoryApproximation1930a}. Also remember that `smoothing' a function, i.e., replacing its derivative with a continuous function, does not give faster convergence for free in general, as its derivative will become steep in the points where we smooth out discontinuities, and this translates to a high Lipschitz constant: a~clear example is given by Eq.~\ref{eq:lipschitz-constant-recurrence-solution}, but in that case, fortunately, nothing depends on the size of the input $N$, and thus does not influence the asymptotic query complexity of Algorithm~\ref{alg:prop-sampling-qsp}, although the constant factor can become large even for $p = 20$. From a theoretical point of view, this work shows that, for any $\eta > 0$, there is an algorithm with query complexity 
$$\Tilde{\bigO}\left(\frac{1}{\bar{c}^{\frac{1}{2} + \eta}} \frac{1}{\epsilon^\eta} \right)$$
solving the proportional-sampling problem. This statement seems to suggest there exists an algorithm which directly solves the problem with $\eta = 0$, and an open question would be to find such algorithm.


It is also interesting to remark that Theorems~\ref{thm:haah-construction},~\ref{thm:haah-completion} indeed allow the construction for any $\phi$, even complex-valued, provided that its absolute value is reciprocal.

One could think that, in Section~\ref{sec:prop-sampling}, instead of using the linear function in the phase-extraction subroutine, we could approximate the square root and then apply the transformation directly on $e^{i \pi c(x)}$. However, in the case of proportional sampling this would be inconvenient, as the derivative of the square root function has a discontinuity with an infinite jump around 0, and we could not choose a constant $\delta$ if we had values of the oracle that are too close to $0$.

% Bibliography
\bibliographystyle{ACM-Reference-Format}
\bibliography{bibliography}

\clearpage
% Appendix
\appendix


\bgroup
\setlength{\tabcolsep}{1.3mm}
\begin{tabular}{lrrrrrcl}
    \toprule
    \cthead{Dataset}                                           & \cthead{\( n \)} & \cthead{\( v \)} & \cthead{\( k \)} & \cthead{\( n_\text{small} \)} & \cthead{\( n_\text{big} \)} & \cthead{Dim.\ }                    & \cthead{Licence} \\ \cmidrule(lr){1-1} \cmidrule(lr){2-8}
    NoisyMNIST~\cite{lecunGradientbasedLearningApplied1998}   & \( 70000 \)     & \( 2 \)         & \( 10 \)        & \( 6313 \)                   & \( 7877 \)                 & \( (28 \times 28)^{2} \)          & CC BY-SA 3.0 \\
    NoisyFashion~\cite{xiaoFashionMNISTNovelImage2017}        & \( 70000 \)     & \( 2 \)         & \( 10 \)        & \( 7000 \)                   & \( 7000 \)                 & \( (28 \times 28)^{2} \)          & MIT \\
    EdgeMNIST~\cite{lecunGradientbasedLearningApplied1998}    & \( 70000 \)     & \( 2 \)         & \( 10 \)        & \( 6313 \)                   & \( 7877 \)                 & \( (28 \times 28)^{2} \)          & CC BY-SA 3.0 \\
    EdgeFashion~\cite{xiaoFashionMNISTNovelImage2017}         & \( 70000 \)     & \( 2 \)         & \( 10 \)        & \( 7000 \)                   & \( 7000 \)                 & \( (28 \times 28)^{2} \)          & MIT \\
    COIL-20~\cite{neneColumbiaObjectImage1996}               & \( 480 \)       & \( 3 \)         & \( 20 \)        & \( 24 \)                     & \( 24 \)                   & \( (64 \times 64)^{3} \)          & None \\
    Caltech7~\cite{fei-feiLearningGenerativeVisual2007}       & \( 1474 \)      & \( 6 \)         & \( 7 \)         & \( 34 \)                     & \( 798 \)                  & \( 48, 40, 254, 1984, 512, 928 \) & CC BY 4.0 \\
    Caltech20~\cite{fei-feiLearningGenerativeVisual2007}      & \( 2386 \)      & \( 6 \)         & \( 20 \)        & \( 33 \)                     & \( 798 \)                  & \( 48, 40, 254, 1984, 512, 928 \) & CC BY 4.0 \\
    PatchedMNIST~\cite{lecunGradientbasedLearningApplied1998} & \( 21770 \)     & \( 12 \)        & \( 3 \)         & \( 6903 \)                   & \( 7877 \)                 & \( (28 \times 28)^{12} \)         & CC BY-SA 3.0 \\
    \bottomrule
\end{tabular}

\egroup


\section{Experimental setup}
%\subsection{Implementation details}
\subsection{Baseline details}
\label{sup:baseline}
MIL-NCE~\citep{miech2020end}, which utilizes S3D~\citep{xie2018rethinking} and word2vec~\citep{mikolov2013efficient} to project two modalities into a common space, is chosen as the standard baseline for this task; 
CoMMA~\citep{tan2021look}, the best-performing model for spatial representations in self-supervised learning (we denote CoMMA$\dagger$ to represent the model that uses weights shared by the author\footnote{We thank the authors for providing code and weights.}); CLIP~\citep{radford2021learning}, an image-text model trained with transformer architecture, is further applied as the backbone and trained with~\citep{tan2021look} to construct CoMMA$\ddagger$; GLIP~\citep{li2022grounded} and RegionCLIP~\citep{zhong2022regionclip}, state-of-the-art image-text grounding models that combine large-scale image caption pretraining and object detection fine-tuning, which we consider weakly supervised as the bounding box proposal network was trained on other human-annotated data.
We further construct a strong baseline out of the best methods for temporal and spatial localization, MIL-NCE+RegionCLIP, where we use MIL-NCE for temporal localization and RegionCLIP for spatial grounding following the inference pipeline of Figure \ref{fig:inference} without additional training. 

\subsection{Backbones and Training}
\label{backbone_and_training}
We evaluate the proposed method on backbones, CLIP \citep{radford2021learning} and S3D-word2vec \citep{miech2020end}.
We described the detailed setup as well as the training in the following.

\noindent \textbf{CLIP models.} For both the visual and text backbone, we use the pretrained weights from CLIP \citep{radford2021learning} with transformer ViT-B/32 and fix the encoder. Both the visual and text encoder has a final embedding size of 512. We apply them to video segments with 12-28 seconds, processing 1 frame per second. An evaluation of how many frames to process (identical to the number of seconds) is shown in Table \ref{tab:frames}. We sampled the video with 5 fps. It shows the best results when we start with 80 possible frames $U$ (as described in Section \ref{frame_sampling}), from which $T$ = 16 frames are selected for training. Ablation of the number of frames $T$ used for training is shown in Table \ref{subtab:ablations2}.
We used a batch size of $B$ = 64 video clips.

\noindent \textbf{S3D-word2vec models.}
For the video backbone, we follow~\citep{tan2021look} and use S3D initialized by MIL-NCE on HowTo100M~\citep{miech2020end} at the rate of 5 frames per second and fix the video encoder. 
The global video clip features were max-pooled over time and projected into embeddings of dimension 512. We used the mean-pooled S3D spatio-temporal features to represent the global representation of the video following the S3D architecture \citep{xie2018rethinking}.
For the text feature, we follow ~\citep{miech2019howto100m} using a GoogleNews pre-trained word2vec model~\citep{mikolov2013efficient} and max-pooling over words in a given sentence to acquire the text global feature.  
We follow \citep{miech2020end} to use the max-pooled word embedding to represent the sentence (global representation) since there is no [CLS] token. Also, the sentence feature is used for the query word selection instead of the [CLS] token. 
We use a batch size of $B$ = 96 video clips.


\noindent \textbf{Training.} For the training of both backbone settings, we use an Adam optimizer~\citep{kingma2015adam} with a learning rate of $1\mathrm{e}{-4}$. 
In the setting of fintining CLIP, we set a learning rate of $1\mathrm{e}{-7}$ for the CLIP backbone. 
The model is trained for 10 epochs on 4 V100 GPUs, which takes about two days. 


\subsection{Inference}
\label{inference_sup}
\noindent \textbf{Inference for the proposed model and CoMMA.} For inference in the case of temporal grounding, as shown in Figure \ref{fig:inference}(a), we first normalize the global feature for video and text. We used a (temporal) threshold $\theta$ = 0.5 to separate detections from the background. In spatial grounding, we acquire an attention heatmap using the attention rollout \citep{abnar2020quantifying} described in Section \ref{inference_section}. We set a spatial threshold $\tau$ = 0.01 to create the mask, as shown in Figure \ref{fig:inference}(b). The choice of this spatial threshold is evaluated in Table \ref{tab:thre}. 



\noindent \textbf{GLIP, RegionCLIP baseline inference.} In spatial grounding, we are given a text query and need to localize it in the frame. GLIP and RegionCLIP predict multiple bounding boxes corresponding to the text query. We select the predicted bounding box with the highest confidence score as the prediction result. We use the center point of the predicted bounding box for the pointing game evaluation as the model prediction. For \textit{mAP} evaluation, we use the predicted bounding box to compute IoU with the ground truth bounding box. In spatio-temporal grounding, we input all possible action description labels as candidates similar to Figure \ref{fig:inference}(a). We pick the class with the highest confidence score as the predicted label. If the model made no prediction, we would predict it as ``background''. The spatial inference is the same as the spatial grounding setting.

\noindent \textbf{TubeDETR, STCAT baseline inference.} TubeDETR and STCAT are spatio-temporal grounding models trained to predict a single spatio-temporal tube per video. 
In both cases, TubeDETR and STCAT, we use models trained on the Vid-STG dataset with 448x448 resolution and evaluate them for the task of spatial grounding. 
Since this dataset contains mostly short videos ($<$30sec), we observed that both methods will also only predict a trajectory tube in this temporal range ($<$30sec), no matter how long the input video is. To allow us to apply them to longer videos ($>$30sec), we split the longer videos based on sliding windows of 5-sec for better performance.




\noindent \textbf{MIL-NCE, CLIP baseline inference.} Both models are trained based on global representations for both input modalities, videos/images and text. We can, therefore, directly compute a sentence-to-video-frame similarity to perform the temporal grounding for Figure \ref{fig:inference}(a), following the same process as the proposed method for temporal grounding. For spatial grounding, we compute sentence-to-region feature similarity. Both visual backbones produce a 7x7 grid feature. We normalize the sentence and region features, then select a spatial threshold $\tau$ = 0.5 to create the mask for the \textit{mAP} evaluation.

\subsection{Evaluation metrics}
\label{eval_metric}


\noindent (i) \textbf{Spatio-temporal grounding in untrimmed video} is evaluated on our annotated GroundingYoutube dataset. We combined the spatial and temporal grounding evaluation as before \citep{kuehne2019mining,akbari2019multi} to form the spatio-temporal evaluation. The entire video and the respective pool of action instructions were provided. The model needs to localize each action step in temporal (start-time/end-time) and spatial (location in the video) as described in Figure \ref{fig:inference}. 
% \noindent\textbf{Inferencing.} The model will need to predict the action label per frame by feature similarity between the video and action classes similar to \citep{Zhukov2019CrossTask}. Later, the model will use its predicted action label as the query to perform spatial grounding to localize the action in the video frame.
We evaluate in two metrics: \textbf{IoU+Pointing game} combines the evaluation setting from the spatial grounding \citep{akbari2019multi} and temporal grounding \citep{kuehne2019mining} metrics. For each video frame, the prediction is correct when the model predicts the correct action for the frame. Also, given the predicted action as a query, the maximum point of the heatmap aims to lie within the desired bounding box. We then compute the Intersection over Union (IoU) over all the predictions with the GT to acquire the final score. 
We also compute \textbf{video mAP} following previous evaluation \citep{gu2018ava}, where we set IoU threshold between GT and predicted spatio-temporal tubes. A prediction is correct when it surpasses the IoU threshold. We then compute the mAP over all classes. We form a 3D prediction mask following Figure \ref{fig:inference} and compute IoU between our 3D heatmap and 3D tube.

\noindent (ii) \textbf{Spatial grounding} is given a text query description to localize the corresponding region in the trimmed video. We use GroundingYoutube, Youcook-Interaction, V-HICO, and Daly for evaluation. %Note that the evaluation is spatial only. It evaluates the results for each frame separately without considering the temporal information. 
This task is evaluated using the \textbf{pointing game accuracy}. Given the query text and video, we compute the attention heatmap on the video as described in Figure \ref{fig:inference}(b). If the highest attention similarity score lies in the ground truth bounding box, the result counts as a ``hit" and counts as ``miss" otherwise. The final accuracy is calculated as a ratio between hits to the total number of predictions $\frac{\text{\# hits}}{\text{\# hits} + \text{\# misses}}$. 
We report the mean average precision \textbf{(mAP)} following the settings from V-HICO \citep{li2021weakly}. Given a human-object category as the text query, we aim to localize the spatial location in the video frame.
The predicted location is correct if their Intersection over-Union (IoU) with ground truth bounding boxes is larger than 0.3. 
Since we do not use any bounding box proposal tools or supervision, we create an attention heatmap as described in Figure \ref{fig:inference}(b) to create a mask for IoU computation. 
We follow \citep{li2021weakly} and compute the mAP over all verb-object classes.


\noindent (iii) \textbf{Temporal grounding} \label{temporal_grounding}
provides videos with the respective actions and their ordering, including the background. The goal is to find the correct frame-wise segmentation of the video. We follow the inference procedure in \citep{kuehne2019mining} to compute the alignment given our similarity input matrix. The task is evaluated by intersection over detection (IoD), defined as $\frac{G \cap D}{D}$ the ratio between the intersection of ground-truth action $G$ and prediction $D$ to prediction $D$, and the Jaccard index, which is an (IoU) given as $\frac{G \cap D}{G \cup D}$.


\section{Deformation Modeling}
\label{ap:deformation_modeling}
In this section, we present additional details on the deformation model $\netdeformation$ used to render articulated objects such as humans. Given an articulated object, we assume its kinematic tree is known and that the transformation $[\tensrotation_j|\vectranslation_j]$ from each joint $j \in 1,...,\numjoints$ to the parent joint is part of the object's properties. From these we can follow the kinematic tree to derive transformations $[\tensrotation'_j|\vectranslation'_j]$ for each joint from the bounding box coordinate system to the canonical coordinate system. Intuitively, these transformations represent how to map a point $\vecpointbbox$ in the bounding box coordinate system belonging to the joint $j$ to the corresponding point $\vecpointcanon$ in the canonical space.

We implement a deformation procedure based on linear blend skinning (LBS) \cite{lewis2000pose} that establishes correspondences between points in the canonical space $\vecpointcanon$ and in the deformed bounding box space $\vecpointbbox$ by introducing blending weights $\vecblendweights$ for each point in the canonical space. These weights can be interpreted as the degree to which that point moves according to the transformation associated with that joint.
\begin{equation}
\label{eq:lbs}
\vecpointbbox = \sum_{j=1}^{\numjoints} w_j(\vecpointcanon)\left(\tensrotation_j^{\prime-1}  \vecpointcanon - \tensrotation_j^{\prime-1} \vectranslation'_j\right).
\end{equation}

During volumetric rendering, however, we sample points $\vecpointbbox$ in the bounding box space and query the canonical volume in the corresponding canonical space point $\vecpointcanon$. Doing so requires solving Eq.~\eqref{eq:lbs} for $\vecpointcanon$, which is prohibitively expensive \cite{li2022tava}. Inspired by HumanNeRF \cite{weng2022humannerf}, instead of modeling LBS weights $\vecblendweights$, we introduce inverse linear blending weights $\vecblendweights^b$:
\begin{equation}
\label{eq:invlbs}
    \vecblendweights_j^b(\vecpointbbox) = \frac{\vecblendweights_j(\tensrotation'_j \vecpointbbox + \vectranslation'_j)}{\sum_{j=1}^{\numjoints} \vecblendweights_j(\tensrotation'_p \vecpointbbox + \vectranslation'_j)}.
\end{equation}
such that the canonical point can be approximated as:
\begin{equation}
\label{eq:lbs_hnerf}
\vecpointcanon = \sum_{j=1}^{\numjoints} \vecblendweights_j^b(\vecpointbbox)\left(\tensrotation'_j \vecpointbbox + \vectranslation'_j\right).
\end{equation}


We parametrize the function $\vecblendweights$ mapping spatial locations in the canonical space to blending weights as a neural network. Similarly to $\netcanonical$, we employ 3D convolutions to map a fixed noise volume $\tensblendweightssmall \in \mathbb{R}^{\numfieldfeatsmall \times \numblendweightsheightsmall \times \numblendweightswidthsmall \times 
\numblendweightsdepthsmall}$ to a volume of blending weights $\tensblendweights \in \mathbb{R}^{\numjoints + 1 \times \numblendweightsheight \times \numblendweightswidth \times 
\numblendweightsdepth}$, where each channel represents the blending weights for each part, with an extra weight modeling the background. The volume channels are normalized using softmax, so that they sum to one, and can efficiently be queried using trilinear sampling.
To facilitate convergence, we exploit the known kinematic tree to build a prior over the blending weights that increases blending weights in the area surrounding each limb \cite{weng2022humannerf}.
\section{Additional Details}
\label{sec:details}

\subsection{Implementation Details}
\label{sec:imp}
Below we provide all the implementation details of our method, detailed in Section 3 in the main paper.

\subsubsection*{Grid-Based Volumetric Representation}
We use 100 images uniformly sampled from upper hemisphere poses along with corresponding camera intrinsic and extrinsic parameters to train our initial grid. We follow the standard ReLU Fields~\cite{karnewar2022relu} training process using their default settings aside from two modifications: 
\begin{enumerate}
    \item We change the result grid size from the standard $128^3$ to $160^3$ to increase the output render quality.
    \item As detailed in the main paper, we limit the order of spherical harmonics to be zero order only to avoid undesirable view-dependent effects (we further illustrate these effects in Section \ref{sec:sh}). 
\end{enumerate}

 

\subsubsection*{Text-guided Object Editing}
We perform 8000 training iterations during the object editing optimization stage. During each iteration, a random pose is uniformly sampled from an upper hemisphere and an image is rendered from our edited grid $G_e$ according to the sampled pose and the rendering process described in ReLU Fields \cite{karnewar2022relu}. Noise is then added to the rendered image according to the time-step sampled from the fitting distribution. 

We use an annealed SDS loss which gradually decreases the maximal time-step we draw $t$ from. Formally, this annealed SDS loss introduces three additional hyper-parameters to our system: a starting iteration $i_{start}$, an annealing frequency $f_a$ and an annealing factor $\gamma_a$. With these hyper-parameters set, we change our time-step distribution to be:
\begin{equation}
    t \sim U[t_0 + \varepsilon, t_{final}*k_i + \varepsilon],
\end{equation}
\begin{equation}
    k_i = 
    \begin{cases}
    1, & \text{if } i < i_{start} \\
    k_{i-1}*\gamma_a, & \text{else if } i\ \% \ f_a = 0 \\
    k_{i-1}, & \text{otherwise}
    \end{cases}
\end{equation}
In all our experiments, the values we use for $\varepsilon$, $i_{start}$, $f_a$ and $\gamma_a$ are 0.02, 4000, 600, and 0.75. Additionally, we stop annealing the time-step when it reaches a value of 0.35. %
The latent diffusion model we use in our experiments is "StableDiffusion 2.1" by Stability AI\href{https://huggingface.co/stabilityai/stable-diffusion-2-1}. %

We use a weight of $200$ to balance the two terms (multiplying $\mathcal{L}_\text{reg3D}$ by this weight value). The volumetric regularization term operates only on the density features of the editing grid. The optimizer we used in this (and all other stages) is the Adam optimizer~\cite{adamoptimizer} with a learning rate of 0.03 and betas 0.9, 0.999. The resolution of the images rendered from our grid is 266$\times$266. We add a "a render of" prefix to all of our editing prompts as we found that this produced more coherent results (and the images the LDM receives are indeed renders).



\subsubsection*{Spatial Refinement via 3D Cross-Attention}
The diffusion model we use for this stage is \href{ https://huggingface.co/CompVis/stable-diffusion-v1-4 }{"StableDiffusion 1.4" by CompVis} and it consists of several %
cross-attention layers at resolutions 32, 16, and 8. To extract a single attention map for each token we interpolate each cross attention map from each layer and attention head to our image resolution (266x266) and take an average per each token. %
The time-step we use to generate the attention maps is 0.2 (the actual step being 0.2 * $N_{steps}$ = 200). 

The cross-attention grids $A_{e}$ and $A_{obj}$ contain a density feature and an additional one-dimensional feature $a$, which represents the cross-attention value at a given voxel and can be interpreted and rendered as a grayscale luma value. We initialize the density features in these grids to the density features of the editing grid's (the former stage's output) and freeze them.
At each refinement iteration we generate two 2D cross-attention maps from the LDM, one for the object and one for the edit. After obtaining the 2D cross-attention maps, we render gray-scale heatmaps %
from $A_{e}$ and $A_{obj}$ and use $L1$ loss to encourage similarity between the rendered attention images and their corresponding attention maps extracted from the diffusion model. We repeat this process for 1500 iterations, sampling a random upper-hemisphere pose each time. As in the former optimization stage, we use the Adam optimizer with a learning rate of 0.03 and betas 0.9 and 0.999 and generate images in 266$\times$266 resolution.

After obtaining the two grids $A_{e}$ and $A_{obj}$, we perform element-wise softmax on their $a$ values to obtain probabilities for each voxel belonging to either the object, denoted by $P_{obj}(v)$, or the edit, denoted by $P_e(v)$. 
We then proceed to calculate the binary refinement volumetric mask. To do this we define a graph in which each non-zero density voxel in our edited grid $G_e$ is a node. We define "edit" and "object" labels as the \emph{source} and \emph{drain} nodes, such that a node connected to the source node is marked as an "edit" node and a node connected to the drain node is marked as an "object" node. We rank the nodes according to their $P_e(v)$ values and connect the top $N_{init-edit}$ nodes to the source node. We then rank the nodes according to their $P_{obj}(v)$ value and connect the top $N_{init-object}$ nodes to the drain node.
We then connect the non-terminal nodes to each-other in a 6-neighborhood with the capacity of each edge being $w_{pq}$ as detailed in the main paper.


We set the hyper-parameters $N_{init-edit}$ and $N_{init-object}$ to be 300 and 200. %
To perform graph-cut~\cite{boykov2001fast}, we used the \href{ https://github.com/pmneila/PyMaxflow }{PyMaxflow} implementation of the max-flow / min-cut algorithm. %

\subsection{Evaluation Protocol}
To evaluate our results quantitatively, we constructed a test set composed of three scenes: Dog, Cat and Kangaroo, and six editing prompts: (1) A $\left<object\right>$ wearing big sunglasses, (2) A $\left<object\right>$ wearing a Christmas sweater, (3) A $\left<object\right>$ wearing a birthday party hat, (4) A yarn doll of a $\left<object\right>$, (5) A wood carving of a $\left<object\right>$, (6) A claymation $\left<object\right>$. This yields 18 edited scenes in total. We render each edited scene from 100 different poses distributed evenly along a $360^{\circ}$ ring. In addition to these 18 scenes we also render 100 images from the same poses on the initial (reconstruction) grid $G_i$ for each input scene. When comparing our result with other 3D textual editing papers we evaluate our results using two CLIP-based metrics. %
The CLIP model we used for both of these metrics is \href{https://github.com/openai/CLIP}{ViT-B/32} and the input image text prompts used to calculate the directional CLIP metric is ``A render of a $\left<object\right>$". $CLIP_{Dir}$ is calculated for each edited image in relation to the corresponding image in the reconstruction scene.
To quantitatively evaluate ablations we use two additional metrics using FID \cite{Seitzer2020FID}. For this we use the \href{ https://github.com/mseitzer/pytorch-fid }{pytorch implementation} given by the authors with the standard settings. 

\subsubsection*{$360^\circ$ \emph{Real Scenes}} For the $360^\circ$ \emph{Real Scenes} edits we follow the same implementation details as outlined previously, with two modifications: 
\begin{enumerate}
    \item Our input poses are  created in a spherical manner and when rendering we sample linearly in inverse depth rather than in depth as seen in the official implementation of NeRF \href{https://github.com/bmild/nerf}.
    \item We perform 5000 training iterations during the object editing optimization stage and the values we use for $\varepsilon$, $i_{start}$, $f_a$ and $\gamma_a$ are 0.02, 3000, 400, and 0.75.
\end{enumerate}

\subsection{3D Object Editing Techniques} 
Below we provide additional details on the alternative 3D object editing techniques we compare against. All of the techniques we compare against use only an un-textured mesh and an editing prompt as input. As such, we used the meshes our inputs were rendered from as input for the editing methods. Additionally, we tested an additional scenario in which we imported the 'horse' mesh from the \href{https://github.com/threedle/text2mesh/blob/main/data/source_meshes/horse.obj}{Text2Mesh GitHub repository} to blender, added a grey-matte material to it and rendered images of it to use as input for our system. This scenario used four prompts - (1) A wood carving of a horse, (2) A horse wearing a Santa hat, (3) A donkey, (4) A carousel horse, and was used for qualitative comparisons only.

\subsubsection*{Text2Mesh} 
When comparing to Text2Mesh we used the \href{https://github.com/threedle/text2mesh}{code provided by the authors} and the input settings given in the "run\_horse.sh" demo file.



\subsubsection*{SketchShape}
 In this comparison we again use the \href{https://github.com/eladrich/latent-nerf}{code provided by the authors}. And the input parameters used are the default parameters in the 'train\_latent\_nerf.py' script  \href{https://github.com/eladrich/latent-nerf/tree/main/scripts}{'train\_latent\_nerf.py' script} with 10,000 training steps (as opposed to the default 5,000).

\subsubsection*{Latent-Paint}
We compared our method to Latent-Paint only qualitatively as this method outputs edits that transform only the appearance of the input mesh, rather than appearance and geometry. As in SketchShape we used the code provided by the authors and used the default input settings provided for latent paint, which are given in the  \href{https://github.com/eladrich/latent-nerf/tree/main/scripts}{'train\_latent\_paint.py' script}.

\subsection{2D Image Editing Techniques}

When comparing to InstructPix2Pix and SDEdit we constructed two image sets for each scene / prompt combination we wanted to test. Both sets were created by rendering one of our inputs in evenly spaced poses along a $360^{\circ}$ ring, one set was rendered over a white background and the other over a 'realistic' image of a brick wall. We used these sets as input for each 2D editing method along with an editing prompt and compared the results to rendered outputs from our result grids. When comparing to InstructPix2Pix we used the standard \href{https://huggingface.co/docs/diffusers/api/pipelines/stable_diffusion/pix2pix}{InstructPix2Pix pipeline} with 16bit floating point precision, a guidance scale of 1 and 20 inference steps. When giving prompts to InstructPix2Pix we rephrased our prompts as instructions, for example turning "a dog wearing sunglasses" to "put sunglasses on this dog". When comparing to SDEdit we used the \href{https://huggingface.co/docs/diffusers/using-diffusers/img2img}{standard SDEdit pipeline} with guidance scale of 0.75 and a strength of 0.6.



\ignorethis{
\begin{enumerate}
    \item Real-scenes?
    \item cross-attention maps - \textbf{maybe a correction figure?} also, what timesteps are we using, any other important details?
    \item explain ablations and comparisons (all details needed to reproduce experiments)
    \item all hyperparameters
    \item Dataset details - the meshes, all the prompts
\end{enumerate}
}



% TODO: Decide what to do with this info

% \section{Training Details}

% The model was trained for 100,000 iterations with a batch size of 24 (24 tesla V100 GPUs over 8 machines). To stablize training, we add one sided label smoothing \cite{gantutorial} to the adversarial loss. We use the ADAM \cite{adam} optimizer with $\beta_1=0.5$ and $\beta_2 = 0.99$. We first pretrain the style encoder for one epoch with learning rate $1e-4$. After, the generator and discriminator are trained with $lr=1e-4$ and the style encoder with $lr=1e-6$. For evaluation, we compute the exponential moving average of model parameters for the generator and style encoder \cite{gan_ema}. Weights are initialized to $N(0,0.1)$ and biases to $0$.

%\textbf{Generator Architecture}. Our Generator consists of one downsampling block, eight intermediate blocks, and one upsampling block, all of which use pre-activation residual blocks \cite{residual}. We use weight modulation and demodulation \cite{Karras2019stylegan2} on every layer and no other normalization. Weight demodulation is disabled on the final output block \cite{Karras2019stylegan2}. Leaky ReLU \cite{Maas2013RectifierNI} is used after each block. A learned style code is injected into each weight modulation layer. 

%\textbf{Style Encoder Architecture}. The style encoder consists of four 2x downsampling pre-activation residual blocks \cite{residual} and an adaptive average pooling layer that pools to a $1\times1\times512$ style code. The style code is then normalized to mean 0 and standard deviation 1, providing the normalized style representation of an image. During pre-training, a linear classifier head with a binary cross entropy loss is trained on the style code to predict real or fake. The learning rate of the loss is reduced to 0.01 after pretraining. This practice prevents the style code from collapsing to zero or a single value. 

%\textbf{Discriminator Architecture} The discriminator is based on a PatchGAN discriminator \cite{isola2017_pix2pix} and consists of four 2x downsampline pre-activation residual blocks with leaky ReLU \cite{Maas2013RectifierNI} and Instance Normalization \cite{instancenorm} at each block. We observe that instance normalization in the discriminator is critical to stable training on our diverse and complex sim2real dataset.
\section{Inference Details}
\label{ap:inference_details}

\subsection{Inference Speed}
\label{ap:inference_speed}
Our synthesis model renders images at 2.96fps over a single A100 GPU. We can parallelize inference by generating batches of 8 consecutive frames on separate GPUs for 23.7fps.
The animation model \change{has a throughput} of 1.08fps using 1000 diffusion sampling timesteps, measured by dividing the number of generated frames by the computation time at the end of the diffusion process. Meng \etal \cite{meng2022on} show that a reduction to 16 timesteps is possible with no or minimal loss in quality for a projected performance of 67.5fps. Hence, we believe our framework can be made real-time, which is a scope for future works. 

\subsection{Animation Model Inference Details}
\label{ap:animation_inference}

At inference time, the user is presented with a fully-masked, empty sequence $\vecsequencecond=0$, $\vecmask=0$, $\vectextactioncond=``"$, $\vecmaskaction=0$. Any object property can be specified as a conditioning signal in $\vecsequencecond$ and text action descriptions for any sequence timesteps can be provided in $\vectextactioncond$, with masks updated accordingly. The desired framerate $\framerate$ is also specified.

The text encoder $\nettext$ produces text embeddings $\vecactionemb$ as in Eq.~\eqref{eq:text_encoder}. Successively, the \emph{reverse process} is started at diffusion time $\difftimestep=\numdifftimesteps$, with $\vecsequenceprednoise_{\numdifftimesteps}$ sampled from the normal distribution. The DDPM sampler \cite{ho2020ddpm} queries the temporal model 
according to Eq.~\eqref{eq:temporal_model} to progressively denoise $\vecsequenceprednoise_\difftimestep$ and obtain the predicted sequence $\vecsequencepred=\vecsequenceprednoise_0$. The final sequence is obtained as $\vecsequence = \vecsequencepred + \vecsequencecond$, following Eq.~\eqref{eq:sequence_composition}.

\subsubsection{High Framerate Generation}
\label{ap:high_framerate_generation}
To produce sequences at the dataset framerate, we devise a two-stage sampling procedure designed to prevent an excessive increase in the sequence length. In the first stage, we sample the desired sequence at a low framerate $\framerate_1$. In the second stage, we exploit the masking mechanism and framerate conditioning to increase the framerate and, consequently, the length of the generated sequence. After the first stage, we consider a higher framerate $\framerate_2$ and extend the sampled sequence $\vecsequence$ with new states between existing ones, that we call keyframes, until the sequence length corresponding to $\framerate_2$ is reached. This sequence constitutes the new $\vecsequencecond$. Any previous action conditioning is copied in a new $\vectextactioncond$ in the corresponding keyframe locations. Masks are updated to be 1 in the position of the keyframes and 0 elsewhere. The sampling process is then repeated with the new conditioning signals and a sequence $\vecsequence$ is produced at the final framerate $\framerate_2$. To avoid an explosion in the length of the sequence, we exploit keyframes to divide the sequence into shorter chunks beginning and terminating at a keyframe, and sampling is performed separately on each chunk.

\subsubsection{Autoregressive Generation}
\label{ap:augoregressive_generation}

Our masking mechanism can be used to produce predictions autoregressively, enabling long sequence generation. Autoregressive generation can be obtained by considering a sequence $\vecsequencecond$ and removing the states corresponding to the first $t$ timesteps. $t$ timesteps are then added at the end of the sequence and a mask $\vecmask$ is created to zero out these additional $t$ steps. Conditioning signals can then be specified as desired for the last $t$ timesteps. When sampling $\vecsequencepred$, a prediction is produced for the additional timesteps and the procedure can be repeated.
For this chapter, fix a prime $p$. We first discuss deformations of coalgebras from $\F_{p}$
to the $p$-adic integers and further to the $p$-completed sphere $\S_{p}^{\wedge}$ which leads
us to the question of how coalgebras behave with respect to $p$-completion. We introduce the
notion of a $p$-complete coalgebra and show that this is well behaved with respect to the
deformation theory discussed in the previous chapter. We then use this to iterate
Proposition~\ref{witt} and prove our main results, namely the existence of Witt Vectors
and spherical Witt Vectors for formally \'etale coalgebras. Then we specialize to the case
of homology coalgebras, show that for a finite space $X$ the coalgebra $\F_{p}[X]$ is formally
\'etale, and answer our initial question about the relation between $\S[X]^{\wedge}_{p}$
and $\F_{p}[X]$

\subsection{Coalgebras and $p$-completion}

We have seen that the functors that interest us are all \textit{nilcomplete}. For a nilcomplete
functor $X:\rm{CAlg}^{\rm{cn}} \to \cl{S}$ and a connective $\bb{E}_{\infty}$-ring $R$, we can construct
lifts from $X(\pi_{0}R)$ to $X(R)$ inductively along the Postnikov tower
\[ \dots \to \tau_{\leq2}R \to \tau_{\tau\leq 1}R \to \tau_{\leq0} R =\pi_{0}R.\]
This is however not quite enough to obtain our goal of lifting from $\F_{p}$ to the
$p$-completed sphere, we first need to pass to $\Z_{p}= \pi_{0}\S_{p}^{\wedge}$.
Explicitly, this means constructing lifts against the tower
\[\dots \to \Z/p^{3}\to \Z/p^{2}\to \Z/p\to \F_{p}\]
which is clearly presents a different problem. With the machinery developed thus far, we can already
prove the following for a general deformation problem.

\begin{proposition}\label{liftpgen}
  Let $X: \rm{CAlg}^{\rm{cn}} \to \cl{S}$ be a cohesive functor and $A\in X(\F_{p})$
  such that $T_{X_{A}}\simeq 0$. Then there exists a unique lift of $A$ to a point in
  $\flim_{n}X(\Z/p^{n})$.
\end{proposition}
\begin{proof}
  Set $A_{0}= A$, we inductively construct lifts against the tower of square zero extensions
  \[\dots \to \Z/p^{3} \to \Z/p^{2}\to \F_{p}.\]
  Suppose we have already constructed lifts $A_{k}$ for $k\le n$ for some $n$.
  Applying Proposition~\ref{bc} inductively, we get that
  \[T_{X_{A_{n}}}^{\F_{p}} \simeq T^{\F_{p}}_{X_{A_{0}}} \simeq 0.\]
  Thus, since $\Z/p^{n+1}\to \Z/p^{n}$ is a square zero extension with fiber $\F_{p}$,
  Proposition~\ref{deformations} implies that the fiber
  \[X_{A_{n}}^{\Z/p^{n+1}}=\rm{fib}_{A_{n}}(X(\Z/p^{n+1})\to \Z/p^{n})\]
  is contractible and we find an essentially unique lift $A_{n+1}$. This proves the claim.
\end{proof}
 Of course, for an arbitrary functor $X:\rm{CAlg}^{\rm{cn}} \to \cl{S}$ the natural map
$X\to \flim_{n}X(\Z/p^{n})$ might not be an equivalence, meaning that in this generality
we can only construct pro-$p$ objects of $X$ using this inductive method.
In fact, we have that $\rm{cCAlg}_{\Z_{p}}\neq  \flim_{n} \rm{cCAlg}_{\Z/p^{n}}$. To remedy
this problem we show that this limit admits a description via \textit{$p$-complete} coalgebras.
To do this, we first recall some facts about $p$-complete modules.

\begin{definition}
Let $R$ be an $\bb{E}_{\infty}$-ring, then $M \in \rm{Mod}_{R}$ is called
$p$-\textit{complete} if the limit
\[ \lim \left(\dots \rar{\cdot p} M \rar{\cdot p}M \right)\]
vanishes. We denote the full subcategory spanned by the $p$-complete modules by $(\rm{Mod}_{R})_{p}^{\wedge}$.
\end{definition}

\begin{remark}
The inclusion $(\rm{Mod}_{R})_{p}^{\wedge} \rari{} \rm{Mod_{R}}$ admits a left adjoint which takes a module $M$
to its \textit{$p$-completion} given by the limit
\[ \lim \left( \dots \to M/p^{2} \to M/p \right).\]
In fact, $M$ is $p$-complete if and only if the natural map $M \to \lim M/p^{n}$ is an equivalence.
This inherits a natural $R^{\wedge}_{p}$-module structure, thus $p$-completion also gives
an equivalence of categories $(\rm{Mod}_{R})^{\wedge}_{p} \simeq (\rm{Mod}_{R^{\wedge}_{p}})^{\wedge}_{p}$ which
allows us to identify these in what follows.\\
The tensor product of $p$-complete modules is in general not $p$-complete. However, the
category $(\rm{Mod}_{R})_{p}^{\wedge}$ admits a symmetric monoidal structure given by the formula
 \[ M \otimes_{(\rm{Mod}_{R})_{p}^{\wedge}} N := ( M \otimes N )^{\wedge}_{p}.\]
 With this monoidal structure the $p$-completion functor $\rm{Mod}_{R}\to (\rm{Mod}_{R})_{p}^{\wedge}$
 is strong monoidal, while the inclusion is only lax monoidal.
\end{remark}

 \begin{definition}
   Let $R$ be an $\bb{E}_{\infty}$-ring. We define the $\infty$-category of $p$-complete
   $R$-coalgebras is given by.
   \[ {(\rm{cCAlg}_{R})}^{\wedge}_{p}:= \rm{cCAlg}({(\rm{Mod}_{R})}^{\wedge}_{p}).\]
 \end{definition}

 \begin{warning}
   Let $R$ be a $\bb{E}_{\infty}$-ring. Notice that by our definition a $p$-complete $R$-coalgebra
   is the same as a $p$-complete $R^{\wedge}_{p}$-coalgebra and so we do not differentiate between
   the two notions.
   However, this is \textit{not} the same as an $R^{\wedge}_{p}$-coalgebra whose underlying
   spectrum is $p$-complete. The process of $p$-completion does refine to a functor
   $\rm{cCAlg}_{R} \to (\rm{cCAlg}_{R^{\wedge}_{p}})^{\wedge}_{p}$,
   but it does not factor through the category $\rm{cCAlg}_{R^{\wedge}_{p}}$.
 \end{warning}

 We now show check that the assignment $R \mapsto \rm{cCAlg}_{R}^{\rm{cn}}$ is subject to the machinery
 of deformation theory.

 \begin{lemma}\label{conil2}
   The following statements hold:
   \begin{enumerate}
     \item   Suppose we have a pullback diagram of connective $\bb{E}_{\infty}$-rings
   \[\begin{tikzcd}
	R\p & S\p \\
	R & S
	\arrow[from=1-1, to=2-1]
	\arrow[from=2-1, to=2-2]
	\arrow[from=1-2, to=2-2]
	\arrow[from=1-1, to=1-2]
\end{tikzcd}\]
such that the map $\pi_{0}R \to \pi_{0}S$ is surjective. Then the natural map
\[ (\rm{cCAlg}_{R\p}^{\rm{cn}})^{\wedge}_{p} \to (\rm{cCAlg}_{R}^{\rm{cn}})^{\wedge}_{p}\times_{(\rm{cCAlg}_{S}^{\rm{cn}})^{\wedge}_{p}} (\rm{cCAlg}_{S\p}^{\rm{cn}})^{\wedge}_{p}\]
is an equivalence.
     \item For every connective $\bb{E}_{\infty}$-ring $R$, the natural map
           \[ (\rm{cCAlg}_{R}^{\rm{cn}})^{\wedge}_{p} \to\flim_{n} (\rm{cCAlg}_{\tau_{\le n}R}^{\rm{cn}})^{\wedge}_{p}\]
           is an equivalence.
   \end{enumerate}
 \end{lemma}
 \begin{proof}
   Ad 1.: Arguing as in the proof of Proposition~\ref{Mod}, it suffices to show that the
   strong monoidal functor
   \begin{align*}
    (\rm{Mod}_{R\p})^{\wedge}_{p} \to (\rm{Mod}_{R})^{\wedge}_{p}\times_{(\rm{Mod}_{S})^{\wedge}_{p}} (\rm{Mod}_{S\p})^{\wedge}_{p}
   \end{align*}
   is an equivalence. Indeed, given a point $(M,N,h)$ in the pullback, the $R\p$-module $M \times_{M \otimes_{R} S}N$
   is again $p$-complete since $p$-completion commutes with limits. Thus, the inverse functor of
   Proposition~\ref{Mod} also induces a functor on the categories of $p$-complete modules. Moreover,
   we have that
   \[ ((M\times_{M\otimes_{R}S}N)\otimes_{R\p} R)^{\wedge}_{p} \simeq M^{\wedge}_{p} \simeq M\]
   \[ ((M \times_{M\otimes_{R}}N)\otimes_{R\p}S\p)^{\wedge}_{p}\simeq N^{\wedge}_{p} \simeq N,\]
   where the first equivalences hold by Proposition~\ref{Mod}, and the latter since $M$ and $N$ are
   to be $p$-complete. Finally, for $M\in (\rm{Mod}_{R\p})^{\wedge}_{p}$, we compute that
   \[ (M \otimes_{R\p} R)^{\wedge}_{p}\times_{(M \otimes_{R\p} S)^{\wedge}_{p}}(M \otimes_{R\p}S\p)^{\wedge}_{p}
     \simeq \left( M \otimes_{R\p} R \times_{M\otimes_{R\p} S} M \otimes_{R\p} S\p\right)^{\wedge}_{p}
   \simeq M^{\wedge}_{p} \simeq M,\]
 where we have again used the result of Proposition~\ref{Mod} and the fact that $p$-completion commutes
 with limits.\\
 Ad 2: This uses the exact same arguments applied to the equivalence of Corollary~\ref{nilcomplete}.
 \end{proof}

 \begin{corollary}
   For any $n\in \bb{N}$, the functor
   \[ \rm{CAlg}^{\rm{cn}} \to \cl{S} \qquad R \mapsto [(\rm{cCAlg}_{R}^{\rm{cn}})^{\wedge}_{p}]^{\Delta^{n}}\]
   is coherent and nilcomplete.
 \end{corollary}

 We now prove the crucial $p$-completeness result for $\Z_{p}$-modules. As before
 this will enable us to deduce the same result for coalgebras and allow us to tackle the
 actual problem of comparing coalgebras over $\F_{p}$, $\Z_{p}$ and $\S_{p}^{\wedge}$.
\begin{proposition}\label{pcomp}
  Let $\rm{Mod}^{\wedge}_{\Z_p} \subseteq \rm{Mod}_{\Z_{p}}$ denote the full subcategory spanned by the
  $p$-complete $\Z_{p}$-module spectra. Then the natural map
  \[ \rm{Mod}_{\Z_{p}} \to \flim_{n} \rm{Mod}_{\Z/p^{n}} \quad N \mapsto (N\otimes_{\Z_{p}}\Z/p^{n})\]
  restricts to a strong monoidal equivalence
  \[(\rm{Mod}_{\Z_{p}})^{\wedge}_{p} \simeq \flim_{n}\rm{Mod}_{\Z/p^{n}}. \]
\end{proposition}
\begin{proof}
  The functor admits a right adjoint which takes $(M_{n})\in \flim_{n}\rm{Mod}_{\Z/p^{n}}$ to the limit
  $\lim_{n}M_{n}$ taken in the category of $\Z_{p}$-modules. Since $p$-complete modules are closed under
  limits, the essential image of this functor is contained in $\rm{Mod}_{\Z_{p}}^{\wedge}$. Moreover,
  if $M\in \rm{Mod}_{\Z_{p}}^{\wedge}$, then we have that
  \[ \flim_{n}(M \otimes_{\Z_{p}} \Z/p^{n}) \simeq \flim_{n} M/p^{n} \simeq M^{\wedge}_{p}\simeq M.\]
  Hence, the counit of the adjunction is an equivalence on $p$-complete modules.
  Conversely, given $(N_{k})\in \flim_{k}\rm{Mod}_{\Z/p^{k}}$ write $N= \lim_{k}N$. We want
  to show that, for every $n$ the natural map
  \[ N \otimes_{\Z_{p}} \Z/p^{n}\rar{\sim}N_{n}\]
  is an equivalence. Since $N \otimes_{\Z_{p}}Z/p^{n}\simeq N/p^{n}$ and limits are exact, we have an equivalence
  \[N \otimes_{\Z_{p}}\Z/p^{n}\simeq \lim_{k >n}(N_{k}\otimes_{\Z_{p}}\Z/p^{n}).\]
  Thus, the unit of the adjunction may be written as
  \[ \lim_{k>n}(N_{k} \otimes_{\Z_{p}}\Z/p^{n}) \to \lim_{k>n}(N_{k}\otimes_{\Z/p^{k}}\Z/p^{n})\simeq N_{n}\]
  and so has fiber given by
  \[ F_{n}:=\lim_{k>n}\left(N_{k}\otimes_{\Z/p^{k}}\rm{fib}(\Z/p^{k}\otimes_{\Z_{p}}\Z/p^{n}\to \Z/p^{n}) \right).\]
  Now we compute the fiber of $\Z/p^{k}\otimes_{\Z_{p}}\Z/p^{n}\to \Z/p^{n}$ as the module
  \[ \rm{Tor}^{\Z_{p}}(\Z/p^{k}, \Z/p^{n})[1]\simeq \Z/p^{n}[1].\]
  The reduction map $\Z/p^{k}\to \Z/p^{k-1}$ is induced by the map of projective resolutions
\[\begin{tikzcd}
	{\Z_p} & {\Z_p} \\
	{\Z_p} & {\Z_p}
	\arrow["{\cdot p^k}", from=1-1, to=1-2]
	\arrow["\id", from=1-2, to=2-2]
	\arrow["{\cdot p}"', from=1-1, to=2-1]
	\arrow["{\cdot p^{k-1}}"', from=2-1, to=2-2],
\end{tikzcd}\]
hence, on Tor it induces the multiplication by $p$ map
\[ \Z/p^{n}=\rm{Tor}^{\Z_{p}}(\Z/p^{k}, \Z/p^{n})\rar{\cdot p} \rm{Tor}^{\Z_{p}}(\Z/p^{k-1}, \Z/p^{n}) =\Z/p^{n}.\]
Thus, if we have $k\p > k > n$ such that $k\p -k > n$, the transition map
\[ F_{k\p}=N_{k\p} \otimes \rm{Tor}^{\Z_{p}}(\Z/p^{k}, \Z/p^{n})\to N_{k} \otimes \rm{Tor}^{\Z_{p}}(\Z/p^{k-1}, \Z/p^{n})= F_{k}\]
vanishes since the Tor-groups are $p^{n}$-torsion. Choosing a cofinal subset $S\subseteq \bb{N}_{>n}$ such that
$\abs{k\p -k}> n$ for any distinct $k\p,k\in S$, we see that
\[ \lim_{k>n} F_{k}\simeq \lim_{k\in S} F_{k} \simeq 0 \]
vanishes. Thus, since limits are exact, the map $N \otimes_{\Z_{p}} \Z/p^{n}\rar{\sim}N_{n}$ is an equivalence.\\
To see that the functor $\rm{Mod}_{\Z_{p}}^{\wedge} \to \flim_n \rm{Mod}_{\Z/p^{n}}$ is strong monoidal,
we observe that since cofibers and limits are exact, we have for each $n$ equivalences
\begin{align*}
  (M \otimes_{\Z_{p}} N)^{\wedge}_{p} \otimes_{\Z_{p}}\Z/p^{n} &\simeq \lim_{k}(M/p^{k} \otimes_{\Z_{p}}N/p^{k})/p^{n}\\
                                              &\simeq \lim_{k}\left((M/p^{n} \otimes_{\Z_{p}} N/p^{n})\otimes_{Z_{p}}\Z/p^{k}\right) \\
  &\simeq ((N\otimes_{\Z_{p}}\Z/p^{n}) \otimes_{\Z_{p}} (M \otimes_{\Z_{p}}\Z/p^{n}))^{\wedge}_{p}.
\end{align*}
This proves the claim.
\end{proof}

\begin{corollary}\label{pcomp1}
  We have an equivalence of categories
  \[ (\rm{cCAlg}_{\Z_{p}})_{p}^{\wedge} \rar{\sim} \flim_{n} \rm{cCAlg}_{\Z/p^{n}} \quad A \mapsto (A\otimes_{\Z_{p}}\Z/p^{n})\]
  with inverse taking a system of coalgebras $(B_{n})$ to the limit $\lim_{n}B_{n}$ taken in the
  category of ($p$-complete) $\Z_{p}$-modules, equipped with the induced $p$-complete
  $\Z_{p}$-coalgebra structure.
\end{corollary}
\begin{proof}
This follows from Proposition~\ref{pcomp}, arguing as in the proof of Proposition~\ref{Mod}.
\end{proof}

\begin{corollary}\label{obliftzp}
  Let $X(\blank)= (\rm{cCAlg}_{\blank}^{\rm{cn}})^{\Delta^{0}}$ and $A\in X(\F_{p})$ such that $T_{X_{A}}\simeq 0$.
  Then the space of lifts of $A$ to a $p$-complete $\Z_{p}$-coalgebra is contractible
\end{corollary}
 \begin{proof}
 Combine Proposition~\ref{liftpgen} and Corollary~\ref{pcomp1}.
 \end{proof}

\begin{corollary}\label{mapliftzp}
  Let $\varphi: B\to A$ be a map of connective, formally \'etale $\F_{p}$-coalgebras. Then the space of
  lifts of $\varphi$ to a map of $p$-complete $\Z_{p}$-coalgebras $B\p \to A\p$ is contractible.
\end{corollary}
\begin{proof}
    Let $ \cl{X}(\blank)=\rm{cCAlg}_{\blank}^{\rm{cn}}$. By Proposition~\ref{etalchar} the natural map
    \[ T_{\cl{X}^{\Delta^{1}}_{\varphi}} \to T_{\cl{X}^{\Delta^{0}}_{B}}\]
    is an equivalence, but since $B$ is formally \'etale we have $T_{\cl{X}^{\Delta^{0}}_{B}} \simeq 0$.
    Hence, the claim follows by applying Proposition~\ref{liftpgen} to the functor $\cl{X}^{\Delta^{1}}$
    and using Corollary~\ref{pcomp1}.
\end{proof}

Having shown this, we can now construct a functor which is analogous to the classical
Witt-Vectors, which allow us to pass from \'etale $\F_{p}$-algebras to $\Z_{p}$-algebras.

\begin{theorem}
  Let $\cl{C}\subseteq (\rm{cCAlg}_{\Z_{p}}^{\rm{cn}})^{\wedge}_{p}$ denote the full subcategory spanned by those
  coalgebras $A$ for which $A\otimes_{\Z_{p}} \F_{p}$ is formally \'etale. Then the base change functor
  \[ \cl{C} \to \rm{cCAlg}_{\F_{p}}^{\rm{cn}, \rm{f\acute{e}t}}  \qquad A \mapsto A\otimes_{\Z_{p}}\F_{p}\]
  is fully faithful and essentially surjective. In particular, the quasi inverse defines a functor
  \[ W_{p}: \rm{cCAlg}_{\F_{p}}^{\rm{cn,f\acute{e}t}} \to (\rm{cCAlg}_{\Z_{p}}^{\rm{cn}})^{\wedge}_{p}\]
  which is fully faithful and satisfies $W_{p}(A)\otimes_{\Z_{p}}\F_{p} \simeq A$ for every connective, formally
  \'etale $\F_{p}$-coalgebra $A$.
\end{theorem}

\begin{proof}
  Combine Corollary~\ref{obliftzp} and Corollary~\ref{mapliftzp}.
\end{proof}

We now turn our attention to the leap from $\Z_{p}$ to $\S_{p}^{\wedge}$. The following proposition shows that,
for an arbitrary cohesive and nilcomplete functor, a $\Z_{p}$-valued point which has vanishing $\F_{p}$-tangent
complex admits a unique lift to a $\S_{p}^{\wedge}$-valued point. This is surprising, as we do not
actually require any information about the $\Z_{p}$-tangent complex, everything is determined by
what happens modulo $p$.

\begin{proposition}\label{spherelift}
  Let $X: \rm{CAlg}^{\rm{cn}} \to \cl{S}$ be a cohesive and nilcomplete functor and let $A \in X(\Z_{p})$
  such that $T_{X_{A\otimes_{\Z_{p}}\F_{p}}}\simeq 0$. Then $A$ admits an essentially unique lift to $X(\S_{p}^{\wedge})$.
\end{proposition}

\begin{proof}
  We inductively construct lifts against the Postnikov Tower
  \[ \dots \to \tau_{\leq2} \S_{p}^{\wedge}  \to \tau_{\leq 1} \S_{p}^{\wedge} \to \tau_{\leq 0} \S_{p}^{\wedge} \simeq \Z_{p}. \]
  Write $A=A_{0},~S_{n}= \tau_{\leq n}\S_{p}^{\wedge},~ M_{n} = \pi_{n}S_{n}$ and assume we have already constructed
  a unique lift $A_{n}$ to $X(S_{n})$. Consider the square zero extension
  \[ M_{n+1}[n+1] \to S_{n+1}\to S_{n}.\]
  Since $M_{n+1} = \pi_{n+1}S_{n+1}$ is concentrated in a single degree, the $S_{n}$-action factors
  through $S_{0}=\Z_{p}$. Moreover, since $\pi_{n+1}S_{n+1}$ is of finite $p$-torsion, the action
  further factors through $\Z/p^{k}$ for some $k\geq 0$. Thus, Proposition~\ref{bc} implies that
  we have an equivalence
  \[ T_{X_{A_{n}}}^{M_{n+1}[n+1]} \simeq \Sigma^{n}T_{X_{A_{n}}}^{M_{n+1}} \simeq T_{X_{A_{n} \otimes_{S_{n}} \Z/p^{k}}}^{M_{n+1}}.\]
  Arguing as in Proposition~\ref{cofib} with respect to the square zero extension
  \[ \F_{p} \to \Z/p^{k}\to \Z/p^{k-1},\]
  we see that we have a cofiber sequence
  \[  T^{M_{n+1}\otimes_{\Z/p^{k}}\F_{p}}_{X_{A_{n} \otimes_{S_{n}} \Z/p^{k-1}}}
    \to T_{X_{A_{n} \otimes_{S_{n}} \Z/p^{k}}}^{M_{n+1}}
    \to T^{M_{n+1}\otimes_{\Z/p^{k}}\Z/p^{{k-1}}}_{X_{A_{n} \otimes_{S_{n}} \Z/p^{k-1}}}.\]
  For the left hand term, Proposition~\ref{bc} gives the equivalence
  \[ T_{X_{A_{n}\otimes_{S_{n}}\Z/p^{k-1}}}^{M_{n+1}\otimes_{\Z/p^{k}}\F_{p}}
    \simeq T_{X_{A \otimes_{\Z_{p}}\F_{p}}}^{{M_{n+1}\otimes_{\Z/p^{k}}\F_{p}}}
    \simeq T_{X_{A\otimes_{\Z_{p}}\F_{p}}}\otimes_{\F_{p}}( M_{n+1}\otimes_{\Z/p^{k}}\F_{p} ) \simeq 0,\]
  where we have used that, since $M_{n+1}$ is finitely generated, the $\F_{p}$-module
  $M_{n+1}\otimes_{\Z/p^{k}}\F_{p}$ is perfect. For the right hand term we
  replace $M_{n+1}$ with $M_{n+1} \otimes_{\Z/p^{k}}\Z/p^{k-1}$ and repeat the argument,
  inductively yielding equivalences
  \[ T^{M_{n+1}}_{X_{A_{n}\otimes_{S_{n}}\Z/p^{k}}}
    \simeq T^{M_{n+1}\otimes_{\Z/p^{k}}\Z/p^{{k-1}}}_{X_{A_{n-1} \otimes_{S_{n-1}} \Z/p^{k-1}}}
  \simeq \cdots \simeq T^{M_{n+1}\otimes_{\Z/p^{k}} \F_{p}}_{X_{A \otimes_{\Z_{p}}\F_{p}}} \simeq 0.\]
In total, this shows that $T_{X_{A_{n}}}^{M_{n+1}[n+1]} \simeq 0$, and hence $A_{n}$ admits an essentially
unique lift to $X(S_{n+1})$. Thus, the fiber over $A$ of the map
\[ X(\S_{p}^{\wedge})\simeq \flim_{n}X(S_{n})\to X( \Z_{p})\]
is contractible and we are done.
  \end{proof}

  \begin{lemma}\label{pcomparison}
    Write $\cl{X}(\blank)=\rm{cCAlg}^{\rm{cn}}_{\blank}$ and $\cl{Y}(\blank)=
    (\rm{cCAlg}^{\rm{cn}}_{\blank})^{\wedge}_{p}$. Then the $p$-completion map $f:\cl{X}\to \cl{X}\p$
    induces an equivalence
    \[ T^{M}_{(\cl{X}^{\Delta^{n}})_{\xi}} \to  T^{M}_{(\cl{Y}^{\Delta^{n}})_{f(\xi)}}\]
        for every $\F_{p}$-module $M$, $n\in \bb{N}$ and $\xi \in \cl{X}(\F_{p})^{\Delta^{n}}$.
  \end{lemma}
  \begin{proof}
    For any $\F_{p}$-algebra $R$ the $p$-completion map gives an equivalence
    $\rm{Mod}_{R}\rar{\sim} (\rm{Mod}_{R})^{\wedge}_{p}$, since multiplication by some power of $p$
    is nullhomotopic over $\F_{p}$. In particular, this applies to the split square zero
    extension $\F_{p}\oplus M$ for any $M \in \rm{Mod}_{\F_{p}}$ and so the natural map
    $\cl{X}(\F_{p}\oplus M) \to \cl{Y}(\F_{p}\oplus M)$ is an equivalence as well.
    Consequently, we also obtain natural equivalences between the fibers
    \[ (\cl{X}^{\Delta^{n}})_{\xi}^{\F_{p}\oplus M} \to  (\cl{Y}^{\Delta^{n}})_{f(\xi_)}^{\F_{p}\oplus M},\]
    which induces the equivalence of spectra
    \[ T^{M}_{(\cl{X}^{\Delta^{n}})_{\xi}} \to  T^{M}_{(\cl{Y}^{\Delta^{n}})_{f(\xi)}}\]
      as claimed.
  \end{proof}

  \begin{corollary}\label{obliftsp}
    Let $X(\blank)=(\rm{cCAlg}^{\rm{cn}}_{\blank})^{\Delta^{0}}$ and $A \in X(\F_{p})$ such that
    $T_{X_{A}}\simeq 0$, then the space of lifts of $A$ to a $p$-complete $\S_{p}^{\wedge}$-coalgebra
    is contractible.
  \end{corollary}

  \begin{proof}
    Write $Y(\blank)= ((\rm{cCAlg}^{\rm{cn}}_{\blank})^{\wedge}_{p})^{\Delta^{0}}$. Then by Lemma~\ref{pcomparison}
    we have an equivalence $T_{X_{A}}\simeq T_{Y_{A}} \simeq 0$. Hence, we can apply Proposition~\ref{obliftzp} to
    obtain an essentially unique lift $A\p\in Y(Z_{p})$. Further applying Proposition~\ref{spherelift}
    to $A\p$ yields our claim.
  \end{proof}
  Thus, we can pointwise lift $\F_{p}$-coalgebras with vanishing tangent complex to $\S_{p}^{\wedge}$. If
  we moreover consider \textit{formally \'etale coalgebras}, we can make this lifting functorial
  in a coalgebraic analogue of the \textit{Spherical Witt Vectors} construction for
  $\bb{E}_{\infty}$-algebras over $\F_{p}$.

\begin{corollary}\label{mapliftsp}
  Let $\varphi:B\to A$ be a map of $\F_{p}$-coalgebras such that $A$ and $B$ are formally \'etale.
  Then the space of lifts of $\varphi$ to a map $\varphi\p: B\p \to A\p$ of $p$-complete
  $\S_{p}^{\wedge}$-coalgebras is contractible.
\end{corollary}

\begin{proof}
  Let $ \cl{X}(\blank)=\rm{cCAlg}_{\blank}^{\rm{cn}}$ and $\cl{Y}(\blank) =
  (\rm{cCAlg}_{\blank}^{\rm{cn}})^{\wedge}_{p}$. By Proposition~\ref{mapliftzp} the map $\varphi$ admits
  an essentially unique lift to a point $\psi \in \cl{Y}(\Z_{p})^{\Delta^{1}}$. Moreover, Lemma~\ref{pcomparison}
  yields an equivalence $T_{\cl{X}^{\Delta^{1}}_{\varphi}}\simeq T_{\cl{Y}^{\Delta^{1}}_{\varphi}}$. Since both $A$ and $B$ are
  formally \'etale Proposition~\ref{etalchar} gives equivalences
  \[ T_{\cl{X}^{\Delta^{1}}_{\varphi}} \rar{\sim} T_{\cl{X}^{\Delta^{0}}_{B}} \simeq 0\]
  Hence, we can apply Proposition~\ref{spherelift} to the functor $\cl{Y}^{\Delta^{1}}$ and the point
  $\psi \in \cl{Y}^{\Delta^{1}}$, proving the claim.
\end{proof}

\begin{theorem}\label{wittsp}
  Denote by $\cl{C}\subseteq (\rm{cCAlg}_{\S_{p}^{\wedge}}^{\rm{cn}})^{\wedge}_{p} $ the full subcategory spanned by those
  coalgebras $A$ such that $A\otimes_{\S_{p}^{\wedge}}\F_{p}$ is formally \'etale. Then the base change functor
  \[ \cl{C} \to \rm{cCAlg}_{\F_{p}}^{\rm{cn}, \rm{f\acute{e}t}} \qquad A \mapsto A \otimes_{\S_{p}^{\wedge}} \F_{p}\]
  is fully faithful and essentially surjective.
\end{theorem}
\begin{proof}
  Combine Corollary~\ref{obliftsp} and Corollary~\ref{mapliftsp}.
\end{proof}

\begin{remark}
  In the setting of Theorem~\ref{wittsp} the quasi-inverse to $\blank \otimes_{\S^{\wedge}_{p}}\F_{p}$ defines
  a fully faithful functor
  \[ W_{\S_{p}^{\wedge}}: \rm{cCAlg}_{\F_{p}}^{\rm{cn}, \rm{f\acute{e}t}}
    \to (\rm{cCAlg}_{\S_{p}^{\wedge}}^{\rm{cn}})^{\wedge}_{p}\]
  which satisfies $W_{\S_{p}^{\wedge}}(A)\otimes_{\S^{\wedge}_{p}}\F_{p} \simeq A$ for every connective, formally \'etale
  $\F_{p}$-coalgebra $A$. We call $W_{\S_{p}^{\wedge}}(A)$ the \textit{spherical Witt vectors} of $A$.
\end{remark}


\subsection{Homology coalgebras}

As observed in Example~\ref{homology}, for every space $X$ and every $\bb{E}_{\infty}$-ring $R$, the
$R$-homology $R[X]$ carries a natural $R$-coalgebra structure, which is a stronger invariant than its
underlying $R$-module. We now want to apply our results and see what can be said about the deformation
theoretic behavior of homology coalgebras. To do this, we first need to compute the cotangent complex of the
$\F_{p}$-cohomology.

\begin{definition}
  A space $X\in \cl{S}$ is called $p$-finite if the following conditions hold:
  \begin{enumerate}
    \item The space $X$ is truncated.
    \item The set $\pi_{0}X$ is finite.
    \item For each $n\geq 1$ and $x\in X$, we have that $\pi_{n}(X,x)$ is a finite $p$-group.
  \end{enumerate}
  We denote the full subcategory of $\cl{S}$ spanned by the $p$-finite spaces as $\cl{S}_{p}$ and call
 $\cl{S}^{\vee}_{p} =: \rm{Pro}(\cl{S}_{p})$ the category of $p$-\textit{profinite} spaces.
\end{definition}

\begin{remark}
We can regard $\cl{S}_{p}^{\vee}$ as the category of ``formal limits'' of $p$-finite spaces $\varprojlim X_{\alpha}$.
As such there is a functor $\cl{S}^{\vee}_{p}\to \cl{S}$ which takes a formal limit to the actual limit in $\cl{S}$.
This functor admits a left adjoint given by $Y \mapsto \flim_{Y_{\alpha} \to Y} Y_{\alpha}$, where the limit runs over all maps
from a $p$-finite space $Y_{\alpha}$ to $Y$.
\end{remark}

\begin{lemma}
  Let $X$ be a space and $\flim X_{\alpha}$ be its $p$-profinite completion. Then the natural map
  of cohomology rings
  \[ \fcolim \F_{p}^{X_{\alpha}} \to \F_{p}^{X} \]
  is an equivalence.
\end{lemma}
\begin{proof}
  This is immediate since the Eilenberg-MacLane spaces $K(\F_{p},n)$ are $p$-finite.
\end{proof}

\begin{proposition}[Mandell, Lurie]\label{coetal}
  Let $X$ be a space, then the $\F_{p}$-cohomology $\F_{p}^{X}$ is a formally \'etale $\F_{p}$-algebra.
\end{proposition}
\begin{proof}
  Since the functor $R \mapsto L_{R/\F_{p}}$ commutes with colimits, the claim follows from the fact that
  $L_{\F_{p}^{X}/\F_{p}}\simeq 0$ for every $p$-finite space $X$ which is proven
  in~\cite[][Proposition 2.4.12]{dag8}.
\end{proof}

Thus we obtain the following result about the homology coalgebra of a finite space $X$
with coefficients in a connective $\F_{p}$-algebra $R$:

\begin{corollary}\label{goal}
  Let $X$ be a finite space and $R$ be an $\F_{p}$-algebra, then $R[X]$ is a formally
  \'etale $R$-coalgebra.
\end{corollary}
\begin{proof}
  From Proposition~\ref{coetal} we get that
  \[ L_{R^{X}/R}\simeq L_{\F_{p}^{X}/\F_{p}}\otimes_{\F_{p}}R \simeq 0.\]
  Since $X$ is finite, the coalgebra $R[X]$ is dualizable with dual given by $R^{X}$, so the claim
  follows from Proposition~\ref{dualetal}.
\end{proof}

Moreover, for the case $R=\F_{p}$, we can use Theorem~\ref{wittsp} to give a partial answer to our
initial question about lifts of the coalgebra $\F_{p}[X]$.

\begin{corollary}
  Let $X$ be a finite space, then $\F_{p}[X]$ admits a unique lift to a $p$-complete $\S_{p}^{\wedge}$-coalgebra
  given by $W_{\S_{p}^{\wedge}}(\F_{p}[X]) \simeq (\S[X])^{\wedge}_{p}$. Moreover, for any other finite space $Y$
  the natural map
  \[\rm{Map}_{(\rm{cCAlg}_{\S_{p}^{\wedge}})^{\wedge}_{p}}((\S[Y])^{\wedge}_{p}, (\S[X])^{\wedge}_{p})
    \to \rm{Map}_{\rm{cCAlg}_{\F_{p}}}(\F_{p}[Y], \F_{p}[X])\]
  is a homotopy equivalence.
\end{corollary}
\begin{proof}
 Combine Corollary~\ref{goal} and Theorem~\ref{wittsp}.
\end{proof}

\section{Where to go from here}

We finish our discussion by explaining some of the shortcomings of our results and sketch a possible
way to proceed towards a coalgebraic analogue of Mandell's Theorem. The first missing puzzle piece is
the cotangent complex of a coalgebra $A$, which we have been unable to give a solid definition of.
The second and more important one is the relation to the \textit{coalgebra Frobenius}. We conjecture
that the class of \textit{perfect} coalgebras defined via this map give examples of non-dualizable
formally \'etale coalgebras. In particular, this conjecture would imply that the $\F_{p}$-homology
of \textit{any} space $X$ is formally \'etale.

\subsection{The cotangent complex of a coalgebra}
One of the first questions that arose during this project turned out to be one of the most subtle and
tricky ones, namely:

\begin{question}
  What is the cotangent complex of a coalgebra $A$?
\end{question}

Clearly, the existence of a single spectrum controlling the deformation theory of $A$ would be immensely
useful. However, it is not immediately clear what the universal property of such a spectrum should be,
i.e.~which space of derivations it should (co)represent.
Some inspiration can be gleamed from Proposition~\ref{cotangentder}. There we had seen that, for
$\varphi: B \to A$ a map of $R$-coalgebras with $A$ dualizable and $M$ an $R$-module, we have an equivalence
\[ \rm{Der}_{\varphi}(B, C_{A}(M)) \simeq \rm{Map}_{A^{\vee}}(L_{A^{\vee}/R}, \varphi^{\vee}_{\pt}\rm{map}_{R}(B, M)).\]
To get rid of the dependence on the second coalgebra $B$ one is tempted to take $B=R$ such that
$\rm{map}_{R}(B,M)\simeq M$. However, not every coalgebra $A$ admits a map $R\to A$, much less a canonical
one. The only natural choice for a map that is not the initial map would yield the following:

\begin{definition}[Preliminary 1.]
  Let $R$ be an $\bb{E}_{\infty}$-ring and $A\in \rm{cCAlg}_{R}$. The cotangent complex of $A$, if it exists,
  is the $R$-module $L_{A}$ corepresenting the functor
  \[ \rm{Mod}_{R}\to \rm{Mod}_{R} \qquad M \mapsto \rm{der}_{\id}(A, C_{A}(M))\]
\end{definition}

There are however several problems with this. Firstly, it is entirely unclear from the definition
whether $L_{A}$ vanishing would actually imply $A$ being formally \'etale. Moreover, in the dualizable
case it would lead to the rather awkward formula
\[ L_{A} \simeq L_{A^{\vee}/R}\otimes_{A^{\vee}}A.\]
Although somewhat plausible, this again gives us little information about what can actually be
deduced in the case that $L_{A}\simeq 0$.
This leaves us with several options, lest we accept that there is no good notion of one singular
cotangent complex. For one we could work with \textit{coaugmented} coalgebras, namely coalgebras
together with a map $R \to A$. For the purpose of understanding homology coalgebras this would correspond
to considering pointed spaces instead of just spaces, an entirely acceptable compromise, but beyond the
scope of this paper. \\
A different  approach would be to give up on the idea of corepresentability
and instead hope for a colimit preserving functor. For example, the functor
\[ \rm{Mod}_{R}\to \rm{Mod}_{R} \qquad M \mapsto C_{A}(M):=\rm{cofib}( A \rar{\eps} \Omega^{\infty}_{A}M).\]
seems to have no chance of preserving limits, but since colimits of coalgebras are formed underlying,
colimits are not out of the race. This leads us to the following idea:

\begin{definition}[Preliminary 2]\label{dream}
  Let $R$ be an $\bb{E}_{\infty}$-ring and $A\in \rm{cCAlg}_{R}$. We say that $A$ admits a cotangent
  complex $L_{A}:= C_{A}(R)$ if the functor $C_{A}(\blank):\rm{Mod}_{R} \to \rm{Mod}_{R}$ commutes
  with colimits. In this case we have $C_{A}(M)\simeq L_{A}\otimes M$ for every $ M \in \rm{Mod}_{R}$
\end{definition}

This definition is highly speculative, as the only coalgebras we know to admit a cotangent complex
in this sense are the formally \'etale coalgebras, for which the functor $C_{\blank}(A)$ is constant.
Conversely, if $A$ admits a cotangent complex then $L_{A}$ vanishes if and only if $A$ is formally
\'etale. Hence, the spectrum $L_{A}$ is precisely the obstruction to $A$ being formally \'etale,
which is the kind of conceptual clarity we are looking for.
While we lose any direct comparison to the cotangent complex of $A^{\vee}$ this is not entirely surprising,
since the property of being formally \'etale is defined very differently for $A^{\vee}$.
This leaves us with the following:

\begin{question}\label{cotangentdream}
  Let $R$ be an $\bb{E}_{\infty}$-ring. Does every $A \in \rm{cCAlg}_{R}$ admit a cotangent complex in the sense
  of Definition~\ref{dream}?
\end{question}

Regardless of the answer, the takeaway should be that the modules
$C_{A}(M)$ are exactly the obstruction towards $A$ being formally \'etale. Moreover, while the functor
$A\mapsto C_{A}(M)$ is very complicated, the dependence on $M$ should be relatively tame. That is,
for fixed $A$ it should be possible to describe the functor $M \mapsto C_{A}(M)$ in terms of a
formula involving $C_{A}(R)$. However, because $C_{A}(M)$ no longer has a direct relation to any
space of derivations or tangent complex, we cannot leverage results like Proposition~\ref{structure}
to obtain such a formula. We understand this as an indication that for these questions, the formalism may
have reached its limit.

\subsection{The Frobenius}
The most lacking thing about our results is the class of coalgebras that we can currently apply them to.
As of now, we are unable to give examples of formally \'etale coalgebras which are not dualizable. In
particular, we cannot describe the deformation theory of $R[X]$ for spaces $X$ which are not finite.
Attempts to reduce to the dualizable case all seem to fail for the following reason: Even though
we may write $X= \fcolim_{i}X_{i}$ where each $X_{i}$ is finite, giving the formula
$R[X]= \fcolim_{i}R[X_{i}]$, there is no reason why the functor
$\Omega^{\infty}_{\blank}(M): \rm{cCAlg}_{R}\to \rm{cCAlg}_{R}$ should commute with colimits.
Indeed, write $f_{M}:R\to R\oplus M$ for inclusion, then by definition
$\Omega^{\infty}_{\blank}(M) = f_{M,!} f^{\pt}_{M}$. The functor $f^{\pt}_{M}$ commutes with colimits,
and from Proposition~\ref{present} and the converse of the adjoint functor theorem we can deduce
that $f_{M,!}$ commutes with $\kappa$-filtered colimits for some regular cardinal $\kappa$. Thus, the class
of formally \'etale coalgebras is closed under $\kappa$-filtered colimits, but $\kappa$ is, in general, not countable.
% Closely related is the fact the notion of compactness is strangely behaved for coalgebras. For example,
% one can show that $\bb{Q}$ is not a compact object of $\rm{cCAlg}_{\Q}$, see~\cite[][Warning 1.2.15.]{ellII}.
% In particular, this means that
% \[ \rm{cSpec}(\fcolim_{i}\S[X_{i}])(\bb{Q})\neq \fcolim_{i}\rm{cSpec}(\S[X_{i}])(\Q),\]
% so we cannot deduce things about the cospectrum of infinite spaces in this way either. \\
This goes to show that the deformation theory of non-dualizable coalgebras is richer and more
interesting than that of the Ind-completion of dualizable coalgebras and requires additional input.
One contender for this additional input is the \textit{Coalgebra Frobenius} constructed by
Nikolaus:

\begin{theorem}[Nikolaus]
  Let $\cl{C} = (\rm{cCAlg}^{\rm{cn}}_{\S^{\wedge}_{p}})^{\wedge}_{p}$, then there exists a natural transformation
  $\psi_{p}:\id_{\cl{C}}\to \id_{\cl{C}}$ which on an object $A\in \cl{C}$ is given by the composition
  \[ \psi_{p}: A \rar{\Delta_{A}^{\otimes p}} (A^{\otimes p})^{hC_{p}} \rar{\rm{can}} (A^{\otimes p})^{tC_{p}} \rar{\sim} A,\]
  where the final map is the inverse of the \textit{Tate Diagonal}, see~\cite[][Theorem III.1.7]{tch}.
\end{theorem}

Given this map, we are naturally led to define \textit{perfect} coalgebras as follows:

\begin{definition}
  We say that $A \in  (\rm{cCAlg}^{\rm{cn}}_{\S^{\wedge}_{p}})^{\wedge}_{p}$ is \textit{perfect} if the coalgebra
  Frobenius $\psi_{p}: A\to A$ is a homotopy equivalence. We denote the full subcategory spanned by
  the perfect coalgebras by $(\rm{cCAlg}^{\rm{cn}}_{\S^{\wedge}_{p}})^{\wedge ,\rm{perf}}_{p} \subseteq
  (\rm{cCAlg}^{\rm{cn}}_{\S^{\wedge}_{p}})^{\wedge}_{p}$.
\end{definition}

\begin{example}\label{frobchains}
  Let $X$ be any space. Then $(\S[X])^{\wedge}_{p}$ is a perfect coalgebra since we have that
  \[\S[X]^{\wedge}_{p} \simeq (\S_{p}^{\wedge}[\colim_{X}\pt])^{\wedge}_{p} \simeq (\colim_{X} \S_{p}^{\wedge})^{\wedge}_{p}.\]
  On $\S_{p}^{\wedge}$ the map $\psi_{p}$ is necessarily given by the identity, because $\S_{p}^{\wedge}$
  is the terminal $p$-complete $\S_{p}^{\wedge}$-coalgebra. Thus, by naturality $\psi_{p}$ is given
  by the identity on $(\S[X])^{\wedge}_{p}$ as well.
\end{example}

We conjecture that this Frobenius map is related to the deformation theory of coalgebras in a similar
way to the Algebra Frobenius, in that it provides a sufficient condition for a coalgebra to be formally
\'etale.

\begin{conjecture}\label{frobcof}
  Let $A \in (\rm{cCAlg}^{\rm{cn}}_{\S^{\wedge}_{p}})^{\wedge}_{p}$ and write $A\p= A\otimes_{\S^{\wedge}_{p}}\F_{p}$.
  Then for any $M \in \rm{Mod}_{\F_{p}}^{\rm{cn}}$, the coalgebra Frobenius $\psi_p:A\to A$ induces the zero map
  on the $R$-module  $C_{A\p}(M) = \rm{cofib}(A\p \rar{\eta_{A\p}} \Omega^{\infty}_{A}(M))$.
\end{conjecture}

\begin{corollary}
  If Conjecture~\ref{frobcof} holds, then the base change functor
  \[ (\rm{cCAlg}^{\rm{cn}}_{\S^{\wedge}_{p}})^{\wedge ,\rm{perf}}_{p} \to \rm{cCAlg}_{\F_{p}}^{\rm{cn}}
  \qquad A \mapsto A\otimes_{\S_{p}^{\wedge}}\F_{p}\]
is fully faithful and factors through the full subcategory
$\rm{cCAlg}_{\F_{p}}^{\rm{cn}, \rm{f\acute{e}t}}\subseteq \rm{cCAlg}_{\F_{p}}^{\rm{cn}}$.
\end{corollary}
\begin{proof}
  Since $\psi_{p}:A\rar{\sim} A$ is an equivalence it induces an equivalence on $A\otimes_{\S_{p}^{\wedge}}\F_{p}$ and
  thus on $C_{A\otimes_{\S_{p}^{\wedge}}\F_{p}}(M)$ as well. However, since it also induces the zero map on the latter
  we get that $C_{A\otimes_{\S_{p}^{\wedge}}\F_{p}}(M)\simeq 0$. Thus, $A\otimes_{\S_{p}^{\wedge}}\F_{p}$ is formally \'etale and the
  claim follows from Theorem~\ref{wittsp}.
\end{proof}

Combining this with Example~\ref{frobchains} would allow us to fully answer our initial question about
homology coalgebras.

\begin{corollary}\label{dream2}
  If Conjecture~\ref{frobcof} holds, then for any space $X$ the $\F_{p}$-chains $\F_{p}[X]$
  are formally \'etale. In particular $\F_{p}[X]$ admits a unique and functorial lift to a $p$-complete
  $\S_{p}^{\wedge}$-coalgebra given by $\S[X]^{\wedge}_{p}= W_{\S_{p}^{\wedge}}(\F_{p}[X])$.
\end{corollary}

The fact that Conjecture~\ref{frobcof} needs to be checked for every connective $\F_{p}$-module should
be understood as an extension of our failure to find a cotangent complex. Indeed, if $\F_{p}[X]$ admits
a cotangent complex in the sense of Definition~\ref{dream}, then to obtain Corollary~\ref{dream2} it
would suffice to show that $\psi_{p}$ induces the zero map on $C_{A\otimes_{\S_{p}^{\wedge}}\F_{p}}(\F_{p})
= L_{A\otimes_{\S_{p}^{\wedge}}\F_{p}}$. However, even for this specific module the conjecture is difficult
to attack from our present position. The problem is the tricky right adjoint
$\rm{cCAlg}_{\F_{p}\oplus \F_{p}}\to \rm{cCAlg}_{\F_{p}}$ appearing in the definition of
$C_{A\otimes_{\S^{\wedge}_{p}}\F_{p}}(\F_{p})$. Because there is no known formula for this functor, attempts to verify
the conjecture have thus far been unsuccessful in all non-trivial cases. This warrants further investigation
of the coalgebra Frobenius and Conjecture~\ref{dream2}.

 \section{Benchmarks and Evaluation}
\label{sec:eval}

We evaluate \krakenSpace to answer the following set of questions:
\begin{itemize}
\item How much improvement does partial evaluation and our implemented compiler optimizations give \kraken? %(\S \ref{sec:eval2})
\item How much faster is our purely functional f-expr language, \krakenSpace, compared to other implementations of fexprs? %(\S \ref{sec:eval1} - \ref{sec:eval2})
\item How does \kraken's performance, with its fexprs, compare to macros? %(\S \ref{sec:eval1}, \S \ref{sec:eval3})
\item How do the different partial evaluation mechanisms/optimizations in \krakenSpace contribute towards reduction in overall runtime?
%\item What does \krakenSpace do internally when we create a data structure and evaluate it for some function? (\S \ref{sec:casestudy})
\end{itemize}

\textbf{Experimental Setup}: 
We ran these experiments in a reproducible Nix environment on a NixOS install \cite{10.1145/1411203.1411255} (Kernel 6.0.0) on a laptop with 8 cores / 16 threads and 64 GB of RAM.
Our code contains the scripts and Nix Flakes needed to reproduce the exact set of dependencies to run our tests.
%The code can be found at \url{https://github.com/limvot/kraken}.

The Kraken benchmarks were run using both the Wasmtime and WAVM WebAssembly engines for most benchmarks.
The Wasmtime WebAssembly engine is one of the most popular, developed by the Bytecode Alliance itself, and uses the CraneLift code generation backend.
The WAVM WebAssembly engine is interesting for its use of LLVM, and it often produces the fastest code on benchmarks but has a higher startup time.
We eliminated the Cfold Wasmtime benchmark due to problems running out of stack space (a known property of the Cfold benchmark).

\textbf{Benchmarks}: 
To showcase the capability of Kraken, we created benchmarks that are commonly implemented in functional languages and have been used as benchmarks in other papers \cite{reinking2021perceus, 10.1145/3547646}.
The benchmarks are
\begin{itemize}
\item Fib - Calculating the nth Fibonacci number
\item RB-Tree - Inserting n items into a red-black tree, then traversing the tree to sum its values
\item Deriv - Computing a symbolic derivative of a large expression
\item Cfold - Constant-folding a large expression
\item NQueens - Placing n number of queens on the board such that no two queens are diagonal, vertical, or horizontal from each other
\end{itemize}
All benchmarks besides Fibonacci use the fexpr version of match for pattern matching in \kraken, which is equivalent to the macro version in NewLisp. We also RB-Tree using NewLisp's~\cite{mueller2018newlisp} version of fexpr match. We modified the sizes of the problems presented to the benchmark to account for the longer running times of some of the less-optimized implementations.
The code for Kraken and NewLisp is very similar, and we should note that it is very unidiomatic NewLisp.
Our goal was not to compare Kraken and NewLisp as implementation languages for Red-Black Trees, but to stress test a single reasonably complex fexpr/macro, namely pattern matching.
% \textbf{Comparison with other languages}: We evaluated \krakenSpace against a language that contains f-exprs, as well as against itself with various optimizations disabled. The only other language we could find which contains a real f-expr mechanism is NewLisp~\cite{mueller2018newlisp} and so we ported \kraken's benchmark implementation to NewLisp.

%The six state-of-the-art languages are Java 17.0.1, Swift 5.4.2, Koka 2.3.2, C++, Haskell 8.10.7, and OCaml 4.12.
%The language choices were taken directly from Perceus reference-counting paper \cite{reinking2021perceus}.
%The Fibonacci benchmark additionally tests Python 3.9.11 and Chez Scheme 9.5.4.
%Koka, Ocaml and Haskell are good comparison points as statically-typed, compiled, functional programming languages, while Chez Scheme is a good comparison point as a mature and industrial strength dynamically-typed Scheme implementation known for its performance. 
%\subsection{Basic Level Comparison}
\subsection{The Effect of Partial Evaluation on Eval Calls}

\begin{table}[h]
\caption{Number of eval calls with no partial evaluation for Fexprs}
	\begin{tabular}{||c | c c c c c ||} 
		\hline
		&Evals & Eval w1 Calls & Eval w0 Calls & Comp Dyn & Comp Dyn\\ 
        & & & & w1 Calls & w0 Calls\\ [0.5ex] 
		\hline\hline
		Cfold 5 & 10897376 & 2784275 & 879066  & 1 & 0 \\ 
		\hline
		  Deriv 2  & 11708558 & 2990090 & 946500 & 1 & 0 \\ 
        \hline
		  NQueens 7 & 13530241 & 3429161 & 1108393 & 1 & 0 \\ 
    \hline
		  Fib 30 & 119107888 & 30450112 & 10770217 & 1 & 0 \\ 
    \hline
		  RB-Tree 10 & 5032297 & 1291489 & 398104 & 1 & 0 \\ 
		\hline
	\end{tabular}
    \label{npe:calls}
 \end{table}

As mentioned before, using fexprs without partial evaluation will prelude optimization and cause a massive amount of repeated work. Table \ref{npe:calls} and Table \ref{pe:calls} show the number of calls to the \krakenSpace runtime's eval function, the number of times the runtime's eval function executed a call to an applicative with wrap\_level=1, the number of times the runtime's eval function executed a call to an operative with wrap\_level=0, the number of compiled dynamic calls to applicatives with wrap\_level=1, and the number of compiled dynamic calls to operatives with wrap\_level=0.
These are shown for \krakenSpace test cases with partial evaluation turned off and turned on. 
\begin{table}[h]
\caption{Number of eval calls in Partially Evaluated Fexprs}
	\begin{tabular}{||c | c c c c c ||} 
		\hline
		&Evals & Eval w1 Calls & Eval w0 Calls & Comp Dyn & Comp Dyn\\ 
        & & & & w1 Calls & w0 Calls\\ [0.5ex] 
		\hline\hline
		Cfold 5 & 0 & 0 & 0  & 0 & 0 \\ 
		\hline
		  Deriv 2  & 0 & 0 & 0 & 2 & 0 \\ 
        \hline
		  NQueens 7 & 0 & 0 & 0 & 0 & 0 \\ 
    \hline
		  Fib 30 & 0 & 0 & 0 & 0 & 0 \\ 
    \hline
		  RB-Tree 10 & 0 & 0 & 0 & 10 & 0 \\ 
		\hline
	\end{tabular}
    \label{pe:calls}
 \end{table}

\begin{table}[h]
\caption{Number of calls to the runtime's eval function for RB-Tree. The table shows the non-partial evaluation numbers -> partial evaluation numbers.}
	\begin{tabular}{||c | c c c c c ||} 
		\hline
		&Evals & Eval w1 Calls & Eval w0 Calls & Comp Dyn & Comp Dyn\\ 
        & & & & w1 Calls & w0 Calls\\ [0.5ex] 
		\hline\hline
		  RB-Tree 7 & 2952848 -> 0 & 757932 -> 0 & 233513 -> 0 & 1 -> 7 & 0 -> 0\\ 
        \hline
		  RB-Tree 8 & 3532131 -> 0 & 906548 -> 0 & 279379 -> 0 & 1 -> 8 & 0 -> 0\\ 
        \hline
		  RB-Tree 9 & 4278001 -> 0 & 1097965 -> 0 & 3383831 -> 0 & 1 -> 9 & 0 -> 0\\ 
		\hline
	\end{tabular}
    \label{pe:rb}
    \vspace{-4mm}
 \end{table}

Without partial evaluation, no compilation can be done because it is impossible to tell if arguments to calls will be evaluated. In all benchmarks, partial evaluation removed all calls to the runtime's eval function, resulting in a completely compiled program. Looking at RB-Tree, there are over a million calls to combiners with wrap level 1 (normal functions), and 398,000 calls to combiners with wrap level 0 (operatives replacing macros). This massive blowup in the number of calls is due to the repeated and exponential re-execution of macro-like-combiners in the definition of other macro-like-combiners, as discussed in the Introduction.

The non-partially-evaluated benchmarks show 1 compiled dynamic call to an applicative (its the first call into eval) and 0 compiled dynamic calls to operatives, because there is no compilation at all. For the partially evaluated benchmarks, there are a few compiled dynamic calls to applicatives due to higher-order function use in the benchmarks, and there are no compiled dynamic calls to operatives, as all operative use has been eliminated.
We also varied the inputs for RB-Tree shown in Table \ref{pe:rb} to give a sense for how the number scale with respect to input size.

The incredible slowdown implied by these tables comes to full fruition in our RB-Tree test in Fig.~\ref{fig:kraken_nqueens_rbtree}.
We kept this run shorter because Kraken's non-partial-evaluating interpreter takes an incredibly long time even for 100 insertions (40 minutes).
The compounding layers of repeated macro-like operative calls in the non-partially-evaluated Kraken version cause a ~70,000x slowdown relative to the partial evaluated, optimized, and compiled version.
For the remaining benchmarks, we remove the naive interpreted \krakenSpace version, as in each case its performance is so bad as to blow out the graph and make it impossible to do any comparison.
In our optimized Kraken, our partial evaluation algorithm is able to fully collapse these levels of inefficiency, evaluate and inline the results, and give the backend more specialized code to optimize, emitting a compiled version that handily beats not only the NewLisp-fexpr implementation but even the NewLisp-macro implementation, as can be seen in Fig.~\ref{fig:kraken_vs_world_fib}.
We kept the benchmark sizes small in this test because the stack limits of NewLisp prevent sizes larger then ~880, while the Tail Call Elimination performed by the \krakenSpace compiler allows us to run much larger benchmarks, including the run of 4,800,000 inserts to the RB-Tree.
This result shows the dramatic effect of partial evaluation and compiler optimizations on runtime for \kraken. Our technique takes the performance of a fully fexpr based language from being completely infeasible to being faster than a macro-based dynamic scripting language currently in use.
% \begin{center}
% \begin{table}[ht]
% \caption{Number of call to the runtime's eval function for Fib. The table shows the non-partial evaluation numbers -> partial evaluation numbers}
% 	\begin{tabular}{||c | c c c c c ||} 
% 		\hline
% 		&Evals & Eval w1 Calls & Eval w0 Calls & Comp Dyn w1 Calls & Comp Dyn w0 Calls\\ [0.5ex] 
% 		\hline\hline
% 		Fib 10 & 8468 -> 0 & 2167 -> 0  & 777 -> 0 & 1 -> 0 & 0 -> 0 \\ 
% 		\hline
% 		  Fib 15  & 87916 -> 0 & 22478 -> 0 & 7961 -> 0 & 1 -> 0 & 0 -> 0 \\ 
%         \hline
% 		  Fib 20 & 969010 -> 0 & 247731 -> 0 & 87633 -> 0 & 1 -> 0 & 0 -> 0 \\ 
%     \hline
% 		  Fib 25 & 10740492 -> 0 & 2745825 -> 0  & 971209 -> 0 & 1 -> 0 & 0 -> 0 \\ 
% 		\hline
% 	\end{tabular}
%     \label{pe:fib}
%  \end{table}
% \end{center}

\begin{figure}[h]
\caption{Constant Fold and Deriv}
\includegraphics[width=0.45\textwidth]{cfold_table.csv_}
\includegraphics[width=0.45\textwidth]{deriv_table.csv_}
\label{fig:kraken_const_deriv}
\vspace{-6mm}
\end{figure}
\subsection{Comparison between Kraken Versions}
Beyond the massive speedup from partial-evaluation, Fig. \ref{fig:kraken_const_deriv} and \ref{fig:kraken_nqueens_rbtree} show the effect of the various compiler optimizations we described by disabling them one by one.
 Our main four optimizations have a strong positive effect on runtime, with the exception of lazy environment instantiation. Lazy environment instantiation helps massively on fib, and some on Deriv, but generally hurts the rest slightly.


\begin{figure}[h]
\caption{N-Queens}
\includegraphics[width=0.45\textwidth]{nqueens_table.csv_}
\includegraphics[width=0.45\textwidth]{slow_rbtree_table.csv_}
\label{fig:kraken_nqueens_rbtree}
\vspace{-4mm}
\end{figure}


\subsection{Comparison against Others}


To give a general idea of our current performance, we also show a Fibonacci benchmark that mostly exercises pure function-call speed and inlining as seen in Fig. ~\ref{fig:kraken_vs_world_fib}.
We include Python and Chez Scheme to give a general idea for where an exemplar slow and an exemplar fast dynamic language would fall.
With the benefit of our partial evaluation, compilation, and leaning upon mature WebAssembly implementations, we beat both, but this should be taken with a grain of salt, as this is a very limited micro-benchmark only meant to give a general sense of the order of magnitude of our performance.



\label{sec:eval1}
\begin{figure}[h]
\caption{Kraken vs. Others. Ordered by fastest to slowest}
\includegraphics[width=0.45\textwidth]{fib_table.csv_}
\includegraphics[width=0.45\textwidth]{rbtree_table.csv_}
\label{fig:kraken_vs_world_fib}
\end{figure}

%\label{sec:eval_nqueens}
%\begin{figure}[h]
%\caption{N-Queens}
%\includegraphics[width=0.45\textwidth]{nqueens_table.csv_}
%\includegraphics[width=0.45\textwidth]{slow_nqueens_table.csv_}
%\label{fig:kraken_nqueens}
%\end{figure}

%\label{sec:eval_nqueens}
%\begin{figure}[h]
%\caption{Kraken, N-Queens, absolute value and log-scale}
%\includegraphics[width=0.45\textwidth]{nqueens_table.csv_}
%\includegraphics[width=0.45\textwidth]{nqueens_table.csv_log}
%\label{fig:kraken_nqueens}
%\end{figure}
%\label{sec:eval_nqueensp}
%\begin{figure}[h]
%\caption{Kraken, N-Queens, absolute value and log-scale}
%\includegraphics[width=0.45\textwidth]{slow_nqueens_table.csv_}
%\includegraphics[width=0.45\textwidth]{slow_nqueens_table.csv_log}
%\label{fig:kraken_nqueensp}
%\end{figure}

%\label{sec:eval_cfold}
%\begin{figure}[h]
%\caption{C-Fold}
%\includegraphics[width=0.45\textwidth]{cfold_table.csv_}
%\includegraphics[width=0.45\textwidth]{slow_cfold_table.csv_}
%\label{fig:kraken_cfold}
%\end{figure}
%\label{sec:eval_cfold}
%\begin{figure}[h]
%\caption{Kraken, C-Fold, absolute value and log-scale}
%\includegraphics[width=0.45\textwidth]{cfold_table.csv_}
%\includegraphics[width=0.45\textwidth]{cfold_table.csv_log}
%\label{fig:kraken_cfold}
%\end{figure}
%\label{sec:eval_cfoldp}
%\begin{figure}[h]
%\caption{Kraken, C-Fold, absolute value and log-scale}
%\includegraphics[width=0.45\textwidth]{slow_cfold_table.csv_}
%\includegraphics[width=0.45\textwidth]{slow_cfold_table.csv_log}
%\label{fig:kraken_cfoldp}
%\end{figure}

%\label{sec:eval_deriv}
%\begin{figure}[h]
%\caption{Deriv}
%\includegraphics[width=0.45\textwidth]{deriv_table.csv_}
%\includegraphics[width=0.45\textwidth]{slow_deriv_table.csv_}
%\label{fig:kraken_deriv}
%\end{figure}
%\label{sec:eval_deriv}
%\begin{figure}[h]
%\caption{Kraken, Deriv, absolute value and log-scale}
%\includegraphics[width=0.45\textwidth]{deriv_table.csv_}
%\includegraphics[width=0.45\textwidth]{deriv_table.csv_log}
%\label{fig:kraken_deriv}
%\end{figure}
%\label{sec:eval_derivp}
%\begin{figure}[h]
%\caption{Kraken, Deriv, absolute value and log-scale}
%\includegraphics[width=0.45\textwidth]{slow_deriv_table.csv_}
%\includegraphics[width=0.45\textwidth]{slow_deriv_table.csv_log}
%\label{fig:kraken_derivp}
%\end{figure}

%\subsection{Comparison against state-of-the-art languages}
%\label{sec:eval3}

%\begin{figure}[h]
%\caption{Kraken vs. S.o.t.A.}
%\includegraphics[width=0.45\textwidth]{cfold_table.csv_}
%\includegraphics[width=0.45\textwidth]{rbtree_table.csv_}
%\label{fig:kraken_vs_world1}
%\end{figure}

%\begin{figure}[h]
%\caption{Kraken vs. S.o.t.A.}
%\includegraphics[width=0.45\textwidth]{deriv_table.csv_}
%\includegraphics[width=0.45\textwidth]{nqueens_table.csv_}
%\label{fig:kraken_vs_world2}
%\end{figure}

% \begin{figure}[h]
% \caption{Kraken vs. S.o.t.A. (Log)}
% \includegraphics[width=0.45\textwidth]{cfold_table.csv_log}
% \includegraphics[width=0.45\textwidth]{rbtree_table.csv_log}
% \label{fig:kraken_vs_world_log_1}
% \end{figure}
% \begin{figure}[h]
% \caption{Kraken vs. S.o.t.A. (Log)}
% \includegraphics[width=0.45\textwidth]{deriv_table.csv_log}
% \includegraphics[width=0.45\textwidth]{nqueens_table.csv_log}
% \label{fig:kraken_vs_world_log_2}
% \end{figure}

%As we noted before with the Fib(30) microbenchmark in Section \ref{sec:eval1}, we remain significantly slower than state-of-the-art compiled languages.
%This is particularly true for memory-intensive benchmarks due to our naive reference-counting and malloc/free implementations.
%However, our results are of a similar order of magnitude to the difference between the state-of-the-art compiled languages and dynamic scripting languages, like Python's results in the Fib(30) microbenchmark.
%We assert that is not a fundamental limitation because the classic f-expr slowness is being eliminated, as shown by Fig. \ref{fig:kraken_vs_newlisp1} and Fig. \ref{fig:kraken_vs_newlisp2}.
%In future work, we plan to expand our compile-time analysis and optimization to implement a modified, dynamic-language version of Perceus reference counting.
%With this change, we belive \krakenSpace can be competitive with these state-of-the-art languages.

%\subsection{Case Study: Red-Black Tree}
%\label{sec:casestudy}

%\begin{figure}[h]
%\caption{Kraken vs. S.o.t.A. - RB-Tree Focus}
%\includegraphics[width=0.4\textwidth]{rbtree_table.csv_}
%\includegraphics[width=0.4\textwidth]{rbtree_table.csv_log}
%\label{fig:kraken_vs_world_rbtree}
%\end{figure}


%To evaluate our partial evaluation algorithm and compiler, we extracted the benchmarks used by the Koka language project from their code repository and added Kraken versions, as well as implementing a naive Fibonacci microbenchmark ourselves to evaluate pure function call speed.\\
%With partial evaluation and the compiler optimizations listed above, we get fairly strong performance on purely numerical computations, such as the naive Fibonacci microbenchmark.
%Unfortunately, the overhead of our unsophisticated reference counting, dynamic type checking, and bounds checking causes poor performance on benchmarks involving data structures relative to mainstream programming language implementations.
%This is not a fundamental limitation, and will be addressed in future work, as recounted in the next section.
%It should be noted, however, that while the performance relative to established language implementations is very poor for the memory-intensive benchmarks (600-900x slower), we still realize a massive speedup compared to an unoptimized and non-partial-evaluated f-expr implementation (100,000x faster)!

We provide some comments on the growth conditions which constituted the majority of our analysis in sections \ref{sec:Hmixing} and \ref{sec:Hsigma}. In the simplest cases of Lemma \ref{lemma:unstableGrowth}, growth was established in an analogous fashion to the old one-step expansion condition (\ref{eq:oldOneStepExpansion}), finding the relevant Jacobians $M_j$ and checking that their expansion factors $K(M_j)$ satisfy
\begin{equation}
    \label{eq:discussionOneStep}
    \sum_j \frac{1}{K(M_j)} <1.
\end{equation}
For the more complicated cases, the inductive method used to establish growth near the accumulation points in Lemma \ref{lemma:unstableGrowth} and the weakened one-step expansion condition (\ref{eq:oneStep}) both address the same fundamental issue: the splitting of unstable curves by singularities into an unbounded number of small components. They circumvent this obstacle in rather different ways, however. While (\ref{eq:oneStep}) generalises (\ref{eq:discussionOneStep}) to ensure an growth of unstable curves `on average' (see \cite{chernov_statistical_2009} for a precise statement), our inductive method is a more direct adaptation of (\ref{eq:discussionOneStep}), using it to generate contradictory geometric conditions which a hypothetical non-growing unstable curve must satisfy. It may be possible to prove Theorem \ref{sec:Hmixing} using (\ref{eq:oneStep}) as the basis for growth. Since we required (\ref{eq:oneStep}) anyway for proving Theorem \ref{thm:HsigmaExp}, this could potentially condense our analysis, but only to a minor extent. A convenience of the method used in section \ref{sec:Hmixing} is that, by way of the `simple intersection' property, it naturally gives geometric information on the images of manifolds, useful for proving the property \textbf{(M)} of Theorem \ref{thm:katok-strelcyn}.

We expect that essentially analogous analysis can be applied to establish mixing properties in a wide class of piecewise linear non-uniformly hyperbolic maps, including those (like the OTM) which sit on the boundary of ergodicity and beyond. While we have relied on the precise partition structure of $H_\sigma$, its fundamental feature (self-similar sequences of elements $A^k$, sharing boundaries with its neighbours $A^{k-1},A^{k+1}$ and accumulating onto some point $p$) is quite typical to return map systems. See, for example, those of various stadium billiards \cite{chernov_chaotic_2006,chernov_improved_2008,chernov_statistical_2009} and LTMs \cite{springham_polynomial_2014}. Indeed, the same method can be used to prove the Bernoulli property for non-monotonic LTMs \cite{myers_hill_mixing_2022}, where monotonicity of the manifold images cannot be assumed and the classical argument \cite{sturman_mathematical_2006} fails. The OTM is the pointwise limit of these maps as the boundary shrinks to null measure. It further has utility in proving growth conditions for maps which are uniformly hyperbolic but possess regions $A_j$ where the hyperbolicity is very weak, signified by $K(M_j) \approx 1$, so that (\ref{eq:discussionOneStep}) fails. Typically this leads to suboptimal bounds on mixing windows, see e.g. \cite{wojtkowski_model_1981,przytycki_ergodicity_1983,myers_hill_family_2022}. The map $H_{(\eta,\eta)}$ for $\eta \approx 1/2$ is another example, possessing weak hyperbolicity over $A_2, A_3$. Letting $\varepsilon = |\eta-1/2|>0$, there is an upper bound $N = N(\varepsilon)$ on escape times from the intersections $A_2\cap \sigma, A_3 \cap \sigma$. The growth lemma then follows by applying the inductive step roughly $N$ times and can be established for arbitrarily small $\varepsilon$, opening the door to establishing optimal mixing windows.

The above gives two examples of piecewise linear perturbations to $H$ where mixing with respect to Lebesgue is preserved and our methods can be applied. Nonlinear perturbations to the shear profiles complicate the analysis in several ways. Firstly as the map's Jacobians takes on a broader range of values, cone invariance becomes an increasingly harder condition to establish. Cones must be widened, giving looser bounds on expansion factors, which may already be weak due to new regions of weaker stretching. This, together with the change from polygonal to curvilinear return time partition elements and nonlinear local manifolds, adds some complexity to showing growth conditions. This does not rule out certain (small) nonlinear perturbations however. There is some leeway in the inequalities which govern cone invariance and growth of local manifolds, the latter of which is not too dissimilar from the piecewise linear setting (see Lemmas \ref{lemma:piecewiseApprox}, \ref{lemma:componentLength}). Certain small perturbations would not alter the \emph{topological} structure of the return time partition, i.e. which elements share boundaries, the key information needed for setting up the induction. Finally while the partition elements would no longer be polygonal, only coarse geometric information is required for verifying each inductive step. Following the above, a potential perturbation could be to replace the linear portions of each shear by a cubic, perturbing the tent profile
\[  f(t) = \begin{cases} 2t & 0 \leq t \leq 1/2, \\ 2(1-t) & 1/2 \leq t \leq 1 ,\end{cases} \]
of the OTM shears to
\[  f_a(t) = \begin{cases} \frac{1}{8} t \left(16 - a + 6at - 8at^{2} \right) & 0 \leq t \leq 1/2, \\ \frac{1}{8}\left(1-t\right)\left( 16 - a + 6a\left(1-t\right) - 8a\left(1-t\right)^{2}\right)  & 1/2 \leq t \leq 1, \end{cases}   \]
for $a>0$. For small enough $a$ the gradient range $f'(t)$ is restricted to small neighbourhoods of $\{ 2, -2\}$ and the escape time partition retains a similar structure. We illustrate this in Figure \ref{fig:perturbations}, showing escapes from the square $S_3$ under the map $G \circ F$, equivalent to escapes from the perturbed $A_3$ under the $G \circ F$, but with a cleaner geometry for comparison. When $a$ is too large the analogy to the OTM breaks down. At $a=16$ the map is twice differentiable everywhere and features a new source of slowed mixing, the Jacobian is the identity at the corner points $x,y \in \{  0, 1/2 \}$ giving locally parabolic behaviour (visible in the escape time partition). 

\begin{figure}
    \centering
    \includegraphics[width=0.24 \linewidth]{0.png}
    \includegraphics[width=0.24 \linewidth]{4.png}
    \includegraphics[width=0.24 \linewidth]{8.png}
    \includegraphics[width=0.24 \linewidth]{16.png}
    \caption{Partition of escape times from $S_3$ under the mapping $F \circ G$ for $a= 0,4,8,16$. }
    \label{fig:perturbations}
\end{figure}


\end{document}
