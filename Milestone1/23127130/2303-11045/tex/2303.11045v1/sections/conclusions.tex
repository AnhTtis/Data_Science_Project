\section{Conclusions and Open Challenges}
\label{sect:conclusions}

One of the major ongoing challenges in the field of blockchain is making Byzantine consensus applicable to large-scale environments. To address this challenge, a large body of research has focused on developing novel techniques to improve the scalability of BFT consensus, paving the way for a new generation of BFT protocols tailored to the needs of blockchain. 
In this SoK paper, we employed a systematic literature search to explore the design space of recent BFT protocols along with their ideas for scaling up to hundreds or thousands of nodes. 
We created a taxonomy of scalability-enhancing techniques, which categorizes these ideas into communication and coordination strategies, pipelining, cryptographic primitives, independent groups, committee selection, and trusted hardware support. 
As shown in \cref{tab:protocols}, many BFT protocols employ not only one idea but rather a combination of several ideas. We also see that a less vigorously explored research field seems to be the incorporation of trusted execution environments, which is inviting for future research works.
Further, we comprehensively discussed all ideas on an abstract level and pinpointed the design space from which their corresponding BFT protocols originated.

Some open challenges regarding BFT scalability remain:
%
% (1) comparing the effectiveness of techniques and combinations;
% (2) finding unidentified combinations of techniques;
% (3) evaluating the complexity of techniques;
% (4?) selecting a protocol according to application requirements.
%
While we have identified and categorized these scalability techniques, we cannot compare their \emph{effectiveness} solely on the basis of the papers' evaluation results, and a common evaluation platform for these protocols is not yet available~\cite{amiri2022bedrock}.
%
\cref{tab:protocols} shows a wide range of combinations of techniques that have so far been applied; however, \emph{further combinations} may exist that lead to valid, performant, and highly scalable protocols and which have to be identified.
%
Not only the techniques' effectiveness is important, but their \emph{complexity} regarding computational resource requirements, implementation effort, or proof of correctness in a protocol is also relevant as well and may differ widely.
%
Finally, depending on the specific \textit{application requirements}, different agreement protocols may be more suitable for specific deployment settings than others. 
As the BFT protocol landscape is extensive, developing a guideline for selecting the most fitting protocol according to these requirements may be helpful.