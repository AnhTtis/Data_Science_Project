\section{Related Work}
\label{sect:related_work}
% Search keywords "\{consensus blockchain scalability survey\}" in Google scholar, reviewing the first 200 results.
Several surveys systematically analyze blockchain protocols; some contain a more or less detailed treatment of scalability. 

Alsunaidi et al. created a survey of blockchain consensus algorithms, focusing on performance and security~\cite{Alsunaidi19Consensus}. The authors distinguish between \textit{proof-based} (\eg PoW) and \textit{voting-based} (\eg BFT). Scalability is a mentioned challenge, but is not analyzed beyond categorizing proof-based protocols as \enquote{strongly} and vote-based protocols as \enquote{weakly} scalable.

Bano et al. presented an SoK about blockchain consensus protocols~\cite{bano19sok}, where the main contribution is a systematization framework that tracks the chronological evolution of blockchain consensus protocols and a categorization using this framework. While the framework explains that consensus scalability can be achieved by advancing from \textit{hybrid single committee consensus} to \textit{hybrid multiple committee consensus} (\eg through sharding), it does not treat scalability mechanics for consensus in general, \eg for consensus within a single committee.

Berger et al. created a short survey in 2018 that broadly analyzes scalability techniques used in BFT consensus protocols~\cite{berger18scaling}. Possibly, our survey can be best understood as progressing this effort by (1) using a systematic methodology, (2) increasing the level of detail, and (3) applying the analysis to contemporary research works, thus extending the scope by many papers that have been published just recently.

%\sketch{Berrang, Pascal "Survey of Consensus Protocols and Scalability Solutions" to be considered? it's only a draft paper ..}
%Eklund et. al. analyze factors that impact the performance and scalability of blockchain networks

Ferdous et al.'s survey on block\-chain consensus introduced a taxonomy of desirable properties~\cite{ferdous2020blockchain}. While scalability is one such property, the technical aspects are not discussed.

% section H - L
Hafid et al. analyze the scalability of blockchain platforms with a focus on first and second-layer solutions~\cite{hafid2020scaling}, \ie changes to the blockchain, \eg the block structure using DAGs or increasing block sizes, and mechanisms implemented outside of it, \eg side-chains, child-chains, or payment channels.
The authors propose a taxonomy based on committee formation and consensus within a committee and compare sharding-based protocols.
Scalability outside of sharding is not considered.

Huang et al. analyze blockchain surveys and focus on theoretical modeling, analysis models, performance measurements, and experiment tools~\cite{huang2021survey}.
Scalability is only considered with regard to sharding and multi-chain interoperability.

Jennath et al. give a general overview of common blockchain consensus protocols, such as \ac{PoW}, \ac{PoS}, \ac{PoET}, \ac{BFT}, and Federated Byzantine agreements~\cite{jennath2020survey}.
%The survey does not consider the scalability of BFT protocols.
%
Lao et al. consider IoT blockchains and their consensus strategies~\cite{lao2020survey}, for which they compare consensus protocols using similar categories. %as Jennath et al.~\cite{jennath2020survey}, and identify the need for a lightweight, highly energy-efficient consensus.
Neither survey considers BFT scalability.

Liu et al. analyze recent blockchain techniques and claim that consensus-based scaling is limited, especially with Moore's law nearing its end~\cite{liu2020effective}.
They discuss and evaluate scaling concerning topology and hardware assistance, \eg off-chain or parallel-chain computations or sharding.

% section M - V
Meneghetti et al. presented a survey on blockchain scalability~\cite{meneghetti2019survey}; however, they focus more on smart contract executions, particularly sharding, than on the consensus mechanism.

Monrat et al.'s survey on blockchain applications, challenges, and opportunities~\cite{monrat2019survey} provides a blockchain taxonomy and describes potential applications.
The authors also describe common consensus algorithms but do not focus on scalability. 

Salimitari and Chatterjee created a survey on blockchain consensus protocols in IoT~\cite{salimitari2018survey}. They evaluate various blockchain consensus protocols for use in IoT scenarios. The survey categorizes consensus protocols in rough categories such as \ac{PoW}, \ac{PoS}, \ac{BFT}, VRF-based, and sharding-based solutions.

Vukolić contrasts \ac{PoW}-based algorithms to BFT SMR protocols~\cite{vukolic2015quest}. Vukolić identifies scalability to many consensus nodes as a blocker for the adoption of blockchain consensus. As of 2016, Vukolić identifies optimistic BFT protocols and relaxed fault models such as XFT or hybrid fault models with trusted hardware as potential solutions.