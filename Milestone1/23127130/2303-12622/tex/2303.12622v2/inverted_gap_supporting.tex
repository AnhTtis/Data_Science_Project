\documentclass[%
 preprint,
superscriptaddress,
 amsmath,amssymb,
 aps,prl]{revtex4-2}


\usepackage{xurl}
\usepackage[hidelinks,breaklinks]{hyperref}

  
\usepackage[all]{hypcap} 
\usepackage{graphicx}
\usepackage{braket}
\usepackage{amsmath}
\usepackage{amsfonts}

\renewcommand{\theequation}{S\arabic{equation}}
\renewcommand{\thefigure}{S\arabic{figure}}
\renewcommand{\thetable}{S\arabic{table}}
\renewcommand{\bibnumfmt}[1]{[S#1]}
\renewcommand{\citenumfont}[1]{S#1}
\setcounter{section}{0}
\renewcommand{\thesection}{S\arabic{section}}


\title{Supporting Information: Stabilizing the inverted phase of a WSe$_2$/BLG/WSe$_2$ heterostructure via hydrostatic pressure}

\begin{document}
\author{M\'at\'e Kedves}
\affiliation{Department of Physics, Institute of Physics, Budapest University of Technology and Economics, M\H uegyetem rkp.\ 3., H-1111 Budapest, Hungary}
\affiliation{MTA-BME Correlated van der Waals Structures Momentum Research Group, M\H uegyetem rkp.\ 3., H-1111 Budapest, Hungary}

\author{B\'alint Szentp\'eteri}
\affiliation{Department of Physics, Institute of Physics, Budapest University of Technology and Economics, M\H uegyetem rkp.\ 3., H-1111 Budapest, Hungary}
\affiliation{MTA-BME Correlated van der Waals Structures Momentum Research Group, M\H uegyetem rkp.\ 3., H-1111 Budapest, Hungary}

\author{Albin M\'arffy}
\affiliation{Department of Physics, Institute of Physics, Budapest University of Technology and Economics, M\H uegyetem rkp.\ 3., H-1111 Budapest, Hungary}
\affiliation{MTA-BME Superconducting Nanoelectronics Momentum Research Group, M\H uegyetem rkp.\ 3., H-1111 Budapest, Hungary}

\author{Endre T\'ov\'ari}
\affiliation{Department of Physics, Institute of Physics, Budapest University of Technology and Economics, M\H uegyetem rkp.\ 3., H-1111 Budapest, Hungary}
\affiliation{MTA-BME Correlated van der Waals Structures Momentum Research Group, M\H uegyetem rkp.\ 3., H-1111 Budapest, Hungary}

\author{Nikos Papadopoulos}
\affiliation{QuTech and Kavli Institute of Nanoscience, Delft University of Technology, 2600 GA Delft, The Netherlands}

\author{Prasanna K. Rout}
\affiliation{QuTech and Kavli Institute of Nanoscience, Delft University of Technology, 2600 GA Delft, The Netherlands}

\author{Kenji Watanabe}
\affiliation{Research Center for Functional Materials, National Institute for Materials Science, 1-1 Namiki, Tsukuba 305-0044, Japan}

\author{Takashi Taniguchi}
\affiliation{International Center for Materials Nanoarchitectonics,
National Institute for Materials Science, 1-1 Namiki, Tsukuba 305-0044, Japan}

\author{Srijit Goswami}
\affiliation{QuTech and Kavli Institute of Nanoscience, Delft University of Technology, 2600 GA Delft, The Netherlands}

\author{Szabolcs Csonka}
\affiliation{Department of Physics, Institute of Physics, Budapest University of Technology and Economics, M\H uegyetem rkp.\ 3., H-1111 Budapest, Hungary}
\affiliation{MTA-BME Superconducting Nanoelectronics Momentum Research Group, M\H uegyetem rkp.\ 3., H-1111 Budapest, Hungary}

\author{P\'eter Makk}
\email{makk.peter@ttk.bme.hu}
\affiliation{Department of Physics, Institute of Physics, Budapest University of Technology and Economics, M\H uegyetem rkp.\ 3., H-1111 Budapest, Hungary}
\affiliation{MTA-BME Correlated van der Waals Structures Momentum Research Group, M\H uegyetem rkp.\ 3., H-1111 Budapest, Hungary}

\maketitle

\section{\label{sec:device}Device geometry and measurement setup}

The measured sample is shown in Fig.\,\ref{fig:device}. The dry-transfer technique with PC/PDMS hemispheres is employed to stack hBN (35 nm)/WSe$_2$ (19 nm)/BLG/WSe$_2$ (19 nm)/hBN (60 nm)/graphite. To fabricate electrical contacts to the Hall bar, we use e-beam lithography patterning followed by a reactive ion etching step using CHF$_3$/O$_2$ mixture and finally deposit Ti (5nm)/NbTiN (100 nm) by dc sputtering. We deposit Al$_2$O$_3$ (30 nm) using ALD which acts as the gate dielectric. Finally, the top gate is defined by e-beam lithography and deposition of Ti (5 nm)/Au (100 nm). We note that the heterostructure was not etched into a Hall-bar shape after the contacts were deposited. Therefore current can flow along the pristine edges of the BLG layer. For the quantum Hall measurements, this results in a nontrivial sample geometry which could result in the mixing of longitudinal and transverse resistances. During the fabrication process, the alignment of WSe$_2$ layers was not controlled.

Transport measurements were carried out in an Oxford cryostat equipped with a variable temperature insert (VTI) at a base temperature of 1.4 K (unless otherwise stated). Measurements were performed using lock-in technique with 0.1 mV AC voltage excitation at 1171 Hz frequency. Measurements presented in the main text were conducted on device S1: the AC voltage bias was applied between contacts A and D while the four-terminal voltages were measured between B and C.

\begin{figure}
\includegraphics[width=0.8\columnwidth]{fig_S1.pdf}
\caption{\label{fig:device} (a) Optical image of the measure device. The scale bar is 4 $\mu$m. (b) Schematic illustration of the heterostructure, showing the hopping terms in the BLG and the layer dependent SOC terms, induced by the WSe$_2$ layers.}
\end{figure}

To apply hydrostatic pressure, the sample is first bonded to a high pressure sample holder and placed in a piston-cylinder pressure cell as detailed in Ref.\,\cite{Fueloep2021a} where kerosene acts as the pressure mediating medium. To change the applied pressure the sample is warmed up to room temperature where the pressure is applied using a hydraulic press and the sample is cooled down
again.

\section{\label{sec:theory}Continuum model of WSe2/BLG/WSe2}
\subsection{Low-energy Hamiltonian}

We model the studied heterostructure with the low-energy Hamiltonian of bilayer graphene with an additional spin--orbit coupling term, which is different on the two graphene layers, induced by the proximity of the two WSe$_2$ layers. In the basis of the 4 atom unit cell, 
$\left(\ket{C_\mathrm{A1}}, \ket{C_\mathrm{B1}}, \ket{C_\mathrm{A2}}, \ket{C_\mathrm{B2}}\right)\otimes \left(\ket{\uparrow},\ket{\downarrow}\right)$ the Hamiltonian is written as 
\begin{eqnarray}
\mathcal{H}&=&\mathcal{H}_\mathrm{BLG}+\mathcal{H}_\mathrm{SOC},\\
\mathcal{H}_\mathrm{BLG}&=&\left(\begin{array}{cccc}
u/2&v_0\pi^\dagger&-v_4\pi^\dagger&v_3\pi\\
v_0\pi&u/2+\Delta' &\gamma_1&-v_4\pi^\dagger\\
-v_4\pi&\gamma_1&-u/2+\Delta'&v_0\pi^\dagger\\
v_3\pi^\dagger&-v_4\pi&v_0\pi&-u/2
\end{array}
\right)\otimes s_0,
\label{H_BLG}\\
\mathcal{H}_\mathrm{SOC}&=&\left(\begin{array}{cccc}
\xi\lambda_I^{b} s_z/2 & i\lambda_R^{b} s_-^\xi&0&0\\
-i\lambda_R^{b} s_+^\xi&\xi\lambda_I^{b} s_z /2&0&0\\
0&0&\xi\lambda_I^{t} s_z/2 & i\lambda_R^{t} s_-^\xi\\
0&0&-i\lambda_R^{t} s_+^\xi&\xi\lambda_I^{t} s_z /2
\end{array}
\right),
\end{eqnarray}
where $\mathcal{H}_\mathrm{BLG}$ is the Hamiltonian of the BLG\cite{McCann2013} and $\mathcal{H}_\mathrm{SOC}$ is the spin--orbit coupling term describing the proximity induced Ising-type SOC with the parameters of $\lambda_I^i$ and Rashba-type SOC parametrized with $\lambda_R^i$ \cite{Konschuh2012,Zollner2021}. Here, $s_i$, with $i=\{0,x,y,z\}$, are the spin Pauli matrices and $s_\pm^\xi=\frac{1}{2}(s_x+i\xi s_y)$. In $\mathcal{H}_\mathrm{BLG}$, $\gamma_i$, with $i=\{0,1,3,4\}$ describe the intra- and interlayer hoppings in BLG, as illustrated in Fig.~\ref{fig:device}.b, $v_i=\sqrt{3}a\gamma_i/2\hbar$ are effective velocities, with the lattice constant of the graphene $a=2.46$\,\AA\ and $\Delta'$ is the dimer on-site energy. $\gamma_0$ is the nearest neighbour intralayer hopping, $\gamma_1$ is the interlayer hopping between the dimer sites, $\gamma_3$ describes the hopping between the non-dimer sites and $\gamma_4$ is the interlayer hopping between the dimer and non-dimer orbitals. In 
$\mathcal{H}$, $\pi=\hbar(\xi k_x+ik_y)$ and $\pi^\dagger=\hbar(\xi k_x-ik_y)$ are momentum operators measured from the K and K' valleys with the valley indices $\xi=\pm1$. The parameter $u$ is the interlayer potentials difference modelling the effect of an external electric field. 

In our simulations we have used the following parameters: $\gamma_0=2.61$\,eV, $\gamma_1=0.361$\,eV, $\gamma_3=0.283$\,eV, $\gamma_4=0.138$\,eV and $\Delta'=0.015$\,eV \cite{Jung2014}.


In the main text, we show the spectrum near the $K$ valley. Here, we show the difference between the $K$ and $K'$ valleys in Fig~\ref{supfig:K_Kp_compare}. The main difference, besides the opposite tilting due to the trigonal warping, is the opposite spin polarization of the bands, which is the manifestation of the Kramers theorem, since the valley-Zeemann terms generate an opposite effective magnetic field in the two valleys. In the figures, we calculate the spin polarization as
\begin{equation}
\zeta_n=\sum\limits_{X=A1,A2,B1,B2}|c_{X,\uparrow}|^2-|c_{X,\downarrow}|^2.
\end{equation}

\begin{figure}[!hbt]\centering
\includegraphics[width=0.8\columnwidth]{fig_S4.pdf}
\caption{{Calculated band structure using the parameters of $u=3$\,meV and $\lambda_I^{b}=-\lambda_I^{t}=2$\,meV (a) near the $K$ valley and (b) near the $K'$ valley. }}
\label{supfig:K_Kp_compare}
\end{figure}

Besides the spin polarization, the layer polarization $\alpha_n$ is also an important parameter of the model, which is defined as
\begin{equation}
\alpha_n=\sum\limits_{s=\uparrow,\downarrow}|c_{A1,s}|^2+|c_{B1,s}|^2-|c_{A2,s}|^2-|c_{B2,s}|^2.
\end{equation}
As shown in Fig.~\ref{supfig:layerpol}. at $u=0$ the bands have no layer polarization, which can be lifted by increasing $u$. For $|u|>|\lambda_I^{b}|$, the low energy part of the conductance and valence bands becomes layer polarized with the opposite layer polarization of the valence and conduction bands.
\begin{figure}[!hbt]\centering
\includegraphics[width=0.8\columnwidth]{fig_S5.pdf}
\caption{{Calculated band structure using the parameter of $\lambda_I^{b}=-\lambda_I^{t}=2$\,meV (a) at $u=-3$\,meV  and (b) at $u=0$\,meV. The color of the line corresponds to the layer polarization: the blue (red) points are fully polarized to the top (bottom) layer and the purple points are layer degenerated.}}
\label{supfig:layerpol}
\end{figure}

For completeness, the band structure near the $K$-point is shown in Fig.~\ref{supfig:sameSOC}. with $\lambda_I^{b}=\lambda_I^{t}$. In this case, the bands are spin split due to the SOC, which can be considered here as a Zeemann splitting in an effective magnetic field. Moreover, in this case, there is no gap at $u=0$ and a gap only opens if $|u|>|\lambda_I^{b,t}|$. 
\begin{figure}[!hbt]\centering
\includegraphics[width=0.8\columnwidth]{fig_S6.pdf}
\caption{{Calculated band structure using the parameter of $\lambda_I^{b}=\lambda_I^{t}=2$\,meV (a) at $u=-3$\,meV, (b) at $u=-2$\,meV  and (c) at $u=0$\,meV. The color of the lines corresponds to the spin polarization.}}
\label{supfig:sameSOC}
\end{figure}

\subsection{Landau level calculations}
In high perpendicular magnetic field $B$, we replace the canonical impulse with the kinetic momentum $\hbar q_i=\hbar k_i-eA_i$, where $A_i$ is the vector potential and introduce the magnetic ladder operators as $\hat{a}=\sqrt{\frac{\hbar}{2eB}}\left(q_x+iq_y\right)$ and $\hat{a}^\dagger=\sqrt{\frac{\hbar}{2eB}}\left(q_x-iq_y\right)$, which satisfy $[\hat{a},\hat{a}^\dagger]=1$. In Eq. \ref{H_BLG}, we replace $\pi$ and $\pi^\dagger$ with the ladder operators as $\pi=\sqrt{2eB\hbar}\hat{a}$ and $\pi^\dagger=\sqrt{2eB\hbar}\hat{a}^\dagger$ for the $K$ valley and $\pi=-\sqrt{2eB\hbar}\hat{a}^\dagger$ and $\pi^\dagger=-\sqrt{2eB\hbar}\hat{a}$ for the $K'$ valley. We neglect $\gamma_3$ in the Landau level (LL) calculations, which would introduce a small mixing of the zero LLs and the higher LLs. For the $K$ valley the $\mathcal{H}_\mathrm{BLG}$ can be rewritten as
\begin{equation}
\mathcal{H}_\mathrm{BLG}^K=\left(\begin{array}{cccc}
u&v_0\sqrt{2eB\hbar}\hat{a}^\dagger&-v_4\sqrt{2eB\hbar}\hat{a}^\dagger&0\\
v_0\sqrt{2eB\hbar}\hat{a}&u+\Delta' &\gamma_1&-v_4\sqrt{2eB\hbar}\hat{a}^\dagger\\
-v_4\sqrt{2eB\hbar}\hat{a}&\gamma_1&-u+\Delta'&v_0\sqrt{2eB\hbar}\hat{a}^\dagger\\
0&-v_4\sqrt{2eB\hbar}\hat{a}&v_0\sqrt{2eB\hbar}\hat{a}&-u
\end{array}\right)\otimes s_0,
\end{equation}
and for $K'$ valley it is given by
\begin{equation}
\mathcal{H}_\mathrm{BLG}^{K'}=\left(\begin{array}{cccc}
u&-v_0\sqrt{2eB\hbar}\hat{a}&v_4\sqrt{2eB\hbar}\hat{a}&0\\
-v_0\sqrt{2eB\hbar}\hat{a^\dagger}&u+\Delta' &\gamma_1&v_4\sqrt{2eB\hbar}\hat{a}\\
v_4\sqrt{2eB\hbar}\hat{a}^\dagger&\gamma_1&-u+\Delta'&-v_0\sqrt{2eB\hbar}\hat{a}\\
0&v_4\sqrt{2eB\hbar}\hat{a}^\dagger&-v_0\sqrt{2eB\hbar}\hat{a}^\dagger&-u
\end{array}\right)\otimes s_0.
\end{equation}


The full Hamiltonian in $B$ field is also expanded with the Zeeman term $\mathcal{H}_Z=E_Zs_z$, where $E_Z=-\mu_BB$ is the Zeeman energy with the Bohr magneton $\mu_B$. The eigenenergies $E_{\xi,n,s_z}$ of  $\mathcal{H}=\mathcal{H}_\mathrm{BLG}+\mathcal{H}_\mathrm{SOC}+\mathcal{H}_Z $ are defined as
\begin{equation}
\label{eq:LLeig}
\mathcal{H}\ket{\xi,n,\sigma}=E_{\xi,n,s_z}\ket{\xi,n,\sigma},
\end{equation}
with the eigenstates of $\ket{\xi,n,\sigma}$. The ladder operators act on the Landau level wavefunctions as $\hat{a}\ket{n}=\sqrt{n}\ket{n-1}$ and $\hat{a}^\dagger\ket{n+1}=\sqrt{n+1}\ket{n+1}$. Following the footsteps of \citep{Khoo2018}, the same ansatz can be used to solve Eq. \ref{eq:LLeig}, which was used with induced SOC only in one graphene layer.


\begin{figure}[!hbt]\centering
\includegraphics[width=1\columnwidth]{fig_S7.pdf}
\caption{{Single-particle zeroth LL spectrum as a function of $u$ at $B=10$\,T for (a) $\lambda_I^b=-\lambda_I^t=-2$\,meV, (b) $\lambda_I^b=\lambda_I^t=-0$ and (c) $\lambda_I^b=1.8$\,meV and $\lambda_I^t=-2.2$\,meV.}}
\label{supfig:LLspectrum}
\end{figure}
By solving the eigenvalue problem, the single-particle Landau level energies can be obtained. In Fig.~4a. in the main text, we plot the lowest 8 LLs with respect to $u$ at $B=8.5$\,T. These LLs would be degenerate if we set every parameter to zero except $\gamma_0$ and $\gamma_1$, including the interlayer potential. In Fig.~\ref{supfig:LLspectrum} we show the LLs in three different scenarios of $\lambda_I$: in panel (b) the LLs are shown if $\lambda_I=0$. In this case, the LLs are spin split due to the Zeeman term and also split in the orbital index due to $\gamma_0$ and $\Delta'$. A finite $u$ further splits these LLs and their energy is linear in $u$. Since for the two lowest LL the spin and layer index becomes effectively the same, different valleys shift oppositely with a displacement field. By introducing a finite Ising-type SOC $\lambda_I^b$ shifts the energies of the eigenstates of $\ket{K,n,\sigma}$ and $\lambda_I^t$ shifts the energies of the eigenstates of $\ket{K',n,\sigma}$. Comparing panel (a) and (b), if $\lambda_I^t=-\lambda_I^b$, at high magnetic field, the spectrum seems very similar to the case without SOC, however, the order of the spin up and spin down levels flip. Moreover, the positions ($u^*$) where two LLs cross also change, which we defined as $D^*$ crossings in the measurements in the main text. In the third case, shown in panel (c), when $\lambda_I^t$ and $\lambda_I^b$ have an opposite sign but their magnitudes are different, the $u$, $-u$ symmetry is lost leading to a non-zero $\pm u_3^*$ crossings.

\begin{figure}[!hbt]\centering
\includegraphics[width=\columnwidth]{fig_S9.pdf}
\caption{Landau level crossings at $\nu=0$ as a function of $B$ with (a) $\lambda_I^b=-\lambda_I^t=-5.6$\,meV, (b) $\lambda_I^b=-\lambda_I^t=-2$\,meV and (c) $\lambda_I^b=\lambda_I^t=-0$.}
\label{supfig:LLcrossing0}
\end{figure}
In Fig.~\ref{supfig:LLcrossing0}. and Fig.~\ref{supfig:LLcrossing1}. we plot the Landau level crossings, with SOC and without SOC, for $\nu=0$ and $\nu=\pm1$, respectively. Without SOC the crossings go to zero as $B\to0$ as opposed to the case of $\lambda_I^b=-\lambda_I^t\neq0$. Comparing these figures with Fig~4. in the main text, the experiments show similarities to the model: the $\nu=1$ crossings show similar tendency, so do the higher $u^*_0$ branches in the $\nu=0$ crossings. Likely the discrepancy between the model calculated in a single-particle picture and the experiment comes from the fact that in our calculations we neglect electron-electron interactions \citep{Hunt2017}.
\begin{figure}[!hbt]\centering
\includegraphics[width=\columnwidth]{fig_S8.pdf}
\caption{Landau level crossings at $\nu=\pm1$ as a function of $B$ with (a) $\lambda_I^b=-\lambda_I^t=-5.6$\,meV, (b) $\lambda_I^b=-\lambda_I^t=-2$\,meV and with (c) $\lambda_I^b=\lambda_I^t=0$.}
\label{supfig:LLcrossing1}
\end{figure}

\section{\label{sec:level_arms}Determination of the lever arms}
\begin{figure}
\includegraphics[width=\columnwidth]{fig_S12.pdf}
\caption{\label{fig:nD_convert} Conversion from a) gate voltages to b) charge carrier density $n$ and transverse displacement field $D$.}
\end{figure}

As discussed in the main text, to tune the charge carrier density $n$ and the transverse displacement field $D$ in the sample, gate voltages are applied to the metallic topgate ($V_{TG}$) and the graphite bottom gate ($V_{BG}$). The conversion from gate voltages to $n$ and $D$ is shown in Fig.\,\ref{fig:nD_convert} and is given by the following relation:

\begin{equation}
    \begin{split}
        n & = \alpha_{TG}V_{TG}+\alpha_{BG}V_{BG}+n_0 \\
        \frac{D}{\epsilon_0} & =\frac{e}{2\epsilon_0}\left(\alpha_{TG}V_{TG}-\alpha_{BG}V_{BG}\right)+\frac{D_0}{\epsilon_0},
    \end{split}
\end{equation}
where $\epsilon_0$ is the vacuum permittivity, $e$ is the elementary charge, $\alpha_{BG,TG}$ are the lever arms of the bottom and topgate, respectively, while $n_{0}$ and $D_{0}$ are the offset charge carrier density and displacement field. Since the lever arms are subject to change after the hydrostatic pressure is applied, originating from the compression of dielectricts, we determine them experimentally. First, the ratio of lever arms $\alpha_{BG}/\alpha_{TG}$ can be obatined from gate voltage maps of the resistance (e.g. Fig.\,\ref{fig:nD_convert}.a), as it is given by the slope of the charge neutrality line. Secondly, by measuring the fan diagram of Landau levels for $D=0$, we determine the lever arms via the relation $\nu=nh/eB$ between the filling factor $\nu$ and the carrier density $n$ for a given magnetic field $B$, where $h$ is Planck's constant. The values of the lever arms can be found in Table\,\ref{tab:lever_arms}.

\begin{figure}
\includegraphics[width=\columnwidth]{fig_S11.pdf}
\caption{\label{fig:fan_plot} Landau fan daigram of the resistance for a) $p=0$ and b) $p=1.65$\,GPa at $D=0$. Dashed lines correspond to carrier densities with filling factors $\nu=4k$, where $k\in\mathbb{Z}$.}
\end{figure}

\begin{table}
\centering
\begin{tabular}{|c |c |c |} 
 \hline
 $p$ ($GPa$) & $\alpha_{BG}$ ($10^{15}$ V$^{-1}$m$^{-2}$)& $\alpha_{TG}$  ($10^{15}$ V$^{-1}$m$^{-2}$)\\
 \hline
 0 & $2.47\pm0.08$ & $2.81\pm0.10$ \\ 
 \hline
 1.65 & $2.82\pm0.07$ & $3.12\pm0.09$ \\
\hline
\end{tabular}

\caption{\label{tab:lever_arms} Lever arms determined from quantum Hall measurements.}
\end{table}

\section{\label{sec:Activation}Extended activation data}
Raw measurement data obtained while cooling the device is shown in Fig.\,\ref{fig:activation_data} for $p=0$ and $p=1.65$\,GPa. Similar maps were recorded while warming up the device from base temperature (not shown).

\begin{figure}
\includegraphics[width=0.7\columnwidth]{fig_S13.pdf}
\caption{\label{fig:activation_data} Temperature dependence of the resistance $R$ as a function of $D$ at $n=0$ for a) $p=0$ and b) $p=1.65$ GPa.}
\end{figure}


\section{\label{sec:SOI_strength}Determination of the SOI parameters}
\begin{figure}
\includegraphics[width=0.75\columnwidth]{fig_S2.pdf}
\caption{\label{fig:activation} Band gaps determined from thermal activation measurements performed while warming up (red) and cooling down (blue) the device for a) $p=0$ and b) $p=1.65$\,GPa, respectively. Symmetrized curve with respect to $u=0$ is shown in orange (see text for details) and the band gap calculated from the theoretical model is shown with the solid black line.}
\end{figure}

To obtain the SOI strength from thermal activation measurements, electric displacement field $D$ has to be converted to the induced interlayer potential difference $u$. As described in the main text, this is done via the relation $u=-\frac{ed}{\epsilon_0 \epsilon_\mathrm{BLG}}D$. For ambient pressure, $d=3.3$\,\r{A} can be taken, similarly to pristine BLG.  On the other hand, the value of $\epsilon_\mathrm{BLG}$ available in the literature ranges from 2.6\,\cite{Slizovskiy2021} to 6\,\cite{Bessler2019}. Since in the large $u$ limit, the band gap induced by the displacement field is independent of the SOI parameters, by varying $\epsilon_\mathrm{BLG}$, we can effectively "fit" our model to the experimental data. Fig.\,\ref{fig:activation}.a shows the extended thermal activation data (partially presented in the main text) measured while warming up (red symbols) and cooling down (blue symbols) the sample for $p=0$. As it is visible, in the band insulator regimes for $u\ll 0$ and $u\gg 0$, the data have different slopes. We take this effect into account by averaging of the two measurements and symmetrizing it with respect to $u=0$ (solid orange line). In the next step, we determine $\epsilon_\mathrm{BLG}$ by matching the slope of the high-$u$ part of the data to match the slope of the theoretical model (solid black line). Finally, we numerically determine the position $u_{min}$ of the band gap minimum of the symmetrized curve and take the Ising SOI parameters as $\lambda_t=-\lambda_b=u_{min}$. We estimate the lower and upper bound of the SOI parameter by fitting the slope of the theoretical curve to the $u\ll 0$ and $u\gg 0$ parts of averaged band gaps, respectively, yielding $\lambda_t=-\lambda_b=2.2\pm0.4$\,meV for $p=0$.

\begin{figure}
\includegraphics[width=0.75\columnwidth]{fig_S19.pdf}
\caption{\label{fig:activation_fit} a) Band gaps determined from thermal activation measurements performed while cooling down dev S1 at $p=1.65$\,GPa, using a linear fit to the data according to the Arrhenius formula ($\ln R \propto \Delta/2k_B T$) in the temperature range 40\,K$<T<$\,70\,K (blue) and 40\,K$<T<$\,100\,K (red). Error bars represent the uncertainty of the fits. b) Examples for linear fits at $D=0$ in the temperature range 40\,K$<T<$\,70\,K (blue) and 40\,K$<T<$\,100\,K (red) are shown with solid lines and are extended with dashed lines for better visibility. In this case, the difference in the slopes stems from the saturation of the resistance around 100 K.}
\end{figure}

In contrast to ambient pressure, at $p=1.65$\,GPa, we expect both $d$ and $\epsilon_{\mathrm{BLG}}$ to change due to applied pressure. From theoretical predictions\,\cite{Carr2018,Yankowitz2018}, we expect a change in the BLG interlayer distance of $\Delta d<5\%/$GPa. However, to estimate the change in $\epsilon_{\mathrm{BLG}}$ is more challenging. To be able to extract the SOI strength at $p=1.65$\,GPa, we vary the $d/\epsilon_\mathrm{BLG}$ ratio to match the experimental data to the model using the method described above. This way, for $p=1.65$\,GPa, we extract $\lambda_t=-\lambda_b=5.6\pm0.6$\,meV. The extracted increase in SOI strength is comparable to theoretical predictions using ab initio calculations\cite{Fueloep2021}. Furthermore, the SOI parameters extracted from the minima give the same order of magnitude estimate as the extracted gaps at $u=0$. Therefore it is clear, that the relative increase of the positions of band gap minima in $D$ cannot alone stem from changes in the $d/\epsilon_\mathrm{BLG}$ ratio.

To estimate the robustness of our method, we extracted the band gaps from the thermal activation measurements at $p=1.65$\,GPa using linear fits to different temperature ranges. These are shown in Fig.\,\ref{fig:activation_fit}. for ranges of 40\,K$<T<$\,70\,K (blue) and 40\,K$<T<$\,100\,K. The error bars on the figure show the error of the fit for given $u$ and fixed temperature range. From this, we conclude that the uncertainty of the extracted band gaps is $<20\%$. More importantly, the uncertainty of the slope of the $u\ll 0$ and $u\gg 0$ regimes and the positions of band gap minima is even less. Since in our analysis, these are the parameters that determine the extracted SOI strength, this results in an uncertainty of the $d/\epsilon_\mathrm{BLG}$ ratio and the SOI strength of $\sim10\%$ that is comparable to the uncertainty estimated from the difference in the slopes of the $u\ll 0$ and $u\gg 0$ regimes.


\begin{figure}
\includegraphics[width=1\columnwidth]{fig_S10.pdf}
\caption{\label{fig:nu3} a,b) Measurements of LL crossings as a function of $B$ for $\nu = +3$ and $\nu = -3$, respectively, for
$p = 0$. c) Positions of the LL crossings extracted from the maps in a) and b) for $\nu=+3$ (blue) and $\nu=-3$ (red).}
\end{figure}

We also have to note that, although we assumed that $|\lambda_{I}^{b}|=|\lambda_{I}^{t}|$, our method of determining the SOI parameters is only sensitive to the absolute difference of the two parameters since this quantity defines the closing of the band gap. In other words, the minima of the $\Delta(u)$ functions shown in Fig.\,\ref{fig:activation} are insensitive to a difference in the abolute values of $|\lambda_{I}^{b,t}|$ as long as $\left|\lambda_{I}^{t}-\lambda_{I}^{b}\right|/2$ is constant. However, we can estimate the asymmetry of $|\lambda_{I}^{b,t}|$ by measuring the $\nu=\pm3$ LL crossings since, within our model, the positions of these crossings in $u$ are separated by $u^{*}_{+3}-u^{*}_{-3}\approx|\lambda_{I}^{t}|-|\lambda_{I}^{b}|$, nearly insensitive to the magnetic field. Measurements of the LL crossings as a function of $D$ and $B$ for $\nu=+3$ and $\nu=-3$ are shown in Fig.\,\ref{fig:nu3}.a,b, respectively. Fig.\,\ref{fig:nu3}.c shows the extracted positions $D^{*}$ of the crossings for $\nu=\pm3$. As it is visible, for the most part of our magnetic field range, the positions of the crossings are indistinguishable. At higher magnetic field we see crossings at $u^{*}_{\pm3}\neq0$, indicating a small asymmetry of the SOC parameters. From these, we can estimate the upper bound of the asymmetry as $|\lambda_{I}^{t}|-|\lambda_{I}^{b}|<0.4$\,meV for $p=0$. 

\section{Extended quantum Hall measurement data}
Additional data for the quantum Hall measurements at $p=1.65$\,GPa is shown in Fig.\,\ref{fig:crossingsatpress}.
\begin{figure}
\includegraphics[width=1\columnwidth]{fig_S3.pdf}
\caption{\label{fig:crossingsatpress} Extended data for Fig.\,4. of the main text. a) $n$-$D$ map of the resistance measured at $B=8.5$\,T and $p=1.65$\,GPa. b,c) Measurements of LL crossings as a function of $B$ for $\nu = 0$ and $\nu = 1$, respectively, at
$p = 1.65$\,GPa.}
\end{figure}


\section{Additional measurements on two-terminal devices}
We performed additional measurements on two-terminal devices also shown in Fig.\,\ref{fig:device}.a. Two different devices were measured between contacts E-F (dev S2) and F-G (dev S3) in separate pressurization and cool-down cycles. Two-terminal resistance maps of the devices as a function of $n$ and $D$ are shown in Fig.\,\ref{fig:gatemaps_S2_S3} at $p=0$ and $p=1.2$\,GPa. These results are consistent with our observations on dev S1 in that the inverted phase is clearly visible and is enhanced by the applied pressure. Fig.\,\ref{fig:gatemaps_S2_S3}.c and \ref{fig:gatemaps_S2_S3}.f shows the comparison between line traces of the resistance measured at $n=0$ as a function of $D$ at $p=0$ and $p=1.2$\,GPa for dev S2 and S3, respectively. In both cases, the resistance at $D=0$ is increased due to the applied pressure and the location of resistance minima ($D^{*}$) is increased by $\sim 25\%$.

\begin{figure}
\includegraphics[width=1\columnwidth]{fig_S14.pdf}
\caption{\label{fig:gatemaps_S2_S3} a,b) 2-terminal resistance map of dev S2 for a) ambient pressure and b) $p=1.2$\,GPa. c) Line traces of resistance for dev S2 as a function of $D$ at $n=0$ for ambient pressure (blue) and $p=1.2$\,GPa (red). d,e,f) Similar resistance maps and comparison for dev S3.}
\end{figure}

\begin{figure}
\includegraphics[width=1\columnwidth]{fig_S18.pdf}
\caption{\label{fig:activation_data_S2} Temperature dependence of the resistance of device S2 as a function of $D$ at $n=0$ for a) $p=0$. The jump in the resistance around 15 K corresponds to the superconducting phase transition of the NbTiN leads. Corrected data after the contact normal resistance was subtracted for b) $p=0$ and c) $p=1.2$\,GPa.}
\end{figure}


We also performed thermal activation measurements on both devices. For reference, the raw measurement data is shown for dev S2 in Fig.\,\ref{fig:activation_data_S2}.a for $p=0$ obtained while cooling the sample from 100 K to base temperature. Due to the two-terminal geometry of these devices a large jump in resistance is visible that corresponds to the superconducting phase transition of the NbTiN contacts.


In order to extract the band gaps from these measurements additional data processing is required and the normal resistance of the contacts have to be subtracted from the high temperature parts of the data. We subtract $R_{N,S2}=5.90$\,k$\Omega$ and $R_{N,S3}=5.38$\,k$\Omega$ for dev S2 and S3, respectively. The thermal activation data after the subtraction of the contact normal resistance is shown in Fig.\,\ref{fig:activation_data_S2}.b and \ref{fig:activation_data_S2}.c for dev S2 at $p=0$ and $p=1.2$\,GPa, respectively. After the subtraction, a vertical line is visible that originates from the phase transition. We exclude these outliers from further analysis. Although this makes our data analysis less reliable, we can nevertheless extract the SOI strength using the same method as described in section\,\ref{sec:SOI_strength}. This yields $\lambda_t=-\lambda_b=1.6\pm0.4$\,meV for $p=0$ and $\lambda_t=-\lambda_b=2.4\pm0.5$\,meV for $p=1.2$\,GPa and $\lambda_t=-\lambda_b=1.0\pm0.3$\,meV for $p=0$ and $\lambda_t=-\lambda_b=2.0\pm0.4$\,meV for $p=1.2$\,GPa for dev S2 and S3, repsectively. These results also reproduce our findings for dev S1 as a significant increase in the extracted SOI strength is seen in all cases.


\begin{figure}
\includegraphics[width=0.75\columnwidth]{fig_S15.pdf}
\caption{\label{fig:activation_gap_S2} Band gaps determined from thermal activation measurements performed while warming up (red) and cooling down (blue) device S2 for a) $p=0$ and b) $p=1.2$\,GPa, respectively. Symmetrized curve with respect to $u=0$ is shown in orange and the band gap calculated from the theoretical model is shown with the solid black line.}
\end{figure}

\begin{figure}
\includegraphics[width=0.75\columnwidth]{fig_S16.pdf}
\caption{Band gaps determined from thermal activation measurements performed while warming up (red) and cooling down (blue) device S3 for a) $p=0$ and b) $p=1.2$\,GPa, respectively. Symmetrized curve with respect to $u=0$ is shown in orange and the band gap calculated from the theoretical model is shown with the solid black line.}
\end{figure}
\newpage

\section{Additional data from Sample S4}
\begin{figure}
\includegraphics[width=0.5\columnwidth]{fig_S17.pdf}
\caption{\label{fig:gatemaps_noIP} Differential resistance of a two-terminal device as a function of $V_{BG}$ and $V_{TG}$ that shows no signatures of band inversion.}
\end{figure}

As mentioned in section\,\ref{sec:device}, not all samples showed signatures of band inversion. We attribute this to the lack of control over the rotation of WSe$_2$ layers. Based on theoretical predictions\cite{Peterfalvi2022,Li2019}, we argue that for certain rotation angles of the two WSe$_2$ layers (e.g. 0$^{\circ}$), the sign of $\lambda_I^{b}$ and $\lambda_I^{t}$ can be the same which leads to the situation discussed section \ref{sec:theory} and shown in Fig.\,\ref{supfig:sameSOC} where no band gap is present at $D=0$. For reference, in Fig.\,\ref{fig:gatemaps_noIP}, we show measurement data of a two-terminal device measured at 4 K, where no signatures of band inversion can be observed in the resistance map.

\bibliography{inverted_gap_supporting}

\end{document}
