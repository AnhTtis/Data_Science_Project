\documentclass[
reprint,
superscriptaddress,
%groupedaddress,
%unsortedaddress,
%runinaddress,
%frontmatterverbose, 
%preprint,
%preprintnumbers,
%nofootinbib,
%nobibnotes,
%bibnotes,
longbibliography,
%amsmath,amssymb,
%aps,
%pra,
%prb,
%twocolumn,
%rmp,
%prstab,
%prstper,
%floatfix,
]{revtex4-2}

\usepackage{hyperref}
\hypersetup{
    colorlinks=true,
    linkcolor=blue,
    filecolor=magenta,      
    urlcolor=cyan,
    pdftitle={Bilayer graphene inverted gap},
    %pdfpagemode=FullScreen,
    }
  
\usepackage[all]{hypcap} %hyperlink navigates to the top of a figure
\usepackage{graphicx}% Include figure files
\usepackage[section]{placeins}
\usepackage{dcolumn}% Align table columns on decimal point
\usepackage{bm}% bold math

\begin{document}

\preprint{APS/123-QED}

\title{Stabilizing the inverted phase of a WSe$_2$/BLG/WSe$_2$ heterostructure via hydrostatic pressure}



\author{M\'at\'e Kedves}
\affiliation{Department of Physics, Institute of Physics, Budapest University of Technology and Economics, M\H uegyetem rkp.\ 3., H-1111 Budapest, Hungary}
\affiliation{MTA-BME Correlated van der Waals Structures Momentum Research Group, M\H uegyetem rkp.\ 3., H-1111 Budapest, Hungary}

\author{B\'alint Szentp\'eteri}
\affiliation{Department of Physics, Institute of Physics, Budapest University of Technology and Economics, M\H uegyetem rkp.\ 3., H-1111 Budapest, Hungary}
\affiliation{MTA-BME Correlated van der Waals Structures Momentum Research Group, M\H uegyetem rkp.\ 3., H-1111 Budapest, Hungary}

\author{Albin M\'arffy}
\affiliation{Department of Physics, Institute of Physics, Budapest University of Technology and Economics, M\H uegyetem rkp.\ 3., H-1111 Budapest, Hungary}
\affiliation{MTA-BME Superconducting Nanoelectronics Momentum Research Group, M\H uegyetem rkp.\ 3., H-1111 Budapest, Hungary}

\author{Endre T\'ov\'ari}
\affiliation{Department of Physics, Institute of Physics, Budapest University of Technology and Economics, M\H uegyetem rkp.\ 3., H-1111 Budapest, Hungary}
\affiliation{MTA-BME Correlated van der Waals Structures Momentum Research Group, M\H uegyetem rkp.\ 3., H-1111 Budapest, Hungary}

\author{Nikos Papadopoulos}
\affiliation{QuTech and Kavli Institute of Nanoscience, Delft University of Technology, 2600 GA Delft, The Netherlands}

\author{Prasanna K. Rout}
\affiliation{QuTech and Kavli Institute of Nanoscience, Delft University of Technology, 2600 GA Delft, The Netherlands}

\author{Kenji Watanabe}
\affiliation{Research Center for Functional Materials, National Institute for Materials Science, 1-1 Namiki, Tsukuba 305-0044, Japan}

\author{Takashi Taniguchi}
\affiliation{International Center for Materials Nanoarchitectonics,
National Institute for Materials Science, 1-1 Namiki, Tsukuba 305-0044, Japan}

\author{Srijit Goswami}
\affiliation{QuTech and Kavli Institute of Nanoscience, Delft University of Technology, 2600 GA Delft, The Netherlands}

\author{Szabolcs Csonka}
\affiliation{Department of Physics, Institute of Physics, Budapest University of Technology and Economics, M\H uegyetem rkp.\ 3., H-1111 Budapest, Hungary}
\affiliation{MTA-BME Superconducting Nanoelectronics Momentum Research Group, M\H uegyetem rkp.\ 3., H-1111 Budapest, Hungary}

\author{P\'eter Makk}
\affiliation{Department of Physics, Institute of Physics, Budapest University of Technology and Economics, M\H uegyetem rkp.\ 3., H-1111 Budapest, Hungary}
\affiliation{MTA-BME Correlated van der Waals Structures Momentum Research Group, M\H uegyetem rkp.\ 3., H-1111 Budapest, Hungary}

\date{\today}

\begin{abstract}
Bilayer graphene (BLG) was recently shown to host a band-inverted phase with unconventional topology emerging from the Ising-type spin--orbit interaction (SOI) induced by the proximity of transition metal dichalcogenides with large intrinsic SOI. Here, we report the stabilization of this band-inverted phase in BLG symmetrically encapsulated in tungsten-diselenide (WSe$_2$) via hydrostatic pressure. Our observations from low temperature transport measurements are consistent with a single particle model with induced Ising SOI of opposite sign on the two graphene layers. To confirm the strengthening of the inverted phase, we present thermal activation measurements and show that the SOI-induced band gap increases by more than 100\% due to the applied pressure. Finally, the investigation of Landau level spectra reveals the dependence of the level-crossings on the applied magnetic field, which further confirms the enhancement of SOI with pressure.
\end{abstract}

\maketitle

Van der Waals (VdW) engineering provides a powerful method to realize electronic devices with novel functionalities via the combination of multiple 2D materials\,\cite{Geim2013}. An exciting example is the case of graphene connected to materials with large intrinsic spin--orbit interaction (SOI), which allows the generation of an enhanced SOI in graphene via proximity effect\,\cite{Konschuh2011,Gmitra2015,Khoo2017,Garcia2018,Li2019,Zollner2021,Naimer2021,Herling2020,Sierra2021,Peterfalvi2022,Avsar2014,Wang2015,Wang2016,Ghiasi2017,Voelkl2017,Benitez2017,Zihlmann2018,Wakamura2018,Ghiasi2019,Island2019,Wang2019,Omar2019,Tiwari2021,InglaAynes2021,Amann2022}. This, on the one hand, is compelling in the case of spintronics devices since the large spin diffusion length in graphene heterostructures\,\cite{InglaAynes2015,Droegeler2016,Singh2016} could be complemented with electrical tunability\,\cite{Yang2016,Dankert2017,Omar2018} or charge-to-spin conversion effects\,\cite{Garcia2017}. Moreover, it is also interesting from a fundamental point of view since graphene with intrinsic SOI was predicted to be a topological insulator\,\cite{Kane2005}. The observation of increased SOI was demonstrated in the last few years in both single layer\,\cite{Avsar2014,Wang2015,Wang2016,Ghiasi2017,Voelkl2017,Benitez2017,Zihlmann2018,Wakamura2018,Ghiasi2019} and recently in bilayer graphene (BLG)\,\cite{Wang2016,Island2019,Wang2019,Omar2019,Tiwari2021,InglaAynes2021,Amann2022}. It was found that one of the dominating spin--orbit terms is the Ising-type valley-Zeeman term which is an effective magnetic field acting oppositely in the two valleys, and could enable such exciting applications  as a valley-spin valve in BLG\,\cite{Gmitra2017}. Recent compressibility measurements\,\cite{Island2019} have shown that BLG encapsulated in tungsten-diselenide (WSe$_2$) from both sides hosts a band-inverted phase if the sign of induced SOI is different for the two WSe$_2$ layers. In practice, this can be achieved if the twist angle between the two WSe$_2$ layers is 180$^\circ$\,\cite{David2019,Zollner2021,Peterfalvi2022}.

In this paper, we experimentally investigate the SOI induced in BLG symmetrically encapsulated in WSe$_2$ (WSe$_2$/BLG/WSe$_2$) via transport measurements. We present resistance measurements as a function of charge carrier density ($n$) and transverse displacement field ($D$) at ambient pressure and demonstrate the appearance of the inverted phase (IP). In order to stabilize this phase, we employ our recently developed setup\,\cite{Fueloep2021,Szentpeteri2021} to apply a hydrostatic pressure ($p$) which allows us to boost the SOI as we have recently demonstrated on single layer graphene\,\cite{Fueloep2021a}. To confirm the increased SOI, we present thermal activation measurements where the evolution of the SOI-induced band gap can be estimated as a function of $D$ and $p$. Finally, we further investigate the induced SOI with quantum Hall measurements by tracking the Landau level crossings as a function of magnetic field.

To reveal the band-inverted phase arising from the Ising SOI in BLG, we show the low-energy band structure of WSe$_2$/BLG/WSe$_2$ in Fig.\,\ref{fig:bandstructure}, calculated using a continuum model by following in the footsteps of Ref.\,\cite{Zollner2021}. The effect of the WSe$_2$ layers in proximity of BLG can be described by the Ising SOI terms $\lambda_{I}^{t}$ and $\lambda_{I}^{b}$ that couple only to the top or bottom layer of BLG and act as a valley-dependent effective magnetic field. For WSe$_2$ layers rotated with respect to each other with 180$^\circ$, the induced SOI couplings will have opposite sign\,\cite{David2019,Zollner2021,Peterfalvi2022}. This is taken into account by the opposite sign of $\lambda_{I}^{t}$ and $\lambda_{I}^{b}$. The transverse displacement field ($D$) in our measurements can be modelled by introducing an interlayer potential difference $u=\frac{-ed}{\epsilon_0\epsilon_{BLG}}D$, where $e$ is the elementary charge, $\epsilon_0$ is the vacuum permittivity, $d=3.3$\,\AA\ is the separation of BLG layers and $\epsilon_{BLG}$ is the effective out-of-plane dielectric constant of BLG.

Fig.\,\ref{fig:bandstructure}.a-c show the calculated band structure around the \textbf{K}-point for different values of $u$, using the parameter values $\lambda_{I}^{t}=-\lambda_{I}^{b}=2$\,meV. Details of the modeling can be found in the Supporting Information. First of all, for $|u|>|\lambda_{I}^{t}|=|\lambda_{I}^{b}|$, we can see the opening of a band gap (Fig.\,\ref{fig:bandstructure}.a), as expected for BLG in a transverse displacement field\,\cite{McCann2006a,Castro2007}. On the other hand, as opposed to pristine BLG, the bands are spin-split and the direction of this spin splitting is opposite for the valence and conduction bands. This is a direct consequence of the opposite sign of $\lambda_{I}^{t}$ and $\lambda_{I}^{b}$ as the valence and conduction bands are localised on different layers due to the large $u$. The band structure in the \textbf{K'}-valley is similar, except that the spin-splitting is reversed due to time reversal symmetry. For $\left|u\right|=|\lambda_{I}^{t,b}|$ (Fig.\,\ref{fig:bandstructure}.b), the $u$-induced band gap approximately equals the spin splitting induced by the Ising SOI and the bands touch. Finally, for $|u|<|\lambda_{I}^{t,b}|$ (Fig.\,\ref{fig:bandstructure}.c), a band gap re-opens and we observe spin-degenerate bands for $u=0$, separated by a gap comparable in size to the Ising SOI terms $\left(\Delta \approx \left|\lambda_{I}^{t}-\lambda_{I}^{b}\right|/2\right)$. This gapped phase is distinct from the band insulating phase at large $u$ in that the valence and conduction bands are no longer layer polarised, hence it is usually referred to as inverted phase (IP). It is worth mentioning that the IP at $|u|<\left|\lambda_{I}^{t}\right|$ is weakly topological unlike the trivial band insulating phase\,\cite{Penaranda2022,Zaletel2019}.

\begin{figure}
\includegraphics[width=1\columnwidth]{figure1.pdf}% Here is how to import EPS art
\caption{\label{fig:bandstructure} a-c) Calculated band structure around the \textbf{K}-point for different values of the interlayer potential difference $u$. Color scale corresponds to the spin polarization of the bands.}
\end{figure}
%convert wavenumber to meaningful format

%\section{\label{sec:results}Results and discussion}
Our device consists of a BLG flake encapsulated in WSe$_2$ and hexagonal boron nitride (hBN) on both sides, as it is illustrated in Fig.\,\ref{fig:resistance}.a. To enable transport measurements, we fabricated NbTiN edge contacts in a Hall bar geometry. The device also features a graphite bottomgate and a metallic topgate that allow the independent tuning of $n$ and $D$. See Supporting Information for more details about sample fabrication and geometry.

Fig.\,\ref{fig:resistance}.c shows the resistance measured in a four-terminal geometry as a function of $n$ and $D$ at ambient pressure at 1.4 K temperature. As expected for BLG, we observe the opening of a band gap at large displacement fields along the charge neutrality line (CNL) at $n=0$, indicated by an increase of resistance. In accordance with the theoretical model and previous compressibility measurements\,\cite{Island2019}, we also observe two local minima separated by a resistance peak at $D=0$ in agreement with the closing and re-opening of the band gap signalling the transition between the band insulator and the IP. This observation is further emphasized in Fig.\,\ref{fig:resistance}.b, where a line trace (blue) of the resistance is shown as a function of $D$, measured along the CNL.

To boost the induced SOI and stabilize the IP, we applied a hydrostatic pressure of $p=1.65$ GPa and repeated the previous measurement. Fig.\,\ref{fig:resistance}.d shows the $n$--$D$ map of the resistance after applying the pressure. Although the basic features of the resistance map are similar, two consequences of applying the pressure are clearly visible. First, as it is also illustrated in Fig.\,\ref{fig:resistance}.b, the peak resistance in the IP at $D=0$ increased by $\sim$25\%. Secondly, the displacement field required to close the gap of the IP increased significantly, by about 70\%. Both of these observations can be accounted for by an increase of the Ising SOI term that results in a larger gap at $D=0$ and subsequently in a larger displacement field needed to close the gap. Altough the shift of resistance minima could be explained by the increase of $\epsilon_{BLG}$ or the decrease of interlayer separation $d$, these altogether are not expected to have greater effect than $\sim$20\% \cite{Yankowitz2018}. It is also worth mentioning that the lever arms also change due to the applied pressure, changing the conversion from gate voltages to $n$ and $D$, however, we have corrected for this effect by experimentally determining them from quantum Hall measurements (see Supporting Information).

\begin{figure}
\includegraphics[width=1\columnwidth]{figure2.pdf}% Here is how to import EPS art
\caption{\label{fig:resistance} a) Schematic representation of the measured device. Bilayer graphene is symmetrically encapsulated in WSe$_2$ and hBN. b) Line trace of the four-terminal resistance along the CNL for ambient pressure (blue) and $p=1.65$\,GPa (red). c,d) Four-terminal resistance map as a function of charge carrier density $n$ and displacement field $D$ measured at c) ambient pressure and d) an applied pressure of 1.65 GPa. The alternating low and high resistance regions along the CNL indicate the closing and re-opening of a band gap in the bilayer graphene.}
\end{figure}

To quantify the increase of SOI gap due to hydrostatic pressure, we performed thermal activation measurements along the CNL for several values of $D$. Fig.\,\ref{fig:activation}.a demonstrates the evolution of resistance as a function of 1/$T$ for selected values of $D$ at ambient pressure. From this, we extract the band gap using a fit to the high-temperature, linear part of the data where thermal activation -- $\ln(R)\propto \Delta/2k_{B}T$ -- dominates over hopping-related effects\,\cite{Sui2015}. Fig.\,\ref{fig:activation}.b shows the extracted gap values as a function of $D$ with and without applied pressure. First of all, a factor of 2 increase is clearly visible in the gap at $D=0$ for $p=1.65$\,GPa, that is consistent with the observed increase of resistance. Secondly, the higher $D$ needed to reach the gap minima is also confirmed. We also note that the band gap cannot be fully closed which we attribute to spatial inhomogeneity in the sample.

The experimentally determined band gaps allow us to quantify the SOI parameters. By adjusting the theoretical model to match the positions of the gap minima for $p=0$, we extract $\lambda_{I}^{t}=-\lambda_{I}^{b}=2.2\pm0.4$\,meV. Since we do not expect $d$ to change substantially due to pressure, by fixing its value to 3.3\,\AA, we can extract the SOI parameters at $p=1.65$\,GPa as well. For these, we obtain $\lambda_{I}^{t}=-\lambda_{I}^{b}=5.6\pm0.6$\,meV. A more detailed discussion on the extraction and possible errors is given in the Supporting Information.

\begin{figure}
\includegraphics[width=1\columnwidth]{figure3.pdf}% Here is how to import EPS art
\caption{\label{fig:activation} Thermal activation measurements along the charge neutrality line. a) Arrhenius plot of the resistance at ambient pressure for selected values of $D$. Solid lines are fits to the linear parts of the data from which the band gap values were obtained. b) Gap $\Delta$ as a function of displacement field at ambient pressure (blue) and an applied pressure of 1.65 GPa. Arrows indicated the $D$ values for which the activation data is shown in a).}
\end{figure}

The quantum Hall effect in BLG provides us another tool to investigate the Ising SOI induced by the WSe$_2$ layers. The two-fold degeneracy of valley isospin ($\xi=+,-$), the first two orbitals ($N=0,1$) and spin ($\sigma=\uparrow,\downarrow$) give rise to an eight-fold degenerate Landau level (LL) near zero-energy\,\cite{McCann2006,Novoselov2006,Hunt2017}. This degeneracy is weakly lifted by the interlayer potential difference, Zeeman energy, coupling elements between the BLG layers\,\cite{Khoo2018} and the induced Ising SOI\,\cite{Wang2019}. We can obtain the energy spectrum of this set of eight closely-spaced sublevels -- labeled by $|\xi ,N,\sigma \rangle$ -- by introducing a perpendicular magnetic field in our continuum model, as detailed in \cite{Khoo2018}. This is shown in Fig.\,\ref{fig:QHall}.a for $B=8.5$\,T as a function of the interlayer potential ($u$). LLs with different $\xi$ reside on different layers of the BLG, therefore $u$ induces a splitting between these levels. Secondly, the finite magnetic field causes the Zeeman-splitting of levels with different $\sigma$. Finally, the Ising SOI induces an additional effective Zeeman field associated to a given layer, further splitting the levels. The key feature that should be noted here, is that for a given filling factor $\nu$, crossings of LLs can be observed and the position of crossing points along the $u$ axis depend on SOI parameters as well as on the magnetic field. These level crossings manifest as sudden changes of resistance in our transport measurements as it is illustrated in Fig.\,\ref{fig:QHall}.b. Here, the $n-D$ map of the resistance is shown as measured at $B=8.5$\,T with fully developed resistance plateaus (due to the unconvetional geometry, see Supporting Information) corresponding to the sublevels of $\nu \in \left[ -4,4\right]$. For a given filling factor $\nu$, we observe $4-|\nu|$ different $D$ values where the resistance deviates from the surrounding plateau corresponding to the crossing of LLs, as expected from the model. 

\begin{figure*}
\includegraphics[width=2\columnwidth]{figure4.pdf}% Here is how to import EPS art
\caption{\label{fig:QHall}a) Low energy Landau level spectrum at $B=8.5$ T obtained from single-particle continuum model with $\lambda_{I}^{t}=-\lambda_{I}^{b}=2$\,meV. b) Four-terminal resistance as a function of $n$ and $D$ measured at $B=8.5$\,T out-of-plane magnetic field and ambient pressure. Resistance plateaus correspond to different $\nu$ filling factors. Abrupt changes in resistance at a given $\nu$ as a function of $D$ indicate the crossings of LLs. c,e) Measurements of LL crossings as a function of $B$ for $\nu=0$ and $\nu=1$, respectively, for $p=0$. Symbols denote LL crossings shown in (a). d,f) Critical displacement field $D^{*}$ corresponding to LL crossings for $\nu=0$ and $\nu=1$ extracted from $D-B$ maps measured at $p=0$ (blue, see (c,e)) and $p=1.65$ GPa (red).}
\end{figure*}

The evolution of LL crossings with $B$ can be observed by performing measurements at fixed filling factors, as it is shown in Fig.\,\ref{fig:QHall}.c and \ref{fig:QHall}.e for $\nu=0$ and $\nu=1$, respectively. During the latter measurement, the carrier density $n$ was tuned such that the filling factor given by $\nu=nh/eB$ was kept constant. On both panels, we can observe $4-\nu$ LL crossings that evolve as $B$ is tuned, until they disappear at low magnetic fields where we can no longer resolve LL plateaus. This $B$-dependent behaviour enables us to investigate the effect of SOI on the LL structure. Fig.\,\ref{fig:QHall}.d and \ref{fig:QHall}.f shows the critical displacement field $D^{*}$ values -- where LL crossings can be observed -- extracted from Fig.\,\ref{fig:QHall}.c and \ref{fig:QHall}.e and similar maps measured at $p=1.65$\,GPa (see Supporting Information). For $\nu=0$ (Fig.\,\ref{fig:QHall}.d), the most important observation is that the crossing points do not extrapolate to zero as $B\rightarrow 0$\,T which is a direct consequence of the induced Ising SOI. It is also clearly visible that due to the applied pressure, $|D^{*}|$ is generally increased, especially at lower $B$-fields, indicating that the Ising SOI has increased, in agreement with our thermal activation measurements. For $\nu=1$ (Fig.\,\ref{fig:QHall}.f), similar trends can be observed. The two LL crossings at finite $D$ saturate for small $B$, while the third crossing remains at $D=0$. We note that the $D^{*}(B)$ curves for $p=1.65$ GPa cannot be scaled down to the $p=0$ curves, which confirms that our observations cannot simply be explained by an increased $\epsilon_{BLG}$ or decreased interlayer separation distance, but are the results of enhanced SOI. We also point out that some lines which extrapolate to $D=0$ can also be observed (e.g. Fig.\,\ref{fig:QHall}.e, grey arrow). This could also be explained by sample inhomogeneity. It is also important to note that our single-particle model fails to quantitatively predict the $B$-dependence of the LL crossings indicating the importance of electron-electron interactions (see Supporting Information). 

%\section{Conclusions}
In conclusion, we showed that the IP observed in BLG symmetrically encapsulated between twisted WSe$_2$ layers can be stabilized by applying hydrostatic pressure which enhances the proximity induced SOI. We presented thermal activation measurements as a means to quantify the Ising SOI parameters in this system and showed an increase of approximately 150\% due to the applied pressure. In order to gain more information on the twist angle dependence of the SOI, a more systematic study with several samples with well-controlled twist angles is needed. The enhancement of Ising SOI with pressure was further confirmed from quantum Hall measurements. However, to extract SOI strengths from these measurements, more complex models are needed that also take into account interaction effects. Our study shows that the hydrostatic pressure is an efficient tuning knob to control the induced Ising SOI, thereby the topological phase in WSe$_2$/BLG/WSe$_2$. 

The IP has a distinct topology from the band insulator phase at large $D$, and the presence of edge states are expected\,\cite{Penaranda2022}. The presence of these states should be studied in better defined sample geometries\,\cite{SanchezYamagishi2016,Veyrat2020} or using supercurrent interferometry\,\cite{Hart2014,Indolese2018}. Opposed to the weak protection of the edge states in this system, a strong topological insulator phase is predicted in ABC trilayer graphene\,\cite{Li2012,Zaletel2019}. Furthermore, pressure could also be used in case of magic-angle twisted BLG, in which topological phase transitions between different Chern insulator states are expected as a function of SOI strength\,\cite{Wang2020}.

\section*{Author contributions}
N.P. and P.K.R fabricated the device. Measurements were performed by M.K., B.Sz., P.K.R. with the help of M.A., P.M. M.K. and B.Sz. did the data analysis. B.Sz. did the theoretical calculation. M.K., B.Sz. and E.T. and P.M. wrote the paper and all authors discussed the results and worked on the manuscript. K.W. and T.T. grew the hBN crystals. The project was guided by Sz.Cs., S.G. and P.M.

\begin{acknowledgments}
This work acknowledges support from the Topograph, MultiSpin and 2DSOTECH FlagERA networks, the OTKA K138433 and PD 134758 grants and the VEKOP 2.3.3-15-2017-00015 grant. This research was supported by the Ministry of Culture and Innovation and the National Research, Development and Innovation Office within the Quantum Information National Laboratory of Hungary (Grant No. 2022-2.1.1-NL-2022-00004), by SuperTop QuantERA network, by the FET Open AndQC netwrok. We acknowledge COST Action CA 21144 superQUMAP. P.M. and E.T. received funding from Bolyai Fellowship. This project was supported by the ÚNKP-22-3-II New National Excellence Program of the Ministry for Innovation and Technology from the source of the National Research, Development and Innovation Found. K.W. and T.T. acknowledge support from JSPS KAKENHI (Grant Numbers 19H05790, 20H00354 and 21H05233). The authors thank Pablo San-Jose, Elsa Prada and Fernando Pe\~{n}aranda for fruitful discussions.
\end{acknowledgments}

\bibliography{inverted_gap}

\end{document}
