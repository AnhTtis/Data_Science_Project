%\documentclass[aps,prl,preprint,superscriptaddress]{revtex4-1}
%\documentclass[aps,prl,twocolumn,showpacs,superscriptaddress]{revtex4-1}

\documentclass[notitlepage, nofootinbib,
 reprint,
superscriptaddress,
%groupedaddress,
%unsortedaddress,
%runinaddress,
%frontmatterverbose,
%preprint,
%showpacs,preprintnumbers,
%nofootinbib,
%nobibnotes,
%bibnotes,
 amsmath,amssymb,
 aps,prl,
%pra,
%prb,
%rmp,
%prstab,
%prstper,
%floatfix,
onecolumn
]{revtex4-1}
%\usepackage{graphicx}% Include figure files
\usepackage{dcolumn}% Align table columns on decimal point
\usepackage{bm}% bold math
\usepackage{gensymb}
\usepackage{color}
\usepackage{float}
\usepackage{multirow}
\usepackage{svg}
\usepackage{float}
\usepackage{graphicx}
\usepackage{siunitx}
\DeclareUnicodeCharacter{2009}{\,} 
\DeclareUnicodeCharacter{3B1}{\ensuremath{\alpha}}
\DeclareSIUnit\angstrom{\text {Å}}
\renewcommand{\thefigure}{S\arabic{figure}}
\usepackage{upgreek}
\usepackage{romannum}
\usepackage[super]{nth}

\begin{document}
%\preprint{APS/123-QED}
\title{Supplemental Material: Molecular Rheology of Nanoconfined Polymer Melts}
\author{Ahmet Burak Yıldırım}
\affiliation{Department of Mechanical Engineering, Bilkent University, 06800 Ankara, Turkey}

\author{Aykut Erbaş}
\affiliation{UNAM - National Nanotechnology Research Center and Institute of Materials Science \& Nanotechnology, Bilkent University, 06800 Ankara, Turkey}
\affiliation{Institute of Physics, University of Silesia, Katowice, Poland}

\author{Luca Biancofiore}
\affiliation{Department of Mechanical Engineering, Bilkent University, 06800 Ankara, Turkey}
\affiliation{UNAM - National Nanotechnology Research Center and Institute of Materials Science \& Nanotechnology, Bilkent University, 06800 Ankara, Turkey}

\date{\today}% It is always \today, today,
             %  but any date may be explicitly specified

\maketitle
\section{Simulation Details} 
The work includes two types of simulations: (i) diffusion to determine the relaxation times and (ii) shear to determine the viscoelastic response. For both types, the interactions between all atoms are modelled by the standard 12/6 Lennard-Jones potential using D\textsc{reiding} force field \cite{mayo_dreiding_1990, das_solvation_1996}, with the cutoff radius of $r_c = 9.85$ $\mathrm{A}^\circ$ and the time-integration is performed for the timestep of $\tau = 1$ fs. 

For (i), each simulation box contains $N=216$ charge-neutral linear or star $\mathrm{C_{25} H_{52}}$ polymers equilibrated at $T = 372$~K at a pressure at $p = 1$ atm ~\cite{mccabe_examining_2001}. The resultant melt density of $\rho = 0.63$ $\mathrm{g/cm^3}$ is expected as D\textsc{reiding} potential is known to underestimate the density of the polymer melts while giving accurate viscosity estimations \cite{ewen_comparison_2016}. Preserving the density does not guarantee that hydrostatic pressures are also identical between the systems; the hydrostatic pressures are also prone to variations in the confinement level, chain topology, and rigidity. However, since we are mainly interested in acquiring identical geometric confinements and densities for all of our melt systems, we disregard the variations in hydrostatic pressure between the systems \cite{kontopoulou_basic_2011} but rather track the change in the flow-dependent (i.e., dynamic) components of the stress tensor. This helps us to construct $N_1$, $N_2$, $\bm{\upsigma}^{\left ( \mathrm{VE} \right )}$ independently of the hydrostatic pressure to study the rheological response. 
We track the mean-square end-to-end vectors for $t=10^8~\tau$ under NVE dynamics regulated at $T=372$~K using the Langevin thermostat with a damping coefficient of $100~\tau$. We later fit the correlation function following $ {\langle\mathbf{r_{ee}}(t) \cdot \mathbf{r_{ee}}(0)\rangle}/{\langle\mathbf{r_{ee}}(0) \cdot \mathbf{r_{ee}}(0)\rangle} = A\exp \left ( -t/\lambda \right )$, with constant $A$ ($\approx 1$ for each fit).

For (ii), the simulation boxes contain varying $N$ polymers, depending on the box size or confinement level, while keeping the density $N/V$ constant for the volume $V$. For the bulk system, we run shear simulations by deforming the cubic simulation box ($h = 5.82$ nm, resulting in $p = 1$ atm for unaltered star polymer) under constant volume and temperature by solving S\textsc{llod} equations of motion \cite{raghavan_shear-thinning_2017} at $T=372$~K with a damping coefficient of $100~\tau$. For the confined melts, we run shear simulations under NVE dynamics at $T=372$ K using the Langevin thermostat ($N = 216$ for box dimensions: 6.35 nm $\times$ 5.57 nm $\times$ 6.97 nm) with a damping coefficient of $10~\tau$, which is achieved by moving the top and bottom surfaces at constant velocity in the x-direction \cite{morciano_nonequilibrium_2017}. While investigating the confinement effect on $N_1$ and $N_2$, we changed the number of polymers $N$, accordingly (i.e., $N=216, h = 5.57$ nm). The thermostat is only applied to the non-sheared directions to preserve the increase in velocity due to shear correctly. The fixed and rigid wall configuration consists of five hematite $ \left ( \alpha-\mathrm{Fe_{2} O_{3}} \right ) $ layers on the top and bottom surfaces \cite{ewen_nonequilibrium_2016}, which  results in a wall thickness of 1.31 nm to ensure a negligible wall deformation compared to thin and deformable walls. Unlike soft walls with variable film thickness, rigid walls are known to experience higher load variations, making them a better choice for observing the magnified confinement effects \cite{israelachvili_intermolecular_nodate}. A Couette flow is simulated for $10^6-2\times10^7\tau$, ensuring that a typical polymer chain passes through the periodic boundaries at least three times, eliminating the transient response. The damping coefficients are selected to ensure that the temperature remains $\sim 372\pm 0.1$~K throughout the simulations. 

Note that the volume of the bulk and confined systems differ due to wall interactions; hence $p = 1$ atm is preserved for unaltered star polymers while determining the volume of both linear and star polymer systems. This enables us to validate some of the results for star polymers with Ref.~\cite{mccabe_examining_2001}.

\section{Crystallization}
\begin{figure}[H]
    \centering
    \includegraphics[width=0.99\textwidth]{Jan23/suppmat/crystallization.pdf}
    \caption{Snapshots of the crystalline structure formation of unaltered linear polymer melt ($k_{\theta} = \epsilon$) under shear for different simulation box sizes. The shear rate is $\dot{\gamma}=6\times10^8 s^{-1}$, corresponding to $Wi = 12.48$. Size of the simulation boxes for (a) is $(h,h,h)$, (b) is $(2h,2h,h)$, for (c) is $(h,4h,h)$, and (d) is $(h,6h,h)$ where $h \approx 5.82 \si{\angstrom}$ to retain the identical density of $\rho = 0.63$ $\mathrm{g/cm^3}$. Under identical settings, star polymer melts do not exhibit crystallization. For each simulation box size, molecular and bond configurations are illustrated separately, where the coloring of the bonds is random.} 
    \label{fig:fig1}
\end{figure}

\section{Slip}

\begin{figure}[H]
    \centering
    \includegraphics[width=0.99\textwidth]{Jan23/suppmat/velocityProfileLinear.pdf}
    \caption{The steady-state spatial configuration and the velocity distribution of the atoms within the shear of confined linear polymer melts for $h = 5.57 \si{\angstrom}$. The shear rate is (a) $\dot{\gamma}=6\times10^8 s^{-1}$, (b) $\dot{\gamma}=6\times10^9 s^{-1}$, (c) $\dot{\gamma}=6\times10^{10} s^{-1}$, (d) $\dot{\gamma}=6\times10^{11} s^{-1}$, and (e) $\dot{\gamma}=6\times10^{12} s^{-1}$ , as a function of the normalized vertical component $\underline{y}$. The corresponding Weissenberg numbers are (a) $Wi = 12.48$, (b) $Wi = 124.8$, (c) $Wi = 1248$, (d) $Wi = 12480$ and (e) $Wi = 124800$. The snapshots of the simulation box given on the left illustrate the carbon-carbon bond configuration of the system, where the coloring of the bonds is random.}
    \label{fig:fig2}
\end{figure}
\begin{figure}[H]
    \centering
    \includegraphics[width=0.99\textwidth]{Jan23/suppmat/velocityProfileStar.pdf}
    \caption{The steady-state spatial configuration and the velocity distribution within the shear of confined star polymer melts where $h = 5.57 \si{\angstrom}$. The shear rate is (a) $\dot{\gamma}=6\times10^8 s^{-1}$, (b) $\dot{\gamma}=6\times10^9 s^{-1}$, (c) $\dot{\gamma}=6\times10^{10} s^{-1}$, (d) $\dot{\gamma}=6\times10^{11} s^{-1}$, and (e) $\dot{\gamma}=6\times10^{12} s^{-1}$ , as a function of the normalized vertical component $\underline{y}$. The corresponding Weissenberg numbers are (a) $Wi = 5.46$, (b) $Wi = 54.6$, (c) $Wi = 546$, (d) $Wi = 5460$, and (e) $Wi = 54600$. The snapshots of the simulation box given on the left illustrate the carbon-carbon bond configuration of the system, where the coloring of the bonds is random.}
    \label{fig:fig3}
\end{figure}

\section{Monomers}
\begin{figure}[H]
    \centering
    \includegraphics[width=0.66\textwidth]{Jan23/suppmat/monomerViscosity.pdf}
    \caption{The normalized viscosity of monomers under confinement, obtained by breaking all the carbon-carbon bonds of the linear polymer melt, together with linear and star polymer melts in bulk and under confinement. Linear and star polymers have the unaltered configuration ($k_{\theta} = \epsilon$), whereas $k_{\theta}$ does not apply for the monomers.  \Romannum{1}, \Romannum{2}, and \Romannum{3} indicate the \nth{1} Newtonian plateau, the shear-thinning regime, and the \nth{2} Newtonian plateau for bulk polymers, respectively.} 
    \label{fig:fig4}
\end{figure}

\section{Radius of Gyration}

\begin{figure}[H]
    \centering
    \includegraphics[width=1\textwidth]{Jan23/suppmat/confinedRadiusOfGyrationXYZMag.pdf}
    \caption{Steady-state radii of gyration (${R_G}$) for confined linear and star polymer melts where $h = 5.57 \si{\angstrom}$. The $x$, $y$, and $z$ components of the radius of gyration, together with the overall magnitude, are given for linear (a) and star (b) polymers. The overall magnitudes $R_{G,\mathrm{mag}}$ do not show significant dependency with $\dot{\gamma}$, the average values read $R_{G,\mathrm{linear}}^{(0.1\epsilon)}=9.34\mathrm{\AA}$,~$R_{G,\mathrm{linear}}^{(\epsilon)}=9.50\mathrm{\AA}$,~$R_{G,\mathrm{linear}}^{(10\epsilon)}=9.52\mathrm{\AA}$;~$R_{G,\mathrm{star}}^{(0.1\epsilon)}=6.32\mathrm{\AA}$,~$R_{G,\mathrm{star}}^{(\epsilon)}=6.31\mathrm{\AA}$, and ~$R_{G,\mathrm{star}}^{(10\epsilon)}=6.36\mathrm{\AA}$.}
    \label{fig:fig5}
\end{figure}

\begin{figure}[H]
    \centering
    \includegraphics[width=0.75\textwidth]{Jan23/suppmat/confinedRadiusOfGyrationXY.pdf}
    \caption{Near/away from the surface steady-state radii of gyration (${R_G}$) for confined linear and star polymer melts where $h = 5.57 \si{\angstrom}$. The $x$ and $y$ components of the radius of gyration are given near the surface for linear (a) and star (b) polymers; away from the surface for linear (c) and star (d) polymers. Especially for star polymers, the relation between viscoelastic load and near-surface dynamics cannot be captured using radii of gyration. The snapshots depict the regions where the components of the radii of gyrations are calculated based on the positions of the center of mass of each chain, whose approximate positions are shown in yellow for a single timestep.}
    \label{fig:fig6}
\end{figure}

\bibliographystyle{h-physrev}

\bibliography{Bib1}

\end{document}