\newcommand{\RR}{\mathbb{R}}
\newcommand{\NN}{\mathbb{N}}
\newcommand{\nrm}[1]{\lvert #1 \rvert}
\newcommand{\WW}{\mathcal{W}}
\newcommand{\cX}{\mathcal{X}}
\newcommand{\cY}{\mathcal{Y}}
\newcommand{\cN}{\mathcal{N}}
\newcommand{\cA}{\mathcal{A}}
\newcommand{\cD}{\mathcal{D}}

\DeclareMathOperator{\ev}{ev}
\DeclareMathOperator{\relu}{ReLU}
\DeclareMathOperator{\diag}{diag}
\DeclareMathOperator{\flatten}{vec}
\DeclareMathOperator{\tr}{tr}
\DeclareMathOperator{\success}{success}
\DeclareMathOperator{\Gr}{Gr}
\DeclareMathOperator{\sign}{sign}


\numberwithin{equation}{section}

\theoremstyle{plain}
\newtheorem{theorem}[equation]{Theorem}
\newtheorem{lemma}[equation]{Lemma}
\newtheorem{corollary}[equation]{Corollary}
\newtheorem{proposition}[equation]{Proposition}
\newtheorem{conjecture}[equation]{Conjecture}
\newtheorem{finding}[equation]{Finding}

\theoremstyle{definition}
\newtheorem{definition}[equation]{Definition}

\theoremstyle{remark}
\newtheorem{remark}[equation]{Remark}
\newtheorem{calculation}[equation]{Calculation}
\newtheorem{observation}[equation]{Observation}
\newtheorem{example}[equation]{Example}
\newtheorem{examples}[equation]{Examples}
\newtheorem{question}[equation]{Question}