\documentclass[a4paper, 12pt]{article} %{{{

% \usepackage{lineno}
% \linenumbers
\usepackage{dcolumn}
\usepackage[margin=1in]{geometry}
\usepackage[utf8]{inputenc}
\usepackage{graphicx}
\usepackage{subfigure}
\usepackage{authblk}
% \usepackage[flushleft]{threeparttable}

%----Helper code for dealing with external references----
% (by cyberSingularity at http://tex.stackexchange.com/a/69832/226)

\usepackage[style=nature,backend=biber,sortcites=true]{biblatex}
\addbibresource{main_current.bib}
% Ignore irrelevant biblatex fields
\AtEveryBibitem{%
 \clearfield{url}%
 \clearfield{month}%
 \clearfield{issn}%
 \clearfield{doi}%%
 % \clearfield{address}%%
}

\usepackage{xcite}
\usepackage{xr}
\makeatletter

\newcommand*{\addFileDependency}[1]{% argument=file name and extension
\typeout{(#1)}% latexmk will find this if $recorder=0
% however, in that case, it will ignore #1 if it is a .aux or 
% .pdf file etc and it exists! If it doesn't exist, it will appear 
% in the list of dependents regardless)
%
% Write the following if you want it to appear in \listfiles 
% --- although not really necessary and latexmk doesn't use this
%
\@addtofilelist{#1}
%
% latexmk will find this message if #1 doesn't exist (yet)
\IfFileExists{#1}{}{\typeout{No file #1.}}
}\makeatother

\newcommand*{\myexternaldocument}[1]{%
\externaldocument{#1}%
\addFileDependency{#1.tex}%
\addFileDependency{#1.aux}%
}
%------------End of helper code--------------

% put all the external documents here!
% \myexternaldocument{SI_current}

% Math
\usepackage{amsmath}
\usepackage{amssymb}
\usepackage{newtxmath}
\DeclareMathAlphabet{\mathpzc}{T1}{pzc}{m}{it}
\DeclareMathOperator*{\argmin}{\arg\!\min}
\DeclareMathOperator*{\argmax}{\arg\!\max}
\usepackage{bm}
\def\tnull{{\text{null}}}
\def\vec#1{{\bm #1}}
\def\mat#1{\mathbf{#1}}

% Figure and table captions
\usepackage[labelfont=bf]{caption}
\captionsetup{font=footnotesize}
\usepackage{floatrow}
\floatsetup[table]{capposition=top}



\newcommand{\secref}[1]{Section~\ref{sec:#1}}
\newcommand{\figref}[1]{Fig.~\ref{fig:#1}}
\newcommand{\tabref}[1]{Table~\ref{tab:#1}}
\newcommand{\todo}[1]{{\leavevmode\color{orange}[TODO: #1]}}

% abbreviations
\def\etal{\emph{et~al}.\ }
\def\eg{e.g.,~} 
\def\ie{i.e.,~}
\def\cf{cf.\ }
\def\viz{viz.\ }
\def\vs{vs.\ }


% Comments
\usepackage[dvipsnames]{xcolor}
\definecolor{dkgreen}{rgb}{0,0.6,0}
\definecolor{gray}{rgb}{0.5,0.5,0.5}
\newcommand{\editHY}[1]{{\textcolor{dkgreen}{#1}}}
\newcommand{\noteHY}[1]{\textbf{\textcolor{dkgreen}{{\scriptsize{HY:}}#1}}}

\usepackage{xurl}
\usepackage[utf8]{inputenc}
\usepackage{changepage}

% % Figure caption
% \usepackage{setspace}
% \usepackage[font=small,labelfont=bf]{caption}
% \captionsetup[subfigure]{font={bf,small}, skip=1pt, singlelinecheck=false}

\newenvironment{sciabstract}{%
\begin{quote} \bf}
{\end{quote}}


\renewcommand\Authfont{\fontsize{15}{14.4}\selectfont}
\renewcommand\Affilfont{\fontsize{10}{9}\itshape}

\newcommand{\figdir}{figs}
\newcommand{\tabdir}{tabs}


\title{
   Nested Skills in Labor Ecosystems: A Hidden Dimension of Human Capital
}
% Mind skill-gap
%Hidden Dimension to Human Capital: understanding the dependency structure of skills
% The Anatomy of Human Capital: understanding the dependency structure of skills
% The skill anatomy of human capital: understanding the dependency structure of skills
% Skill hierarchies - from human capital specificity to nestedness
% Skill hierarchies: seeing human capital through the lens of directed networks
% Skill hierarchies: understanding the dependency structure in human capability
% Bearing on Common Roots:  On the Shoulders of the Common: ...
%Unveiling the Nested Structure: A New Perspective on Skill Dependencies in Labor Ecosystems.
%"Nested Structures in Skill Landscapes: Unveiling the Hidden Complexity of Labor Ecosystems."
%Hidden Dimension to Labor Ecosystems: understanding the nested dependency in skill structures
%"Nested Hierarchies in Skills: Revealing Hidden Aspects of Labor Ecosystems."
%"Nested Dependencies in Labor Ecosystems: Unraveling the Hidden Layers of Skill Structures."
%"Human Capital Unpacked: Exploring the Nested Dependencies in Labor Skill Structures."
%"The Hidden Architecture of Labor: Understanding Nested Dependencies in Skill Structures."
%"Skills and Dependencies: Revealing Hidden Dimensions of Labor Ecosystems."
%"Beyond the Visible: Understanding Nested Skill Structures in the Labor Market."
% ``Hidden Dimension to Labor Ecosystems: understanding the nested dependency in skill structures''
%"Unraveling the Nested Structure of Skills: A New Perspective on Human Capital"
% Mind skill-gap
% "The Hidden Layers of Human Capital: Unpacking the Nested Structure of Skills"
% The Anatomy of Human Capital: understanding the dependency structure of skills
% The skill anatomy of human capital: understanding the dependency structure of skills
% Skill hierarchies - from human capital specificity to nestedness
% Bearing on Common Roots:  On the Shoulders of the Common: ...

% suggested reviewers
% Balazs Lengyel, Christopher Esposito, Lingfei Wu, James McNerney; Muhammed Yildirim; 
 

\author[1,2,3]{Moh Hosseinioun}
\author[4]{Frank Neffke}
\author[5]{Letian (LT) Zhang}
\author[1,2,6]{Hyejin Youn \thanks{Correspondence can be sent to hyejin.youn@kellogg.northwestern.edu.}}


\affil[1]{Kellogg School of Management, Northwestern University, Evanston, IL, USA}
\affil[2]{Northwestern Institute on Complex Systems, Evanston, IL, USA}
\affil[3]{Department of Information and Decision Sciences, University of Illinois, Chicago, IL, USA}
\affil[4]{Complexity Science Hub Vienna, Vienna, Austria}
\affil[5]{Harvard Business School, Harvard University, Cambridge, MA, USA}
\affil[6]{Santa Fe Institute, Santa Fe, NM, USA}

\date{\today}



\begin{document}

\maketitle
\thispagestyle{empty}


\vspace{-0.5cm}
%%%%%%%%%%%%%%%%%%%%%%%%%%%%%%%%%%%%%%%%%%%%%%%%%%%%%%%%%%%%%%%%%%
%%%% Below is the 150-word abstract (250 is left in the main) %%%%
 %Modern economies operate through globally interconnected networks. 
 %As economies become more complex, so do these networks, coordinating increasingly diverse portfolios of specialized efforts and knowledge. 
 %Here, we infer an interdependency tree underlying the fabric of skill portfolios. Hierarchically constructed, this skill tree starts from widely needed, foundational abilities, constituting the root, and extends to highly specialized, niche skills required by select jobs at the extremities. 
 %The directionality is defined by the asymmetrical conditional probabilities of the presence of one skill given the existence of another. 
 %Examining 70 million job transitions, we observe individuals tend to delve deeper into these nested specialization paths as they ascend the career ladder to enjoy higher wage premiums.  
 %Nevertheless, the role of foundational skills for such ascent remains pivotal; without reinforcing them, the anticipated wage premiums may vanish, suggesting the critical role of foundational skills for specialization, and the need for balanced skill development strategies in complex economies.
 %We differentiate 'nested' skills, building on common prerequisites, from others, to examine their disparities across regions and demographic groups as to wage premiums.
 %Our temporal analyses reveal a growing fragmentation between these skill groups over the past decades, suggesting further job polarization.  
%%%% Above is the 150-word abstract (250 is left in the main) %%%
%%%%%%%%%%%%%%%%%%%%%%%%%%%%%%%%%%%%%%%%%%%%%%%%%%%%%%%%%%%%%%%%%%

%%%%%%%%%%%%%%%%%%%%%%%%%%%%%%%%%%%%%%%%%%%%%%%%%%%%%%%%%%%%%%%%%%
%%%%      Below is the 250-word abstract                      %%%%
%%%%%%%%%%%%%%%%%%%%%%%%%%%%%%%%%%%%%%%%%%%%%%%%%%%%%%%%%%%%%%%%%%
\begin{abstract} \label{sec: abstract}
%Modern economies generate immensely diverse complex goods and services through coordinating our efforts and know-how in densely interwoven networks that span across the globe. As these economies grow increasingly complex, participation in these networks grows more intricate and requires an increasingly specialized skill portfolio. With the ever-increasing complexity of these economies, these networks grow more intricate and demand an ever-expanding portfolio of specialized skills. 
%Modern economies generate a vast array of goods and services through complex, globally interdependent networks of coordinated efforts and specialized knowledge. As these economies become more complex, so do the networks, necessitating a more diverse set of specialized skills for participants to acquire.
%Modern economies, characterized by their vast output of goods and services, operate through globally interconnected networks that coordinate efforts and specialized knowledge. As economies become more complex, so do these networks, necessitating an increasingly diverse portfolio of specialized skills for network participants.
Modern economies, characterized by their vast output of goods and services, operate through globally interconnected networks. 
As economies become more complex, so do these networks, coordinating increasingly diverse portfolios of specialized efforts and knowledge. 
In this study, we analyze U.S. survey data (2005--2019) to infer an underlying interdependency tree within the fabric of skill portfolios. Hierarchically constructed, this skill tree starts from widely needed, foundational abilities, constituting the root, and extends to highly specialized, niche skills required by select jobs at the extremities. 
The directionality is defined by the asymmetrical conditional probabilities of the presence of one skill given the existence of another. 
Examining 70 million job transitions in resumes and national surveys, we observe that individuals tend to delve deeper into these nested specialization paths as they ascend the career ladder to enjoy higher wage premiums.  
Nevertheless, the role of foundational skills for such ascent remains pivotal; without reinforcing them, the anticipated wage premiums may vanish. 
Hence, we further differentiate \textit{nested} skills from others, with the former building on common prerequisites while the latter does not, 
and analyze disparities in these skill gaps across different geographic locations, genders, and racial/ethnic groups, observing how these variations in absorptive capacity impact wage premiums. Our analysis reveals a growing and concerning fragmentation in the divide between these two skill groups over the past two decades, suggesting further polarization within the job landscape \cite{Autor2013}.
Our findings highlight the critical role of robust foundational skills as a stepping stone to specialization and the economic advantages it can confer, reinforcing the need for balanced skill development strategies in complex economies \cite{Althobaiti2022}.
%Our findings underscore the importance of a strong foundation of general skills, which are essential for specialization and the subsequent rewards it can offer.
%ones hovering on their own and lacking the foundational roots for wage premiums, and we find this divide has not only grown but also has alarmingly fragmented over two decades, potentially pointing to job market polarization. 
%%----------------------------------------------------------------------------%%
\end{abstract}
\newpage
\pagenumbering{arabic}


\section*{Introduction}
% GUIDELINE: First paragraph: Increasing complexity: from stone tools to smartphones (the same size but capable of so many things) 
% First paragraph is about society, and the second paragraph is about individuals responding the society
% Complexity and specialization are two defining characteristics of modern economies.
%  Modern economies, marked by complexity and specialization, have been on an exponential growth trajectory ever since the dawn of the first stone tool. Instead, it requires globally coordinated networks of specialized individual expertise and effort. 
% Frank_neffke's: I added references to Henrich, Boyd/Richerson, and Turchi, all about the rise of complex societies, but in the literature on cultural evolution. The division of labor argument is made by Boyd/Richersen and Henrich. For Turchin, dol is not a big deal, although he mentions it. I think we can drop him. Still, he does argue that societies are becoming more and more complex. As to reference, There is probably a better citation in the complexity literature here. This paper is a model of trade that follows an O-ring structure, meaning that the more complex the product becomes, the higher the quality of the participants in the value chain has to become to avoid failure.
% This daunting complexity isn't a recent phenomenon. 
%Everyday items like toasters or smartphones perfectly illustrate this paradigm shift. 
Complexity and specialization are two defining characteristics of economic growth \cite{Carneiro1986, Hidalgo2015, henrich2015secret, richerson1999complex, BenJ2009, MitchellMelanie2009}. From the time humans shaped the first stone tool, the growth in technological complexity has snowballed to the point where creating and maintaining societal artifacts no longer rests within the realm of individual capabilities. This paradigm shift is exemplified by everyday items such as smartphones, toasters, and even pencils. The assembly of an iPhone, for instance, engages over 200 suppliers, each with its own workforce \cite{Barboza2016}. 
Likewise, the Toaster Project vividly demonstrates the limitations faced by an individual when tasked with building a simple toaster from scratch in today's economy \cite{Thwaites}. 
This daunting complexity isn't a recent phenomenon. 
Indeed, as far back as 1958, the humble pencil served as a powerful testament to the intricate interdependence of global manufacturers in the creation of even the simplest items \cite{Read1958, Dubner2016}. 


The socio-economic complexity demands a continuous progression towards greater specialization of skills and knowledge and eventually coordination with countless other specialized individuals in expansive, complex global networks \cite{BenJ2009, Wuchty2007, Becker1992, BeckerG.S1992TDoL, McNerney2022, Azoulay2020, VanderWouden2023}. This ongoing evolution of the division of labor and knowledge lays the groundwork for the dynamic reconfiguration of skill dependencies, and sophisticated technologies that integrate individual inputs, thus relieving us from the need to acquire specific embedded skills \cite{Gamble2002}. 
As such, work has inevitably transitioned into a collective enterprise. The coordinated efforts of many, harnessed within global networks, contribute to the creation of the countless products and services that underpin our daily lives. 

Though the collective nature of work isn't new to us, the growing complexity and its resulting impacts on individuals remain an ongoing area of exploration. To meet the socio-economic complexity, we find ourselves not only increasing investments in education and training to sharpen our skills but also scrambling to acquire the \textit{right} set of skills, knowledge, and abilities to participate effectively in these productive networks and bargain for a fair share of the output they create \cite{Mincer1974, Becker1962, Lucas1988, Neffke2013, Neffkeeaax3370}. 
In this effort, certain skills can be gained without any prerequisites, hinging on the repetition and routinization of narrowly defined tasks, while other skills are not immediately attainable, requiring a sequential and cumulative learning path conditional on an individual's previous attainments. The prerequisites for understanding calculus, for instance, are grounded in the knowledge of algebra and geometry. Similarly, professional skills are accrued systematically over a career \cite{Gibbons2004, AndersonKatharineA2017Snam, Cunha2007, Li2021, Arrow1974book, Kuhn1962book}. These inherent dependencies, evident in our educational and career trajectories, provide a roadmap for our learning journeys and consequently shape our social and economic structures \cite{Autor2013, SchwabeHenrik2020Awsa, NelsonDylan2022, Goldin2008}. 
This leads to research questions: What does the structure of these interdependencies look like? And, more importantly, what implications does this structure carry? 
Though the framework may seem intuitive, it is essential to note that the hierarchical layout of skills reflected in job roles has often been assumed rather than empirically evidenced, as have its implications for individuals and society at large. 



In this study, we analyze skill portfolios and their underlying structures using national surveys. 
Initially, we distinguish the generality of skills---those demanded by select occupations and those widely required---and differentiate a general skill group from a specific skill group based on their demand distribution profiles (Fig. 1). Next, we determine the asymmetrical conditional probabilities of the presence of one skill given the existence of another, thereby inferring a dependency direction between the two \cite{Jo2020}. 
By aggregating these dependency relationships, we construct a \emph{nested} hierarchical structure where the gradient from general to specific skills is vertically embedded (Fig. 2). 
The resulting structure presents an unbalanced tree, revealing its partially nested architecture \cite{Saavedra2011, Baldwin2014}. 
Indeed, as mentioned earlier, not all skills necessitate a dense web of dependencies. 
Therefore, we further differentiate between \textit{nested} skills, which are underpinned by dense prerequisites, and those without such foundational support. 


We examine three different datasets (median occupational ages, synthetic cohorts of individuals, and 70 million job transitions in resumes) and observe that as individuals progress up the career ladder, they tend to explore deeper into nested specialization paths and enjoy higher wage premiums (Fig. 3 and Fig. 4). Interestingly, these nested specialization paths, unlike the unnested ones, involve strengthening their foundational, general prerequisite skills. The distinction between nested and un-nested skills proves pivotal when predicting the level of education, experience, and wages associated with an occupation. It also indicates spatial labor division across U.S. regions and consequential disparities of race/ethnicity and gender (Fig. 5 and Fig. 6). Furthermore, given the significance of nested and un-nested skills for future career developments and wage premiums, structural changes in the hierarchical tree network over two decades are concerning, as they reveal an economy growing more nested and specialized (Fig. 7), with an increasingly widening gap between the two.

%\cite{AndersonKatharineA2017Snam,Neffkeeaax3370}, the impact of automation on the world of work \cite{Alabdulkareem2018} and the role of complementarity in teams of coworkers \cite{Neffkeeaax3370}.

%% directionality of investment in skills and path-dependence shows up in Arrow's Limit of Organization
% The existence of such an interdependency web is not in doubt as it manifests in education and career paths and thus shapes the social and economic systems \cite{Autor2013, SchwabeHenrik2020Awsa}.
% Unlike the latter, the former is perhaps not the kind of division of labor that Adam Smith envisioned \cite{smith1937wealth}.
%Sometimes, we have to acquire long training to get to be able to do (nested), and sometimes we are naturally capable of it. What does the structure of those look like, and what's the implication of the structure? 
%Alternative: Since the division of labor by adam smith and Alfred Marshall, tasks are divided such as to increase productivity. But what Adam Smith and Marshal did not know was the complexity also increases not only the division of labor but also the division of knowledge to share the needed functions with many different people. Toyota example?
% Unlike the first concept of division of labor, where tasks are divided in a way to be performed by anyone substitutable, even children (ref: children labor), there have been many skills and knowledge needed for jobs that need a long training. This begs the question: which skills are "nested" and which skills are not nested, and how hierarchically are they structured? 
% The directionality of skill interdependence is not only in the educational curricula but also in carrier paths.
%There is little doubt that such a structure is also influenced by external forces



%% Structure: towards to end of the learning spectrum is specialization %% more specialized skills carry higher returns.
%Understanding the underlying structure of capabilities required to undertake economic activity is of vast interest.
% The recent studies uncovered network structure explaining fragmentation and job mobility \cite{Alabdulkareem2018, Xu2021, Lin2022, Neffkeeaax3370, DelRio-Chanona2021}. 
%Therefore, a comprehensive skill structure disentangles mechanisms of specialty, in which knowledge and technological advances increase the investment necessary to contain the depth of knowledge in an individual \cite{BenJ2009}, and diversity, in which the increasingly complex economy \cite{Hidalgo2009, Hidalgo2015, MitchellMelanie2009} requires more diverse capabilities for a unit of economic activity.


%..... may include this later part... let me think .....
%  Recent work on skill-interdependencies has modeled the skill structure as a network of co-occurrences \cite{AndersonKatharineA2017Snam, Alabdulkareem2018, Xu2021, Lin2022, Neffkeeaax3370, DelRio-Chanona2021}.
% On one hand, the hierarchical structure is well known in explaining wage premiums and so on \cite{Mincer1974, Becker1962} 
%However, because the role of specialization and diversification in shaping skill co-occurrences is rarely teased out \cite{Neffkeeaax3370}, such networks miss the directionality of skill accumulation and how it is influenced by advances in knowledge and technology.
% For instance, one cannot explain why programming is in one cluster and not the other beyond its pattern of bundling in occupations.
% Nor can we deduct the values of skills from the relationships or the structure of the network.
% Therefore, these networks cannot capture changes in the knowledge structure or inform policy, such as education, economic development, or worker reskilling.
% Focusing on a dichotomy of skill clusters, while these skill networks have predicted wage and education, they fail to incorporate specialization and diversification, which are pivotal attributes of skill evolution..
% Therefore, we expect the skill structure not only to branch off horizontally but also to grow vertically in depth [like a growing tree, where some skills act as the root and trunk, and further, smaller branches form].






%%----------------------------------------------------------------------------%%
\section*{Results}

\subsection*{Skill Generality (Individual Occurrences)} \label{sec: skill hierarchy}

We first group skills into categories of generality, determined by the distribution of demand across occupations, ranging from niche skills required by a select few occupations to skills broadly demanded. The conceptual distinction between general and specialized skills is well-documented, though a systematic demonstration has been lacking due to limited data availability \cite{Becker1962, Poletaev2008, Gathmann2010, FergusonJohn-Paul2013, Leung2014, Merluzzi2016, Byun2018, Fini2022, Byun2023}. Ideal empirical data for our purpose would record how individuals acquire skills over time, the aggregate of which will uncover the structure of skills' interdependency. This structure, however, is latent and not readily observed in reality. Instead, our study turns to publicly available surveys conducted by the U.S. Bureau of Labor Statistics (BLS), which record the importance and intensity of each skill, knowledge, or ability required in detailed occupational categories. 

To empirically distinguish these skills, we proceed with the assumption that general skills experience ubiquitous demand across the economy compared to specialized skills. 
For instance, Fig. 1 (a) demonstrates that while only a handful of jobs require high-level Fine Arts skills, a significant level of English proficiency is universally necessary across many occupations. 
Therefore, we classify skills into categories of generality based on their demand profile shapes. Figure 1 shows the aggregated shapes in each category as the average occupation counts for given skill levels accompanied by quartile boxes (see Methods and SI for details). Insets present additional examples for each category, including dynamic flexibility and programming in the specific skill group, mathematics and negotiation in the intermediate, and critical thinking and oral expressions in the general group.  
The full list of each category in SI seems overall in good agreement with conventionally defined basic and specialized skills. 
Furthermore, increasing medians of the demand profiles in the skill categories are also consistent with the skill generality---0.87 for specific, 2.37 for intermediate, and 3.34 for general skills.  
In addition, we provide local reaching centrality as an alternative measure for generality in the next section, which is consistent with the measures above. Finally, we test our results against different group sizes and different clustering algorithms to ensure the robustness of our findings (see SI
% section \ref{supsec:skill clustering}
for details). 

%  (1) rank skills by its average mean and median? (2) calculate a correlation with centrality (basically y-axis in Fig 2). This should give us very high correlation score (3) average of average means (and medians) within each group (general, intermediate, and specific)."Correlations (I will fill in the main text):
%average and median level: 0.98
%average level and LRC: 0.71
%median level and LRC: 0.70
%Group level info:
%Skill Groups Mean of Average Levels Mean of Median Levels Mean of LRC
%1      General               3.343942             3.3552941  0.38189874
%2 Intermediate               2.371088             2.3727907  0.10813977
%3     Specific               1.135680             0.8747436  0.04179466



\begin{figure*}[!h]
    \centering
    \includegraphics[width=\textwidth]{Nature_HB_2023/figNHB/Figure_1_-_Jul_5_2023.png}
    \caption{\textbf{Skill level profiles.}  
    \textbf{(a)} The number of occupations that require given levels of Fine Arts (left), Psychology (middle), and English skill (right). The shape of each skill-level profile represents the distribution of occupational demand for varying levels of these skills.  As depicted, high-level Fine Arts skills are needed only in a few select jobs, while a relatively high level of English proficiency is commonly required across many occupations.
    \textbf{(b)} Skills are grouped into the three characteristic shapes, exemplified above, and labeled as \textit{General} (a group of 31 skills with similar skill-level-profile shapes), \textit{Intermediate} (43 skills), and \textit{Specific} (46 skills).
    The graphs include the averaged occupation counts for given skill levels within each category, accompanied by quartile boxes (see SI
    % section \ref{supsec:skill clustering}
    for details.)
    Insets include additional examples from each skill group.
    }
    \label{fig:Figure 1}
\end{figure*}


% Mathematics skill, using mathematics to solve problems, is demanded across the economy, while Mathematical Reasoning, defined as the ability to choose the right mathematical methods or formulas to solve a problem, is important to certain occupations and irrelevant to others. 
% As such, our results complement existing work that has found evidence of a spectrum of human capital specificity at the individual level, e.g., \cite{Becker1962, Gathmann2010, Poletaev2008, Gibbons2004}.
% In search of a more systematic view of returns to human capital, we further examine horizontal skill interdependence.
% This is motivated by the intuition that skill's closely dependent on various fundamental levels may require longer trajectories, hence, requiring higher investments and leading to higher economic returns, as a result \cite{Mincer1974, Lucas1988, Becker1962, Schultz1961}.
% node size (ranging from less than high school to post-doctorate)

\begin{figure*}[!h]
    \centering
    \includegraphics[width=.85\textwidth]{Nature_HB_2023/figNHB/Figure_2_-_Aug_5_2023.png}
    \caption{\textbf{Skill Dependency Hierarchy.} 
    \textbf{(a)} Schematic illustrates inferring directionality from the asymmetric conditional probability of a skill, contingent on another skill being statistically significantly present in occupations. 
    For instance, if the requirement of math skill is more probable given the presence of programming (as opposed to the reverse), the direction is given between the two, math $ \rightarrow $ programming, weighted with their level of asymmetry (see Methods).  
    Likewise, oral expression $ \rightarrow $ English language and systems analysis $ \rightarrow $ negotiations. 
    \textbf{(b)} Dependency hierarchy is constructed from the aggregated weighted directions of all skill pairs. Node sizes are proportional to education levels, and colored by the groups in Fig \ref{fig:Figure 1}.  
    A node's horizontal and vertical positions are, respectively, its educational attainment and local reaching centrality.  Defined as the proportion of the skills reachable from each node or the number of interdependent skills, the centrality is a reasonable indicator for skill generality \cite{Mones2012}. 
    \textbf{(c)} Nestedness scores of specific (blue) and intermediate skills (gray) and their associated education, and automation risk indexes (size) \cite{Frey2017, Frank2019, Frank2022}. 
    The vertical line at zero separates, hereafter, nested and unnested skills. 
    \textbf{(d-f)} Reachability (arrival probability) from each skill to programming, negotiation, and repairing (highlighted) \cite{norris1998markov}. Dark hues indicate a higher likelihood of arriving at the focal skill (see Methods). Contrary to the well-nested programming and negotiation, repairing does not predominantly rely on general skills, indicating its unnested nature. 
    }
    \label{fig:Figure 2}
\end{figure*}
% Fig. 2-c Caption: skill-dependency network is represented as a nested matrix where skills within each group are sorted by their dependency counts (see Methods). 
%The empirical matrix (right) differs from the pure nested and pure random matrices (left), indicating the skill hierarchy comprises mixed nested with un-nested structures. 
% Figure~\ref{fig:Figure 2}~(c) embeds skills in a nested matrix where skills are sorted by their interactions within each group defined by the Stochastic Block Model (see SI-\ref{suppsec:nestedness} for details). 
%The nested skill matrix indicates that the skill structure deviates from a pure nestedness shape (triangular) when entities are sorted based on the number of their interactions, leaving us with two subtypes, the nested and un-nested skills, whose groups are aligned with the left and the right branches in Fig 2 (b). 
%In the following sections, we differentiate nested from un-nested specific skills, with the former building on general skills while the latter do not exhibit this characteristic, and examine their implications for individuals' career development, wage premiums, and skill gaps across geographic locations, genders, and racial/ethnic groups.

\subsection*{Skill Hierarchy and Nestedness (interdependency)} 

The extent to which general versus specific skills are relevant in many occupations is a longstanding topic of discussion in fields like labor economics, sociology, and management 
\cite{Becker1962, Neal1995, Parent2000, Poletaev2008, Gathmann2010, FergusonJohn-Paul2013, Leung2014, Merluzzi2016, Byun2018, Fini2022, Byun2023, Leahey2007, Teodoridis2018, Heiberger2021}. 
From our experience, it is conceivable, and thus we propose here, that certain general skills serve as the foundation for acquiring a wide range of specific skills, as reflected in the structure of most college curricula where fundamental courses precede specialized ones. 
Figure 1 (b) provides select examples of these potential interdependencies. 
Take negotiation, for instance. 
The development of negotiation skills in the intermediate group hinges on improving one's general skills of critical thinking and oral expression, classified under the general skill group. 
Similar prerequisite chains extend from critical thinking to math to programming encompassing all three skill groups depicted in Fig 1 (b). 
Allocation of skills across our empirically derived classes resembles certain biological mutualistic interactions wherein specialist species preferentially interact with generalists, suggesting integrations of general skills with more specialized ones under a nested structure \cite{Bascompte2003, Saavedra2011, Saavedra2009}. Nevertheless, not every skill is perfectly nested. Certain specialized skills, such as dynamic flexibility (also in Fig. 1 inset), don't conform to a nested structure. Furthermore, certain skills that are highly relevant to specific occupations may hold no value in others. These intricate dependencies across skill categories hint at a nuanced, hierarchical structure within these relationships. 

As such, we build a vertical hierarchical tree, extending from general to specialized skills, incorporating dependencies and their respective directions. Figure~\ref{fig:Figure 2}~(a) illustrates the inferred weighted dependencies between pairs of skills and, consequently, their directionalities by leveraging the asymmetrical conditional probabilities of skills within occupations, given the existence of another \cite{Jo2020}. Aggregating these dependencies across all potential skill pairings yields a nested hierarchical structure whose backbone network is shown in Fig. 2 (b). 
Nodes are colored according to their assigned groups and positioned relative to their educational requirements (horizontal) and local reaching centralities (vertical), illustrating the gradient from general to specialized skills. 
The local reaching centrality is defined as the proportion of the nodes in the hierarchy structure that is reachable from the focal node via outgoing edges, indicating how much other skills are dependent on the focal skill \cite{Mones2012}. As such, this measure offers additional indicators of skill generality to the average level, and the correlation between the two is as high as 0.71. 
%corr(average, median level)=0.98; corr(average level, LRC)=0.71; (median level,LRC)= 0.70

The resultant structure manifests as an uneven tree, thereby hinting at a partially nested architecture \cite{Saavedra2011, Baldwin2014}. 
Indeed, as previously noted, not all skills necessitate a dense web of dependencies. 
Certain intermediate and specific skills in the left branch appear to rely more heavily on general skills than their counterparts in the right branch, whose coarsening structure of two clusters is consistent with the findings from previous seminal studies \cite{Neffkeeaax3370, AndersonKatharineA2017Snam, Alabdulkareem2018, Frank2018, DelRio-Chanona2021, Moro2021}. 
However, the additional dimension of directionality reveals an important aspect of career development that may impact job disparity, as shown in the next section. 


Much like nested mutualistic interactions in ecology \cite{Saavedra2011, Saavedra2009}, where specialist species engage preferentially with generalists \cite{Bascompte2003}, skills may also exist within a nested, with more general types of human capital serving as the foundation for acquiring more specific ones. 
It is perhaps those skills that make the labor ecosystem distinctively nested, manifested as vertically deeper dense dependencies for fruitful nested specializations in terms of wages, education, demographic, and regional disparity (see the next Sections for details) \cite{Autor2014}. 
To test this premise, we calculate to what extent a specific skill (blue nodes in Fig. 2 b) contributes to the nested structure of the occupation-skill ecosystem and divide them into those that contribute a lot (nested), and those that do not (unnested).  


A skill's individual contribution to the nested structure, $c_s$, is defined as a deviation from a null expectation in which links of a focal node $s$ are randomly re-assigned, $c_s = (N - <N^{\ast}_s> ) / {\sigma_{N^{\ast}_s}}$.
$N$ is an observed nestedness, and $<N^{\ast}_s>$ and $\sigma_{N^{\ast}_s}$ are the means and standard deviations of $s$'s counterfactual nestedness \cite{Saavedra2011}. For each focal skill, $s$, we run 5,000 iterations of the null model \cite{SergeiMaslov2002}. And for $N$, we employ the overlap index ($N_c$), checkerboard score, Temperature, and NODF, nestedness commonly used in ecology, to quantify nestedness $N$ \cite{write1992, write1992, Saavedra2011} (See SI
% Sec. \ref{suppsec:nestedness}
). Figure 2 (c) presents skills' nestedness contribution and required education levels, divided by the vertical line at the threshold into two subtypes: the nested and unnested skills.   
Correspondingly, these groups predominantly appear in the left and right branches in Fig 2 (b).
In figurative terms, nested skills are akin to offshoots sprouting from a deeply rooted, sturdy trunk (a dense network of common prerequisites), whereas unnested skills lack such foundational support. 

Figure 2 (d-f) highlights the influence of nested dependencies on skills such as programming, negotiation, and repairing, by color-coding nodes according to their reachabilities, calculated as arrival probabilities, to the focal skill nodes \cite{norris1998markov} (see Methods). 
Unlike the well-nested programming and negotiation skills, only a handful of other skills are reachable from repairing skills, which are mostly in the unnested parts of the skill tree. 
In the following sections, we differentiate nested from un-nested skills, with the former building on general skills while the latter do not, and examine their implications for individuals' career development, wage premiums, and skill gaps across geographic locations, genders, and racial/ethnic groups.


\begin{figure*}[!h]
    \centering
    \includegraphics[width=.9\textwidth]{Nature_HB_2023/figNHB/figure_3_-_Jul_14_2023.png}
    \caption{
    % slopes: nested specific = 0.46 + 0.018 * age, R^2 = 0.031; un-nested specific = 1.3 - 0.01 * age, R^2 = 0.0045; residualized nested specific = 0.3 - 0.01 * age, R^2 = 0.018; residualized un-nested specific = -0.8 + 0.016 * age, R^2 = 0.0015;
    \textbf{Skill Compositions with Occupational Ages and Career Trajectory.}
    \textbf{(a-c)} Average skill levels of occupations (and 95\% confidence intervals), segmented by employees' median ages. Levels of general and nested skills rise with an occupation's median age, while unnested skills do not vary across median-age groups.
    \textbf{(d-f)} Average skill levels (and 95\% confidence intervals) against age in synthetic cohorts. The insets isolate cohorts born in 1967, whereas the main figures average across all cohorts. Notably, general and nested skills rise markedly until around age 30, with declining unnested skills. Moreover, gender gaps also become more pronounced around 30. 
    \textbf{(g-h)} Average skill levels (and 95\% confidence intervals) over identified job sequences as documented in resumes for general, nested, and unnested skills. 
    \textbf{(i)} Changes in skill levels in consecutive job transitions. Skill profiles are typically stabilized within the initial five jobs. 
    The first two transitions are predominated by growth in general skills, paving the way for nested skills to catch up with the general. However, the growth in general skills remains notably constant throughout the nested specialization paths. Unnested skills, in contrast, are decreasing over workers' careers and especially early in these careers. These patterns disappear when the sequences of jobs are randomized, indicated by the grey triangles.
    }
    \label{fig:age}
\end{figure*}


%%%% DONT FORGET CORRELATION BETWEEN GENERAL and NESTED, and UNNESTED. 
%% Cor(General, nested specific) = 0.646
%% Cor(General, un-nested specific) = -0.379
%% Cor(nested specific, un-nested specific) = -0.095

% Such dynamics are driven by general skills being prerequisites to nested specific skills.
% The latter relationship highlights the fact that an important outcome of education is the acquisition of general skills.

%% i. career trajectory: BG and age & economic return
%The degree to which general versus specific skills play a role in many occupations acknowledges a long-standing debate in labor economics, sociology, and management \cite{Becker1962, Neal1995, Parent2000, Poletaev2008, Gathmann2010, FergusonJohn-Paul2013, Leung2014, Merluzzi2016, Byun2018, Fini2022, Byun2023, Leahey2007, Teodoridis2018, Heiberger2021}. 
% using DATA: https://www.bls.gov/cps/demographics.htm\#age} 
% the median age for each occupation in 2019
%  CPS has only 2 years of observation that is overwhelmed by short-term mobility and hence part-time jobs. 
% To the extent that such learning of skills correlates with age, controlling for the pre-requisite (i.e., general skills) should account for any relationship between%. However, the vice versa should not hold: controlling for the dependent (i.e., nested specific) skills should not eradicate the relationship between age and the pre-requisite (i.e., general) skills. The dependent (i.e., nested specific skills).
% However, apart from the challenges of obtaining individual skill data, information captured in intermediately used such sources, for instance, resumes, is likely biased towards reporting specific skills. 
%Include this in data section: Compared to the Current Population Survey, the Burning Glass data is more skewed (ref the paper Daehyun provided) 
% For each move, we link the source and destination occupations to skills from O*NET in 2019.
% we have cohorts born before 1980 but show all cohorts' skill compositions in the period of 1980 and 2022.
%\subsection*{Workplace skill acquisition through career trajectories}
%\subsection*{Occupational Skill Compositions in Career Trajectories}
\subsection*{Skill Categories in Career Trajectories}



We now examine how the derived skill structure uncovers individual career trajectories through three empirical observations: median ages for occupations, synthesized cohorts from individual surveys, and job transitions in resumes.
Each data source provides its unique strengths and weaknesses, which, when combined, complement each other and sketch a coherent picture of career paths. 
We begin our analysis with occupational ages because the assumption is that specializations are likely to demand a substantial investment of time and a dense set of prerequisites. Thus, it is reasonable to expect them to correlate closely with age.

Accordingly, we compute the median age for each occupation using the Current Population Survey (CPS) and the levels of general, nested, and unnested skills in occupations, segmented by their median ages (see Methods).
As shown in Fig. \ref{fig:age} (a-c), the outcomes align consistently with our predictions. Occupations with median ages of over 30 demand high levels of both general and nested skills, while unnested skills, supposedly lacking interdependencies, do not demonstrate any significant correlations with ages, as shown in Fig. 3 (c).


We examine if our results hold true across career trajectories by constructing synthetic cohorts using the CPS microdata. The data provide yearly repeated cross-sectional surveys, but respondents cannot be longitudinally traced. 
Therefore, we construct a ``synthetic panel'' that mimics the career trajectories of respondents, by connecting snapshots of surveys through their birth years. 
Suppose we construct a 1967 cohort for Fig. 3 (d-f) insets. 
We first identify all observations of individuals born in 1967 from the dataset and sort them as if we observed the same individuals as they aged. 
We repeat this for different birth cohorts, excluding observations of respondents of non-full-time and below age 17 or above 55. 


Figure \ref{fig:age} (d-f) shows the skill composition of synthetic cohorts from 1980 to 2022, with insets for the 1967 cohort. Consistent with the findings in Fig. (a-c), the outcomes from the synthetic cohorts provide further insights. 
Age 30 is again a significant transition point. 
General and nested skills concurrently increase sharply until around 30, when unnested skills experience a moderate decrease. Over the age of 30, the rise in overall skill levels stabilizes.


The advantage of this second dataset is that it includes not only the age of individuals but also other demographic information, allowing us to decompose our findings by gender. 
Differentiating skill trends of men and women uncovers a gender gap in specializations that emerges around 30. 
Men continue to grow their general and nested skills until well into their 40s, whereas, for women, the increase in these skills hits a plateau in their early 30s, the typical age range for first-time mothers in the US.
These findings are robust to conditioning out yearly economic conditions, as shown in SI
% Fig. \ref{fig:individuals' age and skill - year effects}
. 
In the following Sections and in SI
% Fig. \ref{fig:Skill Age Gender Race Trends - year effects}
, we offer more detailed breakdowns of these gender disparity trends with respect to race and ethnicity.


Lastly, we complement our findings using additional datasets sourced from resumes that record individual job transitions. 
The data encompass over 70 million job transitions documented in over 20 million resumes,  eliminating the need for synthetic cohorts by providing a direct record of job sequences of individual workers. 
However, this dataset does not include age or gender information for detailed analyses and is known for its biased sampling, favoring more nested job roles. 
Hence, while these data are valuable for corroborating the previous findings, they cannot replace previous datasets.


Figure 3 (g-h) shows the average skill levels required in job sequences held across career paths in which jobs are ordered by date, and Fig. 3 (i) displays changes in skill requirements for the $i$th job transition, $\Delta_{i}$, excluding job transitions within the same occupation, because in these transitions $\Delta_{i}= 0$ by construction. 
On aggregate, career journeys unfold with increasing stocks of both general and nested skills over their careers, in a way that suggests that nested specialization paths require simultaneous increases in nested specific skills along with their dependency skills. 
In addition, skill portfolios typically stabilize within the first five job transitions, namely, $\Delta_{i > 5} \approx 0$. 
In the first three jobs ($i<3$), nested skills require more general skills than later, that is, $\Delta^{general}_{i<3} \gg \Delta^{nested}_{i<3}$, 
after which they become comparable $\Delta^{general}_{i > 3} \approx \Delta^{nested}_{i > 3}$. 
This continued growth in general skills across career paths is intriguing, suggesting that these skills need to be continuously enhanced regardless of how advanced we are in our career journeys.
As a benchmark, we create bootstrapped job sequences denoted in gray marks around zero, which randomize the order of jobs as if there is no career development. We confirm that the empirically observed trends indeed are attributed to career developments (see SI for details). 

All three empirical observations consistently depict nested specializations (i.e., growth in both general and nested skills) throughout career trajectories, while unnested skills are left relatively underdeveloped. 
However, the final analysis of resumes in Fig. 3 offers the first direct evidence of a recurring yet counterintuitive pattern previously observed in occupational age and synthetic cohorts: specialization goes hand in hand with generalization. 
This suggests that the conventional model, where basic general skills precede advanced specialized skills, is not entirely accurate. 
Instead, career paths tend to unfold with increasing emphasis on general skills and their dependent, nested skills.

One might argue that our findings are driven by management/administration jobs, which are typically undertaken later in careers with higher wages. 
However, even after repeating the entire analysis without these jobs, we found consistent results (see SI
% section \ref{sec:robustness check: no managers}
). Furthermore, we repeated the entire analysis, excluding social skills, to determine if they were driving our findings. Once again, our results remained robust, suggesting that our findings are attributed to the intrinsic structure of skills rather than the influence of particular social skills or managerial jobs (see SI
% Section \ref{sec:social skills}
).
%In essence, individuals typically experience rapid growth in general skills, followed by nested specializations, as if they are prerequisites. 
%Although swiftly diminished, however, these nested specialization paths continue to involve a strengthening of their foundational, general skills. 




\begin{figure*}[!h]
    \centering
    \includegraphics[width=\textwidth]{Nature_HB_2023/figNHB/Figure_4_-_Education_-_Wage_Figure_-_Jul_14_2023.png}
    \caption{\textbf{Skill Wage Premiums and Educational Requirements}. Occupations' average annual wage \textbf{(a)} and 
    required educations \textbf{(b)} with skill levels (and 95\% confidence intervals), 
    and their respective slopes (blue bars) in \textbf{(c)} and \textbf{(d)}, and (and standard errors). 
    The dramatic wage premiums and higher educational requirements for nested specializations disappear (shaded bars) after controlling for general skill levels (insets), implying that the bulk of investments and returns to specialization are accounting for the accumulation of general skills. The initial wage penalty for unnested specializations turns into a wage premium once we control for general skill levels. 
    % slops after scaling: wage = 4.5 + 0.19 * nested specific, R^2 = 0.38; wage = 4.9 - 0.084 * un-nested specific, R^2 = 0.12; wage = 4.8 - 0.0071 * residualized nested specific, R^2 = 0.00022; wage = 4.8 + 0.023 * residualized un-nested specific , R^2 = 0.0065;
    }
    \label{fig:Wage}
\end{figure*}




%% ESTABLISH INTERDEPENDENCE
% Establish the term nested specialization
%% Wage Results: Investment pays off...
% We supplement the above analysis with an explicit examination of how nested interdependencies correlate with education and influence the value of skills.
%These relationships are commonsense \cite{Alabdulkareem2018}: occupations that possess "valued" skills command higher premiums and require more education. The question is whether specific or general skills drive this pattern.    %in two setups: unconditional and when conditioned on general skills' endowment. Comparing the unconditional and conditional coefficients for nested specific skills shows that much of their association with education and wages is due to general skills.
% Accumulating absorptive capacity in one period will permit its more efficient accumulation in the next. By having already developed some absorptive capacity in a particular area, a firm may more readily accumulate what additional knowledge it needs in the subsequent periods in order to exploit any  critical external knowledge that may become available wage premiums and require more education. 
%We analyze the differential influence of skill categories on occupational wages and investments in education. 

%Moreover, the initial wage penalties for unnested specializations have turned into small positive wage premiums.  
%A closer look at Fig. \ref{fig:Wage} (a) provides insight into this non-trivial phenomenon. 
%Unlike the monotonic decline in educational level with unnested specializations, annual wages start rising again at very high levels of un-nested specific skills. This suggests that high-level specialization eventually provides benefits surpassing what broad skills and education can offer. 
%This observation is consistent with human capital theory \cite{Becker1962}. 
%Sailors, for instance, who command ships, have a greater level of unnested skills, both raw and residualized, thus earning more than those who load the ships.

\subsection*{Skill Categories and Wage Premiums} 

The differentiation between general and nested/unnested skills turns out to be vital when predicting the levels of education, experience, and wages associated with a given occupation. 
Figure. \ref{fig:Wage} (a-b) shows that the required education for occupations and their return as average annual wages tend to rise (fall) in tandem with their nested (unnested) specializations. 
Aligned with our premise of wage premiums for nested specializations, the results are not entirely surprising. 
However, these observed wage premiums (blue bars) in Fig. \ref{fig:Wage} (c) almost fully disappear when required general skills are controlled (shaded bars), while the observed wage penalty for unnested skills turns into a comparable wage premium, suggesting the driving force of general skills behind the wage effect of nested skills. 
Further analyses in the SI show that our findings are robust to adding control variables for experience, training, and workplace experiments, persist across major occupational groups, and are not driven by managerial occupations or social skills (see SI
% Section \ref{supsec: add - returns to skill}, SI-Table \ref{tab:wage reg on skill endowment}, Fig. \ref{fig:SI_education_skill_level}-\ref{fig:SI_wage_skill_level}, Fig. \ref{fig:Figure 3 full | major occupation groups}, Fig. \ref{fig:returns_to_skills_hierachy_gen_dependence_cor_no_manager}, and Fig. \ref{fig:social skills}.
) 

These findings support a labor market model in which the primary educational investment and wage rewards for specializations presuppose simultaneous accumulated and refined general skills. 
Without such strengthening of general skills, specialization (now in specific skills) does not yield higher wages. 
The development of general skills is perhaps instrumental to accruing absorptive capacity \cite{Cohen1990}, enhancing further skill accumulation in later periods. 
Having already established absorptive capacity in a specific domain, gathering additional knowledge required in future periods becomes easier.
This cumulative process is essential to taking advantage of any key external knowledge that may become accessible and necessitate further education to achieve future wage premiums.
In addition, results support the significant emphasis placed on early childhood education as a cornerstone for future academic achievement and labor market success \cite{Cunha2007}. 



%%---------------------------------------------------------------------------%%
\begin{figure*}[!h]
    \centering
    \includegraphics[width=\textwidth]{Nature_HB_2023/figNHB/Figure_7_-_Demographic_Ratios_-_log-y_-_Skills,_Education,_and_Wages_-_1980-2022_-_Jul_11_2023.png}
    \caption{\textbf{Skill Disparity in Demographic Distribution} of race/ethnicity and gender. 
    \textbf{(a)} The relative average skill level for Asian, Black, and Hispanic/Latinx, to White workers (a ratio of the pairs), for each skill category, education level, and weekly wages, respectively. 
    \textbf{(b)} The relative average skill level for female to male workers for each skill category, education level, and weekly wages.  
    Bootstrapping subsamples, 95\% confidence intervals are calculated for each estimated ratio (see Methods). Further temporal evolution of skill, education, and wage gaps are included in SI
    % Fig. \ref{fig:Temporal Race Gaps - Skills, Education, Wages} and \ref{fig:Temporal Gender Gaps - Skills, Education, Wages}
    .  
    } \label{fig:Figure 7} 
\end{figure*}


\begin{figure*}[!h]
    \centering
    \includegraphics[width=0.95\textwidth]{Nature_HB_2023/figNHB/Fig_6_-_Jul_11_2023.png}
    \caption{\textbf{Spatial Distribution of Skill Categories.}
     \textbf{(a)} General, \textbf{(b)} Nested, and \textbf{(c)} Un-nested skill levels of each county's occupational composition, using their standard score (z-score) relative to the national level (see Methods). 
     The most populated counties in each state are enclosed in a box, and the top five and bottom five U.S. counties are highlighted in italics.
     There is a noticeable concentration of general and nested skills in densely populated areas, while rural areas demonstrate a higher level of un-nested skills.
     \textbf{(d)} and \textbf{(e)} illustrate the average skill levels (and 95\% confidence intervals) of each skill category in relation to population size and manufacturing industries, respectively.}
    \label{fig:Geography}
\end{figure*}


% [I'm leaving quite a bit of detail on the maps here; we can move most of those details to SI later.]
%% iv. demographic and geographic
%Using the regional employment of occupations published by the current population survey (CPS), one can map the geographical distribution of skills by aggregating occupational employment information to the skill categories in this paper.
% Taking an average for each US county using the county employment of occupations as weights, we derive a regional measure of skill endowment for each skill category.
% The largest cities from each state (in terms of population) are shown as boxed labels, and the top and bottom five counties, in terms of their general skill endowment, are shown in italics.
% Furthermore, we divide cities into four groups based on their population (from the 2010 Census) and show skill levels across groups in Fig. ~\ref{fig:Geography}~(d).
% to estimate the distribution of different skills over four race/ethnicity (\textit{White, Asian, Black/African-American, Hispanic/Latinx}) and gender (\textit{Female and Male}) categories. We infer individuals' skills from the skill requirement of their detailed occupation according to O*NET 2019 and calculate a skill endowment for a given race and gender in each skill subtype. female(skill)/male(skill) mapped on a log-scale so that it shows the ratio out of 1.)


%Moh Hosseinioun: To form confidence intervals, I took samples of 10% from the sub-population of interest (say, when comparing the gender gap: Asian female, and Asian male) and recalculated the ratio of interest, for instance, log(wage asian female)/log(wage asian male).)  I repeated this sampling and calculation process 10,000 times to find the distribution of the estimated value. This way we obtain the distribution of the measure of interest. The distribution gives us the values at 95 percentiles and hence, the 95% confidence intervals.
%To do so, we rely on CPS microData between 1980 and 2022, deriving skill endowments across race/ethnicity (White, Black, Hispanic/Latinx, and White) and gender (Female and Male) groups for full-time employed workers in each skill category.
% You: Yes! BUT why do we need to take log for a ratio? Can we just do (wage asian female)/(wage asian male)? instead of log (wage asian female)/log (wage asian male)?
% Moh Hosseinioun: for wages, not sure why even ratios are log(wage_x)/log(wage_y). This is the convention.


% opportunity cost arguments? nested specializations vs. unnested specializations
% While general skills also show such gaps, gaps are wider in nested skills.
% Such skill gaps persist even though women (apart from Asians) report higher educational attainment.
\subsection*{Disparity in Geographic and Demographic Groups}


We examine skill categories across various demographic groups to gain a more comprehensive understanding of their roles in inequalities. 
Figure \ref{fig:Figure 7} (a) compares skill, education, and wage differences across race/ethnic groups with White peers as the baseline group, suggesting disparities for Black and Hispanic workers are mainly attributed to lower nested specializations. In addition, greater unnested specializations for Hispanic employees also account for this disparity. These findings suggest that closing wage gaps for blacks requires different solutions than for Hispanics. 


Next, Fig. \ref{fig:Figure 7} (b) shows marked differences in gender gaps across social groups. 
The most pronounced disparities are seen in both nested and unnested specializations, which contribute to the wage gap between men and women. Encouragingly, this gap has been narrowing over time, as demonstrated in SI
% Fig. \ref{fig:Temporal Gender Gaps - Skills, Education, Wages}
.
Nevertheless, the disconnection between education and wages is puzzling. Women invest more in education with higher accumulated general skills than men do, with Asian women being a notable exception, but they do not seem to translate well into higher wages. 
This disconnect is perhaps attributed to a lack of occupational specializations, either nested or unnested. These analyses highlight that traditional approaches focusing solely on skill gaps with educational attainment may overlook all relevant aspects of skill disparities.
Although a deeper analysis of the causes and consequences of these disparities is beyond the scope of the current paper, given the complex interaction between wages and skill types, such an analysis may provide valuable insights for labor market policies, considering the intricate interplay between wages and skill types.


Lastly, we present skill maps in Fig. ~\ref{fig:Geography} with the distributions of general, nested, and unnested specific skills across U.S. counties (a-c), cities of various sizes (d), and cities with different concentrations of manufacturing (e) (see Methods).
Broadly speaking, the maps show a clear concentration of general skills in densely populated urban areas, reflecting the diverse and complex economic activities found in these locales \cite{Youn2016, gomez2016explaining, Hong2020, Balland2020, Bettencourt2014, Gomez-Lievano2021}. Large cities tend to have higher levels of general and nested skills. However, the starkest disparities between smaller and larger cities are seen in the prevalence of unnested skills, which are significantly less common in cities with over a million inhabitants, a known threshold for cities transitioning towards innovative economic specializations \cite{Hong2020}.


Upon grouping cities by manufacturing employment relative to the national average, we find that cities highly specialized in manufacturing tend to exhibit lower levels of nested specialization but higher levels of unnested specializations. This shows that cities indeed specialize in distinct directions. Interestingly, skill patterns shift in a non-linear fashion across cities with increasing concentrations of manufacturing employment. Both a strong dependence on and a complete absence of, manufacturing correlate with adverse skill bases, i.e., skill bases dominated by unnested skills and a lower prevalence of general and nested skills. Conversely, skills that typically command high wage premiums are overrepresented in cities with intermediate levels of manufacturing activity.


%Such barriers, in turn, exacerbate the aggregate availability of skilled workers required to staff the growing technological demands and widen the wage gap by restricting labor mobility to better-paid jobs \cite{Goldin2008, AcemogluDaron2012WDHC, Aeppli2022}.
% However, specific skills are more sensitive to local production requirements. 
% Beyond urban areas, the skill map shows the accumulation of general skills in certain other areas, such as St. Mary County (Maryland) or Chatham (North Carolina), which is driven by the need for specialized nested skills.
% St. Mary County is an air force and aerospace hub with companies such as Lockheed Martin, and Boeing, and military naval air station Patuxent River among the top employers, and Chatham, and its neighboring counties, Durham (hosting Duke University), Orange (hosting the University of Carolina at Chapel-Hill), and Person, which have fostered one of the fastest growing tech sectors.
%%----------------------------------------------------------------------------%%

\begin{figure*}[!h]
    \centering
    \includegraphics[width=\textwidth]{Nature_HB_2023/figNHB/Historical_Comparison_of_Skills_-_Jul_14_2023_+_arrows.png}
    \caption{\textbf{Historical Changes in the Skill Structure}
    % \editHY{Idea: switch b and c with a, and de-emphasize the current (a) as a supporting of current histograms (b) and c}
    \textbf{(a)} changed in the demand distribution of skills groups in 2005 and 2019. The arrow shows the shift in the average skill levels from 2005 to 2019. Unlike the positive shifts in general skills, the shift in specific skills is not unnoticeable.
    \textbf{(b)} changes the demand distribution of nested and unnested skills in 2005 and 2019. The arrow shows the shift in the average skill levels from 2005 to 2019. Unlike the nested that follow the shift in general skills, the demand for un-nested skills has fallen.
    \textbf{(c)} compares the skill hierarchy structures between 2005 and 2019. The changes in the structure of the skill hierarchies over time highlight an increasing divide in the dependencies of nested and un-nested skills and the widening gap.}
    % Most skills (such as \textit{Oral Expression, Social Perceptiveness, Science, and Dynamic Flexibility}) preserved their position in the (y-axis of the) hierarchy.
    \label{fig:historical skill change}
\end{figure*}
% Nested jump: social scientists and economists  on the one hand, and Telemarketers and Technical Writers on the other hand? We can mention,  as a side note, the CEO and Human resource.
% Occupations of high nested/ and changed higher over the decade: Medical and Health Services Managers become more nested while Computer User Support Specialists become unnested.  Medical and Health Services Managers(nestedness change: 1.0278089) and Computer User Support Specialists (nestedness change: -0.9725729).  




%%-------------------------------------------------------------------------%%
\subsection*{Widening gap in the skill structures}

Figure \ref{fig:historical skill change} paints a somewhat concerning picture, given the important roles that nested and unnested specializations play in both career progression and demographic and regional disparity. 
Fig. \ref{fig:historical skill change} (a) indicates an increased demand for general skills (the shift from dotted to solid distribution) and higher wage premiums over recent decades (see SI
% Fig.~\ref{fig:Wage and education 2003}
).
Moreover, the seemingly static distribution of specific skills belies a countervailing specialization pressure in the opposite direction, an increase in nested specializations, and a decrease in unnested specializations, shown in Fig. \ref{fig:historical skill change}. 
At the same time, the skill structure has become even more nested, decreasing the checkerboard score and temperature from 438.67 to 356.4 and from 40.07 and 31.89, respectively, and increasing NODF and $N_c$ from 39.06 and 573,873 to 41.72 and 651,030, respectively, between the years 2005 and 2019 \cite{Stone1990, Almeida-neto2008}. 
Note that the lower the checkerboard score and the higher NODF indicate the more nested the structure. 
As a result, the skill dependency structure in Fig. \ref{fig:historical skill change} (c) illustrates a wider gap between the nested and unnested branches over the decades. 
The nested branches widen horizontally with increasing depth, suggesting an increased complexity and interdependency in specializations \cite{DemingDavidJ2020EDCJ, Tong2021}.


Echoing previous studies, over the last two decades, the chasm between the two types of specializations has alarmingly broadened within the educational domain \cite{Xu2021, Lin2022, Althobaiti2022}. 
Given the significance of nested and un-nested specializations for future career developments and wage premiums, structural changes in the hierarchical tree network are concerning, as they reveal an economy growing more nested and specialized with
an increasingly widening gap between the two, implying strongly rooted chronic disparity. 

% Network structure: 
% horizontal interdependencies among specific skills have remained largely unchanged.
% For instance, Biology, Medicine, and Chemistry remain in the relatively same positions, evident in the stronger concentration of general skills at the center of the nested community.
% examples, Oral Comprehension and Information Ordering, using computers?
%Despite the growing importance of social skills \cite{Liu2013}, our findings suggest that they do not drive the effect of general skills (or any other skill subtype). Instead, they hinge on a broader array of general skills (see SI Section \ref{fig:social skills}).



% specialization has increased the depth of knowledge in each domain, increasing the length of a typical learning trajectory from basic to advanced topics \cite{BenJ2009}. 
% more effort before a novice can become an expert. % increasing the depth between the basic and the advanced topics.

%%--------------------------------------------------------------------------%%
\section*{Discussion} % role of MBA program? ;) 
%The central goal of this paper is to uncover the underlying dependency structure among skills used in the U.S. labor market by analyzing a large-scale skill survey that documents the human capital requirements of hundreds of U.S. occupations. The resulting structure with two distinct skill communities is not only in good agreement with a distinction between manual and cognitive skills but also adds a new dimension to the structure, directionality, to what had been shown in prior research \cite{Alabdulkareem2018, Frank2018, Moro2021}. Adding directionality to the network, connecting skills according to inferred prerequisite-dependency relations, yields a   hierarchical network that suggests a different skill taxonomy, highlighting the difference between general skills, on the one hand, and nested and un-nested specific skills, on the other. Remarkably, this taxonomy emerges from information only on how skills co-occur within occupations, without using any substantive information on the skills or occupations themselves. The derived taxonomy therewith combines elements from a recent body of work on economic complexity that represents skill structure as networks \cite{Neffkeeaax3370, AndersonKatharineA2017Snam, Alabdulkareem2018, Xu2021, Lin2022, althobaiti2022, DelRio-Chanona2021, Tong2021} and traditional research in economics on human capital specificity \cite{Gibbons2004, Poletaev2008, Gathmann2010, BeckerG.S1992TDoL}. 



% LT Zhang: A well-established concept in economics is general versus firm specific skills. The former refers to skills that are useful across firms and the latter are those useful to only a particular firm. Our construct treats all skills on a continuum, ranging from the most foundational and general to the most nested and specialized. In our framework, firm specific skills would include those that are extremely specialized skills, such that they only apply to a particular firm. 
% cite: Lazear, Edward P. "Firm-specific human capital: A skill-weights approach." Journal of political economy 117, no. 5 (2009): 914-940\cite{Lazear2009}
% The most interesting finding to me is perhaps the importance of combining foundational and nested/specialized skills, especially at the higher level. I believe this is an important point and resonate well with other work showing the increasing role of social skills in recent years (especially at later stage of one's career). 
%Skilled workers represent an important class of capabilities, making understanding the process of skill acquisition, deepening, and diversification closely connected to the fundamental narrative of the literature on economic complexity.
 %The former is widely useful, while the latter raises productivity at only a particular firm, not elsewhere, thus setting up a bilateral monopoly situation between the worker and firm.
%Accumulation patterns differ across skill types: general and nested skills tend to grow over most of a worker’s career, while reliance on un-nested specific skills decreases with age. 


% As economic complexity grows, the nature and intensity of skill dependencies evolve correspondingly. 
% \cite{Borner2018}

The hierarchical structure and its inherent \emph{directionalities} add a new dimension to the rising field of economic complexity, providing a deeper understanding of how knowledge is accumulated within a population and how it is expressed in the economic activities of a firm, city, region, or country  \cite{Hidalgo2015, OClery2021, Balland2020, Hidalgo2021, Harmand2015, Hausmann2011, Hidalgo2007, Hidalgo2009,tacchella2012new, Park2019}. 
Their specializations, contingent on their unique existing capabilities and circumstances, will thus lead to idiosyncratic paths. And yet, observed developmental paths seem to recapitulate the universal path of related diversification of so-called product or industry spaces \cite{Hong2020}. In addition, while the network representations in these studies are mostly undirected, diversification seems to follow a direction, increasing in complexity \cite{mcnerney2021bridging, OClery2021, yang2022scaling, hidalgo2018principle}.  The directional dependencies that we propose break the symmetry in traditional co-occurrence networks for a better understanding of structural changes in economic complexity. 



%%%% robust to management! %%%%
%%% mention the descriptor here and also sec 1. We identified six \textit{social skills} in the list. They are \textit{Social perceptiveness}, the skill of being aware of other's reactions and understanding why they react as they do; \textit{Coordination}, the skill to adjust actions in relation to others' actions; \textit{Persuasion}, the skill to persuade others to change their minds or behavior; \textit{Negotiation}, bringing others together and trying to reconcile differences; \textit{Instructing}, the skill to teach others how to do something; 
%and \textit{Service orientation}, actively looking for ways to help people.

In increasingly complex, large teams, social skills become crucial when specialization necessitates workers to coordinate with team members possessing different specialized skills \cite{Wuchty2007, yoon2023, Neffkeeaax3370, Wu2019Science, Borner2018}. Indeed, social skills and managerial occupations have been increasingly valuable, especially later in one's career with higher wage expectations \cite{Liu2013, Deming2017, Borghans2014, Weinberger2014, Lindqvist2011, Kuhn2005, Deming2018, kogan2021, VanderWouden2023}.  
Our framework identifies and locates them in the skill structure along with general and nested skills (see SI
% Fig. \ref{fig:social skills}-a
), explaining their recent growths and significant role in wage premiums (see SI
% Fig. \ref{fig:social skills}-b
). 
Nevertheless, our results go beyond what social skills and managerial occupations contribute to wage premium, meaning the results are robust to their absence in analyses (see SI
% Sections \ref{sec:social skills} and \ref{sec:robustness check: no managers}
). 
Therefore, social skills are valuable not just because of their role in sociality but because of their structural importance serving as foundational building blocks of human capital, thus increasing the absorptive capacity to enable further valuable specialization and thus more complex organizations \cite{Muthukrishna2014}. 


The nested structural categories also add a new dimension to the theory of human capital \cite{Lazear2009}. 
By differentiating general human capital into a structured spectrum, from the most foundational and general to the most specialized, human capital is comparable at different scales of organizations, which is essential for policy implications. 
More importantly, nested structure reiterates valuable insight that skills are not acquired in isolation. 
This perspective challenges the simplified view of acquiring general skills early in life, later supplemented by specialized skills. 
Life-long learning requires more than just acquiring new specialized skills; it necessitates complementing such investments with deepening one’s general skills, a point echoed by recent recommendations from Stanford and Harvard bringing back math courses over applied alternatives like data science and statistics \cite{Feinstein2023}.
To learn from someone you interact with, you must know the foundational knowledge as a prerequisite for the knowledge being learned with fluid reasoning abilities, known as the ``Flynn Effect" \cite{Clement2022,Hermo2022}.


Finally, just as skill acquisition at the individual level is embedded in a hierarchical web of prerequisite dependencies, so may capability acquisition at the macro level of economies such as firms, cities, and nations. 
This perspective bridges the micro-level of jobs and wages with the macro-level of the economic system as a whole. 
This intrinsic learning sequence suggests that economies may be unable to follow optimal growth trajectories \cite{FinkTMA2019Hmcw, Fink2017a, Fink2017b}. 
As a result, it uncovers structural disparity across different social groups that would have remained underappreciated, had we relied on information on educational attainment or wages \cite{Autor2014}.  


%There are several limitations to our study. 
Throughout the paper, we implicitly assumed that the additional dimension of directedness is uncovered without relying on other occupational characteristics, such as required education, wages, or regional and demographic factors. 
This is partially true for practical purposes and requires further scrutiny with detailed data and qualitative complementary studies.
The dependency structure is incomplete and could be confounded by extraneous factors for skills to co-occur in occupations, providing a limited mechanistic explanation. 
The comprehensive understanding calls for detailed observations of individuals' skill endowments, for example, by conducting extensive surveys of employees rather than jobs asking how they navigate a delicate balance accounting for the required complexity, complementarity, and coordination among heterogeneous skills, while also considering the ease of skill acquisition by individuals \cite{Clement2022}. 
In addition, our data primarily describe the U.S. labor market, which has idiosyncracies in its education system, industrial composition, and urban structure. 
How well these findings generalize to other work settings and economies, especially those at different stages of development, remains a task for future research \cite{Autor2022}. 






%In general, job skill requirements must navigate a delicate balance. 
%They need to account for the required complexity, complementarity, and coordination among heterogeneous skills, while also considering the ease of skill acquisition by individuals. 


%One way education systems can achieve progress is by speeding up the acquisition of general skills. 
%The possibility of this will be clear to anyone who compares textbooks of basic calculus to the original texts on which they build. Simplifying the exposition in teaching basic skills will allow for the further deepening of these skills and therewith provide the basis for the nested specific skills that allow for a deepening division of labor.

% In order to produce the most complex products, national economies must accumulate a web of interdependent capabilities \cite{Hidalgo2009, Hausmann2011, Hidalgo2007}.
% There is a directionality of acquiring capabilities \cite{OClery2021, Imbs2003}
% Such capabilities allow "complex" economies to dominate the production of the most valuable products.
% Growing complexity and specialization similarly motivate examining the underlying structure of human skills \cite{DemingDavidJ2020EDCJ, Tong2021}.
% % [[Placeholder: we will analyze statistical significant link from co-occurrence data to extract interdependency out of hierarchy of abundance]]
% On the one hand, specialization has increased the depth of knowledge in each domain.
% % , expanding what we call \textit{vertical skill interdependence}.
% In other words, specialization has increased the length of a typical learning trajectory from basic to advanced topics \cite{BenJ2009}. % more effort before a novice can become an expert. % increasing the depth between the basic and the advanced topics.
% On the other hand, specialization has accompanied a growing complexity of economic tasks, requiring a diversity of skills and knowledge \cite{Hidalgo2021}.
% % Nonetheless, because job tasks often require several skills, workers must accumulate 'appropriate' bundles of skills, and as a result, horizontal skill interdependence has also intensified.
% That is, increasingly complex job tasks have necessitated the accumulation of 'appropriate' bundles of skills \cite{OClery2019, Hidalgo2018}.
% % , which we call \textit{horizontal skill interdependence}.
% % As a result, the presence of a diverse set of complementary skills, what we call \textit{horizontal skill interdependence}, has also intensified.


% In particular, we do not know if one actually acquires or advances general skills when they move up the occupation level (acquiring the more specific skills).
% The structure in Fig~\ref{fig:Figure 2} is constructed by cross-sectional conditional probability and thus does not guarantee longitudinal occurrences in individual acquisitions.

% Workers' limited attention, time, and resources force them to trade off skill diversity and specialty \cite{Leung2014}.
% [alternative paragraph (to be modified with the previous and the latter)] 
% Vitally, forces of diversity and specialty create an inherent trade-off in the choice of what skills to bundle \cite{Leung2014} which affects workers who must invest time and effort in skills, policy-makers interested in improving aggregate and individual welfare, and social and economic agents who desire an optimal and fair distribution of work.
% Knowledge and technological advances make it impossible for an individual to contain the depth of knowledge at once \cite{BenJ2009}, requiring more investment to reach advanced levels.
%On the one hand, a growing literature highlights the importance of bundling complementary skills \cite{AndersonKatharineA2017Snam, Alabdulkareem2018, Neffke2013, Balland2020}.
%On the other hand, however, the literature on specialization advocates the productivity gains of specialized skills \cite{Mincer1974, Lucas1988, Gibbons2004, smith1937wealth, Frank2018, BeckerG.S1992TDoL, Poletaev2008, Gathmann2010}.
%Further complicating matters is a stream of research that champions the importance of the less specific skills \cite{Hermo2022, Liu2013, epstein2019range}, citing the "Flynn effect" as a sign of the increasing value of more fluid skills over crystallized knowledge \cite{FlynnJamesR.2018IdaP}.
% In conclusion, uncovering the skill structure and articulating the relation between workers’ skills and value requires disentangling forces of complementarity and specialization.

% The skill interdependency web creates a different level of the value of skills, results in job inequality/disparity, and explains the knowledge diffusion process \cite{Moro2021, AndersonKatharineA2017Snam, Alabdulkareem2018, Li2021, Tong2021}. 
% In order to produce the most complex products, national economies must accumulate a web of interdependent capabilities \cite{Hidalgo2009, Hausmann2011, Hidalgo2007}.
% There is a directionality of acquiring capabilities \cite{OClery2021, Imbs2003}
% Such capabilities allow "complex" economies to dominate the production of the most valuable products.



% Existing work has discovered that performing job tasks simultaneously requires several (complementary) skills \cite{Alabdulkareem2018}, and that complementarity of skills, what we call \textit{horizontal interdependence}, increase the overall workers' value \cite{AndersonKatharineA2017Snam, Neffkeeaax3370}.
% Nonetheless, our understanding of complementarity has relied on skill's co-occurrences in occupations \cite{Alabdulkareem2018}, neglecting the asymmetry in skill dependence.
% Intuitively, the investment necessary, hence the payoff, to acquire a skill increases with the number of its prerequisites \cite{Mincer1974, Lucas1988, Becker1962, Schultz1961}.
% We call such dependency on prerequisites \textit{vertical interdependence}.



% The central goal of this paper was to uncover the underlying dependency structure among skills that are used on the US labor market. Our analyses of multiple data sources, we have empirically derived a classification of skills consistent with the notions of \textit{general} and \textit{specific} human capital \cite{Becker1962}. Moreover, we have constructed a skill hierarchy based on the conditional probabilities of skill co-occurrences across the economy. This hierarchy resembles an unbalanced tree, with general skills serving as common prerequisites for some of the more specific skills. The emergence of this skill tree stems from the semi-nested structure of occupational skills, which introduces heterogeneities in the web of dependencies across different skills. Our framework distinguishes skills such as \textit{programming, medicine, and physics} that rely on a dense and nested web of prerequisites, from other skills like \textit{repairing, stamina, and construction} that lack such a structure.
% Skill interdependence and nestedness also carry implications for aggregate economic growth.
% Growing complexity and specialization similarly motivate examining the underlying structure of human skills \cite{DemingDavidJ2020EDCJ, Wilmers2020, Tong2021}.
% Some knowledge requires a basis to travel from one human to another\cite{Coscia2020, VanderWouden2023}.
% While others may not need such a basis.
% While specialization creates challenges in cross-occupational communication and knowledge transfer \cite[ch. 3]{Arrow1974book}, a strong stock of general skills common among occupations, will reduce the cost of transmitting information, and as a result, collaboration and complementarity of workers of different specialties \cite[ch. 2]{Arrow1974book}.
% Our findings suggest investments in certain skills and knowledge, therefore, help further accumulate them as if \textit{capital} \cite{Becker1962, Schultz1961}.
% This is consistent with the empirically observed increases in the education premiums \cite{Janzen2022} and the role of general skills as inter-occupational complementarity \cite{Neffkeeaax3370}.
% In addition, an important part of accumulating human capital is specific to the firm \cite{BeckerG.S1992TDoL, Wilmers2020}, wherein specialized workers, among other things, learn about the firm's communication codes \cite{Arrow1974book}. 
% Therefore our evidence of increased general skills with higher education indicates firms offer premiums to workers with higher education to compensate for their contribution to the intra-firm complementarity.
% % Relatedly, the very basis that facilitates knowledge transfer may act as a communication channel that enables complementarity between workers endowed with the basis however different in their specialized skills \cite[ch. 2]{Arrow1974book}.

% Where does the structure come from, and in what form if it exists? 
% On the one hand, growth in knowledge and know-how leads to longer chains of prerequisites within each specialty branch. 
% Suffering from a heavy knowledge burden on us \cite{BenJ2009}, we specialize to become more productive \cite{smith1937wealth, Wilmers2020} and able to make more advanced products \cite{Hidalgo2021}, increasing our bargaining power \cite{Davidson1898book, Fini2022, FergusonJohn-Paul2013}.
% %% cited work Wilmers2020, in fact, states the opposite that: certain specializations may reduce bargaining power.
% On the other hand, the products (and technologies) have become even more complex, branching off skills into many specialty prongs. Undertaking complex tasks that our modern society needs requires specialized skills, knowledge, and coordinated workforces \cite{Hidalgo2015}.  
% %Therefore, we expect the skill structure to extend both in depth, as a tree gets taller to represent the growth of collective knowledge, and in breadth, a branches of a tree would to represent growing diversity. 



% \cite{VanDam2021} finds that in certain ecologies, two disjointed sub-ecosystems may form that share few required resources. This resembles our nested and un-nested structure. A linkage to the paper and the idea can attract people from that domain.]


% Our geographic and demographic analyses further corroborate the dependence of valuable nested skills on general skills and highlight the unequal distribution of general skills across localities and racial groups. These disparities contribute to the widening wealth distribution \cite{Hong2020} and increased rigidity of social classes \cite{Goldin2008}. Regiones and sub-populations with bundles of general and nested skills are better positioned to engage in complex activities and achieve higher economic outcomes. However, such skill bundling may create barriers to knowledge transfer in regions and demographic groups, lacking the necessary foundation for acquiring nested skills. As economic well-being becomes an increasingly important factor in education \cite{Chau2023, Li2021}, the existing distribution of skills may shape reinforcing factors in the possibility of reskilling and future wealth distribution. Therefore, policymakers must recognize the overlap between factors that create value and those that contribute to the widening gap between socioeconomic strata. %Our findings have implications for planners and policymakers seeking to mitigate the impacts of technological job changes. 


% %% complexity %%
% Complexity and specialization have been the two sides of the coin of economic growth.
% These dual forces have allowed the organization of work and distribution of knowledge so that the collective can produce more diverse and complex products and services \cite{Hidalgo2021, Hidalgo2015, Harmand2015, Hausmann2011, Hidalgo2007, Hidalgo2009}.
% Society has become so complex that not a single person can make a pencil \cite{Read1958, Dubner2016}. 
% % Can you add pencil example and references here? I will revise them later. 
% However, they have also coincided with structural deficiencies in the form of increasingly unfair wealth distribution.
% While specialization has increased the length of a typical learning trajectory from basic to advanced topics \cite{BenJ2009}, the growing complexity of job tasks has necessitated the accumulation of more skills \cite{OClery2019, Hidalgo2018}, adding to the burden of bundling skills.
% Because workers possess a limited time budget and cognitive capacity, they must trade off specialization and diversification.

% Our skill structure bridges the findings of a recent body of work on economic complexity that represents skill structure as networks \cite{AndersonKatharineA2017Snam, Alabdulkareem2018, Xu2021, Lin2022, althobaiti2022, Neffkeeaax3370, DelRio-Chanona2021, Balland2020} and the economic stream on human capital \cite{Mincer1974, Lucas1988, Gibbons2004, smith1937wealth, M.R.2018, BeckerG.S1992TDoL, Poletaev2008, Gathmann2010}.
% % , and the pyschological literature on intelligence \cite{} about the value of skills
% While the economic complexity stream focuses on the diversity and complementarity of skills, labor economic research emphasizes the directionality of dependence that arises from specialization.
% % In contrast to the latter, the psychological literature advocates the importance of more common components of human skills.
% Our work considers skills such as \textit{programming, medicine, and physics} are bundled along the depth of the skill hierarchy.
% These are nested skills as they cannot be obtained without sufficient stock of their prerequisites.
% The fact that they come in together in the career journey (i.e., resumes) implies they are bundled to accomplish complex job tasks.
% Indeed, we find that the jobs that require nested skills obtain the highest economic return.
% As such, drawing from both mentioned streams, we argue jobs with nested skills carry both vertical (specialization) and horizontal (complementarity) skill interdependence.
% In contrast, jobs that mainly rely on un-nested skills lack depth, and as such, do not command high wages.


% Similarly, the skill structure has implications for science and technology regarding how science is needed for technological progress \cite{BenJ2009, Sorenson2004}.
% Put differently, there is directionality among components of the analogies such as  Lego blocks or Alphabets in the Scrabble \cite{FinkTMA2019Hmcw} for the economic building blocks.
% % analogies are not independent and equal.
% % There is a directionality among them.
% We may have to learn 'A' before learning 'Z' to solve the analog of a Scrabble game in economic growth.
% This is consistent with prior findings that national economies that can produce the most complex products also produce simpler products, while the reverse is rarely observed \cite{Hidalgo2007}.
% Furthermore, this intrinsic learning sequence may imply agents may not be able to adapt optimal growth trajectory even if they wanted \cite{FinkTMA2019Hmcw, Fink2017a, Fink2017b}.
% The nature of nestedness will introduce a nontrivial diffusion process in product space \cite{Hidalgo2009}. Knowledge cannot travel without prerequisites \cite{Balland2020, Moro2021, AndersonKatharineA2017Snam, Alabdulkareem2018, Li2021, Tong2021}. % (some knowledge should be bundled to travel, not alone). \\
% That is why in order to produce the most complex products, national economies must accumulate a web of interdependent capabilities \cite{Hidalgo2009, Hausmann2011, Hidalgo2007}.
% Such directionality in acquiring capabilities \cite{OClery2021, Imbs2003} allows "complex" economies to dominate the production of the most valuable products.






% %% limitation %%%
% Of course, our study has limitations that need to be considered. Our empirical findings may not provide a direct answer to why some skills bundle in an occupation or why some skills are needed together in an occupation [a good place to include bargaining and other micro literature on skill acquisition]. The observed skills bundled together many times in a single occupation must be the most productive recipe for economic activities under constraints. The constraints include transaction costs among the associated complementary tasks that trade-off with a single human's limited capability, such as limited attention, time, and resources \cite{Leung2014}. There is only so much we can learn within a working cycle, no matter how high the desired bundle's payoff would be. We have known that when the skills and knowledge are too complex to be bundled within a person, we form teams or firms to \textit{organize} (tacit and non-tacit) knowledge needed to accomplish tasks \cite{Ahmadpoor2019, Wu2019Science, Milojevic2014, Boerner2010, Hunter2008, Wuchty2007}. Our work shows that some types of knowledge are more similar to each other to be learned easily, such as English and German, compared to English and Korean. 

% The structure we find is incomplete and does not isolate the pre-requisition dependence of skills from other possible mechanisms. Nevertheless, we found and presented evidence that the hierarchy is a good first approximation to the hidden fundamental structure in the landscape of human capital.
% ** We should discuss the caveat that our claim that \textit{specific skills require general skills} comes with a lack of micro-level evidence of temporal acquisition of skills: do general skills come before specific skills? Ideally, we can observe this as individuals acquire skills, but we can approximate it with the survey. [include in survey LT is preparing]
% And yet our framework holds.

% % Our findings carry implications for a broad range of fields, from psychology to policy. 
% %We find that the general skills of occupations carry significant predictive power in determining wages. 
% The importance of general skills in enabling the execution of complex tasks is a factor often overlooked. Particularly, many retraining programs are designed around instilling ``valuable'' specific skills. Our results suggest stocks of specific skills on their own are insufficient to create sustained value, and thus the general skills of occupations carry significant predictive power in determining wages. Instead, a longer-term approach to facilitating education availability for acquiring general 
% and nested skills may be required. 
% We provide robust empirical evidence for the growth of general skills in individuals' careers through resume data, the increased importance and level of general skills over time when comparing occupational skills from 2005 to 2019, and an analysis of a synthetic age cohort, which reveals that occupations with a higher median worker age exhibit higher general skills. 
% The observed increase in general skills aligns with the "Flynn Effect" \cite{Hermo2022}, which describes the rise in fluid reasoning abilities in developed countries since the twentieth century. 
% Indeed, there has been a growing recognition of general skills, including education. For instance, in 2022 Stanford, and in 2023 Harvard updated their recommended preparations for their prospective applicants, emphasizing conceptual math courses such as algebra and geometry and de-emphasizing applied alternatives like data science and statistics with the reasoning that these courses are ``not equal'' in how well they prepare individuals to progress in different specialties \cite{Feinstein2023}.
% As such, we also offer an alternative explanation for the empirically observed stagnant payoff to education \cite{Pritchett2001, KruegerAlanB2001EfGW, HanushekEricA2008Troc} based on the necessary co-development of general and nested skills.


% To summarise, we have formed a topology of the skill structure, wherein skills' generality measured by total level across occupations is plotted against two different network measures of prestige (Hub score \cite{} and Pagerank \cite{} on an extracted directional backbone \cite{Jo2020}) and skill's correspondence with general skills as measured by the average Pearson correlation of the focal skills and general skills.




% Put differently, Lego Block and Alphabets in the Scrabble analogy are not independent and equal. There is a directionality among them. We may have to learn 'A' before learning 'Z'. And this intrinsic learning sequence may imply why we cannot adapt the optimum strategy even if we want \cite{FinkTMA2019Hmcw, Fink2017a, Fink2017b}. 



% It may have implications for science and technology in terms of how science is needed for technological progress (ref Ben Johns, US. Patent and Science papers). 

% 0. Implication in product space/knowledge diffusion. The nature of nestedness will introduce a nontrivial diffusion process in product space \cite{Hidalgo2009}. Knowledge cannot travel without prerequisites. % (some knowledge should be bundled to travel, not alone). \\
% Some knowledge requires a basis to travel from one human to another.
% While others may not need such a basis.
% Relatedly, the very basis that facilitates knowledge transfer may act as a communication channel that enables complementarity between workers endowed with the basis however different in their specialized skills \cite[ch. 2]{Arrow1974book}.

% 1. There are skills that are bundled in Fig \ref{fig:skills cor wage, total generality, cor gen skills} those are nested skills. The fact that they come in together in the journey of career (resume) implies they are bundled to accomplish a complex job. We find that these jobs that are bundled together give the highest economic return. This is perhaps how human species has become successful (?)\\
% 1. There are skills that are bundled along the depth of the skill hierarchy. These are nested skills as they cannot be obtained without sufficient stock of their prerequisites. The fact that they come in together in the career journey (i.e., resumes) implies they are bundled to accomplish a complex job. We find that the jobs that require nested skills give the highest economic return. %This is perhaps how human species has become successful (?)\\
% What does this mean for economic growth and equality?

% 2. Our geographic and demographic analyses show disparities in the distribution of such general skills.
% As a result, on the one hand, locality and sub-population endowed with bundles of nested skills may be in a better position to undertake complex activities and gain higher economic outcomes.
% On the other hand, such bundling may forge barriers to transferring such valuable knowledge to geographies and demographic strata that lack the required basis to obtain nested skills.
% As economic well-being becomes an increasingly important factor in education \cite{}, the existing distribution of skills may shape reinforcing factors in the possibility of reskilling and the future distribution of wealth.


% 3. potential stretch to science and technology
% Even technological advances fueled by basic science 
% \cite{Sorenson2004} %(Sorenson and Felming 2004)

% 4. Can we explain the "Flynn Effect?" Increases in fluid reasoning appear to emphasize our general skills. Find relevant info in \cite{Hermo2022}



%%
%%----------------------------------------------------------------------------%%
% https://www.bls.gov/oes/
% O*NET also includes work activity items we disregard as they associate more closely with work tasks than worker skills \cite{Nedelkoska2018}.  

\section*{Data and Methods} \label{sec: method}

%%--------------------------------------------
%O*NET consists of job-oriented attributes and worker-oriented descriptors, respectively. Job-oriented attributes include wage, employment, educational requirement, workplace experience, and training. Worker-oriented descriptors include 120 work-relevant knowledge, abilities, and skills (labeled \textit{skills} throughout the text for brevity)

%Occupation units for both datasets follow the Standard Occupational Classification (SOC), the federal statistical standard used by federal agencies to classify workers into occupational categories. However, their granularity differs by the dataset. While BLS categorizes occupations by 6-digit SOC codes (774 unique occupations), the O*NET detailed them into 8-digit SOC codes (968 unique occupations).

%There are two ways to measure skill levels required for effective work performance: the level of sophistication (or intensity) (ranging from 0 to 7) and the importance or vitality of those requirements (ranging from 1 to 5).
%These two skill measures are highly correlated (0.94); thus, the results are robust to the choice of measurements.
%We chose the skill level for the main text because we reason the level that captures the necessary sophistication and adeptness is more consistent with our hypothesis of skill progression and hierarchy.
%Still, the correlation and the fundamental classification of skills are robust to using importance measures— see supplementary information.

\subsubsection*{Skill Generality Categories} 
%%---------------Occupational-specific information is obtained from two datasets: the Bureau of Labor Statistics (BLS) and the Occupational Information Network (O*NET) \cite{Peterson1999ONET}.
%%% For example, the skill "speaking" is important for both lawyers and paralegals. However, lawyers (who frequently argue cases before judges and juries) are required to have a higher Level of speaking skill, while paralegals only need an average Level of this skill. https://www.onetonline.org/help/online/scales#foot2
%%%%
The Occupational Information Network (O*NET) includes survey records of job-oriented attributes and worker-oriented descriptors \cite{Peterson1999ONET}. Job-oriented attributes include wage, employment, educational requirement, workplace experience, and training. Worker-oriented descriptors include 120 work-relevant knowledge, abilities, and skills (labeled \textit{skills} throughout the text for brevity).
Each occupation includes a list of skills with their sophistication levels (or intensity) and the importance of those requirements, each resulting in an occupation-skill matrix. 
Although we chose the former because of our hypothesis of skill progression and hierarchy, the two measures are highly correlated (0.94); thus, the results are robust to the choice of measurements.

Each skill, therefore, lists the required levels for each occupation whose distribution shape illustrates its generality across occupations, shown in Fig. 1 (a). 
As such, we group skills by their similar distribution shapes by $k$-mean clustering algorithms with correlation metrics. Figure 1 (b) shows the characteristic shapes of each skill group.  
We provide three statistical tests for optimal $k$ and show the findings are qualitatively robust to some variations (see SI
% Sec. \ref{supsec:skill clustering}
). Throughout analyses, we mainly analyze the effects of general and specific skills to filter possible noises. 

In addition to shapes and averaged skill levels (see the main text), we provide 
local reaching centrality as an alternative measure for skill generality, and use them to vertically embed nodes in Fig. 2 (b). 
The local reaching centrality is defined as the proportion of the skill hierarchy structure that is reachable from a skill via outgoing edges \cite{Mones2012}. The higher reaching centrality in the hierarchy structure is, therefore, the more interdependent skills. As such, this measure offers additional indicators of skill generality. 
% LRC is calculated by networkx global_reaching_centrality with an option for local.


%%--------------------------------------------
\subsubsection*{Conditional Probabilities for Skill Hierarchy Structure}


The conditional probability that infers the directionality operates on binary values, but skill levels are recorded in continuous variables. We, therefore, apply the disparity filter (statistically significant links keeping heterogeneity of node degrees) to an occupation-skill matrix whose entry is a required skill level for each occupation \cite{Serrano6483}.
Parameters are chosen such that i) the rank of skill terms in the strength (from the weighted network) and degree (in the binary network) is preserved, ii) the rank of occupations' skills of each category in the weighted network is preserved in the binary network (see SI
% section \ref{supsec:conditional dependencies}
) discusses details and compares the state of data before and after the transformation.

We then calculate conditional probabilities of every pair of skills in the matrix to infer directions between two skills, as illustrated in Fig. 2 (a). 
Figure 2 (b) presents a backbone structure of the aggregated all skill pairs, according to \cite{Jo2020}.  
Please see SI
% section \ref{supsec:conditional dependencies} 
and \cite{Jo2020} for the detailed procedures and choices of parameters and thresholds. 

% First, we use a disparity filter to make a binary network of only statistically significant edges relative to randomness.
% Second, the direction and strength of the dependency using conditional probabilities $P(u|v)$ and $P(v|u)$
% for $P(u|v)<P(v|u)$ and directed edge $u \rightarrow v$ with a weight $\alpha(u,v) $.
% We exercised caution in interpreting the outcome of this step as $v$ being a dependent of $u$.
%\begin{equation} \label{eq: LRC}
    %C_R(u) = \frac{1}{N-1}\Sigma_{v:0<d^{out}(u,v)<\infty}(\frac{\Sigma_{k=1}^{d^{out}(u,v)}\alpha^{(k)}(u,v)}{d^{out}(u,v)})
%\end{equation}
%Where $N$ is the number of nodes in the network, $d^{out}(u,v)$ is the length of the directed path that goes from $u$ to $v$ via out-going edges, and $\alpha^{(k)}(u,v)$ is the weight on the $k^{th}$ edge along that path, as it is derived in the skill hierarchy.
% To calculate the number of paths of length $l$, we employ the weighted-directed adjacency matrix, $M$, and raise it to the power of $l$, yielding $M^l_{i,j}$.


\subsubsection*{Reacheability with Arrival Probability} %--
To quantify what are the chances of getting to the focal skill $j$ given the pre-requisite skill $i$, we calculate reachability from one skill to a focal skill.  It is basically arrival probability, or a version of hitting probability, of a random walk \textit{arriving} at $j$ from node $i$ given 
the weighted skill dependency network \cite{norris1998markov}.
For source and target skills $i \neq j$, this is numerically equivalent to first deriving the probability of random walks of length $l$ by raising the weighted-directed adjacency matrix (skill dependency network in Fig. 2), $M$, to power $l$, and then calculating  $R_{i,j} = \Sigma_l M^l_{i,j}$.
We obtain the final arrival probability by summing over a sufficient number of path lengths until reaching saturation points. To compute arrival probabilities for focal skills (such as programming, negotiation, and repairing) in Fig 2 (b-f), we apply the R package \textit{markovchain} \cite{MarkovchainRPackage}.

% Note that: The package uses skill-occ, not occ-skill matrix (where skills and occupations correspond to species and sites in ecology). Note that the skill-occ matrix is the matrix after the disparity filter because the package needs binary entries. 
% Nestedness: Coordination, AdministrativNODFe & Management, Social Perceptiveness, Service Orientation.
% Given the form $c_s$ takes, we use the threshold of 0. Therefore, we designate skills with nestedness contribution beyond the threshold as \textit{nested} and others \textit{un-nested}.
% We use checkerboard score as our primary measure, but NODF also gives good agreement.

%\subsubsection*{Nested Structure in Skill Hierarchy} 
\subsubsection*{Nested and Unnested Skill Categories} 
Nestedness is a structural characteristic that describes interactions in an ecological system, where specialist species often interact with a subset of generalists. 
Unlike ecological systems, however, in SI
% -Fig. \ref{fig:occ_skill_nestedness_mat}
we show the skill-occupation matrix is a noisy nested structure far from the perfect upper-left triangle when sorted by marginal totals (fills).
This imperfect nested structure may account for the constraints on occupations (limited carrying capacity), introducing severe competition between skill species. Indeed, in SI
% -Fig. \ref{fig:occ_vs_skill_importance_avg_cos}
 we show unlike broad skill generality, occupation's scope is narrowly distributed, indicating 
that the total amount of skill levels embodied in an occupation is not much different from each other, regardless of how much they are paid and how advanced education is needed (see SI
% Sec. \ref{suppsec:nestedness}
).


We attribute occupations' limited scope of skills to the limited attention and cognition/physiological capacity that individual workers can offer. There is only so much a single person can equip and do for a single job \cite{BenJ2009, DUNBAR1992}. Thus, individuals' capacity restricts how many skills an occupation can bundle. This constraint explains the process of specializations needed for a complex job. The structure now includes not only nested structure but also mutually exclusive presences, possibly due to competition between skills within an occupation. 
In contrast to occupations, skills do not have such constraints. Therefore, for limited occupation scope, we only consider the skills' contribution to nested structure.  

 
This constraint distinguishes the nestedness of extensive economies such as nations, regions, and urban areas from the nestedness of occupations in that specializations dominate the evolution labor market while others are dominated by diversification. 
As a result, the skill-occupation matrix is expected to be modular as well as nested with mutually exclusive modules. \textit{Nested-modular matrix} is a complicated structure and will be beyond our current scope \cite{Fortuna2010, VanDam2021}. Here, we will focus on individual skills' contributions to the nested structure and differentiate skills that contribute to the nested structure from those that do not. 

Therefore, we quantify a skill's contribution to the nested structure, $c_s$, defined as a deviation from a null model where the edges of a focal node $s$ to occupations are randomly reassigned, that is, $c_s = (N - <N^{\ast}_s> ) / {\sigma_{N^{\ast}_s}}$.
$N$ is a nestedness score, and $<N^{\ast}_s>$ and $\sigma_{N^{\ast}_s}$ are the means and standard deviation derived from the null model \cite{Saavedra2011}. For each focal skill $s$, we run 5,000 iterations for each skill \cite{SergeiMaslov2002}. We employ the overlap index checkerboard score, Temperature, and NODF, nestedness scores commonly used in ecology, to quantify nestedness $N$ \cite{write1992, Stone1990, Atmar1993, Saavedra2011}. In addition, we only consider skill's contribution and do not occupation's contribution. 


Figure 2-(c) shows $c_s$ of specific skills (blue nodes in Fig 2-b) with the divide between the two groups, those that contribute to the nestedness (nested) and those that do not (unnested). Only specific skills are considered in this differentiation in order to compare skills at the same generality level to avoid comparing apples to oranges. It is not fair for specific skills to be compared with general skills as they have more edges. 


% If you mean highest in terms of nested specific skills, here are the top 5 in 2019:
%1. Physical Medicine Rehab Physicians
%2. Anthropology and Archeology Teachers (postsecondary(
%3. Biomedical Engineerings
%4. Archeologists
%5. Surgeons

%We then calculate an occupation's average nested skill, for example, as $\Sigma_{s \in nested} \; m_{o,s}$ divided by the number of nested skills where $m_{o,s}$ is the occupation-skill matrix that was used above. 
%SI Section \ref{supsec: add - returns to skill} includes additional analysis using the average of occupations' top five skills, instead of the entire skills within each category, in all analyses and found consistent results. 



% M above is a hierarchical tree where m_o,s is occupation-skill matrix. 

%Education, workplace experience, and training are available at the level of 8-digit SOC. However, wage and employment are available at 6-digit SOC codes. Therefore, to link detailed worker attributes with the former job attributes, we match BLS and O*NET data using 8-digit SOC.
%We match the two datasets by aggregating sub-occupations into the 6-digit codes to link wages, education, and employment with detailed worker attributes. 
%From O*NET, we also obtain educational requirements for occupations— at 8-digit SOC codes.
%\subsubsection*{Education and Wage Information}
\subsubsection*{Skill Categories and Educations}
Education variables in O*NET are categorized into twelve discrete grades, ranging from below high school (1) to post-doctorate (12). 
Each occupation includes the proportion to which corresponding sampled employees had to have a given educational level to be hired. 
With this information, we calculated an occupation's associated education variable as a weighted average of the employees. 
For instance, Chief Executives' expected education variable $<edu>_o$ is calculated as $\Sigma_e f_e \cdot edu_e $ where $f_e$ is a fraction of CEO whose education is $e$, and $edu_e$ is a corresponding value of education category, ranging 1 for below high school to 12 for post-doctorate.
For an educational requirement to a skill $s$, $<edu>_s$, we average the skill's education levels of occupations, $<edu>_o$, weighted by the importance of skill, $\text{importance}$, that is $\frac{\Sigma_o \; <edu>_o \cdot \: \text{importance}_{o,s} }{\Sigma_o \text{importance}_{e,o}}$.

%We obtain annual wages in the years 2019 and 2005 from the BLS. However, unlik education wage information is available at the level of 6-digit SOC codes.
%Therefore, all wage analysis use annual wage information at the level of 6-digit (663 items with available wage and skill information in our 2019 sample) occupations and aggregate skills by averaging over the corresponding 8-digit occupations for which we have skill information (789 items with wage and skill information).
%For instance, Chief Executives (SOC: 11-1011) corresponds to 8-digit occupations: Chief Executives (SOC: 11-1011.00) and Chief Sustainability Officers (SOC: 11-1011.03). For all wage analysis, the skill information of Chief Executives and Chief Sustainability Officers are averaged for Chief Executives (SOC: 11-1011).


% \footnote{\tiny\url{https://www.bls.gov/cps/demographics.htm#age}}f
% BLS crosswalk https://www.census.gov/topics/employment/industry-occupation/guidance/code-lists.html
% (mapping 968 to 542 occupation codes) Out of the unmatched 628 O*NET occupations, we link 55 more occupations to their CPS counterpart using text analysis and matching of occupations' titles in O*NET and CPS. Our results are robust to the presence or absence of the latter 55 occupations. See supplementary information for more detail. # we no longer do this extra matching.
%  We aggregate occupation skills for each skill category (\textit{general, nested intermediate and specific, and un-nested intermediate and specific}). Taking an average for each US county using the county employment of occupations as weights, we derive a regional measure of skill endowment for each skill sub-type.
% For each demographic subgroup (e.g., Asian male workers), we derive a skill endowment by linking their occupation as coded in CPS to their corresponding skills in O*NET 2019.
% CPS also records education and income wages. We adjust wages for inflation, and account for the number of hours worked, also recorded by CPS, by computing an adjusted weekly wage that is more readily comparable across the population.
% Among demographic variables, Whites, Blacks, and Asians, which constitute the bulk of the sample. CPS data also contains a separate (from race) variable for identifying Hispanic individuals. 
%occupational employment for Metropolitan and nonmetropolitan areas\footnote{\tiny\url{https://www.bls.gov/oes/}} published by the Bureau of Labor Statistics (BLS) at the level of 6-digit SOC occupations.
% The geographical units in the data are Core-based Statistical Areas (CSAs).
%US counties follow the Federal Information Processing System (FIPS) taxonomy.
%To obtain employment at the level of FIPS and map skill information onto US counties, we used a crosswalk also provided by BLS\footnote{\tiny\url{https://www.bls.gov/oes/current/msa_def.htm}}.
% The data also contains the date range spent on corresponding jobs in most cases.
% We extract the start and end date from the job range field using a Python 3 package called \textit{datesparser} version 1.1.4— we preprocessed the field to improve the performance of the package.
% We omit any record without a properly extracted start and end date.
% Sorting occupations that appear in each resume based on the start and end dates allows us to form a career trajectory for each resume.
% We omit problematic job moves, including job changes where the source and target occupations are the same (i.e., moving from one company to another without changing the occupation), jobs shorter than a year, and multiple jobs (more than one job at the same time) (i.e., someone is a teacher and a technician at the same time. This is problematic for our calculation. We keep the job at which the person started first, or stayed longer).
% The decision to remove such occupations arises from the oddity we observed in most such jobs. For instance, various janitors or models became a CEO immediately or with overlapping periods.
% That leaves us with over 10 million job moves and over 5 million unique resumes.
\subsubsection*{Demographic and Geographic Distribution of Skills}

Median occupational ages are derived from the Current Population Survey (CPS) of the year 2019, and synthetic cohorts from individuals born in each year are created from  the individuals' survey conducted jointly by the U.S. Census Bureau and the Bureau of Labor Statistics \cite{Flood2022}. Different occupational taxonomies between the two datasets are mapped by the BLS crosswalk.

\textbf{Demographic analysis}: CPS microdata also include gender and race/ethnicity demographic information. 
We chose four categories, Whites, Blacks, Asian and Hispanic, as they are the bulk of the sample, and any individuals of Hispanic background are included in that category for Fig. 5. 
To avoid attrition and early retirement, we include only full-time workers, employed at the time of the survey, who are between 18 and 55. 
For each demographic category, the average skill level is calculated for their occupational composition.
The microdata records individuals' education, income wages, and the number of hours worked. 
We adjust wages for inflation and account for the number of hours worked, to compute an adjusted weekly wage, which is readily comparable across the population. 
The race/ethnic disparities in Fig. 5 are a ratio of each demographic quantity (general level, nested level, unnested levels, education, and weekly wages) to those of White workers, following \cite{Tong2021} identifying a dominant social group, a social group if it is at least 1.5 times more likely to be employed in the focal occupation. 
Likewise, the gender gap within each race/ethnicity is measured as a ratio of those quantities to those of male workers.
Because we do not have a matched sample, we obtain 95\% confidence intervals by a random sub-sampling. In each iteration, we take 10\% of the subpopulation of interest, for instance, Asian male and Asian female workers, and estimate all corresponding measures. 
Repeating this sampling and estimation process in 10,000 iterations, we obtain the distribution for each estimation and derive the 95\% confidential intervals.
The skill, education, and wage estimations of Fig. \ref{fig:Figure 7} average over the years. However, dynamics over time also carry information. In SI,
% Fig. \ref{fig:Temporal Race Gaps - Skills, Education, Wages} and \ref{fig:Temporal Gender Gaps - Skills, Education, Wages}
we show such temporal dynamics. 


%% Geographic Data ---------------------------
\textbf{Geographic Analysis}: To construct skill maps in Fig. 6, we calculate each category's skill levels in the U.S. county. 
The BLS provides occupational compositions for each county, from which the average skill level is calculated. 
We then calculate the national average and the standard deviation for each skill category to derive a standard score (also known as z-score). 
For Fig. \ref{fig:Geography}~(d), we group cities (core-based statistical areas) by populations [$<$ 10 thousand, $<$ 50 thousand, $<$ 1 million, and $>$ one million] based on the 2010 Census population estimates.
Figure \ref{fig:Geography}~(e), we group cities by the intensity of their manufacturing industries, using the U.S. Census County Business Patterns in 2019. 
At the 2-digit NAICS codes, we take 31-33 as manufacturing industries and calculate the location quotient of manufacturing employment (the ratio of manufacturing employment from the metro area total employment over the nationwide ratio).
Matching metro areas to counties, we designate counties with no manufacturing employment to group `None', and group the rest based on quotient 33\% and 66\% quantiles of the measure into bottom, middle, and top. 



\subsubsection*{Skill Compositions in Career Trajectories}
The expected skill levels of each category in the career sequences. 
We studied over 70 million job sequences (8-digit SOC) in 20 million individual resumes between 2007 and 2020 from Burning Glass Institute.  
We then calculate the expected skill levels in $i$th job by averaging the skill levels of those occupations appearing in $i$th sequences, shown in Fig 3 (g-h). 
From these sequences of averaged skill levels, we calculate skill level changes in $i$th job transition levels, $\Delta_i$, shown in Fig. 3 (i).  

We exclude job transitions shorter than one year or within an occupation (i.e., moving from one company to another without changing the occupation) for our primary analyses. 
The decision to remove such occupations arises from the oddity we observed in most such jobs. For instance, various janitors or models became a CEO immediately or with overlapping periods.
Nevertheless, our findings are robust to this decision (see SI
% Sec \ref{supsec: skill dependencies and age}
for details). 

To see if the observed trends are truly attributed to career trajectories, we shuffle job history in resumes, bootstrapping the job sequences, to produce a benchmark, and compare it with the skill changes we empirically observed in career moves in Fig. 3 (i), confirming that the empirically observed trends are unique to the career trajectories.  





%Since its first inception in 1998, the O*NET skill data has undergone several taxonomy transitions. 
% Furthermore, although BLS updates O*NET each year, the information about only a subset of occupations is revised at each new version, leaving others unchanged. We also used an older version of O*NET to make a historical comparison.
% \footnote{\tiny\url{https://www.onetcenter.org/taxonomy.html}}
\subsubsection*{Temporal evolution of skill structure} 


We utilize this evolution of skill structure to demonstrate the implication of our constructed nestedness skill structure. 
We choose two sufficiently apart datasets to capture the structural difference, that is, 
version 9.0 in 2005 because it is the first version comparable to the most recent version while offering satisfactory coverage of occupational information (such as education and wage), and version 24.1 in 2019 because it is the most recent version without the potential contamination of irregular patterns due to the pandemic. 
The empirical challenge is that the classification system is continuously updated in response to technological progress, economic transformation, and social reconfiguration \cite{Park2020}. 


%2005 O*NET complies with \textit{O*NET SOC 2000}, while 2019 O*NET relies on \textit{O*NET SOC 2010}, with two other taxonomy changes in between— in 2006 and 2009.
%Therefore, identically encoded occupations may not be comparable across these two years, and matching them requires a crosswalk.
%While O*NET reports crosswalks between each consecutive taxonomy, a direct crosswalk does not exist between 2005 and 2019.
We created a crosswalk between occupation classifications in 2005 and 2019 that is not immediately available but only between two consecutive years.  
Occupation codes in 2005 are matched to those in 2006, and then those in 2006 to 2009, ... to 2019. Our crosswalk automatically matches 968 occupations in 2019 skill data and 941 unique occupations in 2005 skill data, and the rest are manually matched (See SI
% -Tab. \ref{tab:list of onet soc changes}
).
Using these occupations and their skill levels in 2005, we construct the skill structure of 2005 in Fig. ~\ref{fig:historical skill change} (c), using comparable parameters and layouts for both years to make the networks most comparable (see SI). 

% We calculate the difference of the averaged skill level of a skill $s$ between the years 2005 and 2019 as in equation \ref{eq: skill change}:
%\begin{equation} \label{eq: skill change}
    %\Delta(s)_{2019,2005} = Level(s)_{2019} - Level(s)_{2005}
%\end{equation}


%%--------------------------------------------
\section*{Acknowledgement}
H. Y. and M. H. acknowledge the support of the National Science Foundation Grant Award Number EF-2133863.
The authors are grateful to Yong-Yeol Ahn, Inho Hong, Hyunuk Kim, Balazs Lengyel, Muhammed Yildirim, James McNerney, Morgan Frank, Christopher Esposito, Ulrich Schetter, Serguei Saavedra, James Evans, and Brian Uzzi for their valuable discussions and feedback.
N. F. gratefully acknowledges financial support from the Austrian Research Agency (FFG), project \#873927 (ESSENCSE).

% \bibliographystyle{naturemag}
% \bibliography{scibib.bib}

\printbibliography
\end{document}

