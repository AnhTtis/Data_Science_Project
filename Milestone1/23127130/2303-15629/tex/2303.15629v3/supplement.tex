\documentclass[a4paper, 12pt]{article} %{{{

% \usepackage{lineno}
% \linenumbers

\usepackage[utf8]{inputenc}
\usepackage[english]{babel}
\usepackage{csquotes}
\usepackage{authblk}
\usepackage{graphicx}
\usepackage{mathptmx}
\usepackage{amsmath}
\usepackage{mathtools}
\usepackage{bm}
\usepackage[singlespacing]{setspace}
\usepackage[headheight=1in,margin=1in]{geometry}
% \usepackage[sorting=none, style=science]{biblatex}
\usepackage{booktabs}
\usepackage[font=footnotesize]{caption}
\usepackage[font=footnotesize]{subcaption}

%% For fitting sidewaysfigures in a page well
\newcommand{\source}[1]{\ttfamily #1}
\newsavebox{\tempbox}
\newlength{\tempdima}
\newlength{\tempdimb}
\newlength{\tempdimc}


\usepackage[multiple]{footmisc} % for multiple consecutive footnotes
\usepackage[labelfont=bf]{caption}
% \usepackage{xcite}
% \usepackage{xr-hyper} %% for cross-referencing
% \usepackage[hyperfootnotes=false]{hyperref}
\usepackage[dvipsnames]{xcolor}
\usepackage{longtable} %% for long tables
\usepackage{rotating}
\usepackage{dcolumn}

\usepackage{xcite}
\usepackage{xr}
\makeatletter
\newcommand*{\addFileDependency}[1]{% argument=file name and extension
  \typeout{(#1)}% latexmk will find this if $recorder=0 (however, in that case, it will ignore #1 if it is a .aux or .pdf file etc and it exists! if it doesn't exist, it will appear in the list of dependents regardless)
  \@addtofilelist{#1}% if you want it to appear in \listfiles, not really necessary and latexmk doesn't use this
  \IfFileExists{#1}{}{\typeout{No file #1.}}% latexmk will find this message if #1 doesn't exist (yet)
}
\makeatother

\newcommand*{\myexternaldocument}[1]{%
    \externaldocument{#1}%
    \addFileDependency{#1.tex}%
    \addFileDependency{#1.aux}%
}
\myexternaldocument{maincurrent}

\newcommand{\beginsupplement}{%
        \setcounter{table}{0}
        \renewcommand{\thetable}{S\arabic{table}}%
        \setcounter{figure}{0}
        \renewcommand{\thefigure}{S\arabic{figure}}%
     }
     
% \addbibresource{scibib.bib}
\usepackage[backend=biber,style=nature]{biblatex}
\addbibresource{maincurrent.bib}
% Ignore irrelevant biblatex fields
\AtEveryBibitem{%
 \clearfield{url}%
 \clearfield{month}%
 \clearfield{issn}%
 \clearfield{doi}%%
 % \clearfield{address}%%
}
\newcommand{\figdir}{figs}
\newcommand{\tabdir}{Nature_HB_2023/tabsNHB}


\renewcommand\Authfont{\fontsize{15}{14.4}\selectfont}
\renewcommand\Affilfont{\fontsize{10}{9}\itshape}


\title{%
   Supplementary Information: \\
   Nested Skills in Labor Ecosystems: A Hidden Dimension of Human Capital
}

\usepackage[utf8]{inputenc}

\author[1,2,3]{Moh Hosseinioun}
\author[4,5]{Frank Neffke}
\author[1,2,6]{Hyejin Youn \thanks{hyejin.youn@kellogg.northwestern.edu}}
\author[7]{Letian (LT) Zhang}


\affil[1]{Kellogg School of Management, Northwestern University, Evanston, IL, USA}
\affil[2]{Northwestern Institute on Complex Systems, Evanston, IL, USA}
\affil[3]{Department of Information and Decision Sciences, University of Illinois, Chicago, IL, USA}
\affil[4]{Complexity Science Hub Vienna, Vienna, Austria}
\affil[5]{Harvard Kennedy School, Harvard University, Cambridge, MA, USA}
\affil[6]{Santa Fe Institute, Santa Fe, NM, USA}
\affil[7]{Harvard Business School, Harvard University, Cambridge, MA, USA}


\date{\today}

\begin{document}
\maketitle
\beginsupplement
\thispagestyle{empty} 


\newpage
\tableofcontents
\thispagestyle{empty} 



\clearpage
%% Introduction of SI ---------------------------
\section*{Introduction to the Supplementary Information}
\thispagestyle{empty} 

% In this introduction, we offer a guide to the content of the supplementary informtation document and the correspondance between each section to the materials offered in the main text.


Section \ref{supsec:skill clustering} offers details on the statistical derivations and robustness checks corresponding to the results on Generality in the main text (Figs.~\ref{fig:determining_k_level}-\ref{fig:skill_level_dist_cor_k=4}), and the resulting skill groups (Tab.~\ref{tab:skill_groups}) introduced in the main Fig.~\ref{fig:Figure 1}~(a-c) and used across the paper.

Section \ref{suppsec:nestedness} expands on the nestedness of occupation-skill networks in part shown in Fig.~\ref{fig:nestedness} and used throughout the paper.
It describes the rationale for (Fig.~\ref{fig:occ_vs_skill_importance_avg_cos} and \ref{fig:occ_skill_nestedness_mat}), the methodology of measuring skill-level contribution to nestedness, alternative measurement of skills' contributions, the results based on different measures (Fig.~\ref{fig:Skill Nestedness Contribution - C}-\ref{fig:Skill Nestedness Contribution - NODF}), and the resulting split of skills based on nestedness we used throughout the paper (Tab.~\ref{tab:skill_split_C_nestedness}).
It also includes an alternative approach to splitting skills based on correlation, which yields consistent results(Tab.~\ref{tab:skill_split_alt}).


Section \ref{supsec:conditional dependencies} articulates the construction of the skill hierarchy of the main Fig.~\ref{fig:Figure 2}~(a and b).
It describes how we derive conditional probabilities between pairs of skills (which is briefly introduced in Fig.~\ref{fig:Figure 1}~(d), the choice of parameters (Figs.~\ref{fig:comparison_skill_deg_before_after_binarization}-\ref{fig:YY z-score distribution}), and sensitivity analysis (Figs.~\ref{fig:a-threshold and z-threshold sensitivity analysis}).
% -\ref{fig:YY_param_robustness_zth_5.5}).
% For a variety of parameters a similar hierarchical structure emerges, but withdrew visualizing the network with alternative parameter choices for the sake of space-- these visuals can be shared upon request.
Figs.~\ref{fig:full_figure_2b_labeled} and \ref{fig:figure_2b_labeled} show the full and backboned skill hierarchy network with all skill labels attached.
The section highlights the linkage between our skill hierarchy and a skill co-occurrence network (Fig.~\ref{fig:Network_Skill_Complementarity_RCA_modularity_colored}), and offers two cases based on comparing registered nurses with nurse practitioners (Fig.~\ref{fig:RN vs. NP}), and the skill entrapment of some immigrants (Figs.~\ref{fig:skills_of_different_Hispanics}-\ref{fig:hispanics language skills}), to showcase how the skill hierarchy captures career progress.



Section \ref{supsec: skill dependencies and age} expands on the temporal analyses reported in the main Fig.~\ref{fig:age}. We have explicated the preparation process (Figs.~\ref{fig:BG_skill_change_fig_full}-\ref{fig:BG_fullfig_cleaned} and Tab.~\ref{tab:odd job sequences}) Bootstrapping of the job sequences in resume data (Fig.~\ref{fig:BG_skill_change_single_bootstrap}-\ref{fig:BG_skill_change_bootstraps}), and included the result for all skill cateogires (Fig.~\ref{fig:BG_fullfig_dynamic}).
We also include details about the analysis of median age of workers (Fig.~\ref{fig:occupations' median age and skill - full fig}), and the analysis of synthetic birth cohorts based on CPS (Fig.~\ref{fig:individuals' age and skill - year effects}).
As robustness checks, we also show that the skill development observed in the main Fig.~\ref{fig:age} continues long after education (Fig.~\ref{fig:individuals' age and skill and education}) and also emerges for individuals without college education (Fig.~\ref{fig:individuals' age and skill - no college}).


Section \ref{supsec: add - returns to skill} expands the analyses of the main Fig.~\ref{fig:Wage}, capturing the correlation of occupational wages, educational requirement, and experience with their average levels of each skill category (Fig.~\ref{fig:SI_education_skill_level}-\ref{fig:SI_wage_skill_level}), supplements these results by robustness checks, using alternative measures of skill levels (Fig.~\ref{fig:SI_education_skill_level_top5}-\ref{fig:SI_wage_skill_level_top5}), and regression analyses (Tab.~\ref{tab:wage reg on skill endowment}).
We show the robustness of the main wage findings across major occupational groups (Fig.~\ref{fig:Figure 3 full | major occupation groups}), replicate the main Fig.~\ref{fig:Wage} based on the data of year 2005 (Fig.~\ref{fig:Wage and education 2003}), and finally show the correlation between levels of each skill categories and occupational automation risk \cite{Frey2017} (Fig.~\ref{fig:occ_FOautomation_skill}).


Section \ref{section: add - geographical distribution of skills} offers a descriptive geographic analysis of skill distribution.
We offer evidence that part of the urban wage premiums is explained by the distribution of general and nested skills (Tab.~\ref{tab:urban wage premium}, and Figs.~\ref{fig:geo_dist_level_general_skills_employment_weighted}-\ref{fig:skill_and_manufacturing_full}), but leave an in-depth study of the topic for future work.
% , and manufacturing intensity (Figs.~\ref{fig:Geography}-\ref{fig:skill_and_manufacturing_map}).


Section \ref{section: add - demographic distribution of skills} extends the demographic skill analysis of the main Fig.~\ref{fig:Figure 7}, in Fig.~\ref{fig:racial and gender skill endowments}.
Figs.~\ref{fig:Temporal Race Gaps - Skills, Education, Wages} and \ref{fig:Temporal Gender Gaps - Skills, Education, Wages} capture temporal patterns of these factors, exhibiting the gaps have narrowed over time.
Fig.~\ref{fig:Skill Age Gender Race Trends} depicts that racial/ethnic and gender differentials in skills follow similar age trends observed in the main Fig.~\ref{fig:age}, and are robust to time-variant economic factors (Fig.~\ref{fig:Skill Age Gender Race Trends - year effects}).
In Fig.~\ref{fig:Parenthood_Male_vs_Female}, we highlight the differential influence of parenthood on male and female workers, observed in the diverging growth of general and nested skills in the main Fig.~\ref{fig:age}.


Section \ref{supsec: historical skill change} expands on the changes in occupational skill requirements between 2005 and 2019 (the main Fig. \ref{fig:historical skill change}), and the resulting changes in the skill hierarchy (Figs.~\ref{fig:occupation groups historical changes to skill cluster and type levels}-\ref{fig:Figure_7_supp}). 
The section also provides a brief discussion of the changes in the occupational taxonomy (Fig.~\ref{fig:soc match coverage}).


Section \ref{supsec:robustness checks} offers a battery of robustness checks on whether administrative and managerial occupations (Figs.~\ref{fig:determining_k_kmeans_70bins_correlation_no_manager_start}-\ref{fig:skill_and_age_no_manager} and Tab.~\ref{tab:list of manager occupations}) or social skills (Figs.~\ref{fig:social skills}) derive the increasingly important role of general skills.

% %%----------------------------------------------------------------------------%%
%\newpage
%\section{Occupational Skills} \label{subsec:notation}
%
%One can construct for an individual $i$ a vector $H_i = [h_{s_1}, h_{s_2},...,h_{s_M}]$, where each element captures a specific aspect of human capital or skills, $s \in [s_1, s_2, ..., s_M]$, from population $M$ of skills. In this case, a system's economy is a function of 
%$\bm{H} = [H_1, H_2, ..., H_N]$
%% \begin{bsmallmatrix}
%%   H_1\\
%%   H_2\\
%%   ...\\
%%   H_N \end{bsmallmatrix}
%, in which each element denote the human capital vector of a worker $i \in [1, 2,..., N]$ from the population $N$ of workers.
%
%
%With the above perspective in mind, we delve into the occupational Information Network (O*NET) which forms the primary source of our skills.
%Accommodating O*NET data with the above model requires recognizing that occupations, as opposed to individuals, are units of observation. 
%O*NET contains 120 skill items and 968 occupations— identified by 8-digit Standard Occupational Classification (SOC) codes. Cast as the model discussed above, O*NET skill data, $\bm{H} = [H_1, H_2, ..., H_{N_o}]$, 
%% \begin{bsmallmatrix}
%%   H_1\\
%%   H_2\\
%%   ...\\
%%   H_{N_o}
%% \end{bsmallmatrix},
%is the matrix formed by stacking human capital vectors of occupations, $H_o = [h_{s_1}, h_{s_2},...,h_{s_M}]$. Since occupations, as opposed to individuals, are units of observation in O*NET, each element $h_{o,s}$ of $\bm{H}$, shows the intensity with which collection of workers $i \in o$ use any one skill $s$. In O*NET, each skill is linked to every occupation, wherein the intensity of use is documented with two measures: \textit{Importance}, $H^{<Importance>}_{o,s}$, and \textit{Level}, $H^{<Level>}_{o,s}$.
%
%
%
%
%
%With the above perspective in mind, we delve into the occupational Information Network (O*NET) which forms the primary source of our skills.
%Accommodating O*NET data with the above model requires recognizing that occupations, as opposed to individuals, are units of observation. 
%O*NET contains 120 skill items and 968 occupations— identified by 8-digit Standard Occupational Classification (SOC) codes. Cast as the model discussed above, O*NET skill data,
%$\bm{H} = [H_1, H_2, ..., H_{N_o}]$
%, is the matrix formed by stacking human capital vectors of occupations, $H_o = [h_{s_1}, h_{s_2},...,h_{s_M}]$. Since occupations, as opposed to individuals, are units of observation in O*NET, each element $h_{o,s}$ of $\bm{H}$, shows the intensity with which collection of workers $i \in o$ use any one skill $s$. In O*NET, each skill is linked to every occupation, wherein the intensity of use is documented with two measures: \textit{Importance}, $H^{<Importance>}_{o,s}$, and \textit{Level}, $H^{<Level>}_{o,s}$.
%
%Two projections of $\bm{H}$ are possible: 1. \textit{Occupational projection}, $\bm{H}^{<O>}$, denotes occupations' skill distributions, obtained by considering rows of $\bm{H}$ as units. This projection describes occupations' by the intensity with which they use skills. 2. \textit{Skill projection}, $\bm{H}^{<S>}$, which contains skills' usage distributions, obtained by considering the columns of $\bm{H}$ as units. They characterize skills by the intensity with which they are used across the economy. It is the second projection that we consider here.
%
%All 120 skills, knowledge, and abilities (from here on skills for short) in the O*NET data are linked to every occupation found in the skills dataset—a total of 968. 
%There are two ways to measure skill levels required for effective work performance: the level of sophistication (ranging between 0 and 7), and the level of importance/vitality those requirements are (ranging between 1 and 5). These two skill levels are highly correlated ($Pearson Correlation = 0.94$).
%
%We use the level at which O*NET reports occupations use a skill to form a demand distribution for each skill by assuming that basic, general skills are broadly needed, while specialized skills are in much narrower demand.
%Hence, the shape of a skill's demand distribution will reveal its generality versus specificity.
%We chose the skill level as our main measure of relating occupations and skills because we reason the level that captures the necessary sophistication and adeptness is more consistent with our hypothesis of skill progression and hierarchy.
%Still, the correlation and the fundamental classification of skills are robust to using importance measures.
%
%%%----------------------------------------------------------------------------%%
%% By observing the shape similarity and semantics of skills, we concluded the clustering offered by the Cosine techniques is superior.

\clearpage
\pagenumbering{arabic} 
\section{Skill Groups} \label{supsec:skill clustering}

%\subsection*{Clustering Algorithms}
We obtain data-driven categories of skill generality by grouping skills based on their Level Distributions. We employ a $k$-means clustering algorithm (see Fig.~\ref{fig:Figure 1} in the main) and supplement the results with two more measures of skill generality (the average skill level and occupation counts), explained later in this and the next sections. Here, we discuss clustering skills based on their distribution shapes, as Fig. 1 shows in the main text. 


The $k$-means clustering algorithm requires two inputs, a distance metric and the number of clusters, $k$. We choose the correlation distance (as in equation 1) for the former and $k=3$ for the latter. We did not use Euclidean distance because it does not differentiate the shape distributions as inputs compared to correlation. 
Second, we choose $k=3$ because it seems to be in the range of optimal numbers (2-4) from various statistical tests shown in Fig.~\ref{fig:determining_k_level}. Finally, we provide two alternative categories of skill generality, which are consistent with the results of the $k$-means clustering. 


To measure the correlation similarity among the distribution shapes, we binned the distribution with intervals of 0.1. For instance, the skill level ranges from 0 to 7, resulting in a vector of 35 entries, each corresponding to bins of [0,0.10), [0.10, 0.20),... 
Correlation similarities are measured across these vectors. 
Table S1 shows the assignment of skills resulting from $k$-means clustering (based on correlation similarity and $k=3$) used in the main text. 

\begin{equation}
    d = 1 - \frac{\Sigma_i x_i y_i - \frac{1}{n}\Sigma_i x_i \Sigma_i y_i}
    {\sqrt{\Sigma_i x_i^2 - \frac{1}{n}(\Sigma_i x_i})^2 \sqrt{\Sigma_i y_i^2 - \frac{1}{n}(\Sigma_i y_i})^2}
\end{equation}


We use three statistical tests to determine the optimal $k$. These include the elbow method, silhouette analysis, and gap statistics, as shown in Fig.~\ref{fig:determining_k_level}. These results suggest optimal numbers from 2 to 4. We provide the clusters resulting from each choice of $k$ in Figs.~\ref{fig:skill_level_dist_cor_k=3}-\ref{fig:skill_level_dist_cor_k=4}.
 

The conventional \textit{Elbow method} calculates the within-cluster sum of squares for different numbers of clusters $k$ in order to find a sharp decline from one $k$ to another followed by a more gradual decrease in slope, where we find $k=3$ is the best. 
\textit{Silhouette analysis} \cite{ROUSSEEUW198753} measures the similarity of each observation with the cluster to which it is assigned, producing a metric that ranges from -1 (dissimilar) to 1 (similar). In determining the optimal $k$, one looks for the value at which the average (silhouette width) is maximum, providing $k=2$ for the optimal number. 
The \textit{Gap statistic} \cite{Tibshirani2001} compares the total intracluster variation for different $k$ with their expected values under a null model (i.e., a distribution with no obvious clustering, generated using (1,000 iterations of) Monte Carlo simulations of the sampling process,) wherein maximal intracluster variation is desired, providing $k=4$ for the optimal number.
To determine the optimal number of clusters, $k$, based on Gap Statistic, we used the criterion proposed by \cite{Tibshirani2001}, wherein the smallest $k$ such that the \textit{change} in intracluster variation, $f$, is smaller than an (error-adjusted) standard deviation, $s$ of the null model ($f(k+1) - f(k) \geq s_{k+1}$).


%% Determining Number of Clusters - levels
\begin{figure*}[!h]
    \centering
    \includegraphics[width=\textwidth]{Nature_HB_2023/SI_figNHB/Determining_k_-Kmeans_with_correlation-Level-_70_bins.png}
    \caption{\textbf{Statistical tests to determine the optimal $k$ for $k$-mean clustering algorithms}. The figure shows the results of Elbow method, Gap statistic, and Silhouette analysis}
    \label{fig:determining_k_level}
\end{figure*}


%% Determining Number of Clusters - importance
% I commented out this importance analysis to make the read simpler.
%\begin{figure*}[!h]
    %\centering
    %\includegraphics[width=\textwidth]{Nature_HB_2023/SI_figNHB/determining k - cosine - Importance - binwidth = 0.2.png}
    %\caption{Determining Number of Clusters for Skill Demand Distributions based on Importance Values.}
    %\label{fig:determining_k_importance}
%\end{figure*}


Figures \ref{fig:skill_level_dist_cor_k=3}-\ref{fig:skill_level_dist_cor_k=4} show individual skills within categories.
The number of groups does not change the order of generality of skills, which is central to our analysis. The context of our study encourages a focus on the most and the least general skills because those epitomize two skill categories of broad theoretical interest: general skills and specialized skills. Therefore, it is a practical choice for us to start with three clusters, focus primarily on the two extremes, and subject the skills in the remaining cluster to secondary examination.
Given the visual shapes of distributions and the semantic benefit of differentiating the most general and moderately general skills (so-called intermediate skills), we continue using $k=3$ in the main text.
%One can make arguments for and against other numbers of clusters.
% the one hand, using $k=2$, while theoretically appealing, appears to bundle skill distributions that are visibly different.
% the other hand, using $k=4$, enables a more fine-grained division of skills at the cost of semantic burden.
To reduce any inherent noise due to the skills between general and specifics, we choose $k=3$ and focus on general and specifics.
Table \ref{tab:skill_groups} shows the resulting split and offers some supporting statistics.

%% Distribution shapes and assignment - Cosine Regime - k = 3 (kernel width = 0.1)
\begin{figure*}[!h]
    \centering
    \includegraphics[width=\textwidth]{Nature_HB_2023/SI_figNHB/Clustering_Results_-Kmeans_with_correlation-Level-k=3-_70_bins.png}
    \caption{\textbf{Skill Level Distribution with $k = 3$.} Skills are in descending order of generality. The depicted distribution of skills is used in the main text.}
    \label{fig:skill_level_dist_cor_k=3}
\end{figure*}

%% Distribution shapes and assignment - Cosine Regime - k = 2 (kernel width = 0.1)
\begin{figure*}[!h]
    \centering
    \includegraphics[width=.9\textwidth]{Nature_HB_2023/SI_figNHB/Clustering_Results_-Kmeans_with_correlation-Level-k=2-_70_bins.png}
    \caption{\textbf{Skill Level Distribution with $k = 2$.} Skills are in descending order of generality.}
    \label{fig:skill_level_dist_cor_k=2}
\end{figure*}


%% Distribution shapes and assignment - Cosine Regime - k = 4  (kernel width = 0.1)
\begin{figure*}[!h]
    \centering
    \includegraphics[width=.9\textwidth]{Nature_HB_2023/SI_figNHB/Clustering_Results_-Kmeans_with_correlation-Level-k=4-_70_bins.png}
    \caption{\textbf{Skill Level Distribution with $k = 4$.} Skills are in descending order of generality.}
    \label{fig:skill_level_dist_cor_k=4}
\end{figure*}

%% Distribution shapes and assignemnt - Importance Regime - k = 2
%\begin{figure*}[!h]
    %\centering
    %\includegraphics[width=\textwidth]{Nature_HB_2023/SI_figNHB/Ridge plot - cosine - importance - k = 2 - binwidth = 0.3 - Jun18 2023.png}
    %\caption{Skill Importance Demand Distribution and their Division by K-means given $k = 2$. Clusters and skills are organized (by order of generality) from left to right, and bottom to the top, respectively. We used a kernel width of 0.3 for the demand distributions.}
    %\label{fig:skill_importance_dist_cos_k=2}
%\end{figure*}
%
%%% Distribution shapes and assignemnt - Importance Regime - k = 3
%\begin{figure*}[!h]
    %\centering
    %\includegraphics[width=\textwidth]{Nature_HB_2023/SI_figNHB/Ridge plot - cosine - importance - k = 3 - binwidth = 0.3 - Jun18 2023.png}
    %\caption{Skill Importance Demand Distribution and their Division by K-means given $k = 3$. Clusters and skills are organized (by order of generality) from left to right, and bottom to the top, respectively. We used a kernel width of 0.3 for the demand distributions.}
    %\label{fig:skill_importance_dist_cos_k=3}
%\end{figure*}
%
%%% Distribution shapes and assignemnt - Importance Regime - k = 4
%\begin{figure*}[!h]
    %\centering
    %\includegraphics[width=\textwidth]{Nature_HB_2023/SI_figNHB/Ridge plot - cosine - importance - k = 4 - binwidth = 0.3 - Jun18 2023.png}
    %\caption{Skill Importance Demand Distribution and their Division by K-means given $k = 4$. Clusters and skills are organized (by order of generality) from left to right, and bottom to the top, respectively. We used a kernel width of 0.3 for the demand distributions.}
    %\label{fig:skill_importance_dist_cos_k=4}
%\end{figure*}



\clearpage
%% Resulting Skill Assignment to Groups - Table \label{tab:skill_groups}
\footnotesize
\begin{longtable}{@{\extracolsep{5pt}} 
    p{2 cm}|p{13cm}}
    \caption{\textbf{The Skill Group Assignment Resulting from Clustering Skills based on their Usage Distribution Shapes.}. Skills are ordered by their generality (their average levels demanded across occupations, and their marginal totals in the binary skill-occupation network).} 
    \label{tab:skill_groups}
\\[-1.8ex]\hline 

\hline \\[-1.8ex] 
Skill Group & Skill Titles (skill generality, occupation counts) \\ 
\hline \\[-1.8ex]
 
\endhead
\hline \\[-1.8ex] 
General \\(3.43, 563.61) & Oral Comprehension (3.96, 682), Oral Expression (3.9, 680), Customer and Personal Service (3.85, 604), English Language (3.76, 590), Written Comprehension (3.68, 598), Reading Comprehension (3.65, 593), Near Vision (3.64, 703), Problem Sensitivity (3.58, 571), Deductive Reasoning (3.58, 553), Critical Thinking (3.57, 594), Education and Training (3.55, 539), Active Listening (3.54, 593), Inductive Reasoning (3.48, 533), Speaking (3.45, 550), Mathematics Knowledge (3.43, 538), Written Expression (3.4, 532), Monitoring (3.38, 589), Computers and Electronics (3.38, 582), Information Ordering (3.37, 616), Writing (3.29, 514), Speech Clarity (3.28, 474), Speech Recognition (3.26, 589), Category Flexibility (3.23, 610), Judgment and Decision Making (3.21, 520), Active Learning (3.19, 531), Coordination (3.16, 534), Complex Problem Solving (3.15, 498), Administration and Management (3.15, 507), Social Perceptiveness (3.1, 486), Time Management (3.05, 477), Clerical (3.03, 492) \\

% \vspace{0.1 cm} \\


\hline \\[-1.8ex] 
Intermediate \\(2.44, \\281.42) & Selective Attention (2.99, 496), Fluency of Ideas (2.94, 451), Far Vision (2.94, 451), Flexibility of Closure (2.92, 420), Instructing (2.92, 435), Originality (2.88, 440), Visualization (2.86, 415), Service Orientation (2.85, 442), Learning Strategies (2.84, 387), Persuasion (2.79, 366), Perceptual Speed (2.76, 317), Systems Analysis (2.64, 321), Systems Evaluation (2.63, 337), Management of Personnel Resources (2.6, 245), Mathematics Skills (2.57, 284), Negotiation (2.57, 277), Finger Dexterity (2.56, 352), Public Safety and Security (2.55, 298), Number Facility (2.54, 264), Time Sharing (2.54, 164), Mathematical Reasoning (2.52, 267), Psychology (2.47, 305), Visual Color Discrimination (2.45, 207), Speed of Closure (2.44, 128), Memorization (2.34, 120), Personnel and Human Resources (2.33, 234), Engineering and Technology (2.33, 296), Law and Government (2.32, 241), Production and Processing (2.29, 302), Operation Monitoring (2.28, 250), Auditory Attention (2.23, 175), Communications and Media (2.21, 200), Arm-Hand Steadiness (2.21, 327), Quality Control Analysis (2.19, 216), Control Precision (2.17, 318), Sales and Marketing (2.09, 230), Manual Dexterity (2.08, 302), Hearing Sensitivity (2.04, 85), Trunk Strength (2.01, 238), Depth Perception (1.84, 122), Transportation (1.79, 119), Economics and Accounting (1.77, 129), Operations Analysis (1.75, 128) \\



% \vspace{0.1 cm} \\


\hline \\[-1.8ex] 
Specific \\(1.22, \\93.3) & Mechanical (2.4, 354), Design (2.08, 244), Chemistry (1.97, 204), Multilimb Coordination (1.88, 264), Operation and Control (1.79, 192), Physics (1.76, 147), Geography (1.71, 159), Sociology and Anthropology (1.65, 153), Static Strength (1.64, 211), Extent Flexibility (1.59, 234), Troubleshooting (1.58, 126), Reaction Time (1.53, 189), Science (1.52, 154), Management of Material Resources (1.5, 32), Telecommunications (1.48, 53), Building and Construction (1.47, 137), Biology (1.42, 152), Therapy and Counseling (1.42, 119), Management of Financial Resources (1.37, 69), Response Orientation (1.32, 72), Philosophy and Theology (1.31, 102), Medicine and Dentistry (1.26, 108), Stamina (1.26, 79), Gross Body Coordination (1.24, 55), Dynamic Strength (1.21, 56), Wrist-Finger Speed (1.2, 21), Rate Control (1.18, 93), Equipment Selection (1.09, 15), Equipment Maintenance (1.06, 95), Foreign Language (1.03, 17), Gross Body Equilibrium (1.03, 33), Technology Design (1.02, 19), Repairing (1.02, 91), History and Archeology (0.93, 52), Speed of Limb Movement (0.86, 11), Programming (0.84, 29), Spatial Orientation (0.83, 21), Glare Sensitivity (0.74, 13), Fine Arts (0.6, 43), Sound Localization (0.6, 3), Peripheral Vision (0.59, 8), Food Production (0.59, 44), Night Vision (0.53, 2), Explosive Strength (0.48, 4), Installation (0.37, 11), Dynamic Flexibility (0.15, 2) \\
% \vspace{0.1 cm} \\


\hline \\[-1.8ex] 
\end{longtable} 
\normalsize




%%----------------------------------------------------------------------------%%
\normalsize
\clearpage
\section{Skill Nestedness} \label{suppsec:nestedness}

\subsection{Nested Modular structure in Skills and Occupations}
% \label{suppsec:nestedness}

As the scope of knowledge expands, the need for specialization grows.
Unlike findings of the economic complexity about the nested landscape of national, regional, and urban capabilities \cite{Hidalgo2009, Bustos2022}, occupations often bundle few skills and therefore encompass much narrower knowledge domains.
While the main focus of our paper is revealing the underlying structure of workplace skills, noting the distinction is vital.
Here, we empirically offer evidence of the difference between occupation and skill scopes in two ways before discussing nestedness in the skill space.
We find higher variation among skills (in the number of occupations that demand a skill) than among occupations (in the number of skills an occupation demands), as seen in Fig.~\ref{fig:occ_vs_skill_importance_avg_cos}), suggesting a non-trivial nested structure (Fig.~\ref{fig:occ_skill_nestedness_mat}). 

Figure \ref{fig:occ_vs_skill_importance_avg_cos} shows the Level Distribution of skills (red) and the distribution of the total skill amounts in occupations (blue). To obtain the skill Level Distribution, one measures the demand for each skill and makes a distribution. For example, how much English skills are needed for the entire labor market or how much Physics skills are needed across occupations. The former is more broadly used (i.e., general) and therefore has a higher demand than the latter. A skill's demand is calculated by summing skill levels/importances in the occupations (red). 
Similarly, by adding the total levels/importance of each occupation, one obtains occupations' skill endowments, the total level of skills needed to undertake the job's tasks. 


Figure \ref{fig:occ_vs_skill_importance_avg_cos} shows, unlike broad skill generality, occupation's endowment is narrowly distributed. This narrow distribution indicates that the total amount of skills needed for an occupation is not much different from each other, regardless of how much they are paid and how advanced education is needed. We attribute occupations' limited scope of skills to the limited scope or attention that individual workers can offer. There is only so much a single person can equip and do. Thus, individuals' capacity restricts how many skills occupations can bundle. This constraint explains the process of specializations needed for a complex job. In contrast to occupations, skills do not have such constraints. While some skills are niche, general skills epitomize expertise of widespread demand, as they are needed in most occupations.

%% Distribution of Occupations' vs. Skill Average Importance
\begin{figure*}[!h]
    \centering
    \includegraphics[width=0.7\textwidth]{Nature_HB_2023/SI_figNHB/Comparison_of_Endowment_Density_Distribution_of_Skills_and_Occupations.png}
    \caption{\textbf{Scaled Density Distribution of Skill and Occupation Endowments.} Endowment for skills and occupations is calculated by averaging the intensity values for each column and row, respectively. We compute endowment using both intensity measures of Importance and Level. The results contrast for endowment distribution of skills from occupations. Unlike skills, occupations show closer average Importance values. This finding implies occupation's attention is constrained. Hence, they must allocate their limited attention to skills.}
    \label{fig:occ_vs_skill_importance_avg_cos}
\end{figure*}


This stark difference in the scope of occupations and skills requires quantifying the nestedness structure of the skill-occupation matrix differently. In ecological terms, there is no site/area/biome (occupation) that is large enough to nest other sites (occupations), whereas there are species (skills) that can nest other species (skills) as they can appear anywhere. 
This explains why Fig.~\ref{fig:occ_skill_nestedness_mat} shows the noisy nested structure in the skill-occupation matrix, far from the perfect nested triangle. The skill-occupation structure allows mutually exclusive presences, possibly due to competition between skills within an occupation. 

We construct and measure a nested structure of a skill-occupation matrix in Fig.~\ref{fig:occ_skill_nestedness_mat}. The original skill-occupation matrix's entry is a continuous variable (indication of the degree or point along a continuum to which a particular descriptor is required or needed to perform the occupation). But most conventional nestedness analyses require binary entries, and thus, we employ a disparity filter to make the matrix binary entries of statistically significant presences (see Sec.~\ref{supsec:skill-occ}). 
We then sort the matrix entries in descending orders of marginal totals \cite{Bascompte2003}. 
As Fig.~\ref{fig:occ_skill_nestedness_mat} shows, the result deviates from the perfect nested structure as an upper-left triangle. Nevertheless, the upper left is highly populated, indicating a nested structure. This imperfect nested structure may account for the constraints on occupations (limited carrying capacity), introducing severe competition between skill species. This constraint distinguishes the nestedness of extensive economies of nations, regions, and urban areas from occupation's nestedness, for which specializations dominate the evolution more than diversification. 
As a result, the skill-occupation matrix is expected to be modular as well as nested with mutually exclusive modules. \textit{Nested-modular matrix} is a complicated structure and will be beyond our current scope \cite{Fortuna2010}. Here, we will focus on individual skills' contributions to the nested structure and differentiates skills that contribute to the nested structure from those that do not. 
%But we would like to show here some results to build our logic for our current framework.
%The presence of specialization can be better highlighted if groups of occupations using similar skills, and groups of skills demanded by similar occupations were bundled in the matrix.
%In the right panel of Fig.~\ref{fig:occ_skill_nestedness_mat}, we first conduct Stochastic Block Model (SBM) on the skill-occupation space (to avoid the influence of binary transformation on the performance of SBM, we use the Level values as reported by O*NET as input to SBM) and then organized the blocks (and skills and occupations within those blocks) by their number of interactions.
%The result highlights specialization in non-general skills as skill-occupations blocks.
%Figure \ref{fig:occ_skill_nestedness_mat} shows that some occupations rely more heavily on general skills than others (the occupations on the left end).
%The implication of Fig.~\ref{fig:occ_skill_nestedness_mat} is that specialization plays a role in the structure of workplace skills.
%A secondary observation is the \textit{partially} nested structure, supplemented by a modularity measure.
%Nonetheless, the upper fill in both panels of \ref{fig:occ_skill_nestedness_mat} appears denser than a \textit{nested-modular matrix} \cite{Fortuna2010}.

%% skill-occupation Nestedness Matrix
\begin{figure*}[!h]
    \centering
    \includegraphics[width=.5\textwidth]{Nature_HB_2023/SI_figNHB/Occupation-skills_Nestedness_Matrices_-_Jul_19_2023.png}
    \caption{ \textbf{The skill-occupation matrix} The occupations and skills of the matrix are arranged in descending order of their marginal fills (along the x and y axes, respectively.)}
    \label{fig:occ_skill_nestedness_mat}
\end{figure*}




% %% Skill-skill Nestedness Matrix
% \begin{figure*}[!h]
%     \centering
%     \includegraphics[width=\textwidth]{Nature_HB_2023/SI_figNHB/Skill-skill Nestedness Matrices - Jun 21 2023.png}
%     \caption{Associations in the Skills Space.}
%     \label{fig:skill_skill_nestedness_mat}
% \end{figure*}

% %% Skill-skill Nestedness Matrix Labelled
% \begin{figure*}[!h]
%     \centering
%     \includegraphics[width=\textwidth]{Nature_HB_2023/SI_figNHB/Nestedness Matrix with colored labels - Jun 21 2023.png}
%     \caption{Lebelled Skill Association Matrix Organized by SBM and Number of Interactions. Color coding of labels follows our skill clusters of the main Fig.~\ref{fig:Figure 1}.}
%     \label{fig:skill_skill_nestedness_mat_labelled}
% \end{figure*}


%Part of this space, likely corresponding to general skills and a subset of more specialized skills, manifests a significant degree of nestedness.
\subsection{Skill's contribution to Nestedness}

The above evidence reveals that the landscape of human capital is partially nested due to matching increasing complexity with specialization (perhaps, niche constructions), resulting in a nested-modular structure. 
We speculate that increasing complexity mainly generates nested structure, and specialization mainly generates modular structure. We think mathematical modeling of a labor ecosystem according to this insight can be extremely interesting, and we leave this for future work while we focus on empirical observations and quantifications for now. 


Due to the structural complications, described above, conventional approaches for quantifying nested structure (sorting the matrix to observe an upper triangle or calculating presences/absences accounting for a well-defined nested structure) are likely imperfect.
Instead, we look for skills' individual contributions to the current nested structure compared to their counterfactual contributions under a null model. 
For instance, \cite{Saavedra2011} proposes such an approach based on the idea of randomizing edges for a focal skill and comparing the nestedness in the simulated network with the observed value in the system. 
In our case, we create counterfactual worlds as if a focal skill can appear equally likely in any occupation. This equally probable null hypothesis randomly chooses a focal skill's occupations (edges) without considering education, domain knowledge, industrial requirements, or historical contingency, imposing the current socio-economic structure. 
Then, we measure an increase/decrease in nestedness by destroying the current imposition.  
For simplicity, this method is only available for a presence/absence bipartite network \cite{Bastolla2009, Almeida-neto2008}. We use the disparity filter \cite{Serrano6483} because the method preserves degree heterogeneity, which is crucial to distinguishing general from niche skills.
We explain this method in more detail in the supplementary section \ref{supsec:conditional dependencies}.


We use three commonly used metrics of nestedness (checkerboard score, Temperature, and NODF) to quantify nestedness $N$ at the level of the skill-occupation matrix.
Checkerboard score measures the deviation from nestedness as checkerboard appearance of fills \cite{Stone1990}. This score is consistent with the well-known nestedness index, $N_c$, counting the number of times that a species' presence at a site correctly predicts its presence at richer sites and sums these counts across species and sites \cite{write1992}. 
The presence of a checkerboard, an empty site when the nested site predicts the fill, decreases the nestedness. 
Temperature measures as the total number of ``surprises'' on the assumption of a perfectly nested matrix as temperature increases thermal noises to destroy perfect structure \cite{Atmar1993}. Although this is a great measure, this index has its underlying assumption that the system is actually following the mechanism for a perfect nested structure if there is no temperature. 
NODF quantifies nested overlaps, the notion that all species in a poor habitat are present in richer habitats, and decreasing fill (marginal totals of interactions between habitats and species) \cite{Almeida-neto2008}.


Now that we identify the null hypothesis to generate a focal skill's counterfactuals and nestedness indexes let's calculate skills' contributions to nestedness, $c_s$. 
For each skill, we run at least 1,000 simulations, wherein, the focal skills' ties to occupations are randomly shuffled, keeping the number of ties constant. Therefore, all ties of the focal skill (meaning the skill's generality) are preserved. Then, we measured a nestedness index of the generated matrix mentioned above, as $N^{\ast}$.
We quantify a skill $s$'s contribution as:

\begin{equation} \label{eq:nestedness contribution}
    c_s = \frac{N - <N^{\ast}_s>}{\sigma_{N^{\ast}_s}}
\end{equation}

$<N^{\ast}_s>$ and $\sigma_{N^{\ast}_s}$ denote the mean and standard deviation of the nestedness of the simulated matrix, in which skill $s$'s edges were randomized.




%% Skill Nestedness Contribution - RCA
\begin{figure*}[!h]
    \centering
    \includegraphics[width=\textwidth]{Nature_HB_2023/SI_figNHB/Occupation-skills_C-Score_Contribution_-_Jul_15_2023.png}
    \caption{\textbf{Nestedness Contribution of Skills based on checkerboard} \cite{Stone1990}.}
    \label{fig:Skill Nestedness Contribution - C}
\end{figure*}

\begin{figure*}[!h]
    \centering
    \includegraphics[width=\textwidth]{Nature_HB_2023/SI_figNHB/Occupation-skills_Binmatnest_Contribution_-_Jul_15_2023.png}
    \caption{\textbf{Nestedness Contribution of Skills based on Temperature} \cite{Atmar1993}.}
    \label{fig:Skill Nestedness Contribution - BINMATNEST}
\end{figure*}

\begin{figure*}[!h]
    \centering
    \includegraphics[width=\textwidth]{Nature_HB_2023/SI_figNHB/Occupation-skills_row_NODF_Contribution_-_Jul_20_2023.png}
    \caption{\textbf{Nestedness Contribution of Skills based on NODF}\cite{Almeida-neto2008}. Extreme values arise from large denominators for some of the skills.}
    \label{fig:Skill Nestedness Contribution - NODF}
\end{figure*}


%% ----------
\subsection{Nested and Un-nested Skills}
Figs \ref{fig:Skill Nestedness Contribution - C}, \ref{fig:Skill Nestedness Contribution - BINMATNEST}, and \ref{fig:Skill Nestedness Contribution - NODF} show the nestedness contribution of skills using checkerboard score, Temperature, and NODF, respectively.
We are particularly interested in examining the relationship between nestedness contribution and the position of skills in our hierarchy whose vertical position is a local reaching centrality and the horizontal position is education attainment. 
In addition, we would like to compare skills at the same generality level to avoid comparing apples to oranges. 
For example, it is not fair to compare general skills to specific skills as they have more edges. 
Given that general skills mass at the root of those dense dependency webs, we show more closely examine the nestedness contribution of intermediate and specific skills, and use the measurement to split them into categories of nested and un-nested. 

Table \ref{tab:skill_split_C_nestedness} shows the resulting split of skills into categories or subtypes based on the checkerboard score \cite{Stone1990} also shown in Fig.~\ref{fig:Skill Nestedness Contribution - C}.
To be clear, we refer to the result of our skill clustering based on generality \textit{skill clusters} (general, intermediate, and specific) and refer to the further split made based on nestedness \textit{skill categories} or \textit{skill subtypes} (general, nested intermediate, nested specific, un-nested intermediate, un-nested specific.). 

\footnotesize
% \label{tab:skill_split_C_nestedness}
\input{\tabdir/Jul_15_2023_Skill_Split_C-score_Nestedness_Simul_Jan_2023.tex}
\normalsize


\subsection{Alternative Approach for Deriving Skill Categories} \label{supsec:alt splitting skills}
We split the skills of each cluster (general, intermediate, and specific) based on their correspondence with general skills.
We measure such correspondence $C$ by calculating the correlation between the importance of given skill $i$, and the importance of each of the general skills $j$:

\begin{equation}
    C^{<Level>}_{i,j\in <\text{general}>} = \frac{\Sigma_o(Level_{i,o}- \mu_{Level_{i}}) (Level_{j,o}- \mu_{Level_{j}})}{\sqrt{\Sigma_o(Level_{i,o}- \mu_{Level_{i}})^2 \Sigma_o(Level_{j,o}- \mu_{Level_{j}})^2}}
\end{equation}

Aggregating values of $C_{i,j}$ over general skills $j$, we obtain a measure of correspondence between skill $i$ and the set of general skills, $C_{i,<general>}$.
Then, we compare skill $i$ to other skills $l$ in the same cluster $k$ to which $i$ belongs— given our assignment from supplementary section \ref{supsec:skill clustering}.
To do so, for skills $l$ of cluster $k \in \{\text{intermediate}, \text{specific} \}$, we calculate the mean correlation to general skills:

\begin{equation}
    C^{<mean>}_{k} = mean_{l \in k} {C_{i,<general>}}
\end{equation}

Finally, we suggest a skill $i$ of cluster $k \in {Specific, Intermediate}$ is nested if it depends on general skills above the mean level and call it '\textit{nested}' if $C_{i,<general>} \geq C^{<mean>}_{k}$, and suggest it is independent of general skills and call it '\textit{un-nested}', otherwise.
Table \ref{tab:skill_split_alt} shows the resulting assignment of skills based on this approach.
One obtains a similar split of skills if the Importance measure instead of Level is used.

\clearpage
%% Aug 15 - Skill Split - Correlation Dependence.tex
%% \label{tab:skill_split_alt}
\footnotesize
\input{\tabdir/Jul_15_2023-_Alternative_Skill_Split-_Avg_Correlation_with_General_Skill.tex}
\normalsize








%%----------------------------------------------------------------------------%%
\clearpage
\section{Conditional Skill Dependencies} \label{supsec:conditional dependencies}

To obtain the skill structure, as seen in main Fig.~\ref{fig:Figure 2}, we extract conditional probabilities of the appearance of a skill $u$, given the appearance of another, $v$, in the skill-occupation matrix, which was used for nested structure in the previous section \cite{Jo2020}.

\subsection{Skills-Occupation Matrix} \label{supsec:skill-occ}
The original skill-occupation matrix's entry is a continuous variable (indication of the degree, or point along a continuum, to which a particular descriptor is required or needed to perform the occupation). But most conventional nestedness analyses, used in section \ref{suppsec:nestedness}, and conditional probability measures for main Fig.~\ref{fig:Figure 2}, require binary entries.
Thus, we employ a disparity filter to make the matrix binary entries of statistically significant presences \cite{Serrano6483}. 

We chose this method for two reasons.
First, it allows the user to set different restrictions on the skill and occupation sides of the bipartite network. This feature is desirable given the differences in the strength and degree distributions of occupations and skills.
Second, it accommodates heterogeneous degree distribution, which we know is a key characteristic of our skill side.
In choosing the parameters, we ensured the resulting binary network satisfied the following conditions. First, the filtered network has to remain faithful to the skill and occupation degree distributions (macro-level features). We show that the filter indeed kept the distribution shapes in Fig.~\ref{fig:comparison_skill_deg_before_after_binarization} for the skills strength distribution and Fig.~\ref{fig:comparison_occ_deg_before_after_binarization} for occupations' strength correlations.
The Pearson correlation between skills' strengths (sum of edge weights) and their transformed degree is 0.95. The ranking of skills across these two measures is also preserved (correlation is 0.97).
Note that we used comparisons between node strengths and node degrees because of our idiosyncratic empirical data structure. Each occupation includes a survey for every skill, resulting in every occupation having every skill entry with numbers ranging from 0 to 7. 
The surviving skills preserve the distribution and ranking of occupations (correlations between occupations' strength and ranking before and after transformation are 0.79, and 0.79, respectively). 



In the end, the parameter pair ($\alpha_{in} = 0.4$, $\alpha_{out} = 0.275$) results in 33,865 (29\%) edges. 
%out of 116,160 edges in the skill-occupation matrix.
We also conducted validity checks on the choice of parameters by examining the sampled results (5\% of occupations).
The test compares the survived and eliminated skills to common sense. For example, have the links between ``Surgeon" and the skill ``Medicine and Dentistry", and ``Programmer" and skill ``Programming" survived? Conversely, has the link between ``Mathematician" and "Explosive Strength", defined as \textit{The ability to use short bursts of muscle force to propel oneself (as in jumping or sprinting), or to throw an object}, been eliminated?
The goal of this exercise is to ensure the parameters are not set too strictly or too lenient, and that the retained information in ties conforms to expectations.

\begin{figure*}[!h]
    \centering
    \includegraphics[width=\textwidth]{Nature_HB_2023/SI_figNHB/Skill_Dist_of_Binary_Skill-Occupation_Bipartite_-a_la_Serrano_et_al_2009-alpha_in=_0.4,alpha_out=0.275-_Jul_15_2023.png}
    \caption{\textbf{Skill Degrees as Validity Check on Choosing Parameters of Obtaining Skill-occupation Network Backbone}. The figure compares the degree distribution of skills in each skill group before and after the transformation. Our emphasis is on the distinction between the distribution of three types of skills (their overlap) and their relative position to the raw data. Indeed, the Pearson correlation between skills' strengths (sum of edge weights) and their transformed degree is 0.95. The ranking of skills across these two measures is also preserved (correlation is 0.97).}
    \label{fig:comparison_skill_deg_before_after_binarization}
\end{figure*}


\begin{figure*}[!h]
    \centering
    \includegraphics[width=\textwidth]{Nature_HB_2023/SI_figNHB/Occupation_Dist_of_Binary_Skill-Occupation_Bipartite_-a_la_Serrano_et_al_2009-alpha_in=_0.4,alpha_out=0.275-_Jul_15_2023.png}
    \caption{\textbf{Occupational Degrees as Validity Check on Choosing Parameters of Obtaining Skill-occupation Network Backbone}. The figures compare the degree distribution of occupations before and after the transformation. Our emphasis is faithfulness across each of the five skill subtypes.}
    \label{fig:comparison_occ_deg_before_after_binarization}
\end{figure*}


%%% ---------------------------------------------
\subsection{Skill Dependency from Conditional Probabilities} 
% \label{supsec:conditional dependencies}
%The algorithm uses two primary steps.
%In the first, only the edges between nodes whose co-appearance is significantly higher than their appearances (independently) survive.
We first account for the significant conditional appearances, and discount for noise from independent co-appearances (of two skills in occupation) by random chance with a z-score threshold, $z_{th}$ \cite{Jo2020}. 
That is, we account only for those skills that appear together more than randomly expected by $z_{th}$ magnitude. 
Here, $z_{th}$ is a threshold for the extent to which we eliminate chance from two skills appearing in the same occupation.
\begin{equation} \label{eq: YY z-score}
    z_{u,v} = \frac{N(u,v) - \mu}{\sigma} > z_{th}
\end{equation}

Where $\sigma^2 = \frac{N(u).N(v)}{\|O\|} \frac{{\|O\|}- N(v)}{\|O\|} \frac{{\|O\|}- N(u)}{\|O\| - 1 }$ and $\mu = \frac{N(u).N(v)}{\|O\|}$, are the standard deviation and mean of a hypergeometric distribution for the expected co-occurrence of skills (that arise the under the null model of a bipartite configuration model that preserves skill degrees \cite{Jo2020}.) $N(u)$ and $N(v)$ denote the number of occupations that demand skill $u$ and $v$, respectively and $\|O\|$ denotes the total number of occupations.
 
We now estimate conditional probabilities $P(u|v)$ and $P(v|u)$ and assign a direction to them.
The direction, $u \rightarrow v$, is determined when $P(u|v)$ is \textit{substantially} greater than $P(v|u)$. 
Once again, we wouldn't consider every $P(u|v)$ that is insignificantly greater (smaller) than $P(v|u)$, but only those that are sufficiently greater (smaller) to be considered as a \textit{dependent} structure. $\alpha_{th}$ sets the minimum difference between two conditional probabilities so that they are considered to have directional dependence. 
This threshold has to be differentially applied to each skill pair due to the heterogeneous skill node degrees.  
Therefore, the threshold $\alpha_{th}$ is weighted by $(\frac{k_{max}}{min(k_u, k_v)})$ to be applied to filter $[P(u|v)-P(v|u)] \neq 0 $ 

\begin{equation} \label{eq: a_th}
     |P(u|v)-P(v|u)| \: > \: (\frac{k_{max}}{min(k_u, k_v)}) \times \alpha_{th}
\end{equation}
%\begin{equation} \label{eq: a_th}
     %\frac{min(k_u, k_v)}{k_{max}}[P(u|v)<P(v|u)] > \alpha_{th}
%\end{equation}
% \alpha_{u,v}
Where $k$ denotes the number of other skills with ties to the focal skill, and $k_{max}$ denotes the biggest degree observed among skills.
% Authors make a case that $\alpha_{u,v}$ represents the strength of conditional dependency of $v$ on $u$.


The magnitude of the dependence between $u$ and $v$, $w_{u \rightarrow v}$, follows the parametric function introduced by \cite{Jo2020}:

\begin{equation} \label{eq: dependency weight}
    w_{u \rightarrow v} = \frac{min(k_u, k_v)}{k_{max}}\Bigl( \frac{N(u,v)}{N(v)} - \frac{N(u,v)}{N(u)} \Bigr)
\end{equation}


In simple terms, the direction of arrows shows whether by observing skill $v$ in an occupation, it is (more) likely also to observe skill $u$ (than the other way around).
The magnitude of dependence, used as weights in the main Fig.~\ref{fig:Figure 2}, is a parameteric function of the difference between the conditional probabilities of observing $u$ and $v$, and the null model that corresponds to the estimated number of shared occupations between them, given the degrees of $u$ and $v$.
% We exercised caution in interpreting the outcome of this step as $v$ being a dependent of $u$.
The final network is shown in Fig.~\ref{fig:full_figure_2b_labeled} and used across all analysis, but for the main Fig.~\ref{fig:Figure 2} and \ref{fig:historical skill change} that depict the parsimonious versions, from a directed acyclic graph (DAG) \cite{Jo2020}. 
% Nevertheless, when we calculate and analyze, we use the full version. 





\subsubsection*{Choice of Parameters}
There are two parameter choices  $z_{th}$ and $\alpha_{th}$. Here, we present results across different parameters to ensure the robustness of our findings. 
We choose the first parameter, $z_{th}$, in such a way that we remove about two-thirds of the edges.
Fig.~\ref{fig:YY z-score distribution} shows the distribution of z-scores for all skill co-appearance edges.

\begin{figure*}[!h]
    \centering
    \includegraphics[width=0.5\textwidth]{Nature_HB_2023/SI_figNHB/Z-Scores-Distribution_Joe_et_al_2020_Serrano_alpha_in_=_0.4,alpha_out=_0.275_filtered.png}
    \caption{\textbf{Histogram of Z-scores Resulting from Equation \ref{eq: YY z-score} for Skill Co-appearances.} The x-axis shows the $z_{th}$ values derived from equation \ref{eq: YY z-score} on our data, and the y-axis shows the number of co-appearance links falling into a given range of $z_{th}$ values. We focus on $z_{th}$ values between 4 and 6.}
    \label{fig:YY z-score distribution}
\end{figure*}


Ideally, one chooses $\alpha_{th}$ as strictly as possible to remove insignificant links without the loss of skills. If all edges associated with a given skill are removed as a result of the two steps of the algorithm, the skill is eliminated from the resulting dependency network.
However, as our sensitivity analysis shows, retaining too many statistically insignificant links weakens our ability to extract conditional dependencies robustly.
% Several skills become isolated rather quickly, at points where still a significant portion of edges remain.
Therefore, retaining more statistically significant edges inevitably impose the cost of losing several skills.
Fig.~\ref{fig:a-threshold and z-threshold sensitivity analysis} offers a sensitivity analysis on the interaction of $z_{th}$ and $\alpha_{th}$.
Ideally, no more than 5\% of skills are eliminated, while about 95\% of ties between skills were removed as statistically insignificant.
The combination $z_{th} = 4.75$ and $\alpha_{th} = 0.05$ is a possible solution used in the main text.
At this level, only five skills are eliminated from the network.


%% YY param - sensitivity_results - Jul 15 2023
\begin{figure*}[!h]
    \centering
    \includegraphics[width=.95\textwidth]{Nature_HB_2023/SI_figNHB/YY_param_-sensitivity_results-_Jul_15_2023.png}
    \caption{\textbf{Sensitivity Analysis on Parameters Used for Obtaining Significant Skill Dependencies.} The x-axis shows the $\alpha_{threshold}$ values. Each panel shows a certain $z_{threshold}$ as in equation \ref{eq: YY z-score}, and the y-axis shows the rate of node survival or edge elimination. Ideally, no more than 5\% of skills are eliminated, while about 95\% of ties between skills were removed as statistically insignificant. $z_{threshold} = 4.75$ and $\alpha_{threshold} = 0.05$ is a possible solution used in the main text. The resulting backbone from a number of combinations is offered in the following.}
    \label{fig:a-threshold and z-threshold sensitivity analysis}
\end{figure*}



% Hence, for a $z_{threshold}$, we increase $\alpha_{threshold}$ up to a value in which no more than five (out of the total of 120) skills are eliminated.
% Despite preserving more skills, a lenient choice of $\alpha_{threshold}$ includes many redundant dependency links.
% Retaining too many redundant dependencies backbone (to obtain a DAG), many edges are removed, and the resulting backbone networks have extremely low densities— notice figures with lenient $\alpha_{threshold}$ in the robustness checks below.

% In the main Fig.~\ref{fig:Figure 2}, we used $z_{threshold} = 4.75$ and $\alpha_{threshold} = 0.05$.

A consideration is whether the shape of the skill dependency in Fig.~\ref{fig:Figure 2}~(b and c) is influenced by the choice of parameters. 
We conduct a robustness check wherein we visualize the resulting networks from the combination of values of $z_{th}$ between 4 and 5.5 and values of $\alpha_{th}$ between 0.01 and 0.1.
% and show the results in Figs \ref{fig:YY_param_robustness_zth_4} through \ref{fig:YY_param_robustness_zth_5.5}.


% For visualizing these networks, we use Python 3 Networkx package and utilize the Graphviz \footnote{\url{https://pypi.org/project/graphviz/}} algorithm (version 0.20.1) with the method \textit{dot}, the default approach, which operates on directed graphs and visualizes them in a hierarchical order.
% The node colors follow our skill groups (red: general skills, gray: intermediate skills, blue: specific skills), and size is proportional to the skill's association with education— as defined in the main text methodology.
Throughout, a disjointed structure emerges, wherein a set of specialized skills (blue and gray) have closer connections with the general skills (red), than other specialized skills.
Even in networks obtained from a lenient $\alpha_{th}$, the hierarchical structure is visible, and in most, one can distinguish between a more closely knit web of skills that manifest stronger dependence on generals skills (manifest higher connection to red nodes), and a second set of skills, decoupled from the first, which manifest a comparatively shallow dependency web.
We withdrew the visualized network for the sake of brevity. These visuals are sharable upon request.

% Our final choice of parameters used to create the main text Fig.~\ref{fig:Figure 2}, $z_{threshold} = 4.75$ and $\alpha_{threshold} = 0.05$, satisfies the condition mentioned above.

% %% z-th = 4
% \begin{figure}
%     \centering
%     \includegraphics[width=\textwidth]{Nature_HB_2023/SI_figNHB/Simple_Parsimonious_Graph_Joe_et_al_2020_Serrano_alpha_in = 0.4, alpha_out = 0.275_filtered_z-thres_4.0_range_a_thresh - Jul 15 2023.png}
%     \caption{\textbf{Backbones of the Skill Dependence Network as Robustness Checks for Parameters Used in Extracting Skills' Conditional Dependence: $z_{th} = 4.00$}. For visualizing the networks, we used Python 3 Networkx package and utilized the Graphviz package.}
% \label{fig:YY_param_robustness_zth_4}
% \end{figure}

% %% z-th = 4.25
% \begin{figure}
%     \centering
%     \includegraphics[width=\textwidth]{Nature_HB_2023/SI_figNHB/Simple_Parsimonious_Graph_Joe_et_al_2020_Serrano_alpha_in = 0.4, alpha_out = 0.275_filtered_z-thres_4.25_range_a_thresh - Jul 15 2023.png}
% \caption{\textbf{Backbones of the Skill Dependence Network as Robustness Checks for Parameters Used in Extracting Skills' Conditional Dependence: $z_{th} = 4.25$}. For visualizing the networks, we used Python 3 Networkx package and utilized the Graphviz package.}
% \label{fig:YY_param_robustness_zth_4.25}
% \end{figure}

% %% z-th = 4.5
% \begin{figure}
%     \centering
%     \includegraphics[width=\textwidth]{Nature_HB_2023/SI_figNHB/Simple_Parsimonious_Graph_Joe_et_al_2020_Serrano_alpha_in = 0.4, alpha_out = 0.275_filtered_z-thres_4.5_range_a_thresh - Jul 15 2023.png}
% \caption{\textbf{Backbones of the Skill Dependence Network as Robustness Checks for Parameters Used in Extracting Skills' Conditional Dependence: $z_{th} = 4.5$}. For visualizing the networks, we used Python 3 Networkx package and utilized the Graphviz package.}
% \label{fig:YY_param_robustness_zth_4.5}
% \end{figure}

% %% z-th = 4.75
% \begin{figure}
%     \centering
%     \includegraphics[width=\textwidth]{Nature_HB_2023/SI_figNHB/Simple_Parsimonious_Graph_Joe_et_al_2020_Serrano_alpha_in = 0.4, alpha_out = 0.275_filtered_z-thres_4.75_range_a_thresh - Jul 15 2023.png}
% \caption{\textbf{Backbones of the Skill Dependence Network as Robustness Checks for Parameters Used in Extracting Skills' Conditional Dependence: $z_{th} = 4.75$}. For visualizing the networks, we used Python 3 Networkx package and utilized the Graphviz package.}
% \label{fig:YY_param_robustness_zth_4.75}
% \end{figure}

% %% z-th = 5
% \begin{figure}
%     \centering
%     \includegraphics[width=\textwidth]{Nature_HB_2023/SI_figNHB/Simple_Parsimonious_Graph_Joe_et_al_2020_Serrano_alpha_in = 0.4, alpha_out = 0.275_filtered_z-thres_5.0_range_a_thresh - Jul 15 2023.png}
% \caption{\textbf{Backbones of the Skill Dependence Network as Robustness Checks for Parameters Used in Extracting Skills' Conditional Dependence: $z_{th} = 5$}. For visualizing the networks, we used Python 3 Networkx package and utilized the Graphviz package.}
% \label{fig:YY_param_robustness_zth_5}
% \end{figure}

% %% z-th = 5.25
% \begin{figure}
%     \centering
%     \includegraphics[width=\textwidth]{Nature_HB_2023/SI_figNHB/Simple_Parsimonious_Graph_Joe_et_al_2020_Serrano_alpha_in = 0.4, alpha_out = 0.275_filtered_z-thres_5.25_range_a_thresh - Jul 15 2023.png}
% \caption{\textbf{Backbones of the Skill Dependence Network as Robustness Checks for Parameters Used in Extracting Skills' Conditional Dependence: $z_{th} = 5.25$}. For visualizing the networks, we used Python 3 Networkx package and utilized the Graphviz package.}
% \label{fig:YY_param_robustness_zth_5.25}
% \end{figure}

% %% z-th = 5.5
% \begin{figure}
%     \centering
%     \includegraphics[width=\textwidth]{Nature_HB_2023/SI_figNHB/Simple_Parsimonious_Graph_Joe_et_al_2020_Serrano_alpha_in = 0.4, alpha_out = 0.275_filtered_z-thres_5.5_range_a_thresh - Jul 15 2023.png}
% \caption{\textbf{Backbones of the Skill Dependence Network as Robustness Checks for Parameters Used in Extracting Skills' Conditional Dependence: $z_{th} = 5.5$}. For visualizing the networks, we used Python 3 Networkx package and utilized the Graphviz package.}
% \label{fig:YY_param_robustness_zth_5.5}
% \end{figure}



\clearpage
\subsubsection*{Visualizing the Skill Hierarchy}
The main Fig.~\ref{fig:Figure 2} is the backbone of a network with parameters $z_{th} = 4.75$ and $\alpha_{th} = 0.05$.
Fig.~\ref{fig:full_figure_2b_labeled} shows the full skill network, which contains 115 nodes and 1,796 dependency relationships.
The following skills are eliminated from the graph because neither of their dependency relationships passed the statistical significance test: \textit{Installation, Explosive Strength, Sound Localization, Food Production, Public Safety, and Security}.
In the backbone network, any direct path is eliminated where there exists an indirect path through dependencies. As a result, the backbone contains only 395 edges, accommodating visualization.
Nonetheless, we perform all calculations on the skill network and not its backbone.


For both the skill network and its backbone, we use a layout technique inspired by \cite{Mones2012, Kosack2018}.
We determine the vertical placement of a skill based on its local reaching centrality \cite{Mones2012}, defined as the number of nodes achievable from the focal node. This highly correlates with a skills' demand, defined as the total level values, across occupations (Pearson correlation 0.89). 
The horizontal position is proportional to the skill's association with education, calculated as the weighted education of occupations (using levels as weights).
We pass normalized values for both the vertical and horizontal axes through a Lambert cylindrical projection to Gephi for visualization.
 Fig.~\ref{fig:figure_2b_labeled} shows the network in the main Fig.~\ref{fig:Figure 2}~(b) with all nodes labeled and the position of nodes adjusted to accommodate labels.


\begin{figure*}[!h]
    \centering
    \includegraphics[width=\textwidth]{Nature_HB_2023/SI_figNHB/Jul_17__2019_Network__Lambert_Proj__Edu-LRC__4.75_0.05_.4_.275__labels_with_layout_overlay.png}
    \caption{\textbf{Skill Dependency Network.} The layout is adjusted to accommodate the labels. The original layout (without overlap) is shown in the top right corner.}
    \label{fig:full_figure_2b_labeled}
\end{figure*}

\newpage
\begin{figure*}[!h]
    \centering
    \includegraphics[width=\textwidth]{Nature_HB_2023/SI_figNHB/Jul_9__Equirectangular_Proj__Edu-LRC__Parsi__4.75_0.05_.4_.275__adjusted_for_labels.png}
    \caption{\textbf{Labeled Skill Dependency Backbone.} Node positions are adjusted to accommodate labels.}
    \label{fig:figure_2b_labeled}
\end{figure*}


%%----------------------------------------------------------------------------%%

% \clearpage
% \newpage
\subsection{Linkage to Skill Co-occurrence Networks} \label{supsec:skill co-occurrence}
% \subsection*{Network Formation Methodology}

Our work builds on a vast literature that conceptualizes the landscape of skills as a co-occurrence network \cite{Neffkeeaax3370, AndersonKatharineA2017Snam, Alabdulkareem2018, Frank2019}.
Indeed, without directionally, the hierarchical network is in excellent agreement with such previously constructed skill networks.
Here, we follow the approach used in \cite{Alabdulkareem2018} that identifies communities of cognitive versus physical from a pairwise co-occurrence network. We obtain such a network in two steps (using O*NET skill data from 2019):

\begin{enumerate}
    \item Measuring the "effective use of skill" by occupation based on RCA as follows:
    \begin{equation}
        RCA(s,j)=\frac{Importance(s,j)/\Sigma_{s'\in S}Importance(s,j')}{\Sigma_{j'\in J}Importance(s',j')/\Sigma_{s'\in S,j' \in J}Importance(s',j')}
    \end{equation}
    where $s$ denotes a given skill, and $j$ a given occupation. $S$ and $J$ denote the population of skills and occupations respectively. An skill-occupation is 'effective'— i.e., $e(i,j)=1$ if $RCA(s,j)$— and is not— i.e., $e(i,j)=0$, otherwise.
    
    \item Using $e(i,j)$ values, authors derive pairwise skill "complementarity" proportional to the number of times skills $s$ and $s'$ co-appeared in an occupation as follows:
    \begin{equation}
        \theta(s,s')=\frac{\Sigma_{j\in J}e(s,j).e(s',j)}{max\big(\Sigma_{j\in J}e(s,j), \Sigma_{j\in J}e(s',j)\big)}
    \end{equation}
\end{enumerate} 


%% May 31 2022 - Network Figure.png

There are two key messages.
First, we explain that the cluster of \textit{General} skills resides at the center of such a skill co-occurrence network— in fact, the ordering of skill specificity based on our skill clusters is predictive of how far the skills lie towards the fringes of the skill co-occurrence network.
Second, the dichotomy of cognitive versus non-cognitive skills has tight connections with the disjointed structures we found and called nested and un-nested skills, respectively.

% Visualizing $\theta(s,s')$, lead to bi-modal distributions. Our replications of the methodology of \cite{Alabdulkareem2018}, as also discussed above, on the O*NET skill from 2019 produced such bi-modal distributions, as shown in Fig.~\ref{fig:Skill_Complementarity_RCA}. The distribution resulting from Importance values is almost identical to figure 1B in \cite{Alabdulkareem2018}. Furthermore, using Level values instead of Importance leads to a similar bi-modal distribution.

% %% Alabdulkareem et al (2018) Pairwise Skill Complementarity - RCA Values
% \begin{figure*}[!h]
%     \centering
%     \includegraphics[width=\textwidth]{\figdir/Alabdulkareem et al (2018) Pairwise Skill Complementarity - RCA Values.png}
%     \caption{Distribution of Pairwise Skill Complementarity. The values are derived alike \cite{Alabdulkareem2018} and resemble the bi-modal form described by authors. }
%     \label{fig:Skill_Complementarity_RCA}
% \end{figure*}

Fig.~\ref{fig:Network_Skill_Complementarity_RCA_modularity_colored} shows a network representation
% \footnote{
% In this figure, edges denote skill complementarity relationship if they passed the threshold of $\theta(s,s')>0.5$— authors used a threshold of $\theta(s,s')>0.6$ which leads to several isolated nodes in 2019 data. We assign an average education value to each skill based on the educational requirements of occupations. A skill's associated education level is only impacted by occupations that use the skill 'effectively'— i.e., $\forall j:  RCA(s,j)$, nodes are colored based on their modularity communities
% }
of skills based on the pairwise "complementarity" values manifests the bi-modal structure reported by \cite{Alabdulkareem2018}.
% \footnote{
% Louvain network community detection, for instance, identifies two communities.}.
There are several departure points, however.
First, we restrict our workplace skills to the so-called \textit{knowledge, abilities, and skills}, disregarding \textit{work activities}, while the latter is commonly used in co-occurrence networks constructed using O*NET.
Our rationale for not including work activities is that they are job descriptions (i.e., generalized forms of job tasks that are specific to jobs).
In contrast, abilities, knowledge, and skills are characteristics of workers' expertise, which are our primary concern.

Second, the community on the right is an ensemble of "cognitive" skills, while the left group corresponds to mostly "physical" skills, which are consistent with our nested and un-nested skills, respectively.
However, several skills, such as \textit{Physics, Design, and Chemistry} seem out of place at the bottom end of the left community, and are classified by our approach as nested skills— which appear consistent with their wage and educational associations.
The key advantage of our method is that we can predict numerous implications of skills based solely on the informationed embedded in the occupation-skill networks, without the need for knowing the content of the skill, as labeled by cognitive or physical skills.
Our skill hierarchy, in effect, offers an explanation for why certain skills, known to be cognitive, are more valuable, based on the investments necessary to satify their dense and nested web of dependencies.

%% Alabdulkareem et al (2018) Network of Pairwise Skill Complementarity - Importance RCA Values - Colored by Modularity Classes
\begin{figure*}[!h]
    \centering
    \includegraphics[width=.7\textwidth, trim={0 5.5cm 0 5.5cm},clip]{Nature_HB_2023/SI_figNHB/Alabdulkareem_et_al__2018__Network_of_Pairwise_Skill_Complementarity__Importance_RCA_Values__Colored_by_Modularity_Classes.pdf}
    \caption{\textbf{Network of Pairwise Skill "Complementarity."} In this figure, edges denote skill complementarity relationship if they passed the threshold of $\theta(s,s')>0.5$— authors used a threshold of $\theta(s,s')>0.6$ which leads to several isolated nodes in 2019 data. We assign an average education value to each skill based on the educational requirements of occupations. A skill's associated education level is only impacted by occupations that use the skill 'effectively'— i.e., $\forall j:  RCA(s,j)$, nodes are colored based on their modularity communities.}
    \label{fig:Network_Skill_Complementarity_RCA_modularity_colored}
\end{figure*}

% \subsubsection*{Connections between the Dichotomous Perspective and our Skill Cluster}
% Interesting patterns emerge from applying our skill clusters, derived from unsupervised machine learning, upon this network as node colors, as in Fig.~\ref{fig:Network_Skill_Complementarity_RCA_ourclusters_colored}. The colors correspond to the cluster colors as in Fig.~\ref{fig:skill_importance_dist_aggr_cos_repeat}— \textit{General skills: red, Somewhat General skills: green, Somewhat Specific skills: cyan, and Specific skills: purple}. Most \textit{General skills} emerge at the center of the cognitive community (on the right). On the other extreme, \textit{Specific skills}, which we know are mostly manual, are mostly placed in the physical community (on the left). There are exceptions to the latter, such as \textit{Science, Biology, Medicine, and Dentistry, etc.}. Other clusters of skills (Somewhat Specific and Somewhat General) are more evenly distributed between the poles.

% %% Alabdulkareem et al (2018) Network of Pairwise Skill Complementarity - Importance RCA Values - Colored by Our Clusters
% \begin{figure*}[!h]
%     \centering
%     \includegraphics[width=\textwidth, trim={0 5.5cm 0 5.5cm},clip]{\figdir/Alabdulkareem et al (2018) Network of Pairwise Skill Complementarity - Importance RCA Values - Colored by Our Clusters.pdf}
%     \caption{Network of Pairwise Skill Complementarity Encoding Skill Clusters. The figure is identical to Fig.~\ref{fig:Network_Skill_Complementarity_RCA_modularity_colored} aside from the node colors: \textit{General skills: red, Somewhat General skills: green, Somewhat Specific skills: cyan, and Specific skills: purple}. Most \textit{General skills} emerge at the center of the cognitive community (on the right). On the other extreme, \textit{Specific skills}, which we know are mostly manual, are mostly placed in the physical community (on the left). There are exceptions to the latter, such as \textit{Science, Biology, Medicine, and Dentistry, etc.}. Other clusters of skills (Somewhat Specific and Somewhat General) are more evenly distributed between the poles.}
%     \label{fig:Network_Skill_Complementarity_RCA_ourclusters_colored}
% \end{figure*}

% Conducting a sensitivity analysis on the visualization threshold, such as Fig.~\ref{fig:skill_complementarity_thresholds}, can better manifest the role of skills from our proposed skill clusters. With the decreasing restriction on the strength of the links visualized, the placement of "General" skills, red nodes, as defined by our unsupervised learning at the center of the network becomes clearer. On the other hand, a more restrictive visualization (higher threshold) shows that General skills are strongly co-appearing with certain specific (that is cognitive) skills that happen to associate with higher wages. The latter point is further shown in Fig.~\ref{fig:Importance_General_vs.Cognitive_Physical_Specific} that captures the relationship between the Importance of General skills for occupations against the Importance of each subgroup of Specific skills. The slopes show that General skills tend to be more important in occupations to whom Cognitive Specific skills are critical. On the flip side, the importance of General skills is comparatively slim for occupations that put emphasis on Physical Specific skills.

% %% Sensitivity Analysis on the Visualization Threshold.
% \begin{figure}
% \centering
% \begin{subfigure}{.5\textwidth}
%   \centering
%   \includegraphics[width=\linewidth]{\figdir/Alabdulkareem et al (2018) Network Skill Complementarity - Importance RCA Values - Colored by Our Clusters - threshold 0.6 - small labels.png}
%   \caption{$\theta(s,s')>0.6$}
% \end{subfigure}%
% \begin{subfigure}{.5\textwidth}
%   \centering
%   \includegraphics[width=\linewidth]{\figdir/Alabdulkareem et al (2018) Network Skill Complementarity - Importance RCA Values - Colored by Our Clusters - threshold 0.5 - small labels.png}
%   \caption{$\theta(s,s')>0.5$}
% \end{subfigure}
% \begin{subfigure}{.5\textwidth}
%   \centering
%   \includegraphics[width=\linewidth]{\figdir/Alabdulkareem et al (2018) Network Skill Complementarity - Importance RCA Values - Colored by Our Clusters - threshold 0.4 - small labels.png}
%   \caption{$\theta(s,s')>0.4$}
% \end{subfigure}%
% \begin{subfigure}{.5\textwidth}
%   \centering
%   \includegraphics[width=\linewidth]{\figdir/Alabdulkareem et al (2018) Network Skill Complementarity - Importance RCA Values - Colored by Our Clusters - threshold 0.3 - small labels.png}
%   \caption{$\theta(s,s')>0.3$}
% \end{subfigure}

% \caption{\textbf{Dynamics of Skill Complementarity Network Change as a Function of Visualization Threshold.} With the decreasing restriction on the strength of the links visualized, the placement of "General" skills, red nodes, as defined by our unsupervised learning at the center of network becomes clearer.}
% \label{fig:skill_complementarity_thresholds}
% \end{figure}

%%----------------------------------------------------------------%
% \clearpage
% \subsection*{Further Analysis of the Skill Inter-dependency Network}
% In Fig.~\ref{fig: btwm-cluster edges - skill network - Jo et al 2020 - Louvain 1.0}, we modify the backbone outcome from the method devised by \cite{Jo2020}, remove all edges within skill clusters, and try to identify the clusters that arise from inter-dependencies across rungs of the skill hierarchy using Louvain's method \cite{BlondelVincentD2008Fuoc}.


% %% Skill Hierarchy btwn-cluster edges only - cluster 1.0 colored - Level Mean filtered - thresh 10 - a la Jo et al 2020 - labels - alpha = 0.pdf
% \begin{figure*}[!h]
%     \centering
%     \includegraphics[width=\textwidth, trim={0 2cm 0 3cm},clip]{\figdir/Skill Hierarchy btwn-cluster edges only - cluster 1.0 colored - Level Mean filtered - thresh 10 - a la Jo et al 2020 - labels - alpha = 0.pdf}
%     \caption{Sub-clusters within Skill Inter-dependency backbone. We modify the backbone outcome from the method devised by \cite{Jo2020}, remove all edges within skill clusters, and try to identify the clusters that arise from inter-dependencies across rungs of the skill hierarchy using Louvain's method \cite{BlondelVincentD2008Fuoc}. Apart from two isolated nodes (Fine Arts and Telecommunication), three clusters emerge. The top cluster associates with Physical and Engineering skills, while the nested region (bottom) appear to have split into two ad-hoc clusters.}
%     \label{fig: btwm-cluster edges - skill network - Jo et al 2020 - Louvain 1.0}
% \end{figure*}

% Apart from two isolated nodes (Fine Arts and Telecommunication), three clusters emerge. The top cluster associates with Physical and Engineering skills, while the nested region (bottom) appear to have split into two ad-hoc clusters.
% In Fig.~\ref{fig: btwm-cluster edges - skill network - Jo et al 2020 - Louvain 1.0} we force more clusters— by lowering the parameter from 1 to 0.5.
% Once again while un-nested skills form a robust cluster, nested skills' division appear ad-hoc.


% %% Skill Hierarchy btwn-cluster edges only - cluster 0.5 colored - Level Mean filtered - thresh 10 - a la Jo et al 2020 - labels - alpha = 0.pdf
% \begin{figure*}[!h]
%     \centering
%     \includegraphics[width=\textwidth, trim={0 2cm 0 3cm},clip]{\figdir/Skill Hierarchy btwn-cluster edges only - cluster 0.5 colored - Level Mean filtered - thresh 10 - a la Jo et al 2020 - labels - alpha = 0.pdf}
%     \caption{Sub-clusters within Skill Inter-dependency Backbone with More Clusters. Using the setup of Fig.~\ref{fig: btwm-cluster edges - skill network - Jo et al 2020 - Louvain 1.0}, we force more clusters— by lowering the parameter from 1 to 0.5.
% Once again while un-nested skills form a robust cluster, nested skills' division appear ad-hoc.}
%     \label{fig: btwm-cluster edges - skill network - Jo et al 2020 - Louvain 0.5}
% \end{figure*}



\subsection{Skill Hierarchy Captures Career Progress (Specialization)} \label{supsec:RN vs. NP}

A key advantage of integrating the conceptual distinction between general and niche skills with a structural network approach to studying skills is that the aggregation of pairwise skill interdependencies reveals pathways of progress (what has come to be known as “specialization”). However, the structure of our skill hierarchy implies that progress entails co-development in certain niche skills and the prerequisites, often more general skills.

\begin{figure*}[!h]
    \centering
    \includegraphics[width=.85\textwidth]{Nature_HB_2023/SI_figNHB/Nurse_Transition_Figure_Mar_25_2024.png}
    \caption{\textbf{Transition between Registered Nurses (RNs) and Nurse Practitioners (NPs).}
    \textbf{(a)} uses resume data from Burning Glass Technology to capture the transition statistics between RNs and NPs. We restrict the analysis to individuals with at least five listed occupations in their resume and define their early career occupations as the most appeared occupation in the first three jobs, similarly late career occupations as the most appeared in the fourth jobs and onward. We disregard individuals whose early and late careers are neither RN nor NP. Including these individuals would not change the result but significantly complicate the exposition. One expects that higher wages for NPs would attract RNs \textbf{(e)}. Indeed, most NPs were RNs early on. However, only a subset of RNs progresses to NP jobs, suggesting barriers to entry, summarized in higher experience and educational requirements \textbf{(c-d)}.
    \textbf{(b)} captures the skill requirements of RNs and NPs, highlighting the advantage of integrating the conceptual distinction between general and niche skills with a structural network approach to studying skills in revealing pathways of progress (also known as “specialization”). The structure of our skill hierarchy also implies that progress entails co-development in certain niche skills and the prerequisite, often more general skills.}
    \label{fig:RN vs. NP}
\end{figure*}


Here, we explore a case study of such progress based on the skill requirement differentials of registered nurses (RNs) versus nurse practitioners (NPs). Compared to RNs, NPs prescribe medicine and diagnostic tests and command higher wages (Fig.~\ref{fig:RN vs. NP} e). Without any cost, someone equipped with the skills of an RN would ideally prefer to work as a nurse practitioner to benefit from higher payoffs.
However, as Fig.~\ref{fig:RN vs. NP}~(a) shows, only a subset of individuals who are RNs early on in their careers (i.e., for whom RN appears most in their first three jobs listed in their Burning Glass Technology resume) manage the switch to the better-paid NP jobs later in their careers (i.e., NP appears most after their third jobs listed in their Burning Glass Technology resume). The fact that most NPs were initially RNs (81.5\%) corroborates our interpretation of the path from RN to NP as one that entails career progress. The transition statistics captured in Fig.~\ref{fig:RN vs. NP}~(a) are also consistent with the higher experience and the more extended training needed for nurse practitioners to develop the necessary skills (Fig.~\ref{fig:RN vs. NP} c-d).

The correlation between education and wages observed at the cross-section of RNs and NPs agrees with the economic theory narrative. However, only a structural approach can reveal the skill development involved in such a transition, highlighting skill growth pathways, seen in Fig.~\ref{fig:RN vs. NP}~(b).
While RNs require high levels of medicine and therapy (niche skills), psychology (intermediate skill), coordination, and social perceptiveness (among many other general skills), the transition into an NP requires further levels of those skills as well as significant development of science, biology, chemistry (among other niche skills), persuasion (intermediate skill), as well as higher knowledge of math, time management, complex problems solving, administrative and computer skills (among other general skills). In contrast, arm-hand steadiness used at high levels by RNs is not as intensely utilized by NPs. 

Comparing RNs’ with NPs’ skills showcases that our approach teases out meaningful progression (or specialization) pathways embedded in the skill requirement of occupations. The co-development of niche and the relevant general skills underpin what we call a nested specialization path. In the following, we offer evidence that the pattern observed in RNs’ and NPs’ careers emerges across individuals in other occupations.



\subsection{Skill Hierarchy Captures Skill Entrapment} \label{supsec:hispanic skill entrapment}

\begin{figure*}[!h]
    \centering
    \includegraphics[width=.75\textwidth]{Nature_HB_2023/SI_figNHB/Recent_Immigrants_vs_OtherHispanic_vs_White_Skills_Feb22_2024.png}
    \caption{\textbf{Comparison of the Skill Levels of Hispanic Immigrants and White Workers} We distinguish between four groups of workers (i. foreign-born Hispanics who have migrated less than a year to the US from the time of survey, ii. foreign-born Hispanics who have been in the US for more than a year, iii. US-born Hispanics, and iv. the White workers) and map their average skill levels for each skill category. Recently migrated foreign-born Hispanics have the least levels of general and nested skills and most un-nested skills.
    }
    \label{fig:skills_of_different_Hispanics}
\end{figure*}

As the main Fig.~\ref{fig:Figure 7} and SI Sec.~\ref{section: add - demographic distribution of skills} show, Hispanics tend to possess relatively high levels of un-nested skills but are underprivileged in gender nested skills.
This unbalance leads to skill entrapments with possibly early rewards and long-term wage penalties, as SI Fig.~\ref{fig:wage curves}.
Our skill hierarchy allows us to explore one possible driver of this skill unbalance for Hispanics.



We suspect language skills are barriers to some hispanic workers, particularly early on in their careers in the US, hampering the acquisition of (language-related) general and (the downstream) nested specific skills, but less so the acquisition of un-nested skills.
To test this, we split the sample of individuals from the CPS into four subgroups, ordered based on their likely level of English proficiency: \textit{Hispanics born outside of the US who immigrated less than a year before the survey, Hispanics born outside the US who have been in the US for more than a year, Hispanics born inside of the US, and White workers}.
We map the average skill levels of each of the above subgroups for each skill category in Fig.~\ref{fig:skills_of_different_Hispanics}, below. As hypothesized,the foreign-born Hispanics who recently migrated to the US have the lowest levels of general and nested skills and have the highest unnested skills. The suspected ranking of English proficiency of each subgroup is consistent with their ranking in terms of general, nested and unnested skills.
Next, we investigate the role of language skills directly Fig.~\ref{fig:hispanics language skills}.



Our network allows us to directly identify which nested skills more closely depend on language general skills. To do so, we first identify six general skills as “language-related”: i.~English Language, ii.~Oral Expression, iii.~Oral Comprehension, iv.~Written Expression, v.~Written Comprehension, and vi.~Speaking. One can quantify the dependence of each nested skill, i, on each of the mentioned language skills, j, by deriving the arrival probability of a random walk starting from the mentioned language general skills, $P_{i,j}^{<arrival>}$. Aggregating these probabilities over the language general skills, we obtain $P_i^{<arrival>}=\Sigma_j P_{i,j}^{<arrival>}$.
We flag nested specific skills at the top 25\% of skills in terms of their average arrival probability, $P_{i,j}^{<arrival>}$, obtaining the following skills: i.~History \& Archeology, ii.~Management of Material Resources, iii.~Management of Financial Resources, iv.~Programming, v.~Philosophy \& Theology. Splitting general and nested skills by their language associations (general skills into Language-related and Non-language skills, and nested skills into Language dependent and Language independent), we obtain the average skill levels of individuals for the previously defined subgroups of workers (Hispanic and White based on their place of birth and time since immigration). In Fig.~\ref{fig:hispanics language skills}, we show the ratios of skills levels for the different Hispanic subpopulation groups relative to White workers for the Language-related and non-language general skills and Language-dependent.

\begin{figure*}[!h]
    \centering
    \includegraphics[width=.78\textwidth]{Nature_HB_2023/SI_figNHB/Recent_Immigrants_vs_OtherHispanics_ratios_White_Skills_Feb29_2024.png}
    \caption{\textbf{Language Barriers Manifest in Lower Levels of Language-related Nested Skills for Hispanics}. The figure depicts, for the Language-related and non-language general skills and Language-dependent (defined as the skills in the top 25\% arrival probability to the mentioned language skills) and Language-independent nested skills, the ratios of skills levels for the different Hispanic subgroups relative to White workers. The results depict that the language-dependent nested skills vary significantly more across the Language-dependent subset, supporting our suspicion that Hispanic workers, at least in part, suffer from their language skills, which prevents them from acquiring/applying downstream skills.
    }
    \label{fig:hispanics language skills}
\end{figure*}

The results show that the skill gaps between Hispanic subpopulations and White workers mimic the implied language gradient: the less proficient in English a subgroup will be, the larger the gap is to White workers in language-dependent nested skills but not in language-independent specific skills. This supports our hypothesis that the skill gaps for Hispanic workers as a whole are, at least in part, due to language barriers.

%%----------------------------------------------------------------------------%%

\clearpage
\section{Skill Categories in Career Trajectories} \label{supsec: skill dependencies and age}
Main Fig.~\ref{fig:age} supplements our inference of the skill structure from O*NET, which relies on cross-sectional data, with longitudinal evidence in line with the notion that one actually acquires or advances general skills when they progress in their career and acquire more specific skills.
Here, we provide additional evidence and robustness checks on the analysis of main Fig.~\ref{fig:age}, based on resume (Burning Glass) data, occupational median age, and skill acquisition reflected in synthetic birth cohorts we created using CPS microdata.


\subsection{Resume Data} \label{section: add - burning glass}

Unlike O*NET, Burning Glass resume data offers longitudinal observation of skill acquisition and will allow us to conduct a more strict test of our skill structure.
We keep track of one's occupations in the resume data, from which longitudinal skill acquisition is inferred. 
% We hope to find some evidence that general skills are acquired before specific skills by analyzing resumes.
% We hope to find evidence of general and nested skills movement in tandem by analyzing resumes.
% In addition, we hope to show that people have to strengthen their general skills to further advance their specific skills.
% In addition, we wish to show that people have to strengthen their general and nested skills to further their career trajectory.



\subsubsection*{Preparing Burning Glass Data}

The following discussion describes choices made in cleaning the data, revealing robustness to such choices in terms of the direction of the results, although the magnitude may vary slightly.
We studied over 20 million resumes from the Burning Glass data, which amounts to over 70 million job moves.
For each move, we link the source and destination occupations to skills from O*NET in 2019.
Excluding all within-occupation moves— which amount to no skill change— we calculate a skill level change across our skill categories and show the result. Fig.~\ref{fig:BG_skill_change_fig_full} as the distribution of career moves for resumes in the Burning Glass sample— after removing within-occupational career moves.


%% Burning Glass full fig - Oct 23 2022.png
\begin{figure*}[!h]
    \centering
    \includegraphics[width=\textwidth]{Nature_HB_2023/SI_figNHB/SI__Burning_Glass_Distributions__unclean__Jul_15_2023.png}
    \caption{\textbf{Changes in Skill Levels in Individuals' Career Moves.} The distribution shows the Burning Glass resume data. A minority of career moves produce extreme values, stretching the skill change distributions' tails due to imperfect data.
    }
    \label{fig:BG_skill_change_fig_full}
\end{figure*}

As can be seen in Fig.~\ref{fig:BG_skill_change_fig_full},
A minority of career moves produce extreme values, stretching the skill change distributions' tails.
Table \ref{tab:odd job sequences} shows a few such cases from the data.
For instance, the resume with the ID \textit{652855}, serves as a janitor for a short period (4 months) before seemingly claiming a Chief Executive role. Resume with ID \textit{1723696} held overlapping jobs as a Medical Health Technician and a Middle School Teacher.
Studying the career moves that correspond to such skill changes, we noticed a significant proportion arise from short job stints and coinciding jobs— some seemingly voluntary part-time commitments.


% \newpage

We removed such jobs from our resume sample.
Particularly, we kept jobs if they lasted at least 12 months— we arrived at the threshold after studying the career moves that correspond to the thousand largest absolute skill changes.
Furthermore, sorting jobs for each resume based on starting date and end date, we removed any job that had a shorter length and overlapped with another— that is, we remove a job $j_r$ from a resume $r$, if it had a later or equal start date with another job $j'_r$, but did not have a later end date.
We also removed jobs for which we could not extract the start and end date— we used \textit{Python's \textit{dateparser} version 1.1.1.} for the extraction.
The resulting sample was 9,382,602 career moves and 5,361,751 resumes.
Fig.~\ref{fig:BG_skill_change_fig_clean} shows the resulting skill change distributions.



%% Distribution of Career Moves.png
\begin{figure*}[!h]
    \centering
    \includegraphics[width=\textwidth]{Nature_HB_2023/SI_figNHB/SI__Burning_Glass_Distributions__clean__Jul_15_2023.png}
    \caption{\textbf{Changes in Skill Levels in Individuals' Career Moves.}
    Most career moves amount to small changes in skills. On aggregate, general, nested skills experience increases on aggregate, while un-nested skills record non-positive changes.
    Nested skill changes closely correlate with changes in general skills. In contrast, there is almost no noticeable relationship between changes in general and un-nested skills.
    Importantly, randomizing the sequence of job transitions (bootstrap) eradicats the direction of skill acquisition in the observed data.
    }
    \label{fig:BG_skill_change_fig_clean}
\end{figure*}


The main text analyzes the levels and patterns of skill change across general and (nested and un-nested) specific skills.
Fig.~\ref{fig:BG_fullfig_cleaned} shows the net effects, i.e., the average change in levels resulting from job moves across all skill categories, and the correlation between the change in the levels of general skills and changes in the level of other skill categories resulting from job transitions.
Fig.~\ref{fig:BG_fullfig_dynamic} supplements main Fig.~\ref{fig:age} (i) by providing the changes in all skill category levels resulting from consecutive job transitions.

%% Jul 15 2023 - Weird Job Sequences.tex
\footnotesize
% \label{tab:odd job sequences}
% \begin{longtable}{@{\extracolsep{5pt}} clccl} 
%   \caption{
%   A Select List of Job Sequences in Burning Glass data that yield extreme skill changes. For instance, resume with the ID \textit{652855}, serves as a janitor for a short period (4 months) before seemingly claiming a Chief Exective role. Resume with ID \textit{1723696} held overlapping jobs as a Medical Health Technician and a Middle School Teacher.
%   } 
\begin{longtable}{@{\extracolsep{5pt}} 
    % {\centering}m{.1\textwidth}>
    % {\centering}m{.1\textwidth}>
    % {\raggedright\arraybackslash}m{.4\textwidth}>
    % {\centering}m{.1\textwidth}>
    % {\raggedright\arraybackslash}m{.4\textwidth}
    % cclcl}
    p{0.2cm}p{1.2cm}p{1.45cm}p{1.45cm}p{8.7cm}}
    \caption{A Select List of Job Sequences in Burning Glass data that yield extreme Skill Changes.} 
  \label{tab:odd job sequences} 
\\[-1.8ex]\hline 

\hline \\[-1.8ex] 
 & ID & Start & End & Occupation \\ 
\hline \\[-1.8ex] 


1 & 652855 & Oct 2011 & Feb 2012 & Janitors \& Cleaners, Except Maids \& Housekeeping Cleaners \\ 
2 &   & Jun 2012 & Aug 2012 & Chief Executives \\ 
3 & 1723696 & Sep 1981 & Sep 1991 & Janitors \& Cleaners, Except Maids \& Housekeeping Cleaners \\ 
4 &   & Sep 1991 & Sep 1994 & Medical Records \& Health Information Technicians \\ 
5 &   & Sep 1993 & Sep 1993 & Middle School Teachers, Except Special \& Career or Technical Education \\ 
6 &   & Sep 1994 & Sep 1999 & Lodging Managers \\ 
7 & 18075175 & Jun 2007 & Jun 2007 & Chief Executives \\ 
8 &   & Aug 2009 & Aug 2009 & Janitors \& Cleaners, Except Maids \& Housekeeping Cleaners \\ 
9 &   & Aug 2010 & Aug 2010 & Cashiers \\ 
10 &   & Aug 2011 & Aug 2011 & Retail Salespersons \\ 
11 &   & Dec 2012 & Dec 2012 & Retail Salespersons \\ 
12 &   & Feb 2013 & Feb 2013 & Cashiers \\ 
13 & 18325881 & Jun 2022 & Oct 2022 & Medical \& Health Services Managers \\ 
14 &   & Oct 2022 & Jan 2022 & Medical \& Health Services Managers \\ 
15 &   & Jan 2022 & May 2022 & Human Resources Specialists \\ 
16 &   & May 2022 & Sep 2022 & Models \\ 
\hline \\[-1.8ex] 
\end{longtable} 



\normalsize


\begin{figure*}[!h]
    \centering
    \includegraphics[width=\textwidth]{Nature_HB_2023/SI_figNHB/SI__Burning_Glass_Static__Jul_15_2023.png}
    \caption{\textbf{Expected Changes of Skill Levels for Each Career Moves.}
    For each career move, we linked the source and destination occupations to skills from O*NET in 2019. We calculate a skill level change across our five skill subtypes.
    For each skill sub-type, we measure changes in skill levels, $\Delta_s$, corresponding to each career move as the average of differences between the skill levels of the target and source occupations.
    \textbf{(a)} shows average changes in skill levels for skill subtypes. On aggregate, general, nested skills experience increases on aggregate, while un-nested skills record non-positive changes.
    \textbf{(b)} shows the correlation between general skills and each skill category resulting from individuals' career moves. Nested skill changes are closely related to changes in general skills. In contrast, there is almost no noticeable relationship between changes in general and un-nested skills.
    }
    \label{fig:BG_fullfig_cleaned}
\end{figure*}

\newpage
%% SI - Burning Glass Dynamic - Full - Jul 15 2023.png
\begin{figure*}[!h]
    \centering
    \includegraphics[width=.7\textwidth]{Nature_HB_2023/SI_figNHB/SI__Burning_Glass_Dynamic__Full__Jul_15_2023.png}
    \caption{\textbf{Changes in all skill category levels in consecutive job transitions.}
    }
    \label{fig:BG_fullfig_dynamic}
\end{figure*}



\newpage
\subsubsection*{Expected Skill Change from Random Job Transition} \label{supsec: bootstrapping BG}
Furthermore, we bootstrapped our resume sample to produce a benchmark and compare it with the skill changes we obtained from observed career moves.
For each resume in our sample, we randomly permuted the order of career moves and measured the skill changes again.
 Fig.~\ref{fig:BG_skill_change_single_bootstrap} shows one such bootstrap.
It is visible in Fig.~\ref{fig:BG_skill_change_single_bootstrap} that the randomization eradicated the direction of skill changes we had obtained from the observed career moves— in Fig.~\ref{fig:BG_skill_change_fig_clean}.


%% Single BG Career Move Bootrstap Result - Oct 26 2022.png
\begin{figure*}[!h]
    \centering
    \includegraphics[width=.7\textwidth]{Nature_HB_2023/SI_figNHB/SI__Burning_Glass_Single_Bootstrap_Distributions__Jul_15_2023.png}
    \caption{\textbf{Changes in Skill Levels in \textit{Bootstrapped} Individuals' Career Moves.}
    The distribution of changes in skill levels visibly differs from what we obtain from the observed career moves.
    }
    \label{fig:BG_skill_change_single_bootstrap}
\end{figure*}


Fig.~\ref{fig:BG_skill_change_bootstraps} further shows the distribution of average skill changes for 100 bootstraps.
The fact that resulting skill changes from a null model differ significantly from our observed results ensures our results are meaningful signals of individuals' career moves, pointing to the dependencies between (general and nested) skills.

%% Bootstrapped vs. Observed Change in Skill Level during Move - Oct 29 2022.png
\begin{figure*}[!h]
    \centering
    \includegraphics[width=.7\textwidth]{Nature_HB_2023/SI_figNHB/Bootstrapped_vs._Observed_Change_in_Skill_Level_during_Move__Oct_29_2022.png}
    \caption{\textbf{Distribution Skill Changes from Bootstrapped Career Moves.}
    }
    \label{fig:BG_skill_change_bootstraps}
\end{figure*}

%%----------------------------------------------------------------------------%%
\clearpage
% \subsection*{Skills and Occupation's Median Age}

% Here, we ask whether inter-dependencies between skills manifest in how different skills are acquired over time.
% Ideally, one would track individuals' skill accumulation data to detect patterns in the co-accumulation of skills.
% However, apart from the challenges of obtaining individual skill data, information captured in commonly used sources, for instance, resumes, is likely biased towards reporting specific skills.
% Instead, we offer two analyses.


% First, we form synthetic birth cohorts by examining the cross-variations in occupations' median ages at the economic level.
% When examining the relationship between skills and age, one would generally expect cohorts of a higher age to have higher skills across the board— consistent with what we find in panels A and B of Figs \ref{fig: Synthetic Cohort Analysis on Occupation Skills - general} and \ref{fig: Synthetic Cohort Analysis on Occupation Skills - specific}.
% Furthermore, as our theory predicts, if general skills are prerequisites of specific (nested) skills, the relationship between general skills and age should sustain controlling for specific skills.
% A corollary is that controlling for general skills should reverse the relationship between age and specific (nested) skills.
% The latter arises from our hypothesis that obtaining higher levels of nested specific skills requires acquiring more sophisticated general skills. 
% To the extent that such learning of skills correlates with age, controlling for the prerequisite (i.e., general skills) should account for any relationship between the dependent (i.e., nested specific skills).
% However, the vice versa should not hold: controlling for the dependent (i.e., nested specific) skills should not eradicate the relationship between age and the prerequisite (i.e., general) skills.

% Second, we form synthetic birth cohorts from participants in Current Population Survey (CPS) born in a given year.
% Ideally, one would follow a given individual and track their skill acquisition as they age.
% Instead, we can track the cohort of individuals born in a given year and track the composition of their skills as they age.
% We infer individuals' skills from the occupations they hold and link those occupations to the O*NET occupational skill requirement of 2019.
% Even though this choice may impose the skill structure of a certain year, it should carry no bearing on the changing composition of skills as a group of individuals age.


% To analyze the relationship between skills and age, we obtain two datasets. The first contains median age data for occupations from Current Population Survey (CPS)\footnote{\tiny\url{https://www.bls.gov/cps/demographics.htm#age}}for the year 2019, and the second contains household survey conducted jointly by the U.S. Census Bureau and the Bureau of Labor Statistics \cite{Flood2022}.
% We use the former to examine the relationship between occupations' median age and skills. We utilize the latter to form synthetic birth cohorts from individuals born in a given year and investigate their skill composition as they age to infer the accumulation of skills over time.
% For all our analyses using household data, we consider only observations in the workforce.
% CPS and O*NET use different occupational taxonomies. As a result, one needs to map CPS and O*NET occupations. We use a crosswalk offered by BLS\footnote{\tiny\url{https://www.census.gov/topics/employment/industry-occupation/guidance/code-lists.html}}, which maps a CPS occupation to 542 out of 968 occupations in O*NET 8-digit SOC codes.
% CPS household data also contain gender and racial/ethnic demographic information.

\subsection{Occupational Median Age}
Fig.~\ref{fig:occupations' median age and skill - full fig} shows the trends of average skill levels and the average levels of the top 5 skills in each category against occupations' median age.
This analysis supplements the main Fig.~\ref{fig:age} (a-c).

%% Fig. 3 & 4 - Occupational Age and Skill Levels and Education - Oct 7 2022.png
\begin{figure*}[!h]
    \centering
    \includegraphics[width=.95\textwidth]{Nature_HB_2023/SI_figNHB/SI__Occupation_Skill_and_Median_Age__Full_Figure__Jul_15.png}
    \caption{
    \textbf{Median Age of Workers in Occupation and Changes in Skill Categories.}
    }
    \label{fig:occupations' median age and skill - full fig}
\end{figure*}

% We find evidence supporting both implications and show them in panels A and B of Fig.~\ref{fig: Synthetic Cohort Analysis on Occupation Skills - main}.
% Plotting general skills, residualized by different types of specific skills, against occupations' median age, we observe that positive slopes persist— panel A.
% Intuitively, controlling for nested specific skills weakens the signal compared to un-nested specific skills, however, does not eradicate the signal.
% Plotting occupations' nested and un-nested specific skills, residualized by the other type of specific and general skills, against their median age, the positive slope for nested specific skills reverses to modestly negative— panel B.
% Such dynamics are driven by general skills being prerequisites to nested specific skills.
% Comparing the left figures of panels A and B, when general skills are residualized by nested specific skills versus when nested specific skills are residualized by general skills, corroborates our theory that general skills act as prerequisites to nested specific skills.
% A less significant result is that, once the effect of general and nested specific skills are controlled, un-nested specific skills are also positively related to age.
% One way to interpret such a relationship is that acquiring un-nested specific skills, too, requires time if we consider that they do not require general skills.
% In panel C of Fig.~\ref{fig: Synthetic Cohort Analysis on Occupation Skills - main}, we show that, as expected, occupations with higher median age also have higher levels of education.
% The latter relationship highlights that an important outcome of education is acquiring general skills.

% %% Jun 21 2022 - Occupational Age and Skill Levels and Education
% \begin{figure*}[!h]
%     \centering
%     \includegraphics[width=0.95\textwidth]{\figdir/Jun 21 2022 - Occupational Age and Skill Levels and Education.png}
%     \caption{Synthetic Cohort Analysis on Occupation Skills.
%     Panel A shows occupations' general skills, residualized by different types of specific skills, against their median age. We observe positive slopes persist. However, as expected, controlling for nested specific skills weakens the signal compared to un-nested specific skills.
%     Panel B shows occupations' nested and un-nested specific skills, residualized by the other type of specific and general skills, against their median age. In this case, the positive slope for nested specific skills becomes modestly negative.
%     Panel C shows occupations with higher median age also have higher levels of education. This relationship highlights general skills as key outcomes of education.
%     }
%     \label{fig: Synthetic Cohort Analysis on Occupation Skills - main}
% \end{figure*}


% Figs \ref{fig: Synthetic Cohort Analysis on Occupation Skills - general} and \ref{fig: Synthetic Cohort Analysis on Occupation Skills - specific} capture the complete analysis, starting with raw skills and adding controls. Panels C of these figures are used in the main Fig.~\ref{fig: Synthetic Cohort Analysis on Occupation Skills - main}.

% %% Jun 21 2022 - Occupational Age and Skill Levels - General skills
% \begin{figure*}[!h]
%     \centering
%     \includegraphics[width=0.95\textwidth]{\figdir/Jun 21 2022 - Occupational Age and Skill Levels - General skills.png}
%     \caption{Synthetic Cohort Analysis on Occupation General Skills.
%     Panel A shows occupations' raw general skill scales against median age, wherein a positive relationship is observed.
%     In panel B, we residualized general skills by both nested and un-nested specific skills. The positive slope between the residualized general skills and age persists.
%     Finally, in each sub-figure of panel C, we residualized general skills by one type of specific skills. While the positive relationship remains, controlling for nested specific skills weakens the age signal.
%     }
%     \label{fig: Synthetic Cohort Analysis on Occupation Skills - general}
% \end{figure*}


% %% Jun 21 2022 - Occupational Age and Skill Levels - Specific skills
% \begin{figure*}[!h]
%     \centering
%     \includegraphics[width=0.95\textwidth]{\figdir/Jun 21 2022 - Occupational Age and Skill Levels - Specific skills.png}
%     \caption{Synthetic Cohort Analysis on Occupation Skills.
%     Panel A shows occupations' raw specific skill scales against median age, wherein nested and un-nested skills differ in their respective positive and negative relationships with age.
%     In panel B, we residualized each type of specific skill by the other type. Adding such control does not change the qualitative relationship between each specific skill type and age.
%     Finally, in panel C, we also add general skills as a control for each type of specific skill. As a result of this exercise, the relationships reverse. We interpret this reversal as evidence supporting our theory that general skills are prerequisites of nested specific skills.}
%     \label{fig: Synthetic Cohort Analysis on Occupation Skills - specific}
% \end{figure*}


% \newpage
\subsection{Individuals' Age and Skills}
Fig.~\ref{fig:individuals' age and skill - year effects} shows the trends of average skill levels and the average levels of the top 5 skills in each category as individuals age, accounting for the year effect.
This analysis supplements the main Fig.~\ref{fig:age} (d-f) by controlling for varying annual economic situations.
The top 5 skills are determined based on the highest levels of skills in each category and are inferred for the individual based on their occupation.
The results are consistent with the main figure.

%% Figure 3 & 4 - Occupational Age and Skill Levels and Education - Oct 7 2022.png
\begin{figure*}[!h]
    \centering
    \includegraphics[width=.95\textwidth]{Nature_HB_2023/SI_figNHB/SI__Individual_Skill_and_Age__conditioned_on_year_effect__Jul_15.png}
    \caption{
    \textbf{Individuals' Skill Acquisition and Age.}
    }
    \label{fig:individuals' age and skill - year effects}
\end{figure*}


As robustness checks, we also show that the skill development observed in the main Fig.~\ref{fig:age} continues long after education (Fig.~\ref{fig:individuals' age and skill and education}) and also emerges for individuals without a college education (Fig.~\ref{fig:individuals' age and skill - no college}).

\begin{figure*}[!h]
    \centering
    \includegraphics[width=\textwidth]{Nature_HB_2023/SI_figNHB/Skill_age_education_comment_4.3_Feb22_2024.png}
    \caption{
    \textbf{Evolution of skill, age and education.} To measure education, we have used educational attainment and the fraction of individuals who attend school as functions of age, both taken from the Current Population Survey (CPS). The education attainment variable ranges from 2 (i.e., no schooling) to 125 (i.e., doctorate degree). To obtain the fraction of the sample attending school, we utilized the information in the CPS variable SCHOOLCOL that documents attending high school (1 or 2) or college/university (3 or 4) or not attending school (5). We transformed the information so that if an individual attends school (1,2,3 or 4), it receives a value of 1, and if not attending, it has a value of 0. Even though by the age of 30, education plateaus and school attendance drops significantly, skill growth continues, manifesting the presence of other mechanisms for skill accumulation apart from education.
    }
    \label{fig:individuals' age and skill and education}
\end{figure*}


\begin{figure*}[!h]
    \centering
    \includegraphics[width=.95\textwidth]{Nature_HB_2023/SI_figNHB/Skill_age_people_with_no_college.png}
    \caption{
    \textbf{Skill Acquisition and Age for Individuals with no College Education.} The figure replicated the skill-age analysis (Fig. \ref{fig:age}) for the subset of individuals who have obtained no more than a high school diploma (values of less than or equal to 073 on the CPS education attainment variable.) The patterns resemble the skill accumulation across the population, even though the levels of general and nested skills are lower compared to the population-level estimates.
    }
    \label{fig:individuals' age and skill - no college}
\end{figure*}

%%----------------------------------------------------------------------------%%
\clearpage
\section{Skill Investment and Payoffs} \label{supsec: add - returns to skill} 

%% ALL SKILLS  -------------------------------------------------------------
\subsection{Investment and Payoffs of Skill Subtypes}

Figure \ref{fig:wage curves} shows “wage curves” that depict wages as a function of age for individuals in the most nested and the most un-nested occupations.
The figures capture entrapment due to un-nested skills.
% (Argote & Eppel, 1990, Jovanovic et al., 1995, Jovanovic & Nyarko, 1997, Lange, 2007, Nagypál, 2007, Kahn & Lange, 2014), 
To obtain wage curves, we averaged over the levels of nested and un-nested skills of each occupation in our sample. We picked occupations at the top 20\% of the nested skills as the most nested, and occupations at the top 20\% of the un-nested skills as the most un-nested. Matching these occupations to the individuals in the CPS, we can obtain estimates of wages for individuals in these occupations at different ages.
To avoid conflating long-run economic factors, we show the wage-age curves for four distinct periods of 5-years: 1983-1987, 1993-1997, 2003-2007, 2013-2017.
In three of the four periods, un-nested jobs have an early wage lead, which quickly evaporates with age. The pattern is consistent with the notion that learning is steeper in occupations with more complex tasks \cite{Jovanovic1997,Nedelkoska2015}. To arrive at a complete picture, one would need to account for the higher cost of education associated with nested occupations. Hence, the wage offsets observed in the figure may occur later in individuals’ lives in terms of real earnings once the cost of education is accounted for.



\begin{figure*}[!h]
    \centering
    \includegraphics[width=.7\textwidth]{Nature_HB_2023/SI_figNHB/Experience_Curves_20_perc_skills_no_errorbar_Feb22-2024}
    \caption{\textbf{Wage Curves for Occupations with Primarily Nested vs. Primarily Un-nested Skills.}
    We average over the levels of nested and un-nested skills of each occupation in our sample and pick occupations at the top 20\% of the nested skills as the most nested and occupations at the top 20\% of the un-nested skills as the most un-nested. Matching these occupations to the individuals in the CPS, we can obtain estimates of wages for individuals in these occupations at different ages. To avoid conflating long-run economic factors, we show the wage-age curves for four 5-year periods: 1983-1987, 1993-1997, 2003-2007, 2013-2017. Un-nested jobs have an early wage lead which quickly evaporates with age.}
    \label{fig:wage curves}
\end{figure*}


Figures \ref{fig:SI_education_skill_level}-\ref{fig:SI_wage_skill_level} capture a similar analysis to the main Fig.~\ref{fig:Wage} for all skill subtypes, separating the relationship between skills and occupation educational requirement, occupation workplace experience, and wages, respectively.
In each figure, the upper panel depicts the bivariate relationship between each nested or un-nested and intermediate or specific skill subset and a corresponding work measure (educational requirement, workplace experience, and wages).
The inset shows the relationship between general skills.
The lower panels control for general skills when regressing the work measure on the corresponding skill subset. The residualized form shows the partial association between the skill subtype and work measure.
For nested skills, the relationship with education weakens but for experience and wages it almost disappears, consistent with the main text and our intuition that general skills derive a large part of the signal.
For un-nested skills, the predominantly negative relationships reverse to modest positive, consistent with the intuition that cetris paribus, un-nested skills behave as if human capital.
Hence, they require training, accumulate experience, and contribute to wages, albeit modestly.


\begin{figure*}[!h]
    \centering
    \includegraphics[width=\textwidth]{Nature_HB_2023/SI_figNHB/SI__Full_Education_Scatter_Plots__Jul_15_2023.png}
    \caption{\textbf{Relationship between Occupations' Educational Requirement and Skill Subtypes.}}
    \label{fig:SI_education_skill_level}
\end{figure*}

\begin{figure*}[!h]
    \centering
    \includegraphics[width=\textwidth]{Nature_HB_2023/SI_figNHB/SI__Full_Experience_Scatter_Plots__Jul_15_2023.png}
    \caption{\textbf{Relationship between Occupations' Workplace Experience and Skill Subtypes.}}
    \label{fig:SI_experience_skill_level}
\end{figure*}

\begin{figure*}[!h]
    \centering
    \includegraphics[width=\textwidth]{Nature_HB_2023/SI_figNHB/SI__Full_Wage_Scatter_Plots__Jul_15_2023.png}
    \caption{\textbf{Relationship between Occupational Wages and Skill Subtypes.}}
    \label{fig:SI_wage_skill_level}
\end{figure*}


%% TOP 5 -------------------------------------------------------------
Figs. \ref{fig:SI_education_skill_level_top5}-\ref{fig:SI_wage_skill_level_top5} repeat the above analyses with the minor difference that the skill level is calculated not as the average of all skills that belong to a subtype, but as the average level of each occupations' top 5 skills in each skill category.
The nature of the relationships is robust to this change— while slopes vary modestly.

\begin{figure*}[!h]
    \centering
    \includegraphics[width=\textwidth]{Nature_HB_2023/SI_figNHB/SI__Full_Education_Scatter_Plots__Top_5__Jul_15_2023.png}
    \caption{\textbf{Relationship between Educational Requirement and Occupation's Top 5 Skills in Subtypes.}}
    \label{fig:SI_education_skill_level_top5}
\end{figure*}

\begin{figure*}[!h]
    \centering
    \includegraphics[width=\textwidth]{Nature_HB_2023/SI_figNHB/SI__Full_Experience_Scatter_Plots__Top_5__Jul_15_2023.png}
    \caption{\textbf{Relationship between Workplace Experience and Occupation's Top 5 Skills in Subtypes.}}
    \label{fig:SI_experience_skill_level_top5}
\end{figure*}

\begin{figure*}[!h]
    \centering
    \includegraphics[width=\textwidth]{Nature_HB_2023/SI_figNHB/SI__Full_Wage_Scatter_Plots__Top_5__Jul_15_2023.png}
    \caption{\textbf{Relationship between Wages and Occupation's Top 5 Skills in Subtypes.}}
    \label{fig:SI_wage_skill_level_top5}
\end{figure*}


Table \ref{tab:wage reg on skill endowment} supplements previous figures by comparing the partial effect of nested and un-nested categories for each skill group on wages. It also introduces conventional control variables of human capital, such as education, experience, and training.
The slopes are consistent with previous results and are robust (both statistically and in magnitude) to adding human capital controls.
Note that we do not run a regression including all subtypes because of the biases introduced by adding pre-treatment variables— general skills are prerequisites to nested skills. 

%% Wage Regression Table
\begin{table}[!h]
    \centering
    % \begin{threeparttable}
    \caption{\textbf{Wage Regression on Skill Endowment.}}
    \resizebox{\columnwidth}{!}{
    \input{\tabdir/Jul_15_2023__Wage_Regression_on_Skill_Endowments.tex}}
    \label{tab:wage reg on skill endowment}
\end{table}
%%

\newpage
\subsection{Automation Risk and Skills}

Given the broad interest in understanding human capital and automation risk, we plot occupations' automation risk index \cite{Frey2017} against their average levels in each skill category in Fig.~\ref{fig:occ_FOautomation_skill}.

%% Wage ~ Skill Type - Level - new categorization.png
\begin{figure*}[!h]
    \centering
    \includegraphics[width=\textwidth]{Nature_HB_2023/SI_figNHB/FO_Automation_Risk_vs._Skill_Levels__Skill_Cat.png}
    \caption{\textbf{Occupations' Automation Risk Index Against their Average Levels in each Skill Category.}}
    \label{fig:occ_FOautomation_skill}
\end{figure*}

%% SKILL PAYOFFS TO DIFFERENT OCCUPATIONS -------------------------------------------------------------
% \newpage
\subsection{Skill Payoffs for Different Occupations}

Fig.~\ref{fig:Figure 3 full | major occupation groups} relates returns to skills for each major occupational group— 1-digit SOC. The key pattern is that all occupational groups, despite varying in their skill endowments, benefit from higher levels of nested skills. 
However, un-nested skills only improve wages of Professional occupations and Skilled traders. 

%% Wage ~ Skill Type - Level - new categorization.png
\begin{figure*}[!h]
    \centering
    \includegraphics[width=\textwidth]{Nature_HB_2023/SI_figNHB/Jul_15_2023__Wage__Skill_Type__major_groups__Level__new_categorization.png}
    \caption{\textbf{Wage Returns to Different Types of Skill Endowment for each Major Occupational Group}. Each point corresponds to an occupation. The setup supplements the main Fig.~\ref{fig:Wage}, highlighting the benefits of higher levels of nested skills. However, un-nested skills only improve wages of Professional occupations and Skilled traders. This in itself underpins multi-dimensionality skills.}
    \label{fig:Figure 3 full | major occupation groups}
\end{figure*}

Interestingly, managerial occupations command high general skills.
Section \ref{sec:robustness check: no managers} of the supplementary document examines (and finds evidence against) the possibility that the returns to general skills are largely a managerial phenomenon.



%% SKILL PAYOFFS TO DIFFERENT OCCUPATIONS -------------------------------------------------------------
% \newpage
\subsection{Skill Investment and Payoffs in 2005}

In Fig.~\ref{fig:Wage and education 2003}, we repeat our analysis of investment and payoffs to skills (main Fig.~\ref{fig:Wage}) for 2005, finding results consistent with the growing importance of general skills.
The figures show lower associated education and payoffs to general skills than the main figure (\ref{fig:Wage}).

%% Skill Wage against Skill Generality and Correlation with General Skills.png
\begin{figure*}[!h]
    \centering
    \includegraphics[width=\textwidth]{Nature_HB_2023/SI_figNHB/SI__2005_Wage_Replicate__Apr_1_2024.png}
    \caption{\textbf{Investment and Payoffs of Different Specific Skills in 2005.}}
    \label{fig:Wage and education 2003}
\end{figure*}




%%----------------------------------------------------------------------------%%
\clearpage
\section{Skills' Geographic Distribution} \label{section: add - geographical distribution of skills}

An in-depth analysis of how skills interface with urban growth is beyond the scope of this work, we provide a brief descriptive analysis, here. Overall, urban areas are more endowed with general skills.
In contrast, rural areas are less likely to carry general skills.
This is consistent with the concentration of more innovative and complex economic activity \cite{Hong2020, Balland2020} and the concentration of managerial and administrative occupations in larger cities.
We test and find support for the hypothesis that skills, in particular concentration of general skills, explain away part of the urban wage premiums.
Upon grouping cities by manufacturing employment relative to the national average, we find that cities highly specialized in manufacturing tend to exhibit lower levels of nested specialization but higher levels of unnested specializations (Fig.~\ref{fig:skill_and_manufacturing_full}).
This shows that cities indeed specialize in distinct directions. Interestingly, skill patterns shift in a non-linear fashion across cities with increasing concentrations of manufacturing employment.
Both a strong dependence on and a complete absence of manufacturing correlate with adverse skill bases, i.e., skill bases dominated by unnested skills and a lower prevalence of general and nested skills. Conversely, skills that typically command high wage premiums are overrepresented in cities with intermediate levels of manufacturing activity.


\subsection{Counties' Skill Endowments}
Using the occupational employment for Metropolitan and nonmetropolitan areas\footnote{\tiny\url{https://www.bls.gov/oes/}} published by the Bureau of Labor Statistics (BLS), one can map the geographical distribution of skills.
BLS uses Core-based Statistical Areas (CSAs) as geographic units, which are more coarse than the county level.
US counties follow the Federal Information Processing System (FIPS) taxonomy.
To obtain employment at the level of FIPS and map skill information onto US counties, we used a crosswalk also provided by BLS \footnote{\tiny\url{https://www.bls.gov/oes/current/msa_def.htm}}.
We aggregate occupation skills at the level of \textit{general, nested intermediate and specific, and un-nested intermediate and specific}.
Taking an average for each US county using the county employment of occupations as weights, we derive a regional measure of skill endowment for each skill sub-type.


% We then compactly present skill maps in Fig.~\ref{fig:Geography} with the distributions of general, nested, and unnested specific skills across U.S. counties (a-c), cities of various sizes (d), and cities with different concentrations of manufacturing (e).
Overall, our analysis (Figs.~\ref{fig:geo_dist_level_general_skills_employment_weighted} through \ref{fig:geo_dist_level_un-nested_specific_skills_employment_weighted}) show a clear concentration of general skills in densely populated urban areas, reflecting the diverse and complex economic activities found in these locales \cite{Glaeser1999, Wheeler2001, Youn2016, gomez2016explaining, Hong2020, Balland2020, Bettencourt2014, Gomez-Lievano2021}. Large cities tend to have higher levels of general and nested skills (also seen in Fig.~\ref{fig:skill_and_population_full}).
For instance, New York and Washington D.C. harbor significant financial and state employment.
Moreover, even in states with comparatively rural structures, such as Indiana, Iowa, Nebraska, and Kansans, state capitals, where the local state is likely to reside, command a higher level of general skills— than their neighboring counties.
A secondary driver of the abundance of general skills in urban areas is the specialization needed for accomplishing complex economic tasks.
For instance, Boston, Seattle, and San Francisco (the latter not shown on the map) are tech hubs and command a strong stock of general (, and as seen in Fig.~\ref{fig:geo_dist_level_nested_specific_skills_employment_weighted}, specific) skills\footnote{
Finer-grained insights can also be obtained from these maps.
For instance, the most extreme concentration of general skills (or lack thereof) is observed in less populated cities that are specialized in a certain industry.
The significant proportion of the focal industries' workers relative to the total employment highlights the skills used by those workers.
The most extreme concentrations of general skills (or lack thereof) are observed in less populated cities that are specialized in a certain industry.
The significant proportion of the focal industries' workers relative to the total employment highlights the skills used by those workers.
The five most and least endowed counties with general skills are shown on the map— as italicized text.
For instance, St. Mary County (Maryland) is an air force and aerospace hub with companies such as Lockheed Martin and Boeing, and military naval air station Patuxent River among the top employers.
Another example is Chatham and its neighboring counties, Durham (hosting Duke University), Orange (hosting the University of Carolina at Chapel-Hill), and Person, which have fostered one of the fastest growing tech sectors in the US, earning the nickname of \textit{Research Triangle}.
Other notable concentration points of general skills are Limestone and Madison (Alabama), hosting numerous aerospace and automobile manufacturing facilities, and Washtenaw county (Michigan) hosting the University of Michigan Ann Arbor and its off-sprung businesses.
In contrast, Madera (California), and its neighboring counties, Tulare, Kings, and Monterey)Highlands (Florida), Yuma (Arizona), Hall (Georgia), and Kalawao (Hawaii) are primarily designated agricultural areas, accruing unnested skills.
}.
However, the starkest disparities between smaller and larger cities are seen in the prevalence of unnested skills, which are significantly less common in cities with over a million inhabitants, a known threshold for cities transitioning towards innovative economic specializations \cite{Hong2020}.
While most workers with nested skills need general s kills, the concentration of managerial and other supporting roles also needs high levels of general skills.
Hence, examine and find evidence consistent with the hypothesis that the accumulation of general skills indeed explains part of the value generated in large cities (Tab.~\ref{tab:urban wage premium}).


%% Wage Regression Table
\begin{table}
    \centering
    % \begin{threeparttable}
    \caption{\textbf{General Skills Explain Urban Wage Premiums.}}
    \resizebox{\columnwidth}{!}{
    \begin{tabular}{@{\extracolsep{5pt}}lD{.}{.}{-3} D{.}{.}{-3} D{.}{.}{-3} D{.}{.}{-3} } 
\\[-1.8ex]\hline 
\hline \\[-1.8ex] 
 & \multicolumn{4}{c}{\textit{Dependent variable:}} \\ 
\cline{2-5} 
\\[-1.8ex] & \multicolumn{4}{c}{Log(Wage)} \\ 
\\[-1.8ex] & \multicolumn{4}{c}{\textit{OLS}} \\ 
\\[-1.8ex] & \multicolumn{1}{c}{(1)} & \multicolumn{1}{c}{(2)} & \multicolumn{1}{c}{(3)} & \multicolumn{1}{c}{(4)}\\ 
\hline \\[-1.8ex] 
 Population > 1M & 0.082^{***} & 0.054^{***} & 0.056^{***} & 0.059^{***} \\ 
  & \multicolumn{1}{c}{(0.080$, $0.084)} & \multicolumn{1}{c}{(0.053$, $0.056)} & \multicolumn{1}{c}{(0.054$, $0.058)} & \multicolumn{1}{c}{(0.057$, $0.060)} \\ 
  & & & & \\ 
 General Skills &  & 0.269^{***} &  & 0.281^{***} \\ 
  &  & \multicolumn{1}{c}{(0.268$, $0.270)} &  & \multicolumn{1}{c}{(0.278$, $0.283)} \\ 
  & & & & \\ 
 Nested Specific Skills &  &  & 0.248^{***} & 0.007^{***} \\ 
  &  &  & \multicolumn{1}{c}{(0.246$, $0.250)} & \multicolumn{1}{c}{(0.005$, $0.010)} \\ 
  & & & & \\ 
 Un-nested Specific Skills &  &  & -0.073^{***} & 0.026^{***} \\ 
  &  &  & \multicolumn{1}{c}{(-0.074$, $-0.072)} & \multicolumn{1}{c}{(0.025$, $0.027)} \\ 
  & & & & \\ 
 Constant & 4.671^{***} & 3.787^{***} & 4.471^{***} & 3.709^{***} \\ 
  & \multicolumn{1}{c}{(4.670$, $4.673)} & \multicolumn{1}{c}{(3.783$, $3.792)} & \multicolumn{1}{c}{(4.469$, $4.474)} & \multicolumn{1}{c}{(3.702$, $3.717)} \\ 
  & & & & \\ 
\hline \\[-1.8ex] 
Observations & \multicolumn{1}{c}{635,554} & \multicolumn{1}{c}{635,554} & \multicolumn{1}{c}{635,554} & \multicolumn{1}{c}{635,554} \\ 
R$^{2}$ & \multicolumn{1}{c}{0.012} & \multicolumn{1}{c}{0.200} & \multicolumn{1}{c}{0.141} & \multicolumn{1}{c}{0.203} \\ 
Adjusted R$^{2}$ & \multicolumn{1}{c}{0.012} & \multicolumn{1}{c}{0.200} & \multicolumn{1}{c}{0.141} & \multicolumn{1}{c}{0.203} \\ 
Residual Std. Error & \multicolumn{1}{c}{0.368 (df = 635552)} & \multicolumn{1}{c}{0.331 (df = 635551)} & \multicolumn{1}{c}{0.343 (df = 635550)} & \multicolumn{1}{c}{0.331 (df = 635549)} \\ 
F Statistic & \multicolumn{1}{c}{7,845.032$^{***}$ (df = 1; 635552)} & \multicolumn{1}{c}{79,439.180$^{***}$ (df = 2; 635551)} & \multicolumn{1}{c}{34,872.520$^{***}$ (df = 3; 635550)} & \multicolumn{1}{c}{40,451.410$^{***}$ (df = 4; 635549)} \\ 
\hline 
\hline \\[-1.8ex] 
\textit{Note:}  & \multicolumn{4}{r}{$^{*}$p$<$0.1; $^{**}$p$<$0.05; $^{***}$p$<$0.01} \\ 
\end{tabular}}
    \label{tab:urban wage premium}
\end{table}




% \begin{figure*}[!h]
%     \centering
%     \includegraphics[width=0.95\textwidth]{Nature_HB_2023/figNHB/Fig 6 - Jul 11 2023.png}
%     \caption{\textbf{Spatial Distribution of Skill Categories.}
%      \textbf{(a)} General, \textbf{(b)} Nested, and \textbf{(c)} Un-nested skill levels of each county's occupational composition, using their standard score (z-score) relative to the national level (see Methods). 
%      The most populated counties in each state are enclosed in a box, and the top five and bottom five U.S. counties are highlighted in italics.
%      There is a noticeable concentration of general and nested skills in densely populated areas, while rural areas demonstrate a higher level of un-nested skills.
%      \textbf{(d)} and \textbf{(e)} illustrate the average skill levels (and 95\% confidence intervals) of each skill category in relation to population size and manufacturing industries, respectively.}
%     \label{fig:Geography}
% \end{figure*}


% Expanding the above figure, Figs.~\ref{fig:geo_dist_level_general_skills_employment_weighted} through \ref{fig:geo_dist_level_un-nested_specific_skills_employment_weighted} show the geographical distribution of each skill category.
% The largest cities from each state (in terms of population) are shown as boxed labels.
% The bottom histogram shows the distribution of the corresponding stock of skills across the US economy.
% The y-axis shows the number of unique Federal Information Processing Standards (FIPS) with the respective skill level.


% Overall, urban areas are more endowed with general skills— seen in red.
% In contrast, rural areas are less likely to carry general skills— seen in blue.
% This is consistent with the concentration of more innovative and complex economic activity \cite{Hong2020, Balland2020} and the concentration of managerial and administrative occupations in cities.
% For instance, New York and Washington D.C. harbor significant financial and state employment.
% Moreover, even in states with comparatively rural structures, such as Indiana, Iowa, Nebraska, and Kansans, state capitals, where the local state is likely to reside, command a higher level of general skills— than their neighboring counties.
% A secondary driver of the abundance of general skills in urban areas is the specialization needed for accomplishing complex economic tasks.
% For instance, Boston, Seattle, and San Francisco (the latter not shown on the map) are tech hubs and command a strong stock of general (, and as seen in Fig.~\ref{fig:geo_dist_level_nested_specific_skills_employment_weighted}, specific) skills.
% Nonetheless, the most extreme concentration of general skills (or lack thereof) is observed in less populated cities that are specialized in a certain industry.
% The significant proportion of the focal industries' workers relative to the total employment highlights the skills used by those workers.


% The most extreme concentrations of general skills (or lack thereof) are observed in less populated cities that are specialized in a certain industry.
% The significant proportion of the focal industries' workers relative to the total employment highlights the skills used by those workers.
% The five most and least endowed counties with general skills are shown on the map— as italicized text.
% For instance, St. Mary County (Maryland) is an air force and aerospace hub with companies such as Lockheed Martin, and Boeing, and military naval air station Patuxent River among the top employers\footnote{\tiny\url{https://commerce.maryland.gov/Documents/ResearchDocument/MajorEmployersInStMarysCounty.pdf}}.
% Another example is Chatham, and its neighboring counties, Durham (hosting Duke University), Orange (hosting the University of Carolina at Chapel-Hill), and Person, which have fostered one of the fastest growing tech sectors\footnote{\tiny\tiny\url{https://www.thecentersquare.com/north_carolina/durham-chapel-hill-nc-has-one-of-the-fastest-growing-tech-sectors/article_f656132a-67c5-5b50-a288-86d5b77372b8.html}} in the US, earning the nickname of \textit{Research Triangle}\footnote{\tiny\url{https://en.wikipedia.org/wiki/Research_Triangle}}.
% Other notable concentration points of general skills are Limestone\footnote{\tiny\url{https://www.naida.com/our-region/limestone-county/}} and Madison (Alabama)\footnote{\tiny\url{https://www.naida.com/our-region/madison-county/}}, hosting numerous aerospace and automobile manufacturing facilities, and Washtenaw county (Michigan) hosting the University of Michigan Ann Arbor and its off-sprung businesses.
% In contrast, Madera (California)\footnote{\tiny\url{https://centralcalifornia.org/location-advantages/madera-county/#:~:text=Major\%20operations\%20include\%3A\%20Gallo\%20Winery,the\%20geographical\%20center\%20of\%20California.}}, and its neighboring counties, Tulare, Kings, and Monterey)Highlands (Florida)\footnote{\tiny\url{https://www.naida.com/our-region/limestone-county/}}, Yuma (Arizona), Hall (Georgia), and Kalawao (Hawaii) are primarily designated agricultural areas, accruing unnested skills.

%% Geographic Distribution of Average General Skills Weighted by Employment.png
% \begin{sidewaysfigure}[ht]
\begin{figure}[ht]
    \centering
    \includegraphics[width=\textwidth, trim={5cm 0 2cm 2cm},clip]{Nature_HB_2023/SI_figNHB/Geography_of_General_Skills__full_fig__Jul_15_2023.png}
    \caption{\textbf{Geographic Distribution of Average General Skills Weighted by Employment.} The bottom histogram shows the distribution of the corresponding stock of skills across the US economy. The y-axis shows the number of unique FIPS with the respective skill level.
    Overall, urban areas are more endowed with general skills— seen in red. In contrast, rural areas are less likely to carry general skills— seen in blue. This is consistent with the concentration of more innovative and complex economic activity \cite{Hong2020, Balland2020} and the concentration of managerial and administrative occupations in cities.}
    \label{fig:geo_dist_level_general_skills_employment_weighted}
\end{figure}

%% Geographic Distribution of Average Nested intermediate Skills Weighted by Employment.png
\begin{figure}[!h]
    \centering
    \includegraphics[width=.875\textwidth, trim={5cm 0 2cm 2cm},clip]{Nature_HB_2023/SI_figNHB/Geography_of_Intermediate_Nested_Skills__full_fig__Jul_15_2023.png}
    \caption{\textbf{Geographic Distribution of Average Nested Intermediate Skills Weighted by Employment.}
    The bottom histogram shows the distribution of the corresponding stock of skills across the US economy. The y-axis shows the number of unique FIPS with the respective skill level.}
    \label{fig:geo_dist_level_nested_common_skills_employment_weighted}
\end{figure}

%% Geographic Distribution of Average Nested Specific Skills Weighted by Employment.png
\begin{figure}[!h]
    \centering
    \includegraphics[width=.875\textwidth, trim={5cm 0 2cm 2cm},clip]{Nature_HB_2023/SI_figNHB/Geography_of_Nested_Specific_Skills__full_fig__Jul_15_2023.png}
    \caption{\textbf{Geographic Distribution of Average Nested Specific Skills Weighted by Employment.}The bottom histogram shows the distribution of the corresponding stock of skills across the US economy. The y-axis shows the number of unique FIPS with the respective skill level.}
    \label{fig:geo_dist_level_nested_specific_skills_employment_weighted}
\end{figure}

%% Geographic Distribution of Average Un-nested Common Skills Weighted by Employment.png
\begin{figure}[!h]
    \centering
    \includegraphics[width=.875\textwidth, trim={5cm 0 2cm 2cm},clip]{Nature_HB_2023/SI_figNHB/Geography_of_Un-nested_Intermediate_Skills__full_fig__Jul_15_2023.png}
    \caption{\textbf{Geographic Distribution of Average Un-nested Intermediate Skills Weighted by Employment.}The bottom histogram shows the distribution of the corresponding stock of skills across the US economy. The y-axis shows the number of unique FIPS with the respective skill level.}
    \label{fig:geo_dist_level_common_specific_skills_employment_weighted}
\end{figure}

%% Geographic Distribution of Average x Skills Weighted by Employment.png
\begin{figure}[!h]
    \centering
    \includegraphics[width=.875\textwidth, trim={5cm 0 2cm 2cm},clip]{Nature_HB_2023/SI_figNHB/Geography_of_Un-nested_Specific_Skills__full_fig__Jul_15_2023.png}
    \caption{\textbf{Geographic Distribution of Average Un-nested Specific Skills Weighted by Employment.}The bottom histogram shows the distribution of the corresponding stock of skills across the US economy. The y-axis shows the number of unique FIPS with the respective skill level.}
    \label{fig:geo_dist_level_un-nested_specific_skills_employment_weighted}
\end{figure}



\subsection{Skills and Population}
% In the main Fig.~\ref{fig:Geography}~(d), 
We divide cities into four mutually exclusive groups by population (below 10 thousand, below 50 thousand, below 1 million, and more than a million inhabitants) based on 2010 Census population estimates.
Skill endowment for each city group is taken as the average of counties, and 95\% confidence intervals are shown.
Fig.~\ref{fig:skill_and_population_full} shows for cities of different size the levels of
% repeats the analysis of main Fig.~\ref{fig:Geography}~(d) for
all skill categories.

\begin{figure*}[!h]
    \centering
    \includegraphics[width=\textwidth]{Nature_HB_2023/SI_figNHB/Population_and_skills__full_figure__Jul_15_2023.png}
    \caption{\textbf{Population Size and Skills}
    The figure shows for cities of different size the levels of all our skill categories, highlighting the statements for nested and un-nested specific skills also hold for the corresponding intermediate skills.}
    \label{fig:skill_and_population_full}
\end{figure*}

We also test the hypothesis that the accumulation of general skills indeed explains part of the value generated in cities \cite{Lucas1988, Bacolod2009, gomez2016explaining}.
we test that hypothesis directly by utilizing the CBSA size (CBSASZ) variable from CPS microdata, which carries information about the size of the metropolitan area in which the surveyed individual resides (since 2004).
The values range from 0: areas of $<100,000$ inhabitants that do not meet the threshold of a metropolitan area to 6: over 5 million inhabitants. We transform these brackets to cities below and above 1M population \cite{Hong2020}.

In the model (1) of Tab.~\ref{tab:urban wage premium}, we first regressed the log wage reported by individuals to CPS on the size of the metropolitan area in which they reside, obtaining partial correlations that signify the urban wage premiums (the baseline is areas of $<1M$ inhabitants.) In the second model of the table, we add general skills of individuals (which we obtain from matching to O*NET the occupation associated with each individual in the CPS microdata). That means that large cities tend to have more people in occupations with general skills. This bias toward more general-skill intensive activities explains over one-third of the urban wage premiums \cite{Lucas1988, Glaeser2001, Bacolod2009, Gomez-Lievano2021}. Adding nested and un-nested specific skills first without and then with general skills in models 3 and 4, respectively, have similar effects.



\subsection{Skills and Manufacturing Industries}
% The main Fig.~\ref{fig:Geography}~(e) 
We divide cities into four mutually exclusive groups based on the intensity of their manufacturing industries.
We use US Census County Business Patterns from 2019 that report industry employment for metropolitan areas to quantify manufacturing presence.
At the 2-digit naics codes, we take 31-33 as manufacturing industries and calculate the location quotient of manufacturing employment (the ratio of manufacturing employment from the metro area total employment over the nationwide ratio).
Matching metro areas to counties, we designate counties with no manufacturing employment to group "None", and group the rest based on quotient 33\% and 66\% quantiles of the measure into bottom, middle, and top.
Fig.~\ref{fig:skill_and_manufacturing_full} shows for cities of different manufacturing concentrations the levels of
all skill categories.
% , and Fig.~\ref{fig:skill_and_manufacturing_map} shows the classification of counties as described above— with the county names in the \textit{top} manufacturing bracket appearing on the US map.


\begin{figure*}[!h]
    \centering
    \includegraphics[width=\textwidth]{Nature_HB_2023/SI_figNHB/Manufacturing_and_skills__full_figure__Jul_15_2023.png}
    \caption{\textbf{Intensity of Manufacturing Industries and Skills}
    The figure shows for cities of different manufacturing concentrations the levels of all skill categories, highlighting the statements for nested and un-nested specific skills hold also for the corresponding intermediate skills.}
    \label{fig:skill_and_manufacturing_full}
\end{figure*}


% \begin{sidewaysfigure}[!h]
%     \centering
%     \includegraphics[width=0.85\textwidth]{Nature_HB_2023/SI_figNHB/Jun 14 2023 - us map - Manufacturing Jobs Quotient.png}
%     \caption{\textbf{Classification of Counties based on the Intensity of their Manufacturing Industries}
%     The top manufacturing counties capture the Indiana, the North Carolina, and the Georgia Industry Hubs.}
%     \label{fig:skill_and_manufacturing_full}
% \end{sidewaysfigure}

% \tempdima=\textwidth% inside \textwidth=\linewidth=\textheight
%   \begin{sidewaysfigure}
%   \textheight=\tempdima
%   \settoheight{\tempdimb}{\includegraphics[width=\textwidth]{Nature_HB_2023/SI_figNHB/Jul 15 2023 - us map - Manufacturing Jobs Quotient.png}}% default height
%   \savebox{\tempbox}{\begin{minipage}{\textwidth}
%     \caption{\textbf{Classification of Counties based on the Intensity of their Manufacturing Industries.} The top manufacturing county bracket captures Indiana, North Carolina, and the Georgia Industry Hubs. Cutoff thresholds and the distribution of manufacturing employment quotient are shown in the inset— we apply hard thresholds, and the overlaps in the inset arise from binning.}
%     \addtocounter{figure}{-1}% reset
%     \rule{\textwidth}{\tempdimb}% same size but faster
%     \end{minipage}}%
%   \tempdima=\ht\tempbox\relax% can't get \dimexpr to work
%   \advance\tempdima by \dp\tempbox\relax
%   \tempdimc=\tempdimb
%   \ifdim\textheight<\tempdima
%     \advance\tempdimb by \textheight
%     \advance\tempdimb by -\tempdima
%   \fi
%   \tempdima=\dimexpr \tempdimb*\textwidth/\tempdimc\relax% compute scaled width
%   \hfil\begin{minipage}{\tempdima}
%   \includegraphics[width=\textwidth]{Nature_HB_2023/SI_figNHB/Jun 14 2023 - us map - Manufacturing Jobs Quotient.png}
%   \caption{\textbf{Classification of Counties based on the Intensity of their Manufacturing Industries.} The top manufacturing county bracket captures Indiana, North Carolina, and Georgia Industry Hubs. Cutoff thresholds and the distribution of manufacturing employment quotient are shown in the inset— we apply hard thresholds, and the overlaps in the inset arise from binning.}
%   \label{fig:skill_and_manufacturing_map}
%   \end{minipage}
% \end{sidewaysfigure}




%%----------------------------------------------------------------------------%%
\clearpage
\section{Skills' Demographic Distribution} \label{section: add - demographic distribution of skills}

Using the CPS household data between 1980 and 2022, we derive the skill endowment across racial (White, Black, Hispanic/Latinx, and White) and gender (Female and Male) groups in each skill category.
Restricting to full-time workers employed at the time of the survey, who are between 18 and 55,
we apply the mentioned features to examine the prevalence of skills among individuals of different gender and racial groups.
Individuals' skills are infered based on their coded occupations in the CPS data by linking it to the occupational skill requirement in O*NET.
The two datasets, however, use different occupational taxonomies.
As a result, one needs to map CPS and O*NET occupations.
We use a crosswalk offered by BLS\footnote{\tiny\url{https://www.census.gov/topics/employment/industry-occupation/guidance/code-lists.html}}, which maps a CPS occupation to 542 out of 968 occupations in O*NET 8-digit SOC codes.
Note that CPS offers various racial categories. We use Whites, Blacks, and Asians, which constitute the bulk of the sample.
CPS data also contains a separate (from race) variable for identifying Hispanic individuals.
We create a fourth racial category for Hispanics and associate any individual of Hispanic background with that category.
Next, we calculate the endowment of each skill category for each of the resulting four demographic categories and (binary) gender groups.


% Adjusting the income wage for inflation and accounting for the number of hours worked, we compute an adjusted weekly wage that is comparable across the population.
Figure~\ref{fig:racial and gender skill endowments} replicates the main Fig.~\ref{fig:Figure 7}, adding the information on intermediate skills and annual wage. We omit the weekly wage results for brevity.



%% Main Figure 5 + All Skills
\begin{figure}[!h]
    \centering
    \includegraphics[width=\textwidth]{Nature_HB_2023/SI_figNHB/SI__Gender_and_Racial_Gaps__full_fig__Jul_15_2023.png}
    \caption{\textbf{Skill Disparity in Demographic Distribution} of race/ethnicity and gender adding the information on intermediate skills and annual wage.}
    \label{fig:racial and gender skill endowments}
\end{figure}


As a robustness check, we used a different measurement of skills for demographics and found similar results, following Tong et al. \cite{Tong2021}.
They group occupations of different skill levels by corresponding workers’ dominant gender and race/ethnicity and calculate skill endowment across occupations from the same group.
In determining occupations’ “dominant” demographic characteristics, we link an occupation to a racial/gender group if it is 1.5 times or more likely to be employed in the focal occupation than its fraction in the sample.
We then aggregated skill endowments across racial and gender categories and show the results in
Fig.~\ref{fig:Tong et al race gender skill distribution}.
The results are consistent with our main Fig.~\ref{fig:Figure 7} and SI Fig.~\ref{fig:racial and gender skill endowments}.

%% Demographic Distribution of Average x Skills Weighted by Employment.png
\begin{figure}[!h]
    \centering
    \includegraphics[width=\textwidth]{Nature_HB_2023/SI_figNHB/SI__Tong_et_al__race_and_gender_ratios__Jul_15_2023.png}
    \caption{\textbf{Skill Disparity in Demographic Distribution} of race/ethnicity and gender with an alternative aggregation. Similar to our main Fig.~\ref{fig:Figure 7}, we use CPS micro data, however, follow the aggregation of \cite{Tong2021}. The results are consistent with our main figure.}
    \label{fig:Tong et al race gender skill distribution}
\end{figure}

% However, the distribution of general and nested skills heavily favors male (over female) White and Asian workers. 
% In contrast, Hispanics and Black possess a considerably smaller level of these valuable skills.
% While being female almost universally is associated with slightly lower endowments, differences vary across racial groups: for instance, White men and women have closer general, nested intermediate skills, while Black women outperform Black men in these skill categories.
% in general and nested skills. However, in un-nested skills, males manifest significantly higher endowments. This is intuitive, given men's propensity to be hired in more physically demanding occupations.

% The figure shows racial and gender differences in the stock of valuable skills.
% White and Asian workers possess significantly higher general and nested skills than Hispanics and Blacks.
% Interestingly, white (male) workers also remain competitive in un-nested skills.
% In most cases, being a female is associated with slightly lower general and nested skills.
% Therefore, it is challenging to differentiate the gender pay gap in their skill statistics at this level— our Fig.~\ref{fig:age} and Fig.~\ref{fig:Skill Age Gender Race Trends} show divergence in the timeline of skill acquisition, which may play a role.
% However, female workers manifest a lower tendency toward un-nested skills across the board.


\subsection{Parenthood and the Diverging Skills of Male and Female Workers} \label{supsec:Parenthood_Male_vs_Female}

An intriguing pattern in the main Figs.~\ref{fig:age}~(d-e) is the diverging general and nested skills of men and women around the age of 30, when one expects some individuals to become parents.
Utilizing the number of children in the surveyed households recorded by CPS microdoa, we split our birth cohort sample into individuals \textit{with} and \textit{without} children. 
We replicate the analysis of the main Fig.~\ref{fig:age}~(d-f) by tracking the skills manifested in the occupational compositions of birth cohorts as they age, splitting individuals based on their binary gender (Male: lower panels; Female: upper panels of the below figure) and their binary parental status (with child: square; without child: triangle in the below figure) at the time of the survey.
Fig.~\ref{fig:Parenthood_Male_vs_Female} shows the result of aggregating skills for each subgroup. Each column shows the levels of a certain category of skills, while the rows show the results for a gender.
The solid line (and triangles) show the pattern for people without children, while the dashed line (and squares) show the pattern for individuals with children.


\begin{figure*}[!h]
    \centering
    \includegraphics[width=\textwidth]{Nature_HB_2023/SI_figNHB/Child_bearing_Feb21_2024_alt2_comment_1.8.png}
    \caption{\textbf{Parenthood and the Diverging Skills of Male and Female Workers}.
    We track the skills manifested in the occupational compositions of birth cohorts as they age, splitting individuals based on their binary gender (Make/Female) and whether they lived with children at the time of the survey, obtaining the below figure. Each column shows the levels of a certain category of skills, while the rows show the results for men and women. The solid line (and triangles) show the pattern for people without children, while the dashed line (and squares) show the pattern for individuals with children, highlighting the drop in general and nested skills of mothers.
    }
    \label{fig:Parenthood_Male_vs_Female}
\end{figure*}

There is a pronounced gap in the general and nested skills between women with and without children. Please note that the later convergence is likely to arise from the fact that at higher ages, the “without" subgroup will mix families who never had any children with families whose children have already left the household.
In the latter families, caregivers may have been disadvantaged in their early careers, leading to lower skill levels at higher ages.
Contrary to the negative correlation with general and nested skills, women with children appear to sort into jobs that require higher un-nested skills (the SI Sec. \ref{supsec:female job sorting} offers partial evidence of female job sorting).
Interestingly, men with children tend to do better.
Especially men in jobs that require general and nested skills tend to be more intriguing and for longer periods compared to their counterparts without children.
The latter pattern for men may arise from sample selection effects or from the fact that the cost of raising children incentivizes acquiring skills that lead to better-paid careers.
Synthetic birth cohorts are not ideal data for this purpose, as they do not allow for tracking individuals over time. However, it is reasonable to believe this approach offers unbiased estimates of the population behavior.



\subsection{Gender and Jobs Sorting} \label{supsec:female job sorting}

Another intriguing pattern in the main Fig.~\ref{fig:age}~(d-f) is the diverging patterns of skill development between men and women, wherein women exhibit high levels of general skills, surpassing their male counterparts at certain ages but do not manifest the high levels of nested skills observed for male workers of the same age.
Lower levels of nested skills for women are also seen in the first column of the regression Tab.~\ref{tab:irregular hours} that predicts the gender of workers based on their general and nested skills in our CPS sample (Female = 1): general skills are associated with greater, but nested skills with smaller shares of women in an occupation.

One explanation for this pattern \cite{Bertrand2009, Goldin2015, Canon2016} is that women may avoid jobs with irregular or long working schedules.
This implies that despite their high levels of general skills and education, women may avoid jobs that require nested skills because of the working conditions of such jobs.
To examine that hypothesis, we examined whether adding descriptors of work schedule to the same regression diminishes the correlation between skills and the gender of the worker, as reported in column 1 of Tab.~\ref{tab:irregular hours}. 


%% Wage Regression Table
\begin{table}[!h]
    \centering
    % \begin{threeparttable}
    \caption{\textbf{Regression analysis of the correlation between gender, skills, and irregular and long work schedule.}
    The first column offers a baseline model that predicts the gender (Female = 1) of the worker based on general and nested skills, showing a negative correlation with nested skills. Adding descriptors of irregular and long schedules in the second model explains away part of the predictive power of nested skills for workers’ gender. As such, part of the reason why women manifest high level s of general skills but comparatively low levels of nested skills is that jobs that require the latter categories of skills likely impose long and irregular work conditions, which have been found to deter female workers.}
    \resizebox{\columnwidth}{!}{
    \begin{tabular}{@{\extracolsep{5pt}}lD{.}{.}{-3} D{.}{.}{-3} } 
\\[-1.8ex]\hline 
\hline \\[-1.8ex] 
 & \multicolumn{2}{c}{\textit{Dependent variable:}} \\ 
\cline{2-3} 
\\[-1.8ex] & \multicolumn{2}{c}{Gender Dummy (Female = 1)} \\ 
\\[-1.8ex] & \multicolumn{2}{c}{\textit{OLS}} \\ 
\\[-1.8ex] & \multicolumn{1}{c}{(1)} & \multicolumn{1}{c}{(2)}\\ 
\hline \\[-1.8ex] 
 General Skills & 0.203^{***} & 0.150^{***} \\ 
  & \multicolumn{1}{c}{(0.201$, $0.204)} & \multicolumn{1}{c}{(0.148$, $0.153)} \\ 
  & & \\ 
 Nested Skills & -0.357^{***} & -0.258^{***} \\ 
  & \multicolumn{1}{c}{(-0.359$, $-0.355)} & \multicolumn{1}{c}{(-0.261$, $-0.256)} \\ 
  & & \\ 
 Irregular Schedule &  & -0.338^{***} \\ 
  &  & \multicolumn{1}{c}{(-0.342$, $-0.334)} \\ 
  & & \\ 
 Long Hours Dummy ($>50$ weekly) &  & -0.176^{***} \\ 
  &  & \multicolumn{1}{c}{(-0.178$, $-0.174)} \\ 
  & & \\ 
 Constant & 0.097^{***} & 0.629^{***} \\ 
  & \multicolumn{1}{c}{(0.092$, $0.101)} & \multicolumn{1}{c}{(0.620$, $0.638)} \\ 
  & & \\ 
\hline \\[-1.8ex] 
Observations & \multicolumn{1}{c}{1,493,142} & \multicolumn{1}{c}{1,096,362} \\ 
R$^{2}$ & \multicolumn{1}{c}{0.072} & \multicolumn{1}{c}{0.108} \\ 
Adjusted R$^{2}$ & \multicolumn{1}{c}{0.072} & \multicolumn{1}{c}{0.108} \\ 
Residual Std. Error & \multicolumn{1}{c}{0.463 (df = 1493139)} & \multicolumn{1}{c}{0.455 (df = 1096357)} \\ 
F Statistic & \multicolumn{1}{c}{57,942.160$^{***}$ (df = 2; 1493139)} & \multicolumn{1}{c}{33,058.290$^{***}$ (df = 4; 1096357)} \\ 
\hline 
\hline \\[-1.8ex] 
\textit{Note:}  & \multicolumn{2}{r}{$^{*}$p$<$0.1; $^{**}$p$<$0.05; $^{***}$p$<$0.01} \\ 
\end{tabular} }
    \label{tab:irregular hours}
\end{table}
%%

To implement this test, we matched individuals in the Current Population Survey (CPS) aged 18 to 55 who were in the workforce between 1980 and 2020 to the following information in the O*NET using their reported occupation code: skill information (namely, general and nested specific skills) and occupational work schedule (irregularity). Work schedule irregularity is collected as a part of the O*NET work context record as a categorical variable with three levels of
\textit{Regular} (established routine, set schedule),
\textit{Irregular} (changes with weather conditions, production demands, or contract duration), and
\textit{Seasonal} (only during certain times of the year).
This variable is reported for all occupations with weights associated with each category. For example, Chief Executive has the majority of weight in category 1, as it is primarily a job with a regular schedule. A surgeon has more weight, in comparison, on the irregular category. Using the weights, we obtained an aggregated “schedule irregularity” score for each occupation, wherein a value closer to 1 denotes a more regular schedule, and a value closer to 3 denotes a more irregular schedule.
Next, to proxy \textit{long working hours}, we follow Cha et al. \cite{Cha2014} to use the number of hours worked during the week in the CPS data and form a dummy variable that is one if the worker had worked more than 50 hours a week, and 0 otherwise.

Adding the descriptors of irregular schedules or long hours (in Tab.~\ref{tab:irregular hours} column 2) indeed diminishes the correlation estimated in the baseline, per the baseline model (column 1).
A unit increase in the nested specific skills required by a job, decreases the chances of the worker being female by 36\%. Adding schedule descriptors reduces that relation by more than one third, to about 26\%.




\subsection{Skills and Wage Gaps Have Narrowed Over Years}


Figures \ref{fig:Temporal Race Gaps - Skills, Education, Wages} and \ref{fig:Temporal Gender Gaps - Skills, Education, Wages} below show the temporal dynamics of skill, education, and wage gaps shown as averages in the main Fig.~\ref{fig:Figure 7}. These figures show the gaps have narrowed over years.

\begin{figure*}[!h]
    \centering
    \includegraphics[width=\textwidth]{Nature_HB_2023/SI_figNHB/Temporal_Racial_Gap_Ratios_MA__annual_wage__Skills_Education_and_Wages__1980-2022__Jul_15_2023.png}
    \caption{\textbf{Skill and Economic Race/Ethnicity Gaps over Time}.
    Following main Fig.~\ref{fig:Figure 7}, we use White workers as the baseline and show each measure of other demographics as a ratio over the values of White workers.
    In all cases, The 95\% confidence intervals are obtained by a random sub-sampling. In each iteration, we take 20\% of the subpopulation of interest at a certain year, for instance, Asian male and Asian female workers in 2020, and estimate all corresponding measures. Repeating this sampling and estimation process in 10,000 iterations, we obtain the distribution for each estimate (of the subpopulation of interest in that year) and derive the 95\% confidential intervals.
    }
    \label{fig:Temporal Race Gaps - Skills, Education, Wages}
\end{figure*}


\begin{figure*}[!h]
    \centering
    \includegraphics[width=\textwidth]{Nature_HB_2023/SI_figNHB/Temporal_Gender_Gap_Ratios_MA__annual_wage__Skills_Education_and_Wages__1980-2022__Jul_15_2023.png}
    \caption{\textbf{Skill and Economic Inequality Across Genders over Time}.
    Following main Fig.~\ref{fig:Figure 7}, we use male workers in each racial group as the benchmark, showing the average value for women over men for each measure and each demographic.
    In all cases, 95\% confidence intervals are created as explained in Fig.~\ref{fig:Temporal Race Gaps - Skills, Education, Wages}.}
    \label{fig:Temporal Gender Gaps - Skills, Education, Wages}
\end{figure*}

\newpage
\subsection{Gender-Age Divergence of Skills across Demographic Groups}

Fig.~\ref{fig:Skill Age Gender Race Trends} replicates Fig.~\ref{fig:age}. However, it teases out time trends in skill acquisition for racial groups.
% A noticeable pattern is (both male and female) white workers' advantages in general and nested skills tend to persist as they age.
The skill differentials between male and female workers that start around the age 30s (main Fig.~\ref{fig:age}) manifest across racial and ethnic groups.
In most cases, female workers' (general and nested) skill accumulation plateaus in their mid-20s to 30s, while their male counterparts' skill stocks expand (even though slowly) up to their 40s and then plateaus.
Section \ref{supsec:Parenthood_Male_vs_Female} addresses the possible role of children in the divergence of skills.
% An exception is Asians. Particularly female Asian workers' general and nested skills increase up to their 30s and decline with age.
% This skill decline with age is likely due to the younger better-educated Asian workers that enter the sample over time.
% The observed stagnation of female workers' skills around their 30s may arise from their propensity to bear children in that period.
Fig.~\ref{fig:Skill Age Gender Race Trends - year effects} replicate the exercise factoring in annual economic circumstances.

\begin{figure*}[!h]
    \centering
    \includegraphics[width=\textwidth, trim={2cm 0 1cm 1cm},clip]{Nature_HB_2023/SI_figNHB/SI__Skill_Acquisition_and_Age__Gender_and_Race__Jul_15_2023.png}
    \caption{\textbf{Skill Acquisition Differences Across Gender and Race.}
    Using CPS household data between 1980 and 2022, we estimate the distribution of different skills over four racial categories, namely \textit{White, Asian, Black/African-American, Hispanic/Latino}, and for men and women.
    We infer individuals' skills from the skill requirement of their detailed occupation according to O*NET 2019 and calculate a skill endowment for a given race and gender in each skill subtype.
    We aggregate skill endowments for racial and gender groups over age, allowing us to estimate the usage of workplace skills for these subgroups as they age. Shaded areas show 95\% confidence intervals.}
    \label{fig:Skill Age Gender Race Trends}
\end{figure*}


%% replicating fig:individuals' age and skill - year effects and adding year effect
\begin{figure*}[!h]
    \centering
    \includegraphics[width=\textwidth, trim={2cm 0 1cm 1cm},clip]{Nature_HB_2023/SI_figNHB/SI__Race_Gender_Skill_and_Age__conditioned_on_year_effect__Jul_15_2023.png}
    \caption{\textbf{Skill Acquisition Differences Across Gender and Race Conditional on Year Effects.}
    The setup follows Fig.~\ref{fig:Skill Age Gender Race Trends} with the minor difference that skill levels are first residualized by a year effect. The results are almost identical.
    }
    \label{fig:Skill Age Gender Race Trends - year effects}
\end{figure*}





%%------------------------------------------------------------------%%
\clearpage
\section{Historical Patterns of Skill Change for Occupations} \label{supsec: historical skill change}
How have occupational skills (Level and Importance) changed over time? 
Do they manifest our theorized co-evolution of skills at different rungs of the skill hierarchy?
In other words, can we observe our inferred conditional dependence and independence on the level and importance of skills?


To answer these questions, we compared the level and importance of occupational skills reported by O*NET in 2019 and 2005.
When comparing levels and importance of skills across the two years, we use a crosswalk, explained in the supplementary section \ref{sec: onet taxonomy change}, to account for the changes in the taxonomy between 2005 and 2019 \cite{Hopson2021}.
We further capture changes in the skill structure by comparing the skill structure of 2019 to a past snapshot of O*NET from 2005.
We produced the mentioned 2005 skill structure using the same methodology and parameters as used for the 2019 skill dependency network.


\subsection{Changes in the Skill Levels}
% As a broad-level analysis, we aggregated skills of each occupation, presenting both 2005 and 2019 skill data, in each of our five skill categories in the year 2005 and in the year 2019, and measure the correlation between changes in each skill category across occupations in Fig.~\ref{fig:correlation_level_2005_2019}. The results are consistent with the coupling of general and nested skills and the decoupling of general and un-nested skills.


% %% Skill-skill Nestedness Matrix
% \begin{figure*}[!h]
%     \centering
%     \includegraphics[width=.5\textwidth]{Nature_HB_2023/SI_figNHB/Skill Change correlation - 2005 to 2019 - Jul 18 2023.png}
%     \caption{\textbf{Correlation between the Level Changes in each Skill Category (across occupations).}}    \label{fig:correlation_level_2005_2019}
% \end{figure*}


Figure \ref{fig:occupation groups historical changes to skill cluster and type levels} shows the changes in the level of each skill across occupations between 2005 and 2019. 
For each occupation and skill, we subtracted the level O*NET reports in 2019 from the level in 2005. For each skill, we show the density plot of occupations based on their corresponding level change.
The white line for each skill denotes the median, and the dashed line corresponds to no change.

%% High level changes in skills from 2005 to 2019
\begin{figure*}[!h]
    \centering
    \includegraphics[width=.8\textwidth]{Nature_HB_2023/SI_figNHB/Detailed_Skill_Level_Change__2005_to_2019__Jul_18_2023.png}
    \caption{\textbf{Changes in the Level of Each Skill across Occupations between 2005 and 2019 as reported by O*NET.}}
    \label{fig:occupation groups historical changes to skill cluster and type levels}
\end{figure*}




\subsection{Changes in the Skill Dependency Network}
Fig.~\ref{fig:parsi_skill_dep_2005_labeled} shows the backbone of the skill dependency network based on 2005, as appears in Fig.~\ref{fig:Figure 7} with skill labels attached.

%% Skill Co-appearance Network for 2003 and 2019
\begin{figure*}[!h]
    \centering
    \includegraphics[width=\textwidth]{Nature_HB_2023/SI_figNHB/Jul_9__2005_Network__Equirectangular_Proj__Edu-LRC__Parsi__4.75_0.05_.4_.275__adjusted_for_labels.png}
    \caption{\textbf{Labeled Backbone of Skill Dependency for the year 2005}.
    }
    \label{fig:parsi_skill_dep_2005_labeled}
\end{figure*}


In Fig.~\ref{fig:Figure_7_supp}, we unpack the changes that manifest in distinct backbones of skill dependency networks in 2019 compared with 2005 (as shown in Fig.~\ref{fig:Figure 7}).
We compare the dependency ties between all skills that are present in both 2005 and 2019 networks (Installation, Explosive strength, Sound localization, Food production, Public safety, and security in 2019 and Memorization, Food production, Chemistry, and Public safety and security in 2005 are eliminated due to a lack of statistically significant ties.)
We distinguish between three types of ties: 1) \textit{New edges}: dependency ties that were statistically insignificant in 2005 and became significant in 2019 (shown in green); 2) \textit{Constant edges}: dependency ties that were statistically significant in both 2005 and 2019 (shown in black); 3) \textit{Lost edges}: dependency ties that lost statistical significance in 2019 while being significant in 2005 (shown in orange).
In Fig.~\ref{fig:Figure_7_supp}, we used the layout of our main Fig.~\ref{fig:Figure 2} (b), adjusting the distance between nodes slightly to visualize edges better. The edges are replaced with the three types described above.
% The lower panel show only one type of edges, and the upper panel assembles all 
The pattern of changes in dependency ties offers insights into the widening gap between the nested and un-nested parts of the skill structure over time.
Most new edges (green) are massed in the nested section. Particularly, a noticeable number of edges tie general skills to the most niche nested skills— these dependency ties are not visible in the backbone for better visualization. However, they are taken into account in all calculations.
There are virtually no new ties formed between the general and un-nested sections.
Few previously existing ties were no longer statistically significant in 2019 (orange).
Therefore, the figure highlights the increasing intensity of dependencies in the nested part of the network, while a collapse of dependencies in the un-nested section.
These patterns emphasize our findings about the changes in the nature of work.

%% Skill Co-appearance Network for 2003 and 2019
\begin{figure*}[!h]
    \centering
    \includegraphics[width=\textwidth]{Nature_HB_2023/SI_figNHB/Organized_change_networks__Mar_30_2024.png}
    \caption{\textbf{Changes in the Skill Dependency Networks between 2005 and 2019}.
    We distinguish between three types of ties: \textbf{(b)} \textit{New edges}: dependency ties that were statistically insignificant in 2005 and became significant in 2019 (shown in green); \textbf{(c)} \textit{Constant edges}: dependency ties that were statistically significant in both 2005 and 2019 (shown in black); \textbf{(d)} \textit{Lost edges}: dependency ties that lost statistical significance in 2019 while being significant in 2005 (shown in orange). \textbf{(a)} The combined figure highlights the increasing intensity of dependencies in the nested part of the network, while a collapse of vertical dependencies in the un-nested section.}
    \label{fig:Figure_7_supp}
\end{figure*}

%%----------------------------------------------------------------------------%%
\newpage

\subsection{Occupation Taxonomy}\label{sec: onet taxonomy change}

Taxonomy has changed over time \cite{Park2020}.
Our historical analysis comparing 2005 and 2019 data must consider such changes.
2005 O*NET complies to \textit{O*NET SOC 2000}, while 2019 O*NET uses \textit{O*NET SOC 2010}, with two other waves of taxonomy change between (2006 and 2009).
Therefore, identically encoded occupations may not be comparable across these two years, and matching them requires a crosswalk.


While O*NET reports\footnote{\url{https://www.onetcenter.org/taxonomy.html}} crosswalks between each consecutive taxonomy, there is no direct crosswalk between 2005 and 2019.
We created such a crosswalk to match occupations in 2005 and 2019 using the consecutive crosswalks mentioned above\footnote{
For instance, if occupation $X_{2000}$ in taxonomy 2000 is linked to $X_{2006}$ in taxonomy 2006, and $X_{2006}$ is matched to $X_{2009}$ in taxonomy 2009, and $X_{2009}$ is matched to $X_{2010}$ in taxonomy 2010, our crosswalk will link $X_{2000}$ to $X_{2010}$.}.


Our crosswalk matches 968 occupations in 2019 skill data and 941 unique occupations in 2005 skill data.
Out of 1,334 records in our crosswalk, 362 correspond to occupations whose SOC codes have changed.
% Table \ref{tab:list of onet soc changes} shows such 362 occupations.
Fig.~\ref{fig:soc match coverage} shows the number and percentage of occupations in the skill data we could match across both years. Groups such as Computer and Math, Engineering, Health, Management, and Business have the most number of occupations with skill information added between 2005 and 2019.
Therefore, it is important to consider the unbalanced nature of the data, when interpreting analysis on the changes of skills using O*NET.

\begin{figure*}[!h]
    \centering
    \includegraphics[width=\textwidth]{Nature_HB_2023/SI_figNHB/Coverage_of_Occupations__Historical_Skill_Change__2005_to_2010.png}
    \caption{\textbf{Percentage and Number of Occupations from Occupational Groups Matched between 2005 and 2010.}}
    \label{fig:soc match coverage}
\end{figure*}



%% May 28 - SOC OCC 2000-2010 Mismatch.tex
%% \caption{List of SOC Code and Title Changes Between SOC 2000 (2003 data) and SOC 2010 (2019 data).} 
%%  \label{tab:list of onet soc changes}
% \footnotesize
% \input{\tabdir/May 28 - SOC OCC 2000-2010 Mismatch.tex}
% \normalsize


%\subsection{Coverage of Occupations} \label{sec: ONET occupation coverage}
%% Distribution of Occupations over Major Groups
%\begin{figure*}[!h]
    %\centering
    %\includegraphics[width=\textwidth]{\figdir/Distribution of Occupations over Major Groups in ONET Skill Data.png}
    %\caption{\textbf{Distribution of Occupations over Major Groups in O*NET Skill Data.} The bars display the number of unique 8-digit occupations included in the O*NET skill data. Two key patterns are observed: 1. There exists significant heterogeneity in the number of occupations from each major occupational group. This heterogeneity arises from the unbalanced granularity of the occupational classification system (SOC) and is inherited by O*NET; 2. O*NET is comprehensive in terms of including a relatively high percentage of occupations from each major group. The inclusion rates vary ranging from 78\% (Community \& Social Service) to 100\% Installation, Maintenance \& repair.}
    %\label{fig:dist_occ_major_groups}
%\end{figure*}



%%----------------------------------------------------------------------------%%

\clearpage
\section{Robustness Checks with Management and Admin Occupations and Social Skills} \label{supsec:robustness checks}

In this section, we offer supplementary analyses to our main findings and produce pieces of evidence refuting several alternative explanations.
We begin by examining whether managerial occupations drive the importance of general skills.
Next, we discuss whether the general skills' effect is driven primarily by social skills, whose importance has been a topic of growing emphasis.
We continue by offering more detailed analyses of historical changes in skill requirements and skill acquisition with age.



%%--------------------------don't forget to include this in the main text -----------------------------------------%%
\subsection{Role of Management and Administrative Occupations} \label{sec:robustness check: no managers}
Here, we test if the importance of general skills in the wage premium (Fig.~\ref{fig:Wage}) is driven by management or administrative occupations.
To do so, we identify such occupations, exclude them from our analyses at various stages, and examine the resulting changes. Table \ref{tab:list of manager occupations} lists these occupations with their annual wage and educational requirements.  
We identify those using the Standard Occupational Code (SOC) at the 2-digit level, wherein ``11'' denotes managerial occupations. 
In addition, we use descriptive terms for these occupation titles (manager, administrator, and director) to identify relevant occupations further using their titles. 
In total, we found and omitted a total of 75 occupations out of 968 occupations and collected them, sorted based on average annual wage and required education, in Table \ref{tab:list of manager occupations}.

\footnotesize
%% tab:list of manager occupations
\begin{longtable}{@{\extracolsep{\fill}} cclcc} 
  \caption{List of Manager Occupations and their Annual Wage and Education Requirements in our Sample.} 
  \label{tab:list of manager occupations} 
\\[-1.8ex]\hline 
\hline \\[-1.8ex] 
 & Code & Title & Wage & Education \\ 
\hline \\[-1.8ex]
\endhead
1 & 11-1011.00 & Chief Executives & \$ 170.5K & $7.540$ \\ 
2 & 11-1011.03 & Chief Sustainability Officers & \$ 170.5K & $6.920$ \\ 
3 & 11-9041.00 & Architectural and Engineering Managers & \$ 135.8K & $6.720$ \\ 
4 & 11-9041.01 & Biofuels/Biodiesel Technology and Product Devel... & \$ 135.8K & $6.480$ \\ 
5 & 11-2021.00 & Marketing Managers & \$ 124.2K & $6.680$ \\ 
6 & 11-3111.00 & Compensation and Benefits Managers & \$ 124.1K & $6.330$ \\ 
7 & 11-3021.00 & Computer and Information Systems Managers & \$ 123.8K & $5.550$ \\ 
8 & 11-2022.00 & Sales Managers & \$ 122.3K & $6.040$ \\ 
9 & 11-9121.00 & Natural Sciences Managers & \$ 122.1K & $8.130$ \\ 
10 & 11-9121.02 & Water Resource Specialists & \$ 122.1K & $6.860$ \\ 
11 & 11-9121.01 & Clinical Research Coordinators & \$ 122.1K & $6.060$ \\ 
12 & 11-3031.01 & Treasurers and Controllers & \$ 117.7K & $7.070$ \\ 
13 & 11-3031.02 & Financial Managers, Branch or Department & \$ 117.7K & $5.440$ \\ 
14 & 11-2011.00 & Advertising and Promotions Managers & \$ 116.3K & $5.210$ \\ 
15 & 11-3061.00 & Purchasing Managers & \$ 114.3K & $6.150$ \\ 
16 & 11-3131.00 & Training and Development Managers & \$ 112K & $6.630$ \\ 
17 & 11-3051.01 & Quality Control Systems Managers & \$ 109.6K & $6.030$ \\ 
18 & 11-3051.04 & Biomass Power Plant Managers & \$ 109.6K & $5.290$ \\ 
19 & 11-3051.03 & Biofuels Production Managers & \$ 109.6K & $4.980$ \\ 
20 & 11-3051.00 & Industrial Production Managers & \$ 109.6K & $4.920$ \\ 
21 & 11-3051.06 & Hydroelectric Production Managers & \$ 109.6K & $4.170$ \\ 
22 & 11-3051.02 & Geothermal Production Managers & \$ 109.6K & $4.060$ \\ 
23 & 11-3121.00 & Human Resources Managers & \$ 109.4K & $6.300$ \\ 
24 & 11-9033.00 & Education Administrators, Postsecondary & \$ 106.1K & $9.250$ \\ 
25 & 11-9111.00 & Medical and Health Services Managers & \$ 104.8K & $6.080$ \\ 
26 & 11-1021.00 & General and Operations Managers & \$ 103.8K & $4.920$ \\ 
27 & 11-9021.00 & Construction Managers & \$ 95.7K & $5.680$ \\ 
28 & 11-3071.03 & Logistics Managers & \$ 94.1K & $6.170$ \\ 
29 & 11-3071.02 & Storage and Distribution Managers & \$ 94.1K & $4.620$ \\ 
30 & 11-3071.01 & Transportation Managers & \$ 94.1K & $4.370$ \\ 
31 & 11-9032.00 & Education Administrators, Elementary and Second... & \$ 91.6K & $7.820$ \\ 
32 & 27-1011.00 & Art Directors & \$ 88.7K & $6.250$ \\ 
33 & 11-9039.01 & Distance Learning Coordinators & \$ 84.6K & $7.550$ \\ 
34 & 11-9039.02 & Fitness and Wellness Coordinators & \$ 84.6K & $6.580$ \\ 
35 & 15-2041.02 & Clinical Data Managers & \$ 83.7K & $6.100$ \\ 
36 & 11-9071.00 & Gaming Managers & \$ 81.6K & $3.680$ \\ 
37 & 11-9161.00 & Emergency Management Directors & \$ 79.8K & $6.120$ \\ 
38 & 11-9131.00 & Postmasters and Mail Superintendents & \$ 77K & $3.030$ \\ 
39 & 13-1011.00 & Agents and Business Managers of Artists, Perfor... & \$ 76.7K & $5.230$ \\ 
40 & 11-9013.01 & Nursery and Greenhouse Managers & \$ 76.6K & $4.960$ \\ 
41 & 11-9013.03 & Aquacultural Managers & \$ 76.6K & $4.420$ \\ 
42 & 11-9013.02 & Farm and Ranch Managers & \$ 76.6K & $3.880$ \\ 
43 & 19-1031.02 & Range Managers & \$ 67.3K & $5.950$ \\ 
44 & 11-9151.00 & Social and Community Service Managers & \$ 66.8K & $6.310$ \\ 
45 & 47-1011.03 & Solar Energy Installation Managers & \$ 66.1K & $3.680$ \\ 
46 & 25-9031.00 & Instructional Coordinators & \$ 64.6K & $7.700$ \\ 
47 & 27-2012.04 & Talent Directors & \$ 62.9K & $6.140$ \\ 
48 & 27-2012.05 & Technical Directors/Managers & \$ 62.9K & $5.770$ \\ 
49 & 27-2012.03 & Program Directors & \$ 62.9K & $5.080$ \\ 
50 & 27-2012.02 & Directors- Stage, Motion Pictures, Television, ... & \$ 62.9K & $4.570$ \\ 
51 & 11-9081.00 & Lodging Managers & \$ 60.9K & $4.890$ \\ 
52 & 11-9141.00 & Property, Real Estate, and Community Associatio... & \$ 60.2K & $5.040$ \\ 
53 & 11-9051.00 & Food Service Managers & \$ 56.7K & $2.540$ \\ 
54 & 39-4031.00 & Morticians, Undertakers, and Funeral Directors & \$ 56.5K & $4.810$ \\ 
55 & 27-2041.01 & Music Directors & \$ 56.2K & $8.210$ \\ 
56 & 11-9031.00 & Education Administrators, Preschool and Childca... & \$ 49.2K & $4.140$ \\ 
57 & 21-2021.00 & Directors, Religious Activities and Education & \$ 48.9K & $6$ \\ 
58 & 11-9199.03 & Investment Fund Managers & - & $7.410$ \\ 
59 & 11-9199.01 & Regulatory Affairs Managers & - & $6.500$ \\ 
60 & 15-1141.00 & Database Administrators & - & $6.440$ \\ 
61 & 11-9199.04 & Supply Chain Managers & - & $6.430$ \\ 
62 & 11-9199.07 & Security Managers & - & $6.150$ \\ 
63 & 11-9199.11 & Brownfield Redevelopment Specialists and Site M... & - & $6.120$ \\ 
64 & 11-2031.00 & Public Relations and Fundraising Managers & - & $6.100$ \\ 
65 & 11-9199.10 & Wind Energy Project Managers & - & $6.090$ \\ 
66 & 11-9199.09 & Wind Energy Operations Managers & - & $5.860$ \\ 
67 & 15-1199.09 & Information Technology Project Managers & - & $5.860$ \\ 
68 & 11-9061.00 & Funeral Service Managers & - & $5.710$ \\ 
69 & 11-9199.02 & Compliance Managers & - & $5.650$ \\ 
70 & 15-1142.00 & Network and Computer Systems Administrators & - & $5.580$ \\ 
71 & 15-1199.03 & Web Administrators & - & $5.350$ \\ 
72 & 11-9199.08 & Loss Prevention Managers & - & $4.950$ \\ 
73 & 39-1021.01 & Spa Managers & - & $4.220$ \\ 
74 & 11-3011.00 & Administrative Services Managers & - & $3.960$ \\ 
75 & 53-1021.01 & Recycling Coordinators & - & $3.890$ \\ 
\hline \\[-1.8ex] 
\end{longtable}
\normalsize

In conclusion, our findings of skill clusters, skills wage premiums, and educational requirements are robust to the presence/absence of managerial occupations.
In Figs \ref{fig:determining_k_kmeans_70bins_correlation_no_manager_start} through \ref{fig:skill_clusters_k_kmeans_70bins_correlation_no_manager_start} we use $k$-mean clustering to group skills into profiles without considering managerial occupations. The results complement the supplementary section \ref{supsec:skill clustering}, establishing the robustness of our skill profiles.
Fig.~\ref{fig:skill_and_age_no_manager} shows that excluding managerial occupations does not diminish the acquisition of general and nested skills over time by analyzing occupational median age, skill composition of synthetic birth cohorts based on CPS microdata, and analysis of our resume sample. 
Fig.~\ref{fig:returns_to_skills_hierachy_gen_dependence_cor_no_manager} examines wage premiums and educational requirements of occupations in general and specific skills while excluding the information on managerial occupations in the analysis. We find these occupations do not drive the wage premiums, and educational requirements persist. These findings supplement the results in Fig.~\ref{fig:Wage}.  

% In addition, we repeated the entire process without management occupations, from constructing the skill hierarchy (Fig. 2) to separating nested and Un-nested skills in order to test the wage premium, and found similar results in the main text, corroborating the robustness of our findings.
%  Fig.~\ref{fig:skill_clusters_k3_cos_no_manager_start} shows the aggregate cluster of skills that emerged from the occupation skill importance values (bar for management occupations).
% Figs \ref{fig:returns_to_skills_hierachy_gen_dependence_cor_no_manager_start} and \ref{fig:returns_to_skills_hierachy_gen_dependence_cor_no_manager_start_gen_split} collect the same set of wage analysis captured in Figs \ref{fig:returns_to_skills_hierachy_gen_dependence_cor_no_manager} and \ref{fig:returns_to_skills_hierachy_gen_dependence_cor_no_manager_gen_split}.


%\subsubsection{Excluding Managers When Identifying Skill Groups}

%% Determining k
%% SI - Determining k - Excluding Managers
%% skill clusters - no managers from the start - cos k=3.png
\begin{figure*}[!h]
    \centering
    \includegraphics[width=\textwidth]{Nature_HB_2023/SI_figNHB/SI__Determining_k__Excluding_Managers__70_bins.png}
    \caption{\textbf{Determining $k$ for $k$-mean Clustering at the Absence of Managerial Occupations.} Reproduction of Fig.~\ref{fig:determining_k_level} without management and administrative occupations. We use the Elbow method, Gap statistic, and Silhouette analysis to test the optimal $k$.}
\label{fig:determining_k_kmeans_70bins_correlation_no_manager_start}
\end{figure*}
% \begin{figure*}[!h]
    %\centering
    %\includegraphics[width=\textwidth]{Nature_HB_2023/SI_figNHB/Determining k - Kmeans with correlation - Level - 70 bins.png}
    %\caption{Statistical tests to determine the optimal $k$ for $k$-mean clustering algorithms: Elbow method, Gap statistic, and Silhouette analysis}
    %\label{fig:determining_k_level}
%\end{figure*}
%% skill clusters - no managers from the start - cos k=3.png
\begin{figure*}[!h]
    \centering
    \includegraphics[width=\textwidth]{Nature_HB_2023/SI_figNHB/SI__Skill_Profiles__Excluding_Managers__70_bins.png}
    \caption{\textbf{Grouping Skills at the Absence of Managerial Occupations.} Reproduction of Fig.~\ref{fig:Figure 1} without management and administration occupations.}
\label{fig:profiles_k_kmeans_70bins_correlation_no_manager_start}
\end{figure*}

%% skill clusters - no managers from the start - cos k=3.png
\begin{figure*}[!h]
    \centering
    \includegraphics[width=.8\textwidth]{Nature_HB_2023/SI_figNHB/SI__Skill_Clusters__Excluding_Managers__70_bins.png}
    \caption{\textbf{Detailed Assignment of Skills among Clusters.} Reproduction of Fig.~\ref{fig:skill_level_dist_cor_k=3} without management and administration occupations.}
\label{fig:skill_clusters_k_kmeans_70bins_correlation_no_manager_start}
\end{figure*}


%\clearpage
%\subsubsection{Wage Premiums and Educational Requirements}
%% Wage ~ Skills - no managers - new categorization.png
\begin{figure*}[!h]
    \centering
    \includegraphics[width=.8\textwidth]{Nature_HB_2023/SI_figNHB/SI__No_managers_-_Education__Wage_Figure__Mar_7_2024.png}
    \caption{\textbf{Reproduction of Fig.~\ref{fig:Wage} in the main text} without management and administration occupations. Managerial occupations do not drive the wage premiums and the educational requirement.}
\label{fig:returns_to_skills_hierachy_gen_dependence_cor_no_manager}
\end{figure*}

\clearpage
%\subsubsection{Skill Acquisition over Time}
%% SI - Figure 3 equivalent - No Managers - Jul 19 2023.pdf
\begin{figure*}[!h]
    \centering
    \includegraphics[width=.9\textwidth]{Nature_HB_2023/SI_figNHB/SI__Figure_3_equivalent__No_Managers__Jul_19_2023.png}
    \caption{\textbf{Reproduction of Fig. 3 without Management and Administration occupations}. The findings are consistent, suggesting managerial occupations are not the primary drivers of increases in general or nested skills over time.
    \textbf{(a-c)} replicate the analysis of main Fig.~\ref{fig:age}~(a-c) on occupations' median age in the absence fo managerial occupations.
    \textbf{(d-f)} follows analysis Fig.~\ref{fig:age}~(d-f) of synthetic birth cohorts identified in CPS microdata except for excluding observations on individuals who held managerial occupations.
    \textbf{(g-i} follows the analysis of Fig.~\ref{fig:age}~(g-i) on resume data except for excluding observations on individuals who held managerial occupations.}
\label{fig:skill_and_age_no_manager}
\end{figure*}




%%--------------------------------------------------------------------%%
\clearpage
\subsection{Role of Social Skills} \label{sec:social skills}
Given the growing importance of social skills \cite{Liu2013}, we examine the extent to which they shape the role we observed for general skills in our work.
There are six \textit{social skills} in O*NET taxonomy. 
They are \textit{Social perceptiveness}, the skill of being aware of other's reactions and understanding why they react as they do; \textit{Coordination}, the skill to adjust actions in relation to others' actions; \textit{Persuasion}, the skill to persuade others to change their minds or behavior; \textit{Negotiation}, bringing others together and trying to reconcile differences; \textit{Instructing}, the skill to teach others how to do something; 
and \textit{Service orientation}, actively looking for ways to help people.

In Fig.~\ref{fig:social skills}-(a), we annotated these social skills in our skill hierarchy of the main text as well as the hierarchy of 2005 data (insets). 
We find these skills fall within our categories of general (Coordination and Social perceptiveness) and nested skill categories (Instructing, Service orientation, Persuasion, and Negotiation); and these skills are more demanded in 2019 than in 2005 as shown in Fig.~\ref{fig:social skills}-(b). 


The average levels at which occupations in 2019 use negotiation, persuasion, social perceptiveness, and service orientation skills surpassed their levels in 2005.
However, social skills' position in the skill dependency network has moderately moved away from the most broadly used skills.
This means their comparative role has specialized moderately despite their absolute demand rise.
 Fig.~\ref{fig:social skills} (c) shows the changes in the level of each skill category resulting from omitting social skills. As these skills belong to general and nested intermediate categories, scores across other categories do not change. However, changes in the affected subgroups are also minimal, leading to a 0.998 correlation before and after social skills omission.
This offers evidence that social skills do not influence the significance of general skills.
We repeated analyses of wage premiums (Fig.~\ref{fig:Wage} in the main text) on categories excluding the aforementioned six social skills, ad our findings are intact as shown in Fig.  \ref{fig:social skills} (d-e).



\begin{figure*}[!h]
    \centering
    \includegraphics[width=\textwidth]{Nature_HB_2023/SI_figNHB/Social_Skills_Analysis_-_Jul_18_2023.png}
    \caption{\textbf{Analysis of the Social Skills and their Relationship with our Work.}
    \textbf{(a)} Annotates the skill hierarchy of main text Fig.~\ref{fig:Figure 2} and that of the year 2005 with social skills included in O*NET.
    \textbf{(b)} Skill profiles of social skills in 2005 (average shown as dashed line) and in 2019 (average shown as a solid line).
    \textbf{(c)} The differences in skill levels when social skills are excluded.
    \textbf{(d-e)} The equivalent plots of Fig. 4 without social skills in analyses. 
    }
    \label{fig:social skills}
\end{figure*}\










\clearpage
% \bibliographystyle{naturemag}
% \bibliography{scibib.bib}
\printbibliography

\end{document}