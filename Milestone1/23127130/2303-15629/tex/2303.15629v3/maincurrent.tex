\documentclass[a4paper, 12pt]{article} %{{{

% \usepackage{lineno}
% \linenumbers
\usepackage{dcolumn}
\usepackage[margin=1in]{geometry}
\usepackage[utf8]{inputenc}
\usepackage{graphicx}
\usepackage{subfigure}
\usepackage{authblk}
% \usepackage[flushleft]{threeparttable}

%----Helper code for dealing with external references----
% (by cyberSingularity at http://tex.stackexchange.com/a/69832/226)

\usepackage[style=nature,backend=biber,sortcites=true]{biblatex}
\addbibresource{maincurrent.bib}
% Ignore irrelevant biblatex fields
\AtEveryBibitem{%
 \clearfield{url}%
 \clearfield{month}%
 \clearfield{issn}%
 \clearfield{doi}%%
 \clearfield{address}%%
}

\usepackage{xcite}
\usepackage{xr}
\makeatletter

\newcommand*{\addFileDependency}[1]{% argument=file name and extension
\typeout{(#1)}% latexmk will find this if $recorder=0
% however, in that case, it will ignore #1 if it is a .aux or 
% .pdf file etc and it exists! If it doesn't exist, it will appear 
% in the list of dependents regardless)
%
% Write the following if you want it to appear in \listfiles 
% --- although not really necessary and latexmk doesn't use this
%
\@addtofilelist{#1}
%
% latexmk will find this message if #1 doesn't exist (yet)
\IfFileExists{#1}{}{\typeout{No file #1.}}
}\makeatother

\newcommand*{\myexternaldocument}[1]{%
\externaldocument{#1}%
\addFileDependency{#1.tex}%
\addFileDependency{#1.aux}%
}
%------------End of helper code--------------

% put all the external documents here!
\myexternaldocument{supplement}

% Math
\usepackage{amsmath}
\usepackage{amssymb}
\usepackage{newtxmath}
\DeclareMathAlphabet{\mathpzc}{T1}{pzc}{m}{it}
\DeclareMathOperator*{\argmin}{\arg\!\min}
\DeclareMathOperator*{\argmax}{\arg\!\max}
\usepackage{bm}
\def\tnull{{\text{null}}}
\def\vec#1{{\bm #1}}
\def\mat#1{\mathbf{#1}}

% Figure and table captions
\usepackage[labelfont=bf]{caption}
\captionsetup{font=footnotesize}
\usepackage{floatrow}
\floatsetup[table]{capposition=top}



\newcommand{\secref}[1]{Section~\ref{sec:#1}}
\newcommand{\figref}[1]{Fig.~\ref{fig:#1}}
\newcommand{\tabref}[1]{Table~\ref{tab:#1}}
\newcommand{\todo}[1]{{\leavevmode\color{orange}[TODO: #1]}}

% abbreviations
\def\etal{\emph{et~al}.\ }
\def\eg{e.g.,~} 
\def\ie{i.e.,~}
\def\cf{cf.\ }
\def\viz{viz.\ }
\def\vs{vs.\ }


% Comments
\usepackage[dvipsnames]{xcolor}
\definecolor{dkgreen}{rgb}{0,0.6,0}
\definecolor{gray}{rgb}{0.5,0.5,0.5}
\newcommand{\editHY}[1]{{\textcolor{dkgreen}{#1}}}
\newcommand{\noteHY}[1]{\textbf{\textcolor{dkgreen}{{\scriptsize{HY:}}#1}}}

\usepackage{xurl}
\usepackage[utf8]{inputenc}
\usepackage{changepage}

% % Figure caption
% \usepackage{setspace}
% \usepackage[font=small,labelfont=bf]{caption}
% \captionsetup[subfigure]{font={bf,small}, skip=1pt, singlelinecheck=false}

\newenvironment{sciabstract}{%
\begin{quote} \bf}
{\end{quote}}


\renewcommand\Authfont{\fontsize{15}{14.4}\selectfont}
\renewcommand\Affilfont{\fontsize{10}{9}\itshape}

\newcommand{\figdir}{figs}
\newcommand{\tabdir}{tabs}


\title{
   Nested Skills in Labor Ecosystems: A Hidden Dimension of Human Capital
}
% The Anatomy of Human Capital: understanding the dependency structure of skills
% The skill anatomy of human capital: understanding the dependency structure of skills
% Skill hierarchies - from human capital specificity to nestedness
% Skill hierarchies: seeing human capital through the lens of directed networks
% Skill hierarchies: understanding the dependency structure in human capability
% Bearing on Common Roots:  On the Shoulders of the Common: ...
%Unveiling the Nested Structure: A New Perspective on Skill Dependencies in Labor Ecosystems.
%"Nested Structures in Skill Landscapes: Unveiling the Hidden Complexity of Labor Ecosystems."
%Hidden Dimension to Labor Ecosystems: understanding the nested dependency in skill structures
%"Nested Hierarchies in Skills: Revealing Hidden Aspects of Labor Ecosystems."
%"Nested Dependencies in Labor Ecosystems: Unraveling the Hidden Layers of Skill Structures."
%"Human Capital Unpacked: Exploring the Nested Dependencies in Labor Skill Structures."
%"The Hidden Architecture of Labor: Understanding Nested Dependencies in Skill Structures."
%"Skills and Dependencies: Revealing Hidden Dimensions of Labor Ecosystems."
%"Beyond the Visible: Understanding Nested Skill Structures in the Labor Market."
% ``Hidden Dimension to Labor Ecosystems: understanding the nested dependency in skill structures''
%"Unraveling the Nested Structure of Skills: A New Perspective on Human Capital"
% Mind skill-gap
% "The Hidden Layers of Human Capital: Unpacking the Nested Structure of Skills"
% The Anatomy of Human Capital: understanding the dependency structure of skills
% The skill anatomy of human capital: understanding the dependency structure of skills
% Skill hierarchies - from human capital specificity to nestedness

% suggested reviewers
% Balazs Lengyel, Christopher Esposito, Lingfei Wu, James McNerney; Muhammed Yildirim; 
 

\author[1,2,3]{Moh Hosseinioun}
\author[4]{Frank Neffke}
\author[5]{Letian (LT) Zhang}
\author[1,2,6]{Hyejin Youn \thanks{Correspondence can be sent to hyejin.youn@kellogg.northwestern.edu.}}


\affil[1]{Kellogg School of Management, Northwestern University, Evanston, IL, USA}
\affil[2]{Northwestern Institute on Complex Systems, Evanston, IL, USA}
\affil[3]{Department of Information and Decision Sciences, University of Illinois, Chicago, IL, USA}
\affil[4]{Complexity Science Hub Vienna, Vienna, Austria}
\affil[5]{Harvard Business School, Harvard University, Cambridge, MA, USA}
\affil[6]{Santa Fe Institute, Santa Fe, NM, USA}

\date{\today}



\begin{document}

\maketitle
\thispagestyle{empty}


\vspace{-0.5cm}
%%%%%%%%%%%%%%%%%%%%%%%%%%%%%%%%%%%%%%%%%%%%%%%%%%%%%%%%%%%%%%%%%%
%%%% Below is the 150-word abstract (250 is left in the main) %%%%
% Modern economies operate through globally interconnected networks. As economies become more complex, so do these networks, coordinating increasingly diverse portfolios of specialized efforts and knowledge. Here, we infer an interdependency tree underlying the fabric of skill portfolios. Hierarchically constructed, this skill tree starts from widely needed, foundational abilities, constituting the root, and extends to highly specialized, niche skills required by select jobs at the extremities. The directionality is defined by the asymmetrical conditional probabilities of the presence of one skill given the existence of another. Examining 70 million job transitions, we observe individuals tend to delve deeper into these nested specialization paths as they ascend the career ladder to enjoy higher wage premiums.  Nevertheless, the role of foundational skills for such ascent remains pivotal; without reinforcing them, the anticipated wage premiums may vanish, suggesting the critical role of foundational skills for specialization, and the need for balanced skill development strategies in complex economies. We differentiate 'nested' skills, building on common prerequisites, from others, to examine their disparities across regions and demographic groups as to wage premiums. Our temporal analyses reveal a growing fragmentation between these skill groups over the past decades, suggesting further job polarization.  
%%%% Above is the 150-word abstract (250 is left in the main) %%%
%%%%%%%%%%%%%%%%%%%%%%%%%%%%%%%%%%%%%%%%%%%%%%%%%%%%%%%%%%%%%%%%%%

%%%%%%%%%%%%%%%%%%%%%%%%%%%%%%%%%%%%%%%%%%%%%%%%%%%%%%%%%%%%%%%%%%
%%%%      Below is the 250-word abstract                      %%%%
%%%%%%%%%%%%%%%%%%%%%%%%%%%%%%%%%%%%%%%%%%%%%%%%%%%%%%%%%%%%%%%%%%
\begin{abstract} \label{sec: abstract}
%Modern economies generate immensely diverse complex goods and services through coordinating our efforts and know-how in densely interwoven networks that span across the globe. As these economies grow increasingly complex, participation in these networks grows more intricate and requires an increasingly specialized skill portfolio. With the ever-increasing complexity of these economies, these networks grow more intricate and demand an ever-expanding portfolio of specialized skills. 
%Modern economies generate a vast array of goods and services through complex, globally interdependent networks of coordinated efforts and specialized knowledge. As these economies become more complex, so do the networks, necessitating a more diverse set of specialized skills for participants to acquire.
%Modern economies, characterized by their vast output of goods and services, operate through globally interconnected networks that coordinate efforts and specialized knowledge. As economies become more complex, so do these networks, necessitating an increasingly diverse portfolio of specialized skills for network participants.
Modern economies, characterized by their vast output of goods and services, operate through globally interconnected networks. 
As economies become more complex, so do these networks, coordinating increasingly diverse portfolios of specialized efforts and knowledge. 
In this study, we analyze U.S. survey data (2005--2019) to infer an underlying interdependency tree within the fabric of skill portfolios. Hierarchically constructed, this skill tree starts from widely needed, foundational abilities, constituting the root, and extends to highly specialized, niche skills required by select jobs at the extremities. 
The directionality is defined by the asymmetrical conditional probabilities of the presence of one skill given the existence of another. 
Examining 70 million job transitions in resumes and national surveys, we observe that individuals tend to delve deeper into these nested specialization paths as they ascend the career ladder to enjoy higher wage premiums.  
Nevertheless, we find the role of foundational skills for such ascent remains pivotal; without reinforcing them, the anticipated wage premiums may vanish. 
Hence, we further differentiate \textit{nested} skills from others, with the former building on common prerequisites while the latter does not, 
and analyze disparities in these skill gaps across different genders and racial/ethnic groups. Our analysis reveals a growing and concerning fragmentation in the divide between these two skill groups over the past two decades, suggesting further polarization within the job landscape \cite{Autor2013}.
Our findings highlight the critical role of robust foundational skills as a stepping stone to specialization and the economic advantages it can confer, reinforcing the need for balanced skill development strategies in complex economies \cite{Althobaiti2022}.
%Our findings underscore the importance of a strong foundation of general skills, which are essential for specialization and the subsequent rewards it can offer.
%ones hovering on their own and lacking the foundational roots for wage premiums, and we find this divide has not only grown but also has alarmingly fragmented over two decades, potentially pointing to job market polarization. 
%%----------------------------------------------------------------------------%%
\end{abstract}
\newpage
\pagenumbering{arabic}


\section*{Introduction}
 
%  Modern economies, marked by complexity and specialization, have been on an exponential growth trajectory ever since the dawn of the first stone tool. Instead, it requires globally coordinated networks of specialized individual expertise and effort. 
% Frank_neffke's: I added references to Henrich, Boyd/Richerson, and Turchi, all about the rise of complex societies, but in the literature on cultural evolution. The division of labor argument is made by Boyd/Richersen and Henrich. For Turchin, dol is not a big deal, although he mentions it. I think we can drop him. Still, he does argue that societies are becoming more and more complex. As to reference, There is probably a better citation in the complexity literature here. This paper is a model of trade that follows an O-ring structure, meaning that the more complex the product becomes, the higher the quality of the participants in the value chain has to become to avoid failure.
% This daunting complexity isn't a recent phenomenon. 
%Everyday items like toasters or smartphones perfectly illustrate this paradigm shift. 
% This paradigm shift is exemplified by everyday items such as smartphones, toasters, and even pencils. The assembly of an iPhone, for instance, engages over 200 suppliers, each with its own workforce \cite{Barboza2016}. 
% Likewise, the Toaster Project vividly demonstrates the limitations faced by an individual when tasked with building a simple toaster from scratch in today's economy \cite{Thwaites}. 
% Indeed, as far back as 1958, the humble pencil served as a powerful testament to the intricate interdependence of global manufacturers in the creation of even the simplest items \cite{Read1958, Dubner2016}. 
% From the time humans shaped the first stone tool, the growth in technological complexity has snowballed to the point where creating and maintaining societal artifacts no longer rests within individual capabilities.
%in expansive, complex global production networks \cite{BenJ2009, Wuchty2007, BeckerG.S1992TDoL, McNerney2022, Azoulay2020, Matouschek2022,Pichler2023, VanderWouden2023}.
%As society evolves, producing and maintaining increasingly sophisticated goods, services, and infrastructure, it is no longer feasible for one to become a universal mastery across all domains to cater to these complex needs as a whole 
% interlocking supply chains. Major manufacturing firms, such as Airbus or GE, depend on production ecosystems.

Complexity and specialization are foundational to the narrative of economic growth and innovation \cite{Carneiro1986, henrich2015secret, richerson1999complex, MitchellMelanie2009}. As society advances, creating and maintaining sophisticated goods, services, and infrastructure, these socio-economic complexities have surpassed what individuals can embody and manage on their own \cite{BenJ2009, Gamble2002}. It is no longer feasible for individuals to master universal expertise across all areas. For economies, this means developing deep divisions of labor and knowledge that first distribute knowledge across people and then coordinate this distributed knowledge in teams, firms, and value chains \cite{BeckerG.S1992TDoL, Hidalgo2015, Azoulay2020, Pichler2023}.  For individuals, this means specializing, and deciding which skills to acquire over long educational and work trajectories has become increasingly important \cite{Acemoglu2020}. As such, human capital is far from an isolated entity but an interdependent ecosystem of skills and knowledge in economies.

This leads to research questions: What does the structure of these interdependencies look like? And, more importantly, what implications does this nested structure carry? Division of labor, division of knowledge, and the existence of such an interdependency web are not in doubt as they manifest in education and career paths in a way we experience every day, shaping social and economic systems \cite{Autor2013, SchwabeHenrik2020Awsa}. However, though the framework may seem intuitive, it is essential to note that the hierarchical layout of skills reflected in job roles has often been assumed rather than empirically evidenced.

Emerging research aimed at understanding the network architecture of human capital has yielded insights into the detailed tasks that individuals perform at work and the skills they require to do so \cite{AndersonKatharineA2017Snam, Borner2018, Alabdulkareem2018, Xu2021, Lin2022, Neffkeeaax3370, DelRio-Chanona2021, Frank2019, Moro2021}. Nevertheless, a granular understanding of workers' skill trajectories and their resulting impacts on individuals remains an ongoing area of exploration. Furthermore, these frameworks aim to capture complementarities or synergies between capabilities, knowledge, and skills \cite{AndersonKatharineA2017Snam, Alabdulkareem2018, Neffkeeaax3370, gomez2016explaining, hidalgo2018principle}. That is, jobs combine skills that complement one another. We contribute to these ongoing efforts by constructing a directed skill network that expresses how skills build or depend on one another, conceptualizing trajectories with conditional probability. 


In this paper, we propose that the skill composition of jobs not only reflects complementarities but also the innate cognitive constraints of how individuals learn. That is, jobs not only combine synergistic skills but also skills that build on one another. This aligns with an understanding of skill acquisition as a cumulative, sequential trajectory that builds pyramidal skill structures where higher-level skills are nested in most basic layers of expertise \cite{WilkSteffanieL1995GtJC}. Students are taught calculus only after they have mastered the basics of algebra and geometry. We infer such dependencies by analyzing how skills co-occur in jobs and the construction of asymmetric skill networks in which the directed arrows describe skill dependencies.


These dependencies turn out to integrate one of the core concepts of traditional human capital theory into the network-based complexity approach: human capital specificity. Since its inception, the distinction between general and specific skills has been a hallmark of human capital theory, explaining why market economies typically underinvest in general skills \cite{becker2009human}, why acquiring specific skills creates hold-up problems \cite{williamson2007economic}, and why workers often face earning losses when they are displaced from their jobs \cite{jacobson1993earnings}. However, this distinction also matters because general skills constitute a foundational layer in an individual's human capital, on top of which more specific skills can be developed. Just like the way mastering calculus requires a prior understanding of algebra and geometry, these education and career paths are both sequential and cumulative, building on each other, and thus create a high-dimensional space of possibilities for job opportunities \cite{WilkSteffanieL1995GtJC, Jovanovic1997}.

The sequential nature of skill trajectories has important implications for professional development and, therewith, socio-economic outcomes because they mean that certain career paths are only feasible after prior investment in foundational skills \cite{heckman2011economics, Autor2013, SchwabeHenrik2020Awsa, NelsonDylan2022, Goldin2008, Azoulay2021}. 
As a consequence, specialization entails not just an increase in the volume of learning and investments in education and training \cite{BenJ2009} but also the existence of structured, sequential, and nested cumulative paths that can either enable or restrict specific career trajectories. These structured pathways systematically shape professional development and thus the socio-economic landscape at large, leading not only to differential rewards but also differential accessibility and feasibility of career options based on earlier choices \cite{Autor2013, SchwabeHenrik2020Awsa, NelsonDylan2022, Goldin2008, Azoulay2021}. Thus, to succeed in this complex environment, individuals must acquire the right set of skills, knowledge, and abilities \cite{Mincer1974, Becker1962, Lucas1988, Neffke2013, Neffkeeaax3370, Stephany2024, Jovanovic1997}. Yet, the most sought-after skills in today's economic and social sphere are often not readily accessible but are instead nested within specific domains, requiring a progressive accumulation of knowledge and expertise to unlock.


In this paper, we show that this hierarchical network yields a description of human capital that not only recovers broad, well-established job categories but also helps predict career transitions and wage curves. To do that, we analyze skill portfolios and their underlying structures using publicly accessible national surveys complemented by a proprietary dataset. We differentiate specialized skills, those required by select occupations, from general skills, those widely required across occupations (Fig.~\ref{fig:Figure 1}). We then construct a nested hierarchical structure of skill dependencies, employing conditional probabilities of the presence of one skill given the existence of another in occupations \cite{Jo2020}. Our method reveals that not every skill is embedded in a nested structure, resulting in a partially nested hierarchical structure among skills (Fig.~\ref{fig:Figure 2}). Therefore, we quantify each skill's contribution to the overall nested architecture of the network and find that skills contributing significantly to the nested architecture are rewarded most (Fig.~\ref{fig:nestedness}), echoing the nesting nature of economic complexity \cite{Hidalgo2009, Saavedra2011}.
 
By examining three different datasets (median occupational ages, synthetic birth cohorts of individuals, and 70 million job transitions in resumes), we uncover that nested branches are evidence of specialization and career advancement. That is, as individuals progress up the career ladder, they need to acquire and apply skills on nested specialization branches (Fig.~\ref{fig:age}).  
Moreover, we find most of the wage premiums for these nested specializations are conditional on foundational, general prerequisite skills they are nested in, unlike unnested specializations without prerequisite skills (Fig.~\ref{fig:wage curves}). 
This pattern suggests deeply rooted structural disparities in race/ethnicity and gender (Fig.~\ref{fig:Figure 7}). 
Finally, we examine structural changes in the skill network over time and find a wider gap between nested and unnested branches, suggesting potential barriers to upward mobility (Fig.~\ref{fig:historical skill change}). 

Structural properties of skill nestedness in human capital can provide actionable insights. The methodologies we employ introduce a scalable metric for skill categorization, enabling our analysis to extend to more granular levels. The nestedness metric effectively captures shifts in dependency intensity, providing a nuanced view of labor market polarization. As data on workplace skills, knowledge, capabilities, and tasks become increasingly granular, our approach extends to analyzing skills at finer resolutions, evaluating their diverse contributions to nestedness. This capability to identify changes in skill requirements across occupations complements the traditional context-informed categories, which may not adjust as readily to these changes; the flexibility and adaptability of our framework are useful for understanding the evolving landscape of skills and its impact on career development and socio-economic disparities. As the labor market continues to evolve, with new skills emerging and older ones becoming obsolete, our model acts as a comprehensive and dynamic tool for tracking these shifts and their wider implications.
%\cite{AndersonKatharineA2017Snam,Neffkeeaax3370}, the impact of automation on the world of work \cite{Alabdulkareem2018} and the role of complementarity in teams of coworkers \cite{Neffkeeaax3370}.

%% directionality of investment in skills and path-dependence shows up in Arrow's Limit of Organization
% The existence of such an interdependency web is not in doubt as it manifests in education and career paths and thus shapes the social and economic systems \cite{Autor2013, SchwabeHenrik2020Awsa}.
% Unlike the latter, the former is perhaps not the kind of division of labor that Adam Smith envisioned \cite{smith1937wealth}.
%Sometimes, we have to acquire long training to get to be able to do (nested), and sometimes we are naturally capable of it. What does the structure of those look like, and what's the implication of the structure? 
%Alternative: Since the division of labor by adam smith and Alfred Marshall, tasks are divided such as to increase productivity. But what Adam Smith and Marshal did not know was the complexity also increases not only the division of labor but also the division of knowledge to share the needed functions with many different people. Toyota example?
% Unlike the first concept of division of labor, where tasks are divided in a way to be performed by anyone substitutable, even children (ref: children labor), there have been many skills and knowledge needed for jobs that need a long training. This begs the question: which skills are "nested" and which skills are not nested, and how hierarchically are they structured? 
% The directionality of skill interdependence is not only in the educational curricula but also in carrier paths.
%There is little doubt that such a structure is also influenced by external forces

% The recent studies uncovered network structure explaining fragmentation and job mobility \cite{Alabdulkareem2018, Xu2021, Lin2022, Neffkeeaax3370, DelRio-Chanona2021}. 
%Therefore, a comprehensive skill structure disentangles mechanisms of specialty, in which knowledge and technological advances increase the investment necessary to contain the depth of knowledge in an individual \cite{BenJ2009}, and diversity, in which the increasingly complex economy \cite{Hidalgo2009, Hidalgo2015, MitchellMelanie2009} requires more diverse capabilities for a unit of economic activity.


%..... may include this later part... let me think .....
%  Recent work on skill-interdependencies has modeled the skill structure as a network of co-occurrences \cite{AndersonKatharineA2017Snam, Alabdulkareem2018, Xu2021, Lin2022, Neffkeeaax3370, DelRio-Chanona2021}.
% On one hand, the hierarchical structure is well known in explaining wage premiums and so on \cite{Mincer1974, Becker1962} 
%However, because the role of specialization and diversification in shaping skill co-occurrences is rarely teased out \cite{Neffkeeaax3370}, such networks miss the directionality of skill accumulation and how it is influenced by advances in knowledge and technology.
% For instance, one cannot explain why Programming is in one cluster and not the other beyond its pattern of bundling in occupations.
% Nor can we deduct the values of skills from the relationships or the structure of the network.
% Therefore, these networks cannot capture changes in the knowledge structure or inform policy, such as education, economic development, or worker reskilling.
% Focusing on a dichotomy of skill clusters, while these skill networks have predicted wage and education, they fail to incorporate specialization and diversification, which are pivotal attributes of skill evolution.
% Therefore, we expect the skill structure not only to branch off horizontally but also to grow vertically in depth [like a growing tree, where some skills act as the root and trunk, and further, smaller branches form].


\section*{Results}

\begin{figure*}[!h]
    \centering
    \includegraphics[width=\textwidth]{Nature_HB_2023/figNHB/Figure_1_-_Apr_15_2024.png}
    \caption{\textbf{Skill Level Distributions and Dependencies.}  
    \textbf{(a-c)} Average number of occupations requiring each skill level for the three groups (see SI Section \ref{supsec:skill clustering} for details.) 
    Skills are grouped based on their characteristic skill level distribution shapes, exemplified by the insets, and labeled as \textit{General} (31 skills), \textit{Intermediate} (43 skills), and \textit{Specific} (46 skills). The shapes indicate that Specific skills (blue) are needed only in a few jobs, while most jobs require high proficiency in General skills (red).  
    \textbf{(d)} Schematic illustrating our inference method for dependency between skill pairs using the asymmetric conditional probability of one skill being required given another. 
    For example, if requiring Math skills is more probable given the presence of Programming (compared to the reverse), we infer a directional dependency: Math $\rightarrow$ Programming, weighted by the level of asymmetry (see Methods). Similarly, Oral Expression $\rightarrow$ Negotiation, but Math $\not\rightarrow$ Dynamic Flexibility, as their presences are independent events, that is, $P(\text{Math}|\text{Dyn. Flex.}) = P(\text{Dyn. Flex.}|\text{Math})$.
    }
    \label{fig:Figure 1}
\end{figure*}



\subsection*{Skill Generality (Individual Occurrences)} \label{sec: skill hierarchy}

The distinction between general and specialized skills is widely acknowledged, but a systematic quantification of this divide has been lacking \cite{Becker1962, Poletaev2008, Gathmann2010, FergusonJohn-Paul2013, Leung2014, Merluzzi2016, Byun2018, Fini2022, Byun2023, RotundoMaria2004Svgs}. 
Therefore, our study starts with examining, quantifying and classifying the generality of skills based on their breadth of application across occupations, using publicly available survey data from the U.S. Bureau of Labor Statistics (BLS).
These surveys provide detailed observations on the job requirements for thousands of occupational titles, including the importance and required level of each skill, knowledge, or ability necessary for workers to perform their occupational tasks.


Figure~\ref{fig:Figure 1} illustrates the existence of skills with varying degrees of occupational demand, characterized by their level distribution shapes across occupations with broad versus narrow applications. 
Here, demand denotes the number of occupations requiring the skill at a given level, ranging from 0 to 7. Specialized skills, such as Fine Arts and Programming (blue), are required only by select occupations, often at high levels (6 or 7), but not across a broad range of occupations.  
This leads to a distribution shape that primarily peaks at the 0-1 levels with a long tail.
In contrast, skills considered general (red), such as Oral Expression and Critical Thinking, are widely needed at elevated levels, with distributions that peak at levels 3-4, indicating their general applicability across most jobs.

To systematically classify skills, we group them based on similar level distribution shapes, which we interpret as indicators of broad versus narrow utility of skills (See Methods).
Figure~\ref{fig:Figure 1}~(a-c) show distribution shapes for the resulting skill groups, calculated by averaging the number of occupations that require the given skill levels within each group, which sketches the distinct level profile curve of that skill group.
The inset examples demonstrate that some skills are specialized, meaning they are not widely required across occupations but are critically needed at high levels in specific job contexts. These skills are identified and grouped into the \textit{specific} skill set. In contrast, skills relevant to a wide spectrum of roles are labeled as the \textit{general} skill set.

These classifications, detailed in SI-Table \ref{tab:skill_groups}, align with our common understanding of general and specialized skill categories. 
Nevertheless, we ensure the robustness of our findings by testing our results against different group sizes and clustering algorithms (see SI Sec.~\ref{supsec:skill clustering}). 
In addition to the distribution-based approach, skill generality can also be measured by the median skill levels required across occupations. For example, the median level for general skills is 3.34, for intermediate skills, it is 2.37, and for specific skills, it is 0.87, reflecting the skewed shape of niche skills.
In the following, we additionally show that these generality measures are consistent with network-based measures of generality \cite{Mones2012}.
Throughout the paper, our results are color-coded for consistency: general (red), intermediary (gray), and specific skills (blue). 



\begin{figure*}[!h]
    \centering
    \includegraphics[width=\textwidth]{Nature_HB_2023/figNHB/Figure_2_-_Mar_12_2024.png}
    \caption{\textbf{Skill Dependency Hierarchy.} 
    \textbf{(a-b)} Dependency hierarchy is constructed from the aggregated weighted directions of all skill pairs. Node sizes are proportional to education levels and colored by the groups in Fig.~\ref{fig:Figure 1}.  
    A node's horizontal and vertical positions are, respectively, its educational attainment and local reaching centrality. 
    Defined as the proportion of the skills reachable from each node or the number of interdependent skills, the centrality is a reasonable indicator for skill generality \cite{Mones2012}. b shows the backbone of the network for better local visualization, while c shows the full network with normalized weights.
    % The vertical line at zero separates, hereafter, nested and unnested skills.     % \textbf{(c)} Nestedness scores of specific (blue) and intermediate skills (gray) and their associated education, and automation risk indexes (size) \cite{. 
    \textbf{(c-d)} Reachability (arrival probability) from each skill to Programming, Negotiation, and Repairing (highlighted) \cite{norris1998markov}. Dark hues indicate a higher likelihood of arriving at the focal skill (see Methods). Contrary to the well-nested Programming and Negotiation, Repairing does not predominantly rely on general skills, indicating its unnested nature. 
    }
    \label{fig:Figure 2}
\end{figure*}

\subsection*{Skill Hierarchy (interdependency)} 


The disparate skill level profiles captured by our empirical generality skill groups suggest a hierarchical structure among skills, with some serving as prerequisites for others. This hierarchy has been a longstanding topic of interest in fields such as labor economics, sociology, and management, but it has not been systematically analyzed   \cite{Becker1962, Neal1995, Parent2000, Poletaev2008, Gathmann2010, FergusonJohn-Paul2013, Leung2014, Merluzzi2016, Byun2018, Fini2022, Byun2023, Leahey2007, Teodoridis2018, Heiberger2021}. 
As such, we propose a method to quantify these relations by calculating how often occupations that require niche skills also require general skills and compare this to the inverse---how often needing general skills predicts the need for certain niche skills. If general skills are indeed prerequisites for niche skills, much like how most college curricula have fundamental courses preceding specialized ones, we should expect to find an asymmetry in these probabilities.

We operationalize the pairwise dependencies between skills using the information asymmetry in occupational skill requirements, following \cite{Jo2020}. The approach involves calculating the conditional probability of requiring one skill ($skill_A$) given the presence of another skill ($skill_B$), denoted as $p(skill_A | skill_B)$, and comparing it to the reverse probability, $p(skill_B | skill_A)$. This comparison allows us to assign directionality to the skill dependencies. If skill $A$ is contingent on skill $B$, meaning that the application or acquisition of skill $A$ is dependent on that of skill $B$, then $p(skill_A | skill_B)$ will be greater than $p(skill_B | skill_A)$.

In cases where skill $A$ and skill $B$ are independent events across occupations, the direction disappears as the conditional probabilities will be equal. This is because when two events are independent, $p(skill_A | skill_B)$ is expressed as $p(skill_A)p(skill_B)$, which is then the same as $p(skill_B | skill_A)$. Similarly, if two skills are rarely applied together within occupations, both base probabilities will be close to zero, $p(skill_A, skill_B) \simeq 0$, indicating no statistical dependency between them.
In both cases, co-occurrences are purely a result of their random independent events of either occupational need or individual workers' properties and, thus, not influenced by any underlying relationship. Therefore, an asymmetry in the conditional probabilities reveals how skill $A$ relies on skill $B$ for its application or acquisition, indicating the importance of the order in which skills are acquired or applied. 

It is important to acknowledge that this directionality does not provide a detailed understanding of the underlying process. The directionality could arise from the acquisition sequence, such as the learning process, or the requirement sequence through job seniority in organizations. What's happening at the individual worker level is inferred rather than directly measured in the current study because our empirical evidence is based on occupational attributes. 
Disentangling these factors would require more micro-level analyses, yet it is a promising avenue for future research. In this study, we focus on providing a phenomenological understanding of the structure of skill dependencies and their consequences for individuals.

Figure~\ref{fig:Figure 1}~(d) illustrates our inference method using select examples. Given the skill level distributions, the conditional probability of math skills given programming skills, $p(skill_{math} | skill_{prog})$, is higher than $p(skill_{prog} | skill_{math})$, resulting in the directional dependency $math \rightarrow programming$.
This direction is consistent with our common understanding and educational curriculum; to understand the complexity of a program, we need to have a minimum knowledge of math. The same holds true for Negotiation skills being conditional on Oral Expression. Moreover, developing and applying Math skills depends on advancements in Deductive and Inductive Reasoning, which are in the general group (red) of Fig~\ref{fig:Figure 1}~(c).
These create dependency branches, suggesting we will expect more than one depth to the hierarchical network.

These cross-group dependencies resemble biological mutualistic interactions where specialist species (i.e., niche skills) preferentially interact with generalists (i.e., general skills), suggesting a nested hierarchical skill integration \cite{Bascompte2003, Saavedra2011, Saavedra2009, Staniczenko2023}. However, the result is not always obvious; not every skill exhibits such dependency chains. Some specialized skills, like Dynamic Flexibility, may not be contingent on more general skills like mathematical prowess, which is again consistent with our common understanding. This can be calculated as $p(skill_{dyn.flx} | skill_{math})$ and $p(skill_{math} | skill_{dyn.flx})$. We find these two are independent events in which both expressions equal $p(skill_{dyn.flx}) p(skill_{math})$, resulting in no directional dependency in our methodological framework.

Figure \ref{fig:Figure 2}~(a) shows the backbone of the resulting hierarchical network obtained by aggregating the empirically derived dependencies across all skill pairs. The network extends from general to specialized skills, incorporating their directional dependencies (the full network is shown in Fig.~\ref{fig:Figure 2}~b).
Nodes are colored by generality group as in Fig.~\ref{fig:Figure 1} and positioned based on educational requirements (x-axis) and Local Reaching Centrality (y-axis), a measure of skill generality denoting the number of other skills reachable from the focal skill \cite{Mones2012}. 
The network reveals distinct specialization paths and a partially nested architecture. Methods and SI Sec.~\ref{supsec:conditional dependencies} provide detailed parameters for statistical filterings and the threshold for directionality and backbone structure for Fig.~\ref{fig:Figure 2}.

Constructing a network structure from these conditional directions provides a methodologically consistent definition of general and specific skills using reaching centrality \cite{Mones2012} as an alternative measure for generality, as this can reflect the mass of interdependent nodes on the focal node (0.71 correlated). 
Chains of dependencies for select examples are also well embedded as expected, such as Deductive Reasoning to Math skills to Programming, exemplifying the nesting of skills in the skill hierarchy. Negotiation has a different set of dependencies compared to Programming, including Systems Analysis.
Supplementary Information Secs.~\ref{supsec:RN vs. NP} and \ref{supsec:hispanic skill entrapment} offer brief case studies highlighting the role of dependency chains in career progress and specialization.
Finally, we include the fully labeled visualizations of Fig.~\ref{fig:Figure 2}~(a-b) in SI Figs.~\ref{fig:figure_2b_labeled} and \ref{fig:full_figure_2b_labeled} for further examinations.
% In addition, SI-Sec.~\ref{supsec:conditional dependencies} includes the correlation between the two measures, as high as 0.71, as well as correlations among variables and additional statistical tests.

%  (1) rank skills by its average mean and median? (2) calculate a correlation with centrality (basically y-axis in Fig 2). This should give us very high correlation score (3) average of average means (and medians) within each group (general, intermediate, and specific)."Correlations (I will fill in the main text):
%average and median level: 0.98
%average level and LRC: 0.71
%median level and LRC: 0.70
%Group level info:
%Skill Groups Mean of Average Levels Mean of Median Levels Mean of LRC
%1      General               3.343942             3.3552941  0.38189874
%2 Intermediate               2.371088             2.3727907  0.10813977
%3     Specific               1.135680             0.8747436  0.04179466

% Mathematics skill, using mathematics to solve problems, is demanded across the economy, while Mathematical Reasoning, defined as the ability to choose the right mathematical methods or formulas to solve a problem, is important to certain occupations and irrelevant to others. 
% As such, our results complement existing work that has found evidence of a spectrum of human capital specificity at the individual level, e.g., \cite{Becker1962, Gathmann2010, Poletaev2008, Gibbons2004}.
% In search of a more systematic view of returns to human capital, we further examine horizontal skill interdependence.
% This is motivated by the intuition that skills closely dependent on various fundamental levels may require longer trajectories, hence requiring higher investments and leading to higher economic returns, as a result, \cite{Mincer1974, Lucas1988, Becker1962, Schultz1961}.
% node size (ranging from less than high school to post-doctorate)

%The nested skill matrix indicates that the skill structure deviates from a pure nestedness shape (triangular) when entities are sorted based on the number of their interactions, leaving us with two subtypes, the nested and un-nested skills, whose groups are aligned with the left and the right branches in Fig 2 (b). 
%In the following sections, we differentiate nested from un-nested specific skills, with the former building on general skills while the latter do not exhibit this characteristic, and examine their implications for individuals' career development, wage premiums, and skill gaps across geographic locations, genders, and racial/ethnic groups.

\subsection*{Skill Nestedness Contributions} 

\begin{figure*}[!h]
    \centering
    \includegraphics[width=0.8\textwidth]{Nature_HB_2023/figNHB/Figure_Nc_Nestedness_colored_by_category_Mar12_2024_annotated.png}
    \caption{
    \textbf{Skill Nestedness Contributions Score.} Skills' nestedness score is highly indicative of their generality (a), risk of automation  (b), and their value (c-d). Skill Nestedness Contributions are measured following \cite{Saavedra2011}. Generality is measured by Local Reaching Centrality, as in Fig.~\ref{fig:Figure 2},
    Automation risk Index and Value for each skill is calculated, following \cite{Frey2017, Frank2019, Frank2022}.
    We divide skills into \textit{nested}, positive contributions, and \textit{un-nested}, negative contributions toward the nested skill structure.
    }
    \label{fig:nestedness}
\end{figure*}

Figure \ref{fig:Figure 2} also illustrates that the alignment of skills within a nested structure is not uniform. While some skills, such as Programming and Negotiation, seamlessly integrate with general skills in a nested pattern, others break from this arrangement, creating an uneven, tree-like hierarchy.  
This reveals a \textit{partially} nested architecture in human capital, indicating that specific skills don't consistently subordinate to general skills \cite{Saavedra2011, Baldwin2014}. 

To systematically quantify and differentiate these observations on skills, we introduce the ecological measure of nestedness and individual contribution scores where specialist species engage preferentially with generalists \cite{Bascompte2003}.  This analogy extends to the skill ecosystem, where general human capital forms the bedrock for the acquisition and application of more specialized skills \cite{Saavedra2011, Saavedra2009}. 
Therefore, we first measure an overarching nested structure in human capital $N$. There are a number of different ways to measure nested structures. We employ several measures commonly used in ecology, such as the overlap index ($N_c$), checkerboard score, Temperature, and NODF, to ensure the analysis withstands the test of different nestedness measurements \cite{Stone1990, Almeida-neto2008, write1992, write1992, Saavedra2011} (See SI Sec.~\ref{suppsec:nestedness} for the full analyses and robustness tests).

Next, we calculate a skill's nestedness contribution score $c_s$ to assess its alignment with the overarching nested structure $N$ \cite{Saavedra2011}. This score is derived by comparing the actual nestedness ($N$) with a null expectation where a focal node $s$ is randomly distributed across occupations without any underlying dependencies such as $p(A|B)$, which is expressed as $c_s = (N - <N_s^{\ast}> ) / {\sigma{N_s^{\ast}}}$. 
Here, $N$ denotes the empirically observed nestedness in our survey dataset, while $<N^{\ast}s>$ and $\sigma{N^{\ast}_s}$ are the average and expected standard deviation of the nestedness of the random condition, respectively \cite{Saavedra2011}.
We conduct 5,000 simulations for $<N^{\ast}s>$ and $\sigma{N^{\ast}_s}$. In each simulation, occupations using the focal skill $s$ are randomly selected, keeping the skill degree constant. This method allows us to maintain consistency with actually observed patterns of niche and general skills but destroy the dependencies such that we identify how dependencies positively/negatively contribute to the overarching nestedness structure. 


Skills with a high nestedness contribution ($c_s$) are foundational to a hierarchical framework of human capital, suggesting a systematic progression from general to specialized skills toward layered learning paths that demand lengthy mastery effort \cite{Saavedra2011, Hausmann2011}. 
Such a pattern suggests a complex process of human capital formation characterized by interdependent skill acquisition pathways. These pathways are possibly essential for the emergence of specialized skills. In addition, they have profound implications for wages and education and contribute to disparities in demographics and opportunities \cite{Autor2014}.


Figure \ref{fig:nestedness}~(a) shows that highly specialized skills (blue) do not contribute equally to the overall nested structure and are thus divided into those with negative and positive contributions. 
As expected from Fig.~\ref{fig:Figure 2}, skills like Programming exhibit a positive impact on nestedness, indicating a strong reliance on vertical dependencies within their application domains.
In contrast, skills like Repairing, which also belong to the group blue in Fig.~\ref{fig:Figure 1}, are not heavily dependent on such structured dependencies and are quantified as having a negative contribution to nestedness.


We corroborate these findings with simulations of arrival probability from each focal skill. 
Figure \ref{fig:Figure 2}~(d-f) highlights the distinct interaction patterns among two types of specific skills: those that are primarily nested under general skills, such as Programming or Negotiation, and those that primarily interact with other niche skills, such as Repairing. We calculate arrival probabilities to the focal skill nodes and color other nodes according to their arrival probabilities to the focal node (see Methods) \cite{norris1998markov}. 
Unlike the well-nested Programming and Negotiation skills, only a handful of other skills are relatively more easily reachable from Repairing than other skills, which are mostly in the same parts of the skill tree.


Figures \ref{fig:nestedness}~(b-d) demonstrate that the nestedness score, a structural attribute, can translated into socio-economic properties. These findings suggest that skills with high nestedness contributions are more likely to be associated with lower risks of automation and higher wages, as they are integral to a deeply interconnected structure that demands considerable investment for mastery \cite{Davidson1898book, Autor2003, Frey2017}. Such skills play a crucial role in creating a distinctively hierarchical human capital with vertically intricate dependencies, fostering specialized niches that potentially affect wages, education, and demographics. In contrast, skills with negative nestedness contributions, such as Repairing, do not exhibit the same level of dependence on structured hierarchies and may be more susceptible to automation and lower wages. This highlights the importance of considering not only the generality of skills but also their position within the skill hierarchy when assessing their socio-economic implications.

The relationship between nestedness contributions and socio-economic outcomes underscores the significance of the skill hierarchy in shaping the labor market. By understanding the structural properties of skills and their interdependencies, we can better predict the impact of technological change on different skill domains and inform policies aimed at promoting skill development and mitigating the risks of job displacement.

For the remainder of this paper, we simplify the exposition by defining skills according to their skill group and the sign of their nestedness score $c_s$. Skills with $c_s > 0$ are indexed as \textit{nested}, while those with $c_s < 0$ are considered \textit{un-nested} skills.
We continue to refer to general skills as such since all skills in that group have positive nested scores.

\begin{figure*}[!h]
    \centering
    \includegraphics[width=\textwidth]{Nature_HB_2023/figNHB/Figure_Age_-_Mar_14_2024.png}
    \caption{
    \textbf{Skill Compositions with Occupational Ages and Career Trajectory.}
    \textbf{(a-c)} Average skill levels of occupations (and 95\% confidence intervals), segmented by occupations' employees' median ages. Levels of general and nested skills rise with an occupation's median age, while unnested skills do not vary across median-age groups.
    \textbf{(d-f)} Average skill levels (and 95\% confidence intervals) against age in synthetic birth cohorts. The insets isolate cohorts born in 1967, whereas the main figures average across all cohorts. Notably, general and nested skills rise markedly until around age 30, with declining unnested skills. Moreover, gender gaps also become more pronounced around 30. 
    \textbf{(g-h)} Average skill levels (and 95\% confidence intervals) over identified job sequences as documented in resumes for general, nested, and unnested skills. 
    \textbf{(i)} Changes in skill levels in consecutive job transitions. Skill profiles are typically stabilized within the initial five jobs.  The grey triangles indicate bootstrapped results where the sequences of jobs are randomized.
    }
    \label{fig:age}
\end{figure*}
    % slopes: nested specific = 0.46 + 0.018 * age, R^2 = 0.031; un-nested specific = 1.3 - 0.01 * age, R^2 = 0.0045; residualized nested specific = 0.3 - 0.01 * age, R^2 = 0.018; residualized un-nested specific = -0.8 + 0.016 * age, R^2 = 0.0015;

%%%% DONT FORGET CORRELATION BETWEEN GENERAL and NESTED, and UNNESTED. 
%% Cor(General, nested specific) = 0.646
%% Cor(General, un-nested specific) = -0.379
%% Cor(nested specific, un-nested specific) = -0.095

% Such dynamics are driven by general skills being conditional on nested-specific skills.

%% i. career trajectory: BG and age & economic return
%The degree to which general versus specific skills play a role in many occupations acknowledges a longstanding debate in labor economics, sociology, and management \cite{Becker1962, Neal1995, Parent2000, Poletaev2008, Gathmann2010, FergusonJohn-Paul2013, Leung2014, Merluzzi2016, Byun2018, Fini2022, Byun2023, Leahey2007, Teodoridis2018, Heiberger2021}. 
% using DATA: https://www.bls.gov/cps/demographics.htm\#age} 
% the median age for each occupation in 2019
%  CPS has only 2 years of observation that is overwhelmed by short-term mobility and hence part-time jobs. 
% To the extent that such learning of skills correlates with age, controlling for the pre-requisite (i.e., general skills) should account for any relationship between%. However, the vice versa should not hold: controlling for the dependent (i.e., nested specific) skills should not eradicate the relationship between age and the pre-requisite (i.e., general) skills. The dependent (i.e., nested specific skills).
% However, apart from the challenges of obtaining individual skill data, information captured in intermediately used such sources, for instance, resumes, is likely biased towards reporting specific skills. 
%Include this in data section: Compared to the Current Population Survey, the Burning Glass data is more skewed (ref the paper Daehyun provided) 
% For each move, we link the source and destination occupations to skills from O*NET in 2019.
% we have cohorts born before 1980 but show all cohorts' skill compositions in the period of 1980 and 2022.
%\subsection*{Workplace skill acquisition through career trajectories}
%\subsection*{Occupational Skill Compositions in Career Trajectories}
\subsection*{Skill Categories in Career Trajectories}

In this section, we examine how the derived skill structure uncovers individual career trajectories through three empirical observations: median ages for occupations, synthesized birth cohorts from individual surveys, and job transitions in resumes. Each data source provides unique strengths and weaknesses, which, when combined, complement each other and sketch a coherent picture of career paths.  

We begin our analysis with occupational ages, as it is reasonable to expect progression and skill development to correlate closely with age due to the substantial investment of time and the dense set of prerequisites they demand \cite{Argote1990,Jovanovic1997,Nedelkoska2015}. 
Figure \ref{fig:age}~(a-c) shows the levels of general, nested, and unnested skills in occupations, segmented by their median ages, computed using the Current Population Survey (CPS) (see Methods).  
The outcomes align consistently with our predictions \cite{Jovanovic1997}. Occupations with median ages over 30 demand high levels of both general and nested skills, while unnested skills, supposedly lacking interdependencies, do not demonstrate any significant correlations with ages.

To examine if our results hold across career trajectories, we construct synthetic birth cohorts using the CPS microdata, which provides yearly repeated cross-sectional surveys but does not allow longitudinal tracing of respondents long enough for us to trace a few decades. Therefore, we connect snapshots of surveys through their birth years to mimic career trajectories   \cite{Acemoglu2011,Hermo2022}. 
For example, we construct a 1967 cohort for Fig.~\ref{fig:age}~(d-f), excluding observations of non-full-time respondents and those below age 17 or above 55. We then repeat this for different birth cohorts. 

Figures \ref{fig:age}~(d-f) show the skill composition of synthetic birth cohorts from 1980 to 2022, with insets for the 1967 cohort. Consistent with the findings in Fig.\ref{fig:age}~(a-c), age 30 emerges as a significant transition point. General and nested skills concurrently increase sharply until around 30, when unnested skills experience a moderate decrease. After the age of 30, the rise in overall skill levels stabilizes.

The advantage of the second dataset is the information on both the age and demographics of individuals, allowing us to decompose our findings by gender. 
Differentiating skill trends by gender uncovers a gap in specializations that emerges around 30. Men continue to grow their general and nested skills until their 50s, whereas for women, the increase in these skills hits a plateau in their early 30s, the typical age range for first-time mothers in the US.
Supplementary Information Secs.~\ref{supsec:Parenthood_Male_vs_Female} and \ref{supsec:female job sorting} further investigate the influence of parenthood on male and female workers by slicing data by those with and without children as well as the influence of sorting into jobs based on schedule and working hours, respectively. 
These findings are robust to conditioning out yearly economic conditions (SI Fig.~\ref{fig:individuals' age and skill - year effects}).
In the following sections and in Fig.~\ref{fig:Skill Age Gender Race Trends - year effects}, we offer more detailed breakdowns of these gender disparity trends with respect to race and ethnicity.
Notably, education does not fully account for the growth in skill documented by our analysis.
As SI Fig.~\ref{fig:individuals' age and skill and education} shows, the share of educational attendance is negligible after the age of 30, while skill growth continues. Similar patterns, in more modest magnitudes, emerge for workers with no more than high school diplomas (SI Fig.~\ref{fig:individuals' age and skill - no college}.)

Lastly, we complement our findings using resume datasets that record individual job transitions, encompassing over 70 million job transitions documented in over 20 million resumes. While these data provide a direct record of individual workers' job sequences, they are not publicly accessible, do not include age or gender information for detailed analyses, and are known for biased sampling, favoring more nested job roles. Hence, while valuable for corroborating previous findings, they cannot replace the previous datasets.

Figures \ref{fig:age}~(g-h) show the average skill levels required in job sequences held across career paths, and Fig.~\ref{fig:age}~(i)  displays changes in skill requirements for the $i$th job transition, $\Delta_i$, excluding job transitions within the same occupation ($\Delta_{i}= 0$).  
On aggregate, career journeys unfold with increasing stocks of both general and nested skills ($c_s > 0$), suggesting that nested specialization paths require simultaneous increases in nested specific skills along with their dependency skills.  In addition, we find skill portfolios typically stabilize within the first five job transitions ($\Delta_{i > 5} \approx 0$), and in the first three jobs ($i < 3$), nested skills require more general skills than later ($\Delta^{general}_{i < 3} \gg \Delta^{nested}_{i < 3}$), after which they become comparable ($\Delta^{general}_{i > 3} \approx \Delta^{nested}_{i > 3}$). 
The continued growth in general skills across career paths suggests that these skills need to be continuously enhanced regardless of career stage. As a benchmark, we create bootstrapped job sequences (gray marks around zero) that randomize the order of jobs as if there were no career development, confirming that the observed trends are indeed attributed to career developments (see SI Sec.~\ref{supsec: bootstrapping BG} for details).


To explore nested specialization, we choose registered nurses (RNs) and nurse practitioners (NPs) by analyzing resume data to understand how skill and wage differences manifest in career trajectories. 
Supplementary Information Fig.~\ref{fig:RN vs. NP} shows the additional skills (necessary to prescribe medicine and diagnostic tests) in higher-paying NP positions appear in nested paths with growth in both general and dependent niche skills such as medicine, therapy, biology, science, and chemistry (see SI-Sec.~\ref{supsec:RN vs. NP} for the detailed analysis).  
In addition, SI Sec.~\ref{supsec:hispanic skill entrapment} makes a case wherein insufficient levels of certain general skills preclude the development of the dependent niche skills, once again highlighting how our framework teases out pathways for developing human capital.


All three empirical observations consistently depict nested specializations (i.e., growth in both general and nested skills) throughout career trajectories, while unnested skills are left relatively underdeveloped. The resume analysis offers direct evidence of a recurring yet counterintuitive pattern: valuable specialization is not just about developing niche skills; it is conditional on advancing the required more general skills. 
This suggests that the conventional model, where basic general skills precede advanced specialized skills, is not entirely accurate. Instead, career paths tend to unfold with increasing emphasis on general skills and their dependent, nested skills. While research has emphasized the role of education, Fig.~\ref{fig:age} (and SI Figs.~\ref{fig:individuals' age and skill and education} and \ref{fig:individuals' age and skill - no college}) reveal that skill advancement continues long after the age of schooling, suggesting nested specialization pathways operate through but also beyond education \cite{Jovanovic1997, Mincer1974, Arrow1962, Lucas1988, Hermo2022}, challenging the commonly held role of education in developing human capital. 

One might argue that our findings are driven by management/administration jobs, which are typically undertaken later in careers with higher wages. 
To ensure they do not drive our findings, we repeated the entire analysis without these factors and found consistent results (see SI Sec.~\ref{sec:robustness check: no managers}). Also, we repeated the entire analysis, excluding social skills, and again
our results remained robust, suggesting that our findings are attributed to the intrinsic structure of skills rather than the influence of particular social skills or managerial jobs (see SI Sec.~\ref{sec:social skills} for the full analyses).
%In essence, individuals typically experience rapid growth in general skills, followed by nested specializations, as if they are prerequisites. 
%Although swiftly diminished, however, these nested specialization paths continue to involve a strengthening of their foundational, general skills. 


%% ESTABLISH INTERDEPENDENCE
% Establish the term nested specialization
%% Wage Results: Investment pays off...
% We supplement the above analysis with an explicit examination of how nested interdependencies correlate with education and influence the value of skills.
%These relationships are commonsense \cite{Alabdulkareem2018}: occupations that possess "valued" skills command higher premiums and require more education. The question is whether specific or general skills drive this pattern.    %in two setups: unconditional and when conditioned on general skills' endowment. Comparing the unconditional and conditional coefficients for nested specific skills shows that much of their association with education and wages is due to general skills.
% Accumulating absorptive capacity in one period will permit its more efficient accumulation in the next. By having already developed some absorptive capacity in a particular area, a firm may more readily accumulate what additional knowledge it needs in the subsequent periods in order to exploit any  critical external knowledge that may become available wage premiums and require more education. 
%We analyze the differential influence of skill categories on occupational wages and investments in education. 

%Moreover, the initial wage penalties for unnested specializations have turned into small positive wage premiums.  
%A closer look at Fig. \ref{fig:Wage} (a) provides insight into this non-trivial phenomenon. 
%Unlike the monotonic decline in educational level with unnested specializations, annual wages start rising again at very high levels of un-nested specific skills. This suggests that high-level specialization eventually provides benefits surpassing what broad skills and education can offer. 
%This observation is consistent with human capital theory \cite{Becker1962}. 
%Sailors, for instance, who command ships, have a greater level of unnested skills, both raw and residualized, thus earning more than those who load the ships.

%%---------------------------------------------------------------%%
\subsection*{Skill Categories and Wage Premiums} 

\begin{figure*}[!h]
    \centering
    \includegraphics[width=\textwidth]{Nature_HB_2023/figNHB/Figure_-_Education_-_Wage_Figure_-_Apr_1_2024.png}
    \caption{\textbf{Skill Wage Premiums and Educational Requirements}. \textbf{(a)} Occupations' average annual wage and \textbf{(b)} required education levels plotted against skill levels (with 95\% confidence intervals), 
    and their respective slopes (blue bars) in \textbf{(c-d)}, and standard errors. 
    The substantial wage premiums and higher educational requirements associated with nested specializations much reduced (shaded bars) after controlling for general skill levels (insets), implying that the bulk of investments in and returns to specialization are conditional on the accumulation of general skills. The initial wage penalty for unnested specializations turns into a wage premium once general skill levels are controlled for.  
    % slops after scaling: wage = 4.5 + 0.19 * nested specific, R^2 = 0.38; wage = 4.9 - 0.084 * un-nested specific, R^2 = 0.12; wage = 4.8 - 0.0071 * residualized nested specific, R^2 = 0.00022; wage = 4.8 + 0.023 * residualized un-nested specific , R^2 = 0.0065;
    }
    \label{fig:Wage}
\end{figure*}



Figure~\ref{fig:Wage}~(a-b) supports our premise that nested specialization patterns are associated with wage premiums. In particular, we find that educational requirements and average annual wages tend to rise with rising requirements of nested skills in an occupation. However, a closer examination of the observed wage premiums for nested skills (blue bar) in Fig.~\ref{fig:Wage}~(c) reveals that such premiums almost fully disappear when we control for the occupation's general skill requirements (shaded bar). This suggests that general skills are integral to the deployment of nested skills. In contrast, unnested skills ($c_s < 0$) seem to be associated with wage penalties. However, controlling for general skill requirements now turns this penalty into a wage premium that is comparable in magnitude to the nested skill premium. This shows that unnested skills are also valued in the labor market. However, their wage premium is not immediately apparent because unnested skills tend to correlate with an \emph{absence} of general skills.   


Further analyses in SI Sec.~\ref{supsec: add - returns to skill} demonstrate that these results are robust to controlling for education, training, and workplace experience and hold across subsamples of major occupational groups. 
Again, the results are not driven by managerial occupations or social skills, usual suspect factors in wage premium (see the results in SI-Table \ref{tab:wage reg on skill endowment}, and SI Figs.~\ref{fig:SI_education_skill_level}-\ref{fig:SI_wage_skill_level},
\ref{fig:Figure 3 full | major occupation groups}, \ref{fig:returns_to_skills_hierachy_gen_dependence_cor_no_manager}, and \ref{fig:social skills}).  

% \begin{figure*}[!h]
%     \centering
%     \includegraphics[width=0.95\textwidth]{Nature_HB_2023/figNHB/Fig 6 - Jul 11 2023.png}
%     \caption{\textbf{Spatial Distribution of Skill Categories.}
%      \textbf{(a)} General, \textbf{(b)} Nested, and \textbf{(c)} Un-nested skill levels of each county's occupational composition, using their standard score (z-score) relative to the national level (see Methods). 
%      The most populated counties in each state are enclosed in a box, and the top five and bottom five U.S. counties are highlighted in italics.
%      There is a noticeable concentration of general and nested skills in densely populated areas, while rural areas demonstrate a higher level of un-nested skills.
%      \textbf{(d)} and \textbf{(e)} illustrate the average skill levels (and 95\% confidence intervals) of each skill category in relation to population size and manufacturing industries, respectively.}
%     \label{fig:Geography}
% \end{figure*}


%Moh Hosseinioun: To form confidence intervals, I took samples of 10% from the sub-population of interest (say, when comparing the gender gap: Asian female and Asian male) and recalculated the ratio of interest, for instance, log(wage Asian female)/log(wage Asian male).)  I repeated this sampling and calculation process 10,000 times to find the distribution of the estimated value. This way, we obtain the distribution of the measure of interest. The distribution gives us the values at 95 percentiles and, hence, the 95% confidence intervals.
%To do so, we rely on CPS microData between 1980 and 2022, deriving skill endowments across race/ethnicity (White, Black, Hispanic/Latinx, and White) and gender (Female and Male) groups for full-time employed workers in each skill category.
% You: Yes! But why do we need to take a log for a ratio? Can we just do (wage asian female)/(wage asian male)? instead of log (wage Asian female)/log (wage Asian male)?
% Moh Hosseinioun: for wages, I am not sure why even ratios are log(wage_x)/log(wage_y). This is the convention.


% Such skill gaps persist even though women (apart from Asians) report higher educational attainment.


%%---------------------------------------------------------------%%
\subsection*{Disparity in Demographic Groups}

\begin{figure*}[!h]
    \centering
    \includegraphics[width=\textwidth]{Nature_HB_2023/figNHB/Figure_7_-_Demographic_Ratios_-_log-y_-_Skills,_Education,_and_Wages_-_1980-2022_-_Jul_11_2023.png}
    \caption{\textbf{Skill Disparity in Demographic Distribution of race/ethnicity and gender}
    \textbf{(a)} The relative average skill level, education level, and weekly wages for Asian, Black, and Hispanic/Latinx workers compared to White workers (expressed as a ratio).
    \textbf{(b)} The relative average skill level, education level, and weekly wages for female workers compared to male workers. 95\% confidence intervals for each estimated ratio are calculated by bootstrapping subsamples (see Methods). These differentials are robust to measurement (SI Fig.~\ref{fig:Tong et al race gender skill distribution}), follow similar age trends seen in Fig.~\ref{fig:age}, and are robust to time-variant economic factors (Fig.~\ref{fig:Skill Age Gender Race Trends - year effects}.) SI Figs. \ref{fig:Temporal Race Gaps - Skills, Education, Wages} and \ref{fig:Temporal Gender Gaps - Skills, Education, Wages}, further show the gaps have narrowed over time.  
    } \label{fig:Figure 7} 
\end{figure*}


To gain a better understanding of the role that skill differences may play in labor market inequalities, we examine how skills vary across demographic groups. Figure \ref{fig:Figure 7}~(a) compares skill, education, and wage differences across race/ethnic groups against their White peers. The results, first of all, show large wage gaps between Black and Hispanic workers on the one hand and  Asian workers and the baseline of White workers on the other hand. These wage gaps are accompanied by employment in jobs with lower requirements of nested skills for Black and Hispanic workers. However, for Hispanic workers, there is another potentially important factor: elevated unnested skill requirements. 

We explore this further in a brief case study of how language-skill requirements may keep workers out of jobs that require certain nested skills (see Supplementary Information Sec.~\ref{supsec:hispanic skill entrapment}). To do so, we leverage the hierarchical nature of our skill network. This allows us to distinguish between nested skills that depend on (general) language skills and nested skills that don't. Within the group of Hispanic workers, we find particularly large gaps in language-dependent nested skills compared to other nested skills for workers who have recently moved to the US. Such workers may instead develop un-nested skills, leading to ``skill traps'' that are associated with long-run wage penalties (SI Fig.~\ref{fig:wage curves}). Taken together, these findings indicate that closing wage gaps for Black workers may require different solutions than for Hispanic workers. 

Figure \ref{fig:Figure 7}~(b) focuses on skill gaps between men and women across social groups. The most pronounced disparities exist in nested and unnested specializations. Except for in the Asian subsample, women tend to work in occupations that require higher levels of education and general skills than men. However, this does not translate into higher levels of nested skills, where women tend to fall behind men. These disparities are likely to contribute to the well-known gender wage gap we observe in the right-most panel. Encouragingly, this gap has narrowed over time, as demonstrated in SI Fig. \ref{fig:Temporal Gender Gaps - Skills, Education, Wages}. However, the disconnect between education and general skills on the one hand and wages and nested skills on the other is puzzling. 
Supplementary Information Sections \ref{supsec:Parenthood_Male_vs_Female} and \ref{supsec:female job sorting} probe deeper into these gender gaps. This analysis suggests that parenthood, as well as the fact that women often work in jobs with more regular and predictable work schedules, impact both wages and skill development \cite{Bertrand2009, Goldin2015, Canon2016}. In fact, whereas having children is associated with reduced general and nested skills for women, men with children tend to have higher levels of general and nested skills than men without children in the household. When it comes to work schedules, Similarly, we find that the association between gender and nested skill requirements at work is reduced by over a third when we control for irregular hours and overtime in an occupation. 

Finally, Section \ref{section: add - geographical distribution of skills} of the Supplementary Information studies the geographic distribution of skills, showing that general skills concentrate in densely populated urban areas. This finding is in line with prior work that highlights the diverse and complex economic activities that are found in large urban economies \cite{Youn2016, gomez2016explaining, Hong2020, Balland2020, Bettencourt2014, Gomez-Lievano2021}. Moreover, this greater concentration of general skills in large cities can account for about one-third of the well-established urban wage premium \cite{Glaeser2001}. 



In summary, the analysis of skill categories across demographic groups reveals a complex interplay between skills, education, and wages that leave an imprint on macro-level labor market disparities between societal groups. Although a deeper analysis of the causes and consequences of these disparities is beyond the scope of the current paper, our results highlight that analyzing skill gaps solely through the lens of educational attainment overlooks aspects of human capital that have an important impact on a variety of labor market disparities. Moreover, the complex interaction between wages and skill types suggests that considering such aspects may provide valuable insights for labor market policies: addressing long-lived disparities in the labor market may require targeted interventions that go beyond traditional educational programs and instead consider how different skill categories shape labor market outcomes.


\subsection*{Widening Gap in the Skill Structures}

\begin{figure*}[!h]
    \centering
    \includegraphics[width=\textwidth]{Nature_HB_2023/figNHB/Historical_Comparison_of_Skills_-_Jul_14_2023_+_arrows.png}
    \caption{\textbf{Historical Changes in the Skill Structure}
    \textbf{(a)} Distribution of skill levels for different skill groups in 2005 and 2019. The arrow indicates the shift in average skill levels from 2005 to 2019. Unlike the positive shifts in general skills, the shift in specific skills is not as noticeable.
    \textbf{(b)} Distribution of skill levels for nested and unnested skills in 2005 and 2019. The arrow shows the shift in average skill levels from 2005 to 2019. While nested skills follow the shift in general skills, the demand for unnested skills has decreased.
    \textbf{(c)} Comparison of skill hierarchy structures between 2005 and 2019. The changes in the structure of skill hierarchies over time highlight an increasing divide in the dependencies of nested and unnested skills and the widening gap between them.} 
    \label{fig:historical skill change}
\end{figure*}
% Nested jump: social scientists and economists  on the one hand, and Telemarketers and Technical Writers on the other hand? We can mention,  as a side note, the CEO and Human resource.
% Occupations of high nested/ and changed higher over the decade: Medical and Health Services Managers become more nested while Computer User Support Specialists become unnested.  Medical and Health Services Managers(nestedness change: 1.0278089) and Computer User Support Specialists (nestedness change: -0.9725729).

The historical changes in the skill structure, as shown in Figure \ref{fig:historical skill change}, raise concerns given the important roles that nested and unnested specializations play in career progression and demographic and regional disparities. These changes reignite the debate over the widening job polarization \cite{Autor2013, Alabdulkareem2018, Althobaiti2022}. 

Figure \ref{fig:historical skill change}~(a) indicates an increase in the demand for general skills, as evidenced by the shift from the dotted to the solid distribution. This increase in demand for general skills corresponds to higher wage premiums over the recent decade, suggesting that the economy has been rewarding workers with a broad set of skills (see SI Fig.~\ref{fig:Wage and education 2003}). However, the seemingly static distribution of specific skills masks underlying changes in the application of nested and unnested skills. As shown in Figure \ref{fig:historical skill change}(b), there has been a rise in the application of nested skills and a decline in the use of unnested skills between 2005 and 2019, reiterating the importance of considering skill interdependencies when analyzing changes in the skill structure. 

These changes have led to a more nested skill structure, as indicated by the decreased checkerboard score (from 438.67 to 356.4) and temperature (from 40.07 to 31.89), and increased NODF (from 39.06 to 41.72) and $N_c$ (from 573,873 to 651,030) between 2005 and 2019 \cite{Stone1990, Almeida-neto2008}. A lower checkerboard score and temperature, along with a higher NODF, signify a more nested structure.

However, the shift towards greater overall nestedness has been uneven across different skill sets. Figure~\ref{fig:historical skill change}~(c) demonstrates this as a widening gap between the nested and unnested branches over the decades. Supplementary Fig.~\ref{fig:Figure_7_supp}~(b) shows a strengthening in the connections among nested skills, indicating their growing complexity and mutual dependence, while SI Fig.~\ref{fig:Figure_7_supp}~(d) reveals weakened dependency chains of unnested skills. All in all, this trend is observed as a broadening and deepening of nested skill branches within the hierarchy in Fig. ~\ref{fig:historical skill change}~(c), reflecting an increase in the complexity and interdependence of these specialized skill areas \cite{DemingDavidJ2020EDCJ, Tong2021}.

Indeed, the chasm between the two types of specializations has alarmingly broadened within the educational domain over the last two decades \cite{Xu2021, Lin2022, Althobaiti2022}. In response to this, the structural changes in the hierarchical tree network are concerning, given the significance of these specializations for future career developments and wage premiums.  They reveal an economy wherein the structure of valuable human capital has grown more nested, reinforcing barriers to workers without the necessary fundamental skills, who are often entrapped in unnested specialization pathways (see SI Figs.~\ref{fig:wage curves} and \ref{fig:hispanics language skills}). The widening gap between the nested and unnested specialization paths could indicate strongly rooted chronic disparity. 

The increased demand for general skills and the shifting balance between nested and unnested skills have important implications for workers and policymakers. While the rising wage premiums associated with general skills suggest that workers who invest in developing a broad skill set may be better positioned to succeed, the growing importance of nested skills and the declining use of unnested skills may exacerbate existing inequalities and create new barriers to entry for certain occupations.

As the skill structure becomes more complex and interdependent, policymakers and educators must develop strategies to ensure that individuals from all demographic groups and regions have access to fundamental skill development opportunities. Failing to do so may exacerbate disparities and hinder economic mobility for certain segments of the population. To mitigate the potential negative consequences of the changing skill structure, it is crucial to invest in education and training programs that foster both general skills and unlock valuable specialized pathways. Our results suggest that providing individuals with a strong foundation in general skills and the opportunity to develop nested specializations is essential for navigating the increasingly complex labor market and achieving better career outcomes.



% Network structure: 
% horizontal interdependencies among specific skills have remained largely unchanged.
% For instance, Biology, Medicine, and Chemistry remain in the relatively same positions, evident in the stronger concentration of general skills at the center of the nested community.
% examples, Oral Comprehension and Information Ordering, using computers?
%Despite the growing importance of social skills \cite{Liu2013}, our findings suggest that they do not drive the effect of general skills (or any other skill subtype). Instead, they hinge on a broader array of general skills (see SI Section \ref{fig:social skills}).


% specialization has increased the depth of knowledge in each domain, increasing the length of a typical learning trajectory from basic to advanced topics \cite{BenJ2009}. 
% more effort before a novice can become an expert. % increasing the depth between the basic and the advanced topics.

%%--------------------------------------------------------------------------%%
\section*{Discussion} 
%The central goal of this paper is to uncover the underlying dependency structure among skills used in the U.S. labor market by analyzing a large-scale skill survey that documents the human capital requirements of hundreds of U.S. occupations. The resulting structure with two distinct skill communities is not only in good agreement with a distinction between manual and cognitive skills but also adds a new dimension to the structure, directionality, to what had been shown in prior research \cite{Alabdulkareem2018, Frank2018, Moro2021}. Adding directionality to the network, connecting skills according to inferred prerequisite-dependency relations, yields a hierarchical network that suggests a different skill taxonomy, highlighting the difference between general skills, on the one hand, and nested and un-nested specific skills, on the other. Remarkably, this taxonomy emerges from information only on how skills co-occur within occupations, without using any substantive information on the skills or occupations themselves. The derived taxonomy therewith combines elements from a recent body of work on economic complexity that represents skill structure as networks \cite{Neffkeeaax3370, AndersonKatharineA2017Snam, Alabdulkareem2018, Xu2021, Lin2022, Althobaiti2022, DelRio-Chanona2021, Tong2021} and traditional research in economics on human capital specificity \cite{Gibbons2004, Poletaev2008, Gathmann2010, BeckerG.S1992TDoL}. 

%TheA growing literature has drawn from complexity  on skills has explained the increasing polarization of jobs in the US labor market by considering skills as a multidimensional construct usingtheories to study human capital, emphasizing the multidimensional nature of skills \cite{AndersonKatharineA2017Snam, Alabdulkareem2018, Neffkeeaax3370, Moro2021, DelRio-Chanona2021, Althobaiti2022}.
%This literature's application of co-occurrence networks to describe skills has shown promise in elucidating labor market frictions and mobility barriers.
% These skill co-occurrence networks have contributed to the debate on human capital by delineating how different types of skills shape the introspective labeling of workers. 
%However, without incorporating the directionality of skill acquisition and application, a critical topic in the economic and sociological debate on human capital, these co-occurrence networks have fallen short of uncovering the underlying mechanisms that drive careers, generate value and build resilience in labor ecosystems.


Human capital has traditionally been quantified in terms of years of schooling or work experience, yielding important insights about wage curves and returns to education \cite{Mincer1974, Becker1962}. With the arrival of detailed data on tasks that people perform at work and the skills they require to do so, a more granular assessment of human capital became feasible, juxtaposing, for instance, cognitive and manual skills, routine and non-routine skills, or STEM and social skills \cite{Autor2003, goos2007lousy, Deming2017}. However, these dichotomies are often ad hoc, tailored to test specific assumptions about trends in the labor market, such as mechanization, computerization, and the rising importance of soft skills.

In contrast, the complexity approach to human capital analyzes labor markets through network analysis, providing a more comprehensive and data-driven perspective. Understanding the network architecture in complex economic systems—spanning technology, input-output, supply-chain, trade, products, and skills—has yielded insights into socioeconomic phenomena \cite{Neffke2013, Baldwin2014, Alabdulkareem2018, McNerney2022, Mariani2019, DelRio-Chanona2021, hidalgo2018principle, Elliott2014, Acemoglu2012, Elliott2022}. These insights both corroborate and contest established theoretical frameworks, including the underlying causes of economic disparities between countries and their potential developmental trajectories by analyzing trade networks \cite{hidalgo2018principle, mcnerney2021bridging}; the pace of technological innovation and economic growth through technology networks \cite{mcnerney2011, McNerney2022}; differences in labor productivity and resilience through the lens of skill and occupation networks \cite{Neffke2013, Neffkeeaax3370}; and economic network resistance and persistence using business networks \cite{Saavedra2011, Moro2021, Frank2018, DelRio-Chanona2021}. These models and methods translate a range of structural properties into quantifiable and actionable insights.


Our empirical study aims to add a new layer to these structural properties by illustrating how connections within these networks are conditional and how structures become increasingly nested as complexity and specialization grow. This method provides structural insights into prior empirical findings that cognitive skills are clustered themselves and valued more highly than physical skills, based on patterns of co-occurrence \cite{AndersonKatharineA2017Snam, Alabdulkareem2018, Xu2021, Lin2022, Neffkeeaax3370, DelRio-Chanona2021}. Within this networked framework, we observe a system where more central nodes, or skills, are rewarded more substantially along the network's nested branches, suggesting that the value attributed to cognitive skills in previous findings is interdependent with the increasingly nested structure of the skill network. Consequently, this finding leads us to move beyond the traditional dichotomy of cognitive versus physical skills towards a structural classification of skills as either nested or unnested. 


Our research thus bridges economic theories that recognize hierarchical structures to explain progression and wage premiums \cite{Mincer1974, Becker1962} and the economic complexity model for understanding economic development \cite{hidalgo2018principle}, where the hierarchical organization of skills and their societal implications have been taken for granted rather than empirically verified \cite{Neffkeeaax3370}. Our work aims to offer an empirical framework as a network of skills in which ties capture skills' directional interdependencies, distinguishing pathways to specialization. Notably, we show that relying solely on the information embedded in the network of skills and occupations allows for a quantification of skills based on the concept of nestedness, independent of economic and social variables and without any presupposed or context-informed labeling of skills. Our analysis of wage and demographic disparities shows the predictive capability of this minimal approach for various socioeconomic factors.

The hierarchical structure and its inherent directionalities add a new dimension to the rising field of economic complexity, providing a deeper understanding of how knowledge is accumulated within a population and how its precedence relations between activities are expressed in the economic activities of a firm, city, region, or country \cite{Hidalgo2015, OClery2021, Balland2020, Hidalgo2021, Harmand2015, Hausmann2011, Hidalgo2007, Hidalgo2009,tacchella2012new, Park2019}. The directional dependencies that we propose break the symmetry in traditional co-occurrence networks for a better understanding of structural changes in economic complexity \cite{mcnerney2021bridging, OClery2021, yang2022scaling, Hong2020, hidalgo2018principle}.

In increasingly complex, large teams, social skills become crucial when specialization requires workers to coordinate with team members possessing different specialized skills \cite{Carlile2002, Wuchty2007, yoon2023, Neffkeeaax3370, Wu2019Science, Borner2018}. Our framework identifies and locates social skills embedded in the skill structure along with general and nested skills (SI Fig. \ref{fig:social skills}-a), explaining their recent growth and significant role in wage premiums (see SI Fig. \ref{fig:social skills}-b) \cite{Liu2013, Deming2017, Borghans2014, Weinberger2014, Lindqvist2011, Kuhn2005, Deming2018, Kogan2021, VanderWouden2023, Evans2024}. Nevertheless, our results go beyond the contributions of social skills and managerial occupations to wage premiums, as the results are robust to their absence in analyses (see SI Sections \ref{sec:social skills} and \ref{sec:robustness check: no managers}). Therefore, social skills are valuable not just because of their role in sociality but because of their structural properties, serving as foundational building blocks of human capital to enable further valuable specialization and more complex organizations.

The structural implications of our findings extend beyond individual careers and their associated rewards, suggesting potential consequences for not only intra-generation career mobility but also perhaps inter-generation career mobility. In this context, Figure~\ref{fig:historical skill change} presents a disconcerting trend, illustrating the widening gaps across nested branches over the span of a decade. We speculate that these growing disparities may be attributed to the increasing complexity of the economy and the deepening of individual specializations. As the skill structure becomes more intricate and the dependencies between skills more pronounced, individuals who successfully navigate these nested pathways may reap significant benefits, while those who struggle to acquire the necessary skills may face limited opportunities for advancement, potentially leading to entrenched inequalities that persist across generations. However, we acknowledge that the current study does not fully underpin these implications due to the lack of detailed datasets and a comprehensive analytical framework. Therefore, we recognize the need for further research to examine these critical questions and unravel the long-term consequences of the evolving skill structure on intra-generation and inter-generation mobility in order to inform policies and interventions aimed at promoting equitable access to skill acquisition, fostering inclusive economic growth and mitigating the potential for widening disparities within and across generations.

There are limitations in inferring the dynamics. First, our current empirical findings do not establish a causal relationship between semantic categories and structural manifestations, presenting an important question for future research using theoretical frameworks and computational models. 
Second, our analysis leverages datasets of occupational ``requirements'' of skills, that is, skills that are applied, which is not a direct measure of skill acquisition.  In essence, the manner in which skills are learned remains outside our observational scope. We assume that skills applied in the workplace have been acquired beforehand but not long before. This suggests that an individual may have competencies in arithmetic, linear algebra, and programming, which might not be fully exploited until they progress in their career. Although possible, such instances are presumed to be rare and not economically sensible, as individuals typically do not seek to acquire skills that are not immediately necessary, which probably pay less. This presumption rests on the belief that individuals strive to optimize their earnings and learning opportunities within their limited time and resources, making the phenomenon of being overqualified for job requirements relatively uncommon. Fundamentally, we suggest that there is a reluctance to engage in learning and skill development without direct application or compensation; thus, they occur relatively together. Related, our unit of analysis is jobs rather than individual employees, limiting our ability to discern the co-occurrence of learning and skill application. Future research could benefit from surveys targeting employees to gather nuanced data on individual skill portfolios as opposed to relying solely on job surveys. Finally,  our data primarily describe the U.S. labor market, which has idiosyncrasies in its education system, industrial composition, and urban structure. 
How well these findings generalize to other work settings, such as entrepreneurship \cite{Murray2023}, and economies, especially those at different stages of development \cite{Autor2022}, remains a task for future research.

Essentially, the underlying assumption is that people are less inclined to learn and develop skills unless these are directly applied or rewarded in their roles. In addition to the implicit mechanism of learning, our unit of analysis is not the employee but the job. The lack of granularity in our empirics makes it hard to identify whether learning goes together with the application of skills. In the future, conducting surveys of employees for detailed observations of individuals' skill endowments, rather than job surveys.  Finally, our data primarily describe the U.S. labor market, which has idiosyncrasies in its education system, industrial composition, and urban structure. 
How well these findings generalize to other work settings, such as entrepreneurship \cite{Murray2023}, and economies, especially those at different stages of development \cite{Autor2022}, remains a task for future research. 

In conclusion, our study introduces a novel approach to understanding the structure of skills in the labor market, shedding light on the pathways to specialization and the mechanisms driving skill value and resilience. While our study has limitations, it lays the groundwork for future research to explore the generalizability of our findings and investigate the relationships between skills, education, and socioeconomic outcomes.




 

%The nested structural categories also add a new dimension to the theory of human capital \cite{Lazear2009}. 
%By differentiating general human capital into a structured spectrum, from the most foundational and general to the most specialized, human capital is comparable at different organizational scales, which is essential for policy implications. More importantly, the nested structure reiterates the valuable insight that skills are not acquired in isolation. This perspective challenges the simplified view of acquiring general skills early in life, later supplemented by specialized skills. Life-long learning requires more than just acquiring new specialized skills; it necessitates complementing such investments with deepening one's general skills, a point echoed by recent recommendations from Stanford and Harvard bringing back math courses over applied alternatives like data science and statistics \cite{Feinstein2023}. 

% Empirical supports to the logic: 
% We would ideally observe the distinct processes of acquisition and application, yet the data reveals the result of both processes in occupational “requirement” of skills. We side with Reviewer 1 in that our temporal analyses of skill usage over individuals’ careers (the median age of occupations, occupational employment of birth cohorts as they age, and in resumes) in Fig. 3 offer evidence that the processes of acquisition and application are revealed in tandem. However, we concede that this is not a direct observation of acquisition. Our writing has been based on the understanding that given a rhetorical dependence of calculus on algebra the progression in the level at which one applies calculus implies increases in the level at which one must know algebra. We have explicated this intuition in the method section that the O*NET offers information about occupational “requirement” of skills, which implies the possession and the capacity for application of expertise at work, rather than learning of skills, alone. We have further revised the mentioned example, which appears early in the paper to elucidate the nature of dependencies we later derived empirically: “The prerequisites for understanding and applying calculus, for instance, are grounded in the knowledge of algebra and geometry.”


% Finally, just as skill acquisition at the individual level is embedded in a hierarchical web of prerequisite dependencies, so may capability acquisition at the macro level of economies such as firms, cities, and nations. This perspective bridges the micro-level of jobs and wages with the macro-level of the economic system as a whole. This intrinsic learning sequence suggests that economies may be unable to follow optimal growth trajectories \cite{FinkTMA2019Hmcw, Fink2017a, Fink2017b}. As a result, it uncovers structural disparity across different social groups that would have remained underappreciated had we relied solely on information on educational attainment or wages \cite{Autor2014}.  It also carries implications for organization and promotion, for example, explaining the failure of promotion based on performance, i.e., the Peter Principle \cite{Benson2019}. 


%%%% robust to management! %%%%
%%% mention the descriptor here and also sec 1. We identified six \textit{social skills} in the list. They are \textit{Social perceptiveness}, the skill of being aware of other's reactions and understanding why they react as they do; \textit{Coordination}, the skill to adjust actions in relation to others' actions; \textit{Persuasion}, the skill to persuade others to change their minds or behavior; \textit{Negotiation}, bringing others together and trying to reconcile differences; \textit{Instructing}, the skill to teach others how to do something; 
%and \textit{Service orientation}, actively looking for ways to help people.

% LT Zhang: A well-established concept in economics is general versus firm-specific skills. The former refers to skills that are useful across firms and the latter are those useful to only a particular firm. Our construct treats all skills on a continuum, ranging from the most foundational and general to the most nested and specialized. In our framework, firm-specific skills would include those that are extremely specialized skills, such that they only apply to a particular firm. 
% cite: Lazear, Edward P. "Firm-specific human capital: A skill-weights approach." Journal of political economy 117, no. 5 (2009): 914-940\cite{Lazear2009}
% The most interesting finding to me is perhaps the importance of combining foundational and nested/specialized skills, especially at the higher level. I believe this is an important point and resonates well with other work showing the increasing role of social skills in recent years (especially at a later stage of one's career). 
%Skilled workers represent an important class of capabilities, making understanding the process of skill acquisition, deepening, and diversification closely connected to the fundamental narrative of the literature on economic complexity.
 %The former is widely useful, while the latter raises productivity at only a particular firm, not elsewhere, thus setting up a bilateral monopoly situation between the worker and firm.
%Accumulation patterns differ across skill types: general and nested skills tend to grow over most of a worker’s career, while reliance on un-nested specific skills decreases with age. 


% As economic complexity grows, the nature and intensity of skill dependencies evolve correspondingly. 
% \cite{Borner2018}



%In general, job skill requirements must navigate a delicate balance. 
%They need to account for the required complexity, complementarity, and coordination among heterogeneous skills, while also considering the ease of skill acquisition by individuals. 


%One way education systems can achieve progress is by speeding up the acquisition of general skills. 
%The possibility of this will be clear to anyone who compares textbooks of basic calculus to the original texts on which they build. Simplifying the exposition in teaching basic skills will allow for the further deepening of these skills and therewith provide the basis for the nested specific skills that allow for a deepening division of labor.

% In order to produce the most complex products, national economies must accumulate a web of interdependent capabilities \cite{Hidalgo2009, Hausmann2011, Hidalgo2007}.
% There is a directionality of acquiring capabilities \cite{OClery2021, Imbs2003}
% Such capabilities allow "complex" economies to dominate the production of the most valuable products.
% Growing complexity and specialization similarly motivate examining the underlying structure of human skills \cite{DemingDavidJ2020EDCJ, Tong2021}.
% % [[Placeholder: we will analyze statistical significant link from co-occurrence data to extract interdependency out of hierarchy of abundance]]
% On the one hand, specialization has increased the depth of knowledge in each domain.
% % , expanding what we call \textit{vertical skill interdependence}.
% In other words, specialization has increased the length of a typical learning trajectory from basic to advanced topics \cite{BenJ2009}. % more effort before a novice can become an expert. % increasing the depth between the basic and the advanced topics.
% On the other hand, specialization has accompanied a growing complexity of economic tasks, requiring a diversity of skills and knowledge \cite{Hidalgo2021}.
% % Nonetheless, because job tasks often require several skills, workers must accumulate 'appropriate' bundles of skills, and as a result, horizontal skill interdependence has also intensified.
% That is, increasingly complex job tasks have necessitated the accumulation of 'appropriate' bundles of skills \cite{OClery2019, Hidalgo2018}.
% % , which we call \textit{horizontal skill interdependence}.
% % As a result, the presence of a diverse set of complementary skills, what we call \textit{horizontal skill interdependence}, has also intensified.


% In particular, we do not know if one actually acquires or advances general skills when they move up the occupation level (acquiring the more specific skills).
% The structure in Fig~\ref{fig:Figure 2} is constructed by cross-sectional conditional probability and thus does not guarantee longitudinal occurrences in individual acquisitions.

% Workers' limited attention, time, and resources force them to trade off skill diversity and specialty \cite{Leung2014}.
% [alternative paragraph (to be modified with the previous and the latter)] 
% Vitally, forces of diversity and specialty create an inherent trade-off in the choice of what skills to bundle \cite{Leung2014} which affects workers who must invest time and effort in skills, policy-makers interested in improving aggregate and individual welfare, and social and economic agents who desire an optimal and fair distribution of work.
% Knowledge and technological advances make it impossible for an individual to contain the depth of knowledge at once \cite{BenJ2009}, requiring more investment to reach advanced levels.
%On the one hand, a growing literature highlights the importance of bundling complementary skills \cite{AndersonKatharineA2017Snam, Alabdulkareem2018, Neffke2013, Balland2020}.
%On the other hand, however, the literature on specialization advocates the productivity gains of specialized skills \cite{Mincer1974, Lucas1988, Gibbons2004, smith1937wealth, Frank2018, BeckerG.S1992TDoL, Poletaev2008, Gathmann2010}.
%Further complicating matters is a stream of research that champions the importance of the less specific skills \cite{Hermo2022, Liu2013, epstein2019range}, citing the "Flynn effect" as a sign of the increasing value of more fluid skills over crystallized knowledge \cite{FlynnJamesR.2018IdaP}.
% In conclusion, uncovering the skill structure and articulating the relation between workers’ skills and value requires disentangling forces of complementarity and specialization.

% Existing work has discovered that performing job tasks simultaneously requires several (complementary) skills \cite{Alabdulkareem2018}, and that complementarity of skills, what we call \textit{horizontal interdependence}, increase the overall workers' value \cite{AndersonKatharineA2017Snam, Neffkeeaax3370}.
% Nonetheless, our understanding of complementarity has relied on skill's co-occurrences in occupations \cite{Alabdulkareem2018}, neglecting the asymmetry in skill dependence.
% Intuitively, the investment necessary, hence the payoff, to acquire a skill increases with the number of its prerequisites \cite{Mincer1974, Lucas1988, Becker1962, Schultz1961}.
% We call such dependency on prerequisites \textit{vertical interdependence}.



% The central goal of this paper was to uncover the underlying dependency structure among skills that are used on the US labor market. Our analyses of multiple data sources, we have empirically derived a classification of skills consistent with the notions of \textit{general} and \textit{specific} human capital \cite{Becker1962}. Moreover, we have constructed a skill hierarchy based on the conditional probabilities of skill co-occurrences across the economy. This hierarchy resembles an unbalanced tree, with general skills serving as common prerequisites for some of the more specific skills. The emergence of this skill tree stems from the semi-nested structure of occupational skills, which introduces heterogeneities in the web of dependencies across different skills. Our framework distinguishes skills such as \textit{Programming, medicine, and physics} that rely on a dense and nested web of prerequisites, from other skills like \textit{Repairing, stamina, and construction} that lack such a structure.
% Skill interdependence and nestedness also carry implications for aggregate economic growth.
% Growing complexity and specialization similarly motivate examining the underlying structure of human skills \cite{DemingDavidJ2020EDCJ, Wilmers2020, Tong2021}.
% Some knowledge requires a basis to travel from one human to another\cite{Coscia2020, VanderWouden2023}.
% While others may not need such a basis.
% While specialization creates challenges in cross-occupational communication and knowledge transfer \cite[ch. 3]{Arrow1974book}, a strong stock of general skills common among occupations, will reduce the cost of transmitting information, and as a result, collaboration and complementarity of workers of different specialties \cite[ch. 2]{Arrow1974book}.
% Our findings suggest investments in certain skills and knowledge, therefore, help further accumulate them as if \textit{capital} \cite{Becker1962, Schultz1961}.
% This is consistent with the empirically observed increases in the education premiums \cite{Janzen2022} and the role of general skills as inter-occupational complementarity \cite{Neffkeeaax3370}.
% In addition, an important part of accumulating human capital is specific to the firm \cite{BeckerG.S1992TDoL, Wilmers2020}, wherein specialized workers, among other things, learn about the firm's communication codes \cite{Arrow1974book}. 
% % Relatedly, the very basis that facilitates knowledge transfer may act as a communication channel that enables complementarity between workers endowed with the basis however different in their specialized skills \cite[ch. 2]{Arrow1974book}.
% On the one hand, growth in knowledge and know-how leads to longer chains of prerequisites within each specialty branch. 
% Suffering from a heavy knowledge burden on us \cite{BenJ2009}, we specialize to become more productive \cite{smith1937wealth, Wilmers2020} and able to make more advanced products \cite{Hidalgo2021}, increasing our bargaining power \cite{Davidson1898book, Fini2022, FergusonJohn-Paul2013}.
% %% cited work Wilmers2020, in fact, states the opposite that: certain specializations may reduce bargaining power.
% On the other hand, the products (and technologies) have become even more complex, branching off skills into many specialty prongs. Undertaking complex tasks that our modern society needs requires specialized skills, knowledge, and coordinated workforces \cite{Hidalgo2015}.  
% %Therefore, we expect the skill structure to extend both in-depth, as a tree gets taller, to represent the growth of collective knowledge, and in breadth, a branch of a tree would represent growing diversity. 



% \cite{VanDam2021} finds that in certain ecologies, two disjointed sub-ecosystems may form that share few required resources. This resembles our nested and un-nested structure. A linkage to the paper and the idea can attract people from that domain.]


% Our geographic and demographic analyses further corroborate the dependence of valuable nested skills on general skills and highlight the unequal distribution of general skills across localities and racial groups. These disparities contribute to the widening wealth distribution \cite{Hong2020} and increased rigidity of social classes \cite{Goldin2008}. Regiones and sub-populations with bundles of general and nested skills are better positioned to engage in complex activities and achieve higher economic outcomes. However, such skill bundling may create barriers to knowledge transfer in regions and demographic groups, lacking the necessary foundation for acquiring nested skills. As economic well-being becomes an increasingly important factor in education \cite{Chau2023, Li2021}, the existing distribution of skills may shape reinforcing factors in the possibility of reskilling and future wealth distribution. Therefore, policymakers must recognize the overlap between factors that create value and those that contribute to the widening gap between socioeconomic strata. %Our findings have implications for planners and policymakers seeking to mitigate the impacts of technological job changes. 


% %% complexity %%
% Complexity and specialization have been the two sides of the coin of economic growth.
% These dual forces have allowed the organization of work and distribution of knowledge so that the collective can produce more diverse and complex products and services \cite{Hidalgo2021, Hidalgo2015, Harmand2015, Hausmann2011, Hidalgo2007, Hidalgo2009}.
% Society has become so complex that not a single person can make a pencil \cite{Read1958, Dubner2016}.  
% However, they have also coincided with structural deficiencies in the form of increasingly unfair wealth distribution.
% While specialization has increased the length of a typical learning trajectory from basic to advanced topics \cite{BenJ2009}, the growing complexity of job tasks has necessitated the accumulation of more skills \cite{OClery2019, Hidalgo2018}, adding to the burden of bundling skills.
% Because workers possess a limited time budget and cognitive capacity, they must trade off specialization and diversification.

% Our skill structure bridges the findings of a recent body of work on economic complexity that represents skill structure as networks \cite{AndersonKatharineA2017Snam, Alabdulkareem2018, Xu2021, Lin2022, Althobaiti2022, Neffkeeaax3370, DelRio-Chanona2021, Balland2020} and the economic stream on human capital \cite{Mincer1974, Lucas1988, Gibbons2004, smith1937wealth, M.R.2018, BeckerG.S1992TDoL, Poletaev2008, Gathmann2010}.
% % , and the pyschological literature on intelligence \cite{} about the value of skills
% While the economic complexity stream focuses on the diversity and complementarity of skills, labor economic research emphasizes the directionality of dependence that arises from specialization.
% % In contrast to the latter, the psychological literature advocates the importance of more common components of human skills.
% Our work considers skills such as \textit{Programming, medicine, and physics} are bundled along the depth of the skill hierarchy.
% These are nested skills as they cannot be obtained without sufficient stock of their prerequisites.
% The fact that they come in together in the career journey (i.e., resumes) implies they are bundled to accomplish complex job tasks.
% Indeed, we find that the jobs that require nested skills obtain the highest economic return.
% As such, drawing from both mentioned streams, we argue jobs with nested skills carry both vertical (specialization) and horizontal (complementarity) skill interdependence.
% In contrast, jobs that mainly rely on un-nested skills lack depth, and as such, do not command high wages.


% Similarly, the skill structure has implications for science and technology regarding how science is needed for technological progress \cite{BenJ2009, Sorenson2004}.
% Put differently, there is directionality among components of the analogies such as  Lego blocks or Alphabets in the Scrabble \cite{FinkTMA2019Hmcw} for the economic building blocks.
% % analogies are not independent and equal.
% % There is a directionality among them.
% We may have to learn 'A' before learning 'Z' to solve the analog of a Scrabble game in economic growth.
% This is consistent with prior findings that national economies that can produce the most complex products also produce simpler products, while the reverse is rarely observed \cite{Hidalgo2007}.
% Furthermore, this intrinsic learning sequence may imply agents may not be able to adapt optimal growth trajectory even if they wanted \cite{FinkTMA2019Hmcw, Fink2017a, Fink2017b}.
% The nature of nestedness will introduce a nontrivial diffusion process in product space \cite{Hidalgo2009}. Knowledge cannot travel without prerequisites \cite{Balland2020, Moro2021, AndersonKatharineA2017Snam, Alabdulkareem2018, Li2021, Tong2021}. % (some knowledge should be bundled to travel, not alone). \\
% That is why in order to produce the most complex products, national economies must accumulate a web of interdependent capabilities \cite{Hidalgo2009, Hausmann2011, Hidalgo2007}.
% Such directionality in acquiring capabilities \cite{OClery2021, Imbs2003} allows "complex" economies to dominate the production of the most valuable products.






% %% limitation %%%
% Of course, our study has limitations that need to be considered. Our empirical findings may not provide a direct answer to why some skills bundle in an occupation or why some skills are needed together in an occupation [a good place to include bargaining and other micro literature on skill acquisition]. The observed skills bundled together many times in a single occupation must be the most productive recipe for economic activities under constraints. The constraints include transaction costs among the associated complementary tasks that trade-off with a single human's limited capability, such as limited attention, time, and resources \cite{Leung2014}. There is only so much we can learn within a working cycle, no matter how high the desired bundle's payoff would be. We have known that when the skills and knowledge are too complex to be bundled within a person, we form teams or firms to \textit{organize} (tacit and non-tacit) knowledge needed to accomplish tasks \cite{Ahmadpoor2019, Wu2019Science, Milojevic2014, Boerner2010, Hunter2008, Wuchty2007}. Our work shows that some types of knowledge are more similar to each other to be learned easily, such as English and German, compared to English and Korean. 

% The structure we find is incomplete and does not isolate the pre-requisition dependence of skills from other possible mechanisms. Nevertheless, we found and presented evidence that the hierarchy is a good first approximation to the hidden fundamental structure in the landscape of human capital.
% ** We should discuss the caveat that our claim that \textit{specific skills require general skills} comes with a lack of micro-level evidence of temporal acquisition of skills: do general skills come before specific skills? Ideally, we can observe this as individuals acquire skills, but we can approximate it with the survey. [include in survey LT is preparing]
% And yet our framework holds.

% % Our findings carry implications for a broad range of fields, from psychology to policy. 
% %We find that the general skills of occupations carry significant predictive power in determining wages. 
% The importance of general skills in enabling the execution of complex tasks is a factor often overlooked. Particularly, many retraining programs are designed around instilling ``valuable'' specific skills. Our results suggest stocks of specific skills on their own are insufficient to create sustained value, and thus the general skills of occupations carry significant predictive power in determining wages. Instead, a longer-term approach to facilitating education availability for acquiring general 
% and nested skills may be required. 
% We provide robust empirical evidence for the growth of general skills in individuals' careers through resume data, the increased importance and level of general skills over time when comparing occupational skills from 2005 to 2019, and an analysis of a synthetic age cohort, which reveals that occupations with a higher median worker age exhibit higher general skills. 
% The observed increase in general skills aligns with the "Flynn Effect" \cite{Hermo2022}, which describes the rise in fluid reasoning abilities in developed countries since the twentieth century. 
% Indeed, there has been a growing recognition of general skills, including education. For instance, in 2022 Stanford, and in 2023 Harvard updated their recommended preparations for their prospective applicants, emphasizing conceptual math courses such as algebra and geometry and de-emphasizing applied alternatives like data science and statistics with the reasoning that these courses are ``not equal'' in how well they prepare individuals to progress in different specialties \cite{Feinstein2023}.
% As such, we also offer an alternative explanation for the empirically observed stagnant payoff to education \cite{Pritchett2001, KruegerAlanB2001EfGW, HanushekEricA2008Troc} based on the necessary co-development of general and nested skills.


% To summarise, we have formed a topology of the skill structure, wherein skills' generality measured by total level across occupations is plotted against two different network measures of prestige (Hub score \cite{} and Pagerank \cite{} on an extracted directional backbone \cite{Jo2020}) and skill's correspondence with general skills as measured by the average Pearson correlation of the focal skills and general skills.


% Put differently, Lego Block and Alphabets in the Scrabble analogy are not independent and equal. There is a directionality among them. We may have to learn 'A' before learning 'Z'. And this intrinsic learning sequence may imply why we cannot adapt the optimum strategy even if we want \cite{FinkTMA2019Hmcw, Fink2017a, Fink2017b}. 



% It may have implications for science and technology in terms of how science is needed for technological progress (ref Ben Johns, US. Patent and Science papers). 

% 0. Implication in product space/knowledge diffusion. The nature of nestedness will introduce a nontrivial diffusion process in product space \cite{Hidalgo2009}. Knowledge cannot travel without prerequisites. % (some knowledge should be bundled to travel, not alone). \\
% Some knowledge requires a basis to travel from one human to another.
% While others may not need such a basis.
% Relatedly, the very basis that facilitates knowledge transfer may act as a communication channel that enables complementarity between workers endowed with the basis however different in their specialized skills \cite[ch. 2]{Arrow1974book}.

% 1. There are skills that are bundled in Fig \ref{fig:skills cor wage, total generality, cor gen skills} those are nested skills. The fact that they come in together in the journey of career (resume) implies they are bundled to accomplish a complex job. We find that these jobs that are bundled together give the highest economic return. This is perhaps how human species has become successful (?)\\
% 1. There are skills that are bundled along the depth of the skill hierarchy. These are nested skills as they cannot be obtained without sufficient stock of their prerequisites. The fact that they come in together in the career journey (i.e., resumes) implies they are bundled to accomplish a complex job. We find that the jobs that require nested skills give the highest economic return. %This is perhaps how human species has become successful (?)\\
% What does this mean for economic growth and equality?

% 2. Our geographic and demographic analyses show disparities in the distribution of such general skills.
% As a result, on the one hand, locality and sub-population endowed with bundles of nested skills may be in a better position to undertake complex activities and gain higher economic outcomes.
% On the other hand, such bundling may forge barriers to transferring such valuable knowledge to geographies and demographic strata that lack the required basis to obtain nested skills.
% As economic well-being becomes an increasingly important factor in education \cite{}, the existing distribution of skills may shape reinforcing factors in the possibility of reskilling and the future distribution of wealth.

% 3. potential stretch to science and technology
% Even technological advances fueled by basic science 
% \cite{Sorenson2004} %(Sorenson and Felming 2004)

% 4. Can we explain the "Flynn Effect?" Increases in fluid reasoning appear to emphasize our general skills. Find relevant info in \cite{Hermo2022}



%%
%%----------------------------------------------------------------------------%%
% https://www.bls.gov/oes/
% O*NET also includes work activity items we disregard as they associate more closely with work tasks than worker skills \cite{Nedelkoska2018}.  

\section*{Data and Methods} \label{sec: method}

%%--------------------------------------------
\subsection*{Datasets}
\textbf{Occupational Information Network (O*NET) }
includes survey records of job-oriented attributes and worker-oriented descriptors conducted by The Bureau of Labor Statistics (BLS) \cite{Peterson1999ONET}. Job-oriented attributes include educational requirements, workplace experience, and training. Worker-oriented descriptors include 120 work-relevant knowledge, abilities, and skills (labeled \textit{skills} throughout the text for brevity).
Each occupation includes a list of skills with their sophistication levels (or intensity) and the importance of those requirements, each resulting in an occupation-skill matrix. 
Our main analysis uses the level, but the other variable is highly corrected (0.94), and therefore, our findings are robust to the choice of measurements.
We have obtained two versions: 2019, to avoid concerns over contaminating data with signals from the COVID-19 pandemic, and 2005, the first version with a consistent skill topology and available education covering a significant number of occupations.

\noindent \textbf{Occupational Employment and Wage Statistics (OEWS)}
offers wages and employment information at different granularity levels (nation-wide, region-specific, and industry-specific).
We have used nationwide, region-specific data for 2005 and 2019 and combined them with their respective year from O*NET.
Note that including and aggregating data from several years before and after 2005 and 2019 does not change our results.
In the resulting combined data, occupational units were aggregated at the 6-digit SOC codes (OEWS is available at the 6-digit level, while O*NET is available at the 8-digit SOC level).


\noindent \textbf{Current Population Survey (CPS)}
is a monthly survey of households conducted by the Bureau of Census for the Bureau of Labor Statistics.
It offers a representative sample of the population obtained in each round that offers statistics on various aspects of the labor force \cite{Flood2022}.
From the Labor Force Statistics component of CPS, we obtain the median age of workers in occupations for 2019.
From the CPS microdata, we acquire employment and demographic information on households between 1980 and 2020, including occupation, wage, hours worked, gender, and race/ethnicity information.
Matching with SOC occupational units requires a crosswalk described in the corresponding section.


\noindent \textbf{Burning Glass Resume Data }
includes 70 million job sequences (8-digit SOC) documented in 20 million individuals' resumes between 2007 and 2020 from Burning Glass (also known as Lightcast).
Burning Glass applied AI tools to submitted resumes, digitizing their text and mapping them to occupational titles consistent with BLS SOC codes, allowing for easy integration with O*NET data.

%O*NET consists of job-oriented attributes and worker-oriented descriptors, respectively. Job-oriented attributes include wage, employment, educational requirements, workplace experience, and training. Worker-oriented descriptors include 120 work-relevant knowledge, abilities, and skills (labeled \textit{skills} throughout the text for brevity)

%Occupation units for both datasets follow the Standard Occupational Classification (SOC), the federal statistical standard used by federal agencies to classify workers into occupational categories. However, their granularity differs by the dataset. While BLS categorizes occupations by 6-digit SOC codes (774 unique occupations), the O*NET detailed them into 8-digit SOC codes (968 unique occupations).

%There are two ways to measure skill levels required for effective work performance: the level of sophistication (or intensity) (ranging from 0 to 7) and the importance or vitality of those requirements (ranging from 1 to 5).
%These two skill measures are highly correlated (0.94); thus, the results are robust to the choice of measurements.
%We chose the skill level for the main text because we reason the level that captures the necessary sophistication and adeptness is more consistent with our hypothesis of skill progression and hierarchy.
%Still, the correlation and the fundamental classification of skills are robust to using importance measures— see supplementary information.

\subsubsection*{Skill Generality Groups} 
%%---------------Occupational-specific information is obtained from two datasets: the Bureau of Labor Statistics (BLS) and the Occupational Information Network (O*NET) \cite{Peterson1999ONET}.
%%% For example, the skill of "speaking" is important for both lawyers and paralegals. However, lawyers (who frequently argue cases before judges and juries) are required to have a higher Level of speaking skill, while paralegals only need an average Level of this skill. https://www.onetonline.org/help/online/scales#foot2
%%%%


For each skill, O*NET reports the required levels needed for workers of each occupation to perform their tasks. We call the distribution of the number of occupations that require skill at varying levels the \textit{level distribution}.
The shape of a skill's level distribution illustrates its generality across occupations, shown in Fig.~\ref{fig:Figure 1} (a). 
As such, we group skills by their similar distribution shapes by $k$-mean clustering algorithms with correlation metrics. Figure \ref{fig:Figure 1} (b) shows the characteristic shapes of each skill group.  
We provide three statistical tests for optimal $k$ and show the findings are qualitatively robust to some variations (see SI Sec.~\ref{supsec:skill clustering}). Throughout analyses, we mainly analyze the effects of general and specific skills to filter possible noises.

This group is consistent with the local reaching centrality measure, which was used to embed nodes vertically in Fig.~\ref{fig:Figure 2} (b). 
The local reaching centrality is defined as the proportion of the skill hierarchy structure that is reachable from a skill via outgoing edges \cite{Mones2012}. The higher reaching centrality in the hierarchy structure is, therefore, the more interdependent skills. As such, this measure offers additional indicators of skill generality. 
% LRC is calculated by networks global_reaching_centrality with an option for local.


%%--------------------------------------------
\subsubsection*{Conditional Probabilities for Skill Hierarchy Structure}


The conditional probability that infers the directionality operates on binary values, but skill levels are recorded in continuous variables [0,7], which makes it hard to apply the conditional probability method. We use the disparity filter to extract a statistically significant presence/absence in an occupation-skill matrix \cite{Serrano6483}.
Parameters are chosen such that i) the rank of skill terms in the strength (from the weighted network) and degree (in the binary network) is preserved, ii) the rank of occupations' skills of each category in the weighted network is preserved in the binary network.
Supplementary Information Section \ref{supsec:skill-occ} discusses details and compares the state of data before and after the transformation.

We then calculate conditional probabilities of every pair of skills in the transformed (binary) matrix to infer dependence and directions between two skills, following \cite{Jo2020}.
%% The expanded description of the method for producing the skill hierarchy (comment 1.2.)
We first account for the significance of conditional appearances, subject to a threshold, $z_{th}$.
Here, $z_{th}$ is a threshold for the extent to which we eliminate chance from two skills being used in the same occupation.
Given the significant skill pair conditional appearances, we estimate conditional probabilities $P(u|v)$ and $P(v|u)$. 
The direction of dependence $v \rightarrow u$ is set when $P(u|v)$ is \textit{substantially} greater than $P(v|u)$, subject to a parameter $\alpha_{th}$, which is differentially weighted for each pair of skills so that it accounts for heterogeneous skill node degrees (see Eq.~\ref{eq: a_th} in SI Section \ref{supsec:conditional dependencies}).
The magnitude of the dependence is a parametric function of the difference between the conditional probabilities of observing $u$ and $v$, and the null model that accounts for the estimated number of shared occupations between them, given the degrees of $u$ and $v$.as shown by Eq.~\ref{eq: dependency weight} in SI Section \ref{supsec:conditional dependencies}.
Figure \ref{fig:Figure 1} (d) broadly illustrates the intuition behind this methodology.
Figure \ref{fig:Figure 2} (a) presents a backbone structure of the aggregated all skill pairs, where the edge weights follow the magnitudes of pairwise dependencies, as described above.
Figure \ref{fig:Figure 2} (b) offers the full network.
Please see SI Section \ref{supsec:conditional dependencies} and \cite{Jo2020} for the detailed procedures and choices of parameters and thresholds. 

% First, we use a disparity filter to make a binary network of only statistically significant edges relative to randomness.
% Second, the direction and strength of the dependency using conditional probabilities $P(u|v)$ and $P(v|u)$
% for $P(u|v)<P(v|u)$ and directed edge $u \rightarrow v$ with a weight $\alpha(u,v) $.
% We exercised caution in interpreting the outcome of this step as $v$ being a dependent of $u$.
%\begin{equation} \label{eq: LRC}
    %C_R(u) = \frac{1}{N-1}\Sigma_{v:0<d^{out}(u,v)<\infty}(\frac{\Sigma_{k=1}^{d^{out}(u,v)}\alpha^{(k)}(u,v)}{d^{out}(u,v)})
%\end{equation}
%Where $N$ is the number of nodes in the network, $d^{out}(u,v)$ is the length of the directed path that goes from $u$ to $v$ via out-going edges, and $\alpha^{(k)}(u,v)$ is the weight on the $k^{th}$ edge along that path, as it is derived in the skill hierarchy.
% To calculate the number of paths of length $l$, we employ the weighted-directed adjacency matrix, $M$, and raise it to the power of $l$, yielding $M^l_{i,j}$.


\subsubsection*{Reachability with Arrival Probability} %--
To quantify what are the chances of getting to the focal skill $j$ given the pre-requisite skill $i$, we calculate reachability from one skill to a focal skill. It is basically arrival probability, or a version of hitting probability, of a random walk \textit{arriving} at $j$ from node $i$ given 
the weighted skill dependency network \cite{norris1998markov}.
For source and target skills $i \neq j$, this is numerically equivalent to first deriving the probability of random walks of length $l$ by raising the weighted-directed adjacency matrix (skill dependency network in Fig.~\ref{fig:Figure 2}), $M$, to power $l$, and then calculating  $R_{i,j} = \Sigma_l M^l_{i,j}$.
We obtain the final arrival probability by summing over a sufficient number of path lengths until reaching saturation points. To compute arrival probabilities for focal skills (such as Programming, Negotiation, and Repairing) in Fig.~\ref{fig:Figure 2} (b-f), we apply the R package \textit{markovchain} \cite{MarkovchainRPackage}.

% Note that: The package uses skill-occ, not occ-skill matrix (where skills and occupations correspond to species and sites in ecology). Note that the skill-occ matrix is the matrix after the disparity filter because the package needs binary entries. 
% Nestedness: Coordination, AdministrativNODFe & Management, Social Perceptiveness, Service Orientation.
% Given the form $c_s$ takes, we use the threshold of 0. Therefore, we designate skills with nestedness contribution beyond the threshold as \textit{nested} and others \textit{un-nested}.
% We use checkerboard score as our primary measure, but NODF also gives good agreement.

%\subsubsection*{Nested Structure in Skill Hierarchy} 
\subsubsection*{Nested and Unnested Skill Categories} 
Nestedness is a structural characteristic that describes interactions in an ecological system, where specialist species often interact with a subset of generalists. 
Unlike ecological systems, however, SI-Fig.~\ref{fig:occ_skill_nestedness_mat} shows the skill-occupation matrix is a noisy nested structure far from the perfect upper-left triangle when sorted by marginal totals (fills).
This imperfect nested structure may account for the constraints on occupations (limited carrying capacity), introducing severe competition between skill species. Indeed, SI-Fig.~\ref{fig:occ_vs_skill_importance_avg_cos} shows, unlike broad skill generality, the occupation's scope is narrowly distributed, indicating 
that the total amount of skill levels embodied in an occupation is not much different from each other, regardless of how much they are paid and how advanced education is needed (see SI Sec.~\ref{suppsec:nestedness}).


We attribute occupations' limited scope of skills to the limited attention and cognition/physiological capacity that individual workers can offer. There is only so much a single person can equip and do for a single job \cite{BenJ2009, DUNBAR1992}. Thus, individuals' capacity restricts how many skills an occupation can bundle. This constraint explains the process of specializations needed for a complex job. The structure now includes not only nested structure but also mutually exclusive presences, possibly due to competition between skills within an occupation. 
In contrast to occupations, skills do not have such constraints. Therefore, for limited occupation scope, we only consider the skills' contribution to nested structure.  

 
This constraint distinguishes the nestedness of extensive economies such as nations, regions, and urban areas from the nestedness of occupations in that specializations dominate the evolution of the labor market while others are dominated by diversification. 
As a result, the skill-occupation matrix is expected to be modular as well as nested with mutually exclusive modules. \textit{Nested-modular matrix} is a complicated structure and will be beyond our current scope \cite{Fortuna2010, VanDam2021}. Here, we will focus on individual skills' contributions to the nested structure and differentiate skills that contribute to the nested structure from those that do not. 

Therefore, we quantify a skill's contribution to the nested structure, i.e., nested score, $c_s$, defined as a deviation from a null model where the edges of a focal node $s$ to occupations are randomly reassigned, that is, $c_s = (N - <N^{\ast}_s> ) / {\sigma_{N^{\ast}_s}}$.
$N$ is a nestedness score, and $<N^{\ast}_s>$ and $\sigma_{N^{\ast}_s}$ are the means and standard deviation derived from the null model \cite{Saavedra2011}. For each focal skill $s$, we run 5,000 iterations \cite{SergeiMaslov2002}. We employ the overlap index, checkerboard score, Temperature, and NODF, nestedness scores commonly used in ecology, to quantify nestedness $N$ \cite{write1992, Stone1990, Atmar1993, Almeida-neto2008}.
In addition, we only consider skill's contribution and do not occupation's contribution.
To obtain discrete categorizes, any non-general skill with $c_s>0$ is called ''nested'' skills, and ''un-nested'' otherwise. The resulting skill categories are shown in Fig.~\ref{fig:nestedness}.
The detailed allocation of skills to these categories are outlined in SI Table \ref{tab:skill_split_C_nestedness}, and SI Sec.~\ref{suppsec:nestedness} offers details and robustness checks.

% If you mean highest in terms of nested specific skills, here are the top 5 in 2019:
%1. Physical Medicine Rehab Physicians
%2. Anthropology and Archeology Teachers (postsecondary(
%3. Biomedical Engineerings
%4. Archeologists
%5. Surgeons

%We then calculate an occupation's average nested skill, for example, as $\Sigma_{s \in nested} \; m_{o,s}$ divided by the number of nested skills where $m_{o,s}$ is the occupation-skill matrix that was used above. 
%SI Section \ref{supsec: add - returns to skill} includes additional analysis using the average of occupations' top five skills, instead of the entire skills within each category, in all analyses and found consistent results. 



% M above is a hierarchical tree where m_o,s is occupation-skill matrix. 
%Education, workplace experience, and training are available at the level of 8-digit SOC. However, wage and employment are available at 6-digit SOC codes. Therefore, to link detailed worker attributes with the former job attributes, we match BLS and O*NET data using 8-digit SOC.
%We match the two datasets by aggregating sub-occupations into the 6-digit codes to link wages, education, and employment with detailed worker attributes. 
%From O*NET, we also obtain educational requirements for occupations— at 8-digit SOC codes.
%\subsubsection*{Education and Wage Information}
\subsubsection*{Educations}
Education variables in O*NET are categorized into twelve discrete grades, ranging from below high school (1) to post-doctorate (12). 
Each occupation includes the proportion to which corresponding sampled employees had to have a given educational level to be hired. 
With this information, we calculated an occupation's associated education variable as a weighted average of the employees. 
For instance, Chief Executives' expected education variable $<edu>_o$ is calculated as $\Sigma_e f_e \cdot edu_e $ where $f_e$ is a fraction of CEO whose education is $e$, and $edu_e$ is a corresponding value of education category, ranging 1 for below high school to 12 for post-doctorate.
For an educational requirement to a skill $s$, $<edu>_s$, we average the skill's education levels of occupations, $<edu>_o$, weighted by the level of skill, $\text{Level}$, that is $\frac{\Sigma_o \; <edu>_o \cdot \: \text{Level}_{o,s} }{\Sigma_o \text{Level}_{e,o}}$.

%We obtain annual wages in the years 2019 and 2005 from the BLS. However, unlik education wage information is available at the level of 6-digit SOC codes.
%Therefore, all wage analysis use annual wage information at the level of 6-digit (663 items with available wage and skill information in our 2019 sample) occupations and aggregate skills by averaging over the corresponding 8-digit occupations for which we have skill information (789 items with wage and skill information).
%For instance, Chief Executives (SOC: 11-1011) corresponds to 8-digit occupations: Chief Executives (SOC: 11-1011.00) and Chief Sustainability Officers (SOC: 11-1011.03). For all wage analysis, the skill information of Chief Executives and Chief Sustainability Officers are averaged for Chief Executives (SOC: 11-1011).


% \footnote{\tiny\url{https://www.bls.gov/cps/demographics.htm#age}}f
% BLS crosswalk https://www.census.gov/topics/employment/industry-occupation/guidance/code-lists.html
% (mapping 968 to 542 occupation codes) Out of the unmatched 628 O*NET occupations, we link 55 more occupations to their CPS counterpart using text analysis and matching of occupations' titles in O*NET and CPS. Our results are robust to the presence or absence of the latter 55 occupations. See supplementary information for more detail. # we no longer do this extra matching.
%  We aggregate occupation skills for each skill category (\textit{general, nested intermediate and specific, and un-nested intermediate and specific}). Taking an average for each US county using the county employment of occupations as weights, we derive a regional measure of skill endowment for each skill sub-type.
% For each demographic subgroup (e.g., Asian male workers), we derive a skill endowment by linking their occupation as coded in CPS to their corresponding skills in O*NET 2019.
% CPS also records education and income wages. We adjust wages for inflation, and account for the number of hours worked, also recorded by CPS, by computing an adjusted weekly wage that is more readily comparable across the population.
% Among demographic variables, Whites, Blacks, and Asians, which constitute the bulk of the sample. CPS data also contains a separate (from race) variable for identifying Hispanic individuals. 
%occupational employment for Metropolitan and nonmetropolitan areas\footnote{\tiny\url{https://www.bls.gov/oes/}} published by the Bureau of Labor Statistics (BLS) at the level of 6-digit SOC occupations.
% The geographical units in the data are Core-based Statistical Areas (CSAs).
%US counties follow the Federal Information Processing System (FIPS) taxonomy.
%To obtain employment at the level of FIPS and map skill information onto US counties, we used a crosswalk also provided by BLS\footnote{\tiny\url{https://www.bls.gov/oes/current/msa_def.htm}}.
% The data also contains the date range spent on corresponding jobs in most cases.
% We extract the start and end date from the job range field using a Python 3 package called \textit{datesparser} version 1.1.4— we preprocessed the field to improve the performance of the package.
% We omit any record without a properly extracted start and end date.
% Sorting occupations that appear in each resume based on the start and end dates allows us to form a career trajectory for each resume.
% We omit problematic job moves, including job changes where the source and target occupations are the same (i.e., moving from one company to another without changing the occupation), jobs shorter than a year, and multiple jobs (more than one job at the same time) (i.e., someone is a teacher and a technician at the same time. This is problematic for our calculation. We keep the job at which the person started first, or stayed longer).
% The decision to remove such occupations arises from the oddity we observed in most such jobs. For instance, various janitors or models became a CEO immediately or with overlapping periods.
% That leaves us with over 10 million job moves and over 5 million unique resumes.
\subsubsection*{Demographic Distribution of Skills}

Median ages of workers in each occupation are derived from the Current Population Survey (CPS) of the year 2019, and synthetic birth cohorts from individuals born in each year are created from the individuals' survey conducted jointly by the U.S. Census Bureau and the Bureau of Labor Statistics \cite{Flood2022}. Different occupational taxonomies between the two datasets are mapped by the BLS crosswalk.

\noindent\textbf{Synthetic birth cohorts}:
The Current Population Survey (CPS) conducts monthly surveys to obtain a representative sample of the population in each round \cite{Flood2022}.  However, this longitudinal survey does not span over a long period of time, which presents a challenge when attempting to analyze long-term trends. To address this issue, we employ the concept of synthetic cohorts. Synthetic cohorts are constructed by stitching together snapshots of individuals born in the same year across different survey rounds. For example, to create a synthetic cohort for those born in 1970, we first identify people whose birth year was 1970 in the CPS surveys conducted in 1995, 1996, 1997, and so on, up to 2015. We then plot the data for this cohort as if we have been following the individuals born in 1970 throughout their ages, as shown in the inset of Figure 4.

It is important to note that this cohort is referred to as a ``synthetic birth cohort'' because it is not a real cohort in the traditional sense. The individuals surveyed by CPS in each round are different, even though they were all born in the same year. By following individuals born in the same year across multiple survey rounds, we can track changes in the behaviors or characteristics of interest as people age, albeit with different individuals representing the cohort at each point in time.

While synthetic cohorts do not provide the same level of individual-level consistency as true longitudinal studies, they offer a valuable tool for analyzing long-term trends and changes within a specific age group when long-running longitudinal data is not available. This approach allows researchers to leverage the representative nature of the CPS surveys to gain insights into the evolution of various social, economic, and demographic characteristics over time, and thus a common practice across various literature \cite{Acemoglu2011, KambourovGueorgui2013ACNO, Hermo2022, Aeppli2022}.

\noindent\textbf{Demographic analysis}: CPS microdata also include gender and race/ethnicity demographic information. 
We chose four categories, Whites, Blacks, Asians, and Hispanic, as they are the bulk of the sample, and any individuals of Hispanic background are included in that category for Fig.~\ref{fig:Figure 7}. 
To avoid attrition and early retirement, we include only full-time workers employed at the time of the survey, earning at least \$10,000 annually, and between 18 and 55. 
For each demographic category, the average skill level is calculated for their occupational composition.
The microdata records individuals' wages and the number of hours worked. 
We adjust wages for inflation and account for the number of hours worked to compute an adjusted weekly wage, which is readily comparable across the population. 
The race/ethnic disparities in Fig.~\ref{fig:Figure 7} are a ratio of each demographic quantity (general level, nested level, unnested levels, education, and weekly wages) to those of White workers, following \cite{Tong2021} identifying a dominant social group, a social group if it is at least 1.5 times more likely to be employed in the focal occupation. 
Likewise, the gender gap within each race/ethnicity is measured as a ratio of those quantities to those of male workers.
Because we do not have a matched sample, we obtain 95\% confidence intervals by random sub-sampling. In each iteration, we take 10\% of the subpopulation of interest, for instance, Asian male and Asian female workers, and estimate all corresponding measures. 
Repeating this sampling and estimation process in 10,000 iterations, we obtain the distribution for each estimation and derive the 95\% confidential intervals.
The skill, education, and wage estimations of Fig.~\ref{fig:Figure 7} average over the years. Supplementary Figs.~\ref{fig:Temporal Race Gaps - Skills, Education, Wages} and \ref{fig:Temporal Gender Gaps - Skills, Education, Wages} capture temporal patterns of these factors, exhibiting the gaps have narrowed over time.
In addition, SI Figs.~\ref{fig:Skill Age Gender Race Trends} and \ref{fig:Skill Age Gender Race Trends - year effects} show the skill differentials between male and female workers that start around the age of 30 (main Fig.~\ref{fig:age}), manifest across racial and ethnic groups.


%% Geographic Data ---------------------------
% \textbf{Geographic Analysis}: To construct skill maps in Fig. 6, we calculate each category's skill levels in the U.S. county. 
% The BLS provides occupational compositions for each county, from which the average skill level is calculated. 
% We then calculate the national average and the standard deviation for each skill category to derive a standard score (also known as z-score). 
% For Fig. \ref{fig:Geography}~(d), we group cities (core-based statistical areas) by populations [$<$ 10 thousand, $<$ 50 thousand, $<$ 1 million, and $>$ one million] based on the 2010 Census population estimates.
%Figure \ref{fig:Geography}~(e), we group cities by the intensity of their manufacturing industries, using the U.S. Census County Business Patterns in 2019. 
% At the 2-digit NAICS codes, we take 31-33 as manufacturing industries and calculate the location quotient of manufacturing employment (the ratio of manufacturing employment from the metro area total employment over the nationwide ratio).
% Matching metro areas to counties, we designate counties with no manufacturing employment to group `None', and group the rest based on quotient 33\% and 66\% quantiles of the measure into bottom, middle, and top. 



\subsubsection*{Skill Compositions in Career Trajectories}
The expected skill levels of each category in the career sequences. 
We studied over 70 million job sequences (8-digit SOC) in 20 million individual resumes from Burning Glass Institute between 2007 and 2020.  
We then calculate the expected skill levels in $i$th job by averaging the skill levels of those occupations appearing in $i$th sequences, shown in Fig \ref{fig:age} (g-h). 
From these sequences of averaged skill levels, we calculate skill level changes in $i$th job transition levels, $\Delta_i$, shown in Fig. \ref{fig:age} (i).  

We exclude job transitions shorter than one year or within an occupation (i.e., moving from one company to another without changing the occupation) for our primary analyses. 
The decision to remove such occupations arises from the oddity we observed in most such jobs. For instance, various janitors or models became CEOs immediately or with overlapping periods.
Nevertheless, our findings are robust to this decision (see SI Sec \ref{supsec: skill dependencies and age} for details). 

To see if the observed trends are truly attributed to career trajectories, we shuffle job history in resumes, bootstrapping the job sequences to produce a benchmark and compare it with the skill changes we empirically observed in career moves in Fig. \ref{fig:age} (i), confirming that the empirically observed trends are unique to the career trajectories.  



\subsubsection*{Temporal Evolution of Skill Structure} 


We utilize this evolution of skill structure to demonstrate the implication of our constructed nestedness skill structure. 
We choose two sufficiently apart datasets to capture the structural difference, that is, 
version 9.0 in 2005 because it is the first version comparable to the most recent version while offering satisfactory coverage of occupational information (such as education and wage), and version 24.1 in 2019 because it is the most recent version without the potential contamination of irregular patterns due to the pandemic. 
The empirical challenge is that the classification system is continuously updated in response to technological progress, economic transformation, and social reconfiguration \cite{Park2020}. 


%2005 O*NET complies with \textit{O*NET SOC 2000}, while 2019 O*NET relies on \textit{O*NET SOC 2010}, with two other taxonomy changes in between— in 2006 and 2009.
%Therefore, identically encoded occupations may not be comparable across these two years, and matching them requires a crosswalk.
%While O*NET reports crosswalks between each consecutive taxonomy, a direct crosswalk does not exist between 2005 and 2019.
We created a crosswalk between occupation classifications in 2005 and 2019 that is not immediately available but only between two consecutive years.  
Occupation codes in 2005 are matched to those in 2006, and then those in 2006 to 2009, ... to 2019. Our crosswalk automatically matches 968 occupations in 2019 skill data and 941 unique occupations in 2005 skill data, and the rest are manually matched.
Using these occupations and their skill levels in 2005, we construct the skill structure of 2005 in Fig. ~\ref{fig:historical skill change} (c), using comparable parameters and layouts for both years to make the networks most comparable (see SI). 

% We calculate the difference of the averaged skill level of a skill $s$ between the years 2005 and 2019 as in equation \ref{eq: skill change}:
%\begin{equation} \label{eq: skill change}
    %\Delta(s)_{2019,2005} = Level(s)_{2019} - Level(s)_{2005}
%\end{equation}


\section*{Acknowledgement}
H. Y. and M. H. acknowledge the support of the National Science Foundation Grant Award Number EF-2133863.
The authors are grateful to Yong-Yeol Ahn, Inho Hong, Hyunuk Kim, Balazs Lengyel, Muhammed Yildirim, James McNerney, Morgan Frank, Ljubica Nedelkoska, Christopher Esposito, Ulrich Schetter, Serguei Saavedra, James Evans, and Brian Uzzi for their valuable discussions and feedback.
F.N. gratefully acknowledges financial support from the Austrian Research Agency (FFG), project \#873927 (ESSENCSE).

% \bibliographystyle{naturemag}
% \bibliography{scibib.bib}

\printbibliography

\end{document}

