\documentclass[11pt]{amsart}
\usepackage[utf8x]{inputenc}
\usepackage[english]{babel}
\usepackage[T1]{fontenc}
\renewcommand{\baselinestretch}{1.2} 
\usepackage{graphicx}
\usepackage{caption}
\usepackage{subcaption}

\usepackage{amssymb,amsmath}
\usepackage[hidelinks]{hyperref}
\usepackage{indentfirst}
\usepackage{enumerate,amsmath,amssymb, mathrsfs,mathtools}
\usepackage{appendix}
\usepackage{latexsym}
\usepackage{url}
\usepackage{color}
\usepackage{accents}
\usepackage{setspace}
\usepackage{pdfpages}
\usepackage{stmaryrd}
\usepackage{amsrefs}

\usepackage[margin=2.5cm]{geometry}
\allowdisplaybreaks
\newcommand{\Acirc}{\accentset{\circ}{A}}
\newcommand{\hcirc}{\accentset{\circ}{h}}
\newcommand{\hbarcirc}{\accentset{\circ}{\bar{h}}}
\newcommand{\sigmacirc}{\accentset{\circ}{\sigma}}
\newcommand{\ubar}[1]{\underaccent{\bar}{#1}}
\BibSpec{article}{%
	+{}{\PrintAuthors} {author}
	+{,}{ } {title}
	+{, }{\textit } {journal}
	+{}{ \parenthesize} {date}
		+{,  }{no. } {volume}
	+{,}{ } {pages}
	+{,}{ } {note}
%	+{, }{\PrintDOI} {url}
}

\ExplSyntaxOn

%\NewExpandableDocumentCommand{\gobblefirst}{m}
%{
%	\tl_tail:n { #1 }
%}

\ExplSyntaxOff
\BibSpec{book}{%
	+{}{\PrintAuthors}  {author}
	+{. }{}{title}
	+{,}{ }{series}
	+{,}{ vol.~}{volume}
	+{. }{\textit}{publisher}
%	+{,}{ \DashPages} {pages}
	+{,}{ ISBN}{isbn} %{ISBN \gobblefirst}{isbn}
}
\BibSpec{collection.article}{%
	+{}{\PrintAuthors}{author}
	+{, }{}{title}
	+{, }{\textit}{booktitle}
	+{, }{ \DashPages}{pages}
	+{,}{ }{series}
	+{, }{}{volume}
	+{, }{\textit}{publisher}
	+{,}{ }{date}
}

\parskip 0.0cm

\mathcode`l="8000
\begingroup
\makeatletter
\lccode`\~=`\l
\DeclareMathSymbol{\lsb@l}{\mathalpha}{letters}{`l}
\lowercase{\gdef~{\ifnum\the\mathgroup=\m@ne \ell \else \lsb@l \fi}}%
\endgroup
\def\Xint#1{\mathchoice
	{\XXint\displaystyle\textstyle{#1}}%
	{\XXint\textstyle\scriptstyle{#1}}%
	{\XXint\scriptstyle\scriptscriptstyle{#1}}%
	{\XXint\scriptscriptstyle\scriptscriptstyle{#1}}%
	\!\int}
\def\XXint#1#2#3{{\setbox0=\hbox{$#1{#2#3}{\int}$ }
		\vcenter{\hbox{$#2#3$ }}\kern-.6\wd0}}
\def\ddashint{\Xint=}
\def\dashint{\Xint-}

\newtheorem{prop}{Proposition}
\newtheorem{thm}[prop]{Theorem}
\newtheorem{lem}[prop]{Lemma}
\newtheorem{coro}[prop]{Corollary}
\newtheorem{question}[prop]{Question}
\newtheorem{rema}[prop]{Remark}
\newtheorem{exam}[prop]{Example}
\newtheorem{defi}[prop]{Definition}
\newtheorem{conj}[prop]{Conjecture}
\newtheorem{pro}[prop]{Problem}
\newtheorem{quest}[prop]{Question}
\newtheorem{claim}[prop]{Claim}
\newtheorem{RK}[prop]{Remark}
\newtheorem{asm}{Assumption}

\title[Schoen's conjecture for limits of isoperimetric surfaces]{Schoen's conjecture for limits of isoperimetric surfaces}
\author{Michael Eichmair}
\address{
	\textnormal{Michael Eichmair \newline  \indent
		University of Vienna \newline \indent
		Faculty of Mathematics  \newline \indent
		Oskar-Morgenstern-Platz 1 \newline \indent
		1090 Vienna, 	Austria  \newline\indent 
		 \href{https://orcid.org/0000-0001-7993-9536}{https://orcid.org/0000-0001-7993-9536} \newline\indent	
		 \href{mailto:michael.eichmair@univie.ac.atm}{michael.eichmair@univie.ac.at}}
}



\author{Thomas Koerber}
\address{\textnormal{Thomas Koerber  \newline \indent
		University of Vienna \newline \indent
		Faculty of Mathematics  \newline \indent
		Oskar-Morgenstern-Platz 1 \newline \indent 1090 Vienna,	Austria \newline\indent 
		 \href{https://orcid.org/0000-0003-1676-0824}{https://orcid.org/0000-0003-1676-0824} \newline \indent
		  \href{mailto:thomas.koerber@univie.ac.atm}{thomas.koerber@univie.ac.at}}
}

\begin{document}

	\date{\today}
	\onehalfspacing
	\begin{abstract}
		Let $(M,g)$ be an asymptotically flat Riemannian manifold of dimension $3\leq n\leq 7$ with non-negative scalar curvature. R.~Schoen has conjectured that $(M,g)$ is isometric to Euclidean space if it admits a non-compact area-minimizing hypersurface $\Sigma \subset M$. This has been proved by O.~Chodosh and the first-named author in the case where $n = 3$. In this paper, we confirm this conjecture in the case where $3<n\leq 7$ and  $\Sigma$ arises as the limit of isoperimetric surfaces. As a corollary, we obtain that large isoperimetric surfaces diverge unless $(M,g)$ is flat. By contrast, we show that, in dimension $3<n\leq 7$,  a large part of spatial Schwarzschild  is foliated by non-compact area-minimizing hypersurfaces.
			\end{abstract}
	\maketitle 
	\section{Introduction}	
	Throughout, we assume that $(M,g)$ is a connected, complete Riemannian manifold. \\ \indent 
The following conjecture of R.~Schoen is related to his proof of the positive mass theorem with S.-T.~Yau in \cites{pmt, montecatini}.

\begin{conj}[Cp.~{\cite[p.~48]{schoentalk}}] \label{conjecture} 
	Let $(M,g)$ be a Riemannian manifold of dimension $ 3\leq n\leq 7$ and asymptotically flat of rate $\tau= n - 2$ with non-negative scalar curvature. Suppose that there exists a non-compact area-minimizing boundary $\Sigma =\partial \Omega$. Then $(M,g)$ is isometric to flat $\mathbb{R}^n$.
\end{conj}

The background on asymptotically flat manifolds, area-minimizing boundaries, and isoperimetric regions used in this paper is recalled in Appendix \ref{adm appendix} and Appendix \ref{iso appendix}. \\ \indent 
Conjecture \ref{conjecture} has been proved in the special case where $n=3$ by O.~Chodosh and the first-named author \cite[Theorem 1.6]{CCE}. A natural way in which non-compact area-minimizing boundaries arise is as the limit of isoperimetric surfaces. The goal of this paper is to settle Conjecture \ref{conjecture} in this case.

\begin{thm}
	\label{main result 2} Let $(M,g)$ be a Riemannian manifold of dimension $3<n\leq 7$ and asymptotically flat of rate $\tau>n-3$ with non-negative scalar curvature. Suppose that there exist a non-compact area-minimizing boundary $\Sigma =\partial \Omega$ and isoperimetric regions $\Omega_1,\,\Omega_2,\hdots\subset M$  with $\Omega_k\to \Omega$ locally smoothly. Then $(M,g)$ is isometric to flat $\mathbb{R}^n$.
\end{thm}

\begin{rema}
Note that the decay rate $\tau>n-3$ guarantees that coordinate hyperplanes in the end of $(M,g)$ are asymptotically flat with mass zero.
\end{rema}
O.~Chodosh, Y.~Shi, H.~Yu, and the first-named author have showed that in asymptotically flat Riemannian three-manifolds with non-negative scalar curvature and positive mass, there is a unique isoperimetric region for every given sufficiently large amount of volume and that these large isoperimetric regions are close to centered coordinate balls in the chart at infinity; see \cite[Theorem 1.1]{CESH}. An alternative proof of this result with a different condition on the scalar curvature was subsequently given by H.~Yu; see \cite[Theorem 1.6]{yu}. As a step towards the characterization of large isoperimetric regions in asymptotically flat Riemannian manifolds of dimension $3 < n \leq 7$, Corollary \ref{main result} shows that the (unique) large components of the boundaries of such regions necessarily diverge.
\begin{coro}\label{main result} Let $(M,g)$ be a Riemannian manifold of dimension $3<n\leq 7$ and asymptotically flat of rate $\tau>n-3$ with non-negative scalar curvature and positive mass. Let $K\subset M$ be a compact set that is disjoint from the boundary of $M$ and suppose that there are isoperimetric regions $\Omega_1,\,\Omega_2,\hdots$ in $(M,g)$ with $|\Omega_k|\to\infty$. Then, for all $k$ sufficiently large, either $K\subset \Omega_k$  or $K\cap \Omega_k=\emptyset$.
\end{coro}
\subsection*{Outline of our arguments}
 Let $(M,g)$ be an asymptotically flat Riemannian manifold of dimension $3\leq n\leq 7$ with non-negative scalar curvature. Suppose that $\Sigma=\partial \Omega$ is a non-compact area-minimizing boundary. In particular, for every open set $U \Subset M$ and every smooth variation $\{\Sigma(s)\}_{|s| < \epsilon}$ of $\Sigma = \Sigma(0)$ with compact support in $U$,
 $$ \frac{d}{ds}\bigg|_{s=0} |\Sigma(s) \cap U| = 0\qquad  \text{ and }\qquad \frac{d^2}{ds^2}\bigg|_{s=0} |\Sigma(s) \cap U| \geq 0.
 $$
 Equivalently, the mean curvature of $\Sigma$ vanishes and the stability inequality
  $$
 \int_{\Sigma} (|h|^2+Ric(\nu,\nu))\,f^2\,\mathrm{d}\mu\leq \int_{\Sigma} |\nabla f|^2\,\mathrm{d}\mu
 $$
 holds for all $f\in C^\infty_c(\Sigma)$. Here, $\mathrm{d}\mu$ is the area element, $\nabla$ the covariant derivative, $\nu$ the outward normal, and $h$ the second fundamental form, all with respect to $\Sigma$. $Ric$ denotes the Ricci tensor of $(M,g)$. We say that $\Sigma$ is stable with respect to asymptotically constant variations if, in addition, 
 \begin{align} \label{stable wrt acv} 
 \int_{\Sigma} (|h|^2+Ric(\nu,\nu))\,(1+f)^2\,\mathrm{d}\mu\leq \int_{\Sigma} |\nabla f|^2\,\mathrm{d}\mu
 \end{align} 
 for all $f\in C^\infty_c(\Sigma)$.\\
 \indent The proof of the positive mass theorem \cite[Theorem 4.2]{montecatini} shows that an asymptotically flat area-minimizing boundary that has mass zero and which is stable with respect to asymptotically constant variations is isometric to flat $\mathbb{R}^{n-1}$ and totally geodesic. Moreover, the scalar curvature of $(M,g)$ vanishes along such a boundary; see Proposition \ref{flat flat}. An important ingredient in the proof of Conjecture \ref{conjecture} in \cite{CCE} in the case where $n = 3$ and specific to three dimensions is that every non-compact area-minimizing boundary is stable with respect to asymptotically constant variations; see, e.g., \cite[p.~54]{pmt}. Using this, O.~Chodosh and the first-named author have showed that $(M,g)$ is foliated by non-compact area-minimizing boundaries. The construction of these boundaries is based on solving Plateau problems with respect to a carefully chosen local perturbation of the metric $g$ and inspired by the proof of a conjecture of J.~Milnor due to G.~Liu \cite{liu}. An adaptation of an argument by M.~Anderson and L.~Rodr\'iguez \cite{andersonrodriguez} shows that the curvature tensor of $(M,g)$ vanishes along each leaf of this foliation and hence on all of $M$. \\ \indent Our next results show that the situation is markedly different in the case where $3<n\leq 7.$
 \begin{thm} 	\label{counterexample}
 Let $3 < n \leq 7$ and $(M,g)$ be spatial Schwarzschild of dimension $n$ with mass $m=2$.  There exists infinitely many mutually disjoint non-compact area-minimizing hypersurfaces in $(M,g)$. 
 \end{thm}
The construction of the Riemannian manifold $(M,g)$ in Theorem \ref{counterexample 2} below is based on the gluing technique developed by A.~Carlotto and R.~Schoen \cite{SchoenCarlotto}.
\begin{thm} \label{counterexample 2}
 Let $3 < n \leq 7$ and $(n-2)/2<\tau<(n-2)$. There exists a Riemannian manifold $(M,g)$ of dimension $n$ that is asymptotically flat of rate $\tau$ with non-negative scalar curvature and positive mass which contains infinitely many mutually disjoint non-compact area-minimizing hypersurfaces  all of which are stable with respect to asymptotically constant variations. 
\end{thm}
\begin{rema}
	Theorem \ref{counterexample} and Theorem \ref{counterexample 2} show that Conjecture \ref{conjecture} is not true without any further assumptions. We note that the area-minimizing hypersurfaces whose existence is asserted in Theorem \ref{counterexample} are not stable with respect to asymptotically constant variations; see Proposition \ref{foliation prop}. By contrast, the Riemannian manifold constructed in the proof of Theorem \ref{counterexample 2} is not asymptotic to Schwarzschild.
\end{rema}
\begin{rema}
	O.~Chodosh and D.~Ketover have showed in \cite{chodoshketover} that in every complete asymptotically flat Riemannian three-manifold $(M, g)$ which does not contain closed embedded minimal surfaces, through every point, there exists a properly embedded minimal plane; see also the subsequent improvement due to L.~Mazet and H.~Rosenberg in \cite{mazetrosenberg}. Note that if the scalar curvature of $(M, g)$ is non-negative, none of these planes are area-minimizing unless $(M, g)$ is flat $\mathbb{R}^3$.
\end{rema}
In the proof of \cite[Theorem 4.2]{montecatini}, R.~Schoen and S.-T.~Yau have showed that if $(M,g)$ is asymptotic to Schwarzschild
\begin{align} \label{schwarzschild}  
g_S=\left(1+\frac{m}{2}\,|x|_{\bar g}^{2-n}\right)^{\frac{4}{n-2}}\,\bar g
\end{align}  with negative mass $m<0$, then $(M,g)$ contains a non-compact area-minimizing boundary that is stable with respect to asymptotically constant variations. Here, $\bar g$ is the Euclidean metric. This boundary is obtained as the limit of solutions to the Plateau problem with prudently chosen boundaries. Our starting point is the following complementary consideration. If $\Sigma=\partial \Omega$  arises as the limit of isoperimetric surfaces, then we expect $\Sigma$  to be stable with respect to asymptotically constant variations.  \\ \indent 
We now describe the proof of Theorem \ref{main result 2}. Suppose that $3<n\leq 7$. A first difficulty not present in the case where $n=3$ is to show that  $\Sigma$ is asymptotically flat with  mass zero. This is complicated by the fact that  $\Sigma$ is not known to be stable with respect to asymptotically constant variations at this point. By contrast, this stability is an additional assumption in the work of A.~Carlotto \cite{Carlottocalcvar}. To remedy this, we prove the explicit estimate
\begin{align} \label{intro area est} 
1-O(r^{-\frac{\tau}{n-1}})\leq \frac{|B_r\cap \Sigma|}{\omega_{n-1}\,r^{n-1}}\leq 1+O(r^{-\tau});
\end{align} 
see Lemma \ref{upper area bound} and Lemma \ref{lemma lower area bound}.
Here, $B_r$, $r>1$, is the bounded region in $(M,g)$  whose boundary corresponds to $\{x\in\mathbb{R}^n:|x|_{\bar g}=r\}$ in the chart at infinity and $\omega_{n-1}$ the Euclidean area of an $(n-1)$-dimensional unit ball. The proof of \eqref{intro area est} is based on the monotonicity formula applied to carefully chosen, off-centered balls. We then use \eqref{intro area est} to prove a precise asymptotic expansion for $\Sigma$; see Proposition \ref{graph estimate}. Using that $\tau>n-3$, it follows that $\Sigma$ is asymptotically flat with  mass zero; see Lemma \ref{vanishing mass}. We also note that these arguments work for any stable properly embedded non-compact minimal hypersurfaces with $r^{1-n} \, |B_r\cap \Sigma|=O(1)$ for $r >1$.\\ \indent
Next, we assume that $\Sigma=\partial \Omega$ where $\Omega$ is the limit of large isoperimetric regions $\Omega_1,\,\Omega_2,\hdots$ with $|\Omega_k|\to\infty$ and prove that $\Sigma$ is stable with respect to asymptotically constant variations. To this end, we consider the second variation of area of $\Omega_k$ with respect to a suitable Euclidean translation that is corrected to be volume-preserving. The stability with respect to asymptotically constant variations then follows by passing to the limit $k\to\infty$, using the asymptotic expansion for $\Sigma$ obtained in Proposition \ref{graph estimate}, the assumption that $\tau>n-3$, and the integration by parts formula in Lemma  \ref{euclidean lemma}; see Proposition \ref{strictly stable}. The arguments from \cite{montecatini} then show that $\Sigma$ is isometric to flat $\mathbb{R}^{n-1}$ and totally geodesic; see Proposition \ref{flat flat}.  \\ \indent Finally, given any point $p\in M$, we construct a new non-compact area-minimizing boundary $\Sigma_p\subset M$ that passes through $p$; see Proposition \ref{many area minimizing}. In view of Theorem \ref{counterexample} and different from the situation in \cite{CCE}, we need to ensure that $\Sigma_p$ is again stable with respect to asymptotically constant variations. To this end, we construct suitable local perturbations of the metric $g$ in Lemma \ref{metric perturbation} and obtain $\Sigma_p$ as the limit of large isoperimetric regions with respect to these perturbations. A crucial ingredient in the construction of $\Sigma_p$ is a result from \cite{CCE}, stated here as Lemma \ref{existence iso}; namely: Asymptotically flat Riemannian manifolds of positive mass admit isoperimetric regions of every sufficiently large volume. Although the area-minimizing boundaries obtained in our construction do not necessarily form a foliation, we show how to adapt the techniques developed in \cites{CCE,andersonrodriguez,liu} to conclude that the curvature tensor of $(M,g)$ vanishes along each of these boundaries. This completes the proof of Theorem \ref{main result 2}.
\subsection*{Outline of related results} J. Metzger and the first-named author \cite{Eichmair-Metzer:2012} have observed that the existence of area-minimizing boundaries in asymptotically flat manifolds is related to the positioning of large isoperimetric regions. In particular, the authors show that, in asymptotically flat Riemannian three-manifolds, the existence of large isoperimetric regions that do not diverge  is not compatible with positive scalar curvature; see \cite[Theorem 1.5]{Eichmair-Metzer:2012}. In subsequent work \cite[Theorem 1.1]{eichmairmetzgerinvent}, they have showed that, if $(M,g)$ is asymptotic to Schwarzschild of dimension $n \geq 3$, large isoperimetric regions are unique and geometrically close to centered coordinate balls. This implies, in particular, that Theorem \ref{main result 2} holds in all dimensions if $(M,g)$ is assumed to be asymptotic to Schwarzschild. We also note that  Conjecture \ref{conjecture} has been proved by A.~Carlotto \cite[Theorem 1 and Theorem 2]{Carlottocalcvar} in the case where $3\leq n\leq 7$ under the additional assumptions that $(M,g)$ is asymptotic to Schwarzschild and that $\Sigma$ is stable with respect to asymptotically constant variations. In this case, Proposition \ref{flat flat} below yields an immediate contradiction once $\Sigma$ is showed to be asymptotically flat with  mass zero.
Finally, we note that the method of O.~Chodosh and the first-named author in \cite[p.~991]{CCE}  has been used by C.~Li to study the polyhedron rigidity conjecture using isoperimetric regions  \cites{chaoli,chaoli2}. 
\subsection*{Acknowledgments}
 The first-named author acknowledges the support of the START-Project Y963 of the Austrian Science Fund. The second-named author acknowledges the support of the Lise-Meitner-Project M3184 of the Austrian Science Fund.


\section{Asymptotic behavior of area-minimizing surfaces}
In this section, we assume that $g$ is a Riemannian metric on $\mathbb{R}^n$  such that
$$
|g-\bar g|_{\bar g}+|x|_{\bar g}\,|\bar D g|_{\bar g}+|x|^2_{\bar g}\,|\bar D^2 g|_{\bar g}=O(|x|_{\bar g}^{-\tau}),
$$
where $3\leq  n\leq 7$ and $0<\tau\leq n-2$. Here, $\bar g$ denotes the Euclidean metric. Geometric quantities are computed with respect to $g$ unless indicated otherwise. \\
\indent Let $\Sigma \subset \mathbb{R}^n$ be a non-compact two-sided properly embedded hypersurface with $\partial \Sigma=\emptyset$. We assume that $\{x\in \mathbb{R}^n:|x|_{\bar g}>1\}\cap\Sigma$ is a stable minimal surface and that
\begin{align} \label{area growth polynomial}  
\limsup_{r\to\infty} \frac{|B^n_r(0)\cap\Sigma|_{\bar g}}{\omega_{n-1}\,r^{n-1}}<\infty.
\end{align} 
 \indent 
The goal of this section is to prove the following result.
 \begin{prop} \label{graph estimate} There exist $r_0>2,$ an integer $N\geq 1$, numbers $a_1,\hdots,a_N\in\mathbb{R}$, functions $u_1,\hdots, u_N\in C^\infty(\mathbb{R}^{n-1}),$ and a rotation $S\in SO(n)$ such that 
$$
S(\Sigma\setminus B^n_{r_0}(0))\subset \bigcup_{i=1}^N\{(y,u_i(y)):y\in\mathbb{R}^{n-1}\}.
$$
Moreover, for all $0<\varepsilon<\tau/2$ and $i=1,\dots,N$,
 		\begin{align} \label{graph estimate optimal 2} 
 			|y|^{-1}_{\bar g}\,|u_i(y)-a_i|+|\bar\nabla u_i|_{\bar g}+|y|_{\bar g}\,|\bar\nabla^2 u_i |_{\bar g}=O(|y|_{\bar g}^{-\tau+\varepsilon }).
 		\end{align} 
 \end{prop}
\begin{rema} In the case where $g$ is asymptotic to the Schwarzschild metric \eqref{schwarzschild}, Proposition \ref{graph estimate} has been proved by A.~Carlotto in \cite[Lemma 18]{Carlottocalcvar} under the additional assumption that $\Sigma$ is stable with respect to asymptotically constant variations in the sense of \eqref{stable wrt acv}.  There, a version of Corollary \ref{hyperplane uniqueness} is obtained using techniques developed by L.~Simon \cite[Theorem 5.7]{simonisolated}; see \cite[p.~10]{Carlottocalcvar}. A version of  estimate \eqref{hoelder decay} is obtained as a consequence  of the stability with respect to asymptotically constant variations; see \cite[pp.~17-18]{Carlottocalcvar}. Here, we provide a new proof of these results in \cite{Carlottocalcvar} based on the monotonicity formula  which does not require the assumption of stability with respect to asymptotically constant variations. 
\end{rema}

We first reduce the proof of Proposition \ref{graph estimate} to the case where $\Sigma$ has only one end.
\begin{lem} \label{components reduction}
	There exists $r_0>1$ and an integer $N\geq 1$ such that $\Sigma\setminus  B^n_{r_0}(0)$ has $N$ connected components $\Sigma_1,\hdots,\Sigma_N\subset\mathbb{R}^n$ each satisfying
	\begin{align} \label{multiplicity one}
	\lim_{r\to\infty}\frac{|B^n_{r}(0)\cap \Sigma_i|_{\bar g}}{\omega_{n-1}\,r^{n-1}}=1.
	\end{align} 
\end{lem}
\begin{proof}
By the work of R.~Schoen and L.~Simon \cite[Theorem 3]{schoensimon},  $h=O(|x|^{-1}_{\bar g})$ where $h$ is the second fundamental form of $\Sigma$ with respect to a choice of unit normal. In conjunction with \eqref{area growth polynomial}, the assumption that $\Sigma$ is non-compact, and the classification of stable minimal cones in $\mathbb{R}^n$ by J.~Simons \cite[\S6]{Simons}, it follows that for each sequence $\{r_k\}_{k=1}^\infty$ of numbers $r_k>0$ with $r_k\to\infty$, $r_k^{-1}\,\Sigma$ converges to a hyperplane locally smoothly in $\mathbb{R}^n\setminus\{0\}$  possibly with multiplicity. In particular, $\Sigma$ intersects $S^{n-1}_r(0)$ transversally for all $r>1$ sufficiently large. It follows that there is $r_0>1$ such that the number of components of $\Sigma\setminus B^n_{r}(0)$ is finite and constant for all $r>r_0$. Let $\Sigma_1$ be a component of $\Sigma\setminus B^n_{r_0}(0)$. Since $\Sigma_1\setminus B^n_r(0)$ is connected for all $r>r_0$, $r_k^{-1}\,\Sigma_1$ converges  to a hyperplane locally smoothly in $\mathbb{R}^n\setminus \{0\}$ with multiplicity one for each sequence $\{r_k\}_{k=1}^\infty$ of numbers $r_k>0$ with $r_k\to\infty$. In conjunction with \eqref{area growth polynomial}, we obtain \eqref{multiplicity one}.
\end{proof}
In view of Lemma \ref{components reduction}, we may and will assume that
\begin{align} \label{area bound at infinity}
	\lim_{r\to\infty}\frac{|B^n_{r}(0)\cap \Sigma|_{\bar g}}{\omega_{n-1}\,r^{n-1}}=1
\end{align}
in the proof of Proposition \ref{graph estimate}. 
We also record the following byproduct of the proof of Lemma \ref{components reduction}.
\begin{lem} \label{blowdown at infinity are planes}
	Let $\{r_k\}_{k=1}^\infty$ be a sequence of numbers $r_k>1$ with $r_k\to\infty$. Then, passing to a subsequence, $r_k^{-1}\,\Sigma$ converges locally smoothly in $\mathbb{R}^n\setminus \{0\}$  with multiplicity one to a hyperplane through the origin. 
\end{lem}
 \indent 
In Lemma \ref{mean curvature estimate} and Lemma \ref{area estimate sigma bar}, we collect basic properties of $\Sigma$.
\begin{lem} There holds, as $|x|_{\bar g}\to\infty$,  \label{mean curvature estimate}
\begin{align*} 
|x|_{\bar g}\,	|\bar H|+|x|^2_{\bar g}\,|\bar\nabla \bar H|_{\bar g}=O(|x|_{\bar g}^{-\tau})
\end{align*}
and
\begin{align*}
	|x|_{\bar g}\,|\bar h|_{\bar g}+|x|^2_{\bar g}\,|\bar\nabla \bar h|_{\bar g}=o(1).
\end{align*}
\end{lem} 
\begin{proof}
	This follows from Lemma \ref{geometric expansions} using  Lemma \ref{blowdown at infinity are planes} and that $H=0$ on $\{x\in\mathbb{R}^n:|x|_{\bar g}>1\}\cap \Sigma$.
\end{proof}
To proceed, we recall the monotonicity formula from \cite{simonlectures}.
\begin{lem}[{\cite[17.4]{simonlectures}}]
	Let $x_0\in\mathbb{R}^n$ and $0<s<t$. There holds
	\begin{equation} \label{monotonicity formula} 
		\begin{aligned} 
			t^{1-n}\,| B^n_{t}(x_0)\cap  \Sigma|_{\bar g}&=s^{1-n}\,| B^n_{s}(x_0)\cap  \Sigma|_{\bar g}+\int_{(B^n_{t}(x_0)\setminus B^n_{s}(x_0))\cap  \Sigma}|x-x_0|_{\bar g}^{1+n}\,\bar g(x-x_0,\bar\nu)^2\,\mathrm{d}\bar\mu\\ &\qquad -\frac{1}{n-1}\,\int_{(B^n_{t}(x_0)\setminus B^n_{s}(x_0))\cap  \Sigma}(t^{1-n}-|x-x_0|_{\bar g}^{1-n})\,\bar g(x-x_0,\bar\nu)\,\bar H\,\mathrm{d}\bar\mu\\ &\qquad  
			-\frac{1}{n-1}\,\int_{B^n_{s}(x_0)\cap  \Sigma}(t^{1-n}-s^{1-n})\,\bar g(x-x_0,\bar\nu)\,\bar H\,\mathrm{d}\bar\mu.
		\end{aligned} 
	\end{equation}
\end{lem} 
\begin{lem} \label{area estimate sigma bar}
	There holds
	$$
	\sup_{x\in\mathbb{R}^n} \sup_{r>0} \frac{|  B^n_{r}(x)\cap \Sigma|_{\bar g}}{\omega_{n-1}\,r^{n-1}}<\infty.
	$$
\end{lem} 
\begin{proof}
Suppose, for a contradiction, that there are sequences $\{r_k\}_{k=1}^\infty$ of numbers $r_k>0$ and $\{x_k\}_{k=1}^\infty$ of points $x_k\in \mathbb{R}^n$ with
\begin{align} \label{contradiction area estimate}  
	\lim_{k\to\infty} \frac{|  B^n_{r_k}(x_k)\cap \Sigma|_{\bar g}}{\omega_{n-1}\,r_k^{n-1}}=\infty.
\end{align} 
Passing to a subsequence and using that  $\Sigma$ is properly embedded, we may assume that either
$$
\lim_{k\to\infty} |x_k|_{\bar g}=\infty\qquad\text{or}\qquad \lim_{k\to\infty} r_k=\infty.
$$
\indent Note that
\begin{align} \label{r vs x} 
	\liminf_{k\to\infty} \frac{|x_k|_{\bar g}}{r_k}\geq 3.
\end{align} 
Indeed,	if not, then $B^n_{r_k}(x_k)\subset B^n_{4\,r_k}(0)$ for a subsequence. This is not compatible with \eqref{contradiction area estimate} and \eqref{area bound at infinity}. \\ \indent 
Let $t_k=|x_k|_{\bar g}/2$.
By Lemma \ref{mean curvature estimate}, we have \begin{align} \label{contr mean est}|x_k|_{\bar g}\,\bar H=O(|x_k|_{\bar g}^{-\tau})
	\end{align}  on $B^n_{t_k}(x_k)\cap \Sigma$. We choose $s_k$ with $r_k\leq s_k\leq t_k$ such that
\begin{align} \label{sup}  
\frac{| B^n_{s_k}(x_k)\cap \Sigma|_{\bar g}}{\omega_{n-1}\,s_k^{n-1}}= \sup_{r_k\leq r\leq t_k} \frac{| B^n_{r}(x_k)\cap \Sigma|_{\bar g}}{\omega_{n-1}\,r^{n-1}}.
\end{align} 
 Using the monotonicity formula \eqref{monotonicity formula} and \eqref{contr mean est}, we have
\begin{align*} 
\frac{| B^n_{t_k}(x_k)\cap \Sigma|_{\bar g}}{\omega_{n-1}\, t_k^{n-1}}&\geq \frac{| B^n_{s_k}(x_k)\cap \Sigma|_{\bar g}}{\omega_{n-1}\,s_k^{n-1}} -O(|x_k|_{\bar g}^{-1-\tau})\,\int_{(B^n_{t_k}(x_k)\setminus B^n_{s_k}(x_k))\cap \Sigma}\,|x-x_k|_{\bar g}^{2-n}\,\mathrm{d}\bar\mu\\&\qquad -O(|x_k|_{\bar g}^{-\tau})\,\frac{| B^n_{s_k}(x_k)\cap \Sigma|_{\bar g}}{\omega_{n-1}\,s_k^{n-1}}. 
\end{align*}
Using Lemma \ref{layer cake} and \eqref{sup}, we have
$$
\int_{(B^n_{t_k}(x_k)\setminus B^n_{s_k}(x_k))\cap \Sigma}\,|x-x_k|_{\bar g}^{2-n}\,\mathrm{d}\bar\mu=O(|x_k|_{\bar g})\,\frac{| B^n_{s_k}(x_k)\cap \Sigma|_{\bar g}}{\omega_{n-1}\,s_k^{n-1}}.
$$ 
In conjunction with \eqref{contradiction area estimate} and \eqref{sup}, we conclude that
$$
\lim_{k\to\infty}\frac{| B^n_{t_k}(x_k)\cap \Sigma|_{\bar g}}{\omega_{n-1}\,t_k^{n-1}}=\infty.
$$
This is not compatible with \eqref{r vs x}.
\end{proof}

Next, we prove refined estimates on the area growth of $ \Sigma$.

\begin{lem} \label{upper area bound}
	As $s\to\infty$,
	$$
	\frac{| B^n_{s}(0)\cap \Sigma|_{\bar g}}{\omega_{n-1}\,s^{n-1}}\leq 1+O(s^{-\tau}).
	$$
\end{lem}
\begin{proof}

	Using the monotonicity formula \eqref{monotonicity formula} and Lemma  \ref{mean curvature estimate},  we have, 	for every $0<s<t$,
	\begin{align*} 
		\frac{|B^n_{s}(0)\cap \Sigma|_{\bar g}}{\omega_{n-1}\,s^{n-1}}&\leq\frac{|B^n_{t}(0)\cap \Sigma|_{\bar g}}{{\omega_{n-1}\,t^{n-1}}}+O(1)\,\int_{(B^n_t(0)\setminus B^n_{s}(0))\cap \Sigma} |x|_{\bar g}^{1-n-\tau}\,\mathrm{d}\bar\mu\\
		 &\qquad +O(1)\,s^{1-n}\,\int_{  B^n_{s}(0)\cap \Sigma}|x|_{\bar g}^{-\tau}\,\mathrm{d}\bar\mu.
	\end{align*} 
 Using Lemma \ref{layer cake}, it follows that 
	$$
	\frac{| B^n_{s}(0)\cap \Sigma|_{\bar g}}{\omega_{n-1}\,s^{n-1}}\leq\frac{|B^n_{t}(0)\cap \Sigma|_{\bar g}}{{\omega_{n-1}\,t^{n-1}}}+O(t^{-\tau})+O(s^{-\tau}).
	$$
	Letting $t\to\infty$ and using  \eqref{area bound at infinity}, the assertion follows.
\end{proof}
\begin{lem} \label{lemma lower area bound} 
As $t\to\infty$,
	$$
\frac{| B^n_{t}(0)\cap \Sigma|_{\bar g}}{{\omega_{n-1}\,t^{n-1}}}\geq 1-O(t^{-\tau/(n-1)}).
	$$
\end{lem}
\begin{proof}
		For $t>1$ large, we choose $x_t\in  \Sigma$ with $|x_t|_{\bar g}=t^{(n-1-\tau)/(n-1)}$. We apply the monotonicity formula \eqref{monotonicity formula} with $x_0=x_t$. Letting $s\to 0$, using that $\Sigma$ is properly embedded and Lemma \ref{mean curvature estimate}, we obtain 
	$$
	\frac{| B^n_{t}(x_t)\cap \Sigma|_{\bar g}}{\omega_{n-1}\,t^{n-1}}\geq 1  +O(1)\,\int_{   B^n_{t}(x_t)\cap \Sigma}|x-x_t|_{\bar g}^{2-n}\,|x|_{\bar g}^{-1-\tau}\,\mathrm{d}\bar\mu.
	$$
Clearly, $|x|_{\bar g}> |x_t|_{\bar g}/2$ for all $x\in B^n_{|x_t|_{\bar g}/2}(x_t)$. Using Lemma \ref{layer cake},	we obtain
	$$
	\int_{ B^n_{|x_t|_{\bar g}/2}(x_t)\cap \Sigma}|x-x_t|_{\bar g}^{2-n}\,|x|_{\bar g}^{-1-\tau}\,\mathrm{d}\bar\mu=O(|x_t|_{\bar g}^{-\tau}).
	$$
	Likewise,  $|x-x_t|_{\bar g}\geq|x_t|_{\bar g}/2$ for all $x\in\mathbb{R}^n\setminus B^n_{|x_t|_{\bar g}/2}(x_t)$. It follows that \begin{align*} 
		\int_{  ( B^n_{t}(x_t)\setminus B^n_{|x_t|_{\bar g}/2}(x_t))\cap \Sigma }|x-x_t|_{\bar g}^{2-n}\,|x|_{\bar g}^{-1-\tau}\,\mathrm{d}\bar\mu=O(1)\,|x_t|_{\bar g}^{2-n}\,\int_{ B^n_{t}(0)\cap \Sigma}|x|_{\bar g}^{-1-\tau}\,\mathrm{d}\bar \mu.
	\end{align*} 
	Using Lemma \ref{layer cake} again,  we find
	$$
	\int_{ B^n_{t}(0)\cap \Sigma}|x|_{\bar g}^{-1-\tau}\,\mathrm{d}\bar\mu=O(t^{n-2-\tau}).
	$$
Since $0<\tau\leq n-2$, 
	 $$
-	\tau\,\frac{n-1-\tau}{n-1}< -\frac{\tau}{n-1}= (2-n)\,\frac{n-1-\tau}{n-1}+(n-2-\tau).
	$$
	We conclude that 
	$$
	\frac{|  B^n_{t}(x_t)\cap \Sigma|_{\bar g}}{\omega_{n-1}\,t^{n-1}}\geq 1 - O(t^{-\tau/(n-1)}).
	$$
Note that
	$$
	| B^n_{t}(x_t)\cap \Sigma|_{\bar g}\leq | B^n_{t\,(1+t^{-\tau/(n-1)})}(0)\cap \Sigma|_{\bar g}.
	$$
	Using also Lemma \ref{area estimate sigma bar},
	$$
	\frac{| B^n_{t\,(1+t^{-\tau/(n-1)})}(0)\cap \Sigma|_{\bar g}}{\omega_{n-1}\,t^{n-1}}\leq \frac{|  B^n_{t\,(1+t^{-\tau/(n-1)})}(0)\cap \Sigma|_{\bar g}}{\omega_{n-1}\,t^{n-1}\,(1+t^{-\tau/(n-1)})^{n-1}}+O(t^{-\tau/(n-1)}).
	$$
	\indent The assertion follows from these estimates.
\end{proof}
\begin{lem}
As $s\to\infty$,
	$$
	\int_{ \Sigma\setminus B^n_{s}(0)}|x|_{\bar g}^{-1-n}\,\bar g(x,\bar\nu)^2\,\mathrm{d}\bar\mu=O(s^{-\tau/(n-1)}).
	$$ \label{excess estimate}
\end{lem}
\begin{proof}
	We apply the monotonicity formula \eqref{monotonicity formula} with $x_0=0$. In conjunction with Lemma \ref{upper area bound} and Lemma \ref{lemma lower area bound}, letting $t\to\infty$, we obtain
	\begin{align*}
		\int_{ \Sigma\setminus B^n_{s}(0)}|x|_{\bar g}^{-1-n}\,\bar g(x,\bar\nu)^2\,\mathrm{d}\bar\mu&=O(1)\,\int_{ \Sigma\setminus B^n_{s}(0)}|x|_{\bar g}^{1-n}\,\bar g(x,\bar\nu)\,\bar H\,\mathrm{d}\bar\mu\\ &\qquad  
		+O(s^{1-n})\,\int_{ B^n_{s}(0)\cap \Sigma}\bar g(x,\bar\nu)\,\bar H\,\mathrm{d}\bar\mu+O(s^{-\tau/(n-1)}).
	\end{align*}
	By Lemma \ref{mean curvature estimate} and Lemma \ref{layer cake}, 
	$$
	\int_{ \Sigma\setminus B^n_{s}(0)}|x|_{\bar g}^{1-n}\,\bar g(x,\bar\nu)\,\bar H\,\mathrm{d}\bar\mu=O(s^{-\tau})
	\qquad \text{and}\qquad
\int_{  B^n_{s}(0)\cap \Sigma}\bar g(x,\bar\nu)\,\bar H\,\mathrm{d}\bar\mu=O(s^{n-1-\tau}).
	$$
	\indent The assertion follows from these estimates.
\end{proof}
Next, we  show that there is only one  tangent plane at infinity that can arise in the setting of Lemma \ref{blowdown at infinity are planes}. To this end, we apply an argument developed by B.~White \cite[pp.~146-147]{whitetangentconeuniqueness} to study the uniqueness of tangent planes at isolated singularities of area-minimizing  surfaces. This argument has been adapted to study the uniqueness of tangent planes at infinity of certain minimal surfaces in $\mathbb{R}^3$ by P.~Gallagher \cite[p.~374]{gallgheruniqueness}.
\begin{lem} \label{area estimate of normal}  Let $F:\Sigma\setminus B^n_1(0)\to S^{n-1}_1(0)$ be given by $$F(x)=\frac{x}{|x|_{\bar g}}.$$ As $s\to\infty$, 
	$$
	|F(\Sigma\setminus B^n_s(0))|_{\bar g}=O(s^{-\tau/(2\,n-2)}).
	$$ 
\end{lem} 
\begin{proof}
By the area formula,
	$$
	|F(\Sigma\setminus B^n_s(0))|_{\bar g}=\int_{\Sigma\setminus B^n_s(0)}|x|_{\bar g}^{-n}\,|\bar g(x,\bar\nu)|\,\mathrm{d}\bar\mu.
	$$
	By the Cauchy-Schwarz inequality, 
	\begin{align*} 
&\left(\int_{\Sigma\setminus B^n_s(0)}|x|_{\bar g}^{-n}\,|\bar g(x,\bar\nu)|\,\mathrm{d}\bar\mu\right)^2	\\&\qquad \leq \sum_{k=0}^\infty\int_{ (B^n_{2^{k+1}\,s}(0)\setminus B^n_{2^{k}s}(0))\cap \Sigma}|x|_{\bar g}^{-1-n}\,\bar g(x,\bar\nu)^2\,\mathrm{d}\bar\mu\,\int_{ (B^n_{2^{k+1}\,s}(0)\setminus B^n_{2^{k}s}(0))\cap \Sigma}|x|_{\bar g}^{1-n}\,\mathrm{d}\bar\mu. 
	\end{align*} 
Invoking Lemma \ref{upper area bound} and Lemma \ref{excess estimate}, we obtain
$$
\left(\int_{\Sigma\setminus B^n_s(0)}|x|_{\bar g}^{-n}\,|\bar g(x,\bar\nu)|\,\mathrm{d}\bar\mu\right)^2\leq O(1)\,\sum_{k=1}^\infty (2^k\,s)^{-\tau/(n-1)}=O(s^{-\tau/(n-1)}).
$$
\end{proof}
\begin{coro} \label{hyperplane uniqueness}
The  tangent planes in Lemma \ref{blowdown at infinity are planes} all agree.
\end{coro}
\begin{proof}
	Suppose, for a contradiction, that  $\pi_1, \pi_2\subset\mathbb{R}^n$ are two different tangent planes at infinity that arise as in Lemma \ref{blowdown at infinity are planes}. Let $\lambda_2>\lambda_1>s>1$ be such that $S^{n-1}_{1}(0)\cap\lambda_1^{-1}\,\Sigma$ and $S^{n-1}_{1}(0)\cap\lambda_2^{-1}\,\Sigma$ are close to $S^{n-1}_1(0)\cap \pi_1$ and $S^{n-1}_1(0)\cap \pi_2$, respectively. Note that by the intermediate value theorem,  $F((B^n_{\lambda_2}(0)\setminus B^n_{\lambda_1}(0))\cap\Sigma)$ contains at least two of the four components of 
	$S^{n-1}_1(0)\setminus (\lambda_1^{-1}\Sigma\cup \lambda_2^{-1}\,\Sigma)$. Using this, it follows that 
	$$
	\liminf_{s\to\infty} |F(\Sigma\setminus B^n_{s}(0))|_{\bar g}>0.
	$$
	This is not compatible with Lemma \ref{area estimate of normal}.
\end{proof}

\begin{proof}[{Proof of Proposition \ref{graph estimate}}]
	Using Lemma \ref{components reduction}, we may assume that \eqref{area bound at infinity} holds.
Using Lemma  \ref{blowdown at infinity are planes} and Corollary \ref{hyperplane uniqueness}, we see that, after a rotation, there are $r_0>1$ and $u\in C^\infty(\mathbb{R}^{n-1})$ with
	$$
	\Sigma\setminus B^n_{r_0}(0)\subset \{(y,u(y)):y\in\mathbb{R}^{n-1}\}.
	$$
	Using Lemma  \ref{blowdown at infinity are planes}, we obtain 
	that 
	\begin{align} \label{preliminary graph estimate} 
		|y|_{\bar g}^{-1}\,|u|+|\bar\nabla u|_{\bar g}+|y|_{\bar g}\,|\bar\nabla^2 u |_{\bar g}=o(1).
	\end{align} 
\indent 
	Let $v\in C^\infty( \Sigma)$ be given by $v=\bar g(x,\bar\nu)$. 
	Using the Codazzi equation and Lemma \ref{mean curvature estimate}, we have
	$$
	\bar\Delta v +|\bar h|^2_{\bar g}\,v=O(|x|_{\bar g}^{-1-\tau}).
	$$
	Let $x\in \Sigma$ and $r=|x|_{\bar g}/4$. Using the interior $L^2$-estimate \cite[Theorem 9.11]{gilbargtrudinger} and the Sobolev embedding theorem, recalling that $3\leq n\leq 7$, we have
	$$
r^{-1}\,\left(r^{1-n}\,\int_{ B^n_{2\,r}(x)\cap \Sigma}v^6\,\mathrm{d}\bar\mu\right)^\frac16\leq O(r^{-1})\,\left(r^{1-n}\,\int_{ B^n_{3\,r}(x)\cap \Sigma}v^2\,\mathrm{d}\bar\mu\right)^\frac12 +O(r^{-\tau}).
	$$
	By Lemma \ref{excess estimate},
	$$
	r^{3-n}\,\int_{  B^n_{3\,r}(x)\cap \Sigma}v^2\,\mathrm{d}\bar\mu=O(r^{-\tau/(n-1)}).
	$$
	It follows that
	$$
r^{-1}\,\left(r^{1-n}\,\int_{  B^n_{2\,r}(x)\cap \Sigma}v^6\,\mathrm{d}\bar\mu\right)^\frac16=O(r^{-\tau/(2\,n-2)}).
	$$
Using the interior $L^6$-estimate \cite[Theorem 9.11]{gilbargtrudinger} and the Sobolev embedding theorem, we have
		$$
r^{-1}\	\left(r^{1-n}\,\int_{  B^n_{r}(x)\cap \Sigma}v^{7}\,\mathrm{d}\bar\mu\right)^\frac1{7}=O(r^{-\tau/(2\,n-2)}).
	$$
	Finally, using the interior $L^7$-estimate \cite[Theorem 9.11]{gilbargtrudinger} and the Sobolev embedding theorem, we 
	 conclude that
	$$
	(\bar\nabla v)(x)=O(|x|_{\bar g}^{-\tau/(2\,n-2)}).
	$$
	Note that
	$
	\bar\nabla v=x^{\bar\top}\lrcorner\bar h.
	$ By  \eqref{preliminary graph estimate}, $|x|_{\bar g}=O(|x^{\bar\top}|_{\bar g})$. In conjunction with Lemma \ref{mean curvature estimate}, we conclude that
	$$
	|x|_{\bar g}\,\bar h=O(|x|_{\bar g}^{-\tau/(2\,n-2)}).
	$$
Using \eqref{preliminary graph estimate}, we obtain the improved estimate
	\begin{align} \label{hoelder decay}
	|y|_{\bar g}\,|\bar\nabla^2u|_{\bar g}=O(|y|_{\bar g}^{-\tau/(2\,n-2)}).
	\end{align} 
\indent Let $\alpha>0$ and suppose that $|y|_{\bar g}\,|\bar\nabla^2u|_{\bar g}=O(|y|_{\bar g}^{-\alpha})$. As in \cite[p.~16]{Carlottocalcvar}, accounting for the weaker decay assumptions on the Riemannian metric $g$ here, we rewrite the minimal surface equation as
$$
|y|_{\bar g}\,\bar\Delta u=O(|y|_{\bar g}\,|\bar\nabla u|_{\bar g}^2\,|\bar\nabla^2 u|_{\bar g})+O(|y|_{\bar g}^{-\tau}).
$$ 
In particular,
$$
|y|_{\bar g}\,|\bar\Delta u|=O(|y|_{\bar g}^{\max\{-3\,\alpha,-\tau\}}).
$$
Proceeding as in \cite[p.~16]{Carlottocalcvar}, we see that, given $\varepsilon>0$, there are $u_1,\,u_2\,\in C^\infty(\Sigma)$ such that $u=u_1+u_2$ where $u_2$ satisfies
$$
|y|_{\bar g}^{-1}\,|u_2|+|\bar\nabla u_2|_{\bar g}+|y|_{\bar g}\,|\bar \nabla ^2 u_2|_{\bar g}=O(|y|_{\bar g}^{\max\{-3\,\alpha,-\tau\}+\varepsilon}).
$$ 
and
 $u_1$ is harmonic with $ u_1-a=O(|y|_{\bar g}^{3-n})$ for some $a\in\mathbb{R}$ if $3<n\leq 7$ and $u_1=O(\log|y|_{\bar g})$ if $n=3$, respectively.
Iterating this argument, we obtain \eqref{graph estimate optimal 2} from \eqref{hoelder decay}.
\end{proof}
\begin{coro} \label{l2 second fundamental form}
	Suppose that $n-3<\tau\leq n-2$. There holds
	$$
	\int_{\Sigma} |h|^2\,\mathrm{d}\mu<\infty\qquad\text{and}\qquad  \int_{\Sigma} |{Ric}(\nu,\nu)|\,\mathrm{d}\mu<\infty.
	$$
\end{coro}
\begin{proof}
	This follows from Proposition \ref{graph estimate} and Lemma \ref{geometric expansions}.
\end{proof}
For the next lemma, recall the definition \eqref{def adm mass} of the mass of an asymptotically flat manifold.
\begin{lem} \label{vanishing mass} 
Suppose that $n-3<\tau\leq n-2$.	Each end of the  Riemannian $(n-1)$-manifold $(\Sigma,g|_{\Sigma})$ is asymptotically flat with mass zero. 
\end{lem}
\begin{proof}
Fix $i \in \{1, \ldots, N\}$ and let $\varphi : \mathbb{R}^{n-1} \to \mathbb{R}^n$ be the chart given by $\varphi(y) = (y, u_i(y))$ where $u_i\in C^\infty(\mathbb{R}^{n-1})$ is as in Proposition \ref{graph estimate}. Note that
\[
(\varphi^* g) (e_a, e_b) = \delta_{ab} + O (|y|_{\bar g}^{-\tau})
\]
for all $a, \, b \in \{1, \ldots, n-1\}$. Since $\tau > n - 3 = (n-1) - 2$, the assertion follows.
\end{proof}

\section{Stability with respect to asymptotically constant variations}
In this section, we assume that $(M,g)$ is  a Riemannian manifold of dimension  $3\leq n\leq 7$ which is  asymptotically flat of rate $\tau$ where
\begin{align}
	\label{tau condition} n-3<\tau\leq n-2.
\end{align}
 \indent Let $\Omega_1,\,\Omega_2,\hdots\subset M$ be isoperimetric regions  with $| \Omega_k|\to\infty$ such that $\Omega_k$ converges locally smoothly to a region $\Omega\subset M$ whose boundary $\Sigma=\partial \Omega$ is non-compact and area-minimizing. \\ \indent 
 Recall from Proposition \ref{smooth local convergence} that $\Sigma$ has one non-compact component and that the other components of $\Sigma$ are contained in the boundary of $M$. Moreover, after a rotation and passing to a subsequence, there holds $\lambda(\partial \Omega_k)^{-1}\,(\Omega_k\setminus B_1)\to B^n_1(e_n)$ locally smoothly in $\mathbb{R}^n\setminus \{0\}$. By Proposition \ref{graph estimate}, $\Sigma$ is  asymptotic to a coordinate hyperplane. Note that the normal of this plane pointing towards $\Omega$ is necessarily $e_n$.
  	\begin{figure}\centering
 	\includegraphics[width=0.7\linewidth]{boundary.pdf}
 	\caption{The hypersurfaces $\Sigma_k=\partial \Omega_k$ converge locally uniformly to $\Sigma=\partial \Omega$. Both the inward-pointing normal of $\Sigma$ and the re-scaled barycenters of $\Omega_k$ asymptote to the vector $e_n$. }
 	\label{boundary}
 \end{figure}
\\  \indent  
The goal of this section is to show that $\Sigma$ is  stable with respect to asymptotically constant variations, i.e., that
$$
\int_{\Sigma} (|h|^2+Ric(\nu,\nu))\,(1+f)^2\,\mathrm{d}\mu\leq\int_{\Sigma}|\nabla f|^2\,\mathrm{d}\mu
$$
for all $f\in C^\infty_c(\Sigma)$. 
 To this end, we will study the second variation of area of $\Sigma_k=\partial\Omega_k$ with respect to a Euclidean translation; see Figure \ref{boundary} \\ \indent  
 Let $\chi\in C^\infty(M,TM)$ be a vector field with $\chi=e_n$ in $M\setminus B_2$ and $\chi=0$ in $B_1$.  Let $u_k,\,v_k\in C^\infty(\Sigma_k)$ be the functions
$$
u_k=g(\chi,\nu)\qquad \text{and}\qquad  v_k=-h(\chi^\top,\chi^\top).
$$ 
We define the area radius of $\Sigma_k$ to be $$\lambda(\Sigma_k)=\left(\frac{|\Sigma_k|}{n\,\omega_{n}}\right)^{\frac{1}{n-1}}.$$
Recall from \eqref{H est} that $\lambda(\Sigma_k)\,H(\Sigma_k)=(n-1)+o(1)$.
\begin{lem}
	As $k\to\infty$, \label{first volume var}
	$$
\lambda(\Sigma_k)^{1-n}\,	\int_{\Sigma_k}u_k\,\mathrm{d}\mu=O(\lambda(\Sigma_k)^{-\tau}).
	$$
\end{lem}
\begin{proof}
Since $\Sigma_k$ converges to $\Sigma$ locally smoothly, there are closed hypersurfaces $\hat{\Sigma}_1,\,\hat{\Sigma}_2,\ldots$  with 
\begin{align*}
	&\circ \qquad  B^n_1(0)\cap \hat \Sigma_k=\emptyset,\qquad\qquad\qquad\qquad\qquad\qquad\qquad\qquad\qquad\\
	&\circ \qquad \hat \Sigma_k\setminus B^n_2(0)=\Sigma_k\setminus B_2, \\
		&\circ \qquad | B^n_2(0)\cap \hat \Sigma_k|_{\bar g}=O(1), \text{ and} \\
			&\circ \qquad  h(\hat\Sigma_k)=O(1).
\end{align*}
Note that
	$$
	\int_{\Sigma_k} g(\chi,\nu)\,\mathrm{d}\mu =\int_{\hat \Sigma_k} g(\chi,\nu)\,\mathrm{d}\mu- \int_{B_2\cap \hat  \Sigma_k} g(\chi,\nu)\,\mathrm{d}\mu+\int_{ B_2\cap \Sigma_k} g(\chi,\nu)\,\mathrm{d}\mu
	$$ 
	Clearly, 
	$$
	 \int_{ B_2\cap \hat  \Sigma_k} g(\chi,\nu)\,\mathrm{d}\mu=O(1).
	$$
	By Lemma \ref{area growth estimate}, $\limsup_{k\to \infty}|B_2\cap \Sigma_k|<\infty$, so that 
	$$
	\int_{ B_2\cap \Sigma_k} g(\chi,\nu)\,\mathrm{d}\mu=O(1).
	$$
Using Lemma \ref{geometric expansions}, we see that
$$
\int_{\hat \Sigma_k} g(\chi,\nu)\,\mathrm{d}\mu=\int_{\hat \Sigma_k} \bar g (e_n,\nu)\,\mathrm{d}\bar \mu+O(1)\,\int_{\hat \Sigma_k} |x|_{\bar g}^{-\tau}\,\mathrm{d}\bar\mu.
$$
By the divergence theorem,
$$
\int_{\hat \Sigma_k} \bar g (e_n,\nu)\,\mathrm{d}\bar \mu=0.
$$ 
Finally, by Lemma \ref{area growth estimate},
$$
\int_{\hat \Sigma_k} |x|_{\bar g}^{-\tau}\,\mathrm{d}\bar\mu=O(\lambda(\Sigma_k)^{n-1-\tau}).
$$
\indent The assertion follows from these estimates. 
\end{proof}
\begin{lem} \label{second volume var} 
	There holds, as $k\to\infty$,
	$$
\lambda(\Sigma_k)^{2-n}\,	\int_{\Sigma_k} v_k+H\,u_k^2\,\mathrm{d}\mu=O(\lambda(\Sigma_k)^{-\tau}).
	$$
\end{lem}
\begin{proof}
	We continue with the notation introduced in the proof of  Lemma \ref{first volume var}. Using the area estimate from Lemma \ref{area growth estimate} and the curvature estimates \eqref{H est} and \eqref{hcirc est}, we see that
	$$
	\int_{B_2\cap \Sigma_k} v_k+H\,u_k^2\,\mathrm{d}\mu=O(1).
	$$
	Similarly, 
	$$
	\int_{B_2\cap \hat \Sigma_k} -h(\chi^\top,\chi^\top)+H\,g(\chi,\nu)^2\,\mathrm{d}\mu=O(1).
	$$
	Using also Lemma \ref{geometric expansions}, we obtain 
	$$
	v_k+H\,u_k^2=-\bar h(e_n^{\bar\top},e_n^{\bar \top})+\bar H\,\bar g(e_n,\bar \nu)^2+O(|x|_{\bar g}^{-1-\tau}).
	$$
	Using Lemma \ref{area growth estimate}, we get
	$$
	\int_{\hat\Sigma_k}|x|_{\bar g}^{-1-\tau}\,\mathrm{d}\bar\mu=O(\lambda(\Sigma_k)^{n-2-\tau}).
	$$
	Combining these equations,
$$
\int_{\Sigma_k} v_k+H\,u_k^2\,\mathrm{d}\mu=\int_{\hat \Sigma_k}-\bar h(e_n^{\bar\top},e_n^{\bar \top})+\bar H\,\bar g(e_n,\bar \nu)^2\,\mathrm{d}\bar \mu+O(\lambda(\Sigma_k)^{n-2-\tau}).
$$ 
Note that, by the first variation formula, 
$$
\int_{\hat \Sigma_k}-\bar h(e_n^{\bar \top},e_n^{\bar \top})+\bar H\,\bar g(e_n,\bar \nu)^2\,\mathrm{d}\bar \mu=0.
$$
\indent The assertion follows from these estimates.
\end{proof}
Fix $\eta\in C^\infty(\mathbb{R})$ with 
\begin{itemize}
	\item[$\circ$]$\eta(t)=0$ if $t\leq 1$, 
	\item[$\circ$]$\eta(t)>0$ if $1<t<2$, and
	\item[$\circ$] $\eta(t)=0$ if $t\geq 2$.
\end{itemize}
 We define
$$
\kappa_k=\lambda(\Sigma_k)^{1-n}\,\int_{\Sigma_k} \eta\left(\lambda(\Sigma_k)^{-1}\,|x|_{\bar g}\right)\,\mathrm{d}\mu
\qquad\text{and}\qquad 
\kappa=\int_{S^{n-1}_1(e_n)}\eta(|x|_{\bar g})\,\mathrm{d}\bar\mu.
$$
Note that $\kappa>0$.
\begin{lem} 
	Passing to a subsequence, there holds
	$
 \kappa_k\to\kappa. 
	$ \label{test integral}
\end{lem}
\begin{proof}
	This follows from Proposition \ref{smooth local convergence}.
\end{proof} 
Let $f\in C_c^\infty(M)$ be a function whose support is disjoint from the boundary of $M$. We define
\begin{align*} 
\tilde u_k(x)=\begin{dcases}
	\alpha_k\,\lambda(\Sigma_k)^{-\tau}\,\eta\left(\lambda(\Sigma_k)^{-1}\,|x|_{\bar g}\right) \qquad&\text{if } x\notin B_1,\\
	0&\text{else}.
\end{dcases}
\end{align*} 
Here, $\alpha_k\in\mathbb{R}$ is chosen such that
\begin{align} \label{tilde u compatibility}
\int_{\Sigma_k}u_k+f+\tilde u_k\,\mathrm{d}\mu=0.
\end{align} 
Using Lemma \ref{first volume var} and Lemma \ref{test integral}, we see that $\alpha_k=O(1)$. Next, we define

\begin{align*}
\tilde v_k(x)=\begin{dcases} \beta_k\,\lambda(\Sigma_k)^{-1-\tau}\,\eta(\lambda(\Sigma_k)^{-1}\,|x|_{\bar g})\qquad&\text{if }x\notin B_1,
	\\0&\text{else},
	\end{dcases}
\end{align*} 
where $\beta_k\in\mathbb{R}$ is chosen such that
\begin{align} \label{tilde vcompatibility}  
	\int_{\Sigma_k} v_k+\tilde v_k+H\,(u_k+f+\tilde u_k)^2\,\mathrm{d}\mu=0.
\end{align} 
Using Lemma \ref{second volume var}, Lemma \ref{test integral}, and \eqref{H est}, we see that  $\beta_k=O(1)$. \\ \indent Note that there is $\varepsilon>0$ such that, for all $s\in(-\varepsilon,\varepsilon)$, $$\Sigma_k(s)=\{\exp_x\left(U_k(x,s)\,\nu(\Sigma_k)(x)\right):x\in \Sigma_k\}$$
where
$$
 U_k(x,s)=	s\,(u_k+f+\tilde u_k)(x)+\tfrac12\,s^2\,(v_k+\tilde v_k)(x)
$$
is an embedded hypersurface in $M$ that bounds a region  $\Omega_k(s)$.
\begin{lem}
	There holds \label{volume preserving}
	$$
	\frac{d}{ds}\bigg|_{s=0}|\Omega_k(s)|=	\frac{d^2}{ds^2}\bigg|_{s=0}|\Omega_k(s)|=0.
	$$
	\end{lem}  
\begin{proof}
	This follows from Lemma \ref{variation of area and volume} using \eqref{tilde u compatibility} and \eqref{tilde vcompatibility}.
\end{proof}
By Proposition \ref{graph estimate},  $\partial B_r$ intersects $\Sigma$ transversally for all $r>1$ sufficiently large. Increasing $r>1$ if necessary, we may arrange that  \begin{align} 
	\label{spt f}\operatorname{spt}(f)\subset B_r. \end{align}  \indent 
For Proposition \ref{gamma limit lemma} below, let $u\in C^\infty(\Sigma)$ be the function given by $u=g(\chi,\nu)$.
\begin{prop}There holds \label{gamma limit lemma} 
	$$
	\int_{ B_r\cap \Sigma} (|h|^2+{Ric}(\nu,\nu))\,(u+f)^2\,\mathrm{d}\mu\leq \int_{B_r\cap \Sigma}|\nabla (u+f)|^2\,\mathrm{d}\mu+O(r^{n-3-\tau}).
	$$
\end{prop}
\begin{proof}

By Lemma \ref{volume preserving}, the variation $\{\Sigma_k(s)\}_{|s|<\varepsilon}$ is volume-preserving up to second order. Since $\Omega_k$ is isoperimetric, it follows that
	$$
	\frac{d^2}{ds^2}\bigg|_{s=0}| \Sigma_k(s)|_{g}\geq 0.
	$$
	By Lemma \ref{variation of area and volume},
	$$
	\frac{d^2}{ds^2}\bigg|_{s=0}| \Sigma_k(s)|_{g}=\int_{\Sigma_k}H\,\ddot{ U}_k+H^2\,\dot{ U}^2_k+|\nabla \dot{ U}_k|^2-(|h|^2+{Ric}(\nu,\nu))\,\dot{ {U}}^2_k\,\mathrm{d}\mu.
	$$
	Note that $\tilde v_k(x)=\tilde u_k(x)=0$ if  $|x|_{\bar g} \leq \lambda(\Sigma_k)$.	Using also the curvature estimates \eqref{H est} and \eqref{hcirc est}, that is, $|x|_{\bar g}\,h=O(1)$,    we check that
	\begin{itemize}
		\item[$\circ$]$H\,\tilde v_k=O(\lambda(\Sigma_k)^{-2-\tau}),$
\item[$\circ$]$H^2\,\tilde u_k\,(u_k+f+\tilde u_k)=O(\lambda(\Sigma_k)^{-2-\tau})$,
\item[$\circ$]$g(\nabla \tilde u_k,\nabla (u_k+f+\tilde u_k))=O(\lambda(\Sigma_k)^{-2-\tau})$, and
\item[$\circ$]$(|h|^2+Ric(\nu,\nu))\,\tilde u_k\,(u_k+f+\tilde u_k)=O(\lambda(\Sigma_k)^{-2-\tau})$.		
	\end{itemize}	 
It follows that
	\begin{align*}
	\frac{d^2}{ds^2}\bigg|_{s=0}| \Sigma_k(s)|_{g}&=\int_{\Sigma_k}H\,v_k+H^2\,(u_k+f)^2+|\nabla (u_k+f)|^2-(|h|^2+{Ric}(\nu,\nu))\,(u_k+f)^2\,\mathrm{d}\mu\\&\qquad +O(\lambda(\Sigma_k)^{n-3-\tau}).
	\end{align*} 
 \indent Note that $f=0$ on $\Sigma_k\setminus B_r$. Using Lemma \ref{area growth estimate}, we have
	$$
	\int_{\Sigma_k\setminus B_{r}} {Ric}(\nu,\nu)\,u_k^2\,\mathrm{d}\mu=O(1)\,\int_{\Sigma_k\setminus B_r}|x|_{\bar g}^{-2-\tau}\,\mathrm{d}\bar \mu=O(\lambda(\Sigma_k)^{n-3-\tau})+O(r^{n-3-\tau}).
	$$
\indent 	Similarly, using also Lemma \ref{geometric expansions}, the curvature estimates \eqref{H est} and \eqref{hcirc est}, and \eqref{spt f}, we check that
	\begin{align*} 
	&\int_{\Sigma_k\setminus B_{r}}H\,v_k+H^2\,u_k^2+|\nabla u_k|^2-|h|^2\,u_k^2\,\mathrm{d}\mu\\&\qquad =\int_{\Sigma_k\setminus B_{r}}\bar H\,\bar v_k+\bar H^2\,\bar u_k^2+|\bar\nabla\bar u_k|_{\bar g}^2-|\bar h|^2_{\bar g}\,\bar u_k^2\,\mathrm{d}\bar \mu
 + O(\lambda(\Sigma_k)^{n-3-\tau})+O(r^{n-3-\tau})
	\end{align*} 
where $\bar u_k,\,\bar v_k \in C^\infty(\Sigma_k)$ are  given by
$
\bar u_k=\bar g(e_n,\bar\nu)$ and $ \bar v_k=-\bar h(e_n^{\bar\top},e_n^{\bar\top}).
$ For instance, $$H\,v_k=\bar H\,\bar v_k+O(|x|_{\bar g}^{-2-\tau})$$ and 
$$
\int_{\Sigma_k\setminus B_r}|x|_{\bar g}^{-2-\tau}\,\mathrm{d}\bar\mu=O(\lambda(\Sigma_k)^{n-3-\tau})+O(r^{n-3-\tau}). 
$$\indent 
 Since $\Sigma_k\to\Sigma$ locally smoothly, we obtain, using also the integration by parts formula from Lemma \ref{euclidean lemma},	
$$
\int_{\Sigma_k\setminus B_{r}}\bar H\,\bar v_k+\bar H^2\,\bar u_k^2+|\bar\nabla\bar u_k|_{\bar g}^2-|\bar h|^2_{\bar g}\,\bar u_k^2\,\mathrm{d}\bar \mu
=O(1)\,\int_{\partial B_{r}\cap \Sigma} |\bar h(\Sigma)|_{\bar g}\,|e_n^{\bar\top}|_{\bar g}\,\mathrm{d}\bar l.
$$
\indent By Proposition \ref{graph estimate},  $|\bar h(\Sigma)|_{\bar g}=O(|y|_{\bar g}^{-1-\tau/2})$, $e_n^{\bar \top}=O(|y|_{\bar g}^{-\tau/2})$, and $| \partial B_r\cap \Sigma|_{\bar g}=O(r^{n-2})$. Thus, 
\begin{align*} 
\int_{\partial B_{r}\cap \Sigma} |\bar h(\Sigma)|_{\bar g}\,|e_n^{\bar\top}|_{\bar g}\,\mathrm{d} \bar l=O(r^{n-3-\tau}).
\end{align*} 
 \indent The assertion follows from the above estimates using \eqref{tau condition}, \eqref{H est}, and that $\Sigma_k\to\Sigma$ locally smoothly.
\end{proof}

\begin{prop} \label{strictly stable} 
Let $f\in C^\infty_c(\Sigma)$.
There holds
$$
\int_{\Sigma} (|h|^2+{Ric}(\nu,\nu))\,(1+f)^2\,\mathrm{d}\mu\leq \int_{\Sigma}|\nabla f|^2\,\mathrm{d}\mu.
$$
\end{prop}
\begin{proof}
We may assume that the support of $f$ is disjoint from the boundary of $M$, which is stable. Letting $r \to\infty$ in Proposition \ref{gamma limit lemma} and using Corollary \ref{l2 second fundamental form} and the dominated convergence theorem, we conclude that 
 \begin{align} \label{fk}  
 \int_{\Sigma} (|h|^2+{Ric}(\nu,\nu))\,(u+f)^2\,\mathrm{d}\mu\leq \int_{\Sigma}|\nabla (u+f)|^2\,\mathrm{d}\mu
 \end{align} 
 for all $f\in C^\infty_c(\Sigma)$. 
Fix  a function $\psi\in C^\infty_c(\mathbb{R})$ with $\psi(t)=1$ if $t\leq 1$. Let $\{f_k\}_{k=1}^\infty$ be the sequence of functions $f_k\in C_c^\infty(\Sigma)$ given by
 $$
 f_k(x)=\begin{dcases}1-u(x)+f(x) \qquad &\text{if } x\in B_1,\\
 \psi(k^{-1}\,|x|_{\bar g})\,(1-u(x))+f(x)&\text{else}.
 \end{dcases}.
 $$ Note that, on $\Sigma\setminus B_1$,  
 \begin{align} \label{dominated convergence} 
 |\nabla f_k|^2\leq O(1)\,\big( |x|_{\bar g}^{-2}\,|u-1|^2+|\nabla u|^2+ |\nabla f|^2\big).
 \end{align} 
 By Proposition \ref{graph estimate},
 $$|u-1|+|x|_{\bar g}\,|\nabla u|=O(|x|_{\bar g}^{-\tau/2}).
 $$ 
It follows that the right-hand side of \eqref{dominated convergence} is integrable. The assertion  now follows from applying \eqref{fk} with $f_k$ in place of $f$, letting $k\to\infty$, and using the dominated convergence theorem.
\end{proof}
\begin{rema}
	Proposition \ref{strictly stable} holds for all $f\in H^1_{loc}(\Sigma)$ for which there are $f_1,\,f_2,\hdots\in C^\infty_c(\Sigma)$ with
	$
	\nabla f_k\to\nabla f\text{ in } L^2(\Sigma)$, as can be seen using the Sobolev inequality.
\end{rema}
The following Proposition is a consequence of  Proposition \ref{strictly stable} using  arguments of R.~Schoen and S.-T.~Yau \cite{pmt} and of A.~Carlotto \cite{Carlottocalcvar}.
\begin{prop} Suppose that $R\geq 0$ along the non-compact component of $\Sigma$. Then \label{flat flat} 
the non-compact component of	$\Sigma$ is totally geodesic and isometric to flat $\mathbb{R}^{n-1}$. Moreover, there holds $R={Ric}(\nu,\nu)=0$ along the non-compact component of $\Sigma$. 
\end{prop}
\begin{proof}
	Let $\Sigma^0$ be the non-compact component of $\Sigma$. 	Arguing as in the proof of \cite[Theorem 2]{Carlottocalcvar} using Lemma \ref{vanishing mass} and the stability with respect to asymptotically constant variations asserted in Proposition \ref{strictly stable}, we see that the scalar curvature of   $(\Sigma^0,g|_{\Sigma^0})$ vanishes. By the rigidity statement of the positive mass theorem \cite[Lemma 3 and Proposition 2]{schoenconformal},  $(\Sigma^0,g|_{\Sigma^0})$ is isometric to flat $\mathbb{R}^{n-1}$. \\ \indent  As explained in \cite[p.~35]{montecatini},  Proposition \ref{strictly stable} implies that
	$$
	\int_{\Sigma^0} |h|^2+{Ric}(\nu,\nu)\,\mathrm{d}\mu\leq 0.
	$$
	Using the Gauss equation and that $\Sigma^0$ is scalar flat, we have
	$$
	|h|^2+{Ric}(\nu,\nu)=\frac12\,(|h|^2+R).
	$$
	In conjunction with $R\geq 0$, we obtain that $h=0$ and that $R={Ric}(\nu,\nu)=0$ along $\Sigma^0$.
\end{proof}

\section{Proof of Theorem \ref{main result 2}}
In this section, $(M,g)$ is a Riemannian manifold of dimension  $3< n\leq 7$ that is asymptotically flat of rate $\tau>n-3$ and which has non-negative scalar curvature and positive mass. \\ \indent We defer the proof of the following lemma to the end of this section.
\begin{lem} \label{metric perturbation}
 Let $p_1,\,p_2$ be points in the interior of $M$. There exists  $U$ open and compactly contained in the interior of $M$ with $p_1,\,p_2\in U$  such that the following holds. For all $r>0$ sufficiently small, there exist $W\subset U$ open with $p_1,\,p_2\in W$ and a   family $\{g_t\}_{t\in(0,1)}$ of Riemannian metrics $g_t$ on $M$ such that
	\begin{align}  
			\label{g properties 2} &\circ\qquad   g_t \to g \text{ smoothly as } t\to  0,\\
	\label{g properties 1} &\circ\qquad  g_t=g  \text{ in }  M\setminus W,\\
	\label{g properties 5} &\circ\qquad  g_t<g \text{ in }W,\text{ and}\\
	\label{g properties 3} &\circ\qquad  R(g_t)>0 \text{ in } \{x\in W:\operatorname{dist}_g(x,p_2)> r\}. 
\end{align} 
\end{lem}
Let $\Omega_1,\,\Omega_2,\hdots\subset M$ be isoperimetric regions   with $| \Omega_k|\to\infty$ and such that $\Omega_k$ converges locally smoothly to a region $\Omega\subset M$ whose boundary $\Sigma=\partial \Omega$ is non-compact and area-minimizing.
\begin{prop}  \label{many area minimizing}  Let $p$ be a point in the interior of $M$. There exist  $U$ open and compactly contained in the interior of $M$, Riemannian metrics $\tilde g_1,\,\tilde g_2,\hdots $ on $M$, and regions  $ \,\tilde \Omega_1,\,\tilde \Omega_2,\hdots$, and $\Omega_p\subset M$  with the following properties.
	\begin{itemize}
		\item[$\circ$] $\operatorname{spt}(\tilde g_k-g)\subset U$ for all $k$ and $\tilde g_k\to g$ smoothly.
		\item[$\circ$] $\tilde \Omega_k$ is an isoperimetric region in $( M,\tilde g_k$) and $|\tilde \Omega_k|_{\tilde g_k}\to\infty$.
		\item[$\circ$] $\tilde \Omega_k\to  \Omega_p$ locally smoothly.
		\item[$\circ$] $\partial \Omega_p$ is a non-compact area-minimizing boundary in $(M,g)$ that is stable with respect to asymptotically constant variations and $p\in \partial\Omega_p$.
	\end{itemize}
\end{prop}
\begin{proof}
	We first assume that the boundary of $M$ is empty. \\ \indent 
  We choose $p_1\in \Sigma$ and let $p_2=p$. Let $U\subset M$ be the  open set from Lemma \ref{metric perturbation} and $r>0$ be small enough so that the conclusion of Lemma \ref{metric perturbation} holds. According to Lemma \ref{metric perturbation}, there is  $W\subset U$ open and a family $\{g_t\}_{t\in(0,1)}$ of Riemannian metrics $g_t$ on $M$ satisfying  \eqref{g properties 2}, \eqref{g properties 1}, \eqref{g properties 5}, and \eqref{g properties 3}. \\ \indent Let    $t\in(0,1)$. Recall the isoperimetric profile $A_g$ defined in \eqref{isoperimetric profile}. Let  $c(t)=|W|_g-|W|_{g_t}$. Note that
  \begin{itemize}
  	\item[$\circ$] $c(t)>0$ by \eqref{g properties 5} and
  	\item[$\circ$] $c(t)=o(1)$ as $t\to  0$ by \eqref{g properties 2}.
  \end{itemize}
  Since $\partial \Omega_k\to \Sigma$ locally smoothly and $\Sigma\cap W\neq \emptyset$, it follows from $\eqref{g properties 5}$ that  there is $\varepsilon(t)>0$ such that, for all $k$ sufficiently large,
    \begin{align} \label{area comparison} |\partial \Omega_k|_{g_t}+\varepsilon(t)<|\partial \Omega_k|_{g}.
     \end{align} 
   Using Lemma \ref{profile properties}, we see that
	\begin{align} \label{profile growth estimate}
		|\partial \Omega_k|_{g}=A_g(|\Omega_k|_{g})\leq A_g(V)+o(1)
	\end{align} 
for any amount of volume $V$ such that    $|\Omega_k|_{g}-c(t)\leq V
\leq |\Omega_k|_{g}+c(t)$. 
	By Lemma \ref{existence iso}, for every $k$ sufficiently large,  there exists an isoperimetric region $\Omega_k(t)\subset M$ with respect to $g_t$ such that $|\Omega_k(t)|_{g_t}=|\Omega_k|_{g_t}$. \\ \indent We claim that there is  $\tilde W\Subset W$ open such that, for all $k$ sufficiently large,
	\begin{align} \label{non empty intersection} \tilde W\cap \partial \Omega_k(t)\neq\emptyset. 
	\end{align}  
	 To see this, assume the contrary. Passing to a subsequence and using \eqref{g properties 1} and Lemma \ref{area growth estimate}, we see that 
	\begin{align} \label{comparison} 
	|\partial \Omega_k(t)|_{g_t}=|\partial \Omega_k(t)|_{g}+o(1).
\end{align} 
	Using \eqref{g properties 5}, we have
	$$
	|\Omega_k(t)|_g\leq |\Omega_k(t)|_{g_t}+c(t)=|\Omega_k|_{g_t}+c(t)\leq |\Omega_k|_g+c(t).
	$$
	Similarly,
	$
	|\Omega_k(t)|_g\geq  |\Omega_k|_g-c(t).
	$
Using \eqref{profile growth estimate} with $V=|\Omega_k(t)|_g$, we conclude that $$|\partial \Omega_k(t)|_{g}\geq |\partial \Omega_k|_{g}-o(1).$$ This is not compatible with \eqref{area comparison} and \eqref{comparison}. \\ \indent 
	By Proposition \ref{smooth local convergence}, $ \Omega_k(t)$ converges locally smoothly to a region $\Omega(t)$ with non-compact boundary $\Sigma(t)$ that is area-minimizing with respect to $g_t$. Using \eqref{non empty intersection}, we see that $\Sigma(t)$ intersects the closure of  $\tilde W$. 
	Using Proposition  \ref{flat flat} and \eqref{g properties 3}, it follows that, in fact,   $ \{x\in M:\operatorname{dist}(x,p)<r\}\cap \Sigma(t)\neq \emptyset$. \\ \indent 
	Passing to a subsequence, we see that, as $t\to 0$, $\Omega(t)$ converges locally smoothly to a a region $\Omega_p(r)\subset M$ with non-compact boundary $\Sigma_p(r)$ that is  area-minimizing with respect to $g$  such that $ \{x\in M:\operatorname{dist}(x,p)\leq r\}\cap \Sigma_p(r)\neq\emptyset$. Moreover, passing to a subsequence, we see that, as $r\to 0$, $\Omega_p(r)$ converges locally smoothly to a region $\Omega_p$ with non-compact boundary $\Sigma_p$ that is area-minimizing with respect to $g$ and such that $p\in  \Sigma_p$. \\ \indent  
	By a diagonal argument, we see that there are Riemannian metric $\tilde g_1,\,\tilde g_2,\hdots $ on $M$ and regions  $ \,\tilde \Omega_1,\,\tilde \Omega_2,\hdots\subset M$ with  the asserted properties.
Repeating the argument that led to Proposition \ref{strictly stable}, we see that $\Sigma_p$ is stable with respect to asymptotically constant variations in the sense of Proposition \ref{strictly stable}.  This completes the proof in the case where the boundary of $M$ is empty.  \\ \indent
In the case where the boundary of $M$ is an outermost minimal surface, we note that the boundary of $M$ is also an outermost minimal surface with respect to $g_t$ for all sufficiently small $t>0$. The rest of the proof only requires formal modifications.
\end{proof}
\begin{proof}[Proof of Theorem \ref{main result 2}] 
	Let $(M,g)$ be asymptotically flat of rate $\tau>n-3$ with non-negative scalar curvature. Suppose that there are isoperimetric regions $\Omega_1,\,\Omega_2,\hdots$ in $(M,g)$ with $|\Omega_k|\to \infty$
	 such that $\Omega_k$ converges locally smoothly to a region $\Omega\subset M$ whose boundary $\Sigma=\partial \Omega$ is non-compact and area-minimizing.
	\\ \indent
	Suppose for a contradiction, that $(M,g)$ has positive mass $m$. \\ \indent 
	We first assume that the boundary of $M$ is empty. \\ \indent 
Our goal is to show that the curvature tensor ${Rm}$ vanishes everywhere.\\ \indent  Let $p\in M$. We first assume that $p\in \Sigma$. Using Proposition \ref{many area minimizing}, Proposition \ref{flat flat}, and standard convergence results from geometric measure theory, we see that there are mutually distinct connected non-compact   area-minimizing boundaries $\Sigma_1,\,\Sigma_2,\hdots\subset M$ such that
	\begin{itemize}
		\item[$\circ$] $\Sigma_k$ is isometric to flat $\mathbb{R}^{n-1}$ for all $k$,
		\item[$\circ$] $\Sigma_k$ is totally geodesic for all $k$, and
		\item[$\circ$] $\Sigma_k\to \Sigma$ locally smoothly.
	\end{itemize}   Let $W,\,X,\,Y,\,Z$ be tangent fields along $\Sigma$. Note that
\begin{align} \label{first flat}  
{Rm}(W,\,X,\,Y,\,Z)=0.
\end{align} 
Indeed, this follows from the Gauss equation, using that $\Sigma$ is isometric to flat $\mathbb{R}^{n-1}$ and totally geodesic. Similarly, using the Codazzi equation, we obtain
\begin{align} \label{second flat}  
{Rm}(W,\,X,\,Y,\nu)=0.
\end{align} 
By the Hopf maximum principle, $\Sigma$ and $\Sigma_k$ are either disjoint or  intersect transversally. In the latter case, ${Rm}=0$ along the intersection. We may therefore assume that there is a neighborhood $U$ of $p$ such that $U\cap \Sigma\cap \Sigma_k=\emptyset$ for all $k\geq 1$. \\ \indent 
Since $\Sigma$ and $\Sigma_k$ are totally geodesic, the components of $\Sigma\cap \Sigma_k$ are totally geodesic and therefore hyperplanes of $\Sigma$. Since $\Sigma$ and  $\Sigma_k$ are embedded, the components of $\Sigma\cap \Sigma_k$ must be parallel. It follows that, passing to a subsequence, there are three cases.
\begin{itemize}
	\item[$\circ$] For every $k$, $\Sigma\cap \Sigma_k=\emptyset$.
	\item[$\circ$] For every $k$, $\Sigma\cap \Sigma_k$ consists of a single hyperplane in $\Sigma$.
	\item[$\circ$] For every $k$, $\Sigma\cap \Sigma_k$ consists of at least two parallel hyperplanes in $\Sigma$.
\end{itemize}
In the second and the third case, let $\Sigma^0_k$ be the closure of the component of $\Sigma\setminus \Sigma_k$ that contains $p$. Note that $\Sigma^0_k$ is isometric to a half-space in the second case and isometric to a slab in the third case. Since $U\cap \Sigma\cap \Sigma_k=\emptyset$, it follows that
$
\liminf_{k\to\infty}\operatorname{dist}(p,\partial \Sigma^0_k)>0.
$\\\indent
In the first case, we may argue exactly as in \cite[p.~993]{CCE}. Specifically, we may represent $\Sigma_k$ as the graph of a positive function $u_k$ over larger and larger domains, exhausting  $\Sigma$ as $k\to\infty$. By the Harnack inequality \cite[\S8.8]{gilbargtrudinger}, $u_k$ is bounded by a multiple of $u_k(p)$ locally in $\Sigma$ as $k\to\infty$. 
Using the first variation of the second fundamental form and proceeding as in \cite[p.~333]{simonstrict},  
we obtain a positive function $f\in C^\infty(\Sigma)$  such that
\begin{align} \label{f PDE} 
	\nabla_{X,Y}^2f+{Rm}(X,\nu,\nu,Y)\,f=0
\end{align} 
for all tangent fields $X,\,Y$ of $\Sigma$. Tracing \eqref{f PDE} and using that ${Ric}(\nu,\nu)=0$ along $\Sigma$, we see that $f$ is harmonic. By the Liouville theorem,  $f$ is equal to a  constant. It follows that ${Rm}(X,\nu,\nu,Y)=0$. In conjunction with \eqref{first flat} and \eqref{second flat}, we conclude that ${Rm}=0$ along $\Sigma$ and in particular at $p$.
\\ \indent 
In the second case, if $\limsup_{k\to\infty}\operatorname{dist}(p,\partial \Sigma^0_k)=\infty$, we may argue as in the first case. If $\operatorname{dist}(p,\partial \Sigma^0_k)=O(1)$, then, passing to a subsequence, we may assume that $\Sigma_k^0$ converges locally smoothly to a half-space $\Sigma^0\subset \Sigma$. As before, we may represent $\Sigma^0_k$ as the graph of a smooth function $u_k$ over larger and larger domains, exhausting $\Sigma^0$ as $k\to\infty$.  Arguing as in the first case, using also the Boundary Harnack inequality (see, e.g., \cite[Theorem 11.5]{caffarelli}), we obtain a harmonic function $f\in C^\infty(\Sigma^0)$ that satisfies \eqref{f PDE} in $\Sigma^0$, $f=0$ on $\partial \Sigma^0$, and $f>0$ away from $\partial \Sigma^0$. By \cite[Theorem I]{shortproofs}, $f$ is a linear function. As before, it follows that ${Rm}=0$ along $\Sigma$ and in particular at $p$. \\ \indent 
Finally, suppose for a contradiction that the third case arises. We may assume that, passing to a subsequence, $\Sigma_k^0$ converges locally to a slab $\Sigma^0\subset \Sigma$. As in the previous case, we obtain a harmonic function $f\in C^\infty(\Sigma^0)$ that satisfies \eqref{f PDE} in $\Sigma^0$, $f=0$ on $\partial \Sigma^0$, and $f>0$ away from $\partial \Sigma^0$.  Using \eqref{f PDE}, we see that $|\nabla f|$ is bounded from above by a positive constant on $\partial \Sigma^0$. This is not compatible with Lemma \ref{harmonic on slab}. \\ \indent 
 Now, let $p\in M\setminus \Sigma$. By Proposition \ref{many area minimizing}, there exists a non-compact area-minimizing boundary $\Sigma_p$ with $p\in \Sigma_p$ that is stable with respect to asymptotically constant variations. Repeating the argument above with $\Sigma_p$ in place of $\Sigma$, we see that $Rm=0$ along $\Sigma_p$. Since $(M,g)$ is asymptotically flat, it follows that $(M,g)$ is isometric to flat $\mathbb{R}^n$.  This is not compatible with $m>0$.  
This completes  the proof in the case where the boundary of $M$ is empty. \\ \indent The case where the boundary of $M$ is an outermost minimal surface only requires formal modifications.
	\end{proof} 
\begin{proof}[Proof of Corollary \ref{main result}]
	Let $(M,g)$ be asymptotically flat of rate $\tau>n-3$  with non-negative scalar curvature and positive mass $m$. Suppose, for a contradiction, that there are isoperimetric regions $\Omega_1,\,\Omega_2,\hdots$ in $(M,g)$ with $|\Omega_k|\to \infty$ and a compact set $K\subset M$ disjoint from the boundary of $M$  such that $K\cap\partial \Omega_k\neq \emptyset$ for all $k$.
	\\ \indent By Proposition \ref{smooth local convergence},  $\Omega_k$ converges locally smoothly to a region $\Omega\subset M$ whose boundary $\Sigma=\partial \Omega$ is non-compact and area-minimizing. By Theorem \ref{main result 2}, $(M,g)$ is isometric to flat $\mathbb{R}^n$. This is not compatible with $m>0$.
\end{proof}
\begin{proof}[{Proof of Lemma \ref{metric perturbation}}]
Arguing as in \cite[p.~21]{chaoli}, we see that there is $r_0>0$ depending only on $(M,g)$ with the following property. Given $0<r<r_0$ and $q\in M$, there exists a function $v_{r,q}\in C^\infty(M)$ satisfying
	\begin{align}  
		\label{v properties 1} &\circ\qquad  v_{r,q}=0  \text{ in }  \{x\in M:\operatorname{dist}_g(x,q)\geq 6\,r\},\qquad\qquad\qquad\qquad\qquad\\
		\label{v properties 2} &\circ\qquad  v_{r,q}<0  \text{ in }  \{x\in M:\operatorname{dist}_g(x,q)< 6\,r\},\text{ and}\\
		\label{v properties 3} &\circ\qquad  \Delta v_{r,q}<0 \text{ in } \{x\in M:r<\operatorname{dist}_g(x,q)<6\,r\}.		
	\end{align} 
	Decreasing $r>0$ if necessary, we may choose points $q_1,\hdots,q_N\in M$ with $q_1=p_1$ and $q_N=p_2$ such that  
	\begin{align}
		\label{ball property 2} &\{x\in M:\operatorname{dist}(x,q_i)\leq r\}\subset \{x\in M:3\,r\leq \operatorname{dist}(x,q_{i+1})\leq 5\,r\} \text{ for all }i\leq N-1\text{ and}\\ \notag
		&\{x\in M:\operatorname{dist}(x,q_i)\leq 6\,r\}\text{ is disjoint from the boundary of }M\text{ for all }1\leq i\leq N; 
	\end{align}
see Figure \ref{perturbation}.
 	\begin{figure}\centering
	\includegraphics[width=0.7\linewidth]{perturbation.pdf}
	\caption{An illustration of the construction in Lemma \ref{metric perturbation}. The open set $W$ is bounded by the solid black line. The function $v$ is super-harmonic outside of the hatched region. }
	\label{perturbation}
\end{figure}
	Define $a_1,\hdots,a_N$ recursively by $a_1=1$ and, for $i= 2,\hdots,N$,
	$$
	a_i=1+\frac{\sup\{\Delta v_{r,q_{i-1}}(x):\operatorname{dist}(x,q_{i-1})<r\}}{|\inf\{\Delta v_{r,q_i}(x):3\,r\leq \operatorname{dist}(x,q_{i})\leq 5\,r\}|}\,a_{i-1}.
	$$
	Let
	$$
	W=\bigcup_{i=1}^N\{x\in M:\operatorname{dist}(x,q_i)< 6\,r\}
	\qquad \text{and} \qquad v=\sum_{i=1}^N a_i\,v_{r,q_i}.
	$$
	Note that
	\begin{itemize}
		\item[$\circ$] $v=0$ in $M\setminus W$ by \eqref{v properties 1},
		\item[$\circ$] $v<0$ in $W$ by \eqref{v properties 2}, and 
		\item[$\circ$] $\Delta v<0$ in $\{x\in W:\operatorname{dist}(x,p_2)\geq r\}$ by \eqref{v properties 3} and \eqref{ball property 2}.
	\end{itemize}\indent  \indent 
	For all $\delta>0$ sufficiently small, the Riemannian metrics $$g_t=(1+t\,\delta\,v)^{\frac{4}{n-2}}\,g$$ where $t\in(0,1)$ satisfy  the properties asserted in the lemma.\end{proof} 

\section{Proof of Theorem \ref{counterexample} }
\label{unstable section}
  Let $3<n\leq 7$ and $(M,g)$ be spatial Schwarzschild with mass $m=2$, i.e., 
  \begin{align*}
  M=\{x\in\mathbb{R}^n:|x|_{\bar g}\geq 1\}\qquad \text{and} \qquad g=\phi^\frac{4}{n-2}\,\bar g
  \end{align*}  where $\phi\in C^\infty(M)$ is given by $$
\phi(x)=1+| x|_{\bar g}^{2-n}.
$$
The goal of this section is to show that there are infinitely many mutually disjoint non-compact area-minimizing hypersurfaces in $(M,g)$. \\
 \indent  Let $t,\,s>0$. For the next lemma, we choose the following orientations.
\begin{itemize}
	\item[$\circ$]  $M\cap(\mathbb{R}^{n-1}\times\{t\})$ is oriented by the normal in direction of $-e_n$. 
	\item[$\circ$]$M\cap(S^{n-2}_s(0)\times\mathbb{R})$ is  oriented by the normal pointing away from the vertical axis.
\end{itemize}

\begin{lem} Let $t,\,s>0$. The following hold. \label{barriers} 
	\begin{itemize}
		\item[$\circ$]  $M\cap (\mathbb{R}^{n-1}\times\{t\})$ is strictly mean convex.
		\item[$\circ$] $M\cap (S^{n-2}_s(0)\times\mathbb{R})$ is strictly mean convex.
		\item[$\circ$] $S^{n-1}_1(0)$ is minimal.
	\end{itemize}
\end{lem}

\indent Let  $r>2,$ $z>0$,  and $\Sigma_{r,z}$ be the least area hypersurface in $(M,g)$ with $$\partial \Sigma_{r,z}= S^{n-2}_r(0)\times\{z\}.$$ 
By the convex hull property and Lemma \ref{barriers}, $\Sigma_{r,z
}$ is a vertical graph with finite slope near $\partial \Sigma_{r,z}$.
\begin{lem} \label{axial symmetry} 
	$\Sigma_{r,z}$ is axially symmetric with respect to $e_n$.
\end{lem}
\begin{proof}
	Let $\pi\subset\mathbb{R}^{n}$ be a hyperplane through the origin with $e_n\in\pi$. Let $\Pi:\mathbb{R}^n\to\mathbb{R}^n$ be the reflection through $\pi$, $\mathbb{R}^n_{\pm}$ be the components of $\mathbb{R}^n\setminus \pi$, and $\Sigma_{r,z}^{\pm}=\mathbb{R}^n_{\pm}\cap\Sigma_{r,z}$. Note that $M\cap\pi$ is minimal in $(M,g)$. By the Hopf maximum principle, it follows that $\Sigma_{r,z}$ and $\pi$ intersect transversally. In particular, 
	\begin{align} \label{area comparison 2} 
		|\Sigma_{r,z}^+|+|\Sigma_{r,z}^-|= |\Sigma_{r,z}|. \end{align}  Moreover,  $\tilde \Sigma_{r,z}=\Sigma_{r,z}^+\cup \Pi(\Sigma_{r,z}^+)\cup (\Sigma_{r,z}\cap \pi)$ is a Lipschitz surface with $\partial\tilde \Sigma_{r,z}=\partial \Sigma_{r,z}$. Using that $g$ is rotationally symmetric, the  area-minimizing property of $\Sigma_{r,z}$, and \eqref{area comparison 2}, we conclude that  $2\,|\Sigma_{r,z}^+|=2\,|\Sigma_{r,z}^-|= |\Sigma_{r,z}|$. It follows that   $|\tilde \Sigma_{r,z}|=|\Sigma_{r,z}|$. Using again that $\Sigma_{r,z}$ is area-minimizing, we see that $\tilde \Sigma_{r,z}$ intersects $\pi$ orthogonally. In particular, $\tilde \Sigma_{r,z}$ is smooth. By unique continuation \cite[p.~235]{uniquecontinuation},  $\tilde \Sigma_{r,z}=\Sigma_{r,z}$. The assertion follows. \end{proof}
\begin{lem}
	$\Sigma_{r,z}\setminus \partial \Sigma_{r,z}$ is contained in the cylinder  \label{cylinder}
	$
	B^{n-1}_r(0)\times(z,\infty).
	$
\end{lem}
\begin{proof}
This follows from Lemma \ref{barriers} and  the maximum principle.
\end{proof}
\begin{lem} \label{graph} 
	$\Sigma_{r,z}$ is a vertical graph over $\{y\in\mathbb{R}^{n-1}:|y|_{\bar g}\leq r \}$.
\end{lem}
\begin{proof}
	Suppose, for a contradiction, that $\Sigma_{r,z}\setminus \partial \Sigma_{r,z}$ is not a graph over $B^{n-1}_{r}(0)$. Using Lemma \ref{axial symmetry} and Lemma \ref{cylinder}, it follows that $\Sigma_{r,z}$ touches $S^{n-2}_s(0)\times\mathbb{R}$ from the inside for some $s\in(0,r)$. This is not compatible with Lemma \ref{barriers} and the maximum principle.
\end{proof}

By Lemma \ref{axial symmetry} and Lemma \ref{graph}, there is $f_{r,z}\in C^\infty(\mathbb{R})$ with $f_{r,z}(r)=z$ and $f'_{r,z}(0)=0$ such that  $$\Sigma_{r,z}=\{\left(y,f_{r,z}(|y|_{\bar g})\right):y\in\mathbb{R}^{n-1}\text{ with }|y|_{\bar g}\leq r \}.$$ 
\begin{lem} \label{f ode}
	There holds, for all $t\in(0,r)$,
	$$
	\bigg(t^{n-2}\frac{f'_{r,z}(t)}{\sqrt{1+f'_{r,z}(t)^2}}\bigg)'=-\frac{2\,(n-1)}{1+(t^2+f_{r,z}(t)^2)^{-\frac{n-2}{2}}}\,\frac{t^{n-2}}{(t^2+f_{r,z}(t)^2)^{\frac{n}{2}}}\,\frac{f_{r,z}(t)-f_{r,z}'(t)\,t}{\sqrt{1+f'(t)^2}}.
	$$
\end{lem}
\begin{proof}
	This follows from the conformal transformation formula of the mean curvature,
	\begin{align} \label{conformal transformation}  
	\phi^{\frac{2}{n-2}}\,H=\bar H+\frac{2\,(n-1)}{n-2}\phi^{-1}\,\bar D_{\bar\nu}\phi,
	\end{align} 
	using that $\Sigma_{r,z}$ is minimal. Here, $\bar\nu$ is the Euclidean unit normal of $\Sigma_{r,z}$ pointing in direction of $e_n$. 
\end{proof}

\begin{lem}  There holds \label{height bound}
	\begin{align*}
0\leq	\sup_{z>0}	\limsup_{r\to\infty}\left[ f_{r,z}(4\,(n-1))-z\right]<\infty.
	\end{align*}
Moreover, as $z\to 0$,  
	$$ \limsup_{r\to\infty}f_{r,z}(z^{-2})=z+o(z).
	$$
\end{lem}
\begin{proof}
	Using Lemma \ref{barriers} and the maximum principle, we see that $f'_{r,z}(t)\leq 0$ for all $t\in(0,r)$.
Using Lemma \ref{f ode} and the Cauchy-Schwarz inequality, we have, for all $t\in(0,r)$,
	\begin{align*}
		\bigg(t^{n-2}\,\frac{f_{r,z}'(t)}{\sqrt{1+f_{r,z}'(t)^2}}\bigg)'\geq-2\,(n-1)\,\frac{t^{n-2}}{(t^2+f_{r,z}(t)^2)^{\frac{n-1}{2}}}.
	\end{align*}
Integrating, we obtain
	\begin{align} \label{first odeineq}
	\frac{f_{r,z}'(t)}{\sqrt{1+f_{r,z}'(t)^2}}\geq-2\,(n-1)\,t^{2-n}\,\int_0^t \frac{s^{n-2}}{(s^2+f_{r,z}(s)^2)^\frac{n-1}{2}}\,\mathrm{d}s.
\end{align}
Note that $s^2+f_{r,z}(s)^2\geq 1$. Consequently, for all  $t\geq 1$,
\begin{align} \label{second odeineq} 
\int_0^t \frac{s^{n-2}}{(s^2+f_{r,z}(s)^2)^\frac{n-1}{2}}\,\mathrm{d}s\leq \int_{0}^1\,\mathrm{d}s+\int_1^t\,s^{-1}\,\mathrm{d}s =1+\log(t). 
\end{align} 
It follows that 
$$
2\,(n-1)\,t^{2-n}\,\int_0^t \frac{s^{n-2}}{(s^2+f_{r,z}(s)^2)^\frac{n-1}{2}}\,\mathrm{d}s\leq \frac12
$$
provided that $t\geq 4\,(n-1)$.  Note that
$$
\frac{1}{2}\,f'_{r,z}(t)\geq \frac{f'_{r,z}(t)}{\sqrt{1+f'_{r,z}(t)^2}}
\qquad\text{provided that}\qquad 
\frac{f'_{r,z}(t)}{\sqrt{1+f'_{r,z}(t)^2}}\geq -\frac12.
$$
In conjunction with \eqref{first odeineq} and \eqref{second odeineq}, we obtain
$$
f_{r,z}'(t)\geq -4\,(n-1)\,t^{2-n}\,(\log(t)+1)
$$
provided that $t\geq 4\,(n-1)$. Since $f_{r,z}(r)=z$, we conclude that
\begin{align*} 
z\leq f_{r,z}(4\,(n-1))&\leq z+4\,(n-1)\,\int^r_{4\,(n-1)}t^{2-n}\,(\log(t)+1)\,\mathrm{d}t
\end{align*} 
for all $r\geq 4\,(n-1)$. Note that, since $n\geq 4$, 
\begin{align*} 
\int^r_{4\,(n-1)}t^{2-n}\,(\log(t)+1)\,\mathrm{d}t\leq \int^\infty_{4\,(n-1)}t^{2-n}\,(\log(t)+1)\,\mathrm{d}t<\infty.
\end{align*} 
Likewise, we obtain,  as $z\to 0$,
$$
\limsup_{r\to\infty} f_{r,z}(z^{-2})\leq z+4\,(n-1)\,\int^\infty_{z^{-2}} t^{2-n}\,(\log(t)+1)\,\mathrm{d}t=z+o(z).
$$
\indent The assertion follows.
\end{proof}
\begin{lem} \label{convergence to boundary} 
Let $z>0$. There exists a sequence $\{r_k\}_{k=1}^\infty$ with $r_k\to\infty$ such that $\Sigma_{z,r_k}$ converges locally smoothly to a non-compact  area-minimizing hypersurface $\Sigma_z\subset M$ that is axially symmetric with respect to $e_n$ and satisfies \begin{align} \inf\{x^n:x\in \Sigma_z\}=z.\label{asymptotics}\end{align} 
\end{lem}
\begin{proof}
This follows from  Lemma \ref{cylinder}, Lemma \ref{height bound}, and standard compactness results from  geometric measure theory.
\end{proof}
\begin{prop}
The following hold. \label{foliation prop} 
\begin{itemize}
	\item[$\circ$] As $z\to0$, $\Sigma_z$ converges locally smoothly to a non-compact area-minimizing hypersurface $\Sigma_0\subset M$ with $\inf\{x^n:x\in \Sigma_0\}=0$.
	\item[$\circ$] As $z\to\infty$, $-z+\Sigma_z$ converges locally smoothly to a Euclidean plane. 
	\item[$\circ$] The family $\{\Sigma_z\}_{z> 0}$ forms a smooth foliation of  the component of $M\setminus \Sigma_0$ that lies above $\Sigma_0$.
	\item[$\circ$] $\Sigma_z$ is not stable with respect to asymptotically constant perturbations for any $z\geq 0$.  
\end{itemize}
\end{prop}
\begin{proof} Let $z_1,\,z_2>0$ with $z_1\neq z_2$. Since $\Sigma_{z_1}$ and $\Sigma_{z_2}$ are area-minimizing and axially symmetric with respect to $e_n$,  $\Sigma_{z_1}$ and $\Sigma_{z_2}$ are disjoint. 	Using also \eqref{asymptotics}, we see that, as $z\to 0$, $\Sigma_z$ converges locally smoothly to a non-compact area-minimizing hypersurface $\Sigma_0\subset M$ with $\inf\{x^n:x\in \Sigma_0\}=0$. Moreover, using Lemma \ref{height bound}, we see that, as $z\to\infty$, $-z+\Sigma_z$ converges locally smoothly to a non-compact hypersurface $\Sigma$ that is area-minimizing with respect to $\bar g$. By Lemma \ref{lem:characterizationlocallyisoperimetric}, $\Sigma$ is a Euclidean plane. \\ \indent 
Let $z>0$.	By \eqref{conformal transformation}, $t+\Sigma_z$ is strictly mean-convex for every $t>0$ when oriented by the unit normal in direction of $e_n$. In conjunction with  \eqref{asymptotics} and the maximum principle, it follows that  $\Sigma_{z'}\to \Sigma_{z}$ locally smoothly as $z'\searrow z$. Let $\{z_i\}_{i=1}^\infty$ be a sequence of numbers $0<z_i<z$ with $z_i\to z$. Passing to a subsequence and using \eqref{asymptotics}, we see that $\Sigma_{z_i}$ converges locally smoothly to a non-compact area-minimizing hypersurface $\Sigma'_{z}$ that lies below $\Sigma_z$, is axially symmetric with respect to $e_n$, and satisfies $\inf\{x^n:x\in \Sigma'_z\}=z$. Since both $\Sigma_{r,z}$, $r\geq 2$, and $\Sigma'_{z}$ are area-minimizing and axially symmetric with respect to $e_n$, $\Sigma_{r,z}$ lies below $\Sigma'_z$ for every $r\geq 2$. It follows that $\Sigma'_z=\Sigma_z$. Consequently, $\{\Sigma_z\}_{z> 0}$ forms a smooth foliation; see Figure \ref{counterexample figure}.\\ \indent
Finally, suppose for a contradiction, that $\Sigma_z$ is stable with respect to asymptotically constant perturbations for some $z\geq 0$.  By Proposition \ref{flat flat}, $\Sigma_z$ is totally geodesic. It follows that $\Sigma_z=S^n_1(0)$. This is not compatible with $\Sigma_z$ being non-compact. 
\\ \indent The assertion follows.  
\end{proof}
		\begin{figure}\centering
		\includegraphics[width=1\linewidth]{foliation.pdf}
		\caption{An illustration of the foliation $\{\Sigma_{z}\}_{z\geq 0}$ whose leaves are depicted by the solid black lines.  The horizon is depicted by the gray line. $\Sigma_0$ asymptotes to the plane $\{x^n=0:x\in M\}$  depicted by the dashed black line.}
		\label{counterexample figure}
	\end{figure}
	 \section{Proof of Theorem \ref{counterexample 2}}
	 Let $3<n\leq 7$ and $(n-2)/2<\tau<n-2$. In this section, we use the gluing technique developed by A.~Carlotto and R.~Schoen \cite{SchoenCarlotto} to construct a Riemannian manifold of dimension $n$ which is asymptotically flat of rate $\tau$ with non-negative scalar curvature and positive mass and which admits a non-compact area-minimizing boundary that is stable with respect to asymptotically constant variations \eqref{stable wrt acv}. 
	 \begin{lem} \label{localization}
	 	 There exists a Riemannian metric $g$ on $\mathbb{R}^n$ which is asymptotically flat of rate $ \tau$ with non-negative scalar curvature and positive mass such that $$g=\bar g \text{ on } \{x\in\mathbb{R}^n:x^n\leq 2+|x-x^n\,e_n|_{\bar g}\}.$$
	 \end{lem}
	 \begin{proof}
The Riemannian metric
	 	$$
	\hat g=\left[1+(1+|x|_{\bar g}^{2\,n-4})^{-\frac12}\right]^{\frac{4}{n-2}}\,\bar g
	 	$$
 is asymptotically flat of rate $n-2$ with non-negative scalar and  mass $2$. The assertion follows from localizing $(\mathbb{R}^n,\hat g)$ to the cone $\{x\in\mathbb{R}^n:x^n> 2+|x-x^n\,e_n|_{\bar g}\}$ using  \cite[Theorem 2.3]{SchoenCarlotto}.
	 \end{proof}
	 \begin{lem}
	 	Let $(n-2)/2<\tilde \tau<\tau$ and $g$ be as in Lemma \ref{localization}.	There exists a conformally flat Riemannian metric $\tilde g$ on $\mathbb{R}^n$ that is asymptotically flat of rate $\tilde \tau$ with the following properties:
	 	\begin{itemize}
	 		\item[$\circ$] $\tilde g\leq g$.
	 		\item[$\circ$] $\tilde g<g$ on  $\{x\in\mathbb{R}^n:x^n\geq 2+|x-x^n\,e_n|_{\bar g}\}$.
	 		\item[$\circ$] $\tilde g=\bar g $ on $\{x\in\mathbb{R}^n:x^n\leq 0\}$.
	 		\item[$\circ$] $\tilde g$ is axially symmetric with respect to $e_n$.
	 		\item[$\circ$] There holds, for all $x\in\mathbb{R}^n$, $$
	 		\sum_{i=1}^{n-1}x^i\,\bar g(\bar D_{e_i}\tilde g,\bar g)\geq 0.
	 		$$
	 	\end{itemize}
	 \end{lem}
	 \begin{proof}
	 Using that $g$ is asymptotically flat of rate $\tau$, we see that, provided $\delta>0$ is sufficiently small,
	 $$
	 g(x)>\left[1-(1-\delta)\,(1+\delta\,|x|^2_{\bar g})^{-\tilde \tau/2}\right]\,\bar g
	 $$	
for every $x\in\mathbb{R}^n$.
	 	 Let $\eta \in C^\infty(\mathbb{R})$ with 
	 	\begin{itemize}
	 		\item[$\circ$] $0\leq \eta\leq 1$,
	 		\item[$\circ$] $\eta(t)=0$ if $t\leq 1/2$,
	 		\item[$\circ$] $\eta(t)=1$ if $t\geq 1$, and
	 		\item[$\circ$] $\eta'(t)\geq 0$ for all $t\in\mathbb{R}$.
	 	\end{itemize}
	 	The metric $$
	 	\tilde g=\left[1-(1-\delta)\,\eta\big((1+|x-x^n\,e_n|^2_{\bar g})^{-1/2}\,x^n\big)\,(1+\delta\,|x|^2_{\bar g})^{-\tilde \tau/2}\right ]\,\bar g
	 	$$
	 	has the asserted properties.
	 \end{proof}
	 \indent Let  $r>2,$ $z<0$,  and $\tilde \Sigma_{r,z}$ be the least area hypersurface in $(\mathbb{R}^n,\tilde g)$ with $$\partial \tilde \Sigma_{r,z}= S^{n-2}_r(0)\times\{z\}.$$  \indent
	 Repeating the proofs of Lemma \ref{axial symmetry}, Lemma \ref{cylinder}, and Lemma \ref{graph},  we see that there is $\tilde f_{r,z}\in C^{\infty}(\mathbb{R})$ with $\tilde f_{r,z}(r)=z$ and $\tilde f'_{r,z}(0)=0$ such that
	 $$
	 \tilde \Sigma_{r,z}=\{(y,\tilde f_{r,z}(|y|_{\bar g})):y\in\mathbb{R}^{n-1}\text{ with }|y|_{\bar g}\leq r\}.
	 $$
	 \begin{lem} \label{second height bound} 
	 	There are $t_0$ and $c_0>1$ depending only on $n$ such that, provided that $r>2$ is sufficiently large,
	 	$$
	 	z\leq \tilde f_{r,z}(t)\leq c_0+z
	 	$$
	 	for all $t\in(t_0,r)$ and $z<0$.
	 \end{lem}
 \begin{proof}
 	Repeating the proof of Lemma \ref{f ode}, we see that there is $c>1$ depending only on $n$ such that
 	$$
 		\bigg(t^{n-2}\frac{\tilde f'_{r,z}(t)}{\sqrt{1+\tilde f'_{r,z}(t)^2}}\bigg)'\geq -c\,\frac{t^{n-2}}{1+(t^2+\tilde f_{r,z}(t)^2)^{(\tilde \tau+1)/2}}.
 	$$
 	The assertion now follows as in the proof of Lemma \ref{height bound}, using that $\tilde \tau>(n-2)/2\geq 1$.
 \end{proof}
\begin{lem} \label{final bound}  
	There holds
	$$
	\tilde \Sigma_{r,z}=\{y\in\mathbb{R}^{n-1}:|y|_{\bar g}\leq r\}\times\{z\}
	$$
provided that $r>2$ is sufficiently large and that 
\begin{align} \label{z condition}  
z<-c_0- t_0^{1/(n-2)}\,\int_1^\infty\frac{1}{\sqrt{s^{2n-4}-1}}\,\mathrm{d}s.
\end{align} 
\end{lem}
\begin{proof}
By Lemma \ref{second height bound} and \eqref{z condition}, there is a least $t_{r,z}\in[0,t_0)$  such that $\tilde f_{r,z}(t)<0$ for all $t\in(t_{r,z},r)$ provided that  $r>2$ is sufficiently large. Consequently,
$$
\bigg(t^{n-2}\frac{\tilde f'_{r,z}(t)}{\sqrt{1+\tilde f'_{r,z}(t)^2}}\bigg)'=0
$$
for all $t\in(t_{r,z},r)$. Let 
$$
a_{r,z}=-t_{r,z}^{n-2}\frac{\tilde f'_{r,z}(t_{r,z})}{\sqrt{1+\tilde f'_{r,z}(t_{r,z})^2}}
$$
and note that $a_{r,z}< t_{r,z}^{n-2}< t_0^{n-2}$. Integrating, we obtain 
$$
\tilde f(t)=\tilde f_{r,z}(t_{r,z})-\int_{t_{r,z}}^t \frac{a_{r,z}}{\sqrt{s^{2\,n-4}-a_{r,z}^2}}\,\mathrm{d}s.
$$
Consequently, using \eqref{z condition},
$$
\tilde f_{r,z}(t_{r,z})=-z+\int_{t_{r,z}}^r\frac{a_{r,z}}{\sqrt{s^{2\,n-4}-a^2_{r,z}}}\,\mathrm{d}s\leq -z+t(n)^{1/(n-2)}\,\int_1^\infty\frac{1}{\sqrt{s^{2n-4}-1}}\,\mathrm{d}s<0.
$$
It follows that  $t_{r,z}=0$ so that $\tilde \Sigma_{r,z}\subset \{x\in\mathbb{R}^n:x^n\leq 0\}$. Using that $\tilde g=\bar g$ in $\{x\in\mathbb{R}^n:x^n\leq 0\}$, the assertion follows.  
\end{proof}
\begin{proof}[Proof of Theorem \ref{counterexample 2}]
	Let 
	$$
	z<-c_0- t_0^{1/(n-2)}\,\int_1^\infty\frac{1}{\sqrt{s^{2n-4}-1}}\,\mathrm{d}s
	$$
	and $r>2$ be sufficiently large such that the conclusion of Lemma \ref{final bound} applies. Let $\Sigma_{r,z}$ be the least area hypersurface in $(\mathbb{R}^n,g)$ with $$\partial \Sigma_{r,z}= S^{n-2}_r(0)\times\{z\}.$$
	Using that $\tilde g\leq g$ with strict inequality where $g\neq \bar g$ and Lemma \ref{final bound}, it follows that $\Sigma_{r,z}=\tilde \Sigma_{r,z}$. Passing to a limit as $r\to\infty$, we see that $\mathbb{R}^{n-1}\times\{z\}$ is area-minimizing in $(\mathbb{R}^n,g)$. Since $g=\bar g$ near  $\mathbb{R}^{n-1}\times \{z\}$, it follows that  $\mathbb{R}^{n-1}\times \{z\}$ is stable with respect to asymptotically constant variations.
\end{proof}
\begin{appendices}

\section{Asymptotically flat manifolds} \label{adm appendix}
In this section, we recall background on asymptotically flat manifolds. \\ \indent 
Let $(M, g)$ be a connected, complete Riemannian manifold of dimension $n \geq 3$. \\
We say that $(M, g)$ is asymptotically flat of rate $\tau > (n-2)/2$ if there is a non-empty compact set $K \subset M$ and a diffeomorphism
\begin {equation} \label{chartatinfinity}
 \{ x \in \mathbb{R}^n : |x|_{\bar g} > 1/2\}\to M \setminus K
\end {equation}
such that, in the corresponding coordinate system,
\[
|g - \bar g|_{\bar g} + |x|_{\bar g} \, |\bar D g|_{\bar g} + |x|_{\bar g}^2\, |\bar D^2 g|_{\bar g} = O (|x|_{\bar g}^{-\tau}).
\]
Here, $\bar g$ is the Euclidean metric on $\mathbb{R}^n$ and a bar indicates that a geometric quantity is computed with respect to $\bar g$. In addition, we require the scalar curvature  of $(M,g)$ to be integrable. If the boundary of $M$ is non-empty, we require that the boundary is a minimal surface and that every closed minimal hypersurface in $(M, g)$ is contained in the boundary. \\ \indent The particular choice of diffeomorphism \eqref{chartatinfinity} is usually fixed and referred to as the chart at infinity of $(M, g)$. Given $r\geq 1$, we use $B_r$ to denote the open bounded set whose boundary corresponds to $S^{n-1}_r(0)=\{x\in\mathbb{R}^n:|x|_{\bar g}=r\}$ in this chart.
\\ \indent 
If $(M, g)$ is asymptotically flat, the quantity
\begin{align} \label{def adm mass} 
	m=\frac{1}{2\,(n-1)\,n\,\omega_{n}}\,\lim_{\lambda\to\infty}\lambda^{-1}\,\int_{S^{n-1}_{\lambda}(0)}\sum_{i,\,j=1}^nx^i\,\left[(\bar D_{e_j}g)(e_i,e_j)-(\bar D_{e_i}g)(e_j,e_j)\right]\,\mathrm{d}\bar\mu
\end{align} 
is called the mass of $(M, g)$; see \cite{ADM}. 	Here, $e_1,\dots,e_n$ is the standard basis of $\mathbb{R}^n$.  The existence of the limit in \eqref{def adm mass} follows from the integrability of the scalar curvature and the decay assumptions on $g$. It is independent of the particular choice of chart at infinity; see \cite[Theorem 4.2]{bartnik}. Note that the mass vanishes if $\tau > n-2$. 

\section {Isoperimetric regions and their limits}
\label{iso appendix}
In this section, we collect results on large isoperimetric regions. \\ \indent 
Let $(M, g)$ be a connected, complete Riemannian manifold of dimension $3 \leq n \leq 7$ that is asymptotically flat.

Recall that the boundary of $M$ is either empty or an outermost minimal surface. Let $\hat M$ be an open manifold in which $M$ is embedded.

A subset $\Omega \subset M$ is called a region if 
\[
\hat \Omega = \Omega \cup (\hat M \setminus M)
\] 
is a properly embedded $n$-dimensional submanifold of $\hat M$. Note that the boundary of $\hat \Omega$ is a properly embedded hypersurface of $M$. It does not depend on the  choice of extension $\hat M$ and will be denoted by $\partial \Omega$.

Let $\Omega \subset M$ be a region. The second fundamental form $h$ and the mean curvature scalar $H$ of $\partial \Omega$ are computed with respect to the normal pointing out of $\Omega$. 

We are interested in three special types of regions in this paper.

\begin{enumerate} [$\circ$] 
	\item A region $\Omega \subset M$ is isoperimetric if it is compact and
	\[
	|\partial \Omega| \leq |\partial \tilde \Omega|
	\]
	for every compact region $\tilde \Omega \subset M$ with $|\tilde \Omega| = | \Omega|$.
	\item A region $\Omega \subset M$ has area-minimizing boundary if, for every $U \Subset M$ open, there holds
	\[
	|U \cap \partial \Omega | \leq |U \cap \partial \tilde \Omega|
	\]
	for every region $\tilde \Omega \subset M$ with $\tilde \Omega \, \triangle \, \Omega \Subset U$.
	\item 
	A region $\Omega \subset M$ is locally isoperimetric if, for every $U \Subset M$ open, there holds
	\[
	|U \cap \partial \Omega | \leq |U \cap \partial \tilde \Omega|
	\]
	for every region $\tilde \Omega \subset M$ with $\Omega \, \triangle \, \tilde \Omega \Subset U$ and $|U \cap \tilde \Omega| = |U \cap \Omega|$.
\end{enumerate}

\begin {rema}  \label{regularity remark}
Alternatively, we could introduce these notions using sets with locally finite perimeter and their reduced boundaries instead of properly embedded $n$-dimensional submanifolds and their boundaries. Standard regularity theory shows that the reduced boundary of a locally isoperimetric such set is smooth; see, e.g., the survey of results in \cite[Section 4]{eichmairmetzger}.
\end {rema}

\begin {lem}  \label{area growth estimate}  There holds
\begin{align} \label{appendix area est} 
\sup_{k\geq 1} \, \sup_{r > 1} r^{1 - n} \, |B_r \cap \partial \Omega_k| < \infty.
\end{align} 
Moreover, for every $\alpha>0$, there is $c>1$ such that
$$
\limsup_{k\to\infty}\sup_{1<s<t} \left(t^{n-1-\alpha}+s^{n-1-\alpha}\right)^{-1}\,\int_{(B_t\setminus B_{s})\cap  \partial \Omega_k}|x|_{\bar g}^{-\alpha}\,\mathrm{d}\bar\mu <\infty. 
$$
\end{lem}
\begin {proof}
To prove \eqref{appendix area est}, for $r > 1$ sufficiently large and such that $\partial  B_r$ and $\partial \Omega_k$ intersect transversely, we may obtain comparison regions $\tilde \Omega_k$ by cutting $B_r \cap \Omega_k$ from $\Omega_k$ and pasting in a ball of the removed amount of volume. Using also Lemma \ref{layer cake}, the assertion follows.
\end {proof}


\begin {lem}
Let $\Omega \subset M$ be a locally isoperimetric region of $(M, g)$ with $\partial \Omega \neq \emptyset$. Then $\partial \Omega$ has constant mean curvature. The mean curvature is zero when the boundary is area-minimizing.
\end {lem}

\begin {lem} \label{lem:characterizationlocallyisoperimetric}
Assume that $\check \Omega \subset \mathbb{R}^n$ is a locally isoperimetric region with $\partial \check \Omega \neq \emptyset$. Then $\partial \check \Omega$ is either a hyperplane or a coordinate sphere.
\begin {proof} 
This is \cite [Proposition 1]{Morgan-Ros:2010}. 
\end {proof}
\end {lem}

\begin {rema}
The same characterization holds for sets of locally finite perimeter that are locally isoperimetric in $\mathbb{R}^n \setminus \{0\}$. The potential singularity in the reduced boundary of such a set at $0$ is removable; cp.~Remark \ref{regularity remark}.
\end {rema}

Let $\Omega_1,\, \Omega_2, \ldots \subset M$ be locally isoperimetric regions of $(M, g)$ with $\partial \Omega_k \neq \emptyset$.

\begin {lem} \label{lem:locisoconverge}
Assume that $\limsup_{k\to\infty} |H (\partial \Omega_k)| < \infty$. Then $\limsup_{k\to\infty } |h (\partial \Omega_k)| < \infty$. Moreover, there exists a locally isoperimetric region $\Omega \subset M$ such that, passing to a subsequence, $\Omega_k \to \Omega$ locally smoothly.
\begin {proof}
This is a standard result from geometric measure theory; see, e.g., the survey in \cite[Section 4]{eichmairmetzger}.
\end {proof}
\end {lem}

\begin {lem} \label{lem:locallyisoperimetricboundary}
Let $\Omega \subset M$ be a locally isoperimetric region with non-compact boundary. Then $\Omega$ has area-minimizing boundary. All components of $\partial \Omega$ except for one are components of the boundary of $M$.

\begin {proof}

Suppose, for a contradiction, that $H (\partial \Omega) \neq 0$. Let $\{x_\ell\}_{\ell = 1}^\infty $ be a sequence of points $x_\ell \in \partial \Omega \setminus B_1$ with $|x_\ell|_{\bar g} \to \infty$. It follows from Lemma \ref{lem:locisoconverge} that, passing to a subsequence,
\[
- x_\ell + \Omega \setminus B_1  \to \check {\Omega} \text{ locally smoothly in } \mathbb{R}^n
\] 
where $\check \Omega \subset \mathbb{R}^n$ is a locally isoperimetric region with $\partial \check \Omega \neq \emptyset$ and $\bar H (\partial \check \Omega) = H (\partial \Omega)$. By Lemma \ref{lem:characterizationlocallyisoperimetric}, $\partial \check \Omega$ is a coordinate sphere of radius $(n-1) \, |H(\partial \Omega)|^{-1}$. It follows that $\Omega$ has infinitely many bounded components, each one close to a coordinate ball of radius $(n-1) \, |H (\partial \Omega)|$.  Such a configuration contradicts the Euclidean isoperimetric inequality. Thus $H (\partial \Omega) = 0$.

Let $\{s_\ell\}_{\ell = 1}^\infty$ be a sequence of numbers $s_\ell>1$ with $s_\ell \to \infty$. By Lemma \ref{lem:locisoconverge}, passing to a subsequence,
\[
s_\ell^{-1} \, (\Omega \setminus B_1) \to \breve \Omega \text{ locally smoothly in } \mathbb{R}^n \setminus \{0\}
\]
where $\breve \Omega \subset \mathbb{R}^n$ is a locally isoperimetric region whose boundary is non-empty with mean curvature zero. By Lemma \ref{lem:characterizationlocallyisoperimetric}, $\breve \Omega$ is a half-space. It follows that $\partial \Omega$ has only one unbounded component. To see that the boundary of $\Omega$ is area-minimizing, observe that there are constants $\delta, c > 0$ such that, for every $\ell$ sufficiently large and all $v \in (- \delta, \delta),$ there is a region $\tilde \Omega \subset M$ with $\tilde \Omega \,  \triangle \, \Omega \Subset \{x \in \mathbb{R}^n : s_\ell < |x|_{\bar g} < 2 \, s_\ell\} = U_\ell$ and such that
\begin{eqnarray*} 
	s_\ell^{- n} \, \Big(  | U_\ell \cap \Omega| - |U_\ell \cap \tilde \Omega| \Big) = v  \qquad \text{ and } \qquad s_\ell^{1-n} \, \Big| |U_\ell \cap \partial \Omega| - |U_\ell \cap \partial \tilde \Omega | \Big| \leq c \, v^2.
	\end {eqnarray*}
	Thus, we can   add and subtract an amount of volume $V$ from $\Omega$ at the cost of changing the boundary area by an amount of order $s_\ell^{-1} \, V^2$. Since $s_\ell^{-1} \, V^2 = o(1)$, it follows that $\Omega$ has area-minimizing boundary. See \cite[Appendix C]{eichmairmetzger} for additional details on this argument in the case where $n = 3$.
	
	The preceding argument also shows that, in the case where the boundary of $M$ is empty, $\partial \Omega$ has no bounded components, since deleting such components and compensating for the loss of volume far out decreases area. In the case where the boundary of $M$ is non-empty, bounded components of $\partial \Omega$, being closed minimal surfaces, are contained in the boundary of $M$.
	\end {proof}
	\end {lem}
	
	We now assume that $\Omega_k \subset M$ are isoperimetric regions with $|\Omega_k| \to \infty$. 
	
	\begin {lem} \label{lem:coarseisobound} 
	There holds
	\begin {eqnarray}
	\lim_{k \to \infty}  \lambda (\partial \Omega_k)^{-n} \, |\Omega_k| = \omega_n
	\end {eqnarray}
	\begin {proof} 
	By comparison with coordinate balls far out in the asymptotically flat end, we see that
	\[
	\liminf_{k \to \infty}  \lambda (\partial \Omega_k)^{-n} \, |\Omega_k| \geq \omega_n.
	\] 
\indent Let $\varepsilon>0$ and $r>1$ be large such that $\partial B_r$ and $\partial \Omega_k$ intersect  transversally for all $k$ sufficiently large. Let $\Omega_{k,r}=\Omega_k\setminus B_r$. Note that 
	\begin{align}\label{cut and paste comparison}
	| \Omega_{k,r}|\geq |\Omega_k|-|B_r|\qquad \text{and}\qquad |\partial \Omega_k|\geq |\partial  \Omega_{k,r}|-|\partial B_r|.  
	\end{align} 
	By the Euclidean isoperimetric inequality,
	$$
	\bar \lambda (\partial \Omega_{k,r})^{-n}\,|\Omega_{k,r}|_{\bar g}\leq \omega_n.
	$$
	Increasing $r>1$, if necessary, we obtain
	$$
	 \lambda (\partial \Omega_{k,r})^{-n}\,|\Omega_{k,r}|\leq \omega_n+\varepsilon
	$$ 
	for all $k$ sufficiently large. Letting $k\to\infty$ and using \eqref{cut and paste comparison}, we conclude that
	\[
	\limsup_{k \to \infty}  \lambda (\partial \Omega_k)^{-n} \, |\Omega_k| \leq \omega_n+\varepsilon.
	\] 
	The assertion follows. 
	\end {proof}
	\end {lem}
	
	It follows from Lemma \ref{lem:coarseisobound} that $\lambda (\partial \Omega_k) \to \infty$.
	
	\begin {lem}
	If the boundary of $M$ is non-empty, then $H(\partial \Omega_k) >0$ for all $k$. If the boundary of $M$ is empty, then $H (\partial \Omega_k) > 0$ for all sufficiently large $k$.
	\end {lem}
	
	We now assume that $H (\partial \Omega_k) > 0$ for all $k$.
	
	\begin {lem}
	There holds $\limsup_{k \to \infty} \lambda (\partial \Omega_k) \, H(\partial \Omega_k) < \infty$.
	\begin {proof}
	Suppose, for a contradiction, that $\lambda (\partial \Omega_k) \, H (\partial \Omega_k) \to \infty$. Let $x_k \in \partial \Omega_k$ be such that $\liminf_{k \to\infty} \lambda (\partial \Omega_k)^{-1} \, |x_k|_{\bar g} >0$ and $r_k = (n-1) \, H(\partial \Omega_k)^{-1}$.  Passing to a subsequence,
	\[
	r_k^{-1} \, (- x_k + \Omega_k \setminus B_1) \to \{x \in \mathbb{R}^n : |x|_{\bar g} \leq 1\} \text{ locally smoothly in } \mathbb{R}^n. 
	\]
 It follows that the component of $\Omega_k$ which contains $x_k$ is close to a coordinate ball of radius $r_k = o (\lambda (\partial \Omega_k))$. Using Lemma \ref{lem:coarseisobound}, we see that $\Omega_k$ contains a second such component. By the usual cut-and-paste argument, see, e.g., the proof of Lemma \ref{lem:locallyisoperimetricboundary}, this contradicts the assumption that $\Omega_k$ is isoperimetric.
	\end {proof}
	\end {lem}
	
	\begin {lem}  \label{large blowdown1}
	Assume that there is a sequence $\{x_k\}_{k = 1}^\infty$ of points $x_k \in \partial \Omega_k \setminus B_1$ with $\lambda (\partial \Omega_k)^{-1} \, |x_k|_{\bar g} \to \infty$. The component of $\Omega_k$ which contains $x_k$ is close to a coordinate ball of radius $(n-1)\,H(\partial \Omega_k)^{-1}$.
	\end {lem}
	It follows from Lemma \ref{large blowdown1} that $\Omega_k$ has at most one component that is, on the scale of $\lambda (\partial \Omega_k)$, far from $B_1$.
	
	\begin {lem}  \label{large blowdown2}
	Assume that there are $x_k \in \partial \Omega_k \setminus B_1$ with $0 < \liminf_{k \to \infty} \lambda (\partial \Omega_k)^{-1} \, |x_k|_{\bar g} <\infty$. There is $\xi \in \mathbb{R}^n$ such that, passing to a subsequence,
	\[
	\lambda (\partial \Omega_k)^{-1} \,( \Omega_k \setminus B_1) \to \{x \in \mathbb{R}^n : |x - \xi|_{\bar g} \leq 1\}  \text{ locally smoothly in } \mathbb{R}^n \setminus \{0\}.
	\]
	\begin {proof} 
	The assumption implies locally smooth convergence of a subsequence to a non-trivial locally isoperimetric region in $\mathbb{R}^n$, the area of whose boundary is at most $n \, \omega_n$. Such a region is a ball of radius $0 < r \leq 1$. Note that 
	$
	(n-1) \, / \, r = \lim_{k \to \infty} \lambda (\partial \Omega_k) \, H (\partial \Omega_k)
	$.
	Suppose, for a contradiction, that $r < 1$. Using Lemma \ref{large blowdown1}, we see that $\Omega_k$ contains has at least one additional large component that lies far out. By the usual cut-and-paste argument, this contradicts the assumption that $\Omega_k$ is isoperimetric.
	\end {proof}
	\end {lem}
	
	\begin {coro} 
	There holds 
	\begin {eqnarray} \label{H est}
	\lim_{k \to \infty} \lambda (\partial \Omega_k) \, H (\partial \Omega_k) = n-1.
	\end {eqnarray}
	\end {coro}
	
	Now, we assume in addition that there is $K \subset M$ compact and disjoint from the boundary of $M$ such that, for all $k$,
	\[
	K \cap \partial \Omega_k \neq \emptyset.
	\]
	
	\begin {prop} \label{smooth local convergence}
	There is a region $\Omega \subset M$ with non-compact area-minimizing boundary such that, passing to a subsequence, $\Omega_k \to \Omega$ locally smoothly. There is $\xi \in \mathbb{R}^n$ with $|\xi|_{\bar g} = 1$ such that, passing to a subsequence, $\lambda (\partial \Omega_k)^{-1} \, (\Omega_k \setminus B_1) \to \{ x \in \mathbb{R}^n : |x - \xi|_{\bar g} \leq 1\}$ locally smoothly in $\mathbb{R}^n \setminus \{0\}$.
	\begin {proof}
	This follows from Lemma \ref{lem:locisoconverge}, Lemma \ref{lem:locallyisoperimetricboundary}, and Lemma \ref{large blowdown2}.
	\end {proof}
	\end {prop}
	
	In the following lemma, $\hcirc$ denotes the traceless second fundamental form. 
	
	\begin {lem}
	Let $\{x_k\}_{k=1}^\infty$ be a sequence of points $x_k \in \partial \Omega_k \setminus B_1$ with $|x_k|_{\bar g} \to \infty$. Then 
	\begin{align} \label{hcirc est} 
	\limsup_{k\to\infty} |x_k|_{\bar g} \, | \hcirc (\partial \Omega_k) (x_k)| =0. 
	\end{align}
	\begin {proof}
	Suppose first that $\lim_{k \to \infty}  \lambda (\partial \Omega_k)^{-1} |x_k|_{\bar g} = 0$. By Lemma Lemma \ref{lem:characterizationlocallyisoperimetric} and \ref{lem:locisoconverge}, passing to a subsequence, 
	$|x_k|_{\bar g}^{-1} \, ( \Omega_k \setminus B_1)$ converges to a half-space locally smoothly in $\mathbb{R}^n\setminus\{0\}$.
	Suppose second that $\liminf_{k \to\infty} \, \lambda (\partial \Omega_k)^{-1} |x_k|_{\bar g} >0$. By  Proposition \ref{smooth local convergence}, passing to a subsequence, $|x_k|_{\bar g}^{-1} \, (\Omega_k \setminus B_1)$ converges to a ball locally smoothly in $\mathbb{R}^n\setminus \{0\}$. Either way, the assertion follows.
	\end {proof}
	\end {lem}
	
	The isoperimetric profile of $(M, g)$ is the function  $A : (0, \infty) \to (0, \infty)$ given by
	\begin{align}  \label{isoperimetric profile} 
	A(V) = \inf \{ |\partial \Omega| : \Omega \subset M \text{ is a compact region with } |\Omega| = V \}.
	\end{align}
	
	\begin {lem} \label{profile properties} The isoperimetric profile of $(M, g)$ is absolutely continuous. As $V \to \infty$,
	\[
	\left(\omega_n^{-1} \, V\right)^{\frac{1-n}{n}}\,A(V)=n\,\omega_n+o(1)
	\]
	and, almost everywhere,
	\[
	\left(\omega_n^{-1} \, V\right)^{\frac{1}{n}}\,A'(V)=(n-1)+o(1).
	\]
	If the boundary of $M$ is non-empty, then $A$ is a strictly increasing function.
	\begin {proof}
	See, e.g., \cite[Appendix A]{CESH} and \cite[Proposition 4]{eichmairmetzger}.
	\end {proof}
	\end {lem}
	
	\begin {lem} [{\cite[Theorem 1.12]{CCE}}]\label{existence iso}
	Assume that $(M, g)$ has positive mass. For every sufficiently large amount of volume $V>0$, there exists an isoperimetric region $\Omega \subset M$ with $|\Omega| = V$.
	\end {lem}
	




	\section{Variation of area and volume}
	\label{stable cmc} 
	In this section, we recall  the first and second variational formulae for area and volume and the definition of a stable constant mean curvature surface; see, e.g., \cite[Appendix H]{CCE}. \\ \indent 
	Let $(M,g)$ be a Riemannian manifold without boundary of dimension $n\geq 3$.  Let $\Sigma\subset M$ be a closed hypersurface bounding a compact region $\Omega$. We denote by $\mathrm{d}\mu$ the area element, by $\nu$ the outward pointing unit normal, and by $h$ and $H$ the second fundamental form and mean curvature, respectively, computed with respect to $\nu$. \\ \indent
	Let $\varepsilon>0$ and $U\in C^\infty(\Sigma\times(-\varepsilon,\varepsilon))$ with $U(x,0)=0$ for all $x\in \Sigma$. Decreasing $\varepsilon$ if necessary, we obtain a smooth family $\{\Sigma(s):s\in(-\varepsilon,\varepsilon)\}$ of hypersurfaces $\Sigma(s)\subset M$ where 
	$$
\Sigma(s)=\{\exp_x(U(x,s)\,\nu(x)):x\in \Sigma\}.
	$$
	We define the initial velocity $u\in C^\infty(\Sigma)$ and the initial acceleration $v\in C^\infty(\Sigma)$ by
	$$
	u(x)=\dot{U}(x,0)\qquad\text{and}\qquad v=\ddot{U}(x,0).
	$$  \indent 
	Let $\Omega(s)$ be the compact region bounded by $\Sigma(s)$.
	\begin{lem} \label{variation of area and volume} 
		There holds
		\begin{align} 
\notag		\frac{d}{ds}\bigg|_{s=0}|\Sigma(s)|&=\int_{\Sigma}H\,u\,\mathrm{d}\mu,\\
	\label{second area} 	\frac{d^2}{ds^2}\bigg|_{s=0}|\Sigma(s)|&=\int_{\Sigma}H\,v+H^2\,u^2+|\nabla u|^2-(|h|^2+{Ric}(\nu,\nu))\,u^2\,\mathrm{d}\mu.
		\end{align} 
	Moreover,
	\begin{align*} 
		\frac{d}{ds}\bigg|_{s=0}|\Omega(s)|&=\int_{\Sigma}u\,\mathrm{d}\mu,\\
		\frac{d^2}{ds^2}\bigg|_{s=0}|\Omega(s)|&=\int_{\Sigma}v+H\,u^2\,\mathrm{d}\mu.
	\end{align*} 
	\end{lem}
Assume that for every such variation  satisfying also
$$
\frac{d}{ds}\bigg|_{s=0}|\Omega(s)|=\frac{d^2}{ds^2}\bigg|_{s=0}|\Omega(s)|=0,
$$
there holds
$$
\frac{d}{ds}\bigg|_{s=0}|\Sigma(s)|=0\qquad\text{and}\qquad \frac{d^2}{ds^2}\bigg|_{s=0}|\Sigma(s)|\geq 0.
$$
Note that the mean curvature $H$ of  $\Sigma$ is constant in this case. We say that $\Sigma$ is a stable constant mean curvature surface \\ \indent 
Note that each component of the boundary of an isoperimetric region $\Omega\subset M$ is a stable constant mean curvature surface with the same constant mean curvature.
	      \\ \indent 
	We record the following integration by parts formula for the second variation of area formula \eqref{second area} with respect to a Euclidean translation. 
	\begin{lem} \label{euclidean lemma}
Let $ \Sigma\subset \mathbb{R}^n$ be a closed oriented hypersurface with boundary $\partial  \Sigma$. Let $\bar\nu$ be a unit normal of $\Sigma$ and $\bar\omega$ the outward-pointing conormal of $\partial \Sigma$. Let  $\bar u, \bar v:\Sigma\to\mathbb{R}$ be given by 
$$
\bar u=\bar g(e_n,\bar\nu)\qquad\text{and}\qquad \bar v=-\bar h(e_n^{\bar\top},e_n^{\bar\top}).
$$
 There holds 
		$$
		\int_{ \Sigma}\bar H\,\bar v+\bar H^2\,\bar u^2+|\bar\nabla\bar u|_{\bar g}^2-|\bar h|^2_{\bar g}\,\bar u^2\,\mathrm{d}\bar \mu=\int_{\partial \Sigma}\bar h(\bar\omega,e_n^{\bar\top})\,\bar u\,\mathrm{d}\bar{l}-\int_{ \partial \Sigma}\bar g(e_n^{\bar\top},\bar\omega) \bar H\,\bar u\,\mathrm{d}\bar{l}.
		$$
	\end{lem}
	\begin{proof}
		Using that
		$
		\bar\Delta \bar u+|\bar h|^2_{\bar g}\,\bar u=(\bar\nabla_{ e_n^{\bar\top}}\bar H)\,\bar u,
		$ we obtain
		$$
		\int_{\Sigma} |\bar\nabla u|_{\bar g}^2\,\mathrm{d}\bar \mu=\int_{\partial \Sigma}\bar h(\bar\omega,e_n^{\bar\top})\,\bar u\,\mathrm{d}\bar{l}+\int_{\Sigma} |\bar h|^2_{\bar g}\,\bar u^2-(\bar\nabla_{e_n^{\bar\top}}\bar H)\,\bar u\,\mathrm{d}\bar \mu.
		$$
		Using the Codazzi equation $\bar\nabla_{e_n^{\bar\top}}\bar H=(\bar{\operatorname{div}}\,\bar h)(e_n^{\bar\top})$ and integrating by parts again, we have
		$$
		\int_{ \Sigma} (\bar\nabla_{e_n^{\bar\top}}\bar H)\,\bar u\,\mathrm{d}\bar \mu=\int_{ \partial \Sigma}\bar g(e_n^{\bar\top},\bar\omega) \bar H\,\bar u\,\mathrm{d}\bar{l}+\int_{ \Sigma} \bar H^2\,\bar u^2+\bar H\,\bar v\,\mathrm{d}\bar \mu.
		$$
	\end{proof}
	
		\section{Hypersurface geometry in an asymptotically flat end}
	In this section, we assume that $g$ is a Riemannian metric on $\mathbb{R}^n$ where $n\geq 3$ and that, for some $\tau>0$, $$|g-\bar g|+|x|_{\bar g}\,|\bar D g|_{\bar g}=O(|x|_{\bar g}^{-\tau}).$$
	Let $\Sigma\subset \mathbb{R}^n$ be a two-sided  hypersurface with area element $\mathrm{d}\mu$, designated normal $\nu$, and second fundamental form $h$ and mean curvature $H$ with respect to $\nu$.  The corresponding Euclidean quantities are denoted with a bar.
	\begin{lem} As $x\to\infty$,
		\label{geometric expansions}
		\begin{align*}
			\nu=&\,\bar\nu+O(|x|_{\bar g}^{-\tau}),\\
			\mathrm{d}\mu=&\,(1+O(|x|_{\bar g}^{-\tau}))\,\mathrm{d}\bar\mu,
			\\	|x|_{\bar g}\,h=&|x|_{\bar g}\,\bar h +O(|x|_{\bar g}^{-\tau})+O(|x|_{\bar g}^{1-\tau}\,|\bar h|_{\bar g}),\text{ and}
			\\	|x|_{\bar g}\,H=&|x|_{\bar g}\,\bar H +O(|x|_{\bar g}^{-\tau})+O(|x|_{\bar g}^{1-\tau}\,|\bar h|_{\bar g}).
		\end{align*}
	\end{lem}
	\begin{lem} \label{layer cake}
		Suppose that $\partial \Sigma=\emptyset$. 	Let  $\alpha>0$, $0<s<t,$ and suppose that, for some $c\geq1$,
		$$
		r^{1-n}\,	|B^n_{r}  (0)\cap  \Sigma|_{\bar g}\leq c
		$$
		for all $s<r<t$.  There holds
		$$
	\int_{(B^n_t(0)\setminus B^n_{s}(0))\cap  \Sigma}|x|_{\bar g}^{-\alpha}\,\mathrm{d}\bar\mu\leq c\,t^{-\alpha}+\frac{c\,\alpha}{n-1-\alpha}\,\left(t^{n-1-\alpha}+s^{n-1-\alpha}\right).
		$$
	\end{lem}



\section{A Liouville theorem on the slab}
In this section, we prove a Liouville theorem for harmonic functions on a slab.
\begin{lem} \label{harmonic on slab} 
Let $n\geq 2$.	Let $f\in C^\infty(\mathbb{R}^{n-1}\times[0,2])$ be a non-negative harmonic function with $f(x)=0$ for all $x\in\mathbb{R}^{n-1}\times \{0,\,2\}$.  Assume that 
	\begin{align} \label{bound} \sup\{|(\bar D f)(x)|_{\bar g}:x\in\mathbb{R}^{n-1}\times\{0\}\}<\infty.\end{align}
	Then $f=0$. 
\end{lem}
\begin{proof}
	Let $x_0\in \mathbb{R}^{n-1}\times\{1\}$ and $v\in C^\infty(\mathbb{R}^{n-1}\times[0,2])$ be given by $v(x)=f(x_0)\,x^n$. Note that $v$ is harmonic. By the Boundary Harnack comparison principle, see, e.g., \cite[Theorem 11.6]{caffarelli}, there is a constant $c>0$ depending only on $n$ such that $v\leq c\,f$ on $B^{n}_1(x_0-e_n)\cap (\mathbb{R}^{n-1}\times[0,2])$. In particular,
	$$
	f(x_0)=(\partial_{e_n} v)(x_0-e_n)\leq c\,(\partial_{e_n} f)(x_0-e_n).
	$$
	In conjunction with \eqref{bound}, we see that $f$ is bounded on $\mathbb{R}^{n-1}\times\{1\}$. By the Boundary Harnack inequality, see, e.g., \cite[Theorem 11.5]{caffarelli}, it follows that $f$ is bounded in all of $\mathbb{R}^{n-1}\times[0,2]$. We extend $f$ to a bounded periodic harmonic function $\tilde f\in C^\infty(\mathbb{R}^{n})$.  By the Liouville theorem, $\tilde f$ is  constant.
\end{proof}
\begin{rema}
	The function $f\in C^\infty(\mathbb{R}^{n-1}\times[0,2])$ given by $f(x)=\exp(x^1/(2\,\pi))\,\sin(x^n/(2\,\pi))$ is non-negative, harmonic, satisfies $f(x)=0$ for all $x\in\mathbb{R}^{n-1}\times\{0,\,2\},$ but violates \eqref{bound}.
\end{rema}


\end{appendices}
%\bibliographystyle{amsplainnat}
%\bibliography{literature}
% \bib, bibdiv, biblist are defined by the amsrefs package.
\begin{bibdiv}
	\begin{biblist}
		
		\bib{andersonrodriguez}{article}{
			author={Anderson, Michael~T.},
			author={Rodr\'{\i}guez, Lucio},
			title={Minimal surfaces and {$3$}-manifolds of nonnegative {R}icci
				curvature},
			date={1989},
			ISSN={0025-5831},
			journal={Math. Ann.},
			volume={284},
			number={3},
			pages={461\ndash 475},
			url={https://doi.org/10.1007/BF01442497},
			review={\MR{1001714}},
		}
		
		\bib{ADM}{article}{
			author={Arnowitt, Richard},
			author={Deser, Stanley},
			author={Misner, Charles},
			title={Coordinate invariance and energy expressions in general
				relativity},
			date={1961},
			ISSN={0031-899X},
			journal={Phys. Rev. (2)},
			volume={122},
			pages={997\ndash 1006},
			review={\MR{127946}},
		}
		
		\bib{uniquecontinuation}{article}{
			author={Aronszajn, Nachman},
			title={A unique continuation theorem for solutions of elliptic partial
				differential equations or inequalities of second order},
			date={1957},
			ISSN={0021-7824},
			journal={J. Math. Pures Appl. (9)},
			volume={36},
			pages={235\ndash 249},
			review={\MR{92067}},
		}
		
		\bib{bartnik}{article}{
			author={Bartnik, Robert},
			title={The mass of an asymptotically flat manifold},
			date={1986},
			ISSN={0010-3640},
			journal={Comm. Pure Appl. Math.},
			volume={39},
			number={5},
			pages={661\ndash 693},
			url={https://doi.org/10.1002/cpa.3160390505},
			review={\MR{849427}},
		}
		
		\bib{shortproofs}{article}{
			author={Boas, Harold~P.},
			author={Boas, Ralph~P.},
			title={Short proofs of three theorems on harmonic functions},
			date={1988},
			ISSN={0002-9939},
			journal={Proc. Amer. Math. Soc.},
			volume={102},
			number={4},
			pages={906\ndash 908},
			url={https://doi.org/10.2307/2047332},
			review={\MR{934865}},
		}
		
		\bib{caffarelli}{book}{
			author={Caffarelli, Luis},
			author={Salsa, Sandro},
			title={A geometric approach to free boundary problems},
			series={Graduate Studies in Mathematics},
			publisher={American Mathematical Society, Providence, RI},
			date={2005},
			volume={68},
			ISBN={0-8218-3784-2},
			url={https://doi.org/10.1090/gsm/068},
			review={\MR{2145284}},
		}
		
		\bib{Carlottocalcvar}{article}{
			author={Carlotto, Alessandro},
			title={Rigidity of stable minimal hypersurfaces in asymptotically flat
				spaces},
			date={2016},
			ISSN={0944-2669},
			journal={Calc. Var. Partial Differential Equations},
			volume={55},
			number={3},
			pages={Art. 54, 20 pp.},
			url={https://doi.org/10.1007/s00526-016-0989-4},
			review={\MR{3500292}},
		}
		
		\bib{CCE}{article}{
			author={Carlotto, Alessandro},
			author={Chodosh, Otis},
			author={Eichmair, Michael},
			title={Effective versions of the positive mass theorem},
			date={2016},
			ISSN={0020-9910},
			journal={Invent. Math.},
			volume={206},
			number={3},
			pages={975\ndash 1016},
			url={https://doi.org/10.1007/s00222-016-0667-3},
			review={\MR{3573977}},
		}
		
		\bib{SchoenCarlotto}{article}{
			author={Carlotto, Alessandro},
			author={Schoen, Richard},
			title={Localizing solutions of the {E}instein constraint equations},
			date={2016},
			ISSN={0020-9910},
			journal={Invent. Math.},
			volume={205},
			number={3},
			pages={559\ndash 615},
			url={https://doi.org/10.1007/s00222-015-0642-4},
			review={\MR{3539922}},
		}
		
		\bib{CESH}{article}{
			author={Chodosh, Otis},
			author={Eichmair, Michael},
			author={Shi, Yuguang},
			author={Yu, Haobin},
			title={Isoperimetry, scalar curvature, and mass in asymptotically flat
				{R}iemannian 3-manifolds},
			date={2021},
			ISSN={0010-3640},
			journal={Comm. Pure Appl. Math.},
			volume={74},
			number={4},
			pages={865\ndash 905},
			url={https://doi.org/10.1002/cpa.21981},
			review={\MR{4221936}},
		}
		
		\bib{chodoshketover}{article}{
			author={Chodosh, Otis},
			author={Ketover, Daniel},
			title={Asymptotically flat three-manifolds contain minimal planes},
			date={2018},
			ISSN={0001-8708},
			journal={Adv. Math.},
			volume={337},
			pages={171\ndash 192},
			url={https://doi.org/10.1016/j.aim.2018.08.010},
			review={\MR{3853048}},
		}
		
		\bib{Eichmair-Metzer:2012}{article}{
			author={Eichmair, Michael},
			author={Metzger, Jan},
			title={On large volume preserving stable {CMC} surfaces in initial data
				sets},
			date={2012},
			ISSN={0022-040X},
			journal={J. Differential Geom.},
			volume={91},
			number={1},
			pages={81\ndash 102},
			url={http://projecteuclid.org/euclid.jdg/1343133701},
			review={\MR{2944962}},
		}
		
		\bib{eichmairmetzger}{article}{
			author={Eichmair, Michael},
			author={Metzger, Jan},
			title={Large isoperimetric surfaces in initial data sets},
			date={2013},
			ISSN={0022-040X},
			journal={J. Differential Geom.},
			volume={94},
			number={1},
			pages={159\ndash 186},
			url={http://projecteuclid.org/euclid.jdg/1361889064},
			review={\MR{3031863}},
		}
		
		\bib{eichmairmetzgerinvent}{article}{
			author={Eichmair, Michael},
			author={Metzger, Jan},
			title={Unique isoperimetric foliations of asymptotically flat manifolds
				in all dimensions},
			date={2013},
			ISSN={0020-9910},
			journal={Invent. Math.},
			volume={194},
			number={3},
			pages={591\ndash 630},
			url={https://doi.org/10.1007/s00222-013-0452-5},
			review={\MR{3127063}},
		}
		
		\bib{gallgheruniqueness}{article}{
			author={Gallagher, Paul},
			title={A criterion for uniqueness of tangent cones at infinity for
				minimal surfaces},
			date={2019},
			ISSN={1050-6926},
			journal={J. Geom. Anal.},
			volume={29},
			number={1},
			pages={370\ndash 377},
			url={https://doi.org/10.1007/s12220-018-9994-5},
			review={\MR{3897017}},
		}
		
		\bib{gilbargtrudinger}{book}{
			author={Gilbarg, David},
			author={Trudinger, Neil~S.},
			title={Elliptic partial differential equations of second order},
			series={Classics in Mathematics},
			publisher={Springer-Verlag, Berlin},
			date={2001},
			ISBN={3-540-41160-7},
			note={Reprint of the 1998 edition},
			review={\MR{1814364}},
		}
		
		\bib{yu}{article}{
			author={Haobin, Yu},
			title={Isoperimetry for asymptotically flat 3-manifolds with positive
				{ADM} mass},
			date={2022},
			journal={Mathematische Annalen},
			pages={18 pp.},
		}
		
		\bib{chaoli}{article}{
			author={Li, Chao},
			title={The dihedral rigidity conjecture for n-prisms},
			date={2019},
			journal={arXiv preprint arXiv:1907.03855},
			note={to appear in J. Differential Geom.},
		}
		
		\bib{chaoli2}{article}{
			author={Li, Chao},
			title={Dihedral rigidity of parabolic polyhedrons in hyperbolic spaces},
			date={2020},
			journal={SIGMA Symmetry Integrability Geom. Methods Appl.},
			volume={16},
			pages={Paper No. 099, 8 pp.},
			url={https://doi.org/10.3842/SIGMA.2020.099},
			review={\MR{4158684}},
		}
		
		\bib{liu}{article}{
			author={Liu, Gang},
			title={3-manifolds with nonnegative {R}icci curvature},
			date={2013},
			ISSN={0020-9910},
			journal={Invent. Math.},
			volume={193},
			number={2},
			pages={367\ndash 375},
			url={https://doi.org/10.1007/s00222-012-0428-x},
			review={\MR{3090181}},
		}
		
		\bib{mazetrosenberg}{article}{
			author={Mazet, Laurent},
			author={Rosenberg, Harold},
			title={Minimal planes in asymptotically flat three-manifolds},
			date={2022},
			ISSN={0022-040X},
			journal={J. Differential Geom.},
			volume={120},
			number={3},
			pages={533\ndash 556},
			url={https://doi.org/10.4310/jdg/1649953568},
			review={\MR{4408291}},
		}
		
		\bib{Morgan-Ros:2010}{article}{
			author={Morgan, Frank},
			author={Ros, Antonio},
			title={Stable constant-mean-curvature hypersurfaces are area minimizing
				in small {$L^1$} neighborhoods},
			date={2010},
			ISSN={1463-9963},
			journal={Interfaces Free Bound.},
			volume={12},
			number={2},
			pages={151\ndash 155},
			url={https://doi.org/10.4171/IFB/230},
			review={\MR{2652015}},
		}
		
		\bib{schoenconformal}{article}{
			author={Schoen, Richard},
			title={Conformal deformation of a {R}iemannian metric to constant scalar
				curvature},
			date={1984},
			ISSN={0022-040X},
			journal={J. Differential Geom.},
			volume={20},
			number={2},
			pages={479\ndash 495},
			url={http://projecteuclid.org/euclid.jdg/1214439291},
			review={\MR{788292}},
		}
		
		\bib{schoensimon}{article}{
			author={Schoen, Richard},
			author={Simon, Leon},
			title={Regularity of stable minimal hypersurfaces},
			date={1981},
			ISSN={0010-3640},
			journal={Comm. Pure Appl. Math.},
			volume={34},
			number={6},
			pages={741\ndash 797},
			url={https://doi.org/10.1002/cpa.3160340603},
			review={\MR{634285}},
		}
		
		\bib{pmt}{article}{
			author={Schoen, Richard},
			author={Yau, Shing~Tung},
			title={On the proof of the positive mass conjecture in general
				relativity},
			date={1979},
			ISSN={0010-3616},
			journal={Comm. Math. Phys.},
			volume={65},
			number={1},
			pages={45\ndash 76},
			url={http://projecteuclid.org/euclid.cmp/1103904790},
			review={\MR{526976}},
		}
		
		\bib{montecatini}{incollection}{
			author={Schoen, Richard~M.},
			title={Variational theory for the total scalar curvature functional for
				{R}iemannian metrics and related topics},
			date={1989},
			booktitle={Topics in calculus of variations ({M}ontecatini {T}erme, 1987)},
			series={Lecture Notes in Math.},
			volume={1365},
			publisher={Springer, Berlin},
			pages={120\ndash 154},
			url={https://doi.org/10.1007/BFb0089180},
			review={\MR{994021}},
		}
		
		\bib{schoentalk}{article}{
			author={Schoen, Richard~M.},
			title={Geometric questions in general relativity},
			date={December 3 2013},
			journal={Chen-Jung Hsu Lecture 2, Academia Sinica, ROC,},
			volume={88},
			note={slides available at
				\url{http://w3.math.sinica.edu.tw/HSU/page/2013/Dec.3.pdf}},
		}
		
		\bib{simonlectures}{book}{
			author={Simon, Leon},
			title={Lectures on geometric measure theory},
			series={Proceedings of the Centre for Mathematical Analysis, Australian
				National University},
			publisher={Australian National University, Centre for Mathematical Analysis,
				Canberra},
			date={1983},
			volume={3},
			ISBN={0-86784-429-9},
			review={\MR{756417}},
		}
		
		\bib{simonisolated}{incollection}{
			author={Simon, Leon},
			title={Isolated singularities of extrema of geometric variational
				problems},
			date={1985},
			booktitle={Harmonic mappings and minimal immersions ({M}ontecatini, 1984)},
			series={Lecture Notes in Math.},
			volume={1161},
			publisher={Springer, Berlin},
			pages={206\ndash 277},
			url={https://doi.org/10.1007/BFb0075139},
			review={\MR{821971}},
		}
		
		\bib{simonstrict}{article}{
			author={Simon, Leon},
			title={A strict maximum principle for area minimizing hypersurfaces},
			date={1987},
			ISSN={0022-040X},
			journal={J. Differential Geom.},
			volume={26},
			number={2},
			pages={327\ndash 335},
			url={http://projecteuclid.org/euclid.jdg/1214441373},
			review={\MR{906394}},
		}
		
		\bib{Simons}{article}{
			author={Simons, James},
			title={Minimal varieties in {R}iemannian manifolds},
			date={1968},
			ISSN={0003-486X},
			journal={Ann. of Math. (2)},
			volume={88},
			pages={62\ndash 105},
			url={https://doi.org/10.2307/1970556},
			review={\MR{233295}},
		}
		
		\bib{whitetangentconeuniqueness}{article}{
			author={White, Brian},
			title={Tangent cones to two-dimensional area-minimizing integral
				currents are unique},
			date={1983},
			ISSN={0012-7094},
			journal={Duke Math. J.},
			volume={50},
			number={1},
			pages={143\ndash 160},
			url={https://doi.org/10.1215/S0012-7094-83-05005-6},
			review={\MR{700134}},
		}
		
	\end{biblist}
\end{bibdiv}
\end{document} 