\begin{abstract}
Human affective behavior analysis focuses on analyzing human expressions or other behaviors to enhance the understanding of human psychology. 
The CVPR 2023 Competition on Affective Behavior Analysis in-the-wild (ABAW) is dedicated to providing high-quality and large-scale Aff-wild2 for the recognition of commonly used emotion representations, such as Action Units (AU), basic expression categories~(EXPR), and Valence-Arousal (VA). The competition is committed to making significant strides in improving the accuracy and practicality of affective analysis research in real-world scenarios.
In this paper, we introduce our submission to the CVPR 2023: ABAW5. 
Our approach involves several key components. First, we utilize the visual information from a Masked Autoencoder~(MAE) model that has been pre-trained on a large-scale face image dataset in a self-supervised manner. 
Next, we finetune the MAE encoder on the image frames from the Aff-wild2 for AU, EXPR and VA tasks, which can be regarded as a static and uni-modal training. 
Additionally, we leverage the multi-modal and temporal information from the videos and implement a transformer-based framework to fuse the multi-modal features. Our approach achieves impressive results in the ABAW5 competition, with an average F1 score of 55.49\% and 41.21\% in the AU and EXPR tracks, respectively, and an average CCC of 0.6372 in the VA track. Our approach ranks first in the EXPR and AU tracks, and second in the VA track. Extensive quantitative experiments and ablation studies demonstrate the effectiveness of our proposed method. 
\end{abstract}
