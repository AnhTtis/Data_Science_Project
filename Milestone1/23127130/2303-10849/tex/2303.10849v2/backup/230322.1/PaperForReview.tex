% CVPR 2023 Paper Template
% based on the CVPR template provided by Ming-Ming Cheng (https://github.com/MCG-NKU/CVPR_Template)
% modified and extended by Stefan Roth (stefan.roth@NOSPAMtu-darmstadt.de)

\documentclass[10pt,twocolumn,letterpaper]{article}

%%%%%%%%% PAPER TYPE  - PLEASE UPDATE FOR FINAL VERSION
% \usepackage[review]{cvpr}      % To produce the REVIEW version
%\usepackage{cvpr}              % To produce the CAMERA-READY version
\usepackage[pagenumbers]{cvpr} % To force page numbers, e.g. for an arXiv version

% Include other packages here, before hyperref.
\usepackage{graphicx}
\usepackage{amsmath}
\usepackage{amssymb}
\usepackage{booktabs}
\usepackage{multirow}

% It is strongly recommended to use hyperref, especially for the review version.
% hyperref with option pagebackref eases the reviewers' job.
% Please disable hyperref *only* if you encounter grave issues, e.g. with the
% file validation for the camera-ready version.
%
% If you comment hyperref and then uncomment it, you should delete
% ReviewTempalte.aux before re-running LaTeX.
% (Or just hit 'q' on the first LaTeX run, let it finish, and you
%  should be clear).
\usepackage[pagebackref,breaklinks,colorlinks]{hyperref}


% Support for easy cross-referencing
\usepackage[capitalize]{cleveref}
\crefname{section}{Sec.}{Secs.}
\Crefname{section}{Section}{Sections}
\Crefname{table}{Table}{Tables}
\crefname{table}{Tab.}{Tabs.}


%%%%%%%%% PAPER ID  - PLEASE UPDATE
\def\cvprPaperID{*****} % *** Enter the CVPR Paper ID here
\def\confName{CVPR}
\def\confYear{2023}


\begin{document}

%%%%%%%%% TITLE - PLEASE UPDATE
%\title{Facial Affective Analysis based on MAE and Multi-modal Information for 5th ABAW Competition}
\title{MAE-based Multi-modal Facial Affective Analysis in ABAW5 Competition}

% \author{First Author\\
% Institution1\\
% Institution1 address\\
% {\tt\small firstauthor@i1.org}
% % For a paper whose authors are all at the same institution,
% % omit the following lines up until the closing ``}''.
% % Additional authors and addresses can be added with ``\and'',
% % just like the second author.
% % To save space, use either the email address or home page, not both
% \and
% Second Author\\
% Institution2\\
% First line of institution2 address\\
% {\tt\small secondauthor@i2.org}
% }
\author{Wei Zhang, 
Bowen Ma, 
Feng Qiu,  
Yu Ding\footnotemark[2]\thanks{Corresponding Author.}\\
Virtual Human Group, Netease Fuxi AI Lab\\
{\tt\small \{zhangwei05,mabowen01,qiufeng,dingyu01\}@corp.netease.com}
}



\maketitle


%%%%%%%%% ABSTRACT
\begin{abstract}
Human affective behavior analysis focuses on analyzing human expressions or other behaviors, which helps improve the understanding of human psychology. 
The CVPR 2023 Competition on Affective Behavior Analysis in-the-wild (ABAW) is dedicated to providing high-quality and large-scale Aff-wild2 and Hume-Reaction datasets for the recognition of commonly used emotion representations, such as Action Units (AU), basic expression categories, Valence-Arousal (VA), and Emotional Reaction Intensity (ERI). The competition strives to make significant efforts toward improving the accuracy and applicability of affective analysis research in real-world scenarios. 
In this paper, we introduce our submission to the CVPR 2023: ABAW5. 
Our approach involves several key components. First, we utilize the visual information from an MAE model that has been pre-trained on a large-scale face image dataset in a self-supervised manner. 
Next, we fine-tune the MAE encoder on the ABAW challenges using single frames from the Aff-wild2 dataset.
Additionally, we leverage the multi-modal and temporal information from the videos and implement a transformer-based framework to fuse the multi-modal features. To further enhance model generalization, we introduce a novel two-branch collaboration training strategy that randomly interpolates the logits space. Our approach is supported by extensive quantitative experiments and ablation studies conducted on the Aff-Wild2 dataset and Hume-Reaction dataset, which demonstrate the effectiveness of our proposed method. 






   
\end{abstract}

%%%%%%%%% BODY TEXT
\section{Introduction}
\label{sec:intro}
In recent years, there has been a growing interest in the research of human affective behavior analysis due to its potential to provide a more accurate understanding of human emotions, which can be applied to design more friendly human-computer interaction. 
The commonly used human expression representations include Action Unit~(AU), basic expression categories, Valence-Arousal~(VA) and Emotional Reaction Intensity~(ERI). Specifically, AU is first proposed by Paul Ekman and Wallace Friesen in the 1970s~\cite{ekman1978facs}. It depicts the local regional movement of faces which can be used as the smallest unit to describe the expression. Basic expression categories divide expressions into a limited number of groups according to the emotion categories, e.g., happiness, sadness, etc. VA contains two continuous values Valence~(V) and Arousal~(A), which are ranged from [-1,1]. They can be used to describe the human emotional state. V represents the degree of positivity or negativity of emotion, while A describes the level of intensity or activation of emotion. ERI typically comprises a sequence of values representing multiple emotional dimensions that reflect the intensity of an individual's emotional response to a specific stimulus.


The fifth Competition on Affective Behavior Analysis in-the-wild~(ABAW5)~\cite{kollias2023abaw} is organized to focus on handling the obstacles in the process of human affective behavior analysis. It makes great efforts to construct large-scale multi-modal video datasets Aff-wild~\cite{zafeiriou2017aff,kollias2019deep,kollias2019face} and Aff-wild2~\cite{kollias2019expression,kollias2020analysing,kollias2021affect,kollias2021analysing,kollias2021distribution,kollias2022abaw2}. Aff-wild2 contains 598 videos and most of them have the three kinds of frame-wise annotated labels: AU, basic expression categories and VA. There are three challenges of ABAW5 for detecting these three kinds of expression representations. Besides, ABAW5 builds up a Hume-Reaction dataset which consists of about 75 hours of video recordings, recorded via a webcam, in the subjects’ homes. Each video in it has been self-annotated by the subjects themselves for the ERI intensity of 7 emotional experiences.


In this paper, we introduce our submission to the ABAW5. First of all, we train a Masked Autoencoder (MAE)~\cite{he2022masked,ma2022maeface} on our private large-scale face dataset in a self-supervised manner. Then, we choose the MAE encoder as our vision feature extractor to capture the visual features of faces. Given the extensive quantity of faces included in the dataset, the extracted features of the MAE encoder demonstrate strong generalization capabilities. We also finetune the MAE encoder on the specific tasks of AU detection, basic expression recognition~(EXPR) and VA estimation on the single frame. 
After that, to further exploit the temporal information and multi-modal information, we divide the videos into several short clips and perform clip-wise training on the downstream tasks. Specifically, we use the finetuned MAE encoder to extract visual features from each frame and employ the pre-trained audio models~(Hubert~\cite{hubert}, Wav2vec2.0~\cite{baevski2020wav2vec}, vggish~\cite{vggish}) to capture acoustic features. 
The concatenated features of visual and acoustic features are sent into a Transformer structure to acquire the temporal information for downstream tasks. Moreover, we design a dual-branch structure that contains Basic Learning Branch~(BLB) and Collaboration Learning Branch~(CLB). BLB and CLB have the same structure and shared feature extractors. By randomly interpolating the logit space of BLB and CLB, the model can enrich the feature space by implicitly creating some potential samples, which further enhance the model generalization.



% \section{Related Works}
% In this section, we introduce some recent works about the related tasks in the CVPR2023: ABAW5 competitions, including AU detection, expression recognition, VA estimation, and ERI estimation in the wild. We also briefly introduce the commonly used self-supervised learning method. 
% \subsection{AU Detection}
% \label{sec:AU_relat}
% For AU detection in the wild, there exist some challenges regarding the limited identity information and interference of diversity poses, illumination or occlusions. These disturbances restrain the model generation and cause the overfitting to the noise information.
% Several studies propose the use of a multi-task framework to incorporate additional auxiliary information as regularization, which introduces extra label constraints. Specifically, Zhang et al. \cite{multi-task1} proposed a streaming model that simultaneously performs AU detection, expression recognition, and VA regression. Similarly, Jin et al. \cite{multi-task2} and Thinh et al. \cite{multi-task3} combine the tasks of AU detection with expression recognition. JAA-Net~\cite{shao2021jaa} performs landmarks detection and AU detection at the same time. 


% Another effective approach to enhance the model generalization is to utilize the related tasks' pre-trained backbones. 
% Jiang et al. \cite{AU_2} use IResnet100~\cite{iresnet} that pre-trains on Glint360K~\cite{an2022pfc} and some private commercial datasets before conducting the AU detection task in the ABAW3. 
% Savchenko et al.~\cite{EXPR4} utilize the EfficientNet~\cite{tan2019efficientnet} that pre-trained on the face recognition task as the backbone.
% Zhang et al.\cite{multi-task1,zhang2022} introduce the pre-trained expression embedding model as the backbone and win the first prizes in ABAW2 and ABAW3.

% Multi-modal information is also involved in ABAW competitions. 
% Zhang et al.~\cite{zhang2022} capture three modalities of information - vision, acoustic, and text - and fused them using a transformer decoder structure. 
% Jin et al.~\cite{jin2021multi} extract the vision features from IResNet100 and the audio features from Mel Spectrogram. They also use the transformer structure for the fusion of multi-modal features.


% \subsection{Expression Recognition}
% The goal of expression recognition is to classify an input image into one of the basic emotion classes, such as happiness or sadness.
% Similar approaches to exploit the extra information regularization are mentioned in Sec.~\ref{sec:AU_relat}. Zhang et al.~\cite{multi-task1} utilize the prior expression embedding model and propose a multi-task framework. 
% Phan et al.\cite{EXPR5} employ the pre-trained model RegNet\cite{radosavovic2020designing} as the backbone and add the Transformer~\cite{vaswani2017attention} structure to extract the temporal information. 
% Kim \textit{et al.} \cite{EXPR6} use Swin transformer \cite{liu2021swin} as the backbone and exploit the extra auxiliary from the audio modal. 
% Wang et al. \cite{wang2021multi} propose a semi-supervised framework to predict pseudo-labels for unlabeled data, which helps improve the model's generalization to some extent.
% Xue \textit{et al.}~\cite{EXPR3} develop a CFC network that uses different branches to train the easy-distinguished and hard-distinguished emotion categories.  




% \subsection{VA Estimation}




% \subsection{ERI Estimation}

% \subsection{Self-supervised Learning in Affective Analysis}
% It has been pointed out~\cite{zafeiriou2017aff} that annotating the corresponding emotion/AU/VA labels from real-world facial images costs a large amount of time/labor, which limits the development of the affective analysis. It is a potential solution to make use of the self-supervised learning~(SSL) method to exploit more knowledge from the existing large-scale unlabelled data. There have been several works to develop SSL methods to enhance the accuracy in the area of affective analysis. 
% Shu et al.\cite{shu2022revisiting} explore different strategies in the choice of positives and negatives to enforce the expression-related features and reduce the interference of other facial attributes. They improve the accuracy of expression recognition based on the contrastive SSL methods (e.g. SimCLR~\cite{chen2020simple})
% Ma et al.~\cite{ma2022maeface} pre-train the Masked Autoencoder~(MAE) structure on the large-scale face images and finetune it on the AU detection, which achieves the state-of-the-art performance on the BP4D~\cite{zhang2014bp4d} and DISFA~\cite{mavadati2013disfa}.








\begin{figure*}
    \centering
    \includegraphics[ width=1\linewidth]{images/pipeline.pdf}
    %\vspace{-1.5em}
    \caption{Pipeline}
    \label{fig:pipeline}
\end{figure*}





\section{Method}
\subsection{MAE Pre-train}
Different from the traditional MAE, our MAE is pre-trained on the facial image dataset to focus on learning the facial vision features. We construct a large-scale facial image dataset that contains images from the existing facial image datasets, e.g., AffectNet~\cite{AffectNet}, CASIA-WebFace~\cite{CASIA-Webface}, CelebA~\cite{CelebA} and IMDB-WIKI~\cite{IMDB-WIKI}.
Then we pre-train the MAE model on the dataset in a self-supervised manner. Specifically, our MAE consists of a ViT-Base encoder and a ViT decoder based on the structure of Vision Transformer (ViT)~\cite{ViT}. 
The MAE pre-training procedure follows a masking-then-reconstruct method, whereby images are first divided into a series of patches (16x16) and 75\% of them are randomly masked. 
These masked images are sent to the MAE encoder and the complete images should be reconstructed by MAE decoder~(See Fig.~\ref{fig:mae}(a)). The loss function of MAE pre-training is the pixel-wise L2 loss to make the reconstructed images close to the target images.

\begin{figure}
    \centering
    \includegraphics[ width=1\linewidth]{images/MAE1.pdf}
    %\vspace{-1.5em}
    \caption{Description of MAE self-supervised training and finetuning on the downstreaming tasks. (a) The input face image is randomly masked 75\% and MAE requires to reconstruct the complete image. (b) After SSL, MAE encoder is used to finetune on the Aff-wild2 for AU detection, expression recognition and VA estimation.}
    \label{fig:mae}
\end{figure}

Once self-supervised learning is complete, we remove the MAE decoder and replace it with a fully connected layer attached to the MAE encoder (See Fig.~\ref{fig:mae}(b)). This allows us to fine-tune downstream tasks: AU detection, expression recognition, and VA estimation on the Aff-wild2 dataset. It is important to note that this process is based on frame-wise training, without taking into account temporal or other modal information. The corresponding loss functions for these three tasks are as follows:

\begin{equation}
    \mathcal{L}_{\textit{AU\_CE}}=-\frac{1}{12}\sum_{j=1}^{12} W_{au_j}[y_{j}\log\hat{y}_{j} 
     + (1-y_{j})\log(1-\hat{y}_{j})]. 
\label{eq:1}
\end{equation}
\begin{equation}
    \mathcal{L}_{\textit{EXPR\_CE}}=-\frac{1}{8}\sum_{j=1}^{8}W_{exp_j} z_{j}\log\hat{z}_{j}.
\label{eq:2}
\end{equation}

\begin{equation}
\begin{split}
\mathcal{L}_{\textit{VA\_CCC}}= 1 - CCC(\hat{v}_{batch_i},v_{batch_i}) \\
     + 1 - CCC(\hat{a}_{batch_i},a_{batch_i}) 
\end{split}
\label{eq:3}
\end{equation}
\begin{equation}
\label{CCC}
    \textit{CCC}(\mathcal{X},\mathcal{\hat{X}})=\frac{2\rho_{\mathcal{X\hat{X}}}\delta_{\mathcal{X}}\delta_{\mathcal{\hat{X}}}}{\delta_{\mathcal{X}}^2+\delta_{\mathcal{\hat{X}}}^2+(\mu_{\mathcal{X}}-\mu_{\mathcal{\hat{X}}})^{2}}.
\end{equation}
where $\hat{y}$, $\hat{z}$, $\hat{v}$ and $\hat{a}$ denote the model's predictions for AU, expression category, Valence, and Arousal, respectively. The symbols without hats refer to the ground truth. $\delta_{\mathcal{X}}$,$\delta_{\mathcal{\hat{X}}}$ indicate the standard deviations of $\mathcal{X}$ and $\mathcal{\hat{X}}$, respectively. $\mu_{\mathcal{X}}$ and $\mu_{\mathcal{\hat{X}}}$ are the corresponding means and  $\rho_{\mathcal{X\hat{X}}}$ is the correlation coefficient.
For the AU and EXPR tasks, we utilize weighted cross-entropy as the loss function. The weights for different categories, represented by $W_{au_j}$ and $W_{exp_j}$, are inversely proportional to the class number in the training set. 

\subsection{Temporal and Multi-modal Features Extraction}
To further exploit the temporal and multi-modal features for AU, EXPR and VA tasks, we design the sequence-based model which combines the audio features. To concretely, we first divide the videos into several short clips, each having an equivalent frame number of K. 
We construct the Basic Learning Branch~(BLB) to perform the sequence-wise training which can be seen in Fig~\ref{fig:pipeline}. 

Given a video clip $C_i$ and the corresponding audio clips $A_i$, we use the finetuned MAE encoder and some existing pre-train audio embedding models~(e.g. Hubert~\cite{hubert}, Wav2vec2.0~\cite{baevski2020wav2vec}, vggish~\cite{vggish}.) to extract the vision and acoustic features $F^{i}_{vis}$ and $F^{i}_{aud}$ for each frame separately. Then we concatenate $F^{i}_{vis}$ and $F^{i}_{aud}$ and sent them into a Transformer~\cite{vaswani2017attention} encoder structure to exploit the temporal correlations between them. 
The Transformer encoder comprises of four encoder layers with a dropout ratio of 0.3. The output of the Transformer encoder is then directed towards a fully connected layer to resize the final output size, which is tailored to fit various tasks. In the training process of BLB, we flatten the sequence result of a clip and use the same loss function as equations~\ref{eq:1}, \ref{eq:2}, \ref{eq:3}.



\subsection{Dual Branch structure}
To further enhance the model generalization, we propose a two-branch structure to perform collaboration training. 
After the BLB training, we fix its parameters and share the multi-modal feature extraction module~(MAE Encoder and Audio Feature Extractor) with the Collaboration Learning Branch~(CLB). CLB has the same structure as BLB but different train data distribution $D_{CLB}$. Specifically, we first filter the hard training samples which are hard to converge after training. For example, we exclude samples that continue to display a large sequence loss even after the completion of training. We add these hard samples into $D_{CLB}$ in order to focus on learning more hard samples.
To preserve the original training data distribution of $D_{init}$, we augment $D_{CLB}$ with a selection of random samples from $D_{init}$ until $D_{CLB}$ contains the same number of samples as $D_{init}$. 


After building up $D_{CLB}$, we commence with the collaboration training by our proposed Dual-branch collaboration Learning~(DCL). Given a sample~($C_i$,$A_i$) from $D_{init}$ and a sample~($C_j$,$A_j$) from $D_{CLB}$, the corresponding logits outputs of BLB and CLB are $h^{i}_{BLB}$ and $h^{j}_{CLB}$, respectively. Then we perform the randomly linear interpolation in the logits space.
The interpolated logit can be denoted as:
\begin{equation}
\begin{split}
h = \alpha h^{i}_{BLB} \oplus(1-\alpha) h^{j}_{CLB},
\end{split}
\label{eq:res_integrate}
\end{equation}
where $\oplus$ denotes the element-wise sum and $\alpha$ is randomly sampled from the Beta distribution controlled by the hyper-parameters $\tau_1$ and $\tau_2$. We denote the final output after logit $h$ as $\hat{o}$. The final loss function of DCL is as follows:
\begin{equation}
\begin{split}
\mathcal{L}_{DLC} = \alpha L(\hat{o},y_{i}) + (1-\alpha) L(\hat{o},y_{j}),
\label{eq: loss_stage2}
\end{split}
\end{equation}
where $y$ denotes the corresponding labels of different tasks, $L(,)$ represents the corresponding loss functions~\ref{eq:1}, \ref{eq:2}, \ref{eq:3} mentioned before. 

By performing this linear interpolation, we effectively augment the logits space and thereby construct a greater number of potential unseen samples. This approach has the benefit of enriching the feature space and enhancing the model generalization.


\section{Experiment}
\subsection{Experimental Setting}
We processed all videos in the Aff-Wild2 datasets into frames by OpenCV and employ the OpenFace~\cite{baltrusaitis2018openface} detector to crop all facial images into $224\times 224$ scale. 
We pre-train MAE on the large-scale face image dataset for 800 epochs with the AdamW~\cite{loshchilov2017adamw} optimizer. We set the batch size as 4096 and the learning rate as 0.0024. Our training process is implemented based on PyTorch and trained on 8 NVIDIA A30 GPUs.
For the single BLB training, we set the clip length as 100. The batch size is set to 32, and the learning rate is set to 0.0001. The BLB training process takes around 20 epochs using the AdamW optimizer. In the DCL training, we set the $\alpha$ to follow the distribution of $Beta(2,2)$, other experimental settings are the same as single BLB training.


Besides, we also utilize some training tricks during the training process. To concretely, we adjust the learning rate according to the CosineAnnealing policy. Also, to obtain robust training, we also take advantage of the Exponential Weighted Average~(EMA) policy. Besides, we leverage the model soup~\cite{wortsman2022model} to further enhance the performance on the validation set.


\subsection{Metric}
For AU detection and expression classification, we calculate the F1-Score~(F1) for each class to evaluate the prediction results. For VA estimation, we calculate the Concordance Correlation Coefficient (CCC) for valence and arousal respectively. The definition of CCC can be seen equ.~\ref{CCC}. In the case of ERI estimation, we utilize Pearson's Correlation Coefficient (PCC) for each class as the metric. The specific definitions for each challenge are as follows:
\begin{equation}
S_{AU} = \frac{1}{N_{au}}\sum_{}^{}  F1_{au_{i}} 
\label{equ:s_au}
\end{equation}

\begin{equation}
S_{EXP}= \frac{1}{N_{exp}}\sum_{}^{}  F1_{exp_{i}}
\label{equ:s_exp}
\end{equation}

\begin{equation}
S_{VA} = 0.5*(CCC(\hat{v},v)+CCC(\hat{a},a))
\end{equation}

\begin{equation}
S_{ERI} = \frac{1}{N_{exp}}\sum_{}^{} PCC(\hat{p}_{exp_{i}},p_{exp_{i}})
\end{equation}

\begin{equation}
    PCC = \frac{Cov({x},{\hat{x}})}{\delta _{{x}}\delta _{{\hat{x}}}}
    \label{equ:pcc}
\end{equation}
where $Cov(,)$ represents the covariance.




\subsection{Results on validation set}
\subsubsection{AU Challenge}
We show our experimental results of different stages of our framework on the official validation set in Tab. \ref{tab:val_AU}.  We evaluate the model by the average F1 metric in equ.~\ref{equ:s_au}. To enhance the model generalization, we also perform the 5-fold cross-validation according to random video split in the existing labeled data. The final prediction of test set comes from the ensemble results of these models. 
\begin{table}[]
\small
    \centering
    \setlength{\tabcolsep}{1.0mm}{
    \begin{tabular}{c|cccccc}
    \hline
       \multirow{2}{*}{Method} & \multicolumn{6}{c}{Validation Set} \\ \cline{2-7} 
        & Official & fold1 & fold2 & fold3 & fold4 & fold5 \\
       \hline\hline
       MAE finetune  & 0.5527 & 0.5430 & 0.5724 & 0.5440 & 0.5179  &  \textbf{0.5416} \\
       Our-BLB &  0.5501  & 0.5547 & 0.5842 & 0.5441 & 0.5240 &  0.5337 \\
       Our-DCL &  \textbf{0.5667}  & \textbf{0.5647} & \textbf{0.5929} & \textbf{0.5460} & \textbf{0.5345} & 0.5411 \\
       \hline
    \end{tabular}
    \caption{Average AU F1 of the official and 5-fold validation set.}
    \label{tab:val_AU}    
    }
\end{table}





\subsubsection{EXPR Challenge}
We show our experimental results of different stages of our framework on the official validation set in Tab. \ref{tab:val_expr}.  We evaluate the model by the average F1 metric in equ.~\ref{equ:pcc}. To enhance the model generalization, we also perform the 5-fold cross-validation according to random video clip in the existing labeled data. The final prediction of test set comes from the ensemble results of these models. 


\begin{table}[]
\small
    \centering
    \setlength{\tabcolsep}{1.0mm}{
    \begin{tabular}{c|cccccc}
    \hline
       \multirow{2}{*}{Method} & \multicolumn{6}{c}{Validation Set} \\ \cline{2-7} 
        & Official & fold1 & fold2 & fold3 & fold4 & fold5 \\
       \hline\hline
       MAE finetune  & 0.4679 & 0.4203 & 0.4709 & 0.4241 & 0.5066  &  0.3493 \\
       Our-BLB & 0.4817  & 0.4646 & 0.5323 & 0.4624 & 0.5547 &  0.3953 \\
       Our-DCL & \textbf{0.4952} & \textbf{0.4758} & \textbf{0.5376} & \textbf{0.4634} & \textbf{0.5589} & \textbf{0.3981} \\
       \hline
    \end{tabular}
    \caption{Average EXP F1 of the official and 5-fold validation set.}
    \label{tab:val_expr}    
    }
\end{table}


\subsubsection{VA Estimation}
We show our experimental results of different stages of our framework on the official validation set in Tab. \ref{tab:val_AU}.  We evaluate the model by the CCC of Valence and Arousal in equ.~\ref{CCC}. To enhance the model generalization, we also perform the 5-fold cross-validation according to random video split in the existing labeled data. The final prediction of test set comes from the ensemble results of these models. In VA task, we find the improvement of our CLB is slight. This may caused by the differences between the classification and regression tasks.

\begin{table}[]
\small
    \centering
    \setlength{\tabcolsep}{1.0mm}{
    \begin{tabular}{c|cccccc}
    \hline
       \multirow{2}{*}{Method} & \multicolumn{6}{c}{Validation Set} \\ \cline{2-7} 
        & Official & fold1 & fold2 & fold3 & fold4 & fold5 \\
       \hline\hline
       MAE finetune  & \textbf{0.4758} & 0.5496 & 0.5097 & 0.5333 & 0.5005  &  0.6000 \\
       Our-BLB &  0.4643  & \textbf{0.5927} & \textbf{0.5647} & 0.5679 & 0.5567 &  0.6478  \\
       Our-DCL & 0.4698 & 0.5861 & 0.5632 & \textbf{0.5703} & \textbf{0.5581} & \textbf{0.6641}  \\
       \hline
    \end{tabular}
    \caption{CCC of Valence on the official and 5-fold validation set.}
    \label{tab:val_expr}    
    }
\end{table}


\begin{table}[]
\small
    \centering
    \setlength{\tabcolsep}{1.0mm}{
    \begin{tabular}{c|cccccc}
    \hline
       \multirow{2}{*}{Method} & \multicolumn{6}{c}{Validation Set} \\ \cline{2-7} 
        & Official & fold1 & fold2 & fold3 & fold4 & fold5 \\
       \hline\hline
       MAE finetune  & 0.6208 & 0.6216 & 0.5805 & 0.6692&  0.6054  & 0.6657 \\
       Our-BLB &  0.6407 & \textbf{0.6542} & \textbf{0.6267} & \textbf{0.6959} & 0.6456 & 0.7056  \\
       Our-DCL & \textbf{0.6443} & 0.6513 & 0.6247 & 0.6729 & \textbf{0.6478}  & \textbf{0.7110}  \\
       \hline
    \end{tabular}
    \caption{CCC of Arousal on the official and 5-fold validation set.}
    \label{tab:val_expr}    
    }
\end{table}


\subsubsection{ERI Estimation}
We show our experimental results of different stages of our framework on the official validation set in Tab. \ref{tab:val_expr}.  We evaluate the model by the Pearson's Correlations Coefficient (PCC) metric in equ.~\ref{equ:pcc}. To enhance the model generalization, we also perform the 5-fold cross-validation according to random video split in the existing labeled data. The final prediction of test set comes from the ensemble results of these models. 

\begin{table}[]
\small
    \centering
    \setlength{\tabcolsep}{1.0mm}{
    \begin{tabular}{c|cccccc}
    \hline
       \multirow{2}{*}{Method} & \multicolumn{6}{c}{Validation Set} \\ \cline{2-7} 
        & Official & fold1 & fold2 & fold3 & fold4 & fold5 \\
       \hline\hline
    Ours  & 0.4120 & 0.4229 & 0.4199 & 0.4229&  0.4266  & 0.4049 \\

       \hline
    \end{tabular}
    \caption{PCC on the official and 5-fold validation set.}
    \label{tab:val_eri}    
    }
\end{table}


%-------------------------------------------------------------------------
% \subsection{Language}

% All manuscripts must be in English.

% \subsection{Dual submission}

% Please refer to the author guidelines on the \confName\ \confYear\ web page for a
% discussion of the policy on dual submissions.

% \subsection{Paper length}
% Papers, excluding the references section, must be no longer than eight pages in length.
% The references section will not be included in the page count, and there is no limit on the length of the references section.
% For example, a paper of eight pages with two pages of references would have a total length of 10 pages.
% {\bf There will be no extra page charges for \confName\ \confYear.}

% Overlength papers will simply not be reviewed.
% This includes papers where the margins and formatting are deemed to have been significantly altered from those laid down by this style guide.
% Note that this \LaTeX\ guide already sets figure captions and references in a smaller font.
% The reason such papers will not be reviewed is that there is no provision for supervised revisions of manuscripts.
% The reviewing process cannot determine the suitability of the paper for presentation in eight pages if it is reviewed in eleven.

% %-------------------------------------------------------------------------
% \subsection{The ruler}
% The \LaTeX\ style defines a printed ruler which should be present in the version submitted for review.
% The ruler is provided in order that reviewers may comment on particular lines in the paper without circumlocution.
% If you are preparing a document using a non-\LaTeX\ document preparation system, please arrange for an equivalent ruler to appear on the final output pages.
% The presence or absence of the ruler should not change the appearance of any other content on the page.
% The camera-ready copy should not contain a ruler.
% (\LaTeX\ users may use options of cvpr.sty to switch between different versions.)

% Reviewers:
% note that the ruler measurements do not align well with lines in the paper --- this turns out to be very difficult to do well when the paper contains many figures and equations, and, when done, looks ugly.
% Just use fractional references (\eg, this line is $087.5$), although in most cases one would expect that the approximate location will be adequate.


% \subsection{Paper ID}
% Make sure that the Paper ID from the submission system is visible in the version submitted for review (replacing the ``*****'' you see in this document).
% If you are using the \LaTeX\ template, \textbf{make sure to update paper ID in the appropriate place in the tex file}.


% \subsection{Mathematics}



% Please number all of your sections and displayed equations as in these examples:
% \begin{equation}
%   E = m\cdot c^2
%   \label{eq:important}
% \end{equation}
% and
% \begin{equation}
%   v = a\cdot t.
%   \label{eq:also-important}
% \end{equation}
% It is important for readers to be able to refer to any particular equation.
% Just because you did not refer to it in the text does not mean some future reader might not need to refer to it.
% It is cumbersome to have to use circumlocutions like ``the equation second from the top of page 3 column 1''.
% (Note that the ruler will not be present in the final copy, so is not an alternative to equation numbers).
% All authors will benefit from reading Mermin's description of how to write mathematics:
% \url{http://www.pamitc.org/documents/mermin.pdf}.

% \subsection{Blind review}

% Many authors misunderstand the concept of anonymizing for blind review.
% Blind review does not mean that one must remove citations to one's own work---in fact it is often impossible to review a paper unless the previous citations are known and available.

% Blind review means that you do not use the words ``my'' or ``our'' when citing previous work.
% That is all.
% (But see below for tech reports.)

% Saying ``this builds on the work of Lucy Smith [1]'' does not say that you are Lucy Smith;
% it says that you are building on her work.
% If you are Smith and Jones, do not say ``as we show in [7]'', say ``as Smith and Jones show in [7]'' and at the end of the paper, include reference 7 as you would any other cited work.

% An example of a bad paper just asking to be rejected:
% \begin{quote}
% \begin{center}
%     An analysis of the frobnicatable foo filter.
% \end{center}

%    In this paper we present a performance analysis of our previous paper [1], and show it to be inferior to all previously known methods.
%    Why the previous paper was accepted without this analysis is beyond me.

%    [1] Removed for blind review
% \end{quote}


% An example of an acceptable paper:
% \begin{quote}
% \begin{center}
%      An analysis of the frobnicatable foo filter.
% \end{center}

%    In this paper we present a performance analysis of the  paper of Smith \etal [1], and show it to be inferior to all previously known methods.
%    Why the previous paper was accepted without this analysis is beyond me.

%    [1] Smith, L and Jones, C. ``The frobnicatable foo filter, a fundamental contribution to human knowledge''. Nature 381(12), 1-213.
% \end{quote}

% If you are making a submission to another conference at the same time, which covers similar or overlapping material, you may need to refer to that submission in order to explain the differences, just as you would if you had previously published related work.
% In such cases, include the anonymized parallel submission~\cite{Authors14} as supplemental material and cite it as
% \begin{quote}
% [1] Authors. ``The frobnicatable foo filter'', F\&G 2014 Submission ID 324, Supplied as supplemental material {\tt fg324.pdf}.
% \end{quote}

% Finally, you may feel you need to tell the reader that more details can be found elsewhere, and refer them to a technical report.
% For conference submissions, the paper must stand on its own, and not {\em require} the reviewer to go to a tech report for further details.
% Thus, you may say in the body of the paper ``further details may be found in~\cite{Authors14b}''.
% Then submit the tech report as supplemental material.
% Again, you may not assume the reviewers will read this material.

% Sometimes your paper is about a problem which you tested using a tool that is widely known to be restricted to a single institution.
% For example, let's say it's 1969, you have solved a key problem on the Apollo lander, and you believe that the CVPR70 audience would like to hear about your
% solution.
% The work is a development of your celebrated 1968 paper entitled ``Zero-g frobnication: How being the only people in the world with access to the Apollo lander source code makes us a wow at parties'', by Zeus \etal.

% You can handle this paper like any other.
% Do not write ``We show how to improve our previous work [Anonymous, 1968].
% This time we tested the algorithm on a lunar lander [name of lander removed for blind review]''.
% That would be silly, and would immediately identify the authors.
% Instead write the following:
% \begin{quotation}
% \noindent
%    We describe a system for zero-g frobnication.
%    This system is new because it handles the following cases:
%    A, B.  Previous systems [Zeus et al. 1968] did not  handle case B properly.
%    Ours handles it by including a foo term in the bar integral.

%    ...

%    The proposed system was integrated with the Apollo lunar lander, and went all the way to the moon, don't you know.
%    It displayed the following behaviours, which show how well we solved cases A and B: ...
% \end{quotation}
% As you can see, the above text follows standard scientific convention, reads better than the first version, and does not explicitly name you as the authors.
% A reviewer might think it likely that the new paper was written by Zeus \etal, but cannot make any decision based on that guess.
% He or she would have to be sure that no other authors could have been contracted to solve problem B.
% \medskip

% \noindent
% FAQ\medskip\\
% {\bf Q:} Are acknowledgements OK?\\
% {\bf A:} No.  Leave them for the final copy.\medskip\\
% {\bf Q:} How do I cite my results reported in open challenges?
% {\bf A:} To conform with the double-blind review policy, you can report results of other challenge participants together with your results in your paper.
% For your results, however, you should not identify yourself and should not mention your participation in the challenge.
% Instead present your results referring to the method proposed in your paper and draw conclusions based on the experimental comparison to other results.\medskip\\

% \begin{figure*}[t]
%   \centering
  
%   \fbox{\rule{0pt}{2in} \rule{0.9\linewidth}{0pt}}
%    %\includegraphics[width=0.8\linewidth]{egfigure.eps}

%    \caption{Example of caption.
%    It is set in Roman so that mathematics (always set in Roman: $B \sin A = A \sin B$) may be included without an ugly clash.}
%    \label{fig:onecol}
% \end{figure*}

% \subsection{Miscellaneous}

% \noindent
% Compare the following:\\
% \begin{tabular}{ll}
%  \verb'$conf_a$' &  $conf_a$ \\
%  \verb'$\mathit{conf}_a$' & $\mathit{conf}_a$
% \end{tabular}\\
% See The \TeX book, p165.

% The space after \eg, meaning ``for example'', should not be a sentence-ending space.
% So \eg is correct, {\em e.g.} is not.
% The provided \verb'\eg' macro takes care of this.

% When citing a multi-author paper, you may save space by using ``et alia'', shortened to ``\etal'' (not ``{\em et.\ al.}'' as ``{\em et}'' is a complete word).
% If you use the \verb'\etal' macro provided, then you need not worry about double periods when used at the end of a sentence as in Alpher \etal.
% However, use it only when there are three or more authors.
% Thus, the following is correct:
%    ``Frobnication has been trendy lately.
%    It was introduced by Alpher~\cite{Alpher02}, and subsequently developed by
%    Alpher and Fotheringham-Smythe~\cite{Alpher03}, and Alpher \etal~\cite{Alpher04}.''

% This is incorrect: ``... subsequently developed by Alpher \etal~\cite{Alpher03} ...'' because reference~\cite{Alpher03} has just two authors.


% % Update the cvpr.cls to do the following automatically.
% % For this citation style, keep multiple citations in numerical (not
% % chronological) order, so prefer \cite{Alpher03,Alpher02,Authors14} to
% % \cite{Alpher02,Alpher03,Authors14}.


% \begin{figure*}
%   \centering
%   \begin{subfigure}{0.68\linewidth}
%     \fbox{\rule{0pt}{2in} \rule{.9\linewidth}{0pt}}
%     \caption{An example of a subfigure.}
%     \label{fig:short-a}
%   \end{subfigure}
%   \hfill
%   \begin{subfigure}{0.28\linewidth}
%     \fbox{\rule{0pt}{2in} \rule{.9\linewidth}{0pt}}
%     \caption{Another example of a subfigure.}
%     \label{fig:short-b}
%   \end{subfigure}
%   \caption{Example of a short caption, which should be centered.}
%   \label{fig:short}
% \end{figure*}

% %------------------------------------------------------------------------
% \section{Formatting your paper}
% \label{sec:formatting}

% All text must be in a two-column format.
% The total allowable size of the text area is $6\frac78$ inches (17.46 cm) wide by $8\frac78$ inches (22.54 cm) high.
% Columns are to be $3\frac14$ inches (8.25 cm) wide, with a $\frac{5}{16}$ inch (0.8 cm) space between them.
% The main title (on the first page) should begin 1 inch (2.54 cm) from the top edge of the page.
% The second and following pages should begin 1 inch (2.54 cm) from the top edge.
% On all pages, the bottom margin should be $1\frac{1}{8}$ inches (2.86 cm) from the bottom edge of the page for $8.5 \times 11$-inch paper;
% for A4 paper, approximately $1\frac{5}{8}$ inches (4.13 cm) from the bottom edge of the
% page.

% %-------------------------------------------------------------------------
% \subsection{Margins and page numbering}

% All printed material, including text, illustrations, and charts, must be kept
% within a print area $6\frac{7}{8}$ inches (17.46 cm) wide by $8\frac{7}{8}$ inches (22.54 cm)
% high.
% %
% Page numbers should be in the footer, centered and $\frac{3}{4}$ inches from the bottom of the page.
% The review version should have page numbers, yet the final version submitted as camera ready should not show any page numbers.
% The \LaTeX\ template takes care of this when used properly.



% %-------------------------------------------------------------------------
% \subsection{Type style and fonts}

% Wherever Times is specified, Times Roman may also be used.
% If neither is available on your word processor, please use the font closest in
% appearance to Times to which you have access.

% MAIN TITLE.
% Center the title $1\frac{3}{8}$ inches (3.49 cm) from the top edge of the first page.
% The title should be in Times 14-point, boldface type.
% Capitalize the first letter of nouns, pronouns, verbs, adjectives, and adverbs;
% do not capitalize articles, coordinate conjunctions, or prepositions (unless the title begins with such a word).
% Leave two blank lines after the title.

% AUTHOR NAME(s) and AFFILIATION(s) are to be centered beneath the title
% and printed in Times 12-point, non-boldface type.
% This information is to be followed by two blank lines.

% The ABSTRACT and MAIN TEXT are to be in a two-column format.

% MAIN TEXT.
% Type main text in 10-point Times, single-spaced.
% Do NOT use double-spacing.
% All paragraphs should be indented 1 pica (approx.~$\frac{1}{6}$ inch or 0.422 cm).
% Make sure your text is fully justified---that is, flush left and flush right.
% Please do not place any additional blank lines between paragraphs.

% Figure and table captions should be 9-point Roman type as in \cref{fig:onecol,fig:short}.
% Short captions should be centred.

% \noindent Callouts should be 9-point Helvetica, non-boldface type.
% Initially capitalize only the first word of section titles and first-, second-, and third-order headings.

% FIRST-ORDER HEADINGS.
% (For example, {\large \bf 1. Introduction}) should be Times 12-point boldface, initially capitalized, flush left, with one blank line before, and one blank line after.

% SECOND-ORDER HEADINGS.
% (For example, { \bf 1.1. Database elements}) should be Times 11-point boldface, initially capitalized, flush left, with one blank line before, and one after.
% If you require a third-order heading (we discourage it), use 10-point Times, boldface, initially capitalized, flush left, preceded by one blank line, followed by a period and your text on the same line.

% %-------------------------------------------------------------------------
% \subsection{Footnotes}

% Please use footnotes\footnote{This is what a footnote looks like.
% It often distracts the reader from the main flow of the argument.} sparingly.
% Indeed, try to avoid footnotes altogether and include necessary peripheral observations in the text (within parentheses, if you prefer, as in this sentence).
% If you wish to use a footnote, place it at the bottom of the column on the page on which it is referenced.
% Use Times 8-point type, single-spaced.


% %-------------------------------------------------------------------------
% \subsection{Cross-references}

% For the benefit of author(s) and readers, please use the
% {\small\begin{verbatim}
%   \cref{...}
% \end{verbatim}}  command for cross-referencing to figures, tables, equations, or sections.
% This will automatically insert the appropriate label alongside the cross-reference as in this example:
% \begin{quotation}
%   To see how our method outperforms previous work, please see \cref{fig:onecol} and \cref{tab:example}.
%   It is also possible to refer to multiple targets as once, \eg~to \cref{fig:onecol,fig:short-a}.
%   You may also return to \cref{sec:formatting} or look at \cref{eq:also-important}.
% \end{quotation}
% If you do not wish to abbreviate the label, for example at the beginning of the sentence, you can use the
% {\small\begin{verbatim}
%   \Cref{...}
% \end{verbatim}}
% command. Here is an example:
% \begin{quotation}
%   \Cref{fig:onecol} is also quite important.
% \end{quotation}

% %-------------------------------------------------------------------------
% \subsection{References}

% List and number all bibliographical references in 9-point Times, single-spaced, at the end of your paper.
% When referenced in the text, enclose the citation number in square brackets, for
% example~\cite{Authors14}.
% Where appropriate, include page numbers and the name(s) of editors of referenced books.
% When you cite multiple papers at once, please make sure that you cite them in numerical order like this \cite{Alpher02,Alpher03,Alpher05,Authors14b,Authors14}.
% If you use the template as advised, this will be taken care of automatically.

% \begin{table}
%   \centering
%   \begin{tabular}{@{}lc@{}}
%     \toprule
%     Method & Frobnability \\
%     \midrule
%     Theirs & Frumpy \\
%     Yours & Frobbly \\
%     Ours & Makes one's heart Frob\\
%     \bottomrule
%   \end{tabular}
%   \caption{Results.   Ours is better.}
%   \label{tab:example}
% \end{table}

% %-------------------------------------------------------------------------
% \subsection{Illustrations, graphs, and photographs}

% All graphics should be centered.
% In \LaTeX, avoid using the \texttt{center} environment for this purpose, as this adds potentially unwanted whitespace.
% Instead use
% {\small\begin{verbatim}
%   \centering
% \end{verbatim}}
% at the beginning of your figure.
% Please ensure that any point you wish to make is resolvable in a printed copy of the paper.
% Resize fonts in figures to match the font in the body text, and choose line widths that render effectively in print.
% Readers (and reviewers), even of an electronic copy, may choose to print your paper in order to read it.
% You cannot insist that they do otherwise, and therefore must not assume that they can zoom in to see tiny details on a graphic.

% When placing figures in \LaTeX, it's almost always best to use \verb+\includegraphics+, and to specify the figure width as a multiple of the line width as in the example below
% {\small\begin{verbatim}
%    \usepackage{graphicx} ...
%    \includegraphics[width=0.8\linewidth]
%                    {myfile.pdf}
% \end{verbatim}
% }


% %-------------------------------------------------------------------------
% \subsection{Color}

% Please refer to the author guidelines on the \confName\ \confYear\ web page for a discussion of the use of color in your document.

% If you use color in your plots, please keep in mind that a significant subset of reviewers and readers may have a color vision deficiency; red-green blindness is the most frequent kind.
% Hence avoid relying only on color as the discriminative feature in plots (such as red \vs green lines), but add a second discriminative feature to ease disambiguation.

% %------------------------------------------------------------------------
% \section{Final copy}

% You must include your signed IEEE copyright release form when you submit your finished paper.
% We MUST have this form before your paper can be published in the proceedings.

% Please direct any questions to the production editor in charge of these proceedings at the IEEE Computer Society Press:
% \url{https://www.computer.org/about/contact}.


%%%%%%%%% REFERENCES
{\small
\bibliographystyle{ieee_fullname}
\bibliography{egbib}
}

\end{document}
