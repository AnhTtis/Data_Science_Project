% This must be in the first 5 lines to tell arXiv to use pdfLaTeX, which is strongly recommended.
% \pdfoutput=1
% In particular, the hyperref package requires pdfLaTeX in order to break URLs across lines.

\documentclass[11pt]{article}

% Remove the "review" option to generate the final version.
\usepackage[final]{neurips_2023}
\usepackage{natbib}
% Standard package includes
\usepackage[dvipsnames]{xcolor}
\usepackage{times}
\usepackage{wrapfig}
\usepackage{latexsym}
\usepackage[colorlinks=true,linkcolor=blue,citecolor=blue]{hyperref} %https://tex.stackexchange.com/questions/50747/options-for-appearance-of-links-in-hyperref
\usepackage{amsmath}
\usepackage{graphicx}
\usepackage{booktabs}
\usepackage{multirow}
\usepackage{enumitem} % for customized listed item label
\usepackage{amssymb}
\usepackage{diagbox}
\usepackage{caption}
\usepackage{subcaption}
\usepackage{pifont}% http://ctan.org/pkg/pifont
\usepackage{xspace}
\usepackage[export]{adjustbox}
\newcommand{\cmark}{\ding{51}}%
\newcommand{\xmark}{\ding{55}}%
%\usepackage{colortbl}

% For proper rendering and hyphenation of words containing Latin characters (including in bib files)
\usepackage[T1]{fontenc}
% For Vietnamese characters
% \usepackage[T5]{fontenc}
% See https://www.latex-project.org/help/documentation/encguide.pdf for other character sets

% This assumes your files are encoded as UTF8
\usepackage[utf8]{inputenc}

% This is not strictly necessary, and may be commented out,
% but it will improve the layout of the manuscript,
% and will typically save some space.
\usepackage{microtype}
% If the title and author information does not fit in the area allocated, uncomment the following
%
%\setlength\titlebox{<dim>}
%
% and set <dim> to something 5cm or larger.
% \newcommand{\titlestr}{Paraphrasing evades methods for detecting AI-generated text, \\ but semantic retrieval is an effective defense}
\newcommand{\titlestr}{Paraphrasing evades detectors of AI-generated text, \\ but retrieval is an effective defense}

\title{\titlestr}


% The \author macro works with any number of authors. There are two
% commands used to separate the names and addresses of multiple
% authors: \And and \AND.
%
% Using \And between authors leaves it to LaTeX to determine where to
% break the lines. Using \AND forces a line break at that point. So,
% if LaTeX puts 3 of 4 authors names on the first line, and the last
% on the second line, try using \AND instead of \And before the third
% author name.
% \author{
%   \AND
%   Kalpesh Krishna\thanks{Work done as a PhD student in UMass Amherst, and partially as a student researcher in Google Research. John Wieting and Mohit Iyyer contributed equally as advisors.
% }\\
%   Google \\
%   \texttt{kalpeshk@google.com}\\
%   \And
%   Yixiao Song\\
%   UMass Amherst\\
%   \texttt{yi@umass.edu}
%   \And
%   Marzena Karpinska\\
%   UMass Amherst\\
%   \texttt{mkarpinska@cs.umass.edu}
%   \And
%   John Wieting\\
%   Google DeepMind\\
%   \texttt{jwieting@google.com}\\
%   \And
%   Mohit Iyyer\\
%   UMass Amherst\\
%   \texttt{miyyer@cs.umass.edu}\\
% }
%\renewcommand*{\thefootnote}{\fnsymbol{footnote}}
% If you would like to switch back to Arabic numbering, you can do it by

% \renewcommand*{\thefootnote}{\arabic{footnote}}

\author{Kalpesh Krishna$^{\spadesuit\heartsuit}$\thanks{Work done as a PhD student at UMass, and partially as a student researcher in Google Research.} \quad Yixiao Song$^{\spadesuit}$ \quad Marzena Karpinska$^\spadesuit$ \\  {\bf John Wieting}$^{\diamondsuit\,\dagger}$ \quad {\bf Mohit Iyyer}$^{\spadesuit}$\thanks{John Wieting and Mohit Iyyer contributed equally as advisors.} \vspace{8pt}\\
$^\spadesuit$University of Massachusetts Amherst, $^\heartsuit$Google, $^\diamondsuit$Google DeepMind \\ \texttt{\small \{kalpeshk,jwieting\}@google.com} \\ \texttt{\small yixiaosong@umass.edu} \quad \texttt{\small\{mkarpinska,miyyer\}@cs.umass.edu}
}

\newcommand{\lexical}{\textsc{lex}}
\newcommand{\syntax}{\textsc{ord}}
\newcommand{\model}{\textsc{dipper}}
\newcommand{\simmetric}{\textsc{sim}}
\newcommand{\ppl}{\textsc{ppl}}
\newcommand{\booktranslate}{\textsc{Par3}}
\newcommand{\parthree}{\textsc{Par3}}
\newcommand{\namedref}[2]{\hyperref[#2]{#1~\ref*{#2}}}
\newcommand{\spavg}{\textsc{P-SP}\xspace}

\newcommand{\sectionref}[1]{\namedref{Section}{#1}}
\newcommand{\tableref}[1]{\namedref{Table}{#1}}
\newcommand{\figureref}[1]{\namedref{Figure}{#1}}
\newcommand{\appendixref}[1]{\namedref{Appendix}{#1}}

\newcommand{\micomment}[1]{\textcolor{red}{\bf \small [ #1 --MI]}}
\newcommand{\kkcomment}[1]{\textcolor{purple}{\bf \small [ #1 --KK]}}
\newcommand{\jwcomment}[1]{\textcolor{blue}{\bf \small [ #1 --JW]}}
\newcommand{\yscomment}[1]{\textcolor{orange}{\bf \small [#1 --YS]}}
\newcommand{\mkcomment}[1]{\textcolor{gold}{\bf \small [ #1 --MK]}}

%\newcommand{\micomment}[1]{\textcolor{red}{}}
%\newcommand{\kkcomment}[1]{\textcolor{blue}{}}
%\newcommand{\jwcomment}[1]{\textcolor{purple}{}}
%\newcommand{\yscomment}[1]{\textcolor{orange}{}}
%\newcommand{\mkcomment}[1]{\textcolor{gold}{}}

\begin{document}
\maketitle



Over the past few years, there has been a significant amount of research focused on studying the ReLU activation function, with the aim of achieving neural network convergence through over-parametrization. However, recent developments in the field of Large Language Models (LLMs) have sparked interest in the use of exponential activation functions, specifically in the attention mechanism.

Mathematically, we define the neural function $F: \R^{d \times m} \times  \mathbb{R}^d \rightarrow \mathbb{R}$ using an exponential activation function. Given a set of data points with labels $\{(x_1, y_1), (x_2, y_2), \dots, (x_n, y_n)\} \subset \mathbb{R}^d \times \mathbb{R}$ where $n$ denotes the number of the data. Here $F(W(t),x)$ can be expressed as $F(W(t),x) := \sum_{r=1}^m a_r \exp(\langle w_r, x \rangle)$, where $m$ represents the number of neurons, and $w_r(t)$ are weights at time $t$. It's standard in literature that $a_r$ are the fixed weights and it's never changed during the training. We initialize the weights $W(0) \in \mathbb{R}^{d \times m}$ with random Gaussian distributions, such that $w_r(0) \sim \mathcal{N}(0, I_d)$ and initialize $a_r$ from random sign distribution for each $r \in [m]$.

Using the gradient descent algorithm, we can find a weight $W(T)$ such that $\| F(W(T), X) - y \|_2 \leq \epsilon$ holds with probability $1-\delta$, where $\epsilon \in (0,0.1)$ and $m = \Omega(n^{2+o(1)}\log(n/\delta))$. To optimize the over-parametrization bound $m$, we employ several tight analysis techniques from previous studies [Song and Yang arXiv 2019, Munteanu, Omlor, Song and Woodruff ICML 2022]. 

 

\section{Introduction}
\label{sec:introduction}
% \begin{itemize}
%     % Diffusion of FL
%     \item {\st{Diffusion of FL}}
%     % Security threats to FL
%     \item {\st{Security threats to FL with particular focus on model poisoning}}
%     % Limitations of existing countermeasures
%     \item {\st{Current countermeasures (e.g., KRUM) and their limitations}}
%     % Proposed method and its advantages
%     \item {\st{Intuitive description of the proposed method and its difference (i.e., advantages) w.r.t. state of the art}}
%     % Main contributions
%     \item {\st{Summary of the main contributions of this work}}
%     % Paper's structure and organization
%     \item {\st{Paper's structure and organization}}
% \end{itemize}

% Diffusion of FL
Recently, {\em federated learning} (FL) has emerged as the leading paradigm for training distributed, large-scale, and privacy-preserving machine learning (ML) systems~\cite{mcmahan2017googleai,mcmahan2017aistats}. 
The core idea of FL is to allow multiple edge clients to collaboratively train a shared, global model without disclosing their local private training data.
%Specifically, an FL system consists of a central server and many edge clients; 
A typical FL round involves the following steps: {\em(i)} the server randomly picks some clients and sends them the current, global model; {\em(ii)} each selected client locally trains its model with its own private data; then, it sends the resulting local model to the server;\footnote{Whenever we refer to global/local model, we mean global/local model {\em parameters}.} {\em(iii)} the server updates the global model by computing an \emph{aggregation function}, usually the average (FedAvg), on the local models received from clients.
% \begin{enumerate}
%     \item[{\em(i)}] the server sends the current, global model to the clients and appoints some of them for training;
%     \item[{\em(ii)}] each selected client locally trains its copy of the global model with its own private data; then, it sends the resulting local model back to the server;\footnote{Whenever we refer to global/local model, we mean global/local model {\em parameters}.}
%     \item[{\em(iii)}] the server updates the global model by computing an \emph{aggregation function} on the local models received from clients (by default, the average, also referred to as FedAvg~\cite{mcmahan2017aistats}).
% \end{enumerate}
This process goes on until the global model converges. %(e.g., after a certain number of rounds or other similar stopping criteria).
%\\
% The advantages of FL over the traditional, centralized learning paradigm are undoubtedly clear in terms of flexibility/scalability (clients can join/disconnect from the FL network dynamically), network communications (only model weights\footnote{We will use \textit{parameters} and \textit{weights} interchangeably.} are exchanged between clients and server), and privacy (each client's private training data is kept local at the client's end and not uploaded to the server).
\\
% Security threats to FL
%However, the growing adoption of FL also raises security concerns~\cite{costa2022covert}, particularly about its confidentiality, integrity, and availability.
Although its advantages over standard ML, FL also raises security concerns~\cite{costa2022covert}. %, particularly about its confidentiality, integrity, and availability~\cite{costa2022covert}.
% OLD, LONG VERSION
% Indeed, some work deals with privacy leakage that may expose the local data of some clients~\cite{melis2019sp}. 
% A large body of work, instead, investigates attacks that usually aim to detriment the predictive accuracy of the learned global model. For instance, \emph{data poisoning} attacks achieve this goal by letting an adversary pollute the training set of some corrupt FL clients with maliciously crafted examples~\cite{jagielski2018sp}.
% Similarly, in \emph{model poisoning} the attacker attempts to tweak the global model weights~\cite{bhagoji2019pmlr} by directly perturbing the local model's weights of some infected FL clients before these are sent to the central server for aggregation, usually via so-called Byzantine attacks. 
% It turns out that Byzantine model poisoning attacks severely impact standard FedAvg; therefore, more robust aggregation functions must be designed to make FL systems secure.
Here, we focus on \emph{untargeted model poisoning} attacks~\cite{bhagoji2019pmlr}, where an adversary attempts to tweak the global model weights %\footnote{We will use the terms \textit{parameters} and \textit{weights} interchangeably.} 
by directly perturbing the local model's parameters of some infected clients before these are sent to the central server for aggregation.
In doing so, the adversary aims to jeopardize the global model \textit{indiscriminately} at inference time.
Such model poisoning attacks severely impact standard FedAvg; therefore, more robust aggregation functions must be designed to secure FL systems.
\\
% In this paper, we focus on designing a novel robust aggregation scheme at the server's end to contrast the effect of Byzantine model poisoning attacks.
%
% Current countermeasures and their limitations
%Several countermeasures have been proposed in the literature to combat model poisoning attacks on FL systems.
% Some methods use simple statistics more robust than plain average to smooth the impact of malicious updates (e.g., Trimmed Mean and FedMedian~\cite{yin2018icml}). 
% Other defenses implement outlier detection techniques to discard malicious updates from the aggregation performed at the server's end. Those are either based on heuristics (e.g., Krum/Multi-Krum~\cite{blanchard2017nips} and Bulyan~\cite{mhamdi2018pmlr}) or data-driven approaches (e.g., K-means clustering~\cite{shen2016acm} or DnC via spectral analysis~\cite{shejwalkar2021ndss}). 
% Finally, some strategies rely on a centralized ``source of trust'' to spot potential malicious updates (e.g., FLTrust~\cite{cao2020fltrust}).
% Several countermeasures have been proposed in the literature to combat model poisoning attacks on FL systems, i.e., to discard possible malicious local updates from the aggregation performed at the server's end. 
% These techniques range from simple statistics more robust than plain average (e.g., Trimmed Mean and FedMedian~\cite{yin2018icml}) to outlier detection heuristics (e.g., Krum/Multi-Krum~\cite{blanchard2017nips} and Bulyan~\cite{mhamdi2018pmlr}) or data-driven approaches (e.g., spectral analysis via K-means clustering~\cite{shen2016acm} or spectral analysis), or methods based on ``source of trust'' (e.g., FLTrust~\cite{cao2020fltrust}).
% OLD, LONG VERSION
%Several countermeasures have been proposed in the literature to combat Byzantine model poisoning attacks on FL systems.
% Descriptive statistics
% For example, Trimmed Mean and FedMedian aggregate local model updates using more robust statistics than standard average~\cite{yin2018icml}.
%
% % Heuristics for outlier detection
% Many existing Byzantine-resilient strategies implement some outlier detection heuristics to discard the model updates sent by potentially malicious clients from the input of the aggregation function.
% One of the most popular heuristics is Krum~\cite{blanchard2017nips}.
% This strategy tries to mitigate the impact of Byzantine attacks by selecting as a global model the local model with the smallest sum of Euclidean distances to {\em all} the other local models.
% Although powerful, Krum requires the server to know (or, at least, estimate) the number of malicious FL clients upfront, which is generally impossible in a realistic attack scenario. %
% Moreover, Krum may become ineffective for complex, high-dimensional model parameter spaces due to the curse of dimensionality.
% Bulyan~\cite{mhamdi2018pmlr} tries to overcome this issue by combining Krum with a variant of Trimmed Mean.
% % Data-driven outlier detection
% Other strategies use data-driven outlier detection techniques -- e.g., via K-means clustering~\cite{shen2016acm} -- to spot potential malicious local model updates. 
% %For instance, Shen et al. propose to cluster local model updates with K-means and thus identify outliers.
%
% % Other techniques
% As far as the server is concerned, any local model received can be from a potential malicious client. 
% FLTrust~\cite{cao2020fltrust} assumes the server acts as a client, i.e., trains a local model on an additional {\em trustworthy} dataset at the server's end and compares it against all the local models from other clients. 
% This way, the server can rely on some ``source of trust'' when discarding potentially malicious clients.
%\\
% Limitations of existing Byzantine-resilient strategies
Unfortunately, existing defense mechanisms either rely on simple heuristics (e.g., Trimmed Mean and FedMedian by~\cite{yin2018icml}) or need strong and unrealistic assumptions to work effectively (e.g., foreknowledge or estimation of the number of malicious clients in the FL system, as for Krum/Multi-Krum~\cite{blanchard2017nips} and Bulyan~\cite{mhamdi2018pmlr}, which, however, cannot exceed a fixed threshold).
Furthermore, outlier detection methods using K-means clustering~\cite{shen2016acm} or spectral analysis like DnC~\cite{shejwalkar2021ndss} do not directly consider the temporal evolution of local model updates received.
Finally, strategies like FLTrust~\cite{cao2020fltrust} require the server to collect its own dataset and act as a proper client, thereby altering the standard FL protocol.
\\
% OLD, LONG VERSION
% Overall, existing Byzantine-resilient strategies are either simple heuristics (e.g., FedMedian) or, if they are more complex, they rely on strong and unrealistic assumptions to work effectively (e.g., knowing the number of malicious clients in the FL system in advance, as for Krum and alike).
% Furthermore, data-driven outlier detection methods do not consider the temporary evolution of local model updates received (e.g., K-means clustering). 
% Finally, strategies like FLTrust requires the server to collect its own dataset and act as a proper client, thereby altering the standard FL protocol.
%
% Description of the proposed method
This work introduces a novel pre-aggregation \textit{filter} robust to untargeted model poisoning attacks. Notably, this filter $(i)$ operates without requiring prior knowledge or constraints on the number of malicious clients and $(ii)$ inherently integrates temporal dependencies. 
The FL server can employ this filter as a preprocessing step before applying \textit{any} aggregation function, be it standard like FedAvg or robust like Krum or Bulyan.
Specifically, we formulate the problem of identifying corrupted updates as a multidimensional (i.e., matrix-valued) time series anomaly detection task. 
The key idea is that legitimate local updates, resulting from well-calibrated iterative procedures like stochastic gradient descent (SGD) with an appropriate learning rate, show \textit{higher predictability} compared to malicious updates. This hypothesis stems from the fact that the sequence of gradients (thus, model parameters) observed during legitimate training exhibit regular patterns, as validated in Section~\ref{subsec:intuition}. %until convergence. 
%This regularity may be more pronounced for smooth convex loss functions, but it can still be captured within an appropriate time window, even for more complex and convoluted loss surfaces. 
%We provide evidence of this claim in Appendix~B, where we show that the average mutual information (i.e., ``predictability''), calculated over pairs of legitimate model updates sent at different FL rounds, is significantly higher than the corresponding computation for a malicious client.
\\
Inspired by the matrix autoregressive (MAR) framework for multidimensional time series forecasting~\cite{chen2021je}, we propose the FLANDERS ({\em \textbf{F}ederated \textbf{L}earning meets \textbf{AN}omaly \textbf{DE}tection for a \textbf{R}obust and \textbf{S}ecure}) filter.
The main advantages of FLANDERS over existing strategies like FLDetector~\cite{zhao2020multivariate} are its resilience to large-scale attacks, where $50\%$ or more FL participants are hostile, and the capability of working under realistic non-iid scenarios.
We attribute such a capability to two key factors: $(i)$ FLANDERS works without knowing a priori the ratio of corrupted clients, and $(ii)$ it embodies temporal dependencies between intra- and inter-client updates, quickly recognizing local model drifts caused by evil players. Below, we summarize our main contributions:

\begin{itemize}
\item[{\em(i)}]
We provide empirical evidence that the sequence of models sent by legitimate clients is more predictable than those of malicious participants performing untargeted model poisoning attacks.
\\
\item[{\em(ii)}] 
We introduce FLANDERS, the first pre-aggregation filter for FL robust to untargeted model poisoning based on multidimensional time series anomaly detection.
\\
\item[{\em(iii)}] 
We integrate FLANDERS into Flower,\footnote{\scriptsize{\url{https://flower.dev/}}} a popular FL simulation framework for reproducibility.
\\
\item[{\em(iv)}] 
We show that FLANDERS improves the robustness of the existing aggregation methods under multiple settings: different datasets, client's data distribution (non-iid), models, and attack scenarios.
\\
\item[{\em(v)}] 
We publicly release all the implementation code of FLANDERS along with our experiments.\footnote{\scriptsize{\url{https://anonymous.4open.science/r/flanders_exp-7EEB}}}
\end{itemize}

% Paper's structure and organization
The remainder of the paper is structured as follows. %some related work and the current state-of-the-art solutions to security issues that FL entails. 
Section~\ref{sec:background} covers background and preliminaries. 
In Section~\ref{sec:related}, we discuss related work.
Section~\ref{sec:problem} and Section~\ref{sec:method} describe the problem formulation and the method proposed. % to tackle it. 
Section~\ref{sec:experiments} gathers experimental results. %, and Section~\ref{sec:limitations} discusses some limitations of this work.
Finally, we conclude in Section~\ref{sec:conclusion}.
 %discusses the limitations of this work and draws future research directions.
%reports conclusions and draws perspectives for future research directions.

%%%%%%% OLD %%%%%%%
%to overcome the resilience of Byzantine failures in distributed Stochastic Gradient Descent computations. 
% The strength of Krum is its time complexity, which is linear in the gradient dimension. 
% However, the robustness of the approach is guaranteed for gradient-based learning applications only when the majority of the clients are not compromised. 
% Besides, the aggregation mechanism of Krum, as well as that of similar methods, is robust from a coarse-grained perspective and does not provide solutions to errors and perturbations that may occur at inference time.
%A related approach to~\cite{blanchard2017nips} is the work of Su et al.~\cite{su2016dc}. Here, the authors propose an iterated approximate agreement to tackle a multi-layer scenario attacked by Byzantine agents. 
%However, the method works efficiently on the sole discrete context and it is inapplicable to continuous state environments.
%\gabri{Maybe, we should just talk about the main limitations of existing countermeasures without digging into their details (or, we can just mention Krum as this is the most popular one). I will move the description of all these methods to the Related Work section.}
% \section{Background on Network Calculus}
\label{sec: background}


\begin{figure*}[tbh]
\centering
\begin{subfigure}[b]{0.3\textwidth}
    \centering
    \includegraphics[width=\linewidth]{images/in-out.png}
    \caption{Arrival and departure data and their relation with delay $d(t)$ and backlog $b(t)$. For a FIFO system, the delay is the horizontal distance between $R(t)$ and $R^*(t)$ but some other multiplexing techniques may shift the data to a later priority, causing a longer delay.}
    \label{fig: data in-out}
\end{subfigure}
\hfill
\begin{subfigure}[b]{0.35\textwidth}
    \centering
    \includegraphics[width=\linewidth]{images/arrival-service.png}
    \caption{Characteristics of an arrival curve and a service curve. From any point of observation, the arriving data never exceeds its arrival curve; the departure data is also never less than the service curve with respect to the data arrival.}
    \label{fig: arrival-service curves}
\end{subfigure}
\hfill
\begin{subfigure}[b]{0.33\textwidth}
    \centering
    \includegraphics[width=\linewidth]{images/bound.png}
    \caption{Delay and backlog bounds of a system. Backlog is the maximum vertical distance between $\alpha(t)$ and $\beta(t)$; FIFO delay is their maximum horizontal distance; but for arbitrary multiplexing, the delay guarantee is when the system clears its buffer, thus it's the intersection of $\alpha(t)$ and $\beta(t)$.}
    \label{fig: system bounds}
\end{subfigure}
\caption{Network calculus framework. We let $R(t)$ and $R^*(t)$ be the arrival and departure data flow of a system; $\alpha(t)$ be the piecewise linear concave arrival curve and $\beta(t)$ be the piecewise linear convex service curve of a system.}
% \hossein{Better to show piece-wise linear concave arrival curve and piece-wise linear convex service curve instead of token-bucket and rate-latency.}}
\end{figure*}

We recall some of the network calculus essentials for a better understanding of the framework used in Saihu. In the following context, we use the following notation: $\mbb{R}^+$ is the set of non-negative real numbers; $[x]_+$ denotes $\max(0, x)$

The data flow is by convention modeled as a left-continuous wide-sense increasing function $R(t): \mbb{R}^+ \mapsto \mbb{R}^+$ with respect to time $t$~\cite{ncbook2001leboudec}. 

A system $\mcal{S}$ receives arrival data described as a cumulative function $R(t)$ and delivers departure data as another cumulative function $R^*(t)$. Figure~\ref{fig: data in-out} illustrates such a system $\mcal{S}$. The benefit of representing a system like this is that we can observe system backlog and delay with such a model. 

\begin{definition}[Backlog and Delay~\cite{ncbook2001leboudec}]
    The backlog of a system at time~$t$ is
    \begin{equation}
        b(t) = R(t) - R^*(t)
    \end{equation}
    
    The virtual delay of a FIFO system at time $t$ is
    \begin{equation}
        d_{FIFO}(t) = \inf \lbp \tau \geq 0 : R(t) \leq R^*(t+\tau) \rbp
    \end{equation}
\end{definition}



The backlog of a system can be viewed as the vertical distance between $R$ and $R^*$. The FIFO (\textit{First-in First-out}) delay is the horizontal distance between $R$ and $R^*$. One may obtain other delay values if the multiplexing technique is not FIFO.

% \begin{figure}
%     \centering
%     \includegraphics[width=0.9\linewidth]{images/in-out.png}
%     \caption{In/out data flow; delay and backlog}
%     \label{fig: data in-out}
% \end{figure}

Since we are interested in the system guarantee instead of a single instance of data flow, we would like to have general bounds to the arrival and departure data flows. Therefore, we define \textit{arrival curve} and \textit{service curve} as the bounds of arrival and departure data flows.

\begin{definition}[Arrival Curve~\cite{ncbook2001leboudec}]
    Given a wide-sense increasing function $\alpha: \mbb{R}^+ \mapsto \mbb{R}^+$, we say that a flow $R(t)$ is $\alpha$-constrained if and only if for all $s \leq t$:
    \begin{equation}
        R(t) - R(s) \leq \alpha(t-s)
    \end{equation}
    We say $R(t)$ has $\alpha$ as an arrival curve.
\end{definition}

\begin{definition}[Service Curve~\cite{ncbook2001leboudec}]
    Given a wide-sense increasing function $\beta: \mbb{R}^+ \mapsto \mbb{R}^+$ and $\beta(0) = 0$. A system $\mcal{S}$ having $R(t)$ and $R^*(t)$ as its arrival and departure flows. We say $\mcal{S}$ offers a service curve $\beta$ if and only if
    \begin{equation}
        R^*(t) \geq (R \otimes \beta)(t) =: \inf_{s \leq t} \lbp R(s) + \beta(t-s) \rbp
    \end{equation}
    where $\otimes$ denotes the min-plus convolution
\end{definition}

Figure~\ref{fig: arrival-service curves} illustrates the arrival and service curves. Any segment of arrival flow $R(t)$ is constrained by arrival curve $\alpha$ and the output curve $R^*(t)$ is always no less than the curve $R\otimes\beta$. As a result, an arrival curve upper bounds the incoming traffic, and a service curve lower bounds the outgoing traffic.

% \begin{figure}
%     \centering
%     \includegraphics[width=\linewidth]{images/arrival-service.png}
%     \caption{Arrival/Service curve}
%     \label{fig: arrival-service curves}
% \end{figure}

We consider 2 special types of curves throughout this paper, \textit{token-bucket} (or sometimes called \textit{leaky-bucket}) curve and \textit{rate-Latency} curve.

\begin{definition}[Token-bucket and Rate-latency~\cite{ncbook2001leboudec}]
    A token-bucket curve $\gamma_{r,b}$ with arrival rate $r$ and burst $b$ is defined as
    \begin{equation}
        \gamma_{r,b}(t) = b + rt
    \end{equation}

    A rate-latency curve $\beta_{R,T}$ with service rate $R$ and latency $T$ is defined as
    \begin{equation}
        \beta_{R,T}(t) = R \lb t - T \rb_+
    \end{equation}
\end{definition}

A token-bucket curve is determined by a burst $b$ and an arrival rate~$r$. Burst represents the maximum possible data volume that can arrive simultaneously, and arrival rate represents the maximum long-term data rate~\cite{bouillard2022tradeoff}.
A rate-latency curve is determined by a latency~$T$ and a service rate~$R$. Latency represents the time a server needs before starting to process the incoming data, and service rate represents the minimum rate to process data after the initial latency.

With the help of arrival and service curves, we can derive delay and backlog bounds for a system $\mcal{S}$ illustrated in Figure~\ref{fig: system bounds}. Suppose a system $\mcal{S}$ has arrival curve $\alpha$ and service curve~$\beta$, its worst-case backlog $b^*$ is the maximum vertical distance between~$\alpha$ and~$\beta$. Similarly, depending on the multiplexing technique applied to the system, its worst-case delay bound $d^*$ is the maximum horizontal distance between $\alpha$ and $\beta$ if $\mcal{S}$ is a FIFO system. If we don't have any information about its multiplexing technique, referred to as arbitrary multiplexing, the best we can say is that when $\alpha$ and $\beta$ intersect each other, where all data has been delivered out of the system. Consequently, the worst-case delay bound for arbitrary multiplexing is the time required for $\mcal{S}$ to clear its buffer.

% \begin{figure}
%     \centering
%     \includegraphics[width=\linewidth]{images/bound.png}
%     \caption{System delay/backlog bounds}
%     \label{fig: system bounds}
% \end{figure}

While a service curve captures the slowest possible output speed of a system, a link's transmission capacity limits the speed as well. Hence, we model this phenomenon using a \textit{greedy shaper} with a sub-additive function $\sigma: \mbb{R}^+ \mapsto \mbb{R}^+$ concatenated with a server. We consider a concatenation as shown in Figure \ref{fig: system}. By convention we assume $\sigma(0) = 0$ and $\beta(t) \leq \sigma(t), \forall t \in \mbb{R}^+$, meaning that the buffer is cleared at the beginning and the service never exceed its physical limitation. With the above definition, such greedy shaper conserves the service provided by the system due to theorem \ref{thm: shaping}.

\begin{figure}[thb]
    \centering
    \includegraphics[width=0.7\linewidth]{images/system.png}
    \caption{Shaping of departure data. A flow that has an arrival curve $\alpha$ feeds into a server with an arrival data flow $R(t)$. The server having service curve $\beta$ takes $R(t)$ and gives a departure data flow $R^*(t)$ to a shaper with shaping function $\sigma$. The shaper takes $R^*(t)$ and shape the data flow as another departure $D(t)$.}
    \label{fig: system}
\end{figure}


\begin{theorem}[Shaping conserves service \cite{ncbook2001leboudec}]
\label{thm: shaping}
Following the system shown in Figure \ref{fig: system}, we have
\begin{equation}
     D = R^* \otimes \sigma \geq \lp R \otimes \beta \rp \otimes \sigma = R \otimes \lp \beta \otimes \sigma \rp = R \otimes \beta
\end{equation}
\end{theorem}

In the following context, we model the shaping function $\sigma$ as a token-bucket curve $\gamma_{C,L}$ with transmission capacity $C$ and the packet size $L$ to capture the link capacity and packetization~\cite{bouillard2022tradeoff}.

\section{Background on detectors of AI-generated text}
\label{sec:detection-background-shorten}
\label{sec:sadasivan-shorten}

In this section, we provide a brief overview of existing algorithms for detecting AI-generated text detection (see \appendixref{sec:detection-background} for a detailed version). We also contrast our work to~\citet{sadasivan2023aigenerated}, a concurrent effort which notes the efficacy of paraphrasing attacks but does not consider a retrieval-based defense in its pessimistic conclusion about the fate of AI-generated text detection.

%\footnote{A comprehensive overview can be found in \appendixref{sec:detection-background}.}

%Given the recent success of large language models (LLMs) like ChatGPT~\citep{schulman2022chatgpt} in producing long-form human-like text, there's an increasing fear that LLMs will be used for malicious purposes like fake news generation and cheating in academic writing assignments. This has led to the development of several mitigation techniques, which detect if a long-form text was human or machine generated. Methods to detect machine-generated text roughly fall under two categories: watermarking \& outlier detection.\\

% \subsection{Watermarking language model outputs}
% \label{sec:watermark}

%\textbf{A watermarking algorithm} \textit{watermark}s model-generated outputs by modifying the text in a way that can be detected by a statistical algorithm but being imperceptible to human readers. Effective watermarks should have little effect on the quality of generated text while being difficult to remove. Prior work attempted to watermark natural language using syntax tree manipulations~\citep{topkara2005natural, meral2009natural}. Most recently,~\citet{kirchenbauer2023watermark} propose a simple algorithm that only requires access to the LLM's logits at each time step to add watermarks. The watermark can then be verified with only blackbox access to the LM and knowledge of a specific hash function. 

A \textbf{watermark} is a modification to the generated text that can be detected post-hoc by an algorithm while remaining imperceptible to human readers. Effective watermarks are difficult to remove and have little effect on the quality of generated text. Prior work has watermarked natural language using syntax tree manipulations~\citep{topkara2005natural, meral2009natural}, and this area has received renewed interest with the advent of LLMs~\citep{abdelnabi2021adversarial, grinbaum2022ethical}. Most recently,~\citet{kirchenbauer2023watermark} proposed a simple algorithm that watermarks LLMs by slightly perturbing its probability distribution while generating text. 
% This watermark can be detected without white-box access to the language model.

% \subsection{Statistical outlier detection methods}
% \label{sec:outlier-detection}

 \textbf{Statistical outlier detection methods} detect AI-generated text based on its artifacts~\citep{see-etal-2019-massively, holtzman2020curious} instead of modifying the generative algorithm. Early methods detect statistical irregularities in entropy~\citep{lavergne2008detecting} and perplexity~\citep{beresneva2016computer}, while \citet{gehrmann-etal-2019-gltr} introduced the GLTR visualizer to assist humans in detecting AI-generated text. The release of ChatGPT prompted the development of two new tools: closed-source GPTZero~\citep{GPTZero} and open-source DetectGPT~\citep{mitchell2023detectgpt}. The latter uses the observation that AI-generated text tends to have  significantly higher LLM likelihood than meaningful perturbations of it.
 
 %Thus, it constructs multiple perturbations of the model generated text using a mask-and-fill strategy, and compares the log probability of the perturbations with the unperturbed generation.

\textbf{Classifier methods} train models to distinguish human-written text from AI-generated text. Early efforts used classifiers to detect fake reviews~\citep{hovy-2016-enemy} and news~\citep{zellers2019defending}, while others examined classifier performance across domains~\citep{bakhtin2019real} and decoding strategies~\citep{ ippolito-etal-2020-automatic}. Most recently, OpenAI fine-tuned a GPT model to perform this task and released it as a web interface~\citep{AITextClassifier}. Their model uses generations from 34 LLMs, with the human-written text from Wikipedia, WebText, and their internal human demonstration data.

\noindent \textbf{Comparison to Sadasivan et al. (2023)}: In recent concurrent work, ~\citet{sadasivan2023aigenerated} also demonstrate the utility of paraphrasing attacks against AI-generated text detectors. 
% While they use off-the-shelf sentence-level paraphrasers, our \model\ model possesses advanced discourse-level rewriting capabilities as well as fine-grained diversity control, which is much more suited to long-form text generated by LLMs.
Our experiments encompass more tasks, detection algorithms, and larger LMs like GPT3.5. Additionally, we propose a discourse-level paraphrase model (\model) that is much more suited to long-form text than the off-the-shelf sentence-level paraphrasers used in their paper. More importantly, our retrieval-based defense \emph{directly contradicts} the ``impossibility result'' of~\citet{sadasivan2023aigenerated} and its associated proof, which states that an optimal detector will perform at random as the quality of LLM-generated text approaches that of human-written text. The quality of generated text is irrelevant to our detector's accuracy because it relies only on a corpus search, and thus the proof is inapplicable. Other concurrent work~\citep{chakraborty2023possibilities} has also shown the proof's invalidity in practical settings.
\section{Proposed Framework: {\ourmodel}}
\label{model}


In this section, we introduce a novel self-supervised co-training framework {\ourmodel}.
The proposed framework is illustrated in Figure~\ref{fig:intro_model} and works in three phases.
Phase one automatically generates two sets of pseudo labels.
We use a combination of off-the-shelf pre-trained POS and NER taggers, knowledge graph, and GPT-2 scorer for generating the first set of pseudo labels automatically without any hand-crafted rules for matching the slot values.
The other set of pseudo labels is acquired through a zero-shot slot filling model~\cite{liu2020coach}, trained on the out-of-domain dataset.
It is critical to emphasize that both sets of labels are noisy and incomplete which poses serious challenges to training effective models for the task of open-domain slot filling.
Phase two fine-tunes the pre-trained BERT to the slot filling task that effectively transfers the knowledge from the pre-trained language model~(LM) to overcome the issue of label incompleteness to some extent. 
Further, we employ the early stopping technique to minimize the noise in the labels.
The output of this phase is two BERT models that can generate soft labels for self-supervision during co-training in phase three.
Phase three leverages the fine-tuned models and further trains them in an iterative fashion.
Specifically, the proposed peer training approach facilitates high-confidence soft label selection for the other peer to perform training. This phase progressively reduces the noise in the labels and enables effective model fitting. 



\subsection{Phase One: Automatic Label Generation}
To acquire the first set of labels, we perform the following steps.
First of all, off-the-shelf trained POS and NER taggers are used to predict initial estimates of the slot values irrespective of the slot types. Then, the type information of the slot values is queried from the KG and the slot value is tagged for the most appropriate slot in the target domain.
This approach, however, produces low recall. 
To expand the candidate slot values, we generate n-grams of the natural language text and employ a partial matching scheme to query the KG for type information (e.g., \myspecial{Jason} \myspecial{Aldean} = \myspecial{American} \myspecial{singer}) of the n-grams if the entry exists.
This process generates multiple overlapping hypotheses about the slot values.
We replace a span of text that corresponds to a slot value by its type information and a GPT-2 based scorer (see Section~\ref{sec:nlpmodels}) is used to select the best candidate based on the fluency of the text.
Naturally, if a token (or span of tokens) is replaced by its type, the sentence should score higher as compared to the case where an inappropriate substitution is performed. 
We select the best hypothesis if the score is greater than the threshold.
Intuitively, the candidate selection threshold can automatically be searched based on a small validation set from the target domain, making the label generation process fully automatic. 
The other set of noisy labels is acquired by the zero-shot slot filling model~\cite{liu2020coach} that has been trained using an out-of-domain dataset. It is important to highlight that the zero-shot slot filling model does not require any labeled in-domain training example. 
To summarize the automatic label generation phase, both sets of labels are acquired in a fully automatic fashion without any hand-crafting.


In contrast to previous work in weak supervision~\cite{ren2015clustype,he2017autoentity,fries2017swellshark,giannakopoulos2017unsupervised} that obtains a single set of noisy labels and then propose techniques to overcome the challenge of fitting an effective model to the noisy labels, we acquire two sets of complementary labels.
The choice of these two sets of labels is guided by the intuition that they should be complementary and the models trained on these sets of labels should be able to share complementary information with the other to improve the performance in the later phases of the framework.
Essentially, the first set of labels carries information from external knowledge sources, whereas the labels generated through the pre-trained zero-shot slot filling model capture how the slot values are mentioned in other domains.
%
To further elaborate on the motivation and our process for the first set of labels (i.e., labels using KG and other NLP models), the pre-trained LMs have been shown to have a great deal of knowledge~\cite{petroni2019language}, thus should be capable of generating automatic labels with no need of external KG. 
To the best of our knowledge, there exists no work that shows that accurate token-level automatic labeling (e.g., slot filling task) is possible with pre-trained LMs. 
Moreover, such approaches would require heavy prompting in each new target domain, whereas our label generation process is fully automatic and only relies on the readily-available pre-trained NLP models and external KG.

\subsection{Phase Two: LM-assisted Weak Supervision}
Since we do not have access to dataset $\{(\mathbf{X}_n,\mathbf{Y}_n)\}_{n=1}^N$ with true ground-truth labels.
We use pseudo labels generated in phase one, $\{(\mathbf{X}_n,\mathbf{D}_n)\}_{n=1}^N$, to learn 
$f_{m,c}(\cdot; \cdot)$ that outputs the probability of the $m$-th token to take on class $c$. 
We learn $f_{m,c}(\cdot; \cdot)$ by minimizing the following loss over the noisy dataset $\{(\mathbf{X}_n,\mathbf{D}_n)\}_{n=1}^N$: 
$$
\hat\theta = \argmin_{\theta}\frac{1}{N}\sum_{n=1}^{N} \ell(\mathbf{D}_n, f(\mathbf{X}_{n}; \theta)),
\label{eq:stage1}
$$
where $\ell(\mathbf{D}_n, f(\mathbf{X}_{n}; \theta)) = \frac{1}{M} \sum_{m=1}^{M} -\log{f_{m,d_{n, m}}(\mathbf{X}_{n}; \theta)}$. 
We employ the pre-trained multilingual BERT with token-level classification head that uses Adam optimizer \cite{kingma2014adam,Liu2019} with early stopping and multiple random initializations. 


Since slot filling task is similar to the MLM training objective of the BERT, we employ pre-trained BERT as the backbone model.
That is, MLM's goal is to predict the masked tokens using bidirectional contexts. Similarly, slot filling tries to predict the label for a token leveraging both left and right contexts simultaneously, which makes the pre-trained BERT an ideal model of choice that greatly facilitates minimizing incomplete labels.
It is important to highlight that our automatically generated labels are not only incomplete but also potentially wrong.
The training strategies employed in this phase minimize the noise in the label to some extent. 
Specifically, early stopping can provide a strong regularization and would not let the model overfit to the noisy labels, especially wrong labels. 
Moreover, early stopping does not let the model forget the knowledge in the pre-trained model.
Similarly, multiple random initializations enforce robustness. 
Since the model is fine-tuned on the noisy labels, averaging the predictions of multiple models for each token ensures that wrong labels end up with low probabilities and true labels consistently achieve high probabilities.
Using the above-mentioned strategies, we train two slot filling models, which we call the peers. The peer one is trained on the first set of pseudo labels that were generated using POS and NER taggers, KG, and the GPT-2 scorer in phase one. Similarly, peer two is trained using the predictions of the zero-shot slot filling model~\cite{liu2020coach}.
Both models have the same architecture and follow the same training procedures.

\begin{table*}[t!]
\centering
\caption{Dataset statistics.}
\vspace{-7pt}
\label{tab:dataset}
\begin{tabular}{lccccc}
\toprule
\textbf{Dataset}  & \textbf{Dataset Size} & \textbf{Vocab. Size} & \textbf{Avg. Length} & \textbf{\# of Domains} & \textbf{\# of Slots} \\ \hline
\textbf{SGD}      & 188K                  & 33.6K                & 13.8                 & 20                     & 240                  \\
\textbf{MultiWoZ} & 67.4K                 & 10.5K                & 13.3                 & 8                      & 61 \\
\bottomrule
\end{tabular}
\vspace{-7pt}
\end{table*}

\subsection{Phase Three: Self-supervised Co-training}
We introduce an iterative peer training algorithm where both peers generate high-confidence soft labels for training the other peer in the next iteration. 
Theoretically, these peers can be anything, but in this work, 
we explore two of the most promising directions that have shown the promise to minimize the need for manual labeling for the task: zero-shot learning and distant supervision.
This phase uses a self-supervised co-training scheme to exploit the patterns of slot values from other domains through the labels generated by the zero-shot filling model (i.e., peer two)~\cite{liu2020coach} as well as utilize the knowledge in external KGs and pre-trained models via labels provided by the peer one.
Specifically, we initialize the peers trained in phase two and use their pseudo labels to kick-start training in this phase.
Specifically, peer one $f_{m,c}(\cdot; \theta_{\textrm{p1}})$ would generate labels $\{\tilde{\mathbf{Y}}^{(t)}_n = [\tilde{y}_{n,1}^{(t)}, ..., \tilde{y}_{n,m}^{(t)}]\}_{n=1}^{N}$ for peer two $f_{m,c}(\cdot; \theta_{\textrm{p2}})$ at the $t$-th iteration by:
$$
\tilde{y}_{n,m}^{(t)} = \argmax_{c}{f_{m,c}(\mathbf{X}_n; \theta_{\textrm{p1}}^{(t)})}. 
\label{eq:pseudo}
$$

Based on these labels, the peer two can be fine-tuned by: 
$$
\hat\theta_{\textrm{p2}}^{(t+1)} = \argmin_{\theta}\frac{1}{N}\sum_{n=1}^N \ell(\tilde{\mathbf{Y}}_n^{(t)}, f(\mathbf{X}_{n}; \theta)).
\label{eq:self_train1}
$$

Similarly, peer two $f_{m,c}(\cdot; \theta_{\textrm{p2}})$ would generate pseudo labels for peer one $f_{m,c}(\cdot; \theta_{\textrm{p1}})$ that are used to fine-tune peer one. 
We also notice that it is beneficial to stop early during this phase as well, to improve the model fitting and gradually reduce the noise associated with the automatically generated labels.
Since pseudo labels are refined gradually in an iterative way, both peers can benefit from the knowledge contained within the labels of the other while avoiding overfitting.
Furthermore, as an alternative to pseudo labels, we also generate soft labels that are used for confidence re-weighting. 
The high-confidence soft label selection strategy enables better model fitting and efficient learning via better quality of the automatic labels.
Specifically, for the given $m$-th token in the $n$-th training example, the probability for all classes $C$ is $[f_{m,1}(\mathbf{X}_n;\theta),...,f_{m,C}(\mathbf{X}_n;\theta)]$. 
Following ~\cite{xie2016unsupervised}, at $t$-th iteration, peer one generates soft labels, $\{\mathbf{S}_n^{(t)} = [\mathbf{s}_{n,m}^{(t)}]_{m=1}^M \}_{n=1}^N$, as given below:
$$
\mathbf{s}_{n,m}^{(t)} = [s_{n,m,c}^{(t)}]_{c=1}^{C} = \Bigg[  \frac{f_{m,c}^2(\mathbf{X}_n;\theta_{\textrm{peer1}}^{(t)})/p_{c}}{\sum_{c'=1}^C f_{m,c'}^2(\mathbf{X}_n;\theta_{\textrm{peer1}}^{(t)})/p_{c'}}\Bigg]_{c=1}^{C}
\label{eq:soft}
$$ 
where $p_{c} = \sum_{n=1}^N \sum_{m=1}^M f_{m,c}(\mathbf{X}_n;\theta_{\textrm{p1}}^{(t)})$ computes the frequency of the tokens for the $c$-th class. 
Then, peer two $f(\cdot; \theta_{\textrm{p2}}^{(t+1)})$ is fine-tuned by:
$$
\theta_{\textrm{p2}}^{(t+1)} = \argmin_{\theta} \frac{1}{N} \sum_{n=1}^{N} \ell_{\rm KL}(\mathbf{S}_n^{(t)}, f(\mathbf{X}_{n}; \theta)),
$$
where $\ell_{\rm KL}(\cdot,\cdot)$ is the KL-divergence-based loss:
$$
\ell_{\rm KL}(\mathbf{S}_n^{(t)}, f(\mathbf{X}_{n}; \theta))=\frac{1}{M}\sum_{m=1}^M\sum_{c=1}^C - s_{n,m,c}^{(t)} \log f_{m,c}(\mathbf{X}_{n}; \theta).
\label{eq:klloss}
$$

Moreover, we also investigate selecting tokens that have high confidence. 
For instance, we pick high-confidence tokens from the $m$-th input example at the $t$-th iteration by  
$
H^{(t)}_n = \{m : \max_{c} s_{n,m,c}^{(t)} > \epsilon \},
$
where $\epsilon\in [0,1]$ is a threshold that can be searched based on a small validation set. 
Then, peer two $f(\cdot; \theta_{\textrm{p2}}^{(t+1)})$ is fine-tuned by:
$$
\theta_{\textrm{p2}}^{(t+1)} %&= \argmin_{\theta} \frac{1}{N} \sum_{n=1}^{N} \ell_{\rm S-KL}(\bS_n^{(t)}, f(\bX_{n}; \theta)) \\
= \argmin_{\theta} \frac{1}{N|H^{(t)}_n|}\sum_{n=1}^{N} \sum_{m\in H^{(t)}_n}\sum_{c=1}^C - s_{n,m,c}^{(t)} \log f_{m,c}(\mathbf{X}_{n}; \theta).
$$

This phase improves the robustness to effectively fit the model for tokens with high confidence. 
Both peers keep sharing information and their confidence by producing soft labels for their counterparts until they approximate to the true labels while employing early stopping and scheduled learning rates.
It is important to remind that phase three is the most important phase that progressively reduces noise from the labels to a great extent and enables superior performance for the task of open-domain slot filling.
\section{Experiments attacking detection algorithms with \model}
\label{sec:attacks}

%experiments attacking existing AI-generated text detectors with \model. We  first detail our

In this section, we describe our experimental setup in \sectionref{sec:eval-metrics-attacks}-\ref{sec:expt-setup} and present our results in \sectionref{sec:attack-expts}. Overall, we find that \model\ evades all detectors across three LLMs (including GPT3.5). %and two tasks (open-ended generation and long-form QA).


\begin{table}[t!]
\caption{Performance of detection algorithms (at 1\% FPR) before and after \model\ paraphrasing on \textbf{open-ended generation} using Wikipedia prompts (300 generated tokens). As the diversity (L,O) increases, detection rates decrease across algorithms, with nearly perfect semantic similarity (Sim). *GPT3.5 DetectGPT scores computed using 200 samples at 20\% FPR as it scores 0\% at a 1\% FPR.}
\label{tab:watermark-attacks}
\vspace{\baselineskip}

\small
\centering
\begin{tabular}{@{}lrrrrrr@{}} 
 \toprule
  %\multicolumn{7}{c}{\textbf{Paraphrase attacks on open-ended generation} (Wikipedia prompts, 300 generated tokens)}\vspace{0.1in} \\ 
  Metric\hspace{0.09in} $\rightarrow$ & Sim $\uparrow$ & \multicolumn{5}{c}{Detection Accuracy $\downarrow$} \\
  \cmidrule{3-7}
 Detector $\rightarrow$ & & Watermarks & DetectGPT  & OpenAI & GPTZero & RankGen  \\
 %& & \citeyearpar{kirchenbauer2023watermark} & \citeyearpar{mitchell2023detectgpt}  & \citeyearpar{AITextClassifier} & \citeyearpar{GPTZero} & \citeyearpar{krishna-etal-2022-rankgen}\\

% Metric $\rightarrow$ & Sim $\uparrow$ & Acc $\downarrow$ & S\&A $\uparrow$ & Acc $\downarrow$ & S\&A $\uparrow$ & Acc $\downarrow$ & S\&A $\uparrow$ & Acc $\downarrow$ & S\&A $\uparrow$ & Acc $\downarrow$ & S\&A $\uparrow$ \\
 
 \midrule
 GPT2-1.5B  & - & 100.0 & 70.3\phantom{*} & 21.6 & 13.9  & \textbf{13.5} \\
 % sim scores = using detectgpt sim, since it's the worst SIM and good to have a conservative estimate here
 + \model\ 20L  & 99.2 & 97.1 & 28.7\phantom{*} &  19.2  & 9.1 & 15.8 \\
 + \model\ 40L & 98.4 & 85.8 & 15.4\phantom{*} & 17.8 & 7.3 & 18.0 \\
 + \model\ 60L & 96.9 & 68.9 & 8.7\phantom{*} & \textbf{13.3} & 7.1 & 19.8  \\
 + \model\ 60L, 60O & 94.3 & \textbf{57.2} & \textbf{4.6}\phantom{*} & 14.8 & \textbf{1.2} & 28.5 \\
 \midrule
 OPT-13B & - &  99.9 & 14.3\phantom{*} & 11.3 & 8.7 & \textbf{3.2}  \\
 + \model\ 20L & 99.1 & 96.2 & 3.3\phantom{*} & 11.8  & 5.4 & 5.2 \\
 + \model\ 40L & 98.6  & 84.8 &  1.2\phantom{*}  & 11.6 & 3.8 & 6.6  \\
 + \model\ 60L & 97.1 & 63.7 & 0.8\phantom{*} & \textbf{9.1} & 6.3 & 9.3   \\
 % sim = 96.9
 + \model\ 60L, 60O & 94.6 & \textbf{52.8}  & \textbf{0.3}\phantom{*} & 10.0 & \textbf{1.0} & 13.5\\
 \midrule 
 GPT-3.5-175B, davinci-003  & - & - & 26.5* & 30.0  & 7.1  & \textbf{1.2} \\
 + \model\ 20L & 97.6 & -& 12.5*  & 20.6  & 4.3 & 1.7  \\
 + \model\ 40L & 96.7 & - & 8.0*  & 22.4  & 4.8  &  2.0 \\
 + \model\ 60L & 94.2 & - & 7.0*  & \textbf{15.6} & 6.1  & 3.9  \\
 + \model\ 60L, 60O & 88.4 & - & \textbf{4.5}*  & \textbf{15.6} & \textbf{1.8} & 7.3\\
 \midrule
 % 1.7
 Human Text & - & 1.0 &  1.0\phantom{*} & 1.0 & 1.0 & 1.0 \\
\bottomrule
\end{tabular}
\vspace{-0.1in}
\end{table}

\subsection{Evaluation metrics}
\label{sec:eval-metrics-attacks}

\textbf{Detection accuracy}: Our first metric measures how often the input text is correctly detected as AI-generated. Since detection rates are heavily dependent on the chosen detection threshold, the AUC-ROC metric is commonly used to measure detector performance~\citep{mitchell2023detectgpt}, which considers the range of all possible thresholds. However, in this application, it is critical that the \emph{false positive rate} (FPR) is low; in other words, human-written text must almost never be classified as machine-generated~\citep{AITextClassifier, kirchenbauer2023watermark}. Hence, we fix the FPR to 1\% for all detection algorithms (although even 1\% is likely too high in practice), and adjust the detection threshold accordingly while reporting detection accuracies. Additionally, we also plot ROC curves focusing on the 0-1\% FPR region. Overall, we expect detection rates to plummet on paraphrased text.
%at a constant FPR of 1\%,


\textbf{Semantic similarity (Sim)}: Detection accuracy is an insufficient evaluation of our attack's success. We also need to measure whether the original and paraphrased generations share approximately the same semantics. We measure semantic similarity using the state-of-the-art semantic similarity model \spavg from~\citet{wieting-etal-2022-paraphrastic}, an embedding averaging model trained on a large corpus of filtered paraphrase data~\citep{wieting-gimpel-2018-paranmt}.  \spavg is a well-calibrated metric that performs well on semantic calibration tests as well as plagiarism detection in STS benchmarks~\citep{agirre-etal-2016-semeval}.  \spavg is also robust against topically similar non-paraphrases. We found that \spavg\ it scores just 0.09 on random pairs of paragraphs from the same book (topically similar paragraphs but not paraphrases) in the \booktranslate\  dataset~\citep{thai2022booktranslate}. In contrast, the average \spavg score of actual human paraphrase pairs in \booktranslate\ is 0.76. We consider semantics to be approximately preserved if the \spavg score is greater than this average human paraphrase score of 0.76.

Besides semantic similarity, we conduct several automatic evaluations, ablation studies, and human evaluations of intrinsic paraphrase quality in \appendixref{sec:intrinsic-evaluation}.

\subsection{Models, datasets \& detection algorithms}
\label{sec:expt-setup}

\textbf{Base language models}: We conduct attacks on three language models of varying sizes that belong to different model families. We consider the GPT2-XL model (1.5B parameters)~\citep{radford2019language}, the OPT-13B model~\citep{zhang2022opt}, and the \texttt{text-davinci-003} variant from the GPT-3.5 family~\citep{brown2020language}, which has 175B parameters and has additionally been instruction tuned using reinforcement learning from human feedback~\citep{ouyang2022training}. For all LMs, we sample generations that are 300 tokens long before passing them through \model\ for the attack experiments.\footnote{For GPT2-XL and OPT-13B, we generate text using nucleus sampling~\citep{holtzman2020curious} with $p=0.9$. For \texttt{davinci-003}, we use the default hyperparameters on the API Playground (temperature = 0.7).}


\textbf{Natural language generation tasks}: We experiment with two long-form text generation tasks, since most malicious applications (e.g., fake article creation) are associated with long-form outputs. First, we consider \emph{open-ended generation}, where an LM generates a continuation to a two-sentence prompt. To obtain our prompts, we sample 3K contiguous two-sentence chunks from the validation split of WikiText-103~\citep{meritypointer} and use the next 300 tokens as the ``human-written'' continuation. Second, we evaluate \emph{long-form question answering}~\citep{fan2019eli5}, in which an LM answers a question with a 300-word answer (dataset details in \appendixref{appendix:attack-details}). For our main results, the human reference answers or continuations are only used to adjust detection thresholds of studied methods to maintain a 1\% FPR.\footnote{Alternatively, if a random half of the human-written data was used for threshold adjustment, we find the other half has a FPR of 0.8\%-1.2\% across random splits, and this deviation will reduce with a bigger dataset.} Note that we are not removing human-written text from our test set. Our metric is equivalent to having a test set with a 50-50 split between machine/human-written text for the same prompts, and observing the FPR$=$1\% point in the ROC curve (also provided in \appendixref{appendix:more-roc-curves}).

%\footnote{Evaluating how well the studied LMs perform on these two tasks is a challenging problem in its own right that could make additional use of the human references, but this is irrelevant to our paper.}

\begin{wrapfigure}{r}{0.5\textwidth}
\caption{Detector performance (at 1\% FPR) on \textbf{long-form QA} before/after paraphrasing. As diversity (L,O) increases, detection rates decrease with very high semantic preservation (Sim). WM: Watermark, D.GPT: DetectGPT, O.AI: OpenAI. *GPT3.5 D.GPT uses 100 samples at 20\% FPR to show attack success, as it scores 0\% at 1\% FPR.}
\label{tab:attacks-lfqa}
\vspace{0.05in}

\small
\centering
\begin{tabular}{ lrrrr } 
 \toprule
 % \multicolumn{5}{c}{\textbf{Long-form Question Answering}, 300 generated words}\vspace{0.1in} \\ 
 Metric $\rightarrow$ & Sim $\uparrow$ & \multicolumn{3}{c}{Detection Accuracy $\downarrow$} \\
 \cmidrule{3-5}
& & W.M. & D.GPT  & O.AI \\
 \midrule
% \multicolumn{4}{l}{\emph{Long-form QA (300 generated tokens)}}\vspace{0.15cm} \\
 GPT2-1.5B  & - & 100.0  & 74.9\phantom{*} &  59.2 \\
 + \model\ 20L & 99.5 & 98.9 & 45.7\phantom{*} & 35.3   \\
 + \model\ 40L  & 99.0 & 90.7 & 28.0\phantom{*} & 34.4 \\
 + \model\ 60L & 97.5 & 71.1 & 15.8\phantom{*} & \textbf{31.3} \\
 % sim scores = , , , 98.6
 ~~~+ 60L, 60O & 96.2 & \textbf{55.8} & \textbf{7.6}\phantom{*} & 32.7 \\
 \midrule
 OPT-13B & - & 100.0 & 29.8\phantom{*} & 33.5 \\
  % sim scores = 100.0, 99.6, 99.8
 + \model\ 20L & 99.6 & 98.3 & 15.0\phantom{*} & 24.5 \\
  % sim scores = 98.6, 99.4, 100.0
 + \model\ 40L & 99.4 & 87.3 & 6.4\phantom{*} & 24.1 \\
  % sim scores = 97.2, 97.3, 99.4
 + \model\ 60L & 96.5 & 65.5 & 3.2\phantom{*} & \textbf{21.6}   \\
 % sim scores = 94.7
 ~~~+ 60L, 60O & 92.9 & \textbf{51.4} & \textbf{1.5}\phantom{*} & \textbf{21.6} \\
 \midrule 
 GPT-3.5-175B \\
 davinci-003  & - &- & 67.0* & 40.5 \\
 + \model\ 20L & 99.9 &- & 54.0* & 43.1 \\
 + \model\ 40L & 99.8 &- & 36.0* & 43.1 \\
 + \model\ 60L & 99.5 & -& 23.0* & 40.1  \\
 ~~~+ 60L, 60O & 98.3 & - & \textbf{14.0}* & \textbf{38.1} \\
 \midrule
 % 1.7
 Human Text & - & 1.0 & 1.0\phantom{*} & 1.0 \\
\bottomrule
\end{tabular}
\vspace{-0.1in}
\end{wrapfigure}

\textbf{Detection algorithms}: We attack five detectors:\footnote{We consider both model-specific and model-agnostic detectors, as justified in \appendixref{appendix:why-study-model-specific}.} (1) watermarking~\citep{kirchenbauer2023watermark}; (2) DetectGPT~\citep{mitchell2023detectgpt}; (3) GPTZero~\citep{GPTZero}; (4) OpenAI's text classifier~\citep{AITextClassifier};\footnote{This classifier was taken down in July 2023 by OpenAI due to its low accuracy.} and (5) RankGen-XL-all~\citep{krishna-etal-2022-rankgen}.\footnote{While RankGen was not explicitly optimized for this task, it was trained to identify well-written continuations, so we hypothesize that it could also act as a reasonable AI-generated text detector.} We use the default hyperparameters for each detector. We also respect their minimum length specifications, discarding instances where any of the AI-generated text, human-written text, or paraphrased text is shorter than the minimum length. 



\textbf{Paraphrasing AI-generated text}: We pass the prompts for each task and AI-generated responses to those prompts through \model. Specifically, we feed the model input of the form,
\begin{quote}
lexical = L, order = O prompt
\texttt{<p>} generated-text \texttt{</p>},
\end{quote}
 where $L$ and $O$ represent the paraphraser diversity control codes and the \texttt{<p>} and \texttt{</p>} special tokens mark the boundaries of the text to be paraphrased. We use $L = 20, 40, 60$ and $O = 0, 60$ in our main attack experiments. After paraphrasing, we ensure that the AI-generated sequence, paraphrased sequence, and human-written sequence have an equal number of words by truncating them to the length of the shortest among the three. To ensure higher semantic preservation, we iteratively paraphrase long sequences three sentences at a time, keeping already paraphrased text in the context of the generation. To show the effectiveness of our attack, we only \textbf{paraphrase each generation once}, rather than draw multiple samples until it evades detection.\footnote{We discuss this attack briefly in \sectionref{sec:multiple-samples}.}



\subsection{Attacking AI-generated text detectors}
\label{sec:attack-expts}


We present our results in \tableref{tab:watermark-attacks} and \figureref{tab:attacks-lfqa}. Overall we find that:



\textbf{Paraphrasing significantly lowers detection accuracy while preserving input semantics}. Across all LMs, detectors,\footnote{Except RankGen, which scores paraphrases as AI-generated more often than non-paraphrased text. We attribute this to paraphrases being poorer continuations to the prompt compared to the original (\appendixref{sec:intrinsic-evaluation}), an aspect RankGen bases its score on. However, it has low overall performance since it is not trained for this task.} and tasks, paraphrasing significantly lowers detection accuracy across all diversity control codes. For instance, paraphrasing GPT2-XL open-ended generations reduces watermark detection accuracy from 100\% to 57.2\%, and DetectGPT accuracy from 70.3\% to just 4.6\%. Trends are similar even for large LMs like GPT-3.5, for which paraphrasing reduces OpenAI's text classifier accuracy from 30.0\% to 15.6\%. Additionally, \model\ preserves semantics effectively, as 88\%-99\% paraphrases achieve a \spavg\ score higher than the median score of human-written paraphrases. High semantic preservation is supported by careful human evaluations in \appendixref{appendix:HumanEval}. Overall, we find that watermarking is the most resilient detector to paraphrasing.



\begin{wrapfigure}{r}{0.43\textwidth}
    \centering
    \includegraphics[width=0.43\textwidth]{figures/gpt2_xl_fpr_1}
    \caption{ROC plots (0-1\% FPR) for GPT2-XL using different detectors, before (solid lines) and after paraphrasing (dashed). Paraphrasing reduces detection rate across FPRs, and our detector \emph{retrieval} detects paraphrases best; full plots in \appendixref{appendix:more-roc-curves}.}
    \label{fig:auc-plots}
    \vspace{-0.4in}
\end{wrapfigure}

\textbf{Non-watermarking detectors are generally ineffective.} We observe that all detectors apart from watermarking struggle with text generated by larger models like OPT-13B and GPT-3.5, achieving detection accuracies $<$ 50\%. While DetectGPT is effective on the smaller GPT2-XL model (74.9\% on long-form QA), its accuracy drops to just 29.8\% on OPT-13B. Furthermore, GPTZero and RankGen perform the worst among the five detectors on all tested LMs (\tableref{tab:watermark-attacks}), as they are only able to detect < 15\% of non-paraphrased AI-generated text. Thus, we recommend against using these detectors.




\textbf{ROC plots confirm the trends at different false positive rates}. In \figureref{fig:auc-plots}, we plot the detection accuracy (true positive rate) at different values of FPR between 0\% and 1\% for GPT2-XL. Overall, paraphrasing significantly drops detection rates across all FPR thresholds (more plots in \appendixref{appendix:more-roc-curves}).




\subsection{Alternative paraphrasing attacks} 
\label{sec:alt-paraphrasers}
\label{sec:multiple-samples}

%Here, we discuss two other (untested) ways to attack AI-generated text detectors via paraphrasing, which further showcase the brittleness of existing detectors.



\textbf{Paraphrasing multiple times:} Our presented attacks use just a single paraphrase generated by \model\ to evade detection. A simple way to further improve the effectiveness of a paraphrase attack is to sample multiple times\footnote{Precisely, compute $f_\text{dipper}(x)$ for different random seeds while sampling text. Alternatively, an attacker could also compute $f_\text{dipper}(f_\text{dipper}(...f_\text{dipper}(x)))$, but this will lead to excessive semantic drift from $x$.} from \model\ and choose a paraphrase that evades the detector while also preserving semantics. We do not perform this attack as it can only be done if an attacker has access to a detector, which may be a strong assumption (see \appendixref{sec:limitations-retrieval}). That being said, using multiple paraphrase samples can make the attacks even more potent against publicly available detectors.

\textbf{Non-\model\ paraphrasers:} A second alternative is to use non-\model\ paraphrasers that operate at the sentence level. These models can be deployed for long-form text inputs by paraphrasing the inputs sentence by sentence, ignoring prompt context. While the concurrent work of~\citet{sadasivan2023aigenerated} shows that this method can also evade detection, our ablations in \appendixref{sec:intrinsic-evaluation} show that non-contextual versions of \model\ have lower quality and are less compatible with the prompt as \model\ paraphrasers. Moreover, most existing paraphrasers lack fine-grained diversity control and multi-sentence input support (survey in \appendixref{sec:appendix:papersurvey}), two desired properties from an attacker's point of view: attackers want to modify long multi-sentence responses \emph{just enough} to evade detection.

A more interesting option is to use an LLM like ChatGPT to perform few-shot contextual paraphrasing. While this method is likely to provide accurate paraphrases,\footnote{In initial experiments, we observed that \model\ performs competitively with the much larger and more powerful GPT-3.5 davinci-003 model in terms of paraphrase quality, and significantly better at controlling diversity. This finding shows that specialized smaller models can outperform LLMs in paraphrasing tasks.} they may be detectable by strategies like watermarking (whether using the same API as the original LLM or a different one). We thus expect a sophisticated adversary to use their own private paraphraser (like \model) to evade detection.



\section{Defense against paraphrase attacks using retrieval}
\label{sec:defenses}
\label{sec:defense-results}

In \sectionref{sec:attack-expts}, we observed that paraphrasing is an effective attack against AI-generated text detection algorithms.  How can API providers who serve outputs from large language models (LLMs) defend against these attacks? In this section, we propose \emph{retrieval} over previously-generated sequences as a defense against paraphrase attacks. At a high level (\figureref{fig:defense-idea}), an API provider first stores every sequence generated by their LLM in a database. The API provider offers an interface that allows users to enter candidate AI-generated text as a query. The interface searches over the entire database of previously-generated text, trying to find a sequence that approximately matches the content of the input query. This search can be done using a semantic similarity scorer like SP~\citep{wieting-etal-2022-paraphrastic} or a retriever like BM25~\citep{robertson1995okapi}. Since paraphrasing approximately preserves input semantics, we expect such a defense to still be able to map paraphrased generations to their source. 

\begin{table*}[t!]
\small
\begin{center}
\begin{tabular}{ lrrrrrrrrr } 
\toprule
\multicolumn{10}{c}{\textbf{Long-form Question Answering} (300 generated tokens)}\vspace{0.1in} \\ 
  & \multicolumn{3}{c}{GPT2-XL} & \multicolumn{3}{c}{OPT-13B} & \multicolumn{3}{c}{GPT-3.5 (davinci-003)}\\
\cmidrule(lr){2-4} \cmidrule(lr){5-7} \cmidrule(lr){8-10} 
& Original & + 60L & + 60L,60O & Original & + 60L &  + 60L,60O & Original & + 60L & + 60L,60O \\
\midrule
 \multicolumn{5}{l}{\emph{Baseline methods}:} \vspace{0.15cm} \\
Watermark & 100.0 & 71.1 & 55.8 & 100.0 & 65.5 & 51.4 & - & - & -\\
DetectGPT & 74.9 & 15.8 & 7.6 & 29.8 & 3.2 & 1.5 & 1.0 & 0.0 & 0.0 \\
OpenAI & 59.2 & 31.3 & 32.7 & 33.5 & 21.6 & 21.6 & 40.5 & 40.1 & 38.1 \\
\midrule
\multicolumn{10}{l}{\emph{(Ours)} Retrieval over corpus of 3K generations from model itself, with retriever:} \vspace{0.15cm} \\
%~~~SP & 100.0 & 95.0 & 92.8 & 100.0 & 94.2 & 85.4 & 100.0 & 95.4 & 89.4 \\
%~~~BM25 &  100.0 & 99.6 & 98.9 & 100.0 & 99.6 & 98.8 & 100.0 & 99.4 & 98.8  \\
~~~SP & 100.0 & 95.6 & 87.7 & 100.0 & 94.8 & 85.3 & 100.0 & 94.2 & 85.1 \\
~~~BM25 & 100.0 & 99.2 & 97.8 & 100.0 & 99.3 & 97.3 & 100.0 & 98.6 & 96.2 \\
\midrule
\multicolumn{10}{l}{\emph{(Ours)} Retrieval over corpus of 9K generations pooled from all three models, with retriever:} \vspace{0.15cm} \\
%~~~SP & \\
%~~~BM25 & 100.0 & 99.4 & 98.6 & 100.0 & 99.5 & 98.5 & 100.0 & 99.4 & 98.6 \\
~~~SP & 100.0 & 88.9 & 75.4 & 100.0 & 89.6 & 76.4 & 100.0 & 93.8 & 84.6\\
~~~BM25 & 100.0 & 98.3  & 95.2  & 100.0 & 98.5 & 94.4 & 100.0 & 98.5 & 96.0\vspace{0.05in}\\
\midrule
 \multicolumn{10}{c}{\textbf{Open-ended text generation with Wikipedia prompts} (300 generated tokens)}\vspace{0.1in} \\ 
 & \multicolumn{3}{c}{GPT2-XL} & \multicolumn{3}{c}{OPT-13B} & \multicolumn{3}{c}{GPT-3.5 (davinci-003)}\\
\cmidrule(lr){2-4} \cmidrule(lr){5-7} \cmidrule(lr){8-10} 
& Original & + 60L & + 60L,60O & Original & + 60L &  + 60L,60O & Original & + 60L & + 60L,60O \\
\midrule
\multicolumn{5}{l}{\emph{Baseline methods}:} \vspace{0.15cm} \\
Watermark & 100.0 & 68.9 & 57.2 & 99.9 & 63.7 & 52.8 & - & - & -\\
DetectGPT &  70.3 & 8.7 & 4.6 & 14.3 & 0.8 & 0.3 & 2.0 & 0.5 & 0.0 \\
OpenAI & 21.6 & 13.3 & 14.8 & 11.3 & 9.1 & 10.0 & 30.0 & 15.6 & 15.6 \\
\midrule
\multicolumn{10}{l}{\emph{(Ours)} Retrieval over corpus of 3K generations from model itself, with retriever:} \vspace{0.15cm} \\
%~~~SP & 100.0 & 83.0 & 80.8 & 100.0 & 81.8 & 80.0 & 100.0 & 63.4 & 47.4 \\
%~~~BM25 & 100.0 & 99.1 & 98.0 & 100.0 & 97.2 & 95.3 & 100.0 & 58.8 & 37.4 \\
~~~SP & 100.0 & 86.4 & 81.5 & 100.0 & 84.4 & 77.7 & 100.0 & 65.9 & 49.5 \\
~~~BM25 & 100.0 & 99.0 & 98.0 & 100.0 & 97.2 & 95.3 & 100.0 & 58.8 & 37.4\\
\midrule
\multicolumn{10}{l}{\emph{(Ours)} Retrieval over corpus of 9K generations pooled from all three models, with retriever:} \vspace{0.15cm} \\
%~~~SP & 100.0 & 82.8 & 80.0 & 100.0 & 81.8 & 80.0 & 100.0 & 76.6 & 60.0 \\
%~~~BM25 & 100.0 & 98.9 & 97.8 & 100.0 & 97.1 & 95.3 & 100.0 & 58.8 & 37.4 \\
~~~SP & 100.0 & 72.1 & 63.2 & 100.0 & 74.6 & 65.6 & 100.0 & 63.1 & 45.6 \\
~~~BM25 & 100.0 & 85.0 & 78.7 & 100.0 & 87.2 & 79.1 & 100.0 & 58.8 & 37.4 \\
\bottomrule
\end{tabular}
\end{center}
\caption{Our proposed defense (retrieval) significantly improves AI-generated text detection accuracy (at false positive rate 1\%) over baselines on all settings, including our most diverse paraphrase attacks (+60L, +60L,60O).}
\label{tab:watermark-defense}
\end{table*}



In this section we first formalize our retrieval defense in \sectionref{sec:retrieval-formulation}. We then perform controlled comparisons of retrieval with other detection algorithms (\sectionref{sec:control-defense-results}) and evaluate our method at scale using a large retrieval corpus of 15M generations (\sectionref{sec:scale-defense-results}).

\subsection{Formulating the retrieval defense}
\label{sec:retrieval-formulation}
\label{sec:retriever-choice}

Let $f_\text{LM}$ be an LLM API (e.g., GPT-3.5) that takes a prompt $x$ as input and returns a continuation $y$. Let $f_\text{ret}$ be an encoder (e.g., TF-IDF, neural network) that embeds variable-length sequences into fixed-size vectors that represent the input semantics. Then, we do the following:\\

\noindent 1. \textbf{Building the database}: Let $x_1,..., x_N$ be the set of prompts that have been fed as input to the API in the past, where $N$ can potentially be very large for popular APIs (we study up to $N=$ 15M). We construct our database $\mathbf{Y}$ by:
\begin{align*}
    y_i &= f_{\text{LM}}(x_i)~~~~~~\text{for } i = 1,...,N \\
    \mathbf{y}_i &= f_{\text{ret}}(y_i)~~~~~~\text{for } i = 1,...,N \\
    \mathbf{Y} &= [\mathbf{y}_1, ... \mathbf{y}_N]
\end{align*}
The database $\mathbf{Y}$ is dynamically updated and stored on the API side. It is inaccessible to clients except via the API described in the next step.
\vspace{0.1in}

\noindent 2. \textbf{Querying the database}: Let $y'$ be a candidate text, and suppose a client wishes to know whether $y'$ was generated by the API $f_\text{LM}$. The API provider can check this by computing whether the maximum similarity score of $y'$ to an entry in the database exceeds some threshold:
\begin{align*}
    \mathbf{y}' &= f_{\text{ret}}(y') \\
    \text{score} &= \max_{i} \frac{\mathbf{y}' \cdot \mathbf{y}_i}{|\mathbf{y}'|~|\mathbf{y}_i|} \\
\text{output} &= \text{score} > L
\end{align*}

Here, $L$ is the detection threshold chosen by the API provider. We expect unperturbed machine-generated text to always get a score of 1.0, while paraphrasing the text will lower the detection score. Hence, lowering $L$ will increase the detection rate of heavily-paraphrased text but also increase the false positive rate (i.e., human-written text that resembles sequences previously generated by the LLM API can be falsely flagged). Since $N$ can be very large, the score can also be approximated using efficient nearest neighbor libraries like FAISS~\citep{johnson2019billion}. However, in this work we only compute exact inner products.

\vspace{0.1in}

\noindent \textbf{Choice of retriever $f_\text{ret}$}: We experiment with two choices for $f_\text{ret}$ including \spavg from ~\citet{wieting-etal-2022-paraphrastic} and BM25~\citep{robertson1995okapi}. We implement BM25 using the \texttt{retriv} library from~\citet{retriv2022}. In order to normalize and calibrate BM25 scores, we compute the unigram token overlap (using the evaluation script from~\citealp{rajpurkar-etal-2016-squad}) between the candidate $y'$ and the best retrieval $y*$ to get a detection score in $[0, 1]$.




\subsection{Controlled comparisons of retrieval with other AI-generated text detectors}
\label{sec:control-defense-results}


First, we conduct a controlled comparison between the detection algorithms evaluated in \sectionref{sec:attack-expts} and our retrieval method. We construct two retrieval corpora for this experiment: (a) a corpus of the 3K sequences generated by a specific LM for one of the tasks; and (b) a corpus of 9K sequences formed by concatenating the generations from all three LMs considered in this paper. We expect (b) to be a more difficult test for our method than (a), since the retriever needs to distinguish between multiple generations from different models given the same prompt. Next, we perform retrieval over this corpus using the original AI-generated text, its paraphrase (generated by \model\ using control codes L60 and L60,O60), and human-written text as queries. While in this experiment, we only consider queries that are at least 50 token long, we discuss the effect of input length in \sectionref{sec:detect-length-effect}. 
% In \tableref{tab:watermark-defense}, we observe the following across both corpora:


\begin{figure}[t!]
    \centering
    \includegraphics[width=0.48\textwidth]{figures/scale_retrieval}
    \caption{The effectiveness of retrieval as a detection strategy as a function of retrieval corpus size. In all settings we note high detection accuracy of paraphrases (at 1\% FPR), which only slightly degraded as the corpus is scaled from 1M to 15M generations.}
    \label{fig:semantic-search}
\end{figure}


\vspace{0.05in}

\tableref{tab:watermark-defense} shows that \textbf{across all LMs, retrieval is a much more effective detector than baseline detectors}. On unperturbed machine-generated text, retrieval has a 100\% detection accuracy due to exact match with the retrieval corpus. On paraphrased text, retrieval with BM25 is quite effective, detecting 97.8\% of the highest-diversity paraphrases (L60,O60) on GPT2-XL, 97.3\% on OPT-13B and 96.2\% on GPT-3.5 in long-form question answering. This is significantly better than the next best alternative with competing detectors (55.8\%, 51.4\%, 38.1\%, respectively). Even on our harder augmented database of 9K generations, detection rates continue to be high (95.2\%, 94.4\%, 96.0\%). Finally, we observe that BM25 is a more effective retriever than \spavg, scoring 95.2\% vs 75.4\% on the augmented setting in GPT2-XL with long-form question answering. These trends are consistent across different false positive rate thresholds, as shown in our ROC curves (\figureref{fig:auc-plots}).

\subsection{Is retrieval an effective detector with a large retrieval corpus?}
\label{sec:scale-defense-results}
In the previous section, we conducted experiments using the set of 9K sequences generated by all three models as the retrieval corpus. However, this is more of a toy experiment: in practice, a popular LLM API may serve millions of queries a day. As the corpus grows larger, the  false positive rate (i.e., human-written text falsely detected as machine-generated) will grow. How well do retrieval-based detectors scale?  To answer this question, we need access to a large corpus of machine-generated text. We utilize the training data used to train RankGen~\citep{krishna-etal-2022-rankgen}, which contains over 70M machine-generated sequences. We use the Project Gutenberg and Wikipedia splits of the training data, each of which contain 15M sequences generated by a T5-XXL model~\citep{raffel2020exploring} fine-tuned on the different documents in the same domain. We discard generations which are shorter than 50 tokens, and paraphrase a small subset of 2K generations to evaluate retrieval.
 \vspace{0.1in}

\noindent \textbf{Retrieval is effective even with a corpus size of 15M generations.} In \figureref{fig:semantic-search}, we plot the detection accuracy as a function of retrieval database size. Overall, we observe that detection accuracy remains consistently high across different corpus sizes (varying from 1M generations to 15M generations). We do observe slight drops in performance as the corpus size increases: just 1\% (98.3 to 97.3) on Project Gutenberg (PG19) and 9.6\% (90.0 to 80.4) on Wikipedia. Consistent with the results in \sectionref{sec:control-defense-results}, BM25 continues to outperform \spavg\ across different retrieval corpus sizes.




\begin{figure}[t!]
    \centering
    \includegraphics[width=0.48\textwidth]{figures/scale_retrieval_length}
    \caption{The variation in retrieval performance as a detector with different query lengths. Overall, retrieval performs best with queries of length 50 tokens or more. }
    \label{fig:retrieval-length}
\end{figure}



\vspace{0.1in}

\noindent \textbf{Retrieval detection works best with 50 or more tokens of generated text}. Another important factor for our retrieval-based detector is the query length: shorter queries are likely to have more matches (many of them spurious) compared to longer ones. In \figureref{fig:retrieval-length}, we plot the detection accuracy of paraphrased sequences at various query lengths by  truncating each sequence to its first $X$ words before using it as a query for BM25. We use a retrieval corpus of 2M generations for this experiment. We observe that BM25 struggles to detect paraphrased text with a query length of 20 (less than 25\% accuracy), but the detection rate rapidly increases and begins to plateau at 50 tokens. 
\label{sec:detect-length-effect}


\subsection{Ideas to make retrieval detection work well at an even larger scale}
\label{sec:suggestions-retrieval-scale}

In \sectionref{sec:scale-defense-results}, we observed that our proposed retrieval detector is effective even with a large corpus of 15M previously-generated sequences. While we do not have access to a larger corpus of generations (billion-scale), in this section we describe some ideas to improve retrieval detection at such a scale.


\begin{enumerate}
    \item \textbf{Timestamp filtering in retrieval corpus.} To reduce the large search space, the detector interface could provide users with an option to restrict retrieval to only a fixed time period during which the text was likely to be generated. For instance, a common use-case of AI-generated text detection might be when teachers attempt to catch plagiarism in college essays. Teachers could restrict retrieval to only those generations  created during the assignment window.
    % \item \textbf{Recommend users to input longer queries.} Like all AI-generated text detectors, retrieval works best with longer queries (\sectionref{sec:detect-length-effect}). Clients should use as long a query as possible to maximize the chances for detecting paraphrases.
    \item \textbf{More sophisticated retrieval strategies.} In our work, we only explore simple retrieval strategies like BM25. However, several more sophisticated retrieval strategies exist, which are known to boost performance~\citep{thakur2021beir} and could be useful here. These include methods like re-ranking of top-$k$ retrievals~\citep{khattab2020colbert} or dense retrieval~\citep{karpukhin2020dense}. We do note that these more complex methods are also slower, and latency is likely to be a pressing concern for API providers.
    \item \textbf{Fine-tuning dense retrievers for the detection task.} The retrievers in our work are not fine-tuned for the task of AI-generated text detection. However, we hypothesize that fine-tuning retrievers on this task can help retrievers adapt better to the retrieval corpus and detection task. Specifically, a contrastive learning approach could be adopted here: positive pairs are paraphrased or otherwise noised sequences paired with their generations, while negative pairs are human-written continuations paired with the machine-generated text.
\end{enumerate}



\subsection{Limitations of retrieval for detection}
\label{sec:limitations-retrieval}

While retrieval over previously-generated sequences is an effective defense against paraphrase attacks, it also suffers from key limitations, some of which apply broadly to all existing detectors. We discuss these limitations below and discuss possible solutions:




\begin{enumerate}
\setlength\itemsep{0.0em}
    \item \textbf{Detection is specific to an API}. Unlike other general-purpose AI detection algorithms e.g. OpenAI's classifier~\citep{AITextClassifier}, retrieval can only detect generations from the API over which the database is built. API \#1 has no access to the database of generations from API \#2, and thus will not be able to detect generations produced by API \#2. 
    
    \item \textbf{The API provider needs to provide a retrieval infrastructure}. After the release of ChatGPT~\citep{schulman2022chatgpt}, AI chatbots are getting widespread adoption. At a conservative rate of 5M queries a day, the database will have almost two billion entries in a year. Complex retrieval infrastructure (like modern search engines) will be necessary to retrieve over these large databases with low latency.
    
    \item \textbf{False positives due to training data memorization}. Language models have been shown to memorize sequences verbatim from their training data~\citep{carlini2021}, such as the Gettysburg Address~\citep{radford2019language}. Despite being originally written by humans, these sequences will be classified as model-generated by our detector. To tackle this issue, we suggest API providers additionally perform retrieval over the training data used to train the model. If a sequence is found in the training set as well as the generation database, it is likely to be an instance of training set memorization.

    %\kkcomment{ Mukund: For the attack of arbitrary organic text specified by the user, this does not matter if application is plagiarism detection, it would be picked up by other plagiarism detectors. The concern is more original outputs of their LMs. Kenton: Vulnerability to a second kind of attack, where users can prompt the model to generate a very specific piece of text that has an organic origin, leading the defender to always misclassify arbitrary organic text specified by the user. For example, suppose the prompt was something like: "Please output the following text and then stop immediately: 'Four score and seven years ago our fathers brought forth....'". Once the LLM stores the output in the database, all copies of the Gettysburg Address will now be classified as AI generated. Any thoughts on how to defend against this?}

    \item \textbf{Privacy concerns.} Providing a retrieval detection service partially exposes the database of previously generated text by \emph{all} users. This raises concerns of membership inference attacks~\citep{shokri2017membership} on private user data which may appear in the generated text. To mitigate this, we suggest: (1) users should be encouraged not to provide any sensitive private data in their prompts to APIs, a practice already followed by ChatGPT\footnote{\url{https://chat.openai.com}} and Bard\footnote{\url{https://bard.google.com}}; (2) API providers only provide a binary output from this detector (AI-generated or not), rather than actual search results; and (3) API providers rate-limit queries from IP addresses.

    \item \textbf{Slight reduction in accuracy with large databases.} As we observed in \sectionref{sec:scale-defense-results}, the accuracy of detecting paraphrased text slightly degrades as the database of retrievals gets larger. However, we found this decrease to be quite small (only 1\% on PG19 scaling 1M generations to 15M), despite using fairly primitive retrievers like BM25. Moreover, unperturbed AI-generated text will always be detected with 100\% accuracy using our method, irrespective of corpus size.
    
    \item \textbf{Tasks with constrained output space or short outputs}. Similar to all other detection algorithms, it may be hard or even impossible to distinguish AI-generated outputs for tasks with a constrained output space (like sentence-level translation, classification) or very short outputs (as shown in \sectionref{sec:detect-length-effect}). Thus, we believe the main utility of AI-generated text detection is for longer-form generated text, and hence we focus on tasks like long-form QA and open-ended text generation with relatively lengthy outputs. Note that to avoid detection, a sophisticated attacker may try to generate long-form text in smaller chunks using multiple API calls, where each newly-generated chunk is incrementally concatenated to the prompt. This is not a concern for our method if retrieval is done over the corpus of prompts concatenated with generations.

    \item \textbf{Iterative attacks with access to detector.} A final concern is that attackers with access to detection algorithms will iteratively modify their perturbations until they avoid detection. While this is a valid concern for all detectors, we believe retrieval has an important advantage over the alternatives. Since the corpus of previously-generated text is proprietary, only the API provider can provide access to this detection service - it is impossible for attackers to locally reproduce this detector. This allows API providers to adopt several mitigation strategies such as (1) rate-limiting queries to avoid iterative attacks; (2) providing retrieval access only to verified users (e.g., teachers); and (3) detecting possible iterative attacks by analyzing previously queries to the retriever.

\end{enumerate}



\begin{table*}[t!]
\small
\begin{center}
\begin{tabular}{ lrrrrrr } 
 \toprule
   \multicolumn{7}{c}{\textbf{Open-ended generation with GPT2-XL on Wikipedia prompts}}\vspace{0.1in} \\ 
   & \multicolumn{2}{c}{\textsc{RankGen-XL}} & \multicolumn{2}{c}{GPT3.5 davinci-003 perplexity} & \multicolumn{2}{c}{unigram overlap with prompt} \\
 \cmidrule(lr){2-3}  \cmidrule(lr){4-5} \cmidrule(lr){6-7}
 Control & rewrite A &  rewrite B  & rewrite A &  rewrite B  &  rewrite A &  rewrite B \\
 \midrule
 \multicolumn{7}{l}{\textbf{Experiment 1}: \emph{Is context helpful for paraphrasing?}} \vspace{0.05in} \\
 \multicolumn{7}{l}{rewrite A = \model\ with  context}\\
 \multicolumn{7}{l}{rewrite B = \model\ no context} \vspace{0.05in}\\
 20L &  \textbf{65}\% {\scriptsize 10.2} & 35\% {\scriptsize 9.2} & \textbf{71}\% {\scriptsize 11.5} & 29\% {\scriptsize 12.6} & \textbf{55}\% {\scriptsize 41.3} & 45\% {\scriptsize 40.7}\\
 40L & \textbf{64}\% {\scriptsize \phantom{0}9.8} & 36\% {\scriptsize 8.5} & \textbf{70}\% {\scriptsize 11.9} & 30\% {\scriptsize 13.0} & \textbf{57}\% {\scriptsize 40.7} & 43\% {\scriptsize 39.9}\\
 60L & \textbf{67}\% {\scriptsize \phantom{0}9.6} & 33\% {\scriptsize 7.6} & \textbf{68}\% {\scriptsize 12.3} & 32\% {\scriptsize 13.6} & \textbf{56}\% {\scriptsize 39.9} & 44\% {\scriptsize 39.2}\\
 60L,60O & \textbf{65}\% {\scriptsize \phantom{0}8.3} & 35\% {\scriptsize 6.4} & \textbf{75}\% {\scriptsize 12.9} & 25\% {\scriptsize 15.0} & \textbf{58}\% {\scriptsize 39.4} & 42\% {\scriptsize 38.2}\\
 \midrule
  \multicolumn{7}{l}{\textbf{Experiment 2}: \emph{Is it helpful to paraphrase multiple sentences at a time?}} \vspace{0.05in} \\
 \multicolumn{7}{l}{rewrite A = \model\ 3 sentences at a time} \\
 \multicolumn{7}{l}{rewrite B = \model\ 1 sentence at a time} \vspace{0.05in}\\
  20L & \textbf{58}\% {\scriptsize 9.2} & 42\% {\scriptsize 8.6} & \textbf{86}\% {\scriptsize 12.6} & 14\% {\scriptsize 15.3} & 48\% {\scriptsize 40.7} & \textbf{52}\% {\scriptsize 40.9} \\
 40L & \textbf{56}\% {\scriptsize 8.5} & 44\% {\scriptsize 8.1} & \textbf{83}\% {\scriptsize 13.0} & 17\% {\scriptsize 15.8} & 45\% {\scriptsize 39.9} & \textbf{55}\% {\scriptsize 40.4}\\
 60L & \textbf{54}\% {\scriptsize 7.6} & 46\% {\scriptsize 7.5} & \textbf{79}\% {\scriptsize 13.6} & 21\% {\scriptsize 15.7} & 45\% {\scriptsize 39.2} & \textbf{55}\% {\scriptsize 39.9}\\
 60L,60O & \textbf{50}\% {\scriptsize 6.4} & \textbf{50}\% {\scriptsize 6.4} & \textbf{85}\% {\scriptsize 15.0} & 15\% {\scriptsize 19.6} & 42\% {\scriptsize 38.2} & \textbf{58}\% {\scriptsize 39.5}\\
 \midrule
  \multicolumn{7}{l}{\textbf{Experiment 3}: \emph{Does paraphrasing preserve the quality of the original text?}} \vspace{0.05in}\\
 \multicolumn{7}{l}{rewrite A = no paraphrasing} \\
 \multicolumn{7}{l}{rewrite B = \model} \vspace{0.05in}\\
  20L & \textbf{50}\% {\scriptsize 10.4} & \textbf{50}\% {\scriptsize 10.2} & \textbf{61}\% {\scriptsize 11.1} & 39\% {\scriptsize 11.5} & \textbf{51}\% {\scriptsize 41.6} & 49\% {\scriptsize 41.3}\\
 40L & \textbf{57}\% {\scriptsize 10.4} & 43\% {\scriptsize \phantom{0}9.8} & \textbf{67}\% {\scriptsize 11.1} & 33\% {\scriptsize 11.9} & \textbf{55}\% {\scriptsize 41.6} & 45\% {\scriptsize 40.7}\\
 60L & \textbf{58}\% {\scriptsize 10.4} & 42\% {\scriptsize \phantom{0}9.6} & \textbf{73}\% {\scriptsize 11.1} & 27\% {\scriptsize 12.3} & \textbf{58}\% {\scriptsize 41.6} & 42\% {\scriptsize 39.9}\\
 60L,60O & \textbf{68}\% {\scriptsize 10.4} & 32\% {\scriptsize \phantom{0}8.3} & \textbf{79}\% {\scriptsize 11.1} & 21\% {\scriptsize 12.9} & \textbf{61}\% {\scriptsize 41.6} & 39\% {\scriptsize 39.4} \\
\bottomrule
\end{tabular}
\end{center}
\vspace{-0.1in}
\caption{Ablation experiments demonstrate the high quality of \model's paraphrases compared to alternatives. Displayed scores are the percentage of cases in which rewrite A is preferred over B by one of the three metrics, with subscripts showing absolute average scores on each metric across the dataset. Overall, \model\ benefits from context outside the input (Experiment 1), multi-sentence paraphrasing (Experiment 2), and is not too far behind non-paraphrased text in terms of quality (Experiment 3).}
\vspace{-0.1in}
\label{tab:paraphrase-ablations}
\end{table*}

%\kkcomment{defenses against paraphrasing attacks, we may want to focus on aspects paraphrasers cannot easily modify.. one of them could be people's names in story generation ---- upweight watermarks on character names which weren't specified in the prompt, semantically-specified watermarks}

%\kkcomment{overall i'm thinking the only defense against paraphrase attacks could be some kind of "semantic" watermarking.. in many prompts the LLM has a choice about what kind of content to generate.. the particular content it chooses needs to be watermarked somehow, rather than a surface-form watermark in the Goldstein paper.. perhaps some kind of semantic / content vectors can be used, like what's done in diffusion LMs / GANs? The initial random image / seed determines the final choice of content produced, while still adhering to the prompt.. ofc this won't work on very constrained prompts ("Who is president of USA"), but i guess no watermark is going to work in those cases.. btw a simple defense here is store every generation produced by an API as semantic vector(s), and then do MIPS search over semantic vectors during detection.. kind of like retrieval over the set of generations by the LLM}


%\yscomment{A Benchmark Corpus for the Detection of Automatically Generated Text in Academic Publications https://aclanthology.org/2022.lrec-1.501.pdf Arkham: /data/yixiao/paraphrase-eval/GeneratedTextDetection-main FullyGenerated papers : 100 generated (average len 1243), 100 human original. generation prompt is chosen from the original abstract. Their length is variable as they can be composed from the abstract alone to more sections such as introduction, related work and conclusion. fine tuned gpt2 with latest 100 papers from computation and language HybridAbstractDataset: original (human written) abstracts with some sentences being substituted with machine generated sentences, 100 generated (average len 177), 100 human original, Artificial Intelligence, model generated proposal and conclusion. This generation is done with human intervention, so that it is biased towards the objective strategy of making the generated content difficult to detect.}


\begin{table*}[t]
\small
\centering
\resizebox{0.97\textwidth}{!}{%
\begin{tabular}[b]{@{}p{0.5cm}p{6.2cm}p{4.5cm}p{4cm}@{}}
\toprule
  \textbf{L} &
  \textbf{Original} &
  \textbf{Paraphrase} &
  \textbf{Annotator Comment}\\\midrule
  40 &
  \textbf{When tailoring his 1970 composition "I'm the Greatest" for Starr to record on the 1973 album Ringo, Lennon referenced the song title with the lines [$\ldots$]} Lennon's original \textcolor{OrangeRed}{line} was changed to "Back off Boogaloo" to \textcolor{OrangeRed}{reference the song's title, creating} the line \textcolor{OrangeRed}{that} Starr sang \textcolor{OrangeRed}{in} the finished \textcolor{OrangeRed}{version.} &
  Lennon's original \textcolor{ForestGreen}{reference to "I'm the Greatest"} was changed to "Back off Boogaloo" to \textcolor{ForestGreen}{create} the line Starr sang \textcolor{ForestGreen}{on} the finished \textcolor{ForestGreen}{recording.} &
  \textbf{4---Nearly equivalent}\newline The original does not specify that the line in question was "I'm the Greatest." However, this can be inferred from the rest of the passage. Otherwise, the paraphrase and the original are quite similar.\\\midrule
  60 &
  \textbf{Heaving wildly, the Hunyani began to descend rapidly. [\ldots]} The crew \textcolor{OrangeRed}{was unable to} regain \textcolor{OrangeRed}{control} and the Hunyani \textcolor{OrangeRed}{crashed} into the ground \textcolor{OrangeRed}{at 17:14, killing all onboard.}  The \textcolor{OrangeRed}{crash} was investigated by the Civil Aviation Authority of \textcolor{OrangeRed}{Zimbabwe and the investigation concluded} that the crash was caused by a combination of pilot error, \textcolor{OrangeRed}{fatigue}, and \textcolor{OrangeRed}{weather conditions.} The \textcolor{OrangeRed}{pilot} had \textcolor{OrangeRed}{failed to} properly \textcolor{OrangeRed}{assess} the weather \textcolor{OrangeRed}{conditions before takeoff, instead relying} on a forecast from a private \textcolor{OrangeRed}{weather} service. The weather \textcolor{OrangeRed}{conditions worsened as the plane flew and with the pilots already fatigued, they were unable to appropriately react to the changing conditions. The combination of these factors led to the plane losing control and crashing.} &
  The crew \textcolor{ForestGreen}{could not} regain \textcolor{ForestGreen}{control,} and \textcolor{ForestGreen}{at 17:14} the Hunyani \textcolor{ForestGreen}{plunged} into the ground \textcolor{ForestGreen}{and all those on board were killed.} The \textcolor{ForestGreen}{accident} was investigated by the Civil Aviation Authority of \textcolor{ForestGreen}{Zimbabwe, which found} that the crash was caused by a combination of pilot error, \textcolor{ForestGreen}{fatigue} and \textcolor{ForestGreen}{the weather.} The \textcolor{ForestGreen}{pilots} had \textcolor{ForestGreen}{not} properly \textcolor{ForestGreen}{examined} the weather \textcolor{ForestGreen}{before take-off, and instead had relied} on a forecast from a private \textcolor{ForestGreen}{meteorological} service. The weather \textcolor{ForestGreen}{deteriorated during the flight, and the exhausted pilots were unable to react properly to the deteriorating conditions.} & \textbf{5---Approx. equivalent}\newline Well chosen words make the alternate version different but retain the original meaning well. The only difference is that the paraphrased version does not have the concluding sentence, but it does not alter the meaning in any way as the facts have already been included previously.\\\midrule
  60 &
  \textbf{Sonic's main means of attack is the Spin Attack, in which he curls into a ball and rapidly spins his body, damaging enemies and certain obstacles upon collision. This may be performed by jumping or by rolling on the ground.} \textcolor{OrangeRed}{Sonic can also} use a variety of other \textcolor{OrangeRed}{moves}, such as the \textcolor{OrangeRed}{Homing Attack, Slide Attack, and Wall Jump. He also has access to power-ups like the Speed Shoes, Invincibility, and the Super Peel Out. In some games, Sonic can also use} special items or vehicles to traverse \textcolor{OrangeRed}{levels.} &
  \textcolor{ForestGreen}{Also, the character can} use a variety of other \textcolor{ForestGreen}{attacks,} such as the \textcolor{ForestGreen}{homing attack, the slide attack, and the wall jump. In addition, he can acquire power-ups such as the Speed Boost, Super Speed, and the invincibility meter, and in certain games, he may make use of} special items or vehicles to traverse \textcolor{ForestGreen}{the world.} &
  \textbf{2---Topically related}\newline In the second part of the paraphrase, the writer ignores the actual moves and skills of Sonic and invents some entirely different ones that not mentioned in the original text. The method of performing the attack moves has also been missed out.\\\bottomrule
\end{tabular}%
}
\caption{Representative model outputs of each lexical code (L), along with Likert ratings and justifications from the annotators (native English teachers and editors). The bolded text in the original paragraphs are the context. \textcolor{OrangeRed}{Red words} represent the content being changed in the original text (manually marked by the authors), and \textcolor{ForestGreen}{green words} show the corresponding text in \model's paraphrases.}
\label{tab:cherry-pickedexamples}
\end{table*}
%https://docs.google.com/spreadsheets/d/10LtBZLLuTut2kDdyiCkCkYWovnmgTESOABqse-P30EU/edit?usp=sharing
\begin{table*}[t]
\centering
\resizebox{\textwidth}{!}{%
\begin{tabular}{@{}rrrrrr@{}}
\toprule
\multicolumn{1}{c}{\multirow{2}{*}{\textbf{L}}} &
  \multicolumn{1}{c}{\multirow{2}{*}{\textbf{Sum of 4 and 5}}} &
  \multicolumn{1}{c}{\textbf{5}} &
  \multicolumn{1}{c}{\textbf{4}} &
  \multicolumn{1}{c}{\textbf{3}} &
  \multicolumn{1}{c}{\textbf{2}} \\
\multicolumn{1}{c}{} &
  \multicolumn{1}{c}{} &
  \multicolumn{1}{c}{\textbf{Approx. equivalent}} &
  \multicolumn{1}{c}{\textbf{Nearly equivalent}} &
  \multicolumn{1}{c}{\textbf{Somewhat equivalent}} &
  \multicolumn{1}{c}{\textbf{Topically related}} \\
  \midrule
20                              & 95.0\% & 63.3\% & 31.7\% & 5.0\%    & 0.0\% \\
40                              & 78.3\% & 45.0\%   & 33.3\% & 21.7\% & 0.0\% \\
60                              & 70.0\% & 28.3\% & 41.7\% & 28.3\% & 1.7\% \\\midrule
\textbf{Total} & 81.1\% & 45.6\% & 35.6\% & 18.3\% & 0.6\%\\\bottomrule
\end{tabular}%
}
\caption{This table shows how often each point in the Likert scale was chosen by 3 annotators for the pairs of original and paraphrased texts. Twenty text pairs are randomly selected for each lexical code (L). 81.8\% of the time, our model \model\ provides a paraphrase which is nearly equivalent to the input in terms of semantic meaning.}
\label{tab:humanevalpercentage}
\end{table*}

\section{Experiments measuring intrinsic paraphrase generation quality}
\label{sec:intrinsic-evaluation}

Our experiments in \sectionref{sec:attacks} and \sectionref{sec:defense-results} focused on attacking AI-generated text detectors with paraphrases and defending against these paraphrase attacks. We used \model\ as the underlying paraphrase generation model for all of these experiments. Are \model's paraphrases actually good enough to make the attack worthwhile, and can simpler paraphrasers be just as effective as \model? In this section, we conduct careful ablation experiments (\sectionref{sec:ablation-dipper}) and human evaluations (\sectionref{sec:human-evaluation}) to validate the effectiveness of \model\ at preserving the semantics of the input generation. Our results show that \model\ effectively leverages surrounding context to paraphrase multiple sentences while preserving input semantics.

\subsection{Ablation studies on \model}
\label{sec:ablation-dipper}

In this section, we perform automatic evaluations to confirm the efficacy of \model\ as a paraphraser. From a survey of existing paraphrasers that we carry out in \appendixref{sec:appendix:papersurvey}, \model\ possess two unique features that differentiate it from other paraphrasers: (1) its ability to leverage context from \emph{outside} of the text to be paraphrased (such as the prompt); and (2) its ability to paraphrase multiple sentences at once. How useful are these features while paraphrasing long sequences of text?

To answer this question, we first train an ablated version of \model\ by constructing a training dataset (\sectionref{sec:paraphrase-data-build}) without any left or right context, and then fine-tuning T5-XXL using the same hyperparameters as in \sectionref{sec:model}. We call this model \model-no-ctx. We paraphrase 1K open-ended generations from GPT2-XL using both \model\ and \model-no-ctx, using each of the four configurations of diversity control codes studied in this paper. We then evaluate the quality of the paraphrased text using three metrics: (1) GPT3.5-davinci-003 perplexity~\citep{brown2020language} of the prompt concatenated with the paraphrased continuation; (2) \textsc{RankGen} compatibility between the prompt and the paraphrased continuation~\citep{krishna-etal-2022-rankgen}; and (3) unigram token overlap between the paraphrased continuation and the prompt.

\vspace{0.05in}

\noindent \textbf{Contextual paraphrasing leads to higher quality paraphrases}. In \tableref{tab:paraphrase-ablations} (Experiment 1), we observe that across all four control code configurations and all three metrics, paraphrases from \model\ are preferred over paraphrases from \model-no-ctx. Specifically, with the lexical and order control codes set to 60\% (most diverse), \model\ paraphrases are preferred by GPT3.5 perplexity 75\% of the time compared to non-contextual paraphrases (average perplexity drop of 12.9 vs 15.0).

\vspace{0.05in}




\noindent \textbf{Paraphrasing multiple sentences at a time is better than paraphrasing individual sentences.} Next, we use our \model-no-ctx model to compare two settings: paraphrasing 3 sentences at a time vs paraphrasing 1 sentence at a time before concatenating. We hypothesize that the former will produce higher quality paraphrases since we expect it to better connect discourse elements across the text. Indeed, in \tableref{tab:paraphrase-ablations} (Experiment 2) across all control codes, GPT3.5 and \textsc{RankGen} usually prefer multi-sentence paraphrases over the single-sentence baseline. This preference is 79\% or higher for all control codes when evaluating with GPT-3.5 perplexity, reaching 85\% for L60,O60.

\vspace{0.1in}

\noindent \textbf{\model\ paraphrases are close to the unperturbed GPT-2 XL generations}. Finally, we compare \model\ with the original GPT2-XL generations (without paraphrasing) on the same three metrics. While we expect metrics to prefer non-paraphrased text, a strong paraphraser will produce text that is close to the original in terms of these metrics.
\tableref{tab:paraphrase-ablations} (Experiment 3) confirms our hypothesis: at L20, \textsc{RankGen} has a 50-50 preference between the two outputs, while GPT3.5 prefers the non-paraphrased generations just 61\% of the time, with an average perplexity gain of just 0.4 (11.1 to 11.5). At more diverse control codes, preference for GPT2-XL generations does go up (58\% \textsc{RankGen}, 73\% GPT3.5 for L60), but absolute scores continue to be close (11.1 vs 12.3 GPT-3.5 perplexity). Note that while all of these ablations use just a single paraphrase sample, it is easy for an attacker to obtain multiple samples from \model\ and choose the sample that maximizes these metrics (as discussed in \sectionref{sec:attack-expts}). 

\subsection{Human evaluation of semantic preservation using \model}
\label{sec:human-evaluation}

The automatic semantic similarity scores in \tableref{tab:watermark-attacks} and \ref{tab:attacks-lfqa} indicate that \model~generates paraphrases that are faithful to the original input paragraphs. To confirm this result with human evaluation, we hire three native English teachers and/or editors on Upwork\footnote{\url{https://www.upwork.com}} to evaluate the semantic fidelity of the paraphrases.
As human evaluation is expensive, we focus on the impact of the lexical diversity. We evaluate paraphrases with the lexical codes $L20$, $L40$, and $L60$, corresponding to moderate, medium, and high lexical diversity. For each code, we randomly select 20 original and paraphrased text pairs for our annotators to evaluate semantic similarity on a five-point Likert scale (see \appendixref{appendix:HumanEval} for more details of the scale and interface).\footnote{The original texts are preceded by their context. The annotators see the same amount of text as \model.} The rate at which each point on the scale is chosen is reported in \tableref{tab:humanevalpercentage}. Annotators also must provide free-form comments justifying their ratings. Over 80\% of the time, our annotators rate \model's paraphrases as nearly equivalent (4 out of 5) or approximately equivalent (5 out of 5). Details about inter-annotator agreement are in \appendixref{appendix:HumanEval}.

A qualitative analysis of the free-form annotator comments reveals systemic strengths and shortcomings of \model. We provide a brief analysis below and include three representative examples in \tableref{tab:cherry-pickedexamples}. A  comprehensive discussion with more examples can be found in \appendixref{appendix:HumanEval}. 

\vspace{0.05in}

\noindent\textbf{Strengths:} The first case in \tableref{tab:cherry-pickedexamples} exemplifies \model's ability to leverage information from context to increase diversity while maintaining coherence. No prior paraphraser (see \tableref{tab:prior-eval-metric} for a list) is capable of doing so. The paraphrase in the second row highlights \model's ability to make significant changes to original texts with a high lexical diversity code while preserving their semantic meaning.

\vspace{0.05in}

\noindent\textbf{Shortcomings:} One shortcoming of \model~is that, when the original text contains unique proper nouns, such as \textit{Homing Attack} and \textit{Slide Attack} in row 3, applying \model\ with a high lexical code can modify such nouns. This causes an undesirable semantic drift, as the original proper nouns must be maintained in any valid paraphrase. However, this shortcoming can be mitigated by decreasing the lexical code as shown in \appendixref{appendix:HumanEval}. 

% Overall, our human study confirms that \model~is good at meaning preservation and introducing diversity. With help of the controllable lexical and reordering codes, \model\ produces paraphrases of high quality, which makes 





% \begin{table*}[t!]
% \small
% \begin{center}
% \begin{tabular}{ lrrrrrrrr } 
%  \toprule
%   & \multicolumn{2}{c}{\textsc{RankGen-XL-PG19}} & \multicolumn{3}{c}{\textsc{GPT2-XL} perplexity} & \multicolumn{3}{c}{\textsc{1-gram}} (with context tokens only)\\
%  \cmidrule(lr){2-3}  \cmidrule(lr){4-6} \cmidrule(lr){7-9}
%  Model & prefix-ctx & suffix-ctx & prefix-ctx & suffix-ctx & both-ctx & prefix-ctx & suffix-ctx & any-ctx \\
%  \midrule
%  \multicolumn{2}{l}{(\emph{Decode $p$ = 0.75})} \vspace{0.15cm}\\
%  \model-ctx & \textbf{68}\% {\scriptsize (9.0)} & \textbf{70}\% {\scriptsize (8.7)} & \textbf{73}\% {\scriptsize (23.7)} & \textbf{79}\% {\scriptsize (29.8)} & \textbf{77}\% {\scriptsize (30.0)} & \textbf{41}\% {\scriptsize{(3.8)}} & \textbf{40}\% {\scriptsize{(3.8)}} & \textbf{52}\% {\scriptsize{(6.4)}} \\
%  \model-no-ctx & 31\% {\scriptsize (6.4)} & 29\% {\scriptsize (6.6)} & 26\% {\scriptsize (37.0)} & 20\% {\scriptsize (35.0)} & 22\% {\scriptsize (34.5)} & 23\% {\scriptsize (2.4)} & 24\% {\scriptsize{(2.4)}} & 27\% {\scriptsize{(4.1)}} \\
%  tie & 1\% \phantom{{\scriptsize (4.4)}} & 1\% \phantom{{\scriptsize (4.4)}} & 1\% \phantom{{\scriptsize (14.4)}} & 1\% \phantom{{\scriptsize (14.4)}} & 1\% \phantom{{\scriptsize (14.4)}} & 36\% \phantom{{\scriptsize (4.4)}} & 36\% \phantom{{\scriptsize (4.4)}} & 21\% \phantom{{\scriptsize (4.4)}}  \\ 
%  \midrule
%  \multicolumn{2}{l}{(\emph{Decode $p$ = 0.90})} \vspace{0.15cm} \\
%  \model-ctx & \textbf{69}\% {\scriptsize (8.8)} & \textbf{70}\% {\scriptsize (8.5)} & \textbf{73}\% {\scriptsize (26.4)} & \textbf{80}\% {\scriptsize (30.0)} & \textbf{78}\% {\scriptsize (30.9)} & \textbf{46}\% {\scriptsize (4.4)} & \textbf{44}\% {\scriptsize (4.3)} & \textbf{55}\% {\scriptsize (7.4)} \\
%  \model-no-ctx & 31\% {\scriptsize (6.1)} & 30\% {\scriptsize (6.3)} & 27\% {\scriptsize (40.8)} & 20\% {\scriptsize (35.3)} & 22\% {\scriptsize (35.7)} & 27\% {\scriptsize (2.9)} & 27\% {\scriptsize (3.0)} & 30\% (\scriptsize{4.9}) \\
%  tie & 0\% \phantom{{\scriptsize (8.8)}} & 0\% \phantom{{\scriptsize (8.8)}} & 0\% \phantom{{\scriptsize (18.8)}} & 0\% \phantom{{\scriptsize (18.8)}} & 0\% \phantom{{\scriptsize (18.8)}} & 27\% \phantom{{\scriptsize (2.9)}} & 29\% \phantom{{\scriptsize (4.4)}}  & 15\% \phantom{{\scriptsize (4.4)}} \\
% \bottomrule
% \end{tabular}
% \end{center}
% \vspace{-0.1in}
% \caption{Intrinsic paraphrase generation quality. Results show that paraphrasing with context is more effective than paraphrasing without context.\kkcomment{todo, measuring usage of context, add discussion on higher code giving higher performance} \kkcomment{use GPT-3 perplexity} \kkcomment{does rankgen prefer original vs paraphrase?}}
% \vspace{-0.1in}
% \label{tab:context-vs-no-context}
% \end{table*}

% \begin{table*}[t!]
% \small
% \begin{center}
% \begin{tabular}{ lrrrrrrrrrrr } 
%  \toprule
%  Detector $\rightarrow$ & & \multicolumn{2}{c}{Watermarks} & \multicolumn{2}{c}{DetectGPT}  & \multicolumn{2}{c}{OpenAI} & \multicolumn{2}{c}{GPTZero} & \multicolumn{2}{c}{RankGen}  \\
%  & & \multicolumn{2}{c}{\shortcite{kirchenbauer2023watermark}} & \multicolumn{2}{c}{\shortcite{mitchell2023detectgpt}}  & \multicolumn{2}{c}{\shortcite{AITextClassifier}} & \multicolumn{2}{c}{\shortcite{GPTZero}} & \multicolumn{2}{c}{\shortcite{krishna-etal-2022-rankgen}}\\

%   \cmidrule(lr){3-4} \cmidrule(lr){5-6} \cmidrule(lr){7-8} \cmidrule(lr){9-10} \cmidrule(lr){11-12}
%  Metric $\rightarrow$ & S $\uparrow$ & Acc $\downarrow$ & S\&A $\uparrow$ & Acc $\downarrow$ & S\&A $\uparrow$ & Acc $\downarrow$ & S\&A $\uparrow$ & Acc $\downarrow$ & S\&A $\uparrow$ & Acc $\downarrow$ & S\&A $\uparrow$ \\
 
%  \midrule
%  \multicolumn{9}{l}{\emph{Wikipedia prompts (300 tokens generated with $p=0.9$)}}\vspace{0.15cm} \\
%  GPT2-1.5B  & 100.0 & 100.0 & 0.0 & 53.1 & 46.9 & 18.9 & 81.1 & 13.9 & \textbf{86.1} & 14.1 & 85.9 \\
%  % sim scores = , , 99.5
%  + \model\ 20L &  96.3 & 96.1 & 3.4 & 15.4 & 81.2 & 12.0  & 87.6 & 9.1 & 85.5 \\
%  % sim scores = , , 97.4
%  + \model\ 40L  & 92.3 & 80.3 & 17.9 & 6.9 & \textbf{85.7} & 9.6 & \textbf{88.1}  & 7.3 & 83.6  \\
%   % sim scores = 89.6, , 90.5, 80.9
%  + \model\ 60L & 82.6 & \textbf{64.3} & \textbf{33.6} & \textbf{4.7} & 78.2 & \textbf{6.6} & 84.2 & \textbf{7.1} & 74.2 & 24.3 & 58.4 \\
%  \midrule
%  OPT-13B & 100.0 &  &  & 9.4 & 90.6 & 11.6 & 88.4 & 8.7 & 91.3 & \textbf{3.6} & \textbf{96.4} \\
%  + \model\ 20L & 96.2 &  &  & 1.7 & \textbf{94.7} & 8.6 & \textbf{91.0} & 5.4 & \textbf{91.4} & 7.0 & 87.5 \\
%  + \model\ 40L & 92.1  &  &  & 0.6 & 91.4 & 6.7 & 90.2 & \textbf{3.8} & 89.6 & 9.8 & 79.6 \\
%  + \model\ 60L & 82.1 &  &  & \textbf{0.3} & 81.7 & \textbf{5.2} & 84.5 & 6.3 & 75.5 & 12.6 & 66.9  \\
%  \midrule 
%  GPT-3.5-175B \\
%  davinci-003  & 100.0 & -&- & 2.0* & \textbf{98.0}* & 39.1 & 60.9 & \textbf{5.6} & \textbf{94.4} & \textbf{1.4} & \textbf{98.6}\\
%  + \model\ 20L & 95.6 & -&- & 0.5* & 93.5* & 27.7 & 70.1 & 2.0 & 93.2 & 2.3 & 93.9 \\
%  + \model\ 40L & 93.2 & -&- & 0.0* & 92.5* & 22.1 & 74.6 & 2.0 & 90.0 & 3.4 & 89.1 \\
%  + \model\ 60L & 88.1 & -&- & 0.5* & 88.0* & \textbf{15.9} & \textbf{75.4}& 3.6  & 82.8 & 4.8 & 80.6  \\
%  \midrule
%  % 1.7
%  Human Text & - & 1.0 & - & 1.0 & - & 1.0 & - & 1.0 & - & 1.0 & - \\
% \bottomrule
% \end{tabular}
% \end{center}
% \caption{Performance of various AI detection algorithms at a 1\% false positive rate before and after \model\ paraphrasing. As the lexical diversity (L) increases, detection rate (Acc) decreases across algorithms, at a small cost of semantic similarity (S).\kkcomment{todo - get more data points for GPTZero}}
% \label{tab:watermark-attacks}
% \end{table*}
\begin{table}[!t]\begin{center}
\caption{\textbf{Analysis of offset mechanisms in 360Attention and backbone variants} on 360BEV-Matterport dataset.}
\vskip -1ex
\label{tab:analysis}
\setlength{\tabcolsep}{1mm}
\renewcommand{\arraystretch}{1.2}
\resizebox{\columnwidth}{!}{
    \begin{tabular}{ l l | c | c | l}
    \toprule[1pt]
    \textbf{Methods} & \textbf{Backbone} & \textbf{\#Param} & \textbf{FLOPs} & \textbf{mIoU} \\ \midrule\midrule
    
    \circled{1} Ours (360Attention offset) & MiT-B0 & 04.60M  & 248.57G & 36.98     \\
    \circled{2} Ours (360Attention offset) & MiT-B2 & 26.30M & 283.94G & 44.32 \\ 
    \circled{3} Ours (360Attention offset) & MiT-B4  & 62.91M & 341.34G &  \textbf{45.53}    \\  \midrule
    \circled{4} Ours (Multi-scale offset) & MiT-B2  & 26.43M  &284.17G &43.65~\obf{-0.67}   \\
    \circled{5} Ours (Fixed-range offset) & MiT-B2  & 26.30M & 283.44G &  43.28~\obf{-1.04}\\
    \circled{6} Ours (Separate offset) & MiT-B2 & 26.19M & 279.18G &  42.82~\obf{-1.50}\\\midrule
    \circled{7} Ours (360Attention offset) & MSCA-B  & 27.69M &274.59G & \textbf{46.31}~\gbf{+1.99} \\ 

    \bottomrule
    \end{tabular}
}
\end{center}
\vskip -4ex
\end{table}
\section{Conclusion}\label{sec:conclusion}
In this work, we focus on addressing the fundamental challenge of OOD detection tasks, which is how to fully understand the semantic discrepancy between the ID/OOD samples. We reveal that the key to success in the realistic SCOOD task is to allocate as many ID samples in the unlabeled set correctly as possible. To this end, we propose a novel uncertainty-aware optimal transport scheme that introduces class-specific energy scores as guidance for effective label assignment. Experimental results show that our method achieves better performance than previous state-of-the-art methods on SCOOD benchmarks.

\textbf{Limitations.} In addition to temperature scaling, other techniques such as feature clipping applied in ReAct~\cite{sun2021react} also enhance the performance of energy score, so how to obtain an OOD score that best fits the SCOOD task can be further explored. Moreover, a setting highly related to SCOOD has been proposed in \cite{katz2022training} and formulated as a constrained optimization problem. We will also theoretically analyze these practical OOD settings in our feature work.

% \section*{Acknowledgments}
\textbf{Acknowledgments.} 
This work is supported by National Key R\&D Program of China under Grant 2020AAA0105701, National Natural Science Foundation of China (NSFC) under Grants 61872327, Major Special Science and Technology Project of Anhui, National Natural Science Foundation of China (62033012) and Ant Group through Ant Research Intern Program.

\chapter*{Acknowledgement}
\addcontentsline{toc}{chapter}{Acknowledgement}
The authors thank Andrzej Kupsc, Sergey Barsuk, Olivier Callot and Wolfgang K{\"u}hn for their contribution on the CDR draft.
%The authors thank the international review committee XXX for their great effort in reading the CDR draft and providing valuable suggestions. 
The STCF working group thanks all 
the colleagues in the world-wide community for many profitable discussions
and expresses gratitude to the Hefei Comprehensive National Science Center for their strong support.  This work is supported by: international 
partnership program of the Chinese Academy of Sciences Grant No. 211134KYSB20200057.

% Entries for the entire Anthology, followed by custom entries
\bibliography{anthology,custom,bib/journal-full,bib/arr2022,bib/paraphrase}
% options for bibliographystyle: plain ([1]), https://www.overleaf.com/learn/latex/Natbib_bibliography_styles
% \bibliographystyle{rusnat}
\bibliographystyle{abbrvnat}
%\bibliographystyle{plain}
\newpage
\appendix

\section{Appendix for Proofs}

\paragraph{Proof of Theorem \ref{thm:main}.}

\begin{proof}
\label{proof:main}
Our proof has two steps. In Step 1, we will show that SimCLR is equivalent to minimizing the cross entropy loss defined in Eqn.~(\ref{eqn:cross-entropy}). 
In Step 2, we will show  that minimizing the cross-entropy loss 
is equivalent to spectral clustering on $\bfpi$. 
Combining the two steps together, we have proved our theorem. 

\textbf{Step 1: } SimCLR is equivalent to minimizing the cross entropy loss.

The cross-entropy loss takes expectation over 
$\bfW_\bfX\sim \mathbb{P}(\cdot ; \bfpi)$, 
which means $\bfW_\bfX$ has exactly one non-zero entry in each row $i$. By Lemma~\ref{lem:multinomial}, we know every row $i$ of $\bfW_\bfX$ is independent of other rows. Moreover, 
$\bfW_{\bfX,i}\sim \mathcal{M}(1, \bfpi_i/\sum_j \bfpi_{i,j})=\mathcal{M}(1, \bfpi_i)$, because $\bfpi_i$ itself is a probability distribution.
Similarly, we know $\bfW_\bfZ$ also has the row-independent property by sampling over $\mathbb{P}(\cdot;\bfK_\bfZ)$.
Therefore, by Lemma~\ref{lem:cross_split}, we know Eqn.~(\ref{eqn:cross-entropy}) is equivalent to:
\[
 -\sum_{i=1}^n \mathbb{E}_{\bfW_{\bfX,i}}[\log \mathbb{P}(\bfW_{\bfZ,i}=\bfW_{\bfX,i};\bfK_\bfZ)],
\]

This expression takes expectation over $\bfW_{\bfX,i}$ for the given row $i$. Notice that 
$\bfW_{\bfX,i}$ has exactly one non-zero entry, which equals $1$ (same for $\bfW_{\bfZ,i}$). 
As a result
we expand the above expression to be:
\begin{equation}
 -\sum_{i=1}^n \sum_{j\neq i} \Pr(\bfW_{\bfX,i,j}=1)\log \Pr(\bfW_{\bfZ,i,j}=1).
\label{eqn:detailed-expansion}    
\end{equation}


By Lemma~\ref{lem:multinomial}, $\Pr(\bfW_{\bfZ,i,j}=1)=\bfK_{\bfZ,i,j}/\|\bfK_{\bfZ,i}\|_1$ for $j\neq i$. Recall that $\bfK_\bfZ=(k(\bfZ_i-\bfZ_j))_{(i,j)\in[n]^2}$, which means 
$\bfK_{\bfZ,i,j}/\|\bfK_{\bfZ,i}\|_1=\frac{\exp(-\|\bfZ_i-\bfZ_j\|^2/{2\tau})}{\sum_{k\neq i}
\exp(-\|\bfZ_i-\bfZ_k\|^2/{2\tau})
}$ for $j\neq i$, when $k$ is the Gaussian kernel with variance $\tau$. 

Notice that $\bfZ_i=f(\bfX_i)$, so we know
\begin{equation}
-\log \Pr(\bfW_{\bfZ,i,j}=1)=
-\log \frac{\exp(-\|f(\bfX_i)-f(\bfX_j)\|^2/{2\tau})}{\sum_{k\neq i}
\exp(-\|f(\bfX_i)-f(\bfX_k)\|^2/{2\tau}),
}
\label{eqn:infonce-equivalence}    
\end{equation}


The right hand side is exactly the InfoNCE loss defined in Eqn.~(\ref{eqn:infonce}).
Inserting Eqn.~(\ref{eqn:infonce-equivalence}) into Eqn.~(\ref{eqn:detailed-expansion}), we get the SimCLR algorithm, which first samples augmentation pairs $(i,j)$ with $\Pr(\bfW_{\bfX,i,j}=1)$ for each row $i$, and then optimize the InfoNCE loss. 

\textbf{Step 2: } minimizing the cross entropy loss 
is equivalent to spectral clustering on $\bfpi$.


By Lemma~\ref{lem:convert_to_spectral}, we may further convert the loss to 
\begin{equation}
\label{eqn:main-theorem-repul-attr}
\min_{\bfZ}
-\sum_{(i,j)\in [n]^2} \mathbf{P}_{i,j}
\log k (\bfZ_i-\bfZ_j)+\log \mathbf{R}(\bfZ).
\end{equation}
Since $k$ is the Gaussian kernel, this reduces to \[
\min_\bfZ \mathrm{tr}(\bfZ^\top \mathbf{L}(\bfpi) \bfZ)
+\log \mathbf{R}(\bfZ),
\]

where we use the fact that $\mathbb{E}_{\bfW_\bfX\sim \mathbb{P}(\cdot; \bfpi)}[\mathbf{L}(\bfW_\bfX)]
=\mathbf{L}(\bfpi)
$, because the Laplacian operator is linear and $
\mathbb{E}_{\bfW_\bfX\sim \mathbb{P}(\cdot; \bfpi)}(\bfW_\bfX)=\bfpi
$.
\end{proof}

\paragraph{Proof of Theorem \ref{thm:clip}.}
\begin{proof}
Since $\bfW_\bfX\sim \mathbb{P}(\cdot;\bfpi_{\mathbf{A}, \mathbf{B}})$, we know 
$\bfW_\bfX$ has exactly one non-zero entry in each row, denoting the pair that got sampled. 
A notable difference compared to the previous proof is we now have $n_\mathcal{A}+n_\mathcal{B}$ objects in our graph. CLIP deals with this by taking a mini-batch of size $2N$, 
such that $n_\mathcal{A}=n_\mathcal{B}=N$, and adding the $2N$ InfoNCE losses together. We label the objects in $\mathcal{A}$ as $[n_\mathcal{A}]$, and the objects in $\mathcal{B}$ as $\{n_\mathcal{A}+1, \cdots, n_\mathcal{A}+n_\mathcal{B}\}$. 

Notice that $\bfpi_{\mathbf{A}, \mathbf{B}}$ is a bipartite graph, so the edges of objects in $\mathcal{A}$ will only connect to object in $\mathcal{B}$ and vice versa. We can define the similarity matrix in $\cZ$ as $\bfK_\bfZ$, 
where $\bfK_\bfZ(i, j+n_\mathcal{A})=\bfK_\bfZ(j+n_\mathcal{A},i)= k(\bfZ_i-\bfZ_j)$ for $i\in [n_\mathcal{A}], j\in [n_\mathcal{B}]$, and otherwise we set $\bfK_\bfZ(i,j)=0$. 
The rest is same as the previous proof. 
\end{proof}

\paragraph{Proof of Theorem \ref{thm:exponential}.}

\begin{proof}
\label{proof:exponential}
Since the objective function consists of a linear term combined with an entropy regularization, which is a strongly concave function, the maximization problem is a convex optimization problem. Owing to the implicit constraints provided by the entropy function, the problem is equivalent to having only the equality constraint. We then introduce the Lagrangian multiplier $\lambda$ and obtain the following relaxed problem:

$$
\widetilde{E}(\boldsymbol{\alpha})=\psi_{1}-\sum_{i=1}^n \alpha_{i} \psi_{i}+\tau \sum_{i=1}^n \alpha_{i}\log \alpha_{i}+\lambda\left(\boldsymbol{\alpha}^{\top} \mathbf{1}_n-1\right).
$$

As the relaxed problem is unconstrained, taking the derivative with respect to $\alpha_{i}$ yields

$$
\frac{\partial \widetilde{E}(\boldsymbol{\alpha})}{\partial \alpha_{i}}=-\psi_{i}+\tau\left(\log \alpha_{i}+\alpha_{i} \frac{1}{\alpha_{i}}\right)+\lambda=0.
$$

Solving the above equation implies that $\alpha_{i}$ takes the form
$
\alpha_{i}=\exp \left(\frac{1}{\tau} \psi_{i}\right) \exp \left(\frac{-\lambda}{\tau}-1\right).
$ Since $\alpha_{i}$ lies on the probability simplex, the optimal $\alpha_{i}$ is explicitly given by
$
\alpha^{*}_{i}=\frac{\exp \left(\frac{1}{\tau} \psi_{i}\right)}{\sum_{i^{\prime}=1}^n \exp \left(\frac{1}{\tau} \psi_{i^{\prime}}\right)} .
$ Substituting the optimal point into the objective function, we obtain
$$
\begin{aligned}
E\left(\boldsymbol{\alpha}^*\right)  &=\psi_1-\sum_{i=1}^n \frac{\exp \left(\frac{1}{\tau} \psi_{i}\right)}{\sum_{i^{\prime}=1}^n \exp \left(\frac{1}{\tau} \psi_{i^{\prime}}\right)} \psi_{i}+\tau \sum_{i=1}^n \frac{\exp \left(\frac{1}{\tau} \psi_{i}\right)}{\sum_{i^{\prime}=1}^n \exp \left(\frac{1}{\tau} \psi_{i^{\prime}}\right)}\log \frac{\exp \left(\frac{1}{\tau} \psi_{i}\right)}{\sum_{i^{\prime}=1}^n \exp \left(\frac{1}{\tau} \psi_{i^{\prime}}\right)} \\
& =\psi_1 - \tau \log \left(\sum_{i=1}^n \exp \left(\frac{1}{\tau} \psi_{i}\right)\right).
\end{aligned}
$$
Thus, the Lagrangian dual function is given by
\begin{equation*}
-E\left(\boldsymbol{\alpha}^*\right)= -\tau \log \frac{\exp \left(\frac{1}{\tau} \psi_{1}\right)}{\sum_{i=1}^n \exp \left(\frac{1}{\tau} \psi_{i}\right)}.\qedhere
\end{equation*}
\end{proof}



\section{More on Experiments} \label{section: experiment_details}

\paragraph{CIFAR-10 and CIFAR-100} CIFAR-10 ~\citep{krizhevsky2009learning} and CIFAR-100 ~\citep{krizhevsky2009learning} are well-known classic image classification datasets. Both CIFAR-10 and CIFAR-100 contain a total of 60k $32 \times 32$ labeled images of different classes, with 50k for training and 10k for testing. CIFAR-10 is similar to CIFAR-100, except there are 10 different classes in CIFAR-10 and 100 classes in CIFAR-100.

\paragraph{TinyImageNet} TinyImageNet ~\citep{le2015tiny} is a subset of ImageNet ~\citep{deng2009imagenet}. There are 200 different object classes in TinyImageNet, with 500 training images, 50 validation images, and 50 test images for each class. All the images in TinyImageNet are colored and labeled with a size of $64 \times 64$.

\textbf{Pseudo-code.} Algorithm \ref{alg:Training Procedure} presents the pseudo-code for our empirical training procedure.

\begin{algorithm}[!htbp]
\caption{Training Procedure}
\label{alg:Training Procedure}
\begin{algorithmic}[1]
\REQUIRE trainable encoder network $f$, batch size $N$, augmentation strategy \textit{aug}, loss function $L$ with hyperparameters \textit{args}
\FOR {sampled minibatch ${x_i}_{i=1}^N$}
\FORALL{$i \in { 1, ..., N }$}
\STATE draw two augmentations $t_i = \textit{aug}\left(x_i\right) $, $t_i' = \textit{aug}\left(x_i\right) $
\STATE $z_i = f\left(t_i\right)$, $z_i' = f\left(t_i'\right)$
\ENDFOR
\STATE compute loss $\mathcal{L} = L(N, z, z', \textit{args})$
\STATE update encoder network $f$ to minimize $\mathcal{L}$
\ENDFOR
\STATE \textbf{Return} encoder network $f$
\end{algorithmic}
\end{algorithm}

We also provide the pseudo-code for our core loss function used in the training procedure in Algorithm \ref{alg:Core loss}. The pseudo-code is almost identical to SimCLR's loss function, with the exception of an extra parameter $\gamma$.

\begin{algorithm}[!htbp]
\caption{Core loss function $\mathcal{C}$}
\label{alg:Core loss}
\begin{algorithmic}[1]
\REQUIRE batch size $N$, two encoded minibatches $z_1, z_2$, $\gamma$, temperature $\tau$
\STATE $z = \textit{concat}\left(z_1, z_2\right)$
\FOR {$i \in {1, ..., 2N }, j \in {1, ..., 2N}$ }
\STATE $s_{i,j} = \Vert z_i - z_j \Vert_2^{\gamma}$
\ENDFOR
\STATE \textbf{define} $l(i, j)$ \textbf{as} $l(i, j) = - \log \frac{exp\left(s_{i,j}/\tau \right)}{\sum_{k=1}^{2N} \mathbf{1}{[k \ne i]} exp\left(s{i, j} / \tau \right)} $
\STATE \textbf{Return} $\frac{1}{2N} \sum_{k=1}^N\left[l(i, i+N) + l(i+N, i)\right]$
\end{algorithmic}
\end{algorithm}

Utilizing the core loss function $\mathcal{C}$, we can define all kernel loss functions used in our experiments in Table \ref{table: loss definition}. For all $z_i \in z$ with even dimensions $n$, we define $z_{L_i} = z_i\left[0:n/2\right]$ and $z_{R_i} = z_i\left[n/2:n\right]$.

\begin{table}[ht]
\centering
\begin{tabular}{{@{}l|l@{}}}
Kernel  &  Loss function \\ \midrule
Laplacian & $\mathcal{C}\left(N, z, z', \gamma=1, \tau\right)$\\ \midrule
Sum       & $\lambda * \mathcal{C}\left(N, z, z', \gamma=1, \tau_1\right) + (1-\lambda) * \mathcal{C}\left(N, z, z', \gamma=2, \tau_2\right)$  \\ \midrule
Concatenation Sum&$\lambda * \mathcal{C}\left(N, z_L, z'_L, \gamma=1, \tau_1\right) + (1-\lambda) * \mathcal{C}\left(N, z_R, z'_R, \gamma=2, \tau_2\right)$\\ \midrule
$\gamma = 0.5$ & $\mathcal{C}\left(N, z, z', \gamma=0.5, \tau\right)$          \\ 

\end{tabular}

\caption{Definition of kernel loss functions in our experiments}
\label {table: loss definition}
\end{table}

\textbf{Baselines.} We reproduce the SimCLR algorithm using PyTorch Lightning~\citep{PytorchLightning}.

\textbf{Encoder details.}
The encoder $f$ consists of a backbone network and a projection network. We employ ResNet50~\citep{ResNet} as the backbone and a 2-layer MLP (connected by a batch normalization~\citep{ioffe2015batch} layer and a ReLU \cite{nair2010rectified} layer) with hidden dimensions 2048 and output dimensions 128 (or 256 in the concatenation kernel case).

\textbf{Encoder hyperparameter tuning.}
For each encoder training case, we randomly sample 500 hyperparameter groups (sample details are shown in Table \ref{table: Hyperparameter sample}) and train these samples simultaneously using Ray Tune ~\citep{RayTune}, with the ASHA scheduler~\citep{li2018massively}. Ultimately, the hyperparameter group that maximizes the online validation accuracy (integrated in PyTorch Lightning) within 5000 validation steps is chosen for the given encoder training case.

\begin{table}[ht]
\centering

\begin{tabular}{@{}l|l|l@{}}
\midrule
Hyperparameter  & Sample Range & Sample Strategy \\ \midrule
start learning rate & $\left[10^{-2}, 10\right]$ & log uniform \\ \midrule
$\lambda$       & $\left[0, 1\right]$ & uniform \\ \midrule
$\tau$, $\tau_1$, $\tau_2$ & $\left[0, 1\right]$ & log uniform \\ \midrule
\end{tabular}

\caption{Hyperparameters sample strategy}
\label {table: Hyperparameter sample}
\end{table}

\textbf{Encoder training.} 
We train each encoder using the LARS optimizer~\citep{LARSOptimizer}, LambdaLR Scheduler in PyTorch, momentum 0.9, weight decay $10^{-6}$, batch size 256, and the aforementioned hyperparameters for 400 epochs on a single A-100 GPU.

\textbf{Image transformation.} The image transformation strategy, including augmentation, is identical to the default transformation strategy provided by PyTorch Lightning.

\textbf{Linear evaluation.}
The linear head is trained using the SGD optimizer with a cosine learning rate scheduler, batch size 64, and weight decay $10^{-6}$ for 100 epochs. The learning rate starts at $0.3$ and ends at $0$.

\textbf{Moco Experiments.} We also tested our method based on MoCo~\citep{he2019moco}. The results are summarized in Table \ref{tab:results-moco}. Here we choose ResNet18~\citep{ResNet} as the backbone and set a temperature of $0.1$ as default. For our simple sum kernel, we set $\lambda=0.8$. The results show that our method outperforms the original MoCo method.

\begin{table}[thb]
\centering
\caption{MoCo Experiment Results on CIFAR-10 and CIFAR-100.}
\label{tab:results-moco}
\resizebox{\textwidth}{!}{%
\begin{tabular}{@{}c|ccc|ccc@{}}
\toprule
\multirow{3}{*}{Method} & \multicolumn{3}{c|}{CIFAR-10} & \multicolumn{3}{c}{CIFAR-100} \\ \cmidrule(lr){2-4} \cmidrule(lr){5-7} 
                        & 200 epochs & 400 epochs    & 1000 epochs   & 200 epochs & 400 epochs & 1000 epochs         \\ \midrule
MoCo (repro.)         & $76.41 \pm 0.12$    & $80.01 \pm 0.15$          & $84.45 \pm 0.08$    & $\mathbf{47.02 \pm 0.11}$ & $52.50 \pm 0.07$ & $57.62 \pm 0.15$            \\
\midrule
Laplacian Kernel        & ${78.09 \pm 0.10}$    & $\mathbf{83.85 \pm 0.09}$          & $\mathbf{88.34 \pm 0.16}$    & $46.12 \pm 0.22$   & $53.44 \pm 0.17$ & $59.10 \pm 0.14$        \\
Simple Sum Kernel & $\mathbf{78.12 \pm 0.15}$   & $83.23 \pm 0.18$ & $87.50 \pm 0.20$ & $46.65 \pm 0.06$ & $\mathbf{53.62 \pm 0.19}$ & $\mathbf{59.83 \pm 0.12}$\\
\bottomrule
\end{tabular}
}
\end{table}



\section{More Experiments on Synthetic Data}


Consider a scenario with $n$ clusters, each containing $k$ vertices. Let the probability of vertices $u$ and $v$ from the same cluster belonging to $\bfpi$ be $p$. Conversely, for vertices $u$ and $v$ from different clusters, let the probability of belonging to $\pi$ be $q$. We generate the graph $\bfpi$ randomly, based on $p$ and $q$. We experiment with values of $k=100$ and $n=6$ for ease of visualization, embedding all points in a two-dimensional space. Each vertex's initial position originates from a normal distribution. In each iteration, we sample a subgraph of $\bfpi$ uniformly, ensuring each vertex has an out-degree of $1$. We then optimize the corresponding vectors using InfoNCE loss with an SGD optimizer and iterate until convergence. Our experimental setup consists of an SGD learning rate of $1$, an InfoNCE loss temperature of $0.5$, and a batch size of $50$. We evaluate two scenarios with different $p$ and $q$ values: $p=1$, $q=0$, and $p=0.75$, $q=0.2$. The results of these experiments are visualized in Figure \ref{fig:vis-spectral-cluster}. The obtained embeddings exhibit the hallmark pattern of spectral clustering of graph $\bfpi$.

\begin{figure}[!tb]
\centering
\subfigure{
\includegraphics[width=1\textwidth]{Figures/cluster_pi.png}
\label{fig:vis-cluster}
}
\subfigure{
\includegraphics[width=1\textwidth]{Figures/noised_cluster_pi.png}
\label{fig:vis-noised-cluster}
}
\caption{Visualizations of the optimization process using InfoNCE Loss on the vectors corresponding to $\bfpi$. Points of identical color belong to the same cluster within $\bfpi$. To showcase the internal structure of $\bfpi$, we randomly select 10 vertices from each cluster to display the edge distribution of $\bfpi$.}
\label{fig:vis-spectral-cluster}
\end{figure}


\end{document}
