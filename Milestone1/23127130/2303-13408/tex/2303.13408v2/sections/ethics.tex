\section{Ethical Considerations}
\label{appendix:ethical}

Our goal in this paper is not to provide a recipe for potential attackers (e.g., college students wishing to use ChatGPT in their essays) to evade AI text detection systems. Rather, we wish to bring awareness to the wider community about the vulnerabilities of current AI-generated text detectors to simple paraphrase attacks. These detectors are not useful in their current state given how easy they are to evade. We encourage the research community to stress test their detectors against paraphrases, and to develop new detectors which are robust against these attacks. To facilitate such research, we open source our paraphraser and associated data / code.

Furthermore, we propose not just an attack but also a potentially strong defense against this attack. Our detection strategy is simple, relying on retrieval over a corpus of previously-generated sequences. We empirically show that such a detection algorithm could work at scale and provide extensive discussion on possible methods to improve performance (\appendixref{sec:suggestions-retrieval-scale}), as well as discussing possible limitations and  approaches to tackling them (\appendixref{sec:limitations-retrieval}). We hope that retrieval-based AI-generated text detectors rapidly improve and are eventually deployed in conjunction with other detection methods like watermarking / classifiers.
