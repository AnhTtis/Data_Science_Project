\usepackage[utf8]{inputenc}
\usepackage[english]{babel}



\usepackage{lmodern}
\usepackage{amssymb}
\usepackage{amsmath}
\usepackage{mathtools}
\usepackage{enumitem}
\usepackage{booktabs}
\usepackage{subcaption}


\usepackage[svgnames]{xcolor}
\usepackage{tikz}
\usetikzlibrary{cd, backgrounds, patterns}
\usepackage{forest}
% \usepackage[colorinlistoftodos]{todonotes}

\tikzcdset{scale cd/.style={every label/.append style={scale=#1},
      cells={nodes={scale=#1}}}}

\usepackage[autostyle]{csquotes}
\usepackage[style=alphabetic, sorting=nyt, maxnames=10, maxalphanames=5]{biblatex}
\addbibresource{data.bib}
\RenewDocumentCommand{\bibfont}{}{\normalfont\small}

% https://tex.stackexchange.com/a/57622/14965
\let\oldmultinamedelim\multinamedelim
\let\oldfinalnamedelim\finalnamedelim
\RenewDocumentCommand{\multinamedelim}{}{--}
\RenewDocumentCommand{\finalnamedelim}{}{--}
\AtBeginBibliography{
  \RenewDocumentCommand{\multinamedelim}{}{\oldmultinamedelim}
  \RenewDocumentCommand{\finalnamedelim}{}{\oldfinalnamedelim}
}

% \NewDocumentCommand{\renato}{sO{}m}{\todo[\IfBooleanT{#1}{inline}, #2, color=green]{R: #3}}
% \NewDocumentCommand{\najib}{sO{}m}{\todo[\IfBooleanT{#1}{inline}, #2, color=LightSkyBlue]{N: #3}}

\usepackage{hyperref}
\hypersetup{colorlinks, urlcolor=blue!65!black, citecolor=green!65!black, linkcolor=red!65!black}
\usepackage{amsthm, thmtools}

\numberwithin{equation}{section}
\declaretheorem[sibling = equation]{theorem}
\declaretheorem[sibling = equation]{proposition}
\declaretheorem[sibling = equation]{corollary}
\declaretheorem[sibling = equation]{lemma}
\declaretheorem[sibling = equation]{conjecture}
\declaretheorem[sibling = equation, style = definition]{definition}
\declaretheorem[sibling = equation, style = definition]{convention}
\declaretheorem[sibling = equation, style = definition]{notation}
\declaretheorem[sibling = equation, style = remark]{example}
\declaretheorem[sibling = equation, style = remark]{remark}

\declaretheorem[name = Theorem]{theoremIntro}
\RenewDocumentCommand{\thetheoremIntro}{}{\Alph{theoremIntro}}
\declaretheorem[numbered = no, name = Remark, style = remark]{remarkIntro}

% Operads & operators
\DeclareMathOperator{\id}{id}
\DeclareMathOperator{\pr}{pr}
\DeclareMathOperator{\sgn}{sgn}
\DeclareMathOperator{\Hom}{Hom}
\DeclareMathOperator{\Conf}{Conf}

\NewDocumentCommand{\opP}{}{\mathcal{P}}
\NewDocumentCommand{\opQ}{}{\mathcal{Q}}
\NewDocumentCommand{\D}{}{\mathcal{D}}
\NewDocumentCommand{\C}{}{\mathcal{C}}
\NewDocumentCommand{\SC}{}{\mathcal{SC}}
\NewDocumentCommand{\vor}{}{\mathrm{vor}}
\NewDocumentCommand{\SCv}{}{\SC^{\vor}}

\NewDocumentCommand{\catC}{}{\mathsf{C}}
\NewDocumentCommand{\Top}{}{\mathsf{Top}}
\NewDocumentCommand{\Ch}{}{\mathsf{Ch}}

% Colors
\NewDocumentCommand{\colc}{}{\mathfrak{c}}
\NewDocumentCommand{\colo}{}{\mathfrak{o}}

% Fields
\NewDocumentCommand{\R}{}{\mathbb{R}}
\NewDocumentCommand{\Q}{}{\mathbb{Q}}
\NewDocumentCommand{\K}{}{\mathbb{K}}
\NewDocumentCommand{\Z}{}{\mathbb{Z}}
\NewDocumentCommand{\cK}{}{\mathcal{K}}
\RenewDocumentCommand{\S}{}{\mathbb{S}}

% iota, beta, etc. Arguments:
% 1. the symbol (alpha, beta...), mandatory
% 2. a star (optional), puts a hat on the operator
% 3. a dash (optional), puts a bar on the operator
% 4. optional subscript
% 5. mandatory exponent; if it is the single letter "Q", then replace by \mathcal{Q}
\NewDocumentCommand{\OurOperators}{m s t- o m}{{
\IfBooleanTF{#2}{\hat{#1}}{
  \IfBooleanTF{#3}{\bar{#1}}{
    #1}}
^{\ifx#5Q \mathcal{Q} \else #5 \fi}
\IfValueT{#4}{_{#4}}
}}

\def\pro{\OurOperators{\mu}}
\def\inc{\OurOperators{\iota}}
\def\act{\OurOperators{\alpha}}
\def\push{\OurOperators{\beta}}
\def\loo{\OurOperators{\ell}}
\def\spec{\OurOperators{\eta}}
\def\gam{\OurOperators{\gamma}}



\title{Non-formality of Voronov's Swiss-Cheese operads}
\author{Najib Idrissi\thanks{Université Paris Cité and Sorbonne Université, CNRS, IMJ-PRG, F-75013 Paris, France.} \and Renato Vasconcellos Vieira\thanks{Universidade de São Paulo, ICMC, São Carlos, Brasil.}}
\date{March 2023}
