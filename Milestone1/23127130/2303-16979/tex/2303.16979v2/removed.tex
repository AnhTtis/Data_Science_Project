\section{Singular chain complexes and particular elements}

I think we should choose the $N$-dimensional euclidean space spanned by:
$$\{\vec e_0, \vec e_1, \dots, \vec e_{N-1}\}.$$

Let $\S^0:=\{-, +\}$, $\underline{N-1} := \{1, \dots, N-1\}$ and let $\vec b_{\epsilon i}:=\epsilon\vec e_i$ where $\epsilon \in \{-, +\}$.

Consider $A_N := \S^0\times\underline{N-1}$ with all elements open. We equip it with the $\mathbb Z_{2N-2}$-action that ``rotates by a quarter turn'', i.e., it is generated by:
\begin{align*}
  \epsilon i\cdot 1:=\begin{cases}
                       \epsilon(i+1), & \text{if } 1\leq i\leq N-2; \\
                       -\epsilon 1,   & \text{if } i=N-1.
                     \end{cases}
\end{align*}
This action is regular (i.e., free and transitive), so that for all $\epsilon i, \epsilon'i'\in A$ there is a unique element $\epsilon' i'-\epsilon i\in\mathbb Z_{2N-2}$ such that $\epsilon i\cdot (\epsilon' i'-\epsilon i)=\epsilon'i'$. Note that $\epsilon i\cdot (N-1)=-\epsilon i$.

We are going to define several elements of $\SCv_N$.
Recall from Definition~\ref{def:scv} that an element $x \in \SCv_N(A)$ (where $A$ is some bicolored set) by its components $x_a \in D^N \times [0, 1]$ for $a \in A$, where the first coordinate of $x_a$ represents the center of a little (half-)disk and the second coordinate represents its radius.
Let us first define the element $a \in \SCv_N(\{\colc, \colo\})$ by:
$$
  a_\colc:= \Bigl(\frac{3}{4}\vec e_0, \frac{1}{4} \Bigr),
  \quad
  a_\colo := \Bigl( \vec 0, \frac{1}{2} \Bigr),
$$
i.e., we consider a half-disk of radius $1/4$ centered at the origin and a little disk of radius $1/4$ centered $3/4$ units above the origin.
Next, we define the element $m \in \SCv_N(A_N)$ by:
$$
  m_{\epsilon i} := \Bigl( \frac{3}{4}\vec b_{\epsilon i}, \frac{1}{4}\Bigr),
$$
i.e., we consider a configuration of $N-1$ pairs of half-disks, one for each coordinate axis of $\{0\} \times \R^{N-1}$, centered on both sides of the origin.
Finally, we define the $(N-1)$-chain $l : \mathbb{S}^{N-1} \to \D_N(\mathbb{S}^0)$ by:
$$
  l_\epsilon(\vec x) := \bigl(\epsilon\frac{1}{2}\vec x, \frac{1}{2}\bigr), \qquad \forall \vec x \in \mathbb{S}^{N-1}.
$$

Let $A^+:=\{ \mathfrak c \}\sqcup A$. For each $\langle \epsilon_i\rangle\in \Pi_{\underline{N-1}}\S^0$ let
\begin{gather*}
  \beta^{\langle \epsilon_i\rangle}\in C_{N-1}\SC_NA^+\\
  \beta^{\langle \epsilon_i\rangle}_{ \mathfrak c }\vec t:=
  \left(
  \frac{3}{4}
  \left(
    \frac{1+3t_0}{4}\vec e_0
    +
    \sum_{j=1}^{N-1}t_j\vec b_{\epsilon_i i}
    \right)
  , \frac{1}{4(4-3t_0)}\right)\\
  \beta^{\langle\epsilon_i\rangle}_{\epsilon ' i' }\vec t
  :=\begin{cases}\left(\frac{3}{4+4t_0}\vec b_{\epsilon 'i'} , \frac{1}{(1+t_{i'})(4+4t_{0})}\right),
     & \epsilon' =\epsilon_{i'}    \\
    \left(\frac{3}{4+4t_0}\vec b_{\epsilon' i'} , \frac{1}{4+4t_{0}}\right),
     & \epsilon'\neq \epsilon_{i'}
  \end{cases}
\end{gather*}


\begin{figure}
  \centering
  \begin{tikzpicture}
    \node at (0, 2.9) {\includegraphics[scale=0.125]{BetaSC100.png}};
    \node at (-3.25, -1.5) {\includegraphics[scale=0.125]{BetaSC010.png}};
    \node at (3.25, -1.5) {\includegraphics[scale=0.125]{BetaSC001.png}};
    \node at (0, 0) {\includegraphics[scale=0.125]{BetaSC1:2.png}};
    \draw[thick, ->] (-0.1, 1.6)--(-1.9, -0.9);
    \draw[thick, ->] (-1.8, -1)--(1.8, -1);
    \draw[thick, ->] (0.1, 1.6)--(1.9, -0.9);

    \node at (0, 1.8) {\footnotesize$\vec e_0$};
    \node at (-1.95, -1.05) {\footnotesize$\vec e_1$};
    \node at (1.95, -1.05) {\footnotesize$\vec e_2$};
  \end{tikzpicture}
  \caption{$\beta^{++}$ for $N=3$.}
  \label{fig:my_label}
\end{figure}

\renato*{The figure is just a suggestion. We can alter it if you have any ideas on how to improve the illustration.
  If you want to mess around with the geogebra code I made a link:
  \url{https://www.geogebra.org/classic/p67vekna}}
\najib*{The figure is good! Just a little remark, the blue spheres are not of the same size in the middle figure, is it okay?}



Defining
\begin{align*}
  \eta & :=\sum_{\Pi_{\underline{N-1}}\S^0}
  \left(\prod_{\underline{N-1}}\epsilon_i\right)
  \left(a \circ_{\mathfrak o}\beta^{\langle\epsilon_i\rangle}
  +\beta^{\langle\epsilon_i\rangle}\circ_{-\epsilon_{N-1} (N-1)}a\right)
\end{align*}

I think we can build a homotopy between $l$ and $\eta+\eta\cdot(1_{\mathfrak c} 2_{\mathfrak c})$.

Pretty sure this is equivalent to the case $N=2$ in my thesis.

\renato*{Here I am writing an alternative set of cells that I think might work better. Instead of $m$ being a configuration of $2(N-1)$ half disks I think we can use just $N$. Not only there are fewer things moving, now there is only one possible choice of where one of the closed little disks is staying when the other is moving}

\najib{This looks promising! One thing that I hoped to achieve is to write some kind of inductive formula ; the $\beta$ in dimension $N+1$ should be obtained from the one in dimension $N$ glued nicely, but I didn't really manage to complete it.}

Let $\underline{N-1}_*=\{0, 1, \dots N-1\}$, $\vec b_0=\vec 0$ and $\vec b_i=\vec e_i$ for $i\in\underline{N-1}$.

Let $m\in\mathcal{SC}_N^{\text{vor}}(\underline{N-1}_*)$ be defined by:
$$
  m_i=\left(\frac{2}{3}\vec b_i, \frac{1}{3}\right)
$$
For each $i\in\underline{N-1}_*$ let $\beta^i\in \mathcal{SC}_N^{\text{vor}}(\{\mathfrak c\}\sqcup \underline{N-1}_*)$ be defined by
\begin{gather*}
  \beta^i_{\mathfrak c }\vec t:=
  \left(
  \frac{1+2t_0}{4}\vec e_0
  +
  \frac{2}{3}\left(\sum_{j=1}^{N-1}t_j\vec b_{j+i-1}\right)
  , \frac{1}{4(3-2t_0)}\right)\\
  \beta^i_j\vec t
  :=\begin{cases}\left(\frac{2-t_0}{3}\vec b_j , \frac{2-t_0}{(1+t_{j-i+1})6}\right),
     & j\neq i-1 \\
    \left(\frac{2-t_0}{3}\vec b_j , \frac{2-t_0}{6}\right),
     & j= i-1
  \end{cases}
\end{gather*}

\renato*[caption=Easier to generalize]{For $N=2$ the cycle homologous to $a\circ_{\mathfrak c}l$ is
  \begin{gather*}
    a\circ_{\mathfrak o}\beta^0+\beta^1\circ_0 a-(\beta^0\circ_1a+a\circ_{\mathfrak o}\beta^1)\cdot(1_{\mathfrak c}2_{\mathfrak c})\\
    +(a\circ_{\mathfrak o}\beta^0+\beta^1\circ_0 a)\cdot(1_{\mathfrak c}2_{\mathfrak c})-(\beta^0\circ_1a+a\circ_{\mathfrak o}\beta^1)
  \end{gather*}
  We can define
  \begin{align*}
    \eta:= & a\circ_{\mathfrak o}\beta^0    -a\circ_{\mathfrak o}\beta^1 \\
           & -\beta^0\circ_1a+\beta^1\circ_0a
  \end{align*}
  so that the above cycle is
  $$
    \eta+\eta\cdot (1_{\mathfrak c}2_{\mathfrak c})
  $$
  This isn't exactly what I did in my thesis, but I think this will be easier to generalise
}

Let
$$
  \eta:=\sum_{j=0}^{N-1} (-1)^{j(N-1)}(a\circ_{\mathfrak o}\beta^j-\beta^j\circ_{j-1}a)
$$
\renato*{
  I'm not 100\% sure about the signs. And I think $\partial(\eta +\eta\cdot(1_{\mathfrak c}2_{\mathfrak c}))$ isn't going to be exactly 0 when $N>2$. Might need some fixing}

For $0<i< N-1$ and $0\leq j\leq N-1$ we have by construction that
$$
  \partial_i\beta^j=\partial_{i+1}\beta^{j+1}
$$
and
$$
  \partial_{N-1}\beta^j=\partial_1\beta^{j-1}
$$
which implies

\begin{align*}
  \partial \eta & =\sum_{i=0}^{N-1}(-1)^i\partial_i\eta                                                                                 \\
                & = \sum_{i=0}^{N-1}(-1)^i\partial_i\left(
  \sum_{j=0}^{N-1} (-1)^{j(N-1)}(a\circ_{\mathfrak o}\beta^j-\beta^j\circ_{j-1}a)\right)                                                \\
                & =\sum_{i=0}^{N-1}\sum_{j=0}^{N-1}(-1)^{i+j(N-1)}(a\circ_{\mathfrak o}\partial_i\beta^j-\partial_i\beta^j\circ_{j-1}a) \\
                & =\left(\sum_{i=0}^{N-1}\sum_{j=0}^{N-1}(-1)^{i+j(N-1)}a\circ_{\mathfrak o}\partial_i\beta^j\right)
  -\left(\sum_{i=0}^{N-1}\sum_{j=0}^{N-1}(-1)^{i+j(N-1)}\partial_i\beta^j\circ_{j-1}a\right)                                            \\
                & =\left(\sum_{j=0}^{N-1}(-1)^{i+j(N-1)}a\circ_{\mathfrak o}\partial_0\beta^j\right)
  -\left(\sum_{i=0}^{N-1}\sum_{j=0}^{N-1}(-1)^{i+j(N-1)}\partial_i\beta^j\circ_{j-1}a\right)
\end{align*}


\renato*{Started a draft of the calculations. Not much so far}

% \section{General approach to non-formality}

% \najib*{Here we can explain how non-formality can be deduced from finding chains that bound cycles and produce non-trivial homology classes when composed.}
