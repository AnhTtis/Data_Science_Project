\vspace{-0.36cm}
\section{Introduction}

% In this section, the value of optimizing the multi-layer bandwidth control problem is first highlighted.

% Then the problem is described concisely. 

% And the existing solutions to above problem are presented.

% Then the introduction of proposed method is presented and summarize our technical contributions.

\label{sec: intro}
With the rapid development of computing techniques such as cloud computing, machine learning, high performance computing, etc., computer systems are supposed to perform different computing workloads concurrently. Besides, new storage media emerge in endlessly in recent years, which makes the storage architecture become more complex than ever. Among recent advanced storage architectures, the Hierarchical Storage System (HSS)~\cite{Vengerov08, 2014Proactive}, 
which is also known as tiered storage, is considered as an ideal model to meet the cost-performance demand and thus draws considerable attention from both academic researchers and industrial practitioners. Specifically, HSS is comprised of various storage media which are different in terms of speed, efficiency, capacity and price such as Flash, Solid State Disk (SSD), Hard Disk Drive (HDD), etc. To achieve the best cost-performance goal, various types of storage medium are placed at different tiers within HSS. In such a storage system, data file will be migrated based on access frequency between different tiers to enhance the system performance, which is called data migration. Most of previous research~\cite{2005An, Vengerov08} about HSS focuses on studying the data migration policy that 
determines which data file to migrate to which tier, such as Least Recently Used replacement (LRU), Size-Temperature Replacement (STR), Heuristic Threshold (STEP) etc. However, in recent HSS, the maximum amount of data migration is adjustable, of which the corresponding task is named as bandwidth control. Few research about HSS pays attention to this topic.
% which is called bandwidth control. 

The bandwidth control is certainly worthy of being investigated in HSS since it is highly related to the performance and security of the whole system. Specifically, data migration can change the data distribution (such as hot/cold/garbage data) in HSS and the capacity utilization of various storing tiers, which ultimately determines the system performance. The data migration is directly affected by the bandwidth control because the bandwidth control limits the maximum amount of data migration per unit time. Hence, the bandwidth control has huge impact on system performance, such as the throughput, throughput stability and tail latency. Besides, in HSS if one or more tiers are overfilled, the data migration between tiers will be blocked. Even worse, it might cause the system to collapse. Thus, the bandwidth control is supposed to protect each storing tier from being overload, namely to ensure security of HSS. Previous research about bandwidth control~\cite{2015ROBUS, Jaehyung2017Selective, 2019LBICA} also draw similar conclusions but on other scenarios (caching and networking). They propose different bandwidth control policies (such as LBICA~\cite{2019LBICA} and SIB~\cite{Jaehyung2017Selective}) for cache and networking optimization. However, they cannot directly be applied in HSS since they may only take effect on simple tiered storage system or single bandwidth control task (e.g. throughput of cache or networking) without further consideration of data migration between multiple (more than two) tiers.
% In HSS there are many data migration tasks, related to bandwidth control, such as throughput and garbage collection.
% If these tasks  adopt the bandwidth control policy used in cache optimization, it might degrade the HSS performance.

% But what never changes is that people always want to achieve the best performance at the lowest cost. To this end, the Hierarchical Storage Systems (HSS) (also called tiered storage) was proposed and widely studied~\cite{1995Three, Vengerov08, 2014Proactive}. HSS is comprised of various storage medias which are different in terms of speed, efficiency, capacity and price such as Flash, SSD, HDD, etc. The storage medias are placed at different tiers within HSS to meet the cost-performance demand. In such a storage system, data file can migrate between different tiers according to access frequency.

% Most of previous research~\cite{1993The, 2005An} about HSS focuses on investigating the data migration policy that controls the data file transfers in the system, such as Least Recently Used replacement (LRU), Size-Temperature Replacement (STR), Heuristic Threshold (STEP) etc. In research of recent years, there are also some intelligent policies using machine learning technologies. Actually, the data migration will consume a certain amount of system bandwidth. The total bandwidth consumption cannot exceeds a given limitation. The previous work~\cite{1993The, 2005An} always assume that the bandwidth limitation within HSS is constant. They hardly consider how the bandwidth limitation (also called bandwidth control) could affect the performance of HSS. However, it has started to draw attention from the researchers of cache optimization. \cite{2015ROBUS, Jaehyung2017Selective, 2019LBICA} suggest that the bandwidth control on cache has significant impact on the cache performance. Nevertheless, the bandwidth control policies used in cache cannot be directly adapted to HSS. Because these policies usually take effect on single storage medium or single bandwidth control task without consideration of data migration between multiple tiers. In HSS there are many background tasks, related to bandwidth control, such as throughput, flush and garbage collection. If these tasks directly adopt the bandwidth control policy used in cache optimization, it might degrade the HSS performance.

It is challenging to design a bandwidth control policy that can achieve good performance and meanwhile assure the robustness for HSS. Firstly, it is hard to formalize the bandwidth control problem for HSS with a unified mathematical model. Secondly, due to many data migration tasks and their complex interactions in HSS, the required policy needs to cooperatively control the bandwidth of different tasks. Besides, there are usually multiple performance goals (such as high throughput and low tail latency) in the system, which are required to be  optimized simultaneously. 
% The required control policy is also supposed to achieve multiple optimization goals simultaneously. 
To this end, we propose a learning-aided cooperative bandwidth control policy in this paper, which is named \Pascal{} in memory of the \textit{Pascal}'s law.
An abstract indicator, namely pressure, is created, based on which the HSS is illustrated as a pressure balancing system. Moreover, machine learning techniques are applied to further optimize the performance of \Pascal{} by enabling it to control the HSS in a cooperative way.
% , which is inspired by the \textit{Pascal}'s law. The multi-tier HSS resembles a hydraulic system that consists of multiple various containers; The data migration within HSS resembles the liquid flow in the hydraulic system. 
% Compared with existing work~\cite{2015ROBUS, Jaehyung2017Selective, 2019LBICA}, \Pascal{} controls the bandwidth of different background tasks in a cooperative way. 
To summarize, the technical contribution of this paper is four-fold:


% due to heterogeneous storage media of HSS. Thus the bandwidth control for HSS becomes a problem that needs to be investigated thoroughly.




% Nowadays, more and more enterprise storage systems adopt the hierarchical storage architecture, which is also known as multi-layer storage system~\cite{1995Three, 2014Proactive, Vengerov08}. The architecture is comprised of various storage medias which are different in the terms of speed, efficiency, capacity and price, such as tape, HDD, SSD, Flash, etc.
% The storage medias are placed at different levels of the architecture to meet the demand of both performance and cost. For example, the fast storage tiers usually are expensive and smaller in capacity, such as SSD and Flash. Whereas the slow ones are relatively cheap and significantly large in capacity, such as disk and tape. Within such a hierarchical storage architecture, stored data are distributed at above storage tiers according to some specific indicators (such as hotness level~\cite{Vengerov08}). The architecture is considered to be an ideal model which could achieve the best performance with the lowest cost simultaneously~\cite{DBLP:journals/fcsc/XuLLZSDZWZCXY14, DBLP:journals/corr/abs-1207-0147}. 
% The combination of medias with characteristic function makes the architecture an ideal model that could achieve excellent performance with relatively low cost simultaneously~\cite{DBLP:journals/fcsc/XuLLZSDZWZCXY14, DBLP:journals/corr/abs-1207-0147}. 

% In a hierarchical storage system, data file is moved not only from the outside, but also form one part of it to another part of it, which is so called data migration. Data migration is a crucial task for storage system, as it is capable of improving the performance and meanwhile ensuring the security of the system. For example,  tierdown, whose function is moving the data from SSD to HDD, is supposed to keep the data requested most frequently(hot data) in SSD, while move the others to HDD. By this meas, the response time of read request is reduced significantly while the risk of exceeding capacity limit is minimized.


 
% Figure~\ref{fig: problem_description} gives an example of three-tier hierarchical storage architecture, which is defined by the access speed, storage capacity and corresponding price. The fastest tier is comprised of memory or Flash, which are the most expensive and the smallest in capacity; The middle tier is made of SSD, which is relatively less expensive and larger in capacity; The slowest tier consists of HDD, which is the cheapest and largest in capacity. In general, it is desirable to place the `hot' data, that is requested most frequently in recent periods, in fast tiers while keeping `cold' data that is requested less frequently at slow tiers. The stored data could be moved via different tasks (such as throughput, flush, tierdown, garbage collection shown in Figure~\ref{fig: problem_description}) between each tiers, i.e., data migration, within the hierarchical storage architecture to improve the availability of data and performance of the whole system. Many data migration policies for hierarchical storage system~\cite{Vengerov08, 2014Proactive} have been proposed to optimally migrate data in order to minimize the average system response time. In these work~\cite{Vengerov08, 2014Proactive}, the authors always assume that there is a fixed bandwidth limit for each tier. However, the recent storage systems are capable of adjusting bandwidth at run time. There are many previous works that investigated the bandwidth control for the cache~\cite{DBLP:conf/sigmod/KunjirFMB17, 2018LBICA, Jaehyung2017Selective}. But few researchers pay attention to bandwidth control in hierarchical storage system. Thus this paper dedicates to investigate the problem that is how to optimally control the bandwidth for each tier given a specific data migration policy in a hierarchical storage system.



%  Generally, the forthcoming storage product of HUAWEI is with hierarchical storage architecture, where flash, SSD and HDD are placed in different layers of the three-layer architecture so as to maximize the hardware performance. One I/O request entering into the system will flow each layer of the system (see Figure~\ref{fig:problem_description}). 
\begin{itemize}
% [leftmargin=*]
    \item We propose a well-defined stochastic programming model to formalize the bandwidth control problem in HSS.
    \item We design a learning-aided bandwidth control policy for HSS, named \Pascal{}, which can learn to control the bandwidth of different tasks in a cooperative way.
    \item To validate the effectiveness of proposed control policy, we implement \Pascal{} and three strong baselines on a commercial hierarchical storage system and compare their performance over a group of workloads.
    \item Comprehensive experimental results suggest that \Pascal{} can effectively decrease the tail latency (1.95X $\downarrow$) and greatly improve the throughput stability (2X $\uparrow$), and meanwhile keep the throughput at a relatively high level.
\end{itemize}

\vspace{-2ex}
