\section{Proof of Theorem \ref{thm:experiment_design_assumptions_hold}}
\label{sec:experimental_design_proofs}

%\subsection{Proof of Theorem \ref{thm:experiment_design_assumptions_hold}}
%\label{subsec:experimental_design_theorem_proof}

Let $\mathcal{E}_{1}, \ \mathcal{E}_{2},\  \mathcal{E}_{3},\  \mathcal{E}_{4},\  \mathcal{E}_{5}$ denote the event that horizontal span inclusion (Assumptions \ref{ass:donor_set_identification} (a)), incoherence of donor units (Assumption \ref{ass:incoherence}), linear span inclusion (\ref{ass:donor_set_identification}) (b), balanced spectrum (Assumption \ref{ass:balanced_spectrum}) and subspace inclusion (Assumption \ref{ass:rowspace_inclusion}) hold, respectively. Let $\mathcal{E}_{H} = \mathcal{E}_1 \cap \mathcal{E}_2$ and $\mathcal{E}_{V} = \mathcal{E}_3 \cap \mathcal{E}_4 \cap \mathcal{E}_5$ denote the events that the conditions required for horizontal and vertical regression are satisfied respectively. We proceed by showing that both events $\mathcal{E}_{H}$ and $\mathcal{E}_{V}$ hold with high probability. Throughout the proof of the Theorem, and all lemmas, we use $c,c',c'',c''',C,C'$ to refer to universal constants that might change from line to line. 
\vspace{2mm}

\noindent \emph{Step 1: $\mathcal{E}_H$ high probability bound.} To show $\mathcal{E}_H$ holds with high probability, we state the following two lemmas proved in Appendix \ref{subsec:experiment_design_lemmas_proof_horizontal_regression}. 

\begin{lemma} [Horizontal Span Inclusion]
\label{lem:experiment_design_horizontal_span_inclusion} 
Let the set-up of Theorem \ref{thm:experiment_design_assumptions_hold} hold. Then, horizontal span inclusion (Assumption \ref{ass:donor_set_identification}(a)) holds for every unit $u \in \mathcal{I}$ with probability at least $1 - \gamma/4$.
\end{lemma}


\begin{lemma} [Incoherence of Donor Units]
\label{lem:experiment_design_incoherence} Let the set-up of Theorem \ref{thm:experiment_design_assumptions_hold} hold. Then, $\bchi(\Pi_{\mathcal{I}})$ satisfies the incoherence condition: 
\begin{equation*}
    \lVert \frac{\bchi(\Pi_{\mathcal{I}})^T\bchi(\Pi_{\mathcal{I}})}{|\Pi_{\mathcal{I}}|} - \mathbf{I}_{2^p} \rVert_{\infty} \leq \frac{C}{s}
\end{equation*}
for a universal constant $C > 0$ with probability at least $1 - \gamma/4$. 
\end{lemma}

\noindent Applying Lemmas \ref{lem:experiment_design_horizontal_span_inclusion} and \ref{lem:experiment_design_incoherence} alongside Demorgan's law and the union bound gives us
\begin{equation}
\label{eq:experiment_design_horizontal_events_high_probability}
    \mathbb{P} \left(\mathcal{E}^c_H \right) =  \mathbb{P} \left(\mathcal{E}^c_1 \cup \mathcal{E}^c_2 \right) \leq \mathbb{P} \left(\mathcal{E}^c_1\right) +\mathbb{P} \left(\mathcal{E}^c_2\right) \leq \gamma/2
\end{equation}

%if $|\Pi_\mathcal{I}| \geq Cs^2\log(s^2\log(2^p|\mathcal{I}|)/\gamma)$ for a universal constant $C >0$
\vspace{2mm}

\noindent \emph{Step 2: $\mathcal{E}_V$ high probability bound.} Next, we show that $\mathcal{E}_V$ holds with high probability. To that end, we state the following two lemmas proved in Appendix \ref{subsec:experiment_design_lemmas_proof_vertical_regression}. We define some necessary notation for the following lemma, let $\mathcal{A}_{\mathcal{I}} = [\balpha_u : u \in \mathcal{I}] \in \mathbb{R}^{|\mathcal{I}| \times 2^p}$. 

\begin{lemma} [Linear Span Inclusion]
\label{lem:experiment_design_linear_span_inclusion}
Let the set-up of Theorem \ref{thm:experiment_design_assumptions_hold} hold. Then, we have that
\begin{equation}
\label{eq:experimental_design_donor_set_balanced_spectrum}
\frac{C|\mathcal{I}|}{r} \geq s_1 \left(\mathcal{A}_{\mathcal{I}}^T \mathcal{A}_{\mathcal{I}} \right)  \geq s_r \left(\mathcal{A}_{\mathcal{I}}^T \mathcal{A}_{\mathcal{I}} \right) \geq \frac{C'|\mathcal{I}|}{r}
\end{equation}
for universal constants $C,C' > 0$ with probability at least $ 1 - \gamma/8$.
\end{lemma}
\noindent 

\noindent Since $\mathcal{A}_{\mathcal{I}}$ has rank $r$, we have that $\text{rank}(\mathcal{A}_{\mathcal{I}}) = \text{rank}(\mathcal{A})$. Further, since  $\mathcal{A}_{\mathcal{I}}$ is a sub-matrix of $\mathcal{A}$, linear span inclusion (i.e., $\mathcal{E}_3$) holds with probability at least $ 1 - \gamma/8$. Next, we show that balanced spectrum and subspace inclusion hold. 

\begin{lemma} [Balanced Spectrum]
\label{lem:experiment_design_balanced_spectrum}
Let the set-up of Theorem \ref{thm:experiment_design_assumptions_hold} and \eqref{eq:experimental_design_donor_set_balanced_spectrum} hold. Then, for universal constants $c,c' > 0$, we have 
$s_r(\E[\bY_{\mathcal{I}}^{(\Pi_N)}])/s_1(\E[\bY_{\mathcal{I}}^{(\Pi_N)}]) \geq c$ and $\lVert \E[\bY_{\mathcal{I}}^{(\Pi_N)}]\rVert^2_F \geq c'|\Pi_N||\mathcal{I}|$ with probability at least $1 - \gamma/8$. That is, balanced spectrum (Assumption \ref{ass:balanced_spectrum}) holds with high probability. 
\end{lemma}

\begin{lemma} [Subspace Inclusion] 
\label{lem:experiment_design_subspace_inclusion}
Let the set-up of Theorem \ref{thm:experiment_design_assumptions_hold} and \eqref{eq:experimental_design_donor_set_balanced_spectrum} hold. Then, $\E[\bY_{\mathcal{I}}^{(\pi)}]$ lies in the row-space of $\E[\bY_{\mathcal{I}}^{(\Pi_N)}]$ holds with probability at least $1 - \gamma/8$. That is, subspace inclusion (Assumption \ref{ass:rowspace_inclusion}) holds with high probability.  
\end{lemma}

\noindent Let $\mathcal{E}_6$ denote the event that \eqref{eq:experimental_design_donor_set_balanced_spectrum} holds. As a result of Lemma \ref{lem:experiment_design_subspace_inclusion}, we have that $\P(\mathcal{E}_4 ~ | ~ \mathcal{E}_6) \geq 1 - \gamma/8$ and  $\P(\mathcal{E}_5 ~ | ~ \mathcal{E}_6) \geq 1 - \gamma/8$. Additionally, note that $\mathcal{E}_3 \subset \mathcal{E}_6$. Hence, we have
\begin{align}
    \P(\mathcal{E}_V) & = \P(\mathcal{E}_3 \cap \mathcal{E}_4 \cap \mathcal{E}_5)  \geq \P(\mathcal{E}_4 \cap \mathcal{E}_5 \cap \mathcal{E}_6) \nonumber \\
    & = \P(\mathcal{E}_4 \cap \mathcal{E}_5  ~ | ~ \mathcal{E}_6) \P( \mathcal{E}_6) \nonumber \\
    & = \left(\P(\mathcal{E}_4   ~ | ~ \mathcal{E}_6) + \P(\mathcal{E}_5   ~ | ~ \mathcal{E}_6) - \P(\mathcal{E}_4 \cup \mathcal{E}_5  ~ | ~ \mathcal{E}_6)\right) \P( \mathcal{E}_6)  \nonumber \\
    & \geq \left(\P(\mathcal{E}_4   ~ | ~ \mathcal{E}_6) + \P(\mathcal{E}_5   ~ | ~ \mathcal{E}_6) \right) \P( \mathcal{E}_6)  - 1 \nonumber \\
    & =  \P(\mathcal{E}_4 ~ | ~ \mathcal{E}_6) \P( \mathcal{E}_6)  +  \P(\mathcal{E}_5 ~ | ~ \mathcal{E}_6) \P( \mathcal{E}_6)  - 1 \nonumber \\
    & \geq 2(1 - \gamma/8)^2 - 1\geq 1 - \gamma/2 \label{eq:experiment_design_vertical_regression_hold}
\end{align}

    %s_r \left(\E[\bY_{\mathcal{I}}^{(\Pi)}]^T \E[\bY_{\mathcal{I}}^{(\Pi)}]\right) \geq \frac{C'}{r}

\noindent \emph{Step 3: Collecting Terms.} Let $\mathcal{E} = \mathcal{E}_H \cap \mathcal{E}_V$. To complete the proof, it suffices to show that $\P(\mathcal{E}) \geq 1 - \gamma$. Applying \eqref{eq:experiment_design_horizontal_events_high_probability} and \eqref{eq:experiment_design_vertical_regression_hold} gives us 
\begin{equation*}
    \P(\mathcal{E}^c) =  \P(\mathcal{E}_H^c \cup \mathcal{E}_V^c) \leq  \P(\mathcal{E}_H^c) + \P(\mathcal{E}_V^c) \leq 1 - \gamma
\end{equation*}
This completes the proof. 


\section{Proofs of Helper Lemmas for  Theorem \ref{thm:experiment_design_assumptions_hold}}
\label{subsec:experimental_design_helper_lemmas}

Throughout these proofs, we use $c,c',c'',c''',C,C'$ to refer to positive universal constants that can change from line to line. 


\subsection{Proof of Lemmas for Horizontal Regression}
\label{subsec:experiment_design_lemmas_proof_horizontal_regression}

In this section, we provide proofs that horizontal span inclusion and incoherence of the donor unit Fourier characteristics hold with high probability under our experimental design mechanism. That is, we prove Lemmas \ref{lem:experiment_design_horizontal_span_inclusion} and \ref{lem:experiment_design_incoherence} used in Theorem \ref{thm:experiment_design_assumptions_hold}.  


\subsubsection{Proof of Lemma \ref{lem:experiment_design_horizontal_span_inclusion}}
We begin by defining some necessary notation. For a given  unit $u \in \mathcal{I}$, and let $\mathcal{S}_u = \{S \subset [p] ~ | ~ \alpha_{u,S} \neq 0 \}$. 
%
That is, $\mathcal{S}_u$ denotes the subset of coordinates where $\balpha_u \in \mathbb{R}^{2^p}$ is non-zero. 
%
Note that by Assumption \ref{ass:observation_model},  $|\mathcal{S}_u| \leq s$. 
%
For any combination $\pi$, let $\bchi_{\mathcal{S}_u}^{\pi} = [\chi^{\pi}_{S} : S \in \mathcal{S}_u] \in \{-1,1\}^{|\mathcal{S}_u|}$ denote the projection of $\bchi^\pi$ to the non-zero coordinates of $\balpha_u$. 
%
For example, if $\balpha_u = (1,1, \ldots ,0)$ and $\bchi^{\pi} = (1,1,\ldots ,1)$, then $\bchi_{\mathcal{S}_u}^{\pi} = (1,1)$. 
%
Finally, let $\bchi_{\mathcal{S}_u}(\Pi_{\mathcal{I}}) = [\bchi_{\mathcal{S}_u}^{\pi}: \pi \in \Pi_{\mathcal{I}}]$ $\in \{-1,1\}^{|\Pi_{\mathcal{I}}| \times |\mathcal{S}_u|}$. 



\noindent To proceed, we  first show horizontal span inclusion holds for a given unit $u \in \mathcal{I}$, and then show that it extends to the entire donor set $\mathcal{I}$ via the union bound. To that end, fix a donor unit $u \in \mathcal{I}$ and then we proceed by quoting the following result required for our proof, which is based on the matrix Bernstein's inequality \cite[Theorem 1.4]{tropp2012user}.

\begin{theorem}[{\cite[Theorem 5.41]{vershynin_2012}}]
Let $A$ be an $N \times s$ matrix whose rows $A_i$ are independent isotropic random vectors in $\mathbb{R}^s$. Let $m$ be such that $||{A_i}||_2 \le \sqrt{m}$ almost surely for all $i \in [N]$. Then, for every $t \ge 0$, one has
\[\sqrt{N} - t\sqrt{m} \le s_{\text{min}}(A) \le s_{\text{max}}(A) \le \sqrt{N} + t\sqrt{m}\]
with probability at least $1 - 2s\cdot\exp(-ct^2)$.
\end{theorem}

\noindent To apply the theorem, we first show that $\bchi_{\mathcal{S}_u}^{\pi}$ 
is isotropic for any combination $\pi \in \Pi_{\mathcal{I}}$. That is, we show  $\E[\bchi_{\mathcal{S}_u}^{\pi}(\bchi_{\mathcal{S}_u}^{\pi})^T] = \mathbf{I}_{\mathcal{S}_u}$ where the expectation is taken over the randomness in choosing $\pi$ uniformly and independently at random from $\Pi$. The isotropy of $\bchi_{\mathcal{S}_u}^{\pi}$  follows since for any two subsets $S,S'\subset[p]$, we have 
\begin{align*}
\E[\chi^{\pi}_{S}\chi^{\pi}_{S'}] 
&= \frac{1}{2^p}\sum_{\pi \in \Pi} \chi^{\pi}_{S}\chi^{\pi}_{S'} \\
&= \left\langle \chi_{S}, \chi_{S'} \right\rangle_B \\
& = \mathbbm{1}\{S=S'\}.
\end{align*}
Next, we verify that $\|\bchi_{\mathcal{S}_u}^{\pi}\|_2 \le \sqrt{s}$ almost surely. This follows since $\bchi_{\mathcal{S}_u}^{\pi} \in \{1,-1\}^{|\mathcal{S}_u|}$, so
\[\left\|\bchi_{\mathcal{S}_u}^{\pi}\right\|_2 \le \sqrt{|\mathcal{S}_u|}\left\|\bchi_{\mathcal{S}_u}^{\pi}\right\|_\infty \le \sqrt{|\mathcal{S}_u|}.\]
Thus, applying the theorem with $N= |\Pi_{\mathcal{I}}|$ gives
\[\sqrt{|\Pi_{\mathcal{I}}|} - t\sqrt{|\mathcal{S}_u|} \le s_{\text{min}}(\bchi_{\mathcal{S}_u}(\Pi_{\mathcal{I}}))\]
with probability at least $1 - 2|\mathcal{S}_u|\cdot\exp(-ct^2)$.
Choosing $t = \sqrt{|\Pi_{\mathcal{I}}|/4|\mathcal{S}_u|}$ gives 
\begin{equation}
\label{eq:lower_bound_min_singular_value_fourier_characteristic}
    \sqrt{|\Pi_{\mathcal{I}}|/4} \le s_{\text{min}}(\bchi_{\mathcal{S}_u}(\Pi_{\mathcal{I}}))
\end{equation}
with probability at least $1 - 2|\mathcal{S}_u|\cdot\exp(-c|\Pi_{\mathcal{I}}|/4|\mathcal{S}_u|)$. Using our assumption that $|\Pi_{\mathcal{I}}| \geq \frac{Cr^2s^2\log(|\mathcal{I}|2^p/\gamma)}{\delta^2} \geq C|\mathcal{S}_u|\log(|\mathcal{S}_u||\mathcal{I}|/\gamma)$
establishes that $\bchi_{\mathcal{S}_u}(\Pi_{\mathcal{I}})$ has full-rank with probability at least $1 - \gamma/|\mathcal{I}|$. Since, $\bchi_{\mathcal{S}_u}(\Pi_{\mathcal{I}})$ has full-rank, we have that horizontal span inclusion holds for unit $u$. Taking a union bound over all donor units $u \in \mathcal{I}$ completes the proof. 

\subsubsection{Proof of Lemma \ref{lem:experiment_design_incoherence}}
\label{ref:subsubsec:experiment_design_incoherence_proof}
We begin the proof by defining some notation. For a subset $S \subset [p]$, let $\bchi^{\Pi_I}_{S} = [\chi^{\pi}_{S} : \pi \in \Pi_{\mathcal{I}} ] \in \{-1,1 \}^{|\Pi_{\mathcal{I}}|}$. For any two distinct subsets $S,S' \in [p]$, define $z_i = \chi^{\pi_i}_{S}\chi^{\pi_i}_{S'}/|\Pi_{\mathcal{I}}| \in \{-\frac{1}{|\Pi_{\mathcal{I}}|},\frac{1}{|\Pi_{\mathcal{I}}|} \}$ for $1 \leq i \leq |\Pi_\mathcal{I}|$. Let $S \ \Delta \ S'$ denote the symmetric difference of  two subsets $S,S'$.  Then, 
\begin{align*}
    \E[z_i] = \E\left[\frac{1}{|\Pi_\mathcal{I}|} \chi^{\pi_i}_{S}\chi^{\pi_i}_{S'}\right]  & =  \E\left[\frac{1}{|\Pi_\mathcal{I}|} \left(\prod_{j \in S} v(\pi_i)_j \right) \left(\prod_{j' \in S'} v(\pi_i)_{j'} \right)\right] \\
    & = \E\left[\frac{1}{|\Pi_\mathcal{I}|} \left(\prod_{j \in S \ \Delta \ S'} v(\pi_i)_j \right)\right]
\end{align*}
where the expectation is taken over choosing $\pi_i$ uniformly and independently at random. Since each $\pi \in \Pi_{\mathcal{I}}$ is chosen in this fashion, we have that the random variables $v(\pi)_j$ are independent and identically distributed being over over $\{-1,1\}$. Plugging this into the equation above, we have that  
\begin{align*}
    \E[z_i] & = \E\left[\frac{1}{|\Pi_\mathcal{I}|} \left(\prod_{j \in S \ \Delta \ S'} v(\pi_i)_j \right)\right] \\
    & = \frac{1}{|\Pi_\mathcal{I}|} \prod_{j \in S  \Delta  S'} \E[v(\pi_i)_j] = 0
\end{align*}

\noindent Next, observe that since each $\pi \in \Pi_{\mathcal{I}}$ is chosen independently and uniformly and random, we have that $z_1 \ldots z_{|\Pi_\mathcal{I}|}$ are independent and identically distributed bounded random variables. Hence, we can apply Hoeffding's inequality for any $ t > 0$ to give us, 
\begin{equation}
\label{eq:high_probabilty_bound_inner_product_fourier_characteristic}
    \P\left(\sum^{|\Pi_{\mathcal{I}}|}_{i = 1} z_i \geq t \right) \leq 2\exp(-\frac{|\Pi_I\mathcal{}|t^2}{2})
\end{equation}


\noindent Then, for any distinct subsets $S,S' \subset [p]$, we have
\begin{align*}
    \frac{1}{|\Pi_\mathcal{I}|} \langle \bchi^{\Pi_I}_{S}, \bchi^{\Pi_I}_{S'} \rangle = \sum_{\pi_i \in \Pi_{\mathcal{I}}} \chi^{\pi_i}_{S}, \chi^{\pi_i}_{S'} = \sum^{|\Pi_\mathcal{I}|}_{i = 1} z_i
\end{align*}
Therefore, we have that 
\begin{align*}
    \P\left(|\max_{S \neq S'} \frac{1}{|\Pi_\mathcal{I}|}  \langle \bchi^{\Pi_I}_{S}, \bchi^{\Pi_I}_{S'} \rangle | \geq t \right) & \leq \sum_{S \neq S'} \P\left( \frac{1}{|\Pi_\mathcal{I}|}  | \langle \bchi^{\Pi_I}_{S}, \bchi^{\Pi_I}_{S'} \rangle | \geq t \right) \\
     & \leq 2^{2p+1}\exp(-\frac{|\Pi_I\mathcal{}|t^2}{2})
\end{align*}
where we use \eqref{eq:high_probabilty_bound_inner_product_fourier_characteristic} for the last inequality. Choosing $t = C'/s$, and using our assumption that $|\Pi_{\mathcal{I}}| \geq \frac{Cr^2s^2\log(|\mathcal{I}|2^p/\gamma)}{\delta^2} \geq Cs^2\log(2^p/\gamma)$ gives us 
\begin{equation}
\label{eq:high_probabilty_bound_max_inner_product_fourier_characteristic}
     \P\left( \max_{S \neq S'}  |\frac{1}{|\Pi_\mathcal{I}|}  \langle \bchi^{\Pi_I}_{S}, \bchi^{\Pi_I}_{S'} \rangle | \geq \frac{C'}{s} \right) \leq \gamma 
\end{equation}
for an appropriately chosen constant $C > 0$. To proceed, observe that $\langle \bchi^{\Pi_I}_{S}, \bchi^{\Pi_I}_{S} \rangle/|\Pi_{\mathcal{I}}| = 1$. Hence, we have that 
\begin{equation*}
    \lVert \frac{\bchi(\Pi_\mathcal{I})\bchi(\Pi_\mathcal{I})^T}{|\Pi_{\mathcal{I}}|} - \mathbf{I}_{2^p} \rVert_{\infty} = \max_{S \neq S'}  |\frac{1}{|\Pi_\mathcal{I}|}  \langle \bchi^{\Pi_I}_{S}, \bchi^{\Pi_I}_{S'}  \rangle |
\end{equation*}
Substituting this into \eqref{eq:high_probabilty_bound_max_inner_product_fourier_characteristic} completes the proof.

%Denote $\bchi_N(\Pi_\mathcal{I}) = \bchi(\Pi_\mathcal{I})/\sqrt{|\Pi_{\mathcal{I}}|}$. Note that every entry of $\bchi_N(\Pi_\mathcal{I}) = \pm 1/\sqrt{|\Pi_{\mathcal{I}}|}$, hence every column of  $\bchi_N(\Pi_\mathcal{I})$ is normalized, i.e., has $l_2$ norm $1$. Note that every column of $\bchi_N(\Pi_\mathcal{I})$ corresponds to a subset of $S \subset [p]$. Then, let $\bchi^{\Pi_I}_{N,S} = [\chi^{\pi}_{S}/\sqrt{|\Pi_{\mathcal{I}}|} : \pi \in \Pi_{\mathcal{I}} ] \in \{-\frac{1}{\sqrt{|\Pi_{\mathcal{I}}|}},\frac{1}{\sqrt{|\Pi_{\mathcal{I}}|}} \}^{|\Pi_{\mathcal{I}}|}$.  the inner product of any two distinct columns of $\bchi_N(\Pi_\mathcal{I})$ is the sum of $|\Pi_\mathcal{I}|$ independent and identically distributed bounded random variables. Hence, for any two distinct subsets $S,S'$ ,


\subsection{Proofs of Lemmas for Vertical Regression}
\label{subsec:experiment_design_lemmas_proof_vertical_regression}



In this section, we provide proofs that vertical span inclusion and subspace inclusion holds with high probability under our experimental design mechanism. That is, we prove Lemmas \ref{lem:experiment_design_linear_span_inclusion}, \ref{lem:experiment_design_balanced_spectrum}, and \ref{lem:experiment_design_subspace_inclusion} used in Theorem \ref{thm:experiment_design_assumptions_hold}.  

\subsubsection{Proof of Lemma \ref{lem:experiment_design_linear_span_inclusion}}
 Let $\mathcal{A}_{\mathcal{I}} = [\balpha_u: u \in \mathcal{I}] \in \mathbb{R}^{|\mathcal{I}| \times 2^p }$ denote the matrix of Fourier coefficients of the uniformly sampled donor set. By Assumption \ref{ass:observation_model} (a), it suffices to show that $\mathcal{A}_{\mathcal{I}}$ has rank $r$ with high probability. 
 To proceed, note that $\bchi(\Pi)/2^{p/2}$ is an orthogonal matrix since $(\bchi(\Pi)/2^{p/2})^T \bchi(\Pi)/2^{p/2} = \mathbf{I}_{2^p}$. Further, observe that $\E[\bY_{N}^{(\Pi)}] = \bchi(\Pi)\mathcal{A}^T$. Then, since multiplication by an orthogonal matrix preserves singular values, we have
\begin{equation}
\label{eq:fourier_coefficient_outcome_matrix_singular_value_equality}
   2^p s_i (\mathcal{A}^T\mathcal{A})  = 2^p s_i \left(\left(\frac{\bchi(\Pi)}{2^{p/2} }\right)\mathcal{A}^T\mathcal{A} \left(\frac{\bchi(\Pi)}{2^{p/2}}\right)^T\right)  = s_i(\E[\bY_{N}^{(\Pi)}]\E[\bY_{N}^{(\Pi)}]^T) 
\end{equation}
Then, by Assumption \ref{ass:restricted_balanced_spectrum} and the equality above, we have 
\begin{equation}
\label{eq:fourier_matrix_singular_value_bound}
    \frac{CN}{r} \ge s_1(\mathcal{A}^T\mathcal{A}) \ge s_r(\mathcal{A}^T\mathcal{A})  \ge \frac{cN}{r}.
\end{equation}


\noindent Now, let $\mathcal{\Tilde{A}} = [\Tilde{\balpha}_n : n \in [N]] \in \mathbb{R}^{|N| \times p'}$ be formed by removing all the zero columns of $\mathcal{A}$. Analogously, define $\mathcal{\Tilde A}_{\mathcal{I}} =  [\Tilde{\balpha}_u : u \in \mathcal{I}] \in \mathbb{R}^{|\mathcal{I}| \times p'}$ by removing all the zero columns of $\mathcal{A}_{\mathcal{I}}$. Then clearly, for every $i \in [r]$, we have that $s_{i}(\mathcal{\tilde{A}}^T\mathcal{\tilde{A}}) = s_i (\mathcal{A}^T\mathcal{A}) $.  Moreover, note that
\begin{equation*}
    \E[\tilde\balpha_u\tilde\balpha_u^\top] = \frac{1}{N}\sum_{n=1}^N\tilde\balpha_n\tilde\balpha_n^\top = \mathcal{\tilde{A}}^T\mathcal{\tilde{A}}/N.
\end{equation*}

\noindent where the expectation is taken over the randomness in uniformly and independently sampling donor units $u$ from $[N]$. That is, the expectation is taken over the randomness in choosing $\balpha_u$ uniformly and random from $\mathcal{A}$. To bound the number of non-zero columns $p'$, we state the following Lemma proved in Appendix \ref{subsubsec:proof_lemma_nonzeros}. 

\begin{lemma}
\label{lem:number_nonzeros}
The number of nonzero columns of $\mathcal{A}$, which we have denoted $p'$, satisfies $p' \le rs$.
\end{lemma}

\noindent Further, since multiplication by an orthogonal matrix preserves the $2$-norm of a vector, for every unit $n \in [N]$, we have
\begin{equation}
\label{eq:2_norm_fourier_coefficient}
    ||\balpha_n||_2  =  2^{-p/2}||\bchi(\Pi)\balpha_n||_2 =  2^{-p/2}||{\E[\bY_n^{(\Pi)}]}||_2 \leq 1 
\end{equation}
where we used the assumption $\E[Y_n^{(\pi)}] \in [-1,1]$ for every unit-combination pair. To proceed, we state the following result. 

\vspace{2mm}

\begin{theorem} [Theorem 5.44 in \cite{vershynin_2012}]
\label{thm:matrix_bernstein}
Let $X$ be an $q \times p'$ matrix whose rows $X_i$ are independent random vectors in $\mathbb{R}^n$ with the common second moment matrix $\Sigma = \mathbb{E}[X_iX_i^{\top}]$. Let $m$ be a number such that $\|X_i\|_2 \le \sqrt{m}$ almost surely for all $i$. Then for every $t\ge 0$, the following inequality holds with probability at least $1-p' \cdot \exp(-ct^2)$:
\[\left\|\frac{1}{q}X^{\top}X - \Sigma \right\| \le t\sqrt{\frac{m||\Sigma||}{q}} \vee \frac{t^2 m}{q}\]
Here $c > 0$ is an absolute constant.
\end{theorem}


\noindent Then, we can apply Theorem \ref{thm:matrix_bernstein} with $X = \mathcal{\tilde A_{\mathcal{I}}}$, $q = |\mathcal{I}|$, $m = 1$ (obtained from \eqref{eq:2_norm_fourier_coefficient}), $\Sigma = \E[\tilde\balpha_u\tilde\balpha_u^\top]$, $||\Sigma||= C/r$ (obtained from \eqref{eq:fourier_matrix_singular_value_bound}), and $t = \sqrt{\log(rs/\gamma)/c'}$ to give us that 
\begin{equation}
\label{eq:matrix_bernstein_fourier_matrix_bound}
    \lVert \frac{1}{|\mathcal{I}|}\mathcal{\tilde A_{\mathcal{I}}^T}\mathcal{\tilde A_{\mathcal{I}}} - \Sigma \rVert \leq \sqrt{\frac{\log(8rs/\gamma)}{c'r|\mathcal{I}|}} \vee \frac{\log(8rs/\gamma)}{c''|\mathcal{I}|}
\end{equation}
holds with probability at least $1-\gamma/8$. Hence, we have that 
\begin{align*}
    s_1(\mathcal{\tilde A_{\mathcal{I}}^T}\mathcal{\tilde A_{\mathcal{I}}}/|\mathcal{I}|) & \leq \frac{C}{r} +  \sqrt{\frac{\log(8rs/\gamma)}{c'r|\mathcal{I}|}} \vee \frac{\log(8rs/\gamma)}{c''|\mathcal{I}|} \\
\end{align*}
Choosing $|\mathcal{I}| \geq Cr\log(8rs/\gamma)$ for an appropriate constant $C > 0$ gives us that 
\begin{equation}
\label{eq:experimental_design_donor_set_largest_singular_value_upper_bound}
     s_1(\mathcal{\tilde A_{\mathcal{I}}^T}\mathcal{\tilde A_{\mathcal{I}}}/|\mathcal{I}|) \leq \frac{C}{r}
\end{equation}
with probability at least $1 - \gamma/8$. To bound the smallest singular value, we use Weyl's inequality (Theorem \ref{thm:weyl_inequality}), \eqref{eq:fourier_matrix_singular_value_bound}, and \eqref{eq:matrix_bernstein_fourier_matrix_bound}, to give us 
\begin{align*}
    s_r(\mathcal{\tilde A_{\mathcal{I}}^T}\mathcal{\tilde A_{\mathcal{I}}}/|\mathcal{I}|) & \geq  s_r(\mathcal{\tilde A^T}\mathcal{\tilde A}/N) - \lVert \frac{1}{|\mathcal{I}|}\mathcal{\tilde A_{\mathcal{I}}^T}\mathcal{\tilde A_{\mathcal{I}}} - \Sigma \rVert \\
    & \geq \frac{c}{r} - \sqrt{\frac{\log(8rs/\gamma)}{c'r|\mathcal{I}|}} \vee \frac{\log(8rs/\gamma)}{c''|\mathcal{I}|}. 
\end{align*}
Substituting $|\mathcal{I}| \geq Cr\log(8rs/\gamma)$ for an appropriate constant $C > 0$ gives us that 
\begin{equation}
\label{eq:experimental_design_donor_set_singular_value}
    s_r(\mathcal{\tilde A_{\mathcal{I}}^T}\mathcal{\tilde A_{\mathcal{I}}}/|\mathcal{I}|) \geq \frac{c'''}{r}
\end{equation}
for some universal constant $c''' > 0$ with probability at least $1 - \gamma/8$. Next, observe that $ s_i(\mathcal{\tilde A_{\mathcal{I}}^T}\mathcal{\tilde A_{\mathcal{I}}}/|\mathcal{I}|) =  s_i(\mathcal{ A_{\mathcal{I}}^T}\mathcal{ A_{\mathcal{I}}}/|\mathcal{I}|)$ for any $i \in [r]$. Combining \eqref{eq:experimental_design_donor_set_largest_singular_value_upper_bound} and \eqref{eq:experimental_design_donor_set_singular_value} completes the proof.

\subsection{Proof of Lemma \ref{lem:experiment_design_balanced_spectrum}}
\label{subsec:proof_experiment_design_balanced_spectrum}

We proceed by quoting the following result proved in \ref{subsubsec:experiment_design_pcr_balanced_spectrum_proof}.  
\begin{lemma}
\label{lem:experiment_design_pcr_balanced_spectrum}
Let the set-up of Theorem \ref{thm:experiment_design_assumptions_hold}, and \eqref{eq:experimental_design_donor_set_balanced_spectrum} hold. Then, 
\begin{equation}
\label{eq:experiment_design_pcr_balanced_spectrum}
\frac{C|\mathcal{I}|}{r} \geq s_{1}(\E[\bY_{\mathcal{I}}^{(\Pi_N)}]^T\E[\bY_{\mathcal{I}}^{(\Pi_N)}]/|\Pi_N|) \geq  s_{r}(\E[\bY_{\mathcal{I}}^{(\Pi_N)}]^T\E[\bY_{\mathcal{I}}^{(\Pi_N)}]/|\Pi_N|) \geq \frac{C'|\mathcal{I}|}{r} 
\end{equation}
holds with probability at least $1 - \gamma/8$. 
\end{lemma}

Conditional on \eqref{eq:experiment_design_pcr_balanced_spectrum}, we have that $s_{r}(\E[\bY_{\mathcal{I}}^{(\Pi_N)}]^T\E[\bY_{\mathcal{I}}^{(\Pi_N)}])/s_{1}(\E[\bY_{\mathcal{I}}^{(\Pi_N)}]^T\E[\bY_{\mathcal{I}}^{(\Pi_N)}])\geq c$ for some universal constant $c > 0$. Further, conditional on \eqref{eq:experiment_design_pcr_balanced_spectrum}, we have that $\lVert \E[\bY_{\mathcal{I}}^{(\Pi_N)} ~ | ~ \mathcal{A}_{\mathcal{I}}] \rVert^2_F \geq r s_r(\E[\bY_{\mathcal{I}}^{(\Pi_N)}]^T\E[\bY_{\mathcal{I}}^{(\Pi_N)}]) \geq c' |\Pi_N| |\mathcal{I} |$. Hence, balanced spectrum (Assumption \ref{ass:balanced_spectrum}) holds with probability at least $1 - \gamma/8$, which completes the proof. 

%\begin{align*}
%   \lVert \E[\bY_{\mathcal{I}}^{(\Pi_N)} ~ | ~ \mathcal{A}_{\mathcal{I}}] \rVert^2_F \geq r s_r(\E[\bY_{\mathcal{I}}^{(\Pi_N)}]^T\E[\bY_{\mathcal{I}}^{(\Pi_N)}]) \geq c' |\Pi_N| |\mathcal{I} |
%\end{align*}

\subsection{Proof of Lemma \ref{lem:experiment_design_subspace_inclusion}}
It suffices to show that $\E[\bY_{\mathcal{I}}^{(\Pi_N)}] = [\E[\bY_{\mathcal{I}}^{(\pi)}]: \pi \in \Pi_N] \in \mathbb{R}^{|\Pi_N| \times |\mathcal{I}|}$ has rank $r$ with high probability. This immediately follows from \eqref{eq:experiment_design_pcr_balanced_spectrum}, hence completing the proof. 


\subsubsection{Proof of Lemma \ref{lem:number_nonzeros}}
\label{subsubsec:proof_lemma_nonzeros}
Let $\mathcal{A}_t$ denote the sub-matrix of $\mathcal{A}$ formed by taking the first $t$ rows. Further, let $p'_t$ denote the number of nonzero columns of $\mathcal{A}_t$. We proceed by induction, that is we show for all $t$, 
\begin{equation*}
    p'_t - s \cdot \mathrm{rank}(\mathcal{A}_t) \le 0
\end{equation*}

In the case of $t=1$, either the first row is the zero vector, in which case $p'_1 = \mathrm{rank}(\mathcal{A}_1) = 0$, or else $\mathrm{rank}(\mathcal{A}_1) = 1$ and $p'_1 \leq s$ since the first row is at most $s$-sparse. Then, for general $t$, note that 
\[\mathrm{rank}(\mathcal{A}_{t-1}) \le  \mathrm{rank}(\mathcal{A}_t) \le \mathrm{rank}(\mathcal{A}_{t-1}) + 1.\] If $\mathrm{rank}(\mathcal{A}_{t}) =  \mathrm{rank}(\mathcal{A}_{t-1})$ then we must also have $p'_{t-1} = p'_t$ and the inductive hypothesis holds. 
Otherwise, $\mathrm{rank}(\mathcal{A}_{t}) =  \mathrm{rank}(\mathcal{A}_{t-1}) + 1$. Note that the $t^\text{th}$ row of $\mathcal{A}$ has only $s$ nonzero entries, so $p'_t \leq p'_{t-1} + s$. In this case, we have that 
\begin{align*}
    p'_{t} -  s \cdot \mathrm{rank}(\mathcal{A}_t) & =  p'_{t} -  s \cdot ( \mathrm{rank}(\mathcal{A}_{t-1}) + 1) \\
    & \leq (p'_{t-1} + s) - s \cdot ( \mathrm{rank}(\mathcal{A}_{t-1}) + 1) \leq 0
\end{align*}
where the last inequality holds due to the inductive hypothesis. Since $\mathrm{rank}(\mathcal{A}) = r$, we have that $p' \leq rs$.  
This completes the proof. 

\subsubsection{Proof of Lemma \ref{lem:experiment_design_pcr_balanced_spectrum}}
\label{subsubsec:experiment_design_pcr_balanced_spectrum_proof}

 We begin by establishing that $\E[\bY_{\mathcal{I}}^{(\Pi)}]$ has a balanced spectrum. To proceed, note the since $\bchi(\Pi)/2^{p/2}$ is an orthogonal matrix,  $ \E[\bY_{\mathcal{I}}^{(\Pi)}] = \bchi(\Pi)\mathcal{A}_{\mathcal{I}}^T$, and that multiplication by an orthogonal matrix preserves singular values, we have that
\begin{equation*}
     2^p s_i (\mathcal{A}_{\mathcal{I}}^T\mathcal{A}_{\mathcal{I}})  =  s_i \left(\left(\frac{\bchi(\Pi)}{2^{p/2} }\right)\mathcal{A}_{\mathcal{I}}^T\mathcal{A}_{\mathcal{I}}\left(\frac{\bchi(\Pi)}{2^{p/2}}\right)^T\right) =  s_i(\E[\bY_{\mathcal{I}}^{(\Pi)}]\E[\bY_{\mathcal{I}}^{(\Pi)}]^T)  
\end{equation*}
Using the equation above and by assuming \eqref{eq:experimental_design_donor_set_balanced_spectrum} in the set-up of the Lemma, we have 
\begin{equation}
\label{eq:rth_singular_value_donor_set}
    s_r(\E[\bY_{\mathcal{I}}^{(\Pi)}]^T\E[\bY_{\mathcal{I}}^{(\Pi)}]) = 2^p  s_r (\mathcal{A}_{\mathcal{I}}^T\mathcal{A}_{\mathcal{I}}) \geq \frac{c|\mathcal{I}|2^p}{r}
\end{equation}
for a universal constant $C > 0$. Next, using \eqref{eq:experimental_design_donor_set_balanced_spectrum} again, we have that
\begin{equation}
\label{eq:top_singular_value_donor_set}
 s_1(\E[\bY_{\mathcal{I}}^{(\Pi)}]^T\E[\bY_{\mathcal{I}}^{(\Pi)}]) = 2^p s_1 (\mathcal{A}_{\mathcal{I}}^T\mathcal{A}_{\mathcal{I}})  \leq \frac{C|\mathcal{I
 }|2^p}{r}
\end{equation}
Putting both \eqref{eq:rth_singular_value_donor_set} and \eqref{eq:top_singular_value_donor_set} together, we get
\begin{equation}
\label{eq:subspace_inclusion_singular_value_bound}
    \frac{C|\mathcal{I}|2^p}{r} \ge s_1(\E[\bY_{\mathcal{I}}^{(\Pi)}]^T\E[\bY_{\mathcal{I}}^{(\Pi)}]) \ge s_r(\E[\bY_{\mathcal{I}}^{(\Pi)}]^T\E[\bY_{\mathcal{I}}^{(\Pi)}])  \ge  \frac{c|\mathcal{I}|2^p}{r}
\end{equation}
which establishes that $\E[\bY_{\mathcal{I}}^{(\Pi)}]$ has a balanced spectrum. 
Further, observe that for $\pi \in \Pi$ chosen uniformly at random, we have
\begin{equation*}
    \E\left[\E[\bY_{\mathcal{I}}^{(\pi)}]^T\E[\bY_{\mathcal{I}}^{(\pi)}]\right] = \frac{1}{2^p}\sum_{\pi \in \Pi} \E[\bY_{\mathcal{I}}^{(\pi)}]^T\E[\bY_{\mathcal{I}}^{(\pi)}] = \E[\bY_{\mathcal{I}}^{(\Pi)}]^T\E[\bY_{\mathcal{I}}^{(\Pi)}]/2^p
\end{equation*}
where the outer expectation is taken with respect to the randomness in choosing $\pi$ uniformly and independently at random from $\Pi$. Then, we can apply Theorem \ref{thm:matrix_bernstein} with $X = \E[\bY_{\mathcal{I}}^{(\Pi_N)}]$, $m = \lVert \E[\bY_{\mathcal{I}}^{\pi}] \rVert \leq \sqrt{|\mathcal{I}|}$ for any $\pi \in \Pi$, $\Sigma = \E[\bY_{\mathcal{I}}^{(\Pi)}]^T\E[\bY_{\mathcal{I}}^{(\Pi)}]/2^p$, $||\Sigma|| = C|\mathcal{I}|/r$ (obtained from \eqref{eq:top_singular_value_donor_set}), $t = \sqrt{\log(8|\mathcal{I}|/\gamma)/c'}$ to give us that 
\begin{equation}
\label{eq:matrix_bernstein_subspace_inclusion_bound}
     \left \lVert \frac{1}{|\Pi_N|}\E[\bY_{\mathcal{I}}^{(\Pi_N)}]^T\E[\bY_{\mathcal{I}}^{(\Pi_N)}] - \Sigma \right \rVert \leq \sqrt{\frac{|\mathcal{I}|^2\log(8|\mathcal{I}|/\gamma)}{c'r|\Pi_N|}} \vee \frac{|\mathcal{I}|\log(8|\mathcal{I}|/\gamma)}{c''|\Pi_N|}
\end{equation}
holds with probability at least $1 - \gamma/8$.  Hence, we have that the following holds with probability at least $1 - \gamma/8$. 
\begin{equation*}    s_1(\E[\bY_{\mathcal{I}}^{(\Pi_N)}]^T\E[\bY_{\mathcal{I}}^{(\Pi_N)}]/|\Pi_N|) \leq \frac{C|\mathcal{I}|}{r} + \sqrt{\frac{|\mathcal{I}|^2\log(8|\mathcal{I}|/\gamma)}{c'r|\Pi_N|}} \vee \frac{|\mathcal{I}|\log(8|\mathcal{I}|/\gamma)}{c''|\Pi_N|}
\end{equation*}
Choosing $|\Pi_N|\geq Cr\log(|\mathcal{I}|/\gamma)$ for appropriate constant $C > 0$, gives us 
\begin{equation}
\label{eq:experiment_design_top_singular_value_balanced_spectrum_bound}
s_1(\E[\bY_{\mathcal{I}}^{(\Pi_N)}]^T\E[\bY_{\mathcal{I}}^{(\Pi_N)}]/|\Pi_N|) \leq \frac{C|\mathcal{I}|}{r}
\end{equation}

with probability at least $1 - \gamma/8$. Next, we lower bound $s_r$ as follows. Using Theorem \ref{thm:weyl_inequality}, \eqref{eq:subspace_inclusion_singular_value_bound}, and \eqref{eq:matrix_bernstein_subspace_inclusion_bound}, we have 
\begin{align*}
s_{r}(\E[\bY_{\mathcal{I}}^{(\Pi_N)}]^T\E[\bY_{\mathcal{I}}^{(\Pi_N)}]/|\Pi_N|) & \geq s_{r}(\E[\bY_{\mathcal{I}}^{(\Pi)}]^T\E[\bY_{\mathcal{I}}^{(\Pi)}]/2^p|) -   \lVert \frac{1}{|\Pi_N|}\E[\bY_{\mathcal{I}}^{(\Pi_N)}]^T\E[\bY_{\mathcal{I}}^{(\Pi_N)}] - \Sigma \rVert \\
& \geq \frac{c|\mathcal{I}|}{r} - \sqrt{\frac{|\mathcal{I}|^2\log(8|\mathcal{I}|/\gamma)}{c'r|\Pi_N|}} \vee \frac{|\mathcal{I}|\log(8|\mathcal{I}|/\gamma)}{c''|\Pi_N|} \\
\end{align*}
Using $|\Pi_N|\geq Cr\log(|\mathcal{I}|/\gamma)$ for an appropriate constant $C>0$ implies
\begin{equation}
\label{eq:experiment_design_min_singular_value_balanced_spectrum_bound}
s_{r}(\E[\bY_{\mathcal{I}}^{(\Pi_N)}]^T\E[\bY_{\mathcal{I}}^{(\Pi_N)}]/|\Pi_N|) \geq \frac{C'|\mathcal{I}|}{r} 
\end{equation}
with probability at least $ 1- \gamma/8$. Combining \eqref{eq:experiment_design_top_singular_value_balanced_spectrum_bound}, and \eqref{eq:experiment_design_min_singular_value_balanced_spectrum_bound} completes the proof. 


%$ s_{r}(\E[\bY_{\mathcal{I}}^{(\Pi_N)}]^T\E[\bY_{\mathcal{I}}^{(\Pi_N)}]/|\Pi_N|) > 0$ with probability at least $1 - \gamma/8$. This completes the proof. 