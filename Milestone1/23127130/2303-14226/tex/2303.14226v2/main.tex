\documentclass[12pt]{article}

\usepackage{amsmath,amsfonts,amssymb}
\usepackage{graphicx,psfrag,epsf,comment,float,lipsum,titlefoot}
\usepackage{enumerate}
\usepackage{enumitem}
\usepackage{bbm}
\usepackage{mathtools}

% \usepackage{natbib}
\usepackage[numbers]{natbib}
\bibliographystyle{agsm}
\usepackage{authblk}


\usepackage[linktoc=all,colorlinks=true,linkcolor=black,citecolor=blue,urlcolor=blue,bookmarks=false,hypertexnames=true]{hyperref}
\usepackage{url}

\usepackage{xcolor}
\usepackage{algorithm2e}
\usepackage{mathtools}

\usepackage{subcaption}
\usepackage[font=footnotesize,labelfont=bf]{caption} %footnotesizeL

% for plots
\usepackage{tabularx}
\usepackage{array}

% for tables
\usepackage{booktabs}
\usepackage{longtable}
\usepackage{colortbl}

% for multiple bibliography files
\usepackage[resetlabels]{multibib}
\newcites{appendix}{Appendix References}

% DON'T change margins - should be 1 inch all around.
\addtolength{\oddsidemargin}{-.5in}%
\addtolength{\evensidemargin}{-1in}%
\addtolength{\textwidth}{1in}%
\addtolength{\textheight}{1.7in}%
\addtolength{\topmargin}{-1in}%
% \usepackage[margin=1in]{geometry}
\usepackage[subtle]{savetrees}

% reduce spacing between bibtex entries
\setlength{\bibsep}{0pt plus 0.3ex}

% for icons
\usepackage{fontawesome}

\usepackage{pifont}% http://ctan.org/pkg/pifont
\newcommand{\cmark}{\ding{51}}%
\newcommand{\xmark}{\ding{55}}%

%\pdfminorversion=4
% NOTE: To produce blinded version, replace "0" with "1" below.
\newcommand{\blind}{1}

% Bold symbols

\newcommand{\bl}{\mathbf{l}}
\newcommand{\bv}{\mathbf{v}}
\newcommand{\bw}{\mathbf{w}}
\newcommand{\bx}{\mathbf{x}}
\newcommand{\by}{\mathbf{y}}
\newcommand{\bz}{\mathbf{z}}
\newcommand{\bV}{\mathbf{V}}
\newcommand{\bU}{\mathbf{U}}
\newcommand{\bS}{\mathbf{S}}


\newcommand{\bbeta}{\boldsymbol{\beta}}
\newcommand{\bepsilon}{\boldsymbol{\epsilon}}
\newcommand{\bdelta}{\boldsymbol{\delta}}
\newcommand{\bchi}{\boldsymbol{\chi}}
\newcommand{\beps}{\boldsymbol{\epsilon}}
\newcommand{\blambda}{\boldsymbol{\lambda}}
\newcommand{\bpi}{\boldsymbol{\pi}}
\newcommand{\bmu}{\boldsymbol{\mu}}
\newcommand{\bnu}{\boldsymbol{\nu}}
\newcommand{\balpha}{\boldsymbol{\alpha}}

\newcommand{\bW}{\mathbf{W}}
\newcommand{\bX}{\mathbf{X}}
\newcommand{\bY}{\mathbf{Y}}
\newcommand{\bZ}{\mathbf{Z}}
\newcommand{\bI}{\mathbf{I}}


% Common symbols
\newcommand{\R}{\mathbb{R}}
\renewcommand{\P}{\mathbb{P}}
\newcommand{\E}{\mathbb{E}}
\newcommand{\indicator}{\mathbf{1}}
\newcommand{\Var}{\textnormal{Var}}
\newcommand{\Cov}{\textnormal{Cov}}
\newcommand{\KL}{\textnormal{KL}}
\DeclareMathOperator*{\argmin}{arg\,min}
\newcommand{\indep}{\perp \!\!\! \perp}

%
\newcommand{\method}{Synthetic Combinations}

% Paired operators
%\DeclarePairedDelimiter{\braces}{\lbrace}{\rbrace}
%\DeclarePairedDelimiter{\paren}{(}{)}
%\DeclarePairedDelimiter{\floor}{\lfloor}{\rfloor}
%\DeclarePairedDelimiter{\abs}{|}{|}
%\DeclarePairedDelimiter{\norm}{\|}{\|}
%\DeclarePairedDelimiter{\inprod}{\langle}{\rangle}

% Specialized symbols
\newcommand{\data}[1][n]{\mathcal{D}_{#1}}
\newcommand{\semiempirical}[1][n]{\tilde{\mu}_{#1}}
\newcommand{\partition}{\mathfrak{p}}
\newcommand{\collection}{\mathfrak{C}}
\newcommand{\cell}{\mathfrak{t}}
\newcommand{\operator}{\mathcal{T}}
\newcommand{\node}{\mathfrak{t}}
\renewcommand{\dim}{d}
\newcommand{\nsamples}{n}
\newcommand{\coordindices}{[\dim]}
\newcommand{\sampindices}{[\nsamples]}
\newcommand{\hypercube}[1][d]{\braces{0,1}^{#1}}
\newcommand{\uniform}{\mu}
\newcommand{\measure}{\nu}
\newcommand{\risk}{\mathcal{R}}
\newcommand{\oraclerisk}{\mathcal{R}^*}
\newcommand{\vol}{\textnormal{Vol}}
\newcommand{\hamball}{B}
% Misc
\newcommand\blfootnote[1]{%
  \begingroup
  \renewcommand\thefootnote{}\footnote{#1}%
  \addtocounter{footnote}{-1}%
  \endgroup
}
\newcommand{\mybinom}[2]{\Bigl(\begin{array}{@{}c@{}}#1\\#2\end{array}\Bigr)}


% Editing macros
%\newcommand{\comment}[1]{\textcolor{red}{#1}}

\definecolor{cyan}{cmyk}{1, 0.2, 0, 0} 
\newcommand{\ndaa}[1]{{\color{cyan}[Abhi: #1]}}

\newcommand\independent{\protect\mathpalette{\protect\independenT}{\perp}}
\def\independenT#1#2{\mathrel{\rlap{$#1#2$}\mkern2mu{#1#2}}}

\newtheorem{theorem}{Theorem}[section]
\newtheorem{lemma}[theorem]{Lemma}
\newtheorem{proposition}[theorem]{Proposition}
\newtheorem{corollary}[theorem]{Corollary}

\newtheorem{definition}[theorem]{Definition}
\newtheorem{assumption}[theorem]{Assumption}
\newtheorem{remark}[theorem]{Remark}
\newtheorem{proof}[theorem]{Proof}
\DeclarePairedDelimiter{\braces}{\lbrace}{\rbrace}
\begin{document}

\def\spacingset#1{\renewcommand{\baselinestretch}%
{#1}\small\normalsize} \spacingset{1}


%%%%%%%%%%%%%%%%%%%%%%%%%%%%%%%%%%%%%%%%%%%%%%%%%%%%%%%%%%%%%%%%%%%%%%%%%%%%%%

\if1\blind
{\title{\vspace{-10mm}\bf \method: A Causal Inference Framework for Combinatorial Interventions}
  \author{
    \normalsize Abhineet Agarwal \vspace{-1em}\\
    \normalsize Department of Statistics, University of California, Berkeley\\
    \vspace{0.5em}
    Anish Agarwal\footnote{Part of this work was completed while Anish was a post-doc at Amazon, Core AI.}\\
    \normalsize Department of Industrial Engineering and Operations Research, Columbia University\\
   \vspace{0.5em}
    Suhas Vijaykumar\\
    \normalsize Amazon, Core AI
    }
  \maketitle
} \fi

\if0\blind
{
  \bigskip
  \bigskip
  \bigskip
  \begin{center}
    {\LARGE\bf \method: A Causal Inference Framework for Combinatorial Interventions}
\end{center}
  \medskip
} \fi

\begin{abstract}
\label{sec:abstract}
Consider a setting where there are $N$ heterogeneous units and $p$ interventions.
%
Our goal is to learn unit-specific potential outcomes for any combination of these $p$ interventions, i.e., $N \times 2^p$ causal parameters. 
%
Choosing a combination of interventions is a problem that naturally arises in a variety of applications such as factorial design experiments, recommendation engines, combination therapies in medicine, conjoint analysis, etc.  
%
Running $N \times 2^p$ experiments to estimate the various parameters is likely expensive and/or infeasible as $N$ and $p$ grow.
%
Further, with observational data there is likely confounding, i.e., whether or not a unit is seen under a combination is correlated with its potential outcome under that combination.
%
To address these challenges, we propose a novel latent factor model that imposes structure across units (i.e., the matrix of potential outcomes is approximately rank $r$), and combinations of interventions (i.e., the coefficients in the Fourier expansion of the potential outcomes is approximately $s$ sparse). 
%
We establish identification for all $N \times 2^p$ parameters despite unobserved confounding.
%
We propose an estimation procedure, Synthetic Combinations, and establish it is finite-sample consistent and asymptotically normal under precise conditions on the observation pattern.
%
Our results imply consistent estimation given $\text{poly}(r) \times \left( N + s^2p\right)$ observations, while previous methods have sample complexity scaling as $\min(N \times s^2p, \ \ \text{poly(r)} \times (N + 2^p))$.
%
We use Synthetic Combinations to propose a data-efficient experimental design. 
%
Empirically, Synthetic Combinations outperforms competing approaches on a real-world dataset on movie recommendations. 
%
Lastly, we extend our analysis to do causal inference where the intervention is a permutation over $p$ items (e.g., rankings). 
%

 
\end{abstract}
%is based on a class of new predictive models, \rfmethod,
\noindent%
{\it Keywords:} Matrix Completion, Fourier Analysis of Boolean Functions, Latent Factor Models, Synthetic Controls, Learning to Rank
\vfill

\newpage
\spacingset{1.75}
% \spacingset{1.9} % DON'T change the spacing!

\section{Introduction}

The increasing complexity of source code poses a key challenge to the reliability of large-scale software systems. Software bugs in these systems can lead to safety issues~\cite{bug_safety} for users around the world as well as cause non-negligible financial losses~\cite{bug_loss}. As such, developers have to spend a large amount of time and effort on bug fixing. Consequently, \aprfull (\apr), designed to automatically generate patches to fix software bugs, has attracted wide attention from both academia and industry~\cite{long2016prophet, legoues2012genprog, long2015spr, lou2020can, tufano2018empstudy}. 


To achieve \apr, one popular approach is known as Generate-and-Validate (G\&V)~\cite{qi2015gv, ghanbari2019prapr, lou2020can, le2016hdrepair, legoues2012genprog, wen2018capgen, hua2018sketchfix, martinez2016astor, koyuncu2020fixminder, liu2019tbar, liu2019avatar}, which is typically based on the following pipeline: First, fault localization techniques~\cite{wong2016fl, abreu2007ochiai, zhang2013injecting, papadakis2015metallaxis, li2019deepfl, li2017transforming} are applied to determine the suspicious locations in programs where bugs are likely to exist. Then, the buggy locations are used by the \apr tools to generate a list of patches that replace buggy lines with correct lines. Afterward, each patch is validated against the original test suite to identify any \emph{plausible patches} (i.e., passing all tests in the test suite). Finally, to determine the \emph{correct patches}, developers examine the list of plausible patches to see if any of them can correctly fix the bug. 

Traditional \apr tools can mainly be categorized into heuristic-based~\cite{legoues2012genprog, le2016hdrepair, wen2018capgen}, constraint-based~\cite{mechtaev2016angelix, le2017s3, demacro2014nopol, long2015spr} and \template~\cite{ghanbari2019prapr, hua2018sketchfix, martinez2016astor, liu2019tbar, liu2019avatar}. Among these traditional tools, \template \apr tools~\cite{ghanbari2019prapr, liu2019tbar, benton2020effectiveness} have been able to achieve state-of-the-art results. \Template \apr tools typically leverage pre-defined templates (e.g., adding a nullness check) for bug fixing. However, since these fix templates are typically handcrafted, the number and types of bugs they are able to fix can be limited. 



To address the limitations of traditional \apr, researchers have proposed various \learning \apr tools~\cite{li2020dlfix, chen2018sequencer, jiang2021cure, lutellier2020coconut, zhu2021recoder, ye2022rewardrepair} based on the \nmtfull (\nmt) architecture~\cite{sutskever2014mt} where the input is the buggy code snippets and the goal is to translate the buggy code snippets into a fixed version. To accomplish this, \learning \apr tools require supervised training datasets with pairs of both buggy and fixed code snippets in order to learn how to perform this translation step. These training data are usually obtained by mining historical bug fixes using heuristics/keywords~\cite{dallmeier2007benchmark}, which can be imprecise for identifying bug-fixing commits; even the actual bug-fixing commits can include irrelevant code changes, leading to further pollution in the dataset~\cite{xia2022alpharepair}.
% 
Moreover, it can be hard for such \apr tools to generalize and fix bug types unseen during training. 



To better leverage recent advances in \plmfull{s} (\plm{s}), researchers~\cite{xia2022alpharepair, xia2023repairstudy, kolak2022patch, prenner2021codexws} have directly applied \plm{s} to generate patches without bug-fixing datasets. These \llm-based \apr tools work by either directly generating a complete code function~\cite{prenner2021codexws, xia2023repairstudy} or predict/infill the correct code snippet given its surrounding context~\cite{xia2022alpharepair, xia2023repairstudy}. By directly using \llm{s} that are pre-trained on billions of open-source code snippets, \llm-based \apr tools can achieve state-of-the-art performance on many repair datasets~\cite{xia2022alpharepair}. 


% 
%
%

Traditional \apr tools have long used the insight of the \emph{plastic surgery hypothesis}~\cite{barr2014plastic} where it states that the code ingredients to fix a bug already exist within the same project. Traditional \apr tools have manually designed pattern-~\cite{ghanbari2019prapr, saha2017elixir} or heuristic-based~\cite{jiang2018simfix, legoues2012genprog} approaches to finding and using such relevant code ingredients to generate fixes for bugs. However, the plastic surgery hypothesis has been largely ignored in \llm-based \apr. In fact, \llm provides a unique opportunity to fully automate the plastic surgery hypothesis idea via fine-tuning (learning project-specific information via model updates from the buggy project) and prompting (directly providing relevant code ingredients to the model), and make it directly applicable to different languages (since the \llm{s} are typically multi-lingual).%
Moreover, despite the intensive manual efforts involved, traditional \apr tools still cannot fully leverage project-specific information due to large search space for leveraging/composing existing code ingredients. In contrast, the project-specific information can effectively leveraged by \llm{s} due to their power in code understanding/vectorization, e.g., even partial/imprecise information may still guide \llm{s} in correct patch generation!
 To this end, we ask the question: \emph{How useful is the plastic surgery hypothesis in the era of \plm{s}}?








\mypara{Our Work.} To answer the question, we present \ourtech{\xspace} -- a \llm-based approach that automatically utilizes the plastic surgery hypothesis by systematically combining multiple fine-tuning and prompting strategies for \apr. \ourtech fine-tunes \plm{s} using two novel domain-specific training strategies: \textbf{\epfinetune} -- we fine-tune using the original buggy project by aggressively masking out a high percentage of tokens, which allows \plm to learn project-specific code tokens and programming styles; and \textbf{\rofinetune} -- which only masks out a single continuous code sequence per training sample, allowing the model to get used to the final \csapr task of predicting a single continuous code sequence. Furthermore, we directly leverage the ability for \plm{s} to understand natural language instructions and introduce a novel prompting strategy, \textbf{\idprompting}, which uses information retrieval and static analysis to obtain a list of relevant identifiers for the buggy lines. While such relevant identifiers are critical for fixing some difficult bugs, they may not be seen by the \llm during inference due to limited context window size. Through the use of prompting, we directly tell the model to use these extracted identifiers (relevant code ingredients) to generate the correct code. Finally, to perform repair, we combine all four model variants (including the base model, both fine-tuned models and the base model with prompting) for the final repair.





While our insight of leveraging the plastic surgery hypothesis for \llm-based \apr is generalizable across different types of \plm{s}, to implement \ourtech, we choose a recent \plm{\xspace}, \ctfive~\cite{wang2021codet5}, which is pre-trained on millions of open-source code snippets. \ctfive is an encoder-decoder model trained using \mspfull (\msp) objective where a percentage of tokens are masked out and each continuous masked token sequence is referred to as a masked span. Also, although we only extract relevant identifiers from the current buggy project (since this paper focuses on the plastic surgery hypothesis), our work can be easily extended to obtain other code information (such as relevant statements or functions) from other sources, such as  the massive pre-training corpora~\cite{husain2020codesearchnet} or historical bug-fixing datasets~\cite{jiang2019infer}, which can provide more coding knowledge for \llm{s}. Besides, although we mainly focus on using traditional string comparison algorithms for information retrieval in this paper, these techniques can be easily replaced by other frequency-based retrieval~\cite{robertson2009probabilistic} and neural search (or embedding-based search)~\cite{reimers2019sentence}.
  In summary, this paper makes the following contributions:


%


\begin{itemize}[noitemsep, leftmargin=*, topsep=0pt]
    \item \textbf{Dimension.} This paper is the first to revisit the important plastic surgery hypothesis in the era of \llm{s}. It opens up a new dimension for \llm-based \apr to incorporate previously neglected information from the buggy project itself to boost \apr performance. Furthermore, it demonstrates the promising future of retrieval-based prompting for modern \llm-based \apr.
    \item \textbf{Implementation.} We implement \ourtech based on the recent \ctfive model. We augment the model using two novel fine-tuning strategies: \epfinetune and \rofinetune, along with a novel prompting strategy based on information retrieval and static analysis: \idprompting. We combine the patches generated by all four models together and perform patch ranking to speed up \apr.% 
    \item \textbf{Evaluation Study.} We conduct an extensive evaluation against state-of-the-art \apr tools. On the widely studied \dfj 1.2 and 2.0 datasets~\cite{just2014dfj}, \ourtech is able to achieve the new state-of-the-art results of 89 and 44 correct bug fixes (15 and 8 more than best baseline) respectively.  Furthermore, we perform a broad ablation study to justify our design. \ourtech demonstrates for the first time that the plastic surgery hypothesis can substantially boost \llm-based \apr and advance state-of-the-art \apr, while being fully automated and general. Moreover, even partial/imprecise code ingredients may still effectively guide \llm{s} for \apr!
\end{itemize}


\section{Related work}
% There is extensive recent work on speeding up analytical queries due to the need for consistent execution times in the face of the explosive growth in the volume of available data.
% In this section, we divide existing work into two categories: maintaining data freshness in MVs (\Cref{sec:server_side}) and optimizations for minimizing ad-hoc query latency (\Cref{sec:client_side}).

% \subsection{Maintaining Data Freshness in MVs}
% \label{sec:server_side}
% There exists a variety of data warehousing applications aimed at supporting low-latency analytical queries on fresh data.
% In particular, these applications require efficiency in the propagation of newly ingested data into downstream MVs.
 
\mypara{Efficient MV Refresh}
Incremental view maintenance (IVM) aims to update MVs to reflect newly ingested data, taking advantage of already computed results to perform the update in a manner more efficient than computing from scratch (full refresh)
~\cite{ahmad2012dbtoaster,mcsherry2013differential,armbrust2013generalized,zeng2016iolap, palpanas2002incremental, griffin1995incremental, agiwal2021napa, braun2015analytics}. 
There is an abundance of work in IVM, including incremental updates on duplicate values~\cite{griffin1995incremental}, non-distributive aggregate functions~\cite{palpanas2002incremental}, higher-order views~\cite{ahmad2012dbtoaster}, and sliding windows~\cite{braun2015analytics}. 
More recent works also investigate the scalability aspect of IVM, proposing scale-independent updates~\cite{armbrust2013generalized} and sampled views~\cite{zeng2016iolap}. Since \system is applicable to arbitrary SQL statements, \system is orthogonal to and is fully compatible with existing IVM techniques.

\mypara{MV Refresh Scheduling}
There exist works on scheduling the refresh of a MV set focusing on resolving cyclic dependencies~\cite{folkert2005optimizing}, minimizing weighted average staleness~\cite{golab2009scheduling}, and prioritizing MVs with the highest speedups on predicted future queries~\cite{ahmed2020automated}.
\system's scheduling to speed up the end-to-end refresh of the MV set is not addressed in existing works.

\mypara{DAG Workflow Scheduling}
The execution of workloads consisting of individual jobs with acyclic dependencies is a well-studied topic~\cite{apacheoozie,sparkdag,marchal2018parallel,bathie2020revisiting,baruah2022ilp}; many of these techniques can be applied to MV refresh runs studied in this paper.
Existing workflow scheduling systems such as Apache Oozie~\cite{apacheoozie}, Apache Airflow~\cite{airflow}, and Spark DAG scheduler~\cite{sparkdag} automate the execution of user-defined workflows following a topological order.
There are recent works aimed at finding more optimal execution orders in terms of peak memory usage~\cite{marchal2018parallel, bathie2020revisiting} and execution time on parallel platforms~\cite{baruah2022ilp}.
While \system is designed for use with MV refresh runs/workloads, our technique on joint scheduling and optimization can be reasonably applied to general workloads as a possible future direction.

% \paragraph{Incremental MV indexing}
% Update-optimized indices such as the log-structured merge-trees (LSM)~\cite{o1996log} are used for indexing MVs due to frequent updates induced by data ingestion~\cite{gupta2016mesa,agiwal2021napa}.
% \system is orthogonal to indexing: \system is capable of efficiently performing MV refresh runs regardless of whether the individual MVs are indexed or not.

% \subsection{Ad-hoc Query Latency Reduction}
% \label{sec:client_side}

% The minimization of ad-hoc analytical query response times is a well-studied topic due to latency being negatively correlated with the productivity of a data analyst during a data analysis session~\cite{liu2014effects}.
% Sessions are commonly conducted within visualization systems that contain a variety of optimization techniques to ensure that query response times fall within a certain latency tolerance.

% \mypara{Data prefetching}
% Data is often loaded into memory on a by-need basis in visualization systems to minimize interference with user-issued query computations~\cite{mani2017effective,xin2021enhancing,galakatos2017revisiting, yan2020auto, battle2016dynamic, crotty2016case, jalaparti2018netco}. 
% Query-time data retrieval can be significantly expedited by anticipating the data usage of the user in future queries and pre-loading the data into memory during the downtime between user queries (`think time'). SMART~\cite{mani2017effective} prefetches data for modified versions of current user-issued queries with different filters and dimensions. A-WARE~\cite{crotty2016case} maintains a sample store constantly refined through ingesting data based on speculations of future plots.
% ForeCache~\cite{battle2016dynamic} uses an SVM to predict the user's current analysis phase and accordingly prefetches data tiles partitioned based on different numerical values. NetCo predicts future queries via log analysis, and solves an ILP formulation to prefetch data to maximize the number of SLO-meeting queries~\cite{jalaparti2018netco}.
% In the case of MV refresh workloads, `think time' is nonexistent as individual MVs are refreshed back-to-back, rendering data prefetching techniques non-applicable.

\mypara{Intermediate Data Caching}
Some existing data visualization systems cache user-defined variables to support the typical incremental construction of data visualizations~\cite{zgraggen2016progressive, eichmann2020idebench} during data analysis sessions~\cite{jupyter, rstudio, colab}. 
Recent work proposes a management scheme for these cached variables under a memory constraint that greedily keeps variables with the highest estimated time savings based on predicted future user behavior via neural networks~\cite{xin2021enhancing}.
While useful for data visualization, a greedy approach to memory management fails to achieve satisfactory results compared to \system.

\mypara{Intermediate Result Reuse}

There exist works on storing intermediate results from computations to speedup future computations~\cite{yang2018intermediate, dursun2017revisiting, nagel2013recycling, michiardi2019memory, galakatos2017revisiting}.
Studied topics include the identification of reuse opportunities by finding overlaps in computation graphs of successive jobs~\cite{yang2018intermediate, michiardi2019memory},
selective storage under a space constraint with heuristics such as reuse probability~\cite{dursun2017revisiting}, expected savings~\cite{yang2018intermediate}, and recompute-storage cost difference~\cite{nagel2013recycling},
and rewriting incoming jobs to take advantage of stored intermediates~\cite{galakatos2017revisiting}.
These works share similarity with \system in their selection of items to store under a memory constraint, however, \system's problem setting requires it to uniquely consider the joint (re)ordering of job executions along with the selection of items.

% work that considers both job execution (re)order as well as intermediate result caching with a bounded amount of memory. but notably lack the joint aspect of \system and cannot be used to achieve immediate speedup on an incoming MV refresh run if no intermediates are stored beforehand. 

\mypara{Incremental Query Processing} Incremental processing (IQP) is useful for cases where not all data required for a query is immediately available. Similar to online aggregation~\cite{hellerstein1997online}, initial results of a query are computed on a subset of required data and progressively refined as the rest of the required data arrives in a predictable pattern~\cite{tang2019intermittent,wangtempura}. Tang et al. propose a dynamic programming formulation to pick intermediate states to store in memory given a limited memory budget~\cite{tang2019intermittent}. Tempura rewrites the query plan for more efficient execution based on predicted data arrival patterns~\cite{wangtempura}. While similarities exist between the problem setting of IQP and \system, such as management of bounded memory, \system notably includes additional joint optimization for the order of MV updates.

% \paragraph{Sampling}
% Sampling has seen wide use in visualization systems for reducing the computation time of ad-hoc queries by computing an approximate result over a subset of data as exact results are not always required by the user~\cite{crotty2016case, mani2017effective, zgraggen2014panoramicdata, kraska2021northstar, galakatos2017revisiting, kandula2016quickr}. 
% Commonly studied topics in sampling for ad-hoc queries include complex query sampling~\cite{kandula2016quickr}, rare event aggregation~\cite{kraska2021northstar, galakatos2017revisiting}, and maintaining consistency between related sampled visualizations~\cite{zgraggen2014panoramicdata}.
% Sampling server-side at the MV level compromises the assumptions of downstream applications and is thus not considered in \system.

% \paragraph{Progressive visualization}
% The latency tolerance for time-consuming queries can be circumvented by presenting a partially-computed visualization to the user within the tolerance, which is then incrementally refined until it is fully accurate~\cite{rahman2017ve, zgraggen2016progressive, crotty2015vizdom, kraska2021northstar, kamat2017infiniviz}.
% Example plots which benefit from progressive visualization include bar charts~\cite{kamat2017infiniviz} and heatmaps~\cite{rahman2017ve}.
% Similar to sampling, study on this topic is orthogonal to \system as pushing out partially-updated MVs compromises downstream assumptions.
\section{Proposed Framework: {\ourmodel}}
\label{model}


In this section, we introduce a novel self-supervised co-training framework {\ourmodel}.
The proposed framework is illustrated in Figure~\ref{fig:intro_model} and works in three phases.
Phase one automatically generates two sets of pseudo labels.
We use a combination of off-the-shelf pre-trained POS and NER taggers, knowledge graph, and GPT-2 scorer for generating the first set of pseudo labels automatically without any hand-crafted rules for matching the slot values.
The other set of pseudo labels is acquired through a zero-shot slot filling model~\cite{liu2020coach}, trained on the out-of-domain dataset.
It is critical to emphasize that both sets of labels are noisy and incomplete which poses serious challenges to training effective models for the task of open-domain slot filling.
Phase two fine-tunes the pre-trained BERT to the slot filling task that effectively transfers the knowledge from the pre-trained language model~(LM) to overcome the issue of label incompleteness to some extent. 
Further, we employ the early stopping technique to minimize the noise in the labels.
The output of this phase is two BERT models that can generate soft labels for self-supervision during co-training in phase three.
Phase three leverages the fine-tuned models and further trains them in an iterative fashion.
Specifically, the proposed peer training approach facilitates high-confidence soft label selection for the other peer to perform training. This phase progressively reduces the noise in the labels and enables effective model fitting. 



\subsection{Phase One: Automatic Label Generation}
To acquire the first set of labels, we perform the following steps.
First of all, off-the-shelf trained POS and NER taggers are used to predict initial estimates of the slot values irrespective of the slot types. Then, the type information of the slot values is queried from the KG and the slot value is tagged for the most appropriate slot in the target domain.
This approach, however, produces low recall. 
To expand the candidate slot values, we generate n-grams of the natural language text and employ a partial matching scheme to query the KG for type information (e.g., \myspecial{Jason} \myspecial{Aldean} = \myspecial{American} \myspecial{singer}) of the n-grams if the entry exists.
This process generates multiple overlapping hypotheses about the slot values.
We replace a span of text that corresponds to a slot value by its type information and a GPT-2 based scorer (see Section~\ref{sec:nlpmodels}) is used to select the best candidate based on the fluency of the text.
Naturally, if a token (or span of tokens) is replaced by its type, the sentence should score higher as compared to the case where an inappropriate substitution is performed. 
We select the best hypothesis if the score is greater than the threshold.
Intuitively, the candidate selection threshold can automatically be searched based on a small validation set from the target domain, making the label generation process fully automatic. 
The other set of noisy labels is acquired by the zero-shot slot filling model~\cite{liu2020coach} that has been trained using an out-of-domain dataset. It is important to highlight that the zero-shot slot filling model does not require any labeled in-domain training example. 
To summarize the automatic label generation phase, both sets of labels are acquired in a fully automatic fashion without any hand-crafting.


In contrast to previous work in weak supervision~\cite{ren2015clustype,he2017autoentity,fries2017swellshark,giannakopoulos2017unsupervised} that obtains a single set of noisy labels and then propose techniques to overcome the challenge of fitting an effective model to the noisy labels, we acquire two sets of complementary labels.
The choice of these two sets of labels is guided by the intuition that they should be complementary and the models trained on these sets of labels should be able to share complementary information with the other to improve the performance in the later phases of the framework.
Essentially, the first set of labels carries information from external knowledge sources, whereas the labels generated through the pre-trained zero-shot slot filling model capture how the slot values are mentioned in other domains.
%
To further elaborate on the motivation and our process for the first set of labels (i.e., labels using KG and other NLP models), the pre-trained LMs have been shown to have a great deal of knowledge~\cite{petroni2019language}, thus should be capable of generating automatic labels with no need of external KG. 
To the best of our knowledge, there exists no work that shows that accurate token-level automatic labeling (e.g., slot filling task) is possible with pre-trained LMs. 
Moreover, such approaches would require heavy prompting in each new target domain, whereas our label generation process is fully automatic and only relies on the readily-available pre-trained NLP models and external KG.

\subsection{Phase Two: LM-assisted Weak Supervision}
Since we do not have access to dataset $\{(\mathbf{X}_n,\mathbf{Y}_n)\}_{n=1}^N$ with true ground-truth labels.
We use pseudo labels generated in phase one, $\{(\mathbf{X}_n,\mathbf{D}_n)\}_{n=1}^N$, to learn 
$f_{m,c}(\cdot; \cdot)$ that outputs the probability of the $m$-th token to take on class $c$. 
We learn $f_{m,c}(\cdot; \cdot)$ by minimizing the following loss over the noisy dataset $\{(\mathbf{X}_n,\mathbf{D}_n)\}_{n=1}^N$: 
$$
\hat\theta = \argmin_{\theta}\frac{1}{N}\sum_{n=1}^{N} \ell(\mathbf{D}_n, f(\mathbf{X}_{n}; \theta)),
\label{eq:stage1}
$$
where $\ell(\mathbf{D}_n, f(\mathbf{X}_{n}; \theta)) = \frac{1}{M} \sum_{m=1}^{M} -\log{f_{m,d_{n, m}}(\mathbf{X}_{n}; \theta)}$. 
We employ the pre-trained multilingual BERT with token-level classification head that uses Adam optimizer \cite{kingma2014adam,Liu2019} with early stopping and multiple random initializations. 


Since slot filling task is similar to the MLM training objective of the BERT, we employ pre-trained BERT as the backbone model.
That is, MLM's goal is to predict the masked tokens using bidirectional contexts. Similarly, slot filling tries to predict the label for a token leveraging both left and right contexts simultaneously, which makes the pre-trained BERT an ideal model of choice that greatly facilitates minimizing incomplete labels.
It is important to highlight that our automatically generated labels are not only incomplete but also potentially wrong.
The training strategies employed in this phase minimize the noise in the label to some extent. 
Specifically, early stopping can provide a strong regularization and would not let the model overfit to the noisy labels, especially wrong labels. 
Moreover, early stopping does not let the model forget the knowledge in the pre-trained model.
Similarly, multiple random initializations enforce robustness. 
Since the model is fine-tuned on the noisy labels, averaging the predictions of multiple models for each token ensures that wrong labels end up with low probabilities and true labels consistently achieve high probabilities.
Using the above-mentioned strategies, we train two slot filling models, which we call the peers. The peer one is trained on the first set of pseudo labels that were generated using POS and NER taggers, KG, and the GPT-2 scorer in phase one. Similarly, peer two is trained using the predictions of the zero-shot slot filling model~\cite{liu2020coach}.
Both models have the same architecture and follow the same training procedures.

\begin{table*}[t!]
\centering
\caption{Dataset statistics.}
\vspace{-7pt}
\label{tab:dataset}
\begin{tabular}{lccccc}
\toprule
\textbf{Dataset}  & \textbf{Dataset Size} & \textbf{Vocab. Size} & \textbf{Avg. Length} & \textbf{\# of Domains} & \textbf{\# of Slots} \\ \hline
\textbf{SGD}      & 188K                  & 33.6K                & 13.8                 & 20                     & 240                  \\
\textbf{MultiWoZ} & 67.4K                 & 10.5K                & 13.3                 & 8                      & 61 \\
\bottomrule
\end{tabular}
\vspace{-7pt}
\end{table*}

\subsection{Phase Three: Self-supervised Co-training}
We introduce an iterative peer training algorithm where both peers generate high-confidence soft labels for training the other peer in the next iteration. 
Theoretically, these peers can be anything, but in this work, 
we explore two of the most promising directions that have shown the promise to minimize the need for manual labeling for the task: zero-shot learning and distant supervision.
This phase uses a self-supervised co-training scheme to exploit the patterns of slot values from other domains through the labels generated by the zero-shot filling model (i.e., peer two)~\cite{liu2020coach} as well as utilize the knowledge in external KGs and pre-trained models via labels provided by the peer one.
Specifically, we initialize the peers trained in phase two and use their pseudo labels to kick-start training in this phase.
Specifically, peer one $f_{m,c}(\cdot; \theta_{\textrm{p1}})$ would generate labels $\{\tilde{\mathbf{Y}}^{(t)}_n = [\tilde{y}_{n,1}^{(t)}, ..., \tilde{y}_{n,m}^{(t)}]\}_{n=1}^{N}$ for peer two $f_{m,c}(\cdot; \theta_{\textrm{p2}})$ at the $t$-th iteration by:
$$
\tilde{y}_{n,m}^{(t)} = \argmax_{c}{f_{m,c}(\mathbf{X}_n; \theta_{\textrm{p1}}^{(t)})}. 
\label{eq:pseudo}
$$

Based on these labels, the peer two can be fine-tuned by: 
$$
\hat\theta_{\textrm{p2}}^{(t+1)} = \argmin_{\theta}\frac{1}{N}\sum_{n=1}^N \ell(\tilde{\mathbf{Y}}_n^{(t)}, f(\mathbf{X}_{n}; \theta)).
\label{eq:self_train1}
$$

Similarly, peer two $f_{m,c}(\cdot; \theta_{\textrm{p2}})$ would generate pseudo labels for peer one $f_{m,c}(\cdot; \theta_{\textrm{p1}})$ that are used to fine-tune peer one. 
We also notice that it is beneficial to stop early during this phase as well, to improve the model fitting and gradually reduce the noise associated with the automatically generated labels.
Since pseudo labels are refined gradually in an iterative way, both peers can benefit from the knowledge contained within the labels of the other while avoiding overfitting.
Furthermore, as an alternative to pseudo labels, we also generate soft labels that are used for confidence re-weighting. 
The high-confidence soft label selection strategy enables better model fitting and efficient learning via better quality of the automatic labels.
Specifically, for the given $m$-th token in the $n$-th training example, the probability for all classes $C$ is $[f_{m,1}(\mathbf{X}_n;\theta),...,f_{m,C}(\mathbf{X}_n;\theta)]$. 
Following ~\cite{xie2016unsupervised}, at $t$-th iteration, peer one generates soft labels, $\{\mathbf{S}_n^{(t)} = [\mathbf{s}_{n,m}^{(t)}]_{m=1}^M \}_{n=1}^N$, as given below:
$$
\mathbf{s}_{n,m}^{(t)} = [s_{n,m,c}^{(t)}]_{c=1}^{C} = \Bigg[  \frac{f_{m,c}^2(\mathbf{X}_n;\theta_{\textrm{peer1}}^{(t)})/p_{c}}{\sum_{c'=1}^C f_{m,c'}^2(\mathbf{X}_n;\theta_{\textrm{peer1}}^{(t)})/p_{c'}}\Bigg]_{c=1}^{C}
\label{eq:soft}
$$ 
where $p_{c} = \sum_{n=1}^N \sum_{m=1}^M f_{m,c}(\mathbf{X}_n;\theta_{\textrm{p1}}^{(t)})$ computes the frequency of the tokens for the $c$-th class. 
Then, peer two $f(\cdot; \theta_{\textrm{p2}}^{(t+1)})$ is fine-tuned by:
$$
\theta_{\textrm{p2}}^{(t+1)} = \argmin_{\theta} \frac{1}{N} \sum_{n=1}^{N} \ell_{\rm KL}(\mathbf{S}_n^{(t)}, f(\mathbf{X}_{n}; \theta)),
$$
where $\ell_{\rm KL}(\cdot,\cdot)$ is the KL-divergence-based loss:
$$
\ell_{\rm KL}(\mathbf{S}_n^{(t)}, f(\mathbf{X}_{n}; \theta))=\frac{1}{M}\sum_{m=1}^M\sum_{c=1}^C - s_{n,m,c}^{(t)} \log f_{m,c}(\mathbf{X}_{n}; \theta).
\label{eq:klloss}
$$

Moreover, we also investigate selecting tokens that have high confidence. 
For instance, we pick high-confidence tokens from the $m$-th input example at the $t$-th iteration by  
$
H^{(t)}_n = \{m : \max_{c} s_{n,m,c}^{(t)} > \epsilon \},
$
where $\epsilon\in [0,1]$ is a threshold that can be searched based on a small validation set. 
Then, peer two $f(\cdot; \theta_{\textrm{p2}}^{(t+1)})$ is fine-tuned by:
$$
\theta_{\textrm{p2}}^{(t+1)} %&= \argmin_{\theta} \frac{1}{N} \sum_{n=1}^{N} \ell_{\rm S-KL}(\bS_n^{(t)}, f(\bX_{n}; \theta)) \\
= \argmin_{\theta} \frac{1}{N|H^{(t)}_n|}\sum_{n=1}^{N} \sum_{m\in H^{(t)}_n}\sum_{c=1}^C - s_{n,m,c}^{(t)} \log f_{m,c}(\mathbf{X}_{n}; \theta).
$$

This phase improves the robustness to effectively fit the model for tokens with high confidence. 
Both peers keep sharing information and their confidence by producing soft labels for their counterparts until they approximate to the true labels while employing early stopping and scheduled learning rates.
It is important to remind that phase three is the most important phase that progressively reduces noise from the labels to a great extent and enables superior performance for the task of open-domain slot filling.
For this chapter, fix a prime $p$. We first discuss deformations of coalgebras from $\F_{p}$
to the $p$-adic integers and further to the $p$-completed sphere $\S_{p}^{\wedge}$ which leads
us to the question of how coalgebras behave with respect to $p$-completion. We introduce the
notion of a $p$-complete coalgebra and show that this is well behaved with respect to the
deformation theory discussed in the previous chapter. We then use this to iterate
Proposition~\ref{witt} and prove our main results, namely the existence of Witt Vectors
and spherical Witt Vectors for formally \'etale coalgebras. Then we specialize to the case
of homology coalgebras, show that for a finite space $X$ the coalgebra $\F_{p}[X]$ is formally
\'etale, and answer our initial question about the relation between $\S[X]^{\wedge}_{p}$
and $\F_{p}[X]$

\subsection{Coalgebras and $p$-completion}

We have seen that the functors that interest us are all \textit{nilcomplete}. For a nilcomplete
functor $X:\rm{CAlg}^{\rm{cn}} \to \cl{S}$ and a connective $\bb{E}_{\infty}$-ring $R$, we can construct
lifts from $X(\pi_{0}R)$ to $X(R)$ inductively along the Postnikov tower
\[ \dots \to \tau_{\leq2}R \to \tau_{\tau\leq 1}R \to \tau_{\leq0} R =\pi_{0}R.\]
This is however not quite enough to obtain our goal of lifting from $\F_{p}$ to the
$p$-completed sphere, we first need to pass to $\Z_{p}= \pi_{0}\S_{p}^{\wedge}$.
Explicitly, this means constructing lifts against the tower
\[\dots \to \Z/p^{3}\to \Z/p^{2}\to \Z/p\to \F_{p}\]
which is clearly presents a different problem. With the machinery developed thus far, we can already
prove the following for a general deformation problem.

\begin{proposition}\label{liftpgen}
  Let $X: \rm{CAlg}^{\rm{cn}} \to \cl{S}$ be a cohesive functor and $A\in X(\F_{p})$
  such that $T_{X_{A}}\simeq 0$. Then there exists a unique lift of $A$ to a point in
  $\flim_{n}X(\Z/p^{n})$.
\end{proposition}
\begin{proof}
  Set $A_{0}= A$, we inductively construct lifts against the tower of square zero extensions
  \[\dots \to \Z/p^{3} \to \Z/p^{2}\to \F_{p}.\]
  Suppose we have already constructed lifts $A_{k}$ for $k\le n$ for some $n$.
  Applying Proposition~\ref{bc} inductively, we get that
  \[T_{X_{A_{n}}}^{\F_{p}} \simeq T^{\F_{p}}_{X_{A_{0}}} \simeq 0.\]
  Thus, since $\Z/p^{n+1}\to \Z/p^{n}$ is a square zero extension with fiber $\F_{p}$,
  Proposition~\ref{deformations} implies that the fiber
  \[X_{A_{n}}^{\Z/p^{n+1}}=\rm{fib}_{A_{n}}(X(\Z/p^{n+1})\to \Z/p^{n})\]
  is contractible and we find an essentially unique lift $A_{n+1}$. This proves the claim.
\end{proof}
 Of course, for an arbitrary functor $X:\rm{CAlg}^{\rm{cn}} \to \cl{S}$ the natural map
$X\to \flim_{n}X(\Z/p^{n})$ might not be an equivalence, meaning that in this generality
we can only construct pro-$p$ objects of $X$ using this inductive method.
In fact, we have that $\rm{cCAlg}_{\Z_{p}}\neq  \flim_{n} \rm{cCAlg}_{\Z/p^{n}}$. To remedy
this problem we show that this limit admits a description via \textit{$p$-complete} coalgebras.
To do this, we first recall some facts about $p$-complete modules.

\begin{definition}
Let $R$ be an $\bb{E}_{\infty}$-ring, then $M \in \rm{Mod}_{R}$ is called
$p$-\textit{complete} if the limit
\[ \lim \left(\dots \rar{\cdot p} M \rar{\cdot p}M \right)\]
vanishes. We denote the full subcategory spanned by the $p$-complete modules by $(\rm{Mod}_{R})_{p}^{\wedge}$.
\end{definition}

\begin{remark}
The inclusion $(\rm{Mod}_{R})_{p}^{\wedge} \rari{} \rm{Mod_{R}}$ admits a left adjoint which takes a module $M$
to its \textit{$p$-completion} given by the limit
\[ \lim \left( \dots \to M/p^{2} \to M/p \right).\]
In fact, $M$ is $p$-complete if and only if the natural map $M \to \lim M/p^{n}$ is an equivalence.
This inherits a natural $R^{\wedge}_{p}$-module structure, thus $p$-completion also gives
an equivalence of categories $(\rm{Mod}_{R})^{\wedge}_{p} \simeq (\rm{Mod}_{R^{\wedge}_{p}})^{\wedge}_{p}$ which
allows us to identify these in what follows.\\
The tensor product of $p$-complete modules is in general not $p$-complete. However, the
category $(\rm{Mod}_{R})_{p}^{\wedge}$ admits a symmetric monoidal structure given by the formula
 \[ M \otimes_{(\rm{Mod}_{R})_{p}^{\wedge}} N := ( M \otimes N )^{\wedge}_{p}.\]
 With this monoidal structure the $p$-completion functor $\rm{Mod}_{R}\to (\rm{Mod}_{R})_{p}^{\wedge}$
 is strong monoidal, while the inclusion is only lax monoidal.
\end{remark}

 \begin{definition}
   Let $R$ be an $\bb{E}_{\infty}$-ring. We define the $\infty$-category of $p$-complete
   $R$-coalgebras is given by.
   \[ {(\rm{cCAlg}_{R})}^{\wedge}_{p}:= \rm{cCAlg}({(\rm{Mod}_{R})}^{\wedge}_{p}).\]
 \end{definition}

 \begin{warning}
   Let $R$ be a $\bb{E}_{\infty}$-ring. Notice that by our definition a $p$-complete $R$-coalgebra
   is the same as a $p$-complete $R^{\wedge}_{p}$-coalgebra and so we do not differentiate between
   the two notions.
   However, this is \textit{not} the same as an $R^{\wedge}_{p}$-coalgebra whose underlying
   spectrum is $p$-complete. The process of $p$-completion does refine to a functor
   $\rm{cCAlg}_{R} \to (\rm{cCAlg}_{R^{\wedge}_{p}})^{\wedge}_{p}$,
   but it does not factor through the category $\rm{cCAlg}_{R^{\wedge}_{p}}$.
 \end{warning}

 We now show check that the assignment $R \mapsto \rm{cCAlg}_{R}^{\rm{cn}}$ is subject to the machinery
 of deformation theory.

 \begin{lemma}\label{conil2}
   The following statements hold:
   \begin{enumerate}
     \item   Suppose we have a pullback diagram of connective $\bb{E}_{\infty}$-rings
   \[\begin{tikzcd}
	R\p & S\p \\
	R & S
	\arrow[from=1-1, to=2-1]
	\arrow[from=2-1, to=2-2]
	\arrow[from=1-2, to=2-2]
	\arrow[from=1-1, to=1-2]
\end{tikzcd}\]
such that the map $\pi_{0}R \to \pi_{0}S$ is surjective. Then the natural map
\[ (\rm{cCAlg}_{R\p}^{\rm{cn}})^{\wedge}_{p} \to (\rm{cCAlg}_{R}^{\rm{cn}})^{\wedge}_{p}\times_{(\rm{cCAlg}_{S}^{\rm{cn}})^{\wedge}_{p}} (\rm{cCAlg}_{S\p}^{\rm{cn}})^{\wedge}_{p}\]
is an equivalence.
     \item For every connective $\bb{E}_{\infty}$-ring $R$, the natural map
           \[ (\rm{cCAlg}_{R}^{\rm{cn}})^{\wedge}_{p} \to\flim_{n} (\rm{cCAlg}_{\tau_{\le n}R}^{\rm{cn}})^{\wedge}_{p}\]
           is an equivalence.
   \end{enumerate}
 \end{lemma}
 \begin{proof}
   Ad 1.: Arguing as in the proof of Proposition~\ref{Mod}, it suffices to show that the
   strong monoidal functor
   \begin{align*}
    (\rm{Mod}_{R\p})^{\wedge}_{p} \to (\rm{Mod}_{R})^{\wedge}_{p}\times_{(\rm{Mod}_{S})^{\wedge}_{p}} (\rm{Mod}_{S\p})^{\wedge}_{p}
   \end{align*}
   is an equivalence. Indeed, given a point $(M,N,h)$ in the pullback, the $R\p$-module $M \times_{M \otimes_{R} S}N$
   is again $p$-complete since $p$-completion commutes with limits. Thus, the inverse functor of
   Proposition~\ref{Mod} also induces a functor on the categories of $p$-complete modules. Moreover,
   we have that
   \[ ((M\times_{M\otimes_{R}S}N)\otimes_{R\p} R)^{\wedge}_{p} \simeq M^{\wedge}_{p} \simeq M\]
   \[ ((M \times_{M\otimes_{R}}N)\otimes_{R\p}S\p)^{\wedge}_{p}\simeq N^{\wedge}_{p} \simeq N,\]
   where the first equivalences hold by Proposition~\ref{Mod}, and the latter since $M$ and $N$ are
   to be $p$-complete. Finally, for $M\in (\rm{Mod}_{R\p})^{\wedge}_{p}$, we compute that
   \[ (M \otimes_{R\p} R)^{\wedge}_{p}\times_{(M \otimes_{R\p} S)^{\wedge}_{p}}(M \otimes_{R\p}S\p)^{\wedge}_{p}
     \simeq \left( M \otimes_{R\p} R \times_{M\otimes_{R\p} S} M \otimes_{R\p} S\p\right)^{\wedge}_{p}
   \simeq M^{\wedge}_{p} \simeq M,\]
 where we have again used the result of Proposition~\ref{Mod} and the fact that $p$-completion commutes
 with limits.\\
 Ad 2: This uses the exact same arguments applied to the equivalence of Corollary~\ref{nilcomplete}.
 \end{proof}

 \begin{corollary}
   For any $n\in \bb{N}$, the functor
   \[ \rm{CAlg}^{\rm{cn}} \to \cl{S} \qquad R \mapsto [(\rm{cCAlg}_{R}^{\rm{cn}})^{\wedge}_{p}]^{\Delta^{n}}\]
   is coherent and nilcomplete.
 \end{corollary}

 We now prove the crucial $p$-completeness result for $\Z_{p}$-modules. As before
 this will enable us to deduce the same result for coalgebras and allow us to tackle the
 actual problem of comparing coalgebras over $\F_{p}$, $\Z_{p}$ and $\S_{p}^{\wedge}$.
\begin{proposition}\label{pcomp}
  Let $\rm{Mod}^{\wedge}_{\Z_p} \subseteq \rm{Mod}_{\Z_{p}}$ denote the full subcategory spanned by the
  $p$-complete $\Z_{p}$-module spectra. Then the natural map
  \[ \rm{Mod}_{\Z_{p}} \to \flim_{n} \rm{Mod}_{\Z/p^{n}} \quad N \mapsto (N\otimes_{\Z_{p}}\Z/p^{n})\]
  restricts to a strong monoidal equivalence
  \[(\rm{Mod}_{\Z_{p}})^{\wedge}_{p} \simeq \flim_{n}\rm{Mod}_{\Z/p^{n}}. \]
\end{proposition}
\begin{proof}
  The functor admits a right adjoint which takes $(M_{n})\in \flim_{n}\rm{Mod}_{\Z/p^{n}}$ to the limit
  $\lim_{n}M_{n}$ taken in the category of $\Z_{p}$-modules. Since $p$-complete modules are closed under
  limits, the essential image of this functor is contained in $\rm{Mod}_{\Z_{p}}^{\wedge}$. Moreover,
  if $M\in \rm{Mod}_{\Z_{p}}^{\wedge}$, then we have that
  \[ \flim_{n}(M \otimes_{\Z_{p}} \Z/p^{n}) \simeq \flim_{n} M/p^{n} \simeq M^{\wedge}_{p}\simeq M.\]
  Hence, the counit of the adjunction is an equivalence on $p$-complete modules.
  Conversely, given $(N_{k})\in \flim_{k}\rm{Mod}_{\Z/p^{k}}$ write $N= \lim_{k}N$. We want
  to show that, for every $n$ the natural map
  \[ N \otimes_{\Z_{p}} \Z/p^{n}\rar{\sim}N_{n}\]
  is an equivalence. Since $N \otimes_{\Z_{p}}Z/p^{n}\simeq N/p^{n}$ and limits are exact, we have an equivalence
  \[N \otimes_{\Z_{p}}\Z/p^{n}\simeq \lim_{k >n}(N_{k}\otimes_{\Z_{p}}\Z/p^{n}).\]
  Thus, the unit of the adjunction may be written as
  \[ \lim_{k>n}(N_{k} \otimes_{\Z_{p}}\Z/p^{n}) \to \lim_{k>n}(N_{k}\otimes_{\Z/p^{k}}\Z/p^{n})\simeq N_{n}\]
  and so has fiber given by
  \[ F_{n}:=\lim_{k>n}\left(N_{k}\otimes_{\Z/p^{k}}\rm{fib}(\Z/p^{k}\otimes_{\Z_{p}}\Z/p^{n}\to \Z/p^{n}) \right).\]
  Now we compute the fiber of $\Z/p^{k}\otimes_{\Z_{p}}\Z/p^{n}\to \Z/p^{n}$ as the module
  \[ \rm{Tor}^{\Z_{p}}(\Z/p^{k}, \Z/p^{n})[1]\simeq \Z/p^{n}[1].\]
  The reduction map $\Z/p^{k}\to \Z/p^{k-1}$ is induced by the map of projective resolutions
\[\begin{tikzcd}
	{\Z_p} & {\Z_p} \\
	{\Z_p} & {\Z_p}
	\arrow["{\cdot p^k}", from=1-1, to=1-2]
	\arrow["\id", from=1-2, to=2-2]
	\arrow["{\cdot p}"', from=1-1, to=2-1]
	\arrow["{\cdot p^{k-1}}"', from=2-1, to=2-2],
\end{tikzcd}\]
hence, on Tor it induces the multiplication by $p$ map
\[ \Z/p^{n}=\rm{Tor}^{\Z_{p}}(\Z/p^{k}, \Z/p^{n})\rar{\cdot p} \rm{Tor}^{\Z_{p}}(\Z/p^{k-1}, \Z/p^{n}) =\Z/p^{n}.\]
Thus, if we have $k\p > k > n$ such that $k\p -k > n$, the transition map
\[ F_{k\p}=N_{k\p} \otimes \rm{Tor}^{\Z_{p}}(\Z/p^{k}, \Z/p^{n})\to N_{k} \otimes \rm{Tor}^{\Z_{p}}(\Z/p^{k-1}, \Z/p^{n})= F_{k}\]
vanishes since the Tor-groups are $p^{n}$-torsion. Choosing a cofinal subset $S\subseteq \bb{N}_{>n}$ such that
$\abs{k\p -k}> n$ for any distinct $k\p,k\in S$, we see that
\[ \lim_{k>n} F_{k}\simeq \lim_{k\in S} F_{k} \simeq 0 \]
vanishes. Thus, since limits are exact, the map $N \otimes_{\Z_{p}} \Z/p^{n}\rar{\sim}N_{n}$ is an equivalence.\\
To see that the functor $\rm{Mod}_{\Z_{p}}^{\wedge} \to \flim_n \rm{Mod}_{\Z/p^{n}}$ is strong monoidal,
we observe that since cofibers and limits are exact, we have for each $n$ equivalences
\begin{align*}
  (M \otimes_{\Z_{p}} N)^{\wedge}_{p} \otimes_{\Z_{p}}\Z/p^{n} &\simeq \lim_{k}(M/p^{k} \otimes_{\Z_{p}}N/p^{k})/p^{n}\\
                                              &\simeq \lim_{k}\left((M/p^{n} \otimes_{\Z_{p}} N/p^{n})\otimes_{Z_{p}}\Z/p^{k}\right) \\
  &\simeq ((N\otimes_{\Z_{p}}\Z/p^{n}) \otimes_{\Z_{p}} (M \otimes_{\Z_{p}}\Z/p^{n}))^{\wedge}_{p}.
\end{align*}
This proves the claim.
\end{proof}

\begin{corollary}\label{pcomp1}
  We have an equivalence of categories
  \[ (\rm{cCAlg}_{\Z_{p}})_{p}^{\wedge} \rar{\sim} \flim_{n} \rm{cCAlg}_{\Z/p^{n}} \quad A \mapsto (A\otimes_{\Z_{p}}\Z/p^{n})\]
  with inverse taking a system of coalgebras $(B_{n})$ to the limit $\lim_{n}B_{n}$ taken in the
  category of ($p$-complete) $\Z_{p}$-modules, equipped with the induced $p$-complete
  $\Z_{p}$-coalgebra structure.
\end{corollary}
\begin{proof}
This follows from Proposition~\ref{pcomp}, arguing as in the proof of Proposition~\ref{Mod}.
\end{proof}

\begin{corollary}\label{obliftzp}
  Let $X(\blank)= (\rm{cCAlg}_{\blank}^{\rm{cn}})^{\Delta^{0}}$ and $A\in X(\F_{p})$ such that $T_{X_{A}}\simeq 0$.
  Then the space of lifts of $A$ to a $p$-complete $\Z_{p}$-coalgebra is contractible
\end{corollary}
 \begin{proof}
 Combine Proposition~\ref{liftpgen} and Corollary~\ref{pcomp1}.
 \end{proof}

\begin{corollary}\label{mapliftzp}
  Let $\varphi: B\to A$ be a map of connective, formally \'etale $\F_{p}$-coalgebras. Then the space of
  lifts of $\varphi$ to a map of $p$-complete $\Z_{p}$-coalgebras $B\p \to A\p$ is contractible.
\end{corollary}
\begin{proof}
    Let $ \cl{X}(\blank)=\rm{cCAlg}_{\blank}^{\rm{cn}}$. By Proposition~\ref{etalchar} the natural map
    \[ T_{\cl{X}^{\Delta^{1}}_{\varphi}} \to T_{\cl{X}^{\Delta^{0}}_{B}}\]
    is an equivalence, but since $B$ is formally \'etale we have $T_{\cl{X}^{\Delta^{0}}_{B}} \simeq 0$.
    Hence, the claim follows by applying Proposition~\ref{liftpgen} to the functor $\cl{X}^{\Delta^{1}}$
    and using Corollary~\ref{pcomp1}.
\end{proof}

Having shown this, we can now construct a functor which is analogous to the classical
Witt-Vectors, which allow us to pass from \'etale $\F_{p}$-algebras to $\Z_{p}$-algebras.

\begin{theorem}
  Let $\cl{C}\subseteq (\rm{cCAlg}_{\Z_{p}}^{\rm{cn}})^{\wedge}_{p}$ denote the full subcategory spanned by those
  coalgebras $A$ for which $A\otimes_{\Z_{p}} \F_{p}$ is formally \'etale. Then the base change functor
  \[ \cl{C} \to \rm{cCAlg}_{\F_{p}}^{\rm{cn}, \rm{f\acute{e}t}}  \qquad A \mapsto A\otimes_{\Z_{p}}\F_{p}\]
  is fully faithful and essentially surjective. In particular, the quasi inverse defines a functor
  \[ W_{p}: \rm{cCAlg}_{\F_{p}}^{\rm{cn,f\acute{e}t}} \to (\rm{cCAlg}_{\Z_{p}}^{\rm{cn}})^{\wedge}_{p}\]
  which is fully faithful and satisfies $W_{p}(A)\otimes_{\Z_{p}}\F_{p} \simeq A$ for every connective, formally
  \'etale $\F_{p}$-coalgebra $A$.
\end{theorem}

\begin{proof}
  Combine Corollary~\ref{obliftzp} and Corollary~\ref{mapliftzp}.
\end{proof}

We now turn our attention to the leap from $\Z_{p}$ to $\S_{p}^{\wedge}$. The following proposition shows that,
for an arbitrary cohesive and nilcomplete functor, a $\Z_{p}$-valued point which has vanishing $\F_{p}$-tangent
complex admits a unique lift to a $\S_{p}^{\wedge}$-valued point. This is surprising, as we do not
actually require any information about the $\Z_{p}$-tangent complex, everything is determined by
what happens modulo $p$.

\begin{proposition}\label{spherelift}
  Let $X: \rm{CAlg}^{\rm{cn}} \to \cl{S}$ be a cohesive and nilcomplete functor and let $A \in X(\Z_{p})$
  such that $T_{X_{A\otimes_{\Z_{p}}\F_{p}}}\simeq 0$. Then $A$ admits an essentially unique lift to $X(\S_{p}^{\wedge})$.
\end{proposition}

\begin{proof}
  We inductively construct lifts against the Postnikov Tower
  \[ \dots \to \tau_{\leq2} \S_{p}^{\wedge}  \to \tau_{\leq 1} \S_{p}^{\wedge} \to \tau_{\leq 0} \S_{p}^{\wedge} \simeq \Z_{p}. \]
  Write $A=A_{0},~S_{n}= \tau_{\leq n}\S_{p}^{\wedge},~ M_{n} = \pi_{n}S_{n}$ and assume we have already constructed
  a unique lift $A_{n}$ to $X(S_{n})$. Consider the square zero extension
  \[ M_{n+1}[n+1] \to S_{n+1}\to S_{n}.\]
  Since $M_{n+1} = \pi_{n+1}S_{n+1}$ is concentrated in a single degree, the $S_{n}$-action factors
  through $S_{0}=\Z_{p}$. Moreover, since $\pi_{n+1}S_{n+1}$ is of finite $p$-torsion, the action
  further factors through $\Z/p^{k}$ for some $k\geq 0$. Thus, Proposition~\ref{bc} implies that
  we have an equivalence
  \[ T_{X_{A_{n}}}^{M_{n+1}[n+1]} \simeq \Sigma^{n}T_{X_{A_{n}}}^{M_{n+1}} \simeq T_{X_{A_{n} \otimes_{S_{n}} \Z/p^{k}}}^{M_{n+1}}.\]
  Arguing as in Proposition~\ref{cofib} with respect to the square zero extension
  \[ \F_{p} \to \Z/p^{k}\to \Z/p^{k-1},\]
  we see that we have a cofiber sequence
  \[  T^{M_{n+1}\otimes_{\Z/p^{k}}\F_{p}}_{X_{A_{n} \otimes_{S_{n}} \Z/p^{k-1}}}
    \to T_{X_{A_{n} \otimes_{S_{n}} \Z/p^{k}}}^{M_{n+1}}
    \to T^{M_{n+1}\otimes_{\Z/p^{k}}\Z/p^{{k-1}}}_{X_{A_{n} \otimes_{S_{n}} \Z/p^{k-1}}}.\]
  For the left hand term, Proposition~\ref{bc} gives the equivalence
  \[ T_{X_{A_{n}\otimes_{S_{n}}\Z/p^{k-1}}}^{M_{n+1}\otimes_{\Z/p^{k}}\F_{p}}
    \simeq T_{X_{A \otimes_{\Z_{p}}\F_{p}}}^{{M_{n+1}\otimes_{\Z/p^{k}}\F_{p}}}
    \simeq T_{X_{A\otimes_{\Z_{p}}\F_{p}}}\otimes_{\F_{p}}( M_{n+1}\otimes_{\Z/p^{k}}\F_{p} ) \simeq 0,\]
  where we have used that, since $M_{n+1}$ is finitely generated, the $\F_{p}$-module
  $M_{n+1}\otimes_{\Z/p^{k}}\F_{p}$ is perfect. For the right hand term we
  replace $M_{n+1}$ with $M_{n+1} \otimes_{\Z/p^{k}}\Z/p^{k-1}$ and repeat the argument,
  inductively yielding equivalences
  \[ T^{M_{n+1}}_{X_{A_{n}\otimes_{S_{n}}\Z/p^{k}}}
    \simeq T^{M_{n+1}\otimes_{\Z/p^{k}}\Z/p^{{k-1}}}_{X_{A_{n-1} \otimes_{S_{n-1}} \Z/p^{k-1}}}
  \simeq \cdots \simeq T^{M_{n+1}\otimes_{\Z/p^{k}} \F_{p}}_{X_{A \otimes_{\Z_{p}}\F_{p}}} \simeq 0.\]
In total, this shows that $T_{X_{A_{n}}}^{M_{n+1}[n+1]} \simeq 0$, and hence $A_{n}$ admits an essentially
unique lift to $X(S_{n+1})$. Thus, the fiber over $A$ of the map
\[ X(\S_{p}^{\wedge})\simeq \flim_{n}X(S_{n})\to X( \Z_{p})\]
is contractible and we are done.
  \end{proof}

  \begin{lemma}\label{pcomparison}
    Write $\cl{X}(\blank)=\rm{cCAlg}^{\rm{cn}}_{\blank}$ and $\cl{Y}(\blank)=
    (\rm{cCAlg}^{\rm{cn}}_{\blank})^{\wedge}_{p}$. Then the $p$-completion map $f:\cl{X}\to \cl{X}\p$
    induces an equivalence
    \[ T^{M}_{(\cl{X}^{\Delta^{n}})_{\xi}} \to  T^{M}_{(\cl{Y}^{\Delta^{n}})_{f(\xi)}}\]
        for every $\F_{p}$-module $M$, $n\in \bb{N}$ and $\xi \in \cl{X}(\F_{p})^{\Delta^{n}}$.
  \end{lemma}
  \begin{proof}
    For any $\F_{p}$-algebra $R$ the $p$-completion map gives an equivalence
    $\rm{Mod}_{R}\rar{\sim} (\rm{Mod}_{R})^{\wedge}_{p}$, since multiplication by some power of $p$
    is nullhomotopic over $\F_{p}$. In particular, this applies to the split square zero
    extension $\F_{p}\oplus M$ for any $M \in \rm{Mod}_{\F_{p}}$ and so the natural map
    $\cl{X}(\F_{p}\oplus M) \to \cl{Y}(\F_{p}\oplus M)$ is an equivalence as well.
    Consequently, we also obtain natural equivalences between the fibers
    \[ (\cl{X}^{\Delta^{n}})_{\xi}^{\F_{p}\oplus M} \to  (\cl{Y}^{\Delta^{n}})_{f(\xi_)}^{\F_{p}\oplus M},\]
    which induces the equivalence of spectra
    \[ T^{M}_{(\cl{X}^{\Delta^{n}})_{\xi}} \to  T^{M}_{(\cl{Y}^{\Delta^{n}})_{f(\xi)}}\]
      as claimed.
  \end{proof}

  \begin{corollary}\label{obliftsp}
    Let $X(\blank)=(\rm{cCAlg}^{\rm{cn}}_{\blank})^{\Delta^{0}}$ and $A \in X(\F_{p})$ such that
    $T_{X_{A}}\simeq 0$, then the space of lifts of $A$ to a $p$-complete $\S_{p}^{\wedge}$-coalgebra
    is contractible.
  \end{corollary}

  \begin{proof}
    Write $Y(\blank)= ((\rm{cCAlg}^{\rm{cn}}_{\blank})^{\wedge}_{p})^{\Delta^{0}}$. Then by Lemma~\ref{pcomparison}
    we have an equivalence $T_{X_{A}}\simeq T_{Y_{A}} \simeq 0$. Hence, we can apply Proposition~\ref{obliftzp} to
    obtain an essentially unique lift $A\p\in Y(Z_{p})$. Further applying Proposition~\ref{spherelift}
    to $A\p$ yields our claim.
  \end{proof}
  Thus, we can pointwise lift $\F_{p}$-coalgebras with vanishing tangent complex to $\S_{p}^{\wedge}$. If
  we moreover consider \textit{formally \'etale coalgebras}, we can make this lifting functorial
  in a coalgebraic analogue of the \textit{Spherical Witt Vectors} construction for
  $\bb{E}_{\infty}$-algebras over $\F_{p}$.

\begin{corollary}\label{mapliftsp}
  Let $\varphi:B\to A$ be a map of $\F_{p}$-coalgebras such that $A$ and $B$ are formally \'etale.
  Then the space of lifts of $\varphi$ to a map $\varphi\p: B\p \to A\p$ of $p$-complete
  $\S_{p}^{\wedge}$-coalgebras is contractible.
\end{corollary}

\begin{proof}
  Let $ \cl{X}(\blank)=\rm{cCAlg}_{\blank}^{\rm{cn}}$ and $\cl{Y}(\blank) =
  (\rm{cCAlg}_{\blank}^{\rm{cn}})^{\wedge}_{p}$. By Proposition~\ref{mapliftzp} the map $\varphi$ admits
  an essentially unique lift to a point $\psi \in \cl{Y}(\Z_{p})^{\Delta^{1}}$. Moreover, Lemma~\ref{pcomparison}
  yields an equivalence $T_{\cl{X}^{\Delta^{1}}_{\varphi}}\simeq T_{\cl{Y}^{\Delta^{1}}_{\varphi}}$. Since both $A$ and $B$ are
  formally \'etale Proposition~\ref{etalchar} gives equivalences
  \[ T_{\cl{X}^{\Delta^{1}}_{\varphi}} \rar{\sim} T_{\cl{X}^{\Delta^{0}}_{B}} \simeq 0\]
  Hence, we can apply Proposition~\ref{spherelift} to the functor $\cl{Y}^{\Delta^{1}}$ and the point
  $\psi \in \cl{Y}^{\Delta^{1}}$, proving the claim.
\end{proof}

\begin{theorem}\label{wittsp}
  Denote by $\cl{C}\subseteq (\rm{cCAlg}_{\S_{p}^{\wedge}}^{\rm{cn}})^{\wedge}_{p} $ the full subcategory spanned by those
  coalgebras $A$ such that $A\otimes_{\S_{p}^{\wedge}}\F_{p}$ is formally \'etale. Then the base change functor
  \[ \cl{C} \to \rm{cCAlg}_{\F_{p}}^{\rm{cn}, \rm{f\acute{e}t}} \qquad A \mapsto A \otimes_{\S_{p}^{\wedge}} \F_{p}\]
  is fully faithful and essentially surjective.
\end{theorem}
\begin{proof}
  Combine Corollary~\ref{obliftsp} and Corollary~\ref{mapliftsp}.
\end{proof}

\begin{remark}
  In the setting of Theorem~\ref{wittsp} the quasi-inverse to $\blank \otimes_{\S^{\wedge}_{p}}\F_{p}$ defines
  a fully faithful functor
  \[ W_{\S_{p}^{\wedge}}: \rm{cCAlg}_{\F_{p}}^{\rm{cn}, \rm{f\acute{e}t}}
    \to (\rm{cCAlg}_{\S_{p}^{\wedge}}^{\rm{cn}})^{\wedge}_{p}\]
  which satisfies $W_{\S_{p}^{\wedge}}(A)\otimes_{\S^{\wedge}_{p}}\F_{p} \simeq A$ for every connective, formally \'etale
  $\F_{p}$-coalgebra $A$. We call $W_{\S_{p}^{\wedge}}(A)$ the \textit{spherical Witt vectors} of $A$.
\end{remark}


\subsection{Homology coalgebras}

As observed in Example~\ref{homology}, for every space $X$ and every $\bb{E}_{\infty}$-ring $R$, the
$R$-homology $R[X]$ carries a natural $R$-coalgebra structure, which is a stronger invariant than its
underlying $R$-module. We now want to apply our results and see what can be said about the deformation
theoretic behavior of homology coalgebras. To do this, we first need to compute the cotangent complex of the
$\F_{p}$-cohomology.

\begin{definition}
  A space $X\in \cl{S}$ is called $p$-finite if the following conditions hold:
  \begin{enumerate}
    \item The space $X$ is truncated.
    \item The set $\pi_{0}X$ is finite.
    \item For each $n\geq 1$ and $x\in X$, we have that $\pi_{n}(X,x)$ is a finite $p$-group.
  \end{enumerate}
  We denote the full subcategory of $\cl{S}$ spanned by the $p$-finite spaces as $\cl{S}_{p}$ and call
 $\cl{S}^{\vee}_{p} =: \rm{Pro}(\cl{S}_{p})$ the category of $p$-\textit{profinite} spaces.
\end{definition}

\begin{remark}
We can regard $\cl{S}_{p}^{\vee}$ as the category of ``formal limits'' of $p$-finite spaces $\varprojlim X_{\alpha}$.
As such there is a functor $\cl{S}^{\vee}_{p}\to \cl{S}$ which takes a formal limit to the actual limit in $\cl{S}$.
This functor admits a left adjoint given by $Y \mapsto \flim_{Y_{\alpha} \to Y} Y_{\alpha}$, where the limit runs over all maps
from a $p$-finite space $Y_{\alpha}$ to $Y$.
\end{remark}

\begin{lemma}
  Let $X$ be a space and $\flim X_{\alpha}$ be its $p$-profinite completion. Then the natural map
  of cohomology rings
  \[ \fcolim \F_{p}^{X_{\alpha}} \to \F_{p}^{X} \]
  is an equivalence.
\end{lemma}
\begin{proof}
  This is immediate since the Eilenberg-MacLane spaces $K(\F_{p},n)$ are $p$-finite.
\end{proof}

\begin{proposition}[Mandell, Lurie]\label{coetal}
  Let $X$ be a space, then the $\F_{p}$-cohomology $\F_{p}^{X}$ is a formally \'etale $\F_{p}$-algebra.
\end{proposition}
\begin{proof}
  Since the functor $R \mapsto L_{R/\F_{p}}$ commutes with colimits, the claim follows from the fact that
  $L_{\F_{p}^{X}/\F_{p}}\simeq 0$ for every $p$-finite space $X$ which is proven
  in~\cite[][Proposition 2.4.12]{dag8}.
\end{proof}

Thus we obtain the following result about the homology coalgebra of a finite space $X$
with coefficients in a connective $\F_{p}$-algebra $R$:

\begin{corollary}\label{goal}
  Let $X$ be a finite space and $R$ be an $\F_{p}$-algebra, then $R[X]$ is a formally
  \'etale $R$-coalgebra.
\end{corollary}
\begin{proof}
  From Proposition~\ref{coetal} we get that
  \[ L_{R^{X}/R}\simeq L_{\F_{p}^{X}/\F_{p}}\otimes_{\F_{p}}R \simeq 0.\]
  Since $X$ is finite, the coalgebra $R[X]$ is dualizable with dual given by $R^{X}$, so the claim
  follows from Proposition~\ref{dualetal}.
\end{proof}

Moreover, for the case $R=\F_{p}$, we can use Theorem~\ref{wittsp} to give a partial answer to our
initial question about lifts of the coalgebra $\F_{p}[X]$.

\begin{corollary}
  Let $X$ be a finite space, then $\F_{p}[X]$ admits a unique lift to a $p$-complete $\S_{p}^{\wedge}$-coalgebra
  given by $W_{\S_{p}^{\wedge}}(\F_{p}[X]) \simeq (\S[X])^{\wedge}_{p}$. Moreover, for any other finite space $Y$
  the natural map
  \[\rm{Map}_{(\rm{cCAlg}_{\S_{p}^{\wedge}})^{\wedge}_{p}}((\S[Y])^{\wedge}_{p}, (\S[X])^{\wedge}_{p})
    \to \rm{Map}_{\rm{cCAlg}_{\F_{p}}}(\F_{p}[Y], \F_{p}[X])\]
  is a homotopy equivalence.
\end{corollary}
\begin{proof}
 Combine Corollary~\ref{goal} and Theorem~\ref{wittsp}.
\end{proof}

\section{Where to go from here}

We finish our discussion by explaining some of the shortcomings of our results and sketch a possible
way to proceed towards a coalgebraic analogue of Mandell's Theorem. The first missing puzzle piece is
the cotangent complex of a coalgebra $A$, which we have been unable to give a solid definition of.
The second and more important one is the relation to the \textit{coalgebra Frobenius}. We conjecture
that the class of \textit{perfect} coalgebras defined via this map give examples of non-dualizable
formally \'etale coalgebras. In particular, this conjecture would imply that the $\F_{p}$-homology
of \textit{any} space $X$ is formally \'etale.

\subsection{The cotangent complex of a coalgebra}
One of the first questions that arose during this project turned out to be one of the most subtle and
tricky ones, namely:

\begin{question}
  What is the cotangent complex of a coalgebra $A$?
\end{question}

Clearly, the existence of a single spectrum controlling the deformation theory of $A$ would be immensely
useful. However, it is not immediately clear what the universal property of such a spectrum should be,
i.e.~which space of derivations it should (co)represent.
Some inspiration can be gleamed from Proposition~\ref{cotangentder}. There we had seen that, for
$\varphi: B \to A$ a map of $R$-coalgebras with $A$ dualizable and $M$ an $R$-module, we have an equivalence
\[ \rm{Der}_{\varphi}(B, C_{A}(M)) \simeq \rm{Map}_{A^{\vee}}(L_{A^{\vee}/R}, \varphi^{\vee}_{\pt}\rm{map}_{R}(B, M)).\]
To get rid of the dependence on the second coalgebra $B$ one is tempted to take $B=R$ such that
$\rm{map}_{R}(B,M)\simeq M$. However, not every coalgebra $A$ admits a map $R\to A$, much less a canonical
one. The only natural choice for a map that is not the initial map would yield the following:

\begin{definition}[Preliminary 1.]
  Let $R$ be an $\bb{E}_{\infty}$-ring and $A\in \rm{cCAlg}_{R}$. The cotangent complex of $A$, if it exists,
  is the $R$-module $L_{A}$ corepresenting the functor
  \[ \rm{Mod}_{R}\to \rm{Mod}_{R} \qquad M \mapsto \rm{der}_{\id}(A, C_{A}(M))\]
\end{definition}

There are however several problems with this. Firstly, it is entirely unclear from the definition
whether $L_{A}$ vanishing would actually imply $A$ being formally \'etale. Moreover, in the dualizable
case it would lead to the rather awkward formula
\[ L_{A} \simeq L_{A^{\vee}/R}\otimes_{A^{\vee}}A.\]
Although somewhat plausible, this again gives us little information about what can actually be
deduced in the case that $L_{A}\simeq 0$.
This leaves us with several options, lest we accept that there is no good notion of one singular
cotangent complex. For one we could work with \textit{coaugmented} coalgebras, namely coalgebras
together with a map $R \to A$. For the purpose of understanding homology coalgebras this would correspond
to considering pointed spaces instead of just spaces, an entirely acceptable compromise, but beyond the
scope of this paper. \\
A different  approach would be to give up on the idea of corepresentability
and instead hope for a colimit preserving functor. For example, the functor
\[ \rm{Mod}_{R}\to \rm{Mod}_{R} \qquad M \mapsto C_{A}(M):=\rm{cofib}( A \rar{\eps} \Omega^{\infty}_{A}M).\]
seems to have no chance of preserving limits, but since colimits of coalgebras are formed underlying,
colimits are not out of the race. This leads us to the following idea:

\begin{definition}[Preliminary 2]\label{dream}
  Let $R$ be an $\bb{E}_{\infty}$-ring and $A\in \rm{cCAlg}_{R}$. We say that $A$ admits a cotangent
  complex $L_{A}:= C_{A}(R)$ if the functor $C_{A}(\blank):\rm{Mod}_{R} \to \rm{Mod}_{R}$ commutes
  with colimits. In this case we have $C_{A}(M)\simeq L_{A}\otimes M$ for every $ M \in \rm{Mod}_{R}$
\end{definition}

This definition is highly speculative, as the only coalgebras we know to admit a cotangent complex
in this sense are the formally \'etale coalgebras, for which the functor $C_{\blank}(A)$ is constant.
Conversely, if $A$ admits a cotangent complex then $L_{A}$ vanishes if and only if $A$ is formally
\'etale. Hence, the spectrum $L_{A}$ is precisely the obstruction to $A$ being formally \'etale,
which is the kind of conceptual clarity we are looking for.
While we lose any direct comparison to the cotangent complex of $A^{\vee}$ this is not entirely surprising,
since the property of being formally \'etale is defined very differently for $A^{\vee}$.
This leaves us with the following:

\begin{question}\label{cotangentdream}
  Let $R$ be an $\bb{E}_{\infty}$-ring. Does every $A \in \rm{cCAlg}_{R}$ admit a cotangent complex in the sense
  of Definition~\ref{dream}?
\end{question}

Regardless of the answer, the takeaway should be that the modules
$C_{A}(M)$ are exactly the obstruction towards $A$ being formally \'etale. Moreover, while the functor
$A\mapsto C_{A}(M)$ is very complicated, the dependence on $M$ should be relatively tame. That is,
for fixed $A$ it should be possible to describe the functor $M \mapsto C_{A}(M)$ in terms of a
formula involving $C_{A}(R)$. However, because $C_{A}(M)$ no longer has a direct relation to any
space of derivations or tangent complex, we cannot leverage results like Proposition~\ref{structure}
to obtain such a formula. We understand this as an indication that for these questions, the formalism may
have reached its limit.

\subsection{The Frobenius}
The most lacking thing about our results is the class of coalgebras that we can currently apply them to.
As of now, we are unable to give examples of formally \'etale coalgebras which are not dualizable. In
particular, we cannot describe the deformation theory of $R[X]$ for spaces $X$ which are not finite.
Attempts to reduce to the dualizable case all seem to fail for the following reason: Even though
we may write $X= \fcolim_{i}X_{i}$ where each $X_{i}$ is finite, giving the formula
$R[X]= \fcolim_{i}R[X_{i}]$, there is no reason why the functor
$\Omega^{\infty}_{\blank}(M): \rm{cCAlg}_{R}\to \rm{cCAlg}_{R}$ should commute with colimits.
Indeed, write $f_{M}:R\to R\oplus M$ for inclusion, then by definition
$\Omega^{\infty}_{\blank}(M) = f_{M,!} f^{\pt}_{M}$. The functor $f^{\pt}_{M}$ commutes with colimits,
and from Proposition~\ref{present} and the converse of the adjoint functor theorem we can deduce
that $f_{M,!}$ commutes with $\kappa$-filtered colimits for some regular cardinal $\kappa$. Thus, the class
of formally \'etale coalgebras is closed under $\kappa$-filtered colimits, but $\kappa$ is, in general, not countable.
% Closely related is the fact the notion of compactness is strangely behaved for coalgebras. For example,
% one can show that $\bb{Q}$ is not a compact object of $\rm{cCAlg}_{\Q}$, see~\cite[][Warning 1.2.15.]{ellII}.
% In particular, this means that
% \[ \rm{cSpec}(\fcolim_{i}\S[X_{i}])(\bb{Q})\neq \fcolim_{i}\rm{cSpec}(\S[X_{i}])(\Q),\]
% so we cannot deduce things about the cospectrum of infinite spaces in this way either. \\
This goes to show that the deformation theory of non-dualizable coalgebras is richer and more
interesting than that of the Ind-completion of dualizable coalgebras and requires additional input.
One contender for this additional input is the \textit{Coalgebra Frobenius} constructed by
Nikolaus:

\begin{theorem}[Nikolaus]
  Let $\cl{C} = (\rm{cCAlg}^{\rm{cn}}_{\S^{\wedge}_{p}})^{\wedge}_{p}$, then there exists a natural transformation
  $\psi_{p}:\id_{\cl{C}}\to \id_{\cl{C}}$ which on an object $A\in \cl{C}$ is given by the composition
  \[ \psi_{p}: A \rar{\Delta_{A}^{\otimes p}} (A^{\otimes p})^{hC_{p}} \rar{\rm{can}} (A^{\otimes p})^{tC_{p}} \rar{\sim} A,\]
  where the final map is the inverse of the \textit{Tate Diagonal}, see~\cite[][Theorem III.1.7]{tch}.
\end{theorem}

Given this map, we are naturally led to define \textit{perfect} coalgebras as follows:

\begin{definition}
  We say that $A \in  (\rm{cCAlg}^{\rm{cn}}_{\S^{\wedge}_{p}})^{\wedge}_{p}$ is \textit{perfect} if the coalgebra
  Frobenius $\psi_{p}: A\to A$ is a homotopy equivalence. We denote the full subcategory spanned by
  the perfect coalgebras by $(\rm{cCAlg}^{\rm{cn}}_{\S^{\wedge}_{p}})^{\wedge ,\rm{perf}}_{p} \subseteq
  (\rm{cCAlg}^{\rm{cn}}_{\S^{\wedge}_{p}})^{\wedge}_{p}$.
\end{definition}

\begin{example}\label{frobchains}
  Let $X$ be any space. Then $(\S[X])^{\wedge}_{p}$ is a perfect coalgebra since we have that
  \[\S[X]^{\wedge}_{p} \simeq (\S_{p}^{\wedge}[\colim_{X}\pt])^{\wedge}_{p} \simeq (\colim_{X} \S_{p}^{\wedge})^{\wedge}_{p}.\]
  On $\S_{p}^{\wedge}$ the map $\psi_{p}$ is necessarily given by the identity, because $\S_{p}^{\wedge}$
  is the terminal $p$-complete $\S_{p}^{\wedge}$-coalgebra. Thus, by naturality $\psi_{p}$ is given
  by the identity on $(\S[X])^{\wedge}_{p}$ as well.
\end{example}

We conjecture that this Frobenius map is related to the deformation theory of coalgebras in a similar
way to the Algebra Frobenius, in that it provides a sufficient condition for a coalgebra to be formally
\'etale.

\begin{conjecture}\label{frobcof}
  Let $A \in (\rm{cCAlg}^{\rm{cn}}_{\S^{\wedge}_{p}})^{\wedge}_{p}$ and write $A\p= A\otimes_{\S^{\wedge}_{p}}\F_{p}$.
  Then for any $M \in \rm{Mod}_{\F_{p}}^{\rm{cn}}$, the coalgebra Frobenius $\psi_p:A\to A$ induces the zero map
  on the $R$-module  $C_{A\p}(M) = \rm{cofib}(A\p \rar{\eta_{A\p}} \Omega^{\infty}_{A}(M))$.
\end{conjecture}

\begin{corollary}
  If Conjecture~\ref{frobcof} holds, then the base change functor
  \[ (\rm{cCAlg}^{\rm{cn}}_{\S^{\wedge}_{p}})^{\wedge ,\rm{perf}}_{p} \to \rm{cCAlg}_{\F_{p}}^{\rm{cn}}
  \qquad A \mapsto A\otimes_{\S_{p}^{\wedge}}\F_{p}\]
is fully faithful and factors through the full subcategory
$\rm{cCAlg}_{\F_{p}}^{\rm{cn}, \rm{f\acute{e}t}}\subseteq \rm{cCAlg}_{\F_{p}}^{\rm{cn}}$.
\end{corollary}
\begin{proof}
  Since $\psi_{p}:A\rar{\sim} A$ is an equivalence it induces an equivalence on $A\otimes_{\S_{p}^{\wedge}}\F_{p}$ and
  thus on $C_{A\otimes_{\S_{p}^{\wedge}}\F_{p}}(M)$ as well. However, since it also induces the zero map on the latter
  we get that $C_{A\otimes_{\S_{p}^{\wedge}}\F_{p}}(M)\simeq 0$. Thus, $A\otimes_{\S_{p}^{\wedge}}\F_{p}$ is formally \'etale and the
  claim follows from Theorem~\ref{wittsp}.
\end{proof}

Combining this with Example~\ref{frobchains} would allow us to fully answer our initial question about
homology coalgebras.

\begin{corollary}\label{dream2}
  If Conjecture~\ref{frobcof} holds, then for any space $X$ the $\F_{p}$-chains $\F_{p}[X]$
  are formally \'etale. In particular $\F_{p}[X]$ admits a unique and functorial lift to a $p$-complete
  $\S_{p}^{\wedge}$-coalgebra given by $\S[X]^{\wedge}_{p}= W_{\S_{p}^{\wedge}}(\F_{p}[X])$.
\end{corollary}

The fact that Conjecture~\ref{frobcof} needs to be checked for every connective $\F_{p}$-module should
be understood as an extension of our failure to find a cotangent complex. Indeed, if $\F_{p}[X]$ admits
a cotangent complex in the sense of Definition~\ref{dream}, then to obtain Corollary~\ref{dream2} it
would suffice to show that $\psi_{p}$ induces the zero map on $C_{A\otimes_{\S_{p}^{\wedge}}\F_{p}}(\F_{p})
= L_{A\otimes_{\S_{p}^{\wedge}}\F_{p}}$. However, even for this specific module the conjecture is difficult
to attack from our present position. The problem is the tricky right adjoint
$\rm{cCAlg}_{\F_{p}\oplus \F_{p}}\to \rm{cCAlg}_{\F_{p}}$ appearing in the definition of
$C_{A\otimes_{\S^{\wedge}_{p}}\F_{p}}(\F_{p})$. Because there is no known formula for this functor, attempts to verify
the conjecture have thus far been unsuccessful in all non-trivial cases. This warrants further investigation
of the coalgebra Frobenius and Conjecture~\ref{dream2}.

\section{Proof of Theorem \ref{thm:identification}}
\label{subsec:identification_thm_proof}
%
\begin{proof} Below, the symbol $\stackrel{AX}{=}$ and $\stackrel{DX}{=}$ imply that the equality follows from Assumption $X$ and Definition $X$, respectively. 
%
We begin with the proof of Theorem \ref{thm:identification} (a). 
%
\noindent \emph{Proof of Theorem \ref{thm:identification} (a):} For a donor unit $u \in \mathcal{I}$ and $\pi \in \Pi \setminus \Pi_u$, we have
\begin{align*}
    \E[Y^{(\pi)}_{u}~|~\mathcal{A}] & \stackrel{A\ref{ass:observation_model}}{=} \E[\langle \balpha_u, \bchi^{\pi} \rangle + \epsilon_u^{\pi} ~|~\mathcal{A}] \\ 
    & \stackrel{A\ref{ass:observation_model}(c)}{=} \langle \balpha_u, \bchi^\pi \rangle ~| ~\mathcal{A} \\
    & \stackrel{}{=} \langle \balpha_u, \bchi^\pi \rangle ~| ~\mathcal{A}, \ \mathcal{D} \\
    &  \stackrel{A\ref{ass:observation_model}(b)}{=} \langle \balpha_{u}, \tilde{\bchi}_{u}^{\pi} \rangle ~|~ \mathcal{A}, \ \mathcal{D}\\
    & \stackrel{A\ref{ass:donor_set_identification}(a)}{=} \langle \balpha_{u}, \sum_{\pi_u \in  \Pi_u} \beta_{\pi_{u}}^{\pi}\tilde{\bchi}_{u}^{\pi_u} \rangle  ~ | ~  \mathcal{A}, \ \mathcal{D} \\ 
    & = \sum_{\pi_u \in \Pi_u} \beta^{\pi}_{\pi_u} \langle \balpha_{u}, \tilde{\bchi}_{u}^{\pi_u}  \rangle  ~|~ \mathcal{A}, \ \mathcal{D} \\
    & \stackrel{A\ref{ass:observation_model}(c), A\ref{ass:selection_on_fourier}}{=}  \sum_{\pi_u \in \Pi_u} \beta^{\pi}_{\pi_u} \E[\langle \balpha_{u}, \tilde{\bchi}_{u}^{\pi_u}  \rangle + \epsilon_u^{\pi_u} ~ | ~ \mathcal{A}, \ \mathcal{D} ] \\
    & \stackrel{A\ref{ass:observation_model}}{=} \sum_{\pi_u \in \Pi_u} \beta^{\pi}_{\pi_u} \E[Y_{u,\pi_u} ~ | ~ \mathcal{A}, \ \mathcal{D}]
\end{align*}
%
\noindent \emph{Proof of Theorem \ref{thm:identification} (b):} For a donor unit $n \in [N] \setminus I^{D}$ and  $\pi \in \Pi \setminus \Pi_n$, we have
\begin{align*}
    \E[Y^{(\pi)}_{n}~|~\mathcal{A}] & \stackrel{A\ref{ass:observation_model}}{=} \E[\langle \balpha_n, \bchi^{\pi} \rangle + \epsilon_n^{\pi} ~|~\mathcal{A}] \\ 
    & \stackrel{A\ref{ass:observation_model}(c)}{=} \langle \balpha_n, \bchi^\pi \rangle ~| ~\mathcal{A} \\
    & \stackrel{}{=} \langle \balpha_n, \bchi^\pi \rangle ~| ~\mathcal{A}, \  \mathcal{D} \\
    & \stackrel{A\ref{ass:donor_set_identification}(b)}{=} \langle \sum_{u \in \mathcal{I}}w_u^n\balpha_u, \bchi^{\pi} \rangle  ~ | ~  \mathcal{A}, \ \mathcal{D} \\ 
    & = \sum_{u \in \mathcal{I}}w_u^n \langle \balpha_{u}, \bchi^{\pi} \rangle  ~|~ \mathcal{A}, \ \mathcal{D} \\
    & \stackrel{A\ref{ass:observation_model}(c), A\ref{ass:selection_on_fourier}}{=}  \sum_{u \in \mathcal{I}} w_u^n \E[\langle \balpha_{u}, \bchi^{\pi} \rangle + \epsilon_u^{\pi} ~ | ~ \mathcal{A}, \ \mathcal{D} ] \\
    & \stackrel{A\ref{ass:observation_model}}{=} \sum_{u \in \mathcal{I}} w_u^n \E[Y^{(\pi)}_{u} ~ | ~ \mathcal{A}, \ \mathcal{D}] \\
    & = \sum_{u \in \mathcal{I}} \sum_{\pi_u \in \Pi_u}  w_u^n \beta^{\pi}_{\pi_u} \E[Y_{u,\pi_u} ~ | ~ \mathcal{A},\  \mathcal{D}]
\end{align*}
where the last equality follows from Theorem \ref{thm:identification} (a). 
\end{proof}


\section{The \method~Estimator}\label{sec:estimator_descripton}
%
We now describe the \method~estimator, a simple and flexible two-step procedure for estimating our target causal parameter. 
%
In Figure \ref{fig:estimator}, we provide a pictorial representation of the estimator.
%

\vspace{2mm}
\noindent \textbf{Step 1: Horizontal Regression.}
For notational simplicity, we denote the vector of observed responses $\bY_{n,\Pi_n} = [Y_{n\pi} : \pi \in \Pi_n] \in \mathbf{R}^{|\Pi_n|}$ for any unit $n$ as $\bY_{\Pi_n}$.
%
Then, for every unit $u$ in the donor set $\mathcal{I}$, we estimate $\E[Y_u^{(\pi)}]$ via the Lasso, i.e., by solving the following convex program with penalty parameter $\lambda_u$: 
\begin{align}\label{eq:Lasso_estimator}
 \hat{\balpha}_u=  \argmin_{\balpha} \ \frac{1}{|\Pi_u|}\lVert \bY_{\Pi_u} - \bchi(\Pi_u)\balpha \rVert^2_2 + \lambda_u \lVert \bm{\alpha} \rVert_1
\end{align}
%
where recall that $\bchi(\Pi_u) = [\bchi^\pi: \pi \in \Pi_u] \in \mathbb{R}^{|\Pi_u| \times 2^p}$.
%
Then, for any donor unit-combination pair $(u,\pi)$, let $\hat{\E}[Y_u^{(\pi)}] = \langle  \hat{\balpha}_u, \bchi^{\pi} \rangle$ denote the estimate of the potential outcome $\E[Y_u^{(\pi)}]$.
%\begin{equation}
%\label{eq:Lasso_predicted_potential_outcome}
%    \hat{\E}[Y_u^{(\pi)}] = \langle  \hat{\balpha}_u, \bchi^{\pi} \rangle, 
%\end{equation}
 %

\vspace{2mm}
\noindent \textbf{Step 2: Vertical Regression.} 
%
Next, we estimate potential outcomes for all units $n \in [N] \setminus \mathcal{I}$.
%
To do so, we define some additional required notation. 
%
For $\Pi_S \subset \Pi$, define the vector of estimated potential outcomes $\hat{\E}[\bY^{(\Pi_S)}_{u}] = [ \hat{\E}[Y_u^{(\pi)}]: \pi \in \Pi^S] \in \R^{|\Pi_S|}$. 
%
Additionally, let $\hat{\E}[\bY^{(\Pi_S)}_{\mathcal{I}}] = [\hat{\E}[\bY^{(\Pi_S)}_{u}]: u \in \mathcal{I}] \in R^{|\Pi_S| \times |\mathcal{I}|}$.
%

\vspace{2mm}
\noindent \emph{Step 2(a): Principal Component Regression.} Perform a singular value decomposition (SVD) of $\hat{\E}[\bY^{(\Pi_n)}_{\mathcal{I}}]$ to get $\hat{\E}[\bY^{(\Pi_n)}_{\mathcal{I}}] = \sum^{\min(|\Pi_n|,|\mathcal{I}|)}_{l = 1} \hat{s_l}\hat{\bmu}_{l}\hat{\bnu}^T_{l}$. Using a hyper-parameter $\kappa \leq \min(|\Pi_n|,|\mathcal{I}|)$\footnote{Both $\lambda$ and $\kappa$ can be chosen in a data-driven manner (e.g., via cross-validation) as discussed in \cite{chetverikov2021cross} and \cite{agarwal2020principal} respectively.}, compute $\hat{\bw}^{n} \in \mathbb{R}^{|\mathcal{I}|}$ as follows:
\begin{equation}
\label{eq:pcr_linear_model_def}
    \hat{\bw}^{n} = \left(\sum^{\kappa}_{l = 1} \hat{s_l}^{-1}\hat{\bnu}_{l}\hat{\bmu}^T_{l}\right)\bY_{\Pi_n} 
\end{equation}

\noindent \emph{Step 2(b): Estimation.} Using $\hat{\bw}^{n} = [\hat{w}_u^n : u \in \mathcal{I}]$, we have the following estimate for any intervention $\pi \in \Pi$
\begin{equation}
\label{eq:potential_outcome_estimate_vertical_regression}
     \hat{\E}[Y_n^{(\pi)}] = \sum_{u \in \mathcal{I}} \hat{w}_u^{n} \hat{\E}[Y_u^{(\pi)}]
\end{equation}
%

\begin{figure}[htbp]
    \centering
    \includegraphics[width = 0.9\textwidth]{figures/estimator_v7.png}
    \caption{A visual description of \method. Figure \ref{fig:estimator}(a) depicts an example of a particular observation pattern with outcome for unit-combination pair $(n,\pi)$ missing. Figure \ref{fig:estimator}(b) demonstrates horizontal regression for donor unit $u$ where we estimate potential outcome $\E[Y_u^{(\pi)}]$. Figure  \ref{fig:estimator}(c) visualizes vertical regression where we transfer estimated outcomes from the donor set $\mathcal{I}$ to the unit-combination pair $(n,\pi)$. }
    \label{fig:estimator}
\end{figure}

\noindent \textbf{Suitability of Lasso and PCR.} Lasso is appropriate for the horizontal regression step because it adapts to the sparsity of $\balpha_u$. However, it can be replaced with other algorithms that adapt to sparsity (see below for a larger discussion). For vertical regression, PCR is appropriate because  $\mathcal{A}$ is low rank. As \cite{agarwal2019robustness,agarwal2020principal} show, PCR implicitly regularizes the regression by adapting to the rank of the covariates ($\bY_{\Pi_n}$), i.e., the out-of-sample error of PCR scales with $r$ rather than the ambient covariate dimension. 
\vspace{1mm}

%
\noindent \textbf{Determining Donor Set $\mathcal{I}$.} 
%
\method~requires the existence of a subset of units $\mathcal{I} \subset [N]$ such that we are able to (i) accurately estimate their potential outcomes under all possible combinations, and (ii) transfer these estimated outcomes to a unit $n \in [N] \setminus \mathcal{I}$.
%
Theoretically, we detail sufficient conditions on the observation pattern such that we are able to perform (i) and (ii) accurately via the Lasso and PCR respectively. 
%
In practice, we recommend the following practical guidance to determining the donor set $\mathcal{I}$.
%
For every unit $n \in [N]$, learn a separate Lasso model $\balpha_n$ and assess its performance through $k$-fold cross-validation (CV). 
%
Assign units with low CV error (with a pre-determined threshold) as the donor set $\mathcal{I}$, and estimate outcomes $\hat{\E}[Y_u^{(\pi)}]$ for every unit $u \in \mathcal{I}$ and $\pi \in \Pi$. 
%
For any non-donor unit $n \in [N] \setminus \mathcal{I}$, we learn a model via PCR as discussed in step 2 of \method~and assess the performance of PCR through CV. 
%
For units with low PCR error, linear span inclusion (Assumption \ref{ass:donor_set_identification}(b)) and the assumptions required for the generalization for PCR likely hold, and hence we estimate their potential outcomes as in \eqref{eq:potential_outcome_estimate_vertical_regression}. 
%
For units with large PCR error, it is either unlikely that these set of assumptions holds or that $|\Pi_n|$ is not large enough (i.e., additional experiments need to be run for this unit), and hence we do not recommend estimating their counterfactuals. 
%

\vspace{2mm}
\noindent \textbf{Horizontal Regression Model Selection.} \method~allows for any ML algorithm (e.g., random forests, neural networks, ensemble methods) to be used in the first step. 
%
We provide an example of this flexibility by showing how the horizontal regression can also be done via CART in Appendix \ref{sec:CART_horizontal_regression}.
%
Our theoretical analysis of CART shows that it leads to better finite-sample rates under stronger regularity conditions on the potential outcomes $\E[Y_n^{(\pi)}]$ (see Corollary \ref{cor:potential_outcome_convergence_rate_CART}). 
%
This model-agnostic approach allows the analyst to tailor the horizontal learning procedure to the data at hand and include additional structural information for better performance. 
%
However, for simplicity, we focus on the Lasso for the remainder of this paper. 



%we estimate the target causal parameter an estimate of the target causal  causal parameter for any unit $u \in \mathcal{I}$, and $\pi \in \Pi$ can be constructed as follows: 
%\begin{equation}
%\label{eq:potential_outcome_estimate_horizontal_regression}
%    \hat{\E}[Y_u^{(\pi)}] = \hat{f}_u (\pi), 
%\end{equation} 

%{\color{red} Has $Y_{\pi_u}$ been defined? Not following this object - $\{\pi_u, Y_{\pi_u}\}^{|\Pi_u|}_{u=1}$}

\input{content/finite_sample_analysis_tightened}
\section{Experiment Design}
\label{sec:experimental_design}
%
In this section, we show how \method~can be used to design experiments that allow for combinatorial inference (i.e., learning all $N \times 2^p$ causal parameters). 

\vspace{2mm}
\noindent \textbf{Key Assumptions Behind Finite Sample Consistency of \method.} \method~requires the existence of a donor set $\mathcal{I}$ such that we are able to perform accurate horizontal regression for all donor units, and then transfer these estimated outcomes to non-donor units via PCR. 
%
The enabling conditions for accurate horizontal regression  are (i) horizontal span inclusion (Assumption \ref{ass:donor_set_identification} (a)), and (ii) incoherence of the Fourier characteristics (Assumption \ref{ass:incoherence}).   
%
Similarly, the critical assumptions required for consistency of PCR are (i) linear span inclusion (Assumption  \ref{ass:donor_set_identification} (b)), (ii) well-balanced spectrum (Assumption \ref{ass:balanced_spectrum}), and (iii) subspace inclusion (Assumption \ref{ass:rowspace_inclusion}). 
%
In terms of experiment design, the donor set and treatment assignments must be carefully chosen such that these key assumptions hold.
%
To that end, we introduce the following design. 
%
%For concreteness, we assume that horizontal regression is done via the Lasso, however our experimental design scheme easily applies to the CART estimator. 
%

\vspace{2mm}
\noindent \textbf{Experimental Design Mechanism.} Fix a probability threshold $\gamma \in (0,1)$ and estimation error threshold $\delta \in (0,1)$. Our design mechanism (see Figure \ref{fig:experiment_design_observation_pattern} for a a visual description) then proceeds as follows.

\vspace{1mm}
\noindent
 \emph{Step 1: Donor set selection.} Choose the donor set $\mathcal{I} \subset [N]$ by sampling a subset of units independently and uniformly at random with size satisfying $\Omega\left(r\log(rs/\gamma) \right)$. 

\vspace{1mm}
\noindent 
\emph{Step 2: Donor set treatment assignment.} Sample $\Pi_{\mathcal{I}} \subset \Pi$ combinations independently and uniformly at random with size satisfying $|\Pi_{\mathcal{I}}| = \Omega\left(\frac{r^3s^2\log(|\mathcal{I}|2^p/\gamma)}{\delta^2}\right)$. Assign all donor units $u \in \mathcal{I}$ this set of combinations. 
%

\vspace{1mm}
\noindent
\emph{Step 3: Non-donor unit treatment assignment.} Randomly sample $\Pi_N \subset \Pi$ combinations independently and uniformly at random  of size $|\Pi_N| = 
\Omega\left(r\log(|\mathcal{I}|/\gamma) \vee r^4/\delta^4 \right)$. Assign all non-donor units $n \in [N] \setminus \mathcal{I}$ combinations $\Pi_N$. 
%

\begin{figure}[htbp]
    \centering
    \includegraphics [width = \textwidth]{figures/experiment_design_figure_paper.pdf}
    \caption{Observation pattern induced by experiment design mechanism. }
    \label{fig:experiment_design_observation_pattern}
\end{figure}


\vspace{1mm}
\noindent
\emph{Potential outcome estimation.} Given the observation pattern described above, estimate outcomes for each unit-combination pair via \method~as described in Section \ref{sec:estimator_descripton}. 
\vspace{0.5mm}
%

It turns out that this simple design mechanism satisfies the key assumptions presented above with high probability. 
%
In fact, the proposed mechanism ensures these key conditions hold under very limited assumptions.
%
The only required assumptions are that: (i) the potential outcome model is satisfied (Assumption \ref{ass:observation_model}), (ii) $\E[Y_n^{(\pi)}]$ is bounded (Assumption \ref{ass:boundedness_potential_outcome}). 
%
Additionally, only a weakened version of the balanced spectrum condition is needed (Assumption \ref{ass:balanced_spectrum}):
%
\begin{assumption} [Restricted Balanced Spectrum] 
\label{ass:restricted_balanced_spectrum}
%
Let $s_{1} \ldots s_{r}$ denote the non-zero singular values of $\E[\bY_N^{(\Pi)}]$. 
%
Assume that its singular values are well-balanced, i.e., for universal constants  $c,c' > 0$, we have that $s_{r}/s_{1} \geq c$, and $\lVert \E[\bY^{(\Pi)}_{N} ~ | ~ \mathcal{A} ]\rVert^2_F \geq c'N2^p$
%
\end{assumption}
%
\noindent As compared to the original balanced spectrum condition, Assumption \ref{ass:restricted_balanced_spectrum} only requires that the singular values are balanced for the entire potential outcome matrix as opposed to a collection of submatrices. 
%
We then have the following result.
%
\begin{theorem} 
\label{thm:experiment_design_assumptions_hold}
Let Assumptions \ref{ass:observation_model}, \ref{ass:boundedness_potential_outcome}, and \ref{ass:restricted_balanced_spectrum} hold. 
%
Then, the proposed experimental design mechanism ensures satisfies Assumption \ref{ass:selection_on_fourier}, and the following conditions simultaneously with probability at least $1 - \gamma$: 
%
(i) horizontal and linear span inclusion (Assumption \ref{ass:donor_set_identification}), 
%
(ii) incoherence of donor unit Fourier characteristics (Assumption \ref{ass:incoherence}), well-balanced spectrum (Assumption \ref{ass:balanced_spectrum}) and subspace inclusion (Assumption \ref{ass:rowspace_inclusion}).
%
\end{theorem}
%
\begin{corollary} 
\label{cor:experimental_design_error_rate}
Let the set-up of Theorem \ref{thm:experiment_design_assumptions_hold} hold. Then, for every unit-combination pair $(n,\pi)$, we have $|\E[Y_n^{(\pi)}] - \hat{\E}[Y^{(\pi)}_n]| = \Tilde{O}_p (\delta)$.  
\end{corollary}
%
\noindent  Theorem \ref{thm:experiment_design_assumptions_hold}  implies that the key enabling conditions for \method~are satisfied for every unit-combination pair $(n,\pi)$. 
%
% That is, the required assumptions on the observation pattern for meaningful estimation of the entire potential outcome matrix $\E[\bY_N^{(\Pi)}]$ are met.
%
Corollary \ref{cor:experimental_design_error_rate} further establishes that with this design, $O(\delta)$ error is achievable for all $N \times 2^p$ causal parameters.  
%
In contrast, the results in the observational setting do not guarantee accurate estimation of {\em all} $N \times 2^p$ parameters.
%
Instead, they establish that $\E[Y_n^{(\pi)}]$ can be learned for any specific unit-combination pair $(n,\pi)$ that satisfies the required assumptions on the observation pattern. 
%
Additionally, given the discussion in Section \ref{subsec:sample_complexity_synth_combo}, one can verify that the number of observations required by this experiment design mechanism scales as $\tilde{O}\left(\text{poly}(r/\delta)\times \left(N + s^2p \right) \right)$. 
%
In practice, this design requires knowledge of $r,$ and $s$.
%
To overcome this, one can sequentially sample donor units and their observations until the rank and lasso error stabilizes. 
%
This provides an estimate of $r,$ and $s$; a formal analysis of this procedure is left as future work.

%(ii) there is selection on Fourier coefficients (Assumption \ref{ass:selection_on_fourier}),} 
\allowdisplaybreaks

\section{Asymptotic Normality}
\label{sec:asymptotic_normality_supp}

In this section, we provide proofs for Proposition \ref{prop:horizontal_asymptotic_normality} and Theorem \ref{thm:vertical_regression_normality}.


%we first discuss the required conditions for the SR estimator introduced in Section \ref{sec:asymptotic_normality} to achieve support recovery (condition (e) in Proposition %\ref{prop:horizontal_asymptotic_normality}), i.e., for $\P(\hat{\mathcal{S}}_u \neq \mathcal{S}_u) = O(e^{-|\Pi_u|^{c_3}})$.
%%
%Then, we provide proofs for Proposition \ref{prop:horizontal_asymptotic_normality} and Theorem \ref{thm:vertical_regression_normality}.
%


%\subsection{Assumptions for Support Recovery}
%\label{subsec:support_recovery}



%we present a result by \cite{zhao2006model} and the required assumptions to establish condition (e) of Proposition \ref{prop:horizontal_asymptotic_normality}.



 %\item [(d)] $\lambda_{\min}\left(K_u \right) \geq c_{3}$ where $c_{3}$ is a positive constant,
    %\item [(e)] For $\eta > 0$, assume $\max_{S \in \mathcal{S}^c_u} \lVert K_u^{-1} \bchi^T_{\mathcal{S}_u}\bchi_{S}  \rVert_1 \leq \eta|\Pi_u|$,

\subsection{Proof of Proposition \ref{prop:horizontal_asymptotic_normality}}
For ease of exposition, we suppress conditioning on $\mathcal{A}$.
%
Then, from the law of total expectation, this yields, 
\begin{align*}
    &\hat{\E}_{SR}[Y_u^{(\pi)}] - \E[Y_u^{(\pi)}] 
    %
    \\ &= \left( \hat{\E}_{SR}[Y_u^{(\pi)} | \hat{\mathcal{S}}_u = \mathcal{S}_u] -  \E[Y_u^{(\pi)}] \right) \P(\hat{\mathcal{S}}_u = \mathcal{S}_u) + \left( \hat{\E}_{SR}[Y_u^{(\pi)} | \hat{\mathcal{S}}_u \neq \mathcal{S}_u]  - \E[Y_u^{(\pi)}]\right) \P(\hat{\mathcal{S}}_u \neq \mathcal{S}_u) 
    %
    \\ & =  \left( \hat{\E}_{SR}[Y_u^{(\pi)} | \hat{\mathcal{S}}_u = \mathcal{S}_u] -  \E[Y_u^{(\pi)}] \right) \P(\hat{\mathcal{S}}_u = \mathcal{S}_u) +  \hat{\E}_{SR}[Y_u^{(\pi)} | \hat{\mathcal{S}}_u \neq \mathcal{S}_u]  \P(\hat{\mathcal{S}}_u \neq \mathcal{S}_u)  - \E[Y_u^{(\pi)}] \P(\hat{\mathcal{S}}_u \neq \mathcal{S}_u) 
 \end{align*}

\noindent We scale each of three terms in the equation above by $\sqrt{|\Pi_u|}/\sqrt{\sigma^2(\bchi^{\pi})^T\mathbf{K}^{-1}_u\bchi^{\pi}}$, and analyze each of them separately. 

\noindent \emph{Term 1.} 
To proceed, we state the following lemma.
\begin{lemma}
\label{lem:fourier_coefficient_asymptotic_normality}
Let the set-up of Proposition \ref{prop:horizontal_asymptotic_normality} hold. Then, as $|\Pi_u| \rightarrow \infty$, we have, 
\begin{equation}
\label{eq:fourier_coefficient_asymptotic_normality}
     \sqrt{\frac{|\Pi_u|}{\sigma^2(\bchi^{\pi})^T\mathbf{K}^{-1}_u\bchi^{\pi}}}\left( \hat{\E}_{SR}[Y_u^{(\pi)} | \hat{\mathcal{S}}_u = \mathcal{S}_u ] - \E[Y_u^{(\pi)}] \right)\xrightarrow{d} N(0,1).
\end{equation}
\end{lemma}

   
\noindent Next, from condition (e) of Proposition \ref{prop:horizontal_asymptotic_normality}, as  $|\Pi_u| \rightarrow \infty$, $\P(\hat{\mathcal{S}}_u = \mathcal{S}_u) \rightarrow 1$.
%
Then, using this observation, substituting \eqref{eq:fourier_coefficient_asymptotic_normality} into term 1, and using Slutsky's theorem, gives the following
\begin{equation}
\label{eq:horizontal_regression_term1_normality}
      \sqrt{\frac{|\Pi_u|}{\sigma^2(\bchi^{\pi})^T\mathbf{K}^{-1}_u\bchi^{\pi}}}\left( \hat{\E}_{SR}[Y_u^{(\pi)} | \hat{\mathcal{S}}_u = \mathcal{S}_u] -  \E[Y_u^{(\pi)}] \right) \P(\hat{\mathcal{S}}_u = \mathcal{S}_u) \xrightarrow{d} N (0,1),
\end{equation}
as $|\Pi_u| \rightarrow \infty$.
\vspace{1mm}


\noindent \emph{Term 2.} We introduce the following lemma for the analysis of term 2. 
\begin{lemma}
\label{lem:largest_eigenvalue_covariance_matrix}
Let the set-up of Proposition \ref{prop:horizontal_asymptotic_normality} hold. Then, we have 
\begin{equation*}
      \sqrt{\frac{|\Pi_u|}{\sigma^2(\bchi^{\pi})^T\mathbf{K}^{-1}_u\bchi^{\pi}}} = O\left(\sqrt{\frac{s|\Pi_u|}{2^p}} \right)
\end{equation*}
    
\end{lemma}

\noindent Substituting the result of Lemma \ref{lem:largest_eigenvalue_covariance_matrix}, and conditions (d) and (e) into the expression for term 2 yields,
\begin{equation}
\label{eq:horizontal_regression_term2_intermediate1}
    \sqrt{\frac{|\Pi_u|}{\sigma^2(\bchi^{\pi})^T\mathbf{K}^{-1}_u\bchi^{\pi}}} \hat{\E}_{SR}[Y_u^{(\pi)} |\hat{\mathcal{S}}_u  \neq \mathcal{S}_u]  \P(\hat{\mathcal{S}}_u \neq \mathcal{S}_u) = o\left(\sqrt{\frac{|\Pi_u|^{1+c_2}}{2^p} }e^{-|\Pi_u|^{c_3}} \hat{\E}_{SR}[Y_u^{(\pi)} | \hat{\mathcal{S}}_u  \neq \mathcal{S}_u] \right)
\end{equation}
     
\noindent We continue by stating the following lemma.

\begin{lemma}
\label{lem:ridge_bound}
Let the set-up of Proposition \ref{prop:horizontal_asymptotic_normality} hold. Then, we have that
\begin{equation*}
   \hat{\E}_{SR}[Y_u^{(\pi)} | \hat{\mathcal{S}}_u  \neq \mathcal{S}_u]= O_p \left(|\Pi_u| \sqrt{2^{p}} \right)
\end{equation*}
\end{lemma}

\noindent Then, substituting the result of Lemma \ref{lem:ridge_bound} into \eqref{eq:horizontal_regression_term2_intermediate1} gives us 
\begin{equation}
\label{eq:horizontal_regression_term2}
    \sqrt{\frac{|\Pi_u|}{\sigma^2(\bchi^{\pi})^T\mathbf{K}^{-1}_u\bchi^{\pi}}} \hat{\E}_{SR}[Y_u^{(\pi)} | \hat{\mathcal{S}}_u  \neq \mathcal{S}_u]  \P(\hat{\mathcal{S}}_u \neq \mathcal{S}_u) = o_p \left(\sqrt{|\Pi_u|^{3+c_2}} e^{-|\Pi_u|^{c_3}} \right) = o_p(1)
\end{equation}

\vspace{1mm}


\noindent \emph{Term 3.} By Lemma \ref{lem:largest_eigenvalue_covariance_matrix}, Assumption \ref{ass:boundedness_potential_outcome}, conditions (d) and (e) of Proposition \ref{prop:horizontal_asymptotic_normality}, we have 
\begin{equation}
\label{eq:horizontal_asymptotic_normalty_term3_bound}
     \sqrt{\frac{|\Pi_u|}{\sigma^2(\bchi^{\pi})^T\mathbf{K}^{-1}_u\bchi^{\pi}}} \E[Y_u^{(\pi)}]\P(\hat{\mathcal{S}}_u \neq \mathcal{S}_u) =  o\left(\sqrt{\frac{|\Pi_u|^{1+c_2}}{2^p}} e^{-|\Pi_u|^{c_3}}\right) = o(1), 
\end{equation}

\vspace{1mm}

\noindent Collecting \eqref{eq:horizontal_regression_term1_normality}, \eqref{eq:horizontal_regression_term2}, and \eqref{eq:horizontal_asymptotic_normalty_term3_bound} gives us the claimed result.

\subsection{Helper Lemmas for Proposition \ref{prop:horizontal_asymptotic_normality}}


\subsubsection{Proof of Lemma \ref{lem:fourier_coefficient_asymptotic_normality}}
\noindent We establish some notation required for the proof. 
%
Define $\balpha_{\mathcal{S}_u} = [\alpha_{u,S}: S \in \mathcal{S}_u] \in \mathbb{R}^{|\mathcal{S}_u|}$, and $\hat{\balpha}^{SR}_{\mathcal{S}_u} = [\hat{\alpha}^{SR}_{u,S}: S \in \mathcal{S}_u] \in \mathbb{R}^{|\mathcal{S}_u|}$.
%
Then, we begin our proof by stating the following lemma, and state the delta theorem. 

\begin{lemma} [Theorem 3 of \cite{liu2013asymptotic}]
\label{lem:fourier_coefficient_asymptotic_normality_liu}
Let the set-up of Proposition \ref{prop:horizontal_asymptotic_normality} hold. Then, as $|\Pi_u| \rightarrow \infty$, we have that 
\begin{equation*}
    \sqrt{|\Pi_u|} \left( \hat{\balpha}^{SR}_{\mathcal{S}_u} - \balpha_{\mathcal{S}_u} \right) \xrightarrow{d} N\left(0,\sigma^2\mathbf{K}^{-1}_u\right)
\end{equation*}
\end{lemma}

\begin{theorem} [Delta Theorem \cite{ding2024linear}] 
\label{thm:delta_theorem}
Let $f(\mathbf{z})$ be a function from $\mathbb{R}^p \rightarrow \mathbb{R}$, and $\frac{\partial f(\bz)}{\partial \bz} \in \mathbb{R}^p$ denote the partial derivative of $f$ with respect to $\bz$.
%
Additionally, suppose $\sqrt{n}(\mathbf{Z}_n - \mathbf{\theta}) \xrightarrow{d} N(0,\Sigma)$ as $n \rightarrow \infty$. 
%
Then, we have
\begin{equation*}
    \sqrt{n}\left(f(\mathbf{Z}_n) - f(\mathbf{\theta}) \right) \xrightarrow{d} N(0,(\frac{\partial f(\bz)}{\partial \bz})^T \Sigma \frac{\partial f(\bz)}{\partial \bz} )
\end{equation*}
as $n \rightarrow \infty$. 
\end{theorem}

\noindent Applying Lemma \ref{lem:fourier_coefficient_asymptotic_normality}, and Theorem \ref{thm:delta_theorem} (with $f(\cdot) = \langle \cdot, \bchi^\pi \rangle$) gives us 
\begin{equation*}
     \sqrt{|\Pi_u|} \left(\hat{\E}_{SR}[Y_u^{(\pi)} | \hat{\mathcal{S}}_u =\mathcal{S}_u ]  - \E[Y_u^{(\pi)}] \right) =  \sqrt{|\Pi_u|} \left(\langle  \hat{\balpha}^{SR}_{\mathcal{S}_u}, \bchi^{\pi} \rangle - \langle  \balpha_{\mathcal{S}_u}, \bchi^{\pi} \rangle  \right) \xrightarrow{d} N(0, \sigma^2 (\bchi^{\pi})^T\mathbf{K}^{-1}_u \bchi^{\pi} ),
\end{equation*}
as $|\Pi_u| \rightarrow \infty$. 
%
Scaling both sides of the equation above by $\sqrt{1/\sigma^2 (\bchi^{\pi})^T\mathbf{K}^{-1}_u \bchi^{\pi}}$ finishes the proof. 

\subsubsection{Proof of Lemma \ref{lem:largest_eigenvalue_covariance_matrix}} 

We begin by defining some notation.
%
For a square matrix $\bX$, let $\lambda_{\min}(\bX)$ and let $\lambda_{\max}(\bX)$ denote the minimum and maximum eigenvalues of $\bX$ respectively. 
%
Additionally, for a positive-definite invertible matrix $\bX$, recall the fact that $\lambda_{\min}(\bX^{-1}) = 1/\lambda_{\max}(\bX)$.
%
Finally, for a general matrix $\mathbf{A}$, let $s_{\max}(\mathbf{A})$ denote its largest singular value. 
%
Then, this gives

\begin{align}
    \sqrt{\frac{|\Pi_u|}{\sigma^2(\bchi^{\pi})^T\mathbf{K}^{-1}_u\bchi^{\pi}}}  & \leq \sqrt{\frac{|\Pi_u|}{\sigma^2 \lambda_{\min}(\mathbf{K}^{-1}_u) (\bchi^{\pi})^T \bchi^{\pi}}}  \nonumber \\
    & = \sqrt{\frac{|\Pi_u| \lambda_{\max}(\mathbf{K}_u) }{\sigma^2 (\bchi^{\pi})^T \bchi^{\pi}}}  \nonumber \\
    & = \sqrt{\frac{|\Pi_u| \lambda_{\max}(\mathbf{K}_u) }{\sigma^2 2^p}} \nonumber \\
    & =  \sqrt{\frac{\lambda_{\max}((\bchi_{\mathcal{S}_u}(\Pi_u))^T \bchi_{\mathcal{S}_u}(\Pi_u)) }{\sigma^2 2^p}} \nonumber \\
    & =  \frac{s_{\max}(\bchi_{\mathcal{S}_u}(\Pi_u))}{\sqrt{\sigma^2 2^p}} ,\label{eq:horizontal_regression_term2_largest_singular_value}
\end{align}
where in the last line we use the fact that for a matrix $\mathbf{A}$, $s_{\max}(\mathbf{A}) = \sqrt{\lambda_{\max}(\mathbf{A}^T\mathbf{A})}$. 
%
To proceed, recall that $s_{\max}(\mathbf{A}) \leq \lVert \mathbf{A} \rVert_{F}$, where $\lVert \cdot \rVert_{F}$ denote the Frobenius norm.
%
Since $\bchi_{\mathcal{S}_u}(\Pi_u) \in \{-1,1\}^{|\Pi_u| \times |\mathcal{S}_u|}$, we have $\lVert \bchi_{\mathcal{S}_u}(\Pi_u) \rVert_{F} = \sqrt{|\Pi_u| \times |\mathcal{S}_u|} \leq \sqrt{s|\Pi_u|}$. 
%
Substituting $s_{\max}(\bchi_{\mathcal{S}_u}(\Pi_u)) \leq \sqrt{s|\Pi_u|}$ into \eqref{eq:horizontal_regression_term2_largest_singular_value} gives us
\begin{equation*}
     \sqrt{\frac{|\Pi_u|}{\sigma^2(\bchi^{\pi})^T\mathbf{K}^{-1}_u\bchi^{\pi}}}  = O\left(\sqrt{\frac{s|\Pi_u|}{2^p}}\right),
\end{equation*}
which is the claimed result. 

\subsubsection{Proof of Lemma \ref{lem:ridge_bound}}
Using Cauchy-Schwarz gives,
\begin{equation*}
    \hat{\E}_{SR}[Y_u^{(\pi)} | \hat{\mathcal{S}}_u \neq \mathcal{S}_u] = \langle \hat{\balpha}_u^{SR}, \bchi^{\pi} \rangle \leq \lVert  \hat{\balpha}_u^{SR} \rVert_2 \lVert \bchi^{\pi} \rVert_2 =  \sqrt{2^p} \lVert  \hat{\balpha}^{SR}_{u} \rVert_2 
\end{equation*}
\noindent Next, we upper bound $\lVert  \hat{\balpha}^{SR}_{u} \rVert_2 $.
%
Let $\mathbf{U}_{\hat{\mathcal{S}}_u} \mathbf{D}_{\hat{\mathcal{S}}_u} \mathbf{V}^T_{\hat{\mathcal{S}}_u}$ denote the SVD of $\bchi_{\hat{\mathcal{S}}_u}(\Pi_u)$. 
%
Then, substituting the SVD of $\bchi_{\hat{\mathcal{S}}_u}(\Pi_u)$ into the definition of $\hat{\balpha}^{SR}_{u}$ (see \eqref{eq:Ridge_estimator}), it is easy to obtain 
\begin{equation} \label{eq:ridge_SVD_1}
    \hat{\balpha}^{SR}_{u} = \mathbf{V}_{\hat{\mathcal{S}}_u}\left(\mathbf{D}^2_{\hat{\mathcal{S}}_u} + \frac{1}{|\Pi_u|} \mathbf{I}_{|\hat{\mathcal{S}}_u|}\right)^{-1}\mathbf{D}_{\hat{\mathcal{S}}_u} \mathbf{U}^T_{\hat{\mathcal{S}}_u} \bY_{\Pi_u}
\end{equation}

\noindent To proceed, define the diagonal matrix $\mathbf{D}'_{\hat{\mathcal{S}}_u} =  \left(\mathbf{D}^2_{\hat{\mathcal{S}}_u} + \frac{1}{|\Pi_u|} \mathbf{I}_{|\hat{\mathcal{S}}_u|}\right)^{-1} \mathbf{D}_{\hat{\mathcal{S}}_u} $.
%
Then, using \eqref{eq:ridge_SVD_1}, we obtain the following upper bound for $\lVert  \hat{\balpha}^{SR}_{u} \rVert^2_2$.
\begin{align}
    \lVert  \hat{\balpha}^{SR}_{u} \rVert^2_2 & = \left(\mathbf{V}_{\hat{\mathcal{S}}_u}  \mathbf{D}'_{\hat{\mathcal{S}}_u} \mathbf{U}^T_{\hat{\mathcal{S}}_u} \bY_{\Pi_u} \right)^T \left(\mathbf{V}_{\hat{\mathcal{S}}_u}  \mathbf{D}'_{\hat{\mathcal{S}}_u} \mathbf{U}^T_{\hat{\mathcal{S}}_u} \bY_{\Pi_u} \right) \nonumber \\
    &  = \bY^T_{\Pi_u} \mathbf{U}_{\hat{\mathcal{S}}_u} \mathbf{D}'_{\hat{\mathcal{S}}_u} \mathbf{V}^T_{\hat{\mathcal{S}}_u} \mathbf{V}_{\hat{\mathcal{S}}_u}  \mathbf{D}'_{\hat{\mathcal{S}}_u} \mathbf{U}^T_{\hat{\mathcal{S}}_u} \bY_{\Pi_u} \nonumber\\
    & = \bY^T_{\Pi_u} \mathbf{U}_{\hat{\mathcal{S}}_u} (\mathbf{D}'_{\hat{\mathcal{S}}_u})^2 \mathbf{U}^T_{\hat{\mathcal{S}}_u} \bY_{\Pi_u} \nonumber \\
    & \leq s_{\max}\left((\mathbf{D}'_{\hat{\mathcal{S}}_u})^2\right) \lVert \mathbf{U}^T_{\hat{\mathcal{S}}_u} \bY_{\Pi_u} \rVert^2_2 \nonumber \\
    & \leq  s_{\max}\left((\mathbf{D}'_{\hat{\mathcal{S}}_u})^2\right) \lVert \mathbf{U}_{\hat{\mathcal{S}}_u}  \rVert^2_2 
 \lVert \bY_{\Pi_u} \rVert^2_2 \nonumber \\
    & \leq s_{\max}\left((\mathbf{D}'_{\hat{\mathcal{S}}_u})^2\right) \lVert \bY_{\Pi_u} \rVert^2_2 \label{eq:ridge_SVD_2}
\end{align}
%
\noindent where the last inequality follows from the fact that $\mathbf{U}_{\hat{\mathcal{S}}_u}$ is a orthonormal matrix. 
%


For the rest of the proof, some additional notation is required.
%
Define $\bepsilon^{\Pi_u} = [\epsilon_n^{\pi}: \pi \in \Pi_u] \in \R^{|\Pi_u|}$.
%
Let $s_{i}(\bchi_{\hat{\mathcal{S}}_u}({\Pi_u}))$ denote the $i$th singular value of $\bchi_{\hat{\mathcal{S}}_u}(\Pi_u)$ for $i = 1 \ldots |\hat{\mathcal{S}}_u|$.
%
For simplicity, suppress dependence on $\bchi_{\hat{\mathcal{S}}_u}(\Pi_u)$, and denote  $s_{i}(\bchi_{\hat{\mathcal{S}}_u}({\Pi_u}))$ as $s_i$.
%

To proceed, observe that the matrix $(\mathbf{D}'_{\hat{\mathcal{S}}_u})^2$ is diagonal with elements $s^2_i/(s^2_i + 1/|\Pi_u|)^2$ for $i = 1 \ldots |\hat{\mathcal{S}}_u|$.
%
Next, noticing that $(a+b)^2 \geq ab$ for any $a,b \geq 0$, we have $s^2_i/(s^2_i + 1/|\Pi_u|)^2 \leq |\Pi_u|$ for all $i = 1 \ldots |\hat{\mathcal{S}}_u|$.
%
As a result, 
$s_{\max}\left((\mathbf{D}'_{\hat{\mathcal{S}}_u})^2\right) \leq |\Pi_u|$.
%
Substituting this inequality into \eqref{eq:ridge_SVD_2}, and simplifying further gives us, 
\begin{align}
    \lVert \hat{\balpha}^{SR}_{u} \rVert^2_2 & \leq |\Pi_u| \lVert \bY_{\Pi_u} \rVert^2_2 \nonumber \\ 
    & =|\Pi_u| \lVert \E[\bY_u^{(\Pi_u)}] + \bepsilon^{\Pi_u} \rVert^2_2 \nonumber \\
    & \leq 2|\Pi_u|\lVert \E[\bY_u^{(\Pi_u)}] \rVert^2_2 + 2|\Pi_u|\lVert \bepsilon^{\Pi_u} \rVert^2_2 \nonumber \\
    & \leq 2|\Pi_u|^2 + 2|\Pi_u| \lVert \bepsilon^{\Pi_u} \rVert^2_2 \label{eq:ridge_SVD_3},
\end{align}

\noindent where the last inequality uses Assumption \ref{ass:boundedness_potential_outcome}. 
%
Finally, it follows from Theorem 3.1.1 of \cite{vershynin_2018} that $\lVert \bepsilon^{\Pi_u} \rVert^2_2 = O_p (|\Pi_u|)$. 
%
Substituting this into \eqref{eq:ridge_SVD_3} completes the proof. 









%Further, define the diagonal matrix $\mathbf{D}'_{\hat{\mathcal{S}}_u} =  \left(\mathbf{D}^2_{\hat{\mathcal{S}}_u} + \frac{1}{|\Pi_u|} \mathbf{I}_{|\hat{\mathcal{S}}_u|}\right)^{-1}$.
%
%Using \eqref{eq:ridge_SVD_1}, and the notation defined above we have the following expression for $\lVert  \hat{\balpha}_{\mathcal{S}_u} \rVert^2_2$
%\begin{align*}
%    \lVert  \hat{\balpha}_{\mathcal{S}_u} \rVert^2_2 = \left( \right)
%\end{align*}
%Let $s_{i}(\bchi_{\hat{\mathcal{S}}_u}(\Pi_u))$ denote the $i$th singular value of $\bchi_{\hat{\mathcal{S}}_u}(\Pi_u)$. 
%Note that $\mathbf{D}'_{\hat{\mathcal{S}}_u}$ has elements $(s^2_i + 1/|\Pi_u|)^{-1}$ for $i = 1 \ldots |\hat{\mathcal{S}}_u|$.

\subsubsection{Proof of Lemma \ref{lem:fourier_coefficient_asymptotic_normality}}
The proof of this lemma follows immediately from adapting notation of Theorem 3 in \cite{liu2013asymptotic} to this paper. 
%
In particular, $Y = \bY_{\Pi_u}$, $X = \bchi(\Pi_u)$, $p = 2^p$, $\beta = \balpha_u$, $\Tilde{\beta}_{\text{Select + Ridge}} = \hat{\balpha}_u^{SR}$, $\beta_S = \balpha_{\mathcal{S}_u}$, $\Tilde{\beta}_{\text{Select + Ridge}, \mathcal{S}_u}= \hat{\balpha}^{SR}_{\mathcal{S}_u} $,  $\mu_n = 1/|\Pi_u|$, where the left hand side of each equality is the notation used in \cite{liu2013asymptotic}.
%




\subsection{Proof of Theorem \ref{thm:vertical_regression_normality}}



In this section, we provide the proof of Theorem \ref{thm:vertical_regression_normality}, i.e., establish asymptotic normality of the vertical regression step of \method. We also recall the following notation, let $\Delta^n_w = \hat{\bw}^n - \Tilde{\bw}^n \in \R^{|\mathcal{I}|}$, and $\Delta_{\mathcal{I}}^{\pi} = \hat{\E}[\bY_{\mathcal{I}}^{(\pi)}] - \E[\bY_{\mathcal{I}}^{(\pi)}] \in \R^{|\mathcal{I}|}$. 

\noindent Using Lemma \ref{lem:w_tilde_transfer_outcomes}, and the notation established above, we have
\begin{align}
     \hat{\E}[Y_n^{(\pi)}] - \E[Y_{n}^{(\pi)}] & = \langle \hat{\E}[\bY_{\mathcal{I}}^{(\pi)}], \hat{\bw}^n \rangle - \langle \E[\bY_{\mathcal{I}}^{(\pi)}], \Tilde{\bw}^n \rangle  \nonumber \\
     & = \langle \Delta_{\mathcal{I}}^{\pi}, \Tilde{\bw}^n \rangle  +   \langle \E[\bY_{\mathcal{I}}^{(\pi)}] ,\Delta^n_w   \rangle  +\langle \Delta^n_w, \Delta_{\mathcal{I}}^{\pi} \rangle. \nonumber
\end{align}
From Assumption \ref{ass:rowspace_inclusion}, it follows that $\E[\bY_{\mathcal{I}}^{(\pi)}] = \mathcal{P}_{V_{\mathcal{I}}^{(\Pi_n)}} \E[\bY_{\mathcal{I}}^{(\pi)}]  $ , where $\bV_{\mathcal{I}}^{(\Pi_n)}$ are the right singular vectors of $\E[\bY_{\mathcal{I}}^{(\Pi_n)}]$. Plugging this into the equation above gives us
\begin{equation}
\label{eq:three_term_asymptotic_normality}
    \hat{\E}[Y_n^{(\pi)}] - \E[Y_{n}^{(\pi)}] = \langle \Delta_{\mathcal{I}}^{\pi}, \Tilde{\bw}^n \rangle  + \langle   \E[\bY_{\mathcal{I}}^{(\pi)}] ,\mathcal{P}_{V_{\mathcal{I}}^{(\Pi_n)}}\Delta^n_w   \rangle  +\langle \Delta^n_w, \Delta_{\mathcal{I}}^{\pi} \rangle. 
\end{equation}
Next, we scale the left-hand side (LHS) of \eqref{eq:three_term_asymptotic_normality}  by $ \left( \sum_{u \in \mathcal{I}} \hspace{0.5mm} \Tilde{\sigma}^2_u \right)^{-1/2}$, and analyze term 1, and terms 2 \& 3 together  on the right-hand side (RHS) of \eqref{eq:three_term_asymptotic_normality} separately. 

\noindent \emph{Term 1.} Expanding the first term gives us
\begin{equation} \label{eq:vertical_asymptotic_normality_term1_intermediate1}
   \frac{1}{\left( \sum_{u \in \mathcal{I}} \hspace{0.5mm} \Tilde{\sigma}^2_u \right)^{1/2}}  \langle \Delta_{\mathcal{I}}^{\pi}, \Tilde{\bw}^n \rangle = \frac{1}{\left( \sum_{u \in \mathcal{I}} \hspace{0.5mm} \Tilde{\sigma}^2_u \right)^{1/2}}  \sum_{u \in \mathcal{I}}  \Tilde{w}^n_u \left(\hat{\E}[Y_u^{(\pi)}] - \E[Y_u^{(\pi)}] \right)
\end{equation}

\noindent From \eqref{eq:asymptotic_normality_condition_2}, as $|\Pi_u| \rightarrow \infty$, we have that 
\begin{equation*}
     \Tilde{w}^n_u \left(\hat{\E}[Y_u^{(\pi)}] - \E[Y_u^{(\pi)}] \right) \xrightarrow{d} N(0,\Tilde{\sigma}^2_u)
\end{equation*}

\noindent Since $\Tilde{w}^n_u \left(\hat{\E}[Y_u^{(\pi)}] - \E[Y_u^{(\pi)}] \right)$ is independent across $u$, as $M \rightarrow \infty$, the equation above implies
\begin{equation*}
    \sum_{u \in \mathcal{I}}
     \Tilde{w}^n_u \left(\hat{\E}[Y_u^{(\pi)}] - \E[Y_u^{(\pi)}] \right) \xrightarrow{d} N(0,\sum_{u \in \mathcal{I}}\Tilde{\sigma}^2_u)
\end{equation*}

\noindent Substituting the equation above into \eqref{eq:vertical_asymptotic_normality_term1_intermediate1} yields the following, 
\begin{equation} \label{eq:vertical_asymptotic_normality_term1}
   \frac{1}{\left( \sum_{u \in \mathcal{I}} \hspace{0.5mm} \Tilde{\sigma}^2_u \right)^{1/2}}  \sum_{u \in \mathcal{I}}  \Tilde{w}^n_u \left(\hat{\E}[Y_u^{(\pi)}] - \E[Y_u^{(\pi)}] \right) \xrightarrow{d} \mathcal{N}(0,1),
\end{equation}
as $M \rightarrow \infty$.

\noindent \emph{Terms 2 and 3.} We scale the right hand side of \eqref{eq:collecting_terms_simplified_log} by $ \left( \sum_{u \in \mathcal{I}} \hspace{0.5mm} \Tilde{\sigma}^2_u \right)^{-1/2}$ which yields the following,
\begin{equation} \label{eq:vertical_asymptotic_normality_term23_intermediate1}
      \frac{1}{\left( \sum_{u \in \mathcal{I}} \hspace{0.5mm} \Tilde{\sigma}^2_u \right)^{1/2}} \left(\langle \E[\bY_{\mathcal{I}}^{(\pi)}] ,\Delta^n_w   \rangle   +\langle \Delta^n_w, \Delta_{\mathcal{I}}^{\pi} \rangle \right) = O_p\left(\frac{\log^{3}(|\Pi_n||\mathcal{I}|)}{\left( \sum_{u \in \mathcal{I}} \hspace{0.5mm} \Tilde{\sigma}^2_u \right)^{1/2}}\left(r^2_n\sqrt{\frac{{s^2p}}{{M}}} +  \frac{r_n}{|\Pi_n|^{1/4}}\right) \right)
\end{equation}

\vspace{1mm}

\noindent To proceed, we state the following lemma 

\begin{lemma} \label{lem:scaled_variance_upper_bound}
Let the set-up of Theorem \ref{thm:vertical_regression_normality} hold. Then, we have
\begin{equation*}
  \left( \sum_{u \in \mathcal{I}}  \Tilde{\sigma}^2_u \right)^{-1/2} = O\left(\frac{1}{\lVert \Tilde{\bw}^n \rVert_2} \right) 
\end{equation*}
  
\end{lemma}

\noindent Substituting the result of Lemma \ref{lem:scaled_variance_upper_bound} into \eqref{eq:vertical_asymptotic_normality_term23_intermediate1}, and recalling \eqref{eq:asymptotic_normality_condition_1} yields
\begin{equation} \label{eq:asymptotic_normality_term23_op_bound}
     \frac{1}{\left( \sum_{u \in \mathcal{I}} \hspace{0.5mm} \Tilde{\sigma}^2_u \right)^{1/2}} \left(\langle \E[\bY_{\mathcal{I}}^{(\pi)}] ,\Delta^n_w   \rangle   +\langle \Delta^n_w, \Delta_{\mathcal{I}}^{\pi} \rangle \right) = o_p(1)
\end{equation}

\noindent Collecting \eqref{eq:vertical_asymptotic_normality_term1}, and \eqref{eq:asymptotic_normality_term23_op_bound} gives us the result. 


\subsection{Proof of Lemma \ref{lem:scaled_variance_upper_bound}}
\label{subsec:scale_variance_proof}
%
For a donor unit $u \in \mathcal{I}$, we have, 
\begin{align}
    \Tilde{\sigma}^{2}_u & = \frac{(\bchi^{\pi})^T \mathbf{K}^{-1}_u \bchi^{\pi} \left( \Tilde{w}^n_u 
    \right)^2 }{|\Pi_u|}  \nonumber \\
    & \geq \frac{\lambda_{\min}(\mathbf{K}^{-1}_u)(\bchi^{\pi})^T \bchi^{\pi} \left( \Tilde{w}^n_u 
    \right)^2}{|\Pi_u|} \nonumber \\
    & = \frac{\lambda_{\min}(\mathbf{K}^{-1}_u)2^p \left( \Tilde{w}^n_u 
    \right)^2}{|\Pi_u|} \nonumber \\
    & = \frac{2^p \left( \Tilde{w}^n_u 
    \right)^2 }{\lambda_{\max}(\mathbf{K}_u)|\Pi_u|} \nonumber \\
    & \geq \frac{\left( \Tilde{w}^n_u 
    \right)^2}{\lambda_{\max}(\mathbf{K}_u)} \label{eq:sigma_tilde_largest_eigenvalue},
\end{align}
where in the last line we use the fact that $|\Pi_u| \leq 2^p$. 
%
To proceed, we upper bound $\lambda_{\max}(\mathbf{K}_u)$.
%
To do so, we define some notation. 
%
Let $\mathbf{I}_m \in \mathbb{R}^{m \times m}$ denote the identity matrix of dimension $m$.
%
Additionally, recall the following fact: for a matrix $\mathbf{A} \in \mathbb{R}^{m \times m}$, we have that $\lVert \mathbf{A} \rVert_2 \leq m \lVert \mathbf{A} \rVert_{\infty}$.
%
Using this and Assumption \ref{ass:incoherence}, we have that
\begin{align*}
    \lVert \mathbf{K}_u - \mathbf{I}_{|\mathcal{S}_u|} \rVert_2 & = \lVert \frac{(\bchi_{\mathcal{S}_u}(\Pi_u))^T\bchi_{\mathcal{S}_u}(\Pi_u)}{|\Pi_u|} - \mathbf{I}_{|\mathcal{S}_u|} \rVert_2 \\ 
    & \leq s \lVert \frac{(\bchi_{\mathcal{S}_u}(\Pi_u))^T\bchi_{\mathcal{S}_u}(\Pi_u)}{|\Pi_u|} - \mathbf{I}_{|\mathcal{S}_u|} \rVert_\infty \\
    & \leq  s \lVert \frac{(\bchi(\Pi_u))^T\bchi(\Pi_u)}{|\Pi_u|} - \mathbf{I}_{2^p} \rVert_\infty \leq C
\end{align*}
%
\noindent for a positive constant $C > 0$.
%
Next, using the equation above and Weyl's inequality \cite{wainwright2019high}, we have that 
\begin{equation*}
    \lambda_{\max}(\mathbf{K}_u)  \leq \lVert \mathbf{K}_u - \mathbf{I}_{|\mathcal{S}_u|}  \rVert_2 +  \lambda_{\max}(\mathbf{I}_{|\mathcal{S}_u|}) \leq 1 + C
\end{equation*}


\noindent Substituting the result of Lemma \ref{lem:largest_eigenvalue_covariance_matrix} into \eqref{eq:sigma_tilde_largest_eigenvalue} gives us $\Tilde{\sigma}^{2}_u \geq \left( \Tilde{w}^n_u \right)^2/(1 + C)$. 
%    
Hence, we have $\left( \sum_{u \in \mathcal{I}} \hspace{0.5mm} \Tilde{\sigma}^2_u \right)^{-1/2} = O(1/\lVert \Tilde{\bw}^n \rVert_2)$, which is the claimed result. 
%









































































%To proceed, we state the following Lemma
%\begin{equation}
%\label{eq:horizontal_regression_term1_intermediate_1}
%      \sqrt{\frac{|\Pi_u|}{\sigma^2(\bchi^{\pi})^T\mathbf{K}^{-1}_u\bchi^{\pi}}}\left( \hat{\E}_{SR}[Y_u^{(\pi)} | \hat{\mathcal{S}}_u = \mathcal{S}_u] -  \E[Y_u^{(\pi)}] \right) \P(\hat{\mathcal{S}}_u = \mathcal{S}_u) =   \sqrt{\frac{|\Pi_u|}{\sigma^2(\bchi^{\pi})^T\mathbf{K}^{-1}_u\bchi^{\pi}}} \langle \hat{\balpha}^{SR}_{\mathcal{S}_u} - \balpha_{\mathcal{S}_u} , \bchi^{\pi} \rangle  \P(\hat{\mathcal{S}}_u = \mathcal{S}_u)
%\end{equation}




%\begin{align}
%  \sqrt{\frac{|\Pi_u|}{\sigma^2(\bchi^{\pi})^T\mathbf{K}^{-1}_u\bchi^{\pi}}} \hat{\E}_{SR}[Y_u^{(\pi)} | \hat{\mathcal{S}}_u \neq \mathcal{S}_u]  \P(\hat{\mathcal{S}}_u \neq \mathcal{S}_u) & \leq \sqrt{|\Pi_u|} \langle \hat{\balpha}^{SR}_u,\bchi^{\pi} \rangle \P(\hat{\mathcal{S}}_u \neq \mathcal{S}_u) \nonumber \\ 
%    & \leq \left(\sqrt{|\Pi_u|} \lVert \hat{\balpha}^{SR}_u \rVert_2 \lVert \bchi^{\pi} \rVert_2 \right)\P(\hat{\mathcal{S}}_u \neq \mathcal{S}_u) \nonumber \\
%    & \leq \left(\lVert  \hat{\balpha}^{SR}_u \rVert_2 \sqrt{|\Pi_u| 2^p} \right) \P(\hat{\mathcal{S}}_u \neq \mathcal{S}_u) \label{eq:horizontal_regression_term2_intermediate_1}
%\end{align}


%\noindent where the first inequality follows from Cacuhy-Schwarz, and the second follows from the fact that $\bchi^{\pi} \in \{-1,1\}^{2^p}$. Next, using condition (d) of Proposition \ref{prop:horizontal_asymptotic_normality}, we have $2^p = O(2^{|\Pi_u|^{c_3}})$. 
%
%Plugging in  $2^p = O(2^{|\Pi_u|^{c_3}})$, and condition (e) into \eqref{eq:horizontal_regression_term2_intermediate_1}, gives us
%\begin{equation}
%     \hat{\E}_{SR}[Y_u^{(\pi)} | \hat{\mathcal{S}}_u \neq \mathcal{S}_u] = \lVert \hat{\balpha}^{SR}_u \rVert_2 o(\sqrt{|\Pi_u|} 2^{\frac{|\Pi_u|^{c_3}}{2}} e^{-|\Pi_u|^{c_3}})
%\end{equation}

%\begin{equation*}
%    \sqrt{|\Pi_u|} \left( \hat{\E}_{SR}[Y_u^{(\pi)}] - \E[Y_u^{(\pi)}] \right) \xrightarrow{d} N\left(0,(\bchi^{\pi})^T\mathbf{K}^{-1}_u\bchi^{\pi}\right), %\end{equation*}


%\begin{lemma} 
%\label{lem:op_sequence_bound}
%For a strictly positive sequence $a_n$, and positive constants  $c, %\hspace{0.25mm} C$, we have that 
%\begin{equation*}
%    o_p(a^{c}_n  e^{-a_n^{C}}) = o_p(1)
%\end{equation*}
%as $a_n \rightarrow \infty$.
%\end{lemma}



%\begin{equation} \label{eq:term1_asymptotic_normality}
%    \sum_{u \in \mathcal{I}} \sqrt{\frac{|\Pi_u|}{\Tilde{\sigma}^2_u|\mathcal{I}|}} \langle \Delta_{\mathcal{I}}^{\pi}, \Tilde{\bw}^n \rangle =  \sum_{u \in \mathcal{I}}   \sqrt{\frac{|\Pi_u|}{\Tilde{\sigma}^2_u|\mathcal{I}|}} \left( \right)
%    = \frac{\langle \hat{\E}[\bY_{\mathcal{I}}^{(\pi)}] - \E[\bY_{\mathcal{I}}^{(\pi)}] , \Tilde{\bw}^n \rangle }{\lVert \Tilde{\bw}^n \rVert_2}
%\end{equation}

%
%\noindent For the second and third term, we scale the RHS of \eqref{eq:collecting_terms_simplified_log} by $1/\lVert \Tilde{\bw}^n \rVert_2$, which gives us
%\begin{equation*}
%     \frac{1}{\lVert \Tilde{\bw}^n \rVert_2}\left(\langle \E[\bY_{\mathcal{I}}^{(\pi)}] ,\Delta^n_w   \rangle   +\langle \Delta^n_w, \Delta_{\mathcal{I}}^{\pi} \rangle \right) =  O_p\left(\log^{3}(|\Pi_n||\mathcal{I}|) r^2_n\sqrt{\frac{{s^2p}}{{M}}} +  \frac{r_n}{|\Pi_n|^{1/4}} \right).
%\end{equation*}

%\noindent To proceed, we substitute the result of 
%Then, substituting \eqref{eq:asymptotic_normality_condition_1} into the RHS of the equation above, gives us 
%\begin{equation}\label{eq:asymptotic_normality_term2_3_op_bound}

\section{Numerical Experiments}
\label{sec:sims}
In this section, we corroborate our theoretical findings with numerical simulations in both the observational setting and under our experimental design mechanism described in Section \ref{sec:experimental_design}. 
%
We compare \method~to the Lasso (i.e., running a separate horizontal regression  for every unit) and benchmark matrix completion algorithms. 
%
Specifically, we compare \method~to the following matrix completion algorithms: (i) \texttt{SoftImpute} \citep{mazumder2010spectral}, (ii) \texttt{IterativeSVD} \citep{troyanskaya2001missing} . 
%
We note that we also tried running our simulation experiments for nuclear-norm minimization method introduced in \citep{candes2012exact}, but the method was unable to be run within a reasonable time frame (6 hours) for any of our simulation settings. 
%
Code for implementing \method~and reproducing our experiments can be found at \url{https://github.com/aagarwal1996/synth_combo}. 

\subsection{Observational Setting}
\label{subsec:observational_sims}
For this numerical experiment, we simulate an observation pattern commonly found in applications such as recommendation engines, where most users tend to provide ratings for combinations of goods they either particularly liked or disliked, i.e., there is confounding.
%
The precise experimental set-up for this setting is as follows.  

\vspace{2mm}
\noindent \textbf{Experimental Set-up.} We consider $N = 100$ units, and vary the number of interventions $p \in \{10,11, \ldots 15\}$. 
%
We choose $r = 3$, and $s = rp^{3/2}$.
%
To proceed, we first describe how we generate the potential outcomes, followed by how we generate the observation pattern. 

\vspace{2mm}
\noindent
\emph{Generating Potential Outcomes.} For every $i \in [r],$ we generate $\balpha_i$ by first sampling $p^{3/2}$ non-zero coefficients at random.  
%
Then, we sample every non-zero coefficient of $\balpha_i$ from a standard normal distribution. 
%
Denote $\mathcal{A}_r = [\balpha_i : i \in [r]] \in \mathbb{R}^{r \times 2^p}$.  
%
Next, we construct $\mathcal{A}_{N - r}= \mathbf{B}\mathcal{A}_r \in \mathbb{R}^{(N - r) \times 2^p}$, where the entries in $\mathbf{B} \in \mathbb{R}^{(N - r) \times r}$ are sampled i.i.d from a standard Dirichlet distribution. 
%
This ensures that the Fourier coefficients of $\mathcal{A}_{N-r}$ lie in the span of the Fourier coefficients in $\mathcal{A}_r$.
%
We then define $\mathcal{A} = [\mathcal{A}_r, \mathcal{A}_{N-r}]$. 
%
By construction, $\text{rank}(\mathcal{A}) = r$, and $\lVert \balpha_n \rVert_0 \leq  rp^{3/2} = s$ for every unit. 
%
The expected potential outcomes are simply constructed by computing $\bchi(\Pi) \mathcal{A}^T$.
%
Finally, we generate noisy potential outcomes by adding normally distributed noise $\epsilon \sim N(0,\sigma^2)$ to the potential outcome $\E[Y_n^{(\pi)}]$ for every unit $n$ and combination $\pi$. 
%
We choose $\sigma^2$ such that the percentage of variance explained (PVE), defined by the formula $\text{PVE} = \text{Var}(\E[Y_N^{(\Pi)}])/(\sigma^2 + \text{Var}(\E[Y_N^{(\Pi)}]))$ is equal to $0.9$, and where $\text{Var} (\cdot)$ denotes the variance. 
%
PVE is proportional to signal to noise ratio, and can be thought of as the fraction of variance that is explained by the underlying potential outcome matrix. 

\vspace{2mm}
\noindent
\emph{Observation Pattern.} Next, we discuss how we generate the observation pattern. 
%
Let $p_{n,\pi}$ denote the probability that the expected potential outcome for unit $n$ and combination $\pi$ is revealed. 
%
In this simulation, we define $p_{n,\pi} = |\E[Y_n^{(\pi)}]|/\sum_{\pi \in \Pi} |\E[Y_n^{(\pi)}]|$.
%
Our definition of $p_{n,\pi}$ induces the missingness pattern where we are more likely to see outcomes with larger absolute values. 
%
In the context of recommendation engines, this can be interpreted as saying that we are only likely to observe the ratings for combinations that users either strongly like or dislike.
%
Next, we choose $|\mathcal{I}| = 2r$ donor units. 
%
For a donor unit $u$, we generate $2p^{5/2}$ observations where a combination $\pi$ is picked with probability $p_{u,\pi}$.
%
We do the same for non-donor units but generate only $2r^4$ observations.
%

\vspace{2mm}
\noindent
\emph{Hyper-parameter Choices for Estimation Methods}. For the Lasso, we tune the regularization parameter $\lambda$ via $5$-fold cross-validation (CV). 
%
For PCR (i.e., step 2 of \method), we tune $\kappa$, the number of singular values retained, via 5-fold CV as well. 
%
The matrix completion techniques that we compare to require that the rank $r$ of the underlying matrix to be recovered is provided as a hyper-parameter.
%
In this simulation, we provide the true rank $r = 3$ as the hyper-parameter to these matrix completion algorithms. 
\vspace{2mm}


\noindent \textbf{Results.} We measure the mean squared error (MSE) averaged over 5 repetitions between the estimated potential outcome matrix and the true potential outcome matrix for each method. 
%
The MSE for each method as we vary the number of interventions is visualized in Figure \ref{fig:sim_results} (a).  
%
The results show that \method~outperforms all other methods as the number of interventions grow. 
%
Further, the gap in performance between \method~and the Lasso enforce the utility of using PCR for non-donor units that do not have sufficient measurements.
%
We also note that the sample sizes chosen here for the donor set are less than that required by our theoretical results. Nonetheless, \method~performs well, suggesting that our theoretical results may not be entirely tight.  

\subsection{Experimental Design Simulations}
\label{subsec:experimental_design_sims}
We begin by describing the simulation set-up for the observation pattern induced by our experimental design mechanism.

\vspace{2mm}

\noindent \textbf{Experimental Set-up.} We generate potential outcomes, and noisy potential outcomes as we do in the observational setting. 
%
We generate the observation pattern as described by the experimental design mechanism in Section \ref{sec:experimental_design}.
%
Finally, hyper-parameters of all estimation methods are chosen as described in the experimental set-up for the simulations in the observational setting. 

\vspace{2mm}

\noindent \textbf{Results.} We plot the MSE (averaged over 5 repetitions) for the different methods as we increase the number of interventions $p \in \{10,11,\ldots 15\}$ in Figure \ref{fig:sim_results} (b). 
%
We see that \method~significantly outperforms all methods for the observation pattern induced by our design mechanism. 
%
The performance of \method~corroborates our theoretical findings that this experimental design mechanism is able to utilize the strengths of \method~effectively. 
%


\vspace{4mm}


\begin{figure}[htbp]
    \centering
    \begin{minipage}{0.5\textwidth}
        \centering
        \includegraphics[width=1\textwidth]{simulation_figures/observational_design_results_0.9.png}  
        \caption*{(a) Observational setting simulations.}
    \end{minipage}\hfill
    % Remove or comment out this line
    \begin{minipage}{0.5\textwidth}
        \centering
        \includegraphics[width=1\textwidth]{simulation_figures/experimental_design_results_0.9.png} % second figure itself
        \caption*{(b) Experimental design simulations.}
    \end{minipage}
    \caption{MSE for different methods as we vary the number of interventions in (a) the observational setting and (b) under our proposed experimental design mechanism. \method~outperforms all other methods in both settings. }
    \label{fig:sim_results}
\end{figure}

\section{Real World Case Study}
\label{sec:real_world_case_study}
This section details a real-world data experiment on recommendation systems for sets of movie ratings, comparing \method~to the algorithms listed above.
%
Further, we empirically validate that our key modeling assumptions (i.e., low-rank condition of $\mathcal{A}$ and sparsity of donor unit Fourier coefficients) hold in this dataset. 
%
For all methods, hyper-parameters are chosen via $5-$fold CV.
%
Additionally, the donor set $\mathcal{I}$ is chosen via the procedure described in Section \ref{sec:estimator_descripton}.

\noindent \textbf{Data and Experimental Set-up.} We use data collected in \cite{sharma2019sets} which consists of user ratings of sets of movies. 
%
Specifically, users were asked to provide a rating of $1$-$5$ on a set of $5$ movies chosen at random. 
%
This resulted in a total of ratings from $854$ users over $29,516$ sets containing $12,549$ movies.
%
$80\%$ of each user's ratings are chosen as the training set, and the other $20\%$ as the test set.
%



\noindent \textbf{Results.} We measure the RMSE averaged over 5 repetitions for all methods, and display the results in the Table below. 
with  \method~outperforming all other approaches.
%
The performance gap between \method~and other methods demonstrates the benefit of only performing the horizontal regression on the units with sufficient observations (i.e., the donor units), and using PCR for the non-donor units that have an insufficient number of measurements.

\begin{table} [H]
\centering
\footnotesize
\begin{tabular}{lrrrr}
\toprule
Method   & \textbf{\method} & \texttt{SoftImpute} & \texttt{IterativeSVD} & Lasso \\
\midrule
RMSE & \textbf{0.30} $\pm$ 0.03 &  0.38 $\pm$ 0.02  &  0.38 $\pm$ 0.02 & 0.43 $\pm$ 0.05 \\
\bottomrule
\end{tabular}
\label{tab:ratings_results}
\caption{\method~outperforms other approaches for the real-world experiment on sets of ratings for movies.}
\end{table}

\noindent \textbf{Key Assumptions of \method~hold.} We also verify that our two key modeling assumptions (i.e., low-rankness and sparsity of $\mathcal{A}$) hold in this dataset. 
%
For the low-rank condition,the singular value spectrum of movies rated by all users is plotted on a log-scale in Figure \ref{fig:singular_spectrum}. 
%
The plot shows the outcomes (and hence the Fourier coefficients) are low-rank. 
%
To investigate sparsity, we examine $\hat{\balpha}_u$ for the donor set. 
%
The MSE averaged across all donor units on the test set was 0.22, indicating that the estimated Fourier coefficient is an accurate representation of the true underlying Fourier coefficient. 
%
Further, the estimated donor unit Fourier coefficients are indeed sparse, and on average have 8.7\% non-zero coefficients. 


\begin{figure}[htbp]
    \centering
    \includegraphics [width = 0.6\textwidth]{figures/singular_value_spectrum.png}
    \caption{Singular value spectrum of user ratings on sets of movies. Inspecting the singular spectrum shows that the matrix of observed ratings is low-rank.}
    \label{fig:singular_spectrum}
\end{figure}

%the gap between \method~and the Lasso shows the benefit of first estimating the outcomes of the donor set, and then transferring these estimated outcomes to non-donor units which have a number of insufficient measurements.

%that highlight the empirical effectiveness of our approach. 
%The data is publicly available at \url{https://grouplens.org/datasets/learning-from-sets-of-items-2019/}, and more details about the data collection process can be found in \cite{sharma2019sets}. 
%


%\noindent \textbf{Comparison Methods.} As in the numerical simulations, we compare \method~to matrix completion algorithms: \texttt{SoftImpute} \cite{mazumder2010spectral}, and \texttt{IterativeSVD} \cite{troyanskaya2001missing}. 
%
%These methods require that the rank of the underlying matrix be provided as a hyper-parameter. This was chosen via $5$-fold cross-validation (CV). 
%
%We also compare \method~to the Lasso, where we tune the regularization parameter $\lambda$ via $5$-fold CV. 
%
%For \method, we tune all hyper-parameters via $5$-fold CV. 
%
%Additionally, we choose the donor set via the approached outlined in the manuscript.
%two of the key assumptions, the low-rank condition on the matrix of outcomes and sparsity of donor unit Fourier coefficients, hold in this real-world dataset. 
% (using a log-scale for the magnitude of the spectrum) \textbf


%For the sparsity condition, we investigate the Lasso model that was learnt for the donor units. 
%
\section{Conclusion}\label{sec:conclusion}
In this work, we focus on addressing the fundamental challenge of OOD detection tasks, which is how to fully understand the semantic discrepancy between the ID/OOD samples. We reveal that the key to success in the realistic SCOOD task is to allocate as many ID samples in the unlabeled set correctly as possible. To this end, we propose a novel uncertainty-aware optimal transport scheme that introduces class-specific energy scores as guidance for effective label assignment. Experimental results show that our method achieves better performance than previous state-of-the-art methods on SCOOD benchmarks.

\textbf{Limitations.} In addition to temperature scaling, other techniques such as feature clipping applied in ReAct~\cite{sun2021react} also enhance the performance of energy score, so how to obtain an OOD score that best fits the SCOOD task can be further explored. Moreover, a setting highly related to SCOOD has been proposed in \cite{katz2022training} and formulated as a constrained optimization problem. We will also theoretically analyze these practical OOD settings in our feature work.

% \section*{Acknowledgments}
\textbf{Acknowledgments.} 
This work is supported by National Key R\&D Program of China under Grant 2020AAA0105701, National Natural Science Foundation of China (NSFC) under Grants 61872327, Major Special Science and Technology Project of Anhui, National Natural Science Foundation of China (62033012) and Ant Group through Ant Research Intern Program.

\if1\blind{
    \section{Acknowledgements}
\label{sec:acknowledgements}
We thank Alberto Abadie, Peng Ding, Giles Hooker, Devavrat Shah, Vasilis Syrgkanis, and Bin Yu for
useful discussions and feedback.
We also thank Austin Serif for his help in implementing \method. 
} \fi


\bibliography{rf.bib}

\newpage

\newpage
\bibliographystyleappendix{agsm}
%\bibliographyappendix{rf.bib}

\section{Appendix for Proofs}

\paragraph{Proof of Theorem \ref{thm:main}.}

\begin{proof}
\label{proof:main}
Our proof has two steps. In Step 1, we will show that SimCLR is equivalent to minimizing the cross entropy loss defined in Eqn.~(\ref{eqn:cross-entropy}). 
In Step 2, we will show  that minimizing the cross-entropy loss 
is equivalent to spectral clustering on $\bfpi$. 
Combining the two steps together, we have proved our theorem. 

\textbf{Step 1: } SimCLR is equivalent to minimizing the cross entropy loss.

The cross-entropy loss takes expectation over 
$\bfW_\bfX\sim \mathbb{P}(\cdot ; \bfpi)$, 
which means $\bfW_\bfX$ has exactly one non-zero entry in each row $i$. By Lemma~\ref{lem:multinomial}, we know every row $i$ of $\bfW_\bfX$ is independent of other rows. Moreover, 
$\bfW_{\bfX,i}\sim \mathcal{M}(1, \bfpi_i/\sum_j \bfpi_{i,j})=\mathcal{M}(1, \bfpi_i)$, because $\bfpi_i$ itself is a probability distribution.
Similarly, we know $\bfW_\bfZ$ also has the row-independent property by sampling over $\mathbb{P}(\cdot;\bfK_\bfZ)$.
Therefore, by Lemma~\ref{lem:cross_split}, we know Eqn.~(\ref{eqn:cross-entropy}) is equivalent to:
\[
 -\sum_{i=1}^n \mathbb{E}_{\bfW_{\bfX,i}}[\log \mathbb{P}(\bfW_{\bfZ,i}=\bfW_{\bfX,i};\bfK_\bfZ)],
\]

This expression takes expectation over $\bfW_{\bfX,i}$ for the given row $i$. Notice that 
$\bfW_{\bfX,i}$ has exactly one non-zero entry, which equals $1$ (same for $\bfW_{\bfZ,i}$). 
As a result
we expand the above expression to be:
\begin{equation}
 -\sum_{i=1}^n \sum_{j\neq i} \Pr(\bfW_{\bfX,i,j}=1)\log \Pr(\bfW_{\bfZ,i,j}=1).
\label{eqn:detailed-expansion}    
\end{equation}


By Lemma~\ref{lem:multinomial}, $\Pr(\bfW_{\bfZ,i,j}=1)=\bfK_{\bfZ,i,j}/\|\bfK_{\bfZ,i}\|_1$ for $j\neq i$. Recall that $\bfK_\bfZ=(k(\bfZ_i-\bfZ_j))_{(i,j)\in[n]^2}$, which means 
$\bfK_{\bfZ,i,j}/\|\bfK_{\bfZ,i}\|_1=\frac{\exp(-\|\bfZ_i-\bfZ_j\|^2/{2\tau})}{\sum_{k\neq i}
\exp(-\|\bfZ_i-\bfZ_k\|^2/{2\tau})
}$ for $j\neq i$, when $k$ is the Gaussian kernel with variance $\tau$. 

Notice that $\bfZ_i=f(\bfX_i)$, so we know
\begin{equation}
-\log \Pr(\bfW_{\bfZ,i,j}=1)=
-\log \frac{\exp(-\|f(\bfX_i)-f(\bfX_j)\|^2/{2\tau})}{\sum_{k\neq i}
\exp(-\|f(\bfX_i)-f(\bfX_k)\|^2/{2\tau}),
}
\label{eqn:infonce-equivalence}    
\end{equation}


The right hand side is exactly the InfoNCE loss defined in Eqn.~(\ref{eqn:infonce}).
Inserting Eqn.~(\ref{eqn:infonce-equivalence}) into Eqn.~(\ref{eqn:detailed-expansion}), we get the SimCLR algorithm, which first samples augmentation pairs $(i,j)$ with $\Pr(\bfW_{\bfX,i,j}=1)$ for each row $i$, and then optimize the InfoNCE loss. 

\textbf{Step 2: } minimizing the cross entropy loss 
is equivalent to spectral clustering on $\bfpi$.


By Lemma~\ref{lem:convert_to_spectral}, we may further convert the loss to 
\begin{equation}
\label{eqn:main-theorem-repul-attr}
\min_{\bfZ}
-\sum_{(i,j)\in [n]^2} \mathbf{P}_{i,j}
\log k (\bfZ_i-\bfZ_j)+\log \mathbf{R}(\bfZ).
\end{equation}
Since $k$ is the Gaussian kernel, this reduces to \[
\min_\bfZ \mathrm{tr}(\bfZ^\top \mathbf{L}(\bfpi) \bfZ)
+\log \mathbf{R}(\bfZ),
\]

where we use the fact that $\mathbb{E}_{\bfW_\bfX\sim \mathbb{P}(\cdot; \bfpi)}[\mathbf{L}(\bfW_\bfX)]
=\mathbf{L}(\bfpi)
$, because the Laplacian operator is linear and $
\mathbb{E}_{\bfW_\bfX\sim \mathbb{P}(\cdot; \bfpi)}(\bfW_\bfX)=\bfpi
$.
\end{proof}

\paragraph{Proof of Theorem \ref{thm:clip}.}
\begin{proof}
Since $\bfW_\bfX\sim \mathbb{P}(\cdot;\bfpi_{\mathbf{A}, \mathbf{B}})$, we know 
$\bfW_\bfX$ has exactly one non-zero entry in each row, denoting the pair that got sampled. 
A notable difference compared to the previous proof is we now have $n_\mathcal{A}+n_\mathcal{B}$ objects in our graph. CLIP deals with this by taking a mini-batch of size $2N$, 
such that $n_\mathcal{A}=n_\mathcal{B}=N$, and adding the $2N$ InfoNCE losses together. We label the objects in $\mathcal{A}$ as $[n_\mathcal{A}]$, and the objects in $\mathcal{B}$ as $\{n_\mathcal{A}+1, \cdots, n_\mathcal{A}+n_\mathcal{B}\}$. 

Notice that $\bfpi_{\mathbf{A}, \mathbf{B}}$ is a bipartite graph, so the edges of objects in $\mathcal{A}$ will only connect to object in $\mathcal{B}$ and vice versa. We can define the similarity matrix in $\cZ$ as $\bfK_\bfZ$, 
where $\bfK_\bfZ(i, j+n_\mathcal{A})=\bfK_\bfZ(j+n_\mathcal{A},i)= k(\bfZ_i-\bfZ_j)$ for $i\in [n_\mathcal{A}], j\in [n_\mathcal{B}]$, and otherwise we set $\bfK_\bfZ(i,j)=0$. 
The rest is same as the previous proof. 
\end{proof}

\paragraph{Proof of Theorem \ref{thm:exponential}.}

\begin{proof}
\label{proof:exponential}
Since the objective function consists of a linear term combined with an entropy regularization, which is a strongly concave function, the maximization problem is a convex optimization problem. Owing to the implicit constraints provided by the entropy function, the problem is equivalent to having only the equality constraint. We then introduce the Lagrangian multiplier $\lambda$ and obtain the following relaxed problem:

$$
\widetilde{E}(\boldsymbol{\alpha})=\psi_{1}-\sum_{i=1}^n \alpha_{i} \psi_{i}+\tau \sum_{i=1}^n \alpha_{i}\log \alpha_{i}+\lambda\left(\boldsymbol{\alpha}^{\top} \mathbf{1}_n-1\right).
$$

As the relaxed problem is unconstrained, taking the derivative with respect to $\alpha_{i}$ yields

$$
\frac{\partial \widetilde{E}(\boldsymbol{\alpha})}{\partial \alpha_{i}}=-\psi_{i}+\tau\left(\log \alpha_{i}+\alpha_{i} \frac{1}{\alpha_{i}}\right)+\lambda=0.
$$

Solving the above equation implies that $\alpha_{i}$ takes the form
$
\alpha_{i}=\exp \left(\frac{1}{\tau} \psi_{i}\right) \exp \left(\frac{-\lambda}{\tau}-1\right).
$ Since $\alpha_{i}$ lies on the probability simplex, the optimal $\alpha_{i}$ is explicitly given by
$
\alpha^{*}_{i}=\frac{\exp \left(\frac{1}{\tau} \psi_{i}\right)}{\sum_{i^{\prime}=1}^n \exp \left(\frac{1}{\tau} \psi_{i^{\prime}}\right)} .
$ Substituting the optimal point into the objective function, we obtain
$$
\begin{aligned}
E\left(\boldsymbol{\alpha}^*\right)  &=\psi_1-\sum_{i=1}^n \frac{\exp \left(\frac{1}{\tau} \psi_{i}\right)}{\sum_{i^{\prime}=1}^n \exp \left(\frac{1}{\tau} \psi_{i^{\prime}}\right)} \psi_{i}+\tau \sum_{i=1}^n \frac{\exp \left(\frac{1}{\tau} \psi_{i}\right)}{\sum_{i^{\prime}=1}^n \exp \left(\frac{1}{\tau} \psi_{i^{\prime}}\right)}\log \frac{\exp \left(\frac{1}{\tau} \psi_{i}\right)}{\sum_{i^{\prime}=1}^n \exp \left(\frac{1}{\tau} \psi_{i^{\prime}}\right)} \\
& =\psi_1 - \tau \log \left(\sum_{i=1}^n \exp \left(\frac{1}{\tau} \psi_{i}\right)\right).
\end{aligned}
$$
Thus, the Lagrangian dual function is given by
\begin{equation*}
-E\left(\boldsymbol{\alpha}^*\right)= -\tau \log \frac{\exp \left(\frac{1}{\tau} \psi_{1}\right)}{\sum_{i=1}^n \exp \left(\frac{1}{\tau} \psi_{i}\right)}.\qedhere
\end{equation*}
\end{proof}



\section{More on Experiments} \label{section: experiment_details}

\paragraph{CIFAR-10 and CIFAR-100} CIFAR-10 ~\citep{krizhevsky2009learning} and CIFAR-100 ~\citep{krizhevsky2009learning} are well-known classic image classification datasets. Both CIFAR-10 and CIFAR-100 contain a total of 60k $32 \times 32$ labeled images of different classes, with 50k for training and 10k for testing. CIFAR-10 is similar to CIFAR-100, except there are 10 different classes in CIFAR-10 and 100 classes in CIFAR-100.

\paragraph{TinyImageNet} TinyImageNet ~\citep{le2015tiny} is a subset of ImageNet ~\citep{deng2009imagenet}. There are 200 different object classes in TinyImageNet, with 500 training images, 50 validation images, and 50 test images for each class. All the images in TinyImageNet are colored and labeled with a size of $64 \times 64$.

\textbf{Pseudo-code.} Algorithm \ref{alg:Training Procedure} presents the pseudo-code for our empirical training procedure.

\begin{algorithm}[!htbp]
\caption{Training Procedure}
\label{alg:Training Procedure}
\begin{algorithmic}[1]
\REQUIRE trainable encoder network $f$, batch size $N$, augmentation strategy \textit{aug}, loss function $L$ with hyperparameters \textit{args}
\FOR {sampled minibatch ${x_i}_{i=1}^N$}
\FORALL{$i \in { 1, ..., N }$}
\STATE draw two augmentations $t_i = \textit{aug}\left(x_i\right) $, $t_i' = \textit{aug}\left(x_i\right) $
\STATE $z_i = f\left(t_i\right)$, $z_i' = f\left(t_i'\right)$
\ENDFOR
\STATE compute loss $\mathcal{L} = L(N, z, z', \textit{args})$
\STATE update encoder network $f$ to minimize $\mathcal{L}$
\ENDFOR
\STATE \textbf{Return} encoder network $f$
\end{algorithmic}
\end{algorithm}

We also provide the pseudo-code for our core loss function used in the training procedure in Algorithm \ref{alg:Core loss}. The pseudo-code is almost identical to SimCLR's loss function, with the exception of an extra parameter $\gamma$.

\begin{algorithm}[!htbp]
\caption{Core loss function $\mathcal{C}$}
\label{alg:Core loss}
\begin{algorithmic}[1]
\REQUIRE batch size $N$, two encoded minibatches $z_1, z_2$, $\gamma$, temperature $\tau$
\STATE $z = \textit{concat}\left(z_1, z_2\right)$
\FOR {$i \in {1, ..., 2N }, j \in {1, ..., 2N}$ }
\STATE $s_{i,j} = \Vert z_i - z_j \Vert_2^{\gamma}$
\ENDFOR
\STATE \textbf{define} $l(i, j)$ \textbf{as} $l(i, j) = - \log \frac{exp\left(s_{i,j}/\tau \right)}{\sum_{k=1}^{2N} \mathbf{1}{[k \ne i]} exp\left(s{i, j} / \tau \right)} $
\STATE \textbf{Return} $\frac{1}{2N} \sum_{k=1}^N\left[l(i, i+N) + l(i+N, i)\right]$
\end{algorithmic}
\end{algorithm}

Utilizing the core loss function $\mathcal{C}$, we can define all kernel loss functions used in our experiments in Table \ref{table: loss definition}. For all $z_i \in z$ with even dimensions $n$, we define $z_{L_i} = z_i\left[0:n/2\right]$ and $z_{R_i} = z_i\left[n/2:n\right]$.

\begin{table}[ht]
\centering
\begin{tabular}{{@{}l|l@{}}}
Kernel  &  Loss function \\ \midrule
Laplacian & $\mathcal{C}\left(N, z, z', \gamma=1, \tau\right)$\\ \midrule
Sum       & $\lambda * \mathcal{C}\left(N, z, z', \gamma=1, \tau_1\right) + (1-\lambda) * \mathcal{C}\left(N, z, z', \gamma=2, \tau_2\right)$  \\ \midrule
Concatenation Sum&$\lambda * \mathcal{C}\left(N, z_L, z'_L, \gamma=1, \tau_1\right) + (1-\lambda) * \mathcal{C}\left(N, z_R, z'_R, \gamma=2, \tau_2\right)$\\ \midrule
$\gamma = 0.5$ & $\mathcal{C}\left(N, z, z', \gamma=0.5, \tau\right)$          \\ 

\end{tabular}

\caption{Definition of kernel loss functions in our experiments}
\label {table: loss definition}
\end{table}

\textbf{Baselines.} We reproduce the SimCLR algorithm using PyTorch Lightning~\citep{PytorchLightning}.

\textbf{Encoder details.}
The encoder $f$ consists of a backbone network and a projection network. We employ ResNet50~\citep{ResNet} as the backbone and a 2-layer MLP (connected by a batch normalization~\citep{ioffe2015batch} layer and a ReLU \cite{nair2010rectified} layer) with hidden dimensions 2048 and output dimensions 128 (or 256 in the concatenation kernel case).

\textbf{Encoder hyperparameter tuning.}
For each encoder training case, we randomly sample 500 hyperparameter groups (sample details are shown in Table \ref{table: Hyperparameter sample}) and train these samples simultaneously using Ray Tune ~\citep{RayTune}, with the ASHA scheduler~\citep{li2018massively}. Ultimately, the hyperparameter group that maximizes the online validation accuracy (integrated in PyTorch Lightning) within 5000 validation steps is chosen for the given encoder training case.

\begin{table}[ht]
\centering

\begin{tabular}{@{}l|l|l@{}}
\midrule
Hyperparameter  & Sample Range & Sample Strategy \\ \midrule
start learning rate & $\left[10^{-2}, 10\right]$ & log uniform \\ \midrule
$\lambda$       & $\left[0, 1\right]$ & uniform \\ \midrule
$\tau$, $\tau_1$, $\tau_2$ & $\left[0, 1\right]$ & log uniform \\ \midrule
\end{tabular}

\caption{Hyperparameters sample strategy}
\label {table: Hyperparameter sample}
\end{table}

\textbf{Encoder training.} 
We train each encoder using the LARS optimizer~\citep{LARSOptimizer}, LambdaLR Scheduler in PyTorch, momentum 0.9, weight decay $10^{-6}$, batch size 256, and the aforementioned hyperparameters for 400 epochs on a single A-100 GPU.

\textbf{Image transformation.} The image transformation strategy, including augmentation, is identical to the default transformation strategy provided by PyTorch Lightning.

\textbf{Linear evaluation.}
The linear head is trained using the SGD optimizer with a cosine learning rate scheduler, batch size 64, and weight decay $10^{-6}$ for 100 epochs. The learning rate starts at $0.3$ and ends at $0$.

\textbf{Moco Experiments.} We also tested our method based on MoCo~\citep{he2019moco}. The results are summarized in Table \ref{tab:results-moco}. Here we choose ResNet18~\citep{ResNet} as the backbone and set a temperature of $0.1$ as default. For our simple sum kernel, we set $\lambda=0.8$. The results show that our method outperforms the original MoCo method.

\begin{table}[thb]
\centering
\caption{MoCo Experiment Results on CIFAR-10 and CIFAR-100.}
\label{tab:results-moco}
\resizebox{\textwidth}{!}{%
\begin{tabular}{@{}c|ccc|ccc@{}}
\toprule
\multirow{3}{*}{Method} & \multicolumn{3}{c|}{CIFAR-10} & \multicolumn{3}{c}{CIFAR-100} \\ \cmidrule(lr){2-4} \cmidrule(lr){5-7} 
                        & 200 epochs & 400 epochs    & 1000 epochs   & 200 epochs & 400 epochs & 1000 epochs         \\ \midrule
MoCo (repro.)         & $76.41 \pm 0.12$    & $80.01 \pm 0.15$          & $84.45 \pm 0.08$    & $\mathbf{47.02 \pm 0.11}$ & $52.50 \pm 0.07$ & $57.62 \pm 0.15$            \\
\midrule
Laplacian Kernel        & ${78.09 \pm 0.10}$    & $\mathbf{83.85 \pm 0.09}$          & $\mathbf{88.34 \pm 0.16}$    & $46.12 \pm 0.22$   & $53.44 \pm 0.17$ & $59.10 \pm 0.14$        \\
Simple Sum Kernel & $\mathbf{78.12 \pm 0.15}$   & $83.23 \pm 0.18$ & $87.50 \pm 0.20$ & $46.65 \pm 0.06$ & $\mathbf{53.62 \pm 0.19}$ & $\mathbf{59.83 \pm 0.12}$\\
\bottomrule
\end{tabular}
}
\end{table}



\section{More Experiments on Synthetic Data}


Consider a scenario with $n$ clusters, each containing $k$ vertices. Let the probability of vertices $u$ and $v$ from the same cluster belonging to $\bfpi$ be $p$. Conversely, for vertices $u$ and $v$ from different clusters, let the probability of belonging to $\pi$ be $q$. We generate the graph $\bfpi$ randomly, based on $p$ and $q$. We experiment with values of $k=100$ and $n=6$ for ease of visualization, embedding all points in a two-dimensional space. Each vertex's initial position originates from a normal distribution. In each iteration, we sample a subgraph of $\bfpi$ uniformly, ensuring each vertex has an out-degree of $1$. We then optimize the corresponding vectors using InfoNCE loss with an SGD optimizer and iterate until convergence. Our experimental setup consists of an SGD learning rate of $1$, an InfoNCE loss temperature of $0.5$, and a batch size of $50$. We evaluate two scenarios with different $p$ and $q$ values: $p=1$, $q=0$, and $p=0.75$, $q=0.2$. The results of these experiments are visualized in Figure \ref{fig:vis-spectral-cluster}. The obtained embeddings exhibit the hallmark pattern of spectral clustering of graph $\bfpi$.

\begin{figure}[!tb]
\centering
\subfigure{
\includegraphics[width=1\textwidth]{Figures/cluster_pi.png}
\label{fig:vis-cluster}
}
\subfigure{
\includegraphics[width=1\textwidth]{Figures/noised_cluster_pi.png}
\label{fig:vis-noised-cluster}
}
\caption{Visualizations of the optimization process using InfoNCE Loss on the vectors corresponding to $\bfpi$. Points of identical color belong to the same cluster within $\bfpi$. To showcase the internal structure of $\bfpi$, we randomly select 10 vertices from each cluster to display the edge distribution of $\bfpi$.}
\label{fig:vis-spectral-cluster}
\end{figure}




\end{document}
