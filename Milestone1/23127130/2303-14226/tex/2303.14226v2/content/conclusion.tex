\section{Conclusion}\label{sec:discussion}
%
This paper introduces a causal inference framework for combinatorial interventions, a setting that is ubiquitous in practice. 
%
We propose a model that imposes both unit-specific combinatorial structure and latent similarity across units.  
%
Under this model, \method, an estimation procedure, is introduced. 
%
\method~leverages the sparsity and low-rankness of the Fourier coefficients to efficiently estimate all $N \times 2^p$ causal parameters while implicitly allowing for unobserved confounding.
%
Theoretically, finite-sample consistency and asymptotic normality of \method~is established. 
%
A novel experiment design mechanism is proposed, which ensures that the key assumptions required for the estimator to accurately recover all $N \times 2^p$ mean potential outcomes of interest hold. 
%
The empirical effectiveness of \method~is demonstrated through numerical simulations and a real-world case study on movie ratings.
%
We discuss how \method~can be adapted to estimate counterfactuals under different permutations of items, i.e., rankings. 
%
This work suggests future directions for research such as providing an analysis of \method~that is agnostic to the horizontal regression algorithm used and deriving estimators that can achieve the sample complexity lower bound discussed.
%
More broadly, we hope this work serves as a bridge between causal inference and the Fourier analysis of Boolean functions.


%In this work, we formulate a causal inference framework for combinatorial interventions, a setting that is ubiquitous in practice. 
%
%We propose a model that imposes both unit-specific combinatorial structure and latent similarity across units.  
%
%Under this model, we propose an estimation procedure, \method, that exploits the sparsity and low-rankness of the Fourier coefficients to efficiently estimate all $N \times 2^p$ causal parameters and implicitly allows for unobserved confounding.
%
%We formally establish finite-sample consistency {\color{red} and asymptotic normality} of \method. 
%
%We use \method~to provide an experimental design mechanism for inferring all $N \times 2^p$ potential outcomes. 
%
%Next, we establish the empirical utility of \method~using both numerical simulations, and a real-world case study on user ratings for sets of movies. 
%
%We discuss how \method~can be adapted to estimate counterfactuals under different permutations of items, i.e., rankings. 
%
%Our work suggests future directions for research such as providing an analysis of \method~that is agnostic to the horizontal regression algorithm used and deriving estimators that can achieve the sample complexity lower bound discussed. 