We consider a setting where there are $N$ heterogeneous units and $p$ interventions.
%
Our goal is to learn unit-specific potential outcomes for any combination of these $p$ interventions, i.e., $N \times 2^p$ causal parameters. 
%
Choosing a combination of interventions is a problem that naturally arises in a variety of applications such as factorial design experiments, recommendation engines (e.g., showing a set of movies that maximizes engagement for a given user), combination therapies in medicine, selecting important features for machine learning models, etc.  
%
Running $N \times 2^p$ experiments to estimate the various parameters is likely expensive and/or infeasible as $N$ and $p$ grow.
%
Further, with observational data there is likely confounding, i.e., whether or not a unit is seen under a combination is correlated with its potential outcome under that combination.
%
To address these challenges, we propose a novel model that imposes latent structure across {\em both} units and combinations of interventions.
%
Specifically, we assume latent similarity in potential outcomes across units (i.e., the matrix of potential outcomes is approximately rank $r$) and regularity in how combinations of interventions interact (i.e., the coefficients in the Fourier expansion of the potential outcomes is approximately $s$ sparse). %low-rank structure
%
In this setting we establish identification for all $N \times 2^p$ parameters despite unobserved confounding.
%
We propose an estimation procedure, \method, and establish finite-sample consistency under precise conditions on the observation pattern. 
%
Our results imply that \method~is able to consistently estimate unit-specific potential outcomes given a total of $\text{poly}(r) \times \left( N + s^2p\right)$ observations.
%
In comparison, previous methods that do not exploit structure across both units and combinations have sample complexity scaling as $\min(N \times s^2p, \ \ r \times (N + 2^p))$.
%
We then use \method~to propose a data-efficient experimental design mechanism for combinatorial causal inference. 
%
We corroborate our theoretical findings with numerical simulations \footnote{Code for \method~and replicating experiments can be found at \url{https://github.com/aagarwal1996/synth_combo}}.



%
%We corroborate our theoretical findings with simulations.
%
% Our results also imply better sample  results for matrix completion where columns represent combinations of items rather than single items, which may be of independent interest.


%\section*{To do}
%\begin{itemize}
%    \item \st{Intro + Related Work [Suhas]}
%    \item \st{Agree on ordering of sections / high-level changes [Abhi]}
%    \item General Editing [Anish/Suhas]
%    \item Experiment Design  [Abhi]
%    \item Simulations [Abhi]
%    \item Full-Paper read through [Suhas/Anish/Abhi]
%    \item Proof-Checking [Suhas/Anish]
%\end{itemize}

%\begin{itemize}
%    \item \st{Combine assumptions}
%    \item \st{Move low-rank above 5.3}
%    \item \st{r types of people - running exmple from 5.3 all the way to tilde(w)}
%    \item Figures [Abhi]
%\end{itemize}
