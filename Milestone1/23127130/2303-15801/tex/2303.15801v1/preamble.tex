\usepackage[utf8]{inputenc}
\usepackage[T1]{fontenc}
\usepackage[english]{babel}

%\usepackage{authblk}

\usepackage{lineno}
\usepackage{standalone}
\usepackage{caption}
\usepackage{subcaption}
% \usepackage{floatrow}
% \newfloatcommand{capbtabbox}{table}[][\FBwidth]
%\usepackage{amsmath}
% \DeclareCaptionLabelFormat{andtable}{#1~#2  \&  \tablename~\thetable}
\usepackage{stmaryrd}
\usepackage{amssymb,amsmath,amstext,empheq}
\usepackage{upgreek}
\usepackage{graphicx}
\usepackage{booktabs}

% \usepackage{dcolumn}

% \usepackage[bookmarks]{hyperref}
% \hypersetup{
% 	colorlinks=true,
% }

% tikz
\usepackage{tikz}
% \usepackage{mathpazo}
\usetikzlibrary{decorations.pathmorphing,decorations.pathreplacing,decorations.markings,patterns,calc}
\usetikzlibrary{shapes.geometric, shapes.arrows, arrows, angles, quotes}
\usetikzlibrary{spy,backgrounds}
\usetikzlibrary{external}
\usetikzlibrary{intersections}
\usetikzlibrary{tikzmark}

\tikzstyle{startstop} = [rectangle, rounded corners, minimum width=3cm, minimum height=1cm,text centered, draw=black, fill=red!30]
\tikzstyle{io} = [trapezium, trapezium left angle=70, trapezium right angle=110, minimum width=3cm, minimum height=1cm, text centered, draw=black, fill=blue!30]
\tikzstyle{process} = [rectangle, minimum width=3cm, minimum height=1cm, text centered, draw=black, fill=orange!30]
\tikzstyle{decision} = [diamond, minimum width=3cm, minimum height=1cm, text centered, draw=black, fill=green!30]
\tikzstyle{arrow} = [->,>=stealth]

% inkscape figures
\usepackage{import}
\usepackage{pdfpages}
\usepackage{transparent}
\usepackage{xcolor}
\newcommand{\incfig}[2][1]{%
	\def\svgwidth{#1\columnwidth}
	\import{./figures/}{#2.pdf_tex}
}



\usepackage[normalem]{ulem}
\usepackage{textcomp}
\usepackage{siunitx}
\usepackage[most]{tcolorbox}
\newcommand\bmmax{2}
% \usepackage{bbold}
\usepackage{bm}
\usepackage[]{datatool}
\usepackage{calc}
\usepackage{textgreek}
\usepackage{bigints}
\usepackage{physics}
\usepackage{interval}
\usepackage{accents}
\usepackage{mathtools,eqparbox}
\usepackage{indentfirst}
\usepackage[algo2e,ruled,linesnumbered,vlined,commentsnumbered,longend]{algorithm2e}
\SetKw{Continue}{continue}
\SetKw{Break}{break}


% \usepackage{calrsfs}
% \DeclareMathAlphabet{\pazocal}{OMS}{zplm}{m}{n}


% \usepackage{algorithm,algorithmicx,algpseudocode}
% \newcommand{\nosemic}{\renewcommand{\@endalgocfline}{\relax}}% Drop semi-colon ;
% \newcommand{\dosemic}{\renewcommand{\@endalgocfline}{\algocf@endline}}% Reinstate semi-colon ;
% \newcommand{\pushline}{\Indp}% Indent
% \newcommand{\popline}{\Indm\dosemic}% Undent
% \usepackage{multirow}
\usepackage{enumerate}


\usepackage{relsize}
\usepackage[column=O]{cellspace}
\setlength{\cellspacetoplimit}{5pt}
\setlength{\cellspacebottomlimit}{5pt}

\usepackage{pgfplots}
\pgfplotsset{compat=newest}
% \DeclareUnicodeCharacter{2212}{−}
\usepgfplotslibrary{groupplots,dateplot}
\usepgfplotslibrary{colormaps}
\usepgfplotslibrary{colorbrewer}
% \usepgfplotslibrary{groupplots}
\usepgfplotslibrary{polar}
% \usepgfplotslibrary{external} 
\usepgfplotslibrary{fillbetween}

\usepackage[absolute,overlay]{textpos}

\usepackage{tikz-dimline}

\usepackage[autostyle]{csquotes}

% \usepackage[
% 	backend=biber,
% 	% style=unsrt,
% 	sortlocale=de_DE,
% 	natbib=true,
% 	url=false,
% 	doi=true,
% 	eprint=false
% ]{biblatex}


\usepackage[colorinlistoftodos]{todonotes}


\usepackage{hyperref}
% \hypersetup{pdftex,colorlinks=true,allcolors=blue}
\usepackage{hypcap}
\usepackage{bookmark}


\DeclareMathOperator*{\argmin}{arg\,min}
\DeclareMathOperator*{\locargmin}{loc\,arg\,min}
\DeclareMathOperator*{\argmax}{arg\,max}
\DeclareMathOperator*{\mineig}{min\,eig}
\DeclareMathOperator*{\minimize}{minimize}
\DeclareMathOperator*{\Div}{div}
\DeclareMathOperator*{\Vor}{Vor}
\DeclareMathOperator{\sgn}{sgn}
\DeclareMathOperator{\sign}{sign}
\DeclareMathOperator{\Var}{Var}
\DeclareMathOperator{\diag}{diag}
\DeclareMathOperator\supp{supp}
\DeclareMathOperator\diam{diam}
\DeclareMathOperator{\Grad}{grad}
%\setlength\parindent{0pt}

\newcommand{\tn}{\mathrm{n}} % normal traction
\newcommand{\ts}{\mathrm{s}} % tangential traction
\newcommand{\bG}{\boldsymbol{G}}
\newcommand{\bH}{\mathbf{H}}
\newcommand{\bv}{\boldsymbol{v}}
\newcommand{\bw}{\boldsymbol{w}}
\newcommand{\bq}{\boldsymbol{q}}
\newcommand{\bu}{\boldsymbol{u}} % continuous displacement tensor
\newcommand{\du}{\undertilde{\mathbf{u}}} % discretized displacement vector
\newcommand{\bn}{\boldsymbol{n}} % normal tensor
\newcommand{\bx}{\boldsymbol{x}} % spacial point
\newcommand{\ubx}{\mathbf{x}} % spacial point
\newcommand{\by}{\mathbf{y}} % design point
\newcommand{\bY}{\mathbf{Y}} % design samples
\newcommand{\uby}{\mathbf{y}} % spacial point
\newcommand{\ud}{\undertilde{\mathbf{u}}} % discretized displacement vector
\newcommand{\fd}{\undertilde{\mathbf{f}}} % discretized force vector
\newcommand{\bR}{\mathbf{R}} % discretized residual
\newcommand{\bz}{\mathbf{z}} % design vector
\newcommand{\cL}{\mathcal{L}}
\newcommand{\bC}{\boldsymbol{C}} % fourth-order elasticity tensor
\newcommand{\bSigma}{\boldsymbol{\Sigma}} % fourth-order elasticity tensor
\newcommand{\btau}{\boldsymbol{\tau}} % cohesive traction
\newcommand{\balpha}{\boldsymbol{\alpha}}
\newcommand{\blambda}{\boldsymbol{\uplambda}}
\newcommand{\bmu}{\boldsymbol{\upmu}}
\newcommand{\bnu}{\boldsymbol{\nu}}
\newcommand{\ba}{\boldsymbol{a}}
\newcommand{\bxi}{\boldsymbol{\xi}}
\newcommand{\bsigma}{\boldsymbol{\sigma}} % stress tensor
\newcommand{\bepsilon}{\boldsymbol{\varepsilon}} % strain tensor
\newcommand{\bg}{\mathbf{g}}
\newcommand{\tc}{\mathrm{c}}
\newcommand{\tC}{\mathrm{C}}
\newcommand{\bc}{\boldsymbol{c}}
\newcommand{\bh}{\boldsymbol{h}}
\newcommand{\uh}{\undertilde{\mathbf{h}}}
\newcommand{\uH}{\mathbf{H}}
\newcommand{\bbh}{\boldsymbol{h}}
\newcommand{\bb}{\boldsymbol{b}}
\newcommand{\bdelta}{\boldsymbol{\delta}}
\newcommand{\jump}[1]{\left\llbracket #1 \right\rrbracket} % jump
\newcommand{\ubK}{\mathbf{K}} % global elastic stiffness matrix
\newcommand{\ubM}{\mathbf{M}} % global mass matrix
\newcommand{\bphi}{\boldsymbol{\varphi}}
\newcommand{\bpsi}{\boldsymbol{\psi}}
\newcommand{\tJ}{\undertilde{J}}
\newcommand{\bomega}{\mathbf{\omega}}
\newcommand{\ddelta}{\undertilde{\bm{\updelta}}} % discretized crack openings
\newcommand{\rom}[1]{\uppercase\expandafter{\romannumeral #1\relax}}


\intervalconfig {
	soft open fences ,
}

% theorem
\usepackage{amsthm}

\theoremstyle{plain}
\newtheorem{theorem}{Theorem}[section]
\newtheorem{corollary}{Corollary}[theorem]
\newtheorem{lemma}[theorem]{Lemma}

\theoremstyle{definition}
\newtheorem{definition}{Definition}[section]

\theoremstyle{remark}
\newtheorem{remark}{Remark}

% https://tex.stackexchange.com/questions/84127/correctly-align-vertical-text-on-a-baseline-in-pgfplots
\def\mystrut{\vphantom{hg}}

% https://tex.stackexchange.com/questions/204395/add-custom-entry-into-legend-in-pgfplot
\pgfplotsset{
	legend image with text/.style={
			legend image code/.code={%
					\node[anchor=center] at (0.3cm,0cm) {#1};
				}
		},
}
