
%% AASTeX requires revtex4-1.cls and other external packages such as
%% latexsym, graphicx, amssymb, longtable, and epsf.  Note that as of 
%% Oct 2020, APS now uses revtex4.2e for its journals but remember that 
%% AASTeX v6+ still uses v4.1. All of these external packages should 
%% already be present in the modern TeX distributions but not always.
%% For example, revtex4.1 seems to be missing in the linux version of
%% TexLive 2020. One should be able to get all packages from www.ctan.org.
%% In particular, revtex v4.1 can be found at 
%% https://www.ctan.org/pkg/revtex4-1.


%\documentclass[preprint]{aastex631}
\documentclass[twocolumn]{aastex631_modif}

%% The default is a single spaced, 10 point font, single spaced article.
%% There are 5 other style options available via an optional argument. They
%% can be invoked like this:
%%
%% \documentclass[arguments]{aastex631}
%% 
%% where the layout options are:
%%
%%  twocolumn   : two text columns, 10 point font, single spaced article.
%%                This is the most compact and represent the final published
%%                derived PDF copy of the accepted manuscript from the publisher
%%  manuscript  : one text column, 12 point font, double spaced article.
%%  preprint    : one text column, 12 point font, single spaced article.  
%%  preprint2   : two text columns, 12 point font, single spaced article.
%%  modern      : a stylish, single text column, 12 point font, article with
%% 		  wider left and right margins. This uses the Daniel
%% 		  Foreman-Mackey and David Hogg design.
%%  RNAAS       : Supresses an abstract. Originally for RNAAS manuscripts 
%%                but now that abstracts are required this is obsolete for
%%                AAS Journals. Authors might need it for other reasons. DO NOT
%%                use \begin{abstract} and \end{abstract} with this style.
%%
%% Note that you can submit to the AAS Journals in any of these 6 styles.
%%
%% There are other optional arguments one can invoke to allow other stylistic
%% actions. The available options are:
%%
%%   astrosymb    : Loads Astrosymb font and define \astrocommands. 
%%   tighten      : Makes baselineskip slightly smaller, only works with 
%%                  the twocolumn substyle.
%%   times        : uses times font instead of the default
%%   linenumbers  : turn on lineno package.
%%   trackchanges : required to see the revision mark up and print its output
%%   longauthor   : Do not use the more compressed footnote style (default) for 
%%                  the author/collaboration/affiliations. Instead print all
%%                  affiliation information after each name. Creates a much 
%%                  longer author list but may be desirable for short 
%%                  author papers.
%% twocolappendix : make 2 column appendix.
%%   anonymous    : Do not show the authors, affiliations and acknowledgments 
%%                  for dual anonymous review.
%%
%% these can be used in any combination, e.g.
%%
%% \documentclass[twocolumn,linenumbers,trackchanges]{aastex631}
%%
%% AASTeX v6.* now includes \hyperref support. While we have built in specific
%% defaults into the classfile you can manually override them with the
%% \hypersetup command. For example,
%%
%% \hypersetup{linkcolor=red,citecolor=green,filecolor=cyan,urlcolor=magenta}
%%
%% will change the color of the internal links to red, the links to the
%% bibliography to green, the file links to cyan, and the external links to
%% magenta. Additional information on \hyperref options can be found here:
%% https://www.tug.org/applications/hyperref/manual.html#x1-40003
%%
%% Note that in v6.3 "bookmarks" has been changed to "true" in hyperref
%% to improve the accessibility of the compiled pdf file.
%%
%% If you want to create your own macros, you can do so
%% using \newcommand. Your macros should appear before
%% the \begin{document} command.
%%
\newcommand{\vdag}{(v)^\dagger}
\newcommand\aastex{AAS\TeX}
\newcommand\latex{La\TeX}

%% Reintroduced the \received and \accepted commands from AASTeX v5.2
%\received{March 1, 2021}
%\revised{April 1, 2021}
\accepted{\today}

%% Command to document which AAS Journal the manuscript was submitted to.
%% Adds "Submitted to " the argument.
%\submitjournal{RNAAS}



%%%%%%%%%%%%%%%%%%%%%%%%%%%%%%%%%%%%%%%%%%%%%%%%%%%%%%%%%%%%%%%%%%%%%%%%%%%%%%%%
%%
%% The following section outlines numerous optional output that
%% can be displayed in the front matter or as running meta-data.
%%
%% If you wish, you may supply running head information, although
%% this information may be modified by the editorial offices.
\shorttitle{School students observe Venus with IRTF}
\shortauthors{Peralta et al.}
%%
%% You can add a light gray and diagonal water-mark to the first page 
%% with this command:
%% \watermark{text}
%% where "text", e.g. DRAFT, is the text to appear.  If the text is 
%% long you can control the water-mark size with:
%% \setwatermarkfontsize{dimension}
%% where dimension is any recognized LaTeX dimension, e.g. pt, in, etc.
%%
%%%%%%%%%%%%%%%%%%%%%%%%%%%%%%%%%%%%%%%%%%%%%%%%%%%%%%%%%%%%%%%%%%%%%%%%%%%%%%%%
%\graphicspath{{./}{figures/}}
%% This is the end of the preamble.  Indicate the beginning of the
%% manuscript itself with \begin{document}.

\begin{document}

\title{Secondary School Students observe Venus with NASA Infrared Telescope Facility (IRTF)}



\correspondingauthor{Javier Peralta}
\email{jperalta1@us.es}

\author[0000-0002-6823-1695]{Javier Peralta}
\affiliation{Facultad de Física, Universidad de Sevilla \\
Avenida de la Reina Mercedes s/n \\
Sevilla, Spain}

\author[0000-0001-9607-0190]{Juan A. Prieto}
\affiliation{Colegio Huerta de la Cruz \\
Calle Vicente de Paul, 7 \\
Algeciras (C{\' a}diz), Spain}

\author{Pilar Orozco-S{\' a}enz}
\affiliation{Colegio Huerta de la Cruz \\
Calle Vicente de Paul, 7 \\
Algeciras (C{\' a}diz), Spain}

\author{Jes{\' u}s Gonz{\' a}lez}
\affiliation{Colegio Huerta de la Cruz \\
Calle Vicente de Paul, 7 \\
Algeciras (C{\' a}diz), Spain}

\author{Gonzalo Trujillo}
\affiliation{Colegio Huerta de la Cruz \\
Calle Vicente de Paul, 7 \\
Algeciras (C{\' a}diz), Spain}

\author{Luc{\' i}a Torres}
\affiliation{Colegio Huerta de la Cruz \\
Calle Vicente de Paul, 7 \\
Algeciras (C{\' a}diz), Spain}

\author{Alberto S{\' a}nchez}
\affiliation{Colegio Huerta de la Cruz \\
Calle Vicente de Paul, 7 \\
Algeciras (C{\' a}diz), Spain}

\author{Manuel Arnedo}
\affiliation{Colegio Huerta de la Cruz \\
Calle Vicente de Paul, 7 \\
Algeciras (C{\' a}diz), Spain}



%\collaboration{20}{Manuel Arnedo and Luc{\' i}a Torres}



%% Note that the \and command from previous versions of AASTeX is now
%% depreciated in this version as it is no longer necessary. AASTeX 
%% automatically takes care of all commas and "and"s between authors names.

%% AASTeX 6.31 has the new \collaboration and \nocollaboration commands to
%% provide the collaboration status of a group of authors. These commands 
%% can be used either before or after the list of corresponding authors. The
%% argument for \collaboration is the collaboration identifier. Authors are
%% encouraged to surround collaboration identifiers with ()s. The 
%% \nocollaboration command takes no argument and exists to indicate that
%% the nearby authors are not part of surrounding collaborations.

%% Mark off the abstract in the ``abstract'' environment. 
\begin{abstract}
Astronomy and astrophysics are regarded as highly motivating topics for students in primary and secondary schools, and they have been a recurrent and effective resource to inspire passion about science. In fact, during the last years we have witnessed a boost of facilities providing small robotic telescopes for teachers and students to remotely undertake their own observing projects. A step forward is presented here, where we describe the experience of secondary school students attending professional observations of Venus at NASA's Infrared Telescope Facility (IRTF) and, in a second observing run, conducting the observations by themselves. In addition to quickly mastering the basic operation of the control software for the SpeX instrument, the students successfully performed different types of data acquisition, including drift scan imaging.
\end{abstract}

%% Keywords should appear after the \end{abstract} command. 
%% The AAS Journals now uses Unified Astronomy Thesaurus concepts:
%% https://astrothesaurus.org
%% You will be asked to selected these concepts during the submission process
%% but this old "keyword" functionality is maintained in case authors want
%% to include these concepts in their preprints.
\keywords{Astronomy education (2165), Observational astronomy (1145), Infrared observatories (791), Astronomical techniques (1684), Direct imaging(387), Drift scan imaging (410), Venus (1763), Planetary atmospheres (1244), Atmospheric clouds (2180)}



\section{Introduction} \label{sec:intro}
Astronomy is one of the oldest sciences known. For millennia, it has captivated most of the cultures in the world, and it yet remains at the forefront of the attention and interest of public \citep{Bailey2003}. Many teachers have used astronomy to counter the problem that many students find the science content of middle years of schooling uninteresting \citep{Salimpour2021}. In light of the global push to get students engaged in science and technology, many aspects of astronomy have become popular and introduced in school curricula for decades \citep{Lelliott2010}, leveraging the many examples in astrophysics with direct links to Physics, Chemistry, Mathematics and even Biology \citep{Salimpour2021}.\\


%% The "ht!" tells LaTeX to put the figure "here" first, at the "top" next
%% and to override the normal way of calculating a float position
%\begin{figure}[hb!]
\begin{figure*}[ht!]
\centering
%\plotone{Screen_2022-03-11_2022A038.png}
\includegraphics[width=0.9\textwidth]{Screen_2022-03-11_2022A038.png}
\caption{Capture of screen during observing run conducted by the students in 13 of March 2022.
\label{fig:observation}}
\end{figure*}
%\end{figure}

In this context, there has been an significant growth in the number robotic telescopes with observing time fully or partially devoted to educational purposes and friendly user interface to ease the remote control by school students \citep{Gomez2017}. This has allowed school teachers to easily implement inquiry-based learning approaches and students to have authentic science experiences, start international collaborations and even make discoveries and publish the results \citep{Salimpour2018,Fitzgerald2018}.\\

In this work, we describe how a team of secondary school students performed professional observations of the planet Venus with the spectrograph and imager SpeX \citep{Rayner2003} at the National Aeronautics and Space Administration Infrared Telescope Facility (NASA/IRTF) on Mauka Kea (Hawaii). Venus is a captivating target since it exhibits the consequences of a runaway greenhouse effect and it has been recently in the spotlight of the search for life \citep{Greaves2021}.\\


\section{Methodology} \label{sec:method}
The three participating students (J. Gonz{\' a}lez, A. S{\' a}nchez and G.~Trujillo) were aged 15-16 and studied at \textit{Huerta de la Cruz}, a catholic school located in the city of Algeciras (Spain). These students volunteered to carry out a project on Venus \citep[see \textit{work}]{Peralta2023zenodo}, motivated and coordinated by J.~A. Prieto and P. Orozco-S{\' a}enz, professors with successful experience engaging students with scientific research \citep{Prieto2020,Prieto2022sons,Orozco2021}. The objectives of this work comprised acquiring a basic knowledge of Venus and its atmosphere, a discussion about its habitability, and knowing the methods for observing Venus \citep[see \textit{log file}]{Peralta2023zenodo}.\\ 

With regards to the latter objective, the students were invited to participate in professional observations at NASA/IRTF between January and March of 2022. Two programs for observing Venus were scheduled\footnote{\url{http://irtfweb.ifa.hawaii.edu/observing/schedule.php}}: programs 2022A058 (with E.~F. Young as Principal Investigator) and 2022A038 (J. Peralta as Principal Investigator). Although with different scientific goals, both programs shared similar techniques of Venus data acquisition with SpeX \citep[see \textit{work}]{Peralta2023zenodo}: sets of Venus images with several filters using the guide camera (hereafter \textit{guidedog}), hyperspectral data with the slit of the spectrograph (hereafter \textit{bigdog}) scanning from north to south the disk of Venus, guiding on a star to obtain calibration spectra, or inference of flats, darks and bias. The full imagery dataset from these two programs has already been employed in recent Venus research \citep{Peralta2023AyA}, and it will become publicly available at NASA/IPAC Infrared Science Archive.\\

Prior to undertaking observations of Venus with SpeX, the students used the bibliography to became introduced to the clouds of Venus \citep{Peralta2019Icarus}, the SpeX instrument \citep{Rayner2003}, and the graphical user interfaces to control \textit{guidedog} (\textit{Guidedog X-windows User Interface} or GXUI, and \textit{Guidedog Data Viewer} or GDV) and the T3 remote TCS widget to adjust the pointing coordinates and the focuser \citep{Rayner2021}. After this literature review, the students benefited from a training session by remotely attending an observing run during 17 of February 2022 (2022A058). Finally, the students joined a last run during 11 of March 2022 (2022A038) to conduct the Venus observations by themselves (see Figure \ref{fig:observation}).\\

\newpage




\section{Experience}\label{sec:result}
During the observing run of 11 of March 2022, the students conducted the remote observations of Venus with IRTF/SpeX for about 2 hours, and 1 hour was recorded\footnote{Video available upon reasonable request and for educational purposes only.}. Jes{\' u}s Gonz{\' a}lez was selected to make the VNC connection and control GXUI, GDV and the T3 remote TCS widget, supervised by Javier Peralta. In addition to quickly understand most of the explications, J. Gonz{\' a}lez also performed the following operations with success: 
\begin{itemize}
    \item \textit{Fast acquisition of images with different filters and integration times.}
    \item \textit{Acquisition of hyperspectral data.} The student used T3 remote to introduce corrections for the position of the slit during the drift scan of the night and day side of Venus.
    \item \textit{Acquisition of images/spectra of the sky.} The student used T3 remote to shift the slit to a location of the sky at a convenient angular separation.
    \item \textit{IR guiding with SpeX guider/slit viewer.} The student properly employed GXUI and GDV to move a target star into the telescope A beam box and start the guiding with the Auto GuideBox Setup \citep{Rayner2021}.
\end{itemize}




\section{Conclusions}
We have shown in this work that a group of motivated secondary school students have been able to understand basic and more elaborated strategies to observe a planet of the solar system, and also learn to use a complex telescope interface to undertake the observations scheduled for a professional program. Encouraged by the positive experience, we plan to expand this recognised work \citep{Prieto2022cea} with new research projects adapted to secondary education which may include an active participation in the design of professional observing proposals and/or leading novel research under the supervision of postdoctoral researchers, an initiative successfully implemented in other projects \citep{Dunn2018}.\\



\section*{Author Contributions}
Conceptualization, software, resources, data curation, writing—original draft preparation and funding acquisition, by J.P.; methodology, writing—review and editing, supervision and validation, by J.P., J.A.P. and P.O-S.; investigation, visualization and formal analysis, by J.G., M.A., A.S. and G.T.\\



\begin{acknowledgments}
 J.P. thanks funding by the program EMERGIA, Junta de Andalucía (Spain), Grant EMERGIA20{\_}00414. J.A.P. and P.O-S. acknowledge the support from Asociaci{\' o}n Amigos de la Ciencia, Diverciencia. We also thank \textit{E.~F. Young} and \textit{M.~A. Bullock} for allowing the students to attend one of their observing runs. We acknowledge the support from \textit{Callie Matulonis} (Telescope Operator during the run with students) and the rest of staff at the Infrared Telescope Facility, which is operated by the University of Hawaii under contract 80HQTR19D0030 with the National Aeronautics and Space Administration. 
\end{acknowledgments}

%% To help institutions obtain information on the effectiveness of their 
%% telescopes the AAS Journals has created a group of keywords for telescope 
%% facilities.


\vspace{5mm}
\facilities{IRTF(SpeX)}




%% For this sample we use BibTeX plus aasjournals.bst to generate the
%% the bibliography. The sample631.bib file was populated from ADS. To
%% get the citations to show in the compiled file do the following:
%%
%% pdflatex sample631.tex
%% bibtext sample631
%% pdflatex sample631.tex
%% pdflatex sample631.tex


%\bibliography{bibliography}{}
\begin{thebibliography}{}
\expandafter\ifx\csname natexlab\endcsname\relax\def\natexlab#1{#1}\fi
\providecommand{\url}[1]{\href{#1}{#1}}
\providecommand{\dodoi}[1]{doi:~\href{http://doi.org/#1}{\nolinkurl{#1}}}
\providecommand{\doeprint}[1]{\href{http://ascl.net/#1}{\nolinkurl{http://ascl.net/#1}}}
\providecommand{\doarXiv}[1]{\href{https://arxiv.org/abs/#1}{\nolinkurl{https://arxiv.org/abs/#1}}}

\bibitem[{{Bailey} \& {Slater}(2003)}]{Bailey2003}
{Bailey}, J.~M., \& {Slater}, T.~F. 2003, Astronomy Education Review, 2, 20,
  \dodoi{10.3847/AER2003015}

\bibitem[{{Dunn} {et~al.}(2018){Dunn}, {Chubb}, {Tessenyi}, {Heward},
  {Tennyson}, {James}, {Darby}, {Niculescu-Duvaz}, {Meyer}, {Holdship},
  {Baker}, {Smutna}, {Virdee}, {Brannan}, {Tinetti}, {Sousa-Silva},
  {Mckemmish}, {Gorman}, \& {Rivlin}}]{Dunn2018}
{Dunn}, W., {Chubb}, K., {Tessenyi}, M., {et~al.} 2018, in European Planetary
  Science Congress, EPSC2018--970

\bibitem[{{Fitzgerald} {et~al.}(2018){Fitzgerald}, {Cutts}, {Salimpour}, \&
  {Slater}}]{Fitzgerald2018}
{Fitzgerald}, M.~T., {Cutts}, R., {Salimpour}, S., \& {Slater}, S. 2018,
  Robotic Telescope, Student Research and Education Proceedings, 1, 1

\bibitem[{{Gomez} \& {Fitzgerald}(2017)}]{Gomez2017}
{Gomez}, E.~L., \& {Fitzgerald}, M.~T. 2017, The Astronomical Review, 13, 28,
  \dodoi{10.1080/21672857.2017.1303264}

\bibitem[{{Greaves} {et~al.}(2021){Greaves}, {Richards}, {Bains}, {Rimmer},
  {Sagawa}, {Clements}, {Seager}, {Petkowski}, {Sousa-Silva}, {Ranjan},
  {Drabek-Maunder}, {Fraser}, {Cartwright}, {Mueller-Wodarg}, {Zhan},
  {Friberg}, {Coulson}, {Lee}, \& {Hoge}}]{Greaves2021}
{Greaves}, J.~S., {Richards}, A. M.~S., {Bains}, W., {et~al.} 2021, Nature
  Astronomy, 5, 655, \dodoi{10.1038/s41550-020-1174-4}

\bibitem[{{Lelliott} \& {Rollnick}(2010)}]{Lelliott2010}
{Lelliott}, A., \& {Rollnick}, M. 2010, International Journal of Science
  Education, 32, 1771, \dodoi{10.1080/09500690903214546}

\bibitem[{{Orozco-S{\' a}enz}(2021)}]{Orozco2021}
{Orozco-S{\' a}enz}, M.~P. 2021, {Habitable zone: where to search for life in
  the Solar System}, Talk at Online Workshop MW-Gaia: Bringing the Milky Way to
  schools 2021.
\newblock
  \url{http://mao.tfai.vu.lt/schools/index.php?option=com_content&task=view&id=191&Itemid=294}

\bibitem[{{Peralta} {et~al.}(2023{\natexlab{a}}){Peralta}, {Prieto},
  {Orozco-S{\' a}enz}, {Gonz{\' a}lez}, {Arnedo}, {S{\' a}nchez}, \&
  {Trujillo}}]{Peralta2023zenodo}
{Peralta}, J., {Prieto}, J.~A., {Orozco-S{\' a}enz}, M.~P., {et~al.}
  2023{\natexlab{a}}, {Supplemental Material for "Secondary School Students
  observe Venus with NASA Infrared Telescope Facility (IRTF)"}, Zenodo,
  \dodoi{10.5281/zenodo.7702389}

\bibitem[{{Peralta} {et~al.}(2019){Peralta}, {S{\'a}nchez-Lavega},
  {Horinouchi}, {McGouldrick}, {Garate-Lopez}, {Young}, {Bullock}, {Lee},
  {Imamura}, {Satoh}, \& {Limaye}}]{Peralta2019Icarus}
{Peralta}, J., {S{\'a}nchez-Lavega}, A., {Horinouchi}, T., {et~al.} 2019,
  Icarus, 333, 177, \dodoi{10.1016/j.icarus.2019.05.026}

\bibitem[{{Peralta} {et~al.}(2023{\natexlab{b}}){Peralta}, {Cidad{\~a}o},
  {Morrone}, {Foster}, {Bullock}, {Young}, {Garate-Lopez},
  {S{\'a}nchez-Lavega}, {Horinouchi}, {Imamura}, {Kardasis}, {Yamazaki}, \&
  {Watanabe}}]{Peralta2023AyA}
{Peralta}, J., {Cidad{\~a}o}, A., {Morrone}, L., {et~al.} 2023{\natexlab{b}},
  Astronomy \& Astrophysics, \dodoi{10.1051/0004-6361/202244822}

\bibitem[{{Prieto} {et~al.}(2022){Prieto}, {Gonz{\' a}lez}, {Trujillo}, {S{\'
  a}nchez}, \& {Peralta}}]{Prieto2022cea}
{Prieto}, J.~A., {Gonz{\' a}lez}, J., {Trujillo}, G., {S{\' a}nchez}, A., \&
  {Peralta}, J. 2022, {Ocultos en la Noche de Venus}, Awarded with Honorable
  Mention at Ciencia en Acci{\' o}n XXIII in Viladecans (Spain).
\newblock \url{https://cienciaenaccion.org/premiados/premiadospresencial/}

\bibitem[{{Prieto} \& {Orozco-S{\' a}enz}(2020)}]{Prieto2020}
{Prieto}, J.~A., \& {Orozco-S{\' a}enz}, M.~P. 2020, {Habitabily Zone: The
  challenge of the five kingdoms}, Best astronomical research project at EAAE
  Catch a Star 2020.
\newblock
  \url{https://eaae-astronomy.org/catch-a-star/cas-winners/cas-winners-2020}

\bibitem[{{Prieto} \& {Orozco-S{\' a}enz}(2022)}]{Prieto2022sons}
---. 2022, {Astrobiology: a new tool to engage primary school students into
  science}, Winning project at Science on Stage 2022 Prague.
\newblock \url{https://www.sons2022.eu/en/home/winning-projects}

\bibitem[{{Rayner} {et~al.}(2021){Rayner}, {Bond}, {Bonnet}, {Jaffe}, {Muller},
  \& {Tokunaga}}]{Rayner2021}
{Rayner}, J., {Bond}, T., {Bonnet}, M., {et~al.} 2021, {SpeX Observing Manual},
  NASA IRTF

\bibitem[{{Rayner} {et~al.}(2003){Rayner}, {Toomey}, {Onaka}, {Denault},
  {Stahlberger}, {Vacca}, {Cushing}, \& {Wang}}]{Rayner2003}
{Rayner}, J.~T., {Toomey}, D.~W., {Onaka}, P.~M., {et~al.} 2003, The
  Publications of the Astronomical Society of the Pacific, 115, 362,
  \dodoi{10.1086/367745}

\bibitem[{{Salimpour} {et~al.}(2018){Salimpour}, {Bartlett}, {Fitzgerald},
  {Cutts}, {James}, {Miller}, {Danaia}, {Cabezon}, {Faye}, {Baudouin},
  {Birkenbauma}, {Anjos}, {Wu}, {Chu}, \& {Slater}}]{Salimpour2018}
{Salimpour}, S., {Bartlett}, S., {Fitzgerald}, M.~T., {et~al.} 2018, Robotic
  Telescope, Student Research and Education Proceedings, 1, 33

\bibitem[{{Salimpour} {et~al.}(2021){Salimpour}, {Bartlett}, {Fitzgerald},
  {McKinnon}, {Cutts}, {James}, {Miller}, {Danaia}, {Hollow}, {Cabezon},
  {Faye}, {Tomita}, {Max}, {de Korte}, {Baudouin}, {Birkenbauma}, {Kallery},
  {Anjos}, {Wu}, {Chu}, {Slater}, \& {Ortiz-Gil}}]{Salimpour2021}
---. 2021, Research in Science Education, 51, 975,
  \dodoi{10.1007/s11165-020-09922-0}

\end{thebibliography}


\bibliographystyle{aasjournal}

%% This command is needed to show the entire author+affiliation list when
%% the collaboration and author truncation commands are used.  It has to
%% go at the end of the manuscript.
%\allauthors

%% Include this line if you are using the \added, \replaced, \deleted
%% commands to see a summary list of all changes at the end of the article.
%\listofchanges

\end{document}


