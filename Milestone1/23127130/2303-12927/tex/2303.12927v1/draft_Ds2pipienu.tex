%%%%%%%%%%%%%%%%%%%%%%%%%%%%%%%%%
%%\RequirePackage[mathlines]{lineno} % Display line numbers
\documentclass[aps,twocolumn,showpacs,prl,reprint,nofootinbib]{revtex4-1}
\usepackage{epsfig}
\usepackage{graphicx}% Include figure files
\usepackage{dcolumn}% Align table columns on decimal point
\usepackage{bm}% bold math
\usepackage{overpic}
\usepackage{subfigure}
\usepackage{float}
\usepackage{color}
\usepackage{amsmath}
\usepackage{mathcomp}
\usepackage[linktoc=all]{hyperref}
%\linespread{1.3}

%%%%%%%%%%%%%%%%%%%%%%%%%%%%%%%%%%%
%%%define new varible
\newcommand{\gevcc}{\ensuremath{{\mathrm{GeV/}c^2}~}}
\newcommand{\gevgevcccc}{\ensuremath{{\mathrm{GeV^2/}c^4}~}}
%%%%%%%%%%%%%%%%%%%%%%%%%%%%%%%%%%%

%%%%%%%%%%%%%%%%%%%%%%%%%%%%%%
%%%show linenumber before equation
%\let\oldequation\equation
%\let\oldendequation\endequation
%\renewenvironment{equation}{\linenomathNonumbers\oldequation}{\oldendequation\endlinenomath}
%%%%%%%%%%%%%%%%%%%%%%%%%%%%%%

\begin{document}
\title{ \boldmath Study of the $f_{0}(980)$ through the decay $D_{s}^{+} \to \pi^{+} \pi^{-} e^{+} \nu_{e}$ }
% author list 03/07/2023

\newcommand*{\ANL}{Argonne National Laboratory, Argonne, Illinois 60439}
\newcommand*{\ANLindex}{1}
\affiliation{\ANL}
\newcommand*{\CSUDH}{California State University, Dominguez Hills, Carson, CA 90747}
\newcommand*{\CSUDHindex}{2}
\affiliation{\CSUDH}
\newcommand*{\CANISIUS}{Canisius College, Buffalo, NY}
\newcommand*{\CANISIUSindex}{3}
\affiliation{\CANISIUS}
\newcommand*{\SACLAY}{IRFU, CEA, Universit\'{e} Paris-Saclay, F-91191 Gif-sur-Yvette, France}
\newcommand*{\SACLAYindex}{4}
\affiliation{\SACLAY}
\newcommand*{\CNU}{Christopher Newport University, Newport News, Virginia 23606}
\newcommand*{\CNUindex}{5}
\affiliation{\CNU}
\newcommand*{\UCONN}{University of Connecticut, Storrs, Connecticut 06269}
\newcommand*{\UCONNindex}{6}
\affiliation{\UCONN}
\newcommand*{\DUKE}{Duke University, Durham, North Carolina 27708-0305}
\newcommand*{\DUKEindex}{7}
\affiliation{\DUKE}
\newcommand*{\DUQUESNE}{Duquesne University, 600 Forbes Avenue, Pittsburgh, PA 15282 }
\newcommand*{\DUQUESNEindex}{8}
\affiliation{\DUQUESNE}
\newcommand*{\FU}{Fairfield University, Fairfield CT 06824}
\newcommand*{\FUindex}{9}
\affiliation{\FU}
\newcommand*{\FERRARAU}{Universita' di Ferrara , 44121 Ferrara, Italy}
\newcommand*{\FERRARAUindex}{10}
\affiliation{\FERRARAU}
\newcommand*{\FIU}{Florida International University, Miami, Florida 33199}
\newcommand*{\FIUindex}{11}
\affiliation{\FIU}
\newcommand*{\FSU}{Florida State University, Tallahassee, Florida 32306}
\newcommand*{\FSUindex}{12}
\affiliation{\FSU}
\newcommand*{\GWUI}{The George Washington University, Washington, DC 20052}
\newcommand*{\GWUIindex}{13}
\affiliation{\GWUI}
\newcommand*{\GSIFFN}{GSI Helmholtzzentrum fur Schwerionenforschung GmbH, D-64291 Darmstadt, Germany}
\newcommand*{\GSIFFNindex}{14}
\affiliation{\GSIFFN}
\newcommand*{\INFNFE}{INFN, Sezione di Ferrara, 44100 Ferrara, Italy}
\newcommand*{\INFNFEindex}{15}
\affiliation{\INFNFE}
\newcommand*{\INFNFR}{INFN, Laboratori Nazionali di Frascati, 00044 Frascati, Italy}
\newcommand*{\INFNFRindex}{16}
\affiliation{\INFNFR}
\newcommand*{\INFNGE}{INFN, Sezione di Genova, 16146 Genova, Italy}
\newcommand*{\INFNGEindex}{17}
\affiliation{\INFNGE}
\newcommand*{\INFNRO}{INFN, Sezione di Roma Tor Vergata, 00133 Rome, Italy}
\newcommand*{\INFNROindex}{18}
\affiliation{\INFNRO}
\newcommand*{\INFNTUR}{INFN, Sezione di Torino, 10125 Torino, Italy}
\newcommand*{\INFNTURindex}{19}
\affiliation{\INFNTUR}
\newcommand*{\INFNPAV}{INFN, Sezione di Pavia, 27100 Pavia, Italy}
\newcommand*{\INFNPAVindex}{20}
\affiliation{\INFNPAV}
\newcommand*{\ORSAY}{Universit'{e} Paris-Saclay, CNRS/IN2P3, IJCLab, 91405 Orsay, France}
\newcommand*{\ORSAYindex}{21}
\affiliation{\ORSAY}
\newcommand*{\Juelich}{Institute fur Kernphysik (Juelich), Juelich, Germany}
\newcommand*{\Juelichindex}{22}
\affiliation{\Juelich}
\newcommand*{\JMU}{James Madison University, Harrisonburg, Virginia 22807}
\newcommand*{\JMUindex}{23}
\affiliation{\JMU}
\newcommand*{\KNU}{Kyungpook National University, Daegu 41566, Republic of Korea}
\newcommand*{\KNUindex}{24}
\affiliation{\KNU}
\newcommand*{\LAMAR}{Lamar University, 4400 MLK Blvd, PO Box 10046, Beaumont, Texas 77710}
\newcommand*{\LAMARindex}{25}
\affiliation{\LAMAR}
\newcommand*{\MIT}{Massachusetts Institute of Technology, Cambridge, Massachusetts  02139-4307}
\newcommand*{\MITindex}{26}
\affiliation{\MIT}
\newcommand*{\MISS}{Mississippi State University, Mississippi State, MS 39762-5167}
\newcommand*{\MISSindex}{27}
\affiliation{\MISS}
\newcommand*{\ITEP}{National Research Centre Kurchatov Institute - ITEP, Moscow, 117259, Russia}
\newcommand*{\ITEPindex}{28}
\affiliation{\ITEP}
\newcommand*{\UNH}{University of New Hampshire, Durham, New Hampshire 03824-3568}
\newcommand*{\UNHindex}{29}
\affiliation{\UNH}
\newcommand*{\NMSU}{New Mexico State University, PO Box 30001, Las Cruces, NM 88003, USA}
\newcommand*{\NMSUindex}{30}
\affiliation{\NMSU}
\newcommand*{\NSU}{Norfolk State University, Norfolk, Virginia 23504}
\newcommand*{\NSUindex}{31}
\affiliation{\NSU}
\newcommand*{\OHIOU}{Ohio University, Athens, Ohio  45701}
\newcommand*{\OHIOUindex}{32}
\affiliation{\OHIOU}
\newcommand*{\ODU}{Old Dominion University, Norfolk, Virginia 23529}
\newcommand*{\ODUindex}{33}
\affiliation{\ODU}
\newcommand*{\JLUGiessen}{II Physikalisches Institut der Universitaet Giessen, 35392 Giessen, Germany}
\newcommand*{\JLUGiessenindex}{34}
\affiliation{\JLUGiessen}
\newcommand*{\ROMAII}{Universita' di Roma Tor Vergata, 00133 Rome Italy}
\newcommand*{\ROMAIIindex}{35}
\affiliation{\ROMAII}
\newcommand*{\MSU}{Skobeltsyn Institute of Nuclear Physics, Lomonosov Moscow State University, 119234 Moscow, Russia}
\newcommand*{\MSUindex}{36}
\affiliation{\MSU}
\newcommand*{\SCAROLINA}{University of South Carolina, Columbia, South Carolina 29208}
\newcommand*{\SCAROLINAindex}{37}
\affiliation{\SCAROLINA}
\newcommand*{\TEMPLE}{Temple University,  Philadelphia, PA 19122 }
\newcommand*{\TEMPLEindex}{38}
\affiliation{\TEMPLE}
\newcommand*{\JLAB}{Thomas Jefferson National Accelerator Facility, Newport News, Virginia 23606}
\newcommand*{\JLABindex}{39}
\affiliation{\JLAB}
\newcommand*{\UTFSM}{Universidad T\'{e}cnica Federico Santa Mar\'{i}a, Casilla 110-V Valpara\'{i}so, Chile}
\newcommand*{\UTFSMindex}{40}
\affiliation{\UTFSM}
\newcommand*{\INSUBRIA}{Universit\`{a} degli Studi dell'Insubria, 22100 Como, Italy}
\newcommand*{\INSUBRIAindex}{41}
\affiliation{\INSUBRIA}
\newcommand*{\BRESCIA}{Universit`{a} degli Studi di Brescia, 25123 Brescia, Italy}
\newcommand*{\BRESCIAindex}{42}
\affiliation{\BRESCIA}
\newcommand*{\UCR}{University of California Riverside, 900 University Avenue, Riverside, CA 92521, USA}
\newcommand*{\UCRindex}{43}
\affiliation{\UCR}
\newcommand*{\GLASGOW}{University of Glasgow, Glasgow G12 8QQ, United Kingdom}
\newcommand*{\GLASGOWindex}{44}
\affiliation{\GLASGOW}
\newcommand*{\YORK}{University of York, York YO10 5DD, United Kingdom}
\newcommand*{\YORKindex}{45}
\affiliation{\YORK}
\newcommand*{\VIRGINIA}{University of Virginia, Charlottesville, Virginia 22901}
\newcommand*{\VIRGINIAindex}{46}
\affiliation{\VIRGINIA}
\newcommand*{\WM}{College of William and Mary, Williamsburg, Virginia 23187-8795}
\newcommand*{\WMindex}{47}
\affiliation{\WM}
\newcommand*{\YEREVAN}{Yerevan Physics Institute, 375036 Yerevan, Armenia}
\newcommand*{\YEREVANindex}{48}
\affiliation{\YEREVAN}
 

\newcommand*{\NOWANL}{Argonne National Laboratory, Argonne, Illinois 60439}
\newcommand*{\NOWJLAB}{Thomas Jefferson National Accelerator Facility, Newport News, Virginia 23606}
 %%%%%%%%%%%%%%% END OF Latex Macros for institute addresses  %%%%%%%%%%%%%%%%%%%%%%%%% 

\author {S.~Diehl} 
\affiliation{\JLUGiessen}
\affiliation{\UCONN}
\author {N.~Trotta} 
\affiliation{\UCONN}
\author {K.~Joo} 
\affiliation{\UCONN}
\author {P.~Achenbach} 
\affiliation{\JLAB}
\author {Z.~Akbar} 
\affiliation{\VIRGINIA}
\affiliation{\FSU}
\author {W.R.~Armstrong} 
\affiliation{\ANL}
\author {H.~Atac} 
\affiliation{\TEMPLE}
\author {H.~Avakian} 
\affiliation{\JLAB}
\author {L.~Baashen} 
\affiliation{\FIU}
\author {N.A.~Baltzell} 
\affiliation{\JLAB}
\author {L.~Barion} 
\affiliation{\INFNFE}
\author {M.~Bashkanov} 
\affiliation{\YORK}
\author {M.~Battaglieri} 
\affiliation{\INFNGE}
\author {I.~Bedlinskiy} 
\affiliation{\ITEP}
\author {F.~Benmokhtar} 
\affiliation{\DUQUESNE}
\author {A.~Bianconi} 
\affiliation{\BRESCIA}
\affiliation{\INFNPAV}
\author {A.S.~Biselli} 
\affiliation{\FU}
\author {F.~Boss\`u} 
\affiliation{\SACLAY}
\author {K.-T.~Brinkmann} 
\affiliation{\JLUGiessen}
\author {W.J.~Briscoe} 
\affiliation{\GWUI}
\author {D.~Bulumulla} 
\affiliation{\ODU}
\author {V.~Burkert} 
\affiliation{\JLAB}
\author {R.~Capobianco} 
\affiliation{\UCONN}
\author {D.S.~Carman} 
\affiliation{\JLAB}
\author {J.C.~Carvajal} 
\affiliation{\FIU}
\author {A.~Celentano} 
\affiliation{\INFNGE}
\author {G.~Charles} 
\affiliation{\ORSAY}
\affiliation{\ODU}
\author {P.~Chatagnon} 
\affiliation{\JLAB}
\affiliation{\ORSAY}
\author {V.~Chesnokov} 
\affiliation{\MSU}
\author {G.~Ciullo} 
\affiliation{\INFNFE}
\affiliation{\FERRARAU}
\author {P.L.~Cole} 
\affiliation{\LAMAR}
\author {M.~Contalbrigo} 
\affiliation{\INFNFE}
\author {G.~Costantini} 
\affiliation{\BRESCIA}
\affiliation{\INFNPAV}
\author {V.~Crede} 
\affiliation{\FSU}
\author {A.~D'Angelo} 
\affiliation{\INFNRO}
\affiliation{\ROMAII}
\author {N.~Dashyan} 
\affiliation{\YEREVAN}
\author {R.~De~Vita} 
\affiliation{\INFNGE}
\author {A.~Deur} 
\affiliation{\JLAB}
\author {C.~Djalali} 
\affiliation{\OHIOU}
\affiliation{\SCAROLINA}
\author {R.~Dupre} 
\affiliation{\ORSAY}
\author {M.~Ehrhart} 
\altaffiliation[Current address:]{\NOWANL}
\affiliation{\ORSAY}
\author {A.~El~Alaoui} 
\affiliation{\UTFSM}
\author {L.~El~Fassi} 
\affiliation{\MISS}
\author {S.~Fegan} 
\affiliation{\YORK}
\author {A.~Filippi} 
\affiliation{\INFNTUR}
\author {G.~Gavalian} 
\affiliation{\JLAB}
\author {D.I.~Glazier} 
\affiliation{\GLASGOW}
\author {A.A. Golubenko} 
\affiliation{\MSU}
\author {G.~Gosta} 
\affiliation{\BRESCIA}
\affiliation{\INFNPAV}
\author {R.W.~Gothe} 
\affiliation{\SCAROLINA}
\author {Y.~Gotra} 
\affiliation{\JLAB}
\author {K.~Griffioen}
\affiliation{\WM}
\author {K.~Hafidi} 
\affiliation{\ANL}
\author {H.~Hakobyan} 
\affiliation{\UTFSM}
\author {M.~Hattawy} 
\affiliation{\ODU}
\affiliation{\ANL}
\author {T.B.~Hayward} 
\affiliation{\UCONN}
\author {D.~Heddle} 
\affiliation{\CNU}
\affiliation{\JLAB}
\author {A.~Hobart} 
\affiliation{\ORSAY}
\author {M.~Holtrop} 
\affiliation{\UNH}
\author {I.~Illari}
\affiliation{\GWUI}
\author {D.G.~Ireland} 
\affiliation{\GLASGOW}
\author {E.L.~Isupov} 
\affiliation{\MSU}
\author {H.S.~Jo} 
\affiliation{\KNU}
\author {R.~Johnston} 
\affiliation{\MIT}
\author {D.~Keller} 
\affiliation{\VIRGINIA}
\author {M.~Khachatryan} 
\affiliation{\ODU}
\author {A.~Khanal} 
\affiliation{\FIU}
\author {A.~Kim} 
\affiliation{\UCONN}
\author {W.~Kim} 
\affiliation{\KNU}
\author {V.~Klimenko} 
\affiliation{\UCONN}
\author {A.~Kripko} 
\affiliation{\JLUGiessen}
\author {V.~Kubarovsky} 
\affiliation{\JLAB}
\author {S.E.~Kuhn} 
\affiliation{\ODU}
\author {V.~Lagerquist} 
\affiliation{\ODU}
\author {L. Lanza} 
\affiliation{\INFNRO}
\affiliation{\ROMAII}
\author {M.~Leali} 
\affiliation{\BRESCIA}
\affiliation{\INFNPAV}
\author {S.~Lee} 
\affiliation{\ANL}
\author {P.~Lenisa} 
\affiliation{\INFNFE}
\affiliation{\FERRARAU}
\author {X.~Li} 
\affiliation{\MIT}
\author {I .J .D.~MacGregor} 
\affiliation{\GLASGOW}
\author {D.~Marchand} 
\affiliation{\ORSAY}
\author {V.~Mascagna} 
\affiliation{\BRESCIA}
\affiliation{\INSUBRIA}
\affiliation{\INFNPAV}
\author {G.~Matousek}
\affiliation{\DUKE}
\author {B.~McKinnon} 
\affiliation{\GLASGOW}
\author {C.~McLauchlin} 
\affiliation{\SCAROLINA}
\author {Z.E.~Meziani} 
\affiliation{\ANL}
\affiliation{\TEMPLE}
\author {S.~Migliorati} 
\affiliation{\BRESCIA}
\affiliation{\INFNPAV}
\author {R.G.~Milner} 
\affiliation{\MIT}
\author {T.~Mineeva} 
\affiliation{\UTFSM}
\author {M.~Mirazita} 
\affiliation{\INFNFR}
\author {V.~Mokeev} 
\affiliation{\JLAB}
\author {P.~Moran} 
\affiliation{\MIT}
\author {C.~Munoz~Camacho} 
\affiliation{\ORSAY}
\author {P.~Naidoo} 
\affiliation{\GLASGOW}
\author {K.~Neupane} 
\affiliation{\SCAROLINA}
\author {S.~Niccolai} 
\affiliation{\ORSAY}
\author {G.~Niculescu} 
\affiliation{\JMU}
\author {M.~Osipenko} 
\affiliation{\INFNGE}
\author {P.~Pandey} 
\affiliation{\ODU}
\author {M.~Paolone} 
\affiliation{\NMSU}
\affiliation{\TEMPLE}
\author {L.L.~Pappalardo} 
\affiliation{\INFNFE}
\affiliation{\FERRARAU}
\author {R.~Paremuzyan} 
\affiliation{\JLAB}
\affiliation{\UNH}
\author {S.J.~Paul} 
\affiliation{\UCR}
\author {W.~Phelps} 
\affiliation{\CNU}
\affiliation{\GWUI}
\author {N.~Pilleux} 
\affiliation{\ORSAY}
\author {M.~Pokhrel} 
\affiliation{\ODU}
\author {J.~Poudel} 
\altaffiliation[Current address:]{\NOWJLAB}
\affiliation{\ODU}
\author {J.W.~Price} 
\affiliation{\CSUDH}
\author {Y.~Prok} 
\affiliation{\ODU}
\author {A. Radic} 
\affiliation{\UTFSM}
\author {B.A.~Raue} 
\affiliation{\FIU}
\author {T.~Reed} 
\affiliation{\FIU}
\author {J.~Richards} 
\affiliation{\UCONN}
\author {M.~Ripani} 
\affiliation{\INFNGE}
\author {J.~Ritman} 
\affiliation{\GSIFFN}
\affiliation{\Juelich}
\author {P.~Rossi} 
\affiliation{\JLAB}
\affiliation{\INFNFR}
\author {F.~Sabati\'e} 
\affiliation{\SACLAY}
\author {C.~Salgado} 
\affiliation{\NSU}
\author {S.~Schadmand} 
\affiliation{\GSIFFN}
\author {A.~Schmidt} 
\affiliation{\GWUI}
\affiliation{\MIT}
\author {Y.G.~Sharabian} 
\affiliation{\JLAB}
\author {U.~Shrestha} 
\affiliation{\UCONN}
\affiliation{\OHIOU}
\author {D.~Sokhan} 
\affiliation{\SACLAY}
\affiliation{\GLASGOW}
\author {N.~Sparveris} 
\affiliation{\TEMPLE}
\author {M.~Spreafico} 
\affiliation{\INFNGE}
\author {S.~Stepanyan} 
\affiliation{\JLAB}
\author {I.~Strakovsky}
\affiliation{\GWUI}
\author {S.~Strauch} 
\affiliation{\SCAROLINA}
\author {M.~Turisini} 
\affiliation{\INFNFR}
\author {R.~Tyson} 
\affiliation{\GLASGOW}
\author {M.~Ungaro} 
\affiliation{\JLAB}
\author {S.~Vallarino} 
\affiliation{\INFNFE}
\author {L.~Venturelli} 
\affiliation{\BRESCIA}
\affiliation{\INFNPAV}
\author {H.~Voskanyan} 
\affiliation{\YEREVAN}
\author {E.~Voutier} 
\affiliation{\ORSAY}
\author {D.P.~Watts}
\affiliation{\YORK}
\author {X.~Wei} 
\affiliation{\JLAB}
\author {R.~Williams} 
\affiliation{\YORK}
\author {R.~Wishart} 
\affiliation{\GLASGOW}
\author {M.H.~Wood} 
\affiliation{\CANISIUS}
\author {M.~Yurov}
\affiliation{\MISS}
\author {N.~Zachariou} 
\affiliation{\YORK}
\author {Z.W.~Zhao} 
\affiliation{\DUKE}
\affiliation{\ODU}
\author {M.~Zurek} 
\affiliation{\ANL}

\collaboration{The CLAS Collaboration}
\noaffiliation

 






\date{\today}

%%%%%%摘要
\begin{abstract}
%\linenumbers
Using $e^+e^-$ collision data corresponding to an integrated luminosity of 7.33~${\rm fb^{-1}}$ recorded by the BESIII detector
at center-of-mass energies between 4.128 and 4.226~${\rm GeV}$,
we present an analysis of the decay $D_{s}^{+} \to f_{0}(980)e^{+}\nu_{e}$ with $f_{0}(980) \to \pi^{+}\pi^{-}$, where the $D_s^+$ is produced via the process $e^+e^- \to D_{s}^{*\pm}D_{s}^{\mp}$.
We observe the $f_{0}(980)$ in the $\pi^+\pi^-$ system and the branching fraction of 
the decay $D_{s}^{+} \to f_{0}(980)e^{+}\nu_{e}$ with $f_0(980)\to\pi^+\pi^-$ is measured to be $(1.72 \pm 0.13_{\rm stat} \pm 0.10_{\rm syst}) \times10^{-3}$,
where the uncertainties are statistical and systematic, respectively.
The dynamics of the $D_{s}^{+} \to f_{0}(980)e^{+}\nu_{e}$ decay are studied with the simple pole parameterization of the hadronic form factor
and the Flatt\'e formula describing the $f_0(980)$ in the differential decay rate,
and the product of the form factor $f^{f_0}_{+}(0)$ and the $c\to s$ Cabibbo-Kobayashi-Maskawa matrix element $|V_{cs}|$
is determined for the first time to be $f^{f_0}_+(0)|V_{cs}|=0.504\pm0.017_{\rm stat}\pm0.035_{\rm syst}$.
%\\
%\vspace{0.95cm}
\end{abstract}
\maketitle
%\linenumbers

%%%%%%Introduction

%%%%%%%%%%%%%%%%%%%%shulei%%%%%%%%%%%%%%%%%

Quantum Chromodynamics (QCD), the fundamental theory of the Strong Interaction, has been established for almost half a century.
However, there are some features that still need to be understood,
such as quark confinement and dynamics in the non-perturbative regime. 
The light scalar mesons $f_0(500)$, $f_0(980)$, and $a_0(980)$ play a crucial role in the dynamics of the spontaneous breaking of QCD chiral symmetry and in the origin of pseudoscalar meson masses~\cite{Pelaez2016,weiwang2010},
and consequently can be used to probe the confinement of quarks~\cite{Jaffe1977}.
Furthermore, our understanding of the nature of light hadrons is still poor since QCD is non-perturbative
in the low-energy region.
Investigating the structure of the light scalar mesons provides key input to these issues.
In spite of the striking success of the constituent quark model,
the nontrivial quark structure of these mesons has remained controversial for many years~\cite{pdg}.
Their mass ordering cannot be explained by a $q\bar{q}$ configuration in the naive quark model,
leaving open the possibility that they are mixtures of $q\bar{q}$ 
states~\cite{Jaffe1977,qqbar1,qqbar2,QCDSR1,QCDSR2,LCSR1,LCSR2,LF-RQM,Soni2020}.
Other interpretations are diquark-antidiquark states (tetraquark)~\cite{tetraquark} 
and meson-meson bound states (molecule)~\cite{molecule}.
Therefore, more conclusive experimental measurements of these scalar states are highly desired.

Since the leptons and hadrons in the final state interact only weakly with each other,
the semileptonic (SL) decays of charm mesons provide a unique and clean platform to probe
the constituent $q\bar{q}$ components in the wave functions of light scalar states~\cite{Achasov2012}. 
Here, only the spectator light quarks are related to the formation of these states and the quark flavor content can be specified through Cabibbo-favored and -suppressed processes~\cite{Oset2015}.
Additionally, the dynamics of the SL charmed meson decays can be studied by measuring the hadronic form factor (FF) 
that describes the strong interaction between the final-state quarks, including all the non-perturbative effects.
This provides an excellent opportunity to test the different theoretical methods of solving the QCD non-perturbative problem.
Since the FFs and branching fractions (BFs) of the SL charmed meson decays are highly sensitive to the internal structure of light scalar states, 
studies of the dynamics of these decays are also important to understand their nature~\cite{Soni2020}.

In previous studies, the BESIII collaboration has reported measurements of
the decays $D^{0(+)}\to a_{0}(980)^{-(0)}e^{+}\nu_{e}$~\cite{bes3-D2a0enu}, $D^{+}\to f_{0}(500) e^{+}\nu_{e}$~\cite{bes3-D2f0enu},
and $D_{s}^{+}\to f_{0}(980)e^{+}\nu_{e}, f_{0}(980)\to \pi^{0}\pi^{0}$~\cite{bes3-Ds2f0enu},
and searches for the decays $D^{+}\to f_{0}(980) e^{+}\nu_{e}$ and $D_{s}^{+}\to a_{0}(980)^{0}e^{+}\nu_{e}$~\cite{bes3-D2f0enu,bes3-Ds2a0enu}.
With negligible contamination from the $D_{s}^{+} \to \rho^{0} e^+ \nu_{e}$ channel, the decay $D_{s}^+\to \pi^{+}\pi^{-}e^{+}\nu_{e}$
enables us to study the structure of $f_{0}(980)$ in a clean environment.
Previously, only the CLEO collaboration measured the BF of the decay
$D_{s}^{+}\to f_{0}(980)e^{+}\nu_{e}, f_{0}(980)\to \pi^{+}\pi^{-}$ 
\cite{cleo2009-Ds2semilep,cleo2009-Ds2f0enu,cleo2015-Ds2semilep} with data taken at a center-of-mass (CM) energy ($E_{\rm CM}$) near 4.170 GeV.
With a data sample more than 10 times larger, 
we report a significantly improved measurement of the BF and the first measurement of the transition FF.
The obtained results are important tests of theoretical predictions based on different models~\cite{qqbar2,QCDSR1,QCDSR2,LCSR1,LCSR2,LF-RQM,Soni2020}. 
Throughout this letter, charge-conjugate channels are implied.
%%%%%%%%%%%%%%%%%%%%%%%%%%%%%%%%%%%%%%%%%%%

%%%%%%Method, Data and MC sample
For the BF measurements of SL decays, we use the same double-tag technique of Refs~\cite{bes3-Ds2pnbar,bes3-Ds2a0enu,bes3-Ds2f0enu}.
Our measurements are performed based on $e^+e^-$ collision data corresponding to an integrated luminosity of 7.33~fb$^{-1}$
collected with the BESIII detector at $E_{\rm CM}=4.128 - 4.226$~GeV~\cite{material}.
Details about the BESIII detector design and performance are provided in Refs.~\cite{BESIII1,BESIII2,BESIII3}. 

Simulated data samples produced with a {\sc geant4}-based~\cite{GEANT4} Monte Carlo (MC) package,
which includes the geometric description of the BESIII detector~\cite{BESIII4} and the detector response,
are used to determine detection efficiencies and to estimate backgrounds.
The simulation models the beam energy spread and initial state radiation (ISR) in the $e^+e^-$ annihilations with the generator {\sc kkmc}~\cite{KKMC}.
The inclusive MC sample includes the production of open-charm processes, the ISR production of vector charmonium(-like) states,
and the continuum processes incorporated in {\sc kkmc}~\cite{KKMC}.
All particle decays are modelled with {\sc evtgen}~\cite{EVTGEN} using BFs either taken from the Particle Data Group~\cite{pdg},
when available, or otherwise estimated with {\sc lundcharm}~\cite{LUNDCHARM}.
Final state radiation from charged final state particles is incorporated using the {\sc photos} package~\cite{PHOTOS}.
The signal detection efficiencies and signal shapes are obtained from the signal MC samples,
in which the $D_{s}^{-}$ decays inclusively to all known decay channels and the signal $D_{s}^{+}$ decays to $\pi^{+}\pi^{-}e^{+}\nu_{e}$
with the $S$-wave contribution simulated according to previous measurements~\cite{bes3-Dp2kpienu,bes3-D2f0enu}.
The amplitudes for the $f_{0}(980)$ is modeled by the Flatt\'e formula 
with its parameters fixed to the BESII measurement~\cite{bes2f980}.  
%%%%%%%%%%%%%%%%%%%%%%%%%%%%%%%%%%%%%%%%%%%

%%%%%%Tag Selection
The tag $D^-_s$ candidates are reconstructed with $K^{\pm}$, $\pi^{\pm}$, $\rho^{-}$, $\rho^{0}$, $\pi^{0}$, $\eta^{(')}$, and $K_{S}^{0}$ mesons in twelve tag modes:
$K^{+}K^{-}\pi^{-}$,
$K^0_{S}K^{-}$,
$\pi^{-}\eta$,
$\pi^{-}\eta^{\prime}_{\pi^{+}\pi^{-}\eta}$,
$K^{+}K^{-}\pi^{-}\pi^{0}$,
$\pi^{+}\pi^{-}\pi^{-}$,
$K^0_{S}K^{+}\pi^-\pi^-$,
$\rho^{-}\eta$,
$\pi^{-}\eta^{\prime}_{\gamma\rho^0}$,
$K^{+}\pi^{-}\pi^{-}$,
$K^{0}_{S}K^{-}\pi^{0}$,
and $K^0_{S}K^{-}\pi^+\pi^-$.
A detailed description of the selection criteria for all tag candidates except $\rho^{-}\eta$
can be found in Ref~\cite{Suntong}. 
The $\rho^-$ candidates are reconstructed from $\pi^-\pi^0$ combinations within an invariant mass interval (0.625, 0.925) GeV/$c^2$.
Requirements on the recoiling mass ${m_{\rm rec}}$ against the tag $D_{s}^{-}$ candidates are applied to the tag candidates
in order to identify the process $e^+e^- \to D_{s}^{*\pm}D_{s}^{\mp}$.
If there are multiple candidates for a specific tag mode per charge, 
the one with ${m_{\rm rec}}$ closest to the known $D_{s}^{*\pm}$ mass~\cite{pdg} is chosen.
For each tag mode, the tag yield is extracted from the fit to the tag $D_{s}^{-}$ mass spectrum ($M_{D_{s}^{-}}$).
The signals are modeled with the MC-simulated signal shape convolved with a Gaussian function
to account for the resolution difference between data and MC simulation,
while the combinatorial backgrounds are parameterized with a first-order or second-order Chebyshev polynomial.
For the tag mode $D^-_s\to K^0_SK^-$, the peaking background from $D^-\to K^0_S\pi^-$ decay is described by the MC-simulated shape that is smeared
with the same Gaussian function as used in the signal, with the background yield determined from the fit.
Summing over various tag modes and energy points, we obtain the total tag yield $N^{\rm tot}_{\rm tag}=771101\pm3445$.
For more details about tag candidates, such as selection regions and reconstruction efficiencies, see Ref.~\cite{material}.

%%%%%% Signal Selection and Analysis
After a $D_{s}^{-}$ candidate is identified, we reconstruct the decay $D_{s}^{+}\to \pi^+\pi^- e^{+}\nu_{e}$ recoiling against the tag side,
requiring three charged tracks identified as a $\pi^+\pi^-$ pair with the same selection criteria as on the tag side 
and $e^+$ (opposite sign to the tag $D_{s}^{-}$) following Ref.~\cite{bes3-Dp2omegaenu}.
Using the same kinematic fit method of Refs~\cite{bes3-Ds2pnbar,bes3-Ds2a0enu,bes3-Ds2f0enu},
we reconstruct the transition photon from the main decay $D_{s}^{*\pm}\to\gamma D_{s}^{\pm}$. 

For the real $D^{*\pm}_sD^{\mp}_s$ events, the square of the recoiling mass (${M^2_{\rm rec}}$) against the transition photon
and the tag $D_{s}^{-}$ is expected to peak at the known $D_{s}^{+}$ mass squared.
To improve the resolution, the decay products of the tag $D^-_s$ are constrained to the known $D^{+}_s$ mass~\cite{pdg}.
We require ${M^2_{\rm rec}}$ to be within (3.78, 4.05) $\gevgevcccc$
to suppress the backgrounds from non-$D_{s}^{*\pm}D_{s}^{\mp}$ processes.
The missing neutrino information is inferred by the missing mass squared which is defined as
\begin{equation}
M^{2}_{\rm miss}=(\bm{p}_{\rm CM}-\bm{p}_{\rm tag}-\bm{p}_{\pi^+}-\bm{p}_{\pi^-}-\bm{p}_{e}-\bm{p}_{\gamma})^2,
\label{eq:MM2}
\end{equation}
where $\bm{p}_{\rm CM}$ is the four-momentum of the $e^+e^-$ center-of-mass system, $\bm{p}_{\rm tag}$ for the tag $D_{s}^{-}$, $\bm{p}_{\pi^+(\pi^-,e)}$ for the SL final state,
and $\bm{p}_{\gamma}$ for the transition photon from the $D_{s}^{*\pm}$ decay.
Here, the measured momenta of the tag $D^-_s$ and the transition photon are corrected with the kinematic fit to improve the resolution.
In order to further reject backgrounds, we require $|M^{2}_{\rm miss}|< {\rm 0.06~GeV^{2}}/c^{4}$.

To study the $f_{0}(980)$, we require the $\pi^+\pi^-$ invariant mass ($M_{\pi^+\pi^-}$, see Fig.~\ref{fit:hadMass-data}) to be within the interval (0.6, 1.6) ${\rm GeV}/c^{2}$.
The weighted signal efficiency is (35.44$\pm$0.07)\%, 
which is estimated as $\sum_{i}[(N_{\rm tag}^{i}/{N_{\rm tag}^{\rm tot}})\times(\epsilon_{\rm tag}^{i}/\epsilon_{\rm tag, sig}^{i})]$,
where $N_{\rm tag}^{\rm tot}$ is the total tag yield, 
and $\epsilon_{\rm tag}^{i}$ and $\epsilon_{\rm tag, sig}^{i}$ are the tag and SL efficiencies for the $i$-th tag mode, respectively.
The non-peaking background distribution from the inclusive MC sample is verified using events from the data sideband region 
(about 2$\sigma$ away from the signal region of the tag $D_{s}^{-}$ mass and having the same interval as signal region)
of the tag $M_{D_{s}^{-}}$ distribution.
The peak around 0.75~\gevcc is mainly caused by the decay $D^+_s\to\eta^{\prime}(\gamma\pi^+\pi^-)e^+\nu_e$.
An unbinned maximum likelihood fit to the $M_{\pi^+\pi^-}$ distribution is performed to extract the SL signal yield of $D_{s}^{+}\to f_{0}(980)e^{+}\nu_{e},~f_{0}(980)\to \pi^{+}\pi^{-}$ decay.
In the fit, the signal is modeled with an MC-simulated lineshape convolved with a Gaussian resolution function,
and the background is described by the inclusive MC shape convolved with the same Gaussian function as the signal.
From the fit, which is shown in Fig.~\ref{fit:hadMass-data}, 
we obtain $439\pm33$ signal events.
Using the tag and SL efficiencies provided in Ref.~\cite{material}, 
we obtain $\mathcal{B}(D^+_s\to f_0(980)e^{+}\nu_{e}, f_0(980)\to\pi^+\pi^-) = (1.72\pm0.13_{\rm stat}\pm0.10_{\rm syst})\times 10^{-3}$,
where the uncertainties are statistical and systematic, respectively.
 
\begin{figure}[htp]
\begin{center}
\includegraphics[width=0.455\textwidth]{Ds2f0enu_hadMass_data.eps}
\caption{ Fit to the $M_{\pi^+\pi^-}$ distribution of the accepted candidates for the decay $D_{s}^{+} \to f_0(980) e^{+}\nu_{e}$.
The points with error bars are data, and the blue line is the total fit.
The red dotted and violet dashed lines are the signal and background shapes, respectively.
}
\label{fit:hadMass-data}
\end{center}
\end{figure}

The systematic uncertainties of the tracking or particle identification (PID) efficiencies of $\pi^{\pm}$ and $e^{+}$ are studied 
with control samples of $e^+e^- \to K^+K^-\pi^{+}\pi^{-}$ and $e^+e^- \to \gamma e^+e^-$ processes.
For $e^{+}$ both the tracking and PID uncertainties are assigned to be 0.5\%.
For $\pi^{\pm}$ both uncertainties are also assigned to be 0.5\% per track, or 1.0\% for the $\pi^{+}\pi^{-}$ pair.
The uncertainty from the quoted BF of $D^{*\pm}_s\to\gamma D^{\pm}_s$ decay is 0.7\%~\cite{pdg}.
The uncertainty due to the transition photon reconstruction is estimated to be 2.0\% using
the control sample of $e^{+}e^{-}\to D_{s}^{*\pm}D_{s}^{\mp}$ events, where $D_{s}^{-}$ decays via a tag mode,
while $D_{s}^{+}$ decays via one of the two hadronic channels: $D_{s}^{+}\to K_{S}^{0}K^{+}$ or $D_{s}^{+}\to K^{+}K^{-}\pi^{+}$.
The uncertainty in the total number of the tag $D_{s}^{-}$ mesons is assigned to be 0.3\% by 
examining the changes of the fit yields when varying the signal shape, background shape, and taking into account the background fluctuation in the fit.
The uncertainty associated with the signal MC model is estimated to be 4.4\% by replacing the $f_0(980)$ lineshape
from BESII~\cite{bes2f980} with the one from LHCb~\cite{LHCb} in generating the signal MC samples.
The uncertainties of the $M_{\pi^+\pi^-}$ fit is estimated to be 2.1\% by altering the nominal MC background shape.
Firstly, we use alternative MC shapes where the relative fractions of backgrounds from continuum and non-$D_{s}^{*\pm} D_{s}^{\mp}$ open-charm processes are varied by $\pm30$\%
according to the uncertainties of their assigned cross sections in the inclusive MC sample.
Secondly, we vary the relative fraction ($\pm 1\sigma$) of the peaking background channel $D^+_s\to\eta^{\prime}(\gamma\pi^+\pi^-)e^+\nu_e$~\cite{pdg}.
The total systematic uncertainty is 5.6\%, obtained by adding all contributions in quadrature.

%%Form factor
The dynamics of $D_{s}^{+}\to f_{0}(980)e^{+}\nu_{e}$ decay is studied by dividing the SL candidate events into four intervals of $q^2$ 
(four-momentum transfer square of $e^{+}\nu_{e}$).
Using the measured and expected partial decay rates of the $i$-th $q^{2}$ interval,
$\Delta\Gamma^{i}_{\rm mea}$ and $\Delta\Gamma^{i}_{\rm exp}$, the FF is determined by constructing and minimizing a $\chi^2$ as 
\begin{equation}
\chi^{2}=\sum_{ij} (\Delta\Gamma^{i}_{\rm mea} - \Delta\Gamma^{i}_{\rm exp})C_{ij}^{-1}(\Delta\Gamma^{j}_{\rm mea} - \Delta\Gamma^{j}_{\rm exp}),
\label{eq:chi}
\end{equation}
where $C_{ij}$ is the covariance matrix to consider correlations of $\Delta\Gamma^{i}_{\rm mea}$ among $q^2$ intervals.

The $\Delta\Gamma^{i}_{\rm exp}$ is calculated by integrating the following double differential decay rate~\cite{FF-rate-sq}
%\begin{equation}
%\begin{eqnarray}
\begin{align}
\frac{d^{2}\Gamma(D^+_s\to f_0(980)e^+\nu_e)}{dsdq^2}=&\frac{G^2_F|V_{cs}|^{2}}{192\pi^{4}m_{D^+_s}^{3}}\lambda^{3/2}(m_{D^+_s}^{2},s,q^2) \nonumber \\
& \times |f^{f_0}_+(q^2)|^2 P(s),
\label{eq:rate}
\end{align}
%\end{eqnarray}
%\end{equation}
where $s$ is the square of $M_{\pi^+\pi^-}$, $G_F$ is the Fermi constant~\cite{pdg}, 
$|V_{cs}|$ is the Cabibbo Kobayashi-Maskawa matrix element, $m_{D^+_s}$ is the known $D^+_s$ mass~\cite{pdg},
$\lambda(x,y,z)=x^2+y^2+z^2-2xy-2xz-2yz$,
and $P(s)$ is based on the relativistic Flatt\'e formula~\cite{bes2f980} due to the open $K^+K^-$ channel in data as follows:
\begin{equation}
P(s)=\frac{g_1\rho_{\pi\pi}}{|m^2_0-s-i(g_1\rho_{\pi\pi} + g_2\rho_{K\bar{K}})|^2}.
\label{eq:Ps}
\end{equation}
Here $m_0$ denotes the $f_0(980)$ mass;
the constants $g_1$ and $g_2$ are the $f_0(980)$ couplings to $\pi^+\pi^-$ and $K^+K^-$ final states, respectively;
and $\rho_{\pi\pi}$ and $\rho_{K\bar{K}}$ are individual phase space factors.
Using the decay widths in the different $q^2$ intervals, the FF $|f^{f_0}_+(q^2)|$ can be extracted.
In this work, the FF is modeled with the simple pole parameterization~\cite{FF-simple}:
\begin{equation}
f^{f_0}_+(q^2)=\frac{f^{f_0}_+(0)}{1-q^2/M^2_{\rm pole}},
\label{eq:SPD}
\end{equation}
where $f^{f_0}_+(0)$ is the FF evaluated at $q^2=0$, and the pole mass $M_{\rm pole}={\rm 2.46~GeV}/c^2$~\cite{pdg,ref-Ds1}.

The measured partial decay rate $\Delta\Gamma_{\rm mea}^{i}$ is determined by 
$\Delta\Gamma_{\rm mea}^{i} \equiv \int_{i}\int_{s} \frac{d^{2}\Gamma}{dsdq^{2}}dsdq^{2}=N^i_{\rm pro}/(\tau N^{\rm tot}_{\rm tag}{\cal B}_{\gamma})$,
where ${\cal B}_{\gamma}$ represents the BF of $D_{s}^{*\pm}\to\gamma D_{s}^{\pm}$,
$\tau$ is the $D_{s}^{+}$ meson lifetime~\cite{pdg,lifetime} and $N_{\rm pro}^{i}$ is the SL signal yield produced in the $i$-th $q^2$ interval,
obtained as $N_{\rm pro}^{i}=\sum_{j=1}^{4} \epsilon^{-1}_{ij} N_{\rm obs}^{j}$.
Here $N_{\rm obs}^{j}$ is the observed SL decay yield obtained from the similar fit to the corresponding $M_{\pi^{+}\pi^{-}}$ distribution as described previously,
and $\epsilon_{ij}$ is the efficiency matrix determined from the signal MC samples via 
$\epsilon_{ij}=\sum_{k}[(1/N_{\rm tag}^{\rm tot}) \times (N_{\rm rec}^{ij}/N_{\rm gen}^{j})_{k} \times (N_{\rm tag}^{k}/\epsilon_{\rm tag}^{k})]$,
where $N_{\rm rec}^{ij}$ is the SL decay yield   reconstructed in the $i$-th $q^2$ interval and generated in the $j$-th $q^2$ interval,
$N_{\rm gen}^{j}$ is the total signal yield generated in the $j$-th $q^2$ interval, and $k$ sums over all tag modes.
The details of the divisions, $N_{\rm obs}^{i}$ and $\Delta\Gamma_{\rm mea}^{i}$ of various $q^2$ intervals are given in Ref.~\cite{material}.

The statistical and systematic covariance matrices are constructed as 
$C_{ij}^{\rm stat}=(1/\tau N_{\rm tag}^{\rm tot})^{2}\sum_{\alpha}\epsilon_{i\alpha}^{-1}\epsilon_{j\alpha}^{-1}[\sigma(N^{\alpha}_{\rm obs})]^{2}$ 
and $C_{ij}^{\rm syst}=\delta(\Delta\Gamma_{\rm mea}^{i})\delta(\Delta\Gamma_{\rm mea}^{j})$, respectively,
where $\sigma(N^{\alpha}_{\rm obs})$ and $\delta(\Delta\Gamma_{\rm mea}^{i})$ are the statistical and systematic uncertainties in the $i$-th $q^2$ interval.
The $C_{ij}^{\rm syst}$ is obtained by summing all the covariance matrices for all systematic uncertainties, 
where the systematic uncertainty of $\tau$, 0.8\%~\cite{pdg,lifetime}, 
is involved besides those in the BF measurement.
The obtained $C_{ij}^{\rm stat}$ and $C_{ij}^{\rm syst}$ are shown in Ref.~\cite{material}.
The resulting $C_{ij}$ is obtained as $C_{ij}=C_{ij}^{\rm stat}+C_{ij}^{\rm syst}$.

The statistical and systematic uncertainties related to $C_{ij}$ and $\Delta\Gamma_{\rm mea}^{i}$
are estimated to be 3.4\% and 2.6\% by following Ref~\cite{bes3-D02kenu}.   
Besides, the input parameters $m_0$, $g_1$, and $g_2$~\cite{bes2f980} related to $\Delta\Gamma^{i}_{\rm exp}$
are also considered by varying them within $\pm 1 \sigma$ from their central values.
The largest deviations of the FF, respectively 2.2\%, 1.2\% and 6.0\%, are taken as systematic uncertainties.
The quadrature sum of the above uncertainties is 6.9\%, which is taken as the total systematic uncertainty. 

The fit to the differential decay rate of the channel $D_{s}^{+}\to f_{0}(980)e^{+}\nu_{e}$ and the FF projection are shown in Fig.~\ref{fit:ff-width}.
Using the FF parameterization of Eq.~\ref{eq:rate} and the Flatt\'e formula Eq.~\ref{eq:Ps} for the $f_0(980)$ decay in the fit,
the product of the FF and $|V_{cs}|$ is determined to be $f^{f_0}_+(0)|V_{cs}|=0.504\pm0.017_{\rm stat}\pm0.035_{\rm syst}$.
The fit result is shown in Fig.~\ref{fit:ff-width} (a), while Fig.~\ref{fit:ff-width} (b) shows the same fit in projection to the FF $f^{f_{0}}_{+}(q^2)$.
The goodness of fit $\chi^2/{\rm NDF}$ is 0.8,
where ${\rm NDF}$ is the number of degrees of freedom.

\begin{figure}[htp]
\begin{center}
\includegraphics[width=0.485\textwidth]{Ds2f0enu_ff_data_noEnd.eps}
\caption{ Fit to the differential decay rate as a function of $q^2$ (a) and projection to the FF $f^{f_{0}}_{+}(q^2)$ (b). 
The points with error bars are data, and the red line is the fit.}
\label{fit:ff-width}
\end{center}
\end{figure}

%%%%%%In summary
In summary, using $e^+e^-$ collision data corresponding to an integrated luminosity of 
7.33${~\rm fb^{-1}}$ collected at $E_{\rm CM}= {\rm 4.128 - 4.226~GeV}$ by the BESIII detector, 
we measure the BF of $D^+_s\to f_0(980)e^{+}\nu_{e},~f_0(980)\to\pi^+\pi^-$ decay to be $(1.72\pm0.13_{\rm stat}\pm0.10_{\rm syst})\times 10^{-3}$,
which is 2.6 times more accurate than the previous measurement~\cite{cleo2015-Ds2semilep}.
Using the relation between the BF and the mixing angle $\phi$ involved in the $q\bar{q}$ mixture picture for $f_0(980)$ as ${\rm sin}\phi\frac{1}{\sqrt{2}}(u\bar{u}+d\bar{d})+{\rm cos}\phi s\bar{s}$~\cite{QCDSR2,LF-RQM}, we find that the $s\bar{s}$ component is dominant. 
This conclusion disagrees with that in Ref.~\cite{LF-RQM} where their calculation is based on the CLEO measurement~\cite{cleo2009-Ds2semilep}.

Furthermore, we determine $f^{f_0}_+(0)|V_{cs}|=0.504\pm0.017_{\rm stat}\pm0.035_{\rm syst}$ for the first time
by analyzing the dynamics of $D^+_s\to f_0(980)e^{+}\nu_{e},~f_0(980)\to\pi^+\pi^-$ decay.
Using $|V_{cs}|=0.97349\pm0.00016$~\cite{pdg}, we obtain $f^{f_0}_+(0)=0.518\pm0.018_{\rm stat}\pm0.036_{\rm syst}$.
In Table~\ref{comFF} the measured FF result at $q^2=0$ is compared with different theoretical predictions.
Our measurement agrees with the theoretical results in Refs.~\cite{qqbar2,QCDSR1,QCDSR2},
but is much higher than the theoretical results in Refs.~\cite{LCSR1,LF-RQM,Soni2020}.
It is notable that most predicted FF $f^{f_0}_{+}(0)$ and BF depend on the angle $\phi$ known with large uncertainty.
So, the measured FF and BF are both important to constrain this angle and probe the quark component in $f_0(980)$~\cite{LF-RQM}.
Although most theoretical predictions for $f^{f_0}_{+}(0)$ 
have a large uncertainty due to the $\phi$ uncertainty, the measured FF lineshape is a powerful tool to distinguish different models. 
These results are important to understand the nature of the light scalar states $f_0(980)$ and the non-perturbative dynamics of charm meson decays.

%%%%%%Acknowledge
The BESIII Collaboration thanks the staff of BEPCII and the IHEP computing center for their strong support. 
The authors are grateful to De-Liang Yao, Shan Cheng and Xian-Wei Kang for valuable discussions.
This work is supported in part by National Key R\&D Program of China under Contracts Nos. 2020YFA0406300, 2020YFA0406400; National Natural Science Foundation of China (NSFC) under Contracts Nos. 11635010, 11735014, 11835012, 11935015, 11935016, 11935018, 11961141012, 12022510, 12025502, 12035009, 12035013, 12061131003, 12192260, 12192261, 12192262, 12192263, 12192264, 12192265; 
Natural Science Foundation of Hunan Province, China under Contract No.~2021JJ40036
and the Fundamental Research Funds for the Central Universities under Contract No. 020400/531118010467;
the Chinese Academy of Sciences (CAS) Large-Scale Scientific Facility Program; the CAS Center for Excellence in Particle Physics (CCEPP); Joint Large-Scale Scientific Facility Funds of the NSFC and CAS under Contract No. U1832207; CAS Key Research Program of Frontier Sciences under Contracts Nos. QYZDJ-SSW-SLH003, QYZDJ-SSW-SLH040; 100 Talents Program of CAS; The Institute of Nuclear and Particle Physics (INPAC) and Shanghai Key Laboratory for Particle Physics and Cosmology; ERC under Contract No. 758462; European Union's Horizon 2020 research and innovation programme under Marie Sklodowska-Curie grant agreement under Contract No. 894790; German Research Foundation DFG under Contracts Nos. 443159800, 455635585, Collaborative Research Center CRC 1044, FOR5327, GRK 2149; Istituto Nazionale di Fisica Nucleare, Italy; Ministry of Development of Turkey under Contract No. DPT2006K-120470; National Research Foundation of Korea under Contract No. NRF-2022R1A2C1092335; National Science and Technology fund of Mongolia; National Science Research and Innovation Fund (NSRF) via the Program Management Unit for Human Resources \& Institutional Development, Research and Innovation of Thailand under Contract No. B16F640076; Polish National Science Centre under Contract No. 2019/35/O/ST2/02907; The Royal Society, UK under Contract No. DH160214; The Swedish Research Council; U. S. Department of Energy under Contract No. DE-FG02-05ER41374.

\onecolumngrid

\begin{table}[!htp]
\centering
\caption{Comparison of the FF at $q^2=0$ between our measurement and various theoretical predictions.
}
\vspace{0.25cm}
%\resizebox{\textwidth}{!}
%{
\scalebox{0.90}{
\begin{tabular}{c|c|c|c|c|c|c|c|c}
\hline
\hline
                     & This work     &CLFD~\cite{qqbar2}  &DR~\cite{qqbar2}  &QCDSR~\cite{QCDSR1} &QCDSR~\cite{QCDSR2} &LCSR~\cite{LCSR1} &LFQM~\cite{LF-RQM} &CCQM~\cite{Soni2020} \\
\hline
$f^{f_0}_{+}(0)$     & $ 0.518\pm0.018_{\rm stat}\pm0.036_{\rm syst} $ &0.45  &0.46  &$0.50\pm0.13$ & $0.48\pm0.23$ &$0.30\pm0.03$ & $0.24\pm0.05$ & $ 0.39\pm0.02 $ \\
\hline
Difference ($\sigma$)& --- & ---  & ---  &0.1           &0.2            & 4.3          & 4.3            & 2.8 \\
\hline
$\phi$ in theory     & --- & $(32\pm4.8)^\circ$  & $(41.3\pm5.5)^\circ$ &$35^\circ$ & $(8^{+21}_{-8})^\circ$ & --- &  $(56\pm7)^\circ$ & $31^\circ$ \\
\hline
\hline
\end{tabular}
\label{comFF}
}
\end{table}
\twocolumngrid

%%%%%%Reference
\begin{thebibliography}{**}
\bibitem{Pelaez2016} J. R. Pelaez, \href{https://doi.org/10.1016/j.physrep.2016.09.001}{Phys. Rept. {\bf 658}, 1 (2016).}
\bibitem{weiwang2010} W. Wang and C. D. Lu, \href{https://doi.org/10.1103/PhysRevD.82.034016}{Phys. Rev. D {\bf 82}, 034016 (2010).}
\bibitem{Jaffe1977} R. L. Jaffe, \href{https://journals.aps.org/prd/abstract/10.1103/PhysRevD.15.267}{Phys. Rev. D {\bf 15}, 267 (1977).}
\bibitem{pdg} R. L. Workman {\it et al.} (Particle Data Group), \href{https://doi.org/10.1093/ptep/ptac097}{Prog. Theor. Exp. Phys. {\bf 2022}, 083C01 (2022).}
\bibitem{qqbar1} 
V. V. Anisovich, L. Montanet, and V. N. Nikonov, \href{https://doi.org/10.1016/S0370-2693(00)00370-1}{ Phys. Lett. B {\bf 480}, 19-22(2000).}
\bibitem{qqbar2}
B. El-Bennich, O. Leitner, J. P. Dedonder, and B. Loiseau, \href{https://doi.org/10.1103/PhysRevD.79.076004}{Phys. Rev. D {\bf 79}, 076004 (2009).}
\bibitem{QCDSR1} 
I. Bediaga, F. S. Navarra, and M. Nielsen, \href{https://doi.org/10.1016/j.physletb.2003.10.102}{Phys. Lett. B {\bf 579}, 59-66 (2004).}
\bibitem{QCDSR2} 
T. M. Aliev and M. Savci, \href{https://doi.org/10.1209/0295-5075/90/61001}{EPL {\bf 90}, 61001 (2010).}
\bibitem{LCSR1} 
P. Colangelo, F. D. Fazio, and W. Wang, \href{https://doi.org/10.1103/PhysRevD.81.074001}{Phys. Rev. D {\bf 81}, 074001 (2010).}
\bibitem{LCSR2} 
Y. J. Shi and W. Wang, \href{https://doi.org/10.1103/PhysRevD.92.074038}{Phys. Rev. D {\bf 92}, 074038 (2015).}
\bibitem{LF-RQM} 
H. W. Ke, X. Q. Li, and Z. T. Wei, \href{https://doi.org/10.1103/PhysRevD.80.074030}{Phys. Rev. D {\bf 80}, 074030 (2009).}
\bibitem{Soni2020}       
N. R. Soni, A. N. Gadaria, J. J. Patel, and J. N. Pandya, \href{https://doi.org/10.1103/PhysRevD.102.016013}{Phys. Rev. D {\bf 102}, 016013 (2020).}
\bibitem{tetraquark}
J. R. Pel\'aez, \href{https://doi.org/10.1103/PhysRevLett.92.102001}{Phys. Rev. Lett. {\bf 92}, 102001 (2004);}
N. N. Achasov and A. V. Kiselev, \href{https://doi.org/10.1103/PhysRevD.73.054029}{Phys. Rev. D {\bf 73}, 054029 (2006);}
G.’t Hooft, G. Isidori, L. Maiani, A. D. Polosa, and V. Riquer, \href{https://doi.org/10.1016/j.physletb.2008.03.036}{Phys. Lett. B {\bf 662}, 424 (2008);}
A. H. Fariborz, R. Jora, and J. Schechter, \href{https://doi.org/10.1103/PhysRevD.79.074014}{Phys. Rev. D {\bf 79}, 074014 (2009);}
S. Weinberg, \href{https://doi.org/10.1103/PhysRevLett.110.261601}{Phys. Rev. Lett. {\bf 110}, 261601 (2013);}
H. C. Kim, K. S. Kim, M. K Cheoun, Daisuke Jido, and Makoto Oka, \href{https://doi.org/10.1103/PhysRevD.99.014005}{Phys. Rev. D {\bf 99}, 014005 (2019).}
\bibitem{molecule}
J. Weinstein and N. Isgur, \href{https://doi.org/10.1103/PhysRevD.41.2236}{Phys. Rev. D {\bf 41}, 2236 (1990);}
T. Branz, T. Gutschea, and V. Lyubovitskij, \href{https://doi.org/10.1140/epja/i2008-10635-1}{Eur. Phys. J. A {\bf 37}, 303–317 (2008);}
L. Y. Dai, X. G. Wang, and H. Q. Zheng, \href{https://doi.org/10.1088/0253-6102/58/3/15}{Commun. Theor. Phys. {\bf 58}, 410 (2012);}
L. Y. Dai and M. R. Pennington  \href{https://doi.org/10.1016/j.physletb.2014.07.005}{Phys. Lett. B {\bf 736}, 11-15 (2014);}
T. Sekihara and S. Kumano, \href{https://doi.org/10.1103/PhysRevD.92.034010}{Phys. Rev. D {\bf 92}, 034010 (2015);}
D. L. Yao, L. Y. Dai, H. Q. Zheng, and Z. Y. Zhou, \href{https://iopscience.iop.org/article/10.1088/1361-6633/abfa6f}{Rep. Prog. Phys. {\bf 84}, 076201 (2021);}
Z. Q. Wang, X. W. Kang, J. A. Oller, and L. Zhang, \href{https://doi.org/10.1103/PhysRevD.105.074016}{Phys. Rev. D {\bf 105}, 074016 (2022).} 
\bibitem{Achasov2012}
N. N. Achasov and A. V. Kiselev, \href{https://doi.org/10.1103/PhysRevD.86.114010}{Phys. Rev. D {\bf 86}, 114010 (2012).}
\bibitem{Oset2015}
T. Sekihara and E. Oset, \href{https://doi.org/10.1103/PhysRevD.92.054038}{Phys. Rev. D {\bf 92}, 054038 (2015).}
\bibitem{bes3-D2a0enu} 
M. Ablikim {\it et al.} (BESIII Collaboration), \href{https://doi.org/10.1103/PhysRevLett.121.081802}{Phys. Rev. Lett. {\bf 121}, 081802 (2018).}
\bibitem{bes3-D2f0enu} 
M. Ablikim {\it et al.} (BESIII Collaboration), \href{https://doi.org/10.1103/PhysRevLett.122.062001}{Phys. Rev. Lett. {\bf 122}, 062001 (2019).}
\bibitem{bes3-Ds2f0enu} 
M. Ablikim {\it et al.} (BESIII Collaboration), \href{https://doi.org/10.1103/PhysRevD.105.L031101}{Phys. Rev. D {\bf 105}, L031101 (2022).}
\bibitem{bes3-Ds2a0enu} 
M. Ablikim {\it et al.} (BESIII Collaboration), \href{https://doi.org/10.1103/PhysRevD.103.092004}{Phys. Rev. D {\bf 103}, 092004 (2021).}
\bibitem{cleo2009-Ds2semilep} 
J. Yelton {\it et al.} (CLEO Collaboration), \href{https://doi.org/10.1103/PhysRevD.80.052007}{Phys. Rev. D {\bf 80}, 052007 (2009).}
\bibitem{cleo2009-Ds2f0enu} 
K. M. Ecklund {\it et al.} (CLEO Collaboration), \href{https://doi.org/10.1103/PhysRevD.80.052009}{Phys. Rev. D {\bf 80}, 052009 (2009).}
\bibitem{cleo2015-Ds2semilep} 
J. Hietala, D. Cronin-Hennessy, T. Pedlar, and I. Shipsey, \href{https://doi.org/10.1103/PhysRevD.92.012009}{Phys. Rev. D {\bf 92}, 012009 (2015).}
\bibitem{bes3-Ds2pnbar} 
M. Ablikim {\it et al.} (BESIII Collaboration), \href{https://doi.org/10.1103/PhysRevD.99.031101}{Phys. Rev. D {\bf 99}, 031101(R) (2019).}
\bibitem{material} See Supplemental Material at [URL] for additional analysis information.
\bibitem{BESIII1} 
M. Ablikim {\it et al.} (BESIII Collaboration), \href{https://www.sciencedirect.com/science/article/pii/S0168900209023870?via\%3Dihub}{Nucl. Instrum. Methods Phys. Res., Sect. A {\bf 614}, 345 (2010).}
\bibitem{BESIII2} 
M. Ablikim {\it et al.} (BESIII Collaboration), \href{https://iopscience.iop.org/article/10.1088/1674-1137/44/4/040001}{Chin. Phys. C {\bf 44}, 040001 (2020).}
\bibitem{BESIII3} 
X. Li {\it et al.}, \href{https://link.springer.com/article/10.1007/s41605-017-0014-2}{Radiat. Detect. Technol. Methods 1, 13 (2017);}
Y. X. Guo {\it et al.}, \href{https://link.springer.com/article/10.1007/s41605-017-0012-4}{Radiat. Detect. Technol. Methods 1, 15 (2017).}
\bibitem{BESIII4} K. X. Huang {\it et al.}, \href{https://doi.org/10.1007/s41365-022-01133-8}{Nucl. Sci. Tech. {\bf 33}, 142 (2022).}
\bibitem{GEANT4} S. Agostinelli {\it et al.} (GEANT4 Collaboration), \href{https://doi.org/10.1016/S0168-9002(03)01368-8}{Nucl. Instrum. Meth. A {\bf 506}, 250 (2003).}
\bibitem{KKMC} 
S. Jadach, B. F. L. Ward and Z. Was, \href{https://www.sciencedirect.com/science/article/pii/S0010465500000485?via\%3Dihub}{Comput. Phys. Commun. {\bf 130}, 260 (2000);} \href{https://journals.aps.org/prd/abstract/10.1103/PhysRevD.63.113009}{Phys. Rev. D {\bf 63}, 113009 (2001).}
\bibitem{EVTGEN} 
D. J. Lange, \href{https://www.sciencedirect.com/science/article/pii/S0168900201000894?via\%3Dihub}{Nucl. Instrum. Meth. A {\bf 462}, 152 (2001);}\\
R. G. Ping, \href{https://iopscience.iop.org/article/10.1088/1674-1137/32/8/001}{Chin. Phys. C {\bf 32}, 599 (2008).}
\bibitem{LUNDCHARM} 
J. C. Chen, G. S. Huang, X. R. Qi, D. H. Zhang and Y. S. Zhu, \href{https://journals.aps.org/prd/pdf/10.1103/PhysRevD.62.034003}{Phys. Rev. D {\bf 62}, 034003 (2000);}
R. L. Yang, R. G. Ping and H. Chen, \href{https://iopscience.iop.org/article/10.1088/0256-307X/31/6/061301/pdf}{Chin. Phys. Lett. {\bf 31}, 061301 (2014).}
\bibitem{PHOTOS} 
E. Richter-Was, \href{https://www.sciencedirect.com/science/article/abs/pii/037026939390062M?via\%3Dihub}{Phys. Lett. B {\bf 303}, 163 (1993).}
\bibitem{bes3-Dp2kpienu} 
M. Ablikim {\it et al.} (BESIII Collaboration), \href{https://journals.aps.org/prd/abstract/10.1103/PhysRevD.94.032001}{Phys. Rev. D {\bf 94}, 032001 (2016).}
\bibitem{bes2f980} M. Ablikim {\it et al.} (BES Collaboration), \href{https://doi.org/10.1016/j.physletb.2004.12.041}{Phys. Lett. B {\bf 607}, 243 (2005).}
\bibitem{Suntong} M. Ablikim {\it et al.} (BESIII Collaboration), \href{https://link.springer.com/article/10.1007/JHEP04\%282022\%29058}{J. High Energ. Phys. {\bf 04}, 058 (2022).}
\bibitem{bes3-Dp2omegaenu} M. Ablikim {\it et al.} (BESIII Collaboration), \href{https://journals.aps.org/prd/abstract/10.1103/PhysRevD.92.071101}{Phys. Rev. D {\bf 92}, 071101(R) (2015).}
\bibitem{LHCb} R. Aaij {\it et al.} (LHCb Collaboration), \href{https://journals.aps.org/prd/abstract/10.1103/PhysRevD.86.052006}{Phys. Rev. D {\bf 86}, 052006 (2012).}
\bibitem{FF-rate-sq} W. Wang, \href{https://doi.org/10.1016/j.physletb.2016.06.007}{Phys. Rev. B {\bf 759}, 501 (2016).}
The form of our double differential decay rate is a little different but consistent with this reference 
by redefining the parematers of Flatt\`e formula.  
\bibitem{ref-Ds1} N. N. Achasov, A. V. Kiselev, and G. N. Shestakov, \href{https://journals.aps.org/prd/abstract/10.1103/PhysRevD.102.016022}{Phys. Rev. D {\bf 102}, 016022 (2020).}
\bibitem{lifetime} R. Aaij {\it et al.} (LHCb Collaboration), \href{https://journals.aps.org/prl/abstract/10.1103/PhysRevLett.119.101801}{Phys. Rev. Lett. {\bf 119}, 101801 (2017).}
\bibitem{bes3-D02kenu} 
M. Ablikim {\it et al.} (BESIII Collaboration), \href{https://journals.aps.org/prd/abstract/10.1103/PhysRevD.92.072012}{Phys. Rev. D {\bf 92}, 072012 (2015).}
\bibitem{FF-simple} D. Becirevic and A. B. Kaidalov, \href{https://www.sciencedirect.com/science/article/pii/S0370269300002902?via\%3Dihub}{Phys. Lett. B {\bf 478}, 417 (2000).}
\end{thebibliography}
\end{document}

