\documentclass[accepted]{uai2024} % for initial submission
%\documentclass[accepted]{uai2024} % after acceptance, for a revised version; 
% also before submission to see how the non-anonymous paper would look like 
                        
%% There is a class option to choose the math font
% \documentclass[mathfont=ptmx]{uai2024} % ptmx math instead of Computer
                                         % Modern (has noticeable issues)
% \documentclass[mathfont=newtx]{uai2024} % newtx fonts (improves upon
                                          % ptmx; less tested, no support)
% NOTE: Only keep *one* line above as appropriate, as it will be replaced
%       automatically for papers to be published. Do not make any other
%       change above this note for an accepted version.

%% Choose your variant of English; be consistent
\usepackage[american]{babel}
% \usepackage[british]{babel}

%% Some suggested packages, as needed:
\usepackage{natbib} % has a nice set of citation styles and commands
    \bibliographystyle{plainnat}
    \renewcommand{\bibsection}{\subsubsection*{References}}
\usepackage{mathtools} % amsmath with fixes and additions
% \usepackage{siunitx} % for proper typesetting of numbers and units
\usepackage{booktabs} % commands to create good-looking tables
\usepackage{tikz} % nice language for creating drawings and diagrams


%% Packages from original manuscript
\usepackage{hyperref}
\usepackage{url}
\usepackage{multirow}
\usepackage{graphicx}
\usepackage[utf8]{inputenc} % allow utf-8 input
\usepackage[T1]{fontenc}    % use 8-bit T1 fonts
% \usepackage{hyperref}       % hyperlinks
\usepackage{url}            % simple URL typesetting
\usepackage{booktabs}       % professional-quality tables
\usepackage{amsfonts}       % blackboard math symbols
\usepackage{nicefrac}       % compact symbols for 1/2, etc.
\usepackage{microtype}      % microtypography

\usepackage{multirow}
\usepackage{graphicx}
% \usepackage[table,xcdraw]{xcolor}
\usepackage{amsmath}
\usepackage{amssymb}
\usepackage{subcaption}
\usepackage{balance}
\usepackage{diagbox}
\usepackage{algorithm} %%% added for the algorithm 
\usepackage[noend]{algpseudocode}
% \usepackage{flushend}
\usepackage{balance}

\newcommand{\bbox}{\text{bbox}}
\newcommand{\alphapck}{\alpha_\bbox}
\newcommand{\kcycle}{\text{k-CyPCK}}
\newcommand{\cycle}{\text{-CyPCK}}

\newcommand{\I}{\mathbf{I}}
\newcommand{\Ia}{\I^\text{a}}
\newcommand{\Ib}{\I^\text{b}}
\newcommand{\Iatob}{\I^\text{a $\rightarrow$ b}}
\newcommand{\F}{\mathbf{F}}
\newcommand{\Fa}{\F^\text{a}}
\newcommand{\Fb}{\F^\text{b}}
\newcommand{\f}{\mathbf{f}}
\newcommand{\fa}{\f^\text{a}}
\newcommand{\fb}{\f^\text{b}}
\newcommand{\p}{\mathbf{p}}
\newcommand{\pa}{\p^\text{a}}
\newcommand{\pb}{\p^\text{b}}
\newcommand{\A}{\boldsymbol{\Phi}_\text{align}}
\newcommand{\G}{\mathbf{G}}
\newcommand{\C}{\mathbf{C}}
\newcommand{\Ca}{\C^\text{a}}
\newcommand{\Cb}{\C^\text{b}}
\newcommand{\cc}{\mathbf{c}}
\newcommand{\cca}{\cc^\text{a}}
\newcommand{\ccb}{\cc^\text{b}}
\newcommand{\Irec}{\I_\text{Recon}}
\newcommand{\M}{\mathbf{M}}
\newcommand{\Mrec}{\M_\text{Recon}}
\newcommand{\loss}{\mathcal{L}}
\newcommand{\T}{\mathcal{T}}
\newcommand{\W}{\mathcal{W}}
\newcommand{\Id}{\mathcal{I}}


%% Provided macros
% \smaller: Because the class footnote size is essentially LaTeX's \small,
%           redefining \footnotesize, we provide the original \footnotesize
%           using this macro.
%           (Use only sparingly, e.g., in drawings, as it is quite small.)

%% Self-defined macros
\newcommand{\swap}[3][-]{#3#1#2} % just an example

\title{Bayesian Pseudo-Coresets via Contrastive Divergence}

% The standard author block has changed for UAI 2024 to provide
% more space for long author lists and allow for complex affiliations
%
% All author information is authomatically removed by the class for the
% anonymous submission version of your paper, so you can already add your
% information below.
%
% Add authors
% \author[1]{\href{mailto:<jj@example.edu>?Subject=Your UAI 2024 paper}{Jane~J.~von~O'L\'opez}{}}
\author[1]{Piyush~Tiwary}
\author[1]{Kumar~Shubham}
\author[1]{Vivek~V.~Kashyap}
\author[1]{Prathosh~A.P.}
% \author[3,1]{Further~Coauthor}
% Add affiliations after the authors
\affil[1]{%
    Department of Electrical Communuication Engineering\\
    
    Indian Institute of Science\\
    Bengaluru, Karnataka 560012, India
}
% \affil[2]{%
%     Second Affiliation\\
%     Address\\
%     …
% }
% \affil[3]{%
%     Another Affiliation\\
%     Address\\
%     …
%   }
  
  \begin{document}
\maketitle

\begin{abstract}
  Bayesian methods provide an elegant framework for estimating parameter posteriors and quantification of uncertainty associated with probabilistic models. However, they often suffer from slow inference times. To address this challenge, Bayesian Pseudo-Coresets (BPC) have emerged as a promising solution. BPC methods aim to create a small synthetic dataset, known as pseudo-coresets, that approximates the posterior inference achieved with the original dataset. This approximation is achieved by optimizing a divergence measure between the true posterior and the pseudo-coreset posterior.
    Various divergence measures have been proposed for constructing pseudo-coresets, with forward Kullback-Leibler (KL) divergence being the most successful. However, using forward KL divergence necessitates sampling from the pseudo-coreset posterior, often accomplished through approximate Gaussian variational distributions. Alternatively, one could employ Markov Chain Monte Carlo (MCMC) methods for sampling, but this becomes challenging in high-dimensional parameter spaces due to slow mixing.
    In this study, we introduce a novel approach for constructing pseudo-coresets by utilizing contrastive divergence. Importantly, optimizing contrastive divergence eliminates the need for approximations in the pseudo-coreset construction process. Furthermore, it enables the use of finite-step MCMC methods, alleviating the requirement for extensive mixing to reach a stationary distribution.
    To validate our method's effectiveness, we conduct extensive experiments on multiple datasets, demonstrating its superiority over existing BPC techniques. 
    Our implementation is available at \href{https://github.com/backpropagator/BPC-CD}{https://github.com/backpropagator/BPC-CD}
\end{abstract}

\section{Introduction}\label{sec:intro}
In recent years, contemporary deep learning models have demonstrated exceptional effectiveness in a wide array of applications, spanning computer vision, natural language processing, and speech analysis~\citep{krizhevsky2017imagenet,devlin2018bert,amodei2016deep,he2016deep,dosovitskiy2020image,radford2021learning}. 
Conventional deep learning methods rely on one-time training of models providing point estimates~\citep{szegedy2013intriguing}. These point estimates are prone to overfitting and often provide overconfident or under-confident outputs~\citep{gawlikowski2023survey,kabir2018neural}. This prohibits the use of deep learning models in critical applications such as medical, finance, etc~\citep{ker2017deep,cavalcante2016computational}.
Bayesian methods furnish a systematic framework for parameter estimation and quantification of associated uncertainty. Bayesian inference entails sampling from parameter posterior distributions using Markov Chain Monte Carlo (MCMC) techniques~\citep{robert1999monte,robert2011short}. However, conducting inference based on parameter posterior conditioned on the entire dataset is computationally demanding, particularly as the dataset size, denoted by $N$, increases. The computational complexity of MCMC methods scales with $N$ as $\Theta(NS)$, where $S$ denotes the number of samples~\citep{campbell_tamara_greedy_2018}. This complexity becomes prohibitively high for large $N$. To mitigate this, one often resorts to using a random subset of $M \ll N$ data points for likelihood computation at each iteration~\citep{bardenet2017markov,korattikara2014austerity,maclaurin2014firefly,welling2011bayesian,ahn2012bayesian,bierkens2019zig,pollock2020quasi}. However, such approximations introduce errors and lead to slow mixing of Markov chains~\citep{johndrow2020no,nagapetyan2017true,betancourt2015fundamental}.

Bayesian coresets~\citep{huggins_tamara_logistic} were introduced to solve the aforementioned problem. Particularly,~\citet{huggins_tamara_logistic} proposed to select a subset of original dataset (also called a coreset) that uniformly approximates the  log-likelihood of the original dataset. These coresets are significantly smaller in size than the original dataset, leading to vastly improved sampling efficiency. Further,~\cite{campbell_variational_bayesiancoreset} proposed to identify the coreset by minimizing Kullback-Leibler (KL) divergence between full data posterior and coreset posterior. However, most of such methods do not scale with data dimension~\citep{BPC_OG}. Particularly, the KL-divergence between the (optimal) coreset posterior and true posterior increases with data dimension. Meaning that for large data dimension, even with the optimal coresets, the KL-divergence is far from optima (zero), implying that the true posterior is not approximated correctly. However, recently few methods have tried to overcome the scalability issue by using lightweight coresets~\citep{bachem2018scalable}.

The Bayesian Pseudo-Coreset (BPC)~\citep{BPC_OG} approach, as a distinct category of methods, has been proposed to \textit{synthesise} a smaller dataset from the original one, as opposed to selecting a subset, which is the case with coreset methods. The fundamental idea of BPC involves solving an optimization problem over the data space, leading to creation of a `pseudo' dataset that appropriately approximates the true posterior. The said optimization problem pertains to minimization of a divergence metric between the true posterior and pseudo-coreset posterior. 
Such a framework removes the constraint for the pseudo-coresets to be a subset of the original dataset. This additional degree of freedom aids in better optimization of the divergence measure.
Further, in addition to better approximation of true posterior, pseudo-coreset also come with privacy benefits. Specifically, since pseudo-coresets are not part of original dataset, one can outsource these pseudo-coreset without revealing the original dataset for an user to run inference. \cite{BPC_OG} also provided theoretical guarantees showing pseudo-coresets are differentially private.


Recently,~\cite{BPC} analyzed the BPC construction under different divergence measures such as reverse-KL and Wasserstein divergence. Their analysis revealed that BPC methods under different divergence measures are equivalent to their non-bayesian counterparts. These non-bayesian frameworks are often referred to as `Dataset Condensation' or `Dataset Distillation'~\citep{GM,DM,MTT,CAFE,KIP}. In particular,~\cite{BPC} showed that minimization of reverse-KL is equivalent to gradient matching~\citep{GM} and minimization of wasserstein measure is equivalent to matching training trajectory~\citep{MTT}. 
They also proposed to use forward-KL for better pseudo-coreset construction due to its ability to capture the support of the distribution, contrasting with reverse-KL, which tends to focus on the distribution's modes. However, computing the gradient of forward-KL requires sampling from the intractable pseudo-coreset posterior. While this can be achieved using MCMC methods, the extensive mixing time of MCMC in high-dimensional parameter spaces renders this approach impractical. As a remedy, Gaussian variational approximation around SGD solutions was employed to simplify and expedite the sampling process.  However, the quality of such approximation is unknown and remains a matter of concern.  
Here, we note that the definition of contrastive divergence as presented in \citet{BPC} does not align with the original definition in \citet{CD}. Specifically, while \citet{BPC} uses forward-KL divergence to arrive at an objective referred to as the contrastive divergence (Section 3.3 of \citet{BPC}), the original definition (in \citet{CD})  involves difference between two forward-KL divergences. In this work, we resort to the original definition of contrastive divergence as noted in \citet{CD}.

% Here, we highlight a technical error in \citet{BPC}. Despite their assertion that their method aligns with the use of contrastive divergence (Section 3.3 of \citet{BPC}), it does not. Their chosen objective function is normal forward-KL divergence, whereas contrastive divergence involves the difference between two forward-KL divergences~\citep{CD}.


In our current work, we propose a novel approach: using contrastive divergence instead of forward-KL divergence for pseudo-coreset learning. This has two advantages: (1) It eliminates the need for approximating the pseudo-coreset posterior, enabling the straightforward use of MCMC methods, (2) The Markov chain used in this approach does not require extensive mixing to reach a stationary distribution; only a finite number of steps is needed. These advantages effectively address the challenges associated with using forward-KL divergence. Furthermore, our rigorous experiments demonstrate that our proposed method significantly outperforms previous state-of-the-art BPC methods, thereby confirming that the pseudo-coreset posterior using contrastive divergence better approximates the true posterior. Our contributions can be summarized as follows:
\begin{itemize}
    \item We propose a new framework for the construction of Bayesian Pseudo-Coreset  using contrastive divergence.
    \item The proposed method avoids any approximation of pseudo-coreset posterior and facilitates the use of finite step MCMC methods during learning phase.
    \item Extensive experimentation reveals that our method surpasses state-of-the-art BPC methods by substantial margins, affirming the better approximation of the true posterior using contrastive divergence.
\end{itemize}

\section{Related Work}
\label{sec:related_works}


%Apart from the given methods, dataset subset selection has been explored in other fields like continual learning~\citep{coresets_continual}, active learning~\citep{active2022clinical} and hyperparameter optimization~\citep{automata}.
% The coreset or dataset selection method is a special technique for reducing the amount of data required to train a given neural network. The main idea is to select a subset of the training dataset that can generate accuracy comparable to the original dataset. However, finding such a subset from the given training dataset is an NP-hard problem. Various studies have attempted to address this issue using simple heuristic methods. For example, \citep{welling2009herding} searches greedily for subset of data points whose feature centroid matches the original dataset. Similarly, \citep{mirzasoleiman2020coresets} proposes a technique for selecting a subset by minimising the gradient difference between the original dataset and the given subset.
\subsection{Bayesian Inference and Optimization}
The objective of Bayesian methods is the model the parameter posterior distribution of a probabilistic model. However, apart from some simple models, the exact posterior distributions are generally intractable ~\citep{campbell_tamara_hilbert_2019}. In such scenarios, one often relies on inference techniques like MCMC methods~\citep{robert1999monte,robert2011short} and Variational Inference (VI)~\citep{jordan1998introduction,wainwright2008graphical}. Historically, these inference techniques require model-specific tuning based on the path-length parameters, step size~\citep{neal2011handbook}, and the choice of the variational families~\citep{jaakkola1997variational,jordan1999introduction}. Recent methods~\citep{ranganath2014black,kucukelbir2017automatic,hoffman2014no} have circumvented these issues by introducing a black box approach that requires only basic specifications about the model. For instance, the traditional variational inference methods~\citep{jaakkola1997variational,jordan1999introduction} relied on closed form gradients of the model~\citep{ranganath2014black} and an approximate distribution for the posterior of the data. \citet{ranganath2014black,baydin2018automatic,kucukelbir2017automatic} addressed these issues by employing standard transformation over a multivariate Gaussian distribution and used automatic differentiation techniques to calculate the associated gradients. Similarly, for MCMC methods like Hamiltonian Monte Carlo (HMC)~\citep{neal2011handbook} traditional practices involved manually tuning of parameters like step size and path length to achieve accurate posterior estimation. \citet{hoffman2014no} addressed this challenge by automatically estimating both of these parameters.   

% \textcolor{red}{Historically, these inference techniques were developed separately for each specific model. Additionally, they relied heavily on expert knowledge to design transition kernels or gradient updates in variational inference~\citep{campbell_tamara_hilbert_2019}. However, recent work in automated tools like  automatic differentiation~\citep{baydin2018automatic,kucukelbir2017automatic}, black box based gradient estimates~\citep{ranganath2013black}, hamiltonian transition kernel~\citep{neal2011handbook} have alleviated the requirement of expert input and made it accessible to common practitioners.}

In many modern applications, these methods are required to scale with the size of the datasets. The standard MCMC algorithms are computationally expensive for large datasets, and the sampling process scales linearly with the data size. Recent works~\citep{bardenet2017markov,korattikara2014austerity,maclaurin2014firefly,welling2011bayesian,ahn2012bayesian,bierkens2019zig,pollock2020quasi}, have tried to mitigate the computational cost associated with inference models by considering only a random subset of data points during MCMC iterations. One of the initial studies in this direction has been conducted by~\citet{welling2011bayesian} where the authors proposed to use stochastic gradient langevin dynamics (SGLD). This iterative learning algorithm utilizes mini-batches of dataset for Bayesian inference. However, unlike other MCMC methods, their approach often leads to a slow mixing rate. \citet{ahn2012bayesian} addressed this issue by sampling from the Gaussian approximation of posterior for a high mixing rate and mimicking the behavior of SGLD using a pre-conditioner matrix for a slow mixing rate. However, \citet{korattikara2014austerity,bardenet2014towards} have shown that such a sampling approach often leads to a stationary distribution that can have bounded errors under strong conditions of rapid mixing~\citep{maclaurin2014firefly}.
In contrast, they proposed a new accept/reject strategy to select a subset of the dataset for Bayesian inference. On a similar line,~\citet{maclaurin2014firefly} proposed to use a collection of Bernoulli latent variables to select a subset of the dataset for likelihood estimation.  \citet{bierkens2019zig,pollock2020quasi}  have further proposed to use a zig-zag process and quasi-stationary distribution along with the subsampling approaches for bayesian inference. 


% However, such a sampling method generates a noisy approximation and can result in slow mixing of Markov chain~\citep{johndrow2020no,nagapetyan2017true,betancourt2015fundamental}.  

% Bayesian methods are key tool in many parameter and uncertainty approximation techniques. Currently the gold standard for bayesian inference is MCMC method~\citep{robert1999monte,robert2011short} which cast the stationary distribution to be same as the bayesian posterior and simulate markov chain to sample the parameter from this distribution. However, many modern methods deal with large dataset, for which such sampling techniques quickly becomes intractable. Taking inspiration from recent stochastic variational inference techniques~\citep{hoffman2013stochastic,ranganath2013black}, many methods tries to reduce down the computational cost associated with the inference by only considering a random subset of data point during the markov chain iterations~\citep{bardenet2017markov,korattikara2014austerity,maclaurin2014firefly,welling2011bayesian,ahn2012bayesian,bierkens2019zig,pollock2020quasi}. Such methods reduces the overall computational cost associated with per sample  
\subsection{Bayesian Coresets}
\label{sec:bayesian-coreset}
Bayesian coresets~\citep{huggins_tamara_logistic,campbell_tamara_greedy_2018,campbell_variational_bayesiancoreset,campbell_tamara_hilbert_2019,zhang_bayesiancoreset,naik2022fast,chen2022bayesian} present an alternative strategy to address aforementioned challenges by selecting a small weighted subset of the original dataset which can closely approximate the posterior of the full dataset~\citep{zhang_bayesiancoreset,huggins_tamara_logistic}. 
The idea was introduced in~\citet{huggins_tamara_logistic}, where a weighted subset of original data was selected to approximate the log-likelihood of the entire dataset up to some multiplicative error over the parameter space. However, the subset produced by such a technique underestimates the posterior distribution and can result in large approximation errors for some models regardless of the coreset size. \citet{campbell_tamara_greedy_2018} addressed this issue using greedy iterative geodesic ascent (GIGA), that optimally scales the log-likelihood of the coreset to better approximate the entire log-likelihood of the dataset. It further provided a uniform bounded error for all the models. To further enhance the scalability, \citet{campbell_tamara_hilbert_2019} tackled the model and data-specific assumptions made in prior work regarding coreset construction. They constructed Bayesian coreset by solving a sparse vector sum based approximation using frank-wolfe~\citep{frank1956algorithm} based solvers. Recent works~\citep{zhang_bayesiancoreset,naik2022fast,chen2022bayesian} have focused on improving the speed of coreset construction using accelerated optimization methods, quasi-newton refinement, and sparse-hamiltonian flows. However, since the KL divergence between the posteriors of the optimal coreset and the original dataset increases with the data dimensionality~\citep{BPC_OG}, these methods do not easily scale up in high-dimensions. 

\subsection{Bayesian Pseudo-Coreset}
\label{sec:BPC}
\citet{BPC_OG} proposed to use a collection of synthetic data to scale the Bayesian inference to high dimensional datasets. Particularly, they frame the problem as divergence minimization between the posteriors associated with the synthetic and the original dataset. The synthetic set generated through this technique is called `Bayesian Pseudo-Coreset' (BPC). Compared to Bayesian coresets, these methods scale more efficiently with data dimensions and yield a more accurate posterior approximation.

\citet{BPC_OG} formalized the given problem by minimizing the reverse-KL divergence between the posterior of original data and the posterior of synthetic data. On similar lines,~\citet{BPC} demonstrated that other divergence metrics, such as Wasserstein distance and forward-KL divergence, can be used to generate pseudo-coreset. In contrast to reverse-KL, which primarily focuses on the modes of the distributions, forward-KL provides a mechanism to better capture the support of the posterior distribution. To efficiently calculate the forward-KL divergence~\citet{BPC} used a Gaussian variational approximation of the posterior distribution. However, the quality of such an approximation and its impact on the overall performance of the pseudo-coreset is unknown. Further, computing the gradient of forward-KL requires sampling from an intractable posterior of pseudo-coreset using MCMC methods, which is not straightforward in practice.

\subsection{Coresets and Dataset Condensation}
\label{sec:coreset}

While Bayesian coreset focuses on selecting data points to facilitate Bayesian inference, coreset selection strategies have been proposed for other algorithms like geometric approximation~\citep{agarwal2005geometric}, mixture models~\citep{feldman2011scalable}, K-means clustering~\citep{feldman2011unified,feldman2020turning,bachem2016approximate} and DP means~\citep{bachem2015coresets}. Similarly, for deep learning models, \citet{mirzasoleiman2020coresets,killamsetty2021grad,killamsetty2021glister} have introduced subset selection techniques that leverage gradient matching and meta-learning algorithms. Recent works~\citep{welling2009herding, castro_herding, icarl_herding,scail_herding,sener2017active,farahani2009facility}, have further proposed strategies to choose a representative and diverse set of samples from the original dataset. These methods aim to create a generic subset by removing redundant data points. Herding-based coreset methods~\citep{welling2009herding, castro_herding, icarl_herding,scail_herding} select such samples by minimizing the distance between the feature centroids of the coreset, and the original dataset. While K-center-based coreset techniques~\citep{sener2017active,farahani2009facility,guo2022deepcore} pick the most diverse and representative samples by optimizing a submodular function~\citep{farahani2009facility}. Contrary to K-center and herding-based coreset selection methods, forgetting-based coreset~\citep{Forgetting} removes the easily forgettable samples from the training dataset. 

Rather than selecting a subset of data points from the training set, dataset condensation methods aim to generate a synthetic set that emulates the characteristics of the original dataset. For example, in gradient based dataset condensation techniques~\citep{GM,DD_Survey, DC_CS,jiang2022delving} the synthetic samples are generated by aligning the gradients of a model trained using original and synthetic datasets. Similarly, meta-learning based methods~\citep{DD,Rem_Past,KIP,RFAD,FREPO} generate these synthetic samples by matching the validation performance of a model trained using the entire dataset with the performance of a model trained using the synthetic set. \citet{MTT,DD_PP,du2022minimizing} propose generating the synthetic dataset using long-horizon trajectories, ensuring that the models learn similar trajectories during optimization. While distribution matching methods~\citep{DM,CAFE,zhao2022synthesizing,Improve_DM} generate a condensed synthetic set with a similar feature distribution as the original dataset. Recent works~\citep{liu2023dream, zhang2022accelerating,cazenavette2023generalizing} have further focused on improving the performance and computational complexity of existing dataset condensation techniques by using representative samples from the training set, model augmentation techniques, and generative model for learning the synthetic set. While dataset condensation and BPC might seem to do the same, they are fundamentally different from each other. One can look at Bayesian Pseudo-Coresets as bayesian counterparts of dataset condensation methods. We provide detailed account of differences between dataset condensation methods and BPC methods in Appendix.


\section{Proposed Methodology} 
\label{sec:proposed-methodology}


\subsection{Bayesian Pseudo-Coresets}
\label{sec:BPC-BG}
Consider a  dataset 
$(\rvx,\rvy) = \{(\rvx_i,y_i)\}^{n}_{i=1}$ consisting of $n$ data points. 
Now consider a synthetic (learnable) dataset $(\td{\rvx},\td{\rvy}) = \{\td{\rvx}_i,\tdy_i\}_{i=1}^{m}$ such that $\rvy$ and $\td{\rvy}$ share the same label space and $m \ll n$.
Let, $\theta \in \Theta$ be the parameter of a discriminative / classification model. Then the parameter posteriors corresponding to original and synthetic data, $\pi(\theta|\rvx)$ and $\pi(\theta|\td{\rvx})$ are given by
% The goal of BPC is to construct a sythetic dataset $\td{\rvx} = \{\td{\rvx}_i,\tdy_i\}_{i=1}^{|\td{\rvx}|}$ such that $\{y_i\}$ and $\{\tdy_i\}$ share the same label space and $|\td{\rvx}| \ll n$; further, $\td{\rvx}$ should provide classification performance that is comparable to that of original training set $\rvx$.
% For this, consider the space of parameters of a discriminative / classification model ($\Theta$). Now, let $\pi(\theta|\rvx)$ and $\pi(\theta|\td{\rvx})$ be the density of optimal parameters induced due to the original training set ($\rvx$) and the synthetic set ($\td{\rvx}$), respectively. Formally, 
\begin{align}
    \pi_{\rvx}  \triangleq \pi(\theta|\rvx) & = \frac{\pi_0(\theta)}{Z(\rvx)} \exp{\left(\sum\limits_{i=1}^{n} \log \pi(y_i|\rvx_i,\theta)\right)} \label{eq:true-posterior}\\
    \pi_{\td{\rvx}}  \triangleq \pi(\theta|\td{\rvx}) &= \frac{\pi_0(\theta)}{Z(\td{\rvx})} \exp{\left(\sum\limits_{i=1}^{m} \log \pi(\td{y}_i|\td{\rvx}_i,\theta)\right)}\\
    &= \frac{\pi_0(\theta)}{Z(\td{\rvx})} \exp{\left(-E(\td{\rvx},\theta)\right)} \label{eq:pc-posterior}
\end{align}
where, 
% $Z(\rvx) = \int_{\Theta} \pi_0(\theta)\exp{\left(\sum_{i=1}^{n} \log \pi(y_i|\rvx_i,\theta)\right)} d\theta$ and $Z(\td{\rvx}) = \int_{\Theta} \pi_0(\theta)\exp{\left(-E(\td{\rvx},\theta)\right)} d\theta$
\begin{align}
    Z(\rvx) &= \int_{\Theta} \pi_0(\theta)\exp{\left(\sum_{i=1}^{n} \log \pi(y_i|\rvx_i,\theta)\right)} d\theta\\  Z(\td{\rvx}) &= \int_{\Theta} \pi_0(\theta)\exp{\left(-E(\td{\rvx},\theta)\right)} d\theta
\end{align}
are appropriate normalizing constants. Here, $\pi_0(\theta)$ is the prior distribution and $E(\td{\rvx},\theta) = -\sum\limits_{i=1}^{m} \log \pi(\td{y}_i|\td{\rvx}_i,\theta) $ is the sum of negative log-likelihoods which can be treated as a generic potential or energy function. Since $n$ is often very large, the posterior estimation using $\pi_\rvx$ is computationally expensive and infeasible. However, an appropriate approximation such as $\pi_{\td{\rvx}}$ where $m \ll n$, allows one to overcome this hurdle. In particular, this approximation is carried out by solving the following optimization problem:
\begin{align}
    \td{\rvx}^* = \underset{\td{\rvx}}{\arg\min}\hspace{2mm} D\left( \pi_{\rvx}, \pi_{\td{\rvx}} \right)
\end{align}
where, $D(\cdot,\cdot)$ is a divergence measure between two distributions. Recently, \cite{BPC} showed the results for above optimization problem under different divergence metrics. Specifically, they analyzed the results with reverse-KL and wasserstein divergence; consequently drawing equivalence with dataset condensation methods like gradient matching~\citep{GM} and MTT~\citep{MTT}. Further, they propose an alternative solution by using forward-KL divergence as it encourages a model to cover the entire target distribution in contrast to reverse-KL which encourages mode capturing models. The gradient of the forward-KL divergence, as derived in~\cite{BPC}, is expressed as follows:
\begin{align}
    \nabla_{\td{\rvx}} D_{KL}\left( \pi_{\rvx} || \pi_{\td{\rvx}} \right) = \E_{\pi_{\td{\rvx}}}\left[ -\nabla_{\td{\rvx}}  E(\td{\rvx},\theta)  \right] + \nabla_{\td{\rvx}} \E_{\pi_{\rvx}}\left[ E(\td{\rvx},\theta) \right]
\label{eq:fkl-loss}
\end{align}
This gradient computation necessitates the calculation of expectations with respect to the probability distributions $\pi_{\rvx}$ and $\pi_{\td{\rvx}}$. However, the presence of intractable partition functions ($Z(\rvx)$ and $Z(\td{\rvx})$) poses challenges in efficiently sampling from these posterior distributions. 
One can resort to MCMC methods such as langevin dynamics or hamiltonian monte-carlo for sampling, however, due to large dimension of $\Theta$-space, the mixing-time of these methods is very large and in-efficient in practice. 
To overcome this issue,~\cite{BPC} employs gaussian variational approximations for these posteriors, rendering the sampling process computationally feasible. Specifically, gaussian distributions are used, centered around parameters obtained from Stochastic Gradient Descent (SGD) trajectories of $\rvx$ and $\td{\rvx}$ (cf.~\citep{BPC} for details). 
\par In practice, since $m$ (number of samples in pseudo-coreset) is generally very small, the SGD trajectories of $\td{\rvx}$ might overfit, leading to erroneous approximations. 
Hence, it can be seen that there is a clear trade-off between `posterior approximation quality' and `computational efficiency' in the previous methods (cf. Table~\ref{tab:mcmc-comp} for quantitative numbers).
Therefore, it is preferable to bypass such approximations and sample directly from the exact posteriors. In this work, we propose to work with contrastive divergence~\citep{CD} instead of forward-KL to construct the pseudo-coreset. Specifically, using contrastive divergence leads to a loss objective where  $\pi_{\td{\rvx}}$ can be used as it is without any approximation. The key idea behind this is that instead of minimizing forward-KL, contrastive divergence minimizes difference between two forward-KL terms, that results in cancellation of expectation w.r.t $\pi_{\td{\rvx}}$ allowing us to circumvent this approximation. We describe this in detail in next section.

\subsection{Contrastive Divergence for BPC}
\label{sec:CD-for-BPC}
As mentioned earlier, we propose to work with contrastive divergence instead of forward-KL for construction of pseudo-coresets. The concept of contrastive divergence was initially introduced by seminal work in~\cite{CD}.The central premise behind contrastive divergence hinges on a straightforward insight: whereas minimizing forward KL divergence necessitates a term that involves sampling from $\pi_{\td{\rvx}}$, minimizing the difference between two forward KL divergences leads to the nullification of this term. More explicitly, the contrastive divergence is defined as:
\begin{align}
    \gL_{CD} = D_{KL}(\pi_{\rvx} || \pi_{\td{\rvx}}) - D_{KL}(\Pi_{E}^{k}\pi_{\rvx} || \pi_{\td{\rvx}})
\label{eq:final_cd}
\end{align}
where, $\Pi_{E}^{k}(\cdot)$ is an MCMC transition kernel for $\pi_{\td{\rvx}}$ and $\Pi_{E}^{k}\pi_{\rvx}$ represents $k$ sequential MCMC transitions starting from $\pi_{\rvx}$. 
Here, we use transition kernel in the context of Markov process. Particularly, a one-step transition kernel is a map that takes a state as input and generates the next state after one step. Similarly, a $k$-step transition kernel takes a state as input and generates the state after $k$-steps. This is akin to the role of transition matrix in markov chains with finite states. 
In context of the proposed method, the $k$-step transition kernel takes a state sampled from $\pi_\rvx$, and generates a state after $k$-steps. Additionally, we note that Eq.~\ref{eq:final_cd} is minimized to zero only if $\pi_\rvx$ = $\pi_{\tilde{\rvx}}$. This is a well known result noted in \citet{CD} (cf Page 4 of \citet{CD}).

For brevity, let us denote $\Bar{\pi}_{\rvx}$ as $\Pi_{E}^{k}\pi_{\rvx}$. As shown in~\cite{CD}, the gradient of the above objective is approximately given by~\footnote{Please note that the signs in this expression are opposite to those presented in \citet{BPC}, primarily due to differences in the treatment of the energy function. In \citet{BPC}, the energy function (referenced as Eq. 2 in their paper) is considered positive, whereas in our work, we adopt a convention where the energy function (as represented in Eq.~\ref{eq:pc-posterior} of our paper) is negative. This choice is made for convenience, aligning with the convention in physics literature where lower energy states are typically considered stable.}:
\begin{align}
    \nabla_{\td{\rvx}} \gL_{CD} = \E_{\pi_{\rvx}}\left[\nabla_{\td{\rvx}} E(\td{\rvx},\theta)\right] - \E_{\Bar{\pi}_{\rvx}}\left[\nabla_{\td{\rvx}} E(\td{\rvx},\theta)\right]
\end{align}
It is worth noting that the gradient estimation in the above equation does not necessitate sampling from $\pi_{\td{\rvx}}$. Instead, it calls for sampling from $\pi_{\rvx}$ and $\Bar{\pi}_{\rvx}$. In this context, we can employ a variational posterior to approximate  $\pi_{\rvx}$ and use MCMC sampling techniques (e.g. langevin dynamics~\citep{LD}) starting from $\pi_{\rvx}$ to sample from $\Bar{\pi}_{\rvx}$. Notably, unlike in Eq.~\ref{eq:fkl-loss}, the MCMC sampling utilized here only needs to run for finite $k$ steps, alleviating the requirement for substantial Markov chain mixing.

In particular, we use gaussian variational posterior ($q_{\rvx}$) to approximate $\pi_{\rvx}$. Then, a $k$-step MCMC starting from $q_{\rvx}$ should be used as a variational substitute for $\Bar{\pi}_{\rvx}$:
\begin{align}
    q_{\rvx}(\theta) = \gN(\theta; \theta_{\rvx}, \Sigma_{\rvx}),\hspace{4mm} \Bar{q}_{\rvx}(\theta) = \Pi_{E}^{k}~ q_{\rvx}(\theta)
\label{eq:var_approx}
\end{align}
where, $\theta_{\rvx}$ is the MAP solution computed for $\rvx$.
Here, one can note that making an approximation for $\pi_{\rvx}$ is enough unlike previous methods where additional approximations for $\pi_{\td{\rvx}}$ is also required. Hence, the final gradient estimate is obtained as
\begin{align}
    &\nabla_{\td{\rvx}} \gL_{CD} \approx \E_{q_{\rvx}}\left[\nabla_{\td{\rvx}} E(\td{\rvx},\theta)\right] - \E_{\Bar{q}_{\rvx}}\left[\nabla_{\td{\rvx}} E(\td{\rvx},\theta)\right] \\
    &\approx \nabla_{\td{\rvx}}\frac{1}{N} \sum_{j=1}^{N} \left[ E\left(\td{\rvx}, \theta_{\rvx} + \Sigma_{\rvx}^{1/2}\varepsilon_{\rvx}^{(j)}\right) - E\left(\td{\rvx}, \texttt{sg}\left(\Bar{\theta}^{(j)}\right)\right) \right] \label{eq:final-loss}
\end{align}
where, $\texttt{sg}(\cdot)$ denotes stop-gradient operator, $\varepsilon_{\rvx}^{(j)} \sim \gN(0,I)$ and $\Bar{\theta}^{(j)}$ is obtained via running $k$-step MCMC starting from $\left(\theta_{\rvx} + \Sigma_{\rvx}^{1/2}\varepsilon_{\rvx}^{(j)}\right)$. 
\par Here, we note that one can theoretically learn $\Sigma_\rvx$ by treating it as a learnable parameter. However, this would  require gradient of determinant of the covariance matrix. Given the high dimensional parameter space, this operation would lead to computational inefficiencies. Hence, we treat it as a hyperparameter and keep it fixed. 
\par Further, for computational efficiency, we assess the parameter posterior with $\rvx$ using expert trajectories similar to~\citet{BPC}. In essence, expert trajectories represent sequences of parameters obtained while training a model on the dataset ($\rvx$, $\rvy$). Each of these sequences is termed as `parameter trajectory,' and the collection of these trajectories, acquired through various training instances, is known as `expert trajectories.' This  eliminates the need to compute MAP solutions for $\rvx$ ($\theta_\rvx$) at each training step. During training, we randomly pick a parameter from these trajectories to calculate the objective function.




\section{Experiments and results}
\label{sec:experiments}
\subsection{Evaluation Details}

We evaluate our method both quantitatively and qualitatively on several BPC-benchmark datasets with different compression ratios, i.e., the number of images generated per class (ipc).
In particular, we perform our experiments on six different datasets, namely, CIFAR10~\citep{CIFAR10}, SVHN~\citep{SVHN}, MNIST~\citep{MNIST}, FashionMNIST~\citep{FashionMNIST}, CIFAR100~\citep{CIFAR10} and Tiny Imagenet (T-Imagenet)~\citep{tinyimagenet}. All the experiments perform multi-class classification tasks with ipc=1, 10, and 50 which is in line with previous baselines.We employ Langevin dynamics~\citep{neal2011mcmc,teh2003energy} during training as well as inference and report accuracy (Acc) and negative log-likelihood (NLL) with respect to the ground truth labels. For our primary experiments, we use a CNN architecture (ConvNet) exactly as described in the previous works \citep{kim2022dataset,BPC_OG,cazenavette2023generalizing} (cf. Appendix for details) for a fair comparison. 
% We choose the categorical cross-entropy as the choice for energy function for these experiments unless mentioned otherwise. 

Further, we assess the robustness of the BPC methods on  out-of-distribution dataset and against adversarial attacks in Section~\ref{sec:ood-results}. We also examine the cross-architecture performance of the proposed method in Section~\ref{sec:cross-arch}. Next, since bayesian methods are often sensitive to the number of parameters being sampled from the posterior, we observe the effect of number of parameters  on the proposed method and compare it with previous BPC baselines in Section~\ref{sec:diff-arch}. Further, we provide an ablation study of the proposed method against different hyperparameters in Section~\ref{sec:ablations}. Lastly, we provide a quantitative comparison of quality of posterior parameter samples in Section~\ref{sec:comp-post-qual}.   We refer the reader to Appendix for details regarding these experiments.


 \subsection{Baselines and Comparisons}
 \label{sec:baseline}

 We consider the state-of-the-art BPC methods using reverse-KL (BPC-rKL), forward-KL (BPC-fKL), and Wasserstein distance (BPC-W)~\citep{BPC,BPC_OG} for comparison. Further comparison with other coreset methods and dataset condensation is provided in the Appendix. All the baselines are implemented using the official codebase provided by respective methods if available, otherwise, we directly take the reported numbers. In cases, neither the codebase nor the numbers are reported, we exclude them from our tables. 
 
 


 \subsection{Results and Comparison}
 \label{sec:performance_low_res}

\begin{table*}[!t]
\caption{\label{sota-comp-BPC}Comparison of the proposed method with BPC baselines. The results are noted in the form of (mean $\pm$ std. dev) where we have obtained test accuracy over five independent runs on the pseudo-coreset. The best performer across all methods is denoted in bold ($\boldsymbol{x\pm s}$). }
%For ease of comparison, we color the second-best performer with \textcolor{blue}{blue} color.}
 \resizebox{\textwidth}{!}
 {
 \renewcommand{\arraystretch}{1.1}
 
\begin{tabular}{l|rr|rr|rr|rr|rr|rr}
\toprule
 &
  \textbf{ipc} &
  \textbf{Ratio(\%)} &
  \multicolumn{2}{|c} {\textbf{BPC-rKL(sghmc)}} &
  \multicolumn{2}{|c} {\textbf{BPC-W (sghmc)}} &
  \multicolumn{2}{|c} {\textbf{BPC-fKL (hmc)}} &
  \multicolumn{2}{|c} {\textbf{BPC-fKL (sghmc)}} &
  \multicolumn{2}{|c}{\textbf{Ours}} \\
 & 
  \multicolumn{1}{l} {}&
  \multicolumn{1}{l}{} &
  \multicolumn{1}{|c} {\textbf{Acc($\uparrow$)}} &
  \multicolumn{1}{c} {\textbf{NLL($\downarrow$)}} &
  \multicolumn{1}{|c} {\textbf{Acc($\uparrow$)}} &
  \multicolumn{1}{c} {\textbf{NLL($\downarrow$)}} &
  \multicolumn{1}{|c} {\textbf{Acc($\uparrow$)}} &
  \multicolumn{1}{c}{\textbf{NLL($\downarrow$)}} &
  \multicolumn{1}{|c}{\textbf{Acc($\uparrow$)}} &
  \multicolumn{1}{c}{\textbf{NLL($\downarrow$)}} &
  \multicolumn{1}{|c}{\textbf{Acc($\uparrow$)}} &
  \multicolumn{1}{c}{\textbf{NLL($\downarrow$)}} \\ \midrule
 &
  1 &
  0.017 &
  $74.80 \pm 1.17$ &
  $1.90 \pm 0.01$ &
  $83.59 \pm 1.49$ &
  $1.91 \pm 0.02$ &
  $90.46 \pm 1.50$ &
  $1.54\pm 0.03$ &
  $82.98 \pm 2.20$ &
  $1.87 \pm 0.03$ &
  \boldsymbol{$93.42 \pm 0.09$} &
  \boldsymbol{$1.53\pm 0.01$} \\
 &
  10 &
  0.17 &
  $95.27 \pm 0.17$ &
  $1.53 \pm 0.01$ &
  $91.72 \pm 0.55$ &
  $1.52 \pm 0.01$ &
  $89.80 \pm 0.82$ &
  $1.52 \pm 0.01$ &
  $92.05 \pm 0.42$ &
  \boldsymbol{$1.51 \pm 0.02$} &
  \boldsymbol{$97.71 \pm 0.24$} &
  $1.57 \pm 0.02$ \\
 \textbf{\multirow{-3}{*}{	\textbf{MNIST}}} &
  50 &
  0.83 &
  $94.18 \pm 0.26$ &
  $1.36 \pm 0.02$ &
  $93.72 \pm 0.55$ &
  $1.48 \pm 0.02$ &
  $95.58 \pm 1.63$ &
  $1.37 \pm 0.02$ &
  $93.63 \pm 1.80$ &
  $1.36 \pm 0.02$ &
  \boldsymbol{$98.91 \pm 0.22$} &
  \boldsymbol{$1.36 \pm 0.01$} \\ \midrule
 &
  1 &
 0.017 &
  $70.53 \pm 1.09$ &
  $2.47 \pm 0.02$ &
  $72.39 \pm 0.87$ &
  $2.10 \pm 0.01$ &
  \boldsymbol{$78.24 \pm 1.02$} &
  $1.95 \pm 0.04$ &
  $72.51 \pm 2.53$ &
  $2.30 \pm 0.02 $ &
  $77.29 \pm 0.50$ &
  \boldsymbol{$1.90\pm 0.03$} \\
 &
  10 &
  0.17 &
  $78.81 \pm 0.17$ &
  $1.64 \pm 0.01$ &
  $83.69 \pm 0.51$ &
  $1.64 \pm 0.03$ &
  $82.06 \pm 0.44$ &
  \boldsymbol{$1.53 \pm 0.02 $} &
  $83.29 \pm 0.55$ &
  $1.54 \pm 0.03 $ &
  \boldsymbol{$88.40 \pm 0.21$} &
  $1.56\pm0.01$ \\ 
\multirow{-3}{*}{\textbf{FMNIST}} &
  50 &
  0.83 &
  $76.97 \pm 0.59$ &
  $1.48 \pm 0.02$ &
  $74.41 \pm 0.48$ &
  $1.52 \pm 0.03$ &
  $82.40 \pm 0.35$ &
  $1.32 \pm 0.02$ &
  $74.82 \pm 0.52$ &
  $1.47 \pm 0.02 $ &
  \boldsymbol{$89.47 \pm 0.06$} &
  \boldsymbol{$1.30\pm0.02$} \\ \midrule
 &
  1 &
  0.014 &
  $18.34 \pm 1.79$ &
  $3.01 \pm 0.02$ &
  $33.52 \pm 1.15$ &
  $2.89 \pm 0.01$ &
  $48.02 \pm 5.62$ &
  $2.44 \pm 0.03$ &
  $21.48 \pm 6.58$ &
  $ 2.57 \pm 0.02 $ &
  \boldsymbol{$66.74 \pm 0.09$} &
  \boldsymbol{$2.38\pm0.04$} \\
 &
  10 &
  0.14 &
  $60.68 \pm 5.07$ &
  $2.00 \pm 0.01$ &
  $74.75 \pm 1.27$ &
  $1.95 \pm 0.02$ &
  $65.64 \pm 2.92$ &
  $2.13 \pm 0.01$ &
  $75.49 \pm 0.84$ &
  $1.84 \pm 0.01 $ &
  \boldsymbol{$82.32 \pm 0.56$} &
  \boldsymbol{$1.81\pm0.01$} \\
\multirow{-3}{*}{\textbf{SVHN}} &
  50 &
  0.7 &
  $78.27 \pm 0.62$ &
  $1.89 \pm 0.01$ &
  $79.49 \pm 0.54$ &
  $1.90 \pm 0.01$ &
  $79.60 \pm 0.53$ &
  $1.86 \pm 0.01$ &
  $77.08 \pm 1.80$ &
  \boldsymbol{$1.72 \pm 0.01 $ }&
  \boldsymbol{$88.41 \pm 0.12$} &
  $1.88\pm0.02$ \\ \midrule
 &
  1 &
  0.02 &
  $21.62 \pm 0.83$ &
  $2.57 \pm 0.01$ &
  $29.34 \pm 1.21$ &
  $2.14 \pm 0.03$ &
  $35.57 \pm 0.95$ &
  $1.97 \pm 0.03$ &
  $29.30 \pm 1.10$ &
  $ 2.10 \pm 0.03 $ &
  \boldsymbol{$46.87 \pm 0.20$} &
  \boldsymbol{$1.87\pm0.02$} \\
 &
  10 &
  0.2 &
  $37.89 \pm 1.54$ &
  $2.13 \pm 0.02$ &
  $48.90 \pm 1.72$ &
  $1.73 \pm 0.02$ &
  $43.07 \pm 1.06$ &
  $1.89\pm 0.02$ &
  $49.85 \pm 1.37$ &
  $ 1.73 \pm 0.01$ &
  \boldsymbol{$56.39 \pm 0.70$} &
  \boldsymbol{$1.72\pm0.03$} \\
\multirow{-3}{*}{\textbf{Cifar10}} &
  50 &
  1 &
  $37.54 \pm 1.32$ &
  $ 1.93 \pm 0.03$ &
  $46.17 \pm 0.67$ &
  $1.62 \pm 0.02$ &
  $50.92 \pm 1.49$ &
  $1.70\pm0.03$ &
  $42.30 \pm 2.87$ &
  \boldsymbol{$ 1.54 \pm 0.01 $} &
  \boldsymbol{$71.93 \pm 0.17$} &
  $1.57\pm 0.03$ \\ \midrule
 &
  1 &
  0.2 &
  $3.56 \pm 0.04$ &
  $4.69 \pm 0.02$ &
  $12.19 \pm 0.22$ &
  $4.20 \pm 0.01 $ &
  $7.57 \pm 0.54$ &
  $4.25\pm0.04$ &
  $12.07 \pm 0.16$ &
  $4.27 \pm 0.02$ &
  \boldsymbol{$23.97 \pm 0.11$} &
  \boldsymbol{$4.01\pm 0.02$} \\
\multirow{-2}{*}{\textbf{Cifar100}} &
  10 &
  2 &
  - &
   &
  - &
   &
  - &
   &
  - &
  - &
  \boldsymbol{$28.42 \pm 0.24$} &
  \boldsymbol{$3.14\pm 0.02$ }\\ \midrule
 &
  1 &
  0.2 &
  - &
   &
  - &
   &
  - &
   &
  - &
  - &
  \boldsymbol{$8.39 \pm 0.07$} &
  \boldsymbol{$4.72\pm 0.01$} \\
\multirow{-2}{*}{\textbf{T-ImageNet}} &
  10 &
  2 &
  - &
   &
  - &
   &
  - &
   &
  - &
  - &
  \boldsymbol{$17.82 \pm 0.39$} &
  \boldsymbol{$3.64\pm0.05$} \\
\bottomrule
\end{tabular}
}
\end{table*}



\begin{figure*}[!t]
\centering
   \begin{subfigure}[t]{0.239\textwidth}
   \begin{minipage}{\textwidth}
   \includegraphics[keepaspectratio, width=\textwidth]{figs/mnist/mnist-ipc1.png}
   \caption{Images generated with ipc=1 for MNIST}
   \includegraphics[keepaspectratio, width=\textwidth]{figs/mnist/mnist-ipc10.png}
       \caption{Images generated with ipc=10 for MNIST}
   
    \end{minipage}
    % \caption{MNIST}
   % \label{fig:Ng1} 
\end{subfigure}
~
\begin{subfigure}[!t]{0.239\textwidth}
\begin{minipage}{\textwidth}
    
       \includegraphics[keepaspectratio, width=\textwidth]{figs/fmnist/fmnist-ipc1.png}
       \caption{Images generated with ipc=1 for FMNIST}
   \includegraphics[keepaspectratio, width=\textwidth]{figs/fmnist/fmnist-ipc10.png}
   \caption{Images generated with ipc=10 for FMNIST}

   % \label{fig:Ng2}
\end{minipage}
% \caption{FMNIST}
\end{subfigure}
~
\begin{subfigure}[t]{0.239\textwidth}
\begin{minipage}{\textwidth}
   
   \includegraphics[keepaspectratio, width=\textwidth]{figs/svhn/svhn-ipc1.png}
   \caption{Images generated with ipc=1 for SVHN}
   \includegraphics[keepaspectratio, width=\textwidth]{figs/svhn/svhn-ipc10.png}
   \caption{Images generated with ipc=10 for SVHN}
   % \label{fig:Ng2}
\end{minipage}
 % \caption{SVHN}
\end{subfigure}
~
\begin{subfigure}[t]{0.239\textwidth}
\begin{minipage}{\textwidth}

      \includegraphics[keepaspectratio, width=\textwidth]{figs/cifar10/cifar10-ip1.png}
      \caption{Images generated with ipc=1 for CIFAR10}
   \includegraphics[keepaspectratio, width=\textwidth]{figs/cifar10/cifar10-ipc10.png}
    \caption{Images generated with ipc=10 for CIFAR10}
   % \label{fig:Ng2}
\end{minipage}
% \caption{CIFAR10}
\end{subfigure}
\caption{\label{viz-ipc-10}Visualizations of pseudo-coreset generated from our method with one image per class (top) and ten images per class (bottom) for MNIST, FMNIST, SVHN and CIFAR10. It can be seen that the class labels are identifiable to a large extent.}
\end{figure*}

Table~\ref{sota-comp-BPC} presents the results of the comparative analysis between our approach and other BPC baselines.
We observe that the proposed method significantly outperforms all the BPC baselines by large margins. For instance, we observe an improvement of $11.3\%$, $6.54\%$, and $21.01\%$ in accuracy for CIFAR10 with ipc values of 1, 10, and 50, respectively. Additionally, there is a decrease of 0.1 and 0.01 points in negative log-likelihood for ipc values of 1 and 10, respectively, in comparison to the best-performing BPC baseline.  Similarly, on SVHN, we notice an improvement in accuracy and negative log-likelihood. Specifically, we observe gains of $18.72\%$, $6.83\%$ in accuracy and reduction of $0.06$, $0.03$ point in negative log-likelihood for ipc 1, 10 respectively, compared to the BPC counterparts. A similar trend can be seen for MNIST and FMNIST as well~\footnote{We observe that the proposed method has smaller variance compared to other methods. Upon analysis, we attribute this behavior to the rapid decrease in the gradient of the energy function ($(\nabla_{\theta}E(\theta,\tilde{\rvx})$) with respect to parameters in langevin dynamics. This indicates that the sampled parameter stays near the initial parameter value from the expert trajectory, which is based on confident predictions derived from SGD trajectories trained on the full dataset. Therefore, the reduced variance observed is likely a result of this confidence.}. 
%The proposed method also outperforms previous baselines by large margins on bigger datasets like CIFAR100 and Tiny-ImageNet. For instance, we observe an increment of $11.78\%$ on CIFAR100 with an ipc of 1. 
We attribute this boost in performance to the flexible formulation of the proposed method. 


We present the qualitative visualizations for MNIST, FMNIST, SVHN, and CIFAR10 datasets with 1 image per class and 10 image per class in Fig.~\ref{viz-ipc-10}. It can be seen that the constructed pseudo-coreset is identifiable but inherits some artifacts due to the constraints on the dataset size. As the number of images per class increases, the model can induce more variations across all the classes and thus produce a diverse pseudo-coreset. Additional qualitative visualizations for pseudo-coreset generated with 50 images per class on CIFAR100 and T-ImageNet dataset are presented in the Appendix.

\subsection{Results on Out of Distribution (OOD) dataset}
\label{sec:ood-results}
We present the results of the proposed method on out-of-distribution (OOD) dataset in Table~\ref{tab:ood-performance}.
% Table~\ref{tab:ood-performance} presents the results of our experiment on the out-of-distribution dataset.
We use CIFAR10-C~\citep{CIFAR10-C} dataset for this experiment.
In particular, we sample the parameters from the pseudo-coreset posterior obtained using clean CIFAR10 (ipc=10) and perform inference on the corrupted CIFAR10-C, which consists of CIFAR10 images afflicted with different types of corruption including Gaussian Blur, Gaussian Noise, etc.
It is evident from Table~\ref{tab:ood-performance} that our method demonstrates robustness to various types of corruption and exhibits superior performance compared to other baselines. Notably, for corruptions like Gaussian Blur, our method achieves a $1.63\%$ increase in accuracy and a $0.17$-point reduction in negative log-likelihood compared to the best-performing BPC baseline. Likewise, for JPEG Compression, Zoom Blur, and Defocus Blur, our method yields an improvement of $0.07\%$, $2.08\%$, and $0.43\%$ in accuracy, along with a reduction of $0.14$, $0.05$, and $0.06$ points in negative log-likelihood, respectively. The robustness of the proposed method to different forms of corruption highlights its ability to provide a better approximation of underlying posterior distribution when compared to other baselines. \\
We further test all the BPC methods against $\ell_{\infty}$ adversarial attack~\citep{robustbench}. We report the clean accuracy and robust accuracy respectively for each method. Our observations could be found in Table~\ref{tab:adv-attack}. We see that under $\ell_{\infty}$ attack, the performance of all BPC methods drop significantly. This is perhaps due to the fact that these methods don't explicitly take robustness into account while constructing pseudo-coresets. However, we observe even under performance drop, our method gives best robust accuracy as compared to other BPC methods.


\begin{table*}[!t]
\centering
\caption{Comparison of the proposed method with BPC baselines for the performance on  out-of-distribution data. The classifier model is trained on pseudo-coresets generated using CIFAR10. However, the model is evaluated on CIFAR10-C dataset with different types of corruption.}
\resizebox{0.95\textwidth}{!}{%
\begin{tabular}{l|rr|rr|rr|rr|rr}
\toprule
\multirow{2}{*}{\textbf{Corruption}} & \multicolumn{2}{c|}{\textbf{BPC-rKL (sghmc)}} & \multicolumn{2}{c|}{\textbf{BPC-W (sghmc)}} & \multicolumn{2}{c|}{\textbf{BPC-fKL (hmc)}} & \multicolumn{2}{c|}{\textbf{BPC-fKL (sghmc)}} & \multicolumn{2}{c}{\textbf{Ours}}                                    \\ 
                            & \textbf{Acc($\uparrow$)}               & \textbf{NLL($\downarrow$)}              & \textbf{Acc($\uparrow$)}     & \textbf{NLL($\downarrow$)}    & \textbf{Acc($\uparrow$)}     & \textbf{NLL($\downarrow$)}    & \textbf{Acc($\uparrow$)}      & \textbf{NLL($\downarrow$)}     & \textbf{Acc($\uparrow$)}                  & \textbf{NLL($\downarrow$)}                 \\ \midrule
Gaussian Blur               & $31.02\pm2.65$        & $2.13\pm0.77$       & $35.66\pm1.21$       & $2.04\pm0.12$       & $34.76\pm1.86$       & $1.89\pm0.04$    & $39.73\pm2.72$        & $1.94\pm0.05$       & \boldsymbol{$41.36\pm 0.72$} & \boldsymbol{$1.73 \pm 0.83$} \\
Gaussian Noise              & $25.49\pm1.89 $       &$ 2.28\pm0.08$        & $33.21\pm0.89$       & $2.11\pm0.03$       & $36.70\pm1.01$         & $1.86\pm0.02$       & $35.71\pm2.29$        & $2.00\pm0.05$        & \boldsymbol{$38.01\pm 1.26$} & \boldsymbol{$1.82 \pm 0.11$} \\
JPEG Compression            & $30.40\pm0.90 $        &$ 2.13\pm0.02$        & $26.33\pm1.34$       & $2.26\pm0.04$       & $36.20\pm1.92$        & $1.85\pm0.03$       & $37.26\pm2.87$        & $1.95\pm0.06$        & \boldsymbol{$37.33\pm 0.19$} & \boldsymbol{$1.71 \pm 0.03$} \\
Snow                        & $26.85\pm1.71$        &$ 2.20\pm0.07$         & $37.50\pm3.50$         & $1.93\pm0.08$       & $33.99\pm1.91$       & \boldsymbol{$1.91\pm0.03$}       & $35.68\pm2.71$        & $2.00\pm0.07$         & \boldsymbol{$37.84\pm 0.64$} & $1.91 \pm 0.05$ \\
Impulsive Noise             & $28.39\pm1.48$        &$ 2.15\pm0.06$        & $36.71\pm1.93$       & $1.96\pm0.05$       & $33.81\pm1.58$       & $1.94\pm0.02$       & \boldsymbol{$38.26\pm2.34$}        & $1.92\pm0.05$        & $37.98\pm 2.15$ & \boldsymbol{$1.89 \pm 0.07$} \\
Zoom Blur                   & $31.74\pm1.24$        &$ 2.09\pm0.04$        & $36.22\pm2.08$       & $1.99\pm0.05$       & $31.30\pm3.64$        & $1.98\pm0.08$       & $35.05\pm2.90$        & $2.04\pm0.07$        & \boldsymbol{$38.30\pm 0.77$} & \boldsymbol{$1.93 \pm 0.13$} \\
Pixelate                    & $28.98\pm2.26$        &$ 2.19\pm0.07$        & $27.98\pm1.77$       & $2.20\pm0.05$       & $35.59\pm1.94$       & \boldsymbol{$1.88\pm0.03$}       & \boldsymbol{$39.14\pm3.15$}        & $1.93\pm0.06$        & $38.97\pm 1.51$ & $1.92 \pm 0.07$ \\
Speckle Noise               & $29.88\pm0.59$        &$ 2.09\pm0.02$        & $33.33\pm2.18$       & $2.05\pm0.05$       & $34.37\pm2.02$       & $1.90\pm3.57$       & $40.54\pm1.93$        & \boldsymbol{$1.89\pm0.04$}        & \boldsymbol{$42.66\pm 0.83$} & $1.95 \pm 0.03$ \\
Defocus Blur                & $27.57\pm1.31$        &$ 2.20\pm0.05$        & $33.80\pm4.21$       & $2.09\pm0.11$       & $33.60\pm2.93$        & $1.93\pm0.06$       & $36.72\pm3.68$        & $1.99\pm0.08$        & \boldsymbol{$37.15\pm 1.03$} & \boldsymbol{$1.87 \pm 0.04$} \\
Motion Blur                 & $17.38\pm2.51$        &$ 2.73\pm0.14$        & $35.22\pm3.35$       & $2.01\pm0.08$       & $34.33\pm1.89$       & $1.92\pm0.04$      & $35.24\pm3.30$        & $2.01\pm0.05$        & \boldsymbol{$37.06\pm 0.49$} & \boldsymbol{$1.92 \pm 0.04$} \\ \bottomrule
\end{tabular}%
}
\label{tab:ood-performance}
\end{table*}

\begin{table*}[!t]
\centering
\caption{Comparison of the proposed method with BPC baselines for the performance against $\ell_{\infty}$ attack on CIFAR10 with 10 ipc.}
\resizebox{0.9\textwidth}{!}{%

\begin{tabular}{rr|rr|rr|rr}
\toprule
                            \multicolumn{2}{c|}{\textbf{BPC-rkl}}    & \multicolumn{2}{c|}{\textbf{BPC-W}}      & \multicolumn{2}{c|}{\textbf{BPC-fkl}}    & \multicolumn{2}{c}{\textbf{Ours}}                   \\ 
                            \textbf{Clean Acc} & \textbf{Robust Acc} & \textbf{Clean Acc} & \textbf{Robust Acc} & \textbf{Clean Acc} & \textbf{Robust Acc} & \textbf{Clean Acc}       & \textbf{Robust Acc}       \\ \midrule
 $37.89 \pm 1.54$   & $13.54 \pm 1.21$    & $48.90 \pm 1.72$    & $15.62 \pm 0.92$    & $49.85 \pm 1.37$   & $19.30 \pm 1.39$    & $\mathbf{56.39 \pm 0.70}$ & $\mathbf{22.81 \pm 1.28}$ \\ \bottomrule
\end{tabular}%
\label{tab:adv-attack}
}
\end{table*}

\begin{table*}[!t]
  \begin{minipage}{0.45\linewidth}
    \centering
    \caption{\label{cross-architecture}Cross-architecture generalization analysis of BPC methods.}

\renewcommand{\arraystretch}{1.3}
\resizebox{\textwidth}{!}{
\begin{tabular}{l|rrrr}
\toprule
 & \textbf{ConvNet} & \textbf{ResNet} & \textbf{VGG} & \textbf{AlexNet} \\ \midrule
    \textbf{Ours} &
  \boldsymbol{$56.39 \pm 0.70$} &
  \boldsymbol{$41.65 \pm 1.03$} &
  \boldsymbol{$47.51 \pm 0.89$} &
  \boldsymbol{$30.58 \pm 1.43$} \\
\textbf{BPC-fKL(hmc)} &
  $44.34 \pm 1.11$ &
  $10.15 \pm 0.21$ &
  $10.43 \pm 0.33$ &
  $12.21 \pm 0.18$ \\
\textbf{BPC-rKL(sghmc)} &
  $34.48 \pm 0.48$ &
  $10.06 \pm 0.08$ &
  $10.26 \pm 0.35$ &
  $11.02 \pm 0.12$ \\ \bottomrule
\end{tabular}}
  \end{minipage}%
  \hspace{0.3cm}
  \begin{minipage}{0.58\linewidth}
    \centering
    \caption{Performance comparison of the proposed method and other BPC baselines for different parameterized architectures.}
\centering
\resizebox{\textwidth}{!}{
\begin{tabular}{l|rrrrrrr}
\toprule
\textbf{Methods} &  &  \textbf{CN-D3W128} & \textbf{CN-D3W256} & \textbf{CN-D5W128} &  \textbf{AlexNet} &  \textbf{VGG11} &  \textbf{ResNet} \\
 &  &  320,010 &1,229,834 &  596,490 &  1,872,202 &  9,231,114 & 11,173,962 \\
 \midrule
 \textbf{Ours} &  & \boldsymbol{$56.39 \pm 0.70$} & \boldsymbol{$55.93\pm 1.30$} & \boldsymbol{$56.01 \pm 0.69$} & \boldsymbol{$52.88 \pm 1.39$} & \boldsymbol{$49.26 \pm 2.33$} & \boldsymbol{$48.67 \pm 0.52$} \\
 \textbf{BPC-rKL (sghmc)} &  & $37.89 \pm 1.54$ & $35.82 \pm 1.88$ & $35.92 \pm 1.88$ & $32.60 \pm 1.45$ &  $27.66 \pm 0.73$ & $24.98 \pm 1.53$ \\
 \textbf{BPC-W (sghmc)} &  & $48.90 \pm 1.72$ & $43.71 \pm 1.42$ & $46.01\pm 0.92$ & $39.01 \pm 0.51$ & $35.11 \pm 1.82$ & $32.84 \pm 1.38$ \\
\textbf{BPC-fKL (hmc)} &  & $49.85 \pm 1.37$ & $45.87 \pm 0.78$ & $47.92 \pm 1.27$ & $41.22 \pm 1.62$ & $37.05 \pm 1.24$ & $35.10 \pm 2.03$ \\
 \bottomrule

\end{tabular}}
\label{tab:diff-arch}
  \end{minipage}
\end{table*}

\subsection{Results on Cross-Architecture Experiments}
\label{sec:cross-arch}
Here, we present the cross-architecture results pertaining to various BPC methods. In these experiments, we construct the pseudo-coreset using the said ConvNet model, while during inference, we use different architectures such as ResNet~\citep{resnet}, VGG-Net~\citep{vgg} and AlexNet~\citep{alexnet} for evaluation. We perform these experiments for CIFAR10 (ipc = 10).
The results of the cross-architecture experiments are presented in Table~\ref{cross-architecture}. It can be seen that previous BPC methods fail to generalize across different network architectures, whereas our method demonstrates the ability to adapt to various architectures. For instance, the performance of BPC-fKL and BPC-rKL drop by $34.19\%$ and $24.42\%$ respectively on ResNet, resulting in random predictions with an accuracy of almost $10\%$, whereas our method observes a drop of only $14.74\%$ while giving an accuracy of $41.65\%$. 
% \textcolor{red}{It is noteworthy that this aspect of the cross-architecture generalization has been explored in dataset condensation literature but not the BPC literature}. 



\subsection{Effect of Different number of Parameters}
\label{sec:diff-arch}
Lastly, we analyze the performance of BPC methods across differently parameterized networks. Specifically, we generate pseudo-coresets for CIFAR10 (ipc=10) by employing ConvNets with different parameter configurations. These configurations encompass ConvNets with different depth and width. We also conduct a comparative analysis with other deep learning architectures, including AlexNet~\citep{alexnet}, VGG11~\citep{vgg}, and ResNet~\citep{resnet}. Bayesian inference techniques generally encounter scalability issues when dealing with large parametric networks~\citep{handsOnBayesian}. This experiment is conducted to ascertain the impact of both large and small architectures on the performance of pseudo-coresets.      

The results for different parameterized architectures are presented in Table~\ref{tab:diff-arch}. Here, CN-DxWy denotes a ConvNet architecture with a depth of `x' and width of `y'. It is evident from the results that the performance of all BPC methods declines as the number of parameters in the architectures increases. However, our model exhibits relatively better performance in comparison to other methods. Specifically, while our method demonstrates a $7.72\%$ decrease in performance for the ResNet architecture, other BPC baselines such as BPC-fKL, BPC-W, and BPC-rKL experience declines of approximately $14.75\%$, $16.06\%$, and $12.91\%$, respectively. This observation underscores the greater tolerance of our method to large parametric models when compared to other baselines. We again highlight that the performance gain achieved by our proposed method can be attributed to the current formulation, which can generate better approximation to the true posterior.

\subsection{Ablation Study}
\label{sec:ablations}
Next, we provide an empirical analysis of the effect of different hyperparameters on the proposed method. Particularly, we observe that there are two key hyperparameters that affect the performance of the proposed method: the covariance matrix $\Sigma_{\rvx}$ (see Eq.~\ref{eq:var_approx})   and the number of MCMC steps $k$ (see Eq.~\ref{eq:final_cd}). Hence, we ablate our method against these two important hyperparameters and report the results in Table~\ref{tab:main-ablation}.

Further, we verify our claim that the proposed method provides a good trade-off between performance and computational cost. Particularly, as seen in Eq.~\ref{eq:fkl-loss}, one can resort to MCMC methods to evaluate the second term of the said expression, however, since this can be computationally expensive in the high-dimensional parameter space, we resort to using contrastive-divergence that allows one to use finite step MCMC steps making the proposed method computationally less expensive. To verify this, we compare our method against pseudo-coresets obtained using Eq.~\ref{eq:fkl-loss} alongwith MCMC methods. Particularly, we employ HMC~\citep{sghmc} and Kronecker-Factorised Laplace (KFL)~\citep{kfl} for this purpose. Our observations are noted in Table~\ref{tab:mcmc-comp}.We compare GPU Memory usage, Iteration time and Accuracy for each of these method against ours. We see that both HMC and KFL consume more memory and time compared to our method. However, HMC provides marginally better result and KFL gives slightly worse result compared to our method. This verifies our claim.

\begin{table}[!t]
    \centering
    \caption{Ablation study of the proposed method against different hyperparameters on CIFAR10 with 10 ipc.}
    \resizebox{\columnwidth}{!}{
    \begin{tabular}{rrrrrr}
        \toprule
        \multicolumn{6}{c}{\textbf{Ablation on $\Sigma_{\rvx}$}}                                                                                                                                                                                     \\ \midrule
        \multicolumn{2}{c|}{\boldsymbol{$\Sigma_{\rvx}^{1/2} = 0.01I$}}                     & \multicolumn{2}{c|}{\boldsymbol{$\Sigma_{\rvx}^{1/2} = 0.001I$}}                    & \multicolumn{2}{c}{\boldsymbol{$\Sigma_{\rvx}^{1/2} = 0.0001I$}}                   \\ 
        \multicolumn{1}{c}{\textbf{Acc}}     & \multicolumn{1}{c|}{\textbf{NLL}}    & \multicolumn{1}{c}{\textbf{Acc}}     & \multicolumn{1}{c|}{\textbf{NLL}}    & \multicolumn{1}{c}{\textbf{Acc}}     & \multicolumn{1}{c}{\textbf{NLL}}    \\ \midrule
        \multicolumn{1}{c}{$50.18 \pm 0.50$} & \multicolumn{1}{c|}{$1.94 \pm 0.05$} & \multicolumn{1}{c}{$\mathbf{56.39 \pm 0.70}$} & \multicolumn{1}{c|}{$\mathbf{1.72 \pm 0.03}$} & \multicolumn{1}{c}{$54.18 \pm 0.23$} & \multicolumn{1}{c}{$1.86 \pm 0.02$} \\ \bottomrule
        \multicolumn{6}{c}{}                                                                                                                                                                                                                        \\ \toprule
        \multicolumn{6}{c}{\textbf{Ablation on $k$ (or $L$ in Algorithm 1)}}                                                                                                                                                                      \\ \midrule
        \multicolumn{2}{c|}{\boldsymbol{$k = 10$}}                                       & \multicolumn{2}{c|}{\boldsymbol{$k = 50$}}                                       & \multicolumn{2}{c}{\boldsymbol{$k = 100$}}                                      \\ 
        \multicolumn{1}{c}{\textbf{Acc}}     & \multicolumn{1}{c|}{\textbf{NLL}}    & \multicolumn{1}{c}{\textbf{Acc}}     & \multicolumn{1}{c|}{\textbf{NLL}}    & \multicolumn{1}{c}{\textbf{Acc}}     & \multicolumn{1}{c}{\textbf{NLL}}    \\ \midrule
        \multicolumn{1}{c}{$43.77 \pm 0.98$} & \multicolumn{1}{c|}{$1.78 \pm 0.02$} & \multicolumn{1}{c}{$51.27 \pm 1.01$} & \multicolumn{1}{c|}{$1.78 \pm 0.02$} & \multicolumn{1}{c}{$\mathbf{56.39 \pm 0.70}$} & \multicolumn{1}{c}{$\mathbf{1.72 \pm 0.03}$} \\ \bottomrule
        \end{tabular}
        }
    \label{tab:main-ablation}
\end{table}

\begin{table}[!t]
    \caption{Efficiency and Accuracy comparison of the proposed method against pseudo-coresets obtained via direct MCMC estimation of second term of Eq.~\ref{eq:fkl-loss} on CIFAR10 with 10 ipc.}
    \centering
    \resizebox{\columnwidth}{!}{
    \begin{tabular}{ccc|ccc|ccc}
        \toprule
         \multicolumn{3}{c|}{\textbf{HMC}~\citep{sghmc}}                           & \multicolumn{3}{c|}{\textbf{KFL}~\citep{kfl}}  & \multicolumn{3}{c}{\textbf{Ours}}                          \\
                                    \textbf{GPU (GB)} & \textbf{Time (s)} & \textbf{Acc} & \textbf{GPU (GB)} & \textbf{Time (s)} & \textbf{Acc} & \textbf{GPU (GB)} & \textbf{Time (s)} & \textbf{Acc} \\ \midrule 
         52.67                    & 13.82             & $57.09$      & 61.05                    & 20.17             & $53.98$      & 38.28                    & 0.75              & $56.39$      \\ \bottomrule
        \end{tabular}%
    }
    \label{tab:mcmc-comp}
\end{table}

\subsection{Comparison of Posterior Quality}
\label{sec:comp-post-qual}
Lastly, we provide a quantitative comparison for the quality of posteriors obtained using our method and other baselines to substantiate our claims. Note that the true parameter posteriors are intractable and are generally unknown for complex deep networks. Hence, it is difficult to make comparisons against such gold standard posterior for complex networks. However, to provide a comprehensive understanding of the quality of our obtained posteriors, we report the Expected Calibration Error (ECE)~\citep{ECE} and Brier score~\citep{brier}. These metrics, akin to those presented in Table~6 of \citet{BPC}, serve as well-established benchmarks for evaluating posterior quality. These results are listed in Table~\ref{tab:callib-results}. We see that the proposed method performs the best amongst all the baselines, further ensuring that the proposed method is effective in the qualitative sense as well. 

\begin{table}[!t]
    \centering
    \caption{Comparison of ECE ($\downarrow$) and Brier Score ($\downarrow$) for the proposed method against other baselines}
    \resizebox{\columnwidth}{!}{
    \begin{tabular}{rr|rr|rr|rr}
        \toprule
                          \multicolumn{2}{c|}{\textbf{BPC-rkl}}     & \multicolumn{2}{c|}{\textbf{BPC-W}}       & \multicolumn{2}{c|}{\textbf{BPC-fkl}}     & \multicolumn{2}{c}{\textbf{Ours}}                        \\ 
                          \multicolumn{1}{c}{\textbf{ECE}}       & \multicolumn{1}{c|}{\textbf{Brier Score}} & \multicolumn{1}{c}{\textbf{ECE}}       & \multicolumn{1}{c|}{\textbf{Brier Score}} & \multicolumn{1}{c}{\textbf{ECE}}       & \multicolumn{1}{c|}{\textbf{Brier Score}} & \multicolumn{1}{c}{\textbf{ECE}}                & \multicolumn{1}{c}{\textbf{Brier Score}}        \\ \midrule
         $0.1183\pm 0.0038$ & $0.7988\pm 0.0038$   & $0.1457\pm 0.0110$ & $0.8030\pm 0.0049$   & $0.1538\pm 0.0049$ & $0.7231\pm 0.0049$   & $\mathbf{0.1092\pm 0.0052}$ & $\mathbf{0.6755\pm 0.0042}$ \\ \bottomrule
    \end{tabular}
    }
    \label{tab:callib-results}
\end{table}


\section{Conclusion}

In this work, we propose a novel approach to generate pseudo-coreset using contrastive divergence. Our approach addresses the need to approximate the posterior of pseudo-coreset and uses a finite number of steps in MCMC methods to sample the parameters from the underlying posterior distribution.
Subsequently, these parameters are used to construct pseudo-coreset via contrastive divergence. The empirical evidence presented in our study illustrates that our proposed method surpasses previous BPC baselines by substantial margins across multiple datasets.


\textbf{\textit{Limitations and Future Work}:}
While our approach effectively removes variational assumptions associated with the pseudo-coreset posterior and utilizes MCMC methods for parameter sampling, our study still relies on certain assumptions about the posterior of the original dataset. Since there remains a significant performance gap between the pseudo-coreset and the original dataset, a potential avenue for future research could be to relax these assumptions to enhance the performance of BPC methods.


\textbf{\textit{Broader Impact}:}
% \textcolor{red}{
BPC methods have positive applications in democratization and privacy-related concerns by reducing the dependence on the original dataset. 
We don't believe that our method has any associated negative societal impact. 






\begin{acknowledgements}
    This work was supported (in part for setting up the GPU compute) by the Indian Institute of Science through a start-up grant. Prathosh is supported by Infosys Foundation Young investigator award. Piyush is supported by Government of India via Prime minister’s research fellowship.
\end{acknowledgements}










































% \section{Back Matter}
% There are a some final, special sections that come at the back of the paper, in the following order:
% \begin{itemize}
%   \item Author Contributions (optional)
%   \item Acknowledgements (optional)
%   \item References
% \end{itemize}
% They all use an unnumbered \verb|\subsubsection|.

% For the first two special environments are provided.
% (These sections are automatically removed for the anonymous submission version of your paper.)
% The third is the ‘References’ section.
% (See below.)

% (This ‘Back Matter’ section itself should not be included in your paper.)


% \begin{contributions} % will be removed in pdf for initial submission 
% 					  % (without ‘accepted’ option in \documentclass)
%                       % so you can already fill it to test with the
%                       % ‘accepted’ class option
%     Briefly list author contributions. 
%     This is a nice way of making clear who did what and to give proper credit.
%     This section is optional.

%     H.~Q.~Bovik conceived the idea and wrote the paper.
%     Coauthor One created the code.
%     Coauthor Two created the figures.
% \end{contributions}

% \begin{acknowledgements} % will be removed in pdf for initial submission,
% 						 % (without ‘accepted’ option in \documentclass)
%                          % so you can already fill it to test with the
%                          % ‘accepted’ class option
%     Briefly acknowledge people and organizations here.

%     \emph{All} acknowledgements go in this section.
% \end{acknowledgements}

% References
\balance
\bibliography{uai2024-template}

















\newpage

\onecolumn

\title{Bayesian Pseudo-Coresets via Contrastive Divergence\\(Supplementary Material)}
\maketitle
\appendix


\section{Training details and Hyper parameters}
\label{sec:app-training-details}
In this section, we provide the implementation details of the proposed method. Our implementation can be found \href{https://github.com/backpropagator/BPC-CD}{https://github.com/backpropagator/BPC-CD}. During the training process, we randomly initialize the synthetic dataset using samples from the original training set. The overall cardinality of these synthetic sets is determined by the number of images considered for every class (ipc). For our experiments, we have considered ipc values of 1, 10 and 50. Furthermore, similar to previous works~\cite{BPC,MTT}, we have used Differentiable Siamese Augmentation (DSA)~\citep{DSA} strategies to enhance the performance of our model. DSA strategies include random crop, random flip, random brightness, random scale, and rotation. At any instant, we apply one of these augmentations to train our network. These augmentation techniques ensure that the model does not overfit on the given synthetic set and generates optimal parameters. DSA is applied to the synthetic set while running langevin dynamics and calculating the contrastive-divergence-based loss function. 
%We observe that DSA plays an important role in boosting the performance of our method. This is shown as ablation in Table~\ref{tab:dsa-ablation}, where we observe the performance of our method with and without DSA on CIFAR10 for ipc 1 and 10. This is in line with the observations made in~\citep{dc-bench}. 

We have also conducted additional experiments to evaluate the effectiveness of our method and other BPC baselines in the absence of DSA. The findings of the experiment are reported in Table~\ref{tab:dsa_comparison}. The results clearly indicate that the DSA  has a positive impact on the performance of all BPC methods, which align with the observation made by~\citet{dc-bench}. Nevertheless, even without the DSA-based augmentation strategy, our method outperforms other BPC baselines for 1 ipc.

As for the network used to calculate the energy, we take inspiration from previous works~\citep{BPC,DD,DC_CS,MTT} and use a ConvNet architecture~\citep{convnet}. This architecture consists of multiple blocks of convolutional layer with filter dimension of  $3\times3$ and channel size of $128$. The network uses instance normalization, maxpool layer with stride $2$, and RELU activation. In our experiments, we have used an architecture with three such blocks of convolution layers.


Next, we create a buffer of trajectories to sample parameters from the posterior of the original dataset. For this, we  generate $100$ different trajectories, each with $50$ epochs trained using SGD optimizer with a batch size of $256$ on the original training set. These parameters are used to obtain the gaussian variational approximation to estimate the loss function. Further, we use diagonal covariance matrix with diagonal entry of $0.001$ for the re-parameterization trick used in gaussian approximation.

The pseudocode of the implementation is presented in Algorithm~\ref{alg:the_alg}. The hyperparameters used are as follows: $P = 2000$, $\lambda = 0.01$, $n = 50$, $L = 100$, $\Sigma_{\rvx}^{1/2} = 0.001I$. These hyper-parameters are fixed across all the datasets. Further, we observe that there $\gamma$ that can be fine-tuned for marginal improvements in performance. Specifically, $\gamma$ is varied between $\{1, 10, 100, 1000\}$. Note that, we use same set of parameter to sample parameters during inference.
All the experiments are conducted on a single NVIDIA RTX A6000 GPUs with 48GB memory.



\begin{table}[!h]
\centering
\caption{Comparison of the proposed method against other BPC baselines without using DSA on CIFAR10 dataset.}
\label{tab:dsa_comparison}
\resizebox{\columnwidth}{!}{%
\begin{tabular}{rr|rr|rr|rr|rr}
\hline
\multicolumn{2}{c|}{\textbf{BPC-rkl (sghmc)}} & \multicolumn{2}{c|}{\textbf{BPC-W (sghmc)}} & \multicolumn{2}{c|}{\textbf{BPC-fkl (hmc)}} & \multicolumn{2}{c|}{\textbf{BPC-fkl (sghmc)}} & \multicolumn{2}{c}{\textbf{Ours}}                    \\ \hline
\multicolumn{1}{c}{ipc =1}                 & \multicolumn{1}{c|}{ipc = 10}              & \multicolumn{1}{c}{ipc =1}               & \multicolumn{1}{c|}{ipc = 10}             & \multicolumn{1}{c}{ipc =1}               & \multicolumn{1}{c|}{ipc = 10}             & \multicolumn{1}{c}{ipc =1}                & \multicolumn{1}{c|}{ipc = 10}              & \multicolumn{1}{c}{ipc =1}                    & \multicolumn{1}{c}{ipc = 10}                  \\ \hline
$19.70 \pm 1.06$       & $36.41 \pm 0.75$      & $27.66 \pm 0.8$      & $39.61 \pm 1.12$     & $32.61 \pm 1.50$     & $38.12 \pm 1.19$     & $28.25 \pm 0.92$      & \boldsymbol{$41.85 \pm 1.47$}      & \boldsymbol{$34.94 \pm 0.72$} & $41.02 \pm 0.66$ \\ \hline
\end{tabular}%
}
\end{table}




\begin{algorithm}
\caption{Proposed Algorithm }
\label{alg:the_alg}
\textbf{Input :} Set of SGD trajectories obtained from original dataset $(\tau)$, Number of langevin steps ($L$) needed to sample parameter from $\pi_{\td{\rvx}}$, Langevin step size  ($\lambda$), Step size to modify pseudo-coreset ($\gamma$), Number of epochs (P)  \\
\begin{algorithmic}[1]
\State Initialize pseudo-coreset ($\td{x}$) using samples from original dataset $x$.

\For {step in [1... P] : }
\State Sample  $\tau_i \sim \tau$ 
\State Sample $\theta_k^+ \sim \tau_i$ where $\theta_k^+$ are parameters associated with $k^{th}$ epoch for $i^{th}$ trajectory.
\State Let $\theta^+$ = $\theta_{k}^{+} + \Sigma_{\rvx}^{1/2}\varepsilon_{\rvx}$, $\varepsilon \sim \gN(0,I)$
\State Let $\theta^{-}_{0}  = \theta^{+} $
\For {t in [0 .... L] :  }
    \State Calculate energy associated with $\td{\rvx}$ and parameter $\theta^{-}_{t}$ i.e. $E(\theta^{-}_{t},\td{\rvx})$)
    
    \State  $\theta^{-}_{t+1} = \theta^{-}_{t} - \lambda(\nabla_{\theta}E(\theta^{-}_{t},\td{\rvx})) + \eta \text{,}\hspace{1.5mm} \eta \sim {\gN}(0,I)$

\EndFor
 \State Let $\theta^-$ = $\theta^-_{L}$
 \State Calculate  $ \gL = E((\theta^{+},\td{\rvx})) - E({(\theta^{-} , \td{\rvx} )}) $
 \State $ \td{\rvx} \gets \td{\rvx} - \gamma \nabla_{\td{\rvx}}  \gL $
 \EndFor

\end{algorithmic}
\end{algorithm}

\section{Experimental Setup}

{\subsection{Baseline Setup}
We primarily present the results for different BPC frameworks. The experiment of Table~\ref{sota-comp-BPC} in the main manuscript uses the original hyperparameters mentioned in the respective papers. In cases where hyperparameters were not explicitly specified, we employed the default hyperparameters of CIFAR10. 
We have presented the results for BPC methods with only 1 and 10 ipc for the CIFAR100 and T-ImageNet datasets. We could not report the result for other scenarios due to the computational limitations. These methods demand a significant amount of GPU memory, which we currently lack, making it impractical to compute the desired results.


\subsection{GPU and Time Consumption}
We assess the computational efficiency of our method relative to other baselines by comparing the GPU memory consumption and the training time required to generate the pseudo-coresets for a single iteration. The findings of our results are presented in Fig.~\ref{gpu-time}, where the iteration time is calculated by taking the average of the total time for 100 different iterations.

As illustrated in Fig.~\ref{gpu-time}, our method requires relatively less time compared to other BPC methods for low ipc values and outperforms BPC-W for higher ipc values. Additionally, in our examination of GPU memory usage, we observe that BPC-W shows linear scaling in GPU memory consumption as the number of images per class increases. In contrast, our method maintains consistent memory usage across all ipc values. In our experiment, we found that our method utilizes only 37GB of memory, even for higher images per class. It's worth noting that other BPC baselines such as BPC-fKL and BPC-rKL are more memory-efficient than our method and deliver consistent performance across all the ipc values. We attribute this observation to the fact that BPC-fkl and BPC-rkl avoid MCMC sampling during training by making use of relevant approximations. Whereas, our method makes use of gradient-based MCMC sampling (langevin dynamics) for estimation of the objective function. For this reason, the GPU consumption of the proposed method is relatively higher than that of BPC-fkl and BPC-rkl.

Further, for a pseudo-coreset of size $m$, we run langevin-dynamics for $k$ steps which leads to a complexity of $\mathcal{O}(km)$. Compared to inference on entire dataset of size $n$, the same would have a complexity of $\mathcal{O}(kn)$. Since $m\ll n$, pseudo-coresets are much more efficient for inference compared to full data. Since the pseudo-coreset size ($m$) is fixed during inference and only final value of each iteration is required in the consequent iteration of langevin dynamics, the memory complexity is just $\mathcal{O}(m)$. Again since $m\ll n$, pseudo-coresets are efficient in term of memory as well compared to full data inference.

 % We also analyze the GPU memory costs required for training different methods with increased images per class. We observe that methods like the BPC-W scale linearly as we increase the images per class, but our method performs identically across all ipc. Our approach can efficiently allocate only 37GB of memory for higher images per class. BPC-fKL and BPC-rKL prove to be much more efficient method than ours, performing equally across all ipc.

\begin{figure*}[!t]
\centering
   \begin{subfigure}[t]{0.49\textwidth}
   \includegraphics[keepaspectratio, width=\textwidth]{figs/gpu/gpu_33.png}
    % \caption{MNIST}
   \caption{\label{fig:gpu-mem}}
   \end{subfigure}
~
    \begin{subfigure}[t]{0.49\textwidth}
   \includegraphics[keepaspectratio, width=\textwidth]{figs/gpu/time_3.png}
   \caption{\label{fig:gpu-iter}}
    \end{subfigure}
\caption{\label{gpu-time}Computing GPU memory costs along with training time for different image per class.}
% \todo{draw another dotted line in 1a showing the memory usage while training a model on whole CIFAR10. Then say that condensing the data is much cheaper that training on whole data making the motivation strong.}
\end{figure*}


\section{Comparison with coreset  methods}
We have conducted a comparative analysis of our method with other coreset techniques such as Herding~\citep{Herding}, K-Center~\citep{sener2017active}, and Forgetting~\citep{Forgetting}. The outcomes of our experiments are listed in Table~\ref{tab:sota-comp-coreset}. The result clearly shows that our method outperforms other coreset techniques on all the dataset.   
\begin{table*}[ht!]
\centering
  \caption{\label{tab:sota-comp-coreset}Comparison of the proposed method with coreset baselines. The results are noted in form of (mean $\pm$ std. dev) where we have obtained test accuracy over five independent runs on the pseudo-coreset. The best performer across all methods is denoted in bold ($\boldsymbol{x\pm s}$). For ease of comparison, we color the second best performer with \textcolor{blue}{blue} color.}
\resizebox{0.75\textwidth}{!}
 {
 \renewcommand{\arraystretch}{1.1}
\begin{tabular}{l|rr|rrr|r}
 \hline
 &
   \multicolumn{1}{c}{\textbf{ipc}} &
   \multicolumn{1}{c|}{\textbf{Ratio(\%)}} &
   \multicolumn{1}{c}{\textbf{Herding}} &
   \multicolumn{1}{c}{\textbf{K-Center}} &
   \multicolumn{1}{c|}{\textbf{Forgetting}} &
   \multicolumn{1}{c}{\textbf{Ours}} 
   \\ \hline
 &
  1 &
  0.017 &
  $89.2 \pm 1.6$ &
  {\color{blue}{$89.3 \pm 1.5$ }}&
  $35.5 \pm 5.6$ &
 \boldsymbol{$93.42 \pm 0.09$} 
   \\
 &
  10 &
  0.17 &
  {\color{blue}{${93.7} \pm 0.3$}} &
  $84.4 \pm 1.7$ &
  $68.1 \pm 3.3$ &
  \boldsymbol{$97.71 \pm 0.24$} 
   \\
 \textbf{\multirow{-3}{*}{\textbf{MNIST}}} &
  50 &
  0.83 &
  $94.8 \pm 0.2$ &
  {\color{blue}{${97.4} \pm 0.3$}} &
  $88.2 \pm 1.2$ &
  \boldsymbol{$98.91 \pm 0.22$} 
   \\ \hline
 &
  1 &
  0.017 &
  {\color{blue}{${67.0} \pm 1.9$}} &
  $66.9 \pm 1.8$ &
  $42.0 \pm 5.5$ &
   \boldsymbol{$77.29 \pm 0.50$}
   \\
 &
  10 &
  0.17 &
  {\color{blue}{${71.1} \pm 0.7$}} &
  $54.7 \pm 1.5$ &
  $53.9 \pm 2.0$ &
  \boldsymbol{$88.40 \pm 0.21$} 
   \\
\multirow{-3}{*}{\textbf{FMNIST}} &
  50 &
  0.83 &
  {\color{blue}{${71.9} \pm 0.8$}} &
  $68.3 \pm 0.8$ &
  $55.0 \pm 1.1$ &
  \boldsymbol{$89.47 \pm 0.06$} 
   \\ \hline
 &
  1 &
  0.014 &
  $20.9 \pm 1.3$ &
  {\color{blue}{${21.0} \pm 1.5$}} &
  $12.1 \pm 1.7$ &
  \boldsymbol{$66.74 \pm 0.09$} 
   \\
 &
  10 &
  0.14 &
  {\color{blue}{${50.5} \pm 3.3$}} &
  $14.0 \pm 1.3$ &
  $16.8 \pm 1.2$ &
  \boldsymbol{$82.32 \pm 0.56$} 
   \\
 \multirow{-3}{*}{\textbf{SVHN}}&
  50 &
  0.7 &
  {\color{blue}{${72.6} \pm 0.8$}} &
  $20.1 \pm 1.4$ &
  $27.2 \pm 1.5$ &
  \boldsymbol{$88.41 \pm 0.12$} 
   \\ \hline
 &
  1 &
  0.02 &
  $21.5 \pm 1.2$ &
  {\color{blue}{${21.5} \pm 1.3$}} &
  $13.5 \pm 1.2$ &
  \boldsymbol{$46.87 \pm 0.20$} 
   \\
 &
  10 &
  0.2 &
  {\color{blue}{${31.6} \pm 0.7$}} &
  $14.7 \pm 0.9$ &
  $23.3 \pm 1.0$ &
  \boldsymbol{$56.39 \pm 0.70$} 
   \\
\multirow{-3}{*}{\textbf{Cifar10}} &
  50 &
  1 &
  $23.3 \pm 1.0$ &
  {\color{blue}{${27.0} \pm 1.4$}} &
  $23.3 \pm 1.1$ &
  \boldsymbol{$71.93 \pm 0.17$} 
   \\ \hline
 &
  1 &
  0.2 &
  {\color{blue}{${8.4} \pm 0.3$}} &
  $8.3 \pm 0.3$ &
  $4.5 \pm  0.2$ &
  \boldsymbol{$23.97 \pm 0.11$} 
   \\
\multirow{-2}{*}{\textbf{Cifar100}} &
  10 &
  2 &
  {\color{blue}{${17.3} \pm 0.3$}} &
  $7.1 \pm 0.2$ &
  $15.1 \pm 0.3$ &
  \boldsymbol{$28.42 \pm 0.24$} 
   \\ \hline
 &
  1 &
  0.2 &
  $2.8 \pm 0.2$ &
  {\color{blue}{${3.03} \pm 0.1$}} &
  $1.6 \pm 0.1$ &
  \boldsymbol{$8.39 \pm 0.07$} 
   \\
\multirow{-2}{*}{\textbf{T-ImageNet}} &
  10 &
  2 &
  $6.3 \pm 0.2$ &
  {\color{blue}{${11.38} \pm 0.1$}} &
  $5.1 \pm 0.2$ &
  \boldsymbol{$17.82 \pm 0.39$} 
   \\ \hline
  
\end{tabular}
}
\end{table*}


\section{Comparison with dataset condensation techniques}

\begin{table*}[!t]
    \centering
    \caption{Efficiency comparison of the proposed method with several dataset condensation methods on CIFAR10.}
    \resizebox{\textwidth}{!}{
            \begin{tabular}{l|rr|rr|rr|rr}
\toprule
\multirow{2}{*}{\textbf{ipc}} & \multicolumn{2}{c|}{\textbf{GM}}                                                      & \multicolumn{2}{c|}{\textbf{DSA}}                                                     & \multicolumn{2}{c|}{\textbf{MTT}}                                                     & \multicolumn{2}{c}{\textbf{Ours}}                                                    \\
                              & \multicolumn{1}{c}{\textbf{GPU Memory (GB)}} & \multicolumn{1}{c|}{\textbf{Time (s)}} & \multicolumn{1}{c}{\textbf{GPU Memory (GB)}} & \multicolumn{1}{c|}{\textbf{Time (s)}} & \multicolumn{1}{c}{\textbf{GPU Memory (GB)}} & \multicolumn{1}{c|}{\textbf{Time (s)}} & \multicolumn{1}{c}{\textbf{GPU Memory (GB)}} & \multicolumn{1}{c}{\textbf{Time (s)}} \\ \midrule
\textbf{1}                    & $38.88$                                      & $0.4$                                  & $43.09$                                      & $0.3$                                 & $44.82$                                      & $1.2$                                  & $\mathbf{37.10}$                                      & $\mathbf{0.71}$                                 \\
\textbf{10}                   & $42.78$                                      & $11.4$                                 & $46.06$                                      & $7.5$                                  & $50.98$                                      & $12.7$                                 & $\mathbf{38.28}$                                      & $\mathbf{0.75}$                                 \\
\textbf{50}                   & $46.39$                                      & $32.9$                                 & $53.20$                                       & $24.4$                                 & $68.10$                                      & $26.8$                                 & $\mathbf{38.44}$                                      & $\mathbf{1.26}$                                 \\ \bottomrule
\end{tabular}%
}
\label{tab:eff-comp-dc}
\end{table*}


 \begin{table*}[!ht]

  \caption{\label{sota-comp-condense}Comparison of the proposed method with dataset-condensation baselines. The results are noted in form of (mean $\pm$ std. dev) where we have obtained test accuracy over five independent runs on the pseudo-coreset. The best performer across all methods is denoted in bold ($\boldsymbol{x\pm s}$). For ease of comparison, we color the second best performer with \textcolor{blue}{blue} color.}
 \resizebox{\textwidth}{!}
 {
 \renewcommand{\arraystretch}{1.1}
\begin{tabular}{l|rr|rrrrrrrrr|r}
\hline
 &
  \multicolumn{1}{c}{\textbf{Img/cls}} &
  \multicolumn{1}{c|}{\textbf{Ratio\%}} &
  \multicolumn{1}{c}{\textbf{DD}} &
  \multicolumn{1}{c}{\textbf{LD}} &
  \multicolumn{1}{c}{\textbf{GM}} &
  \multicolumn{1}{c}{\textbf{DSA}} &
  \multicolumn{1}{c}{\textbf{DM}} &
  \multicolumn{1}{c}{\textbf{CAFE}} &
  \multicolumn{1}{c}{\textbf{CAFE+DSA}} &
  \multicolumn{1}{c}{\textbf{KIP}} &
  \multicolumn{1}{c|}{\textbf{MTT}} &
  \multicolumn{1}{c}{\textbf{Ours}} \\ \hline
 &
  1 &
  0.017 &
  - &
  $60.60 \pm 2.86$ &
  $92.01 \pm 0.25$ &
  $87.60 \pm 0.07$ &
  $88.89 \pm 0.57$ &
  {\color{blue}{${93.10} \pm 0.30$}} &
  $90.80 \pm 0.50$ &
  $85.46 \pm 0.04$ &
  $89.85 \pm 0.01$ &
  \boldsymbol{$93.42 \pm 0.09$} \\
 &
  10 &
  0.17 &
  $79.71 \pm 8.3$ &
  $87.05 \pm 0.50$ &
  $97.58 \pm 0.10$ &
  $97.39 \pm 0.06$ &
  $96.58 \pm 0.11$ &
  $97.20 \pm 0.20$ &
  $97.50 \pm 0.10$ &
  $97.15 \pm 0.11$ &
  {\color{blue}{${97.70} \pm 0.02$}} &
  \boldsymbol{$97.71 \pm 0.24$} \\
\multirow{-3}{*}{\textbf{MNIST}} &
  50 &
  0.83 &
  - &
  $93.30 \pm 0.30$ &
  $98.81 \pm 0.03$ &
  $98.97 \pm 0.04$ &
  $98.22 \pm 0.05$ &
  $98.60 \pm 0.20$ &
  {\color{blue}{${98.90} \pm 0.20$}} &
  $98.36 \pm 0.08$ &
  $98.6 \pm 0.01$ &
  \boldsymbol{$98.91 \pm 0.22$} \\ \hline
 &
  1 &
  0.017 &
  - &
  - &
  $70.83 \pm 0.01$ &
  $70.45 \pm 0.57$ &
  $71.92 \pm 0.70$ &
  $77.10 \pm 0.90$ &
  $73.70 \pm 0.70$ &
  - &
  {\color{blue}{${77.14} \pm 0.01$}} &
  \boldsymbol{$77.29 \pm 0.50$} \\
 &
  10 &
  0.17 &
  - &
  - &
  $81.93 \pm 0.07$ &
  $84.70 \pm 0.11$ &
  $83.25 \pm 0.09$ &
  $83.00 \pm 0.40$ &
  $83.00 \pm 0.30$ &
  - &
  \boldsymbol{${88.76} \pm 0.01$} &
  {\color{blue}{$88.40 \pm 0.21$}} \\
\multirow{-3}{*}{\textbf{FMNIST}} &
  50 &
  0.83 &
  - &
  - &
  $83.26 \pm 0.17$ &
  $88.55 \pm 0.56$ &
  $87.65 \pm 0.03$ &
  $84.80 \pm 0.40$ &
  $88.20 \pm 0.30$ &
  - &
  {\color{blue}{${89.33} \pm 0.15$}} &
  \boldsymbol{$89.47 \pm 0.06$} \\ \hline
 &
  1 &
  0.014 &
  - &
  - &
  $30.49 \pm 0.57$ &
  $31.18 \pm 0.43$ &
  $19.25 \pm 1.39$ &
  $42.60 \pm 3.30$ &
  $42.90 \pm 3.01$ &
  - &
  {\color{blue}{${57.55} \pm 0.02$}} &
  \boldsymbol{$66.74 \pm 0.09$} \\
 &
  10 &
  0.14 &
  - &
  - &
  $75.10 \pm 0.40$ &
  $78.39 \pm 0.3$ &
  $71.42 \pm 1.01$ &
  $75.90 \pm 0.60$ &
  {\color{blue}{${77.90} \pm 0.60$}} &
  - &
  $72.56 \pm 0.01$ &
  \boldsymbol{$82.32 \pm 0.56$} \\
\multirow{-3}{*}{\textbf{SVHN}} &
  50 &
  0.7 &
  - &
  - &
  $81.70 \pm 0.14$ &
  $82.50 \pm 0.34$ &
  $82.41 \pm0.52$ &
  $81.30 \pm 0.30$ &
  $82.30 \pm 0.40$ &
  - &
  {\color{blue}{${83.73} \pm 0.33$}} &
  \boldsymbol{$88.41 \pm 0.12$} \\ \hline
 &
  1 &
  0.02 &
  - &
  $25.38 \pm 0.2$ &
  $28.10 \pm0.56$ &
  $29.01 \pm 0.64$ &
  $26.40 \pm 0.42$ &
  $30.30 \pm 1.10$ &
  $31.60 \pm 0.80$ &
  $40.50 \pm 0.40$ &
  {\color{blue}{${46.08} \pm 0.80$}} &
  \boldsymbol{$46.87 \pm 0.2$} \\
 &
  10 &
  0.2 &
  $39.14 \pm 2.30$ &
  $37.50 \pm 0.60$ &
  $44.14 \pm 0.60$ &
  $51.85 \pm 0.43$ &
  $48.66 \pm 0.03$ &
  $46.30 \pm 0.60$ &
  $50.90 \pm 0.50$ &
  $53.10 \pm 0.50$ &
  \boldsymbol{${64.27} \pm 0.80$} &
  {\color{blue}{$56.39 \pm 0.70$}} \\
\multirow{-3}{*}{\textbf{CIFAR10}} &
  50 &
  1 &
  - &
  $41.70 \pm 0.50$ &
  $53.73 \pm 0.44$ &
  $60.77 \pm 0.45$ &
  $62.70 \pm 0.07$ &
  $55.50 \pm 0.60$ &
  $63.30 \pm 0.40$ &
  $58.60 \pm 0.40$ &
  {\color{blue}{${71.26} \pm 0.50$}} &
  \boldsymbol{$71.93 \pm 0.17$} \\ \hline
 &
  1 &
  0.2 &
  - &
  $11.50 \pm 0.40$ &
  $12.65 \pm 0.32$ &
  $13.88 \pm 0.29$ &
  $11.35 \pm 0.18$ &
  $12.04 \pm 0.01$ &
  $12.90 \pm 0.30$ &
  $14.01 \pm 0.30$ &
  {\color{blue}{${23.62} \pm 0.63$}} &
  \boldsymbol{$23.97 \pm 0.11$} \\
\multirow{-2}{*}{\textbf{CIFAR100}} &
  10 &
  2 &
  - &
  - &
  $25.28 \pm 0.29$ &
  $32.34 \pm 0.40$ &
  $29.38 \pm 0.26$ &
  $29.04 \pm 0.01$ &
  $27.80 \pm 0.30$ &
  {\color{blue}{$31.50 \pm 0.20$}} &
  \boldsymbol{${36.96} \pm 0.15$} &
  $28.42 \pm 0.24$ \\ \hline
 &
  1 &
  0.2 &
  - &
  - &
  $5.27 \pm 0.01$ &
  $5.67 \pm 0.01$ &
  $3.82 \pm 0.01$ &
  - &
  - &
  - &
  {\color{blue}{${8.27} \pm 0.01$}} &
  \boldsymbol{$8.39 \pm 0.07$} \\
\multirow{-2}{*}{\textbf{T-ImageNet}} &
  10 &
  2 &
  - &
  - &
  $12.83 \pm 0.01$ &
  $16.43 \pm 0.02$ &
  $13.51 \pm 0.01$ &
  - &
  - &
  - &
  \boldsymbol{${20.11} \pm 0.02$} &
  {\color{blue}{$17.82 \pm 0.39$}} \\ \hline
\end{tabular}
 }
 \end{table*}

We also compare our method with other data condensation (DC) techniques like Distillation (DD)~\citep{DD}, Flexible Dataset Distillation (LD)~\citep{LD}, Gradient Matching (DC)~\citep{GM}, Differentiable Siamese Augmentation (DSA)~\citep{DSA}, Distribution Matching (DM)~\citep{DM}, Neural Ridge Regression (KIP)~\citep{KIP}, Condensed data to align features (CAFE)~\citep{CAFE} and Matching Training Trajectories (MTT)~\citep{MTT}. 
\par There are few fundamental differences between dataset condensation methods and BPC methods. We first enumerate these differences here for clarity:
\begin{enumerate}
    \item \textbf{Objective function}: One of the key differences between Dataset Condensation (DC) and BPC is the underlying formulation and the loss objective. Dataset condensation methods mostly rely on heurist objective function to `\textbf{match the performance}' of synthetic data and original data; where the measure of performance matching varies for different methods. For e.g., GM~\citep{GM} relies on matching the gradient direction of a model trained on synthetic data with a model trained on the original data; similarly, MTT~\citep{MTT} relies on matching SGD trajectories of the synthetic data and original data. Hence, these methods don't have a principled way of coming up with loss objective. However, BPC methods on the other hand follow a single principle for loss objective - a divergence measure between true posterior and pseudo-coreset posterior, which is much more principled. Different BPC method opt for different divergence measure, in context of our work, we choose to work with contrastive divergence for the reasons outlined in the paper.
    \item \textbf{Bayesian v/s Non-Bayesian}: Another fundamental difference between the two domains is that BPC methods are purely bayesian, particularly, they treat parameters of the network as random variable and work on matching the distribution of this random variable via divergence minimization. On the other hand, DC methods are non-bayesian and work with point estimates of models.
    \item \textbf{Optimization Strategy}: Finally, the pivotal difference between BPC methods and DC methods, that makes the former more efficient is the optimization strategy employed. Since DC methods rely on `performance matching', it is inadvertent to train a model on synthetic and original data, then match the performance using appropriate metric. To learn the synthetic data in this case, one has to backpropagate the gradients through the model training steps as well. This would require computation of second-order derivatives. This is what we refer to as bi-level optimization. However, in case of BPC methods like ours, there is no such need of higher order derivatives. This is because we don't use bi-level optimization for BPC construction. This difference in optimization strategy makes BPC much more efficient in practice. The quantitative results to illustrate this claim can be found in Table~\ref{tab:eff-comp-dc}, where we compare the GPU memory consumed and time taken for every iteration by different methods. It can be seen that dataset condensation consume significant memory and time. 
\end{enumerate}
\par The performance comparison results are shown in Table~\ref{sota-comp-condense}. We find that the performance of our method is better than almost all the DC baselines, whereas MTT stands out to be a close second in most of the cases. This shows that our method, although falling under the category of Bayesian pseudo-coreset, achieves a performance that is comparable to that of heuristic DC methods. It is to be noted that the DC methods are not the direct competitors of our method. However, we have shown that our method, although a BPC, surpasses (or comes very close to) the SoTA DC methods such as MTT~\citep{MTT}.
% This can be ascribed to the ability of our method in incorporating flexible distributions to posteriors, unlike the previous methods.



\section{Visualizations for CIFAR100 and Tiny-ImageNet}
In this section, we present the visualizations for pseudo-coresets of large datasets like CIFAR100 and Tiny-Imagenet datasets. We present generated synthetic images for both 1 and 10 images per class. We provide the visualization for 1 image per class on both datasets in Fig.~\ref{CIFAR100-1} and Fig.~\ref{Tiny-1},  respectively. Fig.~\ref{CIFAR100-10-1} and Fig.~\ref{CIFAR100-10-2} include visualization for CIFAR100 datasets with 10 ipc wherein each image is divided based on the number of classes. Similarly, we split the image into 50 classes for the Tiny-ImageNet dataset for 10 ipc in Fig.~\ref{Tiny-10-1} and Fig.~\ref{Tiny-10-2}.


  
\begin{figure*}[!t]
  \centering
   \includegraphics[keepaspectratio, width=0.8\textwidth]{figs/cifar100/cifar100-ipc1.png}
    \caption{\label{CIFAR100-1}Visualizations of pseudo-coresets for CIFAR100 with 1 ipc.}
   % \label{fig:Ng2}
% \end{figure*}
% \begin{figure*}[!t]
\end{figure*}

\begin{figure*}[!t]
   \centering
    \includegraphics[keepaspectratio, width=0.6\textwidth]{figs/tiny/tiny-ipc1.png}
    \caption{\label{Tiny-1}Visualizations of pseudo-coresets for Tiny ImageNet with 1 ipc.}
   % \label{fig:Ng1} 
\end{figure*}

\begin{figure*}[!t]
   \centering
   \begin{subfigure}[t]{0.24\textwidth}
   % \centering
   \includegraphics[keepaspectratio,width=\textwidth]{figs/cifar100/cifar100-ipc10-0-50.png}
   \caption{\label{CIFAR100-10-1}Classes 0-50}
    % \caption{Tiny-1}
   % \label{fig:Ng1} 
   \end{subfigure}
   % \hfill
   ~~~~~~~~~~~~~~~~~~~~
   \begin{subfigure}[t]{0.24\textwidth}
   % \centering
   \includegraphics[keepaspectratio,width=\textwidth]{figs/cifar100/cifar100-ipc10-50-100.png}
   \caption{\label{CIFAR100-10-2}Classes 50-100}
    % \caption{Tiny-1}
   % \label{fig:Ng1} 
   \end{subfigure}
   \caption{Visualizations for pseudo-coresets for CIFAR100 with 10 ipc}
\end{figure*}


\begin{figure*}[!t]
   \centering
   \begin{subfigure}[t]{0.24\textwidth}
   % \centering
   \includegraphics[keepaspectratio,width=\textwidth]{figs/tiny/tiny-ipc10-0-50.png}
   \caption{Classes 0-50}
    % \caption{Tiny-1}
   % \label{fig:Ng1} 
   \end{subfigure}
   % \hfill
   ~~~~~~~~~~~~~~~~~~~~
   \begin{subfigure}[t]{0.24\textwidth}
   % \centering
   \includegraphics[keepaspectratio,width=\textwidth]{figs/tiny/tiny-ipc10-50-100.png}
   \caption{Classes 50-100}
    % \caption{Tiny-1}
   % \label{fig:Ng1} 
   \end{subfigure}
   
   \caption{\label{Tiny-10-1}Visualizations of psuedo-coresets for Tiny ImageNet with 10 ipc}
\end{figure*}

\begin{figure*}[!t]
   \centering
   \begin{subfigure}[t]{0.24\textwidth}
   % \centering
   \includegraphics[keepaspectratio,width=\textwidth]{figs/tiny/tiny-ipc10-100-150.png}
   \caption{Classes 100-150}
    % \caption{Tiny-1}
   % \label{fig:Ng1} 
   \end{subfigure}
   % \hfill
   ~~~~~~~~~~~~~~~~~~~~
   \begin{subfigure}[t]{0.24\textwidth}
   % \centering
   \includegraphics[keepaspectratio,width=\textwidth]{figs/tiny/tiny-ipc10-150-200.png}
   \caption{Classes 150-200}
    % \caption{Tiny-1}
   % \label{fig:Ng1} 
   \end{subfigure}
   \caption{\label{Tiny-10-2}Visualizations of pseudo-coresets for Tiny ImageNet with  10 ipc}
\end{figure*}

\end{document}
