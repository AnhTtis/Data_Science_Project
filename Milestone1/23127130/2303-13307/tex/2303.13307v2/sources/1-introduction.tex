
\section{Introduction}


% Paragraph 1: Background and motivation
Data visualization is ubiquitous for data exploration and analysis.
With the explosive popularity of mobile devices such as smartphones and smartwatches,
they have become one of the major platforms for people to view visualizations of various data~\cite{lee2021mobile}, which are often sensitive personal data.
For example, people may check a bar chart on a healthcare app showing one's health and fitness data, like sleeping hours and walking steps~\cite{applewatch, garmin}, and explore a pie chart via one's mobile bank service displaying their portfolio components~\cite{BoA}.
Unlike traditional desktop-based visualizations often used in a private environment (e.g., in the office or at home), mobile data visualizations can be explored by users anytime and anywhere.
Accordingly,
data privacy issues arise when people are viewing mobile data visualizations showing sensitive personal data in public areas (e.g., on a bus or a train), as shoulder surfers nearby can also easily peek at those visualizations displayed on the mobile device screen, which is commonly known as \textit{Shoulder Surfing} as shown in Figure~\ref{fig:image_sp}. Additionally, \textit{shoulder surfers} are individuals who engage in shoulder surfing~\cite{abdrabou2022understanding}.
Though it is possible to be more cautious when using mobile data visualizations in public space, prior research has shown that only $7\%$ of people are aware of  Shoulder Surfing cases in the wild~\cite{eiband2017understanding}.


% Paragraph 2: Existing studies
The most popular way to preserve data privacy on mobile devices is to attach a privacy film to their screens.
By restricting the visible range of a screen to a particular viewing angle, it can effectively prevent personal information leakage when shoulder surfers nearby are not within the visible range of angles.
% \alexin{A screen file differs from a screen protector as it not only prevents damage to the device screen but also restricts the user's visible range of the screen to a specific viewing angle, 
% thus effectively preventing personal information when others nearby are not within the visible range.}
Nevertheless, it takes extra cost to purchase a privacy film to attach it to the device screen.
Also, privacy can negatively affect the sensitivity of the screen and lower the visibility of all the applications on mobile devices~\cite{ali2014protecting}, whether the corresponding application data is sensitive or not.
% and it is considered undesirable by users to reduce the quality and sensitivity of the screen as a result of a privacy film~\cite{ali2014protecting}. Moreover, the film must be experienced by all applications on the phone since it cannot be customized.}
% \alexin{
% However, the lowered visibility can negatively impact the user experience across all applications~\cite{ali2014protecting}. Moreover, purchasing the privacy film incurs an additional extra cost.
Prior visualization research has started to address the data privacy issues of visualization~\cite{bhattacharjee2020privacy}, but mostly focuses on anonymizing the underlying individual data items. It is achieved by introducing uncertainty to either the data space~\cite{chou2016privacy,chou2019privacy,wang2017utility} or the visual mapping between data items and visual encodings~\cite{dasgupta2012conceptualizing,chou2016obfuscated,dasgupta2019guess,wang2015ambiguityvis}, making individual items indistinguishable and preventing identity and attribute disclosures.
However, these approaches cannot achieve privacy preservation for
% the \alexout{abstracted visual summary of overall data trend and distribution} 
the visualization itself, i.e., hiding the visualization from Shoulder Surfing,
which is also crucial for data privacy preservation.
Two recent promising studies, IllustionPin~\cite{papadopoulos2017illusionpin} and HideScreen~\cite{chen2019keep}, attempt to protect information on mobile devices.
They either utilize the concept of \textit{hybrid image}~\cite{papadopoulos2017illusionpin} to hide the PIN password~\cite{chen2019keep} or discretize the device screen into grid patterns to make on-screen texts or personal grayscale images blend into the black background.
However, they target grayscale natural images or PIN keypad images and cannot work for data visualizations that are often colorful~\cite{silva2011using} and involve different visual marks~\cite{munzner2014visualization}, like rectangles, circles, and lines.

% \yong{Need to double check if we have fixed all the issues mentioned by the reviewers on the introduction section.}

% Paragraph 3: Our approach

% \alexout{Our approach
% % % \alexout{achieves privacy-preserving} 
% enhances the privacy preservation of mobile data visualization by adjusting the spatial frequency and luminance contrast of colorful visual marks and texts in visualizations.
% Specifically, we 
% % propose leveraging masking schemes to alter the spatial frequency of visual marks, adjusting their visibility at different distances.
% develop adaptive masking schemes to alter the spatial frequency of different visual marks, such as rectangles, circles, and lines, which adjusts their visibility at different distances.
% Our adaptive masking schemes take into account the different properties of various visual marks, e.g., the areas of visual marks like bars and circles are often larger than those of lines,
% and are informed by our prior user studies.
% Accordingly, we propose different masking schemes and processing directions for line-based elements (e.g., lines, text and axes) and area-based elements (e.g., circles and rectangles), striking a balanced trade-off between visibility at proximity and privacy preservation at a short distance.
% Also, given that most visualizations are intrinsically colorful, we propose leveraging the CIE LAB color space~\cite{colorimetry2004report}, which has proven to be perceptually linear~\cite{munzner2014visualization}, to adjust the visualization luminance contrast while preserving the original color hues.}
To fill the research gap,
we propose a privacy-preserving visualization approach for mobile devices, which can guarantee the visibility of the visualization at proximity but hide visualizations when viewed from a certain distance or above (Figure~\ref{fig:image_sp}).
Our approach is inspired by the study of human visual perception~\cite{barten1999contrast}, which suggests that spatial frequency and luminance contrast primarily determine human visibility of visual stimuli~\cite{national1985emergent}. We also take into account the different properties of various visual marks~\cite{munzner2014visualization} and are informed by our preliminary user study to develop customized masking schemes for line-based marks (e.g., line, text, and axes) and area-based marks (e.g., circles and rectangles) in visualizations to enhance the privacy preservation while maintaining proximity visibility. Specifically, our method is a masking scheme that consists of two granularity  levels of operations aimed at enhancing  the privacy preservation of mobile data visualization. At the coarse-grained level, the method transforms the spatial frequency and luminance contrast of the input visualization image. This operation adjusts the visualization visibility and ensures shoulder surfers cannot access the visualization at a distance. Further, at the fine-grained level, we consider different characteristics of visual marks
% such as line-based and area-based marks, 
and propose customized masking schemes for them, balancing the visualization visibility at a close proximity and privacy preservation at a distance.
% This approach accounts for the unique features of different visual marks and provides targeting masking to strike a balance between proximity visibility and privacy preservation at a far viewing distance. 


% \alexin{Our method adjusts the spatial frequency and luminance contrast of visual marks and texts within visualizations to enhance privacy preservation. To this end, we develop adaptive masking schemes that alter the spatial frequency of visual marks to adjust their visibility at varying distances. Additionally, we leverage the CIELAB color space~\cite{colorimetry2004report}, which is perceptually linear~\cite{munzner2014visualization}, to adjust the luminance contrast between the masked visual marks and visualization background while preserving the original color hues. Additionally, we take into account the different properties of various visual marks (e.g., the areas of visual marks like bars and circles are often larger than those of lines) and are informed by our prior user study to customized masking schemes for line-based elements (e.g., lines, text, and axes) and area-based elements (e.g., circles and rectangles) to strike a balance between proximity visibility and privacy preservation at a far viewing distance.}

There is a large design space to alter masking schemes and luminance contrast for privacy preservation of mobile data visualizations.
To narrow the design scope, we first conducted a preliminary user study with 16 participants to identify suitable variable design choices. The study reveals that participants' perception of the resulting visualizations depends on the mask area or the luminance contrast between the visual marks and background.
A suitable range of mask areas and luminance contrast can achieve the best trade-off between visualization visibility in close proximity (e.g., 30cm) and privacy preservation at a far distance (e.g., 60cm).
% \yong{Songheng, please double check the numbers above.}
% Moreover, the participants' feedback has also motivated us further to apply a fine-grained level masking scheme for visualizations.
Also, the participants' feedback has motivated us to design adaptive fine-grained masking schemes for different visual marks.

% \yong{It is meaningless to simply remove the the last part of the sentence.Need a thorough check.}

Guided by our preliminary study's findings,
% we created a set of privacy-preserving visualizations and then carried out a user study to evaluate them with 18 participants. 
we further carried out a user study with another set of 18 participants to evaluate the complete version of our approach with the suitable variable configurations in comparison with baseline approaches (i.e., the original visualizations and the partial version of our approach).
The results show that our approach can achieve similar visibility with the original visualizations in  proximity, but 
enhance the privacy preservation of data visualizations when being viewed from a certain distance (i.e., 90cm), demonstrating its usefulness and effectiveness in terms of visualization privacy preservation on mobile devices.



\begin{figure}[htbp!]
    \centering
    \includegraphics[width=0.8\linewidth]{figures/Diagram_v4.pdf}
    \caption{An example application scenario of our method. The data owner (left side) in proximity can see the privacy-preserving visualization displayed on the smartphone, while the shoulder surfer (right side) at a far distance can not interpret the visualization content.}
    \vspace{-2em}
    \label{fig:image_sp}
\end{figure}


 







In summary, the main contributions of this paper can be summarized as follows:
\begin{itemize}
    \item We propose a novel perception-driven approach
    % to generate privacy-preserving visualizations on mobile devices, 
    for privacy-preserving mobile data visualization
    % which is achieved 
    by adjusting the spatial frequency and luminance contrast of visual marks, which, to the best of our knowledge, is the first of its kind.
    % and texts of a given visualization. 
    % To the best of our knowledge, this is the first approach for privacy-preserving mobile data visualizations.
  
  


    \item We conducted two
    user studies to inform the design of our approach and further extensively evaluate our approach in comparison with baseline approaches across different types of charts. The results demonstrate our approach's usefulness and effectiveness in terms of privacy preservation.
  
  
  \item We summarize the lessons we learned during the development of the proposed approach, which
  % , we hope, 
  can shed light on future research on the privacy-preservation of mobile data visualizations.
\end{itemize}