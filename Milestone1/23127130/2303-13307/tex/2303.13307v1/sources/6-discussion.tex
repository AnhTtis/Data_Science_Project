\section{Discussion}
\alexin{
The two
% \alexout{well-designed} 
user studies} 
% \yong{pls check my overleaf comments.}
demonstrated the effectiveness of our proposed masking scheme on different types of visualizations. However, during the experiment, we also identified some limitations and lessons, which will be discussed in this section. 

\textbf{Trade-off between privacy-preservation and cognitive efforts.} As discussed in Section~\ref{sec:study_2_results}, there is a trade-off between privacy-preservation and cognitive efforts. Our proposed masking scheme successfully achieved privacy-preserving data visualization on smartphones, but it also requires slightly more time and effort for the data owners to read the visualizations processed by our approach.
Given that mobile data visualizations are often used to show personal data (e.g., bank account information and health data), which is often sensitive, we argue that better privacy preservation is much more critical compared with the slight increase of perception time and effort.


\textbf{Limited colors under low luminance.}
CIELAB color space represents a color using three dimensions, i.e., L, A, B, where L is the luminance~\cite{connolly1997study} and A, B represents the color hue. By adjusting the value of L, we can easily control the luminance contrast between the background and visual marks. However, since CIELAB color space is based on the human vision system, the luminance level and color hue are correlated~\cite{kim2009modeling}. More specifically, with the decrease in luminance L, the available color hue space is also reduced. Consequently, the change of luminance also modifies the color hue, resulting in a slightly different color~\cite{braun1999paradigm}. However, the users are still able to identify the visual marks~\cite{munzner2014visualization}. 


% \textbf{Unknown performance for low-vision users.} Our studies only involved participants with normal vision capability and demonstrated effectiveness. Since our technique is built upon the human vision system, people with low vision might show different perception properties, so the proposed scheme's performance is unknown. \alexout{On the other hand, we are thinking about whether our technique can be applied to general visualization scenarios to help low-vision people gain a better visual perception.} \alexin{On the other hand, we can use our technique to maximize the residual vision of low-vision users when they read the visualizations. Due to the reduced response of the HVS of low-vision users to visual stimuli~\cite{rubin1989psychophysics}, we can utilize their contrast sensitivity function~\cite{chung2016comparing} to tailor contrast enhancements to crucial frequencies (e.g., text and visual marks) to improve their reading capability on visualizations.}

% \alexin{\textbf{More representative participants.} Both of our study participants are young adults. Despite their rating and response demonstrating the effectiveness of our method, we plan to carry out a more comprehensive user study in the future in which elders and teenagers will be invited to evaluate our method fully. In addition to age, these studies include many more male than female participants. While gender imbalance of participants is common in HCI user studies~\cite{offenwanger2021diagnosing}, we plan to include more female participants in our future evaluation to ensure an equal and inclusive gender practice.}

\textbf{Parameter configurations for different devices.}
% \alexout{Due to varying screen sizes and resolutions, 
% our privacy-preserving visualization approach requires suitable parameter configurations on a different devices, which is similar to prior privacy-preservation methods~\cite{papadopoulos2017illusionpin,chen2019keep}.
% However, it is one-off effort. Once the suitable parameters are found and configured, they can be used for all the subsequent visualizations of the same type.}
% To simplify testing conditions,
\alexin{To control the number of variables in the user studies,
our evaluation of the proposed approach was conducted on a single mobile device and
the viewing angle of participants are fixed, i.e., participants were sitting right in front of the mobile device (with different distances).
% maintained a consistent viewing angle for participants when viewing the mobile visualization.
}
\alexin{However, similar to prior privacy-preservation methods~\cite{papadopoulos2017illusionpin,chen2019keep},
our privacy-preserving visualization approach also requires suitable parameter configurations (e.g., luminance contrast and mask areas) on different devices due to the varying screen sizes and resolutions.
We argue that it is a one-off effort, and the observations of our study results (especially Study 1) can guide the optimal configuration on new devices.
Also, it will be interesting to develop an approach to achieve automated optimal parameter configuration across mobile devices with varying screen sizes and resolutions and also take different viewing angles into account, which is left as future work.
% However, it is important to note that the effectiveness of the approach may vary across different device types due to variations in screen size and resolution. Specifically, the number of pixels occupied by a visual element changes with different screen sizes, thereby affecting the calculation of spatial frequency. In addition, the maximum brightness of different devices can impact the human perception
% % of luminance contrast from 
% of
% visual marks. Despite these potential limitations, it is possible to treat screen resolution and brightness as adjustable parameters during the design of privacy-preserved visualizations. While this may require additional effort, such parameter configurations can be applied to subsequent visualizations on the same device type, thereby reducing the overall workload. To gain a more comprehensive understanding of the method's performance across different devices, our future work will extend the evaluation to multiple device types.
}

\textbf{Generalizability.}
\alexin{
Our proposed method provides a flexible solution to enhance privacy of mobile data visualizations without restricting the input visualization designs. It is capable of processing visualization images, is available for four commonly used visualizations, and does not rely on any specific visualization generation tool/package.
% \yong{What do you want to say here?}
Additionally, though our method is primarily designed for mobile visualization, it can be extended to visualizations on other non-mobile contexts with appropriate parameter settings, for example, visualizations on desktops used in public areas.
}



% \textbf{Poor performance on the heatmap.}
% In addition to the four common visualization types investigated in this paper, the heatmap is also widely used for some mobile visualization applications~\cite{reda2018graphical}. A heatmap usually utilizes a single color tone with monotonically increasing luminance to encode the information. However, our masking scheme also relies on the modification of luminance to achieve privacy-preserving, which will damage the original information of the heatmap. As analyzed in study 2, compared to luminance contrast, the change of spatial frequency significantly impacts privacy-preserving performance. Thus, we plan to explore new masking schemes that only consider spatial frequency for heatmap-based privacy preservation.

% \textbf{Automatic visibility calculation}
% In study 1, we revealed the masking scheme effects on visualization readability. Additionally, we created a set of privacy-preserving visualizations based on study 1's feedback and then demonstrated these visualization's effectiveness in study 2. Nevertheless, since the privacy-preserving visualization generation requires human involvement, it cannot automatically generate privacy-preserving visualizations based on the required viewing distance for safety. However, our method is applicable to mobile applications. The privacy preservation effect of the masking scheme depends on the size of a visualization image and the resolution of the display screen of the device. For most applications, the size of the visualization is fixed and the resolution categories are limited. Once a privacy-preserving visualization is successfully tuned at one resolution setting, the visualization can be applied to all devices with the exact resolution. As a result, the time cost of our method is acceptable.

% In this section, we will discuss  lessons learned in developing our approach and carrying out studies, the generability and potential applications, and the limitations of our method. 

% \subsection{Lessons}

% \textbf{Trade-off between glanceable and privacy-preserving visualization}
% The mobile data visualization is glanceable and intuitive to interpret to enable users quickly obtain the necessary information, ~\cite{lee2021mobile}. The characteristics of mobile visualization cut both ways. It makes it easier for users to identify and understand data from the visualization, but it also makes it easier for peekers to see the mobile visualization.  For example, visualization is often used in public places, such as conference rooms and buses, where the user and others are close together. The nearby people can clearly see the user's data visualization in a glimpse, thereby leading to privacy leakage\dm{the above are repeated}. Our approach ensures that the user can see the visualization, but it is invisible to others at a slight distance (e.g., 65 cm), but it impedes the visualization efficiency in terms of information communication. Because users need to take more time to understand the visualizations, increasing their cognitive effort~\cite{matthews2006designing}. Therefore, there is a trade-off between mobile data visualization efficiency and privacy preservation. We discuss the lessons learnt in the method development and user studies to facilitate future studies in the area.

% \textbf{More granular privacy preservation in mobile visualization}\dm{how about 'sensitivity-aware privacy preservation'}\dm{maybe delete the first paragraph, just put the trade-off at the beginning of this paragraph. Then, the sensitivity of the data is the key criterion to balance the trade-off.}
% No single method can be applied to all data mobile visualizations. 
% We should instead determine the level of privacy protection for the visualization based on the sensitivity of the data and the preferences of the users. For instance, many participants in our interview indicated that the privacy protection they require in mobile visualization depends on the sensitivity of the information being visualized. If the data is relevant to a personal financial or health condition, they strongly demand privacy-preserving visualizations. According to one participant, he did not want nearby people to be able to see his visualized monthly income (e.g., bar charts) when he opened a mobile application for a financial institute. As a result, he was willing to spend more time seeing the privacy-preserving visualizations. In contrast, they emphasized the efficiency of mobile visualizations in terms of sports data, such as real-time heart rate and running mileage, since they can quickly refocus on the road ahead by reducing the time spent on viewing mobile visualizations. To fully satisfy the different requirements of users regarding mobile visualization, privacy preservation on mobile visualization should be user-centered.

% \textbf{Text and visual elements}
% Text and visual marks (such as bars and dots) are indispensable in data visualization. Considering geometric shapes, we consider the text and line marks as one category and process them both in the line-masking pattern. But from the perspective of information encoding, the text and visual marks are separable. For example, suppose there is no text in a visualization. In that case, the visualization is just a combination of geometries, and users have no idea what the geometries stand for, resulting in a meaningless visualization.
% On the other hand, users cannot obtain specific information about underlying data from mere text without visual marks. Although our method processes both text and visual marks, the privacy preservation approach can focus either on text or  visual marks in a mobile visualization because if either is invisible to the peekers, they cannot obtain valid information from the visualization. Due to the reduction in restrictions, researchers have more design space to balance visualization efficiency with privacy protection.

% \subsection{Generalizability and Potential Applications}
% \textbf{Generalized to mobile devices} To ensure the consistency of the experimental setting, we used  a fixed model of a smartphone. But our method can be applied to any mobile device in practice since it involves changing the spatial frequencies of a visualization image through a making scheme. The masking scheme only manipulates the image pixels, and no hardware is involved. 

% \textbf{Generalized to common visualizations} Our method classifies visualization marks into line-based and area-based and then applies different masking schemes to them accordingly. This classification is based on the spatial dimension occupied by the visual markers. Our approach can cover most visualizations since they are composed of lines or two-dimensional geometry (e.g., circles and rectangles). Although some visualizations use 3D markers, they are rare in daily usage~\cite{munzner2014visualization}. 

% \textbf{Potential applications}
% Our privacy-preserving visualizations reduce readability to the peekers. Meanwhile, our evaluation verifies its effectiveness in common visualization types (i.e., bar, pie, line, and scatter). Our method is, therefore, applicable to most commercial applications that use common visualizations and require privacy preservation. For example, mobile banking will show a user's cash flow in the line chart. A line chart displays the user's variation in income, expenditures, and budget over time, and the displayed content is personal and sensitive. With our method, the line chart can be transformed into a privacy-preserving one, allowing mobile banking users to access content without worrying about their information's confidentiality.

% \subsection{Limitations}
% Though the evaluation proves the effectiveness of our method, there are some limitations to be aware of.

% \textbf{CIELAB color space}
% The CIELAB provides a precise formula to model common human perception of color difference ~\cite{connolly1997study}.
% Therefore it is widely utilized in color science and engineering applications. However, since CIELAB color space is based on human vision, it also follows color boundaries in the human. The colors that a human can perceive depend on luminance~\cite{kim2009modeling}. Since study 1 aims to measure the luminance contrast effect on human visibility, we changed the luminance channel in CIELAB but kept the channels for color hue. When mapping the color to the image, the changed luminance and kept hue may not exist in the CIELAB color space. Therefore, the hue will be corrected to generate a substitute color that looks like the desired color~\cite{braun1999paradigm}. The adjusted hue does not affect human perception of the visualization because hue contrast has little effect on human detection ability~\cite{munzner2014visualization}. 
% raise RGB example, perfect cube
%However, since the boundary of human vision is not fall into a perfect cube 

% \textbf{Low-vision users}
% Our theoretical model~\cite{barten1999contrast} and conducted experiments are based on normal vision users. As a result, it is assumed that the target users of our approach have normal vision. People with low vision, however, perceive contrast and spatial frequency differently and thus may not be suitable for our method~\cite{chung2016comparing}. They may not be able to obtain helpful information from our privacy-preserving visualizations.

% \textbf{Continuous colormap}
% Some visualizations use continuous colormap with monotonically increasing luminance to encode the information, such as heatmap~\cite{reda2018graphical}. Our method adjusts the spatial frequency of visualizations, changing human eye sensitivity to color. Therefore, the adjusted spatial frequency will affect the effectiveness of the colormap encoding. Since our method considers the most common visualizations (e.g., pie, line)~\cite{battle2018beagle}, it can satisfy users' daily visualization usage. In addition, we plan to solve this problem in future work.
% \textbf{Automatic visibility calculation}
% In study 1, we revealed the masking scheme effects on visualization readability. Additionally, we created a set of privacy-preserving visualizations based on study 1's feedback and then demonstrated these visualization's effectiveness in study 2. Nevertheless, since the privacy-preserving visualization generation requires human involvement, it cannot automatically generate privacy-preserving visualizations based on the required viewing distance for safety. However, our method is applicable to mobile applications. The privacy preservation effect of the masking scheme depends on the size of a visualization image and the resolution of the display screen of the device. For most applications, the size of the visualization is fixed and the resolution categories are limited. Once a privacy-preserving visualization is successfully tuned at one resolution setting, the visualization can be applied to all devices with the exact resolution. As a result, the time cost of our method is acceptable.
% \textbf{}