
\section{Related work}


% 或者如果有相关的survey paper, 你可以直接引用人家的taxonomy,然后分类介绍每一部分工作
% Related work我个人认为比较好的一种组织方式:
% 1. 一两句话介绍这个方向大致做什么,相关工作可以分成几类。
% 2. 大概介绍每一类工作的代表性文章,可以酌情考虑说明这些工作的limitation
% 3. 另起一段说明我们的工作和现有工作的关系:可以是说我们改进了之前工作,解决了他们的limitation;也可以是我们是基于现有工作,借鉴了其中的想法或者思想



% The related work in this paper can be divided into three domains: Privacy-Preserving Data Visualization, Shoulder-surfing Protection on Mobile Devices, and Hybrid-image Visualization. 

% In this section, we review existing studies concerning data anonymization in visualization, shoulder-surfing protection on mobile devices.




\subsection{Data Anonymization in Visualization}
% \yong{How can you have only five references for one subsection?
% You may check the survey here:
% https://onlinelibrary.wiley.com/doi/abs/10.1111/cgf.14032  }                                              
\alexin{
Protecting sensitive information from potential leakage is a critical task for many real-world tasks
% technical challenge in modern society in data transformation
~\cite{bhattacharjee2020privacy}.
With the wide application of data visualization, there has been a growing trend of research on the privacy preservation of data visualizations.
% Because visualization analysis may result in the disclosure of personal information, some studies apply 
Specifically, many
data anonymization techniques have been developed to encrypt identifiers of individual data items that can connect an individual data items to
% a data point 
different visual elements
in a visualization~\cite{oksanen2015methods,wang2018graphprotector,dasgupta2012conceptualizing,dasgupta2019guess}.
According to Bhattacharjee~\textit{et al.}~\cite{bhattacharjee2020privacy}, these data anonymization techniques can be divided into two groups: (1) data uncertainty and (2) visual uncertainty. Data uncertainty-based methods remove or modify a portion of the original dataset to ensure that a specific number of data records are indistinguishable and thereby stop sensitive the exposure of sensitive individual data items~\cite{sweeney2002k,bayardo2005data}.
}
% As a result, visualizations used with the modified dataset are unable to reveal sensitive data. 
% Some studies apply this kind of method to visualization in the visualization field. 
For example, Chou \textit{et al.}~\cite{chou2019privacy} proposed a data-based clustering algorithm on sequential data and prevented the individual data items from being revealed by visualization (e.g., the Sankey diagram).
% that represents the event sequence data. 
Similarly, Okansen \textit{et al.}~\cite{oksanen2015methods} introduced a privacy-preserving heatmap for trajectory data, where three data processing techniques are presented to prevent the data owner's identity disclosure from being identified in the heatmap. 
In contrast to data uncertainty, visual uncertainty based data anonymization approaches involve uncertainty in the mapping between data points and visualizations.
For instance, Dasgupta~\textit{et al.}~\cite{dasgupta2013measuring} proposed a method for clustering similar data points, binning them according to their value ranges, and displaying these data in parallel coordinate views. Chou~\textit{et al.}~\cite{chou2016obfuscated} obfuscated data rendering in scientific visualizations, and the visualizations can prevent unauthorized viewers from seeing the detailed data items.
% the data's content.
\alexin{Although these methods can avoid the revelation of individual data items,
% \alexout{they cannot prevent unauthorized third-party from viewing the visualizations.} 
they cannot prevent shoulder surfers from viewing the visualization itself.}

% \yong{Songheng, pls check my Chinese comments!}

\subsection{Shoulder-surfing Protection on Mobile Devices}

% \dm{what not introduce shoulder-surfing in the intro, this is exactly the scenario you are trying to address.}. 

% In public areas (e.g., trains, cafes), shoulder surfacers can easily view the screen of another person's mobile device and gather sensitive information from the screen. To address the shoulder surfacing attacks on mobile devices, researchers have proposed different approaches and they 
According to Chen~\textit{et al.}~\cite{chen2019keep},
prior studies on shoulder-surfing protection can be categorized into three types: interpretation barrier, shoulder surfer alter, and information blocking.
The interpretation barrier method does not stop shoulder surfers from observing sensitive information but slows down the information leakage on the mobile screen. As a result, sensitive data is displayed in a specific format that the data owner can only understand. For example, Von Zezschwitz~\textit{et al.}~\cite{von2016you} recommended using graphic distortion filters to protect private images in smartphone photo galleries. Distortion filters made it easy for the user (i.e., the image owner) to recognize the content but difficult for the shoulder surfer to comprehend. Similarly, Gouveia~\textit{et al.}~\cite{gouveia2016exploring} designed an alternative metaphor (i.e., a growth garden) to represent physical activity progress. Alternatively, the methods of shoulder surfer alter  use sensors in a mobile device to detect if anyone is nearby and trying to peek at the user's mobile device screen~\cite{ryu2017electronic,ali2014protecting}. For example, Ryu~\textit{et al.}~\cite{ryu2017electronic} leverages face recognition technology and the front camera to detect whether people around the user are peeking at the mobile screen.
% iScreen utilizes the font camera and light sensor of a mobile device to detect the external environment and then adjusts the displayed content settings accordingly (e.g., font size, screen brightness)~\cite{lei2021iscreen}. Thus, the modified display settings may reduce the shoulder surfer's readability, but users can read the content with little difficulty. Furthermore, the information blocking techniques can enable users to conceal their on-screen information from shoulder surfers. For example, adding a privacy film to a mobile device screen can protect users' information on the screen, but it incurs additional expenses. 
The information blocking method aims to conceal the actual information on the screen when shoulder surfers view it. For instance, adding a privacy film to a mobile device screen can protect users' information on the screen, but it incurs additional expenses.  
In contrast to the hardware method, IllusionPin~\cite{papadopoulos2017illusionpin} blends real and fake keypads using the hybrid image~\cite{oliva2006hybrid}. As a result, users can read the actual keypad, while shoulder surfers can only see the fake keypad.  In addition to PIN keypad protection, HideScreen utilizes the optical system to blend screen content into the background, thus preventing shoulder surfing~\cite{chen2019keep}. Nevertheless, IllusionPin is only available for a specific image type (i.e., mobile keypads), and HideScreen only works with grayscale nature images that sacrifice color information.
Unlike prior studies above, our approach targets privacy preservation for mobile data visualizations, which are colorful and different from natural images.
% To bridge the research gap between privacy protection and colored mobile visualizations, we propose a novel masking scheme for the visualization. The scheme can transform a standard visualization into a visualization that prevents shoulder surfing attacks. In addition to working for colored visualization, the method is applicable to a variety of visualization types and hence is not limited to a single visualization type.
% In light of this, we propose a novel masking scheme that allows a user to view the information clearly (at close proximity) and prevents unauthorized third parties from viewing the screen (from a distance).
 
 






