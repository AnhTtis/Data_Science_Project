





\section{Study 1: Variable Effects}
\label{study:study1}

% After applying the proposed masking scheme, we can reduce the visual marks visibility of shoulder surfers from a distance. 

\alexin{
By using the proposed masking scheme to process input visualizations, we can reduce the visibility of visual marks for shoulder surfers at a distance.
}
Meanwhile, we also need to guarantee that visualization owners nearby can still perceive the information from the visualizations. 
As observed from Figure~\ref{fig:csf}, the boundary is determined by both the spatial frequency and luminance contrast. 
\alexin{
Thus, we conducted Study 1
to investigate how these two factors jointly affect human perception of visual marks at different distances.
% we design Study 1 with three objectives as below: 
Specifically, there are three primary objectives for Study 1:
}

% Having designed the masking scheme, we aimed to measure its effect on human visibility to the resulting visualization marks at close and far distances. Therefore, we conducted an experiment to determine the effect of the two variables of masking: mask area and luminance contrast. The subsequent three goals are intended to be achieved in Study 1:
\begin{itemize}
  \item Identifying how different mask areas affect human perception of the resulting visual marks.
  \item Examining how different luminance contrasts between visual marks and background affect human perception of the visual marks.
  \item Studying how different viewing distances affect the visual marks readability.
\end{itemize}

% We wanted to estimate our proposed two variables' effects on the human perception of the resulting privacy-preserving visualization and determine whether our proposed formula can align with the actual human visibility of the visualization at a given viewing distance. Therefore, we conducted a study to evaluate the variable effects and formula by asking participants to rate visualizations at a given viewing distance in terms of visibility.


\begin{figure}[ht!]
    \centering
    \includegraphics[width=0.8\linewidth]{viewingdistancev2.pdf}
    \caption{The viewing distance in various scenarios: (a) seat intervals and (b) shoulder surfer's distance from the user's phone while sitting side-by-side. 30cm denotes the typical distance between a mobile phone and users; 50cm represents average shoulder width; 60cm is the distance between a shoulder surfer and the user's phone when seated together; 90cm refers to the distance where shoulder surfer sits behind the user in a public transportation (e.g., bus).}
    \vspace{-2em}
    \label{fig:view_distance}
\end{figure}

\subsection{Participants and Apparatus}
% \alexout{Participants were recruited through 
% % word-of-mouth and by 
% % researchers' 
% the authors' personal
% connections on a university campus.}
\alexin{All the participants are students recruited from a local university.} 
% \alex{Is it reasonable for our recruitment?}
% \yong{Pls remember to upload our IRB approval letter as a support.}
% \alexin{In total, we} 
In total, 16 participants with normal or corrected-to-normal vision (3 females, age: 24-28) 
% between 24 and 28 years old 
participated in the study. 
% Thirteen were Ph.D. students, and the remaining three had at least bachelor's degrees. 
% All of them had a fundamental knowledge of information visualization (e.g., utilizing the visualization in their work) and had experience viewing visualizations on mobile devices. 
% \alexin{All participants had normal or corrected-to-normal vision.}
None of them was color blind.
The study was conducted in a campus meeting room with good lighting conditions.
% , a quiet, office-like environment with well-lighting. 
Participants were requested to view a certain number of visualizations, and a short break was taken every 5 minutes to avoid visual fatigue. The whole process took around 40 minutes on average. Each participant was compensated with about \$11. Before starting the study, participants read and signed an IRB-approved consent form.

% to complete the entire study. Participants could take a five-minute break after viewing a certain number of visualizations.


 A 6.67-inches mobile phone with 1080 x 2400 pixels was placed on an adjustable stand on a table to accommodate different participants' heights. The authors switched the visualizations displayed on the mobile phone, and the participants sitting on a chair gave audio feedback indicating the visibility of each visualization.





% All visualizations were shown to the participants by a 6.67-inches mobile phone with 1080 x 2400 pixels resolution. Before the evaluation began, the mobile phone was mounted on a stand above a table and adjusted to the same height as the participants so they could comfortably view the screen. In the study, participants remained motionless in a chair while the researchers adjusted the phone's position in order to adjust the distance between them and the phone while also switching the displayed visualizations .

\subsection{Stimulus}


As the bar chart, pie chart, scatter plot, and line chart are considered the most popular visualization types~\cite{battle2018beagle}, we evaluate the mask area and luminance contrast effect on all of them. For each visualization type, we consider seven mask area values for area-based marks (1, 3, 5, 7, 9, 11, 13) and line-based marks (1, 5, 9, 13, 17, 21, 25), respectively. In addition, we selected 5 luminance contrasts (0, 25, 50, 75, 100). To assure consistent visualization settings (e.g., size), we generated these testing visualizations
% \alexin{image} 
using \href{https://vega.github.io/editor/}{Vega-Lite}. Moreover, since the text is line-based by nature, we remove the text so that participants can concentrate on the marks. As a result, the test set consists of 140 visualizations (4 visualization types  $\times$ 7 mask areas  $\times$ 5 luminance contrasts). \alexin{Some visualization examples used in the study
% , which have been processed using three methods, can be observed in 
are shown in
Figure~\ref{fig:study1a_eg} in Appendix~\ref{sec:appendix}.}


% As shown in Figure~\ref{fig:view_distance}(b), 
% in a sitting position, 
According to the prior study~\cite{yoshimura2017smartphone},
the viewing distance between users and the mobile phone screen is about 30cm. The length of the forward-facing seat is approximately 90cm (Figure~\ref{fig:view_distance}(a)), which is normal in public transportation (e.g., bus)~\cite{SeatWidth2021}.
Thus, we
set the distance between the participants and the smartphone with two values: 30cm (the viewing distance of the phone owner) and 90cm (the viewing distance of the shoulder surfer), as shown in Figure~\ref{fig:view_distance}(a, b).

% \begin{figure}[h!]
%     \hspace{-2em}
%     % \centering
%     \includegraphics[width=1.1\linewidth]{figures/study1a_example.pdf}
%     \caption{Comparison of four (a) original visualizations and (b) transformed visualizations with adjustment on both spatial frequency and luminance contrast. Note: due to the varying screen resolutions, the displayed visualization effect may differ from that on the phones in Study 1.}
%     % \caption{Examples of original visualizations and visualizations that are created by combining two variables. (a) It shows the original visualizations used in Study 1. (b) It shows the processed visualizations by two variable combinations. These visualizations may be different from real images shown on the evaluation mobile phone because the colors/images displayed may vary with devices.}
%     \vspace{-1em}
%     \label{fig:study1a_eg}
% \end{figure}


\subsection{Procedures and Evaluation Criteria}
Since the visibility of visual marks
% visibility 
is the basis for visualization exploration,
% participant was asked 
we asked each participant
to view all visualizations 
% (examples shown in Figure~\ref{fig:study1a_eg})
one by one, and then rate the visibility of the marks. When reviewing relevant work~\cite{chen2019keep,papadopoulos2017illusionpin,lei2021iscreen}, we found that no prior work had developed
% scale 
metrics to measure visibility.
Therefore, we proposed rating the visualization visibility with a 5-point Likert scale.
\alexin{To ensure a more precise measurement of participants' subjective assessment of visualization visibility, our 5-point scale goes beyond simply indicating whether the visualization is visible or not. Instead,
we also consider the difficulty levels of identifying and perceiving the visualization in terms of the necessary effort and time when designing the criteria of the 5-point Likert scale. 
% this scale also considers the degree of difficulty associated with viewing the visualization, including the effort and time required to do so. 
}
Specifically, the criteria for the 5-point scale are as follows: 

\begin{enumerate}
  \item[1:] I cannot recognize any visual marks from the visualization.
  \item[2:] I can identify a little visual marks from the visualization.
  \item[3:] I can identify a large percentage of the marks from the visualization.
  \item[4:] I need some time and effort to identify all visualization marks from the visualizations.
  \item[5:] I can easily recognize all the visual marks at a glance.
\end{enumerate}

To calibrate participants' ratings, we provided participants with five different charts corresponding to the five scale scores before starting the real tests.
Participants were encouraged to explain why they rated the visualizations and what factors affect their visibility of the visualizations in a think-aloud manner. 
Given that we have set two viewing distances (30cm and 90cm), the viewing order might affect the rating. For instance, if the participant first viewed at 30cm, he/she is likely to see all the marks on the visualization, which may affect his/her evaluation of the visibility when viewing from 90cm.
% The memory can make it easier for him when viewing at 90cm. 
To counterbalance this issue, we divided the participants into two groups, where the participants of Group 1 first viewed at 30cm and then 90cm, and vice versa for the participants of Group 2. Additionally, the visualizations are displayed to each participant in a random order to control order effects~\cite{strack1992order}.



\begin{figure*}[ht!]
    \centering
    \includegraphics[width=0.8\linewidth]{study1aratingv3.pdf}
    \caption{Average ratings obtained from Study 1, where each row corresponds to one visualization type. Different graphs in each row are for different luminance contrast. Each graph shows the effect of mask area size on visibility at close (30) and far (90cm) distances.  }
    \vspace{-2em}
    \label{fig:study1a_rating}
\end{figure*}

\subsection{Result} Figure~\ref{fig:study1a_rating} shows the participants' average ratings under different settings (varying mask area and luminance contrast) for each type of visualization, from which we
have the following observations:
% can observe the following findings:



\textbf{Impact of Mask Area:} First, we can observe that participant ratings (i.e., visibility) decrease with the increase of mask area, regardless of the viewing distance, luminance contrast, as well as visualization type. The observation is reasonable as the increase of mask area actually enlarges the spatial frequency of the image, and human inherently shows poorer capability in perceptualizing high-frequency content. Second, the decreasing rate of the visibility ratings for the two viewing distances is different in general. Specifically, viewing from a larger distance leads to a sharper decreasing rate, suggesting that spatial frequency has a more critical impact at a far distance.




\textbf{Impact of Luminance Contrast:}
Based on the CSF curve presented in Figure~\ref{fig:csf}, the visibility ratings are expected to decline with the decrease of the luminance contrast. However, as we can observe from Figure~\ref{fig:study1a_rating}, the ratings with a luminance contrast of 100 are consistently lower than those with a contrast of (75 or 50). The underlying reason is that: visualizations (e.g., bar charts) usually contain multiple visual marks with the same shape but with different colors (e.g., two bars correspond to different categories). With extremely low luminance, bars with different colors tend to exhibit the black color. As a result, although the visual marks are distinguished from the background, the information carried by different colors is lost. This implies that the CSF curve cannot be directly adopted in our visualization scenario, and we need to carefully select the optimal luminance contrast based on the experiment.



% \yong{Reach here.}
\textbf{Impact of Visualization Type:} When comparing the graphs (Figure~\ref{fig:study1a_rating}) in different rows, we can observe that although the overall trend of the curves is similar, the points where the curves start to converge are different. Thus, we have to select different variable values for different visualization types. Since our masking scheme aims to reduce the visibility at a far distance while maintaining the visibility in proximity, the optimal variable value corresponds to the case where the ratings between 30cm and 90cm have the most significant gap. Based on the result, the values of (mask area and luminance contrast) selected for bar, pie, scatter, and line chart are (13, 75), (7, 75), (5, 75), (21, 25).




% \textbf{Need for Auxiliary Border: } From Figure~\ref{fig:study1a_rating}, we can observe that for pie and scatter charts, the visibility ratings under close and far distances are very similar, implying the low potential to apply our masking scheme to these visualization types.
% \yong{How can we keep these comments here?? It will definitely backfire. Please go through the whole paper carefully and avoid such claims!}
% We then analyze the feedback from the participants and find that the main reason is the low discernibility of different elements in these visualizations. For instance, participant P7 said \textit{"The dot is small by nature, and when a dot was divided into several smaller pixels, I was not sure whether these pixels belonged to the same dot or not"}. P3 commented that \textit{"Even though I can recognize the visualization is a pie chart, I cannot identify each slice within the pie chart. The visual ambiguity lowers my rating score."}. In addition to P3, some participants reported similar issues with the pie chart. Participant P1 advised adding a border to the bar: \textit{"Even though I detected the height difference between the bars, it would be better if you provide a horizontal line above each bar as a bar's height reference"}. \alexout{Thus, we concluded that this is a general issue for area-based visualization types and proposed adding an auxiliary border to increase the invisibility at a short distance. The detailed technique is described in Section~\ref{sec: further_improve_area}.} 
% \alexin{Based on feedback from participants, we discovered that cognitive factors (i.e., the Gestalt rules~\cite{todorovic2008gestalt}) can affect humans' perception of area-based elements in addition to visual stimuli. This finding motivates us to improve the masking scheme for area-based elements by adding an auxiliary border. The detailed technique is described in Section~\ref{sec: further_improve_area}.}


\alexin{
\textbf{Indistinguishable Borders of Area-based Marks:}
Figure~\ref{fig:study1a_rating} shows that participants' visibility ratings for pie chart and scatter plot are quite similar across close and far distances.
By analyzing the feedback from participants, we find that the main reason is the indistinguishable borders of area-based marks in these visualizations.
Since the same area-based masking scheme (Figure~\ref{fig:study2_sample}(c) in Appendix \ref{sec:appendix}) is leveraged to process the whole area-based marks without specifically processing the borders, the borders of area-based marks are overly discretized, making it harder to distinguish the boundary of area-based marks.
For instance, P7 said \textit{"The dots [in the scatter plots] are small by nature. When a dot is divided into sparse pixels, it takes me more time to confirm whether these pixels belong to the same dot or not when viewing it at a close distance"}. P3 commented that \textit{"Even though I can recognize a pie chart [when viewing it at a close distance],
% I cannot identify each slice within the pie chart. 
it was not easy for me to quickly identify each slice of the pie chart. Thus, I have lowered my visibility rating score at the close viewing distance."} 
% The visual ambiguity lowers my rating score."}. 
Other participants have also reported similar issues with bar chart.
Such an observation is consistent with the Gestalt Principles~\cite{todorovic2008gestalt} such as the Law of proximity and continuity.
It motivates us to propose customized masking for area-based marks, as introduced in Section~\ref{sec-mask4areamarks}. 
% \yong{pls fill the empty figure reference.}
}

% \begin{figure}[h!]
%     \hspace{-2em}
%     % \centering
%     \includegraphics[width=1.1\linewidth]{figures/study2_sample_v3.pdf}
%     \caption{Visualization samples used in Study 2. (a) Original visualizations, (b) visualizations with mask and border technique applied), (c) visualizations with a mask applied. Since the border is only available for the area-based mark, the displayed line chart is the same in (b) and (c). Note: due to the varying screen resolutions, the displayed visualization effect may differ from that on the phones in Study 2.}
%     \vspace{-1em}
%     \label{fig:study2_sample}
% \end{figure}


% \section{Study 2: Usability Evaluation}
% \yong{Songheng, pls make sure you really know the difference between usability and effectiveness.}

\alexin{\section{Study 2: Privacy Preservation Effectiveness}}\label{sec:study2}
Our second user study evaluated the effectiveness of our method in preserving privacy regarding mobile visualizations. This study aims to assess the complete version of our approach with suitable variables in comparison to the baseline approach (i.e., our approach's  original visualization and partial version).

% \begin{figure*}[ht!]
%     \centering
%     \includegraphics[width=0.8\linewidth]{figures/study2_sample_v3.pdf}
%     \caption{Visualization samples used in Study 2. (a) Original visualizations, (b) visualizations with mask and border technique applied), (c) visualizations with mask applied. Since the border is only available for the area-based mark, the displayed line chart is the same in (b) and (c). Note: due to the varying screen resolutions, the displayed visualization effect may differ from that on the phones in Study 2.}
%     \vspace{-1em}
%     \label{fig:study2_sample}
% \end{figure*}
% \subsection{Participants and Apparatus}

Following Study 1's recruitment process, we recruited 18 participants (5 females, Age: 22-28) with normal or corrected-to-normal vision from local university, ensuring none were color blind.
We maintained the same mobile device and room settings  and the IRB-approved consent form signing process. Each participant received a compensation of about \$11 for their time in our study.

\subsection{Stimulus}
\alexin{Similar to Study 1, bar charts, pie charts, scatter plots, and line charts were used in Study 2. Unlike Study 1, where we utilized one visualization chart for each visualization type, we used six charts for each visualization type to justify our method's effectiveness comprehensively. These chart images are generated by the Vega-Lite as well. In addition, we determined the variable values (i.e., mask area and luminance contrast) for each visualization type based on the results of Study 1. The variable values of mask area and luminance contrast is determined by its privacy preservation ability, namely, a high rating at a close viewing distance (30cm) and a low rating at a far viewing distance (90cm), as shown in Figure~\ref{fig:study1a_rating}. The variable values for each chart type are: bar charts (mask area: 13, luminance contrast: 75), pie charts (mask area: 7, luminance contrast: 25), scatter plots (mask area: 5, luminance contrast: 75), and line charts (mask area: 21, luminance contrast: 25).}

% \yong{If you really revised some parts, please highlight them.}

Aside from the two viewing distances tested in Study 1, we intended to further validate our method by adding a different viewing distance that had not been tested. A common shoulder surfing scenario occurs when a user sits side-by-side with a peeker~\cite{says_fedup_2017}, so we add a different viewing distance (60cm) to account for such a scenario.
The reason why the viewing distance between the peeker and the phone is about 60cm is the average shoulder width is approximately 50cm~\cite{NASA} and the viewing distance of users is 30cm~\cite{yoshimura2017smartphone}, as shown in Figure~\ref{fig:view_distance}(a).

Our method aims to deter shoulder surfers from accessing visualization information on mobile devices. In this experiment, we used standard daily-use visualizations containing visual marks, axes, titles, and labels. There are three methods in the study:
\begin{itemize}
    \item \textbf{Visualization without masking:} the original visualization is not processed by the privacy-preserving approach. 
    \item \textbf{Visualization with masking:} the visualization is generated by area- and line-based masking.
    \item \textbf{Visualization with masking and border:} besides the area- and line-based masking, the visualization is supplemented by contour borders in area-based marks.

\end{itemize}

As a result, the test set consists of 60 visualizations (4 visualization
types × 5 charts × 3 methods). \alexin{Some visualization examples used in the study
% , which have been processed using three methods, 
% can be observed 
are shown
in Figure~\ref{fig:study2_sample} in Appendix~\ref{sec:appendix}.}

\begin{figure}[ht!]
    \centering
    \includegraphics[width=\linewidth]{illustrationfinalv2.pdf}
        \caption{\alexin{An illustration of our privacy preservation approach's effect on a visualization (with masking and border) viewed at varying distances (30, 60, 90 cm), with the original bar chart shown on the left side.}}
    % a privacy-preserving visualization observed at different distances (30, 60, 90 cm) in our experiment. The test  visualization example is on the left side .}}
    % \vspace{-2em}
    \label{fig:final_effect}
\end{figure}

\begin{table}[!ht]
% \begin{tabular}{llp{2cm}p{2cm}}
\centering
\begin{tabular}{p{1.5cm}|p{0.8cm}|p{5cm}}
\hline
\multirow{4}{*}{Motivation} &
\multicolumn{1}{l|}{Q1}  & Are you concerned about information leakage when using mobile visualization on a mobile device? If so, in what kind of condition? If not, why?        
\\ \cline{3-2}
 & 
\multicolumn{1}{l|}{Q2}  & Do you think our approach is helpful for privacy preservation while using mobile visualization on mobile devices?        
\\ \hline
\multirow{8}{*}{Effectiveness} &
\multicolumn{1}{l|}{Q3}  & From a close distance, are you able to see all the necessary information from our privacy-preserving visualizations? Why and why not?              \\ \cline{3-2} 
& 
\multicolumn{1}{l|}{Q4}  & From a long distance, are you able to see all the information? Why and why not?   
\\ \cline{3-2}
&
\multicolumn{1}{l|}{Q5}  & Can the border in the area-based visualization (i.e., pie, bar, scatter) can help you identify and receive the visual element?
\\ \cline{3-2}
&
\multicolumn{1}{l|}{Q6}  & Can you see the processed text in the close and far distance, respectively? Why and why not?
\\ \hline
% \alexout{Assessment} 
\alexin{Pros\& Cons} &
\multicolumn{1}{l|}{Q7}  & What are the pros and cons of our overall approach?
\\ \hline
\end{tabular}
\caption{The questions in the post-study questionnaire of Study 2.}
\label{table:study2}
\end{table}


\begin{figure}[ht!]
    \centering
    \includegraphics[width=\linewidth]{study2chartv6.pdf}
    \caption{The average ratings from Study 2 for visual marks at (a) 30cm, (b) 60cm, and (c) 90cm viewing distances.  }
    \vspace{-1em}
    \label{fig:study2_chart}
\end{figure}




\subsection{Procedures and Evaluation Criteria}
We compared our complete privacy-preserving visualizations with the other two methods at three viewing distances. We conducted a within-subject study where tasks and methods were the factors.


The authors first introduced the five rating criteria to calibrate participants' ratings. The participants sat on a chair as the authors presented a series of visualizations
at three distances: 30cm, 60cm, and 90cm \alexin{(Figure~\ref{fig:final_effect})}. Participants provided subjective evaluations for visual encoding and text in the visualization at each distance on the 5-point scale. 
Furthermore, the displayed visualizations are in random order to control the order effects. 
According to the research by Poco~\textit{et al.}~\cite{poco2017reverse}, visualization consists of text (including axes) and visual marks. If the participants can clearly identify both text and marks, they are able to comprehend the visualization and vice versa. Therefore, there are two tasks for rating: visual mark visibility rating and text readability rating.

Participants entered their ratings into a mobile-based survey \href{https://www.qualtrics.com/}{tool}. The survey tool included a timer to record participants' time for each visualization rating. Participants were timed from when they started a visualization rating to when they finished it. The time is regarded as the participants' reading time to the visualization. A post-study interview was conducted to assess the quality of the study. The participants wrote their comments and suggestions in response to the open-ended questions in Table~\ref{table:study2} regarding our method's significance, usability, and overall feedback. All three distance combinations were conducted among participants in a counterbalanced manner. The study took about 60 minutes. The criteria for the 5-point scale are the same as study 1 (i.e., from 1-cannot see any data information to 5-see all data information clearly at a glance).




\subsection{Result}
Overall, participants can clearly see the visualization (i.e., visual marks and text) at a close viewing distance, but hardly discern it at a far distance. Additionally, they need more time and effort to understand the visualization at a close distance. We will elaborate on the quantitative and qualitative results in the following.

\subsubsection{Quantitative Result}
\label{sec:study_2_results}
In visual mark visibility rating, Figure~\ref{fig:study2_chart} shows the participants' average rating for each of the three methods. Regardless of distances, participants can see the original visualizations. However, privacy-preserving visualizations (i.e., (masking + border) visualization) can prevent long-distance observing, as shown in Figure~\ref{fig:study2_chart} (b,c). For example, when participants are 60cm or 90cm away from the visualization, the average ratings of all visualization types are less than 3. According to the 5-point scale, it means that participants were able to view little visual marks on the privacy-preserving visualizations. In contrast, when the viewing distance reaches about 30cm, the average rating of our method is almost equal to 4, indicating that participants can obtain the visual marks without any difficulties. Furthermore, participants spent a minor amount of additional time (i.e., $\mu$=2.89s, $\sigma$=2.01s) getting information from (masking + border) visualization compared to the reading time (i.e., $\mu$=0.82s, $\sigma$=1.07s) for original visualization. Also, Figure~\ref{fig:study2_chart}(a) demonstrates that the incorporation of borders can increase participants' visibility to the visual marks in close viewing distance, not decrease too much privacy-preservation of our method as shown in Figure~\ref{fig:study2_chart} (b,c).

As for text readability, Figure~\ref{fig:study2_text} displays participants' ratings for the text in the original visualizations and processed text by the line-based masking scheme. Figure~\ref{fig:study2_text} (a) verifies that participants can read the same content in the masked text as in the original text because the average rating significantly overcomes 4. As shown in Figure~\ref{fig:study2_text} (b,c), participants had difficulty identifying the text once the viewing distance increased, resulting in dramatic drops in their ratings. Similar to the chart, the reading time of the masked text (i.e., $\mu$=2.42s, $\sigma$=1.81s) is greater than the original text (i.e., $\mu$=1.04s, $\sigma$=1.05s), but the increase is acceptable in practice.

\begin{figure}[ht!]
    \centering
    \includegraphics[width=\linewidth]{textratingv7.pdf}
    \caption{The average ratings obtained from Study 2 for texts at three viewing distances: (a) 30cm, (b) 60cm, and (c) 90cm.}
    \vspace{-2em}
    \label{fig:study2_text}
\end{figure}


\alexin{
\subsubsection{Qualitative Feedback}
}


Overall, our privacy-preserving visualization received an enthusiastic response from the participants. Figures~\ref{fig:study2_chart} and~\ref{fig:study2_text} demonstrate the usefulness of privacy-preserving visualization. We summarized participants' feedback and categorized them into three groups in the following:

\alexin{
\textbf{The information leakage of mobile visualization is a common concern.}
 }
Among all the eighteen participants,
fourteen of them expressed their concerns about the leakage of personal information shown as mobile data visualizations (e.g., it is not a good idea for anyone to view my screen regarding personal information). Seven participants (P7,11,13-16) expressed concern that the mobile visualization would disclose their financial information,

 \textbf{The proposed approach can guarantee privacy preservation after a distance.} 
All participants appreciated that privacy-preserving visualization helps protect their privacy on mobile visualizations. P11 could not distinguish the visualization from its background when viewing at a distance. P18, who had never viewed sensitive data on her phone due to a lack of trust in the phone's privacy protection, commented that "(Privacy-preserving visualizations) increase (my) trust in privacy being maintained (on the mobile devices)." 

 \textbf{Area-based visualizations benefit from the added border.} Most participants recognized the usefulness of the addition border. P10 noted that when viewing at a distance, he could identify the visualization in detail (e.g., slices in a pie chart), but when he was far from the device, he only identified the visualization type. P16 emphasized the border effects on a pie chart, which enables him to determine each slice's size in the chart. P8 concluded that "the border can show the bounds of the information of data (encoded by the area-based marks)."


 % \textbf{privacy preserving visualization affects user's experience}