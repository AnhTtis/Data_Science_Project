\section{Conclusion}
In this work, we present a privacy-preserving approach to protect mobile data visualizations (e.g., financial and personal health data) from shoulder surfers in public spaces. Specifically, based on the characteristics of the human vision system, i.e., spatial frequency and luminance contrast jointly affect the human perception of an image when viewing from different ranges, we design a masking scheme that can be applied to process typical visualization charts like a bar chart, pie chart, scatter plot and line chart. 
\alexin{
We conducted two
% \alexout{well-designed} 
user studies}
with 16 and 18 participants, respectively, to investigate the impact of different parameters, refine the devised masking scheme, and evaluate the real-world performance of our proposed method. The results validated the effectiveness of our approach in preventing information leakage from shoulder surfers.

In future work, we plan to extend our method to
more types of visualizations (e.g., customized visualizations) to further demonstrate its effectiveness in privacy preservation for data visualizations.
\alexin{
Also, it will be interesting to explore how we can leverage the screen information of mobile devices (e.g., sizes and resolutions) to achieve automated optimal parameter configuration for our approach on different mobile devices.
}

% and investigate the feasibility of using the knowledge of human vision system to augment low-vision users' accessibility to mobile visualization.

% We propose a new method of protecting mobile device content privacy. Our approach can transform a regular visualization into a privacy-preserving visualization. Hence, users close to the devices can observe the visualization without obstacles, while those at a distance cannot see it. Our method leverages a characteristic of the human vision system (HVS): at close distances, humans are sensitive to content's high-frequency information, but at a distance, they are sensitive to low-frequency information. Therefore, we convert critical components in visualizations (e.g., marks, axes, and text) into high frequency.
% Additionally, we change the luminance contrast between the components and the visualization background, which allows users to distinguish the components from the background, but others who are far away hardly recognize them. Additionally, we propose a calculation that can indicate how many frequencies in visualization are perceptible by the human eye at a given distance. With 16 participants, we conducted a user study to assess the effects of the method's variables on the clarity of visualization by humans at close and far distances. The study also demonstrates that the proposed formula calculation is consistent with actual human visibility to visualization. Based on the participants' feedback and study results, we improved our method by adding preserving-preserving borders, which can increase visibility at a close but not decrease privacy preservation. We conducted another usability evaluation study with 20 participants to verify the privacy-preserving visualizations. \alex{add results}.

% We summarized lessons learned from method development and experiments to advance the community in mobile visualization privacy protection. In the future, we aim to extend our method to more visualization types. We will also continue to utilize the human vision system characteristics to augment low-vision users' accessibility to mobile visualization.


% \yong{Where is your future work???}