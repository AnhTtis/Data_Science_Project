\section{Background}
\label{sec:background}
In this section, we introduce the background of our research, including image frequency (Section~\ref{background:sp}), luminance contrast (Section~\ref{background:contrast}), and contrast sensitivity function (Section~\ref{background:csf}).


\begin{figure}[ht!]
    \centering
    % \vspace*{-3em}
    \includegraphics[width=\linewidth]{backgroundspv3.pdf}
    \caption{Representations of images (gratings) in the spatial  domain (upper row) which is composed of white and black bars, and frequency domain (lower row) in which white squares represent the spatial frequencies. The denser the gratings, the higher the spatial frequency (white squares farther to the center in the frequency domain). Gratings with higher spatial frequency are more difficult to identify from a fixed viewing distance.
    % \yong{The font style of the texts in all the figures are different from that of the text in the paper.}
    }
    \vspace{-2em}
    \label{fig:bg_freq}
\end{figure}

\subsection{Image Frequency}\label{background:sp}

A two-dimensional (2D) array of pixels represents a digital image (the spatial domain), and each pixel corresponds to a particular color value (e.g., RGB). The representation of 2D pixels can be transformed into a sum of different frequency components in the frequency domain~\cite{bracewell2004fourier}. With the Fourier transformation, we can convert a digital image from a 2D spatial domain to a 2D frequency domain \cite{bracewell1986fourier} as shown in Figure~\ref{fig:bg_freq}. The frequency domain can be used to present the frequency distribution of an image. Additionally, the frequency of the image is determined by how fast the image pixel value changes. For example, in a fixed size figure shown in Figure~\ref{fig:bg_freq}, when there are more  black and white bars alternatively occurring in the grating, the grating's black and white pixel value changes faster, so the frequency of the grating becomes higher. As image frequency increases, the human vision system 
(abbreviated as HVS) is harder to respond to the frequency. Thus, it is difficult for humans to distinguish when the number of alternative black and white bars increases at a fixed image size (Figure~\ref{fig:bg_freq}).
% The Humans are less sensitive to high frequencies at a distance. Thus, it is difficult to distinguish the black and white bars from high-frequency gratings.
% As a result, we can consider an image to be a composite of frequency components. Additionally, the low frequency component represents coarse image information (e.g., image outline), and high frequency details the image. Figure~\ref{fig:bg_freq} exemplifies that the image becomes clearer as more high-frequency information is included to an image.
% Unlike the grating, image normally has different 
% According to the multichannel model of human vision, the HVS can perform a frequency decomposition of a stimulus (e.g., an image seen by a human) where frequency components are detected independently~\cite{chandler2013seven}. 


% If the perceived frequency exceeds the HVS threshold, it will be ignored.

% HVS will disregard a frequency if its perceived contrast cannot meet the necessary contrast threshold. Built upon this discovery, we can transform an image into a frequency-domain representation and then analyze the frequency distribution to estimate the likelihood that the image will be visible to a human. Additionally, we can change the image's visibility to a human by manipulating the frequency components of the image.


\begin{figure}[ht!]
    \centering
    % \vspace*{-3em}
    \includegraphics[width=\linewidth]{backgroundcontrastv3.pdf}
    \caption{The effect of luminance contrast on human perception. Lower luminance contrast at a fixed viewing distance leads to poor visibility of the black and white gratings. }
    \vspace{-2em}
    \label{fig:bg_contrast}
\end{figure}



\subsection{Luminance Contrast}\label{background:contrast}
Luminance contrast refers to the luminance difference between pixels in an image. The HVS is more sensitive to the luminance contrast between pixels than the absolute luminance value of pixels. High contrast between an object and its background will enable the human to distinguish the object from the background. If the contrast does not reach the HVS detection threshold, human cannot identify the object~\cite{lubin1997human}. For example, as shown in Figure~\ref{fig:bg_contrast}, when the luminance contrast between the white and black bars in the grating decreases, we hardly identify the black and white bars. When the contrast reaches zero, we cannot perceive any bar in the image.


\begin{figure}[ht!]
    \centering
    \includegraphics[width=0.8\linewidth]{csfoutputv2.pdf}
    \caption{The contrast sensitivity function curve depicts the coupling effect between spatial frequency and luminance contrast on human visual perception.}
    \vspace{-2em}
    \label{fig:csf}
\end{figure}



\subsection{Contrast Sensitivity Function}\label{background:csf}

Though the HVS is affected by the contrast and frequency of the image, these factors do not influence human perception independently.
% \alexin{It's a transition from contrast and luminance to the CSF subsection.} 
Viewing distance also affects human perception. To consider all these factors, researchers proposed a contrast sensitivity function (CSF).
%Instead of 
Different from
image frequency, CSF utilizes the spatial frequency, which refers to the number of pairs of white and bar (Figure~\ref{fig:bg_freq}) on the retina within a given distance~\cite{sekuler1985perception}. For example, the image frequency can be regarded as the human view of the grating at zero distance. When the human observes the grating and moves away from it, the fixed-size gratings become smaller in the human eye, increasing the spatial frequency. In addition to the viewing distance, CSF assesses the HVS's contrast threshold over a range of spatial frequency~\cite{barten1999contrast}. Specifically, CSF indicates that the HVS exhibits different contrast thresholds at different spatial frequencies. As shown in Figure~\ref{fig:csf}, when the spatial frequency of grating becomes very high, the HVS requires a great contrast between bars. Otherwise, the human cannot identify those bars in the invisible area. Inspired by the CSF, our method leverages human vision characteristics to generate privacy-preserving visualization. 

% The contrast threshold of the HVS is determined by perceived stimuli spatial frequency. Not only determined by the cycle number of black and white bars within an image (Figure~\ref{fig:bg_freq}), spatial frequency is also influenced by the viewing distance, and it . For example, when a human observes a grating image and moves away from it, the fixed-size gratings in the image take up fewer degrees of visual angles (the gratings become smaller in the human eye). In contrast, when a human observes the fixed-size image and moves closer, the gratings will take up greater visual angle degrees (the gratings will appear larger). Therefore, the spatial frequency of a fixed-size image depends on the viewing distance. 



% Inspired by the CSF, we design a masking scheme to adjust the frequency and contrast of the visualization. As a result, a shoulder surfer at a distance perceives the masked visualization as being of a high spatial frequency, and the modified contrast does not match the shoulder surfer's contrast sensitivity. However, the frequency of visualization received by users is relatively low, and the contrast can satisfy their sensitivity. Consequently, only the user is able to view the masked visualization.
% As shown in Figure~\ref{fig:csf} (a), CSF indicates that a human usually reaches peak sensitivity to low spatial frequencies at about 1-4 cycles per degree (c/d). With the spatial frequency increasing, the sensitivity drops quickly and will close to 0 after 60 cycles per degree. As a result, very high frequencies are invisible to the HVS regardless of contrast.
% Figure~\ref{fig:csf} (b) shows varying gratings to represent the visual stimuli to HVS. For instance, because the gratings in the top right corner have high frequencies and low contrast, humans are unable to distinguish between them. In contrast, since the gratings have high contrast and low frequencies, humans can distinguish them easily in the bottom middle. The CSF also contends that a high spatial frequency (e.g., 30 c/d) requires greater contrast than a low spatial frequency (e.g., five c/d) for the HVS to perceive.

% Since spatial frequencies are measured in cycles per degree, these frequencies vary with viewing distance. For example, when a human observes Figure~\ref{fig:csf} (b) and moves away from it, the fixed-size gratings in the image take up fewer degrees of visual angles (the gratings become smaller in the human eye). In contrast, when a human observes the fixed-size image and moves closer, the gratings will take up greater visual angle degrees (the gratings will appear larger). Therefore, the spatial frequency of a fixed-size image depends on the viewing distance and can be calculated as follows~\cite{isenberg2013hybrid}:

% \begin{equation}\label{formula:cpd}
%     cpd  =\frac{\pi}{\left(360 \cdot \text{tan}^{-1} \left( \frac{psize }{2d\cdot ppc} \right) \right)}
% \end{equation}

% where \textit{psize} denotes the size of a pixel in a display, \textit{d} represents the viewing distance, and \textit{ppc} is an abbreviation for "pixels per cycles". The \textit{ppc}, which stands for how many pixels are occupied by a frequency in a cycle, can be found in an image using the Fourier transform.




% \subsection{Marks and Visual Channels}\label{background:vis}

% Information visualization uses visual encodings to characterize datasets so that humans can see the visual encodings, comprehend the dataset behind the encodings, and gain further insight from the data~\cite{munzner2008process}. 
% Visual encodings consist of marks and visual channels.
% Marks are essential elements and refer to geometrically fundamental objects (e.g., points in a scatterplot, bars in a bar chart), and visual channels determine the appearance of marks such as position, size, and shape~\cite{munzner2014visualization}.
% Since all mark types occupy a particular region in the 2D visualization in visual perception, we apply a novel masking scheme to these marks to ensure corresponding visual channels remain the same at a close viewing distance. As a result, the visual encodings are preserved because the marks and visual channels are unchanged after the maks processing. Therefore, the processed visualization, namely privacy-preserving visualization, has the same visual encoding as the original visualization. Users in proximity to the visualization are able to comprehend the dataset from the privacy-preserving visualization.

% users can obtain the same information without much information loss from privacy-preserving visualizations. 

% Users in proximity to the visualization are then able to view the privacy-preserving visualizations, which consist of marks and channels. 


\begin{figure}[ht!]
    % \centering
    \includegraphics[width=\linewidth]{overviewv5.pdf}
    \caption{\alexin{An overview of the proposed method. 
    % The proposed masking scheme 
    It takes a visualization image as input and generates the privacy-preserving visualization.
    It comprises two levels of processing: coarse-grained masking and fine-grained masking.
    % The coarse-grained masking adjusts the spatial frequency and luminance of the input visualization. 
    % Fine-grained masking considers the difference between visual marks (i.e., line-based and area-based visual marks) and further enhances their privacy-preserving effect.
    }
    % Additionally, the masking scheme takes into account different properties of visual marks (i.e., line-based and area-based elements) and proposes customized schemes for different elements as fine-granular masking. 
    % \yong{1. Is it newly added? If so, why not highlighting it? 2. What is ``coarse-granular'' and ``Fine-granular''? Have you double-checked such terms? Please revise the figure and check it throughout the paper.}
    }
    \vspace{-2em}
    \label{fig:overiew}
\end{figure}