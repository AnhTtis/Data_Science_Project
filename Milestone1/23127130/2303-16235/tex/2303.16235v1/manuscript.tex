% CVPR 2023 Paper Template
% based on the CVPR template provided by Ming-Ming Cheng (https://github.com/MCG-NKU/CVPR_Template)
% modified and extended by Stefan Roth (stefan.roth@NOSPAMtu-darmstadt.de)

\documentclass[10pt,twocolumn,letterpaper]{article}

%%%%%%%%% PAPER TYPE  - PLEASE UPDATE FOR FINAL VERSION
% \usepackage[review]{cvpr}      % To produce the REVIEW version
\usepackage{cvpr}              % To produce the CAMERA-READY version
%\usepackage[pagenumbers]{cvpr} % To force page numbers, e.g. for an arXiv version
% Include other packages here, before hyperref.
\usepackage{graphicx}
\usepackage{amsmath}
\usepackage{amssymb}
\usepackage{booktabs}
\usepackage{enumitem}
\usepackage[accsupp]{axessibility}

% It is strongly recommended to use hyperref, especially for the review version.
% hyperref with option pagebackref eases the reviewers' job.
% Please disable hyperref *only* if you encounter grave issues, e.g. with the
% file validation for the camera-ready version.
%
% If you comment hyperref and then uncomment it, you should delete
% ReviewTempalte.aux before re-running LaTeX.
% (Or just hit 'q' on the first LaTeX run, let it finish, and you
%  should be clear).
\usepackage[pagebackref,breaklinks,colorlinks]{hyperref}


% Support for easy cross-referencing
\usepackage[capitalize]{cleveref}
\crefname{section}{Sec.}{Secs.}
\Crefname{section}{Section}{Sections}
\Crefname{table}{Table}{Tables}
\crefname{table}{Tab.}{Tabs.}
\newcommand{\TZ}[1]{\textcolor{cyan}{\textbf{TZ: #1}}}
\newcommand{\WYH}[1]{\textcolor{red}{\textbf{WYH: #1}}}

%%%%%%%%% PAPER ID  - PLEASE UPDATE
\def\cvprPaperID{8301} % *** Enter the CVPR Paper ID here
\def\confName{CVPR}
\def\confYear{2023}
\def\Methodname{STSSL}

\begin{document}



%%%%%%%%% TITLE - PLEASE UPDATE
% \title{ Point Cloud Self-supervised Learning in the wild  }
\title{Spatiotemporal Self-supervised Learning for Point Clouds in the Wild}


\author{
Yanhao Wu $^{1}$ ~
Tong Zhang$^{2}$ ~
Wei Ke$^{1}$ ~
Sabine Süsstrunk$^{2}$ ~
Mathieu Salzmann$^{2}$\\
$^1$ School of Software Engineering, Xi'an Jiaotong University, China \\
$^2$ School of Computer and Communication Sciences, EPFL Switzerland \\
}

% \author{First Author\\
% Institution1\\
% Institution1 address\\
% {\tt\small firstauthor@i1.org}
% % For a paper whose authors are all at the same institution,
% % omit the following lines up until the closing ``}''.
% % Additional authors and addresses can be added with ``\and'',
% % just like the second author.
% % To save space, use either the email address or home page, not both
% \and
% Second Author\\
% Institution2\\
% First line of institution2 address\\
% {\tt\small secondauthor@i2.org}
% }
\maketitle

%%%%%%%%% ABSTRACT
\begin{abstract}
Self-supervised learning (SSL) has the potential to benefit many applications, particularly those where manually annotating data is cumbersome. One such situation is the semantic segmentation of point clouds. In this context, existing methods employ contrastive learning strategies and define positive pairs by performing various augmentation of point clusters in a single frame. As such, these methods do not exploit the temporal nature of LiDAR data. In this paper, we introduce an SSL strategy that leverages positive pairs in both the spatial and temporal domain. To this end, we design (i) a point-to-cluster learning strategy that aggregates spatial information to distinguish objects; and (ii) a cluster-to-cluster learning strategy based on unsupervised object tracking that exploits temporal correspondences. We demonstrate the benefits of our approach via extensive experiments performed by self-supervised training on two large-scale LiDAR datasets and transferring the resulting models to other point cloud segmentation benchmarks. Our results evidence that our method outperforms the state-of-the-art point cloud SSL methods.~\footnote{Our code and pretrained models will be found at \href{https://github.com/YanhaoWu/STSSL}{https://github.com/YanhaoWu/STSSL}. Correspondence to Ke Wei.}

% Our code will be available at
% \href
  % {https://github.com/YanhaoWu/CVPR2023_STSSL}
  % {https://github.com/YanhaoWu/CVPR2023\_STSSL}



%, including the single-frame ones.

   
%especially the ones that need a tremendous amount of human annotation. Semantic segmentation on the point cloud is one of the tasks needed to conduct such pre-training.  Compared to image-based pre-training, self-supervised learning on point clouds is less studied due to its difficulty in finding invariance terms. Therefore, most of the existing works focus on indoor scenes, where different views share the same regions. However, such correspondences will seldom occur in outdoor scenes and thus the methods may fail. We thus propose a novel self-supervised learning framework that seeks correspondences in both spatial and temporal dimensions.  To that end, we design a point-to-cluster learning strategy to firstly aggregate spatial information to distinguish objects, followed by a temporal cluster-to-cluster learning strategy to further refine the object-level features. We conduct our training on two large-scale LiDAR datasets and transfer the model to other datasets for segmentation tasks. Our extensive experiments demonstrate that our method achieves state-of-the-art pre-training on outdoor scenes, suggesting self-supervised learning on point clouds in the wild is a promising direction for future works.
\end{abstract}

%%%%%%%%% BODY TEXT
\vspace{-0.75em}

\section{Introduction}
\vspace{-0.5em}

\label{sec:intro}
Semantic segmentation from LiDAR point clouds can be highly beneficial in practical applications, e.g., for self-driving vehicles to safely interact with their surroundings. Nowadays, state-of-the-art methods~\cite{segmentation1, segmentation2, segmentation3} achieve this with deep neural networks. While effective, the training of such semantic segmentation networks requires large amounts of annotated data, which is prohibitively costly to acquire, particularly for point-level LiDAR annotations~\cite{depthcontrast}. By contrast, with the rapid proliferation of self-driving vehicles, large amounts of \emph{unlabeled} LiDAR data are generated. Here, we develop a method to exploit such unlabeled data in a self-supervised learning framework.

\begin{figure}[t]
  \centering
%   \fbox{\rule{0pt}{2in} \rule{0.9\linewidth}{0pt}}
  \includegraphics[width=1\linewidth]{figure_pdf/overview2.pdf}
  \caption{{\bf Our method vs existing ones.} {\bf (Top)} Previous methods create positive pairs for SSL by applying different augmentations, $\tau_{1}$ and $\tau_{2}$ (e.g., random flipping, clipping), to a single frame. {\bf (Bottom)} By contrast, we leverage both spatial and temporal information via a point-to-cluster and an inter-frame SSL strategy. Points in the same color are from the same cluster in the latent space.  
  %For better visualization, only the points belonging to the specified cluster are colorized. 
  }
%   \WYH{The main difference between our method and previous methods. Points with the same color are adjacent in feature space. With features extracted by the previous method pre-trained model, the cluster is divided into several colors. By contrast our method yields features such that most of the cluster points have the same color. $\tau_{1}$ and $\tau_{2}$ are subsets of $\tau$ (a series of data augmentation transforms such as random flipping, random clipping, etc.) For better visualization, only the points belonging to the specified cluster are colorized.}}
  \vspace{-0.6cm}

  \label{fig:overview}
\end{figure}

\begin{figure}[t]
  \centering
%   \fbox{\rule{0pt}{2in} \rule{0.9\linewidth}{0pt}}
  \includegraphics[width=1\linewidth]{figure/angle_different_1.jpg}
  \caption{Cars in the {\bf same frame but under different illumination angles}. Note that the main source of difference between the two instance point clouds arises from the different illumination angles.}
  \label{fig:angle_different_1}
  \vspace{-0.6cm}

\end{figure}

Self-supervised learning (SSL) aims to learn features without any human annotations~\cite{depthcontrast, pointcontrast, segcontrast, pointssl1, pointssl2, pointssl3, pointssl4, pointssl5} but so that they can be effectively used for fine-tuning on a downstream task with a small number of labeled samples. This is achieved by defining a pre-task that does not require annotations. While many pre-tasks have been proposed~\cite{ssl_survey}, contrastive learning has nowadays become a highly popular choice~\cite{depthcontrast, gcc3d, proposalcontrast, segcontrast, pointcontrast}. In general, it aims to maximize the similarity of positive pairs while potentially minimizing that of negative ones. In this context, most of the point cloud SSL literature focuses on indoor scenes, for which relatively dense point clouds are available. Unfortunately, for outdoor scenes, such as the ones we consider here, the data is more complex and much sparser, and creating effective pairs remains a challenge.
%Motivated by the difficulty of creating valuable negative pairs, BYOL \cite{BYOL} focuses only on positive ones.
%, outperforming other contrastive SSL in the image classification task. However, such success does not migrate to dense prediction due to its dependence on how the positives are constructed.

% outperforming other contrastive SSL techniques. However, the success of this strategy strongly depends on how the positive pairs are constructed, and~\cite{BYOL} only studied this for image-related tasks. 
% Previous method choose whole-scene \cite{STRL}, points\cite{pointcontrast} or clusters\cite{segcontrast} as positive pairs, 



% \WYH{Nowadays, there are SSL approaches for LiDAR point cloud data such as SegContrast\cite{segcontrast} and DepthContrast{\cite{depthcontrast}}. SegContrast and DepthContrast selects clusters or scenes as positive samples as shown at the top of Fig.\ref{fig:overview}, which has weak constraints on any individual's points and is not friendly to point level downstream tasks. Furthermore, it is very important to sample a variety of intra-class instances as positive samples. We found that the main source of intra class changes in the lidar point cloud is the illumination angle of the sensor as shown in Fig.\ref{fig:angle_different_1}, which cannot be simulated through data augmentation \cite{augmentation5}. So it is difficult for them to construct effective positive pairs without annotations.}

Several approaches~\cite{segcontrast,depthcontrast} have nonetheless been proposed to perform SSL on outdoor LiDAR point cloud data.  As illustrated in the top portion of Fig.~\ref{fig:overview}, they construct positive pairs of point clusters or scenes by applying augmentations to a single frame. As such, they neglect the temporal information of the LiDAR data.
%aim to find the positive correspondent clusters or scenes within a single frame by employing augmentations, however, the features extracted per scene or cluster are not suited for point segmentation. Furthermore, they neglect the nature of LiDAR, a continuous sequence with strong temporal information. 
By contrast, in this paper, we introduce an SSL approach to LiDAR point cloud segmentation based on extracting effective positive pairs in both the spatial \emph{and temporal} domain. 

To achieve this without requiring any pose sensor as in~\cite{pointcontrast,STRL}, we introduce (i) a {\bf point-to-cluster (P2C)} SSL strategy that maximizes the similarity between the features encoding a cluster and those of its individual points, thus encouraging the points belonging to the same object to be close in feature space; (ii) a {\bf cluster-level inter-frame self-supervised learning} strategy that tracks an object across consecutive frames in an unsupervised manner and encourages feature similarity between the different frames. These two strategies are depicted in the bottom portion of Fig.~\ref{fig:overview}. 

Note that the illumination angle of one object seen in two different frames typically differs. As shown in Fig.~\ref{fig:angle_different_1}, this is also the main source of difference between two objects of the same class in the same frame. Therefore, our inter-frame SSL strategy lets us encode not only temporal information, but also the fact that points from different objects from the same class should be close to each other in feature space. As simulating different illumination angles via data augmentation is challenging, our approach yields positive pairs that better reflects the intra-class variations in LiDAR point clouds than existing single-frame methods~\cite{segcontrast,depthcontrast}.




%Our approach is based on the intuitions that (i) points belonging to the same object should be close in feature space; and (ii) points from different objects from the same class should be closer to each other than points from different classes. Turning these intuitions into an SSL approach, however, is challenged by the fact that individual points typically contain only limited semantic information, and that identifying different objects from the same class is a difficult task in itself in the absence of annotations.




% Moreover, it is challenging to create inter-frame positive pairs due to the change of the illumination angle of the sensor as shown in Fig.\ref{fig:angle_different_1}.

% Thus, it leads to a failure of finding temporal associations.


% More importantly, such property cannot be simulated through data augmentation \cite{augmentation5}, and it could in turn significantly benefits to the SSL by creating temporal positive pairs properly.



%\emph{What kind of feature space do we need to get from pre-training.} We think {\bf 1)} points from one object should be close in feature space and {\bf 2)} points from the same class should be closer than other classes \cite{crossimage}. 

%To address (i), we note that individual points typically contain only limited semantic information~\cite{pointssl5} and selecting the most informative points is a challenging task in itself. We therefore introduce a {\bf Point-to-Segment(P2S)} SLL strategy that leverages a clustering-based over-segmentation of the point cloud to maximize the similarity between the  features encoding a segment and those of its individual points. To tackle (ii)




%To address this, we first propose to reason at the level of clusters and build a clustering-based over-segmentation of the input point cloud. This lets us tackle (i) via a {\bf Point-to-Cluster (P2C)} SSL strategy that maximizes the similarity between the features encoding a cluster and those of its individual points. Second, having the consistent feature representation of each point in each frame thanks to P2C, we therefore address (ii) by performing {\bf cluster-level inter-frame self-supervised learning}, tracking an object across consecutive frames in an unsupervised manner, and encouraging feature similarity between different frames, for which the illumination angles typically differ as shown in Fig.~\ref{fig:angle_different_1}. Such property cannot be simulated through data augmentation, and in turn it significantly benefits to the SSL tasks.
%Importantly, and in contrast with~\cite{pointcontrast,STRL}, our method aggregates spatiotemporal information without requiring any pose sensor. The main difference between our method and previous methods is shown in Fig.\ref{fig:overview}.





% we note that the main source of intra-class variations in LiDAR point clouds is the illumination angle of the sensor. Considering that we observe dynamic scenes, 

%, and it is difficult to select a valuable point from a large number of points, which makes choosing two points as positive samples has little benefit for feature learning. We need to find a training level that can both constrain the features of points and allow positive samples to have enough semantic information. So we proposed a novel training level called {\bf Point-to-Segment(P2S)} which selects segments and points as positive samples respectively. We cluster points to generate segments, which usually have enough semantic information(such as vehicles, pedestrians, etc.), so that segments and their corresponding points are taken as positive samples. By setting reasonable parameters, we make the clustering results nearly all in the over-segmentation state \cite{seggruop}; in other words the points of one particular segment must be closely to each other, with similar semantic information. In this way, by maximizing the similarity between segment features and point features, we not only achieve the constraint on point-wise features, but also have greater benefits for feature learning.

%In order to achieve the goal of {\bf 2)}, there should be support overlap between positive samples and all other intra-class instances \cite{augmentation5}. In image domain, data augmentation(e.g. cropping, scaling, color distortion, etc.) is used to make samples be more alike to other intra-class images \cite{moco}. However, Lidar point clouds have scale invariance and no color. Furthermore, we found that the main reason for the difference in the representation of intra-class instances is the illumination angle of the Lidar. For example, when a car is in front of and behind the Lidar, it will have an almost completely different representation, as shown in Fig.\ref{fig:angle_different_1}. So it is difficult to use data augmentation to improve the degree of overlap between them. For this reason, we propose to use {\bf object-level inter-frame contrastive learning} to fill this gap. Due to the relative displacement between the Lidar and the environment, we can obtain the representation of the same object in different frames and different illumination angles by unsupervised tracking. We choose the same object in different frames as the positive samples and maximize the similarity between them. 

% The main difference between our method and previous methods is shown in Fig.\ref{fig:overview}. Previous methods only focus on the relations between intra-frame clusters,thus, it lacks the robustness of view changes. Differently, our proposed method not only focuses on the intra-frame point-to-cluster relations, but also considers inter-frame cluster relations. 

% \WYH{The main difference between our method and previous methods is shown in Fig.\ref{fig:overview}. Previous methods only focus on spatial features aggregation between intra-frame clusters as illustrated in the upper of Fig.\ref{fig:overview}, thus, it lacks the robustness of view changes. Differently, our proposed method not only focuses on the intra-frame point-to-cluster relations, but also considers temporal information by aggregating inter-frame cluster features as shown in to bottom of Fig.\ref{fig:overview}}.
% \WYH{ Different from previous methods, our proposed method not only focuses on the intra-frame point-to-cluster relations, but also considers temporal information by aggregating inter-frame cluster features as shown in to bottom of Fig.\ref{fig:overview}}.


Our contribution can be summarized as follows:
\begin{itemize}
\vspace{-0.5em}
\item We introduce an SSL strategy for point cloud segmentation based only on positive pairs. It does not require any external information, such as pose, GPS, and IMU.\vspace{-0.5em}
\item We propose a novel Point-to-Cluster (P2C) training paradigm that combines the advantages of point-level and cluster-level representations to learn a structured point-level embedding space.
%semantic 
\vspace{-0.5em}
\item We introduce the use of cluster-level inter-frame self-supervised leaning on point clouds generated by a LiDAR sensor, which introduces a new way to integrate temporal information into SSL.
% It essentially fill the gap that data augmentation cannot improve the overlap between intra-class instances in point cloud data.
\vspace{-0.5em}
\end{itemize}
Our experiments on several datasets, including KITTI\cite{kitti}, nuScene\cite{nuscene}, SemanticKITTI\cite{semantickitti} and SemanticPOSS\cite{semanticposs}, evidence that our method outperforms the state-of-the-art SSL techniques for point cloud data.

\begin{figure*}[t]
  \centering
%   \fbox{\rule{0pt}{2in} \rule{0.9\linewidth}{0pt}}
  \includegraphics[width=1\linewidth]{figure_pdf/overview_detail2.pdf}
  \vspace{-0.6cm}

  \caption{{\bf Overview of our \Methodname.} Given a sequence of LiDAR point clouds, we first perform clustering and unsupervised tracking to associate clusters in different frames. 
  %This yields a set of indices indicating to which cluster each point belongs. 
  At each training iteration, we select two frames and apply augmentations to generate two views for each frame (i.e., $P^{1}_{m}$, $P^{1}_{n}$, $P^{2}_{m}$, $P^{2}_{n}$). A feature extractor (Backbone) is then used to obtain point-wise features in the four views, and we collect the features belonging to each cluster. In $P^{2}_{m}$, $P^{2}_{n}$, we further apply a cluster-wise pooling layer to the features to generate cluster-wise features. Finally, we minimize the distance between the point features and the corresponding cluster features from $P^{1}_{m}$, $P^{1}_{n}$, and between the cluster features obtained from associated clusters in $P^{2}_{m}$, $P^{2}_{n}$. $\tau_{1}$ and $\tau_{2}$ are data augmentations, such as random flipping and random clipping.
  }
 \vspace{-0.4cm}
  \label{fig:overview_detail}
\end{figure*}
%--------------------------------------------------------------------------------
\vspace{-0.5em}

\section{Related Work}
\vspace{-0.5em}

\label{sec:related works}

%This paper concentrates on self-supervised learning for 3D data. We therefore review methods first for self-supervised learning for images and then for 3D data.

{\bf Self-supervised learning for images.} Self-supervised learning for images has developed at a fast pace in recent years~\cite{imagessl1, imagessl2 ,imagessl3, imagessl4, imagessl5, imagessl6, imagessl7}. Existing methods follow different paradigms, such as generation-based methods~\cite{generatemodel}, clustering methods~\cite{clustering1,clustering2,clustering3,clustering4} and contrastive learning methods~\cite{contrastivemode1, contrasivemodel2, contrastivemodel3}. Currently, BYOL~\cite{BYOL}, a self-supervised learning method that uses only positive pairs in its loss function, constitutes the state of the art. Intrigued by the success of such contrastive learning strategies, several works have studied the principles behind this approach, with a particular focus on the role of data augmentation~\cite{augmentation1, augmentation2, augmentation3, augmentation4}. In~\cite{augmentation5}, it was observed that data augmentation creates a certain degree of “chaos” between the intra-class samples that helps them to become more similar. Similarly, LoGo~\cite{imagessl8} also introduce local and global crops differently to handle the variance due to the augmentation.
%And contrastive learning methods cluster intra-class samples in feature space by aligning positive samples. 
Our method is inspired by BYOL but targets 3D data. Because of the fundamentally different nature of 2D images and 3D point clouds, data augmentation designed for images does not directly apply to the 3D domain.

{\bf Self-supervised learning for 3D data.}
As in the image case, the number of self-supervised learning methods for 3D data has grown rapidly~\cite{depthcontrast, pointcontrast, segcontrast, pointssl1, pointssl2, pointssl3, pointssl4, pointssl5}, with examples such as DepthContrast~\cite{depthcontrast}, PointContrast~\cite{pointcontrast}, GCC-3D\cite{gcc3d},  ProposalContrast~\cite{proposalcontrast}, STRL~\cite{STRL} and SegContrast~\cite{segcontrast}. Nevertheless, these methods still suffer from severe limitations. In particular, many methods~\cite{pointcontrast, STRL} need the camera pose in each frame to find correspondences to use as positive pairs. 
%STRL~\cite{STRL} argues that these methods ignore the temporal nature of the data, treating the individual frames separately. To overcome this, it projects the sequential coordinate changes back to the first frame as an augmentation to help the network to learn such invariance. 
While effective for indoor scenes, the points in outdoor scenes are much sparser, and even with the ground-truth poses, correspondences between points are hard to obtain. By contrast, SegContrast, ProposalContrast, and DepthContrast~\cite{segcontrast,proposalcontrast, depthcontrast} specifically tackle the outdoor scenario, without requiring camera poses. However, they aggregate features in each region through either max or average pooling, and pull region-level features from different views together, therefore, it does not have constraint for each point. More importantly, they fail to find a way to associate the points in different time frames. By contrast, our method only utilizes point cloud data and does not rely on camera calibration to aggregate spatial and temporal features. Furthermore, we propose a point-to-cluster training paradigm that combines the advantages of point-level and cluster-level discrimination.

% While~\cite{pointcontrast, STRL} exploit such temporal information, \cite{pointcontrast} only encourages instance discrimination at the point-level, thus neglecting spatial context~\cite{pointssl5}, and

% STRL~\cite{STRL} considers the whole point cloud as an instance, i.e., performs scene-level discrimination, resulting in instances that do not have a precise semantic meaning, since point cloud scenes typically comprise multiple objects.\TZ{Furthermore, STRL~\cite{STRL} adds sequential continuous coordinates' change as an augmentation to help the network to learn such invariance. To that end, they require the camera pose in each frame to obtain a unified coordinates. }

% While~\cite{pointcontrast, } exploit such temporal information, \cite{pointcontrast} only encourages instance discrimination at the point-level, thus neglecting spatial context~\cite{pointssl5}, and

% \WYH{PointContrast\cite{pointcontrast} projects points into global coordinates and find points with same coordinates in different frames as positive samples so that camera calibration is needed. And \cite{pointcontrast} only encourages instance discrimination at the point-level, thus neglecting spatial context~\cite{pointssl5}}.
% Furthermore,~\cite{pointcontrast, STRL} take advantage of camera calibration; for instance, they cannot be applied to RedWood-vid~\cite{RedWood} for which camera calibration is unavailable. 
% Finally, the existing methods~\cite{pointcontrast, depthcontrast, STRL} focus on indoor 3D scene, 
%S3DIS\cite{S3DIS} and Synthia4D\cite{synthia4D} for example, which are not applicable to outdoor 3D data such as SemanticKITTI\cite{semantickitti}, 
% and tend to fail for outdoor scenes in which the 3D information is much sparser and contains multiple objects, in which they can not find correspondence even the camera pose is given. Thus, it amplify the shortcomings of scene-level and point-level discrimination. 
% To overcome this, \cite{segcontrast} and \cite{proposalcontrast} rely on cluster-level and proposal-level discrimination, respectively. 
% %using clustering or grouping. 
% However, cluster-level and proposal-level discrimination cannot guarantee spatially adjacent points to have similar features
% which plays an important role in point cloud data~\cite{gcc3d}.
% \TZ{Those features are obtained by aggregating features in each region, therefore, those methods do not require every point from the same region to have a consensus}.
% For example, \cite{segcontrast} only requires that the clusters as positive samples have similar features, while the points in the clusters can be distinct. 
% By contrast, our method only utilizes point cloud data and does not rely on camera calibration to aggregate Spatial and temporal features. Furthermore, we propose a point-to-cluster training paradigm that combines the advantages of point-level and cluster-level discrimination.




% \begin{figure}[t]
%   \centering
% %   \fbox{\rule{0pt}{2in} \rule{0.9\linewidth}{0pt}}
%   \includegraphics[width=1\linewidth]{figure/angle_different_1.jpg}
%   \caption{Representation of cars obtained under different sensor illumination angles}
%   \vspace*{-5pt}
%   \label{fig:angle_different_1}
% \end{figure}


\vspace{-0.5em}

\section{Method}
\vspace{-0.5em}

\label{sec:Method}
% Let us now introduce our Spatiotemporal Self-supervised Learning ({\Methodname}) approach for point clouds in the wild.
%takes the training strategy of BYOL~\cite{BYOL}. It 
% Based on BYOL~\cite{BYOL} , we devise our method, \Methodname, which
The overall framework is depicted in Fig.~\ref{fig:overview_detail} and contains three parts: clustering and unsupervised tracking, point-wise and cluster-wise feature extraction, and spatialtemporal feature aggregation. Below, we discuss these components in detail.

\subsection{Clustering and Unsupervised Tracking}

%{\color{red}Note: T frames. $k$-th frame has $N_k$ points and is divided to $M_k$ clusters with point-level feature $F$ and cluster-level feature $C$. }

Let $P = \left\{P^{1},P^{2},...,P^{T}\right\}$ denote a sequence of LiDAR point clouds with $T$ frames, where $P^k = \left\{p^{k}_{1},p^{k}_{2},...,p^{k}_{N_k}\right\}$ represents the $k$-th point cloud with $N_k$ 3D points $p^{k}_i \in {\mathbb{R}^{3}}$. The segmentation map of each $P^k$ is obtained by applying cluster to the non-ground points, where the ground points are eliminated by RANSAC~\cite{RANSAC}. Thanks to the over segmentation property of DBSCAN~\cite{DBSCAN}, each point has high possibility to represent the same semantic meaning with other points in the same cluster. This process yields a set of $M_k$ clusters 
%\cite{seggruop} denoting as 
$S^{k}=\left\{S_{1}^{k}, S_{2}^{k}, ..., S_{M_k}^{k} \right\}$.

We will leverage these clusters to define a point-to-cluster loss for SSL, encoding a notion of spatial similarity. 
Furthermore, we will also exploit them to create temporal positive pairs for SSL via the unsupervised tracking strategy described below. Thus, the mechanism allows similar clusters to be merged into the same one in later stage to achieve final segmentation.


% We obtain segmentation of each $P^k$ by using RANSAC~\cite{RANSAC} to filter out the ground points and then applying DBSCAN~\cite{DBSCAN} to cluster the non-ground points. Th

% , which each cluster contains points that are close enough to each other and, thanks to the over-segmentation of DBSCAN, the points in the same cluster represent the same semantic meaning with high possibility.}

% in which each cluster contains points that are close enough to each other to represent the same
% %a similar
% semantic content.


%The over-segmentation allows us to apply the point-to-segment framework for training which is unreachable for scene-level~\cite{???} and proposal-level~\cite{???} frameworks because they cannot precisely find points with similar semantic information.
% $ M=\left\{ s_{1}, s_{2},...,s_{N} \right\}, s_{i} \in Z$, and $-1 \leq s_{i} \leq N_{s}$. $N_{s}$ is the number of segments in the point cloud. $s_{i}=j$ means that point $i$ is a part of the segment $j$.
% By setting reasonable parameters, we make the clustering results nearly all in the over-segmentation state \cite{seggruop}; in other words the points of one particular segment must be closely to each other, with similar semantic information. 

Specifically, unsupervised tracking is achieved by matching the clusters in two adjacent frames, \eg, frames $k$ with $M_{k}$ clusters and $(k+1)$ with $M_{k+1}$ clusters. To this end, we define a matching degree matrix $D \in R^{M_{k} \times M_{k+1}}$ as
\begin{equation}
    D = D_{loc} + \alpha * D_{feat}, 
    \label{eq:tracking}
\end{equation}
where $D_{loc}$ is the matrix of pairwise Euclidean distances between the cluster centers in the two frames, $D_{feat}$ is the matrix of pairwise feature distance, and $\alpha \in (0, 1)$ is a weight balancing the two matrices. The center 
%$C_{j}^{k}$ 
of cluster $j$ in frame $k$ is taken as the average of all 3D points belong to this cluster. More details regarding the cluster features is provided in Section~\ref{subsec:feature}.
We then use $D$ to match the clusters in both frames using the Hungarian algorithm\cite{hungraian}. For the unmatched clusters, we will create trajectories for the one just appears in current frame, and abandon the trajectories of the clustering no longer exists. More details can be found in the supplementary

% Some clusters will not be matched. They may be new clusters in frame $k+1$, or clusters that will leave the scene in frame $k$. We create new trajectories for the new clusters, and end the trajectories of clusters leaving the scene. More details about unsupervised tracking can be found in supply materials.

%The center coordinate $C_{j}^{k}$ of cluster $S_{j}^{k}$ is the average coordinate of all points in this cluster, defined as 
%\begin{equation}
%C_j^k = \frac{\sum {\delta_{i,j}^k * p_i^k}}{\sum {\delta_{i,j}^k}},
%\end{equation}
%where $\delta_{i,j}^k = 1$ if ${p_i^k}$ belongs to $S_{j}^{k}$ and $\delta_{i,j}^k = 0$ otherwise. 

% which is expressed as $L^{k}=\left\{l_{1}^{m}, l_{2}^{m}, ..., l_{N}^{m} \right\}$, $l_{i}^{m} \in R^{3}$ in frame $m$, and $N$ is the number of clusters in this point cloud. Take cluster $j$ as an example, let $I_{j}=\left\{i_{1}, i_{2}, ..., l_{N} \right\}$, $i_{k} \in [0, 1], k \in [1, N]$, and 
% \begin{equation} 
% \left\{ \begin{array}{l}
% {i_{k}} = 1,\quad if \; s_{k} = j\\
% {i_{k}} = 0,\quad if \; s_{k} \ne j
% \end{array} \right.\
% \end{equation}


% As the training goes on, the network is now able to extract geometric features. Matching a specific cluster in different views is our next goal. First, We use the method mentioned above to extract a feature for each cluster to calculate the similarity between them. Let the feature of each cluster in frame $m$ as $C^{m}=\left\{C_{1}^{m}, C_{2}^{m}, ..., C_{N}^{m} \right\}$. Similarly, we can calculate the coordinate $L^{n}$ in the next frame $i(i=m+1)$, and extract the feature $C^{i}$. Second, to achieve better matching results, we also include the spatial distance. We calculate the feature similarity and spatial distance of any two clusters within two frames and normalize them respectively to get $F_{mi}, F_{mi} \in R^{N_{m} \times N_{i}}$ and $D_{mi}, D_{mi} \in R^{N_{m} \times N_{i}}$. Then we define the matching degree of clusters as $S_{mi} = D_{mi} + \alpha * F_{mi}, \alpha \in (0, 1)$. In doing so, we can now use the Hungarian algorithm\cite{hungraian} to match the object with the highest matching degree between two frames. 

%In Fig.~\ref{fig:track_show}, we visualize clusters tracked across frames with our approach.
Thanks to the combination of 3D information and learned representations, this matching strategy allows us to robustly track a cluster across multiple frames. This lets us construct long-range positive pairs where a cluster is observed under different illumination angles, thus corresponding to a challenging positive sample for SSL. We will discuss how we exploit such pairs in Section~\ref{sec:spatiotemporal}.

\vspace{-0.25em}
\subsection{Feature Extraction}
\vspace{-0.25em}

\label{subsec:feature}

As discussed above, we extract learned features from the input point clouds. Specifically, we extract two types of features: point-level ones and cluster-level ones. To this end, given an input point cloud $P^k$, we first apply data augmentation to obtain two view $\tilde{P}^{k}$ and $\bar{P}^{k}$. One view will be used to extract point-level features and the other for cluster-level features. This will let us create more challenging point-to-cluster pairs for the SSL strategy discussed in Section~\ref{sec:spatiotemporal}.

%We randomly select two point clouds $P_{m}$ and $P_{n}$ from frames $m$ and $n$ and apply data augmentations to them to obtain views $P^{1}_{m}$ , $P^{2}_{m}$, $P^{1}_{n}$, $P^{2}_{n}$.

\textbf{Point-level Features.}
Following the BYOL~\cite{BYOL} format, let $f$ denote the backbone encoder. In our case, $f$ is MinkUnet~\cite{minkunet}. We forward pass $\tilde{P}^{k}$ through the backbone encoder to obtain a feature vector $y^k_{q} = f(\tilde{p}^{k}_{q})$ for every 3D point. We then group these representations according to the cluster to which each point belongs, giving us a set
$F^{k} = \left\{F^{k}_{1}, F^{k}_{2}, ..., F^{k}_{M_{k}} \right\}$, where $F^{k}_{i} \in \mathbb{R}^{N_{k,i} \times d}$, with $N_{k,i}$ the number of points in cluster $i$ from point cloud $k$, and $d$ the feature dimension of each $y^k_q$.

\textbf{Cluster-level Features.}
To extract cluster-level features, we first process $\bar{P}^{k}$ as above to extract $d$-dimensional point-level features. However, instead of simply grouping these features according to the clusters, we max-pool them according to the clusters. This yields a set of cluster-level features $C^{k} = \left\{c^{k}_1, c^{k}_{2}, ..., c^{k}_{M_{k}} \right\} $, where $c^k_i \in \mathbb{R}^{1\times d}$. 

% {\color{red}Since the framework is based on BYOL \cite{BYOL}, there are two extractors with the same structure but different parameters, so the extractors used in feature extraction are not exactly the same, and sometimes features need to pass through the predictor additionally. }
% A detailed description of this section can be found in supply materials.

\vspace{-0.25em}

\subsection{Spatialtemporal Feature Aggregation}
\vspace{-0.25em}

\label{sec:spatiotemporal}
Let us now describe our spatiotemporal SSL framework. It relies on two loss functions encoding two goals: \textbf {i)} points from one object should be close in feature space; \textbf {ii)} points from the same class should be closer to each other than other classes. We materialize these two objectives via the \textbf{Point-to-Cluster} and \textbf{Cluster-level Inter-frame Self-supervised Leaning} strategies discussed below.
%\cite{crossimage}.

\textbf{Point-to-Cluster Learning Strategy.} To encourage points from the same object to be close to each other, we minimize the distance between the point features of $\tilde{P}^{k}$ and the corresponding cluster features in view $\bar{P}^{k}$. Given the features discussed in Section~\ref{subsec:feature}, this is achieved via the loss function
\begin{equation}
L_{p2c} = \sum_{i=1}^{M_{k}} \sum_{j=1}^{N_{k,i}}
\left\| \frac{f^k_{i,j}}{\| f^k_{i,j} \|_2} - \frac{c^k_i}{\| c^k_i \|_2} \right\|_2^2\;,
\label{con:loss1}
\end{equation}
where $f^k_{i,j} \in \mathbb{R}^{1 \times d}$ denotes the feature vector from $F^k_{i}$ corresponding to point $j$ in cluster $i$.
In essence, this encourages the network to learn similar features for all points in the same cluster while being robust to different views of the point cloud.

%of objects from different views and recognize that points in the same cluster should have similar semantic features, which plays an important role in extracting features from a point cloud. Take frame $m$ as an example, after obtaining point-wise features $F_{m}=\left\{F_{m}^{1}, F_{m}^{2}, ..., F_{m}^{N_{m}} \right\}$ and cluster-wise features $C_{m} = \left\{C_{m}^{1}, C_{m}^{2}, ..., C_{m}^{N_{m}} \right\} $, we minimize L2 loss between $F_{m}^{i}$ and $C_{m}^{i}$. Since the number of $F_{m}^{i}$ and $C_{m}^{i}$ does not match, we copy-and-paste $C_{m} ^{i}$ to the same number of $F_{m}^{i}$ to calculate L2 loss. The loss function is defined as:
%\begin{equation}
%L_{p2c} = \sum_{i=0}^{N_{m}}
%{\left\Vert \frac{F_m^i}{\Vert F_m^i \Vert_2} - \frac{C_m^i}{\Vert C_m ^i \Vert_2} \right \Vert_2}^2
%\label{con:loss1}
%\end{equation}

% \begin{table*}[t]
% \centering
% \begin{tabular}{ccccccccccc}
% \hline
% Method name       & mIoU  & Car   & road   & sidewalk & building & fence & vegetation & terrain & parking & pole  \\ \hline
% From scratch  & 29.17 & 82.61 & 74.32  & 52.06    & 78.99    & 19.29 & 83.13      & 68.20  & 9.04  & 30.09  \\ 
% % STRL w \cite{STRL}  &17.04 & 41.65  & 41.78 & 30.66 & 57.58    & 11.89  & 71.21 & 43.20  & 0  & 10.17  \\
% STRL \cite{STRL} &16.64 & 47.66  & 56.17 & 23.17 & 58.63    & 13.68  & 69.96 & 41.91  & 0  & 3.12  \\
% DepthContrast \cite{depthcontrast} &30.91 & 88.80  & 69.51 & 49.87 & 82.67    & 22.70    & 83.36 & 67.38  & 9.32  & 48.69  \\
% SegContrast  \cite{segcontrast}     & 34.01  & 89.22 & 78.72 & 57.19    & 82.80    & 21.99 & 83.42      & 67.26   & 14.06 & 50.91 \\ 
% \hline
% SegContrast-BYOL & 33.94  & 89.34 & 78.56 & 57.29    & 82.86 & 21.19 & 83.26      & 67.30   & 14.09 & 50.54 \\ 
% Segcontrast-Inter &  36.03  &  90.52 &  81.45  &  62.90  &  83.54    &  21.46 &  84.37    & 72.52   & 17.05 &  {\bf 51.78} \\
% P2C  & 35.48  & 90.18 & 80.90 & 61.91    & 83.92    & 25.45 & 84.30  & 71.35  & 18.73 & 46.95 \\
% \Methodname~(ours) & {\bf 37.71}  & {\bf 91.19} & {\bf 85.75}  & {\bf 66.34}    & {\bf 85.64}    &  {\bf 27.14} & {\bf 85.42}  & {\bf 72.09}   & {\bf 23.63} &  47.74 \\ \hline
% % \Methodname(ours) & {\bf 37.19}  & {\bf 90.66} & {\bf 85.27}  & {\bf 66.77}    & {\bf 84.32}    &  {\bf 26.48} & {\bf 85.24}  & {\bf 73.25}   & {\bf 20.64} &  48.28 \\ \hline
% \end{tabular}
% \caption{Per-class IoU fine-tuning with 0.1 $\%$ labels. SegContrast-BYOL: Build SegContrast on the basis of BYOL. SegContrast-Inter: SegContrast with a additional inter-frame self-supervised learning stage after the original SegContrast. P2C: our approach with only the point-to-cluster loss function.}
% \label{table:per class IoU}
% \end{table*}


\begin{table*}[t]
\centering
\begin{tabular}{ccccccccccc}
\hline
Method name       & mIoU  & car   & road   & sidewalk & building & fence & vegetation & terrain & parking & pole  \\ \hline
From scratch  & 29.17 & 82.61 & 74.32  & 52.06    & 78.99    & 19.29 & 83.13      & 68.20  & 9.04  & 30.09  \\ 
% STRL w \cite{STRL}  &17.04 & 41.65  & 41.78 & 30.66 & 57.58    & 11.89  & 71.21 & 43.20  & 0  & 10.17  \\
STRL \cite{STRL} &16.64 & 47.66  & 56.17 & 23.17 & 58.63    & 13.68  & 69.96 & 41.91  & 0  & 3.12  \\
DepthContrast \cite{depthcontrast} &30.91 & 88.80  & 69.51 & 49.87 & 82.67    & 22.70    & 83.36 & 67.38  & 9.32  & 48.69  \\
SegContrast  \cite{segcontrast}     & 34.01  & 89.22 & 78.72 & 57.19    & 82.80    & 21.99 & 83.42      & 67.26   & 14.06 & {\bf 50.91} \\ 
\hline
% \textbf{\Methodname~(ours)} & {\bf 37.71}  & {\bf 91.19} & {\bf 85.75}  & {\bf 66.34}    & {\bf 85.64}    &  {\bf 27.14} & {\bf 85.42}  & {\bf 72.09}   & {\bf 23.63} &  47.74 \\ \hline
\textbf{\Methodname~(ours)} & {\bf 37.71}  & {\bf 91.11} & {\bf 85.34}  & {\bf 66.09}    & {\bf 85.43}    &  {\bf 25.63} & {\bf 84.79}  & {\bf 72.57}   & {\bf 22.61} &  48.67 \\ \hline


\end{tabular}
\caption{\bf{Per-class IoU when fine-tuning with 0.1 $\%$ labels.}}
\label{table:per class IoU}
\vspace{-0.3cm}
\end{table*}

\begin{figure*}[t]
 \centering
\includegraphics[width=1\linewidth]{figure_pdf/segmentation2.pdf}
\caption{{\bf Segmentation results on different frames (rows)}. The models are fine-tuned with 0.1\% labels on KITTI. We compare SegContrast~\cite{segcontrast}, {\Methodname~(ours)} and training from scratch (without pre-training). Our method better distinguishes the different structures shown in the highlighted area (red circle).}
\label{fig:segmentation_result}
\vspace{-0.5cm}
\end{figure*}



\textbf{Cluster-level Inter-frame Self-supervised Leaning.} To encourage points from the same class to be close to each other, we build on the observation that the main source of differences between two objects from the same class in point cloud data is the illumination angles under which they are observed. Thanks to the unsupervised tracking strategy, we can extract pairs of clusters in two distant frames, where the object is then seen under different illumination angles. Given two frames $m$ and $n$, let $N^{mn}$ denote the number of matched clusters across the two frames. Then, we use the cluster-level features to write the loss
\begin{equation}
L_{inter-frame} = \sum_{i=0}^{N^{mn}} \left\| \frac{c^m_{i}}{\| c_{i}^{m} \|_2} - \frac{c^n_{i}}{\| c^n_{i} \|_2} \right\|_2^2\;,
\end{equation}
\label{con:loss2}where $c^m_{i}$ and $c^n_{i}$ are the cluster-level feature vectors of two matched clusters in frame $m$ and $n$. Hence, the total loss can be written as
\begin{equation}\label{con:total_loss}
L_{total} = L_{p2c} + \lambda L_{inter-frame},
\end{equation}
where $\lambda$ is a weight balancing the two loss terms. In practice, the inter-frame information can be better used with the feature of SSL on intra-frame. Thus, we choose a strategy of progressively increasing the $\lambda$.

% inter-frame information needs to build upon the intra-frame initialization. Thus, we set the $\lambda$ to be 0 in the early stage and increasing in the later stage.

% to be more challenging than point-to-cluster SSL, and we therefore set $\lambda$ to 0 in the early training stages and a higer value in the later stages. 

%The most important thing is to pull intra-class instances with different representation together. Since the main difference in the representation of intra-class instances is the illumination angle of LiDAR, we pull features of clusters on the same tracking trajetory in different frames together. We select two frames of $m$ and $n$
%($m>n$ and $(m-n)<10$) 
%and generate the feature of each cluster in two frames so that we get $C_{m} = \left\{C_{1}^{m}, C_{2}^{m},...,C_{N_{m}}^{m}\right\}$ and $C^{n} = \left\{C_{1}^{n}, C_{2}^{n},...,C_{N_{n}}^{n}\right\}$. $N_{m}$ and $N_{n}$ are the number of clusters in two frames, respectively. Some clusters between two frames are successfully matched, and we retain the features of these clusters so that features in frame $m$ and frame $n$ are $C^{m'} = \left\{C_{1}^{m'}, C_{2}^{m'},...,C_{N_{mn}}^{m'}\right\}$ and $C^{n'} = \left\{C_{1}^{n'}, C_{2}^{n'},...,C_{N_{mn}}^{n'}\right\}$ respectively, $N_{mn}$ is the number of clusters successfully matched. Our purpose is to minimize the L2 loss between $C_{i}^{m'}$ and $C_{i}^{n'}, i \in [1, N_{mn}]$. Due to the relative displacement between the lidar and the environment, the representation of a cluster in different frames is collected under different illumination angles. Minimizing the loss between them can make features from the intra-class instances more "compact" in the feature space, which is conducive to downstream tasks. This fundamentally fill the gap that the current data augmentation methods cannot make the features of similar objects under different illumination angles "compact" in the feature space. The loss function is 
%\begin{equation}
%L_{inter-frame} = \sum_{i=0}^{N_{mn}} {\left\Vert \frac{C_{i} ^{m'}}{\Vert C_{i}^{m'} \Vert_2} - \frac{C_{i}^{n'}}{\Vert C_{i} ^{n'} \Vert_2} \right \Vert_2}^2
%\end{equation}\label{con:loss2}

% Inter-frame self-supervised learning is more difficult for models, so we use $L_{p2c}$ in the early stage of training, and use $L_{inter-frame}$ in the late stage of training. 


%-------------------------------------------------------------------------
\vspace{-0.5em}
\section{Experiments}
\vspace{-0.5em}
We first describe our experimental settings, including datasets, unsupervised tracking, and implementation details. Then, we demonstrate the benefits of our self-supervised pre-trained model on downstream tasks, and finally analyze different aspects of our method.
% and pre-training \Methodname \space on KITTI dataset~\cite{kitti} and nuScene dataset\cite{nuscene}  in Sec.~\ref{sec:detail}. Then, we demonstrate the benefits of our self-supervised pre-trained model on downstream tasks, and finally analyze different aspects of our method.


\subsection{Experimental Settings}
\vspace{-0.25em}

\label{sec:detail}
\textbf{Datasets.} We use the KITTI~\cite{kitti} and nuScene~\cite{nuscene} datasets for pre-training, and SemanticKITTI~\cite{semantickitti} and SemanticPOSS~\cite{semanticposs} for the down-stream tasks.

KITTI \cite{kitti} has 21 sequences, and its sampling rate is 10hz. Following~\cite{segcontrast}, we use only the point clouds captured by the Velodyne LiDAR sensor rather than all the information obtained from the position sensors. The sequences 0-10 are used for pre-training, with the exception of sequence 8, which we use as validation data. 
nuScene~\cite{nuscene} is much larger than KITTI. It comprises 1000 scenes and is divided into 10 sequences. The LiDAR data is acquired at 20hz. Because of limited computational resources, we only use the point clouds captured by the Velodyne LiDAR sensor in sequence 1 and 2 (scenes 0 - 149) for pre-training. 
SemanticKITTI~\cite{semantickitti} provides dense point-wise annotations for almost every point in KITTI~\cite{semantickitti}. A total of 23,201 scans are annotated on sequences 0-10 of KITTI for training and validation.
SemanticPOSS \cite{semanticposs} is also used for semantic segmentation, and contains 2988 diverse and complicated LiDAR scans with 14 classes. The scans are divided into 6 splits, with 500 scans per split. Splits 4 and 5 are used for testing and the other ones are used for training. 
% The distribution of the instances in SemanticPOSS is more dynamic than in SemanticKITTI.

{\bf Unsupervised Tracking.} We set the relative weight between spatial distance and feature distance in~\cref{eq:tracking} to $\alpha = 0.5$, and the threshold of RANSAC distance to 0.25 as in \cite{segcontrast}. The threshold of DBSCAN distance is set to 0.25 in KITTI and 0.5 in nuScene. In each frame, we drop clusters with fewer than 200 points or more than 20000 points to filter out noise and retain up to 50 clusters. 


\textbf{Implementation Details.} We compare our approach with DepthContrast~\cite{depthcontrast}, STRL~\cite{STRL}, SegContrast~\cite{segcontrast}, and training from scratch. We use MinkUnet~\cite{minkunet} as backbone for all approaches and build our approach on the basis of BYOL\cite{BYOL}. 
We pre-train the backbone on KITTI and nuScene for 200 epochs using an SGD optimizer with a momentum of 0.9, and set the weight decay to 0.0004 following SegContrast~\cite{segcontrast}. The learning rate is initially set to 0.036 with a linear annealing scheme with a minimum learning rate equal to 0.009. In the early training stages, we set $\lambda$ to 0, and to 4 in the later stages. When incorporating the inter-frame loss term, we re-initialize different MLPs after the backbone networks to avoid information leaks from spatial. The batch size is set to 8 for each GPU, and we use $8 \times$GTX3090 GPUs to pre-train the models, which leads to the total batch size of 64.

For fine-tuning on the down-stream semantic segmentation task, we use an SGD optimizer with a cosine learning rate schedule. The fine-tuned models are evaluated on the validation sequences, i.e., sequence 8 for SemanticKITTI and sequences 4 and 5 for SemanticPOSS~\cite{segcontrast}. The batch size is set to 2 for each GPU, and $4 \times$GTX2080Ti GPUs are used for the experiments.

\textbf{Evaluation Metrics.} We evaluate point cloud semantic segmentation using the mean intersection over
union (mIoU) and the overall point classification accuracy (Acc).

% \begin{table}[]
% \centering
% \begin{tabular}{ccccc}
% \hline
%                           & 0.1\% & 1\%    & 10\%         & 100\% \\ \hline
% From scratch                & 29.17 & 48.11  & 51.00        & 52.76 \\
% % STRL w \cite{STRL}                   & 17.04 & 34.08      & 30.17   & 45.68 \\
% STRL\cite{STRL}                  & 16.64 & 31.88      & 30.88   & 46.23 \\
% DepthContrast \cite{depthcontrast}              & 30.91 & 42.41  & 42.38        & 43.22 \\
% SegContrast \cite{segcontrast}                & 34.01 & 48.02  & 52.26        & 53.58 \\ \hline
% \Methodname(Ours)                       & {\bf 37.19}    & {\bf52.07}   & {\bf53.48}       & {\bf54.93} \\ \hline
% \end{tabular}
% \caption{Pre-trained on KITTI and evaluate the fine-tuning model at different label regimes on SemanticKITTI for semantic segmentation (mIoU). }
% \label{table:finetune at kitti segmentation}
% \end{table}

\begin{table}[h]

\centering
\begin{tabular}{ccccc}
\hline
                          & 0.1\% & 1\%    & 10\%         & 100\% \\ \hline
From Scratch                & 29.17 & 48.11  & 51.00        & 56.14 \\
% STRL w \cite{STRL}                   & 17.04 & 34.08      & 30.17   & 45.68 \\
STRL\cite{STRL}                  & 16.64 & 31.88      & 30.88   & 55.71 \\
DepthContrast \cite{depthcontrast}              & 30.91 & 42.41  & 42.38        & 45.48 \\
SegContrast \cite{segcontrast}                & 34.01 & 48.02  & 52.26        & 55.45 \\ \hline
\textbf{\Methodname~(ours) }                      & {\bf 37.71}    & {\bf52.60}   & {\bf54.51}       & {\bf57.33} \\ \hline
\end{tabular}
\vspace{-0.2cm}

\caption{Pre-training on {\bf KITTI} and evaluating the fine-tuned models in different label regimes on \textbf{SemanticKITTI} for semantic segmentation. We report the mIoU.}
\label{table:finetune_at_kitti_segmentation}
\vspace{-0.2cm}

\end{table}



\begin{table}[h]
\centering
\tabcolsep=0.35cm
\begin{tabular}{ccc}
\hline
mIoU / Acc               & seq1              & seq 1-2 \\ \hline
From Scratch       & 29.17 / 82.57                     & 29.17 / 82.57 \\
% STRL w \cite{STRL}                     & 18.82 / 70.70         & 18.72 / 70.79    \\
STRL\cite{STRL}                    & 19.11 / 74.56         & 18.74 / 70.85    \\
SegContrast \cite{segcontrast}               & 33.91 / 84.88         & 34.28 / 85.20    \\ \hline
\textbf{\Methodname~(ours) }         & {\bf 34.43 / 85.34 }        & {\bf 35.08 / 85.75} \\ \hline
\end{tabular}
\vspace{-0.2cm}

\caption{Pre-training on \textbf{nuScene} and evaluating the fine-tuned models in the 0.1\% label regime on \textbf{SemanticKITTI} for semantic segmentation. We report mIoU/Acc.}
\label{table:nuscene_finetune_at_kitti_segmentation}
\vspace{-0.2cm}
\end{table}

% \begin{table}[]
% \centering
% \tabcolsep=0.45cm
% \begin{tabular}{ccc}

% \hline pre-train dataset & KITTI      & nuScene(seq 1)  \\ \hline
% From Scratch                & 39.64      & 39.64 \\
% % STRL w                     & 38.43      & 35.91 / 38.99\\ 
% STRL\cite{STRL}                     & 38.43      & 36.63 \\ 
% SegContrast \cite{segcontrast}      & 43.88      & 42.86  \\ \hline
% \textbf{\Methodname~(ours) }     &  {\bf 44.27} & {\bf 43.11}  \\ \hline
% \end{tabular}
% \caption{Pre-training on \textbf{KITTI and nuScene}, and evaluating the fine-tuned models on \textbf{SemanticPOSS} for semantic segmentation. We report the mIoU.}
% \label{table:kittt and nuscene finetune at poss segmentation}
% \end{table}

\begin{table}[h]
\centering
\tabcolsep=0.3cm

\begin{tabular}{ccc}

\hline pre-train dataset & KITTI      & nuScene(seq 1)  \\ \hline
From Scratch                & 39.64 / 88.66      & 39.64 / 88.66 \\
% STRL w                     & 38.43      & 35.91 / 38.99\\ 
STRL\cite{STRL}                     & 38.43 / 88.32      & 36.63 / 87.53 \\ 
SegContrast \cite{segcontrast}      & {\bf 43.88 /  89.64}      & 42.86 / 89.28 \\ \hline
\textbf{\Methodname~(ours) }     &  43.84 / 89.47 & {\bf 43.55 / 89.38}  \\ \hline
\end{tabular}
\vspace{-0.2cm}

\caption{Pre-training on \textbf{KITTI and nuScene}, and evaluating the fine-tuned models on \textbf{SemanticPOSS} for semantic segmentation. We report the mIoU/Acc.}
\label{table:kittt and nuscene finetune at poss segmentation}
\vspace{-0.2cm}

\end{table}




% \begin{table}[]
% \centering
% \begin{tabular}{ccc}
% \hline pre-train dataset & KITTI      & nuScene  \\ \hline
% From Scratch                & 39.64      & 39.64 \\
% % STRL w                     & 38.43      & 35.91 / 38.99\\ 
% STRL\cite{STRL}                     & 38.43      & 36.63 \\ 
% SegContrast \cite{segcontrast}      & 43.88      & 42.86  \\ \hline
% \Methodname~(ours)                &  {\bf 44.27} & {\bf 43.11}  \\ \hline
% \end{tabular}
% \caption{Pre-training on KITTI and nuScene, and evaluating the fine-tuning model at SemanticPOSS for semantic segmentation (mIoU).}
% \label{table:kittt and nuscene finetune at poss segmentation}
% \vspace{-0.1cm}
% \end{table}

\begin{table}[]
\centering
\tabcolsep=0.3cm
\begin{tabular}{ccc}
\hline
mIoU / Acc                           & 0.1\%  & 1\%    \\ \hline
From Scratch                & 29.17 / 82.57 & 48.11 / 89.94\\
% STRL w \cite{STRL}                   & 17.04 / 69.30 & 34.23 / 85.06     \\
STRL\cite{STRL}                  & 16.64 / 69.36 & 31.88 / 85.35      \\
DepthContrast \cite{depthcontrast}              & 30.91 / 82.82 & 42.41 / 88.95  \\
SegContrast \cite{segcontrast}                & 34.01 / 84.72 & 48.02 / 88.84  \\ \hline
P2C     & {\bf 35.48 / 86.18}& {\bf 50.83 / 90.14} \\ \hline
\end{tabular}
\vspace{-0.2cm}

\caption{Ablation study on the \textbf{pre-training strategy} with 0.1\% and 1\% labels. We report mIoU/Acc.}
\label{table: ablation on frameworks}
\vspace{-0.6cm}
\end{table}

\vspace{-0.25em}
\subsection{Outdoor Scene Understanding}
\vspace{-0.25em}

\label{sec:Outdoor_scene}

% For STRL, it takes too much time to upsample points in outdoors scenes so we do not use it. We tested whether to use normalization in data augmentation, which is recorded as STRL w and STRL o respectively. 



{\bf  Label Efficiency.} To assess the label efficiency of our \Methodname~ approach, we fine-tune the model pre-trained on KITTI on SemanticKITTI. Following~\cite{segcontrast}, SemanticKITTI is divided into different regimes corresponding to different percentages of labels. Specifically, we use 0.1\%, 1\%, 10\%, and 100\% of the training data to fine-tune the pre-trained model for semantic segmentation.
% and using them to fine-tune the pre-trained model into the semantic segmentation task since it is a large dataset.

In Table \ref{table:per class IoU}, we compare the mIoU and per-class IoU of the proposed \Methodname~ and of the state-of-the-art approaches when using 0.1\% of the labels. Our method outperforms the baselines by an impressive margin, yielding an mIoU of 37.71\%, which is 3.7\% better than SegContrast\cite{segcontrast} and 8.54\% than training from scratch. Fig.~\ref{fig:segmentation_result} evidences that our method yields more complete and accurate segmentation masks than the other methods.


% Fig.\ref{fig:segmentation_result} shows the segmentation results on different frames. It can be observed that our method can better distinguish the different structures.

The per-class comparison shows that our approach greatly improves the network's performance on most classes when there are few annotations. Our {\Methodname} yields much better results than the baselines, especially for car, building, vegetation, terrain, and parking.
%our approach outperforms SegContrast \cite{segcontrast} and all other methods with a large margin, because 
We attribute this to the fact that these classes have clearly different appearances under different illumination angles as shown in Fig.\ref{fig:angle_different_1}, which is exactly the problem that our inter-frame self-supervised learning addresses.

% displays mIoU and per-class IoU at 0.1\% label regime for all approaches. Due to space limitation, only part of per-class results are placed here, and the complete results can be found in the supply materials. 
% It is illustrated that our {\Methodname} performs much better than other approaches in most classes. 


% which shows that our method works for most classes.
The results obtained by fine-tuning with different percentages of training data are provided in Table~\ref{table:finetune_at_kitti_segmentation}. Our method consistently outperforms the state-of-the-art self-supervised approach for all label regimes. Specifically, the mIoU of our approach outperforms the SegContrast and From Scratch ones by 2.25\% and 3.51\%, with 10\% labels.

% by 3.7\% and 8.54\%, 4.49\% and 4.58\%, 3.51\% and 2.25\%, 1.19\% and 1.88\%, with 0.1\%, 1\%, 10\%, and 100\% labels, respectively.
% which shows our approach can greatly help the network to improve performance when there are few annotations.
% The results show that the features learned by {\Methodname} can better fit downstream tasks, greatly improving the effectiveness of data. 
  
% \begin{table}[]
% \caption{Pre-trained on KITTI and evaluate the fine-tuning model at different label regimes on SemanticKITTI for semantic segmentation (mIoU). }
% \centering
% \begin{tabular}{ccccc}
% \hline
%                           & 0.1\% & 1\%    & 10\%         & 100\% \\ \hline
% From Scrath                & 29.17 & 48.11  & 51.00        & 56.56 \\
% STRL w \cite{STRL}                   & 17.04 & 34.08      & 30.17   & 53.32 \\
% STRL o \cite{STRL}                  & 16.64 & 31.88      & 30.88   & 53.22 \\
% DepthContrast \cite{depthcontrast}              & 30.91 & 42.41  & 42.38        & 47.92 \\
% SegContrast \cite{segcontrast}                & 34.00 & 48.02  & 52.26        & 53.58 \\ \hline
% \Methodname(Ours)                       & {\bf 37.19}    & {\bf52.07}   & {\bf53.48}       & {\bf58.02} \\ \hline
% \label{table:finetune at kitti segmentation}
% \end{tabular}
% \end{table}

\begin{figure*}[t]
 \centering
\includegraphics[width=1\linewidth]{figure_pdf/P2S_big.pdf}
\vspace{-0.6cm}
\caption{\textbf{Comparison of the features generated by different pre-trained models}. (a) Results with P2C. (b) Zoomed in car from (a). (c) Zoomed in car from (d). (d) Results with SegContrast\cite{segcontrast}. The points with the same color are in the same cluster(clustering in feature space). The colors of (a, b) and (c, d) are independent and have no relationship with each other. For better visualization, we only colorize the car in (b) and (c). }
\label{fig:p2s_feature_map}
\vspace{-0.3cm}
\end{figure*}

\begin{table*}[t]
\centering
\begin{tabular}{ccccccccccc}
\hline
Method name       & mIoU  & car   & road   & sidewalk & building & fence & vegetation & terrain & parking & pole  \\ \hline
SegContrast  \cite{segcontrast}     & 34.01  & 89.22 & 78.72 & 57.19    & 82.80    & 21.99 & 83.42      & 67.26   & 14.06 & 50.91 \\ 
SegContrast-BYOL & 33.94  & 89.34 & 78.56 & 57.29    & 82.86 & 21.19 & 83.26      & 67.30   & 14.09 & 50.54 \\ 
SegContrast-Inter &  36.03  &  90.52 &  81.45  &  62.90  &  83.54    &  21.46 &  84.37    & 72.52   & 17.05 &  {\bf 51.78} \\ 
P2C  & 35.48  & 90.18 & 80.90 & 61.91    & 83.92    & 25.45 & 84.30  & 71.35  & 18.73 & 46.95 \\\hline
\textbf{\Methodname~(ours)} & {\bf 37.71}  & {\bf 91.11} & {\bf 85.34}  & {\bf 66.09}    & {\bf 85.43}    &  {\bf 25.63} & {\bf 84.79}  & {\bf 72.57}   & {\bf 22.61} &  48.67 \\ \hline
\end{tabular}
\vspace{-0.2cm}
\caption{\textbf{Per-class IoU when fine-tuning with 0.1 $\%$ labels}. SegContrast-BYOL: SegContrast built on the basis of BYOL. SegContrast-Inter: SegContrast with a additional inter-frame self-supervised learning stage after the original SegContrast. P2C: Our approach with only the point-to-cluster loss function.}
\label{table:per_class_IoU2}
\vspace{-0.6cm}
\end{table*}

{\bf Feature Representation Transferability.} To confirm the transferability of the features learned by our approach, 
%we employ the model pre-trained on nuScene and fine-tuned on either SemanticKITTI~\cite{semantickitti} or SemanticPOSS~\cite{semanticposs}.
we pre-train our models on nuScene and design two settings for pre-training: i) only using sequence 1 (seq1), and ii) using sequence 1 and 2 (seq 1-2) with uniform down-sampling to keep the number of frames consistent with seq1 and the frame rate consistent with KITTI.

As shown in Table~\ref{table:nuscene_finetune_at_kitti_segmentation}, our method outperforms training from scratch and Segcontrast~\cite{segcontrast} when using only seq 1 from nuScene for pre-training and fine-tuning on SemanticKITTI~\cite{semantickitti}. Our approach improves the segmentation performance by 5.26\% in mIoU and 2.77\% in Acc. When seq 1 and 2 are used for pre-training, our approach improves the segmentation performance by 5.91\% in mIoU and 3.18\% in Acc. We also fine-tune the models pre-trained on nuScene or KITTI to the semantic segmentation task using SemanticPOSS~\cite{semanticposs}. 
% SemanticPOSS uses the same data format as SemanticKITTI with much smaller size and is collected in Peking University. 
As SemanticPOSS is small, we fine-tune on the entire dataset. Table~\ref{table:kittt and nuscene finetune at poss segmentation} shows that our method yields better mIoU results than training from scratch. When the network is pre-trained on KITTI, our approach improves the mIoU by 4.20\% compared to the network without pre-training.
% when seq 1 and 2 are used for pre-training, the improvements in mIou and ACC are 5.69\% and 3.17\%, respectively.


% \begin{table}[t]
% \centering

% \caption{Pre-trained on nuScene and evaluate the fine-tuning model at different label regimes on SemanticKITTI for semantic segmentation (mIoU)}
% \begin{tabular}{ccc}
% \hline
% seq1 / seq 1-2               & 0.1\%              & 1\% \\ \hline
% From Scrath       & 29.17              & 48.11 \\
% STRL w \cite{STRL}                     & 18.82 / 18.72        & 32.45 / 34.50    \\
% STRL o \cite{STRL}                    & 19.11 / 18.90        & 32.65 / 31.35    \\
% SegContrast \cite{segcontrast}               & 33.91 / 34.28         & 50.00 / {\bf 51.56}    \\ \hline
% \Methodname(Ours)          & {\bf 34.10 / 34.86}        & {\bf 50.03} / 49.47 \\ \hline

% \label{table:nuscene finetune at kitti segmentation}
% \end{tabular}
% \end{table}







% \begin{figure*}[t]
%  \centering
 
%   \begin{subfigure}{0.24\linewidth}
%     \includegraphics[width=1\linewidth]{figure/P2S_a.jpg}
%     \caption{}
%     \label{fig:P2S_left}
%   \end{subfigure}
%   \begin{subfigure}{0.24\linewidth}
%     \includegraphics[width=1\linewidth]{figure/P2S_b.jpg}
%     \caption{}
%     \label{fig:P2S_right}
%   \end{subfigure}
%   \begin{subfigure}{0.24\linewidth}
%     \includegraphics[width=1\linewidth]{figure/P2S_c.jpg}
%     \caption{}
%     \label{fig:P2S_right}
%   \end{subfigure}
%   \begin{subfigure}{0.24\linewidth}
%     \includegraphics[width=1\linewidth]{figure/P2S_d.jpg}
%     \caption{}
%     \label{fig:P2S_right}
%   \end{subfigure}
% \caption{{\bf Compare the results of pre training in different frameworks}. (a):Use the model pre-trained with SegContrast. (b): A cyclist in partial enlarged view (a). (c): A cyclist in partial enlarged view (d). (d):Use the model pre-trained with \Methodname. The same color indicates that they are in the same cluster(clustering in feature space). The colors of (a, b) and (c, d) are independent and have no relationship with each other. For better visualization, we remove the ground line in  (c) and  (d). }
% \end{figure*}
% \label{fig:p2s_feature_map}



\begin{figure*}[t]
  \centering
%   \fbox{\rule{0pt}{2in} \rule{0.9\linewidth}{0pt}}
  \includegraphics[width=1\linewidth]{figure_pdf/k_means_features_big.pdf}
  \caption{\textbf{Comparison of models pre-trained with different intervals between two frames}. Left: SegContrast. Middle: (\Methodname, interval=0). Right: (\Methodname, interval=4). The pre-trained models are used to extract point features for visualization, and the points with the same color are adjacent in feature space. For better visualization, we only colorize the points belonging to the specified clusters. }
  \vspace*{-0.5cm}
  \label{fig:track_feature_show}
\end{figure*}

%-------------------------------------------------------------------------
\vspace{-0.25em}
\subsection{Analysis}
\vspace{-0.25em}
\label{sec:analysis}
\textbf{Guaranteed over-segmentation assumptions.}
 P2C SSL strategy relies on the over-segmentation assumptions and 
the hyper-parameter leading to the over-segmented clusters is easy to set. To show this, we performed the following experiment. We varied the DBSCAN distance threshold from 0.15 to 0.45 and measured the proportion of clusters having at least 90 $\%$ of their points from the same semantic class. The higher this proportion, the higher the chance that the clusters are over-segmented and not under-segmented. In the worst case, we found 73.45$\%$ of the clusters being over-segmented witch evidences the stability of the hyper-parameter of DBSCAN. 

\textbf{Performance of unsupervised tracking.}
We measured that 63.73$\%$ of the clusters are tracked for at least 3 frames, and 31.33$\%$ for at least 8 frames. Such tracking times are sufficient for us to create positive pairs of clusters observed under different illumination angles as shown in Fig.~\ref{fig:track_show}.

\begin{figure}[t]
  \centering
%   \fbox{\rule{0pt}{2in} \rule{0.9\linewidth}{0pt}}
  \includegraphics[width=1\linewidth]{figure_pdf/Track_show2.pdf}
    \vspace*{-1em}
  \caption{{\bf Associated clusters in multiple frames}. Some clusters with clear semantic information are selected for display. The same color represents the same cluster. Red: car; Blue: motorcycle; Orange: person.}
  \vspace*{-1em}
  \label{fig:track_show}
\end{figure}


\vspace*{-0.5em}
\subsection{Ablation Study}
  \vspace*{-0.5em}

\label{sec:ablation}
%In this section, we conduct an in-depth analysis of \Methodname. 

\begin{figure}[h]
  \centering
%   \fbox{\rule{0pt}{2in} \rule{0.9\linewidth}{0pt}}
  \includegraphics[width=1\linewidth]{figure_pdf/IoU.pdf}
  \vspace{-0.8cm}
  \caption{Comparison mIoU on SemanticKITTI dataset, which is fine-tuned by the models pre-trained with \textbf{different intervals} between two frames.}
  \vspace{-0.3cm}
  \label{fig:different_frames}
\end{figure}


Experiment in this section, if it is not specially stated, the model pre-training on KITTI, and fine-tuning and evaluating on SemanticKITTI.

% {\bf Comparison to Scene-/Cluster-level pre-training.}
\textbf{Scene- vs. Cluster-level Pre-training.}
To demonstrate the effectiveness of the Point-to-Cluster learning strategy proposed in Section~\ref{sec:spatiotemporal}, we conduct an experiment with only the point-to-cluster loss function of
~\cref{con:loss1} for pre-training. We dub this setting P2C. As shown in Table~\ref{table: ablation on frameworks} and Table~\ref{table:per_class_IoU2}, a cluster-level pre-trained SegContrast performs better than scene-level pre-trained DepthContrast and STRL. Our proposed P2C pre-training method achieves better results than SegContrast. To better understand the advantages of P2C, we have designed the following visualization. We select a point cloud frame, use the pre-trained model (i.e., SegContrast and P2C) to extract features for each point, and use K-Means~\cite{Kmeans} to cluster these features. Specifically, we use 20 clusters, which corresponds to the number of categories in the annotations. 
% (note that features are clustered in the feature space, which is different from Section \ref{sec:Method}). 
In Fig.~\ref{fig:p2s_feature_map}, we visualize the points by coloring them according to the clusters. 
% and so points with same colors are adjacent in the feature space. 
With features extracted by the SegContrast pre-trained model, the car zoomed in in Fig.~\ref{fig:p2s_feature_map} (c) is divided into several colors. This indicates that the points from the same category can be distant in the learned feature space.
%It means that the features of points in an identity cluster are not adjacent in the feature space. 
By contrast, encouraging the points in the same cluster to have similar features, 
%which plays an important role in point cloud data. As shown in
P2C yields feature such that most of the car points are in the same cluster, as shown in Fig.~\ref{fig:p2s_feature_map} (b), thus indicating that the features are more representative of the object instances.

%\WYH{almost} only one color is rendered, which means the features of each point are adjacent and stable in the feature space.

% \begin{center}
% \begin{table}[H]
% \caption{Ablation on frameworks.SegContrast-BYOL: Build SegContrast on BYOL}
% \centering
% \begin{tabular}{ccccc}
% \hline
% mIoU                 & 0.1\%      & 1\%     & 10\%  & 100\%  \\ \hline From Scrath                & 29.17 & 48.11  & 51.00        & 52.76 \\
% SegContrast                & 34.01 & 48.02  & 52.26        & 53.58 \\ 
% SegContrast-BYOL    & 33.94    & 49.64     & ?    &? \\ \hline
% \Methodname(Ours)                       & {\bf 37.19}    & {\bf52.07}   & {\bf53.48}       & {\bf54.93} \\ \hline
% \end{tabular}
% \end{table}
% \end{center}
% \label{table: ablation on framework}


% \begin{center}
% \begin{table}[H]
% \caption{Ablation on frameworks.SegContrast-BYOL: Build SegContrast on BYOL}
% \centering
% \begin{tabular}{cccc}
% \hline
% Method                 & SegContrast      & Seg-BYOL     & Ours  \\ \hline 
% mIoU/Acc                  & 34.01 / 84.72  & 33.94 / 84.67  & {\bf 37.19 / 87.48}         \\ \hline 
% \end{tabular}
% \end{table}
% \end{center}
% \label{table: ablation on framework}


\vspace{-0.2em}

% {\bf Effectiveness of inter-frame contrastive learning.}
\textbf{Effectiveness of Inter-frame Self-supervised Learning.}
To better demonstrate the role of inter-frame self-supervised learning, we add a new stage after the original SegContrast, in which we activate our inter-frame loss function, as in our method. Note that we maintain the total number of training epochs to 200.
% into two stages. The first stage is the same as the original SegContrast, and the second stage is inter-frame self-supervised learning like our \Methodname, 
We dub the resulting model SegContrast-inter, and show its per-class IoU in Table~\ref{table:per_class_IoU2}. For most classes, SegContrast-inter performs much better than SegContrast. However, SegContrast performs better in the fence class. Fence has almost the same appearance  when they are viewed from different angles (the z axis remains unchanged). This experiment also supports our motivation that the illumination angle is an important factor, making the appearance of an object differ. 
%It is very insightful to design positive pairs from different frames with different illumination angles.


% {\bf Explore the effects caused by the different distances between two frames.} 
\textbf{Interval between Two Frames.} 
Since we choose positive pairs from different frames, the interval between the two selected frames is a hyper-parameter affecting the results. In our approach, we gradually increase the interval between two frames to avoid having to set such a hyper-parameter. Here we evaluate the impact of the interval on a model using a fixed interval between two frames. The experiments are repeated for 3 times to reduce randomness, and we report the average. As shown in Fig.~\ref{fig:different_frames}, 
with the increase of interval, the performance first increases and then decreases. We think it is because greater interval can reduce the impact of more illumination angles, but the number of clusters that can be tracked across frames decreases as the interval increases.  The best-performing model corresponds to an interval of 5, reaching an mIoU of 36.05\%  when it reaches the balance of illumination angles change and number of tracking clusters. Note that the worst performance (interval of 9) is still better than that of SegContrast. 
% We hope to use objects from different angles as positive samples as much as possible, the distance is set 4,
% Therefore, in order to balance the two requirements, we choose 4, 
% which is half of the average tracking length.
% , as the distance between the two sampled frames. 
% But in all distance settings, our method is better than SegContrast.
 
In Fig.~\ref{fig:track_feature_show}, we visualize a frame with multiple cars under different LiDAR illumination angles.
% Further more, we have selected a special frame, in which there are multiple cars under different Lidar irradiation angles.
We also render the points in the same cluster with the same color. 
% use the previously mentioned method to visualize the frame, as shown in Fig.~\ref{fig:track_feature_show}.
% We have used four pre-trained models to extract features.
Note that SegContrast and {\Methodname} with interval=0 cannot extract similar features for the cars under different illumination angles. By contrast, {\Methodname} with interval=4 yields similar features for all cars, benefiting from inter-frame matching.


\begin{table}[]
\centering
\tabcolsep=0.35cm

\begin{tabular}{ccc}
\hline
mIoU / Acc              & 0.1 \%      & 1 \% \\ \hline 
From Scratch         & 29.17 / 82.57     & 48.11 / 89.94 \\
SegContrast \cite{segcontrast}                & 34.01 / 84.72 & 48.02 / 88.84  \\
{\textbf{\Methodname-n} }    & 36.01 / 87.44     & 51.06 / 89.68 \\  \hline
{\textbf{\Methodname}}       & {\bf 37.71 / 87.67} & {\bf 52.60 / 90.24}  \\ \hline
\end{tabular}
\caption{\textbf{Ablation study on feature similarity for tracking}. {\Methodname-n} only considers location similarity in tracking stage.}
\label{table: ablation on match_byol}
\vspace{-2 em}

\end{table}



% {\bf Is it beneficial to use model to assist tracking.}
\textbf{Tracking with vs. without a Model.}
In ~\cref{eq:tracking}, we use both location and feature similarities to compute the matching degree matrix for tracking. To illustrate the effectiveness of the feature similarity, we re-conduct the tracking with only location similarity. This is indicated by {\Methodname-n} in Table \ref{table: ablation on match_byol}.
% the model to assist the unsupervised tracking process. Now we do not use the model for assistance, and reconduct the experiment, which is named as {\Methodname-n}.  The results are shown in the Table \ref{table: ablation on match_byol}. 
Note that the resulting method still outperforms SegContrast but not our complete \Methodname. 

\begin{table}[t]
\centering
\tabcolsep=0.46cm

\begin{tabular}{ccc}
\hline
Method           & mIoU (\%)       & Acc (\%)   \\ \hline
SegContrast\cite{segcontrast}      & 34.01          & 84.72          \\
SegContrast-BYOL & 33.94          & 84.67          \\ \hline
\textbf{\Methodname~(Ours)} & {\bf 37.71}    & {\bf 87.67} \\ \hline
\end{tabular}
\caption{\textbf{Ablation study on the framework}. SegContrast-BYOL: SegContrast built on the basis of BYOL.}
\label{table: ablation on framework}
\vspace{-0.6cm}

\end{table}

% {\bf Whether the performance improvement compared with SegContrast comes from BYOL.} 
\textbf{Effect of BYOL.}
To evidence that the benefits of our approach over SegContrast are not only due to our use of BYOL instead of MoCo but truly to our training formalism, we replace the MoCo in SegContrast with BYOL. As shown in Table~\ref{table: ablation on framework}, SegContrast-BYOL performs on par with the original SegContrast (with MoCo) in the 0.1\% label regime. It implies that the networks are not the key factor in point clouds SSL. Due to the over-segmentation, we used in our method, where each cluster could belong to the same class, building negative pairs for each point in our point-to-cluster strategy will harm the optimization. By contrast, it is more intuitive for us to only build positive pairs to avoid pushing the nearby clusters away, which offers a chance to merge the cluster with the help of temporal information.

% could be from the same class in a single frame, 

% Our {\Methodname} employ the BYOL because there are multiple objects belonging to the same class in one frame. And we don't want to simply classify the objects not matched as negative samples, so we use BYOL that takes only positive samples into consideration.


\vspace{-0.5em}
\section{Conclusion and Limitation}
\vspace{-0.5em}

% In this paper, we have introduced an SSL strategy for point cloud segmentation based only on positive pairs. It contains a novel Point-to-Cluster(P2C) training paradigm and introduced to use inter-frame self-supervised learning. P2C combines the advantages of point-level and cluster-level representations and inter-frame self-supervised learning essentially fill the gap that data augmentation cannot improve the overlap between intra-class instances in point cloud data. The experiments suggest that our approach outperforms the state-of-the-art SSL techniques for point cloud data.

In this paper, we have introduced an SSL strategy for point cloud segmentation without external supervision. It relies on a novel Point-to-Cluster (P2C) training paradigm to exploit spatial information, and further introduces an inter-frame self-supervised learning strategy to capture temporal information. Altogether, our approach provides a practical tool for pre-training with point clouds in the wild. Experiments evidence that our approach outperforms the state-of-the-art SSL techniques for point cloud in the wild. However, such an improvement does not materialize for objects whose appearance is invariant to viewpoint changes, such as fences. In the future, we will therefore focus on how to improve the segmentation ability of such objects.

\small{\textbf{Acknowledgement.} This work was supported in part by the National Natural Science Foundation of China under Grant No. 62006182 and the Swiss National Science Foundation via the Sinergia grant CRSII5-180359.}

%%%%%%%%% REFERENCES
{\small
\bibliographystyle{ieee_fullname}
\bibliography{egbib}
}

\end{document}
