\section{Related Work}
\label{sec:litre}
Diffusion Probabilistic Models, which iteratively improve a given image to obtain image quality that is on par with or better than other types of generative models, have achieved state-of-the-art results in image generation and synthesis tasks~\cite{song2020score,batzolis2021conditional,dhariwal2021diffusion,singh2022high,rombach2022high}.
%
DDPM~\cite{ho2020denoising} is a generative model based on a Markov chain, which models the data distribution by simulating a diffusion process that evolves the input data towards a target distribution.
%
LDMs~\cite{esser2021taming,rombach2022high} proposed to conduct the diffusion process on a compressed latent space of lower dimensionality with cheaper computational cost, faster inference and competitive synthesis quality.
%
While Diffusion models have achieved great success in low-level vision, their potential for high-level vision has yet to be fully explored. A few pioneer works tried to adopt the diffusion model for image segmentation and object detection tasks~\cite{amit2021segdiff,wolleb2022diffusion,gu2022diffusioninst,chen2022diffusiondet}. 
%
For example, Chen \etal~\cite{chen2022diffusiondet} applied a diffusion model for object detection by denoising the random box. CARD~\cite{han2022card} adopted the diffusion process as uncertainty estimation to improve the regression performance. 
%SegDiff~\cite{amit2021segdiff} leveraged a diffusion model to calibrate the segmentation map from both the current estimate and the input image. 
%
However and despite increasing attention to perception tasks, there are no previous solutions that adopt diffusion models for medical image classification. 
%While most segmentation methods are still processed in an image-to-image fashion, image classification, a prediction problem, apparently works in a more demanding way.

