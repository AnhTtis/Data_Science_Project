\section{Introduction}
\label{sec:intro}

%The implications of digital medical imaging in modern healthcare have led to the indispensable role of medical image analysis in clinical therapy~\cite{de2016machine}. Medical image classification, which is a fundamental step in medical image analysis, aims to distinguish medical images according to a certain criterion, such as clinical pathologies or imaging modalities. A reliable medical image classification system can assist doctors in the fast and accurate interpretation of medical images. A large number of solutions for medical image classification have been developed over the past decades in the literature, most of which are based on deep neural networks ranging from popular convolutional neural networks to vision transformers~\cite{Esteva2017,Esteva2019,Rajpurkar2022,Shamshad2022}.  These methods have the potential to reduce the time and effort required for manual classification and improve the consistency and accuracy of results. However, medical images with diverse modalities still challenge existing methods due to the presence of various ambiguous lesions and fine-grained tissues, such as ultrasound (US), dermatoscopic, and fundus images. Moreover, generating medical images under hardware limitations can cause noisy and blurry effects, which can degrade image quality and thus demand a more effective feature representation modeling for robust classifications.
Medical image analysis plays an indispensable role in clinical therapy because of the implications of digital medical imaging in modern healthcare~\cite{de2016machine}. Medical image classification, a fundamental step in the analysis of medical images, strives to distinguish medical images from different modalities based on certain criteria. An automatic and reliable classification system can help doctors interpret medical images quickly and accurately.
Massive solutions for medical image classification have been developed over the past decades in the literature, most of which adopt deep learning ranging from popular CNNs to vision transformers~\cite{Esteva2017,Esteva2019,Rajpurkar2022,Shamshad2022}.  These methods have the potential to reduce the time and effort required for manual classification and improve the accuracy and consistency of results. However, medical images with diverse modalities still challenge existing methods due to the presence of various ambiguous lesions and fine-grained tissues, such as ultrasound (US), dermatoscopic, and fundus images. Moreover, generating medical images under hardware limitations can cause noisy and blurry effects, which can degrade image quality and thus demand a more effective feature representation modeling for robust classifications.

%
Recently, Denoising Diffusion Probabilistic Models (DDPM)~\cite{ho2020denoising} have achieved excellent results in image generation and synthesis tasks~\cite{pmlr-v139-nichol21a,batzolis2021conditional,dhariwal2021diffusion,singh2022high} by iteratively improving the quality of a given image.
%
Specifically, DDPM is a generative model based on a Markov chain, which models the data distribution by simulating a diffusion process that evolves the input data towards a target distribution.
%
%
Although a few pioneer works tried to adopt the diffusion model for image segmentation and object detection tasks~\cite{amit2021segdiff,wolleb2022diffusion,chen2022diffusiondet,han2022card}, their potential for high-level vision has yet to be fully explored.
%


Motivated by the achievements of diffusion probabilistic models in generative image modeling, \textbf{1) we present a novel Denoising Diffusion-based model named DiffMIC} for accurate classification of diverse medical image modalities. As far as we know, we are the first to propose a Diffusion-based model for general medical image classification. Our method can appropriately eliminate undesirable noise in medical images as the diffusion process is stochastic in nature for each sampling step.
%
\textbf{2) In particular, we introduce a Dual-granularity Conditional Guidance (DCG) strategy} to guide the denoising procedure, conditioning each step with both global and local priors in the diffusion process. By conducting the diffusion process on smaller patches, our method can distinguish critical tissues with fine-grained capability. 
\textbf{3) Moreover, we introduce Condition-specific Maximum-Mean Discrepancy (MMD) regularization} to learn the mutual information in the latent space for each granularity, enabling the network to model a robust feature representation shared by the whole image and patches.
%
\textbf{4) We evaluate the effectiveness of DiffMIC on three 2D medical image classification tasks} including placental maturity grading, skin lesion classification, and diabetic retinopathy grading. The experimental results demonstrate that our diffusion-based classification method consistently and significantly surpasses state-of-the-art methods for all three tasks. 
%Our DiffMIC provides a promising solution for accurate and robust classification of diverse medical image modalities.
% In summary, the technical contributions of our work are four-fold:
% \begin{itemize}
%     \item We develop a DDPM-based model for general medical image classification. 
%     \item We design a novel guidance strategy to condition each step by dual-granularity priors.
%     \item We introduce maximum-mean discrepancy regularization for each conditional prior to learn mutual information in the iterative sampling process.
%     \item Experimental results on three medical image classification benchmark datasets have shown that our DiffMIC clearly outperforms state-of-the-art medical image classification methods.
% \end{itemize}   