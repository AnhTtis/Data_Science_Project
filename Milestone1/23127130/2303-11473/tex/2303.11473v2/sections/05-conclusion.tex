\section{Conclusion}
\label{sec:conclusion}
%\vspace{-.2cm}

In this work, we extend the sandwich architecture to video compression by carefully designing a video codec proxy and training it with neural pre- and post-processors in an end-to-end fashion. The proposed sandwich model outperforms the standard codec HEVC under various settings, including YUV 4:0:0 and YUV 4:4:4 LR formats; and under $\ell_2$ and LPIPS distortion, with gains of 8 dB (4:0:0), 6.5 dB (4:4:4 LR), and $\sim 30\%$ improvements in rate (LPIPS), respectively. We show that slim, light-weight networks with $57$K parameters can be used to closely approximate these results. The sandwich system can not only achieve a rate-distortion performance that is substantially superior to the standard video codec in these scenarios but it can do so without compromising computational efficiency through the use of light-weight networks. Our results clearly demonstrate that the sandwich system can re-purpose a standard codec to compression scenarios outside its immediate scope of design, from seamlessly increasing its resolution to optimizing it for a leading perceptual quality metric, all with significant improvements.

