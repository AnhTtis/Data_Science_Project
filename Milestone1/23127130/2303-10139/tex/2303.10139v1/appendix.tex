\clearpage
\newpage

\section{Proofs}
    \label{append:proofs}
    
    \begin{proof}[Proof of Lemma~\ref{theo:bound_unfaithfulness}]
        For node $u$, it is known, by definition, that
        \begin{equation*}
            \begin{split}
                &\left\lVert \Psi(\mathcal{G}_{u})- \Psi(t(\mathcal{G}_{u}, \mathcal{E}_u))\right\rVert_2 \\
                =\ & \left\lVert \sigma\left(\widetilde A^L X \Theta\right)_u - \sigma\left(\widetilde E^L X \Theta\right)_u \right\rVert_2 \\
                =\ &\left\lVert \sigma\left[\left(\widetilde A^L X \Theta\right)_u\right] - \sigma\left[\left(\widetilde E^L X \Theta\right)_u\right] \right\rVert_2,
            \end{split}
        \end{equation*}
        if we call $\sigma$ the $\text{softmax}$ function and $\widetilde E^L$ the powered, normalized adjancency matrix $\widetilde A^L$ after applying the explanation $\mathcal{E}_u$.
        Because $\text{softmax}$ is a Lipschitz continuous function with Lipschitz constant $1$ with respect to norm $\lVert \cdot \rVert_2$ \citep{gao_properties_2018} and considering induced matrix norm compatibility property \citep[Lemma 1.3.8.7]{van_de_geijn_advanced_2022},
        \begin{equation*}
            \begin{split}
                &\left\lVert \Psi(\mathcal{G}_{u})- \Psi(t(\mathcal{G}_{u}, \mathcal{E}_u))\right\rVert_2 \\
                =\ &\left\lVert \sigma\left[\left(\widetilde A^L X \Theta\right)_u\right] - \sigma\left[\left(\widetilde E^L X \Theta\right)_u\right] \right\rVert_2 \\
                \le\ &\left\lVert (X \Theta)^\intercal \right\rVert_2 \left\lVert \widetilde A_u^L - \widetilde E_u^L \right\rVert_2.
            \end{split}
        \end{equation*}
        Similarly, for a perturbation $\mathcal{G}_u'$ of $\mathcal{G}_u$ given by $\Sigma_{u}'$ being added to $X$,
        \begin{equation*}
            \begin{split}
                &\left\lVert \Psi(\mathcal{G}_{u}')- \Psi(t(\mathcal{G}_{u}', \mathcal{E}_u))\right\rVert_2 \\
                \le\ &\lVert [\left(X + \Sigma_{u}'\right) \Theta]^\intercal \rVert_2 \left\lVert \widetilde A_u^L - \widetilde E_u^L \right\rVert_2.
            \end{split}
        \end{equation*}
        Computing the mean over $\mathcal{K} \cup \{\mathcal{G}_u\}$:
        \begin{equation*}
            \begin{split}
                &\frac{1}{|\mathcal{K}| + 1} \sum_{\mathcal{G}_u' \in \mathcal{K} \cup \{\mathcal{G}_u\}} \left\lVert \Psi(\mathcal{G}_{u}')- \Psi(t(\mathcal{G}_{u}', \mathcal{E}_u))\right\rVert_2 \\
                \le\ &  \frac{1}{|\mathcal{K}| + 1} \Biggl[ \lVert (X \Theta)^\intercal \rVert_2 \left\lVert \widetilde A_u^L - \widetilde E_u^L \right\rVert_2 \\
                & \ \ \ \ \ \ \ \ \ \ \ \ \ \ \ \left. + \sum_{\mathcal{G}_u' \in \mathcal{K}}  \lVert [\left(X + \Sigma_{u}'\right) \Theta]^\intercal \rVert_2 \left\lVert \widetilde A_u^L - \widetilde E_u^L \right\rVert_2 \right] \\
                \le\ &  \lVert\Theta^\intercal\rVert_2 \left\lVert \widetilde A_u^L - \widetilde E_u^L \right\rVert_2  \frac{1}{|\mathcal{K}| + 1} \Biggl(\lVert X^\intercal \rVert_2 \\
                & \ \ \ \ \ \ \ \ \ \ \ \ \ \ \ \ \ \ \ \ \ \ \ \ \ \ \ \ \ \ \ \ \ \ \ \ \ \ \ + \left.\sum_{\mathcal{G}_u' \in \mathcal{K}} \lVert (X + \Sigma_{u}')^\intercal \rVert_2 \right) \\
                \le\ &  \gamma_\Theta \left\lVert \underset{\mathcal{E}_u}{\Delta} \widetilde A_u^L \right\rVert_2 \max \left(\{\lVert X^\intercal \rVert_2\} \cup \{\lVert (X + \Sigma_{u}')^\intercal \rVert_2\}_{\mathcal{G}_u' \in \mathcal{K}}\right) \\
                =\ &  \gamma_{\Theta, X, \Sigma} \left\lVert \underset{\mathcal{E}_u}{\Delta} \widetilde A_u^L \right\rVert_2.
            \end{split}
        \end{equation*}
        
        When $\Sigma$ is limited, the constant $\gamma$ may not depend on $\Sigma$.
    \end{proof}
    
    \begin{proof}[Proof of Theorem~\ref{theo:bound_unfaithfulness_2}]
        We know that
        \begin{equation*}
            \begin{split}
                &\left\lVert \Phi(\mathcal{G}_{u}')- \Phi(t(\mathcal{G}_{u}', \mathcal{E}_u))\right\rVert_2 \\
                =\ &\lVert \Phi(\mathcal{G}_{u}') - \Psi(\mathcal{G}_{u}') + \Psi(\mathcal{G}_{u}') - \Psi(t(\mathcal{G}_{u}', \mathcal{E}_u)) \\
                &+ \Psi(t(\mathcal{G}_{u}', \mathcal{E}_u)) - \Phi(t(\mathcal{G}_{u}', \mathcal{E}_u))\rVert_2.
            \end{split}
        \end{equation*}
        So, by using triangle inequality and \autoref{theo:bound_unfaithfulness},
        \begin{equation*}
            \begin{split}
                &\frac{1}{|\mathcal{K}| + 1} \sum_{\mathcal{G}_u' \in \mathcal{K} \cup \{\mathcal{G}_u\}} \left\lVert \Phi(\mathcal{G}_{u}')- \Phi(t(\mathcal{G}_{u}', \mathcal{E}_u))\right\rVert_2 \\
                \le\ & \frac{1}{|\mathcal{K}| + 1} \sum_{\mathcal{G}_u' \in \mathcal{K} \cup \{\mathcal{G}_u\}} \left\lVert \Phi(\mathcal{G}_{u}') - \Psi(\mathcal{G}_{u}')\right\rVert_2 \\
                &+ \frac{1}{|\mathcal{K}| + 1} \sum_{\mathcal{G}_u' \in \mathcal{K} \cup \{\mathcal{G}_u\}} \left\lVert \Psi(\mathcal{G}_{u}') - \Psi(t(\mathcal{G}_{u}', \mathcal{E}_u))\right\rVert_2 \\
                &+ \frac{1}{|\mathcal{K}| + 1} \sum_{\mathcal{G}_u' \in \mathcal{K} \cup \{\mathcal{G}_u\}} \left\lVert \Psi(t(\mathcal{G}_{u}', \mathcal{E}_u)) - \Phi(t(\mathcal{G}_{u}', \mathcal{E}_u))\right\rVert_2 \\
                \le\ & \gamma \left\lVert \underset{\mathcal{E}_u}{\Delta} \widetilde A_u^L \right\rVert_2 + 2\alpha.
            \end{split}
        \end{equation*}
    \end{proof}
    
    \begin{proof}[Proof of Lemma~\ref{theo:prob_bound_unfaithfulness}]
        From the proof of Lemma~\ref{theo:bound_unfaithfulness}, we know that
        \begin{equation*}
            \begin{split}
                &\frac{1}{|\mathcal{K}| + 1} \sum_{\mathcal{G}_u' \in \mathcal{K} \cup \{\mathcal{G}_u\}} \left\lVert \Psi(\mathcal{G}_{u}')- \Psi(t(\mathcal{G}_{u}', \mathcal{E}_u))\right\rVert_2 \le\ \\
                 &  \lVert\Theta^\intercal\rVert_2 \left\lVert \underset{\mathcal{E}_u}{\Delta} \widetilde A_u^L \right\rVert_2  \frac{1}{|\mathcal{K}| + 1} (\lVert X^\intercal \rVert_2 + \sum_{\mathcal{G}_u' \in \mathcal{K}} \lVert (X + \Sigma_{u}')^\intercal \rVert_2 )
            \end{split}
        \end{equation*}
        By using triangle inequality, we can write
        \begin{equation*}
            \begin{split}
                &\frac{1}{|\mathcal{K}| + 1} \sum_{\mathcal{G}_u' \in \mathcal{K} \cup \{\mathcal{G}_u\}} \left\lVert \Psi(\mathcal{G}_{u}')- \Psi(t(\mathcal{G}_{u}', \mathcal{E}_u))\right\rVert_2 \\
                \le\ &  \lVert\Theta^\intercal\rVert_2 \left\lVert \underset{\mathcal{E}_u}{\Delta} \widetilde A_u^L \right\rVert_2 \lVert X^\intercal \rVert_2 \\
                &+ \lVert\Theta^\intercal\rVert_2 \left\lVert \underset{\mathcal{E}_u}{\Delta} \widetilde A_u^L \right\rVert_2 \frac{1}{|\mathcal{K}| + 1} \sum_{\mathcal{G}_u' \in \mathcal{K}} \lVert (\Sigma_{u}')^\intercal \rVert_2.
            \end{split}
        \end{equation*}
        We can work in the perturbation summand:
        \begin{equation*}
            \begin{split}
                \sum_{\mathcal{G}_u' \in \mathcal{K}} \lVert (\Sigma_{u}')^\intercal \rVert_2 \le\ & \sum_{\mathcal{G}_u' \in \mathcal{K}} \lVert \Sigma_{u}' \rVert_\text{F} \\
                =\ & \sum_{\mathcal{G}_u' \in \mathcal{K}} \sqrt{\sum_{i=1}^{nd} \epsilon_{u',i}^2} \\
                =\ & \sigma \sum_{\mathcal{G}_u' \in \mathcal{K}} \sqrt{\sum_{i=1}^{nd} Z_{u',i}^2} \\
                \le\ & \sigma \sum_{\mathcal{G}_u' \in \mathcal{K}} \max\left(1, \ {\sum_{i=1}^{nd} Z_{u',i}^2}\right) \\
                \le\ & \sigma \sum_{\mathcal{G}_u' \in \mathcal{K}} \left(1 + {\sum_{i=1}^{nd} Z_{u',i}^2}\right) \\
                =\ & \sigma |\mathcal{K}| + \sigma Q,
            \end{split}
        \end{equation*}
        with
        \begin{itemize}
            \item $\epsilon_{j, i}~\sim~\mathcal{N}(0, \sigma^2)$;
            \item $Z_{j, i}~\sim~\mathcal{N}(0, 1)$;
            \item $Q~\sim~\chi_{|\mathcal{K}|nd}^2$;
            \item $(n, d)~=~\text{dim}(X)$.
        \end{itemize}
        
        If $\alpha_p$ is the $p$-percentile of $Q$, then the probability of the last inequality is $p$:
        \begin{equation*}
            \begin{split}
                &\frac{1}{|\mathcal{K}| + 1} \sum_{\mathcal{G}_u' \in \mathcal{K} \cup \{\mathcal{G}_u\}} \left\lVert \Psi(\mathcal{G}_{u}')- \Psi(t(\mathcal{G}_{u}', \mathcal{E}_u))\right\rVert_2 \\
                \le\ &  \lVert\Theta^\intercal\rVert_2 \left\lVert \underset{\mathcal{E}_u}{\Delta} \widetilde A_u^L \right\rVert_2 \lVert X^\intercal \rVert_2 \\
                &+ \lVert\Theta^\intercal\rVert_2 \left\lVert \underset{\mathcal{E}_u}{\Delta} \widetilde A_u^L \right\rVert_2 \frac{1}{|\mathcal{K}| + 1} \sum_{\mathcal{G}_u' \in \mathcal{K}} \lVert (\Sigma_{u}')^\intercal \rVert_2 \\
                \le\ & \gamma_1 \left\lVert \underset{\mathcal{E}_u}{\Delta} \widetilde A_u^L \right\rVert_2 + \lVert\Theta^\intercal\rVert_2 \left\lVert \underset{\mathcal{E}_u}{\Delta} \widetilde A_u^L \right\rVert_2 \frac{1}{|\mathcal{K}| + 1} \sigma (Q + |\mathcal{K}|) \\
                \le\ & \gamma_1 \left\lVert \underset{\mathcal{E}_u}{\Delta} \widetilde A_u^L \right\rVert_2 + \lVert\Theta^\intercal\rVert_2 \left\lVert \underset{\mathcal{E}_u}{\Delta} \widetilde A_u^L \right\rVert_2 \frac{1}{|\mathcal{K}| + 1} \sigma (\alpha_p + |\mathcal{K}|) \\
                =\ &   \gamma_1 \left\lVert \underset{\mathcal{E}_u}{\Delta} \widetilde A_u^L \right\rVert_2 + \gamma_2 \left\lVert \underset{\mathcal{E}_u}{\Delta} \widetilde A_u^L \right\rVert_2 \sigma (\alpha_p + |\mathcal{K}|)
            \end{split}
        \end{equation*}
        Notice that, when $\sigma$ approaches zero, the bound's probabilistic characteristic  becomes negligible.
        
        Finally, if we ask the bound to be $\xi$, then the probability is
        \[p = F_{\chi_{|\mathcal{K}|nd}^2}\left(\frac{\xi - \gamma_1 \left\lVert \underset{\mathcal{E}_u}{\Delta} \widetilde A_u^L \right\rVert_2}{\gamma_2 \left\lVert \underset{\mathcal{E}_u}{\Delta} \widetilde A_u^L \right\rVert_2 \sigma} - |\mathcal{K}| \right),\]
        $F_{\chi_{|\mathcal{K}|nd}^2}$ being the c.d.f. of $Q$.
    \end{proof}
    
    \begin{proof}[Proof of \autoref{theo:convexity}]
        The objective function of the problem in \autoref{eq:e2} can be written as
        \begin{equation*}
            \begin{split}
                f(\mathcal{E}) =\ & \left\lVert \widetilde A_i^L \text{diag}(\mathcal{E}) X \Theta - \widetilde A_i^L X \Theta\right\rVert_2^2 \\
                =\ &\left( \widetilde A_i^L \text{diag}(\mathcal{E}) X \Theta - \widetilde A_i^L X \Theta\right) \cdot \\
                &\ \cdot \left( \widetilde A_i^L \text{diag}(\mathcal{E}) X \Theta - \widetilde A_i^L X \Theta\right)^\intercal \\
                =\ &\mathcal{E}^\intercal \text{diag}\left[\left(\widetilde A_i^L\right)^\intercal\right] X \Theta \Theta^\intercal X^\intercal \text{diag}\left[\left(\widetilde A_i^L\right)^\intercal\right] \mathcal{E} \\
                &- 2 \mathcal{E}^\intercal \text{diag}\left[\left(\widetilde A_i^L\right)^\intercal\right] X \Theta \Theta^\intercal X^\intercal \left(\widetilde A_i^L\right)^\intercal \\
                &\ + \left\lVert\widetilde A_i^L X \Theta\right\rVert_2^2 \\
                =\ &\frac{1}{2}\mathcal{E}^\intercal Q \mathcal{E} + \mathcal{E}^\intercal c + \delta.
            \end{split}
        \end{equation*}
        Note also that \[Q = P^\intercal P \text{ with } \ P = \sqrt{2} \Theta^\intercal X^\intercal \text{diag}\left[\left(\widetilde A_i^L\right)^\intercal\right],\]
        thus $Q$ is symmetric and positive semidefinite. Since both the objective function and feasible set are convex, the optimization problem is also convex. 
    \end{proof}
    
    % \begin{theorem}[Relaxed optimization problem solution]
    %     \label{theo:solution}
    %     The optimization problem of Equation~\ref{eq:e2}, with the relaxation of $\mathcal{E}$ belonging to the hyperplane generated by $\Delta$ (i.e., $\mathcal{E}$ sums up to one but can have negative values) has solutions of the form $\mathcal{E} = Zy + d$, where $Z$ and $d$ are constant matrix and vector (with respect to $X, \Theta, \widetilde A^L$) and $y$ is a solution for a problem of the form $\widetilde Q y = \widetilde c$,
    %     where $\widetilde Q$ and $\widetilde c$ depend on $X, \Theta, \widetilde A^L$.
    %     $\widetilde Q$ is positive semidefinite.
    %     Assuming $\Theta^\intercal X^\intercal \text{diag}\left[\left(\widetilde A_i^L\right)^\intercal\right]$ has rank equal to the number of nodes in the explanation $\mathcal{E}$, $\widetilde Q$ is invertible and $\mathcal{E}$ is unique.
    % \end{theorem}
        
    % \begin{proof}[Sketch proof]
    %     We first rewrite the objective function of the relaxation of Equation~\ref{eq:e2} as a canonical quadratic problem with equality constraints.
    %     After that, we remove the equality constraints by a variable transformation.
    %     We solve this simpler quadratic problem and state the result.
    %     When the matrices of the original problem are such that $\Theta^\intercal X^\intercal \text{diag}\left[\left(\widetilde A_i^L\right)^\intercal\right]$ has nullity zero, we can conclude $\widetilde Q$ is invertible and the solution $\mathcal{E}$ is unique.
    %     For the details, refer to the complete proof.
    % \end{proof}
    
    % \begin{proof}
        % The objective function of the problem of Equation~\ref{eq:e2} can be restated as
        % \begin{equation*}
        %     \begin{split}
        %         f(\mathcal{E}) =\ & \left\lVert \widetilde A_i^L \text{diag}(\mathcal{E}) X \Theta - \widetilde A_i^L X \Theta\right\rVert_2^2 \\
        %         =\ &\left( \widetilde A_i^L \text{diag}(\mathcal{E}) X \Theta - \widetilde A_i^L X \Theta\right) \cdot \\
        %         &\ \cdot \left( \widetilde A_i^L \text{diag}(\mathcal{E}) X \Theta - \widetilde A_i^L X \Theta\right)^\intercal \\
        %         =\ &\mathcal{E}^\intercal \text{diag}\left[\left(\widetilde A_i^L\right)^\intercal\right] X \Theta \Theta^\intercal X^\intercal \text{diag}\left[\left(\widetilde A_i^L\right)^\intercal\right] \mathcal{E} \\
        %         &- 2 \mathcal{E}^\intercal \text{diag}\left[\left(\widetilde A_i^L\right)^\intercal\right] X \Theta \Theta^\intercal X^\intercal \left(\widetilde A_i^L\right)^\intercal \\
        %         &\ + \left\lVert\widetilde A_i^L X \Theta\right\rVert_2^2 \\
        %         =\ &\frac{1}{2}\mathcal{E}^\intercal Q \mathcal{E} + \mathcal{E}^\intercal c + \delta.
        %     \end{split}
        % \end{equation*}
        % Because \[Q = P^\intercal P,\ P = \sqrt{2} \Theta^\intercal X^\intercal \text{diag}\left[\left(\widetilde A_i^L\right)^\intercal\right],\]
        % $Q$ is symmetric and positive semidefinite.
        % Therefore, the optimization problem can be rewritten as
        % \begin{equation*}
        %     \begin{split}
        %         \min\ \ &\frac{1}{2}x^\intercal Q x + x^\intercal c, \\
        %         \text{subject to}\ \ &\sum_{i = 1}^n x_i = 1, x \in \mathbb{R}^n.
        %     \end{split}
        % \end{equation*}
        % Note that this is equivalent to
        % \begin{equation*}
        %     \begin{split}
        %         \min\ \ &\frac{1}{2}x^\intercal Q x + x^\intercal c, \\
        %         \text{subject to}\ \ &
        %         \begin{bmatrix}
        %                 x_1 \\
        %                 \vdots \\
        %                 x_{n-1} \\
        %                 x_n
        %         \end{bmatrix} = 
        %         \begin{bmatrix}
        %                 y_1 \\
        %                 \vdots \\
        %                 y_{n-1} \\
        %                 1 - \sum_{i = 1}^{n-1} y_i
        %         \end{bmatrix}, \\
        %             &\ \ x \in \mathbb{R}^n, y \in \mathbb{R}^{n-1},
        %     \end{split}
        % \end{equation*}
        % because $\sum x_i = 1$.
        % We can write
        % \[x = Zy + d \coloneqq \begin{bmatrix}
        %     1 & \cdots & 0 \\
        %     \vdots & \ddots & \vdots \\
        %     0 & \cdots & 1 \\
        %     -1 & \cdots & -1
        % \end{bmatrix} y + \begin{bmatrix}
        %     0 \\
        %     \vdots \\
        %     0 \\
        %     1
        % \end{bmatrix},\]
        % so that the problem becomes
        % \begin{equation*}
        %     \begin{split}
        %         \min\ \ &\frac{1}{2}(Zy + d)^\intercal Q (Zy + d) + (Zy + d)^\intercal c, \\
        %         \text{subject to}\ \ & y \in \mathbb{R}^{n-1}.
        %     \end{split}
        % \end{equation*}
        % This new objective function can be written
        % \begin{equation*}
        %     \begin{split}
        %         &\frac{1}{2}(Zy + d)^\intercal Q (Zy + d) + (Zy + d)^\intercal c \\
        %         =\ &\frac{1}{2}y^\intercal(Z^\intercal QZ)y + \left[Z^\intercal\left(Qd + c\right)\right]^\intercal y + \frac{1}{2}d^\intercal Qd + c^\intercal d \\
        %         =\ &\frac{1}{2}y^\intercal \widetilde Q y + \tilde c^\intercal y + \gamma.
        %     \end{split}
        % \end{equation*}
        % Finally, by setting the gradient to zero, the solution to this new unconstrained quadratic programming problem is a solution to
        % \[\widetilde Q y + \tilde c = 0 \Rightarrow \widetilde Q y = - \tilde c.\]
        
        % If we assume $P = \Theta^\intercal X^\intercal \text{diag}\left[\left(\widetilde A_i^L\right)^\intercal\right]$ has rank equal to the number of nodes in the explanation $\mathcal{E}$, then $P$ has nullity zero.
        % Therefore,
        % \[y \ne 0 \Rightarrow Zy \ne 0 \Rightarrow PZy \ne 0\]
        % and, if $y \ne 0$,
        % \[y^\intercal \widetilde Q y = y^\intercal (Z^\intercal P^\intercal P Z) y = \lVert PZy \rVert_2^2 > 0.\]
        % So $\widetilde Q$, which is symmetric, is also positive definite, therefore invertible, which leads to $y = -\widetilde Q^{-1} \tilde c$.
        % In this case, the unique solution is
        % \[\mathcal{E} = x = Zy + d = -Z\widetilde Q^{-1} \tilde c + d.\]
    % \end{proof}


 

\section{Datasets and implementation details}\label{append:implementation}
\label{sec:details_datasets}

\subsection{Datasets}

Bitcoin-Alpha and Bitcoin-OTC are real-world networks comprising 3783 and 5881 nodes (user accounts), respectively. Users rate their trust in each other using a score between $-10$ (total distrust) and $10$ (total trust). Then, user accounts are labeled as trusted or not based on how other users rate them. Accounts (nodes) have features such as average ratings. Ground-truth explanations for each node are provided by experts.
%
The synthetic datasets are available in \citep{GNNexplainer} and \citep{PGMExplainer}.
%
\autoref{tab:details_datasets} shows summary statistics for all datasets.


\begin{table}[!htb]
\centering
\caption{Statistics of the datasets used in our experiments.}
\adjustbox{width=0.35\textwidth}{
\begin{tabular}{lccc}
\toprule
\textbf{Dataset} & \textbf{nodes} & \textbf{edges} & \textbf{labels}\\
\midrule
BA-House & 700 & 4110 & 4\\
BA-Community & 1400 & 8920 & 8\\
BA-Grids & 1020 & 5080 & 2\\
Tree-Cycles & 871 & 1950 & 2\\
Tree-Grids & 1231 & 3410 & 2 \\
BA-Bottle & 700 & 3948 & 4\\
Bitcoin-Alpha & 3783 & 28248 & 2\\
Bitcoin-OTC & 5881 & 42984 & 2\\
\bottomrule
\end{tabular}}
\label{tab:details_datasets}
\end{table}


\subsection{Implementation details}
\begin{table*}[!t]
\centering
\caption{Performance (mean and standard deviation of accuracy) in node classification tasks for the models to be explained.}

\adjustbox{width=\textwidth}{
\begin{tabular}{lccccccccc}
\toprule
  %& syn1  &  syn2 & syn3 & syn4 & syn5 & syn6 & bitcoin-alpha & bitcoin-otc \\ \toprule
  & \textbf{BA-House} &\textbf{ BA-Community}  &\textbf{ BA-Grids} & \textbf{Tree-Cycles} & \textbf{Tree-Grids} & \textbf{ BA-Bottle} & \textbf{Bitcoin-Alpha} & \textbf{Bitcoin-OTC} \\ \toprule
   %&\multicolumn{7}{c}{GCN} \\ \hline

 GCN & $97.9 \pm{1.6}$  & $85.6\pm{1.7}$  & $99.9\pm{0.2}$ &  $97.8\pm{1.2}$  & $88.9\pm{1.8}$ &  $99.0\pm{0.1}$ &  $93.3\pm{0.4}$  & $93.2\pm{0.6}$ \\
  ARMA &  $98.1\pm{2.3}$ & $92.8 \pm{2.3}$& $99.5 \pm{0.6}$& $96.9 \pm{1.5}$&$92.4 \pm{2.3}$& $99.6 \pm{0.9}$& $93.6\pm{1.4}$ & $92.9\pm{1.1}$\\

 %GAT & $100.0 \pm{}$ & $77.8\pm{}$ & $100.0\pm{}$ & $97.7\pm{}$ & $93.4\pm{}$ &$98.5\pm{}$ & $89.6\pm{}$ & $89.6\pm{}$\\
 GATED & $98.0\pm{1.3}$ & $92.3\pm{2.7}$ & $99.9 \pm{0.2}$& $97.8\pm{2.2}$ & $94.4 \pm{3.0}$& $99.4\pm{0.6}$ & $94.4\pm{1.3}$& $93.8\pm{0.7}$\\
 GIN & $95.6 \pm{5.0}$  & $87.0 \pm{1.5}$ & $99.4\pm{0.6}$ &  $97.8 \pm{1.1}$ & $91.4 \pm{2.6}$ &  $98.8\pm{1.0}$ &  $93.4 \pm{1.0}$ & $92.6\pm{1.1}$ \\
 
\bottomrule
\end{tabular}}
\label{tab:classification}
\end{table*}

We ran all experiments using a laptop with an Intel Xeon 2.20 GHz CPU,  13 GB RAM DDR5, and RTX 3060ti 16GB GPU.

The architecture of the GNNs to be explained in this work are: 
$i$) a GCN model with 3 layers (20 hidden units) followed by a two-layer MLP with 60 hidden units; $ii$) an ARMA model with 3 layers (20 hidden units each) followed by a two-layer MLP with 60 hidden units; $iii$) a GIN model with 3 layers (20 hidden units each); and $iv$) a GATED model with 3 layers (100 hidden units) followed by a two-layer MLP with 300 hidden units. We train all these GNNs using with a learning rate of 0.001. All models use ReLU activations in their hidden units. We also use the one-hot vector of the degree as node features.

For computational reasons, on Bitcoin datasets, we provide results for 500 test nodes whenever the candidate explanation set contemplates 3-hop neighborhood (matching the GNNs we want to explain). For 1-hop cases, we use the full set of 2000 nodes as in the original setup.



\paragraph{Node and edge-level explanations.} DnX and FastDnX were originally designed to generate node-level explanations like PGM-Explainer but some baselines such as PGexplainer and GNNexplainer provide edge-level explanations. For a fair comparison with these baselines, we use a procedure to convert edge-level explanations to node-level ones (and vice-versa). To convert node scores to edge scores, we sum the scores of endpoints of each edge, i.e., an edge $(u,v)$ gets score $s_{u,v} = s_u + s_v$ where $s_u$ and $s_v$ are node scores. For the reverse, we say $s_u = {|\mathcal{N}_u|}^{-1}\sum_{j \in \mathcal{N}_u} s_{u, j}$ where $\mathcal{N}_u$ is the neighborhood of $u$.


 \section{Additional experiments}
\label{append:sup_experiments} 

\vspace{-1pt}
\paragraph{GNN performance.} \autoref{tab:classification} shows the classification accuracy of each model we explain in our experimental campaign. Means and standard deviations reflect the outcome of 10 repetitions. All classifiers achieve accuracy $>85\%$.
\begin{table}[!htb]
\centering
\caption{Distillation accuracy and elapsed time for ARMA, GATED and GIN. For all cases, accuracy $>88\%$ and learning $\Psi$ takes less than a minute.} 
\adjustbox{width=\columnwidth}{
\begin{tabular}{llcc}
\toprule
\textbf{Model} &\textbf{Dataset} & \textbf{Accuracy} & \textbf{Time (s)} \\ \midrule
%\multicolumn{3}{c}{\textbf{ARMA}} \\\midrule
%\textbf{Dataset} & \textbf{Accuracy} & \textbf{Time (s)} \\ \midrule
\multirow{8}{*}{\shortstack{ARMA}}

&BA-House & $97.7 \pm{0.01}$ & 7.254\\
&BA-Community  & $96.7\pm{0.03}$ & 22.848 \\
&BA-Grids & $100.0\pm{0.00}$ & 2.701 \\
&Tree-Cycles & $98.8\pm{0.03}$ & 6.331 \\
&Tree-Grids & $91.7\pm{0.15}$ & 4.152\\
&BA-Bottle & $100.0\pm{0.00}$ & 6.312\\
&Bitcoin-Alpha & $91.3\pm{0.01}$ & 28.613 \\
&Bitcoin-OTC & $91.2\pm{0.03}$ & 39.832\\ 


\midrule 
\multirow{8}{*}{\shortstack{GATED}}

%\multicolumn{3}{c}{\textbf{GATED}} \\\midrule
%\textbf{Dataset} & \textbf{Accuracy} & \textbf{Time (s)} \\ \midrule
&BA-House & $98.0 \pm{0.01}$ &  5.954\\
&BA-Community  & $92.1\pm{0.04}$ & 27.847 \\
&BA-Grids & $100.0\pm{0.00}$ & 2.679 \\
& Tree-Cycles & $99.4\pm{0.04}$ & 6.643\\
&Tree-Grids & $90.9\pm{0.09}$ & 4.609 \\
&BA-Bottle & $100.0\pm{0.00}$ & 6.171 \\
&Bitcoin-Alpha & $91.3\pm{0.02}$ & 26.611 \\
&Bitcoin-OTC & $91.1\pm{0.01}$ & 42.626\\ 
\midrule 

\multirow{8}{*}{\shortstack{GIN}}

%&\multicolumn{3}{c}{\textbf{GIN}} \\\midrule
%\textbf{Dataset} & \textbf{Accuracy} & \textbf{Time (s)} \\ \midrule
&BA-House & $93.4 \pm{0.46}$ & 11.461 \\
&BA-Community  & $90.1\pm{0.13}$ & 24.204 \\
&BA-Grids & $100.0\pm{0.00}$ & 3.884 \\
& Tree-Cycles & $95.8\pm{0.02}$ & 3.335\\
&Tree-Grids & $88.1\pm{0.06}$ & 5.198\\
&BA-Bottle & $100.0\pm{0.01}$ & 6.970\\
&Bitcoin-Alpha & $95.5\pm{0.03}$ & 44.359\\
&Bitcoin-OTC & $89.8\pm{2.04}$ & 55.860\\ 

\bottomrule
\end{tabular}
}
\label{tab:distillation_models}
\end{table}

\vspace{-1pt}
\paragraph{Distillation.} \autoref{tab:distillation_models} shows the extent to which the distiller network $\Psi$ agrees with the GNN $\Phi$. We measure agreement as accuracy, using the predictions of $\Phi$ as ground truth. The means and standard deviations reflect the outcome of 10 repetitions. Additionally, \autoref{tab:distillation_models} shows the  time elapsed during the distillation step. For all cases, we observe accuracy values over $88\%$.



\vspace{-1pt}
\paragraph{Results for edge-level explanations.} \autoref{tab:result_exp_edges} and \autoref{tab:result_exp_bitcoin_edges} complement Tables 1 and 2 in the main paper, showing results for edge-level explanations. All experiments were repeated ten times. For all datasets, we measure performance in terms of AUC, following \citet{PGExplainer}. Both tables corroborate our findings from the main paper. In most cases, FastDnX is the best or second-best model for all models and datasets. For GCN and GATED, PGExplainer yields the best results. Overall, both Dnx and FastDnX outperform GNNExplainer and PGM-Explainer. Remarkably FastDnX and DnX's performance is steady, with small fluctuations depending on the model we are explaining. The same is not true for the competing methods, e.g., PGExplainer loses over $15\%$ AUC for the BA-Community (cf., GCN and ARMA). Note also that FastDnX and DnX are the best models on the real-world datasets.

\begin{table*}[thb]
\centering
\caption{Performance (mean and standard deviation of AUC) of explainer models on synthetic datasets for edge-level explanations. \textcolor{royal}{Blue} and \textcolor{light}{Green} numbers denote the best and second-best methods, respectively.}
\adjustbox{width=\textwidth}{
\begin{tabular}{ccccccccc}
\toprule

 \textbf{Model} & \textbf{Explainer} & \textbf{BA-House} & \textbf{BA-Community}  & \textbf{BA-Grids} &\textbf{ Tree-Cycles} & \textbf{Tree-Grids} &  \textbf{BA-Bottle} \\ \toprule
 %GCN
\multirow{5}{*}{\shortstack{GCN}}

&GNNExplainer & $82.4\pm{6.0}$  & $71.1\pm{6.0}$ &  $80.9\pm{1.0}$ & $58.4\pm{1.0}$ & $53.9\pm{2.0}$ &$82.8\pm{1.0}$  \\
&PGExplainer & \textcolor{royal}{$\bm{99.9\pm{.01}}$}  & \textcolor{royal}{$\bm{99.9\pm{.01}}$} &  $94.1\pm{9.0}$ & \textcolor{royal}{$\bm{92.3\pm{5.0}}$} & \textcolor{royal}{$\bm{79.4\pm{2.0}}$} &\textcolor{royal}{$\bm{99.9\pm{.01}}$}  \\
&PGMExplainer & $56.2\pm{0.3}$ & $53.0\pm{0.4}$ & $68.9\pm{0.2}$ & $62.1\pm{0.2}$ & $66.3\pm{0.4}$ & $53.4\pm{0.5}$\\

\cmidrule{2-8}
&DnX & $95.7\pm{.09}$ & $87.5\pm{.09}$ & \textcolor{royal}{$\bm{97.3\pm{.03}}$} & $80.5\pm{.02}$ & $72.6\pm{0.1}$ & $94.5\pm{0.1}$ \\
&FastDnX & \textcolor{light}{$\bm{99.4\pm{\text{NA}}}$} & \textcolor{light}{$\bm{93.0\pm{\text{NA}}}$} & \textcolor{light}{$\bm{96.7\pm{\text{NA}}}$} & \textcolor{light}{$\bm{89.1\pm{\text{NA}}}$} & \textcolor{light}{$\bm{77.4\pm{\text{NA}}}$} & \textcolor{light}{$\bm{99.5\pm{\text{NA}}}$} \\ \midrule \midrule
 
\multirow{5}{*}{\shortstack{ARMA}}
 %ARMA
 & GNNExplainer & $80.8\pm{1.0}$  & $75.1\pm{1.0}$ &  $69.4\pm{0.1}$ & $59.2\pm{2.0}$ & $59.9\pm{1.0}$ & $85.8\pm{.01}$  \\
 & PGExplainer & $93.3\pm{1.0}$  & $82.7\pm{.01}$ &  $93.1\pm{.01}$ & \textcolor{royal}{$\bm{92.4\pm{2.0}}$} & \textcolor{royal}{$\bm{84.0\pm{.03}}$} & \textcolor{light}{$\bm{98.0\pm{.02}}$}  \\
 &PGMExplainer & $82.2\pm{0.2}$  & $49.6\pm{0.7}$  &  $40.4\pm{0.2}$ & $64.7\pm{0.2}$ & $69.0\pm{0.3}$ & $63.1\pm{0.3}$  \\

\cmidrule{2-8}
 & DnX & \textcolor{light}{$\bm{95.7\pm{.06}}$} & \textcolor{light}{$\bm{87.9\pm{.07}}$} & \textcolor{light}{$\bm{97.3\pm{.04}}$} & $80.4\pm{0.3}$ & $72.8\pm{0.1}$ & $94.4\pm{.08}$ \\
 & FastDnX & \textcolor{royal}{$\bm{99.7\pm{\text{NA}}}$} & \textcolor{royal}{$\bm{96.8\pm{\text{NA}}}$} & \textcolor{royal}{$\bm{97.9\pm{\text{NA}}}$} & \textcolor{light}{$\bm{87.1\pm{\text{NA}}}$} & \textcolor{light}{$\bm{78.4\pm{\text{NA}}}$} & \textcolor{royal}{$\bm{99.7\pm{\text{NA}}}$} \\ \midrule \midrule

\multirow{5}{*}{\shortstack{GATED}}

% GATED
&GNNExplainer & $79.1\pm{2.0}$  & $59.2\pm{3.0}$ &  $69.4\pm{2.0}$ & $62.6\pm{2.0}$ & $53.6\pm{1.0}$ &$73.9\pm{2.0}$  \\
&PGExplainer & \textcolor{royal}{$\bm{99.9\pm{2.0}}$}  & \textcolor{royal}{$\bm{ 99.1 \pm{2.0}}$} &  \textcolor{royal}{$\bm{99.8\pm{1.0}}$} & $76.5\pm{9.0}$ & \textcolor{royal}{$\bm{97.0\pm{3.0}}$} &\textcolor{royal}{$\bm{99.9\pm{0.1}}$}  \\
&PGMExplainer & $49.3\pm{0.4}$  & $47.2\pm{0.5}$  &  $46.2\pm{0.3}$ & $45.8\pm{0.6}$ & $53.1\pm{0.8}$ &$49.1\pm{0.7}$  \\

\cmidrule{2-8}
&DnX & $95.8\pm{.07}$ & $84.4\pm{.09}$ & \textcolor{light}{$\bm{97.3\pm{.02}}$} & \textcolor{light}{$\bm{81.1\pm{0.1}}$} & $72.8\pm{0.1}$ & $94.4\pm{.09}$ \\ 
&FastDnX & \textcolor{light}{$\bm{99.5\pm{\text{NA}}}$} & \textcolor{light}{$\bm{93.7\pm{\text{NA}}}$} & $96.6\pm{\text{NA}}$ & \textcolor{royal}{$\bm{89.2\pm{\text{NA}}}$} & \textcolor{light}{$\bm{78.6\pm{\text{NA}}}$} & \textcolor{light}{$\bm{99.6\pm{\text{NA}}}$} \\ \midrule \midrule

\multirow{3}{*}{\shortstack{GIN}}

  %GIN
% & GNNExplainer &  &  &   &  &  &  \\
% & PGExplainer &  &  &   &  &  &   \\
  & PGMExplainer & $52.4\pm{0.3}$ &  $52.0\pm{0.4}$ & $52.2\pm{1.0}$ &  $67.4\pm{0.2}$& $63.9\pm{0.2}$ & $53.3\pm{0.4}$ \\
  \cmidrule{2-8}
 & DnX &  \textcolor{light}{$\bm{95.7\pm{0.1}}$} & \textcolor{light}{$\bm{87.2\pm{0.1}}$} & \textcolor{royal}{$\bm{97.3\pm{0.1}}$} & \textcolor{light}{$\bm{81.8\pm{0.2}}$} & \textcolor{light}{$\bm{72.0\pm{0.1}}$} & \textcolor{light}{$\bm{94.2\pm{0.1}}$}\\
 
 & FastDnX & \textcolor{royal}{$\bm{97.9\pm{\text{NA}}}$} & \textcolor{royal}{$\bm{91.0\pm{\text{NA}}}$}& \textcolor{light}{$\bm{97.1\pm{\text{NA}}}$} &  \textcolor{royal}{$\bm{86.3\pm{\text{NA}}}$} & \textcolor{royal}{$\bm{72.3\pm{\text{NA}}}$} & \textcolor{royal}{$\bm{97.0\pm{\text{NA}}}$}\\ 


% syn1
% Accuracy:  0.9965
% Precision:  0.9965
% Auc:  0.979241758013846
% syn2
% Accuracy:  0.94725
% Precision:  0.94725
% Auc:  0.9106418121195393
% syn3
% Accuracy:  0.9401234567901234
% Precision:  0.948171206225681
% Auc:  0.9717303349361439
% syn4
% Accuracy:  0.7847222222222222
% Precision:  0.7847222222222222
% Auc:  0.8634922969150769
% syn5
% Accuracy:  0.7748456790123457
% Precision:  0.7773649171698406
% Auc:  0.7230109768658525
% syn6
% Accuracy:  0.9915
% Precision:  0.9915
% Auc:  0.9705748969506456


\bottomrule
\end{tabular}}
%\end{adjustbox}
\label{tab:result_exp_edges}
\end{table*}






% \begin{table}[!htb]
% \centering
% \caption{Performance (accuracy) of explanatory models for synthetic datasets for the GCN model (edge level)}
% \adjustbox{width=0.5\textwidth}{
% \begin{tabular}{cccccccc}
% \toprule
% %  & syn 1  & syn 2  & syn 3 & syn 4 & syn 5 &  syn 6 \\ \toprule
%   &BA-House & BA-Community  & BA-Grids & Tree-Cycles & Tree-Grids &  BA-Bottle \\ \toprule
% GNNExplainer & $0.824\pm{}$  & $0.711\pm{}$ &  $0.809\pm{}$ & $0.584\pm{}$ & $0.539\pm{}$ &$0.828\pm{}$  \\
% PGExplainer & $0.999\pm{}$  & $0.999\pm{}$ &  $0.941\pm{}$ & $0.423\pm{}$ & $0.794\pm{}$ &$0.999\pm{}$  \\
% PGM-Explainer \\


% %PGM-Explainer & $0.516\pm{0.00}$  & $0.498\pm{0.00}$  &  $0.661\pm{0.00}$ & $0.660\pm{0.00}$ & $0.710\pm{0.00}$ &$0.524\pm{0.00}$  \\
% %ConveX & 0.982  & 0.911 & 0.908 & 0.849 & 0.819 &0.983\\
% our & $0.994\pm{0.00}$ & $0.930\pm{0.00}$ & $0.967\pm{0.00}$ & $0.891\pm{0.00}$ & $0.774\pm{0.00}$ & $0.995\pm{0.00}$ \\
% ConveX & $0.957\pm{0.0009}$ & $0.873\pm{0.00}$ & $0.973\pm{0.0003}$ & $0.805\pm{0.00}$ & $0.726\pm{0.00}$ & $0.947\pm{0.00}$ \\

% \bottomrule
% \end{tabular}}
% %\end{adjustbox}
% \label{tab:result_exp}
% \end{table}

% \begin{table}[!htb]
% \centering
% \caption{Performance (accuracy) of explanatory models for synthetic datasets for the GATED model (edge level)}
% \adjustbox{width=0.5\textwidth}{
% \begin{tabular}{cccccccc}
% \toprule
% %  & syn 1  & syn 2  & syn 3 & syn 4 & syn 5 &  syn 6 \\ \toprule
%   &BA-House & BA-Community  & BA-Grids & Tree-Cycles & Tree-Grids &  BA-Bottle \\ \toprule
% GNNExplainer & $0.791\pm{0.02}$  & $0.592\pm{0.03}$ &  $0.694\pm{0.02}$ & $0.626\pm{0.02}$ & $0.536\pm{0.01}$ &$0.739\pm{0.02}$  \\
% PGExplainer & $0.999\pm{0.02}$  & $\pm{}$ &  $0.998\pm{0.00}$ & $0.765\pm{0.11}$ & $0.970\pm{0.19}$ &$0.999\pm{0.00}$  \\
% PGM-Explainer & $0.496\pm{0.00}$  & $0.441\pm{0.00}$  &  $0.468\pm{0.00}$ & $0.450\pm{0.00}$ & $0.518\pm{0.00}$ &$0.476\pm{0.00}$  \\

% %ConveX & 0.982  & 0.911 & 0.908 & 0.849 & 0.819 &0.983\\
% our & $0.995\pm{0.00}$ & $0.937\pm{0.00}$ & $0.966\pm{0.00}$ & $0.892\pm{0.00}$ & $0.786\pm{0.00}$ & $0.996\pm{0.00}$ \\
% ConveX & $0.957\pm{0.00}$ & $0.844\pm{0.00}$ & $0.973\pm{0.00}$ & $0.813\pm{0.00}$ & $0.727\pm{0.00}$ & $0.945\pm{0.00}$ \\

% \bottomrule
% \end{tabular}}
% %\end{adjustbox}
% \label{tab:result_exp}
% \end{table}


% \begin{table}[!htb]
% \centering
% \caption{Performance (accuracy) of explanatory models for synthetic datasets for the GAT model (node level)}
% \adjustbox{width=0.5\textwidth}{
% \begin{tabular}{cccccccc}
% \toprule
% %  & syn 1  & syn 2  & syn 3 & syn 4 & syn 5 &  syn 6 \\ \toprule
%   &BA-House & BA-Community  & BA-Grids & Tree-Cycles & Tree-Grids &  BA-Bottle \\ \toprule
% GNNExplainer \\ %& 0.767 & 0.706 &  0.835 & 0.779 & 0.745 & 0.764   \\
% PGExplainer \\ %& 0.977 & 0.730 & 0.843 & 0.924& 0.879 &0.964 \\
% PGM-Explainer  &  0.474 & 0.415 & 0.614  & 0.493 & 0.479 & 0.475  \\
% ConveX &  &  &  & &  & &\\
% our & 0.998 & 0.925 & 0.966 & 0.877 & 0.777 & 0.996 \\

% \bottomrule
% \end{tabular}}
% %\end{adjustbox}
% \label{tab:result_exp}
% \end{table}


\begin{table}[p]
\centering
\caption{Performance (AUC) of edge-level explanations for real-world datasets, measured over 500 test nodes.}
%\adjustbox{width=0.48\textwidth}
{
\begin{tabular}{cl|c|c}
\toprule
%&\multicolumn{1}{c}{} &\multicolumn{1}{c}{Bitcoin-Alpha} & \multicolumn{1}{c}{Bitcoin-OTC}\\
\textbf{Model} & \textbf{Explainer} &  \textbf{Bitcoin-Alpha}  &  \textbf{Bitcoin-OTC}\\ \hline
\multirow{5}{*}{\shortstack{GCN \\ (3-hop)}}

& GNNex  &  $94.0$ & $97.3$ \\
& PGEx   & $59.7$ & $54.2$ \\
& PGMEx  & $75.8$ & $52.7$ \\
& DnX  & \textcolor{light}{$\bm{97.1}$} &\textcolor{light}{$\bm{98.1}$}\\
& FastDnX  & \textcolor{royal}{$\bm{97.3}$} & \textcolor{royal}{$\bm{99.1}$} \\

\midrule \midrule 
\multirow{5}{*}{\shortstack{ARMA \\ (3-hop)}}
& GNNex  &  $79.6$ & $92.3$ \\
& PGEx   & $47.8$ & $45.4$ \\
& PGMEx  & $76.2$ & $73.8$ \\
& DnX  & \textcolor{light}{$\bm{96.4}$} &\textcolor{light}{$\bm{98.7}$} \\
& FastDnX  & \textcolor{royal}{$\bm{97.4}$} &\textcolor{royal}{$\bm{99.1}$} \\

\midrule \midrule
\multirow{5}{*}{\shortstack{GATED \\ (3-hop)}}

& GNNex  & $92.5$ & $93.6$ \\
& PGEx   & $34.7$ & $33.7$ \\
& PGMEx  & $86.3$ & $83.1$ \\
& DnX  & \textcolor{light}{$\bm{96.3}$} & \textcolor{light}{$\bm{98.6}$} \\
& FastDnX  & \textcolor{royal}{$\bm{97.2}$} & \textcolor{royal}{$\bm{99.1}$} \\

\midrule \midrule 
\multirow{3}{*}{\shortstack{GIN \\ (3-hop)}} 
% & GNNex  &  - & - \\
% & PGEx   & - & - \\
& PGMEx  & $52.8$ & $76.1$ \\
& DnX  &  \textcolor{light}{$\bm{96.4}$} &\textcolor{light}{$\bm{98.7}$} \\
& FastDnX  & \textcolor{royal}{$\bm{97.5}$} &\textcolor{royal}{$\bm{98.9}$} \\

\bottomrule
\end{tabular}}

\label{tab:result_exp_bitcoin_edges}
\end{table}




\vspace{-1pt}
\paragraph{More results for node-level explanations.}
\autoref{tab:result_exp_bitcoin_node_others} shows results for ARMA, GATED, and GIN. As observed in \autoref{tab:result_exp_bitcoin_node} for GCNs, DnX FastDnX are the best methods.


\begin{table}[htb]
\centering
\caption{Performance (average precision) of node-level explanations for real-world datasets, measured over 500 test nodes.}
\adjustbox{width=0.48\textwidth}{
\begin{tabular}{cl|ccc|ccc}
\toprule
&\multicolumn{1}{c}{} &\multicolumn{3}{c}{\textbf{Bitcoin-Alpha}} & \multicolumn{3}{c}{\textbf{Bitcoin-OTC}}\\
\midrule

\textbf{Model} & \textbf{Explainer} &    \textbf{top 3} & \textbf{top 4} &\textbf{ top 5}  &  \textbf{top 3} & \textbf{top 4} & \textbf{top 5}\\ \midrule


\multirow{5}{*}{\shortstack{ARMA \\ (3-hop)}} 
& GNNEx  &  $80.9$ & $78.9$ & $74.5$  & $73.2$  & $69.5$ & $64.0$\\
& PGEx   & $72.5$ &  $67.8$ & $65.0$ &  $69.7$ & $68.7$ & $61.45$ \\
& PGMEx  & $73.8$ & $66.4$ & $58.3$ & $69.1$ & $62.8$ &  $54.9$\\
\cmidrule{2-8}
& DnX  & \textcolor{royal}{$\bm{95.0}$} & \textcolor{royal}{$\bm{90.1}$} & \textcolor{royal}{$\bm{84.9}$} & \textcolor{royal}{$\bm{93.6}$} & \textcolor{royal}{$\bm{88.1}$} & \textcolor{royal}{$\bm{83.2}$} \\
& FastDnX  & \textcolor{light}{$\bm{90.2}$} & \textcolor{light}{$\bm85.8$} & \textcolor{light}{$\bm81.0$} & \textcolor{light}{$\bm{87.7}$} & \textcolor{light}{$\bm{83.3}$} & \textcolor{light}{$\bm{79.2}$} \\

\midrule \midrule 

\multirow{5}{*}{\shortstack{GATED \\ (3-hop)}} 
& GNNEx  &  $80.1$ & $75.1$ & $769.6$  & $75.9$  & $70.9$ & $66.0$\\
& PGEx   & $75.4$ &  $74.8$ & $67.8$ &  $72.5$ & $70.5$ & $65.1$ \\
& PGMEx  & $80.2$ & $76.5$ & $72.4$ & $77.5$ & $72.1$ &  $67.3$\\
\cmidrule{2-8}

& DnX  & \textcolor{royal}{$\bm{94.4}$}  & \textcolor{royal}{$\bm{89.6}$} & \textcolor{royal}{$\bm{84.4}$} & \textcolor{royal}{$\bm{93.4}$} & \textcolor{royal}{$\bm{88.9}$} & \textcolor{royal}{$\bm{83.5}$}\\
& FastDnX  & \textcolor{light}{$\bm{89.6}$} & \textcolor{light}{$\bm{85.3}$} & \textcolor{light}{$\bm{80.0}$} & \textcolor{light}{$\bm{88.0}$} & \textcolor{light}{$\bm{83.4}$} & \textcolor{light}{$\bm{79.0}$} \\




\midrule \midrule
\multirow{3}{*}{\shortstack{GIN \\ (3-hop)}} 
& PGM-Ex  & $58.7$ & $49.2$ & $40.8$ & $57.6$ & $47.8$ &  $40.2$\\
\cmidrule{2-8}

& DnX  & \textcolor{royal}{$\bm{94.3}$} & \textcolor{royal}{$\bm{88.9}$} & \textcolor{royal}{$\bm{83.2}$} & \textcolor{royal}{$\bm{94.0}$} & \textcolor{royal}{$\bm{89.2}$} & \textcolor{royal}{$\bm{83.6}$} \\
& FastDnX  & \textcolor{light}{$\bm{85.0}$} & \textcolor{light}{$\bm{80.4}$} & \textcolor{light}{$\bm{74.6}$} & \textcolor{light}{$\bm{82.6}$} & \textcolor{light}{$\bm{77.1}$} & \textcolor{light}{$\bm{71.3}$} \\


\bottomrule
\end{tabular}}

\label{tab:result_exp_bitcoin_node_others}
\end{table}



