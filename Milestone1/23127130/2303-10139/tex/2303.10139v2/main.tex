\documentclass[twoside]{article}

\usepackage[accepted]{aistats2023}
\special{papersize = 8.5in, 11in}
\setlength{\pdfpageheight}{11in}
\setlength{\pdfpagewidth}{8.5in}
% \usepackage{aistats2023}
\usepackage[round]{natbib}
\renewcommand{\bibname}{References}
\renewcommand{\bibsection}{\subsubsection*{\bibname}}

\usepackage{algpseudocode}
\usepackage{algorithm}
\usepackage[usenames,dvipsnames]{xcolor} 
\usepackage{float}
\usepackage{listings}
\usepackage{enumitem}

\newcommand{\pc}{PC}

\newcommand{\inputraw}{x}
\newcommand{\bow}{\textbf{x}}
\newcommand{\vocabulary}{\mathcal{V}}
\newcommand{\bowlen}{N}

\newcommand{\assignment}{\mathsf{A}}

\newcommand{\corpus}{D}
\newcommand{\corpusdoc}{\bow}
\newcommand{\corpusdoclen}{\left|\corpusdoc\right|}

\newcommand{\topicwordprior}{\beta}
\newcommand{\topicdocumentprior}{\alpha}

\newcommand{\pdf}{\text{PDF}}
\newcommand{\submission}{\bow}
\newcommand{\submissions}{\mathcal{S}}

\newcommand{\reviewersset}{\mathcal{R}}
\newcommand{\reviewersubset}{R}
\newcommand{\surroundingreviewers}{R}
\newcommand{\reviewer}{r}
\newcommand{\archive}{A}

\newcommand{\reviewerload}{\mathsf{L}_{\reviewer}}
\newcommand{\paperload}{\mathsf{L}_{\submission}}

\newcommand{\select}{\text{sel}}
\newcommand{\reject}{\text{rej}}

\newcommand{\requestedreviewers}{\reviewersubset_{\select}}
\newcommand{\rejectedreviewers}{\reviewersubset_{\reject}}

\newcommand{\bid}{b}


\newcommand{\reviewerwindow}{\omega}
\newcommand{\revieweroffset}{\upsilon}
\newcommand{\reviewerwordsmax}{\nu}

\newcommand{\reviewerwordsmass}{Q}
\newcommand{\reviewerwords}{\hat{\topicworddist}}
\newcommand{\reviewertopics}{\tau}


\newcommand{\topic}{t}
\newcommand{\topics}{T}
\newcommand{\topicssubset}{\tau}
\newcommand{\topicspace}{\mathcal{T}}

\newcommand{\stepsize}{k}
\newcommand{\nosuccessors}{M}

\newcommand{\loss}{\ell}

\newcommand{\cardinalitytopicssubset}{k}
\newcommand{\cardinalitytopics}{\left| \topics \right|}

\newcommand{\topicsubsetmarginprob}{\lambda}
\newcommand{\topicsubsetminprob}{\nu}

\newcommand{\extractor}{\Phi}
\newcommand{\topicextractor}{\Gamma}
\newcommand{\topicworddist}{\phi}
\newcommand{\topicworddisthat}{\hat{\topicworddist}}
\newcommand{\topicdocumentdist}{\theta}
\newcommand{\ldadocument}{d}
\newcommand{\word}{w}

\newcommand{\prob}{P}

\newcommand{\beamwidth}{B}
\newcommand{\maxitr}{I}
\newcommand{\clusters}{C}
\newcommand{\switches}{S}

\newcommand{\attackbudgetscale}{\sigma}

\newcommand{\modifications}{\delta}
\newcommand{\modificationsmannorm}{{\left| \left| \modifications \right| \right|}_1}
\newcommand{\modificationsinfnorm}{{\left| \left| \modifications \right| \right|}_\infty}

\newcommand{\maxmannorm}{L_1^\text{max}}
\newcommand{\maxinfnorm}{L_\infty^\text{max}}

\newcommand{\margin}{\gamma}

\newcommand{\submissionpdf}{\submission}

% Everything related with problem space
\newcommand{\Dom}{\ensuremath{\mathcal{Z}}\xspace}
\newcommand{\F}{\ensuremath{\mathcal{F}}\xspace}
\newcommand{\preprocessing}{\ensuremath{\rho}\xspace}
\newcommand{\inputpdf}{\ensuremath{z}\xspace}

% Code transformations, search strategy
\newcommand{\transformation}{\ensuremath{\omega}\xspace}
\newcommand{\transformations}{\ensuremath{\Omega}\xspace}
\usepackage{mathtools}
\usepackage{amsthm, amssymb}
\newcommand{\propositionautorefname}{Proposition}
% \renewcommand{\sectionautorefname}{Section}
\newcommand{\first}[1]{\mathbf{#1}}
\newcommand{\second}[1]{\underline{#1}}
\newcommand{\RomanNumeralCaps}[1]{\MakeUppercase{\romannumeral #1}}

% \newtheorem*{remark}{Remark}
\newtheorem{definition}{Definition}
\newtheorem{lemma}{Lemma}
\newtheorem{proposition}{Proposition}
\newtheorem{theorem}{Theorem}
\newtheorem{corollary}{Corollary}
\newtheorem{remark}{Remark}
\providecommand*\lemmaautorefname{Lemma}
\usepackage{multirow}

\definecolor{royal}{HTML}{1560BD}
\definecolor{light}{HTML}{1ea2a1}

\usepackage{hyperref}
\hypersetup{
    linkbordercolor=Melon,
    urlbordercolor=magenta,
    citebordercolor=Dandelion,
}

\begin{document}

\twocolumn[

\aistatstitle{Distill n' Explain: explaining graph neural networks using simple surrogates}

\aistatsauthor{ Tamara Pereira$^1$, Erik Nascimento$^1$, Lucas E. Resck$^2$, Diego Mesquita$^2$, Amauri Souza$^{1,3}$}


\aistatsaddress{ $^1$Federal Institute of Cear\'a $^2$Getulio Vargas Foundation $^3$Aalto university } ]

% \aistatsauthor{  Tamara Pereira \And Erik Nascimento \And  Lucas E. Resck \And Diego Mesquita \And Amauri Souza}

% \aistatsaddress{ Federal Institute of Cear\'a \And Federal Institute of Cear\'a  \And  Getulio Vargas Foundation \And Aalto university \And Federal Institute of Cear\'a } ]



\begin{abstract}
Explaining node predictions in graph neural networks (GNNs) often boils down to finding graph substructures that preserve predictions. Finding these structures usually implies back-propagating through the GNN, bonding the complexity (e.g., number of layers) of the GNN to the cost of explaining it. This naturally begs the question: \emph{Can we break this bond by explaining a simpler surrogate GNN?} To answer the question, we propose \emph{Distill n' Explain} (DnX). First, DnX learns a surrogate GNN via \emph{knowledge distillation}. Then, DnX extracts node or edge-level explanations by solving a simple convex program. We also propose FastDnX, a faster version of DnX that leverages the linear decomposition of our surrogate model. Experiments show that DnX and FastDnX often outperform state-of-the-art GNN explainers while being orders of magnitude faster. Additionally, we support our empirical findings with theoretical results linking the quality of the surrogate model (i.e., distillation error) to the faithfulness of explanations. 
\end{abstract}

\section{Introduction}

Graph neural networks (GNNs) \citep{Gori2005,scarselli2009} have become the pillars of representation learning on graphs. Typical GNNs resort to message passing on input graphs to extract meaningful node/graph representations for the task at hand. Despite the success of GNNs in many domains \citep{antibiotic_design,Gilmer2017,recommendersystems,ComplexPhysics}, their architectural design often results in models with limited interpretability. This naturally makes it hard to diagnose scenarios in which GNNs are fooled by confounding effects or align poorly with expert knowledge.

To mitigate this lack of interpretability, a popular strategy is to use post-hoc explanation methods \citep{Ribeiro2016,shap2017,Hima2022, Hima2022b, GraphLIME}. The idea is to increase model transparency by highlighting input/model elements that are particularly important for predictions, helping users to understand what is happening under the hood. 


There has been a recent outbreak of methods for explaining GNNs \citep{Yuan2022}. 
%
Although GNN explanations can come in different flavors \citep{GNNexplainer,subgraphx_icml21,Wang2021,Lucic2022,xgnn_kdd20}, they usually take the form of (minimal) substructures of input graphs that are highly influential to the prediction we want to explain. The seminal work of \citet[GNNExplainer]{GNNexplainer} proposes learning a \emph{soft} mask to weigh graph edges. To find meaningful masks, GNNExplainer maximizes the mutual information between the GNN predictions given the original graph and the masked one. To alleviate the burden of optimizing again whenever we want to explain a different node, \citet[PGExplainer]{PGExplainer} propose using node embeddings to parameterize the masks, i.e., amortizing the inference. 
Nonetheless, GNNExplainer and PGExplainer impose strong assumptions on our access to the GNN we are trying to explain. The former assumes we are able to back-propagate through the GNN. The latter further assumes that we can access hidden activations of the GNN. \citet[PGMExplainer]{PGMExplainer} relieve these assumptions by approximating the local behavior of the GNN with a probabilistic graphical model (PGM) over components, which can be used to rank the relevance of nodes and edges. On the other hand, getting explanations from PGMExplainer involves learning the structure of a PGM, and may not scale well.

In this work, we adopt the same black-box setting of \citet{PGMExplainer} but severely cut down on computational cost by extracting explanations from a \emph{global} surrogate model. In particular, we propose \emph{Distill n' Explain} (DnX). DnX uses knowledge distillation to learn a simple GNN $\Psi$, e.g. simple graph convolution~\citep[SGC]{SGC}, that mimics the behavior of the GNN $\Phi$ we want to explain. 
Then, it solves a simple convex program to find a mask that weighs the influence of each node in the output of $\Psi$.
%
We also propose FastDnX, a variant of DnX that leverages the linear nature of our surrogate to speed up the explanation procedure.
%
Notably, we only require evaluations of $\Phi$ to learn the surrogate $\Psi$ and, after $\Psi$ is fixed, we can use it to explain any node-level prediction. To back up the intuition that explaining a surrogate instead of the original GNN is a sensible idea, we provide a theoretical result linking the distillation quality to the faithfulness of our explanations.

Experiments on eight popular node classification benchmarks show that DnX and FastDnX often outperform GNN-, PG-, and PGM-Explainers. We also demonstrate that both DnX and FastDnX are much faster than the competitors. Remarkably, FastDnX presents a speedup of up to $65K\times$ over GNNExplainer. 
%
Finally, we discuss the limitations of current benchmarks and show that explainers capable of leveraging simple inductive biases can ace them.

\noindent\textbf{Our contributions} are three-fold:
\begin{enumerate}[itemsep=0pt]
    \item we propose a new framework for GNN explanations that treats  GNNs as black-box functions and hinges on explaining a simple surrogate model obtained through knowledge distillation;   
    \item we provide theoretical bounds on the quality of explanations based on these surrogates, linking the error in the distillation procedure to the faithfulness of the explanation;
    \item we carry out extensive experiments, showing that our methods outperform the prior art while running orders of magnitude faster.
\end{enumerate}


\section{Method}
\label{sec:method}

% \ml{``Inconsistent'' to ``large variation''}

% In this section, we propose our methods based on the observations in Section \ref{sec:motivation}.
In this section, we propose two techniques to further enhance the strong baseline to capture the variation of activation distributions better.
We first introduce spatial re-scaling to adapt the network to pixel-to-pixel variation.
We then propose channel-wise shifting and re-scaling to better capture the channel-to-channel variation.
Meanwhile, as both of the two methods are image-dependent, the image-to-image variation can be captured naturally.
By combining the two methods with our strong baseline, we build our enhanced BNN for SR, named EBSR.

% Because the activation distributions among pixels, channels and images have large variations \red{**are highly inconsistent} in SR networks, we introduce spatial re-scaling to adapt to pixel-wise variations and channel shift and re-scaling to adapt to channel-wise variations. And both of them are image-dependent to adapt to image-wise variations, which means during inference our network re-scales and shifts the distributions of activations flexibly for different input images. Based on these methods, we build an enhanced binary neural network for image super-resolution (EBSR).

% According to [3], the difference of activation magnitudes indicates different scaling factors are needed for each pixel.

\subsection{Spatial Re-scaling}
% It is better to use different scaling factors for different pixels to reduce the quantization error and retain more detailed information for image super-resolution. 

% \ml{In the main method, we do not need to introduce the previous works but can focus on introducing our own method. Channel rescaling in Real-to-binary Net is not relevant in this context.}

% Re-scaling the output of binary convolutions was proposed at the birth of BNN in XNOR-Net \cite{rastegari2016xnor} to reduce quantization error and improve accuracy for image classification tasks.
% It is computed as below:
% \begin{equation}
% \mathcal{A} * \mathcal{W} \approx(\operatorname{sign}(\mathcal{A}) \circledast \operatorname{sign}(\mathcal{W})) \odot \mathcal{K} \alpha
% \label{eq:xnor-net rescale}
% \end{equation}
% where $\circledast$ denotes the binary convolution and $\odot$ denotes the element-wise multiplication.
% $\mathcal{A}$, $\mathcal{W}$, $\alpha$, and $\mathcal{K}$ denote the activation, weight, weight scaling factor, and activation scaling factor, respectively.
%  Later in XNOR-Net++ \cite{bulat2019xnor}, Bulat et al. fuse the activation and weight scaling factors into a single one that is learned end-to-end based on gradients and this improves the classification accuracy on ImageNet dataset.

% % It is computed as Eq.~\ref{eq:xnor-net rescale}, where $\circledast$ denotes 
% %  the binary convolution and $\odot$ denotes the element-wise multiplication. The binary convolution of $\mathcal{A}$ and $\mathcal{W}$ is rescaled by the weight scaling factor $\alpha$ and the activation scaling factor $\mathcal{K}$, both of which are calculated analytically.


% \zc{Similarly, you should explain the meaning of A, W and the operators $\circledast$ in the formula}
% Then in Real-to-binary Net \cite{martinez2020training}, Martinez et al. used a data-driven channel re-scaling module that takes the pre-convolution activations as input to predict the activation scaling factor. Unlike that in XNOR-Net++ \cite{bulat2019xnor}, these scaling factors are not fixed during inference but rather inferred from data. By doing this, they further improved the classification accuracy on ImageNet over XNOR-Net++. 
As is shown in Figure \ref{fig:pixel}, activation distributions have large pixel-to-pixel variation in SR networks
and the difference of activation magnitudes indicates different scaling factors are preferred for different pixels.
Inspired by \cite{martinez2020training}, we propose spatial re-scaling to better adapt the network to the spatial variation
of activation distributions in SR networks.
% fit the various pixel-wise distributions in SR networks.
We take the real-valued activations $A$ before convolution as input and predict pixel-wise scaling factors $S(A)$, which re-scale the binary convolution output. Spatial re-scaling process can be formulated as follows:
\begin{equation}
A * W \approx(\operatorname{sign}(A) \circledast \operatorname{sign}(W)) \odot \alpha \odot S(A)
\label{eq:spatial rescale}
\end{equation}
where $\circledast$ denotes 
the binary convolution and $\odot$ denotes the element-wise multiplication. $A$, $W$, $\alpha$, and $S\left(A\right)$ denote real-valued activations, weights, the scaling factor of weights, and the spatial-wise scaling factor of activations respectively. $S\left(A\right) \in \mathbb{R}^{1\times H\times W}$ can be calculated with a convolution and a sigmoid function.
% as $\sigma\left( CONV\left(A\right)\right)$. 
As shown in Figure \ref{fig:method}(a), real-valued activations first go through a convolution layer,
which has an input channel of $C$ and an output channel of 1, 
and then pass through a sigmoid function to produce the scaling factors $S(A)$ along the spatial dimension.
During inference, the scaling factor will change dynamically according to different input feature maps.
By re-scaling binary convolution output using $S(A)$, we can reduce the quantization error and the original pixel-wise information in FP activation
will be preserved much better.
Spatial re-scaling leads to a large PSNR improvement of 0.24 dB (from 30.30 dB to 31.54 dB) on Set5 and 0.22 dB (from 25.09 dB to 25.31 dB)
on Urban100 compared with our strong baseline. 

\subsection{Channel-wise Shifting and Re-scaling}

\begin{table}[!tb]
\centering
\caption{Comparison between whether to fuse channel-wise shifting and re-scaling or not based on our baseline with spatial re-scaling. }
\label{tab:fusing}

\scalebox{0.65}{
\begin{tabular}{c|cc|cc|cc}
\hline
\multirow{2}{*}{Method}     & \multirow{2}{*}{OPs} & \multirow{2}{*}{Params} & \multicolumn{2}{c|}{Set5} & \multicolumn{2}{c}{Urban100} \\ \cline{4-7} 
                            &                      &                         & PSNR        & SSIM        & PSNR          & SSIM         \\ \hline
Baseline + spatial re-scale & 2.16G                & 0.05M                   & 31.54       & 0.883       & 25.31         & 0.759        \\
+ channel-wise shift and re-scale             & 2.34G                & 0.09M                   & 31.61       & 0.885       & 25.35         & 0.761        \\
+ w/ fusing                   & 2.27G                & 0.08M                   & \textbf{31.64}       & \textbf{0.885}       & \textbf{25.36}         & \textbf{0.761}        \\ \hline
\end{tabular}
}
\end{table}

In SR networks, activation distributions exhibit larger channel-to-channel variation (Figure \ref{fig:chl}).
Both the mean and magnitude of the activation distributions vary significantly across channels.
% Thus we use channel-wise shifting and re-scaling to adapt to various channel-wise distributions. 
\cite{martinez2020training} has proposed the data-driven channel re-scaling, 
but our method differs from them in further introducing data-driven thresholds to handle the channel-wise variation of both mean and magnitude.
Since the blocks to generate the scaling factors and thresholds are very similar, we further propose to fuse them into one module.
% and fusing channel-wise shifting and re-scaling into one module.
We evaluate the effect of fusing the two blocks in Table \ref{tab:fusing}.
With channel-wise shifting and re-scaling fused, our models have fewer operations and parameters overhead and slightly higher performance.

For the specific process, we take the real-valued activations as input and predict different thresholds and scaling factors for each channel. They are also image dependent, e.g., $\beta_{i}$ in Eq.\ref{eq:act_binarize} is no longer fixed during inference but generated according to different input feature maps. Channel-wise shifting and re-scaling can be formulated as follows:
\begin{equation}
A * W \approx(\operatorname{sign}(A-C_s(A)) \circledast \operatorname{sign}(W)) \odot \alpha \odot C_r(A)
\label{eq:channel-wise_shift_and_rescale}
\end{equation}
where $\circledast$ denotes 
the binary convolution and $\odot$ denotes the element-wise multiplication. $C_s(A), C_r(A) \in \mathbb{R}^{C\times1\times1}$ denote the channel-wise threshold and scaling factor, respectively. 
We show the block diagram in Figure \ref{fig:method}(b).
The real-valued input feature map is first squeezed to a ${C\times1\times1}$ vector by a global average pooling (GAP) layer.
The subsequent fully connected layers and ReLU learn the channel-wise information and output a ${2C\times1\times1}$ vector.
Then the ${2C\times1\times1}$ vector is split into two ${C\times1\times1}$ vectors.
We use the first $C$ channels as the channel-wise bias and pass the last $C$ channels through a sigmoid layer 
as the channel-wise scaling factor, which are used to shift the real-valued activations and re-scale the binary convolution output, respectively. 


% \ml{We can mention previously, channel-wise re-scale has been proposed. We propose to fuse them. Add the comparison between fuse v.s. no fuse.}

\begin{figure}[!tbp]%
  \centering
    \includegraphics[width=0.4\textwidth]{fig/methods.png}
  
% \subfloat[channel-wise shifting\&re-scale]{
%     \label{subfig:channel-wise shifting and re-scale}
%     \includegraphics[width=0.2\textwidth]{fig/chl shift and rescale.png}
%   }

  \caption{Block diagram for spatial re-scaling, and channel-wise shifting and re-scaling.} 
  % Input A is the real-valued activation tensor and C, H, and W denote its dimension. GAP stands for global average pooling. The reduction ratio r is set to 16 for a better trade-off between the performance and the number of operations and parameters.}
  \label{fig:method}
\end{figure}


\subsection{Network Structure}

Combining the spatial re-scaling and the channel-wise shifting and re-scaling methods, we construct the enhanced convolution layer (E-Conv).
Then we build our EBSR model based on E-Conv.
In Figure \ref{fig:E-conv}, we compare the binary convolution layer used in the baseline network and our proposed E-Conv.
We use spatial and channel-wise scaling factors to re-scale the binary convolution output,
and use channel-wise shifting to learn appropriate thresholds for each channel before binarization.
The scaling factors and threshold used in E-Conv are learnable and depend on the real-valued input activations.
In this way, our proposed EBSR can adapt to pixel-to-pixel, channel-to-channel, and image-to-image variations
to reduce the large binarization error and preserve more details.
% In this way, our proposed E-Conv reduces the large quantization error caused by binarization and keeps the original information of input feature maps to a large extent.


\begin{figure}[!tb]%
  \centering

    \includegraphics[width=0.5\textwidth]{fig/E-conv.png}

  \caption{Comparison of (a) the binary convolution layer with a skip connection used in our baseline network and (b) the proposed E-Conv.}
  \label{fig:E-conv}
\end{figure}


Figure \ref{fig:network} shows the basic block based on the E-Conv and our EBSR composed of the basic blocks. Following existing works, the convolution layers in the head and tail modules are not binarized. We choose the lightweight EDSR which has 16 basic blocks and 64 channels, and EDSR which has 32 basic blocks and 256 channels as our backbones, which correspond to EBSR-light and EBSR, respectively.

\begin{figure}[!tb]%
  \centering
  {
    \includegraphics[width=0.35\textwidth]{fig/network.png}
  }
  
  \caption{The structure of our proposed EBSR.  Convolution layers in purple are real-valued vanilla 3x3 convolutions.}
  \label{fig:network}
\end{figure}

% \newpage
\section{Analysis}
    In this section, we discuss the theoretical and computational aspects of our method.
    We first provide theoretical results supporting the hypothesis that good explanations of a global surrogate $\Psi$ also characterize good explanations of $\Phi$ --- in terms of faithfulness.
    Then, we discuss the convexity of the optimization problem DnX solves to extract explanations.
    We delegate proofs to the \autoref{append:proofs}.


        \label{sec:faith}

        Let $\mathcal{G}_u$ denote the subgraph of $\mathcal{G}$ induced by the $L$-hop neighborhood around node $u$.
        %
        We say an explanation $\mathcal{E}_u$ for a node $u$ is faithful with respect to $\Phi$ if: i) $\Phi$ outputs approximately the same predictions for $u$ regardless of using $\mathcal{E}_u$ to weigh the nodes of $\mathcal{G}_u$ or not; and ii) the same holds under small perturbations of $\mathcal{G}_u$. 
        %
        We can define a perturbation $\mathcal{G}_{u}^\prime$ of $\mathcal{G}_u$  by adding noise to $u$'s features or by randomly rewiring node $u$'s incident edges \citep{agarwal_probing_2022}.
        %
        In this work, we consider perturbations over node features. More precisely, this entails that $V(\mathcal{G}_{u}^\prime) = V(\mathcal{G}_u)$, $E(\mathcal{G}^\prime_u) = E(\mathcal{G}_u)$, and that features are corrupted by noise, i.e., $X^\prime_i = X_i + \epsilon_i$ for $i \in V(\mathcal{G}_u)$ and $\epsilon_i \in \mathbb{R}^{d}$. 
        

        \begin{definition}[Faithfulness]
            \label{def:faithfulness}
            Given a set $\mathcal{K}$ of perturbations of $\mathcal{G}_u$, an explanation $\mathcal{E}_u$ is \textit{faithful} to a model $f$ if
            \[\frac{1}{|\mathcal{K}| + 1} \sum_{ \mathcal{G}_{u}^\prime \in \mathcal{K} \cup \{\mathcal{G}_u\}} \left\lVert f(\mathcal{G}_{u}^\prime)- f(t(\mathcal{G}_{u}^\prime, \mathcal{E}_u))\right\rVert_2 \le \delta,\]
            where $\mathcal{G}_{u}^\prime$ is a possibly perturbed version of $\mathcal{G}_u$, $t$ is a function that applies the explanation $\mathcal{E}_u$ to the graph $\mathcal{G}_{u}^\prime$, and $\delta$ is a small constant \citep{agarwal_probing_2022}.
        \end{definition}



        \autoref{theo:bound_unfaithfulness} provides an upper bound on the  unfaithfulness of $\mathcal{E}_u$ with respect to the surrogate model $\Psi$. \autoref{theo:bound_unfaithfulness_2} extends this result to obtain a bound for $\mathcal{E}_u$ with respect to the model we originally want to explain, i.e., $\Phi$.
    
        \begin{lemma}[Unfaithfulness with respect to $\Psi$]
            \label{theo:bound_unfaithfulness}
            Given a node $u$ and a set $\mathcal{K}$ of perturbations, the unfaithfulness of the explanation $\mathcal{E}_u$ with respect to the prediction $Y_u^{(\Psi_\Theta)}$ of node $u$ is bounded as follows:
            \[\frac{1}{|\mathcal{K}| + 1} \sum_{\substack{ \mathcal{G}^\prime_{u} \in \\ \mathcal{K} \cup \{\mathcal{G}_u\}}} \left\lVert \Psi(\mathcal{G}_{u}^\prime)- \Psi(t(\mathcal{G}_{u}^\prime, \mathcal{E}_u))\right\rVert_2 \le \gamma \left\lVert \underset{\mathcal{E}_u}{\Delta} \widetilde A_u^L \right\rVert_2,\]
            where $\mathcal{G}_{u}^\prime$ is a possibly perturbed version of $\mathcal{G}_u$, $t$ is a function that applies the explanation $\mathcal{E}_u$ to the graph $\mathcal{G}_{u}^\prime$, $\gamma$ is a constant that depends on the model weights $\Theta$, node features $X$, and perturbation $\epsilon$. Furthermore, $\underset{\mathcal{E}_u}{\Delta} \widetilde A_u^L$ is the $u$-th row of the difference of the powered, normalized adjacency matrix $\widetilde A^L$ before and after applying the explanation $\mathcal{E}_u$.
        \end{lemma}
    
        \begin{proof}[Sketch of the proof.]
            We first show that
            \[\left\lVert \Psi(\mathcal{G}_{u})- \Psi(t(\mathcal{G}_{u}, \mathcal{E}_u))\right\rVert_2 \le \lVert (X \Theta)^\intercal \rVert_2 \left\lVert \widetilde A_u^L - \widetilde E_u^L \right\rVert_2\]
            by using Lipschitz continuity of the $\text{softmax}$ function and the compatibility property of the $L_2$ matrix norm.
            We repeat for $\mathcal{G}_u^\prime \in \mathcal{K}$, take the mean in $\mathcal{K} \cup \{\mathcal{G}_u\}$ and isolate $\left\lVert \underset{\mathcal{E}_u}{\Delta} \widetilde A_u^L \right\rVert_2 = \left\lVert \widetilde A_u^L - \widetilde E_u^L \right\rVert_2$.
            The complete proof is available in \autoref{append:proofs}.
        \end{proof}

        \begin{theorem}[Unfaithfulness with respect to $\Phi$]
            \label{theo:bound_unfaithfulness_2}
            Under the same assumptions of \autoref{theo:bound_unfaithfulness} and assuming the $L_2$ distillation error is bounded by $\alpha$, the unfaithfulness of the explanation $\mathcal{E}_u$ for the original model $\Phi$'s node $u$ prediction is bounded as follows:
            \begin{equation*}
                \begin{split}
                    \frac{1}{|\mathcal{K}| + 1} \sum_{\substack{\mathcal{G}_u^\prime \in \\ \mathcal{K} \cup \{\mathcal{G}_u\}}} \left\lVert \Phi(\mathcal{G}_{u}^\prime)- \Phi(t(\mathcal{G}_{u}^\prime, \mathcal{E}_u))\right\rVert_2 \le\ &\gamma \left\lVert \underset{\mathcal{E}_u}{\Delta} \widetilde A_u^L \right\rVert_2 \\
                    &+ 2\alpha.
                \end{split}
            \end{equation*}
        \end{theorem}

        Note that \autoref{theo:bound_unfaithfulness_2} establishes a bound on faithfulness that depends directly on the distillation error $\alpha$. Importantly, when $\Psi$ is a perfect approximation of $\Phi$, we retrieve upper-bound on the RHS of \autoref{theo:bound_unfaithfulness}.

        We note that Theorem 1 by \citet{agarwal_probing_2022} covers an upper bound for the unfaithfulness of GNN explanation methods. However, they do not cover the case in which the explanation is a (weighted) subset of nodes in the $L$-hop neighborhood of $u$, as in our method.

        
        For completeness, we also extend \autoref{theo:bound_unfaithfulness} and \autoref{theo:bound_unfaithfulness_2} to account for the (very often) probabilistic nature of the noise, i.e., for the case in which $\epsilon_i$ are random variables.
    
        \begin{lemma}[Probability bound on unfaithfulness \emph{w.r.t.} $\Psi$]
            \label{theo:prob_bound_unfaithfulness}
            Given a node $u$ and a set $\mathcal{K}$ of perturbations and assuming the perturbations are i.i.d. with distribution $\epsilon_i~\sim~\mathcal{N}(0, \sigma^2)$, the unfaithfulness of the explanation $\mathcal{E}_u$ with respect to the prediction $Y_u^{(\Psi_\Theta)}$ of node $u$ is bounded in probability as follows:
            \begin{equation*}
                \begin{split}
                    \mathbb{P}\left(
                    \frac{1}{|\mathcal{K}| + 1} \sum_{\substack{ \mathcal{G}^\prime_{u} \in \\ \mathcal{K} \cup \{\mathcal{G}_u\}}} \left\lVert \Psi(\mathcal{G}_{u}^\prime)- \Psi(t(\mathcal{G}_{u}^\prime, \mathcal{E}_u))\right\rVert_2 \le \xi \right) \ge \\
                    \ge F_{\chi_{|\mathcal{K}|nd}^2}\left(\frac{\xi - \gamma_1 \left\lVert \underset{\mathcal{E}_u}{\Delta} \widetilde A_u^L \right\rVert_2}{\gamma_2 \left\lVert \underset{\mathcal{E}_u}{\Delta} \widetilde A_u^L \right\rVert_2 \sigma} - |\mathcal{K}|\right)
                \end{split}
            \end{equation*}
            where $\gamma_1$ is a constant that depends on the model weights $\Theta$ and node features $X$, $\gamma_2$ is a constant that depends on the model weights $\Theta$, and $F_{\chi_{|\mathcal{K}|nd}^2}$ is the c.d.f. of a chi-square r.v. with $|\mathcal{K}| \times n \times d$ degrees of freedom where $(n, d)$ are the row- and column-wise dimensions of $X$.  
        \end{lemma}
        
        
        \begin{theorem}[Probability bound on unfaithfulness \emph{w.r.t.} $\Phi$]
            Under the same assumptions of \autoref{theo:prob_bound_unfaithfulness} and assuming the $L_2$ distillation error is bounded by $\alpha$, the unfaithfulness of the explanation $\mathcal{E}_u$ for the original model $\Phi$'s node $u$ prediction is bounded in probability as follows:
            \begin{equation*}
                \begin{split}
                    \mathbb{P}\left(
                    \frac{1}{|\mathcal{K}| + 1} \sum_{\substack{ \mathcal{G}^\prime_{u} \in \\ \mathcal{K} \cup \{\mathcal{G}_u\}}} \left\lVert \Phi(\mathcal{G}_{u}^\prime)- \Phi(t(\mathcal{G}_{u}^\prime, \mathcal{E}_u))\right\rVert_2 \le \xi \right) \ge \\
                    \ge F_{\chi_{|\mathcal{K}|nd}^2}\left(\frac{\xi - \gamma_1 \left\lVert \underset{\mathcal{E}_u}{\Delta} \widetilde A_u^L \right\rVert_2 - 2\alpha}{\gamma_2 \left\lVert \underset{\mathcal{E}_u}{\Delta} \widetilde A_u^L \right\rVert_2 \sigma} - |\mathcal{K}|\right)
                \end{split}
            \end{equation*}
            \label{theo:prob_bound}
        \end{theorem}

In \autoref{theo:prob_bound_unfaithfulness} and \autoref{theo:prob_bound}, when the variance $\sigma^2$ approaches zero, $\xi$ relinquishes its random nature and the probability in the RHS converges to one.
 We note that numerators in the RHS must be non-negative.

Recall DnX/FastDnX's pipeline  involves two steps: model distillation (\autoref{eq:destilador}) and explanation extraction (\autoref{eq:e2}). The former is done only once to learn the surrogate $\Psi$. The latter, however, must be executed for each node whose prediction we want to explain.  Then, gauging the cost of the extraction step may become a genuine concern from a practical point of view, especially for DnX, which implies solving an optimization problem repeatedly. Fortunately, the loss landscape of our extraction problem depends only on the shape of $\Psi$, and not on the original GNN $\Phi$ as in GNNExplainer. Since $\Psi$ is  an SGC, \autoref{eq:e2} is a convex program (\autoref{theo:convexity}) and we reach global optima using, e.g., gradient-based algorithms.
        
        \begin{theorem}[Convexity of DnX]
            \label{theo:convexity}
            The optimization problem of \autoref{eq:e2} is convex.
        \end{theorem}

\section{Additional related works}

\paragraph{Explanations for GNNs.}

The ever-increasing application of GNNs to support high-stake decisions on critical domains \citep{antibiotic_design,Luna2020,Pinion2021} has recently boosted interest in explainability methods for graph models.
%
\citet{Pope2019} first extended classical gradient-based explanation methods for GNNs. 
%
Importantly, \citet{GNNexplainer} introduced GNNExplainer and synthetic benchmarks that have been widely adopted to assess GNN explainers. 
%
Building on parameterized explainers by \citet{PGExplainer}, \citet{Wang2021} proposed ReFine to leverage both global information (e.g., class-wise knowledge) via pre-training and local one (i.e., instance specific patterns) using a fine-tuning process.  
%
\citet{Lucic2022,Bajaj2021} investigated counterfactual explanations for GNNs, aiming to find minimal perturbations to the input graph such that the prediction changes, e.g., using edge deletions.
%
\citet{DEGREE} proposed measuring the contribution of different components of the input graph to the GNN prediction by decomposing the information generation and aggregation mechanism of GNNs.
%
Recently, \citet{Games2022} introduced a structure-aware scoring function derived from cooperative game theory to determine node importance.
%
Explainability methods for GNNs have also been approached through the lens of causal inference \citep{Lin2021,Lin2022}.
%
For a more comprehensive coverage of the literature, we refer the reader to \citet{Yuan2022}.


\paragraph{Knowledge distillation.} Since the pivotal work of \citet{Hinton2015}, condensing the knowledge from  a possibly complex \emph{teacher} model into a simpler \emph{student} surrogate has been an active research topic~\citep[e.g.][]{Vadera, Malinin2020Ensemble, ryabinin2021scaling,Ba2022, Clayer}. Nonetheless, despite numerous works using  distillation in image domains~\citep[e.g.][]{Lamp2017, Arthur, Object}, the distillation of GNNs is still a blooming direction. \citet{destillGCN1} proposed the first method for GNN distillation, using a structure-preserving module to explicitly factor in the topological structure embedded by the teacher.
\citep{GCRD} proposed using contrastive learning to implicitly align the node embeddings of the student and the teacher in a common representation space.
%
\citet{Jing} combined the knowledge of complementary teacher networks into a single student using a dedicated convolutional operator and topological attribution maps.
%
\citet{Mscale} used an attention mechanism to weigh different teachers depending on the local topology of each node.


We present in section~\ref{ssec:faces} an application of PnP-HVAE on face images, using a pretrained state-of-the-art hierarchical VAE. 
Next, we study the application of our framework to natural images. To that end, we introduce  in section~\ref{ssec:patchVDVAE}  a patch hierachical VAE architecture, that is able to model natural images of different resolutions. In section~\ref{ssec:app_nat}, we provide deblurring, super-resolution and inpainting experiments to demonstrate the relevance of the proposed method.

Additional results are presented in Appendix~\ref{app:add}. All experiments can be reproduced using the code available at \url{https://github.com/jprost76/PnP-HVAE}.



\subsection{Face Image restoration (FFHQ)}\label{ssec:faces}
We first demonstrate the effectiveness of PnP-HVAE on highly structured data, by performing face image restoration.
Latent variable generative models can accurately model structured images such as face images \cite{karras2019style,vahdat2020nvae,child2021very,kingma2018glow}, and then be used to produce high quality restoration of such data. 
In our experiments, we use the VDVAE model of~\cite{child2021very}, pre-trained on the FFHQ dataset~\cite{karras2019style}, as our hierarchical VAE prior.
VDVAE has $L=66$ latent variable groups in its hierarchy and generates images at resolution $256\times256$.

We compare PnP-HVAE with the intermediate layer optimization algorithm (ILO)~\cite{daras2021intermediate} that is based on a different class of generative models than HVAE. ILO is a GAN inversion method which optimizes the image latent code along with the intermediate layer representation of a StyleGAN to generate an image consistent with a degraded observation.
We use the official implementation of ILO, along with a StyleGAN2 model~\cite{karras2020analyzing, stylegan2pytorch}, that was trained for 550k iterations on images of resolution $256\times256$ from FFHQ.  
As VDVAE and StyleGAN models are not trained on the same train-test split of FFHQ, we chose to evaluate the methods on a subset of 100 images from the CelebA dataset~\cite{liu2018large}. 
For super-resolution, the degradation model corresponds to the application of a gaussian low-pass filter followed by a $\times 4$ sub-sampling, and the addition of a gaussian white noise with $\sigma=3$.
For the deblurring, we considered motion blur and  gaussian kernels, both with a noise level $\sigma=8$. %

We provide quantitative comparisons in table~\ref{table:comp_ILO}, along with a visual comparison of the results in figure~\ref{fig:face_restoration}.
PnP-HVAE has the best  PSNR and SSIM results for all the considered restoration tasks, while ILO provides better results  for the perceptual distance.
By jointly optimizing the image and its latent variable, PnP-HVAE provides  results that are both realistic and consistent with the degraded observation.
On the other hand,  ILO  only optimizes on an extended latent space. This method generates  sharp and realistic images with better LPIPS scores,   
but the results lack  of consistency with respect to the observation, which explains the overall lower PSNR performance. 






\subsection{PatchVDVAE: a HVAE for natural images}\label{ssec:patchVDVAE}
Available generative models in the literature operate on images of  fixed resolutions and
are either restrained to datasets of limited diversity, or even to registered face images~\cite{kingma2018glow,child2021very, vahdat2020nvae, karras2019style}, or requiring additional class information~\cite{brock2018large, dhariwal2021diffusion, song2020score, luhman2022optimizing}.
Fitting an unconditional model on natural images appears to be a more difficult task, as their resolution can change, and their content is highly diverse.
The complexity of the problem can be reduced by learning a prior model on patches of reduced dimension. 
For image restoration problems, the patch model can be reused on images of higher dimensions~\cite{zoran2011learning,prost2021learning,altekruger2022patchnr}. When the model is a full CNN, the prior on the set of the  patches can  be computed efficiently by applying the network on the full image~\cite{prost2021learning}.

We thus introduce  patchVDVAE, a fully convolutional hierarchical VAE.
Contrary to existing HVAE models whose resolution is constrained by the constant tensor at the input of the top-down block, patchVDVAE can generate images of different resolutions by controlling the dimension of the input latent. 
This amounts to defining a prior on patches whose dimension corresponds to the receptive field of the VAE. A similar model is used for image denoising in~\cite{prakash2021interpretable}.

 
For PatchVDVAE architecture, we use the same bottom-up and top-down blocks as VDVAE~\cite{child2021very}, and replace the constant trainable input in the first top-down block by a latent variable, to make the model fully convolutional (details on the  architecture are given in Appendix~\ref{app:details}). 
The training dataset is composed of $128\times 128$ patches extracted from a combination of DIV2K~\cite{agustsson2017ntire} and Flickr2K~\cite{Lim_2017_CVPR_workshops} datasets.
We perform data augmentation by extracting  patches at $3$ resolutions: HR-images and $\times 2$ and $\times 4$ downscaled images. 
The model is trained for $7.10^5$ iterations with a batch size of $64$. Following the recommendation of~\cite{hazami2022efficient}, we use Adamax optimizer with an exponential moving average and gradient smoothing of the variance.
We set the decoder model to be a gaussian with diagonal covariance, as in~\cite{luhman2022optimizing}.
PatchVDVAE is fully convolutional and can generate images of dimension that are multiples of $64$ as illustrated by
figure~\ref{fig:vdvae}.

\newlength{\patchwidth}
\setlength{\patchwidth}{0.135\columnwidth}
\begin{figure}[!ht]
    \centering
    \begin{subfigure}[t]{.34\columnwidth}\hspace{0.1cm}
        \setlength{\tabcolsep}{0.02pt}
\renewcommand{\arraystretch}{0}
        \begin{tabular}{*{2}{p{1.03\patchwidth}}}
            \includegraphics[width=\patchwidth]{figures_arxiv/patchVDVAE/samples/generated/64x64/setup-5-image-0018.png} &
            \includegraphics[width=\patchwidth]{figures_arxiv/patchVDVAE/samples/generated/64x64/setup-5-image-0016.png} \\
            \includegraphics[width=\patchwidth]{figures_arxiv/patchVDVAE/samples/generated/64x64/setup-5-image-0008.png} &
            \includegraphics[width=\patchwidth]{figures_arxiv/patchVDVAE/samples/generated/64x64/setup-5-image-0019.png}   
        \end{tabular}
    \end{subfigure}\hspace{-0.15cm}
    \begin{subfigure}[t]{.64\columnwidth}
\begin{tabular}{cc}\vspace{-0.1cm}
\includegraphics[width=2\patchwidth]{figures_arxiv/patchVDVAE/samples/generated/256x256/setup-2-image-0009.png}&
        \includegraphics[width=2\patchwidth]{figures_arxiv/patchVDVAE/samples/generated/256x256/setup-2-image-0002.png}\end{tabular}

    \end{subfigure}
    \caption{\label{fig:vdvae} Left: $64\times64$ patches samples from our patchVDVAE model trained on patches from natural images.
    Right: PatchVDVAE is fully convolutional and it can generate images of higher resolution (here: $128\times128$).\vspace{-0.2cm}}
\end{figure}

\subsection{Natural images restoration}\label{ssec:app_nat}
We  evaluate PnP-HVAE on natural image restoration.
For each task, we report the average value of the PSNR, the SSIM, and the LPIPS metrics on $20$ images from the test set of the BSD dataset~\cite{MartinFTM01}.\\


\noindent
{\bf Image deblurring.}
In the experiments, we consider $2$ gaussian kernels and $2$ motion blur kernels from~\cite{levin2009understanding}, with $3$ different noise levels 
$\sigma \in \{2.55, 7.65, 12.75\}$.
As a baseline we consider  EPLL~\cite{zoran2011learning}, which learns a prior on image patches with a gaussian mixture model.
We also compare PnP-HVAE  with PnP-MMO and GS-PnP, $2$ competing convergent Plug-and-Play methods based on CNN denoisers.
PnP-MMO~\cite{pesquet2021learning} restricts the denoiser to be contraction in order to guarantee the convergence of the PnP forward-backard algorithm. GS-PnP~\cite{hurault2022gradient} considers a gradient step denoiser and reaches state-of-the-art performances of non converging methods~\cite{zhang2021plug}.
We set the temperature $\tau$  in our method as $0.95$, $0.8$ and $0.6$ for noise levels $2.55$, $7.65$ and $12.75$ respectively, and we let it run for a maximum of $50$ iterations. 
For the three compared methods we use the official implementations and pre-trained models provided by the respective authors. 
Details on the choice of hyperparameters for the concurrent methods are provided in the Appendix~\ref{app:details}
Figure~\ref{fig:deblurring_bsd} illustrates that our method provides correct deblurring results. 

According to table~\ref{tab:deb}, the performance of PnP-HVAE is between those of EPLL and GS-PnP and it outperforms PnP-MMO for large noise levels.\\

\begin{table}
\begin{center}\footnotesize
    \begin{tabular}{>{\centering}m{.3cm}*{5}{c}}
    $\sigma$ &Method & PSNR$\uparrow$ & SSIM$\uparrow$ & LPIPS$\downarrow$  \\ 
    \hline
    \multirow{4}{*}{\vcell{$2.55$}}
    & PnP-HVAE & $27.75$ & $0.79$ & $0.31$\\
    & GS-PNP \cite{hurault2022gradient} & $\mathbf{29.59}$ & $\mathbf{0.84}$ & $\mathbf{0.22}$\\
    & EPLL \cite{zoran2011learning} & $26.49$ & $0.71$ & $0.36$\\ 
    & PnP-MMO \cite{pesquet2021learning} & $\underbar{29.50}$ & $\underbar{0.83}$ & $\underbar{0.20}$ \\ \hline
    \multirow{4}{*}{\vcell{$7.65$}}
    & PnP-HVAE & $\underbar{26.36}$ & $\underbar{0.72}$ & $\underbar{0.40}$\\
    & GS-PNP \cite{hurault2022gradient} & $\mathbf{27.33}$ & $\mathbf{0.77}$ & $\mathbf{0.31}$\\
    & EPLL \cite{zoran2011learning} & $24.04$ & $0.66$ & $0.45$ \\ 
    & PnP-MMO \cite{pesquet2021learning} & $25.34$ & $0.69$ & $0.34$\\
    \hline
    \multirow{4}{*}{\vcell{$12.75$}}
    & PnP-HVAE & $\underbar{25.12}$ & $\mathbf{0.73}$ & $\underbar{0.47}$\\
    & GS-PNP \cite{hurault2022gradient} & $\mathbf{26.32}$ & $\mathbf{0.73}$ & $\mathbf{0.37}$\\
    & EPLL \cite{zoran2011learning} & $23.28$ & $0.61$ & $0.51$ \\ 
    & PnP-MMO \cite{pesquet2021learning} & $22.42$ & $0.53$& $0.54$ \\
    \hline
    &\vspace*{-.3cm}\\
            \multicolumn{2}{c}{Blur and motion kernels}& \multicolumn{3}{c}{
        \includegraphics*[scale=1]{figures_arxiv/kernels/4.png}\;\includegraphics*[scale=1]{figures_arxiv/kernels/7.png}\;\includegraphics*[scale=1]{figures_arxiv/kernels/9.png}\;\includegraphics*[scale=1]{figures_arxiv/kernels/11.png}} 
    \end{tabular}
        \caption{\label{tab:deb}Comparison  of PnP-HVAE  and other restoration methods on deblurring. Results are averaged on $4$ kernels.\vspace{-0.2cm}}% on image deblurring.}
    \end{center}
\end{table}

\begin{figure}
    
    \begin{subfigure}[h]{\linewidth}
        \centering
        \includegraphics*[width=\columnwidth]{figures_arxiv/deb_s255_k7.pdf}\vspace{-0.1cm}
        \caption{Gaussian blur, $\sigma=2.55$}
    \end{subfigure}
    \begin{subfigure}[h]{\linewidth}
        \centering
        \includegraphics*[width=\columnwidth]{figures_arxiv/deb_s765_k11.pdf}\vspace{-0.1cm}
        \caption{Motion blur, $\sigma=7.65$}
    \end{subfigure}\vspace*{-0.1cm}
    \caption{\label{fig:deblurring_bsd} Natural image deblurring\vspace{-0.1cm}}
\end{figure}

\noindent {\bf Effect of the temperature.}
PnP-HVAE gives control on the temperature of the prior over the latent space.
In figure~\ref{fig:temp_effect}, we illustrate that reducing the temperature increases the strength of the regularization prior. In this example the tuning $\tau=0.7$ produces the best performance.\\
\begin{figure}[!ht]
   
    \includegraphics[width=\columnwidth]{figures_arxiv/demo_temp.pdf}\vspace{-0.15cm}
    \caption{ \label{fig:temp_effect} Effect of the temperature in PnP-VAE on a deblurring problem, with $\sigma=7.65$.\vspace{-0.15cm}}
\end{figure}


\noindent
{\bf Image inpainting.}
Next we consider the task of noisy image inpainting. 
We compose a test-set of 10 images from the validation set of BSD~\cite{MartinFTM01} and we create masks
  by occluding diverse objects of small size in the images. 
A gaussian white noise with $\sigma=3$ is added to the images.
As a comparaison, we still consider GS-PnP and EPLL.
For PnP-HVAE, the temperature is set to $\tau=0.6$, and the algorithm is run for a maximum of $200$ iterations, unless the residual $||\x_{k+1}-\x_k||$ is on a plateau.
We provide on Table~\ref{tab:inpainting_bsd} the distortion metrics with the ground truth, as well as a visual
\begin{table}



\begin{center}
    \begin{tabular}{cccc}
        & PSNR$\uparrow$ & SSIM$\uparrow$ &LPIPS$\downarrow$ \\\hline
        PnP-HVAE  & $\mathbf{29.54}$ & $\mathbf{0.93}$ & $\mathbf{0.06}$\\
        GS-PNP & $28.52$ & $\mathbf{0.93}$ & $0.09$\\
        EPLL & $\underline{29.16}$ & $\mathbf{0.93}$ & $\mathbf{0.06}$\\
    \end{tabular}
    \caption{\label{tab:inpainting_bsd}Quantitative evaluation for inpainting on BSD.}
    \end{center}
\end{table}
comparison on figure~\ref{fig:inpainting_bsd}. 
With its hierarchical structure,  PnP-HVAE outperforms the compared methods. \vspace{0.05cm}



\begin{figure}[!h]
    \includegraphics[width=\columnwidth]{figures_arxiv/demo_inp_bsd2.pdf}\vspace{-0.1cm}
    \caption{\label{fig:inpainting_bsd}Natural image inpainting\vspace{-0.3cm}}
\end{figure}












We provide some comments on the growth conditions which constituted the majority of our analysis in sections \ref{sec:Hmixing} and \ref{sec:Hsigma}. In the simplest cases of Lemma \ref{lemma:unstableGrowth}, growth was established in an analogous fashion to the old one-step expansion condition (\ref{eq:oldOneStepExpansion}), finding the relevant Jacobians $M_j$ and checking that their expansion factors $K(M_j)$ satisfy
\begin{equation}
    \label{eq:discussionOneStep}
    \sum_j \frac{1}{K(M_j)} <1.
\end{equation}
For the more complicated cases, the inductive method used to establish growth near the accumulation points in Lemma \ref{lemma:unstableGrowth} and the weakened one-step expansion condition (\ref{eq:oneStep}) both address the same fundamental issue: the splitting of unstable curves by singularities into an unbounded number of small components. They circumvent this obstacle in rather different ways, however. While (\ref{eq:oneStep}) generalises (\ref{eq:discussionOneStep}) to ensure an growth of unstable curves `on average' (see \cite{chernov_statistical_2009} for a precise statement), our inductive method is a more direct adaptation of (\ref{eq:discussionOneStep}), using it to generate contradictory geometric conditions which a hypothetical non-growing unstable curve must satisfy. It may be possible to prove Theorem \ref{sec:Hmixing} using (\ref{eq:oneStep}) as the basis for growth. Since we required (\ref{eq:oneStep}) anyway for proving Theorem \ref{thm:HsigmaExp}, this could potentially condense our analysis, but only to a minor extent. A convenience of the method used in section \ref{sec:Hmixing} is that, by way of the `simple intersection' property, it naturally gives geometric information on the images of manifolds, useful for proving the property \textbf{(M)} of Theorem \ref{thm:katok-strelcyn}.

We expect that essentially analogous analysis can be applied to establish mixing properties in a wide class of piecewise linear non-uniformly hyperbolic maps, including those (like the OTM) which sit on the boundary of ergodicity and beyond. While we have relied on the precise partition structure of $H_\sigma$, its fundamental feature (self-similar sequences of elements $A^k$, sharing boundaries with its neighbours $A^{k-1},A^{k+1}$ and accumulating onto some point $p$) is quite typical to return map systems. See, for example, those of various stadium billiards \cite{chernov_chaotic_2006,chernov_improved_2008,chernov_statistical_2009} and LTMs \cite{springham_polynomial_2014}. Indeed, the same method can be used to prove the Bernoulli property for non-monotonic LTMs \cite{myers_hill_mixing_2022}, where monotonicity of the manifold images cannot be assumed and the classical argument \cite{sturman_mathematical_2006} fails. The OTM is the pointwise limit of these maps as the boundary shrinks to null measure. It further has utility in proving growth conditions for maps which are uniformly hyperbolic but possess regions $A_j$ where the hyperbolicity is very weak, signified by $K(M_j) \approx 1$, so that (\ref{eq:discussionOneStep}) fails. Typically this leads to suboptimal bounds on mixing windows, see e.g. \cite{wojtkowski_model_1981,przytycki_ergodicity_1983,myers_hill_family_2022}. The map $H_{(\eta,\eta)}$ for $\eta \approx 1/2$ is another example, possessing weak hyperbolicity over $A_2, A_3$. Letting $\varepsilon = |\eta-1/2|>0$, there is an upper bound $N = N(\varepsilon)$ on escape times from the intersections $A_2\cap \sigma, A_3 \cap \sigma$. The growth lemma then follows by applying the inductive step roughly $N$ times and can be established for arbitrarily small $\varepsilon$, opening the door to establishing optimal mixing windows.

The above gives two examples of piecewise linear perturbations to $H$ where mixing with respect to Lebesgue is preserved and our methods can be applied. Nonlinear perturbations to the shear profiles complicate the analysis in several ways. Firstly as the map's Jacobians takes on a broader range of values, cone invariance becomes an increasingly harder condition to establish. Cones must be widened, giving looser bounds on expansion factors, which may already be weak due to new regions of weaker stretching. This, together with the change from polygonal to curvilinear return time partition elements and nonlinear local manifolds, adds some complexity to showing growth conditions. This does not rule out certain (small) nonlinear perturbations however. There is some leeway in the inequalities which govern cone invariance and growth of local manifolds, the latter of which is not too dissimilar from the piecewise linear setting (see Lemmas \ref{lemma:piecewiseApprox}, \ref{lemma:componentLength}). Certain small perturbations would not alter the \emph{topological} structure of the return time partition, i.e. which elements share boundaries, the key information needed for setting up the induction. Finally while the partition elements would no longer be polygonal, only coarse geometric information is required for verifying each inductive step. Following the above, a potential perturbation could be to replace the linear portions of each shear by a cubic, perturbing the tent profile
\[  f(t) = \begin{cases} 2t & 0 \leq t \leq 1/2, \\ 2(1-t) & 1/2 \leq t \leq 1 ,\end{cases} \]
of the OTM shears to
\[  f_a(t) = \begin{cases} \frac{1}{8} t \left(16 - a + 6at - 8at^{2} \right) & 0 \leq t \leq 1/2, \\ \frac{1}{8}\left(1-t\right)\left( 16 - a + 6a\left(1-t\right) - 8a\left(1-t\right)^{2}\right)  & 1/2 \leq t \leq 1, \end{cases}   \]
for $a>0$. For small enough $a$ the gradient range $f'(t)$ is restricted to small neighbourhoods of $\{ 2, -2\}$ and the escape time partition retains a similar structure. We illustrate this in Figure \ref{fig:perturbations}, showing escapes from the square $S_3$ under the map $G \circ F$, equivalent to escapes from the perturbed $A_3$ under the $G \circ F$, but with a cleaner geometry for comparison. When $a$ is too large the analogy to the OTM breaks down. At $a=16$ the map is twice differentiable everywhere and features a new source of slowed mixing, the Jacobian is the identity at the corner points $x,y \in \{  0, 1/2 \}$ giving locally parabolic behaviour (visible in the escape time partition). 

\begin{figure}
    \centering
    \includegraphics[width=0.24 \linewidth]{0.png}
    \includegraphics[width=0.24 \linewidth]{4.png}
    \includegraphics[width=0.24 \linewidth]{8.png}
    \includegraphics[width=0.24 \linewidth]{16.png}
    \caption{Partition of escape times from $S_3$ under the mapping $F \circ G$ for $a= 0,4,8,16$. }
    \label{fig:perturbations}
\end{figure}

\section{Conclusion}
This work proposes \emph{DnX} as a simple and intuitive two-step framework for post-hoc explanation of GNNs. First, we distill the GNN into a simpler and more interpretable one, that serves as a global surrogate. Then, we leverage the simple structure of the surrogate to extract explanations. Experiments show that (Fast)DnX outperforms the prior art on a variety of benchmarks. Remarkably, our simple design allows FastDnX to run at least $200\times$ faster than relevant baselines on real-world tasks. 
%
Additionally, we provide theoretical results that justify our framework and support our empirical findings.
%
Besides advancing the current art, we hope this work will motivate other researchers to focus on developing compute-efficient explainability methods.

\section*{Acknowledgments}

This work was supported by the Silicon Valley Community Foundation (SVCF) through the Ripple impact fund, the Funda\c{c}\~ao de Amparo \`a Pesquisa do Estado do Rio de Janeiro (FAPERJ), the Funda\c{c}\~ao Cearense de Apoio ao Desenvolvimento Científico e Tecnológico (FUNCAP), the Coordena\c{c}\~ao de Aperfei\c{c}oamento de Pessoal de N\'ivel Superior (CAPES), and the Getulio Vargas Foundation's school of applied mathematics (FGV EMAp).


 \newpage

\bibliographystyle{plainnat}
\bibliography{references}

\appendix
\section{Appendix for Proofs}

\paragraph{Proof of Theorem \ref{thm:main}.}

\begin{proof}
\label{proof:main}
Our proof has two steps. In Step 1, we will show that SimCLR is equivalent to minimizing the cross entropy loss defined in Eqn.~(\ref{eqn:cross-entropy}). 
In Step 2, we will show  that minimizing the cross-entropy loss 
is equivalent to spectral clustering on $\bfpi$. 
Combining the two steps together, we have proved our theorem. 

\textbf{Step 1: } SimCLR is equivalent to minimizing the cross entropy loss.

The cross-entropy loss takes expectation over 
$\bfW_\bfX\sim \mathbb{P}(\cdot ; \bfpi)$, 
which means $\bfW_\bfX$ has exactly one non-zero entry in each row $i$. By Lemma~\ref{lem:multinomial}, we know every row $i$ of $\bfW_\bfX$ is independent of other rows. Moreover, 
$\bfW_{\bfX,i}\sim \mathcal{M}(1, \bfpi_i/\sum_j \bfpi_{i,j})=\mathcal{M}(1, \bfpi_i)$, because $\bfpi_i$ itself is a probability distribution.
Similarly, we know $\bfW_\bfZ$ also has the row-independent property by sampling over $\mathbb{P}(\cdot;\bfK_\bfZ)$.
Therefore, by Lemma~\ref{lem:cross_split}, we know Eqn.~(\ref{eqn:cross-entropy}) is equivalent to:
\[
 -\sum_{i=1}^n \mathbb{E}_{\bfW_{\bfX,i}}[\log \mathbb{P}(\bfW_{\bfZ,i}=\bfW_{\bfX,i};\bfK_\bfZ)],
\]

This expression takes expectation over $\bfW_{\bfX,i}$ for the given row $i$. Notice that 
$\bfW_{\bfX,i}$ has exactly one non-zero entry, which equals $1$ (same for $\bfW_{\bfZ,i}$). 
As a result
we expand the above expression to be:
\begin{equation}
 -\sum_{i=1}^n \sum_{j\neq i} \Pr(\bfW_{\bfX,i,j}=1)\log \Pr(\bfW_{\bfZ,i,j}=1).
\label{eqn:detailed-expansion}    
\end{equation}


By Lemma~\ref{lem:multinomial}, $\Pr(\bfW_{\bfZ,i,j}=1)=\bfK_{\bfZ,i,j}/\|\bfK_{\bfZ,i}\|_1$ for $j\neq i$. Recall that $\bfK_\bfZ=(k(\bfZ_i-\bfZ_j))_{(i,j)\in[n]^2}$, which means 
$\bfK_{\bfZ,i,j}/\|\bfK_{\bfZ,i}\|_1=\frac{\exp(-\|\bfZ_i-\bfZ_j\|^2/{2\tau})}{\sum_{k\neq i}
\exp(-\|\bfZ_i-\bfZ_k\|^2/{2\tau})
}$ for $j\neq i$, when $k$ is the Gaussian kernel with variance $\tau$. 

Notice that $\bfZ_i=f(\bfX_i)$, so we know
\begin{equation}
-\log \Pr(\bfW_{\bfZ,i,j}=1)=
-\log \frac{\exp(-\|f(\bfX_i)-f(\bfX_j)\|^2/{2\tau})}{\sum_{k\neq i}
\exp(-\|f(\bfX_i)-f(\bfX_k)\|^2/{2\tau}),
}
\label{eqn:infonce-equivalence}    
\end{equation}


The right hand side is exactly the InfoNCE loss defined in Eqn.~(\ref{eqn:infonce}).
Inserting Eqn.~(\ref{eqn:infonce-equivalence}) into Eqn.~(\ref{eqn:detailed-expansion}), we get the SimCLR algorithm, which first samples augmentation pairs $(i,j)$ with $\Pr(\bfW_{\bfX,i,j}=1)$ for each row $i$, and then optimize the InfoNCE loss. 

\textbf{Step 2: } minimizing the cross entropy loss 
is equivalent to spectral clustering on $\bfpi$.


By Lemma~\ref{lem:convert_to_spectral}, we may further convert the loss to 
\begin{equation}
\label{eqn:main-theorem-repul-attr}
\min_{\bfZ}
-\sum_{(i,j)\in [n]^2} \mathbf{P}_{i,j}
\log k (\bfZ_i-\bfZ_j)+\log \mathbf{R}(\bfZ).
\end{equation}
Since $k$ is the Gaussian kernel, this reduces to \[
\min_\bfZ \mathrm{tr}(\bfZ^\top \mathbf{L}(\bfpi) \bfZ)
+\log \mathbf{R}(\bfZ),
\]

where we use the fact that $\mathbb{E}_{\bfW_\bfX\sim \mathbb{P}(\cdot; \bfpi)}[\mathbf{L}(\bfW_\bfX)]
=\mathbf{L}(\bfpi)
$, because the Laplacian operator is linear and $
\mathbb{E}_{\bfW_\bfX\sim \mathbb{P}(\cdot; \bfpi)}(\bfW_\bfX)=\bfpi
$.
\end{proof}

\paragraph{Proof of Theorem \ref{thm:clip}.}
\begin{proof}
Since $\bfW_\bfX\sim \mathbb{P}(\cdot;\bfpi_{\mathbf{A}, \mathbf{B}})$, we know 
$\bfW_\bfX$ has exactly one non-zero entry in each row, denoting the pair that got sampled. 
A notable difference compared to the previous proof is we now have $n_\mathcal{A}+n_\mathcal{B}$ objects in our graph. CLIP deals with this by taking a mini-batch of size $2N$, 
such that $n_\mathcal{A}=n_\mathcal{B}=N$, and adding the $2N$ InfoNCE losses together. We label the objects in $\mathcal{A}$ as $[n_\mathcal{A}]$, and the objects in $\mathcal{B}$ as $\{n_\mathcal{A}+1, \cdots, n_\mathcal{A}+n_\mathcal{B}\}$. 

Notice that $\bfpi_{\mathbf{A}, \mathbf{B}}$ is a bipartite graph, so the edges of objects in $\mathcal{A}$ will only connect to object in $\mathcal{B}$ and vice versa. We can define the similarity matrix in $\cZ$ as $\bfK_\bfZ$, 
where $\bfK_\bfZ(i, j+n_\mathcal{A})=\bfK_\bfZ(j+n_\mathcal{A},i)= k(\bfZ_i-\bfZ_j)$ for $i\in [n_\mathcal{A}], j\in [n_\mathcal{B}]$, and otherwise we set $\bfK_\bfZ(i,j)=0$. 
The rest is same as the previous proof. 
\end{proof}

\paragraph{Proof of Theorem \ref{thm:exponential}.}

\begin{proof}
\label{proof:exponential}
Since the objective function consists of a linear term combined with an entropy regularization, which is a strongly concave function, the maximization problem is a convex optimization problem. Owing to the implicit constraints provided by the entropy function, the problem is equivalent to having only the equality constraint. We then introduce the Lagrangian multiplier $\lambda$ and obtain the following relaxed problem:

$$
\widetilde{E}(\boldsymbol{\alpha})=\psi_{1}-\sum_{i=1}^n \alpha_{i} \psi_{i}+\tau \sum_{i=1}^n \alpha_{i}\log \alpha_{i}+\lambda\left(\boldsymbol{\alpha}^{\top} \mathbf{1}_n-1\right).
$$

As the relaxed problem is unconstrained, taking the derivative with respect to $\alpha_{i}$ yields

$$
\frac{\partial \widetilde{E}(\boldsymbol{\alpha})}{\partial \alpha_{i}}=-\psi_{i}+\tau\left(\log \alpha_{i}+\alpha_{i} \frac{1}{\alpha_{i}}\right)+\lambda=0.
$$

Solving the above equation implies that $\alpha_{i}$ takes the form
$
\alpha_{i}=\exp \left(\frac{1}{\tau} \psi_{i}\right) \exp \left(\frac{-\lambda}{\tau}-1\right).
$ Since $\alpha_{i}$ lies on the probability simplex, the optimal $\alpha_{i}$ is explicitly given by
$
\alpha^{*}_{i}=\frac{\exp \left(\frac{1}{\tau} \psi_{i}\right)}{\sum_{i^{\prime}=1}^n \exp \left(\frac{1}{\tau} \psi_{i^{\prime}}\right)} .
$ Substituting the optimal point into the objective function, we obtain
$$
\begin{aligned}
E\left(\boldsymbol{\alpha}^*\right)  &=\psi_1-\sum_{i=1}^n \frac{\exp \left(\frac{1}{\tau} \psi_{i}\right)}{\sum_{i^{\prime}=1}^n \exp \left(\frac{1}{\tau} \psi_{i^{\prime}}\right)} \psi_{i}+\tau \sum_{i=1}^n \frac{\exp \left(\frac{1}{\tau} \psi_{i}\right)}{\sum_{i^{\prime}=1}^n \exp \left(\frac{1}{\tau} \psi_{i^{\prime}}\right)}\log \frac{\exp \left(\frac{1}{\tau} \psi_{i}\right)}{\sum_{i^{\prime}=1}^n \exp \left(\frac{1}{\tau} \psi_{i^{\prime}}\right)} \\
& =\psi_1 - \tau \log \left(\sum_{i=1}^n \exp \left(\frac{1}{\tau} \psi_{i}\right)\right).
\end{aligned}
$$
Thus, the Lagrangian dual function is given by
\begin{equation*}
-E\left(\boldsymbol{\alpha}^*\right)= -\tau \log \frac{\exp \left(\frac{1}{\tau} \psi_{1}\right)}{\sum_{i=1}^n \exp \left(\frac{1}{\tau} \psi_{i}\right)}.\qedhere
\end{equation*}
\end{proof}



\section{More on Experiments} \label{section: experiment_details}

\paragraph{CIFAR-10 and CIFAR-100} CIFAR-10 ~\citep{krizhevsky2009learning} and CIFAR-100 ~\citep{krizhevsky2009learning} are well-known classic image classification datasets. Both CIFAR-10 and CIFAR-100 contain a total of 60k $32 \times 32$ labeled images of different classes, with 50k for training and 10k for testing. CIFAR-10 is similar to CIFAR-100, except there are 10 different classes in CIFAR-10 and 100 classes in CIFAR-100.

\paragraph{TinyImageNet} TinyImageNet ~\citep{le2015tiny} is a subset of ImageNet ~\citep{deng2009imagenet}. There are 200 different object classes in TinyImageNet, with 500 training images, 50 validation images, and 50 test images for each class. All the images in TinyImageNet are colored and labeled with a size of $64 \times 64$.

\textbf{Pseudo-code.} Algorithm \ref{alg:Training Procedure} presents the pseudo-code for our empirical training procedure.

\begin{algorithm}[!htbp]
\caption{Training Procedure}
\label{alg:Training Procedure}
\begin{algorithmic}[1]
\REQUIRE trainable encoder network $f$, batch size $N$, augmentation strategy \textit{aug}, loss function $L$ with hyperparameters \textit{args}
\FOR {sampled minibatch ${x_i}_{i=1}^N$}
\FORALL{$i \in { 1, ..., N }$}
\STATE draw two augmentations $t_i = \textit{aug}\left(x_i\right) $, $t_i' = \textit{aug}\left(x_i\right) $
\STATE $z_i = f\left(t_i\right)$, $z_i' = f\left(t_i'\right)$
\ENDFOR
\STATE compute loss $\mathcal{L} = L(N, z, z', \textit{args})$
\STATE update encoder network $f$ to minimize $\mathcal{L}$
\ENDFOR
\STATE \textbf{Return} encoder network $f$
\end{algorithmic}
\end{algorithm}

We also provide the pseudo-code for our core loss function used in the training procedure in Algorithm \ref{alg:Core loss}. The pseudo-code is almost identical to SimCLR's loss function, with the exception of an extra parameter $\gamma$.

\begin{algorithm}[!htbp]
\caption{Core loss function $\mathcal{C}$}
\label{alg:Core loss}
\begin{algorithmic}[1]
\REQUIRE batch size $N$, two encoded minibatches $z_1, z_2$, $\gamma$, temperature $\tau$
\STATE $z = \textit{concat}\left(z_1, z_2\right)$
\FOR {$i \in {1, ..., 2N }, j \in {1, ..., 2N}$ }
\STATE $s_{i,j} = \Vert z_i - z_j \Vert_2^{\gamma}$
\ENDFOR
\STATE \textbf{define} $l(i, j)$ \textbf{as} $l(i, j) = - \log \frac{exp\left(s_{i,j}/\tau \right)}{\sum_{k=1}^{2N} \mathbf{1}{[k \ne i]} exp\left(s{i, j} / \tau \right)} $
\STATE \textbf{Return} $\frac{1}{2N} \sum_{k=1}^N\left[l(i, i+N) + l(i+N, i)\right]$
\end{algorithmic}
\end{algorithm}

Utilizing the core loss function $\mathcal{C}$, we can define all kernel loss functions used in our experiments in Table \ref{table: loss definition}. For all $z_i \in z$ with even dimensions $n$, we define $z_{L_i} = z_i\left[0:n/2\right]$ and $z_{R_i} = z_i\left[n/2:n\right]$.

\begin{table}[ht]
\centering
\begin{tabular}{{@{}l|l@{}}}
Kernel  &  Loss function \\ \midrule
Laplacian & $\mathcal{C}\left(N, z, z', \gamma=1, \tau\right)$\\ \midrule
Sum       & $\lambda * \mathcal{C}\left(N, z, z', \gamma=1, \tau_1\right) + (1-\lambda) * \mathcal{C}\left(N, z, z', \gamma=2, \tau_2\right)$  \\ \midrule
Concatenation Sum&$\lambda * \mathcal{C}\left(N, z_L, z'_L, \gamma=1, \tau_1\right) + (1-\lambda) * \mathcal{C}\left(N, z_R, z'_R, \gamma=2, \tau_2\right)$\\ \midrule
$\gamma = 0.5$ & $\mathcal{C}\left(N, z, z', \gamma=0.5, \tau\right)$          \\ 

\end{tabular}

\caption{Definition of kernel loss functions in our experiments}
\label {table: loss definition}
\end{table}

\textbf{Baselines.} We reproduce the SimCLR algorithm using PyTorch Lightning~\citep{PytorchLightning}.

\textbf{Encoder details.}
The encoder $f$ consists of a backbone network and a projection network. We employ ResNet50~\citep{ResNet} as the backbone and a 2-layer MLP (connected by a batch normalization~\citep{ioffe2015batch} layer and a ReLU \cite{nair2010rectified} layer) with hidden dimensions 2048 and output dimensions 128 (or 256 in the concatenation kernel case).

\textbf{Encoder hyperparameter tuning.}
For each encoder training case, we randomly sample 500 hyperparameter groups (sample details are shown in Table \ref{table: Hyperparameter sample}) and train these samples simultaneously using Ray Tune ~\citep{RayTune}, with the ASHA scheduler~\citep{li2018massively}. Ultimately, the hyperparameter group that maximizes the online validation accuracy (integrated in PyTorch Lightning) within 5000 validation steps is chosen for the given encoder training case.

\begin{table}[ht]
\centering

\begin{tabular}{@{}l|l|l@{}}
\midrule
Hyperparameter  & Sample Range & Sample Strategy \\ \midrule
start learning rate & $\left[10^{-2}, 10\right]$ & log uniform \\ \midrule
$\lambda$       & $\left[0, 1\right]$ & uniform \\ \midrule
$\tau$, $\tau_1$, $\tau_2$ & $\left[0, 1\right]$ & log uniform \\ \midrule
\end{tabular}

\caption{Hyperparameters sample strategy}
\label {table: Hyperparameter sample}
\end{table}

\textbf{Encoder training.} 
We train each encoder using the LARS optimizer~\citep{LARSOptimizer}, LambdaLR Scheduler in PyTorch, momentum 0.9, weight decay $10^{-6}$, batch size 256, and the aforementioned hyperparameters for 400 epochs on a single A-100 GPU.

\textbf{Image transformation.} The image transformation strategy, including augmentation, is identical to the default transformation strategy provided by PyTorch Lightning.

\textbf{Linear evaluation.}
The linear head is trained using the SGD optimizer with a cosine learning rate scheduler, batch size 64, and weight decay $10^{-6}$ for 100 epochs. The learning rate starts at $0.3$ and ends at $0$.

\textbf{Moco Experiments.} We also tested our method based on MoCo~\citep{he2019moco}. The results are summarized in Table \ref{tab:results-moco}. Here we choose ResNet18~\citep{ResNet} as the backbone and set a temperature of $0.1$ as default. For our simple sum kernel, we set $\lambda=0.8$. The results show that our method outperforms the original MoCo method.

\begin{table}[thb]
\centering
\caption{MoCo Experiment Results on CIFAR-10 and CIFAR-100.}
\label{tab:results-moco}
\resizebox{\textwidth}{!}{%
\begin{tabular}{@{}c|ccc|ccc@{}}
\toprule
\multirow{3}{*}{Method} & \multicolumn{3}{c|}{CIFAR-10} & \multicolumn{3}{c}{CIFAR-100} \\ \cmidrule(lr){2-4} \cmidrule(lr){5-7} 
                        & 200 epochs & 400 epochs    & 1000 epochs   & 200 epochs & 400 epochs & 1000 epochs         \\ \midrule
MoCo (repro.)         & $76.41 \pm 0.12$    & $80.01 \pm 0.15$          & $84.45 \pm 0.08$    & $\mathbf{47.02 \pm 0.11}$ & $52.50 \pm 0.07$ & $57.62 \pm 0.15$            \\
\midrule
Laplacian Kernel        & ${78.09 \pm 0.10}$    & $\mathbf{83.85 \pm 0.09}$          & $\mathbf{88.34 \pm 0.16}$    & $46.12 \pm 0.22$   & $53.44 \pm 0.17$ & $59.10 \pm 0.14$        \\
Simple Sum Kernel & $\mathbf{78.12 \pm 0.15}$   & $83.23 \pm 0.18$ & $87.50 \pm 0.20$ & $46.65 \pm 0.06$ & $\mathbf{53.62 \pm 0.19}$ & $\mathbf{59.83 \pm 0.12}$\\
\bottomrule
\end{tabular}
}
\end{table}



\section{More Experiments on Synthetic Data}


Consider a scenario with $n$ clusters, each containing $k$ vertices. Let the probability of vertices $u$ and $v$ from the same cluster belonging to $\bfpi$ be $p$. Conversely, for vertices $u$ and $v$ from different clusters, let the probability of belonging to $\pi$ be $q$. We generate the graph $\bfpi$ randomly, based on $p$ and $q$. We experiment with values of $k=100$ and $n=6$ for ease of visualization, embedding all points in a two-dimensional space. Each vertex's initial position originates from a normal distribution. In each iteration, we sample a subgraph of $\bfpi$ uniformly, ensuring each vertex has an out-degree of $1$. We then optimize the corresponding vectors using InfoNCE loss with an SGD optimizer and iterate until convergence. Our experimental setup consists of an SGD learning rate of $1$, an InfoNCE loss temperature of $0.5$, and a batch size of $50$. We evaluate two scenarios with different $p$ and $q$ values: $p=1$, $q=0$, and $p=0.75$, $q=0.2$. The results of these experiments are visualized in Figure \ref{fig:vis-spectral-cluster}. The obtained embeddings exhibit the hallmark pattern of spectral clustering of graph $\bfpi$.

\begin{figure}[!tb]
\centering
\subfigure{
\includegraphics[width=1\textwidth]{Figures/cluster_pi.png}
\label{fig:vis-cluster}
}
\subfigure{
\includegraphics[width=1\textwidth]{Figures/noised_cluster_pi.png}
\label{fig:vis-noised-cluster}
}
\caption{Visualizations of the optimization process using InfoNCE Loss on the vectors corresponding to $\bfpi$. Points of identical color belong to the same cluster within $\bfpi$. To showcase the internal structure of $\bfpi$, we randomly select 10 vertices from each cluster to display the edge distribution of $\bfpi$.}
\label{fig:vis-spectral-cluster}
\end{figure}



\end{document}
x`