\documentclass{article}
\pdfoutput=1

% Recommended, but optional, packages for figures and better typesetting:
\usepackage[utf8x]{inputenc}
\usepackage[T1]{fontenc}

\usepackage[accepted]{sty/icml2023}

\usepackage{microtype}
\usepackage{subfigure}
\usepackage{booktabs} % for professional tables
%% Useful packages
\usepackage{amsfonts, amsmath, amssymb, amsthm}
\usepackage{xfrac}
\usepackage{graphicx, xcolor, colortbl}
\usepackage{footnote}
\usepackage{tablefootnote}
\usepackage{natbib}
\usepackage{dsfont}
\usepackage{bbm}
\usepackage{algorithm}
\usepackage{algorithmic}
\usepackage{import}
\usepackage{mathtools}
\usepackage{bm}
\usepackage{xspace}
\usepackage{thmtools,thm-restate} %to restate lemma
\usepackage{accents}
\usepackage{framed}
\usepackage{enumitem}
\usepackage{booktabs}
\usepackage{color, colortbl}
\usepackage{nicefrac}


% hyperref makes hyperlinks in the resulting PDF.
% If your build breaks (sometimes temporarily if a hyperlink spans a page)
% please comment out the following usepackage line and replace
% \usepackage{icml2022} with \usepackage[nohyperref]{icml2022} above.
%\usepackage{hyperref}


% % Attempt to make hyperref and algorithmic work together better:
% \newcommand{\theHalgorithm}{\arabic{algorithm}}

% Use the following line for the initial blind version submitted for review:
%\usepackage{sty/icml2023}

% If accepted, instead use the following line for the camera-ready submission:


%notations
\usepackage{sty/notations}
%%%%%%%%%%%%%%%%%%%%%%%%%%%%%%%%
% THEOREMS
%%%%%%%%%%%%%%%%%%%%%%%%%%%%%%%%
\theoremstyle{plain}
\newtheorem{theorem}{Theorem}[section]
\newtheorem{proposition}[theorem]{Proposition}
\newtheorem{lemma}[theorem]{Lemma}
\newtheorem{corollary}[theorem]{Corollary}
\theoremstyle{definition}
\newtheorem{definition}[theorem]{Definition}
\newtheorem{assumption}[theorem]{Assumption}
\theoremstyle{remark}
\newtheorem{remark}[theorem]{Remark}

% % Todonotes is useful during development; simply uncomment the next line
% %    and comment out the line below the next line to turn off comments
% %\usepackage[disable,textsize=tiny]{todonotes}
% \usepackage[textsize=tiny]{todonotes}




% For  footnote reuse (https://tex.stackexchange.com/questions/35043/reference-different-places-to-the-same-footnote)
%\usepackage{scrextend}

% for multirow in table
% for multirow in table
\usepackage{multirow}
\usepackage{framed}
\usepackage{booktabs}
\usepackage{colortbl}
\usepackage{pifont}


% % working together
% \usepackage{todonotes}
% \newcommand{\todoDa}[1]{\todo[color=violet!40, inline]{\small Daniil: #1}}
% \newcommand{\todoPi}[1]{\todo[color=yellow!40, inline]{\small Pierre: #1}}
% \newcommand{\todopp}[1]{\todo[color=gray!40, inline]{\small Pierre P: #1}}
% \newcommand{\todom}[1]{\todo[color=green!40, inline]{\small Michal: #1}}
% \newcommand{\todoDe}[1]{\todo[color=orange!40, inline]{\small Denis: #1}}
% \newcommand{\todoYu}[1]{\todo[color=blue!40, inline]{\small Yunhao: #1}}
% \newcommand{\todoAlex}[1]{\todo[color=red!40, inline]{\small Alex: #1}}
% \newcommand{\todoRemi}[1]{\todo[color=cyan!40, inline]{\small R\'emi: #1}}
% \newcommand{\new}[1]{\textcolor{red}{#1}}

% for table of contents (Appendix only)
\usepackage{minitoc}
\renewcommand \thepart{}
\renewcommand \partname{}
% The \icmltitle you define below is probably too long as a header.
% Therefore, a short form for the running title is supplied here:
%\icmltitlerunning{Fast Rates for Maximum Entropy Exploration}

\begin{document}

\twocolumn[
\icmltitle{Fast Rates for Maximum Entropy Exploration}
% It is OKAY to include author information, even for blind
% submissions: the style file will automatically remove it for you
% unless you've provided the [accepted] option to the icml2022
% package.

% List of affiliations: The first argument should be a (short)
% identifier you will use later to specify author affiliations
% Academic affiliations should list Department, University, City, Region, Country
% Industry affiliations should list Company, City, Region, Country

% You can specify symbols, otherwise they are numbered in order.
% Ideally, you should not use this facility. Affiliations will be numbered
% in order of appearance and this is the preferred way.
%\icmlsetsymbol{equal}{*}

\begin{icmlauthorlist}
\icmlauthor{Daniil Tiapkin}{hse,airi}
\icmlauthor{Denis Belomestny}{essen,hse}
\icmlauthor{Daniele Calandriello}{deepmind}
\icmlauthor{\' Eric Moulines}{polytechnique,mbzuai}
\icmlauthor{R\'emi Munos}{deepmind}
\icmlauthor{Alexey Naumov}{hse}
\icmlauthor{Pierre Perrault}{idemia}
\icmlauthor{Yunhao Tang}{deepmind}
\icmlauthor{Michal Valko}{deepmind}
\icmlauthor{Pierre M\' enard}{lyon}
\end{icmlauthorlist}

\icmlaffiliation{hse}{HSE University}
\icmlaffiliation{airi}{Artificial Intelligence Research Institute}
\icmlaffiliation{essen}{Duisburg-Essen University}
\icmlaffiliation{idemia}{IDEMIA}
\icmlaffiliation{polytechnique}{\' Ecole Polytechnique}
\icmlaffiliation{mbzuai}{Mohamed Bin Zayed University of AI}
\icmlaffiliation{deepmind}{Google DeepMind}
\icmlaffiliation{lyon}{ENS Lyon}

\icmlcorrespondingauthor{Daniil Tiapkin}{dtyapkin@hse.ru}
%\icmlcorrespondingauthor{Firstname2 Lastname2}{first2.last2@www.uk}

% You may provide any keywords that you
% find helpful for describing your paper; these are used to populate
% the "keywords" metadata in the PDF but will not be shown in the document
\icmlkeywords{Reinforcement Learning, Maximum Entropy Exploration, Regularization in RL}

\vskip 0.3in
]
% this must go after the closing bracket ] following \twocolumn[ ...

% This command actually creates the footnote in the first column
% listing the affiliations and the copyright notice.
% The command takes one argument, which is text to display at the start of the footnote.
% The \icmlEqualContribution command is standard text for equal contribution.
% Remove it (just {}) if you do not need this facility.

\printAffiliationsAndNotice{}  % leave blank if no need to mention equal contribution
%\printAffiliationsAndNotice{\icmlEqualContribution} % otherwise use the standard text.

% For TOC in appendix (https://tex.stackexchange.com/a/419290)
\doparttoc % Tell to minitoc to generate a toc for the parts
\faketableofcontents % Run a fake tableofcontents command for the partocs

\begin{abstract}
We address the challenge of exploration in reinforcement learning (RL) when the agent operates in an unknown environment with sparse or no rewards. 
In this work, we study the maximum entropy exploration problem of two different types. The first type is \textit{visitation entropy maximization}  previously considered by \citet{hazan2019provably} in the discounted setting. For this type of exploration, we propose a game-theoretic algorithm that has $\tcO(H^3S^2A/\varepsilon^2)$ sample complexity thus improving the $\varepsilon$-dependence upon existing results, where $S$ is a number of states, $A$ is a number of actions, $H$ is an episode length, and $\varepsilon$ is a desired accuracy. The second type of entropy we study is the \textit{trajectory entropy}. This objective function is closely related to the entropy-regularized MDPs, and we propose a simple algorithm that has a sample complexity of order $\tcO(\poly(S,A,H)/\varepsilon)$. Interestingly, it is  the first theoretical result in RL literature that establishes the potential statistical advantage of regularized MDPs for exploration.
Finally, we apply developed regularization techniques to reduce sample complexity of visitation entropy maximization to $\tcO(H^2SA/\varepsilon^2)$, yielding a statistical separation between maximum entropy exploration and reward-free exploration.
\end{abstract}

\section{Introduction}

The increasing complexity of source code poses a key challenge to the reliability of large-scale software systems. Software bugs in these systems can lead to safety issues~\cite{bug_safety} for users around the world as well as cause non-negligible financial losses~\cite{bug_loss}. As such, developers have to spend a large amount of time and effort on bug fixing. Consequently, \aprfull (\apr), designed to automatically generate patches to fix software bugs, has attracted wide attention from both academia and industry~\cite{long2016prophet, legoues2012genprog, long2015spr, lou2020can, tufano2018empstudy}. 


To achieve \apr, one popular approach is known as Generate-and-Validate (G\&V)~\cite{qi2015gv, ghanbari2019prapr, lou2020can, le2016hdrepair, legoues2012genprog, wen2018capgen, hua2018sketchfix, martinez2016astor, koyuncu2020fixminder, liu2019tbar, liu2019avatar}, which is typically based on the following pipeline: First, fault localization techniques~\cite{wong2016fl, abreu2007ochiai, zhang2013injecting, papadakis2015metallaxis, li2019deepfl, li2017transforming} are applied to determine the suspicious locations in programs where bugs are likely to exist. Then, the buggy locations are used by the \apr tools to generate a list of patches that replace buggy lines with correct lines. Afterward, each patch is validated against the original test suite to identify any \emph{plausible patches} (i.e., passing all tests in the test suite). Finally, to determine the \emph{correct patches}, developers examine the list of plausible patches to see if any of them can correctly fix the bug. 

Traditional \apr tools can mainly be categorized into heuristic-based~\cite{legoues2012genprog, le2016hdrepair, wen2018capgen}, constraint-based~\cite{mechtaev2016angelix, le2017s3, demacro2014nopol, long2015spr} and \template~\cite{ghanbari2019prapr, hua2018sketchfix, martinez2016astor, liu2019tbar, liu2019avatar}. Among these traditional tools, \template \apr tools~\cite{ghanbari2019prapr, liu2019tbar, benton2020effectiveness} have been able to achieve state-of-the-art results. \Template \apr tools typically leverage pre-defined templates (e.g., adding a nullness check) for bug fixing. However, since these fix templates are typically handcrafted, the number and types of bugs they are able to fix can be limited. 



To address the limitations of traditional \apr, researchers have proposed various \learning \apr tools~\cite{li2020dlfix, chen2018sequencer, jiang2021cure, lutellier2020coconut, zhu2021recoder, ye2022rewardrepair} based on the \nmtfull (\nmt) architecture~\cite{sutskever2014mt} where the input is the buggy code snippets and the goal is to translate the buggy code snippets into a fixed version. To accomplish this, \learning \apr tools require supervised training datasets with pairs of both buggy and fixed code snippets in order to learn how to perform this translation step. These training data are usually obtained by mining historical bug fixes using heuristics/keywords~\cite{dallmeier2007benchmark}, which can be imprecise for identifying bug-fixing commits; even the actual bug-fixing commits can include irrelevant code changes, leading to further pollution in the dataset~\cite{xia2022alpharepair}.
% 
Moreover, it can be hard for such \apr tools to generalize and fix bug types unseen during training. 



To better leverage recent advances in \plmfull{s} (\plm{s}), researchers~\cite{xia2022alpharepair, xia2023repairstudy, kolak2022patch, prenner2021codexws} have directly applied \plm{s} to generate patches without bug-fixing datasets. These \llm-based \apr tools work by either directly generating a complete code function~\cite{prenner2021codexws, xia2023repairstudy} or predict/infill the correct code snippet given its surrounding context~\cite{xia2022alpharepair, xia2023repairstudy}. By directly using \llm{s} that are pre-trained on billions of open-source code snippets, \llm-based \apr tools can achieve state-of-the-art performance on many repair datasets~\cite{xia2022alpharepair}. 


% 
%
%

Traditional \apr tools have long used the insight of the \emph{plastic surgery hypothesis}~\cite{barr2014plastic} where it states that the code ingredients to fix a bug already exist within the same project. Traditional \apr tools have manually designed pattern-~\cite{ghanbari2019prapr, saha2017elixir} or heuristic-based~\cite{jiang2018simfix, legoues2012genprog} approaches to finding and using such relevant code ingredients to generate fixes for bugs. However, the plastic surgery hypothesis has been largely ignored in \llm-based \apr. In fact, \llm provides a unique opportunity to fully automate the plastic surgery hypothesis idea via fine-tuning (learning project-specific information via model updates from the buggy project) and prompting (directly providing relevant code ingredients to the model), and make it directly applicable to different languages (since the \llm{s} are typically multi-lingual).%
Moreover, despite the intensive manual efforts involved, traditional \apr tools still cannot fully leverage project-specific information due to large search space for leveraging/composing existing code ingredients. In contrast, the project-specific information can effectively leveraged by \llm{s} due to their power in code understanding/vectorization, e.g., even partial/imprecise information may still guide \llm{s} in correct patch generation!
 To this end, we ask the question: \emph{How useful is the plastic surgery hypothesis in the era of \plm{s}}?








\mypara{Our Work.} To answer the question, we present \ourtech{\xspace} -- a \llm-based approach that automatically utilizes the plastic surgery hypothesis by systematically combining multiple fine-tuning and prompting strategies for \apr. \ourtech fine-tunes \plm{s} using two novel domain-specific training strategies: \textbf{\epfinetune} -- we fine-tune using the original buggy project by aggressively masking out a high percentage of tokens, which allows \plm to learn project-specific code tokens and programming styles; and \textbf{\rofinetune} -- which only masks out a single continuous code sequence per training sample, allowing the model to get used to the final \csapr task of predicting a single continuous code sequence. Furthermore, we directly leverage the ability for \plm{s} to understand natural language instructions and introduce a novel prompting strategy, \textbf{\idprompting}, which uses information retrieval and static analysis to obtain a list of relevant identifiers for the buggy lines. While such relevant identifiers are critical for fixing some difficult bugs, they may not be seen by the \llm during inference due to limited context window size. Through the use of prompting, we directly tell the model to use these extracted identifiers (relevant code ingredients) to generate the correct code. Finally, to perform repair, we combine all four model variants (including the base model, both fine-tuned models and the base model with prompting) for the final repair.





While our insight of leveraging the plastic surgery hypothesis for \llm-based \apr is generalizable across different types of \plm{s}, to implement \ourtech, we choose a recent \plm{\xspace}, \ctfive~\cite{wang2021codet5}, which is pre-trained on millions of open-source code snippets. \ctfive is an encoder-decoder model trained using \mspfull (\msp) objective where a percentage of tokens are masked out and each continuous masked token sequence is referred to as a masked span. Also, although we only extract relevant identifiers from the current buggy project (since this paper focuses on the plastic surgery hypothesis), our work can be easily extended to obtain other code information (such as relevant statements or functions) from other sources, such as  the massive pre-training corpora~\cite{husain2020codesearchnet} or historical bug-fixing datasets~\cite{jiang2019infer}, which can provide more coding knowledge for \llm{s}. Besides, although we mainly focus on using traditional string comparison algorithms for information retrieval in this paper, these techniques can be easily replaced by other frequency-based retrieval~\cite{robertson2009probabilistic} and neural search (or embedding-based search)~\cite{reimers2019sentence}.
  In summary, this paper makes the following contributions:


%


\begin{itemize}[noitemsep, leftmargin=*, topsep=0pt]
    \item \textbf{Dimension.} This paper is the first to revisit the important plastic surgery hypothesis in the era of \llm{s}. It opens up a new dimension for \llm-based \apr to incorporate previously neglected information from the buggy project itself to boost \apr performance. Furthermore, it demonstrates the promising future of retrieval-based prompting for modern \llm-based \apr.
    \item \textbf{Implementation.} We implement \ourtech based on the recent \ctfive model. We augment the model using two novel fine-tuning strategies: \epfinetune and \rofinetune, along with a novel prompting strategy based on information retrieval and static analysis: \idprompting. We combine the patches generated by all four models together and perform patch ranking to speed up \apr.% 
    \item \textbf{Evaluation Study.} We conduct an extensive evaluation against state-of-the-art \apr tools. On the widely studied \dfj 1.2 and 2.0 datasets~\cite{just2014dfj}, \ourtech is able to achieve the new state-of-the-art results of 89 and 44 correct bug fixes (15 and 8 more than best baseline) respectively.  Furthermore, we perform a broad ablation study to justify our design. \ourtech demonstrates for the first time that the plastic surgery hypothesis can substantially boost \llm-based \apr and advance state-of-the-art \apr, while being fully automated and general. Moreover, even partial/imprecise code ingredients may still effectively guide \llm{s} for \apr!
\end{itemize}


\section{Experiments}\label{sec:experiments}

In this section, DISTRO is evaluated for a variety of robust ID and OOD tests and is compared to previous approaches.
As baseline, we consider the pre-trained models\footnote{\href{https://github.com/AlexMeinke/Provable-OOD-Detection}{https://github.com/AlexMeinke/Provable-OOD-Detection}} from \citet{prood}.
The normal trained (\textbf{Plain}) and outlier exposure (\textbf{OE})~\cite{oe} models share the same ResNet18~\cite{resnet} architecture and hyperparameters as \textbf{ProoD}~\cite{prood}.
\textbf{GOOD}~\cite{good} uses a 'XL' convolutional neural network.
Additionally, we evaluate the pretrained DenseNet101~\cite{densenet} models for \textbf{ATOM}~\cite{atom} and \textbf{ACET}~\cite{acet}; and the standard OOD detection methods: \textbf{VOS}\footnote{\href{https://github.com/deeplearning-wisc/vos}{https://github.com/deeplearning-wisc/vos}}~\cite{vos} and \textbf{LogitNorm}\footnote{\href{https://github.com/hongxin001/logitnorm_ood}{https://github.com/hongxin001/logitnorm\_ood}}~\cite{logitnorm} with the pretrained WideResNet40~\cite{wideresnet} models provided in the respective works.
We consider \textbf{DDS}~\cite{dds} with a pre-trained diffusion model\footnote{\href{https://github.com/openai/improved-diffusion}{https://github.com/openai/improved-diffusion}} from \citet{nichol2021improved} in front of the OE classifier.
With \textbf{DISTRO}, we incorporate the same pre-trained diffusion model of DDS before the main classifier of ProoD, and maintain its discriminator.
The diffusion models have been used with the settings described in \citet{dds}.
In the context of $\ell_\infty$, we set $\sigma = \sqrt{d} \cdot \epsilon$.

We evaluate all methods on the standard datasets \texttt{CIFAR10/100}~\cite{cifar} as ID.
For the OOD detection evaluation we consider the following set of datasets: 
\texttt{CIFAR100/10}, \texttt{SVHN}~\cite{svhn}, LSUN~\cite{lsun} cropped (\texttt{LSUN\_CR}) and resized (\texttt{LSUN\_RS}),  TinyImageNet~\cite{tiny} cropped (\texttt{TinyImageNet\_CR}), \texttt{Textures}~\citep{textures} and synthetic (\texttt{Gaussian} and \texttt{Uniform}) noise distributions.
We use a random but fixed subset of 1000 images for all datasets considered as a test for OOD.
For ID, we consider the entire dataset.
We run all our experiments on a single NVIDIA A100. 

\subsection{In-Distribution Results}\label{sec:id-results}

Here, we compare clean, adversarial, and certified accuracy for ID samples.
Adversarial accuracy is evaluated with AutoAttack~\citep{apgd} for $\ell_\infty$-norm attacks of budget $\epsilon \in \{\nicefrac{2}{255}, \nicefrac{8}{255}\}$.
We ran the standard version of AutoAttack without additional hyper-parameters. 
Certified accuracy is evaluated for $\ell_2$-norm robustness of deviation $\sigma \in \{0.12, 0.25\}$.
To this end, random smoothing is performed on 10'000 Gaussian distributed samples around the input with a failure probability of $0.001$.
All $R>0$ are considered for the certified accuracy.
In the context of DISTRO and DDS we run 100 evaluation of the entire test set of \texttt{CIFAR10} to estimate the clean accuracy and report the average.
Further, we ran AutoAttack in both \textit{rand} and \textit{standard} modes, and considered the lowest results for DISTRO and DDS.


\begin{table}[htb]
\vspace{-0.5em}
    \centering
    \caption{\textbf{ID Accuracy}: Results of clean, adversarial and certified accuracy (\%) on the \texttt{CIFAR10} test set.
    The grayed-out models have an accuracy drop greater than $3\%$ relative to the model with the highest accuracy.}
    \label{tab:in-distribution}
    \begin{adjustbox}{width=0.5\textwidth,center}
        \begin{tabular}{llccccc}
            \toprule
            \multirow{2}{*}{Method} &\multirow{2}{*}{Clean} &\multicolumn{2}{c}{Adversarial ($\ell_\infty$)} &\multicolumn{2}{c}{Certified ($\ell_2$)} \\
            & &$\epsilon = \nicefrac{2}{255}$ &$\epsilon = \nicefrac{8}{255}$ &$\sigma=0.12$ &$\sigma = 0.25$ \\
            \midrule
            Plain$^*$       &95.01  &2.16   &0.00   &28.14  &14.17 \\
            OE$^*$          &95.53 &1.97   &0.00   &31.48  &10.88 \\
            VOS$^\dag$      &94.62  &2.24   &0.00   &13.13   &10.02       \\
            LogitNorm$^\ddag$  &94.48  &2.65   &0.00   &12.53  &10.25 \\
            \gray{ATOM$^*$}    &\gray{92.33}  &\gray{0.00}   &\gray{0.00}   &\gray{0.00}   &\gray{0.00}  \\
            \gray{ACET$^*$}    &\gray{91.49}  &\gray{69.01}  &\gray{6.04}   &\gray{57.13}  &\gray{12.48} \\
            \gray{GOOD$^*_{80}$} &\gray{90.13}  &\gray{11.65}  &\gray{0.23}   &\gray{17.33}  &\gray{10.31} \\
            ProoD$^*$ $\Delta=3$  &95.46  &2.69   &0.00   &33.92  &13.50 \\
            DDS                   &\textbf{95.55} &72.97 &24.09 &82.26 &64.58 \\
            DISTRO (our)          &95.47  &\textbf{73.34} &\textbf{27.14}  &\textbf{82.77}   &\textbf{65.63} \\
            \bottomrule
        \end{tabular}
    \end{adjustbox}
    \scriptsize{$*$ Pre-trained models from \citet{prood}, $\dagger$ Pre-trained from \citet{vos}, \\ $\ddag$ Pre-trained from \citet{logitnorm}.
    }
\vspace{-2em}
\end{table}

In \autoref{tab:in-distribution}, we show the results.
As expected, Plain and OE are not robust to adversarial attacks.
This applies to ProoD as well, since OE is its primary classifier.
Similarly, standard OOD detection methods, as LogitNorm and VOS, show poor robustness for ID data.
GOOD demonstrates better results than ProoD for adversarial attacks and worse in terms of certified accuracy.
Suprisingly, ACET reveals strong adversarial and certified accuracy despite of its reduced clean accuracy.
Meanwhile, ATOM results in zero for all tests since any slight perturbation of the input triggers the last neuron used for OOD detection.

\subsubsection*{Discussion}

It is clear that diffusion models can enhance adversarial and certified robustness while maintaining high clean accuracy.
As diffusion introduces variance into gradient estimators, standard attacks become much less effective.
Nevertheless, robustness accuracy of diffusion models varies over different runs for the same input, so it should be defined differently from deterministic accuracy, e.g. as expectation.
Luckily, one-shot diffusion introduces such a tiny variance that throughout a few of runs, our results were similar.
\section{Visitation entropy}
\label{sec:visitation_entropy}
In this section we focus on maximizing the visitation entropy defined below.

\paragraph{Visitation entropy} We define the visitation entropy of a policy $\pi$ denoted by  as the sum of the visitation distribution entropies at each steps
\vspace{-0.3cm}
\[
\VE(d^\pi) \triangleq \sum_{h=1}^H \cH(d_h^\pi)\,\cdot
\]
We denote by $\pistarVE\in\argmax_{\pi} \VE(d^\pi)$ a policy that maximizes the visitation entropy. 

% %\todom{can we say which one is preferable? was it easier for them
% %to show something of their AVG version?}
% %\todopp{I am not sure that the visitation entropy of \citet{hazan2019provably} is what is stated here. Maybe I am wrong, but from my understanding, it should be$\frac{1}{H}\sum_{(h,s,a)} d_h^\pi(s,a) \log\frac{1}{\frac{1}{H}\sum_{h} d_h^\pi(s,a)}$. The inequality is still true by concavity though}
% \todoPi{
% %What is the difference between the entropy you wrote and the one below it seems they are the same. 
% For me the difference between averaging the visitation distribution or not is a matter of do we treat each step state action $(h,s,a)$ separately or we only consider only $(s,a)$. The second option makes more sens in the discounted setting where the transition does not depends on the step and there is a natural limit visitation distribution. In the episodic case is not clear if we want to 'merge' all the visitation distributions}
% %\todopp{Bellow, inside the log, the sum is also over $s,a$ which I think is not correct.}
% %\todoPi{Yes you are right!}
% \todom{Thanks Pierre for noticing this! Given this 'discrepancy' of Hazan's defintion, can does it maka Hazan's less natural, since it is not $\sum_{i\in[n]} p_i\log p_i$ for the same $p$, if I understood correctly? 
% And this is something that we can remark? (I'm fishing here for some reason to justify why we do not follow Hazan 100\%)} 
% \todopp{Does our result hold for both entropies? If, for the Hazan's one, we take as rewards $-\log\big(\frac{1}{H}\sum_{h}d_h^t(s,a)\big)$ in \algMVEE}
% \end{remark}

\paragraph{Maximum visitation entropy exploration} In MVEE the agent interacts with the reward-free MDP $\cM$ as follows. At the beginning of episode $t$, the agent picks a policy $\pi^t$ based only on the transitions collected up to episode $t-1$. Then a new reward-free trajectory is sampled following the policy~$\pi^t$ and observed by the agent. At the end of each episode the agent can decide to stop collecting new data, according to a random stopping time $\tau$, the stopping rule, and outputs a (general) policy $\hpi$ based on the observed transitions. An agent for MVEE is therefore made of a triplet $((\pi^t)_{t\in\N},\tau,\hpi)$. 
%\todom{why is $((\pi^t)_{t\in\N},\tau,\hpi)$ and agent? }
%\todoPi{Non-Markovian output}
\begin{definition} (PAC algorithm for MVEE) An algorithm $((\pi^t)_{t\in\N},\tau,\hpi)$ is $(\epsilon,\delta)$-PAC for MVEE if 
{\small\[
\P\Big( \VE\big(d^{\pistarVE}\big) - \VE(d^{\hpi}) \leq \epsilon\Big) \geq 1-\delta.
\]}
\end{definition}
Our goal is to design an algorithm that is $(\epsilon,\delta)$-PAC for MVEE with as sample complexity $\tau$ as small as possible.

\subsection{MVEE by solving game}
\label{sec:MVEE_game}

Following the general framework of \citet{hazan2019provably, zahavy2021reward}, it is possible to solve MVEE by applying the Frank-Wolfe algorithm to a smoothed version of the visitation entropy. Interestingly, \citet{abernethy2017frankwolfe} showed that this procedure is equivalent to computing the Nash equilibrium of a particular game induced by the Legendre-Fenchel transform of the smoothed entropy. In fact, as noted by \citet{grunwald2002game}, there exists another game naturally linked to MVEE, stated next.

\paragraph{Prediction game} Maximum visitation entropy is the value of the following prediction game 
\begin{align*}
    \max_{d\in\cK_p} \VE(d) &= \max_{d\in\cK_p} \min_{\bd\in\cK}\sum_{(h,s,a)} d_h(s,a) \log \frac{1}{\bd_h(s,a)}\\
    &=  \min_{\bd\in\cK} \max_{d\in\cK_p} \sum_{(h,s,a)} d_h(s,a) \log \frac{1}{\bd_h(s,a)}\CommaBin
\end{align*}
see Lemma~\ref{lem:prediction_game} in Appendix~\ref{app:technical} for a proof.
This game can be interpreted as follows. On the one hand, the min player, or forecaster player, tries to predict which state-action pairs the max player will visit to minimize $\KL(d_h,\bd_h)$.  On the other hand, the max player, or sampler player, is rewarded for visiting state-action pairs that the forecaster player did not predict correctly.


We now describe the algorithm \algMVEE\ for MVEE. In this algorithm, we let a forecaster player and a sampler player compete for $T$ episodes long. Let us first define the two players.
\paragraph{Forecaster-player} As forecaster-player we use the Mixture-Forecaster for a logarithmic loss, see Section~9 in \citep{cesabianchi2006prediction}. Fix a prior count $n_0$ and their sum $t_0 = S A n_0$. The forecaster-player predicts at episode $t$ the distributions $\bd^t\in\cK$ with $\bd_h^t(s,a) = \bn_h^{t-1}(s,a) /(t+t_0)$ where the pseudo counts are $ \bn_h^t(s,a) = n_h^t(s,a)+n_0$ and $n_h^t(s,a)$ the counts of state-action pairs visited by the sampler-player.
Note that $\bd_h^t$ can be seen as the posterior mean under a Dirichlet distribution on $\cS\times\cA.$ 
% \todopp{Do we put a remark here saying that the Hazan's entropy case can be dealt using the AVG over h of the $\bd_h^t(s,a)$ ?}
% \todoPi{Explain the Bayesian interpretation}



\paragraph{Sampler-player} As sampler-player we choose the optimistic best-response. Define the optimistic Bellman equations 
{\small
\begin{align}
\begin{split}\label{eq:optimistic_planning_VE}
\uQ_h^t(s,a) &=  \log\frac{1}{\bd_h^{t+1}(s,a)} + \hp_h^{\,t} \uV^t_{h+1}(s,a) +b_h^t(s,a) \\
\uV_h^t(s) &= \clip\big( \max_{a\in\cA}\uQ_h^t(s,a), 0, \log(t/n_0+SA)H\big) 
\end{split}
\end{align}\\}
where  $V_{H+1}^t = 0$ and $b_h^t$ are some Hoeffding-like bonuses defined in \eqref{eq:sampler_exploration_bonus} of Appendix~\ref{app:visitation_entropy_proofs}. The sampler player then plays $d^{\pi^{t+1}}$ where $\pi^{t+1}$ is greedy with respect to the optimistic Q-values, that is, $\pi_h^{t+1}(s) \in\argmax_{\pi\in\Delta_A} \pi \uQ_h^{t}(s)$. 

\paragraph{Sampling rule} At each episode $t$ the policy $\pi^t$ of the sampler-player is used as a sampling rule to generate a new trajectory.

\paragraph{Decision rule} After $T$ episodes we output a non-Markovian policy $\hpi$ defined as the mixture of the policies $\{\pi^t\}_{t\in[T]}$, that is, to obtain a trajectory from $\hpi$ we first sample uniformly at random $t\in[T]$ and then follow the policy $\pi^t$. Note that the visitation distribution of $\hpi$ is exactly the average $d^{\hpi} = (1/T)\sum_{t\in[T]} d^{\pi^t}$. 



Remark that the stopping rule of \algMVEE is deterministic and equals to $\tau = T$. The complete procedure is detailed in Algorithm~\ref{alg:ourMVEE}.

\begin{algorithm}[h!]
\centering
\caption{\algMVEE}
\label{alg:ourMVEE}
\begin{algorithmic}[1]
  \STATE {\bfseries Input:} Number of episodes $T$, prior counts $n_0$.
      \FOR{$t \in[T]$}
      \STATE \textcolor{blue}{\# Forecaster-player}
      \STATE Update pseudo counts $\bn_h^{t-1}(s,a)$ and predict $\bd_h^t(s,a)$. 
      \STATE \textcolor{blue}{\# Sampler-player}
      \STATE Compute $\pi^t$ by optimistic planning \eqref{eq:optimistic_planning_VE} with rewards $\log\big(1/ \bd_h^t(s,a)\big)$.
    \STATE \textcolor{blue}{\# Sampling}
      \FOR{$h \in [H]$}
        \STATE Play $a_h^t\sim \pi_h^t(s_h^t)$
        \STATE Observe $s_{h+1}^t\sim p_h(s_h^t,a_h^t)$
      \ENDFOR
    \STATE{ Update counts and transition estimates.}
   \ENDFOR
   \STATE Output $\hpi$ the uniform mixture of $\{\pi^t\}_{t\in[T]}$.
\end{algorithmic}
\end{algorithm}




\begin{theorem}
\label{th:MVEE_sample_complexity}
Fix  $\epsilon > 0,$  $\delta\in(0,1)$ and $n_0=1.$ Then under the choice 
\[
T = \tcO\left( \frac{H^4 S^2 A}{\varepsilon^2} + \frac{HSA}{\varepsilon} \right)
\]
the algorithm \algMVEE is $(\epsilon,\delta)$-PAC. See Theorem~\ref{th:MVEE_sample_complexity_full} in Appendix~\ref{app:visitation_entropy_proofs} for a precise bound.
\end{theorem}
Thus the sample complexity of \algMVEE is of order $\tcO(H^4S^2A/\epsilon^2)$. In particular, this result significantly improves over the previous rate for MTEE, see Table~\ref{tab:sample_complexity}. Note that, by using Bernstein-like bonuses~\citep{azar2017minimax} instead of Hoeffding-like ones for the sampler-player would gives a sample complexity of order $\tcO(H^3S^2A/\epsilon^2)$ saving one factor $H$. 

\paragraph{Space and time complexity} Since \algMVEE relies on a model-based algorithm for the sampler-player, its space complexity is of order $\cO(HS^2A)$. Because of the value iteration performed by the sampler-player, the time-complexity of one iteration of \algMVEE is of order $\cO(HS^2A)$.

\begin{remark}
\label{rem:hazan_entropy_vs_us} Note that our definition of the visitation entropy slightly differs from the one considered by \citet{hazan2019provably}. Indeed, their definition, translated to the episodic setting, is the entropy of the average of the visitation distributions which is an upper bound on the average of the entropies by concavity of the entropy 
{\small
\[
\cH\left(\frac{1}{H} \sum_{h=1}^H d_h^\pi\right)\geq \frac{1}{H}\VE(d^\pi)\,.\]\vspace{-0.25cm}\\
} 
Even if both definitions make sense in the episodic setting, we think ours is slightly more appropriate in the case of step-dependent transition probabilities. 
Indeed, in this case we want visitation distributions to be close to the uniform distribution over state-action pairs \emph{for all steps}.
 Nevertheless, \EntGame can be adapted to optimize the visitation entropy  used in \citet{hazan2019provably} simply by predicting $\bd_h^t(s,a) = \sum_{h'=1}^H \bn_{h'}^{t-1}(s,a)/(H(t+t_0))$ for the forecaster-player. We conjecture that the sample complexity of this adaptation of \EntGame for the alternative entropy is again of order $\tcO(HS^2A/\epsilon^2)$.
\end{remark}

\paragraph{Comparison with \MaxEnt  and \MetaEnt} All three algorithms, \algMVEE, \MetaEnt\citep{zahavy2021reward}, \MaxEnt\citep{hazan2019provably} rely on the same principle of computing, implicitly or explicitly, the equilibrium of a well chosen game and deduce from it an optimal policy for MVEE. One first difference between \algMVEE and its competitors lies in the choice of the game. While \MetaEnt, \MaxEnt consider the game induced by the Legendre-Fenchel conjugate of a smoothed visitation entropy \citep{zahavy2021reward}, \algMVEE leverages the prediction game which looks more natural for MVEE. One advantage of using this game, is that it allows to avoid the need to smooth the visitation entropy because it is done implicitly by the forecaster-agent with the pseudo-counts. % In fact,  they have to choose a more conservative equivalent of prior count of order $n_0=\cO(\sqrt{T})$ whereas $n_0=1$ is enough for \algMVEE.
More importantly, \MaxEnt and \MetaEnt both needs to accurately estimate at each episode the visitation distributions of the sampler-player $d_h^{\pi^{t}}$, leading to an extra $1/\epsilon^3$ term in the sample complexity. Whereas \algMVEE  needs one trajectory from $\pi^t$ since it only involves the estimation of the averages $1/T \sum_{t=1}^T d_h^{\pi^t}$.

\section{Trajectory entropy}
\label{sec:trajectory_entropy}


In this section we focus on another type of entropy, the trajectory entropy, that can be efficiently maximized. The entropy of paths of a (Markovian) stochastic process is introduced by \citet{ekroot1993entropy}. It quantifies the randomness of realizations with fixed initial and final states. Later it was extended \citep{savas2019entropy}  to realizations that reach a certain set of states, rather than a fixed final state.   This type of entropy is also closely related to the so-called entropy rate of a stochastic process.

\paragraph{Trajectory entropy} We define the trajectory entropy of a policy $\pi$ as the entropy of a trajectory generated with the policy $\pi$ 
\[
\TE(q^\pi) \triangleq \cH(q^\pi) = \sum_{m\in\cT} q^\pi(m) \log\frac{1}{q^\pi(m)}\,.
\]

We denote by $\pistarTE\in\argmax_{\pi} \TE(q^\pi)$ a policy that maximizes the trajectory entropy. 

\paragraph{Maximum trajectory entropy exploration} MTEE differs from MVEE only in the choice of entropy. In particular an algorithm $((\pi^t)_{t\in\N_+},\tau,\hpi)$ for MTEE is also a combination of a time dependent policy $(\pi^t)_{t\in\N_+}$, a stopping rule $\tau$, and a decision rule $\hpi$. 
\begin{definition}
(PAC algortihm for MTEE) An algorithm $((\pi^t)_{t\in\N},\tau,\hpi)$ is $(\epsilon,\delta)$-PAC for MTEE if 
\[
\P\left( \TE\big(q^{\pistarTE}\big)- \TE(q^{\hpi}) \leq \epsilon \right) \leq 1-\delta\,.
\]
\end{definition}

\vspace{-0.25cm}
As noted by \citet{eysenbach2019if}, MTEE can  also be connected to a prediction game. In this game, the forecaster-player aims to predict the whole trajectory that the sampler-player will generate. Remark that predicting the trajectory implies to predict, in particular, the visited state-action pairs but the reverse is not true in general \footnote{Indeed $d_h^\pi$ are only the marginals of $q^\pi$.}. We could then apply the same strategy as in Section~\ref{sec:visitation_entropy} to solve MTEE. Nevertheless, for trajectory entropy, there is a more direct way to proceed.

\paragraph{Entropy regularized Bellman equations} One big difference between MVEE and MTEE is that the optimal policy can be obtained by solving regularized Bellman equations. Indeed, thanks to the chain rule for the entropy, the trajectory entropy of a policy $\pi$ is $\TE(d^\pi) = V_1^\pi(s_1)$ and the maximum trajectory entropy is $\TE\big(d^{\pistarTE}\big)  = \Vstar_1(s_1)$ where the value functions $V^\pi$ and $\Vstar$ satisfy
{\small
\begin{align*}
	Q_h^{\pi}(s,a) &= \cH\big(p_h(s,a)\big) + p_h V_{h+1}^\pi(s,a) \,,\\
    V_h^\pi(s) &= \pi_h Q_h^\pi (s) +\cH\big(\pi_h(s)\big)\,,\\
  Q_h^\star(s,a) &=  \cH\big(p_h(s,a)\big) + p_h V_{h+1}^\star(s,a) \,,\\
  V_h^\star(s) &= \max_{\pi\in\Delta_A}  \pi Q_h^\star (s) + \cH(\pi)\,,
\end{align*}}

\vspace{-0.4cm}
\!where by definition, $V_{H+1}^\star \triangleq V_{H+1}^\pi \triangleq 0$. In particular, the maximum trajectory entropy policy is given by $\pistarTE_h(s) = \argmax_{\pi\in\Delta_A} (\pi\Qstar(s)+\cH(\pi))$.
It can be computed explicitly via $\pistarTE_h(a|s) = \exp\!\big(\Qstar_h(s,a) -\Vstar_h(s)\big)$ as well as the optimal value function $\Vstar_h(s) = \log\left(\sum_{a\in\cA} \rme^{\Qstar_h(s,a)}\right)$.

We now describe our algorithm \algMTEE for MTEE. Since one only needs to solve  regularized Bellman equations to obtain a maximum trajectory entropy policy, we can use an algorithm of the same flavor as the ones proposed for best policy identification \citep{tirinzoni2022optimistic, menard2021fast, kaufmann2020adaptive, dann2019policy}. In particular, \algMTEE is close to the \BPIUCRL algorithm by \citet{kaufmann2020adaptive} and can be characterized by the following rules.


\paragraph{Sampling rule} As sampling rule we use an optimistic policy $\pi^{t+1}$ obtained by optimistic planning in the regularized MDP
{\small
\begin{align}
  \uQ_h^t(s,a) &=  \clip\Big(\cH\big(\hp_h(s,a)\big)+b_h^{\cH,t}(s,a) \nonumber\\
  &\quad+ \hp^t_h \uV_{h+1}^t(s,a)+ b_h^{p,t}(s,a),0,\log(SA)H\Big)\,,\nonumber\\
  \uV_h^t(s) &= \max_{\pi\in\Delta_A}  \pi \uQ_h^t (s) + \cH(\pi)\,,\label{eq:optimistic_planning_TE}\\
  \pi_h^{t+1}(s) &= \argmax_{\pi\in\Delta_A}  \pi \uQ_h^t (s) + \cH(\pi)\nonumber\,,
\end{align}}
with $\uV^t_{H+1} = 0$ by convention  where $b^{\cH,t}$, $b^{p,t}$ are bonuses for the entropy  and the transition probabilities, respectively. Precisely we use bonuses of the form 
{\scriptsize
\begin{align*}b_h^{\cH,t}(s,a) &=  \sqrt{\frac{2 \beta^{\cH}(\delta, n^t_h(s,a))}{n^t_h(s,a)}} + \min\left(\frac{\beta^{\KL}(\delta, n^t_h(s,a))}{n^t_h(s,a)},\, \log(S) \!\!\right)\,,\\
b_h^{p,t}(s,a) &= \sqrt{\frac{2H^2\log^2(SA) \beta^{\conc}(\delta, n^t_h(s,a))}{n^t_h(s,a)}}\,,
\end{align*}}
for some functions $\beta^{\cH}, \beta^{\KL}$ and $\beta^{\conc}$.

\paragraph{Stopping and decision rule} To define the stopping rule, we first recursively build an upper-bound on the difference between the value of the optimal policy and the value of the current policy $\pi^{t+1}$, 
{\scriptsize
\begin{align}
    W_h^t(s,a) &=  4H^2 \log(SA) \frac{\beta^{\KL}(\delta, n^t_h(s,a))}{n^t_h(s,a)} + \left( \!1\! +\! \frac{1}{H}\right) \hp^{\,t}_h G^{t}_{h+1}(s,a)\,,\nonumber\\
    G^{t}_{h}(s) &= \min\!\bigl( \pi_h^{t+1} W_h^t(s)+ \!\frac{1}{2} \big(\uV^t_h(s) - \lV^t_h(s)\!\big)^2\!\!,\, \log(SA)H \bigl)\,,\label{eq:upper_bound_gap}
\end{align}\\}

\vspace{-1.2cm}
where $\lV^t$ is a lower-bound on the optimal value function defined in Appendix~\ref{app:regularized_mdp} and $G_{H+1}^t = 0$ by convention.
Then the stopping time $\tau= \inf\{ t \in \N : G^t_{1}(s_1)  \leq \varepsilon \}$ corresponds to the first episode when this upper-bound is smaller than $\epsilon$. At this episode we return the Markovian policy $\hpi = \pi^{\tau+1}$.

\begin{algorithm}[h!]
\centering
\caption{\algMTEE}
\label{alg:UCBVIEnt}
\begin{algorithmic}[1]
  \STATE {\bfseries Input:} Target precision $\epsilon$, target probability $\delta$, threshold functions $\beta^{\cH},\beta^{p}$, $\beta^{\conc}$.
      \WHILE{ true}
       \STATE Compute $\pi^{t}$ by optimistic planning with \eqref{eq:optimistic_planning_TE}.
       \STATE Compute bound on the gap $G_1^{t-1}(s_1)$ with \eqref{eq:upper_bound_gap}.
       \STATE \textbf{if} $G_1^{t-1}(s_1)\leq \epsilon$ \textbf{then break}
      \FOR{$h \in [H]$}
        \STATE Play $a_h^t\sim \pi_h^t(s^t_h)$
        \STATE Observe $s_{h+1}^t\sim p_h(s_h^t,a_h^t)$
      \ENDFOR
    \STATE{ Update counts $n^t$, transition estimates $\hp^t$ and episode number $t\gets t+1$.}
   \ENDWHILE
   \STATE Output policy $\hpi = \pi^t$.
\end{algorithmic}
\end{algorithm}
The complete procedure is described in Algorithm~\ref{alg:UCBVIEnt}. We now state our main theoretical result for \algMTEE. We prove that for the well calibrated threshold functions $\beta^{\cH}, \beta^{\KL}$ and $\beta^{\conc},$ the \algMTEE is  $(\epsilon, \delta)$-PAC for MTEE and  provide a
high-probability upper bound on its sample complexity. The next result is proved in Appendix~\ref{app:regularized_mdp}.
\begin{theorem}
\label{th:ucbvi_sample_complexiy}
Let $\beta^{\KL}, \beta^{\conc}$ and $\beta^{\cH}$ be defined in Lemma~\ref{lem:proba_master_event} of Appendix~\ref{app:regularized_mdp}. Fix  $\epsilon>0$ and $\delta\in(0,1),$ then the \algMTEE algorithm is $(\epsilon,\delta)$-PAC for MTEE.   Moreover, the optimal policy is given by $\hpi = \pi^{\tau+1}$ where 
\[
\tau = \tcO\left(\frac{H^6S^2A}{\epsilon}\right).
\]
with probability at least $1-\delta.$
Here $\tcO$ hides poly-logarithmic factors in $\epsilon, \delta, H,S,A$. See Theorem~\ref{th:mtee_sample_complexity} in Appendix~\ref{app:regularized_mdp} for a precise bound.
\end{theorem}



\paragraph{Space and time complexity} Since \UCBVIEnt is a model based algorithm, its space-complexity is of order $\cO(HS^2A)$ whereas its time-complexity for one episode is of order $\cO(HS^2A)$ because of the optimistic planning.

\vspace{0.25cm}
\begin{remark} 
\label{rem:regularized_mdp}
(On solving regularized MDPs) Interestingly, our approach for MTEE can be adapted to solve entropy-regularized MDPs. For a reward functions $(r_h)_{h\in[H]}$ and regularization parameter $\lambda>0,$ consider the regularized Bellman equations 
{\small
\begin{align*}
    Q^{\pi}_{\lambda,h}(s,a) &= r_h(s,a) + p_h V^{\pi}_{\lambda,h+1}(s,a)\,,\\
    V^\pi_{\lambda,h}(s) &= \pi_h Q^\pi_{\lambda,h}(s) + \lambda \cH\big(\pi_h(s)\big)\,,\\
    \Qstar_{\lambda,h}(s,a) &= r_h(s,a) + p_h \Vstar_{\lambda,h+1}(s,a)\,,\\
    \Vstar_{\lambda,h}(s) &= \max_{\pi\in\simplex_A} \pi\Qstar_{\lambda,h}(s) + \lambda \cH(\pi)
\end{align*}}

where $V_{\lambda, H+1}^\pi = \Vstar_{\lambda, H+1} = 0$. Note that these are the Bellman equations used by \SoftQlearning \citep{fox2016taming,schulman2017equivalence,haarnoja2017reinforcement} and \SAC \citep{haarnoja2018soft} algorithms. We are interested in the best policy identification for this regularized MDP. That is finding an algorithm that will output an $\epsilon$-optimal policy $\hpi$ such that with probability $1-\delta$, it holds $\Vstar_{\lambda,1}(s_1) - V^{\hpi}_{\lambda,1}(s_1) \leq \epsilon$ after a minimal number $\tau$ of trajectories sampled from the MDP $\cM^r = (\cS,\cA, H ,(p_h)_{h\in[H]},(r_h)_{h\in[H]},s_1)$. 
By using  similar sampling  and stopping rules as in \algMTEE, we get an algorithm for BPI in the entropy-regularized MDP that also enjoys the fast rate of order $\tcO\big(H^6S^2A/(\epsilon \lambda)\big)$. Refer to Appendix~\ref{app:regularized_mdp} for precise statements and proofs.

We observe that the sample complexity for solving the regularized MDP is strictly smaller\footnote{For small enough $\epsilon$.} than the sample complexity for solving the original MDP. Indeed, one needs at least $\tcO(H^3SA/\epsilon^2)$ trajectory \citep{domingues2021episodic} to learn a policy $\pi$ which value in the (unregularized) MDP is $\epsilon$-close to the optimal value. Nevertheless, regularizing the MDP introduces a bias in the value function. Precisely we have for all $\pi$, $0\leq V^\pi_{\lambda,1}(s_1)-  V^\pi_1(s_1) \leq \tcO(\lambda H)$ where 
$V^\pi_1(s_1)$ is the value function of $\pi$ at the initial state and MDP $\cM^r$. Thus, to solve BPI in $\cM^r$ through BPI in the regularized MDP, one needs to take $\lambda = \tcO(1/(H\epsilon))$, leading to a sample complexity of order $\tcO(H^7S^2A/\epsilon^2)$. In particular, our fast rate for BPI in regularized MDP does not contradict the lower bound for BPI in the original MDP. However, our analysis shows that regularization is an effective way to trade-off bias for sample complexity.
\end{remark}


\paragraph{Visitation entropy vs trajectory entropy}
We can compare the visitation entropy and the trajectory entropy with
\[
\TE(q^\pi)\leq  \underbrace{\KL(q^\pi,\otimes_{h=1}^H d^\pi_h) + \TE(q^\pi)}_{\VE(d^\pi)} \leq H \TE(q^\pi)\,,
\]
see Lemma~\ref{lem:comparison_ent_traj_visit} in Appendix~\ref{app:technical} for a proof. Note also that in general the visitation distributions of an optimal policy for maximum trajectory entropy will be less 'spread' than the one of an optimal policy for MTEE, see Section~\ref{sec:experiments} for an example. In particular one can prove that the optimal policy for MTEE is the uniform policy if the transitions are deterministic, see Lemma~\ref{lem:MTEE_deterministic} of Appendix~\ref{app:technical}.







%\todopp{I am not sure I actually used the contraction property of the entropy for the 2nd inequality, but rather a "grouping" property of the entropy (for me contraction is something like H(X|Y)<=H(X)). If the grouping argument stated in appendix is fine, then we should change the word "contraction" by "grouping" here, otherwise we have to change the proof.}





\subsection{Proof sketch}
In this section we sketch the proof of Theorem~\ref{th:mtee_sample_complexity}. 

\paragraph{Properties of entropy}

We start from analysing several properties of the entropy function. First, we notice that the well-known log-sum-exp function is a convex conjugate to the negative entropy defined only on the probability simplex \cite{boyd2004convex}
{\small
\[
    F(x) \triangleq \log\left( \sum_{a \in \cA} \rme^{x_a} \right) = \max_{\pi \in \simplex_A} \langle \pi, x \rangle + \cH(\pi)
\]}
% \todoPi{This is not the usual convex conjugate since we are optimizing on the simplex.}
% \todoDa{Added note that we define entropy on the the probability simplex, thus it is a correct formula for conjugate.}
and extend its action to $Q$-functions
{\small
\[
    F(Q)(s) \triangleq \max_{\pi \in \simplex_A} \pi Q(s) + \cH(\pi). 
\]
}
This definition is useful because we can rewrite the optimal value function for MTEE and $\uV^t$ as follows
\begin{equation}\label{eq:V_star_using_F}
    \Vstar_{h}(s) = F(\Qstar_h)(s), \quad \uV^t_h(s) = F(\uQ^t_h)(s).
\end{equation}
Additionally, we notice that the gradient of $F$ is equal to the soft-max policy that maximizes the expressions above
\[
    \pistar_h(s) = \nabla F(\Qstar_h)(s) , \quad \pi^{t+1}_h(s) = \nabla F(\uQ^t_h)(s),
\]
and, moreover, since the negative entropy $-\cH(\pi)$ is $1$-strongly convex with respect to $\ell_1$ norm, gradients of $F$ is $1$-Lipschitz with respect to $\ell_\infty$ norm by the properties of the convex conjugate \cite{kakade2009duality}. Combining the gradient properties with  the smoothness definition to $\Qstar$ and $\uQ$ we obtain
\begin{small}
\begin{align}\label{eq:F_smooth_Q}
    \begin{split}
    F(\Qstar_h)(s) &\leq F(\uQ^t_h)(s) + \pi^{t+1}_h\left( \Qstar_h- \uQ^t_h\right)(s) \\
    &+ \frac{1}{2} \norm{ \Qstar_h - \uQ^t_h}^2_\infty(s),
    \end{split}
\end{align}\end{small}
where \begin{small}$\norm{ \Qstar_h - \uQ^t_h}_\infty(s) = \max_{a\in\cA} \vert \Qstar_h(s,a) -\uQ^t_h(s,a) \vert$.\end{small}
% \todoPi{We can say directly that since $\cH$ is strongly convex then it's convex conjugate is smooth. Btw why it is not 1 the smoothness constant (inverse of the strongly convex)?}
% \todoDa{As I remember, strong convexity defined as $\mu$ such that $f(y) \geq f(x) - \langle \nabla f(x), y-x \rangle + \frac{\mu}{2} \norm{y-x}^2$ and then $\mu$ is strong convexity constant (for entropy $\mu=1$ is from Pinsker's inequality), and the smoothness is also defined with $1/2$-constant. Added a reference with exact formulations. }
% \todoPi{Maybe we can just state this last inequality and avoid to describe precisely all the properties of entropy to save space?}
% \todoDa{Removed the smoothness definition, but I think that showing representation of V-functions and policies using F is crucial.}

\paragraph{Bound on the policy error}

Next we apply the key inequality \eqref{eq:F_smooth_Q} to analyze the error between the optimal policy and a policy $\pi^{t+1}$. Using~\eqref{eq:V_star_using_F}
\begin{small}
\begin{align*}
    \Vstar_h(s) - V^{\pi^{t+1}}_h(s) &\leq \uV^t_h(s) - V^{\pi^{t+1}}_h(s)  + \pi^{t+1}_h \big(\Qstar_h - \uQ^t_h \big)(s) \\
    &+ \frac{1}{2} \max_{a \in \cA} \left( \uQ^t_h(s,a) - \Qstar_h(s,a) \right)^2.
\end{align*}
\end{small}

\vspace{-0.4cm}
Next, by definition of $\pi^{t+1}$  we have
\begin{small}
$
    \uV^t_h(s) = \cH(\pi^{t+1}_h(s)) +  \pi^{t+1}_h \uQ^t_h(s),
$
\end{small}
therefore by the regularized Bellman equations
\begin{small}
\begin{align}\label{eq:policy_error_ineq}
    \begin{split}
    \Vstar_h(s) - V^{\pi^{t+1}}_h(s) & \leq \pi^{t+1}_h p_h \left( \Vstar_{h+1} - V^{\pi^{t+1}}_{h+1} \right)(s) \\
    &+\frac{1}{2} \max_{a \in \cA} \left( \uQ^t_h(s,a) - \Qstar_h(s,a) \right)^2,
    \end{split}
\end{align}\\
\end{small}

\vspace{-0.9cm}
and, finally, since our estimate $\uQ$ is optimistic on some high-probability event by construction of bonuses, we can show 
\begin{small}$
    \max\limits_{a \in \cA} \left( \uQ^t_h(s,a) - \Qstar_h(s,a) \right)^2 \leq \big(\uV^t_h(s) - \lV^t_h(s)\big)^2,
$\end{small}
where $\lV^t_h$ is a lower confidence bound on $\Qstar_h$. In other words, in the regularized setting we are able to propagate the policy error from step $h$ to step $h+1$ by the cost of only a squared over-estimation error. It is the most crucial part of the proof because this square allows us to make all the first-order terms of type $1/\sqrt{n^t_h(s,a)}$  the second-order terms of type $1/n^t_h(s,a)$ and obtain fast rates.

To finalize all these formulas to the computable decision rule, we notice that we can exchange $p_h$ with $\hp^t_h$ in the right-hand side of \eqref{eq:policy_error_ineq} by Lemma~\ref{lem:reg_directional_concentration} paying the second-order term and $(1+1/H)$-factor. Overall, we obtain
\begin{small}
\begin{equation}\label{eq:policy_error_via_gap}
    \Vstar_1(s_1) - V^{\pi^{t+1}}_1(s_1) \leq G^t_1(s_1)
\end{equation}\\
\end{small}

\vspace{-0.9cm}
for $G^t_h(s)$ defined in \eqref{eq:upper_bound_gap}. This inequality automatically justifies that \UCBVIEnt is $(\varepsilon,\delta)$-PAC of MTEE because for $\hpi = \pi^{\tau+1},$ we have $G^{\tau}_1(s_1) \leq \varepsilon$, and \eqref{eq:policy_error_via_gap} for $t=\tau$ concludes the statement.

% \begin{remark}
% Another key difference between our approach and approaches to identifying the best policy in ordinary MDPs \citep{menard2021fast, kaufmann2020adaptive, dann2019policy} is that smoothing the log-sum-exp function allows us to define the upper bounds on the policy error not for the difference in $Q$-functions, but only for the policy error.
% \end{remark}

\begin{remark}
Another key difference between our approach and BPI is that smoothness of the log-sum-exp function allows us to define the upper bounds on the policy error not in terms of the difference of $Q$-functions, but of value-functions.
\end{remark}


% \paragraph{Sample complexity}
% \todoDe{May give too many details in this paragraph ...}
% To derive an upper bound on sample complexity we have to provide an upper bound on $G^t_1(s_1)$. To do it, we can change  $\hp^t_h$ back to $p_h$ with a second-order penalty and by rolling out conditional expectations
% \[
%     G^t_1(s_1) \leq \E_{\pi^{t+1}}\biggl[ \sum_{h=1}^H \frac{\tcO(H^2S)}{n^t_h(s_h,a_h)} + \frac{\rme^2}{2}(\uV^t_h - \lV^t_h)^2(s_h)  \Big| s_1 \biggl].
% \]
% Next we analyze the second term. The difference $\uV^t_h - \lV^t_h$ is a standard object in the analysis of \UCBVI-like algorithms and could be upper bounded as follows
% \begin{small}
%     \begin{align*}
%         (\uV^t_h &- \lV^t_h)(s_h) \leq \sqrt{\E_{\pi^{t+1}}\left[ \sum_{h'=h}^H \frac{\tcO(H^3)}{n^t_{h'}(s_{h'},a_{h'})} \bigg| s_h\right]} \\
%         &+ \tcO(H) \cdot \E_{\pi^{t+1}}\left[ \sum_{h'=h}^H \min\left(\frac{\tcO(HS)}{n^t_{h'}(s_{h'},a_{h'})}, 1\right) \bigg| s_h\right].
%     \end{align*}\\\end{small}
% In the setting of unregularized MDPs the term that is similar to the first one in the upper bound above has the leading role because it scales as sum of $1/\sqrt{n^t_h(s,a)}$ and leads to $\tcO(1/\varepsilon^2)$ sample complexity, In our case, this term becomes the second-order one too that leads to improvement in sample complexity.
    
% Taking the square and applying Jensen's inequality several times we can obtain
% \[
%     G^t_1(s_1) \leq \E_{\pi^{t+1}}\left[ \sum_{h=1}^H \frac{\tcO(H^5 S)}{n^t_h(s,a)} \bigg| s_1  \right].
% \]
% Notice that for all $t < \tau$ we have $G^t_1(s_1) > \varepsilon$, thus summing up the inequality above for all $t < \tau$
% \[
%     (\tau-1)\varepsilon \leq  \sum_{h,s,a} \sum_{t=1}^{\tau-1} d^{\pi^{t+1}}_h(s,a) \frac{\tcO(H^5 S)}{n^t_h(s,a)}  = \tcO\left( H^6 S^2 A \right),
% \]
% that yields the fast rate for the MTEE problem.


\vspace{-0.25cm}
\section{Faster Rates for Visitation Entropy}
\label{sec:faster_rates_mvee}


In this section, we show how to combine the regularization techniques developed in Section~\ref{sec:trajectory_entropy} with \EntGame algorithm presented in Section~\ref{sec:MVEE_game}.

The new algorithm \regalgMVEE is based on exactly the same game-theoretical framework as \EntGame, but uses a \textit{regularized sampler player} instead of the usual one.


\paragraph{Regularized sampler-player} For the sampler player, we shall take advantage of strong convexity of the visitation entropy. Beforehand, we construct an estimate of the model $\{ \hp_h \}_{h\in[H]}$ by reward-free exploration, using $HSN_0$ samples to compute a policy $\pi^{\mathrm{mix}}$ and $N$ samples to estimate transitions. Next, define the empirical regularized Bellman equations 
{\small
\begin{align*}
\begin{split}
\hQ_h^t(s,a) &=  \log\left(\frac{1}{\bd_h^{t+1}(s)}\right) + \hp_h \hV^t_{h+1}(s,a),\\
\hV_h^t(s) &= \max_{\pi\in\Delta_{\cA}}\{ \pi \hQ_h^t(s) + \cH(\pi) \},
\end{split}
\end{align*}\\}
where  $\hV_{H+1}^t = 0$. The sampler player then follows the distribution $d^{\pi^{t+1}}$ where $\pi^{t+1}$ is greedy with respect to the regularized empirical Q-values, that is, $\pi_h^{t+1}(s) \in\argmax_{\pi\in\Delta_A}\{ \pi\hQ_h^t(s) + \cH(\pi) \}$.


\begin{theorem}
\label{th:fast_MVEE_sample_complexity}
Fix $\epsilon > 0$ and $\delta\in(0,1)$. For $n_0=1,$ 
\begin{small}
\[
    N_0 = \Omega\left( \frac{H^7 S^3 A  \cdot L^3}{\varepsilon}\right),
    \quad
    N = \Omega\left( \frac{ H^6 S^3 A L^5 }{\varepsilon}\right), 
    \]
\end{small}
\!and \begin{small}
\[
    T = \Omega\left( \frac{H^3 S A L^3}{\varepsilon^2} + \frac{H^2 S^2 A^2 L^2}{\varepsilon}\right).
\]
\end{small}
\!with $L = \log(SAH/\delta\varepsilon)$ the algorithm \regalgMVEE is $(\epsilon,\delta)$-PAC. The total sample complexity is equal to $SH \cdot N_0 + N + T,$ that is,
\begin{small}
\[
    \tau = \tcO\left( \frac{H^2 SA}{\varepsilon^2} + \frac{H^8 S^4 A}{\varepsilon} \right).
\]
\end{small}
\end{theorem}

\vspace{-0.2cm}
\!Thus, the sample complexity of \regalgMVEE is of order $\tcO(H^2SA/\epsilon^2)$ for $\varepsilon$ large enough. In particular, this result significantly improves over  the previous rates for MVEE, see Table~\ref{tab:sample_complexity}. Moreover, this result shows a rate separation between reward-free exploration \citep{jin2020reward-free}, where the established lower bound on sample complexity $\Omega(H^3S^2A/\varepsilon^2)$ scales with $S^2$, and the visitation entropy maximization problem. 

\paragraph{Proof idea} The main proof idea is to exploit not just strong convexity of the visitation entropy with respect to Euclidean distance but its strong convexity \textit{with respect to trajectory entropy} \citep{bauschke2016descent}. It allows us to use entropy regularization as described in Section~\ref{sec:proof_mtee} for the sampler player resulting in an averaged regret less than $\varepsilon$ for only $\tcO(\poly(S,A,H)/\varepsilon)$ samples. Thus, the density estimation error becomes the leading term in the full error decomposition. For more details refer to Appendix~\ref{app:reg_visitation_entropy_proofs}.

We present in section~\ref{ssec:faces} an application of PnP-HVAE on face images, using a pretrained state-of-the-art hierarchical VAE. 
Next, we study the application of our framework to natural images. To that end, we introduce  in section~\ref{ssec:patchVDVAE}  a patch hierachical VAE architecture, that is able to model natural images of different resolutions. In section~\ref{ssec:app_nat}, we provide deblurring, super-resolution and inpainting experiments to demonstrate the relevance of the proposed method.

Additional results are presented in Appendix~\ref{app:add}. All experiments can be reproduced using the code available at \url{https://github.com/jprost76/PnP-HVAE}.



\subsection{Face Image restoration (FFHQ)}\label{ssec:faces}
We first demonstrate the effectiveness of PnP-HVAE on highly structured data, by performing face image restoration.
Latent variable generative models can accurately model structured images such as face images \cite{karras2019style,vahdat2020nvae,child2021very,kingma2018glow}, and then be used to produce high quality restoration of such data. 
In our experiments, we use the VDVAE model of~\cite{child2021very}, pre-trained on the FFHQ dataset~\cite{karras2019style}, as our hierarchical VAE prior.
VDVAE has $L=66$ latent variable groups in its hierarchy and generates images at resolution $256\times256$.

We compare PnP-HVAE with the intermediate layer optimization algorithm (ILO)~\cite{daras2021intermediate} that is based on a different class of generative models than HVAE. ILO is a GAN inversion method which optimizes the image latent code along with the intermediate layer representation of a StyleGAN to generate an image consistent with a degraded observation.
We use the official implementation of ILO, along with a StyleGAN2 model~\cite{karras2020analyzing, stylegan2pytorch}, that was trained for 550k iterations on images of resolution $256\times256$ from FFHQ.  
As VDVAE and StyleGAN models are not trained on the same train-test split of FFHQ, we chose to evaluate the methods on a subset of 100 images from the CelebA dataset~\cite{liu2018large}. 
For super-resolution, the degradation model corresponds to the application of a gaussian low-pass filter followed by a $\times 4$ sub-sampling, and the addition of a gaussian white noise with $\sigma=3$.
For the deblurring, we considered motion blur and  gaussian kernels, both with a noise level $\sigma=8$. %

We provide quantitative comparisons in table~\ref{table:comp_ILO}, along with a visual comparison of the results in figure~\ref{fig:face_restoration}.
PnP-HVAE has the best  PSNR and SSIM results for all the considered restoration tasks, while ILO provides better results  for the perceptual distance.
By jointly optimizing the image and its latent variable, PnP-HVAE provides  results that are both realistic and consistent with the degraded observation.
On the other hand,  ILO  only optimizes on an extended latent space. This method generates  sharp and realistic images with better LPIPS scores,   
but the results lack  of consistency with respect to the observation, which explains the overall lower PSNR performance. 






\subsection{PatchVDVAE: a HVAE for natural images}\label{ssec:patchVDVAE}
Available generative models in the literature operate on images of  fixed resolutions and
are either restrained to datasets of limited diversity, or even to registered face images~\cite{kingma2018glow,child2021very, vahdat2020nvae, karras2019style}, or requiring additional class information~\cite{brock2018large, dhariwal2021diffusion, song2020score, luhman2022optimizing}.
Fitting an unconditional model on natural images appears to be a more difficult task, as their resolution can change, and their content is highly diverse.
The complexity of the problem can be reduced by learning a prior model on patches of reduced dimension. 
For image restoration problems, the patch model can be reused on images of higher dimensions~\cite{zoran2011learning,prost2021learning,altekruger2022patchnr}. When the model is a full CNN, the prior on the set of the  patches can  be computed efficiently by applying the network on the full image~\cite{prost2021learning}.

We thus introduce  patchVDVAE, a fully convolutional hierarchical VAE.
Contrary to existing HVAE models whose resolution is constrained by the constant tensor at the input of the top-down block, patchVDVAE can generate images of different resolutions by controlling the dimension of the input latent. 
This amounts to defining a prior on patches whose dimension corresponds to the receptive field of the VAE. A similar model is used for image denoising in~\cite{prakash2021interpretable}.

 
For PatchVDVAE architecture, we use the same bottom-up and top-down blocks as VDVAE~\cite{child2021very}, and replace the constant trainable input in the first top-down block by a latent variable, to make the model fully convolutional (details on the  architecture are given in Appendix~\ref{app:details}). 
The training dataset is composed of $128\times 128$ patches extracted from a combination of DIV2K~\cite{agustsson2017ntire} and Flickr2K~\cite{Lim_2017_CVPR_workshops} datasets.
We perform data augmentation by extracting  patches at $3$ resolutions: HR-images and $\times 2$ and $\times 4$ downscaled images. 
The model is trained for $7.10^5$ iterations with a batch size of $64$. Following the recommendation of~\cite{hazami2022efficient}, we use Adamax optimizer with an exponential moving average and gradient smoothing of the variance.
We set the decoder model to be a gaussian with diagonal covariance, as in~\cite{luhman2022optimizing}.
PatchVDVAE is fully convolutional and can generate images of dimension that are multiples of $64$ as illustrated by
figure~\ref{fig:vdvae}.

\newlength{\patchwidth}
\setlength{\patchwidth}{0.135\columnwidth}
\begin{figure}[!ht]
    \centering
    \begin{subfigure}[t]{.34\columnwidth}\hspace{0.1cm}
        \setlength{\tabcolsep}{0.02pt}
\renewcommand{\arraystretch}{0}
        \begin{tabular}{*{2}{p{1.03\patchwidth}}}
            \includegraphics[width=\patchwidth]{figures_arxiv/patchVDVAE/samples/generated/64x64/setup-5-image-0018.png} &
            \includegraphics[width=\patchwidth]{figures_arxiv/patchVDVAE/samples/generated/64x64/setup-5-image-0016.png} \\
            \includegraphics[width=\patchwidth]{figures_arxiv/patchVDVAE/samples/generated/64x64/setup-5-image-0008.png} &
            \includegraphics[width=\patchwidth]{figures_arxiv/patchVDVAE/samples/generated/64x64/setup-5-image-0019.png}   
        \end{tabular}
    \end{subfigure}\hspace{-0.15cm}
    \begin{subfigure}[t]{.64\columnwidth}
\begin{tabular}{cc}\vspace{-0.1cm}
\includegraphics[width=2\patchwidth]{figures_arxiv/patchVDVAE/samples/generated/256x256/setup-2-image-0009.png}&
        \includegraphics[width=2\patchwidth]{figures_arxiv/patchVDVAE/samples/generated/256x256/setup-2-image-0002.png}\end{tabular}

    \end{subfigure}
    \caption{\label{fig:vdvae} Left: $64\times64$ patches samples from our patchVDVAE model trained on patches from natural images.
    Right: PatchVDVAE is fully convolutional and it can generate images of higher resolution (here: $128\times128$).\vspace{-0.2cm}}
\end{figure}

\subsection{Natural images restoration}\label{ssec:app_nat}
We  evaluate PnP-HVAE on natural image restoration.
For each task, we report the average value of the PSNR, the SSIM, and the LPIPS metrics on $20$ images from the test set of the BSD dataset~\cite{MartinFTM01}.\\


\noindent
{\bf Image deblurring.}
In the experiments, we consider $2$ gaussian kernels and $2$ motion blur kernels from~\cite{levin2009understanding}, with $3$ different noise levels 
$\sigma \in \{2.55, 7.65, 12.75\}$.
As a baseline we consider  EPLL~\cite{zoran2011learning}, which learns a prior on image patches with a gaussian mixture model.
We also compare PnP-HVAE  with PnP-MMO and GS-PnP, $2$ competing convergent Plug-and-Play methods based on CNN denoisers.
PnP-MMO~\cite{pesquet2021learning} restricts the denoiser to be contraction in order to guarantee the convergence of the PnP forward-backard algorithm. GS-PnP~\cite{hurault2022gradient} considers a gradient step denoiser and reaches state-of-the-art performances of non converging methods~\cite{zhang2021plug}.
We set the temperature $\tau$  in our method as $0.95$, $0.8$ and $0.6$ for noise levels $2.55$, $7.65$ and $12.75$ respectively, and we let it run for a maximum of $50$ iterations. 
For the three compared methods we use the official implementations and pre-trained models provided by the respective authors. 
Details on the choice of hyperparameters for the concurrent methods are provided in the Appendix~\ref{app:details}
Figure~\ref{fig:deblurring_bsd} illustrates that our method provides correct deblurring results. 

According to table~\ref{tab:deb}, the performance of PnP-HVAE is between those of EPLL and GS-PnP and it outperforms PnP-MMO for large noise levels.\\

\begin{table}
\begin{center}\footnotesize
    \begin{tabular}{>{\centering}m{.3cm}*{5}{c}}
    $\sigma$ &Method & PSNR$\uparrow$ & SSIM$\uparrow$ & LPIPS$\downarrow$  \\ 
    \hline
    \multirow{4}{*}{\vcell{$2.55$}}
    & PnP-HVAE & $27.75$ & $0.79$ & $0.31$\\
    & GS-PNP \cite{hurault2022gradient} & $\mathbf{29.59}$ & $\mathbf{0.84}$ & $\mathbf{0.22}$\\
    & EPLL \cite{zoran2011learning} & $26.49$ & $0.71$ & $0.36$\\ 
    & PnP-MMO \cite{pesquet2021learning} & $\underbar{29.50}$ & $\underbar{0.83}$ & $\underbar{0.20}$ \\ \hline
    \multirow{4}{*}{\vcell{$7.65$}}
    & PnP-HVAE & $\underbar{26.36}$ & $\underbar{0.72}$ & $\underbar{0.40}$\\
    & GS-PNP \cite{hurault2022gradient} & $\mathbf{27.33}$ & $\mathbf{0.77}$ & $\mathbf{0.31}$\\
    & EPLL \cite{zoran2011learning} & $24.04$ & $0.66$ & $0.45$ \\ 
    & PnP-MMO \cite{pesquet2021learning} & $25.34$ & $0.69$ & $0.34$\\
    \hline
    \multirow{4}{*}{\vcell{$12.75$}}
    & PnP-HVAE & $\underbar{25.12}$ & $\mathbf{0.73}$ & $\underbar{0.47}$\\
    & GS-PNP \cite{hurault2022gradient} & $\mathbf{26.32}$ & $\mathbf{0.73}$ & $\mathbf{0.37}$\\
    & EPLL \cite{zoran2011learning} & $23.28$ & $0.61$ & $0.51$ \\ 
    & PnP-MMO \cite{pesquet2021learning} & $22.42$ & $0.53$& $0.54$ \\
    \hline
    &\vspace*{-.3cm}\\
            \multicolumn{2}{c}{Blur and motion kernels}& \multicolumn{3}{c}{
        \includegraphics*[scale=1]{figures_arxiv/kernels/4.png}\;\includegraphics*[scale=1]{figures_arxiv/kernels/7.png}\;\includegraphics*[scale=1]{figures_arxiv/kernels/9.png}\;\includegraphics*[scale=1]{figures_arxiv/kernels/11.png}} 
    \end{tabular}
        \caption{\label{tab:deb}Comparison  of PnP-HVAE  and other restoration methods on deblurring. Results are averaged on $4$ kernels.\vspace{-0.2cm}}% on image deblurring.}
    \end{center}
\end{table}

\begin{figure}
    
    \begin{subfigure}[h]{\linewidth}
        \centering
        \includegraphics*[width=\columnwidth]{figures_arxiv/deb_s255_k7.pdf}\vspace{-0.1cm}
        \caption{Gaussian blur, $\sigma=2.55$}
    \end{subfigure}
    \begin{subfigure}[h]{\linewidth}
        \centering
        \includegraphics*[width=\columnwidth]{figures_arxiv/deb_s765_k11.pdf}\vspace{-0.1cm}
        \caption{Motion blur, $\sigma=7.65$}
    \end{subfigure}\vspace*{-0.1cm}
    \caption{\label{fig:deblurring_bsd} Natural image deblurring\vspace{-0.1cm}}
\end{figure}

\noindent {\bf Effect of the temperature.}
PnP-HVAE gives control on the temperature of the prior over the latent space.
In figure~\ref{fig:temp_effect}, we illustrate that reducing the temperature increases the strength of the regularization prior. In this example the tuning $\tau=0.7$ produces the best performance.\\
\begin{figure}[!ht]
   
    \includegraphics[width=\columnwidth]{figures_arxiv/demo_temp.pdf}\vspace{-0.15cm}
    \caption{ \label{fig:temp_effect} Effect of the temperature in PnP-VAE on a deblurring problem, with $\sigma=7.65$.\vspace{-0.15cm}}
\end{figure}


\noindent
{\bf Image inpainting.}
Next we consider the task of noisy image inpainting. 
We compose a test-set of 10 images from the validation set of BSD~\cite{MartinFTM01} and we create masks
  by occluding diverse objects of small size in the images. 
A gaussian white noise with $\sigma=3$ is added to the images.
As a comparaison, we still consider GS-PnP and EPLL.
For PnP-HVAE, the temperature is set to $\tau=0.6$, and the algorithm is run for a maximum of $200$ iterations, unless the residual $||\x_{k+1}-\x_k||$ is on a plateau.
We provide on Table~\ref{tab:inpainting_bsd} the distortion metrics with the ground truth, as well as a visual
\begin{table}



\begin{center}
    \begin{tabular}{cccc}
        & PSNR$\uparrow$ & SSIM$\uparrow$ &LPIPS$\downarrow$ \\\hline
        PnP-HVAE  & $\mathbf{29.54}$ & $\mathbf{0.93}$ & $\mathbf{0.06}$\\
        GS-PNP & $28.52$ & $\mathbf{0.93}$ & $0.09$\\
        EPLL & $\underline{29.16}$ & $\mathbf{0.93}$ & $\mathbf{0.06}$\\
    \end{tabular}
    \caption{\label{tab:inpainting_bsd}Quantitative evaluation for inpainting on BSD.}
    \end{center}
\end{table}
comparison on figure~\ref{fig:inpainting_bsd}. 
With its hierarchical structure,  PnP-HVAE outperforms the compared methods. \vspace{0.05cm}



\begin{figure}[!h]
    \includegraphics[width=\columnwidth]{figures_arxiv/demo_inp_bsd2.pdf}\vspace{-0.1cm}
    \caption{\label{fig:inpainting_bsd}Natural image inpainting\vspace{-0.3cm}}
\end{figure}











\section{Conclusion}\label{sec:conclusion}
In this work, we focus on addressing the fundamental challenge of OOD detection tasks, which is how to fully understand the semantic discrepancy between the ID/OOD samples. We reveal that the key to success in the realistic SCOOD task is to allocate as many ID samples in the unlabeled set correctly as possible. To this end, we propose a novel uncertainty-aware optimal transport scheme that introduces class-specific energy scores as guidance for effective label assignment. Experimental results show that our method achieves better performance than previous state-of-the-art methods on SCOOD benchmarks.

\textbf{Limitations.} In addition to temperature scaling, other techniques such as feature clipping applied in ReAct~\cite{sun2021react} also enhance the performance of energy score, so how to obtain an OOD score that best fits the SCOOD task can be further explored. Moreover, a setting highly related to SCOOD has been proposed in \cite{katz2022training} and formulated as a constrained optimization problem. We will also theoretically analyze these practical OOD settings in our feature work.

% \section*{Acknowledgments}
\textbf{Acknowledgments.} 
This work is supported by National Key R\&D Program of China under Grant 2020AAA0105701, National Natural Science Foundation of China (NSFC) under Grants 61872327, Major Special Science and Technology Project of Anhui, National Natural Science Foundation of China (62033012) and Ant Group through Ant Research Intern Program.


\section*{Acknowledgements}
The work of D. Tiapkin, A. Naumov, and D. Belomestny were supported by the grant for research centers in the field of AI provided by the Analytical Center for the Government of the Russian Federation (ACRF) in accordance with the agreement on the provision of subsidies (identifier of the agreement 000000D730321P5Q0002) and the agreement with HSE University No. 70-2021-00139. D. Belomestny acknowledges the financial support from Deutsche Forschungsgemeinschaft (DFG), Grant Nr.497300407.
P. M\'enard acknowledges the Chaire SeqALO (ANR-20-CHIA-0020-01).


\bibliographystyle{sty/icml2023}
\bibliography{ref.bib}


\newpage
\appendix
\onecolumn

\part{Appendix}

\vspace{-0.15cm}
\parttoc
\newpage
% \documentclass[a4paper]{amsart}%[a4paper]
% %%%%% GENERAL MATH COMMANDS
% Reals
\newcommand{\R}{{\mathbb R}}
% Integers
\newcommand{\Z}{{\mathbb Z}}
% Naturals
\newcommand{\N}{{\mathbb N}}
% Expectation
\DeclareMathOperator*{\E}{\mathbb{E}}
% ^th notation
\newcommand{\tth}{^{\text{th}}}
% Small dots for integer range [a .. b]
\newcommand{\sdots}{\,..\,}
% Vectorized version of matrix
\newcommand{\matvec}{\mbox{vec}}

% := sign
\newcommand{\defeq}{\vcentcolon=}
% Zero function
\newcommand{\zf}{\mathbf{0}}
% Vector of ones
\newcommand{\ones}{\mathbf{1}}

% Argmin and argmax definitions
\DeclareMathOperator*{\argmax}{arg\,max}
\DeclareMathOperator*{\argmin}{arg\,min}


%%%%% PROBLEM STATEMENT NOTATION 
% \newcommandtwoopt{\St}[2][t][]{{S_{#1}^{#2}}} % State
\newcommand{\task}[1][i]{{\mathcal{T}_{#1}}} % Task, optionally takes index
\newcommand{\tasks}{\{ \task \}_{i=1}^N}
\newcommand{\losst}[1][i]{{l_{#1}}}
\newcommand{\lossv}[1][i]{{l_{#1}^{\textrm{val}}}}
\newcommand{\tasktarget}{{\mathcal{T}_{\textrm{target}}}}
\newcommand{\lossttarget}{l_{\textrm{target}}}
\newcommand{\lossvtarget}{l_{\textrm{target}}^{\textrm{val}}}
\newcommand{\lossttargetit}{l_{\textrm{target}}^{(k)}}
\newcommand{\losstotal}{l^{\textrm{total}}}
\newcommand{\lossopt}{l^*}

\newcommand{\thetait}[2]{\theta_{#1}^{(#2)}}
\newcommand{\phit}[1]{\phi^{(#1)}}
\newcommand{\hist}[2]{S_{#1}^{(#2)}}
\newcommand{\grad}[2]{G_{#1}^{(#2)}}

\newcommand{\Alg}{\textup{\textbf{Opt}}}
\newcommand{\MetaAlg}{\textup{\textbf{MetaOpt}}}

%%%%% Theorems
\newtheoremstyle{mytheoremstyle} % name
    {\topsep}                    % Space above
    {\topsep}                    % Space below
    {\itshape}                   % Body font
    {}                           % Indent amount
    {\scshape}                   % Theorem head font
    {.}                          % Punctuation after theorem head
    {.5em}                       % Space after theorem head
    {}  % Theorem head spec (can be left empty, meaning ‘normal’)
\theoremstyle{mytheoremstyle}
\theoremstyle{plain}
\newtheorem{theorem}{Theorem}
\newtheorem{proposition}{Proposition}
\newtheorem{assumption}{Assumption}
\newtheorem{definition}{Definition}
\newtheorem{lemma}{Lemma}
\theoremstyle{remark}
\newtheorem{remark}{Remark}

%
% \begin{document}
% \section{notation}\label{sec:notation}
For a positive integer $d$, we define $[d]:=\{1,2,\ldots,d\}$. 
The set of non-negative integers is denoted by $\NN:=\{0,1,2,\ldots\}$.
The cardinality of a set $S$ is denoted by $|S|$.
%Operations on $[d]$ cyclically.

Our \emph{graphs} are finite and undirected. We allow multiple edges and loops. A \emph{simple graph} is a graph without multiple edges or loops. 


A \emph{plane map} is a connected planar graph drawn in the plane without edge crossing, considered up to continuous deformation. 
The \emph{faces} of a plane map are the connected components of the complement of the graph. The infinite face is called \emph{outer face}, and the finite faces are called \emph{inner faces}. The vertices and edges incident to the outer face are called \emph{outer} while the other are called \emph{inner}. 
The numbers $\vv$, $\ee$ and $\ff$ of vertices, edges and faces of a plane map are related by the \emph{Euler relation}  $\vv+\ff=\ee+2$. 


We now define the class of plane maps which will be relevant for this article.
\begin{definition}\label{def:d-adapted}
A \emph{$d$-map} is a plane map such that the inner faces have degree at most $d$, and the outer face has degree $d$ and is incident to $d$ distinct vertices (in other words, the contour of the outer face is a simple cycle). 
We will assume that the outer vertices of a $d$-map are labeled $v_1,v_2,\ldots, v_d$ in clockwise order along the boundary of the outer face. %, as in Figure \ref{???}.\\
A \emph{$d$-adapted map} is a $d$-map such that any simple cycle which is not the contour of a face has length at least $d$.\\
\end{definition}
We point out that $d$-adapted maps are necessarily 2-connected (because a cut point in a $d$-map $G$ implies the existence of a simple cycle of length strictly less than the degree of an inner face of $G$, which shows that $G$ is not $d$-adapted).


In a plane map, a \emph{corner} is the sector delimited by two consecutive (half-)edges around a vertex. It is called an \emph{inner corner} if it lies in an inner face, and an \emph{outer corner} otherwise.
The \emph{degree} of a vertex or face is its number of incident corners. A  \emph{$d$-angulation} is a plane map with all faces of degree $d$. A \emph{$d$-angulation of the $k$-gon} is a plane map with inner faces of degree $d$, and outer face of degree $k$. 
A graph is \emph{bipartite} if it admits a bicoloring of its vertices such that adjacent vertices have different colors. It is known that a plane map is bipartite if and only if all its faces have even degree. For $k\geq 2$, a graph is called \emph{$k$-connected} if it is connected and the deletion of any subset of $(k-1)$ vertices does not disconnect it (loops are forbidden for $k\geq 2$, multiple edges are forbidden for $k\geq 3$). 




Let $G$ be an undirected graph. An \emph{arc} of $G$ is an edge $e$ of $G$ together with a chosen orientation of $e$ (so each edge of $G$ correspond to two arcs). The arc \emph{opposite} to an arc $a$, denoted by $-a$, is the arc corresponding to the same edge as $a$ but with the opposite direction. 
The endpoints of an arc $a$ are called the \emph{initial} and \emph{terminal} vertices of $a$ (with $a$ oriented from the initial vertex to the terminal vertex).  If $v$ is the initial (resp. terminal) vertex of the arc $a$, then we say that $a$ is an \emph{outgoing arc} (resp. \emph{ingoing arc}) at $v$. 
\\

%In a graph, a \emph{walk} (of length $k$) is a sequence $v_1,e_1,v_2,\ldots,e_k,v_{k+1}$ that alternates vertices and edges, such that $e_i$ connects $v_i$ to $v_{i+1}$ for $i\in[k]$. It is called a \emph{closed walk} if $v_1=v_{k+1}$. 
%\OB{Made a change in the def of walk (talking about arcs instead). Should we call them ``paths'' rather than ``walks''?}
A \emph{path} in an undirected graph $G$ is a sequence of arcs $a_1,a_2,\ldots,a_k$ such that the terminal vertex of $a_i$ is the initial vertex of $a_{i+1}$ for all $i\in[k-1]$. It is called a \emph{closed path} if the terminal vertex of $a_k$ is the initial vertex of $a_1$. A \emph{cycle} is a (cyclically ordered) closed path. A path or cycle is called \emph{simple} if it does not pass twice by the same vertex. The \emph{girth} of a graph is the minimum length of its simple cycles.   In a plane map, a closed path formed by the arcs around a face is called \emph{contour} of that face. It is known that face contours are simple cycles if the plane map is 2-connected. 
A simple cycle on a plane map is called \emph{counterclockwise} (resp. \emph{clockwise}) if the direction of arcs is counterclockwise (resp. clockwise) around the cycle.

Let $G$ be a graph.  Given an orientation of $G$, a \emph{directed path} (resp. \emph{directed cycle}) is a path (resp. cycle) $a_1,a_2,\ldots,a_k$ such that every arc $a_i$ is oriented according to the orientation of $G$.
A \emph{weighted orientation} of $G$ is an assignment of a non-negative integer to each arc of $G$. Given a weighted orientation $\cW$ of $G$, we call \emph{weight} of an edge the sum of the weights of the two corresponding arcs. 
Weighted orientations are a generalization of the classical (unweighted) orientations of $G$. Indeed the ``unweighted'' orientations of $G$ can be identified to the weighted orientations of $G$ such that the weight of every edge is 1 (for each edge, the arc of weight 1 is taken as the orientation of the edge). The \emph{outgoing weight} (shortly, the \emph{weight}) of a vertex $v$ is the sum of the weights of the arcs going out of $v$. Given a weighted orientation, we call \emph{positive path} (resp. \emph{positive cycle}) a path (resp. cycle) $a_1,a_2,\ldots,a_k$ such that the weight of every arc is positive (this generalizes the notion of \emph{directed path} and \emph{directed cycle}).  




A \emph{tree} is a connected, acyclic graph. For a tree $T$ with a vertex $v$ distinguished as its \emph{root}, we apply the usual ``genealogy'' vocabulary about trees, where $v$ is an \emph{ancestor} of all the other vertices, and every non-root vertex incident to $T$ has a \emph{parent} in $T$, etc. 
We say that we \emph{orient the tree $T$ toward its root} by orienting every edge from child to parent. With this orientation, every non-root vertex of $T$ is incident to one outgoing edge in $T$ (the edge leading to its parent).
%\OB{changed: calling ``subtree'' instead of ``tree''}
A \emph{subtree} of a graph $G$ is a subset of edges of $G$ such that this set of edges together with the incident vertices forms a tree. A \emph{spanning tree} of $G$ is a subtree of $G$ incident to every vertex of $G$. 





%\end{document}

\newpage
\section{Proofs for Visitation Entropy}
\label{app:visitation_entropy_proofs}

We first define the regrets of each players obtained by playing $T$ times the games. For the forecaster-player, for any $\bd\in\cK$ we define 
\[
\regret_{\fore}^T(\bd) \triangleq \sum_{t=1}^T \sum_{h,s,a} \td_h^t(s,a) \left(\log\frac{1}{\bd_h^t(s,a)} -\log\frac{1}{\bd_h(s,a)}\right)
\]
where $\td_h^t(s,a) \triangleq \ind\big\{(s_h^t,a_h^t)=(s,a)\big\}$ is a sample from $d_h^{\pi^t}(s,a)$.
Similarly for the sampler-player, for any $d\in\cK_p$ we define 
\[
\regret_{\samp}^T(d)\triangleq \sum_{t=1}^T \sum_{h,s,a} \big(d_h(s,a) - d_h^{\pi^t}(s,a) \big) \log\frac{1}{\bd_h^t(s,a)}\,.
\]

Recall that the visitation distribution of the policy $\pi$ returned by \algMVEE is the average of the visitation distributions of the sampler-player 
$d_h^{\hpi}(s,a) = \hd^{\,T}_h(s,a) \triangleq (1/T) \sum_{t=1}^T d_h^{\pi^t}(s,a)$.  We also denote by $\rd^T_h(s,a)\triangleq (1/T) \sum_{t=1}^T \td^t(s,a)$ the average of the 'sample' visitation distributions.

We now relate the difference between the optimal visitation entropy and the visitation entropy of the outputted policy $\hpi$ with the regrets of the two players. Indeed, using $\cH(p) = \sum_{i\in[n]} p_i \log(1/q_i) -\KL(p,q) \leq  \sum_{i\in[n]} p_i \log(1/q_i)$ for all $(p,q)\in(\Delta_n)^2$, we obtain 
\begin{align*}
T\big(\VE(d^{\pistar}) -\VE(d^{\hpi})\big) &\leq \sum_{t=1}^T \sum_{h,a,s} d_h^{\pistar}(s,a) \log\frac{1}{\bd_h^t(s,a)}  - \td^t_h(s,a) \log\frac{1}{\rd_h^{\,T}(s,a)} + T\big(\VE(\rd^T) - \VE(\hd^T)\big)\\
& = \regret_{\samp}^T(d^{\pistar})+ \underbrace{\sum_{t=1}^T \sum_{h,s,a} \big(d_h^{\pi^t}(s,a) - \td_h^t(s,a) \big) \log\frac{1}{\bd_h^t(s,a)}}_{\mathrm{Bias}_1} + \regret_{\fore}^T(\rd^T) \\
&\quad+ \underbrace{T\big(\VE(\rd^T) - \VE(\hd^T)\big)}_{\mathrm{Bias}_2}\,.
\end{align*}
It remains to upper bound each terms separately in order to obtain a bound on the gap. We first bound the two regrets terms. The first bias term is  martingale and can easily be bounded with a deviation inequality, whereas for the second one we introduce just instrumentally smoothing of the entropy.

\subsection{Regret of the Forecaster-Player}

We prove in this section a regret-bound for the mixture forecaster.
\begin{lemma}
\label{lem:regret_forecaster}
For $n_0=1$, for any $\bd\in\cK$ it holds almost surely 
\[
\regret_{\fore}^T(\bd) \leq  HSA\log\big(\rme(T+1)\big) - T\sum_{h=1}^H\KL(\rd_h^T,\bd_h)\,.
\]
\end{lemma} 
\begin{proof}
We will bound the regret at step $h$,
\[
\regret^T_{\fore,h}(\bd) \triangleq \sum_{t=1}^T \sum_{s,a} \td_h^t(s,a) \left(\log\frac{1}{\bd_h^t(s,a)} -\log\frac{1}{\bd_h(s,a)}\right)
\]
and then sum the upper bounds. Recall 
\[
    \bd^t_h(s,a) = \frac{n^{t-1}_h(s,a) + 1}{t-1 + SA},
\]
and for $(s,a) = (s^t_h, a^t_h)$ and any $t\in[T], h \in [H]$ we have $n^{t-1}_h(s,a) + 1 = n^t_h(s,a)$. Since $n_0 = 1$, we have $\bd^t_h(s^t_h,a^t_h) = n^t_h(s^t_h, a^t_h) / (SA + t-1)$. Armed with this observation we can rewrite the regret as follows 
\begin{align*}
    \regret^T_{\fore,h}(\bd) &= -T\KL(\rd_h^T,\bd_h) - T\cH(\rd_h^T) -\sum_{t=1}^T \log\big( \bd_h^t(s_h^t,a_h^t)\big)\\
    &= -T\KL(\rd_h^T,\bd_h) - T\cH(\rd_h^T) -\log\left( \prod_{t=1}^T \bd_h^t(s_h^t,a_h^t)\right).
\end{align*}
Then we have an explicit formula for the product of $\bd^t_h$
\begin{align*}
    \prod_{t=1}^T \bd^t_h(s^t_h, a^t_h) &= \prod_{t=1}^T \frac{n^t_h(s^t_h, a^t_h)}{SA + t-1} = \frac{(SA-1)!}{(SA+T-1)!} \prod_{(s,a)\in \cS \times \cA}  [n^T_h(s, a)]! \\
    &= \frac{1}{\binom{T}{(n_h^T(s,a))_{(s,a)\in\cS\times\cA}}}\frac{1}{\binom{T+SA-1}{SA-1}}\\
    &\geq \exp\left(-T\cH(\rd_h^T) - (T+SA-1)\cH\left(\frac{SA-1}{T+SA-1}\right)\right)
\end{align*}
where in the last inequality we used Theorem 11.1.3 by \citet{cover2006elements} and overload the entropy notation $\cH(p) =- p\log(p) - (1-p)\log(1-p)$ for $p\in[0,1]$.
 Putting all together we get
\[
\regret^T_{\fore,h}(\bd) \leq (T+SA-1)\cH\left(\frac{A-1}{T+A-1}\right)-T\KL(\rd_h^T,\bd_h)\,.
\]
Bounding the entropic term
\begin{align*}
(T+SA-1)\cH\left(\frac{SA-1}{T+SA-1}\right) &= (SA-1)\log\frac{T+SA-1}{SA-1}+T\log\frac{T+SA-1}{T}\\
&\leq(SA-1)\log\frac{T+SA-1}{SA-1}+T\log\left(1+\frac{SA-1}{T}\right)\\
&\leq (SA-1)\log\frac{\rme(T+SA-1)}{SA-1}\\
&\leq SA\log\big(\rme(T+1)\big)\,,
\end{align*}
and summing over $h$ allows us to conclude.
\end{proof}

\subsection{Regret of the Sampler-Player}

We start from introducing new notation. Let $\cM_t = (\cS, \cA, \{ p_h \}_{h\in[H]}, \{r^t_h\}_{h\in[H]}, s_1)$ be a sequence of MDPs where reward defined as follows $r^t_h(s,a) = \log(1/ \bd^t_h(s,a))$. Define $Q^{\pi, t}_h(s,a)$ and $V^{\pi, t}_h(s,a)$ as a action-value and value functions of a policy $\pi$ on a MDP $\cM_t$. Notice that the value-function of initial state in this case could be written as follows
\[
    V^{\pi,t}_1(s_1) = \sum_{h,s,a} d^{\pi}_h(s,a) \log\left( \frac{1}{\bd^t_h(s,a)} \right)
\]
therefore, the regret for the sampler-player could be rewritten in the terms of the regret for this sequence of MDPs
\[
    \regret_{\samp}^T(d^\pi) =  \sum_{t=1}^T V^{\pi,t}_1(s_1) - V^{\pi^t,t}_1(s_1).
\]
Since the rewards are changing in each episode and depending on the full history on interaction during previous episodes, we have to handle more uniform approach as in usual \UCBVI proofs \cite{azar2017minimax}.


\paragraph{Concentration}

Let $\alpha^{\KL}, \alpha^{\cnt}: (0,1) \times \R_{+} \to \R_{+}$ be some functions defined later on in Lemma \ref{lem:sampler_proba_master_event}. We define the following favorable events
\begin{align*}
\cE^{\KL}(\delta) &\triangleq \Bigg\{ \forall t \in \N, \forall h \in [H], \forall (s,a) \in \cS\times\cA: \quad \KL(\hp^{\,t}_h(s,a), p_h(s,a)) \leq \frac{\alpha^{\KL}(\delta, n^{\,t}_h(s,a))}{n^{\,t}_h(s,a)} \Bigg\},\\
\cE^{\cnt}(\delta) &\triangleq \Bigg\{ \forall t \in \N, \forall h \in [H], \forall (s,a) \in \cS\times\cA: \quad n^t_h(s,a) \geq \frac{1}{2} \upn^t_h(s,a) - \alpha^{\cnt}(\delta) \Bigg\},
\end{align*}
\begin{lemma}\label{lem:sampler_proba_master_event}
For any $\delta \in (0,1)$ and for the following choices of functions $\alpha,$
\begin{align*}
    \alpha^{\KL}(\delta, n)  \triangleq \log(2SAH/\delta) + S\log\left(\rme(1+n) \right),  \quad
    \alpha^{\cnt}(\delta) \triangleq \log(2SAH/\delta), 
\end{align*}
it holds that
\begin{align*}
\P[\cE^{\KL}(\delta)] \geq 1-\delta/2, \qquad \P[\cE^\cnt(\delta)]\geq 1-\delta/2
\end{align*}
In particular, $\P[\cG(\delta)] \geq 1-\delta$.
\end{lemma}
\begin{proof}
Applying Theorem~\ref{th:max_ineq_categorical} and the union bound over $h \in [H], (s,a) \in \cS \times \cA$ we get $\P[\cE^{\KL}(\delta)]\geq 1-\delta/2$.  By Theorem~\ref{th:bernoulli-deviation} and union bound,  $\P[\cE^{\cnt}(\delta)]\geq 1 - \delta/2$. The union bound yields $\P[\cG(\delta)] \geq 1- \delta$.
\end{proof}

\paragraph{Optimism}
Next we define the exploration bonuses $b^t_h(s,a)$ for the sampler-player for $n_0 = 1$
\begin{equation}\label{eq:sampler_exploration_bonus}
    b^t_h(s,a) = \sqrt{\frac{2 H^2 \log^2(t+SA) \cdot \alpha^{\KL}(\delta, n^t_h(s,a))}{n^t_h(s,a)}}
\end{equation}


\begin{lemma}\label{lem:sampler_optimism}
    For any $t \in [T]$ and any policy $\pi$, the following holds on event $\cG(\delta)$
    \[
        \uQ^t_h(s,a) \geq Q^{\pi, t+1}_h(s,a), \qquad \uV^t_h(s) \geq V^{\pi, t+1}_h(s).
    \]
\end{lemma}
\begin{proof}
    Proceed by backward induction over $h$. For $h = H+1$ the statement trivially holds. Next we assume that the statement holds for any $h' > h$. Then we have by induction hypothesis and Hölder's inequality
    \begin{align*}
        \uQ^t_h(s,a) - Q^{\pi,t+1}_h(s,a) &= \hp^t_h \uV^t_{h+1}(s,a) - p_h V^{\pi,t+1}_h(s,a) + b^t_h(s,a)\\
        & \geq [\hp^t_h - p_h] V^{\pi,t+1}_h(s,a) + b^t_h(s,a) \geq - \norm{V^{\pi,t+1}_h}_\infty \norm{\hp^t_h - p_h}_1 + b^t_h(s,a).
    \end{align*}
    The fact that $\norm{V^{\pi,t+1}_h}_\infty \leq H \log(t+SA)$, Pinsker's inequality and the definition of the event $\cE^{\KL}(\delta)$ yields
    \[
        \norm{V^{\pi,t+1}_h}_\infty \norm{\hp^t_h - p_h}_1 \leq H\log(t+SA) \sqrt{\frac{2\alpha^{\KL}(\delta,n^t_h(s,a))}{n^t_h(s,a)}}  = b^t_h(s,a)
    \]
    that shows $\uQ^t_h(s,a) - Q^{\pi,t+1}_h(s,a) \geq 0$. The inequality on $V$-functions could be derived as follows
    \[
        \uV^t_h(s) \geq \pi \uQ^t_h(s) \geq \pi Q^{\pi,t+1}_h(s) = V^{\pi,t+1}_h(s).
    \]
\end{proof}


\paragraph{Regret Bound}

\begin{lemma}\label{lem:regret_sampler}
    Let $\pi$ be any fixed policy. Then for any $\delta \in(0,1)$ with probability at least $1-\delta$ the following holds
    \[
        \regret_{\samp}^T(d^\pi)  \leq 10\log(T+SA)\sqrt{2H^4 S A T \cdot \left( \log(2SAH/\delta) + S\log(\rme T)\right)\log(T)}.
    \]
\end{lemma}
\begin{proof}
    Assume that the event $\cG(\delta)$ holds.   By Lemma~\ref{lem:sampler_optimism} for any $t \in [T]$ and $h\in[H]$ we have
    \[
        V^{\pi, t}_t(s_h) - V^{\pi^t, t}_h(s^t_h) \leq \uV^{t-1}_h(s^t_h) - V^{\pi^t, t}_h(s^t_h) = \pi^t_h(\uQ^{t-1}_h - Q^{\pi^t, t}_h)(s),
    \]
    thus we can define $\delta^t_h(s,a) = \uQ^{t-1}_h(s,a) - Q^{\pi^t, t}_h(s,a)$ and upper bound the regret as follows
    \[
        \regret_{\samp}^T(d^\pi) \leq \sum_{t=1}^T \pi^t_1 \delta^t_1(s_1).
    \]

    Next we analyze $\delta^t_h(s^t_h)$. By the same argument as in Lemma~\ref{lem:sampler_optimism}
    \[
        \delta^t_h(s,a) = [\hp^{t-1}_h - p_h] \uV^{t-1}_{h+1}(s,a) + b^t_h(s,a) + p_h[\uV^{t-1}_{h+1} - V^{\pi^t, t}_{h+1}](s,a) \leq 2b^{t-1}_h(s,a) + p_h\pi^{t}_{h+1} [\uQ^{t-1}_{h+1} - Q^{\pi^t, t}_{h+1}](s,a)
    \]
    that could be rewritten as follows
    \[
        \delta^t_h(s,a) \leq \E_{\pi^t}\left[ 2b^{t-1}_h(s,a) + \delta^t_{h+1}(s_{h+1}, a_{h+1}) | (s_h, a_h) = (s,a)\right],
    \]
    thus, rolling out the initial bound on regret we have
    \[
        \regret_{\samp}^T(d^\pi) \leq H\log(T+SA) \sum_{t=1}^{T-1} \E_{\pi^t}\left[ \sum_{h=1}^H 2\sqrt{\frac{2\alpha^{\KL}(\delta, n^t_h(s_h, a_h))}{n^t_h(s_h,a_h)} \wedge 1} \bigg| s_1 \right] + H\log(T+SA).
    \]
    By Lemma~\ref{lem:cnt_pseudo} and Jensen's inequality we have
    \[
        \regret_{\samp}^T(d^\pi) \leq 5 H^{3/2}\log(T+SA) \sqrt{2T} \sqrt{ \sum_{h,s,a} \sum_{t=1}^{T-1} d^{\pi^t}_h(s,a)  \frac{\alpha^{\KL}(\delta, \upn^t_h(s, a)) }{\upn^t_h(s,a) \vee 1}}.
    \]
    Notice that $d^{\pi^t}_h(s,a) = \upn^{t+1}_h(s,a) - \upn^t_h(s,a)$ and $\alpha^{\KL}(\delta, \upn^t_h(s,a)) \leq \alpha^{\KL}(\delta, T-1)$. Combined with Lemma~\ref{lem:sum_1_over_n} it implies
    \[
        \regret_{\samp}^T(d^\pi) \leq 10\log(T+SA)\sqrt{2H^4 S A T \cdot \left( \log(2SAH/\delta) + S\log(\rme T)\right)\log(T)}.
    \]

    Finally, the fact that $\P[\cG(\delta)] \geq 1 -\delta$ concludes the statement of theorem.
\end{proof}
\begin{remark}
    It is possible to improve the $H$-dependence by introducing Bernstein-type bonuses, however, we are focused on improvement in a dependence in $\varepsilon^{-1}$ and leave this regret bound as simple as possible.
\end{remark}


\subsection{Bias Terms}

\begin{lemma}\label{lem:bias_terms}
    Let $\delta \in (0,1)$ and $n_0 = 1$. Then with probability at least $1-\delta$ the following two bounds hold
    \begin{align*}
        \mathrm{Bias}_1 &\triangleq \sum_{t=1}^T \sum_{h,s,a} \big(d_h^{\pi^t}(s,a) - \td_h^t(s,a) \big) \log\frac{1}{\bd_h^t(s,a)} \leq  \log(T+SA) \sqrt{2TH \log(2/\delta)} \\
        \mathrm{Bias}_2 &\triangleq T(\VE(\rd^T) - \VE(\hd^T)) \leq \log(SAT)\left(\sqrt{2TH\log(2/\delta)} + 3H\sqrt{SAT\log(3T)} \right).
    \end{align*}
\end{lemma}
\begin{proof}
    Let us define the lexicographic order on the set $[T] \times [H]$ with an additional convention $(t,0) = (t-1, H)$.
    
    Then we can define a filtration $\cF_{t,h} = \sigma\left\{ (s^{t'}_{h'}, a^{t'}_{h'}) \ \forall t \leq t,  \forall h' \in [H] \} \cup \{ (s^t_{h'}, a^t_{h'})\  \forall h' \leq h \} \right\}$ that consists of the all history of interactions of the \algMVEE algorithm with an environment up to the $h$-th step of the episode $t$. The most important fact is that $\pi^t$ and $\bd^t_h(s,a)$ are $\cF_{t,h-1}$-measurable for $h > 1$ and $\cF_{t-1,H}$-measurable for $h=1$. 
    
    Therefore,  for any $t \in [T], h \in [H]$
    \[
        \E\left[ \sum_{s,a} (d_h^{\pi^t}(s,a) - \td_h^t(s,a)) \log \frac{1}{\bd_h^t(s,a)} \bigg| \cF_{t,h-1} \right] = 0.
    \]
    Therefore $X_{t,h} = \sum_{s,a} (d_h^{\pi^t}(s,a) - \td_h^t(s,a)) \log \frac{1}{\bd_h^t(s,a)}$ is a martingale-difference sequence adapted to the filtration $\cF_{t,h}$. Also we notice that a.s. the following bound holds
    \[
        \vert X_{t,h} \vert \leq \log(T+SA)
    \]

    All these facts combined with Azuma-Hoeffding inequality implies that with probability at least $1-\delta/2$
    \[
        \mathrm{Bias}_1 = \sum_{t=1}^T \sum_{h=1}^H X_{t,h} \leq \log(T+SA) \sqrt{2TH \log(2/\delta)}.
    \]

    To show the second part of the statement we notice that
    \[
        \mathrm{Bias}_2 = T \sum_{h=1}^H (\cH(\rd^T_h) - \cH(\hd^T_h)).
    \]
    Let us introduce the smoothed entropy as it was done by \citet{hazan2019provably}. 
    \[
        \forall d \in \simplex_{SA}: \cH_\sigma(d) = \sum_{s,a} d(s,a) \log \frac{1}{d(s,a) + \sigma}.
    \]
    The key difference with our approach and approach of \citet{hazan2019provably} that we need the smoothing only instrumentally to provide a bound on $\mathrm{Bias}_2$.
    
    It is easy to see that $\cH_\sigma$ is concave and, moreover $d \in \simplex_{SA}$
    \[
        \vert \cH(d) - \cH_\sigma(d)  \vert \leq \sum_{s,a} d(s,a) \log \frac{d(s,a) +\sigma}{d(s,a) }  \leq \sigma SA,
    \]
    where we used inequality $\log(1+x) \leq x$ for all $x \geq 0$, and also for $\sigma < \rme^{-1}$
    \[
        \norm{ \nabla \cH_\sigma(d) }_\infty = \sup_{x \in (0,1)} \left\vert \log(x + \sigma) + \frac{x}{x + \sigma} \right\vert \leq \log(\sigma^{-1}).
    \]
    By replacing an entropy with a smoothed entropy
    \[
        \mathrm{Bias}_2 \leq T\sum_{h\in H} (\cH_\sigma(\rd^T_h) - \cH_\sigma(\hd^T_h)) + 2\sigma \cdot TSAH.
    \]
    To analyze the first term we use that $\cH_\sigma$ is concave, therefore
    \[
        \sum_{h=1}^H \cH_\sigma(\rd^T_h) - \cH_\sigma(\hd^T_h) \leq \sum_{h=1}^H \langle \nabla \cH_\sigma(\hd^T_h), \rd^T_h - \hd^T_h \rangle = \frac{1}{T} \sum_{s,a}\sum_{t=1}^T \sum_{h=1}^H (\td_h^t(s,a) - d_h^{\pi^t}(s,a)) \cdot \nabla \cH_\sigma(\hd^T_h)(s,a)
    \]

     For this term situation is more involved than for $\mathrm{Bias}_1$ because $\hd^T_h$ is dependent on all generated policies. Therefore we have to preform uniform bounds. Define $\cW = \{ w \in \R^{HSA} \mid \vert w_h(s,a) \vert \leq 1\} $ as a unit $\ell_\infty$-ball in $\R^{HSA}$. Then we have
     \[
        T\sum_{h=1}^H \cH_\sigma(\rd^T_h) - \cH_\sigma(\hd^T_h) \leq \log(\sigma^{-1}) \cdot \sup_{w \in \cW } \sum_{t=1}^T \sum_{h=1}^H \left( \sum_{s,a} (\td_h^t(s,a) - d_h^{\pi^t}(s,a)) \cdot w_h(s,a) \right).
     \]
     
     Define $N(\varepsilon, \norm{\cdot}_\infty, \cW)$ as $\epsilon$-covering number for a set $\cW$ with $\ell_\infty$-norm as a distance, and $\cW_\varepsilon$ as a minimal $\varepsilon$-net. Next we can use the well-known result on upper bound on the covering number (e.g. see Exercise 5.5 by \citet{van2014probability})
    \[
        N(\epsilon, \norm{\cdot}_\infty, \cW) \leq \left(\frac{3}{\varepsilon}\right)^{SAH},
    \]
    and replace our maximization problem with the maximization over $\varepsilon$-net
    \[
        \sup_{w \in \cW } \sum_{t=1}^T \sum_{h=1}^H \left( \sum_{s,a} (\td_h^t(s,a) - d_h^{\pi^t}(s,a)) \cdot w_h(s,a) \right) \leq \sup_{\hat w \in \cW_\varepsilon } \sum_{t=1}^T \sum_{h=1}^H \left( \sum_{s,a} (\td_h^t(s,a) - d_h^{\pi^t}(s,a)) \cdot \hat w_h(s,a) \right) + \varepsilon TH.
    \]
    For any fixed $\hat w \in \cW_{\varepsilon}$ we apply Azuma-Hoeffding inequality exactly in the same manner as in the bound for $\mathrm{Bias}_1$-term. We have that with probability at least $1-\delta/(2N_\varepsilon)$ for $N = N(\varepsilon, \norm{\cdot}_\infty, \cW)$ we have
    \[
        \sum_{t=1}^T \sum_{h=1}^H \left( \sum_{s,a} (\td_h^t(s,a) - d_h^{\pi^t}(s,a)) \cdot \hat w_h(s,a) \right) \leq \sqrt{2TH\log(2 N_{\varepsilon} / \delta)}.
    \]
    Thus, by union bound we have with probability at least $1-\delta/2$
    \[
         \sup_{w \in \cW } \sum_{t=1}^T \sum_{h=1}^H \left( \sum_{s,a} (\td_h^t(s,a) - d_h^{\pi^t}(s,a)) \cdot w_h(s,a) \right) \leq \sqrt{2TH(\log(2 / \delta) + SAH\log(3/\varepsilon)) } + \varepsilon TH.
    \]
    Taking $\varepsilon = 1/T$ we have
    \[
        \mathrm{Bias}_2 \leq \log(\sigma^{-1})(\sqrt{2TH(\log(2 / \delta) + SAH\log(3T)) } + H) + \sigma SATH.
    \]
    Next we choose $\sigma = 1/SAT$ and by inequality $\sqrt{a+b} \leq \sqrt{a} +\sqrt{b}$ obtain
    \[
        \mathrm{Bias}_2 \leq \log(SAT)\left(\sqrt{2TH\log(2/\delta)} + 3H\sqrt{SAT\log(3T)} \right).
    \]
\end{proof}

\subsection{Proof of Theorem~\ref{th:MVEE_sample_complexity}}

We state the version of this theorem with all prescribed dependencies factors.
\begin{theorem}\label{th:MVEE_sample_complexity_full}
For all $\epsilon > 0$ and $\delta\in(0,1)$. For $n_0=1$ and 
\[
T \geq 1 + \frac{648 (\log(SA) + L)H^4 SA \cdot (\log(4SAH/\delta) + S + L) \cdot L}{\varepsilon^2} + \frac{2 HSA(2+L)}{\varepsilon}
\]
for $L = 9 \log\left( 1010 \sqrt{H^4 S^{8/3} A^{8/3} \log(4SAH/\delta)} / \varepsilon \right)$ the algorithm \algMVEE is $(\epsilon,\delta)$-PAC.
\end{theorem}
\begin{proof}
    We start from writing down the decomposition defined in the beginning of the appendix
    \[
        T(\VE(d^{\pistarVE}) - \VE(d^{\hpi})) \leq \regret_{\samp}^T(d^{\pistarVE}) + \regret_{\fore}^T(\rd^T) + \mathrm{Bias}_1 + \mathrm{Bias}_2.
    \]
    By Lemma~\ref{lem:regret_sampler} with probability at least $1-\delta/2$ it holds
    \[
        \regret_{\samp}^T(d^{\pistarVE}) = 10\log(T+SA)\sqrt{2H^4 S A T \cdot \left( \log(4SAH/\delta) + S\log(\rme T)\right)\log(T)}
    \]
    By Lemma~\ref{lem:regret_forecaster} 
    \[
        \regret_{\fore}^T(\rd^T) \leq HSA\log\big(\rme(T+1)\big).
    \]
    By Lemma~\ref{lem:bias_terms} with probability at least $1-\delta/2$
    \[
         \mathrm{Bias}_1 + \mathrm{Bias}_2 \leq 3\log(SAT)\left(\sqrt{TH\log(4/\delta)} + H\sqrt{SAT\log(3T)} \right).
    \]
    By union bound all these inequalities hold simultaneously with probability at least $1-\delta$. Combining all these bounds we get
    \[
        T(\VE(d^{\pistarVE}) - \VE(d^{\hpi})) \leq 18 \log(SAT) \sqrt{ H^4 SAT (\log(4SAH/\delta) + S \log(\rme T)) \log(T) } + HSA\log(\rme (T+1)).
    \]
    Therefore, it is enough to choose $T$ such that $\VE(d^{\pistarVE}) - \VE(d^{\hpi})$ is guaranteed to be less than $\varepsilon$. In this case \algMVEE become automatically $(\varepsilon,\delta)$-PAC.
    
    It is equivalent to find a maximal $T$ such that
    \[
        \varepsilon T \leq 18 \log(SAT) \sqrt{ H^4 SAT (\log(4SAH/\delta) + S \log(\rme T)) \log(T) } + HSA\log(\rme (T+1))
    \]
    and add $1$ to it.

    
    We start from obtaining a loose bound to eliminate logarithmic factors in $T$.
    
    First, we assume that $T \geq 1$, thus $T+1 \leq 2T$. Additionally, let us use inequality $\log(x) \leq x^{\beta}/\beta$ for any $x > 0$ and $\beta > 0$. We obtain
    \begin{align*}
        \varepsilon T &\leq 18 \frac{(SAT)^{1/3}}{1/3} \sqrt{ H^4 S^2 AT \log(4SAH/\delta) \frac{(\rme T)^{1/18}}{1/18} \frac{T^{1/18}}{1/18} } + HSA \frac{(2\rme T)^{8/9}}{8/9}\\
        &\leq T^{8/9} \left( 1010 \sqrt{H^4 S^2 A^2 \log(2SAH/\delta)} \right),
    \end{align*}
    thus we can define $L = 9 \log\left( 1010 \sqrt{H^4 S^{8/3} A^{8/3} \log(4SAH/\delta)} / \varepsilon \right)$ for which $\log(T) \leq L$. Thus we have
    \[
        \varepsilon T \leq 18 (\log(SA) + L) \sqrt{H^4 SAT(\log(4SAH/\delta) + S + L) L} + HSA(2+ L).
    \]
    Solving this quadratic inequality, we obtain the minimal required $T$ to guarantee $\VE(d^{\pistarVE}) - \VE(d^{\hpi}) \leq \varepsilon$. In particular,
    \[
        T \geq 1 + \frac{648 (\log(SA) + L)H^4 SA \cdot (\log(4SAH/\delta) +S + L) \cdot L}{\varepsilon^2} + \frac{2 HSA(2+L)}{\varepsilon}.
    \]
\end{proof}
\newpage
\section{Regularized Bellman Equations}\label{app:reg_bellman_eq}

In this section we provide complete proofs for regularized Bellman Equations in the general setting. Let $\Phi \colon \Delta_{\cA} \to \R$ be a strictly convex function.

Then we can define the regularized value function as follows
\begin{equation}\label{eq:reg_value_func}
    V^{\pi}_{\lambda,h}(s) \triangleq \E_\pi\left[ \sum_{h'=h}^H r_{h'}(s_{h'}, a_{h'}) -\lambda \Phi(\pi_h(s_{h'}))  \mid s_h = s \right].
\end{equation}
Notably, for a specific choice of rewards $r_h(s,a) = \cH(p_h(s,a))$, the regularizer is equal to the negative entropy $\Phi(\pi) = -\cH(\pi)$, and $\lambda = 1$ we have $V^{\pi}_{\lambda,1}(s_1) = \TE(q^\pi)$. In more general setting let $r_h(s,a)$ be equal to the sum of deterministic reward and $\lambda \cH(p_h(s,a))$. In this case we have $V^{\pi}_{\lambda,1}(s_1) = V^\pi_1(s_1) + \lambda\TE(q^\pi)$ in terms of a usual non-regularized value function.

Let us define an entropy action-value function as follows
\begin{equation}\label{eq:reg_q_func}
    Q^{\pi}_{\lambda,h}(s,a) \triangleq \E_{\pi}\left[ r_h(s_h,a_h) + \sum_{h'=h+1}^H \left[  r_{h'}(s_{h'}, a_{h'}) - \lambda \Phi(\pi_h(s_{h'})) \right] \mid (s_h,a_h) = (s,a) \right].
\end{equation}

Additionally, we define an optimal entropy-regularized value functions a follows
\[
    \Vstar_{\lambda,h}(s) \triangleq \max_{\pi} V^{\pi}_{\lambda,h}(s), \quad \Qstar_{\lambda,h}(s,a) \triangleq \max_{\pi} Q^{\pi}_{\lambda,h}(s,a) \quad \forall (s,a,h) \in \cS \times \cA \times [H]
\]

\subsection{Proof of Entropy-Regularized Bellman Equations}\label{app:proof_entropic_bellman_eq}

\begin{theorem}[Regularized Bellman Equations]
    For any stochastic policy $\pi$ the following decomposition of the entropy-regularized value function holds
    \begin{align}\label{eq:reg_bellman_equation}
    \begin{split}
        Q^{\pi}_{\lambda,h}(s,a) &= r_h(s,a) + p_h V^{\pi}_{\lambda,h+1}(s,a), \\
        V^{\pi}_{\lambda, h}(s) &= \pi_h Q^{\pi}_{\lambda,h}(s) - \lambda \Phi(\pi_h(s)).
    \end{split}
    \end{align}

    Moreover, for optimal $Q$- and $V$-functions we have
    \begin{align}\label{eq:opt_reg_bellman_equation}
    \begin{split}
        \Qstar_{\lambda,h}(s,a) &= r_h(s,a) + p_h \Vstar_{\lambda,h+1}(s,a), \\
        \Vstar_{\lambda,h}(s) &= \max_{\pi \in \Delta_{\cA}}\left\{ \pi \Qstar_h(s) - \lambda \Phi(\pi) \right\}.
    \end{split}
    \end{align}
\end{theorem}
\begin{remark}
    For the case of interest $\Phi(\pi) = - \cH(\pi)$ the expression for the $V$-function allows the closed-form formula by a well-known LogSumExp smooth maximum approximation
    \[
        \Vstar_{\lambda,h}(s) = \lambda \log\left( \sum_{a \in \cA} \exp\left( 
        \frac{1}{\lambda} \Qstar_{\lambda,h}(s,a) \right) \right),
    \]
    and as $\lambda \to 0$ we see that entropy-regularized value function tends to a usual value function without regularization.
\end{remark}
\begin{proof}
    We proceed by induction. For $h=H+1$ the equation is trivial. By definition and tower property of conditional expectation
    \begin{align*}
        Q^{\pi}_{\lambda,h}(s,a) &= r_h(s,a) + \E\left[ \sum_{t=h+1}^{H} r_t(s_t, a_t) - \lambda \Phi(\pi_t(s_t)) \biggl| s_h=s, a_h=a \right] \\
        &= r(s,a) + \E\left[ \E\left[\sum_{t=h+1}^{H} r_t(s_t, a_t) - \lambda \Phi(\pi_t(s_t)) \biggl| s_{h+1}\right] \biggl| s_h=s, a_h=a \right] \\
        &= r_h(s,a) + p_h V^{\pi}_{\lambda, h+1}(s,a).
    \end{align*}
    Next we provide the second Bellman equation by tower property and the definition of the regularized $Q$-function
    \begin{align*}
        V^{\pi}_{\lambda,h}(s) &=  - \lambda \Phi(\pi_h(s)) + \E\left[ r_h(s_h, a_h) + \sum_{t=h+1}^{H} r_t(s_t, a_t) - \lambda \Phi(\pi_t(s_t))  \biggl| s_h=s \right] \\
        &= \pi_h Q^{\pi}_{\lambda,h}(s) - \lambda \Phi(\pi_h(s)).
    \end{align*}

    For optimal Bellman equation we proceed by induction. For $h=H+1$ the equation is also trivial. By Bellman equations
    \[
        \Qstar_{\lambda,h}(s,a) = \max_{\pi} \left\{ r_h(s,a) + p_h V^{\pi}_{\lambda,h+1}(s,a) \right\} = r_h(s,a) + p_h \Vstar_{\lambda,h+1}(s,a),
    \]
    and, finally
    \[
        \Vstar_{\lambda,h}(s) = \max_{\pi_1,\ldots,\pi_H \in \simplex_{\cA}}\left\{ \pi_h \Qstar_{\lambda,h}(s)  - \lambda \Phi(\pi_h(s))\right\} = \max_{\pi \in \Delta_{\cA}}\left\{ \pi \Qstar_{\lambda,h}(s) - \lambda \Phi(\pi) \right\}.
    \]
\end{proof}



\subsection{A Bellman-type Equations for Variance}\label{app:Bellman_variance}
For a stochastic policy $\pi$ we define Bellman-type equations for the variances as follows
\begin{align*}
  \Qvar_{\lambda,h}^{\pi}(s,a) &\triangleq \Var_{p_h}{V_{\lambda,h+1}^{\pi}}(s,a) + p_h \Vvar^{\pi}_{\lambda,h+1}(s,a)\\
  \Vvar_{\lambda,h}^{\pi}(s) &\triangleq \Var_{\pi_h}{Q^{\pi}_{\lambda,h}}(s) + \pi_h \Qvar^{\pi}_{\lambda,h} (s)\\
  \Vvar_{\lambda,H+1}^{\pi}(s)&\triangleq0,
\end{align*}
where $\Var_{p_h}(f)(s,a) \triangleq \E_{s' \sim p_h(\cdot | s, a)} \big[(f(s')-p_h f(s,a))^2\big]$ denotes the \emph{variance operator over transitions} and $\Var_{\pi_h}(f)(s) \triangleq \E_{a' \sim \pi_h(s)}\big[ (f(s,a') - \pi_h f(s))^2 \big]$ denoted the \emph{variance operator over the policy}.
 In particular, the function $s \mapsto \Vvar_{\lambda,1}^{\pi}(s)$ represents the average sum of the local variances $\Var_{p_h}{V_{\lambda, h+1}^{\pi}}(s,a)$ and $ \Var_{\pi_h}{Q^{\pi}_{\lambda,h}}(s) $ over a trajectory following the policy $\pi$, starting from $(s, a)$. Indeed, the definition above implies that
 \[\Vvar_{\lambda,1}^{\pi}(s_1) = \sum_{h=1}^H \sum_{s\in\cS} d_h^\pi(s) \Var_{\pi_h}{Q^{\pi}_{\lambda,h}}(s) +  \sum_{h=1}^H\sum_{s,a} d_h^\pi(s,a) \Var_{p_h}(V_{\lambda,h+1}^{\pi})(s,a).
 \]
 The lemma below shows that we can relate the global variance of the cumulative reward over a trajectory to the average sum of local variances.
\begin{lemma}[Law of total variance]\label{lem:law_of_total_variance}  For any stochastic policy $\pi$ and for all $h\in[H]$,
\begin{align*}
   \Qvar_{\lambda,h}^{\pi}(s,a) &= \E_\pi\!\left[  \left(\sum_{h'=h}^H \left( r_{h'}( s_{h'},a_{h'}) -\lambda \Phi(\pi_{h'}(s_{h'})) \right) - \left( Q_{\lambda,h}^{\pi}(s_h,a_h) -\lambda \Phi(\pi_h(s_h)) \right) \right)^{\!\!2}\middle| (s_h,a_h)=(s,a) \right], \\
  \Vvar_{\lambda,h}^{\pi}(s) &= \E_\pi\!\left[  \left(\sum_{h'=h}^H \left( r_{h'}( s_{h'},a_{h'}) -\lambda \Phi(\pi_{h'}(s_{h'})) \right) - V_{\lambda,h}^{\pi}(s_h) \right)^{\!\!2}\middle| s_h=s \right].
\end{align*}
\end{lemma}
\begin{proof}
	We proceed by induction. The statement in Lemma~\ref{lem:law_of_total_variance} is trivial for $h=H+1$. We now assume that it holds for $h+1$ and prove that it also holds for $h$. For this purpose, we compute
	\begin{align*}
		&
		%\Qvar_h^\pi(s,a) =
		\E_\pi\left[\!
		 \left(\sum_{h'=h}^H \left( r_{h'}( s_{h'},a_{h'}) -\lambda \Phi(\pi_{h'}(s_{h'})) \right) -  \left( Q_{\lambda,h}^{\pi}(s_h,a_h) -\lambda \Phi(\pi_h(s_h))\right) \right)^{\!\!2}
		 \middle| (s_h,a_h) \right] \\
		& =
		\E_\pi\left[\! \left( V_{\lambda, h+1}^{\pi}(s_{h+1}) - p_h V_{\lambda, h+1}^{\pi}(s_h,a_h) + \sum_{h'=h+1}^H \left( r_{h'}( s_{h'},a_{h'}) -\lambda \Phi(\pi_{h'}(s_{h'})) \right) - V_{\lambda, h+1}^{\pi}(s_{h+1})\right)^{\!\!2}
		\middle| (s_h,a_h) \right] \\
		& =
		\E_\pi\left[\!
			\left(  V_{\lambda, h+1}^{\pi}(s_{h+1}) - p_h V_{\lambda, h+1}^{\pi}(s_h,a_h) \right)^{\!\!2}
		\middle| (s_h,a_h) \right] \\
		&
		+ \E_\pi\left[\!
		\left( \sum_{h'=h+1}^H \left( r_{h'}( s_{h'},a_{h'}) -\lambda \Phi(\pi_{h'}(s_{h'})) \right) - V_{\lambda, h+1}^{\pi}(s_{h+1})\right)^{\!\!2}
		\middle| (s_h,a_h) \right] \\
		& + 2 \E_\pi\left[\!
			\left(  \sum_{h'=h+1}^H \left( r_{h'}( s_{h'},a_{h'}) -\lambda \Phi(\pi_{h'}(s_{h'})) \right) - V_{\lambda, h+1}^{\pi}(s_{h+1}) \right)
			\left( V_{\lambda, h+1}^{\pi}(s_{h+1}) - p_h V_{\lambda, h+1}^{\pi}(s_h,a_h) \right)
		\middle| (s_h,a_h) \right].
	\end{align*}
	The definition of  $V_{\lambda,h+1}^\pi(s_{h+1})$ implies that \[\E_\pi\!\left[  \sum_{h'=h+1}^H \left( r_{h'}( s_{h'},a_{h'}) -\lambda \Phi(\pi_{h'}(s_{h'})) \right) - V_{\lambda, h+1}^{\pi}(s_{h+1})	\middle| s_{h+1} \right] = 0.\]
	Therefore, the tower property of conditional expectation gives us
	\begin{align*}
		&\E_\pi\left[\!
		 \left(\sum_{h'=h}^H \left( r_{h'}( s_{h'},a_{h'}) -\lambda \Phi(\pi_{h'}(s_{h'})) \right) -  \left( Q_{\lambda,h}^{\pi}(s_h,a_h) -\lambda \Phi(\pi_h(s_h))\right) \right)^{\!\!2}
		 \middle| (s_h,a_h) \right]  \\ 
        &= \E_\pi\left[\!
			\left(  V_{\lambda, h+1}^{\pi}(s_{h+1}) - p_h V_{\lambda, h+1}^{\pi}(s_h,a_h) \right)^{\!\!2}
		\middle| (s_h,a_h) \right] \\
		&
		+ \E\left[\! \E_\pi\left[
		\left( \sum_{h'=h+1}^H \left( r_{h'}( s_{h'},a_{h'}) -\lambda \Phi(\pi_{h'}(s_{h'})) \right) - V_{\lambda, h+1}^{\pi}(s_{h+1})\right)^{\!\!2}
		\middle| s_{h+1} \right] \middle| (s_h,a_h) \right] \\
		&  = \Var_{p_h}{V_{\lambda, h+1}^{\pi}}(s_h,a_h) + p_h \Vvar^{\pi}_{\lambda,h+1}(s_h,a_h) = \Qvar_{\lambda,h}^{\pi}(s_h,a_h)
	\end{align*}
	where in the third equality we used the inductive hypothesis and the definition of $\sigma V_{h+1}^{\pi}$. To prove the second equation we use the entropy-regularized Bellman equations
    \begin{align*}
        & \E_\pi\!\left[  \left(\sum_{h'=h}^H \left( r_{h'}( s_{h'},a_{h'}) -\lambda \Phi(\pi_{h'}(s_{h'})) \right) - V_{\lambda,h}^{\pi}(s_h) \right)^{\!\!2}\middle| s_h=s \right] \\
        &= \E_\pi\!\left[  \left(\sum_{h'=h}^H \left( r_{h'}( s_{h'},a_{h'}) -\lambda \Phi(\pi_{h'}(s_{h'})) \right) - (Q_{\lambda,h}^{\pi}(s_h, a_h) - \lambda \Phi(\pi_{h}(s_h)) ) \right)^{\!\!2}\middle| s_h=s \right] \\
        &+ 2\E_\pi\!\left[  \left(\sum_{h'=h}^H \left( r_{h'}( s_{h'},a_{h'}) -\lambda \Phi(\pi_{h'}(s_{h'})) \right) - (Q_{\lambda,h}^{\pi}(s_h, a_h) - \lambda \Phi(\pi_{h}(s_h)) )\right) \left( \pi_h (Q_{\lambda,h}^{\pi}(s_h) - (Q_{\lambda,h}^{\pi}(s_h, a_h)  \right)\middle| s_h=s \right] \\
        &+ \E_{\pi}\!\left[\left( \pi_h Q_{\lambda,h}^{\pi}(s_h) - Q_{\lambda,h}^{\pi}(s_h, a_h)  \right)^{2} \middle| s_h=s\right].
    \end{align*}
    By definition of $Q^{\pi}_{\lambda,h}$ we have
    \[
        \E_\pi\left[ \left(\sum_{h'=h}^H \left( r_{h'}( s_{h'},a_{h'}) -\lambda \Phi(\pi_{h'}(s_{h'})) \right) - (Q_{\lambda,h}^{\pi}(s_h, a_h) - \lambda \Phi(\pi_{h}(s_h)) )\right) \middle| (s_h,a_h) = (s,a) \right] = 0,
    \]
    thus, by the tower property
    \begin{align*}
        &\E_\pi\!\left[  \left(\sum_{h'=h}^H \left( r_{h'}( s_{h'},a_{h'}) -\lambda \Phi(\pi_{h'}(s_{h'})) \right) - V_{\lambda,h}^{\pi}(s_h) \right)^{\!\!2}\middle| s_h=s \right] = \pi_h \Qvar^{\pi}_{\lambda,h}(s_h) + \Var_{\pi_h} Q^{\pi}_{\lambda, h}(s_h) = \Vvar^{\pi}_{\lambda, h}(s_h).
    \end{align*}
\end{proof}

\newpage
\section{Sample Complexity for Regularized MDPs}\label{app:regularized_mdp}

In this section we describe the general setting of regularized MDPs, not only entropy-regularized.

\subsection{Preliminaries}

First we define class of regularizers we are interested in. For more exposition on this definition, see \cite{bubeck2015convex}. 
\begin{definition}\label{def:mirror_map}
    Let $\Phi \colon \simplex_{\cA} \to \R$ be a proper closed strongly-convex function. We will call $\Phi$ a mirror-map if the following holds 
\begin{itemize}
    \item $\Phi$ is $1$-strongly convex with respect to norm $\norm{\cdot}$;
    \item $\nabla \Phi$ takes all possible values in $\R^{\cA}$;
    \item $\nabla \Phi$ diverges on the boundary of $\simplex_{\cA}$: $\lim_{x \in \partial \simplex_{\cA}} \norm{\nabla \Phi(x)} = +\infty$;
\end{itemize}
\end{definition}

We explain three main examples of a such regularizers.
\begin{itemize}
    \item The negative Shannon entropy $\Phi(\pi) = -\cH(\pi)$ for $\cH(\pi) = \sum_{a\in\cA} \pi_a \log\left(\frac{1}{\pi_a}\right)$ satisfies the Definition~\ref{def:mirror_map} for $\ell_1$-norm;
    \item The negative Tsallis entropy $\Phi(\pi) = - \frac{1}{q}\cT_{q}(\pi)$ for $\cT_q(\pi) = \frac{1}{q - 1} \left(1 - \sum_{a\in\cA} \pi_a^{q} \right)$ satisfied the Definition~\ref{def:mirror_map} for $\ell_2$ norm for every $q\in(0,1)$. In particular, $q=0.5$ corresponds to the choice by \citet{grill2019planning} in Appendix E that is tightly connected to the UCB algorithm;
    \item For any other fixed policy $\pi'\in\simplex_{\cA}$ we can choose $\Phi(\pi) = \KL(\pi, \pi') = \sum_{a\in\cA} \pi_a \log\left( \frac{\pi_a}{\pi'_a}\right)$ that inherits all the properties from the choice of the negative entropy.
\end{itemize}


Let $\cM = (\cS, \cA, \{p_h\}_{h\in[H]}, \{r_h\}_{h\in[H]}, s_1)$ be a finite-horizon MDP, where $r_h(s,a)$ is a deterministic reward function. For simplicity we assume that $0 \leq r_h(s,a) \leq r_{\max}$ for any $(h,s,a) \in [H] \times \cS \times \cA$. 

Then we can define entropy-augmented rewards as follows
\[
    r_{\mu,h}(s,a) = r_h(s,a) + \mu \cH(p_h(s,a)).
\]
This definition is required to cover the following case of practical interest. Let $\mu = \lambda$ and $\Phi(\pi) = -\cH(\pi)$, then we obtain the following representation for the $\lambda$-regularized value function
\[
    V^{\pi}_{\lambda,1}(s_1) = V^\pi_1(s_1) + \lambda \TE(q^\pi),
\]
where $V^\pi_1(s_1)$ is a usual value function for a MDP $\cM$. For $r_{\max} = 0$  and $\mu=\lambda=1$ we recover just a trajectory entropy $V^{\pi}_{\lambda,1}(s_1) = \TE(q^\pi)$.



Next we define a convex conjugate to $\lambda \Phi$ as $F_\lambda \colon \R^{\cA} \to \R$
\[
    F_\lambda (x) = \max_{\pi \in \simplex_{\cA}} \{ \langle \pi, x \rangle - \lambda \Phi(\pi) \}
\]
and, with a sight abuse of notation extend the action of this function to the $Q$-function as follows
\[
    \Vstar_{\lambda,h}(s) = F_{\lambda}(\Qstar_{\lambda,h})(s) = \max_{\pi \in \Delta_{\cA}}\left\{ \pi \Qstar_{\lambda,h}(s) - \lambda \Phi(\pi) \right\}.
\]

Thanks to the fact that $\Phi$ satisfies Definition~\ref{def:mirror_map}, we have exact formula for the optimal policy by Fenchel-Legendre transform
\[
    \pi^\star_h(s) = \argmax_{\pi \in \Delta_{\cA}}\left\{ \pi \Qstar_{\lambda,h}(s) - \lambda \Phi(\pi) \right\} = \nabla F_\lambda(\Qstar_{\lambda,h}(s,\cdot)).
\]
Notice that we have $\nabla F_\lambda(\Qstar_{\lambda,h}(s,\cdot)) \in \simplex_{\cA}$ since the gradient of $\Phi$ diverges on the boundary of $\simplex_{\cA}$. For entropy regularization this formula become the softmax function.

Finally, it is known that the smoothness property of $F_{\lambda}$ plays a key role in reduced sample complexity for planning in regularized MDPs \cite{grill2019planning}. For our general setting we have that since $\lambda \Phi$ is $\lambda$-strongly convex with respect to $\norm{\cdot}$, then $F_{\lambda}$ is $1/\lambda$-strongly smooth with respect to a dual norm $\norm{\cdot}_*$
\[
    F_\lambda(x) \leq F_\lambda(x') + \langle \nabla F_\lambda(x'), x-x' \rangle + \frac{1}{2\lambda} \norm{x - x'}_*^2.
\]
Let us define $\Rphi$ as a maximal possible value of $\vert \Phi\vert$. Without loss of generality assume that $\Phi \leq 0$. In this case we define $\Rmax = r_{\max} + \mu \log(S) + \lambda \Rphi $ as an upper bound of an about of reward obtain at the one step. By this definition we have $0 \leq V^{\pi}_{\lambda,h}(s) \leq H \Rmax$ for any $h\in[H], s \in \cS$ and any policy $\pi$.

Also, since all norms in $\R^\cA$ are equivalent, we define a constant $r_A$ that defined for a dual norm $\norm{\cdot}_*$ as follows
\[
    \norm{\cdot }_* \leq r_A \cdot \norm{\cdot}_\infty.
\]
For example, for $\ell_2$-norm $r_A = \sqrt{A}$ and for $\ell_1$-norm $r_A = A$. In the case $\Phi = -\cH$ we have $r_A = 1$ since the entropy is $1$-strongly convex with respect to a $\ell_1$-norm, thus the dual norm is exactly a $\ell_\infty$-norm.

The rest of this section is devoted to obtain the sample complexity guarantees for \UCBVIEnt algorithm with two different choices of the stopping rule:
\begin{itemize}
    \item Regularization-aware stopping rule \eqref{eq:def_tau_lambda} with the gap notion \eqref{eq:def_gap_lambda}. In this case we can prove Theorem~\ref{th:reg_aware_sample_complexity} that gives $\tcO\left(\frac{H^6S^2A}{\lambda \varepsilon}\right)$ sample complexity guarantee ignoring $\Rmax$,$r_A$ and poly-logarithmic factors;
    \item Regularization-agnostic stopping rule \eqref{eq:def_tau_agnostic} with the gap notion \eqref{eq:def_gap_agnostic}. In this case Theorem~\ref{th:reg_agnostic_sample_complexity} gives us $\tcO\left(\frac{H^3SA}{\varepsilon^2} + \frac{H^3 S^2A}{\varepsilon}\right)$ sample complexity guarantee ignoring $\Rmax$ and poly-logarithmic factors. Notably, this sample complexity result does depend directly on $\lambda$ and $r_A$.
\end{itemize}
As a corollary of these general results, we obtain algorithm for the MTEE problem with sample complexity $\tcO\left( H^6 S^2 A / \varepsilon \right)$ by taking as a regularizer negative entropy,  $\lambda = \mu = 1$ and $r_{\max} = 0$.

\newpage
\subsection{Concentration Events}
Following the ideas of \cite{menard2021fast}, we define the following concentration events. 

Let $\beta^{\KL}, \beta^{\conc}, \beta^{\cnt}, \beta^{\cH}: (0,1) \times \N \to \R_{+}$ be some functions defined later on in Lemma \ref{lem:proba_master_event}. We define the following favorable events
\begin{align*}
\cE^{\KL}(\delta) &\triangleq \Bigg\{ \forall t \in \N, \forall h \in [H], \forall (s,a) \in \cS\times\cA: \quad \KL(\hp^{\,t}_h(s,a), p_h(s,a)) \leq \frac{\beta^{\KL}(\delta, n^{\,t}_h(s,a))}{n^{\,t}_h(s,a)} \Bigg\},\\
\cE^{\conc}_H(\delta) &\triangleq \Bigg\{\forall t \in \N, \forall h \in [H], \forall (s,a)\in\cS\times\cA: \\
&\qquad|(\hp_h^t -p_h) \Vstar_{\lambda, h+1}(s,a)| \leq \sqrt{\frac{2 H^2 \Rmax^2 \beta^{\conc}(\delta,n_h^t(s,a))}{n_h^t(s,a)}} \Bigg\},\\
\cE^{\conc}_B(\delta) &\triangleq \Bigg\{\forall t \in \N, \forall h \in [H], \forall (s,a)\in\cS\times\cA: \\
&\qquad|(\hp_h^t -p_h) \Vstar_{\lambda,h+1}(s,a)| \leq \sqrt{2 \Var_{p_h}(\Vstar_{\lambda,h+1})(s,a)\frac{\beta^{\conc}(\delta,n_h^t(s,a))}{n_h^t(s,a)}} + 3 H \Rmax \frac{\beta^{\conc}(\delta,n_h^t(s,a))}{n_h^t(s,a)}\Bigg\},\\
\cE^{\cnt}(\delta) &\triangleq \Bigg\{ \forall t \in \N, \forall h \in [H], \forall (s,a) \in \cS\times\cA: \quad n^t_h(s,a) \geq \frac{1}{2} \upn^t_h(s,a) - \beta^{\cnt}(\delta) \Bigg\},\\
\cE^{\cH}(\delta) &\triangleq \Bigg\{\forall t \in \N, \forall h \in [H], \forall (s,a)\in\cS\times\cA: \\
&\qquad  \vert \cH(\hp^t_h(s,a)) - \cH(p_h(s,a)) \vert \leq  \sqrt{\frac{2 \beta^{\cH}(\delta, n^t_h(s,a))}{n^t_h(s,a)}} + \left(\frac{\beta^{\KL}(\delta, n^t_h(s,a))}{n^t_h(s,a)} \wedge \log(S) \right)
\Bigg\}.
\end{align*}
We also introduce two intersections of these events of interest, $\cG_H(\delta) \triangleq \cE^{\KL}(\delta) \cap \cE^{\conc}_H(\delta) \cap \cE^{\cnt}(\delta) \cap \cE^{\cH}(\delta)$ and $\cG_B(\delta) \triangleq \cE^{\KL}(\delta) \cap \cE^{\conc}_B(\delta) \cap \cE^{\cnt}(\delta) \cap \cE^{\cH}(\delta)$. We  prove that for the right choice of the functions $\beta^{\KL}, \beta^{\conc},\beta^{\cnt}, \beta^{\cH}$ the above events hold with high probability.
\begin{lemma}
\label{lem:proba_master_event}
For any $\delta \in (0,1)$ and for the following choices of functions $\beta,$
\begin{align*}
    \beta^{\KL}(\delta, n) & \triangleq \log(4SAH/\delta) + S\log\left(\rme(1+n) \right), \\
    \beta^{\conc}(\delta, n) &\triangleq \log(4SAH/\delta) + \log(4\rme n(2n+1)) ,\\
    \beta^{\cnt}(\delta) &\triangleq \log(4SAH/\delta), \\
    \beta^{\cH}(\delta,n) &\triangleq \log^2(n)\left(\log(4SAH/\delta) + \log(n(n+1))\right),
\end{align*}
it holds that
\begin{align*}
\P[\cE^{\KL}(\delta)]&\geq 1-\delta/4, \qquad \P[\cE^{\conc}_H(\delta)]\geq 1-\delta/4, \qquad \P[\cE^{\conc}_B(\delta)]\geq 1-\delta/4, \\
\P[\cE^\cnt(\delta)]&\geq 1-\delta/4,  \qquad \P[\cE^{\cH}(\delta)]\geq 1-\delta/4.
\end{align*}
In particular, $\P[\cG_H(\delta)] \geq 1-\delta$ and $\P[\cG_B(\delta)] \geq 1-\delta$.
\end{lemma}
\begin{proof}
Applying Theorem~\ref{th:max_ineq_categorical} and the union bound over $h \in [H], (s,a) \in \cS \times \cA$ we get $\P[\cE^{\KL}(\delta)]\geq 1-\delta/4$. 

Let us call the empirical model constructed based on $n$ sampled as $\hp^{[n]}_h$. Let us fix $n,s,a,h$. Then by Azuma-Hoeffding inequality we have
\[
    \P\left[ \vert [\hp^{[n]}_h - p_h]\Vstar_{\lambda, h+1}(s,a) \vert > \sqrt{\frac{2 H^2 \Rmax^2\beta^{\conc}(\delta,n)}{n}} \right] \leq \frac{\delta}{4SAH \cdot 2n(n+1)}.
\]
Union bound over $n,s,a,h$ concludes show that $\P[\cE^{\conc}_H(\delta)] \geq 1 - \delta/4$.

Next, Theorem~\ref{th:bernstein} and the union bound over $h \in [H], (s,a) \in \cS \times \cA$ yield $\P[\cE^{\conc}_B(\delta)]\geq 1 - \delta/4$. By Theorem~\ref{th:bernoulli-deviation} and union bound,  $\P[\cE^{\cnt}(\delta)]\geq 1 - \delta/4$.  Finally, by Theorem~\ref{th:entropy_concentration} and union bound over $(s,a,h) \in \cS \times \cA \times [H]$ $\P[\cE^{\cH}(\delta)]\geq 1-\delta/4$. The union bound over four prescribed events concludes $\P[\cG_H(\delta)] \geq 1 - \delta$ and $\P[\cG_B(\delta)] \geq 1 - \delta$.
\end{proof}


\begin{lemma}\label{lem:reg_directional_concentration}
      Assume conditions of Lemma \ref{lem:proba_master_event}. Then on event $\cE^{\KL}(\delta)$, for any $f \colon \cS \to [0, H \Rmax]$, $t \in \N, h \in [H], (s,a) \in \cS \times \cA$,
      \begin{align*}
            [p_h - \hp_h^t]f(s,a) &\leq \frac{1}{H} \hp^t_h f(s,a) + 2H\Rmax \left(\frac{2H \beta^{\KL}(\delta, n^{\,t}_h(s,a))}{n^{\,t}_h(s,a)} \wedge 1 \right), \\
            [\hp_h^t -p_h]f(s,a) &\leq \frac{1}{H} p_h f(s,a) + 2H\Rmax \left(\frac{2H \beta^{\KL}(\delta, n^{\,t}_h(s,a))}{n^{\,t}_h(s,a)} \wedge 1 \right). 
      \end{align*}
\end{lemma}
\begin{proof}
    Let us start from the first statement.  We apply the second inequality of Lemma~\ref{lem:Bernstein_via_kl} and Lemma~\ref{lem:switch_variance_bis} to obtain
    \begin{align*}
        [p_h - \hp_h^t]f(s,a) &\leq \sqrt{2\Var_{p_h}[f](s,a) \cdot \KL(\hp_h^t, p_h) } \\
        &\leq 2\sqrt{\Var_{\hp^t_h}[f](s,a) \cdot \KL(\hp_h^t, p_h) } +  3 H \Rmax \KL(\hp^t_h, p_h).
    \end{align*}
    Since $0 \leq f(s) \leq  H\Rmax$ we get
    \[
        \Var_{\hp^t_h}[f](s,a) \leq \hp^t_h[f^2](s,a) \leq  H\Rmax \cdot \hp^t_h f(s,a).
    \]
    Finally, applying $2\sqrt{ab} \leq a+b, a, b \geq 0$, we obtain the following inequality
    \begin{align*}
        (\hp_h^t -p_h)f(s,a) &\leq \frac{1}{H} \hp^t_h f(s,a) + 4H^2\Rmax \KL(\hp_h^t, p_h).
    \end{align*}
    Definition of $\cE^{\KL}(\delta)$ implies the part of the statement. At the same time we have a trivial bound since $f(s) \in [0, H \Rmax]$
    \[
        [p_h - \hp^t_h] f(s,a) \leq 2H\Rmax \leq \frac{1}{H} \hp^t_h f(s,a) + 2H\Rmax.
    \]

    To prove the second statement, apply the first inequality of Lemma~\ref{lem:Bernstein_via_kl} and proceed similarly.
\end{proof}


\begin{lemma}\label{lem:empirical_bernstein}
    Assume conditions of Lemma \ref{lem:proba_master_event} and assume that $\beta^{\conc}_B(\delta) \leq \beta^{\KL}(\delta)$. Then conditioned on event $\cG_B(\delta)$, for any $U \colon \cS \to [0, H \Rmax]$, $t \in \N, h \in [H], (s,a) \in \cS \times \cA$,
      \begin{align*}
            \vert (\hp_h^t -p_h)\Vstar_{\lambda,h+1}(s,a) \vert &\leq 3 \sqrt{\Var_{\hp^t_{h+1}}(U)(s,a) \frac{\beta^{\conc}(\delta,n_h^t(s,a))}{n_h^t(s,a)}} + \frac{9H^2\Rmax \beta^{\KL}(\delta, n^t_h(s,a))}{n^t_h(s,a)} \\
            &+ \frac{1}{H} \hp^t_h \vert U - \Vstar_{\lambda,h+1} \vert(s,a).
      \end{align*}
\end{lemma}
\begin{proof}
    First, we apply the definition of event $\cE^{\conc}(\delta)$
    \[
        \vert (\hp_h^t -p_h)\Vstar_{\lambda,h+1}(s,a) \vert \leq \sqrt{2 \Var_{p_h}(\Vstar_{\lambda,h+1})(s,a)\frac{\beta^{\conc}(\delta,n_h^t(s,a))}{n_h^t(s,a)}} + 3 H \Rmax \frac{\beta^{\conc}(\delta,n_h^t(s,a))}{n_h^t(s,a)}.
    \]
    Next we apply Lemma~\ref{lem:switch_variance_bis} and Lemma~\ref{lem:switch_variance} and obtain
    \begin{align*}
        \Var_{p_h}(\Vstar_{\lambda,h+1})(s,a) &\leq 2 \Var_{\hp^t_h}(\Vstar_{\lambda,h+1})(s,a) + 4H^2 \Rmax^2 \KL(\hp^t_h(s,a), p_h(s,a)) \\
        &\leq 4\Var_{\hp^t_{h+1}}(U)(s,a) + 4H\Rmax \hp^t_h \vert U - \Vstar_{\lambda,h+1} \vert(s,a) +  4H^2 \Rmax^2 \KL(\hp^t_h(s,a), p_h(s,a)).
    \end{align*}
    Thus, by inequality $\sqrt{a+b} \leq \sqrt{a} + \sqrt{b}$.
    \begin{align*}
        \vert (\hp_h^t -p_h)\Vstar_{\lambda,h+1}(s,a) \vert &\leq 3 \sqrt{\Var_{\hp^t_{h+1}}(U)(s,a) \frac{\beta^{\conc}(\delta,n_h^t(s,a))}{n_h^t(s,a)}} + 3 \sqrt{H\Rmax \hp^t_h \vert U - \Vstar_{\lambda,h+1}\vert(s,a) \cdot \frac{\beta^{\conc}(\delta,n_h^t(s,a))}{n_h^t(s,a)}} \\
        &+ 3H\Rmax\sqrt{\KL(\hp^t_h(s,a), p_h(s,a)) \cdot \frac{\beta^{\conc}(\delta,n_h^t(s,a))}{n_h^t(s,a)}} + 3H\Rmax \frac{\beta^{\conc}(\delta,n_h^t(s,a))}{n_h^t(s,a)}.
    \end{align*}
    By inequality $2\sqrt{ab} \leq a+b$ we have
    \[
        3 \sqrt{H\Rmax \hp^t_h \vert U - \Vstar_{\lambda,h+1}\vert(s,a) \cdot \frac{\beta^{\conc}(\delta,n_h^t(s,a))}{n_h^t(s,a)}} \leq \frac{1}{H}\hp^t_h \vert U - \Vstar_{\lambda,h+1}\vert(s,a) + \frac{9 H^2 \Rmax\beta^{\conc}(\delta,n_h^t(s,a))}{4 n^t_h(s,a)}.
    \]
    By the definition of the event $\cE^{\KL}(\delta)$ and the fact $\beta^{\conc }(\delta) \leq \beta^{\KL}(\delta)$ we have
    \[
        \sqrt{\KL(\hp^t_h(s,a), p_h(s,a)) \cdot \frac{\beta^{\conc}(\delta,n_h^t(s,a))}{n_h^t(s,a)}} \leq \frac{\beta^{\KL}(\delta, n^t_h(s,a))}{n^t_h(s,a)}.
    \]
\end{proof}

\subsection{Confidence Intervals}

Similar to \citet{azar2017minimax,Zanette19Euler,menard2021fast}, we define the upper confidence bound for the optimal regularized  Q-function of two types: with Hoeffding bonuses and with Bernstein bonuses. 

Let us define empirical estimate of entropy-augmented rewards as follows
\[
    \hat r^t_{\mu,h}(s,a) = r_h(s,a) + \mu \cH(\hp^t_h(s,a)).
\]
Then we have the following sequences defined as follows
\begin{align*}
    \uQ^{t}_{h}(s,a) &= \clip\left( \hat r^t_{\mu,h}(s,a) + \hp^t_h \uV^t_h(s,a) + b^{p,t}_h(s,a) + \mu b^{\cH, t}_h(s,a),0,H\Rmax \right) \\
    \pi^{t+1}_h(s) &= \max_{\pi \in \simplex_\cA} \{ \pi \uQ^t_h(s) -\lambda \Phi(\pi)\}, \\
    \uV^t_h(s) &= \cH(\pi^{t+1}_h(s)) + \pi^{t+1}_h \uQ^t_h(s)  \\
    \uV^t_{H+1}(s) &= 0,
\end{align*}
and the lower confidence bound as follows
\begin{align*}
    \lQ^{t}_{h}(s,a) &= \clip\left( \hat r^t_{\mu,h}(s,a) + \hp^t_h \lV^t_h(s,a) - b^{p,t}_h(s,a) - \mu b^{\cH, t}_h(s,a),0,H \Rmax\right) \\
    \lV^t_h(s) &= \max_{\pi \in \simplex_\cA }\{ \pi \lQ^t_h(s) - \lambda\Phi(\pi)  \} \\
    \lV^t_{H+1}(s) &= 0,
\end{align*}
where we have two types of transition bonuses that will be specified before use. The Hoeffding bonuses defined as follows
\begin{align}\label{eq:hoeffding_transition_bonuses}
    b^{p,t}_h(s,a) &\triangleq \sqrt{\frac{2H^2\Rmax^2 \beta^{\conc}(\delta, n^t_h(s,a))]}{n^t_h(s,a)}},
\end{align}
and the Bernstein bonuses are defined as follows
\begin{align}\label{eq:bernstein_transition_bonuses}
    \begin{split}
        b^{p,t}_h(s,a) &\triangleq b^{B,t}_h(s,a) + b^{\corr,t}_h(s,a),\\
        b^{B,t}_h(s,a) &\triangleq 3\sqrt{\Var_{\hp^t_h}(\uV^t_{h+1})(s,a) \frac{\beta^{\conc}(\delta, n^t_h(s,a))}{n^t_h(s,a)}} + \frac{9H^2 \Rmax \beta^{\KL}(\delta, n^t_h(s,a))}{n^t_h(s,a)}, \\
        b^{\corr,t}_h(s,a) &\triangleq \frac{1}{H} \hp^t_h(\uV^t_{h+1} - \lV^t_{h+1})(s,a).
    \end{split}
\end{align}
The entropy bonuses are defined bellow
\begin{align}\label{eq:entropy_bonuses}
    b^{\cH, t}_h(s,a) &\triangleq \sqrt{\frac{2 \beta^{\cH}(\delta, n^t_h(s,a))}{n^t_h(s,a)}} + \left(\frac{\beta^{\KL}(\delta, n^t_h(s,a))}{n^t_h(s,a)} \wedge \log(S) \right).
\end{align}

\begin{theorem}\label{th:reg_confidence_intervals}
    Let $\delta \in (0,1)$. Assume Hoeffding bonuses \eqref{eq:hoeffding_transition_bonuses}. Then on event $\cG_H(\delta)$ for any $t \in \N$, $(h,s,a) \in [H]\times \cS \times \cA$ it holds
    \[
        \lQ^t_h(s,a) \leq \Qstar_{\lambda,h}(s,a) \leq \uQ^t_h(s,a), \qquad \lV^t_h(s) \leq \Vstar_{\lambda,h}(s) \leq \uV^{t}_h(s,a).
    \]
    The same inequalities hold on event $\cG_B(\delta)$ if we assume Bernstein bonuses \eqref{eq:bernstein_transition_bonuses}.
\end{theorem}
\begin{proof}
    Proceed by induction over $h$. For $h = H+1$ the statement is trivial. Now we assume that inequality holds for any $h' > h$ for a fixed $h \in [H]$. Fix a timestamp $t \in \N$ and a state-action pair $(s,a)$ and assume that $\uQ^t_h(s,a) < H\Rmax$, i.e. no clipping occurs. Otherwise the inequality $\Qstar_{\lambda,h}(s,a) \leq \uQ^t_h(s,a)$ is trivial. In particular, it implies $n^t_h(s,a) > 0$.

    In this case by Bellman equations \eqref{eq:opt_reg_bellman_equation} we have
    \begin{align*}
        [\uQ^t_h - \Qstar_{\lambda,h}](s,a) &= \underbrace{r_h(s,a) + \mu \cH(\hp^t_h(s,a)) - r_h(s,a) - \mu \cH(p_h(s,a)) + \mu b^{\cH,t}_h(s,a)}_{T_1} \\
        &+ \underbrace{\hp^t_h \uV^t_{h+1}(s,a) - p_h \Vstar_{\lambda,h+1}(s,a) + b^{p,t}_h(s,a)}_{T_2}.
    \end{align*}
    By the definition of event $\cE^{\cH}(\delta)$ that is subset of both $\cG_H(\delta)$ and $\cG_B(\delta)$ we have $T_1 \geq 0$. 
    To show that $T_2 \geq 0$, we start from induction hypothesis
    \begin{align*}
        T_2 &\geq [\hp^t_h - p_h] \Vstar_{\lambda,h+1}(s,a) + b^{p,t}_h(s,a).
    \end{align*}
    In the case of Hoeffding bonuses the inequality $T_2 \geq 0$ automatically holds from the definition of $\cE^{\conc}_{H}(\delta) \subseteq \cG_H(\delta)$.
    
    In the case of Bernstein bonuses we apply Lemma~\ref{lem:empirical_bernstein} with $U = \uV^t_{h+1}$ and definition of transition bonuses we have 
    \[
        T_2 \geq - \frac{1}{H} \hp^t_h \vert \uV^t_{h+1} - \Vstar_{\lambda, h+1} \vert (s,a) + \frac{1}{H} \hp^t_h [\uV^t_{h+1} - \lV^t_{h+1}](s,a).
    \]
    By induction hypothesis we have $\uV^t_{h+1}(s) \geq  \Vstar_{\lambda, h+1}(s) \geq \lV^t_{h+1}(s)$, thus $T_2 \geq 0$.

    To prove the second inequality on $Q$-function, we assume $\lQ^t_h(s,a) > 0$ and, as a consequence, $n^t_h(s,a) > 0$. Thus we have
    \begin{align*}
        [\lQ^t_h - \Qstar_{\lambda,h}](s,a) = \underbrace{\cH(\hp^t_h(s,a)) - \cH(p_h(s,a)) - b^{\cH,t}_h(s,a)}_{T_1'} + \underbrace{\hp^t_h \lV^t_{h+1}(s,a) - p_h \Vstar_{\lambda, h+1}(s,a) - b^{p,t}_h(s,a)}_{T_2'}.
    \end{align*}
    Again, by the definition of event $\cE^{\cH}(\delta)$ we have $T_1' \leq 0$ and, by induction hypothesis
    \[
        T_2' \leq [\hp^t_h - p_h] V^{\cH,\star}_{h+1}(s,a) - b^{p,t}_h(s,a).
    \]
    In the case of Hoeffding bonuses $T_2'\leq 0$ be event $\cE^{\conc}_H(\delta)$. In the case of Bernstein bonuses we again apply Lemma~\ref{lem:empirical_bernstein} with $U = \uV^t_{h+1}$
    \[
        T_2' \leq \frac{1}{H} \hp^t_h \vert \uV^t_{h+1} - \Vstar_{\lambda,h+1} \vert (s,a) - \frac{1}{H} \hp^t_h[\uV^t_{h+1} - \lV^t_{h+1}](s,a).
    \]
    We conclude the statement by induction hypothesis for $h' = h+1$.

    Finally, we have to show the inequality for $V$-functions. To do it, we use the fact that $V$-functions are computed by $F_{\lambda}$ applied to $Q$-functions
    \[
        \lV^t_h(s) = F_{\lambda}(\lQ^t_h)(s), \quad \Vstar_{\lambda,h}(s) = F_{\lambda}(\Qstar_{\lambda,h})(s), \quad \uV^t_h(s) = F_{\lambda}(\uQ^t_h)(s).
    \] 
    Notice that $\nabla F_\lambda$ takes values in a probability simplex, thus, all partial derivatives of $F_\lambda$ are non-negative and therefore $F_\lambda$ is monotone in each coordinate. Thus, since $\lQ^t_h(s,a) \leq \Qstar_{\lambda,h}(s,a) \leq \uQ^t_h(s,a)$, we have the same inequality $\lV^t_h(s) \leq \Vstar_{\lambda,h}(s) \leq \uV^t_h(s)$.
\end{proof}


\subsection{Regularization-Aware Stopping Rule}
In this section we provide guarantees for the \textit{regularization-aware gap} that highly depends on the parameter $\lambda$. The stopping rule defined with this notion of the gap for $\lambda = \Omega(1)$ allows to obtain algorithm with $\tcO(1/\varepsilon)$ sample complexity.


Let us define new regularization-aware gap recursively. We start from $G^{t}_{\lambda,H+1} \triangleq 0$ and
\begin{align}\label{eq:def_gap_lambda}
    \begin{split}
        W^t_{\lambda,h}(s,a) &=  \left( 1 + \frac{1}{H}\right)\hp^t_h G^{t}_{\lambda,h+1}(s) + \frac{4H^2 \Rmax \beta^{\KL}(\delta, n^t_h(s,a))}{n^t_h(s,a)} \\
        G^{t}_{\lambda,h}(s) &= \clip\biggl( \pi^{t+1}_h W^t_{\lambda,h}(s) + \frac{r_A^2}{2\lambda} \left(\uV^t_h(s) - \lV^t_h(s)\right)^2,
        0, H\Rmax \biggl)
    \end{split}
\end{align}
and the corresponding stopping time
\begin{align}\label{eq:def_tau_lambda}
    \tau_\lambda = \inf\{ t \in \N : G^t_{\lambda, 1 }(s_1)  \leq \varepsilon \}.
\end{align}

The next lemma justifies this choice of the stopping rule.
\begin{lemma}\label{lem:reg_aware_stopping_rule}
    Assume the choice of Hoeffding bonuses \eqref{eq:hoeffding_transition_bonuses} and let the event $\cG_H(\delta)$ defined in Lemma~\ref{lem:proba_master_event} holds. Then for any $t\in\N$, $s\in \cS, h \in [H]$ 
    \[
        \Vstar_{\lambda,h}(s) - V^{ \pi^{t+1}}_{\lambda,h}(s) \leq G^{t}_{\lambda,h}(s).
    \]
\end{lemma}
\begin{proof}
    Let us proceed by induction. For $h=H+1$ the statement is trivial. Assume that for any $h' > h$ the statement holds. Also assume that $G^t_{\lambda,h}(s) < H \Rmax$, otherwise the inequality on the policy error holds trivially. In particular, it holds that $n^t_h(s,a) > 0$ for all $a \in \cA$.

    We can start analysis from understanding the policy error by applying the smoothness of $F_\lambda$.    
    \begin{align*}
        \Vstar_{\lambda,h}(s) - V^{\pi^{t+1}}_{\lambda,h}(s) &= F_\lambda(\Qstar_{\lambda,h}(s, \cdot)) - \left(\pi^{t+1}_h Q^{\pi^{t+1}}_{\lambda,h}(s, \cdot)  -\lambda \Phi(\pi^{t+1}_h(s)) \right) \\
        &\leq F_\lambda(\uQ^{t}_h)(s) + \langle \nabla F_\lambda(\uQ^{t}_h(s,\cdot)), \Qstar_{\lambda,h}(s,\cdot) - \uQ^t_h(s,\cdot)  \rangle + \frac{1}{2\lambda} \norm{\uQ^t_h - \Qstar_{\lambda,h}}_*^2(s) \\
        &- \left(\pi^{t+1}_h Q^{\pi^{t+1}}_{\lambda,h}(s, \cdot)  -\lambda \Phi(\pi^{t+1}_h(s)) \right).
    \end{align*}
    Next we recall that
    \[
         \pi^{t+1}_h(s) = \nabla F(\uQ^{t}_h(s,\cdot)), \quad F(\uQ^{t}_h)(s)  =  \pi^{t+1}_h \uQ^{t}_h(s) - \lambda \Phi(\pi^{t+1}_h(s)),
    \]
    thus we have
    \[
        F(\uQ^t_h)(s) - \left( \pi^{t+1}_h Q^{\pi^{t+1}}_{\lambda,h}(s, \cdot) - \lambda \Phi(\pi^{t+1}_h(s))  \right) = \pi^{t+1}_h [ \uQ^t_h - Q^{\pi^{t+1}}_{\lambda,h}](s)
    \]
    and, by Bellman equations
    \begin{align*}
        \Vstar_{\lambda,h}(s) - V^{\pi^{t+1}}_{\lambda,h}(s) &\leq \pi^{t+1}_h \left[ \Qstar_{\lambda,h} - Q^{\pi^{t+1}}_{\lambda,h} \right] (s) + \frac{1}{2\lambda} \norm{\uQ^t_h - \Qstar_{\lambda,h}}_*^2(s) \\
        &\leq \pi^{t+1}_h p_h \left[ \Vstar_{\lambda,h+1} - V^{\pi^{t+1}}_{\lambda,h+1} \right] (s) + \frac{1}{2\lambda} \norm{\uQ^t_h - \Qstar_{\lambda,h}}_*^2(s).
    \end{align*}
    By induction hypothesis we have
    \[
        \Vstar_{\lambda,h}(s) - V^{\pi^{t+1}}_{\lambda,h}(s) \leq \pi^{t+1}_h p_h G^{t}_{\lambda,h+1}(s) + \frac{1}{2\lambda} \norm{\uQ^t_h - \Qstar_{\lambda,h}}^2_*(s).
    \]
    Next, we apply Lemma~\ref{lem:reg_directional_concentration}
    \[
        p_h^t G_{\lambda, h+1}(s) = \hp^t_hG_{\lambda, h+1}(s) +  [p_h - \hp^t_h] G^{t}_{\lambda,h+1}(s) \leq \left(1 + \frac{1}{H}\right) \hp^t_h G^{t}_{\lambda,h+1}(s) + \frac{4H^2 \Rmax\beta^{\KL}(\delta, n^t_h(s,a))}{n^t_h(s,a)} = W^t_{\lambda,h}(s,a),
    \]
    thus
    \begin{align*}
        \Vstar_{\lambda,h}(s) - V^{\pi^{t+1}}_{\lambda,h}(s) &\leq \pi^{t+1}_h W^t_{\lambda,h}(s) + \frac{1}{2\lambda} \norm{\uQ^t_h - \Qstar_{\lambda,h}}^2_*(s).
    \end{align*}
    Next we are going to bound $\ell_\infty$ of norm and upper bound this difference by something simpler. First, by Theorem~\ref{th:reg_confidence_intervals}
    \begin{align*}
        \max_{a \in \cA}[\uQ^t_h - \Qstar_{\lambda,h} ]^2(s,a) &= \left( \max_{a \in \cA}[\uQ^t_h - \Qstar_{\lambda,h}](s,a) \right)^2,
    \end{align*}
    and, therefore
    \begin{align*}
         \max_{a \in \cA}[\uQ^t_h - \Qstar_{\lambda,h}](s,a) &= \max_{q \in \simplex_{\cA}}q[\uQ^t_h - \Qstar_{\lambda,h}](s)  \\
         &= \max_{q \in \simplex_{\cA}}\left\{ q\uQ^t_h  - \lambda \Phi(q) - q\Qstar_{\lambda,h} + \lambda \Phi(q) \right\} \\
         &\leq \uV^t_h(s) - \Vstar_{\lambda,h}(s) \leq \uV^t_h(s) - \lV^t_h(s).
    \end{align*}
    Finally, we replace $\norm{\cdot}_*$ by $\norm{\cdot}_\infty$ and obtain
    \begin{align*}
        \Vstar_{\lambda,h}(s) - V^{\pi^{t+1}}_{\lambda,h}(s) &\leq \pi^{t+1}_h W^t_{\lambda,h}(s) + \frac{r_A^2}{2\lambda} \left(\uV^t_h(s) - \lV^t_h(s)\right)^2 = G^{t}_{\lambda,h}(s).
    \end{align*}
\end{proof}


\begin{theorem}\label{th:reg_aware_sample_complexity}
    Let $\varepsilon > 0$,$\delta \in (0,1)$, $S\geq 2$ and $\lambda \leq H^3 S r^2_A$. Then $\UCBVIEnt$ algorithm with Hoeffding bonuses and a regularization-aware stopping rule $\tau_\lambda$ is $(\varepsilon,\delta)$-PAC for the best policy identification in regularized MDPs. 
    
    Moreover, the stopping time $\tau_\lambda$ is bounded as follows
    % \[
    %     \tau_\lambda \leq 1 + \frac{256\rme^4 H^6 S A r^2_A \Rmax^2(\log(3SAH/\delta) + S(1+L)) \cdot L^3}{\varepsilon\lambda},
    % \]
    \[
        \tau_\lambda = \cO\left( \frac{H^6SA r_A^2\Rmax^2 \cdot (\log(SAH/\delta) + SL) \cdot L^3}{\varepsilon \lambda} \right)
    \]
    % where $L = 3\log(54A + \sqrt{18AB})= \cO(\log(\nicefrac{SAH\Rmax}{\varepsilon\lambda}) + \log\log(SAH/\delta) )$, where $A = \nicefrac{256\rme^4 H^6 S A \Rmax^2}{\lambda \varepsilon}$ and $B = \log(3SAH/\delta) + S$. 
    where $L = \cO(\log(\nicefrac{SAH\Rmax}{\varepsilon\lambda}) + \log\log(SAH/\delta))$.
\end{theorem}
\begin{proof}
    To show that $\UCBVIEnt$ is $(\varepsilon,\delta)$-PAC we notice that on event $\cG_H(\delta)$ for $\hpi = \pi^{\tau}$ by Lemma~\ref{lem:reg_aware_stopping_rule}
    \[
        \Vstar_{\lambda,1}(s_1) - V^{\hat \pi}_{\lambda,1}(s_1) \leq G^{\tau_{\lambda}}_1(s_1) \leq \varepsilon,
    \]
    and the event $\cG_H(\delta)$ holds with probability at least $1-\delta$. Next we show that the sample complexity is bounded by the mentioned quantity.


    \textbf{Step 1. Bound for $G^{t}_{\lambda,1}(s_1)$}
    First, we start from bounding $W^t_{\lambda,h}(s,a)$ and $G^t_{\lambda, h}(s)$. By Lemma~\ref{lem:reg_directional_concentration} we can define the following upper bound for $W^t_{\lambda,h}(s,a)$
    \begin{align*}
        W^t_{\lambda,h}(s,a) \leq \left( 1 + \frac{2}{H}\right)p_h G^{t}_{\lambda, h+1}(s,a) + \frac{8H^2 \Rmax \beta^{\KL}(\delta, n^t_h(s,a))}{n^t_h(s,a)}
    \end{align*}
    Therefore we obtain
    \begin{align*}
        G^t_{\lambda, h}(s)  \leq \E_{\pi^{t+1}}\left[ \left( 1 + \frac{2}{H}\right) G^{t}_{\lambda, h+1}(s_{h+1}) + \frac{8H^2 \Rmax \beta^{\KL}(\delta, n^t_h(s_h,a_h))}{n^t_h(s_h,a_h)} + \frac{r_A^2}{2\lambda} \left(\uV^t_h(s_h) - \lV^t_h(s_h)\right)^2 \bigg| s_h = s\right] ,
    \end{align*}
    By rolling out this expression
    \begin{align*}
        G^t_{\lambda, 1}(s_1) &\leq \E_{\pi^{t+1}} \left[ \sum_{h=1}^H \left( 1 + \frac{2}{H} \right)^h \frac{8H^2 \Rmax \beta^{\KL}(\delta, n^t_h(s_h,a_h))}{n^t_h(s_h,a_h)} + \left( 1 + \frac{2}{H} \right)^h  \frac{r_A^2}{2\lambda} \left(\uV^t_h(s_h) - \lV^t_h(s_h)\right)^2 \bigg| s_1 \right].
    \end{align*}
    Using the fact that $(1+2/H)^h \leq \rme^2$, we have
    \begin{align*}
        G^t_{\lambda, 1}(s_1) &\leq \underbrace{8\rme^2 H^2 \Rmax \E_{\pi^{t+1}}\left[\sum_{h=1}^H \frac{\beta^{\KL}(\delta, n^t_h(s_h,a_h))}{n^t_h(s_h,a_h)} \right]}_{\termA} + \frac{\rme^2 r^2_A}{2\lambda} \underbrace{\E_{\pi^{t+1}}\left[ \sum_{h=1}^H \left( \uV^t_h(s_h) - \lV^t_h(s_h) \right)^2 \right]}_{\termB}.
    \end{align*}
    The analysis of the term $\termA$ follows \cite{menard2021fast}: we switch counts to pseudocounts by Lemma~\ref{lem:cnt_pseudo} and obtain
    \[
        \termA \leq 32\rme^2 H^2 \Rmax \sum_{h=1}^H \sum_{(s,a) \in \cS \times \cA} d^{\pi^{t+1}}_h(s,a)  \frac{\beta^{\KL}(\delta, \upn^t_h(s,a))}{\upn^t_h(s,a) \vee 1}. 
    \]
    For the term $\termB$ we start from expression for one term. First, we have
    \[
        \uV^t_h(s_h) - \lV^t_h(s_h) \leq 2 \pi^{t+1}_h b^{p,t}_h(s_h) + 2\mu \pi^{t+1}_h b^{\cH, t}_h(s_h) + \pi^{t+1}_h \hp^t_h [\uV^t_{h+1} - \lV^t_{h+1}](s_h).
    \]
    By Lemma~\ref{lem:reg_directional_concentration}
    \begin{align*}
        \uV^t_h(s_h) - \lV^t_h(s_h) &\leq 2 \pi^{t+1}_h b^{p,t}_h(s_h) + 2\mu \pi^{t+1}_h b^{\cH, t}_h(s_h) + \pi^{t+1}_h 2H\Rmax \left(\frac{2H \beta^{\KL}(\delta, n^{\,t}_h(s,\cdot))}{n^{\,t}_h(s,\cdot)} \wedge 1 \right) \\
        &+ \left(1 + \frac{1}{H} \right) \pi^{t+1}_h p_h [\uV^t_{h+1} - \lV^t_{h+1}](s_h).
    \end{align*}
    By definition of Hoeffding bonuses \eqref{eq:hoeffding_transition_bonuses} and entropy bonuses \eqref{eq:entropy_bonuses} 
    \begin{align*}
        \mu b^{\cH,t}_h(s_h) &\leq \mu \log(S) \left( \sqrt{\frac{2 \beta^{\cH}(\delta, n^t_h(s,a))}{n^t_h(s,a)}} + \left(\frac{\beta^{\KL}(\delta, n^t_h(s,a))}{n^t_h(s,a)} \wedge 1 \right)  \right) \\
        &\leq 2H\Rmax \sqrt{\frac{2 \beta^{\cH}(\delta, n^t_h(s,a))}{n^t_h(s,a)}} + H\Rmax \left(\frac{2H \beta^{\KL}(\delta, n^{\,t}_h(s,\cdot))}{n^{\,t}_h(s,\cdot)} \wedge 1 \right). 
    \end{align*}
    Notice that $\beta^{\KL}(\delta, n) \leq \log^2(n) \cdot \beta^{\conc}(\delta, n)$, thus, rolling out this recursion and using the fact that $(1+1/H)^h \leq \rme$ for $h \leq H$, we have
    \begin{align*}
        \uV^t_h(s_h) - \lV^t_h(s_h) &\leq 4\rme H\Rmax \cdot \E_{\pi^{t+1}} \Biggl[ \sum_{h'=h}^H \sqrt{\frac{2\log^2(n^t_{h'}(s_{h'}, a_{h'}))\beta^{\conc}(\delta, n^t_{h'}(s_{h'}, a_{h'})) }{n^t_{h'}(s_{h'}, a_{h'})}} \\
        &+  \left(\frac{2H \beta^{\KL}(\delta, n^{\,t}_{h'}(s_{h'},a_{h'}))}{n^{\,t}_{h'}(s_{h'},a_{h'})} \wedge 1 \right)   \bigg| s_h \Biggl].
    \end{align*}
    By Lemma~\ref{lem:cnt_pseudo} and Jensen inequality we have
    \begin{align*}
        \uV^t_h(s_h) - \lV^t_h(s_h) &\leq 2\rme H^{3/2}\Rmax \sqrt{ \E_{\pi^{t+1}} \left[ \sum_{h'=h}^H  \frac{2\log^2(\upn^t_{h'}(s_{h'}, a_{h'})) \beta^{\conc}(\delta, \upn^t_{h'}(s_{h'}, a_{h'})) }{\upn^t_{h'}(s_{h'}, a_{h'}) \vee 1} | s_h \right]} \\
        &+ 2\rme H \Rmax \E_{\pi^{t+1}}\left[\sum_{h'=h}^H  \left( \frac{2H\beta^{\KL}(\delta, \upn^t_{h'}(s_{h'},a_{h'}))}{\upn^t_{h'}(s_{h'},a_{h'}) \vee 1} \wedge 1 \right) \big| s_h \right].
    \end{align*}
    By inequality $(a+b)^2 \leq 2a^2 + 2b^2$ 
    \begin{align*}
        \termB &= \frac{\rme^2 r_A^2}{\lambda} \cdot \E_{\pi^{t+1}}\left[ \sum_{h=1}^H (\uV^t_h(s_h) - \lV^t_h(s_h))^2 \bigg| s_1 \right]  \\
        &\leq \frac{8\rme^4 H^3 \Rmax^2 r_A^2}{\lambda} \underbrace{\E_{\pi^{t+1}}\left[\sum_{h=1}^H \E_{\pi^{t+1}}\left[\sum_{h'=h}^H \frac{2\log^2(\upn^t_{h'}(s_{h'}, a_{h'}))\beta^{\conc}(\delta, \upn^t_{h'}(s_{h'}, a_{h'})) }{\upn^t_{h'}(s_{h'}, a_{h'}) \vee 1} \bigg| s_h \right] \bigg| s_1 \right]}_{\termC} \\
        &+ \frac{8\rme^4 H^2 \Rmax^2 r_A^2}{\lambda}  \underbrace{\E_{\pi^{t+1}}\left[\sum_{h=1}^H \left( \E_{\pi^{t+1}}\left[  \sum_{h'=h}^H  \left( \frac{2H\beta^{\KL}(\delta, \upn^t_h(s_{h'},a_{h'}))}{\upn^t_{h'}(s_{h'},a_{h'}) \vee 1} \wedge 1 \right)\bigg| s_h \right]  \right)^2 \bigg| s_1 \right]}_{\termD}.
    \end{align*}
    For term $\termC$ we apply the telescoping property of conditional expectation and obtain the following bound
    \begin{align*}
        \termC &= \E_{\pi^{t+1}}\left[\sum_{h=1}^H \sum_{h'=h}^H \frac{2\log^2(\upn^t_{h'}(s_{h'}, a_{h'})) \beta^{\conc}(\delta, \upn^t_{h'}(s_{h'}, a_{h'})) }{\upn^t_{h'}(s_{h'}, a_{h'}) \vee 1}  \bigg| s_1 \right] \\
        &\leq H \E_{\pi^{t+1}}\left[\sum_{h=1}^H \frac{2\log^2(\upn^t_{h}(s_{h}, a_{h})) \beta^{\conc}(\delta, \upn^t_{h}(s_{h}, a_{h}))}{\upn^t_{h}(s_{h}, a_{h}) \vee 1}  \bigg| s_1 \right].
    \end{align*}
    For term $\termD$ we first apply Jensen's inequality and the telescoping property of conditional expectation
    \[
        \termD \leq \E_{\pi^{t+1}}\left[ \sum_{h=1}^H \left( \sum_{h'=h}^H  \left( \frac{2H\beta^{\KL}(\delta, \upn^t_h(s_{h'},a_{h'}))}{\upn^t_{h'}(s_{h'},a_{h'}) \vee 1} \wedge 1 \right) \right)^2 \bigg| s_1 \right].
    \]
    Notice that the expression under the square maximizes at $h=1$. Thus, applying Jensen's inequality
    \[
        \termD \leq H^3 \E_{\pi^{t+1}}\left[ \left( \sum_{h=1}^H \frac{1}{H}   \left( \frac{2H\beta^{\KL}(\delta, \upn^t_h(s_{h'},a_{h'}))}{\upn^t_{h'}(s_{h'},a_{h'}) \vee 1} \wedge 1 \right) \right)^2 \bigg| s_1 \right] \leq H^2 \E_{\pi^{t+1}}\left[ \sum_{h=1}^H  \left( \frac{2H\beta^{\KL}(\delta, \upn^t_h(s_{h'},a_{h'}))}{\upn^t_{h'}(s_{h'},a_{h'}) \vee 1} \wedge 1 \right)^2 \bigg| s_1 \right].
    \]
    Finally, representing maximum as a product of two equivalent terms and using two different upper bounds based on minimum operation we have
    \[
        \termD \leq 2H^3 \E_{\pi^{t+1}}\left[ \sum_{h=1}^H  \frac{\beta^{\KL}(\delta, \upn^t_h(s_{h},a_{h}))}{\upn^t_{h}(s_{h},a_{h}) \vee 1} \bigg| s_1 \right].
    \]
    The final bound for an initial gap follows
    \begin{align*}
        G^t_{\lambda, 1}(s_1) &\leq 32\rme^2 H^2 \Rmax \sum_{h=1}^H \sum_{(s,a) \in \cS \times \cA} d^{\pi^{t+1}}_h(s,a)  \frac{\beta^{\KL}(\delta, \upn^t_h(s,a))}{\upn^t_h(s,a) \vee 1} \\
        &+ \frac{16\rme^4 H^4 \Rmax^2 r_A^2}{\lambda} \sum_{h=1}^H \sum_{(s,a) \in \cS \times \cA} d^{\pi^{t+1}}_h(s,a)  \frac{\log^2(\upn^t_{h}(s, a)) \beta^{\conc}(\delta, \upn^t_h(s,a))}{\upn^t_h(s,a) \vee 1} \\
        &+ \frac{16\rme^4 H^5 \Rmax^2 r_A^2}{\lambda} \sum_{h=1}^H  \sum_{(s,a) \in \cS \times \cA} d^{\pi^{t+1}}_h(s,a) \frac{\beta^{\KL}(\delta, \upn^t_h(s,a))}{\upn^t_h(s,a) \vee 1} .
    \end{align*}

    \textbf{Step 2. Sum over $t < \tau_{\lambda}$.} Assume $\tau_{\lambda} > 0$. In the case $\tau_{\lambda} = 0$ the bound is trivially true. Notice that for any $t < \tau_{\lambda}$ we have
    \[
        G^t_{\lambda,1}(s_1) > \varepsilon,
    \]
    thus, summing upper bounds on $G^t_{\lambda,1}(s_1)$ over all $t < \tau_{\lambda}$ we have
    \begin{align*}
        \varepsilon (\tau_{\lambda} - 1) < \sum_{t=1}^{\tau_{\lambda}-1} G^{t}_{\lambda,1}(s_1) &\leq 32\rme^2 H^2 \Rmax \sum_{h=1}^H \sum_{(s,a) \in \cS \times \cA} \sum_{t=1}^{\tau_{\lambda}-1}d^{\pi^{t+1}}_h(s,a)  \frac{\beta^{\KL}(\delta, \upn^t_h(s,a))}{\upn^t_h(s,a) \vee 1} \\
        &+ \frac{16\rme^4 H^4 \Rmax^2 r_A^2}{\lambda} \sum_{h=1}^H \sum_{(s,a) \in \cS \times \cA} \sum_{t=1}^{\tau_{\lambda}-1} d^{\pi^{t+1}}_h(s,a)  \frac{\log^2(\upn^t_{h}(s, a))  \beta^{\conc}(\delta, \upn^t_h(s,a))}{\upn^t_h(s,a) \vee 1} \\
        &+ \frac{16\rme^4 H^5 \Rmax^2 r_A^2}{\lambda} \sum_{h=1}^H  \sum_{(s,a) \in \cS \times \cA}  \sum_{t=1}^{\tau_{\lambda}-1} d^{\pi^{t+1}}_h(s,a)   \frac{\beta^{\KL}(\delta, \upn^t_h(s,a)) }{\upn^t_h(s,a) \vee 1}.
    \end{align*}

    Notice that $\beta^{\KL}(\delta,\cdot)$ and $\beta^{\conc}(\delta, \cdot)$ are monotone and maximizes at $\tau_{\lambda}$, and $d^{\pi^{t+1}}_h(s,a) = \upn^{t+1}_h(s,a) - \upn^t_h(s,a)$. Thus, applying Lemma~\ref{lem:sum_1_over_n}, we have
    \begin{align*}
        \varepsilon(\tau_{\lambda} - 1) &< 128\rme^2 H^3 SA \Rmax \cdot  \beta^{\KL}(\delta, \tau_{\lambda}-1) \log(\tau_{\lambda})   \\
        &+ \frac{64\rme^4 H^5 SA \Rmax^2 r_A^2 \cdot \beta^{\conc}(\delta, \tau_{\lambda}-1) \log^3(\tau_{\lambda})}{\lambda} \\
        &+ \frac{64\rme^4 H^6 SA \Rmax^2 r_A^2 \cdot \beta^{\KL}(\delta, \tau_{\lambda} - 1) \log(\tau_{\lambda})}{\lambda}.
    \end{align*}
    Notice that since $S>2$ we have $\beta^{\KL}(\delta, \tau_{\lambda} - 1) \geq \beta^{\conc}(\delta, \tau_{\lambda}-1)$, thus the third term dominates the second one. Assume that $\lambda \leq H^3 S r^2_A$. Then by definition of $\beta^{\KL}$
    \[
        \varepsilon(\tau_{\lambda}-1) \leq \frac{256\rme^4 H^6 S A r^2_A \Rmax^2}{\lambda} \cdot (\log(3SAH/\delta) + S \log(\rme \tau_{\lambda})) \cdot \log^3(\tau_{\lambda}).
    \]

    \textbf{Step 3. Solving the recurrence.}
    Define $A = \nicefrac{192\rme^4 H^6 S A r^2_A \Rmax^2}{\lambda \varepsilon}$ and $B = \log(3SAH/\delta) + S$. Our goal is to upper bound solutions to the following inequality
    \[
        \tau \leq 1 + A (S\log(\tau) + B) \cdot \log^3(\tau).
    \]
    First we obtain loose solution by using inequality $\log(\tau) \leq \tau^\beta / \beta$ that holds for any $\tau \geq 1$. Taking $\beta = 1/3$ inside the brackets and $\beta=1/9$ outside we have
    \[
        \tau \leq 1 + 9A(3S\cdot \tau^{1/3} + B ) \cdot \tau^{1/3}.
    \]
    Also we may assume that $\tau \geq 2$, thus $1 \leq \tau/2$ and we achieve
    \[
        \tau^{2/3} \leq 18A (3S \tau^{1/3} + B).
    \]
    Solving this quadratic inequality in $\tau^{1/3}$, we have
    \[
        \tau \leq \left(\frac{54AS + \sqrt{(54AS)^2 + 72AB}}{2} \right)^3 \leq \left( 54AS + \sqrt{18 AB} \right)^{3}.
    \]
    Define $L = 3 \log\left( 54AS + \sqrt{18AB} \right)$. Then we can easily upper bond the initial inequality as follows
    \[
        \tau \leq 1 + A(B + SL) L^3.
    \]
\end{proof}

After this general result we state the bound for the MTEE problem that was stated in the main text.
\begin{theorem}\label{th:mtee_sample_complexity}
    For all $\varepsilon > 0$ and $\delta \in (0,1)$ the \UCBVIEnt algorithm is $(\varepsilon,\delta)$-PAC for MTEE. Moreover, with probability at least $1-\delta$
    \[
        \tau \leq \cO\left( \frac{H^6 SA \log^2(SA) \cdot (\log(SAH/\delta) + SL) L^3}{\varepsilon} \right),
    \]
    where $L = \log(SAH/\varepsilon) + \log\log(SAH/\delta)$.
\end{theorem}
\begin{proof}
    Fix $\Phi(\pi) = -\cH(\pi), \mu=\lambda=1$ and $r_{\max} = 0$. Since $\cH(\pi)$ is $1$-strongly convex with respect to $\ell_1$-norm, we have $r_A = 1$. Also we automatically have $\Rmax = \log(SA)$. In this setting, Theorem~\ref{th:reg_aware_sample_complexity} yields the desired statement.
\end{proof}
\begin{remark}
    Note that the obtained sample complexity corresponds only to the second-order term in the complexity of the algorithm \UCBVIBPI for identifying the best policy \cite{menard2021fast}. Improving $S$-dependence turns out to be a major challenge for the MTEE problem, as it requires reducing the second-order bonus within the expression for the entropy bonus $b^{\cH,t}_h(s,a)$, which scales with $S/n^t_h(s,a)$ and therefore leads to $\tcO(HS ^2 A /\varepsilon)$ sample complexity. However, if we assume that we have access to the true entropy of the transitions, it is possible to improve the $S$-dependence by applying the $Q$-learning type algorithm \cite{jin2018is,zhang2020advantage}.
\end{remark}


\subsection{Regularization-Agnostic Stopping Rule}

In this section we provide guarantees for the so-called \textit{regularization-agnostic gap}: this notion of gap does not influenced by regularization except the changing of the range of value functions and basically mimics \UCBVIBPI algorithm by \citet{menard2021fast} in definition of the similar gap.

Let us define $G^t_{H+1}(s,a) \triangleq 0$ for all $s,a$ and
\begin{align}\label{eq:def_gap_agnostic}
    \begin{split}
    G^t_h(s,a) &\triangleq \clip\biggl( 2 b^{B,t}_h(s,a) + \frac{4 H^2\Rmax \beta^{\KL}(\delta, n^t_h(s,a))}{n^t_h(s,a)} + 2 \mu b^{\cH, t}_h(s,a) + \left(1 + \frac{3}{H} \right) \hp^t_h \left[\pi^{t+1}_{h+1} G^t_{h+1}\right](s,a), \\
    &\qquad 0, H\Rmax \biggl),
    \end{split}
\end{align}
where $b^{B,t}_h(s,a)$ is defined in \eqref{eq:bernstein_transition_bonuses}. For this notion of the gap we can define the stopping rule as follows
\begin{align}\label{eq:def_tau_agnostic}
    \tau = \min\{ t \in \N : \pi^{t+1}_1 G^t_1(s_1) \leq \varepsilon \}.
\end{align}

The lemma below justifies this choice of the stopping time.
\begin{lemma}\label{lem:reg_agnostic_stopping_rule}
    Assume the choice of Bernstein bonuses \eqref{eq:bernstein_transition_bonuses} and let the event $\cG_B(\delta)$ defined in Lemma~\ref{lem:proba_master_event} holds. Then for all $t \in \N$ we have
    \[
        \Vstar_{\lambda,1}(s_1) - V^{\pi^{t+1}}_{\lambda,1}(s_1) \leq \pi^{t+1}_1 G^t_1(s_1).
    \]
\end{lemma}
\begin{proof}
    Following \cite{menard2021fast}, we start by defining the following quantities
    \begin{align*}
        \tQ^t_h(s,a) &\triangleq \clip\left( \hat r^t_{\mu,h}(s,a) + \hp^t_h \tV^t_{h+1}(s,a) - b^{p,t}_h(s,a) - \mu b^{\cH,t}_h(s,a) , 0, r_{\mu,h}(s,a) + p_h \tV^t_{h+1}(s,a) \right),  \\
        \tV^t_h(s) &\triangleq  \pi^{t+1}_h \tQ^t_h(s) - \lambda \Phi(\pi^{t+1}_h(s)), \\
        \tV^t_{H+1}(s) &\triangleq 0.
    \end{align*}
    By Theorem~\ref{th:reg_confidence_intervals} and Lemma~\ref{lem:tQ_properties} we have
    \begin{align*}
        \Vstar_{\lambda,1}(s_1) - V^{\pi^{t+1}}_{\lambda,1}(s_1) &\leq \uV^{t}_1(s_1) - V^{\pi^{t+1}}_{\lambda,1}(s_1) \leq \uV^{t}_1(s_1) - \tV^t_1(s_1) \\
        &=  \pi^{t+1}_1 \uQ^t_1(s_1) - \lambda \Phi(\pi^{t+1}_1(s_1))- \pi^{t+1}_1 \tQ^t_1(s_1) + \lambda \Phi(\pi^{t+1}_1(s_1)) = \pi^{t+1}_1 [\uQ^t_1 - \tQ^t_1](s_1).
    \end{align*}
    Therefore, it is enough to show that for any $(h,s,a) \in [H] \times \cS \times \cA$
    \[
        [\uQ^t_h - \tQ^t_h](s,a) \leq G^t_h(s,a), \qquad  [\uV^t_h - \tV^t_h](s) \leq \pi^{t+1}_h G^t_h(s).
    \]
    Proceed by backward induction over $h$. The case $h=H+1$ is trivial, thus we may assume that the statement holds for any $h' > h$ for a fixed $h$. Also fix $(s,a) \in \cS \times \cA$.

    Notice that if $G^t_h(s,a) = H\Rmax$, then the inequality is trivially true. Therefore we may assume that $G^t_h(s,a) < H\Rmax$ and, consequently, $n^t_h(s,a) > 0$. Now we have to separate cases.
    \paragraph{First case.}
    In this case we have $\tQ^t_h(s,a) = r_{\mu,h}(s,a) + p_h \tV^t_{h+1}(s,a)$, i.e. maximal clipping occurs. Therefore
    \begin{align*}
        \uQ^t_h(s,a) - \tQ^t_h(s,a) &= r_h(s,a) + \mu \cH(\hp^t_h(s,a)) + \mu b^{\cH,t}_h(s,a) - r_h(s,a) - \mu\cH(p_h(s,a)) \\
        &+ \hp^t_h \uV^t_{h+1}(s,a) - p_h \tV^t_{h+1}(s,a) + b^{p,t}_h(s,a).
    \end{align*}
    By the definition of the event $\cE^{\cH}(\delta) \subseteq \cG_B(\delta)$ we have
    \[
        \mu \cH(\hp^t_h(s,a)) + \mu b^{\cH,t}_h(s,a)  - \mu\cH(p_h(s,a)) \leq 2\mu b^{\cH,t}_h(s,a),
    \]
    for the next term we have
    \[
        \hp^t_h \uV^t_{h+1} - p_h \tV^t_{h+1}(s,a) = \hp^t_h [\uV^t_{h+1} - \tV^t_{h+1}(s,a)] + [\hp^t_h - p_h] \Vstar_{\lambda,h+1} (s,a) + [p_h - \hp^t_h][ \Vstar_{\lambda,h+1}- \tV^t_{h+1} ](s,a).
    \]
    By induction hypothesis 
    \[
        \hp^t_h [\uV^t_{h+1} - \tV^t_{h+1}(s,a)] \leq \hp^t_h [ \pi^{t+1}_{h+1} G^t_{h+1}](s,a).
    \]
    Next we apply Lemma~\ref{lem:empirical_bernstein} with $U = \uV^t_{h+1}(s,a)$ and Theorem~\ref{th:reg_confidence_intervals}
    \[
        [\hp^t_h - p_h] \Vstar_{\lambda,h+1}(s,a) \leq b^{p,t}_h(s,a).
    \]
    Finally, we apply Lemma~\ref{lem:reg_directional_concentration} and obtain
    \[
        [p_h - \hp^t_h][\Vstar_{\lambda,h+1}- \tV^t_{h+1} ](s,a) \leq \frac{1}{H} \hp^t_h[\Vstar_{\lambda,h+1} - \tV^t_{h+1} ](s,a) + \frac{4 H^2 \Rmax \cdot \beta^{\KL}(\delta,n^t_h(s,a))}{n^t_h(s,a)}.
    \]
    Summing all these bounds up, we have
    \begin{align*}
        \uQ^t_h(s,a) - \tQ^t_h(s,a) &\leq \hp^t_h [ \pi^{t+1}_{h+1} G^t_{h+1}](s,a) + 2\mu b^{\cH,t}_h(s,a) + 2b^{p,t}_h(s,a) \\
        &+ \frac{1}{H} \hp^t_h[ \Vstar_{\lambda, h+1}- \tV^t_{h+1} ](s,a) + \frac{4 H^2 \Rmax \cdot \beta^{\KL}(\delta,n^t_h(s,a))}{n^t_h(s,a)}.
    \end{align*}
    Notice that by Theorem~\ref{th:reg_confidence_intervals} and the induction hypothesis 
    \[
        \frac{1}{H} \hp^t_h[\Vstar_{\lambda, h+1}- \tV^t_{h+1} ](s,a) \leq \frac{1}{H} \hp^t_h[\uV^{t}_{h+1}- \tV^t_{h+1} ](s,a) \leq  \frac{1}{H} \hp^t_h [ \pi^{t+1}_{h+1} G^t_{h+1}](s,a),
    \]
    and by decomposing the transition bonus \eqref{eq:bernstein_transition_bonuses} to Bernstein bonus and correction term and applying Lemma~\ref{lem:tQ_properties}
    \begin{align*}
        b^{p,t}_h(s,a) &= b^{B,t}_h(s,a) + \frac{1}{H} \hp^t_h[\uV^t_{h+1} - \lV^t_{h+1}](s,a) \leq  b^{B,t}_h(s,a) + \frac{1}{H} \hp^t_h[\uV^{t}_{h+1}- \tV^t_{h+1} ](s,a) \\
        &\leq  b^{B,t}_h(s,a) + \frac{1}{H}\hp^t_h [ \pi^{t+1}_{h+1} G^t_{h+1}](s,a),
    \end{align*}
    thus
    \begin{align*}
        \uQ^t_h(s,a) - \tQ^t_h(s,a) &\leq \left(1 + \frac{3}{H} \right)\hp^t_h [ \pi^{t+1}_{h+1} G^t_{h+1}](s,a) + 2 \mu b^{\cH,t}_h(s,a) + 2b^{B,t}_h(s,a) \\
        &+ \frac{4 H^2 \Rmax \cdot \beta^{\KL}(\delta,n^t_h(s,a))}{n^t_h(s,a)} = G^t_h(s,a).
    \end{align*}
    
    \paragraph{Second case.} In this case we have $\tQ^t_h(s,a) = \hat r^t_{\lambda,h}(s,a) + \hp^t_h \tV^t_{h+1}(s,a) - b^{p,t}_h(s,a) - \mu b^{\cH,t}_h(s,a)$. Thus

    \[
        \uQ^t_h(s,a) - \tQ^t_h(s,a) \leq 2 \mu b^{\cH,t}_h(s,a) + 2 b^{B,t}_h(s,a) + \hp^t_h[\uV^{t}_{h+1} - \tV^t_{h+1}](s,a) + \frac{2}{H}\hp^t_h[\uV^{t}_{h+1} - \lV^t_{h+1}](s,a).
    \]
    By Lemma~\ref{lem:tQ_properties} and induction hypothesis we have 
    \[
        \uQ^t_h(s,a) - \tQ^t_h(s,a) \leq 2\mu b^{\cH,t}_h(s,a) + 2 b^{B,t}_h(s,a) + \left(1 + \frac{2}{H} \right)\hp^t_h [\pi^{t+1}_{h+1} G^t_{h+1}](s,a)  \leq G^t_h(s,a).
    \]

    \paragraph{Conclusion.} From the two cases above we conclude
    \[
        [\uQ^t_h - \tQ^t_h](s,a) \leq G^t_h(s,a).
    \]
    Moreover, we have
    \[
        \uV^t_h(s) - \tV^t_h(s) = \pi^{t+1}_h \uQ^t_h(s) -\lambda\Phi(\pi^{t+1}_h(s))  \pi^{t+1}_h \tQ^t_h(s) + \lambda\Phi(\pi^{t+1}_h(s))= \pi^{t+1}_h [\uQ^t_h - \tQ^t_h](s) \leq \pi^{t+1}_h G^t_h(s).
    \]
    The last inequality concludes the statement of Lemma~\ref{lem:reg_agnostic_stopping_rule}.
\end{proof}


\begin{lemma}\label{lem:tQ_properties}
    Under the choice of Bernstein bonuses \eqref{eq:bernstein_transition_bonuses}, on event $\cG_B(\delta)$ for any $t \in \N$ and any $(h,s,a) \in [H] \times \cS \times \cA$
    \[
        \tQ^t_h(s,a) \leq \min\{ Q^{\pi^{t+1}}_{\lambda,h}(s,a), \lQ^t_h(s,a) \}, \qquad
        \tV^t_h(s) \leq \min\{ V^{\pi^{t+1}}_{\lambda,h}(s), \lV^t_h(s) \}.
    \]
\end{lemma}
\begin{proof}
    Proceed by backward induction over $h$. The case $h=H+1$ is trivially true, assume that the statement holds for any $h' > h$ for a fixed $h$. Also let us fix $t,s,a$.  By induction hypothesis we have
    \[
        \tQ^t_h(s,a) \leq r_{\mu,h}(s,a) + p_h \tV^t_{h+1} \leq r_{\mu,h}(s,a) + p_h V^{\pi^{t+1}}_{\lambda, h+1}(s,a) = Q^{\pi^{t+1}}_{\lambda,h}(s,a).
    \]
    In the same manner
    \begin{align*}
        \tQ^t_h(s,a) &\leq \hat r^t_{\mu,h}(s,a) + \hp^t_h \tV^t_{h+1} - b^{p,t}_h(s,a) - \mu b^{\cH,t}_h(s,a) \\
        &\leq \hat r^t_{\mu,h}(s,a) + \hp^t_h \lV^t_{h+1} - b^{p,t}_h(s,a) - \mu b^{\cH,t}_h(s,a) = \lQ^t_h(s,a).
    \end{align*}
    Next, we prove the same inequalities for $V$-functions
    \[
        \tV^t_h(s) = \pi^{t+1}_h(s) \tQ^t_h(s,a) - \lambda \Phi(\pi^{t+1}_h(s))\leq  \pi^{t+1}_h(s) Q^{\pi^{t+1}}_{\lambda,h}(s,a) -\lambda \Phi(\pi^{t+1}_h(s)) = V^{\pi^{t+1}}_{\lambda,h}(s),
    \]
    and
    \begin{align*}
        \tV^t_h(s) = \pi^{t+1}_h \tQ^t_h(s)  -\lambda \Phi(\pi^{t+1}_h(s)) \leq  \pi^{t+1}_h \lQ^{t}_h(s) -\lambda \Phi(\pi^{t+1}_h(s)) \leq \max_{\pi \in \simplex_{\cA}}\left\{ \pi \lQ^{t}_h(s) -\lambda \Phi(\pi)\right\} = \lV^t_h(s).
    \end{align*}
\end{proof}

After defining a proper quantity for a stopping rule we may proceed with the final proof for sample-complexity of the presented algorithm \UCBVIEnt.

\begin{theorem}\label{th:reg_agnostic_sample_complexity}
    Let $\delta \in (0,1)$. Then $\UCBVIEnt$ algorithm with Bernstein bonuses \eqref{eq:bernstein_transition_bonuses} and a regularization-agnostic stopping rule $\tau$ is $(\varepsilon,\delta)$-PAC for the best policy identification in regularized MDPs. 
    
    Moreover, with probability at least $1-\delta$ the stopping time $\tau$ is bounded as follows
    \[
        \tau = \cO\left( \frac{H^3SA \Rmax^2 \log(SAH/\delta) L^4}{\varepsilon^2} + \frac{H^3SA (\log(SAH/\delta) + SL) \cdot L}{\varepsilon} \right),
    \]
    where $L = \cO(\log(SAH\Rmax/\varepsilon)) + \log\log(SAH/\delta))$.
\end{theorem}
\begin{proof}
    Notice that if $\tau = 0$, then our sample complexity bound is trivial, thus we assume that $\tau > 0$.
    Let us start from deriving an upper bound for $G^t_h(s,a)$ for $t < \tau, h \in [H], (s,a) \in \cS \times \cA$. 
    \begin{align*}
        G^t_h(s,a) &\leq 2 b^{B,t}_h(s,a) + 2 \mu b^{\cH,t}_h(s,a) + \frac{4 H^2 \Rmax \beta^{\KL}(\delta, n^t_h(s,a))}{n^t_h(s,a)} + \left(1 + \frac{3}{H} \right) \hp^t_h [\pi^{t+1}_{h+1} G^t_{h+1}](s,a) \\
        &\leq 6\sqrt{\Var_{\hp^t_h}[\uV^t_{h+1}](s,a) \frac{\beta^{\conc}(\delta, n^t_h(s,a))}{n^t_h(s,a)}} + 2\mu \sqrt{\frac{2\beta^{\cH}(\delta, n^t_h(s,a))}{n^t_h(s,a)}} + \frac{23 H^2 \Rmax \beta^{\KL}(\delta, n^t_h(s,a))}{n^t_h(s,a)}  \\
        &+ \left(1 + \frac{3}{H} \right)[\hp^t_h - p_h] [\pi^{t+1}_{h+1} G^t_{h+1}](s,a) + \left(1 + \frac{3}{H} \right)p_h [\pi^{t+1}_{h+1} G^t_{h+1}](s,a).
    \end{align*}
    By Lemma~\ref{lem:reg_directional_concentration}
    \[
        [\hp^t_h - p_h] [\pi^{t+1}_{h+1} G^t_{h+1}](s,a) \leq \frac{1}{H} p_h [\pi^{t+1}_{h+1} G^t_{h+1}](s,a) + \frac{4 H^2 \Rmax \beta^{\KL}(\delta, n^t_h(s,a))}{n^t_h(s,a)}.
    \]
    Also we have to replace the variance of the empirical model with the real variance of the value function for $\pi^{t+1}$ in order to apply the law of total variance (Lemma~\ref{lem:law_of_total_variance}). 

    Apply Lemma~\ref{lem:switch_variance} and Lemma~\ref{lem:switch_variance_bis}
    \begin{align*}
        \Var_{\hp^t_h}[\uV^t_{h+1}](s,a) &\leq 2\Var_{p_h}[\uV^t_{h+1}](s,a) + \frac{4H^2\Rmax^2\beta^{\KL}(\delta,n^t_h(s,a))}{n^t_h(s,a)} \\
        &\leq 4 \Var_{p_h}[V^{ \pi^{t+1}}_{\lambda, h+1}](s,a) + 2H\Rmax p_h [ \uV^t_{h+1} - V^{\pi^{t+1}}_{\lambda, h+1}](s,a) + \frac{4H^2\Rmax^2\beta^{\KL}(\delta,n^t_h(s,a))}{n^t_h(s,a)} .
    \end{align*}
    In the proof of Lemma~\ref{lem:reg_agnostic_stopping_rule} it was proven that
    \[
        [ \uV^t_{h+1} - V^{\pi^{t+1}}_{\lambda, h+1}](s) \leq \pi^{t+1}_{h+1} G^t_{h+1}(s),
    \]
    thus, combining with an inequality $\sqrt{a+b} \leq \sqrt{a} + \sqrt{b}$
    \begin{align*}
        6\sqrt{\Var_{\hp^t_h}[\uV^t_{h+1}](s,a) \frac{\beta^{\conc}(\delta, n^t_h(s,a))}{n^t_h(s,a)}} &\leq 12 \sqrt{\Var_{p_h}[V^{\pi^{t+1}}_{\lambda,h+1}](s,a) \frac{\beta^{\conc}(\delta, n^t_h(s,a))}{n^t_h(s,a)}} \\
        &+ 6\sqrt{ p_h[\pi^{t+1}_{h+1} G^t_{h+1}](s,a) \frac{2H\Rmax\beta^{\conc}(\delta,n^t_h(s,a))}{n^t_h(s,a)}}\\
        &+\frac{12 H \Rmax \beta^{\KL}(\delta, n^t_h(s,a)}{n^t_h(s,a)} 
    \end{align*}
    To bound the second term, we use inequality $2\sqrt{ab} \leq a + b$
    \[
        6\sqrt{ p_h[\pi^{t+1}_{h+1} G^t_{h+1}](s,a) \frac{2H\Rmax\beta^{\conc}(\delta,n^t_h(s,a))}{n^t_h(s,a)}} \leq \frac{3}{H}p_h[\pi^{t+1}_{h+1} G^t_{h+1}](s,a) + \frac{3H^2\Rmax\beta^{\conc}(\delta,n^t_h(s,a))}{n^t_h(s,a)}.
    \]
    Finally, we have the following bound on $G^t_{h}(s,a)$
    \begin{align*}
        G^t_h(s,a) &\leq 12 \sqrt{\Var_{p_h}[V^{\pi^{t+1}}_{\lambda, h+1}](s,a) \frac{\beta^{\conc}(\delta, n^t_h(s,a))}{n^t_h(s,a)}} + 2\mu\sqrt{\frac{2\beta^{\cH}(\delta, n^t_h(s,a))}{n^t_h(s,a)}} \\
        &+ \frac{54 H^2 \Rmax \beta^{\KL}(\delta, n^t_h(s,a))}{n^t_h(s,a)} + \left(1 + \frac{10}{H}\right) p_h[\pi^{t+1}_{h+1} G^t_{h+1}](s,a).
    \end{align*}
    Notice that his inequality could be rewritten in the following form
    \begin{align*}
        G^t_h(s,a) &\leq \E_{\pi^{t+1}}\biggl[12 \sqrt{\Var_{p_h}[V^{\pi^{t+1}}_{\lambda,h+1}](s,a) \frac{\beta^{\conc}(\delta, n^t_h(s,a))}{n^t_h(s,a)}} + 2\mu\sqrt{\frac{2\beta^{\cH}(\delta, n^t_h(s,a))}{n^t_h(s,a)}} \\
        &+ \frac{54 H^2 \Rmax \beta^{\KL}(\delta, n^t_h(s,a))}{n^t_h(s,a)} + \left(1 + \frac{10}{H} \right) G^t_{h+1}(s_{h+1},a_{h+1}) \biggl| (s_h,a_h) = (s,a) \biggl ],
    \end{align*}
    thus by rolling out we have
    \begin{align*}
        \pi^{t+1}_1 G^t_1(s_1) &\leq \E_{\pi^{t+1}}\biggl[ \underbrace{12 \sum_{h=1}^H \left(1 + \frac{10}{H}\right)^{h} \sqrt{\Var_{p_h}[V^{\pi^{t+1}}_{\lambda,h+1}](s_h,a_h) \frac{\beta^{\conc}(\delta, n^t_h(s_h,a_h))}{n^t_h(s_h,a_h)}}}_{\termA} \\
        &+ \underbrace{2\mu\sum_{h=1}^H \left(1 + \frac{10}{H}\right)^{h} \sqrt{\frac{2\beta^{\cH}(\delta, n^t_h(s_h,a_h))}{n^t_h(s_h,a_h)}}}_{\termB} \\
        &+ \underbrace{\sum_{h=1}^H \left(1 + \frac{10}{H}\right)^{h} \frac{54 H^2 \Rmax \beta^{\KL}(\delta, n^t_h(s_h,a_h))}{n^t_h(s_h,a_h)}}_{\termC} \biggl| s_1\biggl ],
    \end{align*}
    where $(1+10/H)^h \leq \rme^{10}$ for any $h \in [H]$. Now we bound each term separately. 

    \paragraph{Term $\termA$.} To bound this term, we apply Cauchy-Schwarz inequality
    \begin{align*}
        \termA &\leq 12 \rme^{10} \sum_{(h,s,a) \in [H] \times \cS \times \cA} d^{\pi^{t+1}}_h(s,a) \sqrt{\Var_{p_h}[V^{\pi^{t+1}}_{\lambda,h+1}](s,a) \frac{\beta^{\conc}(\delta, n^t_h(s,a))}{n^t_h(s,a)}} \\
        &\leq 12 \rme^{10} \sqrt{\sum_{(h,s,a) \in [H] \times \cS \times \cA} d^{\pi^{t+1}}_h(s,a) \Var_{p_h}[V^{\pi^{t+1}}_{\lambda,h+1}](s,a)} \cdot \sqrt{\sum_{(h,s,a) \in [H] \times \cS \times \cA} d^{\pi^{t+1}}_h(s,a) \frac{\beta^{\conc}(\delta, n^t_h(s,a))}{n^t_h(s,a)}}.
    \end{align*}
    For the first multiplier we apply the law of total variance (Lemma~\ref{lem:law_of_total_variance})
    \begin{align*}
        \sum_{h,s,a} d^{\pi^{t+1}}_h(s,a) \Var_{p_h}[V^{\pi^{t+1}}_{\lambda,h+1}](s,a) &\leq \sum_{h,s,a} d^{\pi^{t+1}}_h(s,a) \Var_{p_h}[V^{\pi^{t+1}}_{\lambda,h+1}](s,a) + \sum_{h,s}d^{\pi^{t+1}}_h(s) \Var_{\pi^{t+1}_h}[Q^{\pi^{t+1}}_{\lambda,h}](s) \\
        &= \Vvar^{\cH, \pi^{t+1}}_1(s_1) \leq H^2\Rmax^2.
    \end{align*}
    Therefore, 
    \[
        \termA \leq 24\rme^{10} H\Rmax\sqrt{\sum_{(h,s,a) \in [H] \times \cS \times \cA} d^{\pi^{t+1}}_h(s,a) \frac{\beta^{\conc}(\delta, n^t_h(s,a))}{n^t_h(s,a)}}.
    \]

    \paragraph{Term $\termB$.} For this term we may apply Jensen's inequality
    \begin{align*}
        \termB &\leq 2\mu H\rme^{10} \E_{\pi^{t+1}} \left[ \frac{1}{H}\sum_{h=1}^H \sqrt{\frac{2\beta^{\cH}(\delta, n^t_h(s_h,a_h))}{n^t_h(s_h,a_h)}} \bigg| s_1 \right] \leq \mu \sqrt{8H}\rme^{10}\sqrt{\sum_{h,s,a} d^{\pi^{t+1}}_h(s,a) \frac{\beta^{\cH}(\delta, n^t_h(s,a))}{n^t_h(s,a)}}.
    \end{align*}

    By summing up and replacing counts by pseudo-counts by Lemma~\ref{lem:cnt_pseudo} we obtain
    \begin{align*}
        \pi^{t+1}_1 G^t_1(s_1) &\leq 48\rme^{10} H\Rmax\sqrt{\sum_{(h,s,a) \in [H] \times \cS \times \cA} d^{\pi^{t+1}}_h(s,a) \frac{\beta^{\conc}(\delta, \upn^t_h(s,a))}{\upn^t_h(s,a) \vee 1}} \\
        &+ 4\mu \rme^{10} \sqrt{2H}\sqrt{\sum_{h,s,a} d^{\pi^{t+1}}_h(s,a) \frac{\beta^{\cH}(\delta, \upn^t_h(s,a))}{\upn^t_h(s,a) \vee 1}} \\
        &+  4\rme^{10} H^2 \Rmax  \sum_{h,s,a} d^{\pi^{t+1}}_h(s,a)  \frac{\beta^{\KL}(\delta, \upn^t_h(s,a))}{\upn^t_h(s,a) \vee 1}.
    \end{align*}

    The last step is to notice that for $t < \tau$ we have $\pi^{t+1}_1 G^t_1(s_1) \geq \varepsilon$, thus summing over all $t < \tau$ we have
    \begin{align*}
        (\tau-1) \varepsilon &\leq 48\rme^{10} H\Rmax \sum_{t=1}^{\tau-1}\sqrt{\sum_{(h,s,a) \in [H] \times \cS \times \cA} d^{\pi^{t+1}}_h(s,a) \frac{\beta^{\conc}(\delta, \upn^t_h(s,a))}{\upn^t_h(s,a) \vee 1}} \\
        &+ 4\mu \rme^{10} \sqrt{2H} \sum_{t=1}^{\tau-1}\sqrt{\sum_{h,s,a} d^{\pi^{t+1}}_h(s,a) \frac{\beta^{\cH}(\delta, \upn^t_h(s,a))}{\upn^t_h(s,a) \vee 1}} \\
        &+  4\rme^{10} H^2 \Rmax  \sum_{t=1}^{\tau-1} \sum_{h,s,a} d^{\pi^{t+1}}_h(s,a)  \frac{\beta^{\KL}(\delta, \upn^t_h(s,a))}{\upn^t_h(s,a) \vee 1}.
    \end{align*}

    Also we notice that $\beta^{\KL}(\delta, \cdot), \beta^{\conc}(\delta, \cdot), \beta^{\cH}(\delta, \cdot)$ are monotone, thus we may replace $\upn^t_h(s,a)$ with a stopping time $\tau$. Thus, by Jensen's inequality
    \begin{align*}
        (\tau-1) \varepsilon &\leq 48\rme^{10} H\Rmax \sqrt{(\tau-1) \cdot \beta^{\conc}(\delta, \tau-1)}\sqrt{\sum_{t=1}^{\tau-1} \sum_{(h,s,a) \in [H] \times \cS \times \cA} d^{\pi^{t+1}}_h(s,a) \frac{1}{\upn^t_h(s,a)\vee 1} } \\
        &+ 4\mu \rme^{10} \sqrt{2H \beta^{\cH}(\delta, \tau) \cdot (\tau-1)} \sqrt{\sum_{t=1}^{\tau-1} \sum_{h,s,a} d^{\pi^{t+1}}_h(s,a) \frac{1}{\upn^t_h(s,a) \vee 1}} \\
        &+  4\rme^{10} H^2 \Rmax\beta^{\KL}(\delta, \tau-1)  \sum_{t=1}^{\tau-1} \sum_{h,s,a} d^{\pi^{t+1}}_h(s,a)  \frac{1}{\upn^t_h(s,a) \vee 1}.
    \end{align*}
    Furthermore, notice 
    \[
        \sum_{t=1}^{\tau-1} d^{\pi^{t+1}}_h(s,a)  \frac{1}{\upn^t_h(s,a) \vee 1} = \sum_{t=1}^{\tau-1} \frac{\upn^{t+1}_h(s,a) - \upn^t_h(s,a)}{\upn^t_h(s,a) \vee 1},
    \]
    thus Lemma~\ref{lem:sum_1_over_n} is applicable:
        \begin{align*}
        (\tau-1) \varepsilon &\leq 96\rme^{10} \Rmax \sqrt{(\tau-1)H^3SA \log(\tau) \cdot \beta^{\conc}(\delta, \tau-1)} \\
        &+ 2\mu\rme^{10} \sqrt{2H^2 SA \log^3(\tau) \beta^{\cH}(\delta, \tau) \cdot (\tau-1)} \\
        &+  8\rme^{10} H^3SA \log(\tau) \Rmax\beta^{\KL}(\delta, \tau-1).
    \end{align*}
    By the definitions of $\beta^{\KL}, \beta^{\conc}, \beta^{\cH}$ we have the following inequality
    \begin{align*}
        (\tau-1) \varepsilon &\leq 96\rme^{10} \Rmax \sqrt{(\tau-1)H^3SA \log(\tau) \cdot (\log(16SAH/\delta) + 2 \log(\rme \tau)) } \\
        &+ 12\mu\rme^{10} \sqrt{H^2 SA (\tau-1) \log^3(\tau) \cdot (\log(4SAH/\delta) + 2\log(\rme \tau) ) } \\
        &+  16\rme^{10} H^3SA \log(\tau) \Rmax (\log(4SAH/\delta) + S\log(\rme \tau)).
    \end{align*}
    Under assumption $\tau \geq 2$ we can proceed with the further simplifications
    \begin{align}\label{eq:tau_inequality}
        \begin{split}
            \tau \varepsilon &\leq 216\rme^{10} \Rmax \sqrt{\tau H^3SA \log^3(\tau) \cdot (\log(16 SAH/\delta) + 2 \log(\rme \tau)) } \\
        &+  32\rme^{10} H^3SA \log(\tau) \Rmax (\log(16SAH/\delta) + S\log(\rme \tau)).
        \end{split}
    \end{align}

    Let us define the following constants
    \[
        A = 216\rme^{10} \Rmax \cdot \sqrt{\frac{H^3SA}{\varepsilon^2}}, \quad B =\log(16 SAH/\delta), \quad C = \frac{32\rme^{10} \cdot H^3 SA \Rmax }{\varepsilon}.
    \]
    Then inequality~\eqref{eq:tau_inequality} has the following form
    \[
        \tau \leq A\sqrt{\tau(B + 2\log(\rme \tau)) \cdot \log^3(\tau)} + C(B + S \log(\rme \tau))\log \tau.
    \]
    First, we obtain a loose inequality on $\tau$. Let us use the inequality $\log(x) \leq x^\beta / \beta$ for any $x > 0, \beta > 0$ with different $\beta$ for each logarithm
    \begin{align*}
        \tau &\leq A \sqrt{216\tau (B + 4(\rme \tau)^{1/4}) \tau^{1/2}}+ 4C(B + 8S/3 (\rme \tau)^{3/8}) \tau^{1/4} \\
        \Rightarrow \tau^{3/4} &\leq \tau^{3/8} \left( 6A\sqrt{6(B + 4 \rme^{1/4})} + 12CS \rme^{3/8} \right) + 4CB.
    \end{align*}
    
    Notice that the solution to the inequality $x^2 \leq ax + b$ could be upper-bounded as follows
    \[
        x \leq \frac{a + \sqrt{a^2 + 4b}}{2} \leq a + \sqrt{b},
    \]
    thus
    \[
        \tau^{3/8} \leq \left( 6A\sqrt{6(B + 4 \rme^{1/4})} + 12CS \rme^{3/8} \right) + 2\sqrt{CB}.
    \]
    Define $L = 8/3 \log\left( 6A\sqrt{6(B + 4 \rme^{1/4})} + 12CS \rme^{3/8} + 2\sqrt{CB}\right) = \cO(\log(SAH\Rmax/\varepsilon) + \log\log(SAH/\delta) )$ and we have $\log(\tau) \leq L$. Then we have that the solution to \eqref{eq:tau_inequality} is a subset of solutions to
    \[
        \tau \leq A\sqrt{\tau(B + 2(1+L)) \cdot L^3} + C(B + S(1+L))L,
    \]
    solving this inequality we obtain the bound
    \[
        \tau \leq 2A^2(B + 2(1+L))L^3 + 2CB(S(1+L))L.
    \]    
\end{proof}


\newpage
\section{Fast Rates for MTEE and Regularized MDPs}\label{app:fast_rates_regularized}

In this section we describe an algorithm that will achieve $\tcO(\poly(S,A,H)/\varepsilon)$ sample complexity for regularized MDPs. Additionally, we show that this algorithm could be used for reward-free exploration under regularization.

\subsection{\RFExploreEnt Algorithm}

We leverage the reward-free exploration approach by \citet{jin2020reward-free}. Our algorithm is split into two phases: the first phase is devoted to reward-free exploration, and on the second phase the collected samples are used to build estimates of transition probabilities and entropy of transitions. The main idea is that regularization allows us to collect much smaller number of samples to control the policy error.


\paragraph{Exploration phase} We first learn a policy that visit uniformly the MDP. To this aim for each state $s'\in\cS$ at each step $h'\in[H]$ we build the reward that put one on this state at step $h$ and zeros everywhere else $r_h(s,a) = \ind\{(s,h)=(s',h')\}$. We note that the reward function does not depend on action taken.
We then run the \EULER algorithm for $N_0$ episodes in the MDP equipped with the reward $r$ and denote by $\tilde{\Pi}_{s',h'}$ the set of $N_0$ policies used by \EULER to interact with the MDP. We modify this set of policies by forcing to act uniformly at the goal state $s'$ into the set 

\[\Pi_{s',h'} = \Bigg\{\pi'_{h}(a|s) = \begin{cases} 1/A &\text{if } s=s', h=h'\\ \pi_h(s,a)&\text{else} \end{cases}:\ \pi\in\tilde{\Pi}_{s',h'}\Bigg\}\,.\]
We define the (non-Markovian) policy $\pi^{\mathrm{mix}}$ as the uniform mixture of the policies $\{\pi\in\Pi_{s,h}, (s,h)\in\cS\times[H]\}$ we just constructed. As proved by \citet{jin2020reward-free} the policy $\pi^{\mathrm{mix}}$ is built such that it will visit almost uniformly all the states that can be reached in the MDP from the initial state. Before precising this property we need to introduce the notion of significant state.

\begin{definition}\label{def:significant_states}
    A state $s$ at step $h$ is called $\varepsilon'$-significant if there exists a policy $\pi$ such that the visitation probability of $s$ under policy $\pi$ is greater than $\varepsilon'$:
    \[
        \max_{\pi} d^\pi_h(s) \geq \varepsilon'.
    \]
    The set of all $\varepsilon'$-significant state-step pairs is called $S_{\varepsilon'}$.
\end{definition}
We reproduce here the result by \citet{jin2020reward-free} that shows that the policy $\pi^{\mathrm{mix}}$ will visit any significant state with a large enough probability.
\begin{theorem}[Theorem 3.3 by \citealt{jin2020reward-free}]\label{th:rf_explore_sampling} There exists an absolute constant $c > 0$ such that for any $\varepsilon' > 0$ and $\delta \in (0,1)$, if we set the parameter $N_0 \geq c S^2 A H^4 L/\varepsilon'$ where $L = \log(SAH/(\delta \varepsilon'))$, then with probability at least $1-\delta/3$ the following event holds
\[
    \cE^{\RFExplore}(\delta, \varepsilon')  = \left\{ \forall (s,h) \in S_{\varepsilon'}, \forall a \in \cA, \forall \pi:  \frac{d^\pi_h(s,a)}{\mu_h(s,a)} \leq 2 SAH \right\},
\]
where we denote the visitation distribution of policy $\pi^{\mathrm{mix}}$ by $\mu_h(s,a)= d^{\pi^{\mathrm{mix}}}_h(s,a)$.
\end{theorem}
\begin{remark} Note that the space complexity of \RFExploreEnt is very large since we need to store all the intermediate policies in order to construct $\pi^{\mathrm{mix}}$.
\end{remark}

This policy $\pi^{\mathrm{mix}}$ is then used to collect $N$ new independent trajectories $(z_n)_{n\in[N]}$ where $z_n=  (s^n_1,a^n_1,\ldots,s^n_H, a^n_H, s^n_{H+1})$ by following the policy $\pi^{\mathrm{mix}}$ in the original MDP. Next define the set $\cD= \{(s^n_h,a^n_h,s^n_{h+1}) ,h\in[H], n \in[N]\}$ consisting of the transitions in the sampled trajectories.




% \todoDa{Add sample complexity information}
% \todoPi{TODO Put the definition of the event in the theorem. And define $\pi^{\mathrm{mix}}$ and $\mu_h(s,a) = d_h^{\pi^{\mathrm{mix}}}(s,a)$.}
% \db{what is $N_0$ ? how is it related to complexity of the algorithm ?}
% \todoDa{Parameter of the algorithm, will be additional $SHN_0$ (basically, it is a number of iteration of Euler algorithm with reward that will be $1$ on visitation of state-step pair $(s,h)$)}
% \db{What does it mean ``$N$ trajectories $\{ z_n\}_{n=1}^N$ sampled i.i.d. from a distribution $\mu$'' ? what are the values of $z_n$, pairs ?}
% \todoDa{It is trajectories: $z_n = (s_1, a_1, \ldots, s_H, a_H, s_{H+1})$. $\mu_h$ is a marginal distribution for state-action pairs of a mixture policy by algorithm of \cite{jin2020reward-free}. }
% \todoPi{Ok maybe we need more context from \citet{jin2020reward-free} before giving the theorem. Then we can talk about the datset only in terms of transitions (and rescale quantities by $H$.}
% \db{yes, more context would be good. otherwise many things are not clear, e.g. mixture policy }


\paragraph{Planning phase} Given the transitions collected in the exploration phase we estimate the transition probability distributions and then plan in the estimated MDP with the Bellman equations for MTEE to obtain a maximum trajectory entropy policy.

Using the dataset $\cD$, we construct estimates of transition probabilities $\{\hp_h\}_{h\in[H]}$. We first define the number of visits of a state action pair $(s,a)$ at step $h$ and the number of transitions for $(s,a)$ at step $h$ to a states $s'$ observed in the dataset $\cD$,
\[
    n_h(s,a) = \sum_{n=1}^N \ind\{ (s^n_h, a^n_h) = (s,a) \} \quad  n_h(s'|s,a) = \sum_{n=1}^N \ind\{ (s^n_h, a^n_h, s^n_{h+1}) = (s,a,s') \},
\]
The transitions are estimated using the maximum likelihood method:
\begin{align}\label{eq:hp_construction}
    \hp_h(s'|s,a) = \begin{cases}
        \frac{n_h(s'|s,a)}{n_h(s,a)} & n_h(s,a) > 0 \\
        \frac{1}{S} & n_h(s,a) = 0
    \end{cases}\,.
\end{align}
% \db{how exactly ? because the trajectories are obtained using iid data, you get constant conditional probabilities or ?}
% \todoDa{We can treat $(s_h,a_h)$ generated from any policy $\pi$ as $(s_h,a_h) \sim d^\pi_h$ i.i.d. over trajectories because trajectories are independent. So, at each layer we will have independent sample}
% \db{Does it mean that $(s_h,a_h)$ are independent for different $h$ ? I still do not understand how you  recover the conditional probabilities $s_{h+1}$ given $s_h$ and $a_h$ from $d^\pi_h.$  }
% \todoDa{They will be dependent of course. If you know policy and $d^\pi_h$, you cannot reconstruct the model (at least it is not clear to me), but why it is needed? We can reconstruct the model from the joint distribution over trajectories: $q^{\pi}(m) = \pi_1(s_1) \prod_{i=2}^H p_{i-1}(s_{i+1}|s_i, a_i)$. I don't understand what is unclear there.}
% and solve regularized Bellman equations with respect to this policy. 
% For a given policy $\pi$, we call $\hQ^\pi_h(s,a)$ the Q-values computed with estimated transitions and estimated entropy of the transitions. They could be defined by the following version of Bellman equations
Given these estimates, we can define an empirical version of the regularized Bellman equations for MTEE
\begin{align}\label{eq:hQ_definition}
    \begin{split}
        \hQ^{\pi}_{\lambda,h}(s,a) &= r_h(s,a) + \kappa \cH(\hp_h(s,a)) + \hp_h \hV^{\pi}_{\lambda, h+1}(s,a) \\
        \hV^{\pi}_{\lambda, h}(s) &= \pi \hQ^{\pi}_{\lambda, h}(s) - \lambda \Phi(\pi).
    \end{split}
\end{align}

Then the output policy $\hpi$ is the solution to the optimal regularized Bellman equations
\begin{align}\label{eq:hQ_opt_definition}
    \begin{split}
        \hQ^{\star}_{\lambda,h}(s,a) &= r_h(s,a) + \kappa \cH(\hp_h(s,a)) + \hp_h \hV^{\star}_{\lambda, h+1}(s,a) \\
        \hV^{\star}_{\lambda, h}(s) &= \max_{\pi}\left\{ \pi \hQ^{\star}_{\lambda, h}(s) - \lambda \Phi(\pi) \right\} \\
        \hpi_h(s) &= \argmax_{\pi}\left\{ \pi \hQ^{\star}_{\lambda, h}(s) - \lambda \Phi(\pi) \right\}.
    \end{split}
\end{align}
We call this algorithm \RFExploreEnt. Notably, we can extend this algorithm to the setting of the changing rewards by solving \eqref{eq:hQ_opt_definition} with new reward functions $r_h(s,a)$. The detailed description of the algorithm is presented in Algorithm~\ref{alg:RFExploreEnt}. The only difference between our algorithm and \RFExplore by \citet{jin2020reward-free} is the use of a smaller number of trajectories $N$ and solving regularized Bellman equations instead of usual one.


\subsection{Concentration Events}

In this section we describe all required concentration events.

Let $\beta^{\conc}\colon (0,1) \times \N \to \R_{+}$ and $\beta^{\cnt} \colon (0,1) \to \R_+$ be some functions defined later on in Lemma \ref{lem:fast_traj_proba_master_event}. We define the following favorable events
\begin{align*}
\cE^{\conc}(N,\delta) &\triangleq \Bigg\{ \forall h\in [H], \forall G \colon \cS \to [0, H\Rmax], \forall \nu \colon \cS \to \cA: \\
&\qquad\quad \E_{(s,a) \sim \mu_{h}}\left[ \left(\left[ \hp_{h'} - p_{h'}  \right] G(s,a) \right)^2 \ind\{\nu(s) = a \} \right] \leq  \frac{CH^2 \Rmax^2 S \cdot \beta^{\conc}(\delta, N)}{N}\Bigg\}\,,\\
\cE^{\cH}(N,\delta) &\triangleq \Bigg\{\forall h \in [H]: \E_{(s,a) \sim \mu_{h}}\left[ (\cH(\hp_{h}(s,a)) - \cH(p_{h}(s,a)))^2 \right] \leq \frac{12S^2 A \log^2(SN) \cdot \beta^{\cH}(\delta)}{N}
\Bigg\}\,,
\end{align*}
where $C$ is a some absolute constant.
We also introduce two intersections of these events of interest and $\cE^{\RFExplore}(\delta)$, defined in Theorem~\ref{th:rf_explore_sampling}: $\cG(N,\delta,\varepsilon') \triangleq \cE^{\conc}(N,\delta) \cap \cE^{\cH}(\delta) \cap \cE^{\RFExplore}(\delta, \varepsilon')$. We  prove that for the right choice of the functions $\beta^{\conc}, \beta^{\cH}$ the above events hold with high probability.
\begin{lemma}
\label{lem:fast_traj_proba_master_event}
For any $\delta \in (0,1), \varepsilon' > 0, N \in \N$ and for the following choices of functions $\beta,$
\begin{align*}
    \beta^{\conc}(\delta, N) &\triangleq \log(3AH\Rmax N/\delta),\\
    \beta^{\cH}(\delta) &\triangleq \log(12SAH/\delta),
\end{align*}
it holds that
\begin{align*}
 \P[\cE^{\conc}(\delta)]\geq 1-\delta/3, \qquad \P[\cE^{\cH}(\delta)]\geq 1-\delta/3.
\end{align*}
In particular, $\P[\cG(\delta, N,\varepsilon')] \geq 1-\delta$.
\end{lemma}
\begin{proof}
    Holds from an application of Lemma~\ref{lem:sampling_square_value_error_bound}, Lemma~\ref{lem:sampling_entropy_bound}, Theorem~\ref{th:rf_explore_sampling} and union bound.
\end{proof}


\subsection{Sample Complexity Proof}

In this section we provide the sample complexity result of \RFExploreEnt algorithm in the simple BPI setting and in the reward free setting.

\begin{theorem}\label{th:rf_explore_ent_sample_complexity}
    Algorithm \RFExploreEnt with parameters $N_0 = \Omega\left( \frac{H^7 S^3 A r_A^2  \cdot \Rmax^2 \cdot L}{\varepsilon \lambda}\right)$ and $N \geq \Omega\left( \frac{ H^6 S^3 A r_A^2 \Rmax^2 L^3 }{\varepsilon \lambda}\right)$ is $(\varepsilon,\delta)$-PAC for the best policy identification in regularized MDPs, where $L = \log(SAH/(\varepsilon \lambda \delta))$. The sample complexity is bounded by
    \[
        \tcO\left( \frac{H^8 S^4 A r_A^2 \Rmax^2}{\varepsilon \lambda} \right).
    \]
\end{theorem}
\begin{proof}
    Let us start from exploiting the strong convexity of the regularizer. This property is given by Lemma~\ref{lem:policy_error_decomposition}
    \[
        \Vstar_{\lambda,1}(s_1) - V^{\hpi}_{\lambda,1}(s_1) \leq \frac{r_A^2}{2\lambda} \sum_{h=1}^H \E_{\hpi}\left[\max_{a\in \cA } \left( \hQ^{\hpi}_{\lambda,h} - \Qstar_{\lambda,h} \right)^2(s_h,a) \middle| s_1 \right].
    \]
    Next we study each separate term in this decomposition. By the definition of $\hpi$ and $\pistar$ we have
    \[
    \hQ^{\pistar}_{\lambda,h}(s,a) - Q^{\pistar}_{\lambda,h}(s,a) \leq \hQ^{\hpi}_{\lambda,h}(s,a) - \Qstar_{\lambda,h}(s,a) \leq \hQ^{\hpi}_{\lambda,h}(s,a) - Q^{\hpi}_{\lambda,h}(s,a),
    \]
    thus
    \[
        \left( \hQ^{\pistar}_{\lambda,h}(s,a) - Q^{\pistar}_{\lambda,h}(s,a)\right)^2 \leq \max\left\{ \left( \hQ^{\hpi}_{\lambda,h}(s,a) - \Qstar_{\lambda,h}(s,a) \right)^2, \left(\hQ^{\hpi}_{\lambda,h}(s,a) - Q^{\hpi}_{\lambda,h}(s,a) \right)^2  \right\},
    \]
    and by an inequality $\max\{a,b\} \leq a+b$ for positive $a,b$ we have
    \begin{align*}
        \left( \hQ^{\hpi}_{\lambda,h}(s,a) - \Qstar_{\lambda,h}(s,a) \right)^2 &\leq \left(\hQ^{\hpi}_{\lambda,h}(s,a) - Q^{\hpi}_{\lambda,h}(s,a) \right)^2  + \left( \hQ^{\pistar}_{\lambda,h}(s,a) - Q^{\pistar}_{\lambda,h}(s,a)  \right)^2.
    \end{align*}
    Therefore, the policy error decomposes as follows
    \begin{align*}
        \Vstar_{\lambda,1}(s_1) - V^{\hpi}_{\lambda,1}(s_1) &\leq \frac{r_A^2}{2\lambda}\sum_{h=1}^H \biggl( \E_{\hpi}\left[ \max_{a\in \cA}\left(\hQ^{\hpi}_{\lambda,h}(s_h,a) - Q^{\hpi}_{\lambda,h}(s_h,a) \right)^2  \mid s_1 \right] \\
        &\qquad\qquad + \E_{\hpi}\left[ \max_{a\in \cA}\left( \hQ^{\pistar}_{\lambda,h}(s_h,a) - Q^{\pistar}_{\lambda,h}(s_h,a)  \right)^2  \mid s_1 \right]\biggl).
    \end{align*}
    Next we assume that the event $\cG(N,\delta,\varepsilon')$ holds for the values $N$ and $\varepsilon'$ that will be specified later.  Then Lemma~\ref{lem:q_square_bound} applied $2H$ times yields
    \begin{align*}
        \Vstar_{\lambda,1}(s_1) - V^{\hpi}_{\lambda,1}(s_1) &\leq \frac{r_A^2}{\lambda}\biggl( \frac{48 S^3 H^4 A \Rmax^2 \log^2(N) \log(12SAH/\delta)}{N} \\
        &+ \frac{4C H^6 \Rmax^2 S^2 A \cdot (\log(3AH\Rmax/\delta) + \log(N))}{N} + 2 S H^3 \Rmax^2 \cdot \varepsilon' \biggl).
    \end{align*}
    Next we take
    \[
        \varepsilon' = \frac{\lambda \varepsilon}{4r_A^2 S H^3 \Rmax^2},
    \]
    that requires to take $N_0 \geq \frac{cH^7 S^3 A r_A^2  \cdot \Rmax^2 \cdot L}{\varepsilon \lambda}$ for an absolute constant $c > 0$ and $L = \log(SAH/\delta) + \log(1/(\varepsilon \lambda))$. This yields $\tcO(H^8 S^4 A r_A^2 \Rmax^2 / (\varepsilon \lambda))$ sample complexity of the first phase, since we need $N_0$ samples for each $(s,h) \in \cS \times [H]$. Under this choice, we have
    \[
        \Vstar_{\lambda,1}(s_1) - V^{\hpi}_{\lambda,1}(s_1) \leq \varepsilon/2 + \frac{r^2_A}{\lambda N} \cdot \left( (48 + 4C) S^3 A H^6 \Rmax^2 \log^2(N) \cdot \log(12SAH \Rmax/\delta) \right).
    \]
    To make the second part smaller than $\varepsilon$, we have to analyze the following inequality
    \[
        \log^2(N) / N \cdot B \leq \varepsilon
    \]
    and upper bound its minimal solution that we will call $N^\star$. To do it, we first use a simple numeric bound $\log(N) \leq 4 N^{1/4}$ and obtain a simple estimate $N \geq 256 B^2 / \varepsilon^2$, thus the minimal solution $N^\star \leq 256 B^2 / \varepsilon^2$.
    Therefore, we can assume that $\log(N^\star) \leq 2\log(16B/\varepsilon) = \cO(\log(SAH\Rmax/\varepsilon + \log(1/\delta))$, thus taking
    \[
        N \geq N^\star = \Omega\left( \frac{ H^6 S^3 A r_A^2 \Rmax^2 \log^2(SAH\Rmax/\varepsilon + \log(1/\delta)) \cdot \log(SAH \Rmax/\delta) }{\varepsilon \lambda}\right).
    \]
    is enough to guarantee that the policy error is smaller than $\varepsilon$.
\end{proof}

Notice that in the proof we do not rely on one particular reward function, since the only we need is conditioning on event $\cG(\delta)$ that does not depend on the particular reward function.
\begin{corollary}\label{cor:rf_explore_ent_rf_sample_complexity}
    Algorithm \RFExploreEnt for a  choice $N_0 = \Omega\left( \frac{H^7 S^3 A r_A^2  \cdot \Rmax^2 \cdot L}{\varepsilon \lambda}\right)$ and $N \geq \Omega\left( \frac{ H^6 S^3 A r_A^2 \Rmax^2 L^3 }{\varepsilon \lambda}\right)$ outputs $\varepsilon$-optimal policies for an arbitrary number of reward functions in regularized MDPs. The sample complexity is bounded by
    \[
        \tcO\left( \frac{H^8 S^4 A r_A^2 \Rmax^2}{\varepsilon \lambda} \right).
    \]
\end{corollary}

Finally, we provide a formal proof for application of this algorithm to the MTEE problem, that is a simple application of the results above.
\begin{theorem}\label{th:mtee_fast_rates}
    Algorithm \RFExploreEnt with parameters $N_0 = \Omega\left( \frac{H^7 S^3 A \cdot L^3}{\varepsilon}\right)$ and $N = \Omega\left( \frac{ H^6 S^3 A  L^5 }{\varepsilon}\right)$ is $(\varepsilon,\delta)$-PAC for the MTEE problem, where $L = \log(SAH/(\varepsilon \delta))$. The total sample complexity $SHN_0 + N$ is bounded by
    \[
        \tcO\left( \frac{H^8 S^4 A}{\varepsilon} \right).
    \]
\end{theorem}
\begin{proof}
    Fix $\Phi(\pi) = -\cH(\pi), \kappa = \lambda = 1$ and $r_{\max} = 0$. By 1-strong convexity of $-\cH(\pi)$ with respect to $\ell_1$-norm, its dual is 1-strongly convex with respect to $\ell_\infty$ norm, yielding $r_A = 1$. Also we have $\Rmax = \log(SA)$, thus by \thmref{th:rf_explore_ent_sample_complexity} we conclude the statement.
\end{proof}


\subsection{Technical Lemmas}


\begin{lemma}\label{lem:policy_error_decomposition}
    Let $\pi$ be a greedy policy with respect to regularized Q-values $\uQ_{\lambda,h}(s,a): \pi_h(s) = \argmax_{\pi} \{ \pi \uQ_{\lambda,h}(s) - \lambda \Phi(\pi) \}$. Then the following error decomposition holds
    \[
        \Vstar_{\lambda,1}(s_1) - V^{\pi}_{\lambda,1}(s_1) \leq \frac{r_A^2}{2\lambda}\E_{\pi}\left[ \sum_{h=1}^H \max_{a\in \cA } \left( \uQ_{\lambda,h} - \Qstar_{\lambda,h} \right)^2(s_h,a) \mid s_1 \right],
    \]
    where $r_A$ is a constant defined in \eqref{eq:norm_equivalence}.
\end{lemma}
\begin{proof}
    First, we formulate the statement dependent of $h$
    \begin{equation}\label{eq:policy_error_h}
        \Vstar_{\lambda,h}(s) - V^{\pi}_{\lambda,h}(s) \leq \frac{r_A^2}{2\lambda}\E_{\pi}\left[ \sum_{h'=h}^H \max_{a\in \cA } \left( \uQ_{\lambda,h'} - \Qstar_{\lambda,h'} \right)^2(s_{h'},a) \mid s_h=s \right].
    \end{equation}
    Notice that for $h=1$ and $s=s_1$ the initial statement is recovered. We proceed by induction over $h$. The initial case $h=H+1$ is trivial, next we assume that the statement \eqref{eq:policy_error_h} is true for any $h' > h$.

    We start the analysis from understanding the policy error by applying the smoothness of $F_\lambda$ for any $h$.
    \begin{align*}
        \Vstar_{\lambda,h}(s) - V^{\pi}_{\lambda,h}(s) &= F_\lambda(\Qstar_{\lambda,h}(s, \cdot)) - \left(\pi_h Q^{\pi}_{\lambda,h}(s, \cdot)  -\lambda \Phi(\pi_h(s)) \right) \\
        &\leq F_\lambda(\uQ_h)(s) + \langle \nabla F_\lambda(\uQ_h(s,\cdot)), \Qstar_{\lambda,h}(s,\cdot) - \uQ_h(s,\cdot)  \rangle + \frac{1}{2\lambda} \norm{\uQ_h - \Qstar_{\lambda,h}}_*^2(s) \\
        &- \left(\pi_h Q^{\pi}_{\lambda,h}(s, \cdot)  -\lambda \Phi(\pi_h(s)) \right).
    \end{align*}
    Next we recall that
    \[
         \pi_h(s) = \nabla F(\uQ_h(s,\cdot)), \quad F(\uQ_h)(s)  =  \pi_h \uQ_h(s) - \lambda \Phi(\pi_h(s)),
    \]
    thus we have
    \[
        F(\uQ_h)(s) - \left( \pi_h Q^{\pi}_{\lambda,h}(s, \cdot) - \lambda \Phi(\pi_h(s))  \right) = \pi_h [ \uQ_h - Q^{\pi}_{\lambda,h}](s)
    \]
    and, by Bellman equations
    \begin{align*}
        \Vstar_{\lambda,h}(s) - V^{\pi}_{\lambda,h}(s) &\leq \pi_h \left[ \Qstar_{\lambda,h} - Q^{\pi}_{\lambda,h} \right] (s) + \frac{1}{2\lambda} \norm{\uQ_h - \Qstar_{\lambda,h}}_*^2(s) \\
        &\leq \pi_h p_h \left[ \Vstar_{\lambda,h+1} - V^{\pi}_{\lambda,h+1} \right] (s) + \frac{1}{2\lambda} \norm{\uQ_h - \Qstar_{\lambda,h}}_*^2(s).
    \end{align*}
    Applying norm equivalence \eqref{eq:norm_equivalence} we have
    \[
        \Vstar_{\lambda,h}(s) - V^{\pi}_{\lambda,h}(s) \leq \E_{\pi}\left[ \frac{r_A^2}{2\lambda} \norm{\uQ_h - \Qstar_{\lambda,h}}_\infty^2(s_h) + \Vstar_{\lambda,h+1}(s_{h+1}) - V^{\pi}_{\lambda,h+1}(s_{h+1})  \mid s_h = s\right].
    \]
    By induction hypothesis we conclude the statement.
\end{proof}


\begin{lemma}\label{lem:q_square_bound}
    For any policy $\pi$ the following holds on event $\cG(N,\delta,\varepsilon')$ defined in Lemma~\ref{lem:fast_traj_proba_master_event} for any $h \in [H]$
    \begin{align*}
        \E_{\hpi}\left[ \max_{a\in \cA} \left( \hQ^{\pi}_{\lambda,h} - Q^\pi_{\lambda,h} \right)^2(s_h,a) \mid s_1 \right] &\leq \frac{48 S^3 H^3 A \Rmax^2 \log^2(N) \beta^{\cH}(\delta)}{N} + \frac{4C H^5 \Rmax^2 S^2 A \cdot \beta^{\conc}(\delta,N)}{N} \\
        &+ 2 S H^2 \Rmax^2 \cdot \varepsilon'.
    \end{align*}
\end{lemma}
\begin{proof}
    By performance-difference Lemma~\ref{lm:performance_difference} and form of the rewards stated in \eqref{eq:hQ_definition} we have for any $(s,a,h)\in \cS \times\cA\times[H]$
    \begin{align*}
        \hQ^{\pi}_{\lambda,h}(s,a) - Q^\pi_{\lambda,h}(s,a) &= \kappa \E_{\pi}\left[ \sum_{h'=h}^H \cH(\hp_{h'}(s_{h'},a_{h'})) - \cH(p_{h'}(s_{h'}, a_{h'})) \mid (s_h,a_h) = (s,a) \right] \\
        &+ \E_{\pi}\left[ \sum_{h'=h}^H \left[ \hp_{h'} - p_{h'}\right] \hV^{\pi}_{\lambda, h'+1}(s_{h'}, a_{h'}) \mid (s_h,a_h) = (s,a)\right].
    \end{align*}
    
    Next we analyze all required expectation for one fixed value $h\in[H]$. By Jensen's inequality and a simple algebraic inequality $(a+b)^2 \leq 2a^2 + 2b^2$
    \begin{align*}
        \left(\hQ^{\pi}_{\lambda,h}(s,a) - Q^\pi_{\lambda,h}(s,a)\right)^2 &\leq 2\kappa^2 \E_{\pi}\left[ \left(\sum_{h'=h}^H \cH(\hp_{h'}(s_{h'},a_{h'})) - \cH(p_{h'}(s_{h'}, a_{h'})\right)^2 \bigg| (s_h,a_h) = (s,a) \right] \\
        &+ 2\E_{\pi}\left[ \left( \sum_{h'=h}^H \left[ \hp_{h'} - p_{h'}\right] \hV^{\pi}_{\lambda, h'+1}(s_{h'}, a_{h'})\right)^2 \bigg| (s_h,a_h) = (s,a)\right].
    \end{align*}
    By Cauchy–Schwarz inequality we have the final form
    \begin{align}
        \begin{split}\label{eq:performance_difference_dq}
        \left(\hQ^{\pi}_{\lambda,h}(s,a) - Q^\pi_{\lambda,h}(s,a)\right)^2 &\leq 2H \E_{\pi}\biggl[ \sum_{h'=h}^H \kappa^2\left(\cH(\hp_{h'}(s_{h'}, a_{h'})) - \cH(p_{h'}(s_{h'}, a_{h'})) \right)^2 \\
        &+ \sum_{h'=h}^H \left(\left[ \hp_{h'} - p_{h'}\right] \hV^{\pi}_{\lambda, h'+1}\right)^2(s_{h'}, a_{h'}) \mid (s_h,a_h) = (s,a) \biggl]
        \end{split}
    \end{align}
    To connect this conditional expectation in the right-hand side of \eqref{eq:performance_difference_dq} with conditional expectation over $\hpi,$ we define the following policy
    \[
        \tpi_{h'}(a|s) = \begin{cases}
            \hpi_{h'}(a|s) & h' < h \\
            \ind\{ a = \argmax_{a\in \cA} \left\{ \left(\hQ^{\pi}_h(s,a) - Q^\pi_h(s,a)\right)^2  \right\} & h' = h \\
            \pi_h(a|s) & h' > h.
        \end{cases}
    \]
    Since we do not change the policy $\pi$ for steps greater than $h$, we can replace $\pi$ with $\tpi$ in \eqref{eq:performance_difference_dq}. Additionally, since this policy is equal to $\hpi$ for the first $h-1$ steps, the distribution $d^{\hpi}_h(s_h)$ is equal to $d^{\tpi}_h(s_h)$:
    \begin{align*}
        d^{\hpi}_h(s_h) &= \sum_{s_1,a_1,\ldots,s_{h-1},a_{h-1}} \hpi_1(a_1|s_1) \left( \prod_{h'=1}^{h-1} p_{h'-1}(s_{h'}|s_{h'-1},a_{h'-1}) \hpi_{h'}(a_{h'}|s_{h'}) \right) \cdot p_{h-1}(s_h|s_{h-1},a_{h-1}) \\
        &= \sum_{s_1,a_1,\ldots,s_{h-1},a_{h-1}} \tpi_1(a_1|s_1) \left( \prod_{h'=1}^{h-1} p_{h'-1}(s_{h'}|s_{h'-1},a_{h'-1}) \tpi_{h'}(a_{h'}|s_{h'}) \right) \cdot p_{h-1}(s_h|s_{h-1},a_{h-1}) = d^{\tpi}_h(s_h).
    \end{align*}
    Therefore
    \begin{align*}
        \E_{\hpi}\left[ \max_{a\in \cA} \left( \hQ^{\pi}_{\lambda,h} - Q^\pi_{\lambda,h} \right)^2(s_h,a) \mid s_1 \right] &= \sum_{s} d^{\hpi}_h(s) \max_{a\in \cA}\left( \hQ^{\pi}_{\lambda,h} - Q^\pi_{\lambda,h} \right)^2(s,a) \\
        &=\sum_{s} d^{\tpi}_h(s) \max_{a\in \cA}\left( \hQ^{\pi}_{\lambda,h} - Q^\pi_{\lambda,h} \right)^2(s,a) \\
        &=  \E_{\tpi}\left[ \max_{a\in \cA} \left( \hQ^{\pi}_{\lambda,h} - Q^\pi_{\lambda,h} \right)^2(s_h,a) \mid s_1 \right] = \E_{\tpi}\left[  \left( \hQ^{\pi}_{\lambda,h} - Q^\pi_{\lambda,h} \right)^2(s_h,a_h) \mid s_1 \right].
    \end{align*}
    Next we show that in \eqref{eq:performance_difference_dq} we can make a change of policy from $\pi$ to $\tpi$. It is enough to show that the required marginal distributions are equal for all $h' \geq h$, i.e. for any $(s,a) \in \cS\times \cA$ it holds $\P_{\pi}[(s_{h'},a_{h'}) | (s_h,a_h)] = \P_{\tpi}[(s_{h'},a_{h'}) | (s_h,a_h)]$. For $h'=h$ this probability is an indicator on $(s_h,a_h)$, so it does not depend on policy. For the general case $h'>h$ we can use Markov property and imply
    \begin{align*}
        \P_{\pi}[(s_{h'},a_{h'}) | (s_h,a_h)] &= \sum_{(s_{h+1},a_{h+1},\ldots,s_{h'-1},a_{h'-1})} \prod_{i=h}^{h'-1} \pi_{i+1}(a_{i+1}|s_{i+1}) p_{i}(s_{i+1} | s_i, a_i) \\
        &= \sum_{(s_{h+1},a_{h+1},\ldots,s_{h'-1},a_{h'-1})} \prod_{i=h}^{h'-1} \tpi_{i+1}(a_{i+1}|s_{i+1}) p_{i}(s_{i+1} | s_i, a_i) 
        &= \P_{\tpi}[(s_{h'},a_{h'}) | (s_h,a_h)].
    \end{align*}
    Therefore, we can make change of measure in \eqref{eq:performance_difference_dq} and obtain
    \begin{align*}
        \E_{\hpi}\left[ \max_{a\in \cA} \left( \hQ^{\pi}_{\lambda,h} - Q^\pi_{\lambda,h} \right)^2(s_h,a) \mid s_1 \right]
        &\leq 2H \E_{\tpi} \biggl[ \E_{\tpi}\left[\sum_{h'=h}^H \kappa^2\left(\cH(\hp_{h'}(s_{h'}, a_{h'})) - \cH(p_{h'}(s_{h'}, a_{h'})) \right)^2   \mid s_h, a_h\right] \\
        &\quad+ \E_{\tpi}\left[\sum_{h'=h}^H  \left(\left[ \hp_{h'} - p_{h'}\right] \hV^{\pi}_{\lambda, h'+1}\right)^2(s_{h'}, a_{h'})\mid s_h,a_h\right]\ \bigg|\ s_1 \biggl].
    \end{align*}
    By the properties of conditional expectation we can eliminate the inner expectation and obtain the following upper bound
    \begin{align*}
        \E_{\hpi}\left[ \max_{a\in \cA} \left( \hQ^{\pi}_{\lambda,h} - Q^\pi_{\lambda,h} \right)^2(s_h,a) \mid s_1 \right] &\leq 2H\kappa^2 \sum_{h'=h}^H \E_{\tpi}\left[ \left(\cH(\hp_{h'}(s_{h'}, a_{h'})) - \cH(p_{h'}(s_{h'}, a_{h'})) \right)^2  \ \big|\ s_1 \right] \\
        &+ 2H \sum_{h'=h}^H \E_{\tpi}\left[ \left(\left[ \hp_{h'} - p_{h'}\right] \hV^{\pi}_{\lambda, h'+1}\right)^2(s_{h'}, a_{h'}) \ \bigg|\ s_1\right].
    \end{align*}
    Applying Lemma~\ref{lem:change_measure} and the definition of $\cE^{\cH}(\delta)$ we have
    \begin{align*}
        \E_{\tpi}\left[ \left(\cH(\hp_{h'}(s_{h'}, a_{h'})) - \cH(p_{h'}(s_{h'}, a_{h'})) \right)^2  \mid s_1 \right] &\leq 2SAH \E_{(s,a) \sim \mu_{h'}}\left[ \left(\cH(\hp_{h'}(s, a)) - \cH(p_{h'}(s, a)) \right)^2  \right] + S \log^2(S) \varepsilon' \\
        &\leq \frac{24 S^3 H A \log^2(SN) \beta^{\cH}(\delta)}{N} + S \log^2(S) \varepsilon'.
    \end{align*}

    In the same way by Lemma~\ref{lem:change_measure} and the definition of event $\cE^{\conc}(N,\delta)$
    \begin{align*}
        \E_{\tpi}\left[ \left(\left[ \hp_{h'} - p_{h'}\right] \hV^{\pi}_{\lambda, h'+1}\right)^2(s_{h'}, a_{h'}) \mid s_1\right] &\leq 2SAH\E_{(s,a) \sim \mu_{h'}}\left[ \left(\left[ \hp_{h'} - p_{h'}\right] \hV^{\pi}_{\lambda, h'+1}\right)^2(s, a)\right] + SH^2 \Rmax^2 \varepsilon' \\
        &\leq \frac{2C H^3 \Rmax^2 S^2 A \cdot \beta^{\conc}(\delta, N) }{N} + SH^2 \Rmax^2 \varepsilon'.
    \end{align*}
    Combining these two upper bounds, we have
    \begin{align*}
        \E_{\hpi}\left[ \max_{a\in \cA} \left( \hQ^{\pi}_{\lambda,h} - Q^\pi_{\lambda,h} \right)^2(s_h,a) \mid s_1 \right] &\leq \frac{48 S^3 H^3 A \kappa^2 \log^2(SN) \beta^{\cH}(\delta)}{N} + \frac{4C H^5 \Rmax^2 S^2 A \cdot \beta^{\conc}(\delta,N)}{N} \\
        &+ S (H^2 \Rmax^2 + \kappa^2 \log^2(S))\varepsilon'.
    \end{align*}
    Since $\kappa^2 \log^2(s) \leq \Rmax^2$ and $H \geq 1$, we conclude the statement.
\end{proof}

\begin{lemma}\label{lem:change_measure}
    For any bounded function $f \colon \cS \times \cA \to \R_+, f(s,a) \leq B$ for any policy $\pi$ and step $h$ on event $\cE^{\RFExplore}(\delta, \varepsilon')$ the following holds
    \[
        \E_{\pi}\left[ f(s_h,a_h) | s_1 \right] \leq 2SAH\E_{(s,a) \sim \mu_h}\left[ f(s,a) \right] + BS \varepsilon'.
    \]
\end{lemma}
\begin{proof}
    Recall $S_{\varepsilon', h}$ be a set of all $\varepsilon'$-significant states (see Definition~\ref{def:significant_states}) at step $h$. Then we can rewrite this expectation as follows
    \[
        \E_{\pi}\left[ f(s_h,a_h) | s_1 \right] = \sum_{a \in \cA, s \in S_{\varepsilon', h}} d^\pi_h(s,a) f(s,a) + \sum_{a \in \cA, s \not \in S_{\varepsilon', h}} d^\pi_h(s,a) f(s,a).
    \]
    For the first sum by Theorem~\ref{th:rf_explore_sampling} we have $d^\pi_h(s,a) \leq 2SAH \mu_h(s,a)$, thus
    \[
        \sum_{a \in \cA, s \in S_{\varepsilon', h}} d^\pi_h(s,a) f(s,a)  \leq 2SAH \sum_{(s,a) \in \cS \times \cA} \mu_h(s,a) f(s,a) = 2SAH \E_{(s,a) \sim \mu_h}\left[ f(s,a) \right].
    \]
    For the second sum we apply $f(s,a) \leq B$ and the fact that for all states that are not $\varepsilon'$-significant under any policy $d^\pi_h(s) \leq \varepsilon'$:
    \[
         \sum_{a \in \cA, s \not \in S_{\varepsilon', h}} d^\pi_h(s,a) f(s,a) \leq B  \sum_{a \in \cA, s \not \in S_{\varepsilon', h}} d^\pi_h(s,a) = B \sum_{s \not \in S_{\varepsilon', h}} d^\pi_h(s) \leq BS \varepsilon'.
    \]
\end{proof}
\newpage
\section{Faster Rates for Visitation Entropy}
\label{app:reg_visitation_entropy_proofs}

\subsection{Algorithm description}
Let us start from the description of the modified algorithm \regalgMVEE. It has a similar game-theoretical foundation as it aims at solving the following minimax game
\begin{align*}
    \max_{d\in\cK_p} \VE(d) &= \max_{d\in\cK_p} \min_{\bd\in\cK}\sum_{(h,s,a)} d_h(s,a) \log \frac{1}{\bd_h(s,a)} =  \min_{\bd\in\cK} \max_{d\in\cK_p} \sum_{(h,s,a)} d_h(s,a) \log \frac{1}{\bd_h(s,a)}.
\end{align*}
As for usual \algMVEE, there are two players in the  game. On the one hand, the min player, or forecaster player, tries to predict which state-action pairs the max player will visit to minimize $\KL(d_h,\bd_h)$.  On the other hand, the max player, or sampler player, is rewarded for visiting state-action pairs that the forecaster player did not predict correctly.

We now describe the algorithm \regalgMVEE\ for MVEE. In this algorithm, we let a forecaster player and a sampler player compete for $T$ episodes long. Let us first define the two players.
\paragraph{Forecaster-player} The forecaster player remains exactly the same as for usual \algMVEE algorithm, see the corresponding section in the main text.

\paragraph{Regularized Sampler-player} For the sampler player we exploit strong convexity of visitation entropy. The running time of the sampler player will be divided onto two stages, as it was done in \RFExploreEnt algorithm.

\paragraph{Exploration phase}

Before the start of the game, the sampler-player uses some preprocessing time in order to explore the environment to learn a simple (non-markovian) preliminary exploration policy $\pi^{\mathrm{mix}}$. This policy is used to construct an accurate enough estimates of transition probabilities. This policy is obtained, as in \RFExplore and \RFExploreEnt, by learning for each state-action pair $(s,ah)$, a policy that reliably reaches this action pair $(s,a)$ at step $h$. This can be done by running any regret minimization algorithm for the sparse reward function putting reward one at state $s$ at step $h$ and zero otherwise. The policy $\pi^{\mathrm{mix}}$ is defined as the mixture of the aforementioned policies.

\paragraph{Planning phase}

The second phase is starting during the running time of the algorithm. Since \RFExploreEnt algorithm is essentially reward-free in a sense of working with an arbitrary reward functions.

For each episode $t$ during the game we define the empirical regularized Bellman equations 
\begin{align}
\begin{split}\label{eq:empirical_regularized_planning_VE}
\hQ_h^t(s,a) &=  \log\frac{1}{\bd_h^{t+1}(s)} + \hp_h^{\,t} \hV^t_{h+1}(s,a) \\
\hV_h^t(s) &= \max_{\pi \in \simplex_{\cA}}\{ \pi\hQ_h^t(s,a) + \cH(\pi) \},
\end{split}
\end{align}
where  $\hV_{H+1}^t = 0$. The sampler player then follows $d^{\pi^{t+1}}$ where $\pi^{t+1}$ is greedy with respect to the regularized Q-values, that is, $\pi_h^{t+1}(s) \in\argmax_{\pi\in\Delta_A} \{ \pi\hQ_h^t(s) + \cH(\pi) \}$. This choice of sampler player will be clear in the analysis below. 

\paragraph{Sampling rule} At each episode $t,$ the policy $\pi^t$ of the sampler-player is used as a sampling rule to generate a new trajectory.

\paragraph{Decision rule} After $T$ episodes we output a non-Markovian policy $\hpi$ defined as the mixture of the policies $\{\pi^t\}_{t\in[T]}$, that is, to obtain a trajectory from $\hpi$ we first sample uniformly at random $t\in[T]$ and then follow the policy $\pi^t$. Note that the visitation distribution of $\hpi$ is exactly the average $d^{\hpi} = (1/T)\sum_{t\in[T]} d^{\pi^t}$. 



Remark that the stopping rule of \regalgMVEE is deterministic and equals to $\tau = T$. The complete procedure is detailed in Algorithm~\ref{alg:regMVEE}.



\begin{algorithm}[h!]
\centering
\caption{\regalgMVEE}
\label{alg:regMVEE}
\begin{algorithmic}[1]
  \STATE {\bfseries Input:} Number of episodes $T$, number of exploration episodes $N_0$, number of transition samples $N$, prior counts $n_0$.
  \STATE \textcolor{blue}{\# Preliminary exploration}
  \FOR{$(s',h') \in \cS \times [H]$}
        \STATE Form rewards $r_h(s,a) = \ind\{ s=s', h=h'\}$.
        \STATE Run \EULER \citep{zanette2019tighter} with rewards $r_h$ over $N_0$ iterates and collect all policies $\Pi_{s',h'}$.
        \STATE Modify $\pi \in \Pi_{s',h'}:\ \pi_{h'}(a|s') = 1/A$ for all $a\in \cA$.
    \ENDFOR
    \STATE Construct a uniform mixture policy $\pi^{\mathrm{mix}}$ over all $\{ \Pi_{s,h} : (s,h) \in \cS \times [H] \}$.
    \STATE Sample $N$ independent trajectories $\{z_n\}_{n\in[N]}$ using  $\pi^{\mathrm{mix}}$ in the original MDP.
    \STATE Construct from $\{z_n\}_{n\in[N]}$ the estimates $\hp_h$ as in \eqref{eq:hp_construction}.
      \FOR{$t \in[T]$}
      \STATE \textcolor{blue}{\# Forecaster-player}
      \STATE Update pseudo counts $\bn_h^{t-1}(s,a)$ and predict $\bd_h^t(s,a)$. 
      \STATE \textcolor{blue}{\# Sampler-player}
      \STATE Compute $\pi^t$ by regularized planning \eqref{eq:empirical_regularized_planning_VE} with rewards $\log\big(1/ \bd_h^t(s)\big)$ and entropy regularization.
    \STATE \textcolor{blue}{\# Sampling}
      \FOR{$h \in [H]$}
        \STATE Play $a_h^t\sim \pi_h^t(s_h^t)$
        \STATE Observe $s_{h+1}^t\sim p_h(s_h^t,a_h^t)$
      \ENDFOR
    \STATE{ Update counts and transition estimates.}
   \ENDFOR
   \STATE Output $\hpi$ the uniform mixture of $\{\pi^t\}_{t\in[T]}$.
\end{algorithmic}
\end{algorithm}



\subsection{Analysis}


We first define the regrets of each players obtained by playing $T$ times the games. For the forecaster-player, for any $\bd\in\cK,$ we define 
\[
\regret_{\fore}^T(\bd) \triangleq \sum_{t=1}^T \sum_{h,s,a} \td_h^t(s,a) \left(\log\frac{1}{\bd_h^t(s,a)} -\log\frac{1}{\bd_h(s,a)}\right)
\]
where $\td_h^t(s,a) \triangleq \ind\big\{(s_h^t,a_h^t)=(s,a)\big\}$ is a sample from $d_h^{\pi^t}(s,a)$.
Similarly for the sampler-player, for any $d\in\cK_p,$ we define a \textit{regularized regret}
\begin{small}
\[
\regret_{\samp}^T(d)\triangleq \sum_{t=1}^T \left( \sum_{h,s,a} \big[ d_h(s,a) - d_h^{\pi^t}(s,a) \big] \log\frac{1}{\bd_h^t(s,a)} - \sum_{h,s} \left[ d_h(s) \KL(\pi(s), \bar{\pi}^t_h(s)) -  d^{\pi^t}_h(s) \KL(\pi^t_h(s), \bar{\pi}^t_h(s)) \right] \right)\,,
\]
\end{small}
\!where corresponding policies are defined as $\pi_h(a|s) = d_h(s,a) / d_h(s)$ and $\bar{\pi}^t_h(a|s) = \bd^t_h(s,a) / \bd^t_h(s)$ for $d_h(s) = \sum_{a} d_h(s,a)$ and $\bd^t_h(s) = \sum_{a} \bd^t_h(s,a)$.

Recall that the visitation distribution of the policy $\pi$ returned by \regalgMVEE is the average of the visitation distributions of the sampler-player 
$d_h^{\hpi}(s,a) = \hd^{\,T}_h(s,a) \triangleq (1/T) \sum_{t=1}^T d_h^{\pi^t}(s,a)$.  We also denote by $\rd^T_h(s,a)\triangleq (1/T) \sum_{t=1}^T \td^t(s,a)$ the average of the 'sample' visitation distributions.

We now relate the difference between the optimal visitation entropy and the visitation entropy of the outputted policy $\hpi$ to the regrets of the two players. 
Indeed, using $\cH(p) = \sum_{i\in[n]} p_i \log(1/q_i) -\KL(p,q)$ for all $(p,q)\in(\Delta_n)^2$ and
\begin{align*}
    \KL(d^{\pistar}_h, \bd_h^t) &= \sum_{s,a} d^{\pistar}_h(s,a) \log\left( \frac{d^{\pistar}_h(s,a)}{\bd_h^t(s,a)} \right) \\
    &= \sum_{s} d^{\pistar}_h(s) \log\left( \frac{d^{\pistar}_h(s)}{\bd_h^t(s)} \right) + \sum_{s} d^{\pistar}_h(s) \sum_{a} \pistar_h(a|s) \log\left( \frac{\pistar_h(a|s)}{\bar{\pi}_h^t(a|s)} \right) \\
    &\geq \sum_{s} d^{\pistar}_h(s) \KL(\pistar_h(s), \bar{\pi}^t_h(s))\,.
\end{align*}    
This inequality could be treated as a strong convexity of visitation entropy with respect to trajectory entropy since $\KL(d^{\pistar}_h, \bd^t_h)$ is a \textit{Bregman divergence} with respect to $\VE$, and the final average of $\KL(\pistar_h(s), \bpi^t_h(s))$ is a Bregman divergence with respect to $\TE$ (up to linearities).

Applying this inequality, we have
\begin{align*}
T\big(\VE(d^{\pistar}) -\VE(d^{\hpi})\big) &\leq \sum_{t=1}^T \left( \sum_{h,a,s} d_h^{\pistar}(s,a) \log\frac{1}{\bd_h^t(s,a)} - \sum_{h,s}  d_h^{\pistar}(s) \KL(\pistar_h(s), \bar{\pi}^t_h(s))  \right) \\
& \quad- \sum_{t=1}^T \td^t_h(s,a) \log\frac{1}{\rd_h^{\,T}(s,a)} + T\big(\VE(\rd^T) - \VE(\hd^T)\big)\\
& \leq \regret_{\samp}^T(d^{\pistar})+ \underbrace{\sum_{t=1}^T \sum_{h,s,a} \big(d_h^{\pi^t}(s,a) - \td_h^t(s,a) \big) \log\frac{1}{\bd_h^t(s,a)}}_{\mathrm{Bias}_1} \\
& \quad + \regret_{\fore}^T(\rd^T) + \underbrace{T\big(\VE(\rd^T) - \VE(\hd^T)\big)}_{\mathrm{Bias}_2}\,.
\end{align*}
It remains to upper bound each terms separately in order to obtain a bound on the gap. Notably, only the sampler player result changes in comparison to \algMVEE.


\subsection{Regret of the Sampler-Player}

We start from introducing new notation. Let $\cM_t = (\cS, \cA, \{ p_h \}_{h\in[H]}, \{r^t_h\}_{h\in[H]}, s_1)$ be a sequence of entropy-regularized MDPs where reward defined as $r^t_h(s,a) = \log(1/ \bd^t_h(s))$. Define $Q^{\pi, t}_h(s,a)$ and $V^{\pi, t}_h(s,a)$ as a action-value and value functions of a policy $\pi$ on a MDP $\cM_t$. Notice that the value-function of initial state in this case could be written as follows (see Appendix~\ref{app:reg_bellman_eq})
\begin{align*}
    V^{\pi,t}_1(s_1) &= \sum_{h,s,a} d^{\pi}_h(s,a) \log\left( \frac{1}{\bd^t_h(s,a)} \right) - \sum_{h,s} d^\pi_h(s) \KL(\pi_h(s), \bar{\pi}^t_h(s)) \\
    &= \sum_{h,s,a} d^\pi_h(s,a) \log\left( \frac{1}{\bd^t_h(s)} \right) + \sum_{h,s} d^\pi_h(s) \cH(\pi_h(s)),
\end{align*}
also see Appendix~\ref{app:regularized_mdp} for more exposition. Therefore, the regret for the sampler-player could be rewritten in the terms of the regret for this sequence of entropy-regularized MDPs
\[
    \regret_{\samp}^T(d^\pi) =  \sum_{t=1}^T V^{\pi,t}_1(s_1) - V^{\pi^t,t}_1(s_1).
\]

We notice that our approach does not gives a regret minimizer algorithm in a classical sense, however analysis shows us that we can control the sum of policy error with respect to \textit{any} reward function.

\begin{lemma}\label{lem:reg_regret_sampler}
    Let $N_0 = \Omega\left( \frac{H^7 S^3 A  \cdot \log^2(T+SA) \cdot L}{\varepsilon}\right)$ and $N = \Omega\left( \frac{ H^6 S^3 A \log^2(T+SA) L^3 }{\varepsilon}\right).$ Then with probability at least $1-\delta/2,$ the regret of the sampler player is bounded as
    \[
        \regret_{\samp}^T(d^{\pistar}) \leq \varepsilon/2 \cdot T
    \]
    after 
    \[
         \tcO\left( \frac{H^8 S^4 A}{\varepsilon} \right)
    \]
    episodes of pure exploration.
\end{lemma}
\begin{proof}
    From  Corollary~\ref{cor:rf_explore_ent_rf_sample_complexity} under the choice of parameters $\lambda = 1, \kappa = 0$ and reward function $r^t_h(s,a) = \log(1/\bd^t_h(s))$ for each iteration, that is bounded by $\log(T+SA)$, we have that for any reward function the sub-optimality gap is bounded by $\varepsilon/2$. The total number of episodes of pure exploration is equal to $N_0SH + N$.
\end{proof}





\subsection{Proof of Theorem~\ref{th:fast_MVEE_sample_complexity}}

We state the version of this theorem with all prescribed dependencies factors.
\begin{theorem}\label{th:fast_MVEE_sample_complexity_full}
Fix some $\epsilon > 0$ and $\delta\in(0,1)$. Then for $n_0=1,$ 
\[
    N_0 = \Omega\left( \frac{H^7 S^3 A  \cdot L^3}{\varepsilon}\right),
    \quad
    N = \Omega\left( \frac{ H^6 S^3 A L^5 }{\varepsilon}\right), 
    \quad
    T = \Omega\left( \frac{H^2 S A L^3}{\varepsilon^2} + \frac{H^2 S^2 A^2 L^2}{\varepsilon}\right)
    \]
with $L = \log(SAH/\delta\varepsilon),$ the algorithm \regalgMVEE is $(\epsilon,\delta)$-PAC. Its total sample complexity is equal to $SH \cdot N_0 + N + T,$ that is,
\[
    \tau = \tcO\left( \frac{H^2 SA}{\varepsilon^2} + \frac{H^8 S^4 A}{\varepsilon} \right).
\]
\end{theorem}
\begin{proof}
    We start from writing down the decomposition defined in the beginning of the appendix
    \[
        T(\VE(d^{\pistarVE}) - \VE(d^{\hpi})) \leq \regret_{\samp}^T(d^{\pistarVE}) + \regret_{\fore}^T(\rd^T) + \mathrm{Bias}_1 + \mathrm{Bias}_2.
    \]
    By Lemma~\ref{lem:reg_regret_sampler} with probability at least $1-\delta/2$ it holds
    \[
        \regret_{\samp}^T(d^{\pistarVE}) \leq \varepsilon T/2.
    \]
    By Lemma~\ref{lem:regret_forecaster} 
    \[
        \regret_{\fore}^T(\rd^T) \leq HSA\log\big(\rme(T+1)\big).
    \]
    By Lemma~\ref{lem:bias_terms} with probability at least $1-\delta/2$
    \[
         \mathrm{Bias}_1 + \mathrm{Bias}_2 \leq 3\log(SAT)\left(\sqrt{TH\log(4/\delta)} + H\sqrt{SAT\log(3T)} \right).
    \]
    By union bound all these inequalities hold simultaneously with probability at least $1-\delta$. Combining all these bounds we get
    \begin{align*}
        T(\VE(d^{\pistarVE}) - \VE(d^{\hpi})) &\leq  3\log(SAT)\left(\sqrt{TH\log(4/\delta)} + H\sqrt{SAT\log(3T)} \right) \\
        &+ HSA\log(\rme (T+1)) + \varepsilon T / 2.
    \end{align*}
    Therefore, it is enough to choose $T$ such that $\VE(d^{\pistarVE}) - \VE(d^{\hpi})$ is guaranteed to be less than $\varepsilon$. In this case \regalgMVEE become automatically $(\varepsilon,\delta)$-PAC. It is equivalent to find a maximal $T$ such that
    \begin{align*}
        \varepsilon T/2 &\leq 3\log(SAT)\left(\sqrt{TH\log(4/\delta)} + H\sqrt{SAT\log(3T)} \right) +  HSA \log(\rme (T+1))).
    \end{align*}
    and add $1$ to it. We start from obtaining a loose bound to eliminate logarithmic factors in $T$.
    
    First, we assume that $T \geq 1$, thus $T+1 \leq 2T$. Additionally, let us use inequality $\log(x) \leq x^{\beta}/\beta$ for any $x > 0$ and $\beta > 0$. We obtain
    \[
        \varepsilon T \leq 48(SAT)^{1/8}\left(\sqrt{T^{3/2} H\log(4/\delta)} + H\sqrt{4 SAT^{3/2}} \right) +  16/7 \cdot HSA (2\rme T)^{7/8}
    \]
    that could be relaxed as follows
    \[
         \varepsilon T^{1/8} \leq 48 (SA)^{1/8} (H \log(4/\delta)^{1/2} + H (SA)^{5/8} + 11 HSA
    \]
    thus we can define $\gamma = 8 \log\left( (48(SA)^{1/8} (H \log(4/\delta)^{1/2} + H (SA)^{5/8} + 11 HSA)/\varepsilon \right) = \cO(L)$ for which $\log(T) \leq \gamma$. Therefore
    \[ 
        \varepsilon T/2 \leq 3 (\log(SA) + \gamma) \sqrt{T} \left( \sqrt{H\log(4/\delta)} + \sqrt{SAH^2 (\log(3) + L)} \right) + HSA (1 + 2\gamma).
    \]
    Solving this quadratic inequality, we obtain the minimal required $T$ to guarantee $\VE(d^{\pistarVE}) - \VE(d^{\hpi}) \leq \varepsilon$. In particular,
    \[
        T = \Omega\left( \frac{H^2 S A L^3}{\varepsilon^2} + \frac{H^2 S^2 A^2 L^2}{\varepsilon}\right).
    \]
\end{proof}
\newpage
%!TEX root = ../BayesUCBVI.tex
\section{Deviation Inequalities}
\label{app:deviation_ineq}
% We define the following favorable events: $\cE^\star$ where the empirical means of the optimal value functions are close to the true ones, $\cE^\cnt$ the event where the pseudo-counts are close to their expectation, and $\cE^\bias$ the event where we control the deviation of the martingale of the bias of the upper bounds on the optimal value function,
% \begin{align*}
%   \cE^\star &\triangleq \Bigg\{\forall t \in \N, \forall h \in [H], \forall (s,a)\in\cS\times\cA:
%     \Kinf(\hp_h^t(s,a),p_h \Vstar_{h+1}(s,a), \Vstar_{h+1}) \leq  \frac{\beta(\delta,n_h^t(s,a))}{n_h^t(s,a)}\Bigg\}\,,\\
%  \cE^{\cnt} &\triangleq  \left\{ \forall t \in \N, \forall h\in [H],\forall (s ,a)\in\cS\times\cA:\ n_h^t(s,a) \geq \frac{1}{2}\bar n_h^t(s,a)-\beta^{\cnt}(\delta)  \right\}\,,\\
%  \cE^{\bias} &\triangleq \Bigg\{\forall t \in \N, \forall h \in [H], \forall (s,a)\in\cS\times\cA:\\
% &\qquad\sum_{k=1}^t \ind_{\{(s_h^t,a_h^t)=(s,a)\}}(\tp_h^k -p_h)( \uV_{h+1}^{k-1}-\Vstar_{h+1})(s,a) \leq\\
% &\qquad\qquad\sqrt{2 \beta(\delta,n_h^t(s,a))\sum_{k=1}^t \ind_{\{(s_h^t,a_h^t)=(s,a)\}} \Var_{p_h}( \uV_{h+1}^{k-1}-\Vstar_{h+1}) (s,a)} + 3 H \beta(\delta,n_h^t(s,a))\Bigg\}\\
% \cE^{\conc} &\triangleq \Bigg\{\forall t \in \N, \forall h \in [H], \forall (s,a)\in\cS\times\cA: \\
% &\qquad|(\hp_h^t -p_h) \Vstar_{h+1}(s,a)| \leq \sqrt{2 \Var_{p_h}(\Vstar_{h+1})(s,a)\beta(\delta,n_h^t(s,a))} + 3 H \frac{\beta(\delta,n_h^t(s,a))}{n_h^t(s,a)}\\
% &\qquad|(\hp_h^t -p_h) (\Vstar_{h+1})^2(s,a)| \leq 5\sqrt{ H^2\Var_{p_h}(\Vstar_{h+1})(s,a)\beta(\delta,n_h^t(s,a))} + 9 H^2 \frac{\beta(\delta,n_h^t(s,a))}{n_h^t(s,a)}
% \Bigg\}
% \end{align*}
% We also introduce the intersection of these events, $\cG \triangleq \cE \cap \cE^{\cnt}\cap \cE^\star,$ and the intersection of only the first two events, $\cF \triangleq \cE \cap \cE^{\cnt}$ . Note that the event $\cF$ is independent of the reward function $r$. We  prove that for the right choice of the functions $\beta$ the above events hold with high probability.
% \begin{lemma}
% \label{lem:proba_master_event}
% For the following choices of functions $\beta,$
% \begin{align*}
%   \beta(n,\delta) &\triangleq   \log(3SAH/\delta) + S\log \left(8e(n+1)\right),\\
%   \beta^\cnt(\delta) &\triangleq \log\left(3SAH/\delta\right), \quad \text{and}\\
%   \betastar(n,\delta) &\triangleq \log(3SAH/\delta) + \log\left(8e(n+1)\right)\,,\\
% \end{align*}
% it holds that
% \[
% \P(\cE)\geq 1-\delta, \qquad \P(\cE^{\cnt})\geq 1-\delta,  \qquad \text{and} \qquad \P(\cE^\star)\geq 1-\delta\,.
% \]
% In particular, $\P(\cG) \geq 1-\delta$ and $\P(\cF) \geq 1-\delta$.
% \end{lemma}
% \begin{proof}
% First, by Theorem~\ref{th:max_ineq_categorical}, we have that
% \[
% \P(\cE)\geq 1-\frac{\delta}{3}\cdot
% \]
% Second, by Theorem~\ref{th:bernoulli-deviation}, we have that
% \[
% \P(\cE^{\cnt})\geq 1-\frac{\delta}{3}\cdot
% \]
% Finally, by Theorem~\ref{th:bernstein}, we have that
% \[
% \P(\cE^\star)\geq 1-\frac{\delta}{3}\cdot
% \]
% Applying a union to the above three inequalities, we conclude that
% \[
% \P(\cG)\geq 1-\delta \qquad \P(\cE)\geq 1-\delta\,.
% \]
% \noindent \emph{Remark}: Note that we can order %\begin{align*}
%   $1\leq \beta^{\cnt}(\delta)\leq \betastar(n,\delta) \leq \beta(n,\delta).$
% %\end{align*}
% \end{proof}


\subsection{Deviation inequality for categorical distributions}

Next, we state the deviation inequality for categorical distributions by \citet[Proposition 1]{jonsson2020planning}.
Let $(X_t)_{t\in\N^\star}$ be i.i.d.\,samples from a distribution supported on $\{1,\ldots,m\}$, of probabilities given by $p\in\simplex_{m-1}$, where $\simplex_{m-1}$ is the probability simplex of dimension $m-1$. We denote by $\hp_n$ the empirical vector of probabilities, i.e., for all $k\in\{1,\ldots,m\},$
 \[
 \hp_{n,k} \triangleq \frac{1}{n} \sum_{\ell=1}^n \ind\left\{X_\ell = k\right\}.
 \]
 Note that  an element $p \in \simplex_{m-1}$ can be seen as an element of $\R^{m-1}$ since $p_m = 1- \sum_{k=1}^{m-1} p_k$. This will be clear from the context. 
%  We denote by $H(p)$ the (Shannon) entropy of $p\in\Sigma_m$,
%  \[
%  H(p) = \sum_{k=1}^m p_k \log\left(\frac{1}{p_k}\right)\cdot
%  \]
 \begin{theorem} \label{th:max_ineq_categorical}
 For all $p\in\simplex_{m-1}$ and for all $\delta\in[0,1]$,
 \begin{align*}
     \P\left(\exists n\in \N^\star,\, n\KL(\hp_n, p)> \log(1/\delta) + (m-1)\log\left(e(1+n/(m-1))\right)\right)\leq \delta.
 \end{align*}
\end{theorem}



\subsection{Deviation inequality for Shannon entropy}

 We denote by $\cH(p)$ the (Shannon) entropy of $p\in \simplex_{m-1}$,
 \[
    \cH(p) \triangleq \sum_{k=1}^m p_k \log\left(\frac{1}{p_k}\right) .
 \]
We will follow the ideas of \citet{paninski2003estimation}.
 
\begin{theorem}\label{th:entropy_concentration}
    For all $p \in \simplex_{m-1}$ and for all $\delta \in[0,1]$
    \[
        \P\left[ \vert \cH(\hp_n) - \cH(p) \vert \geq  \sqrt{\frac{2 \log^2(n) \cdot \log(2/\delta)}{n}} + \left( \frac{(m-1) \log(\rme (1 + n/(m-1))) + 1}{n} \wedge \log(m) \right) \right] \leq \delta.
    \]
    Moreover,
    \[
        \P\left[ \exists n:  \vert \cH(\hp_n) - \cH(p) \vert \geq  \sqrt{\frac{2 \log^2(n) \cdot (\log(2/\delta) + \log(n(n+1)))}{n}} + \left( \frac{(m-1) \log(\rme (1 + n/(m-1))) + 1}{n} \wedge \log(m) \right)\right] \leq \delta.
    \]
\end{theorem}
\begin{proof}
    We start from application of McDiarmid's inequality to entropy by \citet{antos2001convergence}.
    For all $p \in \Delta_{m-1}$ with probability at least $1-\delta$ we have
    \[
        \P\left[ \vert \cH(\hat p_n) - \E[\cH(\hat p_n)]  \vert \geq \sqrt{\frac{2 \log^2(n) \cdot \log(2/\delta)}{n}}  \right] \leq \delta.
    \]
    To relate $\E[\cH(\hp_n)]$ and $\cH(p)$ we use the following observation
    \[
        \cH(\hp_n) - \cH(p) = - \KL(\hp_n, p) + \sum_{k: p_k > 0} (\hp_{n,k} - p_k) \log(1/p_k),
    \]
    therefore by taking expectation we have
    \[
        \E[\cH(\hp_n)] - \cH(p) = - \E[\KL(\hp_n, p)].
    \]
    In the following our analysis differs from \cite{paninski2003estimation} since we obtain a direct estimate on the KL-divergence using Theorem~\ref{th:max_ineq_categorical},
    \begin{align*}
        \E[n \KL(\hp_n,p)] &= \int_{0}^\infty \P[n \KL(\hp_n, p) > t]\rmd t \leq (m-1) \log(\rme(1 + n/(m-1))) + \int_0^\infty \rme^{-t} \rmd t.
    \end{align*}
    At the same time we have a trivial bound that concludes the first statement
    \[
        \E[\cH(\hp^n)] - \cH(p) \leq \log(m).
    \]

    To show the second statement of Theorem~\ref{th:entropy_concentration}, we apply the first part with $\delta'(n) = \delta/( n(n+1))$,
    \[
        \P\left[ \vert \cH(\hp_n) - \cH(p) \vert \geq  \sqrt{\frac{2 \log^2(n) \cdot \log(2/\delta'(n))}{n}} + \left( \frac{(m-1) \log(\rme (1 + n/(m-1))) + 1}{n} \wedge \log(m) \right)\right] \leq \frac{\delta}{n(n+1)},
    \]
    thus by union bound over $n \in \N$ we conclude the statement.
\end{proof}

% \subsection{Deviation inequality for categorical weighted sum}
% %  Let $(X_t)_{t\in\N^\star}$ be i.i.d.\,samples from a distribution supported on $\{1,\ldots,m\}$, of probabilities given by $p\in\Sigma_m$, where $\Sigma_m$ is the probability simplex of dimension $m-1$. We denote by $\hp_n$ the empirical vector of probabilities, i.e., for all $k\in\{1,\ldots,m\},$
% %  \[
% %  \hp_{n,k} = \frac{1}{n} \sum_{\ell=1}^n \ind\left\{X_\ell = k\right\}.
% %  \]
% %  Note that  an element $p \in \Sigma_m$ can be seen as an element of $\R^{m-1}$ since $p_m = 1- \sum_{k=1}^{m-1} p_k$. 
%  We fix a function $f: \{1,\ldots,m\} \mapsto [0,b]$ and recall the definition of the minimal Kullback-Leibler divergence for $p\in\simplex_{m-1}$  and $u\in\R$
%  \[
% \Kinf(p,u,f) = \inf\left\{  \KL(p,q): q\in\simplex_{m-1}, qf \geq u\right\}\,.
%  \]
% As the Kullback-Leibler divergence this quantity admits a variational formula.
% \begin{lemma}[Lemma 18 by \citet{garivier2018kl}]
% \label{lem:var_form_Kinf} For all $p \in \simplex_{m-1}$, $u\in [0,b)$,
% \[
% \Kinf(p,u,f) = \max_{\lambda \in[0,1]} \E_{X\sim p}\left[ \log\left( 1-\lambda \frac{f(X)-u}{b-u}\right)\right]\,,
%  \]
%  moreover if we denote by $\lambda^\star$ the value at which the above maximum is reached, then
%  \[
%    \E_{X\sim p} \left[\frac{1}{1-\lambda^\star\frac{f(X)-u}{b-u}}\right] \leq 1\,.
%  \]
% \end{lemma}
% \begin{remark} Contrary to \citet{garivier2018kl} we allow that $u=0$ but in this case Lemma~\ref{lem:var_form_Kinf} is trivially true, indeed
%   \[
%   \Kinf(p,0,f) =  0  = \max_{\lambda \in[0,1]} \E_{X\sim p}\left[ \log\left( 1-\lambda \frac{f(X)}{b}\right)\right]\,.
%    \]
% \end{remark}

% We are now ready to state the deviation inequality for the $\Kinf$ which is a self-normalized version of Proposition~13 by \citet{garivier2018kl}.
%  \begin{theorem} \label{th:max_ineq_kinf}
%  For all $p\in\simplex_{m-1}$ and for all $\delta\in[0,1]$,
%  \begin{align*}
%      \P\big(\exists n\in \N^\star,\, n\Kinf(\hp_n, pf, f)> \log(1/\delta) + 3\log(e\pi(1+2n))\big)\leq \delta.
%  \end{align*}
% \end{theorem}

%  \begin{proof}
% First if $pf=b$ then $f(k)=b$ for all $k$ such that $p_k>0$. In this case $\Kinf(\hp_n, pf, f)=0$ for all $n$ and the result is trivially true. We thus assume now that $pf<b$.

% The proof is a combination of the one of Proposition~13 by \citet{garivier2018kl} and the method of mixtures. We first define the martingale
% \[
% M_n^\lambda = \exp\left(\sum_{\ell=1}^n \log\left(1-\lambda \frac{f(X_\ell)-pf}{b-pf}\right)\right)\,,
% \]
% with the convention $M_0^\lambda= 1$. Indeed if we denote by $\cF_n = \sigma(X_1,\ldots,X_n)$ the information available at time $n$, we have
% \begin{align*}
%   \E\left[M_n^\lambda|\cF_{n-1}\right] = \E\left[1-\lambda\frac{f(X_n)-pf}{b-pf}\right] M_{n-1}^\lambda = M_{n-1}^\lambda\,.
% \end{align*}
% We fix a real number $\gamma_j = 1/(2j)$ for $j\in\N^*$and let $S_j$ be the set
% \[
% S_j= \Bigg\{ \frac{1}{2}-\Bigg\lfloor\frac{1}{2\gamma_j}\Bigg\rfloor\gamma_j, \dots,\frac{1}{2}-\gamma_j,\,\frac{1}{2},\,\frac{1}{2}+\gamma_j,\dots,\frac{1}{2}+\Bigg\lfloor\frac{1}{2\gamma_j}\Bigg\rfloor\gamma_j \Bigg\}\,.
% \]
% The cardinality of this set $S_j$ is bounded by $1 + 2j$. We choose a prior on $\lambda$ the mixture of uniform distribution over this grid: $6/\pi^2\sum_{j=1}^{\infty} 1/j^2 \cU(S_j)$. Thus we consider the integrated martingale
%  \begin{align}
%      M_n &= \frac{6}{\pi^2}\sum_{j=1}^{\infty} \frac{1}{j^2} \sum_{\lambda \in S_j} \frac{1}{|S_j|}M_n^{\lambda} \nonumber\\
%      &\geq \frac{6}{\pi^2 n^2 |S_n|} \max_{\lambda \in S_n}M_n^{\lambda}\nonumber\\
%      &\geq \frac{6}{\pi^2 (1+2n)^3}\max_{\lambda \in S_n}M_n^{\lambda}\,.\label{eq:lb_mixture_kinf}
%  \end{align}
% Lemma~\ref{lem:regularity_ln_lambda} below
% indicates that for all $\lambda\in[0,1]$, there exists a $\lambda'\in S_n$ such that for all $x\in[0,b]$,
% \begin{equation}
% \label{eq:2gamma}
% \log\!\Bigg(1-\lambda \, \frac{x-p f}{b-p f } \Bigg)\leq 2\gamma_n
% + \log\!\Bigg(1- \lambda' \frac{x-p f}{b-p f} \Bigg)\,.
% \end{equation}
% Now, a combination of
% the variational formula of Lemma~\ref{lem:var_form_Kinf}
% and of the inequality~\eqref{eq:2gamma} yields a finite maximum as an upper bound on $\Kinf(\hp_n,pf,f)$
% \begin{align*}
% \Kinf(\hp_n,pf,f)
% & = \max_{0\leq \lambda\leq 1} \frac{1}{n}\sum_{\ell=1}^{n} \log\!\Bigg(1-\lambda \frac{X_\ell-pf}{b-pf} \Bigg) \\
% & \leq 2 \gamma_n + \max_{\lambda' \in S_n} \frac{1}{n}\sum_{k=1}^{n}\log\!\Bigg(1-\lambda' \frac{X_\ell-pf}{b-pf} \Bigg)\,.
% \end{align*}
% Thanks to the definition of the martingale $M_n^\lambda$ we obtain
% \[
% \max_{\lambda \in S_n}M_n^{\lambda} \geq e^{-2n\gamma_n} e^{n\Kinf(\hp_n,pf,f)}= e^{-1} e^{n\Kinf(\hp_n,pf,f)}\,.
% \]
% Combining this inequality with \eqref{eq:lb_mixture_kinf} yields
% \[
%  M_n  \geq  \frac{6}{e\pi^2 (1+2n)^3}e^{n\Kinf(\hp_n,pf,f)}\,.
% \]
% Since for any supermartingale we have that
%  \begin{equation}\P\left(\exists n \in \N : M_n > 1/\delta\right) \leq \delta \cdot \E[M_0],\label{eq:supermartingale}\end{equation}
%  which is a well-known property of the method of mixtures \citep{de2004self}, we conclude that
%  \[
%  \P\left(\exists n\in \N^\star,\, n\Kinf(\hp_n,pf,f)> \log(1/\delta)+ 3\log(e\pi(1+2n))\right) \leq \delta\,.
%  \]
% \end{proof}

% \begin{lemma}[Lemma 19 by \citealp{garivier2018kl} and comment below]
% For all $\lambda, \lambda'\in[0,1]$ such that either $\lambda \leq \lambda'\leq 1/2$ or $1/2\leq \lambda'\leq \lambda$, for all real numbers $c\leq 1$,
% \begin{equation*}
% \log(1-\lambda c)-\log(1-\lambda' c)\leq 2|\lambda-\lambda'|\,.
% \end{equation*}
% \label{lem:regularity_ln_lambda}
% \end{lemma}

\subsection{Deviation inequality for sequence of Bernoulli random variables}

Below, we state the deviation inequality for Bernoulli distributions by \citet[Lemma F.4]{dann2017unifying}.
Let $\mathcal F_t$ for $t\in\N$ be a filtration and $(X_t)_{t\in\N^\star}$ be a sequence of Bernoulli random variables with $\P(X_t = 1 | \mathcal F_{t-1}) = P_t$ with $P_t$ being $\mathcal F_{t-1}$-measurable and $X_t$ being $\mathcal F_{t}$-measurable.

\begin{theorem}\label{th:bernoulli-deviation}
	For all $\delta>0$,
	\begin{align*}
	\P \left(\exists n : \,\, \sum_{t=1}^n X_t < \sum_{t=1}^n P_t / 2 -\log\frac{1}{\delta}  \right) \leq \delta.
	\end{align*}
\end{theorem}

% \begin{proof}
% 	$P_t - X_t$ is a martingale difference sequence with respect to the filtration $\mathcal F_t$.
% 	Since $X_t$ is nonnegative and has finite second moment, we have for any $\lambda > 0$ that
% 	$\E\left[e^{-\lambda (X_t - P_t)} | \mathcal F_{t-1} \right] \leq e^{\lambda^2 P_t / 2}$ (Exercise 2.9, \citealp{boucheron2013concentration}).
% 	Hence, we have
% 	\begin{align}
% 	\E\left[ e^{\lambda(P_t - X_t) - \lambda^2 P_t / 2} | \mathcal F_{t-1} \right] \leq 1
% 	\end{align}
% 	and by setting $\lambda \triangleq 1$, we see that
% 	\begin{align}
% 	M_n = e^{\sum_{t=1}^n (-X_t + P_t / 2)}
% 	\end{align}
% 	is a supermartingale.
% 	Therefore by Markov's inequality,
% 	\begin{align}
% 	\P \left( \sum_{t=1}^n (-X_t + P_t / 2) \geq \log\frac{1}{\delta} \right)
% 	=
% 	\P \left( M_n \geq \frac{1}{\delta} \right)
% 	\leq {\delta} \E[M_n] \leq \delta
% 	\end{align}
% 	which gives us
% 	\begin{align}
% 	\P \left( \sum_{t=1}^n X_t \leq \sum_{t=1}^n P_t / 2 -\log\frac{1}{\delta}\right)
% 	\leq \delta
% 	\end{align}
% 	for a fixed $n$.
% 	We define now the stopping time $\tau \triangleq \min\{ t \in \mathbb N \, : \, M_t > \frac{1}{\delta} \}$ and the sequence $\tau_n = \min\{ t \in \mathbb N \, : \, M_t > \frac{1}{\delta} \vee t \geq n \}$.
% 	Applying the convergence theorem for nonnegative supermartingales
% 	(Theorem~5.2.9 by \citealp{durrett2010probability}), we get that $\lim_{t \rightarrow
% 		\infty} M_t$ is well-defined almost surely. Therefore, $M_\tau$ is
% 	well-defined even when $\tau = \infty$.
% 	By the optional stopping theorem for nonnegative supermartingales (Theorem
% 	5.7.6 by \citealp{durrett2010probability}), we have $\E[M_{\tau_n}] \leq \E[M_0] \leq
% 	1$ for all $n$ and applying Fatou's lemma, we obtain
% 	$\E[M_\tau] = \E[\lim_{n \rightarrow \infty} M_{\tau_n}] \leq \lim \inf_{n \rightarrow \infty} \E[M_{\tau_n}] \leq 1$.
% 	Using Markov's inequality, we can finally bound
% 	\begin{align}
% 	\P\left( \exists n:\,\, \sum_{t=1}^n X_t < \frac 1 2 \sum_{t=1}^n P_t - \log\frac{1}{\delta} \right)
% 	\leq
% 	\P ( \tau < \infty) \leq \P ( M_{\tau} > \frac{1}{\delta})
% 	\leq \delta \E[M_{\tau}] \leq \delta.
% 	\end{align}
% \end{proof}


\subsection{Deviation inequality for bounded distributions}
Below, we state the self-normalized Bernstein-type inequality by \citet{domingues2020regret}. Let $(Y_t)_{t\in\N^\star}$, $(w_t)_{t\in\N^\star}$ be two sequences of random variables adapted to a filtration $(\cF_t)_{t\in\N}$. We assume that the weights are in the unit interval $w_t\in[0,1]$ and predictable, i.e. $\cF_{t-1}$ measurable. We also assume that the random variables $Y_t$  are bounded $|Y_t|\leq b$ and centered $\EEc{Y_t}{\cF_{t-1}} = 0$.
Consider the following quantities
\begin{align*}
		S_t \triangleq \sum_{s=1}^t w_s Y_s, \quad V_t \triangleq \sum_{s=1}^t w_s^2\cdot\EEc{Y_s^2}{\cF_{s-1}}, \quad \mbox{and} \quad W_t \triangleq \sum_{s=1}^t w_s
\end{align*}
and let $h(x) \triangleq (x+1) \log(x+1)-x$ be the Cramér transform of a Poisson distribution of parameter~1.

\begin{theorem}[Bernstein-type concentration inequality]
  \label{th:bernstein}
	For all $\delta >0$,
	\begin{align*}
		\PP{\exists t\geq 1,   (V_t/b^2+1)h\left(\!\frac{b |S_t|}{V_t+b^2}\right) \geq \log(1/\delta) + \log\left(4e(2t+1)\!\right)}\leq \delta.
	\end{align*}
  The previous inequality can be weakened to obtain a more explicit bound: if $b\geq 1$ with probability at least $1-\delta$, for all $t\geq 1$,
 \[
 |S_t|\leq \sqrt{2V_t \log\left(4e(2t+1)/\delta\right)}+ 3b\log\left(4e(2t+1)/\delta\right)\,.
 \]
\end{theorem}

% \subsection{Deviation inequality for Dirichlet distribution}
% Below we provide the Bernstein-type inequality for weighted sum of Dirichlet distribution, using a result on upper bound on tails of Dirichlet boundary crossing (see Lemma~\ref{lem:upper_bound_dbc}).

% \begin{lemma}\label{lem:kinf_attains}
%      For any $p \in \simplex_m$, $f \colon \{0,\ldots,m\} \to [0,b]$ such that $f(0) = b$, $p_0 > 0$, and $\mu \in (p f, b)$ there exists a measure  $q \in \simplex_m$ such that $p \ll q $, $q f = \mu$ and \(\Kinf(p, \mu, f) = \KL(p, q)\).
% \end{lemma}
% \begin{proof}
%     By the variational form of $\Kinf$ (Lemma~\ref{lem:var_form_Kinf})
%     \[ 
%         \Kinf(p, \mu, f) = \max_{\lambda \in [0,1]} \E_{X \sim p}\left[ \log\left(1 - \lambda \frac{f(X) - \mu}{b - \mu} \right) \right] = \E_{X \sim p}\left[ \log\left(1 - \lambda^\star \frac{f(X) - \mu}{b - \mu} \right) \right].
%     \]
%     Note that $\PP{f(X) = b} > 0$ implies $\lambda^\star < 1$. Jensen's inequality and $\mu > pf$ imply $\lambda^\star > 0$. It is easy to check that $\lambda^\star$ satisfies
%     \[
%         \E\left[ \frac{1}{1 - \lambda^\star (f(X) - \mu)/(b-\mu)} \right] = \sum_{j=0}^m \frac{p(j)}{1 - \lambda^\star (f(j) - \mu)/(b-\mu)} = 1,
%     \]
%     and 
%     \begin{equation}\label{eq: kinf_attains1}
%         \E\left[ \frac{f(X) - \mu}{1 - \lambda^\star (f(X) - \mu)/(b-\mu)} \right] = \sum_{j=0}^m \frac{p(j) (f(j) - \mu)}{1 - \lambda^\star (f(j) - \mu)/(b-\mu)} = 0.
%     \end{equation}
%     Define $q(j) = \frac{p(j)}{1 - \lambda^\star (f(j) - \mu)/(b-\mu)}, j = 0, \ldots, m$, and let $q = (q_0, \ldots, q_m)$. Clearly, $q \in \simplex_m$, $qf = \mu$ by \eqref{eq: kinf_attains1} and $p \ll q$. Moreover,
%     \[
%         \Kinf(p, \mu, f) = \E_{X \sim p}\left[ \log\left(1 - \lambda^\star \frac{f(X) - \mu}{b - \mu} \right) \right] = \E_{p}\left[ \log \frac{\rmd p}{\rmd q}\right] = \KL(p,q).
%     \]
% \end{proof}


% \begin{lemma}\label{lem:bernstein_dirichlet}
%      For any $\alpha = (\alpha_0, \alpha_1, \ldots, \alpha_m) \in \N^{m+1}$ define  $\up \in \simplex_{m}$ such that $\up(\ell) = \alpha_\ell/\ualpha, \ell = 0, \ldots, m$, where $\ualpha = \sum_{j=0}^m \alpha_j$. Then for any $f \colon \{0,\ldots,m\} \to [0,b]$ such that $f(0) = b$ and $\delta \in (0,1)$
%      \[
%         \P_{w \sim \Dir(\alpha)}\left[wf \geq \up f +  2 \sqrt{ \frac{ \Var_{\up}(f) \log(1/\delta)}{\ualpha}} + \frac{2b\sqrt 2  \cdot \log(1/\delta)}{\ualpha} \right] \leq \delta.
%      \]
% \end{lemma}
% \begin{proof}
% Fix $\delta \in (0,1)$ and let $\mu \in (\up f,b)$ be such that
% \[
%     \Kinf(\up, \mu, f) = \ualpha^{-1} \log (1/\delta). 
% \]
% Note that such $\mu$ exists. Indeed, it follows from the continuity of $\Kinf$ w.r.t. the second argument, see \citet[Theorem 7]{honda2010asymptotically}. By Lemma~\ref{lem:kinf_attains} there exists $q$ such that $\up \ll q $, $q f = \mu$ and $\KL(\up, q) = \ualpha^{-1} \log (1/\delta)$. By Lemma \ref{lem:upper_bound_dbc}   
% \begin{equation}
% \label{eq: bernstein 1}
%  \P_{w \sim \Dir(\alpha)}[wf \geq qf] = \P_{w \sim \Dir(\alpha)}[wf \geq \mu] \leq \exp\left( -\ualpha \Kinf(\up, \mu, f) \right) = \delta.   
% \end{equation}
% By Lemma \ref{lem:Bernstein_via_kl}
% \[
% q f  - \up f \le \sqrt{2\Var_{q}(f)\KL(\up ,q)}.
% \]
% By Lemma \ref{lem:switch_variance_bis},  $\Var_q(f) \leq 2\Var_{\up}(f) +4b^2 \KL(\up ,q)$.
% The last two inequalities and \eqref{eq: bernstein 1} imply that
% \begin{equation*}
%   \P_{w \sim \Dir(\alpha)}\left[wf - \up f \geq \sqrt{4 \Var_{\up}(f) \KL(\up ,q)} + 2 b \sqrt 2 \cdot      \KL(\up,q) \right] \le \delta. 
% \end{equation*}
  
% \end{proof}


\newpage
\begin{tabular}{lcccccc} 
\rowcolor{mygray}
\thickhline
\hline
\multirow{1}{*}{\begin{tabular}[c]{@{}c@{}}\textbf{Ref }\end{tabular}} & \multirow{1}{*}

{\begin{tabular}[c]{@{}c@{}}\textbf{ Method}\end{tabular}} & \multirow{1}{*}

{\begin{tabular}[c]
{@{}c@{}}\textbf{Extractor}\end{tabular}} & \multirow{1}{*}{\begin{tabular}[c]{@{}c@{}}\textbf{L. Function}\end{tabular}} & \textbf{LR} & \textbf{Optimizer} & \textbf{Epochs}  \\ \hline
\cite{wang2014person}\textsubscript{ ECCV} & DVR & HOG3D & Hinge & — & — & — \\
\cite{karanam2015sparse}\textsubscript{ CVPR}& SRID & Schmid, Gabor filters & — & — & — & — 
\\
\cite{liu2015spatio}\textsubscript{ ICCV} & STFV3D & Fisher Vector & — & — & — & — \\
\cite{wu2016deep}\textsubscript{ ARXIV}  & Deep RCN & — & — & — & — & — 
\\
\cite{you2016top}\textsubscript{ CVPR} & TDL & \begin{tabular}[c]{@{}c@{}}HOG3D, Color \\ Histograms, LBP\end{tabular} & Hinge & — & — & — 
\\
\cite{chen2016person}\textsubscript{  IEEE-SRL} & OFEI & LBP & — & — & — & — 
\\
\cite{chen2016person}\textsubscript{  ECCV} & RFA-Net & LBP, HSV, Lab & Softmax & \begin{tabular}[c]{@{}c@{}}0.001 to \\ 0.0001\end{tabular} & — & 400 
\\
\cite{mclaughlin2016recurrent}\textsubscript{  CVPR} & CNN and RNN & Cross Entropy & 
 — &0.001 & SGD & 500 
\\
\cite{zhou2017see}\textsubscript{  CVPR} & JS-TRNN & TAM and SRM & Triplet & — & — & — 
\\
\cite{liu2017quality}\textsubscript{  CVPR} & QAN & — & Softmax and Triplet & — & — & — \\
\cite{xu2017jointly}\textsubscript{  ICCV} & ASTPN & — & CE and Hinge & 0.001 & SGD & 700 
\\
\cite{chung2017two}\textsubscript{  ICCV} & 2SCNN & CNN and RNN & Softmax & 0.001 & SGD & 1000 
\\
\cite{gao2021novel}\textsubscript{ ACM\_MM} & CMA & CNN+RNN & Softmax & 0.001 & SGD & 800 
\\
\thickhline
\end{tabular}
\newpage
We present in section~\ref{ssec:faces} an application of PnP-HVAE on face images, using a pretrained state-of-the-art hierarchical VAE. 
Next, we study the application of our framework to natural images. To that end, we introduce  in section~\ref{ssec:patchVDVAE}  a patch hierachical VAE architecture, that is able to model natural images of different resolutions. In section~\ref{ssec:app_nat}, we provide deblurring, super-resolution and inpainting experiments to demonstrate the relevance of the proposed method.

Additional results are presented in Appendix~\ref{app:add}. All experiments can be reproduced using the code available at \url{https://github.com/jprost76/PnP-HVAE}.



\subsection{Face Image restoration (FFHQ)}\label{ssec:faces}
We first demonstrate the effectiveness of PnP-HVAE on highly structured data, by performing face image restoration.
Latent variable generative models can accurately model structured images such as face images \cite{karras2019style,vahdat2020nvae,child2021very,kingma2018glow}, and then be used to produce high quality restoration of such data. 
In our experiments, we use the VDVAE model of~\cite{child2021very}, pre-trained on the FFHQ dataset~\cite{karras2019style}, as our hierarchical VAE prior.
VDVAE has $L=66$ latent variable groups in its hierarchy and generates images at resolution $256\times256$.

We compare PnP-HVAE with the intermediate layer optimization algorithm (ILO)~\cite{daras2021intermediate} that is based on a different class of generative models than HVAE. ILO is a GAN inversion method which optimizes the image latent code along with the intermediate layer representation of a StyleGAN to generate an image consistent with a degraded observation.
We use the official implementation of ILO, along with a StyleGAN2 model~\cite{karras2020analyzing, stylegan2pytorch}, that was trained for 550k iterations on images of resolution $256\times256$ from FFHQ.  
As VDVAE and StyleGAN models are not trained on the same train-test split of FFHQ, we chose to evaluate the methods on a subset of 100 images from the CelebA dataset~\cite{liu2018large}. 
For super-resolution, the degradation model corresponds to the application of a gaussian low-pass filter followed by a $\times 4$ sub-sampling, and the addition of a gaussian white noise with $\sigma=3$.
For the deblurring, we considered motion blur and  gaussian kernels, both with a noise level $\sigma=8$. %

We provide quantitative comparisons in table~\ref{table:comp_ILO}, along with a visual comparison of the results in figure~\ref{fig:face_restoration}.
PnP-HVAE has the best  PSNR and SSIM results for all the considered restoration tasks, while ILO provides better results  for the perceptual distance.
By jointly optimizing the image and its latent variable, PnP-HVAE provides  results that are both realistic and consistent with the degraded observation.
On the other hand,  ILO  only optimizes on an extended latent space. This method generates  sharp and realistic images with better LPIPS scores,   
but the results lack  of consistency with respect to the observation, which explains the overall lower PSNR performance. 






\subsection{PatchVDVAE: a HVAE for natural images}\label{ssec:patchVDVAE}
Available generative models in the literature operate on images of  fixed resolutions and
are either restrained to datasets of limited diversity, or even to registered face images~\cite{kingma2018glow,child2021very, vahdat2020nvae, karras2019style}, or requiring additional class information~\cite{brock2018large, dhariwal2021diffusion, song2020score, luhman2022optimizing}.
Fitting an unconditional model on natural images appears to be a more difficult task, as their resolution can change, and their content is highly diverse.
The complexity of the problem can be reduced by learning a prior model on patches of reduced dimension. 
For image restoration problems, the patch model can be reused on images of higher dimensions~\cite{zoran2011learning,prost2021learning,altekruger2022patchnr}. When the model is a full CNN, the prior on the set of the  patches can  be computed efficiently by applying the network on the full image~\cite{prost2021learning}.

We thus introduce  patchVDVAE, a fully convolutional hierarchical VAE.
Contrary to existing HVAE models whose resolution is constrained by the constant tensor at the input of the top-down block, patchVDVAE can generate images of different resolutions by controlling the dimension of the input latent. 
This amounts to defining a prior on patches whose dimension corresponds to the receptive field of the VAE. A similar model is used for image denoising in~\cite{prakash2021interpretable}.

 
For PatchVDVAE architecture, we use the same bottom-up and top-down blocks as VDVAE~\cite{child2021very}, and replace the constant trainable input in the first top-down block by a latent variable, to make the model fully convolutional (details on the  architecture are given in Appendix~\ref{app:details}). 
The training dataset is composed of $128\times 128$ patches extracted from a combination of DIV2K~\cite{agustsson2017ntire} and Flickr2K~\cite{Lim_2017_CVPR_workshops} datasets.
We perform data augmentation by extracting  patches at $3$ resolutions: HR-images and $\times 2$ and $\times 4$ downscaled images. 
The model is trained for $7.10^5$ iterations with a batch size of $64$. Following the recommendation of~\cite{hazami2022efficient}, we use Adamax optimizer with an exponential moving average and gradient smoothing of the variance.
We set the decoder model to be a gaussian with diagonal covariance, as in~\cite{luhman2022optimizing}.
PatchVDVAE is fully convolutional and can generate images of dimension that are multiples of $64$ as illustrated by
figure~\ref{fig:vdvae}.

\newlength{\patchwidth}
\setlength{\patchwidth}{0.135\columnwidth}
\begin{figure}[!ht]
    \centering
    \begin{subfigure}[t]{.34\columnwidth}\hspace{0.1cm}
        \setlength{\tabcolsep}{0.02pt}
\renewcommand{\arraystretch}{0}
        \begin{tabular}{*{2}{p{1.03\patchwidth}}}
            \includegraphics[width=\patchwidth]{figures_arxiv/patchVDVAE/samples/generated/64x64/setup-5-image-0018.png} &
            \includegraphics[width=\patchwidth]{figures_arxiv/patchVDVAE/samples/generated/64x64/setup-5-image-0016.png} \\
            \includegraphics[width=\patchwidth]{figures_arxiv/patchVDVAE/samples/generated/64x64/setup-5-image-0008.png} &
            \includegraphics[width=\patchwidth]{figures_arxiv/patchVDVAE/samples/generated/64x64/setup-5-image-0019.png}   
        \end{tabular}
    \end{subfigure}\hspace{-0.15cm}
    \begin{subfigure}[t]{.64\columnwidth}
\begin{tabular}{cc}\vspace{-0.1cm}
\includegraphics[width=2\patchwidth]{figures_arxiv/patchVDVAE/samples/generated/256x256/setup-2-image-0009.png}&
        \includegraphics[width=2\patchwidth]{figures_arxiv/patchVDVAE/samples/generated/256x256/setup-2-image-0002.png}\end{tabular}

    \end{subfigure}
    \caption{\label{fig:vdvae} Left: $64\times64$ patches samples from our patchVDVAE model trained on patches from natural images.
    Right: PatchVDVAE is fully convolutional and it can generate images of higher resolution (here: $128\times128$).\vspace{-0.2cm}}
\end{figure}

\subsection{Natural images restoration}\label{ssec:app_nat}
We  evaluate PnP-HVAE on natural image restoration.
For each task, we report the average value of the PSNR, the SSIM, and the LPIPS metrics on $20$ images from the test set of the BSD dataset~\cite{MartinFTM01}.\\


\noindent
{\bf Image deblurring.}
In the experiments, we consider $2$ gaussian kernels and $2$ motion blur kernels from~\cite{levin2009understanding}, with $3$ different noise levels 
$\sigma \in \{2.55, 7.65, 12.75\}$.
As a baseline we consider  EPLL~\cite{zoran2011learning}, which learns a prior on image patches with a gaussian mixture model.
We also compare PnP-HVAE  with PnP-MMO and GS-PnP, $2$ competing convergent Plug-and-Play methods based on CNN denoisers.
PnP-MMO~\cite{pesquet2021learning} restricts the denoiser to be contraction in order to guarantee the convergence of the PnP forward-backard algorithm. GS-PnP~\cite{hurault2022gradient} considers a gradient step denoiser and reaches state-of-the-art performances of non converging methods~\cite{zhang2021plug}.
We set the temperature $\tau$  in our method as $0.95$, $0.8$ and $0.6$ for noise levels $2.55$, $7.65$ and $12.75$ respectively, and we let it run for a maximum of $50$ iterations. 
For the three compared methods we use the official implementations and pre-trained models provided by the respective authors. 
Details on the choice of hyperparameters for the concurrent methods are provided in the Appendix~\ref{app:details}
Figure~\ref{fig:deblurring_bsd} illustrates that our method provides correct deblurring results. 

According to table~\ref{tab:deb}, the performance of PnP-HVAE is between those of EPLL and GS-PnP and it outperforms PnP-MMO for large noise levels.\\

\begin{table}
\begin{center}\footnotesize
    \begin{tabular}{>{\centering}m{.3cm}*{5}{c}}
    $\sigma$ &Method & PSNR$\uparrow$ & SSIM$\uparrow$ & LPIPS$\downarrow$  \\ 
    \hline
    \multirow{4}{*}{\vcell{$2.55$}}
    & PnP-HVAE & $27.75$ & $0.79$ & $0.31$\\
    & GS-PNP \cite{hurault2022gradient} & $\mathbf{29.59}$ & $\mathbf{0.84}$ & $\mathbf{0.22}$\\
    & EPLL \cite{zoran2011learning} & $26.49$ & $0.71$ & $0.36$\\ 
    & PnP-MMO \cite{pesquet2021learning} & $\underbar{29.50}$ & $\underbar{0.83}$ & $\underbar{0.20}$ \\ \hline
    \multirow{4}{*}{\vcell{$7.65$}}
    & PnP-HVAE & $\underbar{26.36}$ & $\underbar{0.72}$ & $\underbar{0.40}$\\
    & GS-PNP \cite{hurault2022gradient} & $\mathbf{27.33}$ & $\mathbf{0.77}$ & $\mathbf{0.31}$\\
    & EPLL \cite{zoran2011learning} & $24.04$ & $0.66$ & $0.45$ \\ 
    & PnP-MMO \cite{pesquet2021learning} & $25.34$ & $0.69$ & $0.34$\\
    \hline
    \multirow{4}{*}{\vcell{$12.75$}}
    & PnP-HVAE & $\underbar{25.12}$ & $\mathbf{0.73}$ & $\underbar{0.47}$\\
    & GS-PNP \cite{hurault2022gradient} & $\mathbf{26.32}$ & $\mathbf{0.73}$ & $\mathbf{0.37}$\\
    & EPLL \cite{zoran2011learning} & $23.28$ & $0.61$ & $0.51$ \\ 
    & PnP-MMO \cite{pesquet2021learning} & $22.42$ & $0.53$& $0.54$ \\
    \hline
    &\vspace*{-.3cm}\\
            \multicolumn{2}{c}{Blur and motion kernels}& \multicolumn{3}{c}{
        \includegraphics*[scale=1]{figures_arxiv/kernels/4.png}\;\includegraphics*[scale=1]{figures_arxiv/kernels/7.png}\;\includegraphics*[scale=1]{figures_arxiv/kernels/9.png}\;\includegraphics*[scale=1]{figures_arxiv/kernels/11.png}} 
    \end{tabular}
        \caption{\label{tab:deb}Comparison  of PnP-HVAE  and other restoration methods on deblurring. Results are averaged on $4$ kernels.\vspace{-0.2cm}}% on image deblurring.}
    \end{center}
\end{table}

\begin{figure}
    
    \begin{subfigure}[h]{\linewidth}
        \centering
        \includegraphics*[width=\columnwidth]{figures_arxiv/deb_s255_k7.pdf}\vspace{-0.1cm}
        \caption{Gaussian blur, $\sigma=2.55$}
    \end{subfigure}
    \begin{subfigure}[h]{\linewidth}
        \centering
        \includegraphics*[width=\columnwidth]{figures_arxiv/deb_s765_k11.pdf}\vspace{-0.1cm}
        \caption{Motion blur, $\sigma=7.65$}
    \end{subfigure}\vspace*{-0.1cm}
    \caption{\label{fig:deblurring_bsd} Natural image deblurring\vspace{-0.1cm}}
\end{figure}

\noindent {\bf Effect of the temperature.}
PnP-HVAE gives control on the temperature of the prior over the latent space.
In figure~\ref{fig:temp_effect}, we illustrate that reducing the temperature increases the strength of the regularization prior. In this example the tuning $\tau=0.7$ produces the best performance.\\
\begin{figure}[!ht]
   
    \includegraphics[width=\columnwidth]{figures_arxiv/demo_temp.pdf}\vspace{-0.15cm}
    \caption{ \label{fig:temp_effect} Effect of the temperature in PnP-VAE on a deblurring problem, with $\sigma=7.65$.\vspace{-0.15cm}}
\end{figure}


\noindent
{\bf Image inpainting.}
Next we consider the task of noisy image inpainting. 
We compose a test-set of 10 images from the validation set of BSD~\cite{MartinFTM01} and we create masks
  by occluding diverse objects of small size in the images. 
A gaussian white noise with $\sigma=3$ is added to the images.
As a comparaison, we still consider GS-PnP and EPLL.
For PnP-HVAE, the temperature is set to $\tau=0.6$, and the algorithm is run for a maximum of $200$ iterations, unless the residual $||\x_{k+1}-\x_k||$ is on a plateau.
We provide on Table~\ref{tab:inpainting_bsd} the distortion metrics with the ground truth, as well as a visual
\begin{table}



\begin{center}
    \begin{tabular}{cccc}
        & PSNR$\uparrow$ & SSIM$\uparrow$ &LPIPS$\downarrow$ \\\hline
        PnP-HVAE  & $\mathbf{29.54}$ & $\mathbf{0.93}$ & $\mathbf{0.06}$\\
        GS-PNP & $28.52$ & $\mathbf{0.93}$ & $0.09$\\
        EPLL & $\underline{29.16}$ & $\mathbf{0.93}$ & $\mathbf{0.06}$\\
    \end{tabular}
    \caption{\label{tab:inpainting_bsd}Quantitative evaluation for inpainting on BSD.}
    \end{center}
\end{table}
comparison on figure~\ref{fig:inpainting_bsd}. 
With its hierarchical structure,  PnP-HVAE outperforms the compared methods. \vspace{0.05cm}



\begin{figure}[!h]
    \includegraphics[width=\columnwidth]{figures_arxiv/demo_inp_bsd2.pdf}\vspace{-0.1cm}
    \caption{\label{fig:inpainting_bsd}Natural image inpainting\vspace{-0.3cm}}
\end{figure}













\end{document}
