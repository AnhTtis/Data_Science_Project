\renewcommand{\thesection}{\Alph{section}}

\section*{\fontsize{14pt}{\baselineskip}\selectfont Appendix}
\vspace{3mm}

\section{
More 
gain with other base superpixel sizes
}
\begin{figure}[!ht]
    \captionsetup[subfigure]{font=footnotesize,labelfont=footnotesize,aboveskip=0.05cm,belowskip=-0.15cm}
    \centering
    \hspace{-5mm}
    \begin{subfigure}{.49\linewidth}
        \centering
        \begin{tikzpicture}
            \begin{axis}[
                legend style={nodes={scale=0.6}, at={(1.65, 1.16)}},
                legend columns=-1,
                xlabel={The number of clicks},
                ylabel={mIoU (\%)},
                width=1.23\linewidth,
                height=1.23\linewidth,
                ymin=59.,
                ymax=71,
                ytick={58, 60, 62, 64, 66, 68, 70, 72, 74},
                xlabel style={yshift=0.15cm},
                ylabel style={yshift=-0.2cm},
                xmin=90,
                xmax=210,
                label style={font=\scriptsize},
                tick label style={font=\scriptsize},
                xticklabel={$\pgfmathprintnumber{\tick}$k}
            ]
            \addplot[cdeepBP, very thick, mark=diamond*, mark size=2pt, mark options={solid}] table[col sep=comma, x=x, y=am-sp]{Data/limited_budget_cityscapes_64.csv};
            \addplot[cdeepMF, very thick, mark=triangle*, mark size=2pt, mark options={solid}] table[col sep=comma, x=x, y=revisiting]{Data/limited_budget_cityscapes_64.csv};
            
            
            \legend{AMSP+S (Ours), SP~\cite{cai2021revisiting}}
            \end{axis}
        \end{tikzpicture}
        \caption{Base superpixel size of 64}
        \label{fig:(a)-small-region}
    \end{subfigure}
    \hspace{1mm}
    \begin{subfigure}{.49\linewidth}
        \centering
        \begin{tikzpicture}
            \begin{axis}[
                legend style={nodes={scale=0.6}, at={(1.65, 1.16)}},
                legend columns=-1,
                xlabel={The number of clicks},
                ylabel={mIoU (\%)},
                width=1.23\linewidth,
                height=1.23\linewidth,
                ymin=59,
                ymax=71,
                ytick={60, 62, 64, 66, 68, 70, 72, 74},
                xlabel style={yshift=0.15cm},
                ylabel style={yshift=-0.2cm},
                xmin=90,
                xmax=210,
                label style={font=\scriptsize},
                tick label style={font=\scriptsize},
                xticklabel={$\pgfmathprintnumber{\tick}$k}
            ]
            \addplot[cdeepBP, very thick, mark=diamond*, mark size=2pt, mark options={solid}] table[col sep=comma, x=x, y=am-sp]{Data/limited_budget_cityscapes_256.csv};
            \addplot[cdeepMF, very thick, mark=triangle*, mark size=2pt, mark options={solid}] table[col sep=comma, x=x, y=revisiting]{Data/limited_budget_cityscapes_256.csv};
            
            
            \end{axis}
        \end{tikzpicture}
        \caption{Base superpixel size of 256}
        \label{fig:(b)-small-region}
    \end{subfigure}
    \caption{{\em Effect of base superpixel size on Cityscapes.} The performance difference is greater when the superpixel size is smaller.
    }
    \label{fig:supple-region-size-cityscapes}
\end{figure}
\begin{figure}[!ht]
    \captionsetup[subfigure]{font=footnotesize,labelfont=footnotesize,aboveskip=0.05cm,belowskip=-0.15cm}
    \centering
    \hspace{-5mm}
    \begin{subfigure}{.49\linewidth}
        \centering
        \begin{tikzpicture}
            \begin{axis}[
                legend style={nodes={scale=0.6}, at={(1.02, 1.16)}},
                xlabel={Size of base superpixels},
                ylabel={mIoU (\%)},
                width=1.23\linewidth,
                height=1.23\linewidth,
                ymin=54.5,
                ymax=63.5,
                ytick={54, 56, 58, 60, 62, 64, 66},
                xlabel style={yshift=0.15cm},
                ylabel style={yshift=-0.2cm},
                legend columns=2,
                xmin=0.7,
                xmax=6.3,
                label style={font=\scriptsize},
                tick label style={font=\scriptsize},
                xtick=data,
                xticklabels={4,16,64,256,1024,4096},
            ]
            \addplot[cdeepBP, very thick, mark=diamond*, mark size=2pt, mark options={solid}] table[col sep=comma, x=x, y=sm]{Data/supple_region_size_pascal.csv};
            \addplot[cdeepMF, very thick, mark=triangle*, mark size=2pt, mark options={solid}] table[col sep=comma, x=x, y=revisiting] {Data/supple_region_size_pascal.csv};


            
            \legend{AMSP+S (Ours), SP~\cite{cai2021revisiting}}
            \end{axis}
        \end{tikzpicture}
        \caption{PASCAL}
    \end{subfigure}
    \caption{{\em Effect of base superpixel size on PASCAL.} Our method exhibits robustness to large superpixels, while the baseline is sensitive. 
    }
    \label{fig:supple-region-size-pascal}
\end{figure}




For ease of exposition, Figure~\ref{fig:robustness} presents the gain of our method (compared to \textit{SP} \cite{cai2021revisiting}) for a limited set of base superpixel sizes. In this section, we report an additional investigation
suggesting further gain with different base superpixels.


\label{fig:sup-small-cityscapes}
\paragraph{Further gain on Cityscapes.}
In Figure~~\ref{fig:(a)-small-region},
we additionally provide a comparison between
the proposed method (\textit{AMSP+S})
and \textit{SP} \cite{cai2021revisiting}, 
where the experiment setup with
Cityscapes
is identical to 
that in Figure~\ref{fig:(a)-effect} except that the base superpixel size is 64 (Figure~~\ref{fig:(a)-small-region}) instead of 256 (Figure~~\ref{fig:(b)-small-region}).
Our adaptive merging method (\textit{AMSP+S}) is especially effective when the superpixel size is small in Figure~\ref{fig:(a)-small-region}, thanks to the adaptive merging mechanism.
This observation suggests more significant gain of our method with other choices of base superpixel size than that in Figure~\ref{fig:robustness}.



\paragraph{Further gain on PASCAL.}
We also demonstrate a larger gap between 
the proposed method and existing one 
in PASCAL. 
In Figure~\ref{fig:supple-region-size-pascal}, our adaptive merging method (\textit{AMSP+S}) outperforms the baseline (\textit{SP}) for various superpixel sizes
as we observed in Figure~\ref{fig:robustness}.
We stress that the gain of the proposed method is particularly larger than the one reported in Figure~\ref{fig:robustness} when
the base superpixel size is 4096, which is much larger than 256 used in Figure~\ref{fig:robustness}. This is because 
the sieving procedure to 
suppresses
the noise from dominant labeling 
becomes more crucial when querying large superpixels.
The experimental setup used in Figure \ref{fig:supple-region-size-pascal} is identical to that of Figure \ref{fig:(d)-effect}. 




\begin{figure*}[!ht]
    \captionsetup[subfigure]{font=footnotesize}
    \centering
    \begin{subfigure}{.33\linewidth}
        \centering
        \includegraphics[scale=0.322]{Figures/fig9_ascend/sfig_9_1a.png}
    \end{subfigure}
    \begin{subfigure}{.33\linewidth}
        \centering
        \includegraphics[scale=0.322]{Figures/fig9_ascend/sfig_9_1b.png}
    \end{subfigure}
    \begin{subfigure}{.33\linewidth}
        \centering
        \includegraphics[scale=0.322]{Figures/fig9_ascend/sfig_9_1c.png}
    \end{subfigure}

    \begin{subfigure}{.33\linewidth}
        \centering
        \includegraphics[scale=0.322]{Figures/fig9_ascend/sfig_9_2a.png}
        \caption{Merging superpixels with low 10\% uncertainty}
    \end{subfigure}
    \begin{subfigure}{.33\linewidth}
        \centering
        \includegraphics[scale=0.322]{Figures/fig9_ascend/sfig_9_2b.png}
        \caption{Merging superpixels with high 10\% uncertainty}
    \end{subfigure}
    \begin{subfigure}{.33\linewidth}
        \centering
        \includegraphics[scale=0.322]{Figures/fig9_ascend/sfig_9_2c.png}
        \caption{Merging all superpixels}
    \end{subfigure}
    \caption{{\em Qualitative results for partial merging.} The average uncertainty of pixels in the cyan box is 0.30 for the top images and 0.27 for the bottom images, while for pixels in the red box, it is 0.09 for the top images and 0.00 for the bottom images. 
    By merging only a portion of superpixels in the order of high uncertainty, we can reduce time complexity, as it creates merged superpixels (\ie in the cyan boxes) that will be selected by the acquisition function.
    }
    \label{fig:descend}
    \vspace{-3mm}
\end{figure*}


\section{Rationale for line~\ref{line:order} of Algorithm \ref{algorithm2}}
\label{fig:sup-descend}
\begin{table}[!ht]
\centering
\setlength\tabcolsep{6pt}
\begin{tabular}{l|c}
\toprule
Methods & mIoU \\ \midrule
\textit{SP} \cite{cai2021revisiting} & 63.77 \\ \midrule
\textit{AMSP+S} (ascending, 10\%) & 64.33 \\ \midrule
\textit{AMSP+S} (descending, 10\%) & \underline{65.99} \\ \midrule
\rowcolor{Gray}
\textit{AMSP+S} (descending, 100\%) & \textbf{66.53} \\ \midrule
\bottomrule
\end{tabular}
\caption{{\em Various merging order.} Experiments are conducted on Cityscapes dataset with an average superpixel size of 256, using 100k costs for two rounds.}
\label{tab:descending}
\end{table}

We explain the rationale behind traversing nodes in the descending order of uncertainty in line~\ref{line:order} of Algorithm \ref{algorithm2}.
We convert superpixels into a graph for merging, where the time and space increase proportionally to the square of the number of superpixels $|S|$ as we compare pairs of superpixels to build an adjacency matrix, \ie $O(|S|^2)$.
Based on prioritizing superpixels with high uncertainty in the acquisition function, we propose a technique that firstly merges high-uncertainty superpixels.
To reduce the computational complexity,
it is possible to merge a {\it part} of 
base superpixels of highest uncertainty (say top $10\%$)
to generate query candidates, although the proposed method investigates
{\it all} the base superpixels in the descending order of uncertainty.
Table~\ref{tab:descending}
shows the partial merging in the descending order of uncertainty results in only a small gap to the full merging.
This suggests a tip to save the computation resource for practitioners.


In addition, Table~\ref{tab:descending}
and Figure~\ref{fig:descend}
show that it is important to prioritize the merging highly uncertain superpixels.
In Table~\ref{tab:descending},
merging along the ascending order of uncertainty degenerates the performance.
In Figure~\ref{fig:descend},
we examplify the merged superpixels
from the partial merging in the ascending or descending order of uncertainty,
and the full merging, where
the cyan boxes contain
higher values of acquisition function
than the red boxes.
The partial merging with the ascending order of uncertainty regrettably merges
the superpixels that would not be selected in AL, while that with the ascending order
efficiently combines the base superpixels
of which selection is highly like.
This difference indeed results in a huge gap in the final performance as shown in Table~\ref{tab:descending}.



\section{Rationale for the adaptive threshold $\phi(s;\theta)$ in the sieving}
\label{fig:sup-sieving}
\begin{table}[!ht]
\centering
\setlength\tabcolsep{6pt}
\begin{tabular}{l|c}
\toprule
Methods & mIoU \\ \midrule
\textit{SP} \cite{cai2021revisiting} & 63.77 \\ \midrule
\textit{AMSP+S} $(\phi(s;\theta) = 0.0)$ & \underline{65.35} \\ \midrule
\textit{AMSP+S} $(\phi(s;\theta) = 0.2)$ & 61.80 \\ \midrule
\textit{AMSP+S} $(\phi(s;\theta) = 0.4)$ & 57.77 \\ \midrule
\textit{AMSP+S} $(\phi(s;\theta) = 0.6)$ & 45.84 \\ \midrule
\textit{AMSP+S} $(\phi(s;\theta) = 0.8)$ & 38.99 \\ \midrule
\rowcolor{Gray}
\textit{AMSP+S} (Kneedle \cite{satopaa2011finding}) & \textbf{66.53} \\ \midrule
\bottomrule
\end{tabular}
\caption{{\em Various sieving methods.} Experiments are conducted on Cityscapes dataset with an average superpixel size of 256, using 100k costs for two rounds.}
\label{tab:sieving}
\end{table}
\begin{figure}[!ht]
    \captionsetup[subfigure]{font=footnotesize,labelfont=footnotesize,aboveskip=0.05cm,belowskip=-0.15cm}
    \centering
    \hspace{-3mm}
    \begin{subfigure}{.47\linewidth}
        \centering
        \begin{tikzpicture}
            \begin{axis}[
                legend style={nodes={scale=0.6}, at={(1.55, 1.16)}},
                legend columns=-1,
                xlabel={$x$},
                ylabel={$f_\theta(\text{road};x)$},
                width=1.23\linewidth,
                height=1.23\linewidth,
                ymin=-0.1,
                ymax=1.1,
                xlabel style={yshift=0.15cm},
                ylabel style={yshift=-0.2cm},
                xmin=-0.5,
                xmax=18.5,
                label style={font=\scriptsize},
                tick label style={font=\scriptsize},
                xtick=data
            ]
            \addplot[cdeepBP, very thick, mark=diamond*, mark size=2pt, mark options={solid}] table[col sep=comma, x=x, y=y]{Data/knee_road.csv};
            \draw[orange, very thick, dashed] (2,-1) -- (2,2);
            \addlegendimage{dashed, line width=0.4mm, height=1mm, color=orange}
            \legend{data, knee/elbow}
            \end{axis}
        \end{tikzpicture}
        \caption{Road}
    \end{subfigure}
    \hspace{1mm}    
    \begin{subfigure}{.47\linewidth}
        \centering
        \begin{tikzpicture}
            \begin{axis}[
                legend style={nodes={scale=0.35}, at={(0.03, 0.24)}, anchor=west}, 
                xlabel={$x$},
                ylabel={$f_\theta(\text{pole};x)$},
                width=1.23\linewidth,
                height=1.23\linewidth,
                ymin=-0.1,
                ymax=1.1,
                xlabel style={yshift=0.15cm},
                ylabel style={yshift=-0.2cm},
                xmin=-0.5,
                xmax=18.5,
                label style={font=\scriptsize},
                tick label style={font=\scriptsize},
                xtick=data,
            ]

            \addplot[cdeepBP, very thick, mark=diamond*, mark size=2pt, mark options={solid}] table[col sep=comma, x=x, y=y]{Data/knee_pole.csv};
            \draw[orange, very thick, dashed] (4,-1) -- (4,2);
            \addlegendimage{dashed, line width=0.4mm, height=1mm, color=cdeepBP32}
            \end{axis}
        \end{tikzpicture}
        \caption{Pole}
    \end{subfigure}
    \caption{{\em Examples of knee points on Cityscapes.} We obtain (a) a high knee value for the common road class and (b) a low knee value for the rare pole class.}
    \label{fig:sup-knee-points}
    \vspace{-7mm}
\end{figure}




\begin{figure*}[t!]
\captionsetup[subfigure]{font=footnotesize,labelfont=footnotesize,aboveskip=0.05cm,belowskip=-0.15cm}
\centering
\begin{tikzpicture}
    \begin{axis}[
        width  = \textwidth,
        axis y line*=left,
        symbolic x coords={
            \rotatebox{60}{Road},
            \rotatebox{60}{Building},
            \rotatebox{60}{Vegetation},
            \rotatebox{60}{Car},
            \rotatebox{60}{Sidewalk},
            \rotatebox{60}{Sky},
            \rotatebox{60}{Pole},
            \rotatebox{60}{Person},
            \rotatebox{60}{Terrain},
            \rotatebox{60}{Fence},
            \rotatebox{60}{Wall},
            \rotatebox{60}{Sign},
            \rotatebox{60}{Bicycle},
            \rotatebox{60}{Truck},
            \rotatebox{60}{Bus},
            \rotatebox{60}{Train},
            \rotatebox{60}{Light},
            \rotatebox{60}{Rider},
            \rotatebox{60}{Motorcycle},
        },
        axis x line=bottom,
        height = 5.2cm,
        major x tick style = transparent,
        ybar=3*\pgflinewidth,
        bar width=4pt,
        ymajorgrids = true,
        ylabel = {IoU},
        xtick = data,
        scaled y ticks = false,
        enlarge x limits=0.3,
        axis line style={-},
        ymin=0.0,ymax=1,
        legend columns=2,
        legend cell align=left,
        legend style={
                nodes={scale=0.6},
                at={(0.5,1.2)},
                anchor=north,
                column sep=1ex
        },
        label style={font=\scriptsize},
        tick label style={font=\scriptsize}
    ]
        \addplot[style={cdeepBP,fill=cdeepBP,mark=none}] coordinates {
            (\rotatebox{60}{Road}, 0.964601)
            (\rotatebox{60}{Building}, 0.883573)
            (\rotatebox{60}{Vegetation}, 0.897434)
            (\rotatebox{60}{Car}, 0.908026)
            (\rotatebox{60}{Sidewalk}, 0.750152)
            (\rotatebox{60}{Sky}, 0.906192)
            (\rotatebox{60}{Pole}, 0.482772)
            (\rotatebox{60}{Person}, 0.729143)
            (\rotatebox{60}{Terrain}, 0.55855)
            (\rotatebox{60}{Fence}, 0.492545)
            (\rotatebox{60}{Wall}, 0.461659)
            (\rotatebox{60}{Sign}, 0.614888)
            (\rotatebox{60}{Bicycle}, 0.67332)
            (\rotatebox{60}{Truck}, 0.559177)
            (\rotatebox{60}{Bus}, 0.733037)
            (\rotatebox{60}{Train}, 0.601527)
            (\rotatebox{60}{Light}, 0.468385)
            (\rotatebox{60}{Rider}, 0.483054)
            (\rotatebox{60}{Motorcycle}, 0.473352)
        };
        \addplot[style={orange,fill=orange,mark=none}] coordinates {
            (\rotatebox{60}{Road}, 0.955517)
            (\rotatebox{60}{Building}, 0.852214)
            (\rotatebox{60}{Vegetation}, 0.880035)
            (\rotatebox{60}{Car}, 0.851916)
            (\rotatebox{60}{Sidewalk}, 0.89579)
            (\rotatebox{60}{Sky}, 0.896663)
            (\rotatebox{60}{Pole}, 0.324208)
            (\rotatebox{60}{Person}, 0.523996)
            (\rotatebox{60}{Terrain}, 0.479043)
            (\rotatebox{60}{Fence}, 0.376733)
            (\rotatebox{60}{Wall}, 0.211172)
            (\rotatebox{60}{Sign}, 0.431976)
            (\rotatebox{60}{Bicycle}, 0.503692)
            (\rotatebox{60}{Truck}, 0.097781)
            (\rotatebox{60}{Bus}, 0.401534)
            (\rotatebox{60}{Train}, 0.128532)
            (\rotatebox{60}{Light}, 0.058156)
            (\rotatebox{60}{Rider}, 0.04937)
            (\rotatebox{60}{Motorcycle}, 0.01)
        };
        \legend{AMSP+S (Kneedle \cite{satopaa2011finding}), AMSP+S $( \phi(s) = 0.6 )$}
    \end{axis}
\end{tikzpicture}
\caption{{\em Class-wise IoU according to $\phi(s;\theta)$.} Applying the same $\phi(s)$ of 0.6 to all pixels results in excessive sieving for relatively rare classes, leading to decreased performance for these classes (\eg Light, Rider, and Motorcycle). Based on the ground-truth, class labels are organized in order of the total pixel count for each class.}
\label{fig:sup-class-dist}
\end{figure*}
\begin{figure*}[t!]
    \captionsetup[subfigure]{font=footnotesize}
    \centering
    \begin{subfigure}{.33\linewidth}
        \centering
        \includegraphics[scale=0.322]{Figures/aachen_000000_000019_seg.png}
        \caption{Semantic segmentation}
        \label{subfig:sementic-seg}
    \end{subfigure}
    \begin{subfigure}{.33\linewidth}
        \centering
        \includegraphics[scale=0.322]{Figures/aachen_000000_000019_gt.jpg}
        \caption{Panoptic segmentation}
    \end{subfigure}
    \begin{subfigure}{.33\linewidth}
        \centering
        \includegraphics[scale=0.322]{Figures/aachen_000000_000019_gt_cc.jpg}
        \caption{Oracle superpixels (ours)}
        \label{subfig:oralce-seg}
    \end{subfigure} 
    \caption{{\em Difference between conventional segmentations and oracle superpixels.} (a) When sharing the same class label, they are depicted as identical superpixels (\ie green color on separate trees).  (b) Although a building is divided by a pole, it is represented as a single superpixel (\ie cyan color). (c) We consider a building as two distinct superpixels (\ie cyan and light yellow colors).}
    \label{fig:sup-oracle-superpixels}
\end{figure*}


We provide the reason for introducing 
the threshold function $\phi(s)$
personalized for each superpixel $s$, described in Section \ref{sec:sieving-technique}.
We obtain the dominant label $\text{D}(s)$ for a queried superpixel $s$, however, we only propagate the label to pixels $x \in s$ that are predicted to have a positive impact on the training of model $\theta$ as:
\begin{equation}
h(s;\theta) := \{ x \in s: f_\theta \big( \text{D}(s); x \big) \geq \phi(s; \theta) \} \;,
\end{equation}
where $f_\theta \left( \text{D}(s);x \right)$ implies the confidence of pixel $x$ to dominant label $\text{D}(s)$ given $\theta$ and $\phi(s;\theta)$ determines the degree of sieving.
In Table \ref{tab:sieving}, we study the effect of various $\phi(s;\theta)$.
When the same $\phi(s;\theta)$ is applied to all pixels, it causes class imbalance by leaving relatively easy classes as described in Figure \ref{fig:sup-class-dist}.
To avoid this issue, we utilize the Kneedle algorithm \cite{satopaa2011finding} to obtain different $\phi(s;\theta)$ for each superpixel $s$.
Specifically, $\phi(s; \theta)$ is a knee point of the cumulative distribution function of values of $f_\theta \big( \text{D}(s); x \big)$ in superpixel $x \in s$.
However, for the Kneedle algorithm to work accurately, the curve of cumulative distribution must be either convex or concave. 
In addition, the algorithm may provide inaccurate knee points on very smooth curves.
To address this issue, we use a subset of uniformly sampled values based on $f_\theta(\text{D}(s);x)$, instead of using the distribution for all pixels.
We sample 20 and 5 pixels for Cityscapes and PASCAL datasets, respectively.
In Figure \ref{fig:sup-knee-points}, different knee points are detected according to the dominant class of superpixels.







\begin{figure*}[!ht]
    \captionsetup[subfigure]{font=scriptsize,labelfont=scriptsize,aboveskip=0.05cm,belowskip=-0.15cm}
    \centering
    \hspace{-1mm}
    \begin{subfigure}{.235\linewidth}
        \centering
        \begin{tikzpicture}
            \begin{axis}[
                legend style={nodes={scale=0.35}, at={(0.03, 0.24)}, anchor=west}, 
                xlabel={ASA$(S;G)$},
                ylabel={mIoU (\%)},
                width=1.23\linewidth,
                height=1.23\linewidth,
                ymin=50.8,
                ymax=63.2,
                ytick={51, 53, 55, 57, 59, 61, 63},
                xlabel style={yshift=0.15cm},
                ylabel style={yshift=-0.2cm},
                legend columns=2,
                xmin=0.877,
                xmax=0.971,
                label style={font=\scriptsize},
                tick label style={font=\scriptsize},
                x tick label style={
                    /pgf/number format/.cd,
                        fixed,
                }
            ]
            \addplot[cdeepBP, only marks] table[col sep=comma, x=ASASG, y=mIoU]{Data/correlation_asasg.csv};
            \addplot[very thick, orange] table[col sep=comma, x=ASASG, y={create col/linear regression = {y=mIoU}}
            ]{Data/correlation_asasg.csv};
            \draw (0.5\linewidth, 0.35\linewidth) node {\scriptsize$\text{Corr} = 0.05$};
            \end{axis}
        \end{tikzpicture}
    \end{subfigure}
    \hspace{1mm}
    \begin{subfigure}{.235\linewidth}
        \centering
        \begin{tikzpicture}
            \begin{axis}[
                legend style={nodes={scale=0.35}, at={(0.03, 0.24)}, anchor=west}, 
                xlabel={ASA$(G;S)$},
                ylabel={mIoU (\%)},
                width=1.23\linewidth,
                height=1.23\linewidth,
                ymin=50.8,
                ymax=63.2,
                xlabel style={yshift=0.15cm},
                ylabel style={yshift=-0.2cm},
                ytick={51, 53, 55, 57, 59, 61, 63},
                legend columns=2,
                xmin=-0.05,
                xmax=0.657,
                label style={font=\scriptsize},
                tick label style={font=\scriptsize},
                x tick label style={
                    /pgf/number format/.cd,
                        fixed,
                }
            ]
            \addplot[cdeepBP, only marks] table[col sep=comma, x=ASAGS, y=mIoU]{Data/correlation_asags.csv};
            \addplot[very thick, orange] table[col sep=comma, x=ASAGS, y={create col/linear regression = {y=mIoU}}
            ]{Data/correlation_asags.csv};
            \draw (0.5\linewidth, 0.35\linewidth) node {\scriptsize$\text{Corr}=0.71$};
            \end{axis}
        \end{tikzpicture}
    \end{subfigure}
    \hspace{1mm}
    \begin{subfigure}{.235\linewidth}
        \centering
        \begin{tikzpicture}
            \begin{axis}[
                legend style={nodes={scale=0.35}, at={(0.03, 0.24)}, anchor=west}, 
                xlabel={AP$(S;G)$},
                ylabel={mIoU (\%)},
                width=1.23\linewidth,
                height=1.23\linewidth,
                ymin=50.8,
                ymax=63.2,
                ytick={51, 53, 55, 57, 59, 61, 63},
                xlabel style={yshift=0.15cm},
                ylabel style={yshift=-0.2cm},
                legend columns=2,
                xmin=0.87,
                xmax=0.97,
                label style={font=\scriptsize},
                tick label style={font=\scriptsize},
                x tick label style={
                    /pgf/number format/.cd,
                        fixed,
                }
            ]
            \addplot[cdeepBP, only marks] table[col sep=comma, x=APSG, y=mIoU]{Data/correlation_apsg.csv};
            \addplot[very thick, orange] table[col sep=comma, x=APSG, y={create col/linear regression = {y=mIoU}}
            ]{Data/correlation_apsg.csv};
            \draw (0.5\linewidth, 0.35\linewidth) node {\scriptsize$\text{Corr} = -0.22$};
            \end{axis}
        \end{tikzpicture}
    \end{subfigure}
    \hspace{1mm}
    \begin{subfigure}{.235\linewidth}
        \centering
        \begin{tikzpicture}
            \begin{axis}[
                legend style={nodes={scale=0.35}, at={(0.03, 0.24)}, anchor=west}, 
                xlabel={AR$(S;G)$},
                ylabel={mIoU (\%)},
                width=1.23\linewidth,
                height=1.23\linewidth,
                ymin=50.8,
                ymax=63.2,
                xlabel style={yshift=0.15cm},
                ylabel style={yshift=-0.2cm},
                ytick={51, 53, 55, 57, 59, 61, 63},
                legend columns=2,
                xmin=0,
                xmax=0.057,
                label style={font=\scriptsize},
                tick label style={font=\scriptsize},
                x tick label style={
                    /pgf/number format/.cd,
                        fixed,
                    /tikz/.cd,
                }
            ]
            \addplot[cdeepBP, only marks] table[col sep=comma, x=ARSG, y=mIoU]{Data/correlation_arsg.csv};
            \addplot[very thick, orange] table[col sep=comma, x=ARSG, y={create col/linear regression = {y=mIoU}}
            ]{Data/correlation_arsg.csv};
            \draw (0.5\linewidth, 0.35\linewidth) node {\scriptsize$\text{Corr}=-0.02$};
            \end{axis}
        \end{tikzpicture}
    \end{subfigure}
    \hspace{-5mm}
    \begin{subfigure}{.235\linewidth}
        \centering
        \begin{tikzpicture}
            \begin{axis}[
                legend style={nodes={scale=0.35}, at={(0.03, 0.24)}, anchor=west}, 
                xlabel={AF$(S;G)$},
                ylabel={mIoU (\%)},
                width=1.23\linewidth,
                height=1.23\linewidth,
                ymin=50.8,
                ymax=63.2,
                ytick={51, 53, 55, 57, 59, 61, 63},
                xlabel style={yshift=0.15cm},
                ylabel style={yshift=-0.2cm},
                legend columns=2,
                xmin=0,
                xmax=0.082,
                label style={font=\scriptsize},
                tick label style={font=\scriptsize},
                x tick label style={
                    /pgf/number format/.cd,
                        fixed,
                    /tikz/.cd,
                }
            ]
            \addplot[cdeepBP, only marks] table[col sep=comma, x=AFSG, y=mIoU]{Data/correlation_afsg.csv};
            \addplot[very thick, orange] table[col sep=comma, x=AFSG, y={create col/linear regression = {y=mIoU}}
            ]{Data/correlation_afsg.csv};
            \draw (0.5\linewidth, 0.35\linewidth) node {\scriptsize$\text{Corr} = -0.01$};
            \end{axis}
        \end{tikzpicture}
    \end{subfigure}
    \hspace{1mm}
    \begin{subfigure}{.235\linewidth}
        \centering
        \begin{tikzpicture}
            \begin{axis}[
                legend style={nodes={scale=0.35}, at={(0.03, 0.24)}, anchor=west}, 
                xlabel={AP$(G;S)$},
                ylabel={mIoU (\%)},
                width=1.23\linewidth,
                height=1.23\linewidth,
                ymin=50.8,
                ymax=63.2,
                xlabel style={yshift=0.15cm},
                ylabel style={yshift=-0.2cm},
                ytick={51, 53, 55, 57, 59, 61, 63},
                legend columns=2,
                xmin=0.35,
                xmax=0.74,
                label style={font=\scriptsize},
                tick label style={font=\scriptsize},
                x tick label style={
                    /pgf/number format/.cd,
                        fixed,
                }
            ]
            \addplot[cdeepBP, only marks] table[col sep=comma, x=APGS, y=mIoU]{Data/correlation_apgs.csv};
            \addplot[very thick, orange] table[col sep=comma, x=APGS, y={create col/linear regression = {y=mIoU}}
            ]{Data/correlation_apgs.csv};
            \draw (0.5\linewidth, 0.35\linewidth) node {\scriptsize$\text{Corr}=-0.43$};
            \end{axis}
        \end{tikzpicture}
    \end{subfigure}
    \hspace{1mm}
    \begin{subfigure}{.235\linewidth}
        \centering
        \begin{tikzpicture}
            \begin{axis}[
                legend style={nodes={scale=0.35}, at={(0.03, 0.24)}, anchor=west}, 
                xlabel={AR$(G;S)$},
                ylabel={mIoU (\%)},
                width=1.23\linewidth,
                height=1.23\linewidth,
                ymin=50.8,
                ymax=63.2,
                ytick={51, 53, 55, 57, 59, 61, 63},
                xlabel style={yshift=0.15cm},
                ylabel style={yshift=-0.2cm},
                legend columns=2,
                xmin=0.21,
                xmax=0.696,
                label style={font=\scriptsize},
                tick label style={font=\scriptsize},
                x tick label style={
                    /pgf/number format/.cd,
                        fixed,
                }
            ]
            \addplot[cdeepBP, only marks] table[col sep=comma, x=ARGS, y=mIoU]{Data/correlation_args.csv};
            \addplot[very thick, orange] table[col sep=comma, x=ARGS, y={create col/linear regression = {y=mIoU}}
            ]{Data/correlation_args.csv};
            \draw (0.5\linewidth, 0.35\linewidth) node {\scriptsize$\text{Corr} = 0.57$};
            \end{axis}
        \end{tikzpicture}
    \end{subfigure}
    \hspace{1mm}
    \begin{subfigure}{.235\linewidth}
        \centering
        \begin{tikzpicture}
            \begin{axis}[
                legend style={nodes={scale=0.35}, at={(0.03, 0.24)}, anchor=west}, 
                xlabel={AF$(G;S)$},
                ylabel={mIoU (\%)},
                width=1.23\linewidth,
                height=1.23\linewidth,
                ymin=50.8,
                ymax=63.2,
                xlabel style={yshift=0.15cm},
                ylabel style={yshift=-0.2cm},
                ytick={51, 53, 55, 57, 59, 61, 63},
                legend columns=2,
                xmin=0.173,
                xmax=0.371,
                label style={font=\scriptsize},
                tick label style={font=\scriptsize},
                x tick label style={
                    /pgf/number format/.cd,
                        fixed,
                }
            ]
            \addplot[cdeepBP, only marks] table[col sep=comma, x=AFGS, y=mIoU]{Data/correlation_afgs.csv};
            \addplot[very thick, orange] table[col sep=comma, x=AFGS, y={create col/linear regression = {y=mIoU}}
            ]{Data/correlation_afgs.csv};
            \draw (0.5\linewidth, 0.35\linewidth) node {\scriptsize$\textbf{Corr}=\textbf{0.95}$};
            \end{axis}
        \end{tikzpicture}
    \end{subfigure}
    \caption{{\em Relationship between metrics and mIoU.} The correlation between AF$(G;S)$ and mIoU is especially high. For the correlation calculation, \textit{Oracle} in Table \ref{tab:quantitative} is excluded.}
    \label{fig:sup-correlation}
\end{figure*}
\begin{figure*}[t!]
    \captionsetup[subfigure]{font=footnotesize,labelfont=footnotesize,aboveskip=0.05cm,belowskip=-0.15cm}
    \centering
    \hspace{-1mm}
    \begin{subfigure}{.235\linewidth}
        \centering
        \begin{tikzpicture}
            \begin{axis}[
                legend style={nodes={scale=0.35}, at={(0.03, 0.24)}, anchor=west}, 
                xlabel={AF$(G;S)$},
                ylabel={mIoU (\%)},
                width=1.23\linewidth,
                height=1.23\linewidth,
                ymin=50.8,
                ymax=63.2,
                xlabel style={yshift=0.15cm},
                ylabel style={yshift=-0.2cm},
                ytick={51, 53, 55, 57, 59, 61, 63},
                legend columns=2,
                xmin=0.06,
                xmax=0.42,
                label style={font=\scriptsize},
                tick label style={font=\scriptsize},
                x tick label style={
                    /pgf/number format/.cd,
                        fixed,
                }
            ]
            \addplot[cdeepBP, only marks] table[col sep=comma, x=AFGS, y=mIoU]{Data/correlation_afgs_semantic.csv};
            \addplot[very thick, orange] table[col sep=comma, x=AFGS, y={create col/linear regression = {y=mIoU}}
            ]{Data/correlation_afgs_semantic.csv};
            \draw (0.5\linewidth, 0.35\linewidth) node {\scriptsize $\text{Corr}=0.57$};
            \end{axis}
        \end{tikzpicture}
        \caption{Semantic segmentation}
    \end{subfigure}
    \hspace{1mm}
    \begin{subfigure}{.235\linewidth}
        \centering
        \begin{tikzpicture}
            \begin{axis}[
                legend style={nodes={scale=0.35}, at={(0.03, 0.24)}, anchor=west}, 
                xlabel={AF$(G;S)$},
                ylabel={mIoU (\%)},
                width=1.23\linewidth,
                height=1.23\linewidth,
                ymin=50.8,
                ymax=63.2,
                xlabel style={yshift=0.15cm},
                ylabel style={yshift=-0.2cm},
                ytick={51, 53, 55, 57, 59, 61, 63},
                legend columns=2,
                xmin=0.21,
                xmax=0.39,
                label style={font=\scriptsize},
                tick label style={font=\scriptsize},
                x tick label style={
                    /pgf/number format/.cd,
                        fixed,
                }
            ]
            \addplot[cdeepBP, only marks] table[col sep=comma, x=AFGS, y=mIoU]{Data/correlation_afgs_panoptic.csv};
            \addplot[very thick, orange] table[col sep=comma, x=AFGS, y={create col/linear regression = {y=mIoU}}
            ]{Data/correlation_afgs_panoptic.csv};
            \draw (0.5\linewidth, 0.35\linewidth) node {\scriptsize$\textbf{Corr}=\textbf{0.95}$};
            \end{axis}
        \end{tikzpicture}
        \caption{Panoptic segmentation}
    \end{subfigure}
    \hspace{1mm}
    \begin{subfigure}{.235\linewidth}
        \centering
        \begin{tikzpicture}
            \begin{axis}[
                legend style={nodes={scale=0.35}, at={(0.03, 0.24)}, anchor=west}, 
                xlabel={AF$(G;S)$},
                ylabel={mIoU (\%)},
                width=1.23\linewidth,
                height=1.23\linewidth,
                ymin=50.8,
                ymax=63.2,
                xlabel style={yshift=0.15cm},
                ylabel style={yshift=-0.2cm},
                ytick={51, 53, 55, 57, 59, 61, 63},
                legend columns=2,
                xmin=0.173,
                xmax=0.371,
                label style={font=\scriptsize},
                tick label style={font=\scriptsize},
                x tick label style={
                    /pgf/number format/.cd,
                        fixed,
                }
            ]
            \addplot[cdeepBP, only marks] table[col sep=comma, x=AFGS, y=mIoU]{Data/correlation_afgs.csv};
            \addplot[very thick, orange] table[col sep=comma, x=AFGS, y={create col/linear regression = {y=mIoU}}
            ]{Data/correlation_afgs.csv};
            \draw (0.5\linewidth, 0.35\linewidth) node {\scriptsize$\textbf{Corr}=\textbf{0.95}$};
            \end{axis}
        \end{tikzpicture}
        \caption{Oracle superpixels}
    \end{subfigure}
    \caption{{\em Relationship between AF$(G;S)$ and mIoU varying $G$.} AF$(G;S)$ and mIoU exhibit a high correlation when ground-truth $G$ is represented by the panoptic segmentation and oracle superpixels in Figure \ref{fig:sup-oracle-superpixels}. For the correlation calculation, \textit{Oracle} in Table \ref{tab:quantitative} is excluded.}
    \label{fig:sup-afgs-g}
\end{figure*}


\begin{figure*}[t!]
    \captionsetup[subfigure]{font=footnotesize}
    \centering
    \begin{subfigure}[!ht]{.245\linewidth}
        \centering
        \includegraphics[scale=0.238]{Figures/fig13_round/bochum_000000_025833_r1.png}
    \end{subfigure}
    \begin{subfigure}[!ht]{.245\linewidth}
        \centering
        \includegraphics[scale=0.238]{Figures/fig13_round/bochum_000000_025833_r2.png}
    \end{subfigure}
    \begin{subfigure}[!ht]{.245\linewidth}
        \centering
        \includegraphics[scale=0.238]{Figures/fig13_round/bochum_000000_025833_r3.png}
    \end{subfigure}
    \begin{subfigure}[!ht]{.245\linewidth}
        \centering
        \includegraphics[scale=0.238]{Figures/fig13_round/bochum_000000_025833_r4.png}
    \end{subfigure}

    
    \begin{subfigure}[!ht]{.245\linewidth}
        \centering
        \includegraphics[scale=0.238]{Figures/fig13_round/hanover_000000_058189_r1.png}
    \end{subfigure}
    \begin{subfigure}[!ht]{.245\linewidth}
        \centering
        \includegraphics[scale=0.238]{Figures/fig13_round/hanover_000000_058189_r2.png}
    \end{subfigure}
    \begin{subfigure}[!ht]{.245\linewidth}
        \centering
        \includegraphics[scale=0.238]{Figures/fig13_round/hanover_000000_058189_r3.png}
    \end{subfigure}
    \begin{subfigure}[!ht]{.245\linewidth}
        \centering
        \includegraphics[scale=0.238]{Figures/fig13_round/hanover_000000_058189_r4.png}
    \end{subfigure}
    
    
    \begin{subfigure}[!ht]{.245\linewidth}
        \centering
        \includegraphics[scale=0.238]{Figures/fig13_round/zurich_000117_000019_r1.png}
        \caption{Adaptive merged $(t=1)$}
    \end{subfigure}
    \begin{subfigure}[!ht]{.245\linewidth}
        \centering
        \includegraphics[scale=0.238]{Figures/fig13_round/zurich_000117_000019_r2.png}
        \caption{Adaptive merged $(t=2)$}
    \end{subfigure}
    \begin{subfigure}[!ht]{.245\linewidth}
        \centering
        \includegraphics[scale=0.238]{Figures/fig13_round/zurich_000117_000019_r3.png}
        \caption{Adaptive merged $(t=3)$}
    \end{subfigure}
    \begin{subfigure}[!ht]{.245\linewidth}
        \centering
        \includegraphics[scale=0.238]{Figures/fig13_round/zurich_000117_000019_r4.png}
        \caption{Adaptive merged $(t=4)$}
    \end{subfigure}
    \caption{{\em Qualitative results with varying round.}
    (a-d) Superpixels generated with proposed adaptive merging at rounds 1 to 4.
    Thanks to the improved model, we observe that the merging becomes more accurate as the round increases. We use the model reported in Figure~\ref{fig:(a)-effect}.}
    \label{fig:sup-round}
    \vspace{-2mm}
\end{figure*}


\begin{figure*}[t!]
    \captionsetup[subfigure]{font=footnotesize}
    \centering
    \begin{subfigure}[!ht]{.245\linewidth}
        \centering
        \includegraphics[scale=0.238]{Figures/fig14_eps/bremen_000310_000019_00.png}
    \end{subfigure}
    \begin{subfigure}[!ht]{.245\linewidth}
        \centering
        \includegraphics[scale=0.238]{Figures/fig14_eps/bremen_000310_000019_01.png}
    \end{subfigure}
    \begin{subfigure}[!ht]{.245\linewidth}
        \centering
        \includegraphics[scale=0.238]{Figures/fig14_eps/bremen_000310_000019_015.png}
    \end{subfigure}
    \begin{subfigure}[!ht]{.245\linewidth}
        \centering
        \includegraphics[scale=0.238]{Figures/fig14_eps/bremen_000310_000019_02.png}
    \end{subfigure}

    
    \begin{subfigure}[!ht]{.245\linewidth}
        \centering
        \includegraphics[scale=0.238]{Figures/fig14_eps/zurich_000012_000019_005.png}
    \end{subfigure}
    \begin{subfigure}[!ht]{.245\linewidth}
        \centering
        \includegraphics[scale=0.238]{Figures/fig14_eps/zurich_000012_000019_01.png}
    \end{subfigure}
    \begin{subfigure}[!ht]{.245\linewidth}
        \centering
        \includegraphics[scale=0.238]{Figures/fig14_eps/zurich_000012_000019_015.png}
    \end{subfigure}
    \begin{subfigure}[!ht]{.245\linewidth}
        \centering
        \includegraphics[scale=0.238]{Figures/fig14_eps/zurich_000012_000019_02.png}
    \end{subfigure}
    
    
    \begin{subfigure}[!ht]{.245\linewidth}
        \centering
        \includegraphics[scale=0.238]{Figures/fig14_eps/zurich_000036_000019_005.png}
        \caption{Adaptive merged $(\epsilon=0.05)$}
    \end{subfigure}
    \begin{subfigure}[!ht]{.245\linewidth}
        \centering
        \includegraphics[scale=0.238]{Figures/fig14_eps/zurich_000036_000019_01.png}
        \caption{Adaptive merged $(\epsilon=0.1)$}
    \end{subfigure}
    \begin{subfigure}[!ht]{.245\linewidth}
        \centering
        \includegraphics[scale=0.238]{Figures/fig14_eps/zurich_000036_000019_015.png}
        \caption{Adaptive merged $(\epsilon=0.15)$}
    \end{subfigure}
    \begin{subfigure}[!ht]{.245\linewidth}
        \centering
        \includegraphics[scale=0.238]{Figures/fig14_eps/zurich_000036_000019_02.png}
        \caption{Adaptive merged $(\epsilon=0.2)$}
    \end{subfigure}
    \caption{{\em Qualitative results with varying $\epsilon$.}
    (a-d) Superpixels are generated with proposed adaptive merging with $\epsilon$: 0.05, 0.1, 0.15, 0.2. %
    We observe that an increase in $\epsilon$ gives more aggressive merging. Merging is conducted on Cityscapes with a base superpixel size of 256. }
    \label{fig:sup-epsilon}
    \vspace{-3mm}
\end{figure*}
\begin{figure*}[t!]
    \captionsetup[subfigure]{font=footnotesize}
    \centering
    \begin{subfigure}{.33\linewidth}
        \centering
        \includegraphics[scale=0.322]{Figures/fig12_qual/fig_12_1a.png}
    \end{subfigure}
    \begin{subfigure}{.33\linewidth}
        \centering
        \includegraphics[scale=0.322]{Figures/fig12_qual/fig_12_1b.png}
    \end{subfigure}
    \begin{subfigure}{.33\linewidth}
        \centering
        \includegraphics[scale=0.322]{Figures/fig12_qual/fig_12_1c_n.png}
    \end{subfigure}

    
    \begin{subfigure}{.33\linewidth}
        \centering
        \includegraphics[scale=0.322]{Figures/fig12_qual/fig_12_2a.png}
    \end{subfigure}
    \begin{subfigure}{.33\linewidth}
        \centering
        \includegraphics[scale=0.322]{Figures/fig12_qual/fig_12_2b.png}
    \end{subfigure}
    \begin{subfigure}{.33\linewidth}
        \centering
        \includegraphics[scale=0.322]{Figures/fig12_qual/fig_12_2c_n.png}
    \end{subfigure}

    \begin{subfigure}{.33\linewidth}
        \centering
        \includegraphics[scale=0.322]{Figures/fig12_qual/fig_12_3a.png}
        \caption{Base superpixels~\cite{van2012seeds}}
    \end{subfigure}
    \begin{subfigure}{.33\linewidth}
        \centering
        \includegraphics[scale=0.322]{Figures/fig12_qual/fig_12_3b.png}
        \caption{Merged superpixels (Ours)}
    \end{subfigure}
    \begin{subfigure}{.33\linewidth}
        \centering
        \includegraphics[scale=0.322]{Figures/fig12_qual/fig_12_3c_n.png}
        \caption{Oracle superpixels}
    \end{subfigure}
    \caption{{\em Qualitative results of adaptive superpixels.} (a) Base superpixel generated by SEEDS~\cite{van2012seeds} with size 256. (b) Superpixels generated with proposed adaptive merging at round 4. (c) Oracle superpixels generated from the ground truth.}
    \label{fig:sup-same-paper}
    \vspace{3mm}
\end{figure*}

\section{Further discussion on the oracle superpixels}
\label{fig:sup-oracle}

In Section \ref{para:oracle-superpixels},
we introduce the oracle superpixels,
which we believe is an achievable optimal set of superpixels for active learning.
For clarification,  
we provide the detailed process of generating the proposed oracle superpixels.
In addition, we provide further insights
into the achievable notion of optimal superpixels.

The Cityscapes dataset is equipped with the ground-truth annotations for semantic segmentation, represented by dense pixel-wise labels: \ie., each pixel in an annotated image is assigned an ID that represents a ground-truth semantic category~(Figure~\ref{subfig:sementic-seg}). In such annotation, each group of pixels that share the same ID aligns perfectly with the boundary of semantic objects. However, each such group is not guaranteed to be a single-connected component of pixels.
For example, different cars in Figure~\ref{subfig:sementic-seg} are assigned the same blue color despite being physically separated, and a car divided into two parts due to an obstructing pole is still colored blue. 
This is opposed to what we hope to achieve by merging two adjacent superpixels repeatedly.
To address this issue, we subdivide each superpixel as necessary to ensure that every pixel within a superpixel is adjacent to each other.
We utilize OpenCV~\cite{opencv_library} and Shapely~\cite{shapely2007} to identify the maximal connected component of pixels sharing the same semantic. 
We apply the same procedure to annotated images in the PASCAL dataset
Figure~\ref{fig:sup-oracle-superpixels} illustrates the distinction between conventional semantic and panoptic segmentation and our oracle superpixels.

The Cityscapes and PASCAL datasets are divided into 327k and 16k oracle superpixels, respectively.
It is worth noting that the PASCAL has a lower number of oracle superpixels due to the smaller number of classes per image.
In other words, only a few objects are of interest in each image, and the rest are simply treated as the background.

\newpage
\section{Further discussion on the achievable metrics}
\label{fig:sup-metrics}



In Table \ref{tab:quantitative}, we evaluate various superpixels using eight metrics with oracle superpixels as ground-truth $G$.
Figure \ref{fig:sup-correlation} shows the correlation between each metric and mIoU. 
We observe that our AF$(G;S)$ can be utilized to look-ahead a model's performance in active learning without training. 
In addition, we examine how different ground-truth $G$ impacts AF$(G;S)$.
In the field of semantic segmentation, two conventional segmentations, semantic and panoptic segmentations in Figure \ref{fig:sup-oracle-superpixels}, are widely used as ground-truth.
Figure \ref{fig:sup-afgs-g} indicates that using panoptic segmentation and oracle superpixels for $G$ results in higher correlation between AF$(G;S)$ and mIoU than semantic segmentation.
However, obtaining panoptic segmentation requires more costs than semantic segmentation since it utilizes additional instance information.
It is worth noting that our oracle superpixels (Figure \ref{subfig:oralce-seg}) can be easily generated even in cost-limited practical situations as they are produced from semantic segmentation (Figure \ref{subfig:sementic-seg}).




\newpage
\section{Additional qualitative adaptive superpixels}
\label{fig:sup-qual}
To facilitate comprehension of the merged superpixels, we display superpixels generated across diverse settings.
The appearance of merged superpixels is mainly determined by the model's performance and $\epsilon$. 
Figure \ref{fig:sup-round} highlights that as the round progresses, the model's performance improves, leading to more accurate merging. 
With the model fixed at round 4, Figure \ref{fig:sup-epsilon} shows the impact of adjusting $\epsilon$.
As $\epsilon$ grows, the merging process intensifies, ultimately decreasing the overall number of superpixels.
In addition, Figure \ref{fig:sup-same-paper} shows further examples of our merged superpixels.







\clearpage
\begin{table*}[t!]
\centering
\setlength\tabcolsep{6pt}
\begin{tabular}{c|l}
\toprule
Notations & Description \\ \midrule
$\mathcal{I}$ & the set of unlabeled images \\ \midrule
$\mathcal{C}$ & the set of class labels \\ \midrule
$t$ & a round \\ \midrule
$x$ & a pixel \\ \midrule
$s$ & a superpixel \\ \midrule
$S_t(i)$ & the set of superpixels in an image $i$ in round $t$ \\ \midrule
$\mathcal{S}_t$ & the set of superpixels in all images in round $t$, $\mathcal{S}_t := \bigcup_{i \in \mathcal{I}} S_t(i)$ \\ \midrule
$B$ & the query budget per round \\ \midrule
$\mathcal{B}_t$ & the set of $B$ selected superpixels in round $t$, $B_t \subset \mathcal{S}_t, |B_t| = B$\\ \midrule
$\theta_t$ & the model at the end of round $t$ \\ \midrule
$y_\theta(x)$ & the estimated dominant label of pixel $x$ given $\theta$ \\ \midrule
$\text{D}(s)$ & the true dominant label of superpixel $s$ \\ \midrule
$\text{D}_\theta(s)$ & the estimated dominant label of superpixel $s$ given $\theta$ \\ \midrule
\multirow{2}{*}{$\mathcal{G}(S) := (S, \mathcal{E}(S))$} & the graph consisting of the superpixels in $S$ as nodes and
the edge set $\mathcal{E}(S)$ \\
& such that $(s, n) \in \mathcal{E}(S)$ for each pair of adjacent superpixels $s, n \in S$.  \\ \midrule
$\epsilon$      & the hyperparameter for merging in \eqref{eq:jsd} \\
\bottomrule
\end{tabular}
\caption{{\em Notations.} The notations used in the paper are defined.}
\label{tab:notations}
\end{table*}

