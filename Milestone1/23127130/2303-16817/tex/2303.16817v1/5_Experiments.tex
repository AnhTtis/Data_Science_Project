\section{Experiments}



\subsection{Experimental setup}
\label{para:oracle-superpixels}
\paragraph{Datasets.}
We use two semantic segmentation datasets: Cityscapes~\cite{Cordts2016Cityscapes} and PASCAL VOC 2012 (PASCAL)~\cite{pascal-voc-2012}.
Cityscapes comprises 2,975 training and 500 validation images with 19 classes, while PASCAL consists of 1,464 training and 1,449 validation images with 20 classes.

\begin{figure*}[t!]
    \captionsetup[subfigure]{font=footnotesize,labelfont=footnotesize,aboveskip=0.05cm,belowskip=-0.15cm}
    \centering
    \hspace{-5mm}
    \begin{subfigure}{.23\linewidth}
        \centering
        \begin{tikzpicture}
            \begin{axis}[
                legend style={nodes={scale=0.6}, at={(2.15, 1.16)}},
                legend columns=-1,
                xlabel={The number of clicks},
                ylabel={mIoU (\%)},
                width=1.23\linewidth,
                height=1.23\linewidth,
                ymin=62.5,
                ymax=74.6,
                ytick={64, 66, 68, 70, 72, 74},
                xlabel style={yshift=0.15cm},
                ylabel style={yshift=-0.2cm},
                xmin=90,
                xmax=260,
                label style={font=\scriptsize},
                tick label style={font=\scriptsize},
                xticklabel={$\pgfmathprintnumber{\tick}$k}
            ]
            \addplot[cCL, very thick, mark=pentagon*, mark size=2pt, mark options={solid}] table[col sep=comma, x=x, y=oracle-avg]{Data/limited_budget_cityscapes.csv};
            \addplot[cdeepBP, very thick, mark=diamond*, mark size=2pt, mark options={solid}] table[col sep=comma, x=x, y=am-sp-avg]{Data/limited_budget_cityscapes.csv};
            \addplot[cdeepBP32, very thick, mark=square*, mark size=2pt, mark options={solid}] table[col sep=comma, x=x, y=m-sp-avg]{Data/limited_budget_cityscapes.csv};
            \addplot[cdeepMF, very thick, mark=triangle*, mark size=2pt, mark options={solid}] table[col sep=comma, x=x, y=revisiting-avg]{Data/limited_budget_cityscapes.csv};
            \addplot[cdeepBP32, very thick, mark=square*, mark size=2pt, mark options={solid}] table[col sep=comma, x=x, y=m-sp-avg]{Data/limited_budget_cityscapes.csv};
            \addplot[cdeepBP, very thick, mark=diamond*, mark size=2pt, mark options={solid}] table[col sep=comma, x=x, y=am-sp-avg]{Data/limited_budget_cityscapes.csv};
            \addplot[cCL, very thick, mark=pentagon*, mark size=2pt, mark options={solid}] table[col sep=comma, x=x, y=oracle-avg]{Data/limited_budget_cityscapes.csv};
            
            \addplot[name path=oracle-l, draw=none, fill=none] table[col sep=comma, x=x, y=oracle-l]{Data/limited_budget_cityscapes.csv};
            \addplot[name path=oracle-u, draw=none, fill=none] table[col sep=comma, x=x, y=oracle-u]{Data/limited_budget_cityscapes.csv};
            \addplot[cCL, fill opacity=0.15] fill between[of=oracle-l and oracle-u];
            \addplot[name path=revisiting-l, draw=none, fill=none] table[col sep=comma, x=x, y=revisiting-l]{Data/limited_budget_cityscapes.csv};
            \addplot[name path=revisiting-u, draw=none, fill=none] table[col sep=comma, x=x, y=revisiting-u]{Data/limited_budget_cityscapes.csv};
            \addplot[cdeepMF, fill opacity=0.15] fill between[of=revisiting-l and revisiting-u]; 
            \addplot[name path=m-sp-l, draw=none, fill=none] table[col sep=comma, x=x, y=m-sp-l]{Data/limited_budget_cityscapes.csv};
            \addplot[name path=m-sp-u, draw=none, fill=none] table[col sep=comma, x=x, y=m-sp-u]{Data/limited_budget_cityscapes.csv};
            \addplot[cdeepBP32, fill opacity=0.15] fill between[of=m-sp-l and m-sp-u]; 
            \addplot[name path=am-sp-l, draw=none, fill=none] table[col sep=comma, x=x, y=am-sp-l]{Data/limited_budget_cityscapes.csv};
            \addplot[name path=am-sp-u, draw=none, fill=none] table[col sep=comma, x=x, y=am-sp-u]{Data/limited_budget_cityscapes.csv};
            \addplot[cdeepBP, fill opacity=0.15] fill between[of=am-sp-l and am-sp-u]; 
            
            \legend{Oracle, AMSP+S (Ours), MSP+S, SP~\cite{cai2021revisiting}}
            \end{axis}
        \end{tikzpicture}
        \caption{Cityscapes}
        \label{fig:(a)-effect}
    \end{subfigure}
    \hspace{1mm}    
    \begin{subfigure}{.23\linewidth}
        \centering
        \begin{tikzpicture}
            \begin{axis}[
                legend style={nodes={scale=0.35}, at={(0.03, 0.24)}, anchor=west}, 
                xlabel={The number of clicks},
                ylabel={mIoU (\%)},
                width=1.23\linewidth,
                height=1.23\linewidth,
                ymin=58,
                ymax=70.1,
                ytick={60, 62, 64, 66, 68, 70},
                xlabel style={yshift=0.15cm},
                ylabel style={yshift=-0.2cm},
                legend columns=2,
                xmin=9,
                xmax=26,
                label style={font=\scriptsize},
                tick label style={font=\scriptsize},
                xticklabel={$\pgfmathprintnumber{\tick}$k}
            ]
            \addplot[cCL, very thick, mark=pentagon*, mark size=2pt, mark options={solid}] table[col sep=comma, x=x, y=oracle]{Data/limited_budget_pascal.csv};
            \addplot[cBP, very thick, mark=square*, mark size=2pt, mark options={solid}] table[col sep=comma, x=x, y=m-sp]{Data/limited_budget_pascal.csv};
            \addplot[cdeepMF, very thick, mark=triangle*, mark size=2pt, mark options={solid}] table[col sep=comma, x=x, y=revisiting]{Data/limited_budget_pascal.csv};
            \addplot[cdeepBP32, very thick, mark=square*, mark size=2pt, mark options={solid}] table[col sep=comma, x=x, y=m-sp]{Data/limited_budget_pascal.csv};
            \addplot[cdeepBP, very thick, mark=diamond*, mark size=2pt, mark options={solid}] table[col sep=comma, x=x, y=am-sp]{Data/limited_budget_pascal.csv};
            
            \addplot[name path=oracle-l, draw=none, fill=none] table[col sep=comma, x=x, y=oracle-l]{Data/limited_budget_pascal.csv};
            \addplot[name path=oracle-u, draw=none, fill=none] table[col sep=comma, x=x, y=oracle-u]{Data/limited_budget_pascal.csv};
            \addplot[cCL, fill opacity=0.15] fill between[of=oracle-l and oracle-u];
            \addplot[name path=revisiting-l, draw=none, fill=none] table[col sep=comma, x=x, y=revisiting-l]{Data/limited_budget_pascal.csv};
            \addplot[name path=revisiting-u, draw=none, fill=none] table[col sep=comma, x=x, y=revisiting-u]{Data/limited_budget_pascal.csv};
            \addplot[cdeepMF, fill opacity=0.15] fill between[of=revisiting-l and revisiting-u];
            \addplot[name path=m-sp-l, draw=none, fill=none] table[col sep=comma, x=x, y=m-sp-l]{Data/limited_budget_pascal.csv};
            \addplot[name path=m-sp-u, draw=none, fill=none] table[col sep=comma, x=x, y=m-sp-u]{Data/limited_budget_pascal.csv};
            \addplot[cdeepBP32, fill opacity=0.15] fill between[of=m-sp-l and m-sp-u];            
            \addplot[name path=am-sp-l, draw=none, fill=none] table[col sep=comma, x=x, y=am-sp-l]{Data/limited_budget_pascal.csv};
            \addplot[name path=am-sp-u, draw=none, fill=none] table[col sep=comma, x=x, y=am-sp-u]{Data/limited_budget_pascal.csv};
            \addplot[cdeepBP, fill opacity=0.15] fill between[of=am-sp-l and am-sp-u]; 
            
            \end{axis}
        \end{tikzpicture}
        \caption{PASCAL}
        \label{fig:(b)-effect}
    \end{subfigure}
    \hspace{1mm}
    \begin{subfigure}{.23\linewidth}
        \centering
        \begin{tikzpicture}
            \begin{axis}[
                legend style={nodes={scale=0.6}, at={(2.1, 1.16)}},
                legend columns=-1,
                xlabel={Size of base superpixels},
                ylabel={mIoU (\%)},
                width=1.23\linewidth,
                height=1.23\linewidth,
                ymin=54.2,
                ymax=68.6,
                ytick={54, 56, 58, 60, 62, 64, 66, 68, 70},
                xlabel style={yshift=0.15cm},
                ylabel style={yshift=-0.2cm},
                xmin=0.7,
                xmax=4.3,
                label style={font=\scriptsize},
                tick label style={font=\scriptsize},
                xtick=data,
                xticklabels={64,256,1024,4096},
            ]
            \addplot[cCL, very thick, mark=pentagon*, mark size=2pt, mark options={solid}] table[col sep=comma, x=x, y=oracle]{Data/region_size_cityscapes.csv};
            \addplot[cdeepBP, very thick, mark=diamond*, mark size=2pt, mark options={solid}] table[col sep=comma, x=x, y=am-sp]{Data/region_size_cityscapes.csv};
            \addplot[cMV, very thick, mark=*, mark size=2pt, mark options={solid}] table[col sep=comma, x=x, y=s-sp]{Data/region_size_cityscapes.csv};
            \addplot[cdeepMF, very thick, mark=triangle*, mark size=2pt, mark options={solid}] table[col sep=comma, x=x, y=sp] {Data/region_size_cityscapes.csv};
            \addplot[cMV, very thick, mark=*, mark size=2pt, mark options={solid}] table[col sep=comma, x=x, y=s-sp]{Data/region_size_cityscapes.csv};
            \addplot[cdeepBP, very thick, mark=diamond*, mark size=2pt, mark options={solid}] table[col sep=comma, x=x, y=am-sp]{Data/region_size_cityscapes.csv};

            \addplot[name path=oracle-l, draw=none, fill=none] table[col sep=comma, x=x, y=oracle-l]{Data/region_size_cityscapes.csv};
            \addplot[name path=oracle-u, draw=none, fill=none] table[col sep=comma, x=x, y=oracle-u]{Data/region_size_cityscapes.csv};
            \addplot[cCL, fill opacity=0.15] fill between[of=oracle-l and oracle-u];
            \addplot[name path=sp-l, draw=none, fill=none] table[col sep=comma, x=x, y=sp-l]{Data/region_size_cityscapes.csv};
            \addplot[name path=sp-u, draw=none, fill=none] table[col sep=comma, x=x, y=sp-u]{Data/region_size_cityscapes.csv};
            \addplot[cdeepMF, fill opacity=0.15] fill between[of=sp-l and sp-u];
            \addplot[name path=s-sp-l, draw=none, fill=none] table[col sep=comma, x=x, y=s-sp-l]{Data/region_size_cityscapes.csv};
            \addplot[name path=s-sp-u, draw=none, fill=none] table[col sep=comma, x=x, y=s-sp-u]{Data/region_size_cityscapes.csv};
            \addplot[cMV, fill opacity=0.15] fill between[of=s-sp-l and s-sp-u];
            \addplot[name path=am-sp-l, draw=none, fill=none] table[col sep=comma, x=x, y=am-sp-l]{Data/region_size_cityscapes.csv};
            \addplot[name path=am-sp-u, draw=none, fill=none] table[col sep=comma, x=x, y=am-sp-u]{Data/region_size_cityscapes.csv};
            \addplot[cdeepBP, fill opacity=0.15] fill between[of=am-sp-l and am-sp-u];
            
            \legend{Oracle,AMSP+S (Ours),SP+S,SP~\cite{cai2021revisiting}}
            \end{axis}
        \end{tikzpicture}
        \caption{Cityscapes}
        \label{fig:(c)-effect}
    \end{subfigure}
    \hspace{1mm}    
    \begin{subfigure}{.23\linewidth}
        \centering
        \begin{tikzpicture}
            \begin{axis}[
                legend style={nodes={scale=0.35}, at={(0.03, 0.24)}, anchor=west}, 
                xlabel={Size of base superpixels},
                ylabel={mIoU (\%)},
                width=1.23\linewidth,
                height=1.23\linewidth,
                ymin=53.8,
                ymax=66.6,
                ytick={54, 56, 58, 60, 62, 64, 66},
                xlabel style={yshift=0.15cm},
                ylabel style={yshift=-0.2cm},
                legend columns=2,
                xmin=0.7,
                xmax=4.3,
                label style={font=\scriptsize},
                tick label style={font=\scriptsize},
                xtick=data,
                xticklabels={4,16,64,256},
            ]
            \addplot[cdeepMF, very thick, mark=triangle*, mark size=2pt, mark options={solid}] table[col sep=comma, x=x, y=sp] {Data/region_size_pascal.csv};
            \addplot[cMV, very thick, mark=*, mark size=2pt, mark options={solid}] table[col sep=comma, x=x, y=s-sp]{Data/region_size_pascal.csv};
            \addplot[cdeepBP, very thick, mark=diamond*, mark size=2pt, mark options={solid}] table[col sep=comma, x=x, y=am-sp]{Data/region_size_pascal.csv};
            \addplot[cCL, very thick, mark=pentagon*, mark size=2pt, mark options={solid}] table[col sep=comma, x=x, y=oracle]{Data/region_size_pascal.csv};

            \addplot[name path=sp-l, draw=none, fill=none] table[col sep=comma, x=x, y=sp-l]{Data/region_size_pascal.csv};
            \addplot[name path=sp-u, draw=none, fill=none] table[col sep=comma, x=x, y=sp-u]{Data/region_size_pascal.csv};
            \addplot[cdeepMF, fill opacity=0.15] fill between[of=sp-l and sp-u];
            \addplot[name path=s-sp-l, draw=none, fill=none] table[col sep=comma, x=x, y=s-sp-l]{Data/region_size_pascal.csv};
            \addplot[name path=s-sp-u, draw=none, fill=none] table[col sep=comma, x=x, y=s-sp-u]{Data/region_size_pascal.csv};
            \addplot[cMV, fill opacity=0.15] fill between[of=s-sp-l and s-sp-u];
            \addplot[name path=am-sp-l, draw=none, fill=none] table[col sep=comma, x=x, y=am-sp-l]{Data/region_size_pascal.csv};
            \addplot[name path=am-sp-u, draw=none, fill=none] table[col sep=comma, x=x, y=am-sp-u]{Data/region_size_pascal.csv};
            \addplot[cdeepBP, fill opacity=0.15] fill between[of=am-sp-l and am-sp-u]; 
            \addplot[name path=oracle-l, draw=none, fill=none] table[col sep=comma, x=x, y=oracle-l]{Data/region_size_pascal.csv};
            \addplot[name path=oracle-u, draw=none, fill=none] table[col sep=comma, x=x, y=oracle-u]{Data/region_size_pascal.csv};
            \addplot[cCL, fill opacity=0.15] fill between[of=oracle-l and oracle-u];
            \end{axis}
        \end{tikzpicture}
        \caption{PASCAL}
        \label{fig:(d)-effect}
    \end{subfigure}
    \caption{{\em Effect of adaptive superpixels.} (a, b) mIoU versus the number of clicks as budget. (c, d) mIoU versus the size of base superpixels. Each experiment is conducted with three trials and the shaded region indicates ranges.}
    \label{fig:robustness}
    \vspace{-2mm}
\end{figure*}


\begin{figure*}[t!]
    \captionsetup[subfigure]{font=footnotesize}
    \centering
    \begin{subfigure}{.33\linewidth}
        \centering
        \includegraphics[scale=0.322]{Figures/fig4_a_1.png}
    \end{subfigure}
    \begin{subfigure}{.33\linewidth}
        \centering
        \includegraphics[scale=0.322]{Figures/fig4_b_1.png}
    \end{subfigure}
    \begin{subfigure}{.33\linewidth}
        \centering
        \includegraphics[scale=0.322]{Figures/fig_4_1c_n.png}
    \end{subfigure}
    \begin{subfigure}{.33\linewidth}
        \centering
        \includegraphics[scale=0.322]{Figures/fig4_a_2.png}
        \caption{Base superpixels~\cite{van2012seeds}}
        \label{(a)-qualitative}
    \end{subfigure}
    \begin{subfigure}{.33\linewidth}
        \centering
        \includegraphics[scale=0.322]{Figures/fig4_b_2.png}
        \caption{Merged superpixels (Ours)}
        \label{(b)-qualitative}
    \end{subfigure}
    \begin{subfigure}{.33\linewidth}
        \centering
        \includegraphics[scale=0.321]{Figures/fig_4_2c_n.png}
        \caption{Oracle superpixels}
        \label{(c)-qualitative}
    \end{subfigure}
    \caption{{\em Qualitative results of adaptive superpixels.} (a) Base superpixel generated by SEEDS~\cite{van2012seeds} with size 256. (b) Superpixels generated with proposed adaptive merging at round 4. (c) Oracle superpixels generated from the ground truth.}
    \label{fig:qualitative}
    \vspace{-3mm}
\end{figure*}


            
            
            
            

            


            

            


\vspace{-4mm}
\paragraph{Implementation details.}
We adopt DeepLab-v3+ architecture with Xception-65~\cite{chen2018encoder} as our segmentation backbone.
During training, we use the SGD optimizer with a momentum of 0.9 and set a base learning rate to 7e-3.
We decay the learning rate by polynomial decay with a power of 0.9.
For Cityscapes, we resize training images to 769 $\times$ 769 and train a model for 60k iterations with a mini-batch size of 4.
Similarly, for PASCAL, we resize training images to 513 $\times$ 513 and train a model for 30k iterations with a mini-batch size of 12.
Unless specified, we set the value of $\epsilon$ to 0.1.


\vspace{-4mm}
\paragraph{Baseline methods.} 
We compare our algorithm to \textit{SP} \cite{cai2021revisiting}, the state-of-the-art superpixel-based active segmentation method.
Our algorithm applies two proposed processes in each round: merging and sieving.
We call our complete method including adaptive merging as \textit{AMSP+S}, while the partial version that only uses the sieving without the merging is called \textit{SP+S}.
Additionally, we evaluate the modified version of our method that performs merging only once in the second round, called \textit{MSP+S}.
Note that \textit{AMSP+S} and \textit{MSP+S} are identical until the second round.






\vspace{-4mm}
\paragraph{Oracle baseline.}
The adaptive superpixel aims to merge every connected region with the same class labels.
Thus, the upper bound of it is to consider each region separated by the ground truth mask as a superpixel.
We refer to such ideal regions as oracle superpixels in Figure~\ref{(c)-qualitative}.
An active learning model trained using the oracle superpixels is called \textit{Oracle}.
Details are in Appendix \ref{fig:sup-oracle}.
As the number of oracle superpixels is limited, all of them are eventually labeled as the round progresses, and the performance of the trained model becomes equivalent to that of the pixel-wise fully supervised model.
We report 100\% and 90\% of the \textit{Oracle} performance for Cityscapes and PASCAL, respectively.


\vspace{-4mm}
\paragraph{Evaluation protocol.}
We set the average size of the superpixels to 256 and 64 pixels on Cityscapes and PASCAL, respectively, for all experiments except for one where we adjust the size.
Following \textit{SP}~\cite{cai2021revisiting}, we use the number of clicks as the labeling budget.
We conduct 5 rounds of data sampling, where we allocate a budget of 50k and 5k for each round on Cityscapes and PASCAL, respectively.
In the first round, we randomly select superpixels to train a model, ensuring that all methods start at the same performance.
We evaluate the trained model with mean Intersection-over-Union~\cite{everingham2015pascal} on the validation images.
We emphasize that the average size of superpixels containing 64 pixels is more efficient on Cityscapes, as detailed in Appendix \ref{fig:sup-small-cityscapes}.











\subsection{Effect of adaptive superpixels}
\label{sec:effect-of-adaptive}

\paragraph{Multi-round scenario.}
In Figures~\ref{fig:(a)-effect} and \ref{fig:(b)-effect}, we compare the performance of the proposed method to \textit{SP}~\cite{cai2021revisiting} varying budget for both of Cityscapes and PASCAL.
Note that the performance for round 0, \ie., 50K budget, is omitted as each method has the same performance at the warm-up round.
The results show that our adaptive superpixel (\textit{AMSP+S}) clearly outperforms the previous art in every budget setting on both of the datasets.
In particular, the \textit{AMSP+S} with only 150k clicks outperforms the previous art with 250k clicks in Cityscapes.
In the final round, the proposed method recovers 97\% and 92\% of the \textit{Oracle} performance for Cityscapes and PASCAL, respectively.
To show the effectiveness of our adaptive approach, we compare \textit{AMSP+S} to its one-shot merging version \textit{MSP+S} in Figures~\ref{fig:(a)-effect} and \ref{fig:(b)-effect}.
On both datasets, adaptive feature of \textit{AMSP+S} shows performance gain especially for the last two rounds.

\begin{table*}[t!]
  \centering
  \setlength\tabcolsep{4pt}
  \begin{tabular}{l|cc|ccc|ccc|c}
    \toprule
    Methods & ASA$(S;G)$ & ASA$(G;S)$ & AP$(S;G)$ & AR$(S;G)$ & AF$(S;G)$ & AP$(G;S)$ & AR$(G;S)$ & AF$(G;S)$ & mIoU \\ 
    \midrule
    $\text{SLIC}_{4096}$ & 0.887 & 0.082 & 0.897 & 0.046 & 0.066 & 0.695 & 0.259 & 0.185 & 53.18 \\
    $\text{SEEDS}_{4096}$ & 0.909 & 0.082 & 0.900 & 0.050 & 0.070 & 0.665 & 0.309 & 0.221 & 57.61 \\
    $\text{SLIC}_{256}$ & 0.956 & 0.013 & 0.958 & 0.007 & 0.012 & 0.400 & 0.622 & 0.278 & 58.04 \\
    $\text{SEEDS}_{256}$ & 0.961 & 0.014 & 0.960 & 0.007 & 0.012 & 0.395 & 0.647 & 0.297 & 58.97 \\
    \rowcolor{Gray}
    $\text{Merged}_2$ & 0.898 & 0.515 & 0.883 & 0.042 & 0.063 & 0.553 & 0.472 & 0.333 & \underline{60.00} \\
    \rowcolor{Gray}
    $\text{Merged}_4$ & 0.898 & 0.496 & 0.883 & 0.042 & 0.062 & 0.548 & 0.484 & 0.340 & \textbf{61.36} \\
    \midrule
    $\text{Merged}^*$ & 0.899 & 0.597 & 0.880 & 0.045 & 0.066 & 0.547 & 0.510 & 0.359 & 61.85 \\
    Oracle & 1.000 & 1.000 & 1.000 & 1.000 & 1.000 & 1.000 & 1.000 & 1.000 & 70.81 \\
    \bottomrule
  \end{tabular}
  \caption{{\em Evaluation metrics of superpixels.}
  The subscript indicates the average size of the superpixel for SLIC~\cite{achanta2012slic} and SEEDS~\cite{van2012seeds}, while it indicates the round for Merged.
  $\text{Merged}^*$ indicates superpixel merged by a model trained with full supervision.
  To compute the mIoU, we train a model with 100k randomly selected superpixels.}
  \label{tab:quantitative}
  \vspace{-3mm}
\end{table*}

\vspace{-5mm}
\paragraph{Multi-size scenario.}
The size of superpixels is an essential hyperparameter in superpixel-based AL, affecting both the quantity and quality of labels.
In Figures~\ref{fig:(c)-effect} and \ref{fig:(d)-effect}, we compare the proposed method to \textit{SP}~\cite{cai2021revisiting} varying the base superpixel size for both of Cityscapes and PASCAL, in the second round.
Our adaptive superpixel (\textit{AMSP+S}) outperforms the previous art in various superpixel sizes on both of the datasets.
We also evaluate sieving only version (\textit{SP+S}) of our method, which quantifies contribution of each components in our method.
The performance improvement between \textit{SP} and \textit{SP+S} shows our sieving is especially helpful for large superpixels, and the performance gap between \textit{SP+S} and \textit{AMSP+S} shows our merging is especially effective for small superpixels.
Thanks to the proposed sieving and merging, \textit{AMSP+S} are comparably robust to the change of the superpixel size than \textit{SP}.

\vspace{-4.5mm}
\paragraph{Qualitative results.}
The quality of the proposed adaptive superpixel is illustrated in Figure~\ref{fig:qualitative}.
As shown in Figure~\ref{(a)-qualitative}, superpixels used in the previous study \cite{cai2021revisiting} have uniform sizes for all areas regardless of their content.
In contrast, Figure~\ref{(b)-qualitative} demonstrates that adaptive superpixels accurately reflect the actual size of the content in images, carefully capturing small object classes while efficiently covering large background classes.
Additional examples are in Appendix \ref{fig:sup-qual}.





