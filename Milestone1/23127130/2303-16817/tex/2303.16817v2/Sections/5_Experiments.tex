\section{Experiments}
% \begin{table}
%   \centering
%   \begin{tabular}{@{}lc@{}}
%     \toprule
%     Method & Budgets \\
%     \midrule
%     EntropyBox+ & 10.25$\%$ \\
%     MetaBox+    & 10.47$\%$ \\
%     SEEDS+AL & 7.85$\%$ \\
%     EAM+AL & \\
%     \midrule 
%     Oracle & \\
%     Ours & \\
%     \bottomrule
%   \end{tabular}
%   \caption{Budgets to obtain 95\% accuracy on Cityscapes}
%   \label{tab:example}
% \end{table}



\subsection{Experimental setup}
\label{para:oracle-superpixels}
\iffalse
\smallskip\noindent\textbf{Evaluation datasets.}
%To control various constraints including the size of superpixels and annotation budgets,
We use two widely known semantic segmentation datasets: Cityscapes \cite{Cordts2016Cityscapes} and PASCAL VOC 2012 (PASCAL) \cite{pascal-voc-2012}. 
Cityscapes consists of 2,975 training and 500 validation images with 19 classes, while PASCAL consists of 1,464 training and 1,449 validation images with 20 classes.
Regardless of the resolution of images, we set the average superpixel size, i.e., the number of pixels in superpixels, to 256 for all experiments except for one where we adjust the size.
After training a model with the superpixel selected by an acquisition function, we evaluate the mean Intersection-over-Union \cite{everingham2015pascal} of the model on the validation images.
\fi
\smallskip\noindent\textbf{Datasets.}
We use two semantic segmentation datasets: Cityscapes~\cite{Cordts2016Cityscapes} and PASCAL VOC 2012 (PASCAL)~\cite{pascal-voc-2012}.
Cityscapes comprises 2,975 training and 500 validation images with 19 classes, while PASCAL consists of 1,464 training and 1,449 validation images with 20 classes.

\begin{figure*}[t!]
    % \captionsetup[subfigure]{font=footnotesize,labelfont=footnotesize,aboveskip=0.05cm,belowskip=-0.15cm}
    \centering
    \hspace{-5mm}
    \begin{subfigure}{.23\linewidth}
        \centering
        \begin{tikzpicture}
            \begin{axis}[
                % legend style={nodes={scale=0.5}, at={(0.5, 0.19)}, anchor=west}, 
                legend style={nodes={scale=0.6}, at={(2.15, 1.16)}},
                legend columns=-1,
                xlabel={The number of clicks},
                ylabel={mIoU (\%)},
                width=1.23\linewidth,
                height=1.23\linewidth,
                ymin=62.5,
                ymax=74.6,
                ytick={64, 66, 68, 70, 72, 74},
                xlabel style={yshift=0.15cm},
                % ylabel style={yshift=-0.6cm},
                ylabel style={yshift=-0.2cm},
                % legend columns=1,
                xmin=90,
                xmax=260,
                label style={font=\scriptsize},
                tick label style={font=\scriptsize},
                xticklabel={$\pgfmathprintnumber{\tick}$k}
            ]
            % Oracle
            \addplot[cCL, very thick, mark=pentagon*, mark size=2pt, mark options={solid}] table[col sep=comma, x=x, y=oracle-avg]{Data/limited_budget_cityscapes.csv};
            % AM-SP
            \addplot[cdeepBP, very thick, mark=diamond*, mark size=2pt, mark options={solid}] table[col sep=comma, x=x, y=am-sp-avg]{Data/limited_budget_cityscapes.csv};
            % M-SP
            \addplot[cdeepBP32, very thick, mark=square*, mark size=2pt, mark options={solid}] table[col sep=comma, x=x, y=m-sp-avg]{Data/limited_budget_cityscapes.csv};
            % SP
            \addplot[cdeepMF, very thick, mark=triangle*, mark size=2pt, mark options={solid}] table[col sep=comma, x=x, y=revisiting-avg]{Data/limited_budget_cityscapes.csv};
            % M-SP
            \addplot[cdeepBP32, very thick, mark=square*, mark size=2pt, mark options={solid}] table[col sep=comma, x=x, y=m-sp-avg]{Data/limited_budget_cityscapes.csv};
            % AM-SP
            \addplot[cdeepBP, very thick, mark=diamond*, mark size=2pt, mark options={solid}] table[col sep=comma, x=x, y=am-sp-avg]{Data/limited_budget_cityscapes.csv};
            % Oracle
            \addplot[cCL, very thick, mark=pentagon*, mark size=2pt, mark options={solid}] table[col sep=comma, x=x, y=oracle-avg]{Data/limited_budget_cityscapes.csv};
            
            % Oracle
            \addplot[name path=oracle-l, draw=none, fill=none] table[col sep=comma, x=x, y=oracle-l]{Data/limited_budget_cityscapes.csv};
            \addplot[name path=oracle-u, draw=none, fill=none] table[col sep=comma, x=x, y=oracle-u]{Data/limited_budget_cityscapes.csv};
            \addplot[cCL, fill opacity=0.15] fill between[of=oracle-l and oracle-u];
            % SP
            \addplot[name path=revisiting-l, draw=none, fill=none] table[col sep=comma, x=x, y=revisiting-l]{Data/limited_budget_cityscapes.csv};
            \addplot[name path=revisiting-u, draw=none, fill=none] table[col sep=comma, x=x, y=revisiting-u]{Data/limited_budget_cityscapes.csv};
            \addplot[cdeepMF, fill opacity=0.15] fill between[of=revisiting-l and revisiting-u]; 
            % M-SP
            \addplot[name path=m-sp-l, draw=none, fill=none] table[col sep=comma, x=x, y=m-sp-l]{Data/limited_budget_cityscapes.csv};
            \addplot[name path=m-sp-u, draw=none, fill=none] table[col sep=comma, x=x, y=m-sp-u]{Data/limited_budget_cityscapes.csv};
            \addplot[cdeepBP32, fill opacity=0.15] fill between[of=m-sp-l and m-sp-u]; 
            % AM-SP
            \addplot[name path=am-sp-l, draw=none, fill=none] table[col sep=comma, x=x, y=am-sp-l]{Data/limited_budget_cityscapes.csv};
            \addplot[name path=am-sp-u, draw=none, fill=none] table[col sep=comma, x=x, y=am-sp-u]{Data/limited_budget_cityscapes.csv};
            \addplot[cdeepBP, fill opacity=0.15] fill between[of=am-sp-l and am-sp-u]; 
            
            \legend{Oracle, AMSP+S (Ours), MSP+S, SP~\cite{cai2021revisiting}}
            \end{axis}
        % \node[above] at (3.7, 3.41) {\small Performance for varying budget};
        \end{tikzpicture}
        \caption{Cityscapes}
        \label{fig:(a)-effect}
    \end{subfigure}
    \hspace{1mm}    
    \begin{subfigure}{.23\linewidth}
        \centering
        \begin{tikzpicture}
            \begin{axis}[
                legend style={nodes={scale=0.35}, at={(0.03, 0.24)}, anchor=west}, 
                xlabel={The number of clicks},
                ylabel={mIoU (\%)},
                width=1.23\linewidth,
                height=1.23\linewidth,
                ymin=58,
                ymax=70.1,
                ytick={60, 62, 64, 66, 68, 70},
                xlabel style={yshift=0.15cm},
                % ylabel style={yshift=-0.6cm},
                ylabel style={yshift=-0.2cm},
                legend columns=2,
                xmin=9,
                xmax=26,
                label style={font=\scriptsize},
                % xticklabels={\scriptsize 10k, 15k, 20k, 25k},
                tick label style={font=\scriptsize},
                xticklabel={$\pgfmathprintnumber{\tick}$k}
            ]
            % Oracle
            \addplot[cCL, very thick, mark=pentagon*, mark size=2pt, mark options={solid}] table[col sep=comma, x=x, y=oracle]{Data/limited_budget_pascal.csv};
            % M-SP
            \addplot[cBP, very thick, mark=square*, mark size=2pt, mark options={solid}] table[col sep=comma, x=x, y=m-sp]{Data/limited_budget_pascal.csv};
            % SP
            \addplot[cdeepMF, very thick, mark=triangle*, mark size=2pt, mark options={solid}] table[col sep=comma, x=x, y=revisiting]{Data/limited_budget_pascal.csv};
            % M-SP
            \addplot[cdeepBP32, very thick, mark=square*, mark size=2pt, mark options={solid}] table[col sep=comma, x=x, y=m-sp]{Data/limited_budget_pascal.csv};
            % AM-SP
            \addplot[cdeepBP, very thick, mark=diamond*, mark size=2pt, mark options={solid}] table[col sep=comma, x=x, y=am-sp]{Data/limited_budget_pascal.csv};
            
            % Oracle
            \addplot[name path=oracle-l, draw=none, fill=none] table[col sep=comma, x=x, y=oracle-l]{Data/limited_budget_pascal.csv};
            \addplot[name path=oracle-u, draw=none, fill=none] table[col sep=comma, x=x, y=oracle-u]{Data/limited_budget_pascal.csv};
            \addplot[cCL, fill opacity=0.15] fill between[of=oracle-l and oracle-u];
            % SP
            \addplot[name path=revisiting-l, draw=none, fill=none] table[col sep=comma, x=x, y=revisiting-l]{Data/limited_budget_pascal.csv};
            \addplot[name path=revisiting-u, draw=none, fill=none] table[col sep=comma, x=x, y=revisiting-u]{Data/limited_budget_pascal.csv};
            \addplot[cdeepMF, fill opacity=0.15] fill between[of=revisiting-l and revisiting-u];
            % M-SP
            \addplot[name path=m-sp-l, draw=none, fill=none] table[col sep=comma, x=x, y=m-sp-l]{Data/limited_budget_pascal.csv};
            \addplot[name path=m-sp-u, draw=none, fill=none] table[col sep=comma, x=x, y=m-sp-u]{Data/limited_budget_pascal.csv};
            \addplot[cdeepBP32, fill opacity=0.15] fill between[of=m-sp-l and m-sp-u];            
            % AM-SP
            \addplot[name path=am-sp-l, draw=none, fill=none] table[col sep=comma, x=x, y=am-sp-l]{Data/limited_budget_pascal.csv};
            \addplot[name path=am-sp-u, draw=none, fill=none] table[col sep=comma, x=x, y=am-sp-u]{Data/limited_budget_pascal.csv};
            \addplot[cdeepBP, fill opacity=0.15] fill between[of=am-sp-l and am-sp-u]; 
            
            \end{axis}
        % \node[above] at (1.7, 3.35) {\footnotesize PASCAL VOC 2012};s
        \end{tikzpicture}
        \caption{PASCAL}
        \label{fig:(b)-effect}
    \end{subfigure}
    \hspace{1mm}
    \begin{subfigure}{.23\linewidth}
        \centering
        \begin{tikzpicture}
            \begin{axis}[
                % legend style={nodes={scale=0.5}, at={(0.5, 0.19)}, anchor=west}, 
                legend style={nodes={scale=0.6}, at={(2.1, 1.16)}},
                legend columns=-1,
                xlabel={Size of base superpixels},
                ylabel={mIoU (\%)},
                width=1.23\linewidth,
                height=1.23\linewidth,
                ymin=54.2,
                ymax=68.6,
                ytick={54, 56, 58, 60, 62, 64, 66, 68, 70},
                xlabel style={yshift=0.15cm},
                % ylabel style={yshift=-0.6cm},
                ylabel style={yshift=-0.2cm},
                % legend columns=1,
                xmin=0.7,
                xmax=4.3,
                label style={font=\scriptsize},
                tick label style={font=\scriptsize},
                xtick=data,
                xticklabels={64,256,1024,4096},
            ]
            % Oracle
            \addplot[cCL, very thick, mark=pentagon*, mark size=2pt, mark options={solid}] table[col sep=comma, x=x, y=oracle]{Data/region_size_cityscapes.csv};
            % AM-SP
            \addplot[cdeepBP, very thick, mark=diamond*, mark size=2pt, mark options={solid}] table[col sep=comma, x=x, y=am-sp]{Data/region_size_cityscapes.csv};
            % S-SP
            \addplot[cMV, very thick, mark=*, mark size=2pt, mark options={solid}] table[col sep=comma, x=x, y=s-sp]{Data/region_size_cityscapes.csv};
            % SP
            \addplot[cdeepMF, very thick, mark=triangle*, mark size=2pt, mark options={solid}] table[col sep=comma, x=x, y=sp] {Data/region_size_cityscapes.csv};
            % S-SP
            \addplot[cMV, very thick, mark=*, mark size=2pt, mark options={solid}] table[col sep=comma, x=x, y=s-sp]{Data/region_size_cityscapes.csv};
            % AM-SP
            \addplot[cdeepBP, very thick, mark=diamond*, mark size=2pt, mark options={solid}] table[col sep=comma, x=x, y=am-sp]{Data/region_size_cityscapes.csv};

            % Oracle
            \addplot[name path=oracle-l, draw=none, fill=none] table[col sep=comma, x=x, y=oracle-l]{Data/region_size_cityscapes.csv};
            \addplot[name path=oracle-u, draw=none, fill=none] table[col sep=comma, x=x, y=oracle-u]{Data/region_size_cityscapes.csv};
            \addplot[cCL, fill opacity=0.15] fill between[of=oracle-l and oracle-u];
            % SP
            \addplot[name path=sp-l, draw=none, fill=none] table[col sep=comma, x=x, y=sp-l]{Data/region_size_cityscapes.csv};
            \addplot[name path=sp-u, draw=none, fill=none] table[col sep=comma, x=x, y=sp-u]{Data/region_size_cityscapes.csv};
            \addplot[cdeepMF, fill opacity=0.15] fill between[of=sp-l and sp-u];
            % S-SP
            \addplot[name path=s-sp-l, draw=none, fill=none] table[col sep=comma, x=x, y=s-sp-l]{Data/region_size_cityscapes.csv};
            \addplot[name path=s-sp-u, draw=none, fill=none] table[col sep=comma, x=x, y=s-sp-u]{Data/region_size_cityscapes.csv};
            \addplot[cMV, fill opacity=0.15] fill between[of=s-sp-l and s-sp-u];
            % AM-SP
            \addplot[name path=am-sp-l, draw=none, fill=none] table[col sep=comma, x=x, y=am-sp-l]{Data/region_size_cityscapes.csv};
            \addplot[name path=am-sp-u, draw=none, fill=none] table[col sep=comma, x=x, y=am-sp-u]{Data/region_size_cityscapes.csv};
            \addplot[cdeepBP, fill opacity=0.15] fill between[of=am-sp-l and am-sp-u];
            
            \legend{Oracle,AMSP+S (Ours),SP+S,SP~\cite{cai2021revisiting}}
            \end{axis}
        % \node[above] at (3.6, 3.34) {\small Performance for varying superpixel size};
        \end{tikzpicture}
        \caption{Cityscapes}
        \label{fig:(c)-effect}
    \end{subfigure}
    \hspace{1mm}    
    \begin{subfigure}{.23\linewidth}
        \centering
        \begin{tikzpicture}
            \begin{axis}[
                legend style={nodes={scale=0.35}, at={(0.03, 0.24)}, anchor=west}, 
                xlabel={Size of base superpixels},
                ylabel={mIoU (\%)},
                width=1.23\linewidth,
                height=1.23\linewidth,
                ymin=53.8,
                ymax=66.6,
                ytick={54, 56, 58, 60, 62, 64, 66},
                xlabel style={yshift=0.15cm},
                % ylabel style={yshift=-0.6cm},
                ylabel style={yshift=-0.2cm},
                legend columns=2,
                xmin=0.7,
                xmax=4.3,
                label style={font=\scriptsize},
                tick label style={font=\scriptsize},
                xtick=data,
                xticklabels={4,16,64,256},
            ]
            \addplot[cdeepMF, very thick, mark=triangle*, mark size=2pt, mark options={solid}] table[col sep=comma, x=x, y=sp] {Data/region_size_pascal.csv};
            \addplot[cMV, very thick, mark=*, mark size=2pt, mark options={solid}] table[col sep=comma, x=x, y=s-sp]{Data/region_size_pascal.csv};
            \addplot[cdeepBP, very thick, mark=diamond*, mark size=2pt, mark options={solid}] table[col sep=comma, x=x, y=am-sp]{Data/region_size_pascal.csv};
            \addplot[cCL, very thick, mark=pentagon*, mark size=2pt, mark options={solid}] table[col sep=comma, x=x, y=oracle]{Data/region_size_pascal.csv};

            % SP
            \addplot[name path=sp-l, draw=none, fill=none] table[col sep=comma, x=x, y=sp-l]{Data/region_size_pascal.csv};
            \addplot[name path=sp-u, draw=none, fill=none] table[col sep=comma, x=x, y=sp-u]{Data/region_size_pascal.csv};
            \addplot[cdeepMF, fill opacity=0.15] fill between[of=sp-l and sp-u];
            % S-SP
            \addplot[name path=s-sp-l, draw=none, fill=none] table[col sep=comma, x=x, y=s-sp-l]{Data/region_size_pascal.csv};
            \addplot[name path=s-sp-u, draw=none, fill=none] table[col sep=comma, x=x, y=s-sp-u]{Data/region_size_pascal.csv};
            \addplot[cMV, fill opacity=0.15] fill between[of=s-sp-l and s-sp-u];
            % AM-SP
            \addplot[name path=am-sp-l, draw=none, fill=none] table[col sep=comma, x=x, y=am-sp-l]{Data/region_size_pascal.csv};
            \addplot[name path=am-sp-u, draw=none, fill=none] table[col sep=comma, x=x, y=am-sp-u]{Data/region_size_pascal.csv};
            \addplot[cdeepBP, fill opacity=0.15] fill between[of=am-sp-l and am-sp-u]; 
            % Oracle
            \addplot[name path=oracle-l, draw=none, fill=none] table[col sep=comma, x=x, y=oracle-l]{Data/region_size_pascal.csv};
            \addplot[name path=oracle-u, draw=none, fill=none] table[col sep=comma, x=x, y=oracle-u]{Data/region_size_pascal.csv};
            \addplot[cCL, fill opacity=0.15] fill between[of=oracle-l and oracle-u];
            % % AM-SP
            % \addplot[name path=am-sp-l, draw=none, fill=none] table[col sep=comma, x=x, y=am-sp-l]{Data/region_size_pascal.csv};
            % \addplot[name path=am-sp-u, draw=none, fill=none] table[col sep=comma, x=x, y=am-sp-u]{Data/region_size_pascal.csv};
            % \addplot[cdeepBP, fill opacity=0.15] fill between[of=am-sp-l and am-sp-u];
            % % S-SP
            % \addplot[name path=s-sp-l, draw=none, fill=none] table[col sep=comma, x=x, y=s-sp-l]{Data/region_size_pascal.csv};
            % \addplot[name path=s-sp-u, draw=none, fill=none] table[col sep=comma, x=x, y=s-sp-u]{Data/region_size_pascal.csv};
            % \addplot[cMV, fill opacity=0.15] fill between[of=s-sp-l and s-sp-u];
            % % S-SP
            % \addplot[name path=s-sp-l, draw=none, fill=none] table[col sep=comma, x=x, y=s-sp-l]{Data/region_size_pascal.csv};
            % \addplot[name path=s-sp-u, draw=none, fill=none] table[col sep=comma, x=x, y=s-sp-u]{Data/region_size_pascal.csv};
            % \addplot[cMV, fill opacity=0.15] fill between[of=s-sp-l and s-sp-u];
            \end{axis}
        \end{tikzpicture}
        \caption{PASCAL}
        \label{fig:(d)-effect}
    \end{subfigure}
    % \caption{{\em Effect of adaptive superpixels.} (a, b) As the round progresses, \textit{AM-SP} shows better performance than \textit{M-SP}. (c, d) \textit{AM-SP} is robust to the size of superpixels, while \textit{SP} is sensitive. Experiments are conducted three times.}
    \caption{{\em Effect of adaptive superpixels.} (a, b) mIoU versus the number of clicks as budget. (c, d) mIoU versus the size of base superpixels. Each experiment is conducted with three trials and the shaded region indicates ranges.}
    % \hsh{(a, b) mIoU versus varying budgets as the amount of clicks. (c, d) mIoU versus the size of superpixels. Each experiment is conducted with three trials and the shaded region indicates ranges.
    \label{fig:robustness}
    % \vspace{-2mm}
\end{figure*}




\iffalse
\smallskip\noindent\textbf{Implementation details.}
We adopt DeepLab-v3+ architecture with Xception-65 \cite{chen2018encoder} as the backbone of our segmentation model.
During training, we use the SGD optimizer with a momentum of 0.9 and set a base learning rate to 7e-3.
We decay the learning rate by polynomial decay with a power of 0.9.
For Cityscapes, we resize training images to 769 $\times$ 769 and train a model for 60k iterations with a batch size of 4.
Similarly, for PASCAL, we resize training images to 513 $\times$ 513 and train a model for 30k iterations with a batch size of 12.
We keep $\epsilon$ at 0.1 for all experiments, except for the one where we adjust it.
\fi

% \begin{figure*}[t!]
%     \captionsetup[subfigure]{font=footnotesize,labelfont=footnotesize,aboveskip=0.05cm,belowskip=-0.15cm}
%     \centering
%     \hspace{-5mm}
%     \begin{subfigure}{.23\linewidth}
%         \centering
%         \begin{tikzpicture}
%             \begin{axis}[
%                 legend style={nodes={scale=0.5}, at={(0.53, 0.19)}, anchor=west}, 
%                 xlabel={Amount of Clicks},
%                 ylabel={mIoU (\%)},
%                 width=1.23\linewidth,
%                 height=1.23\linewidth,
%                 ymin=62.5,
%                 ymax=74.6,
%                 xlabel style={yshift=0.15cm},
%                 ylabel style={yshift=-0.6cm},
%                 legend columns=1,
%                 xmin=90,
%                 xmax=260,
%                 label style={font=\scriptsize},
%                 tick label style={font=\scriptsize},
%             ]
%             % Oracle
%             \addplot[cCL, very thick, mark=pentagon*, mark size=2pt, mark options={solid}] table[col sep=comma, x=x, y=oracle-avg]{Data/limited_budget_cityscapes.csv};
%             % AM-SP
%             \addplot[cdeepBP, very thick, mark=diamond*, mark size=2pt, mark options={solid}] table[col sep=comma, x=x, y=am-sp-avg]{Data/limited_budget_cityscapes.csv};
%             % M-SP
%             \addplot[cBP, very thick, mark=square*, mark size=2pt, mark options={solid}] table[col sep=comma, x=x, y=m-sp-avg]{Data/limited_budget_cityscapes.csv};
%             % SP
%             \addplot[cdeepMF, very thick, mark=triangle*, mark size=2pt, mark options={solid}] table[col sep=comma, x=x, y=revisiting-avg]{Data/limited_budget_cityscapes.csv};
%             % M-SP
%             \addplot[cBP, very thick, mark=square*, mark size=2pt, mark options={solid}] table[col sep=comma, x=x, y=m-sp-avg]{Data/limited_budget_cityscapes.csv};
%             % AM-SP
%             \addplot[cdeepBP, very thick, mark=diamond*, mark size=2pt, mark options={solid}] table[col sep=comma, x=x, y=am-sp-avg]{Data/limited_budget_cityscapes.csv};
            
%             % Oracle
%             \addplot[name path=oracle-l, draw=none, fill=none] table[col sep=comma, x=x, y=oracle-l]{Data/limited_budget_cityscapes.csv};
%             \addplot[name path=oracle-u, draw=none, fill=none] table[col sep=comma, x=x, y=oracle-u]{Data/limited_budget_cityscapes.csv};
%             \addplot[cCL, fill opacity=0.15] fill between[of=oracle-l and oracle-u];
%             % SP
%             \addplot[name path=revisiting-l, draw=none, fill=none] table[col sep=comma, x=x, y=revisiting-l]{Data/limited_budget_cityscapes.csv};
%             \addplot[name path=revisiting-u, draw=none, fill=none] table[col sep=comma, x=x, y=revisiting-u]{Data/limited_budget_cityscapes.csv};
%             \addplot[cdeepMF, fill opacity=0.15] fill between[of=revisiting-l and revisiting-u]; 
%             % M-SP
%             \addplot[name path=m-sp-l, draw=none, fill=none] table[col sep=comma, x=x, y=m-sp-l]{Data/limited_budget_cityscapes.csv};
%             \addplot[name path=m-sp-u, draw=none, fill=none] table[col sep=comma, x=x, y=m-sp-u]{Data/limited_budget_cityscapes.csv};
%             \addplot[cBP, fill opacity=0.15] fill between[of=m-sp-l and m-sp-u]; 
%             % AM-SP
%             \addplot[name path=am-sp-l, draw=none, fill=none] table[col sep=comma, x=x, y=am-sp-l]{Data/limited_budget_cityscapes.csv};
%             \addplot[name path=am-sp-u, draw=none, fill=none] table[col sep=comma, x=x, y=am-sp-u]{Data/limited_budget_cityscapes.csv};
%             \addplot[cdeepBP, fill opacity=0.15] fill between[of=am-sp-l and am-sp-u]; 
            
%             \legend{Oracle,AM-SP,M-SP,SP}
%             \end{axis}
%         \end{tikzpicture}
%         \caption{Cityscapes}
%         \label{fig:(a)-effect}
%     \end{subfigure}
%     \hspace{1mm}    
%     \begin{subfigure}{.23\linewidth}
%         \centering
%         \begin{tikzpicture}
%             \begin{axis}[
%                 legend style={nodes={scale=0.35}, at={(0.03, 0.24)}, anchor=west}, 
%                 xlabel={Amount of Clicks},
%                 ylabel={mIoU (\%)},
%                 width=1.23\linewidth,
%                 height=1.23\linewidth,
%                 ymin=58,
%                 ymax=69.8,
%                 xlabel style={yshift=0.15cm},
%                 ylabel style={yshift=-0.6cm},
%                 legend columns=2,
%                 xmin=9,
%                 xmax=26,
%                 label style={font=\scriptsize},
%                 tick label style={font=\scriptsize},
%             ]
%             % Oracle
%             \addplot[cCL, very thick, mark=pentagon*, mark size=2pt, mark options={solid}] table[col sep=comma, x=x, y=oracle]{Data/limited_budget_pascal.csv};
%             % M-SP
%             \addplot[cBP, very thick, mark=square*, mark size=2pt, mark options={solid}] table[col sep=comma, x=x, y=m-sp]{Data/limited_budget_pascal.csv};
%             % SP
%             \addplot[cdeepMF, very thick, mark=triangle*, mark size=2pt, mark options={solid}] table[col sep=comma, x=x, y=revisiting]{Data/limited_budget_pascal.csv};
%             % M-SP
%             \addplot[cBP, very thick, mark=square*, mark size=2pt, mark options={solid}] table[col sep=comma, x=x, y=m-sp]{Data/limited_budget_pascal.csv};
%             % AM-SP
%             \addplot[cdeepBP, very thick, mark=diamond*, mark size=2pt, mark options={solid}] table[col sep=comma, x=x, y=am-sp]{Data/limited_budget_pascal.csv};
            
%             % Oracle
%             \addplot[name path=oracle-l, draw=none, fill=none] table[col sep=comma, x=x, y=oracle-l]{Data/limited_budget_pascal.csv};
%             \addplot[name path=oracle-u, draw=none, fill=none] table[col sep=comma, x=x, y=oracle-u]{Data/limited_budget_pascal.csv};
%             \addplot[cCL, fill opacity=0.15] fill between[of=oracle-l and oracle-u];
%             % SP
%             \addplot[name path=revisiting-l, draw=none, fill=none] table[col sep=comma, x=x, y=revisiting-l]{Data/limited_budget_pascal.csv};
%             \addplot[name path=revisiting-u, draw=none, fill=none] table[col sep=comma, x=x, y=revisiting-u]{Data/limited_budget_pascal.csv};
%             \addplot[cdeepMF, fill opacity=0.15] fill between[of=revisiting-l and revisiting-u];
%             % M-SP
%             \addplot[name path=m-sp-l, draw=none, fill=none] table[col sep=comma, x=x, y=m-sp-l]{Data/limited_budget_pascal.csv};
%             \addplot[name path=m-sp-u, draw=none, fill=none] table[col sep=comma, x=x, y=m-sp-u]{Data/limited_budget_pascal.csv};
%             \addplot[cBP, fill opacity=0.15] fill between[of=m-sp-l and m-sp-u];            
%             % AM-SP
%             \addplot[name path=am-sp-l, draw=none, fill=none] table[col sep=comma, x=x, y=am-sp-l]{Data/limited_budget_pascal.csv};
%             \addplot[name path=am-sp-u, draw=none, fill=none] table[col sep=comma, x=x, y=am-sp-u]{Data/limited_budget_pascal.csv};
%             \addplot[cdeepBP, fill opacity=0.15] fill between[of=am-sp-l and am-sp-u]; 
            
%             \end{axis}
%         \end{tikzpicture}
%         \caption{PASCAL VOC 2012}
%         \label{fig:(b)-effect}
%     \end{subfigure}
%     \hspace{1mm}
%     \begin{subfigure}{.23\linewidth}
%         \centering
%         \begin{tikzpicture}
%             \begin{axis}[
%                 legend style={nodes={scale=0.5}, at={(0.53, 0.19)}, anchor=west}, 
%                 xlabel={Size of superpixels},
%                 ylabel={mIoU (\%)},
%                 width=1.23\linewidth,
%                 height=1.23\linewidth,
%                 ymin=44,
%                 ymax=69.99,
%                 xlabel style={yshift=0.15cm},
%                 ylabel style={yshift=-0.6cm},
%                 legend columns=1,
%                 xmin=0.7,
%                 xmax=4.3,
%                 label style={font=\scriptsize},
%                 tick label style={font=\scriptsize},
%                 xtick=data,
%                 xticklabels={64,256,1024,4096},
%             ]
%             % Oracle
%             \addplot[cCL, very thick, mark=pentagon*, mark size=2pt, mark options={solid}] table[col sep=comma, x=x, y=oracle]{Data/region_size_cityscapes.csv};
%             % AM-SP
%             \addplot[cdeepBP, very thick, mark=diamond*, mark size=2pt, mark options={solid}] table[col sep=comma, x=x, y=am-sp]{Data/region_size_cityscapes.csv};
%             % S-SP
%             \addplot[cMV, very thick, mark=*, mark size=2pt, mark options={solid}] table[col sep=comma, x=x, y=s-sp]{Data/region_size_cityscapes.csv};
%             % SP
%             \addplot[cdeepMF, very thick, mark=triangle*, mark size=2pt, mark options={solid}] table[col sep=comma, x=x, y=sp] {Data/region_size_cityscapes.csv};
%             % S-SP
%             \addplot[cMV, very thick, mark=*, mark size=2pt, mark options={solid}] table[col sep=comma, x=x, y=s-sp]{Data/region_size_cityscapes.csv};
%             % AM-SP
%             \addplot[cdeepBP, very thick, mark=diamond*, mark size=2pt, mark options={solid}] table[col sep=comma, x=x, y=am-sp]{Data/region_size_cityscapes.csv};

%             % Oracle
%             \addplot[name path=oracle-l, draw=none, fill=none] table[col sep=comma, x=x, y=oracle-l]{Data/region_size_cityscapes.csv};
%             \addplot[name path=oracle-u, draw=none, fill=none] table[col sep=comma, x=x, y=oracle-u]{Data/region_size_cityscapes.csv};
%             \addplot[cCL, fill opacity=0.15] fill between[of=oracle-l and oracle-u];
%             % SP
%             \addplot[name path=sp-l, draw=none, fill=none] table[col sep=comma, x=x, y=sp-l]{Data/region_size_cityscapes.csv};
%             \addplot[name path=sp-u, draw=none, fill=none] table[col sep=comma, x=x, y=sp-u]{Data/region_size_cityscapes.csv};
%             \addplot[cdeepMF, fill opacity=0.15] fill between[of=sp-l and sp-u];
%             % S-SP
%             \addplot[name path=s-sp-l, draw=none, fill=none] table[col sep=comma, x=x, y=s-sp-l]{Data/region_size_cityscapes.csv};
%             \addplot[name path=s-sp-u, draw=none, fill=none] table[col sep=comma, x=x, y=s-sp-u]{Data/region_size_cityscapes.csv};
%             \addplot[cMV, fill opacity=0.15] fill between[of=s-sp-l and s-sp-u];
%             % AM-SP
%             \addplot[name path=am-sp-l, draw=none, fill=none] table[col sep=comma, x=x, y=am-sp-l]{Data/region_size_cityscapes.csv};
%             \addplot[name path=am-sp-u, draw=none, fill=none] table[col sep=comma, x=x, y=am-sp-u]{Data/region_size_cityscapes.csv};
%             \addplot[cdeepBP, fill opacity=0.15] fill between[of=am-sp-l and am-sp-u];
            
%             \legend{Oracle,AM-SP,S-SP,SP}
%             \end{axis}
%         \end{tikzpicture}
%         \caption{Cityscapes}
%         \label{fig:(c)-effect}
%     \end{subfigure}
%     \hspace{1mm}    
%     \begin{subfigure}{.23\linewidth}
%         \centering
%         \begin{tikzpicture}
%             \begin{axis}[
%                 legend style={nodes={scale=0.35}, at={(0.03, 0.24)}, anchor=west}, 
%                 xlabel={Size of superpixels},
%                 ylabel={mIoU (\%)},
%                 width=1.23\linewidth,
%                 height=1.23\linewidth,
%                 ymin=54,
%                 ymax=67,
%                 xlabel style={yshift=0.15cm},
%                 ylabel style={yshift=-0.6cm},
%                 legend columns=2,
%                 xmin=0.7,
%                 xmax=4.3,
%                 label style={font=\scriptsize},
%                 tick label style={font=\scriptsize},
%                 xtick=data,
%                 xticklabels={4,16,64,256},
%             ]
%             \addplot[cdeepMF, very thick, mark=triangle*, mark size=2pt, mark options={solid}] table[col sep=comma, x=x, y=sp] {Data/region_size_pascal.csv};
%             \addplot[cMV, very thick, mark=*, mark size=2pt, mark options={solid}] table[col sep=comma, x=x, y=s-sp]{Data/region_size_pascal.csv};
%             \addplot[cdeepBP, very thick, mark=diamond*, mark size=2pt, mark options={solid}] table[col sep=comma, x=x, y=am-sp]{Data/region_size_pascal.csv};
%             \addplot[cCL, very thick, mark=pentagon*, mark size=2pt, mark options={solid}] table[col sep=comma, x=x, y=oracle]{Data/region_size_pascal.csv};

%             % SP
%             \addplot[name path=sp-l, draw=none, fill=none] table[col sep=comma, x=x, y=sp-l]{Data/region_size_pascal.csv};
%             \addplot[name path=sp-u, draw=none, fill=none] table[col sep=comma, x=x, y=sp-u]{Data/region_size_pascal.csv};
%             \addplot[cdeepMF, fill opacity=0.15] fill between[of=sp-l and sp-u];
%             % S-SP
%             \addplot[name path=s-sp-l, draw=none, fill=none] table[col sep=comma, x=x, y=s-sp-l]{Data/region_size_pascal.csv};
%             \addplot[name path=s-sp-u, draw=none, fill=none] table[col sep=comma, x=x, y=s-sp-u]{Data/region_size_pascal.csv};
%             \addplot[cMV, fill opacity=0.15] fill between[of=s-sp-l and s-sp-u];
%             % AM-SP
%             \addplot[name path=am-sp-l, draw=none, fill=none] table[col sep=comma, x=x, y=am-sp-l]{Data/region_size_pascal.csv};
%             \addplot[name path=am-sp-u, draw=none, fill=none] table[col sep=comma, x=x, y=am-sp-u]{Data/region_size_pascal.csv};
%             \addplot[cdeepBP, fill opacity=0.15] fill between[of=am-sp-l and am-sp-u]; 
%             % Oracle
%             \addplot[name path=oracle-l, draw=none, fill=none] table[col sep=comma, x=x, y=oracle-l]{Data/region_size_pascal.csv};
%             \addplot[name path=oracle-u, draw=none, fill=none] table[col sep=comma, x=x, y=oracle-u]{Data/region_size_pascal.csv};
%             \addplot[cCL, fill opacity=0.15] fill between[of=oracle-l and oracle-u];

%             % % AM-SP
%             % \addplot[name path=am-sp-l, draw=none, fill=none] table[col sep=comma, x=x, y=am-sp-l]{Data/region_size_pascal.csv};
%             % \addplot[name path=am-sp-u, draw=none, fill=none] table[col sep=comma, x=x, y=am-sp-u]{Data/region_size_pascal.csv};
%             % \addplot[cdeepBP, fill opacity=0.15] fill between[of=am-sp-l and am-sp-u];
%             % % S-SP
%             % \addplot[name path=s-sp-l, draw=none, fill=none] table[col sep=comma, x=x, y=s-sp-l]{Data/region_size_pascal.csv};
%             % \addplot[name path=s-sp-u, draw=none, fill=none] table[col sep=comma, x=x, y=s-sp-u]{Data/region_size_pascal.csv};
%             % \addplot[cMV, fill opacity=0.15] fill between[of=s-sp-l and s-sp-u];
            
%             % % S-SP
%             % \addplot[name path=s-sp-l, draw=none, fill=none] table[col sep=comma, x=x, y=s-sp-l]{Data/region_size_pascal.csv};
%             % \addplot[name path=s-sp-u, draw=none, fill=none] table[col sep=comma, x=x, y=s-sp-u]{Data/region_size_pascal.csv};
%             % \addplot[cMV, fill opacity=0.15] fill between[of=s-sp-l and s-sp-u];

            
%             \end{axis}
%         \end{tikzpicture}
%         \caption{PASCAL VOC 2012}
%         \label{fig:(d)-effect}
%     \end{subfigure}
%     \caption{{\em Effect of adaptive superpixels.} (a, b) As the round progresses, \textit{AMSP+S} shows better performance than \textit{MSP+S}. (c, d) \textit{AMSP+S} is robust to the size of superpixels, while \textit{SP} is sensitive. Experiments are conducted three times.}
%     \label{fig:robustness}
%     \vspace{-3mm}
% \end{figure*}


\smallskip\noindent\textbf{Implementation details.}
We adopt DeepLab-v3+ architecture with Xception-65~\cite{chen2018encoder} as our segmentation backbone.
During training, we use the SGD optimizer with a momentum of 0.9 and set a base learning rate to 7e-3.
We decay the learning rate by polynomial decay with a power of 0.9.
For Cityscapes, we resize training images to 769 $\times$ 769 and train a model for 60k iterations with a mini-batch size of 4.
Similarly, for PASCAL, we resize training images to 513 $\times$ 513 and train a model for 30k iterations with a mini-batch size of 12.
% We set $\epsilon$ to 0.1 for all quantitative experiments except for the one where we adjust it.
Unless specified, we set the value of $\epsilon$ to 0.1.

\iffalse
\smallskip\noindent\textbf{Baseline methods.} 
We compare our adaptive merging algorithm to the state-of-the-art in superpixel-based active learning \cite{cai2021revisiting} called \textit{SP}, the abbreviation of superpixel.
Our algorithm comprises two main components: a sieving technique and a merging algorithm.
We refer to the method that uses only the sieving technique as \textit{SP+S}, while the one that employs both sieving and adaptive merging is called \textit{AMSP+S}.
Furthermore, we compare our adaptive merging approach with \textit{MSP+S}, which performs merging only in the second round.
Note that \textit{MSP+S} and \textit{AMSP+S} are identical until round 2.
\fi

\smallskip\noindent\textbf{Baseline methods.} 
We compare our algorithm to \textit{SP} \cite{cai2021revisiting}, the state-of-the-art superpixel-based active segmentation method.
Our algorithm applies two proposed processes in each round: merging and sieving.
We call our complete method including adaptive merging as \textit{AMSP+S}, while the partial version that only uses the sieving without the merging is called \textit{SP+S}.
Additionally, we evaluate the modified version of our method that performs merging only once in the second round, called \textit{MSP+S}.
% Furthermore, we compare our adaptive merging approach with \textit{MSP+S}, which performs merging only in the second round.
% We refer to the method that uses only the sieving technique as \textit{SP+S}, while the one that employs both sieving and adaptive merging is called \textit{AMSP+S}.
% Furthermore, we compare our adaptive merging approach with \textit{MSP+S}, which performs merging only in the second round.
Note that \textit{AMSP+S} and \textit{MSP+S} are identical until the second round.

\iffalse
\smallskip\noindent\textbf{Oracle superpixels.}
To measure the maximum achievable performance of a model, we introduce an oracle superpixel baseline named \textit{Oracle}.
Usually, the performance of a model trained with full pixel-wise annotations of all images is considered the upper bound.
However, the fully supervised baseline is inconsistent with active learning since it disregards budget and round limitations.
Therefore, we suggest an oracle superpixel baseline that leverages ground truth as superpixels illustrated in Figure \ref{fig:(d)_adaptive_merged_superpixels}.
For example, if a car is obstructed by a pole and divided into two parts, it is treated as two superpixels.
The Cityscapes and PASCAL datasets contain 408K and 16k oracle superpixels, respectively.
It is worth noting that the PASCAL has a lower number of oracle superpixels due to the smaller number of classes per image.
For a fair comparison, we select the same number of superpixels per round from the oracle superpixels.
As a round progresses, all oracle superpixels are selected, and our oracle superpixel baseline is equal to the fully supervised baseline.
We report 95\% accuracy for the oracle baseline for comparable comparison.
\fi




\smallskip\noindent\textbf{Oracle baseline.}
The adaptive superpixel aims to merge every connected region with the same class labels.
% \omh{(It might be understood as referring to Ground-truth or Pseudo Label in Table \ref{tab:merging-methods})} 
Thus, the upper bound of it is to consider each region separated by the ground truth mask as a superpixel.
We refer to such ideal regions as oracle superpixels in Figure~\ref{(c)-qualitative}.
An active learning model trained using the oracle superpixels is called \textit{Oracle}.
Details are in Appendix \ref{fig:sup-oracle}.
% A detailed analysis of an oracle baseline is in the appendix.
As the number of oracle superpixels is limited, all of them are eventually labeled as the round progresses, and the performance of the trained model becomes equivalent to that of the pixel-wise fully supervised model.
We report 100\% and 90\% of the \textit{Oracle} performance for Cityscapes and PASCAL, respectively.
% Following the convention of the previous research \cite{cai2021revisiting}, we report 95\% of the \textit{Oracle} performance.


\smallskip\noindent\textbf{Evaluation protocol.}
% Regardless of the resolution of the images,
We set the average size of the superpixels to 256 and 64 pixels on Cityscapes and PASCAL, respectively, for all experiments except for one where we adjust the size.
% In the first round, we randomly select superpixels to train a model, ensuring that all methods, except for the \textit{Oracle}, start at the same point.
% In the following rounds, we adaptively merge superpixels and choose superpixels with the acquisition function in \eqref{acquisition_function}.
Following \textit{SP}~\cite{cai2021revisiting}, we use the number of clicks as the labeling budget.
We conduct 5 rounds of data sampling, where we allocate a budget of 50k and 5k for each round on Cityscapes and PASCAL, respectively.
% We allocate a budget of 50k and 5k for each round on Cityscapes and PASCAL, respectively.
% Once we obtain the dominant labels of the selected superpixels, we sieve the labels.
% We finally train a model with the sieved dataset and evaluate the mean Intersection-over-Union \cite{everingham2015pascal} of the model on the validation images.
In the first round, we randomly select superpixels to train a model, ensuring that all methods start at the same performance.
We evaluate the trained model with mean Intersection-over-Union~\cite{everingham2015pascal} on the validation images.
We emphasize that the average size of superpixels containing 64 pixels is more efficient on Cityscapes, as detailed in Appendix~\ref{sec:base-superpixel-sizes}.
% we use a budget of 50k and 5k for each round on Cityscapes and PASCAL, respectively.
% After training a model with the superpixel selected by an acquisition function, we evaluate the mean Intersection-over-Union \cite{everingham2015pascal} of the model on the validation images.
% \hsh{[TODO] budget, round}



% \begin{figure}[t!]
%     \captionsetup[subfigure]{font=footnotesize,labelfont=footnotesize,aboveskip=0.05cm,belowskip=-0.15cm}
%     \centering
%     \hspace{-5mm}
%     \begin{subfigure}{.47\linewidth}
%         \centering
%         \begin{tikzpicture}
%             \begin{axis}[
%                 legend style={nodes={scale=0.5}, at={(0.03, 0.91)}, anchor=west}, 
%                 xlabel={$\epsilon$},
%                 ylabel={mIoU (\%)},
%                 width=1.23\linewidth,
%                 height=1.23\linewidth,
%                 ymin=62.8,
%                 ymax=68,
%                 xlabel style={yshift=0.15cm},
%                 ylabel style={yshift=-0.6cm},
%                 legend columns=2,
%                 xmin=0.03,
%                 xmax=0.22,
%                 label style={font=\scriptsize},
%                 tick label style={font=\scriptsize},
%                 x tick label style={
%                     /pgf/number format/.cd,
%                         fixed,
%                 }
%             ]
%             \addplot[cBP, thick, mark=pentagon*, mark size=2pt, mark options={solid}] table[col sep=comma, x=x, y=revisiting] {Data/wrong_merging_cityscapes.csv};
%             \addplot[cMV, thick, mark=pentagon*, mark size=2pt, mark options={solid}] table[col sep=comma, x=x, y=ours]{Data/wrong_merging_cityscapes.csv};
%             \legend{SP,M-SP}
%             \end{axis}
%         \end{tikzpicture}
%         \caption{Cityscapes}
%     \end{subfigure}
%     \hspace{1mm}
%     \begin{subfigure}{.47\linewidth}
%         \centering
%         \begin{tikzpicture}
%             \begin{axis}[
%                 legend style={nodes={scale=0.35}, at={(0.03, 64)}, anchor=west}, 
%                 xlabel={$\epsilon$},
%                 ylabel={mIoU (\%)},
%                 width=1.23\linewidth,
%                 height=1.23\linewidth,
%                 ymin=61,
%                 ymax=63.5,
%                 xlabel style={yshift=0.15cm},
%                 ylabel style={yshift=-0.6cm},
%                 legend columns=2,
%                 xmin=0.02,
%                 xmax=0.22,
%                 label style={font=\scriptsize},
%                 tick label style={font=\scriptsize},
%                 x tick label style={
%                     /pgf/number format/.cd,
%                         fixed,
%                 }
%             ]
%             \addplot[cBP, thick, mark=pentagon*, mark size=2pt, mark options={solid}] table[col sep=comma, x=x, y=revisiting] {Data/wrong_merging_pascal.csv};
%             \addplot[cMV, thick, mark=pentagon*, mark size=2pt, mark options={solid}] table[col sep=comma, x=x, y=ours]{Data/wrong_merging_pascal.csv};
%             \end{axis}
%         \end{tikzpicture}
%         \caption{PASCAL VOC 2012}
%     \end{subfigure}
%     \caption{{\em Robustness to $\epsilon$.} As $\epsilon$ increases, the number of noisy labels from wrong merging increases, and accordingly, the performance decreases. Our sieving technique relaxes the noisy label, resulting in robust results.}
%     \label{fig:wrong-merging}
% \end{figure}


\iffalse
\subsection{Robustness to various superpixels}
Creating a superpixel is a crucial step before initiating active learning, and the size of the superpixel is an important hyperparameter to consider. 
Since a smaller size results in fewer labels and a larger size results in lower-quality labels, we balance the tradeoff between quantity and quality of labels.
In this study, we investigate the impact of various superpixel sizes.
Initially, we train a model using 50k and 5k superpixels randomly selected from Cityscapes and PASCAL, respectively, in round 1.
We then experiment with different superpixel sizes in round 2 with the same budget used in round 1.
Figures \ref{fig:(a)-robustness} and \ref{fig:(b)-robustness} demonstrate that our merged superpixels are robust, while the baseline is sensitive to size.
Furthermore, our algorithm generates distinct superpixels depending on the value of $\epsilon$. 
Similar to the superpixel size, a large epsilon increases the number of labels by boosting the average size of the superpixels, while a small epsilon improves the quality of the labels by reducing the number of incorrect merges. 
In Figures \ref{fig:(c)-robustness} and \ref{fig:(d)-robustness}, we observe that our merging approach not only exhibits robustness to $\epsilon$ but also outperforms the baseline over a wide range of $\epsilon$.
\fi
% \subsection{Robustness to Various Superpixels}
% \hsh{The size of superpixel is an essential hyperparameter in conventional superpixel-based active learning, affecting both the quality and quantity of labels, e.g., a decrease in size results in fewer labels, and an increase in size leads to lower quality labels.}
% % The size of superpixels is a crucial hyperparameter in superpixel-based active learning, which impacts both the quality and the quantity of the labels, e.g., a smaller size results in fewer labels, and a larger size results in lower-quality labels.
% % In this study, we investigate the impact of various superpixel sizes.
% \hsh{To evaluate the sensitivity to the choice of the superpixel size, we train models with different superpixel sizes in round 2.
% As shown in Fig. \ref{fig:(a)-robustness} and Fig. \ref{fig:(b)-robustness}, our merged superpixels are robust, while the baseline is sensitive to the size of superpixel.}
% % Initially, we train a model using 50k and 5k superpixels randomly selected from Cityscapes and PASCAL, respectively, in round 1.
% % We then experiment with different superpixel sizes in round 2 with the same budget used in round 1.
% % Figures \ref{fig:(a)-robustness} and \ref{fig:(b)-robustness} demonstrate that our merged superpixels are robust, while the baseline is sensitive to size.

% \hsh{We also evaluate the sensitivity of our method to the choice of $\epsilon$, which determines the amount of the merging process.}
% % Furthermore, our algorithm generates distinct superpixels depending on the value of $\epsilon$. 
% % Similar to the superpixel size, a large epsilon increases the number of labels by boosting the average size of the superpixels, while a small epsilon improves the quality of the labels by reducing the number of incorrect merges. 
% In Fig. \ref{fig:(c)-robustness} and Fig. \ref{fig:(d)-robustness}, we observe that our merging approach not only exhibits robustness to $\epsilon$ but also outperforms the baseline over a wide range of $\epsilon$.

% Especially merged superpixels with a sieving technique are effective in the presence of large superpixels.
% We merge adjacent superpixels into a single superpixel if their features are closer than a specified threshold $\epsilon$.
% Since a model and the feature obtained from the model are inaccurate, we sometimes combine two superpixels with different domain labels.
% This phenomenon occurs frequently, especially as $\epsilon$ is larger, which can lead to performance degradation of the model.
% Figure \ref{fig:wrong-merging} shows
% Appendix, epsilon, superpixel




\subsection{Effect of adaptive superpixels}
\label{sec:effect-of-adaptive}
\iffalse
\smallskip\noindent\textbf{Multi-round scenario.}
Active learning typically allocates its budgets across multiple rounds, with performance gradually enhancing as we identify useful superpixels in each round, guided by the current model.
Given a total budget of 250k and 50k for Cityscapes and PASCAL, respectively, we spend budgets equally over 5 rounds.
In the first round, we train a model by randomly selecting over-segmented superpixels common to all methods, and from the second round, the superpixels used by each method change.
A baseline \textit{SP} continues to use the over-segmented superpixels, our \textit{MSP+S} uses the merged superpixels generated with the model trained in round 1, and our \textit{AMSP+S} uses adaptively merged superpixels in every round.
Here, both our methods applied a sieving technique.
For an upper bound, we compare with \textit{Oracle} utilizing oracle superpixels.
In Figure \ref{fig:multi-rounds}, we observe that our 
\fi

\begin{figure*}[t!]
    % \captionsetup[subfigure]{font=footnotesize}
    \centering
    \begin{subfigure}{.33\linewidth}
        \centering
        \includegraphics[scale=0.322]{Figures/fig4_a_1.png}
        % \caption{$\text{ASA}(S;G)=0.021, \; \text{AF}(G;S)=0.355$}
        % \vspace{2mm}
        % \label{(a)-qualitative}
    \end{subfigure}
    \begin{subfigure}{.33\linewidth}
        \centering
        \includegraphics[scale=0.322]{Figures/fig4_b_1.png}
        % \caption{$\text{ASA}(S;G)=0.89, \; \text{AF}(G;S)=0.283$}
        % \vspace{2mm}
        % \label{(b)-qualitative}
    \end{subfigure}
    \begin{subfigure}{.33\linewidth}
        \centering
        \includegraphics[scale=0.322]{Figures/fig_4_1c_n.png}
        % \caption{$ASA(S;G)=1.00, \; AF(G;S)=1.00$}
        % \vspace{2mm}
        % \label{(c)-qualitative}
    \end{subfigure}
    \begin{subfigure}{.33\linewidth}
        \centering
        \includegraphics[scale=0.322]{Figures/fig4_a_2.png}
        \caption{Base superpixels~\cite{van2012seeds}}
        \label{(a)-qualitative}
    \end{subfigure}
    \begin{subfigure}{.33\linewidth}
        \centering
        \includegraphics[scale=0.322]{Figures/fig4_b_2.png}
        \caption{Merged superpixels (Ours)}
        \label{(b)-qualitative}
    \end{subfigure}
    \begin{subfigure}{.33\linewidth}
        \centering
        \includegraphics[scale=0.322]{Figures/fig_4_2c_n.png}
        % \caption{$ASA(S;G)=1.00, \; AF(G;S)=1.00$}
        \caption{Oracle superpixels}
        \label{(c)-qualitative}
    \end{subfigure}
    % \caption{{\em Qualitative results of adaptive superpixels.} As the round progresses, (a) over-segmented superpixels becomes (b) adaptively merged ones, and they resemble (c) oracle superpixels, especially for the classes that the model is confident about.}
    \caption{{\em Qualitative results of adaptive superpixels.} (a) Base superpixel generated by SEEDS~\cite{van2012seeds} with size 256. (b) Superpixels generated with proposed adaptive merging at round 4. (c) Oracle superpixels generated from the ground truth.}
    \label{fig:qualitative}
    % \vspace{-3mm}
\end{figure*}

% \vspace{-3mm}
\iffalse
\smallskip\noindent\textbf{Multi-round scenario.}
Active learning typically allocates its budgets across multiple rounds, with performance gradually enhancing as we identify useful superpixels in each round.
Following the previous work \cite{cai2021revisiting}, we evaluate the performance during five rounds.
From the second round, the superpixels used by each method change.
A baseline \textit{SP} continues to use the over-segmented superpixels, \textit{MSP+S} uses the merged superpixels generated with the model trained in the first round, and \textit{AMSP+S} uses adaptively merged superpixels in every round.
For an upper bound, we compare with \textit{Oracle} utilizing oracle superpixels.
In Figures~\ref{fig:(a)-effect} and \ref{fig:(b)-effect}, we observe the performance improvement of adaptive merging.
\fi
\smallskip\noindent\textbf{Multi-round scenario.}
In Figures~\ref{fig:(a)-effect} and \ref{fig:(b)-effect}, we compare the performance of the proposed method to \textit{SP}~\cite{cai2021revisiting} varying budget for both of Cityscapes and PASCAL.
Note that the performance for round 0, \ie., 50K budget, is omitted as each method has the same performance at the warm-up round.
The results show that our adaptive superpixel (\textit{AMSP+S}) clearly outperforms the previous art in every budget setting on both of the datasets.
In particular, the \textit{AMSP+S} with only 150k clicks outperforms the previous art with 250k clicks in Cityscapes.
In the final round, the proposed method recovers 97\% and 92\% of the \textit{Oracle} performance for Cityscapes and PASCAL, respectively.
% Active learning typically allocates its budgets across multiple rounds, with performance gradually enhancing as we identify useful superpixels in each round.
% Following the previous work \cite{cai2021revisiting}, we evaluate the performance during five rounds.
% From the second round, the superpixels used by each method change.
% A baseline \textit{SP} continues to use the over-segmented superpixels, \textit{MSP+S} uses the merged superpixels generated with the model trained in the first round, and \textit{AMSP+S} uses adaptively merged superpixels in every round.
% For an upper bound, we compare with \textit{Oracle} utilizing oracle superpixels.
% In Figures~\ref{fig:(a)-effect} and \ref{fig:(b)-effect}, we observe the performance improvement of adaptive merging.
To show the effectiveness of our adaptive approach, we compare \textit{AMSP+S} to its one-shot merging version \textit{MSP+S} in Figures~\ref{fig:(a)-effect} and \ref{fig:(b)-effect}.
On both datasets, adaptive feature of \textit{AMSP+S} shows performance gain especially for the last two rounds.
The experiments conducted for additional rounds can be found in Appendix~\ref{sec:base-superpixel-sizes}.


\iffalse
\begin{table*}[t!]
  \centering
  \setlength\tabcolsep{4pt}
  \begin{tabular}{l|cc|ccc|ccc|c}
    \toprule
    Methods & ASA$(S;G)$ & ASA$(G;S)$ & AP$(S;G)$ & AR$(S;G)$ & AF$(S;G)$ & AP$(G;S)$ & AR$(G;S)$ & AF$(G;S)$ & mIoU \\ 
    \midrule
    % SLIC-512 & 87.8 & 7.70 & 47.7 \\
    % SLIC-2048 & 92.3 & 3.10 & 47.7 \\
    $\text{SLIC}_{4096}$ & 0.885 & 0.101 & 0.897 & 0.058 & 0.082 & 0.700 & 0.260 & 0.195 & 53.18 \\
    $\text{SEEDS}_{4096}$ & 0.908 & 0.099 & 0.904 & 0.063 & 0.085 & 0.666 & 0.307 & 0.229 & 57.61 \\
    % $\text{SLIC}_{1024}$ & 0.935 & ? & 0.941 & 0.025 & 0.037 & 0.559 & 0.435 & 0.269 & - \\
    % $\text{SEEDS}_{1024}$ & 0.948 & ? & 0.946 & 0.026 & 0.038 & 0.539 & 0.484 & 0.297 & - \\
    $\text{SLIC}_{256}$ & 0.965 & 0.016 & 0.967 & 0.009 & 0.015 & 0.391 & 0.639 & 0.287 & 58.04 \\
    % SEEDS-512  & 87.8 & 7.70 & 47.7 & - \\
    % SEEDS-2048 & 93.6 & 3.10 & 48.3 & - \\
    $\text{SEEDS}_{256}$ & 0.969 & 0.018 & 0.969 & 0.009 & 0.015 & 0.390 & 0.660 & 0.304 & 58.97 \\
    % $\text{Merged}_1$ & 0.877 & 0.540 & 0.887 & 0.051 & 0.074 & 0.557 & 0.445 & 0.325 & 55.09(1) \\
    \rowcolor{Gray}
    $\text{Merged}_2$ & 0.879 & 0.547 & 0.885 & 0.052 & 0.076 & 0.558 & 0.452 & 0.333 & \underline{60.00} \\
    \rowcolor{Gray}
    $\text{Merged}_4$ & 0.881 & 0.527 & 0.885 & 0.052 & 0.076 & 0.553 & 0.465 & 0.340 & \textbf{61.36} \\
    % $\text{Merged}_5$ & 0.851 & 0.767 & 0.881 & 0.054 & 0.076 & 0.610 & 0.425 & 0.338 & - \\
    \midrule
    $\text{Merged}^*$ & 0.879 & 0.630 & 0.880 & 0.055 & 0.079 & 0.552 & 0.487 & 0.356 & 61.85 \\
    Oracle & 1.000 & 1.000 & 1.000 & 1.000 & 1.000 & 1.000 & 1.000 & 1.000 & 70.81 \\
    \bottomrule
  \end{tabular}
  % \caption{{\em Evaluation metrics of superpixels.} Superpixels are generated using SLIC \cite{achanta2012slic} and SEEDS \cite{van2012seeds}, with the subscript indicating the average size of superpixels. Our merged superpixels are evaluated, with the subscript value implying the round that used the superpixels and * representing full supervision. To compute the mIoU, we train a model with 100k randomly selected superpixels.}
  \caption{{\em Evaluation metrics of superpixels.}
  % Superpixel quality is measured by various evaluation metrics, where each superpixel is generated from  SLIC~\cite{achanta2012slic}, SEEDS~\cite{van2012seeds}, our adaptive merging (Merged), and the ground-truth (Oracle).
  The subscript indicates the average size of the superpixel for SLIC~\cite{achanta2012slic} and SEEDS~\cite{van2012seeds}, while it indicates the round for Merged.
  $\text{Merged}^*$ indicates superpixel merged by a model trained with full supervision.
  To compute the mIoU, we train a model with 100k randomly selected superpixels.}
  \label{tab:quantitative}
  \vspace{-3mm}
\end{table*}
\fi


\smallskip\noindent\textbf{Multi-size scenario.}
% The size of superpixels is an essential hyperparameter in superpixel-based AL, affecting both the quantity and quality of labels, \eg, a decrease in size results in fewer labels, and an increase in size leads to lower quality labels.
The size of superpixels is an essential hyperparameter in superpixel-based AL, affecting both the quantity and quality of labels.
% In superpixel-based AL, the size of superpixel affects both the quantity and quality of labels, where a decrease in size results in fewer labels, and an increase in size leads to lower quality labels.
In Figures~\ref{fig:(c)-effect} and \ref{fig:(d)-effect}, we compare the proposed method to \textit{SP}~\cite{cai2021revisiting} varying the base superpixel size for both of Cityscapes and PASCAL, in the second round.
Our adaptive superpixel (\textit{AMSP+S}) outperforms the previous art in various superpixel sizes on both of the datasets.
% We also evaluate , which shows that our adaptive merging is especially effective for small superpixels and our sieving is especially effective for large superpixels.
We also evaluate sieving only version (\textit{SP+S}) of our method, which quantifies contribution of each components in our method.
The performance improvement between \textit{SP} and \textit{SP+S} shows our sieving is especially helpful for large superpixels, and the performance gap between \textit{SP+S} and \textit{AMSP+S} shows our merging is especially effective for small superpixels.
Thanks to the proposed sieving and merging, \textit{AMSP+S} are comparably robust to the change of the superpixel size than \textit{SP}.
% To evaluate the sensitivity to the choice of the superpixel size, we train models with different superpixel sizes in the second round.
% Initially, we train a model using 50k and 5k superpixels randomly selected from Cityscapes and PASCAL, respectively, in round 1.
% We then experiment with different superpixel sizes in round 2 with the same budget used in round 1.
% Figures \ref{fig:(a)-robustness} and \ref{fig:(b)-robustness} demonstrate that our merged superpixels are robust, while the baseline is sensitive to size.

\smallskip\noindent\textbf{Qualitative results.}
The quality of the proposed adaptive superpixel is illustrated in Figure~\ref{fig:qualitative}.
As shown in Figure~\ref{(a)-qualitative}, superpixels used in the previous study \cite{cai2021revisiting} have uniform sizes for all areas regardless of their content.
In contrast, Figure~\ref{(b)-qualitative} demonstrates that adaptive superpixels accurately reflect the actual size of the content in images, carefully capturing small object classes while efficiently covering large background classes.
% As the round increases, the quality of the adaptive superpixel increases, and gradually approaches the oracle superpixel depicted in Fig. \ref{fig:(d)_adaptive_merged_superpixels}.
% As shown in Fig. \ref{(b)-adaptive} and \ref{(c)-adaptive}, as the round increases, 
% As shown in Figure 1b-c, as the round increases, the quality of the adaptive superpixel increases, and gradually approaches the oracle superpixel drawn on figure 1d.
More examples are in Appendix \ref{fig:sup-qual}.
% as shown in Fig. \ref{(b)-adaptive} and \ref{(c)-adaptive}, adaptive superpixel reflects the actual size of the content in images, delicately capturing small object classes while efficiently covering large background classes.
% Figure \ref{fig:adaptive_merged_superpixels} shows our adaptive superpixels along with their corresponding oracle superpixels that server as the ground-truth.
% In contrast to previous work \cite{cai2021revisiting}, which employs a constant number of superpixels for all images, our merged superpixels vary in number, depending on the objects within each image.
% As round progresses, the distribution of the number of superpixels becomes increasingly similar to that of the oracle superpixel.
% 
% 
% 
% size vs noise ratio

% A typical metric F1-score is calculated as the harmonic mean of Precision and Recall as:
% \begin{equation}
% \text{F1-score}(S, G) = \frac{2}{\frac{1}{\text{Precision}(S, G)} + \frac{1}{\text{Recall}(S, G)}} \;.
% \end{equation}

% \subsection{Robustness to Region Size}
% Round 1, color, proposed algorithm, transform feature embedding space into color, xy embedding space, 
% Round 2, feature, round 1 fixed

%% Don;t normalize?
%% Scale font size.
% \begin{figure*}[!]
%     \centering
%     \includegraphics[width=\textwidth]{Figures/matching-region-distribution.pdf}
%     \caption{Changing distribution of regions at each cycle of active learning. The region distribution converges to that of ground-truth annotations during training.}
%     \label{fig:matching-region-distribution}
% \end{figure*}


