\appendix

% \setcounter{section}{0}
% \renewcommand{\thesection}{\Alph{section}}

\section*{\fontsize{14pt}{\baselineskip}\selectfont Appendix}
\vspace{3mm}

% \section{Notations}
% The notations used in the paper are defined in Table \ref{tab:notations}, where the subscript $t$ corresponds to round $t$.
% {
% \hypersetup{linkcolor=red}
% \tableofcontents
% \setcounter{tocdepth}{0}
% }

\section{
More 
gain with other base superpixel sizes
}
\label{sec:base-superpixel-sizes}


\begin{figure}[!ht]
    \captionsetup[subfigure]{font=footnotesize,labelfont=footnotesize,aboveskip=0.05cm,belowskip=-0.15cm}
    \centering
    \hspace{-5mm}
    \begin{subfigure}{.49\linewidth}
        \centering
        \begin{tikzpicture}
            \begin{axis}[
                % legend style={nodes={scale=0.5}, at={(0.5, 0.19)}, anchor=west}, 
                legend style={nodes={scale=0.6}, at={(1.65, 1.16)}},
                legend columns=-1,
                xlabel={The number of clicks},
                ylabel={mIoU (\%)},
                width=1.23\linewidth,
                height=1.23\linewidth,
                ymin=59.,
                ymax=71,
                ytick={58, 60, 62, 64, 66, 68, 70, 72, 74},
                xlabel style={yshift=0.15cm},
                % ylabel style={yshift=-0.6cm},
                ylabel style={yshift=-0.2cm},
                % legend columns=1,
                xmin=90,
                xmax=210,
                label style={font=\scriptsize},
                tick label style={font=\scriptsize},
                xticklabel={$\pgfmathprintnumber{\tick}$k}
            ]
            % AM-SP
            \addplot[cdeepBP, very thick, mark=diamond*, mark size=2pt, mark options={solid}] table[col sep=comma, x=x, y=am-sp]{Data/limited_budget_cityscapes_64.csv};
            % SP
            \addplot[cdeepMF, very thick, mark=triangle*, mark size=2pt, mark options={solid}] table[col sep=comma, x=x, y=revisiting]{Data/limited_budget_cityscapes_64.csv};
            
            % SP
            % \addplot[name path=revisiting-l, draw=none, fill=none] table[col sep=comma, x=x, y=revisiting-l]{Data/limited_budget_cityscapes.csv};
            % \addplot[name path=revisiting-u, draw=none, fill=none] table[col sep=comma, x=x, y=revisiting-u]{Data/limited_budget_cityscapes.csv};
            % \addplot[cdeepMF, fill opacity=0.15] fill between[of=revisiting-l and revisiting-u];
            % AM-SP
            % \addplot[name path=am-sp-l, draw=none, fill=none] table[col sep=comma, x=x, y=am-sp-l]{Data/limited_budget_cityscapes.csv};
            % \addplot[name path=am-sp-u, draw=none, fill=none] table[col sep=comma, x=x, y=am-sp-u]{Data/limited_budget_cityscapes.csv};
            % \addplot[cdeepBP, fill opacity=0.15] fill between[of=am-sp-l and am-sp-u]; 
            
            \legend{AMSP+S (Ours), SP~\cite{cai2021revisiting}}
            \end{axis}
        % \node[above] at (3.7, 3.41) {\small Performance for varying budget};
        \end{tikzpicture}
        \caption{Base superpixel size of 64}
        \label{fig:(a)-small-region}
    \end{subfigure}
    \hspace{1mm}
    \begin{subfigure}{.49\linewidth}
        \centering
        \begin{tikzpicture}
            \begin{axis}[
                % legend style={nodes={scale=0.5}, at={(0.5, 0.19)}, anchor=west}, 
                legend style={nodes={scale=0.6}, at={(1.65, 1.16)}},
                legend columns=-1,
                xlabel={The number of clicks},
                ylabel={mIoU (\%)},
                width=1.23\linewidth,
                height=1.23\linewidth,
                ymin=59,
                ymax=71,
                ytick={60, 62, 64, 66, 68, 70, 72, 74},
                xlabel style={yshift=0.15cm},
                % ylabel style={yshift=-0.6cm},
                ylabel style={yshift=-0.2cm},
                % legend columns=1,
                xmin=90,
                xmax=210,
                label style={font=\scriptsize},
                tick label style={font=\scriptsize},
                xticklabel={$\pgfmathprintnumber{\tick}$k}
            ]
            % Oracle
            % \addplot[cCL, very thick, mark=pentagon*, mark size=2pt, mark options={solid}] table[col sep=comma, x=x, y=oracle-avg]{Data/limited_budget_cityscapes.csv};
            % AM-SP
            \addplot[cdeepBP, very thick, mark=diamond*, mark size=2pt, mark options={solid}] table[col sep=comma, x=x, y=am-sp]{Data/limited_budget_cityscapes_256.csv};
            % M-SP
            % \addplot[cBP, very thick, mark=square*, mark size=2pt, mark options={solid}] table[col sep=comma, x=x, y=m-sp-avg]{Data/limited_budget_cityscapes.csv};
            % SP
            \addplot[cdeepMF, very thick, mark=triangle*, mark size=2pt, mark options={solid}] table[col sep=comma, x=x, y=revisiting]{Data/limited_budget_cityscapes_256.csv};
            % M-SP
            % \addplot[cBP, very thick, mark=square*, mark size=2pt, mark options={solid}] table[col sep=comma, x=x, y=m-sp-avg]{Data/limited_budget_cityscapes.csv};
            % AM-SP
            % \addplot[cdeepBP, very thick, mark=diamond*, mark size=2pt, mark options={solid}] table[col sep=comma, x=x, y=am-sp-avg]{Data/limited_budget_cityscapes.csv};
            
            % Oracle
            % \addplot[name path=oracle-l, draw=none, fill=none] table[col sep=comma, x=x, y=oracle-l]{Data/limited_budget_cityscapes.csv};
            % \addplot[name path=oracle-u, draw=none, fill=none] table[col sep=comma, x=x, y=oracle-u]{Data/limited_budget_cityscapes.csv};
            % \addplot[cCL, fill opacity=0.15] fill between[of=oracle-l and oracle-u];
            % SP
            % \addplot[name path=revisiting-l, draw=none, fill=none] table[col sep=comma, x=x, y=revisiting-l]{Data/limited_budget_cityscapes.csv};
            % \addplot[name path=revisiting-u, draw=none, fill=none] table[col sep=comma, x=x, y=revisiting-u]{Data/limited_budget_cityscapes.csv};
            % \addplot[cdeepMF, fill opacity=0.15] fill between[of=revisiting-l and revisiting-u]; 
            % M-SP
            % \addplot[name path=m-sp-l, draw=none, fill=none] table[col sep=comma, x=x, y=m-sp-l]{Data/limited_budget_cityscapes.csv};
            % \addplot[name path=m-sp-u, draw=none, fill=none] table[col sep=comma, x=x, y=m-sp-u]{Data/limited_budget_cityscapes.csv};
            % \addplot[cBP, fill opacity=0.15] fill between[of=m-sp-l and m-sp-u]; 
            % AM-SP
            % \addplot[name path=am-sp-l, draw=none, fill=none] table[col sep=comma, x=x, y=am-sp-l]{Data/limited_budget_cityscapes.csv};
            % \addplot[name path=am-sp-u, draw=none, fill=none] table[col sep=comma, x=x, y=am-sp-u]{Data/limited_budget_cityscapes.csv};
            % \addplot[cdeepBP, fill opacity=0.15] fill between[of=am-sp-l and am-sp-u]; 
            
            % \legend{AMSP+S (Ours), SP~\cite{cai2021revisiting}}
            \end{axis}
        % \node[above] at (3.7, 3.41) {\small Performance for varying budget};
        \end{tikzpicture}
        \caption{Base superpixel size of 256}
        \label{fig:(b)-small-region}
    \end{subfigure}
    \caption{{\em Effect of base superpixel size on Cityscapes.} The performance difference is greater when the superpixel size is smaller.
    % Each experiment is conducted with three trials and the shaded region indicates range
    }
    \label{fig:supple-region-size-cityscapes}
\end{figure}
\begin{figure}[!ht]
    \captionsetup[subfigure]{font=footnotesize,labelfont=footnotesize,aboveskip=0.05cm,belowskip=-0.15cm}
    \centering
    \hspace{-5mm}
    \begin{subfigure}{.49\linewidth}
        \centering
        \begin{tikzpicture}
            \begin{axis}[
                % legend style={nodes={scale=0.35}, at={(0.03, 0.24)}, anchor=west},
                legend style={nodes={scale=0.6}, at={(1.02, 1.16)}},
                xlabel={Size of base superpixels},
                ylabel={mIoU (\%)},
                width=1.23\linewidth,
                height=1.23\linewidth,
                ymin=54.5,
                ymax=63.5,
                ytick={54, 56, 58, 60, 62, 64, 66},
                xlabel style={yshift=0.15cm},
                % ylabel style={yshift=-0.6cm},
                ylabel style={yshift=-0.2cm},
                legend columns=2,
                xmin=0.7,
                xmax=6.3,
                label style={font=\scriptsize},
                tick label style={font=\scriptsize},
                xtick=data,
                xticklabels={4,16,64,256,1024,4096},
            ]
            \addplot[cdeepBP, very thick, mark=diamond*, mark size=2pt, mark options={solid}] table[col sep=comma, x=x, y=sm]{Data/supple_region_size_pascal.csv};
            \addplot[cdeepMF, very thick, mark=triangle*, mark size=2pt, mark options={solid}] table[col sep=comma, x=x, y=revisiting] {Data/supple_region_size_pascal.csv};
            % \addplot[cMV, very thick, mark=*, mark size=2pt, mark options={solid}] table[col sep=comma, x=x, y=s-sp]{Data/region_size_pascal.csv};
            % \addplot[cCL, very thick, mark=pentagon*, mark size=2pt, mark options={solid}] table[col sep=comma, x=x, y=oracle]{Data/region_size_pascal.csv};

            % AM-SP
            % \addplot[name path=am-sp-l, draw=none, fill=none] table[col sep=comma, x=x, y=sm-d]{Data/supple_region_size_pascal.csv};
            % \addplot[name path=am-sp-u, draw=none, fill=none] table[col sep=comma, x=x, y=sm-u]{Data/supple_region_size_pascal.csv};
            % \addplot[cdeepBP, fill opacity=0.15] fill between[of=am-sp-l and am-sp-u]; 

            % SP
            % \addplot[name path=sp-l, draw=none, fill=none] table[col sep=comma, x=x, y=revisiting-d]{Data/supple_region_size_pascal.csv};
            % \addplot[name path=sp-u, draw=none, fill=none] table[col sep=comma, x=x, y=revisiting-u]{Data/supple_region_size_pascal.csv};
            % \addplot[cdeepMF, fill opacity=0.15] fill between[of=sp-l and sp-u];
            % S-SP
            % \addplot[name path=s-sp-l, draw=none, fill=none] table[col sep=comma, x=x, y=s-sp-l]{Data/region_size_pascal.csv};
            % \addplot[name path=s-sp-u, draw=none, fill=none] table[col sep=comma, x=x, y=s-sp-u]{Data/region_size_pascal.csv};
            % \addplot[cMV, fill opacity=0.15] fill between[of=s-sp-l and s-sp-u];
            
            % Oracle
            % \addplot[name path=oracle-l, draw=none, fill=none] table[col sep=comma, x=x, y=oracle-l]{Data/region_size_pascal.csv};
            % \addplot[name path=oracle-u, draw=none, fill=none] table[col sep=comma, x=x, y=oracle-u]{Data/region_size_pascal.csv};
            % \addplot[cCL, fill opacity=0.15] fill between[of=oracle-l and oracle-u];
            % % AM-SP
            % \addplot[name path=am-sp-l, draw=none, fill=none] table[col sep=comma, x=x, y=am-sp-l]{Data/region_size_pascal.csv};
            % \addplot[name path=am-sp-u, draw=none, fill=none] table[col sep=comma, x=x, y=am-sp-u]{Data/region_size_pascal.csv};
            % \addplot[cdeepBP, fill opacity=0.15] fill between[of=am-sp-l and am-sp-u];
            % % S-SP
            % \addplot[name path=s-sp-l, draw=none, fill=none] table[col sep=comma, x=x, y=s-sp-l]{Data/region_size_pascal.csv};
            % \addplot[name path=s-sp-u, draw=none, fill=none] table[col sep=comma, x=x, y=s-sp-u]{Data/region_size_pascal.csv};
            % \addplot[cMV, fill opacity=0.15] fill between[of=s-sp-l and s-sp-u];
            % % S-SP
            % \addplot[name path=s-sp-l, draw=none, fill=none] table[col sep=comma, x=x, y=s-sp-l]{Data/region_size_pascal.csv};
            % \addplot[name path=s-sp-u, draw=none, fill=none] table[col sep=comma, x=x, y=s-sp-u]{Data/region_size_pascal.csv};
            % \addplot[cMV, fill opacity=0.15] fill between[of=s-sp-l and s-sp-u];
            \legend{AMSP+S (Ours), SP~\cite{cai2021revisiting}}
            \end{axis}
        \end{tikzpicture}
        \caption{PASCAL}
    \end{subfigure}
    \caption{{\em Effect of base superpixel size on PASCAL.} Our method exhibits robustness to large superpixels, while the baseline is sensitive. 
    % Each experiment is conducted with three trials and the shaded region indicates range
    }
    \label{fig:supple-region-size-pascal}
\end{figure}




For ease of exposition, Figure~\ref{fig:robustness} presents the gain of our method (compared to \textit{SP} \cite{cai2021revisiting}) for a limited set of base superpixel sizes. In this section, we report an additional investigation
suggesting further gain with different base superpixels.


% \label{fig:sup-small-cityscapes}
\smallskip\noindent\textbf{Further gain on Cityscapes.}
In Figure~~\ref{fig:(a)-small-region},
we additionally provide a comparison between
the proposed method (\textit{AMSP+S})
and \textit{SP} \cite{cai2021revisiting}, 
where the experiment setup with
Cityscapes
is identical to 
that in Figure~\ref{fig:(a)-effect} except that the base superpixel size is 64 (Figure~~\ref{fig:(a)-small-region}) instead of 256 (Figure~~\ref{fig:(b)-small-region}).
Our adaptive merging method (\textit{AMSP+S}) is especially effective when the superpixel size is small in Figure~\ref{fig:(a)-small-region}, thanks to the adaptive merging mechanism.
This observation suggests more significant gain of our method with other choices of base superpixel size than that in Figure~\ref{fig:robustness}.


% The experimental setup used in Figure \ref{fig:(b)-small-region} is identical to that of Figure~\ref{fig:(a)-effect}. 

\begin{figure}[!ht]
    \centering
    \begin{tikzpicture}
        \begin{axis}[
            % legend style={nodes={scale=0.5}, at={(0.5, 0.19)}, anchor=west}, 
            legend style={nodes={scale=0.57}, at={(0.925, 1.2)}},
            legend columns=-1,
            xlabel={The number of clicks},
            ylabel={mIoU (\%)},
            width=1.05\linewidth,
            height=0.5\linewidth,
            ymin=62.5,
            ymax=73.2,
            ytick={64, 66, 68, 70, 72, 74},
            xlabel style={yshift=0.15cm},
            % ylabel style={yshift=-0.6cm},
            ylabel style={yshift=-0.2cm},
            xmin=90,
            xmax=410,
            label style={font=\scriptsize},
            tick label style={font=\scriptsize},
            xticklabel={$\pgfmathprintnumber{\tick}$k}
        ]
        % 95%
        \addplot[gray, very thick] table[col sep=comma, x=x, y=sup]{Data/Rebuttal/limited_budget_cityscapes_rebuttal.csv};
        % 95%
        \addplot[cdeepBP, very thick, mark=diamond*, mark size=2pt, mark options={solid}] table[col sep=comma, x=x, y=am-sp]{Data/Rebuttal/limited_budget_cityscapes_rebuttal.csv};
        % SP
        \addplot[cdeepMF, very thick, mark=triangle*, mark size=2pt, mark options={solid}] table[col sep=comma, x=x, y=revisiting]{Data/Rebuttal/limited_budget_cityscapes_rebuttal.csv};
        \legend{95\% Fully-supervised, AMSP+S (Ours), SP \cite{cai2021revisiting}}
        \end{axis}
    \end{tikzpicture}
    % \vspace{-2mm}
    \caption{{\em Additional rounds experiments on Cityscapes.} We extend the experiments in Figure~\ref{fig:(a)-effect} up to a budget of 400k. The performance improvement remains consistent across various additional budgets.}
    \label{fig:95}
\end{figure}

\begin{table}[!ht]
\centering
\setlength\tabcolsep{6pt}
\begin{tabular}{l|c}
\toprule
Methods & mIoU \\ \midrule
\textit{SP} \cite{cai2021revisiting} & 63.77 \\ \midrule
\textit{AMSP+S} (bottom 10\%) & 64.33 \\ \midrule
\textit{AMSP+S} (top 10\%) & \underline{65.99} \\ \midrule
\rowcolor{Gray}
\textit{AMSP+S} (complete 100\%) & \textbf{66.53} \\ \midrule
\bottomrule
\end{tabular}
\caption{{\em Various levels of partial merging.} Experiments are conducted under the same setting of Figure~\ref{fig:(a)-effect} with 100k clicks (Cityscapes, superpixel size of 256).}
\label{tab:sup-descending}
\end{table}

\iffalse
\begin{table}[!ht]
\centering
\setlength\tabcolsep{6pt}
\begin{tabular}{l|c}
\toprule
Methods & mIoU \\ \midrule
\textit{SP} \cite{cai2021revisiting} & 63.77 \\ \midrule
\textit{AMSP+S} (ascending, 10\%) & 64.33 \\ \midrule
\textit{AMSP+S} (descending, 10\%) & \underline{65.99} \\ \midrule
\rowcolor{Gray}
\textit{AMSP+S} (descending, 100\%) & \textbf{66.53} \\ \midrule
\bottomrule
\end{tabular}
\caption{{\em Various merging order.} Experiments are conducted on Cityscapes dataset with an average superpixel size of 256, using 100k costs for two rounds.}
\label{tab:descending}
\end{table}
\fi

\smallskip\noindent\textbf{Further gain on PASCAL.}
We also demonstrate a larger gap between 
the proposed method and existing one 
in PASCAL. 
% explore the effectiveness of the proposed framework by altering the base superpixel size on PASCAL.
In Figure~\ref{fig:supple-region-size-pascal}, our adaptive merging method (\textit{AMSP+S}) outperforms the baseline (\textit{SP}) for various superpixel sizes
as we observed in Figure~\ref{fig:robustness}.
We stress that the gain of the proposed method is particularly larger than the one reported in Figure~\ref{fig:robustness} when
the base superpixel size is 4096, which is much larger than 256 used in Figure~\ref{fig:robustness}. This is because 
the sieving procedure to 
suppresses
the noise from dominant labeling 
becomes more crucial when querying large superpixels.
The experimental setup used in Figure \ref{fig:supple-region-size-pascal} is identical to that of Figure \ref{fig:(d)-effect}. 

%\khy{
\smallskip\noindent\textbf{Further rounds on Cityscapes.}
To demonstrate the efficacy of our method across various budgets, we experiment by gradually increasing the budget as illustrated in Figure~\ref{fig:95} on Cityscapes.
The experimental setting in Figure~\ref{fig:95} remains consistent with that of Figure~\ref{fig:(a)-effect}.
The advantage of our method over SP~\cite{cai2021revisiting} is continued in further rounds.
We remark that the proposed method nearly achieves the 95\% mIoU of the fully
supervised model (71.95\%) at 300k clicks, whereas SP does at 400k clicks.
%}

% \label{fig:sup-descend}

\iffalse
\begin{figure*}[!ht]
    \captionsetup[subfigure]{font=footnotesize}
    \centering
    \begin{subfigure}{.33\linewidth}
        \centering
        \includegraphics[scale=0.322]{Figures/fig9_ascend/sfig_9_1a.png}
    \end{subfigure}
    \begin{subfigure}{.33\linewidth}
        \centering
        \includegraphics[scale=0.322]{Figures/fig9_ascend/sfig_9_1b.png}
    \end{subfigure}
    \begin{subfigure}{.33\linewidth}
        \centering
        \includegraphics[scale=0.322]{Figures/fig9_ascend/sfig_9_1c.png}
        % \caption{$ASA(S;G)=1.00, \; AF(G;S)=1.00$}
    \end{subfigure}

    \begin{subfigure}{.33\linewidth}
        \centering
        \includegraphics[scale=0.322]{Figures/fig9_ascend/sfig_9_2a.png}
        \caption{Merging superpixels with low 10\% uncertainty}
    \end{subfigure}
    \begin{subfigure}{.33\linewidth}
        \centering
        \includegraphics[scale=0.322]{Figures/fig9_ascend/sfig_9_2b.png}
        \caption{Merging superpixels with high 10\% uncertainty}
    \end{subfigure}
    \begin{subfigure}{.33\linewidth}
        \centering
        \includegraphics[scale=0.322]{Figures/fig9_ascend/sfig_9_2c.png}
        % \caption{$ASA(S;G)=1.00, \; AF(G;S)=1.00$}
        \caption{Merging all superpixels}
    \end{subfigure}
    \caption{{\em Qualitative results for partial merging.} The average uncertainty of pixels in the cyan box is 0.30 for the top images and 0.27 for the bottom images, while for pixels in the red box, it is 0.09 for the top images and 0.00 for the bottom images. 
    % High acquisition values are observed within the cyan boxes, while the red boxes hold low acquisition values.
    By merging only a portion of superpixels in the order of high uncertainty, we can reduce time complexity, as it creates merged superpixels (\ie in the cyan boxes) that will be selected by the acquisition function.
    % (a) Only 10 \% of superpixels are merged in the order of low uncertainty. 
    % (b) Only 10 \% of superpixels are merged in the order of high uncertainty. 
    % (c) For merging, our adaptive merging method explores all superpixels in an image.
    }
    \label{fig:descend}
    \vspace{-3mm}
\end{figure*}
\fi

\section{Rationale for line~\ref{line:order} of Algorithm \ref{algorithm2}}
\label{sec:rationale-merging}

\begin{figure*}[!t]
    \captionsetup[subfigure]{font=footnotesize}
    \centering
    \begin{subfigure}{.33\linewidth}
        \centering
        \includegraphics[scale=0.322]{Figures/fig9_ascend/sfig_9_1a.png}
    \end{subfigure}
    \begin{subfigure}{.33\linewidth}
        \centering
        \includegraphics[scale=0.322]{Figures/fig9_ascend/sfig_9_1b.png}
    \end{subfigure}
    \begin{subfigure}{.33\linewidth}
        \centering
        \includegraphics[scale=0.322]{Figures/fig9_ascend/sfig_9_1c.png}
        % \caption{$ASA(S;G)=1.00, \; AF(G;S)=1.00$}
    \end{subfigure}

    \begin{subfigure}{.33\linewidth}
        \centering
        \includegraphics[scale=0.322]{Figures/fig9_ascend/sfig_9_2a.png}
        \caption{Merging superpixels with low 10\% uncertainty}
        \label{fig:(a)-partial}
    \end{subfigure}
    \begin{subfigure}{.33\linewidth}
        \centering
        \includegraphics[scale=0.322]{Figures/fig9_ascend/sfig_9_2b.png}
        \caption{Merging superpixels with high 10\% uncertainty}
        \label{fig:(b)-partial}
    \end{subfigure}
    \begin{subfigure}{.33\linewidth}
        \centering
        \includegraphics[scale=0.322]{Figures/fig9_ascend/sfig_9_2c.png}
        % \caption{$ASA(S;G)=1.00, \; AF(G;S)=1.00$}
        \caption{Merging all superpixels}
        \label{fig:(c)-partial}
    \end{subfigure}
    \caption{{\em Qualitative results for partial merging.} 
    % The average uncertainty of pixels in the cyan box is 0.30 for the top images and 0.27 for the bottom images, while for pixels in the red box, it is 0.09 for the top images and 0.00 for the bottom images. 
    % High acquisition values are observed within the cyan boxes, while the red boxes hold low acquisition values.
    The cyan boxes encompass superpixels exhibiting the highest 10\% uncertainty, while the red boxes encompass superpixels with the lowest 10\% uncertainty.
    (b) By merging only a portion of superpixels in the order of high uncertainty, we can reduce time complexity, as it creates similar merged superpixels compared with the cyan box in (c).
    % (\ie in the cyan boxes of (b) and (c)) with high acquisition values. 
    % (a) Only 10 \% of superpixels are merged in the order of low uncertainty. 
    % (b) Only 10 \% of superpixels are merged in the order of high uncertainty. 
    % (c) For merging, our adaptive merging method explores all superpixels in an image.
    }
    \label{fig:descend}
\end{figure*}







\begin{table}[!ht]
\centering
\setlength\tabcolsep{6pt}
\begin{tabular}{l|c}
\toprule
Methods & mIoU \\ \midrule
\textit{SP} \cite{cai2021revisiting} & 63.77 \\ \midrule
\textit{AMSP+S} $(\phi(s;\theta) = 0.0)$ & \underline{65.35} \\ \midrule
\textit{AMSP+S} $(\phi(s;\theta) = 0.2)$ & 61.80 \\ \midrule
\textit{AMSP+S} $(\phi(s;\theta) = 0.4)$ & 57.77 \\ \midrule
\textit{AMSP+S} $(\phi(s;\theta) = 0.6)$ & 45.84 \\ \midrule
\textit{AMSP+S} $(\phi(s;\theta) = 0.8)$ & 38.99 \\ \midrule
% Class-wise & 00.00 \\ \midrule
\rowcolor{Gray}
\textit{AMSP+S} (Kneedle \cite{satopaa2011finding}) & \textbf{66.53} \\ \midrule
\bottomrule
\end{tabular}
\caption{{\em Various sieving methods.} Experiments are conducted on Cityscapes dataset with an average superpixel size of 256, using 100k costs for two rounds.}
\label{tab:sieving}
\end{table}


% We utilize a graph to merge superpixels by converting them into nodes and edges.
% However,
We explain the rationale behind traversing nodes in the descending order of uncertainty in line~\ref{line:order} of Algorithm \ref{algorithm2}.
Our merging process requires a linear time complexity proportional to the size of the base superpixels graph.
However, due to the advantage of merging in descending uncertainty order, we are able to acquire merged superpixels with considerable uncertainty at the beginning of merging.
To reduce merging time complexity, we only merge the top $10\%$ of base superpixels with the highest uncertainty as query candidates.
Table~\ref{tab:sup-descending} shows that it is important to prioritize the merging highly uncertain superpixels,
and
merging along the ascending order of uncertainty degenerates the performance.

In Figure~\ref{fig:descend},
we exemplify the merged superpixels
from the partial merging in the ascending or descending order of uncertainty,
and the full merging, where
the cyan boxes contain
higher values of acquisition function
than the red boxes.
The partial merging with the ascending order of uncertainty regrettably merges
the superpixels that would not be selected in AL, while that with the ascending order
efficiently combines the base superpixels
of which selection is highly like.
This difference indeed results in a huge gap in the final performance as shown in Table~\ref{tab:sup-descending}.
% However, thanks to the descending order of uncertainty and the high uncertainty superpixels prioritization of the acquisition function, we can interrupt merging process as we already obtain
% We first convert superpixels into a graph for merging, where
% \khy{
% In order to complete the merging process for an image, we require a linear time complexity in the size of the base superpixels graph.
% However, this complexity can be significantly reduced by employing a subgraph comprising solely of base superpixels with high uncertainties.
% }
% the square of the number of superpixels $|S|$ as we compare pairs of superpixels to build an adjacency matrix, \ie $O(|S|^2)$.
% For experiments, we divide a Cityscapes image into 8,192 superpixels, however, the number of superpixels can vary depending on the image resolution.
% Based on prioritizing superpixels with high uncertainty in the acquisition function, we propose a technique that firstly merges high-uncertainty superpixels.
% To alleviate this dependency, we only merge high-uncertainty superpixels.
% Note that the acquisition function prioritizes superpixels with high uncertainty.
% we note that the acquisition function prioritizes superpixels with high uncertainty, and only merging these superpixels can be enough. 
% as the superpixels with low uncertainty are unnecessary to the model.

% \khy{
% Given a budget $b$ for a round, we only convert $\frac{b}{N}$ nodes, the average number of superpixels for $N$ images, into a graph, where the complexity reduces to $O(\frac{b^2}{N^2})$.
% We convert $\frac{b}{N}$ superpixels, instead of $|S|$, into a graph,
% where $b$ is a budget and $N$ is the total number of images.
% The complexity reduces to $O(\frac{b^2}{N^2})$, where $|S| \gg b$ in active learning.
% }

% Details are in the appendix.
% To reduce the computational complexity,
% it is possible to merge a {\it part} of 
% base superpixels of highest uncertainty (say top $10\%$)
% to generate query candidates, although the proposed method investigates
% {\it all} the base superpixels in the descending order of uncertainty.
% although we propose 
% Instead of exploring all nodes in a graph, we only inspect nodes in order of high uncertainty to reduce the time complexity of graph preprocessing.
%To compare our algorithm which explores all nodes in descending order of uncertainty, 
% Table~\ref{tab:descending}
% shows the partial merging in the descending order of uncertainty results in only a small gap to the full merging.
% This suggests a tip to save the computation resource for practitioners.
% \khy{
% Upon closer examination, we find that by employing the base superpixels from the upper 10\% uncertainty, the time complexity decreases significantly by a factor of 25.98, resulting in a remarkable reduction from 12.42 seconds to 0.48 seconds per image.
% }

% we examine 10\% of nodes in ascending and descending order, respectively.
% Table~\ref{tab:descending} shows that merging only for high-uncertainty nodes achieves comparable performance to our approach.
% In Figure~\ref{fig:descend}, the superpixels used for each method are showed.


\section{Rationale for the adaptive threshold $\phi(s;\theta)$ in the sieving}
\label{fig:sup-sieving}
\begin{figure}[!t]
    \captionsetup[subfigure]{font=footnotesize,labelfont=footnotesize,aboveskip=0.05cm,belowskip=-0.15cm}
    \centering
    \hspace{-3mm}
    \begin{subfigure}{.47\linewidth}
        \centering
        \begin{tikzpicture}
            \begin{axis}[
                % legend style={nodes={scale=0.5}, at={(0.5, 0.19)}, anchor=west}, 
                legend style={nodes={scale=0.6}, at={(1.55, 1.16)}},
                legend columns=-1,
                xlabel={$x$},
                ylabel={$f_\theta(\text{road};x)$},
                width=1.23\linewidth,
                height=1.23\linewidth,
                ymin=-0.1,
                ymax=1.1,
                % ytick={64, 66, 68, 70, 72, 74},
                xlabel style={yshift=0.15cm},
                % ylabel style={yshift=-0.6cm},
                ylabel style={yshift=-0.2cm},
                % legend columns=1,
                xmin=-0.5,
                xmax=18.5,
                label style={font=\scriptsize},
                tick label style={font=\scriptsize},
                xtick=data
                %xticklabel={1,2,3,4,5,6,7,8,9,10}
            ]
            % Road
            \addplot[cdeepBP, very thick, mark=diamond*, mark size=2pt, mark options={solid}] table[col sep=comma, x=x, y=y]{Data/knee_road.csv};
            \draw[orange, very thick, dashed] (2,-1) -- (2,2);
            \addlegendimage{dashed, line width=0.4mm, height=1mm, color=orange}
            \legend{data, knee/elbow}
            % \legend{AMSP+S (Ours), SP~\cite{cai2021revisiting}}
            \end{axis}
        % \node[above] at (3.7, 3.41) {\small Performance for varying budget};
        \end{tikzpicture}
        \caption{Road}
    \end{subfigure}
    \hspace{1mm}    
    \begin{subfigure}{.47\linewidth}
        \centering
        \begin{tikzpicture}
            \begin{axis}[
                legend style={nodes={scale=0.35}, at={(0.03, 0.24)}, anchor=west}, 
                xlabel={$x$},
                ylabel={$f_\theta(\text{pole};x)$},
                width=1.23\linewidth,
                height=1.23\linewidth,
                ymin=-0.1,
                ymax=1.1,
                % ytick={64, 66, 68, 70, 72, 74},
                xlabel style={yshift=0.15cm},
                % ylabel style={yshift=-0.6cm},
                ylabel style={yshift=-0.2cm},
                % legend columns=1,
                xmin=-0.5,
                xmax=18.5,
                label style={font=\scriptsize},
                tick label style={font=\scriptsize},
                xtick=data,
                % xticklabels={4,16,64,256,1024,4096},
            ]

            % Pole
            \addplot[cdeepBP, very thick, mark=diamond*, mark size=2pt, mark options={solid}] table[col sep=comma, x=x, y=y]{Data/knee_pole.csv};
            \draw[orange, very thick, dashed] (4,-1) -- (4,2);
            \addlegendimage{dashed, line width=0.4mm, height=1mm, color=cdeepBP32}
            % \legend{data, knee/elbow}
            \end{axis}
        \end{tikzpicture}
        \caption{Pole}
    \end{subfigure}
    \caption{{\em Examples of knee points on Cityscapes.} We obtain (a) a high knee value for the common road class and (b) a low knee value for the rare pole class.}
    \label{fig:sup-knee-points}
\end{figure}

\begin{figure*}[t!]
\captionsetup[subfigure]{font=footnotesize,labelfont=footnotesize,aboveskip=0.05cm,belowskip=-0.15cm}
\centering
\begin{tikzpicture}
    \begin{axis}[
        width  = \textwidth,
        axis y line*=left,
        symbolic x coords={
            \rotatebox{60}{Road},
            \rotatebox{60}{Building},
            \rotatebox{60}{Vegetation},
            \rotatebox{60}{Car},
            \rotatebox{60}{Sidewalk},
            \rotatebox{60}{Sky},
            \rotatebox{60}{Pole},
            \rotatebox{60}{Person},
            \rotatebox{60}{Terrain},
            \rotatebox{60}{Fence},
            \rotatebox{60}{Wall},
            \rotatebox{60}{Sign},
            \rotatebox{60}{Bicycle},
            \rotatebox{60}{Truck},
            \rotatebox{60}{Bus},
            \rotatebox{60}{Train},
            \rotatebox{60}{Light},
            \rotatebox{60}{Rider},
            \rotatebox{60}{Motorcycle},
        },
        axis x line=bottom,
        height = 5.2cm,
        major x tick style = transparent,
        %axis on top,
        ybar=3*\pgflinewidth,
        bar width=4pt,
        ymajorgrids = true,
        ylabel = {IoU},
        % xlabel = {Class},
        xtick = data,
        scaled y ticks = false,
        enlarge x limits=0.3,
        axis line style={-},
        ymin=0.0,ymax=1,
        legend columns=2,
        legend cell align=left,
        legend style={
                nodes={scale=0.6},
                at={(0.5,1.2)},
                anchor=north,
                column sep=1ex
        },
        label style={font=\scriptsize},
        tick label style={font=\scriptsize}
    ]
        \addplot[style={cdeepBP,fill=cdeepBP,mark=none}] coordinates {
            (\rotatebox{60}{Road}, 0.964601)
            (\rotatebox{60}{Building}, 0.883573)
            (\rotatebox{60}{Vegetation}, 0.897434)
            (\rotatebox{60}{Car}, 0.908026)
            (\rotatebox{60}{Sidewalk}, 0.750152)
            (\rotatebox{60}{Sky}, 0.906192)
            (\rotatebox{60}{Pole}, 0.482772)
            (\rotatebox{60}{Person}, 0.729143)
            (\rotatebox{60}{Terrain}, 0.55855)
            (\rotatebox{60}{Fence}, 0.492545)
            (\rotatebox{60}{Wall}, 0.461659)
            (\rotatebox{60}{Sign}, 0.614888)
            (\rotatebox{60}{Bicycle}, 0.67332)
            (\rotatebox{60}{Truck}, 0.559177)
            (\rotatebox{60}{Bus}, 0.733037)
            (\rotatebox{60}{Train}, 0.601527)
            (\rotatebox{60}{Light}, 0.468385)
            (\rotatebox{60}{Rider}, 0.483054)
            (\rotatebox{60}{Motorcycle}, 0.473352)
        };
        \addplot[style={orange,fill=orange,mark=none}] coordinates {
            (\rotatebox{60}{Road}, 0.955517)
            (\rotatebox{60}{Building}, 0.852214)
            (\rotatebox{60}{Vegetation}, 0.880035)
            (\rotatebox{60}{Car}, 0.851916)
            (\rotatebox{60}{Sidewalk}, 0.89579)
            (\rotatebox{60}{Sky}, 0.896663)
            (\rotatebox{60}{Pole}, 0.324208)
            (\rotatebox{60}{Person}, 0.523996)
            (\rotatebox{60}{Terrain}, 0.479043)
            (\rotatebox{60}{Fence}, 0.376733)
            (\rotatebox{60}{Wall}, 0.211172)
            (\rotatebox{60}{Sign}, 0.431976)
            (\rotatebox{60}{Bicycle}, 0.503692)
            (\rotatebox{60}{Truck}, 0.097781)
            (\rotatebox{60}{Bus}, 0.401534)
            (\rotatebox{60}{Train}, 0.128532)
            (\rotatebox{60}{Light}, 0.058156)
            (\rotatebox{60}{Rider}, 0.04937)
            (\rotatebox{60}{Motorcycle}, 0.01)
        };
        % \addplot[style={cdeepMF,fill=cdeepMF,mark=none}]
             % coordinates {(Road, 19.22) (Building, 21.29) (Vegetation, 21.58)};
        \legend{AMSP+S (Kneedle \cite{satopaa2011finding}), AMSP+S $( \phi(s) = 0.6 )$}
    \end{axis}
\end{tikzpicture}
\caption{{\em Class-wise IoU according to $\phi(s;\theta)$.} Applying the same $\phi(s)$ of 0.6 to all pixels results in excessive sieving for relatively rare classes, leading to decreased performance for these classes (\eg Light, Rider, and Motorcycle). Based on the ground-truth, class labels are organized in order of the total pixel count for each class.}
\label{fig:sup-class-dist}
\end{figure*}

\begin{figure*}[t!]
    \captionsetup[subfigure]{font=footnotesize}
    \centering
    \begin{subfigure}{.33\linewidth}
        \centering
        \includegraphics[scale=0.322]{Figures/aachen_000000_000019_seg.png}
        \caption{Semantic segmentation}
        \label{subfig:sementic-seg}
    \end{subfigure}
    \begin{subfigure}{.33\linewidth}
        \centering
        \includegraphics[scale=0.322]{Figures/aachen_000000_000019_gt.jpg}
        \caption{Panoptic segmentation}
    \end{subfigure}
    \begin{subfigure}{.33\linewidth}
        \centering
        \includegraphics[scale=0.322]{Figures/aachen_000000_000019_gt_cc.jpg}
        \caption{Oracle superpixels (ours)}
        \label{subfig:oralce-seg}
    \end{subfigure} 
    \caption{{\em Difference between conventional segmentations and oracle superpixels.} (a) When sharing the same class label, they are depicted as identical superpixels (\ie green color on separate trees).  (b) Although a building is divided by a pole, it is represented as a single superpixel (\ie cyan color). (c) We consider a building as two distinct superpixels (\ie cyan and light yellow colors).}
    \label{fig:sup-oracle-superpixels}
\end{figure*}



We provide the reason for introducing 
the threshold function $\phi(s)$
personalized for each superpixel $s$, described in Section \ref{sec:sieving-technique}.
We obtain the dominant label $\text{D}(s)$ for a queried superpixel $s$, however, we only propagate the label to pixels $x \in s$ that are predicted to have a positive impact on the training of model $\theta$ as:
\begin{equation}
h(s;\theta) := \{ x \in s: f_\theta \big( \text{D}(s); x \big) \geq \phi(s; \theta) \} \;,
\end{equation}
where $f_\theta \left( \text{D}(s);x \right)$ implies the confidence of pixel $x$ to dominant label $\text{D}(s)$ given $\theta$ and $\phi(s;\theta)$ determines the degree of sieving.
In Table \ref{tab:sieving}, we study the effect of various $\phi(s;\theta)$.
When the same $\phi(s;\theta)$ is applied to all pixels, it causes class imbalance by leaving relatively easy classes as described in Figure \ref{fig:sup-class-dist}.
To avoid this issue, we utilize the Kneedle algorithm \cite{satopaa2011finding} to obtain different $\phi(s;\theta)$ for each superpixel $s$.
Specifically, $\phi(s; \theta)$ is a knee point of the cumulative distribution function of values of $f_\theta \big( \text{D}(s); x \big)$ in superpixel $x \in s$.
However, for the Kneedle algorithm to work accurately, the curve of cumulative distribution must be either convex or concave. 
In addition, the algorithm may provide inaccurate knee points on very smooth curves.
To address this issue, we use a subset of uniformly sampled values based on $f_\theta(\text{D}(s);x)$, instead of using the distribution for all pixels.
We sample 20 and 5 pixels for Cityscapes and PASCAL datasets, respectively.
In Figure \ref{fig:sup-knee-points}, different knee points are detected according to the dominant class of superpixels.

% \khy{
\smallskip\noindent\textbf{Effect of sieving.} 
Our sieving method exhibits a significant effect on larger superpixels, as illustrated in Figure \ref{fig:(c)-effect} and Figure \ref{fig:supple-region-size-pascal}.
Especially, in Figure \ref{fig:supple-region-size-pascal} with a large base superpixel size of 4096, the first sieving excises 45.87\% of the mislabeled pixels that disagree with their dominant labels.
Furthermore, we observe that the sieving
is progressively refined round by round.
For instance, in Figure \ref{fig:(a)-effect}, the portion of the mislabeled labels removed by the sieving increases over four rounds as
follows: 3.58\%, 8.54\%, 10.46\%, and 12.43\%.
Our sieving technique enhances label quality by retaining only high-confidence labels and continuously improves through multiple rounds.
% }


\begin{figure*}[t!]
    \captionsetup[subfigure]{font=footnotesize,labelfont=footnotesize,aboveskip=0.05cm,belowskip=-0.15cm}
    \centering
    \hspace{-1mm}
    \begin{subfigure}{.235\linewidth}
        \centering
        \begin{tikzpicture}
            \begin{axis}[
                legend style={nodes={scale=0.35}, at={(0.03, 0.24)}, anchor=west}, 
                xlabel={AF$(G;S)$},
                ylabel={mIoU (\%)},
                width=1.23\linewidth,
                height=1.23\linewidth,
                ymin=50.8,
                ymax=63.2,
                xlabel style={yshift=0.15cm},
                % ylabel style={yshift=-0.6cm},
                ylabel style={yshift=-0.2cm},
                ytick={51, 53, 55, 57, 59, 61, 63},
                legend columns=2,
                xmin=0.06,
                xmax=0.42,
                label style={font=\scriptsize},
                tick label style={font=\scriptsize},
                x tick label style={
                    /pgf/number format/.cd,
                        fixed,
                }
            ]
            \addplot[cdeepBP, only marks] table[col sep=comma, x=AFGS, y=mIoU]{Data/correlation_afgs_semantic.csv};
            \addplot[very thick, orange] table[col sep=comma, x=AFGS, y={create col/linear regression = {y=mIoU}}
            ]{Data/correlation_afgs_semantic.csv};
            \draw (0.5\linewidth, 0.35\linewidth) node {\scriptsize $\text{Corr}=0.57$};
            \end{axis}
        \end{tikzpicture}
        \caption{Semantic segmentation}
    \end{subfigure}
    \hspace{1mm}
    \begin{subfigure}{.235\linewidth}
        \centering
        \begin{tikzpicture}
            \begin{axis}[
                legend style={nodes={scale=0.35}, at={(0.03, 0.24)}, anchor=west}, 
                xlabel={AF$(G;S)$},
                ylabel={mIoU (\%)},
                width=1.23\linewidth,
                height=1.23\linewidth,
                ymin=50.8,
                ymax=63.2,
                xlabel style={yshift=0.15cm},
                % ylabel style={yshift=-0.6cm},
                ylabel style={yshift=-0.2cm},
                ytick={51, 53, 55, 57, 59, 61, 63},
                legend columns=2,
                xmin=0.21,
                xmax=0.39,
                label style={font=\scriptsize},
                tick label style={font=\scriptsize},
                x tick label style={
                    /pgf/number format/.cd,
                        fixed,
                }
            ]
            \addplot[cdeepBP, only marks] table[col sep=comma, x=AFGS, y=mIoU]{Data/correlation_afgs_panoptic.csv};
            \addplot[very thick, orange] table[col sep=comma, x=AFGS, y={create col/linear regression = {y=mIoU}}
            ]{Data/correlation_afgs_panoptic.csv};
            \draw (0.5\linewidth, 0.35\linewidth) node {\scriptsize$\textbf{Corr}=\textbf{0.95}$};
            \end{axis}
        \end{tikzpicture}
        \caption{Panoptic segmentation}
    \end{subfigure}
    \hspace{1mm}
    \begin{subfigure}{.235\linewidth}
        \centering
        \begin{tikzpicture}
            \begin{axis}[
                legend style={nodes={scale=0.35}, at={(0.03, 0.24)}, anchor=west}, 
                xlabel={AF$(G;S)$},
                ylabel={mIoU (\%)},
                width=1.23\linewidth,
                height=1.23\linewidth,
                ymin=50.8,
                ymax=63.2,
                xlabel style={yshift=0.15cm},
                % ylabel style={yshift=-0.6cm},
                ylabel style={yshift=-0.2cm},
                ytick={51, 53, 55, 57, 59, 61, 63},
                legend columns=2,
                xmin=0.173,
                xmax=0.371,
                label style={font=\scriptsize},
                tick label style={font=\scriptsize},
                x tick label style={
                    /pgf/number format/.cd,
                        fixed,
                }
            ]
            \addplot[cdeepBP, only marks] table[col sep=comma, x=AFGS, y=mIoU]{Data/correlation_afgs.csv};
            \addplot[very thick, orange] table[col sep=comma, x=AFGS, y={create col/linear regression = {y=mIoU}}
            ]{Data/correlation_afgs.csv};
            \draw (0.5\linewidth, 0.35\linewidth) node {\scriptsize$\textbf{Corr}=\textbf{0.95}$};
            \end{axis}
        \end{tikzpicture}
        \caption{Oracle superpixels}
    \end{subfigure}
    \caption{{\em Relationship between AF$(G;S)$ and mIoU varying $G$.} AF$(G;S)$ and mIoU exhibit a high correlation when ground-truth $G$ is represented by the panoptic segmentation and oracle superpixels in Figure \ref{fig:sup-oracle-superpixels}. For the correlation calculation, \textit{Oracle} in Table \ref{tab:quantitative} is excluded.}
    \label{fig:sup-afgs-g}
\end{figure*}



\begin{figure*}[!ht]
    \captionsetup[subfigure]{font=scriptsize,labelfont=scriptsize,aboveskip=0.05cm,belowskip=-0.15cm}
    \centering
    \hspace{-1mm}
    \begin{subfigure}{.235\linewidth}
        \centering
        \begin{tikzpicture}
            \begin{axis}[
                legend style={nodes={scale=0.35}, at={(0.03, 0.24)}, anchor=west}, 
                xlabel={ASA$(S;G)$},
                ylabel={mIoU (\%)},
                width=1.23\linewidth,
                height=1.23\linewidth,
                ymin=50.8,
                ymax=63.2,
                ytick={51, 53, 55, 57, 59, 61, 63},
                xlabel style={yshift=0.15cm},
                % ylabel style={yshift=-0.6cm},
                ylabel style={yshift=-0.2cm},
                legend columns=2,
                xmin=0.877,
                xmax=0.971,
                label style={font=\scriptsize},
                tick label style={font=\scriptsize},
                x tick label style={
                    /pgf/number format/.cd,
                        fixed,
                }
            ]
            \addplot[cdeepBP, only marks] table[col sep=comma, x=ASASG, y=mIoU]{Data/correlation_asasg.csv};
            \addplot[very thick, orange] table[col sep=comma, x=ASASG, y={create col/linear regression = {y=mIoU}}
            ]{Data/correlation_asasg.csv};
            \draw (0.5\linewidth, 0.35\linewidth) node {\scriptsize$\text{Corr} = 0.05$};
            \end{axis}
        \end{tikzpicture}
        % \caption{ASA(S;G) vs mIoU}
    \end{subfigure}
    \hspace{1mm}
    \begin{subfigure}{.235\linewidth}
        \centering
        \begin{tikzpicture}
            \begin{axis}[
                legend style={nodes={scale=0.35}, at={(0.03, 0.24)}, anchor=west}, 
                xlabel={ASA$(G;S)$},
                ylabel={mIoU (\%)},
                width=1.23\linewidth,
                height=1.23\linewidth,
                ymin=50.8,
                ymax=63.2,
                xlabel style={yshift=0.15cm},
                % ylabel style={yshift=-0.6cm},
                ylabel style={yshift=-0.2cm},
                ytick={51, 53, 55, 57, 59, 61, 63},
                legend columns=2,
                xmin=-0.05,
                xmax=0.657,
                label style={font=\scriptsize},
                tick label style={font=\scriptsize},
                x tick label style={
                    /pgf/number format/.cd,
                        fixed,
                }
            ]
            \addplot[cdeepBP, only marks] table[col sep=comma, x=ASAGS, y=mIoU]{Data/correlation_asags.csv};
            \addplot[very thick, orange] table[col sep=comma, x=ASAGS, y={create col/linear regression = {y=mIoU}}
            ]{Data/correlation_asags.csv};
            \draw (0.5\linewidth, 0.35\linewidth) node {\scriptsize$\text{Corr}=0.71$};
            \end{axis}
        \end{tikzpicture}
    \end{subfigure}
    \hspace{1mm}
    \begin{subfigure}{.235\linewidth}
        \centering
        \begin{tikzpicture}
            \begin{axis}[
                legend style={nodes={scale=0.35}, at={(0.03, 0.24)}, anchor=west}, 
                xlabel={AP$(S;G)$},
                ylabel={mIoU (\%)},
                width=1.23\linewidth,
                height=1.23\linewidth,
                ymin=50.8,
                ymax=63.2,
                ytick={51, 53, 55, 57, 59, 61, 63},
                xlabel style={yshift=0.15cm},
                % ylabel style={yshift=-0.6cm},
                ylabel style={yshift=-0.2cm},
                legend columns=2,
                xmin=0.87,
                xmax=0.97,
                label style={font=\scriptsize},
                tick label style={font=\scriptsize},
                x tick label style={
                    /pgf/number format/.cd,
                        fixed,
                }
            ]
            \addplot[cdeepBP, only marks] table[col sep=comma, x=APSG, y=mIoU]{Data/correlation_apsg.csv};
            \addplot[very thick, orange] table[col sep=comma, x=APSG, y={create col/linear regression = {y=mIoU}}
            ]{Data/correlation_apsg.csv};
            \draw (0.5\linewidth, 0.35\linewidth) node {\scriptsize$\text{Corr} = -0.22$};
            \end{axis}
        \end{tikzpicture}
        % \caption{ASA(S;G) vs mIoU}
    \end{subfigure}
    \hspace{1mm}
    \begin{subfigure}{.235\linewidth}
        \centering
        \begin{tikzpicture}
            \begin{axis}[
                legend style={nodes={scale=0.35}, at={(0.03, 0.24)}, anchor=west}, 
                xlabel={AR$(S;G)$},
                ylabel={mIoU (\%)},
                width=1.23\linewidth,
                height=1.23\linewidth,
                ymin=50.8,
                ymax=63.2,
                xlabel style={yshift=0.15cm},
                % ylabel style={yshift=-0.6cm},
                ylabel style={yshift=-0.2cm},
                ytick={51, 53, 55, 57, 59, 61, 63},
                legend columns=2,
                xmin=0,
                xmax=0.057,
                label style={font=\scriptsize},
                tick label style={font=\scriptsize},
                x tick label style={
                    /pgf/number format/.cd,
                        fixed,
                    /tikz/.cd,
                }
            ]
            \addplot[cdeepBP, only marks] table[col sep=comma, x=ARSG, y=mIoU]{Data/correlation_arsg.csv};
            \addplot[very thick, orange] table[col sep=comma, x=ARSG, y={create col/linear regression = {y=mIoU}}
            ]{Data/correlation_arsg.csv};
            \draw (0.5\linewidth, 0.35\linewidth) node {\scriptsize$\text{Corr}=-0.02$};
            \end{axis}
        \end{tikzpicture}
    \end{subfigure}
    \hspace{-5mm}
    \begin{subfigure}{.235\linewidth}
        \centering
        \begin{tikzpicture}
            \begin{axis}[
                legend style={nodes={scale=0.35}, at={(0.03, 0.24)}, anchor=west}, 
                xlabel={AF$(S;G)$},
                ylabel={mIoU (\%)},
                width=1.23\linewidth,
                height=1.23\linewidth,
                ymin=50.8,
                ymax=63.2,
                ytick={51, 53, 55, 57, 59, 61, 63},
                xlabel style={yshift=0.15cm},
                % ylabel style={yshift=-0.6cm},
                ylabel style={yshift=-0.2cm},
                legend columns=2,
                xmin=0,
                xmax=0.082,
                label style={font=\scriptsize},
                tick label style={font=\scriptsize},
                x tick label style={
                    /pgf/number format/.cd,
                        fixed,
                    /tikz/.cd,
                }
            ]
            \addplot[cdeepBP, only marks] table[col sep=comma, x=AFSG, y=mIoU]{Data/correlation_afsg.csv};
            \addplot[very thick, orange] table[col sep=comma, x=AFSG, y={create col/linear regression = {y=mIoU}}
            ]{Data/correlation_afsg.csv};
            \draw (0.5\linewidth, 0.35\linewidth) node {\scriptsize$\text{Corr} = -0.01$};
            \end{axis}
        \end{tikzpicture}
        % \caption{ASA(S;G) vs mIoU}
    \end{subfigure}
    \hspace{1mm}
    \begin{subfigure}{.235\linewidth}
        \centering
        \begin{tikzpicture}
            \begin{axis}[
                legend style={nodes={scale=0.35}, at={(0.03, 0.24)}, anchor=west}, 
                xlabel={AP$(G;S)$},
                ylabel={mIoU (\%)},
                width=1.23\linewidth,
                height=1.23\linewidth,
                ymin=50.8,
                ymax=63.2,
                xlabel style={yshift=0.15cm},
                % ylabel style={yshift=-0.6cm},
                ylabel style={yshift=-0.2cm},
                ytick={51, 53, 55, 57, 59, 61, 63},
                legend columns=2,
                xmin=0.35,
                xmax=0.74,
                label style={font=\scriptsize},
                tick label style={font=\scriptsize},
                x tick label style={
                    /pgf/number format/.cd,
                        fixed,
                }
            ]
            \addplot[cdeepBP, only marks] table[col sep=comma, x=APGS, y=mIoU]{Data/correlation_apgs.csv};
            \addplot[very thick, orange] table[col sep=comma, x=APGS, y={create col/linear regression = {y=mIoU}}
            ]{Data/correlation_apgs.csv};
            \draw (0.5\linewidth, 0.35\linewidth) node {\scriptsize$\text{Corr}=-0.43$};
            \end{axis}
        \end{tikzpicture}
    \end{subfigure}
    \hspace{1mm}
    \begin{subfigure}{.235\linewidth}
        \centering
        \begin{tikzpicture}
            \begin{axis}[
                legend style={nodes={scale=0.35}, at={(0.03, 0.24)}, anchor=west}, 
                xlabel={AR$(G;S)$},
                ylabel={mIoU (\%)},
                width=1.23\linewidth,
                height=1.23\linewidth,
                ymin=50.8,
                ymax=63.2,
                ytick={51, 53, 55, 57, 59, 61, 63},
                xlabel style={yshift=0.15cm},
                % ylabel style={yshift=-0.6cm},
                ylabel style={yshift=-0.2cm},
                legend columns=2,
                xmin=0.21,
                xmax=0.696,
                label style={font=\scriptsize},
                tick label style={font=\scriptsize},
                x tick label style={
                    /pgf/number format/.cd,
                        fixed,
                }
            ]
            \addplot[cdeepBP, only marks] table[col sep=comma, x=ARGS, y=mIoU]{Data/correlation_args.csv};
            \addplot[very thick, orange] table[col sep=comma, x=ARGS, y={create col/linear regression = {y=mIoU}}
            ]{Data/correlation_args.csv};
            \draw (0.5\linewidth, 0.35\linewidth) node {\scriptsize$\text{Corr} = 0.57$};
            \end{axis}
        \end{tikzpicture}
        % \caption{ASA(S;G) vs mIoU}
    \end{subfigure}
    \hspace{1mm}
    \begin{subfigure}{.235\linewidth}
        \centering
        \begin{tikzpicture}
            \begin{axis}[
                legend style={nodes={scale=0.35}, at={(0.03, 0.24)}, anchor=west}, 
                xlabel={AF$(G;S)$},
                ylabel={mIoU (\%)},
                width=1.23\linewidth,
                height=1.23\linewidth,
                ymin=50.8,
                ymax=63.2,
                xlabel style={yshift=0.15cm},
                % ylabel style={yshift=-0.6cm},
                ylabel style={yshift=-0.2cm},
                ytick={51, 53, 55, 57, 59, 61, 63},
                legend columns=2,
                xmin=0.173,
                xmax=0.371,
                label style={font=\scriptsize},
                tick label style={font=\scriptsize},
                x tick label style={
                    /pgf/number format/.cd,
                        fixed,
                }
            ]
            \addplot[cdeepBP, only marks] table[col sep=comma, x=AFGS, y=mIoU]{Data/correlation_afgs.csv};
            \addplot[very thick, orange] table[col sep=comma, x=AFGS, y={create col/linear regression = {y=mIoU}}
            ]{Data/correlation_afgs.csv};
            \draw (0.5\linewidth, 0.35\linewidth) node {\scriptsize$\textbf{Corr}=\textbf{0.95}$};
            \end{axis}
        \end{tikzpicture}
    \end{subfigure}
    \caption{{\em Relationship between metrics and mIoU.} The correlation between AF$(G;S)$ and mIoU is especially high. For the correlation calculation, \textit{Oracle} in Table \ref{tab:quantitative} is excluded.}
    \label{fig:sup-correlation}
\end{figure*}











% To address this issue, we sample the values uniformly by confidence order for all pixels inside the superpixel, without using the distribution for all pixels, which is described in Figure \ref{fig:sup-knee-points}.
% We sample 20 and 5 pixels for Cityscapes and PASCAL datasets, respectively.
% hyperparameter for sieving, easy class, difficult class, different confidence
% When we set the same threshold for all pixels, only the common classes remain, and the IoU of rare classes decreases in Figure \ref{fig:sup-class-dist}.




\begin{figure*}[t!]
    \captionsetup[subfigure]{font=footnotesize}
    \centering
    \begin{subfigure}[!ht]{.245\linewidth}
        \centering
        % \includegraphics[scale=0.078]{Figures/aachen_000000_000019.jpg}
        \includegraphics[scale=0.238]{Figures/fig13_round/bochum_000000_025833_r1.png}
        % \includegraphics[scale=0.0565]{Figures/fig1/fig_1_a.png}
        % \includegraphics[scale=0.121]{Figures/fig1/fig1_a.png}
    \end{subfigure}
    \begin{subfigure}[!ht]{.245\linewidth}
        \centering
        % \includegraphics[scale=0.078]{Figures/cityscapes_merged_round1.jpg}
        \includegraphics[scale=0.238]{Figures/fig13_round/bochum_000000_025833_r2.png}
        % \includegraphics[scale=0.0565]{Figures/fig1/fig_1_b_v2.png}
        % \includegraphics[scale=0.121]{Figures/fig1/fig1_b.png}
    \end{subfigure}
    \begin{subfigure}[!ht]{.245\linewidth}
        \centering
        % \includegraphics[scale=0.078]{Figures/cityscapes_merged_round1.jpg}
        \includegraphics[scale=0.238]{Figures/fig13_round/bochum_000000_025833_r3.png}
        % \includegraphics[scale=0.0565]{Figures/fig1/fig_1_c_v2.png}
        % \includegraphics[scale=0.121]{Figures/fig1/fig1_c.png}
    \end{subfigure}
    \begin{subfigure}[!ht]{.245\linewidth}
        \centering
        % \includegraphics[scale=0.078]{Figures/cityscapes_merged_round1.jpg}
        \includegraphics[scale=0.238]{Figures/fig13_round/bochum_000000_025833_r4.png}
        % \includegraphics[scale=0.0565]{Figures/fig1/fig_1_d_v3.png}
        % \includegraphics[scale=0.121]{Figures/fig1/fig1_d.png}
    \end{subfigure}

    
    \begin{subfigure}[!ht]{.245\linewidth}
        \centering
        % \includegraphics[scale=0.078]{Figures/aachen_000000_000019.jpg}
        \includegraphics[scale=0.238]{Figures/fig13_round/hanover_000000_058189_r1.png}
        % \includegraphics[scale=0.0565]{Figures/fig1/fig_1_a.png}
        % \includegraphics[scale=0.121]{Figures/fig1/fig1_a.png}
    \end{subfigure}
    \begin{subfigure}[!ht]{.245\linewidth}
        \centering
        % \includegraphics[scale=0.078]{Figures/cityscapes_merged_round1.jpg}
        \includegraphics[scale=0.238]{Figures/fig13_round/hanover_000000_058189_r2.png}
        % \includegraphics[scale=0.0565]{Figures/fig1/fig_1_b_v2.png}
        % \includegraphics[scale=0.121]{Figures/fig1/fig1_b.png}
    \end{subfigure}
    \begin{subfigure}[!ht]{.245\linewidth}
        \centering
        % \includegraphics[scale=0.078]{Figures/cityscapes_merged_round1.jpg}
        \includegraphics[scale=0.238]{Figures/fig13_round/hanover_000000_058189_r3.png}
        % \includegraphics[scale=0.0565]{Figures/fig1/fig_1_c_v2.png}
        % \includegraphics[scale=0.121]{Figures/fig1/fig1_c.png}
    \end{subfigure}
    \begin{subfigure}[!ht]{.245\linewidth}
        \centering
        % \includegraphics[scale=0.078]{Figures/cityscapes_merged_round1.jpg}
        \includegraphics[scale=0.238]{Figures/fig13_round/hanover_000000_058189_r4.png}
        % \includegraphics[scale=0.0565]{Figures/fig1/fig_1_d_v3.png}
        % \includegraphics[scale=0.121]{Figures/fig1/fig1_d.png}
    \end{subfigure}
    
    
    \begin{subfigure}[!ht]{.245\linewidth}
        \centering
        % \includegraphics[scale=0.078]{Figures/aachen_000000_000019.jpg}
        \includegraphics[scale=0.238]{Figures/fig13_round/zurich_000117_000019_r1.png}
        % \includegraphics[scale=0.0565]{Figures/fig1/fig_1_a.png}
        % \includegraphics[scale=0.121]{Figures/fig1/fig1_a.png}
        \caption{Adaptive merged $(t=1)$}
    \end{subfigure}
    \begin{subfigure}[!ht]{.245\linewidth}
        \centering
        % \includegraphics[scale=0.078]{Figures/cityscapes_merged_round1.jpg}
        \includegraphics[scale=0.238]{Figures/fig13_round/zurich_000117_000019_r2.png}
        % \includegraphics[scale=0.0565]{Figures/fig1/fig_1_b_v2.png}
        % \includegraphics[scale=0.121]{Figures/fig1/fig1_b.png}
        \caption{Adaptive merged $(t=2)$}
    \end{subfigure}
    \begin{subfigure}[!ht]{.245\linewidth}
        \centering
        % \includegraphics[scale=0.078]{Figures/cityscapes_merged_round1.jpg}
        \includegraphics[scale=0.238]{Figures/fig13_round/zurich_000117_000019_r3.png}
        % \includegraphics[scale=0.0565]{Figures/fig1/fig_1_c_v2.png}
        % \includegraphics[scale=0.121]{Figures/fig1/fig1_c.png}
        \caption{Adaptive merged $(t=3)$}
    \end{subfigure}
    \begin{subfigure}[!ht]{.245\linewidth}
        \centering
        % \includegraphics[scale=0.078]{Figures/cityscapes_merged_round1.jpg}
        \includegraphics[scale=0.238]{Figures/fig13_round/zurich_000117_000019_r4.png}
        % \includegraphics[scale=0.0565]{Figures/fig1/fig_1_d_v3.png}
        % \includegraphics[scale=0.121]{Figures/fig1/fig1_d.png}
        \caption{Adaptive merged $(t=4)$}
    \end{subfigure}
    % \caption{{\em Round} (a) We begin active learning with over-segmented superpixels. (b, c) In each round $t$, we merge superpixels in an adaptive manner using the model from the previous round. % $t-1$. 
    % (d) As the round progresses, adaptive superpixels look similar to oracle ones.}
    \caption{{\em Qualitative results with varying round.}
    (a-d) Superpixels generated with proposed adaptive merging at rounds 1 to 4.
    Thanks to the improved model, we observe that the merging becomes more accurate as the round increases. We use the model reported in Figure~\ref{fig:(a)-effect}.}
    \label{fig:sup-round}
\end{figure*}


\begin{figure*}[t!]
    \captionsetup[subfigure]{font=footnotesize}
    \centering
    \begin{subfigure}[!ht]{.245\linewidth}
        \centering
        % \includegraphics[scale=0.078]{Figures/aachen_000000_000019.jpg}
        \includegraphics[scale=0.238]{Figures/fig14_eps/bremen_000310_000019_00.png}
        % \includegraphics[scale=0.121]{Figures/fig1/fig1_a.png}
    \end{subfigure}
    \begin{subfigure}[!ht]{.245\linewidth}
        \centering
        % \includegraphics[scale=0.078]{Figures/cityscapes_merged_round1.jpg}
        \includegraphics[scale=0.238]{Figures/fig14_eps/bremen_000310_000019_01.png}
        % \includegraphics[scale=0.121]{Figures/fig1/fig1_b.png}
    \end{subfigure}
    \begin{subfigure}[!ht]{.245\linewidth}
        \centering
        % \includegraphics[scale=0.078]{Figures/cityscapes_merged_round1.jpg}
        \includegraphics[scale=0.238]{Figures/fig14_eps/bremen_000310_000019_015.png}
        % \includegraphics[scale=0.121]{Figures/fig1/fig1_c.png}
    \end{subfigure}
    \begin{subfigure}[!ht]{.245\linewidth}
        \centering
        % \includegraphics[scale=0.078]{Figures/cityscapes_merged_round1.jpg}
        \includegraphics[scale=0.238]{Figures/fig14_eps/bremen_000310_000019_02.png}
        % \includegraphics[scale=0.121]{Figures/fig1/fig1_d.png}
    \end{subfigure}

    
    \begin{subfigure}[!ht]{.245\linewidth}
        \centering
        % \includegraphics[scale=0.078]{Figures/aachen_000000_000019.jpg}
        \includegraphics[scale=0.238]{Figures/fig14_eps/zurich_000012_000019_005.png}
        % \includegraphics[scale=0.121]{Figures/fig1/fig1_a.png}
    \end{subfigure}
    \begin{subfigure}[!ht]{.245\linewidth}
        \centering
        % \includegraphics[scale=0.078]{Figures/cityscapes_merged_round1.jpg}
        \includegraphics[scale=0.238]{Figures/fig14_eps/zurich_000012_000019_01.png}
        % \includegraphics[scale=0.0565]{Figures/fig1/fig_1_b_v2.png}
        % \includegraphics[scale=0.121]{Figures/fig1/fig1_b.png}
    \end{subfigure}
    \begin{subfigure}[!ht]{.245\linewidth}
        \centering
        % \includegraphics[scale=0.078]{Figures/cityscapes_merged_round1.jpg}
        \includegraphics[scale=0.238]{Figures/fig14_eps/zurich_000012_000019_015.png}
        % \includegraphics[scale=0.0565]{Figures/fig1/fig_1_c_v2.png}
        % \includegraphics[scale=0.121]{Figures/fig1/fig1_c.png}
    \end{subfigure}
    \begin{subfigure}[!ht]{.245\linewidth}
        \centering
        % \includegraphics[scale=0.078]{Figures/cityscapes_merged_round1.jpg}
        \includegraphics[scale=0.238]{Figures/fig14_eps/zurich_000012_000019_02.png}
        % \includegraphics[scale=0.0565]{Figures/fig1/fig_1_d_v3.png}
        % \includegraphics[scale=0.121]{Figures/fig1/fig1_d.png}
    \end{subfigure}
    
    
    \begin{subfigure}[!ht]{.245\linewidth}
        \centering
        % \includegraphics[scale=0.078]{Figures/aachen_000000_000019.jpg}
        \includegraphics[scale=0.238]{Figures/fig14_eps/zurich_000036_000019_005.png}
        % \includegraphics[scale=0.0565]{Figures/fig1/fig_1_a.png}
        % \includegraphics[scale=0.121]{Figures/fig1/fig1_a.png}
        \caption{Adaptive merged $(\epsilon=0.05)$}
    \end{subfigure}
    \begin{subfigure}[!ht]{.245\linewidth}
        \centering
        % \includegraphics[scale=0.078]{Figures/cityscapes_merged_round1.jpg}
        \includegraphics[scale=0.238]{Figures/fig14_eps/zurich_000036_000019_01.png}
        % \includegraphics[scale=0.0565]{Figures/fig1/fig_1_b_v2.png}
        % \includegraphics[scale=0.121]{Figures/fig1/fig1_b.png}
        \caption{Adaptive merged $(\epsilon=0.1)$}
    \end{subfigure}
    \begin{subfigure}[!ht]{.245\linewidth}
        \centering
        % \includegraphics[scale=0.078]{Figures/cityscapes_merged_round1.jpg}
        \includegraphics[scale=0.238]{Figures/fig14_eps/zurich_000036_000019_015.png}
        % \includegraphics[scale=0.0565]{Figures/fig1/fig_1_c_v2.png}
        % \includegraphics[scale=0.121]{Figures/fig1/fig1_c.png}
        \caption{Adaptive merged $(\epsilon=0.15)$}
    \end{subfigure}
    \begin{subfigure}[!ht]{.245\linewidth}
        \centering
        % \includegraphics[scale=0.078]{Figures/cityscapes_merged_round1.jpg}
        \includegraphics[scale=0.238]{Figures/fig14_eps/zurich_000036_000019_02.png}
        % \includegraphics[scale=0.0565]{Figures/fig1/fig_1_d_v3.png}
        % \includegraphics[scale=0.121]{Figures/fig1/fig1_d.png}
        \caption{Adaptive merged $(\epsilon=0.2)$}
    \end{subfigure}
    % \caption{{\em Epsilon.} (a) We begin active learning with over-segmented superpixels. (b, c) In each round $t$, we merge superpixels in an adaptive manner using the model from the previous round. % $t-1$. 
    % (d) As the round progresses, adaptive superpixels look similar to oracle ones.}
    \caption{{\em Qualitative results with varying $\epsilon$.}
    (a-d) Superpixels are generated with proposed adaptive merging with $\epsilon$: 0.05, 0.1, 0.15, 0.2. % $t-1$. 
    We observe that an increase in $\epsilon$ gives more aggressive merging. Merging is conducted on Cityscapes with a base superpixel size of 256. }
    \label{fig:sup-epsilon}
\end{figure*}


\begin{figure*}[t!]
    \captionsetup[subfigure]{font=footnotesize}
    \centering
    \begin{subfigure}{.33\linewidth}
        \centering
        \includegraphics[scale=0.322]{Figures/fig12_qual/fig_12_1a.png}
        % \caption{$\text{ASA}(S;G)=0.021, \; \text{AF}(G;S)=0.355$}
        % \vspace{2mm}
        % \label{(a)-qualitative}
    \end{subfigure}
    \begin{subfigure}{.33\linewidth}
        \centering
        \includegraphics[scale=0.322]{Figures/fig12_qual/fig_12_1b.png}
        % \includegraphics[scale=0.5]{Figures/fig4_b_1.png}
        % \caption{$\text{ASA}(S;G)=0.89, \; \text{AF}(G;S)=0.283$}
        % \vspace{2mm}
        % \label{(b)-qualitative}
    \end{subfigure}
    \begin{subfigure}{.33\linewidth}
        \centering
        \includegraphics[scale=0.322]{Figures/fig12_qual/fig_12_1c_n.png}
        % \includegraphics[scale=0.5]{Figures/fig4_c_1.png}
        % \caption{$ASA(S;G)=1.00, \; AF(G;S)=1.00$}
        % \vspace{2mm}
        % \label{(c)-qualitative}
    \end{subfigure}

    
    \begin{subfigure}{.33\linewidth}
        \centering
        \includegraphics[scale=0.322]{Figures/fig12_qual/fig_12_2a.png}
        % \includegraphics[scale=0.5]{Figures/fig4_a_2.png}
    \end{subfigure}
    \begin{subfigure}{.33\linewidth}
        \centering
        \includegraphics[scale=0.322]{Figures/fig12_qual/fig_12_2b.png}
        % \includegraphics[scale=0.5]{Figures/fig4_b_2.png}
    \end{subfigure}
    \begin{subfigure}{.33\linewidth}
        \centering
        \includegraphics[scale=0.322]{Figures/fig12_qual/fig_12_2c_n.png}
        % \includegraphics[scale=0.5]{Figures/fig4_c_2.png}
        % \caption{$ASA(S;G)=1.00, \; AF(G;S)=1.00$}
    \end{subfigure}

    \begin{subfigure}{.33\linewidth}
        \centering
        \includegraphics[scale=0.322]{Figures/fig12_qual/fig_12_3a.png}
        % \includegraphics[scale=0.5]{Figures/fig4_a_2.png}
        \caption{Base superpixels~\cite{van2012seeds}}
    \end{subfigure}
    \begin{subfigure}{.33\linewidth}
        \centering
        \includegraphics[scale=0.322]{Figures/fig12_qual/fig_12_3b.png}
        % \includegraphics[scale=0.5]{Figures/fig4_b_2.png}
        \caption{Merged superpixels (Ours)}
    \end{subfigure}
    \begin{subfigure}{.33\linewidth}
        \centering
        \includegraphics[scale=0.322]{Figures/fig12_qual/fig_12_3c_n.png}
        % \includegraphics[scale=0.5]{Figures/fig4_c_2.png}
        % \caption{$ASA(S;G)=1.00, \; AF(G;S)=1.00$}
        \caption{Oracle superpixels}
    \end{subfigure}
    %\caption{{\em Qualitative results of adaptive superpixels.} As the round progresses, (a) over-segmented superpixels becomes (b) adaptively merged ones, and they resemble (c) oracle superpixels, especially for the classes that the model is confident about.}
    \caption{{\em Qualitative results of adaptive superpixels.} (a) Base superpixel generated by SEEDS~\cite{van2012seeds} with size 256. (b) Superpixels generated with proposed adaptive merging at round 4. (c) Oracle superpixels generated from the ground truth.}
    \label{fig:sup-same-paper}
\end{figure*}


\section{Further discussion on the oracle superpixels}
\label{fig:sup-oracle}

In Section \ref{para:oracle-superpixels},
we introduce the oracle superpixels,
which we believe is an achievable optimal set of superpixels for active learning.
For clarification,  
we provide the detailed process of generating the proposed oracle superpixels.
In addition, we provide further insights
into the achievable notion of optimal superpixels.
%In order to demonstrate effectiveness of our merging process, we require an upper bound of AL performance we aim to achieve. To this end, we generate ideal superpixels, each without any noise. Such superpixels are called oracle superpixels.

The Cityscapes dataset is equipped with the ground-truth annotations for semantic segmentation, represented by dense pixel-wise labels: \ie., each pixel in an annotated image is assigned an ID that represents a ground-truth semantic category~(Figure~\ref{subfig:sementic-seg}). In such annotation, each group of pixels that share the same ID aligns perfectly with the boundary of semantic objects. However, each such group is not guaranteed to be a single-connected component of pixels.
% and hence is not a proper superpixel.
For example, different cars in Figure~\ref{subfig:sementic-seg} are assigned the same blue color despite being physically separated, and a car divided into two parts due to an obstructing pole is still colored blue. 
% In Figure~\ref{subfig:sementic-seg}, for example, different cars are away from each other, but are assigned the same blue color. 
% In addition, a car in Figure~\ref{subfig:sementic-seg} is divided into two parts due to an obstructing pole, but is also colored blue. 
This is opposed to what we hope to achieve by merging two adjacent superpixels repeatedly.
To address this issue, we subdivide each superpixel as necessary to ensure that every pixel within a superpixel is adjacent to each other.
% so that every pixel within a superpixel is adjacent to each other. 
We utilize OpenCV~\cite{opencv_library} and Shapely~\cite{shapely2007} to identify the maximal connected component of pixels sharing the same semantic. 
We apply the same procedure to annotated images in the PASCAL dataset
% to identify maximal connected components.
% Similarly, PASCAL dataset provides semantic annotations in dense pixel-wise assignment of classes. To identify boundary-preserving, maximal connnected components, we apply the aforementioned procedure to the annotated images. 
Figure~\ref{fig:sup-oracle-superpixels} illustrates the distinction between conventional semantic and panoptic segmentation and our oracle superpixels.
\iffalse
The Cityscapes dataset is equipped with ground-truth annotations for semantic segmentation, represented by polygon vertices that differentiate between polygons as much as possible.
% in the form of polygon vertices, distinguishing polygons as much as possible. 
Each semantic object (\eg., car) is associated with its label and a set of vertices on its contour. 
To partition an image to oracle superpixels, we first consider two distinct polygons with the same label as separate. 
% we first treat two distinct polygons as different even though they are assigned the same label.
% Now, we have partitions of an image with each superpixel aligned with the exact boundary of a semantic object. 
This results in image partitions where each superpixel aligns perfectly with the boundary of a semantic object. 
However, there still remains a factor that makes such partition unrealistic: \ie., the inability to guarantee that each superpixel is a single-connected component of pixels.
% However, there still remains what makes such partition unrealistic --- \ie., each superpixel is not guaranteed to be a single-connected component of pixels.
For example, a car that is divided into two parts due to an obstructing pole is treated as a single superpixel.
% even if a car is obstructed by a pole and divided into two parts, they are treated as a single superpixel. 
This is opposed to what we hope to achieve by merging two adjacent superpixels repeatedly.
To overcome this limitation, we subdivide each superpixel if necessary so that every pixel within a superpixel is adjacent to each other. 
We utilize OpenCV~\cite{opencv_library} and Shapely~\cite{shapely2007} to identify the maximal connected component of pixels that share the same semantic.
Unlike Cityscapes, PASCAL dataset represents semantic annotations in the form of PNG images, where each pixel is assigned a semantic category. 
To identify boundary-preserving, maximal connected components, we apply the aforementioned procedure to the PNG images.
Figure \ref{fig:sup-oracle-superpixels} illustrates the distinction between conventional semantic and instance segmentations and our oracle superpixels.
% Therefore, we simply apply to the PNG images, the aforementioned procedure in order to find boundary-preserving, maximal connected components.
\fi

The Cityscapes and PASCAL datasets are divided into 327k and 16k oracle superpixels, respectively.
% , respectively to 408K and 16k oracle superpixels.
It is worth noting that the PASCAL has a lower number of oracle superpixels due to the smaller number of classes per image.
In other words, only a few objects are of interest in each image, and the rest are simply treated as the background.
% \ie., in each image, only few objects are of interest and the rest are simply treated as a background. 
% \todo{Detailed implementation can be found on \url{our.path.to.repository}}.
% \st{oracle statistics}, \todo{why pascal get high mIoU}, additional division vs ground-truth (\todo{Figure?}), \st{generation cityscapes, pascal}

\iffalse
\section{Compatibility with other base superpixels}
\begin{table}[!ht]
\centering
\setlength\tabcolsep{6pt}
\begin{tabular}{l|l|c}
\toprule
Superpixel algorithm & Method & mIoU \\ \midrule
\multirow{2}{*}{SEEDS} & \textit{SP}~\cite{cai2021revisiting} & 63.77 \\
                       & \textit{AMSP+S} (Ours) & \textbf{66.53} \\ \midrule
\multirow{2}{*}{SLIC}  & \textit{SP}~\cite{cai2021revisiting} & 65.97 \\
                       & \textit{AMSP+S} (Ours) & \textbf{67.56} \\
\bottomrule
\end{tabular}
\caption{{\em Other base superpixels.} 
Experiments are carried out under the identical settings, as presented in Table \ref{tab:sieving}.}
\label{tab:slic}
\vspace{-6mm}
\end{table}
\fi

% \begin{table}[!ht]
% \centering
% \setlength\tabcolsep{6pt}
% \begin{tabular}{l|cc}
% \toprule
% Methods & SEEDS & SLIC \\ \midrule
% \textit{SP} \cite{cai2021revisiting} & 63.77 & 65.97 \\ \midrule
% \rowcolor{Gray}
% \textit{AMSP+S} (Ours) & \textbf{66.53} & \textbf{67.56} \\ \midrule
% \bottomrule
% \end{tabular}
% \caption{{\em Other base superpixels.} 
% Experiments are carried out under the identical settings, as presented in Table \ref{tab:sieving}.}
% \label{tab:slic}
% \vspace{-3mm}
% \end{table}

\iffalse
\khy{
For a fair comparison, we employ the same superpixel algorithm called SEEDS \cite{van2012seeds} used in the previous study \cite{cai2021revisiting}.
However, our algorithm works independently of the choice of base superpixels since our merging and sieving processes can be applied on top of any base superpixels, \eg in Figure \ref{fig:(c)-effect} and \ref{fig:supple-region-size-pascal}, our method consistently shows gains over SP \cite{cai2021revisiting} when using base superpixels of different sizes.
Furthermore, we conduct an additional experiment on Cityscapes with SLIC \cite{achanta2012slic} instead of SEEDS.
We observe that the merging and sieving are also effective with SLIC as they were with SEEDS.
Furthermore, without relying on specific superpixel algorithms like SEEDS or SLIC, our proposed method can work with any base superpixels, which are image partitions obtained from unsupervised segmentation methods~\cite{ke2022unsupervised} or foundation models~\cite{kirillov2023segment}.
This flexibility opens up interesting avenues for exploring the effectiveness of our proposed approach with different types of superpixels.
}
\fi

\begin{figure*}[t!]
\captionsetup[subfigure]{font=footnotesize,labelfont=footnotesize,aboveskip=0.05cm,belowskip=-0.15cm}
\centering
\begin{tikzpicture}
    \begin{axis}[
        width  = \textwidth,
        axis y line*=left,
        symbolic x coords={
            \rotatebox{60}{Road},
            \rotatebox{60}{Building},
            \rotatebox{60}{Vegetation},
            \rotatebox{60}{Car},
            \rotatebox{60}{Sidewalk},
            \rotatebox{60}{Sky},
            \rotatebox{60}{Pole},
            \rotatebox{60}{Person},
            \rotatebox{60}{Terrain},
            \rotatebox{60}{Fence},
            \rotatebox{60}{Wall},
            \rotatebox{60}{Sign},
            \rotatebox{60}{Bicycle},
            \rotatebox{60}{Truck},
            \rotatebox{60}{Bus},
            \rotatebox{60}{Train},
            \rotatebox{60}{Light},
            \rotatebox{60}{Rider},
            \rotatebox{60}{Motorcycle},
        },
        axis x line=bottom,
        height = 5.2cm,
        major x tick style = transparent,
        %axis on top,
        ybar=3*\pgflinewidth,
        bar width=4pt,
        ymajorgrids = true,
        ylabel = {Ratio (\%)},
        % xlabel = {Class},
        xtick = data,
        scaled y ticks = false,
        enlarge x limits=0.3,
        axis line style={-},
        ymin=0.0,ymax=13,
        legend columns=2,
        legend cell align=left,
        legend style={
                nodes={scale=0.6},
                at={(0.5,1.2)},
                anchor=north,
                column sep=1ex
        },
        label style={font=\scriptsize},
        tick label style={font=\scriptsize}
    ]
        \addplot[style={cdeepBP,fill=cdeepBP,mark=none}] coordinates {
            (\rotatebox{60}{Road}, 8.784105735)
            (\rotatebox{60}{Building}, 12.85866043)
            (\rotatebox{60}{Vegetation}, 12.25386072)
            (\rotatebox{60}{Car}, 5.79546044)
            (\rotatebox{60}{Sidewalk}, 5.881853763)
            (\rotatebox{60}{Sky}, 3.392110938)
            (\rotatebox{60}{Pole}, 0.717479588)
            (\rotatebox{60}{Person}, 1.680341609)
            (\rotatebox{60}{Terrain}, 6.53641597)
            (\rotatebox{60}{Fence}, 3.555288862)
            (\rotatebox{60}{Wall}, 0.41288854)
            (\rotatebox{60}{Sign}, 5.917555467)
            (\rotatebox{60}{Bicycle}, 3.593630835)
            (\rotatebox{60}{Truck}, 7.823520075)
            (\rotatebox{60}{Bus}, 7.91594498)
            (\rotatebox{60}{Train}, 6.611388774)
            (\rotatebox{60}{Light}, 2.617009992)
            (\rotatebox{60}{Rider}, 1.569643873)
            (\rotatebox{60}{Motorcycle}, 2.082839412)
        };
        \addplot[style={orange,fill=orange,mark=none}] coordinates {
            (\rotatebox{60}{Road}, 6.181659949)
            (\rotatebox{60}{Building}, 11.38706875)
            (\rotatebox{60}{Vegetation}, 10.88211877)
            (\rotatebox{60}{Car}, 6.476668129)
            (\rotatebox{60}{Sidewalk}, 6.440680182)
            (\rotatebox{60}{Sky}, 3.333062128)
            (\rotatebox{60}{Pole}, 0.747530823)
            (\rotatebox{60}{Person}, 1.661918156)
            (\rotatebox{60}{Terrain}, 6.061194039)
            (\rotatebox{60}{Fence}, 3.871472033)
            (\rotatebox{60}{Wall}, 0.337991765)
            (\rotatebox{60}{Sign}, 6.279293782)
            (\rotatebox{60}{Bicycle}, 3.896345415)
            (\rotatebox{60}{Truck}, 8.368686854)
            (\rotatebox{60}{Bus}, 9.450410149)
            (\rotatebox{60}{Train}, 7.468290798)
            (\rotatebox{60}{Light}, 3.133602892)
            (\rotatebox{60}{Rider}, 1.782747288)
            (\rotatebox{60}{Motorcycle}, 2.239258099)
        };
        % \addplot[style={cdeepMF,fill=cdeepMF,mark=none}]
             % coordinates {(Road, 19.22) (Building, 21.29) (Vegetation, 21.58)};
        \legend{without class-balance, with class-balance}
    \end{axis}
\end{tikzpicture}
\caption{{\em Effect of class-balanced acquisition function.} According to the ground-truth, class labels are arranged based on the total pixel count for each class, \ie classes become rarer in images as you move from left to right along the x-axis. We observe that classes on the left are selected less with the class-balanced term, while classes on the right are selected more.}
\label{fig:class-balanced}
\end{figure*}


\section{Further analysis on the achievable metrics}
\label{fig:sup-metrics}
% \begin{figure}[!ht]
%     \captionsetup[subfigure]{font=footnotesize}
%     \centering
%     \begin{subfigure}{\linewidth}
%         \centering
%         \includegraphics[scale=0.3]{Figures/confusion_matrix_2.JPG}
%     \end{subfigure}
%     \caption{{\em Ascending 10\%, Descending 10\%, Descending 100\%.}}
%     \label{fig:descend}
%     \vspace{-2mm}
% \end{figure}

\iffalse
We describe the formulas for metrics that are present in Table \ref{tab:quantitative}, but excluded from the main text.
To properly evaluate superpixels in active learning, it is crucial to consider their size, which is related to the quantity of labels, in addition to their accuracy.
\fi


\iffalse
For example, achievable segmentation accuracy (ASA) measures the achievable accuracy of superpixels $s \in S$ in comparison to their largest overlap in oracle superpixels $g \in G$ as:
\begin{equation}
\text{ASA}(S;G) := \frac{\sum_{s \in S} \max_g |s \cap g|}{\sum_{s \in S} |s|} \;.
\label{eq:asasg}
\end{equation}
To take into account the quantity of labels, we modify the evaluation by changing the subject:
\begin{equation}
\text{ASA}(G;S) := \frac{\sum_{g \in G} \max_s |g \cap s|}{\sum_{g \in G} |g|} \;,
\end{equation}
where the denominator is the same as in \eqref{eq:asasg}, but the numerator is linked to the achievable number of labels for each oracle superpixel $g \in G$.
In addition, we can define the achievable precision (AP), recall (AR) and F1-score (AF) 
by comparing $G$ to $S$ (rather than $S$ to $G$)
as follows:
\begin{equation}
\text{AP}(G;S) := \frac{1}{|G|} \sum_{g \in G} \frac{\max_s |g \cap s|}{|g|} \;,
\end{equation}
% where $|g \cap s|$ is related to true positive (TP) and $|g|$ is related to the sum of TP and false positive (FP) between oracle $g$ and generated superpixel $s$.
\begin{equation}
\text{AR}(G;S) := \frac{1}{|G|} \sum_{g \in G} \frac{\max_s |g \cap s|}{|s'|} \;,
\end{equation}
\begin{equation}
\text{AF}(G;S) := \frac{2}{|G|} \sum_{g \in G} \frac{\max_s |g \cap s|}{|g| + |s'|} \;,
\end{equation}
where $s' := \argmax_{s \in S} \left|g \cap s \right|$ is related to the maximum amount of labels that we can obtain for oracle superpixel $g$.
\fi
In Table \ref{tab:quantitative}, we evaluate various superpixels using eight metrics with oracle superpixels as ground-truth $G$.
% We utilize oracle superpixels as ground-truth G and evaluate various superpixels with eight metrics in Table \ref{tab:quantitative}.
Figure \ref{fig:sup-correlation} shows the correlation between each metric and mIoU. 
We observe that our AF$(G;S)$ can be utilized to look-ahead a model's performance in active learning without training. 
In addition, we examine how different ground-truth $G$ impacts AF$(G;S)$.
In the field of semantic segmentation, two conventional segmentations, semantic and panoptic segmentations in Figure \ref{fig:sup-oracle-superpixels}, are widely used as ground-truth.
Figure \ref{fig:sup-afgs-g} indicates that using panoptic segmentation and oracle superpixels for $G$ results in higher correlation between AF$(G;S)$ and mIoU than semantic segmentation.
However, obtaining panoptic segmentation requires more costs than semantic segmentation since it utilizes additional instance information.
It is worth noting that our oracle superpixels (Figure \ref{subfig:oralce-seg}) can be easily generated even in cost-limited practical situations as they are produced from semantic segmentation (Figure \ref{subfig:sementic-seg}).

% \section{Effect of pixel-level class popularity}
% uncertainty

% class-balanced

% size-aware class-balanced



\begin{table*}[t!]
\centering
\setlength\tabcolsep{6pt}
\begin{tabular}{c|l}
\toprule
Notations & Description \\ \midrule
$\mathcal{I}$ & the set of unlabeled images \\ \midrule
$\mathcal{C}$ & the set of class labels \\ \midrule
$t$ & a round \\ \midrule
$x$ & a pixel \\ \midrule
$s$ & a superpixel \\ \midrule
$S_t(i)$ & the set of superpixels in an image $i$ in round $t$ \\ \midrule
$\mathcal{S}_t$ & the set of superpixels in all images in round $t$, $\mathcal{S}_t := \bigcup_{i \in \mathcal{I}} S_t(i)$ \\ \midrule
$B$ & the query budget per round \\ \midrule
$\mathcal{B}_t$ & the set of $B$ selected superpixels in round $t$, $B_t \subset \mathcal{S}_t, |B_t| = B$\\ \midrule
$\theta_t$ & the model at the end of round $t$ \\ \midrule
$y_\theta(x)$ & the estimated dominant label of pixel $x$ given $\theta$ \\ \midrule
$\text{D}(s)$ & the true dominant label of superpixel $s$ \\ \midrule
$\text{D}_\theta(s)$ & the estimated dominant label of superpixel $s$ given $\theta$ \\ \midrule
\multirow{2}{*}{$\mathcal{G}(S) := (S, \mathcal{E}(S))$} & the graph consisting of the superpixels in $S$ as nodes and
the edge set $\mathcal{E}(S)$ \\
& such that $(s, n) \in \mathcal{E}(S)$ for each pair of adjacent superpixels $s, n \in S$.  \\ \midrule
% $\mathcal{E}(S)$      & the set of edges in graph $\mathcal{G}(S)$ \\ \midrule
$\epsilon$      & the hyperparameter for merging in \eqref{eq:jsd} \\
\bottomrule
\end{tabular}
\caption{{\em Notations.} The notations used in the paper are defined.}
\label{tab:notations}
\end{table*}    


\section{Additional qualitative adaptive superpixels}
% \section{Additional qualitative analyses on the adaptive merging}
% \section{Qualitative analyses}
\label{fig:sup-qual}
To facilitate comprehension of the merged superpixels, we display superpixels generated across diverse settings.
The appearance of merged superpixels is mainly determined by the model's performance and $\epsilon$. 
Figure \ref{fig:sup-round} highlights that as the round progresses, the model's performance improves, leading to more accurate merging. 
With the model fixed at round 4, Figure \ref{fig:sup-epsilon} shows the impact of adjusting $\epsilon$.
As $\epsilon$ grows, the merging process intensifies, ultimately decreasing the overall number of superpixels.
In addition, Figure \ref{fig:sup-same-paper} shows further examples of our merged superpixels.

\section{Class-balanced sampling}
% \khy{
% Our acquisition function in \eqref{acquisition_function} gives priority to uncertain superpixels of rare classes. 
To observe the impact of the class-balanced acquisition function in~\eqref{acquisition_function}, we analyze the class distribution of selected superpixels both with and without the class-balanced term. 
In Figure~\ref{fig:class-balanced}, where class labels are sorted such that the left (road) and right (motorcycle) ends represent the most and least popular classes, it is evident that the class-balanced term results in a higher selection of rarer classes, as intended.
% }

% additional examples similar to figure 4, varying epsilon, 
% qualitative images with our metric value
% Distribution of number of superpixels
% Matching the distribution with that of oracle superpixels (or labelTrainIds..? since we are learning on them).

% \section{Relation between region size and noise}
% \begin{figure*}[!ht]
%     \captionsetup[subfigure]{font=footnotesize,labelfont=footnotesize,aboveskip=0.05cm,belowskip=-0.15cm}
%     \centering
%     \includegraphics[scale=0.6]{Figures/size_noise.JPG}
%     \caption{test}
% \end{figure*}

% Noise distribution, Heatmap hist, Correlation

\iffalse
\section{Acquisition function}
acquisition function
\fi



\iffalse
\begin{table*}[t!]
  \centering
  \setlength\tabcolsep{4pt}
  \begin{tabular}{l|ccc|c}
    \toprule
    Methods & (a) AF$(G;S)$ & (b) AF$(G;S)$ & (c) AF$(G;S)$ & mIoU \\ 
    \midrule
    $\text{SLIC}_{4096}$ & 0.196 & 0.222 & - & 53.18 \\
    $\text{SEEDS}_{4096}$ & 0.220 & 0.264 & - & 57.61 \\
    $\text{SLIC}_{256}$ & 0.082 & 0.280 & - & 58.04 \\
    $\text{SEEDS}_{256}$ & 0.088 & 0.296 & - & 58.97 \\
    \rowcolor{Gray}
    $\text{Merged}_2$ & 0.368 & 0.353 & - & \underline{60.00} \\
    \rowcolor{Gray}
    $\text{Merged}_4$ & 0.367 & 0.359 & - & \textbf{61.36} \\
    \midrule
    $\text{Merged}^*$ & 0.400 & 0.377 & - & 61.85 \\
    Oracle & 1.000 & 1.000 & 1.000 & 70.81 \\
    \bottomrule
  \end{tabular}
  % \caption{{\em Evaluation metrics of superpixels.} Superpixels are generated using SLIC \cite{achanta2012slic} and SEEDS \cite{van2012seeds}, with the subscript indicating the average size of superpixels. Our merged superpixels are evaluated, with the subscript value implying the round that used the superpixels and * representing full supervision. To compute the mIoU, we train a model with 100k randomly selected superpixels.}
  % \label{tab:quantitative}
  \vspace{-3mm}
\end{table*}
\fi

