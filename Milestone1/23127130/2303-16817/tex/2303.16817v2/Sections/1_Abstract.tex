\begin{abstract}
Learning semantic segmentation requires pixel-wise annotations, which can be time-consuming and expensive. To reduce the annotation cost, we propose a superpixel-based active learning (AL) framework, which collects a dominant label per superpixel instead. To be specific, it consists of adaptive superpixel and sieving mechanisms, fully dedicated to AL. At each round of AL, we adaptively merge neighboring pixels of similar learned features into superpixels. We then query a selected subset of these superpixels using an acquisition function assuming no uniform superpixel size. This approach is more efficient than existing methods, which rely only on innate features such as RGB color and assume uniform superpixel sizes. Obtaining a dominant label per superpixel drastically reduces annotators' burden as it requires fewer clicks. However, it inevitably introduces noisy annotations due to mismatches between superpixel and ground truth segmentation. To address this issue, we further devise a sieving mechanism that identifies and excludes potentially noisy annotations from learning. Our experiments on both Cityscapes and PASCAL VOC datasets demonstrate the efficacy of adaptive superpixel and sieving mechanisms.


% ==== version 1 ====
\iffalse
Learning-based semantic segmentation requires pixel-wise annotations, which can be time-consuming and expensive. To reduce the annotation cost, we propose a superpixel-based active learning (AL) framework, which collects a dominant label per superpixel instead. To be specific, it consists of adaptive superpixel and sieving mechanisms, fully dedicated to AL. At each round of AL, we adaptively merge neighboring pixels of similar learned features into superpixels. We then query a selected subset of these superpixels using an acquisition function assuming no uniform superpixel size. This approach is more efficient than existing methods, which rely only on innate features such as RGB color and assume uniform superpixel sizes. Obtaining a dominant label per superpixel drastically reduces annotators' burden as it requires fewer clicks. However, it inevitably introduces noisy annotations due to mismatches between superpixel and ground truth segmentation. To address this issue, we further devise a sieving mechanism that identifies and excludes noisy annotations from learning. In our experiments on both Cityscapes and PASCAL VOC datasets, we demonstrate the efficacy of adaptive superpixel and sieving mechanisms.
\fi


%\vspace{7cm}
% Active learning selectively obtains labels for the most informative samples, which can achieve comparable performance with full supervision while reducing the time and cost of annotations.
% Region-based are shown to offer better performance than whole image sampling, however, under realistic click-based annotations costs, region size becomes an important consideration.
% Superpixels can facilitate labeling by grouping surrounding pixels into perceptually meaningful regions, but their over-segmentation and size can lead to higher labeling costs and lower-quality dominant labels. To address this, the authors propose an adaptive approach to updating the pool of superpixels by merging multiple superpixels into one based on a model that is updated after each round and introducing a new evaluation metric based on achievable segmentation accuracy.

% Active learning selectively obtains labels for the most informative samples, which can achieve comparable performance with full supervision while reducing the time and cost of annotations.
% In active learning for semantic segmentation, superpixels, which group surrounding pixels into perceptually meaningful regions, have been shown to be more efficient in reducing click-based annotation costs compared to using rectangles. 
% However, the over-segmentation of superpixels can result in higher labeling costs for large regions. 
% To address this, we propose an adaptive approach to update the pool of superpixels by merging multiple superpixels based on a model that is updated after each round. Furthermore, we introduce a superpixel-wise denoising technique and a new evaluation metric for merged superpixels. 
% The results demonstrate that the proposed method achieves high segmentation accuracy while reducing annotation costs, making it a promising approach for active learning in semantic segmentation.
\end{abstract}