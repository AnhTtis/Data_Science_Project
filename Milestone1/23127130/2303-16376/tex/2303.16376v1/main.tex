% \% This is samplepaper.tex, a sample chapter demonstrating the
% LLNCS macro package for Springer Computer Science proceedings;
% Version 2.20 of 2017/10/04

% https://www.overleaf.com/3388282336khjfwqmchbjf

%
\documentclass[runningheads]{llncs}
%
\usepackage{graphicx}
\usepackage{cite}
\usepackage{times}
\usepackage{epsfig}
\usepackage{amsmath}
\usepackage{amssymb}
\usepackage{subcaption}
\usepackage{mwe}
\usepackage{acro}
\usepackage{amssymb}
\usepackage{xcolor,colortbl}
\usepackage{tabularx}
\usepackage{relsize}
\usepackage{pifont}
\usepackage{booktabs} 
\usepackage{multirow}
\usepackage{multicol}
\usepackage{adjustbox}
\usepackage{float}
\usepackage{tabularx}
\usepackage{array}
\DeclareMathSymbol{\shortminus}{\mathbin}{AMSa}{"39}
\DeclareMathOperator*{\argmax}{\arg\!\max}
\DeclareMathOperator*{\argmin}{\arg\!\min}
\usepackage[colorlinks,
            linkcolor=red,  
            anchorcolor=blue, 
            citecolor=green,   
            ]{hyperref}
%\usepackage[style=splncs04]{biblatex} %Imports biblatex package


\newcommand{\etal}{\textit{et al}. }
\newcommand{\ie}{\textit{i}.\textit{e}., }
\newcommand{\eg}{\textit{e}.\textit{g}. }
% Used for displaying a sample figure. If possible, figure files should
% be included in EPS format.
%
% If you use the hyperref package, please uncomment the following line
% to display URLs in blue roman font according to Springer's eBook style:
% \renewcommand\UrlFont{\color{blue}\rmfamily}

\begin{document}
%
\title{A Unified Single-stage Learning Model for Estimating Fiber Orientation Distribution Functions on Heterogeneous Multi-shell Diffusion-weighted MRI}
%
%\titlerunning{Abbreviated paper title}
% If the paper title is too long for the running head, you can set
% an abbreviated paper title here
% \author{submission 2306}
% \author{Tianyuan Yao\inst{1} \and
% index{Landman, Bennett\inst{1}}
% index{Huo, Yuankai\inst{1}}

\author{Tianyuan Yao\inst{1} \and
Nancy Newlin\inst{1} \and
Praitayini Kanakaraj\inst{1} \and
Vishwesh Nath\inst{3} \and
Leon Y Cai\inst{1} \and
Karthik Ramadass\inst{1} \and
Kurt Schilling\inst{2} \and
Bennett A. Landman\inst{1} \and
Yuankai Huo\inst{1}}

%
% \authorrunning{H. Yang et al.}
% % First names are abbreviated in the running head.
% % If there are more than two authors, 'et al.' is used.
% %
 % \author{submission}
 % \institute{******}
\institute{Vanderbilt University, Nashville TN 37215, USA \and
Vanderbilt University Medical Center, Nashville TN 37215, USA \and
NVIDIA Corporation, Santa Clara and Bethesda, USA\\
}
% \email{yuankai.huo@vanderbilt.edu}
%
\maketitle              % typeset the header of the contribution
%
\begin{abstract}
Diffusion-weighted (DW) MRI measures the direction and scale of the local diffusion process in every voxel through its spectrum in q-space, typically acquired in one or more shells. Recent developments in micro-structure imaging and multi-tissue decomposition have sparked renewed attention to the radial b-value dependence of the signal. Applications in tissue classification and micro-architecture estimation, therefore, require a signal representation that extends over the radial as well as angular domain. Multiple approaches have been proposed that can model the non-linear relationship between the DW-MRI signal and biological microstructure. In the past few years, many deep learning-based methods have been developed towards faster inference speed and higher inter-scan consistency compared with traditional model-based methods (e.g., multi-shell multi-tissue constrained spherical deconvolution). However, a multi-stage learning strategy is typically required since the learning process relied on various middle representations, such as simple harmonic oscillator reconstruction (SHORE) representation. In this work, we present a unified dynamic network with a single-stage spherical convolutional neural network, which allows efficient fiber orientation distribution function (fODF) estimation through heterogeneous multi-shell diffusion MRI sequences. We study the Human Connectome Project (HCP) young adults with test-retest scans. From the experimental results, the proposed single-stage method outperforms prior multi-stage approaches in repeated fODF estimation with shell dropoff and single-shell DW-MRI sequences.

\keywords{DW-MRI\and multi-shell Deep learning.}
\end{abstract}
%
%
%
\section{Introduction}
% Diffusion-weighted magnetic resonance imaging (DW-MRI) is essential for the non-invasive reconstruction of the microstructure of the human \textit{in vivo} brain. These images are sensitized to the underlying organization of the tissue at a millimetric scale. Multiple approaches have been proposed that can model the non-linear relationship between the DW-MRI signal and biological microstructure with the most common being diffusion tensor imaging (DTI) \cite{BASSER_DTI}. Substantial efforts have shown that other advanced approaches can recover more elaborate reconstruction of the microstructure \cite{MSMTCSDjeurissen2014multi}\cite{MAPMRIozarslan2013mean}\cite{HARDIdescoteaux1999high} and these methods are collectively referred to as high angular resolution diffusion imaging (HARDI). 

Diffusion-weighted magnetic resonance imaging (DW-MRI) is essential for the non-invasive reconstruction of the microstructure of the human \textit{in vivo} brain~\cite{basser1994estimation,van2012human,glasser2013minimal}. Substantial efforts have shown that other advanced approaches can recover more elaborate reconstruction of the microstructure \cite{MSMTCSDjeurissen2014multi, MAPMRIozarslan2013mean, HARDIdescoteaux1999high} and these methods are collectively referred to as high angular resolution diffusion imaging (HARDI). HARDI methods have been broadly proposed in two categories of single-shell acquisitions and multi-shell acquisitions (i.e., using multiple diffusivity values). A majority of single-shell HARDI methods utilize spherical harmonics (SH) based modeling as in q-ball imaging (QBI) \cite{QBItuch2004q}, constrained spherical deconvolution (CSD) \cite{tournier2007robust}, and many others. However, SH-based modeling cannot directly leverage additional information provided by multi-shell acquisitions as the SH transformation does not allow for a representation of the radial complexity that is introduced by the diffusivity value. SH has been combined with other bases to represent multi-shell data, e.g., solid harmonics \cite{SOLIDHARMONICSdescoteaux2011multiple}, simple harmonic oscillator reconstruction (SHORE) \cite{SHOREcheng2011theoretical}, and spherical polar Fourier imaging \cite{SPFIcheng2010model}.

% From these HARDI methods, DW-MRI parameters can be derived to reveal the microstructure. The computation of these parameters from DW-MRI data is known as DW-MRI parameter estimation, which has traditionally been achieved with model fitting. However, the fidelity of DW-MRI parameters estimated in this way is limited by relatively high noise in the data, requiring more measurements (higher gradient directions or multiple diffusivity values) to be acquired as compared to a clinical acquisition. Advanced tractography methods, such as high-definition fiber tractography (HDFT) \cite{HDFTfernandez2012high}, have been applied for utilization for neurosurgery guidance. A caveat is that HDFT requires multi-shell DW-MRI (multiple diffusivity values) acquisitions which are expensive and require more time relatively as compared to a single-shell acquisition, high b-value acquisitions also require upgraded scanner hardware. 

Deep learning (DL) has revolutionized many different domains in medical imaging, and DW-MRI parameter estimation is no different. Lots of DW-MRI methods have been developed that utilize the powerful data-driven capabilities of deep learning, yielding improved accuracy over conventional fitting when the acquisition scheme has a limited number of measurements. However, most methods are only focused on the translation of single-shell data to DW-MRI parameters and in contrast, the multi-shell methods get neglected due to the complexity associated with multi-shell data. Moreover, the SHORE-based DL methods typically used a multi-stage design. For instance, the algorithm must first optimize a specific optimal SHORE representation and then optimize the fiber orientation distribution function (fODF) estimation. Such methods are prone to overfitting, lower inference time, and complicated parameter tuning.

% Deep learning has become a powerful tool for learning approximate functions between a set of inputs and outputs where a non-linear mapping exists. Although deep learning has been quite useful in other medical imaging domains, it is still in its blossoming stage for multi-shell diffusion inputs.  Recent developments in microstructure modeling and multi-tissue decomposition have sparked renewed attention to the radial b-value dependence of the signal. 



% There is therefore a need for compact representations of the DW-MRI signal that extend over the radial as well as angular domain.

As shown in Fig~\ref{fig:problem}, in this paper, we propose a single-stage dynamic network with both the q-space and radial space signal based on a spherical convolutional neural network. We evaluated the resultant representation by targeting it to multi-shell multi-tissue CSD (MSMT-CSD). Both fiber orientation estimation and recovery of tissue volume fraction are evaluated. The contribution of this paper is three-fold:

$\bullet$ We proposed a unified dynamic network with the single-stage spherical convolutional neural network that can recover/predict microstructural measures.

$\bullet$ The proposed method is universally applicable to perform learning-based fODF estimation using a single deep model for various combinations of multiple shells.

$\bullet$ The proposed method achieved an overall superior performance compared with model-based and data-driven benchmarks.

\begin{figure}[t]
\begin{center}
\includegraphics[width=1\linewidth]{fig/problemidea.pdf}
\end{center}
   \caption{Utilizing multi-shell DW-MRI signals in deep learning usually requires independent models trained for each specific shell configuration as conventional SH-based modeling cannot directly leverage additional information (radial space) provided by multi-shell acquisitions. In our study, the dynamic head aims to improve the network expressiveness by learning and adaptively adjusting the first convolution layer for different shell configurations. }
\label{fig:problem}
\end{figure}


\section{Related work}
\subsection{Multi-Shell Multi-Tissue Constrained Spherical Deconvolution}
Spherical deconvolution \cite{SDanderson2005measurement} is a set of methods to reconstruct the local fODF from DW-MRI data. They have become a popular choice for recovering the fiber orientation due to their ability to resolve fiber crossings with small inter-fiber angles in datasets acquired within a clinically feasible scan time. SD methods are based on the assumption that the acquired diffusion signal in each voxel can be modeled as a spherical convolution between the fODF and the fiber response function (FRF), which describes the common signal profile from the white matter (WM) bundles contained in the voxel. The term "constrained" in CSD \cite{tournier2007robust} refers to constraints that are placed on the solution. The goal of these constraints is to ensure that the estimated fiber orientations are biologically plausible and to avoid overfitting noise in the data.


Multi-Shell Multi-Tissue Constrained Spherical Deconvolution (MSMT-CSD) \cite{MSMTCSDjeurissen2014multi} is a technique developed to overcome the limitations of traditional single-shell diffusion MRI methods, which are unable to resolve the complex fiber orientations of multiple tissue types in the brain. MSMT-CSD is able to separate the contribution of different tissue types (such as gray matter, white matter, and cerebrospinal fluid) to the diffusion signal by modeling the diffusion signal as a combination of multiple shells with different b-values. This modeling-based method has been a conventional method for multi-tissue micro-architecture estimation. 


\subsection{Learning-based estimation}
Recently, machine learning (ML) and deep learning (DL) techniques have demonstrated their remarkable abilities in neuroimaging. Such approaches have been applied to the task of microstructure estimation \cite{nath2019deep}, aiming to directly learn the mapping between input DW-MRI scans and output fiber tractography~\cite{schilling2021fiber,cai2023convolutional} while maintaining the necessary biological characteristics and reproducibility for clinical translation. Such studies have illustrated that DL is a promising tool that uses nonlinear transforms to extract features from high-dimensional data. Data-driven approaches can be useful in validating the hypothesis of the existence of untapped information because they generalize toward the ground truth.

\subsection{Model based representation}
The SHORE basis function has been shown to capture the representation of multi-shell DW-MRI with minimal representation error and ensure the same when modeling single-shell DW-MRI. SHORE \cite{SHOREcheng2011theoretical} provides a more detailed characterization of the diffusion process by modeling the water diffusion profile as a combination of multiple frequencies, each representing a different spatial scale of motion. This allows for a more accurate and detailed representation of the underlying tissue microstructure. In our study, we applied the SHORE basis function as the baseline representation for multi-shell DW-MRI.
 


\section{Methods}
\subsection{Preliminaries}
Traditional deep learning frameworks are not generalizable to new acquisition schemes. This complicates the application of a DL model to data acquired from multiple sites. Our model aims to train a DL framework that can be adapted to an arbitrary number of available multi-shell DW-MRI sequences. To serve this motivation, we employ a dynamic head (DH) design to handle the multi-shell problem on the three most common diffusivity values: 1000, 2000, and 3000 $s/mm^{2}$. Additionally, to tackle the problem of a varying number of gradient directions on each shell (diffusivity value), we leverage the spherical CNNs with the traditional 'modeling then feeding to FCN' strategy. In this study, we employ the fODF estimation as our chosen task to perform assessments on different methods.
 
\subsection{Dynamic head design}
In our model, a single set of model parameters $\theta$ is learned to handle all permutations of a different number of shells with the most commonly known b-values. Note that with $K$ shells there are $2^K\shortminus1$ configurations. To improve the network expressiveness, we devise a dynamic head $D$ to adaptively generate model parameters conditioned on the availability of input shells. We use a binary code $m\in\mathbb{R}^K$ where 0/1 represent the absence/presence of each shell. To mitigate the large input variation caused by artificially zero-ed channels, we use the dynamic head to generate the parameters for the first convolutional layer.

\subsection{Spherical Convolution}
To extract features from DW-MRI signals, the first and most common deep learning network architecture applied to dMRI is the FCN~\cite{aliotta2019highly, nath2019inter}, Conventionally these have been implemented following:

\begin{equation}\label{loss}
\begin{split}
        t = F_{FCN}(s; \theta_{FCN})\\
\end{split}
\end{equation}

where $F_{FCN}$ is a fully-connected network with trainable parameters $\theta_{FCN}$. The dMRI signals serve as the input for the network, and it does not consider the acquisition information, making the network unaware of the acquisition scheme. This lack of knowledge poses an issue when incorporating new data acquired at a different location with a distinct acquisition scheme. The accuracy of estimation from a new set of DW-MRIs depends on the consistency of the acquisition scheme with the training set. Additionally, the FCN's design does not account for rotational equivariance, which could result in requiring a varied range of tissue microstructure orientations in the training dataset for accurate estimation independent of fiber orientation.

Theoretically, Spherical CNNs offer an advantage over FCNs regarding both the robustness of the gradient scheme and the distribution of training data~\cite{ goodwin2022can,sedlar2021spherical}. The Spherical CNN's architecture differs from FCNs, but not in the conventional sense. Instead of convolution across multiple voxels, Spherical CNNs perform convolution over the spherical image space. Hence, like FCNs, they are voxelwise networks. At each voxel, the spherical image is created from the dMRI signals and their corresponding gradient scheme. This architecture can naturally address the limitations of FCNs in two ways. Firstly, Spherical CNNs are aware of the gradient scheme in their input, which is not the case with FCNs, as demonstrated in the equation below:

\begin{equation}\label{loss}
\begin{split}
        t = F_{S-CNN}(s, G; \theta_{S-CNN})\\
\end{split}
\end{equation}

where $F_{S-CNN}$ is a Spherical CNNs with trainable parameters $\theta_{S-CNN}$. In Spherical CNNs, the gradient scheme is explicitly taken into account in the input, enabling them to better handle variations in the gradient scheme that can occur between different acquisition protocols or imaging sites. Additionally, Spherical CNNs can better handle the distribution of training data, as they are naturally suited to working with data on a sphere. Overall, these benefits of spherical convolution can lead to improved accuracy and robustness in the analysis of diffusion MRI signals.

Suppose $L_{t}$ is a loss function designed for the downstream task.  Let $x$ be the input of the framework. During training, the shared network $F_{S-CNN}$ is trained using the input data and there are $2^K\shortminus1$ unique shell configurations, represented by: 

 \begin{equation}
    \tilde{x}^{k}=\delta^{k} x^{k},
%    i \in \{1,...,2^{K}\shortminus1\},
    k \in \{1,...,K\},
\end{equation}
 
 By targeting different diffusion properties $y$ and with the dynamic head setting. The learning objective can be expressed as: 
 
 \begin{equation}\label{loss}
\begin{split}
 \argmin_{\theta_{s}} L_{t}(y, F(\tilde{x}_i|\theta_{S-CNN}))
 \end{split}
\end{equation}


\section{Experiments}



\subsection{Data and Implementation Details}
We have chosen DW-MRI from the Human Connectome Project - Young Adult (HCP-ya) dataset~\cite{van2013wu, glasser2013minimal}, 45 subjects with the scan-rescan acquisition were used (a total of 90 images). The acquisitions had b-values of 1000, 2000, 3000 $s/mm^{2}$ with 90 gradient directions on each shell. A T1 volume of the same subject was used for WM segmentation using SLANT\cite{SLANT}. All HCP-ya DW-MRI was distortion corrected with top-up and eddy\cite{jenkinson2012fsl}. 30 subjects are used as training data while 10 subjects were used as evaluation and 5 subjects as testing data.

We performed shell extraction on all the data. Every subject has seven different shell configurations which are the permutations of all three b-values $\{$$\{1K, 2K, 3K\}$, $\{1K, 2K\}$, $\{2K, 3K\}$, $\{1K, 3K\}$, $\{1K\}$, $\{2K\}$, $\{3K\}$$\}$. Ground-truth fODF maps were computed from MSMT-CSD using the DIPY library with the default settings ~\cite{DIPY}. $8^{th}$ order SH were chosen for data representation with the 'tournier07' basis~\cite{tournier2007robust}. The white matter fODF and the volume fraction which refers to the proportion of the volume of the voxel that is occupied by each tissue type, are combined together as the targeted sequence. The $6^{th}$ radial order SHORE basis is employed as a baseline representation, the SHORE scaling factor $\zeta$ defined in units of $ mm^{-2}$ as $\zeta = 1/8\pi^{2}\tau MD$ is calculated based on the mean diffusivity (MD) obtained from the data.

\subsection{Experimental setting}
We first trained separate models for each shell configuration. The models consist of four fully connected layers with ReLU activation function. The number of neurons per layer is 400, 48, 200, and 48. The input is the $1 \times 50$ vector of the shore basis signal ODF, and the output is the combination $1 \times 45$ vectors of the SH basis WM fODF and the $1 \times 3$ vector of tissues fraction. The models are then tested on the different shell configurations. By simply feeding all the shell configuration data (all labeled with reconstructed fODF from data with all shells) to the FCN as a baseline 'unified' deep learning model.

We assess the impact of dynamic head strategy by evaluating the performance of the unified models against independent models trained for each specific shell configuration. Furthermore, the generalizability of the different representations with dynamic head designs was assessed. For the spherical convolution, we used an architecture known as the hybrid spherical CNN as described in ~\cite{cobb2020efficient}. After the rotational invariant features are extracted. They are concatenated and fed into fully connected layers(the same hidden size as above) which perform the final estimation.


\subsection{Evaluation metric}
To evaluate the predictions from the proposed methods, we calculated the mean squared error of the volume fractions with ground truth sequences. Then we compute the angular correlation coefficient (ACC, Eq.~\ref{ACC}) between the predicted fODF and the ground truth fODF over the white matter region. ACC is a generalized measure for all fiber population scenarios. It assesses the correlation of all directions over a spherical harmonic expansion. In brief, it provides the estimate of how closely a pair of fODFs are related on a scale of -1 to 1, where 1 is the best measure. Here ‘u’ and ‘v’ represent sets of SH coefficients.

\begin{equation}\label{ACC}
\begin{split}
ACC= \frac{\sum_{k=1}^{L}\sum_{m=-k}^{k}(u_{km})(v^*_{km})}{[\sum_{k=1}^{L}\sum_{m=-k}^{k}|u_{km}|^2]^{0.5}\cdot[\sum_{k=1}^{L}\sum_{m=-k}^{k}|v_{km}|^2]^{0.5}} 
\end{split}
\end{equation}

\section{Experimental Results}
We compared the performances of the unified models against independent models trained for each specific shell configuration. A qualitative result of fODF predictions and GT are shown in Fig~\ref{fig: vis}. As shown in Table~\ref{table:model1}, the independent models that are trained are thus more likely to outperform others in their own shell configuration and these models can be considered as the upper bounds for each shell configuration. With the dynamic head settings, the unified model with spherical convolution outperforms the other models in the single shell configuration. Additionally, the ACC is a sensitive generalized metric, the performances need further evaluation. We assessed how good our prediction was by evaluating the scan/rescan consistency and volume fraction prediction~\ref{table:model2}.


\begin{figure}[t]
\begin{center}
\includegraphics[width=0.95\linewidth]{fig/vis.pdf}
\end{center}
   \caption{This is a visualization of the fODF prediction and the correlation with the GT in different views. The background of the zoom-in patches shows the ACC spatial map with the GT signals.}
\label{fig: vis}
\end{figure}


\newcolumntype{P}[1]{>{\centering\arraybackslash}p{#1}}
\begin{table}[htbp]
\caption{Performances of the unified models against independent models in different shell configurations}
\centering
\small
\begin{tabular}{|P{2.6cm}|P{1cm}|P{1cm}|P{1cm}|P{1cm}|P{1cm}|P{1cm}|P{1cm}|P{1.4cm}|}
 \hline
 \textbf{Model} & \textbf{$C_{1}$} & \textbf{$C_{2}$} & \textbf{$C_{3}$} & \textbf{$C_{1,2}$} & \textbf{$C_{2,3}$} & \textbf{$C_{1,3}$} & \textbf{$C_{1,2,3}$} & \textbf{Ave.} \\ \hline 
 $M_{1}$ & \textcolor{blue}{0.808} & 0.725 & 0.732 & 0.752 & 0.734 & 0.751 & 0.788 & 0.756 \\
$M_{2}$ & 0.762 & \textcolor{blue}{0.815} & 0.756 & 0.744 & 0.749 & 0.745 & 0.774 & 0.764 \\
$M_{3}$ & 0.757 & 0.724 & \textcolor{blue}{0.814} & 0.734 & 0.753 & 0.760 & 0.779 & 0.760 \\
$M_{1,2}$ & 0.745 & 0.732 & 0.743 & \textcolor{red}{ 0.831} & 0.788 & 0.778 & 0.789 & 0.772 \\
$M_{2,3}$ & 0.734 & 0.744 & 0.738 & 0.802 & \textcolor{blue}{0.825} & 0.786 & 0.786 & 0.774 \\
$M_{1,3}$ & 0.737 & 0.745 & 0.745 & 0.785 & 0.793 & \textcolor{red}{ 0.832} & 0.784 & 0.774 \\
$M_{1,2,3}$ & 0.752 & 0.734 & 0.742 & 0.762 & 0.756 & 0.772 & \textcolor{red}{ 0.853} & 0.767 \\ \hline
All Data Feeding & 0.789 & 0.793 & 0.794 & 0.801 & 0.799 & 0.803 & 0.814 & 0.799 \\
DH w. SHORE~\cite{SHOREcheng2011theoretical} & 0.782 & 0.788 & 0.784 & 0.823 & 0.817 & \textcolor{blue}{0.825} & \textcolor{blue}{0.843} & 0.809 \\ DH w. SH & 0.805 & 0.809 & \textcolor{blue}{0.814} & 0.818 & 0.812 & 0.812 & 0.832 & \textcolor{blue}{0.815} \\
DH w. SC (Ours) & \textcolor{red}{ 0.816} & \textcolor{red}{ 0.82} & \textcolor{red}{ 0.816} & \textcolor{blue}{0.827} & \textcolor{red}{ 0.828} & 0.824 & 0.837 & \textcolor{red}{ 0.824} \\ \hline
\end{tabular}
\caption*{$M_{i}$, where i $\in [1,2,3]$ indicates the model is only trained on that shell configuration. $C_{i}$ indicates the testing data in that shell configuration. The best and second best performances are denoted by the \textcolor{red}{red} mark and \textcolor{blue}{blue} mark. The average metrics of ACC are listed in the last column. }
\label{table:model1}
\end{table}

\begin{table}[htbp]
\caption{FODF prediction assessment}
\centering
\small
\begin{tabular}{|P{2.5cm}|P{3cm}|P{2.5cm}|P{3.5cm}|}
 \hline
 \textbf{Model} & \textbf{Shell configuration} & \textbf{Tissue proportion prediction} & \textbf{Scan-rescan consistency}  \\ \hline 
 \multirow{6}{*}{Single model} & 1K & 8.45E-04 & 0.862 \\
  & 2K & 7.92E-04 & 0.865 \\
 & 3K & 8.63E-04 &  0.857 \\
 & 1K, 2K & 7.32E-04 & 0.856 \\
 & 2K, 3K & 7.49E-04 & 0.86 \\
 & 1K, 3K & 8.02E-04 & 0.862 \\
 & 1K, 2K, 3K & 6.38E-04 & 0.865 \\   \hline
\multirow{6}{*}{DH w. SC}  & 1K & 7.27E-04 & 0.855 \\
  & 2K & 7.12E-04 & 0.86 \\ 
  & 3K & 7.35E-04 & 0.858  \\
  & 1K, 2K & 7.01E-04 & 0.858 \\ 
 & 2K, 3K & 6.79E-04 & 0.86 \\ 
 & 1K, 3K & 6.82E-04 & 0.864 \\
 & 1K, 2K, 3K & 5.92E-04 & 0.861 \\    \hline
\multicolumn{2}{|c|}{Silver standard : MSMT-CSD~\cite{MSMTCSDjeurissen2014multi}} &  & \textcolor{red}{0.856} \\ \hline
\end{tabular}
\label{table:model2}
\caption*{Reconstruction result from msmt-CSD are applied as silver standard in the evaluation. Wilcoxon signed-rank test is applied as statistical assessment for scan-rescan consistency evaluation. It has significant difference ($p < 0.001$) compared with WM fODF. The MSE is reported for evaluation of VF predictions. The ACC between scan/rescan DW-MRI over WM regions is reported.}
\end{table}


\section{Conclusion}
In this paper, we propose a single-stage dynamic network with both the q-space and radial space signal based on a spherical convolutional neural network. Integrating dynamic head and spherical convolution removes the need to retrain a new network for a known diffusivity value of DW-MRI. Besides, adjusting the last multi-layer regression network to different targets, this plug-and-play design of our method is potentially applicable to a wider range of diffusion properties in neuroimaging.

% ~\\
% \noindent\textbf{Acknowledgements}. *************





% \textbf{Acknowledgements}:
% *****


% ---- Bibliography ----
%
% BibTeX users should specify bibliography style 'splncs04'.
% References will then be sorted and formatted in the correct style.
%
% \clearpage
\bibliographystyle{splncs04}
\bibliography{main}
%\printbibliography %Prints bibliography
%
% \begin{thebibliography}{8}
% \bibitem{ref_article1}
% Author, F.: Article title. Journal \textbf{2}(5), 99--110 (2016)

% \bibitem{ref_lncs1}
% Author, F., Author, S.: Title of a proceedings paper. In: Editor,
% F., Editor, S. (eds.) CONFERENCE 2016, LNCS, vol. 9999, pp. 1--13.
% Springer, Heidelberg (2016). \doi{10.10007/1234567890}

% \bibitem{ref_book1}
% Author, F., Author, S., Author, T.: Book title. 2nd ed. Publisher,
% Location (1999)

% \bibitem{ref_proc1}
% Author, A.-B.: Contribution title. In: 9th International Proceedings
% on Proceedings, pp. 1--2. Publisher, Location (2010)

% \bibitem{ref_url1}
% LNCS Homepage, \url{http://www.springer.com/lncs}. Last accessed 4
% Oct 2017
% \end{thebibliography}
\end{document}
