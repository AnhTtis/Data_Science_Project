
\clearpage

\setcounter{page}{1}
\setcounter{section}{0}
\setcounter{equation}{0}
\setcounter{table}{0}
\setcounter{figure}{0}

\renewcommand{\thepage}{S\arabic{page}}
\renewcommand{\thesection}{S\arabic{section}}
\renewcommand{\theequation}{S\arabic{equation}}
\renewcommand{\thetable}{S\arabic{table}}
\renewcommand{\thefigure}{S\arabic{figure}}

\onecolumngrid
\section*{Supplemental Material: \\
QMC-consistent static spin and density local field factors for the uniform electron gas}
\twocolumngrid

\tableofcontents

\onecolumngrid

\section{Fitting procedure \label{sec:fit_method}}

To fit the local field factors (LFFs) $G_\pm$, we performed a least squares fit using the SciPy package \cite{virtanen2020}.
The sum of squared residuals
\begin{equation}
  \chi_\pm^2 = \sum_{i,j} \left| \frac{G_\pm(\rs^{(i)},q_j) - G_\pm^\text{QMC}(\rs^{(i)},q_j)}{\delta G_\pm^\text{QMC}(\rs^{(i)},q_j)} \right|^2
\end{equation}
was minimized.
$G_\pm^\text{QMC}$ is the LFF computed from QMC, and $\delta G_\pm^\text{QMC}$ is its uncertainty.
For $G_+$, we fit to the Kukkonen-Chen \cite{kukkonen2021} data for $G_+(0 \leq q_j/\kf < 2.5)$ at $\rs^{(i)} \in \{ 1, 2\} $; and to the Moroni \textit{et al.} \cite{moroni1995} data for $G_+(1 < q_j/\kf < 4.25)$ at $\rs^{(i)} \in \{ 5, 10\}$.
For $G_-$, we fit only to the Kukkonen-Chen \cite{kukkonen2021} data for $G_-(0 \leq q_j/\kf < 2.5)$ at $\rs^{(i)} \in \{ 1, 2, 3, 4, 5 \}$.

\begin{table}[h]
  \begin{tabular}{rrrrrr} \hline
    & QMC \cite{chen2019,kukkonen2021} & \multicolumn{2}{c}{PW92} & \multicolumn{2}{c}{This work} \\
    $r_\mathrm{s}$ & & $\chi_s/\chi_P$ & PD (\%) & $\chi_s/\chi_P$ & PD (\%) \\ \hline
    1 & 1.152(2) & 1.153425 & 0.12 & 1.153466 & 0.13 \\
    2 & 1.296(6) & 1.299474 & 0.27 & 1.299030 & 0.23 \\
    3 & 1.438(9) & 1.442503 & 0.31 & 1.439717 & 0.12 \\
    4 & 1.576(9) & 1.583653 & 0.48 & 1.575237 & -0.05 \\
    5 & 1.683(15) & 1.723687 & 2.39 & 1.705048 & 1.30 \\
  \hline
  \end{tabular}
  \caption{Values of the spin-susceptibility enhancement calculated in Refs. \cite{chen2019,kukkonen2021}, here by Eq. (\ref{eq:chi_enh}) using the Perdew-Wang (PW92) parameterization \cite{perdew1992} of the UEG correlation energy density, and in this work using a revised parameterization of the Perdew-Wang form.
  The percent difference (PD) in quantities $x$ and $y$ is defined here as $(200\%)(x-y)/(x+y)$, i.e., the difference of $x$ and $y$ weighted by their average.}
  \label{tab:chi_enh}
\end{table}

To fit the correlation spin-stiffness $\alpha\suc$, we performed a least-squares \cite{virtanen2020} minimization of the objective function
\begin{align}
  \sigma =& \sum_i \left|
    \frac{\widetilde{\chi}_\text{approx}(\rs^{(i)}) - \widetilde{\chi}_\text{QMC}(\rs^{(i)})}{\delta \widetilde{\chi}_\text{QMC}(\rs^{(i)})}
    \right|^2
    + \sum_i \left|
    \frac{\alpha\suc(\rs^{(i)}) - \alpha\suc^\text{QMC}(\rs^{(i)})}{\delta \alpha\suc^\text{QMC}(\rs^{(i)})}
  \right|^2.
\end{align}
$\widetilde{\chi}_\text{QMC}=\chi_s^\text{QMC}/\chi_s^{(0)}$, and $\delta \widetilde{\chi}_\text{QMC}$ is its uncertainty, for $\rs^{(i)} = 1,2,3,4,5$.
$\widetilde{\chi}_\text{approx} = \chi_s^\text{approx}/\chi_s^{(0)}$ is computed using Eqs. (\ref{eq:chi_enh}) and (\ref{eq:alpha_c_pw92}).
Using the Perdew-Zunger \cite{perdew1981} ansatz for the spin-dependence of the correlation energy,
\begin{align}
  \varepsilon\suc(\rs,\zeta) &= \varepsilon\suc(\rs,0)
    + f(\zeta)\left[ \varepsilon\suc(\rs,1)
    - \varepsilon\suc(\rs,0) \right], \\
  f(\zeta) &= \frac{(1 + \zeta)^{4/3} + (1 - \zeta)^{4/3} - 2}{2^{4/3} - 2},
\end{align}
we have approximated
\begin{align}
  \alpha\suc^\text{QMC}(\rs) &= f''(0)\left[
    \varepsilon\suc^\text{QMC}(\rs,1)
  - \varepsilon\suc^\text{QMC}(\rs,0) \right], \\
  \delta \alpha\suc^\text{QMC}(\rs) &= f''(0)\left\{
    \left[\delta \varepsilon\suc^\text{QMC}(\rs,1)\right]^2
    + \left[\delta \varepsilon\suc^\text{QMC}(\rs,0)\right]^2
    \right\}^{1/2},
\end{align}
with $\varepsilon\suc^\text{QMC}$ the accurate correlation energies from Table VI of Ref. \cite{azadi2022}, and $\delta \varepsilon\suc^\text{QMC}$ their uncertainties.
A few values of the spin susceptibility enhancement predicted by QMC, PW92, and the present work are presented in Table \ref{tab:chi_enh}.

\vspace{5mm}
\twocolumngrid

\section{Evaluation of the fit quality at all values of $\rs$}

This section presents figures analogous to Figs. \ref{fig:gplus_rs_2} and \ref{fig:gminus_rs_4} of the main text, but for the other values of $\rs$ used to fit $G_\pm(q)$.
For $G_+(\rs,q)$, these are for $\rs \in \{ 1, 5, 10 \}$ in Figs. \ref{fig:gp_rs_1}--\ref{fig:gp_rs_10}.
For $G_-(\rs,q)$, these are for $\rs \in \{ 1,2,3,5\}$ in Figs. \ref{fig:gm_rs_1}--\ref{fig:gm_rs_5}.

\subsection{Static density local field factor}

\nopagebreak
\begin{figure}[h]
  \centering
  \includegraphics[width=\columnwidth]{./figs/gplus_rs_1_2p.pdf}
  \caption{
  Comparison of the model $G_+$ of Eq. (\ref{eq:g_app}) (blue curve) and Table \ref{tab:fpars} with the QMC data of Ref. \cite{kukkonen2021} (black points with vertical uncertainties) for $\rs = 1$.
  Panel (a) presents $G_+$ and (b) $4\pi G_+ (\kf/q)^2 = \kf^2 f\suxc(q)$.
  Also shown are the LFFs of Corradini \textit{et al.} \cite{corradini1998} (gray, dash-dotted), which is fitted to the data of Ref. \cite{moroni1995}, and of RA \cite{richardson1994} (red, dashed).
  The small-$q$ expansion (SQE) of Eq. (\ref{eq:gp_small_q}) (orange, dotted) and large-$q$ expansion (LQE) of Eq. (\ref{eq:gp_large_q}) (green, dotted) are also shown.
  }
  \label{fig:gp_rs_1}
\end{figure}

\begin{figure}[h]
  \centering
  \includegraphics[width=\columnwidth]{./figs/gplus_rs_5_2p.pdf}
  \caption{
  Comparison of the model $G_+$ of Eq. (\ref{eq:g_app}) (blue curve) and Table \ref{tab:fpars} with the QMC data of Ref. \cite{moroni1995} (magenta points with vertical uncertainties) for $\rs=5$.
  Panel (a) presents $G_+$ and (b) $4\pi G_+ (\kf/q)^2 = \kf^2 f\suxc(q)$.
  Also shown are the LFFs of Corradini \textit{et al.} \cite{corradini1998} (gray, dash-dotted), which is fitted to the data of Ref. \cite{moroni1995}, and of RA \cite{richardson1994} (red, dashed).
  The small-$q$ expansion (SQE) of Eq. (\ref{eq:gp_small_q}) (orange, dotted) and large-$q$ expansion (LQE) of Eq. (\ref{eq:gp_large_q}) (green, dotted) are also shown.
  }
  \label{fig:gp_rs_5}
\end{figure}

\begin{figure}[h]
  \centering
  \includegraphics[width=\columnwidth]{./figs/gplus_rs_10_2p.pdf}
  \caption{
  Same as Fig. \ref{fig:gp_rs_5}, but for $\rs = 10$.
  }
  \label{fig:gp_rs_10}
\end{figure}

\clearpage
\subsection{Static spin local field factor}

\begin{figure}[h]
  \centering
  \includegraphics[width=\columnwidth]{./figs/gminus_rs_1_2p.pdf}
  \caption{
  Comparison of the model $G_-$ of Eq. (\ref{eq:g_app}) (blue curve) and Table \ref{tab:fpars} with the QMC data of Ref. \cite{kukkonen2021} (black points with vertical uncertainties) for $\rs = 1$.
  Panel (a) presents $G_-$ and (b) $4\pi G_- (\kf/q)^2 = \kf^2 f\suxc(q)$.
  The RA expression for $G_-$ \cite{richardson1994} (red, dashed), the small-$q$ expansion (SQE) of Eq. (\ref{eq:gp_small_q}) (orange, dotted), and large-$q$ expansion (LQE) of Eq. (\ref{eq:gp_large_q}) (green, dotted) are also shown.
  }
  \label{fig:gm_rs_1}
\end{figure}

\nopagebreak
\begin{figure}
  \centering
  \includegraphics[width=\columnwidth]{./figs/gminus_rs_2_2p.pdf}
  \caption{
  Same as Fig. \ref{fig:gm_rs_1}, but for $\rs = 2$.
  }
  \label{fig:gm_rs_2}
\end{figure}

\begin{figure}
  \centering
  \includegraphics[width=\columnwidth]{./figs/gminus_rs_3_2p.pdf}
  \caption{
  Same as Fig. \ref{fig:gm_rs_1}, but for $\rs = 3$.
  }
  \label{fig:gm_rs_3}
\end{figure}

\begin{figure}
  \centering
  \includegraphics[width=\columnwidth]{./figs/gminus_rs_5_2p.pdf}
  \caption{
  Same as Fig. \ref{fig:gm_rs_1}, but for $\rs = 5$.
  }
  \label{fig:gm_rs_5}
\end{figure}

\clearpage

\section{Quality of extrapolation}

This section presents the \textit{predictions} of the model LFFs for the shapes of $G_\pm(q)$ at values of $\rs$ for which they are not fitted.
This gauges the quality of extrapolation and reliability of this model for jellium at any density.

For both $G_\pm(q)$, we show extrapolations to an extremely high density, $\rs = 0.1$ in Figs. \ref{fig:gp_rs_0p1} and \ref{fig:gm_rs_0p1}, and to an extremely low density, $\rs = 100$ in Figs. \ref{fig:gp_rs_100} and \ref{fig:gm_rs_100}.

\subsection{Static density local field factor}

\begin{figure}[h]
  \centering
  \includegraphics[width=\columnwidth]{./figs/gplus_rs_0.1_2p.pdf}
  \caption{
  Extrapolation of the model $G_+$ to $\rs=0.1$.
  Panel (a) presents $G_+$ and (b) $4\pi G_+ (\kf/q)^2 = \kf^2 f\suxc(q)$.
  Also shown are the LFFs of Corradini \textit{et al.} \cite{corradini1998} (gray, dash-dotted), which is fitted to the data of Ref. \cite{moroni1995}, and of RA \cite{richardson1994} (red, dashed).
  The small-$q$ expansion (SQE) of Eq. (\ref{eq:gp_small_q}) (orange, dotted) and large-$q$ expansion (LQE) of Eq. (\ref{eq:gp_large_q}) (green, dotted) are also shown.
  }
  \label{fig:gp_rs_0p1}
\end{figure}

\begin{figure}[h]
  \centering
  \includegraphics[width=\columnwidth]{./figs/gplus_rs_100_2p.pdf}
  \caption{
  Same as Fig. \ref{fig:gp_rs_0p1}, but for $\rs = 100$.
  }
  \label{fig:gp_rs_100}
\end{figure}


\clearpage
\subsection{Static spin local field factor}

\begin{figure}[h]
  \centering
  \includegraphics[width=\columnwidth]{./figs/gminus_rs_0.1_2p.pdf}
  \caption{
  Extrapolation of the model $G_-$ to $\rs = 0.1$.
  Panel (a) presents $G_-$ and (b) $4\pi G_- (\kf/q)^2 = \kf^2 f\suxc(q)$.
  The RA expression for $G_-$ \cite{richardson1994} (red, dashed), the small-$q$ expansion (SQE) of Eq. (\ref{eq:gp_small_q}) (orange, dotted), and large-$q$ expansion (LQE) of Eq. (\ref{eq:gp_large_q}) (green, dotted) are also shown.
  }
  \label{fig:gm_rs_0p1}
\end{figure}

\begin{figure}[h]
  \centering
  \includegraphics[width=\columnwidth]{./figs/gminus_rs_100_2p.pdf}
  \caption{
  Same as Fig. \ref{fig:gm_rs_0p1}, but for $\rs = 100$.
  }
  \label{fig:gm_rs_100}
\end{figure}

\clearpage
\onecolumngrid

\section{Surface plots of the local field factors}

This section presents surface plots of $G_+(\rs,q)$ as a function of $q/\kf$ and $\rs$, with comparisons to the Corradini \textit{et al.} LFF in Fig. \ref{fig:surf_gp_corr}, and to the Richardson-Ashcroft (RA) LFF in Fig. \ref{fig:surf_gp_ras}.
The model of $G_-(\rs,q)$ developed here and the model of RA are compared in Fig. \ref{fig:surf_gm_ras}.

\begin{figure}[h]
  \includegraphics[width=0.8\columnwidth]{./figs/gp_corr.pdf}
  \caption{Surface plot of (a) the model $4\pi G_+(\rs,q)(\kf/q)^2$ of this work and (b) of Corradini \textit{et al.} \cite{corradini1998}.
  Both are shown as functions of $0 \leq q/\kf \leq 4$ and in the metallic range $2 \leq \rs \leq 10$.
  }
  \label{fig:surf_gp_corr}
\end{figure}

\begin{figure}[h]
  \includegraphics[width=0.8\columnwidth]{./figs/gp_RAS.pdf}
  \caption{Surface plot of (a) the model $4\pi G_+(\rs,q)(\kf/q)^2$ of this work and (b) of Richardson and Ashcroft (RA) \cite{richardson1994}.
  Both are shown as functions of $0 \leq q/\kf \leq 4$ and in the metallic range $2 \leq \rs \leq 10$.
  }
  \label{fig:surf_gp_ras}
\end{figure}

\begin{figure}[h]
  \includegraphics[width=0.8\columnwidth]{./figs/gm_RAS.pdf}
  \caption{Surface plot of (a) the model $4\pi G_-(\rs,q)(\kf/q)^2$ of this work and (b) of Richardson and Ashcroft (RA) \cite{richardson1994}.
  Both are shown as functions of $0 \leq q/\kf \leq 4$ and in the metallic range $2 \leq \rs \leq 10$.
  }
  \label{fig:surf_gm_ras}
\end{figure}

\section{Computation of correlation energies \label{sec:eps_c_method}}

To compute correlation energies per electron for a spin-unpolarized jellium, $\varepsilon\suc(\rs,\zeta=0)$, we use the standard coupling-constant integration \cite{lein2000}
\begin{equation}
  \varepsilon\suc(\rs,\zeta=0) = -3 \int_0^\infty d \left(\frac{q}{\kf} \right) \int_0^1 d \lambda \int_0^\infty
    d \left( \frac{u}{\kf^2} \right) \frac{\left[\chi_0(q,i u) \right]^2 f_\mathrm{Hxc}^{(\lambda)}(q,i u)}{1 - \chi_0(q,i u) f_\mathrm{Hxc}^{(\lambda)}(q,i u)}.
    \label{eq:acfdt}
\end{equation}
$\chi_0$ is the non-interacting or Kohn-Sham response function.
When evaluated for the UEG, it also known as the Lindhard function \cite{lindhard1954},
\begin{equation}
  \chi_0(q,i u) = \frac{\kf}{2\pi^2} \left\{
    \frac{z^2 - U^2 - 1}{4 z} \ln \left[\frac{U^2 + (z + 1)^2}{U^2 + (z - 1)^2} \right]
    -1 + U \arctan \left(\frac{1 + z}{U} \right)
    + U \arctan \left(\frac{1 - z}{U} \right)
  \right\},
\end{equation}
where $z = q/(2 \kf)$ and $U \equiv u/(q\kf)$.
$f_\mathrm{Hxc}^{(\lambda)}$ is the sum of Hartree,
\begin{equation}
  f_\mathrm{H}(q) = \frac{4\pi}{q^2},
\end{equation}
and exchange-correlation kernels evaluated at the coupling-constant $\lambda$.
From Ref. \cite{lein2000}, we may obtain this expression from the coupling-constant scaled LFF
\begin{equation}
  f_\mathrm{Hxc}^{(\lambda)}(q,i u) = \frac{4\pi \lambda}{q^2} \left[1 - G_+\left(\lambda \rs, \frac{q}{\lambda},\frac{i u}{\lambda^2} \right) \right].
  \label{eq:fhxc}
\end{equation}

Developing a method to reliably perform the three-dimensional integration needed in Eq. (\ref{eq:acfdt}) without combinatorial explosion is challenging.
To do this, we first computed approximate random phase approximation (RPA) correlation energies by integrating up to two cutoffs, called $x\suc \equiv q\suc/\kf$ and $v\suc \equiv u\suc/\kf^2$,
\begin{equation}
  \varepsilon\suc^\text{RPA}(\rs) \approx -3 \int_0^{x\suc} dx \int_0^1 d \lambda \int_0^{v\suc}
    d v \frac{4\pi \lambda\left[\chi_0(q,i u) \right]^2 q^{-2}}{1 - 4\pi \lambda \chi_0(q,i u)/q^{-2}}.
  \label{eq:eps_c_rpa_approx}
\end{equation}
As $G_+^\text{RPA}=0$, the right- and left-hand-sides of Eq. (\ref{eq:eps_c_rpa_approx}) become exactly equal in the limit that $x\suc, \, v\suc \to \infty$.
These integrals were computed using globally-adaptive, Gauss-Kronrod quadrature.
See the computational details of Refs. \cite{perdew2021} and \cite{kaplan2022} for more details.

The cutoffs were adjusted to give agreement to within, ideally, 1\% error of the PW92-parameterized RPA correlation energies \cite{perdew1992}.
These cutoffs were then approximately parameterized as continuous functions of $\rs$,
\begin{equation}
  x\suc(\rs) \approx \left\{
  \begin{array}{ll}
    c_{x0} + c_{x1} \rs, & \rs \leq 5 \\
    c_{x0} + 5 c_{x1} + c_{x2} (\rs - 5) + c_{x3} (\rs - 5)^2, & 5 < \rs \leq 60 \\
    c_{x0} + 5 c_{x1} + 55 c_{x2} + 3025 c_{x3} + c_{x4}(\rs - 60), & 60 < \rs
  \end{array}
  \right. ,
\end{equation}
with $c_{x0} = 3.928319$, $c_{x1} = 0.540168$, $c_{x2} = 0.042225$, $c_{x3} = 0.001810$, and $c_{x4} =2.501585$.
Analogously,
\begin{equation}
  v\suc(\rs) \approx \left\{
  \begin{array}{ll}
    c_{v0} + c_{v1} \rs^{c_{v2}}, & \rs \leq 40 \\
    c_{v0} + c_{v1}(40)^{c_{v2}} + (\rs - 40)^{c_{v3}}, & 40 < \rs
  \end{array}
  \right. ,
\end{equation}
with $c_{v0} = 1.227277$, $c_{v1} = 5.991171$, $c_{v2} = 0.283892$, and $c_{v3} = 0.379981$.

To recover the error lost in using finite integration bounds, we then perform a set of coordinate remappings.
Let $f(x)$ be a generic function of $x$, and $g(v)$ a generic function of $v$.
Then the mappings used are
\begin{align}
  \int_0^\infty dx \, f(x) &= \int_0^{x\suc} dx \, f(x) + \int_0^{1/x\suc} dt \frac{f(1/t)}{t^2} \\
  \int_0^\infty dv \, g(v) &= \int_0^{v\suc} dv \, g(v) + \int_0^1 dw \frac{g(v\suc - \ln(1 - w))}{1 - w}.
\end{align}
These mappings are, in principle, exact.
For the range of $0 < x < x\suc$, we use 100-point Gauss-Legendre quadrature, and for the range of $0 < t < 1/x\suc$, we use 50-point Gauss-Legendre quadrature.
The same number of points were used for the corresponding ranges of $v$ and $w$, respectively.
100-point Gauss-Legendre quadrature was used for the coupling-constant, $\lambda$, integration.
Table \ref{tab:RPA_sanity} shows that this method becomes asymptotically exact as $\rs \to 0$, and, in the metallic range $ 1 \leq \rs \leq 10$, gives generally negligible percent deviations from the Perdew-Wang parameterization of the RPA correlation energy, PW-RPA \cite{perdew1992}.
Indeed, for all $\rs \leq 120$, this method yields percent deviations less than 1\% from PW-RPA.

\begin{table}
  \centering
  \begin{tabular}{rrrr} \hline
    $\rs$ & $\varepsilon\suc^\mathrm{RPA}(\rs)$ & $\varepsilon\suc^\mathrm{PW-RPA}(\rs)$ & Percent Deviation (\%) \\ \hline
    0.1 & -0.143815 & -0.143819 & 0.00 \\
    0.5 & -0.097155 & -0.097221 & 0.07 \\
    1.0 & -0.078631 & -0.078741 & 0.14 \\
    2.0 & -0.061651 & -0.061797 & 0.24 \\
    3.0 & -0.052619 & -0.052774 & 0.29 \\
    4.0 & -0.046673 & -0.046827 & 0.33 \\
    5.0 & -0.042343 & -0.042491 & 0.35 \\
    10.0 & -0.030549 & -0.030661 & 0.37 \\
    20.0 & -0.021288 & -0.021367 & 0.37 \\
    40.0 & -0.014385 & -0.014454 & 0.48 \\
    60.0 & -0.011300 & -0.011367 & 0.59 \\
    80.0 & -0.009472 & -0.009542 & 0.74 \\
    100.0 & -0.008236 & -0.008311 & 0.90 \\
    120.0 & -0.007345 & -0.007413 & 0.93 \\
  \hline
  \end{tabular}
  \caption{Comparison of the RPA correlation energies computed using the method described here, and with the Perdew-Wang approximation for the RPA correlation energy, PW-RPA \cite{perdew1992}.
  The PW-RPA approximation is simply a parameterization of the accurate RPA data of Vosko, Wilk, and Nusair \cite{vosko1980}.
  Percent deviations, $100\% \cdot (1 - \varepsilon\suc^\mathrm{RPA}/\varepsilon\suc^\mathrm{PW-RPA})$, are shown in the last column.
  }
  \label{tab:RPA_sanity}
\end{table}


\section{Corrected expressions for the Richardson-Ashcroft local field factors \label{sec:RA_corrected}}

The work of Richardson and Ashcroft \cite{richardson1994} is extremely important, as it is the first work to directly compute the individual LFFs $G_s$, $G_a$, and $G_n$ at a range of wavevectors, frequencies, and densities.
Moreover, they provided sensible parameterizations of these functions that are unfortunately hindered by typographical errors, as realized by Lein \textit{et al.} \cite{lein2000}.
We provide further corrections here.
The density and spin LFFs are computed as
\begin{align}
  G_+(\rs,q,\omega) &= G_s(\rs,q,\omega) + G_n(\rs,q,\omega) \\
  G_-(\rs,q,\omega) &= G_a(\rs,q,\omega) + G_n(\rs,q,\omega).
\end{align}
As before, $q>0$ is a wavevector, and $\omega$ is a complex-valued frequency.
The following dimensionless variables are used in the Richardson-Ashcroft work
\begin{align}
  z &= q/(2\kf) \\
  u &= \frac{1}{2\kf^2} \mathrm{Im} \, \omega.
\end{align}
A few $\rs$-dependent functions are used to define the low- and high-frequency regimes of the LFFs, $\lambda_i^{(j)}$, where $i= s, \, a, \, n$ and $j = 0, \, \infty$.
Richardson and Ashcroft parameterized the relationship between the $u \to 0$ behaviors of $G_a$ and $G_n$ as
\begin{equation}
  \frac{\lambda_n^{(0)}}{\lambda_n^{(0)} + \lambda_a^{(0)}}
    \approx \frac{-(0.11) \rs}{1 + (0.33) \rs} \equiv \mathcal{F}(\rs).
\end{equation}
Their sum is rigorously computed using Eq. (RA:39) of Ref. \cite{lein2000},
\begin{equation}
  \lambda_n^{(0)} + \lambda_a^{(0)} = 1 - 3 \left(\frac{2\pi}{3} \right)^{2/3}
    \rs \frac{\partial^2 \varepsilon\suc}{\partial \zeta^2} (\rs,0),
\end{equation}
where $\varepsilon\suc(\rs,\zeta)$ is in \textit{Hartree} units, and not Rydberg units as in Ref. \cite{richardson1994} or Eq. (RA:39) of Ref. \cite{lein2000}.
Thus
\begin{align}
  \lambda_n^{(0)} &= \mathcal{F}(\rs) \left[
    1 - 3 \left(\frac{2\pi}{3} \right)^{2/3}
      \rs \frac{\partial^2 \varepsilon\suc}{\partial \zeta^2} (\rs,0)
  \right], \\
  \lambda_a^{(0)} &= \frac{1 - \mathcal{F}(\rs)}{\mathcal{F}(\rs)} \lambda_n^{(0)}.
\end{align}
The $u \to 0$ limit of the spin-symmetric, noninteracting LFF is then
\begin{equation}
  \lambda_s^{(0)} = - \lambda_n^{(0)} + 1
  + \frac{2\pi}{3}a\sux \rs^2 \frac{\partial \varepsilon\suc}{\partial \rs}(\rs,0) - \frac{\pi}{3} a\sux \rs^3 \frac{\partial^2 \varepsilon\suc}{\partial \rs^2}(\rs,0)
\end{equation}
again with $\varepsilon\suc(\rs,\zeta)$ in Hartree.
$a\sux = [4/(9\pi)]^{1/3}$ is the inverse of the factor that relates the Fermi momentum to the Wigner-Seitz radius, $\rs = (a\sux \kf)^{-1}$.

Although not defined explicitly in Ref. \cite{richardson1994}, the high-frequency limit of the spin-antisymmetric, noninteracting LFF is
\begin{equation}
  \lambda_a^{(\infty)} = \frac{2g(\rs) - 1}{3},
\end{equation}
where again, $g(\rs)$ is the on-top pair distribution function.
The high-frequency limit of the occupation number LFF is given as
\begin{equation}
  \lambda_n^{(\infty)} = 6\pi a\sux \rs \frac{\partial }{\partial \rs}\left[
    \rs \, \varepsilon\suc(\rs,0) \right],
\end{equation}
and the corresponding limit of the spin-symmetric, noninteracting LFF from Eq. (RA:39) of Ref. \cite{lein2000},
\begin{equation}
  \lambda_s^{(\infty)} = \frac{3}{5} - \frac{4\pi a\sux}{5}\left[
    \rs^2 \frac{\partial \varepsilon\suc}{\partial \rs} (\rs,0)
    + 2\rs \varepsilon\suc(\rs,0) \right].
\end{equation}

Finally, we give the expression for the spin-symmetric, noninteracting LFF as
\begin{align}
  \gamma_s & \equiv \frac{9}{16[1 - g(\rs)]} \lambda_s^{(\infty)} + \frac{4\alpha_s - 3}{4\alpha_s} \\
  a_s(u) &= \lambda_s^{(\infty)}
    + \frac{\lambda_s^{(0)} - \lambda_s^{(\infty)}}{1 + (\gamma_s u)^2} \\
  c_s(u) &= \frac{3 \lambda_s^{(\infty)}}{4[ 1 - g(\rs) ]}
    - \left(
      1 + \gamma_s u \right)^{-1}
      \left[\frac{4}{3} - \frac{1}{\alpha_s}
      + \frac{3\lambda_s^{(\infty)}}{4[1 - g(\rs)]}
    \right] \\
  b_s(u) &= a_s(u) \left\{
    3 a_s(u)(1 + u)^4 - \frac{8}{3}[1 - g(\rs)](1 + u)^3 - 2c_s(u)[1 - g(\rs)](1 + u)^4
  \right\}^{-1} \\
  G_s(z,i u) &= z^2 \frac{a_s(u) + 2 [1 - g(\rs)] b_s(u)z^6/3}{1 + c_s(u)z^2 + b_s(u)z^8}.
\end{align}
$\alpha_s = 0.9$ is a fit parameter.

Likewise, the spin-antisymmetric, noninteracting LFF is parameterized as
\begin{align}
  \gamma_a &= \frac{9}{8} \lambda_a^{(\infty)} + \frac{1}{4}, \\
  a_a(u) &= \lambda_a^{(\infty)}
    + \frac{\lambda_a^{(0)} - \lambda_a^{(\infty)}}{1 + (\gamma_a u)^2} \\
  c_a(u) &= \frac{3}{2} \lambda_a^{(\infty)} - \left[ 1 + (\gamma_a u)^2 \right]^{-1}
    \left[ \frac{1}{3} + \frac{3}{2} \lambda_a^{(\infty)} \right] \\
  \beta_a(u) &= \frac{4g(\rs) - 1}{3}
    - \lambda_a^{(\infty)}\, \frac{(\gamma_a u)^2}{1 + (\gamma_a u)^2} \\
  b_a(u) &= a_a(u) \left[  3a_a(u)(1 + u)^4 - 4\beta_a(u)(1 + u)^3 - 3 c_a(u) \beta_a(u)(1 + u)^4
  \right]^{-1} \\
  G_a(z,i u) &= \lambda_a^{(\infty)} \frac{(\gamma_a u)^2}{1  + (\gamma_a u)^2}
     + z^2 \frac{a_a(u) + b_a(u) \beta_a(u) z^6}{1 + c_a(u) z^2 + b_a(u) z^8}.
\end{align}

Last, the occupation number LFF is parameterized as
\begin{align}
  a_n(u) &= \lambda_n^{(\infty)} + \frac{\lambda_n^{(0)} - \lambda_n^{(\infty)}}{1 + (\gamma_n u)^2} \\
  c_n(u) &= \frac{3 \gamma_n u}{(1.18)(1 + \gamma_n u)} - [1 + (\gamma_n u)^2]^{-1}
    \left[ \frac{ 3\lambda_n^{(0)} + \lambda_n^{(\infty)} }{3 \lambda_n^{(0)} + 2\lambda_n^{(\infty)} } + \frac{3\gamma_n u}{(1.18)(1 + \gamma_n u)} \right] \\
  d_n(u) &= a_n(u) + \lambda_n^{(\infty)} + \frac{2}{3} \lambda_n^{(\infty)} c_n(u)(1 + \gamma_n u) \\
  b_n(u) &= -\frac{3}{2\lambda_n^{(\infty)}(1 + \gamma_n u)^2}\left\{
    d_n(u) + \left[ d_n(u)^2 + \frac{4}{3} \lambda_n^{(\infty) } a_n(u) \right]^{1/2}
  \right\} \\
  G_n(z, i u) &= z^2 \frac{a_n(u) - \lambda_n^{(\infty)} b_n(u) z^4/3}{1 + c_n(u) z^2 + b_n(u) z^4}.
\end{align}
$\gamma_n = 0.68$ is another fit parameter.
