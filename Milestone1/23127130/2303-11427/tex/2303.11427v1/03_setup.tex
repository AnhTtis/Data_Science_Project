
%%%%%%%%%%%%%%%%%%%%%%%%%%%%%%%%%%%%%%%%%%%%%%%%%%%%%%%%%%
% REMEMBER STORYTELLING
%%%%%%%%%%%%%%%%%%%%%%%%%%%%%%%%%%%%%%%%%%%%%%%%%%%%%%%%%%

\section{Satellite Communication Setup \& Notations}
\label{sec:setup}

This section introduces the applied \gls{los} channel model with errors regarding the measurement of user and satellite positions. Further, we explain two common precoding techniques for satellite communications.


\subsection{System Model}
\label{sec:systemmodel}

In this paper, we consider a multi-user downlink scenario with \( \numsatellites \) \gls{leo} satellites, each equipped with a \gls{ula} of \(\numantennasper\) antennas, serving \(\numusers\) handheld users with one receive antenna each and low receive antenna gain \( \usergain \). It is assumed that all satellites are provided with the data symbols of all users and perform joint precoding, \ie all satellites perform the precoding together.
The data symbol $\datasymbol_\useridx$ of user $\useridx$ is weighted by a precoding vector $\precodingvec_\useridx \in \mathbb{C}^{\numsatellites \numantennasper \times 1}$ and transmitted over the \gls{los} channel \(\csivector_\useridx \in \mathbb{C}^{1\times \numsatellites \numantennasper}\) with \gls{awgn} $n_\useridx \sim \mathcal{CN}(0, \noisepower)$. Hence, the receive signal $y_\useridx$ follows as
%
\begin{align}
\label{eq:transmission}
\textstyle
    y_\useridx = \csivector_\useridx\precodingvec_\useridx \datasymbol_\useridx + \csivector_\useridx\sum^\numusers_{\otheruseridx \neq \useridx}  \precodingvec_\otheruseridx \datasymbol_l + n_\useridx .
\end{align}
%
The \gls{los} channel vector ~\( \csivector_{\useridx, \satidx} \in \mathbb{C}^{1\times \numantennasper} \) from satellite \(\satidx\) to user \(\useridx\) with $\csivector_\useridx = [\csivector_{\useridx,1} \dots \csivector_{\useridx, \numsatellites}]$ is given by
%A corresponding \gls{los} channel vector \( \csivector_{\satidx,\useridx} \) from satellite \(\satidx\) to user \(\useridx\) is given by
%
\begin{align}
\label{eq:channel_perfect}
    \csivector_{\useridx, \satidx}(\aod_{\useridx, \satidx}) = 
     \frac{\wavelength\sqrt{\usergain \satgain }}{4\pi \dist_{\useridx, \satidx}}  \text{e}^{-j \varphi_{\useridx, \satidx} }  \steeringvec_{\useridx, \satidx} ( \cos(\aod_{\useridx, \satidx})),
\end{align}
%
where $\satgain$ denotes the satellite antenna gain, $\wavelength$ is the wavelength and $\dist_{\useridx, \satidx}$ is the distance between the $\satidx$-th satellite and the $\useridx$-th user. The overall phase shift from satellite $\satidx$ to user $\useridx$ is given by $\varphi_{\useridx, \satidx} \in [0, 2 \pi]$. The relative phase shifts between the $\numantennasper$ antennas of a satellite $\satidx$ to user $\useridx$ are described by the steering vector $\steeringvec_{\useridx, \satidx}( \cos(\aod_{\useridx, \satidx})) \in \mathbb{C}^{1 \times N}$, where $\aod_{\useridx, \satidx}$ is the \gls{aod} from satellite $\satidx$ to user $\useridx$. The $\antidx$-th entry of the steering vector $\steeringvec_{\useridx, \satidx}(\cos(\aod_{\useridx, \satidx}))$ is given as
%
\begin{align}
\label{eq:steering_vec}
     \steeringentry^\antidx_{\useridx, \satidx}(\cos(\aod_{\useridx, \satidx})) = \text{e}^{-j\pi \frac{\antdist}{\wavelength} (\numantennasper +1 - 2\antidx) \cos(\aod_{\useridx, \satidx})}  ,
\end{align}
%
where $\antdist$ is the inter-antenna-distance between the $\numantennasper$ antennas per satellite. 
Given that the inter-antenna-distance $\antdist$ is fixed and known, the phase differences in \eqref{eq:steering_vec} between antenna $\antidx$ and $\antidx'$ are only determined by the $\useridx$\nobreakdash-th user position, reflected in the space angle $\cos(\aod_{\useridx, \satidx})$.
%The steering vectors $\steeringvec_{\useridx, \satidx}(\cos(\aod_{\useridx, \satidx}))$ and the channel vectors $\csivector_{\useridx, \satidx}(\aod_{\useridx, \satidx})$ for all $\numusers$ users and $\numsatellites$ satellites are, accordingly, primarily characterized by the \glspl{aod} $\aod_{\useridx, \satidx}$. The \gls{aod} $\aod_{\useridx, \satidx}$ provides satellite $\satidx$ with position knowledge of user $\useridx$.
In practice, a precise estimation of the user's position may not be feasible due to the high velocities of \gls{leo} satellites. %\maik{Such estimation error of the position leads to imperfect \gls{csit}.}
We assume that user $\useridx$'s position can be well estimated within a certain region. Therefore, we model the estimation error on the space angle $\cos(\aod_{\useridx, \satidx})$ as a uniformly distributed additive error~${\erroraod_{\useridx, \satidx} \sim \mathcal{U}(-\errorbound, + \errorbound)}$~\cite{MaikICC}.
%We therefore model the influence of erroneous \gls{aod} estimates on the cosine of the \glspl{aod} $\cos(\aod_{\useridx, \satidx})$ as a uniformly distributed additive error~${\erroraod_{\useridx, \satidx} \sim \mathcal{U}(-\errorbound, + \errorbound)}$ \cite{MaikICC}. 
%The error $\erroraod_{\useridx, \satidx}$ follows the same distribution for all $\numsatellites$ satellites and $\numusers$ users. 
Since the error $\erroraod_{\useridx, \satidx} $ is added on the cosine of the \glspl{aod} and given the definition of the steering vector \eqref{eq:steering_vec}, the erroneous channel vector~\( \tilde{\csivector}^{(1)}_{\useridx,\satidx}(\aod_{\useridx, \satidx}) \) can be expressed as an overall multiplicative error $\steeringvec_{\useridx, \satidx}(\erroraod_{\useridx, \satidx})\in \mathbb{C}^{1 \times \numantennasper} $ on the true channel ${\csivector}_{\useridx,\satidx}(\aod_{\useridx, \satidx})$
%
\begin{align}
\label{eq:error}
    \tilde{\csivector}^{(1)}_{\useridx, \satidx}( \aod_{\useridx, \satidx}, \erroraod_{\useridx,\satidx}) = \csivector_{\useridx, \satidx}( \aod_{\useridx, \satidx}) \circ \steeringvec_{\useridx, \satidx}\big(\erroraod_{\useridx, \satidx}\big) 
\end{align}
%
with the $\antidx$-th entry of the error vector $\steeringvec_{\useridx, \satidx}(\erroraod_{\useridx, \satidx})$
\begin{align}
\label{eq:steering_error}
     \steeringentry^\antidx_{\useridx, \satidx}(\erroraod_{\useridx, \satidx}) = \text{e}^{-j\pi \frac{\antdist}{\wavelength} (\numantennasper +1 - 2\antidx) \erroraod_{\useridx, \satidx}}  .
\end{align}

Joint precoding with multiple satellites requires tight inter-satellite synchronization. If such synchronization is imperfect, the signals from different satellites arrive with different phases. To model such synchronization mismatch, we assume an error $\errorphase_{\useridx, \satidx} \sim \mathcal{N}(0,\errorphasevariance)$ on the overall phase shift $\overallphase_{\useridx, \satidx}$. The resulting erroneous channel estimation with imperfect synchronization and position knowledge is given as
%For joint precoding, each satellite requires the positions of all users \emph{as well as} the positions of all satellites.
%We model flawed information on the satellite positions as 
%If the information on the satellite positions is flawed, it corresponds to
%an error $\errorphase_{\useridx, \satidx} \sim \mathcal{N}(0,\errorphasevariance)$ on the overall phase shift $\overallphase_{\useridx, \satidx}$. If errors of imperfections occur on both satellite and user positions, the resulting erroneous channel estimation~$\tilde{\csivector}^{(2)}_{\useridx, \satidx}(\aod_{\useridx, \satidx}, \erroraod_{\useridx,\satidx}, \errorphase_{\useridx,\satidx})$ is given as
%
\begin{align}
\label{eq:error2}
    \tilde{\csivector}^{(2)}_{\useridx, \satidx}( \aod_{\useridx, \satidx}, \erroraod_{\useridx,\satidx}, \errorphase_{\useridx,\satidx} )= \text{e}^{-j \errorphase_{\useridx, \satidx}}
    \cdot
    \tilde{\csivector}^{(1)}_{\useridx, \satidx}( \aod_{\useridx, \satidx}, \erroraod_{\useridx,\satidx}).
\end{align}
%
In this paper, we analyze the influence of both error models, \eqref{eq:error} and \eqref{eq:error2}, on the precoding performance.
%Based on the estimates $\tilde{\csivector}_{\useridx, \satidx}( \cos(\aod_{\useridx, \satidx}))$ of the channel vector ${\csivector}_{\useridx, \satidx}( \cos(\aod_{\useridx, \satidx}))$ the different precoding methods in the subsequent sections are implemented.
We evaluate the performance of our different precoding techniques by comparing their corresponding sum rates
%
\begin{align}
\label{eq:sumRate}
    \sumrate = \sum_{\useridx = 1}^{\numusers}  \log \Bigg(1+\frac{\left|\csivector_\useridx\precodingvec_{\useridx}\right|^2}{\noisepower+ \sum_{\otheruseridx \neq \useridx}^\numusers|\csivector_\useridx \precodingvec_{\otheruseridx}|^2}\Bigg) 
\end{align}
%
 Our goal is to maximize the sum rate~\eqref{eq:sumRate} for various satellite-to-user constellations while enhancing robustness against imperfect user position knowledge at the satellites~\eqref{eq:error} as well as imperfect satellite position knowledge~\eqref{eq:error2}. We propose to learn a robust precoding algorithm using the \gls{sac} method, to be introduced in \refsec{sec:sac}. Further, we analyze two common approaches explained in the subsequent section.

\subsection{Baseline Precoding Techniques}
\label{sec:SETUPTOPIC2}

%This section presents two common precoding approaches for satellite downlink communications. First, we introduce the conventional \gls{sdma} \gls{mmse} precoder, then we specify an \gls{oma} approach as a baseline comparison.
This subsection presents two common precoding approaches for satellite downlink communications: the conventional \gls{mmse} precoder for \gls{sdma}, and an \gls{oma} approach assuming orthogonal time or frequency resources. 

\subsubsection{\gls{mmse}}
\label{sec:MMSE}
%Conventional \gls{sdma} precoding spatially separates the user messages by assigning a precoding vector $\precodingvec_\useridx$ to each user's transmit symbol $\symbolentry_\useridx$.
For optimal \gls{sdma} precoding, the precoding vector $\precodingvec_\useridx$ steers a beam with maximal power into the direction of the corresponding user $\useridx$ while minimizing \gls{iui}. The \gls{mmse} precoder has been proven to be a reliable precoder in this manner and is widely used \cite{windpassinger2004detection, MMSEspace}. For \gls{mmse}, the precoding matrix $\precodingmatrix^\text{MMSE} = \big[\precodingvec_1^\text{MMSE} \dots \precodingvec_\numusers^\text{MMSE}\big]$ for a given channel estimate $\tilde{\csimatrix} = [\tilde{\csivector}_1 \dots \tilde{\csivector}_\numusers]^\text{T}$ is calculated as follows
%
\begin{align}
	\label{eq:MMSE}
	\begin{split}
		&\precodingmatrix^\text{MMSE}= \sqrt{\frac{\transmitpower}{\text{tr} \{ {{\precodingmatrix^{\prime}}^{\text{H}}} \precodingmatrix^{\prime} \} }} \cdot \precodingmatrix^{\prime}  
\\[0.8ex]
	&\precodingmatrix^{\prime} = \Big[ \mathbf{\tilde{\csimatrix}}^\text{H} \mathbf{\tilde{\csimatrix}} + \noisepower \cdot \frac{\numusers}{\transmitpower} \cdot \mathbf{I}_{\numsatellites \numantennasper} \Big]^{-1} \mathbf{\tilde{\csimatrix}}^\text{H} 
	\end{split} ,
\end{align}
%
where $\transmitpower$ denotes the total amount of transmit power. In this paper, we further constrain the maximum transmit power per satellite $\transmitpower_\satidx$ to be equally distributed among all $\numsatellites$ satellites, such that $\transmitpower_\satidx \leq \transmitpower/\numsatellites$ applies. Note that the \gls{mmse} precoding approach does not necessarily maximize the sum rate $\sumrate$ \eqref{eq:sumRate}.


\subsubsection{\gls{oma}}

In contrast to \gls{sdma}, \gls{oma} uses orthogonal time or frequency resources. In this case there is no \gls{iui} between the user channels. Therefore, the optimal precoder is a \gls{mrt} precoder that solely maximizes the transmit power steered into the direction of a given user $k$
%
\begin{align}
    \precodingvec_\useridx^\mathrm{MRT} = \sqrt{\transmitpower} \cdot \frac{\tilde{\csivector}_\useridx^\mathrm{H}}{\|\tilde{\csivector}_\useridx\|} .
\end{align}
%
Here the total transmit power $\transmitpower$ is used for user $k$. Due to the absence of \gls{iui}, the sum rate for \gls{oma} calculates as
%
\begin{align}
\label{eq:oma}
    \sumrate^\mathrm{OMA} = \frac{1}{\numusers} {\sum_{\useridx = 1}^{\numusers}}  \log \Bigg(1+\frac{\left|\csivector_\useridx \precodingvec_{\useridx}\right|^2}{\noisepower}\Bigg) .
\end{align}
%
To guarantee a fair comparison between \gls{oma} and the different \gls{sdma} approaches, the overall rate is divided by the number of users $\numusers$, taking into account that in \gls{sdma} time and frequency resources are shared between all users.
The \gls{mmse} and the \gls{oma} approach will serve as baselines for our proposed learning algorithm. 
%for the evaluation of our proposed learning algorithm.

In the next section, we will discuss how to iteratively learn a robust precoding algorithm that approximately optimizes the sum rate \eqref{eq:sumRate} based on the current channel state estimate.


%%%%%%%%%%%%%%%%%%%%%%%%%%%%%%%%%%%%%%%%%%%%%%%%%%%%%%%%%%
%% EXAMPLE TABLE
%\begin{table}[!t]
%	\renewcommand{\arraystretch}{1.3}
%	\caption{A Simple Example Table}
%	\label{tab:table_example}
%	\centering
%	\rowcolors{2}{white}{gray!10} 
%	\begin{tabular}{cc}
%		\hline
%		\bfseries First & \bfseries Next\\
%		\hline\hline
%		1.0 & 2.0\\
%		1.0 & 2.0\\
%		\hline
%	\end{tabular}
%\end{table}

%% EXAMPLE FIGURE
%\begin{figure}[!t]
%	\centering
%	\includegraphics[width=2.5in]{myfigure}
%	\input{figures/myfigure.pgf}
%	\caption{text}
%	\label{fig:figure_example}
%\end{figure}
