
%%%%%%%%%%%%%%%%%%%%%%%%%%%%%%%%%%%%%%%%%%%%%%%%%%%%%%%%%%
% REMEMBER STORYTELLING
%%%%%%%%%%%%%%%%%%%%%%%%%%%%%%%%%%%%%%%%%%%%%%%%%%%%%%%%%%


\section{Conclusions}
\label{sec:conclusions}

We have applied the \gls{sac} \gls{rl} method to the problem of sum rate optimal precoding for cooperative multibeam satellite communications in the presence of erroneous channel knowledge. Our results show that we are able to learn both effective and robust precoding algorithms with no assumptions on the underlying error model.
We therefore expect this approach to scale well on real life satellites, particularly with sights on dedicated \gls{ml} hardware being deployed on future satellites~\cite{giuffrida_cloudscout_2020}.



%%%%%%%%%%%%%%%%%%%%%%%%%%%%%%%%%%%%%%%%%%%%%%%%%%%%%%%%%%
%% EXAMPLE TABLE
%\begin{table}[!t]
%	\renewcommand{\arraystretch}{1.3}
%	\caption{A Simple Example Table}
%	\label{tab:table_example}
%	\centering
%	\rowcolors{2}{white}{gray!10} 
%	\begin{tabular}{cc}
%		\hline
%		\bfseries First & \bfseries Next\\
%		\hline\hline
%		1.0 & 2.0\\
%		1.0 & 2.0\\
%		\hline
%	\end{tabular}
%\end{table}

%% EXAMPLE FIGURE
%\begin{figure}[!t]
%	\centering
%	\includegraphics[width=2.5in]{myfigure}
%	\input{figures/myfigure.pgf}
%	\caption{text}
%	\label{fig:figure_example}
%\end{figure}
