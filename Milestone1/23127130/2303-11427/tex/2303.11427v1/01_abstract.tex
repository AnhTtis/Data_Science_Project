
%%%%%%%%%%%%%%%%%%%%%%%%%%%%%%%%%%%%%%%%%%%%%%%%%%%%%%%%%%
% REMEMBER STORYTELLING
%%%%%%%%%%%%%%%%%%%%%%%%%%%%%%%%%%%%%%%%%%%%%%%%%%%%%%%%%%


\begin{abstract}
	Direct \acrlong{leo} satellite-to-handheld links are expected to be part of a new era in satellite communications.
    \acrlong{sdma} precoding is a technique that reduces interference among satellite beams, therefore increasing spectral efficiency by allowing cooperating satellites to reuse frequency.
    Over the past decades, optimal precoding solutions with perfect channel state information have been proposed for several scenarios, whereas robust precoding with only imperfect channel state information has been mostly studied for simplified models. In particular, for \acrlong{leo} satellite applications such simplified models might not be accurate.
    %While optimal precoding solutions have been found for, \eg power minimization in unicast transmissions with perfect channel knowledge, future systems are expected to be robust in the presence of imperfect estimations.
    In this paper, we use the function approximation capabilities of the \acrlong{sac} deep \acrlong{rl} algorithm to learn robust precoding with no knowledge of the system imperfections.
\end{abstract}

\keywords{
	Multi-user beamforming, 3D networks, \acrfull{leo}, \acrfull{ml}, deep \acrfull{rl}
}

\glsresetall  % Resets the "first use" counter of acronym package
