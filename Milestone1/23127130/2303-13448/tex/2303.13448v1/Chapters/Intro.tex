The nature of kaons and antikaons is closely related to the chiral symmetry breaking pattern of low-energy QCD~\cite{Tolos:2020aln,Hyodo:2022xhp}. Therefore, the study of kaon properties and their modification in dense nuclear matter has received the attention of the scientific community in the past decades.
In particular, the hadronic interactions of charged kaons, K$^+$ and K$^-$, with nucleons (N) were investigated using kaonic atoms~\cite{BATTY1997385,friedman2017k,curceanu2019modern} and by studying the interaction or the production of kaons in light~\cite{PhysRevD.1.1267,BC,MICHAEL199629,PhysRevC.92.045204,doce2016k,Piscicchia:2018rez,del2019hbox,PhysRevC.51.669,PhysRevC.49.2569} and heavy nuclei~\cite{20006,PhysRevC.75.024906,PhysRevC.90.025210,PhysRevC.65.014603,PhysRevLett.96.072301,PhysRevLett.123.022002,ANKE,PhysRevC.99.014904,PhysRevLett.102.182501}. Such measurements demonstrated the repulsive and attractive nature of the K$^+$N and K$^-$N strong interactions~\cite{Tolos:2020aln,Hyodo:2022xhp}, respectively. Information on the in-medium modification of the K$^+$ and K$^-$ potentials with the increasing baryon density was extracted from the comparison of kaon production yields and flow observables measured in nucleus--nucleus collisions~\cite{20006,PhysRevC.75.024906,PhysRevC.90.025210} and pion- and proton-induced reactions~\cite{PhysRevC.65.014603,PhysRevLett.96.072301,PhysRevLett.123.022002,PhysRevLett.102.182501,ANKE,PhysRevC.99.014904} with the expectations from transport models. Several K$^+$ nuclear potentials have been tested and the best agreement with the data results in a repulsive strength of 20--40 MeV at nuclear saturation density. 
On the other hand, the K$^-$N interaction is known to be sufficiently attractive in the isospin I~=~0 channel to dynamically generate the $\Lambda(1405)$ just below the K$^-$p threshold~\cite{Mai2021}. Such a state is interpreted as a quasi-bound antikaon--nucleon $\mathrm{\overline{K}}$N system which couples strongly to the $\pi\Sigma$ channel, giving rise to a sizable K$^-$ absorption~\cite{SekiharaPRC}.
In the case of K$^-$ nuclear interaction, the influence of single- and multi-nucleon absorption processes in nuclei currently prevents extracting firm conclusions on the strength of the K$^-$ in-medium attractive potential~\cite{Tolos:2020aln}.
This ambiguity triggered a longstanding debate in the literature about the possible existence of exotic kaonic bound objects with nucleons (see~\cite{Tolos:2020aln} and references therein).
Recently, the E15 Collaboration reported the first experimental evidence of the $\mathrm{\overline{K}}$NN state with a binding energy of about 42 MeV and a decay width of about 100 MeV~\cite{yamaga2020observation}.
From the theoretical side, 
 binding energies in the range 9--95 MeV and decay widths between 16 and 110 MeV are expected~\cite{yamazaki2002k,shevchenko2007faddeev,shevchenko2007k,ikeda2007strange,dote2008k,ikeda2009resonance,wycech2009variational,koike2009k,ikeda2010energy,barnea2012realistic,bayar2013k,revai2014faddeev,dote2015application,sekihara2016structure,ohnishi2017few,dote2017fully}.
The uncertainties in the models mainly arise from scarce knowledge of the full $\mathrm{\overline{K}}$NN three-body effects, such as three-body coupled channels and two-nucleon absorption processes.
Additionally, three-body forces, which are relevant in the calculation of the nuclear binding energies~\cite{RevModPhys.85.197,marcucci2020hyperspherical}, are currently not included in kaonic bound state models. 
Further experimental investigations on the K$^+$NN and K$^-$NN three-body dynamics, in addition to the studies of K$^+$ and K$^-$ interaction in nuclei, are required to isolate and quantify the contribution from genuine three-body effects in such systems.

An alternative method to explore the three-body dynamics of K$^+$ and K$^-$ with nucleons is to employ the femtoscopy technique at high-energy collider facilities. In small colliding systems, such as pp and p--Pb at the Large Hadron Collider (LHC),
%the hadrons are produced at distances ranging 
the inter-hadron distances at the time of the particle emission range
from a few femtometers down to scales compatible with the nucleon size. This leads to an enhancement of the strength of the signal due to the short-range strong interaction in the measured correlation function~\cite{Fabbietti:2020bfg}. The femtoscopy method was proven to be able to test and constrain the hadron--hadron interaction for various two-particle systems~\cite{ALICE:Run1,Acharya:2019bsa,acharya2020investigation,ALICE:LL, ALICE:pXi,ALICE:pOmega,ALICE:2021cpv,ALICE:2021cyj,ALICE:2022yyh,ALICE:2021njx,ALICE:2022uso,ALICE:2022enj}, providing data with unprecedented precision on the hadronic interactions with strangeness at low relative momenta, down to the energy threshold of the produced pairs. Furthermore, the measured femtoscopic correlation functions are sensitive to the presence of bound states in the energy region below the threshold. Thus, they were also used to constrain the parameters of bound hadron--hadron systems~\cite{ALICE:LL,Chizzali:2022pjd}. Recently, the ALICE Collaboration has extended the method to explore the correlation among three baryons, such as \pppCF and \ppLCF, to study the dynamics of three-body systems~\cite{ALICE:2022vzr}. The analysis exploited Kubo's cumulant expansion method~\cite{Kubo} and the projector method~\cite{DelGrande:2021mju} to isolate the genuine three-particle correlations from the measured correlation functions. As a result, a first experimental hint of genuine three-body effects in the unbound \pppCF system was found. The study in this Letter applies the same analysis procedure adopted in~\cite{ALICE:2022vzr} to the case of \ppKp and \ppKm particle triplets to explore possible genuine three-body effects in the correlation functions induced by the strong interaction and bound state formation. The main advantage of the femtoscopy method with respect to the previous experimental techniques is the possibility to investigate for the first time the K$^+$ and K$^-$ three-body dynamics with nucleons, free from additional effects induced by the presence of the surrounding nucleons in nuclei. 

