The final state interactions (FSIs) among the hadrons emitted in the collisions can be explored using correlation functions~\cite{lednicky2004correlation,Lisa:2005dd} in momentum space, defined 
as
\begin{equation}
   C_3(\mathbf{p}_1,\mathbf{p}_2,\mathbf{p}_3) = \frac{P_3(\mathbf{p}_1,\mathbf{p}_2,\mathbf{p}_3)}{P_1(\mathbf{p}_1) P_1(\mathbf{p}_2) P_1(\mathbf{p}_3)} \ ,
\end{equation}
where $\mathbf{p}_i$ is the momentum vector of the $i$-th particle and $P_3(\mathbf{p}_1,\mathbf{p}_2,\mathbf{p}_3)$ and $P_1(\mathbf{p}_i)$  are the probabilities of finding three particles and one particle with the corresponding momentum, respectively.
If the particles are not correlated $P_3(\mathbf{p}_1,\mathbf{p}_2,\mathbf{p}_3) = P_1(\mathbf{p}_1) P_1(\mathbf{p}_2) P_1(\mathbf{p}_3)$ and then the correlation function becomes identical to unity.
The presence of FSIs induces a correlation signal imprinted in the correlation function which causes deviations from unity depending on the nature of the interactions, attractive or repulsive, as well as on the properties of the emitting source (see~\cite{Fabbietti:2020bfg,ALICE:2020ibs} for the details).
In the case of three-particle systems, the correlation function can be evaluated in terms of $Q_3$. Hence, three-particle correlation functions are experimentally obtained as 
\begin{equation}
    C_3(Q_3) = \mathcal{N}\ \frac{N_\mathrm{same}(Q_3)}{N_\mathrm{mixed}(Q_3)} \ ,
\end{equation}
where $\mathcal{N}$ is a normalisation parameter, $N_\mathrm{same}(Q_3)$ is the $Q_3$ distribution of the particle triplets emitted in the same collision and  $N_\mathrm{mixed}(Q_3)$ is the distribution of uncorrelated triplets. The latter are obtained by taking the three particles needed to form a triplet from three different collisions, thus called mixed event sample.
Only events with similar multiplicity and position of the primary vertex along the beam axis are mixed.
The number of events used for the mixing is set to 30. To account for the two-track merging and splitting effects due to the finite two-track resolution in the same event sample, a minimum value of the distance between the same charge tracks on the azimuthal--polar angle plane $\Delta \eta$--$\Delta \varphi$ is applied to both the same and mixed event samples. The default selection is $\Delta\eta^2 + \Delta \varphi^2 \geq (0.017)^2$ for~ p--p tracks and $\Delta\eta^2/(0.012)^2 + \Delta \varphi^2/(0.04)^2 \geq 1$ for p--K$^+$ tracks. A systematic variation of about $+$10\% for the values of the minimum distance is applied in the analysis.
The correlation function is normalised to unity in the range $1.0 < Q_3 < 1.2$ GeV/$c$, where no signal from FSIs is expected.
Variations of the normalisation range are performed and included in the systematic uncertainties. 
In the case of a particle triplet X--Y--Z, pairwise correlations with one uncorrelated particle (X--Y)--Z, X--(Y--Z) and (Z--X)--Y also occur in the system. Genuine three-particle correlations were isolated in~\cite{ALICE:2022vzr} by applying the cumulant expansion method to the femtoscopic correlation functions. The three-particle cumulant $c_3 (Q_3)$, which incorporates information about the three-body effects in the system, is computed from the measured correlation function $C_3(Q_3)$ as follows
\begin{equation}
    c_3 (Q_3) = C_{3}(Q_3) - C_\mathrm{two\text{-}body}(Q_3) \ ,
    \label{eq:cumulant}
\end{equation}
where pairwise correlations are evaluated following the cumulant decomposition as
\begin{equation}
    C_\mathrm{two\text{-}body}(Q_3) = C_\mathrm{{}_{(X-Y)-Z}}(Q_3) + C_\mathrm{{}_{X-(Y-Z)}}(Q_3) + C_\mathrm{{}_{(Z-X)-Y}}(Q_3) - 2  \ .
    \label{eq:loc}
\end{equation}
Each component $C_\mathrm{{}_{(i-j)-k}}(Q_3)$ of the lower-order contributions in Eq. \eqref{eq:loc} is computed using two methods~\cite{ALICE:2022vzr,DelGrande:2021mju}: $i)$ a data-driven approach based on the event mixing technique, building triplets in which the correlated (i--j) pairs are emitted in the same collision and the uncorrelated particle --k from another collision; $ii)$ the projector method which uses as input the measured two-particle correlation function $C(k^*)$ for the correlated pair (i--j), evaluated in terms of the relative momentum $k^*$ in the pair rest frame (PRF), and calculates all the possible $k^*$ configurations in the phase space for each $Q_3$ value of the i--j--k triplet. The input correlation functions $C(k^*)$ used in the projector method are obtained by selecting p--p,  p--K$^+$ and p--K$^-$ pairs using the same dataset as for the three-body analysis. Unlike the p--p and p--K$^+$ pairs, the p--K$^-$ correlation is significantly affected by the contribution from jet-like events~\cite{ALICE:2011kmy}. To use particles emitted in collisions with the same event shape as in the case of the analysed triplets, p--K$^-$ pairs are selected from \ppKm triplets with $Q_3 < 1$ GeV/$c$, while for p--p and p--K$^+$ pairs such additional requirement is not used as the correlation functions with and without the $Q_3$ selection are in agreement. A $\pm 0.1$ GeV/$c$ variation of the $Q_3$ limit is included in the systematic uncertainties of the projector method and it is combined with the uncertainties propagated from the data points of the p--p, p--K$^+$ and p--K$^-$ correlation functions. The latter include the variations in the selection criteria for particles, variations in the track splitting/merging rejection criteria, and normalisation range.

\begin{figure}[!h]
    \centering
%    [width=0.48\textwidth]
        \includegraphics[width=0.48\textwidth]{Figures/ppKplus_9.pdf}
    \includegraphics[width=0.48\textwidth]{Figures/ppKminus_9.pdf}
    \caption{Correlation functions for (p--p)--K$^+$ (panel $\mathrm{a_1}$), (p--p)--K$^-$ (panel $\mathrm{a_2}$), p--(p--K$^+$) (panel $\mathrm{b_1}$), p--(p--K$^-$) (panel $\mathrm{b_2}$) as well as the total lower-order contributions for \ppKp (panel $\mathrm{c_1}$) and \ppKm (panel $\mathrm{c_2}$) as a function of $Q_3$. The points represent the results obtained using the data-driven approach, with the statistical and systematic uncertainties represented by the error bars and the green boxes, respectively. The grey bands are the expectations of the projector method, with the band width representing the combined statistical and systematic uncertainties.}
    \label{fig:locPPK}
\end{figure}
For the \ppKp and \ppKm systems, the lower-order contributions to the three-particle correlation functions are shown in Fig.~\ref{fig:locPPK}, using the data-driven approach (data points) and the projector method (grey bands). 
The three-particle correlation functions of two correlated protons and an uncorrelated kaon, denoted as (p--p)--K$^+$ and (p--p)--K$^-$, are shown in panels $\mathrm{a_1}$) and $\mathrm{a_2}$), respectively. 
In both cases, the correlation functions are larger than unity in the low $Q_3$ region as a result of the attractive p--p strong interaction, combined with the repulsive Coulomb and quantum statistics effects (see~\cite{ALICE:Run1,acharya2020investigation} for more details).
In the case of correlated p--K$^+$ pairs with uncorrelated protons, p--(p--K$^+$), the correlation function shown in panel $\mathrm{b_1}$) is lower than unity as a result of the repulsive p--K$^+$ strong and Coulomb interactions, consistent with the measurement reported in~\cite{Acharya:2019bsa}.
In the p--(p--K$^-$) correlation function, shown in panel $\mathrm{b_2}$), the main features of the p--K$^-$ interaction are visible~\cite{Acharya:2019bsa}: the cusp structure due to the opening of the $\mathrm{\overline{K}^0}$n channel as well as the bump due to the $\mathrm{\Lambda(1520) \rightarrow p K^{-}}$ decay appear at $Q_3 \approx$ 0.15 GeV/$c$ and $Q_3 \approx$ 0.7 GeV/$c$, respectively.
The total lower-order contributions to the \ppKp, panel $\mathrm{c_1}$), and \ppKm, panel $\mathrm{c_2}$), correlation functions 
evaluated with the data-driven approach and the projector method are in agreement. 
The reduced $\chi^2$ between the two methods, evaluated in the $Q_3$ range shown in Fig.~\ref{fig:locPPK} (21 degrees of freedom) by adding in quadrature the statistical and systematic uncertainties of the data, is 1.6 and 1.4 for \ppKp and \ppKm, respectively. Since the projector method does not depend on the third particle mixing, it provides significantly smaller uncertainties than the data-driven approach and hence is used to extract the cumulants. This choice does not affect the final results of the analysis.
Figure~\ref{fig:CFs} shows the measured \ppKp (left panel) and \ppKm (right panel) correlation functions (data points) compared to the lower-order contributions evaluated using the projector method (grey bands). 
\begin{figure}[!h]
    \centering
    \includegraphics[width=0.495\textwidth]{Figures/ppKplus_CF_6.pdf}
    \includegraphics[width=0.495\textwidth]{Figures/ppKminus_CF_6.pdf}
    \caption{Correlation functions (data points) for \ppKp (left panel) and \ppKm (right panel) compared to the lower-order contributions evaluated using the projector method (grey bands). Statistical and systematic uncertainties are represented by error bars and green boxes, respectively. The band widths represent the combined statistical and systematic uncertainties propagated from the two-particle correlation functions used as input to the projector method. }
    \label{fig:CFs}
\end{figure}

The corresponding cumulants are extracted using Eq. \eqref{eq:cumulant}. 
The measured cumulants include the correlations of the primary particles as well as the feed-down from resonances and particle misidentifications, which need to be accounted for as discussed in~\cite{ALICE:2022vzr}. 
The number of correctly identified primary triplets is about 66\% of the total sample for both \ppKp and \ppKm. The remaining contribution mainly stems from the feed-down of the $\Lambda$ (17\%) and $\Sigma^+$ (7\%) hyperon decays into protons and from the $\phi$ meson (4\%) decay into charged kaons. These numbers are obtained combining the primary and secondary fractions with particle purities. The primary and secondary fractions are extracted using Monte Carlo template fits to the measured distributions of the DCA to the primary vertex~\cite{ALICE:Run1}. The fraction of charged kaons from $\phi$ decays is calculated using the expectations from a thermal model~\cite{VOVCHENKO2019295}.  The correction of the measured cumulants to obtain the contribution from primary triplets is performed following the decomposition procedure adopted in~\cite{ALICE:2022vzr}. 
Due to the absence of theoretical and experimental information on the genuine p--$\Lambda$--K$^\pm$, p--$\Sigma^{+}$--K$^\pm$ and p--p--$\phi$ correlations, the corresponding feed-down contributions to the p--p--K$^{\pm}$ cumulants are considered to be flat, without any specific dependence on $Q_3$. This assumption was tested in the case of p--p--p correlations~\cite{ALICE:2022vzr}, where a similar contribution (about 19\%) for the $\Lambda$ feed-down into protons was found. The results obtained by using the flat and non-flat corrections are found to be in agreement within the uncertainties.
A flat feed-down contribution in the cumulant corresponds to negligible correlations from genuine three-body effects in the mother particle systems X--Y--Z decaying into p--p--K$^{\pm}$. The dominant feed-down contributions coming from pairwise correlations in the triplets, such as (X--Y)--Z and X--(Y--Z), are already removed as they are included in the lower-order correlations. 
The resulting cumulants of the correctly identified primary \ppKp and \ppKm particle triplets are shown in left and right panels of Fig.~\ref{fig:cumulants}, respectively.
\begin{figure}
    \centering
    \includegraphics[width=0.495\textwidth]{Figures/cumulant_ppKplus_5.pdf}
    \includegraphics[width=0.495\textwidth]{Figures/cumulant_ppKminus_5.pdf}
    \caption{Cumulants for the \ppKp (left panel) and \ppKm (right panel) primary triplets. The error bars and the blue boxes on the data points represent the statistical and systematic uncertainties, respectively. The $n_\sigma$ deviations from zero in each bin are shown in the bottom panels, adding in quadrature the statistical and systematic uncertainties of the data.}
    \label{fig:cumulants}
\end{figure}

In the absence of genuine three-particle correlations in the triplets, the measured three-particle correlation function would be identical to the lower-order contributions $C_{3}(Q_3) = C_\mathrm{two\text{-}body}(Q_3)$. Consequently, from Eq. \eqref{eq:cumulant}, the cumulant would be compatible with zero within the experimental uncertainties. In the case of the \ppKp and \ppKm systems, the agreement of the measured cumulants with zero is evaluated by performing a $\chi^2$ test in the region $Q_3 < 0.4$ GeV/$c$. In this region, the relative momentum for all the pairs in the particle triplets is lower than 0.2 GeV/$c$, which is the kinematic region sensitive to the strong interaction. Larger $Q_3$ values are dominated by kinematic configurations in which one of the pairs has a relative momentum larger than 0.2 GeV/$c$, meaning that only two of the three particles in the triplet can be in a kinematic configuration which is favourable for the strong interaction. Thus, at $Q_3 > 0.4$ GeV/$c$ the contributions from the two-body interaction dominate. The local statistical significance at $Q_3 < 0.4$ GeV/$c$ is obtained from the p-value of the $\chi^2$ distribution, which is converted in a number $n_\sigma$ of Gaussian standard deviations. As the systematic uncertainties in different $Q_3$ bins are correlated, the $\chi^2$ is calculated using the cumulants extracted from each systematic variation with the corresponding statistical uncertainties. Finally, the average $\chi^2$ in the interval $Q_3 < 0.4$ GeV/$c$ is extracted.
The corresponding $n_\sigma$ values are 2.5 and 1.5 for \ppKp and \ppKm, respectively. As the $Q_3$ range of the genuine three-body effects is not known a priori, the $n_\sigma$ values have been evaluated as well in the region $Q_3 < 0.2$ GeV/$c$ and they are found to be 2.7 and 1.4 for \ppKp and \ppKm, respectively.  Such results indicate that the measured \ppKp and \ppKm correlation functions are compatible, within the quoted significance levels, with the assumption of pairwise correlations in the triplets without additional contributions from genuine three-body effects. More solid conclusions require a larger data sample to reduce the statistical uncertainties and full-fledged theoretical calculations for the three-particle correlation functions.
In order to provide a comparison of the kinematic range accessible with this measurement to the study of K$^+$ and K$^-$ interactions in light nuclei, the momenta of the three particles in the triplets with $Q_3 < 0.4$ GeV/$c$ are evaluated in the PRF of the two protons. The momentum of each proton in the PRF is $p_\mathrm{p}^* < 180$ MeV/$c$, compatible with the typical Fermi momenta of nucleons in light nuclei (e.g.~the Fermi momentum in Carbon-12 is about 220 MeV/$c$). In the rest frame of the two protons, the momentum of the K$^+$ and K$^-$ ranges from 30 MeV/$c$ (in the first bin in Fig.~\ref{fig:cumulants}) to 130 MeV/$c$ (eighth bin in Fig.~\ref{fig:cumulants}). 
This demonstrates that even for low kaon momenta, i.e. at energies close to the \ppKp and \ppKm thresholds, three-body effects such as kaonic bound state formation below threshold or three-body interactions do not contribute significantly to the measured correlation functions. 
These results provide additional experimental information for theoretical models aiming to understand the role of genuine three-body effects in \ppKp and \ppKm systems. 
The LHC Run 3 data taking will deliver a larger data set for more detailed studies of the K$^+$ and K$^-$ three-body dynamics in the low momentum region, down to the energy threshold of the triplets.



