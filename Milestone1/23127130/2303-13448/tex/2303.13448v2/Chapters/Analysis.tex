The analysed data sample of  \pp collisions at \thirteen was recorded by ALICE~\cite{ALICE, ALICEperf,ALICE:2022wpn} during the LHC Run 2 (2015--2018) data-taking period. In the following, the information from the V0 detector system~\cite{Abbas:2013taa}, the Inner Tracking System (ITS)~\cite{ALICEITS}, the Time Projection Chamber (TPC)~\cite{TPC} and the Time-Of-Flight (TOF) detector~\cite{TOF} is used. The V0 detector is employed to trigger on high-multiplicity (HM) events. This trigger selects events within the 0.17\% largest charged-track multiplicity of the INEL~$>0$ class, which is defined as inelastic collisions with at least one measured charged particle in the pseudorapidity range $|\eta|<1$~\cite{Abbas:2013taa}. This condition results in an average of about 30 charged particles in the range $|\eta|<0.5$~\cite{ALICE:pOmega}, hence increasing the probability of finding triplets of the desired particle species with respect to the minimum-bias sample.
The primary vertex position is reconstructed with the combined information of the ITS
and the TPC, and, independently, with track segments in the two innermost layers of the ITS.
Only events with a reconstructed primary vertex position along the beam axis within
10 cm from the centre of the ALICE detector are selected.
A total of 10$^9$ HM events are used in this analysis. 
The \ppKp and \ppKm data samples are built 
by combining three charged-particle tracks (triplets) reconstructed with the TPC.  
Assuming the same interactions in the particle and antiparticle systems, triplets of particles and the corresponding antiparticles are combined: \ppKp $\equiv$ (~\ppKp~$\oplus$~\appKp) and \ppKm $\equiv$ (~\ppKm~$\oplus$~\appKm).
The agreement of the corresponding correlation functions confirmed the validity of this assumption. 


Particles and antiparticles are identified using the same kinematic and topological selections.  
Protons and kaons are selected in the range $|\eta| < 0.8$ and in the transverse momentum intervals $\pt \in$ (0.5--4.05)\,\GeVc and $\pt \in$ (0.2--2.5)\,\GeVc, respectively. 
A minimum of 80 space points (hits) inside the TPC, out of the total 159, is required to guarantee track quality and good momentum resolution. 
Particle identification (PID) is conducted by requiring that the measured energy loss (\dEdx) in the TPC gas is compatible with the expected one from protons and kaons within three standard deviations ($\sigma$).         
For high momentum particles, the \dEdx information is combined with the time-of-flight measurement provided by the TOF, using a 3$\sigma$ selection on the expected value for a given particle hypothesis at a given momentum.
The PID selection for protons is described in detail in~\cite{ALICE:Run1}. The selection of kaons is based on the procedure described in ~\cite{ALICE:2022yyh} with several changes as explained in the following. 
TPC reconstructed tracks are identified as kaons either by using TPC PID information for momenta lower than 0.85 GeV/$c$, or, if a signal in the TOF is matched to the TPC track, by using the combined PID information from TPC and TOF up to a momentum of 2.5 GeV/$c$. 
To improve the purity of the kaon selection, the TPC and TOF information are also used to reject candidates that are compatible with the pion or electron hypothesis.
To reject particles that are non-primary or come from pile-up collisions, the distance of closest approach (DCA) to the primary vertex of the tracks is required to be less than 0.1~cm in the transverse plane and less than 0.2~cm along the beam axis. The purity of the proton and kaon candidates is estimated employing Monte Carlo simulations based on PYTHIA 8~\cite{Sjostrand:2014zea} (Monash 2013 Tune) event generator with a dedicated high-multiplicity selection to mimic the V0 high-multiplicity trigger and the \geant package~\cite{Agostinelli:2002hh,Uzhinsky:2011zz}. The purity averaged over the \pt ranges of the identified protons and kaons is about $98\%$ for both particle species.

The number of selected and analysed triplets amounts to 4530 for \ppKp, 3161 for \appKp, 6200 for \ppKm and 4937 for \appKm in the femtoscopic region $Q_3 < 0.4$ \GeVc. The kinematic variable $Q_3$, which is used in three-body analyses~\cite{ALICE:2022vzr,Dhevan}, is a Lorentz-invariant scalar defined as $Q_3=\sqrt{-q_{12}^2-q_{23}^2 -q_{31}^2}$, where $q_{ij}$ is the norm of the relative four-momentum of the pair $ij$ in the triplet $q_{ij}^\mu = 2\ [ m_j/(m_i~+~m_j) ~p_i^\mu~-~m_i/(m_i + m_j)\ p_j^\mu ]$. 
The systematic uncertainties of the correlation functions are evaluated by performing simultaneous variations of the selection criteria for particles and antiparticles. For protons and antiprotons, the same variations for the track selection and PID criteria as in~\cite{ALICE:2022vzr} are performed. Similar variations are used for the kaons and antikaons. In order to account for the correlations between the systematic uncertainties, the variations are randomly combined in sets in which at least one selection criterion is changed. Large statistical fluctuations introduced by the variations of the selection criteria are avoided by requiring that 
the yield of the triplets changes by less than 10\% with respect to the standard selection in the femtoscopic region $Q_3 < 0.4$ \GeVc. The main contribution at low $Q_3$ ($\approx 0.15$ \GeVc) is given by the variation of the DCA selection and it is found to be smaller than 5\%.

