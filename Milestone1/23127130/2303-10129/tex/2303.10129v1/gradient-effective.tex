
% As a first example,
Let us start with the linear anomalous Hall conductivity given by
second term in \eq{sigma-10}. To evaluate it we need the net Berry
curvature $\Omega^{ab}_{\rm A}\equiv\sum_n^{\rm A}\,\Omega^{ab}$ of the
occupied (A) states at each $\k$, where the Berry curvature of an
individual state is
$\Omega^{ab}_n=-2\Im\lip{\partial_a n}{\partial_b n}$ according to
\eq{curv-n-def} (the index $\k$ has been omitted for brevity). Using
the standard result from first-order non-degenerate perturbation theory, one
obtains~\cite{xiao-rmp10,vanderbilt-book18}
%
\beq
\ket{\partial_a n}=-i\alpha_n\ket{n}+
\sum_{j\not= n}\,\ket{j}
\frac{\me{j}{\partial_a \Ham\eff}{n}}{\en-\varepsilon_j}
\eql{dn-1}
\eeq
%
where $\alpha_n$ can be any real number.
%It is easy to find that the $l=n$ term does not contribute to 
%Berry Curvature. 
Inserting the completeness relation $\one=\sum_l\all\,\ket{j}\bra{j}$
The Berry curvature of a single band is 
\beq
\Omega_n^{ab}=-2\Im\sum_{j\not= n}\,
\frac{\me{n}{\partial_a \Ham\eff}{j}\me{j}{\partial_b \Ham\eff}{n}}
{\left(\en-\ej\right)^2}\,.
\eql{Omega-kn}
\eeq
%The summation contains $N-1$ terms, which is a small number for simple
%models. \Eq{Omega-kn} is therefore a very convenient means of evaluating the Berry curvature in
%effective models.

In typical applications one is more interested in the overall
properties of groups of energy eigenstates than in the properties of
individual eigenstates. For \eqr{sigma-10}{sigma-30}, groups A and B comprise all states
below and above the Fermi level at a given $\k$, respectively.
For the linear anomalous Hall conductivity 
%given by second term in \eq{sigma-10}; there, the quantity of interest is 
the net Berry curvature
%
%\beq
%\Omega^{ab}_{\rm A}=\sum_n^{\rm A}\,\Omega^{ab}_n
%\eeq
%
of the occupied states
%Using \eq{curv-n-def} and inserting \eq{completeness}
reads
%the double summation over states belonging to the same subspace (A or B) vanishes leaving
%
\beq
\Omega_{\rm A}^{ab}=-2\Im\sum_n^{\rm A}\sum_l^{\rm B}\,
\frac{\me{n}{\partial_a \Ham\eff}{l}\me{l}{\partial_b \Ham\eff}{n}}
{\left(\en-\el\right)^2}\,.
\eql{Omega-A-1}
\eeq
%
where the double summation over states belonging to set A cancels out.
Naturally, \eq{Omega-A-1} reduces to \eq{Omega-kn} when group A contains a single state.

By virtue of having $(\en-\el)^2$ in the denominator, \eq{Omega-A-1}
becomes resonantly enhanced in regions of the BZ where the gap between
strongly-coupled A and B states is small.  Conversely, the lack of
energy denominators involving pairs of A states or pairs of B states
means that $\Omega_{\rm A}^{ab}=-\Omega_{\rm B}^{ab}$ does not react
strongly to band crossings and avoided crossings within each sector.

All of the above are familiar results.
% properties of the Berry curvature.
Consider now the gradient of the Berry curvature,
% $\partial_c\calF^{ab}_{\rm A}$,
which
governs the quadratic anomalous Hall conductivity given by the second
term in \eq{sigma-20}. Like $\Omega_{\rm A}^{ab}$ itself,
$\partial_c\Omega_{\rm A}^{ab}$ should only react strongly to small
energy gaps between the two groups A and B, not
% to % internal
% (quasi)degeneracies
within each group. However, direct differentiation of \eq{Omega-A-1}
leads to an expression
% for $\partial_c\calF_{\rm A}^{ab}$
containing
energy denominators between pairs of states within the same group. The
problematic terms appear when differentiating the eigenstates
% in \eq{Omega-A-1}
using \eq{dn-1}.
% nondegenerate perturbation theory, which
% gives~\cite{vanderbilt-book18}
% %
% \ism{Replaced $n^\prime\rightarrow j$.}
% %
% \beq
% \lket{n_{,a}}=-i\alpha_n\lket{n}+
% \sum_{j\not= n}\,\lket{j}
% \frac{\lme{j}{H_{,a}}{n}}{\en-\epsilon_j}
% \eql{dn-1}
% \eeq
% %
% where $\alpha_n$ can be any real number.
If $n\in{\rm A}$, the summation over $j$ includes not only the B states
but also the other A states, which can be arbitrarily close in energy
to state $n$. Such unwanted terms cancel each other out in the final
result for $\partial_c\Omega_{\rm A}^{ab}$, but making that cancelation
explicit is not as straightforward as in the case of
\eq{Omega-A-1}. For $\partial^2_{cd}\Omega_{\rm A}^{ab}$ and higher
derivatives, achieving that cancellation becomes increasingly more
difficult.

To circumvent the above problem, we will develop a systematic
procedure for differentiating geometric quantities in such a way that
the spurious terms are absent by construction.
% In the case of
% When used
% to evaluate
% \eq{Omega-A-grad-def}
% $\partial_c\calF_{\rm A}^{ab}$,
% starting from \eq{Omega-A-1} for
% $\Omega_{\rm A}^{ab}$,
That procedure yields
%
\ism{Refer here to a later section where this expression is derived.}
%
%\begin{widetext}
%\begin{multline}
%\partial_c\Omega_{\rm A}^{ab}=-2\Im\sum_n^{\rm A}\sum_l^{\rm B}\,
%\frac{1}{\left(\en-\el\right)^2}
%\Bigg\{
%\me{n}{\partial_a \Ham\eff}{l}\me{l}{\partial^2_{bc}\Ham\eff}{n}\\
%+\me{n}{\partial_a\Ham\eff}{l}
%\Bigg[
%\sum_{n'}^{\rm A}\,
%\frac{\me{l}{\partial_b \Ham\eff}{n'}}{\el-\varepsilon_{n'}}\me{n'}{\partial_c \Ham\eff}{n}
%-\sum_{l'}^{\rm B}\,\me{l}{\partial_b \Ham\eff}{l'}
%\frac{\me{l'}{\partial_c \Ham\eff}{n}}{\varepsilon_{l'}-\en}
%+(b\leftrightarrow c)\Bigg]-(a\leftrightarrow b)\Bigg\}\,.
%\eql{Omega-A-grad-1}
%\end{multline}
%\end{widetext}
%
\tents{
\begin{widetext}
\begin{multline}
\partial_c\Omega_{\rm A}^{ab}=-2\Im\sum_n^{\rm A}\sum_l^{\rm B}\,
\frac{1}{\left(\en-\el\right)^2}
\Bigg\{
\Hcoma{a}{n}{l}\Hcomatwo{bc}{l}{n}
+\Hcoma{a}{n}{l}
\Bigg[
\sum_{n'}^{\rm A}\,
\frac{\Hcoma{b}{l}{n'} \Hcoma{c}{n'}{n}}{\el-\varepsilon_{n'}}
-\sum_{l'}^{\rm B}\,
\frac{\Hcoma{b}{l}{l'} \Hcoma{c}{l'}{n}}{\varepsilon_{l'}-\en}
+(b\leftrightarrow c)\Bigg]-(a\leftrightarrow b)\Bigg\}\,.
\eql{Omega-A-grad-1}
\end{multline}
\end{widetext}
%
% This sum-over-states formula
for the gradient of the Berry curvature
of an effective-Hamiltonian model, where we have used a shortened notation
$\Hcoma{a}{n}{l}\equiv\me{n}{\partial_a\Ham}{l}$
}
% is our first result.
Like \eq{Omega-A-1}, \eq{Omega-A-grad-1} contains energy denominators
between A and B states only, making it well-suited for numerical work.
\tents{Interestingly, \Eq{Omega-A-grad-1} is symmetric under $b\leftrightarrow c$, 
which is connected to the known property of the berry curvature dipole \eqs{dp-sea}{dp-surf}
to have zero trace. (see \aref{traceless} for details)}

\tents{In the following sections we will demonstrate how to systematically derive 
the derivatives of any order of Berry curvature, orbital magnetic moment,
quantum metric and similar quantities, and at the end of \sref{der-F-H} the derivation of \eq{Omega-A-grad-1} will emerge. }





%%% Local Variables:
%%% mode: latex
%%% TeX-master: "pap"
%%% End:
