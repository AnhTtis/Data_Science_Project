%======================
\section{Introduction}
\secl{intro}
%======================

\ism{Maybe start by saying that covariant derivatives in $\k$ space
  play a central role in the microscopic theory of nonlinear {\it
    optical} responses.  Here, we show that nonlinear {\it transport}
  responses are also most naturally formulated in terms of covariant
  derivatives of geometric quantities characterizing the occupied
  Bkloch states.}

The absence of spatial inversion and/or time-reversal symmetry gives
rise to various nonlinear
% charge-
transport phenomena in
solids~\cite{tokura-nc18},
%
\ism{Add the two 2021 reviews here?}
%
including
unidirectional magnetoresistance~\cite{rikken-prl01,avci-natphys15} and
nonlinear Hall
effects~\cite{deyo-arxiv09,sodemann-prl15,ma-nature19,kang-natmater19,huang-arxiv21,zhang-arxiv20}.
% \Red{Addititonal
%   nonlinear effects appear if time-reversal symmetry is also
%   broken. An example is unidirectional
%   magnetoresistance~\cite{avci-natphys15}, which can be viewed as a
%   ``spontaneous'' version of the magnetochiral anisotropy induced by a
%   magnetic field.}
%
Such effects are encoded in the expansion of the
% charge-
current density in powers of the applied electric and magnetic fields,
%
\begin{align}
j_a&=\sigma^{10}_{ab}E_b+\sigma^{11}_{ab\alpha}E_b B_\alpha+
\sigma^{12}_{ab\alpha\beta}E_b B_\alpha B_\beta \nn
&+\sigma^{20}_{abc}E_bE_c+\sigma^{21}_{abc\alpha}E_b E_c B_\alpha \nn
&+\sigma^{30}_{abcd}E_bE_cE_d \nn
&+\ldots\,.
\eql{sigma-expansion}
\end{align}
%
% where the subscripts denote Cartesian indices.
%
% In the limit of static or low frequency fields,
At low frequencies,
the expansion coefficients
% \Red{(magneto)}
% conductivity tensors
% % appearing
% in \eq{sigma-expansion} 
can be evaluated using
semiclassical~\cite{xiao-rmp10,gao-fp19} as well as fully
quantum-mechanical methods~\cite{morimoto-prb16,watanabe-prr20}. The
resulting expressions contain two types of terms:
%
% \ism{The terms ``intraband'' and ``interband'' usually mean something
%   different: intraband vs interband optical conductivity, depending on
%   how the optical frequency compares with the fundamental gap. Do we
%   need to clarify that this is {\it not} what we mean?}
%
(i)
intraband terms that only involve intrinsic geometric properties of
the unperturbed Bloch states (Berry curvature, effective mass,
magnetic moment, and quantum metric);
%
% \ism{This definition of intraband terms includes the ``horizontal
%   mixing'' corrections depicted in Fig.~1 of
%   Ref.~\cite{gao-fp19}. These contain the quantum metric as well as
%   its first $\k$ derivative, which appears in the Christoffel symbol.}
%
% and
(ii) interband terms that take into account how the Bloch states
change, via vertical band mixing, under the applied
fields~\cite{gao-fp19}.

This paper deals with the evaluation of intraband nonlinear
conductivities, which typically contain {\it derivatives} of
% Berry curvature
% (and related quantities)
geometric quantities with respect to crystal momentum $\k$.  While the
focus will be on {\it ab initio} implementations based on Wannier
functions, the formalism presented here can also be combined with
effective-Hamiltonian methods such as $\k\cdot{\bf p}$ and
tight-binding.

% As a concrete example,
Consider the current response at zero magnetic field,
%
\beq
j_a=\sigma^{10}_{ab}E_b+\sigma^{20}_{abc}E_bE_c+
\sigma^{30}_{abcd}E_bE_cE_d+\ldots
\eql{sigma-expansion-E}
\eeq
%
Working in the constant relaxation time approximation and neglecting
interband contributions, one obtains~\cite{zhang-arxiv20}
%
% \ism{By writing the Berry curvature as an antisymmetric tensor these
%   expressions acquire a nice symmetry, I think. But later on we can
%   convert to axial-vector form, if it simplifies things. (The vector
%   form is useful to discuss the tracelessness of the Berry dipole, for
%   example.)}
%
\begin{align}
\sigma^{10}_{ab}&=\frac{e^2}{\hbar}\int\dk\sum_n\,f_0(\enk)
\left[(\tau/\hbar)\partial^2_{ab}\enk -\Omega^{ab}\bnk\right]\,,
\eql{sigma-10}\\
\sigma^{20}_{abc}&=\frac{e^3\tau}{\hbar^2}\int\dk\sum_n\,f_0(\enk)
\left[-(\tau/\hbar)\partial^3_{abc}\enk+\partial_c\Omega^{ab}\bnk\right]\,,
\eql{sigma-20}\\
\sigma^{30}_{abcd}&=\frac{e^4\tau^2}{\hbar^3}\int\dk\sum_n\,f_0(\enk)
\bigg((\tau/\hbar)\partial^4_{abcd}\enk\nn
&\,\,\,\,\,\,\,\,\,\,\,\,\,\,\,\,\,\,\,\,\,\,\,\,\,\,\,\,\,\,\,\,\,\,\,\,\,\,
\,\,\,\,\,\,\,\,\,\,
%\,\,\,\,\,\,\,\,\,
-\frac{3}{4}\partial^2_{cd}\Omega^{ab}\bnk
-\frac{1}{4}\partial^2_{bc}\Omega^{ad}\bnk
\bigg)\,.
\eql{sigma-30}
\end{align}
%
where $\tau$ is the relaxation time, $\dk\equiv d^3k/(2\pi)^3$,
$\partial_a\equiv\partial/\partial k_a$, and the integrals are over
the first Brillouin zone (BZ). The band energy is denoted $\enk$, $f_0(\enk)$
is the Fermi-Dirac distribution function,
and
%
\beq
\Omega^{ab}\bnk=-2\Im\ip{\partial_a\unk}{\partial_b\unk}
% =-\calF^{ba}\bnk
\eql{curv-n-def}
\eeq
%
is the Berry curvature, where $\ket{\unk}$ is the periodic part of a
Bloch state $\ket{\psi\bnk}$. \tents{\Eqr{sigma-10}{sigma-30} are written in the so-called "Fermi-sea" form,
meaning that all states below the Fermi level contribute to the integral. }
%
%
% In writing
% these expressions, we have used the notation
% %
% \beq
% {\cal O}^{,a}\equiv\frac{\partial{\cal O}}{\partial k_a}\,,\quad
% {\cal O}^{,ab}\equiv\frac{\partial^2{\cal O}}{\partial k_a\partial k_b}\,,\quad
% {\cal O}^{ab,c}\equiv\frac{\partial{\cal O}^{ab}}{\partial k_c}
% \eql{der-notation}
% \eeq
% %
% for the partial derivatives of various
% % $\k$-dependent
% objects.

\Eq{sigma-10} gives the linear conductivity: the first term is the
Ohmic Drude conductivity expressed in terms of the inverse effective
mass of the occupied states; the second term, given by the net Berry
curvature of the occupied states, describes an intrinsic anomalous
Hall effect in magnetic conductors~\cite{nagaosa-rmp10}.
\Eq{sigma-20} gives the quadratic conductivity: the first term is
Ohmic and it describes unidirectional magnetoresistance in magnetic
acentric conductors~\cite{zelezny-arxiv21}, while the second term
describes an anomalous Hall effect in nonmagnetic acentric
conductors~\cite{deyo-arxiv09,sodemann-prl15,ma-nature19,kang-natmater19}.
\Eq{sigma-30} describes cubic Ohmic and anomalous Hall responses,
which so far have only been studied
theoretically~\cite{parker-prb19,zhang-arxiv20}.  Note that
higher-order conductivities contain higher-order derivatives
of % either
the band dispersion
% or
and of the Berry
curvature.
% $\enk$ and $\Omega^{ab}\bnk$.
For magnetoconductivities such as
$\sigma^{11}_{ab\alpha}$, $\sigma^{12}_{ab\alpha\beta}$ and
$\sigma^{21}_{abc\alpha}$ in \eq{sigma-expansion}, the intrinsic
magnetic moment
% of the Bloch states
% ${\bf m}\bnk$
and its derivatives are needed as
well~\cite{xiao-rmp10,gao-fp19,morimoto-prb16,lahiri-arxiv21}.
%
\ism{Added citation to the recent preprint that worked out the $E^2B$
  Berry-Boltzmann terms.}
%
% \ism{I guess the quantum metric will appear as well at higher orders
%   in the fields, do you agree?}

% First-principles calculations of nonlinear conductivities and
% magnetoconductivites are still in their infancy, calling for the
% development of improved methodologies.
The first-principles evaluation of both terms in the linear
conductivity~\eqref{eq:sigma-10} is by now a fairly routine task.
% In the case of the linear conductivity given by \eq{sigma-10}, it is
% well known how to compute the needed ingredients from first
% principles.
A popular approach
% that is both accurate and efficient
is
Wannier interpolation~\cite{wang-prb06,yates-prb07}, where a
Slater-Koster type of interpolation is carried out for the quantities
of interest after mapping the low-energy {\it ab initio} electronic
structure onto a basis of localized Wannier functions.

When it comes to nonlinear conductivities, {\it ab initio}
calculations are still quite recent.
In the case of intraband
responses such as \eqs{sigma-20}{sigma-30}, a possible strategy is as
follows. First compute the inverse effective mass and Berry curvature
on a dense $\k$ mesh by Wannier interpolation, and then evaluate their
$\k$ derivatives by finite differences. This strategy was used in
several recent studies of nonlinear anomalous Hall
effects~\cite{zhang-prb18,zhang-2dmater18,zhang-arxiv20,he-arxiv21}.


\tents{Another common strategy is to employ integration by parts in 
\eqr{sigma-20}{sigma-30} and thus transfer the derivative  
from the Berry curvature $\Omega^{ab}\bnk$ to the Fermi-Dirac distribution
$f_0(\enk)$. For instance, the second term of \eq{sigma-20} is governed by 
the so-called "Berry curvature dipole"~\cite{sodemann-prl15} which in the Fermi-sea formulation is 
given by
\beq
\mathcal{D}_{cd}^{\rm sea} = \epsilon_{abd} \int \dk \sum_n \partial_c \Omega_{\k n}^{ab} f_0(\enk)~,
\eql{dp-sea}
\eeq 
Using integration by parts it may be rewritten in as 
\beq
\mathcal{D}_{cd}^{\rm surf} = \epsilon_{abd} \int \dk \sum_n  \Omega_{\k n}^{ab} \partial_c\enk  \left(- \frac{\partial f_0}{\partial \varepsilon}\right)_{\varepsilon = \enk}  ~,
\eql{dp-surf}
\eeq 
At low temperature the derivative of the distribution function $f_0'$ is a narrow peak, which ensures that only electronic states that are close to the Fermi level contribute to the integral. Therefore, such formulations are called ``Fermi-surface'' integrals. 
As we will show, in numerical simulations, such integrals require denser sampling of the Brillouin zone, compared to the ``Fermi-sea'' integrals. Moreover, for magnetoconductivities magnetoconductivities such as
$\sigma^{11}_{ab\alpha}$, $\sigma^{12}_{ab\alpha\beta}$ and
$\sigma^{21}_{abc\alpha}$ it is not always possible to get rid of derivatives of Berry curvature and orbital magnetic moment simultaneously. 
}

In this work, we develop an alternative approach where
derivatives of geometric quantities
% such
% as $\partial_c\calF^{ab}$
are evaluated perturbatively,
% at each grid point,
% by Wannier interpolation,
without resorting to finite differences.  Importantly, the expressions
we obtain are oblivious to band crossings and avoided crossings away
from the Fermi level, as expected on physical grounds.  We emphasize
that this is not the case if one differentiates 
% for the expressions obtained via a
% straightforward differention of
% the formulas for
the Berry curvature in a naive way:
% and related geometric quantities:
the resulting expression contains spurious terms that react strongly
to remote level crossings (see for example
Ref.~\cite{morimoto-prb16}).  To obtain well-behaved gradient formulas
we start from gauge-covariant matrix objects such as the non-Abelian
Berry curvature, and differentiate them using a 
%generalized derivative
``covariant derivative''
%
\ism{I think we should just call it ``covariant derivative'' (see my
  notes from November 2021).}
%
that preserves gauge covariance.  The nonlinear conductivities are
then expressed as gauge-invariant traces.
% of the resulting gauge-covariant matrices.


The manuscript is structured as follows: \tents{In \sref{preliminary}, 
we discuss the calculation of the derivative of the Berry curvature 
for an effective model in order to demonstrate the essence of the problem. 
In \sref{generalized-der} we introduce the covariant derivative and discuss its properties. 
In \sref{geometric-quantity} we demonstrate, how multiple geometrical quantities of interest may be formulated in a gauge-covariant way and reduced to two objects $\Fcov^{ab}_{mn}$ and  $\Hcov^{ab}_{mn}$. 
\sref{wannier-interpolation} explains how to evaluate those objects using Wannier interpolation, or for a tight-binding model. 
Finally, in \sref{result-Te}, we present the first-principle simulation of the Berry curvature dipole in Te, where we show that the "Fermi-sea" formula with the covariant gradient of the Berry curvature has better convergence and is more robust than the "Fermi-surface" formula.
}
%%% Local Variables:
%%% mode: latex
%%% TeX-master: "pap"
%%% End:
