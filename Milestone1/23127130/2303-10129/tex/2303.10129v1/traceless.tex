\section{Tracelessness of Berry dipole}
\secl{traceless}
As noted before, \eq{Omega-A-grad-1} is symmetric under permutation of indices $b\leftrightarrow c$,
%
\beq
\Omega_{\rm A}^{ab,c}=\Omega_{\rm A}^{ac,b}\,.
\eql{Omega-symm}
\eeq
%
To make sense of this symmetry, let us recast the Berry curvature an a
pseudovector, $\Omega^c_{\rm A}=\epsilon_{abc}\Omega^{ab}_{\rm A}/2$.
Applying $\partial_c$ to both sides of this relation and then using
\eq{Omega-symm} gives
%
\beq
\bnabla_\k\cdot\bOmega_{\rm A}=0\,.
\eql{div}
\eeq
%
Thus, the net Berry curvature of a group of states is divergence-free
everywhere in the BZ. This is a known result for a single band, because 
Berry curvature is a curl of  Berry connection 
$\bOmega_n = i\bnabla_\k \times \ip{n}{\bnabla_\k n}$,
and a divergence of a curl is always zero. 
  This is true even at chiral touching points,
where the Berry curvature of an individual band has nonzero
divergence.  The reason is that each chiral node acts as a monopole
source of Berry curvature for one of the touching bands and as a sink
for the other; but since degenerate states must belong to the same
group A or B (recall that the two groups were assumed to be separated
in energy), the monopoles cancel each other out.  It follows from
\eq{div} that the ``Berry curvature dipole'
tensor'' \eqref{eq:dp-sea}
%
is always traceless. Moreover the integrand of \eq{dp-sea} is 
traceless at every k-point, therefore the resulting tensor has a zero trace 
even if integration is done on a coarse $\kk$-grid. In turn, the integrand
of \eq{dp-surf} is not traceless, therefore the tensor $\mathcal{D}_{cd}^{\rm surf}$
becomes traceless only after integration is performed with high precision. 
Naturally, in an accurate calculation  $\mathcal{D}_{cd}^{\rm surf}=\mathcal{D}_{cd}^{\rm sea}$,
however that level of precision is not always easy to achieve.
%