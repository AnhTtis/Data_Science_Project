\section{Role of Wannier Centers}
\secl{wannier-centers}

Interesting to note that quantities \eqref{eq:O-R} were defined in \cite{wang-prb06} 
and \cite{lopez-prb12} (and in Wannier90 code) with $\bt_i=0$.
As far as inclusion of $\bt$ in the phase factors correspond to a phase choice 
of the basis set, the computed physical observables should not depend on the value of $t$.
Below we will show that the total value of $\Fcov^{ab}_{mn}$ and $\Hcov^{ab}_{mn}$ 
 defined by \eqs{Fab-wanint-all}{Hab-wanint-all} does not depend on $\bt_i$. 
However, the particular value of "internal" and "external" terms will depend on $\bt$. 

In particular, setting  $\bt_i$ equal to the the Wannier charge center $\me{0j}{\hat{\rr}}{0j}$
will make  $\Fcov^{ab,\rm (ext)}_{mn}$ vanish in the "tight-binding" limit, 
where $\me{\RR i}{\hat{\rr}}{\RR'j} = \delta_{\RR\RR'}\delta_{ij}$. 
\stm{what about 
$\Hcov^{ab}_{mn}$ ? }
Also, in the ab initio calculations, the computationally heavy "external" terms 
become smaller with this choice,thus allowing to evaluate them on a coarser grid, 
to achieve the same absolute accuracy of the total Berry curvature.


 Denoting qqunatities defined by Eqs. \ref{eq:Oww-2} at $\bt_i=0$ 
 as $\zerot{\mathbb{A}}$,   $\zerot{\mathbb{B}}$,  $\zerot{\mathbb{C}}$,  $\zerot{\mathbb{F}}$ 
we find the following relations:
\beas{conventions-connection}
\mathbb{A}^a_{ij}(\RR) &=& \zerot{\mathbb{A}}^a_{ij}(\RR) - \delta_{ij}\delta_{R,0}t_j^a\\
\mathbb{B}^a_{ij}(\RR) &=& \zerot{\mathbb{B}}^a_{ij}(\RR) - \zerot{\cal H}_{ij}(\R) t_j^a\\
\mathbb{C}^{ab}_{ij}(\RR) &=& \zerot{\mathbb{C}}^{ab}_{ij}(\RR) 
    - t_i^a\zerot{\mathbb{B}}_{ij}^b(\RR) - \zerot{\mathbb{B}}_{ji}^{a*}(-\RR) t_j^b\nonumber\\
&&    + t_i^a \zerot{\cal H}_{ij}(\R) t_j^b\\
\mathbb{F}^{ab}_{ij}(\RR) &=& \zerot{\mathbb{F}}^{ab}_{ij}(\RR) 
    - t_i^a\zerot{\mathbb{A}}_{ij}^b(\RR) - t_j^b\zerot{\mathbb{A}}_{ij}^a(\RR) \nonumber\\
    &&+ \delta_{ij}\delta_{\R,0} t_i^a t_i^b 
\eeas
%
Defining  $\overline{t}^a_{ij}  = U^*_{j'i}t_{j'}^a U_{j'i}$ in accordance with \eq{O-mn},
we arrive at the following relationsfpr the bar quantities:
%
\beas{conventions-connection-hamgauge}
\overline{A}^a_{ij} &=& \zerot{\overline{A}}^a_{ij} - \overline{t}^a_{ij}\\
\overline{D}^a_{ij} &=& \zerot{\overline{D}}^a_{ij} - i\overline{t}^a_{ij}\\
\overline{B}^a_{ij} &=& \zerot{\overline{B}}^a_{ij} - \zerot{\overline{H}}_{ij'}\overline{t}^a_{j'j}\\
\overline{C}^{ab}_{ij} &=& \zerot{\overline{C}}^{ab}_{ij} - (\zerot{\overline{B}}^{a\dagger})_{ij'}\overline{t}^b_{j'j} 
  - \overline{t}^a_{ij'} \zerot{\overline{B}}^{b}_{j'j} 
  - \overline{t}^a_{ii'}  \zerot{\overline{H}}_{i'j'} \overline{t}^b_{j'j'}
\\
\overline{F}^{ab}_{ij} &=& \zerot{\overline{F}}^{ab}_{ij} - \zerot{\overline{A}}^{a}_{ij'}\overline{t}^b_{j'j} 
- \overline{t}^a_{ij'} \zerot{\overline{A}}^{b}_{j'j} - \overline{t}^a_{ij'} \overline{t}^b_{j'j}
\eeas
%
Note, that summation here is performed over all wannierised states $i',j'$, no matter to which set ($A$ or $B$) 
belong the states $i,j$.
Substituting these relations into \eq{Fab-wanint-all} it is straightforward to show that 
$\Fcov^{ab}_{mn}=\zerot{\Fcov}^{ab}_{mn}$. However, for $\Fcov^{ab}_{mn}$ we get 

\begin{multline}
\Hcov^{ab}_{mn}-\zerot{\Hcov}^{ab}_{mn} = t^a_{mm'}\left(H_{m'n'}A_{n'n}^b - B_{m'n}^b\right) + \conjabmn 
\end{multline}

However this differnece vanishes if we apply \eq{B-frozen}. We should note that the relation 
\eqref{eq:B-frozen} is not automatically sattisfied upon Wanier interpolation, and therefore 
it is important to enforce it by hand, and follow the procedure prescribed by \aref{Bfrozen}

Due to independence of the final result on the values of $\bt_i$,
the definition of "Wannier center" may be understood broadly --- 
not only as the Wanier charge center $\me{0j}{\hat{\rr}}{0j}$, but also, e.g., 
as the position of an atom, on which the Wannier function is located.




%%% Local Variables:
%%% mode: latex
%%% TeX-master: "pap"
%%% End:
