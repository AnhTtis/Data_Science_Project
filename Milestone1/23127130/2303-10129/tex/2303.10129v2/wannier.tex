%\section{Berry curvature and its gradient in the minimal tight-binding formalism} 
\subsection{Wannier functions and effective models}
\secl{Wannier}
In this section we introduce the necessary notation, and 
briefly recall the spirint of Wannier interpolation of Berry curvature, 
closely following Ref.~\cite{wang-prb06}.

Wannier functions $\ket{\RR j}$ form a localized orthonormal basis for the description of electron bandstructure.\cite{Marzari-MLWF-rmp}
The eigenvalues of the Wannier Hamiltonian 
%
\beq
\Ham\w_{ij}(\kk)=\sum_\R\,e^{i\k\cdot(\R+\bt_j-\bt_i)}\Ham\w_{ij}(\R)\,.
\eql{H-k}
\eeq
%
accurately reproduce the eigenvalues of the Bloch bands computed from first principles. 
Here $\RR$ are lattice vectors, and the matrix elements $\Ham\w_{ij}(\R)$ are computed as 
%
\beq
\Ham\w_{ij}(\RR)=\me{\R' i}{\hat{\Ham}}{\R'+\R,j}=
\me{{\bf 0}i}{\hat \Ham}{\R j}\,,
\eql{H-R}
\eeq
%
and the orthonormality condition reads
\beq
\ip{\RR'i}{\RR j} = \delta_{\RR\RR'} \delta_{ij} \eql{orthonormal}
\eeq
%
The electron energies and wavefunctions
at any arbitrary wavevector $\kk$ are obtained from the secular equation
%
\beq
\Ham\w(\k) U\bnk=\enk U\bnk\,,
\eql{eig-wann}
\eeq
%
to find the eigenenergies $\enk$ and  column vectors $\ket{ n}$
of coefficients $U_{jn}(\kk)$  of the expansion of the wavefunctions
\beq
\ket{u_{j\kk}}=\ket{u\ww_{j'\kk}} U_{j'j}(\kk)
\eeq
in the Bloch basis $\ket{u\ww_{j\kk}}$  constructed from Wannier functions as
\beq
\ket{u\ww_{j\kk}} = \sum_\RR e^{i\kk\cdot(\RR+\bt_j-\rr)}\ket {\RR j}\,.
\eeq
We have included Wannier centers $\bt_i$ in the phase factors
the phase factor, as this is the most convenient convention for
handling Berry-phase quantities~\cite{vanderbilt-book18}.(See \aref{wannier-centers} for details )

The derivative of states $\ket{n}$ of the Wannier Hamiltonian is given by \eq{dn-1} 
the and taking inner product with $\bra{ l}$ yields the anti-Hermitian matrix
%
\beq
D^a_{ln}=\ip{l}{\partial_a n}=
\begin{cases}
\dis
\frac{\lme{l}{\partial_a \Ham\w}{n}}{\en-\el},& \text{if $l\not= n$}\\
-i\alpha_n,& \text{if $l=n$}
\end{cases}\,.
\eql{D-a}
\eeq
which is convenient for the evaluation of the derivative of Bloch states as 
\beq
\ket{\partial_b u_n} =  \ket{\partial_b u_{j'}\ww}U_{j'n} +  \ket{u_{j'}}D_{{j'}n}^b
\eql{der-u}
\eeq
Inserting \eq{der-u} into \eq{curv-n-def} and summing over states in set A one gets the net Berry curvature
\beas{berry-curv-wannier-int}
\Omega^{ab}_\mathrm{A} &=& \Omega^{ab}_\mathrm{A,int}+ \Omega^{ab}_\mathrm{A,ext}~;\\
\Omega^{ab}_\mathrm{A,int}&=& 2\Im\sum^A_n\sum^B_l D^a_{nl} D^b_{ln}~;
\eql{Omega-DD}\\
\Omega^{ab}_\mathrm{A,ext} &=& 
 -2\Im \sum^A_n \overline{F}^{ab}_{nn}
-4 \Re \sum^A_n\sum^B_l  D^a_{nl} \overline{A}^b_{ln}~,
\eql{Omega-DD-AD}
\eeas
which we have separated into ``internal'' and ``external'' terms for a reason that will be explained shortly.
Hereinafter, with an overline we denote a transform of any matrix object in the Wannier gauge ${\cal O}\w_\k$ 
into to the Hamiltonian gauge: 
%
\beq
\overline{\cal O}_{mn} = \me{m}{{\cal O}\w_\k}{n}= \left( U^\dagger \cdot {\cal O} \cdot U\right)_{mn}
\eql{O-mn}
\eeq
and the corresponding Wannier gauge matrices $A\w$ and $F\w$ are defined in \eqs{Aww-2}{Fww-2} based on the real-space matrix elements \eqref{eq:O-R}.

In the derivations above, one could replace the Wannier functions to any other set  of orthonormal
%
\footnote{Generalization to non-orthogonal localized basis was done in \citep{Wang_2019-nonorthogonal,Jin_2021-nonorthogonal}.
Our formalism of covariant derivatives can also be generalized to that case, but we leave it out of the scope of the present article.
}
%
 localized basis states $\ket{\RR j}$. In fact, the derivation would be the same for an empirical tight-binding model, with the only difference 
 that the matrix elements $\Ham_{ij}(\RR)$, $\mathbb{A}^a_{ij}(\RR)$ and $\mathbb{F}^a_{ij}(\RR)$ would not be computed via \eqs{H-R}{O-R}, but rather fitted to bandstructure or chosen empirically. It is a common practice in tight-binding models to neglect matrix elements \eqref{eq:O-R}, and work only with the ``hoppings'' $\Ham(\RR)$.  Also, when working with an effective $\kk\cdot\mathbf{p}$ model, there are no localized functions, 
 but instead the Hamiltonian matrix is assumed to be written in a basis that does not  depend on $\kk$, and therefore the first term in \eq{der-u} vanishes. In these cases the Berry curvature is given only by $\Omega^{ab}_\mathrm{A,int}$ which is identical to \eq{Omega-A-1}. 
The terms that survive in the case of an effective model are called \textit{internal} because they are the property of a Hamiltonian. In turn, the terms that depend on additional matrix elements are called \textit{external}, and they are important for an accurate \textit{ab initio} description of
electronic properties.  In the rest of the present article we will keep this separation.


%%% Local Variables:
%%% mode: latex
%%% TeX-master: "pap"
%%% End:
