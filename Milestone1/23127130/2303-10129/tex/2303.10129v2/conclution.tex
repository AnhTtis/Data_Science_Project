\section{Conclusions}

In this paper, we present a Wannier interpolation scheme of the non-Abelian 
Berry curvature and orbital magnetic moment matrices for a group of bands.
This method involves grouping the bands of interest together in specific energy range,
which not only reduces computational complexity but also avoids convergence 
difficulties caused by intersections of bands within the group.
By computing the trace of interpolated quantities, the sum of the quantities of the band in 
the group are obtained.

When studying higher-order transport phenomena using ``Berry-Boltzmann equation'', 
we can use integration by parts to transfer the derivative from the distribution function 
to other quantities, thus converting the conversion from ``Fermi surface'' integration to 
``Fermi sea'' integration. The advantage of ``Fermi sea'' integration is that 
all electron states below the Fermi level contribute to the integral, 
and there is no need to evaluate the Fermi surface with dense k-sampling, 
which results in fewer k-points needed to obtain accurate results. 
In our \textit{ab initio} simulation of the Berry curvature dipole in Te, 
the ``Fermi sea'' integration demonstrates good stability and convergence 
compared to the ``Fermi surface'' integration.

However, the application of the developed method is not limited to 
improving convergence of a Berry curvature dipole. In the study of 
magnetotransport within Berry-Boltzmann formalism one gets terms where the derivatives of Berry curvature 
and orbital moment cannot be avoided. In particular, that allowed us to evaluate the recently measured \cite{Rikken19,calavalle2022gate}
electrical magnetochiral anisotropy (eMChA) in tellurium. These calculations are described in \cite{Te-eMChA-to-be-published}
%Since Wannier functions have ``gauge freedom, we use ``gauge-covariant'' derivatives 
%when interpolating the derivative matrix to protect the it. 
%The ``gauge invariant'' trace of band group are also preserved.



%%% Local Variables:
%%% mode: latex
%%% TeX-master: "pap"
%%% End:
