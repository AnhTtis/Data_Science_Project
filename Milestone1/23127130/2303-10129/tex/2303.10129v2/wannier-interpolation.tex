\section{Wannier interpolation}

\secl{wannier-interpolation}
%\section{Berry curvature and its gradient in the minimal tight-binding formalism} 
\subsection{Wannier functions and effective models}
\secl{Wannier}
In this section we introduce the necessary notation, and 
briefly recall the spirint of Wannier interpolation of Berry curvature, 
closely following Ref.~\cite{wang-prb06}.

Wannier functions $\ket{\RR j}$ form a localized orthonormal basis for the description of electron bandstructure.\cite{Marzari-MLWF-rmp}
The eigenvalues of the Wannier Hamiltonian 
%
\beq
\Ham\w_{ij}(\kk)=\sum_\R\,e^{i\k\cdot(\R+\bt_j-\bt_i)}\Ham\w_{ij}(\R)\,.
\eql{H-k}
\eeq
%
accurately reproduce the eigenvalues of the Bloch bands computed from first principles. 
Here $\RR$ are lattice vectors, and the matrix elements $\Ham\w_{ij}(\R)$ are computed as 
%
\beq
\Ham\w_{ij}(\RR)=\me{\R' i}{\hat{\Ham}}{\R'+\R,j}=
\me{{\bf 0}i}{\hat \Ham}{\R j}\,,
\eql{H-R}
\eeq
%
and the orthonormality condition reads
\beq
\ip{\RR'i}{\RR j} = \delta_{\RR\RR'} \delta_{ij} \eql{orthonormal}
\eeq
%
The electron energies and wavefunctions
at any arbitrary wavevector $\kk$ are obtained from the secular equation
%
\beq
\Ham\w(\k) U\bnk=\enk U\bnk\,,
\eql{eig-wann}
\eeq
%
to find the eigenenergies $\enk$ and  column vectors $\ket{ n}$
of coefficients $U_{jn}(\kk)$  of the expansion of the wavefunctions
\beq
\ket{u_{j\kk}}=\ket{u\ww_{j'\kk}} U_{j'j}(\kk)
\eeq
in the Bloch basis $\ket{u\ww_{j\kk}}$  constructed from Wannier functions as
\beq
\ket{u\ww_{j\kk}} = \sum_\RR e^{i\kk\cdot(\RR+\bt_j-\rr)}\ket {\RR j}\,.
\eeq
We have included Wannier centers $\bt_i$ in the phase factors
the phase factor, as this is the most convenient convention for
handling Berry-phase quantities~\cite{vanderbilt-book18}.(See \aref{wannier-centers} for details )

The derivative of states $\ket{n}$ of the Wannier Hamiltonian is given by \eq{dn-1} 
the and taking inner product with $\bra{ l}$ yields the anti-Hermitian matrix
%
\beq
D^a_{ln}=\ip{l}{\partial_a n}=
\begin{cases}
\dis
\frac{\lme{l}{\partial_a \Ham\w}{n}}{\en-\el},& \text{if $l\not= n$}\\
-i\alpha_n,& \text{if $l=n$}
\end{cases}\,.
\eql{D-a}
\eeq
which is convenient for the evaluation of the derivative of Bloch states as 
\beq
\ket{\partial_b u_n} =  \ket{\partial_b u_{j'}\ww}U_{j'n} +  \ket{u_{j'}}D_{{j'}n}^b
\eql{der-u}
\eeq
Inserting \eq{der-u} into \eq{curv-n-def} and summing over states in set A one gets the net Berry curvature
\beas{berry-curv-wannier-int}
\Omega^{ab}_\mathrm{A} &=& \Omega^{ab}_\mathrm{A,int}+ \Omega^{ab}_\mathrm{A,ext}~;\\
\Omega^{ab}_\mathrm{A,int}&=& 2\Im\sum^A_n\sum^B_l D^a_{nl} D^b_{ln}~;
\eql{Omega-DD}\\
\Omega^{ab}_\mathrm{A,ext} &=& 
 -2\Im \sum^A_n \overline{F}^{ab}_{nn}
-4 \Re \sum^A_n\sum^B_l  D^a_{nl} \overline{A}^b_{ln}~,
\eql{Omega-DD-AD}
\eeas
which we have separated into ``internal'' and ``external'' terms for a reason that will be explained shortly.
Hereinafter, with an overline we denote a transform of any matrix object in the Wannier gauge ${\cal O}\w_\k$ 
into to the Hamiltonian gauge: 
%
\beq
\overline{\cal O}_{mn} = \me{m}{{\cal O}\w_\k}{n}= \left( U^\dagger \cdot {\cal O} \cdot U\right)_{mn}
\eql{O-mn}
\eeq
and the corresponding Wannier gauge matrices $A\w$ and $F\w$ are defined in \eqs{Aww-2}{Fww-2} based on the real-space matrix elements \eqref{eq:O-R}.

In the derivations above, one could replace the Wannier functions to any other set  of orthonormal
%
\footnote{Generalization to non-orthogonal localized basis was done in \citep{Wang_2019-nonorthogonal,Jin_2021-nonorthogonal}.
Our formalism of covariant derivatives can also be generalized to that case, but we leave it out of the scope of the present article.
}
%
 localized basis states $\ket{\RR j}$. In fact, the derivation would be the same for an empirical tight-binding model, with the only difference 
 that the matrix elements $\Ham_{ij}(\RR)$, $\mathbb{A}^a_{ij}(\RR)$ and $\mathbb{F}^a_{ij}(\RR)$ would not be computed via \eqs{H-R}{O-R}, but rather fitted to bandstructure or chosen empirically. It is a common practice in tight-binding models to neglect matrix elements \eqref{eq:O-R}, and work only with the ``hoppings'' $\Ham(\RR)$.  Also, when working with an effective $\kk\cdot\mathbf{p}$ model, there are no localized functions, 
 but instead the Hamiltonian matrix is assumed to be written in a basis that does not  depend on $\kk$, and therefore the first term in \eq{der-u} vanishes. In these cases the Berry curvature is given only by $\Omega^{ab}_\mathrm{A,int}$ which is identical to \eq{Omega-A-1}. 
The terms that survive in the case of an effective model are called \textit{internal} because they are the property of a Hamiltonian. In turn, the terms that depend on additional matrix elements are called \textit{external}, and they are important for an accurate \textit{ab initio} description of
electronic properties.  In the rest of the present article we will keep this separation.


%%% Local Variables:
%%% mode: latex
%%% TeX-master: "pap"
%%% End:




\subsection{Wannier interpolation of 
$\Fcov^{ab}_{mn}$ and  $\Hcov^{ab}_{mn}$ }
\secl{wannier-interpolation-FH}

\ism{In case we discuss internal vs external terms of the orbital
  moment matrix, we should try to make contact with the $g$-factors
  literature, where a similar distinction is made: see Eq.~(19) in
  Phys. Rev. Research {\bf 2}, 033256 (2020). \tenxl{ Do you mean Eq.~18 ?}}

In this section we derive the Wannier interpolation of quantities given by \eqs{Ftilde}{Htilde}.
% the follwoing covariant quantities:
%which we rewrite as 
%%%
%\bea
%\Fcov^{ab}_{mn}&=\ip{\wt\partial_a u_m}{\wt\partial_b u_n} &=\left(\Fcov^{ba}_{nm}\right)^*\,, \eql{Fab}\\
%\Hcov^{ab}_{mn}&=\me{\wt\partial_a u_m}{\Ham}{\wt\partial_b u_n}&=\left(\Hcov^{ba}_{nm}\right)^*\,, \eql{Hab}
%\eea
%where $\wt\partial_a \equiv Q\partial_a$, and recall that 
%%
%The approach closely follows Ref. \cite{wang-prb06}. 
%And let us consider Bloch states in another gauge
%\beq
%\ket{u_{j\kk}}=\ket{u\ww_{n\kk}} U_{j'j}(\kk)
%\eeq
%These may be the eigenstate of the Hamiltonian (Hamiltonian gauge) or any other 
%gauge, that is related to the hamiltonian gauge by a transformation not mixing 
%the subspaces $A$ and $B$. 
%According to eq. 26 of  \cite{wang-prb06} the derivative with respect to $\kk$ vector may be taken as
%%
%\beq
%\ket{\partial_b u_n} =  \ket{\partial_b u_{j'}\ww}U_{j'n} +  \ket{u_{j'}}D_{{j'}n}^b
%\eeq
Recalling that  $Q=1-\ket{u_{n'}}\bra{u_{n'}}$  from \eq{der-u} we obtain
\begin{multline}
Q\ket{ \partial_b u_n} = Q \ket{\partial_b u_{j'}\ww}U_{{j'}n} +  \ket{u_{l'}}D_{l'n}^b =  \\
\ket{\partial_b u_{j'}\ww} U_{j'n}
- \ket{u_{n'}} \ip{u_{n'}}{\partial_b u_{j'}\ww}U_{j'n}
%
+   \ket{u_{l'}}D_{l'n}^b=
\\
\ket{\partial_b u_{j'}\ww}U_{j'n} + i\sum_{n'}\inn \ket{u_{n'}} \overline{A}^b_{n'n}+ \ket{u_{l'}}D_{l'n}^b
\end{multline}
%
and using and $Q^2=Q$ we obtain the following expressions:
%
\beas{Fab-wanint-all}
\Fcov^{ab}_{mn}              &=& \Fcov^{ab}_{mn,\textrm{int}} + \Fcov^{ab}_{mn,\textrm{ext}} \eql{Fab-wanint}\\
\Fcov^{ab}_{mn,\textrm{int}} &=& - D_{nl'}^a D_{l'm}^b \eql{Fab-wanint-int}\\
\Fcov^{ab}_{mn,\textrm{ext}} &=&  
\left[\left(iD^a_{nl'}\Abar^b_{l'm}\right)+\conjabmn   \right]
\nonumber \\ &&
+\Fbar^{ab}_{mn}-  \Abar^a_{mn'} \Abar^b_{n'n} 
\eql{Fab-wanint-ext}
\eeas
and
\beas{Hab-wanint-all}
\Hcov^{ab}_{mn}&=& \Hcov^{ab}_{mn,\textrm{int}} + \Hcov^{ab}_{mn,\textrm{ext}} \eql{Hab-wanint}\\
\Hcov^{ab}_{mn,\textrm{int}}&=&- D_{nl'}^a \Ham_{l'l''} D_{l''m}^b  \eql{Hab-wanint-int}\\
\Hcov^{ab}_{mn,\textrm{ext}}&=& 
 \left[\left(i D^a_{nl'}\overline{B}^b_{l'm}\right)+\conjabmn \right] +
\nonumber\\&&
+\Hbar^{ab}_{mn} -  \Abar^a_{mm'} \Ham_{m'n'} \Abar^b_{n'n} 
\eql{Cab-wanint}
\eeas
Similar to \eq{berry-curv-wannier-int} we have separated the result into internal and external terms. 
Here the bar above a matrix follows \eq{O-mn}.
%\beq
%\overline{\cal O}_{ij}=\left[ U^\dagger{\cal O}\ww U\right]_{ij}  \,,
%\eql{O-bar}
%\eeq
To describe the external terms, And following \cite{wang-prb06} and \cite{lopez-prb12} 
we have defined a series of quantities defined in the Wannier gauge:\footnote{Note that in \cite{lopez-prb12} the quantity of our \eq{Hww} was denoted as $C\ww$}
\beas{Oww-all}
\left(A\ww\right)^{a}_{ij}&\equiv&  i\ip{ u_i\ww}{\partial_a u_j\ww} \eql{Aww} \\% = \nonumber\\ 
%   &=& \sum_\mathbf{R} e^{i\kk(\R+\bt_j-\bt_i)}\memnR{\hat{r}_a-\RR-\bt_j } =
%      e^{i\kk(\bt_j-\bt_i)}\sum_\mathbf{R} e^{i\kk\R}  \mathbb{A}^a_{ij}(\RR)  \\
\left(B\ww\right)^{a}_{ij}&\equiv&  i\me{ u_i\ww}{\Ham}{\partial_a u_j\ww} \eql{Bww}\\
%  =  \sum_\mathbf{R} e^{i\kk\mathbf{R}}\mathbb{B}^a_{ij}(\RR)   \\
\Hwan^{ab}_{ij}&\equiv&  \me{\partial_a u_i\ww}{\Ham}{\partial_b u_j\ww} \eql{Hww} \\
% = \sum_\mathbf{R} e^{i\kk\mathbf{R}}\mathbb{C}^{ab}_{ij}(\RR) \\
\Fwan^{ab}_{ij}&\equiv&  \ip{\partial_a u_i\ww}{\partial_b u_j\ww}  \eql{Fww}
 % = \sum_\mathbf{R} e^{i\kk\mathbf{R}} \mathbb{F}^{ab}_{ij}(\RR)\\
\eeas
Evaluation of these quantities is given in \aref{wannier-gauge}
%

Inserting \Eqs{Fab-wanint-all}{Hab-wanint-all} into corresponding equations from \sref{geometric-quantity} one can obtain Wannier interpolation of the needed geometrical quantity. For instance, combining \eqref{eq:Fab-wanint-all}, \eqref{eq:curv-cov} and \eqref{eq:trace-cov} one obtains the known result for Berry curvature \eqref{eq:berry-curv-wannier-int}.


\subsection{Covariant  derivatives of %  
$\Fcov^{ab}_{mn}$ and $\Hcov^{ab}_{mn}$}
\secl{der-F-H}


Now we are ready to take the covariant derivatives of \eqs{Fab-wanint}{Cab-wanint}, using the product rule defined in \eq{gender-prod-rule}.

\beas{Fab-wanint-der}
\Fcov^{ab:d}_{mn}&=& \Fcov^{ab:d}_{mn,\textrm{int}} + \Fcov^{ab:d}_{mn,\textrm{ext}} \eql{Fab-wanint-der-all}\\
\Fcov^{ab:d}_{mn,\textrm{int}}&=& D_{ml'}^{a} D_{l'n}^{b:d} +\conjabmn \eql{Fab-wanint-der-int}\\
\Fcov^{ab:d}_{mn,\textrm{ext}}&=& - \Bigl[\Bigl( i\Abar^a_{ml'}D^{b:d}_{l'n} + iD^a_{ml'}\Abar^{b:d}_{l'n}\nonumber \\
 & &  +\Abar_{mn'}^a\Abar_{n'n}^{b:d} \Bigr) +\conjabmn \Bigr]+ \Fbar^{ab:d}_{mn}   \eql{Fab-wanint-der-ext}
\eeas
%
\beas{Hab-wanint-der}
\Hcov^{ab:d}_{mn}&=& \Hcov^{ab:d}_{mn,\textrm{int}} + \Hcov^{ab:d}_{mn,\textrm{ext}} \eql{Hab-wanint-der-all}\\
\Hcov^{ab:d}_{mn,\textrm{int}}&=&
			        - \Bigl[ D_{ml'}^{a} {\Ham_{l'l''}} D_{l''n}^{b:d}  +\conjabmn \Bigr] 
			\nonumber\\ &&
			-  D_{ml'}^{a} \Ham_{l'l''}^{,d} D_{l''n}^b  			        
			        \eql{Hab-wanint-der-int}\\
\Hcov^{ab:d}_{mn,\textrm{ext}}&=& \Bigl[
                 i(\overline{B}^\dagger)^a_{ml'}D^{b:d}_{l'n} + iD^a_{ml'}\overline{B}^{b:d}_{l'n}  \nonumber\\&&
        -  \Abar_{mm'}^a \Ham_{m'n'} \Abar_{n'n}^{b:d}  
         +\conjabmn  \Bigr]  -  
                      \nonumber\\ &&
                 -\Abar_{mm'}^a \Ham_{m'n'}^{,d} \Abar_{n'n}^{b}  + \Hbar^{ab:d}_{mn} 
\eql{Hab-wanint-der-ext}
\eeas
%
Now we need to find the generalized derivatives of the ingredients of this equation. 

The derivatives $\Abar^{b:d}_{ln}$, $\overline{B}^{b:d}_{ln}$, 
$\overline{C}^{b:d}_{ln}$ and $\overline{F}^{b:d}_{ln}$ are evaluated in a general way.  
Consider a quantity of the form of \eq{O-mn}, \eq{gender-Xnn} for
instance reads
%
\beq
\overline{\cal O}^{:d}_{nn'}=\overline{\cal O}^{,d}_{nn'}
-\sum_l\out D^d_{nl}{\cal O}_{ln'}+\sum_l\out {\cal O}_{nl}D^d_{ln'}\eql{gender-Onn}
\eeq
%
where in accordance with \eq{operator-coma-derivative}  we defined
%
\beq
\overline{\cal O}^{,d} \equiv\overline{\partial_d {\cal O}}=
U^\dagger\left(\partial_d {\cal O}\ww\right) U\,. \eql{commader}
\eeq
Some technical notes on the evaluation on the evaluation of $\overline{B}^{b:d}_{ln}$ are left for \aref{B-frozen}.
In order to derive $D_{ln}^{b:d}$  let's rewrite \eq{D-a} for $D^b_{ln}$ as 
\beq
  D_{ln'}^b \Ham_{n'n} -  \Ham_{ll'}D_{l'n}^b = \Ham_{ln}^{,b}~.
 \label{eq:DHmHD}
\eeq
 % 
Now we take the covariant derivative of
both sides of this equation and employing the product rule
\eq{gender-prod-rule} we get
%
\begin{multline}
  D_{ln'}^b \Ham_{n'n}^{:d}  + D_{ln'}^{b:d}  \Ham_{n'n} 
- \Ham_{ll'}D_{l'n}^{b:d}  -  \Ham_{ll'}^{:d}D_{l'n}^b =\\
 \Ham_{ln}^{,bd} - D_{ln'}^{d} \Ham_{n'n}^{,b} +  \Ham_{ll'}^{,b} D_{l'n}^d 
\end{multline}
%
where we also used \eq{gender-Xln} for $\Ham_{ln}^{,b:d}$ and
\eq{H-gender}

Now, consider this equation in the Hamiltonian gauge,
collect all terms except $D_{nl'}^{b:d}$ in the RHS and
divide by $(\en-\el)\equiv \varepsilon_{nl}$ to get the following  expression 
\beq
D^{b:d}_{ln}=\frac{1}{\varepsilon_{nl}}
\left[
\Ham^{,bd}_{ln}+
\left(\Ham^{,b}_{ll'} D^d_{l'n} - D^d_{ln'}\Ham^{,b}_{n'n}  + (b \leftrightarrow d)\right)
\right]
\eql{gender-sumrule}
\eeq

It can be seen that the derived equations do not contain the $D$ matrix states A, as well as between states B.  As far as spaces A and B are 
By construction, this behaviour will be preserved in evaluation of higher derivatives (see \aref{second-derivative}).

Inserting \Eqs{Fab-wanint-der-all}{Hab-wanint-der-all} into corresponding equations from \sref{geometric-quantity} one can obtain Wannier interpolation of thederivatives of the needed geometrical quantities. For instance, combining \eqref{eq:Fab-wanint-der-int}, \eqref{eq:der-trace}, \eqref{eq:curv-cov} and \eqref{eq:gender-sumrule} one obtains the known result for derivative of the Berry curvature of occupied states \eqref{eq:Omega-A-grad-1}, that we gave in \sref{preliminary} without proof.





%%% Local Variables:
%%% mode: latex
%%% TeX-master: "pap"
%%% End:
