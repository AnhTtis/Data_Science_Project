\section{First principles results: trigonal Tellurium}
\secl{result-Te}

%%% Local Variables:
%%% mode: latex
%%% TeX-master: "pap"
%%% End:TR
Trigonal Tellurium has a crystal structure composed of homotropic 
triple helix Te-chains with space group $P3_121$ (right handed) or 
$P3_221$ (left handed), the right handed structure of Te is shown in \fref{Te-Cell}. 
The three Te atoms in each unit cell are evenly distributed along the helix. 
The screw structure breaks inversion symmetry and creates a Berry curvature dipole.
\begin{figure}[t]
\begin{center}
\includegraphics[width=0.9\columnwidth]{Te_cell_long.png}
    \caption{Crystal structure of tellurium with space group $P3_121$.(a) The tellurium chains spiral along z-axes of the crystal lattice. (b) Top view of Te spiral chains.}
    \figl{Te-Cell}
\end{center}
\end{figure}

The electronic structure is calculated by employing the HSE06 hybrid functional \cite{paier2006screened} implemented in the VASP code \cite{Kresse:1996,kresse1996efficiency,Kresse:1999} .
The maximaly localized Wannier functions are generated using Wannier90 \cite{Pizzi:2020}, and disentanglement is performed using a frozen window below $ \varepsilon_{\rm F} + 3{\rm eV} $ with $s, p$ orbitals of Te as projections. 
Tellurium has a band gap of about 0.3eV with conduction band (CB) minimum and valence band (VB) maximum around the H and H$^\prime$ points of the brillouin zone (BZ).
The material is usually p-doped, and the VB presents more interest. However, for our demonstration the CB is more interesting because there is a Weyl point (WP) at the H point, which presents a computational challenge for the evaluation of 
Berry dipole via \eq{dp-surf}. Moreover, this WP is predicted to give a significant contribution to switch the sign of circular photogalvanic effect at high temperatures and low doping \cite{tsirkin-gyrotropic}.
The methodology derived in this manuscript has been implemented within the open-source code WanierBerri\cite{wannierberri}. 
% There are Weyl points located at the $H$ or $H'$ points with energy 0.312eV above the Fermi level. 
%All Fermi surfaces at the energy of the Weyl point are contributed by Rashba-splitting-type band structures around the Weyl points, as shown in \fref{single-band-compare-Te}(c). 
\begin{figure}
\begin{center}
\includegraphics[width=1.0\columnwidth]{single-band-compare-Te.png}
\caption{(c)  Energy band of Tellurium, the k-path of 
    which is a small range around H point of K-H-K line. 
    There is a Weyl point (WP) at the H point located at 0.312eV 
    above Fermi energy. (b)(d) Berry curvature z-component 
    of the conduction band (CB) 1 and 2. (a)(e) Derivative 
    of Berry curvature z-component of CB1 and CB2. Solid black 
    lines are calculated with \eqs{Fab-wanint-der-int}{Fab-wanint-der-ext}. 
    And the dashed color lines are calculated by the finite difference of Berry curvature.}
\figl{single-band-compare-Te}
\end{center}
\end{figure}
By using Wannier interpolation, 
the interpolated Berry curvature $\wt \Omega$ of conduction bands CB1 and CB2  can be calculated with
\beq
\wt \Omega^c_n = -\epsilon_{abc} {\rm Im} \wt F^{ab}_{nn}~,
\eeq
where $\wt F^{ab}_{nn}$ is introduced in \eq{Fab-wanint}.
According to \eq{Omega-kn}, the Berry curvature of a single band follows an inverse-square law with respect to the energy difference with other bands,
and therefore it changes rapidly in the vicinity of the WP, as shown in \fref{single-band-compare-Te}(b,d). 
Following \eq{Fab-wanint-der-all}, the interpolated derivatives of Berry curvature are calculated with
\beq
\wt \Omega^{c:d}_n = -\epsilon_{abc} {\rm Im} \wt F^{ab:d}_{nn},
\eeq
which are shown in \fref{single-band-compare-Te}(a,e) as solid lines. 
The dashed colored lines are plotted using finite differences based on the interpolated Berry curvature data in \fref{single-band-compare-Te}(b,d), 
using denser k sampling. The solid curves in \fref{single-band-compare-Te}(b,d) show good agreement with dashed colored lines, 
indicating that our Wannier interpolation method with covariant gradient works well around WPs in real materials.
\begin{figure}[t]
\begin{center}
\includegraphics[width=0.9\columnwidth]{plane-Te-dp.png}
    \caption{(a)(c) Berry curvature z-component $\Omega^z_{\rm A}$ and 
    derivative of Berry curvature zz-component  $\partial_z\Omega^z_{\rm A}$ 
    of all occupied bands below Weyl point (WP) energy on a plane submanifold 
    of Brillouin zone. (b)(d) The zoom-in of the dashed square range in (a)(c). }
\figl{plane-Te-dp}
\end{center}
\end{figure}

%Due to time-reversal symmetry, the berry curvature and its gradient obey the following relation
%\beas{Omega-k}
%\wt \Omega^z_{\k,{\rm n}} &= -\wt \Omega^z_{-\k,{\rm n}} \eql{Omega+-k}\\
%\partial_z\wt\Omega^z_{\k,{\rm n}} &= \partial_z\wt\Omega^z_{-\k,{\rm n}} \eql{dOmega+-k}
%\eeas
%shown in \fref{plane-Te-dp}(a). 
In the previous research, 
evaluating  $ \Ham_{\k n}^{:a} \wt \Omega_{\k n}^b$ on the Fermi surface is 
a widely used method to calculate dimensionless Berry curvature 
dipole tensor via \eq{dp-surf}. 
Typically, one evaluates it as a direct summation 
\beq
\mathcal{D}_{ab}^{\rm surf} = \frac{1}{N_\kk V_{\rm u.c.}}\sum_{\kk n}  \partial_a \enk \Omega_{\k n}^b \left(- \frac{\partial f_0}{\partial \varepsilon}\right)_{\varepsilon = \enk}~.
\eql{dp-surf}
\eeq
However, at low temperature, the derivative of the distribution function with energy $f_0'$ 
is a narrow peak function. This means only the k points that lie closely on the Fermi surface contribute to the integral.
However, a big number of points needs to be evaluated before the narrow peaks from individual $\kk$-points merge into a smooth curve. 
In \cite{Singh-MoTe}  $\mathcal{D}_{ab}$ was by first computing the Fermi surface by employing the
tetrahedron method at a given k grid, and then the Berry curvature was sampled only at the reduced grid points near the Fermi
surface. However, they also noted a slow convergence of the integral. 

Instead, in \eqref{eq:dp-sea} all k-points (which have bands below Fermi level) contribute, and therefore we are integrating a smoother function. 
%By using integration by parts, the derivative can be moved from the distribution function $f_0$ to Berry curvature $\Omega_{\k n}$ \eq{dp-sea}
%\beq
%\mathcal{D}_{ab}^{\rm sea} = \int \dk \sum_n \wt \Omega_{\k n}^{b:a} f_0(\enk)~,
%\eql{dp-sea}
%\eeq
%And the evaluation method involves a Fermi sea integral, which requires summing up the
%\tenxl{$\Omega^{b:a}_{\k n}$} values of all occupied bands over the entire first Brillouin zone. 
%All points on the k-grid are used in the calculation of these quantities. 
 %, which relies on only a few k-points.
%By using the Wannier interpolation method with covariant gradient, we can interpolate the matrix $\wt \Omega_{\k n}^{b:a}$ 
%on the k-grid to produce smooth results, in contrast to the jagged data obtained from finite differences.
%And this interpolation method aids in achieving convergence.

Another reason for the slow convergence of \eq{dp-surf}
is the sharp peak of the Berry curvature near a WP.  Close to theWP the divergent part of the BC is equal in magnitude and opposite in sign for the two subbands. Therefore, if the Fermi level is above (below) both subbands, in the Fermi-sea integral the divergent parts cancel out (do not appear). In turn, in the Fermi-surface integral both subbands contribute with their divergent Berry curvature multiplied by different velocities and distribution functions, and therefore no cancellation occurs. 


\begin{figure}[t]
\begin{center}
\includegraphics[width=0.95\columnwidth]{fsur-sea-dipole.png}
\caption{(a) Integral of Berry curvature dipole using Fermi surface 
    \eq{dp-surf} and Fermi sea \eq{dp-sea} integral function. 
    The colors show different k-points sampling numbers (unit M is million) 
    in all Brillouin zone. WP is the energy where the Weyl point 
    is located. (b) The standard deviation compared with the converged result. }
\figl{fsur-sea-dipole}
\end{center}
\end{figure}

Thus, the Fermi-sea integral should have a better convergence with respect to the density of the $\kk$-grid.
To demonstrate this, we calculated the Berry dipole using both \eq{dp-surf} and \eq{dp-sea} at temperature 50K, as shown in \fref{fsur-sea-dipole}(a),
with varying numbers of k-points.
When using \eq{dp-surf} and a smaller number of k-points, 
there is a clear divergence in the Fermi-surface curve at the Weyl point energy.
And the curve does not converge until 47 million k-points are used.
In contrast, the Fermi-sea curve exhibits good convergence and is almost converged with only 0.7 million k-points,
without any divergence at the Weyl point energy.
The standard deviation of the results also supports these findings, as shown in \fref{fsur-sea-dipole}(b).

 

%Any slight loss of symmetry in the Wannier function will exacerbate the impact of the Weyl point. 
%Although the Wannier function generally agrees well with results from density functional theory, 
%some non-negligible changes may arise for a single state at a specific k point
%such as a gapped degenerate point, unusual Berry curvature, or a non-symmetric Fermi surface.
%In \eq{dp-surf}, when using a lower k-grid, any changes to a k-point on the Fermi surface due to symmetry breaking cannot be ignored. 
%However, in \eq{dp-sea}, the sum over all occupied bands in the Brillouin zone enables a high tolerance 
%for any small symmetry losses.
%This makes \eq{dp-sea} vastly more robust than \eq{dp-surf}.

\stm{For future, we may add discussion and figures on how (badly) the fermi-surface Berry dipole converges to tracelesness.
In turn, how the value of fermi-sea integrals converges to zero when the fermi level is in the gap. }

