
\section{Catalogue of geometric quantities}
\secl{geometric-quantity}

\tents{Before proceeding with derivation of explicit equations for Wannier interpolation of multiple geometrical quantities of interest,
we first define them in a gauge-covariant way and show that they all may be reduced to two objects $\Fcov^{ab}_{mn}$ and  $\Hcov^{ab}_{mn}$  defined below. }
%---------------------------------------------------
\subsection{Gauge-covariant matrices}
%---------------------------------------------------
\stm{the "covariant" definition here looks disconnected from previous 
section from first glance. Add some words to connect "P's" and "Q's" to
the definition of \eq{gender}}
Let $P=\sum_n^{\rm A}\,\ket{u_n}\bra{u_n}$ and $Q=1-P$.
% In the following, indices $m,n,n'$ run over the active space A.
The states $\ket{u_n}$ within the active space A are not assumed to be
energy eigenstates,\footnote{However, the A states $\{\ket{u_n}\}$ are
  assumed to be unitarily related to energy eigenstates.} so that in
general the Hamiltonian matrix
%
\beq
\Ham_{mn}\equiv\me{u_m}{\hat \Ham}{u_n}
\eeq
%
is not diagonal.
%
Following Eqs.~(6-8) of Ref.~\cite{lopez-prb12} We write the three gage-covariant quantities
%
\begin{align}
\wt F^{ab}_{mn}&\equiv\me{u_m}{(\partial_a P) Q (\partial_b P)}{u_n}=
\me{\partial_a u_m}{Q}{\partial_b u_n}\,,
\eql{Ftilde}\\
\wt G^{ab}_{mn}&\equiv
\frac{1}{2}\me{u_m}{\Ham (\partial_a P) Q (\partial_b P)+(\partial_a P) Q (\partial_b P)\Ham}{u_n}=\nonumber\\
&=\frac{1}{2}\left(\Ham_{mn'}\wt F^{ab}_{n'm}+
\wt F^{ab}_{mn'}\Ham_{n'n}\right)\,,
\eql{Gtilde}\\
\wt H^{ab}_{mn}&\equiv\me{u_m}{(\partial_a P)Q\Ham Q(\partial_b P)}{u_n}=
\me{\partial_a u_m}{Q\Ham Q}{\partial_b u_n}\,,
\eql{Htilde}
\end{align}
%
and also
%
\beq
\wt S^{ab}_{mn}\equiv\frac{\hbar^2}{m_e}\delta_{ab}\delta_{mn}+
2\left(\wt G^{ab}_{mn}-\wt H^{ab}_{mn}\right)\,.
\eql{Stilde}
\eeq
%
The four matrices $\wt O^{ab}_{mn}$ with $O=F,G,H,S$ are Hermitian
in the sence that $\wt O^{ab}_{mn} = \left(\wt O^{ba}_{nm}\right)^*$, 
and they (as well as $\Ham_{mn}$)
transform covariantly under unitary gauge transformations of the form
$\ket{u_n}\rightarrow\ket{u'_n}=\sum_m^{\rm A}\ket{u_m}U_{mn}$, that
is, $\wt O^{ab}\rightarrow U^\dagger \wt O^{ab} U$.

$\wt F^{ab}_{mn}$ is the metric-curvature tensor, from which the
covariant quantum metric and Berry curvature tensors
% $\wt{\cal F}^c_{mn}$ and
% $\wt f^{ab}_{mn}$
can be obtained as
\xlm{Change symbol of quantum matrix and inverse effective mass tensor from $f$ and $s$ to $\mathfrak F$ and $\mathfrak S$. $f$ is distribution function above.}
\beq
\wt{\mathfrak F}^{ab}_{mn}\equiv\frac{1}{2}\wt F^{ab}_{mn}+\frac{1}{2}\wt F^{ba}_{mn}
\eql{metric-cov}
\eeq
%
and
%
\beq
\wt{\Omega}^c_{mn}\equiv i\abc\wt F^{ab}_{mn}\,,
\eql{curv-cov}
\eeq
%
respectively. In the same way, we can extract from $\wt S^{ab}_{mn}$
two new covariant tensors
%
\beq
\wt {\mathfrak S}^{ab}_{mn}\equiv\frac{1}{2}\wt S^{ab}_{mn}+\frac{1}{2}\wt S^{ba}_{mn}
\eql{mass-cov}
\eeq
%
and
%
\beq
\wt{\cal S}^c_{mn}\equiv i\abc\wt S^{ab}_{mn}\,.
\eql{moment-cov}
\eeq
%
$\wt {\mathfrak S}^{ab}_{mn}$ is a generalized inverse effective mass tensor in a
sense to be clarified shortly, and $\wt{\cal S}^c_{mn}$ is a
generalized orbital magnetic moment tensor in the following sense: if
the A states are all degenerate with energy $\varepsilon_{\rm A}$, so
that $\Ham_{mn}=\varepsilon_{\rm A}\delta_{mn}$ in any gauge, the matrices
$\wt m^c_{mn}=(e/4\hbar)\wt{\cal S}^c_{mn}$ and
$\wt L^c_{mn}=-(m_e/2\hbar)\wt{\cal S}^c_{mn}$ reduce to the orbital
moment and orbital angular momentum matrices as defined in Eq.~(51) of
Ref.~\cite{chang-jpcm08},
%
\begin{multline}
\frac{2\hbar}{e}\wt m^c_{mn}=-\frac{\hbar}{m_e}\wt L^c_{mn}=
i\abc\left(\varepsilon_{\rm A}\wt F^{ab}_{mn}-\wt H^{ab}_{mn}\right)= \\=
i\abc\me{\partial_a u_m}{Q(\varepsilon_{\rm A}-\Ham)Q}{\partial_b u_n}\,.
\end{multline}
%

Finally, we also introduce a $\k$-resolved orbital magnetization
matrix
%
\begin{align}
\frac{2\hbar}{e}\wt M^c_{mn}\equiv -i\abc
\left(\wt H^{ab}_{mn}+\wt G^{ab}_{mn}-2\ef\wt F^{ab}_{mn}\right)=\\=
\frac{2\hbar}{e}\wt m^c_{mn}+
2i\abc\left(\ef\wt F^{ab}_{mn}-\wt G^{ab}_{mn}\right)\,.
\eql{Morb}
\end{align}

\subsection{Gauge-invariant traces}

% \subsection{General case}

Given a covariant matrix $\wt O_{mn}$ with $m,n\in{\rm A}$, we write
its (gauge-invariant) trace as
%
\beq
O_{\rm A}\equiv\Tr_{\rm A}\,\wt O=\sum_n^{\rm A}\,\wt O_{nn}\,.
\eql{trace-cov}
\eeq
%
Thus ${\mathfrak F}^{ab}_{\rm A}={\mathfrak F}^{ba}_{\rm A}$ and ${\Omega}^c_{\rm A}$ are the
net quantum metric and Berry curvature of the A states respectively,
$M^c_{\rm A}$ is their net $\k$-resolved orbital magnetization, and
${\mathfrak S}^{ab}_{\rm A}={\mathfrak S}^{ba}_{\rm A}$ is the sum of the inverse effective
masses (times $\hbar^2$) of the A band states,
%
\beq
{\mathfrak S}^{ab}_{\rm A}:=\sum_n^{\rm A}\,\partial^2_{ab}\en\,.
\eeq
%
Here the symbol $:=$ denotes an equality whose right-hand-side
only holds in a gauge where the Hamiltonian matrix is diagonal:
$\Ham_{mn}=\varepsilon_m\delta_{mn}$.\footnote{Note that if $X:=Y$ and
  $Z:=Y$, then $X=Z$.}

% \subsection{Single-band limit and band additivity}

If space A contains a single band $n$, we simplify the notation as
$\wt O^{ab}_{mn}\rightarrow O^{ab}_n$. In that limit
\eqr{Ftilde}{Htilde} reduce to
%
\begin{align}
F^{ab}_n&=\ip{\partial_a u_n}{\partial_b u _n}-\ip{\partial_a u _n}{u_n}\ip{u_n}{\partial_b u _n}\,,\\
G^{ab}_n&=\en F^{ab}_n\,,\\
H^{ab}_n&=\me{\partial_a u _n}{\Ham}{\partial_b u _n}-
\en\ip{\partial_a u _n}{u_n}\ip{u_n}{\partial_b u _n}\,,
\end{align}
%
and \eq{Stilde} becomes
%
\beq
S^{ab}_n=\frac{\hbar^2}{m_e}\delta_{ab}+
2\me{\partial_a u _n}{(\en-\Ham)}{\partial_b u _n}\,,
\eeq
%
which agrees with the expression given in Ref.~\cite{gao-prb15} (see
the 2nd column of p.~3 therein).

In the same limit \eqs{metric-cov}{curv-cov} reduce to the single-band
quantum metric and Berry curvature, respectively,
%
\begin{align}
{\mathfrak F}^{ab}_n&=\Re\,F^{ab}_n=
\Re\ip{\partial_a u _n}{\partial_b u _n}-\ip{\partial_a u _n}{u_n}\ip{\partial_b u _n}{u_n}
\,,\\
{\Omega}^c_n&=-\abc\Im\,F^{ab}_n=
% -\abc\Im\,\ip{\partial_a u _n}{\partial_b u _n}\,,
-\Im\,\bra{\nabla_\k u_n}\times\ket{\nabla_\k u_n}_c\,,
\end{align}
%
\eqs{mass-cov}{moment-cov} become proportional to the single-band
inverse effective mass and orbital moment, respectively,
%
\begin{multline}
{\mathfrak S}^{ab}_n=\Re\,S^{ab}_n=\\=
\frac{\hbar^2}{m_e}\delta_{ab}+2\Re\me{\partial_a u _n}{(\en-\Ham)}{\partial_b u _n}
=\partial^2_{ab}\en\,,
\end{multline}
%
\begin{multline}
{\cal S}^c_n =-\abc\Im\,S^{ab}_n=\\=
%-2\abc\Im\me{\partial_a u _n}{(\epsilon_n-\Ham)}{\partial_b u _n}=
-2\Im\bra{\nabla_\k u_n}\times (\en-\Ham)\ket{\nabla_\k u_n}_c=
\frac{4\hbar}{e}m^c_n\,,
\end{multline}
%
and \eq{Morb} becomes the single-band $\k$-resolved orbital
magnetization,
%
\beq
\frac{2\hbar}{e}M^c_n=\Im\bra{\nabla_\k}\times
\left(\Ham+\en-2\varepsilon_{\rm F}\right)\ket{\nabla_\k u_n}_c\,,
\eeq
%
or equivalently,
%
\beq
M^c_n=m^c_n+\frac{e}{\hbar}(\varepsilon_{\rm F}-\en){\Omega}^c_n\,.
\eeq

Suppose that space A is entirely made up of bands that never touch one
another. Then its net Berry curvature, inverse effective mass, and
orbital magnetization are  equal to the sums over bands of the
corresponding single-band quantities,
%
%\begin{align}
%{\Omega}^c_{\rm A}&=\sum_n^{\rm A}\,\Omega^c_n\,,\\
%{\mathfrak S}^{ab}_{\rm A}&=\sum_n^{\rm A}\,{\mathfrak S}^{ab}_n\,,\\
%M^c_{\rm A}&=\sum_n^{\rm A}\,M^c_n\,,
%\end{align}
\beq
{\Omega}^c_{\rm A}=\sum_n^{\rm A}\,\Omega^c_n\,,\quad
{\mathfrak S}^{ab}_{\rm A}=\sum_n^{\rm A}\,{\mathfrak S}^{ab}_n\,,\quad
M^c_{\rm A} =\sum_n^{\rm A}\,M^c_n\,,
\eeq
%
and such quantities are said to be ``band additive.'' Note that the
quantum metric and the orbital moment are {\it not} band additive,
since in general
%
\beq
{\mathfrak F}^{ab}_{\rm A}\not=\sum_n^{\rm A}\,{\mathfrak F}^{ab}_n\,,\quad
{\cal S}^c_{\rm A}\not=\sum_n^{\rm A}\,{\cal S}^c_n\,.
\eeq


%%% Local Variables:
%%% mode: latex
%%% TeX-master: "pap"
%%% End:
