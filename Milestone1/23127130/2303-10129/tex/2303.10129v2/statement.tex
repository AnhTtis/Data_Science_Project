
%=================================
\section{Statement of the problem}
\secl{preliminary}
%=================================
%
% \ism{Previous title: ``Preliminary discussion''. Added two
%   subsections. OK?}


\begin{figure}
\begin{center}
% \includegraphics[width=0.5\columnwidth]{A-B.pdf}
%\includegraphics[width=0.95\columnwidth]{A-B-sketch.pdf}
\includegraphics[width=0.95\columnwidth]{A-B2.png}
\caption{Partition of the energy levels of an effective Hamiltonian
  $\Ham_\k$ into an ``active'' group A containing the levels of
  interest, and its complement B. The two groups are separated in
  energy, but degeneracies may be present within each group.}
\figl{A-B}
\end{center}
\end{figure}

Before detailed derivations, let us present our scheme in the
simplest possible setting.  We consider a system described by a
% $\k$-dependent
an effective Hamiltonian $\Ham\eff_\k$ (e.g., a tight-binding or
$\k\cdot\mathbf{p}$ Hamiltonian) with a finite number of
eigenvectors $\ket{m\k}$ and eigenvalues $\emk$ at each $\k$,
%
\ism{Later on, when specializing to TB models and Wannier
  interpolation, the effective Hamiltonian $\Ham\eff_\k$ will be
  denoted as $H\w_\k$. Here I want to distinguish it from the {\it ab
    initio} Bloch Hamiltonian $\Ham_\k$. or maybe we should denote the
  latter as $\Ham_\k^{\rm KS}$ (Kohn-Sham) and its eigenvalues
  $\enk^{\rm KS}$, for extra clarity.}
\stm{I do not see a need to distinguish between the "effective" and "Kohn Sham" Hamiltonians. I suggest to remove "eff" supoerscript here also}

%
\beq
\Ham\eff_\k\ket{m\k} = \emk\ket{m\k}\,.
\eeq
%
In practice, one is typically interested in groups of eigenstates, not
in individual states. Let us denote the ``active'' group of
interest by A and its complement by B, as depicted in \fref{A-B}. It
is assumed that the A and B groups are well separated in energy, but
degeneracies may be present within each of them. The completeness relation is 
\beq
\one=\sum_m^{\rm A}\,\ket{m}\bra{m}+\sum_{l}^{\rm B}\,\ket{l}\bra{l} \equiv\sum_j^{\rm all}\,\ket{j}\bra{j} \,.
\eql{completeness}
\eeq
In this paper we will consistently use indices $i,j,j',\ldots$ to 
denote the whole set of Wannier states, while $m,n,n',n'',\ldots$ will denote states 
of subspace A, and $p,l,l',l'',\ldots$ for states in subspace B.
We will assume that the \textit{repeated primed}
indices are summed (with the values running over the corresponding subspace), while the non-primed
indices are not summed unless written explicitly (for the Trace quantities).
Thus, from \sref{geometric-quantity}, we shorten the equation by omiting the $\sum$ symbols.

% (we will refer to them as ``composite groups'').
% For \eqr{sigma-10}{sigma-30}, groups A and B at a given $\k$ comprise
% all states below and above the Fermi level, respectively.

%%% Local Variables:
%%% mode: latex
%%% TeX-master: "pap"
%%% End:
