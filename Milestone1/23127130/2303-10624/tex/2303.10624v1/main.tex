\documentclass[conference]{IEEEtran}
% \IEEEoverridecommandlockouts
% The preceding line is only needed to identify funding in the first footnote. If that is unneeded, please comment it out.
\usepackage{cite}
%\usepackage{algorithmic}
\usepackage{graphicx}
\usepackage{textcomp}
\usepackage{xcolor}
\usepackage{url,afterpage,amssymb,amsfonts}
\usepackage{enumitem,makecell}
\usepackage{booktabs}
\usepackage{multirow}
\usepackage{subfig}
\usepackage{graphicx}
\usepackage{amsmath}
\usepackage{verbatim}
\usepackage{dblfloatfix}
\usepackage{color}
\usepackage{colortbl}
\graphicspath{{./img/}}
% \usepackage[linesnumbered,noend]{algorithm2e}
\usepackage{listings}
\usepackage{soul}
\usepackage{enumitem,makecell}
\usepackage{colortbl}
\usepackage{float}
\usepackage{wrapfig}
\usepackage{pgfgantt}
\usepackage[thinc]{esdiff}
\usepackage{algorithm,algpseudocode}
\algrenewcommand\textproc{}
\usepackage{tabularx,booktabs}
\usepackage{xcolor}
% defined centered version of "X" column type:
\newcolumntype{C}{>{\centering\arraybackslash}X} 
\setlength{\extrarowheight}{1pt} % for a bit more open "look"



%Environments begin
\makeatletter
\newenvironment{breakablealgorithm}
  {% \begin{breakablealgorithm}
   \begin{center}
     \refstepcounter{algorithm}% New algorithm
     \hrule height.8pt depth0pt \kern2pt% \@fs@pre for \@fs@ruled
     \renewcommand{\caption}[2][\relax]{% Make a new \caption
       {\raggedright\textbf{\fname@algorithm~\thealgorithm} ##2\par}%
       \ifx\relax##1\relax % #1 is \relax
         \addcontentsline{loa}{algorithm}{\protect\numberline{\thealgorithm}##2}%
       \else % #1 is not \relax
         \addcontentsline{loa}{algorithm}{\protect\numberline{\thealgorithm}##1}%
       \fi
       \kern2pt\hrule\kern2pt
     }
  }{% \end{breakablealgorithm}
     \kern2pt\hrule\relax% \@fs@post for \@fs@ruled
   \end{center}
  }
\makeatother


\newcounter{savealgorithm}
\newenvironment{subalgorithms}
 {%
  \stepcounter{algorithm}%
  \edef\currentthealgorithm{\thealgorithm}%
  \setcounter{savealgorithm}{\value{algorithm}}%
  \setcounter{algorithm}{0}%
  \renewcommand{\thealgorithm}{\currentthealgorithm\alph{algorithm}}%
 }
 {%
  \setcounter{algorithm}{\value{savealgorithm}}%
 }

%Environments end 




\def\BibTeX{{\rm B\kern-.05em{\sc i\kern-.025em b}\kern-.08em
    T\kern-.1667em\lower.7ex\hbox{E}\kern-.125emX}}
    

\newcommand\todo[1]{\textcolor{red}{#1}}

\begin{document}

\title{PFSL: Personalized \& Fair Split Learning with Data \& Label Privacy  for thin clients}


\author{\IEEEauthorblockN{Manas Wadhwa, Gagan Raj Gupta, Ashutosh Sahu, Rahul Saini, Vidhi Mittal}
\IEEEauthorblockA{\textit{Department of Electrical Engineering and Computer Science} \\
\textit{Indian Institute of Technology, Bhilai}\\
 Bhilai, India \\
\{manasw, gagan, ashutoshsahu, rahuls, vidhimittal\}@iitbhilai.ac.in}
}


\maketitle
\begin{abstract}
The traditional framework of federated learning (FL) requires each client to re-train their models in every iteration, making it infeasible for resource-constrained mobile devices to train deep-learning (DL) models. Split learning (SL) provides an alternative by using a centralized server to offload the computation of activations and gradients for a subset of the model but suffers from problems of slow convergence and lower accuracy. 

In this paper, we implement PFSL, a new framework of distributed split learning where a large number of thin clients perform transfer learning in parallel, starting with a pre-trained DL model without sharing their data or labels with a central server. We implement a lightweight step of personalization of client models to provide high performance for their respective data distributions. Furthermore, we evaluate performance fairness amongst clients under a work fairness constraint for various scenarios of non-i.i.d. data distributions and unequal sample sizes. Our accuracy far exceeds that of current SL algorithms and is very close to that of centralized learning on several real-life benchmarks. It has a very low computation cost compared to FL variants and promises to deliver the full benefits of DL to extremely thin, resource-constrained clients. 

\end{abstract}

\begin{IEEEkeywords}
Distributed Machine Learning, U Shaped Split Learning, Federated Learning, non-i.i.d., Personalization, Fairness, Image Classification
\end{IEEEkeywords}

% Importance and appeal of children's drawings
Children's depictions of the human figure are highly expressive and varied.
As one of the very first subjects children attempt to draw, the representation begins as an almost unintelligible cloud of scribbles. 
As the child grows, their representation of the human figure becomes more developed and is extended to graphically represent many different types of characters: people, animals, and even personified objects (see Figure 1).

Who among us has not wished, either as a child or as an adult, to see such figures come to life and move around on the page?
Sadly, while it is relatively fast to produce a single drawing, creating the sequence of images necessary for animation is a much more tedious endeavor, requiring discipline, skill, patience, and sometimes complicated software.
As a result, most of these figures remain static upon the page.

% We built a system to animate them.
Inspired by the importance and appeal of the drawn human figure, we design and build a system to automatically animate it given an in-the-wild photograph of a child's drawing. 
Our system is fast, intuitive, and robust to much of the variation present in these types of drawings, making it well-suited to allow our target audience--children--to see their own characters coming to life.
The system is comprised of four stages: figure detection, segmentation masking, pose estimation/rigging, and animation. 
We describe each stage and identify common causes of failure in each. 
For object detection and pose estimation, we make use of existing computer vision models designed to detect human figures and joints in photographs; we fine-tune these models for use with children's drawings.
For segmentation, we present a straightforward, image processing-based method that, for animation purposes, is more useful and accurate than segmentation masks obtained from a fine-tuned object detection model.
During the animation step, we take advantage of the \textit{twisted perspective} commonly seen in children’s drawings to retarget motion capture data onto the character in a novel and appealing way.

% We use existing machine learning models. However, given the wide domain gap it's not clear how much fine-tuning data was needed. So we ran some experiments to find out and report it.
While our system leverages existing models and techniques, most are not directly applicable to the task due to the many differences between photographic images and simple pen and paper representations. 
To this end, we couple the presentation of our system with a set of experiments exploring the relationship between fine-tuning training set size and success rates.
We also include a perceptual study validating viewer preference for incorporating \textit{twisted perspective} into the motion retargeting step.

We validate the desirability and appeal of our system by building and publicly releasing a version of it as the \AD Demo \,\cite{animateddrawings}.
Launched in December 2021, this demo has been used by millions of people around the world to animate their children's drawings.
Inspired by this reception, our second contribution is The Amateur Drawings Dataset: \hjs{180,000 drawings and user-accepted annotations collected, with consent, through the demo. See Section \ref{sec:UI} for a description of how the annotations were generated.}
We believe this dataset will be a resource to researchers from various fields seeking to better understand the space of amateur drawings, evaluate new algorithms in this domain, or develop new drawing-based tools in general.

To summarize, our contributions are as follows:
\begin{enumerate}
    \item 
    We explore the problem of automatic sketch-to-animation for children's drawings of human figures and present a framework that achieves this effect. We also present a set of experiments determining the amount of training data necessary to achieve high levels of success and a perceptual study validating the usefulness of our motion retargeting technique.
    \item To encourage additional research in the domain of amateur drawings, we present a first-of-its-kind dataset of 180,000 user-submitted amateur drawings, along with user-accepted bounding box, segmentation mask, and joint location annotations.
\end{enumerate}

Upon acceptance of this paper, we plan to publicly release the Amateur Drawings Dataset, project code, and fine-tuned model weights.

\section{Related Work}

\textbf{Topological Map in Exploration and Navigation.} Inspired by the animal and human psychology~\cite{tolman1948cognitive}, a large amount of work has recently proposed to build topological map to represent an environment~\cite{graphtopoexplore,Murphy08ICRA,neural_topomap,beeching2020learning,savinov2018semiparametric,VisualGraphMem_ICCV21,MRTopoMap,Savarese-RSS-19}. They use the topological map for tasks such as navigation~\cite{neural_topomap,learn2explore_iclr20,savinov2018semiparametric,VisualGraphMem_ICCV21,Savarese-RSS-19}, exploration~\cite{learn2explore_iclr20,graphtopoexplore,Murphy08ICRA,savinov2018semiparametric,MRTopoMap,TSGM} and planning~\cite{beeching2020learning}. To build the topological map, they combine various sensors such as RGB image, depth map~\cite{TSGM,Savarese-RSS-19}, pose~\cite{neural_topomap,learn2explore_iclr20,beeching2020learning} and even LiDAR scanner~\cite{MRTopoMap,graphtopoexplore}. Some of them further adopt data-hungry and computation-demanding Reinforcement Learning~(RL) techniques to train the model to construct the topological map~\cite{neural_topomap,learn2explore_iclr20,VisualGraphMem_ICCV21}. Kwon \textit{et al.}~\cite{VisualGraphMem_ICCV21} combine imitation learning~(IL) and RL to train the model. Some of these methods~\cite{neural_topomap,learn2explore_iclr20,beeching2020learning} involve metric information to construct the topological map. N.~Savinov~\textit{et. al.}~\cite{savinov2018semiparametric} use the random walk to construct the topological map, which inevitably leads to an inefficient topological map. TSGM~\cite{TSGM} jointly adds surrounding objects during topological map construction. Unlike these prior works, our \textit{\acronym{}} is completely metric-free and simple in experimental configuration~(just RGB image, much smaller expert demonstration size).

\textbf{Hallucinating Future Feature.} The idea of hallucinating future latent features has been discussed in other application domains. Previous work has utilized this idea of visual anticipation in video prediction/human action prediction~\cite{16Vondrick,17Zeng,20Chang,21Fernando,Suris2021LearningTP}, and researchers have applied similar ideas to robot motion and path planning~\cite{Jain2016RecurrentNN, Koppula2016, Carlone2019, Park2016}. As stated in~\cite{16Vondrick,17Zeng,Suris2021LearningTP}, visual features in the latent space provide an efficient way to encode semantic/high-level information of scenes, allowing us to do planning in the latent space, which is considered more computationally efficient when dealing with high-dimensional data as input~\cite{Lippi2020,Ichter2019}. Different from previous robotics work, we take advantage of this efficient representation by adding deep supervision when anticipating the next visual feature, which was computationally intractable if we were to operate at the pixel level.

\textbf{Deeply-Supervised Learning} has been extensively explored~\cite{deeply_supervised_nets,knowledge_synergy,li2017deep,li2018deep} during the past several years. The main idea is to add extra supervision to various intermediate layers of a deep neural network in order to more effectively train deeper neural networks. In our work, we adopt a similar idea to deeply supervise the training of feature hallucination and action generation.


\begin{figure*}[t]
    \centering
    \includegraphics[width=0.8\linewidth]{Topo_map.pdf}
    \caption{\textbf{Training and inference for task and motion imitation.} Feature extractor $g_\psi$ takes image $I_t$ as input and generates the corresponding feature vector $f_t$. \textit{TaskPlanner} $\pi_{\theta_T}$ is a recurrent neural network (RNN) consuming a sequence of features $\{ f_{t-10}, \cdots, f_t\}$ to hallucinate the next best feature to visit $\hat{f}_{t+1}$. \textit{MotionPlanner} $\pi_{\theta_M}$ consumes the concatenation (denoted by $\bigoplus$) of ${f}_{t}$ and $\hat{f}_{t+1}$ and generates the action to move the agent towards the hallucinated feature. During training, we supervise all the intermediate outputs including the intermediate hallucinated features $\{ \hat{f}_{t-9}, \cdots, \hat{f}_{t} \}$ and the intermediate actions $\{ \hat{a}_{t-10}, \cdots, \hat{a}_{t-1} \}$, in addition to the final output $\hat{f}_{t+1}$ and $\hat{a}_t$. During inference, current observation $I_t$ is firstly encoded and fed into $\pi_{\theta_T}$ to hallucinate $\hat{f}_{t+1}$, and then $\hat{f}_{t+1}$ combined with the ${f}_{t}$ is fed into $\pi_{\theta_M}$ for motion planning. $\mathcal{L}_T$ is $L_2$ loss and $\mathcal{L}_M$ is cross entropy loss (the subscripts $T$ and $M$ denote \textbf{T}ask and \textbf{M}otion respectively). $h_t$ denotes the hidden state of RNN.} 
    \label{fig:pipeline}
    \vspace{-5mm}
\end{figure*}

\textbf{Task and Motion Planning.} Task and motion planning (TAMP) divides a robotic planning problem into high-level task allocation (task planning) and low-level action for task execution (motion planning). This hierarchical framework is adopted in many robotic tasks such as manipulation \cite{chitnis2016guided,mcdonald2022guided} exploration~\cite{Cao-RSS-21} and navigation \cite{lo2018petlon,thomas2021mptp}. Such a framework allows us to leverage high-level information about the scenes to tackle challenges in local control techniques~\cite{bansal2019-lb-wayptnav}. In this work, to perform active topological mapping of a novel environment, the agent firstly reasons at the highest level about the regions to navigate: hallucinate the next best feature point to visit. Afterward, the agent takes an action to get to the target feature. The whole procedure is totally implemented in feature space without any metric information.

\textbf{Imitation Learning} aims to mimic human behavior or expert demonstrations for a given specific task~\cite{imitation_learning,il_legged,il_planning}. The agent is trained to perform tasks by directly observing demonstrations~\cite{il_legged,il_planning}. In our work, the expert demonstration is a set of image-action pair sequences that an agent would observe along a route that efficiently covers an environment. It is widely accessible in either real-world or simulated environments~(e.g. from human experts or maps of environments). 
\begin{figure*}[t]
    \centering
    \includegraphics[width=\textwidth]{Images/ProblemFormV2.png}
    \caption{A two node UAV network. At each time step, the link cost between $D_{B}$ and $D_{F}$ increases while the link cost between $D_{B}$ and $D_{F}$ is constant. The B.A.T.M.A.N. routing protocol will continue to select the route with the lower link cost, branch $(b)$. The proposed solution, branch $(a)$, would preemptively switch routes to manage network congestion and select an alternate route despite being unfavorable at the current time step.}
    \label{fig:SysMod}
\end{figure*} 

The initial system, shown in Fig.~\ref{fig:SysMod}, is a simple two-node UAV network, which will be extended to a multi-node heterogeneous UAV network in future work. The network employs the B.A.T.M.A.N. routing protocol for communication between UAVs. The aim of this study, illustrated branch $(a)$ in Fig.~\ref{fig:SysMod}, is to demonstrate the feasibility of an ML-aided B.A.T.M.A.N. protocol to improve network congestion by predicting when to switch routes, even if switching to a route with a higher cost is not immediately beneficial. 

\subsection{B.A.T.M.A.N. Protocol}
As a baseline, the network uses the B.A.T.M.A.N routing protocol~\cite{batman}. B.A.T.M.A.N. was designed to address the challenges of routing in mobile ad-hoc networks (MANETs), such as frequent topology changes and the lack of a central authority to coordinate routing. Rather than maintain information about the global network topology, B.A.T.M.A.N. only requires nodes to maintain information about the best next hop to its immediate neighbors. The network is flooded with originator messages (OGMs). OGMs routed through good paths are received by nodes quicker than those transmitted on poor quality routes, informing the nodes in the network which immediate neighbor has the best route to transmit across. The routing tables are configured by selecting the best next hop to the originator node \cite{b11}.

However, there are a few drawbacks to the B.A.T.M.A.N. routing algorithm. For one, if the network contains a substantial number of nodes, B.A.T.M.A.N. can generate a large amount of overhead, as each node must re-broadcast the OGM to its neighbors. This can lead to increased network congestion and reduced overall efficiency. Additionally, in scenarios where the source or destination of the packet is in motion, B.A.T.M.A.N. can suffer from higher delays, which is undesirable if the network topology is highly dynamic. Finally, the drawback we focus on in this work is that B.A.T.M.A.N. is a threshold-based routing protocol, demonstrated in Fig.~\ref{fig:SysMod}. As a result, the node will always choose the next hop with the best route, even if conditions on the current best route are degrading. Waiting to change routes until the threshold is met can cause bottlenecks in the network \cite{b12}. 

%

%
\usepackage{algorithm}
\usepackage{algorithmicx}
\usepackage[noend]{algpseudocode}
%
%
%
%
%
%

%
%
\renewcommand{\algorithmiccomment}[1]{\hfill$\triangleright$ \emph{#1}}

\algnewcommand{\algorithmicswitch}{\textbf{switch}}
\algdef{SE}[SWITCH]{Switch}{EndSwitch}[1]{\algorithmicswitch\ #1\ \algorithmicdo}{\algorithmicend\ \algorithmicswitch}%
\algtext*{EndSwitch}%

\algnewcommand{\algorithmiccase}{\textbf{case}}
\algdef{SE}[CASE]{Case}{EndCase}[1]{\algorithmiccase\ #1}{\algorithmicend\ \algorithmiccase}%
\algtext*{EndCase}%

\algnewcommand{\algorithmicon}{\textbf{on}}
\algdef{SE}[ON]{On}{EndOn}[1]{\algorithmicon\ #1\ \algorithmicdo}{\algorithmicend\ \algorithmicon}%
\algtext*{EndOn}%

\algnewcommand{\algorithmicat}{\textbf{at}}
\algdef{SE}[AT]{At}{EndAt}[1]{\algorithmicat\ #1\ \algorithmicdo}{\algorithmicend\ \algorithmicat}%
\algtext*{EndAt}%

\algnewcommand{\algorithmicrealfunction}{\textbf{function}}
\algdef{SE}[REALFUNCTION]{RealFunction}{EndRealFunction}[1]{\algorithmicrealfunction\ #1\ \algorithmicdo}{\algorithmicend\ \algorithmicrealfunction}%
\algtext*{EndRealFunction}%

\algnewcommand{\algorithmicthroughout}{\textbf{do throughout}}
\algdef{SE}[Throughout]{Throughout}{EndThroughout}[1]{\algorithmicthroughout\ #1\ \algorithmicdo}{\algorithmicend\ \algorithmicthroughout}%
\algtext*{EndThroughout}%

\algrenewcommand{\algorithmicdo}{}
\algrenewcommand{\algorithmicthen}{}

\algnewcommand{\algorithmicgoto}{\textbf{goto}}%
\algnewcommand{\Goto}[1]{\algorithmicgoto~\ref{#1}}%

\algnewcommand{\algorithmicassert}{\textbf{assert}}%
\algnewcommand{\Assert}[1]{\algorithmicassert~{#1}}%

\algnewcommand{\algorithmicbreak}{\textbf{break}}%
\algnewcommand{\Break}[0]{\algorithmicbreak}%

%
%

\algnewcommand{\algorithmicwaiton}{\textbf{wait on}}%
\algnewcommand{\WaitOn}[1]{\algorithmicwaiton~{#1}}%

\algnewcommand{\LineComment}[1]{\State \(\triangleright\) \textit{#1}}
\algnewcommand{\InlineRequire}[1]{\textbf{require} {#1}}



\section{Experiments and Results}
\label{sec:experiments_and_results}
We have implemented our framework, PFSL, in PyTorch. Our implementation is modular and allows specifying the number of clients, DNN model, number of the front, central and back layers, number of data points per client, etc. We have implemented state-of-the-art DML algorithms: FL, FL\_TL \cite{fl_tl}, SL \cite{split_main}, SFLv1, and SFLv2 \cite{split_fed} as described in their papers and reproduced their results for the validating our implementations. For experimentation, we used our workstation with Intel(R) Xeon(R) Gold 5218 CPU @ 2.30GHz, with NVIDIA RTX A6000 GPU.

In all our experiments, we use ResNet-18 \cite{resnet18 } with pre-trained weights, which provides a good balance between accuracy and compute resource requirements. We use 2 ophthalmic imaging datasets for blindness detection: APTOS\cite{aptos2019blindness} and EyePACS \cite{kaggleKaggleDataset} and normalize image size to 224x224. We have used several image classification benchmarks: MNIST with 10 classes of hand-written digits \cite{MNIST},  F-MNIST\cite{fmnist} with 10 classes of fashion products, and CIFAR-10\cite{cifar-100} with 10 classes of objects. The size of input images in these benchmarks is 32x32. Since ResNet-18 was pre-trained with 224x224 size and weights of the initial layers were frozen for all algorithms with transfer learning (PFSL, FL\_TL), these images had to be scaled to 224x224 for them to achieve good performance. 

In the real world, good quality labeled data is a scarce resource. The clients are heterogeneous and their data distributions of clients are often non-i.i.d. Similar to Federated Learning, PFSL is applicable for diverse applications. Thus, we carefully evaluate performance and fairness in practical scenarios/settings. We consider the following settings:
\begin{itemize}
\item{S1: Labels are uniformly distributed, number of labeled data points per client is the same but small.}
\item{S2: Labels are non-uniformly distributed, the number of labeled data points per client is the same but small.}
\item{S3: Labels are uniformly distributed. One client has a high number of labeled data points and the remaining have small.}
\item{S4: All clients have a high number of labeled data points for MNIST, F-MNIST, and CIFAR-10.}
\item{S5: 1000 clients in the system each participating only for 1 epoch, and 50\% of them dropping out before completing training.}
\item{S6: Clients have data coming from different benchmarks: APTOS and EyePACS.}
\end{itemize}

For scenarios S1, S2, S3, and S5, we use CIFAR-10 as it is quite complex, and achieving good accuracy is non-trivial. Apart from the existing DML algorithms, for performance comparison, we also include a centralized model assuming that it has access to the labeled data from all the clients. It is expected to have the highest performance but can compromise data privacy. Hence, the goal of DML algorithms is to equal their performance without compromising accuracy. We have two versions of Central models (with and without TL) to compare with the performance of relevant algorithms. To quantify the benefit of DML, we also compare it with the client's performance by performing training locally without participating in DML. We call this individual training. For individual clients, starting from pre-trained weights is best, so only that is reported (as Individual\_TL). 

\begin{figure}[H]
\includegraphics[width=\linewidth]{img/Setting1_final_plot.png}
\caption{S1: Test Accuracy for DML algorithms vs Datapoints} 
\label{Test_setting1}
\end{figure}
\vspace{-1 em}

\subsection{Setting 1: Small Sample Size (Equal), i.i.d.}
Each client has a very small number of labeled data points, and all these samples are distributed identically across clients. The number of data points per client is slowly increased to study the benefits of having more data per client on the global model's accuracy. For each algorithm, we keep the global epochs fixed at 100 and the number of participating clients at 10. We evaluate each client on a CIFAR-10 test set having 2000 data points with the same distribution and report the average test accuracy. Since the test and train data distributions across all clients are the same, we run our algorithm only till Generalization Phase 1. For FL\_TL, the same number of layers were unfrozen at the client's side as PFSL for a fair comparison.

There are several interesting observations from Fig \ref{Test_setting1}. PFSL achieves accuracy close to the Central model (with TL) with the complete dataset. Its accuracy exceeds that of FL\_TL by 10\%. It is clear from this experiment, that for realistic settings with small sample size, transfer learning is very helpful, since Individual\_TL also exceeds the accuracy of FL, SL, SFLv1, and SFLv2 which attempt to train from scratch.

\begin{figure}[H]
\centering
\includegraphics[width=0.6\linewidth]{img/Setting2_final_plot.png}
\caption{Data Distribution among clients where larger circle indicates more number of training samples for that particular class and client  } 
\label{Setting1_distribution_plot}
\end{figure}

\subsection{Setting 2: Small Sample Size (Equal), non-i.i.d.}

In this scenario, we consider the clients to have a small sample size and different data distributions, as shown in Fig \ref{Setting1_distribution_plot}. There are a total of 10 clients, and each client has 500 labeled data points. We model a situation where every client has more labeled data points from a subset of classes \emph{(prominent classes)} and less from the remaining classes. We chose to experiment with heavy label imbalance and diversity. The test distribution of each client is the same as its train distribution providing an incentive for them to run \textbf{Personalization Phase 2} as well. To evaluate algorithms in this setting, we consider the average F1 scores of the \emph{prominent classes} for each client on its respective test set and then average these over all the clients. As the data distribution across all clients differs, we run our algorithm completely until Phase 2. We run Phase 1 till convergence and Phase 2 again till convergence. 

\begin{table}[H]
    \centering
    \begin{tabular}{|l|l|l|l|}
    \hline
        \textbf{Algorithm} & \textbf{F1 score} & \textbf{Epochs} & \textbf{Test Standard Deviation}\\ \hline
        FL & 0.505 & 100 & 8\\ \hline
        SL & 0.794 & 50 & 4.25\\ \hline
        SFLv1 & 0.749 & 50 & 5\\ \hline
        SFLv2 & 0.773 & 50 & 4.25\\ \hline
        PFSL$_{BL\_3} (gen)$ & \textbf{0.816}  & \textbf{24} & \textbf{5.25}\\ \hline
        PFSL$_{BL\_3} (pers)$ & \textbf{0.953} & \textbf{26} & \textbf{2}\\ \hline
        PFSL$_{BL\_2} (gen)$ & \textbf{0.79}  & \textbf{24} & \textbf{5.75}\\ \hline
        PFSL$_{BL\_2} (pers)$ & \textbf{0.95} & \textbf{30} & \textbf{2.1}\\ \hline
        \end{tabular}
    \caption{S2: Average F1 Scores of 10 clients}
    \label{Setting1: f1 table}
\end{table}

From Table \ref{Setting1: f1 table}, we can observe that the average F1 score of Split learning and its variants is much better than Federated learning. Our algorithm, PFSL, gives the best average F1 scores. This happens because phase 2 allows the clients to train unique models suited to their own train datasets. Moreover, the generalization phase of our algorithm ends at the 25th epoch with better average F1 scores than the other algorithms achieve at 100 epochs. Our lightweight personalization phase results in a significant boost of 0.14 in average F1 scores. PFSL$_{BL\_3}$ has used three back layers at the client while  PFSL$_{BL\_2}$ used only two back layers at the client, which is our default. We observe that increasing the number of layers at the back model only affects the number of epochs needed to converge as the F1 scores converge to the same value in both cases. We notice that once the clients have converged to a generalized model in our framework, its personalization is very quick (2 global epochs in $PFSL_{BL\_3}$ and 6 global epochs in $PFSL_{BL\_2}$). 

The Test Accuracy Standard deviation column shows the values of Performance Fairness of different DML algorithms using the standard deviation of test performance of the clients in different frameworks. We can observe from Table \ref{Setting1: f1 table} that PSFL shows comparable performance fairness to others in the generalization phase. But after running the personalization phase, the fairness increases dramatically, indicating the ability of all the clients to converge easily on their respective personalized models, given the starting generalized weights. This also shows that the weights obtained from the generalization phase need to be fine-tuned to fit individual data better, as described in \cite{ditto_main, P1, P2, P3, P4} to improve fairness. We believe that in a practical setting, the ability to personalize private models is very attractive, and the performance gain and high-performance fairness provide an incentive to clients to work together for the common good while still enjoying the competitive edge.

\subsection{Setting 3: Small Sample Size (Unequal), i.i.d.}
Very often, the participating clients may have a different sample size. Most clients usually have fewer data points, and a select few have a much larger dataset. In this situation, the clients with more data tend to think that participation in distributed training with the clients (who may have fewer data) won't greatly benefit them, and they may lose their competitive edge. Thus, they can decide to forego participating in training completely.

To simulate the above-described situation, we consider 11 clients where the \emph{Large client} has 2000 labeled data points while the other ten small clients have 150 labeled data points, each distributed identically. Note that the class distributions among all the clients are the same. For evaluation purposes, we consider a test set having 2000 data points with an identical distribution of classes as the train set. Firstly, we calculate the average test accuracy when only 10 small clients $C_1 - C_{10}$ with 150 data points each participate. In the next case, all 11 clients (small and Large) participate in training, and we calculate the first client's test accuracy individually and the remaining clients' average test accuracy. We also calculate the performance achieved by the \emph{Large client} if it begins transfer learning on a pre-trained model utilizing only its 2000 labeled data points. To give incentive to the Large client, its test accuracy must increase significantly by participating in the learning with other clients with fewer data.

Existing works have not focused on achieving work fairness when the number of training data points per client is imbalanced. We consider that it is unfair to make the \emph{Large client} work ten times more than each small client if it has ten times more data than them. As we have explained before, in PFSL, we ensure that the number of batches and the batch size with each client in a global epoch remain the same. We calculate the metrics of this setting on our algorithm after ensuring this fairness constraint. Also, since the data distributions across all clients are the same, we run our algorithm only till Phase 1.

We present the Average test and train performance when only the small clients participate in DML training in Table \ref{setting2_t1}. It is evident that PFSL achieves the highest accuracy and minimizes over-fitting. Next, we present the Average test performance of small and \emph{Large client} when all the 11 clients participate in DML training in Table \ref{setting2_t2}. We can see a significant increment in their test accuracies in all the algorithms. This shows that the small clients greatly benefit from the \emph{Large client}. 

If \emph{Large client} is only trained on its dataset of 2000 data points, using a pre-trained model and transfer learning, then the test accuracy is 83.  Table \ref{setting2_t2} shows that only PFSL can give an incentive to such clients by providing a 2\% increase in accuracy. Moreover, our results are calculated after incorporating work fairness. We do not force the \emph{Large client} with more data to do more work in a local epoch than the other smaller clients, yet provide tangible benefits and incentives to participate in DML. We observe that when we make the \textit{Large client} do more work, its accuracy does not increase significantly (2\% over the previous) but its work done per global epoch increases by 16 times. Thus, by work fairness, we ensure high performance for this client with a far lesser workload.  

\begin{table}
\setlength{\tabcolsep}{2.5pt}
\centering
    \begin{tabular}{|l|l|l|}
    \hline
        \textbf{Algorithm} & \textbf{Avg $C_1 - C_{10}$ Test} & \textbf{Avg $C_1 - C_{10}$ Train} \\ \hline
        FL & 48.56 & 86.25 \\ \hline
        SL & 52.42 & 88.01 \\ \hline
        SFLv2 & 50.43 & 85.35 \\ \hline
        SFLv1 & 37.37 & 81.23 \\ \hline
        \textbf{PFSL} & \textbf{81.42} & 100 \\ \hline
    \end{tabular}
    \caption{ S3: Test and Train accuracies when only  $C_1 - C_{10}$ participate in DML}
    \label{setting2_t1}
    \bigskip % or \bigskip\bigskip
\begin{tabular}{|l|l|l|}
    \hline
        \textbf{Algorithm} & \textbf{Avg $C_1 - C_{10}$ Test } & \textbf{\emph{Large Client} Test} \\ \hline
        FL & 59.02  & 59.15 \\ \hline
        SL & 63.41  & 64.64 \\ \hline
        SFLv2 & 61.01  & 61.96 \\ \hline
        SFLv1 & 56.76 & 62.12 \\ \hline
        \textbf{PFSL} & \textbf{84.62}  & \textbf{84.64} \\ \hline
    \end{tabular}
    \caption{S3: Test Accuracies of Clients $C_1 - C_{10}$ and \emph{Large Client} when all participate in DML}
    \label{setting2_t2}
    
\end{table}

\subsection{Setting 4: A large number of data samples}
For the sake of comparison with most existing works, where a large number of samples per client are assumed, we experimented with three different image classification datasets: MNIST, FMNIST, and CIFAR-10. We also tune the hyperparameters of our algorithm on the validation set of the MNIST dataset as shown in Fig. \ref{val_plots}. This helps in giving an estimate of the parameters that work best for achieving optimal performance of all clients in our framework. Five clients were used in the experiment and each client had 10k datapoints for CIFAR-10, 12k MNIST and 12k for FMNIST. These clients then participate in distributed learning and try to achieve performance close to centralized learning. We evaluate all the clients by their accuracies on a common test set. We report the averaged train and test accuracies for each algorithm. 

Table \ref{tab:pop_bm_1} reports the average test accuracies of the participating clients across all the algorithms on CIFAR-10, FMNIST, and MNIST datasets when each of them has a large number of training data points. Our algorithm gives the best performance on all the datasets. Notably, it performs significantly better than the other algorithms on the CIFAR-10 dataset, which is more complex than the other two. 

Table \ref{tab : pop_bm_2} shows the average train and test accuracies of 5 clients and the amount of overfitting (difference between train and test accuracies) across all algorithms on the CIFAR-10 dataset. We can observe that other algorithms heavily overfit their training data and thus perform poorly on their test data. In contrast, our algorithm shows the least amount of over-fitting because of a carefully designed weight-averaging process. 

\begin{figure}[H]
\includegraphics[width= \linewidth]{img/Validation.png}
\caption{S4: Hyper-parameter tuning using the validation set for MNIST} 
\label{val_plots}
\end{figure}

\begin{table}
\setlength{\tabcolsep}{2.5pt}
\centering
\begin{tabular}{|l|l|l|l|}
\hline
\textbf{Algorithm} & \textbf{CIFAR-10} & \textbf{FMNIST} & \textbf{MNIST} \\ \hline
$d_i$ & 10K & 12K & 12K \\ \hline
FL  & 81.52 & 91.38  & 99.26 \\ \hline
SL & 82 & 91.08 & 99.14 \\ \hline
SFLv1 \cite{split_fed}  & 78.82 & 90.55  & 98.7\\ \hline
SFLv2 \cite{split_fed} & 73.82 & 90.93 & 99.08 \\ \hline
\textbf{PFSL} & \textbf{92} & \textbf{93} & \textbf{99.46}  \\ \hline
\end{tabular}
\caption{S4: Average Test Accuracy, five clients on three important image classification benchmarks}
\label{tab:pop_bm_1}
\bigskip % or \bigskip\bigskip
\begin{tabular}{|l|l|l|l|}
    \hline
        \textbf{Algorithm} & \textbf{Train Accuracy} & \textbf{Test Accuracy} & \textbf{Overfitting} \\ \hline
        FL & 92 & 81.52 & 10.48 \\ \hline
        SL & 96.94 & 82 & 14.94 \\ \hline
        SFLv1 & 92.73 & 78.82 & 13.91 \\ \hline
        SFLv2 & 96.5 & 73.82 & 22.68 \\ \hline
        \textbf{PFSL} & \textbf{99.8} & \textbf{92} & \textbf{7.8} \\ \hline
    \end{tabular}
    \caption{S4: Overfitting on CIFAR-10, 5 clients}
    \label{tab : pop_bm_2}
  
\end{table}


\subsection{Setting 5: System simulation with 1000 clients}

\begin{figure}[H]
\includegraphics[width=\linewidth]{img/sys_simulation_plots.png}
\caption{S5: Test Accuracy with number of epochs} 
\label{fig: sys_sim}
\end{figure}

In this setting, we simulate 1000 clients.  We allow only 10 clients to simultaneously perform the training. Each client stays in the system only for 1 round which lasts only 1 epoch. Thus, we evaluate our system for the worst possible scenario when every client cannot stay in the system for long and can only afford to make a minimal effort to participate. We assume that each client has 50 labeled data points sampled randomly but unique to the client. Within each round, we simulate a dropout, where clients begin training but are not able to complete the weight averaging. We keep the dropout probability at 50\%. In this experiment, we used the CIFAR-10 dataset. We find that the final test accuracy after 100 rounds (where each client begins the training exactly once), was 90\% for PFSL (see Fig.\ref{fig: sys_sim}). This is very close to that observed in an ideal scenario with 5 clients having 10K data points each (see Table \ref{tab : pop_bm_2}). This setting clearly demonstrates the high scalability of our framework in realistic scenarios with extremely resource-constrained devices participating in the distributed learning process. 


\subsection{Setting 6: Different Diabetic Retinopathy Datasets}
This experiment describes the realistic scenario when healthcare centers have different sets of raw patient data for the same disease. We have used two datasets EyePACS \cite{eyePacs} and APTOS\cite{aptos2019blindness}. The images in EyePACS were captured under various conditions by various devices at multiple primary care sites throughout California and elsewhere. The APTOS dataset, on the other hand, has been collected across various healthcare sites in India. For the experiment, we created a set-up where there was a total of 10 clients, in which the first 5 clients were provided the APTOS dataset and the next 5 clients (Client 5- Client 9) were provided the EyePACS dataset. The sample size of data for each client was kept at 500. The images had to be classified into three categories-No DR(0), Moderate(1), and Severe(2). All clients ran the personalization phase to maximize their test accuracies on their respective datasets.

Table \ref{tab:algo_f1_score} shows the average test accuracies achieved by the first set of clients and the second set of clients separately. We have also shown the F1 Scores of one representative client having the APTOS dataset and one of the clients having the EyePACS dataset across all the algorithms.
\begin{table}[H]
\centering
\begin{tabular}{|l|l|l|l|l|l|}
\hline
\textbf{Algorithm} &  \textbf{$C_0$ F1 } &\textbf{$C_5$ F1 } & \textbf{$C_0-C_4$ Test}  & \textbf{$C_5-C_9$ Test}  \\ \hline

FL  & 0.73 & 0.36	 & 81.7	& 70.14 \\ \hline
SL & 0.72 & 0.41 & 80.37 &	69.57 \\ \hline
SFLv1 \cite{split_fed}  & 0.61 & 0.32 &	77.79 &	69.55\\ \hline
SFLv2 \cite{split_fed} & 0.74 & 0.43 & 80.21 &	66.53  \\ \hline
\textbf{PFSL} & \textbf{0.78}  & \textbf{0.58} & \textbf{85.36} & \textbf{70.86}  \\ \hline

\end{tabular}
\caption{S6: F1 Score of Client 0 (having APTOS dataset) and F1 Score of Client 5 (having EyePACS dataset) and Avg Test Accuracies of Client 0-4 (APTOS dataset) and Client 5-9 (EyePACS dataset)}
\label{tab:algo_f1_score}
\end{table}

From the results, we see that even though the two datasets are very different, the PFSL algorithm achieves the highest F1 scores for Client 0 and Client 5 and also the highest average test accuracies for the first and second sets of clients. This shows that our algorithm achieves the best, personalized results, wherein each set of clients performs very well on their own dataset.

\subsection{Work Fairness Analysis}

Table \ref{tab : wfair} shows the work done by each client in different DML algorithms. As we can see, there is a difference in values of $C_i$ for S3 and S6 for FL, SL, SFLv1, SFLv2, because of the change in the size of input images from 32x32 in S3 to 224x224 in S6. For PFSL, we had to use 224x224 in both S3 and S6 because of using transfer learning on ResNet-18 pre-trained model. This could have been reduced considerably by using a customized pre-trained model for 32x32. It is evident that FL algorithms are very resource intensive as compared to all the SL variants. In S3, we can see the clear benefit of work fairness for the large client (O\_L\_c) under PFSL. All other DML algorithms cause the large client to overwork heavily for marginal gains. As we mentioned, a better approach would be to let the large client participate in multiple rounds to increase its accuracy gradually. Finally, in Setting 6, we can see the clear benefits of using PFSL over all other DML algorithms.

\begin{table}[H]
    \centering
    \begin{tabular}{|p{1.5cm}|l|l|p{1cm}|p{0.8cm}|l|}
    \hline
        \textbf{Algorithm } & \textbf{S3:C\_i} & \textbf{S6:C\_i} & \textbf{S3: O\_L\_c} & \textbf{S3: O\_S\_c } & \textbf{S6: O\_i} \\ \hline
        FL  & 14.28 & 700.2 &44268 & 2856 & 105030 \\ \hline
        SL, SFLv1, SFLv2 & 2.76 &  135.9 & 8556& 552 & 10192 \\ \hline
        PFSL  & 15.5 & 15.5 & 775& 775 & 744 \\ \hline
    \end{tabular}
    \caption{Work Fairness Analysis for settings S3 and S6. C refers to the work done per iteration and O refers to the total work done till convergence is reached. L\_c refers to the Large client in S3 and S\_c refers to the small clients in S3. All values are computed in GFlops using a profiler \cite{profiler}}
    \label{tab : wfair}
\end{table}

%\section{}
%\label{sec:resDir}


\section{Conclusion}
\label{sec:conclusion}
% <>
Since its advent in 1931, Koopman operator theory \cite{koopman:1931} has only recently been actively utilized for solving practical problems, thanks to the introduction of the DMD algorithm in 2008 \cite{schmid:2008}. Since then, a multitude of DMD algorithm variations have risen to prominence and found utility across various fields. A notable feature of our survey paper was reviewing and categorizing the results of over 100 research papers based on both application and algorithm type in smart mobility and vehicle engineering  (see Table~\ref{tab1} and Section~\ref{sec:vehicApp}).  Additionally, this survey paper identified potential research gaps in smart mobility and vehicular engineering applications (Remarks~\ref{remGap1}--\ref{remGap6}). Finally, this review paper discussed theoretical aspects of Koopman operator theory that have been largely neglected by the smart mobility and vehicle engineering community and yet have large potential for contributing to solving open problems in these areas (see Section~\ref{subsec:theorIssue}).

\noindent{\textbf{Future Research Directions.}}	Given the emergence of cyber-threats against connected and autonomous vehicles as well as robotic systems (see, e.g.,~\cite{nekouei2021randomized,mohammadi2022generation}), a future research direction might include utilizing Koopman operator-based algorithms for designing cyber-resilient vehicular and smart mobility applications (see, e.g.,~\cite{taheri2022data} for a related line of research). Another potential research direction is using Koopman operator-based algorithms for predicting the motion of vulnerable road users (VRUs), e.g., pedestrians and cyclists (see, e.g.,~\cite{pool2019context,scholler2020constant}). Finally, rehabilitation robotics and robotic exoskeletons can be the benefactors of the predictive capabilities of Koopman operator-based algorithms for detecting tripping events and/or system  identification in various modes of locomotion (see, e.g.,~\cite{kumar2019extremum,aprigliano2019pre}).



%Fig. 1 depicts the accumulation of such algorithms since 2014, which are particular to vehicle engineering and smart mobility, i.e., the focus of this review. Table 1 summarizes the varieties of relevant algorithms developed in those studies. Furthermore, we have highlighted theoretical issues, whose expansion will have potential applications to the wide research area of smart mobility and vehicle engineering.  

%Although fairly comprehensive, we have found several gaps in this research area. In particular, we could not find any studies related to elevators, robots/vehicles employing crawling, slithering, hopping or peristaltic locomotion, arctic or special-terrain vehicles such as those employing screws or tracks, hovercraft and other amphibious vehicles or subsystems which tolerate flexible environments, classification or guidance systems related to vehicles for drilling or agriculture, or for current-ripple, power-split, battery health monitoring, nuclear propulsion, exoskeletons/prosthetics, personal mobility, motorsports, specialized rovers or similar open problems in emerging areas.  These examples are, of course, not exhaustive.  
%
%The purely data-driven nature of Koopman operators holds the promise of capturing unknown and complex dynamics for reduced-order model generation and system identification, through which the rich machinery of linear control techniques can be utilized. The emergent nature of the smart mobility and vehicular-related applications, where  the Koopman operator  in each particular application needs to be approximated, implies that the development of various Koopman operator approximation  algorithms is expected to grow along with the vehicular problems they aim to solve.  Given the ongoing development of this research area and the many existing open problems in the fields of smart mobility and vehicle engineering, a survey of techniques and open challenges of applying Koopman operator theory to this vibrant area is warranted.  To the best of our knowledge, this survey paper is the \emph{first of its kind} reviewing the applications of Koopman operator theory within a focused research area, namely, smart mobility and vehicle engineering applications. A \emph{notable feature} of our survey paper is reviewing and categorizing the results of over 100 research papers based on both application and algorithm type  (see Tables~\ref{tab1}--~\ref{tab4} and Section~\ref{sec:vehicApp}) that are concerned with the applications of Koopman operator theory to the field of smart mobility and vehicular engineering. Such a \emph{comprehensive and  detailed categorization} will be beneficial to the research practitioners working in the field.  Furthermore, this review paper discusses theoretical aspects of Koopman operator theory that have been largely neglected by the smart mobility and vehicle engineering community and yet have large potential for contributing to solving open problems in these areas. Additionally, our survey paper seeks to \emph{identify gaps} in the smart mobility and vehicle engineering research where new and existing Koopman operator-based methods have the potential to further develop and address unsolved problems  potentially benefiting from the perspectives of nonlinear system identification, control, global linearization, and the predictive powers that Koopman operator theory has to offer (see, e.g., Remarks~\ref{remGap1}--\ref{remGap6}). 


\bibliographystyle{IEEEtran}
\bibliography{ref}
\newpage
\appendix
\section{Skew Equations}
We will justify and show the three equations used in Lemma \ref{skew rel} to narrow our search for these skew axial algebras. Although they do not provide much use to understanding how these algebras could be constructed, they do make the proof easier.

Suppose $v$ is an $\mu$-eigenvector of an axis, $x$, where $\mu\neq1$. Then the projection on that axis should be equal to 0; that is, $\lm_x(v)=0$. Coincidentally, nearly all of the eigenvectors in Lemma \ref{eigen a} and \ref{eigen b} satisfy that rule. However we have
\begin{equation*}
 0=\lm_b\left(-\frac{P}{\bt}a+Pb+c\right) = -\frac{P}{\bt}\lmf_1+P+\lmf_2.
\end{equation*}
Whence we get Equation (\ref{proof1}).

\begin{defn}
Let $x$ be a $\mon{\al,\bt}$-axis in $A$, $\lm\in \{1,0, \al, \bt\}$ and $v\in A$. We denote $[v]^x_\lm$ to be the component of $v$ in $ A_\lm(x)$. 
\end{defn}
\begin{lem}
Let $w:=\frac{1}{2}(b-c)$. We have $[a]^a_\bt=0$, $[b]^a_\bt=w$, $[c]^a_\bt=-w$, $[\sg]^a_\bt=0$. Further, $[ab]^a_\bt=\bt w$, $[ac]^a_\bt=-\bt w$, $[bc]^a_\bt=0$, $[a\sg]^a_\bt=0$, $[b\sg]^a_\bt=\dt^fw$, $[c\sg]^a_\bt=-\dt^fw$ and $[\sg^2]^a_\bt=0$.
\end{lem}
\proof
As $a\in A_1(a)$, it has no $\bt$-component in $A_\bt(a)$ and $[a]^a_\bt=0$. As $\sg\in A_{\{1,0,\al\}}(a)$, it has no $\bt$-component in $A_\bt(a)$ and $[\sg]^a_\bt=0$. We can express $b$ in terms of the eigenvectors of $\text{ad}_a$ in Lemma \ref{eigen a}. The reader can check
\[ b= \lm_1 a+ \frac{1}{\al}\left(\ep a+\frac{1}{2}(\al-\bt)(b+c)-\sg\right)+ \frac{1}{\al}\left(\gm a +\frac{1}{2}\bt(b+c)+\sg\right)+\frac{1}{2}(b-c).\]
Thus $[b]_\bt^a=w$. As $c=b^{\tu{a}}$, we get $[c]_\bt^a=-w$.

Let $x, y \in A_{\{0,1,\al\}}(a)$ and notice $x^2, xy\in A_{\{1,0,\al\}}(a)$ and so has no $\bt$-component in $A_\bt(a)$. Therefore $[\sg^2]^a_\bt=[a\sg]^a_\bt=0$. Also
\[ [bc]_\bt^a=P\left([a]_\bt^a+\frac{1}{\bt}[\sg]_\bt^a\right)=0.\]
Note that
\[ [ab]_\bt^a=[\sg]_\bt^a+\bt[a]_\bt^a+\bt[b]_\bt^a=\bt w\]
and 
\[ [b\sg]_\bt^a=(\al-\bt)[\sg]_\bt^a+\bt(\al-\bt)[a]_\bt^a+dt^f[b]_\bt^a=\dt^f w.\]
Applying $\tu{a}$, we get $[ac]_\bt^a$ and $[c\sg]_\bt^a$. \qed



Let $u:= (b -\al)a - \bt b=\sg -(\al-\bt)a$. As $A_\bt(b)=\{0\}$, we have that $u\in A_{\{1,0\}}(b)$. By Lemma \ref{Seress}, the following holds
\[b(au)=(ba)u.\]
Notice
\[ au = a(\sg -(\al-\bt)a)=(\dt -(\al-\bt))a+\frac{1}{2}\bt(\al-\bt)(b+c)+(\al-\bt)\sg\]
and so
\begin{eqnarray*}
[b(au)]_\bt^a &=& (\dt -(\al-\bt))[ab]_\bt^a+\frac{1}{2}\bt(\al-\bt)([b]_\bt^a+[bc]_\bt^a)+(\al-\bt)[b\sg]_\bt^a\\
& =& \left(\bt(\dt -(\al-\bt))+\frac{1}{2}\bt(\al-\bt)+(\al-\bt)\dt^f\right)w
\end{eqnarray*}
We also have 
\begin{eqnarray*}
[(ba)u]_\bt^a&=&[(\sg+\bt a +\bt b)(\sg -(\al-\bt)a)]_\bt^a\\
&=& [\sg^2]_\bt^a -(\al-2\bt)[a\sg]_\bt^a +\bt [b\sg]_\bt^a -\bt(\al-\bt)[a]_\bt^a - \bt(\al-\bt)[ab]_\bt^a\\
&=& (\bt\dt^f -\bt^2(\al-\bt)) w
\end{eqnarray*}
By Lemma \ref{Seress}, we have $0=(ba)u-b(au)$ moreover $0=[(ba)u]_\bt-[b(au)]_\bt$. Looking at the coefficient of $w$, we have
\begin{eqnarray*} 
0&=& (\bt\dt^f-\bt^2(\al-\bt))\\
& -& \left(\bt \dt -\bt(\al-\bt)+\frac{1}{2}\bt(\al-\bt)+(\al-\bt)\dt^f\right)\\
&=&-\bt^2(\al-\bt) -\bt\dt+\frac{1}{2}\bt(\al-\bt)-(\al-2\bt)\dt^f.
\end{eqnarray*}
Rearranging we get Equation (\ref{proof2}).

Let $v:=Pa+\frac{P}{\bt}\sg -\al c=c(b-\al)$. Notice that $v \in A_{\{1,0\}}(b)$. Again by Lemma \ref{Seress}, the following holds
\[b(av)=(ba)v.\]
We have
\begin{eqnarray*}
av &=& Pa +\frac{P}{\bt}\left(\dt a + \frac{1}{2}\bt(\al-\bt)(b+c) +(\al-\bt)\sg\right)\\
& -&\al(\bt a +\bt c +\sg)\\
&=&\left(P +\frac{P}{\bt}\dt -\al\bt\right)a+\left(\frac{1}{2}(\al-\bt)P\right)b\\
&+&\left(\frac{1}{2}(\al-\bt)P-\al\bt\right)c+\left(\frac{P}{\bt}(\al-\bt)-\al\right)\sg.
\end{eqnarray*}
Therefore
\begin{eqnarray*}
[b(av)]_\bt^a &=&\left(P +\frac{P}{\bt}\dt -\al\bt\right)[ab]_\bt^a+\left(\frac{1}{2}(\al-\bt)P\right)[b]_\bt^a\\
&+&\left(\frac{1}{2}(\al-\bt)P-\al\bt\right)[bc]_\bt^a+\left(\frac{P}{\bt}(\al-\bt)-\al\right)[b\sg]_\bt^a.\\
&=&\left(\bt \left(P +\frac{P}{\bt}\dt -\al\bt\right)+\dt^f\left(\frac{P}{\bt}(\al-\bt)-\al\right)\right)w
\end{eqnarray*}
We also have
\begin{eqnarray*}
[(ba)v]_\bt^a&=&\left[\left(\bt a +\bt b +\sg\right)\left(Pa+\frac{P}{\bt}\sg -\al c\right)\right]_\bt^a\\
&=&2P[a\sg]_\bt^a +\frac{P}{\bt}[\sg^2]_\bt^a -\al [c \sg]_\bt^a + \bt P [a]_\bt^a -\al\bt [ac]_\bt^a\\
&+&\bt P [ab]_\bt^a +P[b\sg]_\bt^a -\al\bt [bc]_\bt^a\\
&=&\left(\al \dt^f +\al\bt^2 +\bt^2 P  +P\dt^f\right)w
\end{eqnarray*}
By Lemma \ref{Seress}, $0=[b(av)]^a_\bt-[(ba)v]^a_\bt$ and looking at the coefficient of $w$, we get 
\begin{eqnarray*}
0&=&[b(av)]_\bt-[(ba)v]_\bt\\
&=&\left(\bt P +\dt P -\al\bt^2+\frac{1}{2}(\al-\bt)P+\frac{P}{\bt}(\al-\bt)\dt^f -\al\dt^f\right)\\
&-&\left(\bt^2P +P\dt^f+\al\dt^f +\al\bt^2 \right)\\ 
&=&\left(\frac{P}{\bt}\left[\bt^2 +\bt\dt+\frac{1}{2}\bt(\al-\bt)+(\al-2\bt)\dt^f-\bt^3\right]-2\al(\dt^f+\bt^2)\right).
\end{eqnarray*}
From Equation (\ref{proof2}), we get that
\begin{eqnarray*}
0&=&\frac{P}{\bt}\left[\bt^2 -\bt^2(\al-\bt) +\frac{1}{2}\bt(\al-\bt)-(\al-2\bt)\dt^f\right.\\
&+&\left.\frac{1}{2}\bt(\al-\bt)+(\al-2\bt)\dt^f-\bt^3\right]-2\al(\dt^f+\bt^2)\\
&=&\frac{P}{\bt}\left[\bt^2 -\bt^2(\al-\bt) +\bt(\al-\bt)-\bt^3\right]-2\al(\dt^f+\bt^2)\\
&=&\frac{P}{\bt}\al\bt\left[1-\bt\right]-2\al(\dt^f+\bt^2).
\end{eqnarray*}
Hence we get Equation (\ref{proof3}).

\section*{Acknowledgements}
I would like to thank Professor Sergey Shpectorov for his guidance throughout my PhD studies so far and pushing me to complete this paper. I would also like to thank my family for their continuing support. 

\end{document}