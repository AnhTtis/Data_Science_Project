% \appendices
\appendix


%%%%%%%%%%%%%%%%%%%%%%%%%%%%%%%%%
\subsection{Existence of limit \texorpdfstring{$\overline X$}{}}
\label{sec:Xbar}

In this section, we show that the limit $\overline X$ in \eqref{eq_Xbar} exists under most practical policies if $X(t)$ is bounded in the sense of \eqref{eq_zeta}.
The key is to show that the PDMP is irreducible in the sense of \cite[p.5]{meyn1993survey}.
The irreducibility argument in the discrete state space is straightforward. For the continuous state space, note that, under most practical policies, the state $x=[0,0]$ (or $u=\{0,0\}$) can be exactly attained from any initial condition in finite time with a positive probability. Hence, the PDMP is irreducible. Since the PDMP is also bounded in the sense of \eqref{eq_zeta}, the PDMP is positive by Harris \cite[Theorem 7]{meyn1993survey}, and thus the limit in \eqref{eq_Xbar} exists.





\section*{Acknowledgment}
The authors appreciate the SJTU undergraduate students that helped with this work: Qi Dai contributed to the theoretical results, Tianhao Yao, Yunqian Yao, Jingye Lin contributed to simulation, and Siyuan Lin, Yufeng Yan contributed to the experiments.