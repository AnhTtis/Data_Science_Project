\section{Multi-OD Intersections}

In this section, we extend the capacity guarantees in Theorem~\ref{thm_two} to multi-OD intersections. Specifically, we consider three typical configurations,  i.e. four ODs with turning, four ODs without turning, and general twelve ODs; see Fig.~\ref{fig:multi}. Since these configurations are more complex than two ODs, we will focus on capacity guarantees; travel delay will also be considered if insightful results are available.
%
\begin{figure}[hbt!]
    \centering
    \subfigure[Four ODs with turning.]
    {\includegraphics[width=0.2\textwidth]{images/four_turning_config.png}
    \label{fig:four_turning_config}
    }
    \subfigure[Four ODs without turning.]
    {\includegraphics[width=0.2\textwidth]{images/four_no_turning_config.png}
    \label{fig:four_no_turning_config}
    }
    \subfigure[Twelve ODs.]
    {\includegraphics[width=0.4\textwidth]{images/twelve_config.png}
    \label{fig:twelve_config}
    }
    \caption{Multi-OD intersections studied in this paper.}
    \label{fig:multi}
\end{figure}
%

By analogy with the two-OD system in Section~\ref{sec:two}, we define PDMPs for multi-OD systems as follows. Note that the notations in this subsection are self-contained. Let $\mathcal K$ be the set of traffic classes. Class-$k$ vehicles arrive as a Poisson process of rate $\lambda_k$. The service time of a vehicle consists of an inter-vehicle headway and a crossing time defined as follows. The headway is specified by a matrix $\Theta\in\mathbb R_{\ge0}^{n\times n}$, where $n$ is the number of ODs. We assume that $\theta_{i,k}\le\theta_{j,k}$ if $i$ and $k$ are from the same origin and if $j$ and $k$ are from different origins. The crossing time is a rv with $\overline R$, variance $\sigma_R^2$, and MGF $g_R(\alpha)$, all of which are assumed to be bounded.
Let $G(t)\in\mathcal G=\cup_{n=0}^\infty(\mathcal K^n\times\mathbb Z_{>0}^n)$ be the sequence of crossing, and $U(t)\in\cup_{n=0}^\infty\mathbb Z_{\ge0}^n$ be the residual service times. 
If the sequencing at time $t$ only depends on $(G(t_-),U(t_-))$, then $\{(G(t),U(t));t\ge0\}$ is an RCLL PDMP. 
Let $\mathcal K_\kappa$ be the set of ODs with the origin $\kappa$, where $\kappa\in\{W,N,E,S\}$.
We also consider the simplified state $X(t)=[X_W(t),X_N(t),X_E(t),X_S(t)]\in\mathbb R_{\ge0}^4$, where $X_\kappa(t)=\sum_{k\in\mathcal K_\kappa}\sum_{i=1}^{N_k(t)}U_{k,i}(t)$.
Again, we assume single lanes for each direction, and thus vehicles from the same origin must follow the FIFO discipline; relaxing this assumption would involve en-route over-taking and lane-changing, which we leave for future work.
We are interested in the existence of $\overline X<\infty$ such that $\overline X=\lim_{t\to\infty}(1/t)\int_{s=6}^t\|X(s)\|_1ds$ a.s., which is defined as stability of the PDMP.

The rest of this section is devoted to detailed analysis of the three configurations in Fig.~\ref{fig:multi}.

%%%%%%%%%%%%%%%%%%%%%%%%%%%%%%%%%%%
\subsection{Four ODs with turning}

Consider the four-OD intersection in Fig.~\ref{fig:four_turning_config} with $\mathcal K=\{WE,WS,NS,NE\}$. 
Analogous to the two-OD system, we consider three typical policies, viz. {\bf FIFO}, {\bf MS}, and {\bf LQF}.  
FIFO can be formulated with the simplified state $X(t)$ and a discrete state $Y(t)\in\mathcal H=\mathcal K\cup\{0\}$, which tracks the class of the last vehicle entering the system before time $t$; $Y(t)=0$ means that the system is empty.
MS can be formulated with the simplified state $X(t)$ and a discrete state $Z(t)\in\mathcal K$, where $Z(t)$ tracks the class of the vehicle crossing the intersection at time $t$. A major difference from two-OD MS is that, since vehicles from the same origin must follow FIFO, the MS policy actually determines whether to discharge traffic in $\mathcal K_W=\{WE,WS\}$ or $\mathcal K_N=\{NS,NE\}$ rather than the traffic in a specific class.
LQF requires tracking the full state $(G(t),U(t))$.

This configuration can be viewed as a variation of the two-OD configuration with arrival rates $\bar\lambda_W=\lambda_{WE}+\lambda_{WS}$ and $\bar\lambda_N=\lambda_{NE}+\lambda_{NS}$ and crossing times
$$
\bar\theta_{\kappa_1,\kappa_2}=\sum_{k\in\mathcal K_{\kappa_1}}\sum_{\ell\in\mathcal K_{\kappa_2}}\frac{\lambda_k\lambda_\ell}{\lambda_{\kappa_1}\lambda_{\kappa_2}}\theta_{k,\ell},
\quad \kappa_1,\kappa_2\in\{W,N\}.
$$
Let
\begin{align}
    \bar\lambda=\begin{bmatrix}
    \bar\lambda_W\\
    \bar\lambda_N
    \end{bmatrix},
    \ 
    \bar\Theta=\begin{bmatrix}
    \bar\theta_{WW} & \bar\theta_{WN}\\
    \bar\theta_{NW} & \bar\theta_{NN}
    \end{bmatrix}.
    \label{eq_lambdabar}
\end{align}
The four-OD-with-turning configuration has a strong analogy to the two-OD configuration:

\begin{theorem}
\label{thm_four1}
The capacity and delay guarantees at a four-OD intersection with turning is equivalent to those given in Theorem~\ref{thm_two} under the respective policies, with $\lambda$ and $\Theta$ in Theorem~\ref{thm_two} replaced by $\bar\lambda$ and $\bar\Theta$ defined in \eqref{eq_lambdabar}.
\end{theorem}

For numerical examples in this section, we consider
\begin{align*}
    \Theta=\left[
    \begin{array}{cccc}
        0.5 & 0.5 & 1 & 1 \\
        0.75 & 0.75 & 1.25 & 1.25 \\
        1 & 1 & 0.5 & 0.5 \\
        1.25 & 1.25 & 0.75 & 0.75
    \end{array}
    \right],
\end{align*}
which has the unit of seconds.

\emph{Proof of Theorem~\ref{thm_four1}}.
It suffices to show that $\bar\lambda$ and $\bar\Theta$ contribute to the mean drift equivalently with $\lambda$ and $\Theta$ in Theorem~\ref{thm_two}. For FIFO, consider the Lyapunov function
$$
\bar V(y,x)=\frac12\|x\|_1^2+a_y\|x\|.
$$
Over the covering $O_m=\{x\in\mathbb R_{>0}:\|x\|_2^2<m\}$, $m=1,2,\ldots$, we have
\begin{align*}
    \mathscr A_m\bar V(y,x)=\Big(-1+\sum_{k\in\mathcal K}\lambda_k(\theta_{y,k}+\overline{R}+a_k-a_y)\Big)\|x\|_1+\mbox{constant.}
\end{align*}
By Lemma~\ref{lmm_a}, there exists $\{a_y;y\in\mathcal H\}$ such that
\begin{align*}
    \sum_{k\in\mathcal K}\lambda_k(\theta_{y,k}+\overline{R}+a_k-a_y)
\le\sum_{\kappa\in\{W,N\}}\bar\lambda_\kappa\sum_{\kappa'\in\{W,N\}}\frac{\bar\lambda_\kappa'}{\|\bar\lambda\|_1}\bar\theta_{\kappa',\kappa},
\quad y\in\mathcal H,
\end{align*}
where $\bar\lambda$ and $\bar\Theta$ are defined in \eqref{eq_lambdabar}.
The rest of the proof, including the arguments associated with MS and LQF, is analogous to Theorem~\ref{thm_two}.
\hfill$\square$

Simulation results.

Experiment results.

%%%%%%%%%%%%%%%%%%%%%%%%%%%%%%%%%%%
\subsection{Four OD without turning}

Consider the four-OD configuration in Fig.~\ref{fig:four_no_turning_config} with $\mathcal K=\{WE,EW,NS,SN\}$. 
Sequencing for this configuration is considerably different from (and more complex than) the two-OD one in the following aspects.
First, {FIFO} in this configuration does not have to be strict. Since opposite ODs (i.e., WE vs. EW and NS vs. SN) do not directly interfere, no FIFO constraint will be imposed between such ODs; see Fig.~\ref{}. We call this policy the {\bf relaxed FIFO} policy. We use $k\cap \ell=\emptyset$ (resp. $k\cap \ell\ne\emptyset$) to indicate that the orbits of classes $k$ and $\ell$ do not interfere (resp. do interfere); see Table~\ref{tab:interference}. 
\begin{table}[hbt]
    \centering
    \begin{tabular}{c|c c c c}
         & WE & NS & EW & SN  \\
         \hline
         WE & N/A & Yes & No & Yes \\
         NS & Yes & N/A & Yes & No \\
         EW & No & Yes & N/A & Yes \\
         SN & Yes & No & Yes & N/A 
    \end{tabular}
    \caption{Interference matrix for ODs in Fig.~\ref{fig:four_no_turning_config}. ``Yes'' (resp. ``No'') means that two ODs interfere (resp. do not interfere) with each other.}
    \label{tab:interference}
\end{table}
Second, {\bf MS} in this configuration switches the directions to discharge if both directions being discharged are cleared; see Fig.~\ref{}.
Third, LQF in this configuration will involve synchronization of sequences in four directions, which is very complicated and thus not discussed.

\begin{theorem}
\label{thm_four2}
The four-OD configuration without turning is stabilized by
\begin{enumerate}
    \item[\ref{thm_four2}.1] relaxed FIFO if and only if
    \begin{align}
        &\lambda_k\Bigg(\lambda_\ell\Big(\sum_{\substack{j=k \mbox{ or}\\j\cap k\ne\emptyset}}\frac{\lambda_j}{\sum_{i\in\mathcal K_k}\lambda_i}\theta_{j,k}+\overline{R}\Big)+\overline{R}\Bigg)\nonumber\\
        &
        +\sqrt{\sum_{\ell:\ell\cap k=\emptyset}\Big(\lambda_\ell(\sum_{\substack{j=\ell \mbox{ or}\\j\cap \ell\ne\emptyset}}\frac{\lambda_j}{\sum_{i\in\mathcal K_k}\lambda_i}\theta_{j,k}+\overline{R})\Big)^2}<1,
        \quad \forall k\in\mathcal K,
        \label{eq_fifo4wo}
    \end{align}
    
    \item[\ref{thm_four2}.2] MS if
    \begin{align}
        \lambda_k(\theta_{k,k}+\overline{R})+\sqrt{\sum_{\ell:\ell\cap k=\emptyset}\Big(\lambda_\ell(\theta_{\ell,\ell}+\overline{R})\Big)^2}<1,
        \quad \forall k\in\mathcal K.
    \end{align}
    
    % \item MD if
    % \begin{align}
    %     \lambda_k(\max_j\theta_{k,j}+\overline{R})+\sqrt{\sum_{\ell:\ell\cap k=\emptyset}\Big(\lambda_\ell(\max_j\theta_{\ell,j}+\overline{R})\Big)^2}<1,
    %     \quad \forall k\in\mathcal K.
    % \end{align}
\end{enumerate}
\end{theorem}

For numerical examples in this section, we consider
\begin{align*}
    \Theta=\left[
    \begin{array}{cccc}
        0.5 & 0.5 & 1 & 1 \\
        0.75 & 0.75 & 1.25 & 1.25 \\
        1 & 1 & 0.5 & 0.5 \\
        1.25 & 1.25 & 0.75 & 0.75
    \end{array}
    \right],
\end{align*}
which has the unit of seconds.

The proof of Theorem~\ref{thm_four2} in the FIFO case is analogous to that of Theorem~\ref{thm_two}.
The MS case is more sophisticated due to the lack of ``coupling'' between opposite ODs, which invalidates quadratic functions as in \eqref{eq_V2}. To resolve this challenge, we used a refined Lyapunov function as follows:

The root form in the above is motivated by the technique introduced in \cite{foley2001join}, which is invented for parallel queues. Specifically, this form essentially eliminates the impact of class-$k$ arrivals if $x_k=0$. The $\epsilon$ term fixes a technical gap between the Lyapunov function in \cite{foley2001join} and the peculiar behavior of our PDMP model.

\emph{Proof of Theorem~\ref{thm_four2}.1}.
Without loss of generality, suppose that $x_1>0$ and that a class-1 vehicle is crossing.
Consider the Lyapunov function
\begin{align}
    V_3(y,x)=&\frac12\Big(\sqrt{x_1^2+x_3^2+\epsilon(x_2^2+x_4^2)}+\sqrt{x_2^2+x_4^2+\epsilon(x_1^2+x_3^2)}\Big)^2\nonumber\\
    &+a_y\Big(\sqrt{x_1^2+x_3^2+\epsilon(x_2^2+x_4^2)}+\sqrt{x_2^2+x_4^2+\epsilon(x_1^2+x_3^2)}\Big),
\end{align}
where $\epsilon>0$ is a small number.
By analogy of the proof of \cite[Proposition ??]{foley2001join}, one can show that the mean drift is given by
\begin{align*}
    \mathscr LV_3=\Bigg(
    &\frac{(-1+\lambda_1(\theta_{y,1}+\overline{R}))x_1+\lambda_2(\theta_{y,2}+\overline{R})x_2+\epsilon\Big(\lambda_3(\theta_{y,3}+\overline{R})x_3+\lambda_4(\theta_{y,4}+\overline{R})x_4\Big)}{\sqrt{x_1^2+x_3^2+\epsilon(x_2^2+x_4^2)}}\\
    &+\frac{\epsilon\Big((-1+\lambda_1(\theta_{y,1}+\overline{R}))x_1+\lambda_2(\theta_{y,2}+\overline{R})x_2\Big)+\lambda_3(\theta_{y,3}+\overline{R})x_3+\lambda_4(\theta_{y,4}+\overline{R})x_4}{\sqrt{x_1^2+x_3^2+\epsilon(x_2^2+x_4^2)}}\\
    &+\sum_{y'\ne y}\lambda_{y'}(a_{y'}-a_y)+\frac{o(\|x\|_2)}{\|x\|_2}\Bigg)\Big(\sqrt{x_1^2+x_3^2+\epsilon(x_2^2+x_4^2)}+\sqrt{x_2^2+x_4^2+\epsilon(x_1^2+x_3^2)}\Big).
\end{align*}
By \eqref{eq_fifo4wo} and by Lemma~\ref{}, there exist $a_y$ such that
\begin{align*}
    (-1+\lambda_1(\theta_{y,1}+\overline{R})+)
\end{align*}
\hfill$\square$

Simulation 
and experiment results.

%%%%%%%%%%%%%%%%%%%%%%%%%%%%%%%%%%%
\subsection{General configuration}

\begin{theorem}
A twelve-OD intersection is stabilized by the relaxed FIFO policy
\end{theorem}