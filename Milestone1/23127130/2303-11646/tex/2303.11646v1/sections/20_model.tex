\section{Modeling and Formulation}
\label{sec_model}

In this section, we formulate the sequencing problem for the intersection shown in Fig.~\ref{actual}.
\begin{figure*}[hbt]
    \centering
    \subfigure[Actual setting.]{\includegraphics[width=0.3\textwidth]{images/actual2.png}
    \label{actual}}
    %
    \subfigure[PDMP setting.]{\includegraphics[width=0.2\textwidth]{images/virtual.png}
    \label{pdmp}}
    %
    \subfigure[PDMP state variables.]{\includegraphics[width=0.4\textwidth]{images/state.png}
    \label{fig:state}}
    %
    \caption{Modeling intersection as PDMP.}
    \label{fig:two_config}
\end{figure*}
We first specify the PDMP model. Then, we formulate the sequencing policies that we will study. Finally, we define the performance metrics for evaluation and comparison.

%%%%%%%%%%%%%%%%%%%%%%%%%%%%%%%%%%%%%%
\subsection{Piecewise-deterministic Markov model}

We consider random arrivals.
For the two-OD intersection in Fig.~\ref{actual}, let $\mathcal K=\{1,2\}$ be the set of OD pairs (or traffic classes). Class-$k$ vehicles arrive at the approaching zone as the Poisson process of rate $\lambda_k$.
If a class-$k$ vehicle arrives and sees no other vehicles (of either class), then the vehicle will traverse the approaching zone with a nominal traverse time $\tau_k$. Suppose that a class-$k$ vehicle arrives at the approaching zone at time $t_0$, which we call the \emph{actual arrival time}; then, we define $t_1=t_0+\tau_k$ as the \emph{virtual arrival time}, or simply \emph{arrival time}, of the vehicle. That is, if this vehicle were not to be influenced by other vehicles, it would arrive at the crossing zone at time $t_1$. Since the nominal traverse time is constant, if one observes at the crossing zone, the arrival process for class $k$ is also Poisson of rate $\lambda_k$. Hence, instead of the actual kinematics of incoming vehicles, we consider a PDMP formulation for the equivalent, virtual ``queuing process'', where the arrival times are equal to the above-mentioned virtual arrival times; see Fig.~\ref{pdmp}. In Section~\ref{sec_simulation}, we will discuss how sequencing policies formulated in the PDMP can be translated back to instructions implementable in the approaching zone.

The evolution of the PDMP is driven by the arrival and discharge of vehicles. When a class-$k$ vehicle arrives at the crossing zone at time $t$ and sees no other vehicle crossing, then it will finish crossing at time $t+R$, where $R$ is the \emph{crossing time} (with unit [sec]).
To account for vehicle heterogeneity, we assume that $R$ is a random variable (rv) with a cumulative distribution function $F_R(r)$ supported by a bounded interval $[R_{\min},R_{\max}]$; the value of $R$ becomes known to the system operator when a vehicle arrives. In practice, $R$ depends on vehicle size and crossing speed. Since $R$ is bounded, so are the mean $\bar R$, the variance $\sigma_R^2$, and the moment generating function (MGF) $g_R(\rho)$.
If an incoming vehicle sees other vehicles crossing or waiting for crossing, it will have to wait. Note that such queuing-like waiting is virtual in the PDMP; in practice, the waiting time will be absorbed over the approaching zone via trajectory planning; see Section~\ref{sub:two_sim}.
For any two vehicles crossing the intersection consecutively, the headway (with unit [sec]) in between must be no less than the \emph{minimal headway} $\theta_{ij}\ge0$, where $i$ (resp. $j$) are the class of the leading (resp. following) vehicle. 
Thus, we have a headway matrix $\Theta\in\mathbb R_{\ge0}^{2\times2}$. By practical insight, we assume that $\theta_{ij}>\theta_{ii}$ for $j\ne i$ and $i\in\mathcal K$; see Fig.~\ref{fig:two_theta}.
\begin{figure}[hbt]
    \centering
    \includegraphics[width=0.45\textwidth]{images/two_theta.png}
    \caption{Minimal headway depends on the sequence.}
    \label{fig:two_theta}
\end{figure}
Since the \emph{service times} $\theta_{ij}+R$ are not independent and identically distributed, classical queuing theory does not apply to our PDMP here, especially for the purpose of sequencing policy analysis; new tools, which this paper focuses on, are needed.

A complete state-space representation of the PDMP is as follows.
Let $N_k(t)\in\mathbb Z_{\ge0}$ be the number of class-$k$ vehicles waiting for discharge at time $t$, and let $N(t)=N_1(t)+N_2(t)$; see Fig.~\ref{fig:state}. We call $N_k(t)$ the \emph{class-$k$ count} and $N(t)$ the \emph{total count}. We use a tuple $(k,n)$ to label the $n$th class-$k$ vehicle. Vehicles will cross the intersection according to the \emph{sequence of crossing} $G(t)=[G_1(t),G_2(t),\ldots,G_{N(t)}]^T\in(\mathcal K\times\mathbb Z_{>0})^{N(t)}$, where $G_i(t)=(k,n)$ means that, observed at time $t$, vehicle $(k,n)$ will be the $i$th to cross the intersection if no more vehicles are joining the queues.
%
Let $U_{k,n}(t)$ be the \emph{residual service time} for vehicle $(k,n)$ defined as follows. If $(k,n)\ne G_1(t)$, i.e., if vehicle $(n,k)$ is not at the beginning of the sequence, then $U_{k,n}(t)$ is the service time for vehicle $(n,k)$. If $(k,n)=G_1(t)$, then vehicle $(k,n)$ is ``being served'' and will fully cross the intersection at time $t+U_{k,n}(t)$. Let $U(t)$ be the vector of $U_{k,n}(t)$ for $1\le n\le N_k(t)$ and $k\in\mathcal K$. Thus, the evolution of the PDMP can be fully described by the hybrid state $(G(t),U(t))$, where $G(t)$ is discrete and $U(t)$ is continuous. Since the vehicle arrivals are Poisson, if the sequencing policy is also Markovian (i.e., depending on $G(t)$ and $U(t)$ only), $\{(G(t),U(t));t\ge0\}$ is a piecewise-deterministic Markov process \cite{davis1984piecewise}.
% the state space is $\mathcal G\times\mathcal U$, where $\mathcal G=\cup_{n\in\mathbb Z_{\ge0}}(\mathcal K^n\times\mathbb Z_{>0}^n)$ is the discrete state space and $\mathcal U=\cup_{n\in\mathbb Z_{\ge0}}\mathbb R_{\ge0}^n$ is the continuous state space.

For ease of presentation, we also introduce a simplified representation.
Although $G(t)$ and $U(t)$ give a complete state-space representation, they have time-varying dimensions and are thus not easy to use directly.
To resolve this problem, we define a an \emph{aggregate state} $X(t)=[X_1(t),X_2(t)]^T$, where
$$
X_k(t)=\sum_{n=1}^{N_k(t)}U_{k,n}(t),
\quad k=1,2.
$$
If $N_k(t)=0$, we define the above to be zero.
One can interpret $X_k(t)$ as the ``aggregate service time'' or ``temporal queue size'' of class-$k$ traffic.

Finally, the dynamics for the hybrid state $(G(t),U(t))$ and thus the aggregate state $X(t)$ depends on the sequencing policy, which we discuss in the next subsection.

%%%%%%%%%%%%%%%%%%%%%%%%%%%%%%%%%%%%%%
\subsection{Sequencing policies}

The PDMP formulation is incomplete until a sequencing policy is specified. Intuitively, the sequencing policy defines how the hybrid state $(G(t),U(t))$ or the aggregate state $X(t)$ is reset when a vehicle arrives. For ease of presentation, we introduce sequencing policies in terms of the aggregate state $X(t)$.
Between vehicle arrivals, $X(t)$ evolves deterministically as follows:
\begin{align*}
    \frac d{dt}X_k(t)=\begin{cases}
    -1 & \mbox{$(k,1)$ is being discharged,}\\
    0 & \mbox{otherwise,}
    \end{cases}
    \quad k=1,2.
\end{align*}
Arrivals of vehicles will lead to sudden jumps in $X(t)$. The magnitude of such jumps depends on the sequencing policy.
We consider three typical sequencing policies:
\begin{enumerate}
    \item {\bf First-in-first-out (FIFO)} discharges vehicles according to the order of arrivals. To formulate this policy, we define an auxiliary discrete state $Y(t)\in\mathcal K$
    that tracks the class of the vehicle at the end of the sequence: we define $Y(t)=k$ if a class-$k$ vehicle is at the end of the sequence; if $N(t)=0$, then $Y(t)$ is the class of the last discharged vehicle. Thus, $\{Y(t);t\ge0\}$ is a two-state Markov process with transition rates $q_{12}=\lambda_1$ and $q_{21}=\lambda_2$. If a class-$k$ vehicle arrives at time $t$, the continuous state $X(t)$ is then updated as follows:
    \begin{align*}
        &X_k(t)=
        X_k(t_-)+\theta_{Y(t_-),k}+R,\\
        &X_{-k}(t)=X_{-k}(t_-),
    \end{align*}
    where ``$-k$'' means $j\in\mathcal K:j\ne k$ and ``$t_-$'' means left limits, for Poisson processes are right-continuous with left limits (RCLL) \cite{gallager2013stochastic}.
    % The discrete state $Y(t)$ evolves according to the mechanism in Fig.~\ref{fig:two_Yt}.
    % \begin{figure}[hbt]
    %     \centering
    %     \includegraphics[width=0.5\textwidth]{images/two_Yt.png}
    %     \caption{Transition mechanisms for discrete states $Y(t)$ (left) and $Z(t)$ (right). $\lambda_k$ indicates the rate of Poisson transitions, while equalities/inequalities indicate the condition for deterministic transitions.}
    %     \label{fig:two_Yt}
    % \end{figure}
    
    \item {\bf Min-switch (MS)} always clears traffic in one class before switching to the other class. To formulate this policy, we need an auxiliary discrete state $Z(t)\in\mathcal K$ that tracks the class being discharged: we define $Z(t)=k$ if a class-$k$ vehicle is crossing or was the last to cross the intersection. If a class-$k$ vehicle arrives at time $t$, the continuous state $X(t)$ is then updated as follows:
    \begin{align*}
        &X_k(t)=\begin{cases}
        X_k(t_-)+\theta_{k,k}+R,&X(t_-)>0,\\
        \theta_{-k,k}+R & \mbox{otherwise,}
        \end{cases}\\
        &X_{-k}(t)=X_{-k}(t_-).
    \end{align*}
    The discrete state $Z(t)$ is updated only when traffic in class $Z(t_-)$ has been cleared and when there is non-zero traffic in class $-Z(t_-)$ waiting for discharge.
    
    \item {\bf Longer-queue-first (LQF)} always discharges the class with a longer (in a generalized sense) queue. 
    This policy leads to dynamics more sophisticated than the other policies, so we will need the full state $(G(t),U(t))$.
    For ease of presentation, we use $Q(t)\in\mathcal H$ to track which OD has a longer ``queue'':
    \begin{align*}
        Q(t)=\begin{cases}
        1 & X_1(t)>\beta X_2(t),\\
        2 & X_1(t)<\beta X_1(t),\\
        0 & \mbox{otherwise,}
        \end{cases}
    \end{align*}
    where $\beta>0$ is a design parameter.
    Note that here we compare the ``temporal queue size'' $X_k(t)$ instead of the vehicle counts $N_k(t)$, since the former are more relevant for delay balancing.
    If a class-$k$ vehicle arrives at time $t$, the aggregate state $X(t)$ is then updated as follows:
    \begin{align*}
        &X_k(t)=\begin{cases}
        X_k(t_-)+\theta_{k,k}+R& Q(t_-)=k\mbox{ or }0,\\
        X_k(t_-)+\theta_{-k,k}+R & Q(t_-)=-k,
        \end{cases}\\
        &X_{-k}(t)=\begin{cases}
        X_{-k}(t_-) \hspace{1.85cm} Q(t_-)=k\mbox{ or }0,\\
        X_{-k}(t_-)+\theta_{-k,k}-\theta_{-k,-k}\\
        \hspace{3.18cm} Q(t_-)=-k,
        \end{cases}
    \end{align*}
    That is, the new arrival will be placed at the end of the sequence if it belongs to the longer queue. Otherwise, it will be inserted in the middle of the sequence before some vehicles in the longer queue; see Fig.~\ref{fig:two_etat}.
    \begin{figure}[hbt]
        \centering
        \includegraphics[width=0.45\textwidth]{images/two_Gt.png}
        \caption{LQF policy may sequence new arrivals (shaded) before existing vehicles to balance traffic from different directions.}
        \label{fig:two_etat}
    \end{figure}
    % Note that $(Q(t),X(t))$ is not a complete state-space representation and is used only as auxiliary states.
\end{enumerate}

Note that all auxiliary states in the above can be uniquely derived from the PDMP's complete state $(G(t),U(t))$; see Fig.~\ref{representation}.
%
\begin{figure}[hbt]
    \centering
    \includegraphics[width=0.3\textwidth]{images/representation.png}
    \caption{Two representations for PDMP.}
    \label{representation}
\end{figure}
%
Hence, given a sequencing policy, $\{(G(t),U(t));t\ge0\}$ is a PDMP. One can verify that it is RCLL.
Thus, the PDMP dynamics can be compactly characterized with the \emph{generator} $\mathscr A$ defined as follows.
Let $V:\mathcal G\times\mathcal U\to\mathbb R$ be a well-defined (in the sense of \cite[p.521]{meyn1993stability}) function.
Then, $\mathscr AV$ is essentially a measurable function such that for each initial condition $(g,u)\in\mathcal G\times\mathcal U$ and each time $t>0$,
\begin{align*}
    &\mathrm E\bigg[V\Big(G(t),U(t)\Big)\bigg|\Big(G(0),U(0)\Big)=(g,u)\bigg]=V(g,u)\\
    &+\mathrm E\bigg[\int_{s=0}^t\mathscr AV\Big(G(s),U(s)\Big)ds\bigg|\Big(G(0),U(0)\Big)=(g,u)\bigg],\\
    &\mathrm E\bigg[\int_{s=0}^t\mathscr AV\Big(G(s),U(s)\Big)ds\bigg|\Big(G(0),U(0)\Big)=(g,u)\bigg]<\infty.
\end{align*}
The term $\mathscr AV(g,u)$ is typically interpreted as the time derivative of $\mathrm E[V(G(t),U(t))]$ as $(G(t),U(t))=(g,u)$. Hence, $\mathscr AV(g,u)$ is called the \emph{mean drift} of $V$. Stability of the PDMP is closely related to behavior of the mean drift of appropriate Lyapunov functions \cite{meyn1993stability,benaim15,cloez15}.

% The PDMP abstracts practical traffic dynamics at two-OD intersections. In Sections~\ref{sub:two_sim} and \ref{sub:two_exp}, we will show how outputs of the PDMPs can be translated to instructions that can be implemented in practice.

%%%%%%%%%%%%%%%%%%%%%%%%%%%%%%%%%%%%%%
\subsection{Performance metrics}

To obtain analytical performance guarantees (delay and capacity) using the PDMP, we consider the long-time behavior of the aggregate state $X(t)$. We say that the PDMP is \emph{stable} if there exists $\overline{W}<\infty$ such that for any initial condition,
\begin{align}
    \limsup_{t\to\infty}\frac1t\int_{s=0}^t\mathrm E[\|X(s)\|_1]ds\le\overline{W},
    \label{eq_zeta}
\end{align}
where $\|\cdot\|_1$ is the 1-norm in Euclidean spaces.
Furthermore, if the above holds, there typically exists $\overline X<\infty$ such that for any initial condition
\begin{align}
    \lim_{t\to\infty}\frac1t\int_{s=0}^t \|X(s)\|_1ds=\overline X
\quad a.s.;
\label{eq_Xbar}
\end{align}
see Section~\ref{sub_fifo} for discussion on the existence of this limit.

Practically, $\overline X$ can be interpreted as the average \emph{delay} due to vehicle coordination. Computing the exact value of $\overline X$ is in general not easy, so we focus on obtaining an upper bound $\overline{W}$. Stability in the sense of \eqref{eq_zeta} means the delay is within an acceptable range, while instability means congestion will grow unboundedly. We will compare the delays resulting from various sequencing policies.

The \emph{capacity} of the intersection is essentially the maximal demand that can be accommodated in the sense that \eqref{eq_zeta} holds, i.e., that the PDMP is stable. Since the demand is a two-dimensional vector $\lambda=[\lambda_1,\lambda_2]^T$, we cannot use a scalar to characterize the intersection's capacity. Instead, we are interested in identifying the set of $\lambda$ such that the PDMP is stable, which is called the \emph{capacity region}. Note that various sequencing policies lead to different capacity regions, which enables us to compare policies in terms of capacity.

In particular, if we fix the \emph{distribution of demand}, then we can define capacity as a scalar as follows. 
The distribution of demand is characterized by the probability vector $p=[p_1,p_2]$.
With a fixed $p$, we can define capacity $\bar\lambda$ as follows:
\begin{align*}
    \bar\lambda=\max&\quad \|\lambda\|_1\\
    s.t.&\quad \lambda=p\|\lambda\|_1,\ \lambda\ge0,\\
    &\quad \mbox{stability condition holds.}
\end{align*}

The next section studies the analytical guarantees of the above performance metrics.