\section{Introduction}
\label{sec:introduction}

Signal-free (or unsignalized) intersections are a novel traffic management system that utilizes the recent development in connected and autonomous vehicle (CAV) technology to improve capacity and abate delay \cite{pourmehrab2017,guanetti2018control, bian2019cooperation, campisi2021development}. 
The key characteristic of such systems is that vehicles are discharged as discrete ``customers'' as opposed to traffic flows at conventional signalized intersections.
Hence, signal-free intersections are more flexible and thus potentially more efficient than signalized intersections \cite{lioris2017platoons,ZHANG2022103503,wang2021roadrunner}.

A typical signal-free intersection relies on a hierarchical decision-making mechanism as shown in Fig.~\ref{fig:intersection}  \cite{varaiya1993smart,zhang2015state,2018A}.
\begin{figure}[hbt!]
    \centering
    \includegraphics[width=0.49\textwidth]{images/hierarchy.png}
    \caption{A two-OD intersection for CAVs and the hierarchy for decision making. This paper focuses on vehicle sequencing.}
    \label{fig:intersection}
\end{figure}
In this hierarchy, an infrastructure-based decision maker (e.g., a road-side unit installed at the intersection) integrates kinematic information of incoming vehicles and allocates time windows for crossing (``vehicle sequencing'').
The decisions from vehicle sequencing specify the boundary conditions for vehicles to cooperatively plan their trajectories, typically in a fully or partially centralized manner (``trajectory planning'').
The planned trajectories will be implemented by autonomous driving capabilities installed onboard (``autonomous driving'').
Upper layers instruct lower layers, while lower layers constrain upper layers.
Extensive results have been developed recently for the planning layer \cite{kim2014mpc,ahn2017safety,xu2018distributed, li2018near,2018A,mirheli2019consensus,yan2022unified} and driving layer \cite{kong2015kinematic,besselink2017string,kiran2021deep,huang2019data,wabersich2021probabilistic,cichella2020optimal}. There also exists a body of work on the sequencing layer \cite{zhu2015,ZHANG2022103503,miculescu2019polling}, but, to the best of our knowledge, very limited results are available for macroscopic evaluation of various sequencing policies.
Specifically, the following two questions have not been well understood from a theoretical, quantitative perspective:
\begin{enumerate}
    \item How many vehicles on average can the intersection discharge per hour? (Capacity)
    \item How much delay on average is induced at the intersection? (Delay)
\end{enumerate}
These two questions are essential in that we need such macroscopic performance metrics (i) to quantify the benefits of deploying relevant technologies at intersections and (ii) to compare various sequencing algorithms.

%outline & contributions
In this paper, we study the above questions by considering the sequencing problem at signal-free intersections, i.e., determining the order and times (or time windows) for incoming vehicles to cross.
We model the approaching and crossing of vehicles as a
piecewise-deterministic Markov process (PDMP) \cite{davis1984piecewise}. 
We use the Foster-Lyapunov stability theory for Markov processes to quantify key performance metrics, viz. the capacity and delay, attained by typical sequencing policies.
In particular, we identify the set of demand patterns that a certain sequencing policy can accommodate, which characterizes intersection capacity, and estimate the upper bound for delay.
We also implement the PDMP-based policies and validate the results in a standard micro-simulation platform (Simulation of Urban Mobility, SUMO \cite{krajzewicz2010traffic}). 
% \begin{figure}[H]
%     \centering
%     \includegraphics[width=0.3\textwidth]{images/intersection.png}
%     \caption{A two-orbit intersection with CAVs.}
%     \label{fig:intersection}
% \end{figure}

Sequencing is a critical decision that significantly influences intersection efficiency \cite{xu2018distributed, li2018near}, and some related work is as follows. Linear or integer programming-based sequencing algorithms have been studied in \cite{zhu2015,fayazi2018}. Lioris et al \cite{lioris2017platoons} used a classical queuing model to estimate intersection capacity with the introduction of CAVs. Zhang et al \cite{ZHANG2022103503} simulated and evaluated various sequencing policies in typical scenarios. Miculescu and Karaman \cite{miculescu2019polling} proposed an online algorithm that provides guarantees on safety and efficiency under the first-in-first-out (FIFO) policy.
In addition, the rich body of work on adaptive signalized intersections \cite{dresner2004,li2017recasting} also provide insights for the signal-free setting.
Although the above work provides very useful hints for our problem, they do not directly address the question that we consider in this paper: 

\emph{How can we analytically relate various sequencing policies to key macroscopic performance metrics?}\\
Surprisingly, to the best our knowledge, the above question has not been well understood, possibly due to lack of tractable models. This gap prohibits quantification of the macroscopic benefits of signal-free intersections (with respect to conventional intersections) and comparison between various sequencing policies.

%modeling approach
In response to the above gap, we model the traffic at an intersection as a hybrid-state PDMP. 
Every vehicle belongs to a particular class that captures the origin-destination (OD) information.
This paper focuses on the two-OD configuration as shown in Fig.~\ref{fig:intersection}, but the models, methods, and results also provide insights for multi-OD configurations.
Vehicles enter the approaching zone (Fig.~\ref{fig:intersection}) as class-specific Poisson processes and the subsequent waiting and crossing behavior is deterministic.
We define the continuous (resp. discrete) state of the PDMP as the residual system times (resp. OD classes) of all vehicles currently in the system (i.e., in the approaching or the crossing zone).
The minimal time interval between two consecutive crossings depends on the ODs of the two vehicles. The crossing times are independent and identically distributed random variables with a finite variance.
% Auxiliary state variables, which are typically discrete and sequence-dependent, may be needed to formulate certain sequencing policies.
The resultant hybrid-state PDMP model is seemingly analogous to but fundamentally different from classical queuing models \cite{sundarapandian2009probability}: In the former, the sequencing decision affects the crossing times, while in the latter, the service times are independent of service sequence.

%analysis approach
We then use the PDMP model to study key performance metrics associated with three typical sequencing policies: first-in-first-out (FIFO), min-switchover (MS), and longer-queue-first (LQF).
FIFO is the baseline policy \cite{au2010motion}, while
MS and LQF largely resemble the passing group-based policy \cite{yan2012new} and the max-pressure policy \cite{varaiya2013max}, respectively.
The main result (Theorem~\ref{thm_two}) gives criteria for traffic queue stability under various sequencing policies and, if stable, upper bounds for travel delay.
The proof is based on the Foster-Lyapunov stability theory for Markov processes \cite{meyn1993stability,benaim15,cloez15}; this is a generic theory that is conceptually important but does not directly leads to practical results in our setting.
To address this gap, we develop sequencing policy-specific Lyapunov functions that captures behavior peculiar to each policy.
For FIFO, we consider a quadratic Lyapunov function with a switching first-order term.
For MS, we construct a decomposed process that is coupled with the original PDMP but easier to analyze.
Our stability criteria are sharp (i.e., ``if and only if'') for FIFO and MS.
For LQF, we utilize peculiar properties of crossing sequences under this policy to compute the mean drift of a quadratic Lyapunov function, which leads to a sufficient condition for stability.
PDMP-based approaches have been used to study macroscopic traffic control \cite{jin2018analysis,jin2018stability}, and this paper is, to the best of our knowledge, among the first that applies such approaches to intersection control.
Therefore, our techniques themselves also contribute to the theory of traffic control.

The main result directly responds to the questions posed at the beginning of this section. The stability criteria lead to closed-form characterization for intersection capacity regions in the demand space. Using this result, we show that the capacity is the lowest if traffic is evenly distributed over various ODs, which agrees with simulation-based analysis \cite{kamal2013}. In addition, the main result provides upper bounds on the travel delays associated with various policies.
Among the three typical policies, MS attains the highest capacity and the lowest travel delay; this finding is largely consistent with simulation analysis \cite{meng2017analysis,xu2021comparison}. LQF, although providing good fairness, leads to the worst capacity and travel delay.


Finally, we discuss how the PDMP-based decisions can be translated to practically implementable instructions for CAVs and validate the results via simulation-based experiments. In particular, we show that the theoretical capacity regions adequately characterizes the simulated boundary between free flow and congestion. We also show that the theoretical upper bounds for delay are valid for the simulated values.

%Contribution
The main contributions of this paper include:
\begin{enumerate}
    \item A PDMP model of signal-free intersection that can be used to quantitatively analyze capacity and travel delay under various sequencing policies.
    \item A Foster-Lyapunov drift approach that can be used to analyze the above mentioned PDMP model in regard to macroscopic properties.
    \item Analytical characterization of capacity regions associated with typical sequencing policies.
    \item Algorithms for implementing the PDMP-based decisions and insights for vehicle sequencing.
\end{enumerate}

%paper structure
The rest of this paper is organized as follows. Section \ref{sec_model} introduces the PDMP model for intersections and formulates the sequencing policies. Section \ref{sec_performance} analyzes the theoretical properties of the above policies. Section \ref{sec_simulation} validates the theoretical results with simulations on SUMO. Section \ref{sec_conclusion} gives the concluding remarks.

