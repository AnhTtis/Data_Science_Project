\section{Implementation and Validation}
\label{sec_simulation}

In this section, we discuss the implementation of the PDMP-based sequencing policies in practice and validate the theoretical results via simulation. Section \ref{sub:impl} introduces how we translate the PDMP-based control to the sequence and set time windows for incoming vehicles to cross the intersection. Section \ref{sub:two_sim} shows how to implement the set times of arrival in SUMO and discuss the simulation results. Table \ref{table:sim}  lists the variables involved in the implementation of the sequence policies.

\begin{table}[hbt]
\centering
\begin{tabular}{cl}
\hline
Notation & Variable [unit]\\
\hline
$\lambda_k$ & arrival rate [veh/sec]\\
\hline
$\theta_{ij}$ & minimal headway [sec]\\
\hline
$\Theta $& headway matrix [sec]\\
% \hline
% \makecell[c]{$N_k(t)$\\}
% & \makecell[c]{the number of class-$k$ vehicles waiting \\ for discharge  at time $t$ / class-$k$ count [veh]}\\
\hline
$N(t)$ & total count [veh]\\
\hline
$\delta$ & simulation time step size [sec]\\
\hline
$G(t)$ & sequence of crossing [($k_i,n_i$)]\\
\hline
$T^{i}_{\mathrm{ms}}$ &minimal set time for the $i$th vehicle to cross [sec] \\
\hline
$T^{i}_{\mathrm{min}}$ & minimal time for the $i$th vehicle to cross [sec]\\
\hline
$T^{i}_{\mathrm{set}}$ & set time for the $i$th vehicle to cross [sec]\\
\hline
$T^{i}_{\mathrm{e}}$ & time of the $i$th vehicle arrival [sec]\\
\hline
$t$ & current time in simulation [sec]\\
\hline
$L$ & length of approaching zone [m]\\
\hline
$v_{\mathrm{max}}$ & nominal/maximum speed [m/sec]\\
\hline
$v^i$ & speed for the $i$th vehicle [m/sec]\\
\hline
$a_+$ &  acceleration [m/se$\mbox{c}^2$]\\
\hline
$a_-$ & deceleration  [m/se$\mbox{c}^2$]\\
\hline
\end{tabular}
\caption{Notations used in simulation.}
\label{table:sim} 
\end{table}

%%%%%%%%%%%%%%%%%%%%%%%%%%%%%%%%%%%%%%%%%%%%%%%%%
\subsection{Implementation of PDMP-based policies }
\label{sub:impl}
We control the vehicles by setting their times to cross; that is, all vehicles will cross the intersection at their set times. We assume that all vehicles arrive with the same speed $v_{\mathrm{max}}$. The minimal set time for the $i$th vehicle (of all vehicles) to cross can be calculated by  
\begin{equation*}
    T_{\mathrm{ms}}^{G_i}=\frac{L}{v_{\mathrm{max}}}+T_{\mathrm{e}}^{G_i}.
\end{equation*}
Note that $G_i=(k_i,n_i)$, i.e. the $n$th class $k$ vehicle. The set time for each vehicle to cross is decided by the sequence of crossing $G(t)$
\begin{subequations}
    \begin{align}
        T_{\mathrm{set}}^{G_1}&=T_{\mathrm{ms}}^{G_1}, \label{eq:tset1} \\
        T_{\mathrm{set}}^{G_i}&=max(T_{\mathrm{ms}}^{G_i},T_{\mathrm{set}}^{G_{i-1}}+\theta_{k_ik_{i-1}}+\overline{R}) \quad i=2,3... \label{eq:tset2}
    \end{align}
 \end{subequations}
\subsubsection{FIFO}
For the FIFO policy, the sequence of vehicles to cross is the sequence of arrival. Once the sequence of crossing is decided, we can easily calculate the set times by \eqref{eq:tset1}-\eqref{eq:tset2}. Algorithm $\ref{algo:FIFO}$ is the pseudo-code to calculate the crossing time windows for vehicles.

\begin{algorithm}[hbt]
	\caption{Crossing time windows of FIFO} 
        \label{algo:FIFO}
	\LinesNumbered  
	\KwIn{ \emph{$\Theta$}, \emph{$\overline{R}$}}
	\KwOut{\emph{G}, \emph{$t$}, \emph{$T_{\mathrm{set}}$}} 
        $t \leftarrow 0$\\
        $G \leftarrow $  empty array\\
        $T_{\mathrm{ms}} \leftarrow $ empty array\\
        \While {Simulation is running}{
        $t \leftarrow t+1$\\
        \If{new vehicle $(k_i,n_i)$ arrives}{
           $ G_i\leftarrow (k_i,n_i)$ \\
            \eIf{$i=0$}{
            $T_{\mathrm{set}}^{G_i}\leftarrow T_{\mathrm{ms}}^{G_i}$ \\
            }{
            $T_{\mathrm{set}}^{G_i} \leftarrow max(T_{\mathrm{ms}}^{G_i},T_{\mathrm{set}}^{G_{i-1}}+\theta_{k_ik_{i-1}}+\overline{R})$ \\
            }
       
        }
         \textbf{Return} $G$, $t$, $T_{\mathrm{set}}$
        }

        

        

\end{algorithm}


\subsubsection{MS}
For the MS policy, when a new vehicle (labeled as $(k_i,n_i)$) arrives, we need to first find $(k_i,n_{i-1})$.  We assume $G_f=(k_i,n_{i-1})$; that is, $(k_i,n_{i-1})$ is the $f$th vehicle in the sequence of crossing.
\begin{enumerate}
    \item If $T_{\mathrm{ms}}^{G_f} \leq T_{\mathrm{set}}^{G_f}+\theta_{11}+\overline{R}$, then we shift every vehicle following $G_f$ to the next position in the sequence, and let $G_{f+1}=(k_i,n_i)$.
    \item If $T_{\mathrm{ms}}^{G_f} > T_{\mathrm{set}}^{G_f}+\theta_{11}+\overline{R}$, then let $j=f$. We check if $T_{\mathrm{ms}}^{G_f} \leq T_{\mathrm{set}}^{G_{j+1}]} +\theta_{12}+\overline{R}$ and 2*$(\theta_{12}+\overline{R}) \leq T_{\mathrm{set}}^{G_{j+1}}-T_{\mathrm{set}}^{G_j} $. Then we shift every vehicle following $G_j$ to the next position in the sequence, and let $G_{j+1}=(k_i,n_i)$, or else we let $j=j+1$ and repeat this until the  vehicle is inserted into the queue.
\end{enumerate}
 Once the sequence to cross is decided, we can compute $T_{\mathrm{set}}$ using \eqref{eq:tset1}-\eqref{eq:tset2}. The algorithm is analogous to Algorithm $\ref{algo:FIFO}$.


\subsubsection{LQF}
For the LQF policy, each time there is a new vehicle enters or leaves the approaching zone, we will update $G$: we keep shifting vehicles from the longer class to $G$, until current releasing class is shorter than the other class and then we will repeat until all vehicles in the approaching zone is in $G$. In case of a tie, we maintain the class of discharge: this is preferable, since the number of switch-overs is reduced. Once the sequence to cross is decided, we can compute $T_{\mathrm{set}}$ using \eqref{eq:tset1}-\eqref{eq:tset2}. The algorithm is analogous to Algorithm $\ref{algo:FIFO}$.



%%%%%%%%%%%%%%%%%%%%%%%%%%%%%%%%%%%%%%%%%%%%%%%%%
\subsection{Simulation}
\label{sub:two_sim}

We now validate the theoretical results by simulating the sequencing policies at the smart intersection. We applied various sequencing policies in Simulation of Urban Mobility (SUMO) \cite{krajzewicz2010traffic}; see
Fig. \ref{fig:simulation_SUMO}. 
\begin{figure}[hbt]
  \centering
  \includegraphics[width=0.5\textwidth]{images/intersection_SUMO_c.png}
  \caption{Simulation intersection in SUMO.}
  \label{fig:simulation_SUMO}
\end{figure}
The intersection consists of two directions, west-east (WE) and south-north (SN); each direction has an incoming vehicle flow. Vehicles are generated at random times with a minimal inter-arrival time of $\overline{R}$. The simulation step size is 0.1 sec, and a discrete-time Bernoulli process is used to approximate the continuous-time Poisson process considered in the theoretical analysis.



\subsubsection{Trajectory planning}
\begin{figure*}[htbp]
 \centering
 \subfigure[FIFO.]{
 \label{fig:subfig:a} 
 \includegraphics[width=2in]{images/FCFS_D.png}}
%  \hspace{0.01in}
 %
 \subfigure[MS.]{
 \label{fig:subfig:c} 
 \includegraphics[width=2in]{images/MSO_D.png}}
%  \hspace{0.01in}
 %
 \subfigure[LQF.]{
 \label{fig:subfig:b} 
 \includegraphics[width=2in]{images/LQF_D.png}}
%  \hspace{0.01in}
 %
 \caption{Spatio-temporal trajectories of vehicles under various sequencing policies.}

 \label{fig:SUMO_distance_time} 
\end{figure*}

\begin{figure*}[htbp]
 \centering
 \subfigure[FIFO.]{
 \label{fig:subfig:a} 
 \includegraphics[width=1.9in]{images/FIFO_temp.png}}
%  \hspace{0.01in}
 %
 \subfigure[MS.]{
 \label{fig:subfig:c} 
 \includegraphics[width=1.9in]{images/MSO_temp.png}}
%  \hspace{0.01in}
 %
 \subfigure[LQF.]{
 \label{fig:subfig:b} 
 \includegraphics[width=2.3in]{images/LQF_temp.png}}
%  \hspace{0.01in}
 %

 \caption{Heat maps of average delay for different sequencing policies; the color bar applies to all three figures. The intersection is considered to be congested if the average delay attains 10 sec/veh. The white curves indicate the theoretical boundaries of the capacity regions.}

 \label{fig:SUMO_results} 
\end{figure*}

Using the sequence and set times obtained in the previous subsection, we consider a simple trajectory planning scheme such that the designated delay will be absorbed over the approaching zone. This scheme ensures higher crossing speed than holding vehicles at the intersection and is thus preferred: almost all vehicles will cross the intersection at the nominal speed, and the capacity is fully utilized.  After $G$ is decided, we calculate the set time to cross $T_{\mathrm{set}}$ and generate the time series of vehicle speed until crossing. For ease of presentation, we assume uniform acceleration/deceleration of vehicles; note that one can indeed replace such simple scheme with more sophisticated trajectory-planning schemes in the literature (e.g. \cite{ahn2017safety,2018A}). $T_{\mathrm{min}}^{i}$ is the minimal time to go (subject to safety constraints) for the $i$th vehicle to reach the intersection. $T_{\mathrm{min}}$ depends on whether the vehicle can attain the nominal speed before it reaches the intersection. $T_{\mathrm{min}}$ is calculated by the following formula:
\begin{equation*}
    T_{\mathrm{min}}^i=
    \begin{cases}
        \frac{(-v^i+\sqrt{(v^i)^2+2a_+d^i})}{a_+}+t \\
        \hspace{2.5cm} \mbox{if }\frac{v_{\mathrm{max}}^2-(v^i)^2}{2a_+}>d^i,\\
        \frac{(v_{\mathrm{max}}-v^i)}{a_+}+\frac{(d^i-\frac{v_{\mathrm{max}}^2-(v^i)^2}{2a_+})}{v_{\mathrm{max}}}+t \\ 
        \hspace{2.5cm} \mbox{if }\frac{v_{\mathrm{max}}^2-(v^i)^2}{2a_+}\leq d^i.
    \end{cases}
\end{equation*}
For the above formula, the first case is that the vehicle reaches the intersection when it is still accelerating. The second case is that the vehicle reaches the intersection with the speed $v_{\mathrm{max}}$.
The speed of the vehicle is bounded by $[0,v_{\mathrm{max}}]$, where $v_{\mathrm{max}}$ is the nominal speed. 
For each vehicle, if $T_{\mathrm{min}}^i<T_{\mathrm{set}}^i$, the vehicle needs to decelerate in order to avoid interference with the previous vehicle. The vehicle will decelerate at a constant acceleration $a_-$; if $T_{\mathrm{min}}^i>T_{\mathrm{set}}^i$, the vehicle cannot enter the crossing zone at the set time and we will let $T_{\mathrm{min}}^i=T_{\mathrm{set}}^i$; if $T_{\mathrm{min}}^i=T_{\mathrm{set}}^i$, the vehicle can travel through the entire approaching zone without deceleration. It will accelerate at a constant acceleration $a_+$ until it maintains $v_{\mathrm{max}}$ and then travels at uniform speed. Therefore, the acceleration of vehicle $i$ is calculated by:
\begin{equation*}
    a^i=\left\{
    \begin{aligned}
        a_+ \quad & \mbox{if }v^i<v_{\mathrm{max}},T_{\mathrm{min}}^i\geq T_{\mathrm{set}}^i ,\\
        a_- \quad & \mbox{if }v^i>v_{\mathrm{min}},T_{\mathrm{min}}^i < T_{\mathrm{set}}^i,\\
        0 \quad & \mbox{otherwise} .
    \end{aligned}
    \right
    .
\end{equation*}

We use a simple collision detection function to ensure the distance between vehicles is no less than the safety gap.
Vehicles that have traversed the crossing zone will travel uniformly at $v_{\mathrm{max}}$ and be deleted from $G$.



\subsubsection{Results and discussion}
We use average delay of vehicles in each simulation to evaluate the three sequencing policies. The delay experienced by a vehicle is the difference between the actual driving time and the theoretical driving time. The theoretical driving time is assuming there is no other vehicles and the vehicle travels uniformly with the maximum speed.

Fig. \ref{fig:SUMO_distance_time} shows the spatio-temporal trajectories of 10 vehicles under various sequencing policies. MS changes most vehicles' sequence of reaching the crossing zone, LQF changes few vehicles' order of reaching the crossing zone and FIFO does not change the vehicles' order of reaching the crossing zone. As expected, the MS policy leads to the minimal average delay (3.22 sec/veh), since it minimizes the number of switch-overs. Meanwhile, the LQF policy tends to alternate the direction to discharge and thus leads to the maximal average delay (4.79 sec/veh).


Fig. \ref{fig:SUMO_results} shows the average delay under various sequencing policies as well as under various demand patterns. In general, average delay increases with demand and rapidly blows up as the demand approaches a certain threshold. The thresholds are policy-dependent and largely consistent with the boundary of the theoretical capacity region given by Theorem~\ref{thm_two}. Specifically, for smaller demands, the impact of sequencing policy is less significant. As demand increases, the delay under LQF quickly rises and attains the congestion (i.e. red) domain. MS leads to the minimal delay and the maximal capacity, while FIFO gives intermediate performance. The simulation results are compliant with our theoretical analysis in Section \ref{sec_performance} that the intersection has the largest capacity under MS policy and the smallest capacity under LQF policy. 

