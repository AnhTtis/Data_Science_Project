\section{Concluding remarks}
\label{sec_conclusion}

In this paper, we formulate the sequencing of vehicles at a signal-free intersection as a piecewise-deterministic Markov process. Our model captures the characteristics of typical sequencing policies and produces analytical guarantees on macroscopic performance metrics including capacity and delay. We use the Foster-Lyapunov stability theory to analyze the boundedness of traffic state and to obtain closed-form bounds for travel delay under various policies. In particular, we show that the min-switchover (resp. longer-queue-first) policy attains the best (resp. worst) system-wide performance. We also develop algorithms that implement various sequencing policies in practical settings and validate the theoretical results via micro-simulation-based experiments. This work provides useful tools and insights for intersection control. Possible future directions include extension to multi-origin-destination configurations and integration of learning-based methods to obtain adaptivity.

%%%%%%%%%%%%%%%%%%%%%%%%%%%%%%%
\section*{Acknowledgments}
The authors appreciate the discussion with Prof. Z.-P. Jiang and Prof. J. Chao at NYU.
The authors also appreciate the undergraduate students that helped with this work: H. Dai and Q. Dai contributed to the theoretical results, T. Yao, Y. Yao, J. Lin contributed to simulation.