%%%%%%%%%%%%%%%%%%%%%%%%%%%%
\section{Analytical performance guarantees}
\label{sec_performance}

In this section, we derive analytical guarantees on performance metrics under various sequencing policies.
Recall questions 1 and 2 in Section~\ref{sec:introduction}, which were posed in the practical setting. These questions can be mapped to the following questions in the PDMP setting:
\begin{enumerate}
    \item When does the upper bound in \eqref{eq_zeta} exist? (Capacity analysis.)
    \item If an upper bound exists, how can we estimate it? (Delay analysis.)
\end{enumerate}
This section is devoted to answering the above questions and deriving practical insights.

To state the main result, let $\|\cdot\|_p$ be the $p$-norm and $J_{m,n}$ be the $m\times n$ matrix of ones. Let $\mathrm{diag}(\cdot)$ be the row vector of the diagonal components of a matrix. Let $\circ$ be the Hadamard product operator. Let $\mathrm{det}(\cdot)$ be the determinant of a matrix. Define
{
\begin{subequations}
\begin{align}
    &
    A(\lambda)=\nonumber\\
    &\begin{bmatrix}
    $$
    \frac{\lambda_1(\theta_{11}-\theta_{21})+\lambda_2(\theta_{12}-\theta_{22})}{2(\lambda_1+\lambda_2)}
    &
    \frac{\lambda_1(\theta_{11}-\theta_{21})+\lambda_2(\theta_{12}-\theta_{22})}{2(\lambda_1+\lambda_2)}
    \\
    \frac{\lambda_1(\theta_{21}-\theta_{11})+\lambda_2(\theta_{22}-\theta_{12})}{2(\lambda_1+\lambda_2)}
    &
    \frac{\lambda_1(\theta_{21}-\theta_{11})+\lambda_2(\theta_{22}-\theta_{12})}{2(\lambda_1+\lambda_2)}
    $$
    \end{bmatrix},
    \\
    %
    &B(\lambda)=[b_{ij}(\lambda)]\nonumber\\
    &{\scriptsize\begin{bmatrix}
$$
(\theta_{12}+\theta_{21}+\overline{R}+R_{\max})\lambda_1
&
(\theta_{11}+\overline{R})\lambda_1+(\theta_{21}-\theta_{11})\lambda_2-1
\\
(\theta_{12}-\theta_{22})\lambda_1+(\theta_{22}+\overline{R})\lambda_2-1
&
(\theta_{12}+\theta_{21}+\overline{R}+R_{\max})\lambda_1
$$
\end{bmatrix}}.
\label{eq_B}
% \\
% &\gamma(\lambda)=\frac{\max\Big\{b_{11}(\lambda)+\beta b_{12}(\lambda),\beta b_{21}(\lambda)+b_{22}(\lambda)\Big\}}{\sqrt{1+\beta^2}}
\end{align}
\end{subequations}
}%
Then, we state the main result of this paper as follows.

\begin{theorem}
\label{thm_two}
Suppose an intersection with arrival rates $\lambda\in\mathbb R_{\ge0}^2$ and headway matrix $\Theta$. Let $\bar R$ and $\sigma_R^2$ be the mean and variance of the crossing time.
\begin{enumerate}
    \item[(i)] FIFO policy is stabilizing if and only if
    \begin{align}
        \Big(\frac{\lambda^T}{\|\lambda\|_1}\Theta+\bar RJ_{1,2}\Big)\lambda<1.
        \label{eq_fifo}
    \end{align}
    Furthermore, if \eqref{eq_fifo} holds, then
    \begin{align}
    \overline W_0
    \le
        \overline X
        \le \overline W_1,
        \label{eq_Wbar1}
    \end{align}
    where
    {\footnotesize
    \begin{align*}
        &\overline W_0=\mathrm{diag}(\Theta)\frac{\lambda}{\|\lambda\|_1}+\overline R\\
        &\quad+\frac{\Big(\mathrm{diag}(\Theta\circ\Theta)+\mathrm{diag}(\Theta)\overline{R}\Big)\frac{\lambda}{\|\lambda\|_1}+\overline R^2+\sigma_R^2}{\frac2{\|\lambda\|_1}-2\|\lambda\|_1\Big(\mathrm{diag}(\Theta)\frac{\lambda}{\|\lambda\|_1}+\overline R\Big)},
        \\
        &\overline W_1\\
        &=\frac{\Big\|\Big((\Theta+\overline RJ_{2,2})\circ(\frac12(\Theta+\overline RJ_{2,2})+A(\lambda))+\frac12\sigma_R^2J_{1,2}\Big)\lambda\Big\|_\infty}{1-\Big(\frac{\lambda^T}{\|\lambda\|_1}\Theta+\overline RJ_{1,2}\Big)\lambda}.
    \end{align*}}
    
    \item[(ii)] MS policy is stabilizing if and only if
    \begin{align}
        \mathrm{diag}(\Theta+\bar RJ_{2,2})\lambda<1.
        \label{eq_ms}
    \end{align}
    Furthermore, if \eqref{eq_ms} holds, then
    \begin{align}
    \overline W_0
    \le
        \overline X\le \overline W_2,
        \label{eq_Wbar2}
    \end{align}
    where 
    {\footnotesize\begin{align*}
    \overline W_2 =&\frac{\Big\|\Big((\Theta+\overline RJ_{2,2})\circ(\Theta+\overline RJ_{2,2})+\sigma_R^2J_{1,2}\Big)\lambda\Big\|_\infty}{2-2\mathrm{diag}(\Theta+\overline RJ_{2,2})\lambda}\\
    &+\frac{\lambda_1\lambda_2}{\lambda_1+\lambda_2}\sum_{k=1}^2(\theta_{-k,k}-\theta_{k,k}).
    \end{align*}}
    
    \item[(iii)] LQF policy is stabilizing if
    \begin{align}
        % &(\theta_{12}+\theta_{21}+\overline{R}+R_{\max})^2\lambda_1\lambda_2\nonumber
        % \\
        % &<\Big(1-(\theta_{11}+\overline{R})\lambda_1-(\theta_{21}-\theta_{11})\lambda_2\Big)\Big(1-(\theta_{12}\nonumber
        % \\
        % &\quad-\theta_{22})\lambda_1-(\theta_{22}+\overline{R})\lambda_2\Big)
        \mathrm{det}\Big(B(\lambda)\Big)<0
        \label{eq_lqf}
    \end{align}
    and if the comparing parameter $\beta$ satisfies
    \begin{align}
        \frac{b_{11}(\lambda)}{-b_{21}(\lambda)}
        <\beta
        <\frac{-b_{12}(\lambda)}{b_{22}(\lambda)}.
        \label{eq_beta}
    \end{align}
    Furthermore, if \eqref{eq_lqf}--\eqref{eq_beta} hold, then
    \begin{align}
        \overline W_0
    \le
    \overline X\le \overline W_3,
    \label{eq_Wbar3}
    \end{align}
    where
    {\footnotesize\begin{align*}
    \overline W_3 =\sqrt{\frac{(1+\beta^2)}{2}}\frac{\max_k\{b_{kk}^2(\lambda)+(b_{k,-k}(\lambda)+1)^2\}+\sigma_R^2\|\lambda\|_1}{-\max_k\{b_{kk}(\lambda)+\beta b_{k,-k}(\lambda)\}}.
    \end{align*}}
\end{enumerate}
\end{theorem}

The above theorem directly responds to the questions posed at the beginning of this section. 
The stability criteria for FIFO and MS are necessary and sufficient, while that for LQF is sufficient. Hence, we can compute the exact capacity under FIFO and MS and estimate a lower bound for capacity under LQF.
The delay lower bound $\overline{W}_0$ is common for all policies, while the upper bounds are policy-specific.

The rest of this section is devoted to an example-based interpretation (Section~\ref{sub_numerical}) and proof (Section~\ref{sub_proof}) of the theorem.

%%%%%%%%%%%%%%%%%%%%%%%%%%%%%%%%%%%%%%%%%%%%%%%%%%
\subsection{Numerical example}
\label{sub_numerical}
A major use of Theorem~\ref{thm_two} is to compute the capacity regions under various policies. 
Fig.~\ref{fig:two_regimes}
\begin{figure}[hbt]
    \centering
    \includegraphics[width=0.5\textwidth]{images/two_regimes.png}
    \caption{Stability regimes under various policies. MS stabilizes the white, light gray, dark gray regimes, FIFO stabilizes the white and light gray regimes, and LQF stabilizes the white regime only.}
    \label{fig:two_regimes}
\end{figure}%
shows the stability regimes associated with parameters
\begin{align*}
    {\Theta} = 
\begin{bmatrix}
$$
0.5 & 1\\
1 & 0.5
$$
\end{bmatrix} \mbox{[sec]},\ 
\bar R=0.5\ \mbox{[sec]},\ 
\sigma_R^2=0.1.
\end{align*}
That is, it takes every vehicle 0.5 sec on average to go through the crossing zone, and the inter-vehicle headway should be no less than 0.5 or 1 sec, depending on the crossing sequence.
For LQF, we assume that $\beta=1$ and that class 1 is prioritized in case of a tie.
%

As Fig.~\ref{fig:two_regimes} shows, MS (resp. LQF) leads to the largest (resp. smallest) stable regime,  and thus the largest capacity region. Therefore, MS gives higher capacity while LQF gives lower capacity.
The reason is that switching over the direction of crossing requires additional time, and MS minimizes the chance of switchovers. LQF, on the contrary, maximizes the chance of switchovers, for this policy tends to discharge various ODs in an alternate manner so that two traffic queues are balanced. FIFO is to some extent in the middle of MS and LQF.

An important insight from Theorem~\ref{thm_two} is that the capacity of an intersection depends on both the distribution of demand and the sequencing policy. 
The capacity depends on the distribution of demand via $p$ and on the sequencing policy via the stability condition \eqref{eq_fifo} or \eqref{eq_ms} or \eqref{eq_lqf}. As Fig.~\ref{fig:two_lambdabar} shows,
%
\begin{figure}[hbt]
    \centering
    \includegraphics[width=0.4\textwidth]{images/two_lambdabar.png}
    \caption{MS maximizes capacity for any demand distribution.}
    \label{fig:two_lambdabar}
\end{figure}
%
if the demand distribution is balanced (i.e., $p_1\approx p_2$), the differences between the policy-specific capacities are large; if the demand distribution is highly imbalanced (i.e., $p_1\approx1$ or $p_2\approx1$), then such differences are minimal. The reason is that imbalanced demand makes switchovers less frequent, and thus the impact of sequencing policy is less significant.

In conclusion, as far as capacity is concerned,
Theorem~\ref{thm_two} implies the following:
\begin{enumerate}
    \item MS maximizes capacity in that given the headway matrix $\Theta$, no sequencing policy can attain a higher capacity than MS does.
    
    \item FIFO performs as well as MS if $p_1\approx1$ or $p_2\approx1$ but not so well if $p_1\approx p_2$, since $p_1\approx1$ or $p_2\approx1$ means low chance of alternate sequences and thus fewer switchovers, while $p_1\approx p_2$ means high chance of alternate sequences and thus more switchovers.
    
    \item LQF in general gives the lowest capacity, since this policy is not intended to maximize discharge rate. Instead, this policy focuses on balancing traffic with various ODs.
\end{enumerate}

When the intersection is stabilized, we can also use the bounds provided in Theorem~\ref{thm_two} to estimate the average delay. Fig.~\ref{fig:two_Wbar} shows that MS gives not only the maximal capacity but also the minimal delay. In fact, the delay under MS ($\overline{W}_2$ in Fig.~\ref{fig:two_Wbar}) is very close to the theoretical lower bound $\overline{W}_0$, which results from an over-optimistic estimation (see Section~\ref{sub_fifo}).
%
\begin{figure}[hbt]
    \centering
    \includegraphics[width=0.45\textwidth]{images/Wbar2}
    \caption{Simulated average delay $\overline X$ in SUMO and theoretical bounds given by Theorem~\ref{thm_two}. Symmetric demand pattern (i.e., $p_1=p_2=0.5$) is assumed.}
    \label{fig:two_Wbar}
\end{figure}
%
LQF leads to both minimal capacity and maximal delay.

We now use a specific sample path to further explain the above insights. Suppose 8 vehicles already in the system with a given arrival sequence (Fig.~\ref{fig:sequence}).
%
\begin{figure}[hbt]
    \centering
    \includegraphics[width=0.25\textwidth]{images/sequence.png}
    \caption{Sequences of arrival/crossing for a given sample path. Vehicle 1 (resp. 8) is the first (resp. last) arrival.}
    \label{fig:sequence}
\end{figure}
%
We assume that no more vehicle will arrive and focus on the times for each policy to discharge these 8 vehicles. Fig.~\ref{fig:sequence} shows the sequences of crossing under each policy. Because of the switchover time, more turns mean more delay. MS minimizes the number of turns, while LQF creates the most switchovers. The total discharge times are listed in Table~\ref{tab:discharge}.
\begin{table}[hbt]
    \centering
    \begin{tabular}{|c|c c c|}
        \hline
        System time for & FIFO & MS & LQF \\
        \hline
        Last vehicle (i.e., 8) & 9 sec & 8 sec & 11 sec \\
        \hline
        First class-1 vehicle (i.e., 1) & 0.5 sec & 0.5 sec & 0.5 sec \\
        \hline
        First class-2 vehicle (i.e., 4) & 4 sec & 5 sec & 2 sec \\
        \hline
    \end{tabular}
    \caption{System times for sequence in Fig.~\ref{fig:sequence}.}
    \label{tab:discharge}
\end{table}
One can interpret the comparison between MS and LQF as the trade-off between system-wide efficiency and fairness. MS may hold some vehicles for a long time and result in large variance in delay. LQF, however, tends to evenly distribute delay over vehicles. For example, vehicle 1, which is in the front of the class-1 queue, has to wait for 0 sec under all policies, while vehicle 4, which is in the front of the class-2 queue, has to wait for 4 sec, 5 sec, and 2 sec under FIFO, MS, and LQF, respectively.

%%%%%%%%%%%%%%%%%%%%%%%%%%%%%%%%%%%%%%%%%%%%%%%%%%
\subsection{Proof of Theorem~\ref{thm_two}}
\label{sub_proof}

% We construct policy-specific Lyapunov functions that verify the comparison theorem for Markov processes \cite[Theorem 4.3]{meyn1993stability}.

\subsubsection{FIFO}
\label{sub_fifo}
This proof consists of three parts: (i) sufficiency of \eqref{eq_fifo} and the upper bound $\overline{W}_1$, (ii) necessity of \eqref{eq_fifo}, and (iii) the lower bound $\overline{W}_0$.

To show {\bf sufficiency} of \eqref{eq_fifo} and the {\bf upper bound} $\overline{W}_1$, consider the Lyapunov function
\begin{align}
    V_1(g,u)=\frac12\|x\|_1^2+a_y\|x\|_1,
    \label{eq_V1}
\end{align}
where
\begin{align*}
    a_y=\frac{\lambda_1(\theta_{y,1}-\theta_{-y,1})+\lambda_2(\theta_{y,2}-\theta_{-y,2})}{2(\lambda_1+\lambda_2)},
    \quad y\in\mathcal K
\end{align*}
are in fact distinct elements in the matrix $A(\lambda)$; note that $x$ and $y$ can be derived from $(g,u)$.
% We apply the generator $\mathscr A_1$ to the Lyapunov function to compute the mean drift.
The main challenge for the proof is to show that the mean drift satisfies
\begin{align}
    \mathscr A_1V_1(g,u)\le-c_1\|x\|_1+d_1,
    \quad\forall (g,u)\in\mathcal G\times \mathcal U,
    \label{eq_drift1}
\end{align}
for constants $c_1>0$ and $d_1<\infty$; note that the generator also depends on the policy.
% The drift condition \eqref{eq_drift1} is essential to apply the comparison theorem for Markov processes \cite[Theorem 4.3]{meyn1993stability}, which leads to sufficiency of \eqref{eq_fifo} and the upper bound.
We verify \eqref{eq_drift1} over the four qualitatively different regimes in Fig.~\ref{1norm}.
%
\begin{figure}[hbt]
    \centering
    \includegraphics[width=0.25\textwidth]{images/1norm.png}
    \caption{Regimes 1--4 for verifying drift condition under FIFO.}
    \label{1norm}
\end{figure}
%
Regime 1 is the singleton $\{0\}$; based on properties of PDMPs \cite{benaim15,cloez15}, the mean drift is given by
{\begin{align}
    &\mathscr A_1V_1=\sum_{k=1}^2\lambda_k\mathrm E\Big[\frac12(\theta_{y,k}+R)^2+a_y(\theta_{y,k}+R)\Big]
    \nonumber\\
    &=\sum_{k=1}^2\lambda_k\Big(\frac12\theta_{y,k}^2+\theta_{y,k}\overline{R}+\frac12(\overline{R}^2+\sigma_R^2)+a_y(\theta_{y,k}+\overline{R})\Big),\nonumber
\end{align}}%
which leads to
\begin{align}
    \mathscr A_1V_1\le\Big\|&\Big((\Theta+\overline RJ_{2,2})\circ(\frac12(\Theta+\overline RJ_{2,2})+A)\nonumber\\
    &+\frac12\sigma_R^2J_{1,2}\Big)\lambda\Big\|_\infty;
    \label{eq_d1}
\end{align}
In fact, the above holds over all four regimes.
Regimes 2--4 are given by $\{x:x_1=0,x_2>0\}$, $\{x:x_1>0,x_2>0\}$, $\{x:x_1>0,x_2=0\}$, respectively. The main difference between these regimes is that $y$ must be 2 (resp. 1) over regime 2 (resp. 4), while $y$ can be either 1 or 2 over regime 3. The mean drift over regimes 2--4 satisfies
\begin{align}
    &\mathscr A_1V_1=\Big(-1+\sum_{k=1}^2\lambda_k(\theta_{y,k}+\bar R)+\lambda_{-y}(a_{-y}
    \nonumber\\
    &-a_y)\Big)\|x\|_1+\frac12\sum_{k=1}^2\lambda_k\mathrm E[(\theta_{y,k}+R)^2]-a_y.
    \label{eq_AmV1}
\end{align}
Note that $\theta_{y,k}$ in the summations means that when a class-$k$ vehicle arrives, it has to keep a headway of $\theta_{y,k}$ away from the last vehicle, which is of class $y$.
Substitution of $a_y$ into \eqref{eq_AmV1} yields
\begin{align}
    &-1+\sum_{k=1}^2\lambda_k(\theta_{y,k}+\overline R)+\lambda_{-y}(a_{-y}-a_y)
    \nonumber\\
    &=-1+\sum_{\ell=1}^2\frac{\lambda_\ell}{\|\lambda\|_1}\sum_{k=1}^2\lambda_k(\theta_{\ell,k}+\overline R),
    \quad y=1,2.
    \label{eq_-1+sum}
\end{align}
Then, one can obtain from \eqref{eq_d1} and \eqref{eq_-1+sum} that there exist
\begin{align*}
    c_1&:=1-\Big(\frac{\lambda^T}{\|\lambda\|_1}\Theta+\overline RJ_{1,2}\Big)\lambda>0,\\
    d_1&:=\frac12\Big(\mathrm{diag}\Big((\Theta+\overline RJ_{2,2})\circ(\Theta+\overline RJ_{2,2})\Big)+\sigma_R^2J_{1,2}\Big)\lambda\\&<\infty
\end{align*}
such that \eqref{eq_drift1} holds.
Then, we can apply the comparison theorem \cite[Theorem 4.3]{meyn1993stability} to obtain boundedness in the sense of \eqref{eq_zeta} with $\overline{W}=d_1/c_1$ and the upper bound in \eqref{eq_Wbar1}.

Note that the limit $\overline X$ in \eqref{eq_Xbar} exists under most practical policies if $X(t)$ is bounded in the sense of \eqref{eq_zeta}.
The key is to argue that the PDMP is irreducible in the sense of \cite[p.5]{meyn1993survey}.
The irreducibility argument in the discrete state space is straightforward. For the continuous state space, note that, under most practical policies, the state $x=[0,0]$ (or $u=\{0,0\}$) can be exactly attained from any initial condition in finite time with a positive probability. Hence, the PDMP is irreducible. Since the PDMP is also bounded in the sense of \eqref{eq_zeta}, the PDMP is positive Harris \cite[Theorem 7]{meyn1993survey}, and thus the limit in \eqref{eq_Xbar} exists.

To show {\bf necessity} of \eqref{eq_fifo}, we a consider bounded test function
\begin{align*}
    W_1(g,u)=1- e^{-\alpha_1(\|x\|_1+\beta_y)},
\end{align*}
where $\alpha_1,\beta_y$ are positive constants. In the following, we show that 
\begin{align}
    \mathscr A_1W_1\ge0,\quad\forall (g,u)\in\mathcal G\times\mathcal U.
    \label{eq_AWi}
\end{align}
If $x\ne 0$, the mean drift is given by
\begin{align*}
    \mathscr A_1W_1=&- e^{-\alpha_1(\|x\|_1+\beta_y)}(-1)+\sum_{k=1}^2\lambda_k(e^{-(\alpha_1\|x\|_1+\beta_y)}\\
    &-\mathrm E[e^{-\alpha_1(\|x\|_1+\theta_{y,k}+R+\beta_k)}])\\
    =&e^{-\alpha_1(\|x\|_1+\beta_y)}\Big(-1+\sum_{k=1}^2\lambda_k(1\\
    &-e^{-\alpha_1\theta_{y,k}}g_R(-\alpha)e^{-\alpha_1(\beta_k-\beta_y)}\Big)
\end{align*}
where $g_R(-\alpha_1)$ is the MGF of the rv $R$.
We can expand the exponential terms as follows:
\begin{align*}
    &1-e^{-\alpha_1\theta_{y,k}}g_R(-\alpha_1)e^{-\alpha_1(\beta_k-\beta_y)}\\
    &=
    \alpha_1(\theta_{y,k}+\overline R+\beta_k-\beta_y)+o(\alpha_1),
\end{align*}
where $\beta_y$ are balancing terms analogous to $a_y$ in \eqref{eq_V1}.
Hence, if 
$
        ({\lambda^T}/{\|\lambda\|_1}\Theta+\bar RJ_{1,2})\lambda>1
$
then there exists $\alpha>0$ such that
$$
1-e^{-\alpha_1\theta_{y,k}}g_R(-\alpha_1)e^{-\alpha_1(\beta_k-\beta_y)}\ge0,
$$
which leads to \eqref{eq_AWi}.
Thus, we can apply the drift criterion for transience \cite[Theorem 4]{meyn1993survey} and conclude that the PDMP is unstable. The case that 
$
       ({\lambda^T}/{\|\lambda\|_1}\Theta+\bar RJ_{1,2})\lambda=1
$
involves null-recurrence type arguments, which we omit here; in fact, this boundary case is of limited practical relevance, since an arbitrarily small deviation of model parameters will prevent this from happening.

The {\bf lower bound} $\overline W_0$ results from the construction of an M/G/1 process that optimistically estimates the traffic discharging process. The M/G/1 process has an arrival rate of $\|\lambda\|_1$. The service time is the sum of two independent rv s $\Phi$ and $R$, where $\Phi$ has the probability mass function
\begin{align*}
    p_\Phi(\theta_{11})=\frac{\lambda_1}{\|\lambda\|_1},\quad
    p_\Phi(\theta_{22})=\frac{\lambda_2}{\|\lambda\|_1}
\end{align*}
and $R$ has the CDF $F_R(r)$.
This M/G/1 process essentially uses $\theta_{k,k}$ to under-estimate the headway before a class-$k$ vehicle, which thus leads to an under-estimate of delay. The expression for $\overline W_0$ results from Pollaczek-Khinchin formula \cite[p.248]{gallager2013stochastic}, and this lower bound applies to any sequencing policy.

%%%%%%%%%%%%%%%%%%%%%%%%%%%%%%%%%%%%%%%%
\subsubsection{MS}
To show {\bf sufficiency} of \eqref{eq_ms} and the {\bf upper bound} $\overline{W}_2$, we consider a ``decomposed'' process $(\tilde X,F(t))\in\mathbb R_{\ge0}^2\times\mathbb R_{\ge0}$ such that $\|\tilde X(t)\|_1+F(t)=\|X(t)\|_1$ for all $\ge0$, where $\tilde X(t)$ tracks the ``crossing time-to-go'' and $F(t)$ tracks the ``switchover time-to-go''. Therefore, the decomposed process is stable if and only if the original process is stable.
Specifically, the decomposed process initiates according to
$$
\tilde X(0)=X(0),
\ 
F(t)=0.
$$
Then, for $t>0$, if $\tilde X_k(t)>0$ and if $F(t)=0$, a class-$k$ arrival increases $X_k(t)$ by $\theta_{k,k}+R$, and $F(t)$ is unchanged. Let $\epsilon$ be a constant such that $0<\epsilon<1-\mathrm{diag}(\Theta+\overline{R}J_{2\times2})\lambda$; note that \eqref{eq_ms} ensures the existence of $\epsilon$. If $\tilde X_k(t)=0$, then a class-$k$ arrival increases $\tilde X_k(t)$ by $\theta_{k,k}+\epsilon+R$, and $F(t)$ is increased by $(\theta_{-k,k}-\theta_{k,k}-\epsilon)_+$. Whenever $F(t)>0$, a class-$k$ arrival will increase $\tilde X(t)$ by $\theta_{k,k}+\epsilon+R$ and decrease $F(t)$ by $\min\{F(t),\epsilon\}$.
By definition, $\tilde X(t)\in\mathbb R_{\ge0}^2$ and $F(t)\in[0,\bar f]$, where $\bar f=\sum_k(\theta_{-k,k}-\theta_{k,k})$.
Thus, $\tilde X(t)$ is bounded if and only if $X(t)$ is bounded.
Furthermore, $\{Z(t),\tilde X(t),F(t);t>0\}$ itself is a Markov process.
Then, consider the Lyapunov function
\begin{align}
    V_2(z,\tilde x,f)=\frac12\|\tilde x\|_1^2.
    \label{eq_V2}
\end{align}
For any $(z,\tilde x,f)\in\mathcal K\times\mathbb R_{\ge0}^2\times[0,\bar f]$, the mean drift is
\begin{align}
    &\mathscr A_2V_2(z,\tilde x,f)=\Big(-1+\sum_{k=1}^2\lambda_k(\theta_{k,k}+\bar R+\epsilon\mathbb I\{f\nonumber\\
    &>0\})\Big)\|x\|_1+\frac12\sum_{k=1}^2\lambda_k\mathrm E[(\theta_{k,k}+R+\epsilon\mathbb I\{f>0\})^2].
    \label{eq_AmV2}
\end{align}
The use of $\epsilon$ (and thus the decomposed process) is to distribute the switchover-induced increment $(\theta_{k,-k}-\theta_{k,k})$ to the $\ceil{(\theta_{k,-k}-\theta_{k,k})/\epsilon}$ subsequent arrivals, so that the mean drift is still negative upon switchovers.
% The main differences between the above and \eqref{eq_AmV1} include: (i) the headway is $\theta_{k,k}$ in the above, since all incoming vehicles will be placed after a vehicle of the same class; (ii) the balancing terms $a_y$ do not exist.
Then, by analogy with the FIFO case, we can show that there exist
\begin{align*}
    c_2:=&1-\mathrm{diag}(\Theta+\overline RJ_{2,2})\lambda-\epsilon>0,\\
    d_2:=&\frac12\Big(\mathrm{diag}\Big((\Theta+(\overline R+\epsilon)J_{2,2})\circ(\Theta+(\overline R+\epsilon)J_{2,2})\Big)\nonumber\\
    &+\sigma_R^2J_{1,2}\Big)\lambda<\infty
\end{align*}
such that for all $(z,\tilde x,f)\in\mathcal K\times\mathbb R_{\ge0}^2\times[0,\bar f]$,
\begin{align*}
    \mathscr A_2V_2(z,\tilde x,f)\le-c_2\|\tilde x\|_1+d_2.
\end{align*}
% Note that if \eqref{eq_ms} holds, then there exists $\epsilon>0$ such that $c_2>0$.
Also note that we can let $\epsilon\downarrow0$, since there is at most one switchover to go under MS; this argument would be invalid under other policies (e.g., FIFO), since there could be infinitely many switchovers to go. Finally, $\|\tilde X(t)\|_1$ is associated with the upper bound $d_2/c_2$, and thus $\|X(t)\|$ is associated with the upper bound given by $\overline W_2$; the difference between $\overline W_2$ and $d_2/c_2$ is the mean switchover time under FIFO, which is indeed an upper bound for that under MS.

To show {\bf necessity} of \eqref{eq_ms}, consider the test function 
\begin{align*}
    W_2(g,u)=1-e^{\alpha_2\|x\|_1}.
\end{align*}
Then, one can analogously to the FIFO case.

%%%%%%%%%%%%%%%%%%%%%%%%%%%%%%%%%%%%%%%%
\subsubsection{LQF}
Consider the Lyapunov function
\begin{align*}
    V_{3}(g,u)=\frac12\|x\|_2^2.
\end{align*}
To show that $V_{3}$ drifts negatively, we essentially need to show that the ``mean velocity'' of the PDMP points towards the interior of the level curves; see Fig.~\ref{2norm}.
%
\begin{figure}[hbt]
    \centering
    \includegraphics[width=0.3\textwidth]{images/2norm.png}
    \caption{Regimes 1--4 for verifying drift condition under LQF.}
    \label{2norm}
\end{figure}
%
The drift condition involves four qualitatively different regimes, which are labeled 1--4 in Fig.~\ref{2norm} with
\begin{align*}
    \beta_{\min}=\frac{\lambda_1\tilde\theta_{21}}{1-\lambda_2\tilde\theta_{22}},\quad
    \beta_{\max}=\frac{1-\lambda_1\tilde\theta_{11}}{\lambda_2\tilde\theta_{12}};
\end{align*}
note that $0<\beta_{\min}<\beta_{\max}<\infty$ if \eqref{eq_lqf} holds.
Also note that each regime for $x$ corresponds to a unique regime for $(g,u)$.
Next, we consider the four regimes separately.

Regime \textcircled{1}: $x_1=x_2=0$: By analogy with FIFO, we have
    \begin{align}
        \mathscr A_3V_3&\le
        \Big\|\frac12\Big((\Theta+\overline RJ_{2,2})\circ(\Theta+\overline RJ_{2,2})\nonumber\\
        &+\sigma_R^2J_{1,2}\Big)\lambda\Big\|_\infty=:d_3<\infty
    \end{align}
    for all regimes and thus indeed for regime 1.
    
Regime \textcircled{2}: $x_1\ge0$, $x_2>\beta_{\max} x_1$: 
    We require the matrix $B(\lambda)$ as defined in \eqref{eq_B} to characterize the mean drift to account for two peculiarities of the LQF policy:
    \begin{enumerate}
        \item Since class 2 has a longer queue, a new class-2 arrival will be placed at the end of the sequence of crossing, which leads to a service time of $\theta_{22}+R$. A class-1 arrival will be placed either after a class-1 vehicle or after a class-2 vehicle. In the former sub-case, the arrival has a service time of $\theta_{11}+R$. In the latter sub-case, the arrival has a service time of $\theta_{21}+R$. In addition, the arrival may also be inserted in front of a class-2 vehicle, which will further increases the service time of the following class-1 vehicle from $\theta_{22}+R$ to $\theta_{12}+R$; see Fig.~\ref{fig:two_lqf}.
        \begin{figure}[hbt]
    \centering
    \includegraphics[width=0.45\textwidth]{images/two_lqf.png}
    \caption{Under LQF, a new arrival (2,2) may affect the service time of an existing vehicle (1,3).}
    \label{fig:two_lqf}
    \end{figure}
        
        \item Although the LQF policy essentially prioritizes the longer queue for discharging (i.e., class 2 in thisregime), either class 1 or 2 can be discharged in this regime. The reason is that an ongoing crossing cannot be interrupted; the state may switch between regimes as a crossing goes on. To resolve this, we assume that, in this regime, a class-1 arrival leads to an additional ``phantom'' increment of $u_1$ in $x_1$; this increment will decrease and vanish synchronously with the ongoing crossing, so it will also contributes negatively to the mean drift (see Fig.~\ref{2norm}). Since $u_1\le\theta_{21}+R_{\max}$, such increment is also upper-bounded by $\theta_{21}+R_{\max}$.
        
    \end{enumerate}
To sum up, a class-1 arrival will increase $x_1$ by no greater than $\theta_{21}+R+\theta_{12}+R_{\max}$ and increase $x_2$ by no greater than $\theta_{12}-\theta_{22}$; a class-2 arrival will not influence $x_1$ and will increase $x_2$ by $\theta_{22}+R$.
Thus, the mean drift is upper-bounded by
    \begin{align*}
        &\mathscr A_3V_3\le\bigg(\lambda_1(\theta_{21}+R+\theta_{12}+R_{\max})\cos\phi+\Big(\lambda_1(\theta_{12}
        \\
        &\quad-\theta_{22})+\lambda_2(\theta_{22}+\overline{R})-1\Big)\sin\phi+\frac{o(\| x\|_2)}{\| x\|_2}\bigg)\| x\|_2,
    \end{align*}
    where $\phi=\arctan(x_2/x_1)$; see Fig.~\ref{2norm}. Note that
    \begin{align*}
        &\lambda_1(\theta_{21}+R+\theta_{12}+R_{\max})\cos\phi+\Big(\lambda_1(\theta_{12}-\theta_{22})
        \\
        &\quad+\lambda_2(\theta_{22}+\overline{R})-1\Big)\sin\phi
        \\
        &\stackrel{\eqref{eq_lqf}}{\le}\lambda_1(\theta_{21}+R+\theta_{12}+R_{\max})\frac{1}{\sqrt{1+\beta_{\max}^2}}
        \\
        &\quad+\lambda_2(\theta_{22}+\overline{R})-1\Big)\frac{\beta_{\max}}{\sqrt{1+\beta_{\max}^2}}\\
        &\stackrel{\eqref{eq_beta}}{\le}\frac{\mathrm{det}(\tilde\Theta+\overline{R}J_{2,2})\lambda_1\lambda_2+\mathrm{diag}(\tilde\Theta)\lambda-1}{\sqrt{(1-\lambda_1\tilde\theta_{11})^2+(\lambda_2\theta_{12})^2}}\\
        &=:-c_{3,1}<0,
    \end{align*}
    where also defines the constant $c_{3,1}$.
     Thus, if $\|x\|_2$ is sufficiently large, we have
     \begin{align*}
         \mathscr A_3V_3\le-(c_{3,1}-\epsilon)\|x\|_2,
     \end{align*}
     where $\epsilon>0$ is an arbitrarily small number. Hence, given \eqref{eq_lqf}, we have
     \begin{align*}
         \mathscr A_3V_3<-c_{3,1}\|x\|_2+d_3
     \end{align*}
     over regime 2.
    
Regime \textcircled{3}: $x_1>0$, $\beta_{\min} x_1\le x_2\le\beta_{\max} x_1$: Either class is discharged.
    If class 2 is discharged, the proof is analogous to regime 2. If class 1 is discharged, we have
    \begin{align*}
        \mathscr A_3V_3=&\bigg(\lambda_1(\theta_{11}+\overline{R}-1)\cos\phi+\lambda_2\Big((\theta_{12}+\theta_{21}\\
        &-\theta_{11})+\overline{R}\Big)\sin\phi+\frac{o(\| x\|_2)}{\| x\|_2}\bigg)\| x\|_2
        \\
        \le&\bigg(\lambda_1(\theta_{11}+\overline{R}-1)\frac{1}{\sqrt{1+\beta_{\min}^2}}+\lambda_2\Big((\theta_{12}+\theta_{21}\\
        &-\theta_{11})+\overline{R}\Big)\frac{\beta_{\min}}{\sqrt{1+\beta_{\min}^2}}+\frac{o(\| x\|_2)}{\| x\|_2}\bigg)\| x\|_2\\
        \stackrel{\eqref{eq_beta}}\le&\Bigg(\frac{\mathrm{det}(\tilde\Theta+\overline{R}J_{2,2})\lambda_1\lambda_2+\mathrm{diag}(\tilde\Theta)\lambda-1}{\sqrt{(1-\lambda_1\tilde\theta_{11})^2+(\lambda_2\theta_{12})^2}}\\
        &+\frac{o(\| x\|_2)}{\| x\|_2}\Bigg)\|x\|_2\\
        =&-\Bigg(c_{3,2}-\frac{o(\| x\|_2)}{\| x\|_2}\Bigg)\|x\|_2<0,
    \end{align*}
    which also defines the constant $c_{3,2}$.
    Thus, if $\|x\|_2$ is sufficiently large, we have
     \begin{align*}
         \mathscr A_3V_3\le-(c_{3,2}-\epsilon)\|x\|_2,
     \end{align*}
     where $\epsilon>0$ is an arbitrarily small number. Hence, given \eqref{eq_lqf},
     \begin{align*}
         \mathscr A_3V_3<-c_{3}\|x\|_2+d_3
     \end{align*}
     over regime 2, where $c_3=\min\{c_{3,1},c_{3,2}\}$.
     
Regime \textcircled{4}: $x_2\ge0$, $x_1>\beta_{\min}x_1$: analogous to regime 2.

In conclusion, given \eqref{eq_lqf}, we have
\begin{align*}
    \mathscr A_3V_3\le-c_3\|x\|_2+d_3
\end{align*}
over all regimes. By the comparison theorem, we have
\begin{align*}
    d_3/c_3\ge \|x\|_2\ge\|x\|_1/\sqrt2,
\end{align*}
which leads to stability as well as the upper bound $\overline{W}_3$.