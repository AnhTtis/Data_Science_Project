\section{Conclusions} 
We have proposed a new distributional approach to the classic discounted LQR problem. Specifically, we first provided an analytic expression for the exact random return  that depends on infinitely many random variables. 
% 
Since the computation of this expression is difficult in practice, we also proposed an approximate expression for the distribution of the random return that only depends on a finite number of random variables, and have further characterised the error between these two distributions. Finally, we utilised the proposed random return to  obtain an optimal controller for a risk-averse LQR problem using the CVaR as a measure of risk. 
%
To the best of our knowledge, this is a first framework for distributional LQR: it inherits the advantages of DRL methods compared to standard RL methods that rely on the expected return to evaluate the effect of a given policy, but it also provides an analytic expression for the return distribution, an area where current DRL methods significantly lack. 
%
Future research includes analyzing the theoretical convergence of risk-averse policy gradient algorithms and exploring a model-free setup where the system matrices are unknown. 


