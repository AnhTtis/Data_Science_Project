\documentclass[12pt]{l4dc2023}

% The following packages will be automatically loaded:
% amsmath, amssymb, natbib, graphicx, url, algorithm2e


\newcommand{\yulong}[1]{\textcolor{cyan}{(Yulong:  #1)}}

\title[Policy Evaluation in Distributional LQR]{Policy Evaluation in Distributional LQR}
\usepackage{times}
\usepackage{float}
\usepackage{algorithmic}
\usepackage{algorithm}
%\usepackage{algpseudocode}
%\usepackage{algorithmic}
%\usepackage{graphicx}


% \usepackage[ruled,vlined]{algorithm2e}       % algorithm
% Use \Name{Author Name} to specify the name.
% If the surname contains spaces, enclose the surname
% in braces, e.g. \Name{John {Smith Jones}} similarly
% if the name has a "von" part, e.g \Name{Jane {de Winter}}.
% If the first letter in the forenames is a diacritic
% enclose the diacritic in braces, e.g. \Name{{\'E}louise Smith}

% Two authors with the same address
% \coltauthor{\Name{Author Name1} \Email{abc@sample.com}\and
%  \Name{Author Name2} \Email{xyz@sample.com}\\
%  \addr Address}

% Three or more authors with the same address:
% \coltauthor{\Name{Author Name1} \Email{an1@sample.com}\\
%  \Name{Author Name2} \Email{an2@sample.com}\\
%  \Name{Author Name3} \Email{an3@sample.com}\\
%  \addr Address}

% Authors with different addresses:


\author{%
 \Name{Zifan Wang}$^1$ \Email{zifanw@kth.se} \\
 \Name{Yulong Gao}$^2$ \Email{yulong.gao@cs.ox.ac.uk} \\
 \Name{Siyi Wang}$^3$ \Email{siyi.wang@tum.de} \\
 \Name{Michael M. Zavlanos}$^4$ \Email{michael.zavlanos@duke.edu} \\
  \Name{Alessandro Abate}$^2$ \Email{alessandro.abate@cs.ox.ac.uk}\\
  \Name{Karl H. Johansson}$^1$ \Email{kallej@kth.se} 
  \\
  \addr $^1$ Division of Decision and Control Systems, KTH Royal Institute of Technology, Sweden\\
   \addr $^2$ Department of Computer Science, University of Oxford, UK\\
   \addr $^3$ Chair of Information-oriented Control, Technical University of Munich, Germany\\
    \addr $^4$ Department of Mechanical Engineering and Materials Science, Duke University, USA
}

% \author{%
%  \Name{Zifan Wang} \Email{zifanw@kth.se} \\
%  \addr Division of Decision and Control Systems, KTH Royal Institute of Technology, Sweden
% \\
%  \Name{Yulong Gao} \Email{yulong.gao@cs.ox.ac.uk} \\
%  \addr Department of Computer Science, University of Oxford
% \\
%  \Name{Siyi Wang} \Email{siyi.wang@tum.de} \\
%  \addr Chair of Information-oriented Control, Technical University of Munich, Germany
% \\
%  \Name{Michael M. Zavlanos} \Email{michael.zavlanos@duke.edu} \\
%  \addr Department of Mechanical Engineering and Materials Science, Duke University, USA
% \\
%   \Name{Alessandro Abate} \Email{alessandro.abate@cs.ox.ac.uk}\\
%   \addr Department of Computer Science, University of Oxford 
% \\
%   \Name{Karl H. Johansson} \Email{kallej@kth.se} \\
%   \addr Division of Decision and Control Systems, KTH Royal Institute of Technology, Sweden
% }




\begin{document}

\maketitle

\begin{abstract}%
Distributional reinforcement learning (DRL) enhances the understanding of the effects of the randomness  in the environment by letting agents learn the distribution of a random return, rather than its expected value as in standard RL. 
%
At the same time, a main challenge in DRL is that policy evaluation in DRL typically relies on the representation of the return distribution, which needs to be carefully designed.
%
In this paper, we address this challenge for a special class of DRL problems that rely on discounted linear quadratic regulator (LQR) for control, advocating for a new distributional approach to LQR, which we call \emph{distributional LQR}. Specifically, we provide  a closed-form expression of the distribution of the random return which, remarkably, is applicable to all exogenous disturbances on the dynamics, as long as they are independent and identically distributed (i.i.d.).
%
While the proposed exact return distribution consists of infinitely many random variables, we show that this distribution can be approximated by a finite number of random variables, and the associated approximation error can be analytically bounded under mild assumptions. 
%
Using the approximate return distribution, we propose a zeroth-order policy gradient algorithm for risk-averse LQR using the Conditional Value at Risk (CVaR) as a measure of risk. 
%
Numerical experiments are provided to illustrate our theoretical results.
\end{abstract}

\begin{keywords}%
  Distributional LQR, distributional RL, policy evaluation, risk-averse control% 
\end{keywords}



% \begin{itemize}
%   \item Limit the main text (not counting references) to 10 PMLR-formatted pages, using this template.
%   \item Include {\em in the main text} enough details, including proof details, to convince the reviewers of the contribution, novelty and significance of the submissions.
% \end{itemize}

% Acknowledgments---Will not appear in anonymized version



\vspace{-10pt}
%%%%%%%%% BODY TEXT

\section{Introduction}
\label{section:introduction}
%% 1. why should someone care?

%The advent of advanced interactive computer vision systems~\cite{hololens} and recent progress in vision-language and multi-modal models~\cite{} opens doors for such next generation of assistive agents. 
% We envision that the future assistive agents would build up on these visual and language reasoning capabilities of today and empower users to achieve goals in their everyday lives. In particular, such agents would be able to reason about \emph{unseen} human goals... 
% We posit that such agents would require the ability to understand user goals described in natural language at high-level i.e., without complete details about as well as unseen user goals. 

%Recent progress in augmented reality systems~\cite{hololens, magicleap}, as well as vision-language and multi-modal models~\cite{}, opens doors for the next generation of assistive agents. 
Inspired by recent progress in visual systems~\cite{MagicLeap, ungureanu2020hololens}, we consider an assistive egocentric agent capable of reasoning about daily activities. When invoked via natural language commands, for e.g., while baking a cake, the agent understands the steps involved in baking, tracks progress through the various stages of the task, detects and proactively prevents mistakes by making suggestions. Such an agent would empower users to learn new skills and accomplish tasks efficiently.
% One could envision invoking such an agent merely through natural language descriptions of tasks similar to how present day assistants such as Alexa, Siri etc.~\cite{voice_assistants} are invoked. 
%We envision such agents to empower users in daily life by  invoking them naturally through 

%% 2. Why is it challenging? 
%While recent progress in vision-language and multi-modal models~\cite{} opens doors for such next generation of assistive agents, various challenges remain in making such agents a reality. 
%To make such agents a reality, 

Developing such an egocentric agent capable of tracking and verifying everyday tasks based on their natural language specification is challenging for multiple reasons. First, such an agent must reason about various ways of doing a \emph{multi-step} task specified in natural language. This entails decomposing the task into relevant actions, state changes, object interactions as well as any necessary causal and temporal relationships between these entities. Secondly, the agent must ground these entities in egocentric observations to track progress and detect mistakes. Lastly, to truly be useful, such an agent must support tracking and verification for a combination of tasks and, ideally, even unseen tasks. These three challenges -- causal and temporal reasoning about task structure from natural language, visual grounding of sub-tasks, and compositional generalization -- form the core goals of our work.

% %% 3. What are we doing? What is our approach?
% \aks{I think this is a matter of preference, but I personally don't like related work in intro. I would make this paragraph be about EgoTV and NSG. Starting with something like - "To this end, we propose...", ie, your next paragraph.}
% \nk{+1, we should move parts of this para to lit review and delete the rest.}
% Recent research on language modeling enables decomposing tasks into multiple steps from natural language descriptions~\cite{llm_zero_shot_planning,proscript}. However, such \emph{task decompositions} cannot directly be leveraged for task tracking in egocentric agents because of lack of grounding into the visual observations or context. In parallel, the computer vision community has advanced action recognition~\cite{}, object detection and tracking~\cite{}, hand object interaction and object state change detection~\cite{ego_4d,change_it,}, step classification in procedural tasks~\cite{}, and even vision language reasoning~\cite{nsvqa,nscl,star_situated_reasoning,clevrer}, which may help with the grounding challenge. However, majority of current research on identifying actions, objects, steps, or state changes does not account for the overall task structure. Likewise, predominant research on vision language understanding~\cite{} and multi-modal grounding~\cite{} does not consider the temporal and causal constraints that emerge in task tracking and verification. We therefore focus on the order-aware visual grounding problem in our work, with an eye towards compositional generalization to scale usability of these agents. In particular, we aim to achieve visual grounding of the actions and objects corresponding to each step or sub-task obtained from the task description decomposition in an order-aware manner.

%% 4. What are our results/contributions?
As our first contribution, we propose a benchmark -- \emph{\textbf{Ego}centric \textbf{T}ask \textbf{V}erification} (\etv \inlineimg{figures/TV}) -- and a corresponding dataset in the AI2-THOR~\cite{ai2thor} simulator. % \emoji{tv}
Given a natural language (NL) task description and a corresponding egocentric video of an agent, the goal of \etv is to verify whether the task was successfully completed in the video or not.
\etv contains multi-step tasks with \emph{ordering} constraints on the steps and \emph{abstracted} NL task descriptions with omitted low-level task details inspired by the needs of real-world assistants. We also provide splits of the dataset focused on different generalization aspects, e.g., unseen visual contexts, compositions of steps, and tasks (see Figure~\ref{figure:dataset}).
% Next, we create splits of the dataset focused on different aspects of generalization, ranging from generalization to unseen visual context to unseen compositions of steps and tasks. Figure~\ref{figure:dataset} shows an example task and overview of generalization splits from \etv. Succeeding at \etv tasks requires decomposing tasks into partially-ordered steps from the NL description and order-aware visual grounding of these steps into the video. 

Our second contribution is a novel approach for order-aware visual grounding~--~\emph{\textbf{N}euro-\textbf{S}ymbolic \textbf{G}rounding} (NSG), capable of compositional reasoning and generalizing to unseen tasks owing to its ability to leverage abstract NL descriptions and compositional structure of tasks (task decomposition, ordering).~In contrast, state-of-the-art vision-language models~\cite{coca,clip,videoclip,clip_hitchiker} struggle to ground NL descriptions in egocentric videos, and do not generalize to unseen tasks.~NSG outperforms these models by~$\mathbf{33.8}\%$~on compositional generalization and~$\mathbf{32.8}\%$~on abstractly described task verification. Finally, to evaluate \nsg on real-world data, we instantiate \etv on the CrossTask~\cite{cross_task} instructional video dataset. %Specifically, we synthetically create videos with mistakes in CrossTask. 
We find that it also outperforms state-of-the-art models at task verification on CrossTask. We hope that the \etv~benchmark and dataset will enable future research on egocentric agents capable of aiding in everyday tasks.

% We experiment with many for the \etv tasks. We find that while these models generalize well to unseen visual context, they struggle to perform grounding from abstracted task descriptions and to generalize to new compositions of tasks. To deal with these challenges, we take inspiration from recent research on and develop . ~\rd{unclear why neurosymbolic models would do well on abstraction.} 

% To summarize, our main contributions are:~1)~\etv: a benchmark and synthetic dataset to systematically study egocentric task verification.
% 2)~\nsg: a novel neuro-symbolic approach to enable the core reasoning capability for \etv -- order-aware visual grounding. We demonstrate \nsg's capability on our synthetic \etv dataset as well as a real-world dataset derived from CrossTask. We will release both of these datasets and our models for future research on egocentric task tracking and verification. 


% Assistive agents require the ability to track actions and state changes from an egocentric perspective for effective assistance in day-to-day tasks. For example, an agent helping a user prepare a recipe would need to both generate the steps of the recipe (\textit{plan generation}) and track the user's actions to ensure the plan is executed correctly (\textit{plan verification}). We formulate this as a Video Entailment task~\cite{violin_dataset,9710490} \rd{should we call our task video-based goal entailment?}, wherein, given an egocentric video of an agent (or human) performing a task (\textit{premise}) and a NL task description (\textit{hypothesis}), the objective is to learn a model to track whether the given task was successfully executed in the video. 
% An ideal model should also be able to seamlessly generalize to novel compositions (of actions and objects) unseen during training. \rd{add a line about what we mean by abstraction and why is it important.} To this end, we generate a novel Vision-Language dataset on the AI2-THOR simulator~\cite{ai2thor} to study compositional and abstraction-based generalization. Our dataset provides effective evaluation measures in a controlled setting, while closely reflecting the diversity of real-world events. We implement and train a variety of end-to-end models based on existing state-of-the-art approaches. We empirically demonstrate that neural models suffer from overfitting and cannot effectively generalize to novel compositions of actions, objects, and scenes. 
% To address this problem, we propose an end-to-end Neuro-Symbolic (NeSy) framework that performs plan generation and verification. At the heart of our approach is the hypothesis that symbolic reasoning models are good at generalization and capturing compositional substructure, while neural models are good at learning representations from sensory data~\cite{10.5555/3326943.3327039,nscl,clevrer}. \rd{summarize contributions in a bulleted list.} \rd{also add a line about the main result e.g., x\% improvement as compared to end-to-end models}. 

% \rd{we also evaluate NeSy with real-world data: add briefly about CrossTask experiments.}

% % \fbox{\begin{minipage}{\linewidth}
% % \textbf{Problem Statement}

% % Given:
% % (i) Premise: Egocentric video of an agent performing a task.
% % (ii) Hypothesis: NL description of the task.

% % Learn: A model to track whether the premise entails the hypothesis. The output of the model is True if the given task is executed successfully in the video.
% % \end{minipage}}

% \textbf{Contributions:} 
% \begin{itemize}
%     \item We generate a benchmark video-language dataset to study compositional and abstraction-based generalization.
%     \item We evaluate the performance of a variety of state-of-the-art models and show that these (baseline) models cannot effectively generalize to novel compositions of actions.
%     \item We propose a novel end-to-end NeSy approach that significantly outperforms the baselines on some compositional generalization splits while performing on par with them on the rest.
%     \item We also evaluate our NeSy approach with real-world data showing similar performance improvements.
% \end{itemize}


\vspace{-0.2cm}
\section{Problem Statement}
Consider a discrete-time linear dynamical system:
\begin{align}
    x_{t+1} = A x_t + B u_t +v_t ,
\end{align}
where $x_t \in \mathbb{R}^n$, $u_t\in \mathbb{R}^p$, $v_t \in \mathbb{R}^n$ are the system state, control input, and the exogenous disturbance, respectively. 
We assume that the exogenous disturbances $v_t$ with bounded moments, $t\in \mathbb{N}$, are i.i.d. sampled from a distribution $ \mathcal{D}$ of arbitrary form.

\vspace{-0.2cm}
\subsection{Classical LQR}
The canonical  LQR problem aims to find a control policy $\pi: \mathbb{R}^n \rightarrow \mathbb{R}^p$ to minimise the objective
%\begin{align}
 %   J(u) = \mathbb{E}\left[ \sum_{t=0}^{\infty} \gamma^t c_t(x_t,u_t) \right]
 %   = \mathbb{E}\left[ \sum_{t=0}^{\infty} \gamma^t (x_t^T Q x_t + u_t^T R u_t)  \right],
%\end{align}
\begin{align}
    J(u) %= \mathbb{E}\left[ \sum_{t=0}^{\infty} \gamma^t c_t(x_t,u_t) \right]
    = \mathbb{E}\left[ \sum_{t=0}^{\infty} \gamma^t (x_t^T Q x_t + u_t^T R u_t)  \right],
\end{align}
where $Q,R$ are positive-definite constant matrices and $\gamma \in (0,1)$ is a discount parameter. Given a control policy $\pi$, let $V^{\pi}(x) = \mathbb{E}\left[ \sum_{t=0}^{\infty} \gamma^k (x_t^T Q x_t + u_t^T R u_t)  \right]$ denote the expected return from an initial state $x_0 =x$ with $ u_t = \pi( x_t)$.
% \begin{align}
%     V^{\pi}(x) = \mathbb{E}\left[ \sum_{t=0}^{\infty} \gamma^k (x_t^T Q x_t + u_t^T R u_t)  \right],\quad u_t = \pi( x_t),
% \end{align}
% denote the expected return from an initial state $x_0 =x$.
For the static linear policy $\pi(x_t)=K x_t$, the value function $V^{\pi}(x)$ satisfies the Bellman equation
\begin{align}\label{eq:Bellman expectation}
    V^{\pi}(x) %=\mathop{\mathbb{E}}_{X' = (A+BK)x+v_0 }[ c_0 + \gamma V^{\pi}(X')]
    &= x^T (Q + K^T R K) x + \gamma \mathop{\mathbb{E}}_{X' = (A+BK)x+v_0 } [V^{\pi}(X')],  
\end{align} 
where the capital letter $X'$ denotes a random variable over which we take the expectation.

When the exogenous disturbances $v_t$ are normally distributed with zero mean, the value function is known to take the quadratic form  $V^{\pi}(x) = x^T P x +q$, where $P>0$ is the solution of the Lyapunov equation $P = Q+ K^T R K + \gamma A_K^T P A_K$ and $q$ is a scalar related to the variance of $v_t$. In particular, the optimal control feedback gain is obtained as $K^*=-\gamma(R+\gamma B^TPB)^{-1}PA$ and $P$ is the solution to the classic Riccati equation $P = \gamma A^T P A - \gamma^2 A^T P B (R+\gamma B^T P B)^{-1} B^T P A+Q$. 




%When the exogenous disturbances $v_t$ are \emph{\yulong{Gaussian with zero mean}}, the value function is shown to take the quadratic form  $V^{\pi}(x) = x^T P x +q$, where $P>0$ is a solution of the Riccati equation $P = \gamma A^T P A - \gamma^2 A^T P B (R+\gamma B^T P B)^{-1} B^T P A$ and $q$ is a scalar related to the variance of $v_t$.


%When the exogenous disturbances $v_t$ are \emph{\yulong{Gaussian with zero mean}}, the value function is shown to take the quadratic form  $V^{\pi}(x) = x^T P x +q$, where $P>0$ is a solution of the Ricatti equation $P = \gamma A^T P A - \gamma^2 A^T P B (R+\gamma B^T P B)^{-1} B^T P A$ and $q$ is a scalar related to the variance of $v_t$.


% \begin{align}
%     V^{*}(x) &= x^T P x + q,
% \end{align}
% where $P = \gamma A^T P A - \gamma^2 A^T P B (R+\gamma B^T P B)^{-1} B^T P A$ and $q = \frac{  \gamma}{1-\gamma} {\rm{Tr}}(PR_v)$, {\color{blue}where the matrix $R_v$ is the covariance of random variable $v$.  }

\subsection{Distributional LQR}
Motivated by the advantages of DRL in better understanding the effects of the randomness in the environment and in considering more general optimality criteria, in this paper we propose a distributional approach to the LQR problem.
%
Unlike classical reinforcement learning, which relies on expected returns, DRL \citep{bdr2022} relies on the distribution of random returns.
%
The return distribution characterises the probability distribution of different returns generated by a given policy and, as such, it contains much richer information on the performance of a given policy compared to the expected return.
%
In the context of LQR, we
 denote by $G^{\pi}(x)$ the random return using the static control strategy $u_t = \pi( x_t)$ from the initial state $x_0=x$, which is defined as
\begin{align}
    G^{\pi}(x) %=  \sum_{t=0}^{\infty} \gamma^t c_t(x_t,u_t) 
    = \sum_{t=0}^{\infty} \gamma^t (x_t^T Q x_t + u_t^T R u_t) , \quad  u_t = \pi( x_t),x_0 = x.
\end{align}
%
It is straightforward to see that the expectation of $G^{\pi}(x)$ is equivalent to the value function $V^{\pi}(x)$. 
The standard Bellman equation in \eqref{eq:Bellman expectation} decomposes the long-term expected return into an immediate stage cost plus the expected return of future actions starting at the next step. 
%In fact, we can do the analog for the random return $G^{\pi}(x)$ with some modifications. Specifically, 
Similarly, we can define the distributional Bellman equation for the random return as 
\begin{align}\label{eq:rv:Bellman}
    G^{\pi}(x) & \mathop{=}^{D} x^TQx+\pi(x)^TR\pi(x) + \gamma G^{\pi}(X'), \quad X' = Ax+B\pi(x)+v_0.
\end{align}
Here we use the notation $\mathop{=}\limits^{D}$ to denote that two random variables $Z_1,Z_2$ are equal in distribution, i.e., $Z_1 \mathop{=}\limits^{D} Z_2$. Note  that $X'$ denotes a random variable, as in \eqref{eq:Bellman expectation}.
Compared to the expected return in LQR, which is a scalar, here the return distribution is infinite-dimensional and can have a complex form. 
It is  challenging to estimate an infinite-dimensional function exactly with finite data and thus an approximation of the return distribution is  necessary in practice. 
%


In this paper,  we first analytically characterise the random return for the LQR problem. Then we  show how to  approximate the  distribution of the random return using finite random variables, so such that the approximated distribution is computationally tractable and the  approximation error is bounded. The proposed distributional LQR framework allows us to consider more general optimality criteria, which we demonstrate by using the proposed return distribution to develop a policy gradient algorithm for risk-averse LQR.
%




\section{Main Results}
\subsection{Exact Form of the Return Distribution}
In this section, we precisely characterise the distribution of the random return  that satisfies the distributional  Bellman equation \eqref{eq:rv:Bellman}. 
Given a static linear policy $\pi(x_t)=K x_t$, we denote by $G^K(x)$ the random return $G^{\pi}(x)$  under the policy $\pi(x_t)$ from the initial state $x_0=x$ , which is defined as
\begin{align*}
    G^{K}(x) =  \sum_{t=0}^{\infty} \gamma^t x_t^T (Q+K^T R K) x_t  , \quad x_0 = x.
\end{align*}
The random return  $G^{K}(x)$ satisfies the following distributional Bellman equation  
\begin{align}\label{eq:rv:bellman}
    G^{K}(x) & \mathop{=}^{D} x^TQ_Kx + \gamma G^{K}(X'),\quad X'=A_K x+ v_0,
\end{align}
where $A_K:=A+BK$ and $Q_K := Q+ K^T R K$. 
In the following theorem, we provide an explicit expression of the random return $G^K(x)$.

\begin{theorem}
Suppose that the feedback gain  $K$ is stabilizing, i.e., $A_K=A+BK$ is stable. Let  \begin{align}\label{eq:dist_func}
    G^{K}(x) = x^T P x +  \sum_{k = 0 }^{\infty}  \gamma^{k+1}  w_k^T P w_k +  2 \sum_{k = 0 }^{\infty} \gamma^{k+1} w_k^T P A_K^{k+1}x + 
 2  \sum_{k = 1 }^{\infty} \gamma^{k+1} w_k^T P \sum_{\tau=0}^{k-1} A_K^{k-\tau}w_{\tau},
\end{align}
where  $P$ is obtained from the algebraic Riccati equation $P = Q+ K^T R K + \gamma A_K^T P A_K$, and the random variables $w_k \sim \mathcal{D} $ are independent from each other for all $k\in\mathbb{N}$. Then, the random variable $G^{K}(x)$ defined in 
    \eqref{eq:dist_func} is a fixed point solution to the distributional Bellman equation \eqref{eq:rv:bellman}.
\end{theorem}
\begin{proof}
Recall that $X' = A_K x + v_0$, where $v_0$ is a  random variable sampled from the distribution $\mathcal{D}$ and is independent from $w_k$, $k\in \mathbb{N}$, in \eqref{eq:dist_func}. 
Substituting \eqref{eq:dist_func} into the right hand side of the equation \eqref{eq:rv:bellman}, we have that
\begin{align*}
    & x^T(Q+K^T R K)x + \gamma G^{K}(X') \\
    =  & x^TQ_Kx + \gamma X'^T P X' +  \sum_{t=0}^{\infty} \gamma^{t+2} w_t^T P w_t + 2 \sum_{t=0}^{\infty} \gamma^{t+2} w_t^T  P A_K^{t+1}  X' \\
    &+ 2  \sum_{t=1}^{\infty} \gamma^{t+2} w_t^T P A_K \sum_{i=0}^{t-1} A_K^{t-1-i} w_i  \\
    = & x^TQ_Kx + \gamma (A_K x + v_0)^T P (A_K x + v_0) + \gamma^2 \sum_{t=0}^{\infty} \gamma^t w_t^T P w_t  + 2 \gamma^2 \sum_{t=1}^{\infty} \gamma^t w_t^T P \sum_{i=0}^{t-1} A_K^{t-i} w_i \\
    &+ 2\gamma^2 \sum_{t=0}^{\infty} \gamma^t w_t^T  P A_K^{t+1}  (A_K x + v_0) \\
    = & x^T(Q_K + \gamma A_K^T P A_K)x + \underbrace{\gamma v_0^T P v_0 + \gamma^2 \sum_{t=0}^{\infty} \gamma^t w_t^T P w_t }_{:=T_1} +\underbrace{ 2\gamma v_0^T PA_K x + 2\gamma^2 \sum_{t=0}^{\infty} \gamma^t w_t^T  P A_K^{t+2} x}_{:=T_2} \\
    & + \underbrace{2 \gamma^2 \sum_{t=1}^{\infty} \gamma^t w_t^T P  \sum_{i=0}^{t-1} A_K^{t-i} w_i + 2\gamma^2 \sum_{t=0}^{\infty} \gamma^t w_t^T  P  A_K^{t+1} v_0}_{:=T_3}.
\end{align*}
Define $ \xi_0 :=v_0$, $\xi_t = w_{t-1}$, $t=1,2,\ldots$. 
From the definition of the term $T_1$, we have that
\begin{align*}
    T_1 & = \gamma v_0^T P v_0 + \gamma^2 \sum_{t=0}^{\infty} \gamma^t w_t^T P w_t 
     \mathop{=}^{k=t+1}  \gamma \xi_0^T P \xi_0 + \gamma \sum_{k=1}^{\infty} \gamma^k \xi_{k}^T P \xi_{k} 
    =  \gamma \sum_{k=0}^{\infty} \gamma^k \xi_{k}^T P \xi_{k}.
\end{align*}
% For the term $T_2$, applying similar techniques as above, we have that $T_2=2\gamma  \sum_{k=0}^{\infty} \gamma^k \xi_k^T P A_K^{k+1} x$.
For the term $T_2$, we have that
\begin{align*}
    T_2 & = 2\gamma v_0^T PA_K x + 2\gamma^2 \sum_{t=0}^{\infty} \gamma^t w_t^T  P A_K^{t+2} x 
    = 2\gamma \xi_0^T P A_K x + 2\gamma^2 \sum_{t=0}^{\infty} \gamma^t \xi_{t+1}^T  P A_K^{t+2} x \\
    &\mathop{=}^{k=t+1}  2\gamma \xi_0^T P A_K x + 2\gamma \sum_{k=1}^{\infty} \gamma^k \xi_{k}^T  P A_K^{k+1} x 
    =  2\gamma  \sum_{k=0}^{\infty} \gamma^k \xi_k^T P A_K^{k+1} x.
\end{align*}
%
%
%
%
%
Using similar techniques for the term $T_3$, we obtain that
$T_3 = 2 \gamma \sum_{k=1}^{\infty} \gamma^{k} \xi_{k}^T P A_K  \sum_{i=0}^{k-1} A_K^{k-1-i} \xi_i.$ 
% \begin{align*}
%     T_3 =& 2 \gamma^2 \sum_{t=1}^{\infty} \gamma^t w_t^T P A_K \sum_{i=0}^{t-1} A_K^{t-1-i} w_i + 2\gamma^2 \sum_{t=0}^{\infty} \gamma^t w_t^T  P  A_K^{t+1} v_0 \\
%     = & 2 \gamma^2 \sum_{t=1}^{\infty} \gamma^t \xi_{t+1}^T P A_K \sum_{i=0}^{t-1} A_K^{t-1-i} \xi_{i+1} + 2\gamma^2 \sum_{t=1}^{\infty} \gamma^t \xi_{t+1}^T P  A_K^{t+1} \xi_0   +2\gamma^2 \xi_1^T P A_K \xi_0\\
%     = & 2 \gamma \sum_{t=1}^{\infty} \gamma^{t+1} \xi_{t+1}^T P A_K \left(  \sum_{i=0}^{t-1} A_K^{t-1-i} \xi_{i+1} + A_K^t \xi_0\right)  +2\gamma^2 \xi_1^T P A_K \xi_0\\
%     = & 2 \gamma \sum_{t=1}^{\infty} \gamma^{t+1} \xi_{t+1}^T P A_K  \left( \sum_{i=0}^t A_K^{t-i} \xi_i \right) +2\gamma^2 \xi_1^T P A_K \xi_0\\
%     \mathop{=}^{k=t+1} &  2 \gamma \sum_{k=2}^{\infty} \gamma^{k} \xi_{k}^T P A_K  \left( \sum_{i=0}^{k-1} A_K^{k-1-i} \xi_i \right) +2\gamma^2 \xi_1^T P A_K \xi_0\\
%     = & 2 \gamma \sum_{k=1}^{\infty} \gamma^{k} \xi_{k}^T P A_K  \sum_{i=0}^{k-1} A_K^{k-1-i} \xi_i .
% \end{align*}
Due to the fact that $P =Q+ K^T R K + \gamma A_K^T P A_K$, we have 
\begin{align}\label{eq:proof:P1:temp1}
    & x^TQ_Kx + \gamma G^{K}(X') 
    =  x^T P x + T_1 +T_2 +T_3 \nonumber \\
    = & x^T P x + \gamma \sum_{k=0}^{\infty} \gamma^k \xi_{k}^T P \xi_{k} + 2\gamma  \sum_{k=0}^{\infty} \gamma^k x^T P A_K^{k+1} \xi_k + 2 \gamma \sum_{k=1}^{\infty} \gamma^{k} \xi_{k}^T P A_K  \sum_{i=0}^{k-1} A_K^{k-1-i} \xi_i,
\end{align}
which is in the same form as in \eqref{eq:dist_func}.
Since $\{ \xi_k\}_{k=0}^{\infty}$ and $\{ w_k\}_{k=0}^{\infty}$ are i.i.d.,
we have that the two random variables \eqref{eq:dist_func} and \eqref{eq:proof:P1:temp1} have the same distribution, i.e.,
$G^{K}(x)  \mathop{=}\limits^{D}  x^TQ_Kx + \gamma G^{K}(X').$
%The proof is completed. 
\end{proof}

% \begin{proposition}
%     {\color{blue}Suppose that the exogenous disturbance $v_t$ is Gaussian and the feedback gain  $K$ is stabilizing, i.e., $A_K=A+BK$ is stable.  Then, for any $x\in \mathcal{R}^d$, }we have that $${\rm var}(G^{\pi}(x))< \infty.$$  
% \end{proposition}



\subsection{Approximation of the Return Distribution with Finite Parameters}\label{Sec:Approxreturn}

In this section, we show how to approximate the random return defined in \eqref{eq:dist_func} using a finite number of random variables. Considering only the first $N$ terms in the summations in the expression in \eqref{eq:dist_func}  and disregarding the terms for $k$ larger than $N$ yields the following:   
\begin{align}\label{eq:approxreturn}
    {G}^{K}_N(x) =  x^T P x +  \sum_{k = 0 }^{N-1}  \gamma^{k+1}  w_k^T P w_k +  2 \sum_{k = 0 }^{N-1} \gamma^{k+1} w_k^T P A_K^{k+1}x + 
 2 \sum_{k = 1 }^{N-1} \gamma^{k+1} w_k^T P \sum_{\tau=0}^{k-1} A_K^{k-\tau}w_{\tau}.
\end{align}
Let $F^{K}_x$ and ${F}^{K}_{x,N}$ denote the cumulative distribution function (CDF) of $G^{K}(x)$ and ${G}_N^{K}(x)$, respectively. The following theorem provides an upper bound on the difference between $F^{K}_x$ and ${F}^{K}_{x,N}$, and shows that  the sequence $\{{G}^{K}_N(x)\}_{N\in \mathbb{N}}$ converges pointwise in distribution to $G^{K}(x)$,  $\forall x\in\mathbb{R}^n$. 

\begin{theorem}\label{Theorem:approx}
Assume that the probability density functions of $w_k$ exist and are bounded, and satisfy $\mathbb{E}[w_k^T w_k] \leq \sigma_0^2$, $\mathbb{E}[\left\|w_k\right\|_2] \leq \mu_0 $, for $\forall k\in \mathbb{N}$.
Suppose that the feedback gain  $K$ is stabilizing such that $\left\|A_K\right\|_2 = \rho_K <1$. Then, the sup difference between the CDFs $F^K_x$ and ${F}^{K}_{x,N}$ is bounded by
\begin{align}\label{eq:approximat:bound}
     \sup_{z}|{F}^{K}_x(z)-{F}^{K}_{x,N}(z) | \leq C \gamma^N,
\end{align}
where $C$ is a constant that depends on the matrices $A,B,Q,R,K$, the initial state value $x$, and the parameters $\gamma, \rho_K, \sigma_0,\mu_0$. 
\end{theorem}

\begin{proof}
Define $Y_N:= G^K(x) - G^K_N(x)$, we have
\begin{align}\label{eq:upp_bound_temp1}
     &\sup_{z}|{F}^{K}_x(z)-{F}^{K}_{x,N}(z) | 
     =\sup_{z}|\mathbb{P}(G^K_N(x) \leq z) -\mathbb{P}(G^K(x)\leq z) | \nonumber \\
     =& \sup_{z}|\mathbb{P}(G^K_N(x)\leq z) -\mathbb{P}(G^K_N(x) +Y_N\leq z) | \nonumber \\
     =& \sup_{z}\Big|\mathbb{P}(G^K_N(x)\leq z) \int_{-\infty}^{\infty} \mathbb{P}(Y_N = t)dt -\int_{-\infty}^{\infty} \mathbb{P}(G^K_N(x) \leq z-t) \mathbb{P}(Y_N = t) dt \Big| \nonumber \\
     =& \sup_{z}\Big| \int_{-\infty}^{\infty} \mathbb{P}(Y_N = t) \big(  F^K_{x,N}(z) -F^K_{x,N}(z-t) \big) dt \Big|.
\end{align}
Since the random variables $w_t$ are i.i.d for all $t>0$ and the probability density function of $w_t$ exists, the function $F^K_{x,N}$ is continuous and differentiable. 
Applying the mean value theorem, when $t>0$ there exists a point $z'\in[z-t,z]$ such that $F^K_{x,N}(z) -F^K_{x,N}(z-t) = f^K_{x,N}(z') t$, where $f^K_{x,N}$ is the probability density function of $G^K_N(x)$. Since the probability density function of $w_t$ is bounded, it further follows  that $f^K_{x,N}$ is bounded.   Then, we have that $|F^K_{x,N}(z) -F^K_{x,N}(z-t)| = |f^K_{x,N}(z') t| \leq L_0 |t|$, where $L_0$ is an upper bound of the probability function $f^K_{x,N}$. Following a similar argument, we can show that this inequality holds when $t\leq 0$. Substituting this inequality into \eqref{eq:upp_bound_temp1}, we obtain
\begin{align}\label{eq:upp_bound_temp2}
   \sup_{z}|{F}^{K}_x(z)-{F}^{K}_{x,N}(z) | \leq \sup_{z}\Big| \int_{-\infty}^{\infty} \mathbb{P}(Y_N = t) L_0 |t| dt \Big| 
     = L_0 \mathbb{E}|Y_N|.
\end{align}
From the definition of $Y_N$, we obtain that
\begin{align*}
     Y_N =& 
     \sum_{k = N }^{\infty}  \gamma^{k+1}  w_k^T P w_k +  2 \sum_{k = N }^{\infty} \gamma^{k+1} w_k^T P A_K^{k+1}x + 2 \sum_{k = N }^{\infty} \gamma^{k+1} w_k^T P \sum_{\tau=0}^{k-1} A_K^{k-\tau}w_{\tau} \nonumber \\
     \mathop{=}^{t=k-N}&    \gamma^N \Big(  \sum_{t = 0 }^{\infty} \gamma^{t+1}  w_{t+N}^T P w_{t+N} +  2 \sum_{t = 0 }^{\infty} \gamma^{t+1} w_{t+N}^T P A_K^{t+N+1}x \nonumber \\
     &+ 2 \sum_{t = 0 }^{\infty} \gamma^{t+1} w_{t+N}^T P \sum_{\tau=0}^{t+N-1} A_K^{t+N-\tau}w_{\tau}  \Big).
\end{align*}
%\begin{align*}
%      Y_N =& 
%      \sum_{k = N }^{\infty}  \gamma^{k+1}  w_k^T P w_k +  2 \sum_{k = N }^{\infty} \gamma^{k+1} w_k^T P A_K^{k+1}x + 2 \sum_{k = N }^{\infty} \gamma^{k+1} w_k^T P \sum_{\tau=0}^{k-1} A_K^{k-\tau}w_{\tau} \nonumber \\
%      \mathop{=}^{t=k-N}&  \sum_{t = 0 }^{\infty} \gamma^{t+N+1}  w_{t+N}^T P w_{t+N} +  2 \sum_{t = 0 }^{\infty} \gamma^{t+N+1} w_{t+N}^T P A_K^{t+N+1}x \nonumber \\
%      &+ 2 \sum_{t = 0 }^{\infty} \gamma^{t+N+1} w_{t+N}^T P \sum_{\tau=0}^{t+N-1} A_K^{t+N-\tau}w_{\tau} \nonumber \\
%      =&  \gamma^N \Big(  \sum_{t = 0 }^{\infty} \gamma^{t+1}  w_{t+N}^T P w_{t+N} +  2 \sum_{t = 0 }^{\infty} \gamma^{t+1} w_{t+N}^T P A_K^{t+N+1}x \nonumber \\
%      &+ 2 \sum_{t = 0 }^{\infty} \gamma^{t+1} w_{t+N}^T P \sum_{\tau=0}^{t+N-1} A_K^{t+N-\tau}w_{\tau}  \Big).
% \end{align*}
Taking the expectation of the absolute value of $Y_N$, we have
\begin{align*}
    \mathbb{E}|Y_N| \leq &\gamma^N \Big( \sum_{t = 0 }^{\infty} \gamma^{t+1} \mathbb{E} | w_{t+N}^T P w_{t+N} |   +  2 \sum_{t = 0 }^{\infty} \gamma^{t+1} \mathbb{E}| w_{t+N}^T P A_K^{t+N+1}x|\nonumber \\
    &+ 2 \sum_{t = 0 }^{\infty} \gamma^{t+1} \mathbb{E} |w_{t+N}^T P \sum_{\tau=0}^{t+N-1} A_K^{t+N-\tau}w_{\tau} |  \Big) .
\end{align*}
We handle the terms in the above inequality one by one. For the first term, we have that
\begin{align}\label{eq:upp_bound_temp4}
    \sum_{t = 0 }^{\infty} \gamma^{t+1} \mathbb{E} | w_{t+N}^T P w_{t+N} | \leq \sum_{t = 0 }^{\infty} \gamma^{t+1}  \mathbb{E}| \lambda_{\max}(P) w_{t+N}^T w_{t+N} | \leq \lambda_{\max}(P) \sigma_0^2 \frac{\gamma}{1-\gamma}.
\end{align}
For the second term, we have that
\begin{align}\label{eq:upp_bound_temp5}
    &2 \sum_{t = 0 }^{\infty} \gamma^{t+1} \mathbb{E}| w_{t+N}^T P A_K^{t+N+1}x|\leq  2\mu  \sum_{t = 0 }^{\infty} \gamma^{t+1} \left\| P \right\|_2 \left\| A_K^{t+N+1} \right\|_2 \left\|x \right\|_2 \nonumber \\
    \leq& 2\mu  \sum_{t = 0 }^{\infty} \gamma^{t+1} \left\| P\right\|_2 \rho_K^{t+N-1} \left\|x \right\|_2 \leq 
    2\mu \left\| P\right\|_2 |x|\frac{\gamma \rho_K^{N-1}}{1-\gamma \rho_K} \leq 2\mu \left\| P\right\|_2 |x|\frac{\gamma}{1-\gamma \rho_K},
\end{align}
where the second inequality is due to the fact that $\left\|A_K^{t+N+1} \right\|_2\leq (\left\|A_K \right\|_2)^{t+N+1} \leq \rho_K^{t+N+1}$ and the last inequality follows from the fact that $N\geq 1$.
% We define $S:=\sum_{t = 0 }^{\infty} \gamma^{t+1} A_K^{t+N+1}$. Multiplying both sides by $\gamma A_K$, we obtain
% \begin{align}\label{eq:upp_bound_temp6}
%     \gamma A_K S =\sum_{t = 0 }^{\infty} \gamma^{t+2} A_K^{t+N+2}\mathop{=}^{k=t+1} \sum_{k = 1 }^{\infty} \gamma^{k+1} A_K^{k+N+1}= S - \gamma A_K^{N+1}.
% \end{align}
% Since $A_K$ is stable, we have that $I-\gamma A_K$ is full rank and invertible. Rearranging the terms in \eqref{eq:upp_bound_temp6}, we obtain $S=\gamma (I-\gamma A_K)^{-1} A_K^{N+1}$. Substituting $S$ into \eqref{eq:upp_bound_temp5}, it gives that
% \begin{align}\label{eq:upp_bound_temp7}
%     &2 \sum_{t = 0 }^{\infty} \gamma^{t+1} \mathbb{E}| w_{t+N}^T P A_K^{t+N+1}x|\leq  2\gamma \mu \boldsymbol{1}^T P (I-\gamma A_K)^{-1}  A_K^{N+1}  |x| 
% ;;[----\end{align}
For the third term, we have that
\begin{align}\label{eq:upp_bound_temp8}
    &2 \sum_{t = 0 }^{\infty} \gamma^{t+1} \mathbb{E} |w_{t+N}^T P \sum_{\tau=0}^{t+N-1} A_K^{t+N-\tau}w_{\tau} | 
    \leq   2 \sum_{t = 0 }^{\infty} \gamma^{t+1} \mathbb{E} \left[ \left\| w_{t+N}^T\right\|_2  \left\| P \right\|_2  \left\|\sum_{\tau=0}^{t+N-1} A_K^{t+N-\tau} w_{\tau}\right\|_2 \right] \nonumber \\
    \leq &  2 \mu \left\| P \right\|_2  \sum_{t = 0 }^{\infty} \gamma^{t+1}   \mathbb{E} \left[ \left\|\sum_{\tau=0}^{t+N-1} A_K^{t+N-\tau} w_{\tau}\right\|_2 \right]
    \leq  2 \mu \left\| P \right\|_2  \sum_{t = 0 }^{\infty} \gamma^{t+1}   \mathbb{E} \left[ \sum_{\tau=0}^{t+N-1} \left\|A_K^{t+N-\tau} \right\|_2\left\|w_{\tau}\right\|_2 \right] \nonumber\\
    \leq  &2 \mu^2 \left\| P \right\|_2 \sum_{t = 0 }^{\infty} \gamma^{t+1}  \sum_{\tau=0}^{t+N-1}  \rho_K^{t+N-\tau} \leq 2 \mu^2 \left\| P \right\|_2 \sum_{t = 0 }^{\infty} \gamma^{t+1} \frac{\rho_K}{1-\rho_K} \leq 2 \mu^2 \left\| P \right\|_2 \frac{\gamma \rho_K}{(1-\gamma)(1-\rho_K)},
\end{align}
where the second inequality is due to the fact that $w_{\tau}$ and $w_{t+N}$ are independent and the  second to last inequality follows from the fact that $\sum_{\tau=0}^{t+N-1}  \rho_K^{t+N-\tau} = \sum_{\tau=1}^{t+N} \rho_K^{\tau} \leq \frac{\rho_K}{1-\rho_K} $.
Combining \eqref{eq:upp_bound_temp4}, \eqref{eq:upp_bound_temp5} and \eqref{eq:upp_bound_temp8}, we have that
\begin{align}
    &\sup_{z}|{F}^{K}_x(z)-{F}^{K}_{x,N}(z) | \leq  L_0 \mathbb{E}|Y_N| \nonumber \\
    \leq & L_0 \gamma^N \Big(\lambda_{\max}(P) \sigma_0^2 \frac{\gamma}{1-\gamma} + 2\mu \left\| P\right\|_2 |x|\frac{\gamma}{1-\gamma \rho_K}
    +  2 \mu^2 \left\| P \right\|_2 \frac{\gamma \rho_K}{(1-\gamma)(1-\rho_K)} \Big) 
    :=  C \gamma^N . \nonumber 
\end{align}
The proof is complete and also yields the expression of the constant $C$. 
\end{proof}

\begin{remark}
The bound on the distribution approximation in \eqref{eq:approximat:bound} relies on the conditions of Theorem~\ref{Theorem:approx}, which ensure that the PDF of $G^K_N$ is continuous and bounded. Note that these conditions are not particularly strict, and indeed hold for many  noise distributions commonly used in linear dynamical systems, including Gaussian and uniform. Future work will investigate relaxations of these conditions. 
\end{remark}


\subsection{Numerical Experiments on Quality of the Approximation of the Return Distribution}\label{Sec:NemVer}

In the following experiment, we consider a scalar model with matrices $A=B=1$. Similarly, the weighting matrices in the LQR cost are chosen as $Q=R=1$. The exogenous disturbances are standard normal distributions with zero mean. 

% \textcolor{red}{[With some further thoughts, some reviewers might complain that we are comparing our approximation versus another approximation, assuming the latter is closer to the true return distribution. So, they may say, why do we need the truncated distributions to begin with? We should eprhaps argue more for the value of the truncated approximation: like, the MC one is not formal/analytical and so we cannot really use it any further (plus it comes only with a statistical error, whereas we have bounds!); or it takes so many samples to be a good approximation; or similar arguments (see related discussion on sample-based approximation for DRL in the introduction) .... by the way, is it not possible to obtain an analytical form of the true return distribution even in the scalar case? if so this is a good display of how unwieldy such form is in general.]}
% %
Even for this scalar system, it is impossible to simplify the expression of the exact return distribution, which still depends on an infinite number of random variables.  
Thus, as a baseline for the return distribution, we generate an empirical distribution that approximates the true distribution of the random return. More specifically, we use the Monte Carlo (MC) method to obtain 10000 samples of the random return and use the sample frequency over evenly-divided regions as an approximation of the probability density function. According to the law of large numbers, the empirical distribution approaches the real one as the  number of trials increases. 
Note that, although the MC method provides an alternative way to approximate the return distribution, it relies on using sufficiently many samples that can be time-consuming, and its (statistical) approximation error is generally difficult to analyse. Thus, the MC method is not applicable for practical policy evaluation of distributional LQR, and in this experiment, it is used only to verify our approximate return distribution.
In comparison, the approximate return distribution using finite number of random variables in this paper is analytical for policy evaluation and the corresponding approximation error can be bounded: as such, it is further usable for policy optimisation, as shown in Section~\ref{Sec:riskaversecontrol}. We denote here by $f_N$ the distribution of the approximated random return $G^K_N(x_0)$ obtained considering $N$ random variables. 
%as it takes too many samples to obtain a good estimate and no theoretical bound of the estimation error can be provided. In this experiment, the MC method with sufficient samples, which is extremely time-consuming, is used only for the verification of our approximation of the return distribution.}
% We thus consider the empirical distribution return obtained from the Monte Carlo method as the baseline. 


\begin{figure}[t]
    \centering
	\subfigure[$\gamma=0.6$, $x_0=1$.]{
	\includegraphics[scale=0.24]{figures/S1_F1}}
	\subfigure[$\gamma=0.8$, $x_0=1$.]{
	\includegraphics[scale=0.24]{figures/S1_F2}}
        \subfigure[$\gamma=0.6$, $x_0=8$.]{
	\includegraphics[scale=0.24]{figures/S1_F3}}
        \subfigure[$\gamma=0.85$, $x_0=8$.]{
	\includegraphics[scale=0.24]{figures/S1_F4}}
	\caption{Return distribution and its approximation with finite number of random variables for different $\gamma$ and $x_0$. MC denotes the distribution returned by the Monte Carlo method and $f_N$ denotes the distribution of the approximated random return $G^K_N(x_0)$.}\label{fig:S1}
 \vspace{-0.2in}
\end{figure}



We fix the feedback gain as $K= -0.4684$ and select different values of $\gamma$ and $x_0$. The results are shown in Fig.~\ref{fig:S1}. Specifically,  Fig.~\ref{fig:S1} (a) and (c) show that when $\gamma$ is small, the return distribution can be well approximated using only few random variables ($N=3$ works well). 
%
However, when $\gamma$ approaches $1$, more random variables are needed for an accurate approximation: we employ $N=15$ and $N=20$ random variables in the case of $\gamma=0.8$ and $\gamma=0.85$, respectively, as shown in Fig.~\ref{fig:S1} (b) and (d). Moreover, the value of the initial state $x_0$ has an influence on the shape of the return distribution, which can be clearly observed from the scalar case. When $x_0$ is large, the random variable $w_k^T P A_K^{k+1}x_0$ dominates and, therefore, its distribution is close to a Gaussian distribution, as shown in Fig.~\ref{fig:S1} (c) and (d). If instead $x_0$ is  small, then the random variable $w_k^T P w_k$ plays a leading role, so the overall distribution is close to the chi-square one, as shown in Fig.~\ref{fig:S1} (a) and (b). 
In conclusion,  when $N$ is large, the approximate distribution is closer to the distribution obtained from the MC method, and thus to the true distribution. 


\section{Application to Risk-Averse LQR}\label{Sec:riskaversecontrol}
In this section, we consider a risk-averse LQR problem and leverage the closed-form expression of the random return $G^K(x)$ to obtain an optimal policy. Since the distribution of the random return $G^K(x)$ consists of an infinite number of random variables, it is computationally unwieldy. 
Instead, we employ the approximate random return $G^K_N(x)$ proposed in Section~\ref{Sec:Approxreturn}.
As a risk measure for the problem at hand, 
we select the well-known Conditional Value at Risk (CVaR) \citep{rockafellar2000optimization}. 
We then construct an approximate risk-averse objective function, as $\hat{\mathcal{C}}_N(K):={\rm{CVaR}}_{\alpha}\left[ {G}_N^K(x)\right]$.  
For a random variable $Z$ with the CDF $F$ and a risk level $\alpha \in (0,1]$, the ${\rm{CVaR}}$ value is defined as ${\rm{CVaR}}_{\alpha}[Z] = \mathbb{E}_F[Z| Z>Z^{\alpha}]$, where $Z^{\alpha}$ is the $1-\alpha$ quantile of the distribution of the random variable $Z$. 
Given this objective function, the goal is to find the optimal risk-averse controller, that is, to select the feedback gain $K$ that minimises $\hat{\mathcal{C}}_N(K)$. 

\subsection{Risk-Averse Policy Gradient Algorithm}

 %We use the Conditional Value at Risk (CVaR) as the risk measure and construct the risk-averse objective function as $\mathcal{C}(K) := {\rm{CVaR}}_{\alpha}\left[ G^K(x)\right]$. The goal is to use policy gradient method to find the optimal risk-averse controller, that is,  selecting the feedback gain $K$ that minimizes   $\mathcal{C}(K)$.  Since the random return $G^K(x)$ consists of infinite numbers of random variables, its distribution is computationally intractable. Instead, we use the approximated random return $G^K_N(x)$ with a finite numbers of random variables to construct an approximated risk-averse objective function $\hat{\mathcal{C}}_N(K):={\rm{CVaR}}_{\alpha}\left[ {G}_N^K(x)\right]$. Then,  we  consider an alternative problem:  $\min_{K} \hat{\mathcal{C}}_N(K)$. 

In what follows, we propose a policy gradient method to solve this problem. We assume that the matrices $A,B,Q,R$ are known. The first-order gradient descent step is hard to compute as it hinges on the gradient of the CVaR function. Therefore, we rely on zeroth-order optimisation to derive the policy gradient, as detailed in Algorithm~\ref{alg:algorithm_PG}. 

%Policy gradient methods can be used to solve this problem. We assume that the matrices $A,B,Q,R$ are known. However, the first-order gradient descent is still intractable due to the fact that the gradient of the CVaR is hardly computable. Therefore, we use the zeroth-order optimization to implement policy gradient, which is detailed in Algorithm~\ref{Alg:zeroopt}. 

%Random variable $G(K) := G^{\pi}(x) $, ${G}_N(K):={G}^{\pi}_N(x)$. 



\begin{algorithm}[t]
\caption{Risk-Averse Policy Gradient}
\begin{algorithmic}[1]
\REQUIRE  initial values $K_0$, $x$, step size $\eta$, smoothing parameter $\delta$, and dimension $n$
    \FOR {$episode \; t=1,\ldots,T$}
        \STATE Sample $\hat{K}_t = K_t + U_t$, where $U_t$ is drawn at random over matrices whose norm is $\delta$;
        \STATE Compute the distribution of the random variable ${G}_N^{\hat{K}_t}$;
        \STATE Compute $\hat{\mathcal{C}}_N(\hat{K}_t)$;
        \STATE $K_{t+1}= K_t - \eta g_t$, where $g_t= \frac{n}{\delta^2} \Big(\hat{\mathcal{C}}(\hat{K}_t)- \hat{\mathcal{C}}(\hat{K}_{t-1}) \Big) U_t $.
    \ENDFOR
\end{algorithmic}\label{alg:algorithm_PG}
\end{algorithm}

%


Specifically, at each episode $t$, we sample an approximate feedback gain $\hat{K}_t = K_t +U_t$, where $U_t$ is drawn uniformly at random from the set of matrices with norm $\delta$. Given $\hat{K}_t$, we compute the approximate distribution of the random return ${G}_N^{\hat{K}_t}(x)$ in \eqref{eq:approxreturn} and the value of $\hat{\mathcal{C}}_N(\hat{K}_t)$. Then, we can perform the feedback gain update as
  $K_{t+1}= K_t - \eta g_t$,
where $g_t= \frac{n}{\delta^2} \Big(\hat{\mathcal{C}}(\hat{K}_t)- \hat{\mathcal{C}}(\hat{K}_{t-1}) \Big) U_i $. Here, the zeroth-order residual feedback technique proposed in \citet{zhang2022new} is used to reduce the variance.
The theoretical analysis of this algorithm is left as our future work.




\begin{figure}[t]
    \centering
	\subfigure[The $K$ values  when $\alpha=1$.]{
	\includegraphics[scale=0.24]{figures/S2_F1}}
	\subfigure[The ${\rm{CVaR}}$ values  when $\alpha=1$.]{
	\includegraphics[scale=0.24]{figures/S2_F2}}
        \subfigure[The $K$ values  when $\alpha=0.4$.]{
	\includegraphics[scale=0.24]{figures/S2_F3}}
        \subfigure[The ${\rm{CVaR}}$ values when $\alpha=0.4$.]{
	\includegraphics[scale=0.24]{figures/S2_F4}}
	\caption{Risk-averse control  using Algorithm~\ref{alg:algorithm_PG}. The solid lines are averages over 20 runs.} \label{fig:S2}
\end{figure}

\subsection{Numerical Experiments}
Next, we consider a risk-averse LQR  problem and experimentally illustrate the performance of Algorithm~\ref{alg:algorithm_PG}. We illustrate our approach for the same  scalar system with the same cost function as in Section~\ref{Sec:NemVer}. 
The other parameters are selected as $\gamma=0.6$, $\delta=0.1$, $\eta=0.0004$, $N= 10$, respectively. The initial controller is set as $K_0=-0.2$, which is a stable one.


We first set $\alpha=1$: in this case, the risk-averse control problem is reduced to a risk-neutral control problem. Therefore, we can use traditional LQR techniques to compute the optimal feedback gain $K^{*}=-0.4684$. 
We run the proposed risk-averse policy gradient Algorithm~\ref{alg:algorithm_PG} and the simulation results are presented in Fig.~\ref{fig:S2} (a) and (b). Specifically, in Fig.~\ref{fig:S2} (a), the controller $K$ returned by Algorithm 1 converges to $K^{*}$, which verifies our proposed method for the risk-neutral case. Fig.~\ref{fig:S2} (b) illustrates the values of ${\rm{CVaR}}$ achieved by Algorithm~\ref{alg:algorithm_PG}.
%
Additionally, we select $\alpha=0.4$ to find the optimal risk-averse controller. The simulation results are presented in Fig.~\ref{fig:S2} (c) and (d). We see that $K$  converges to $-0.55$, which leads to a smaller $A+BK$ compared to $K^{*}=-0.4684$. 





% \begin{figure}[htbp]
%     \centering
% 	\subfigure[The values of $K$.]{
% 	\includegraphics[scale=0.3]{figures/S2_F3.eps}}
% 	\quad
% 	\subfigure[The values of ${\rm{CVaR}}$.]{
% 	\includegraphics[scale=0.3]{figures/S2_F4.eps}}
% 	\caption{The simulation result achieved by Algorithm~\ref{alg:algorithm_PG} when $\alpha=0.4$. The solid lines are averages over 20 runs.} \label{fig:S2-risk}
% \end{figure}



% {\color{blue} It would be better to explain why smaller $\alpha$ leaders to smaller $K$}

\section{Conclusion}
In conclusion, we introduce a cost aggregation framework for open-vocabulary semantic segmentation, aggregating the cosine-similarity scores between image and text embeddings of CLIP. Through our \ours framework, we fine-tune the encoders of CLIP for its adaptation for the downstream task of segmentation. Our method surpasses the previous state-of-the-art in standard benchmarks and also in scenarios with a vast domain difference. The success in diverse domains underscores the promise and potential of our cost aggregation framework in advancing the field of open-vocabulary semantic segmentation.\\
\vspace{-10pt}\paragraph{Acknowledgement.} This research was supported by the MSIT, Korea (IITP-2023-2020-0-01819, RS-2023-00266509).



\acks{This work is supported in part by the Knut and Alice Wallenberg Foundation, the Swedish Strategic Research Foundation, the Swedish Research Council,  AFOSR under award \#FA9550-19-1-0169, and  NSF under award CNS-1932011.}

%\bibliography{bibfile}

\begin{thebibliography}{22}
	\providecommand{\natexlab}[1]{#1}
	\providecommand{\url}[1]{\texttt{#1}}
	\expandafter\ifx\csname urlstyle\endcsname\relax
	\providecommand{\doi}[1]{doi: #1}\else
	\providecommand{\doi}{doi: \begingroup \urlstyle{rm}\Url}\fi
	
	\bibitem[Barth-Maron et~al.(2018)Barth-Maron, Hoffman, Budden, Dabney, Horgan,
	Tb, Muldal, Heess, and Lillicrap]{barth2018distributed}
	Gabriel Barth-Maron, Matthew~W Hoffman, David Budden, Will Dabney, Dan Horgan,
	Dhruva Tb, Alistair Muldal, Nicolas Heess, and Timothy Lillicrap.
	\newblock Distributed distributional deterministic policy gradients.
	\newblock \emph{arXiv preprint arXiv:1804.08617}, 2018.
	
	\bibitem[Bellemare et~al.(2017)Bellemare, Dabney, and
	Munos]{bellemare2017distributional}
	Marc~G Bellemare, Will Dabney, and R{\'e}mi Munos.
	\newblock A distributional perspective on reinforcement learning.
	\newblock In \emph{Proceedings of International Conference on Machine
		Learning}, pages 449--458. PMLR, 2017.
	
	\bibitem[Bellemare et~al.(2023)Bellemare, Dabney, and Rowland]{bdr2022}
	Marc~G. Bellemare, Will Dabney, and Mark Rowland.
	\newblock \emph{Distributional Reinforcement Learning}.
	\newblock MIT Press, 2023.
	\newblock \url{http://www.distributional-rl.org}.
	
	\bibitem[Chapman and Lessard(2021)]{chapman2021toward}
	Margaret~P Chapman and Laurent Lessard.
	\newblock Toward a scalable upper bound for a {CVaR-LQ} problem.
	\newblock \emph{IEEE Control Systems Letters}, 6:\penalty0 920--925, 2021.
	
	\bibitem[Dabney et~al.(2018{\natexlab{a}})Dabney, Ostrovski, Silver, and
	Munos]{dabney2018implicit}
	Will Dabney, Georg Ostrovski, David Silver, and R{\'e}mi Munos.
	\newblock Implicit quantile networks for distributional reinforcement learning.
	\newblock In \emph{Proceedings of International Conference on Machine
		Learning}, pages 1096--1105. PMLR, 2018{\natexlab{a}}.
	
	\bibitem[Dabney et~al.(2018{\natexlab{b}})Dabney, Rowland, Bellemare, and
	Munos]{dabney2018distributional}
	Will Dabney, Mark Rowland, Marc Bellemare, and R{\'e}mi Munos.
	\newblock Distributional reinforcement learning with quantile regression.
	\newblock In \emph{Proceedings of AAAI Conference on Artificial Intelligence},
	volume~32, 2018{\natexlab{b}}.
	
	\bibitem[Dean et~al.(2020)Dean, Mania, Matni, Recht, and Tu]{dean2020sample}
	Sarah Dean, Horia Mania, Nikolai Matni, Benjamin Recht, and Stephen Tu.
	\newblock On the sample complexity of the linear quadratic regulator.
	\newblock \emph{Foundations of Computational Mathematics}, 20\penalty0
	(4):\penalty0 633--679, 2020.
	
	\bibitem[Fazel et~al.(2018)Fazel, Ge, Kakade, and Mesbahi]{fazel2018global}
	Maryam Fazel, Rong Ge, Sham Kakade, and Mehran Mesbahi.
	\newblock Global convergence of policy gradient methods for the linear
	quadratic regulator.
	\newblock In \emph{Proceedings of International Conference on Machine
		Learning}, pages 1467--1476. PMLR, 2018.
	
	\bibitem[Kim and Yang(2021)]{kim2021distributional}
	Kihyun Kim and Insoon Yang.
	\newblock Distributional robustness in minimax linear quadratic control with
	{Wasserstein} distance.
	\newblock \emph{arXiv preprint arXiv:2102.12715}, 2021.
	
	\bibitem[Kishida and Cetinkaya(2022)]{kishida2022risk}
	Masako Kishida and Ahmet Cetinkaya.
	\newblock Risk-aware linear quadratic control using conditional value-at-risk.
	\newblock \emph{IEEE Transactions on Automatic Control}, 2022.
	
	\bibitem[Li et~al.(2021)Li, Tang, Zhang, and Li]{li2021distributed}
	Yingying Li, Yujie Tang, Runyu Zhang, and Na~Li.
	\newblock Distributed reinforcement learning for decentralized linear quadratic
	control: A derivative-free policy optimization approach.
	\newblock \emph{IEEE Transactions on Automatic Control}, 2021.
	
	\bibitem[Malik et~al.(2019)Malik, Pananjady, Bhatia, Khamaru, Bartlett, and
	Wainwright]{malik2019derivative}
	Dhruv Malik, Ashwin Pananjady, Kush Bhatia, Koulik Khamaru, Peter Bartlett, and
	Martin Wainwright.
	\newblock Derivative-free methods for policy optimization: guarantees for
	linear quadratic systems.
	\newblock In \emph{Proceedings of 22nd International Conference on Artificial
		Intelligence and Statistics}, pages 2916--2925. PMLR, 2019.
	
	\bibitem[Rockafellar et~al.(2000)Rockafellar, Uryasev,
	et~al.]{rockafellar2000optimization}
	R~Tyrrell Rockafellar, Stanislav Uryasev, et~al.
	\newblock Optimization of conditional value-at-risk.
	\newblock \emph{Journal of Risk}, 2:\penalty0 21--42, 2000.
	
	\bibitem[Singh et~al.(2020)Singh, Zhang, and Chen]{singh2020improving}
	Rahul Singh, Qinsheng Zhang, and Yongxin Chen.
	\newblock Improving robustness via risk averse distributional reinforcement
	learning.
	\newblock In \emph{Proceedings of Learning for Dynamics and Control
		Conference}, pages 958--968. PMLR, 2020.
	
	\bibitem[Singh et~al.(2022)Singh, Lee, and Chen]{singh2022sample}
	Rahul Singh, Keuntaek Lee, and Yongxin Chen.
	\newblock Sample-based distributional policy gradient.
	\newblock In \emph{Proceedings of Learning for Dynamics and Control
		Conference}, pages 676--688. PMLR, 2022.
	
	\bibitem[Tang et~al.(2019)Tang, Zhang, and Salakhutdinov]{tang2019worst}
	Yichuan~Charlie Tang, Jian Zhang, and Ruslan Salakhutdinov.
	\newblock Worst case policy gradients.
	\newblock \emph{arXiv preprint arXiv:1911.03618}, 2019.
	
	\bibitem[Tsiamis et~al.(2021)Tsiamis, Kalogerias, Ribeiro, and
	Pappas]{tsiamis2021linear}
	Anastasios Tsiamis, Dionysios~S Kalogerias, Alejandro Ribeiro, and George~J
	Pappas.
	\newblock Linear quadratic control with risk constraints.
	\newblock \emph{arXiv preprint arXiv:2112.07564}, 2021.
	
	\bibitem[Tu and Recht(2018)]{tu2018least}
	Stephen Tu and Benjamin Recht.
	\newblock Least-squares temporal difference learning for the linear quadratic
	regulator.
	\newblock In \emph{Proceedings of International Conference on Machine
		Learning}, pages 5005--5014. PMLR, 2018.
	
	\bibitem[Van~Parys et~al.(2015)Van~Parys, Kuhn, Goulart, and
	Morari]{van2015distributionally}
	Bart~PG Van~Parys, Daniel Kuhn, Paul~J Goulart, and Manfred Morari.
	\newblock Distributionally robust control of constrained stochastic systems.
	\newblock \emph{IEEE Transactions on Automatic Control}, 61\penalty0
	(2):\penalty0 430--442, 2015.
	
	\bibitem[Yaghmaie et~al.(2022)Yaghmaie, Gustafsson, and
	Ljung]{yaghmaie2022linear}
	Farnaz~Adib Yaghmaie, Fredrik Gustafsson, and Lennart Ljung.
	\newblock Linear quadratic control using model-free reinforcement learning.
	\newblock \emph{IEEE Transactions on Automatic Control}, 2022.
	
	\bibitem[Zhang et~al.(2022)Zhang, Zhou, Ji, and Zavlanos]{zhang2022new}
	Yan Zhang, Yi~Zhou, Kaiyi Ji, and Michael~M Zavlanos.
	\newblock A new one-point residual-feedback oracle for black-box learning and
	control.
	\newblock \emph{Automatica}, 136:\penalty0 110006, 2022.
	
	\bibitem[Zheng et~al.(2021)Zheng, Furieri, Kamgarpour, and Li]{zheng2021sample}
	Yang Zheng, Luca Furieri, Maryam Kamgarpour, and Na~Li.
	\newblock Sample complexity of linear quadratic {G}aussian {(LQG)} control for
	output feedback systems.
	\newblock In \emph{Proceedings of Learning for Dynamics and Control
		Conference}, pages 559--570. PMLR, 2021.
	
\end{thebibliography}


\end{document}
