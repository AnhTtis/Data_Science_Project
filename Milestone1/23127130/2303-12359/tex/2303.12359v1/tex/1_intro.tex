\section{Introduction}
Simulating the dynamic properties of large-scale spiking neural networks is challenging due to the massively parallel interactions of their neurons and synapses.
The BrainScaleS neuromorphic architecture proposes a solution to this dilemma by providing inherently parallel computation at nodes operating as neurons and synapses and communicating through asynchronous spikes.
It thereby achieves a constant emulation speed with increasing network sizes~\cite{bruederle10simulator}. %

BrainScaleS implements physical models of neurons and synapses on a CMOS substrate with analog circuits, while the spike communication is digital.
On the one hand, the physical models inherently provide solutions to neuron and synapse dynamics in continuous time, in contrast to the time-discretized and numerically integrated solutions of digital systems and software simulations.
On the other hand, the programmable digital communication of action potentials allows for flexible network topologies and the possibility of using digital logic to feed and read spike events from outside the system.
Furthermore, circuits are operated in strong inversion, targeting dynamics with a typical speedup factor of \num{10000} compared to biological real-time.

The \acrlong{bss1} system utilizes wafer-scale integration to achieve large ASIC counts with energy efficiency and high communication bandwidth.
The structure of its underlying neuromorphic chip and the technology to achieve its wafer-scale integration are introduced in~\cite{schemmel_ijcnn2008,fieres_ijcnn2008,schemmel2010iscas,millner2010vlsi}.
Turning the silicon wafer into a ready-to-use system, though, implicates bringing several additional components, shown in \cref{fig:wafer_module}, to work hand in hand.
For that cause, a commissioning chain is established, which is this paper's focus.

We first illustrate the different components that constitute the system and how they are tested.
Then, we show the steps to assemble the module before it is finally placed in the machine room, as shown in~\cref{fig:machine_room}.
In the second part of the paper, we describe the methods devised to bring such a system into a reliable substrate for neuromorphic experiments: a large number of VLSI analog components inevitably leads to malfunctioning parts and analog variability, for which an underlying fault-tolerant design and suitable handling have to be put in place.
To demonstrate its operation and the successful implementation of these measures, a biologically-motivated network of spiking neurons, a \acrlong{sfc}, is emulated on a fully commissioned \acrlong{bss1} wafer module.

The system belongs to the still-nascent field of neuromorphic computing and remains under continuous development.
Having pioneered a neuromorphic wafer-scale integration of VLSI analog and digital circuits, we also discuss the lessons learned while solving or circumventing the challenges faced along the way.
