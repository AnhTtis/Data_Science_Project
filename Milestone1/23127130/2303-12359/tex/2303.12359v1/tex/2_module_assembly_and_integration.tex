\section{System Components and Individual Tests}
\label{sec:system_components}

A \acrlong{bss1} wafer module is depicted in~\cref{fig:wafer_module}.
Each of its constituent boards is individually tested before its integration into the system, which permits differentiating errors in the parts from those arising from the assembly.
A short description of each component and the tests it undergoes is given in the following.

\begin{figure}
  \hfill
  \subfloat[\label{wafer_module:exploded_view}]{\def\svgwidth{.5\linewidth}\input{img/wmod_explo_rend05_annotated.pdf_tex}}
  \hfill
  \subfloat[\label{wafer_module:picture}]{\includegraphics[height=0.2\textheight]{img/wmod.png}}
  \hspace*{\fill}
  \caption{\label{fig:wafer_module}
    \protect\subref{wafer_module:exploded_view} 3D-schematic of a \acrlong{bss1} wafer module (dimensions: \SI{50}{\centi\meter} \texttimes{} \SI{50}{\centi\meter} \texttimes{} \SI{15}{\centi\meter}) hosting the wafer~(A) and \num{48} communication boards~(B).
    The positioning mask~(C) aligns elastomeric connectors that link the wafer to the large Main PCB~(D).
    Support PCBs provide power supply~(E \& F) for the on-wafer circuits as well as access~(G) to analog dynamic variables such as neuron membrane voltages.
    The connectors for inter-wafer and off-wafer/host connectivity (\num{48} \texttimes{} Gigabit-Ethernet) are distributed over all four edges~(H) of the Main PCB.
    Mechanical stability is provided by an aluminum frame~(I).
    \protect\subref{wafer_module:picture} Photograph of a fully assembled wafer module.
    Taken from~\cite{schmitt2017hwitl}.
  }
\end{figure}

\begin{figure}
  \centering
  \includegraphics[width=0.8\linewidth]{img/racks.jpg}
  \caption{\label{fig:machine_room}The \acrlong{bss1} machine room comprising \num{20} wafer modules organized in \num{5} racks.
  A slot in the middle of each rack hosts the \acrlong{anarm} and the \acrlongpl{macu} of its neighboring wafer modules.
  Gigabit-Ethernet cables connect each wafer module via aggregation switches to the control cluster positioned in the middle rack.
  Taken from~\cite{schmitt2017hwitl}.
  }
\end{figure}

\begin{figure}[!ht]
  \subfloat[\label{fig:wafer}]{%
    \begin{overpic}[width=0.48\linewidth]{img/wafer.jpg}
    \end{overpic}
  }
  \subfloat[\label{fig:main_pcb_bot}]{%
    \begin{overpic}[width=0.48\linewidth]{img/mainpcb_bot.jpg}
      \put(45,45){\includegraphics[scale=4]{img/numbers/num_1.pdf}}
      \put(45,5){\includegraphics[scale=4]{img/numbers/num_2.pdf}}
      \put(45,80){\includegraphics[scale=4]{img/numbers/num_2.pdf}}
      \put(5,45){\includegraphics[scale=4]{img/numbers/num_2.pdf}}
      \put(80,45){\includegraphics[scale=4]{img/numbers/num_2.pdf}}
    \end{overpic}
  }
  \caption{
    \protect\subref{fig:wafer} The \acrlong{bss1} wafer with applied postprocessing to achieve
      wafer-scale integration and to establish its connection to the \protect\subref{fig:main_pcb_bot} bottom side of the Main PCB.
      There, the wafer connects through elastometic connectors to the center, marked with 1.
      In the borders, \num{48} connectors, marked with 2, accommodate the communication boards.
  }\label{fig:wafer_and_main_pcb}
\end{figure}

\subsection{The \acrlong{bss1} Wafer}
\label{sec:hicann-wafer}

The heart of each module is an uncut $20\,\text{cm}$ wafer, displayed in~\cref{fig:wafer}, fabricated in UMC $\SI{180}{\nano\meter}$ technology comprising 384 \cgls{hicann} ASICs.
Each \cgls{hicann} contains \num{512} analog neuron circuits implementing the \cgls{adex} model~\cite{millner2010vlsi}.
Single neuron circuits receive input from up to \num{220} analog synapses.
Since neuron membranes can interconnect in groups of up to \num{64}, a maximum of $\num{14080}$ synapses can provide input to each of these composite neurons.
Synapse weights are stored with \num{4}-bit resolution in local SRAM at each synapse.

Each \cgls{hicann} stores $\num{12384}$ analog quantities for parameterization of its analog circuits in \cgls{fg} CMOS cells that retain their operation levels according to their isolated gate's accumulated charge~\cite{Millner2012,lande96}.
These \acrshortpl{fg} are written via an onboard \num{10}-bit-resolution DAC, enabling reprogramming via incremental loops with feedback.
Then, the stored values get translated to either a voltage or a current using a source follower or a current mirror, respectively, to set neuron parameters and other onboard circuit operation levels.
While these \cglspl{fg} present a low-power, small-space solution to store analog operation settings, they introduce write-cycle to write-cycle variability, as will be further discussed.\\
Wafer-wide communication is achieved with a custom-developed redistribution layer applied post-wafer-production, creating around $\num{160000}$ lateral connections across chip borders~\cite{zoschkeguettler2017rdlembedding}.
These connections provide the modules with on-wafer spike event communication through \cgls{lvds} buses utilizing an asynchronous serial event transmission protocol.
Furthermore, connections through top-layer pads on the wafer provide the modules with parallel per \cgls{hicann} off-wafer communication, which in conjunction with programmable and redundant components, constitute the system's fault tolerance~\cite{schemmel_ijcnn2008}.

\textit{Testing:} In order to assess the effect of wafer post-processing on the digital yield of an entire wafer, initial needle card tests were carried out on two unprocessed\footnote{"unprocessed" in this context means untested wafers straight from the manufacturer, before the custom redistribution layers have been added.} wafers to determine their yield immediately after production.
Since the wafers undergoing these tests cannot be further processed, comparing results on the same wafers before and after the post-processing is, however, not possible.

The setup for these tests in the institute's clean room is shown in \cref{fig:clean_room_wafer_prober}, and the procedure is as follows.
The needle card is used to contact each individual ASIC.
Immediately after contacting and powering up, the total current on the used lab supply is measured to detect potential power shorts.
Henceforward, all digital memory cells on the \cgls{hicann} circuits are tested using a built-in \cgls{jtag} access mode.
During these tests, \num{448} \cglspl{hicann} on each of the two wafers were tested, and \SI{93}{\percent} of them showed no single digital error. %
To compare, UMC's calculator estimates a yield of approximately \SI{85}{\percent} by taking into account the process parameters and circuit size.
However, our results are only an estimation:
On the one hand, the tested digital memory cells only cover a fraction of the whole silicon area, which is dominated by analog circuitry.
Therefore, the digital test yield could be assumed to be too optimistic.
On the other hand, perfect power and signal integrity could not be ensured while connecting the circuits through the needles, leading to a possible detection of false negatives, caused for example by slightly underpowered memory cells.
In addition, only wafers from the initial engineering sample production have been available for testing.
No documentation has been available to relate the production yield data from UMC to small batch-size engineering runs.
Nonetheless, the results match the expectations taking the high level of uncertainty into account.
Also, a yield in the order of, e.g., \SI{85}{\percent} would not indicate that \SI{15}{\percent} of the dies cannot be used.
Instead, advantaging from the fault-tolerant design, and depending on the defect type, it could suffice to disable single neuron or synapse circuits, for example, on affected \cglspl{hicann} that are otherwise fully functional and can remain available for experiments.

\begin{figure}
  \centering
  \subfloat[\label{fig:clean_room_setup_2}]{\includegraphics[width=0.48\linewidth]{img/clean_room_setup_2_scaled}}
  \subfloat[\label{fig:wafer_and_needle_card}]{\includegraphics[width=0.48\linewidth]{img/wafer_and_needle_card_scaled}}
  \caption{\protect\subref{fig:clean_room_setup_2} Photograph of the wafer prober and \protect\subref{fig:wafer_and_needle_card}~a close-up of a wafer under test. Different needle cards have been developed and used for tests carried out before wafer post-processing (\cref{sec:hicann-wafer}, visible in this setup) and before wafer module assembly (\cref{sec:tests_installation_steps}), respectively.}
  \label{fig:clean_room_wafer_prober}
\end{figure}

\subsection{Main PCB}
The Main PCB, displayed in~\cref{fig:main_pcb_bot}, is a \SI{43}{\cm}$\,\times\,$\SI{43}{\cm} passive interconnector board for most parts of the wafer-scale integration system.
Seven of its $\num{14}$ layers are used to distribute $\num{23}$ power rails carrying up to \SI{200}{\ampere} of current.
The rest of the layers are used to route $\num{1152}$ power monitoring, $\num{1472}$ high-speed differential communication, and different sideband signals.
Auxiliary boards, communication infrastructure, and the silicon wafer are connected via various kinds of detachable connectors.
These enable system modularity for development and upgrades, desirable for research and development in dynamic environments over longer timespans.

\textit{Testing:} The manufacturer\footnote{Manufactured by \foreignlanguage{ngerman}{Würth Elektronik}, Germany} performs complete optical inspection and electrical tests of the Main PCB. The \acrlong{bss1} wafer modules are assembled using exclusively fully validated, error-free Main PCBs.

\subsection{Auxiliary Boards}
The wafer module is completed by populating it with \num{48} communication boards and auxiliary boards for power delivery,  control, monitoring, and inter-module communication.

\subsubsection{Communication Boards}
Each communication board\footnote{\label{foot:devAtTud}Developed at the chair of \foreignlanguage{ngerman}{Hochparallele VLSI-Systeme und Neuromikroelektronik} at TU~Dresden}
contains a \cgls{fpga} and connects to one \cgls{hicann} group consisting of \num{8} \cglspl{hicann}.
These boards communicate through separate high-speed \cgls{lvds} interfaces with each of the connected \cglspl{hicann} to configure, monitor, and coordinate the experiment runs; they feed and collect generated spikes into/from the experiments.
Furthermore, they synchronize the start of experiments to allow for wafer wide execution.
Trigger signals generated on these boards also align experiments with analog recordings using the \cgls{anarm}.

\textit{Testing:} The communication boards are tested on a standalone setup that implements loopback connections for the high-speed interfaces.
For this purpose, a test board accommodates and tests four PCBs in parallel, as shown in~\cref{fig:fcp_tud}.
Primarily automated and controlled via software, the tests switch the power supply via \cgls{gpib}. %
Programming is performed via \cgls{jtag} and \cgls{pmbus}.
Tests comprising current consumption measurements, loading and communicating with the \cgls{fpga} design, as well as memory tests are conducted.
In addition, communication with the host computer as well as the links to the wafer and neighboring communication boards are tested.
As per data logs, only \num{18} out of $\num{1404}$ produced PCBs had to be discarded after failed tests.

\subsubsection{Wafer I/O PCB}
Each one of the module's four \cglspl{wio}\footref{foot:devAtTud} attaches to twelve communication boards, aggregating Gbit-Ethernet and connections to other communication boards.

\textit{Testing:} A manual approach is followed as the number of boards is smaller than that of the communication boards.
The board, shown in~\cref{fig:wio_test}, is supplied with power, and the proper functioning of the DC/DC converters is checked with a multimeter.
Individual communication ports are tested.
In addition, the proper transmission of signals using a signal generator and differential probes is measured.
A partial test of the JTAG pins is also carried out.
As per data logs, only \num{2} out of \num{120} produced \cglspl{wio} were discarded after failed tests.

\subsubsection{Main Power Supply}
The \cgls{powerit} has three output channels: Two \SI{1.8}{\volt} outputs as main analog and digital supplies of the wafer with a current limit of \SI{200}{\ampere} each, as well as a \SI{9.6}{\volt} output capable of up to \SI{110}{\ampere} to supply the communication boards.
Multiple custom-milled copper parts ensure a low-resistance screw connection between the \cgls{powerit} and the Main PCB.
Additionally, digital control of the voltages and sensors as near to the wafer as possible allow for compensation of IR-drop.
An integrated microcontroller can measure input and output currents and voltages via shunt resistors, hall sensors, and voltage dividers.

\textit{Testing:} Commissioning of the \cgls{powerit} involves basic functionality tests and calibration of the current and voltage measurement circuits using an external electronic load capable of sinking \SI{4.8}{\kilo\watt} and precision multimeters, see~\cref{fig:prober_powerit}.

\begin{figure}
  \centering
  \subfloat[\label{fig:fcp_tud}]{\includegraphics[width=0.48\linewidth]{img/fcp_tud.jpg}}
  \subfloat[\label{fig:wio_test}]{\includegraphics[width=0.48\linewidth]{img/wio_test.jpg}}\\
  \subfloat[\label{fig:prober_powerit}]{\includegraphics[width=0.48\linewidth]{img/powerit_calib_device_2020_cut.jpg}}
  \subfloat[\label{fig:prober_auxpwr}]{\includegraphics[width=0.48\linewidth]{img/auxpwr_teststand.jpg}}\\
  \subfloat[\label{fig:prober_cure}]{\includegraphics[width=0.48\linewidth]{img/cure_teststand.png}}
  \subfloat[\label{fig:prober_adc}]{\includegraphics[width=0.48\linewidth]{img/analog_top.png}}
  \caption{Auxiliary boards under test.
  \protect\subref{fig:fcp_tud} communication boards test setup and \protect\subref{fig:wio_test} \acrlong{wio} board.
  \protect\subref{fig:prober_powerit} \acrlong{powerit} connected to programmable power supply and electronic load.
  \protect\subref{fig:prober_auxpwr} \acrlong{auxpwr} test stand.
  \protect\subref{fig:prober_cure} \acrlong{cure} test stand.
	Each Power Emulation Systems for Testing (PEST) board emulates the supply voltages of one \cgls{hicann} group.
  \protect\subref{fig:prober_adc} FPGA board of the \acrlong{anarm}. During the calibration, the pins on the top left are connected via a \SI{50}{\ohm} impedance to an external source meter, while the module is connected via USB to the host computer.
  Figures \protect\subref{fig:fcp_tud} and \protect\subref{fig:wio_test} made available by S.~Schiefer, TU-Dresden.
  }
  \label{fig:auxiliary_prober}
\end{figure}

\subsubsection{Auxiliary Power Supply}
The \cgls{auxpwr} designed in~\cite{sterzenbach2014auxpwr}, receives \SI{9.6}{\volt} from the \cgls{powerit} and provides ten different voltage outputs for the wafer module.
The currents drawn at the derived voltages vary from \SI{50}{\milli\ampere}, for the common-mode voltage of the \cgls{lvds} on-wafer communication, to \SI{60}{\ampere} for the synapse driver output.
The board has an L-shape with linear and switching regulators placed on different axes to reduce the coils' electromagnetic-noise induction.
In addition, the usage of intermediate voltages reduces the power dissipation for the voltage scaling.
An onboard microcontroller monitors all the voltages and currents.
Four voltages can be controlled digitally through the \cgls{i2c} protocol.

\textit{Testing:} The \cgls{auxpwr} components' functionality is tested during the calibration process of the board, during which an external voltmeter permits adjusting voltage offsets.
A two-point linear calibration under load is performed for the currents.
The test stand can be seen in~\cref{fig:prober_auxpwr}.

\subsubsection{Control Unit for Reticles}
Since the \acrlong{bss1} wafer is not cut into individual chips, the wafer module must be fault-tolerant to individual \cgls{hicann} problems.
For this purpose, the Main PCB features power-FETs for the supply rails of each \cgls{hicann} group of the wafer; overcurrents manifest as a large voltage drop across these power transistors.
The \cgls{cure} controls the gates of these transistors and monitors the supply voltages of the wafer.
Three microcontrollers manage the measured data and react to fault conditions by shutting off the power of the affected \cgls{hicann} groups.
Thus, the \cgls{cure} allows to identify individual fatal faults and to exclude the respective \cgls{hicann} groups from the usable components.
The term reticle refers to the semiconductor manufacturing process and consists of one \cgls{hicann} group.

\textit{Testing:}  The \cgls{cure} is tested using a custom setup producing the voltages expected inside the actual \acrlong{bss1} wafer module, simulating all possible fault conditions while the response time is measured.
Likewise, the drive strength of the control signals for the power transistors on the Main PCB is quantified.
The test setup is displayed in~\cref{fig:prober_cure}.

\subsubsection{Analog Readout Module}
Further insight into the neuron dynamics can be obtained via measurements of its membrane potential, allowing for a better understanding of experiment results and the implementation of calibration routines.
To this end, each neuron contains a switchable analog output amplifier that connects to one of two \SI{50}{\ohm} output buffers per die.
These two outputs are each short-circuited across dies in the same \acrshort{hicann} group. Therefore, each of these groups has two analog outputs, totaling \num{96} independent analog channels available on each wafer module.

The \cgls{anarm} system consists of twelve \cgls{fpga}-controlled \num{12}-bit ADC modules that allow for the digitization of the membrane voltages on one wafer module per \acrlong{bss1} system rack.
Each of the modules in the \cgls{anarm} system connects through a ribbon cable to one of two \cglspl{anab} mounted on the Main PCB, receiving eight analog signals that are multiplexed into the ADC.
An additional digital signal acts as a trigger; four \cgls{hicann} groups share one, allowing synchronization during an experiment between the involved communication boards, \cglspl{hicann} and the \acrshort{anarm} system.
Overall, the \acrshort{anarm} system can simultaneously sample 12 membrane traces per wafer module.

\textit{Testing:} The \acrshort{fpga} board in the \cgls{anarm}, displayed in~\cref{fig:prober_adc}, undergoes DRAM memory tests and basic functional testing of all its peripheral components.
The analog front end is tested during the calibration of the modules.
This calibration is performed using a source meter to generate a series of ground-truth voltages, which are subsequently measured using each input channel.
A \SI{50}{\ohm} series impedance is used at the output of the source meter to match the impedance of the output buffers on the \cgls{hicann}.
This voltage divider formed by the output and input impedances halves the \SI{1.8}{\volt} span of the \cgls{hicann} output to the \SI{0.9}{\volt} maximum input of the \cgls{anarm}.
A linear function fits the recorded signal to the source meter voltages, and the per board offset and gains are stored in a database.

\subsection{Main Control Unit}
The \cgls{macu} consists of a Raspberry Pi powered by the standby voltage of the \cgls{powerit}.
Using the \cgls{i2c} protocol to communicate with all other wafer module components, it controls the start-up sequence of the system.
Additionally, it monitors the multitude of components of a wafer module, which is crucial to ensure robust operation.
With this in mind, the \cgls{macu} aggregates over \num{1800} metrics per wafer, e.g., supply voltages, temperatures, or the active/inactive status of components.
Most data is of a time-series nature and stored via Graphite~\cite{carbon}, with visualization through Grafana dashboards~\cite{grafana}.
These dashboards are hierarchically structured, allowing an intuitive drill-down navigation of the data.
As it is not practical to manually oversee such a large amount of metrics, alerts are set up to check for unexpected events.
For example, supply voltages are checked to be in a valid range and to remain constant over time.
Furthermore, event data, e.g., powering up components, is handled via the ELK stack~\cite{elasticsearch} but also integrated into Grafana and displayed as marks.
These allow easily matching the events with changes in the time-series data.

\textit{Testing:} The Raspberry Pi computers used for the \acrshortpl{macu} are purchased and commissioned without further tests.
However, the maintenance and deployment of the control and monitoring software they run is part of the system's continuous integration development methodology~\cite{mueller2022operating}.

\section{System Assembly and Integration Tests}
\label{sec:system_assembly}

In addition to the tests devised for the individual components, the \acrlong{bss1} wafer module assembly process is carried out along with additional tests that allow pinpointing problems to the individual steps.
In the following, we discuss the module assembly method and the different tests it undergoes during this phase.

\subsection{Wafer to Main PCB Marriage and Module Integration}
The wafer is connected to a total of \num{11904} pads on the Main PCB via \num{384} elastomeric connectors, shown in~\cref{fig:elastomeric_connectors}.
Mounting the Main PCB and the silicon wafer in custom-milled aluminum brackets allows reaching the compression forces required by the connectors.
The station used to align the two components is shown in \cref{fig:wafer_alignment}.
Electrical resistance tests, described in \cref{sec:assembly_tests}, are performed while compressing the elastomeric connectors to ensure correct positioning and even pressure distribution.
Then, the wafer module is populated with the auxiliary boards and, when fully assembled, connected to the \cgls{macu}.
Afterward, it is put on a test stand for initial full-system tests using the same communication chain later used for experiments.
Following this step, the wafer module is placed in a rack in the machine room and attached to the \cgls{anarm} system.

\begin{figure}
  \centering
  \subfloat[\label{fig:elastomeric_connectors}]{\includegraphics[width=0.7\linewidth]{img/elastomer_mask_detail_view.jpg}}\\
  \subfloat[\label{fig:wafer_alignment}]{\includegraphics[width=0.53\linewidth]{img/alignment_station.png}}
  \subfloat[\label{fig:assembly_test_boards}]{\includegraphics[width=0.445\linewidth]{img/assembly_test_boards_2021.jpg}}\\
  \caption{\protect\subref{fig:elastomeric_connectors} Detail view of the elastomeric connectors that connect the pads on the \acrlong{bss1} wafer with the Main PCB.
          \protect\subref{fig:wafer_alignment} Station used to align the Main PCB to the silicon wafer.
           The Main PCB is fixed by springs that apply a constant force (blue arrows).
           Its position is controlled with a micrometer linear stage (red arrows).
           Angular errors can be corrected by rotating the wafer (purple arrows).
           \protect\subref{fig:assembly_test_boards} Test PCBs mounted on the Main PCB to measure the connectivity to the wafer during assembly.}
  \label{fig:alignment}
\end{figure}

\subsection{Tests at Different Assembly Stages}
\label{sec:tests_installation_steps}

Stage-specific tests allow mapping arising errors to individual assembly steps of the \acrlong{bss1} wafer module, which enables evaluating and improving the procedure.
This section shows the test results obtained for one wafer as an example.

\subsubsection{Pre-Assembly Tests of All \acrshortpl{hicann} on the Wafer}
Before placing a wafer in a module, digital and analog tests are performed on a wafer prober in the institute's clean room, see \cref{fig:clean_room_wafer_prober}.
These tests distinguish production problems from those arising in the wafer module assembly procedure.\\
Similar to the initial needle card tests on the unprocessed wafers, described in \cref{sec:hicann-wafer}, a test system was built using a different needle card connecting to the redistribution layer of a pair of \cglspl{hicann} on a wafer with post-processing.
Extended analog and digital tests are run on the connected dies, a process that is repeated until the entire wafer is analyzed.
These tests serve two purposes: first, to sort out wafers with a high error count that might arise from disrupted connections in the post-processing, and second, to establish a base level for the following assembly tests.
\Cref{fig:prober_switchram} shows the results of a high-level test for all \cglspl{hicann} of one wafer.
The image shows more test results than the number of dies on the picture of the assembled wafer module.
The reason for this was design constraints and limited routing resources on the Main PCB, by which not all \cgls{hicann} groups could be electrically connected and thus used within the module context; those at the edge of the wafer were left out.
For the same reason, the two \cgls{hicann} groups at the center are without high-speed connection.

\subsubsection{Tests During the Assembly Phase}
\label{sec:assembly_tests}
For these additional tests the Main PCB is equipped with test PCBs\footnote{Developed by the group of Yasar Gürbüz at Sabanci University, Istanbul}, shown in \cref{fig:assembly_test_boards}, which measure \acrshort{esd} diode currents and termination resistances between the \cgls{lvds} lines on the wafer.
The tests determine whether a good connection of the wafer to the Main PCB exists.
\Cref{fig:assembly} shows the result of one of these tests, where only the same faulty device on \acrshort{hicann} group 29, also detected in the needle card test, can be seen.
No additional faulty devices validate that the wafer to Main PCB marriage was appropriate.

\subsubsection{Post-Assembly Tests of All \acrshortpl{hicann} on the Wafer}
After the assembly of the wafer module is completed, the same tests run on the pre-assembly phase are conducted, and results are compared.
The results for one test are shown in \cref{fig:module_switchram}.
The errors in \acrshort{hicann} groups \num{15} and \num{29} are still present, while the errors in groups \num{36} and \num{42} are not.
Further investigations could trace these last errors to connection problems of the needle card used in the wafer prober.

\begin{figure}
  \centering
  \subfloat[\label{fig:prober_switchram}]{\includegraphics[width=0.5\linewidth]{img/D9NKP14B6_wp_switchram}}
  \subfloat[\label{fig:assembly}]{\includegraphics[width=0.5\linewidth]{img/D9NKP14B6_F01_assembly_cut}}\\
  \subfloat[\label{fig:module_switchram}]{\includegraphics[width=0.5\linewidth]{img/D9NKP14B6_F01_switchram}}
  \hspace{0.003\linewidth}
  \subfloat[\label{fig:commtest}]{\includegraphics[width=0.455\linewidth]{img/communication_test_result.png}}
  \caption{Test results of one \acrlong{bss1} wafer for the different assembly steps:
	\protect\subref{fig:prober_switchram} Before assembly \protect\subref{fig:assembly} during assembly \protect\subref{fig:module_switchram} after assembly.
	In \protect\subref{fig:prober_switchram} and \protect\subref{fig:module_switchram}, the number in the smallest rectangles shows the amount of errors found on the corresponding \cgls{hicann}.
	Purple or red indicate that all tests were successful or failed, respectively.
	For grey \cglspl{hicann} the test was skipped since no connection could be established using the wafer prober.
	In \protect\subref{fig:assembly}, test results are shown per elastomeric connector and a yellow rectangle indicates a problem in the high-speed communication of one \cgls{hicann}.
	\protect\subref{fig:commtest} Communication test result.
	\cglspl{hicann} without high-speed communication are marked yellow, without JTAG communication red.
	The center two \cgls{hicann} groups have no high-speed interface by design.
	Consequently, they are marked faulty in all tests requiring  high-speed communication to the Main PCB.}
  \label{fig:D9NKP14B6_tests}
\end{figure}
