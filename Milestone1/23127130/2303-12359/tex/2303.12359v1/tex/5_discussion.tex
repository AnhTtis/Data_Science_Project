\section{Discussion}

Starting its development more than ten years ago, the first-generation BrainScaleS wafer-scale neuromorphic system represents a milestone toward a large-scale analog neural network emulation platform.
Over years during which several modules have been commissioned and experiments run, we have learned important lessons on building and handling such a complex system.
We discovered drawbacks in our first implementation; some of them could successfully be circumvented via our commissioning software.
Our second-generation neuromorphic \acrlong{bss2} chip~\cite{schemmel2020accelerated} addresses \acrlong{bss1}'s design weaknesses.
Moreover, it enables the application of advanced learning mechanisms by introducing a digital plasticity processor, neuron multi-compartment capabilities, as well as extended analog to digital conversion capacities.

In this paper, we described the individual components of a \acrlong{bss1} wafer module and showed the necessary steps to assemble it.
A wafer-scale analog system is complex and requires many hardware components working concurrently.
Once a wafer module is assembled, it is often not possible to pinpoint defects in individual components.
To alleviate this, each component must get tested on its own; malfunctioning ones must be repaired or replaced before they are added to the system.
Additional tests during the assembly are also crucial to allow for finding and solving errors that arise during that process.
The remaining problems are handled by the exclusion of affected components or circuits from the availability database to ensure the correct operation of the system.

The importance of the tests and monitoring remains after the wafer module gets placed in the rack.
For example, tight monitoring during system operation is necessary to uncover the wear out of system components.
Automated alerts are fundamental for warning in case of values deviating over time.
Furthermore, the tests executed nightly help keep track of the wafer modules' state.

Concerning the wafer in the core of the \acrlong{bss1} system, the probability of fabrication defects in microelectronics is proportional to the circuit area~\cite{werner2016circumvent}.
Thus, it is unfeasible to build such a large analog system without malfunctioning components.
This will most likely further intensify in the future by utilizing novel materials.
With this in mind, the digital tests introduced are executed nightly to identify such malfunctioning components and exclude them from our availability database.
These tests enable storing different states of the database on disk and allow to differentiate actual malfunctioning components from those not usable due to a dependency.
The users can then utilize reliable components, possibly even using a custom availability database.

An additional challenge using analog hardware is the fixed-pattern noise introduced by unavoidable manufacturing process variations.
In the \acrlong{bss1} system, this is worsened by the design decision to use \cglspl{fg} to store the neuron configuration.
These cells allow for long-term storage of analog parameters without storing digital values onboard.
However, the current implementation introduces write-cycle to write-cycle variability. 
Though small, these variations lead to noticeable errors if they are further enlarged by non-linear dependencies between control signal and observable.
To minimize these effects, we presented our calibration framework, which also allows non-expert users to configure experiments in the biological domain without specific knowledge of the hardware.
We demonstrated the narrowing and centering of the achieved value distribution for exemplary parameters after the calibration was applied, limited by thermal noise and the variations caused by the \cglspl{fg}, nonetheless.
Since single-poly floating-gate cells are non-standard devices and not supported by the manufacture, the second-generation \acrlong{bss2} chip reverts to a digital parameter storage scheme employed in a previous neuromorphic architecture~\cite{schemmel_ijcnn04}, thereby vastly improving analog parameter accuracy.
Since the second generation uses a manufacturing process with much smaller geometry, namely \SI{65}{\nano\meter} vs. \SI{180}{\nano\meter}, the area penalty for the digital parameter storage is manageable.
A further advantage of the novel parameter storage is the reduced programming time~\cite{hock13analogmemory}.
In the presented wafer-scale implementation, the single-poly floating-gate parameter storage was the only feasible solution to achieve the required number of analog parameters for the neuron circuits.

On top of explaining the calibration methodology, we demonstrated the necessity for parallel execution of the calibrations.
The large parameter space of the synapse weight calibration exceeds reasonable runtimes using the current readout system.
In order to circumvent this, we introduced a per wafer calibration which, compared to a per circuit calibration, shows larger errors but can be generated in a reasonable time frame.
To improve this, we developed a new readout system, which will replace the external set of ADCs with on-wafer-module boards, increasing the parallel readout capabilities from \num{12} to \num{96} channels~\cite{ilmberger2017masterthesis}.
Moreover, in the \acrlong{bss2} chip, we introduce a per neuron-circuit ADC system, which allows for a massive parallel calibration~\cite{schemmel2020accelerated}.
A per-circuit calibration before each experiment becomes feasible with such a solution.

Finally, we demonstrated the operation of a fully commissioned \acrlong{bss1} wafer module implementing \cglspl{sfc}.
While small chains portray the capability to fine-tune the network parameters, extending to a long chain of \num{190} links illustrates the possibility to scale up networks.
Successfully mapped to an inherently imperfect substrate, it consists of the largest spiking network emulation run with analog components and individual synapses to date.

Our endeavor in developing and maintaining the \acrlong{bss1} system has demonstrated, while illustrating the field's challenges, that building wafer-scale analog neuromorphic hardware is feasible.
Furthermore, the \acrlong{bss1} wafer module with its operating system laid the foundation for the next-generation systems; all lessons learned from the first generation contribute to the success of future large-scale neuromorphic systems.
