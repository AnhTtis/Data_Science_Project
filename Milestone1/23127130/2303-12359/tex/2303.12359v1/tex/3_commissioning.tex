\section{Commissioning Software}\label{sec:commissioning}
After assembly, additional steps are necessary to bring the \acrlong{bss1} wafer module into readiness for experiments.
These include digital tests to find and exclude malfunctioning components and calibrating the individual neurons to address manufacturing-process-induced circuit mismatches.
Databases store the results from these two steps, allowing serialized data storage to disk.
See~\cite{mueller2022operating} for details.
Furthermore, all steps are fully automated and periodically executed after installation of the module in the machine room to track the systems' current state.

\subsection{Communication Tests}
The first test that is executed on a newly assembled wafer module is the communication test, which is used to find unresponsive \cglspl{hicann}.
Communication problems most likely arise from insufficient connection quality between the Main PCB and the wafer, cf.~\cite{zoschkeguettler2017rdlembedding}, or from scratches or similar defects on the post-processing layers.\\
During the test, an individual connection is established to each of the \num{384} \cglspl{hicann} of one wafer.
The test is split into a high-speed test and a \cgls{jtag} test, which reflects the two possibilities to communicate with the \acrshort{hicann}.
Failures are stored separately in the availability database.
The result of a communication test is shown in \cref{fig:commtest}.
In this example, the result comparison between the test stand and the rack-mounted fully assembled wafer module shows one additional \cgls{hicann} group and \num{3} individual \cglspl{hicann} that cannot communicate via \cgls{jtag}.

\subsection{Memory Tests}\label{sec:blacklisting}
Using a whole uncut wafer, each \acrlong{bss1} wafer module profits from better energy efficiency and higher bandwidth for communication between its ASICs as if these were produced separately and then integrated.
This approach presents a challenge, though, as producing an error-free wafer-scale system in such a way is not possible, as ASICs with manufacturing-induced problems cannot be removed.
The \acrlong{bss1} system addresses this through a digital memory test, which in conjunction with the fault-tolerant system design, enables dynamic handling of malfunctioning components.
Executed after assembly as well as periodically, the test also tracks the state of the systems over time.
Therefore, it allows to operate wafer modules despite a subset of malfunctioning components or connections, consequently increasing the yield of functional systems.\\
The test builds upon the communication test and establishes a connection to a \cgls{hicann} group.
First, it initializes the connected communication board and the \cgls{hicann} under test.
Subsequently, each digital memory is repeatedly write/read-tested using random values.
If a mismatch is found, the largest functional unit that depends on the malfunctioning component is excluded so that it is not utilized in experiments.
\cglspl{hicann} that can communicate only via \cgls{jtag} are exclusively used for spike route-through to and from neighboring \acrshortpl{hicann} on the same wafer.
For these, a routing-specific reduced memory test minimizes the runtime using the slower connection.
In total, more than \SI{42}{\mebi\byte} of digital memory get tested per wafer.
Results for a fully assembled wafer module are shown in \cref{tab:blacklists}.
Tested components and their position on the \acrshort{hicann} are visualized in \cref{fig:hicann_overview}.\\
\begin{table}
  \centering
	\caption{Overview of excluded components extracted from a fully assembled \acrlong{bss1} wafer module.
	"Components" shows the number of components taken into account for the tests and the effective exclusion.
	If two numbers are given, the first one is the number of tested components and the second one is the number of components evaluated for the effective exclusion.
	"Individual" lists the communication and memory test results.
	Buses are marked with \mbox{"-"} because they have no digital memory that could be tested.
	"Effective" shows the results of the effective exclusion of components.
	Here, all components that should not be used during an experiment are included.
	They not necessarily failed a test.}
  \label{tab:blacklists}
  \begin{tabular}{llll}
    \toprule
    Resource & Components & \multicolumn{2}{c}{Excluded}\\ &&Individual & Effective \\
    \midrule
    \cGls{jtag} comm. & 384 & \SI{2.86}{\percent} & \SI{3.39}{\percent} \\
    High-speed comm. & 368/384 & \SI{3.26}{\percent} & \SI{7.81}{\percent} \\
    Synapse drivers & 78320 & \SI{0.04}{\percent} & \SI{0.04}{\percent} \\
    Synapse arrays & 712 & \SI{1.97}{\percent} & \SI{1.97}{\percent} \\
    Synapse rows & 159488 & \SI{0.11}{\percent} & \SI{0.11}{\percent} \\
    Synapses & 40099840 & \SI{0.68}{\percent} & \SI{0.68}{\percent} \\
    \cgls{fg} blocks & 1492 & \SI{0.34}{\percent} & \SI{0.34}{\percent} \\
    External input mergers & 2848/2984 & \SI{0.0}{\percent} & \SI{4.83}{\percent} \\
    Analog outputs & 712 & \SI{0.0}{\percent} & \SI{0.0}{\percent} \\
    Background-generators & 2848 & \SI{0.0}{\percent} & \SI{0.0}{\percent} \\
    Mergers & 5340 & \SI{0.0}{\percent} & \SI{0.0}{\percent} \\
    Buses & -/119360 & - & \SI{2.2}{\percent} \\
    Repeaters & 119360 & \SI{0.21}{\percent} & \SI{0.22}{\percent} \\
    Switches & 2864640 & \SI{0.02}{\percent} & \SI{0.02}{\percent} \\
    \bottomrule
  \end{tabular}
\end{table}

\begin{figure}
  \includegraphics[width=1\linewidth]{img/hicann_overview.png}
  \caption{Left: Picture of the \cgls{hicann} with labeled components and marked areas shown on the right side.
  Top right: Detail of the synapse array.
  Two synapse rows are connected to one synapse driver.
  All synapses of the same column are connected to one neuron circuit.
  Middle right: Left half of the merger tree.
  Neuron input from the top gets routed to the buses on the bottom.
  Several inputs can be merged on the same bus.
  Background-generators are used to inject additional signals generated on-chip.
  Right bottom: Sketch of the bus system. Buses are connected by a sparse switch matrix.
  Repeaters, used to regenerate the signals, connect buses of neighboring \cglspl{hicann}.}
  \label{fig:hicann_overview}
\end{figure}

With \SI{110}{\kibi\byte} per \cgls{hicann}, the configuration registers of the synapses make up the largest part of the tested memory.
They are split into two synapse arrays per \cgls{hicann}, each of which is programmed by a custom on-chip SRAM controller described in~\cite{friedmann13phd}.
In the tests, on \SI{1.97}{\percent} of the synapse arrays, unstable behavior is observed.
This means, consecutive write/read operations with fixed values on a single synapse register show varying results.
Since problems in individual synapse registers are very unlikely and could also derive e.g. from the control chain, a special stability test is introduced.
There, each register is tested several times with the same value.
If a single register shows unstable behavior, the whole synapse array is excluded.
Thereby, at the expense of functional components, only stable programmable synapses are used during experiments.\\

A test with ten write/reads of random data per component and a stability test with ten repetitions takes approximately \SI{70}{\second} per \cgls{hicann}.
Since the tests can be executed in parallel for each \cgls{hicann} group, a full wafer test takes approximately \SI{10}{\minute} and can be executed periodically to track the state of the systems.

\subsection{Effective Exclusion of Components}\label{sec:derived_blacklisting}
In special cases, it is not enough to skip malfunctioning components during an experiment, but it is also important to be aware of hardware specific dependencies that can be linked with these components.
This is achieved through an additional step, the effective exclusion of components, where functional but dependent components are excluded.
Several dependencies lead to an effective exclusion.
Some of them are visualized in \cref{fig:hicann_overview}.
\begin{itemize}
  \item Unstable repeater controller:
	To enhance the signal integrity of spike events that have to be routed across several \acrshortpl{hicann}, the signal is regenerated between dies by repeaters.
	These repeaters are organized in blocks where each block has a custom on-chip controller used to program its repeaters.
	Since failures in the digital memory of the repeaters are very unlikely, more than one failing repeater per block indicates that there could also be a problem in the control chain.
	To ensure no unstable components are used, all repeaters connected to the corresponding repeater block are removed from the availability database in such cases.
  \item Buses connected to malfunctioning repeaters:
	Buses are used to route spike events between neuron circuits.
	On boundaries between two \cglspl{hicann}, the buses are connected to repeaters that regenerate the signal.
	Each repeater is connected to a bus on its own \acrshort{hicann} as well as on a neighboring one.
	If a repeater is failing the memory test, there is no possibility to test if it sends wrong signals to its connected buses.
	To circumvent this, all buses connected to such a repeater are excluded and thus not used during an experiment.
	The same holds for repeaters on \cglspl{hicann} without \cgls{jtag} connection.
	As the repeaters cannot be initialized correctly, all neighboring buses connected to repeaters on the problematic \cgls{hicann} are excluded.
  \item Malfunctioning \cgls{fg} controller:
	The \cglspl{fg} are not only used to configure the neurons but also to supply bias voltages to the spike event routing.
	If an error in the controller programming the \cglspl{fg} is found, the whole \cgls{hicann} is excluded from the availability database and, in the following, treated as if there would be no \cgls{jtag} connection.
	Such a \cgls{hicann} is not used at all in experiments.
  \item Without high-speed:
	\cglspl{hicann} that have no high-speed connection are, due to the higher bandwidth requirements, not used to emulate neurons or external inputs but only used to route spike events.
	This is achieved by removing all neurons and external input mergers from the availability database.
  \item No routing options:
	  To improve the placement and prevent lost connections, the algorithm checks that all the components required to establish a route from each neuron and external input merger are available. %
	  If not, the neuron or the external input merger is excluded and therefore skipped in the process of building a network.
  \item Handling hardware versioning:
	In an earlier version of the post-processing, connections were established to \acrshortpl{hicann} on the edges of the wafer that must not be connected.
	To prevent leakage currents from these dies, the connected buses are excluded.
	Therefore, it is unnecessary to distinguish wafer versions in all the following steps.
\end{itemize}
An overview of removed components before and after the effective exclusion of components can be seen in \cref{tab:blacklists}.
The availability database, used to handle the excluded components, allows for storing different states on disk, so malfunctioning components and effective components can be differentiated afterward.
This is for example important during the initialization of the \acrshortpl{hicann}, where only malfunctioning components have to be handled specifically.
\subsection{Analog Readout Tests}
Before usage, the analog recording system gets verified for correct connectivity and configuration by running a series of tests.
Each \cgls{hicann} is set in sequence to generate two different voltage levels, which the \cgls{anarm} measures.
The voltage levels originate from the configuration of one of the \cglspl{fg}.
A recording that agrees with the settings and whose noise levels are within a tolerance threshold indicates that the system is ready for experiments or calibration runs.


\subsection{Calibration}\label{sec:calibration}
VLSI transistors are subject to manufacturing variations translating into differences in signal response.
This problem and the potential impacts have been noted since the first approaches to neuromorphic computing using VLSI~\cite{mead89analog}.
Consequently, the \cgls{hicann}'s microelectronic analog circuits require correction mechanisms to deliver homogeneous responses.

As the manufacturing variability is stationary within the components' operating ranges, thus termed fixed-pattern noise, it can be reduced by suitable calibration.
To this end, a framework has been developed for the \acrlong{bss1} wafer module that performs a one-time circuit characterization through running sequences of experiments that sweep neuron parameters, measure the effect in the observable, and perform appropriate fits on suitable models.
The process creates a database that holds the calibration results and is loaded on routine hardware usage, allowing for automatic translation between biological-space parameters and \cgls{fg}-stored parameters.
Such a conversion is automated and transparent for the users when running an experiment.
See~\cite{mueller2022operating} for details.

The calibration procedure configures all the neuron circuits at once and then processes the individual measurements to allow for programming the \cglspl{fg} in parallel.
In addition, parallelizing the analysis algorithms on the already measured steps further optimizes the time required for calibration.
Regardless, an increase in the number of calibration steps could improve the quality of the fits, while also parameters that are more sensitive to \cgls{fg} parameter variability benefit from an increase in the number of measurement repetitions.
Consequently, calibration time and precision of the results require balancing.

\subsubsection{Calibration Methodology}

In the \acrlong{bss1} system, the only analog neuron property that can be directly recorded is the membrane voltage.
Accordingly, all parameter calibrations are based on membrane recordings under different parameter configurations.
In general, the calibration of one parameter sweeps over its operating range while maintaining the rest of the parameters constant.
The execution order is relevant, as some calibration routines require an already calibrated subset of parameters.
Furthermore, the calibration accounts for analog readout noise, and measurements can be repeated to factor in \cgls{fg} parameter variability.

\begin{figure}
  \centering
  \includegraphics[width=0.8\linewidth]{img/neuron_synapse_parameters.png}
  \caption{Simplified neuron circuit schematic, displayed on the bottom, with the most relevant calibrated parameters in the \acrlong{bss1} system.
  The leak conductance controlled by $I_\text{gl}$ is constantly driving the membrane potential $V_\text{mem}$ towards the rest voltage $E_\text{leak}$.
  A spike is elicited when the membrane potential reaches the threshold voltage $V_\text{threshold}$.
  After a neuron spikes, its membrane's potential is connected to the reset voltage $V_\text{reset}$ for a period controlled by the parameter $I_\text{pulse}$.
  For simplicity, one synaptic input is displayed out of two through which a neuron integrates excitatory and inhibitory input currents $I_\text{syn}$; this controls a conductance between the reversal potential $E_\text{syn}$ and the membrane with a synaptic time constant controlled by $V_\text{syntc}$.
  Each input receives currents from all the synapses connected to one column in the synaptic array, displayed on top.
  Additional parameters $V_\text{syn}$, $V_\text{convoff}$, and $I_\text{conv}$ provide further control over the synaptic input, as further discussed in the supplementary material.
  }
  \label{fig:neuron_parameters}
\end{figure}

The main neuron calibration parameters are summarized in \cref{fig:neuron_parameters}.
In the following, the calibration procedure is exemplarily shown for the parameter $I_\text{pulse}$, which controls the refractory period $\tau_\text{ref}$, i.e., the time after the emission of a neuron's action potential during which its membrane is clamped to the reset potential and the neuron can elicit no further spike.
The higher $I_\text{pulse}$ is, the shorter the achieved $\tau_\text{ref}$.
Each $I_\text{pulse}$ calibration step sets the resting potential $E_\text{leak}$ above the level at which a spike event is elicited, i.e., $V_\text{threshold}$, which causes the neurons to spike continuously.
The \cgls{isi} is the measurable result.

In the first step, $I_\text{pulse}$ is set to maximum, and the corresponding \cgls{isi} is regarded as \cgls{isi}$_0$, the minimum attainable interval under the current settings.
Larger refractory periods are referenced to \cgls{isi}$_0$ by using

\begin{equation}
  \label{eq:tau_ref_zero}
  \tau_\text{ref}(I_\text{pulse}) = \text{ISI}(I_\text{pulse}) - \text{ISI}_0,
\end{equation}

making the minimum $\tau_\text{ref}$ zero seconds by definition.
Afterward, each step's distinct target \cgls{fg} values of $I_\text{pulse}$ are programmed, causing changes observable in the \cgls{isi} and thus in $\tau_\text{ref}$.
The obtained set of configured parameters and their achieved refractory periods is then fit to a model, which in the case of $\tau_\text{ref}$ corresponds to

\begin{equation}
  \label{eq:Ipl_model}
  I_\text{pulse} = \frac{1}{(c_{0} + c_{1}\cdot \tau_\text{ref})}.
\end{equation}

Such a model derives from transistor-level simulations described in~\cite{schwartz2013diss}.
The resulting fits for five neurons are shown in~\cref{fig:calibration_fit_depiction}.

The pair of constants $c_0$ and $c_1$ corresponding to model~\cref{eq:Ipl_model} is stored in the calibration database for each neuron, which is then used for translation from $\tau_\text{ref}$ in seconds to $I_\text{pulse}$ in digital value.
Further details for each parameter calibration are provided in the supplementary material.

Depending on each parameter's sensitivity to the programmed \cgls{fg} values, some calibrations enable a more precise setting of parameters than others.
An increased sensitivity due to non-linear hardware dependencies is found where small changes in \cgls{fg} values cause large changes in the observables.
Furthermore, for some \cglspl{fg} only a limited range of their available parameter space is used, reducing the ability to set their corresponding parameters precisely.
As can be seen from the measured values in~\cref{fig:calibration_fit_depiction}, such is the case for $I_\text{pulse}$.
For comparison,~\cref{fig:I_pl_E_leak_calib_comparison} shows how the leak potential $E_\text{leak}$, which is easier to control, obtains a more precise calibration than $I_\text{pulse}$.
For this reason, the control precision of several parameters was improved in the second-generation \acrlong{bss2} chip~\cite{schemmel2020accelerated} partly by enabling digital value storage.

\begin{figure}
  \includegraphics[width=0.90\linewidth]{img/I_pl_fit_zoom.png}
  \caption{Exemplified calibration procedure for the refractory period.
           Sample fits obtained for a set of neurons, relating the $I_\text{pulse}$ parameter configured with the Floating Gates with the measured $\tau_\text{ref}$.
  Seven values within the dynamic range of $I_\text{pulse}$ were used for the fits.}
  \label{fig:calibration_fit_depiction}
\end{figure}

\begin{figure}
  \centering
  \subfloat[
            \label{fig:calibration_I_pl}]{\includegraphics[width=0.5\linewidth]{img/I_pl_without_defects.png}}
  \subfloat[
            \label{fig:calibration_E_leak}]{\includegraphics[width=0.5\linewidth]{img/E_l_without_defects.png}}
    \caption{Histograms of achieved parameter settings on all neurons of one \cgls{hicann} for \protect\subref{fig:calibration_I_pl} the refractory time constant controlled by the parameter $I_\text{pulse}$ and \protect\subref{fig:calibration_E_leak} the leak potential controlled by the parameter $E_\text{leak}$.
            Pale and intense colors correspond to the hardware achieved time constants and voltages for different target values (shown in black-dashed lines), before and after the calibration is applied, respectively.
            The narrowing and centering of the achieved value distributions is better for $E_\text{leak}$ than for $\tau_\text{ref}$.}
  \label{fig:I_pl_E_leak_calib_comparison}
\end{figure}

\subsubsection{Synapse Weight Calibration}
The calibration of the synaptic input differs from the other calibrations due to its additional dependency on the synapse drivers.
The strength of a synapse is configured by three hardware parameters.
The \num{4}-bit digital weight $w$ stored per synapse, a scaling factor $gmax\_div$ stored per synapse row, and the \cgls{fg}-stored reference parameter $V_\text{gmax}$.
This last parameter is set per synapse row and selects one of four possible values shared by blocks of 128 neurons.
Calibrating this large parameter space for each of the \num{512} neurons with \num{110} connected synapse drivers using the analog readout system, which allows for measuring \num{12} membrane traces in parallel, is not possible in a reasonable time frame.
Therefore, a per wafer translation is performed, where only some of the components are taken into account to find the average circuit behavior.
The measurement requires the results of all previous calibrations.
Neurons on different \cglspl{hicann} are stimulated by a single spike for different combinations of the three hardware parameters to cover the whole parameter range.
Subsequently, a fit of the conductance based neuron model is applied to the recorded membrane traces to extract the ratio between biological weight and membrane capacitance $\frac{w_\text{bio}}{C_\text{HW}}$.
Since the membrane capacitance is fixed during experiments, it is unnecessary to separately determine both values.
During the fit, the model parameter of the already calibrated reversal potential is fixed.
The reduced $\chi^2$ value of the fit is used to identify and exclude saturation effects of the involved \cgls{ota}$_{1}$, cf. \cref{fig:neuron_parameters}, which might occur for large weight values.
Finally, the weight translation is found by fitting the expected hardware behavior
\begin{linenomath}
\begin{equation}
  \label{eq:weight}
  A(\frac{w \cdot V_\text{gmax}}{gmax\_div} + i_\text{0} + i_\text{1} \cdot w_\text{1} + i_\text{2} \cdot w_\text{2} + i_\text{4} \cdot w_\text{4}+ i_\text{8} \cdot w_\text{8}),
\end{equation}
\end{linenomath}
adapted from~\cite{Koke2017}, to the results of the first fits.
The fit parameters $i_\text{0-8}$ characterize the effect of parasitic capacitances found in the synaptic circuit for each enabled bit of the \num{4}-bit weight value $w$.
\Cref{fig:2d_weight_calib} demonstrates the large parameter space of the synapse weight calibration.
It shows the measurement of a single neuron, stimulated by a single synapse driver for a single $V_\text{gmax}$ value without rewriting the \cglspl{fg}.
The performance of the fit applied on the whole measured parameter space is shown for fixed values of $gmax\_div$ in \cref{fig:weight_fit} and for fixed digital weight values $w$ in \cref{fig:gdiv_fit}.
Although the whole neuron circuit and consequently the expected noise of each individual component is involved, the error of each measurement does not exceed the variations observed in other calibrations.
However, additional deviations arise from rewriting the \cglspl{fg}, which is demonstrated in \cref{fig:rewriting_floating_gates}; this renders the search for a more precise fit function unbeneficial.
In addition, the per wafer calibration opted over a per neuron circuit calibration introduces a dominant error due to the deviations between neuron circuits, shown in \cref{fig:mean_error_all_hicanns}.
A precise weight calibration within a reasonable runtime would be achievable via a parallel measurement of each neuron circuit.
This would also allow to exclude neurons showing unintended behavior.
However, this is not possible with the currently used analog readout system.
Nonetheless, the lack of a perfect weight calibration can be circumvented via in-the-loop training on the \acrlong{bss1} system, as shown for inference tasks in previous results~\cite{schmitt2017hwitl}.

\begin{figure}
  \centering
  \subfloat[
            \label{fig:2d_weight_calib}]{\includegraphics[width=0.8\linewidth]{img/heatmap_weight_measurement_increased_text.pdf}}\\
  \subfloat[
            \label{fig:weight_fit}]{\includegraphics[width=0.49\linewidth]{img/horizontal_cut_weight_measurement.pdf}}
  \subfloat[
            \label{fig:gdiv_fit}]{\includegraphics[width=0.49\linewidth]{img/vertical_cut_weight_measurement.pdf}}\\
  \subfloat[
            \label{fig:rewriting_floating_gates}]{\includegraphics[width=0.5\linewidth]{img/rewrite_variations_weight_calib.pdf}}
  \subfloat[
            \label{fig:mean_error_all_hicanns}]{\includegraphics[width=0.5\linewidth]{img/mean_error_all_hicanns.pdf}}
    \caption{Results of the synapse weight calibration.
            \label{fig:synapse_weight_calib} \protect\subref{fig:2d_weight_calib} Weight measurement for a fixed neuron circuit for different settings of the digital weight $w$ and hardware parameter $gmax\_div$ with $V_\text{gmax}=\SI{700}{LSB}$.
            Horizontal dashed lines indicate cuts with fixed values of the hardware parameter $gmax\_div$ shown in \protect\subref{fig:weight_fit}, vertical dashed lines indicate cuts with fixed digital weight values $w$ shown in \protect\subref{fig:gdiv_fit}.
            In \protect\subref{fig:weight_fit} and  \protect\subref{fig:gdiv_fit}, solid lines represent measured values, dashed lines the results of the fit of \cref{eq:weight} applied on the whole measured parameter space.
            \protect\subref{fig:rewriting_floating_gates} Variations of weight measurements with and without rewriting of the Floating Gates.
	    Values are extracted for \num{3} digital weight parameters $w$ from a fixed neuron with fixed hardware parameters ($V_\text{gmax} = \SI{700}{LSB}$, $gmax\_div = \SI{2}{LSB}$).
            \protect\subref{fig:mean_error_all_hicanns} Comparison of a per wafer and a per neuron weight calibration.
           Measurements for the entire parameter space are performed on a subset of neurons.
           The calibration is then performed for the whole subset or per individual neuron.
           The histogram shows the difference between the measured and expected values using the obtained calibrations.
         }
\end{figure}

\subsubsection{Calibration Based Exclusion of Components}
The operation of the \acrshortpl{hicann} during the calibration is similar to the operation during experiments.
All components have to work correctly for the calibration to succeed.
Failing calibrations indicate unintended behavior.
This allows for testing the whole die, especially the analog circuits that cannot be tested directly.
Additionally, thresholds can be defined to exclude outliers.
Consequently, neurons that do not pass all calibration steps are excluded from the availability database.
Numbers of calibration based excluded neurons on a typical wafer are given in \cref{tab:calib_blacklists}.

\begin{table}
  \centering
  \caption{Overview of calibration based excluded neurons of a fully assembled wafer module.
	In the column labeled "Neurons" the first entry shows the number of neurons taken into account for the calibration, the second entry the number of neurons taken into account for the effective exclusion.}
  \label{tab:calib_blacklists}
  \begin{tabular}{lll}
    \toprule
    Neurons & Excluded by Calibration & Effective Exclusion\\
    \midrule
	  182272/190976 & \SI{10.25}{\percent} & \SI{14.59}{\percent} \\
    \bottomrule
  \end{tabular}
\end{table}
