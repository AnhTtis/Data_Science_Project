\section{Calibration Details for the System's Neuron and Synapse Circuits}\label{sec:appendix}


A calibration procedure is in place for the BrainScaleS-1 system, which compensates for manufacture-induced analog circuit variability.
It accounts for analog readout noise by averaging the features extracted from the membrane traces over time.
In addition, measurements are repeated and then averaged after rewriting the \cglspl{fg}, where stated, to consider \cgls{fg} write-cycle to write-cycle parameter storage variability.

A detailed explanation of each parameter calibration conducted on the wafer module is provided in the following.
We first describe the parameter, explain the calibration approach and the settings used, and show plots illustrating the results.
In addition to the synaptic-weight calibration presented in the main text, these constitute the complete neuron- and synapse-circuit calibrations performed in the system.
Details of the sample points measured, the models utilized, and the average runtimes per parameter are summarized in~\cref{tab:calibration_data}.\\

\textit{Readout shift}: On each \cgls{hicann}, every neuron's membrane trace can be recorded by connecting its switchable analog output amplifier to one of two output buffers.
	     Due to circuit variability, each amplifier adds a constant offset to the recorded traces, the so called readout shift.
	     It has to be determined first, since all further calibrations are influenced by it.
      \begin{itemize}
        \item \textit{How}: Neuron membranes are interconnected in groups of \num{64} (the maximum possible).
        Their individual resting membranes are recorded and every neuron's deviation from the group's mean is stored.
\item \textit{Settings}: $E_\text{leak} = \SI{0.9}{\volt}$, the middle of the range, $V_\text{threshold}$ above resting potential, $I_\text{conv}$ set to \SI{0}{\ampere} to switch off both \cglspl{ota} for the excitatory and inhibitory synaptic input conductances.
        \item \textit{Effects}: The offset is automatically corrected for all subsequent calibrations by loading the calibration backend.
        The distribution of the analog output amplifier offsets of all neurons on one \cgls{hicann} is shown in~\cref{fig:analog_readout_offset}.
      \end{itemize}


  \begin{table}
    \centering
    \caption{Parameter-wise details for the calibration of neurons in one \cgls{hicann}, including the number of steps and repetitions run.
    Extraction refers to the model used to relate and determine variable values from the observables.
    HW-\cgls{fg} refers to the model used to fit parameters between the hardware domain and the programmable \cgls{fg} values.}
    \label{tab:calibration_data}
    \begin{tabular}{l | c | c | S[input-uncertainty-signs=pm,separate-uncertainty=true,table-format = 4.1(1)]}
      \toprule
      Parameter             & \begin{tabular}[x]{@{}c@{}}Target\\values\end{tabular} &  \begin{tabular}[x]{@{}c@{}}Extraction model /\\HW-\cgls{fg} model\end{tabular} & {\begin{tabular}[x]{@{}c@{}}Run time\\(s)\end{tabular}} \\
      \midrule
      Readout shift         & 1                                                      & -                    & 20.5 pm 0.7\\
      $V_\text{reset}$      & 4                                                      & linear / linear      & 87 pm 1\\
      $V_\text{threshold}$  & 5                                                      & linear / linear   & 111 pm 2\\
      $E_\text{syni}$       & 3                                                      & linear / linear      & 59.8 pm 0.8\\
      $I_\text{pulse}$      & 4 rep x 7                                              & \begin{tabular}[x]{@{}c@{}} \acrshort{isi}($I_\text{pulse}$)$-$\acrshort{isi}$_0$ /\\ eq.~(Main-2) \end{tabular} & 920 pm 10\\
      $E_\text{leak}$       & 3                                                      & linear / linear      & 60 pm 1\\
      $V_\text{convoffx,i}$ & 25                                                     & linear / linear      & 313 pm 4\\
      $I_\text{gl}$         & 8                                                      & \begin{tabular}[x]{@{}c@{}}SinglePSP~\cref{eq:psp_model} /\\Softplus~\cref{eq:softplus_function_mem}\end{tabular} & 780 pm 50 \\
      $V_\text{syntcx,i}$   & 10                                                     & \begin{tabular}[x]{@{}c@{}}SinglePSP /\\Softplus~\cref{eq:softplus_function_syn}\end{tabular}   & 810 pm 20\\
      $E_\text{synx}$       & 4 x 4                                                  & linear / linear        & 1300 pm 100\\
      Weight                & 16 x 4 x 4                                             & PSP / eq.~(Main-3) & {$\approx$ 220 x 20000}\\
      \bottomrule
    \end{tabular}
  \end{table}

    $V_\text{reset}$: The potential to which a neuron's membrane is set after a spike is generated.
      It is shared among a group of \num{128} neurons. Each \cgls{hicann} contains four of these groups.
      \begin{itemize}
        \item \textit{How}: Neurons are set to spike continuously by setting their leak potential $E_\text{leak}$ above their threshold potential $V_\text{threshold}$.
         A recording time of \SI{80}{\micro\second} per target value collects an average of \num{39} \cglspl{isi} on each membrane.
        The refractory time $\tau_\text{ref}$ is set to maximum in order to allow for long baseline traces between the spikes.
        The reset voltage is calculated as the average over all the interspike baseline samples to account for readout noise.
    \item \textit{Settings}: $I_\text{conv} = \SI{0}{\ampere}$ for both excitatory and inhibitory synaptic inputs, shutting off the \cgls{ota} of their synaptic conductance.
        $I_\text{gl} = \SI{1.1}{\micro\ampere}$, $I_\text{pulse} = \SI{20}{\nano\ampere}$ to set the refractory time to a high value.
        \item \textit{Sweep}: $V_\text{reset}$, with $E_\text{leak} = V_\text{reset} + \SI{0.4}{\volt}$,
        $V_\text{threshold} = V_\text{reset} + \SI{0.2}{\volt}$
        \item \textit{Effects}: The achieved hardware voltage distribution is shifted towards the correct target value from its original mean, as can be seen in~\cref{fig:calibration_V_reset}.
        The standard deviation does not improve for all targets since the shared nature of the parameter limits the action of the correction over the individual neurons. 
      \end{itemize}
  
    $V_\text{threshold}$: The threshold potential of the \cgls{lif} model, at which an action potential is elicited and the membrane's voltage is forced into the reset potential for the refractory period.
     \begin{itemize}
        \item \textit{How}: Synaptic inputs are minimized in order to isolate the membrane and the threshold detect circuits.
        The threshold potential $V_\text{threshold}$ is set below the leak potential $E_\text{leak}$ to elicit constant spiking.
        The maximum membrane voltage at several spike peaks is averaged and considered the true threshold voltage.
        \item \textit{Settings}: $I_\text{conv} = \SI{0}{\ampere}$, $I_\text{gl} = \SI{1.5}{\micro\ampere}$
        \item \textit{Sweep}: $V_\text{threshold}$, $V_\text{reset} = V_\text{threshold} - \SI{200}{\milli\volt}$, $E_\text{leak} = V_\text{threshold} + \SI{200}{\milli\volt}$
        \item \textit{Effects}: The corrected hardware voltage distribution is centered around the correct target value.
        The standard deviation decreases, as can be seen in \cref{fig:calibration_V_threshold}.
      \end{itemize}
  
      \begin{figure}
        \centering
        \subfloat[
                  \label{fig:analog_readout_offset}]{\includegraphics[width=0.5\linewidth]{img/analog_readout_offset.png}}
        \subfloat[
                  \label{fig:calibration_V_reset}]{\includegraphics[width=0.5\linewidth]{img/V_reset_without_defects.png}}\\
        \subfloat[
                  \label{fig:calibration_V_threshold}]{\includegraphics[width=0.5\linewidth]{img/V_t_without_defects.png}}
        \subfloat[
                  \label{fig:esyni_calib}]{\includegraphics[width=0.5\linewidth]{img/E_syni_calibrated.png}}
	  \caption{\label{fig:calibration_effects_v_fgs} \protect\subref{fig:analog_readout_offset} Analog readout offset distribution for the \num{512} neurons of one \cgls{hicann}.
                  Calibration results for the parameters \protect\subref{fig:calibration_V_reset} $V_\text{reset}$, \protect\subref{fig:calibration_V_threshold} $V_\text{threshold}$ and \protect\subref{fig:esyni_calib} $E_\text{syni}$.
                  Pale and intense colors correspond to the hardware achieved voltages for different target values (shown in black-dashed lines) before and after the calibration is applied, respectively.
		  For $V_\text{reset}$ the correction effect is limited by the parameter being shared by \num{128} neurons.}
      \end{figure}
  
    $E_\textit{syni}$: The inhibitory reversal potential towards which the \cgls{ota} in the inhibitory synaptic input drives the membrane when processing synaptic input.
      \begin{itemize}
        \item \textit{How}:
        $V_\text{convoffi}$ of the inhibitory synaptic input is set to a small value so the bias generator forces the membrane potential to the inhibitory reversal potential.
        No spikes are elicted since the threshold voltage is never reached.
        Once the neuron is at rest the averaged membrane voltage characterizes the reversal potential.
        \item \textit{Settings}: $E_\text{leak} = \SI{0.8}{\volt}$, $I_\text{convx} = \SI{0}{\ampere}$, $I_\text{gl} = \SI{0}{\ampere}$, $V_\text{convoffi} = \SI{0.1}{\volt}$, $V_\text{syntcx,y} = \SI{1.8}{\volt}$, $V_\text{threshold} = \SI{1.2}{\volt}$
        \item \textit{Sweep}: $E_\text{syni}$
        \item \textit{Effects}: The achieved inhibitory reversal potential voltages before and after calibration are shown in~\cref{fig:esyni_calib}.
      \end{itemize}
  
    $I_\text{pulse}$: Bias current that controls how fast the neuron's timing mechanism recovers from the reset state after a spike is generated.
      \begin{itemize}
        \item \textit{How}: Neurons are set to spike continuously by setting $E_\text{leak}$ above $V_\text{threshold}$.
        For the refractory time constant measurements, the baseline traces corresponding to the reset-state of the membranes are extracted.
        $I_\text{pulse}$ is first set to its maximum and the effective refractory period is measured and recorded; this constitutes the minimum achievable period denoted thus $\tau_{0}$.
		      The subsequent measured refractory periods are referenced to $\tau_{0}$ by substracting $\tau_{0}$ from them, and fitting eq.~(Main-2) from the main text.
        \item \textit{Settings}: $E_\text{leak} = \SI{1.2}{\volt}$, $V_\text{threshold} = \SI{0.8}{\volt}$, $E_\text{synx} = \SI{1.2}{\volt}$, $E_\text{syni} = \SI{0.8}{\volt}$,
        $V_\text{reset} = \SI{0.5}{\volt}$
        \item \textit{Sweep}: $I_\text{pulse}$
	\item \textit{Effects}: The achieved refractory time constants' mean is closer to the target value after the calibration is obtained and applied, as can be observed in fig.~11a in the main text.
        The standard deviations reduce.
        In fig.~10 in the main text the limited precision to configure the refractory time constant is demonstrated, as only a fraction of the possible parameter range of $I_\text{pulse}$ results in reasonable configurations.
      \end{itemize}
  
    $E_\text{leak}$: The reference voltage towards which the membrane potential is constantly driven through the leak conductance.
      \begin{itemize}
        \item \textit{How}: Synaptic inputs are minimized and the membranes are read on a resting state.
        \item \textit{Settings}: $I_\text{conv} = \SI{0}{\ampere}$, $V_\text{t} = \SI{1.2}{\volt}$, $V_\text{reset} = \SI{0.9}{\volt}$
        \item \textit{Sweep}: $E_\text{leak}$
        \item \textit{Effects}: The corrected hardware voltage distribution is centered around the correct target value.
        The standard deviation decreases, as can be seen in fig.~11b in the main text.
      \end{itemize}
  
      $V_\text{convoffx}$: Offset voltage for the integrator on the excitatory synaptic input.
      The voltage parameter is used by a bias generator that controls the reference of \cgls{ota}$_{1}$, compensating for mismatches.
      The offset should balance two effects: minimize an undesired permanent current flowing to the membrane, which shifts the neuron's resting potential, against the weakening of the synaptic input caused by a too substantial compensation.
      Consequently, the goal of the calibration is to find the sweet spot in between, where the bias generator compensates precisely for the mismatch of \cgls{ota}$_{1}$.
      \begin{itemize}
        \item \textit{How}:
        The point of interest is the transition from a zero to a non-zero conductance on \cgls{ota}$_{1}$.
        It is measured by the shift of the resting potential arising for different values of $V_\text{convoffx}$.
        The calibrated value of $V_\text{convoffx}$ corresponds to the first value where the resting potential is no longer shifted.
        In addition, the linear range of the relation between the membrane rest-voltage shift and $V_\text{convoffx}$ is characterized.
        Effects from the inhibitory synaptic input are minimized by using low values for $E_\text{syni}$, $I_\text{convi}$ and a high $V_\text{convoffi}$.
        Furthermore, the effect is more pronounced for lower values of $E_\text{leak}$.
        \item \textit{Settings}: $E_\text{leak} = \SI{0.8}{\volt}$, $E_\text{syni} = \SI{0.4}{\volt}$, $E_\text{synx} = \SI{1.2}{\volt}$, $I_\text{convi} = \SI{0}{\ampere}$,
        $I_\text{gl} = \SI{0.2}{\micro\ampere}$ a low value that limits the leakage current from the synapse onto the membrane,
        $V_\text{convoffi} = \SI{1.8}{\volt}$, $V_\text{threshold} = \SI{1.2}{\volt}$, $V_\text{reset} = \SI{0.3}{\volt}$
        \item \textit{Sweep}: $V_\text{convoffx}$
        \item \textit{Effects}: A calibrated $V_\text{convoffx}$ parameter limits the deviations in the effective resting potential arising from leaks through the excitatory synaptic input, as shown in~\cref{fig:V_convoff_effects}.
        Nevertheless, a minimal $I_\text{gl}$ is required to allow the neuron membranes to exhibit uniform effective resting potentials.
      \end{itemize}

      \begin{figure}
        \centering
        \subfloat[
                  \label{fig:V_convoff_effects_uncalibrated}]{\includegraphics[width=0.51\linewidth]{img/V_convoff_uncalibrated.png}}
        \subfloat[
                  \label{fig:V_convoff_effects_calibrated}]{\includegraphics[width=0.49\linewidth]{img/V_convoff_calibrated.png}}
        \caption{\label{fig:V_convoff_effects} Effective resting potential of the neurons on one \cgls{hicann} \protect\subref{fig:V_convoff_effects_uncalibrated} before and \protect\subref{fig:V_convoff_effects_calibrated} after calibration of parameter $V_\text{convoffx}$.}
      \end{figure}

      $V_\text{convoffi}$: Offset voltage for the integrator on the inhibitory synaptic input conductance.
      The calibration principle is the same as for $V_\text{convoffx}$, but it should be performed independently as both inputs introduce leak currents into the membrane.
      \begin{itemize}
        \item \textit{How}: A low $E_\text{synx}$, $I_\text{convx}$ and high $V_\text{convoffx}$ minimize effects from the excitatory synaptic input.
        \item \textit{Settings}: $E_\text{leak} = \SI{0.8}{\volt}$, $E_\text{syni} = \SI{0.4}{\volt}$, $E_\text{synx} = \SI{1.2}{\volt}$, $I_\text{convx} = \SI{0}{\ampere}$,
        $I_\text{gl} = \SI{0.2}{\micro\ampere}$ a low value that limits the leakage current from the synapse onto the membrane,
        $V_\text{convoffx} = \SI{1.8}{\volt}$, $V_\text{threshold} = \SI{1.2}{\volt}$, $V_\text{reset} = \SI{0.3}{\volt}$
        \item \textit{Sweep}: $V_\text{convoffi}$
      \end{itemize}

    The following parameter calibrations use input spikes to generate \cglspl{psp} on the membrane.
    From the shape of the voltage traces, it is possible to approximate parameters related to the time constants of synaptic inputs ($\tau_\text{syn}$) and the membrane ($\tau_\text{mem}$).
    For a single input spike arriving while the membrane of a LIF neuron is in a steady state, the \cgls{psp} shape can be either described by an $\alpha$-function, if both time constants are the same, or by a difference of exponentials if one of the time constants is smaller~\cite{petrovici2016form}.
    This behavior is described by $V(t)\approx$
    \begin{equation}
      \label{eq:psp_model}
      \begin{cases}
          \begin{split}
            &E_{\text{leak}}+
            \theta(t_{\scaleto{0\mathstrut}{4pt}})A\left(\text{exp}\left(\frac{t_{\scaleto{0\mathstrut}{4pt}}-t}{\tau_{\scaleto{1\mathstrut}{4pt}}}\right)-\text{exp}\left(\frac{t_{\scaleto{0\mathstrut}{4pt}}-t}{\tau_{\scaleto{2\mathstrut}{4pt}}}\right)\right)\\
            &\qquad\qquad\qquad\qquad\qquad\qquad\qquad\qquad\quad\text{if }\tau_{\scaleto{1\mathstrut}{4pt}}\neq\tau_{\scaleto{2\mathstrut}{4pt}}\\
            &\\
            &E_{\text{leak}}+\theta(t_{\scaleto{0\mathstrut}{4pt}})\text{exp}\left(1-\frac{t-t_{\scaleto{0\mathstrut}{4pt}}}{\tau_{\scaleto{1\mathstrut}{4pt}}}\right)\frac{t-t_{\scaleto{0\mathstrut}{4pt}}}{\tau_{\scaleto{1\mathstrut}{4pt}}}\\
            &\qquad\qquad\qquad\qquad\qquad\qquad\qquad\qquad\quad\text{if }\tau_{\scaleto{1\mathstrut}{4pt}}=\tau_{\scaleto{2\mathstrut}{4pt}},\\
        \end{split}
      \end{cases}
    \end{equation}
    with
    \begin{equation}
    	A=\frac{h}{\tau^{\frac{1}{1-\tau}}-\tau^{\frac{\tau}{1-\tau}}}
    \end{equation}
    and $\tau=\frac{\tau_{2}}{\tau_{1}}$ a ratio between $\tau_\mathrm{mem}$ and $\tau_\mathrm{syn}$,
    derived in~\cite{bytschok2011shared} and further developed in~\cite{Koke2017}.
    It relates the membrane's voltage course with both relevant time constants and the height $h$ of the \cgls{psp}.
    The fitting algorithm fixes one of the time constants and varies the other.
    Although the \cglspl{psp} are symmetric in $\tau_\text{mem}$ and $\tau_\text{syn}$, the fact that typically $\tau_\text{mem}>\tau_\text{syn}$ is considered.
    Once the parameters are determined from the measurements through fitting the model, a linear fit is used to obtain a calibration relating parameters with \cgls{fg} values, as with the previously treated calibrations.
  
    $I_\text{gl}$: Bias current that controls the membrane's leakage conductance.
      This parameter and the chosen membrane capacitance, which can be set to two different values, determines the membrane time constant.
      \begin{itemize}
        \item \textit{How}: The input spikes should arrive with enough space to allow the membrane to return to a steady-state after each perturbation.
        A strong excitatory synaptic input is set to achieve a better signal-to-noise ratio.
        Fitting~\cref{eq:psp_model} returns both the membrane and the synaptic input time constant from the \cgls{psp} shape.
        A fit of the softplus function
        \begin{equation}
          \label{eq:softplus_function_mem}
           \tau_\text{mem} = \frac{a \cdot \text{log}(1 + \text{exp}(c \cdot (b - I_\text{gl})))}{c} + \text{offset}
        \end{equation}
        is subsequently used to translate between biological-space parameters and \cgls{fg}-stored parameters.
        \item \textit{Settings}: $E_\text{leak} = \SI{0.8}{\volt}$, $E_\text{synx} = \SI{1.3}{\volt}$, $E_\text{syni} = \SI{0.6}{\volt}$, $V_\text{syntcx} = \SI{1.6}{\volt}$,
        $V_\text{convoffi} = \SI{0.9}{\volt}$, $V_\text{convoffx} = \SI{0.9}{\volt}$,
        $V_\text{threshold} = \SI{1.2}{\volt}$, $V_\text{reset} = \SI{0.3}{\volt}$,
        $V_\text{gmax0} = \SI{1}{\volt}$, $\text{gmax\_div} = \SI{30}{LSB}$,
        \textit{big capacitor}, $I_\text{gl}$ speedup normal.
        \item \textit{Sweep}: $I_\text{gl}$
        \item \textit{Effects}: The achieved membrane time constants' mean is closer to the target value after the calibration is obtained and applied, as can be observed in~\cref{fig:calibration_I_gl}.
        The standard deviations are reduced.
        However, as seen in~\cref{fig:I_gl_fits}, the precision to configure $\tau_\text{mem}$ is limited as only a fraction of the possible parameter range of $I_\text{gl}$ results in reasonable configurations.

      \end{itemize}

    \begin{figure}
        \centering
        \subfloat[
                  \label{fig:I_gl_fits}]{\includegraphics[width=0.5\linewidth]{img/I_gl_fit.png}}
        \subfloat[
                  \label{fig:calibration_I_gl}]{\includegraphics[width=0.5\linewidth]{img/I_gl_PSP_without_defects.png}}
        \caption{\protect\subref{fig:I_gl_fits} Fits for the parameter $I_\text{gl}$ against the achieved membrane time constant on five neurons, using a softplus function model and eight measurement steps.
        \protect\subref{fig:calibration_I_gl} Distribution of membrane time constants before and after the $I_\text{gl}$ calibration is applied for all neurons, with pale and intense colors, respectively.}
        \label{fig:I_gl_figures}
      \end{figure}

      \begin{figure}
        \centering
        \subfloat[
                  \label{fig:vsyntcx_calib}]{\includegraphics[width=0.5\linewidth]{img/V_syntcx_calibrated.png}}
        \subfloat[
                  \label{fig:vsyntci_calib}]{\includegraphics[width=0.5\linewidth]{img/V_syntci_calibrated.png}}
        \caption{Distribution of the achieved \protect\subref{fig:vsyntcx_calib} excitatory and \protect\subref{fig:vsyntci_calib} inhibitory synaptic time constants before and after calibration in pale and intense colors, respectively,
        for four different target values (black dashed lines).}
      \end{figure}
  
    $V_\text{syntcx}$: Voltage controlling the excitatory synapse time constant, $\tau_{syn,x}$, by varying the voltage integrator's resistive element.
      Large values of $V_\text{syntcx}$ shift $E_\text{leak}$ towards the reversal potential, since leak currents in the synaptic input integrator inrease for higher voltages.
      \begin{itemize}
        \item \textit{How}: Similar to the $I_\text{gl}$ calibration, input spikes that arrive with enough separation are used.
        \Cref{eq:psp_model} is fitted to extract the time constants.
	Afterwards, a fit of the softplus function
        \begin{equation}
          \label{eq:softplus_function_syn}
           \tau_\text{syn} = \frac{a \cdot \text{log}(1 + \text{exp}(c \cdot (b - V_\text{syntc})))}{c} + \text{offset}
        \end{equation}
        to the extracted values is used to translate between biological-space parameters and \cgls{fg}-stored parameters.
        \item \textit{Settings}: $E_\text{leak} = \SI{0.8}{\volt}$, $E_\text{syni} = \SI{0.6}{\volt}$, $E_\text{synx} = \SI{1.3}{\volt}$, $I_\text{gl} = \SI{0.3}{\micro\ampere}$,
        $V_\text{convoffx} = \SI{1.8}{\volt}$, $V_\text{convoffi} = \SI{1.8}{\volt}$, $V_\text{threshold} = \SI{1.2}{\volt}$, $V_\text{reset} = \SI{0.3}{\volt}$,
        $V_\text{gmax0} = \SI{0.05}{\volt}$, $\text{gmax\_div} = \SI{30}{LSB}$
        \item \textit{Sweep}: $V_\text{syntcx}$
        \item \textit{Effects}: The distribution of the achieved excitatory synaptic time constants, $\tau_{syn,x}$ before and after calibration of $V_{syntcx}$ is shown in~\cref{fig:vsyntcx_calib}.
      \end{itemize}
  
      $V_\text{syntci}$: Voltage controlling the inhibitory synapse time constant, $\tau_{syn,i}$, by varying the voltage integrator's resistive element.
    \begin{itemize}
        \item \textit{How}: Similar to the $V_\text{syntcx}$ calibration.
        \item \textit{Settings}: $E_\text{leak} = \SI{0.8}{\volt}$, $E_\text{syni} = \SI{0.3}{\volt}$, $E_\text{synx} = \SI{1.3}{\volt}$, $I_\text{gl} = \SI{0.3}{\micro\ampere}$,
        $V_\text{convoffx} = \SI{1.8}{\volt}$, $V_\text{convoffi} = \SI{1.8}{\volt}$, $V_\text{threshold} = \SI{1.2}{\volt}$, $V_\text{reset} = \SI{0.3}{\volt}$,
        $V_\text{gmax0} = \SI{0.05}{\volt}$, $\text{gmax\_div} = \SI{30}{LSB}$
        \item \textit{Sweep}: $V_\text{syntci}$
        \item \textit{Effects}: The distribution of the achieved inhibitory synaptic time constants, $\tau_{syn,i}$ before and after calibration of $V_{syntcx}$ is shown in~\cref{fig:vsyntci_calib}.
      \end{itemize}

  \textit{$E_\textit{synx}$}:
  In biologically plausible networks, the excitatory reversal potential is above the threshold and thus never reached by the membrane potential.
  Its calibration is a good showcase for pitfalls during the operation of analog circuits.
  Intuitively, a direct measurement using the membrane potential would be used for both reversal potentials.
  However, similar to their biological counterparts, the circuits of the \cgls{hicann} chip are not designed for the membrane potential to get close to the excitatory reversal potential.
  Thus, the circuits show a non-linear behavior when approaching the reversal potential, as they deviate from the center of their design ranges.
  This can be observed in~\cref{fig:exc_rev_pot}.
  Therefore, the excitatory reversal potential is measured indirectly.
    \begin{itemize}
      \item \textit{How}:
      The height of the \cgls{psp} of a stimulated neuron is measured for different resting potentials in the linear regime of the circuits.
      A linear extrapolation is used to extract the resting potential where the height reaches zero, shown in~\cref{fig:exc_rev_pot}.
      In the conductance based synapse model this resting potential is equal to the reversal potential.
      The measurements are repeated for different reversal potentials to extract the linear dependency between hardware value and applied voltage.
      \item \textit{Settings}: $I_\text{convi} = \SI{0}{\ampere}$, $I_\text{gl} = \SI{E-7}{\second}$, $V_\text{convoffx,i} = \SI{0.9}{\volt}$, $V_\text{syntcx,i} = \SI{2E-7}{\second}$, $V_\text{threshold} = \SI{1.8}{\volt}$, $V_\text{gmax} = \SI{0.9}{\volt}$, $gmax\_div = \SI{2}{LSB}$, $w = \SI{15}{LSB}$
      \item \textit{Sweep}:
      $E_\text{leak}$, $E_\textit{synx}$
      \item \textit{Results}:
      Results of the calibration compared to a direct measurement can be seen in~\cref{fig:result_exc_rev_pot}.
      The disadvantage of the indirect measurement is the increased runtime and the dependency on the shape of the \cgls{psp}.
      Small variations of hardware parameters, most likely due to the necessity to rewrite the \cgls{fg} value of the resting potential, are enlarged by the linear extrapolation performed to find the reversal potential.
      As a result,~\cref{fig:result_exc_rev_pot} shows larger variations for the indirect calibration than the direct measurement.
      Nevertheless, the technique allows for correctly calibrating the excitatory reversal potential without directly measuring it.
    \end{itemize}

  \begin{figure}
        \subfloat[
                  \label{fig:exc_rev_pot}]{\includegraphics[width=0.49\linewidth]{img/E_rev_non_linearity.pdf}}
        \subfloat[
                  \label{fig:result_exc_rev_pot}]{\includegraphics[width=0.49\linewidth]{img/E_rev_calib.pdf}}
        \caption{Excitatory reversal potential calibration. \protect\subref{fig:exc_rev_pot} Indirect measurement of the excitatory reversal potential.
             The height of the \cgls{psp} of a stimulated neuron is extracted for different resting potentials.
             Different colors indicate different hardware settings of the reversal potential in LSB.
             Since the circuits are not designed to reach the excitatory reversal potential, non linear behavior is observed for small distances between membrane potential and reversal potential.
             A linear extrapolation of the linear region (dotted line) is used to extract the correct reversal potential.
             During experiments the neuron is exclusively operated in the linear regime.
             \protect\subref{fig:result_exc_rev_pot} Comparison of direct and indirect measurement of the excitatory reversal potential.
             Because of the non-linear behavior of the circuits close to the reversal potential, the direct measurement provides too small values.
             The indirect measurement has larger errors due to its dependency on the whole neuron circuit and the enlargement by the linear extrapolation.}
  \label{fig:exc_rev_pot_both}
  \end{figure}
