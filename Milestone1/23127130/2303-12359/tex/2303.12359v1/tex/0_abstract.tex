The first-generation of BrainScaleS, also referred to as \acrlong{bss1}, is a neuromorphic system for emulating large-scale networks of spiking neurons.
Following a ``physical modeling'' principle, its VLSI circuits are designed to emulate the dynamics of biological examples: analog circuits implement neurons and synapses with time constants that arise from their electronic components' intrinsic properties.
It operates in continuous time, with dynamics typically matching an acceleration factor of \num{10000} compared to the biological regime.
A fault-tolerant design allows it to achieve wafer-scale integration despite unavoidable analog variability and component failures.
In this paper, we present the commissioning process of a \acrlong{bss1} wafer module, providing a short description of the system's physical components, illustrating the steps taken during its assembly and the measures taken to operate it.
Furthermore, we reflect on the system's development process and the lessons learned to conclude with a demonstration of its functionality by emulating a wafer-scale \acrlong{sfc}, the largest spiking network emulation ran with analog components and individual synapses to date.
