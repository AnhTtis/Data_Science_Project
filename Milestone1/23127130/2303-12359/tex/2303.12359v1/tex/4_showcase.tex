\section{Experiment Showcase - Synchronous Firing Chain}

Previous experiments on the \acrlong{bss1} system relied on a small subset of the available neurons~\cite{schmitt2017hwitl,kungl2019accelerated,goeltz2021fast}.
In this section, we use a \cgls{sfc} to utilize a large number of the available wafer module resources.
We start with a relatively short chain to illustrate the behavior of the network and finally present a longer one that utilizes a large part of a single wafer module. %

\cGlspl{sfc} can filter for synchronous activity and propagate the activity along a chain of neuron groups~\cite{aertsen96, gewaltig01_synfire}.
We choose \cglspl{sfc} since they can easily be scaled up to arbitrary sizes by increasing the chain length as well as the number of neurons in a single group and have been studied extensively in previous publications~\cite{diesmann99, diesmann02, kumar08conditions}.
Furthermore, \cglspl{sfc} were used to showcase the functionality of the predecessor of \acrlong{bss1}~\cite{pfeil2013six} and to characterize the behavior of the current system in software simulations~\cite{petrovici2014characterization}.

\begin{figure}
  \centering
  \includegraphics[width=\linewidth]{img/sfc_model}
  \caption{Structure of the \cglspl{sfc} presented in this section.
           The \cgls{sfc} is made up of several groups of excitatory (blue) and inhibitory (red) populations.
           The inhibitory population connects to the excitatory population within the same group and aims to improve the chain's filtering for synchronous input~\cite{kumar08conditions, kremkow2010functional}.
           Each excitatory population is connected to the excitatory and inhibitory population of the next group.
           By repeating this construction schema (grey), chains of arbitrary length can be realized.
           The network is excited by a stimulus population (orange) which projects to the excitatory and inhibitory population of the first group.}
  \label{fig:synfire}
\end{figure}

\Cref{fig:synfire} displays a \cgls{sfc} with feed-forward inhibition.
Each chain link consists of an excitatory and inhibitory population.
The inhibitory populations are connected to the excitatory population within the same group.
This feed-forward inhibition can enhance the filtering properties of the chain~\cite{kremkow2010functional, kumar08conditions}.
The excitatory population forwards its outputs to both populations within the next group.
External stimulus is injected in the form of Gaussian pulse packages~\cite{diesmann99}. %
The strength $a$ denotes the number of input spikes per stimulus neuron and $\sigma$ the standard deviation of the Gaussian from which the spike times are drawn.
We will use $(a, \sigma)$ to refer to specific packages.

\subsection{Network Behavior}

In a first step we will look at a relatively short chain with six chain links, shown in \cref{fig:sfc_short}, to illustrate how the filtering properties of the chain can be tuned.
\Cref{tab:synfire} summarizes some of the key properties of the network.
We used the manual placement described in~\cite{mueller2022operating} to place the different populations on the wafer.
Specifically, we distribute the external stimulus over several \cglspl{hicann} in order to minimize spike loss due to limited bandwidth.

\begin{table}
  \centering
  \caption{Parameters used for the \cglspl{sfc} presented in this section.}
  \label{tab:synfire}
  \begin{tabular}{lll}
    \toprule
    Parameter                        & Short Chain           & Long Chain            \\
    \midrule
	Chain Length                    & 6                     & 190                   \\
	Stimulus Neurons                & 100                   & 80                    \\
	Excitatory Neurons per Group    & 100                   & 80                    \\
	Inhihibitory Neurons per Group  & 25                    & 20                    \\
	Total Number of Neurons         & 750                   & 19000                 \\
	Total Number of Neuron Circuits & 3000                  & 76000                 \\
	Total Number of Synapses        & $\approx \num{73000}$ & $\approx \num{1.4e6}$ \\
	Used HICANNs                    & 48                    & 230                   \\
    \bottomrule
  \end{tabular}
\end{table}

\begin{figure}
  \centering
  \subfloat[\label{fig:sfc_short_propagation}]{\includegraphics[width=\linewidth]{img/sfc_initial_bss_ffi_spike_trains}}\\
  \subfloat[\label{fig:sfc_short_phase}]{\includegraphics[width=\linewidth]{img/sfc_initial_bss_ffi_phase_diagram}}
  \caption{Hardware emulation of a chain with six chain links.
  		   \protect\subref{fig:sfc_short_propagation} Propagation of pulse packages along the chain.
		   Successful propagation depends both on the strength $a$ as well as the synchronicity $\sigma$ of the initial stimulus, represented by ($a$, $\sigma$).
		   Broad input stimuli synchronize along the chain or do not reach the end of the chain.
		   \protect\subref{fig:sfc_short_phase} Average number of spikes per neuron in the final group $\bar{a}_\text{out}$ of the chain in dependency on the initial strength $a$ and synchronicity $\sigma$.
		   Each input package was presented \num{40} times and the results are averaged over all presentations.
		   The pulse packages propagate if the initial input is strong and synchronous enough.
		   In the region of stable propagation the output strength is almost constant, near the separation of the two regimes the average strength of the final pulse package decreases.
		   This separation line between successful propagation and failure of transmission can be controlled by several parameters such as the synaptic weights.
  }
  \label{fig:sfc_short}
\end{figure}

As mentioned previously, \cglspl{sfc} are able to filter for synchronous input and to synchronize less-synchronous input as it travels along the chain~\cite{diesmann99, gewaltig01_synfire}.
\Cref{fig:sfc_short_propagation} shows the propagation of three different input stimuli along the chain.
In case of a relatively weak and synchronous input $(1,1)$ a single, narrow package travels along the chain.
If the input is stronger and more asynchronous, we observe a broader response in the first groups of the chain which is synchronized as the signal propagates along the chain such that the responses in the final group are comparable.
Too weak and asynchronous input, here $(1,4)$ as an example, dies out and does not cause a response in the final group.
This is in agreement with previous results~\cite{diesmann99, kremkow2010functional, pfeil2013six, petrovici2014characterization}.


\Cref{fig:sfc_short_phase} shows in more detail for which input stimuli the propagation along the chain is successful.
In agreement with the previous observations, weak and asynchronous input is not transmitted to the final group.
The response in the final group is almost uniform.
This indicates that the packages are synchronized as they travel along the chain.
Setting appropriate parameters which reproduce the expected results from simulations relies on the calibration routines, introduced in \cref{sec:calibration}.
The calibration allows to set model parameters in the biological domain and reduces the inherent mismatch between the physical components.


\subsection{Wafer-Scale Network}
The previous section demonstrates the implementation and control of a short \cgls{sfc} on the \acrlong{bss1} system.
This section shows that the commissioning efforts described in \cref{sec:commissioning} also facilitate the implementation of wafer-scale networks.
The properties of this \cgls{sfc} are summarized in \cref{tab:synfire}.

\begin{figure}
  \centering
  \subfloat[\label{fig:sfc_long_mapping}]{\includegraphics[width=0.7\linewidth]{img/sfc_manual_mapping_long_chain}}\\
  \subfloat[\label{fig:sfc_long_propagation}]{\includegraphics[width=\linewidth]{img/sfc_long_bss_manual_spike_trains}}
  
	\caption{Hardware emulation of a chain with \num{19000} neurons.
			 Further parameters of the network can be found in \cref{tab:synfire}.
			 \protect\subref{fig:sfc_long_mapping} Mapping of the network to a \acrlong{bss1} wafer.
			 \cGlspl{hicann} excluded from the availability database are marked in red, cf.\ \cref{sec:blacklisting}.
			 \cGlspl{hicann} which cannot host an entire group are marked in orange and are not used in the experiment.
			 On each \cgls{hicann} colored in blue an entire group of neurons is placed.
			 Colored lines indicate synaptic connections.
			 \protect\subref{fig:sfc_long_propagation} Response of the chain to an input packet of strength $a=1$ and spread $\sigma=1$.
	   }
  \label{fig:sfc_long}
\end{figure}


The complexity of the emulation increases with the size of the model.
While for a relatively short chain it is possible to investigate the behavior of individual neurons and manually detect malfunctioning and bad calibrated entities, this is not feasible for larger experiments.
Therefore, digital tests described in \cref{sec:blacklisting} are essential to automatically avoid these components during the experiment.

To simplify the automatic routing of the abstract network description to physical entities on the wafer, we once again employ manual mapping, see \cref{fig:sfc_long_mapping}.
We place the different groups in a zig-zag pattern starting from the top-left side towards the bottom of the wafer and then back up towards the top-right side.
This placement schema allows the \acrlong{bss1} operating system~\cite{mueller2022operating} to find appropriate connections between the different populations and minimizes synapse loss, i.e.\ synaptic connections that could not be mapped to the hardware.

We were able to successfully emulate a \cgls{sfc} with \num{190} chain links on the \acrlong{bss1} system.
\Cref{fig:sfc_long_propagation} shows an example of a pulse package that travels along the full length of the chain.
The activity of the individual groups still depends on the exact neuron and synapse properties, but the calibration ensures that the pulse package remains compact.
A synchronous pulse reaches the final group after a signal propagation time of about \SI{600}{\milli\second} in the biological regime, which corresponds to \SI{60}{\micro\second} wall-clock time.
