\documentclass[epsfig,12pt]{article}
\usepackage{epsfig}
\usepackage{graphicx}
\usepackage{array}
\usepackage{color}
% \usepackage{bm}
\usepackage[numbers,sort&compress]{natbib}
\usepackage{hyperref}
\hypersetup{
	colorlinks=true,  
	linktoc=page,
	linkcolor=blue
}
\usepackage[nottoc,notlof,notlot]{tocbibind} 
\usepackage[titles]{tocloft} 
\renewcommand{\cftsecfont}{\rmfamily\mdseries\upshape}
\renewcommand{\cftsecpagefont}{\rmfamily\mdseries\upshape} 

\usepackage{amssymb,amsmath,mathtools,physics,float,subcaption}
\allowdisplaybreaks

\def\d{\partial}
\def\CC{\mathbb{C}}
\def\RR{\mathbb{R}}
\def\cp{\mathbb{CP}}
\def\cK{\mathcal{K}}
\def\cL{\mathcal{L}}
\def\cM{\mathcal{M}}
\def\bphi{\overline{\phi}}
\def\vphi{\varphi}
\def\bz{\overline{z}}
\def\Li#1{\operatorname{Li}_{#1}}
\def\acnn#1#2#3{\Gamma^{#1}_{#2#3}}
\def\acnnb#1#2#3{\Gamma^{\overline{#1}}_{\overline{#2}\overline{#3}}}
\def\vol{\operatorname{Vol}}
\def\arccosh{\operatorname{arccosh}}
\def\eqa#1{\begin{equation}\begin{aligned}#1\end{aligned}\end{equation}}

\newcommand {\s}{\vspace{.25in}}
\newcommand{\beq}{\begin{equation}}   
\newcommand{\eeq}{\end{equation}}
\newcommand{\beqn}{\begin{eqnarray}}   
\newcommand{\eeqn}{\end{eqnarray}}
\newcommand{\note}[1]{\marginpar{\tiny #1}}
\newcommand{\pt}{\partial}

\newcommand{\Cc}{{\mathcal C}}
\newcommand{\pdvm}{\partial_{--}}
\newcommand{\gagp}{A_{++}}
\newcommand{\gagm}{A_{--}}
\newcommand{\olra}[1]{\overset\leftrightarrow{#1}}
\newcommand{\dlt}[2]{\delta^{#1}_{#2}}
\newcommand{\mtr}[3]{#1_{#2\bar{#3}}\,}
\newcommand{\imtr}[3]{#1^{#2\bar{#3}}\,}
\newcommand{\mcn}[3]{\Gamma^{#1}_{#2#3}}
\newcommand{\cdp}[4][]{#2^{#3}_{(#4)#1}}

\newcommand{\rc}{{\mathcal R}}
\newcommand{\nc}{{\mathcal N}}
\newcommand{\ntwo}{{\mathcal N}=2}
\newcommand{\nfour}{{\mathcal N}=4}
\newcommand{\none}{{\mathcal N}=1}
\def\ntwot{${\mathcal N}=(2,2)\;$}
\def\ntwoo{${\mathcal N}=(0,2)\;$}

%\newcommand{\cpn}{CP$(N-1)\;$}


\newcommand{\gsim}{\lower.7ex\hbox{$
\;\stackrel{\textstyle>}{\sim}\;$}}
\newcommand{\lsim}{\lower.7ex\hbox{$
\;\stackrel{\textstyle<}{\sim}\;$}}
\setcounter{table}{0}


%\usepackage{showlabels}

\begin{document}

\begin{titlepage}

   \begin{flushright}
    FTPI-MINN-23-05, UMN-TH-4211/23\\
    \end{flushright}




\begin{center}
{\Large \bf Remarks on Baby Skyrmion
Lie-Algebraic  \\[2mm]Generalization }

\end{center}

 \vspace{5mm}
    
    \begin{center}
    { \bf   Chao-Hsiang Sheu$^{a}$ and Mikhail Shifman$^{a, b}$}
    \end {center}
    
    \begin{center}
    
        {\it  $^{a}$Department of Physics,
    University of Minnesota,
    Minneapolis, MN 55455}\\{\small and}\\
    {\it  $^{b}$William I. Fine Theoretical Physics Institute,
    University of Minnesota,
    Minneapolis, MN 55455}\\
    
    \end{center}
    


\vspace{10mm}

\begin{center}
{\bf Abstract}
\end{center} 

We discuss generalized baby Skyrmions emerging in a (1+2)-dimensional $\sigma$ model with the Lie-Algebraic structure. The O(3) symmetry of the target space is lost in the general case, but O(2) is preserved.  Both the topological charge and the soliton mass are determined. Of a special interest is a limiting case of the large deformation parameter. The model under consideration interpolates between CP(1) and the cigar model.



\end{titlepage}

\section{Introduction}
\label{intro}

Baby Skyrmions in D=1+2 attracted much attention in condensed matter physics, especially in magnetic phenomena.\footnote{The original baby Skyrmion
is in fact the Belavin-Polyakov instanton \cite{BP} in the 2D CP(1) model elevated to $D=1+2$ dimensions and reinterpreted as a static soliton excitation.}  
The literature on this topic is enormous, including studies of the baby Skyrmion Hall effect of the texture and the topological Hall effect of the electron  (see e.g. the review \cite{GMT}). In this paper we consider deformed baby Skyrmions based on Lie-algebraic construction of \cite{LMZ,Los}. Deformed in this way 
the baby Skyrmion continuously interpolates between the standard Polyakov-Belavin baby Skyrmion  and the so-called cigar models \cite{FOZ} (for a review see \cite{LZ}).

The standard Heisenberg model -- a prototype model in this range of questions -- can be written as\,\footnote{Also referred to as the O(3) model.}
\beq
{\mathcal L} =\frac{1}{2g^2} (\pt S_i) (\pt S_i)\,,\qquad \vec{S} \vec{S}=1\,.
\label{one}
\eeq
where $\vec S$ is a unit isovector describing spin, $\vec{S} =\{ S_1, S_2, S_3\}$,  and $g^2$ is the coupling constant with mass dimension $[m^{-1}]$. Its target space is $S^2$, two-dimensional sphere.

Various generalizations of the standard  baby Skyrmions were considered previously (e.g. \cite{KH,BS} and references therein). Here
we use a Lie-algebraic construction of \cite{Los,LMZ} to deform the Polyakov-Belavin baby Skyrmion in a special way. 
Assuming that the deformed target space preserves O(2)$\times Z_2$ of the original O(3) we arrive at
\beq
{\mathcal L} =\frac{1}{2g^2(S_3)} (\pt S_i) (\pt S_i)\,,\qquad \vec{S} \vec{S}=1\,.
\label{two}
\eeq
where the coupling $g^2$ becomes a function of $S_3$, the third component of the isovector $\vec S$,
\beq
g^2(S_3) = g^2\cdot  \left(
\textstyle{\frac{1+k}{2}}+\textstyle{\frac{1-k}{2}}\,S_3^2\right) .
\label{three}
\eeq
Moreover,  $k$ is a numerical parameter to be defined below. At $k\to 1$ we return to the O(3) model while 
at $k\to+\infty$ with $g^2\cdot k =$const we approach the cigar model.
Equation (\ref{three}) leads to the following scalar curvature ${\mathcal  R}$ of the target space:
\beq
{\textstyle\frac{1}{ 2}} {\mathcal  R}= g^2\,\frac{(1+k) - (1-k)S_3^2}{(1+k) + (1-k)S_3^2}= g^2 \left[
1+ \frac{2(k-1)S_3^2}{(1+k) + (1-k)S_3^2}\right].
\label{four}
\eeq
When we consider $k\gg 1$ the scalar curvature is small in the equatorial region and almost everywhere else, $\sim g^2\sim 1/k$, with the exception of two polar domains
where $1-S_3^2\sim 1/k$. 
Indeed,
\beq
{\textstyle{\frac{1}{ 2}} {\mathcal  R}} =
 \left\{
\begin{array}{l} 
g^2 \,\,\,{\rm equator},
\\[1mm] g^2 k\,\,\,{\rm poles}.
\end{array}
\right.
\nonumber
\eeq
In the polar domains the scalar curvature $\sim g^2k\sim O(1)$, i.e. the ratio of the polar scalar curvature to equatorial is $\sim k\gg 1$. 

Equations (\ref{two}) and (\ref{three}) do not present the most general extension. The extensions breaking O(2) of the target space (rotations around the third axis in the isospace in (\ref{two}), (\ref{three}))  can be obtained 
in a similar way; see footnote before Eq. \eqref{1one}.

Organization of the paper is as follows. In Sect. \ref{lac} we review the Lie-algebraic construction which lies 
in the basis of the generalization under consideration. In Sect. \ref{bsi} we  calculate the baby Skyrmion mass (equal to the action of the
Polyakov-Belavin instanton in the 2D O(3) model). In Sect. \ref{lkl} we will study a special limit corresponding to the cigar models of \cite{FOZ}
also known as the metric of a 2D Euclidean black hole \cite{EFR,EW}.

\section{Lie-algebraic construction}
\label{lac}

The simplest Lie-algebraic construction is based on the $sl (2)\times sl(2)$ algebra
\cite{Los,LMZ}.
In the CP(1) model the target space is one-dimensional (one complex dimension) and is parametrized by a single complex field $\phi$ and its complex conjugated. Our task is to build a class of Lie-alegebaric (LA)  extensions.
The  $sl(2)\times sl(2)$  generators can be represented in the form
\beqn
&& T^+=-\phi^2\,d_\phi\,,\quad T^0= \phi\,d_\phi\,,\quad T^-= d_\phi\,\nonumber\\[2mm]
&& \bar T^+=-\bar\phi^2\,d_{\bar\phi}\,,\quad \bar T^0= \bar\phi\,d_{\bar\phi}\,,\quad \bar T^-= d_{\bar\phi}\,.
\label{five}
\eeqn
where one of the two $sl(2)$ algebras is holomorphic and the other antiholomorphic.
The commutation relations are standard for $sl(2)$,
 \beq
[T^+,T^-] = 2 T^0\,,\quad [T^+,T^0] = - T^+\,,\quad [T^-,T^0] = + T^-\,,
\label{six}
\eeq 
with the structure constants 
\beq
f^{+-}_{\quad 0}=2, \quad f^{+0}_{\quad +}=-1, \quad  f^{-0}_{\quad -}=1,
\label{seven}
\eeq
and similar for $\bar T$s.

Generically, the Lie-algebraic metric (with the upper indices) must take the form of a quadratic in $T$  combination
\beq
G^{1\bar 1}  d_\phi d_{\bar\phi} = \sum_{a,\bar b} {\mathcal P}_{a\bar b} T^a \bar{T}^{\bar b}
\label{eight}
\eeq
with a set of numeric coefficients $\{ {\mathcal P}_{a\bar b} \}$.

 Assuming  that the target space preserves a residual U(1) symmetry,
 we can  reduce the set  of coefficients $\{ {\mathcal P}_{a\bar b} \}$ to the diagonal form
\beq
\{{\mathcal P}\} = \{
{\mathcal P}_{1\bar 1}\equiv n_1, \quad {\mathcal P}_{2\bar 2}\equiv n_2,\quad {\mathcal P}_{3\bar 3}\equiv n_3\}
\label{ten}
\eeq
with all  off-diagonal ${\mathcal P}_{a\bar b}$ vanishing. Without loss of generality we can choose the set of coefficients (\ref{ten}) real.
Then the metric with the lower indices $G_{1\bar 1}$ takes the form\,\footnote{In the most generial case the Lie-algebraic metric $G_{1\bar 1}$  is parametrized as
$$
G_{1\bar 1}= \frac{1}{n_1+n_2 \bar\phi \phi +n_3 \bar\phi^2\phi^2
+\left(m_1\phi+ m_2\phi^2+m_3 \phi\bar\phi^2 +
{\rm H.c.} \right) }
$$}
\beq
G_{1\bar 1} = \frac{1}{n_1 +n_2\bar\phi \phi + n_3 (\bar\phi\phi)^2}\,\,.
\label{1one}
\eeq
We arrive at the Lagrangian
\beq
{\mathcal L} =G_{1\bar 1}\left(\pt_\mu\bar\phi \pt^\mu\phi\right).
\label{1two}
\eeq
If the coefficients $n_{1,3}$ are nonsingular (i.e. neither 0 nor $\infty$), by rescaling the fields $\phi,\bar\phi$,
\beq
\phi, \bar\phi\to \lambda \phi, \lambda\bar\phi,\qquad \lambda^2 =\sqrt{\frac{n_1}{n_3}}
\eeq
one can always make the first and the third coefficients equal to each other.\footnote{This may not be the case if, say $n_1\to 0 $, see below.}  One can keep this in mind.
If so, one  can conveniently  parametrize $n_{1,2,3}$ as  as follows,
 \beq
n_1= n_3= \frac{g^2}{2}, \quad  n_2 = g^2k\,.
\label{twop}
 \eeq
 The only extra parameter compared to CP(1) is $k$. If $k =1$ we return to CP(1). Another interesting limit to be discussed below is $k\to +\infty$.

The geometry of the space (\ref{1one}) is K\"ahlerian. The K\"ahler potential is
\begin{multline}
    \cK = -\frac{1}{g^2\sqrt{k^2-1}}\Bigg\{ 
        \Bigg[\log(-\frac{\phi\bphi}{\sqrt{k^2-1}+k})
        \log(\phi\bphi+\sqrt{k^2-1}+k)\\[1mm]
        +\Li{2}\left( \frac{\phi\bphi+\sqrt{k^2-1}+k}{\sqrt{k^2-1}+k} \right)\Bigg]
        - \left( k +\sqrt{k^2-1}\to k -\sqrt{k^2-1}\, \right)
    \Bigg\}
    \label{eq:kahler}
\end{multline}
for $k\in [1,\infty]$.
Further geometric data is given by the following expressions:
%\begin{align}
\beqn
\label{eq:curvature}
    &&\Gamma^{1}_{11} = -\frac{2(k+\bphi\phi)\bphi}{1+2k \bphi\phi+3(\bphi\phi)^2}\,,\quad R_{1\bar{1}1\bar{1}} =\! -\frac{4\left[k + 2\phi\bphi+k(\phi\bphi)^2\right]}{g^2\left[(\bphi\phi)^2 + 2k \bphi\phi + \right]^3} \,,\quad\quad\notag\\[2mm]
    &&R_{1\bar{1}}\, = \,\frac{2\left[k + 2\phi\bphi+k(\phi\bphi)^2\right]}{\left[(\bphi\phi)^2 + 2k \bphi\phi + 1\right]^2} \,,
    \quad\notag\\[2mm]
    &&{\mathcal R}\, = \,\frac{2g^2\left[k + 2\phi\bphi+k(\phi\bphi)^2\right]}{(\bphi\phi)^2 + 2k \bphi\phi + 1}\,.
    \eeqn
%\end{align}
Correspondence between $\phi, \bar\phi$  and the O(3) representation through the  unit vector ${\vec S = \{S_i\}}$, $i=1,2,3$   (see Eq. (\ref{one})) is realized through the stereographic projection,
\beq
\phi= \frac{S_1+ iS_2}{1+S_3}\,,\quad \bar\phi= \frac{S_1- iS_2}{1+S_3}\,.
\label{17}
\eeq
Then the following equations ensue:


\beqn
&&\pt_\mu\bar \phi\,\pt^\mu \phi= 
\frac{1}{\left(1+S_3\right)^{2}}\, 
\pt_\mu\vec{S}\,\pt^\mu \vec{S}\,,
\label{18}
\\[2mm]
&&n_1+n_2(\bar\phi\phi ) + n_3(\bar\phi\phi )^2 = g^2 \frac{1}{\left(1+S_3\right)^2}\left[(1+k) + (1-k) S_3^2
\right].
\label{19}
\eeqn
Combining (\ref{18}) and (\ref{19}) we arrive at Eqs. (\ref{two}), (\ref{three}). 


\section{Baby Skyrmions (Instantons in 2D)}
\label{bsi}

In the subsequent discussion, we will focus on $k \geq 1 $. 

The topological charge $Q$ of the 
baby Skrmion (which is the same as that of 2D deformed $\cp(1)$) is defined by the pullback of the K\"ahler form on the target space (with suitable normalization) \cite{Perelomov:1987va}, say,
\begin{align}\label{20}
    Q = \frac{1}{\vol(\cM)}\int_{\cM}\frac{\dd^{2}\phi}{(\bphi\phi)^2 + 2k \bphi\phi + 1}
    = \frac{1}{\vol(\cM)}\int_{S^2}\frac{\abs{\pdv*{\phi}{z}}^2-\abs{\pdv*{\phi}{\bz}}^2}{(\bphi\phi)^2 + 2k \bphi\phi + 1} \, \dd^{2}{z}\,.
\end{align}
Here $\vol(\cM)$ stands for  the volume of the target space $\cM$,
\begin{align}\label{21}
    \vol(\cM) = \frac{2\pi \arccosh{k}}{\sqrt{k^2-1}}
\end{align}
Two spatial dimensions ($x$ and $y$)  are parametrized by complex variables $z,\bz$. 
 The duality equation is the same as in the Polyakov-Belavin analysis \cite{BP}, and so is the instanton solution, see below Eq. \eqref{23}.
Then the corresponding instanton action reads 
\begin{align}
  S_{\rm inst} = \frac{2}{g^2}\int_{\cM}\frac{\d_{\mu}\phi \, \d^{\mu}\bphi{}}{(\bphi\phi)^2 + 2k \bphi\phi + 1}
  \dd^{2}z
  = \frac{2\vol(\cM)}{g^2} \,.
\end{align}
In 1+2 dimensions the 2D instanton action is reinterpreted as the baby Skyrmion mass $M_{\rm baby \, Sk}$; since the baby Skyrmion
saturates the BPS bound we have
\begin{align}
  M_{\rm baby \, Sk} =
    \displaystyle\frac{4\pi}{g^2}\frac{\arccosh{k}}{\sqrt{k^2-1}} \,,
  \quad k \geq 1\,,
  \label{22}
\end{align}
see Fig. \ref{fig:scalarrn2}.

Alternatively, the same result could be obtained directly, with no reference to the topological charge and the BPS saturation. 
Indeed, the duality equation remains the same as in the CP(1) model implying that $\phi$ is an analytic function of $z$ (the sum of poles) and 
$\bar\phi$ is the complex conjugated analytic function of $\bar z$. The minimal soliton presents just a single pole.
The appropriate solution can be chosen as follows,
\begin{figure}[H] 
    \centering 
    \includegraphics[width=8cm]{babyskyrm}
    \caption{\small The baby Skyrmion mass in the units of $4\pi/g^2$. If $k\to\infty$ the Skyrmion mass tends to zero as $k^{-1}\log 2k$.  }
    \label{fig:scalarrn2}
  \end{figure}
\beq
\phi(z) = \frac{a}{z-z_0} 
\label{23}
\eeq
The parameter $|a|$ has the meaning of the overall size of the solution and the phase of this parameter arg($a$) is a collective coordinate
reflecting the U(1) symmetry of the target space. We have 4 collective coordinates overall. After plugging the ansatz  \eqref{23} (with $z_0$ set to zero) in Eqs. (\ref{two}), (\ref{three}) we arrive at
the following  the density,
\begin{align}
    d\rho_{2}(x,y) 
    = d\tilde{z} d\bar{\tilde{z}}\,\, \frac{2}{g^2} \,\frac{1}{1 + 2k \,|{\tilde z}^2|+ |{\tilde z}^2|^2}
    \,,
    \label{24}
\end{align}
where
\beq
\tilde z = \frac{z}{a}\,.
 \label{25}
\eeq
After integrating the density above over the $\{x,y\}$ plane, we obtain the same mass as in \eqref{22}. 
The plots in Fig. \ref{fig:babysk} show the density distribution for different $k$.

Needless to say, if one wants to use the deformed baby Skyrmions discussed in this paper in description of magnetic phenomena their size must be stabilized.
To this end one can add the Dzyaloshinskii-Moriya term and a higher derivative term similar to that used in 4D Skyrmions.

\begin{figure}[H] 
  \centering 
  \includegraphics[width=\linewidth]{density4.png}
  \caption{\small The density distribution of the skyrmion solution for $k=0.5,1,2$. $a$ and $z_0$ are set at $1$ and $0$, respectively.}
  \label{fig:babysk}
\end{figure}


\section{Large-\boldmath{$k$} limit. The cigar model.}
\label{lkl}

In this section we discuss the large-$k$ limit which will lead us to the cigar model \cite{FOZ,LZ}. 

If we introduce
\beq
\gamma^2 =\frac{g^4}{4} \,(k^2-1)\,\,\,{\rm and}\,\,\, \kappa = \frac{g^2}{2} \,(k-1),
\label{26}
\eeq
then our Eq. (\ref{three}) takes the form
\beq
\kappa=\gamma\sqrt{\frac{k-1}{k+1}}\,,\quad  0 \leq \kappa\leq \gamma\,.
\label{27}
\eeq
With these definitions we can rewrite \eqref{two} as follows,
\beq
{\mathcal L}=\frac{1}{2\gamma^2}\,\frac{\pt_\mu\vec S\, \pt^\mu\vec S}{\kappa^{-1} -\kappa \gamma^{-2} S_3^2}
\label{28}
\eeq


A direct match of our result \eqref{28}  with Eq. (5.16) presented in Ref. \cite{LZ} is in order here. First, let us pass from from the three-component vector $\vec S$ as in (\ref{one})
to the standard angular representation which solves the constraint $\vec S^{\,2} =1$,
\begin{align}
  S_1 = \sin{\beta}\cos{\alpha} \,,\quad
  S_2 = \sin{\beta}\sin{\alpha} \,,\quad
  S_3 = \cos{\beta} \,,
\end{align}
where the azimuthal angle $\alpha \in (0,2\pi)$ and the polar angle  $\beta \in (0,\pi)$.
Next, we perform  the coordinate transformation 
\beq
\zeta = -\cos(\beta)
\label{zeta}
\eeq
 as suggested in \cite{LZ} .
Then the corresponding metric reduces to 
  \begin{align}
      \dd{s}^2 
      &= \frac{1}{g_1^2}\left[\frac{(\dd\beta)^2+\sin^2\beta(\dd\alpha)^2}{1-\kappa_1^2\cos^2\beta}\right]
      \label{29}
  \end{align}
  where $g_1$ and $\kappa_1$ are the coupling constant and the deformation parameter, respectively.
As a result,  \eqref{28} conincides with \eqref{29} provided we identify  the parameters $\kappa_1,\,g_1^2$ as follows
  $$
\kappa_1= \kappa\gamma^{-1} ,\quad g_1^2=2\kappa^{-1}\gamma^2\,.
$$

  To make an approximate evaluation of the length of the cigar,  we again start with the metric (5.16) in \cite{LZ} and perform the coordinate transformation
  \begin{align}
    \zeta \,\overset{{\rm def}}{=}\, \tanh{r}
    \label{30}
  \end{align}
  in which $r$ ranges from $-\infty$ to $\infty$ and will be associated with the length of the cylinder; and $\zeta$ is defined in (\ref{zeta}),
  \begin{align}
    \dd{s}^2 = \frac{1}{g_1^2}\left[\frac{2 (\dd{r}^2 + \dd{\alpha}^2)}{(1+\kappa_1^2) + (1-\kappa_1^2)\cosh{2r}}\right]
    \,\overset{?}{\approx}\,\,
    \frac{\rm const}{g_1^2}(\dd{r}^2 + \dd{\alpha}^2).
     \label{31}
  \end{align}
  If the denominator in the square brackets in (\ref{31}) is close to a constant then this metric is close to cylindrical, as noted in the second equality in (\ref{31}).
  This is what happens provided $|r|$ is sufficiently small. How small should $|r|$ be?
  
  If 
   \begin{align}
    \abs{r} \leq \frac{1}{2}\operatorname{arccosh}\left( \frac{1+\kappa_1^2}{1-\kappa_1^2} \right)
    = \frac{1}{2}\log(\frac{1+\kappa_1}{1-\kappa_1})\equiv r_*
     \label{32}
  \end{align}
  then the denominator in  the square brackets in (\ref{31})
  is $2$ in the ``equatorial" domain ($r$ close to zero) and approaches $2(1+\kappa_1^2)\sim 4$ in the ``polar" domains.
  We may view $2r_*$  defined in Eq.   \eqref{32} as the ``length'' of the ``cylinder".
  This defines the approximate constant in the last equation in (\ref{31}).
  \begin{figure}[t] 
    \centering 
    \includegraphics[width=.8\linewidth]{sausage.png}
    \caption{\small Some examples of the overall factor in the metric \eqref{31}. The contours from inside to outside (i.e. red to blue) correspond to $k$ equal to 1.0, 9.5, 200, and 1000, respectively. The  surfaces corresponding to the metric are embedded in the three-dimensional Euclidean space \cite{Belardinelli:1994dq,FOZ}.}
    \label{sausage}
  \end{figure}
  A sketch of the corresponding metric is presented in Fig. \ref{sausage}.
  
Thus, as was mentioned in Sect. \ref{intro}, the target space deforms from a sphere (of the $\mathbb{CP}(1)$ model) at $k=1$ to a cigar/sausage as $k = 200, 1000$.
The relation between $\kappa_1$ and $k$ is trivial, see \eqref{27},
  \begin{align}
    \kappa_1  = \sqrt{\frac{k-1}{k+1}}\,.
     \label{33}
  \end{align}
  We see that in terms of our original deformation parameter $k$ the expression for the cylinder ``ends" is as follows
  \begin{align}
 r_*
    = 
    \frac{1}{2}\log(\frac{\sqrt{k+1}+\sqrt{k-1}}{\sqrt{k+1}-\sqrt{k-1}})
    \sim
  \frac{1}{2}  \log(2 k)
   \label{34}
  \end{align}
  if $k$ is large enough. 
 
\section{Conclusions}
\label{concl}

In this paper first applications of the Lie-algebraic deformations of CP(1) model are considered. 
Hopefully deformed baby Skyrmions emerging in this construction can be used in exotic magnetic phenomena in condensed matter physics.
It is curious that the family of deformations we study includes the CP(1) model on the one hand and the cigar models on the other. 
The latter presents a special limit of a large deformation.

\section*{Acknowledgements}

We are grateful to C. Batista, O. Gamayun, A. Losev, and V. Lukyanov,  for extremely useful communications.

 This work is supported in part by DOE grant DE-SC0011842 and by Fulbright Scholarship. 
 This work was completed during MS visit  at 
the Institute for Particle and Nuclear Physics, Charles University (Prague, Czech Republic). MS thanks his colleagues from Charles University
for kind hospitality.
 



\begin{thebibliography}{99}

 \bibitem{BP}
 A.~M.~Polyakov and A.~A.~Belavin,
 {Metastable States of Two-Dimensional Isotropic Ferromagnets}, 
 JETP Lett. \textbf{22}, 245-248 (1975).

\bibitem{GMT}
B\"orge G\"obel, Ingrid Mertig, Oleg A. Tretiakov
Phys. Rept. {\bf 895}, 1 (2021) [arXiv:2005.01390]

\bibitem{LMZ}
A.~S.~Losev, A.~Marshakov and A.~M.~Zeitlin,
{\em On first order formalism in string theory,}
Phys. Lett. B \textbf{633}, 375-381 (2006)
%doi:10.1016/j.physletb.2005.12.010
[arXiv:hep-th/0510065 [hep-th]].

\bibitem{Los}
A. Losev et al., in preparation.

\bibitem{FOZ}
V.~A.~Fateev, E.~Onofri and A.~B.~Zamolodchikov,
{\em Integrable deformations of the $O(3)$ sigma model. The sausage model},
Nucl. Phys. B \textbf{406}, 521-565 (1993).

\bibitem{LZ}
S.~L.~Lukyanov and A.~B.~Zamolodchikov,
{\em Integrability in 2D fields theory/sigma-models},
 Les Houches Lect. Notes {\bf 106} (2019) 
% [doi:10.1093/oso/9780198828150.003.0006]

\bibitem{KH}
M.~Karliner and I.~Hen,
{\em Rotational Symmetry Breaking in Baby Skyrme Models,} [arXiv:0901.1489 [hep-th]]
% doi:10.1142/9789814280709\_0008


\bibitem{BS}
C.~D.~Batista, M.~Shifman, Z.~Wang and S.~S.~Zhang,
{\em Principal Chiral Model in Correlated Electron Systems,}
Phys. Rev. Lett. \textbf{121}, no.22, 227201 (2018)
%doi:10.1103/PhysRevLett.121.227201
[arXiv:1808.00633 [cond-mat.str-el]].

\bibitem{EFR}
S.~Elitzur, A.~Forge and E.~Rabinovici,
{\em Some global aspects of string compactifications},
Nucl. Phys. B \textbf{359}, 581-610 (1991)
%doi:10.1016/0550-3213(91)90073-7

\bibitem{EW}
E.~Witten,
{\em On string theory and black holes},
Phys. Rev. D \textbf{44}, 314-324 (1991)
%doi:10.1103/PhysRevD.44.314

\bibitem{Perelomov:1987va}
    A.~M.~Perelomov,
    {\em Chiral Models: Geometrical Aspects,}
    Phys. Rept. \textbf{146}, 135-213 (1987).

\bibitem{Belardinelli:1994dq}
L.~Belardinelli and E.~Onofri,
{\em The Numerical sausage,}
[arXiv:hep-th/9404082 [hep-th]].
    



\end{thebibliography}

\end{document}


