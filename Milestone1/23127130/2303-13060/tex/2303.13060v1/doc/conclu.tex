\section{Conclusion}
 We aim to synthesize diverse and legal layout patterns, and propose a practical method named \tool{DiffPattern}. Based on the given design rules our approach allows us to stably generate theoretically infinite legal layout patterns. The experiment results show that \tool{DiffPattern} outperforms previous SOTA methods by a large margin. Meanwhile, \tool{DiffPattern} is easy to extend and can be used on more complex pattern generation scenarios. In future work, we hope to extend our method to a more generic layout pattern generation approach.
 
 
 
 %We aim to transfer knowledge in the pose domain and propose an effective method named FlexPose. Our approach allows us to adapt an existing pose distribution to a different target one by using a few poses from the target dataset and generating theoretically infinite poses following the target distribution. FlexPose can be used on several pose-related works. In future work, we hope to extend our method to a more generic pose domain adaptation approach.