\section{Introduction}
\label{sec:intro}   

Reliable Very-Large-Scale Integration (VLSI) layout pattern libraries are the foundation of various designs for manufacturability (DFM) research, including perfection of design rules, Optical Proximity Correction (OPC) recipes \cite{gao2014mosaic}, lithography simulation \cite{kuang2013efficient,yu2015layout}, layout hotspot detection \cite{chen2019faster}, and so on. With the rapidly growing requirement for layout patterns in machine-learning-based lithography design applications, building a practical large-scale pattern library could be highly time-consuming due to the long logic-to-chip design cycle. 

Various rule-based and learning-based layout pattern generation methods have been proposed to address this issue.
Early rule-based methods \cite{reddy2018enhanced, ye2019lithoroc} first augment a set of predefined basic units by simple enhancement technology, {\it e.g.,} flipping and rotation. Then they randomly select several units and splice them together. However, patterns generated in this way lack diversity and are limited in quantity. Recently, learning-based generative methods \cite{yang2019deepattern, zhang2020layout, wen2022layoutransformer} have shown the ability to produce a large number of diverse layout patterns. Among them, pixel-based methods \cite{yang2019deepattern, zhang2020layout} regard the pattern generation problem as a binary image generation task.
A lossless pattern representation, {\it Squish Pattern}, is proposed by \cite{gennari2014topology} to reduce computational costs. Squish Pattern splits a large layout pattern into a significantly smaller binary topology matrix and two geometric vectors. Pixel-based methods synthesize a topology matrix with continual values and threshold it to get a new binary topology matrix, which wastes computation and hurts model capacity. Also, the sequential-based method 
in \cite{wen2022layoutransformer} models a layout pattern as a sequence of polygons, which is further decomposed into a sequence of vertices and directed edges. To get a new layout pattern, \cite{wen2022layoutransformer} generates a polygon sequence and translates it into a layout pattern. For both kinds of them, the generative model learns a latent regularization from the training set to avoid producing illegal layout patterns that violate the design rules.
We argue that the implicit constraint learned from the training set may be neither flexible nor reliable. Apart from the inconvenience that we need to train a new model on a specific large-scale dataset that follows the new design rules when the design rule changes, a considerable ratio of generated patterns violate the design rules in these methods.

In this paper, we propose \tool{DiffPattern}, a pixel-based practical layout pattern generation framework that consists of three main components: a) Inspired by the great success of diffusion models \cite{ho2020denoising,song2020denoising}, we approach the generation of topology as a denoising task for a random noise image. We utilize a discrete diffusion model to predict the noise that should be removed at every step. During the training and sampling procedure, each entry of the image tensor can be expressed by a discrete state in a finite state space, $\{0,1\}$ for example, and there is no need to manually set a threshold on a continual output range as previous work does. The naturally discrete output provides a reasonable regularization that avoids meaningless overfitting on how to produce a discrete-like output since all the pixels in the training samples are discrete in their values and position.
b) To further enlarge the information density of patterns and reduce computation costs, \tool{DiffPattern} adapts the idea of Squish Pattern Representation and pushes it one step forward. A kind of lossless layout pattern representation method called {\it Deep Squish Pattern} is proposed. Deep Squish Pattern folds a topology matrix into a topology tensor, which shrinks the input size and extends the channel dimension. Due to the fact that the efficiency of the existing diffusion model is more sensitive to the input size \cite{yang2022diffusion} and significantly less to the number of input channels, Deep Squish Pattern provides an easy-to-compute solution and can be efficiently applied to other pixel-based pattern generation methods. c) With newly generated topology matrices, we aim to assign geometric vectors to them and restore legal layout patterns. We develop a nonlinear system that can figure out the legal solution for each topology matrix and can be easily adjusted to every design rule.
Empowered by the interpretable pattern assessment strategy, \tool{DiffPattern} achieves a notable 100\% legality rate on generated layout patterns in our setting.

% In this subsection, we will introduce the detailed generation method of squish patterns based on the diffusion model. Diffusion models usually refer to denoising diffusion probabilistic models (DDPM), which have recently shown great potential in generating high-quality image samples, even outperforming previous VAEs, GANs, and Flow-based models. Due to its unique idea of adding noise and denoising, the diffusion models keep latent variables of the same dimension as the input rather than embedding them in a lower dimension, which also makes the diffusion models have a larger feature representation space than the other generative models mentioned above. In addition, unlike VAEs that require a trade-off between generative diversity and generative similarity, GANs whose training process is unstable due to adversarial training, and Flow-based models that require building complex reversible transformations and specific architectures, the diffusion models can directly use general neural network architectures, have a stable training loss function, and ensure the diversity and similarity of the generated samples at the same time.

% The diffusion models considers the process of adding noise and denoising to the sample, and represents it as a Markov chain. They slowly added random noise to the data in the forward diffusion process, and then learned the corresponding reverse diffusion process through the neural network. The reverse diffusion process can be used to generate data samples from random noise that conform to the original distribution.


Our main contributions can be summarized as follows:
\begin{enumerate}
    \item We develop a novel layout pattern generation method based on discrete denoising for synthesizing layout topology.
    \item We propose a lossless layout pattern representation strategy, Deep Squish Pattern, which accelerates pixel-based layout pattern generation schemes.  
    \item We utilize a nonlinear system for pattern assessment where the system provides a white-box method to legalize the layout patterns.  
    \item We extensively evaluate our methodology on benchmark datasets showing that \tool{DiffPattern} can achieve state-of-the-art (SOTA) performance.
\end{enumerate}









