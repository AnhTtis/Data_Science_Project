\documentclass{JHEP3}
\usepackage{amsmath}
%\input {epsf}
\usepackage{epsfig}
\usepackage{amssymb}
\usepackage{graphics}
%\usepackage[active]{srcltx}
%\usepackage{amsthm}
%\usepackage{shuffle}
\usepackage{url}
%\usepackage{tcolorbox}
%\usepackage{pgf,pgfarrows,pgfnodes,pgfautomata,pgfheaps,pgfshade}

\setlength{\oddsidemargin}{0.75in}
\setlength{\evensidemargin}{0.75in} \setlength{\topmargin}{0.75in}
\setlength{\textwidth}{7.0in} \setlength{\textheight}{8.5in}
\renewcommand{\baselinestretch}{1.2}
\jot=2mm


\newcommand{\lba}{\left[\begin{array}}
	\newcommand{\rba}{\end{array}\right]}
\newcommand{\bea}{\begin{eqnarray}}
	\newcommand{\eea}{\end{eqnarray}}
\newcommand{\lam}{\lambda}
\newcommand{\p}{\Pi}
\newcommand{\bean}{\begin{eqnarray}}
	\newcommand{\eean}{\end{eqnarray}}
\newcommand{\nn}{\nonumber \\}
\newcommand{\mat}[1]{( \matrix{#1} )}
\newcommand{\tmat}[1]{{\scriptsize \mat{#1}}}
\newtheorem{theorem}{\sf THEOREM}
\def\thetheorem{\thesection.\arabic{theorem}}
\DeclareMathOperator{\pslash}{\displaystyle{\not}}


\def\IZ{\mathbb{Z}}
\def\IR{\mathbb{R}}
\def\IQ{\mathbb{Q}}
\def\IC{\mathbb{C}}
\def\IP{\mathbb{P}}
\def\O #1{\overline{#1}}
\def\D #1{\dot{#1}}
\def\W #1{\widetilde{#1}}
\def\WH #1{\widehat{#1}}
\def\und #1{\underline{#1}}

\def\rightaction#1{\stackrel{\rightarrow}{\partial \over \partial #1}}
\def\leftaction#1{\stackrel{\lefttarrow}{\partial \over \partial #1}}
\def\func#1{\mathop{\rm #1}\nolimits}
\def\abs#1{\left| #1\right|}
\def\braket#1{\left\langle #1 \right\rangle}
\def\bra#1{\left\langle #1\right|}
\def\ket#1{\left| #1\right\rangle}
\def\gb #1{ \left\langle #1 \right]}
\def\tgb #1{ \left[ #1 \right\rangle}


\def\bbar#1{ \overline #1}
\def\Tr{\mathop{\rm Tr}}
\def\det{\mathop{\rm det}}
\newcommand{\fref}[1]{Figure~\ref{#1}}
\def\eref#1{(\ref{#1})}
\def\bit#1{\begin{array}{c} \\ \\ \end{array} \hspace{#1 cm}}
\def\d{{\rm d}}
\def\wt{\widetilde}
\def\wtl{\widetilde{\lambda}}
\def\wh{\widehat}
\def\wht{\widehat{x}^0}
\def\whx{\widehat{x}^1}
\def\th{{\theta}}
\def\bth{{\overline{\theta}}}
\def\a{{\alpha}}
\def\ba{{\overline{\alpha}}}
\def\da{{\dot{\alpha}}}
\def\b{{\beta}}
\def\db{{\dot{\beta}}}
\def\c{{\gamma}}
\def\dc{{\dot{\gamma}}}

\def\d{\partial}
\def\rmd{{\rm d}}
\def\la{\lambda}
\def\eps{\epsilon}
\def\lblb{({\bar\la}{\bar\la})}
\def\ald{{\dot\alpha}}
\def\bed{{\dot\beta}}
\def\gad{{\dot\gamma}}
\def\sid{{\dot\rho}}
\def\vev{\braket}
\def\tgb #1{ \left[ #1 \right\rangle}
\def\bra#1{\left\langle #1\right|}
\def\bbra#1{\left[ #1\right|}
\def\ket#1{\left| #1\right\rangle}
\def\bket#1{\left| #1\right]}
\def\bvev#1{\left[ #1 \right]}
\def\Spaa{\vev}
\def\Spbb{\bvev}
\def\Spab{\gb}
\def\Spba{\tgb}

\def\Label#1{\label{#1}%
	\smash{\hbox to0pt{\raise1ex\hbox{\tiny[#1]}\hss}}}
\newcommand{\bbibitem}[1]{\bibitem{#1}\marginpar{#1}}

\newtheorem{lemma}{Proposition}[section]
\newtheorem{corollary}{Proposition}[lemma]

\title{Reduction of two-loop Feynman integrals in parametric representation with syzygy trick }
\author{Hongbin Wang \footnote{Emails: 21836003@zju.edu.cn } \\
	{\small Zhejiang Institute of Modern Physics, Zhejiang University, Hangzhou, 310027, P. R. China }
}
\date{\today}
\abstract{Reduction  of high-loop Feynman integrals is one of the main tasks in scatting amplitude. In this paper, a new representation of Feynman integrals proposed by Chen in \cite{chen1,chen2} is considered.  We combined Chen's method with "syzygy" trick to simplify the IBP relations, and successfully canceled the dimensional shift and the unwanted doubled propagators. Moreover, we improved the method to deal with tensor's structure. We demonstrated our method using 
 three two-loop integrals to show our method, and presented the analytical reduction coefficients in the top-sector.
}



\keywords{Amplitudes, Feynman integrals, loop reduction }



\newpage
\begin{document}
	%%%%%%%%%%%%%%%%%%%%%%%%%%%%%%%%%%%%%%%%%%%%%%%%
	\section{Introduction}
	%
	The analytic calculation of scatting amplitude for a given process is an interesting and important task. Since the complexity for high-loop calculation, many new and novel technique have been proposed in recent years. The most used method is the Integrating-by-Part(IBP) method \cite{smir,ibp1,ibp2}, which reduce the Feynman integrals to the master integrals(MI). However, to get the reduction results, there are usually a great number of IBP relations in practical calculations, which is hard to solve. Finding more efficient reduction methods becomes an important direction.
	%
	%Unitarity cut method is one alternative reduction method which has been proved to be very useful for one-loop integrals  \cite{Bern:1994zx,Bern:1994cg,Britto:2004nc,Cachazo:2004zb,Britto:2005ha,u1,u2,u3,u4,Britto:2009wz,Anastasiou:2006jv,Anastasiou:2006gt}. To complete this method and generalized it to high poles' case, we combined it with  the differential operator $\frac{\partial}{\partial m^2}$ and successfully used it to reduce the one-loop bubbles, triangle, box and pentagon to the scalar basis in our previous work \cite{wang}. For the tensor's case, the well-known PV-reduction method has been proved useful and efficient.
	%
	One way to  calculate and reduce the Feynman integrals is to establish  representation in other space rather than the traditional momentum space. There are many well-known representation such as Feynman parametric representation \cite{Bern:1992em,Bern:1993kr} and  Baikov representation \cite{Baikov:1996rk,Baikov:1996cd}.
	
	In recent years, Chen proposed a new parametric representation for Feynman integrals \cite{chen1,chen2}. One of the benifit in the new representation is that  the number of IBP identities is usually less than he traditional momentum representation. For example, with $n$  propagators  the number of IBP relations in Chen's form is
	 $n+1$. However, to find the recursion relation using the IBP identities in the parametric representation, there will naturally be some terms in different spacetime dimensions. Since we usually care about the reduction in the same dimension $D$, we need to cancel the dimensional shift. The trick  to cancel the unwanted terms is known as solving "syzygy" equation \cite{Kosower,zhang1,zhang2,Larsen:2015ped,Larsen:2016tdk,Zhang:2016kfo,Jiang:2017phk}.
	%
	
	In our recent work \cite{2021General}, we consider the Chen's method \cite{chen1,chen2} at one-loop case, when the homogeneous polynomial $F$ is at degree 2. By this fact  we proposed a method by adding an antisymmetry matrix $\hat K_{A}$ with many parameters. By choosing the parameters, we are able to cancel the  terms with dimension shift in the IBP relation, and at the same time simplified the IBP relations by dropping the integrals with doubled propagators. In this paper, we considered the two-loop case, when the polynomial $F$ is at degree 3. We proposed the syzygy trick in the parametric representation and gave some explicit examples. By the technique, we got the independent IBP relations which does not contain the dimensional shift and the unwanted terms with  higher power for propagators. Further more, we considered the tensor's case. After giving a brief review of the tensor's case in Chen's parametric representation, we gave some examples to show the "syzygy" trick  in reduction of tensor's integrals.
	%
	
	The plan of this paper is following. In section 2, we gave a brief review of Chen's parametric representation of Feynman integrals in scalar's and tensor's cases. In section 3, we show how we improved the IBP recurrence relation to cancel the dimensional shift and simplify the IBP relations  in both scalar's and tensor's case. By the improved IBP identities, in  section 4, 5, and 6, we gave three different examples to show our method and compared it with others. Since the analytic expressions for various quantities are long, some of them have been presented in the Appendix.
\section{Review of Chen's method}

In this section, we will briefly discuss the
method proposed in \cite{chen1,chen2}
by Chen. This method depends crucially on a
different parametrization form, which is obtained by
adding one more integral variable over the
familiar Feynman parametrization.   For example, for the
scalar integrals
\bea
%
I(L;\lambda_1+1,\cdots ,\lambda_n+1)&=&\int d^Dl_1\cdots d^Dl_L\frac{1}{D_1^{\lambda_1+1}\cdots D_{n}^{\lambda_n+1}}~~~\label{Wang-2-2}
%
\eea
%
the familiar Feynman parametrization leads to
\bea
%
I(L;\lambda_1+1,\cdots ,\lambda_n+1)&=&(-i)^{n+ \lambda-\frac{DL}{2}}\Gamma(n+\lambda-\frac{DL}{2})\int dx_1\cdots dx_n\delta (\sum_{j\in S} x_j-1)U^{\lambda_u}f^{\lambda_f}x_1^{\lambda_1}\cdots x_n^{\lambda_n}~~~\label{1.18}
%
\eea
%
where
%
\bea
U(x) & = & Det~A,~~~~~~~~f(x)=-V(x)+U(x)\sum m_i^2x_i\nn
%
\lambda&=&\sum_{i=1}^{n} \lambda_i,~~~~
%
\lambda_u=n+\lambda-\frac{D}{2}(L+1),~~~~
%
%
\lambda_f=-n-\lambda+\frac{DL}{2}~~~\label{Wang-2-8}
%%
\eea
%
and $U(x),V(x)$ are defined through
%
\bea
& & \sum_{i}^{L}\a_iD_i=\sum_{i,j}^{L}A_{ij}l_i\cdot l_j+2\sum_{i=1}^{L}B_i\cdot l_i+C\nn
& & C-\sum A_{ij}^{-1}B_i\cdot B_j\equiv \frac{V(\a)}{U(\a)}-\sum m_i^2\a_i ~~~\label{Wang-2-5}
%
\eea
%%
To go to Chen's parametrization, we use the Mellin transformation
%
\bea
%
A^{\lambda_1}B^{\lambda_2}&=&\frac{\Gamma(-\lambda_1-\lambda_2)}{\Gamma(-\lambda_1)\Gamma(-\lambda_2)}\int _0^{\infty} dx  (A+Bx)^{\lambda_1+\lambda_2}x^{-\lambda_2-1}~~~\label{Wang-2-9}
%
\eea
to rewrite \eref{1.18} as
\cite{chen1,chen2} \footnote{Notice that the $\lambda_i$ is different from the power in the integral in \eref{Wang-2-2}.}
\bea
%
I[\lambda_1+1,\cdots ,\lambda_n+1]&=&(-1)^{n+\lam} i^{L}\pi^{\frac{LD}{2}}\frac{\Gamma(-\lambda_0)}{\Pi_{i=1}^{n+1}\Gamma(\lambda_i+1)}i[\lambda_0;\lambda_1,\cdots ,\lambda_n]~~~~\Label{defI}
%
\eea
%
with \footnote{Here the F is given by $F=U(B^T A^{-1}B -C +x_{n+1})$, with the $A$, $B$ and $C$ are given in \eref{Wang-2-5}
	%determined by \\$\sum_{i=1}^{n}x_iD_i=\sum_{i,j=1}^{L}A_{i,j}l_i\cdot l_j+2\sum_{i=1}^{L}B_i \cdot l_i+C$, and $U=Det[A]$.} \footnote{For more details, see \cite{Wang:2021ury}
	.}
%
\bea
%
&&\lambda_0=-\frac{D}{2},~~~\lam\equiv \sum_{i=1}^{n}\lambda_i,~~\lambda_{n+1}=-n-1-\sum_{i=1}^{n}\lambda_i+\frac{LD}{2},~~~F=U x_{n+1}+f\nn
%
&&i[\lambda_0;\lambda_1,\cdots ,\lambda_n]\equiv \int d\Pi^{(n+1)}F^{\lambda_0}x_1^{\lambda_1}\cdots x_n^{\lambda_n}x_{n+1}^{\lambda_{n+1}},~~~d\Pi^{(n+1)}\equiv dx_1\cdots dx_{n+1}\delta(\sum_{i\in S}x_i-1)~~~~\Label{pin+1}
%
\eea
%%%%
%
%
Here $S$ is an arbitrary non-trivial subset of $\{1,2,\cdots n\}$.
Notice the $\lambda_0$ is the dimensional regularization parameter with the space-time dimension $D$.	

To deal with tensor integrals
\bea I[\lambda_0;\lambda_1+1,\cdots,\lambda_n+1]^{\mu_1\mu_2\cdots \mu_m}&\equiv & \int d^{D}l_1\cdots d^{D}l_L \frac{l_{i_1}^{\mu_1}\cdots l_{i_m}^{\mu_m}}{D_1^{\lambda_1+1}\cdots D_n^{\lambda_n+1}}~~~~~~\Label{tendef}
\eea
we could  use the identity
%
\bea
%%
l_{i_1}^{\mu_1}l_{i_2}^{\mu_2}\cdots l_{i_m}^{\mu_m}&=&\frac{i(-1)^m}{\Gamma(m+1)}\Big\{\frac{\partial }{\partial q_{i_1,\mu_1}}\frac{\partial }{\partial q_{i_2,\mu_2}}\cdots \frac{\partial }{\partial q_{i_m,\mu_m}}\int _0^{\infty}dy \exp[-iy(1+\sum_{i=1}^{L}q_i\cdot l_i)]\Big\}|_{q_i^\mu\to0}
%%
\eea
to rewrite  the \eref{tendef} as
%
\bea
%
&&I[\lambda_0;\lambda_1+1,\cdots\lambda_n+1]^{\mu_1\mu_2\cdots \mu_m}\nn
%
&=&\frac{i(-1)^{m}}{\Gamma(m+1)}\frac{\d }{\d q_{i_1,\mu_1}}\cdots\frac{\d }{\d q_{i_m,\mu_m}}\int \frac{d^{D}l_1\cdots d^{D}l_L}{D_1^{\lambda_1+1}\cdots D_n^{\lambda_{n}+1}}\int _0^{\infty}dy \exp[-iy(1+\sum_{i=1}^Lq_i\cdot l_i)]|_{q_i^{\mu}\to0}~~~~\Label{tensor1}
%
\eea
The integration of $y$ can be carried out to give ${1\over D_0}$ with $D_0=1+\sum_{i=1}^m q_i\cdot l_i$. 
In other words, the $\int dy$ can be interpreted as the Schwinger parametrization of ${1\over D_0}$. 
Doing the Schwinger parametrization for other $D_i, i=1,...,n$, it becomes the scalar integrals discussed in 
\eref{Wang-2-2} and we have 
%It can be interpreted as one additional propagator to the original integrals. Now the form becomes the  scalar integrals
%discussed in above with additional $D_0$ propagator. Doing similar things, i.e., writting
\bea
%
\sum_{i=1}^{n}x_iD_i-y D_0&\equiv &\sum_{i,j=1}^L A_{ij}l_i\cdot l_j+2\sum_{i=1}^L \W B_i \cdot l_i+\W C,~~~~\W B_i=B_i-\frac{y}{2}q_i,~~~~\W C=C-y
%
\eea
%
for which we can read out
\bea
%
U(x)=Det[A],~~~f(q,y)&=&U(\W B^T A^{-1}\W B -\W C),~~F(q,y)=x_{n+1}U+f(q,y)
%
\eea
After doing the integration over $l_1$ to $l_L$
we arrive
\bea
%
I[\lambda_0;\lambda_1+1,\cdots ,\lambda_n+1]^{\mu_1\cdots \mu_m}&=&(-1)^{n+\lam+1}i^{L}\pi^{\frac{LD}{2}}\frac{\Gamma(-\lambda_0)}{\Pi_{i=1}^{n}\Gamma(\lambda_i+1)\Gamma(\lambda_{n+1})}i[\lambda_0;\lambda_1,\cdots ,\lambda_n]^{\mu_1\cdots\mu_m}~~~\Label{Itoi}
%
\eea
%
where
\bea
%%
&&i[\lambda_0;\lambda_1,\cdots,\lambda_n]^{\mu_1\cdots\mu_m}\nn
%
&\equiv &\frac{(-1)^m}{\Gamma(m+1)}\frac{\d }{\d q_{i_1,\mu_1}}\frac{\d }{\d q_{i_2,\mu_2}}\cdots\frac{\d }{\d q_{i_m,\mu_m}}\int dx_1\cdots dx_{n+1}dy\delta(\sum_{j\in S}x_j-1)F(q,y)^{\lambda_0}x_1^{\lambda_1}\cdots x_{n+1}^{\lambda_{n+1}-1}|_{q_i\to0}\nn
%
&=&\frac{(-1)^{m}}{\Gamma(m+1)}\frac{\d }{\d q_{i_1,\mu_1}}\cdots\frac{\d }{\d q_{i_m,\mu_m}}\int d\Pi^{(n+1)}\int _0^{\infty }dy~F(q,y)^{\lambda_0}x_1^{\lambda_1}\cdots x_{n+1}^{\lambda_{n+1}-1}|_{q_i\to0}~~~\Label{tensor2}
%%
\eea
%
It is worth noticing that  here we have totally $n+1$ propagators, so there is a factor $\frac{1}{\Gamma(\lambda_{n+1})}$ not $\frac{1}{\Gamma(\lambda_{n+1}+1)}$ which fixing the power of $x_{n+1}$ with the definition of $\lambda_{n+1}$ as  $\lambda_{n+1}\equiv -(n+1)-\sum_{i=1}^{n}\lambda_i+\frac{LD}{2}$.
%
To continue,  we  decompose the function $F(q,y)$ into the form
%
\bea
%
F(q,y)&=&F(0,0)+yU-yU\sum_{i,j=1}^L[A^{-1}_{ij} B_i\cdot q_j]|_{y=0}+\frac{1}{4}y^2U\sum_{i,j=1}^L[A^{-1}_{ij}q_i\cdot q_j]|_{y=0}\nn
%
&=&F(0,0)+yU+y\sum_{=1}^L b_i \cdot q_i+\sum_{i,j}^L c_{ij}y^2 q_i\cdot q_j~~~\label{q-dep}
%
\eea
%
where $F(0,0)$ is the same one in \eref{pin+1} and  $b$ and $c$ are  polynomials of x's
%
\bea
%
b_i&=&-\sum_{i=1}^L -UA_{ij}^{-1}\W B_j|_{y=0}=-\sum_{i=1}^L U A_{ij}^{-1} B_j,~~~~~~~c_{ij}=\frac{1}{4}UA_{ij}^{-1}|_{y=0}=\frac{1}{4}U A_{ij}^{-1}~~~\Label{b,c}
%
\eea
%%%
Since the $q$ dependence has been explicitly written done in \eref{q-dep},  carrying out
the differential operators $\frac{\d }{\d q_{i_j,\mu_j}}$ in \eref{tensor2} we arrive following sum
\bea
%
\sum_{n_0=\la_0-m}^{\la_0}\sum_{ i_t}%\sum_{\substack{n_0=\lambda_0-m\\i_1,\cdots, i_n}}^{\lambda_0}
c[n_0]^{\mu_1\cdots\mu_m}_{i_1,\cdots, i_n}  \int d\Pi^{(n+1)}\int _0^{\infty}dy \Big\{F(0,0)+yU\Big\}^{n_0}y^{\delta_y}x_1^{\lambda_1+i_1}\cdots x_n^{\lambda_n+i_n}x_{n+1}^{\lam{n+1}-1}~~~~~~~~~~~~~\Label{three}
%
\eea
%
where the sum over all possible choices of $i_t$ with $t=1,...,n$ such that 
%
\bea i_t\geq 0,~~~~~\sum_{t=1}^n i_t=(\lambda_0-n_0)(1+L)-m\eea
%
The coefficients $c[n_0]^{\mu_1\cdots\mu_m}_{i_1,\cdots, i_n}$ can be worked out although  expressions
are little complicated. Another point is that although the sum is up to $\la_0$, one can check that 
with $n_0>\frac{2\lambda_0-m}{2}$ $c[n_0]$  is always zero by the degree of $q_i$s.


%The coefficient $c[n_0]$ with $n_0>\frac{2\lambda_0-m}{2}$ is always zero by the degree of $q_i$s. For our later discussion, the explicit form of $c[n_0]$ is not essential,
%and we have presented the recurrence relation of $c[n_0]^{\mu_1\cdots\mu_m}_{i_1,\cdots, i_n}$ in the appendix.(??????)
% {\bf and in the our following calculation, it is easy to be understood. Also, we gave a operator's version in the next subsetion. }

Recalling the $F(0,0)$ given in \eref{pin+1}, we can combine $x_{n+1}+y=\W x_{n+1}$ or do the replacement
\bea
%
x_{n+1}\to x_{n+1}-y
%
\eea
and the \eref{three} becomes to
%
\bea
%
\frac{(-1)^m}{\Gamma(m+1)}\sum_{\substack{n_0=\lambda_0-m\\i_1+\cdots i_n+m\\=(\lambda_0-n_0)(L+1)}}^{\lambda_0} && c[n_0]^{\mu_1\cdots\mu_m}_{i_1,\cdots, i_n}\int d\Pi^{(n+1)}\int dy\Big\{F(0,0)\Big\}^{n_0}\times y^{\delta_y}x_1^{\lambda_1+i_1}\cdots x_n^{\lambda_n+i_n}(x_{n+1}-y)^{\lambda_{n+1}-1}~~~\Label{tensor4}~~~
%
\eea
%
Using
%
\bea
%
\int  dy y^{\delta_y}(x_{n+1}-y)^{\lambda_{n+1}-1}=\frac{x_{n+1}^{\lambda_{n+1}+\delta_y}\Gamma(1+\delta_y)\Gamma(\lambda_{n+1})}{\Gamma(1+\delta_y+\lambda_{n+1})}
%
\eea
we got
%
\bea
%
&&\frac{(-1)^m}{\Gamma(m+1)}\sum_{\substack{n_0=\lambda_0-m\\i_1+\cdots i_n+m\\=(\lambda_0-n_0)(L+1)}}^{\lambda_0}  c[n_0]^{\mu_1\cdots\mu_m}_{i_1,\cdots, i_n}\frac{\Gamma(1+\delta_y)\Gamma(\lambda_{n+1})}{\Gamma(1+\delta_y+\lambda_{n+1})}\int d\Pi^{(n+1)}F(0,0)^{n_0}x_1^{\lambda_1+i_1}\cdots x_n^{\lambda_n+i_n}x_{n+1}^{\lambda_{n+1}+\delta_y}\nn
%
&=&\frac{(-1)^m}{\Gamma(m+1)}\sum_{\substack{n_0=\lambda_0-m\\i_1+\cdots i_n+m\\=(\lambda_0-n_0)(L+1)}}^{\lambda_0} c[n_0]^{\mu_1\cdots\mu_m}_{i_1,\cdots, i_n}\frac{\Gamma(1+\delta_y)\Gamma(\lambda_{n+1})}{\Gamma(1+\delta_y+\lambda_{n+1})}i[n_0;\lambda_1+i_1,\cdots,\lambda_n+i_n]~~~\Label{sum}
%
\eea
%
%
%
Putting the result \eref{sum} back to \eref{Itoi}, we have
%
\bea
%
I[\lambda_0,\lambda_1+1,\cdots,\lambda_{n}+1]^{\mu_1\cdots\mu_m}&= & (-1)^{n+\lam+m}i^{L+1}\pi^{\frac{LD}{2}}\frac{\Gamma(-\lambda_0)}{\Pi_{i=1}^n\Gamma(\lambda_i+1)}\nn & & \times\sum_{\substack{n_0=\lambda_0-m\\i_1+\cdots i_n+m\\=(\lambda_0-n_0)(L+1)}}^{\lambda_0} c[n_0]^{\mu_1\cdots\mu_m}_{i_1,\cdots, i_n} \frac{1}{\Gamma(m+\lambda_{n+1}+1)}i[n_0;\lambda_1+i_1,\cdots,\lambda_n+i_n]~~~~~~~~~~
\Label{sumI}
%
\eea
where we have used the fact that  after  taking the differential operators and setting the limit of $q_i\to0$,   $\delta_y=m$.
%
%\subsection{A simple example:  One-loop bubble with tensor structure}

To demonstrate the usage of \eref{sumI}, let us present two examples for tensor one-loop bubbles. Let us start
with the tensor rank one, i.e.,
%To demonstrate above discussions for the tensor reduction, let us consider the one-loop bubble with rank one tensor structure, i.e.,
%
\bea
I_2(n_1+1,n_2+1)^{\mu}&=&\int d^Dl\frac{l^{\mu}}{(l^2-m_1^2)^{n_1+1}((l-p)^2-m_2^2)^{n_2+1}}\nn
%
&=&(-1)^{2+n_1+n_2}i\pi^{\frac{D}{2}}\frac{\Gamma(-\lambda_0)}{\Gamma(n_1+1)\Gamma(n_2+1)\Gamma(\lambda_3)}i[\lambda_0;n_1,n_2]^{\mu}~~~\Label{I2111}
%
\eea
%
where 
%
\bea
%
i[\lambda_0;n_1,n_2]^{\mu}&=&\frac{-1}{\Gamma(2)}\frac{\d }{\d q_{1,\mu}}\int d\Pi^{(3)}\int_0^{\infty}dy F(q,y)^{\lambda_0}x_1^{n_1}x_2^{n_2}x_3^{\lambda_3-1}|_{q_1\to 0}~~~\Label{bubblen1n2}
%
\eea
with the polynomial $F(q,y)$
\bea
F(q,y)&=&-p^2 x_1 x_2 + m_1^2 x_1 (x_1 + x_2) +
m_2^2 x_2 (x_1 + x_2) + (x_1 + x_2) x_3 + (x_1 + x_2 + p\cdot  q_1 x_2) y + \frac{q_1^2y^2}{4}~~~~~~~
%
\eea
%
Carrying out  $\frac{\d }{\d q_{1,\mu}}$ the \eref{bubblen1n2} becomes to
%
\bea
i[\lambda_0;n_1,n_2]^{\mu}&=&\frac{-\Gamma(\lambda_3)p^\mu \lam_0}{\Gamma(2+\lambda_3)}i[\lambda_0-1;n_1,n_2+1]~~~\Label{exab}
\eea
%
with $c[\lambda_0-1]^{\mu}_{0,1}=-\lam_0 p^{\mu}$, 
thus 
%
\bea
%
I_2[n_1+1,n_2+1]^{\mu}&=&(-1)^{n_1+n_2}\pi^{\frac{D}{2}}\frac{\Gamma(-\lambda_0)\lam_0p^\mu}{\Gamma(n_1+1)\Gamma(n_2+1)\Gamma(2+\lambda_3)}i[\lambda_0-1;n_1,n_2+1]
%
\eea
%
%By the definition of \eref{sumI}, here $c[\lambda_0-1]^{\mu}_{0,1}=-\lam_0 p^{\mu}$.
%%%%
%
%\subsubsection{Bubble with two Lorentz index}
For bubbles with tensor rank two, by  \eref{tensor1} we have  
%
\bea
%
I_2[n_1+1,n_2+1]^{\mu\nu}&=&(-1)^{2+n_1+n_2}i \pi^{\frac{D}{2}}\frac{\Gamma(-\lam_0)}{\Gamma(n_1)\Gamma(n_2)\Gamma(\lam_3)}i[n_1,n_2]^{\mu\nu}
%
\eea
%
where (by \eref{tensor2})
%
\bea
%
i[n_1,n_2]^{\mu\nu}&=&\frac{1}{\Gamma(3)}\frac{\d }{\d q_{1,\mu}}\frac{\d }{\d q_{1,\nu}}\int d\Pi^{(3)}\int _0^{\infty} dy F(q,y)^{\lam_0}x_1^{n_1}x_2^{n_2}x_3^{\lam_3-1}|_{q_1\to0},~~~~\lam_3=-3-n_1-n_2-2\lam_0~~~~~~~~~
%
\eea
%
wtih  the corresponding function 
% 
%
\bea
%
F(q,y)&=&-p^2 x_1 x_2 + m_1^2 x_1 (x_1 + x_2) +
m_2^2 x_2 (x_1 + x_2) + (x_1 + x_2) x_3 + (x_1 + x_2 + p\cdot  q_1 x_2) y + \frac{q_1^2y^2}{4}~~~~~~~~~~
%
%
\eea
%
Carrying out the derivative, we get two terms
%
%
\bea
%
&&i[n_1,n_2]^{\mu\nu}=\lam_0(\lam_0-1)p^\mu p^\nu \frac{\Gamma(\lam_3)}{\Gamma(\lam_3+3) }i[\lam_0-2;n_1,n_2+2]+\frac{1}{2}\lam_0 g^{\mu\nu}\frac{\Gamma(\lam_3)}{\Gamma(\lam_3+3) } i[\lam_0-1;n_1,n_2]~~~\Label{tensorfram2}
%
\eea
%
where two  coefficients are 
%
\bea
%
c[\lam_0-2]^{\mu\nu}_{0,2}&=& \lam_0(\lam_0-1) p^\mu p^\nu,~~~c[\lam_0-1]^{\mu\nu}_{0,0}=\frac{1}{2}\lam_0 g^{\mu\nu}
%
\eea
%
with $\lam_0=-\frac{D}{2}$.
%
\section{The improved IBP relations with syzygy method}
%
%\subsection{Scalar's case}
Having reduced everything to the integral form \eref{pin+1}, we can establish the corresponding IBP relations in this frame. It is given by  \footnote{Notice that in the first term, the power of $x_{n+1}$ has been shifted before the differential, which keeps the degree of the integrand  $-n-1$. In general, one can choose arbitary $x_i$ to shift the degree, but for simplicity, we chose to shift the $x_{n+1}$ here.}
%
\bea
%
\int d\Pi^{(n+1)}\frac{\partial}{\partial x_i}\Big\{F^{\lambda_0}x_1^{\lambda_1}\cdots x_n^{\lambda_n}x_{n+1}^{\lambda_{n+1}+1}\Big\}+\delta_{\lambda_i,0}\int d\Pi^{(n)}\Big\{F^{\lambda_0}x_1^{\lambda_1}\cdots x_n^{\lambda_n}x_{n+1}^{\lambda_{n+1}+1}\Big\}|_{x_i\rightarrow 0}&=&0,~~i=1,\cdots n+1~~~~~~~\Label{ibpiden}
%
%
\eea
%
where the second term is the boundary term  which contributes to the sub-topology, i.e. the integral with the propagator $D_i$ having been removed. Notice in the first term the sum of the power of $x_i$ is one more than that in $i[\lambda_0;\lambda_1,\cdots , \lambda_n]$. There are total $(n+1)$ independent IBP recurrence relations. 
%To get the complete IBP recurrence relation, we should put the $n+1$ relations altogether. Further more, 
Since the action of differential operator $\frac{\partial }{\partial x_i}$  over the function $F$ will produce the integrals in different dimension, which makes the reduction procedure cumbersome,   we will use the syzygy trick suggested in [cite Kosower, Zhang etc] \footnote{In \cite{}, Zhang has proposed the method in Baikov represent.}.(1,2,7,8,9,10,11,12,13)
To overcome the difficulties and simplify the reduction procedure.

% we will use the syzygy trick suggested in [cite Kosower, Zhang etc] \footnote{In \cite{}, Zhang has proposed the method in Baikov represent.}.(1,2,7,8,9,10,11,12,13)
%

Let us combining the $n+1$ identities \eref{ibpiden}, with each one multiplied a polynomial factor $z_i$ in degree zero\footnote{Here we mean the  term like  $z_1=\frac{x_{i_1}^{\c_1}x_{i_2}^{\c_2}\cdots x_{i_k}^{\a_k}}{x_{n+1}^{\c_1+\c_2+\cdots+\a_k}}$, for the reason that the integrand of \eref{IBPsyz} must be degree $-n-1$, the factor $z_i$ must keep the degree of the integrand.}
%
\bea
%
\int d\Pi^{(n+1)} \sum_{i=1}^{n+1}\frac{\d }{\d x_i}\Big\{z_iF^{\lambda_0}x_1^{\lambda_1}\cdots x_n^{\lambda_n}x_{n+1}^{\lambda_{n+1}+1}\Big\}+\delta_{R}&=&0~~~\Label{IBPsyz}
%
%
\eea
%
Expanding \eref{IBPsyz} we got
%
\bea
%
\int d\Pi^{(n+1)} \Big\{\sum_{i=1}^{n+1} \frac{\d z_i}{\d x_i}+\lambda_0 \frac{\sum_{i=1}^{n+1}z_i\frac{\d F}{\d x_i}}{F}+\sum_{i=1}^{n+1}\frac{\lambda_iz_i}{x_i}+\frac{z_{n+1}}{x_{n+1}}\Big\}F^{\lambda_0}x_1^{\lambda_1}\cdots x_n^{\lambda_n}x_{n+1}^{\lambda_{n+1}+1}+\delta_{R}&=&0~~~\Label{syz2}
%
\eea
%
with the combined  boundary term $\delta_R$ \footnote{Actually in the general case, the parameter $\lambda_{n+1}$ will never be zero, so the summary could be written as $\sum_{i=1}^n$ and drop the case when $i=n+1$.}
%
\bea
%
\delta_R&=&\sum_{i=1}^{n+1}\int d\Pi^{(n)}\Big\{z_iF^{\lambda_0}x_1^{\lambda_1}x_2^{\lambda_2}\cdots x_n^{\lambda_n}x_{n+1}^{\lambda_{n+1}+1}\Big\}|_{x_i\to0}
%
\eea
%
%here the $\tilde \lambda_i$  in the delta function represents the actual power of $x_i$ after $x_i^{\lambda_i}$ muitiplied the factor $z_i$.
Since the second term in \eref{syz2} contains a $F$ in denominator, it will shift the spacetime dimension by two (recalling that  $\lambda_0=-\frac{D}{2}$). In practical calculation  we usually need the IBP relation without dimensional shift. 
%In the \eref{syz2} we could find that the second term will cancel a $F$, leading to the dimensional shift from (Here we assume the dimension is set to D, but in general we could deal with any dimension.) $\lambda_0=-\frac{D}{2}$ to $\lambda_0-1=-\frac{D+2}{2}$. In practical calculation  we usually need the IBP relation without dimensional shift, which means the second term should be canceled. 
To avoid this, we  use the trick of "syzygy" in computational algebranic geometry \cite{zhang1,zhang2,Larsen:2015ped,Larsen:2016tdk,Zhang:2016kfo,Jiang:2017phk}, i.e., choosing the functions $z_i$. 
%One just need to solve the polynomial equations
such that\footnote{Since the degree of both side is equal, the solution of $B$ is a polynomial of degree $-1$.}
\bea
%
\sum_{i=1}^{n+1}z_i\frac{\d F}{\d x_i}&=&BF~~~~\Label{2.5}
%
\eea
%
In general, solving the syzygy of the type \eref{2.5} is highly nontrivial. However, 
since the $F(x)$ is a homogeneous function of $x_1$ to $x_{n+1}$ of degree $L+1$, there is a natural and trivial solution of $z_i$ by the Euler equation for homogeneous function,
%
\bea
%
\sum_{i=1}^{n+1} x_i\frac{\d F}{\d x_i}&=&(L+1)F ~~~\Label{eu}
%
\eea
%
To keep the degree, we multiply a $\frac{1}{x_{n+1}}$ in both side of \eref{eu}\footnote{The choice of denominator is free from $x_1$ to $x_{n+1}$, but the most natural choice is just $x_{n+1}$.}, i.e., the trivial solution $\W z_i=\frac{x_i}{x_{n+1}}$ such that 
\bea
%
\sum_{i=1}^{n+1}\W z_i \frac{\d F}{\d x_i}&=&\frac{L+1}{x_{n+1}}F~~~\label{2.7}
%
\eea
%
With above trivial solution, we can split a general solution $z_i$ as 
%$z_i+{B x_{n+1}\over L+1}\W z_i$ in \eref{2.5} and the  syzygy can be split to 
%By the virtue of \eref{eu} we could make a decomposition to $z_i$
%
\bea
z_i\to z_i+{B x_{n+1}\over L+1}\W z_i~~~\Label{decom}
\eea
%
then syzygy can be split to
%
\bea
%
\sum_{i=1}^{n+1} z_i\frac{\d F}{\d x_i}&=&0,~~~~~~\sum_{i=1}^{n+1} \tilde z_i \frac{\d F}{\d x_i}=BF~~~\Label{2z}
%
\eea
%
%One can easily find the choice \eref{decom} could always be solved, and 
Since the $\tilde z_i$ part always give the trivial IBP relation $0=0$,
% and  will not bring any useful IBP reduction relation in the identities. 
to find useful IBP reduction relations we just need to find the solution for the 
first  equation in \eref{2z}, which could be solved by SINGULAR
%In the following calculation, we just drop the part $\tilde z_i$ and keep the $z_i$ for simplicity. The first equation in \eref{2z} is a syzygy equation which could be solved by SINGULAR.  
in the form of polynomials,
\bea
%
\sum_{i=1}^{n+1} g_{ji}\frac{\d F}{\d x_i}&=&0,~~~j=1\cdots s_{k}~~~\Label{syzy}
%
\eea
%
The solution is called the syzygy of the polynomial ring of $F[\frac{\d F}{\d x_1},\cdots,\frac{\d F}{\d x_{n+1}}]$,
where $s_k$ is the number of generators to the syzygy $g_i$. The $g_{ji}$s are all homogeneous function of $x_1$ to $x_{n+1}$.
%
For two-loop integrals, by our experiences there are usually 10 to 100  independent syzygy generators.
Then the $z_i$ could be chosen as
%
\bea
%
z_i&=&\sum_{j=1}^{s_k}c_j{g_{ji}}{x_{n+1}^{-k_j}}~~~~\Label{2.11}
%
\eea
%
where the addtional factor $x_{n+1}^{-k_j}$ is to make the $z_i$ degree zero.
%  the $k_j$ is the degree of polynomial $g_{ji}$, for example, when $g_{ji}=x_1x_2$, then we should choose $k_j=2$. 
The $c_j$ could be degree zero rational function of $x_i$, but for simplicity, one can choose them to be independent of $x_i$ and be the functions of  kinematic variables $s,t,m_i$ only. 
Even with such a simplification, there are still various choices for $c_i$. As we will show, with
%After the choice, solving IBP equations becomes to choosing the parameters $c_j$. By 
some nice choices of $c_j$s, one can get  IBP realtions,  where there is no  terms with increasing power of propagators.
This will reduce the complexity of reduction a lot.

% By this method, we could get the complete choices of $c_j$ and get all independent reduction relations with less steps.
%
%\subsection{Tensor's case}
%
%
When we consider the tensor reduction, there are some differences. First from 
\eref{sumI}, one can see that for the tensor rank $m$, the values of $n_0$ can be from
$\lambda_0-m$ to $\lambda_0$. In the case $n_0\neq \lambda_0$, the dimension has been shifted. To rewrite the integrals in different shifted dimension into the original dimension $\lam_0=-\frac{D}{2}$, we need to get the IBP relation connecting different dimensions in \eref{syz2}. Now the syzygy \eref{syzy} could be modified to
%To do this , we could use the trick by syzygy. For example, in \eref{syz2}, solving the syzygy equation 
%
\bea
%
\sum_{i=1}^{n+1}g_{ji}\frac{\d F}{\d x_i}&=&g_{j0}x_1^{\c_1}x_2^{\c_2}\cdots x_{n}^{\c_n}x_{n+1}^{\c_{n+1}}     ~~~~\Label{tensortrick1}
%
\eea
where the role of $x_i^{\gamma_i}$ is to shift the power of propagators to the wanted number. 
Putting it to \eref{syzy} we will get a recurrence  relation where  there is one term 
%
\bea
%
\lam_0 g_{j0}\int d\Pi^{(n+1)} F^{\lam_0-1}x_1^{\lam_1+\gamma_1}\cdots x_{n}^{\lam_n+\gamma_n}x_{n+1}^{\lam_{n+1}+\gamma_{n+1}}.
%
\eea
%
This term corresponds to dimension $(D+2)$ while other terms, dimension $D$. 
%is exactly  the coefficient $\lam_0 g_{j0}$ times  $i[\lam_0-1;\lam_1+\gamma_1,\cdots,\lam_n+\gamma_n]$ in dimension parameter $\lam_0-1=-\frac{D+2}{2}$ which contributes to dimension $D+2$   and all other terms  in dimension parameter $\lam_0$. 
Now we successfully got a recurrence relation of integrals between dimension $D$ and $D+2$ (or in general, $D+k+2$ to $D+k$). Repeating this procedure, we will finally reduce integrals in dimension parameter $n_0$ in \eref{sumI} from $n_0=\lam_0-m$ to $\lam_0$.
% 
%




%For tensor's case, the tensor structure contributes to the integrals in different dimension, in the form \eref{sum}. To get the reduction results in  the same dimension, we could use the trick of changing  the syzygy equations \eref{syzy}. For example, we could use the equation
%
%\bea
%
%\sum_{j}g_{ji}\frac{\d F}{\d x_i}&=&g_{j0} x_1^{\c_1}x_2^{\c_2}\cdots x_n^{\c_n}x_{n+1}^{\a_{n+1}}~~~~\Label{syzgamma}
%
%\eea
%
%to reduce the integral $i[\lambda_0-1,\c_1+\lam_1,\cdots,\c_n+\lam_n]$ to dimension $\lambda_0=\frac{-D}{2}$ in \eref{syz2}, since the  second term in the bracket of  \eref{syz2} is not zero. Expanding the equation \eref{syz2}, one could get the relation which contains one term in dimension $\lam_0-1$ and all others in dimension $\lam_0$. The parameter $\a_{n+1}=-(n+1)-\sum_{i=1}^n \c_i-(L+1)\lambda_0$. Notice that in the \eref{syzgamma}, there are several choices of the value of $\gamma_i$ and $\lam_i$ to keep $\gamma_i+\lam_i$, and in the practical calculation, one could pick the choice that makes it easier to solve the syzygy equation.    Repeat the procedure, we could finally reduce all terms into the same dimension $D$.
%
%%%%%%%%%%%%%%%%%%%%%%%
\section{Example 1: massive sunset with different mass}
%%%%%%%%%%%%%%%%%%%%%%%%

Having laid out our general strategy, starting from this section, we will present various examples to demonstrate 
the method. 
In this section, we will consider the simplest two-loop example, i.e., the sunset topology. The integrals can be written as
\bea
%
I_3^{(r_1,r_2)}(n_1,n_2,n_3)&\equiv &\int d^{D}\ell_1d^{D}\ell_2{ \ell_1^{\mu_1}... \ell_1^{\mu_{r_1}}\ell_2^{\nu_1}...\ell_2^{\nu_{r_2}} \over (D_1^{n_1})(D_2^{n_2})(D_3^{n_3})}~~~~\Label{sunset-1}
%
\eea
%
with the propagators
%
\bea
%
D_1&=&\ell_1^2-m_1^2,~~D_2=\ell_2^2-m_2^2,~~D_3=(\ell_1+\ell_2-p)^2-m_3^2~~~~\Label{sunset-2}
%
\eea
%
The corresponding functions are given by
\bea
%
A&=&\lba{cc}
x_1+x_3&x_3\nn
x_3&x_2+x_3
\rba~~~~~
%
B=\lba{c}
-x_3 p\nn
-x_3 p
\rba~~~~~
%
C=-m_1^2x_1-m_2^2x_2+(p^2-m_3^2)x_3\nn
%
U(x)&=&Det[\hat A]=x_1x_2+x_1x_3+x_2x_3\nn
%
%
F(x)&=&U(x)x_4+f(x)\nn
%
&=&m_{1}^2 x_{1} (x_{1} (x_{2}+x_{3})+x_{2} x_{3})+m_{2}^2 x_{2} (x_{1} (x_{2}+x_{3})+x_{2} x_{3})\nn&&+m_{3}^2 x_{1} x_{2} x_{3}+m_{3}^2 x_{1} x_{3}^2+m_{3}^2 x_{2} x_{3}^2-p^2 x_{1} x_{2} x_{3}+x_{1} x_{2} x_{4}+x_{1} x_{3} x_{4}+x_{2} x_{3} x_{4}
%
\eea
%
In the following discussion, we will consider the reduction of several situations respectively, where  for simplicity we always choose the following seven scalar  integrals as the basis
%
\bea
%
I_3^{(0,0)}(2,1,1),~I_3^{(0,0)}(1,2,1),~I_3^{(0,0)}(1,1,2),~I_3^{(0,0)}(1,1,1),~I_3^{(0,0)}(1,1,0),~I_3^{(0,0)}(1,0,1),~I_3^{(0,0)}(0,1,1)~~~~~~~~\Label{massun}.
%
\eea

%

%
%%
%
\subsection{Scalar's case}
Let us consider the reduction of scalar integrals first, i.e., $r_1=r_2=0$.
By SINGULAR, we solved the syzygy equations \eref{2.5} in lexicographical (lp) ordering directly.
%
There are ten generators, which are too long to write here, so we will put them  in the appendix A.
%When solving the syzygy equation \eref{2.5} and \eref{syzy}, there is a technical point. In the sunset example, it is simpler to go back to solve the \eref{2.5}.
%Naively the equation \eref{2.5} is for variables $x_i$ while $s,m_1,m_2,m_3$ are parameters. However, it is found that when enlarging the variables to include $s,m_1,m_2,m_3$, i.e., considering the ideal $<x_1,x_2,x_3,x_4,s,m_1,m_2,m_3>$ with the lexicographical ordering to solve the original \eref{2.5}, the output of generators of the syzygy will be simpler.
%It is found that if we go back to solve
%For current massive sunset, there are ten generators and 
Each generator has five components, i.e., four $z_i$'s and the last one, $B$ in \eref{2.5}. One of them is
\bea
g_1&=&\Big\{0, 0, s x_3^2,
m_1^4 x_1^2 + m_2^4 x_2^2 + m_3^4 x_3^2 - m_3^2 s x_3^2 +
2 m_3^2 x_3  + x_4^2 + 2 m_2^2 x_2 (m_3^2 x_3 + x_4) \nn & & +
2 m_1^2 x_1 (m_2^2 x_2 + m_3^2 x_3 + x_4), -m_1^2 x_1 - m_2^2 x_2 -
m_3^2 x_3 - s x_3 - x_4\Big\}\label{g1}
%
\eea
%The other nine generators are put in the appendix.
Since the first nine are degree two and the last one, degree one, we write 
 $z_i=\frac{\sum_{j=1}^{9}c_jg_{ji}}{x_4^{2}}+\frac{c_{10}g_{10,i}}{x_4}$  as in the \eref{2.11}, and the obtained IBP relations without dimensional shift are
%
%
\bea
%
&&\Big\{\sum_{i=1}^{3}c_{i^{++}}i^{++}+\sum_{i,j=1;i\neq j}^{3}c_{i^{+}j^{+}}i^{+}j^{+}+\sum_{i=1}^{3}c_{i^{+}}i^{+}+\sum_{i,j=1;i\neq j}^{3}c_{i^{++}j^{-}}i^{++}j^{-}\nn
%
&&+\sum_{i,j=1;i\neq j}^{3}c_{i^{+}j^{-}}i^{+}j^{-}+\sum_{i,j,k=1;i\neq j\neq k}^{3}c_{i^{+}j^{+}k^{-}}i^{+}j^{+}k^{-}+c_{0,0,0}\Big\}i[\lambda_0;n_1,n_2,n_3]
+\delta_{bound}=0~~~\Label{3.6}
%
\eea
%
We can rewrite \eref{3.6} as 
%
\bea \sum C^{i_1,i_2, i_3}_{n_1,n_2,n_3}   i[n_1+i_1,n_2+i_2,n_3+i_3] +\delta_{bound}=0~~~~~~~~\Label{Ci1}
\eea
%
where the coefficients $C$'s are defined by, for example,
%
\bea c_{1^{++}} 1^{++} i[\lambda_0;n_1,n_2,n_3]=c_{1^{++}} i[\lambda_0;n_1+2,n_2,n_3]
\equiv C^{2,0,0}_{(n_1,n_2,n_3)}i[\lambda_0;n_1+2,n_2,n_3]\eea
%
Using the explicit expression of $g_i$ we can find\footnote{One can see that the $g_{10}$ is the trivial solution \eref{eu}
	and the corresponding coefficient $c_{10}$ will not appear in \eref{Ci1}. Thus it supports our claim that trivial
	solution will not contribute to nontrivial IBP relation.}
	  %
\bea
%
C^{2,0,0}_{n_1,n_2,n_3}&=&\frac{1}{2}m_1^2 (m_1^4 (c_2 - c_6) + s (-5 m_2^2 + 3 m_3^2 + s) c_6 \nn&&+
m_1^2 (c_1 - s c_2 + 5 m_2^2 c_6 - 3 m_3^2 c_6)) (3 D -
2 (5 +n_1 + n_2 + n_3))
%
\eea
here $c_1$ to $c_9$ are free parameters.
Since the coefficients are too long to write here, we put them in the appendix.

Now we discuss  the choice of free parameters $c_i$. From the expression of \eref{3.6}, we see that the allowed values  of $\{i_1,i_2, i_3\}$
in \eref{Ci1}
are given by the union of following six types of values
%For Simplicity of the following discussion, let us rewrite \eref{Ci1} as
%\footnote{The former coefficients $C_{i_1,i_2,i_3}$ will be concerned with the power of the propagators}
%\bea
%
%\sum_{\{i_1,i_2,i_3\}\in c_1\cup \cdots\cup A_6} C_{i_1,i_2,i_3} ~i[\lambda_0,n_1+i_1,n_2+i_2,n_3+i_3]+\delta_{bound}&=&0~~~\Label{IBPRSUN}
%
%\eea
%
%with the $C_{i_1,i_2,i_3}$ the analytic coefficients of the terms and
\bea
%
&&A_1=\cup_{Permutation}\{2,0,0\},~~A_2=\cup_{Permutation}\{1,1,0\},~~A_3=\cup_{Permutation}\{1,0,0\}\cup\{0,0,0\},\nn
%
&&A_4=\cup_{Permutation}\{2,-1,0\},~~A_5=\cup _{Permutation}\{1,-1,0\},~~A_6=\cup_{Permutation}\{1,1,-1\}
%
\eea
%The $\delta_{bound}$ in \eref{Ci1} contrbutes to the boundary terms.
%
%Notice that the equ\eref{Ci1} only contains the terms in the same dimension $D$.
Among these six groups, we see that $A_1$ and $A_2$ shift the total power of propagators by two.
For the $A_4$ term, although the shift of total power is just one, but the power of one propagator has been shifted by two.
We could choose $x_i$ to simplify the IBP relation \eref{Ci1} by removing as many as possible groups with
shifting total power.  
For example, there is a solution of $c_i$'s such that 
%keeping the coeffcients of the term $i[\lambda_0;n_1+1,n_2+1,n_3]$, while setting coefficients of other terms  with the total shift power two, i.e. solving the equation systems \footnote{Here we want to the IBP relation be as simple as possible, and it is the best choice. If one wants the coefficients of $c_{i_1,i_2,i_3}$ wiht $\{i_1,i_2,i_3\}$ in $A_6$ be zero as well, he will get the trivial IBP recurrence relation, i.e. $0=0$.}
\bea
%
\forall \{i_1,i_2,i_3\}\in A_1\cup A_4\cup A_5\cup\{\{1,0,1\},\{0,1,1\}\} ,~~ C_{n_1,n_2,n_3}^{i_1,i_2,i_3}&=&0~~~\Label{par110}
%
\eea
Taking the solution  in \eref{par110}, we got the simplified IBP relation
%
\bea
%
c_{1,1,0}~i[\lambda_0,n_1+1,n_2+1,n_3]+\sum_{\{i_1,i_2,i_3\}\in A_3 \cup A_6}c_{\{i_1,i_2,i_3\}} i[\lambda_0,n_1+i_1,n_2+i_2,n_3+i_3]+\delta_{1,1,0}&=&0~~~~~~\Label{ibpaft}
%
%
\eea
%
Notice that in the relation \eref{ibpaft}, only the first term shifts  total power by two, and all others  shift the total power at most by one. By this relation, we could  reduce $i[\lambda_0,i_1+1,i_2+1,i_3]$ to simpler integrals. Iterating the
procedure we will get the complete reduction result.  Similarly, by  different choices of the free parameters $c_1$ to $c_9$, we could get
%
\bea
%
c_{2,0,0}i[\lambda_0,n_1+2,n_2,n_3]+\sum_{\{i_1,i_2,i_3\}\in A_3 \cup A_6}c_{i_1,i_2,i_3} i[\lambda_0,n_1+i_1,n_2+i_2,n_3+i_3]+\delta_{2,0,0}&=&0\label{3.13-a}\\
%
c_{0,2,0}i[\lambda_0,n_1,n_2+2,n_3]+\sum_{\{i_1,i_2,i_3\}\in A_3  \cup A_6}c_{i_1,i_2,i_3} i[\lambda_0,n_1+i_1,n_2+i_2,n_3+i_3]+\delta_{0,2,0}&=&0\label{3.13-b}\\
%
c_{0,0,2}i[\lambda_0,n_1,n_2,n_3+2]+\sum_{\{i_1,i_2,i_3\}\in A_3  \cup A_6}c_{i_1,i_2,i_3} i[\lambda_0,n_1+i_1,n_2+i_2,n_3+i_3]+\delta_{0,0,2}&=&0\label{3.13-c}\\
%
c_{1,0,1}i[\lambda_0,n_1+1,n_2,n_3+1]+\sum_{\{i_1,i_2,i_3\}\in A_3  \cup A_6}c_{i_1,i_2,i_3} i[\lambda_0,n_1+i_1,n_2+i_2,n_3+i_3]+\delta_{1,0,1}&=&0\label{3.13-d}\\
%
c_{0,1,1}i[\lambda_0,n_1,n_2+1,n_3+1]+\sum_{\{i_1,i_2,i_3\}\in A_3  \cup A_6}c_{i_1,i_2,i_3} i[\lambda_0,n_1+i_1,n_2+i_2,n_3+i_3]+\delta_{0,1,1}&=&0~~~\Label{ibpaft2}
%%
\eea
%
%Combining \eref{ibpaft2} and \eref{ibpaft}
Using above  six independent IBP relations, one can quickly  reduce the power of propagators to the basis \eref{massun}\footnote{Notice that the boundary term  is different from each other. Since we chose different parameters $c_1$ to $c_9$ in each relation, the $z_1$ to $z_8$ are also different.}. The boundary terms only contribute to the sub-topology, which is the diagram removed by a inner line. We gave the analytic coefficients in the attached files. 
%In our choice of the master integrals in \eref{massun}, we do not need the recurrence relation to reduce $i[n_1+1,n_2,n_3]$, $i[n_1,n_2+1,n_3]$ and  $i[n_1,n_2,n_3+1]$ to $i[n_1,n_2,n_3]$.
%
%
%
%
\subsubsection{The example: $I_3^{(0,0)}(3,1,1)$}
%
To get the results of $I_3^{(0,0)}(3,1,1)$ we need to reduce $i[\lambda_0;2,0,0]$. We could set $n_1=0$, $n_2=0$ and $n_3=0$ in the \eref{3.13-a}, and we will get a relation \footnote{When $n_1=n_2=n_3=0$,  the coefficients $c_{i_1,i_2,i_3}$ with $\{i_1,i_2,i_3\}$ in $A_4$ and $A_5$ are zero automatically.}
%
\bea
%
i[\lambda_0;2,0,0]&=&c_{\{0,0,0\}}i[\lambda_0;0,0,0]+c_{\{1,0,0\}}i[\lambda_0;1,0,0]+c_{\{0,1,0\}}i[\lambda_0;0,1,0]+c_{\{0,0,1\}}i[\lambda_0;0,0,1]+\delta_{boundterm}~~~~~~~~~~~\label{3.18}
%
\eea
%
which is the wanted reduction result already. To translate this result to the familiar master integrals, we need to use the correspondence between 
 $i[\lambda_0,n_1,n_2,n_3]$ and  $I_3(n_1+1,n_2+1,n_3+1)$ as given \eref{defI}, thus we get
%
\bea
%
I_3^{(0,0)}(3,1,1)&=&c_{311\to211}I_3^{(0,0)}(2,1,1)+c_{311\to 121}I_3^{(0,0)}(1,2,1)+c_{311\to112}I_3^{(0,0)}(1,1,2)\nn & & +c_{311\to111}I_3^{(0,0)}(1,1,1)+\cdots~~~~~
%
\eea
%
with the coefficients given in the Appendix A. The boundary part is just the product of two one-loop tadpoles and the corresponding 
reductions have been discussed in [cite previous work], thus we can write down the expressions directly.  
%
%Here we ellip the boundary terms.
 The coefficients  $c_{311\to211}$, $c_{311\to121}$, $c_{311\to112}$, $c_{311\to111}$ are confirmed with the FIRE6.
 One nice point of our method is that to give the reduction results of the top-sector, we only need to use the simplified IBP relation \eref{ibpaft2} once.
 
%  For the boundary term, we could repeat the procedure. The function F for boundary term is much simpler.  For example, when we deal with the $i[\lambda_0,-1,n_2,n_3]$ topology, we just need to remove the Feynman parameter $x_1$ from the function $F(x)$, and then follow the procedure. In this example, to give the reduction results of the top-sector, we only need to use the simplified IBP relation \eref{ibpaft2} {\bf once}.
%
\subsubsection{The example: $I_3^{0,0}(2,2,1)$}
%
In this case, we could set $n_1=n_2=n_3=0$ in \eref{ibpaft}, and we have the solution
%
\bea
%
%
i[\lambda_0,1,1,0]&=&c_{\{1,0,0\}}i[\lambda_0,1,0,0]+c_{\{0,1,0\}}i[\lambda_0,0,1,0]+c_{\{0,0,1\}}i[\lambda_0,0,0,1]+c_{\{0,0,0\}}i[\lambda_0,0,0,0]+\cdots ~~~~~
%
\eea
%
Compared to the definition of $I_3$ in \eref{defI}, we have
%
\bea
%
I_3(2,2,1)&=&c_{221\to211}I_3(2,1,1)+c_{221\to121}I_3(1,2,1)+c_{221\to112}I_3(1,1,2)+c_{221\to111}I_3(1,1,1)+\cdots `~~~~
%
\eea
%
%
Again, to successed reduce the top-sector to the master integrals, we need to use \eref{ibpaft} only once.
The coefficients which given in the appendix are confirmed with FIRE6.
%
%
%
\subsubsection{The scalar's reduction with general power}
%
For the general powers, we could use the recurrence relation \eref{ibpaft} to \eref{ibpaft2} to lower the total power until we reduce the top-sector  to the master integrals and get the analytic coefficients. 
For the boundary part, again we can use discussions in previous work to solve them [cite one-loop work]. 
%As for the boundary terms which contribute to the sub-topology, its parametric form is much more simple. We just need to remove the parameter which contributes to the removed propagator and write down the function $F$. 
%Repeating the procedure we discuss above, we will finally get all the reduction coefficients to the master integrals. 
Depending on different situations, we should choose different recurrence relations to simplify reduction procedure: 
%There are two types of recurrence relation  for lowing the total power:
%
\begin{itemize}
\item When One propagator has the   larger   power   than the others, for example $n_1>n_2, n_3$, we should use \eref{3.13-a}
to lower $n_1$ to $(n_1-2)$ (similarly using \eref{3.13-b} or \eref{3.13-c} to lower $n_2, n_3$).  

%{\bf One propagator has the   larger   power   than the others:}
%
%In this case, for example, when $n_1$ is larger than $n_2$ and $n_3$ , we could use  one of \eref{3.13-a}, \eref{3.13-a} to lower the power $n_1$ to $n_1-2$. The result of this step will give terms whose power of $D_2$ becomes to $n_2+1$ or power of $D_3$ becomes to $n_3+1$. For example, if we want to reduce $i[3,1,1]$, we need to use \eref{3.13-a}. By setting $n_1=n_2=n_3=1$, the recurrence \eref{3.13-a} reduces  $i[3,1,1]$ to $i[2,1,1]$, $i[1,2,1]$, $i[1,1,2]$ and other simpler integrals. 
%
\item When at least two propagator have the   larger   power, for example $n_1=n_2\geq n_3$, we should use 
\eref{ibpaft} to lower $n_1, n_2$ to $(n_1-1, n_2-1)$ (and similarly using \eref{3.13-d} and \eref{ibpaft2})

%{\bf Two propagators in the same power larger than the left one}\\
%
%In this case, we could use \eref{ibpaft}, \eref{3.13-d} and \eref{ibpaft2} to lower total power. For example, when we want to reduce $i[3,3,2]$, we could use \eref{ibpaft} to reduce $i[3,3,2]$ to $i[3,2,2]$, $i[2,3,2]$, $i[2,2,3]$ and other terms with lower total power.
%
\end{itemize}
%
According to above procedure, iteratively using \eref{ibpaft} to \eref{ibpaft2}, we will eventually get all the reduction coefficients to the master integrals. 

\subsection{Tensor's case}
%
%Now we consider the example of the massive sunset
%
Now we consider the  massive sunset with nontrivial tensor structures
%
\bea
%
I_3^{(r_1,r_2)}(n_1,n_2,n_3)&=&\int d^{D}l_1d^Dl_2\frac{l_1^{\mu_1}\cdots l_1^{\mu_{r_1}}l_2^{\nu_1}\cdots l_2^{\nu_{r_2}}}{D_1^{n_1+1}D_2^{n_2+1}D_3^{n_3+1}}
%
\eea
%
The function $F(q,y)$   is given by
%
%
\bea
%
F(q,y)&=&-x_1x_2x_3p^2+(p\cdot q_2x_1+p\cdot q_1x_2)x_3y+m_{1}^2 x_{1} (x_{1} (x_{2}+x_{3})\nn
%
&&+x_{2} x_{3})+\frac{1}{4} \Big(4 m_{2}^2 x_{2} (x_{1} (x_{2}+x_{3})+x_{2} x_{3})+4 m_{3}^2 x_{3} (x_{1} (x_{2}+x_{3})+x_{2} x_{3})\nn
%
&&+q_{1}^2 x_{2} y^2+q_{1}^2 x_{3} y^2-2 q_{1} q_{2} x_{3} y^2+q_{2}^2 x_{1} y^2+q_{2}^2 x_{3} y^2+4 x_{1} x_{2} x_{4}+4 x_{1} x_{3} x_{4}+4 x_{2} x_{3} x_{4}\Big)
%
\eea
%
For simplicity, we choose the scalar basis same as \eref{massun}.
%
\subsubsection{The example: $I_3^{(1,0)}(1,1,1)$}
%
%Let us consider the example $I_3^{(1,0)}(1,1,1)=\int d^D l_1d^Dl_2\frac{l_1^{\mu_1}}{D_1D_2D_3}$.
%%
For this case, using \eref{sumI}  we find %we need to calculate $i[\lambda_0,0,0,0]^{\mu_1}$,
%
\bea
%
i[\lambda_0,0,0,0]^{\mu_1}&=&%\frac{-\lambda_4}{\Gamma(2)}\frac{\d }{\d q_{1,\mu_1}}\int d\Pi^{(4)} ~dy~F(q,y)^{\lambda_0}x_4^{(\lambda_4-1)}|_{q_i\rightarrow 0}\nn
%
%
%&=&\frac{-\lambda_0p^{\mu_1}}{\Gamma(2)(\lambda_4+1)}\int d\Pi^{(4)}~F^{\lambda_0-1}x_2x_3x_4^{\lambda_4+1}
\frac{-\lambda_0 p^{\mu_1}}{-3+\frac{3D}{2}}i[\lambda_0-1;0,1,1]~~~\Label{322}
%
\eea
%
where the $i[\lambda_0-1,0,1,1]$ contributes to the sunset in dimension $D+2$.
%
To get the result of reduction in the dimension $D$, we need to reduce the $i[\lambda_0-1,0,1,1]$ to the $i[\lambda_0,\cdots]$. Different from the proposal given in [cite chen], we use the method discussed in \eref{tensortrick1},
i.e., imposing 
%In Chen's original paper, he proposed a method based on the noncommutative algebra. Here we give a different method.
%
%In the \eref{syz2} we could see that the dimensional shifted terms come from the second term. To get the $i[\lambda_0-1,0,1,1]$ term, we need to  solve the equation \footnote{In the right hand side, the choice is not unique. For example, we could set the right hand side is $x_2x_3x_4$ or $x_2x_3x_4^2$, since the power of $x_4$ will not contribute to the power of the propagators. But for any choice, the polynomial $g_0$ should not contain $x_1$ to $x_3$, for the reason the power of $x_1$ in $i[\lambda_0-1,0,1,1]$ is zero. A clever choice is  $\alpha=0$ in this case, because the $\frac{\d F}{\d x_i}$ is a polynomial with power $2$. This choice lead to the result that the $g_0$ and $g_{ji}$ have the same degree of $x_i$}
\bea
%
\sum_{i=1}^{n+1}g_{i}\frac{\d F}{\d x_i}&=&g_{0}x_2x_3x_4^{\alpha}~~~\Label{234a}
%
\eea
%
One point worth to explain is that to produce $i[\lambda_0-1;0,1,1]$ at the right hand side of \eref{322}, we must 
require the right hand side of \eref{234a} to be the form $g x_2 x_3$ where $g$ is only the function of $x_4$. 
By SINGULAR, we could find the solution with the choice of $\a=1$ and $g_0$ independent of $x_i$. There are 
 totally ten generators of the syzygy in lexicographical (lp) ordering and  one of them is
%
%
\bea
%
g_{1}&=&(4s-2m_1^2-2m_2^2-2m_3^2)x_1+(-3m_2^2)x_2-3m_3^2x_3,~~~~
%
g_2=(-2s-2m_1^2+m_2^2+4m_3^2)x_2+3m_3^2x_3~~~~~~~~~~~~\nn
%
g_3&=&3m_2^2x_2+(-2s-2m_1^2+4m_2^2+m_3^2)x_3\nn
%
g_4&=&(-6sm_1^2+6m_1^4)x_1+(3sm_2^2+9m_1^2m_2^2-3m_2^4-9m_2^2m_3^2)x_2\nn&&+(3sm_3^2+9m_1^2m_3^2-9m_2^2m_3^2-3m_3^4)x_3+(-2s+4m_1^2-2m_2^2-2m_3^2)x_4,~~~~g_0=-6s~~~\Label{tensorsunsetge}
%
\eea
%
 Since each of the generators from $g_1$ to $g_4$ has degree one, we could set $z_j=\frac{g_j}{x_4}, j=1,\cdots,4$ and $z_0={g_0\over x_4}$. Taking the $z_j$ into \eref{syz2} with $\la_1=\la_2=\la_3=0$, we got the recurrence relations between dimension $D+2$ to dimension $D$ as 
%
%\bea
%
%&&\int d\Pi^{(4)} \Big\{\sum_{i=1}^4\frac{\d z_i}{\d x_i}+\lambda_0\frac{\sum_{i=1}^4z_i\frac{\d F}{\d x_i}}{F}+\sum_{i=1}^3\frac{n_i z_i}{x_i}+(\lambda_4+1)\frac{z_4}{x_4}\Big\}F^{\lambda_0}x_1^{n_1}\cdots x_4^{\lambda_4+1}+boundaryterm =0\nn
%
%&& \int d\Pi^{(4)}\Big\{\sum_{i=1}^4\frac{\d z_i}{\d x_i}+\lambda_0\frac{z_0x_2x_3x_4}{F}\Big\}F^{\lambda_0}x_4^{-3+\frac{3D}{2}}+boundaryterm =0~~~\Label{dimensionshiftten}
%
%\eea
%
%The second term in the bracket leads to the integral $i[\lambda_0-1;0,1,1]$. Expanding the \eref{dimensionshiftten}, we get the recurrence relation
%
%
\bea
%
&&3 \Big(-2 (D-3) m_{1}^2+(D-3) m_{2}^2+D m_{3}^2+D s-3 m_{3}^2-2 s\Big)i[\lambda_0,0,0,0]\nn&&+\frac{3}{2} (3 D-8) m_{3}^2 \Big(-3 m_{1}^2+3 m_{2}^2+m_{3}^2-s\Big) i[\lambda_0,0,0,1]+\frac{3}{2} (3 D-8) m_{2}^2 \Big(-3 m_{1}^2+m_{2}^2+3 m_{3}^2-s\Big)i[\lambda_0,0,1,0]\nn&&-3 (3 D-8) m_{1}^2 \Big(m_{1}^2-s\Big)i[\lambda_0,1,0,0]-3Dsi[\lambda_0-1,0,1,1]+{\rm bound}=0
%
\eea
%
with the boundary term
%
\bea
%
{\rm bound}&=&3 \Big(m_{2}^2 (i[\lambda_0,-1,1,0]-i[\lambda_0,0,1,-1]+m_{3}^2 (i[\lambda_0,-1,0,1]-i[\lambda_0,0,-1,1])\Big)
%
\eea
%
Solving the $i[\lambda_0-1;0,1,1]$, we got
\bea
%
&&i[\lambda_0-1,0,1,1]\nn
%
&=&\frac{1}{D s}(-2 (D-3) m_{1}^2+(D-3) m_{2}^2+D m_{3}^2+D s-3 m_{3}^2-2 s) i[\lambda_0,0,0,0]\nn
%
&&+\frac{(3 D-8) m_{3}^2 \Big(-3 m_{1}^2+3 m_{2}^2+m_{3}^2-s\Big)}{2 D s} i[\lambda_0,0,0,1]
%
+\frac{(3 D-8) m_{2}^2 \Big(-3 m_{1}^2+m_{2}^2+3 m_{3}^2-s\Big)}{2 D s} i[\lambda_0,0,1,0]\nn
%
&&+-\frac{(3 D-8) m_{1}^2 \Big(m_{1}^2-s\Big)}{D s} i[\lambda_0,1,0,0]+\frac{m_{3}^2}{D s} i[\lambda_0,-1,0,1]\nn&&-\frac{m_{3}^2}{D s}i[\lambda_0,0,-1,1]
+\frac{m_{2}^2}{D s}i[\lambda_0,-1,1,0]-\frac{m_{2}^2}{D s} i[\lambda_0,0,1,-1]~~~\Label{436}
%
\eea
%
%
Here the $i[\lambda_0,-1,0,1]$, $i[\lambda_0,0,1,-1]$ and $i[\lambda_0,0,-1,1]]$  come from the boundary terms. Notice that the right hand of the \eref{436} all contribute to the basis we chose, which means we got the results of reduction by solving syzygy equation once. Combining \eref{436} and \eref{322} with \eref{defI}, we got the results of reduction
%
\bea
%
I_3^{(1,0)}(1,1,1)&=&c_{10\to211}I_3^{(0,0)}(2,1,1)+c_{10\to121}I_3^{(1,0)}(1,2,1)+c_{10\to112}I_3^{(1,0)}(1,1,2)\nn&&+c_{10\to111}I_3^{(1,0)}(1,1,1)+\cdots ~~~~~~~~~~
%
\eea
%
with the coefficients (without the boundary terms)
%
\bea
%
c_{10\to211}&=&\frac{(-8+3D)m_1^2(s-m_1^2)p^{\mu_1}}{3(D-2)s}\nn
%
c_{10\to121}&=&\frac{(-8+3D)m_2^2(-3m_1^2+m_2^2+3m_3^2-s)p^{\mu_1}}{6(D-2)s}\nn
%
c_{10\to112}&=&\frac{(-8+3D)m_3^2(-3m_1^2+3m_2^2+m_3^2-s)p^{\mu_1}}{6(D-2)s}\nn
%
c_{10\to111}&=&\frac{(3D-8)\Big(-2(D-3)m_1^2+(D-3)m_2^2+(D-3)m_3^2+(D-2)s\Big)}{6(D-2)s}p^{\mu_1}
%
\eea
%
The coefficients are confirmed with FIRE6 \cite{Smirnov:2019qkx}. Here we just got the reduction result of the top-sector. To get the complete results, we need to reduce the boundary terms by repeating the similar procedure above.
%
\subsubsection{The example: $I_3^{(1,1)}(1,1,1)$}
%
Now let us consider a more complex case, i.e., tensor rank $(1,1)$. By directly calculation, we got
%
%
\bea
%
i[\lambda_0,0,0,0]^{\mu_1\nu_1}&=&%\frac{\lambda_4}{\Gamma(3)}\frac{\d^2 }{\d q_{1,\mu_1}\d q_{1,\nu_1}}\int d\Pi^{(4)}dy~F(q,y)^{\lambda_0}x_1^0x_2^0x_3^0x_4^{\lambda_4-1}|_{q_i\rightarrow 0}\nn
%
\frac{\lambda_0}{\Gamma(3)}\frac{g^{\mu_1\nu_1}}{(\lambda_4+1)(\lambda_4+2)}\Big\{i[\lambda_0-1;0,1,0]+i[\lambda_0-1;0,0,1]\Big\}\nn
&&+\frac{\lambda_0(\lambda_0-1)}{\Gamma(3)}\frac{2p^{\mu_1}p^{\nu_1}}{(\lambda_4+1)(\lambda_4+2)}i[\lambda_0-2;0,2,2]~~\Label{iuv}
\eea
%
The two terms in the first line of \eref{iuv} contribute to the integrals in dimension $D+2$, the last term in the second line contributes to the integral in dimension $D+4$. To reduce $i[\lambda_0-2;0,2,2]$, we could solve
%
$
%
\sum_{j=1}^4 g_{j}\frac{\d F}{\d x_j}=g_0 x_2^2x_3^2x_4^{\a}
%
$. 
%
Choosing $\a=0$, we got one of the generators
%
\bea
%
g_1&=&2 x_{3} (m_3^2 x_{3}+x_{4})
(m_1^2 (2 m_2^2-3 s)+m_1^4-2
(m_2^2-s)
(m_2^2-m_3^2+s))\nn
%
g_2&=&2 m_1^2 (-m_2^2 (m_3^2 (2
x_{3} (x_{3}-x_{1})+3 x_{2}^2+2 x_{2}
x_{3})+s (2 x_{1} (x_{2}+x_{3})-x_{2}
(9 x_{2}+2 x_{3}))+2 x_{4} (2
x_{2}+x_{3}))\nn&&+m_2^4 (x_{1}+x_{2})
(x_{2}-x_{3})+s (m_3^2 (-2 x_{1}
x_{3}+3 x_{2}^2+4 x_{2} x_{3}+3
x_{3}^2)-s (x_{1} (3
x_{2}+x_{3})\nn&&+x_{2} (12 x_{2}+7 x_{3}))+3
x_{4} (2 x_{2}+x_{3})))+m_1^4
(m_2^2 (-x_{1} x_{2}+3 x_{1}
x_{3}-8 x_{2}^2+3 x_{2} x_{3})-2 (2
x_{2}+x_{3}) (m_3^2
x_{3}+x_{4})\nn&&+s (15 x_{1} x_{2}+9
x_{1} x_{3}+15 x_{2}^2+13 x_{2}
x_{3}))+m_1^6 (-(6 x_{1}
x_{2}+4 x_{1} x_{3}+3 x_{2}^2+4 x_{2}
x_{3}))\nn&&+4 (m_2^2-s)
(m_2^2-m_3^2+s) (m_2^2
x_{2}^2+(2 x_{2}+x_{3}) (m_3^2
x_{3}+x_{4})-s x_{2} (3 x_{2}+2
x_{3}))\nn
%
%
%
%
g_3&=&m_1^4 (m_2^2 (x_{1} (x_{2}-3
x_{3})+x_{3} (x_{2}+4 x_{3}))-2 x_{3}
(m_3^2 x_{3}+x_{4})-s (15
x_{1} x_{2}+9 x_{1} x_{3}+15 x_{2}
x_{3}+x_{3}^2))\nn&&+2 m_1^2
(m_2^2 (m_3^2 (-x_{3}) (2
x_{1}+x_{3})+2 s x_{1} (x_{2}+x_{3})+s
x_{3} (2 x_{2}-x_{3})-2 x_{3}
x_{4})\nn&&+m_2^4 (x_{1}+x_{3})
(-(x_{2}-x_{3}))+s (2 m_3^2 x_{3}
(x_{1}+x_{3})+s x_{1} (3 x_{2}+x_{3})+s
x_{3} (3 x_{2}-2 x_{3})+3 x_{3}
x_{4}))\nn&&+m_1^6 (6 x_{1}
x_{2}+4 x_{1} x_{3}+6 x_{2}
x_{3}+x_{3}^2)-4 x_{3}
(m_2^2-s)
(m_2^2-m_3^2+s) (m_2^2
x_{3}-m_3^2 x_{3}+s x_{3}-x_{4})\nonumber
\eea
\bea
%
g_4&=&-2 m_1^2 (-m_2^2 (m_3^2
(x_{4} (4 x_{1}+3 x_{2}+5 x_{3})-s
(10 x_{1} x_{2}-2 x_{1} x_{3}+6
x_{2}^2+23 x_{2} x_{3}+x_{3}^2))\nn&&+3
m_3^4 x_{3} (2 x_{1}+x_{2})+s^2 (7
x_{1} x_{2}-3 x_{1} x_{3}+21 x_{2}^2+13
x_{2} x_{3})-s x_{4} (23 x_{2}\nn&&+2
x_{3})+4 x_{4}^2)-m_2^4 (m_3^2
(3 x_{1} (x_{2}-3 x_{3})+6 x_{2}^2+4
x_{2} x_{3}+3 x_{3}^2)+s (7 x_{1}
x_{2}+3 x_{1} x_{3}-20 x_{2}^2\nn&&+6 x_{2}
x_{3})+4 x_{4} (-x_{1}+2
x_{2}+x_{3}))+m_2^6 (3 x_{1}+x_{2})
(x_{2}-x_{3})+s (m_3^2 (x_{4} (4
x_{1}+3 x_{2}+10 x_{3})-3 s (x_{1}
(x_{2}+x_{3})\nn&&+x_{3} (3 x_{2}+2 x_{3})))+3
m_3^4 x_{3} (2 x_{1}+x_{2}+x_{3})+3 s^2
(x_{1} (x_{2}+x_{3})+4 x_{2} x_{3})-s
x_{4} (4 x_{1}+9 x_{2}+4 x_{3})+6
x_{4}^2))\nn&&+m_1^4 (m_2^2
(m_3^2 (5 x_{1} (x_{2}+x_{3})+x_{3}
(14 x_{2}-13 x_{3}))-s x_{2} (36 x_{1}+15
x_{2}+25 x_{3})+x_{4} (8 x_{1}+15 x_{2}-9
x_{3}))\nn&&+m_2^4 (x_{1} (x_{2}-3
x_{3})+5 x_{2} (3 x_{2}-2 x_{3}))+m_3^2
(3 s (x_{1} (3 x_{2}+x_{3})+x_{3} (5
x_{2}-x_{3}))\nn&&+10 x_{3} x_{4})+6 m_3^4
x_{3}^2+3 s^2 x_{1} x_{2}+15 s^2 x_{1}
x_{3}+15 s^2 x_{2} x_{3}-12 s x_{1} x_{4}-4
s x_{3} x_{4}+4 x_{4}^2)\nn&&+m_1^6
(m_2^2 (13 x_{1} x_{2}-3 x_{1}
x_{3}+7 x_{2} x_{3})-3 m_3^2 x_{2} (2
x_{1}+3 x_{3})+6 s x_{1} x_{2}-12 s x_{1}
x_{3}-3 s x_{2} x_{3}\nn&&+4 x_{1} x_{4}-3
x_{2} x_{4}+x_{3} x_{4})+3 m_1^8
x_{1} (x_{3}-x_{2})-4 (m_2^2-s)
(m_2^2-m_3^2+s) (-m_2^2
(m_3^2 x_{3} (x_{3}\nn&&-5 x_{2})+2 s
x_{2} (3 x_{2}+x_{3})-5 x_{2}
x_{4})+m_2^4 x_{2} (2
x_{2}-x_{3})+(m_3^2 x_{3}-s
x_{3}+x_{4}) (3 m_3^2 x_{3}-3 s
x_{2}+2 x_{4}))\nonumber
\eea
\bea
%
%
g_0&=&3 s (m_1^2-2 (m_2^2+s)) (m_1^2 (2 m_2^2-3 s)+m_1^4-2 (m_2^2-s) (m_2^2-m_3^2+s))~~~~\Label{syzoftensor11}
%
\eea
%
Taking the \eref{syzoftensor11} into \eref{syz2}, we reduce the $i[\lambda_0-2;0,2,2]$ to integral in dimension $\lambda_0-1=-\frac{D+2}{2}$. Notice the $\lambda_0$ in \eref{syz2} should be replaced by $\lambda_0-1$. We have the result
%
%
\bea
%
&&i[\lambda_0-2;0,2,2]\nn
&=&c_{i;022\to000}i[\lambda_0-1;0,0,0]+c_{i;022\to100}i[\lambda_0-1;1,0,0]+c_{i;022\to010}i[\lambda_0-1;0,1,0]+c_{i;022\to001}i[\lambda_0-1;0,0,1]\nn
%
&&+c_{i;022\to110}i[\lambda_0-1;1,1,0]+c_{i;022\to101}i[\lambda_0-1;1,0,1]+c_{i;022\to011}i[\lambda_0-1;0,1,1]+c_{i;022\to200}i[\lambda_0-1;2,0,0]\nn
%
&&+c_{i;022\to020}i[\lambda_0-1;0,2,0]+c_{i;022\to002}i[\lambda_0-1;0,0,2]+\cdots~~~~~~~~\Label{334}
%
%
\eea
%
with the coefficients given in appendix.
%
%\bea
%
%c_{i;022\to000}&=&\frac{4 (3 D-1)}{3 (D+2) s (2 (m_{2}^2+s)-m_{1}^2)}\nn
%
%c_{i;022\to100}&=&\frac{4 (1-3 D) m_{1}^2}{3 (D+2) s (m_{1}^2-2 (m_{2}^2+s))}\nn
%
%c_{i;022\to010}&=&\frac{1}{3 (D+2) s (2 (m_{2}^2+s)-m_{1}^2) (m_{1}^2 (2 m_{2}^2-3 s)+m_{1}^4-2
%	(m_{2}^2-s) (m_{2}^2-m_{3}^2+s))}\times %\Big\{D (6 m_{1}^2 (m_{2}^2 (3 m_{3}^2\nn&&-23 s)+8 m_{2}^4-3 s (m_{3}^2-3 s))+45 m_{1}^4 m_{2}^2-9 m_{1}^6-12 (-8 m_{2}^2 s+5
%	m_{2}^4+3 s^2) (m_{2}^2-m_{3}^2+s))\nn&&-2 %(m_{2}^2-s) (3 m_{1}^2+2 m_{2}^2-6 s) (5 m_{1}^2-4
	%(m_{2}^2-m_{3}^2+s))\Big\}\nn
%
%c_{i;022\to001}&=&\frac{1}{3 (D+2) s
%	(2 (m_{2}^2+s)-m_{1}^2) (m_{1}^2 (2 %m_{2}^2-3 s)+m_{1}^4-2 (m_{2}^2-s) (m_{2}^2-m_{3}^2+s))}\times\Big\{m_{1}^4 ((22-27 D) m_{2}^2\nn&&+2 (15 D-8) m_{3}^2+2 (11-6 D) s)+2 m_{1}^2 (m_{2}^2 ((15 D-8) m_{3}^2-6 D s)+2 (6 D+1) m_{2}^4\nn&&+2 s ((8-15
%	D) m_{3}^2+(6 D-11) s))+(3 D-4) m_{1}^6-4 (m_{2}^2-s) (m_{2}^2-m_{3}^2+s) ((15 D-8) m_{3}^2\nn&&-6 D s+4 m_{2}^2+8 s)\Big\}\nn
%	%
	%
	%
%
%c_{i;022\to110}&=&\frac{(3 D-2) m_{1}^2 (3 m_{1}^2+2 m_{2}^2-6 s) (m_{1}^2 (2 m_{3}^2-5 m_{2}^2)+m_{1}^4+(3 m_{2}^2-s)
%	(m_{2}^2-m_{3}^2+s))}{3 (D+2) s (m_{1}^2-2 (m_{2}^2+s)) (m_{1}^2 (2 m_{2}^2-3 s)+m_{1}^4-2
%	(m_{2}^2-s) (m_{2}^2-m_{3}^2+s))}\nn
%
%c_{i;022\to101}&=&-\frac{1}{3 (D+2)
%	s (m_{1}^2-2 (m_{2}^2+s)) (m_{1}^2 (2 m_{2}^2-3 s)+m_{1}^4-2 (m_{2}^2-s) (m_{2}^2-m_{3}^2+s))}\times \Big\{(3 D\nn&&-2) m_{1}^2 (m_{1}^2 (5 m_{2}^2 m_{3}^2-3 m_{2}^4+3 s (m_{3}^2+5 s))-3 m_{1}^4 (m_{2}^2+4 s)+3 m_{1}^6+2
%	(m_{2}^2 (2 m_{3}^2 s\nn&&+6 m_{3}^4-3 s^2)+m_{2}^4 (3 s-9 m_{3}^2)+3 m_{2}^6-3 s (-m_{3}^2 s+2 m_{3}^4+s^2)))\Big\}\nonumber
%\eea
%\bea
%c_{i;022\to011}&=&\frac{1}{3 (D+2) s (2 (m_{2}^2+s)-m_{1}^2) (m_{1}^2 (2 m_{2}^2-3 s)+m_{1}^4-2 (m_{2}^2-s)
	%(m_{2}^2-m_{3}^2+s))}\times \Big\{(3 D-2) (2 m_{1}^2 (m_{2}^2 \nn&&(-23 m_{3}^2 s+3 m_{3}^4+13 s^2)+m_{2}^4 (4 m_{3}^2+6 s)+m_{2}^6-3 s (-3 m_{3}^2 s+m_{3}^4+4
%	s^2))+m_{1}^6 (7 m_{2}^2-3 (3 m_{3}^2+s))\nn&&+m_{1}^4 (m_{2}^2 (14 m_{3}^2-25 s)-10 m_{2}^4+15 s
%	(m_{3}^2+s))+4 (m_{2}^2-s) (m_{2}^2 (-4 m_{3}^2 s+5 m_{3}^4-s^2)+m_{2}^4 (3 s-6 m_{3}^2)\nn&&+m_{2}^6-3 s
%	(m_{3}^2-s)^2))\Big\}\nn
%
%c_{i;022\to200}&=&0\nn
%
%c_{i;022\to020}&=&\frac{(3 D-2) m_{2}^2 (m_{2}^2-s) (3 m_{1}^2+2 m_{2}^2-6 s) (5 m_{1}^2-4 (m_{2}^2-m_{3}^2+s))}{3 (D+2) s (2
	%(m_{2}^2+s)-m_{1}^2) (m_{1}^2 (2 m_{2}^2-3 s)+m_{1}^4-2 (m_{2}^2-s) (m_{2}^2-m_{3}^2+s))}\nn
%
%c_{i;022\to002}&=&-\frac{1}{3 (D+2) s (2(m_{2}^2+s)-m_{1}^2) (m_{1}^2 (2 m_{2}^2-3 s)+m_{1}^4-2 (m_{2}^2-s) (m_{2}^2-m_{3}^2+s))}\times \Big\{(3 D-2) m_{3}^2 (m_{1}^4 \nn&&(13 m_{2}^2-6 m_{3}^2+3 s)+m_{1}^2 (2 m_{2}^2 s-6 m_{2}^4+6 s (m_{3}^2-2 s))\nn&&-4 (m_{2}^2(-2 m_{3}^2 s+3 m_{3}^4-s^2)+m_{2}^4 (3 s-4 m_{3}^2)+m_{2}^6-3 s (m_{3}^2-s)^2))\Big\}\nonumber
%
%\eea%
%
Now we have reduced the situation to the similar one studies in previous subsubsection (see \eref{322}). 
Using the similar method we can finish the reduction completely.

%We can use the similar method to shift the dimension back to $\la_0$ (see \eref{234a}). Since the  

%Then we need to reduce the terms $i[\lambda_0-1;0,0,0]$, $i[\lambda_0-1;1,0,0]$, $i[\lambda_0-1;0,1,0]$, $i[\lambda_0-1;0,0,1]$, $i[\lambda_0-1;1,1,0]$, $i[\lambda_0-1;1,0,1]$, $i[\lambda_0-1;0,1,1]$, $i[\lambda_0-1;2,0,0]$, $i[\lambda_0-1;0,2,0]$, and $i[\lambda_0-1;0,0,2]$ to the integrals in the dimension $\lambda_0$ which contributes to integrals in dimension $D$.
%
\subsubsection{The example: $I_3^{(1,0)}(2,1,1)$}
%
This example has one tensor structure  as well as a high power propagator. Again we have 
%
\bea
%
i[\lam_0;1,0,0]^{\mu_1} %&=&\frac{-\lam_4}{\Gamma(2)}\frac{\d }{\d q_{1,\mu_1}}\int d\Pi^{(4)}~dy F(q,y)^{\lam_0}x_1^{1}x_2^0x_3^0x_4^{\lam_4-1}|_{q_i\rightarrow 0}\nn
%
%&=& \frac{-\lam_4}{\Gamma(2)}\int d\Pi^{(4)}~dy F(q,y)^{\lam_0-1}\frac{\d F(q,y)}{\d q_{1,\mu_1}}x_1 x_4^{\lam_4-1}|_{q_i\to0}\nn
%
&=&\frac{D}{3D-8}p^{\mu_1}i[\lam_0-1;1,1,1]~~~\Label{imu1100}
%
\eea
%
To reduce the $i[\lam_0,1;1,1,1]$, we could  use the solution of the syzygy equation  \eref{tensorsunsetge} in the first example. Choosing the same $z_j=\frac{g_j}{x_4}$ and $\lam_1=1$, $\lam_2=\lam_3=0$ in the recurrence relation \eref{syz2}, 
%we could get
%
%\bea
%
%\int d\Pi^{(4)}\Big\{\sum_{i=1}^4 \frac{\d z_i}{\d x_i}+\lam_0\frac{z_0x_2x_3x_4}{F}\Big\}F^{\lam_0}x_1x_4^{-4+\frac{3D}{2}}+boundaryterm_{\{1,1,1\}} &=&0~~~\Label{recur335}
%
%\eea
%
After some expansion % \eref{recur335}, we could get the relation to reduce the $i[\lam_0-1;1,1,1]$ to the integrals in the dimension $D$. Combining the result of $i[\lam_0-1;1,1,1]$ and \eref{imu1100}, and using \eref{Itoi}, 
we could get the recurrence relation
%
\bea
%
&&I_3^{(1,0)}(2,1,1)\nn
%
&=&c_{t;211\to211}I_3^{(0,0)}(2,1,1)+c_{t;211\to212}I_3^{(0,0)}(2,1,2)+c_{t;211\to221}I_3^{(0,0)}(2,2,1)\nn
%
&&+c_{t;211\to112}I_3^{(0,0)}(1,1,2)+c_{t;1,2,1}I_3^{(0,0)}(1,2,1)+c_{t;211\to311}I_3^{(0,0)}(3,1,1)+\cdots
%
\eea
%
with the coefficients
%
\bea
%
c_{t;211\to211}&=&\frac{\Big((4-D)m_1^2+(D-3)m_2^2+(D-3)m_3^2+(D-4)s\Big)p^{\mu_1}}{(3D-8)s}\nn
%
c_{t;211\to212}&=&\frac{m_3^2(-3m_1^2+3m_2^2+m_3^2-s)p^{\mu_1}}{(3D-8)s}\nn
%
c_{t;211\to221}&=&\frac{m_2^2(-3m_1^2+m_2^2+3m_3^2-s)p^{\mu_1}}{(3D-8)s}\nn
%
c_{t;211\to112}&=&\frac{m_3^2p^{\mu_1}}{(3D-8)s},~~
c_{t;211\to121}=\frac{m_2^2p^{\mu_1}}{(3D-8)s}\nn
%
c_{t;211\to311}&=&\frac{4m_1^2(s-m_1^2)p^{\mu_1}}{(3D-8)s}~~~\Label{I310211}
%
\eea
%
Notice that not all of the six basis in the right hand side of \eref{I310211} are the master integrals, and the three terms $I_3^{(0,0)}(2,1,2)$, $I_3^{(0,0)}(2,2,1)$, $I_3^{(0,0)}(3,1,1)$ are still needed to be reduced. After reducing the three terms and adding the boundary terms, we got the final results which confirmed with FIRE6.
%
%\subsubsection{The tensor's reduction with higher  Lorentz index $r_i$ and general power}
%

Having above examples, we can see that for general integrals with arbitrary tensor structures and arbitrary powers,
first we need to use the differential operators \eref{tensor2} to get the expression \eref{sumI}, where each one is
the scalar integrals, but possibly with different dimension. Secondly, we need to shift dimension back by the trick
\eref{tensortrick1}. After these two steps, we are left with only scalar integrals in the same dimension, but 
different powers. Now we can use the trick \eref{syzy} to reduce these scalar integrals to the master basis.

%There are two steps in algorithm tensor's reduction. Firstly, we need to use the differential operators \eref{tensor2} to get the expression \eref{sumI} in which we need to summary many terms in different dimensions. Secondly  the terms in different dimensions should be reduced  to the master basis in dimension D, since  the results are ususlly considered  in the  dimension $D$.


%In the former subsection, we disscussed how to reduce scalar  integrals in different dimension,  which come from the  tensor structure,  to the dimension $D$. If $r_i$ is larger than one, we will get the expression where the summation in \eref{sumI} contains more integrals in  dimensions different from $D$. To pullback the dimension, we proposed the trick  by solving the syzygy equations in the form \eref{tensortrick1}.    In each step,  one term in dimension $D+k+2$ is expressed   by integrals   dimension $D+k$ with propagators in higher power. Repeating this procedure, we could finally reduce the terms in the summation to integrals in the dimension $D$, which usually have general power of propagators and not our master basis. Then the question becomes to the {\bf reduction of scalar integrals in the same dimension $D$ with general power.}
%By the  trick discussed above, in  each step we just need to solve one syzygy equations {\bf once}.
%
%
\section{Example 2: double triangle diagram}
%
Now we consider an more complicated example, 
%
\bea
%
I_5^{(r_1,r_2)}(n_1,\cdots n_5)&=&\int d^{D}l_1d^{D}l_2  \frac{l_1^{\mu_1}\cdots l_1^{\mu_{r_1}}l_2^{\nu_2}\cdots l_1^{\nu_{r_2}}}{D_1^{n_1}\cdots D_5^{n_5}}
%
\eea
%
with the propagators 
%
\bea
%
D_1&=&l_1^2-m_1^2,~~D_2=l_2^2-m_2^2,~~D_3=(l_1+l_2)^2-m_3^2,~~D_4=(l_1-p)^2-m_4^2,~~D_5=(l_2+p)^2-m_5^2~~~~~~~~~~~~~~~
%
\eea
%
The corresponding functions are
%
\bea
%
F&=&x_1 \Big(x_2 \Big(x_3
(m_{1}^2+m_{2}^2+m_{3}^2)+x_4
(m_{1}^2+m_{4}^2-s)+m_{2}^2 x_5+m_{5}^2
x_5-s x_5+x_6\Big)\nn&&+x_3 \Big(x_5
(m_{1}^2+m_{3}^2+m_{5}^2-s)+x_4
(m_{1}^2+m_{4}^2-s)+x_6\Big)+x_5 (x_4
(m_{1}^2+m_{4}^2-s)\nn&&+m_{5}^2
x_5+x_6)+m_{2}^2 x_2^2+m_{3}^2 x_3^2\Big)+m_{1}^2
(x_2+x_3+x_5) x_1^2+x_2 \Big(x_3 \Big(x_4
(m_{2}^2+m_{3}^2\nn&&+m_{4}^2-s)+x_5
(m_{2}^2+m_{5}^2-s)+x_6\Big)+x_4 \Big(x_5
(m_{2}^2+m_{5}^2-s)+m_{4}^2
x_4+x_6\Big)+m_{3}^2 x_3^2\Big)\nn&&+m_{2}^2 x_2^2
(x_3+x_4)+\Big(x_4 x_5+x_3 (x_4+x_5)\Big)
\Big(m_{3}^2 x_3+m_{4}^2 x_4+m_{5}^2 x_5+x_6\Big)\nn
%
A&=&\left[\begin{array}{cc}
	x_1+x_3+x_4&x_3\nn
	x_3&x_2+x_3+x_5
	\end{array}\right]
%
,~~B=\left[\begin{array}{c}
	-px_4\nn
	px_5
	\end{array}\right]\nn
%
C&=&-m_1^2x_1-m_2^2x_2-m_3^2x_3-m_4^2x_4-m_5^2z_5+p^2(x_4+x_5)~~~\Label{4.3}
%
\eea
%
For simplicity, we always choose the scalar basis $I_{5}^{(0,0)}(1,1,1,1,1)$ as the basis of top-sector in the following calculation.
%
\bea
\eea
%
\subsection{Scalar's case}
%
In the scalar's case, i.e., $r_1=r_2=0$, we solved the syzygy equations \eref{syzy} using degrevlex ordering, and got 8 generators of the syzygy with degree one. Using the similar method in sunset's case, we define $z_i\equiv \frac{\sum_{i=1}^{8}c_ig_{ji}}{x_6}$ and obtain the following IBP recurrence relation without dimensional shift,
%
\bea
%
\sum_{\{i_1,\cdots,i_5\}\in A_0\cup\cdots \cup A_3} C_{n_1,\cdots, n_5}^{i_1\cdots i_5} i[n_1+i_1,\cdots n_5+i_5]+\delta_{bound}&=&0
%
\eea
%
where  the four types of the values of $\{i_1,\cdots ,i_5\}$ are
%
\bea
%
A_0&=&\{0,0,0,0,0\},~~A_1=\cup_{Permutation}\{1,0,0,0,0\},\nn
%
A_2&=&\cup_{Permutation}\{1,-1,0,0,0\},~~A_3=\cup_{Permutation}\{-1,0,0,0,0\}~~~\Label{doubletriangle}
%
\eea
%
By choosing the particular values of $c_1$ to $c_8$, we could get five independent recurrence relation to lower the total power,
\bea
%
%
&&c_{1,0,0,0,0} i[\lam_0,n_1+1,\cdots,\cdots n_5]+\sum_{\substack{\{k_1,\cdots,k_5\}\\ \in A_0\cup A_2\cup A_3}}c_{k_1,\cdots,k_5} i[\lam_0,n_1+k_1,\cdots n_5+k_5]+\delta_{B,1}=0~~\Label{line1dbt}\\
%
&&c_{0,1,0,0,0} i[\lam_0,n_1,\cdots,n_2+1,\cdots n_5]+\sum_{\substack{\{k_1,\cdots,k_5\}\\ \in A_0\cup A_2\cup A_3}}c_{k_1,\cdots,k_5} i[\lam_0,n_1+k_1,\cdots n_5+k_5]+\delta_{B,2}=0~~\Label{line2dbt}\\
%
&&c_{0,0,1,0,0} i[\lam_0,n_1,\cdots,n_3+1,\cdots n_5]+\sum_{\substack{\{k_1,\cdots,k_5\}\\ \in A_0\cup A_2\cup A_3}}c_{k_1,\cdots,k_5} i[\lam_0,n_1+k_1,\cdots n_5+k_5]+\delta_{B,3}=0~~\Label{line3dbt}\\
%
&&c_{0,0,0,1,0} i[\lam_0,n_1,\cdots,n_4+1,\cdots n_5]+\sum_{\substack{\{k_1,\cdots,k_5\}\\ \in A_0\cup A_2\cup A_3}}c_{k_1,\cdots,k_5} i[\lam_0,n_1+k_1,\cdots n_5+k_5]+\delta_{B,4}=0~~\Label{line4dbt}\\
%
&&c_{0,0,0,0,1} i[\lam_0,n_1,\cdots,\cdots n_5+1]+\sum_{\substack{\{k_1,\cdots,k_5\}\\ \in A_0\cup A_2\cup A_3}}c_{k_1,\cdots,k_5} i[\lam_0,n_1+k_1,\cdots n_5+k_5]+\delta_{B,5}=0~~~~~~~\Label{line5dbt}
%
\eea
%
where the boundary terms $\delta_{B,j}$ 
 are too long to be shown here.
%
\subsubsection{The example: $I_5^{(0,0)}(2,1,1,1,1)$}
%
In this example, we need to reduce $i[\lam_0;1,0,0,0,0]$. We could use the first line in \eref{line1dbt} with $n_1=\cdots=n_5=0$ and solve the target $i[\lam_0;1,0,0,0,0]$.  The result is
%
\bea
%
i[\lam_0;1,0,0,0,0]&=&c_{10000\to00000}i[\lam_0;0,0,0,0,0]+\cdots
%
\eea
%
with the coefficients
%
\bea
%
c_{10000\to00000}&=&\frac{-2m_1^2m_5^2+m_2^2(-m_3^2+m_4^2+m_5^2)+m_3^2m_5^2 +m_3^2 s +m_4^2 m_5^2 -m_4^2 s +m_5^2 s -m_5^4}{N_{11111}}\nn
%
N_{11111}&=&3 (m_1^2 (m_2^2(m_3^2-m_4^2-m_5^2 ) -m_3^2(m_5^2+s)  -(m_4^2-m_5^2)(m_5^2-s) ) +m_1^4m_5^2 -m_2^2 (m_3^2(m_4^2+s) \nn
%
&&-(m_4^2-m_5^2)(m_4^2-s)) +m_2^4m_4^2 +m_3^2 (s (m_3^2-m_5^2+s)+m_4^2(m_5^2-s))  )
\eea
%
Then using the \eref{defI} to translate $i[\lam_0;n_1,\cdots,n_5]$ to $I[n_1+1,\cdots,n_5+1]$, we got the reduction coefficients which is confirmed with FIRE6. Here we just used the simplified IBP relation {\bf once}.

\subsubsection{The example: $I_5^{(0,0)}(3,1,1,1,1)$}
%
In this case, we need to reduce $i[\lam_0;2,0,0,0,0]$. In the first step, one could reduce $i[\lam_0;2,0,0,0,0]$ to terms having {\bf total lower power} by \eref{line1dbt}, with the results,
%
\bea
%
i[\lam_0;2,0,0,0,0]&=&c_{2;10000}i[\lam_0;1,0,0,0,0]+c_{2;01000}i[\lam_0;0,1,0,0,0]+c_{2;00100}i[\lam_0;0,0,1,0,0]\nn
%
&&+c_{2;00010}i[\lam_0;0,0,0,1,0]+c_{2;00001}i[\lam_0;0,0,0,0,1]+c_{2;00000}i[\lam_0;0,0,0,0,0]+\cdots 
%
\eea
To get the final reduction results of  $i[\lam_0;2,0,0,0,0]$, we should go on and  reduce $i[\lam_0;1,0,0,0,0]$, $i[\lam_0;0,1,0,0,0]$, $i[\lam_0;0,0,1,0,0]$, $i[\lam_0;0,0,0,1,0]$ and $i[\lam_0;0,0,0,0,1]$ to $i[\lam_0;0,0,0,0,0]$. By using  \eref{line1dbt} to \eref{line5dbt}, we could easily get the reduction coefficients from these five terms to $i[\lam_0;0,0,0,0,0]$. The final result is confirmed with FIRE6.
%
\subsubsection{The scalar's reduction with general power}
%
For the case with general powers in double-triangle, things are similar. Iteratively using  the  five recurrence relations \eref{line1dbt} to \eref{line5dbt}, one can lower the total power one by one to the scalar basis as did in previous two examples. 
%
\subsection{Tensor's case}
%
%The doubld-triangle with tensor structure is defined as
%
For the tensor case, the function $F(q,y)$ is 
%
\bea
%
F(q,y)&=&F(0,0)+y\Big(x_2(x_3+x_4+p \cdot q_1 x_4) + x_1 (x_2+x_3 +x_5-p \cdot q_2 x_5)\nn&& +(1+p\cdot (q_1-q_2)) (x_4 x_5 +x_3(x_4+x_5)) \Big)\nn
%
&&-\frac{1}{4}y^2\Big( -2q_1\cdot q_2 x_3+q_2^2 (x_1+x_3+x_4)+ q_1^2 (x_2+x_3+x_5)  \Big)
%
\eea
where $F(0,0)$ is the function $F$   given in \eref{4.3}.
%
%

\subsubsection{The example: $I_5^{(1,0)}(1,1,1,1,1)$}
%
By \eref{tensor1}, we got 
%
\bea
%
i[\lam_0;0,0,0,0,0]^{\mu_1}&=&\frac{\frac{D}{2}}{(\frac{3D}{2}-6)(\frac{3D}{2}-5)} p^{\mu_1} \Big(i[\lam_0-1;0,1,0,1,0]+i[\lam_0-1;0,0,0,1,1]\nn&&+i[\lam_0-1;0,0,1,1,0] +i[\lam_0-1;0,0,1,0,1]  \Big)~~~~~\Label{tensorexample10}
%
\eea
%
To get the reduction in the  dimension $D$, we need to reduce the four temrs in dimension $D+2$ using the method in \eref{tensortrick1}.  By solving syzygy equation
%
\bea
%
\sum_{i=1}^{6} g_i\frac{\d F}{\d x_i}&=&g_0 (x_2x_4+x_4x_5+x_3x_4+x_3x_5)x_6^{\alpha}
%
\eea
%
using SINGULAR (Here we choose $\alpha=0$), we got the solutions where one generator reads
%
\bea
%
g_1&=&1,~~g_2=0,~~g_3=0,~~g_4=-1,~~g_5=0,~~g_6=-m_1^2+m_4^2-s,~~g_0=-2s
%
\eea
Using the $g_0$ to $g_6$ in the IBP identity \eref{syz2} with $z_i\equiv g_i$, we have
%
\bea
%
&&-2s \Big(i[\lam_0-1;0,1,0,1,0]+ i[\lam_0-1;0,0,0,1,1] +i[\lam_0-1;0,0,1,1,0] +i[\lam_0-1;0,0,1,0,1] \Big) \nn&&+\frac{1}{2}(10-3D)(m_1^2-m_4^2+s) i[\lam_0;0,0,0,0,0] +  boundary~term = 0 ~~~\Label{4.20}
%
\eea
%
Taking \eref{4.20} into \eref{tensorexample10}, we solved the $i[\lam_0;0,0,0,0,0]^{\mu_1}$ in dimension $D$. Then by \eref{sumI}, we got the reduction result
%
\bea
%
I[\lam_0;1,1,1,1,1]^{\mu_1}&=& \frac{p^{\mu_1} (m_1^2-m_4^2+s)}{2s} I_5^{(0,0)}(1,1,1,1,1)
%
\eea
%
where the coefficients are confirmed with FIRE6.
%
%
%
%
\subsubsection{The example: $I_5^{(1,1)}(1,1,1,1,1)$ (to be checked)}
%
By directly calculation, 
%
\bea
%
&&i[\lam_0;0,0,0,0,0]^{\mu\nu}\nn
%
&=&\frac{(\lam_0 g^{\mu\nu})}{2}\times\frac{2}{\lam_6(\lam_6+1)(\lam_6+2)}\int d\Pi^{(6)} F^{\lam_0-1}x_3x_6^{\lam_6+2}-\frac{\lam_0(\lam_0-1)p^{\mu}p^{\nu}}{2} \frac{2}{\lam_6(\lam_6+1)(\lam_6+2)}\nn
%
&&\times \int d\Pi^{(6)}F^{\lam_0-2}\Big(x_2x_4+x_4x_5+x_3(x_4+x_5)\Big)\Big( (x_1+x_4)x_5+x_3(x_4+x_5)\Big)x_6^{\lam_6+2}~~~~~~~~
%
\eea
%
where the terms in the second line contribute to Feynman integrals in dimension parameter $\lam_0-2$.
By syzygy trick  disscussed above, we should  first reduce the integrals in dimension parameter $\lam_0-2$ to dimension parameter $\lam_0-1$. To achieve this, we should solve the following syzygy equation,
%
\bea
%
\sum_{i=1}^{6}g_i\frac{\d F}{\d x_i}&=&g_0\Big(x_2x_4+x_4x_5+x_3(x_4+x_5)\Big)\Big( (x_1+x_4)x_5+x_3(x_4+x_5)\Big) x_6^{\alpha}
%
\eea
%
By choosing $\a=0$, we got the following result
%
\bea
%
g_1&=&0,~~g_2=m_4^2x_4^2-m_3^2(x_3+x_4)x_5+x_4(m_5^2x_5+x_6),g_3=x_4(m_2^2x_3-m_3^2x_3-2sx_3-sx_4+m_1^2(x_1+x_3+x_4))\nn
%
g_4&=&x_4\Big(-((m_1^2+2(-m_2^2+m_3^2+s))x_1-(m_1^2-m_2^2+m_3^2)x_3-(m_1^2-2m_2^2+2m_3^2+s)x_4\Big)\nn
%
g_5&=&-m_1^2x_1x_4-2sx_2x_4+(m_1-m_2^2+m_3^2)x_3x_4-(m_1^2+m_4^2-s)x_4^2+m_3^2x_3x_5+(m_3^2-m_5^2)x_4x_5-x_4x_6\nn
%
g_6&=&(2(m_2^2-m_3^2-s)(s-m_4^2)+m_1^2(-2m_2^2+m_3^2+m_4^2+m_5^2+2s))x_1x_4-2(m_2^4+s(-m_5^2+s)-m_2^2(m_3^2\nn
&&+2s))x_2x_4+(m_2^2(-3m_3^2-m_4^2+m_5^2)-m_1^2(m_3^2-m_4^2+m_5^2)+m_3^2(3m_3^2+m_4^2-m_5^2+6s))x_3x_4+(-5m_2^2m_4^2\nn
%
&&+4m_3^2m_4^2+m_4^2m_5^2+m_1^2(-m_3^2+m_4^2+m_5^2)+m_3^2s+4m_4^2s-m_5^2s)x_4^2+m_3^2(m_2^2-m_5^2+s)x_3x_5\nn
%
&&+(m_5^4+m_2^2(m_3^2-3m_5^2)+3m_5^2s+m_3^2(m_5^2+s))x_4x_5+(-3m_2^2+2m_3^2+m_5^2+3s)x_4x_6\nn
%
g_0&=&2m_3^2 s
%
\eea
%
Defing $z_i\equiv \frac{g_i}{x_6^2}$ in the \eref{syz2} with $\lam_1=\cdots=\lam_5=0$, and  replace $\lam_0$ by $\lam_0-1$. By similar procedure, we reduce the integrals in dimension parameter $\lam_0-1$ to dimension parameter $\lam_0$. 

%
\section{Example 3: Topology F in \cite{Xu:2018eos}}
%
This example is given by
%
\bea
%
I_7^{(r_1,r_2)}(n_1,\cdots n_7)&=&\int d^D l_1 d^D l_2\frac{l_1^{\mu_1}\cdots l_1^{\mu_{r_1}} l_1^{\mu_2}\cdots l_2^{\mu_{r_2}}  }{D_1^{n_1}\cdots D_{7}^{n_7}}
%
\eea
%
where the propagators are given by\footnote{In \cite{} there are $D_8=(l_1+p_3)^2$ and $D_9=(l_2-p_1)^2-m^2$ which is for the tensor structure. Since we use the different 
	way to deal with tensor, we do not include them here.}
%
\bea
%
\{D_1,\cdots D_7\}&=&\{(l_1-p_1)^2,l_1^2,(l_1+p_2)^2,(l_1+l_2-p_1)^2-m^2,l_2^2-m^2,\nn
%
&&(l_2+p_3)^2-m^2,(l_1+l_2+p_2+p_3)^2-m^2,(l_1+p_3)^2,(l_2-p_1)^2-m^2\}
%
\eea
%where $D_8$ and $D_9$ are two ISPs which for simplicity, we elip them. 
All external momenta are set to be on-shell, so the kinematic variables are 
%
\bea
%
s&=&(p_1+p_2)^2,~~t=(p_1+p_3)^2,~~u=(p_2+p_3)^2=(p_1+p_4)^2=-s-t,~~p_1^2=p_2^2=p_3^2=p_4^2=0
\eea
%with this exception we have $s+t+u=0$. All our kinematic invariants are $s,t$ and $m$.

%

%
The corresbonding functions are given by
%
\bea
A&=&\lba{cc}
%
x_1+x_2+x_3+x_4+x_7&x_4+x_7\nn
%
x_4+x_7&x_4+x_5+x_6+x_7\nn
%
%
\rba\nn
%
%
B&=&\lba{c}
%
-(x_1+x_4)p_1+(x_3+x_7)p_2+x_7p_3\nn
%
-x_4p_1+x_7p_2+(x_6+x_7)p_3
%
\rba\nn
%
C&=&-m^2(x_4+x_5+x_6+x_7)+p_{1}^2 (x_{1}+x_{4})+p_{2}^2 (x_{3}+x_{7})+2 p_{2}\cdot p_{3} x_{7}+p_{3}^2 x_{6}+p_{3}^2 x_{7}
\eea
%
The function F is given by
%
\bea
%
F(x)&=&U(x)x_8+f(x)\nn
%
&=&m^2 (x_{4}+x_{5}+x_{6}+x_{7}) (x_{1} (x_{4}+x_{5}+x_{6}+x_{7})+x_{2} (x_{4}+x_{5}+x_{6}+x_{7})+x_{3} x_{4}+x_{3} x_{5}\nn&&+x_{3} x_{6}+x_{3} x_{7}+x_{4} x_{5}+x_{4} x_{6}+x_{5} x_{7}+x_{6} x_{7})-s (x_{1} x_{3} (x_{4}+x_{5}+x_{6}+x_{7})+x_{1} x_{6} x_{7}\nn&&+x_{5} (x_{3} x_{4}-x_{2} x_{7}))+t x_{2} (x_{5} x_{7}-x_{4} x_{6})+x_{8} (x_{1} (x_{4}+x_{5}+x_{6}+x_{7})+x_{2} (x_{4}+x_{5}+x_{6}+x_{7})\nn&&+x_{3} x_{4}+x_{3} x_{5}+x_{3} x_{6}+x_{3} x_{7}+x_{4} x_{5}+x_{4} x_{6}+x_{5} x_{7}+x_{6} x_{7})
%
\eea
For later convenience, we choose the scalar basis in the top topology as
\bea
&&I_7^{(0,0)}(2,1,1,1,1,1,1),~~I_7^{(0,0)}(1,1,1,2,1,1,1),~~I_7^{(0,0)}(1,1,1,1,2,1,1),\nn
%
&&I_7^{(0,0)}(1,1,1,1,1,2,1),~~I_7^{(0,0)}(1,1,1,1,1,1,2),~~I_7^{(0,0)}(1,1,1,1,1,1,1),~~\cdots
\eea
where the "$\cdots$" represents the basis in the subtopologies.
\subsection{Scalar's case}
%
In the scalar's case, we solved the syzygy equations \eref{syzy} in lexicographical order and got 55 generators, where 2 in degree one, 37 in degree two, and 16 in degree three. Using the similar method in the former two examples, we obtain the following IBP recurrence relations without dimensional shift,
%
\bea
\sum_{\{i_1,\cdots ,i7\}\in A_0\cup A_{12}}C_{n_1,\cdots,n_7}^{i_1,\cdots,i_7} i[\lam_0;n_1+i_1,\cdots,n_7+i_7]+\delta_{bound}&=&0
\eea
%
with 13 types of the values of $\{i_1,\cdots ,i_7\}$ as
%
\bea
%
A_1&=&\cup_{Permutation}\{3,0,0,0,0,0,0\},~~A_2=\cup_{Permutation}\{2,1,0,0,0,0,0\},~~
%
%
A_3=\cup_{Permutation}\{1,1,1,0,0,0,0\}\nn
%
%
A_4&=&\cup_{Permutation}\{2,0,0,0,0,0,0\},~~
%
A_5=\cup_{Permutation}\{1,1,0,0,0,0,0\},~~A_6=\cup_{Permutation}\{3,-1,0,0,0,0,0\}\nn
%
A_7&=&\cup_{Permutation}\{2,1,-1,0,0,0,0\},~~
%
A_8=\cup_{Permutation}\{1,1,1,-1,0,0,0\},~~A_9=\cup_{Permutation}\{1,0,0,0,0,0,0\}\nn
%
%
A_{10}&=&\cup_{Permutation}\{2,-1,0,0,0,0,0\},~~A_{11}=\cup_{Permutation}\{1,1,-1,0,0,0,0\},~~A_{12}=\cup_{Permutation}\{1,-1,0,0,0,0,0\}\nn
%
A_{0}&=&\{0,0,0,0,0,0,0\}
%
%
\eea
Note there are three types of generators of syzygy in degree one, two, and three respectively. The generators in degree one read as
%
\bea
g_{1i}&=&\Big\{x_2+x_3,-x_2,2 x_1+2 x_2+x_3,x_7,x_6,2 x_5+x_6,2 x_4+x_7\nn&&,-2 \Big(3 m^2 x_4+3 m^2 x_5+3 m^2 x_6+3 m^2 x_7+2 x_8 \Big)+2 s x_1+s x_3\Big\}\nn
%
g_{2i}&=&\Big\{2 x_1+3 \Big(x_2+x_3\Big),-x_2,-x_3,2 x_4+3 x_7,2 x_5+3 x_6,-x_6,-x_7,\nn&&3 s x_3-2 \Big(3 m^2 x_4+3 m^2 x_5+3 m^2 x_6+3 m^2 x_7+2 x_8\Big)\Big\}
\eea
Defining 
%
\bea
z_{k}&\equiv& \frac{c_{1}g_{1i}+c_{2}g_{2i}}{x_8},~~k=1,\cdots,8
\eea
with $c_1=-c_2$ and taking them into \eref{syz2}, we got two independent recurrence relations
%
\bea
%
&&c_{n_1,\cdots,n_7}^{1000000}i[\lam_0;n_1+1,\cdots,n_7]+c_{n_1,\cdots,n_7}^{0010000}i[\lam_0;\cdots,n_3+1,\cdots]+\delta_{11}=0\nn
%
&&c_{n_1,\cdots,n_7}^{1000000}i[\lam_0;n_1+1,\cdots]+c_{n_1\cdots,n_7}^{0001000}i[\cdots,n_4+1,\cdots]+c_{n_1,\cdots,n_7}^{0000100}i[\cdots,n_5+1,\cdots]\nn&&+c_{n_1,\cdots,n_7}^{0000010}i[\cdots,n_6+1,n_7]+c_{n_1,\cdots,n_7}^{0000001}i[\cdots,n_7+1]+c_{n_1,\cdots,n_7}^{0000000}i[\lam_0;n_1,\cdots,n_7]+\delta_{12}=0  ~~~\Label{step1}
%
\eea 
%
Since the two relations could not be used to lower total power, we need to combine them with the generators in degree two and setting $z_i\equiv \frac{\sum_{j=1}^{37} b_j g_{ji} }{ x_8^{2}}$. By choosing proper values of the  coefficients $b_j$, we could get the simplified IBP relations,
%
\bea
%
c_{n_1,\cdots,n_7}^{i_1,\cdots,i_7}i[\lam_0;n_1+i_1,\cdots n_7+i_7] +\sum_{\substack{\{i_1,\cdots,i_7\} \in A_0\cup A_9\\ \cup A_{10} \cup A_{11} \cup A_{12} } }  c_{n_1,\cdots,n_7}^{i_1,\cdots,i_7} i[\lam_0;n_1+i_1,\cdots,n_7+i_7] +\delta_{2} = 0,\forall\{i_1,\cdots,i_7\}\in A_5~~~~\Label{topoF22}
%
\eea


\subsubsection{The example: $I_7^{(0,0)} (2,1,2,1,1,1,1)$}
In this example, we need to reduce $i[\lam_0;1,0,1,0,0,0,0]$. Setting $n_1=\cdots=n_7=0$, we could use one of the relations  \eref{topoF22}, 
%
\bea
c_{0000000}^{1010000}i[\lam_0;1,0,1,0,0,0,0]+\sum_{j=1,j\neq2}^{7}j^{+}i[\lam_0;0,0,0,0,0,0,0]+c_{0000000}^{0000000}i[\lam_0;0,0,0,0,0,0,0]+\delta_{1,3}&=&0~~~~\Label{step2}~~~~~~~~~
\eea
Note in the relation  \eref{step2} there is no term $i[\lam_0;0,1,0,0,0,0,0]$, but there is one term $i[\lam_0;0,0,1,0,0,0,0]$, which is not our basis. So the next step is to use \eref{step1} to reduce $i[\lam_0;0,0,1,0,0,0,0]$ to $i[\lam_0;1,0,0,0,0,0,0]$.
After the two step, we successfully reduce our target $i[\lam_0;1,0,1,0,0,0,0]$ in the top-sector level. After translating the $i[\lam_0;n_1,\cdots,n_7]$ to $I_7^{(0,0)}(n_1+1,\cdots,n_7+1)$, we obtain the reduction coefficients which are confirmed by FIRE6. In summary,  to get the reduction of the top-sector, we just need to use \eref{step1} and \eref{step2} once respectively.


\subsection{The tensor's case}
%
For the tensor,  the function  $F(q,y)$ is   
%
\bea
%
F(q,y)&=&F(0,0)+\frac{1}{4} y (-2 q_2\cdot  (2 p_2 ((x_1+x_2) x_7-x_3 x_4)-2 p_1 (x_2 x_4+x_3 x_4-x_1 x_7)+2 p_3 (x_3 x_6+x_4 x_6\nn
%
&&+x_7 x_6+x_3 x_7+x_1 (x_6+x_7)+x_2 (x_6+x_7))+q_1 (x_4+x_7) y)+4 q_1 (p_3 (x_4 x_6-x_5 x_7)\nn
%
&&+p_1 (x_4 (x_5+x_6)+x_1 (x_4+x_5+x_6+x_7))-p_2 ((x_5+x_6) x_7+x_3 (x_4+x_5+x_6+x_7)))\nn
%
&&+q_1^2 (x_4+x_5+x_6+x_7) y+q_2^2 (x_1+x_2+x_3+x_4+x_7) y+4 (x_3 x_4+x_5 x_4+x_6 x_4\nn
&&+x_3 x_5+x_3 x_6+x_3 x_7+x_5 x_7+x_6 x_7+x_1 (x_4+x_5+x_6+x_7)+x_2 (x_4+x_5+x_6+x_7)))~~~
%
\eea
%
\subsubsection{The  example: $I_7^{(1,0)}(1,1,1,1,1,1,1)$}

%
By directly calculation and integrate over y, we got 
%
\bea
&&i[\lam_0,0,\cdots,0]^{\mu}\nn
%
&=&\frac{-D}{3D-14}\Big( \Big(-(i[\lam_0-1,0,0,0,1,0,1,0]+i[\lam_0-1,0,0,0,1,1,0,0]+i[\lam_0-1,1,0,0,0,0,0,1]\nn&&+i[\lam_0-1,1,0,0,0,0,1,0]+i[\lam_0-1,1,0,0,0,1,0,0]+i[\lam_0-1,1,0,0,1,0,0,0]) p_1^{\mu}\nn&&+(i[\lam_0-1,0,0,0,0,0,1,1]+i[\lam_0-1,0,0,0,0,1,0,1]+i[\lam_0-1,0,0,1,0,0,0,1]+i[\lam_0-1,0,0,1,0,0,1,0]\nn&&+i[\lam_0-1,0,0,1,0,1,0,0]+i[\lam_0-1,0,0,1,1,0,0,0]) p_2^{\mu}+(i[\lam_0-1,0,0,0,0,1,0,1]\nn&&-i[\lam_0-1,0,0,0,1,0,1,0]) p_3^{\mu}\Big)\Big)~~~~~~~~~~~~~~~~~~
\eea
%
To reduce the term in different dimension, we could solve the syzygy equation
%
\bea
%
\sum_{i=1}^8 g_{ji}\frac{\d F}{\d x_i}&=&g_0 \tilde Y
%
\eea
%
with 
%
\bea
%
\tilde Y&=&x_4 x_5 p_1^{\mu}+x_4 x_6 p_1^{\mu}+x_1 \Big(x_4+x_5+x_6+x_7\Big) p_1^{\mu}-x_5 x_7 p_2^{\mu}-x_6 x_7 p_2^{\mu}-\nn&&x_3 \Big(x_4+x_5+x_6+x_7\Big) p_2^{\mu}+x_4 x_6 p_3^{\mu}-x_5 x_7 p_3^{\mu}
%
\eea
and then defing the $z_i$s as
%
\bea
%
z_i&\equiv &\frac{\sum_{j}c_{j}g_{ji}}{x_8^\alpha}
%
\eea
%
Taking the $z_i$s into the IBP identities, we got the results. Since the results are too complicated to write here, we put them in the attached files. Using the similar method, we finally got the reduction results.
%
%
\section{Conclusion}

In this article, we considered a new  parametric representation of  Feynman integrals proposed by Chen \cite{chen1,chen2}. In the practical calculation of two-loop integrals, there are large number of IBP relations which makes our calculation hard. To simplify the IBP relations, we used the "syzygy" trick to cancel the dimensional shift and the unwanted doubled propagators in the parametric representation. By this trick, as shown in three examples of two-loop diagrams, we successfully simplified the IBP relations, and used them to reduce the diagrams. The main advantage of our method is that, by the simplified IBP relations, we could directly choose which relation we need to lower the total power, and do the reduction more efficiently than the traditional method, so it will be good if such a  procedure could be programmed.
However, as the number of loops going higher, the homogeneous polynomial  $F$ becomes more and more complicated, which makes the syzygy equations hard to solve. Further more, more mathematical properties and constructions are still need to find and utilize, such as the polynomial tangent space. We will focus on them in the near future.
%
\section*{Acknowledgments}
I would like to thank Bo Feng for the inspiring guidance.

\newpage
\appendix
\section{The syzygy generators of sunset's case}
%
In this part, we give the syzygy generators in the section 4.1:

\bean
%
g_1&=&\Big\{0, 0, s x_3^2,
m_1^4 x_1^2 + m_2^4 x_2^2 + m_3^4 x_3^2 - m_3^2 s x_3^2 +
2 m_3^2 x_3  + x_4^2 + 2 m_2^2 x_2 (m_3^2 x_3 + x_4) +
2 m_1^2 x_1 (m_2^2 x_2 + m_3^2 x_3 + x_4)\nn&&, -m_1^2 x_1 - m_2^2 x_2 -
m_3^2 x_3 - s x_3 - x_4\Big\}\eean
%
%
\bean
g_2&=&\Big\{0, 0, -s (m_2^2 + s) x_3^2 + m_1^4 (x_2 x_3 + x_1 (x_2 + x_3)) -
m_1^2 (-2 s x_3^2 + m_2^2 (x_2 x_3 + x_1 (x_2 + x_3))),\nn&&
m_1^6 x_1 (x_1 - x_2) -
m_1^4 (s x_1 (x_1 - x_2) + m_3^2 x_1 x_2 +
m_2^2 x_2 (-4 x_1 + x_2) - 2 m_3^2 x_1 x_3 +
2 m_3^2 x_2 x_3 \nn&&- 3 x_1 x_4 + x_2 x_4) - (m_2^2 +
s) (m_2^4 x_2^2 + m_3^4 x_3^2 + x_4^2 +
2 m_2^2 x_2 (m_3^2 x_3 + x_4) + m_3^2 x_3 (-s x_3 + 2 x_4)) \nn&&+
m_1^2 (m_2^4 x_2 (-x_1 + 3 x_2)+
m_2^2 (-3 s x_1 x_2 + m_3^2 x_2 (x_1 + 6 x_3) - x_1 x_4 +
5 x_2 x_4) +
2 (m_3^4 x_3^2 \nn&&- m_3^2 x_3 (s (x_1 + x_3) - 2 x_4) +
x_4 (-s x_1 + x_4))), -(2 m_1^2 - m_2^2 - s) (m_1^2 x_1 +
m_2^2 x_2 + m_3^2 x_3 + s x_3 + x_4)\Big\}\eean
%
\bean
g_3&=&\Big\{0, x_2^2, -x_3^2, -m_2^2 x_2^2 + m_3^2 x_3^2, -x_2 + x_3\Big\}\eean
%
\bean
g_4&=&\Big\{0, 2 m_2^4 x_2^2 - m_1^4 (x_1 + x_2) x_3 +
m_1^2 (-2 m_2^2 x_2^2 + s (x_1 + x_2) x_3),
x_3 (2 (-m_2^4 + s^2) x_3 + m_1^4 (x_1 + x_3)\nn&& +
m_1^2 (2 m_2^2 x_3 - s (x_1 + 3 x_3))),
2 m_1^2 m_2^2 (-m_1^2 + s) x_1 x_2 - 2 (m_2^6 - m_2^4 s) x_2^2 +
m_1^2 (m_1^2 - s) (m_1^2 \nn&&+ m_2^2 - 3 m_3^2 - s) x_1 x_3 +
2 m_2^2 (m_1^2 - 2 m_3^2) (m_1^2 - s) x_2 x_3 -
2 m_3^2 (-m_2^4 + m_1^2 (m_2^2 + m_3^2 - s) + s (-m_3^2 + s)) x_3^2 \nn&&-
2 (m_1^4 - m_1^2 s) x_1 x_4 +
4 m_2^2 (-m_1^2 + s) x_2 x_4 + (m_1^2 - 4 m_3^2) (m_1^2 - s) x_3 x_4 - 2 (m_1^2 - s) x_4^2, \nn&&-2 m_2^2 (-2 m_1^2 + m_2^2 + s) x_2 - (m_1^4 - 2 m_2^4 + m_1^2 (2 m_2^2 - 2 m_3^2 - 3 s) +
2 s (m_3^2 + s)) x_3 + 2 (m_1^2 - s) x_4\Big\}\eean
%
\bean
g_5&=&\Big\{0, -m_1^2 m_2^2 (m_1^2 - s) (m_1^2 - 2 m_2^2 + s) x_1 x_2 -
m_2^2 (m_1^6 + 2 m_1^4 (m_3^2 - 2 s) +
2 (2 m_2^6 + m_2^4 (2 m_3^2 - 3 s)\nn&& - m_2^2 s^2 + s^2 (-m_3^2 + s)) +
m_1^2 (-6 m_2^4 + (4 m_3^2 - 3 s) s + m_2^2 (-8 m_3^2 + 14 s))) x_2^2 - m_1^2 (m_1^2 - s) (-2 m_2^4 +
m_1^2 (m_2^2 \nn&&+ m_3^2 - s) + (3 m_3^2 - s) s +
m_2^2 (-4 m_3^2 + 3 s)) x_1 x_3 -
m_1^2 (m_1^2 - s) (-2 m_2^4 + m_1^2 (m_2^2 + m_3^2 - s) + (3 m_3^2 - s) s \nn
%
&&+
m_2^2 (-4 m_3^2 + 3 s)) x_2 x_3,
m_1^2 m_2^2 (m_1^2 - s) (m_1^2 - 2 m_2^2 + s) x_1 x_2 +
m_1^2 (m_1^2 - s) (-2 m_2^4 + m_1^2 (m_2^2 + m_3^2 - s) \nn&&+ (3 m_3^2 - s) s +
m_2^2 (-4 m_3^2 + 3 s)) x_1 x_3 +
m_1^2 m_2^2 (m_1^2 - s) (m_1^2 - 2 m_2^2 + s) x_2 x_3 + (m_1^6 (m_2^2 + m_3^2 - s) \nn&&+ 2 (m_2^2 - s) (2 m_2^6 + m_2^4 (2 m_3^2 - s) + 2 m_2^2 (m_3^2 - s) s +
s^2 (-3 m_3^2 + s)) - 2 m_1^4 (-s^2 + m_2^2 (m_3^2 + s)) \nn&&+
m_1^2 (-6 m_2^6 + m_2^2 (16 m_3^2 - 9 s) s + s^2 (-7 m_3^2 + s) +
m_2^4 (-8 m_3^2 + 14 s))) x_3^2, -m_1^2 m_2^2 (m_1^2 - s) (m_1^4 + 2 m_2^4 \nn&&+
m_1^2 (-3 m_2^2 + 3 m_3^2 - 2 s) + (7 m_3^2 - 3 s) s +
5 m_2^2 (-2 m_3^2 + s)) x_1 x_2 +
2 m_2^4 (m_2^2 - s) (m_1^4 \nn
%
&&+ 2 m_2^4 + m_2^2 (2 m_3^2 - s) - 2 m_3^2 s +
m_1^2 (-3 m_2^2 + s)) x_2^2 +
m_1^2 (m_1^2 - s) (-2 m_2^6 + m_1^4 (m_2^2 + m_3^2 - s) - 9 m_3^4 s \nn&&+ s^3 +
m_2^4 (-2 m_3^2 + 5 s) -
m_1^2 (m_2^4 + 3 m_3^4 + m_2^2 (4 m_3^2 - s) - 5 m_3^2 s) +
4 m_2^2 (3 m_3^4 - s^2)) x_1 x_3 +
2 m_2^2 (m_1^6 (m_2^2 - s) \nn&&-
2 m_1^4 (m_2^4 + m_3^4 + m_2^2 (m_3^2 - s) - 2 m_3^2 s) +
2 m_3^2 s ((3 m_3^2 - s) s + m_2^2 (-4 m_3^2 + 2 s)) \nn&&+
m_1^2 (2 m_2^4 s + m_2^2 (8 m_3^4 - 2 m_3^2 s - 3 s^2) +
s (-4 m_3^4 - 2 m_3^2 s + s^2))) x_2 x_3 -
2 m_3^2 (2 m_2^8 + m_1^4 (m_2^4 + m_2^2 (m_3^2 - 2 s) \nn&&+ (m_3^2 - s)^2) +
m_2^6 (2 m_3^2 - 3 s) - m_2^4 s^2 - s^2 (3 m_3^4 - 4 m_3^2 s + s^2) +
m_2^2 s (4 m_3^4 - 7 m_3^2 s + 3 s^2) \nn&&-
m_1^2 (3 m_2^6 + m_2^4 (4 m_3^2 - 6 s) + 2 m_3^2 s (-m_3^2 + s) +
m_2^2 (4 m_3^4 - 8 m_3^2 s + 3 s^2))) x_3^2 +
2 m_1^2 (m_1^2 - s) (m_2^2 (4 m_3^2 - 2 s) \nn&&+ s (-3 m_3^2 + s) +
m_1^2 (-m_3^2 + s)) x_1 x_4 -
m_2^2 (m_1^2 - s) (m_1^4 + m_1^2 (-2 m_2^2 + 4 m_3^2 - 3 s) -
4 (m_2^2 (4 m_3^2 - 2 s) \nn&&+ s (-3 m_3^2 + s))) x_2 x_4 + (m_1^6 (m_2^2 + m_3^2 - s) -
2 m_1^4 (m_2^4 + 2 m_3^4 + m_2^2 (2 m_3^2 - s) - 3 m_3^2 s) +
4 m_3^2 s ((3 m_3^2 - s) s \nn&&+ m_2^2 (-4 m_3^2 + 2 s)) +
m_1^2 (2 m_2^4 s + m_2^2 (16 m_3^4 - 4 m_3^2 s - 3 s^2) +
s (-8 m_3^4 - 3 m_3^2 s + s^2))) x_3 x_4 \nn&&+
2 (m_1^2 - s) (m_2^2 (4 m_3^2 - 2 s) + s (-3 m_3^2 + s) +
m_1^2 (-m_3^2 + s)) x_4^2,
m_2^2 (m_1^6 + 4 m_2^6 + m_1^4 (4 m_3^2 - 6 s) \nn
&&+ m_2^4 (4 m_3^2 - 6 s) +
2 m_2^2 (4 m_3^2 - 3 s) s + 4 s^2 (-2 m_3^2 + s) +
m_1^2 (-6 m_2^4 - 2 m_2^2 (8 m_3^2 - 9 s) + (8 m_3^2 - 3 s) s)) x_2 \nn&&+ (-m_1^6 (m_2^2 + m_3^2 - s) +
2 m_1^4 (m_3^4 - m_3^2 s - s^2 + m_2^2 (m_3^2 + s)) +
2 (-2 m_2^8 + m_2^4 s^2 + m_2^6 (-2 m_3^2 + 3 s)\nn&& +
m_2^2 s (4 m_3^4 + 3 m_3^2 s - 3 s^2) +
s^2 (-3 m_3^4 - 2 m_3^2 s + s^2)) +
m_1^2 (6 m_2^6 + 2 m_2^4 (4 m_3^2 - 7 s) \nn&&+
s (4 m_3^4 + 7 m_3^2 s - s^2) +
m_2^2 (-8 m_3^4 - 12 m_3^2 s + 9 s^2))) x_3 -
2 (m_1^2 - s) (m_2^2 (4 m_3^2 - 2 s)\nn&& + s (-3 m_3^2 + s) +
m_1^2 (-m_3^2 + s)) x_4\Big\}\nn
%
g_6&=&\Big\{0, -m_1^4 x_1 (x_1 + x_2) + m_2^4 x_2 (x_1 + x_2) +
m_2^2 (-s x_2 (3 x_1 + x_2) +
m_3^2 (x_2 x_3 + x_1 (2 x_2 + x_3))) + s (x_1 + x_2) x_4 \nn&&-
m_1^2 (-s x_1 (x_1 + 3 x_2) +
m_3^2 (x_2 x_3 + x_1 (2 x_2 + x_3)) + (x_1 + x_2) x_4),
m_1^4 x_1 (x_1 + x_3) + m_2^4 (x_2 x_3 + x_1 (x_2 + 2 x_3)) \nn&&+
m_2^2 (m_3^2 x_3 (x_1 + x_3) -
s (x_2 x_3 + x_1 (x_2 + 4 x_3))) -
m_1^2 (x_1 + x_3) (s x_1 + m_3^2 x_3 - x_4) +
s (2 s x_1 x_3 - (x_1 + x_3) x_4),\nn&& -m_1^2 (m_1^2 - s) (m_1^2 -
5 m_2^2 + 3 m_3^2 + s) x_1^2 -
m_2^2 (m_2^2 - s) (-5 m_1^2 + m_2^2 + 3 m_3^2 + s) x_1 x_2 +
m_3^2 (-3 m_1^4 - 3 m_2^4 \nn&&+ 2 (m_3^2 - s) s + m_2^2 (-m_3^2 + s) +
m_1^2 (6 m_2^2 - m_3^2 + s)) x_1 x[
3] + (-2 m_1^4 + m_1^2 (7 m_2^2 - 3 m_3^2 - s) +
s (-5 m_2^2 \nn&&+ 3 m_3^2 + s)) x_1 x_4 +
3 (m_2^4 - m_2^2 s) x_2 x_4 +
3 (-m_1^2 + m_2^2) m_3^2 x_3 x_4 - (m_1^2 - 2 m_2^2 + s) x_4^2, (m_1^4 - m_1^2 (4 m_2^2 - 4 m_3^2 \nn&&+ s) -
2 (m_2^4 + m_2^2 (m_3^2 - 4 s) + s (m_3^2 + s))) x_1 -
3 (m_2^4 - m_2^2 s) x_2 +
3 (m_1^2 - m_2^2) m_3^2 x_3 + (m_1^2 - 2 m_2^2 + s) x_4\Big\}\nonumber
%
\eea
\bea
%
g_7&=&\Big\{x_3 (m_3^2 x_3 + x_4),
2 s x_2 x_3 - m_1^2 (x_1 + x_2) x_3 -
m_3^2 x_3 (2 x_2 + x_3) - 2 x_2 x_4 - x_3 x_4,
x_3 (-m_3^2 x_3 + m_1^2 (x_1 + x_3) - x_4),\nn&&
m_1^2 (-m_1^2 + 3 m_2^2 - 3 m_3^2 + s) x_1 x_3 +
2 (m_2^4 - m_2^2 s) x_2 x_3 +
m_3^2 (-3 m_1^2 + 3 m_2^2 - m_3^2 + s) x_3^2 -
2 m_1^2 x_1 x_4 + (-2 m_1^2 \nn&&+ 3 m_2^2 - 3 m_3^2 + s) x_3 x_4 -
2 x_4^2, (m_1^2 - 2 (m_2^2 - 2 m_3^2 + s)) x_3 + 4 x_4\Big\}\nn
%
g_8&=&\Big\{x_3 (-m_2^4 x_2 + m_2^2 (s x_2 - m_3^2 x_3 - x_4) +
m_1^2 (m_3^2 x_3 + x_4)), -m_1^4 (x_1 + x_2) x_3 +
m_2^2 (m_2^2 x_2 x_3 - 3 s x_2 x_3 \nn&&+ m_3^2 x_3 (2 x_2 + x_3) +
2 x_2 x_4 + x_3 x_4) +
m_1^2 (s (x_1 + 3 x_2) x_3 - (2 x_2 + x_3) (m_3^2 x_3 + x_4)),
\nn&&x_3 (2 s^2 x_3 + m_1^4 (x_1 + x_3) + m_2^4 (x_2 + 2 x_3) -
m_1^2 (m_3^2 x_3 + s (x_1 + x_3) + x_4) +
m_2^2 (m_3^2 x_3 - s (x_2 + 4 x_3) + x_4)),\nn&& -m_1^2 (m_1^2 -
s) (m_1^2 - 5 m_2^2 + 3 m_3^2 + s) x_1 x_3 -
m_2^2 (m_2^2 - s) (-5 m_1^2 + m_2^2 + 3 m_3^2 + s) x_2 x_3 +
m_3^2 (-3 m_1^4 - 3 m_2^4 \nn&&+ 2 (m_3^2 - s) s + m_2^2 (-m_3^2 + s) +
m_1^2 (6 m_2^2 - m_3^2 + s)) x_3^2 - 2 (m_1^4 - m_1^2 s) x_1 x_4 -
2 (m_2^4 - m_2^2 s) x_2 x_4 \nn&&- (2 m_1^4 + m_2^4 + m_1^2 (-6 m_2^2 + 3 m_3^2) - 4 m_3^2 s +
m_2^2 (m_3^2 + 3 s)) x_3 x_4 -
2 (m_1^2 - s) x_4^2, (m_1^4 - m_1^2 (4 m_2^2 - 4 m_3^2 + s) \nn&&-
2 (m_2^4 + m_2^2 (m_3^2 - 4 s) + s (m_3^2 + s))) x_3 + (4 m_1^2 - 2 (m_2^2 + s)) x_4\Big\}\nn
%
g_9&=&\Big\{m_2^4 x_2 (-x_2 + x_3) +
m_2^2 (x_2 - x_3) (s x_2 - m_3^2 x_3 - x_4) + s x_2 x_4 -
m_1^2 x_3 (m_3^2 (-x_2 + x_3) + x_4),
m_2^4 x_2 (x_2 - x_3) - m_1^4 (x_1 + x_2) (x_2 - x_3) \nn&&+
m_1^2 (s (x_1 + 3 x_2) (x_2 - x_3) - (2 x_2 +
x_3) (m_3^2 (x_2 - x_3) - x_4)) - s x_2 x_4 -
m_2^2 (3 s x_2 (x_2 - x_3) +
m_3^2 (-2 x_2^2 \nn&&+ x_2 x_3 + x_3^2) + (x_2 + x_3) x_4),
m_1^4 (x_2 - x_3) (x_1 + x_3) +
m_2^4 (x_2^2 + x_2 x_3 - 2 x_3^2) +
m_2^2 (m_3^2 (x_2 - x_3) x_3
\eean
\bean
&&-
s (x_2^2 + 3 x_2 x_3 - 4 x_3^2) + (x_2 + x_3) x_4) +
s (2 s (x_2 - x_3) x_3 - (x_2 + 2 x_3) x_4) +
m_1^2 (-s (x_2 - x_3) (x_1 + x_3) \nn&&+
x_3 (m_3^2 (-x_2 + x_3) + x_4)), -m_1^2 (m_1^2 - s) (m_1^2 -
5 m_2^2 + 3 m_3^2 + s) x_1 x_2 -
m_2^2 (m_2^2 - s) (-5 m_1^2 \nn&&+ m_2^2 + 3 m_3^2 + s) x_2^2 +
m_1^2 (m_1^2 - s) (m_1^2 - 5 m_2^2 + 3 m_3^2 + s) x_1 x_3 - (-m_2^6 + 3 m_1^4 m_3^2 \nn&&+ 2 m_3^2 s (-m_3^2 + s) +
m_2^2 (m_3^2 + s)^2 +
m_1^2 (5 m_2^4 + m_3^4 - m_3^2 s - m_2^2 (6 m_3^2 + 5 s))) x_2 x_3 \nn&&+
m_3^2 (3 m_1^4 + 3 m_2^4 + m_2^2 (m_3^2 - s) +
m_1^2 (-6 m_2^2 + m_3^2 - s) + 2 s (-m_3^2 + s)) x_3^2 +
4 (m_1^4 - m_1^2 s) x_1 x_4 \nn&&+ (-m_1^4 + m_2^4 + s (3 m_3^2 + s) - m_2^2 (m_3^2 + 6 s) +
m_1^2 (7 m_2^2 - 2 (m_3^2 + s))) x_2 x_4 + (2 m_1^4 + m_2^4 + m_1^2 (-6 m_2^2 + 5 m_3^2) \nn&&-
m_2^2 (m_3^2 - 3 s) - 4 m_3^2 s) x_3 x_4 +
4 (m_1^2 - s) x_4^2, (m_1^4 - m_1^2 (4 m_2^2 - 4 m_3^2 + s) -
2 (m_2^4 + m_2^2 (m_3^2 - 4 s) + s (m_3^2 + s))) x_2 \nn&&+ (-m_1^4 + m_1^2 (4 m_2^2 - 4 m_3^2 + s) +
2 (m_2^4 + m_2^2 (m_3^2 - 4 s) + s (m_3^2 + s))) x_3 -
6 (m_1^2 - s) x_4\Big\}\nn
%
g_{10}&=&\Big\{x_1, x_2, x_3, x_4, -3\Big\}
%
\eean
%
%
\section{The coefficients of sunset}
\subsection{The analytical coefficients of sunset example: $I_3^{(0,0)}(3,1,1)$}
The reduction result is
%
\bea
%
I_3^{(0,0)}(3,1,1)&=&c_{311\to211}I_3^{(0,0)}(2,1,1)+c_{311\to121}I_3^{(0,0)}(1,2,1)+c_{311\to112}I_3^{(0,0)}(1,1,2)+c_{311\to111}I_3^{(0,0)}(1,1,1)+\cdots~~~~
%
\eea
with the coefficients
%
\bea
%
c_{311\to211}&=&\frac{n_{311\to211}}{N_{311\to211}},~~c_{311\to121}=\frac{n_{311\to121}}{N_{311\to121}},~~c_{311\to112}=\frac{n_{311\to112}}{N_{311\to112}},~~c_{311\to111}=\frac{n_{311\to111}}{N_{311\to111}}
%
\eea
%
where
%
\bean
%
n_{311\to211}&=&2 m_1^4 (2 m_2^2 ((7 D-22) m_3^2+(D-4) s)+(5 D-18) m_2^4+2 (D-4) m_3^2 s+(5 D-18) m_3^4+3 (3 D-10) s^2)\nn&&-4 m_1^2 (m_2^2 (2 (8 D-29) m_3^2 s-(D-4) m_3^4+(10-3 D) s^2)+m_2^4 (s-(D-4) m_3^2)+(D-4) m_2^6\nn&&+(m_3^2-s)^2 ((D-4) m_3^2+(2 D-7) s))-4 m_1^6 ((3 D-10) m_2^2+(3 D-10) m_3^2+(4 D-13) s)\nn&&+(5 D-16) m_1^8+(D-4) (-2 m_2^2 (m_3^2+s)+m_2^4+(m_3^2-s)^2)^2\nn
%
N_{311\to211}&=&4 m_1^2 (m_1^2+2 m_1 (m_2-m_3)+m_2^2-2 m_2 m_3+m_3^2-s) (m_1^2-2 m_1 m_2+2 m_1 m_3\nn&&+m_2^2-2 m_2 m_3+m_3^2-s) (m_1^2-2 m_1 (m_2+m_3)+m_2^2+2 m_2 m_3+m_3^2-s) (m_1^2+2 m_1 (m_2+m_3)\nn&&+m_2^2+2 m_2 m_3+m_3^2-s)\nn
%
\eean
%
\bea
n_{311\to121}&=&(D-3) m_2^2 (-2 m_1^2 (m_2^2-s) (m_2^2-3 m_3^2+s)+m_1^4 (m_2^2-s)+m_2^2 (m_3^4+3 s^2)\nn&&-m_2^4 (2 m_3^2+3 s)+m_2^6-s (m_3^2-s)^2)\nn
%
N_{311\to121}&=&m_1^2 (m_1^2+2 m_1 (m_2-m_3)+m_2^2-2 m_2 m_3+m_3^2-s) (m_1^2-2 m_1 m_2+2 m_1 m_3\nn&&+m_2^2-2 m_2 m_3+m_3^2-s) (m_1^2-2 m_1 (m_2+m_3)+m_2^2+2 m_2 m_3+m_3^2-s) (m_1^2+2 m_1 (m_2+m_3)\nn&&+m_2^2+2 m_2 m_3+m_3^2-s)\nn
%
n_{311\to112}&=&(D-3) m_3^2 (2 m_1^2 (m_3^2-s) (3 m_2^2-m_3^2-s)+m_1^4 (m_3^2-s)-2 m_2^2 (m_3^4-s^2)+m_2^4 (m_3^2-s)+(m_3^2-s)^3)\nn
%
N_{311\to112}&=&m_1^2 (m_1^2+2 m_1 (m_2-m_3)+m_2^2-2 m_2 m_3+m_3^2-s) (m_1^2-2 m_1 m_2+2 m_1 m_3\nn&&+m_2^2-2 m_2 m_3+m_3^2-s) (m_1^2-2 m_1 (m_2+m_3)+m_2^2+2 m_2 m_3+m_3^2-s) (m_1^2+2 m_1 (m_2+m_3)+m_2^2\nn&&+2 m_2 m_3+m_3^2-s)\nn
%
n_{311\to111}&=&(3-D) (3 d-8) (-m_1^2 (2 m_2^2 (s-5 m_3^2)+m_2^4+2 m_3^2 s+m_3^4-3 s^2)\nn&&-m_1^4 (m_2^2+m_3^2+3 s)+m_1^6+((m_2-m_3)^2-s) (m_2^2+m_3^2-s) ((m_2+m_3)^2-s))\nn
%
N_{311\to111}&=&4 m_1^2 ((m_1+m_2-m_3)^2-s) ((m_1-m_2\nn&&+m_3)^2-s) ((-m_1+m_2+m_3)^2-s) ((m_1+m_2+m_3)^2-s)
%
\eea
%
%
%
%
\\
\\
\\
\subsection{The analytical coefficients of sunset example: $I_3^{(0,0)}(2,2,1)$}
%
The reduction result is
%
\bea
%
I_3^{(0,0)}(2,2,1)&=&c_{221\to211}I_3^{(0,0)}(2,1,1)+c_{221\to121}I_3^{(0,0)}(1,2,1)+c_{221\to112}I_3^{(0,0)}(1,1,2)+c_{221\to111}I_3^{(0,0)}(1,1,1)+\cdots ~~~~~
%
\eea
%
%
with the coefficients
%
\bea
%
c_{221\to211}&=&\frac{n_{221\to211}}{N_{221\to211}},~~c_{211\to121}=\frac{n_{221\to121}}{N_{221\to121}},~~c_{221to112}=\frac{n_{221\to112}}{N_{221\to121}},~c_{221\to111}=\frac{n_{221\to111}}{N_{221\to111}}
%
\eea
%
where
%
\bea
%
n_{221\to211}&=&(D-3) (-m_1^2 (2 m_2^2 (m_3^2-3 s)+5 m_2^4+14 m_3^2 s-7 m_3^4+s^2)+m_1^4 (7 m_2^2-3 m_3^2+5 s)\nn&&-3 m_1^6+m_2^2 (2 m_3^2 s+3 m_3^4+3 s^2)-3 m_2^4 (m_3^2+s)+m_2^6-(m_3^2-s)^2 (m_3^2+s))\nn
%
N_{221\to211}&=&((m_1+m_2-m_3)^2-s) ((m_1-m_2+m_3)^2-s) ((-m_1+m_2+m_3)^2-s) ((m_1+m_2+m_3)^2-s)\nn
%
n_{211\to121}&=&(D-3) (m_1^2 (-2 m_2^2 (m_3^2-3 s)+7 m_2^4+2 m_3^2 s+3 m_3^4+3 s^2)-m_1^4 (5 m_2^2+3 (m_3^2+s))\nn&&+m_1^6+m_2^2 (-14 m_3^2 s+7 m_3^4-s^2)+m_2^4 (5 s-3 m_3^2)-3 m_2^6-(m_3^2-s)^2 (m_3^2+s))\nn
%
N_{211\to121}&=&((m_1+m_2-m_3)^2-s) ((m_1-m_2+m_3)^2-s) ((-m_1+m_2+m_3)^2-s) ((m_1+m_2+m_3)^2-s)\nn
%
n_{221\to112}&=&8 (D-3) m_3^2 (m_3^2-s) (-m_1^2-m_2^2+m_3^2+s)\nn
%
N_{221\to112}&=&((m_1+m_2-m_3)^2-s) ((m_1-m_2+m_3)^2-s) ((-m_1+m_2+m_3)^2-s) ((m_1+m_2+m_3)^2-s)\nn
%
n_{221\to111}&=&(D-3) (3 D-8) (-2 m_1^2 (m_2^2-m_3^2+s)+m_1^4+2 m_2^2 (m_3^2-s)+m_2^4+2 m_3^2 s-3 m_3^4+s^2)\nn
%
N_{221\to111}&=&(m_1^2+2 m_1 (m_2-m_3)+m_2^2-2 m_2 m_3+m_3^2-s) (m_1^2-2 m_1 m_2+2 m_1 m_3+m_2^2-2 m_2 m_3+m_3^2\nn&&-s) (m_1^2-2 m_1 (m_2+m_3)+m_2^2+2 m_2 m_3+m_3^2-s) (m_1^2+2 m_1 (m_2+m_3)+m_2^2+2 m_2 m_3+m_3^2-s)~~~
%
\eea
%
\\
\\
\\
\\
\\
\\
\\
\\
\\
\\
\\
\\
\\
\\
\\
\\
%
\subsection{The coefficients of $i(\lam_0-2;0,2,2)$}
%
\bea
%
c_{i;022\to000}&=&\frac{4 (3 D-1)}{3 (D+2) s (2 (m_{2}^2+s)-m_{1}^2)}\nn
%
c_{i;022\to100}&=&\frac{4 (1-3 D) m_{1}^2}{3 (D+2) s (m_{1}^2-2 (m_{2}^2+s))}\nn
%
c_{i;022\to010}&=&\frac{1}{3 (D+2) s (2 (m_{2}^2+s)-m_{1}^2) (m_{1}^2 (2 m_{2}^2-3 s)+m_{1}^4-2
	(m_{2}^2-s) (m_{2}^2-m_{3}^2+s))}\times \Big\{D (6 m_{1}^2 (m_{2}^2 (3 m_{3}^2\nn&&-23 s)+8 m_{2}^4-3 s (m_{3}^2-3 s))+45 m_{1}^4 m_{2}^2-9 m_{1}^6-12 (-8 m_{2}^2 s+5
m_{2}^4+3 s^2) (m_{2}^2-m_{3}^2+s))\nn&&-2 (m_{2}^2-s) (3 m_{1}^2+2 m_{2}^2-6 s) (5 m_{1}^2-4
(m_{2}^2-m_{3}^2+s))\Big\}\nn
%
c_{i;022\to001}&=&\frac{1}{3 (D+2) s
	(2 (m_{2}^2+s)-m_{1}^2) (m_{1}^2 (2 m_{2}^2-3 s)+m_{1}^4-2 (m_{2}^2-s) (m_{2}^2-m_{3}^2+s))}\times\Big\{m_{1}^4 ((22-27 D) m_{2}^2\nn&&+2 (15 D-8) m_{3}^2+2 (11-6 D) s)+2 m_{1}^2 (m_{2}^2 ((15 D-8) m_{3}^2-6 D s)+2 (6 D+1) m_{2}^4\nn&&+2 s ((8-15
D) m_{3}^2+(6 D-11) s))+(3 D-4) m_{1}^6-4 (m_{2}^2-s) (m_{2}^2-m_{3}^2+s) ((15 D-8) m_{3}^2\nn&&-6 D s+4 m_{2}^2+8 s)\Big\}\nn
%
%
%
%
c_{i;022\to110}&=&\frac{(3 D-2) m_{1}^2 (3 m_{1}^2+2 m_{2}^2-6 s) (m_{1}^2 (2 m_{3}^2-5 m_{2}^2)+m_{1}^4+(3 m_{2}^2-s)
	(m_{2}^2-m_{3}^2+s))}{3 (D+2) s (m_{1}^2-2 (m_{2}^2+s)) (m_{1}^2 (2 m_{2}^2-3 s)+m_{1}^4-2
	(m_{2}^2-s) (m_{2}^2-m_{3}^2+s))}\nn
%
c_{i;022\to101}&=&-\frac{1}{3 (D+2)
	s (m_{1}^2-2 (m_{2}^2+s)) (m_{1}^2 (2 m_{2}^2-3 s)+m_{1}^4-2 (m_{2}^2-s) (m_{2}^2-m_{3}^2+s))}\times \Big\{(3 D\nn&&-2) m_{1}^2 (m_{1}^2 (5 m_{2}^2 m_{3}^2-3 m_{2}^4+3 s (m_{3}^2+5 s))-3 m_{1}^4 (m_{2}^2+4 s)+3 m_{1}^6+2
(m_{2}^2 (2 m_{3}^2 s\nn&&+6 m_{3}^4-3 s^2)+m_{2}^4 (3 s-9 m_{3}^2)+3 m_{2}^6-3 s (-m_{3}^2 s+2 m_{3}^4+s^2)))\Big\}\nonumber
\eea
\bea
c_{i;022\to011}&=&\frac{1}{3 (D+2) s (2 (m_{2}^2+s)-m_{1}^2) (m_{1}^2 (2 m_{2}^2-3 s)+m_{1}^4-2 (m_{2}^2-s)
	(m_{2}^2-m_{3}^2+s))}\times \Big\{(3 D-2) (2 m_{1}^2 (m_{2}^2 \nn&&(-23 m_{3}^2 s+3 m_{3}^4+13 s^2)+m_{2}^4 (4 m_{3}^2+6 s)+m_{2}^6-3 s (-3 m_{3}^2 s+m_{3}^4+4
s^2))+m_{1}^6 (7 m_{2}^2-3 (3 m_{3}^2+s))\nn&&+m_{1}^4 (m_{2}^2 (14 m_{3}^2-25 s)-10 m_{2}^4+15 s
(m_{3}^2+s))+4 (m_{2}^2-s) (m_{2}^2 (-4 m_{3}^2 s+5 m_{3}^4-s^2)+m_{2}^4 (3 s-6 m_{3}^2)\nn&&+m_{2}^6-3 s
(m_{3}^2-s)^2))\Big\}\nn
%
c_{i;022\to200}&=&0\nn
%
c_{i;022\to020}&=&\frac{(3 D-2) m_{2}^2 (m_{2}^2-s) (3 m_{1}^2+2 m_{2}^2-6 s) (5 m_{1}^2-4 (m_{2}^2-m_{3}^2+s))}{3 (D+2) s (2
	(m_{2}^2+s)-m_{1}^2) (m_{1}^2 (2 m_{2}^2-3 s)+m_{1}^4-2 (m_{2}^2-s) (m_{2}^2-m_{3}^2+s))}\nn
%
c_{i;022\to002}&=&-\frac{1}{3 (D+2) s (2
	(m_{2}^2+s)-m_{1}^2) (m_{1}^2 (2 m_{2}^2-3 s)+m_{1}^4-2 (m_{2}^2-s) (m_{2}^2-m_{3}^2+s))}\times \Big\{(3 D-2) m_{3}^2 (m_{1}^4 \nn&&(13 m_{2}^2-6 m_{3}^2+3 s)+m_{1}^2 (2 m_{2}^2 s-6 m_{2}^4+6 s (m_{3}^2-2 s))\nn&&-4 (m_{2}^2
(-2 m_{3}^2 s+3 m_{3}^4-s^2)+m_{2}^4 (3 s-4 m_{3}^2)+m_{2}^6-3 s (m_{3}^2-s)^2))\Big\}\nonumber
%
\eea
%
\nocite{}
\bibliographystyle{JHEP}


\bibliography{reference2}%
%%%%%%%%%%%%%%%%%%%%%%%%%%%%%%%%%%%%%%%%%%%%%
\end{document}
