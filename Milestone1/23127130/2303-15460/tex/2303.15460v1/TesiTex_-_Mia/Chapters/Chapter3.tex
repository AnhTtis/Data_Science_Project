% Chapter 3

\chapter{Second-order ordinary differential equation}\label{ch 3} % Main chapter title

\label{Chapter3} % For referencing the chapter elsewhere, use \ref{Chapter3} 

%----------------------------------------------------------------------------------------

% Define some commands to keep the formatting separated from the content 

%----------------------------------------------------------------------------------------

\section{Variational formulation for $\partial_{tt}u+ \mu u=f$}\label{sect var form}
As a model problem we consider the following second-order linear equation: 
\begin{equation}\label{eq ode}
\partial_{tt}u(t)+ \mu u(t)=f(t), \quad \text{for} \ t \in (0,T), \quad u(0)=\partial_{t}u(t)_{|t=0}=0,
\end{equation}
where $\mu >0$.

The variational formulation of \eqref{eq ode} reads as follows:
\begin{equation}\label{var ode}
\begin{cases}
\text{Find} \ u \in H^1_{0,*}(0,T) \quad \text{such that}\\
a(u,v)=\langle f,v \rangle_{(0,T)} \quad  \forall v \in H^1_{*,0}(0,T),
\end{cases}
\end{equation}
where $T>0$ and $f \in [H^1_{*,0}(0,T)]' $ are given, and where the bilinear form $a(\cdot,\cdot): H^1_{0,*}(0,T) \times H^1_{*,0}(0,T) \longrightarrow \mathbb{R}$ is 
\begin{equation}\label{bil ode}
a(u,v):=-\langle \partial_{t}u, \partial_{t}v \rangle_{L^2(0,T)}+ \mu \langle u,v \rangle_{L^2(0,T)},
\end{equation} 
for all $u \in H^1_{0,*}(0,T)$, $v \in H^1_{*,0}(0,T)$. The notation $\langle \cdot,\cdot \rangle_{(0,T)}$ denotes the duality pairing as extension of the inner product in $L^2(0,T)$, and the Sobolev spaces $H^1_{0,*}(0,T), H^1_{*,0}(0,T)$ are introduced in Section \ref{sec sobolev}. Note that the first initial condition $u(0)=0$ is incorporated in the solution space $H^1_{0,*}(0,T)$, whereas the second initial condition $\partial_{t}u(t)_{|t=0}=0$ is considered as a natural
condition in the variational formulation.

Thanks to the Cauchy-Schwarz inequality and the Poincaré inequality \eqref{Poinc}, the bilinear form $a(\cdot,\cdot)$ is bounded with
\begin{equation}\label{cont of a}
|a(u,v)| \leq \Bigg(1+\frac{4 T^2 \mu }{\pi^2} \Bigg) |u|_{H^1(0,T)}|v|_{H^1(0,T)},
\end{equation}
for all $(u,v) \in H^1_{0,*}(0,T) \times H^1_{*,0}(0,T)$. 

In order to prove the well-posedness of \eqref{var ode} we consider an equivalent variational problem. Thus, we introduce the isomorphism
\begin{gather}\label{HT}
\overline{\mathcal{H}}_T: H^1_{0,*}(0,T) \longrightarrow H^1_{*,0}(0,T)\\
u \mapsto u(T)-u(\cdot),
\end{gather}
where its inverse is given by
\begin{gather*}
\overline{\mathcal{H}}_T^{-1}: H^1_{*,0}(0,T) \longrightarrow H^1_{0,*}(0,T)\\
v \mapsto v(0)-v(\cdot).
\end{gather*}
Since the above mappings are actually isometries with respect to $|\cdot|_{H^1(0,T)}$, then the well-posedness of  \eqref{var ode} is equivalent to the well-posedness (with the same stability constant) of the following problem
\begin{equation}\label{var ode equiv}
\begin{cases}
\text{Find} \ u \in H^1_{0,*}(0,T) \ \text{such that}\\
a(u,\overline{\mathcal{H}}_Tv)=\langle f,\overline{\mathcal{H}}_Tv \rangle_{(0,T)} \quad  \forall v \in H^1_{0,*}(0,T),
\end{cases}
\end{equation}
where the solution and test space coincide. Note that the continuity of the bilinear form in \eqref{var ode equiv} is an immediate consequence of estimate \eqref{cont of a}. Unfortunately, at least for $\mu$ sufficiently large, the bilinear form of \eqref{var ode equiv} is not coercive, hence, we cannot rely on the classical Lax-Milgram Lemma \ref{lax-milgram}, but we need more specific tools. The branch of functional analysis that we exploit is called \textit{Fredholm theory} and studies compact perturbations of linear bounded operators. In particular, there holds the following Theorem.

\begin{theorem}[\cite{Zank2020}]\label{zank}
	For a given $f \in [H^1_{*,0}(0,T)]'$, there exists a unique solution $u \in H^1_{0,*}(0,T)$ of the variational formulation \eqref{var ode equiv}, and the following a priori estimate holds
	\begin{equation}\label{stab ode}
	|u|_{H^1(0,T)}\leq \Bigg(\frac{2+\sqrt{\mu}T}{2}\Bigg) \|f\|_{[H^1_{*,0}(0,T)]'}.
	\end{equation}
	In addition, \eqref{stab ode} is optimal with respect to the order of $\mu$ and $T$.
\end{theorem}
The well-posedness of \eqref{var ode equiv} is an immediate consequence of the results presented in Section \ref{sec prel ode}. Indeed, defining two bounded linear operators $\mathcal{B},\mathcal{D}:H^1_{0,*}(0,T) \rightarrow [H^1_{0,*}(0,T)]'$ by
\begin{gather*}
\langle \mathcal{B}u,v \rangle _{(0,T)} := \langle \partial_tu,\partial_t v\rangle_{L^2(0,T)}, \quad u,v \in H^1_{0,*}(0,T)\\
\langle \mathcal{D}u,v \rangle _{(0,T)} := \langle u,v(T)-v \rangle_{L^2(0,T)}, \quad u,v \in H^1_{0,*}(0,T),
\end{gather*}
 the map $\mathcal{B}+\mu \mathcal{D}: H^1_{0,*}(0,T) \rightarrow [H^1_{0,*}(0,T)]'$ is an injective Fredholm operator of index 0, where the compactness of $\mathcal{D}$ follows from the properties of trace operators. However, estimate \eqref{stab ode} gives an explicit dependency relation of its stability constant on $T,\mu$. In order to prove \eqref{stab ode}, we use the notation $\mathcal{A}$ for the bounded linear operator related to the bilinear form \eqref{bil ode}, i.e.,
\begin{gather*}
\mathcal{A}:  H^1_{0,*}(0,T) \rightarrow [H^1_{*,0}(0,T)]' \quad \text{s.t.}\\
\langle \mathcal{A}u,v \rangle _{(0,T)}:=a(u,v), \quad u \in H^1_{0,*}(0,T), v \in H^1_{*,0}(0,T).
\end{gather*}
Since $\mathcal{A}$ is an isomorphism, its adjoint operator $\mathcal{A}^*:H^1_{*,0}(0,T)\rightarrow  [H^1_{0,*}(0,T)]' $ is an isomorphism with $(\mathcal{A}^*)^{-1}=(A^{-1})^*$, see (\cite{adjoint}). As a consequence, given $g \in [H^1_{0,*}(0,T)]'$, the adjoint problem 
\begin{equation}\label{adj dim}
\begin{cases}
\text{Find} \ z \in H^1_{*,0}(0,T) \quad \text{such that}\\
a(w,z)=\langle g,w \rangle_{(0,T)} \quad \forall w \in H^1_{0,*}(0,T)
\end{cases}
\end{equation}
is well-posed. In particular, it is possible to compute the exact solution of problem \eqref{adj dim} using a Green's function, if the right-hand side  $g \in [H^1_{0,*}(0,T)]'$ depends on a fixed $u \in H^1_{0,*}(0,T)$; for more details we refer to (\cite{Zank2020}). For the optimality of the estimate we refer to Theorem 4.2.6 of (\cite{Zank2020}).

\section{Isogeometric discretization}\label{sec iga}
As discrete spaces for the Galerkin discretization of \eqref{var ode} we choose to consider spline spaces of degree two with maximal regularity, i.e., piecewise polynomials of degree two with global $C^1$ regularity.

Given a positive integer $N$, let $\Xi:=\{t_1, \ldots, t_{N+3}\}$ be an open knot vector in $[0,T]$ with the interior knots that appear just once, i.e., $0=t_1=t_2=t_3 < \ldots  <t_{N+1}=t_{N+2}=t_{N+3}=T$. By means of Cox-de Boor recursion formulas \eqref{cox de boor} we define the quadratic univariate B-spline basis functions ${b}_{i,2}: [0,T] \rightarrow \mathbb{R}$ for $i=1, \ldots, N$ (we omit the subscript $\Xi$ to lighten the notation). Thus, the univariate spline space is defined as 
\begin{equation*}
{S}^2_h:= \text{span}\{{b}_{i,2}\}_{i=1}^N,
\end{equation*}
where $h$ is the mesh-size, i.e., $h:=\max{\{|t_{i+1}-t_i|\ : \ i=1,\ldots,N+2}\}$. 

We introduce the spline space with initial conditions as
\begin{equation}\label{iga space}
\begin{split}
V^h_{0,*} &:= {S}^2_h \cap H^1_{0,*}(0,1)
= \{v_h \in {S}^2_h \ : \ {v}_h(0)=0 \}\\
&= \text{span}\{{b}_{i,2}\}_{i=2}^N,
\end{split}
\end{equation}
which is our isogeometric solution space, and the spline space 
\begin{equation}\label{iga test space}
\begin{split}
{V}^h_{*,0} &:= {S}^2_h \cap H^1_{*,0}(0,1)
= \{ {v}_h \in {S}^2_h \ : \ {v}_h(T)=0 \}\\
&= \text{span}\{ {b}_{i,2}\}_{i=1}^{N-1},
\end{split}
\end{equation}
which is our isogeometric test space. Note that $V^h_{0,*}=\text{span}\{{b}_{i,2}\}_{i=2}^N$  and ${V}^h_{*,0}=\text{span}\{ {b}_{i,2}\}_{i=1}^{N-1}$ since the first and last B-spline basis functions of an open knot vector are interpolating functions.

\begin{oss}
	Typically, in isogeometric discretizations, and in the GeoPDEs library that we are going to use for the numerical experiments, one considers the parametric domain $[0,1]^n$ and the splines/NURBS on this domain. Once these spaces are constructed, 
	one maps, via a NURBS function $F$,	
	the parametric domain to the physical domain of interest, \begin{minipage}[r]{0.6\linewidth} 
		\includegraphics[width=\textwidth]{Figures/isogeo}
	\end{minipage} \\ and the discrete solution and test spaces are the pushforward by $F$ of splines/NURBS spaces  on $[0,1]^n$. \\
	In our case, the map $F$ would be $F: [0,1] \rightarrow [0,T]$ s.t. $\tau \mapsto T\tau$. Therefore, doing the pushforward of the spline spaces on the parametric interval, we would obtain spline spaces on $[0,T]$ with B-spline basis functions that are the pushforward of the B-spline basis functions on $[0,1]$. It is then sufficient for us to construct the discrete spaces directly on $[0,T]$.
\end{oss}

A conforming Galerkin-Petrov isogeometric discretization
of \eqref{var ode} is the following problem
\begin{equation}\label{iga ode}
\begin{cases}
\text{Find} \ u_h \in V^h_{0,*} \quad \text{such that}\\
a(u_h,v_h)=\langle f,v_h \rangle_{(0,T)} \quad  \forall v_h \in V^h_{*,0}.
\end{cases}
\end{equation}
As in the continuous framework, the restricted operator
\begin{gather*}
\overline{\mathcal{H}}_{T_{|V^h_{0,*}}}: V^h_{0,*} \longrightarrow V^h_{*,0}\\
u_h \mapsto u_h(T)-u_h(\cdot)
\end{gather*}
is actually an isometric isomorphism with respect to $|\cdot|_{H^1(0,T)}$. Therefore, the well-posedness of \eqref{iga ode} is  equivalent to the well-posedness (with the same stability constant) of the conforming Galerkin-Bubnov isogeometric discretization of \eqref{var ode equiv}:
\begin{equation}\label{iga ode equiv}
\begin{cases}
\text{Find} \ u_h \in V^h_{0,*} \ \text{such that}\\
a(u_h,\overline{\mathcal{H}}_Tv_h)=\langle f,\overline{\mathcal{H}}_Tv_h \rangle_{(0,T)} \quad  \forall v_h \in V^h_{0,*}.
\end{cases}
\end{equation}
The isogeometric spaces $V^h_{0,*}$ define a dense discrete sequence in $H^1_{0,*}(0,T)$ (as we will see in Section \ref{approx properties spline}), directed in the real parameter $h$. Hence, we can conclude the following Theorem.

\begin{theorem}\label{cond well-posed iga}
	There exists two constants $C(T,\mu),\overline{h}>0$ such that, if $h\leq \overline{h}$, problem \eqref{iga ode equiv} is well-posed, with the stability estimate
	\begin{equation*}
	|u_h|_{H^1(0,T)} \leq C(T,\mu) \|f\|_{[H^1_{*,0}(0,T)]'} \quad f \in [H^1_{*,0}(0,T)]',
	\end{equation*}
	where $u_h$ is the unique solution of \eqref{iga ode equiv}.
	Moreover, if $h\leq \overline{h}$, a quasi-optimality estimate holds
	\begin{equation*}
	|u-u_h|_{H^1(0,T)} \leq \Big(1+C(T,\mu) \|\mathcal{A}\| \Big) \inf_{v_h \in V^h_{0,*}} |u-v_h|_{H^1(0,T)},
	\end{equation*}
	where $u \in H^1_{0,*}(0,T)$ is the unique solution of \eqref{var ode equiv}, and where $\|\cdot\|$ is the norm of the \textit{solution-to-data} operator $\mathcal{A}:H^1_{0,*}(0,T) \rightarrow [H^1_{0,*}(0,T)]'$, related to the bilinear form of the variational formulation \eqref{var ode equiv}.
\end{theorem}
\begin{proof}
	Theorem \ref{cond well-posed iga} is a straightforward consequence of the results presented in Section \ref{sec prel ode}, i.e., Proposition \ref{prop infsup comp}, Corollary \ref{well-pos comp dis} and Corollary \ref{céa comp}.
\end{proof}

\subsection{Approximation properties of spline spaces with an initial boundary condition}\label{approx properties spline}
So far we have shown that, if the IGA discretization is \textit{sufficiently fine}, problem \eqref{iga ode equiv} is well posed, stability holds and we eventually reach convergence. However, we would like to make explicit the threshold on the mesh-size and (possibly) the stability and quasi-optimality constants. In order to get these results we need a priori error estimate in the Sobolev semi-norm $|\cdot|_{H^1(0,T)}$
for approximation in spline spaces of maximal smoothness on grids defined by arbitrary break points. As pointed out in the paper (\cite{n-width}), classical error estimates for spline approximation are expressed in terms of:
\begin{itemize}
	\item [1.] A power of the mesh-size.
	\item [2.] An appropriate semi-norm of the function to be approximated.
	\item [3.] A constant which is independent of the previous quantities.
\end{itemize}

However, we are interested in estimates with the constant of point 3 made explicit. In this respect, article (\cite{n-width}) is relevant to us, since we slightly modify the construction of this work in order to obtain the estimates we need for test and trial spaces of our interest. In particular, the authors study the approximation properties of spline spaces, without boundary conditions, and of periodic spline spaces. We partially extend their work by including an initial boundary condition.

Let $S_\Xi^p(0,T)$ be the spline space of degree $p \geq 0$ and maximal regularity, where $\Xi$ is the open knot vector that defines the B-spline basis functions of $S_\Xi^p(0,T)$. We firstly observe that we use the terms \textit{knot vector} and \textit{break points} as introduced in Section \ref{sec splines}, unlike (\cite{n-width}) in which the sequence of break points is called knot vector. Let $Q_p^q: H^q(0,T) \rightarrow S_\Xi^p(0,T)$, $q=0,\ldots,p$, be a sequence of bounded linear operators such that 
\begin{gather}
Q^0_p:=P_p \quad \text{is the $L^2(0,T)$ orthogonal projection on} \ S^p_\Xi(0,T) \label{Qp0}, \\
\begin{split}
(Q^q_p&u)(t):=u(0) + \int_0^t(Q_{p-1}^{q-1}\partial_t u)(s) \ ds \label{Qpq},\\
& \quad \ 1 \leq q \leq p, \ \forall u \in H^q(0,T),
\end{split}
\end{gather}
%where $c(u)$ is a constant such that
%\begin{equation}
%\int_{0}^{T} (Q_p^qu)(t)dt = \int_0^T u(t) dt, 
%\end{equation}
%i.e., $c(u)$ is the mean of the function $u(\cdot)-\int_0^{(\cdot)}(Q_{p-1}^{q-1}\partial u)(s) ds $
%on the interval $[0,T]$. 
Firstly, let now observe the reason why the operators $Q^p_q$ maps $H^q(0,T)$ into $S^p_\Xi(0,T)$. Let $K$ be the integral operator such that, for $f \in L^2(0,T)$, 
\begin{equation}\label{K}
Kf(t):=\int_0^t f(s) \ ds.
\end{equation}
Recall from Theorem 17 of (\cite{chapter1}) that the spline space $S_\Xi^p(0,T)$ satisfies 
\begin{equation}\label{der spline}
\partial^q_tS_\Xi^p(0,T)=S_\Xi^{p-q}(0,T), \quad \text{for any} \ q=0,\ldots,p,
\end{equation}
where, by abuse of notation, we denote by the letter $\Xi$ both the open knot vector $\{t_1, \ldots, t_{N+p+1}\}$ (with $0=t_1=\ldots=t_{p+1} <\ldots  <t_{N+1}=\ldots =t_{N+p+1}=T$) and the one obtained from $\{t_1, \ldots, t_{N+p+1}\}$ by reducing the external nodes from $p+1$ to $p-q+1$. 
Thus, as a consequence of \eqref{der spline} and of the Fundamental Theorem of Calculus, there holds
\begin{equation}
S_\Xi^p(0,T)=\mathbb{P}_0 + K\Big(S_\Xi^{p-1}(0,T)\Big), \quad \forall p \geq 1
\end{equation}
where $\mathbb{P}_0$ is the space of constant functions. Hence, by an inductive argument, the range of $Q_p^q$ is a subspace of $S_\Xi^p(0,T)$. The linearity and continuity of $Q^p_q$ are straightforward.


\begin{prop}
	The maps $Q_p^q$ defined in \eqref{Qp0} and \eqref{Qpq} are projection operators with range $S_\Xi^p(0,T)$. They also satisfies
	\begin{equation}\label{commut Q}
	\partial_t Q_p^q = Q^{q-1}_{p-1}\partial_t, \quad \text{for all} \ p \geq 1, \ q=1,\ldots,p.
	\end{equation}
\end{prop}

\begin{proof} 
	Equality \eqref{commut Q} is satisfied by definition. 
	
	If $q=0$, $Q^q_p$ is a projection operator by definition. We use an inductive argument in order to prove that $Q^q_p\big(Q^q_pu \big) = Q^q_p u$ for all $u \in H^q(0,T)$ if $p \geq 1$ and $q=1,\ldots,p$. Let $p=q=1$, then
	\begin{equation*}
	\begin{split}
	Q^1_1\big(Q^1_1u \big)(t)&=Q^1_1\Big( u(0) +\int_0^{(\cdot)}(Q_0^0\partial_t u)(s) \ ds \Big)(t) \\
	&=Q^1_1 \big( u(0) \big) + Q^1_1\Big(\int_0^{(\cdot)}(Q_0^0\partial_t u)(s) \ ds \Big) (t) \quad \text{(linearity of $Q^p_q$)} \\
	&=u(0)+ \int_0^t Q^0_0 \Big(\partial_t \int_0^{(\cdot)}(Q_0^0\partial_t u)(s) \ ds \Big)(\tau) \ d\tau \quad \text{(definition of $Q^p_q$)} \\
	&\overset{\eqref{commut Q}}=u(0)+ \int_0^t Q^0_0 \Big(\partial_t \int_0^{(\cdot)}(\partial_t Q_1^1u)(s) \ ds \Big)(\tau) \ d\tau\\
	&=u(0)+ \int_0^t Q^0_0 \big(\partial_t Q^1_1u(\tau)-\partial_t(Q^1_1u(0)) \big) \ d\tau \quad \text{(Fundamental Th. of Calculus)}\\
	&\overset{\eqref{commut Q}}=u(0)+ \int_0^t Q^0_0 \big(Q^0_0\partial_t u(\tau)\big) \ d\tau \\
	&=u(0)+\int_0^t (Q^0_0 \partial_t u)(\tau) \ d\tau \quad \text{($Q^0_0$ is $L^2(0,T)$-projection)}.
	\end{split}
	\end{equation*}
    Let now assume that the statement is true for $p \geq 1, \ q=1,\ldots,p$, and let now consider $p+1, \ q=1,\ldots,p+1$. Hence, as before, there hold the following identities
	\begin{equation*}
	\begin{split}
	Q^q_{p+1}\big(Q^q_{p+1}u \big)(t)&=Q^q_{p+1}\Big( u(0) +\int_0^{(\cdot)}(Q_p^{q-1}\partial_t u)(s) \ ds \Big)(t) \\
	&=u(0)+ \int_0^t Q_p^{q-1} \Big(\partial_t \int_0^{(\cdot)}(Q_p^{q-1}\partial u)(s) \ ds \Big)(\tau) \ d\tau \\
	&=u(0)+\int_0^t (Q_p^{q-1} \partial_t u)(\tau) \ d\tau.
	\end{split}	
	\end{equation*}
If $q=0$, the range of $Q^q_p$ is clearly $S^p_\Xi(0,T)$. Indeed $Q^0_p$ is the $L^2(0,T)$-projection on $S^p_\Xi(0,T)$, hence, in particular, ${Q^0_p}_{|S^p_\Xi(0,T)} \equiv Id$. If $p \geq 1$ and $q=1,\ldots,p$, the fact that the range of $Q^q_p$ is $S^p_\Xi(0,T)$ is straightforward. Indeed, by using an inductive argument and \eqref{der spline}, one can prove that ${Q^q_p}_{|S^p_\Xi(0,T)} \equiv Id$ also for $p \geq 1$ and $q=1,\ldots,p$.
\end{proof}

Let now recall Theorem $1$ of (\cite{n-width}).

\begin{theorem}[E. Sande, C. Manni, H. Speleers]\label{sande,manni,speleers}
	For any sequence of break points that defines the open knot vector $\Xi$ with maximal regularity, let $h$ denote its maximal knot distance, and let $P_p$ denote the $L^2(0,T)$ orthogonal projection onto the spline space $S^p_\Xi(0,T)$. Then, for any function $u \in H^r(0,T)$ with $r \geq 1$,
	\begin{equation}\label{proj L2}
	\|u-P_pu\|_{L^2(0,T)} \leq \Big ( \frac{h}{\pi} \Big)^r |u|_{H^r(0,T)},
	\end{equation}
	for all $p \geq r-1$.
\end{theorem}

\begin{oss}
	Let $P$ be the $L^2(0,T)$ orthogonal projection onto a finite dimensional subspace $\mathcal{X}$ of $L^2(0,T)$. For $A \subseteq L^2(0,T)$ we define
	\begin{equation*}
	E(A,\mathcal{X})=\sup_{u \in A}{\|u-Pu\|_{L^2(0,T)}},
	\end{equation*}
	i.e., $E(A,\mathcal{X})$ is the ``maximal of the minimal distances'' between $A$ and the projection space $\mathcal{X}$.
	The \textit{Kolmogorov $L^2$ n-width} of $A$ is defined as 
	\begin{equation*}
	d_n(A)=\inf_{\mathcal{X} \subset L^2(0,T), dim\mathcal{X}=n} E(A,\mathcal{X}).
	\end{equation*}
	The projection space $\mathcal{X}$ is called an optimal subspace for $d_n(A)$ if $d_n(A)=E(A,\mathcal{X})$.
	
Let $r \geq 1$ and let $A=\{u \in H^r(0,T): \|\partial^r_t{u}\|_{L^2(0,T)} \leq 1 \}$. 
	Clearly, for any finite subspace $\mathcal{X}$ of $L^2(0,T)$ the following estimate holds
	\begin{equation*}
	\|u-Pu\|_{L^2(0,T)} \leq E(A,\mathcal{X})\|\partial^r_t{u}\|_{L^2(0,T)} \quad \forall u \in H^r(0,T),
	\end{equation*}
	and, in (\cite{n-width}), an error estimate of the form
	\begin{equation}\label{proj}
	\|u-Pu\|_{L^2(0,T)} \leq C\|\partial^r_t{u}\|_{L^2(0,T)} \quad \forall u \in H^r(0,T)
	\end{equation}
	is said to be \textit{sharp} if 
	\begin{equation*}
	C=E(A,\mathcal{X}).
	\end{equation*}
Also, in (\cite{n-width}), a projection error estimate of the form \eqref{proj} is said to be \textit{optimal} if the subspace we project onto is optimal for the Kolmogorov $L^2$ n-width of $A$ and the projection error estimate is sharp.

In (\cite{n-width}) the following results are proven.
\begin{itemize}
	\item If $r=1$ and $p=0$ and the sequence of break points that defines $\Xi$ is uniform, then the estimate \eqref{proj L2} is optimal.
	\item If $r=1$ and $p>0$ and the sequence of break points that defines $\Xi$ is uniform, then the estimate \eqref{proj L2} is asymptotically $\big($with respect to the dimension of the spline space $S^p_\Xi(0,T)$ $\big)$ optimal.
\end{itemize}
In general, the authors conjecture that:
\begin{center}
 for all degree $p \geq 0$ there exists a sequence of break points such that for at least an $r=1,\ldots,p+1$, the estimate \eqref{proj L2} is optimal.
 \end{center}
\end{oss}

As a consequence of Theorem \ref{sande,manni,speleers} there holds the following result, partially extending Theorem 3 of (\cite{n-width}).

\begin{theorem}
	Let $u \in H^r(0,T)$ for $r \geq 2$. For any $q=1,\ldots,r-1$ and any sequence of break points that defines the open knot vector $\Xi$ with maximal regularity, let $h$ denote its maximal knot distance, and let $Q^q_p$ be the projection onto $S^p_\Xi(0,T)$ defined in \eqref{Qpq}. Then,
	\begin{equation}\label{err spline}
	\big |u-Q^q_pu \big |_{H^q(0,T)}\leq \Big(\frac{h}{\pi}\Big)^{r-q}\big |u \big |_{H^r(0,T)},
	\end{equation}
	for all $p \geq r-1$.
\end{theorem}

\begin{proof}
Firstly, as a consequence of the Fundamental Theorem of Calculus for absolutely continuous functions, observe that the space $H^r(0,T)$, with $r \geq 1$, satisfies
\begin{equation}\label{Hr}
H^r(0,T)=\mathbb{P}_0+K\big(H^{r-1}(0,T)\big)=\ldots=\mathbb{P}_{r-1}+K^r\big(H^0(0,T)\big),
\end{equation}
where $\mathbb{P}_{r-1}$ is the space of polynomials of degree at most $r-1$ and $K$ is the integral operator defined in \eqref{K}.

From \eqref{Hr} we know that $u \in H^r(0,T)$ can be written as $u=g+K^qv$, for $g \in \mathbb{P}_{q-1} \subset S^{p}_\Xi$, with $q \geq 1$, and $v \in H^{r-q}(0,T)$. Since the projection operator $Q^q_p: H^q(0,T) \rightarrow S^p_\Xi(0,T)$ is surjective and $\partial^q Q^q_p = Q^{q-q}_{p-q}\partial^q=P_{p-q}\partial^q$, as a consequence of Theorem \ref{sande,manni,speleers} there hold the following relations
\begin{equation*}
\begin{split}
\big\|\partial^q\big(u-Q^q_pu \big) \big \|_{L^2(0,T)}&=\big\| v-P_{p-q}v \big \|_{L^2(0,T)} \\
&\overset{(\text{if} \ p-q \geq r-q-1 \geq 0)}\leq \Big(\frac{h}{\pi}\Big)^{r-q} \big \| \partial^{r-q}v \big \|_{L^2(0,T)}=\big \|\partial^r u \big \|_{L^2(0,T)}, 
\end{split}
\end{equation*}
$q \leq r-1$ and $p \geq r-1$. Since $q = 1, \ldots, r-1$, the last inequality holds for $r \geq 2$.
\end{proof}


\begin{oss}
	The density of the family of spline spaces $(V^h_{0,*})_h$ in $H^1_{0,*}(0,T)$ is a consequence of the result \eqref{err spline} with $p=r=2$, $q=1$, and of the density of $C^\infty_c(0,T]$ in $H^1_{0,*}(0,T)$ observed in Section \ref{sec sobolev}. 
	\begin{proof}
		Let $u \in H^1_{0,*}(0,T)$ and let $\epsilon >0$ be fixed. As a consequence of the density of $C^\infty_c(0,T]$ in $H^1_{0,*}(0,T)$, there exists $\phi \in C^\infty_c(0,T]$ such that\\ $|u-\phi|_{H^1(0,T)} \leq \frac{\epsilon}{2}$. As a consequence of \eqref{err spline}, there exists $\overline{h}>0$ such that \\ $\big |\phi-Q^1_2\phi \big |_{H^1(0,T)} \leq \frac{\epsilon}{2}$ for all $h \leq \overline{h}$. We then obtain:
		\begin{equation*}
		\begin{split}
		\inf_{v_h \in V^h_{0,*}}|u-v_h|_{H^1(0,T)} &\leq |u-Q^1_2\phi \big |_{H^1(0,T)} \\
		&\leq |u-\phi|_{H^1(0,T)} + |\phi-Q^1_2\phi \big |_{H^1(0,T)} \leq \epsilon \quad \forall h \leq \overline{h}.
		\end{split}
		\end{equation*}
	\end{proof}
\end{oss}

Let us recall that we have modified the projection operator of (\cite{n-width}) in order to get a projection operator whose restriction to $H^1_{0,*}(0,T) \cap H^q(0,T)$ ($q \geq 1$) assumes values in $V^h_{0,*}$ defined in \eqref{iga space}.

\subsection{Bound on the mesh-size, stability and quasi-optimality constants}
In this Section we give two results of conditioned stability \textit{with an explicit bound on the mesh-size} and we also make explicit the stability and quasi-optimality constants.
\subsubsection{Extension to quadratic IGA with maximal regularity of conditioned stability for piecewise continuous linear FEM}
A first result is an extension to the IGA discretization of Theorem 4.7 of (\cite{Coercive}).

\begin{theorem}\label{teo stab IGA zank}
	Let 
	\begin{equation}\label{h bound}
	h \leq \frac{\pi^2}{\sqrt{2}(2+\sqrt{\mu}T)\mu T}
	\end{equation}
	be satisfied. Then, the bilinear form $a(\cdot,\cdot)$ as defined in \eqref{bil ode} satisfies the inf-sup condition
	\begin{equation}\label{infsup zank}
	\frac{2 \pi^2}{(2+\sqrt{\mu}T)^2(\pi^2+4\mu T^2)} |u_h|_{H^1(0,T)} \leq \sup_{0 \neq v_h \in V^h_{0,*}} \frac{a(u_h,\overline{\mathcal{H}}_T v_h)}{|v_h|_{H^1(0,T)}},
	\end{equation}
	for all $u_h \in V^h_{0,*}$.
\end{theorem}

\begin{proof}
	Let $u_h \in V^h_{0,*}$. As a consequence of Lax-Milgram Lemma \ref{lax-milgram}, let $w \in H^1_{0,*}(0,T)$ be the unique solution of the variational problem
	\begin{equation}\label{1 var}
	-\int_0^T \partial_t w(t) \partial_t(\overline{\mathcal{H}}_T v)(t) \ dt = -\mu \int_0^T u_h(t)(\overline{\mathcal{H}}_T v)(t) \ dt  \quad \forall v \in H^1_{0,*}(0,T).
	\end{equation}
	Hence, recalling that $(\overline{\mathcal{H}}_T v)(t)=v(T)-v(t)$, 
	\begin{equation}\label{auh-w}
	\begin{split}
		a(u_h,\overline{\mathcal{H}}_T(u_h-w))&=-\int_0^T \partial_t u_h(t) \partial_t[(\overline{\mathcal{H}}_T u_h)(t)-(\overline{\mathcal{H}}_T w)(t)] \ dt \\
		 & \quad + \mu \int_0^T u_h(t) [(\overline{\mathcal{H}}_T u_h)(t)-(\overline{\mathcal{H}}_T w)(t)] \ dt\\
	&\overset{\eqref{1 var}}= \int_0^T \partial_t u_h(t) \partial_t[u_h(t)- w(t)] \ dt - \int_0^T \partial_t w(t) [\partial_t u_h(t)-\partial_t w(t)] \ dt\\
	&= \int_0^T [\partial_t u_h(t)-\partial_t w(t)]^2 \ dt=|u_h-w|_{H^1(0,T)}^2.
	\end{split}
	\end{equation}
	Also, thanks to Lax-Milgram Lemma \ref{lax-milgram}, let $z \in H^1_{0,*}(0,T)$ be the unique solution of the following variational problem
	\begin{equation}\label{2 var}
	\begin{split}
	-\int_0^T \partial_t z(t) \partial_t (\overline{\mathcal{H}}_T v)(t) \ dt = -\int_0^T &\partial_t u_h(t) \partial_t(\overline{\mathcal{H}}_T v)(t) \ dt \\
	& + \mu \int_0^T u_h(t)(\overline{\mathcal{H}}_T v)(t)\ dt \quad \forall v \in H^1_{0,*}(0,T)
	\end{split}
	\end{equation}
	Since $w \in H^1_{0,*}(0,T)$ is the solution of \eqref{1 var}, problem \eqref{2 var} is equivalent to
	\begin{equation*}
	-\int_0^T \partial_t [z(t)-(u_h(t)-w(t))] \partial_t(\overline{\mathcal{H}}_T v)(t) \ dt =0 \quad \forall v \in H^1_{0,*}(0,T),
	\end{equation*}
	from which we conclude, by choosing $v=z-(u_h-w) \in H^1_{0,*}(0,T)$, that
	$z(t)=u_h(t)-w(t)$ for all $t \in [0,T]$, since $u_h(0)=w(0)=z(0)=0$. Therefore, from \eqref{auh-w}, we conclude that
	\begin{equation}\label{a=z}
	a(u_h,\overline{\mathcal{H}}_T(u_h-w))=|z|^2_{H^1(0,T)}.
	\end{equation}
	On the other hand, the variational formulation \eqref{2 var} gives
	\begin{equation*}
	\begin{split}
	|z|_{H^1(0,T)} &\overset{\eqref{2 var}}= \frac{a(u_h,\overline{\mathcal{H}}_T z)}{|z|_{H^1(0,T)} } \leq \sup_{0 \neq v \in H^1_{0,*}(0,T)} \frac{a(u_h,\overline{\mathcal{H}}_T v)}{|v|_{H^1(0,T)}} \\
	&\overset{\eqref{2 var}} = \sup_{0 \neq v \in H^1_{0,*}(0,T)} \frac{ \langle \partial_tz,\partial_tv \rangle _{L^2(0,T)}}{|v|_{H^1(0,T)}} \overset{(C-S)}\leq |z|_{H^1(0,T)}.
	\end{split}
	\end{equation*}
	Hence, by using \eqref{stab ode}, there hold the following
	\begin{equation*}
	|z|_{H^1(0,T)}= \sup_{0 \neq v \in H^1_{0,*}(0,T)} \frac{a(u_h,\overline{\mathcal{H}}_T v)}{|v|_{H^1(0,T)}} \overset{\eqref{stab ode}}\geq \frac{2}{2+\sqrt{\mu}T} |u_h|_{H^1(0,T)},
	\end{equation*}
	from which, recalling \eqref{a=z}, we conclude
	\begin{equation}\label{magg a}
	a(u_h,\overline{\mathcal{H}}_T(u_h-w)) \overset{\eqref{a=z}}= |z|^2_{H^1(0,T)} \geq \frac{4}{(2+\sqrt{\mu}T)^2}|u_h|^2_{H^1(0,T)}.
	\end{equation}
	We now discretize the first variational formulation \eqref{1 var}. Let $w_h \in V^h_{0,*}$ be the unique solution of the following variational problem
	\begin{equation}\label{disc dim}
	\int_0^T \partial_t w_h(t) \partial_t v_h(t) dt = -\mu \int_0^T u_h(t) (\overline{\mathcal{H}}_T v_h)(t) \ dt \quad \forall v_h \in V^h_{0,*}.
	\end{equation}
	By using Céa's Lemma \ref{céa} and the error estimate \eqref{err spline} with $r=2$, $p=2$, $q=1$, there hold the following relations
	\begin{equation}\label{w-wh}
	\begin{split}
	|w-&w_h|_{H^1(0,T)}\leq \inf_{0 \neq v_h \in V^h_{0,*}}|w-v_h|_{H^1(0,T)} \quad \text{(Céa)}\\
	& \leq |w-Q^1_2 w|_{H^1(0,T)} \overset{\eqref{err spline}}\leq \frac{h}{\pi}\|\partial_{tt}w\|_{L^2(0,T)} \overset{\eqref{disc dim}}= \mu \frac{h}{\pi}\|u_h\|_{L^2(0,T)}.
	\end{split}
	\end{equation}
	Furthermore, Galerkin orthogonality \eqref{gal ort} is satisfied
	\begin{equation}\label{gal ort dim}
	\int_0^T[\partial_t w(t)-\partial_t w_h(t)] \partial_t v_h(t) \ dt =0 \quad \forall v_h \in V^h_{0,*}.
	\end{equation}
	As a consequence of \eqref{gal ort dim}, we have
	\begin{equation}\label{magg uh}
	\begin{split}
	a(u_h,\overline{\mathcal{H}}_T(w-w_h))&=\int_0^T \partial_t u_h(t)\partial_t[w(t)-w_h(t)] \ dt + \mu \int_0^T u_h(t)[\overline{\mathcal{H}}_T(w-w_h)](t) \ dt \\
	&\overset{\eqref{gal ort dim}}=\mu \int_0^T u_h(t)[\overline{\mathcal{H}}_T(w-w_h)](t) \ dt \\  
	& \leq \mu \|u_h\|_{L^2(0,T)} \|\overline{\mathcal{H}}_T(w-w_h)\|_{L^2(0,T)}.
	\end{split}
	\end{equation}
	As a consequence of Lax-Milgram Lemma \ref{lax-milgram}, let now $\psi \in H^1_{0,*}(0,T)$ be the unique solution of the following variational problem
	\begin{equation}\label{3 var}
	- \int_0^T \partial_t \psi(t) \partial_t [\overline{\mathcal{H}}_Tv](t) \ dt = \int_0^T(\overline{\mathcal{H}}_T(w-w_h))(t)(\overline{\mathcal{H}}_Tv)(t) \ dt \quad \forall v \in H^1_{0,*}(0,T).
	\end{equation}
	In particular, by choosing $v=w-w_h \in H^1_{0,*}(0,T)$ and recalling \eqref{gal ort dim}, we obtain
	\begin{equation*}
	\begin{split}
	\|\overline{\mathcal{H}}_T(w-w_h)\|^2_{L^2(0,T)}&=\int_0^T[\overline{\mathcal{H}}_T(w-w_h)](t)[\overline{\mathcal{H}}_T(w-w_h)](t) \ dt \\
	&\overset{\eqref{3 var}}=-\int_0^T\partial_t \psi(t)\partial_t [\overline{\mathcal{H}}_T(w-w_h)](t) \ dt \\
	&=\int_0^T\partial_t \psi(t)[\partial_t w(t)-\partial_t w_h(t)] \ dt \\
	&\overset{\eqref{gal ort dim}}=\int_0^T \partial_t[\psi(t)-Q^1_2 \psi(t)][\partial_t w(t)-\partial_t w_h(t)] \ dt\\
	& \leq |\psi-Q^1_2 \psi|_{H^1(0,T)}|w-w_h|_{H^1(0,T)}\\
	&\overset{\eqref{err spline},\eqref{w-wh}}\leq \Big(\frac{h}{\pi} \Big)^2\mu\|\partial_{tt} \psi \|_{L^2(0,T)}\|u_h\|_{L^2(0,T)}\\
	&\overset{\eqref{3 var}}= \Big(\frac{h}{\pi} \Big)^2\mu \|\overline{\mathcal{H}}_T(w-w_h)\|_{L^2(0,T)} \|u_h\|_{L^2(0,T)},
	\end{split}
	\end{equation*}
	i.e.,
	\begin{equation*}
	\|\overline{\mathcal{H}}_T(w-w_h)\|_{L^2(0,T)}\leq \frac{h^2}{\pi^2}\mu\|u_h\|_{L^2(0,T)}.
	\end{equation*}
	Therefore, by using \eqref{magg uh} and Poincaré inequality \eqref{Poinc},
	\begin{equation}\label{stim}
     a(u_h,\overline{\mathcal{H}}_T(w-w_h)) \overset{\eqref{magg uh}}\leq \Big(\frac{h}{\pi} \mu \Big)^2  \|u_h\|^2_{L^2(0,T)} \overset{\eqref{Poinc}}\leq \Big(\frac{2 T}{\pi^2} \mu h \Big)^2|u_h|_{H^1(0,T)}^2
  	\end{equation}
  	follows. Hence, as a consequence of \eqref{magg a} and \eqref{stim} we conclude
  	\begin{equation*}
  	\begin{split}
  	a(u_h,\overline{\mathcal{H}}_T(u_h-w_h))&=a(u_h,\overline{\mathcal{H}}_T(u_h-w))+a(u_h,\overline{\mathcal{H}}_T(w-w_h))\\
  	& \overset{\eqref{magg a},\eqref{stim}}\geq \Bigg[ \frac{4}{(2+\sqrt{\mu}T)^2} - \Big(\frac{2 T}{\pi^2} \mu h \Big)^2 \Bigg] |u_h|^2_{H^1(0,T)}\\
  	& \geq \frac{2}{(2+\sqrt{\mu}T)^2} |u_h|^2_{H^1(0,T)},
  	\end{split}
  	\end{equation*}
  	if
  	\begin{equation*}
  	\Big(\frac{2 T}{\pi^2} \mu h \Big)^2 \leq \frac{2}{(2+\sqrt{\mu}T)^2}
  	\end{equation*}
  	is satisfied, i.e.,
  	\begin{equation*}
  	h \leq \frac{\pi^2}{\sqrt{2}(2+\sqrt{\mu}T)\mu T}.
  	\end{equation*}
  	We now consider a lower bound for $|u_h|_{H^1(0,T)}$ dependent of $|u_h-w_h|_{H^1(0,T)}$. Indeed, we have
  	\begin{equation*}
  	\|\partial_t(u_h-w_h)\|_{L^2(0,T)}\leq \|\partial_t u_h\|_{L^2(0,T)}+\|\partial_t w_h\|_{L^2(0,T)},
  	\end{equation*}
  	and, thanks to \eqref{disc dim} and Poincaré inequality \eqref{Poinc},
  	\begin{equation*}
  	\begin{split}
  	\|\partial_t w_h\|^2_{L^2(0,T)}&=-\int_0^T \partial_t w_h(t) \partial_t (\overline{\mathcal{H}}_T w_h)(t) \ dt \overset{\eqref{disc dim}}= -\mu \int_0^T u_h(t)(\overline{\mathcal{H}}_T w_h)(t) \ dt \\
  	&  \leq \mu \|u_h\|_{L^2(0,T)} \|\overline{\mathcal{H}}_T w_h \|_{L^2(0,T)} \overset{\eqref{Poinc}} \leq \frac{4 T^2}{\pi^2}\mu |u_h|_{H^1(0,T)}|w_h|_{H^1(0,T)},
  	\end{split}
  	\end{equation*}
  	i.e.,
  	\begin{equation*}
  	|u_h-w_h|_{H^1(0,T)} \leq \Big(1+\frac{4 T^2}{\pi^2}\mu \Big) |u_h|_{H^1(0,T)}.
  	\end{equation*}
  	Therefore, we obtain the following inequality
  	\begin{equation*}
  	\frac{2 \pi^2}{(2+\sqrt{\mu}T)^2(\pi^2+4\mu T^2)} |u_h|_{H^1(0,T)} \leq \frac{a(u_h,\overline{\mathcal{H}}_T(u_h-w_h))}{|u_h-w_h|_{H^1(0,T)}}.
  	\end{equation*}
\end{proof}

Hence, we obtain the following result.

\begin{theorem}
	Let \eqref{h bound} be satisfied. Then, problem \eqref{iga ode equiv} is well-posed with the stability estimate
	\begin{equation}\label{stab zank}
	|u_h|_{H^1(0,T)} \leq \frac{(2+\sqrt{\mu}T)^2(\pi^2+4\mu T^2)}{2\pi^2}\|f\|_{[H^1_{*,0}(0,T)]'},
	\end{equation}
	where $f \in [H^1_{*,0}(0,T)]'$ and $u_h$ is the unique solution of \eqref{iga ode equiv}. Moreover, a quasi-optimality estimate holds
	\begin{equation}\label{quasi opt zank}
	|u-u_h|_{H^1(0,T)} \leq \Bigg[ 1 + \frac{(2+\sqrt{\mu} T)^2(\pi^2+4 \mu T^2)^2}{2 \pi^4} \Bigg] \inf_{v_h \in V^h_{0,*}} |u-v_h|_{H^1(0,T)},
	\end{equation}
	where $u$ is the unique solution of \eqref{var ode equiv}.
\end{theorem}

\begin{proof}
	The well-posedness with stability estimate \eqref{stab zank} is an immediate consequence of BNB Theorem \ref{BNB}.
	
	In order to prove \eqref{quasi opt zank}, we repeat, with explicit constants, the proof of Proposition \ref{err BNB}. For any $w \in H^1_{0,*}(0,T)$ we define $w_h:= G_h w$ as the Galerkin projection satisfying 
	\begin{equation*}
	a(G_hw,\overline{\mathcal{H}}_T v_h)=a(w,\overline{\mathcal{H}}_T v_h) \quad \forall v_h \in V^h_{0,*},
	\end{equation*}
	which is well defined thanks to the well-posedness of the discrete problem. Hence, by using the stability estimate \eqref{infsup zank} and the continuity of $a(\cdot,\cdot)$ \eqref{cont of a}, there hold
	\begin{equation*}
	|G_h w|_{H^1(0,T)} \leq \frac{(2+\sqrt{\mu}T)^2(\pi^2+4 \mu T^2)^2}{2 \pi^4} |w|_{H^1(0,T)}.
	\end{equation*}
	Indeed,
	\begin{equation*}
	\begin{split}
		\frac{2 \pi^2}{(2+\sqrt{\mu}T)^2(\pi^2+4\mu T^2)} &|G_hw|_{H^1(0,T)}  
		\overset{\eqref{infsup zank}}\leq \sup_{0 \neq v_h \in V^h_{0,*}} \frac{a(G_hw,\overline{\mathcal{H}}_T v_h)}{|v_h|_{H^1(0,T)}}\\
		&=\sup_{0 \neq v_h \in V^h_{0,*}} \frac{a(w,\overline{\mathcal{H}}_T v_h)}{|v_h|_{H^1(0,T)}}\\
		& \overset{\eqref{cont of a}}\leq  \Bigg(1+\frac{4 T^2 \mu }{\pi^2} \Bigg) |w|_{H^1(0,T)}.
	\end{split}
	\end{equation*}
	Since $u_h=G_h u$ and $v_h = G_h v_h$ for all $v_h \in V^h_{0,*}$, we conclude
	\begin{equation*}
	\begin{split}
	|u-u_h|_{H^1(0,T)} &\leq |u-v_h|_{H^1(0,T)}+|G_h(v_h-u)|_{H^1(0,T)} \\
	& \leq \Bigg[ 1 + \frac{(2+\sqrt{\mu} T)^2(\pi^2+4 \mu T^2)^2}{2 \pi^4} \Bigg] |u-v_h|_{H^1(0,T)}.
	\end{split}
	\end{equation*}
\end{proof}

Thus, we are in a position to state a convergence result for the isogeometric solution $u_h$ of the variational formulation \eqref{var ode equiv}.

\begin{cor}
	Let $u \in H^1_{0,*}(0,T)$ and $u_h \in V^h_{0,*}$ be the unique solutions of the variational formulations \eqref{var ode equiv} and \eqref{iga ode equiv}, respectively. Let  $u \in H^3(0,T)$ and \eqref{h bound} be satisfied. Then, there holds true the error estimate
	\begin{equation}\label{ord conv iga zank}
	|u-u_h|_{H^1(0,T)} \leq \frac{1}{\pi^2} \Bigg[ 1 + \frac{(2+\sqrt{\mu} T)^2(\pi^2+4 \mu T^2)^2}{2 \pi^4} \Bigg] h^2 |u|_{H^3(0,T)}.
	\end{equation}
\end{cor}

\begin{proof}
	Estimate \eqref{ord conv iga zank} is a straightforward consequence of quasi-optimality \eqref{quasi opt zank} and of \eqref{err spline} with $r=3,p=2,q=1$.
\end{proof}

\begin{oss}
	Note that the bound on the mesh-size \eqref{h bound} is about $2.015$ times the bound 
	\begin{equation*}
	h \leq \frac{2 \sqrt{3}}{(2+\sqrt{\mu}T)\mu T}
	\end{equation*}
	of (\cite{Steinbach2019}). It's also about $1.81$ times the more accurate bound 
	\begin{equation*}
	h \leq \frac{\sqrt{3}\pi}{\sqrt{2}(2+\mu T)\mu T}
	\end{equation*}	
	of (\cite{Zank2020}). Therefore, in order to get stability and convergence, our mesh-size can be approximately twice the mesh-size of piece-wise linear FEM.
\end{oss}

\subsubsection{Application of Theorem \ref{theo gard} to quadratic IGA with maximal regularity}
Another way to get a bound on the mesh-size, so that, if it is respected, the well-posedness of IGA, stability and convergence (with explicit constants) are guaranteed, is the theory of \textit{Galerkin method applied to G\aa rding-type problems} discussed in Section \ref{sec prel ode}. 

Let now consider problem \eqref{var ode equiv} and its isogeometric discretization \eqref{iga ode equiv}. Actually, these problems lie within the framework of Theorem \ref{theo gard}. Indeed, as a consequence of Rellich-Kondrachov Theorem, the inclusion $H^1_{0,*}(0,T) \subset L^2(0,T)$ is compact. Furthermore, the following result holds.
\begin{lemma}
	Let $b > \frac{\mu T^2}{2}$. The bilinear form defined in \eqref{bil ode} satisfies the G\aa rding inequality:
	\begin{equation}\label{gard ineq iga}
	a(v,\overline{\mathcal{H}}_T v) \geq \Big(1- \frac{\mu T^2}{2b}\Big)|v|_{H^1(0,T)}-\frac{2+b}{2}\mu \|v\|_{L^2(0,T)} \quad \forall v \in H^1_{0,*}(0,T).
	\end{equation}
\end{lemma}
\begin{proof}
	The bilinear form defined in \eqref{bil ode} satisfies the following inequalities
	\begin{equation*}
	\begin{split}
	a(v,\overline{\mathcal{H}}_T v)&= \langle \partial_t v, \partial_t v \rangle _{L^2(0,T)}- \mu \langle v,v \rangle_{L^2(0,T)}+\mu \langle v,v(T) \rangle_{L^2(0,T)} \\
	& \geq |v|^2_{H^1(0,T)}-\mu \|v\|^2_{L^2(0,T)}-\mu \int_0^T |v(t)v(T)| \ dt \\
	& \overset{(C-S)}\geq |v|^2_{H^1(0,T)}-\mu \|v\|^2_{L^2(0,T)} - \mu \|v\|_{L^2(0,T)}\sqrt{T}|v(T)| \\
	& \overset{(F.T.C),(Young)}\geq |v|^2_{H^1(0,T)}-\mu \|v\|^2_{L^2(0,T)} - \frac{\mu b}{2} \|v\|^2_{L^2(0,T)}-\frac{\mu T^2}{2b} |v|^2_{H^1(0,T)}\\
	& = \Big(1- \frac{\mu T^2}{2b}\Big)|v|^2_{H^1(0,T)}-\frac{2+b}{2}\mu \|v\|^2_{L^2(0,T)} ,
	\end{split}
	\end{equation*}
	for all $b >0$, where we used Cauchy Schwarz inequality, Young inequality and the Fundamental Theorem of Calculus. In order to obtain positive coefficients in \eqref{gard ineq iga} we add the constraint $b > \frac{\mu T^2}{2}$.
\end{proof}	
	
Also, Theorem \ref{zank} guarantees that problem \eqref{var ode equiv} is well-posed, then, the only $u_0 \in H^1_{0,*}(0,T)$ such that $a(u_0,\overline{\mathcal{H}}_T v)=0$ for all $v \in H^1_{0,*}(0,T)$ is $u_0=0$. Therefore, we conclude the following result.

\begin{theorem}\label{theo gard iga}
	Let 
	\begin{equation}\label{h bound gard}
	h \leq \frac{\pi^5}{(\pi^2+ 4\mu T^2)[\pi^2+2\mu T^2 (2+\sqrt{\mu} T)]} \sqrt{\frac{2b - \mu T^2}{2b(2+b)\mu}}
	\end{equation}
	be satisfied, with $b > \frac{\mu T^2}{2}$. 
	Then, problem \eqref{iga ode equiv} is well-posed with the stability estimate
	\begin{equation}\label{stab iga gard}
	\|u_h\|_{H^1(0,T)} \leq (2+ \sqrt{\mu} T ) \Bigg[ \frac{3b + \mu T^2\big(\frac{8b}{\pi^2}-\frac{1}{2} \big)}{2b-\mu T^2} \Bigg] \|f\|_{[H^1_{*,0}(0,T)]'},
	\end{equation}
	where $f \in [H^1_{*,0}(0,T)]' $ and $u_h$ is the unique solution of \eqref{iga ode equiv}. Moreover, a quasi-optimality estimate holds
	\begin{equation}\label{opt iga gard}
	|u-u_h|_{H^1(0,T)} \leq \frac{4b}{\pi^2}\frac{ \pi^2+ 4 \mu T^2}{2b-\mu T^2} \inf_{v_h \in V^h_{0,*}} |u-v_h|_{H^1(0,T)},
	\end{equation}
	where $u \in H^1_{0,*}(0,T)$ is the unique solution of \eqref{var ode equiv}.
\end{theorem}

\begin{proof}
	Let $b > \frac{\mu T^2}{2}$ and let $\eta(V^h_{0,*})$ be the parameter defined in \eqref{eta gard} and related to the sequence of isogeometric spaces in \eqref{iga space}. By using the projection operator \eqref{Qpq} with $p=2$, $q=1$, the error estimate \eqref{err spline} with $r=2$, the Poincaré inequality \eqref{Poinc} and the a priori estimate of the abstract problem \eqref{stab ode}, we obtain
	\begin{equation}\label{eta iga}
	\eta(V^h_{0,*}) \leq \frac{h}{\pi} \Bigg[ 1 + \frac{2\mu T^2}{\pi^2}(2+\sqrt{\mu} T) \Bigg].
	\end{equation}
    Let see why \eqref{eta iga} holds. Let $z_g \in H^1_{0,*}(0,T)$ be the solution of the adjoint problem \eqref{adj} with respect to our context, i.e., given $g \in L^2(0,T)$, $z_g$ is the unique element of $H^1_{0,*}(0,T)$ that satisfies
	\begin{equation*}
	a(v,\overline{\mathcal{H}}_T z_g)=(g,v)_{L^2(0,T)} \quad \forall v \in H^1_{0,*}(0,T).
	\end{equation*}
	Therefore, $z:=\overline{\mathcal{H}}_Tz_g \in H^1_{*,0}(0,T)$ is the unique solution of
	\begin{equation*}
	a(v,z)=(g,v)_{L^2(0,T)} \quad \forall v \in H^1_{0,*}(0,T),
	\end{equation*}
	i.e.,
	\begin{equation*}
	-\langle \partial_t v,\partial_t z \rangle_{L^2(0,T)} + \mu \langle v,z \rangle_{L^2(0,T)} = (g,v)_{L^2(0,T)} \quad \forall v \in H^1_{0,*}(0,T).
	\end{equation*}
	As a consequence, the distributional derivative $\partial_{tt} z$ is represented by $g-\mu z \in L^2(0,T)$. Hence, $z \in H^2(0,T)$ and $z_g = \overline{\mathcal{H}}_T^{-1} z = z(0)-z \in H^2(0,T)$ with 
	\begin{equation*}
	\partial_{tt} z_g = \mu \overline{\mathcal{H}}_T z_g - g.
	\end{equation*}
	Note that the adjoint problem has the same stability estimate (w.r.t. the dual norm $\|g\|_{[H^1_{0,*}(0,T)]'}$) of the primal problem, as noted in the second point of Remark \ref{oss gard}. Then, the following relations hold
	\begin{equation*}
	\begin{split}
\inf_{v_h \in V^h_{0,*}} |z_g-v_h|_{H^1(0,T)} &\leq |z_g - Q^1_2 z_g|_{H^1(0,T)} \overset{\eqref{err spline}}\leq \frac{h}{\pi} \|\partial_{tt} z_g\|_{L^2(0,T)}\\
&=\frac{h}{\pi}\|\mu \overline{\mathcal{H}}_T z_g-g\|_{L^2(0,T)}\\
& \leq \frac{h}{\pi} \Big(\|g\|_{L^2(0,T)} + \mu \|\overline{\mathcal{H}}_T z_g\|_{L^2(0,T)} \Big)\\
&\overset{\eqref{Poinc}}\leq \frac{h}{\pi} \Big(\|g\|_{L^2(0,T)} + \frac{2T}{\pi}\mu |z_g|_{H^1(0,T)} \Big)\\
&\overset{\eqref{stab ode}}\leq \frac{h}{\pi} \Big(\|g\|_{L^2(0,T)} + \frac{4T^2}{\pi^2}\mu \frac{2 + \sqrt{\mu} T}{2} \|g\|_{L^2(0,T)} \Big)\\
&=\frac{h}{\pi} \Big[ 1 + \frac{2\mu T^2}{\pi^2}(2+\sqrt{\mu}T) \Big] \|g\|_{L^2(0,T)},
	\end{split}
	\end{equation*}
	which implies estimate \eqref{eta iga}. Therefore, from condition \eqref{cond eta} where \eqref{gard ineq iga} is the G\aa rding inequality of our problem \eqref{iga ode equiv}, we deduce that, if 
	\begin{equation*}
	\frac{h}{\pi} \Bigg[ 1 + \frac{2\mu T^2}{\pi^2}(2+\sqrt{\mu}T) \Bigg] \leq \frac{\pi^2}{\pi^2+4\mu T^2} \sqrt{\frac{2b - \mu T^2}{2b(2+b)\mu}},
	\end{equation*}
i.e.,
\begin{equation*}
h \leq \frac{\pi^5}{(\pi^2+4 \mu T^2)[\pi^2+2\mu T^2 (2+\sqrt{\mu} T)]} \sqrt{\frac{2b - \mu T^2}{2b(2+b)\mu}},
\end{equation*}	
then problem \eqref{iga ode equiv} is well-posed, and conditions \eqref{stab iga gard}, \eqref{opt iga gard} are satisfied. 
\end{proof}

\begin{oss}
	The choice
	\begin{equation}\label{b}
	b=\frac{\mu T^2 + \sqrt{\mu^2 T^4+ 4 \mu T^2}}{2}
	\end{equation}
	maximises $\sqrt{\frac{2b - \mu T^2}{2b(2+b)\mu}}$ in the upper bound of \eqref{h bound gard}.
\end{oss}

As a consequence of Theorem \ref{theo gard iga}, we can state a convergence result for the isogeometric solution $u_h$ of the variational formulation \eqref{var ode equiv}.

\begin{cor}\label{cor gard}
Let $u \in H^1_{0,*}(0,T)$ and $u_h \in V^h_{0,*}$ be the unique solutions of the variational formulations \eqref{var ode equiv} and \eqref{iga ode equiv}, respectively. Let $u \in H^3(0,T)$ and \eqref{h bound gard} be satisfied. Then, there holds true the error estimate
\begin{equation}\label{ord conv iga gard}
|u-u_h|_{H^1(0,T)} \leq \frac{4b}{\pi^4}\frac{ \pi^2+ 4 \mu T^2}{2b-\mu T^2} h^2 |u|_{H^3(0,T)}.
\end{equation}
\end{cor}

\begin{proof}
	Estimate \eqref{ord conv iga gard} is a straightforward consequence of quasi-optimality \eqref{opt iga gard} and of \eqref{err spline} with $r=3,p=2,q=1$.
\end{proof}

\begin{oss}\label{oss grado più alto}
The analysis proposed in this Section, which is based on techniques that exploit the G\aa rding inequality and are alternative to those proposed by O. Steinbach and M. Zank (extended to the IGA case in the previous Section) is useful to understand how the IGA behaves in the case of generic polynomial degree and maximal regularity. Indeed, from the proof of Theorem \ref{theo gard iga} it emerges that the threshold on $h$ \eqref{h bound gard} does not change when the polynomial degree and the regularity of the spline test and trial functions are raised. Only the order of convergence of Corollary \ref{h bound gard} changes: for a generic polynomial degree $p \geq 1$ and an exact solution $u \in H^{p+1}(0,T)$, the order of convergence is $p$. 
On the other hand, how the upper bound \eqref{h bound} behaves in $h$ is not immediately clear from the proof of Theorem \ref{teo stab IGA zank}. This is a consequence of the various auxiliary problems considered, in which the regularity of the corresponding solutions is a key point. 
\end{oss}	

\begin{oss}\label{oss asympt}
	Asymptotically for $\mu \rightarrow \infty$ the optimal value \eqref{b} satisfies $b\simeq\mu T^2$. We then obtain
	\begin{equation*}
	h \leq \overline{h}_2 :=\frac{\pi^5}{(\pi^2+ 4\mu T^2)[\pi^2+2\mu T^2 (2+\sqrt{\mu} T)]} \sqrt{\frac{1}{2\mu(2+\mu T^2)}},
	\end{equation*}
	in place of \eqref{h bound gard} with a general value $b > \frac{\mu T^2}{2}$. 
	
	Let us now recall estimate \eqref{h bound} obtained by extending Theorem $4.7$ of (\cite{Coercive}) to quadratic IGA with maximal regularity, i.e.,
	\begin{equation*}
	h \leq \overline{h}_1 :=\frac{\pi^2}{\sqrt{2}(2+\sqrt{\mu}T)\mu T}.
	\end{equation*}
	Therefore, the techniques of O. Steinbach and M. Zank and those ones using the G\aa rding inequality \eqref{gard} give constraints on $h$ of order
	\begin{gather*}
	\overline{h}_1 \simeq \mathcal{O}(\mu^{-3/2}),\\
	\overline{h}_2 \simeq \mathcal{O}(\mu^{-7/2}),
	\end{gather*}
respectively. Thus, we conclude that, asymptotically, threshold \eqref{h bound gard} is a stronger constraint than \eqref{h bound}. However, asymptotically, stability estimate \eqref{stab iga gard} behaves as $\mathcal{O}(\mu^{3/2})$, whereas, stability estimate \eqref{stab zank} behaves as $\mathcal{O}(\mu^{2})$. Therefore, we conclude that stability estimate \eqref{stab iga gard} has a slower growth than \eqref{stab zank} for $\mu \rightarrow \infty$. 
\end{oss}
	