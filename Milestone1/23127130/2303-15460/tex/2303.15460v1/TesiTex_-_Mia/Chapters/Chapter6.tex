% Chapter 6

\chapter{The wave equation} % Main chapter title

\label{Chapter6} % For referencing the chapter elsewhere, use \ref{Chapter3} 

%----------------------------------------------------------------------------------------

% Define some commands to keep the formatting separated from the content 

%----------------------------------------------------------------------------------------

In this Chapter we present ...

Let us recall that $\Omega \subset \mathbb{R}^d$, with $d=1,2,3$, is an open bounded Lipschitz domain and, for a real value $T>0$, $(0,T)$ is a time interval. Then, the bounded space-time cylinder is defined as $Q:=\Omega \times (0,T) \subset \mathbb{R}^{d+1}$ and $\Sigma:=\partial{\Omega} \times [0,T] \subset \mathbb{R}^{d+1}$ is the lateral boundary of $Q$.

\section{Wave operator as a distribution} 
Since we're interested in efficient numerical resolutions for the variational formulation of the Dirichlet boundary value problem for the wave equation, we define the wave operator $\Box:= \partial_{tt}-\Delta_x$, which is the classical pointwise derivative for sufficiently smooth functions. In general, we can see $\Box$ as the distributional wave operator $\Box: \mathcal{D}'(Q) \rightarrow \mathcal{D}'(Q)$, where, for a distribution $T: \mathcal{D} \rightarrow \mathbb{R}$,
\begin{equation*}
\langle \Box T,\phi\rangle=\langle T, \Box \phi \rangle \quad \forall \phi \in \mathcal{D(Q)}.
\end{equation*}
Naturally, we can extend this definition from the domain $Q$ to all the open subset $A \subset \mathbb{R}^n$, with $n \in \mathbb{N}$.

\section{Sobolev and Bochner spaces} 

\textbf{Sobolev spaces}. With the usual notations, the Hilbert space $H^{p}(\Omega)$, with $p \in \mathbb{N}$, is the Sobolev space of (classes of) real-valued functions endowed with the inner product $\langle \cdot, \cdot \rangle_{H^p(\Omega)}$ and the induced norm $\|\cdot\|_{H^p(\Omega)}$, i.e.,
\begin{gather*}
H^p(\Omega):=\{u \in L^2(\Omega): \ \partial^\alpha_x u \in L^2(\Omega) \  \text{for all multi-index} \ \alpha \ \text{s.t.} \ |\alpha| \leq p \},\\
\langle u,v \rangle_{H^p(\Omega)}:=\int_{\Omega} u(x)v(x) \ dx + \sum_{1\leq|\alpha| \leq p} \int_\Omega \partial^\alpha_x u(x)\partial^\alpha_x v(x) \ dx, \quad u,v \in H^p(\Omega).
\end{gather*}
Also, we will consider the seminorm
\begin{equation*}
|u|_{H^p(\Omega)}:=\sqrt{\sum_{|\alpha| =p}\int_\Omega \partial^\alpha_x u(x)\partial^\alpha_x v(x) \ dx}, \quad u \in H^p(\Omega).
\end{equation*}
As usual, the Hilbert space $H^p_0(\Omega)$ is defined as the closure of the space $C^{\infty}_c(\Omega)$ with respect to the norm $\|\cdot\|_{H^p(\Omega)}$.  

For the subspace $H^1_0(\Omega) \subset H^1(\Omega)$, we consider the inner product
\begin{equation*}
\langle u,v \rangle_{H^1_0(\Omega)}:=\int_{\Omega} \nabla_x u(x) \cdot \nabla_x v(x) \ dx, \quad u,v \in H^1_0(\Omega),
\end{equation*}
and the induced Hilbertian norm
\begin{equation*}
\|u\|_{H^1_0(\Omega)}=|u|_{H^1(\Omega)}, \quad u \in H^1_0(\Omega),
\end{equation*}
which is equivalent to $\|\cdot\|_{H^1(\Omega)}$, by the Poincaré inequality.

The dual space $H^{-1}(\Omega):=[H^1_0(\Omega)]'$ is a Hilbert space characterised as a completion of $L^2(\Omega)$ with respect to the Hilbertian norm $||\cdot||_{H^{-1}(\Omega)}$. \bigskip \\

\textbf{Bochner spaces}. For an introduction to Bochner spaces, see (\cite{evans, brezis_operat}).

Let $X$ be a separable real Banach space, the Bochner space $L^2(\Omega;X)$ is defined as the space of (classes of) measurable vector-valued functions $\mathbf{u}:\Omega \rightarrow X$ such that $\|\mathbf{u}(\cdot)\|_X \in L^2(\Omega)$. The space $L^2(\Omega;X)$, endowed with the norm
\begin{equation*}
\|\mathbf{u}\|_{L^2(\Omega;X)}=\sqrt{\int_\Omega \|\mathbf{u}(y)\|_X^2 \ dy}, \quad \mathbf{u} \in L^2(\Omega;X),
\end{equation*}
is a Banach space. Assuming that $X$ is also reflexive, than the dual space $[L^2(\Omega;X)]'$ and the Bochner space $L^2(\Omega;X')$ are isometric, with the dual product
\begin{equation*}
\int_\Omega \langle \mathbf{f}(y),\mathbf{u}(y) \rangle_{X', X} \  dy, \quad \mathbf{f} \in L^2(\Omega;X'), \ \mathbf{u} \in L^2(\Omega;X).
\end{equation*}

Let $H$ be a separable real Hilbert space, than $L^2(\Omega;H)$ is also a Hilbert space with respect to the natural inner product
\begin{equation*}
\langle \mathbf{u},\mathbf{v} \rangle_{L^2(\Omega;H)}:=\int_\Omega \langle \mathbf{u}(y),\mathbf{v}(y) \rangle_H  \ dy, \quad \mathbf{u},\mathbf{v} \in L^2(\Omega;H).
\end{equation*}

Assuming that $\Omega_1 \subset \mathbb{R}^{d_1}$ and $\Omega_2 \subset \mathbb{R}^{d_2}$ are two open bounded Lipschitz domain, the following important property holds 
\begin{equation*}
L^2(\Omega_1 \times \Omega_2) \simeq L^2(\Omega_2;L^2(\Omega_1)) \simeq L^2(\Omega_1; L^2(\Omega_2)), 
\end{equation*}
i.e. that spaces are isometric, see (\cite{aubin2000applied}). Hence, for a separable Banach space $X \subset L^2(\Omega_2)$, since it holds $L^2(\Omega_1;X) \subset L^2(\Omega_1;L^2(\Omega_2))$, see (\cite{gasinski}), we can make this identification
\begin{equation}\label{identific L^2 boc}
L^2(\Omega_1;X)=\{ u \in L^2(\Omega_1 \times \Omega_2): \ y \mapsto u(\cdot,y) \in L^2(\Omega_1;X)\},
\end{equation}
and it holds analogously for $L^2(\Omega_2;X)$, with $X \subset L^2(\Omega_1)$.\\
\\
For $p \in \mathbb{N}$, we define the following Bochner Sobolev space
\begin{equation*}
H^p(\Omega;X):=\{\mathbf{u} \in L^2(\Omega;X): \partial_{y}^\alpha \mathbf{u} \in L^2(\Omega;X) \ \text{for all multi-index} \ \alpha \ \text{s.t.} \ |\alpha| \leq m\},
\end{equation*}
with $\partial_y$ the distributional vector-valued derivative. $H^p(\Omega;X)$ is a Banach space with respect to the norm 
\begin{equation*}
\|\mathbf{u}\|_{H^p(\Omega;X)}:=\sqrt{\int_{\Omega} \|\mathbf{u}(y)\|_X^2 \ dy + \sum_{1\leq|\alpha|\leq p} \int_\Omega \|\partial_y^\alpha \mathbf{u}(y)\|_X^2 \ dy}, \quad \mathbf{u} \in H^p(\Omega;X).
\end{equation*}

Let $H$ be a separable real Hilbert space, than $H^p(\Omega;H)$ is also a Hilbert space with respect to the natural inner product
\begin{equation*}
\begin{split}
\langle \mathbf{u},\mathbf{v} \rangle_{H^p(\Omega;H)}:=\int_{\Omega} \langle \mathbf{u}(y),\mathbf{v}(y) \rangle _H  \ dy + \sum_{1\leq|\alpha|\leq p} \int_\Omega \langle \partial_y^\alpha \mathbf{u}(y), &\partial_y^\alpha \mathbf{v}(y) \rangle_H \ dy,\\
&  \mathbf{u},\mathbf{v} \in H^p(\Omega;X).
\end{split}
\end{equation*}

Recalling the Abstract Calculus Theorem, see (\cite{evans}), i.e,
\begin{equation}\label{abt}
H^1(0,T;X) \subset C([0,T];X)
\end{equation}
with a continuous embedding, we can define the following two subspaces of $H^1(0,T;X)$, as in Section \ref{sec sobolev}, 
\begin{gather}
H^1_{0,*}(0,T;X):=\{\mathbf{v} \in H^1(0,T;X): \ \mathbf{v}(0)=0 \ \text{in} \ X\},\label{0 in t=0}\\
H^1_{*,0}(0,T;X):=\{\mathbf{v} \in H^1(0,T;X): \ \mathbf{v}(T)=0 \ \text{in} \ X\}.
\end{gather}

\subsection{Sobolev spaces in $Q$}
Assuming $p \in \mathbb{N}$, $q \in \mathbb{N}$, one defines the anisotropic Sobolev space
\begin{equation*}
H^{p,q}(Q):=L^2(0,T;H^p(\Omega)) \cap H^q(0,T;L^2(\Omega)),
\end{equation*}
which is a Hilbert space with respect to the inner product
\begin{equation*}
\langle \mathbf{u},\mathbf{v} \rangle_{H^{p,q}(Q)}:=\langle \mathbf{u},\mathbf{v} \rangle_{L^2(0,T;H^p(\Omega))}+\langle \mathbf{u},\mathbf{v} \rangle_{H^q(0,T;L^2(\Omega))}, \quad \mathbf{u},\mathbf{v} \in H^{p,q}(Q).
\end{equation*}
Recalling the identification \eqref{identific L^2 boc}, and also the identification
\begin{equation*}
H^q(0,T; L^2(\Omega))=L^2(\Omega;H^q(0,T)),
\end{equation*}
see Section 2.4 of (\cite{Zank2020}), the following identification holds
\begin{equation*}
H^{p,q}(Q)=L^2(0,T;H^p(\Omega))\cap L^2(\Omega;H^q(0,T)) \subset L^2(Q),
\end{equation*}
and also
\begin{equation*}
\begin{split}
\langle u,v \rangle_{H^{p,q}(Q)}= \int_0^T \langle u(\cdot,t),v(\cdot,t) \rangle_{H^p(\Omega)}  \ dt+ \int_\Omega \langle u(x,\cdot)&,v(x,\cdot) \rangle_{H^q(0,T)}  \ dx,\\
& u,v \in H^{p,q}(Q)
\end{split}
\end{equation*}

We also have the following characterisation for the anisotropic Sobolev space $H^{p,q}(Q)$:
\begin{equation*}
H^{p,q}(Q)=\{ u \in L^2(Q): \ \partial_x^\alpha u \in L^2(Q) \ \text{for} \ |\alpha|\leq p, \ \partial_t^\beta u \in L^2(Q) \ \text{for} \ |\beta|\leq q\}.
\end{equation*}

\begin{oss}
	\begin{itemize}
		\item [(1)] The map
		\begin{gather*}
		H^1(Q) \longrightarrow H^{1,1}(Q)\\
		u \longmapsto u 
		\end{gather*}
		is an isomorphism, so we can make the identification $H^1(Q)=H^{1,1}(Q)$.
		\item [(2)] There hold 
		\begin{equation*}
		H^{1,1}(Q) \subset C([0,T];L^2(\Omega)),
		\end{equation*}
		with a continuous embedding, see \eqref{abt}.
	\end{itemize}
\end{oss}

The subspace 
\begin{equation}\label{sp cod H10}
H^{1,1}_{0;}(Q):=L^2(0,T; H^1_0(\Omega)) \cap H^1(0,T; L^2(\Omega)) \subset H^{1,1}(Q)
\end{equation}
is endowed with the inner product
\begin{equation}\label{in prod}
\begin{split}
\langle u,v \rangle_{H^{1,1}_{0;}(Q)}:=\int_0^T \int_{\Omega}(\partial_tu(x,t)\partial_tv(x,t)+ \nabla_xu(x,&t) \cdot \nabla_xv(x,t)) \ dxdt\\
& u,v \in H^{1,1}_{0;}(Q),
\end{split}
\end{equation}
and the induced Hilbertian norm
\begin{equation*}
\|u\|_{H^{1,1}_{0;}(Q)}:=|u|_{H^1(Q)}=\sqrt{\int_0^T \int_{\Omega}\Big(|\partial_tu(x,t)|^2+ |\nabla_xu(x,t)|^2 \Big) \ dxdt}, \quad u \in H^{1,1}_{0;}(Q).
\end{equation*}
Recalling the Poincaré inequality for $H^1_0(\Omega)$, in $H^{1,1}_{0;}(Q)$ the norm $|\cdot|_{H^1(Q)}$ is equivalent to the norm $\|\cdot\|_{H^1(Q)}$.

We also define these useful subspaces
\begin{gather}
\label{sp ansatz} H^{1,1}_{0;0,*}(Q):=L^2(0,T;H^1_0(\Omega)) \cap H^1_{0,*}(0,T;L^2(\Omega)), \\
H^{1,1}_{0;*,0}(Q):=L^2(0,T;H^1_0(\Omega)) \cap H^1_{*,0}(0,T;L^2(\Omega))\label{sp test}
\end{gather}
with the inner product \eqref{in prod} and the induced Hilbertian norm $|\cdot|_{H^1(Q)}$.

We can characterise the spaces \eqref{sp cod H10}, \eqref{sp ansatz}, \eqref{sp test} by the extended trace operator
\begin{gather*}
\gamma_{0,x}: L^2(0,T;H^1(\Omega)) \longrightarrow L^2(\Sigma), \\
\text{s.t.} \ \gamma_{0,x}v=v_{|\Sigma} \quad \text{for} \ v \in L^2(0,T;C(\overline{\Omega})),
\end{gather*}
which is a linear continuous function with the same continuity constant of the trace operator $\gamma_{0}: H^1(\Omega) \rightarrow H^{1/2}(\partial \Omega)$. Since the following relation holds
\begin{equation*}
\big\|\gamma_{0,x}v\big\|_{L^2(\Sigma)}=0 \Longleftrightarrow v \in L^2(0,T;H^1_0(\Omega)),
\end{equation*}
we have the representations
\begin{gather*}
H^{1,1}_{0;}(Q)=\Big\{ v \in H^1(Q): \big \|\gamma_{0,x}v\big \|_{L^2(\Sigma)}=0 \Big\},\\
H^{1,1}_{0;0,*}(Q)=\Big\{ v \in H^1(Q): \big\|\gamma_{0,x}v\big\|_{L^2(\Sigma)}=\|v(\cdot,0)\|_{L^2(\Omega)}=0 \Big\},\\
H^{1,1}_{0;*,0}(Q)=\Big\{ v \in H^1(Q):
\big\|\gamma_{0,x}v\big\|_{L^2(\Sigma)}=\|v(\cdot,T)\|_{L^2(\Omega)}=0 \Big\}.
\end{gather*}

The dual spaces $[H^{1,1}_{0;0,*}(Q)]'$ and $[ H^{1,1}_{0;*,0}(Q)]'$ are characterise as completions of $L^2(Q)$ with respect to their Hilbertian norm.