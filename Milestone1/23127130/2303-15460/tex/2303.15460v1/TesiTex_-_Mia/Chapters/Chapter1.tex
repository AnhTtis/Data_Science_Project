% Chapter 1

\chapter{Introduction} % Main chapter title

\label{Chapter1} % For referencing the chapter elsewhere, use \ref{Chapter1} 

%----------------------------------------------------------------------------------------

% Define some commands to keep the formatting separated from the content 
\newcommand{\keyword}[1]{\textbf{#1}}
\newcommand{\tabhead}[1]{\textbf{#1}}
\newcommand{\code}[1]{\texttt{#1}}
\newcommand{\file}[1]{\texttt{\bfseries#1}}
\newcommand{\option}[1]{\texttt{\itshape#1}}

%----------------------------------------------------------------------------------------

The most widely used approaches for the numerical solution of time-de\-pen\-dent linear partial differential equations are based on semi-discretization in space and time, using a time-stepping approach: at each fixed discrete time instant, a corresponding discretization in space is considered. The proposed methodology is applied to parabolic differential equations and to hyperbolic problems in Chapter 6 and Chapter 8 of (\cite{quarteroni2009modellistica}). A possible alternative is the simultaneous discretization in space and time, i.e., a space-time discretization. Space-time methods for time-dependent linear partial differential equations present some advantages when compared with more standard space discretization plus time-stepping. For example, the approximate solutions are available at all times in the interval of interest and it could also be possible to use techniques that are already been used for elliptical problems, such as multigrid methods (\cite{multigrid}) and domain decomposition (Chapter 14 of \cite{quarteroni2009modellistica}), which may allow parallelisation in time (\cite{timeparallel}). A possible drawback is that a global linear system must be solved at once. Therefore, fast solvers and preconditioning become essential. For instance, in the case of space-time isogeometric discretization for parabolic problems see (\cite{parabsang}).

Let us now consider the homogeneous Dirichlet problem for the second-order wave equation:
\begin{equation}\label{eqonde}
\begin{cases}
\partial_{tt}u(x,t)-\Delta_xu(x,t)=g(x,t) \quad (x,t) \in \Omega \times (0,T)\\
u(x,t)=0 \quad (x,t) \in \partial{\Omega} \times [0,T]\\
u(x,0)=\partial_tu(x,t)_{|t=0}=0 \quad x \in \Omega,
\end{cases}
\end{equation}
where $\Omega \subset \mathbb{R}^d$, with $d=1,2,3$, is an open bounded Lipschitz domain and, for a real value $T>0$, $(0,T)$ is a time interval. In (\cite{Coercive}) the authors introduce a space-time variational formulation of \eqref{eqonde}, where integration by parts is also applied with respect to the time variable, and the classic anisotropic Sobolev spaces with homogeneous initial and boundary conditions are employed. Stability of a conforming tensor-product space-time discretization of this variational formulation with piecewise polynomial, continuous solution and test functions, requires a Courant – Friedrichs – Lewy (CFL) condition, i.e., 
\begin{equation}\label{CFL}
h_t \leq C h_x,
\end{equation}
with a constant $C > 0$, depending on the constant of a spatial inverse inequality, where $h_t$ and $h_x$ are the (uniform) mesh-sizes in time and space, e.g, see (\cite{Steinbach2019, Zank2020}, for the piecewise linear case). The CFL condition is necessary for convergence while solving certain partial differential equations (usually hyperbolic PDEs) numerically. It arises in the numerical analysis of explicit time integration schemes. According to that, the time step must be less than a certain value, otherwise the simulation produces incorrect results. The condition is named after Richard Courant, Kurt Friedrichs, and Hans Lewy who described it in (\cite{CFL}).

Several approaches have been proposed in order to overcome restriction \eqref{CFL}. In (\cite{Steinbach2019}) the authors, following the work of (\cite{zlotnik}), introduce a perturbation of the tensor-product space-time piecewise linear discretization of \eqref{eqonde}. As a result, they can prove unconditional stability and optimal convergence rates in space-time norms. In particular, they start by considering the ordinary differential equation 
\begin{equation}\label{ode intro}
\partial_{tt}u(t)+ \mu u(t)=f(t), \quad \text{for} \ t \in (0,T), \quad u(0)=\partial_{t}u(t)_{|t=0}=0,
\end{equation}
where $\mu >0$, and its piecewise linear finite element discretization, since its stability is linked to the stability of the space-time standard FEM discretization of \eqref{eqonde}. They perturb the conforming discrete bilinear form, by considering the $L^2$ orthogonal projection on the piecewise constant finite element space. As a consequence, the new discrete system is unconditionally stable and convergence results hold without restrictions on the mesh-size. Finally, they extend this stabilization to the discretization for the wave-equation \eqref{eqonde}. In (\cite{higherorderzank}), M. Zank generalises this stabilization idea to an arbitrary polynomial degree with global continuity. In particular, he provides numerical examples for a one-dimensional spatial domain,
where the unconditional stability and optimal convergence rates in space-time norms are shown. On the other hand, theoretical considerations showing that such stabilization works are left to future papers. 

In (\cite{löscher2021numerical}) the authors consider a suitable linear transformation that defines an isomorphism between the anisotropic solution and test spaces. In this way they are able to define a Galerkin-Bubnov formulation that is unconditionally stable without further perturbations. In particular, the operator they use is the modified Hilbert transformation introduced in (\cite{Coercive, SteinbachZanknote, Zankexact}). However, in (\cite{löscher2021numerical}), they only give numerical
examples for a one- and a two-dimensional spatial domain, where the unconditional stability and optimal convergence rates in space-time norms are illustrated, and theoretical results are left to a future work. 

As it is proven in (\cite{Zank2020}), although the variational formulation to find the weak solution of \eqref{eqonde} in a suitable anisotropic Sobolev subspace $W$ of $H^1(\Omega \times (0,T))$ is well–defined for the right-hand-side $g$ being in the dual of the anisotropic Sobolev test space $V$, it is not possible to establish unique solvability. Indeed, the \textit{solution-to-data} operator between $W$ and $V'$ is not bijective. This is due to the fact that a stability condition, with respect to the dual norm of the right-hand-side, is not satisfied, see Theorem 4.2.24 of (\cite{Zank2020}). As a consequence, by the \textit{bounded inverse Theorem} (or \textit{inverse mapping Theorem}), the \textit{solution-to-data} linear map cannot be bijective. To ensure existence and uniqueness of a weak solution, we need to assume that $g \in L^2(\Omega \times (0,T))$. This is a
standard assumption to ensure sufficient regularity for the weak solution, and therefore, to obtain linear convergence for piecewise linear finite element approximations, but, as observed before, stability of common finite element discretizations require some CFL condition. In (\cite{steinbach2021generalized}) the authors introduce a new variational setting by enlarging the solution space. In this new framework they can prove that the \textit{solution-to-data} linear map is an isomorphism. Based on these results, they aim to derive a space–time finite element method for the numerical solution of the wave equation that is unconditionally stable. \\ \bigskip

The goal of this thesis is to investigate the first steps towards an unconditionally stable space-time isogeometric method with maximal regularity, using a tensor-product approach, for the wave problem \eqref{eqonde}. In particular, following (\cite{Steinbach2019}), the starting point is the analysis and the stabilization of the conforming discretization of \eqref{ode intro}. 

The choice of isogeometric methods can be advantageous due to the high degree of approximation of B-spline technology (\cite{HUGHES20084104, n-width}) and due to the exact representation of the geometry with non uniform rational B-splines (NURBS), which simplifies mesh refinement, as further communications with CAD are not necessary (\cite{isogeoCAD}). In particular, the choice of isogeometric methods with maximal regularity can be advantageous in the case of wave propagation problems to tackle the so-called \textit{pollution-effect}, which occurs in high-frequency wave problems (\cite{Babuska2000}). Indeed, a typical solution is to raise the order of the method: for the same number of degrees of freedom, methods that use piecewise polynomials of higher degree and regularity should perform better (\cite{HUGHES20084104}). Therefore, the choice of the IGA method with maximal regularity seems to be particularly suitable for this type of problems.

\subsubsection{Outline}
The rest of this thesis is organised as follows: in Chapter \ref{Chapter2} Sobolev spaces, spline spaces and variational methods are fixed and their most important properties are repeated. In Chapter \ref{Chapter3} the quadratic isogeometric method with maximal regularity for the ODE \eqref{ode intro} is investigated. In Chapter \ref{Chapter4} some numerical results are shown. In Chapter \ref{Chapter5} a short summary of the thesis and some suggestions for future work are given.