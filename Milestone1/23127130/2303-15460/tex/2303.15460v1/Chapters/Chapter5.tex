% Chapter 5

\chapter{Conclusions} % Main chapter title

\label{Chapter5} % For referencing the chapter elsewhere, use \ref{Chapter3} 

%----------------------------------------------------------------------------------------

% Define some commands to keep the formatting separated from the content 

%----------------------------------------------------------------------------------------

The goal of this thesis is to investigate the first steps towards an unconditionally stable space-time isogeometric method with maximal regularity, using a tensor-product approach, for the wave problem \eqref{eqonde}. 

Inspired by the work (\cite{Steinbach2019}), our starting point is studying the conditioned stability of the conforming quadratic IGA discretization with global $C^1(0,T)$ regularity of the related ordinary differential problem \eqref{eq ode}. In Chapter \ref{Chapter3} we hence obtain two explicit upper bounds on the mesh-size, which, if respected, guarantee stability. The first one \eqref{h bound}, i.e., 
\begin{equation*}
h \leq \frac{\pi^2}{\sqrt{2}(2+\sqrt{\mu}T)\mu T},
\end{equation*}
is an extension to quadratic IGA with maximal regularity of Theorem 4.7 of (\cite{Coercive}), which is a result for the piecewise continuous linear FEM discretization. In particular, our upper bound is about twice as large as the FEM one, and the discrete stability constant of \eqref{infsup zank}, i.e.,
\begin{equation*}
\beta_1(\mu,T):=\frac{2 \pi^2}{(2+\sqrt{\mu}T)^2(\pi^2+4\mu T^2)},
\end{equation*}
depends on the coefficient $\mu >0$ and on the final time $T>0$ of the ODE \eqref{eq ode} with the same order as the FEM one. The second upper bound \eqref{h bound gard}, i.e.,
\begin{equation*}
h \leq \frac{\pi^5}{(\pi^2+ 4\mu T^2)[\pi^2+2\mu T^2 (2+\sqrt{\mu} T)]} \sqrt{\frac{2b - \mu T^2}{2b(2+b)\mu}}
\end{equation*}
where $b > \frac{\mu T^2}{2}$ is an arbitrary fixed real value, is obtained by the theory of \textit{Galerkin method applied to G\aa rding-type problems}. 

The asymptotic case, i.e., $\mu$ that is significantly large, seems to us the most interesting situation for the problem of instability (Remark \ref{pollut effect}) and for the wave equation. Thus, in Remark \ref{oss asympt} we compare the obtained bounds and the corresponding stability constants for $\mu \rightarrow \infty$. It follows that, for ``a very large'' $\mu$, the first bound \eqref{h bound} is a weaker constraint than the second bound \eqref{h bound gard}. Moreover, some numerical results of Chapter \ref{Chapter4} show that the latter, with the optimal choice for $b$ \eqref{b}, can be a stronger constraint than the former even if $\mu$ is not extremely large.

Let us define 
\begin{equation*}
C_1(\mu,T):=\frac{1}{\beta_1(\mu,T)},
\end{equation*}
where $\beta_1(\mu,T)$ is the discrete inf-sup constant of \eqref{infsup zank} that we recall above. In Remark \ref{oss asympt} we note that the stability constant of \eqref{stab iga gard}, i.e.,
\begin{equation*}
C_2(\mu,T):=\Bigg[ \frac{3b + \mu T^2\big(\frac{8b}{\pi^2}-\frac{1}{2} \big)}{2b-\mu T^2} \Bigg] \quad \Bigg( \text{for a fixed} \ b > \frac{\mu T^2}{2} \Bigg),
\end{equation*}
that arises from the second upper bound \eqref{h bound gard} has a slower growth than $C_1(\mu,T)$ for $\mu \rightarrow \infty$. Thus, the theory of \textit{Galerkin method applied to G\aa rding-type problems} is useful for our problem leading to a lower bound of the discrete inf-sup that, asymptotically, is sharper than that obtained by extending the analysis of O. Steinbach and M. Zank (\cite{Coercive}). In Chapter \ref{Chapter4} there are also some numerical results showing that the inf-sup $\beta_2(\mu,T):=\frac{1}{C_2(\mu,T)}$ can be sharper than $\beta_1(\mu,T)$ even if $\mu$ is not significantly large.

In Chapter \ref{Chapter4} we observe that the two upper bounds \eqref{h bound}, \eqref{h bound gard} are not optimal. However, if the mesh-size is uniform, we manage to numerically find a stability constraint \eqref{emp constraint}, i.e., 
\begin{equation*}
h <  \sqrt{\frac{9}{\mu}},
\end{equation*}
which, from the numerical results that we obtain, seems to be sharp. The upper bound \eqref{emp constraint} is of the same order (w.r.t $\mu$) of the sharp upper bound \eqref{stab Linfty} for the stability of FEM discretization, i.e., 
\begin{equation*}
h < \sqrt{\frac{12}{\mu}}.
\end{equation*}
 
Quadratic IGA discretization of maximal regularity appears to be advantageous over piecewise continuous linear FEM. Indeed, the stability upper bounds on the mesh size of the former are very similar to the FEM ones, and the orders of convergence of the IGA method are of one order more than the FEM ones. Moreover, from the numerical tests we see that the error committed by the IGA is, for each mesh-size $h$, strictly smaller than the FEM one.\\ \vspace{0.8cm}

As observed in Remark \ref{oss grado più alto}, if we raise the degree of the splines to $p>2$ while keeping maximal regularity, the constraint on the mesh-size \eqref{h bound gard} and the resulting stability constant do not change. On the other hand, how the upper bound \eqref{h bound} behaves in $h$ is not immediately clear from the proof of Theorem \ref{teo stab IGA zank}. However, we are sure that the orders of convergence of the discrete solution to the exact solution will be $p$ in $|\cdot|_{H^1(0,T)}$ and $p+1$ in $\|\cdot\|_{L^2(0,T)}$, if the exact solution is sufficiently regular. Also, we could expect that the maximal error, and more generally for all values of $h$, is strictly smaller than the error committed by continuous, linear FEM and quadratic $C^1(0,T)$ IGA discretizations. This behaviour of the error would be a consequence of high degree of approximation of B-spline technology (\cite{HUGHES20084104, n-width}) and of the ``good behaviour'' of high-order methods with respect to wave propagation problems (\cite{Babuska2000}). These error considerations are indeed confirmed by our numerical tests in Figures \ref{fig: err grad 3}, \ref{fig: rel err grad 3}, \ref{fig: err grad 4}, \ref{fig: rel err grad 4}.

As in Chapter \ref{Chapter4}, as a numerical example for the Galerkin-Petrov finite element methods  we consider a uniform discretization of the time interval $(0,T)$ with $T = 10$ and a mesh-size $h = T/N$. For $\mu = 1000$ we consider the strong solution $u(t) = \sin^2\Big(\frac{5}{4}\pi t\Big)$
and we compute the integrals appearing at the right-hand side using high-order integration rules.

\begin{figure}
	\hspace{-1.3cm}
	\begin{minipage}[h!]{9cm}
		\centering
		\includegraphics[width=6.5cm]{Figures/err_ode_iga_grado3}
		\caption{A $\log$-$\log$ plot of errors committed by cubic IGA, with maximal regularity, in \\$|\cdot|_{H^1(0,T)}$ seminorm and in \\$\|\cdot\|_{L^2(0,T)}$ norm, with respect to a uniform mesh-size $h$. Also, the best approximation error in\\ $|\cdot|_{H^1(0,T)}$ seminorm and the bound \eqref{emp constraint} are represented. The square of the wave number is $\mu=1000$}
		\label{fig: err grad 3}
	\end{minipage} 
	\hspace{-1.3cm}
	\begin{minipage}[h!]{9cm}
		\centering
		\includegraphics[width=6.5cm]{Figures/rel_err_ode_iga_grado3}
		\caption{A $\log$-$\log$ plot of relative errors committed by cubic IGA, with maximal regularity, in $|\cdot|_{H^1(0,T)}$ seminorm and in \\$\|\cdot\|_{L^2(0,T)}$ norm, with respect to a uniform mesh-size $h$. Also, the best approximation error in\\ $|\cdot|_{H^1(0,T)}$ seminorm and the bound \eqref{emp constraint} are represented. The square of the wave number is $\mu=1000$}
		\label{fig: rel err grad 3}
	\end{minipage}
\end{figure}

\begin{figure}
	\hspace{-1.3cm}
	\begin{minipage}[h!]{9cm}
		\centering
		\includegraphics[width=6.5cm]{Figures/err_ode_iga_grado4}
		\caption{A $\log$-$\log$ plot of errors committed by IGA of fourth degree, with maximal regularity, in $|\cdot|_{H^1(0,T)}$ seminorm and in $\|\cdot\|_{L^2(0,T)}$ norm, with respect to a uniform mesh-size $h$. Also, the best approximation error in $|\cdot|_{H^1(0,T)}$ seminorm and the bound \eqref{emp constraint} are represented. The square of the wave number is $\mu=1000$}
		\label{fig: err grad 4}
	\end{minipage} 
	\hspace{-1.3cm}
	\begin{minipage}[h!]{9cm}
		\centering
		\includegraphics[width=6.5cm]{Figures/rel_err_ode_iga_grado4}
		\caption{A $\log$-$\log$ plot of relative errors committed by IGA of fourth degree, with maximal regularity, in $|\cdot|_{H^1(0,T)}$ seminorm and in $\|\cdot\|_{L^2(0,T)}$ norm, with respect to a uniform mesh-size $h$. Also, the best approximation error in $|\cdot|_{H^1(0,T)}$ seminorm and the bound \eqref{emp constraint} are represented. The square of the wave number is $\mu=1000$}
		\label{fig: rel err grad 4}
	\end{minipage}
\end{figure}

As one can note in Figures \ref{fig: err grad 3}, \ref{fig: rel err grad 3}, \ref{fig: err grad 4}, \ref{fig: rel err grad 4} the constraint \eqref{emp constraint} seems to remain optimal with respect to the error by raising the degree and regularity of the splines.

Although the errors diminish by raising the degree and regularity of the splines, it is of significant importance to find a method that is stable for every degree and regularity, so that we are not forced to work with dense matrices, which cause a high computational cost. 

Finally, let us note that we expect the IGA discretization of degree $p \geq 2$ and regularity $C^{p-1}(0,T)$ to perform better than the piecewise continuous FEM of degree $p$. Indeed, although the IGA matrices have more non-zero entries, the FEM discretization uses more basis functions. Furthermore, we expect that, while approximating solutions of wave propagation problems, the error committed by the IGA method is smaller than the FEM one (\cite{Babuska2000}).\\ \bigskip

Our proposals of stabilizations are all based on non-consistent perturbations.

In order to stabilize the quadratic IGA with maximal regularity, if the mesh-size is uniform, we propose to consider the perturbed bilinear form \eqref{tent 3 iga}, i.e.,
\begin{equation*}
\begin{split}
a_h(u_h,v_h)=-\langle \partial_t u_h, \partial_t v_h \rangle _{L^2(0,T)} + \mu \langle u_h, &v_h \rangle_{L^2(0,T)} \\
&- \delta \mu h^4\langle \partial_t^2 u_h, \partial_t^2 v_h \rangle _{L^2(0,T)},
\end{split}
\end{equation*}
for all $u_h \in V^h_{0,*}$ and $v_h \in V^h_{*,0}$, where $\delta >0$ is a fixed real value. Our numerical results are very promising in the case of $\delta=\frac{1}{100}$. It would therefore be interesting to study the theory that could explain why this stabilization works and then propose an optimal $\delta$. 

For IGA method with generic polynomial degree $p$ and maximal regularity, we propose to consider the following perturbed bilinear form
\begin{equation}\label{ah grado magg iga}
\begin{split}
a_h(u_h,v_h)=-\langle \partial_t u_h, \partial_t v_h \rangle _{L^2(0,T)} + \mu \langle u_h, &v_h \rangle_{L^2(0,T)} \\
&- \delta \mu h^{2p}\langle \partial_t^p u_h, \partial_t^p v_h \rangle _{L^2(0,T)},
\end{split}
\end{equation}
for all $u_h$ and $v_h$, respectively, in the discrete trial and test spaces, where $\delta >0$ is a fixed real value.

If the mesh-size is non-uniform, we suggest considering
\begin{equation}\label{ah iga mesh generica}
\begin{split}
a_h(u_h,v_h)=-\langle \partial_t u_h, \partial_t v_h \rangle _{L^2(0,T)} + \mu \langle u_h, &v_h \rangle_{L^2(0,T)} \\
&- \delta\mu\sum_{l=1}^{N} h_l^{2p} \langle \partial_t^p u_h, \partial_t^p v_h\rangle_{L^2(\tau_l)},
\end{split}
\end{equation}
for all $u_h$ and $v_h$, respectively, in the discrete trial and test spaces, where $\tau_l$, for $l=1, \ldots,N$, are the subintervals of $(0,T)$ given by the Galerkin discretization and $\delta >0$ is a fixed real value. Actually, \eqref{ah grado magg iga} is a subcase of \eqref{ah iga mesh generica}. 

Finally, let us briefly consider the homogeneous Dirichlet problem for the second-order wave equation \eqref{eqonde}, i.e., 
\begin{equation}
\begin{cases}
\partial_{tt}u(x,t)-\Delta_xu(x,t)=g(x,t) \quad (x,t) \in Q:=\Omega \times (0,T)\\
u(x,t)=0 \quad (x,t) \in \partial{\Omega} \times [0,T]\\
u(x,0)=\partial_tu(x,t)_{|t=0}=0 \quad x \in \Omega,
\end{cases}
\end{equation}
where $\Omega \subset \mathbb{R}^d$, with $d=1,2,3$, is an open bounded Lipschitz domain and, for a real value $T>0$, $(0,T)$ is a time interval. In (\cite{Coercive}) the authors introduce a space-time variational formulation of \eqref{eqonde}, where integration by parts is also applied with respect to the time variable, and the classic anisotropic Sobolev spaces with homogeneous initial and boundary conditions are employed. Inspired by (\cite{Steinbach2019, Zank2020}), a possible unconditionally stable space-time IGA method with maximal regularity based on a tensor-product approach could be obtained by considering the perturbed bilinear form
\begin{equation}\label{onde}
\begin{split}
a_h(u_h,v_h)=-\langle \partial_t u_h, \partial_t v_h \rangle _{L^2(Q)} + &\langle \nabla_x u_h, \nabla_x v_h \rangle_{L^2(Q)} \\
&- \delta\sum_{m=1}^{d} \sum_{l=1}^{N_t} h_l^{2p} \langle \partial_t^p \partial_{x_m} u_h, \partial_t^p \partial_{x_m} v_h\rangle_{L^2(\Omega \times \tau_l)},
\end{split}
\end{equation}
for all $u_h$ and $v_h$, respectively, in the discrete trial and test spaces, where $\tau_l$, for $l=1, \ldots,Nt$, are the subintervals of $(0,T)$ given by the Galerkin discretization and $\delta >0$ is a fixed real value. We expect this stabilization to perform well, given the appreciable numerical results of the perturbation \eqref{tent 3 iga}. However, we have not tested \eqref{onde} yet, since we prefer to give priority to a full analysis of the IGA discretization of our model problem \eqref{var ode}, which, given its link to the wave equation, we expect to be significantly useful.