% Chapter 4

\chapter{Numerical methods for $\partial_{tt}u+ \mu u=f$} % Main chapter title

\label{Chapter4} % For referencing the chapter elsewhere, use \ref{Chapter3} 

In this Chapter we numerically study our model problem \eqref{eq ode}. 

All the simulations are performed in MATLAB on a Intel(R) Core(TM) i3-4005U CPU @ 1.70GHz 1.70 GHz laptop, with 4,00 GB RAM.

All the isogeometric discretizations are performed using GeoPDEs, which is an open source and free package for the research and teaching of Isogeometric Analysis, written in Octave and fully compatible with MATLAB. See (\cite{GeoPDEs}) for a complete explanation of its design and its main features.

%----------------------------------------------------------------------------------------

% Define some commands to keep the formatting separated from the content 

%----------------------------------------------------------------------------------------

\section{Conditioned stability}

\subsection{Errors committed by piecewise continuous linear finite element method}\label{sec errors fem}

In (\cite{Steinbach2019, Zank2020, Coercive}) the authors study the conditioned stability of the piecewise continuous linear finite element discretization of \eqref{eq ode} and introduce a stabilized method (\cite{Steinbach2019, Zank2020}), which they then extend to the wave equation \eqref{eqonde} (\cite{Steinbach2019, Zank2020}). Indeed, as noticed in (\cite{Steinbach2019, Zank2020}), the stability of a conforming (w.r.t. classic anisotropic Sobolev spaces) tensor-product space-time discretization with pie\-ce\-wise linear, continuous solution and test functions of \eqref{eqonde}, requires a Cou\-rant – Friedrichs – Lewy (CFL) condition, i.e., 
\begin{equation}\label{CFL 2}
h_t \leq C h_x,
\end{equation}
with a constant $C > 0$, depending on the constant of a spatial inverse inequality, where $h_t$ and $h_x$ are the uniform mesh-sizes in time and space. In particular, constraint \eqref{CFL 2} follows from the conditioned stability of the piecewise continuous linear FEM applied to \eqref{eq ode} with uniform mesh-size. 

In this Section we briefly recall O. Steinbach and M. Zank's main results for the conditioned stability of FEM discretization of the ODE \eqref{eq ode} and we show some numerical results that we obtain by testing their theoretical considerations. 

Let us define the discrete spaces
\begin{gather}
S^1_{h;0,*}:=\{v_h \in S^1_h(0,T)| \ v_h(0)=0\}=S^1_h(0,T) \cap H^1_{0,*}(0,T), \label{fem trial}\\
S^1_{h;*,0}:=\{v_h \in S^1_h(0,T)| \ v_h(T)=0\}=S^1_h(0,T) \cap H^1_{*,0}(0,T), \label{fem test}
\end{gather}
where $S^1_h(0,T)$ is the classic space of piecewise continuous linear functions on $[0,T]$ with maximal mesh-size $h$.

In Theorem 4.7 of (\cite{Coercive}), which we extend to the isogeometric case with Theorem \ref{teo stab IGA zank}, the authors prove that if the mesh-size $h$ satisfies 
\begin{equation}\label{zank 1 stima}
h \leq \frac{2\sqrt{3}}{(2+\sqrt{\mu}T)\mu T},
\end{equation}
 then the discrete Galerkin-Bubnov formulation of \eqref{eq ode} is well-posed with a uniform (w.r.t $h$) lower bound on the discrete inf-sup, i.e.,
 \begin{equation}\label{infsup fem}
 \frac{8}{(2+\sqrt{\mu}T)^2(4+\mu T^2)} |u_h|_{H^1(0,T)} \leq \sup_{0 \neq v_h \in S^1_{h;0,*}} \frac{a(u_h,\overline{\mathcal{H}}_T v_h)}{|v_h|_{H^1(0,T)}},
 \end{equation}
 for all $u_h \in S^1_{h;0,*}$, where $a(\cdot,\cdot)$ is the bilinear form defined in \eqref{bil ode} and $\overline{\mathcal{H}}_T(\cdot)$ is the isometric isomorphism defined in \eqref{HT}, and where we corrected a missing second power of $\frac{T}{2}$.
 
 \begin{oss}
 	As a consequence of the sharp Poincaré's inequalities \eqref{Poinc}, the estimate \eqref{zank 1 stima} can be slightly improved with the more accurate bound 
 	\begin{equation}\label{zank più sharp}
 	h \leq \frac{\sqrt{3}\pi}{\sqrt{2}(2+\sqrt{\mu} T)\mu T},
 	\end{equation}	
 	as observed in (\cite{Zank2020}). Therefore, we consider this bound in our numerical experiments.
 \end{oss}

\begin{oss}
	The well-posedness of the Galerkin-Bubnov FEM discretization of \eqref{eq ode} is equivalent to the well-posedness (with the same stability constants) of its Galerkin-Petrov FEM discretization. Indeed, as in the isogeometric setting, the restriction of operator  $\overline{\mathcal{H}}_T(\cdot)$ to the discrete trial space $S^1_{h;0,*}$ is actually an isometric isomorphism between $S^1_{h;0,*}$ and the test space $S^1_{h;*,0}$.
\end{oss}

With classic arguments, in (\cite{Coercive}) the authors prove that the discrete solution of the piecewise continuous linear FEM discretization of \eqref{eq ode} converges linearly in $|\cdot|_{H^1(0,T)}$ to the solution $u$ of \eqref{var ode} if $u \in H^2(0,T)$ (Theorem 4.8, \cite{Coercive}). Moreover, using Aubin-Nitsche's trick, it is also possible to prove quadratic convergence in $\|\cdot\|_{L^2(0,T)}$ norm, if the exact solution satisfies $u \in H^2(0,T)$. \\ \bigskip

Under the assumption of a uniform mesh-size, the linear system obtained from the Galerkin-Petrov FEM discretization of \eqref{eq ode} can be seen as a finite difference scheme; see Remark 4.2.8 of (\cite{Zank2020}). In these condition, the stability of the corresponding finite difference scheme holds if and only if
\begin{equation}\label{stab Linfty}
h < \sqrt{\frac{12}{\mu}},
\end{equation}
as a consequence of Chapter III.3 of (\cite{Hairer}). Let us note that the stability considered in \eqref{stab Linfty} is a uniform (w.r.t. $h$) boundness condition for the discrete solution in $\|\cdot\|_{L^\infty(0,T)}$ norm, which is necessary for stability of the discrete solution in classic Sobolev seminorms, due to the fact that the norm $\|\cdot\|_{L^\infty(0,T)}$ is controlled in $H^1(0,T)$, as a consequence of Morrey's Theorem.

\begin{oss}
Constraint \eqref{stab Linfty} is related to the CFL condition for the tensor-product space-time discretization with piecewise continuous linear solution and test functions of the wave propagatin problem \eqref{eqonde}. The details are explained in (\cite{Steinbach2019, Zank2020}). 
\end{oss} 

\vspace{0.8cm}
 As in (\cite{Steinbach2019, Zank2020}), as a numerical example for the Galerkin-Petrov finite element methods  we consider a uniform discretization of the time interval $(0,T)$ with $T = 10$ and a mesh-size $h = T/N$. For $\mu = 1000$ we consider the strong solution $u(t) = \sin^2\Big(\frac{5}{4}\pi t\Big)$
and we compute the integrals appearing at the right-hand side using high-order integration rules.

\begin{figure}[h!]	
	\centering
	\includegraphics[scale=0.3]{Figures/hbound_fem}
	\caption{A $\log$-$\log$ plot of errors committed by piecewise continuous linear FEM in $|\cdot|_{H^1(0,T)}$ seminorm and in $\|\cdot\|_{L^2(0,T)}$ norm, with respect to a uniform mesh-size $h$. Also, the best approximation error in $|\cdot|_{H^1(0,T)}$ seminorm and bounds \eqref{zank più sharp}, \eqref{stab Linfty} are represented. The square of the wave number is $\mu=1000$ and the final time is $T=10$.}
	\label{fig:err fem}
\end{figure}

\begin{figure}[h!]
	\centering
	\includegraphics[scale=0.3]{Figures/rel_err_ode_fem}
	\caption{A $\log$-$\log$ plot of relative errors committed by piecewise continuous linear FEM in $|\cdot|_{H^1(0,T)}$ seminorm and in $\|\cdot\|_{L^2(0,T)}$ norm, with respect to a uniform mesh-size $h$. Also, the best approximation error in $|\cdot|_{H^1(0,T)}$ seminorm and bounds \eqref{zank più sharp}, \eqref{stab Linfty} are represented. The square of the wave number is $\mu=1000$ and the final time is $T=10$.}
\end{figure}

\begin{figure}
	\hspace{-1cm}
	\begin{minipage}[h!]{8.5cm}
		\centering
		\includegraphics[width=6.5cm]{Figures/uex_udis_ode_fem_N64}
		\caption{Exact and discrete solutions by piecewise linear FEM for $\mu = 1000$ and $N = 64$ elements (i.e., $h = 0.1563$).}
	\end{minipage} 
	\hspace{-1cm}
	\begin{minipage}[h!]{8.5cm}
		\centering
		\includegraphics[width=6.5cm]{Figures/uex_udis_ode_fem}
		\caption{Exact and discrete solutions by piecewise linear FEM for $\mu = 1000$ and $N=32768$ elements (i.e., $h=0.0003$).}
	\end{minipage}
\end{figure}

We consider approximation errors since they are a reflection of instability. 

The minimum number of elements chosen is $N = 4$, the maximum number is $N=32768$, as in (\cite{Steinbach2019}). 

From Figure \ref{fig:err fem}, we can see that the maximal error occurs at $N = 64$: it is of order of $10^{23}$ in seminorm $|\cdot|_{H^1(0,T)}$ and $10^{22}$ in norm $\|\cdot\|_{L^2(0,T)}$, as we expect from (\cite{Steinbach2019}). 

As we can note in Figure \ref{fig:err fem}, there is convergence only for sufficiently small mesh-size $h$.  In particular, convergence is linear in $|\cdot|_{H^1(0,T)}$ seminorm and is quadratic in $\|\cdot\|_{L^2(0,T)}$ norm, as we expect. Clearly, the bound \eqref{zank più sharp} is suboptimal, since convergence starts for $h$ much larger than this threshold. Instead, the bound \eqref{stab Linfty} seems to be sharp with respect to the error committed by the finite elements: this suggests that there is a uniform (w.r.t. $h$) inf-sup value already for $h < \sqrt{\frac{12}{\mu}}$. Thus, it remains open to improve assumption \eqref{zank più sharp} to ensure a uniform inf-sup condition of \eqref{infsup fem} type. 

\begin{oss}\label{pollut effect}
Let us define $k:=\sqrt{\mu}$ to follow the notation used in the literature we refer to in this comment. It is well known that the numerical solution of the Helmholtz equation 
\begin{equation*}
-\Delta_x u(x)-k^2 u(x) = f(x)
\end{equation*}
with boundary conditions, obtained by classic Galerkin FEM, differs significantly from the best approximation with increasing wave number $k$. This phenomena is the so-called \textit{pollution effect}. The Galerkin FEM leads to quasi-optimal error estimates in which the constant factor by which the accuracy of the Galerkin solution differs from the best approximation error increases with increasing wave number (\cite{HARARI199159, IHLENBURG19959, BABUSKA1995325, Babuska2000}). On the other hand, it was shown in (\cite{Aziz1988ATP}) that the condition “$k^2 h$ is small” would be sufficient to guarantee that the error of the Galerkin solution is of the same magnitude as the error of the best approximation. However, this condition involves considerable computational complexity in three dimensions (\cite{Babuska2000}). Therefore, many attempts have been made in the mathematical and engineering literature to overcome this lack of robustness of the classic Galerkin FEM with respect to $k$ (\cite{HARARI199159,BABUSKA1995325}).

Our model problem \eqref{eq ode} is a wave propagation problem of Helmholtz type with initial conditions, instead of boundary conditions. Therefore, morally, we could expect ``a certain pollution effect''. Hence, we decide to plot the best approximation error in order to estimate some kind of pollution error, which could be related to the conditioned stability of the classic Galerkin FEM. 

Note that, unlike Helmholtz, in our model problem \eqref{eq ode} we have two types of error-source frequencies. The first one is the frequency of the equation operator, i.e., $\sqrt{\mu}=\sqrt{1000}$ in our example, which dictates the number of oscillations in the unit time of typical solutions (e.g. homogeneous) of the problem. The second one is the specific frequency of the solution we consider, i.e., $f := \frac{\omega}{2\pi} = \frac{10 \pi}{4}\frac{1}{2 \pi}=\frac{5}{4}$ in our example, which dictates the number of oscillations in the unit time of this specific solution. Only the former is the source of a certain pollution error for this problem, whereas the latter is related to the best approximation error. However, both are related to the choice of the mesh-size in order to get a discrete space whose Galerkin error is ``small''.
\end{oss}

\subsection{Inf-sup tests for piecewise continuous linear finite element method}\label{sec infsup fem}

As noticed in Section \ref{sec errors fem}, the bound \eqref{stab Linfty} seems to be sharp with respect to the error committed by the finite elements: this suggests that there is a uniform (w.r.t. $h$) inf-sup value already for $h < \sqrt{\frac{12}{\mu}}$. Therefore, we make inf-sup tests for the discrete bilinear form of the piecewise continuous linear FEM discretization. The idea is to numerically estimate the discrete inf-sup and visualize its behaviour with respect to $(\mu,h)$: we expect that in the region satisfying $h \geq \sqrt{\frac{12}{\mu}}$ there are infinitesimal inf-sup values, whereas we expect a uniformly (w.r.t. $h$) limited behaviour in the complementary region. \\ \bigskip

\textbf{Numerical estimate of the discrete inf-sup.} Let us define:
\begin{equation}\label{beta dis}
\beta(\mu,h,T):=\inf_{u_h \in S^1_{h;0,*}} \sup_{v_h \in S^1_{h;*,0}} \frac{a(u_h,v_h)}{|u_h|_{H^1(0,T)} |v_h|_{H^1(0,T)}},
\end{equation}
where $a(\cdot,\cdot)$ is the bilinear form defined in \eqref{bil ode}. Let us also define the matrices $\mathbf{A},\mathbf{H}_1,\mathbf{H}_2$ such that 
\begin{gather}
\begin{split}
[\mathbf{A}]_{i,j}:=-\langle \partial_t b_{1,j},\partial_t b_{1,i} \rangle_{L^2(0,T)} &+ \mu \langle  b_{1,j}, b_{1,i} \rangle_{L^2(0,T)}\\ \quad \text{for} \ &i=1,\ldots,M-1, \ j=2,\ldots,M 
\end{split}
\\{[\mathbf{H}_1]}_{i,j}:=\langle \partial_t b_{1,i},\partial_t b_{1,j} \rangle_{L^2(0,T)} \quad \text{for} \ i,j=2,\ldots,M, \\
{[\mathbf{H}_2]}_{i,j}:=\langle \partial_t b_{1,i},\partial_t b_{1,j} \rangle_{L^2(0,T)} \quad \text{for} \ i,j=1,\ldots,M-1,
\end{gather}
where $b_{1,k}$ for $k=1,\ldots,M$ are the classic \textit{hat functions} such that $S^1_h(0,T)=\text{span}\{b_{1,k}\}_{k=1}^M$. Through this matrices we can estimate the discrete inf-sup value \eqref{beta dis}.

\begin{prop}\label{estimate infsup}
	The discrete inf-sup value \eqref{beta dis} satisfies
	\begin{equation*}
	\beta(\mu,h,T)=\sqrt{\lambda_{min}},
	\end{equation*}
	where $\lambda_{min} \geq 0$ is the minimum eigenvalue of the generalised eigenvalue problem:
	\begin{equation*}
	\mathbf{A}^T \mathbf{H}_2^{-1}\mathbf{A}\overset{\rightarrow}{x}=\lambda\mathbf{H}_1 \overset{\rightarrow}{x}, \quad \text{for some} \ \overset{\rightarrow}{x} \in \mathbb{R}^{M-1}.
	\end{equation*} 
\end{prop}

\begin{proof}
	Let $u_h \in S^1_{h;0,*}$ and $v_h \in S^1_{h;*,0}$. They can be represented as
	\begin{gather*}
	u_h=\sum_{j=2}^M u_j b_{1,j},\\
	v_h=\sum_{i=1}^{M-1} v_i b_{1,i}.
	\end{gather*}
	Let us define $\overset{\rightarrow}{u}:=(u_2,\ldots,u_M)^T \in \mathbb{R}^{M-1}$, $\overset{\rightarrow}{v}:=(v_1,\ldots,v_{M-1})^T \in \mathbb{R}^{M-1}$. Therefore,
    \begin{equation}\label{vett}
    \begin{cases}
    a(u_h,v_h)=\overset{\rightarrow}{u}^T\mathbf{A}^T \overset{\rightarrow}{v},\\
    |u_h|^2_{H^1(0,T)}=\overset{\rightarrow}{u}^T\mathbf{H}_1 \overset{\rightarrow}{u},\\
    |v_h|^2_{H^1(0,T)}=\overset{\rightarrow}{v}^T\mathbf{H}_2 \overset{\rightarrow}{v}.
    \end{cases}
    \end{equation}
    In order to lighten the notation we will denote $\overset{\rightarrow}{u}$ with $u$ and $\overset{\rightarrow}{v}$ with $v$. By \eqref{vett} the following equality holds
    \begin{equation*}
    \inf_{u_h \in S^1_{h;0,*}} \sup_{v_h \in S^1_{h;*,0}} \frac{a(u_h,v_h)}{|u_h|_{H^1(0,T)} |v_h|_{H^1(0,T)}}= \inf_{u \in \mathbb{R}^{M-1}} \sup_{v \in \mathbb{R}^{M-1}}\frac{{u}^T\mathbf{A}^T {v}}{\sqrt{u^T\mathbf{H}_1 {u}} \sqrt{{v}^T\mathbf{H}_2 {v}}}.
    \end{equation*}
    $\mathbf{H}_2$ is a symmetric real positive-definite matrix, thus $\mathbf{H}_2=\mathbf{H}_2^\frac{1}{2}\mathbf{H}_2^\frac{1}{2}$. With the change of variable $\tilde{v}:=\mathbf{H}_2^\frac{1}{2}v$, the following equalities hold:
    \begin{equation*}
    \sup_{v \in \mathbb{R}^{M-1}}\frac{{u}^T\mathbf{A}^T {v}}{ \sqrt{{v}^T\mathbf{H}_2 {v}}}=\sup_{\tilde{v} \in \mathbb{R}^{M-1}}\frac{{u}^T\mathbf{A}^T \mathbf{H}_2^{-\frac{1}{2}} \tilde{v}}{ \sqrt{\tilde{v}^T \tilde{v}}} = \Big\|\mathbf{H}_2^{-\frac{1}{2}} \mathbf{A} u\Big\|_2,
    \end{equation*}
    where $\|\cdot\|_2$ denote the euclidean norm of a vector. 
    
    $\mathbf{H}_1$ is a symmetric real positive-definite matrix, thus $\mathbf{H}_1=\mathbf{H}_1^\frac{1}{2}\mathbf{H}_1^\frac{1}{2}$. With the change of variable $\tilde{u}:=\mathbf{H}_1^\frac{1}{2}u$, the following equalities hold:
    \begin{equation*}
    \begin{split}
    \Big\|\mathbf{H}_2^{-\frac{1}{2}} \mathbf{A} u\Big\|_2^2&=u^T\mathbf{A}^T\mathbf{H}_2^{-\frac{1}{2}}\mathbf{H}_2^{-\frac{1}{2}} \mathbf{A} u=u^T\mathbf{A}^T\mathbf{H}_2^{-1} \mathbf{A} u\\
    &=\tilde{u}^T\mathbf{H}_1^{-\frac{1}{2}}\mathbf{A}^T\mathbf{H}_2^{-1} \mathbf{A} \mathbf{H}_1^{-\frac{1}{2}} \tilde{u}.
    \end{split}
    \end{equation*}
    Therefore,
    \begin{equation*}
    \begin{split}
    \beta(\mu,h,T)&=\inf_{\tilde{u} \in \mathbb{R}^{M-1}}\sqrt{\frac{\tilde{u}^T\mathbf{H}_1^{-\frac{1}{2}}\mathbf{A}^T\mathbf{H}_2^{-1} \mathbf{A} \mathbf{H}_1^{-\frac{1}{2}} \tilde{u}}{\tilde{u}^T\tilde{u}}}\\
    &=\sqrt{\inf_{\tilde{u} \in \mathbb{R}^{M-1}} \frac{\tilde{u}^T\mathbf{H}_1^{-\frac{1}{2}}\mathbf{A}^T\mathbf{H}_2^{-1} \mathbf{A} \mathbf{H}_1^{-\frac{1}{2}} \tilde{u}}{\tilde{u}^T\tilde{u}}}\\
    &=\sqrt{\lambda_{min}},
    \end{split}
    \end{equation*}
    where $\lambda_{min}$ is the minimum eigenvalue of the symmetric real positive-semide\-fi\-nite matrix $\mathbf{H}_1^{-\frac{1}{2}}\mathbf{A}^T\mathbf{H}_2^{-1} \mathbf{A} \mathbf{H}_1^{-\frac{1}{2}}$. This means that $\lambda_{min}$ is the smallest number that satisfies
    \begin{equation*}
    \mathbf{H}_1^{-\frac{1}{2}}\mathbf{A}^T\mathbf{H}_2^{-1} \mathbf{A} \mathbf{H}_1^{-\frac{1}{2}} \overset{\rightarrow}{x} = \lambda \overset{\rightarrow}{x}, \quad \text{for some} \ \overset{\rightarrow}{x} \in \mathbb{R}^{M-1},
    \end{equation*}
    i.e.,
    \begin{equation*}
    \mathbf{A}^T\mathbf{H}_2^{-1} \mathbf{A} \mathbf{H}_1^{-\frac{1}{2}} \overset{\rightarrow}{x} = \lambda \mathbf{H}_1^{\frac{1}{2}} \overset{\rightarrow}{x}, \quad \text{for some} \ \overset{\rightarrow}{x} \in \mathbb{R}^{M-1}.
    \end{equation*}
    In order to lighten the notation we will denote $\overset{\rightarrow}{x}$ with $x$. With the change of variable $\tilde{x}:=\mathbf{H}_1^{-\frac{1}{2}} x$, $\lambda_{min}$ is the minimum of 
    \begin{equation*}
    \mathbf{A}^T\mathbf{H}_2^{-1} \mathbf{A} \tilde{x}= \lambda \mathbf{H}_1 \tilde{x}, \quad \text{for some} \ \tilde{x} \in \mathbb{R}^{M-1}.
    \end{equation*}
     The thesis is therefore proven.
\end{proof}

\vspace{0.8cm}

We fix the final time $T=10$ and a uniform mesh. We numerically study the behaviour of $\beta(\mu,h)$ of \eqref{beta dis} with respect to $(\mu,h)$ by means of a p-colour plot of $\log(\beta)$ depending on $(\log(\mu),\log(h))$, so as to visualize the development of $\beta(\mu,h)$ more effectively. 

\begin{figure}[h!]
	\centering
	\includegraphics[scale=0.5]{Figures/image_nonstab_FEM_20x20}
	\caption{A p-color plot of $\log(\beta)$ of piecewise continuous linear FEM with respect to $(\log(\mu),log(h))$. The red line is the natural logarithm of the upper bound in \eqref{stab Linfty}.}
	\label{fig:beta_fem}
\end{figure}

Firstly, let us note that the MATLAB function \textit{p-color} sets by default the MATLAB values \textit{-Inf} (i.e., numbers whose absolute value is too large to be represented as conventional floating-point values) to dark blue: we have checked this numerically and it is also clarified by Figure \ref{fig:3D fem}, where the empty regions correspond to $log(\beta)=-Inf$.

\begin{figure}[h!]
	\centering
	\includegraphics[scale=0.3]{Figures/FEM_nonstab_3D}
	\caption{On the left, a three-dimensional plot of $\log(\beta)$ of piecewise continuous linear FEM with respect to $(\log(\mu),\log(h))$. On the right, the contour lines of $\log(\beta)$ with respect to $(\log(\mu),\log(h))$.}
	\label{fig:3D fem}
\end{figure}

The red line of Figure \ref{fig:beta_fem} is the natural logarithm of the upper bound in \eqref{stab Linfty}. We can clearly see in Figure \ref{fig:beta_fem} that it is the separation margin of two regions in which we observe a significantly different behaviour of $\beta(\mu,h)$. In the region below the red line, i.e., for $h < \sqrt{\frac{12}{\mu}}$, we observe a uniformly (w.r.t. $h$) bounded $\beta(\mu,h)$, whereas in the region above the red line, i.e., for $h > \sqrt{ \frac{12}{\mu}}$, we observe a predominantly infinitesimal $\beta(\mu,h)$. This behaviour of the discrete inf-sup is indeed what we expect from Figure \ref{fig:err fem}, in which the bound \eqref{stab Linfty} on $h$ seemed to be sharp, suggesting to us a uniformly (w.r.t. $h$) bounded discrete inf-sup already for $h < \sqrt{\frac{12}{\mu}}$. In the stability region, we can see a dependency of the discrete inf-sup on $\mu$ of order $\mu^{-\frac{1}{2}}$, as noticed in (\cite{Zank2020}).

Note that in Figure \ref{fig:beta_fem} we do not consider the values of discrete inf-sup in the upper-right white region. In this region, the values of $\log(\beta)$ are greater than $10^2$. Actually, as a consequence of $h$ ``coarse'' and ``high'' wave-number, such large values for the discrete inf-sup are natural results, since value \eqref{beta dis} is directly proportional to $\mu$ for wave-number and mesh-size that are ``very large''. This is due to inverse inequalities which allow the term $L^2$ to dominate the derivatives. We are not interested in working under these conditions, since, for those wave numbers, $h$ is too coarse compared to the resolution we expect to need in order to obtain satisfactory numerical results. Therefore, we do not visualize the corresponding inf-sup values.

\subsection{Errors committed by quadratic isogeometric discretization with maximal regularity}\label{sec errs iga}

In this Section we show some numerical results that we obtain by testing our theoretical considerations of Section \ref{sec iga}. 

As in Section \ref{sec errors fem} and in (\cite{Steinbach2019, Zank2020}), as a numerical example for the Galerkin-Petrov quadratic isogeometric discretization with maximal regularity, we consider a uniform discretization of the time interval $(0,T)$ with $T = 10$ and a mesh-size $h = T/N$. For $\mu = 1000$ we consider the strong solution $u(t) = \sin^2\Big(\frac{5}{4}\pi t\Big)$
and we compute the integrals appearing at the right-hand side using high-order integration rules. 

As in Section \ref{sec errors fem} we consider approximation errors since they are a reflection of instability. 

The minimum number of elements chosen is $N = 4$, the maximum number is $N=4096$ (it is not $N = 32768$ due to the memory limits of the laptop used).


\begin{figure}[h!]	
	\centering
	\includegraphics[scale=0.3]{Figures/hbound_confronto}
	\caption{A $\log$-$\log$ plot of errors committed by quadratic IGA, with maximal regularity, in $|\cdot|_{H^1(0,T)}$ seminorm and in $\|\cdot\|_{L^2(0,T)}$ norm, with respect to a uniform mesh-size $h$. Also, the best approximation error in $|\cdot|_{H^1(0,T)}$ seminorm and bounds \eqref{h bound}, \eqref{h bound gard} (with the optimal choice \eqref{b}) are represented. The square of the wave number is $\mu=1000$ and the final time is $T=10$.}
	\label{fig:err iga}
\end{figure}

\begin{figure}[h!]	
	\centering
	\includegraphics[scale=0.3]{Figures/rel_err_ODE_iga}
	\caption{A $\log$-$\log$ plot of relative errors committed by quadratic IGA, with maximal regularity, in $|\cdot|_{H^1(0,T)}$ seminorm and in $\|\cdot\|_{L^2(0,T)}$ norm, with respect to a uniform mesh-size $h$. Also, the best approximation error in $|\cdot|_{H^1(0,T)}$ seminorm and bounds \eqref{h bound}, \eqref{h bound gard} (with the optimal choice \eqref{b}) are represented. The square of the wave number is $\mu=1000$ and the final time is $T=10$.}
	\label{}
\end{figure}

From Figure \ref{fig:err iga}, you can see that the maximal error occurs at $N = 64$: it is of order of $10^{16}$ in seminorm $|\cdot|_{H^1(0,T)}$ and $10^{15}$ in norm $\|\cdot\|_{L^2(0,T)}$. Note that the maximal error, and more generally for all values of $h$, is strictly smaller than the error committed by piecewise continuous linear FEM discretization, see Section \ref{sec errors fem}. This result can be interpreted as a consequence of the good approximation properties of B-spline technology (\cite{HUGHES20084104, n-width}).

\begin{figure}
	\hspace{-1cm}
	\begin{minipage}[h!]{8.5cm}
		\centering
		\includegraphics[width=6.5cm]{Figures/uex_udis_ode_iga_N64}
		\caption{Exact and discrete solutions by quadratic IGA for $\mu = 1000$ and $N = 64$ elements (i.e., $h = 0.1563$).}
	\end{minipage} 
	\hspace{-1cm}
	\begin{minipage}[h!]{9cm}
		\centering
		\includegraphics[width=6.5cm]{Figures/uex_udis_ode_IGA}
		\caption{Exact and discrete solutions by quadratic IGA for $\mu = 1000$ and $N=4096$ elements (i.e., $h=0.0024$).}
	\end{minipage}
\end{figure}


As you can see in Figure \ref{fig:err iga}, there is convergence only for sufficiently small mesh-size $h$.  In particular, convergence is quadratic in $|\cdot|_{H^1(0,T)}$ seminorm, as we expect from Section \ref{sec iga}, and it is cubic in $\|\cdot\|_{L^2(0,T)}$ norm. Note that in our theoretical discussion of Section \ref{sec iga} we did not estimate the order of convergence in $\|\cdot\|_{L^2(0,T)}$ norm, since it was beyond our interest. However, using Aubin-Nitsche's trick, one can verify that the $L^2$-error converges cubically if the exact solution $u$ satisfies $u \in H^3(0,T)$. 

Clearly, both bounds \eqref{h bound}, \eqref{h bound gard} (with the optimal choice \eqref{b}) are suboptimal, since a beginning of convergence is observed for $h$ much larger than these thresholds, see Figure \ref{fig:err iga}. Moreover, the constraint \eqref{h bound gard} is more stringent than \eqref{h bound}, as we expect to happen asymptotically (i.e., $\mu \rightarrow \infty$) from Remark \ref{oss asympt}. Thus, it remains open to improve assumptions \eqref{h bound}, \eqref{h bound gard} to ensure stability conditions of \eqref{stab zank}, \eqref{stab iga gard} type, respectively. That is, we are interested in an upper bound on the mesh-size that provides us with stability and is sharp. 

\subsection{Inf-sup tests for quadratic isogeometric discretization with maximal regularity}

In this Section we numerically estimate the discrete inf-sup value and visualize its behaviour with respect to $(\mu,h)$, with a uniform mesh-size $h$ and final instant $T=10$. Since the development of the error committed by the quadratic IGA is similar to that of the linear FEM, we expect the behaviour of the discrete inf-sup to be similar as well. In particular, if we numerically study $\beta(\mu,h)$ of \eqref{beta dis} with respect to $(\mu,h)$ by means of a p-colour plot of $\log(\beta)$ depending on $(\log(\mu),\log(h))$, we expect to visualize a line that is the separation margin of stability and instability regions. This line corresponds to the sharp bound on $h$ that, if satisfied, ensures the desired stability.

\begin{figure}[h!]
	\centering
	\includegraphics[scale=0.5]{Figures/image_nonstab_IGA_20x20}
	\caption{A p-color plot of $\log(\beta)$ of quadratic IGA with maximal regularity with respect to $(\log(\mu),log(h))$. The red line represents $\log(h)=\frac{1}{2}\log(9)-\frac{1}{2}\log(\mu)$.}
	\label{fig:beta_iga}
\end{figure}

\begin{figure}[h!]
	\centering
	\includegraphics[scale=0.3]{Figures/IGA_nonstab_3D}
	\caption{On the left, a three-dimensional plot of $\log(\beta)$ of quadratic IGA with maximal regularity with respect to $(\log(\mu),\log(h))$. On the right, the contour lines of $\log(\beta)$ with respect to $(\log(\mu),\log(h))$.}
	\label{fig:3D iga}
\end{figure}

\bigskip
We estimate the isogeometric discrete inf-sup using Proposition \ref{estimate infsup} with the discrete solution and test spaces defined in \eqref{iga space}, \eqref{iga test space}, respectively. 

The numerical results in Figure \ref{fig:beta_iga} and Figure \ref{fig:3D iga} confirm what we expect. The red line of Figure \ref{fig:beta_iga} corresponds to the constraint
\begin{equation}\label{emp constraint}
h < \sqrt{\frac{9}{\mu}}.
\end{equation}
Indeed, in the region below the red line, i.e., for $h < \sqrt{\frac{9}{\mu}}$, we observe a uniformly (w.r.t. $h$) bounded $\beta(\mu,h)$, whereas in the region above the red line, i.e., for $h > \sqrt{\frac{9}{\mu}}$, we observe a predominantly infinitesimal $\beta(\mu,h)$. In the stability region, we can see a dependency of the discrete inf-sup on $\mu$ of order $\mu^{-1/2}$, as in the continuous linear FEM case of Section \ref{sec infsup fem}. %which guarantees a slower decrease of the discrete quadratic IGA inf-sup (w.r.t. $\mu \rightarrow \infty$) than that of the FEM case observed in Section \ref{sec infsup fem}.

In Figure \ref{fig:beta_iga} we do not consider the values of discrete inf-sup that are greater than $10^2$. Indeed these values are not a physical phenomenon, but simply the result of an unsuitable discretization of the problem, as observed in Section \ref{sec infsup fem}. 

\begin{figure}[h!]	
	\centering
	\includegraphics[scale=0.3]{Figures/hbound_confronto_constimaempirica}
	\caption{A $\log$-$\log$ plot of errors committed by quadratic IGA, with maximal regularity, in $|\cdot|_{H^1(0,T)}$ seminorm and in $\|\cdot\|_{L^2(0,T)}$ norm, with respect to a uniform mesh-size $h$. Also, the best approximation error in $|\cdot|_{H^1(0,T)}$ seminorm and bounds \eqref{h bound}, \eqref{h bound gard} (with the optimal choice \eqref{b}), \eqref{emp constraint} are represented. The square of the wave number is $\mu=1000$ and the final time is $T=10$.}
	\label{fig:err iga con stima empirica}
\end{figure}
 
In Figure \ref{fig:err iga con stima empirica} one can note the sharpness of the constraint \eqref{emp constraint} with respect to the error committed by the quadratic IGA with maximal regularity. This result is what we actually expect, as a consequence of the inf-sup stability for $h < \sqrt{\frac{9}{\mu}}$.\\ \bigskip

Another interesting numerical test displays, in the stability region given by \eqref{emp constraint}, the numerically estimated isogeometric inf-sup with respect to the mesh-size $h$ at a fixed $\mu$, see Figure \ref{fig: sez infsup depend on h}. This is useful to further see that the discrete inf-sup is actually uniformly limited in the region given by the constraint \eqref{emp constraint}.

\begin{figure}[h!]	
	\centering
	\includegraphics[scale=0.4]{Figures/sez_betanumerica_iga_mu100}
	\caption{A $\log$-$\log$ plot of the isogeometric inf-sup value with respect to a uniform mesh-size $h$, for a fixed $\mu = 100$.}
	\label{fig: sez infsup depend on h}
\end{figure}

Comparing the theoretical inf-sup estimates \eqref{infsup zank}, \eqref{stab iga gard} of Section \ref{sec iga} with the numerical inf-sup could be useful in order to understand how accurate the theoretical bounds are. Therefore, in Figure \ref{fig: sez infsup depend on mu} we visualize the three stability bounds \eqref{emp constraint}, \eqref{h bound}, \eqref{h bound gard} and the relative inf-sup estimates at fixed $h$. It is sufficient to fix $h$ and compare them as functions of the single variable $\mu$, since in the stability region their mesh-size dependence disappears.
In order to represent these three constraints on the mesh-size with respect to $\mu$, we approximate \eqref{h bound}, \eqref{h bound gard} as in Remark \ref{oss asympt}, i.e., 
\begin{gather*}
h < \mu^{-3/2},\\
h < \mu^{-7/2},
\end{gather*}
respectively, which result in
\begin{gather*}
\mu < h^{-2/3},\\
\mu < h^{-2/7}.
\end{gather*}

Since the approximation of \eqref{h bound gard} with $h < \mu^{-7/2}$ corresponds to the slightly suboptimal value $b = \mu T^2$, plotting the inf-sup estimate of Theorem \ref{theo gard iga} with this choice is natural. %However, we could have plotted the inf-sup of Theorem \eqref{h bound gard} also for the optimal value \eqref{b}, since its related stability is guaranteed for every $h$ satisfying the constraint \eqref{h bound gard} with the choice \eqref{b}, and thus for every $h$ satisfying the more stringent constraint. 


As one can note in Figure \ref{fig: sez infsup depend on mu}, the inf-sup estimate of Theorem \ref{theo gard iga} with the choice $b = \mu T^2$ decreases more slowly for $\mu \rightarrow \infty$ than the inf-sup \eqref{infsup zank}, as we expect from Remark \ref{oss asympt}. Moreover, the former is more accurate than the latter, being larger at the same $h$ and $\mu$ for which they are comparable.


\begin{figure}[h!]	
	\centering
	\includegraphics[scale=0.5]{Figures/sez_h_fissato}
	\caption{Three $\log$-$\log$ plots of the isogeometric inf-sup value and of its lower-bounds with respect to $\mu$ for a fixed uniform mesh-sizes $h$.}
	\label{fig: sez infsup depend on mu}
\end{figure}

\section{Unconditional stability}
\subsection{Stabilization of piecewise continuous linear finite element method}\label{stab fem}
In (\cite{Steinbach2019, Zank2020}) the authors introduce a stabilized piecewise continuous linear finite element discretization of \eqref{eq ode} in order to overcome the mesh constraints \eqref{zank più sharp}, \eqref{stab Linfty}. In this Section we briefly recall their main results about this stabilization and we show some numerical experiments that we make to test their theoretical considerations. 

O. Steinbach and M. Zank define a new discrete bilinear form \\ $a_h: S^1_{h;0,*} \times S^1_{h;*,0} \rightarrow \mathbb{R}$ such that
\begin{equation}\label{ah fem}
a_h(u_h,v_h):=-\langle \partial_t u_h, \partial_t v_h \rangle _{L^2(0,T)} + \mu \langle u_h, Q_h^0 v_h \rangle_{L^2(0,T)},
\end{equation}
for all $u_h\in S^1_{h;0,*}$, $v_h \in S^1_{h;*,0}$, where $Q^0_h: L^2(0,T) \rightarrow S^0_h(0, T)$ denotes the $L^2$ orthogonal projection on the piecewise constant finite element space $S^0_h(0, T )$. So basically, they do an under-integration in the $L^2(0,T)$ norm. That is, instead of integrating exactly the $L^2(0,T)$ scalar product $\langle u_h, v_h \rangle_{L^2(0,T)}$, they approximate it by projecting $v_h$ on the piecewise constant finite element space $S^0_h(0, T )$.

As a consequence of the properties of the projection operator $Q^0_h$ and of the piecewise linear nodal interpolation operator $I_h: C[0,T] \rightarrow S^1_h(0,T)$ (Lemma $17.1$, Lemma $17.2$, Lemma $17.3$, Lemma $17.4$, Lemma $17.5$ of \cite{Steinbach2019}), the new bilinear form \eqref{ah fem} satisfies the uniform (w.r.t. $h$) inf-sup condition
\begin{equation}\label{infsup perturbata}
\frac{1}{1+\sqrt{2}\mu T^2} |u_h|_{H^1(0,T)} \leq \sup_{0 \neq v_h \in S^1_{h;*,0}} \frac{a_h(u_h, v_h)}{|v_h|_{H^1(0,T)}},
\end{equation}
see Lemma $17.6$ of (\cite{Steinbach2019}). 

Therefore, O. Steinbach and M. Zank consider the following perturbed problem
\begin{equation}\label{perturbed problem}
\begin{cases}
\text{Find} \ u_h \in S^1_{h;0,*} \quad \text{such that}\\
a_h(u_h,v_h)=\langle f,v_h \rangle_{(0,T)} \quad  \forall v_h \in  S^1_{h;*,0},
\end{cases}
\end{equation}
where $T>0$ and $f \in [H^1_{*,0}(0,T)]' $ are given, and the notation $\langle \cdot,\cdot \rangle_{(0,T)}$ denotes the duality pairing as extension of the inner product in $L^2(0,T)$. Thanks to \eqref{infsup perturbata} estimate, problem \eqref{perturbed problem} is well-posed and the following stability condition holds
\begin{equation*}
|\tilde{u}_h|_{H^1(0,T)} \leq \big(1+\sqrt{2}\mu T^2\big) \|f\|_{[H^1_{*,0}(0,T)]'},
\end{equation*}
 where $\tilde{u}_h \in S^1_{h;0,*} $ is the unique solution related to the fixed right-hand side $f \in [H^1_{*,0}(0,T)]'$.

Using an alternative representation (Corollary 17.1 of \cite{Steinbach2019}) of the bilinear form $a(\cdot,\cdot)$ defined in \eqref{bil ode} and some standard techniques, such as Galerkin orthogonality and interpolation error estimates, also linear convergence in $|\cdot|_{H^1(0,T)}$ of the discrete solution $\tilde{u}_h$ of \eqref{perturbed problem} to the solution $u \in H^1_{0,*}(0,T)$ of \eqref{var ode} are proven, if $u \in H^2(0,T)$; see Theorem 17.1 of (\cite{Steinbach2019}) for more details. Furthermore, quadratic convergence in $\|\cdot\|_{L^2(0,T)}$ is shown, if the exact solution satisfies $u \in H^2(0,T)$; see Theorem 4.2.21 of (\cite{Zank2020}). \\ \bigskip

We fix the final time $T=10$, and we test their stabilization by numerically estimating the discrete inf-sup of the bilinear form \eqref{infsup perturbata} by means of Proposition \ref{estimate infsup} with uniform mesh.

\begin{figure}[h!]
	\centering
	\includegraphics[scale=0.45]{Figures/image_stab_FEM_20x20_confronto}
	\caption{On the left a p-color plot of $\log(\beta)$ with respect to $(\log(\mu),log(h))$, where $\beta$ is the discrete inf-sup value of the bilinear form \eqref{bil ode}. On the right a p-color plot of $\log(\beta)$ with respect to $(\log(\mu),log(h))$, where $\beta$ is the inf-sup value of the bilinear form \eqref{ah fem}. In both Figures the red line is the natural logarithm of the upper bound \eqref{stab Linfty}.}
	\label{fig: beta stab fem}
\end{figure}

\begin{figure}[h!]
	\centering
	\includegraphics[scale=0.45]{Figures/test_inf_sup_FEM}
	\caption{On the left, two three-dimensional plots of $\log(\beta)$ with respect to $(\log(\mu),\log(h))$. On the right, the contour lines of $\log(\beta)$ with respect to $(\log(\mu),\log(h))$. The top two figures correspond to the bilinear form defined in \eqref{bil ode} restricted to the FEM discrete spaces \eqref{fem trial}, \eqref{fem test}, and the bottom two to the stabilised bilinear form \eqref{ah fem}.}
	\label{fig: 3D plot beta stab fem}
\end{figure}

As one can see from Figures \ref{fig: beta stab fem}, \ref{fig: 3D plot beta stab fem}, the behaviour of the discrete inf-sup related to \eqref{ah fem} is uniformly (w.r.t. $h$) bounded. In particular, the red line of Figure \ref{fig: beta stab fem} corresponds to the upper-bound \eqref{stab Linfty} and now, in the stabilized case, it no longer separates two different discrete inf-sup regimes.

As we expect from Remark 4.2.20 of (\cite{Zank2020}), in Figure \ref{fig: beta stab fem} one can note that the optimal discrete inf-sup constant shows a dependency of order $\mu^{-\frac{1}{2}}$, like the discrete inf-sup constant of \eqref{bil ode} in the stability region given by \eqref{stab Linfty}, i.e., defined by
\begin{equation*}
h < \sqrt{\frac{12}{\mu}}.
\end{equation*}

As before, in Figure \ref{fig: beta stab fem} we do not consider the values of the discrete inf-sup that are greater than $10^2$. Indeed these values are not a physical phenomenon, but simply the result of an unsuitable discretization of the problem, as observed in Section \ref{sec infsup fem}. \\ \bigskip

We test O. Steinbach and M. Zank's stabilization also by studying the error committed in the approximation of an exact solution of \eqref{var ode}. As in previous Sections and in (\cite{Steinbach2019, Zank2020}), as a numerical example for the perturbed Galerkin piecewise continuous linear FEM \eqref{perturbed problem}, we consider a uniform discretization of the time interval $(0,T)$ with $T = 10$ and a mesh-size $h = T/N$. For $\mu = 1000$ we consider the strong solution $u(t) = \sin^2\Big(\frac{5}{4}\pi t\Big)$
and we compute the integrals appearing at the right-hand side using high-order integration rules.  

The minimum number of elements chosen is $N = 4$, the maximum number is $N = 32768$, as in Section \ref{sec errors fem}.

\begin{figure}[h!]	
	\centering
	\includegraphics[scale=0.5]{Figures/err_ode_fem_stab}
	\caption{A $\log$-$\log$ plot of errors committed by perturbed piecewise continuous linear FEM \eqref{perturbed problem} in $|\cdot|_{H^1(0,T)}$ seminorm and in $\|\cdot\|_{L^2(0,T)}$ norm, with respect to a uniform mesh-size $h$. Also, the best approximation error in $|\cdot|_{H^1(0,T)}$ seminorm is represented. The square of the wave number is $\mu=1000$.}
	\label{fig: err fem stab}
\end{figure}

\begin{figure}[h!]	
	\centering
	\includegraphics[scale=0.5]{Figures/rel_err_ode_fem_stab}
	\caption{A $\log$-$\log$ plot of relative errors committed by perturbed piecewise continuous linear FEM \eqref{perturbed problem} in $|\cdot|_{H^1(0,T)}$ seminorm and in $\|\cdot\|_{L^2(0,T)}$ norm, with respect to a uniform mesh-size $h$. Also, the best approximation error in $|\cdot|_{H^1(0,T)}$ seminorm is represented. The square of the wave number is $\mu=1000$.}
	\label{fig: rel err fem stab}
\end{figure}

As one can see in Figure \ref{fig: err fem stab}, linear and quadratic convergence, respectively, in $|\cdot|_{H^1(0,T)}$ seminorm and in $\|\cdot\|_{L^2(0,T)}$ norm are confirmed.

\begin{figure}
	\hspace{-1cm}
	\begin{minipage}[h!]{8.5cm}
		\centering
		\includegraphics[width=6.5cm]{Figures/uex_udis_ode_femstab_N64}
		\caption{Exact and discrete solutions of the perturbed problem \eqref{perturbed problem} for $\mu = 1000$ and $N = 64$ elements (i.e., $h = 0.1563$).}
	\end{minipage} 
	\hspace{-1cm}
	\begin{minipage}[h!]{9cm}
		\centering
		\includegraphics[width=6.5cm]{Figures/uex_udis_ode_fem_N64}
		\caption{Exact and discrete solutions by unperturbed piecewise linear FEM  for $\mu = 1000$ and $N = 64$ elements (i.e., $h = 0.1563$).}
	\end{minipage}
\end{figure}

\subsubsection{Empirical reason for the unconditional stability of the perturbed FEM discretization}
The following representation holds (Lemma 17.2 of \cite{Steinbach2019})
\begin{equation}\label{rappre ah}
a_h(u_h,v_h)=-\langle \partial_t u_h, \partial_t v_h \rangle _{L^2(0,T)} - \frac{\mu}{12}\sum_{l=1}^{N} h_l^2 \langle \partial_t u_h, \partial_t v_h\rangle_{L^2(\tau_l)} + \mu \langle u_h, v_h \rangle_{L^2(0,T)},
\end{equation}
for all $u_h\in S^1_{h;0,*}$, $v_h \in S^1_{h;*,0}$, where $\tau_l$, for $l=1, \ldots,N$, are the subintervals of $(0,T)$ given by the Galerkin discretization. Note that \eqref{rappre ah} results in
\begin{equation}\label{rappre ah mesh unif}
a_h(u_h,v_h)=-\Bigg(1+\frac{\mu h^2}{12}\Bigg)\langle \partial_t u_h, \partial_t v_h \rangle _{L^2(0,T)} + \mu \langle u_h, v_h \rangle_{L^2(0,T)},
\end{equation}
for all $u_h\in S^1_{h;0,*}$, $v_h \in S^1_{h;*,0}$, in the case of a uniform mesh-size. 

The representation \eqref{rappre ah mesh unif} suggests another justification to the inf-sup stability of the perturbed bilinear form $a_h(\cdot,\cdot)$ in the case of a uniform mesh-size. Indeed, the bilinear form \eqref{rappre ah mesh unif} corresponds to the perturbed problem
\begin{equation*}
\partial_{tt}u(t)+ \mu_h u(t)=f(t), \quad \text{for} \ t \in (0,T), \quad u(0)=\partial_{t}u(t)_{|t=0}=0,
\end{equation*}
where 
\begin{equation*}
\mu_h := \frac{\mu}{1+\frac{\mu h^2}{12}}.
\end{equation*}
Since the mesh-size $h$ always satisfies
\begin{equation*}
h < \sqrt{\frac{12}{\mu_h}} < \sqrt{\frac{12}{\mu}+h^2},
\end{equation*}
the perturbed bilinear form \eqref{ah fem} with a uniform mesh-size is inf-sup stable with an inf-sup value that has a dependency on $\mu_h$ of order $\mu_h^{-\frac{1}{2}}$, as a consequence of the numerical results of Section \ref{sec infsup fem}. In particular, 
\begin{equation*}
 {\mu}^{-\frac{1}{2}} \leq {\mu_h}^{-\frac{1}{2}} \simeq \inf_{0 \neq u_h \in S^1_{h;0,*}} \sup_{0 \neq v_h \in S^1_{h;*,0}} \frac{a_h(u_h, v_h)}{|v_h|_{H^1(0,T)}},
\end{equation*}
for all $u_h\in S^1_{h;0,*}$, $v_h \in S^1_{h;*,0}$. Note that these arguments are not intended to replace the analysis of (\cite{Steinbach2019, Zank2020}), which is more complete. They are meant to give an empirical motivation that explains why the stabilization proposed by O. Steinbach and M. Zank actually works.

\subsection{Stabilization for quadratic isogeometric discretization with maximal regularity}
In this Section we consider three different stabilization techniques. The first two give poor results, whereas the performances of the last one are satisfactory.

Inspired by \eqref{ah fem}, we firstly try to perturb the quadratic isogeometric discretization with maximal regularity in the same way, i.e., defining
\begin{equation}\label{tent 1 iga}
a_h(u_h,v_h):=-\langle \partial_t u_h, \partial_t v_h \rangle _{L^2(0,T)} + \mu \langle u_h, Q_h^0 v_h \rangle_{L^2(0,T)},
\end{equation}
for all $u_h \in V^h_{0,*}$ of \eqref{iga space} and $v_h \in V^h_{*,0}$ of \eqref{iga test space}, and considering the modified discrete problem
\begin{equation}\label{perturbed iga}
\begin{cases}
\text{Find} \ u_h \in V^h_{0,*} \quad \text{such that}\\
a_h(u_h,v_h)=\langle f,v_h \rangle_{(0,T)} \quad  \forall v_h \in  V^h_{*,0}.
\end{cases}
\end{equation}
\begin{oss}
	Another possible perturbation consists of defining
	\begin{equation*}
	a_h(u_h,v_h):=-\langle \partial_t u_h, \partial_t v_h \rangle _{L^2(0,T)} + \mu \langle u_h, Q_h^1 v_h \rangle_{L^2(0,T)},
	\end{equation*}
	for all $u_h \in V^h_{0,*}$ and $v_h \in V^h_{*,0}$, where $Q^1_h: L^2(0,T) \rightarrow S^1_h(0,T)$ denotes the $L^2(0,T)$ orthogonal projection on the piecewise continuous linear finite element space $S^1_h(0,T)$.
\end{oss}

As in previous Sections, we test perturbation \eqref{tent 1 iga} by studying the error committed in the approximation of an exact solution of \eqref{var ode}. In particular, as before, as a numerical example we consider a uniform discretization of the time interval $(0,T)$ with $T = 10$ and a mesh-size $h = T/N$. For $\mu = 1000$ we choose the strong solution $u(t) = \sin^2\Big(\frac{5}{4}\pi t\Big)$
and we compute the integrals appearing at the right-hand side using high-order integration rules. The minimum number of elements chosen is $N = 4$, the maximum number is $N=4096$, as in Section \ref{sec errs iga}.

The results that we obtain are in Figure \ref{fig: err iga pert Qh0}, which are far from promising, since the errors are even larger than those ones made by the conditionally stable method \eqref{iga ode}. A possible explanation is that we are exceeding in the under-integration if compared to the regularity of the test and trial functions we consider.

\begin{figure}[h!]	
	\centering
	\includegraphics[scale=0.5]{Figures/err_ode_IGA_pert2}
	\caption{A $\log$-$\log$ plot of errors committed by perturbed quadratic IGA with maximal regularity \eqref{tent 1 iga} in $|\cdot|_{H^1(0,T)}$ seminorm and in $\|\cdot\|_{L^2(0,T)}$ norm, with respect to a uniform mesh-size $h$. The square of the wave number is $\mu=1000$.}
	\label{fig: err iga pert Qh0}
\end{figure}

We then decide to change perspective by considering the perturbation \eqref{rappre ah mesh unif}, inspired by the observations at the end of Section \ref{stab fem}. Since we numerically obtain the stability constraint \eqref{emp constraint}, we define the following perturbation
\begin{equation}\label{tent 2 iga}
a_h(u_h,v_h)=-\Bigg(1+\frac{\mu h^2}{9}\Bigg)\langle \partial_t u_h, \partial_t v_h \rangle _{L^2(0,T)} + \mu \langle u_h, v_h \rangle_{L^2(0,T)},
\end{equation}
for all $u_h \in V^h_{0,*}$ and $v_h \in V^h_{*,0}$, if the mesh-size is uniform. We then test this perturbation by studying the error committed by the discrete solution of \eqref{perturbed iga} considering the perturbed bilinear form \eqref{tent 2 iga}. We choose the same exact solution and assumptions of Figure \ref{fig: err iga pert Qh0}. The results that we achieve are in Figure \ref{fig: err iga pert tent 2} and are quite promising, as we could expect from the empirical analysis at the end of Section \ref{stab fem}. Indeed, bounded errors are a consequence of stability. However, we do not obtain the orders of convergence that we expect for quadratic IGA of maximal regularity, since we only get quadratic convergence in $\|\cdot\|_{L^2(0,T)}$ norm.

\begin{figure}[h!]	
	\centering
	\includegraphics[scale=0.5]{Figures/err_ode_IGA_pert1}
	\caption{A $\log$-$\log$ plot of errors committed by perturbed quadratic IGA with maximal regularity \eqref{tent 2 iga} in $|\cdot|_{H^1(0,T)}$ seminorm and in $\|\cdot\|_{L^2(0,T)}$ norm, with respect to a uniform mesh-size $h$. The square of the wave number is $\mu=1000$.}
	\label{fig: err iga pert tent 2}
\end{figure}

\bigskip
In order to gain a method that is unconditionally stable whose orders of convergence in $|\cdot|_{H^1(0,T)}$ and $\|\cdot\|_{L^2(0,T)}$ are what we expect for quadratic splines with $C^1(0,T)$ global regularity, we decide to define the following perturbed bilinear form in the case of a uniform mesh-size 
\begin{equation}\label{tent 3 iga}
\begin{split}
a_h(u_h,v_h)=-\langle \partial_t u_h, \partial_t v_h \rangle _{L^2(0,T)} + \mu \langle u_h, &v_h \rangle_{L^2(0,T)} \\
&- \delta \mu h^4\langle \partial_t^2 u_h, \partial_t^2 v_h \rangle _{L^2(0,T)},
\end{split}
\end{equation}
for all $u_h \in V^h_{0,*}$ and $v_h \in V^h_{*,0}$, where $\delta >0$ is a fixed real value. 

Let us note that both the bilinear forms \eqref{tent 2 iga}, \eqref{tent 3 iga} are defined by a non consistent perturbation, since, in general, if $\tilde{u}_h \in V^h_{0,*}$ is a solution of their related perturbed problem \eqref{perturbed iga}, $\tilde{u}_h $ is not a solution of the non perturbed problem \eqref{iga ode}.

The behaviour of \eqref{tent 3 iga} depends on the choice of the coefficient $\delta$. Therefore, we test this perturbation by studying the error committed by the discrete solution of \eqref{perturbed iga} considering the perturbed bilinear form \eqref{tent 3 iga} with different choice of $\delta$. In particular, we consider $\delta \in \{\frac{1}{100},\frac{1}{10},1,10,100\}$ in order to see how the errors behave for different orders of magnitude of the perturbation coefficient. We choose the same exact solution and assumptions of Figures \ref{fig: err iga pert Qh0}, \ref{fig: err iga pert tent 2}. 

Figure \ref{fig: err and rel err iga delta} represents $\log$-$\log$ plots of relative errors committed by perturbed quadratic IGA with maximal regularity \eqref{tent 3 iga} in $|\cdot|_{H^1(0,T)}$ seminorm and in $\|\cdot\|_{L^2(0,T)}$ norm, with respect to a uniform mesh-size $h$. The smallest global errors are obtained for $\delta = \frac{1}{100}$. In particular, they are satisfactory since the errors in $|\cdot|_{H^1(0,T)}$ and $\|\cdot\|_{L^2(0,T)}$ have a maximum that is smaller than $10^2$, but also because we achieve quadratic convergence in $|\cdot|_{H^1(0,T)}$ and cubic convergence in $\|\cdot\|_{L^2(0,T)}$, as we expect from quadratic IGA discretization.

\begin{figure}[h!]	
	\centering
	\includegraphics[scale=0.4]{Figures/New_Project_2}
	\caption{$\log$-$\log$ plots of relative errors committed by perturbed quadratic IGA with maximal regularity \eqref{tent 3 iga} in $|\cdot|_{H^1(0,T)}$ seminorm and in $\|\cdot\|_{L^2(0,T)}$ norm, with respect to a uniform mesh-size $h$. Also, the best approximation error in $|\cdot|_{H^1(0,T)}$ seminorm is represented. The square of the wave number is $\mu=1000$.}
	\label{fig: err and rel err iga delta}
\end{figure}

\begin{figure}
	\hspace{-1.5cm}
	\begin{minipage}[h!]{9cm}
		\centering
		\includegraphics[width=6.5cm]{Figures/uex_udis_ode_IGA_pert4_N64}
		\caption{Exact and discrete solutions of the perturbed problem \eqref{perturbed iga} with \eqref{tent 3 iga} and $\delta=\frac{1}{100}$, for $\mu = 1000$ and $N = 64$ elements (i.e., h = 0.1563).}
	\end{minipage} 
	\hspace{-1.5cm}
	\begin{minipage}[h!]{8.5cm}
		\centering
		\includegraphics[width=6.5cm]{Figures/uex_udis_ode_iga_N64}
		\caption{Exact and discrete solutions by unperturbed quadratic IGA for $\mu = 1000$ and $N = 64$ elements (i.e., h = 0.1563).}
	\end{minipage}
\end{figure}

Since the new bilinear form \eqref{tent 3 iga} with $\delta = \frac{1}{100}$ returns significantly small errors that converge with ``the right orders'' of convergence, we are now interested in numerically studying how its inf-sup value behaves with respect to $(\mu,h)$. As before, we fix the final time T=10 and a uniform mesh and we numerically estimate the discrete inf-sup of the bilinear form \eqref{tent 3 iga} by means of Proposition \ref{estimate infsup}.

\begin{figure}[h!]
	\centering
	\includegraphics[scale=0.45]{Figures/image_delta1cent_IGA_20x20_confronto}
	\caption{On the left a p-color plot of $\log(\beta)$ with respect to $(\log(\mu),log(h))$, where $\beta$ is the discrete inf-sup value of the bilinear form \eqref{bil ode}. On the right a p-color plot of $\log(\beta)$ with respect to $(\log(\mu),log(h))$, where $\beta$ is the inf-sup value of the bilinear form \eqref{tent 3 iga} with $\delta=\frac{1}{100}$. In both Figures the red line is the natural logarithm of the upper bound \eqref{emp constraint} and the square of the wave number is $\mu=1000$ .}
	\label{fig: beta stab iga}
\end{figure}

\begin{figure}[h!]
	\centering
	\includegraphics[scale=0.45]{Figures/test_inf_sup_IGA_delta1cent}
	\caption{On the left, two three-dimensional plots of $\log(\beta)$ with respect to $(\log(\mu),\log(h))$. On the right, the contour lines of $\log(\beta)$ with respect to $(\log(\mu),\log(h))$. The top two figures correspond to the bilinear form defined in \eqref{bil ode} restricted to the discrete IGA spaces \eqref{iga space}, \eqref{iga test space}, and the bottom two to the stabilised bilinear form \eqref{tent 3 iga} with $\delta=\frac{1}{100}$. The square of the wave number is $\mu=1000$.}
	\label{fig: 3D plot beta stab iga}
\end{figure}

As one can see from Figures \ref{fig: beta stab iga}, \ref{fig: 3D plot beta stab iga}, the behaviour of the discrete inf-sup of \eqref{tent 3 iga} is uniformly (w.r.t. $h$) bounded. In particular, the red line of Figure \ref{fig: beta stab iga} corresponds to the upper-bound \eqref{emp constraint} and now, in the stabilized case, it no longer separates two different discrete inf-sup regimes. 

\begin{figure}[h!]	
	\centering
	\includegraphics[scale=0.4]{Figures/sez_betanumerica_iga_mu100_delta1cent}
	\caption{A $\log$-$\log$ plot of the isogeometric inf-sup values of \eqref{bil ode}, \eqref{tent 3 iga} with respect to a uniform mesh-size $h$, for $\delta = \frac{1}{100}$ and a fixed $\mu = 100$.}
	\label{fig: sez infsup iga depend on h}
\end{figure}

Figure \ref{fig: sez infsup iga depend on h} clearly shows the uniformly bounded behaviour (w.r.t. $h$) of the numerically estimated inf-sup of the perturbed bilinear form \eqref{tent 3 iga}.

In Figure \ref{fig: beta stab iga} one can note that the optimal discrete inf-sup constant shows a dependency on $\mu$ of order $\mu^{-\frac{1}{2}}$, as the discrete inf-sup constant of \eqref{bil ode} in the stability region given by \eqref{emp constraint}, i.e., defined by
\begin{equation*}
h < \sqrt{\frac{9}{\mu}}.
\end{equation*}

As before, in Figure \ref{fig: beta stab iga} we do not consider the values of discrete inf-sup that are greater than $10^2$, since they are not a physical phenomenon, but simply the result of an unsuitable discretization of the problem, as observed in Section \ref{sec infsup fem}. 

It might be interesting to further improve the stabilization by choosing the optimal $\delta$ in relation to the empirical stability constraint \eqref{emp constraint}, but the perturbation of order four that we are operating does not allow us to apply the reasoning explained at the end of Section \ref{stab fem}. However, an improvement of $\delta$ could be obtained by writing it as an appropriate function of $\mu$.
