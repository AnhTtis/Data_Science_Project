In this section we summarize the main concepts underlying the notion
of Nominal Sets; for a more complete treatment we refer the reader to
\cite{Pitts2013-book}. We repeat the basic definitions of group and
group action. A \emph{group} is a set $G$ with a distinguished element
($\epsilon \in G$, the \emph{unit}), a binary operation
($\_\cdot\_\colon G\times G \to G$, the \emph{multiplication}), and a
unary operation ($\_^{-1} \colon G\to G$, the \emph{inverse}),
satisfying the following axioms:
\begin{align*} 
  \text{Associativity:} && g_1 \cdot (g_2 \cdot g_3) \ &= \ (g_1 \cdot g_2) \cdot g_3 && ,\forall g_1,g_2,g_3 \in G\\ 
  \text{Inverse element:} && g\cdot (g^{-1}) \ &= \ \epsilon \ = \ g^{-1} \cdot g  && ,\forall g \in G\\ 
  \text{Identity element:} && \epsilon\cdot g \ &= \ g \ = \ g\cdot \epsilon && ,\forall g \in G
\end{align*}

\noindent Although a group is given by the tuple
$(G,\epsilon,\_\cdot\_,\_^{-1})$ (and the proofs that these operations
satisfy the axioms) we will refer to the group simply by $G$. A
sub-group of $G$ is a subset $H\subseteq G$ such that $\epsilon\in H$
and $H$ is closed under the inverse and multiplication.

Let $G$ be a group. A \emph{$G$-set} is a set $X$ with an operation
$\gactUn \colon G\times X \to X$ (called the \emph{action})
satisfying:
\begin{align*}
\text{Identity:} && \gact{\epsilon}{x} & \ = \ x                            && ,\forall x \in X\\ 
\text{Compatibility:} && \gact{g_1}{(\gact{g_2}{x})} & \ = \ \gact{(g_1\cdot g_2)}{x}  && ,\forall g_1,g_2 \in G, \forall x \in X
\end{align*}

A morphism between $G$-sets $X$ and $Y$ is a function
$F \colon X \to Y$ that commutes with the actions:
\[
  F\,(\gact{g}{x}) \ = \ \gact{g}{F\,x} %
    \quad\qquad ,\forall g \in G, \forall x \in X
\]
These are called \emph{equivariant} functions. Since $id_{X}$ is
equivariant and the composition of equivariant functions yields an
equivariant function we can talk of the category of $G$-Sets.

Any set $X$ can be seen as a $G$-set by letting $\gact{g}{x} = x$;
such a $G$-set is called the \emph{discrete} $G$-set. Moreover any
group acts on itself by the multiplication.

One can form the (in)finitary product of $G$-sets by defining the
action of $G$ on a tuple in a pointwise manner:
\[
  \gact{g}{\tuple{x_1}{\,x_2}} \ = \
  \tuple{\gact{g}{x_1}}{\,\gact{g}{x_2}}
    \quad\qquad ,\forall g \in G, \forall x_1 \in X_1, \forall x_2 \in X_2
\]
The projections and the product morphism $\tuple{F}{H}$ are
equivariant, assuming that $F$ and $H$ are also
equivariant. $G$-set, as a category, also has co-products.

If $X$ and $Y$ are $G$-sets one can endow the set $Y^X$ of functions
from $X$ to $Y$ with the \emph{conjugate} action:
\[
  (\gact{g}{F})\,x \ = \ \gact{g}{(F\,(\gact{g^{-1}}{x}))}
    \quad\qquad ,\forall g \in G, \forall x \in X \enspace .
\]

\paragraph{$G$-sets over the Permutation Group}
The group of symmetries over a set $X$ consists of $G = \Sym{X}$,
where $\Sym{X}$ is the set of bijections on $X$; the multiplication of
$\Sym{X}$ is composition, the inverse is the inverse bijection, and
the unit is the identity.

Let $\Perm{X}$ be the subset of $\Sym{X}$ of bijections that changes
only finitely many elements; i.e., $f\in\Perm{X}$ if
$\supp{f} = \{x\in X \mid f\,x\not=x \}$ is finite. It is
straightforward to prove that $\Perm{X}$ is a sub-group of
$\Sym{X}$. Of course, if $X$ is finite, then $\Perm{X} = \Sym{X}$.
Notice that $X$ itself is a $\Perm{X}$-set with the action being
function application: $\gact{\pi}{x} \ = \ \mathop{\pi}\,x \enspace$.

In particular, the \emph{transposition} (or \textit{swapping}) of a
pair of elements $x, y \in X$ is the finite permutation
$\swap x y \in \Perm{X}$ given by
\[
\swap x y \, z \ = \ \begin{cases}
y & \text{if } z = x \\
x & \text{if } z = y \\
z & \text{otherwise}
\end{cases} \]

A basic result (in \cite{Hungerford} is proved as Theorem 6.3 and
Corollary 6.5) is that every $\pi \in \Perm{X}$ can be expressed as a
composition of \emph{disjoint cycles}
\[
\pi \ = \ (x_1\ x_2\ \ldots\  x_n) \circ \cdots \circ 
(z_1\ z_2\ \ldots\ z_k)
\]
and every cycle can be expressed as a composition of transpositions
\[
(x_1\ x_2\ \ldots\ x_n)\ = \ \swap{x_1}{x_2} \circ \swap{x_2}{x_3} \circ \cdots \circ \swap{x_{n-1}}{x_n}
\]
Therefore every $\pi \in \Perm{X}$ can be expressed as a composition
of transpositions.  We elaborate on the equivalence of the
representations in Sect.~\ref{sec:perm}.  Let us exhibit this with a
concrete example.

\begin{example}
\label{ex:perm}
Let $f\colon \mathbb{N}\to \mathbb{N}$ be defined as
\begin{align*}
    f\,x \ = \ & \begin{cases}
    (x+2) \,\mathop{mod}\, 6 & \text { if } x \leq 5\\
    x & \text{ else }
    \end{cases}
\end{align*}
Function $f$ is a finite permutation, because it has finite support:
$\{ x \in \mathbb{N} \mid 0 \leq x \leq 5 \}$. Therefore it can be
expressed as the composition of two cycles:
$(1\ 3\ 5)\circ (0\ 2\ 4)$, or alternatively, it can also be expressed
as a composition of four transpositions:
$\swap{1}{3}\circ\swap{3}{5}\circ\swap{0}{2}\circ\swap{2}{4}$.
\end{example}


\paragraph{Nominal Sets}

If we let $X$ be the set of variables for the lambda calculus, then a
permutation on $X$ is a renaming; such a permutation can be lifted to
an action over the set of lambda terms (taking care of the bound
variables). In the nominal parlance one says that $X$ is the set of
\emph{atoms} or that variables are atomic names: an atomic name has no
structure in itself. We only assume that a set of atoms is a countable
infinite set with decidable equality; from now on we will use
$\mathbb{A}$ to refer to a set of atoms.

Let $X$ be a $\PermA$-set. We say that $x \in X$ \emph{is supported by} $A\subseteq\mathbb{A}$
if \[
\forall\, \pi.\ (\forall\, a\in A.\ \pi\,a\,=\,a) \ \implies \ \gact{\pi}{x} \,=\, x \enspace .
\]

\noindent We say that $X$ is a \emph{nominal set} if each element of
$X$ is supported by some finite subset of $\mathbb{A}$. Since each
finite permutation can be decomposed as a composition of
transpositions, then one can prove that the above definition is
equivalent to
\[ \forall\, a,a'\in\mathbb{A}\setminus A.\ \gact{\swap{a}{a'}}{x}
  \,=\, x \enspace .
\]

\noindent The following are some examples of nominal sets:
\begin{itemize}
    \item The discrete $Perm(\mathbb{A})$-set $X$ is nominal, because any $x \in X$ is supported by $\emptyset$. 
    \item $\mathbb{A}$ itself is nominal once equipped with the action
      $\gact{\pi}{a} = \pi\,a$, because any $a\in \mathbb{A}$ is
      supported by $\{a\}$. More in general, any
      $S \subseteq \mathbb{A}$ containing name $a$ is a support for
      $a$.
    \item The set $\lambda$\textit{Term} of $\lambda$-calculus terms, inductively defined by $\ t \ ::= \ V(a) \ \mid \ A(t,t) \ \mid \ L(a,t)\ $ where $a \in \mathbb{A}$, equipped with the action $\gactUn \ \colon Perm(\mathbb{A})\times\lambda\textit{Term} \to \lambda\textit{Term}$ such that
      \begin{align*}
        & \gact{\pi}{V(a)}        \ = \ V(\pi\, a)                    \\ 
        & \gact{\pi}{A(t_1,t_2)}  \ = \ A(\gact{\pi}{t_1},\,\gact{\pi}{t_2})     \\ 
        & \gact{\pi}{L(a,t)}      \ = \ L(\pi\, a,\,\gact{\pi}{t})    
      \end{align*}
      is nominal because any $t \in \lambda$\textit{Term} is supported by $supp(t)=FreeVars(t)$.
\end{itemize}


In his book \cite{Pitts2013-book} Pitts uses classical logic to prove
that if $x$ is supported by some finite set $A$, then there exists a
least supporting set, called \emph{the} support of $x$. As shown by
Swan \cite{Swan2017} one cannot define the least support in a
constructive setting; therefore a formalization in a constructive type
theory should ask for ``some'' finite support. This affects the notion
of freshness: in classical logic we have
\[
x \,\textit{ is fresh for }\, y \ \ \Leftrightarrow \ \  \supp{x} \cap \supp{y} = \emptyset, 
\]
with $x\in X$ and $y\in Y$ being elements of different nominal sets;
but in a constructive setting one has to limit this relation to atoms,
that is
\[
a\in\mathbb{A} \,\textit{ is fresh for }\, x \in X \ \ \Leftrightarrow \ \ a\not\in\supp{x}, 
\]
where $\supp{x}$ is the set supporting $x$, not necessarily the least
one. Notice that the definition is the same (``there exists some
finite support for each element''), but in classical logic that is
sufficient to obtain the least support.