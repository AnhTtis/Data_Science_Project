Nominal techniques have been adopted in various developments. We
distinguish developments borrowing some concepts from nominal
techniques to be applied in specific use cases (e.g. formalization of
languages with binders like the $\lambda$ or $\pi$ calculus with their
associated meta-theory) \cite{Bengtson2007, Copello2016, Copello2018,
Copello2018-2} from more general developments aiming to formalize at
least the core aspects of the theory of nominal sets. We are more
concerned with the later type.

The nominal datatype package for Isabelle/HOL \cite{Urban2006}
developed by Urban and Berghofer implements an infrastructure for
defining languages involving binders and for reasoning conveniently
about alpha-equivalence classes. This Isabelle/HOL package inspired
Aydemir et al.~\cite{Aydemir2007} to develop a proof of concept for
the Coq proof assistant, however it had no further development.  In
his Master thesis~\cite{Chou2015-mt}, Choudhury notes that none of the
previous developments following the theory of nominal sets were based
on constructive foundations.  He showed that a considerable portion
(most of the first four chapters of Pitts book \cite{Pitts2013-book})
of the theory of nominal sets can also be developed constructively by
giving a formalization in Agda. Pitts original work is based on
classical logic, and depends heavily on the existence of the smallest
finite support for an element of a nominal set. However, Swan
\cite{Swan2017} has shown that in general this existence cannot be
constructively guaranteed, as it would imply the law of the excluded
middle.

Choudhury works with the notion of \emph{some non-unique support}. In
order to formalize the category of Nominal Sets, Choudhury preferred
setoids instead of postulating functional extensionality. As far as we
know, Choudhury is still the most comprehensive mechanization in terms
of instances of constructions having a nominal structure.

Recently Paranhos and Ventura \cite{Paranhos2022} presented a
constructive formalization in Coq of the core notions of nominal sets:
support, freshness and name abstraction. They follow closely
Choudhury’s work in Agda \cite{Chou2015-mt}, acknowledging the
importance of working with setoids. They claim that by using Coq’s
type class and setoid rewriting mechanism, much shorter and simpler
proofs are achieved, circumventing the ``setoid hell'' described by
Choudhury. In his master thesis \cite{paranhos-thesis} Paranhos
further developed the library.

Both of those two formalizations in type theory take a very pragmatic
approach to finite permutations: a finite permutation is a list of
pairs of names. In our approach, we start with the more general
notion of bijective function from which the finite permutations are
obtained as a special case; moreover having different representations
allowed us to state and prove some theorems that cannot even be stated
in the other formalizations. So far, our main contributions are: the
representation of finite permutations and the normalization of
composition of transpositions; the equivalence between two definitions
of the relation ``$A$ supports the element $x$''; and proving that the
extension of every container type can be enriched with a group action
(notice that this cover lists, trees, etc.).

Our next steps are the definition of freshness. We are studying an
alternative notion of support that would admit having a freshness
relation between elements of two nominal sets (in contrast with other
mechanization that only consider ``the atom $a$ is fresh for $x$'')
and name abstraction. In parallel we hope to be able to prove that
extensions of finite containers on nominal sets are also nominal
sets. We also hope to streamline further some rough corners of our
development.