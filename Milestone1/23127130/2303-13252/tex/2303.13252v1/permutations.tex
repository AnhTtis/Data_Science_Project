As we have already said, a finite permutation on a set $A$ can be
explicitly given by:
\begin{enumerate}
    \item a bijection $f : A \to A$ together with its support $\supp{f} \subseteq_{\mathit{fin}} A$; i.e., $a\in \supp{f}$ if and only if $f\,a\not= a$;
    \item a composition of disjoint cycles; concretely, we can think of this as a finite set $R \subseteq_{\mathit{fin}} A^*$ of disjoint cycles, each of them without repeated elements;
    \item a composition of transpositions; that is, a finite sequence of pairs $p : (A\times A)^*$.
\end{enumerate}

We present our proof that these definitions are equivalent. It
basically boils down to define a predicate on sequences of elements in
$A$ not containing repeated elements ensuring that they are cycles for
$f$. We use the usual notation $\swap a b$ to denote the bijection
$\{(a,b),(b,a)\}$.

\begin{definition}[List of transpositions from a cycle] We define $\mathit{toFP}\colon A\times A^* \to (A\times A)^*$.
\[
    \tofp{a}{\rho} = \begin{cases} 
    [] & \text{ if } \rho=[]\\
    (a,b):\tofp{b}{\rho'} & \text{ if } \rho=b:\rho'\\
    \end{cases}
\]
If we know that $\rho=a:\rho'$, then we also write $\tofpUn{\rho}$
to mean $\tofp{a}{\rho'}$.
\end{definition}

\begin{definition}[Permutation from a list of transpositions] Let $as : (A\times A)^*$, then $\fromFP{as} : A \to A$ is defined by recursion on $as$:
\[
    \fromFP{as} = \begin{cases} 
    \mathit{id} & \text{ if } as=[]\\
    \swap{a}{b} \cdot \fromFP{as'} & \text{ if } as =(a,b):as'\\
    \end{cases}
\]
\end{definition}

\begin{definition}[Prefixes]
We say that a non-empty sequence $\rho=[a_1,\ldots,a_n] : A^*$ is a \emph{prefix with head $a_0$} \emph{for} bijection $f$ if:
\begin{enumerate}
    \item $a_0 \in \supp{f}$,
    \item $f\,a_i=a_{i+1}$, and
    \item $a_0\not\in\rho$.
\end{enumerate}
A prefix $\rho$ is \emph{closed} if $f\,a_n = a_0$. Since $\rho$ is non-empty, we denote with $\lastEl{\rho}$ its last element.
\end{definition}

From this simple definition we can deduce:
\begin{lemma}[Properties of prefixes] Let $\rho$ be a prefix with head $a$.
\begin{enumerate}
\item If $\rho'$ is a prefix with head $\lastEl{\rho}$, then its concatenation $\rho\rho'$ is a prefix with head $a$.
    \item $\rho$ has no duplicates.
    \item If $\rho$ is closed and $b\in (a:\rho)$, then $f\,b = \fromFP{\tofp{a}{\rho}}\,b$.
    \item If $b\not\in(a:\rho)$, then $\fromFP{\tofp{a}{\rho}}\,b = b$.
    \end{enumerate}
\end{lemma}

We can extend this definition to a sequence of sequences: let $R = [(a_1,\rho_1),\ldots,(a_m,\rho_m)] : (A\times A^*)^*$, then $R$ is a
list of prefixes, with its head, if each $\rho_i$ is a prefix and $\rho_i\cap \rho_j = \emptyset$.

\begin{lemma}[Correctness of prefixes]
Let $R = [(a_1,\rho_1),\ldots,(a_m,\rho_m)]$ be a list of closed prefixes, then $\fromFP{ \tofp{a_1}{\rho_1}\,\ldots\,\tofp{a_m}{\rho_m}}\,a = f\,a$.
\end{lemma}

This proves that from a representation with cycles one can get
a representation with transpositions. If we can produce a list
of closed prefixes from a finite permutation (as a bijection 
with its support explicitly given) then we have the equivalence.
First we define a function $\mathit{cycle}_f\colon 
\mathbb{N}\times A\to A^*$ such that $\cycle{f}{n}{a}$ computes 
a prefix with head $a$ of length at most $n+1$ by recursion on $n$:
\[
\begin{aligned}
   \cycle{f}{0}{a} &= [f\,a]\\
   \cycle{f}{n+1}{a} &= \begin{cases}
    \rho & \text{ if } f\,b=a\\
    \rho[f\,b] & \text{ otherwise}
   \end{cases}\\
    \text{ where }& \rho = \cycle{f}{n}{a} \text{ and } b = \lastEl{\rho}
\end{aligned}
\]

\noindent We can extend this definition to compute a list of prefixes from a list of atoms:
\[
\begin{aligned}
   \cycles{f}{n}{[]}{R} &= R\\
   \cycles{f}{n}{a:as}{R} &= \begin{cases}
    \cycles{f}{n}{as}{R} & \text{ if } a\in \bigcup R\\
    \cycles{f}{n}{as}{\rho:R} & \text{otherwise}
   \end{cases}\\
   \text{ where }& \rho = a:\cycle{f}{n}{a}
\end{aligned}
\]

\begin{lemma}[Correctness of computed cycles]
If $f\colon A\to A$ is a bijection and $a\in \supp{f}$, then
$\cycle{f}{n}{a}$ is a prefix with head $a$, for all $n\in\mathbb{N}$. Moreover if $|\supp{f}| \leqslant n$, 
then $\cycle{f}{n}{a}$ is closed.

If $R$ is a list of prefixes and $as\subseteq\supp{f}$,
then $\cycles{f}{n}{as}{R}$ is a list of prefixes; if
$|\supp{f}| \leqslant n$, then $\cycles{f}{n}{as}{R}$
is a list of closed prefixes.
\end{lemma}

\begin{theorem}
If $f\colon A\to A$ is a bijection, then $R= \cycles{f}{|\supp{f}|}{\supp{f}}{[]}$ is a list
of closed prefixes. Therefore $\fromFP{\tofpUnS{R}}\,a = f\,a$, for all $a\in A$. 
\end{theorem}

Notice that a composition of transpositions might mention elements that are not in the
support of the induced permutation; for example, both $\swap 1 1$ and $\swap 1 2 \swap 2 1$ are equal
to the identity permutation. One can get a ``normalized'' representation by composing 
our functions. As a matter of fact, this was our motivation to formalize cycles.
\begin{corollary}[Normalization of transpositions] Let $p$ be a list of transpositions and
$\mathit{ats} = \supp{\tofpUn{p}}$. Moreover, let $R = \cycles{\tofpUn{p}}{|\mathit{ats}|}{\mathit{ats}}{[]}$.
Then $\fromFP{\tofpUnS{R}} = \fromFP{as}$; moreover every atom in $\tofpUnS{R}$ is in
its support.
\end{corollary}
