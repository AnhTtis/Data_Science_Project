% ****** Start of file apssamp.tex ******
%
%   This file is part of the APS files in the REVTeX 4.2 distribution.
%   Version 4.2a of REVTeX, December 2014
%
%   Copyright (c) 2014 The American Physical Society.
%
%   See the REVTeX 4 README file for restrictions and more information.
%
% TeX'ing this file requires that you have AMS-LaTeX 2.0 installed
% as well as the rest of the prerequisites for REVTeX 4.2
%
% See the REVTeX 4 README file
% It also requires running BibTeX. The commands are as follows:
%
%  1)  latex apssamp.tex
%  2)  bibtex apssamp
%  3)  latex apssamp.tex
%  4)  latex apssamp.tex
%
\documentclass[%
 reprint,
%superscriptaddress,
%groupedaddress,
%unsortedaddress,
%runinaddress,
%frontmatterverbose, 
%preprint,
%preprintnumbers,
%nofootinbib,
%nobibnotes,
%bibnotes,
 amsmath,amssymb,
 aps,
%pra,
%prb,
%rmp,
%prstab,
%prstper,
%floatfix,
]{revtex4-2}
\usepackage{graphicx}% Include figure files
\usepackage{dcolumn}% Align table columns on decimal point
\usepackage{bm}
\usepackage{tikz}% bold math
\usepackage{appendix}
\usepackage{hyperref}
\usepackage{comment}
\newcommand{\levicivita}{}% initialize
\def\levicivita#1#{\tensor#1{\epsilon}}% add hypertext capabilities
%\usepackage[mathlines]{lineno}% Enable numbering of text and display math
%\linenumbers\relax % Commence numbering lines

%\usepackage[showframe,%Uncomment any one of the following lines to test 
%%scale=0.7, marginratio={1:1, 2:3}, ignoreall,% default settings
%%text={7in,10in},centering,
%%margin=1.5in,
%%total={6.5in,8.75in}, top=1.2in, left=0.9in, includefoot,
%%height=10in,a5paper,hmargin={3cm,0.8in},
%]{geometry}

\begin{document}

\preprint{APS/123-QED}

\title{Anderson Impurities In Edge States with Nonlinear Dispersion}
\author{Vinayak M. Kulkarni}
\affiliation{Theoretical Sciences Unit, Jawaharlal Nehru Centre for Advanced Scientific Research Jakkur, Bangalore - 560064,India}

\date{\today}% It is always \today, today,
             %  but any date may be explicitly specified

\begin{abstract}
A non-linear dispersion can significantly impact the Kondo problem, resulting in anomalous effects on electronic transport. By analyzing a special bath with a $\theta=\frac{\pi}{3}$ symmetry rotation in the Brillouin zone or 3-fold symmetry in momentum, we derive an effective spin-spin interacting model. Combining the anisotropic Dzyaloshinskii-Moriya (DM) interaction with non-linear dispersion can lead to exceptional points $(E_p)$ in a Hermitian model. Our RG analysis reveals that the spin relaxation time has the signature of coalescence in momentum-resolved couplings and an ideal logarithmic divergence in resistivity over a range of nonlinearity ($\beta$). The effective model at the impurity subspace has a Lie group structure of Dirac matrices. We show nontrivial renormalization within a Poorman approximation with the inclusion of potential scattering, and the invariant obtained will not be altered by potential scattering. We expand the model to a two-impurity Kondo model and investigate the Kondo destruction and anomalous spin transport signature by calculating the spin-relaxation time ($\tau$).Analysis of RG equations zeros and poles show a "Sign Reversion" regime exists for a Hermitian problem with a critical value of nonlinear coupling $J_{k^3}$. Our results show the existence of an out-of-phase RKKY oscillation above and below the critical value of the chemical potential.


\end{abstract}

%\keywords{Suggested keywords}%Use showkeys class option if keyword
                              %display desired
\maketitle

%\tableofcontents
\section{\label{sec:level1}Introduction}

%and a realization can be made through cold atoms. 
%These $\mathcal{PT}$-symmetric models are that they mimic open quantum systems with the many-body interactions.  
%For continuum field theories  only $\mathcal{PT}$ symmetry is important as discussed in articles \cite{Extnhbendor},\cite{Extnerra}. 
The occurrence of the Kondo peak and bound states in Topological Insulators\cite{quat_imp,quant_imp1} (TIs) highly depends on various parameters such as topology and chemical potential. The behavior of the resonance level is distinct from that of simple metals and ordinary insulators. In the case of band inversion\cite{dey2018bulk}, the mixed-valence regime is wider, and the coexistence of both the Kondo peak and in-gap bound states is possible, unlike in ordinary insulators where only one exists. If the impurity energy is far from the chemical potential, the in-gap bound states merge into the bulk. Furthermore, a self-screening Kondo effect may occur due to the interaction between the impurity and the bound-state spin.
The study\cite{PhysRevLett.102.136806} predicts the occurrence of quantized conductance in the presence of strong disorder, even in parameters where the system is a metal in the absence of disorder. Unlike previously studied topological insulators, the Fermi energy is located within a mobility gap, and the existence of ballistic edge states does not rely on band inversion. Further investigation is necessary to determine the correlation between the presence of strong disorder and the location of the topological Anderson insulator in the phase diagram.

The analysis presented in the study\cite{imp_2DTI} investigates the impact of an Anderson impurity on a 2D topological insulator. It demonstrates that the exchange interaction between the impurity and an in-gap bound state can undergo dynamic changes. The temperature dependence of the system exhibits crossover behavior, which could provide experimental evidence for the theoretical analysis. In the weak-coupling regime, both screened and underscreened Kondo effects display a modification in their effective coupling constant as a function of temperature.
The interplay between topology and interactions\cite{igor_RSW} was analyzed; analysis shows an interesting relationship between topology and interactions. The critical strength for the interaction is smaller in situations where the total charge of the system does not change during a transition. The critical strength also does not change with a change in the system's topology. The rearrangement of the charge structure in the ordered phases can be studied through spectroscopy or optical response measurements. Further studies of the relationship between interactions and topology can provide new insights.

The effect of thermal bias on the two-impurity Kondo system was studied\cite{therm}. Findings show that Kondo correlations are destroyed for large thermal bias for the dot connected to the hot reservoir. This leads to suppressed electrical and heat flows as the dots get decoupled. Kondo correlations also influence the interdot coupling. For non-negligible antiferromagnetic spin exchange coupling, thermal bias affects the Kondo-to-AFM crossover. The critical value for the crossover increases as thermal bias increases. These observations can be experimentally tested due to advancements in thermoelectrical transport through nanostructures. This work is also important for some engineered nanostructures for thermoelectric systems.

The study of Two Impurity Models (TIK)\cite{affleck1992exact,mitchell2012two,silva1996particle,o2016holographic,sela2011exact,gan1995solution,gan1995mapping,bayat2012entanglement,mross2009two,cho2006quantum,sire1993theory,campo2004thermodynamics,hallberg1997two} has previously focused on a variety of regimes, including over-screened and unscreened phases, as well as impurity triplet and decoupled single dot states. Unlike the single impurity case, TIK presents a complex problem with numerous possible configurations and interactions between impurities. These models have been studied extensively to understand the interplay between magnetic interactions, Kondo physics, and electronic transport properties in complex systems.

The article\cite{PhysRevB.92.041107} examines the effects of Kondo screening and its interplay with scalar disorder on transport properties in Weyl semimetals, leading to Weyl-Kondo physics. At zero temperature, magnetic impurities generate a Kondo resistivity that scales with the number of impurities. At finite temperatures, the strength of the scalar disorder impacts the Kondo resistivity. In the case of weak scalar disorder, the Kondo resistivity has a minimum, but it may not be observed in strong disorder, even if part of the sample is Kondo screened.
\\
We start with a generic model for the bath, which is well-considered for the Topological Insulators(TI)\cite{PhysRevB.82.045122,igor_RSW,zhang2009topological}.By projecting onto the singly-occupied subspace of the dot, we obtain an effective Hamiltonian, which is an  s-d exchange model. We show its lie algebra and connection to conformal field theory (CFT), and it's topological properties. Later we also extend the projected model to a two impurities case which will be a two-impurity kondo model in a topological system. We performed two-loop RG calculations for one impurity and one-loop for the TIK model within the Poorman scaling approximations. We  analytically and numerically  solved the RG equations, 
found invariants, and explored rich phase diagram of the problem. We highlight transport signatures relevant for momentum-resolved measurements of the impurity-bath systems.
\section{Formalism}
We start with the proposed and studied model\cite{PhysRevB.82.045122} in the context of topological insulators. This was used to explicitly extract the bulk and surface contributions from the model perspective. The model reads as the following, 
\begin{equation}
\label{eq:spin_mod}
   \begin{split}
    \hat{H} &=  \alpha k^2_{\parallel}.\mathbb{I}+\beta  k^3_{\parallel}\cos(3\theta) \sigma_z+ \hat{z}.(\vec{k}\times\vec{\sigma}) 
    \end{split}
\end{equation}
Where in the above we have $\vec{k}=k_x \hat{i}+k_y \hat{j}+k_z \hat{k}$ and $\vec{\sigma}=\sigma_x \hat{i}+\sigma_y \hat{j}+\sigma_z \hat{k}$
This model in the eqn \ref{eq:spin_mod} is well studied in the context of topology with such a nonlinear dispersion; the term spin coupling to momentum is studied in the many body context by introducing spin-orbit coupling. We write the above in the second quantized form for spinful fermions as the following,
 \begin{equation}
   \begin{split}
    H &=\psi^\dag \hat{H}\psi
    \end{split}
    \label{eq:Canderson}
\end{equation}
In the above model, the basis vector $\psi^\dag =\begin{pmatrix} c^\dag_{k\uparrow}& c^\dag_{k\downarrow}\end{pmatrix}$.Now we put an impurity with onsite interaction and hybridization with the edge states Hamiltonian as the following,
\begin{equation}
\label{eq:SIAM}
    H_{siam}=\psi^\dag \hat{H}\psi + H_d +\sum_{k\sigma}V_{k}(c^\dag_{k\sigma}d_{\sigma}+hc)
\end{equation}
\begin{figure}[h!] 
    %\centering
\includegraphics[scale=.8]{dirac.eps}
\caption{The band dispersion around the low k points from diagonalization of the bath represents the emergent chiral bands.In this case we set $\beta=0.3$ and $\theta=\frac{\pi}{3}$ we observed more $\beta$ was flattening the bands.}
\label{fig:fixed_point}
\end{figure}
Where in above single impurity model equation \ref{eq:SIAM} $H_{d}=\sum_{\sigma}\epsilon_{d}d^\dag_{\sigma}d_{\sigma}+Un_{d\uparrow}n_{d\downarrow}$ in the bath Hamiltonian we can replace the $k_\parallel$ with $k$ with a parametrization $k_x \pm ik_y=k_\parallel e^{\pm i\theta}$.We do a unitary operation which is k-dependent in bath operators preserving canonical relations, which yields a nonlinear k dependent coefficients as the following,
\begin{equation}
\label{eq:unitary}
    \begin{split}
      \mathcal{U}= \frac{1}{\sqrt{\mathcal{N}}}
\begin{pmatrix}
    e^{i \frac{\theta}{2} } \alpha_{k1} & e^{-i \frac{\theta}{2} } \alpha_{k2} \\
 -i e^{i \frac{\theta}{2} } \alpha_{k2} & i e^{-i \frac{\theta}{2} } \alpha_{k1}
\end{pmatrix}
    \end{split}
\end{equation}
In the above equation \ref{eq:unitary}, we have 
$\alpha_1= \sqrt{\frac{\Delta }{2}+\beta k^3 \cos3\theta}$ and 
$\alpha_2=\sqrt{\beta k^3 \cos3\theta-\frac{\Delta }{2}}$, where $\Delta=\sqrt{4\beta^2k^6 \cos^2 3\theta +4k^2}$, and the normalization constant is given by $\mathcal{N}=\sqrt{|\alpha_1|^2+|\alpha_2|^2}$.

This unitary transformation will rotate the original bath operators as $\tilde{\psi}_k=\mathcal{U}_k \psi_k$ and such k dependent operation shown in case of weyl multiplicity\cite{k-dependent} exist, which is also a different form of nonlinear dispersion. Also note that the expression $\sqrt{\beta k^3 \cos3\theta-\frac{\Delta }{2}}=-i\sqrt{\frac{\Delta }{2}-\beta k^3 \cos3\theta}$ is also a unitary operator. The reason for choosing this particular form is that it leads to a simpler normalization constant for the unitary, which makes subsequent analysis easier.
\begin{equation}
    \begin{split}
        \tilde{\psi}&=\begin{pmatrix}
            c_{k+}\\c_{k-}
        \end{pmatrix}=\mathcal{U}\psi\\
         c_{k+}&=\frac{1}{ \sqrt{\beta k^3 \cos3\theta}} \bigg(e^{i \frac{\theta}{2} }\sqrt{\frac{\Delta }{2}+\beta k^3 \cos3\theta}  c_{k\uparrow}\\
        &+ e^{-i \frac{\theta}{2}} \sqrt{\beta k^3 \cos3\theta-\frac{\Delta }{2}} c_{k\downarrow}\bigg)\\
        c_{k-}&=\frac{1}{\sqrt{\beta k^3 \cos3\theta}} \bigg(-i e^{i \frac{\theta}{2} } \sqrt{\beta k^3 \cos3\theta-\frac{\Delta }{2}}  c_{k\uparrow} \\
        &+ i e^{-i \frac{\theta}{2} }\sqrt{\frac{\Delta }{2}+\beta k^3 \cos3\theta} c_{k\downarrow}\bigg)
    \end{split}
\end{equation}
Inverting the above unitary operator to express the original spin basis to these new chiral operators,
\begin{equation}
    \begin{split}
        c_{k\uparrow}&=\frac{1}{ \sqrt{\beta k^3 \cos3\theta}} \bigg(e^{-i \frac{\theta}{2} }\sqrt{\frac{\Delta }{2}+\beta k^3 \cos3\theta}  c_{k+}\\
        &+ ie^{-i \frac{\theta}{2}} \sqrt{\beta k^3 \cos3\theta-\frac{\Delta }{2}} c_{k-}\bigg)\\
        c_{k\downarrow}&=\frac{1}{\sqrt{\beta k^3 \cos3\theta}} \bigg( e^{i \frac{\theta}{2} } \sqrt{\beta k^3 \cos3\theta-\frac{\Delta }{2}}  c_{k+} \\
        &-i e^{-i \frac{\theta}{2} }\sqrt{\frac{\Delta }{2}+\beta k^3 \cos3\theta} c_{k-}\bigg)
    \end{split}
\end{equation}
This will effectively generate a chiral model with the eigenergies as $\epsilon_{\pm}=\alpha k^2 \pm \frac{1}{2}\Delta$ and the hybridization as the following,
\begin{equation}
    \begin{split}
        \tilde{H}^{+}_{hyb}&=\sum_k \tilde{V}_k\bigg(e^{i \frac{\theta}{2} }\sqrt{\frac{\Delta }{2}+\beta k^3 \cos3\theta}  c^\dag_{k+} d_{\uparrow}+hc)\\
       & +\sum_k \tilde{V}_k\bigg(e^{i \frac{\theta}{2} } \sqrt{\beta k^3 \cos3\theta-\frac{\Delta }{2}}  c^\dag_{k+} d_{\downarrow}+hc\bigg)\\
        \tilde{H}^{-}_{hyb}&=\sum_k \tilde{V}_k\bigg( -ie^{i \frac{\theta}{2}} \sqrt{\beta k^3 \cos3\theta-\frac{\Delta }{2}} c^\dag_{k-} d_{\uparrow}+hc\bigg)\\
       & +\sum_k \tilde{V}_k\bigg(i e^{i \frac{\theta}{2} }\sqrt{\frac{\Delta }{2}+\beta k^3 \cos3\theta} c^\dag_{k-} d_{\downarrow}+hc\bigg)
    \end{split}
\end{equation}
This will create the $\frac{3}{2}$ singularity of momentum in the hybridization as $\tilde{V}_k=\frac{V_k}{\sqrt{\beta k^3 \cos 3\theta}}$ distinguishing it from the square root singularity by the Rashba coupling studies\cite{zarea,sandler2,sandler3,Zitko}.In order to simplify the symbolic cumbersome, let's introduce two k-dependent constants as $\alpha_{k1}=\sqrt{\frac{\Delta }{2}+\beta k^3 \cos3\theta}$ and  $\alpha_{k2}= \sqrt{\beta k^3 \cos3\theta-\frac{\Delta }{2}}$. If we look at the $\theta=\frac{\pi}{3}$ 3-fold rotations might seem that the hybridization has become complex, but it will introduce a phase factor, but overall hybridization  remains Hermitian. Hybridization in the new notation appears as follows,
\begin{equation}
    \begin{split}
        \tilde{H}^{+}_{hyb}&=\sum_k \tilde{V}_k \alpha_{k1}\bigg(e^{i \frac{\theta}{2} }  c^\dag_{k+} d_{\uparrow}+hc\bigg)\\
       & +\sum_k \tilde{V}_k\alpha_{k2}\bigg(e^{i \frac{\theta}{2} }   c^\dag_{k+} d_{\downarrow}+hc\bigg)\\
        \tilde{H}^{-}_{hyb}&=\sum_k \tilde{V}_k \alpha_{k2}\bigg( -ie^{i \frac{\theta}{2}}  c^\dag_{k-} d_{\uparrow}+hc\bigg)\\
       & +\sum_k \tilde{V}_k \alpha_{k1}\bigg(i e^{i \frac{\theta}{2} } c^\dag_{k-} d_{\downarrow}+hc\bigg)
    \end{split}
\end{equation}

We project this model to impurity subspace using the standard Hewson's projection operator method. The projection operators are $P_0=(1-n_\uparrow)(1-n_\downarrow)$, $P_1=n_\uparrow(1-n_\downarrow)+n_{\downarrow}(1-n_{\uparrow})$ and $P_2=n_\uparrow n_\downarrow$, respectively, for unoccupied, singly occupied and doubly occupied states. We employ the projections and write the following components of the effective model,
\begin{equation}
    \begin{split}
        \tilde{H}^{+}_{10}&=\sum_{k} \tilde{V}_{k}\alpha_{k1}e^{-i\frac{\theta}{2}}d^\dag_{\uparrow}(1-n_{\downarrow})c_{k+}\\
        &+\sum_{k}e^{-i\frac{\theta}{2}} \tilde{V}_{k}\alpha_{k2}d^\dag_{\downarrow}(1-n_{\uparrow})c_{k+}\\
        \tilde{H}^{-}_{10}&=\sum_{k} i\tilde{V}_{k}\alpha_{k2}e^{-i\frac{\theta}{2}}d^\dag_{\uparrow}(1-n_{\downarrow})c_{k-} \\
        &+\sum_{k} -ie^{-i\frac{\theta}{2}} \tilde{V}_{k}\alpha_{k1}d^\dag_{\downarrow}(1-n_{\uparrow})c_{k-}
    \end{split}
\end{equation}
Similarly, we have the projections connecting doubly and singly occupied space.
The $H_{02}$ and $H_{20}$ will vanish. $H_{12}$ can be written for this form of hybridization,
\begin{equation}
    \begin{split}
        \tilde{H}^{+}_{12}&=\sum_{k} \tilde{V}_{k}\alpha_{k1}e^{i\frac{\theta}{2}}c^\dag_{k+}d_{\uparrow}n_{\downarrow}
        +\sum_{k}e^{i\frac{\theta}{2}} \tilde{V}_{k}\alpha_{k2}c^\dag_{k+}d_{\downarrow}n_{\uparrow}\\
        \tilde{H}^{-}_{12}&=-\sum_{k} i\tilde{V}_{k}\alpha_{k2}e^{i\frac{\theta}{2}}c^\dag_{k-}d_{\uparrow}n_{\downarrow} 
        +\sum_{k}ie^{i\frac{\theta}{2}} \tilde{V}_{k}\alpha_{k1}c^\dag_{k-}d_{\downarrow}n_{\uparrow}
    \end{split}
\end{equation}
Remaining components are  $H_{00}= \sum_{k\alpha} \epsilon_{k\alpha} c^\dag_{k\alpha}c_{k\alpha}P_0+\sum_{\sigma}\epsilon_d n_{\sigma}P_0$ and  $H_{22}= \sum_{k\alpha} \epsilon_{k\alpha} c^\dag_{k\alpha}c_{k\alpha}P_2+\sum_{\sigma}\epsilon_d n_{\sigma}P_2+Un_\uparrow n_\downarrow P_2$. Now we have all the components to calculate the effective models in various sub-spaces.
\begin{equation}
    \begin{split}
        H^{0}_{eff}&=H_{00}+\bigg(\sum_{\alpha}H^\alpha_{01}\bigg)\frac{1}{E-H_{11}}\bigg(\sum_{\alpha}H^\alpha_{10}\bigg)\\
        H^{1}_{eff}&=H_{11} +\bigg(\sum_{\alpha}H^\alpha_{10}\bigg)\frac{1}{E-H_{00}}\bigg(\sum_{\alpha}H^\alpha_{01}\bigg)\\
        &+\bigg(\sum_{\alpha}H^\alpha_{12}\bigg)\frac{1}{E-H_{22}}\bigg(\sum_{\alpha}H^\alpha_{21}\bigg)\\
        H^{2}_{eff}&=H_{22}+\bigg(\sum_{\alpha}H^\alpha_{21}\bigg)\frac{1}{E-H_{11}}\bigg(\sum_{\alpha}H^\alpha_{12}\bigg)
    \end{split}
\end{equation}
The singly occupied subspace is the low-energy effective model for the Kondo regime. We want to explore  the non-Kondo regime Hamiltonian to explore the operator structures in the later sections. We detail the calculations in the appendix for an effective sd exchange model. 
\begin{equation}
\label{eq:spineff}
    \begin{split}
       H^{1}_{eff}=H_0 -i\alpha^2_{1}M_{kk'}S_{-}s^{+}_{kk'}+i\alpha^2_{2}M_{kk'}s^{-}_{kk'}S_{+}\\
        \alpha^2_{1}M_{kk'}S_{z}s^{z}_{kk'}+\alpha^2_{2}M_{kk'}s^{z}_{kk'}S_{z}\\
        \alpha_{1}\alpha_{2}M_{kk'}S_{-}s^{z}_{kk'}+\alpha_1\alpha_{2}M_{kk'}s^{z}_{kk'}S_{+}\\
        i\alpha_{1}\alpha_{2}M_{kk'}S_{z}s^{-}_{kk'}-i\alpha_1\alpha_{2}M_{kk'}s^{+}_{kk'}S_{z}
        \end{split}
\end{equation}
In above equation \ref{eq:spineff} $M_{kk'}=\tilde{V}_{k}\tilde{V}_{k'}\bigg(\frac{1}{\epsilon_{k'}-\epsilon_{d}}+\frac{1}{\epsilon_{d}+U-\epsilon_{k}}\bigg)$
The above spin-spin interaction model has an algebraic structure similar to the CFT paper\cite{loure}. We can notice some cross-product terms pop out in such a nonlinear dispersion similar to the $\mathcal{DM}$ interaction term\cite{zarea} for the chiral channel but in the spin basis of the bath operators; 
\begin{equation}
\label{eq:eff}
    \begin{split}
         H^{1}_{eff}&=H_0+\sum_{kk'}J_0 S.s_{kk'}
         +i\sum_{kk'}\vec{J}_{k^3}.(S\times s_{kk'})\\
         &+i\sum_{kk'}\vec{J}_{k}.(S\times s_{kk'})+H_{pot}
    \end{split}
\end{equation}
where in the above $\vec{J}_{k^3}=\frac{M^\theta_{k}}{\beta k^3 \cos3\theta}\Delta\hat{z}$ and $\vec{J}_k= \frac{kM}{\beta k^3 \cos3\theta} \hat{x}+\frac{k\Delta}{\beta k^3 \cos3\theta}M^{\theta}\hat{y}$ 
Also, $H^\dag=H$ since $\alpha_{1,2}$ are real constants and in the limit $\Delta \to 0$, we have a standard hermitian problem with Anderson impurity in a bath. Note that in this limit model no longer has anisotropic$\mathcal{DM}$interaction.  The band electrons' left and right mover can be defined and shown in the figure \ref{fig:pmrl} for the above model. After Poorman scaling is done for the magnitude of the vectors as defined in the equ \ref{eq:eff}, we collect the contributions for RG equations in terms of these magnitudes.
\begin{figure}
    \centering
    \includegraphics[scale=0.55]{3d_dir_two1.eps}
    \caption{A schematic representing two Anderson impurities in edge states. We separate the left and right scatterers with green and blue arrows.$k_\parallel \to k$ and $\theta\to \tan^{-1}\frac{k_y}{k_x}$ see derivation. Scatterings for the $(k,\theta)$ to $(-k,\theta')$ in Poorman scaling are represented schematically in a 3D shell.}
    \label{fig:pmrl}
\end{figure}
\begin{equation}
    \begin{split}
        \frac{dJ_0}{dl}&=J^2_0+J_{k^3}J_k+J^2_{k^3}+J^2_k\\
        \frac{dJ_{k^3}}{dl}&=J^2_k+J_0 J_{k^3}\\
        \frac{dJ_k}{dl}&=J_0 J_k
    \end{split}
\end{equation}
One solution of the above RG equations $ J^2_k -J_k=m J_{k^3}$ where m is an invariant and remaining solution, we detail the analytic solution in the appendix.

\section{Emergence Of Complex Solution}
Notably, the emergence of complex solutions to RG equations due to nonlinear dispersion has nontrivial renormalization. This is also seen in the one-loop Poorman equations, which yield the emergent exceptional point scenario \cite{loure,poor} as RG reversion. Moreover, adding second-order self-energy to the original Hamiltonian leads to non-Hermitian physics \cite{equi}. In open systems, adding finite-order self-energies can also give rise to effective non-Hermitian models \cite{dot_bias, kulkarni2022derivation, ep_dmft}. In a theoretical study, it has been shown that complex solutions emerge from various types of potentials and their symmetries \cite{fring2020complex}.
\begin{figure}[h!] 
    %\centering
\includegraphics[scale=1.1]{EgTOPOKondo.eps}
\caption{The above plots show the emergence of coalescing point inside the fermi-level. These points are solely due to the renormalization of the linear and nonlinear couplings in the spin sector(1,1). Where m is RG invariant which arises  from the one loop poorman scaling.}
\label{fig:small_J}
\end{figure}
\begin{figure}[h!] 
    %\centering
\includegraphics[scale=1.1]{EgTOPOKondo_largeJ.eps}
\caption{The above plots show the spin sector(1,1) eigenvalues for large $J_0 >1.0$. The eigenvalues become flatter as the bare coupling increases, but coalescence remains, but it only disappears for $J_0 \to \infty$}
\label{fig:large_J}
\end{figure}

Based on the plots in Figures \ref{fig:small_J} and \ref{fig:large_J}, one might conclude that the Dirac cone has disappeared, but in reality, it still exists in the bath. We are only visualizing the eigenvalues of the sd model, which is more relevant to impurity and dirac cone exist\ref{fig:small_J} for resonant level case, which depend on the flown couplings. The spectrum becomes gapped in impurity in large $J_0$ limit. By substituting the RG invariant $J_{k}=\frac{1}{2}\pm \frac{1}{2}\sqrt{1+4mJ_{k^3}}$ into the sd model eigenvalues, we see that this is the source of the coalescing points. Interestingly, these points disappear in the limit of $J_0 \to \infty$. To examine the relevance of potential scattering and its effect on particle-hole symmetry, we diagonalize the flown $H^1_{eff}$ at the single occupancy sector for only the sd part. In real materials, analogous exceptional points have been observed, and in calculations, they have been tuned \cite{zhen2015spawning, betancourt2016complex}. Thus, finite nonlinear dispersion can generate a complex solution and significantly modify the RG flow of the Kondo problem, as shown by the Poorman solution.
 \section{Renormalization With Potential Scattering}
 To study the various momentum channel scattering as shown in the fig \ref{fig:poorman} ,which is interestingly yields rich physics. Largely we can write the effective model derived\ref{eq:eff} with scattering terms  as follows,
 \begin{equation}
     \begin{split}
         (H')^1_{eff}&= H^1_{eff}+\\
         &\begin{pmatrix}
             [k,\theta_{+2\frac{\pi}{3}}]\Leftrightarrow [k,\theta_{+\frac{\pi}{3}}] & [k, \theta_{+2\frac{\pi}{3}}]\Leftrightarrow [-k,\theta_{ -\frac{\pi}{3}}]\\
             [-k,\theta_{-2\frac{\pi}{3}}]\Leftrightarrow [k,\theta_{+\frac{\pi}{3}}] & [-k,\theta_{-2\frac{\pi}{3}}]\Leftrightarrow [-k,\theta_{-\frac{\pi}{3}}]
         \end{pmatrix} 
     \end{split}
 \end{equation}
As discussed above, we can rewrite the effective Hamiltonian considering such a potential scattering. We consider the basis of bath in the new scattering as $\psi^\dag=\begin{pmatrix}
    c^\dag_{k_{\theta_{\frac{2\pi}{3}}} +}, &c^\dag_{k_{\theta_{\frac{\pi}{3}}} -}, & c^\dag_{-k_{\theta_{-\frac{2\pi}{3}}} +}, &c^\dag_{-k_{\theta_{-\frac{\pi}{3}}} -}
\end{pmatrix}$ which makes the block structure of the kondo problem and represents the couplings $\vec{J}^{\theta,\theta'}_{k^3}=M^{\theta,\theta'}_{kk'}(\Delta+\beta k^3 \cos(3\theta) )\hat{z}$ and $\vec{J}^{\theta,\theta'}_k= \frac{k}{\beta k^3 \cos3\theta}M^{\theta,\theta'}\hat{x}+\frac{k\Delta}{\beta k^3 \cos3\theta}M^{\theta,\theta'}\hat{y}$ 
\begin{equation}
\label{eq:gen_eff}
    \begin{split}
         H^{1}_{eff}&=H_0+\sum_{kk'}J_0 S.\psi^\dag(\Sigma)\psi\\
         &+i\sum_{kk'}\vec{J}^{(0,0)}_{k^3}.(S\times \psi^\dag(\Sigma)\psi)\\
         &-\sum_{kk'} \vec{J}^{\theta_{\pm\frac{2\pi}{3}},\theta_{\pm\frac{\pi}{3}}}_{k^3}.(S\times \psi^\dag(\Omega)\psi)\\
         &+\sum_{kk'}\vec{J}^{\theta_{\frac{2\pi}{3}},\theta_{-\frac{\pi}{3}}}_{k^3}.(S\times \psi^\dag(\Gamma)\psi)\\
         &+i\sum_{kk'}\vec{J}^{(0,0)}_{k}.(S\times \psi^\dag(\Sigma)\psi)\\
         &-\sum_{kk'}\vec{J}^{\theta_{\pm\frac{2\pi}{3}},\theta_{\pm\frac{\pi}{3}}}_{k}.(S\times \psi^\dag(\Omega)\psi)\\
         &+i\sum_{kk'}\vec{J}^{\theta_{\frac{2\pi}{3}},\theta_{-\frac{\pi}{3}}}_{k}.(S\times \psi^\dag(\Gamma)\psi)
    \end{split}
\end{equation}

\section{Comparison With CFT scaling laws at two loops}
We can see the algebraic consistency in the  earlier works \cite{loure,affleck1992exact,affleck1993exact} using the lie  algebra of matrices, which is similar to the Poorman scaling for the operator structures but CFT will be exact since it incorporate the bath states more accurately. This idea can also be extended to the two impurity problems.After introducing the potential scattering, we have more general model\ref{eq:gen_eff} in the lie matrices, which are defined as follows,
\begin{equation}
\label{eq:lie}
    \begin{split}
        \Sigma=\begin{pmatrix}
            \sigma &\sigma\\
            \sigma & \sigma 
        \end{pmatrix}, \Omega=\begin{pmatrix}
            \sigma &0\\
            0& -\sigma 
        \end{pmatrix}, \Gamma=\begin{pmatrix}
            \sigma &0\\
            0 & \sigma 
        \end{pmatrix}
    \end{split}
\end{equation}
where in the above equation\ref{eq:lie} $\sigma$ are the Pauli matrices 

\begin{equation}
\label{eq:lie_mat}
    \begin{split}
        [\Sigma^a \Sigma^{b},\Sigma^c]&=[\Gamma^a \Gamma^b,\Gamma^c]=[\Omega^a \Omega^b,\Omega^c]\\
        &=0 \\
        [\Sigma^\alpha \Sigma^{\alpha'},\Sigma^\alpha]&=-4i\delta_{\alpha,\alpha'}\Sigma^\alpha\\
        [\Sigma^b \Sigma^b,\Sigma^c]&=\Sigma^b[\Sigma^b,\Sigma^c]+[\Sigma^b,\Sigma^c]\Sigma^b\\
        &=+8i\epsilon^{abc}\Sigma^c\\
        [\Omega^b \Omega^b,\Omega^c]&=\Omega^b[\Omega^b,\Omega^c]+[\Omega^b,\Omega^c]\Omega^b\\
        &=+4i\epsilon^{abc}\Omega^c\\
        [\Omega^b \Omega^b,\Sigma^c]&=\Omega^b[\Omega^b,\Sigma^c]+[\Omega^b,\Sigma^c]\Omega^b\\
        &=+4i\epsilon^{abc}\Omega^c+4i\epsilon^{abc}\mathit{I}_{4X4}\\
        [\Sigma^a\Sigma^b,\Upsilon]&=\Sigma^a[\Sigma^b,\Upsilon]+[\Sigma^a,\Upsilon]\Sigma^b
    \end{split}
\end{equation}
\begin{equation}
    \begin{split}
    \frac{dJ_{0}}{dl}&=J^2_0+J^2_{k^3}+J^2_{k}+J_{k^3}J_{k}+J^2_{k}J_0+J^2_{k^3}J_0+J^3_0\\
    &+f(J_{k^3},g_{2k},g_{2k^3},J_{k})\\
        \frac{dJ_{k^3}}{dl}&=J^2_{k}+J_{0}J_{k^3}+J^2_{k}J_{k^3}+J^2_0J_{k^3}+J^3_{k^3}\\
        \frac{dJ_{k}}{dl}&=J_0J_k+J^2_0J_k+J_kJ^2_{k^3}+J^3_{k}\\
        \frac{dg_{1k^3}}{dl}&=g^3_{1k^3}-g^2_{1k}g_{1k^3}\\
        \frac{dg_{1k}}{dl}&=g^3_{1k}-g^2_{1k^2}g_{1k}\\
        \frac{dg_{2k^3}}{dl}&=-g^3_{2k^3}+g^2_{1k^3}g_{2k^3}\\
        \frac{dg_{2k}}{dl}&=-g^3_{2k}+g^2_{1k}g_{2k}
    \end{split}
\end{equation}
We have given some details on calculations of the above rg equations in Appendix B.
\begin{figure}[h!] 
    %\centering
\includegraphics[scale=.4]{FullFlow_pot.eps}
\caption{We set $g_{1k}=g_{1k^3}=1$ which is invariant derived in appendix for all calculations $J_{1k}=\frac{1}{2}\pm \frac{1}{2}\sqrt{1+4mJ_{1k^3}}$ is used and $m=-4$ we set for all above plots.We plot the grid as $\begin{pmatrix}
    pot & pot1\\
    R^{+}R^{+}  & R^{+}I^{+} \\
    I^{-}R^{-}  & I^{-}I^{-}
\end{pmatrix}$ first row is potential scattering, and the second row is in both beta functions chosen real and imaginary part of the positive root. We see at least two fixed points, two spirals in (RR, RI) row, and one spiral, and one is marginal. The dotted lines in the second plot of the first row are a family of fixed points.}
\label{fig:fixed_point1}
\end{figure}
The spiral fixed points in non-hermitian systems are demonstrated earlier\cite{kulkarni2021functional,han2023complex}; also, in three and higher dimensions, spiral invariants obtained\cite{PhysRevD.69.025004}. Additionally, the potential scattering terms emerged spiral fixed points in all channels more relevant to such a system.
\section{Generalization to Two Impurities}
We generalize this model to two impurity  kondo models with coupling between the impurities. One can also start with the projection for two impurities Anderson models to derive the two impurity kondo model from getting exact matrix elements for coupling between the dots. Two dots with spin-orbit coupling\cite{tkss} is studied earlier, which shows that $\mathcal{DM}$ term exists for only Y-component and linear dispersion in the bath. In this case, it generates a more general form of $S_{1}\times S_2$ term in all components.



\begin{equation}
    \begin{split}
     H^{1}_{eff}&=H_0+\sum_{kk'}J_0 S_{\alpha}.s_{kk'}
         +i\sum_{kk'\alpha}\vec{J}_{k^3}.(S_{\alpha}\times s_{kk'})\\
         &+i\sum_{kk'\alpha}\vec{J}_{k}.(S_{\alpha}\times s_{kk'})+H_{pot}\\
         &+J_{Y}S_{1}.S_{2} + i K.(S_1\times S_2)
    \end{split}
\end{equation}
With the new coupling, we see how it renormalizes the problem; immediately, we can notice that the RKKY coupling $J_Y$ modifies the single impurity invariant. Since the generalized problem  has many coupling, we restrict our-self to analyzing the one-loop RG equations.,
\begin{equation}
\label{eq:RGeqn}
    \begin{split}
        \frac{dJ_0}{dl}&=J^2_0+J_{k^3}J_k+J^2_{k^3}+J^2_k+J_{Y}J_0+KJ_0\\
        \frac{dJ_{k^3}}{dl}&=J^2_k+J_0J_{k^3}+J_YJ_{k^3}+J_YJ_k+KJ_{k^3}\\
        \frac{dJ_k}{dl}&=J_0 J_k+J_YJ_{k^3}+ J_Y J_k+KJ_{k}\\
        \frac{d J_{Y}}{dl}&=J^2_{Y}+K^2+J^2_0+J_{k^3}J_{k}+J^2_{k}+J^2_{k^3}\\
        \frac{d K}{dl}&= K^2+KJ_Y+J_{k^3}J_k
    \end{split}
\end{equation}
Solutions to the above equations are detailed in an appendix in various limits. The beta function zeros for kondo destruction can be seen in the odd-even couplings given in articles\cite{PhysRevLett.58.843,PhysRevLett.61.125}. Since we focus on the nonlinear couplings $J_k , J_{k^3}$, We look at the solutions around the anomalous contributions to Spin-relaxation time and the fixed points in these couplings. 
\section{Kondo Scale in two Impurity}
We integrate the RG equations to get the scales for the problem in various limits so that how the nonlinear coupling affects this problem. First in limit $J_{k^3}=J_k=K=0$ we have the solution $J_0=\frac{e^{-g}}{(1-g)^2}$ by integrating the $J_0,J_Y$ equations, where $g=\frac{J_Y}{J_0}$. We solve when impurity $\mathcal{DM}$ interaction $K\neq 0$  in the appendix. Figure \ref{fig:RKKY_scales} show the variation of $T_K$ with various coupling limits. The two invariants $R, R_1$ modify the TIK problem significantly,  "R" cause the destruction of the TIK bound state, and $R_1$ try to stabilize the bound state which can be seen in the middle plot in figure \ref{fig:RKKY_scales}. 
\begin{figure}
    \includegraphics[scale=.36]{RKKYscales.eps}
    \caption{We plot the scales, as shown above, to show competition between RKKY and impurity$\mathcal{DM}$ interactions.An Inset graph in K small label is an analytic expression derived when impurity $\mathcal{DM}$ interaction is absent and $g=\frac{J_Y}{J_0}$ in the x-axis. Discontinuity in the  scale for K Large limit  is shown with the dotted line where it is ill-defined. }
    \label{fig:RKKY_scales}
\end{figure}
\begin{figure}
    \includegraphics[scale=.38]{RG_fullsoln.eps}
    \caption{We plot the solutions of the RG equations around the critical points from the flow diagram. the x-axis is the bandwidth diverging points corresponding to the kondo scale, and cusp-like behaviors indicate the two impurity model fixed points. The inset graph shows the non-interacting fixed point we got when we set either $J_Y$ or $K$ to negative otherwise, we have two impurities Kondo effect, where all couplings diverge. Sign reversion in both can be seen for the couplings $J_0, K$ when for both $J_{k^3},J_{k}$ given -ve initial conditions(IC) are given. Otherwise, only the $K$ reverses sign with cusp $-\infty\to\infty$.With various IC's, we verified it is necessary to have $J_{k^3}\neq 0$ to get cusp-like sign reversion in the couplings. IC's are mentioned above each plots above.  }
    \label{fig:RKKY}
\end{figure}
\begin{figure}
    \includegraphics[scale=.42]{RKKYRG.eps}
    \caption{RG flows for the two impurity problems when anisotropic $\mathcal{DM}$ interaction due to bath is absent. See the RG solutions in  the appendix. There are three fixed points from these solutions one is TIK point(blue), decoupled dots(Green), and dot triplet state(blue in the right plot). The quadrant of these points depends on the integral constants(either in third or first), and we refer here to the $-ve$ sign for ferromagnetic couplings. }
    \label{fig:RKKY_DM}
\end{figure}

\section{RG Equations Zeros And Poles Analysis} 
We realize that the solutions to couplings  are divergent;  therefore, it becomes trickier to separate the two impurity kondo regime, single impurity kondo regime, and the sign reversion of couplings due to nonlinear coupling $J_{k^3}$. We solve the RG ODE's by allowing complex solutions and analyzing solutions around the fixed points found in flow diagrams.
If there is a phase transition, then couplings will flow to a stable fixed point, and hence we should see poles in functions and at unstable points only zero crossings. This serves as the numerical diagnosis for the identification of various phases.
 \begin{figure}
    \includegraphics[scale=.36]{RG_zero_pole.eps}
    \caption{We plot the RG solutions by allowing the complex solutions of RG nonlinear equations. The dotted and thick lines correspond to real and imaginary coupling values. Sign reversion in all couplings is captured when we have a critical value of the nonlinear coupling $J_{k^3}>2.0$. Between the two sign-reversing phases in plots$a \to f$, there are various 2-pole(b,c,e) and 3-pole(d) regimes. This confirms the spiral fixed points we got are due to the sign reversion in couplings. This happens only at two initial conditions. Scales above the $10^3$ are not labeled in y - the axis and all plots' x-axis is a flow parameter which is either in log/linear scale.}
    \label{fig:scaling}
\end{figure}
\section{Impurity Transport Calculation}
We have derived the anomalous  contributions for the relaxation time in the appendix due to the presence of the nonlinear dispersive bath following the $\mathcal{T}_{kk'}$ formalism\cite{hews} which is detailed in appendix F,
\begin{equation}
    \begin{split}
        \frac{1}{\tau(k)}\propto (1-2J \tilde{g}_{\alpha}(\epsilon_k)-2J\tilde{g}_{\alpha}^* (\epsilon_k)-2J_{k^3} \tilde{g}_{\alpha k^3}(\epsilon_k)\\
        -2J_{k^3}\tilde{g}_{\alpha k^3}^* (\epsilon_k)-2J_k \tilde{g}_{\alpha k}(\epsilon_k)-2J_k \tilde{g}_{\alpha k}^* (\epsilon_k))
    \end{split}
\end{equation}
We are addressing coalescing points from a hermitian model with any transport signature. The appendix evaluates these functions as contour integrals for polar components and definite integrals for radial fermi vector($k_f$). Spin relaxation time will reflect the RG invariant as $\frac{1}{\tau}\propto  \tilde{g}_\epsilon=(1\pm\sqrt{1+4\epsilon})f(\epsilon,\epsilon')$ implying the momentum resolved couplings have the signature of coalescing points which can be seen in spin relaxation time.
\begin{figure}[h!] 
    %\centering
\includegraphics[scale=.42]{tau_scaling.eps}
\caption{We calculated the spin-relaxation time to see the coalescing point signature with the nonlinear coupling $\beta$. as the $\beta$ decreases, ideal log divergence reaches but dips persist in $\frac{1}{\tau}$. Left plot to confirm if the anomalous contribution is from the coalescence, we divided each curve by lowest $\beta=0.01$ curve and scaled for $\pm$ roots as $\tau^{-1}_{\pm r}=\tau^{-1}*(1\pm\sqrt{4(\epsilon-0.25)+1})$ for the y-axis and $D_{eff}*(1\pm\sqrt{4(\epsilon-0.25)+1})$ for the x-axis. Increasing $\beta$, we see polynomial contributions dominate; hence the scaling will be absent.}
\label{fig:transport}
\end{figure}
The matrix elements scale as $M_{kk'} \approx \frac{1}{\sqrt{k^3\cos(3\theta)}\sqrt{(k')^3\cos{3\theta'}}}$ and when $k=-k'$ there is an imaginary coupling which creates a non-Hermitian Kondo problem with a bath that has non-linear dispersion. Various studies of non-Hermitian Kondo systems exist, considering different perspectives, symmetry considerations, and complex spin exchange interactions\cite{poor, loure, yoshimura2020non, kulkarni2022kondo}. In the presence of inversion symmetry, non-Hermiticity can result from $\theta' \to \frac{\pi}{3}-\theta$, where a Dirac point exists in the Brillouin zone. In the appendix, it is discussed that $\alpha_{1k}\alpha_{2k'}$ beyond the poor man's limit (i.e., $k \neq k'$) results in a general non-Hermitian model at impurity due to nonlinear parametric dispersion in the bath.
  \begin{figure}
    \includegraphics[scale=.4]{RKKY.eps}
    \caption{We perform The integrals for $J_Y$ given in the appendix and plot by varying the bandwidth.RKKY is enhanced for $\beta < 1.0$ and approaches to ideal flatt band limit for $\beta >1.0$. As we saw in the relaxation time here, also we can see above, and below a critical value of chemical potential ($\epsilon'=\pm 0.25$), $J_Y$ reverses sign and goes out of phase.}
    \label{fig:my_label}
\end{figure}
 \begin{figure}
    \includegraphics[scale=.4]{EP_qcp.eps}
    \caption{We plot the elliptic functions which arose in calculating the RKKY integrals. For negative chemical potential $\epsilon'$, there is sign reversion everywhere else we see the exceptional point behaviors. The EP's start shifting for increasing +ve $\epsilon'$, but there is no sign of reversion in the real part of these special functions. }
    \label{fig:scaling}
\end{figure}

\section{Discussion}
This work investigates the possibility of coalescing points in a bulk system having magnetic impurities through spin relaxation time calculations in different momentum directions. Our results show that two distinct calculations share a common invariant as renormalization group (RG) solutions, with or without potential scattering. Increasing nonlinearity strength causes bands to flatten in the bath spectrum while the impurity resistivity reaches the ideal log divergence for a $\beta$ strength range. Furthermore, we analyze RG calculations on single and two impurity problems in a homogeneous bath and find that the competition between RKKY and Kondo interactions is only observed when the impurity $\mathcal{DM}$-interaction is absent. We also show that to obtain coalescing points in hermitian Kondo models, anisotropic $\mathcal{DM}$ interactions are necessary due to the $k^3$ and $k$ terms in the dispersion, which naturally yield the roots as the RG invariants. These results are significant as they provide a signature in transport, which we demonstrate through numerical diagnosis for critical values of $J_{k^3}$ and $J_{k}$ in the couplings, confirmed by flow diagrams. The sign-reversion regime in the couplings corresponds to the spiral fixed points of the problem. Out-Of-Phase oscillations in the RKKY are due to the presence of the special bath, as we ruled out the other possibilities by analyzing the elliptic functions EP. Future work could extend these findings to charge and thermal transport calculations and investigate impurity effects in systems with anisotropic $\mathcal{DM}$ interaction and in Weyl and topological systems with nonlinear dispersion. One may expect loss of unitarity in sub system as we have seen here in single occupied subspace if we analyze the system in other subspaces $H^0_{eff},H^2_{eff}$ may be there will be global unitarity which is not done in current work. 

The results of our work demonstrate the potential for more advanced numerical techniques, such as the Numerical Renormalization Group (NRG), to extract valuable information about spin transport in complex systems. This opens up new avenues for future research to investigate the behavior of impurities in systems with anisotropic $\mathcal{DM}$ interactions and in Weyl and topological systems with nonlinear dispersion.It is important to note that we did not explore charge or thermal transport in our study. These are interesting directions for future research, as they could provide a more complete understanding of the transport properties of impurities in topological systems.
Our work contributes to a growing body of research to better understand impurities' behavior in topological systems. This can inform the design of new materials and devices with enhanced transport properties and deepen our fundamental understanding of phenomena associated with open and closed condensed matter systems.


\begin{acknowledgments}
The author extends gratitude to the JNCASR for the supportive research environment provided. Acknowledgment is also given to Prof. N.S. Vidhyadhiraja for his encouragement in publishing the research. The author is thankful to Prof. Henrik Johannesson for his contribution to the conceptualization of the TKSS model and to Prof. Ipsita Mandal and  Dr. Ranjani Seshadri for introducing the concept of Weyl and topological systems.
\end{acknowledgments}


\onecolumngrid

\appendix
\section{Projection details for deriving effective sd model}
\begin{equation}
    \begin{split}
        \tilde{H}^{+}_{12}&=\sum_{k} \tilde{V}_{k}\alpha_{k1}e^{i\frac{\theta}{2}}c^\dag_{k+}d_{\uparrow}n_{\downarrow}
        +\sum_{k}e^{i\frac{\theta}{2}} \tilde{V}_{k}\alpha_{k2}c^\dag_{k+}d_{\downarrow}n_{\uparrow}\\
        \tilde{H}^{-}_{12}&=-\sum_{k} i\tilde{V}_{k}\alpha_{k2}e^{i\frac{\theta}{2}}c^\dag_{k-}d_{\uparrow}n_{\downarrow} 
        +\sum_{k}ie^{i\frac{\theta}{2}} \tilde{V}_{k}\alpha_{k1}c^\dag_{k-}d_{\downarrow}n_{\uparrow}
    \end{split}
\end{equation}
We will derive the components of hamiltonian as done in the text\cite{hews} as follows,
\begin{equation}
    \begin{split}
        H^+_{12}\frac{1}{E-H_{22}}H^{+}_{21}+H^+_{12}\frac{1}{E-H_{22}}H^{-}_{21}+H^-_{12}\frac{1}{E-H_{22}}H^{+}_{21}+H^-_{12}\frac{1}{E-H_{22}}H^{-}_{21} \\
        \bigg(\sum_{k} \tilde{V}_{k}\alpha_{k1}e^{i\frac{\theta}{2}}c^\dag_{k+}d_{\uparrow}n_{\downarrow}
        +\sum_{k}e^{i\frac{\theta}{2}} \tilde{V}_{k}\alpha_{k2}c^\dag_{k+}d_{\downarrow}n_{\uparrow}\bigg)^\dag\frac{1}{E-H_{22}}\bigg(\sum_{k'} \tilde{V}_{k'}\alpha_{k'1}e^{i\frac{\theta'}{2}}c^\dag_{k'+}d_{\uparrow}n_{\downarrow}
        +\sum_{k'}e^{i\frac{\theta'}{2}} \tilde{V}_{k'}\alpha_{k'2}c^\dag_{k'+}d_{\downarrow}n_{\uparrow}\bigg)\\
        \bigg(\sum_{k} \tilde{V}_{k}\alpha_{k1}e^{i\frac{\theta}{2}}c^\dag_{k+}d_{\uparrow}n_{\downarrow}
        +\sum_{k}e^{i\frac{\theta}{2}} \tilde{V}_{k}\alpha_{k2}c^\dag_{k+}d_{\downarrow}n_{\uparrow}\bigg)^\dag\frac{1}{E-H_{22}}\bigg(-\sum_{k'} i\tilde{V}_{k'}\alpha_{k'2}e^{i\frac{\theta'}{2}}c^\dag_{k'-}d_{\uparrow}n_{\downarrow} 
        +\sum_{k'}ie^{i\frac{\theta}{2}} \tilde{V}_{k'}\alpha_{k'1}c^\dag_{k'-}d_{\downarrow}n_{\uparrow}\bigg)\\
         \bigg(-\sum_{k} i\tilde{V}_{k}\alpha_{k2}e^{i\frac{\theta}{2}}c^\dag_{k-}d_{\uparrow}n_{\downarrow} 
        +\sum_{k}ie^{i\frac{\theta}{2}} \tilde{V}_{k}\alpha_{k1}c^\dag_{k-}d_{\downarrow}n_{\uparrow}\bigg)^\dag\frac{1}{E-H_{22}}\bigg(\sum_{k'} \tilde{V}_{k'}\alpha_{k'1}e^{i\frac{\theta'}{2}}c^\dag_{k'+}d_{\uparrow}n_{\downarrow}
        +\sum_{k'}e^{i\frac{\theta'}{2}} \tilde{V}_{k'}\alpha_{k'2}c^\dag_{k'+}d_{\downarrow}n_{\uparrow}\bigg)\\
        \bigg(-\sum_{k} i\tilde{V}_{k}\alpha_{k2}e^{i\frac{\theta}{2}}c^\dag_{k-}d_{\uparrow}n_{\downarrow} 
        +\sum_{k}ie^{i\frac{\theta}{2}} \tilde{V}_{k}\alpha_{k1}c^\dag_{k-}d_{\downarrow}n_{\uparrow}\bigg)^\dag\frac{1}{E-H_{22}}\bigg(-\sum_{k'} i\tilde{V}_{k'}\alpha_{k'2}e^{i\frac{\theta'}{2}}c^\dag_{k'-}d_{\uparrow}n_{\downarrow} 
        +\sum_{k'}ie^{i\frac{\theta'}{2}} \tilde{V}_{k'}\alpha_{k'1}c^\dag_{k'-}d_{\downarrow}n_{\uparrow}\bigg)
    \end{split}
\end{equation} 
We calculate for the $\theta=\theta'$ cases, and the remaining can be included in potential scattering later.
\begin{equation}
    \begin{split}
        &\sum_{k} M^{22}_{k+}\alpha^2_{k1}n_{\downarrow}d^\dag_{\uparrow}c_{k+}c^\dag_{k+}d_{\uparrow}n_{\downarrow}+\sum_{k}M^{22}_{k+}\alpha_{k2}\alpha_{k1}n_{\uparrow}d^\dag_{\downarrow}c_{k+}c^\dag_{k+}d_{\uparrow}n_{\downarrow}\\
        &+\sum_{k} M^{22}_{k+}\alpha_{k1}\alpha_{k2} n_{\downarrow}d^\dag_{\uparrow}c_{k+}c^\dag_{k+}d_{\uparrow}n_{\uparrow}             +\sum_{k}M^{22}_{k+}\alpha^2_{k2}n_{\uparrow}d^\dag_{\downarrow}c_{k+}c^\dag_{k+}d_{\downarrow}n_{\uparrow}\\
        &i\sum_{k} M^{22}_{k-}\alpha^2_{k1}n_{\downarrow}d^\dag_{\uparrow}c_{k+}c^\dag_{k-}d_{\downarrow}n_{\uparrow}-i\sum_{k}M^{22}_{k-}\alpha_{k2}\alpha_{k1}n_{\downarrow}d^\dag_{\uparrow}c_{k+}c^\dag_{k-}d_{\uparrow}n_{\downarrow}\\
        &+i\sum_{k} M^{22}_{k-}\alpha_{k1}\alpha_{k2} n_{\downarrow}d^\dag_{\uparrow}c_{k+}c^\dag_{k-}d_{\downarrow}n_{\uparrow}             -i\sum_{k}M^{22}_{k-}\alpha^2_{k2}n_{\uparrow}d^\dag_{\downarrow}c_{k+}c^\dag_{k-}d_{\uparrow}n_{\downarrow}\\
        &-i\sum_{k} M^{22}_{k+}\alpha^2_{k1}n_{\uparrow}d^\dag_{\downarrow}c_{k-}c^\dag_{k+}d_{\uparrow}n_{\downarrow} +i\sum_{k}M^{22}_{k+}\alpha_{k2}\alpha_{k1}n_{\downarrow}d^\dag_{\uparrow}c_{k-}c^\dag_{k+}d_{\uparrow}n_{\downarrow}\\
        &-i\sum_{k} M^{22}_{k+}\alpha_{k1}\alpha_{k2} n_{\uparrow}d^\dag_{\downarrow}c_{k-}c^\dag_{k+}d_{\downarrow}n_{\uparrow}             +i\sum_{k}M^{22}_{k+}\alpha^2_{k2}n_{\downarrow}d^\dag_{\uparrow}c_{k-}c^\dag_{k+}d_{\downarrow}n_{\uparrow}\\
        &\sum_{k} M^{22}_{k-}\alpha^2_{k1}n_{\uparrow}d^\dag_{\downarrow}c_{k-}c^\dag_{k-}d_{\downarrow}n_{\uparrow}-\sum_{k}M^{22}_{k-}\alpha_{k2}\alpha_{k1}n_{\uparrow}d^\dag_{\downarrow}c_{k-}c^\dag_{k-}d_{\uparrow}n_{\downarrow}\\
        &-\sum_{k} M^{22}_{k-}\alpha_{k1}\alpha_{k2} n_{\downarrow}d^\dag_{\uparrow}c_{k-}c^\dag_{k-}d_{\downarrow}n_{\uparrow}             +\sum_{k}M^{22}_{k-}\alpha^2_{k2}n_{\downarrow}d^\dag_{\uparrow}c_{k-}c^\dag_{k-}d_{\uparrow}n_{\downarrow} + (k \neq k' \text{terms})
    \end{split}
\end{equation} 
Where in above $M^{22}_{k\alpha}=\frac{\tilde{V}^2_{k}}{\epsilon_d+U-\epsilon_{k'\alpha}}\approx M^{22}_{k}(1-\alpha \Delta)$ with this substitution we can seperate the original kondo model and kondo model from the edge states impurity  interaction.Collecting terms with $\alpha^2_{k1},\alpha^2_{2},\alpha_{k1}\alpha_{k2}$ terms and with simplifications using the commutation and some substitution at half filling $n^2_{\sigma}=n_{\sigma}$, $n_{\uparrow}\to (\frac{1}{2}-n_{\downarrow})$,  $n_{\downarrow}\to (\frac{1}{2}-n_{\uparrow})$ and the dot operators $n_{\sigma}d_{\sigma}=d_{\sigma}(1-n_{\sigma})$ can be written as follows,
\begin{equation}
\label{eq:alpha1}
    \begin{split}
      (\alpha^2_{k1}) \longrightarrow  \sum_{k} M^{22}_{k-}\alpha^2_{k1}d^\dag_{\downarrow}c_{k-}c^\dag_{k-}d_{\downarrow}n_{\uparrow}-i\sum_{k} M^{22}_{k+}\alpha^2_{k1}n_{\uparrow}d^\dag_{\downarrow}c_{k-}c^\dag_{k+}d_{\uparrow}n_{\downarrow}\\
        i\sum_{k} M^{22}_{k-}\alpha^2_{k1}n_{\downarrow}d^\dag_{\uparrow}c_{k+}c^\dag_{k-}d_{\downarrow}n_{\uparrow}+\sum_{k} M^{22}_{k+}\alpha^2_{k1}d^\dag_{\uparrow}c_{k+}c^\dag_{k+}d_{\uparrow}n_{\downarrow} \\
        =\sum_{k}M^{22}_{k}\alpha^2_{k1}S^z_{d}s^z_{kk'}-i\sum_{k}M^{22}_{k}\alpha^2_{k1}S^{-}s^{+}_{kk'}+i\sum_{k}M^{22}_{k}\alpha^2_{k1}s^{-}_{kk'}S^{+} + (c^\dag_{k\alpha}c_{k\bar{\alpha}}n_{\uparrow}n_{\downarrow} \text{terms})\\
        i\sum_{k}M^{22}_{k}\Delta\alpha^2_{k1}S^{-}s^{+}_{kk'}-i\sum_{k}M^{22}_{k}\Delta\alpha^2_{k1}s^{-}_{kk'}S^{+} 
    \end{split}
\end{equation}
Note that The edge contribution is only appearing in the cross terms does not contribute to $S^z_ds^{z}_{kk'}$.Similarly for $\alpha^2_{k2}$ terms
\begin{equation}
\label{eq:alpha2}
    \begin{split}
        (\alpha^2_{k2})\longrightarrow \sum_{k}M^{22}_{k+}\alpha^2_{k2}n_{\uparrow}d^\dag_{\downarrow}c_{k+}c^\dag_{k+}d_{\downarrow}n_{\uparrow}-i\sum_{k}M^{22}_{k-}\alpha^2_{k2}n_{\uparrow}d^\dag_{\downarrow}c_{k+}c^\dag_{k-}d_{\uparrow}n_{\downarrow}\\
        +i\sum_{k}M^{22}_{k+}\alpha^2_{k2}n_{\downarrow}d^\dag_{\uparrow}c_{k-}c^\dag_{k+}d_{\downarrow}n_{\uparrow}+\sum_{k}M^{22}_{k-}\alpha^2_{k2}n_{\downarrow}d^\dag_{\uparrow}c_{k-}c^\dag_{k-}d_{\uparrow}n_{\downarrow}\\
        =\sum_{k}M^{22}_{k}\alpha^2_{k2}S^z_{d}s^z_{kk'}-i\sum_{k}M^{22}_{k}\alpha^2_{k2}S^{-}s^{+}_{kk'}+i\sum_{k}M^{22}_{k}\alpha^2_{k2}s^{-}_{kk'}S^{+} + (c^\dag_{k\alpha}c_{k\bar{\alpha}}n_{\uparrow}n_{\downarrow} \text{terms})\\
        i\sum_{k}M^{22}_{k}\Delta\alpha^2_{k2}S^{-}s^{+}_{kk'}-i\sum_{k}M^{22}_{k}\Delta\alpha^2_{k2}s^{-}_{kk'}S^{+} 
    \end{split}
\end{equation}
Finally we can collect all terms with $\alpha_1\alpha_2$
\begin{equation}
\label{eq:alpha12}
    \begin{split}
     (\alpha_{k1}\alpha_{k2})\longrightarrow   \sum_{k}M^{22}_{k+}\alpha_{k2}\alpha_{k1}n_{\uparrow}d^\dag_{\downarrow}c_{k+}c^\dag_{k+}d_{\uparrow}n_{\downarrow}+\sum_{k} M^{22}_{k+}\alpha_{k1}\alpha_{k2} n_{\downarrow}d^\dag_{\uparrow}c_{k+}c^\dag_{k+}d_{\downarrow}n_{\uparrow}\\
        -i\sum_{k}M^{22}_{k-}\alpha_{k2}\alpha_{k1}n_{\downarrow}d^\dag_{\uparrow}c_{k+}c^\dag_{k-}d_{\uparrow}n_{\downarrow}
        +i\sum_{k} M^{22}_{k-}\alpha_{k1}\alpha_{k2} n_{\uparrow}d^\dag_{\downarrow}c_{k+}c^\dag_{k-}d_{\downarrow}n_{\uparrow}\\+i\sum_{k}M^{22}_{k+}\alpha_{k2}\alpha_{k1}n_{\downarrow}d^\dag_{\uparrow}c_{k-}c^\dag_{k+}d_{\uparrow}n_{\downarrow}
        -i\sum_{k}M^{22}_{k+}\alpha_{k2}\alpha_{k1}n_{\uparrow}d^\dag_{\downarrow}c_{k-}c^\dag_{k+}d_{\downarrow}n_{\uparrow}\\
        -\sum_{k} M^{22}_{k-}\alpha_{k1}\alpha_{k2} n_{\downarrow}d^\dag_{\uparrow}c_{k-}c^\dag_{k-}d_{\downarrow}n_{\uparrow}-\sum_{k} M^{22}_{k-}\alpha_{k1}\alpha_{k2} n_{\uparrow}d^\dag_{\downarrow}c_{k-}c^\dag_{k-}d_{\uparrow}n_{\downarrow}
\end{split}
\end{equation}
 We simplify the above and use the Abrikosov representation\cite{Abri} for spin for impurity as $S=\psi^\dag_{d}\sigma\psi_{d}$ and  $s_{kk'}=\psi^\dag_{k}\sigma\psi_{k'}$ for bath operators. Where in the representation $\sigma$ is pauli matrix  and $\psi^\dag_{d}=\begin{pmatrix}d^\dag_{\uparrow}&d^\dag_{\downarrow}
 \end{pmatrix}$     $\psi^\dag_{k}=\begin{pmatrix}c^\dag_{k\uparrow}&c^\dag_{k'\downarrow}
 \end{pmatrix}$
 \begin{equation}
 \label{eq:alpha12simp}
    \begin{split}
     i\sum_{k}M^{22}_{k}\alpha_{k2}\alpha_{k1}(-s^-_{kk'}S^z +S^zs^+_{kk'} )+\sum_{k}M^{22}_{k}\alpha_{k2}\alpha_{k1}(s^z_{kk'}S^+ -S^-s^z_{kk'} )\\
      +i\sum_{k}M^{22}_{k}\Delta\alpha_{k2}\alpha_{k1}(s^-_{kk'}S^z -S^zs^+_{kk'} )-\sum_{k}M^{22}_{k}\Delta\alpha_{k2}\alpha_{k1}(s^z_{kk'}S^+ -S^-s^z_{kk'} )
            \end{split}
        \end{equation}
Similarly we can calculate $ H^+_{10}\frac{1}{E-H_{00}}H^{+}_{01}+H^+_{10}\frac{1}{E-H_{00}}H^{-}_{01}+H^-_{10}\frac{1}{E-H_{00}}H^{+}_{01}+H^-_{10}\frac{1}{E-H_{00}}H^{-}_{01}$ which yield same operator structure except we have $M^{00}_{k}$   
Now substituting back $\alpha_1= \sqrt{\frac{\Delta }{2}+\beta k^3 \cos3\theta}$ and 
$\alpha_2=\sqrt{\beta k^3 \cos3\theta-\frac{\Delta }{2}}$, where $\Delta=\sqrt{4\beta^2k^6 \cos^2 3\theta +4k^2}$ and $\tilde{V}_k=\frac{V_k}{\sqrt{k^3\cos{3\theta}}}$, collecting terms  we get various simplifications for example $\alpha^2_{k1}+\alpha^2_{k2}=\beta k^3 \cos(3\theta)$  and $\alpha_{k1}\alpha_{k2}=k$
\begin{equation}
    \begin{split}
       H^1_{eff}=H_{11}+ \beta k^3\cos{3\theta}\bigg(\sum_{k}M_{k}S^z_{d}s^z_{kk'}-i\sum_{k}M_{k}S^{-}s^{+}_{kk'}+i\sum_{k}M_{k}s^{-}_{kk'}S^{+}\\
        i\sum_{k}M_{k}\Delta S^{-}s^{+}_{kk'}-i\sum_{k}M_{k}\Delta s^{-}_{kk'}S^{+} \bigg)\\
         k\bigg(i\sum_{k}M_{k}(-s^-_{kk'}S^z +S^zs^+_{kk'} )+\sum_{k}M_{k}(s^z_{kk'}S^+ -S^-s^z_{kk'} )\\
      +i\sum_{k}M_{k}\Delta(s^-_{kk'}S^z -S^zs^+_{kk'} )-\sum_{k}M_{k}\Delta(s^z_{kk'}S^+ -S^-s^z_{kk'} )\bigg)
    \end{split}
\end{equation}
Collecting All terms from equations \ref{eq:alpha1},\ref{eq:alpha2}and \ref{eq:alpha12simp} we can simplify and rewrite the derived effective model as,
\begin{equation}
    \begin{split}
         H^1_{eff}&=H_{11}+\sum_{kk'}J_{0}S.s_{kk'}+i\sum_{kk'}\Delta (S\times s_{kk'})_z+ \sum_{kk'}i\frac{k}{\beta k^3 \cos3\theta}M_{kk'}.(S\times s_{kk'})_y + \sum_{kk'}i\frac{k}{\beta k^3 \cos3\theta}\Delta M_{kk'}.(S\times s_{kk'})_x\\
        \therefore H^1_{eff}&=H_{11}+\sum_{kk'}J_{0}S.s_{kk'}+i\sum_{kk'} \vec{J}_{k^3}.(S\times s_{kk'})+i\sum_{kk'} \vec{J}_{k}.(S\times s_{kk'})
    \end{split}
\end{equation}
\section{Including the potential scattering}
The effective model indicates that the matrix elements are scaled as $M^{\theta,\theta'}_{kk'}\propto \frac{V{k}}{\sqrt{k^3 \cos3\theta}}\frac{V_{k'}}{\sqrt{k'^3 \cos3\theta'}}$.If we go beyond poorman's limit $k=k'=k_f$ then this leads to four possible potential scattering scenarios: $\vec{J}^{\theta_{\pm\frac{2\pi}{3}},\theta_{\pm\frac{\pi}{3}}}_{k^3,k}$, $\vec{J}^{\theta{\frac{2\pi}{3}},\theta_{-\frac{\pi}{3}}}_{k^3,k}$, and the original couplings denoted as $\vec{J}^{0,0}_{k^3,k}$, $\vec{J}^{0,0}_{k^3,k}$. It's worth noting that the vector corresponding to non-linear dispersion is only 1-dimensional and contains a z-component. In contrast, the vector corresponding to linear dispersion is 2-dimensional and contains x and y components. This introduces an anisotropic $\mathcal{DM}$-interaction.
\begin{equation}
    \begin{split}
         H^{1}_{eff}&=H_0+\sum_{kk'}J_0 S.\psi^\dag(\Sigma)\psi
         +i\sum_{kk'}\vec{J}^{(0,0)}_{k^3}.(S\times \psi^\dag(\Sigma)\psi)
         -\sum_{kk'} \vec{J}^{\theta_{\pm\frac{2\pi}{3}},\theta_{\pm\frac{\pi}{3}}}_{k^3}.(S\times \psi^\dag(\Omega)\psi)\\
         &+\sum_{kk'}\vec{J}^{\theta_{\frac{2\pi}{3}},\theta_{-\frac{\pi}{3}}}_{k^3}.(S\times \psi^\dag(\Gamma)\psi)
         +i\sum_{kk'}\vec{J}^{(0,0)}_{k}.(S\times \psi^\dag(\Sigma)\psi)
         -\sum_{kk'}\vec{J}^{\theta_{\pm\frac{2\pi}{3}},\theta_{\pm\frac{\pi}{3}}}_{k}.(S\times \psi^\dag(\Omega)\psi)\\
         &+i\sum_{kk'}\vec{J}^{\theta_{\frac{2\pi}{3}},\theta_{-\frac{\pi}{3}}}_{k}.(S\times \psi^\dag(\Gamma)\psi)
    \end{split}
\end{equation}
For notational simplicity we will rewite the coupling as following,
\begin{equation}
    \begin{split}
         H^{1}_{eff}&=H_0+\sum_{kk'}J_0 S.\psi^\dag(\Sigma)\psi
         +i\sum_{kk'}\vec{J}_{k^3}.(S\times \psi^\dag(\Sigma)\psi)
         -\sum_{kk'} \vec{g}_{1k^3}.(S\times \psi^\dag(\Omega)\psi)\\
         &+\sum_{kk'}\vec{g}_{2k^3}.(S\times \psi^\dag(\Gamma)\psi)
         +i\sum_{kk'}\vec{J}_{k}.(S\times \psi^\dag(\Sigma)\psi)
         -\sum_{kk'}\vec{g}_{1k}.(S\times \psi^\dag(\Omega)\psi)\\
         &+i\sum_{kk'}\vec{g}_{2k}.(S\times \psi^\dag(\Gamma)\psi)\\
         &=H_0+\sum_{kk'}J_0 \bigg(S_x \Sigma_x+S_y \Sigma_y+S_z\Sigma_z \bigg)
         +i\sum_{kk'}|\vec{J}_{k^3}|(S_x\Sigma_y-\Sigma_xS_y)
         -\sum_{kk'} |\vec{g}_{1k^3}|(S_x\Omega_y-\Omega_xS_y)\\
         &+\sum_{kk'}|\vec{g}_{2k^3}|(S_x\Gamma_y-\Gamma_xS_y)
         +i\sum_{kk'}|\vec{J}_{k}|\bigg((S_y\Sigma_z-\Sigma_yS_z) +(S_x\Sigma_z-\Sigma_xS_z)\bigg)\\
         &-\sum_{kk'}|\vec{g}_{1k}|\bigg((S_y\Omega_z-\Omega_yS_z) +(S_x\Omega_z-\Omega_xS_z)\bigg)\\
         &+i\sum_{kk'}|\vec{g}_{2k}|\bigg((S_y\Gamma_z-\Gamma_yS_z) +(S_x\Gamma_z-\Gamma_xS_z)\bigg)
    \end{split}
\end{equation}
We use the diagrams\cite{nfl,affleck1993exact,anderson1970poor,solyom1974renormalization,solyom1974scaling} with all permutations of vertices using the following algebra,
\begin{equation}
    \begin{split}
        [\Sigma_{a}\Sigma_{b},\Sigma_c]=\Sigma_{a}[\Sigma_{b},\Sigma_c]+[\Sigma_{a},\Sigma_c]\Sigma_b=0\\
        [\Sigma_{a}\Sigma_{b},\Sigma_a]=\Sigma_{a}[\Sigma_{b},\Sigma_c]+[\Sigma_{a},\Sigma_a]\Sigma_b=8i\Sigma_b\\
        [\Gamma_{a}\Gamma_{b},\Gamma_a]=\Gamma_{a}[\Gamma_{b},\Gamma_c]+[\Gamma_{a},\Gamma_a]\Gamma_b=8i\Gamma_b\\
        [\Omega_{a}\Omega_{b},\Omega_a]=\Omega_{a}[\Omega_{b},\Omega_c]+[\Omega_{a},\Omega_a]\Omega_b=8i\Omega_b\\
        [\Omega_{a}\Sigma_{b},\Sigma_a]=\Omega_{a}[\Sigma_{b},\Sigma_c]+[\Omega_{a},\Sigma_a]\Sigma_b=8i\Sigma_b\\
        [\Sigma_{a}\Omega_{b},\Gamma_a]=\Sigma_{a}[\Omega_{b},\Gamma_c]+[\Sigma_{a},\Gamma_a]\Omega_b=8i\Upsilon
    \end{split}
\end{equation}
With the above algebra, we find the added potential scattering renormalizes only at the second loop, and we derive these equations as follows,
\begin{equation}
    \begin{split}
        \frac{dJ_{0}}{dl}&=J^2_0+J^2_{k^3}+J^2_{k}+J_{k^3}J_{k}+J^2_{k}J_0+J^2_{k^3}J_0+J^3_0-J_{k^3}g_{2k}-J_{k^3}g_{2k^3}-J_{k}g_{2k}+g^2_{2k}J_{k^3}+g^2_{2k^3}J_{k}-J^2_{k}g_{2k^3}\\
        \frac{dJ_{k^3}}{dl}&=J^2_{k}+J_{0}J_{k^3}+J^2_{k}J_{k^3}+J^2_0J_{k^3}+J^3_{k^3}\\
        \frac{dJ_{k}}{dl}&=J_0J_k+J^2_0J_k+J_kJ^2_{k^3}+J^3_{k}\\
        \frac{dg_{1k^3}}{dl}&=g^3_{1k^3}-g^2_{1k}g_{1k^3}\\
        \frac{dg_{1k}}{dl}&=g^3_{1k}-g^2_{1k^2}g_{1k}\\
        \frac{dg_{2k^3}}{dl}&=-g^3_{2k^3}+g^2_{1k^3}g_{2k^3}\\
        \frac{dg_{2k}}{dl}&=-g^3_{2k}+g^2_{1k}g_{2k}
    \end{split}
\end{equation}
From above, we identify various RG invariants as $\frac{g_{1k^3}}{g_{1k}}=m_1$, and we can notice that after adding the potential scattering terms, we still have the invariant $J^2_{k}+J_k=mJ_{k^3}$ and $\frac{g^2_{2k}(g_{2k}+\sqrt{m_2})}{(g_{2k}-\sqrt{m_2})}= \frac{g^2_{2k^3}(g_{2k^3}+\sqrt{m_2/m_1})}{(g_{2k^3}-\sqrt{m_2/m_1})}
$
\section{Rg equations For Two impurity}
To study the renormalization of RKKY and Kondo couplings, we show here to introduce a 4-operator vertex in a dot, specifically $J_{Y}:d^\dag_{\alpha\sigma}d^\dag_{\alpha'\sigma'}d_{\alpha\sigma}d_{\alpha'\sigma'}:$, among all the possible vertices in two dots. By using spin algebra, we can write $[S^+{\alpha},S^-{\alpha'}]=\delta_{\alpha,\alpha'}S^z_{\alpha}-d^\dag_{\alpha\uparrow}d^\dag_{\alpha\downarrow}d_{\alpha\downarrow}d_{\alpha'\uparrow}+d^\dag_{\alpha\uparrow}d^\dag_{\alpha\downarrow}d_{\alpha\downarrow}d_{\alpha'\uparrow}$ in normal ordering, where fermionic algebra is used except for $S^z$, and all terms except for $S^z$ vanish. Summing all one flip vertices, we get $[S^z_{\alpha},S^{\pm}{\alpha'}]=2S^{\pm}\delta{\alpha\alpha'}-d^\dag_{\alpha\uparrow}d^\dag_{\alpha\sigma}d_{\alpha\downarrow}d_{\alpha'\sigma}+d^\dag_{\alpha\uparrow}d^\dag_{\alpha\sigma}d_{\alpha\downarrow}d_{\alpha'\sigma}$. We have written the full operator terms to demonstrate that they obey spin algebra, and the commutator yields one spin vertex with no scattering terms in the dot. However, this is not the case in the bath.

\begin{figure}[h!] 
    %\centering
\includegraphics[scale=0.5]{Poorman.eps}
\caption{To study the renormalization of RKKY and usual Kondo couplings, we can extend the Poorman diagrams to include RKKY vertices represented by crossed circles and the usual J vertices with black circles. Three spin-flip processes are absent in the single impurity case at the second-order process. To distinguish between different types of interactions, we can use a wavy line to denote RKKY interactions and a dotted line to denote impurity-$mathcal {DM}$ interactions.}
\label{fig:PMscaling}
\end{figure}
The above discussion highlights the difficulty in computing all diagrams at third order in the presence of both RKKY and $\mathcal{DM}$ interactions. With ${}^{7}C_{2}$ diagrams possible at second order and ${}^{7}C_{2}+ {}^{7}C_{3}$ diagrams possible at third order, the calculations become extremely tedious. However, one can use algebraic techniques to simplify the computation in scenarios with potential scattering. Lie matrices can be used to compute the contributions in such cases. 
\begin{figure}[h!] 
    %\centering
\includegraphics[scale=.8]{TopoKondoRG.eps}
\caption{The above panels in the first row correspond to the positive root  and $m=-4$  and in the real, imaginary plane, similarly below  for the negative root and $m=-4$, showing the spiral fixed point in RG flow.}
\label{fig:fixed_point1}
\end{figure}
\section{RKKY Interaction}
We can write the RKKY interaction in edge states by expanding in the Fourier series for the spins,
\begin{equation}
    \begin{split}
        J_{Y}(r_i -r_j)&\propto\sum_{kk'hh'}\frac{V_{ij}}{\epsilon_{kh}-\epsilon_{k'h'}}e^{ik(r_i-r_j)}V^*_{ij}e^{-ik'(r_i-r_j)}\\
        & =\int^{2\pi}_{0}\int^{\pi}_{0}\int^{k_f}_{0}\frac{k^2sin\theta e^{ikR} dk d\theta d\phi}{\sqrt{k^6\beta^2 \cos^2 3\theta+k^2}}+(k \neq k',h=h') \hspace{2mm} \text{terms}
    \end{split}
\end{equation}
We solve the above integral the same way as we evaluate in impurity transport sections and show that the leading contribution for $J_Y$ is the sum of special functions as $E_1(\epsilon)+E_2(\epsilon)$ with substitution $\cos \theta=t$ we get following also we don't set $\epsilon_k=\epsilon_{k'}$ as in the Poorman limit instead we consider a chemical potential such that integral does not diverge in this limit,
\begin{equation}
    \begin{split}
       J_Y &\propto\pi\int_k \frac{e^{ikR}}{k\beta}\oint\frac{qdt}{q^2-(4t^3-3t)^2} +\sum_{kk'}\frac{|V_{ij}|}{k^2-(k')^2}e^{ikR}e^{-ik'R} \\
       &\text{where} \hspace{2mm} q=\frac{\sqrt{(k^2-\epsilon)^2-k^2}}{\beta k^3}\\
       \therefore J_Y &\propto \int e^{i\sqrt{\epsilon}R}\frac{\sqrt{(\epsilon-\epsilon')^2-\epsilon}}{\beta \epsilon^{\frac{3}{2}}}\sqrt{\epsilon}d\epsilon+\frac{\sin(\sqrt{\epsilon}R)}{\epsilon^{\frac{3}{2}}}+\frac{\sin(\sqrt{\epsilon}R)}{\sqrt{\epsilon}}..
    \end{split}
\end{equation}
We can immediately see the sum of the roots will vanish for such cubic equations since there is no $t^2$ coefficient, but the quantity in the root is not the same for all residues; hence it does contribute.
\section{Analytic Solution of RG equations}
A natural simplification of $J_{k^3}$ and $J_{k}$ RG equation by eliminating $J_0$ gives the following,
\begin{equation}
    \begin{split}
        \frac{dJ_{k^3}}{dl}-J^2_k=\frac{J_{k^3}}{J_k}\frac{dJ_k}{dl}\\
        \frac{1}{J_{k^3}}\frac{dJ_{k^3}}{dl}-\frac{J^2_k}{J_{k^3}}=\frac{1}{J_k}\frac{dJ_k}{dl}
    \end{split}
\end{equation}
Now with a substitution of $\frac{J^2_k}{J_{k^3}}=x$ which gives upon differentiation $\frac{dx}{dl}=2\frac{J_k}{J_{k^3}}\frac{dJ_k}{dl}-\frac{J^2_k}{J^2_{k^3}}\frac{dJ_{k^3}}{dl}$ using this and rearranging we get solution as $J_k =-\frac{1}{2}\pm \frac{1}{2}\sqrt{1+4mJ_{k^3}}$ this is related to second solution by phase factors 
Now we can derive the full solution as the following,
\begin{equation}
    \begin{split}
        \Delta\frac{dJ_0}{dl}-J^3_0=n, \hspace{2mm}
        (\frac{J^2_k+J_k}{m}+J_k \frac{(J_k+1)^2}{m^2}+J_k\bigg)\frac{dJ_k}{dl} =n
    \end{split}
\end{equation}
This yield solution is as follows; One can show the m has complex roots when the effective bandwidth vanishes. $J^*_{k}\to \frac{\sqrt{3}e^{i\frac{\pi}{3}}}{2}$,$J^*_{k^3}\to \frac{9e^{i\frac{\pi}{3}}}{16}$ and $m=e^{i\frac{\pi}{3}}$ where the $J^*_0$ will have $\tan^{-1}(..)$ quantity vanishing for $J^*_0=\pm1$ in ferromagnetic and antiferromagnetic cases for $n \to 1$ see equations in \ref{eq:Soln1}, also log contribution to scale vanish when $J^*_0=-2$.We get these fixed points when we add the potential scattering terms.
\begin{equation}
\label{eq:Soln1}
    \begin{split}
        \log\bigg(\frac{[m^2+m(J_k-1)+(J_k-1)^2]^{3\gamma}}{J^{2\gamma}_k}\bigg) +\frac{(m-2)}{\sqrt{3}(m^2-m+1)} \tan ^{-1}\left(\frac{m+2 J_k-2}{\sqrt{3} m}\right)=n\log D_{eff}\\
        \frac{\log \left(n^{2/3}+\sqrt[3]{n} (-J_0)+J_0^2\right)-2 \log \left(\sqrt[3]{n}+J_0\right)-2 \sqrt{3} \tan ^{-1}\left(\frac{1-\frac{2 J_0}{\sqrt[3]{n}}}{\sqrt{3}}\right)}{6 \sqrt[3]{n}}=\log D_{eff}\\
       \text{where}\hspace{2mm} \gamma=\frac{-m}{6*(m^2-m+1)} \implies \gamma \to -\infty* \hspace{2mm} \text{for} \hspace{2mm} m=e^{\pm i\frac{ \pi}{3}} 
    \end{split}
\end{equation}
Due to the bath's exceptional dispersion, there may be an unusual impurity screening. The kondo destruction will be at the critical points found above, leading to some complex fixed points. 
The similarity between the RG of a single impurity in edge states and two impurities in conventional Fermi gas.
For this reason, we solve two-impurity problem RG equations by limiting  all couplings to zero except $J_0 \neq J_Y \neq K\neq 0$.We need to solve the equations $\frac{dJ_0}{dl}=J^2_0+J_YJ_0+KJ_0,\frac{dJ_Y}{dl}=J^2_Y+J^2_0+K^2,\frac{dK}{dl}=K^2+KJ_Y$. If we solve $J_0$ and K equations, we get $J_0=\frac{R K}{K-1}$ as a solution where R is a constant or invariant  under renormalization. Using this solution, we solve $J_Y$ and K equations as follows,
\begin{equation}
    \begin{split}
        J_Y \frac{d J_Y}{dl}-J^3_Y=R_1 , \frac{dK}{dl}-K^2= \frac{R_1}{\frac{R^2K}{(1-K)^2}+K}\\
        \frac{-2 \log \left(J_Y+\sqrt[3]{R_1}\right)+\log \left(\sqrt[3]{R_1} \left(-J_Y\right)+J_Y^2+R_1^{2/3}\right)+2 \sqrt{3} \tan ^{-1}\left(\frac{\frac{2 J_Y}{\sqrt[3]{R_1}}-1}{\sqrt{3}}\right)}{6 \sqrt[3]{R_1}}=\log D_{eff}\\
         \frac{-2 \log \left(K+\sqrt[3]{R_1}\right)+\log \left(\sqrt[3]{R_1} \left(-K\right)+K^2+R_1^{2/3}\right)+2 \sqrt{3} \tan ^{-1}\left(\frac{\frac{2 K}{\sqrt[3]{R_1}}-1}{\sqrt{3}}\right)}{6 \sqrt[3]{R_1}}=\log D_{eff}, \hspace{2mm} \text{for} \hspace{2mm} K \to \infty\\
         \frac{-\frac{R^2}{K-1}+R^2 \log (K-1)+\frac{1}{2} (K-1)^2+K}{R_1}=\log D_{eff} , \hspace{2mm} \text{for} \hspace{2mm} K \to 0\\
        (K-1)^{\frac{R}{R_1}} \exp{\frac{1}{R_1}\bigg(-\frac{R^2}{K-1}+\frac{1}{2} (K-1)^2+K\bigg)}=D_{eff}\\
        \implies (KR/J_0)^{\frac{R}{R_1}} e^{\frac{1}{R_1}\bigg(-\frac{R^2}{KR/J_0}+\frac{1}{2} (KR/J_0)^2+K\bigg)}=D_{eff}\\
        \frac{\sqrt[3]{R_1} \left(-J_Y\right)+J_Y^2+R_1^{2/3}}{(J_Y+\sqrt[3]{R_1})^2}e^{2 \sqrt{3} \tan ^{-1}\left(\frac{\frac{2 J_Y}{\sqrt[3]{R_1}}-1}{\sqrt{3}}\right)}=D^{6R^{\frac{1}{3}}_1}_{eff}   
\end{split}
\end{equation}
Solving for $D_{eff}$ we can see how the two impurity kondo scale renormalized in this problem. 
Similarly, We will solve the $J_0$ equation with $J_Y$ to separate the solutions and for $R=R_1=1.0$, we show the flow and critical points. We expect fixed points in different quadrants depending on the sign we choose for these constants.
\begin{figure}[h!] 
    %\centering
\includegraphics[scale=.8]{twoimp_r1_-1_r_1.eps}
\caption{Plots above show the fixed points can be in different quadrants with $R_1 =-1$ and $R=1.0$. We get around the spiral fixed points we get sign reversion in couplings.}
\label{fig:fixed_point1}
\end{figure}

\section{Impurity Transport Calculation}
We follow the $\mathcal{T}_{kk'}$ formalism (without expanding in terms of $\Delta$ as we did for separating RG flow and also not including the potential scatterings) for the derived kondo model after the k-dependent unitary to compute the relaxation time as follows,
\begin{equation}
    \begin{split}
        \frac{1}{\tau(k)}\propto (1-2J \tilde{g}_{\alpha}(\epsilon_k)-2J\tilde{g}_{\alpha}^* (\epsilon_k)-2J_{k^3} \tilde{g}_{\alpha k^3}(\epsilon_k)-2J_{k^3}\tilde{g}_{\alpha k^3}^* (\epsilon_k)-2J_k \tilde{g}_{\alpha k}(\epsilon_k)-2J_k \tilde{g}_{\alpha k}^* (\epsilon_k))
    \end{split}
\end{equation}
\begin{equation}
    \begin{split}
        \tilde{g}_{\alpha}(\epsilon)=\int^{2\pi}_{0}\int^{\pi}_{0}\int^{k_f}_{0}\frac{k^2sin\theta dk d\theta d\phi}{k^2+\alpha\sqrt{k^6\beta^2 \cos^2 3\theta+k^2}-\epsilon}
    \end{split}
\end{equation}
Solving the $\theta$ integral first, we get two pieces as follows,
\begin{equation}
    \begin{split}
        \tilde{g}_{\alpha}(\epsilon)=2\pi\int_k k^2\oint\frac{dt}{k^2+\alpha\sqrt{k^6\beta^2 (4t^3-3t)^2+k^2}-\epsilon} 
    \end{split}
\end{equation}
we rationalize the above integral and write as the following by introducing $q=\frac{\sqrt{(k^2-\epsilon)^2-k^2}}{\beta k^3}$,
\begin{equation}
    \begin{split}
        \tilde{g}_{\alpha}(\epsilon)&=2\pi\int_k \frac{1}{k}\oint\frac{q+ \bar{\alpha}\sqrt{(4t^3-3t)^2+\frac{1}{ \beta^2 k^4}}dt}{q^2-(4t^3-3t)^2}\\
        &=\pi\int_k \frac{1}{kq}\oint\frac{q+ \bar{\alpha}\sqrt{ (4t^3-3t)^2+\frac{1}{\beta^2 k^4}}dt}{q- (4t^3-3t)}+\pi\int_k \frac{1}{kq}\oint\frac{q+ \bar{\alpha}\sqrt{ (4t^3-3t)^2+\frac{1}{\beta^2 k^4}}dt}{q+ (4t^3-3t)}\\
        &=\int_k \frac{\pi}{kq}\bigg( \sum_{res} f(t,q)+\sum_{res} f(t,-q)\bigg)
    \end{split}
\end{equation}
For finding these residues, we will use the roots of the cubic equations $\beta (4t^3-3t)\pm q=0$. We collect positive q roots and negative as follows for doing contour integrals,
\begin{equation}
    \begin{split}
    t^+=
\begin{cases}
\frac{1}{2} \left(\sqrt[3]{\sqrt{q^2-1}+q}+\frac{1}{\sqrt[3]{\sqrt{q^2-1}+q}}\right),\\
-\frac{1}{4} \left(1-i \sqrt{3}\right) \sqrt[3]{\sqrt{q^2-1}+q}-\frac{1+i \sqrt{3}}{4 \sqrt[3]{\sqrt{q^2-1}+q}}\\
-\frac{1}{4} \left(1+i \sqrt{3}\right) \sqrt[3]{\sqrt{q^2-1}+q}-\frac{1-i \sqrt{3}}{4 \sqrt[3]{\sqrt{q^2-1}+q}}
\end{cases}
    t^-=
\begin{cases}
\frac{1}{2} \left(\sqrt[3]{\sqrt{q^2-1}-q}+\frac{1}{\sqrt[3]{\sqrt{q^2-1}-q}}\right),\\
-\frac{1}{4} \left(1-i \sqrt{3}\right) \sqrt[3]{\sqrt{q^2-1}-q}-\frac{1+i \sqrt{3}}{4 \sqrt[3]{\sqrt{q^2-1}-q}}\\
-\frac{1}{4} \left(1+i \sqrt{3}\right) \sqrt[3]{\sqrt{q^2-1}-q}-\frac{1-i \sqrt{3}}{4 \sqrt[3]{\sqrt{q^2-1}-q}}
\end{cases}
    \end{split}
\end{equation}
Similarly, the other contributions are as follows,
\begin{equation}
    \begin{split}
     \tilde{g}_{\alpha k^3}(\epsilon)&=2\pi\beta\int_k k^2\oint\frac{((4t^3-3t)^2)(q+ \bar{\alpha}\sqrt{(4t^3-3t)^2+\frac{1}{ \beta^2 k^4}})}{q^2-(4t^3-3t)^2}dt\\
     \tilde{g}_{\alpha k}(\epsilon)&=\int_k \oint\frac{(q+ \bar{\alpha}\sqrt{(4t^3-3t)^2+\frac{1}{ \beta^2 k^4}})}{q^2-(4t^3-3t)^2}dt
    \end{split}
\end{equation}
after summing over the $\alpha=\pm$ bands we get following simplifications,
\begin{equation}
    \begin{split}
    \tilde{g}_{\alpha}(\epsilon)&=2\pi\int_k \frac{1}{k}\oint\frac{qdt}{q^2-(4t^3-3t)^2}, \tilde{g}_{\alpha k^3}(\epsilon)=2\pi\beta\int_k k^2\oint\frac{(4t^3-3t)^2qdt}{q^2-(4t^3-3t)^2},
     \tilde{g}_{\alpha k}(\epsilon)=\int_k \oint\frac{qdt}{q^2-(4t^3-3t)^2}
    \end{split}
\end{equation}
After contour integrals we can show each integral contributes as $\oint\frac{qdt}{q^2-(4t^3-3t)^2}=q$ hence sum of all $(g_{\alpha})$ contribution as following,
\begin{equation}
    \begin{split}
        \tilde{g}_{\epsilon}=\int (\beta qk^2 -\beta q^4 k +\frac{q}{k}-q )dk
    \end{split}
\end{equation}
We know from derivation $\epsilon \approx k^2$ and the density of states in 3D  as $\rho(\epsilon)\approx \sqrt{\epsilon}$ then above integral yields,
\begin{equation}
    \begin{split}
        \tilde{g}_{\epsilon'}=\int \bigg( \frac{\sqrt{(\epsilon-\epsilon')^2-\epsilon}}{\sqrt{\epsilon}} -\frac{((\epsilon-\epsilon')^2-\epsilon)^2}{\beta^3 \epsilon^{6-\frac{1}{2}}}  +\frac{\sqrt{(\epsilon-\epsilon')^2-\epsilon}}{\beta \epsilon^{2}}-\frac{\sqrt{(\epsilon-\epsilon')^2-\epsilon}}{\beta \epsilon^{\frac{3}{2}}} \bigg)d\epsilon
    \end{split}
\end{equation}
Where in above $\epsilon'$ is the chemical potential can take any values around fermi energy. We perform this above integral exactly in terms of special functions  to extract contribution to $\frac{1}{\tau}$, which indeed scales with RG invariant in the elliptic functions. 

\begin{equation}
    \begin{split}
        &\tilde{g}_{\epsilon} \propto \frac{P_1(\epsilon)+P_{\frac{3}{2}}(\epsilon)+T_1(\epsilon)+E_1(\epsilon)+E_2(\epsilon)+\log (\epsilon)}{\beta ^3}\\
        &P_{1}(\epsilon)=-\frac{3 \epsilon'^4-16 \epsilon'^3 \epsilon+4 \epsilon'^2 \epsilon (9 \epsilon-2)+2 \epsilon^2 \left(2 \epsilon \left(\beta ^2 \sqrt{(\epsilon'-\epsilon)^2-\epsilon} \left(2 \sqrt{\epsilon} (\beta -\beta \epsilon+3)+3\right)-6\right)+3\right)}{12 \epsilon^4}\\
        &P_{\frac{3}{2}}(\epsilon)=\frac{8 \epsilon' \epsilon^2 \left(2 \beta ^3 \epsilon^{3/2} \sqrt{(\epsilon'-\epsilon)^2-\epsilon}-6 \epsilon+3\right)}{12 \epsilon^4}\\
       & T_1(\epsilon)=\beta ^2 \tanh ^{-1}\left(\frac{-2 \epsilon'+2 \epsilon-1}{2 \sqrt{(\epsilon'-\epsilon)^2-\epsilon}}\right)+\frac{(2 \epsilon'+1) \beta ^2 \coth ^{-1}\left(\frac{2 \epsilon' \sqrt{(\epsilon'-\epsilon)^2-\epsilon}}{2 \epsilon' (\epsilon'-\epsilon)-\epsilon}\right)}{2 \epsilon'}\\
       & E_1(\epsilon)=\frac{i \beta ^2 \epsilon \sqrt{\frac{-2 \epsilon'+\sqrt{4 \epsilon'+1}+2 \epsilon-1}{\epsilon}} \sqrt{-\frac{2 \epsilon'+\sqrt{4 \epsilon'+1}-2 \epsilon+1}{\epsilon}} \left(A E\left(i \sinh ^{-1}\left(\frac{\sqrt{-2 \epsilon'-\sqrt{4 \epsilon'+1}-1}}{\sqrt{2} \sqrt{\epsilon}}\right)|\frac{2 \epsilon'-\sqrt{4 \epsilon'+1}+1}{2 \epsilon'+\sqrt{4 \epsilon'+1}+1}\right)\right)}{3 \sqrt{2} \sqrt{-2 \epsilon'-\sqrt{4 \epsilon'+1}-1} \sqrt{(\epsilon'-\epsilon)^2-\epsilon}}\\
       &A=\left(2 \epsilon'+\sqrt{4 \epsilon'+1}+1\right) (2 \epsilon' \beta +\beta +6)\\
        &E_2(\epsilon)=-\frac{B\left(\left(\sqrt{4 \epsilon'+1}+2 \epsilon' \left(\sqrt{4 \epsilon'+1}+2\right)+1\right) \beta +6 \sqrt{4 \epsilon'+1}\right) F\left(i \sinh ^{-1}\left(\frac{\sqrt{-2 \epsilon'-\sqrt{4 \epsilon'+1}-1}}{\sqrt{2} \sqrt{\epsilon}}\right)|\frac{2 \epsilon'-\sqrt{4 \epsilon'+1}+1}{2 \epsilon'+\sqrt{4 \epsilon'+1}+1}\right)}{3 \sqrt{2} \sqrt{-2 \epsilon'-\sqrt{4 \epsilon'+1}-1} \sqrt{(\epsilon'-\epsilon)^2-\epsilon}}\\
        &B=i \beta ^2 \epsilon \sqrt{\frac{-2 \epsilon'+\sqrt{4 \epsilon'+1}+2 \epsilon-1}{\epsilon}} \sqrt{-\frac{2 \epsilon'+\sqrt{4 \epsilon'+1}-2 \epsilon+1}{\epsilon}} 
    \end{split}
\end{equation}
In the above solution, F is an incomplete elliptic function of the first kind, and E is an elliptic function of the second kind. P and T are polynomial and transcendental functions, respectively.Similar elliptic functions are found to solve nonhermitian problem \cite{kulkarni2022derivation} to connect the nonequilibrium self-energy of an open system.

% The \nocite command causes all entries in a bibliography to be printed out
% whether or not they are actually referenced in the text. This is appropriate
% for the sample file to show the different styles of references, but authors
% most likely will not want to use it.
%\nocite{*}
\twocolumngrid
\bibliography{ref}% Produces the bibliography via BibTeX.

\end{document}
%
% ****** End of file apssamp.tex ******
