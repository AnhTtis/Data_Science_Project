\section{Analytical framework}
\label{sec:framework}

In this section, we present several results concerning the continuous minimizers of \eqref{eq:energy_functional_times_kappa2}.


From now on, we assume that the magnetic potential $\MagF$ satisfies 
%
\begin{equation} \label{eq:ass_\MagF_for_H2}
	\MagF \in L^\infty(\Omega,\mathbb{R}^d),
	\qquad
	\div \MagF = 0 \text{ in } \Omega,
	\qquad 
	\MagF \cdot \nu =0  \text{ on } \partial \Omega.
\end{equation}
%
Further, we introduce the dual pairing $\dualp{\sol }{ \testfun } \coloneqq \dualp{\sol }{ \testfun }_{\dualHone, H^1}$, and the bilinear forms
given by
%
	\begin{align} 
				\ipsymLtwo{\sol}{\testfun} \coloneqq \Real \int_{\Omega} \sol \testfun^* \,dx ,
				%
				\qquad
				%
		\abilmag{\sol}{\testfun} \coloneqq  \Real \int_\Omega \bigl( \nabla \sol + \ci \kappa \MagF \sol \bigr) \cdot  \bigl( \nabla \testfun + \ci \kappa \MagF \testfun \bigr)^*  \,dx ,
		\label{eq:def_forms_abil}
	\end{align}
%
as well as the norm $\norm{H^1}{\testfun}^2 \coloneqq \norm{L^2}{\nabla \testfun }^2 + \norm{L^2}{ \testfun }^2$, the scaled norms 
%
\begin{equation} \label{eq:def_norms}
	\Honekappa{\testfun}^2 \coloneqq \norm{L^2}{\nabla \testfun }^2 + \kappa^2 \norm{L^2}{ \testfun }^2, 
	\qquad
	\Htwokappa{\testfun}^2 \coloneqq \norm{H^2}{\testfun}^2 + \kappa^2 \Honekappa{\testfun}^2,
\end{equation}
%
and the induced norm $\Honekappaminus{f} = \sup_{\testfun \in H^1} \frac{f(\testfun)}{\Honekappa{\testfun}}$.
%
We abbreviate $\MagFinfty = \norm{L^\infty}{\MagF}$,
and define the stabilized inner product on $\VS = H^1(\Omega)$ for $\sol,\testfun\in \VS$ by
%
\begin{equation} \label{eq:def_abilmagstabsym}
	\abilmagstabsym{\sol}{\testfun}  \coloneqq \abilmag{\sol}{\testfun} + \stabPar^2 \ipsymLtwo{\sol}{\testfun}_{L^2} ,
	\quad \text{with } \stabPar^2 = \kappa^2 (\MagFinfty^2 +1).
\end{equation}
%
We call it stabilized since this modification enables us to show boundedness and coercicvity of $\abilmagstabsym{\cdot}{\cdot}$ with respect to the
$\HonekappaSpace$-norm defined in \eqref{eq:def_norms}.

\begin{lemma}
	\label{lem:prop_bil}
	There are $\kappa$-independent constants $\Cbnd,\Ccoer>0$ such that
	for all $\testfunTWO,\testfun\in \VS$
	\begin{align}
		\abilmagstabsym{\testfunTWO}{\testfun} &\leq \Cbnd \, \Honekappa{\testfunTWO} \Honekappa{\testfun},
		\qquad
		\text{and}
		\qquad
		%
		\abilmagstabsym{\testfun}{\testfun} \geq  \Ccoer \, \Honekappa{\testfun}^2.
	\end{align}
\end{lemma}

\begin{proof}
The boundedness is a straightforward application of the Cauchy-Schwarz inequality. For the coercivity, we note that by Young's inequality it holds
%
\begin{equation}
|\nabla \testfun + \ci \kappa \MagF \testfun|^2 
\geq
 |\nabla \testfun|^2 - 2   |\nabla \testfun| |\kappa \MagF \testfun| + |\kappa \MagF \testfun|^2
  \geq 
  \frac12 |\nabla \testfun|^2 - \kappa^2 \MagFinfty^2 |\testfun|^2.
\end{equation}
%
By the choice of $\beta$, we conclude the lower bound.
\end{proof}

A straightforward calculation shows that the
energy is  (real-)\Frechet differentiable and satisfies for all $\testfun\in H^1$
%
\begin{align} \label{eq:first_Frechet}
	\dualp{\energy'(\sol)}{\testfun}
	&= 
	\Real  \int_\Omega \bigl( \nabla \sol + \ci \kappa \MagF \sol \bigr)  \cdot \bigl( \nabla \testfun + \ci \kappa \MagF \testfun \bigr)^*  
	+
	\kappa^2 \bigl( |\sol|^2 -1 \bigr)  \sol \testfun^* 
	\,dx .
\end{align}
%
In particular any minimizer $\sol \in H^1$ satisfies $\energy'(\sol) = 0$.
Our first result collects the existence of a minimizer $\sol$ and its properties. 

\begin{theorem} \label{thm:cont_minimizer}
	For every $\kappa \geq 0$ there exists a minimizer $\sol \in H^1$ of  \eqref{eq:energy_functional_times_kappa2}.
	%
	Further, any minimizer fulfills
	%
	\begin{align}
	|\sol(x)| \leq 1 \,\mbox{ for all } x\in \Omega ,
	 \qquad 
	 \Honekappa{\sol}\lesssim \kappa , 
	  \quad 
	 \mbox{and if $\Omega$ is convex then $\sol \in H^2$ and} 
	 \quad
	 \Htwokappa{\sol} \lesssim \kappa^2,
	\end{align}
%
where the hidden constants in the above estimates are independent of $\kappa$ and $\sol$.
\end{theorem}

\begin{proof}
	%
	First note that the energy $\energy$ is continuous in $H^1(\Omega)$, and further weakly lower semi-continuous, see e.g., \cite[Thm.~1.6]{Struwe08}. In addition, a simple calculation shows
	%
	\begin{equation}
		\energy(\sol) = \abilmagstabsym{\sol}{\sol} 
		+
		\frac{\kappa^2}{2}  \int_\Omega
		  \bigl( 1 + \frac{2 \stabPar^2}{\kappa^2} - |\sol|^2 \bigr)^2  
		  +  
		  1
		  -
		  \bigl( 1 + \frac{2 \stabPar^2}{\kappa^2} \bigr)^2 \,dx,
	\end{equation}
	%
	and hence $\energy(\sol) \to \infty$ as $\Honekappa{\sol} \to \infty$.
	The standard arguments then imply the existence of a minimizer, see e.g., \cite[Thm.~1.2]{Struwe08}.
	%
	For the pointwise bound, we refer to \cite[Prop.~3.11]{DuGP92}, which implies a bound in $L^2$ independent of $\kappa$.  We further have 
	%
	\begin{equation}
		\norm{L^2}{\nabla \sol} \leq \norm{L^2}{\nabla \sol + \ci \kappa \MagF \sol} + \kappa \MagFinfty \norm{L^2}{\sol} \lesssim \energy(0)^{1/2} + \kappa  \lesssim \kappa .
	\end{equation}
	%
	Since $\energy'(\sol) = 0  $, we rearrange to
	%
	\begin{align}
	\abilmag{\sol}{\testfun} =
	- \kappa^2 \Real \int_\Omega \bigl( |\sol|^2 -1 \bigr)  \sol \testfun^* 
	\,dx
	%
	= \ipsymLtwo{ f }{\testfun} 
\end{align}
	%
	with $\norm{L^2}{f} \lesssim  \kappa^2 $, and obtain with \eqref{eq:ass_\MagF_for_H2}
	%
	\begin{equation} 
	\Real \int_{\Omega}
	\nabla \sol \cdot \nabla \testfun^* 
	\,dx
	= \ipsymLtwo{ f }{\testfun} 
%
	-
	\Real \int_{\Omega}
	\bigl(
	2 \ci \kappa \MagF \cdot \nabla \sol 
	+ \kappa^2 |\MagF|^2 \sol  \bigr) \testfun^*
	\,dx .
\end{equation}
	%
	If $\Omega$ is convex, standard elliptic regularity theory (cf. \cite{GiT01}) gives us
	\begin{equation}
		\norm{H^2}{\sol} \lesssim  \norm{L^2}{f} + \kappa^2 \norm{L^2}{\sol} + \kappa \norm{L^2}{\nabla \sol} 
		\lesssim \kappa^2,
	\end{equation}
	%
	where we used the $L^2$- and $H^1$-bounds for $\sol$ in the last step.
\end{proof}

Since $u$ is a global minimizer of the energy $\energy$, it must not only hold $\langle E^{\prime}(u) , \testfun\rangle =0$ but also $\langle E^{\prime\prime}(u) \testfun , \testfun \rangle \ge 0$ for all $\testfun\in H^1$. Later we will make use of these conditions. For that we require a corresponding representation of the second \Frechet derivative of $\energy$. This and its properties are summarized in the following lemma.

\begin{lemma} \label{lem:Frechet_functional}
\bulletpoint{a}	The energy is twice (real-)\Frechet differentiable and satisfies for $\testfun,\testfunTWO \in H^1$
	\begin{align}
	\dualp{\energy''(\sol) \testfunTWO}{\testfun}	 &=  
	%
	\Real  \int_\Omega \bigl( \nabla \testfunTWO + \ci \kappa \MagF \testfunTWO \bigr)  \cdot \bigl( \nabla \testfun + \ci \kappa \MagF \testfun \bigr)^*   
	+
	\kappa^2 
	\bigl(  ( |\sol|^2 -1  ) \testfunTWO \testfun^*  + \sol^2 \testfunTWO^* \testfun^* + |\sol|^2 \testfunTWO \testfun^* \bigr)
	\,dx .
	%
\end{align}
%

\bulletpoint{b}	
For  $\testfun,\testfunTWO\in H^1$ it holds
%
\begin{equation} \label{eq:E_prime_prime_sym}
	\dualp{\energy''(\sol) \testfunTWO}{\testfun}	 = \dualp{\energy''(\sol) \testfun}{\testfunTWO}	
	\quad
	\text{and}
	\quad
	|  \dualp{\energy''(\sol)  \testfunTWO}{\testfun}  | 
	\lesssim  \Honekappa{\testfunTWO} \Honekappa{\testfun} .
\end{equation}
%
\end{lemma} 

\begin{proof}
The \Frechet derivative is computed in a straightforward manner, and the symmetry follows from the representation by noting the real part in front of the integral. For the bound, we employ Lemma~\ref{lem:prop_bil} as well as $|\sol| \leq 1$.
\end{proof}

Let $\sol$ be a minimizer of \eqref{eq:energy_functional_times_kappa2},
then by the invariance under complex rotation, also $e^{\ci \phi} \sol$ is a minimizer for any $\phi \in \R$. In particular, one easily shows that 
$	\dualp{\energy''(\sol) \, \ci \sol}{\testfun} = 0 $ holds for all $\testfun \in H^1$.
%
To tackle this indefiniteness,
we define the $\ipsymLtwo{\cdot}{\cdot}$-orthogonal complement of $\ci u$ in $H^1$ by 
%
\begin{equation}
	\Honeperp \coloneqq H^1 \cap ( \ci \sol)^\perp \coloneqq \{ \testfun \in H^1 \, | \, m(\ci \sol , \testfun ) = 0 \}.
\end{equation}
%
In our error analysis we will restrict ourselves to this space. The choice of $\Honeperp$ is further discussed in connection with Assumption \ref{ass:cinfsup} below.

Note that $\Honeperp$ is a closed subspace of $H^1$. 
Since the variational problems in the following proofs are posed on this subspace, we show the following properties of their solutions.



\begin{lemma} \label{lem:wepo_abilmagstab}
	For any $f \in L^2(\Omega)$, there is $\testfunFOUR\in \Honeperp \subset H^1(\Omega)$ such that
	%
	\begin{equation}
		\abilmagstabsym{\testfunFOUR}{\testfun} =  \ipsymLtwo{f}{\testfun}, \quad  \text{ for all } \testfun \in \Honeperp ,
	\end{equation}
	%
	and there hold the bounds
	%
	\begin{equation}
		\Honekappa{\testfunFOUR} 
		\lesssim \Honekappaminus{f}
		\lesssim \frac{1}{\kappa} \norm{L^2}{f}
		\quad 
		\mbox{and, if $\Omega$ is convex, then $\testfunFOUR \in H^2$ and} 
		\
		\Htwokappa{\testfunFOUR} \lesssim  \norm{L^2}{f},
	\end{equation}
	%
	where the (hidden) constants in the bounds are independent of $\kappa$. 
	\end{lemma}




\begin{proof}
	Since $\abilmagstabsym{\cdot}{\cdot}$ is still coercive on $\Honeperp$, we immediately obtain the unique solution, and also the bounds in $\HonekappaSpace$. 
	Furthermore, we have for any $f \in L^2$ that
	%
	\begin{equation} \label{eq:relation_Honeminus_L2}
		\Honekappaminus{f} 
		=
		\sup_{  \Honekappa{\testfun} = 1} \ipsymLtwo{f}{\testfun}
		\leq
		\sup_{  \Honekappa{\testfun} = 1}  \frac{1}{\kappa} \norm{L^2}{f} \kappa \norm{L^2}{\testfun}
		\leq  \frac{1}{\kappa} \norm{L^2}{f} ,
	\end{equation}
	%
	which yields the second inequality. For the bound in the $\HtwokappaSpace$-norm for convex domains, let $\testfun \in H^1$ and decompose as $\testfun = \widehat{\testfun} + \alpha  (\ci \sol) $
	with $\widehat{\testfun} \in \Honeperp$ and
	$\alpha = \ipsymLtwo{\testfun}{\ci \sol} \norm{L^2}{\sol}^{-2} $. Then,
	%
	\begin{align}
		\abilmagstabsym{\testfunFOUR}{\testfun} &= 	\abilmagstabsym{\testfunFOUR}{\widehat{\testfun} } +  \alpha \, \abilmagstabsym{\testfunFOUR}{ \ci \sol }
		=  \ipsymLtwo{f}{\widehat{\testfun} }  +  \alpha \, \abilmagstabsym{\testfunFOUR}{ \ci \sol }
		\\
		&=  \ipsymLtwo{f}{\testfun}  -
		\alpha \, \ipsymLtwo{f}{\ci \sol }+  \alpha \, \abilmag{\testfunFOUR}{ \ci \sol },
	\end{align}
	%
	where we used \eqref{eq:def_abilmagstabsym} in the last step. We first note
	%
	\begin{equation}
		| \ipsymLtwo{f}{\testfun}  -
		\alpha \, \ipsymLtwo{f}{\ci \sol } | \leq 2 \norm{L^2}{f} \norm{L^2}{\testfun} \,,
	\end{equation}
	%
	and then employ $\energy'(\ci \sol) = 0$ to obtain
	%
	\begin{align}
		| \abilmag{\testfunFOUR}{ \ci \sol } | &=
		| \dualp{\energy'(\ci \sol)}{\testfunFOUR} -
		\kappa^2 	\Real  \int_\Omega \bigl( |\sol|^2 -1 \bigr) \ci \sol \testfunFOUR^* 
		\,dx |
		\leq \kappa^2 \norm{L^2}{\sol} \norm{L^2}{\testfunFOUR} \lesssim \norm{L^2}{f},
	\end{align}
	%
	where we exploited $ \kappa^2 \norm{L^2}{\testfunFOUR} \lesssim \norm{L^2}{f}$ in the last line. Altogether we have shown that there exists some $f_\testfunFOUR\in L^2$ such that it holds for all $\testfun \in H^1$
	%
	\begin{align} \label{eq:var_form_extended_test_fun}
		\abilmagstabsym{\testfunFOUR}{\testfun} &=  \ipsymLtwo{f_\testfunFOUR}{\testfun}  ,\quad \norm{L^2}{f_\testfunFOUR} \lesssim \norm{L^2}{f}.
	\end{align}
	%
	We conclude as in Theorem~\ref{thm:cont_minimizer}: We write
	%
	\begin{equation} \label{eq:expansion_ahat}
		\abilmagstabsym{\testfunFOUR}{\testfun} 
		=  	  
		\Real \int_{\Omega}
		\nabla \testfunFOUR\cdot \nabla \testfun^* 
		\,dx
		+
		\Real \int_{\Omega}
		\bigl(\stabPar^2 \testfunFOUR 
		+ 2 \ci \kappa \MagF \cdot \nabla \testfunFOUR
		+ \kappa^2 |\MagF|^2 \testfunFOUR \bigr) \testfun^*
		\,dx ,
	\end{equation}
	%
	and since the second term is in $L^2$, we have $\testfunFOUR\in H^2$ and
	%
	\begin{equation}
		\norm{H^2}{\testfunFOUR} \lesssim  \norm{L^2}{f_\testfunFOUR} + \kappa^2 \norm{L^2}{\testfunFOUR} + \kappa \norm{L^2}{\nabla \testfunFOUR} 
		\lesssim \norm{L^2}{f},
	\end{equation}
	%
	where we used the $L^2$- and $H^1$-bounds for $\testfunFOUR$ in the last step.
\end{proof}
%

We now turn to the key assumption in our analysis. As we have seen above, we cannot expect uniqueness of a minimizer due to the rotation invariance.  However, we assume that apart from this, the minimizer is locally unique. For that we can restrict the energy to an appropriate subspace. To be precise, if $\sol$ is a global minimizer of $\energy$, we know that $\energy^{\prime}(\sol)=0$ and that the spectrum of $\energy^{\prime\prime}(\sol)$ is non-negative. On the other hand, it is easily seen that $\ci \sol$ is an eigenfunction of $\energy^{\prime\prime}(\sol)$ with eigenvalue $0$. This eigenvalue corresponds to the aforementioned invariance of $\energy$ under rotations of the form $e^{\ci \phi}$. By assuming that the remaining spectrum of $\energy^{\prime\prime}(\sol)$ is strictly positive we can hence guarantee that the solution $\sol$ is locally unique (up to rotations). A positive spectrum of $\energy^{\prime\prime}(\sol)$ on the $\ipsymLtwo{\cdot}{\cdot}$-orthogonal complement of the eigenfunction $\ci \sol$ (i.e. the space $\Honeperp$) implies inf-sup stability of $\energy^{\prime\prime}(\sol)$ on $\Honeperp$. This is precisely what the following assumption says.
%

\begin{assumption} \label{ass:cinfsup}
	Let $\sol$ be a minimizer of \eqref{eq:energy_functional_times_kappa2}.  Then, there is a constant $\Csol \gtrsim 1$ such that
	%
	\begin{equation} \label{eq:inf_sup_condition_cont}
		\Csolinv  \leq \inf_{\testfunTWO \in \Honeperp} \sup_{\testfun\in \Honeperp} \frac{\dualp{\energy''(\sol)  \testfunTWO}{\testfun}}{\Honekappa{\testfunTWO} \Honekappa{\testfun}  } .
	\end{equation}
\end{assumption}

Let us note that the condition $\Csol \gtrsim 1$ is not a restriction, since 
one can drop the condition replacing $\Csol$ by $1+ \Csol$ at every occurrence.

\begin{remark}
From our numerical experiments, the precise growth of $\Csolinv$ with respect to $\kappa$ does not become clearly visible. In fact, it turns out to be difficult to numerically compute the inf-sup constants on a space which contains information on the exact solution.
%
In addition, we are not aware of any literature (neither in analysis nor numerics) addressing the (spectral) properties of $\energy''(\sol)$.
%
We are convinced that this an interesting research question which might be pursued in the future, both analytically and numerically.
\end{remark}


From the above assumption, we can conclude solvability and a priori bounds which will play a crucial role in the presented error analysis below.
%
Let us note that the inclusion $\Honeperp \subset H^1$ implies for the dual spaces $\dualHone \subset  (\Honeperp)'$.


\begin{corollary} \label{cor:du_F_inf_sup_solvability}
	Let Assumption~\ref{ass:cinfsup} hold.	
	
	\bulletpoint{a} For any $f \in  (\Honeperp)' $, 
	there is a unique $\testfunFOUR \in \Honeperp$ such that
	%
	\begin{equation} \label{eq:du_F_var_problem}
		\dualp{\energy''(\sol) \testfunFOUR}{\testfun}  = \dualp{f}{\testfun}, \quad \text{for all} \quad \testfun \in \Honeperp ,
	\end{equation} 
	%
	which satisfies the estimate
	%
	\begin{equation}
		\Honekappa{\testfunFOUR} \leq \Csol \Honekappaminus{f} .
	\end{equation}
	
	
	\bulletpoint{b} Let $\testfunFOUR \in \Honeperp$ be the solution of \eqref{eq:du_F_var_problem} with $f \in L^2$.
	 Then, it further holds
	%
	\begin{align}
		\Honekappa{\testfunFOUR} 
		&\leq \frac{\Csol}{\kappa} \norm{L^2}{f}
		\quad
		\mbox{and, if $\Omega$ is convex, then $\testfunFOUR \in H^2$ and} 
		\
		\Htwokappa{\testfunFOUR} \lesssim \Csol  \norm{L^2}{f} .
		%
	\end{align}
\end{corollary}


\begin{proof}
	By standard theory for indefinite differential equations (cf. \cite{Bab7071}), the inf-sup stability in Assumption~\ref{ass:cinfsup} directly gives the \wepo of  \eqref{eq:du_F_var_problem} together with the stability estimate $\Honekappa{\testfunFOUR} \leq \Csol \Honekappaminus{f}$, hence proving (a). The first estimate in (b) is obtained from \eqref{eq:relation_Honeminus_L2}.
	%
	Using this observation, we conclude that $\testfunFOUR\in \Honeperp$ solves
	%
	\begin{equation}
		\abilmagstabsym{\testfunFOUR}{\testfun} =  \ipsymLtwo{\widetilde{f}}{\testfun}, \quad  \text{ for all } \testfun \in \Honeperp \,,
	\end{equation}
%
for some $\widetilde{f}\in L^2$ with $\norm{L^2}{\widetilde{f}} \lesssim \Csol \norm{L^2}{f}$, 
and thus Lemma~\ref{lem:wepo_abilmagstab} gives the claim.
\end{proof}

If one considers domains with smooth boundaries, and uses magnetic vector potential in some higher order Sobolev spaces, higher regularity of the minimizer $\sol$ can be derived. However, for our purposes the $H^2$-regularity is sufficient, and we hence turn to the spatial discretization.





