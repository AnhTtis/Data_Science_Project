%\newpage
\section{Relaxed $\kappa$-dependencies in LOD spaces}
\label{sec:lod}
%
In this final section we present a nonstandard application of the abstract approximation result in Theorem~\ref{thm:main}. For that we consider spaces based on the so-called Localized Orthogonal Decomposition (LOD).  LOD spaces were originally developed in the context of elliptic multiscale problems with rough coefficients to efficiently handle low regularity and unresolved scales \cite{MaP14}. An introduction to the methodology is given in the textbook by M{\aa}lqvist and Peterseim \cite{MaP21} and the review article by Altmann et al. \cite{AHP21acta}. Recently, new applications of these spaces emerged in the field of quantum mechanics where they were used to boost the performance of traditional discretizations \cite{HePer21,HeW22,WuZh22}. As we will see see, the Ginzburg-Landau equation could be yet another promising application of LOD spaces in the context of quantum physics.

To define suitable LOD spaces for the GLE and to characterize its approximation properties in an abstract way, we start from a linear Lagrange finite element space $\VSh$ as defined in \eqref{eq:defVh} and  assume that the underlying triangulation $\Th$ is shape-regular and quasi-uniform. The LOD space is now constructed from $\VSh$ by applying the inverse of a differential operator to the functions of $\VSh$. In our case, we use the differential operator associated with the bilinear form $\abilmagstabsym{\cdot,\cdot}$. The construction is made precise in the following definition.
\begin{definition}[LOD spaces]
\label{definition:LODspace}
Let $\abilmagstabsym{\cdot}{\cdot}$ denote the symmetric, continuous and coercive bilinear form on $H^1(\Omega,\mathbb{C})$ given by \eqref{eq:def_forms_abil} and
let $\hat{\mathcal{A}}_{\kappa}^{-1}$ denote the corresponding solution operator on $L^2$, i.e., for $f \in L^2(\Omega,\C)$ the image $\hat{\mathcal{A}}_{\kappa}^{-1} f \in H^1$ is given by the solution to
\begin{align}
\abilmagstabsym{ \,\hat{\mathcal{A}}_{\kappa}^{-1}f \, }{ \, \testfun \, } = \ipsymLtwo{f }{ \testfun} \qquad \mbox{ for all  }  \testfun \in H^1.
\end{align}
With this definition, the LOD space based on $\abilmagstabsym{\cdot}{\cdot}$ and $\VSh$ is given by
\begin{align}
\VShLOD := \hat{\mathcal{A}}_{\kappa}^{-1} \VSh.
\end{align}
\end{definition}
We note that the above definition of LOD spaces formally differs from the construction given in the classical references \cite{MaP14,HeP13,HeM14}. However, the characterizations are indeed equivalent as can be extracted from e.g., \cite{HaP23} and \cite{AHP21acta}.

From a practical perspective it is also important to note that the space $\VShLOD$ admits a quasi-local basis, i.e., basis functions that are (super-)exponentially decaying in distances of the mesh size $h$. Details on the practical computation/approximation of such basis functions are given in \cite{EHMP19} and recent super-localization strategies are presented in \cite{HaP23}. Corresponding numerical errors that might arise from the approximation of basis functions are well understood \cite{AHP21acta} and will be for brevity disregarded in the following error analysis.
%

The approximation properties of the idealized space $\VShLOD$ are summarized in the following proposition.
%
%
\begin{proposition}[Approximation properties of $\VShLOD$]
\label{proposition:LOD-abstract-est}
Let $\VShLOD$ be the LOD-space from Definition \ref{definition:LODspace} and let $f \in L^2$ be given. 
If $\sol \in H^1$ denotes the solution to
$$
\abilmagstabsym{\sol}{\testfun} = \ipsymLtwo{f }{ \testfun} \qquad \mbox{ for all  }  \testfun \in H^1
$$
and if $\RitzprojLOD \sol \in \VShLOD$ denotes the corresponding $\abilmagstabsym{\cdot}{\cdot}$-Ritz-projection of $\sol$ in $\VShLOD$, then it holds
\begin{align}
\label{eqn:abstract-estimate-LOD}
\Honekappa{ \sol - \RitzprojLOD \sol} \, \lesssim\, h \, \norm{L^2}{f - \Ltwoproj f},
\end{align}
%
where we recall $\Ltwoproj : L^2 \rightarrow \VSh$ as the $L^2$-projection on $\VSh$.
The hidden constant in \eqref{eqn:abstract-estimate-LOD} is generic and depends on the coercivity and continuity constants of $\abilmagstabsym{\cdot}{\cdot}$, as well as the mesh regularity, but it does not depend on $h$ and $\kappa$.

Furthermore, for every $\phi \in H^1$ there exists a unique decomposition such that
\begin{align}
\label{equation:LODdecomposition}
\phi = \phi^{\LOD} + \phi_0, \qquad \mbox{where } \,\,\phi^{\LOD}\in\VShLOD,\quad \Ltwoproj \phi_0 = 0
\quad \mbox{ and } \quad \abilmagstabsym{\phi^{\LOD}}{\phi_0} = 0.
\end{align}
\end{proposition}
The result is standard and can be for instance found in \cite{HeW22} for homogeneous Dirichlet boundary conditions. For generalizations to higher order FE spaces and to only piecewise smooth source terms $f$, we refer to \cite{Mai21}.  

Analogously, to standard Lagrange finite elements, it is also possible to quantify the approximation properties of $\RitzprojLOD $ for general smooth functions. This is done in the following lemma.
%
\begin{lemma}
\label{lemma:RitzLOD-properties}
Let $\VShLOD$ be the LOD-space from Definition \ref{definition:LODspace} and let $\RitzprojLOD: H^1 \rightarrow \VShLOD$ denote the corresponding Ritz-projection w.r.t. $\abilmagstabsym{\cdot}{\cdot}$.
	Then, for every $\testfunTHREE \in H^2$, there is a $f_\testfunTHREE \in L^2$ such that
	%
	\begin{equation}
		\abilmagstabsym{\testfunTHREE}{\testfun} = \ipsymLtwo{f_\testfunTHREE}{\testfun} \quad \mbox{and} \quad \norm{L^2}{f_\testfunTHREE} \lesssim \Htwokappa{\testfunTHREE}.
	\end{equation}
%
Consequently, for all $\testfunTHREE \in H^2$ it holds
%
\begin{align}
	\Honekappa{\testfunTHREE - \RitzprojLOD \testfunTHREE}  \lesssim \h \Htwokappa{\testfunTHREE}.
\end{align}
\end{lemma}

\begin{proof}
	From \eqref{eq:expansion_ahat}, we obtain using integration by parts
	%
	\begin{equation}
		\abilmagstabsym{\testfunTHREE}{\testfun} 
		=  	  
		\Real \int_{\Omega}
	\bigl( - \Delta \testfunTHREE + 
	\stabPar^2 \testfunTHREE  
		+ 2 \ci \kappa \MagF \nabla \testfunTHREE 
		+ \kappa^2 |\MagF|^2 \testfunTHREE  \bigr) \testfun^*
		\,dx  \eqqcolon \ipsymLtwo{f_\testfunTHREE}{\testfun}
	\end{equation}
%
and the bound for $\norm{L^2}{f_\testfunTHREE}$ follows. Proposition \ref{proposition:LOD-abstract-est} finishes the second part of the lemma.
%
\end{proof}
%

From this lemma, we can deduce property \bulletpoint{a} in Assumption~\ref{ass:FEM_space}.
	For the second property, we need a variant of this result given in the next lemma.
		
\begin{lemma}
	\label{lemma:RitzLOD-properties_v2}
	Let $\VShLOD$ be the LOD-space from Definition \ref{definition:LODspace} and let $\RitzprojLOD: H^1 \rightarrow \VShLOD$ denote the corresponding Ritz-projection w.r.t. $\abilmagstabsym{\cdot}{\cdot}$.
	For $f \in L^2$ let $\testfunTHREE \in \Honeperp$ be the solution of
	%
	\begin{equation}
		\abilmagstabsym{\testfunTHREE}{\testfun} = \ipsymLtwo{f}{\testfun} \quad \mbox{for all } \quad \testfun \in \Honeperp.
		\end{equation}
	%
	Then, it holds
	%
	\begin{align}
		\Honekappa{\testfunTHREE - \RitzprojLOD \testfunTHREE}  \lesssim \h \norm{L^2}{f}.
	\end{align}
\end{lemma}

\begin{proof}
As in the proof of Lemma~\ref{lem:wepo_abilmagstab} in \eqref{eq:var_form_extended_test_fun}, we know that $\testfunTHREE$ solves the variational problem also tested against all $\testfun \in H^1$ for some modification of $f$ which is bounded in $L^2$ by $\norm{L^2}{f}$. Hence, the assertion follows from
Proposition~\ref{proposition:LOD-abstract-est}.
\end{proof}

To apply the general error estimates in Theorem \ref{thm:main}, we need to verify Assumption \ref{ass:FEM_space} for the LOD space $\VShLOD$. As $H^2$ is a dense subset of $H^1$, the property (a) follows from the second part of Lemma \ref{lemma:RitzLOD-properties}.
For property (b), we require the following lemma.
%
\begin{lemma}
\label{lemma:ritzprojorth-LOD}
Let again $\RitzprojLOD: H^1 \rightarrow \VShLOD$ denote the Ritz-projection onto the LOD-space $\VShLOD$ and let
\begin{align}
\projLtwoisolLOD : \Honeperp \rightarrow \VShLOD \cap (\ci \sol)^\perp
\end{align}
denote the corresponding Ritz-projection onto $\VShLOD \cap (\ci \sol)^\perp$. 
If $h$ is small enough, in particular $h \lesssim \kappa^{-1}$, then it holds for all $\testfun \in \Honeperp$
\begin{align}
	\Honekappa{\testfun - \projLtwoisolLOD \testfun}  \lesssim \Honekappa{\testfun - \RitzprojLOD \testfun}. 
\end{align}
\end{lemma}


\begin{proof}
To proceed as in the proof of Lemma \ref{lem:proj_Lagrange}, we note that by the LOD-decomposition \eqref{equation:LODdecomposition} we have $\Ltwoproj\left( \ci \sol - \RitzprojLOD (\ci \sol) \right) = 0$. Hence, with the approximation properties of $\Ltwoproj$:
\begin{align}
\label{equation:L2-est-RitzprojLOD}
\norm{L^2}{ \RitzprojLOD (\ci \sol)  - \ci \sol }
\lesssim h \, \Honekappa{ \RitzprojLOD (\ci \sol)  - \ci \sol }
\lesssim h \, \Honekappa{ \ci \sol } \lesssim h \, \kappa.
\end{align}
%
This implies for all $\testfun \in \Honeperp$ 
	%
	\begin{align}
		\Honekappa{ \testfun - \projLtwoisolLOD \testfun } 
		%
		&\lesssim \Honekappa{  \testfun - \Bigl( \RitzprojLOD \testfun - 
		\frac{  \ipsymLtwo{\RitzprojLOD \testfun}{\ci \sol}}{\ipsymLtwo{\RitzprojLOD(\ci \sol)}{\ci \sol} } \RitzprojLOD(\ci \sol) \Bigr) } \\
		%
		&\le  \Honekappa{  \testfun - \RitzprojLOD \testfun } 
		+ 
		\frac{\ipsymLtwo{\RitzprojLOD \testfun - \testfun}{\ci \sol}}{\ipsymLtwo{\RitzprojLOD(\ci \sol)}{\ci \sol}}
		\Honekappa{ \RitzprojLOD(\ci \sol) } 
		%
		\\
		&\hspace{-4pt}\overset{\eqref{equation:L2-est-RitzprojLOD}}{\lesssim} \Honekappa{  \testfun - \RitzprojLOD \testfun } 
		+ 
		\norm{L^2}{ \testfun - \RitzprojLOD \testfun }
		\frac{ \| \sol \|_{L^2} }{
			\ipsymLtwo{\RitzprojLOD(\ci \sol) - \ci \sol}{\ci \sol}
			+  \norm{L^2}{  \sol }^2  }  \Honekappa{ \sol } 
		\\
		%
		&\lesssim  \Honekappa{  \testfun - \RitzprojLOD \testfun } + \frac{\kappa}{1 -  c \kappa \h} \norm{L^2}{  \testfun - \RitzprojLOD \testfun }
		\lesssim \Honekappa{  \testfun - \RitzprojLOD \testfun }.\\[-3.0em]
	\end{align}
%
\end{proof}
%
Lemma \ref{lemma:ritzprojorth-LOD} together with Theorem \ref{thm:main} guarantees that the $\HonekappaSpace$-error between an exact solution $u$ and a corresponding approximation in the LOD-space is bounded by $\Honekappa{ \sol - \RitzprojLOD \sol}$. The next lemma quantifies this error. 


\begin{lemma}
\label{lemma:estimate-u-LODspace}
Let $\sol$ be a minimizer of \eqref{eq:energy_functional_times_kappa2} and let $\RitzprojLOD: H^1 \rightarrow \VShLOD$ be the Ritz-projection onto $\VShLOD$. Then it holds at least
	%
	\begin{align}
		\Honekappa{ \sol - \RitzprojLOD \sol}  \lesssim \kappa^{3}\, \h^{2} %
	\end{align}
and, if $\Omega$ is convex, we have $u\in H^2$ and the estimate improves to
	\begin{align}
		\Honekappa{ \sol - \RitzprojLOD \sol}  \lesssim \kappa^{4}\, \h^{3}. %
	\end{align}
\end{lemma}

\begin{proof}
	We want to apply Proposition \ref{proposition:LOD-abstract-est}. By $\energy'(\sol) = 0 $ we have for every $\phi \in \HonekappaSpace$ that
	%
	\begin{align}
	\abilmagstabsym{\sol}{ \testfun}  &= \stabPar 
	\Real  \int_\Omega \sol \testfun^* 
	\,dx 
	 -
		\kappa^2 	\Real  \int_\Omega \bigl( |\sol|^2 -1 \bigr)  \sol \testfun^* 
		\,dx 
		%
		= \ipsymLtwo{\stabPar^2 \sol - \kappa^2 \bigl( |\sol|^2 -1 \bigr)  \sol  }{\testfun} .
\end{align}
	%
Since $\stabPar^2 \sol - \kappa^2 \bigl( |\sol|^2 -1 \bigr)  \sol$ is at least in $H^1$ and even in $H^2$ for convex domains, one easily verifies that for $s=0,1,2$
	%
	\begin{equation}
		\norm{H^s}{\stabPar^2 \sol - \kappa^2 \bigl( |\sol|^2 -1 \bigr)  \sol }
		\lesssim
		\kappa^2 	\bigl( 	\norm{H^1}{\sol  }^s  + 	\norm{H^s}{\sol  } \bigr) 
 		 \leq \kappa^{s+2},
	\end{equation}
%
where we used the bounds from Theorem \ref{thm:cont_minimizer} and in particular repeatedly $|\sol| \leq 1$. The estimate now follows with Proposition \ref{proposition:LOD-abstract-est} and standard estimates for the $L^2$-projection $\Ltwoproj$ on $P1$ finite element spaces.
\end{proof}
%
By collecting the previous results we obtain our final main result which shows the superapproximation properties of the LOD space, even on nonconvex domains.
%
\begin{theorem} \label{thm:final_bounds_LOD_H1}
Let Assumption~\ref{ass:cinfsup} hold and let $\h$ be sufficiently small in the sense of Theorem \ref{thm:main}. If $\VShLOD$ denotes the LOD-space from Definition \ref{definition:LODspace} and if $\solhLOD \in \VShLOD$ is a corresponding minimizer of the Ginzburg--Landau energy with
\begin{equation} 
	\energy(\solhLOD) = \inf\limits_{\testfun \in \VShLOD } E(\testfun),
\end{equation}
then, there is neighborhood $\Nbh \subset H^1(\Omega)$ of $\solhLOD$ and a unique minimizer $\sol \in \Nbh$ of \eqref{eq:energy_functional_times_kappa2} with $\ipsymLtwo{\solhLOD}{\ci \sol} = 0$ and such that
%%
 \begin{align}
\Csolinv \, \norm{L^2(\Omega)}{ \sol - \solhLOD}  + \h \, \Honekappa{ \sol - \solhLOD} &\lesssim \kappa^{3} \h^{3}   ,
\end{align}
and for convex domains $\Omega$ (and consequently $H^2$-solutions) it even holds
 \begin{align}
\Csolinv \, \norm{L^2(\Omega)}{ \sol - \solhLOD}  + \h \, \Honekappa{ \sol - \solhLOD} &\lesssim \kappa^{4} \h^{4} .
\end{align}
%
\end{theorem}

\begin{proof}
Proposition \ref{proposition:LOD-abstract-est} and Lemmas~\ref{lemma:RitzLOD-properties}, \ref{lemma:RitzLOD-properties_v2}, and \ref{lemma:ritzprojorth-LOD} guarantee that Assumption \ref{ass:FEM_space} is fulfilled for $\VShLOD$. Hence, we can apply Theorem \ref{thm:main} together with  Lemmas~\ref{lemma:ritzprojorth-LOD} and~\ref{lemma:estimate-u-LODspace} to conclude that for all sufficiently small $h$ and for $u \in H^s$ with $s\in \{1,2\}$ it holds
\begin{align} 
\Honekappa{ \sol - \solhLOD} \lesssim \Honekappa{\sol - \projLtwoisolLOD \sol}   \lesssim   \Honekappa{ \sol - \RitzprojLOD \sol}  \lesssim \kappa^{s+2}\, \h^{s+1} ,
\end{align}
and
\begin{align}
\norm{L^2}{ \sol - \solhLOD} \lesssim \Csol \h\, \Honekappa{ \sol - \solhLOD} \lesssim \Csol\, \kappa^{s+2}\, \h^{s+2}.
\end{align}
\end{proof}


\begin{remark}
	It is worth to note that, in LOD-spaces, one can also improve the smallness condition on $\kappa \Csol \h$ required for the inf-sup condition in  Lemma~\ref{lem:du_F_inf_sup_solvability_discrete}. 
	%
	In fact, a precise inspection of the proof leads to a smallness condition on $\kappa^2 \Csol \h^2$, which is in general weaker if $1 \lesssim \Csol$. However, since the abstract result contains a term of the form $1 + \kappa \Csol \h$, one cannot exploit this any further in the error analysis, and we thus refrain from giving the proof here.
\end{remark}

