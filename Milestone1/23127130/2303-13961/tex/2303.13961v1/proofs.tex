%\newpage
\section{Proof of the main result}
\label{sec:proofs}

In this section, we provide the proof of our main results 
Theorem~\ref{thm:main} and Corollary \ref{cor:main_Lagrange}. We first show an abstract convergence result in order to identify possible limits of a sequence of discrete minimizers. Those are then used to establish convergence with rates, if we are sufficiently close to a continuous minimizer. 	
Throughout this section, we let Assumptions~\ref{ass:cinfsup} and \ref{ass:FEM_space} hold.


\subsection{Abstract convergence result}
 
In order to deduce convergence, we first establish bounds on minimizers in the discrete space $\VSh$ which are independent of the spatial parameter $\h$ as formulated in Lemma~\ref{lem:bound_general_uH}.


\begin{proof}[Proof of Lemma~\ref{lem:bound_general_uH}]
	First note that for all $\h>0$ we have $0 \in \VSh$, and thus by the minimizing property, we conclude the bound on the energy
	%
	\begin{equation}
		\energy(\solh) \leq \energy(0) 
		 \leq \frac{\kappa^2 }{2} \vol(\Omega).
	\end{equation}
%
This gives on the one hand
	%
	\begin{equation}
		\norm{L^2}{\nabla \solh + \ci \kappa \MagF \solh } \leq \energy(\solh)^{1/2} \leq \kappa  \vol(\Omega)^{1/2},
	\end{equation}
	%
	and on the other hand we estimate
	%
	\begin{align}
		\frac{\kappa^2}{2} \norm{L^2}{1 - |\solh| \, }^2 \leq 
		\frac{\kappa^2}{2} \int_\Omega   \bigl( 1- |\solh| \bigr)^2 \bigl( 1+  |\solh| \bigr)^2 \,dx \leq  
		\energy (0) = \frac{\kappa^2}{2} \vol(\Omega)^{1/2}  ,
		%
	\end{align}
	%
	and thus conclude
	%
	\begin{equation}
		\norm{L^2}{ \solh} \leq \norm{L^2}{1 - |\solh| } + \vol(\Omega)^{1/2} \leq 2 \vol(\Omega)^{1/2}.
	\end{equation}
	%
	Combining the estimates above, the bound on $\norm{L^2}{\nabla \solh}$ directly follows.
\end{proof}

With the uniform estimates on the discrete minimizers, 
following the approach in \cite{CheGZ10},
we employ the Banach--Alaoglu theorem to obtain some limit
which is an exact minimizer and by Assumption~\ref{ass:cinfsup} locally unique up to complex rotation. 


\begin{proposition} \label{prop:abstract_convergence}
	Denote by $(\solh)_{\h>0}$ a family of minimizers of \eqref{eq:energy_functional_discrete}.
	Then, there exists a minimizer $\sol_0$
	of \eqref{eq:energy_functional_times_kappa2} such that there is a monotonically decreasing sequence 
	$( \h_{n} )_{n \in \mathbb{N}}$ with
	%
	\begin{equation}
	\lim_{n\to \infty}
	\Honekappa{ \sol_0 - \solhn{n} } = 0. 
	\end{equation}

In particular, we can define the twisted approximations
	%
	\begin{equation}
	\solhnTwist{n} \coloneqq e^{\ci \phi_n} \solhn{n}
		\qquad
		\mbox{where } \phi_n \in [-\tfrac{\pi}{2},\tfrac{\pi}{2}] \mbox{ is chosen such that }
		%
		\ipsymLtwo{ \solhnTwist{n} }{\ci \sol_0} = 0 \,,
	\end{equation}
	%
	which 
	 also converge in $H^1$, i.e.,  
	%
	\begin{equation}
		\lim_{j\to \infty}
		\Honekappa{ \solhnTwist{n}  - \sol_0} = 0.
	\end{equation}
	%
	Conversely, for any $n$, the minimizer $\solhn{n}$ is an approximation to $e^{ - \ci \phi_n} \sol_0$.
\end{proposition}

\begin{remark}
	The assertion of Proposition~\ref{prop:abstract_convergence} can be interpreted as follows. Assume that there exists a (sub-)sequence of discrete minimizers that keeps a positive distance to all
	exact minimizers, then this would be a contradiction to Proposition~\ref{prop:abstract_convergence}.
	Hence, for $\h$ sufficiently small, one always arrives at a neighborhood of some minimizer $\sol_0$, which is precisely the claim in Theorem~\ref{thm:main}.
\end{remark}

\begin{proof}[Proof of Proposition~\ref{prop:abstract_convergence}]
	%
	The proof of convergence of a subsequence is along the lines of \cite{CheGZ10} if one takes into account the bounds provided in Lemma~\ref{lem:bound_general_uH} together with the weak lower semi-continuity of $\energy$, see Theorem~\ref{thm:cont_minimizer},
	and Assumption~\ref{ass:FEM_space}.
	
	

For the twisted approximations, we note that we can find some $\phi_n \in [-\frac{\pi}{2},\frac{\pi}{2}]$ such that real part of the inner product with $\ci \sol_0$ vanishes if $n$ is large enough.
Thus, we obtain by the choice of $\phi_n$
%
\begin{align}
	 \sin \phi_n  \ \ipsymLtwo{ \solhn{n}    }{ \solhn{n}   } 
	&=
	 \ipsymLtwo{ e^{\ci \phi_n} \solhn{n}    }{  \ci \solhn{n}   }  
=
	 \ipsymLtwo{ e^{\ci \phi_n} \solhn{n}    }{  \ci \solhn{n} - \ci \sol_0  }  .
\end{align}
%
Since the right-hand side tends to zero, either $\sol_0 = 0$ or
$\phi_n \to 0$ holds.
%
In any case, we have
%
	\begin{equation}
	\Honekappa{  \solhnTwist{n} - \sol_0} \leq 	\Honekappa{\solhn{n}  - \sol_0} + 	
	|1 - e^{ \ci \phi_n} | \,
	\Honekappa{ \solhn{n}} \to 0 ,
\end{equation}
%
which yields the assertion.
\end{proof}


\subsection{Discrete inf-sup stability}

In order to derive the error estimates, we first establish a discrete version of the inf-sup condition in \eqref{eq:inf_sup_condition_cont}. 
In the proof, we need the following consequence of Assumption~\ref{ass:FEM_space}.

\begin{corollary}
Let Assumption~\ref{ass:FEM_space} hold, and let $\testfunFOUR \in \Honeperp$ be the solution of
%
\begin{equation}
	\dualp{\energy''(\sol) \testfunFOUR}{\testfun}%
	=
	%
	\dualp{f}{\testfun} ,
	\quad \text{ for all } \testfun\in \Honeperp.
\end{equation}
Then, it holds the estimate
%
\begin{equation} \label{eq:Honekappa_bound_alternative}
	\Honekappa{\testfunFOUR - \projLtwoisol \testfunFOUR} \lesssim   \Csol \h \norm{L^2}{f} .
\end{equation}
\end{corollary}


\begin{proof}
We rewrite with Lemma~\ref{lem:Frechet_functional}
%
\begin{equation}
	\abilmagstabsym{\testfunFOUR}{\testfun} = \ipsymLtwo{f}{\testfun} - \ipsymLtwo{f_\testfunFOUR}{\testfun}  	\quad \text{ for all } \testfun\in \Honeperp.
\end{equation}
%
Here, $f_\testfunFOUR$ satisfies
%
\begin{equation}
\norm{L^2}{f_\testfunFOUR} \lesssim \kappa^2 \norm{L^2}{z} \lesssim \Csol \norm{L^2}{f},
\end{equation}
%
where we used part (b) in Corollary~\ref{cor:du_F_inf_sup_solvability}, and
the approximation \eqref{eq:ass_bound_H1kappa} in Assumption~\ref{ass:FEM_space}
gives the claim.
\end{proof}




The proof of the next lemma, which states the discrete inf-sup stability,
is inspired by the thesis \cite[Prop.~8.2.7]{Melenk_Diss95}, where this was done for the Helmholtz equation.

\begin{lemma} \label{lem:du_F_inf_sup_solvability_discrete}
	%
	\bulletpoint{a} If $\kappa \Csol \h$ is sufficiently small, it holds
	 for all $\testfunTHREE_\h \in \VShperp$
	%
	\begin{equation}
		\Honekappa{\testfunTHREE_\h} \lesssim \, \Csol  \sup_{\testfunh \in \VShperp} \frac{\dualp{\energy''(\sol)  \testfunTHREE_\h}{\testfunh} } {\Honekappa{\testfunh}},
	\end{equation}
	%
	where the constant is  independent of $\h$ and $\kappa$.
	
	%
	\bulletpoint{b} For any $f \in (\Honeperp)'$, 
	there is a unique $\testfunTHREE_\h \in \VShperp$ such that
	%
	\begin{equation}
		\dualp{\energy''(\sol)  \testfunTHREE_\h}{\testfunh}%
		=
		%
		\dualp{f}{\testfunh} ,
		\quad \text{ for all } \testfunh \in \VShperp
	\end{equation}
	%
	and it holds
	%
	\begin{equation}
		\Honekappa{\testfunTHREE_\h} \lesssim \Csol \Honekappaminus{f} .
	\end{equation}
\end{lemma}


\begin{proof}
Part \bulletpoint{b} is a classical stability bound for inf-sup stable problems, cf. \cite[Thm.~2.1]{Bab7071}. Hence, claim \bulletpoint{b} directly follows once we have shown \bulletpoint{a}. To do so,
we fix $\testfunTHREE_\h \in \VShperp$ and observe for arbitrary $\testfunFOUR \in \Honeperp$
%
\begin{align}
\label{proof:lem5.3:step1}
	\dualp{\energy''(\sol)  \testfunTHREE_\h}{\testfunTHREE_\h + \testfunFOUR } &= \abilmag{\testfunTHREE_\h}{\testfunTHREE_\h}
	+
	\kappa^2 \Real \int_\Omega \bigl( 2 |\sol|^2 -1 \bigr)  \testfunTHREE_\h \testfunTHREE_\h^*  + u^2 \testfunTHREE_\h^* \testfunTHREE_\h^*  
	\,dx
	+
	\dualp{\energy''(\sol)  \testfunFOUR}{\testfunTHREE_\h}  .
\end{align}
%
Now let $\testfunFOUR \in \Honeperp$ be the solution to
	%
	\begin{align}
	\dualp{\energy''(\sol)  \testfunFOUR}{\testfun}   = 	\ipsymLtwo{f}{\testfun} \quad \text{ for all } \testfun \in \Honeperp
	 \qquad \text{with} \quad  f = ( \stabPar^2 + 2 \kappa^2 ) \testfunTHREE_\h 
\end{align}
and insert it into \eqref{proof:lem5.3:step1}.
%
Then, we obtain from \eqref{eq:def_abilmagstabsym} together with Lemma~\ref{lem:prop_bil} and Lemma~\ref{lem:Frechet_functional} that
%
\begin{align}
	\Honekappa{\testfunTHREE_\h}^2 &\lesssim \dualp{\energy''(\sol)  \testfunTHREE_\h}{\testfunTHREE_\h + \testfunFOUR }
	\lesssim
	\dualp{\energy''(\sol)  \testfunTHREE_\h}{\testfunTHREE_\h +  \projLtwoisol \testfunFOUR }
	+ 
	\Honekappa{\testfunTHREE_\h} \Honekappa{ \projLtwoisol \testfunFOUR - \testfunFOUR }
	\\
	%
	&\lesssim 
	\sup_{\testfunh \in \VShperp} \frac{\dualp{\energy''(\sol)  \testfunTHREE_\h}{\testfunh} } {\Honekappa{\testfunh}} 
	\Honekappa{\testfunTHREE_\h + \projLtwoisol \testfunFOUR}
	+
	\Honekappa{\testfunTHREE_\h} \Honekappa{ \projLtwoisol \testfunFOUR - \testfunFOUR }.
\end{align}
%
It remains to study the terms with $\testfunFOUR$. Here, 
we establish with Corollary~\ref{cor:du_F_inf_sup_solvability}
and \eqref{eq:Honekappa_bound_alternative} the bound
%
\begin{equation}
\kappa 	\Honekappa{\testfunFOUR} 
+
\h^{-1} 	\Honekappa{\testfunFOUR - \projLtwoisol \testfunFOUR} 
 \lesssim \Csol \norm{L^2}{f} \lesssim \kappa \Csol \Honekappa{\testfunTHREE_\h}.
\end{equation}
%
From this, we finally conclude 
%
\begin{align}
	\Honekappa{\testfunTHREE_\h}^2 &\lesssim 
	\sup_{\testfunh \in \VShperp} \frac{\dualp{\energy''(\sol)  \testfunTHREE_\h}{\testfunh} } {\Honekappa{\testfunh}} 
	 \Csol \Honekappa{\testfunTHREE_\h}
	+
	\kappa \Csol \h \Honekappa{\testfunTHREE_\h}^2 ,
\end{align}
%
and obtain the assertion \bulletpoint{a} if $\kappa \Csol \h$ is sufficiently small by absorption.
\end{proof}

\subsection{Convergence with rates}
 
After these preparations, we can derive the error equation employing the second \Frechet derivative $\energy''$. To this end, we strive for a representation of the form
%
\begin{align} \label{eq:error_equation}
	\dualp{\energy''(\sol)  ( \projLtwoisol   \sol -\solh)}{\testfunh}	 &= 
	\errHmone (\testfunh) ,
\end{align}
%
for $\testfunh \in \VShperp$ and employ Lemma~\ref{lem:du_F_inf_sup_solvability_discrete} to conclude a bound for $\projLtwoisol \sol -\solh$. The right-hand side $\errHmone$ is studied in the following lemma.



\begin{lemma} \label{lem:rep_error}

Let $\sol$ and $\solh$  be minimizers of \eqref{eq:energy_functional_times_kappa2}
and \eqref{eq:energy_functional_discrete}, respectively. 
	
\bulletpoint{a}
For $\testfunh \in \VShperp$ it holds the representation \eqref{eq:error_equation}
%
where
$\errHmone = \errHmonelin + \errHmonenonlin$ and
%
	%
	\begin{align} \label{eq:rep_error_Lagrange}
%	\begin{aligned}
		%
				\errHmonelin (\testfunh) &=
		\kappa^2 \Real \int_\Omega 
		\Bigl( 
		(|\sol|^2 - 1) ( \projLtwoisol   \sol -\sol) + \sol^2 ( \projLtwoisol   \sol -\sol)^*+ |\sol|^2 ( \projLtwoisol   \sol -\sol)  \Bigr) \testfunh^* \,dx
		\\
		&\qquad - \stabPar^2 \Real \int_\Omega 
		( \projLtwoisol   \sol -\sol)  \testfunh^* \,dx ,
		\\
		\errHmonenonlin (\testfunh )&= 2 \kappa^2
		\Real  \int_\Omega |\sol|^2 \sol \testfunh^* 	\,dx
		+
		\kappa^2
		\Real  \int_\Omega |\solh|^2 \solh \testfunh^* 	\,dx
		-
		\kappa^2
		\Real  \int_\Omega
		2 \bigl(  |\sol|^2 \solh + \sol^2 \solh^* \bigr)  \testfunh^*  \,dx .
	\end{align}


\bulletpoint{b}	The error terms are bounded by
		%
\begin{align}
\Honekappaminus{\errHmonelin}   &\lesssim
\kappa \norm{L^2}{\sol -  \projLtwoisol   \sol},
\\
\Honekappaminus{\errHmonenonlin} &\lesssim \kappa   \bigl( \norm{L^{4}}{\sol - \solh}^2   
+ \norm{L^{6}}{\sol - \solh}^3   \bigr)  ,
\end{align}
	%
where the constants are independent of $\h$ and $\kappa$.
\end{lemma}


\begin{proof}
Inserting the exact solution $\sol$, we decompose $\errHmone$ as
%
\begin{align} 
	\errHmonelin (\testfunh)  = \dualp{\energy''(\sol)  ( \projLtwoisol   \sol -\sol)}{\testfunh},
	\qquad
	%
	\errHmonenonlin (\testfunh)  = \dualp{\energy''(\sol)  ( \sol - \solh)}{\testfunh},
\end{align}
%
and treat the two terms separately. We begin with the linear part and use Lemma~\ref{lem:Frechet_functional}, the definition of $\abilmagstabsym{\cdot}{\cdot}$ in \eqref{eq:def_abilmagstabsym}, and the orthogonality condition of $\projLtwoisol$ to obtain
%
\begin{eqnarray*}
\lefteqn{ \dualp{\energy''(\sol)  ( \projLtwoisol   \sol -\sol)}{\testfunh}}
	\\
	&=&\abilmag{ \projLtwoisol   \sol -\sol}{\testfunh}
	%
	+ \kappa^2 \Real \int_\Omega 
	\Bigl( 
	(|\sol|^2 - 1) ( \projLtwoisol   \sol -\sol) + \sol^2 ( \projLtwoisol   \sol -\sol)^*+ |\sol|^2 ( \projLtwoisol   \sol -\sol)  \Bigr) \testfunh^* \,dx
	\\
	&=&- \stabPar^2 \Real \int_\Omega 
	( \projLtwoisol   \sol -\sol)  \testfunh^* \,dx
	\\
	&\enspace&+  \kappa^2 \Real \int_\Omega 
	\Bigl( 
	(|\sol|^2 - 1) ( \projLtwoisol   \sol -\sol) + \sol^2 ( \projLtwoisol   \sol -\sol)^*+ |\sol|^2 ( \projLtwoisol   \sol -\sol)  \Bigr) \testfunh^* \,dx.
\end{eqnarray*}
%
Using that $\kappa \norm{L^2}{\testfunh} \leq \Honekappa{\testfunh}$  gives the first estimate in part \bulletpoint{b}. 

For the nonlinear part, we note with Lemma~\ref{lem:Frechet_functional} the identity for $\testfunTWO,\testfun\in H^1$
%
\begin{equation}
\dualp{	\energy''(\testfunTWO) \testfunTWO}{\testfun}= \dualp{	\energy'(\testfunTWO)}{\testfun} + 2 \kappa^2 \Real \int_{\Omega}  |\testfunTWO|^2 \testfunTWO \testfun^*  \,dx .
\end{equation}
%
Since $\dualp{	\energy'(\sol)}{ \testfunh } = \dualp{	\energy'(\solh)}{ \testfunh } = 0 $, we expand 
%
	\begin{align}
	\dualp{	\energy''(\sol)  (\sol - \solh)}{\testfunh}	
	%		\\
	&=
	\dualp{	\energy''(\sol)  \sol }{ \testfunh } 
	-
	\dualp{	\energy''(\solh)  \solh }{ \testfunh } 
	+
	\dualp{	\energy''(\solh)  \solh }{ \testfunh } 
	-
	\dualp{	\energy''(\sol)  \solh } { \testfunh } 
	\\
%	\\
	&= 
	2 \kappa^2
	\Real  \int_\Omega |\sol|^2 \sol \testfunh^* 	\,dx
	-
	2 \kappa^2
	\Real  \int_\Omega |\solh|^2 \solh \testfunh^* 	\,dx
	\\
	&\quad + \kappa^2
	\Real  \int_\Omega
	2 \bigl( |\solh|^2 - |\sol|^2 \bigr) \solh \testfunh^*  
	+ \bigl( \solh^2 - \sol^2 \bigr) \solh^* \testfunh^* 
	\,dx
	%
	\\
	&= 
	2 \kappa^2
	\Real  \int_\Omega |\sol|^2 \sol \testfunh^* 	\,dx
	+
	\kappa^2
	\Real  \int_\Omega |\solh|^2 \solh \testfunh^* 	\,dx
	\\
	&\quad -
	\kappa^2
	\Real  \int_\Omega
	 \bigl(  2 |\sol|^2 \solh + \sol^2 \solh^* \bigr)  \testfunh^*  \,dx ,
\end{align}
%
where we collected terms in the last step. For the estimate, we write
$\solh = \sol - \errh$
and compute
%
\begin{align} \label{eq:Taylor_nonlinear_errh}
	\ 2 |\sol|^2 \sol  +  |\solh|^2 \solh  -
	\bigl( 2 |\sol|^2 \solh + \sol^2 \solh^* \bigr) 
%
	&= 2  \sol |\errh|^2 + \errh^2 \sol^* - |\errh|^2 \errh  ,     
\end{align}
%
which together with $|\sol|\leq 1$ and the H{\"o}lder inequality gives the second bound.
\end{proof}

Now we have everything together to prove the first part of Theorem~\ref{thm:main}, i.e., the $\HonekappaSpace$-estimates for the discrete minimizers.

\begin{proposition} \label{prop:convergence_ahat_general}
	Let $\sol$ and $\solh$  be minimizers of \eqref{eq:energy_functional_times_kappa2}
	and \eqref{eq:energy_functional_discrete}, respectively, and assume the orthogonality  $\ipsymLtwo{\solh}{\ci \sol} = 0$.
	
	\bulletpoint{a}  We have for the fully discrete error
	%
	\begin{align}
		\Honekappa{ \sol - \solh}  &\lesssim  
		\Honekappa{\sol- \projLtwoisol \sol  }  
		+
		\kappa \, \CsolH
		\norm{L^2}{\sol- \projLtwoisol \sol  } 
		 + 
		\kappa \, \CsolH
		\bigl( \norm{L^4}{\sol - \solh }^2 
		+
		\norm{L^6}{\sol - \solh}^3 \bigr)
	\end{align}
	
	\bulletpoint{b}  For $\h$ sufficiently small, we have for the (unique) minimizer $\sol$ in  Proposition~\ref{prop:abstract_convergence}
	%
	\begin{align}
		\Honekappa{ \sol - \solh}  &\lesssim 
		\Honekappa{\sol- \projLtwoisol \sol  }  
		+
		\kappa \, \CsolH
		\norm{L^2}{\sol- \projLtwoisol \sol  } .
	\end{align}
\end{proposition}

Let us point out that \bulletpoint{a} holds for any minimizers $\sol$ and $\solh$. But to ensure that the higher order terms are indeed negligible, we need the a priori information from the abstract convergence result in Proposition~\ref{prop:abstract_convergence}.  

\begin{proof}[Proof of Proposition~\ref{prop:convergence_ahat_general}]
\bulletpoint{a} Using the triangle inequality, we obtain
%
\begin{align}
\Honekappa{ \sol - \solh}  &\lesssim 
\Honekappa{\sol-  \projLtwoisol   \sol  }  
+
\Honekappa{  \projLtwoisol   \sol - \solh},
\end{align}
%
and are left to bound the second term.  Lemmas~\ref{lem:du_F_inf_sup_solvability_discrete}
and \ref{lem:rep_error} then give
%
\begin{equation}
\Honekappa{  \projLtwoisol   \sol - \solh} 
\lesssim
 \Csol \Honekappaminus{\errHmone}
 \lesssim
\kappa \Csol  \bigl(   \norm{L^2}{\sol -  \projLtwoisol   \sol}
+
 \norm{L^{4}}{\sol - \solh}^2   
 + \norm{L^{6}}{\sol - \solh}^3    \bigr) ,
\end{equation}
%
and the bound is established.

\bulletpoint{b} With the convergence shown in Proposition~\ref{prop:abstract_convergence} for $\h$ sufficiently small, we can absorb the higher order terms, and obtain the claimed estimate for $\h \leq \h_0$.
\end{proof}

We can further show quadratic convergence in the $L^2$-norm for the discrete minimizers using an Aubin--Nitsche argument.


\begin{lemma} \label{lem:convergence_ahat_general_L2}
	%
	Let $\sol$ and $\solh$  be a minimizers of \eqref{eq:energy_functional_times_kappa2}
	and \eqref{eq:energy_functional_discrete}, respectively, and assume the orthogonality  $\ipsymLtwo{\solh}{\ci \sol} = 0$.
	 We have for the fully discrete error
	 %
	\begin{align}
		\norm{L^2}{ \sol - \solh } &\lesssim  \Csol   
	\h  \Honekappa{\sol - \solh} \\
	&\qquad + 
	\Csol  \kappa \norm{L^2}{\sol - \solh} \bigl(
	\norm{L^3}{\sol - \solh } 
	+
	\norm{L^6}{\sol - \solh}^2  
	\bigr) ,
	\end{align}
	%
	and hence for $\h$ sufficiently small, it holds
	for the (unique) minimizer $\sol$ in  Proposition~\ref{prop:abstract_convergence}
	%
	\begin{align}
		\norm{L^2}{\sol - \solh} 
		\lesssim 
	\Csol 	
	\h 
	\Honekappa{\sol-  \solh  } .
	\end{align}
\end{lemma}

\begin{proof}
	Recall the abbreviation  $\errh = \sol - \solh$, and let $\testfunFOUR \in \Honeperp$ be the solution of 
	%
	\begin{equation}
	\ipsymLtwo{\energy''(\sol)  \testfunFOUR }{\testfun} = \ipsymLtwo{ \errh }{\testfun} ,
	\end{equation}
	%
	and note that Corollary~\ref{cor:du_F_inf_sup_solvability} and
	\eqref{eq:Honekappa_bound_alternative}
	 give the estimate
	%
	\begin{equation} \label{eq:H2_Aubin_Nitsche}
	\kappa \Honekappa{\testfunFOUR} + \h^{-1} \Honekappa{\testfunFOUR - \projLtwoisol \testfunFOUR} \leq  \Csol  \norm{L^2}{ \errh }.
	\end{equation}
	%
	Using the symmetry of $\energy''$, we can decompose the error as
	%
	\begin{align}
	\norm{L^2}{ \errh  }^2 &= 
	\dualp{\energy''(\sol) \, \errh }{ \testfunFOUR - \projLtwoisol \testfunFOUR}
	+
	\dualp{\energy''(\sol) \, \errh }{ \projLtwoisol \testfunFOUR }
	%\\
	%
	%&
	= E^1 +	\errHmonenonlin (\projLtwoisol \testfunFOUR ) ,
	\end{align}
	%
	where $\errHmonenonlin$ is defined in Lemma~\ref{lem:rep_error}.
	%
	We estimate the first term with Lemma~\ref{lem:Frechet_functional} %\eqref{eq:ass_bound_H1kappa}, 
	and \eqref{eq:H2_Aubin_Nitsche} 
	%
	\begin{align}
	E^1 &\lesssim
	 \Honekappa{\errh} \Honekappa{\testfunFOUR - \projLtwoisol \testfunFOUR}
	\lesssim
	 \Csol   \h   \Honekappa{e} \norm{L^2}{ \errh } .
	\end{align}
	%
	For the second term, we use the representation of $\errHmonenonlin$ in
	Lemma~\ref{lem:rep_error} and
	\eqref{eq:Taylor_nonlinear_errh} 
	together with \eqref{eq:H2_Aubin_Nitsche} and the H{\"o}lder equation
	to obtain 
	%
	\begin{align}
	| \errHmonenonlin (\projLtwoisol \testfunFOUR ) | &\lesssim
	  \kappa^2  \norm{L^{2}}{\sol - \solh}  \bigl( \norm{L^{3}}{\sol - \solh}   
	+ \norm{L^{6}}{\sol - \solh}^2   \bigr)  \norm{H^1}{\projLtwoisol \testfunFOUR}
	\\
	&\lesssim 
	\kappa \Csol    \norm{L^{2}}{\sol - \solh}  
	\bigl( \norm{L^{3}}{\sol - \solh}  	+ \norm{L^{6}}{\sol - \solh}^2   \bigr) 
	  \norm{L^2}{\errh} ,
	\end{align}
%
where we used $\kappa \norm{H^1}{\projLtwoisol \testfunFOUR} \lesssim \kappa \Honekappa{\testfunFOUR} \lesssim \Csol \norm{L^2}{\errh}  $ 
in the last step.
%
Combining the two bounds and dividing by $\norm{L^2}{\errh}$ gives the desired estimate.
	%
\end{proof}

A similar trick gives the improved convergence of $\projLtwoisol$ in the $L^2$-norm.

\begin{lemma} \label{lem:proj_L2}
	For $\kappa \h$ small enough, the following bound holds for all $\testfunTHREE \in \Honeperp$
		%
			\begin{equation}
					\norm{L^2}{ \testfunTHREE - \projLtwoisol (\testfunTHREE) } 
					\lesssim \h 	\Honekappa{ \testfunTHREE - \projLtwoisol (\testfunTHREE) } ,
				\end{equation}
		%
		where the constant is independent of $\h$ and $\kappa$.
\end{lemma}

\begin{proof}
We use an Aubin--Nitsche argument and let $\testfunFOUR \in
% \Changes{ H^2 \cap }
  \Honeperp$ be the solution of
	%
	\begin{equation}
		\abilmagstabsym{\testfunFOUR}{\testfun} = \ipsymLtwo{ \testfunTHREE - \projLtwoisol \testfunTHREE}{\testfun}, \quad \text{ for all } \testfun \in \Honeperp .
	\end{equation} 
	%
	Using orthogonality, we have by \eqref{eq:ass_bound_H1kappa} that%{eq:Honekappa_bound_alternative}
	%
	\begin{equation}
	\norm{L^2}{\testfunTHREE - \projLtwoisol \testfunTHREE}^2
	=
	\abilmagstabsym{\testfunFOUR - \projLtwoisol \testfunFOUR}{\testfunTHREE - \projLtwoisol \testfunTHREE} 
	\lesssim
	\h \norm{L^2}{\testfunTHREE - \projLtwoisol \testfunTHREE}	\Honekappa{\testfunTHREE - \projLtwoisol \testfunTHREE},
\end{equation}
	%
	and the claim follows.
\end{proof}


Finally, we provide the error bounds for the energy which behaves in the lowest order as the square of the error in the $\HonekappaSpace$-norm.

\begin{lemma} \label{lem:convergence_energy}
		Let $\sol$ and $\solh$  be minimizers of \eqref{eq:energy_functional_times_kappa2}
	and \eqref{eq:energy_functional_discrete}, respectively.
The error in the energies is bounded by
%
	\begin{equation}
		0\leq 	\energy(\solh) - \energy(\sol) 
		\lesssim
		 \Honekappa{\sol -\solh}^2 \,
		 \bigl( 
		 1
		+
		\kappa^{1/2} \Honekappa{\sol -\solh}
		+
		\kappa \Honekappa{\sol -\solh}^2 \bigr).
	\end{equation}
\end{lemma}

We note that the powers of $\kappa$ can be improved in the case $d =2$, but since the leading order term does not change, we will not give any details here.

\begin{proof}[Proof of Lemma~\ref{lem:convergence_energy}]
Since $\VSh \subset \VS$, we have $\energy(\sol) \leq \energy(\solh)$, and thus the lower bound. In the next step, we derive the representation
%
\begin{equation}\label{eq:rep_energy}
\begin{aligned} 
	\energy(\solh) - \energy(\sol) 
	&=
	\frac12 \abilmag{\sol - \solh}{\sol - \solh}
	\\
	&\qquad +
	\frac{\kappa^2}{4} \Real \int_\Omega
	(1-|\solh|^2)^2 -	(1-|\sol|^2)^2
	+
	4 (|\sol|^2 -1 ) \sol (\sol - \solh)^* \,dx¸
\end{aligned}
\end{equation}
	Let us first note the identity
%
\begin{equation}
	\frac12 \abilmag{\sol - \solh}{\sol - \solh} = 
	\frac12 \abilmag{\solh}{\solh} 
	-
	\frac12 \abilmag{\sol}{\sol} 
	+
	\abilmag{\sol}{\sol - \solh} ,
\end{equation}
%
and rewrite the energies as 
%
\begin{align}
	\energy(\solh) - \energy(\sol)  &= \frac12 \abilmag{\solh}{\solh} 
	-
	\frac12 \abilmag{\sol}{\sol} 
	+
	\frac{\kappa^2}{4}
	\Real \int_\Omega
	(1-|\solh|^2)^2 -	(1-|\sol|^2)^2 \,dx
	\\
	%
	&= \frac12 \abilmag{\sol - \solh}{\sol - \solh} 
	+
	\frac{\kappa^2}{4}
	\Real \int_\Omega
	(1-|\solh|^2)^2 -	(1-|\sol|^2)^2 \,dx
	-
	\abilmag{\sol}{\sol - \solh}.
\end{align}
%
Since $\sol$ is a minimizer, we have $\dualp{\energy'(\sol)}{\sol -\solh} = 0$
and thus by \eqref{eq:first_Frechet}
%
\begin{equation}
	- \abilmag{\sol}{\sol - \solh} =  \kappa^2 \Real \int_\Omega
	(|\sol|^2 -1 ) \sol (\sol - \solh)^* \,dx ,
\end{equation}
%
and hence \eqref{eq:rep_energy} holds. The first term of the representation gives the $\HonekappaSpace$-norm in the estimate,
%
and it remains to study  the nonlinear part. We first investigate the difference of the squares.
%
As before, we write 
$\solh = \sol - \errh$
and obtain
%
\begin{align}
	\bigl(1-|\sol - \errh|^2\bigr)^2  &= 
	\bigl( |\sol|^2 + |\errh|^2 - 1 - 2 \Real (\sol \errh^*)  \bigr)^2 
	\\
	%
	&= |\sol|^4 + 1 - 2 |\sol|^2 + 4 \Real (\sol \errh^*) - 4 |\sol|^2 \Real (\sol \errh^*)  
	%
	+ \mathcal{O}(|\errh|^2 + |\errh|^3 + |\errh|^4) ,
\end{align}
%
which gives
%
\begin{align}
	(1-|\solh|^2)^2 -	(1-|\sol|^2)^2 = 4 \Real (\sol \errh^*) - 4 |\sol|^2 \Real (\sol \errh^*)  
	%
	+ \mathcal{O}(|\errh|^2 + |\errh|^3 + |\errh|^4) .
\end{align}
%
We now show that the part, which is linear in $\errh$, is
canceled by the last term in \eqref{eq:rep_energy}.
In fact, since it holds
%
\begin{equation}
	4 \Real (|\sol|^2 -1 ) \sol (\sol - \solh)^*  =  
	4 |\sol|^2  \Real (\sol \errh^* ) 
	- 4 \Real (\sol \errh^*) ,
\end{equation}
%
we conclude from \eqref{eq:rep_energy}, the fact that $|\sol|\leq 1$ and the H{\"o}lder inequality the bound 
	\begin{equation}
	\energy(\solh) - \energy(\sol) 
	\lesssim \Honekappa{\sol -\solh}^2
	+
	\kappa^2
	\bigl( 
	\norm{L^2}{\sol-\solh}^2 +  \norm{L^3}{\sol-\solh}^3
	+
	\norm{L^4}{\sol-\solh}^4 \bigr).
\end{equation}
%
To show the final estimate, we use interpolation theory, see e.g., \cite[Thm.~2.6]{Lun18}, 
with $\tfrac{1}{3} = \tfrac{\theta}{2} + \tfrac{1-\theta}{6}$ for $\theta = \tfrac12$
to obtain for $\testfunTHREE \in H^1$
%
\begin{equation}
	\kappa^2 \norm{L^3}{\testfunTHREE}^3 \lesssim 
	\kappa^{1/2} \bigl( \kappa \norm{L^2}{\testfunTHREE} \bigr)^{3/2} \norm{L^6}{\testfunTHREE}^{3/2}
	\lesssim \kappa^{1/2} \Honekappa{\testfunTHREE}^3,
\end{equation}
%
and similarly with $\tfrac{1}{4} = \tfrac{\theta}{2} + \tfrac{1-\theta}{6}$ for $\theta = \tfrac14$
%
\begin{equation}
	\kappa^2 \norm{L^4}{\testfunTHREE}^4 \lesssim 
	\kappa \bigl( \kappa \norm{L^2}{\testfunTHREE} \bigr) \norm{L^6}{\testfunTHREE}^3
	\lesssim \kappa \Honekappa{\testfunTHREE}^4,
\end{equation}
%
and the second claim is established.
\end{proof}

We can finally give the proof of our main result.

\begin{proof}[Proof of Theorem~\ref{thm:main}]
We mainly collect the results shown in Proposition~\ref{prop:convergence_ahat_general}, Lemma~\ref{lem:convergence_ahat_general_L2}, 
together with the $L^2$-estimate in
Lemma~\ref{lem:proj_L2},
and Lemma~\ref{lem:convergence_energy},
and the claims are established.
\end{proof}


\subsection{Application to Lagrange finite elements}

In this section, we consider the linear Lagrange finite element space $\VSh$ as defined \eqref{eq:defVh}. In order to derive the corresponding error estimates through verifying the assumptions of Theorem~\ref{thm:main}, we require the $L^2$-orthogonal projection onto the ansatz space $\VSh$ as an auxiliary projection. We recall the $L^2$-projection 
	for $\testfunTWO \in L^2$ as
	%
	\begin{equation}
		\ipsymLtwo{\Ltwoproj \testfunTWO}{\testfunh} = \ipsymLtwo{ \testfunTWO}{\testfunh} \quad \mbox{for all} \quad \testfunh \in \VSh.
	\end{equation}
	%
In the following lemma, we provide corresponding estimates in the $\HonekappaSpace$-norm
which are the first step towards verifying part \bulletpoint{b} in Assumption~\ref{ass:FEM_space}.

\begin{lemma} \label{lem:proj}
	
	\bulletpoint{a} The $L^2$-projection $\Ltwoproj$ is stable in 
	$\HonekappaSpace$, i.e., there hold the bounds
	%
	\begin{align}
		\Honekappa{\Ltwoproj \testfun} &\lesssim \Honekappa{\testfun},
		\qquad
		 \testfun \in H^1,
	\end{align}
	where the constant is independent of $\h$ and $\kappa$.
	
	\bulletpoint{b} 	
	For all $\testfunFOUR\in H^2$ it holds
	%
	\begin{align}
		\Honekappa{\testfunFOUR - \Ltwoproj \testfunFOUR} 
		 &\lesssim \h	\Htwokappa{\testfunFOUR} ,
	\end{align}	
	%
	where the constant is independent of $\h$ and $\kappa$.
	
	\bulletpoint{c}  	
	If $\Omega$ is convex and $\testfunFOUR\in \Honeperp$ satisfies for $f \in L^2$ the equation
	$\abilmagstabsym{\testfunFOUR}{\testfun} = \ipsymLtwo{f}{\testfun}$ for all $\testfun \in \Honeperp$,
	then 
	%
	\begin{align}
		\Honekappa{\testfunFOUR - \Ltwoproj \testfunFOUR} 
		  &\lesssim \h	\norm{L^2}{f}.
	\end{align}	
\end{lemma}
%
\begin{proof}
	Due to \eqref{eq:H1_stab_L2_proj}, standard arguments lead to the bounds on the $L^2$-projection in part \bulletpoint{a} and \bulletpoint{b}.
	%
	Part \bulletpoint{c} is a direct consequence of part \bulletpoint{b} and Lemma~\ref{lem:wepo_abilmagstab}.
\end{proof}

In the next lemma, we relate the orthogonal projection, which takes into account the orthogonality to $\ci \sol$ in $\ipsymLtwo{\cdot}{\cdot}$, to the $L^2$-projection.

\begin{lemma} \label{lem:proj_Lagrange}
	For $\kappa \h$ small enough, it  holds the bound
	%
	\begin{equation}
		\Honekappa{ \testfun - \projLtwoisol (\testfun) } 
		\lesssim
%		  \min \{ 
%		   \Honekappa{ \testfun - \Ritzproj \testfun} ,
		    \Honekappa{  \testfun - \Ltwoproj \testfun } ,
		    \quad
		    \testfun \in \Honeperp.
%		   \}.
	\end{equation}
	%
\end{lemma}

\begin{proof}
	For $\testfunh \in \VSh$ we let $\Ltwotwist: \VSh \rightarrow \VShperp$
	be the mapping that adjusts the angle to $\ci \sol$ via
	%
	\begin{align}
		\Ltwotwist(\testfunh) \coloneqq 
		\testfunh - 
		\frac{  \ipsymLtwo{\testfunh}{\ci \sol}}{
			\ipsymLtwo{\Ltwoproj(\ci \sol)}{\ci \sol} }
		\Ltwoproj(\ci \sol).
	\end{align}
	%
	We this we obtain for any $\testfun \in \Honeperp$
	%
	\begin{align}
		\Honekappa{ \testfun - \projLtwoisol (\testfun) } 
		&\lesssim \Honekappa{  \testfun - (\Ltwotwist \circ \Ltwoproj  )\testfun } 
		\le  \Honekappa{  \testfun - \Ltwoproj \testfun } 
		+ 
		\frac{\ipsymLtwo{\Ltwoproj \testfun - \testfun}{\ci \sol}}{\ipsymLtwo{\Ltwoproj(\ci \sol)}{\ci \sol}}
		\Honekappa{ \Ltwoproj(\ci \sol) } 
		%
		\\
		&\lesssim \Honekappa{  \testfun - \Ltwoproj \testfun } 
		+ 
		\norm{L^2}{ \testfun - \Ltwoproj \testfun }
		\frac{ \| \sol \|_{L^2} }{
			\ipsymLtwo{\Ltwoproj(\ci \sol) - \ci \sol}{\ci \sol}
			+  \norm{L^2}{  \sol }^2  }  \Honekappa{ \sol } 
		\\
		%
		&\lesssim  \Honekappa{  \testfun - \Ltwoproj \testfun } + \frac{\kappa}{1 -  c \kappa \h} \norm{L^2}{  \testfun - \Ltwoproj \testfun }
		\lesssim \Honekappa{  \testfun - \Ltwoproj \testfun },
	\end{align}
%
	where we used in the last step that $\norm{L^2}{\Ltwoproj(\ci \sol) - \ci \sol} \lesssim h \norm{H^1}{\sol} \lesssim \kappa h$ holds. 
\end{proof}

These preparations lead to the error bounds for our first application.

\begin{proof}[Proof of Corollary~\ref{cor:main_Lagrange}]
From Lemmas~\ref{lem:proj} and \ref{lem:proj_Lagrange}, we obtain that Assumption~\ref{ass:FEM_space} holds,
and thus we can use the bounds in Theorem~\ref{thm:main}.
	In addition, we recall that $\Omega$ is assumed to be convex, and, hence, the approximation estimates
	due to
	Lemmas~\ref{lem:proj} and \ref{lem:proj_Lagrange} yield
	%
	\begin{equation}
	\Honekappa{\sol - \projLtwoisol \sol}
		\lesssim 
		\h 
		\Htwokappa{\sol}
		\lesssim \kappa^2 \h ,
	\end{equation}
	%
	where we used Theorem~\ref{thm:cont_minimizer} for the last step.
%
	This establishes the claims.
\end{proof}