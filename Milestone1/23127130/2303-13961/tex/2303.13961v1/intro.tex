
\section{Introduction}

Superconductors are materials that allow to conduct electricity without any electrical resistance.
Letting $\Omega \subset \R^d$,  $d=2,3$, denote a
bounded polyhedral Lipschitz domain occupied by a superconducting material,  
the superconductivity in $\Omega$ can be modeled by a complex-valued wave
function $\sol : \Omega \rightarrow \mathbb{C}$ which is called the order parameter.  
The physical quantity of interest is $|\sol|^2$ which denotes the density of the superconducting electron pairs, where in the appropriate scaling, it holds $0 \leq |\sol|^2 \leq 1$.
%
This means that the material is not superconducting (in normal state) in $x\in\Omega$ if 
$|\sol(x)|^2  = 0$ 
and behaves like a perfect superconductor if $|\sol(x)|^2  = 1$. In between, different degrees
of superconductivity are possible. Of particular interest are mixed normal-superconducting states where both phases coexist in a lattice of quantized vortices \cite{Abr04}. %lattice structure
%

Mathematically, the order parameter can be characterized as a
%
minimizer of the \GL energy (or Gibbs free energy) given by
%
\begin{equation} \label{eq:energy_functional_times_kappa2}
	\energy(\sol) = \frac{1}{2} \int_\Omega |\nabla \sol + \ci  \kappa \MagF \sol |^2 + \frac{\kappa^2}{2} \bigl( 1- |\sol|^2 \bigr)^2\,dx,
\end{equation}
%
where $\MagF\colon \Omega \to \R^d$ is a real-valued magnetic potential and $\kappa$ is the so-called \GL parameter, a material parameter that correlates with the temperature and determines the type of superconductor. By the necessary condition for local extrema, any minimizer $\sol  \in H^1(\Omega)$ must fulfill the condition $E^{\prime}(\sol)=0$, which is known as the \GL equation (GLE) and reads written out (cf. \cite{DuGP92})
%
\begin{align}
\label{eq:ginzburg-landau-eqn}
	\Real  \int_\Omega \bigl( \nabla \sol + \ci \kappa \MagF \sol \bigr)  \cdot \bigl( \nabla \testfun + \ci \kappa \MagF \testfun \bigr)^*  
	+
	\kappa^2 \bigl( |\sol|^2 -1 \bigr)  \sol \testfun^* 
	\,dx = 0 \qquad \mbox{for } \testfun \in H^1(\Omega).
\end{align}
%
The real-valued magnetic potential $\MagF\colon \Omega \to \R^d$ in the GLE is typically unknown and can be inferred from an external magnetic field $H$ through the condition $H = \curl \MagF$ which is then added as a penalty term to the energy. In this work we consider the simplifying case that $\MagF$ is given, where the focus of our analysis is rather the influence of $\kappa$ on the accuracy of numerical approximations. In fact, the size of the parameter $\kappa$ is crucial for the appearance of vortices \cite{Ser99,SaS07,SeS10,SaS12}. On the one hand, if $\kappa$ is too small, no vortices will appear. On the other hand, the larger the value of $\kappa$, the more vortices appear in the lattice and the more point-like they become \cite{Aftalion99,SaS07}. The so-called high-$\kappa$ regime is hence the physically most interesting regime, but numerically it is also the most challenging one because it requires fine meshes to resolve all lattice structures. This raises an important practical question: how fine do we have to select the mesh size relative to the size of $\kappa$ so that the numerical approximations capture the correct vortex pattern?
%
%
Motivated by this question, the main goal of this work is to derive rigorous error bounds for the discrete minimizers with constants that are explicit and optimal in the spatial parameter $\h$ and the \GL parameter $\kappa$.

The first work where the approximation properties of discrete solutions to the stationary GLE were analyzed is the seminal SIAM Review article by Du, Gunzburger and Peterson \cite{DuGP92} (see also \cite{DuGP93} for periodic boundary conditions). The paper considers $H^1$-error estimates in finite element (FE) spaces for both the order parameter $u$ and the magnetic potential $\MagF$. The proof technique considers fixed (compact) intervals of $\kappa$-values and does not trace all $\kappa$-dependencies that enter through the size of these intervals and through uniform bounds for certain operator norms (that are linked to the chosen interval). The proof also considers a modified setting where $\energy^{\prime\prime}(u)$ is assumed to have a trivial kernel. However, the solutions to the GLE \eqref{eq:ginzburg-landau-eqn} are known to be only locally unique up to gauge transformations \cite{DuGP92}. In our case, these transformations are of the form $\sol \mapsto e^{\ci \theta} \sol$ for any $\theta\in\mathbb{R}$. In fact, it is easily seen that $\energy(\sol)=\energy(e^{\ci \theta} \sol)$ for all such $\theta$, which hence leads to a cluster of (qualitatively equivalent) solutions $e^{\ci \theta} \sol$. In turn, we have $\dualp{\energy^{\prime\prime}(\sol) \, e^{\ci \pi/2} \sol}{ \cdot } = 0$ which shows that $\energy^{\prime\prime}(\sol)$ can become singular. Hence, it makes sense to revisit the results \cite{DuGP92} with new proof techniques that allow us to follow all $\kappa$-dependencies and which allow us to avoid an assumption of local uniqueness. To the best of our knowledge, there are only two other works that address the convergence of discrete solutions to the stationary GLE: In \cite{DuNicolaidesWu98} a spatial discretization based on a covolume method is suggested, and in \cite{QuJu05} a finite volume discretization is used to solve the GLE on the sphere. In both papers the convergence of a subsequence of discrete solutions to a continuous minimizer is established, however without rates in $\h$ and $\kappa$.

With the goal to close this gap in the literature for finite element discretizations, our error analysis is performed in a general framework of FE methods, where we state our results under natural assumptions on the discrete spaces.
%
We first establish bounds on the discrete minimizers which are explicit in $\kappa$ and independent of $\h$.
This enables us to provide an abstract convergence result which identifies a suitable, continuous minimizer
of \eqref{eq:energy_functional_times_kappa2}. This a priori information is crucial in the derivation of the error bounds.
%
In order to exploit the structure of the problem, we have to study the properties of the second \Frechet derivative
of the energy $\energy$.
%
In particular, we carry over the inf-sup stability to our discrete setting under a smallness condition related to the product $\kappa \h$.
Let us emphasize that this is not a technical issue, but is indeed observed in our numerical experiments.
%
We employ a problem adapted scalar product and its Ritz projection, which captures the
one-dimensional kernel of $\energy''$, to extract optimal error bounds not only for the $H^1$-norm, but also
new error bounds
for the $L^2$-norm and the energy. Our numerical experiments confirm that the predicted scaling of the
error in $\kappa$ and $\h$ is asymptotically sharp.

It is worth to mention that, aside from stationary Ginzburg--Lindau equations, there has been a lot of work on the numerical analysis of the time-dependent problem that describes the dynamics of superconductors, where we exemplarily refer to \cite{Chen97,CheD01,Du94,Du94b,Du97,DuGray96,DuanZhang22,GJX19,GaoS18,Li17,LiZ17,LiZ15} and the references therein. For works with a particular emphasis on tracing the influence of $\kappa$ in the estimates, we refer to \cite{Bar05,Bartels2005,BarMO11} for the case of vanishing vector potentials $\MagF$. Due to the different nature of the time-dependent problem, we will not discuss the equation any further here. 
%
%
\smallskip

The rest of the paper is organized as follows:
%
In Section~\ref{sec:framework}, we introduce the analytical framework and present some results on continuous minimizers of \eqref{eq:energy_functional_times_kappa2}. In particular, we discuss the assumptions concerning uniqueness of minimizers.
%
For an abstract finite element space discretization, we present in Section~\ref{sec:main} our main results on the existence, boundedness, and approximation of discrete minimizers. An application to linear Lagrange finite elements is also given.
%
Numerical experiments which illustrate our theoretical findings
and confirm the convergence rates as well as the $\kappa$-dependency of our bounds are shown in Section~\ref{sec:num_exp}. 
%
The proofs of our main results are given in Section~\ref{sec:proofs}. Finally, in Section~\ref{sec:lod} we present a nonstandard application of the abstract result to spaces based on the Localized Orthogonal Decomposition.

\subsection*{Notation} For a complex number $z \in \mathbb{C}$, we use $z^*$ for the complex conjugate of $z$. In the whole paper we further denote by $L^2:=L^2(\Omega,\mathbb{C})$ the Hilbert space of $L^2$-integratable complex functions, but equipped with the {\it real} scalar product $\ipsymLtwo{\sol}{v} :=\Real \int_{\Omega} v \, w^* \,dx$ for $v,w\in L^2$. Hence, we interpret the space as a {\it real} Hilbert space. Analogously, we equip the space $H^1:=H^1(\Omega,\mathbb{C})$ with the scalar product $\ipsymLtwo{v}{w}+\ipsymLtwo{\nabla v}{\nabla w}$. This interpretation is crucial so that the Fr\'echet derivatives of $E$ are meaningful and exist on $H^1$. For any space $X$, we denote its dual space by $X'$. Note that this implies, that the elements of the dual space of $H^1$ consist of real-linear functionals, which are not necessarily complex-linear. For example, if $F(v):=m(f,v)$ for some $f \in L^2$, then it holds $F(\alpha \, v)=\alpha\, F(v)$ if $\alpha \in \mathbb{R}$, but in general {\it not} if $\alpha \in \mathbb{C}$.



In the following $C$ will denote a generic constant which is independent of $\kappa$ and the spatial mesh parameter $h$, but might depend on numerical constants as well as $\Omega$ and $\MagF$. 
In particular, we will write $\alpha \lesssim \beta$ if there is a constant $C$ independent of $\kappa$ and $h$ such that $\alpha \leq C \,\beta $.
 %

