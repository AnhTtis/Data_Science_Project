\section{Analytical framework}
\label{sec:framework}

In this section, we present several results concerning the continuous minimizers of \eqref{eq:energy_functional_times_kappa2}.


From now on, we assume that the magnetic potential $\MagF$ satisfies 
%
\begin{equation} \label{eq:ass_\MagF_for_H2}
	\MagF \in L^\infty(\Omega,\mathbb{R}^d),
	\qquad
	\div \MagF = 0 \text{ in } \Omega,
	\qquad 
	\MagF \cdot \nu =0  \text{ on } \partial \Omega.
\end{equation}
%
The above assumption can be rigorously justified on convex and on smooth domains $\Omega$, cf. \cite{DuGP92}. However, we also note that the $L^\infty$-regularity of $A$ might not be available on general complex geometries with re-entrant corners.
 
Further, we introduce the dual pairing $\dualp{\sol }{ \testfun } \coloneqq \dualp{\sol }{ \testfun }_{\dualHone, H^1}$, and the bilinear forms
given by
%
	\begin{align} 
				\ipsymLtwo{\sol}{\testfun} \coloneqq \Real \int_{\Omega} \sol \testfun^* \,dx ,
				%
				\qquad
				%
		\abilmag{\sol}{\testfun} \coloneqq  \Real \int_\Omega \bigl( \nabla \sol + \ci \kappa \MagF \sol \bigr) \cdot  \bigl( \nabla \testfun + \ci \kappa \MagF \testfun \bigr)^*  \,dx ,
		\label{eq:def_forms_abil}
	\end{align}
%
as well as the norm $\norm{H^1}{\testfun}^2 \coloneqq \norm{L^2}{\nabla \testfun }^2 + \norm{L^2}{ \testfun }^2$, the scaled norms 
%
\begin{equation} \label{eq:def_norms}
	\Honekappa{\testfun}^2 \coloneqq \norm{L^2}{\nabla \testfun }^2 + \kappa^2 \norm{L^2}{ \testfun }^2, 
	\qquad
	\Htwokappa{\testfun}^2 \coloneqq \norm{H^2}{\testfun}^2 + \kappa^2 \Honekappa{\testfun}^2,
\end{equation}
%
and the induced norm $\Honekappaminus{f} = \sup_{\testfun \in H^1} \frac{f(\testfun)}{\Honekappa{\testfun}}$.
%
We abbreviate $\MagFinfty = \norm{L^\infty}{\MagF}$,
and define the stabilized inner product on $\VS = H^1(\Omega)$ for $\sol,\testfun\in \VS$ by
%
\begin{equation} \label{eq:def_abilmagstabsym}
	\abilmagstabsym{\sol}{\testfun}  \coloneqq \abilmag{\sol}{\testfun} + \stabPar^2 \ipsymLtwo{\sol}{\testfun}_{L^2} ,
	\quad \text{with } \stabPar^2 = \kappa^2 (\MagFinfty^2 +1).
\end{equation}
%
We call it stabilized since this modification enables us to show boundedness and coercicvity of $\abilmagstabsym{\cdot}{\cdot}$ with respect to the
$\HonekappaSpace$-norm defined in \eqref{eq:def_norms}.

\begin{lemma}
	\label{lem:prop_bil}
	There are $\kappa$-independent constants $\Cbnd,\Ccoer>0$ such that
	for all $\testfunTWO,\testfun\in \VS$
	\begin{align}
		\abilmagstabsym{\testfunTWO}{\testfun} &\leq \Cbnd \, \Honekappa{\testfunTWO} \Honekappa{\testfun},
		\qquad
		\text{and}
		\qquad
		%
		\abilmagstabsym{\testfun}{\testfun} \geq  \Ccoer \, \Honekappa{\testfun}^2.
	\end{align}
	%
\end{lemma}

\begin{proof}
The boundedness is a straightforward application of the Cauchy-Schwarz inequality. For the coercivity, we note that by Young's inequality it holds
%
\begin{equation}
|\nabla \testfun + \ci \kappa \MagF \testfun|^2 
\geq
 |\nabla \testfun|^2 - 2   |\nabla \testfun| |\kappa \MagF \testfun| + |\kappa \MagF \testfun|^2
  \geq 
  \frac12 |\nabla \testfun|^2 - \kappa^2 \MagFinfty^2 |\testfun|^2.
\end{equation}
%
By the choice of $\beta$, we conclude the lower bound.
\end{proof}

A straightforward calculation shows that the
energy is  (real-)\Frechet differentiable and satisfies for all $\testfun\in H^1$
%
\begin{align} \label{eq:first_Frechet}
	\dualp{\energy'(\sol)}{\testfun}
	&= 
	\Real  \int_\Omega \bigl( \nabla \sol + \ci \kappa \MagF \sol \bigr)  \cdot \bigl( \nabla \testfun + \ci \kappa \MagF \testfun \bigr)^*  
	+
	\kappa^2 \bigl( |\sol|^2 -1 \bigr)  \sol \testfun^* 
	\,dx .
\end{align}
%
In particular any minimizer $\sol \in H^1$ satisfies $\energy'(\sol) = 0$.
Our first result collects the existence of a minimizer $\sol$ and its properties. 

\begin{theorem} \label{thm:cont_minimizer}
	For every $\kappa \geq 0$ there exists a minimizer $\sol \in H^1$ of  \eqref{eq:energy_functional_times_kappa2}.
	%
	Further, any minimizer fulfills
	%
	\begin{align}
	|\sol(x)| \leq 1 \,\mbox{ for all } x\in \Omega ,
	 \qquad 
	 \Honekappa{\sol}\lesssim \kappa , 
	  \quad 
	 \mbox{and if $\Omega$ is convex then $\sol \in H^2$ and} 
	 \quad
	 \Htwokappa{\sol} \lesssim \kappa^2,
	\end{align}
%
where the hidden constants in the above estimates are independent of $\kappa$ and $\sol$.
\end{theorem}

\begin{proof}
	%
	First note that the energy $\energy$ is continuous in $H^1(\Omega)$, and further weakly lower semi-continuous, see e.g., \cite[Thm.~1.6]{Struwe08}. In addition, a simple calculation shows
	%
	\begin{equation}
		\energy(\sol) = \abilmagstabsym{\sol}{\sol} 
		+
		\frac{\kappa^2}{2}  \int_\Omega
		  \bigl( 1 + \frac{2 \stabPar^2}{\kappa^2} - |\sol|^2 \bigr)^2  
		  +  
		  1
		  -
		  \bigl( 1 + \frac{2 \stabPar^2}{\kappa^2} \bigr)^2 \,dx,
	\end{equation}
	%
	and hence $\energy(\sol) \to \infty$ as $\Honekappa{\sol} \to \infty$.
	The standard arguments then imply the existence of a minimizer, see e.g., \cite[Thm.~1.2]{Struwe08}.
	%
	For the pointwise bound, we refer to \cite[Prop.~3.11]{DuGP92}, which implies a bound in $L^2$ independent of $\kappa$.  We further have 
	%
	\begin{equation}
		\norm{L^2}{\nabla \sol} \leq \norm{L^2}{\nabla \sol + \ci \kappa \MagF \sol} + \kappa \MagFinfty \norm{L^2}{\sol} \lesssim \energy(0)^{1/2} + \kappa  \lesssim \kappa .
	\end{equation}
	%
	Since $\energy'(\sol) = 0  $, we rearrange to
	%
	\begin{align}
	\abilmag{\sol}{\testfun} =
	- \kappa^2 \Real \int_\Omega \bigl( |\sol|^2 -1 \bigr)  \sol \testfun^* 
	\,dx
	%
	= \ipsymLtwo{ f }{\testfun} 
\end{align}
	%
	with $\norm{L^2}{f} \lesssim  \kappa^2 $, and obtain with \eqref{eq:ass_\MagF_for_H2}
	%
	\begin{equation} 
	\Real \int_{\Omega}
	\nabla \sol \cdot \nabla \testfun^* 
	\,dx
	= \ipsymLtwo{ f }{\testfun} 
%
	-
	\Real \int_{\Omega}
	\bigl(
	- 2 \ci \kappa \MagF \cdot \nabla \sol 
	+ \kappa^2 |\MagF|^2 \sol  \bigr) \testfun^*
	\,dx .
\end{equation}
	%
	If $\Omega$ is convex, standard elliptic regularity theory (cf. \cite{GiT01}) gives us
	\begin{equation}
		\norm{H^2}{\sol} \lesssim  \norm{L^2}{f} + \kappa^2 \norm{L^2}{\sol} + \kappa \norm{L^2}{\nabla \sol} 
		\lesssim \kappa^2,
	\end{equation}
	%
	where we used the $L^2$- and $H^1$-bounds for $\sol$ in the last step.
\end{proof}

Since $u$ is a global minimizer of the energy $\energy$, it must not only hold $\langle E^{\prime}(u) , \testfun\rangle =0$ but also $\langle E^{\prime\prime}(u) \testfun , \testfun \rangle \ge 0$ for all $\testfun\in H^1$. Later we will make use of these conditions. For that we require a corresponding representation of the second \Frechet derivative of $\energy$. This and its properties are summarized in the following lemma.

\begin{lemma} \label{lem:Frechet_functional}
\bulletpoint{a}	The energy is twice (real-)\Frechet differentiable and satisfies for $\testfun,\testfunTWO \in H^1$
	\begin{align}
	\dualp{\energy''(\sol) \testfunTWO}{\testfun}	 &=  
	%
	\Real  \int_\Omega \bigl( \nabla \testfunTWO + \ci \kappa \MagF \testfunTWO \bigr)  \cdot \bigl( \nabla \testfun + \ci \kappa \MagF \testfun \bigr)^*   
	+
	\kappa^2 
	\bigl(  ( |\sol|^2 -1  ) \testfunTWO \testfun^*  + \sol^2 \testfunTWO^* \testfun^* + |\sol|^2 \testfunTWO \testfun^* \bigr)
	\,dx .
	%
\end{align}
%

\bulletpoint{b}	
For  $\testfun,\testfunTWO\in H^1$ it holds
%
\begin{equation} \label{eq:E_prime_prime_sym}
	\dualp{\energy''(\sol) \testfunTWO}{\testfun}	 = \dualp{\energy''(\sol) \testfun}{\testfunTWO}	
	\quad
	\text{and}
	\quad
	|  \dualp{\energy''(\sol)  \testfunTWO}{\testfun}  | 
	\lesssim  \Honekappa{\testfunTWO} \Honekappa{\testfun} .
\end{equation}
%
\end{lemma} 

\begin{proof}
The \Frechet derivative is computed in a straightforward manner, and the symmetry follows from the representation by noting the real part in front of the integral. For the bound, we employ Lemma~\ref{lem:prop_bil} as well as $|\sol| \leq 1$.
\end{proof}

As explained in the introduction, a minimizer of \eqref{eq:energy_functional_times_kappa2} cannot be unique due to the invariance of the energy under complex rotations, i.e. if $\sol$ is a minimizer, then $ \exp( -\ci \phi) \sol$ is also a minimizer for any $\phi \in \R$. This property is known as gauge invariance and the mapping $\sol \mapsto \exp( -\ci \phi) \sol$ is called a gauge transformation. On polygonal domains, minimizers are believed to be locally unique, that means, that in a sufficiently small environment of $\sol$, the only other minimizers are exactly the ones obtained by gauge transformations. However, a general proof for this local uniqueness property is one of the great challenges of the field and has not yet been established. For partial results in various important settings, we refer to \cite{PacRiv00,Riv2002,SaS07,WeiWu2020} and the references therein.

In the following, we hence make the local uniqueness an (reasonable) assumption for our analysis. Furthermore, we later describe how to check the validity of the assumption numerically for any given setting so that we explicitly know if it is fulfilled. 

In the definition below we express the local uniqueness by curves passing through a minimizer. To be precise, we look at the energy level $\ell: t \mapsto E(\, \gamma(t) \, )$ for a smooth curve $\gamma : \mathbb{R} \rightarrow H^1$ with $\gamma(0)=\sol$. If $\sol$ is a minimizer of $E$, then $\ell(t)$ has a local minimum at $t=0$ (i.e. $\ell^{\prime}(0)=0$ and $\ell^{\prime\prime}(0)\ge0$). Furthermore, $\sol$ is locally unique up to gauge, if $\ell^{\prime\prime}(0)>0$ for all directions $\gamma^{\prime}(0)$ that are not parallel to $\ci \sol$. Note that the direction $\ci \sol$ is the tangent in $\sol$ on the circle line $t \mapsto \exp( -\ci t) \sol$, i.e., the direction in which the value of the energy does not change, cf. Figure \ref{illustration-tangent}. Since the energy is constant on the circle line, we naturally have $\ell^{\prime\prime}(0)=0$ in this direction. The following definition formalizes this type of local uniqueness. 

\begin{figure}
\begin{tikzpicture}  %
 [scale=0.6, point/.style = {draw, circle, fill=black, inner sep=0.5pt}]
 \filldraw[fill=light-gray,color=light-gray]  (0,0) node[]{} -- (4,0) node[]{} -- (0,4) node[]{}  -- cycle;
 \filldraw[fill=very-light-gray] (4,0) arc[start angle=0, end angle=360, radius=4]; 
 \node (C) at (0,0) [point]{}; 
 \node (iU) at (0,4) [cross,label=above:{$\ci \sol = \exp( \ci \tfrac{1}{2}\pi)\sol$}]{};
 \node (minusiU) at (0,-4) [cross,label=below:{$-\ci \sol = \exp(- \ci \tfrac{1}{2}\pi)\sol$}]{};
 \node (U) at (4,0) [cross,label=right:$\sol$]{};
 \node (E1) at (4, 4.5) []{}; 
 \node (E2) at (4, -4.5) []{};  
 \node (Utheta) at (-2.828427124748,2.828427124748) [point,label=left:{{$\exp( \ci \tfrac{3}{4}\pi)\sol$}}]{}; 
  \draw[->,thick, color=red] (C) -- (U);
  \draw[->,thick,color=new-blue] (C) -- (iU);
  \draw[->,thick,dashed,color=gray] (C) -- (minusiU);
  \draw[->,thick,dashed,color=gray] (C) -- (-4,0);
  \draw[->,thick, dashed,color=light-blue] (U) -- (E1);
  \draw[->,thick, dashed,color=light-blue] (U) -- (E2);
  \draw[->,thick, dashed,color=gray] (C) -- (Utheta);
  \node at (Utheta) [cross,rotate=30]{};
  \node at (4.5, 3.5) {{\color{light-blue}$\ci \sol$}};
  \node at (4.5, -3.5) {{\color{light-blue}-$\ci \sol$}};
  \node at (0.3, -0.3) {$0$};
 \node at (-6.4,0) {{$\exp(- \ci \pi)\sol=-\sol$}};
\end{tikzpicture}
\caption{For a given minimizer $\sol$, the figure illustrates the circle line parametrized by the $2\pi$-periodic curve $\gamma : t\mapsto \exp(- \ci t) \sol$ for $t \in [-\pi ,\pi)$. The tangent direction in $u$ is given by $\gamma^{\prime}(0) = \ci \sol$ and the energy $E$ is constant and minimal on the whole circle line, i.e. $\frac{\mbox{\rm d}^2}{\mbox{\rm d}t^2} E(\hspace{1pt}\gamma(t)\hspace{1pt})= 0$.}
\label{illustration-tangent}
\end{figure}


\begin{definition}[Local uniqueness up to gauge transformation]
\label{def-local-uniqueness}
Let $\sol$ be a minimizer of \eqref{eq:energy_functional_times_kappa2}. 
We call $u$ a {\rm locally unique minimizer up to gauge transformation} if, for all 
smooth curves $\gamma : [-\pi,\pi) \rightarrow H^1$ with $\gamma(0)=\sol$, it holds 
\begin{eqnarray*}
\frac{\mbox{\rm d}^2}{\mbox{\rm d}t^2} E(\, \gamma(t) \, )\vert_{t=0} = 0
\quad \Longleftrightarrow \quad 
\gamma^{\prime}(0) \in \mbox{\rm span} \{ \ci \sol \} := \{ \alpha \, \ci \sol \, | \, \alpha \in \mathbb{R} \}.
\end{eqnarray*}
Note that $u$ being a minimizer always implies $\frac{\mbox{\rm\tiny d}}{\mbox{\tiny\rm d}t} E(\, \gamma(t) \, )\vert_{t=0} = 0$ and $\frac{\mbox{\tiny\rm d}^2}{\mbox{\tiny\rm d}t^2} E(\, \gamma(t) \, )\vert_{t=0} \ge 0 $, independent if it is locally unique or not.
\end{definition}
From now on, we assume local uniqueness for our error analysis. %
\begin{assumption} \label{ass:cinfsup}
The minimizers $\sol$ of the Ginzburg-Landau energy \eqref{eq:energy_functional_times_kappa2} are locally unique up to gauge transformation in the sense of Definition \ref{def-local-uniqueness}. 
\end{assumption}
For any minimizer $\sol$, the above assumption implies inf-sup stability of $E^{\prime\prime}(\sol)$ on the $\ipsymLtwo{\cdot}{\cdot}$-orthogonal complement of $\mbox{\rm span} \{ \ci \sol \}$ in $H^1$. To formalize and prove this result, we define the corresponding space by
%
\begin{equation}
	\Honeperp \coloneqq H^1 \cap ( \ci \sol)^\perp \coloneqq \{ \testfun \in H^1 \, | \, m(\ci \sol , \testfun ) = 0 \}.
\end{equation}
Note that $\Honeperp$ is a closed subspace of $H^1$ and that our error analysis will be naturally restricted to it. The claimed inf-sup stability is specified in the following proposition.
\begin{proposition}
\label{proposition-inf-sup-stability}
Let $\sol$ be a minimizer of \eqref{eq:energy_functional_times_kappa2} and let Assumption \ref{ass:cinfsup} be fulfilled, i.e., $\sol$ is locally unique up to gauge transformation. Then, there is a constant $\Csol \gtrsim 1$ such that
	%
	\begin{equation} \label{eq:inf_sup_condition_cont}
		\Csolinv  \leq \inf_{\testfunTWO \in \Honeperp} \sup_{\testfun\in \Honeperp} \frac{\dualp{\energy''(\sol)  \testfunTWO}{\testfun}}{\Honekappa{\testfunTWO} \Honekappa{\testfun}  } .
	\end{equation}
Furthermore, $\energy''(\sol) $ is singular in the direction $\ci \sol$, i.e.,
\begin{eqnarray*}
\dualp{\energy''(\sol) \, \ci \sol}{\testfun} = 0 \qquad \mbox{for all } \testfun \in H^1.
\end{eqnarray*}
\end{proposition}
% 
Since Assumption \ref{ass:cinfsup} implies a positive spectrum of $E''(\sol)$ on $\Honeperp$, the inf-sup stability follows from the G{\aa}rding inequality. For completeness we elaborate the short proof below, also with the purpose to emphasize how to check the local uniqueness numerically.
%
\begin{proof}
Let $v \in \Honeperp \setminus \{ 0\}$ be arbitrary and $\gamma : [-\pi,\pi)\rightarrow H^1$ a smooth curve with $\gamma(0)=u$ and $\gamma^{\prime}(0)=v$. Then Assumption \ref{ass:cinfsup} implies
\begin{equation}
\label{positivityEprimeprime}
0 < \frac{\mbox{\rm d}^2}{\mbox{\rm d}t^2} E(\, \gamma(t) \, ) \vert_{t=0} 
= \langle E^{\prime}( \gamma(0) ) ,  \gamma^{\prime\prime}(0)  
\rangle
+
\langle E^{\prime\prime}( \gamma(0) ) \, \gamma^{\prime}(0), \gamma^{\prime}(0) \rangle
= \langle E^{\prime\prime}(\sol ) v, v \rangle.
\end{equation}
Hence, the eigenvalue problem seeking $(w_{\ell}, \lambda_{\ell}) \in H^1 \times \mathbb{R}$ with
\begin{eqnarray}
\label{eigenvalues-secvarE}
\langle E^{\prime\prime}(u) w_{\ell} , v \rangle = \lambda_{\ell} \, m( w_{\ell} , v ) \quad \mbox{for all } v \in H^1
\end{eqnarray}
has a non-negative spectrum. A direct calculation shows that the smallest eigenvalue is $\lambda_1= 0$ with eigenfunction $w_1 = \ci \sol$. Furthermore, since $\langle E^{\prime\prime}(\sol ) \cdot ,\cdot \rangle$ is a symmetric bilinear form on $H^1$, the second smallest eigenvalue $\lambda_2$ is positive due to the Courant--Fischer theorem which yields
$$
\lambda_2 = \underset{v \not=0}{\inf_{v\in \Honeperp}}  \frac{\langle E^{\prime\prime}(u) v , v \rangle}{m(v,v)} \overset{\eqref{positivityEprimeprime}}{>} 0.
$$
Observe that the strict positivity of the second eigenvalue of $E^{\prime\prime}(u)$ is in fact equivalent to the local uniqueness of $\sol$ in the sense of Definition \ref{def-local-uniqueness}. Hence, by computing $\lambda_2$ we can practically verify if a computed minimizer $u$ is locally unique.

Thanks to $\lambda_2>0$, we obtain that $E^{\prime\prime}(u)$ is injective on the $m(\cdot,\cdot)$-orthogonal complement of the first eigenspace, i.e., on $\Honeperp$. Furthermore, it is easily seen that the following G{\aa}rding inequality holds for all $v\in H^1$ (and in particular $v\in \Honeperp$):
\begin{eqnarray}
\label{gardinginequality}
\langle E^{\prime\prime}(u) v , v \rangle \ge \frac{1}{2} \| v \|^2_{H^1} - (1+\kappa^2) (1 + A_{\infty}^2 ) \| v \|^2_{L^2}.
\end{eqnarray}
It is well known (cf. \cite{GaticaHsiao94,Yosida1980}) that the injectivity on $\Honeperp$ and the G{\aa}rding inequality (in combination with the Lax-Milgram theorem) imply the Fredholm alternative for $E^{\prime\prime}(u) : \Honeperp \rightarrow (\Honeperp)'$, i.e., $E^{\prime\prime}(u)$ has a bounded inverse on the orthogonal complement of $\ci \sol$. With this insight, let $v \in \Honeperp$ be arbitrary and let $z \in \Honeperp$ be the unique solution to 
\begin{eqnarray*}
\langle E^{\prime\prime}(u) z , w \rangle = \mu \, m(v, w)\qquad \mbox{for all } w\in \Honeperp
\end{eqnarray*}
for some constant $\mu>(1+\kappa^2) (1 + A_{\infty}^2 )$. Since  $E^{\prime\prime}(u)$ has a bounded inverse it also holds
\begin{eqnarray}
\label{bounded-inverse-estimate-z}
\| z \|_{H^1} \le C \, \mu \, \| v \|_{H^1} 
\end{eqnarray}
for some constant $C=C(u,\kappa)$. In conclusion we obtain
\begin{eqnarray*}
\langle E^{\prime\prime}(u) (v+z) , v \rangle
= \langle E^{\prime\prime}(u) v, v \rangle + \mu \, m(v, v) 
\overset{\eqref{gardinginequality}}{\ge} \frac{1}{2} \| v\|_{H^1}^2 
\overset{\eqref{bounded-inverse-estimate-z}}{\ge} C \, \| v\|_{H^1} \| v+z \|_{H^1}
\end{eqnarray*}
with $C=C(u,\kappa,\mu)>0$. Hence, 
\begin{eqnarray*}
\sup_{\phi \in \Honeperp}  \frac{\langle E^{\prime\prime}(u) \phi , v \rangle }{\| \phi \|_{H^1} \| v \|_{H^1}}
\ge \frac{\langle E^{\prime\prime}(u) (v+z) , v \rangle}{\| v+z\|_{H^1} \| v \|_{H^1}}
\ge C.
\end{eqnarray*} 
Since $v \in \Honeperp$ was arbitrary, the desired inf-sup stability follows from the symmetry of $\langle E^{\prime\prime}(u) \, \cdot , \cdot\rangle$ and the equivalence of the $H^1$-norm and the $H^1_{\kappa}$-norm (up to $\kappa$-constants).
\end{proof}

\begin{remark}[Verifying the local uniqueness numerically]
	\label{rem:loc_uniq}
The proof of Proposition \ref{proposition-inf-sup-stability} shows that Assumption \ref{ass:cinfsup} is fulfilled for a minimizer $u$ if and only if the second smallest eigenvalue, $\lambda_2$, of 
\begin{eqnarray*}
\langle E^{\prime\prime}(u) w_{\ell} , v \rangle = \lambda_{\ell} \, m( w_{\ell} , v ) \quad \mbox{for all } v \in H^1
\end{eqnarray*}
is positive. Given a computed discrete minimizer $u_h$ that approximates $u$, we can check this condition numerically. Since $\| u_h - u \|_{H^1} \le \varepsilon(h) \rightarrow 0$ independent of local uniqueness (cf. Proposition \ref{prop:abstract_convergence} below), the second smallest eigenvalue of $E^{\prime\prime}(u_h)$ converges to the second smallest eigenvalue of $E^{\prime\prime}(u)$. Hence, if $\lambda_2$ is clearly bounded away from zero, $u$ must be locally unique in the sense of Definition \ref{def-local-uniqueness}. The practical assembly of the bilinear form $\langle E^{\prime\prime}(u_h) \, \cdot ,\cdot \rangle$ is discussed in the appendix. In our numerical experiments we present the corresponding values for $\lambda_2$ and we observed that the local uniqueness up to gauge transformations was always fulfilled.
\end{remark}

In the next step, we will derive stability and regularity estimates for solutions to variational problems on $\Honeperp$. The variational problems will later play a crucial role in our error analysis and involve the stabilized bilinear form $\abilmagstabsym{\cdot}{\cdot}$, as well as the inf-sup stable bilinear form $\langle E^{\prime}(u) \, \cdot , \cdot \rangle$.

\begin{lemma} \label{lem:wepo_abilmagstab}
	For any $f \in L^2(\Omega)$, there is $\testfunFOUR\in \Honeperp \subset H^1(\Omega)$ such that
	%
	\begin{equation}
		\abilmagstabsym{\testfunFOUR}{\testfun} =  \ipsymLtwo{f}{\testfun}, \quad  \text{ for all } \testfun \in \Honeperp ,
	\end{equation}
	%
	and there hold the bounds
	%
	\begin{equation}
		\Honekappa{\testfunFOUR} 
		\lesssim \Honekappaminus{f}
		\lesssim \frac{1}{\kappa} \norm{L^2}{f}
		\quad 
		\mbox{and, if $\Omega$ is convex, then $\testfunFOUR \in H^2$ and} 
		\
		\Htwokappa{\testfunFOUR} \lesssim  \norm{L^2}{f},
	\end{equation}
	%
	where the (hidden) constants in the bounds are independent of $\kappa$. 
	\end{lemma}




\begin{proof}
	Since $\abilmagstabsym{\cdot}{\cdot}$ is still coercive on $\Honeperp$, we immediately obtain the unique solution, and also the bounds in $\HonekappaSpace$. 
	Furthermore, we have for any $f \in L^2$ that
	%
	\begin{equation} \label{eq:relation_Honeminus_L2}
		\Honekappaminus{f} 
		=
		\sup_{  \Honekappa{\testfun} = 1} \ipsymLtwo{f}{\testfun}
		\leq
		\sup_{  \Honekappa{\testfun} = 1}  \frac{1}{\kappa} \norm{L^2}{f} \kappa \norm{L^2}{\testfun}
		\leq  \frac{1}{\kappa} \norm{L^2}{f} ,
	\end{equation}
	%
	which yields the second inequality. For the bound in the $\HtwokappaSpace$-norm for convex domains, let $\testfun \in H^1$ and decompose as $\testfun = \widehat{\testfun} + \alpha  (\ci \sol) $
	with $\widehat{\testfun} \in \Honeperp$ and
	$\alpha = \ipsymLtwo{\testfun}{\ci \sol} \norm{L^2}{\sol}^{-2} $. Then,
	%
	\begin{align}
		\abilmagstabsym{\testfunFOUR}{\testfun} &= 	\abilmagstabsym{\testfunFOUR}{\widehat{\testfun} } +  \alpha \, \abilmagstabsym{\testfunFOUR}{ \ci \sol }
		=  \ipsymLtwo{f}{\widehat{\testfun} }  +  \alpha \, \abilmagstabsym{\testfunFOUR}{ \ci \sol }
		\\
		&=  \ipsymLtwo{f}{\testfun}  -
		\alpha \, \ipsymLtwo{f}{\ci \sol }+  \alpha \, \abilmag{\testfunFOUR}{ \ci \sol },
	\end{align}
	%
	where we used \eqref{eq:def_abilmagstabsym} in the last step. We first note
	%
	\begin{equation}
		| \ipsymLtwo{f}{\testfun}  -
		\alpha \, \ipsymLtwo{f}{\ci \sol } | \leq 2 \norm{L^2}{f} \norm{L^2}{\testfun} \,,
	\end{equation}
	%
	and then employ $\energy'(\ci \sol) = 0$ to obtain
	%
	\begin{align}
		| \abilmag{\testfunFOUR}{ \ci \sol } | &=
		| \dualp{\energy'(\ci \sol)}{\testfunFOUR} -
		\kappa^2 	\Real  \int_\Omega \bigl( |\sol|^2 -1 \bigr) \ci \sol \testfunFOUR^* 
		\,dx |
		\leq \kappa^2 \norm{L^2}{\sol} \norm{L^2}{\testfunFOUR} \lesssim \norm{L^2}{f},
	\end{align}
	%
	where we exploited $ \kappa^2 \norm{L^2}{\testfunFOUR} \lesssim \norm{L^2}{f}$ in the last line. Altogether we have shown that there exists some $f_\testfunFOUR\in L^2$ such that it holds for all $\testfun \in H^1$
	%
	\begin{align} \label{eq:var_form_extended_test_fun}
		\abilmagstabsym{\testfunFOUR}{\testfun} &=  \ipsymLtwo{f_\testfunFOUR}{\testfun}  ,\quad \norm{L^2}{f_\testfunFOUR} \lesssim \norm{L^2}{f}.
	\end{align}
	%
	We conclude as in Theorem~\ref{thm:cont_minimizer}: We write
	%
	\begin{equation} \label{eq:expansion_ahat}
		\abilmagstabsym{\testfunFOUR}{\testfun} 
		=  	  
		\Real \int_{\Omega}
		\nabla \testfunFOUR\cdot \nabla \testfun^* 
		\,dx
		+
		\Real \int_{\Omega}
		\bigl(\stabPar^2 \testfunFOUR 
		- 2 \ci \kappa \MagF \cdot \nabla \testfunFOUR
		+ \kappa^2 |\MagF|^2 \testfunFOUR \bigr) \testfun^*
		\,dx ,
	\end{equation}
	%
	and since the second term is in $L^2$, we have $\testfunFOUR\in H^2$ and
	%
	\begin{equation}
		\norm{H^2}{\testfunFOUR} \lesssim  \norm{L^2}{f_\testfunFOUR} + \kappa^2 \norm{L^2}{\testfunFOUR} + \kappa \norm{L^2}{\nabla \testfunFOUR} 
		\lesssim \norm{L^2}{f},
	\end{equation}
	%
	where we used the $L^2$- and $H^1$-bounds for $\testfunFOUR$ in the last step.
\end{proof}
%


Using the inf-sup stability established in Proposition \ref{proposition-inf-sup-stability}, we can formulate an analogous result for variational problems based on $\energy''(\sol)$ in $\Honeperp$. Note here that the inclusion $\Honeperp \subset H^1$ implies $\dualHone \subset  (\Honeperp)'$ for the dual spaces.


%


\begin{corollary} \label{cor:du_F_inf_sup_solvability} %\label{lem:H2_reg_of_du_F}
	Let Assumption~\ref{ass:cinfsup} hold.	
	
	\bulletpoint{a} For any $f \in  (\Honeperp)' $, 
	there is a unique $\testfunFOUR \in \Honeperp$ such that
	%
	\begin{equation} \label{eq:du_F_var_problem}
		\dualp{\energy''(\sol) \testfunFOUR}{\testfun}  = \dualp{f}{\testfun}, \quad \text{for all} \quad \testfun \in \Honeperp ,
	\end{equation} 
	%
	which satisfies the estimate
	%
	\begin{equation}
		\Honekappa{\testfunFOUR} \leq \Csol \Honekappaminus{f} .
	\end{equation}
	
	
	\bulletpoint{b} Let $\testfunFOUR \in \Honeperp$ be the solution of \eqref{eq:du_F_var_problem} with $f \in L^2$.
	 Then, it further holds
	%
	\begin{align}
		\Honekappa{\testfunFOUR} 
		&\leq \frac{\Csol}{\kappa} \norm{L^2}{f}
		\quad
		\mbox{and, if $\Omega$ is convex, then $\testfunFOUR \in H^2$ and} 
%		\mbox{and \Changes{if $\Omega$ is convex} }
		\
		\Htwokappa{\testfunFOUR} \lesssim \Csol  \norm{L^2}{f} .
		%
	\end{align}
\end{corollary}


\begin{proof}
	By standard theory for indefinite differential equations (cf. \cite{Bab7071}), the inf-sup stability in Proposition~\ref{proposition-inf-sup-stability} directly gives the \wepo of  \eqref{eq:du_F_var_problem} together with the stability estimate $\Honekappa{\testfunFOUR} \leq \Csol \Honekappaminus{f}$, hence proving (a). The first estimate in (b) is obtained from \eqref{eq:relation_Honeminus_L2}.
	%
	Using this observation, we conclude that $\testfunFOUR\in \Honeperp$ solves
	%
	\begin{equation}
		\abilmagstabsym{\testfunFOUR}{\testfun} =  \ipsymLtwo{\widetilde{f}}{\testfun}, \quad  \text{ for all } \testfun \in \Honeperp \,,
	\end{equation}
%
for some $\widetilde{f}\in L^2$ with $\norm{L^2}{\widetilde{f}} \lesssim \Csol \norm{L^2}{f}$, 
and thus Lemma~\ref{lem:wepo_abilmagstab} gives the claim.
\end{proof}
