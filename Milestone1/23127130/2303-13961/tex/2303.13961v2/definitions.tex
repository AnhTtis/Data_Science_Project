\usepackage{amsmath}
\usepackage{amssymb}
\usepackage{mathtools}
\usepackage{url}
\usepackage{hyperref}
\usepackage{autonum}
\usepackage{xspace}
\usepackage{xcolor}
\usepackage{hyperref}
\usepackage[msc-links]{amsrefs}

\usepackage{graphicx}
\usepackage{tikz} %Diagramme
\usepackage{pgfplots} 

\usetikzlibrary{calc}
\usetikzlibrary{shapes.misc}

\tikzset{cross/.style={cross out, thick, draw=black, minimum size=2*(#1-\pgflinewidth), inner sep=0pt, outer sep=0pt},
%default radius will be 2.5pt.
cross/.default={2.5pt}}


\usepackage{caption}
\usepackage{subcaption}

\numberwithin{equation}{section}

\usepackage{siunitx}
\usepackage{multirow}

\newcommand{\Zeile}[5]{#1 & #2 & #3 & #4 & #5 }


\numberwithin{equation}{section}
%\usepackage[notref,notcite]{showkeys}


%\renewcommand*\showkeyslabelformat[1]{%
%		\fbox{\parbox[t]{\marginparwidth}{\raggedright\path{#1}}}}

	
%\usepackage{lineno}
%\linenumbers

\hypersetup{
		colorlinks,
		linkcolor={red!50!black},
		citecolor={blue!50!black},
		urlcolor={blue!80!black}
	}


\definecolor{color0}{rgb}{0.12156862745098,0.466666666666667,0.705882352941177}
\definecolor{color1}{rgb}{1,0.498039215686275,0.0549019607843137}
\definecolor{color2}{rgb}{0.172549019607843,0.627450980392157,0.172549019607843}
\definecolor{color3}{rgb}{0.83921568627451,0.152941176470588,0.156862745098039}
\definecolor{color4}{rgb}{0.580392156862745,0.403921568627451,0.741176470588235}
\definecolor{color5}{rgb}{0.549019607843137,0.337254901960784,0.294117647058824}
\definecolor{color6}{rgb}{0.890196078431372,0.466666666666667,0.76078431372549}

\definecolor{very-light-gray}{gray}{0.75}
\definecolor{light-gray}{gray}{0.69}
\definecolor{new-blue}{rgb}{0.0,0.0,0.8}
\definecolor{light-blue}{rgb}{0.4,0.45,1}

\usepackage{enumitem}
\setenumerate{label=\upshape(\alph*),itemsep=3pt,topsep=3pt}


\newcommand{\bulletpoint}[1]{\upshape(#1)}

%%%%%%%%%%%%%%%%%%%%%%%%%%%%%%%%%%%%%%%%%%%%%%%%%%%%%%%%%
\newcommand{\mycode}{https://doi.org/10.5445/IR/1000157006}
%%%%%%%%%%%%%%%%%%%%%%%%%%%%%%%%%%%%%%%%%%%%%%%%%%%%%%%%%

\definecolor{myBlue3}{RGB}{60,124,155} 
\newcommand{\corr}[1]{{\color{myBlue3}{#1}}}

\newcommand{\GL}{Ginzburg--Landau\xspace}

%%%%%%%%%%%%%%%%%%%%%%%%%%%%%%
\newtheorem{theorem}{Theorem}[section]
\newtheorem{lemma}[theorem]{Lemma}
\newtheorem{corollary}[theorem]{Corollary}
\newtheorem{proposition}[theorem]{Proposition}
\newtheorem{definition}[theorem]{Definition}
\newtheorem{remark}[theorem]{Remark}
\newtheorem{assumption}[theorem]{Assumption}
%%%%%%%%%%%%%%%%%%%%%%%%%%%%%%%%%%%%%%%%%%

\newcommand{\Changes}[1]{{\color{blue} #1}}
\newcommand{\Bcomment}[1]{{\color{orange}{\bf B:} #1}}
\newcommand{\Pcomment}[1]{{\color{magenta}{\bf P:} #1}}

\newcommand{\Frechet}{Fréchet\xspace}
\newcommand{\wepo}{well-posedness\xspace}

\DeclareMathOperator{\Real}{Re}
\DeclareMathOperator{\Imag}{Im}
\DeclareMathOperator{\linhull}{span}
\DeclareMathOperator{\vol}{vol}
\newcommand{\ci}{\mathrm{i}}


\newcommand{\kommentare}[1]{}

%%%%%%%%%%%%%%%%%%%%%%%%%%%%%%%%%%%%%%%%%%%%%%%%%%%%

\newcommand{\ip}[2]{ ( #1,#2 ) }
\newcommand{\dualp}[2]{ \langle #1,#2 \rangle}
\newcommand{\abil}[2]{a(#1,#2)}
%
\newcommand{\abilmag}[2]{a_A(#1,#2)}
\newcommand{\abilmagstab}[2]{\tilde{a}_A(#1,#2)}

\newcommand{\abilmagstabsym}[2]{\hat{a}_\kappa(#1,#2)}
\newcommand{\ipsymLtwo}[2]{ m ( #1,#2 ) }


\newcommand{\dualHone}{(H^1)'}

\newcommand{\HonekappaSpace}{H^1_\kappa}
\newcommand{\HtwokappaSpace}{H^2_\kappa}
\newcommand{\Honekappa}[1]{\norm{\HonekappaSpace}{#1}}
\newcommand{\Htwokappa}[1]{\norm{\HtwokappaSpace}{#1}}
%\newcommand{\Honekappaminus}[1]{\norm{H^{-1}_\kappa}{#1}}
\newcommand{\Honekappaminus}[1]{\norm{(\HonekappaSpace)'}{#1}}


\newcommand{\LOD}{{\mbox{\rm\tiny LOD}}}

\newcommand{\h}{h}
\newcommand{\testfun}{\varphi}
\newcommand{\testfunTWO}{v}
\newcommand{\testfunTHREE}{w}
\newcommand{\testfunFOUR}{z}
\newcommand{\testfunh}{\varphi_{\h}}


\newcommand{\projisol}{\Pi_{\ci \sol}}
\newcommand{\Ritzproj}{\textup{R}_{\kappa,\h}} 
\newcommand{\Ltwoproj}{\pi_{\h}}
\newcommand{\Ih}{I_\h}

\newcommand{\projLtwoisol}{\textup{R}_{\kappa,\h}^{\perp}}
%\newcommand{\projLtwoisol}{\Pi_{\kappa,\h}^{\perp}}
\newcommand{\Ltwotwist}{P_{\ci u}^{\perp}}

%\newcommand{\Ritzprojahat}{\text{R}_{\h}^{\widehat{a}}} 


\newcommand{\RitzprojLOD}{\textup{R}_{\kappa,\h}^\LOD} 
\newcommand{\projLtwoisolLOD}{\textup{R}_{\kappa,\h}^{\perp,\LOD}} 
\newcommand{\LtwoprojLOD}{\pi_{\h}^{\LOD}} 


\newcommand{\errh}{e_{\h}}


\newcommand{\errHmone}{\varepsilon_\h}
\newcommand{\errHmonelin}{\errHmone^{\text{lin}}}
\newcommand{\errHmonenonlin}{\errHmone^{\text{nonlin}}}


\newcommand{\Cbnd}{C_{\text{bnd}}}
\newcommand{\Ccoer}{C_{\text{coe}}}
\newcommand{\Cappro}{C_{\text{ap}}}


\newcommand{\cinfsup}{C_{\text{inf,sup}}}
\newcommand{\cinfsupLtwo}{C_{\text{inf,sup},L^2}}
\newcommand{\cinfsupH}{C_{\text{inf,sup,\h}}}
\newcommand{\cinfsupHLOD}{C_{\text{inf,sup,\h,\LOD}}}

\newcommand{\Csol}{C_{\textup{sol}}(\sol,\kappa)}
\newcommand{\Csolinv}{C^{-1}_{\textup{sol}}(\sol,\kappa)}
\newcommand{\CsolH}{C_{\textup{sol},\h}(\kappa)}
\newcommand{\CsolHLOD}{C_{\textup{sol, \LOD}}(\kappa)}


\newcommand{\Honeperp}{H^1_{\ci \sol}}
%\newcommand{\Honeperp}{H^1_{\sol}}
\newcommand{\HoneperpH}{V_{\h,\sol}}
\newcommand{\HoneperpHLOD}{V_{\h,\sol}^\LOD}

\newcommand{\diffop}{L}
\newcommand{\energy}{E}
\newcommand{\energyhat}{\widehat{E}}
\newcommand{\kaptrafo}{\lambda}

\renewcommand{\div}{\operatorname{div}}
\newcommand{\curl}{\operatorname{curl}}

\newcommand{\norm}[2]{\lvert| #2 \rvert|_{#1}}
\newcommand{\normbig}[2]{\bigl\lVert #2 \bigr\rVert_{#1}}

%%%%%%%%%%


\newcommand{\sol}{u}
\newcommand{\solh}{\sol_{\h}}
\newcommand{\solhn}[1]{\sol_{{\h}_{#1}}}
\newcommand{\solhnTwist}[1]{\widetilde{\sol}_{{\h}_{#1}}}


\newcommand{\solhLOD}{\solh^{\LOD}}

%\newcommand{\solhahat}{\solh^{ \widehat{a} }}


\newcommand{\Nbh}{U} %{\mathcal{U}}


\newcommand{\MagF}{A}
\newcommand{\MagFinfty}{A_\infty}
\newcommand{\stabPar}{\beta}




\newcommand{\nonlinLetter}{G}
\newcommand{\nonlinLetterShift}{\widetilde{\nonlinLetter}}
\newcommand{\nonlinhLetter}{G_{\h}}
\newcommand{\nonlin}[1]{\hat{\nonlinLetter}(#1)}
\newcommand{\nonlinfull}[1]{\nonlinLetter (\kappa , #1)}


\newcommand{\nonlinzeroLetter}{F}
\newcommand{\nonlinhzeroLetter}{F_{\h}}
\newcommand{\nonlinzerofull}[1]{\nonlinzeroLetter (\kappa , #1)}
\newcommand{\nonlinhzerofull}[1]{\nonlinhzeroLetter (\kappa , #1)}

%%%%%%%%%
\newcommand{\R}{\mathbb{R}}
\newcommand{\C}{\mathbb{C}}
\newcommand{\VS}{H^1} %V
\newcommand{\VSh}{V_{\h}}
\newcommand{\VShLOD}{V_{\h}^{\LOD}}

\newcommand{\VShperp}{V_{\h}^\perp}

\newcommand{\Th}{\mathcal{T}_\h}
\newcommand{\Pone}{\mathcal{P}_1}

%%%%%%%%%%%%%%%%%%%%%%%%%%%%%%%%%%%

\newcommand{\pt}{\partial_t}