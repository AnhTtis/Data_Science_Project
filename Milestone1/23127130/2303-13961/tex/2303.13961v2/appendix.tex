%
\appendix \section{Matrix representation of $E^{\prime\prime}(u_h)$}
\label{appendix}

In the following we briefly discuss the matrix representation of $E^{\prime\prime}(u_h)$ and the computation of the lower-most eigenvalues. 

For that, let $\mathcal{N}_h = \{ z_1, \cdots, z_N \}$ denote the set of nodes of the mesh $\mathcal{T}_h$ and let $\phi_j \in \VSh$ denote the {\it real} nodal shape function with the property
$$
\phi_j(z_k) = \delta_{jk}  \qquad \mbox{for all } 1 \le j,k \le N.
$$
Consequently, a nodal basis of $\VSh$ is given by the set
\begin{eqnarray*}
\{ \phi_1, \ci  \phi_1 , \phi_2, \ci  \phi_2 ,  \cdots, \phi_N, \ci \phi_N \}.  
\end{eqnarray*}
We are seeking of a matrix representation of the operator $\langle \energy''(\solh) v_h , w_h \rangle $ for arbitrary $v_h,w_h \in \VSh$. By expanding $v_h$ and $w_h$ in terms of the nodal basis functions we obtain
\begin{align}
v_h = \sum_{j=1}^N v_j^{\mbox{\tiny\rm Re}} \, \phi_j +  \sum_{j=1}^N v_j^{\mbox{\tiny\rm Im}} \, \ci \, \phi_j  
\qquad
\mbox{and}
\qquad
w_h = \sum_{j=1}^N w_j^{\mbox{\tiny\rm Re}} \, \phi_j +  \sum_{j=1}^N w_j^{\mbox{\tiny\rm Im}} \, \ci \, \phi_j  
\end{align}
for corresponding coefficient vectors $\mathbf{v}^{\mbox{\tiny\rm Re}},\mathbf{v}^{\mbox{\tiny\rm Im}},\mathbf{w}^{\mbox{\tiny\rm Re}},\mathbf{w}^{\mbox{\tiny\rm Im}}\in \mathbb{R}^N$ with
\begin{eqnarray*}
\mathbf{v}^{\mbox{\tiny\rm Re}} =  
\left(\begin{matrix}
v_1^{\mbox{\tiny\rm Re}} \\
\vdots \\
v_N^{\mbox{\tiny\rm Re}}
\end{matrix}\right),\quad
%
\mathbf{v}^{\mbox{\tiny\rm Im}} =  
\left(\begin{matrix}
v_1^{\mbox{\tiny\rm Im}} \\
\vdots \\
v_N^{\mbox{\tiny\rm Im}}
\end{matrix}\right), 
\quad
%
\mathbf{w}^{\mbox{\tiny\rm Re}} =  
\left(\begin{matrix}
w_1^{\mbox{\tiny\rm Re}} \\
\vdots \\
w_N^{\mbox{\tiny\rm Re}}
\end{matrix}\right),
% \in \mathbb{R}^N,
\quad
%
\mathbf{w}^{\mbox{\tiny\rm Im}} =  
\left(\begin{matrix}
w_1^{\mbox{\tiny\rm Im}} \\
\vdots \\
w_N^{\mbox{\tiny\rm Im}}
\end{matrix}\right).
\end{eqnarray*}
Hence, we can write
\begin{align}
\langle \energy''(\solh) v_h , w_h \rangle  = 
 \left(\begin{matrix}
	\mathbf{w}^{\mbox{\tiny\rm Re}} \\
	\mathbf{w}^{\mbox{\tiny\rm Im}}
\end{matrix}
\right)^{\top}	
\left( \begin{matrix}
\mathbf{A}(u_h)^{\mbox{\tiny\rm RR}} & \mathbf{A}(u_h)^{\mbox{\tiny\rm IR}} \\
\mathbf{A}(u_h)^{\mbox{\tiny\rm RI}} &\mathbf{A}(u_h)^{\mbox{\tiny\rm II}}
\end{matrix} \right)
\left(\begin{matrix}
\mathbf{v}^{\mbox{\tiny\rm Re}} \\
\mathbf{v}^{\mbox{\tiny\rm Im}}
\end{matrix}
\right),
\end{align}
for a block matrix $\mathbf{A}(u_h) \in \R^{2N \times 2N}$. Recalling the representation of $\energy''(\solh)$ as
	\begin{eqnarray*}
\lefteqn{\dualp{\energy''(\solh) v_h}{w_h}}	\\
&=&  
	%
	\Real  \int_\Omega \bigl( \nabla v_h + \ci \kappa \MagF w_h \bigr)  \cdot \bigl( \nabla v_h + \ci \kappa \MagF w_h \bigr)^*   
	+
	\kappa^2 
	\bigl(  ( |\solh|^2 -1  ) v_h w_h^*  + \solh^2 v_h^* w_h^* + |\solh|^2 v_h w_h^* \bigr)
	\,dx,
	%
\end{eqnarray*}
we compute the corresponding blocks of  $\mathbf{A}(u_h) $ as follows. For the upper left block we obtain straightforwardly
\begin{eqnarray*}
\mathbf{A}(u_h)^{\mbox{\tiny\rm RR}}_{jk}
&=& \Real \int_\Omega \nabla \phi_k \cdot \nabla \phi_j + \kappa^2 \left( |\MagF|^2 +  2 |\solh|^2 -1 + \solh^2 \right)\phi_k \phi_j  \,dx \\
&=&  \int_\Omega \nabla \phi_k \cdot \nabla \phi_j + \kappa^2 \left( |\MagF|^2 -1 + 3 \Real(u_h)^2 + \Imag(u_h)^2 \right)\phi_k \phi_j  \,dx.
\end{eqnarray*}
For upper right block we have 
%
\begin{eqnarray*}
\mathbf{A}(u_h)^{\mbox{\tiny\rm IR}}_{jk}
&=& \Real  \int_\Omega \ci \bigl( \nabla \phi_k + \kappa \MagF \ci \phi_k \bigr)  \cdot \bigl( \nabla \phi_j + \ci \kappa \MagF \phi_j \bigr)^*   
	+
	\ci \kappa^2 
	\bigl(  ( |\solh|^2 -1  )  - \solh^2 + |\solh|^2 \bigr) \phi_k \phi_j
	\,dx \\
&=&  \int_\Omega \kappa \left( \phi_j  \MagF \cdot \nabla \phi_k - \phi_k \MagF \cdot  \nabla \phi_j \right)
	-
	\kappa^2 \Real( \ci  
	\solh^2 ) \phi_k \phi_j
	\,dx \\
&=&  \int_\Omega \kappa \left( \phi_j  \MagF \cdot \nabla \phi_k - \phi_k \MagF \cdot  \nabla \phi_j \right)
	+ 2
	\kappa^2 \Real(u_h) \Imag(u_h) \, \phi_k \phi_j
	\,dx . %\\
\end{eqnarray*}%%
%
Analogously, we obtain for upper left block 
\begin{eqnarray*}
\mathbf{A}(u_h)^{\mbox{\tiny\rm RI}}_{jk}
%&=& \Real  \int_\Omega \bigl( \nabla \phi_k + \ci \kappa \MagF \phi_k \bigr)  \cdot \bigl( \nabla \ci \phi_j - \kappa \MagF \phi_j \bigr)^* - \kappa^2 \bigl(  \ci ( |\solh|^2 -1  ) \phi_k \phi_j  + \ci \solh^2 \phi_k \phi_j + \ci |\solh|^2 \phi_k \phi_j \bigr) \,dx \\
%&=& \Real  \int_\Omega \kappa ( \phi_k \MagF \cdot \nabla \phi_j  - \phi_j  \MagF \cdot \nabla \phi_k ) - \int_\Omega \kappa^2 \Real \bigl(  \ci \solh^2 \bigr)  \phi_k \phi_j \,dx \\
&=& \int_\Omega \kappa ( \phi_k \MagF \cdot \nabla \phi_j  - \phi_j  \MagF \cdot \nabla \phi_k )
+ 2 \kappa^2 \Real(u_h) \Imag(u_h) \, \phi_k \phi_j \,dx.
\end{eqnarray*}
Finally, the lower left block is given by
%
\begin{eqnarray*}
\mathbf{A}(u_h)^{\mbox{\tiny\rm II}}_{jk}
%&=& \Real  \int_\Omega \bigl( \nabla \ci \phi_k + \ci \kappa \MagF \ci \phi_k \bigr)  \cdot \bigl( \nabla \ci \phi_j + \ci \kappa \MagF \ci \phi_j \bigr)^*  + \kappa^2  \bigl(  ( |\solh|^2 -1  ) \ci \phi_k( \ci \phi_j)^*  + \solh^2 (\ci \phi_k)^* (\ci \phi_j)^* + |\solh|^2 \ci \phi_k (\ci \phi_j)^* \bigr) \,dx\\
%&=& \Real  \int_\Omega \bigl( \nabla \ci \phi_k - \kappa \MagF \phi_k \bigr)  \cdot \bigl( - \nabla \ci \phi_j - \kappa \MagF \phi_j \bigr)  + \kappa^2  \bigl(  ( |\solh|^2 -1  ) \phi_k \phi_j  - \solh^2 \phi_k \phi_j + |\solh|^2 \phi_k \phi_j \bigr) \,dx\\
&=&  \int_\Omega  \nabla \phi_k \cdot  \nabla \phi_j + \kappa^2 \left(  |\MagF|^2 
-1 + \Real(u_h)^2 + 3 \Imag(u_h)^2)\right) \phi_k \phi_j \,dx.
\end{eqnarray*}
By assembling the blocks in a standard way, we obtain the matrix $\mathbf{A}(u_h)$. Since $\langle E^{\prime\prime}(u_h)\,\cdot,\cdot\rangle$ is symmetric, $\mathbf{A}(u_h)$ must be symmetric too. This is also easily seen by the observations that
\begin{align}
\mathbf{A}(u_h)^{\mbox{\tiny\rm RR}}_{jk}=\mathbf{A}(u_h)^{\mbox{\tiny\rm RR}}_{kj}, \qquad
\mathbf{A}(u_h)^{\mbox{\tiny\rm II}}_{jk}=\mathbf{A}(u_h)^{\mbox{\tiny\rm II}}_{kj},
\qquad 
\mbox{and}
\quad \mathbf{A}(u_h)^{\mbox{\tiny\rm IR}}_{kj} = \mathbf{A}(u_h)^{\mbox{\tiny\rm RI}}_{jk}.
\end{align}
Note that a matrix representation of $m(v_h,w_h)$ is straightforwardly given by
\begin{align}
m(v_h,w_h) = 
 \left(\begin{matrix}
	\mathbf{w}^{\mbox{\tiny\rm Re}} \\
	\mathbf{w}^{\mbox{\tiny\rm Im}}
\end{matrix}
\right)^{\top}\left( \begin{matrix}
\mathbf{M} & 0 \\
0 & \mathbf{M}
\end{matrix} \right)
\left(\begin{matrix}
\mathbf{v}^{\mbox{\tiny\rm Re}} \\
\mathbf{v}^{\mbox{\tiny\rm Im}}
\end{matrix}
\right),
\end{align}
where $\mathbf{M} \in \mathbb{R}^{N \times N}$ is the conventional mass matrix with entries $\mathbf{M}_{jk}= \int_\Omega \phi_k \phi_j \,dx$. Hence, the eigenvalues of $ E^{\prime\prime}(u_h)$ in the sense of definition \eqref{eigenvalues-secvarE} are given by the system 
\begin{align}
\left( \begin{matrix}
\mathbf{A}(u_h)^{\mbox{\tiny\rm RR}} & \mathbf{A}(u_h)^{\mbox{\tiny\rm IR}} \\
\mathbf{A}(u_h)^{\mbox{\tiny\rm RI}} &\mathbf{A}(u_h)^{\mbox{\tiny\rm II}}
\end{matrix} \right)
\left(\begin{matrix}
\mathbf{v}^{\mbox{\tiny\rm Re}}_{\ell} \\
\mathbf{v}^{\mbox{\tiny\rm Im}}_{\ell} 
\end{matrix}
\right)
=
\lambda_{\ell} 
\left( \begin{matrix}
\mathbf{M} & 0 \\
0 & \mathbf{M}
\end{matrix} \right)
\left(\begin{matrix}
\mathbf{v}^{\mbox{\tiny\rm Re}}_{\ell} \\
\mathbf{v}^{\mbox{\tiny\rm Im}}_{\ell}
\end{matrix}
\right).
\end{align}
Due to the symmetry of the matrices, the smallest eigenvalues can be easily computed with a standard method such as the inverse power iteration.
