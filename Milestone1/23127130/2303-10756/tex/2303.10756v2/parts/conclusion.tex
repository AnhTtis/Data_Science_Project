\section{Conclusion}\label{sec:conclusion}
The main aim of this study was to obtain practitioners' perspectives on the AB-BPM method. Using mostly qualitative research methods, we shed light on the requirements for the further development of AB-BPM tools and the underlying method.
Overall, the study participants perceived the methodology as advantageous in comparison to the status quo. The three main conclusions of the study are:
\emph{i)} more possibilities of human intervention, and interaction between the RL agent and the human expert, are a core requirement,;
\emph{ii)} transparency and features for the participation of process participants are needed to make AB-BPM culturally viable;
\emph{iii)} integrated process execution is necessary to facilitate the seamless deployment of parallel process variants and deliver the real-time data needed for dynamic RL and routing.
The openness of the semi-structured interviews facilitated the discovery of future research opportunities, e.g., studying companies carrying out unstructured process tests in a production environment, and tool-driven training of process participants. While we focused on a single process improvement method, practitioners' insights on different kinds of business improvement methods~\cite{malinova2022study} could provide insights for studies that compare AB-BPM to other methodologies.