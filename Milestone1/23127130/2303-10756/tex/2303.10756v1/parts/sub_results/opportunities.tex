\subsubsection*{Opportunities.}\label{sec:opportunities}
The \textbf{sentiment} towards AB-BPM was mainly positive. Multiple consultants brought up that some companies they worked with tried testing new process versions and comparing them with the status quo. However, the tests were mostly unstructured and considered only of a few instances or even no ``real'' instances (i.e., only tests). This means that any drawn conclusions are not dependable, due to the low number of instances and lack of statistical rigor when it comes to controlling confounding factors. Thus, AB testing provides a useful process improvement method that supports the structured testing of alternative versions. 

One question to the study participants was about the \textbf{suitability} of the method regarding contextual factors. The study participants were not presented with a list of categories but were free to elaborate on their intuition. More concretely, they were asked for what type of processes and what surrounding circumstances (company, market, industry) they think the methodology would be well- or ill-suited. Their statements were then mapped to the categorization of BPM contexts by~\cite{brocke_role_2016}. The result can be seen in Figure~\ref{fig:usecases}. The characteristics in italics present special cases for factors where every characteristic was deemed suitable, which we will outline in the following.

\begin{figure}[h]
\centering
\includegraphics[width=\textwidth]{graphics/suitability.png}
\caption{Suitability of method regarding BPM context. Categorization from \cite{brocke_role_2016}. Items in italics present particularly interesting/suitable cases for factors where every characteristic is suitable.}
\label{fig:usecases}
\end{figure}

\begin{description}
\item{\textbf{Focus.}} No agreement could be reached on whether AB-BPM was suitable for radical changes. \cite{Satyal:2019:IS} present the method primarily for evolutionary changes, while some study participants believe it is suitable for both. However, most consider the method more appropriate for small process changes due to the ease and speed of implementation. AB-BPM would be incompatible with fundamental changes that require ``lengthy discussions'' and expensive financial obligations, making rapid testing difficult. Somewhat larger changes within the same information system may be feasible, but smaller changes are generally preferred.
\item{\textbf{Value contribution.}} Using the AB-BPM method might be especially useful in core processes. This is because other processes are found at many companies and cannot be used for meaningful differentiation. One study participant noted that it is advisable to ``differentiate where you differ.'' They said, ``as a sports shoe company, we could strive to have the best finance processes, but that won't make people buy our shoes -- we need better shoes and better shoe quality to win in the marketplace.'' They, therefore, recommended using standard processes for everything but the core processes. This is already common practice and also suggested by academic studies~\cite{lubke_effectively_2019}. For core processes, however, experimentation with the AB-BPM method would be highly favorable.
\item{\textbf{Competitiveness.}} In general, there were no opinions indicating that any level of market competitiveness would lead to less suitability of the method. Study participants noted, however, that highly competitive markets would increase the need for such a tool to allow for faster process iterations, ``to stay competitive.''
\end{description}

The elicitation of \textbf{requirements} for a tool that executes and supports the AB-BPM method was also part of the study, and the identified feature requirements are presented in the following. First, we present the ranking of the items. Afterward, more details on the items ranked as most important are provided.

The ranking is based on the importance Likert scale presented in Section \ref{sec:methodology}. The average importance scores (AVG) are accompanied by the standard deviations (SD) to give an insight into the level of agreement among the experts. Furthermore, the feature requirements have been categorized into presentation, procedure, and support. 
 \emph{Presentation} includes features regarding the presentation of data, or features that are more focused on the front end of the tool in general;
\emph{Procedure} are features regarding the underlying technical or methodological procedure;
\emph{Support} includes features that already exist in the AB-BPM method but that have not been presented to the study participants during the introduction to AB-BPM (see Section \ref{sec:backgr}); they, therefore, support the equivalent suggestions by~\cite{Satyal:2019:IS}.

\begin{figure}[ht]
\centering
\includegraphics[width=\textwidth]{graphics/rank_tofe.png}
\caption{Item list of desired tool features, in order of perceived importance. Colors in ``Code'' column depict categories; in ``SD'' scale of standard deviations (highest: red; lowest: green)}
\label{fig:rank_towi}
\end{figure}

In the following, the three tool features ranked as most important are described in more detail.
\begin{description}
\item{\textbf{See potential impact beforehand (amount and business-wise).}}
According to the study participants, process experts should be able to see estimations on possible impacts beforehand to support an informed decision-making process, e.g., how many customers or what order volume would be affected by the test.
\item{\textbf{BPMN diff viewer.}}
One study participant emphasized the importance of human experts having a clear understanding of changes in the current initiative. They suggested a "diff viewer for the diagrams" - a graphical representation of changes made to a document from one version to another, commonly used in software engineering. In business process management, this could involve versions of a BPMN diagram, with changes highlighted in different colors. Diff viewers are well-researched and applied in this context, for example in~\cite{pietsch_comparison_2012,ivanov_bpmndiffviz_2015}.
\item{\textbf{Communicating process changes efficiently for teaching and enablement of employees.}}
The need for process participants to learn how new versions have to be executed was stressed by multiple interviewees. One study participant stated that ``one needs to notify the people working on steps in the process of the changes.'' More ``enablement is needed to teach employees the changes,'' and another study participant noted that ``seeing how this [aspect of change management] can be integrated would be an interesting question.'' This would go beyond just teaching single steps but also create openness and transparency about goals and project setup, allowing for ``a lot of change, even in parallel, without people being lost.''
Similar to the diff viewer, change notification management is a feature that has already received research attention in the context of business process management software~\cite{yan_business_2012,rosa_apromore_nodate}.
\end{description}