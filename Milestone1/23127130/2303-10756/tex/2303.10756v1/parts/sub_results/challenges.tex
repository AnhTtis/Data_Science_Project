\subsubsection*{Challenges.}\label{sec:challenges}
It is vital to know the core challenges to advance the AB-BPM methodology and adjacent endeavors. Only then can they be addressed and mitigated adequately. The risks and further challenges that the expert panel has voiced are, therefore, outlined below.

A critical goal of this work is to determine the AB-BPM method's principal \textbf{risks} since they hinder its usage and implementation in organizations.
In the following, we will present the results from the experts' ranking and then give a more detailed outline of the most highly ranked individual risks.
The risks, alongside their average risk scores and the standard deviation (SD) of those scores, can be seen in Figure~\ref{fig:rank_risk}. Furthermore, the risks have been categorized as follows.
\emph{Culture} are risks regarding the working culture and employees of the company. \emph{Results} include risks regarding results, decisions, and outcomes; \emph{Operations} consists of risks regarding the implementation and execution of the AB-BPM method itself, but also the normal business operations; \emph{Legal} includes risks regarding the cost and loss of income caused by legal uncertainty~\cite{tsui_experience_2013}.

\begin{figure}[ht]
\centering
\includegraphics[width=\textwidth]{graphics/rank_risk.png}
\caption{Item list of risks, in order of perceived risk. Colors in ``Code'' column describe categories; colors in ``SD'' column represent scale of standard deviations (highest: red; lowest: green).}
\label{fig:rank_risk}
\end{figure}

In the following, details on the three most highly ranked risks are explained in more detail.
\begin{description}
    \item[\textbf{Unclear results due to high process variance and process drift.}] As mentioned before, the execution of business processes can differ from how they were intended to be executed and it is subject to (unintended) changes over time. This phenomenon, called process drift~\cite{sato_survey_2021}, leads to a high variance of executed process versions. This could pose a risk for the AB-BPM method since it is then unclear whether process participants execute the two versions as they are intended. A process participant is a company-internal actor performing tasks of a process~\cite{dumas_fundamentals_2013}, i.e., an employee of the organization executing the process. If the process cases vary from the intended way of execution, it is hard to draw conclusions from the results since they might be based on a change that occurred spontaneously instead of the planned process changes. One example might be that ``people exchange emails instead of following the steps in the process execution software.''
    \item[\textbf{Erroneous machine-generated analysis results are blindly followed.}] \phantom{e}\\ Many interviewees noted that solely relying on the algorithm's interpretation of the data might cause problems. One study participant noted that ``such models are always an abstraction of reality [...] and relying on them completely can lead to mistakes.'' This topic also came up during the discussion of bad prior experiences, when a study participant noted that sometimes wrong decisions were made because of a lack of understanding of data. One potential example is the use of team performance metrics, which are often highly subjective (e.g., workload estimates in some project management methods), without context. Putting data into context and not blindly following statistical calculations is, therefore, a core challenge that needs to be addressed.
    \item[\textbf{Cultural change management problems during variant roll-out.}] \phantom{e}\\ This risk was added after the validation survey since one study participant remarked that this item was missing. It can be understood as incorporating any other cultural change management issues not yet included in the item list (see blue items in Figure~\ref{fig:rank_risk}). The high rating of this item can be seen as an indicator that the human side of the method and adjacent tools must not be left out of research and development efforts. The importance of culture also became very apparent when asked about \textit{prerequisites for the use of the AB-BPM method}. Many study participants noted that the organization would need to have an experiment culture, meaning that they should be open to trying new things and handling failures as learning opportunities. Furthermore, they stated a need for organizational transparency and trust.
\end{description}

The implementation and adoption of AB-BPM as presented in \cite{Satyal:2019:IS} assumes the existence of a Business Process Management System (BPMS) that allows for the direct deployment of BPMN models. A BPMS is an information system that uses an explicit description of an executable process model in the form of a BPMN model to execute business processes and manage relevant resources and data. It presents a centralized, model-driven way of business process execution and intelligence~\cite{dumas_fundamentals_2013}. However, most processes are executed by non-BPMS software, i.e., they are not executed from models directly~\cite{10.1007/978-3-031-07475-2_9}.

Therefore, whether the usage of a BPMS is a requirement for \textbf{technical feasibility} is a research question of this study. Altogether, AB testing of business processes seems technologically feasible without a BPMS, one interviewee noted: ``I do not think that it is a problem that processes are executed over several IT systems since you only need to be able to start either process version. The route they are going afterward, even if it is ten more systems, is no longer relevant.'' However, if we want to use live analytics to route incoming process instantiation requests (e.g., as proposed with RL) without a BPMS, we would need an Extract-Transform-Load (ETL) tool. ETL software is responsible for retrieving relevant data from various sources while bringing the data into a suitable format~\cite{vassiliadis2009survey}. Relying on a BPMS would not only have the benefit of easier data collection and access, it would also make deploying and executing experimental processes more straightforward. Furthermore, such an ETL tool might also be highly complex due to the many systems processes can potentially touch. One study participant noted that when a BPMS does not exist, ``you will have to put a lot of effort into mining performance data; it would be more difficult to get the same data from process mining, covering every path and system.'' In fact, most study participants did deem a BPMS, or something similar, to be a prerequisite. One study participant stated, however, that while some ``central execution platform'' would probably be needed, it remains unclear whether these have to be in the shape of current BPMS. Overall, there seemed to be the notion that the integrated, model-driven way of orchestrating and executing business processes offered by BPMS is the direction the industry should move towards. 

Besides the risks, the study participants also mentioned \textbf{other challenges}.
Here, we highlight some of them.
Regarding the question of \textit{bad prior experiences when conducting BPI initiatives}, some study participants criticized the unclear impact of process improvements during/after BPI projects. This is due to constantly changing environmental factors and the resulting difficulty to compare process data that has been collected at different points in time. This highlights the possible positive impact that AB-BPM could have on BPI efforts, by giving BPM experts a better data basis to evaluate improvement efforts.
Regarding the \textit{prerequisites for the use of the AB-BPM method}, on the more technical side, the interviewees noted that many companies would not offer the level of continuous data metrics needed for the dynamic, RL-driven routing during the experiments.