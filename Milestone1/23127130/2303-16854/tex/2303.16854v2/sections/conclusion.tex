\section{Conclusion}
In this paper, we present AnnoLLM, a novel annotation system powered by LLMs that has the potential to replace traditional crowdsourced annotators. 
AnnoLLM adopts a two-step approach, \textit{explain-then-annotate}. In this method, LLMs are initially employed to generate a few-shot CoT prompt, which is subsequently utilized to prompt LLMs in annotating unlabeled data. 
Our experimental results on three datasets demonstrate the feasibility of using AnnoLLM to substitute crowdsourced annotators. 
Moreover, we introduce the ConIR dataset, which is created using AnnoLLM, to facilitate the research on  conversation-based information retrieval. 

% Furthermore, we create the high-quality ConIR dataset for conversation-based information retrieval with AnnoLLM. 
% which highlights the potential to facilitate the development of using LLMs like GPT-3.5 to annotate data for various NLP tasks. 


\section{Acknowledgments}
This work is supported by HKU-SCF FinTech Academy, Shenzhen-Hong Kong-Macao Science and Technology Plan Project (Category C Project: SGDX20210823103537030), and Theme-based Research Scheme of RGC, Hong Kong (T35-710/20-R). 
We would like to thank the anonymous reviewers for their constructive and informative feedback on this work.