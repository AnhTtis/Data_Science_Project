% This must be in the first 5 lines to tell arXiv to use pdfLaTeX, which is strongly recommended.
\pdfoutput=1
% In particular, the hyperref package requires pdfLaTeX in order to break URLs across lines.

\documentclass[11pt]{article}

% Remove the "review" option to generate the final version.
% \usepackage[review]{EMNLP2023}
\usepackage[]{EMNLP2023}
% Standard package includes
\usepackage{times}
\usepackage{latexsym}

% For proper rendering and hyphenation of words containing Latin characters (including in bib files)
\usepackage[T1]{fontenc}
% For Vietnamese characters
% \usepackage[T5]{fontenc}
% See https://www.latex-project.org/help/documentation/encguide.pdf for other character sets

% This assumes your files are encoded as UTF8
\usepackage[utf8]{inputenc}

% This is not strictly necessary and may be commented out.
% However, it will improve the layout of the manuscript,
% and will typically save some space.
\usepackage{microtype}

% This is also not strictly necessary and may be commented out.
% However, it will improve the aesthetics of text in
% the typewriter font.
\usepackage{inconsolata}

\usepackage{subfigure}
\usepackage{graphicx}
\usepackage{subfigure}
\usepackage{multirow}
\usepackage{makecell}
\usepackage{booktabs}
\usepackage{amsmath}
\usepackage{bm}
\usepackage{etoolbox}
\usepackage{amssymb}

\usepackage{hyperref}
\usepackage{color}
\usepackage{stfloats}

\title{AnnoLLM: Making Large Language Models to Be Better\\ Crowdsourced  Annotators}



\author{%
  Xingwei He\textsuperscript{\rm 1}\thanks{\ \ Work done during internship at Microsoft Research Asia.}, \quad
  Zhenghao Lin\textsuperscript{\rm 2}, \quad
  Yeyun Gong\textsuperscript{\rm 4}, \quad
  A-Long Jin\textsuperscript{\rm 3},  \quad
  Hang Zhang\textsuperscript{\rm 4}, 
  \\
  \textbf{Chen Lin}\textsuperscript{\rm 2}\textbf{,}  \quad
  \textbf{Jian Jiao}\textsuperscript{\rm 5}\textbf{,}  \quad 
  \textbf{Siu-Ming Yiu}\textsuperscript{\rm 1}\thanks{\ \ Corresponding author.}\textbf{,} \quad
  \textbf{Nan Duan}\textsuperscript{\rm 4}\textbf{,} \quad
  \textbf{Weizhu Chen}\textsuperscript{\rm 5}
  \\
  \textsuperscript{\rm 1}The University of Hong Kong,
  \textsuperscript{\rm 2}Xiamen University,\\
  \textsuperscript{\rm 3}Xi'an Jiaotong-Liverpool University,
 \textsuperscript{\rm 4}Microsoft Research Asia,
  \textsuperscript{\rm 5}Microsoft 
  \\
  \texttt{hexingwei15@gmail.com},  
  \texttt{along.jin@xjtlu.edu.cn}, 
  \texttt{smyiu@cs.hku.hk}, \\
  \texttt{zhenghaolin@stu.xmu.edu.cn}, 
  \texttt{chenlin@xmu.edu.cn},  \\
  \texttt{\{yegong, v-zhhang, jian.jiao,  nanduan, wzchen\}@microsoft.com} 
}

\begin{document}
\maketitle
\begin{abstract}
Many natural language processing (NLP) tasks rely on labeled data to train machine learning models with high performance. However, data annotation is time-consuming and expensive, especially when the task involves a large amount of data or requires specialized domains. 
Recently, GPT-3.5 series models have demonstrated remarkable few-shot and zero-shot ability across various NLP tasks. 
In this paper, we first claim that large language models (LLMs), such as GPT-3.5, can serve as an excellent crowdsourced annotator when provided with sufficient guidance and demonstrated examples. 
Accordingly, 
we propose AnnoLLM, 
an annotation system powered by LLMs, which adopts a two-step approach, \textit{explain-then-annotate}. 
Concretely, we first prompt LLMs to provide explanations for why the specific ground truth answer/label was assigned for a given example. 
Then, we construct the few-shot chain-of-thought prompt with the self-generated explanation and employ it to annotate the unlabeled data with LLMs. 
Our experiment results on three tasks, including user input and keyword relevance assessment, BoolQ, and WiC, demonstrate that AnnoLLM surpasses or performs on par with crowdsourced annotators. 
Furthermore, we build the first conversation-based information retrieval dataset employing AnnoLLM.  This dataset is designed to facilitate the development of retrieval models capable of retrieving pertinent documents for conversational text. Human evaluation has validated the dataset's high quality.


\end{abstract}


% 

Over the past few years, there has been a significant amount of research focused on studying the ReLU activation function, with the aim of achieving neural network convergence through over-parametrization. However, recent developments in the field of Large Language Models (LLMs) have sparked interest in the use of exponential activation functions, specifically in the attention mechanism.

Mathematically, we define the neural function $F: \R^{d \times m} \times  \mathbb{R}^d \rightarrow \mathbb{R}$ using an exponential activation function. Given a set of data points with labels $\{(x_1, y_1), (x_2, y_2), \dots, (x_n, y_n)\} \subset \mathbb{R}^d \times \mathbb{R}$ where $n$ denotes the number of the data. Here $F(W(t),x)$ can be expressed as $F(W(t),x) := \sum_{r=1}^m a_r \exp(\langle w_r, x \rangle)$, where $m$ represents the number of neurons, and $w_r(t)$ are weights at time $t$. It's standard in literature that $a_r$ are the fixed weights and it's never changed during the training. We initialize the weights $W(0) \in \mathbb{R}^{d \times m}$ with random Gaussian distributions, such that $w_r(0) \sim \mathcal{N}(0, I_d)$ and initialize $a_r$ from random sign distribution for each $r \in [m]$.

Using the gradient descent algorithm, we can find a weight $W(T)$ such that $\| F(W(T), X) - y \|_2 \leq \epsilon$ holds with probability $1-\delta$, where $\epsilon \in (0,0.1)$ and $m = \Omega(n^{2+o(1)}\log(n/\delta))$. To optimize the over-parametrization bound $m$, we employ several tight analysis techniques from previous studies [Song and Yang arXiv 2019, Munteanu, Omlor, Song and Woodruff ICML 2022]. 

 

\section{Introduction}
\label{sec:introduction}
% \begin{itemize}
%     % Diffusion of FL
%     \item {\st{Diffusion of FL}}
%     % Security threats to FL
%     \item {\st{Security threats to FL with particular focus on model poisoning}}
%     % Limitations of existing countermeasures
%     \item {\st{Current countermeasures (e.g., KRUM) and their limitations}}
%     % Proposed method and its advantages
%     \item {\st{Intuitive description of the proposed method and its difference (i.e., advantages) w.r.t. state of the art}}
%     % Main contributions
%     \item {\st{Summary of the main contributions of this work}}
%     % Paper's structure and organization
%     \item {\st{Paper's structure and organization}}
% \end{itemize}

% Diffusion of FL
Recently, {\em federated learning} (FL) has emerged as the leading paradigm for training distributed, large-scale, and privacy-preserving machine learning (ML) systems~\cite{mcmahan2017googleai,mcmahan2017aistats}. 
The core idea of FL is to allow multiple edge clients to collaboratively train a shared, global model without disclosing their local private training data.
%Specifically, an FL system consists of a central server and many edge clients; 
A typical FL round involves the following steps: {\em(i)} the server randomly picks some clients and sends them the current, global model; {\em(ii)} each selected client locally trains its model with its own private data; then, it sends the resulting local model to the server;\footnote{Whenever we refer to global/local model, we mean global/local model {\em parameters}.} {\em(iii)} the server updates the global model by computing an \emph{aggregation function}, usually the average (FedAvg), on the local models received from clients.
% \begin{enumerate}
%     \item[{\em(i)}] the server sends the current, global model to the clients and appoints some of them for training;
%     \item[{\em(ii)}] each selected client locally trains its copy of the global model with its own private data; then, it sends the resulting local model back to the server;\footnote{Whenever we refer to global/local model, we mean global/local model {\em parameters}.}
%     \item[{\em(iii)}] the server updates the global model by computing an \emph{aggregation function} on the local models received from clients (by default, the average, also referred to as FedAvg~\cite{mcmahan2017aistats}).
% \end{enumerate}
This process goes on until the global model converges. %(e.g., after a certain number of rounds or other similar stopping criteria).
%\\
% The advantages of FL over the traditional, centralized learning paradigm are undoubtedly clear in terms of flexibility/scalability (clients can join/disconnect from the FL network dynamically), network communications (only model weights\footnote{We will use \textit{parameters} and \textit{weights} interchangeably.} are exchanged between clients and server), and privacy (each client's private training data is kept local at the client's end and not uploaded to the server).
\\
% Security threats to FL
%However, the growing adoption of FL also raises security concerns~\cite{costa2022covert}, particularly about its confidentiality, integrity, and availability.
Although its advantages over standard ML, FL also raises security concerns~\cite{costa2022covert}. %, particularly about its confidentiality, integrity, and availability~\cite{costa2022covert}.
% OLD, LONG VERSION
% Indeed, some work deals with privacy leakage that may expose the local data of some clients~\cite{melis2019sp}. 
% A large body of work, instead, investigates attacks that usually aim to detriment the predictive accuracy of the learned global model. For instance, \emph{data poisoning} attacks achieve this goal by letting an adversary pollute the training set of some corrupt FL clients with maliciously crafted examples~\cite{jagielski2018sp}.
% Similarly, in \emph{model poisoning} the attacker attempts to tweak the global model weights~\cite{bhagoji2019pmlr} by directly perturbing the local model's weights of some infected FL clients before these are sent to the central server for aggregation, usually via so-called Byzantine attacks. 
% It turns out that Byzantine model poisoning attacks severely impact standard FedAvg; therefore, more robust aggregation functions must be designed to make FL systems secure.
Here, we focus on \emph{untargeted model poisoning} attacks~\cite{bhagoji2019pmlr}, where an adversary attempts to tweak the global model weights %\footnote{We will use the terms \textit{parameters} and \textit{weights} interchangeably.} 
by directly perturbing the local model's parameters of some infected clients before these are sent to the central server for aggregation.
In doing so, the adversary aims to jeopardize the global model \textit{indiscriminately} at inference time.
Such model poisoning attacks severely impact standard FedAvg; therefore, more robust aggregation functions must be designed to secure FL systems.
\\
% In this paper, we focus on designing a novel robust aggregation scheme at the server's end to contrast the effect of Byzantine model poisoning attacks.
%
% Current countermeasures and their limitations
%Several countermeasures have been proposed in the literature to combat model poisoning attacks on FL systems.
% Some methods use simple statistics more robust than plain average to smooth the impact of malicious updates (e.g., Trimmed Mean and FedMedian~\cite{yin2018icml}). 
% Other defenses implement outlier detection techniques to discard malicious updates from the aggregation performed at the server's end. Those are either based on heuristics (e.g., Krum/Multi-Krum~\cite{blanchard2017nips} and Bulyan~\cite{mhamdi2018pmlr}) or data-driven approaches (e.g., K-means clustering~\cite{shen2016acm} or DnC via spectral analysis~\cite{shejwalkar2021ndss}). 
% Finally, some strategies rely on a centralized ``source of trust'' to spot potential malicious updates (e.g., FLTrust~\cite{cao2020fltrust}).
% Several countermeasures have been proposed in the literature to combat model poisoning attacks on FL systems, i.e., to discard possible malicious local updates from the aggregation performed at the server's end. 
% These techniques range from simple statistics more robust than plain average (e.g., Trimmed Mean and FedMedian~\cite{yin2018icml}) to outlier detection heuristics (e.g., Krum/Multi-Krum~\cite{blanchard2017nips} and Bulyan~\cite{mhamdi2018pmlr}) or data-driven approaches (e.g., spectral analysis via K-means clustering~\cite{shen2016acm} or spectral analysis), or methods based on ``source of trust'' (e.g., FLTrust~\cite{cao2020fltrust}).
% OLD, LONG VERSION
%Several countermeasures have been proposed in the literature to combat Byzantine model poisoning attacks on FL systems.
% Descriptive statistics
% For example, Trimmed Mean and FedMedian aggregate local model updates using more robust statistics than standard average~\cite{yin2018icml}.
%
% % Heuristics for outlier detection
% Many existing Byzantine-resilient strategies implement some outlier detection heuristics to discard the model updates sent by potentially malicious clients from the input of the aggregation function.
% One of the most popular heuristics is Krum~\cite{blanchard2017nips}.
% This strategy tries to mitigate the impact of Byzantine attacks by selecting as a global model the local model with the smallest sum of Euclidean distances to {\em all} the other local models.
% Although powerful, Krum requires the server to know (or, at least, estimate) the number of malicious FL clients upfront, which is generally impossible in a realistic attack scenario. %
% Moreover, Krum may become ineffective for complex, high-dimensional model parameter spaces due to the curse of dimensionality.
% Bulyan~\cite{mhamdi2018pmlr} tries to overcome this issue by combining Krum with a variant of Trimmed Mean.
% % Data-driven outlier detection
% Other strategies use data-driven outlier detection techniques -- e.g., via K-means clustering~\cite{shen2016acm} -- to spot potential malicious local model updates. 
% %For instance, Shen et al. propose to cluster local model updates with K-means and thus identify outliers.
%
% % Other techniques
% As far as the server is concerned, any local model received can be from a potential malicious client. 
% FLTrust~\cite{cao2020fltrust} assumes the server acts as a client, i.e., trains a local model on an additional {\em trustworthy} dataset at the server's end and compares it against all the local models from other clients. 
% This way, the server can rely on some ``source of trust'' when discarding potentially malicious clients.
%\\
% Limitations of existing Byzantine-resilient strategies
Unfortunately, existing defense mechanisms either rely on simple heuristics (e.g., Trimmed Mean and FedMedian by~\cite{yin2018icml}) or need strong and unrealistic assumptions to work effectively (e.g., foreknowledge or estimation of the number of malicious clients in the FL system, as for Krum/Multi-Krum~\cite{blanchard2017nips} and Bulyan~\cite{mhamdi2018pmlr}, which, however, cannot exceed a fixed threshold).
Furthermore, outlier detection methods using K-means clustering~\cite{shen2016acm} or spectral analysis like DnC~\cite{shejwalkar2021ndss} do not directly consider the temporal evolution of local model updates received.
Finally, strategies like FLTrust~\cite{cao2020fltrust} require the server to collect its own dataset and act as a proper client, thereby altering the standard FL protocol.
\\
% OLD, LONG VERSION
% Overall, existing Byzantine-resilient strategies are either simple heuristics (e.g., FedMedian) or, if they are more complex, they rely on strong and unrealistic assumptions to work effectively (e.g., knowing the number of malicious clients in the FL system in advance, as for Krum and alike).
% Furthermore, data-driven outlier detection methods do not consider the temporary evolution of local model updates received (e.g., K-means clustering). 
% Finally, strategies like FLTrust requires the server to collect its own dataset and act as a proper client, thereby altering the standard FL protocol.
%
% Description of the proposed method
This work introduces a novel pre-aggregation \textit{filter} robust to untargeted model poisoning attacks. Notably, this filter $(i)$ operates without requiring prior knowledge or constraints on the number of malicious clients and $(ii)$ inherently integrates temporal dependencies. 
The FL server can employ this filter as a preprocessing step before applying \textit{any} aggregation function, be it standard like FedAvg or robust like Krum or Bulyan.
Specifically, we formulate the problem of identifying corrupted updates as a multidimensional (i.e., matrix-valued) time series anomaly detection task. 
The key idea is that legitimate local updates, resulting from well-calibrated iterative procedures like stochastic gradient descent (SGD) with an appropriate learning rate, show \textit{higher predictability} compared to malicious updates. This hypothesis stems from the fact that the sequence of gradients (thus, model parameters) observed during legitimate training exhibit regular patterns, as validated in Section~\ref{subsec:intuition}. %until convergence. 
%This regularity may be more pronounced for smooth convex loss functions, but it can still be captured within an appropriate time window, even for more complex and convoluted loss surfaces. 
%We provide evidence of this claim in Appendix~B, where we show that the average mutual information (i.e., ``predictability''), calculated over pairs of legitimate model updates sent at different FL rounds, is significantly higher than the corresponding computation for a malicious client.
\\
Inspired by the matrix autoregressive (MAR) framework for multidimensional time series forecasting~\cite{chen2021je}, we propose the FLANDERS ({\em \textbf{F}ederated \textbf{L}earning meets \textbf{AN}omaly \textbf{DE}tection for a \textbf{R}obust and \textbf{S}ecure}) filter.
The main advantages of FLANDERS over existing strategies like FLDetector~\cite{zhao2020multivariate} are its resilience to large-scale attacks, where $50\%$ or more FL participants are hostile, and the capability of working under realistic non-iid scenarios.
We attribute such a capability to two key factors: $(i)$ FLANDERS works without knowing a priori the ratio of corrupted clients, and $(ii)$ it embodies temporal dependencies between intra- and inter-client updates, quickly recognizing local model drifts caused by evil players. Below, we summarize our main contributions:

\begin{itemize}
\item[{\em(i)}]
We provide empirical evidence that the sequence of models sent by legitimate clients is more predictable than those of malicious participants performing untargeted model poisoning attacks.
\\
\item[{\em(ii)}] 
We introduce FLANDERS, the first pre-aggregation filter for FL robust to untargeted model poisoning based on multidimensional time series anomaly detection.
\\
\item[{\em(iii)}] 
We integrate FLANDERS into Flower,\footnote{\scriptsize{\url{https://flower.dev/}}} a popular FL simulation framework for reproducibility.
\\
\item[{\em(iv)}] 
We show that FLANDERS improves the robustness of the existing aggregation methods under multiple settings: different datasets, client's data distribution (non-iid), models, and attack scenarios.
\\
\item[{\em(v)}] 
We publicly release all the implementation code of FLANDERS along with our experiments.\footnote{\scriptsize{\url{https://anonymous.4open.science/r/flanders_exp-7EEB}}}
\end{itemize}

% Paper's structure and organization
The remainder of the paper is structured as follows. %some related work and the current state-of-the-art solutions to security issues that FL entails. 
Section~\ref{sec:background} covers background and preliminaries. 
In Section~\ref{sec:related}, we discuss related work.
Section~\ref{sec:problem} and Section~\ref{sec:method} describe the problem formulation and the method proposed. % to tackle it. 
Section~\ref{sec:experiments} gathers experimental results. %, and Section~\ref{sec:limitations} discusses some limitations of this work.
Finally, we conclude in Section~\ref{sec:conclusion}.
 %discusses the limitations of this work and draws future research directions.
%reports conclusions and draws perspectives for future research directions.

%%%%%%% OLD %%%%%%%
%to overcome the resilience of Byzantine failures in distributed Stochastic Gradient Descent computations. 
% The strength of Krum is its time complexity, which is linear in the gradient dimension. 
% However, the robustness of the approach is guaranteed for gradient-based learning applications only when the majority of the clients are not compromised. 
% Besides, the aggregation mechanism of Krum, as well as that of similar methods, is robust from a coarse-grained perspective and does not provide solutions to errors and perturbations that may occur at inference time.
%A related approach to~\cite{blanchard2017nips} is the work of Su et al.~\cite{su2016dc}. Here, the authors propose an iterated approximate agreement to tackle a multi-layer scenario attacked by Byzantine agents. 
%However, the method works efficiently on the sole discrete context and it is inapplicable to continuous state environments.
%\gabri{Maybe, we should just talk about the main limitations of existing countermeasures without digging into their details (or, we can just mention Krum as this is the most popular one). I will move the description of all these methods to the Related Work section.}
\section{Method}
\label{sec:method}

% \ml{``Inconsistent'' to ``large variation''}

% In this section, we propose our methods based on the observations in Section \ref{sec:motivation}.
In this section, we propose two techniques to further enhance the strong baseline to capture the variation of activation distributions better.
We first introduce spatial re-scaling to adapt the network to pixel-to-pixel variation.
We then propose channel-wise shifting and re-scaling to better capture the channel-to-channel variation.
Meanwhile, as both of the two methods are image-dependent, the image-to-image variation can be captured naturally.
By combining the two methods with our strong baseline, we build our enhanced BNN for SR, named EBSR.

% Because the activation distributions among pixels, channels and images have large variations \red{**are highly inconsistent} in SR networks, we introduce spatial re-scaling to adapt to pixel-wise variations and channel shift and re-scaling to adapt to channel-wise variations. And both of them are image-dependent to adapt to image-wise variations, which means during inference our network re-scales and shifts the distributions of activations flexibly for different input images. Based on these methods, we build an enhanced binary neural network for image super-resolution (EBSR).

% According to [3], the difference of activation magnitudes indicates different scaling factors are needed for each pixel.

\subsection{Spatial Re-scaling}
% It is better to use different scaling factors for different pixels to reduce the quantization error and retain more detailed information for image super-resolution. 

% \ml{In the main method, we do not need to introduce the previous works but can focus on introducing our own method. Channel rescaling in Real-to-binary Net is not relevant in this context.}

% Re-scaling the output of binary convolutions was proposed at the birth of BNN in XNOR-Net \cite{rastegari2016xnor} to reduce quantization error and improve accuracy for image classification tasks.
% It is computed as below:
% \begin{equation}
% \mathcal{A} * \mathcal{W} \approx(\operatorname{sign}(\mathcal{A}) \circledast \operatorname{sign}(\mathcal{W})) \odot \mathcal{K} \alpha
% \label{eq:xnor-net rescale}
% \end{equation}
% where $\circledast$ denotes the binary convolution and $\odot$ denotes the element-wise multiplication.
% $\mathcal{A}$, $\mathcal{W}$, $\alpha$, and $\mathcal{K}$ denote the activation, weight, weight scaling factor, and activation scaling factor, respectively.
%  Later in XNOR-Net++ \cite{bulat2019xnor}, Bulat et al. fuse the activation and weight scaling factors into a single one that is learned end-to-end based on gradients and this improves the classification accuracy on ImageNet dataset.

% % It is computed as Eq.~\ref{eq:xnor-net rescale}, where $\circledast$ denotes 
% %  the binary convolution and $\odot$ denotes the element-wise multiplication. The binary convolution of $\mathcal{A}$ and $\mathcal{W}$ is rescaled by the weight scaling factor $\alpha$ and the activation scaling factor $\mathcal{K}$, both of which are calculated analytically.


% \zc{Similarly, you should explain the meaning of A, W and the operators $\circledast$ in the formula}
% Then in Real-to-binary Net \cite{martinez2020training}, Martinez et al. used a data-driven channel re-scaling module that takes the pre-convolution activations as input to predict the activation scaling factor. Unlike that in XNOR-Net++ \cite{bulat2019xnor}, these scaling factors are not fixed during inference but rather inferred from data. By doing this, they further improved the classification accuracy on ImageNet over XNOR-Net++. 
As is shown in Figure \ref{fig:pixel}, activation distributions have large pixel-to-pixel variation in SR networks
and the difference of activation magnitudes indicates different scaling factors are preferred for different pixels.
Inspired by \cite{martinez2020training}, we propose spatial re-scaling to better adapt the network to the spatial variation
of activation distributions in SR networks.
% fit the various pixel-wise distributions in SR networks.
We take the real-valued activations $A$ before convolution as input and predict pixel-wise scaling factors $S(A)$, which re-scale the binary convolution output. Spatial re-scaling process can be formulated as follows:
\begin{equation}
A * W \approx(\operatorname{sign}(A) \circledast \operatorname{sign}(W)) \odot \alpha \odot S(A)
\label{eq:spatial rescale}
\end{equation}
where $\circledast$ denotes 
the binary convolution and $\odot$ denotes the element-wise multiplication. $A$, $W$, $\alpha$, and $S\left(A\right)$ denote real-valued activations, weights, the scaling factor of weights, and the spatial-wise scaling factor of activations respectively. $S\left(A\right) \in \mathbb{R}^{1\times H\times W}$ can be calculated with a convolution and a sigmoid function.
% as $\sigma\left( CONV\left(A\right)\right)$. 
As shown in Figure \ref{fig:method}(a), real-valued activations first go through a convolution layer,
which has an input channel of $C$ and an output channel of 1, 
and then pass through a sigmoid function to produce the scaling factors $S(A)$ along the spatial dimension.
During inference, the scaling factor will change dynamically according to different input feature maps.
By re-scaling binary convolution output using $S(A)$, we can reduce the quantization error and the original pixel-wise information in FP activation
will be preserved much better.
Spatial re-scaling leads to a large PSNR improvement of 0.24 dB (from 30.30 dB to 31.54 dB) on Set5 and 0.22 dB (from 25.09 dB to 25.31 dB)
on Urban100 compared with our strong baseline. 

\subsection{Channel-wise Shifting and Re-scaling}

\begin{table}[!tb]
\centering
\caption{Comparison between whether to fuse channel-wise shifting and re-scaling or not based on our baseline with spatial re-scaling. }
\label{tab:fusing}

\scalebox{0.65}{
\begin{tabular}{c|cc|cc|cc}
\hline
\multirow{2}{*}{Method}     & \multirow{2}{*}{OPs} & \multirow{2}{*}{Params} & \multicolumn{2}{c|}{Set5} & \multicolumn{2}{c}{Urban100} \\ \cline{4-7} 
                            &                      &                         & PSNR        & SSIM        & PSNR          & SSIM         \\ \hline
Baseline + spatial re-scale & 2.16G                & 0.05M                   & 31.54       & 0.883       & 25.31         & 0.759        \\
+ channel-wise shift and re-scale             & 2.34G                & 0.09M                   & 31.61       & 0.885       & 25.35         & 0.761        \\
+ w/ fusing                   & 2.27G                & 0.08M                   & \textbf{31.64}       & \textbf{0.885}       & \textbf{25.36}         & \textbf{0.761}        \\ \hline
\end{tabular}
}
\end{table}

In SR networks, activation distributions exhibit larger channel-to-channel variation (Figure \ref{fig:chl}).
Both the mean and magnitude of the activation distributions vary significantly across channels.
% Thus we use channel-wise shifting and re-scaling to adapt to various channel-wise distributions. 
\cite{martinez2020training} has proposed the data-driven channel re-scaling, 
but our method differs from them in further introducing data-driven thresholds to handle the channel-wise variation of both mean and magnitude.
Since the blocks to generate the scaling factors and thresholds are very similar, we further propose to fuse them into one module.
% and fusing channel-wise shifting and re-scaling into one module.
We evaluate the effect of fusing the two blocks in Table \ref{tab:fusing}.
With channel-wise shifting and re-scaling fused, our models have fewer operations and parameters overhead and slightly higher performance.

For the specific process, we take the real-valued activations as input and predict different thresholds and scaling factors for each channel. They are also image dependent, e.g., $\beta_{i}$ in Eq.\ref{eq:act_binarize} is no longer fixed during inference but generated according to different input feature maps. Channel-wise shifting and re-scaling can be formulated as follows:
\begin{equation}
A * W \approx(\operatorname{sign}(A-C_s(A)) \circledast \operatorname{sign}(W)) \odot \alpha \odot C_r(A)
\label{eq:channel-wise_shift_and_rescale}
\end{equation}
where $\circledast$ denotes 
the binary convolution and $\odot$ denotes the element-wise multiplication. $C_s(A), C_r(A) \in \mathbb{R}^{C\times1\times1}$ denote the channel-wise threshold and scaling factor, respectively. 
We show the block diagram in Figure \ref{fig:method}(b).
The real-valued input feature map is first squeezed to a ${C\times1\times1}$ vector by a global average pooling (GAP) layer.
The subsequent fully connected layers and ReLU learn the channel-wise information and output a ${2C\times1\times1}$ vector.
Then the ${2C\times1\times1}$ vector is split into two ${C\times1\times1}$ vectors.
We use the first $C$ channels as the channel-wise bias and pass the last $C$ channels through a sigmoid layer 
as the channel-wise scaling factor, which are used to shift the real-valued activations and re-scale the binary convolution output, respectively. 


% \ml{We can mention previously, channel-wise re-scale has been proposed. We propose to fuse them. Add the comparison between fuse v.s. no fuse.}

\begin{figure}[!tbp]%
  \centering
    \includegraphics[width=0.4\textwidth]{fig/methods.png}
  
% \subfloat[channel-wise shifting\&re-scale]{
%     \label{subfig:channel-wise shifting and re-scale}
%     \includegraphics[width=0.2\textwidth]{fig/chl shift and rescale.png}
%   }

  \caption{Block diagram for spatial re-scaling, and channel-wise shifting and re-scaling.} 
  % Input A is the real-valued activation tensor and C, H, and W denote its dimension. GAP stands for global average pooling. The reduction ratio r is set to 16 for a better trade-off between the performance and the number of operations and parameters.}
  \label{fig:method}
\end{figure}


\subsection{Network Structure}

Combining the spatial re-scaling and the channel-wise shifting and re-scaling methods, we construct the enhanced convolution layer (E-Conv).
Then we build our EBSR model based on E-Conv.
In Figure \ref{fig:E-conv}, we compare the binary convolution layer used in the baseline network and our proposed E-Conv.
We use spatial and channel-wise scaling factors to re-scale the binary convolution output,
and use channel-wise shifting to learn appropriate thresholds for each channel before binarization.
The scaling factors and threshold used in E-Conv are learnable and depend on the real-valued input activations.
In this way, our proposed EBSR can adapt to pixel-to-pixel, channel-to-channel, and image-to-image variations
to reduce the large binarization error and preserve more details.
% In this way, our proposed E-Conv reduces the large quantization error caused by binarization and keeps the original information of input feature maps to a large extent.


\begin{figure}[!tb]%
  \centering

    \includegraphics[width=0.5\textwidth]{fig/E-conv.png}

  \caption{Comparison of (a) the binary convolution layer with a skip connection used in our baseline network and (b) the proposed E-Conv.}
  \label{fig:E-conv}
\end{figure}


Figure \ref{fig:network} shows the basic block based on the E-Conv and our EBSR composed of the basic blocks. Following existing works, the convolution layers in the head and tail modules are not binarized. We choose the lightweight EDSR which has 16 basic blocks and 64 channels, and EDSR which has 32 basic blocks and 256 channels as our backbones, which correspond to EBSR-light and EBSR, respectively.

\begin{figure}[!tb]%
  \centering
  {
    \includegraphics[width=0.35\textwidth]{fig/network.png}
  }
  
  \caption{The structure of our proposed EBSR.  Convolution layers in purple are real-valued vanilla 3x3 convolutions.}
  \label{fig:network}
\end{figure}
\section{Experimental Results}
\label{sec:experiments}
\subsection{Training Details}
\cite{Kalantari2017DeepHD} provides the first dataset specifically designed for multi-exposure HDR fusion under large motion. It consists of 74 training sets, which we use to supervise the training of our model. We crop the input images to patches of size \(256 \times 256\) at a step size of 64. This totally generates 20128 training samples. To augment training samples, we randomly rotate and flip the training images. The training adopts Adam optimizer. The learning rate is initialized to \(10^{-4}\) and is reduced to \(10^{-5}\) after 20 epochs. It is observed that 40 epochs are sufficient for the training to converge.    

\subsection{Numerical Evaluation}
We numerically measure the performance of our method on the 15 test sets of \cite{Kalantari2017DeepHD}, by Peak Signal-to-Noise Ratio (PSNR) and Structure Similarity, computed in both tonemapping domain (-\(\mu\)) and HDR linear domain (-L). Visual difference metric HDR-VDP-2 is also adopted, where the parameters are set as same as in previous works \cite{wu2018end} and \cite{niu2021hdrgan}. 

Table \ref{table_metrics} compares our model with state-of-the-art models. For \cite{yan2020nonlocal} and \cite{xiong2021hierarchical}, we use the results reported in the publications. Note that \cite{sen2012robust} and \cite{hu2013hdr} are not machine learning based methods. Moreover,  \cite{Kalantari2017DeepHD} and \cite{wu2018end} apply optical flow and homography transformation to preprocess the input images respectively, and hence entail extra computation. 

Table \ref{table_metrics} shows that our method outperforms competing method in terms of PSNR-L, SSIM-$\mu$, SSIM-L and HDR-VDP-2. It ranks the second best in PSNR-$\mu$, being slightly (0.1dB) inferior to \cite{xiong2021hierarchical}. Note that \cite{xiong2021hierarchical} utilizes a pretrained model to detect ghosting regions for training, whereas our method does not require any pretrained model. The high PSNR and SSIM scores varify that our model has strong HDR reconstruction ability and can accurately restore the radiance and structure of the scene in both tonemapping domain and HDR linear domain. Furthermore, its high performance in term of HDR-VDP-2\cite{mantiuk2011hdr} performance indicates that our method can generate HDR image visually close to the target image.

\begin{table*}[ht]
\centering
\begin{tabular}{l|c|c|c|c|c}
\hline
& PSNR-$\mu$ & PSNR-L & SSIM-$\mu$ & SSIM-L & HDR-VDP-2 \\
\hline
\bfseries Sen & 40.97 & 38.36 & 0.9830 & 0.9746 & 60.60\\
\hline
\bfseries Hu  & 35.65 & 30.80 & 0.9725 & 0.9491 & 58.34\\
\hline
\bfseries Kalantari & 42.69 & 41.22 & 0.9888 & 0.9845 & 65.05\\
\hline
\bfseries DeepHDR& 41.99 & 41.22 & 0.9878 & 0.9859 & \underline{65.91}\\
\hline
\bfseries AHDR & 43.62 & 41.03 & 0.9900  &\underline{0.9883} & 63.85 \\
\hline 
\bfseries NHDRRNet& 42.414 & - & 0.9887 & - & 61.21 \\
\hline 
\bfseries HDR-GAN &43.92 & \underline{41.57} &\underline{0.9905} &0.9865 & 65.45\\
\hline 
\bfseries HFNet & \textbf{44.28} & 41.47 & - & - & - \\
\hline 
\bfseries Ours & \underline{44.18} & \textbf{42.19}&\textbf{0.9912} & \textbf{0.9883}& \textbf{67.07} \\
\hline
\end{tabular}
\caption{Numerical performance of the proposed model, evaluated on the dataset by Kalantari-Ramamoorthi. The best and second best results for each metric are marked in \textbf{bold} and \underline{underlined}, respectively}
\label{table_metrics}
\end{table*}

\subsection{Visual Performance Evaluation}

\begin{figure*}[!htb]
\centering
\includegraphics[width=\textwidth]{experiments/kalantari_test.png}
\caption{Visual comparison on the test set of Kalantari-Ramamoorthi dataset. Zoom-in views of reconstruction by each method are presented on the saturated regions that contain moving objects. Our network built with gated Swin Transformer yields noticeably better visual results than other methods.}
\label{fig_kalantari_test}
\end{figure*}
Fig. \ref{fig_kalantari_test} present the visual performance of our method and comparable methods on two examples from \cite{Kalantari2017DeepHD}. We present the zoom-in views of two challenging cases, where large saturated regions contain substantial non-rigid motion in the reference image. The two patch-based methods do not reconstruct the missing details in the saturated regions, as they heavily rely on the details provided by the reference image for registration. Image reconstructed by the optical flow based method \cite{Kalantari2017DeepHD} suffers motion blur artifacts. This is because the convolutions of DeepHDR and HDR-GAN have limited receptive fields, and hence are hampered to repair missing content in misaligned regions by aligned regions. The gating mechanism of AHDR is only applied to low-level features, so the high-level outliers may deteriorate the HDR fusion. In contrast to comparable methods, our model remarkably overcomes the ghosting artifacts.

\begin{figure}[ht]
\centering
\includegraphics[width=\columnwidth]{experiments/sen_test.pdf}
\caption{Visual performance comparison on example images from the dataset by Sen et al. Zoom in views on challenging areas are presented. Although the ground truth is unavailable, it can be clearly observed that our method visually performs better than comparable methods.}
\label{sen_test}
\end{figure}

\begin{figure}[ht]
\centering
\includegraphics[width=\columnwidth]{experiments/tursun_test.pdf}
\caption{Visual performance comparison on example images from the dataset by Tursun et al. Compared to state of the art methods, our method suffers less ghosting artifact.}
\label{tursun_test}
\end{figure}

Fig.\ref{sen_test} and Fig.\ref{tursun_test} present visual performance of our method on two examples from benchmark datasets \cite{sen2012robust} and \cite{tursun2016objective}. As these test datasets   do not provide ground truth image. we mark the visual difference on the results generated by different methods. It can be seen that our method suffers less artifacts than other methods in various scenes with various motion patterns, achieving better visual results. Our method creates high-quality HDR more robustly and generalizes well. 

\subsection{Ablation Study}

\begin{table}[h]
\centering
\resizebox{\columnwidth}{!}{
\begin{tabular}{l|c|c|c|c|c}
\hline
                         & PSNR-$\mu$ & PSNR-l & SSIM-$\mu$ & SSIM-l & HDR-VDP-2 \\ \hline
restormer(w/o ssim loss) & 44.00  & 41.5   & 0.9906 & 0.9873 & 64.72  \\ \hline
Ours(w/o ssim loss)      & 44.07  & 41.83  & 0.9909 & 0.9879 &  64.78  \\ \hline
Ours                     & 44.18  & 42.19  & 0.9912 & 0.9883 & 67.07      \\ \hline
\end{tabular}
}
\caption{Experimental results of ablation study. We compare using Gated Swin Transformer v.s. Gated Transformer, and the combined loss function v.s. the traditional $l_{1}$ norm loss function.}
\label{table_ablation_block_loss}
\end{table}

We verify various components of our method, including Swin Transformer, loss function, and gating mechanism by ablation study.

\subsubsection{Ablation Study on Block Design}
Our model has similar architecture to Restormer, which uses modified Transformer, whereas we use modified Swin Transformer as the building unit. For comparison, we replace the residual modules in each block in our model with multiple transformer layers as in Restormer, with same number of transformer layers. Table \ref{table_ablation_block_loss} presents the results, which show that using Swin Transformer achieves superior performance in all measures. The reason is that the attention module of Restormer is computed channel-wise, but forgoes the cross-exposure spatial dependency to repair the non-aligned area. 

\subsubsection{Ablation Study on Loss Function}
We trained our model under different loss function configurations, as shown in \ref{table_ablation_block_loss}. The results validate that the SSIM loss benefits detail reconstruction.

\subsubsection{Ablation Study on Gating Mechanism}
\begin{table}[h]
\resizebox{\columnwidth}{!}{
\begin{tabular}{l|c|c|c|c|c}
\hline
           & PSNR-$\mu$ & PSNR-l & SSIM-$\mu$ & SSIM-l & HDR-VDP-2 \\ \hline
w/o gating & 43.14  & 41.03  & 0.9904 & 0.9868 &     64.88      \\ \hline
one gating & 43.44  & 41.42  & 0.9909 & 0.9882 &    67.13   \\ \hline
Ours       & 43.61  & 41.74  & 0.9909 & 0.9881 & 66.96     \\ \hline
\end{tabular}
}
\caption{Ablation experimental results to verify the effectiveness of the gating mechanism}
\label{table_ablation_gating}
\end{table}

The gating mechanism is an important component in our model. Ablation study is conducted in the gating mechanism as follows.

\textbf{w/o gating}: The gating mechanism is not used in the feed forward network of all transformer layers in the model, that it, our GST unit degenerate to the vanilla Swin Transformer.

\textbf{one gating}: The gating mechanism is only used in the first Swin Transformer layers subsequent to the embedding layer, but not used for other layers. 

 Table \ref{table_ablation_gating} shows the results of the ablation experiments, where the model is trained for 20 epochs. By removing the gating mechanism, the network relies on self-attention for image alignment, resulting in the lowest performance. On top of it, adding gates to low level layers notably improves the HDR reconstruction. Furthermore, by integrating the gating mechanism with all Swin Transformer layers, the model effectively inpaints information in non-aligned regions and obtains the highest HDR reconstruction results, thus validates the effectiveness of the gating mechanism in our model.

\section{Related work}
\noindent \textbf{Video foundation models.}
With sufficient computational power and an abundant source of data, there have been attempts to build a single large-scale foundation model that can be adapted to diverse downstream tasks.
Along with the success of foundations models in the natural language processing domain~\cite{brown2020language,chen2021evaluating,devlin2019bert} and in computer vision~\cite{bertasius2021space,jia2021scaling,radford2021learning}, video data has become another data type of interest, as it has grown in scale due to numerous internet video-sharing platforms.
Accordingly, several methods to train a video foundation model have been proposed.
Due to the innate multi-modality of video data, \textit{i.e.}, a combination of visual $\cdot$ vocal $\cdot$ textual context, most works have centered around the variations of the cross-modal attention mechanism \cite{akbari2021vatt,bertasius2021space,gabeur2020multi,luo2020univl,neimark2021video,tan2021look,wei2020multi,yang2021taco}.
In addition, as most video data lack proper labels or descriptions, contrastive learning methods were studied to learn meaningful feature representations or enhance video-text alignment in a self-supervised manner \cite{akbari2021vatt,kuang2021video,luo2020univl,yang2021taco}.

More specifically, MERLOT \cite{zellers2021merlot} proposed a multi-modal representation learning method for visual commonsense reasoning, which also performed well in twelve video reasoning tasks.
VATT \cite{akbari2021vatt} introduced a multi-modal learning method via contrastive learning. 
The pre-trained model performed well in a variety of vision tasks from image classification to video action recognition and zero-shot video retrieval.
Another representative work, UniVL \cite{luo2020univl} proposed a straightforward pre-training method with auxiliary loss functions. 
After fine-tuning on a specific task, the pre-trained model performed outstandingly in a wide range of tasks of text-to-video retrieval, action segmentation, action step localization, video sentiment analysis, and video captioning.
Other foundation models for multiple video tasks include \cite{li2020hero,sun2019learning,sun2019videobert,zhu2020actbert,fu2021violet,wang2022all}. 

\noindent \textbf{Auxiliary learning.}
In order to enhance the performance of one or a multitude of primary tasks, auxiliary learning methods can be incorporated.
\cite{ruder2017overview} introduced Multi-task learning (MTL) to the deep neural networks by training a single model with multiple task losses to assist learning on the main task.
Such a method is generally adapted to pre-train the foundation models in the self-supervised manner~\cite{li2020hero,sun2019learning,sun2019videobert,zhu2020actbert,fu2021violet,wang2022all}.
However, these various pretext task losses used in the pre-training phase are ignored in the fine-tuning phase, and only the primary task loss is minimized.

Recently, meta-learning methods have been introduced for auxiliary learning.
\cite{liu2019self,navon2020auxiliary,shu2019meta} proposed a meta-learning method in which the model learns auxiliary tasks to generalize well to unseen data. 
In these settings, a separate subset of data is held out as the primary task, while the others are used as auxiliary tasks that aid the primary task's performance.
Similar methods were adopted for computer vision tasks such as semantic segmentation \cite{xu2021leveraging}.
Other domain applications include navigation tasks with reinforcement learning \cite{ye2021auxiliary}, or self-supervised learning methods on graph data \cite{hwang2020self}.
\section{Conclusion}\label{sec:conclusion}
In this work, we focus on addressing the fundamental challenge of OOD detection tasks, which is how to fully understand the semantic discrepancy between the ID/OOD samples. We reveal that the key to success in the realistic SCOOD task is to allocate as many ID samples in the unlabeled set correctly as possible. To this end, we propose a novel uncertainty-aware optimal transport scheme that introduces class-specific energy scores as guidance for effective label assignment. Experimental results show that our method achieves better performance than previous state-of-the-art methods on SCOOD benchmarks.

\textbf{Limitations.} In addition to temperature scaling, other techniques such as feature clipping applied in ReAct~\cite{sun2021react} also enhance the performance of energy score, so how to obtain an OOD score that best fits the SCOOD task can be further explored. Moreover, a setting highly related to SCOOD has been proposed in \cite{katz2022training} and formulated as a constrained optimization problem. We will also theoretically analyze these practical OOD settings in our feature work.

% \section*{Acknowledgments}
\textbf{Acknowledgments.} 
This work is supported by National Key R\&D Program of China under Grant 2020AAA0105701, National Natural Science Foundation of China (NSFC) under Grants 61872327, Major Special Science and Technology Project of Anhui, National Natural Science Foundation of China (62033012) and Ant Group through Ant Research Intern Program.

% \section{Limitations and Future Work}

We summarize the limitations we have identified for our method and propose
future research directions.

\textbf{Parallel implementation:} 
With a focus on accuracy and algorithms, our implementation for this work is
serial. Some of the most time-consuming routines in our method can easily
benefit from a parallel implementation, while the same is not obvious for the
SAP solver and the Schur complement computation. Leveraging the power of
parallelization on modern hardware for these computations is an interesting area
for future investigation.

\textbf{Rotational invariance:} 
As with all other linear constitutive models, our linearized model with lagged
rotational component is not rotationally invariant. Thus it is not suitable for
simulation of extreme deformations using large time steps. For those scenarios,
we fall back to traditional nonlinear models with Hessian positive definite
corrections proposed in \cite{bib:teran2005robust}.

\textbf{Self-contact:} 
We do not consider self-contact at the moment due to the lack of support by our
geometry engine. Self-contact can be incorporated into our method by updating the
geometry engine to augment the set of contacts reported.

\textbf{Tunneling at high speeds:} Though our method has a lower computational
cost, it could benefit from continuous collision detection strategies
\cite{bib:li2020ipc} to provide constraints before contact is established. This
would allow to mitigate issues such as objects tunneling past each other at high
speeds. Efficient solution to mitigate this issue is a topic of active research
for the authors.

\textbf{Redundant constraints:} Our geometry engine often introduces a large
number of constraints to resolve contact. Similarly, welding a large number of
deformable mesh vertices to a rigid body (as done in Section
\ref{sec:bubble_gripper}) introduces many constraints. Even though our SAP
solver \cite{bib:castro2022unconstrained} provides existence and uniqueness
guarantees, a large number of constraints hurts performance as can be observed
in the \emph{Soft-bubble} example. We are currently investigating strategies to
significantly reduce the number of constraints without sacrificing accuracy.




% \section*{Limitations}
% EMNLP 2023 requires all submissions to have a section titled ``Limitations'', for discussing the limitations of the paper as a complement to the discussion of strengths in the main text. This section should occur after the conclusion, but before the references. It will not count towards the page limit.  

% The discussion of limitations is mandatory. Papers without a limitation section will be desk-rejected without review.
% ARR-reviewed papers that did not include ``Limitations'' section in their prior submission, should submit a PDF with such a section together with their EMNLP 2023 submission.

% While we are open to different types of limitations, just mentioning that a set of results have been shown for English only probably does not reflect what we expect. 
% Mentioning that the method works mostly for languages with limited morphology, like English, is a much better alternative.
% In addition, limitations such as low scalability to long text, the requirement of large GPU resources, or other things that inspire crucial further investigation are welcome.

% \section*{Ethics Statement}
% Scientific work published at EMNLP 2023 must comply with the \href{https://www.aclweb.org/portal/content/acl-code-ethics}{ACL Ethics Policy}. We encourage all authors to include an explicit ethics statement on the broader impact of the work, or other ethical considerations after the conclusion but before the references. The ethics statement will not count toward the page limit (8 pages for long, 4 pages for short papers).

% \section*{Acknowledgements}
% This document has been adapted by Yue Zhang, Ryan Cotterell and Lea Frermann from the style files used for earlier ACL and NAACL proceedings, including those for 
% ACL 2020 by Steven Bethard, Ryan Cotterell and Rui Yan,
% ACL 2019 by Douwe Kiela and Ivan Vuli\'{c},
% NAACL 2019 by Stephanie Lukin and Alla Roskovskaya, 
% ACL 2018 by Shay Cohen, Kevin Gimpel, and Wei Lu, 
% NAACL 2018 by Margaret Mitchell and Stephanie Lukin,
% Bib\TeX{} suggestions for (NA)ACL 2017/2018 from Jason Eisner,
% ACL 2017 by Dan Gildea and Min-Yen Kan, NAACL 2017 by Margaret Mitchell, 
% ACL 2012 by Maggie Li and Michael White, 
% ACL 2010 by Jing-Shin Chang and Philipp Koehn, 
% ACL 2008 by Johanna D. Moore, Simone Teufel, James Allan, and Sadaoki Furui, 
% ACL 2005 by Hwee Tou Ng and Kemal Oflazer, 
% ACL 2002 by Eugene Charniak and Dekang Lin, 
% and earlier ACL and EACL formats written by several people, including
% John Chen, Henry S. Thompson and Donald Walker.
% Additional elements were taken from the formatting instructions of the \emph{International Joint Conference on Artificial Intelligence} and the \emph{Conference on Computer Vision and Pattern Recognition}.

% Entries for the entire Anthology, followed by custom entries
\bibliography{anthology,custom}
\bibliographystyle{acl_natbib}


\clearpage

\appendix


\section{Prompts for Generating Explanations}\label{sec.appendix.a}
\begin{table*}[b] 
  \centering
 \footnotesize
%   \scriptsize
  % \tiny
   \begin{tabular}{
    m{0.1\textwidth}<{\centering}|
    m{0.84\textwidth}
    }
    \toprule
    \textbf{Input}  & Directions: Given a search engine query: "\textbf{google data studio sharepoint}", first, consider what the user could have in mind when they type in the query and allow for misspellings or other ambiguity, then classify the relevance of keyword: "\textbf{sharepoint migration tool file share}" to the query into one of the following categories: "Not bad", "Bad". The definitions of the categories are\newline
"Not bad": the keyword is relevant to the user's search query. This can include: broader or narrower product selection, competitor or alternative products, accessories, products often purchased together and related topics as well as direct matches to the user's search.\newline
"Bad": the keyword is not relevant to the user's search query. There is no relationship between the query and keyword. This includes but is not limited to: incorrect/unrelated product, unrelated topic, wrong location when location is important, cannot be used in place of query product nor are they commonly purchased together, etc.\newline
Briefly explain why the relevance is "\textbf{Bad}", with a response length not exceeding 100 words.   \\
    \midrule
    \textbf{Output 1}  & The relevance of the keyword "sharepoint migration tool file share" to the search query "google data studio sharepoint" is "Bad". This keyword is not relevant to the user's search query as it pertains to a different tool or technology (SharePoint) that is not mentioned in the search query. The keyword focuses on a migration tool for SharePoint file sharing, while the search query is focused on Google Data Studio. There is no relationship between the two, so the keyword is not relevant to the user's search query.\\
    \midrule
    \textbf{Output 2}  & The relevance of "sharepoint migration tool file share" to the search query "google data studio sharepoint" is "Bad." The keyword is not relevant to the user's search query because it deals with SharePoint migration and file sharing, whereas the user's query is focused on Google Data Studio and SharePoint. There is no relationship between these two topics, and the keyword cannot be used in place of the query product.\\
    \midrule
    \textbf{Output 3}  & The relevance is "Bad". The keyword "sharepoint migration tool file share" is not directly related to the query "google data studio sharepoint". The keyword refers to tools and services for migrating data from SharePoint to another file share platform, whereas the query is about using Google Data Studio to connect with SharePoint data. There is no clear relationship between the two topics.\\
    \midrule
    \textbf{Output 4}  & The relevance of the keyword "sharepoint migration tool file share" to the search query "google data studio sharepoint" is "Bad". The keyword is about a tool for migrating data from SharePoint, which is not directly related to Google Data Studio. Google Data Studio is a data visualization and reporting tool, not a migration tool for SharePoint.\\
    \midrule
    \textbf{Output 5}  & The relevance is "Bad" as the keyword "sharepoint migration tool file share" has no direct relation to the user's search query "google data studio sharepoint". The keyword pertains to a tool used for migrating data within a SharePoint platform, while the query is focused on Google Data Studio and SharePoint. These two concepts are not directly related and therefore, the keyword is not relevant to the user's search query.\\
    \bottomrule
 \end{tabular}
 \caption
 { Illustration of how to use a LLM to generate explanations for the QK task. 
``Input'' and ``Output'' refer to the prompt provided to the LLM and generated explanations, respectively.}\label{tab.qk.explanation} 
\end{table*} 


\begin{table*}[hbp!]
  \centering
 \footnotesize
  % \scriptsize
  % \tiny
   \begin{tabular}{
    m{0.1\textwidth}<{\centering}|
    m{0.8\textwidth}
    }
    \toprule
    \textbf{Input}  & Task description: The task is to identify if the occurrences of "place" in `\textbf{Do you want to come over to my "place" later?}' and `\textbf{A political system with no "place" for the less prominent groups.}' correspond to the same meaning or not. The target word "place" in the first sentence and the second sentence is enclosed in quotation marks. There is no need to consider the tense and form of "place". If "place" in the first sentence and the second sentence correspond to the same meaning, output "true", otherwise output "false".\newline
Briefly explain why the answer is "\textbf{false}", with a response length not exceeding 100 words.\\
    \midrule
    \textbf{Output 1}  & The answer is "false". In the first sentence, "place" refers to a physical location or a person's home. In the second sentence, "place" refers to a position or status in a political system. Therefore, although the spelling and punctuation of "place" are the same in both sentences, the context and meaning of the word are different.\\
    \midrule
    \textbf{Output 2}  & The answer is "false". In the first sentence, "place" refers to a physical location, meaning a house or an apartment. In contrast, in the second sentence, "place" refers to a figurative location, meaning a position or role within a political system. Therefore, the two occurrences of "place" have different meanings and do not correspond to the same meaning.\\
    \midrule
    \textbf{Output 3}  & The answer is "false". In the first sentence, "place" refers to a physical location or someone's home, while in the second sentence, "place" refers to a position or status in a political system. These are different meanings, and therefore, the occurrences of "place" in the two sentences do not correspond to the same meaning.\\
    \midrule
    \textbf{Output 4}  &The answer is "false". In the first sentence, "place" refers to a physical location, as in a person's home. In contrast, in the second sentence, "place" refers to a position or status within a system or society, as in a group's role or standing. Thus, the two uses of "place" have different meanings, and do not correspond to the same meaning.\\
    \midrule
    \textbf{Output 5}  & The answer is "false". In the first sentence, "place" refers to a physical location where the speaker is inviting someone to come over to. In contrast, in the second sentence, "place" refers to a position or role in a political system. Therefore, the two occurrences of "place" have different meanings and cannot be considered to correspond to the same meaning.
\\
    \bottomrule
 \end{tabular}
 \caption{Illustration of how to use LLM to generate explanations for the WiC task. 
``Input'' and ``Output'' refer to the prompt provided to the LLM and generated explanations, respectively.}\label{tab.wic.1} 
\end{table*} 


\begin{table*}
  \centering
 \footnotesize
  % \scriptsize
  % \tiny
   \begin{tabular}{
    m{0.1\textwidth}<{\centering}|
    m{0.84\textwidth}
    }
    \toprule
    \textbf{Input}  & Task description: Based on the information provided in the passage: "\textbf{The Elder Scrolls Online -- As with other games in The Elder Scrolls series, the game is set on the continent of Tamriel. The events of the game occur a millennium before those of The Elder Scrolls V: Skyrim and around 800 years before The Elder Scrolls III: Morrowind and The Elder Scrolls IV: Oblivion. It has a broadly similar structure to Skyrim, with two separate conflicts progressing at the same time, one with the fate of the world in the balance, and one where the prize is supreme power on Tamriel. In The Elder Scrolls Online, the first struggle is against the Daedric Prince Molag Bal, who is attempting to meld the plane of Mundus with his realm of Coldharbour, and the second is to capture the vacant imperial throne, contested by three alliances of the mortal races. The player character has been sacrificed to Molag Bal, and Molag Bal has stolen their soul, the recovery of which is the primary game objective.}", determine if the answer to the question: "\textbf{is elder scrolls online the same as skyrim}" is "yes" or "no". If the answer is affirmative, indicate "true", otherwise indicate "false".\newline
Briefly explain why the answer is "\textbf{false}", with a response length not exceeding 100 words.   \\
    \midrule
    \textbf{Output 1}  & The answer is "false". Although The Elder Scrolls Online is set on the continent of Tamriel, like Skyrim and other games in The Elder Scrolls series, it takes place a millennium before Skyrim and around 800 years before Morrowind and Oblivion. While it has a similar structure to Skyrim, with two conflicts progressing at the same time, the primary struggle in The Elder Scrolls Online is against the Daedric Prince Molag Bal, who has stolen the player character's soul, and the second is to capture the vacant imperial throne. Therefore, The Elder Scrolls Online is not the same game as Skyrim.\\
    \midrule
    \textbf{Output 2}  & The answer is "false". While The Elder Scrolls Online is set in the same continent of Tamriel as Skyrim, and shares a similar structure with two conflicts happening simultaneously, the events of the game occur a millennium before Skyrim, and its primary antagonist is the Daedric Prince Molag Bal, not Alduin as in Skyrim. The gameplay mechanics and overall experience of The Elder Scrolls Online are also distinct from Skyrim, as it is a massively multiplayer online role-playing game (MMORPG), while Skyrim is a single-player game.\\
    \midrule
    \textbf{Output 3}  & The answer is "false". While The Elder Scrolls Online and Skyrim share the same fictional universe and are part of The Elder Scrolls series, they are not the same game. The Elder Scrolls Online is set a millennium before Skyrim and has a different storyline and gameplay mechanics. In The Elder Scrolls Online, the player fights against Molag Bal and attempts to recover their stolen soul, while in Skyrim, the player character is a Dragonborn who fights against Alduin, the World-Eater. Additionally, The Elder Scrolls Online is a massively multiplayer online game, while Skyrim is a single-player game.\\
    \midrule
    \textbf{Output 4}  &The answer is "false". While The Elder Scrolls Online and Skyrim share some similarities, such as being set in the same continent of Tamriel and having a similar structure, they are different games with different stories and timelines. The events in The Elder Scrolls Online take place a millennium before Skyrim and revolve around two conflicts, one against the Daedric Prince Molag Bal and the other for the vacant imperial throne, while Skyrim has its own unique story centered around the return of the dragons to the world.\\
    \midrule
    \textbf{Output 5}  & The answer is "false". While The Elder Scrolls Online and Skyrim share the same fictional world of Tamriel, they are not the same game. The events of The Elder Scrolls Online take place a millennium before Skyrim, and the game features a different storyline and game structure. In The Elder Scrolls Online, the player character is sacrificed to the Daedric Prince Molag Bal and must recover their soul, while in Skyrim, the player character is a Dragonborn on a quest to defeat the dragon Alduin and save Tamriel from destruction.\\
    \bottomrule
 \end{tabular}
 \caption{Illustration of how to use a LLM to generate explanations for the BoolQ task. 
``Input'' and ``Output'' refer to the prompt provided to the LLM and generated explanations, respectively.
}\label{tab.boolq.1} 
\end{table*} 







\begin{table*}
  \centering
 \footnotesize
  % \scriptsize
  % \tiny
   \begin{tabular}{
    m{0.1\textwidth}<{\centering}|
    m{0.84\textwidth}
    }
    \toprule
    \textbf{Input}  & Directions: Given a search engine query: "\textbf{google data studio sharepoint}", first, consider what the user could have in mind when they type in the query and allow for misspellings or other ambiguity, then classify the relevance of keyword: "\textbf{sharepoint migration tool file share}" to the query into one of the following categories: "Not bad", "Bad". The definitions of the categories are\newline
"Not bad": the keyword is relevant to the user's search query. This can include: broader or narrower product selection, competitor or alternative products, accessories, products often purchased together and related topics as well as direct matches to the user's search.\newline
"Bad": the keyword is not relevant to the user's search query. There is no relationship between the query and keyword. This includes but is not limited to: incorrect/unrelated product, unrelated topic, wrong location when location is important, cannot be used in place of query product nor are they commonly purchased together, etc.\newline
Briefly explain the relevance between the keyword and query, with a response length not exceeding 100 words.  \\
    \midrule
    \textbf{Output 1}  & The user's query "google data studio sharepoint" suggests they are looking for information on how to integrate or connect Google Data Studio with SharePoint, possibly for data visualization or reporting purposes. The keyword "sharepoint migration tool file share" is not directly relevant to the user's query as it pertains to a tool used for migrating files between SharePoint instances, rather than integrating Google Data Studio with SharePoint. Therefore, the keyword is considered "Bad" in relation to the user's query.\\
    \midrule
    \textbf{Output 2}  & When a user searches for "Google Data Studio SharePoint," they are likely looking for information on how to integrate Google Data Studio with SharePoint or how to use Google Data Studio to create reports on SharePoint data. The keyword "SharePoint migration tool file share" is not directly relevant to the query as it pertains to a tool for migrating files to SharePoint rather than using Google Data Studio with SharePoint. Therefore, the keyword is considered "bad" in this context.\\
    \midrule
    \textbf{Output 3}  & \textcolor{red}{When a user searches for "google data studio sharepoint," they are likely looking for information on how to integrate or connect Google Data Studio with SharePoint, a popular document management and collaboration platform. The keyword "sharepoint migration tool file share" is somewhat relevant to the query, as it pertains to migrating files from one SharePoint instance to another, which could be useful information for someone looking to integrate the two platforms. Therefore, I would classify this keyword as "Not bad."}\\
    \midrule
    \textbf{Output 4}  & The search query "google data studio sharepoint" likely suggests that the user is looking for information about how to integrate or use Google Data Studio with SharePoint. They may be interested in creating reports or visualizations using data from SharePoint in Google Data Studio. The keyword "sharepoint migration tool file share" is not relevant to the query as it refers to a tool for migrating files from one SharePoint site to another, and does not address the query's focus on integrating Google Data Studio with SharePoint. Therefore, the keyword is considered "Bad" for this search query.\\
    \midrule
    \textbf{Output 5}  & The search engine query "google data studio sharepoint" suggests that the user may be looking for information on how to integrate or use Google Data Studio with SharePoint, which is a web-based collaborative platform used for document management and storage. The keyword "sharepoint migration tool file share" is not directly relevant to the user's search query, as it is related to a tool used for migrating files between different file-sharing platforms, and does not address the user's original intent of using Google Data Studio with SharePoint. Therefore, the keyword is classified as "Bad."\\
    \bottomrule
 \end{tabular}
 \caption{Illustration of how to use a LLM to generate explanations for the user query and keyword relevance assessment task without using the ground truth labels. 
``Input'' and ``Output'' refer to the prompt provided to the LLM and the generated explanations, respectively.
 The red outputs indicate incorrect generated explanations.
 }\label{tab.qk.explanation_wo_label} 
\end{table*} 





















\clearpage
\section{Zero-shot Prompts}\label{sec.appendix.b}
\begin{table*}[hbp!] 
  \centering
 \footnotesize
  % \scriptsize
  % \tiny
   \begin{tabular}{
    m{0.96\textwidth}
    }
    \toprule
    Given a search engine query, first, consider what the user could have in mind when they type in the query and allow for misspellings or other ambiguity, then classify the relevance of keyword to the query into one of the following categories: "Not bad", or "Bad". The definitions of the categories are\\
    "Not bad": the keyword is relevant to the user's search query. This can include: broader or narrower product selection, competitor or alternative products, accessories, products often purchased together and related topics as well as direct matches to the user's search.\\
    "Bad": the keyword is not relevant to the user's search query. There is no relationship between the query and keyword. This includes but is not limited to: incorrect/unrelated product, unrelated topic, wrong location when location is important, cannot be used in place of query product nor are they commonly purchased together, etc.\\
    Please predict whether the keyword is relevant to the query or not. The answer should be exact "Not bad" or "Bad".\\
    \\
    Query: \textbf{\{query\}}\\
    Keyword: \textbf{\{keyword\}}\\
    Answer: \\
    \bottomrule
 \end{tabular}
 \caption{Zero-shot prompt for the QK task.}\label{tab.qk.zero-shot} 
\end{table*} 

\begin{table*}[hbp!]
  \centering
 \footnotesize
  % \scriptsize
  % \tiny
   \begin{tabular}{
    m{0.96\textwidth}
    }
    \toprule
    The goal of this task is to determine whether the targeted word in the first sentence and the second sentence conveys the same meaning. Please note that if the targeted word appears multiple times in the sentences, only the instance of the word surrounded by quotation marks should be considered. Additionally, the tense and form of the targeted word should not be taken into account. If the targeted word in the first sentence and the second sentence corresponds to the same meaning, output "True"; otherwise, output "False".\\
    To complete this task, you will need to predict whether the targeted word "w" in the first sentence "s1" and the second sentence "s2" convey the same meaning. Your answer should be either "True" or "False".\\
    \\
    w: \textbf{\{target word\}}\\
    s1: \textbf{\{first sentence\}}\\
    s2: \textbf{\{second sentence\}}\\
    Answer: \\
    \bottomrule
 \end{tabular}
 \caption{Zero-shot prompt for the WiC task.}\label{tab.wic.zero-shot} 
\end{table*} 

\begin{table*}[hbp!]
  \centering
 \footnotesize
  % \scriptsize
  % \tiny
   \begin{tabular}{
    m{0.96\textwidth}
    }
    \toprule
    Yes/No question-answering consists of a short passage and a Yes/No question about the passage. The questions are provided anonymously and unsolicited by users of the Google search engine, and afterwards paired with a paragraph from a Wikipedia article containing the answer. If there exists evidence in the passage that supports the facts in the question, the answer should be "Yes". If there exists evidence in the passage that denies the facts in the question, the answer should be "No".\\
    Your task is to read the passage and predict whether the answer to the question is "Yes" or "No".\\
    \\
    Passage: \textbf{\{passage\}}\\
    Question: \textbf{\{question\}}\\
    Answer: \\
    \bottomrule
 \end{tabular}
 \caption{Zero-shot prompt for the BoolQ task.}\label{tab.boolq.zero-shot} 
\end{table*} 

\clearpage
\section{Few-shot Prompts}\label{sec.appendix.c}

\begin{table}[!hbp] 
  \centering
 \footnotesize
  % \scriptsize
  % \tiny
   \begin{tabular}{
    m{0.46\textwidth}
    }
    \toprule
    Given a search engine query, first, consider what the user could have in mind when they type in the query and allow for misspellings or other ambiguity, then classify the relevance of keyword to the query into one of the following categories: "Not bad", or "Bad". The definitions of the categories are\\
    "Not bad": the keyword is relevant to the user's search query. This can include: broader or narrower product selection, competitor or alternative products, accessories, products often purchased together and related topics as well as direct matches to the user's search.\\
    "Bad": the keyword is not relevant to the user's search query. There is no relationship between the query and keyword. This includes but is not limited to: incorrect/unrelated product, unrelated topic, wrong location when location is important, cannot be used in place of query product nor are they commonly purchased together, etc.\\
    Please predict whether the keyword is relevant to the query or not. The answer should be exact "Not bad" or "Bad".\\
    \\
    Query: google data studio sharepoint\\
    Keyword: sharepoint migration tool file share\\
    Answer: Bad\\
    \\
    Query: motorhomes sale\\
    Keyword: rv sale used class c\\
    Answer: Not bad\\
    \\
    Query: southern exposure seed exchange company\\
    Keyword: uk poppy seeds\\
    Answer: Not bad\\
    \\
    Query: nissan parts canada\\
    Keyword: purchase tires\\
    Answer: Bad\\
    \\
    Query: alcohol detoxing\\
    Keyword: inpatient drug rehab\\
    Answer: Not bad\\
    \\
    Query: loudmouth clothing sale\\
    Keyword: levis jeans\\
    Answer: Bad\\
    \\
    Query: firefox mac sierra\\
    Keyword: opera browser mac\\
    Answer: Not bad\\
    \\
    Query: google images\\
    Keyword: buy photo\\
    Answer: Bad\\
    \\
    Query: \textbf{\{query\}}\\
    Keyword: \textbf{\{keyword\}}\\
    Answer: \\
    \bottomrule
 \end{tabular}
 \caption{Few-shot exemplars prompt for the QK task.}\label{tab.qk.few-shot} 
\end{table} 


\begin{table}[!hbp] 
  \centering
 \footnotesize
  % \scriptsize
  % \tiny
   \begin{tabular}{
    m{0.46\textwidth}
    }
    \toprule
    The goal of this task is to determine whether the targeted word in the first sentence and the second sentence conveys the same meaning. Please note that if the targeted word appears multiple times in the sentences, only the instance of the word surrounded by quotation marks should be considered. Additionally, the tense and form of the targeted word should not be taken into account. If the targeted word in the first sentence and the second sentence corresponds to the same meaning, output "True"; otherwise, output "False".\\
    To complete this task, you will need to predict whether the targeted word "w" in the first sentence "s1" and the second sentence "s2" convey the same meaning. Your answer should be either "True" or "False".\\
    \\
    w: "place"\\
    s1: Do you want to come over to my "place" later?\\
    s2: A political system with no "place" for the less prominent groups.\\
    Answer: False\\
    \\
    w: "hold"\\
    s1: The general ordered the colonel to "hold" his position at all costs.\\
    s2: "Hold" the taxi.\\
    Answer: True\\
    \\
    w: "summer"\\
    s1: We like to "summer" in the Mediterranean.\\
    s2: We "summered" in Kashmir.\\
    Answer: True\\
    \\
    w: "approach"\\
    s1: "Approach" a task.\\
    s2: To "approach" the city.\\
    Answer: False\\
    \\
    w: "run"\\
    s1: "Run" rogue.\\
    s2: She "ran" 10 miles that day.\\
    Answer: False\\
    \\
    w: "head"\\
    s1: His horse won by a "head".\\
    s2: He is two "heads" taller than his little sister.\\
    Answer: True\\
    \\
    w: "meet"\\
    s1: The company agrees to "meet" the cost of any repairs.\\
    s2: This proposal "meets" my requirements.\\
    Answer: True\\
    \\
    w: "development"\\
    s1: The organism has reached a crucial stage in its "development".\\
    s2: Our news team brings you the latest "developments".\\
    Answer: False\\
    \\
    w: \textbf{\{target word\}}\\
    s1: \textbf{\{first sentence\}}\\
    s2: \textbf{\{second sentence\}}\\
    Answer: \\
    \bottomrule
 \end{tabular}
 \caption{Few-shot exemplars prompt for the WiC task.}\label{tab.wic.few-shot} 
\end{table} 


\begin{table*}[!hbp] 
  \centering
 \footnotesize
  % \scriptsize
  % \tiny
   \begin{tabular}{
    m{0.96\textwidth}
    }
    \toprule
    Yes/No question-answering consists of a short passage and a Yes/No question about the passage. The questions are provided anonymously and unsolicited by users of the Google search engine, and afterwards paired with a paragraph from a Wikipedia article containing the answer. If there exists evidence in the passage that supports the facts in the question, the answer should be "Yes". If there exists evidence in the passage that denies the facts in the question, the answer should be "No".\\
    Your task is to read the passage and predict whether the answer to the question is "Yes" or "No".\\
    \\
    Passage: The Elder Scrolls Online -- As with other games in The Elder Scrolls series, the game is set on the continent of Tamriel. The events of the game occur a millennium before those of The Elder Scrolls V: Skyrim and around 800 years before The Elder Scrolls III: Morrowind and The Elder Scrolls IV: Oblivion. It has a broadly similar structure to Skyrim, with two separate conflicts progressing at the same time, one with the fate of the world in the balance, and one where the prize is supreme power on Tamriel. In The Elder Scrolls Online, the first struggle is against the Daedric Prince Molag Bal, who is attempting to meld the plane of Mundus with his realm of Coldharbour, and the second is to capture the vacant imperial throne, contested by three alliances of the mortal races. The player character has been sacrificed to Molag Bal, and Molag Bal has stolen their soul, the recovery of which is the primary game objective.\\
    Question: is elder scrolls online the same as skyrim\\
    Answer: No\\
    \\
    Passage: Good Samaritan law -- Good Samaritan laws offer legal protection to people who give reasonable assistance to those who are, or who they believe to be, injured, ill, in peril, or otherwise incapacitated. The protection is intended to reduce bystanders' hesitation to assist, for fear of being sued or prosecuted for unintentional injury or wrongful death. An example of such a law in common-law areas of Canada: a good Samaritan doctrine is a legal principle that prevents a rescuer who has voluntarily helped a victim in distress from being successfully sued for wrongdoing. Its purpose is to keep people from being reluctant to help a stranger in need for fear of legal repercussions should they make some mistake in treatment. By contrast, a duty to rescue law requires people to offer assistance and holds those who fail to do so liable.\\
    Question: do good samaritan laws protect those who help at an accident\\
    Answer: Yes\\
    \\
    Passage: Windows Movie Maker -- Windows Movie Maker (formerly known as Windows Live Movie Maker in Windows 7) is a discontinued video editing software by Microsoft. It is a part of Windows Essentials software suite and offers the ability to create and edit videos as well as to publish them on OneDrive, Facebook, Vimeo, YouTube, and Flickr.\\
    Question: is windows movie maker part of windows essentials\\
    Answer: Yes\\
    \\
    Passage: Epsom railway station -- Epsom railway station serves the town of Epsom in Surrey. It is located off Waterloo Road and is less than two minutes' walk from the High Street. It is not in the London Oyster card zone unlike Epsom Downs or Tattenham Corner stations. The station building was replaced in 2012/2013 with a new building with apartments above the station (see end of article).\\
    Question: can you use oyster card at epsom station\\
    Answer: No\\
    \\
    Passage: Da Vinci's Demons -- The series premiered in the United States on Starz on 12 April 2013, and its second season premiered on 22 March 2014. The series was renewed for a third season, which premiered on 24 October 2015. On 23 July 2015, Starz announced that the third season would be the show's last. However Goyer has left it open for a miniseries return.\\
    Question: will there be a season 4 of da vinci's demons\\
    Answer: No\\
    \\
    Passage: Powdered sugar -- Powdered sugar, also called confectioners' sugar, icing sugar, and icing cake, is a finely ground sugar produced by milling granulated sugar into a powdered state. It usually contains a small amount of anti-caking agent to prevent clumping and improve flow. Although most often produced in a factory, powdered sugar can also be made by processing ordinary granulated sugar in a coffee grinder, or by crushing it by hand in a mortar and pestle.\\
    Question: is confectionary sugar the same as powdered sugar\\
    Answer: Yes\\
         \end{tabular}
 \caption{Continued on the next page}
\end{table*} 

\begin{table*}[t]
 \ContinuedFloat
  \centering
 \footnotesize
  % \scriptsize
  % \tiny
   \begin{tabular}{
    m{0.96\textwidth}
    }
    % \toprule
    Passage: Federal judiciary of the United States -- The federal courts are composed of three levels of courts. The Supreme Court of the United States is the court of last resort. It is generally an appellate court that operates under discretionary review, which means that the Court can choose which cases to hear, by granting writs of certiorari. There is therefore generally no basic right of appeal that extends automatically all the way to the Supreme Court. In a few situations (like lawsuits between state governments or some cases between the federal government and a state) it sits as a court of original jurisdiction.\\
    Question: is the federal court the same as the supreme court\\
    Answer: No\\
    \\
    Passage: Bixby letter -- In the 1998 war film Saving Private Ryan, General George Marshall (played by Harve Presnell) reads the Bixby letter to his officers before giving the order to find and send home Private James Francis Ryan after Ryan's three brothers died in battle.\\
    Question: did abraham lincoln write the letter in saving private ryan\\
    Answer: Yes\\
    \\
    Passage: \textbf{\{passage\}}\\
    Question: \textbf{\{question\}}\\
    Answer: \\
    \bottomrule
 \end{tabular}
 \caption{Few-shot exemplars prompt for the BoolQ task.}\label{tab.boolq.few-shot} 
\end{table*} 











% \clearpage
\newpage
\section{Few-shot CoT Prompts}\label{sec.appendix.d}

\begin{table*}[bh] 
  \centering
 \footnotesize
  % \scriptsize
  % \tiny
   \begin{tabular}{
    m{0.96\textwidth}
    }
    \toprule
    Given a search engine query, first, consider what the user could have in mind when they type in the query and allow for misspellings or other ambiguity, then classify the relevance of keyword to the query into one of the following categories: "Not bad", or "Bad". The definitions of the categories are\\
    "Not bad": the keyword is relevant to the user's search query. This can include: broader or narrower product selection, competitor or alternative products, accessories, products often purchased together and related topics as well as direct matches to the user's search.\\
    "Bad": the keyword is not relevant to the user's search query. There is no relationship between the query and keyword. This includes but is not limited to: incorrect/unrelated product, unrelated topic, wrong location when location is important, cannot be used in place of query product nor are they commonly purchased together, etc.\\
    Please predict whether the keyword is relevant to the query or not. The answer should be exact "Not bad" or "Bad".\\
    \\
    Query: google data studio sharepoint\\
    Keyword: sharepoint migration tool file share\\
    Answer: \textbf{\underline{The relevance of the keyword "sharepoint migration tool file share" to the search query "google data} \underline{studio sharepoint" is "Bad".} This keyword is not relevant to the user's search query as it pertains to a different tool or technology (SharePoint) that is not mentioned in the search query. The keyword focuses on a migration tool for SharePoint file sharing, while the search query is focused on Google Data Studio. There is no relationship between the two, so the keyword is not relevant to the user's search query.} \textit{Therefore, the relevance is "Bad".}\\
    \\
    Query: motorhomes sale\\
    Keyword: rv sale used class c\\
    Answer: \textbf{\underline{The relevance of keyword "rv sale used class c" is "Not bad".} The keyword "rv sale used class c" contains relevant information to the user's search query "motorhomes sale". It includes a specific type of motorhome, "class c", that is being sold used, which is a relevant product selection for the user's search. Additionally, the keyword "rv sale used class c" is a direct match to the user's search and is commonly purchased together with the products in the search query.} \textit{Therefore, the relevance is "Not bad".}\\
    \\
    Query: southern exposure seed exchange company\\
    Keyword: uk poppy seeds\\
    Answer: \textbf{\underline{The relevance of the keyword "uk poppy seeds" to the search query "southern exposure seed exchange} \underline{company" is "Not bad".} This is because the search query is about a seed exchange company, and the keyword "uk poppy seeds" is a specific type of seed that could be offered by such a company. Even though the company mentioned in the search query is based in the southern hemisphere, they may still offer seeds from other regions, including the UK, which could include poppy seeds.} \textit{Therefore, the relevance is "Not bad".}\\
    \\
    Query: nissan parts canada\\
    Keyword: purchase tires\\
    Answer: \textbf{\underline{The relevance of "purchase tires" to the query "nissan parts canada" is "Bad".} The keyword "purchase tires" is not directly related to the user's search for Nissan parts in Canada. Tires are not a part of a Nissan car and therefore are not relevant to the search for Nissan parts. Additionally, the keyword does not relate to a common purchase with Nissan parts, and it is not a substitute for the search query.} \textit{Therefore, the relevance is "Bad".}\\
    \\
    Query: \textbf{\{query\}}\\
    Keyword: \textbf{\{keyword\}}\\
    Answer: \\
    \bottomrule
 \end{tabular}
 \caption{Few-shot exemplars for full chain-of-thought prompt for the QK task. The bold text indicates the explanations  generated with the prompt in Table  \ref{tab.qk.explanation}.}\label{tab.qk.few-shot-cot} 
\end{table*} 


\begin{table*}[!hbp] 
  \centering
 \footnotesize
  % \scriptsize
  % \tiny
   \begin{tabular}{
    m{0.96\textwidth}
    }
    \toprule
    Task description: The task is to identify if the occurrences of w in s1 and s2 correspond to the same meaning or not. The target word w in s1 and s2 is enclosed in quotation marks. There is no need to consider the tense and form of w. If w in s1 and s2 corresponds to the same meaning, output "true", otherwise output "false". \\
    \\
    w: "place"
    s1: Do you want to come over to my "place" later?\\
    s2: A political system with no "place" for the less prominent groups.\\
    Explanation: \textbf{The answer is "false". In the first sentence, "place" refers to a physical location or a person's home. In the second sentence, "place" refers to a position or status in a political system. Therefore, although the spelling and punctuation of "place" are the same in both sentences, the context and meaning of the word are different.} Therefore, the answer is "false".\\
    \\
    w: "hold"
    s1: The general ordered the colonel to "hold" his position at all costs.\\
    s2: "Hold" the taxi.\\
    Explanation: \textbf{The answer is "true". In both sentences, "hold" means to maintain or stay in a specific position or location. In the first sentence, the general orders the colonel to hold his position at all costs, meaning the colonel should not retreat or move from his current location. In the second sentence, someone is asking the taxi driver to hold, meaning to wait and stay in the current location until the person returns. Therefore, in both sentences, "hold" is used to convey the idea of staying in a particular position or location.} Therefore, the answer is "true".\\
    \\
    w: "summer"\\
    s1: We like to "summer" in the Mediterranean.\\
    s2: We "summered" in Kashmir.\\
    Explanation: \textbf{The answer is "true". In both sentences, "summer" refers to the activity of spending a summer vacation in a certain location. Although the word form is different ("summer" in the first sentence and "summered" in the second), the meaning is the same. The use of quotation marks in both sentences implies that "summer" is being used as a verb, rather than a season. Therefore, both sentences convey the idea of enjoying a summer vacation in different locations, making the meaning the same.} Therefore, the answer is "true".\\
    \\
    w: "approach"\\
    s1: "Approach" a task.\\
    s2: To "approach" the city.\\
    Explanation: \textbf{The answer is "false". In the first sentence, "Approach" is used as a verb and means to begin or undertake a task. In the second sentence, "approach" is also used as a verb but means to move towards or get closer to the city. Although the spelling and the quotation marks around the word are the same in both sentences, the context and the meaning of the word are different. Therefore, the occurrences of "approach" in the two sentences do not correspond to the same meaning.} Therefore, the answer is "false".\\
    \\
    w: "cover"\\
    s1: "Cover" her face with a handkerchief.\\
    s2: Count the cash in the drawer twice just to "cover" yourself.\\
    Explanation: \textbf{The answer is "false". The word "cover" has different meanings in the two sentences. In the first sentence, "cover" means to hide or conceal something. In contrast, in the second sentence, "cover" means to protect oneself from potential problems or accusations. The context and usage of the word "cover" in the two sentences are different, indicating that the two instances of "cover" do not correspond to the same meaning.} Therefore, the answer is "false".\\
    \\
    w: "head"\\
    s1: His horse won by a "head".\\
    s2: He is two "heads" taller than his little sister.\\
    Explanation: \textbf{The answer is "true". In both sentences, "head" is used as a unit of measurement. In the first sentence, it means that the speaker's horse won by a small margin, specifically the length of the horse's head. In the second sentence, it means that the person is taller than their little sister by a specific amount, namely two head lengths. Therefore, in both cases, "head" is used as a measure of distance or height, indicating that the two occurrences correspond to the same meaning.} Therefore, the answer is "true".\\
    \\
    w: "meet"\\
    s1: The company agrees to "meet" the cost of any repairs.\\
    s2: This proposal "meets" my requirements.\\
    Explanation: \textbf{The answer is "true". Although the verb "meet" is used in different senses in the two sentences, in both cases, it conveys the idea of satisfying a requirement or obligation. In the first sentence, the company agrees to pay for any repairs, which is a requirement or obligation. In the second sentence, the proposal fulfills or satisfies the speaker's requirements. Therefore, even though the two sentences use "meet" in different senses, they convey the same general idea of satisfying an obligation or requirement, making the answer "true".} Therefore, the answer is "true".\\
        % \bottomrule
 \end{tabular}
 \caption{Continued on the next page}
\end{table*} 

\begin{table*}
 \ContinuedFloat
  \centering
 \footnotesize
  % \scriptsize
  % \tiny
   \begin{tabular}{
    m{0.96\textwidth}
    }
    % \toprule
    w: "development"\\
    s1: The organism has reached a crucial stage in its "development".\\
    s2: Our news team brings you the latest "developments".\\
    Explanation: \textbf{The answer is "false". In the first sentence, "development" refers to the growth or progress of an organism. In contrast, in the second sentence, "developments" refer to the latest events or updates on a particular subject. The first sentence refers to a specific stage in the life of an organism, while the second sentence refers to recent events or news related to a particular topic. Therefore, the context and meaning of "development" in both sentences are different, and they cannot be considered the same.} Therefore, the answer is "false".\\
    \\
    w: \textbf{\{target word\}}\\
    s1: \textbf{\{first sentence\}}\\
    s2: \textbf{\{second sentence\}}\\
    Explanation: \\
    \bottomrule
 \end{tabular}
 \caption{Few-shot exemplars for full chain-of-thought prompt for the WiC task. The bold text indicates the explanations generated with the prompt in Table  \ref{tab.wic.1}.}\label{tab.wic.few-shot-cot} 
\end{table*} 




\begin{table*}[!hbp] 
  \centering
 \footnotesize
  % \scriptsize
  % \tiny
   \begin{tabular}{
    m{0.96\textwidth}
    }
    \toprule
    Yes/No question-answering consists of a short passage and a Yes/No question about the passage. The questions are provided anonymously and unsolicited by users of the Google search engine, and afterwards paired with a paragraph from a Wikipedia article containing the answer. If there exists evidence in the passage that supports the facts in the question, the answer should be "Yes". If there exists evidence in the passage that denies the facts in the question, the answer should be "No".\\
    Your task is to read the passage and predict whether the answer to the question is "Yes" or "No".\\
    \\
    Passage: The Elder Scrolls Online -- As with other games in The Elder Scrolls series, the game is set on the continent of Tamriel. The events of the game occur a millennium before those of The Elder Scrolls V: Skyrim and around 800 years before The Elder Scrolls III: Morrowind and The Elder Scrolls IV: Oblivion. It has a broadly similar structure to Skyrim, with two separate conflicts progressing at the same time, one with the fate of the world in the balance, and one where the prize is supreme power on Tamriel. In The Elder Scrolls Online, the first struggle is against the Daedric Prince Molag Bal, who is attempting to meld the plane of Mundus with his realm of Coldharbour, and the second is to capture the vacant imperial throne, contested by three alliances of the mortal races. The player character has been sacrificed to Molag Bal, and Molag Bal has stolen their soul, the recovery of which is the primary game objective.\\
    Question: is elder scrolls online the same as skyrim\\
    Answer: \textbf{The answer is "No". Although The Elder Scrolls Online is set on the continent of Tamriel, like Skyrim and other games in The Elder Scrolls series, it takes place a millennium before Skyrim and around 800 years before Morrowind and Oblivion. While it has a similar structure to Skyrim, with two conflicts progressing at the same time, the primary struggle in The Elder Scrolls Online is against the Daedric Prince Molag Bal, who has stolen the player character's soul, and the second is to capture the vacant imperial throne. Therefore, The Elder Scrolls Online is not the same game as Skyrim.} Therefore, the answer is "No".\\
    \\
    Passage: Good Samaritan law -- Good Samaritan laws offer legal protection to people who give reasonable assistance to those who are, or who they believe to be, injured, ill, in peril, or otherwise incapacitated. The protection is intended to reduce bystanders' hesitation to assist, for fear of being sued or prosecuted for unintentional injury or wrongful death. An example of such a law in common-law areas of Canada: a good Samaritan doctrine is a legal principle that prevents a rescuer who has voluntarily helped a victim in distress from being successfully sued for wrongdoing. Its purpose is to keep people from being reluctant to help a stranger in need for fear of legal repercussions should they make some mistake in treatment. By contrast, a duty to rescue law requires people to offer assistance and holds those who fail to do so liable.\\
    Question: do good samaritan laws protect those who help at an accident\\
    Answer: \textbf{ The answer is "Yes". Good Samaritan laws offer legal protection to people who provide reasonable assistance to those who are injured, ill, in peril, or otherwise incapacitated. The purpose of these laws is to encourage bystanders to provide help without fear of being sued or prosecuted for unintentional injury or wrongful death. In Canada, for example, the good Samaritan doctrine is a legal principle that prevents a rescuer who has voluntarily helped a victim in distress from being successfully sued for wrongdoing. Therefore, individuals who provide reasonable assistance at an accident can be protected by good Samaritan laws.} Therefore, the answer is "Yes".\\
    \\
    Passage: Windows Movie Maker -- Windows Movie Maker (formerly known as Windows Live Movie Maker in Windows 7) is a discontinued video editing software by Microsoft. It is a part of Windows Essentials software suite and offers the ability to create and edit videos as well as to publish them on OneDrive, Facebook, Vimeo, YouTube, and Flickr.\\
    Question: is windows movie maker part of windows essentials\\
    Answer: \textbf{The answer to the question "is Windows Movie Maker part of Windows Essentials?" is Yes. As stated in the passage, Windows Movie Maker is a discontinued video editing software by Microsoft and was a part of the Windows Essentials software suite. Windows Essentials was a collection of free software programs from Microsoft, including Photo Gallery, Movie Maker, Mail, Writer, and OneDrive, among others. Therefore, Windows Movie Maker was one of the programs included in the Windows Essentials suite, which was available for download from Microsoft's website.} Therefore, the answer is "Yes".\\
            % \bottomrule
 \end{tabular}
 \caption{Continued on the next page}
\end{table*} 

\begin{table*}[!hbp] 
 \ContinuedFloat
  \centering
 \footnotesize
  % \scriptsize
  % \tiny
   \begin{tabular}{
    m{0.96\textwidth}
    }
    % \toprule
    Passage: Epsom railway station -- Epsom railway station serves the town of Epsom in Surrey. It is located off Waterloo Road and is less than two minutes' walk from the High Street. It is not in the London Oyster card zone unlike Epsom Downs or Tattenham Corner stations. The station building was replaced in 2012/2013 with a new building with apartments above the station (see end of article).\\
    Question: can you use oyster card at epsom station\\
    Answer: \textbf{The answer is "No". The passage explicitly states that Epsom railway station is not in the London Oyster card zone, unlike Epsom Downs or Tattenham Corner stations. Therefore, you cannot use Oyster card at Epsom railway station.} Therefore, the answer is "No".\\
    \\
    Passage: Da Vinci's Demons -- The series premiered in the United States on Starz on 12 April 2013, and its second season premiered on 22 March 2014. The series was renewed for a third season, which premiered on 24 October 2015. On 23 July 2015, Starz announced that the third season would be the show's last. However Goyer has left it open for a miniseries return.\\
    Question: will there be a season 4 of da vinci's demons\\
    Answer: \textbf{The answer is "No". The passage states that the third season of "Da Vinci's Demons" was the show's last and that the series creator, David S. Goyer, left open the possibility of a miniseries return. However, there is no indication that a fourth season of the show was planned or produced. Therefore, based on the information provided, it can be concluded that there will not be a season 4 of "Da Vinci's Demons".} Therefore, the answer is "No".\\
    \\
    Passage: Powdered sugar -- Powdered sugar, also called confectioners' sugar, icing sugar, and icing cake, is a finely ground sugar produced by milling granulated sugar into a powdered state. It usually contains a small amount of anti-caking agent to prevent clumping and improve flow. Although most often produced in a factory, powdered sugar can also be made by processing ordinary granulated sugar in a coffee grinder, or by crushing it by hand in a mortar and pestle.\\
    Question: is confectionary sugar the same as powdered sugar\\
    Answer: \textbf{The answer is "Yes". Confectioners' sugar is another name for powdered sugar, as mentioned in the passage. It is produced by milling granulated sugar into a fine powder, and usually contains a small amount of anti-caking agent to prevent clumping. Both terms, confectioners' sugar and powdered sugar, can be used interchangeably to refer to the same product.} Therefore, the answer is "Yes".\\
    \\
    Passage: Federal judiciary of the United States -- The federal courts are composed of three levels of courts. The Supreme Court of the United States is the court of last resort. It is generally an appellate court that operates under discretionary review, which means that the Court can choose which cases to hear, by granting writs of certiorari. There is therefore generally no basic right of appeal that extends automatically all the way to the Supreme Court. In a few situations (like lawsuits between state governments or some cases between the federal government and a state) it sits as a court of original jurisdiction.\\
    Question: is the federal court the same as the supreme court\\
    Answer: \textbf{The answer is "No". The federal court is not the same as the Supreme Court. The federal courts consist of three levels of courts, while the Supreme Court is the highest level court of the federal judiciary. The Supreme Court operates under discretionary review and generally only hears cases on appeal. It also has original jurisdiction in certain specific situations, but it is not the same as the federal court.} Therefore, the answer is "No".\\
    \\
    Passage: Batman \& Robin (film) -- With the box office success of Batman Forever in June 1995, Warner Bros. immediately commissioned a sequel. They hired director Joel Schumacher and writer Akiva Goldsman to reprise their duties the following August, and decided it was best to fast track production for a June 1997 target release date, which is a break from the usual 3-year gap between films. Schumacher wanted to homage both the broad camp style of the 1960s television series and the work of Dick Sprang. The storyline of Batman \& Robin was conceived by Schumacher and Goldsman during pre-production on A Time to Kill. Portions of Mr. Freeze's back-story were based on the Batman: The Animated Series episode ``Heart of Ice'', written by Paul Dini.\\
    Question: is batman and robin a sequel to batman forever\\
    Answer: \textbf{The answer is "Yes". Based on the information provided in the passage, Batman \& Robin is a sequel to Batman Forever. Warner Bros. commissioned the sequel after the box office success of Batman Forever, and hired director Joel Schumacher and writer Akiva Goldsman to continue their duties. The film was fast-tracked for a June 1997 release, with Schumacher and Goldsman conceiving the storyline during pre-production of another movie. Therefore, Batman \& Robin is a direct sequel to Batman Forever, released just two years later.} Therefore, the answer is "Yes".\\
    \\
    Passage: \textbf{\{passage\}}\\
    Question: \textbf{\{question\}}\\
    Answer: \\
    \bottomrule
 \end{tabular}
 \caption{Few-shot exemplars for full chain-of-thought prompt for the BoolQ task. The bold text indicates the explanations generated with the prompt in Table  \ref{tab.boolq.1}.}
 \label{tab.boolq.few-shot-cot} 
\end{table*} 



\clearpage
% \newpage
\section{Prompts Used to Test the Stability}\label{sec.appendix.e}
We present the few-shot prompts p1, p2 and p3 in Tables \ref{tab.boolq.few-shot1}, \ref{tab.boolq.few-shot2} and \ref{tab.boolq.few-shot3}, respectively. 
The few-shot prompt p3 is obtained by swapping the order of the ``Question'' and ``Passage'' in Table \ref{tab.boolq.few-shot}. 
While few-shot prompts p1 and p2 have minor variations in their task description compared to p3, we have highlighted the differences in bold. 
The few-shot prompts, p1, p2, and p3, consist of the same demonstrated examples as the original prompt presented in Table \ref{tab.boolq.few-shot}.

We show the few-shot CoT prompts p1, p2 and p3 in Tables \ref{tab.boolq.few-shot-cot1}, \ref{tab.boolq.few-shot-cot2} and \ref{tab.boolq.few-shot-cot3}, respectively. 
The few-shot CoT prompt p3 is obtained by swapping the order of the ``Question'' and ``Passage'' in Table \ref{tab.boolq.few-shot-cot}. 
While few-shot CoT prompts p1 and p2 have minor variations in their task description compared to p3, we have highlighted the differences in bold. 
The few-shot CoT prompts, p1, p2, and p3, consist of the same demonstrated examples as the original prompt presented in Table \ref{tab.boolq.few-shot-cot}.



\begin{table}[!hbp] 
  \centering
 \footnotesize
  % \scriptsize
  % \tiny
   \begin{tabular}{
    m{0.46\textwidth}
    }
    \toprule
    \textbf{Your task is to read the passage and predict whether the answer to the question is "Yes" or "No".}\\
    \\
    \textbf{Question}: is elder scrolls online the same as skyrim\\
    \textbf{Passage}: The Elder Scrolls Online -- As with other games in The Elder Scrolls series, the game is set on the continent of Tamriel. The events of the game occur a millennium before those of The Elder Scrolls V: Skyrim and around 800 years before The Elder Scrolls III: Morrowind and The Elder Scrolls IV: Oblivion. It has a broadly similar structure to Skyrim, with two separate conflicts progressing at the same time, one with the fate of the world in the balance, and one where the prize is supreme power on Tamriel. In The Elder Scrolls Online, the first struggle is against the Daedric Prince Molag Bal, who is attempting to meld the plane of Mundus with his realm of Coldharbour, and the second is to capture the vacant imperial throne, contested by three alliances of the mortal races. The player character has been sacrificed to Molag Bal, and Molag Bal has stolen their soul, the recovery of which is the primary game objective.\\
    Answer: No\\
    \\
    \textbf{......} \\
    \bottomrule
 \end{tabular}
 \caption{Few-shot prompt p1 for the BoolQ task. 
 }
 \label{tab.boolq.few-shot1} 
\end{table} 


\begin{table}[!hbp] 
  \centering
 \footnotesize
  % \scriptsize
  % \tiny
   \begin{tabular}{
    m{0.46\textwidth}
    }
    \toprule
    \textbf{Yes/No question-answering consists of a short passage and a Yes/No question about the passage. The questions are provided anonymously and unsolicited by users of the Google search engine, and afterwards paired with a paragraph from a Wikipedia article containing the answer.}\\
    \textbf{Your task is to read the passage and predict whether the answer to the question is "Yes" or "No".}\\
    \\
    \textbf{Question}: is elder scrolls online the same as skyrim\\
    \textbf{Passage}: The Elder Scrolls Online -- As with other games in The Elder Scrolls series, the game is set on the continent of Tamriel. The events of the game occur a millennium before those of The Elder Scrolls V: Skyrim and around 800 years before The Elder Scrolls III: Morrowind and The Elder Scrolls IV: Oblivion. It has a broadly similar structure to Skyrim, with two separate conflicts progressing at the same time, one with the fate of the world in the balance, and one where the prize is supreme power on Tamriel. In The Elder Scrolls Online, the first struggle is against the Daedric Prince Molag Bal, who is attempting to meld the plane of Mundus with his realm of Coldharbour, and the second is to capture the vacant imperial throne, contested by three alliances of the mortal races. The player character has been sacrificed to Molag Bal, and Molag Bal has stolen their soul, the recovery of which is the primary game objective.\\
    Answer: No\\
    \\
    \textbf{......} \\
    \bottomrule
 \end{tabular}
 \caption{Few-shot prompt p2 for the BoolQ task. 
 }
 \label{tab.boolq.few-shot2} 
\end{table} 



\begin{table}[!hbp] 
  \centering
 \footnotesize
  % \scriptsize
  % \tiny
   \begin{tabular}{
    m{0.46\textwidth}
    }
    \toprule
    \textbf{Yes/No question-answering consists of a short passage and a Yes/No question about the passage. The questions are provided anonymously and unsolicited by users of the Google search engine, and afterwards paired with a paragraph from a Wikipedia article containing the answer. If there exists evidence in the passage that supports the facts in the question, the answer should be "Yes". If there exists evidence in the passage that denies the facts in the question, the answer should be "No".}\\
    \textbf{Your task is to read the passage and predict whether the answer to the question is "Yes" or "No".}\\
    \\
    \textbf{Question}: is elder scrolls online the same as skyrim\\
    \textbf{Passage}: The Elder Scrolls Online -- As with other games in The Elder Scrolls series, the game is set on the continent of Tamriel. The events of the game occur a millennium before those of The Elder Scrolls V: Skyrim and around 800 years before The Elder Scrolls III: Morrowind and The Elder Scrolls IV: Oblivion. It has a broadly similar structure to Skyrim, with two separate conflicts progressing at the same time, one with the fate of the world in the balance, and one where the prize is supreme power on Tamriel. In The Elder Scrolls Online, the first struggle is against the Daedric Prince Molag Bal, who is attempting to meld the plane of Mundus with his realm of Coldharbour, and the second is to capture the vacant imperial throne, contested by three alliances of the mortal races. The player character has been sacrificed to Molag Bal, and Molag Bal has stolen their soul, the recovery of which is the primary game objective.\\
    Answer: No\\
    \\
    \textbf{......} \\
    \bottomrule
 \end{tabular}
 \caption{Few-shot prompt p3 for the BoolQ task. 
 }
 \label{tab.boolq.few-shot3} 
\end{table} 



\begin{table*}[!hbp] 
  \centering
 % \footnotesize
  \scriptsize
  % \tiny
   \begin{tabular}{
    m{0.98\textwidth}
    }
    \toprule
    \textbf{Your task is to read the passage and predict whether the answer to the question is "Yes" or "No".}\\
    \\
    \textbf{Question}: is elder scrolls online the same as skyrim\\
    \textbf{Passage}: The Elder Scrolls Online -- As with other games in The Elder Scrolls series, the game is set on the continent of Tamriel. The events of the game occur a millennium before those of The Elder Scrolls V: Skyrim and around 800 years before The Elder Scrolls III: Morrowind and The Elder Scrolls IV: Oblivion. It has a broadly similar structure to Skyrim, with two separate conflicts progressing at the same time, one with the fate of the world in the balance, and one where the prize is supreme power on Tamriel. In The Elder Scrolls Online, the first struggle is against the Daedric Prince Molag Bal, who is attempting to meld the plane of Mundus with his realm of Coldharbour, and the second is to capture the vacant imperial throne, contested by three alliances of the mortal races. The player character has been sacrificed to Molag Bal, and Molag Bal has stolen their soul, the recovery of which is the primary game objective.\\
    Answer: The answer is "No". Although The Elder Scrolls Online is set on the continent of Tamriel, like Skyrim and other games in The Elder Scrolls series, it takes place a millennium before Skyrim and around 800 years before Morrowind and Oblivion. While it has a similar structure to Skyrim, with two conflicts progressing at the same time, the primary struggle in The Elder Scrolls Online is against the Daedric Prince Molag Bal, who has stolen the player character's soul, and the second is to capture the vacant imperial throne. Therefore, The Elder Scrolls Online is not the same game as Skyrim. Therefore, the answer is "No".\\
    \textbf{......} \\
    \bottomrule
 \end{tabular}
 \caption{Few-shot CoT prompt p1 for the BoolQ task. 
 }
 \label{tab.boolq.few-shot-cot1} 
\end{table*} 


\begin{table*}[!hbp] 
  \centering
 % \footnotesize
  \scriptsize
  % \tiny
   \begin{tabular}{
    m{0.98\textwidth}
    }
    \toprule
    \textbf{Yes/No question-answering consists of a short passage and a Yes/No question about the passage. The questions are provided anonymously and unsolicited by users of the Google search engine, and afterwards paired with a paragraph from a Wikipedia article containing the answer.}\\
    \textbf{Your task is to read the passage and predict whether the answer to the question is "Yes" or "No".}\\
    \\
    \textbf{Question}: is elder scrolls online the same as skyrim\\
    \textbf{Passage}: The Elder Scrolls Online -- As with other games in The Elder Scrolls series, the game is set on the continent of Tamriel. The events of the game occur a millennium before those of The Elder Scrolls V: Skyrim and around 800 years before The Elder Scrolls III: Morrowind and The Elder Scrolls IV: Oblivion. It has a broadly similar structure to Skyrim, with two separate conflicts progressing at the same time, one with the fate of the world in the balance, and one where the prize is supreme power on Tamriel. In The Elder Scrolls Online, the first struggle is against the Daedric Prince Molag Bal, who is attempting to meld the plane of Mundus with his realm of Coldharbour, and the second is to capture the vacant imperial throne, contested by three alliances of the mortal races. The player character has been sacrificed to Molag Bal, and Molag Bal has stolen their soul, the recovery of which is the primary game objective.\\
    Answer: The answer is "No". Although The Elder Scrolls Online is set on the continent of Tamriel, like Skyrim and other games in The Elder Scrolls series, it takes place a millennium before Skyrim and around 800 years before Morrowind and Oblivion. While it has a similar structure to Skyrim, with two conflicts progressing at the same time, the primary struggle in The Elder Scrolls Online is against the Daedric Prince Molag Bal, who has stolen the player character's soul, and the second is to capture the vacant imperial throne. Therefore, The Elder Scrolls Online is not the same game as Skyrim. Therefore, the answer is "No".\\
    \textbf{......} \\
    \bottomrule
 \end{tabular}
 \caption{Few-shot CoT prompt p2 for the BoolQ task. 
 }
 \label{tab.boolq.few-shot-cot2} 
\end{table*} 



\begin{table*}[!hbp] 
  \centering
 % \footnotesize
  \scriptsize
  % \tiny
   \begin{tabular}{
    m{0.96\textwidth}
    }
    \toprule
    \textbf{Yes/No question-answering consists of a short passage and a Yes/No question about the passage. The questions are provided anonymously and unsolicited by users of the Google search engine, and afterwards paired with a paragraph from a Wikipedia article containing the answer. If there exists evidence in the passage that supports the facts in the question, the answer should be "Yes". If there exists evidence in the passage that denies the facts in the question, the answer should be "No".}\\
    \textbf{Your task is to read the passage and predict whether the answer to the question is "Yes" or "No".}\\
    \\
    \textbf{Question}: is elder scrolls online the same as skyrim\\
    \textbf{Passage}: The Elder Scrolls Online -- As with other games in The Elder Scrolls series, the game is set on the continent of Tamriel. The events of the game occur a millennium before those of The Elder Scrolls V: Skyrim and around 800 years before The Elder Scrolls III: Morrowind and The Elder Scrolls IV: Oblivion. It has a broadly similar structure to Skyrim, with two separate conflicts progressing at the same time, one with the fate of the world in the balance, and one where the prize is supreme power on Tamriel. In The Elder Scrolls Online, the first struggle is against the Daedric Prince Molag Bal, who is attempting to meld the plane of Mundus with his realm of Coldharbour, and the second is to capture the vacant imperial throne, contested by three alliances of the mortal races. The player character has been sacrificed to Molag Bal, and Molag Bal has stolen their soul, the recovery of which is the primary game objective.\\
    Answer: The answer is "No". Although The Elder Scrolls Online is set on the continent of Tamriel, like Skyrim and other games in The Elder Scrolls series, it takes place a millennium before Skyrim and around 800 years before Morrowind and Oblivion. While it has a similar structure to Skyrim, with two conflicts progressing at the same time, the primary struggle in The Elder Scrolls Online is against the Daedric Prince Molag Bal, who has stolen the player character's soul, and the second is to capture the vacant imperial throne. Therefore, The Elder Scrolls Online is not the same game as Skyrim. Therefore, the answer is "No".\\
    \textbf{......} \\
    \bottomrule
 \end{tabular}
 \caption{Few-shot CoT prompt p3 for the BoolQ task. 
 }
 \label{tab.boolq.few-shot-cot3} 
\end{table*} 


\newpage
% \clearpage
\section{Prompts for Constructing the Conversation-based Information Retrieval Dataset}\label{sec.appendix.f}

\begin{table}[hbp!]
  \centering
 \footnotesize
  % \scriptsize
  % \tiny
   \begin{tabular}{
    m{0.48\textwidth}
    }
    \toprule
    Now you need to enrich the following text paragraphs with content that is relevant and factually consistent with the content of the text paragraphs.\\
    Passage: \textcolor{blue}{To determine the salary range percentile, you must first calculate the difference between the maximum and minimum salary figures. For example, if the salary range for a particular position is between \$45,000 and \$75,000, the difference between those two figures would be \$30,000.}\\
    \textcolor{red}{Once you have calculated the salary range difference, you can then use this information to determine the salary range percentile. This is a way to compare salaries of individuals in the same profession or industry, and can be helpful in negotiating salaries and assessing the competitiveness of compensation packages.}\\
    \textcolor{red}{The salary range percentile can be calculated by taking an individual's salary and comparing it to the range of salaries for the same position. For instance, if an individual's salary is \$60,000 and the salary range for their position is \$45,000 to \$75,000, their salary falls within the 50th percentile of the salary range.}\\
    \textcolor{red}{It's important to note that salary ranges can vary depending on a number of factors, including location, industry, experience, and education. Employers typically set salary ranges based on market research and benchmarking against similar positions in the same industry and geographic region.}\\
    \textcolor{red}{Additionally, some companies may have a more structured salary range system, while others may be more flexible and open to negotiation. It's important to research and understand the salary range for your position and industry to ensure that you are being compensated fairly and competitively.}\\
    % Passage: \textbf{\{passage\}}\\
    \bottomrule
 \end{tabular}
 \caption{Zero-shot prompt used to enrich text paragraphs (\textcolor{blue}{blue} = input; \textcolor{red}{red} = output).}\label{tab.enrich_text.zero-shot} 
\end{table} 




\begin{table}[hbp!]
  \centering
 \footnotesize
  % \scriptsize
  % \tiny
   \begin{tabular}{
    m{0.48\textwidth}
    }
    \toprule
    I need you to simulate the scene of a conversation between a person and ChatGPT. P1 represents the person, who is responsible for asking questions based on the given text passage, and P2 represents the ChatGPT, who is responsible for answering the questions posed by P1. You need to answer query according to the following passage and form the first round of the dialogue:\\
    Query: \textcolor{blue}{how to determine salary range}\\
    Passage: \textcolor{blue}{Once you have calculated the salary range difference, you can then use this information to determine the salary range percentile. This is a way to compare salaries of individuals in the same profession or industry, and can be helpful in negotiating salaries and assessing the competitiveness of compensation packages.}\\
    \textcolor{blue}{The salary range percentile can be calculated by taking an individual's salary and comparing it to the range of salaries for the same position. For instance, if an individual's salary is \$60,000 and the salary range for their position is \$45,000 to \$75,000, their salary falls within the 50th percentile of the salary range.}\\
    \textcolor{blue}{It's important to note that salary ranges can vary depending on a number of factors, including location, industry, experience, and education. Employers typically set salary ranges based on market research and benchmarking against similar positions in the same industry and geographic region.}\\
    \textcolor{blue}{Additionally, some companies may have a more structured salary range system, while others may be more flexible and open to negotiation. It's important to research and understand the salary range for your position and industry to ensure that you are being compensated fairly and competitively.}\\
    % Query: \textbf{\{query\}}\\
    % Passage: \textbf{\{passage\}}\\
    Next, you need to continue several rounds of dialogue according to the first round of dialogue. Complete the above requirements step by step.\\

    \textcolor{red}{P1: How do you determine the salary range percentile?}\\
    \textcolor{red}{P2: You must first compute the difference between the highest and minimum wage figures in order to establish the salary range percentile.}\\
    \textcolor{red}{P1: Can you give me an example of this calculation?}\\
    \textcolor{red}{P2: Sure. For example, if the salary range for a particular position is between \$45,000 and \$75,000, the difference between those two figures would be \$30,000.}\\
    \textcolor{red}{P1: Is there anything else I need to do to determine the salary range?}\\
    \textcolor{red}{P2: Once you have calculated the difference between the maximum and minimum salary figures, you can use that information to generate the salary range percentile.}\\
    \textcolor{red}{P1: How do I generate the salary range percentile?}\\
    \textcolor{red}{P2: The salary range percentile is generated by comparing the salary range for a particular position to the salaries for similar positions in the same industry and geographic location.}\\
    \textcolor{red}{P1: Is the wage range percentile generated using a specific formula?}\\
    \textcolor{red}{P2: There is no specific formula for generating the salary range percentile, as it can vary depending on the company and industry. However, it typically involves researching salary data for similar positions and adjusting the salary range based on factors such as experience level, education, and geographic location.}\\
    \bottomrule
 \end{tabular}
 \caption{Zero-shot prompt for conversation generation (\textcolor{blue}{blue} = input; \textcolor{red}{red} = output).}\label{tab.conversation_generation.zero-shot} 
\end{table} 





\begin{table*}[bh] 
  \centering
 % \footnotesize
  \scriptsize
  % \tiny
   \begin{tabular}{
    m{0.96\textwidth}
    }
    \toprule
    Given a conversation and passage, first, consider the relevance of the conversation and paragraph, explore the relevant content between the conversation and passage, and then classify the relevance between the conversation and passage into one of the following categories: "Relevant" or "Not Relevant Enough". The definitions of the categories are:\\
    "Not Relevant Enough": The conversation is not relevant enough to the passage, as a small portion of the content may have appeared in the passage, but most of the content in the conversation has not appeared in the passage or is not related to the passage. The content involved in the conversation may not be consistent with the facts described in the passage. For the specific details in the passage, the conversation did not involve relevant details.\\
    "Relevant": Passage and conversation are related. The topic of the conversation is consistent with the paragraph, and the facts mentioned in the conversation are consistent with the facts in the passage. The content of the conversation is mostly related to the paragraph, and can be appropriately expanded.\\
    Here are some examples:\\
    <Passage>\\
    To determine the salary range percentile, you must first calculate the difference between the maximum and minimum salary figures. For example, if the salary range for a particular position is between \$45,000 and \$75,000, the difference between those two figures would be \$30,000.\\
% \end{tabular}
%  \caption{Continued on the next page}
% \end{table*} 

% \begin{table*}[!hbp] 
%  \ContinuedFloat
%   \centering
%  \footnotesize
%   % \scriptsize
%   % \tiny
%    \begin{tabular}{
%     m{0.96\textwidth}
%     }
    <Conversation>\\
    P1: How do you determine the salary range percentile?\\
    P2: You must first compute the difference between the highest and minimum wage figures in order to establish the salary range percentile.\\
    P1: Can you give me an example of this calculation?\\
    P2: Sure. For example, if the salary range for a particular position is between $45,000 and $75,000, the difference between those two figures would be \$30,000.\\
    P1: Is there anything else I need to do to determine the salary range?\\
    P2: Once you have calculated the difference between the maximum and minimum salary figures, you can use that information to generate the salary range percentile.\\
    P1: How do I generate the salary range percentile?\\
    P2: The salary range percentile is generated by comparing the salary range for a particular position to the salaries for similar positions in the same industry and geographic location.\\
    P1: Is the wage range percentile generated using a specific formula?\\
    P2: There is no specific formula for generating the salary range percentile, as it can vary depending on the company and industry. However, it typically involves researching salary data for similar positions and adjusting the salary range based on factors such as experience level, education, and geographic location.\\
    <Answer>\\
    The conversation is discussing how to determine the salary range percentile, which is the topic of the passage. The conversation provides an example that is consistent with the information presented in the passage, and also expands on how to generate the salary range percentile by comparing it to salaries for similar positions. The information in the conversation is relevant and consistent with the information presented in the passage. Therefore, the relecance is "Relevant".\\
    <end>\\
    \\
    <Passage>\\
    An antigen is any molecule against which a specific immune response can be generated. This means that lymphocytes are activated: 1  B-cells are activated to produce antibodies and. 2  T-cells are activated to provide helper function and cytotoxic function.\\
    <Conversation>\\
    P1: What is the function of lymphocytes?\\
    P2: Lymphocytes are a type of white blood cell that are responsible for producing antibodies.\\
    P1: What do antibodies do?\\
    P2: Antibodies attach to the antigens on the surface of bacteria or viruses, which helps the immune system identify and destroy the pathogen.\\
    P1: Can you explain what antigens are?\\
    P2: Antigens are molecules that are found on the surface of bacteria or viruses. They are recognized by the immune system as foreign invaders, and the attachment of antibodies to these antigens helps to destroy the pathogen.\\
    P1: Are different antibodies needed for different types of antigens?\\
    P2: Yes, different organisms have different antigens, so a different antibody is needed to recognize each different type of antigen.\\
    P1: What happens when an antibody attaches to a virus?\\
    P2: When an antibody attaches to a virus, it can either destroy the pathogen or make it easier for a phagocyte to ingest and destroy it. This is an important part of the immune response to viral infections.\\
    <Answer>\\
    Although the conversation is related to the immune system and antibodies, it does not provide any additional information about the functions of B-cells or T-cells as described in the passage. The conversation also does not mention anything about the specific immune response generated by antigens, which is the main topic of the passage. While the conversation provides general information about the immune system and antibodies, it does not expand on the content of the passage or provide any additional relevant details. Therefore, the relecance is "Not Relevant Enough".\\
    <end>\\
    \\
    Please predict whether the conversation is relevant to the passage or not. The answer should be exact "Not Relevant Enough" or "Relevant".\\
    \\
    <Passage>\\
    \textbf{\{passage\}}\\
    <Conversation>\\
    \textbf{\{conversation\}}\\
    \bottomrule
 \end{tabular}
 \caption{Few-shot chain-of-thought prompt used to filter out irrelevant conversations.}\label{tab.data_filter.few-shot-cot} 
\end{table*} 

\clearpage
\section{Details on Human Evaluation}\label{sec.appendix.g}
For the purpose of human evaluation, we begin by presenting annotators with a multi-turn conversation accompanied by a paired passage. Their task involves carefully reading both the conversation and passage, ensuring a comprehensive grasp of the main topics and any significant details. Subsequently, they are required to assess the fluency of the conversation, as well as its relevance and consistency with the provided passage.
\subsection{Fluency}
To evaluate the fluency of the generated conversation, annotators should answer the first question:\\
How fluent do you think the conversation is? 

Following previous study \cite{he-yiu-2022-controllable}, annotators need to score the fluency of the conversation on a 5-point Likert scale from 1 to 5, based on the following rules:\\
    1: The conversation cannot be understood and all segments are not fluent.\\
    2: The conversation cannot be understood, but some segments are fluent.\\
    3: The conversation can be understood to some extent, but with many grammatical errors.\\
    4: The conversation can be understood with several grammatical errors.\\
    5: The conversation is extremely fluent without any grammatical errors.\\

\subsection{Relevance}
To assess the relevance between the generated conversation and the paired passage, graders need to consider the second question:\\
Q2: How relevant do you think the conversation is to the given passage? 

Specifically, graders need to score the relevance between the conversation and the given passage on a 3-point Likert scale from 1 to 3:\\
    1 (Irrelevant): Any topic discussed in the conversation is completely unrelated to the given passage.\\
    2 (Not Relevant Enough): Few topics discussed in the conversation are related to the given passage. \\
    3 (Relevant): Most topics discussed in the conversation are related to the given passage.\\
    
\subsection{Consistency}
As for consistency, graders should answer the following question: \\
Q3: How consistent do you think the conversation is to the given passage? 

To be concrete, graders need to score the consistency between the conversation and the given passage on a 3-point Likert scale from 1 to 3:\\
    1: Any fact mentioned in the conversation does not appear in the given passage.\\
    2: Few facts mentioned in the conversation are supported by the facts in the given passage.\\
    3: Most facts mentioned in the conversation are consistent with the facts in the passage. \\


We show the human evaluation results in Table \ref{table.ConIR.human2}. 
\begin{table}
\footnotesize
  \centering
    \begin{tabular}{
     m{0.06\textwidth}<{\centering}|
     m{0.1\textwidth}<{\centering}
     m{0.1\textwidth}<{\centering}
     m{0.1\textwidth}<{\centering}
     }
    \hline
    \textbf{Score} & \textbf{Fluency}  &\textbf{Relevance} &\textbf{Consistency} \\
    \hline 
    1 & 0 & 5& 16\\
    2 & 0 & 132& 144\\
    3 & 0& 163& 140\\
    4 & 3 & -& -\\
    5 & 297 & -& -\\
    \hline
    Average & 4.99 & 2.53& 2.41\\
    \hline

    \hline
  \end{tabular}
  \caption{Human evaluation results on ConIR. The first five rows display the frequency distribution of each annotation score. The last row represents the average score of the annotations.
  }\label{table.ConIR.human2}
\end{table}

\end{document}
