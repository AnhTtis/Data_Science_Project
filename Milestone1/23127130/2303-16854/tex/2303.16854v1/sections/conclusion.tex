\section{Conclusion}
In this paper, we introduce AnnoLLM, a novel annotation system powered by LLMs that has the potential to replace traditional crowdsourced annotators. 
Additionally, we propose a two-step approach, `explain-then-annotate', to enhance the data annotation capabilities of LLMs. The approach leverages LLMs to generate a few-shot chain-of-thought prompt, which is then used to annotate unlabeled data.
Our experimental results on three datasets demonstrate the feasibility of using LLMs to substitute crowdsourced annotators, which highlights the potential to facilitate the development of using LLMs like GPT-3.5 to annotate data for various NLP tasks. 
