\subsection{Directed graphs}\label{subsec:directed}

Directed graphs are graphs in which for any edge it is also taken into account the directionality, meaning that any edge is defined by a source vertex $s$ and a target vertex $t$. Therefore, for two vertices $u$ and $v$, there can exist edges $(u,v)$ and $(v,u)$. In this context, the classical notion of density \rev{becomes} meaningless, since it does not take into account the directionality. The first notion of density for directed graphs was proposed by Kannan and Vinay \cite{Kannan}:
given a directed \rev{graph} $G=(V,E)$, the Directed Densest Subgraph problem (DDS) requires to find two set of nodes $S$ and $T$ that maximize $f(S,T) = |E(S,T)| / \sqrt[]{|S|\cdot|T|}$, where $E(S,T)$ is the set of edges having its source in $S$ and target in $T$. \rev{This density notion prioritizes pairs of sets of nodes} with a massive number of connections from one to the other; the square root at the denominator ensures that the output is not a single edge, which would take maximum without it.
\rev{The authors} proposed an $\bigO(\log n)$-approximation algorithm for DDS, by relating the objective function to the singular value of the adjacency matrix and applying Monte Carlo algorithm for its computation. Charikar~\cite{Charikar2000} obtained an exact \rev{algorithm} and a 2-approximation algorithm based on $\bigO(n^2)$ LPs. The latter resembles the greedy peeling for DSP running in $\bigO(n+m)$ \rev{time}, yielding \rev{the total time complexity} of $\bigO(n^2(n+m))$. For both algorithms it was \rev{also} observed that with $\bigO(\log n /\epsilon)$ execution of LPs, it is possible to output $(1+\epsilon)$- and  $(2+\epsilon)$-approximations, \rev{respectively}.
Khuller and Saha \cite{Khuller2009Dense} gave the first max-flow-based polynomial-time algorithm for DDS; likewise for DSP, this algorithm requires $\bigO(\log n)$ executions of max flow instances, or $\bigO(1)$ executions of parametrized maximum flow problem. \rev{The authors also} claimed to have obtained a 2-approximation algorithm with time complexity $\bigO(n+m)$, that was recently disproved by Ma et al. \cite{ma2021directed} by a counter-example, \rev{where} the authors reported an alternative 2-approximation algorithm provided by Ma~\cite{saha2029fix}, whose time complexity is $\bigO(n(n+m))$.
Bahmani et al. \cite{bahmani} developed a $2(1 + \epsilon)$-approximation algorithm, that ends in $\bigO(\log_{1+\epsilon}n)$ passes over the input graph.

Sawlani and Wang \cite{sawlani2020dynamic} reduced DDS to HDSP, relying on the knowledge of the ratio between sizes of $|S|$ and $|T|$ for the optimal solution. Since this quantity is unknown, they \rev{proved} that $O(\log n/\epsilon)$ \rev{solutions} of a $(1 + \epsilon/2)$-approximation algorithm for HDSP \rev{can guarantee} a $(1+\epsilon)$-approximate solution for DDS. Furthermore, the authors introduced different proposals for densest subgraph algorithms in \rev{a} fully dynamic setting for undirected graphs and vertex-weighted undirected graphs; therefore, their reduction of DDS to HDSP led to the first algorithm for DDS in \rev{a} fully dynamic setting, that maintains a $(1+\epsilon)$-approximate solution with worst-case time \rev{$O(\poly(\log n, \epsilon^{-1}))$} per update. Chekuri et al. \cite{Chekuri2022supermod} followed this reduction to adapt their DSP flow-based algorithm for DDS, that outputs a $(1+\epsilon)$-approximation in $\tilde \bigO(m /\epsilon^2)$ \rev{time}.

\rev{Ma et al. \cite{ma2021directed,ma2020efficient} obtained different results in this context.} Introducing the notion of $[x,y]$-core, the directed counterpart of the $k$-core notion for undirected graphs, they \rev{proved} that \rev{an optimal solution to DDS can be identified} through it with theoretical guarantees.
As a first direct consequence, these results enable the maximum-flow based exact algorithm to be executed only on a reduced version of the graph, composed by some $[x,y]$-cores, making it faster. Furthermore, they provided a divide-and-conquer strategy to carefully select the optimal value of $|S|/|T|$, that reduces the possible values from $n^2$ to $k$, with $k \ll n^2$.
Furthermore, they proved that a particular instance of $[x,y]$-core is a 2-\rev{approximate solution}, \rev{and thus proposed} an algorithm to find it that runs in $\bigO(\sqrt{m}(n+m))$.
Ma et al. \cite{ma2022convex} designed a new LP formulation for DDS, with a Frank-Wolfe based algorithm to optimize the associate dual. With this new LP, they were able to design a new algorithmic framework to reduce the number of LP instances to solve. More precisely, they provided a $(1+\epsilon)$-\rev{approximate solution} with \rev{$\bigO(T_\mathrm{FW}\log_{1+\epsilon}n)$}, where $T_\mathrm{FW}$ is the time complexity for solving a single instance of their LP.

