\subsection{Multilayer networks}\label{subsec:multilayer}

Multilayer networks are a generalization of the ordinary (i.e., single-layer) graphs.
For positive integer $\ell$, let $[\ell]=\{1,2,\dots, \ell\}$.
Mathematically, a multilayer network is defined as a tuple $(V,(E_i)_{i\in [\ell]})$,
where $V$ is the set of vertices and each $E_i$ ($i=1,2,\dots, \ell$) is a set of edges on $V$.
That is, a multilayer network has a number of edge sets (called layers),
which may encode different types of connections and/or time-dependent connections over the same set of vertices.
As the density value of $S\subseteq V$ varies layer by layer,
there would be several ways to define the objective function of multilayer-network counterparts of DSP.
For $S\subseteq V$ and $i\in [\ell]$, we denote by $d_i(S)$ the density of $S$ in terms of the layer $i$. Jethava and Beerenwinkel~\cite{jethava2015relational} introduced the first optimization problem for dense subgraph discovery in multilayer networks,
which they referred to as the Densest Common Subgraph Problem (DCSP).
In the problem, given a multilayer network $G=(V,(E_i)_{i\in [\ell]})$, we seek a vertex subset $S\subseteq V$ that maximizes
the minimum density over layers, i.e., $\min_{i\in [\ell]} d_i(S)$.
They devised an LP-based polynomial-time heuristic and a $2\ell$-approximation algorithm based on the greedy peeling.
Reinthal et al.~\cite{reinthal2016finding} studied \rev{a} simplex method and \rev{an} interior-point method for the above LP 
and observed that employing the interior-point method can shorten the computation time in practice.
Galimberti et al.~\cite{galimberti2017core,galimberti2020core} introduced a generalization of DCSP,
which they refer to as the multilayer densest subgraph problem.
This problem aims at optimizing a trade-off between the minimum density value over layers and the number of layers having such a density value.

Later, Charikar et al.~\cite{charikar2018common} designed two combinatorial polynomial-time algorithms with approximation ratios
$O(\sqrt{n\log \ell})$ and $O(n^{2/3})$ (irrespective of $\ell$), respectively.
Moreover, they showed some strong inapproximability results for the problem, based on some assumptions.
Specifically, they showed that \rev{DCSP} is at least as hard to approximate as \textsc{MinRep}, a well-studied minimization version of \textsc{Label Cover}, which implies that the problem cannot be approximated to within a factor of $2^{\log^{1-\epsilon}n}$, unless $\NP \subseteq \text{DTIME}(n^{\textsf{polylog}(n)})$.
They also showed that if the planted dense subgraph conjecture is true, the problem cannot be approximated to within a factor of $n^{1/4-\epsilon}$ and even for $\ell=2$, the problem cannot be approximated to within $n^{1/8-\epsilon}$.

Recently, Hashemi et al.~\cite{hashemi2022firmcore} designed a sophisticated core decomposition algorithm for multilayer networks,
which they call the FirmCore decomposition algorithm.
For $k\in \mathbb{Z}_+$ and $\lambda\in [\ell]$, a subgraph $H=(S,(E_i[S])_{i\in [\ell]})$ is called a $(k,\lambda)$-FirmCore
if it is a maximal subgraph in which every vertex has degree no less than $k$ in the subgraph for at least $\lambda$ layers.
They devised a polynomial-time algorithm for finding the set of $(k,\lambda)$-FirmCores for all possible $k$ and $\lambda$.
They proved that the FirmCore decomposition unfolds an approximate solution to the multilayer densest subgraph problem,
with a better approximation ratio than that obtained by~\cite{galimberti2017core} for many instances.

Very recently, Kawase et al.~\cite{kawase2023stochastic} studied stochastic solutions to dense subgraph discovery in multilayer networks.
Their novel optimization problem asks to find a stochastic solution, i.e., a probability distribution over the family of vertex subsets, rather than a single vertex subset,
whereas it can also be used for obtaining a single vertex subset.
The quality of stochastic solutions is measured using the expectation of the following three metrics, the density, the robust ratio, and the regret,
on the layer selected by the adversary.
Therefore, their optimization problem can be seen as (a generalization of) the stochastic version of DCSP.
Unlike DCSP, their optimization problem can be solved exactly in polynomial time;
indeed, they designed an LP-based polynomial-time exact algorithm.
They proved that the output of the proposed algorithm has a useful structure;
the family of vertex subsets with positive probabilities has a hierarchical structure.
This leads to several practical benefits, e.g., the largest size subset contains all the other subsets and the optimal solution obtained by the algorithm has support size at most $n$.

Finally, we take a look at a very special case of multilayer networks called dual networks, i.e., the case of $\ell=2$ in multilayer networks.
Wu et al.~\cite{wu2015dual,Wu+16} introduced an optimization problem of detecting a dense and connected subgraph in dual networks:
given a dual network $G=(V,(E_1,E_2))$, we are asked to find $S\subseteq V$ that maximizes $d_1(S)$, i.e., the density on the first layer,
under the constraint that $(S,E_2[S])$ is connected.
They proved that the problem is \NP-hard and designed a scalable heuristic.
Later, Chen et al.~\cite{Chen+22} considered a variant of the problem,
where $k\in \mathbb{Z}_+$ is given as an additional input, and we seek $S\subseteq V$ that maximizes the minimum degree of vertices on the first layer,
under the constraint that $(S,E_2[S])$ is $k$-edge-connected, 
enabling to control the strength of connectivity on the second layer.
Owing to the use of the minimum degree, this problem can be solved exactly in polynomial time, unlike the above problem by Wu et al.~\cite{wu2015dual,Wu+16}.
\rev{Feng et al.~\cite{feng2021specgreedy} considred another variant, where we seek a vertex subset that 
maximizes a function prioritizing the density on the first layer while discarding some subsets that are dense even on the second layer.
}

Yang et al. \cite{yang2018mining} proposed the density contrast subgraph problem: Given \rev{a dual network, find a vertex subset that maximizes the difference between densities on the two layers}. They \rev{reduced} the maximum clique problem to their problem, proving that the problem is \NP-hard and cannot be approximated within $O(n^{1-\epsilon})$ for any $\epsilon > 0$. They \rev{solved} the problem via a variant of the greedy peeling, providing an $O(n)$-\rev{approximation}.
Lanciano et al. \cite{lanciano2020contrast} proposed a variant of this problem, by considering maximizing the difference in terms of the number of edges, subject to an input penalty term that controls the output of the solution. They showed that this problem can be mapped to an instance of GOQC (\rev{see} Section \ref{sec:negative_weights}).


