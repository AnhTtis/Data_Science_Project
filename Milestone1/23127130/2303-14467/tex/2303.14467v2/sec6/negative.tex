\subsection{Negatively weighted graphs}
\label{sec:negative_weights}
\rev{HDSP, reviewed in the previous section}, was defined on positive weights.
Recently some specific applications (e.g., networks derived from correlation matrices) are requiring to extract dense subgraphs from networks whose edges can be negatively weighted. When we allow negative weights, most of the algorithmic results for DSP no longer \rev{hold}. Cadena et al. \cite{cadena2016dense} were the first to analyze \rev{dense structures} in graphs with negative weights on edges. Building on top of \cite{tsourakakis2013denser} they defined the Generalized Optimal Quasi Clique (GOQC) problem as follows: given a \rev{graph} $G=(V,E)$, a weight function $w: E \rightarrow \mathbb{R}$ and a penalty function $\bar{\alpha}: E \rightarrow \mathbb{R}$, the goal is to find a subset of nodes $S$ that maximizes $f_{\bar{\alpha}}(S) = \sum_{\{u,v\} \in E[S]}(w(u,v) - \bar{\alpha}(u,v))$.
The authors showed that the problem is \NP-hard to approximate within a factor of $\bigO(n^{1/2-\epsilon})$, and proposed an algorithm composed by solving SDP \rev{by refining} the local search algorithm by \cite{tsourakakis2013denser}.

Tsourakakis et al. \cite{tsourakakis2019novel} tackled the Densest Subgraph with Negative Weights \rev{problem (DSNW)} defined as follows: given a graph $G=(V,E)$, a weight function $w: E \rightarrow \mathbb{R}$, the goal is to find a subset of nodes $S$ that maximizes $\sum_{\{u,v\} \in E[S]}w(u,v)/|S|$.
The authors proved that the problem is \NP-hard via a reduction based on the fact that the max-cut problem on graphs with possibly negative edges is \NP-hard.
In order to efficiently solve this problem, they analyzed the performance of the greedy peeling \rev{algorithm} for DSP, proving that for this problem it achieves an approximation of $\frac{\rho^*}{2}-\frac{\Delta}{2}$, where $\rho^*$ is the optimal value and $\Delta$ is the \rev{largest absolute value of the negative degrees}.

