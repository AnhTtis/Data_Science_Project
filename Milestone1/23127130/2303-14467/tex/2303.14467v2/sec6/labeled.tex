\subsection{Edge- and vertex-labeled graphs}\label{subsec:labeled}
An edge-labeled (\rev{vertex-labeled, resp.}) graph is a type of graph where each edge (\rev{vertex, resp.}) is assigned one or more specific labels or attributes.  
The simplest example of an edge- and/or vertex-labeled graph is a graph where each edge and/or each vertex is associated with a scalar attribute, i.e., a weight. \rev{DSP} on weighted graphs was already studied by Goldberg~\cite{goldberg1984finding}, 
but recently dusted-off by Fazzone et al.~\cite{Fazzone2022}, that dubbed it as Heavy and Dense Subgraph Problem (HDSP).
Given a simple undirected graph $(G, V, E, w_V, w_E)$, where $w_V: V \rightarrow \mathbb{R}_+$ and $w_E: E \rightarrow \mathbb{R}_+$,
HDSP requires to find \rev{$S \subseteq V$} that maximizes
$(e[S] + \sum_{s \in S} w_V(s)) /|S|$, where $e[S] = \sum_{e \in E[S]} w_E(e)$.
HDSP is \rev{polynomial-time solvable}: 
Goldberg~\cite{goldberg1984finding} gave an exact polynomial-time algorithm based on a reduction to the $s$-$t$ maximum flow / $s$-$t$ minimum cut problem.
Recently, Fazzone et al.~\cite{Fazzone2022} studied the approximation guarantees of the greedy peeling algorithm (\rev{Section~\ref{subsec:approx}}) when applied to HDSP, and then adapted the iterative greedy peeling of Chekuri et al.~\cite{Chekuri2022supermod} to HDSP.

Anagnostopoulos et al.~\cite{anagnostopoulos2020fair} tackled the algorithmic fairness issue \rev{of DSP} by defining the Fair Densest Subgraph Problem (FDSP) on a graph with binary labeling of vertices. 
FDSP \rev{computes} 
$S \subseteq V$
of maximum density while ensuring that
$S$
contains an equal number of representatives of either label, guaranteeing that the binary-protected attribute is not disparately impacted. 
The authors proved that FDSP is \NP-hard and approximating \rev{FDSP} with a polynomial-time algorithm is at least as hard as Dam$k$S, for which no constant-factor approximation algorithms are known (see Section~\ref{sec:size}).
On the other hand, any $\alpha$-approximation to Dam$k$S is a $2\alpha$-approximation to FDSP. 
 In the case where the input graph \rev{is} fair, the authors \rev{introduced} a polynomial-time $2$-approximation algorithm for the problem and \rev{proved} the tightness of this approximation factor under the small set expansion hypothesis~\cite{raghavendra2010expansion}.
The authors also \rev{devised} an algorithm, based on a spectral embedding, \rev{enabling} to retrieve an approximate solution with theoretical guarantees in polynomial time for the case where the input graph is an expander that contains an almost-regular dense subgraph.

\rev{
Very recently, Miyauchi et al.~\cite{Miyauchi+23} studied the problem of finding a densest diverse subgraph in a graph 
whose vertices have different attribute values/types that they refer to as colors. 
Let $C$ be a set of colors. 
The input graph $G=(V,E)$ is associated with a color function $\ell:V\rightarrow C$ that assigns a color to each vertex. 
They proposed two novel optimization models motivated by different realistic scenarios. 
The first model, called the \emph{Densest Diverse Subgraph Problem} (DDSP), maximizes the density 
while guaranteeing that no color represents more than a given fraction $\alpha\in [1/|C|,1]$ of vertices in the output, which is a generalization of the above FDSP. 
Varying the fraction $\alpha$ enables to range the diversity constraint and to interpolate from a diverse dense subgraph where all colors have to be equally represented 
to an unconstrained densest subgraph. 
They designed a scalable $O(\sqrt{n})$-approximation algorithm for the case where $V$ is a feasible solution. 
The second model, called the \emph{Densest at-least-$\vv{k}$-Subgraph problem} (Dal$\vv{k}$S), is motivated by the setting where any specified color should not be overlooked. 
As its name suggests, Dal$\vv{k}$S is a generalization of Dal$k$S, where $\vv{k}$ represents cardinality demands with one coordinate per color class. 
They designed a $3$-approximation algorithm using LP together with an acceleration technique. 
Computational experiments demonstrate that densest subgraphs in real-world graphs have strong homophily in terms of colors, 
emphasizing the importance of considering the diversity, and the proposed algorithms contribute to extracting dense but diverse subgraphs. 
}

Tsourakakis et al.~\cite{tsourakakis2019novel} considered \rev{solving} DSP in edge-(multi)labeled networks with exclusion queries: given a \rev{graph} $G = (V,E)$ in which any edge \rev{is} associated \rev{with} a subset of labels in the label universe $L$, and an input set $l \subseteq L$, find the densest subgraph that contains only edges whose label is contained in $l$. To solve this problem, they resort to an heuristic based on the Densest Subgraph with Negative Weights problem (see \refsec{negative_weights}).

Rozenshtein et al.~\cite{rozenshtein2020mining} studied the problem of finding a dense edge-induced subgraph in an edge-labeled graph whose edges are similar to each other based on a given similarity function of the labels. The authors modeled the problem \rev{using the} objective function that is the sum of \rev{density and similarity terms}, and \rev{introduced} a Lagrangian relaxation problem, for which they \rev{proposed} an algorithm based on parametric min-cut, requiring $\bigO(m^3 \log m)$ time and $\bigO(m^2)$ space.
