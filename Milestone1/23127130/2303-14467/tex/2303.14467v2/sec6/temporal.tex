\subsection{Temporal networks}
Temporal networks are a special case of multilayer networks, where the different layers \rev{correspond} to different time slices, so that the order of the layers is important. \rev{Another definition of} a temporal network is a graph, where each edge is \rev{associated with} a timestamp from a discrete temporal domain. 
More formally we can define a temporal network as a graph $G = (V, E, T)$ where $V$ denotes the set of vertices, $T = \{0,1,..., t_\text{max}\} \subset \mathbb{N}$ is the time domain, and $E \subseteq V \times V \times T$.

Bodganov et al. \cite{bogdanov2011heavy} were the first to address \rev{a variant of DSP} in temporal networks. Allowing the presence of negatively weighted edges, they defined the problem of finding the heaviest temporal subgraph, i.e., a set of nodes and a time-interval maximizing the edge's weights summation, and \rev{showed} that it is \NP-hard even with edge weights restricted in $\{-1, +1\}$.
The same problem has been recently tackled again by Ma et al. \cite{ma2020temporal}, that proposed a new heuristic \rev{algorithm}. 

Rozenshtein et al. \cite{rozenshtein2017dynamic} proposed to find the densest temporal subgraph in the latter representation of the temporal network, 
by imposing a constraint on the maximum number of timestamp to be included in the time-interval, and another on the maximum length of the span of the time-interval.
Lately, Rozenshtein et al. \cite{Rozenshtein2019segmentation} introduced the $k$-densest-episodes problem, where an episode is defined as a pair $S = \{I, H\}$ with $I = \{i_1, i_2\}$ representing the time-interval and $H \subseteq V$ as a set of nodes. The objective is to find the $k$ episodes that are densest in $k$ disjoint time-intervals.
The same problem has been recently tackled by Dondi and Hosseinzadeh \cite{Dondi2021}, that proposed an heuristic \rev{running} in $O(|T| + (k + \tau) t_\text{DSP})$, where $\tau$ is the maximum number of iterations and $t_\text{DSP}$ is \rev{the time complexity required for solving DSP}.

Angel et al.~\cite{Angel2013} tackled real-time story identification on Twitter via maintenance of densest subgraph in the fully dynamic setting.
Similarly, Bonchi et al. \cite{BonchiBGS16,BonchiBGS19} adopted anomalously dense subgraphs in temporal networks as a way to identify buzzing stories in social media.
To recognize a story as buzzing, it needs to have high density in the interactions (or co-occurrences) among 
all objects (terms or entities) therein and for all time instants in a temporal window.
Thus they defined the density of a subgraph in a given time interval as the minimum degree, among all vertices of the subgraph and all timestamps of the temporal window.
They showed that the problem of finding the densest subgraph, for a given time interval, can be solved exactly by the peeling algorithm to construct the core decomposition, and by returning the innermost core. 
Chu et al. \cite{chu2019online} defined the density bursting subgraph as a subgraph that accumulates its density at the fastest speed in a temporal network, according to their definition of the burstiness of a subgraph.

Semertzidis et al.~\cite{semertzidis2019finding} introduced another generalization of the densest common subgraph problem, called the Best Friends Forever (BFF) problem,
in the context of evolving graphs with a number of snapshots.
The BFF problem is a series of optimization problems that maximize an \emph{aggregate density} over snapshots,
where the aggregate density is set to be the average/minimum value of the average/minimum degree of vertices over layers.
Similarly to the multilayer densest subgraph problem,
they also considered the variant called the On--Off BFF ($\text{O}^2$BFF) problem, which only asks the output to be dense for a part of snapshots.
They investigated the computational complexity of the problems and designed some approximation \rev{and} heuristic algorithms.


Zhu et al. \cite{Zhu2022}
weighted each edge according to the number of time-stamps in which the edge exists and tackled the problem \rev{of maximizing} the density divided by the number of time-stamps in the relative time-interval, by imposing a constraint on the minimum number of time-stamps to take into account. For this problem, the authors designed \rev{exact and heuristic algorithms}.

The most recent contribution in this context is by Qin et al. \cite{qin2023periodic}, which defined a $\sigma$-periodic subgraph to be a subgraph whose occurrences are exactly $\sigma$ in the temporal graph, and the time difference between any pair of consequent occurrences is the same.
They proposed efficient strategies to prune the search space in temporal graphs, in order to run \rev{a max-flow algorithm for} an instance of reduced size.

