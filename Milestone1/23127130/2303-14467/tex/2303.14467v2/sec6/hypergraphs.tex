\subsection{Hypergraphs}\label{subsec:hyper}
Hypergraphs are a generalization of graphs, consisting of a set $V$ of vertices and a set $E$ of \rev{hyperedges}, that are composed by an arbitrary number of vertices. Huang and Kahng \cite{Huang-Kahng95} formally introduced the Densest Subhypergraph problem (DSH), that given \rev{a} hypergraph $G=(V,E)$, requires to find $S\subseteq V$ that maximizes \rev{$d(S) = e[S]/|S|$, where $e[S]=|\{e\in E\mid e\subseteq S\}|$}, and proposed a \rev{polynomial-time} flow-based \rev{exact} algorithm.


\rev{
Very recently, Huang et al.~\cite{Huang+23} generalized DSH to the model with a seed set, 
inspired by the aforementioned ADS (see Section~\ref{subsec:seed}). 
Their model, called the Anchored Densest Subhypergraph (ADSH), is a common generalization of DSH and ADS, 
while the anchored node set is limited to the emptyset. 
Therefore, the model seeks a densest subhypergraph that is close to the given reference node set. 
Its noteworthy feature is the consideration of the so-called locality parameter, 
specifying how the output should be close to the given reference node set. 
The authors presented a flow-based exact algorithm for ADSH, 
based on that for DSP and the recent development of computing a minimum $s$--$t$ cut on hypergraphs~\cite{Veldt+22}. 
This result remains valid even for some generalizations of ADSH, including HDSP (see Section~\ref{subsec:labeled}), 
on hypergraphs with positive and even negative node weights. 
The main algorithmic contribution is to develop a strongly-local exact algorithm for ADSH, 
i.e., an algorithm outputting an optimal solution in time depending only on the parameters of the reference node set. 
The algorithm runs in an iterative fashion: in the first iteration, the algorithm deals with the subhypergraph consisting of the reference nodes and their neighbors, and the later iterations, it computes the minimum $s$--$t$ cut in a hypergraph constructed from the current subhypergraph and expands it using the information of the cut computed. 
}

\rev{Miyauchi et al. \cite{Miyauchi+15} \rev{studied} two general optimization models, in the context of advertising budget allocation, i.e., 
the maximum general-thresholds coverage problem, in which the densest $k$-subhypergraph 
(i.e., the variant of DSH in which the output size is constrained to be equal to $k$) falls, 
and its cost-effective counterpart, in which DSH itself falls. 
For the first model they proposed two different greedy algorithms, while for the second problem they designed a scalable approximation algorithm.
}

Hu et al. \cite{hu2017dynamicsub} \rev{generalized the results by Tsourakakis~\cite{Tsourakakis15}} to hypergraphs, 
by designing \rev{LP-based and flow-based algorithms}. 
Furthermore, they addressed the densest subhypergraph maintenance in \rev{a} dynamic setting, 
by providing two algorithms that maintain a $r(1+\epsilon)$-approximation in the case where there are only edge insertions, 
and a $r^2(1+\epsilon)$-approximation in \rev{a} fully dynamic setting, where $r$ is the rank of the hypergraph, i.e., $r=\max_{e\in E} |e|$. 
Chlamtac et al. \cite{Chlamtac+18} performed a theoretical analysis on \rev{the} densest $k$-subhypergraph problem and provided bounds over 3-uniform hypergraphs; 
Corinzia et al.~\cite{corinzia2022statistical} considered recovering the planted densest $k$-subhypergraph in a $d$-uniform hypergraph (i.e., \rev{a} hypergraph in which for any $e \in E$, $|e| = d$ holds) \rev{and provided} tight statistical bounds on recovering quality and algorithmic bounds based on approximate message passing algorithms.

\rev{The \rev{recent results} by Chekuri et al.~\cite{Chekuri2022supermod} also have generalization to hypergraphs, e.g., 
a fast $(1+\epsilon)$-approximation algorithm based on max flow.
Furthermore, a natural generalization of {\sc Greedy++} (in Section~\ref{subsubsec:iterative_peeling}) to hypergraphs provides a
$(1+\epsilon)$-approximation for DSH.}
The iterative algorithm proposed by Harb et al.~\cite{harb2022faster} \rev{also has a generalization to hypergraphs}; it provides an $\epsilon$-additive approximate solution for DSH in
$\bigO(\sqrt{pr\Delta} / \epsilon)$ iterations,
where each iteration requires $O(p\log r)$ time and admits some level of parallelization. Here $\Delta=\max_{v \in V} |\{e\in E : v \in e\}|$ and $p = \sum_{e \in E} |e|$.

Zhou et al. \cite{zhou2022extracting} \rev{introduced a generalization of DSH by considering the fact that} there might be \rev{some} hyperedges only partially included in the solution, \rev{which} were not counted in the objective function of DSH but might be relevant in \rev{some} specific applications. Therefore, they \rev{defined} a weighting scheme according to the number of vertices of any hyperedge included in the solution, and \rev{a} maximization problem based on it. For this problem they \rev{proposed} exact and $r$-approximation algorithms, where $r$ is the rank of the hypergraph.

Finally, Bera et al. \cite{bera2022dynamicsub} very recently designed a new algorithm for the dynamic setting. They improved the approximation ratio to $(1+\epsilon)$, making it independent \rev{of} the hypergraph rank, with a similar update time to that required in \cite{hu2017dynamicsub}. 


