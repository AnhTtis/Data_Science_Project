This section delves into variants of DSP, defined by means of constraints.
We will begin covering size-constrained problems, which set limits on the desired size of the output \rev{subset}.
Next, we will investigate seed-set problems, where an initial set of nodes --- referred to as seed set --- guides the search for the densest subgraph.
Finally, we will examine problems with connectivity constraints, where the output subgraph must meet specific requirements in terms of connectivity, to prevent potential vertex/edge failures.


\subsection{Size constraints}
\label{sec:size}
Size-constrained versions of DSP are \rev{well studied} in literature, \rev{as they 
find applications in several real-world contexts}. 
While DSP is polynomial-time \rev{solvable}, adding size constraints makes it \NP-hard.

\subsubsection{Densest $k$-subgraph problem}\label{subsubsec:DkS}
Given a simple undirected graph $G=(V,E)$ and a positive integer $k$, the \emph{Densest $k$-Subgraph problem} (D$k$S) requires to find a vertex subset $S\subseteq V$ that maximizes $d(S)=e[S]/|S|$ subject to $|S|=k$.
As the size of solutions is fixed, the objective function can be reduced to $e[S]$.
It is easy to see that the maximum clique problem can be reduced to D$k$S; therefore, the problem is \NP-hard.
D$k$S is known not only as a variant of DSP but also as one of the most fundamental combinatorial optimization problems, and would deserve a survey of its own. 

Feige et al.~\cite{FPK01} proposed a combinatorial polynomial-time $O(n^{1/3-\delta})$-approximation algorithm for some tiny $\delta >0$.
Later, Goldstein and Langberg~\cite{Goldstein2009dense} estimated the above approximation ratio of $O(n^{1/3-\delta})$ and concluded that it is approximately equal to $O(n^{0.3226})$.
In addition, they presented an algorithm with a slightly better approximation ratio of $O(n^{0.3159})$.
Bhaskara et al.~\cite{Bhaskara+10} proposed an $O(n^{1/4+\epsilon})$-approximation algorithm running in $n^{O(1/\epsilon)}$ time, for any $\epsilon >0$.
This approximation ratio is the current \rev{state-of-the-art} for D$k$S.
The algorithm is based on a clever procedure that distinguishes random graphs from \rev{some} random graphs with planted dense subgraphs.

In addition, there are some algorithms with an approximation ratio depending on the parameter $k$.
Asahiro et al.~\cite{Asahiro2000} demonstrated that the straightforward application of the greedy peeling algorithm attains the approximation ratio of $O(n/k)$.
Later, Feige and Langberg~\cite{FL01} employed semidefinite programming (SDP) and achieved an approximation ratio somewhat better than $O(n/k)$.

There exist also approximation algorithms for specific instances of D$k$S.
Arora et al.~\cite{AKK95} proved that there exists a polynomial-time approximation scheme (PTAS) for D$k$S on $G$ with $m=\Omega(n^{2})$ and $k=\Omega(n)$.
Ye and Zhang~\cite{Ye2003approximation} developed an SDP-based polynomial-time $1.7048$-approximation algorithm for D$k$S with $k=n/2$,
which improves some previous results, e.g., a trivial randomized $4$-approximation or $2$-approximation algorithm using LP~\cite{Goemans1996mathematical} or SDP~\cite{Feige1997densest}.
Liazi et al.~\cite{Liazi2008constant} presented a polynomial-time $3$-approximation algorithm for D$k$S with chordal graphs.
Later, Chen et al.~\cite{Chen2010densest} developed a polynomial-time constant-factor approximation algorithm for D$k$S with a variety of classes of intersection graphs, including chordal graphs and claw-free graphs.
They also proposed a PTAS for D$k$S on unit disk graphs, which improves the previous $1.5$-approximation for D$k$S on proper interval graphs, a special case of unit disk graphs~\cite{Backer2010constant}.
Papailiopoulos et al.~\cite{Papailiopoulos2014finding} designed an algorithm for D$k$S that looks into a low-dimensional space of dense subgraphs. The approximation guarantee depends on the graph spectrum, which is effective for many graphs in applications. The algorithm runs in nearly-linear time under some mild assumptions of the graph spectrum, and is highly parallelizable.
Khanna and Louis~\cite{khanna2020planted} introduced semi-random models of instances with a planted dense subgraphs and studied the approximability of D$k$S.
They showed that approximation ratios better than $O(n^{1/4+\epsilon})$ can be achieved for a wide range of parameters of the models.

The literature is also rich in inapproximability results for D$k$S.
It is known that D$k$S has no PTAS under some reasonable computational complexity assumptions.
For example, Feige~\cite{Feige02} assumed that random 3-SAT formulas are hard to refute,
Khot~\cite{Khot06} assumed that \NP does not have any randomized algorithm running in subexponential time,
and Raghavendra and Steurer~\cite{raghavendra2010expansion} assumed a strengthened version of the unique games conjecture (UGC).
\rev{Bhaskara et al.~\cite{Bhaskara+12} studied the inapproximability of D$k$S from the perspective of SDP relaxations and devised lower bounds on the integrality gaps of strong SDP relaxations, showing that beating a factor of $n^{\Omega(1)}$ is a barrier even for the most powerful SDPs.}
Manurangsi~\cite{Manurangsi2017Almost} proved that D$k$S cannot be approximated up to a factor of $n^{\frac{1}{\left(\log \log n\right)^c}}$, 
for some $c>0$ assuming the Exponential Time Hypothesis (ETH)~\cite{Impagliazzo2001Complexity}.
Braverman et al.~\cite{braverman2017eth} ruled out a PTAS in terms of the additive approximation, assuming ETH.
\rev{Very recently, Chuzhoy et al.~\cite{Chuzhoy+23} showed that D$k$S cannot be approximated up to a factor of $2^{\log^\epsilon n}$ for some $\epsilon >0$, assuming a novel conjecture on the hardness of some constraint satisfaction problems.}

Besides the approximability and inapproximability, the parameterized complexity of D$k$S has also been studied
(see e.g., \cite{Cygan2015parameterized} for the foundations of the parameterized complexity).
Cai~\cite{Cai2008parameterized} proved that D$k$S is $\text{W}[1]$-hard with respect to the parameter $k$, meaning that there exists no fixed-parameter tractable algorithm for D$k$S parameterized by $k$, unless $\ensuremath{\mathrm{P}} = \NP$, \rev{which was later strengthened by Komusiewicz and Sorge~\cite{komusiewicz2015algorithmic}}. 
Bourgeois et al.~\cite{Bourgeois2013exact} showed that D$k$S can be solved exactly in $2^\texttt{tw}\cdot n^{O(1)}$ time, where $\texttt{tw}$ is the tree-width of the input graph.
Broersma et al.~\cite{Broersma2013tight} demonstrated that D$k$S can be solved in $k^{O(\texttt{cw})}\cdot n$ time, where $\texttt{cw}$ is the clique-width of the input graph, but it cannot be solved in $2^{o(\texttt{cw}\log k)}\cdot n^{O(1)}$ time, unless ETH fails.
\rev{Komusiewicz and Sorge~\cite{komusiewicz2015algorithmic} developed an algorithm for D$k$S with a time complexity of
$O((4.2(\Delta-1))^{k-1} (\Delta+k)\, k^{2} n)$, where $\Delta$ is the maximum degree of a vertex in the graph.
The algorithm is randomized and can only produce false negatives with probability at most $1/e$.}
Recently, Mizutani and Sullivan~\cite{Mizutani2022parameterized} proved that D$k$S can be solved in any of $f(\texttt{nd})\cdot n^{O(1)}$ time and $O(2^\texttt{cd}\cdot k^2n)$ time, where \texttt{nd} and \texttt{cd} denote, \rev{respectively, two characteristics of the input graph, i.e., the neighborhood diversity~\cite{Lampis12} and the cluster deletion number~\cite{Bocker+13},  while} $f$ is some computable function.
Hanaka~\cite{Hanaka2023computing} independently showed that D$k$S is fixed parameter tractable \rev{using} the neighborhood diversity, \rev{as well as other parameters}.

There are some other exact algorithms for D$k$S,
based on mathematical programming, heuristic search, or graph-theoretic methods.
Billionnet et al.~\cite{Billionnet2009improving} devised a reformulation technique that is applicable to a wide range of quadratic programming problems, including D$k$S.
Malick and Roupin~\cite{Malick2012solving} presented a branch-and-bound method based on SDP relaxations of D$k$S.
Later, Krislock et al.~\cite{Krislock2016computational} improved the above branch-and-bound method using a better bounding procedure.
Komusiewicz and Sommer~\cite{komusiewicz2020fixcon} devised an enumeration-based exact algorithm for a special case of D$k$S, where the subgraph obtained should be connected. 
Gonzales and Migler~\cite{gonzales2019densest} designed an $O(nk^2)$-time exact algorithm for D$k$S on outerplanar graphs.

Kawase and Miyauchi~\cite{Kawase-Miyauchi18} introduced the concept of dense-frontier points of a graph.
Given $G=(V,E)$, plot all the points contained in $\{(|S|,e[S])\mid S\subseteq V\}$.
They referred to the extreme points of the upper convex hull of the above set as the dense frontier points of $G$.
Note that the densest subgraph and the entire graph are typical vertex subsets corresponding to dense frontier points.
\rev{The authors} designed an LP-based algorithm that computes a corresponding vertex subset for every dense frontier point, in polynomial time.
An algorithm designed by Nagano et al.~\cite{Nagano+11} can also be used \rev{for the same purpose}.

Finally, several effective heuristics for D$k$S have been proposed.
Sotirov~\cite{sotirov2020solving} developed coordinate descent heuristics for D$k$S. 
Bombina and Ames~\cite{bombina2020convex} introduced a novel convex programming relaxation for D$k$S using the nuclear norm relaxation of a low-rank and sparse decomposition of the adjacency matrices of subgraphs with $k$ vertices. Using the relaxation, they proved that an optimal solution can be obtained if the input is sampled randomly from a distribution of random graphs constructed to contain a highly dense subgraphs with $k$ vertices with high probability.
Konar and Sidiropoulos~\cite{konar2021exploring} reformulated D$k$S as a submodular function minimization subject to a cardinality constraint, and introduced a relaxation as a Lov\'asz extension minimization over the convex hull of the cardinality constraint. They proposed an effective heuristic for D$k$S by developing a highly scalable algorithm for the relaxation, based on the \rev{Alternating Direction Method of Multipliers} (ADMM). 
\rev{Arrazola et al.~\cite{Arrazola+18} demonstrated that Gaussian boson sampling, one of the limited models of quantum computation, can be used for enhancing randomized algorithms for D$k$S.}


There are some papers dealing with the edge-weighted version of D$k$S, where the edge weights are nonnegative.
Ravi et al.~\cite{Ravi1994heuristic} presented a polynomial-time $4$-approximation algorithm for the problem with edge weights satisfying the triangle inequality.
Later, Hassin et al.~\cite{Hassin1997approximation} proposed an algorithm with a better approximation ratio of $2$.
Recently, Chang et al.~\cite{chang2020hardness} considered a generalized setting, where edge weights only satisfy a relaxed variant of the triangle inequality.
They demonstrated that the problem is \NP-hard for any degree of relaxation of the triangle inequality and extended the above $2$-approximation algorithm to the generalized setting.
Barman~\cite{barman2018approximating} studied a slight variant of D$k$S called the Normalized Densest $k$-Subgraph problem (ND$k$S), where the objective function is normalized to be at most 1, i.e., $e[S]/|S|^2$ is considered. As $|S|$ is fixed to $k$, the inapproximability in terms of the multiplicative approximation is inherited from D$k$S.
Instead, Barman~\cite{barman2018approximating} focused on an additive approximation for ND$k$S, and developed an $\epsilon$-additive approximation algorithm running in $n^{O(\log \Delta / \epsilon^2)}$ time, where $\Delta$ is the maximum degree of $G$.
Barman~\cite{barman2018approximating} also gave an $\epsilon$-additive approximation algorithm for a bipartite graph variant of ND$k$S called the \rev{Densest $k$-Bipartite Subgraph} problem (D$k$BS).
Braverman et al.~\cite{braverman2017eth} studied the inapproximability in terms of the additive approximation of ND$k$S:
\rev{they} ruled out an additive PTAS for ND$k$S under ETH.
Hazan and Krauthgamer~\cite{Hazan2011how} showed that there exits no PTAS for D$k$BS under some computational complexity assumptions.


\subsubsection{Densest at-least-$k$-subgraph problem and densest at-most-$k$-subgraph problem}
There are two relaxations of D$k$S, introduced by Andersen and Chellapilla~\cite{AndersenChellapilla}.
The two problems are called the \rev{Densest at-least-$k$ Subgraph} problem (Dal$k$S) and the \rev{Densest at-most-$k$ Subgraph} problem (Dam$k$S).
As suggested by the names, Dal$k$S and Dam$k$S ask $S\subseteq V$ that maximizes $d(S)$ subject to $|S|\geq k$ and $|S|\leq k$, respectively.
Obviously, similar to D$k$S, Dam$k$S is \NP-hard.
The \NP-hardness of Dal$k$S is not trivial, but it was proved by Khuller and Saha~\cite{Khuller2009Dense} by reducing D$k$S to Dal$k$S.

Andersen and Chellapilla~\cite{AndersenChellapilla} presented a linear-time $3$-approximation algorithm for Dal$k$S
using the greedy peeling algorithm for DSP.
The analysis of the approximation ratio of $3$ is based on the relationship between subgraphs with large density and subgraphs with large minimum degree,
which can be viewed as a generalization of the analysis of $2$-approximation for DSP by Kortsarz and Peleg~\cite{Kortsarz-Peleg94}.
Andersen and Chellapilla~\cite{AndersenChellapilla} also mentioned the hardness of approximation of Dam$k$S:
in particular, they proved that if there exists a polynomial-time $\gamma$-approximation algorithm for Dam$k$S,
then there exists a polynomial-time $8\gamma^2$-approximation algorithm for D$k$S.

Later, Khuller and Saha~\cite{Khuller2009Dense} improved the approximability of Dal$k$S and the hardness of approximation of Dam$k$S.
For Dal$k$S, they designed an LP-based $1/2$-approximation algorithm, 
\rev{which solves a series of $n-k+1$ LPs, constructs $n-k+1$ candidate solutions from the optimal solutions to the LPs, and outputs the best among them.} 
Their analysis of the approximation ratio of $2$ is based on the analysis by Charikar~\cite{Charikar2000}
that DSP can be solved exactly by the LP-based algorithm.
To \rev{decrease the number of LPs to solve from $n-k+1$ to $1$, the authors} suggested incorporating the algorithm by Andersen and Chellapilla into (a lighter version of) the above algorithm.
As for the inapproximability of Dam$k$S, Khuller and Saha~\cite{Khuller2009Dense} proved that
if there exists a polynomial-time $\gamma$-approximation algorithm for Dam$k$S,
then there exists a polynomial-time $4\gamma$-approximation algorithm for D$k$S,
which improves the above result by Andersen and Chellapilla~\cite{AndersenChellapilla} and implies that Dam$k$S is as hard to approximate as D$k$S within a constant factor.
Recently, Zhang and Liu~\cite{zhang2021approximating} developed a randomized bicriteria approximation algorithm for Dam$k$S.


\subsection{Seed set}\label{subsec:seed}
Dai et al.~\cite{dai2022anchored} studied the problem of Anchored Densest Subgraph search (ADS). Given a graph $G=(V,E)$, a reference node set $R$ and an anchored node set $A$, with $A \subseteq R \subseteq V$, the ADS problem consists in finding a set of nodes $S$ that contains all nodes in $A$ \rev{but} maximizes the following quantity:
$(2 e[S] - \sum_{v \in S \setminus R} \deg(v)) / |S|$.
The authors designed an exact algorithm for ADS based on a modified version of the Goldberg~\cite{goldberg1984finding} reduction to
$O\left(\log \sum_{v \in R} \deg(v) \right)$
instances of maximum-flow.
The complexity of the proposed method is bounded by a polynomial of $\sum_{v \in R} \deg(v)$, and is independent of the size of the input graph.

Sozio and Gionis~\cite{Sozio} defined and studied the \emph{cocktail party} problem:
given a graph $G=(V,E)$,
and a set of query nodes $Q \subseteq V$,
the problem consists in finding
a set of nodes $S$ containing all query nodes ($Q\subseteq S$),
 whose induced subgraph
maximizes
a node-monotone non-increasing function
satisfying
a set of monotone non-increasing properties.
The authors considered the minimum degree as the node-monotone non-increasing function to maximize
and a condition on the maximum allowed distance between the solution set $S$ and the query set $Q$ as the monotone non-increasing property to satisfy.
The author proved that a direct adaptation to their problem of the greedy peeling algorithm (Algorithm~\ref{alg:peeling}) always returns an optimal solution.

Fazzone et al.~\cite{Fazzone2022} defined the Dense Subgraphs with Attractors and Repulsers problem (DSAR), whose goal is to find
a dense cluster of nodes $S$, where each node in $S$ is simultaneously close to
a given set $A$ of nodes (Attractors) and far from a given set $R$ of nodes (Repulsers). They proved that DSAR is a special instance of a generalization of DSP on weighted graphs (see Section \ref{subsec:labeled}).


\subsection{Connectivity constraints}
Connectivity constraints on DSP require the output subgraph to satisfy specific requirements in terms of connectivity between the vertices.
These problems are motivated by the fact that densest subgraphs have a structural drawback, that is, they may not be robust to vertex/edge failure, thus not reliable in real-world applications such as network design, transportation, and telecommunications, to name a few.
Indeed, densest subgraphs may not be well-connected, which implies that they may be disconnected by removing only a few vertices/edges within it.
As a toy example, consider a barbell graph consisting of two equally-sized large cliques bridged by only one edge.
In the classical DSP setup, the entire graph would be the densest subgraph, but the failure of the edge connecting the two cliques would disconnect the entire graph.

In this spirit, Bonchi et al.~\cite{bonchi2021finding} introduced two related optimization problems: the densest $k$-vertex-connected subgraph problem and the densest $k$-edge-connected subgraph problem.
In the densest $k$-vertex/edge-connected subgraph problem,
given an undirected graph $G=(V,E)$ and a positive integer $k$, we seek a vertex subset $S\subseteq V$ that maximizes $d(S)$ subject to the constraint that $G[S]$ is $k$-vertex/edge-connected, i.e., the subgraph would be still connected with the removal of any subset of $k$ different vertices/edges.
Bonchi et al.~\cite{bonchi2021finding} first designed polynomial-time $\left(4/\gamma,1/\gamma\right)$-bicriteria approximation algorithms
with parameter $\gamma\in [1,2]$ for these problems.
Note that setting $\gamma=1$, we can obtain $4$-approximation algorithms for the problems.
The algorithms are designed based on a well-known theorem in extremal graph theory, proved by Mader~\cite{Mader72}.
They then designed polynomial-time $19/6$-approximation algorithms for the problems,
which improves the above approximation ratio of $4$ derived directly from the bicriteria approximation ratio.

