\documentclass[10pt]{article}
\definecolor{purple}{rgb}{1, 0, 1}

\newcommand{\ie}{\emph{i.e.,}\xspace}
\newcommand{\eg}{\emph{e.g.,}\xspace}
\newcommand{\abr}{\emph{abbr.}\xspace}
\newcommand{\ea}{\emph{et al.}\xspace}
\newcommand{\gensync}{\emph{GenSync}\xspace}
\newcommand{\colosseum}{\emph{Colosseum}\xspace}
\newcommand{\srep}{\emph{SREP}\xspace} % Set Reconciliation Enhances
\newcommand{\srepsim}{\emph{SREPSim}\xspace}
% Propagation
\newcommand{\esrep}{\emph{E-SREP}\xspace}
\newcommand{\epsrep}{\emph{EP-SREP}\xspace}
\newcommand{\mesrep}{\emph{ME-SREP}\xspace}
\newcommand{\mempoolsync}{\emph{MempoolSync}}

\newcommand{\fref}[1]{Fig.~\ref{#1}}
\newcommand{\tref}[1]{Table~\ref{#1}}
\newcommand{\aref}[1]{Algorithm~\ref{#1}}
\newcommand{\procref}[1]{Procedure~\ref{#1}}
\newcommand{\sref}[1]{Section~\ref{#1}}
\newcommand{\lineref}[1]{line~\ref{#1}}
\newcommand{\appref}[1]{Appendix~\ref{#1}}

% Change \eqref
\LetLtxMacro{\originaleqref}{\eqref}
\renewcommand{\eqref}{Eq.~\originaleqref}

% Theorems and corollaries
\newcounter{theoremcount}
\setcounter{theoremcount}{0}
\DeclareRobustCommand{\theorem}[1]{%
  \refstepcounter{theoremcount}%
  \noindent\textit{\textbf{Theorem \thetheoremcount\label{theorem:#1}: }}%
}
\DeclareRobustCommand{\theoremref}[1]{Theorem~\ref{theorem:#1}}

\DeclareRobustCommand{\proof}{\emph{Proof:}\xspace}
\DeclareRobustCommand{\qqed}{\hfill$\blacksquare$}

\newcounter{corollcount}
\setcounter{corollcount}{0}
\DeclareRobustCommand{\coroll}[1]{%
  \refstepcounter{corollcount}%
  \noindent\textit{\textbf{Corollary \thecorollcount\label{coroll:#1}: }}%
}
\DeclareRobustCommand{\corollref}[1]{Corollary~\ref{coroll:#1}}

\newcounter{lemmacount}
\setcounter{lemmacount}{0}
\DeclareRobustCommand{\lemma}[1]{%
  \refstepcounter{lemmacount}%
  \noindent\textit{\textbf{Lemma \thelemmacount\label{lemma:#1}: }}%
}
\DeclareRobustCommand{\lemmaref}[1]{Lemma~\ref{lemma:#1}}

\newcounter{definitioncount}
\setcounter{definitioncount}{0}
\DeclareRobustCommand{\definition}[1]{%
  \refstepcounter{definitioncount}%
  \noindent\textit{\textbf{Definition \thedefinitioncount\label{definition:#1}: }}%
}
\DeclareRobustCommand{\defref}[1]{Definition~\ref{definition:#1}}

%notes of different authors
\newif\ifnotes
\notestrue
\notesfalse

\newif\ifdiff
\difftrue
\difffalse

\newcommand{\anote}[1]{\ifnotes $\ll$\textsf{\textcolor{purple}{Ari: {#1}}}$\gg$ \fi}
\newcommand{\nnote}[1]{\ifnotes $\ll$\textsf{\textcolor{orange}{Novak: {#1}}}$\gg$ \fi}
\newcommand{\diff}[1]{\ifdiff\textcolor{orange}{#1}\else#1\fi}

%%% Local Variables:
%%% mode: latex
%%% TeX-master: "main"
%%% End:


\begin{document}

\title{A Survey on the Densest Subgraph Problem and Its Variants}

\author[1,3]{Tommaso Lanciano}
\author[2]{Atsushi Miyauchi}
\author[2]{Adriano Fazzone}
\author[2]{Francesco Bonchi\thanks{Contact: \href{mailto:bonchi@centai.eu}{bonchi@centai.eu}}}
\affil[1]{KTH Royal Institute of Technology, Sweden}
\affil[2]{CENTAI Institute, Italy}
\affil[3]{Sapienza University of Rome, Italy}
\date{}
\maketitle \sloppy


\begin{abstract}
The Densest Subgraph Problem requires to find, in a given graph, a subset of vertices whose induced subgraph maximizes a measure of density. The problem has received a great deal of attention in the algorithmic literature since the early 1970s, with many variants proposed and many applications built on top of this basic definition. Recent years have witnessed a revival of research interest \rev{in} this problem with several \rev{important} contributions, including some groundbreaking results, published in 2022 and 2023.
This survey provides a deep overview of the fundamental results and an exhaustive coverage of the many variants proposed in the literature, with a special attention \rev{to} the most recent results. The survey also presents a comprehensive overview of applications and discusses some interesting open problems for this evergreen research topic.
\end{abstract}

\pagebreak

\tableofcontents

\pagebreak



\section{Introduction}
\label{sec:intro}
\section{Introduction}

The increasing complexity of source code poses a key challenge to the reliability of large-scale software systems. Software bugs in these systems can lead to safety issues~\cite{bug_safety} for users around the world as well as cause non-negligible financial losses~\cite{bug_loss}. As such, developers have to spend a large amount of time and effort on bug fixing. Consequently, \aprfull (\apr), designed to automatically generate patches to fix software bugs, has attracted wide attention from both academia and industry~\cite{long2016prophet, legoues2012genprog, long2015spr, lou2020can, tufano2018empstudy}. 


To achieve \apr, one popular approach is known as Generate-and-Validate (G\&V)~\cite{qi2015gv, ghanbari2019prapr, lou2020can, le2016hdrepair, legoues2012genprog, wen2018capgen, hua2018sketchfix, martinez2016astor, koyuncu2020fixminder, liu2019tbar, liu2019avatar}, which is typically based on the following pipeline: First, fault localization techniques~\cite{wong2016fl, abreu2007ochiai, zhang2013injecting, papadakis2015metallaxis, li2019deepfl, li2017transforming} are applied to determine the suspicious locations in programs where bugs are likely to exist. Then, the buggy locations are used by the \apr tools to generate a list of patches that replace buggy lines with correct lines. Afterward, each patch is validated against the original test suite to identify any \emph{plausible patches} (i.e., passing all tests in the test suite). Finally, to determine the \emph{correct patches}, developers examine the list of plausible patches to see if any of them can correctly fix the bug. 

Traditional \apr tools can mainly be categorized into heuristic-based~\cite{legoues2012genprog, le2016hdrepair, wen2018capgen}, constraint-based~\cite{mechtaev2016angelix, le2017s3, demacro2014nopol, long2015spr} and \template~\cite{ghanbari2019prapr, hua2018sketchfix, martinez2016astor, liu2019tbar, liu2019avatar}. Among these traditional tools, \template \apr tools~\cite{ghanbari2019prapr, liu2019tbar, benton2020effectiveness} have been able to achieve state-of-the-art results. \Template \apr tools typically leverage pre-defined templates (e.g., adding a nullness check) for bug fixing. However, since these fix templates are typically handcrafted, the number and types of bugs they are able to fix can be limited. 



To address the limitations of traditional \apr, researchers have proposed various \learning \apr tools~\cite{li2020dlfix, chen2018sequencer, jiang2021cure, lutellier2020coconut, zhu2021recoder, ye2022rewardrepair} based on the \nmtfull (\nmt) architecture~\cite{sutskever2014mt} where the input is the buggy code snippets and the goal is to translate the buggy code snippets into a fixed version. To accomplish this, \learning \apr tools require supervised training datasets with pairs of both buggy and fixed code snippets in order to learn how to perform this translation step. These training data are usually obtained by mining historical bug fixes using heuristics/keywords~\cite{dallmeier2007benchmark}, which can be imprecise for identifying bug-fixing commits; even the actual bug-fixing commits can include irrelevant code changes, leading to further pollution in the dataset~\cite{xia2022alpharepair}.
% 
Moreover, it can be hard for such \apr tools to generalize and fix bug types unseen during training. 



To better leverage recent advances in \plmfull{s} (\plm{s}), researchers~\cite{xia2022alpharepair, xia2023repairstudy, kolak2022patch, prenner2021codexws} have directly applied \plm{s} to generate patches without bug-fixing datasets. These \llm-based \apr tools work by either directly generating a complete code function~\cite{prenner2021codexws, xia2023repairstudy} or predict/infill the correct code snippet given its surrounding context~\cite{xia2022alpharepair, xia2023repairstudy}. By directly using \llm{s} that are pre-trained on billions of open-source code snippets, \llm-based \apr tools can achieve state-of-the-art performance on many repair datasets~\cite{xia2022alpharepair}. 


% 
%
%

Traditional \apr tools have long used the insight of the \emph{plastic surgery hypothesis}~\cite{barr2014plastic} where it states that the code ingredients to fix a bug already exist within the same project. Traditional \apr tools have manually designed pattern-~\cite{ghanbari2019prapr, saha2017elixir} or heuristic-based~\cite{jiang2018simfix, legoues2012genprog} approaches to finding and using such relevant code ingredients to generate fixes for bugs. However, the plastic surgery hypothesis has been largely ignored in \llm-based \apr. In fact, \llm provides a unique opportunity to fully automate the plastic surgery hypothesis idea via fine-tuning (learning project-specific information via model updates from the buggy project) and prompting (directly providing relevant code ingredients to the model), and make it directly applicable to different languages (since the \llm{s} are typically multi-lingual).%
Moreover, despite the intensive manual efforts involved, traditional \apr tools still cannot fully leverage project-specific information due to large search space for leveraging/composing existing code ingredients. In contrast, the project-specific information can effectively leveraged by \llm{s} due to their power in code understanding/vectorization, e.g., even partial/imprecise information may still guide \llm{s} in correct patch generation!
 To this end, we ask the question: \emph{How useful is the plastic surgery hypothesis in the era of \plm{s}}?








\mypara{Our Work.} To answer the question, we present \ourtech{\xspace} -- a \llm-based approach that automatically utilizes the plastic surgery hypothesis by systematically combining multiple fine-tuning and prompting strategies for \apr. \ourtech fine-tunes \plm{s} using two novel domain-specific training strategies: \textbf{\epfinetune} -- we fine-tune using the original buggy project by aggressively masking out a high percentage of tokens, which allows \plm to learn project-specific code tokens and programming styles; and \textbf{\rofinetune} -- which only masks out a single continuous code sequence per training sample, allowing the model to get used to the final \csapr task of predicting a single continuous code sequence. Furthermore, we directly leverage the ability for \plm{s} to understand natural language instructions and introduce a novel prompting strategy, \textbf{\idprompting}, which uses information retrieval and static analysis to obtain a list of relevant identifiers for the buggy lines. While such relevant identifiers are critical for fixing some difficult bugs, they may not be seen by the \llm during inference due to limited context window size. Through the use of prompting, we directly tell the model to use these extracted identifiers (relevant code ingredients) to generate the correct code. Finally, to perform repair, we combine all four model variants (including the base model, both fine-tuned models and the base model with prompting) for the final repair.





While our insight of leveraging the plastic surgery hypothesis for \llm-based \apr is generalizable across different types of \plm{s}, to implement \ourtech, we choose a recent \plm{\xspace}, \ctfive~\cite{wang2021codet5}, which is pre-trained on millions of open-source code snippets. \ctfive is an encoder-decoder model trained using \mspfull (\msp) objective where a percentage of tokens are masked out and each continuous masked token sequence is referred to as a masked span. Also, although we only extract relevant identifiers from the current buggy project (since this paper focuses on the plastic surgery hypothesis), our work can be easily extended to obtain other code information (such as relevant statements or functions) from other sources, such as  the massive pre-training corpora~\cite{husain2020codesearchnet} or historical bug-fixing datasets~\cite{jiang2019infer}, which can provide more coding knowledge for \llm{s}. Besides, although we mainly focus on using traditional string comparison algorithms for information retrieval in this paper, these techniques can be easily replaced by other frequency-based retrieval~\cite{robertson2009probabilistic} and neural search (or embedding-based search)~\cite{reimers2019sentence}.
  In summary, this paper makes the following contributions:


%


\begin{itemize}[noitemsep, leftmargin=*, topsep=0pt]
    \item \textbf{Dimension.} This paper is the first to revisit the important plastic surgery hypothesis in the era of \llm{s}. It opens up a new dimension for \llm-based \apr to incorporate previously neglected information from the buggy project itself to boost \apr performance. Furthermore, it demonstrates the promising future of retrieval-based prompting for modern \llm-based \apr.
    \item \textbf{Implementation.} We implement \ourtech based on the recent \ctfive model. We augment the model using two novel fine-tuning strategies: \epfinetune and \rofinetune, along with a novel prompting strategy based on information retrieval and static analysis: \idprompting. We combine the patches generated by all four models together and perform patch ranking to speed up \apr.% 
    \item \textbf{Evaluation Study.} We conduct an extensive evaluation against state-of-the-art \apr tools. On the widely studied \dfj 1.2 and 2.0 datasets~\cite{just2014dfj}, \ourtech is able to achieve the new state-of-the-art results of 89 and 44 correct bug fixes (15 and 8 more than best baseline) respectively.  Furthermore, we perform a broad ablation study to justify our design. \ourtech demonstrates for the first time that the plastic surgery hypothesis can substantially boost \llm-based \apr and advance state-of-the-art \apr, while being fully automated and general. Moreover, even partial/imprecise code ingredients may still effectively guide \llm{s} for \apr!
\end{itemize}



\section{Fundamental Results}
\label{sec:fundamental}
In this section, we review the fundamental results for DSP,
starting from classical exact and approximation algorithms
and arriving to very recent breakthrough results.

\subsection{Exact algorithms for DSP}
There is a long history of polynomial-time exact algorithms for DSP.
In 1979, Picard and Queyranne~\cite{Picard82} presented the first exact algorithm for DSP in their technical report (officially published in 1982).
The algorithm is designed based on a series of maximum-flow computations.
In 1984, Goldberg~\cite{goldberg1984finding} gave an improved version of the above algorithm in terms of time complexity.
More than 15 years later, Charikar~\cite{Charikar2000} designed an LP-based exact algorithm for DSP.
As maximum-flow computation can be done by solving LP, the existence of an LP-based exact algorithm is trivial.
However, Charikar's \rev{LP-based algorithm} \rev{exploits a different idea}  \rev{and it is of independent interest, because it allows
to see the renowned greedy-peeling approximation algorithm (described later in Section \ref{subsec:approx})
as a primal-dual algorithm for DSP.}


\subsubsection{Goldberg's maximum-flow-based algorithm}
We next review the maximum-flow-based exact algorithm for DSP, designed by Goldberg~\cite{goldberg1984finding}.
Strictly speaking, the algorithm we describe is slightly different from Goldberg's one, but the difference is not essential and it is just for the sake of simplicity of presentation. The algorithm maintains upper and lower bounds on the (unknown) optimal value of the problem,
and tightens the bounds step-by-step using binary search, until the current lower bound is guaranteed to be the optimal value of DSP.
To update upper and lower bounds, the algorithm utilizes maximum-flow computation.
As initial upper and lower bounds, $m/2$ and $0$ can be employed, respectively.
Let $\beta \geq 0$ be the midpoint of the upper and lower bounds kept in the current iteration.
For $G=(V,E)$ and $\beta \geq 0$, the algorithm constructs the following edge-weighted directed graph $(U,A,w_\beta)$: $U=V\cup \{s,t\}$, $A=A_s\cup A_E\cup A_t$, where
$A_s=\{(s,v)\mid v\in V\}$, $A_E=\{(u,v), (v,u)\mid \{u,v\}\in E\}$, and $A_t=\{(v,t)\mid v\in V\}$, \rev{and} $w_\beta \colon A\rightarrow \mathbb{R}_+$ \rev{such that}
\begin{align*}
w_\beta(e) =
\begin{cases}
\frac{\deg(v)}{2} &\text{if } e=(s,v)\in A_s,\\
\frac{1}{2} &\text{if } e\in A_E,\\
\beta &\text{if } e\in A_t. 
\end{cases}
\end{align*}
For $v\in V$, $\deg(v)$ is the degree of $v$ in $G$. 
This graph is illustrated in Figure~\ref{fig:digraph}. 
Here we introduce some terminology and notation.
An $s$--$t$ cut of $(U,A,w_\beta)$ is a partition $(X,Y)$ of $U$ (i.e., $X\cup Y=U$ and $X\cap Y=\emptyset$) such that $s\in X$ and $t\in Y$.
The cost of an $s$--$t$ cut $(X,Y)$, denoted by $\text{cost}(X,Y)$, is defined as the sum of weights of edges going from $X$ to $Y$,
i.e., $\mathrm{cost}(X,Y)=\sum_{(u,v)\in A: u\in X,v\in Y}w_\beta(u,v)$.
An $s$--$t$ cut having the minimum cost is called a minimum $s$--$t$ cut.
The following lemma is useful for updating the upper and lower bounds using a minimum $s$--$t$ cut of $(U,A,w_\beta)$:
\begin{figure}[t]
\centering
\includegraphics[scale=1.15]{./digraph.pdf}
\caption{An edge-weighted directed graph $(U,A,w_\beta)$ constructed from $G$ and $\beta$.}
\label{fig:digraph}
\end{figure}



\begin{lemma}\label{lemma1}
Let $(X,Y)$ be an $s$--$t$ cut of the edge-weighted directed graph $(U,A,w_\beta)$,
and $S=X\setminus \{s\}$.
Then it holds that
$\mathrm{cost}(X,Y)=m+\beta |S|-e[S].$
In particular, when $X=\{s\}$, $\mathrm{cost}(X,Y)=m$ holds.
\end{lemma}
\begin{proof}
Any edge from $X$ to $Y$ is contained in exactly one of the following sets:
$\{(s,v)\in A_s\mid v\in V\setminus S\},\
\{(u,v)\in A_E\mid u\in S,\, v\in V\setminus S\},\
\{(v,t)\in A_t\mid v\in S\}$.
Therefore, the cost of $(X,Y)$ can be evaluated as follows:
\begin{align*}
\text{cost}(X,Y)
&=\sum_{(s,v)\in A_s:\, v\in V\setminus S}w_\beta(s,v)+\sum_{(u,v)\in A_E:\, u\in S,v\in V\setminus S}w_\beta(u,v)
+\sum_{(v,t)\in A_t:\, v\in S}w_\beta(v,t)\\
&= \sum_{v\in V\setminus S} \frac{\text{deg}(v)}{2}+\frac{|\{{\{u,v\}\in E\mid u\in S,\, v\in V\setminus S}\}|}{2}+\beta|S|\\
&=e[V\setminus S]+|\{{\{u,v\}\in E\mid u\in S,\, v\in V\setminus S}\}|+\beta|S|\\
&=m+\beta|S|-e[S].
\end{align*}
\end{proof}
Let $(X,Y)$ be the minimum $s$--$t$ cut of $(U,A,w_\beta)$ computed by the algorithm.
If $X= \{s\}$ does not hold,
then $S=X\setminus \{s\}$ is a vertex subset that satisfies $\beta|S|-e[S]\leq 0$, i.e., $e[S]/|S|\geq \beta$;
hence, the lower bound on the optimal value can be replaced by $\beta$.
On the other hand, if $X=\{s\}$, then we see that there is no $S\subseteq V$ that satisfies $e[S]/|S|> \beta$;
hence, the upper bound can be replaced by $\beta$.
The following lemma is useful for specifying the termination condition of the algorithm.
\begin{lemma}
Let $G=(V,E)$ be an undirected graph.
For any $S_1,S_2\subseteq V$, if $d(S_1)\neq d(S_2)$, then
$|d(S_1)-d(S_2)|\geq \frac{1}{n(n-1)}$.
\end{lemma}
\begin{proof}
Defining $\Delta=\left| d(S_1)-d(S_2)\right|$, we have
$\Delta=\left| \frac{e[S_1]|S_2|-e[S_2]|S_1|}{|S_1||S_2|}\right|$.
If $|S_1|=|S_2|$ holds, then
$\Delta=\left| \frac{e[S_1]-e[S_2]}{|S_1|}\right|$, implying that $\Delta \geq \frac{1}{n}$.
On the other hand, if $|S_1|\neq |S_2|$, then $|S_1||S_2|\leq n(n-1)$; therefore, $\Delta \geq \frac{1}{n(n-1)}$.
\end{proof}
From this lemma, we see that once the difference between the upper and lower bounds becomes less than $\frac{1}{n(n-1)}$,
the current lower bound attains the optimal value of DSP.
The entire procedure is summarized in Algorithm~\ref{alg:flow}.
The following theorem guarantees the solution quality and time complexity.
\begin{algorithm}[t]
\caption{Maximum-flow-based algorithm}\label{alg:flow}
\SetKwInOut{Input}{Input}
\SetKwInOut{Output}{Output}
\Input{\ $G=(V,E)$}
\Output{\ $S\subseteq V$}
$\beta^{(0)}_\text{ub}\leftarrow m/2$, $\beta^{(0)}_\text{lb}\leftarrow 0$, $i\leftarrow 0$\;
\While{$\beta_\mathrm{ub}^{(i)} - \beta_\mathrm{lb}^{(i)}\geq \frac{1}{n(n-1)}$}{
  $\beta^{(i)}\leftarrow \frac{\beta_\text{lb}^{(i)}+\beta_\text{ub}^{(i)}}{2}$\;
  Compute the minimum $s$--$t$ cut $(X^{(i)},Y^{(i)})$ of $(U,A,w_{\beta^{(i)}})$ using maximum-flow computation\;
  \If{$X^{(i)}\neq \{s\}$}{
    $\beta_\text{lb}^{(i+1)}\leftarrow \beta^{(i)}$, $\beta_\text{ub}^{(i+1)}\leftarrow \beta_\text{ub}^{(i)}$\;
  }
  \Else{
    $\beta_\text{lb}^{(i+1)}\leftarrow \beta_\text{lb}^{(i)}$, $\beta_\text{ub}^{(i+1)}\leftarrow \beta^{(i)}$\;
  }
  $i\leftarrow i+1$\;
}
Compute the minimum $s$--$t$ cut $(X,Y)$ of $(U,A,w_{\beta_\text{lb}^{(i)}})$ using maximum-flow computation\;
\Return{$X\setminus \{s\}$}
\end{algorithm}
\begin{theorem}
Algorithm~\ref{alg:flow} returns an optimal solution to DSP in $O(T_\mathrm{Flow}\log n)$ time, where $T_\mathrm{Flow}$ is the time complexity required to compute a minimum $s$--$t$ cut $(X^{(i)},Y^{(i)})$ using maximum-flow computation ($i=0,1,\dots$).
\end{theorem}
\begin{proof}
From the above discussion, we know that the algorithm returns an optimal solution to DSP.
In what follows, we show that the time complexity of the algorithm is given by $O(T_\text{Flow}\log n)$.
The number of iterations $\hat{i}$ of the while-loop is the minimum (nonnegative) integer among $i$'s that satisfy
\begin{align*}
\frac{1}{2^i}(\beta^{(0)}_\text{ub}-\beta^{(0)}_\text{lb})< \frac{1}{n(n-1)}.
\end{align*}
Therefore, we see that
\begin{align*}
\hat{i}=O(\log(mn(n-1)))=O(\log n).
\end{align*}
Obviously, the time complexity of each iteration of the while-loop is dominated by $T_\text{Flow}$.
Thus, we have the theorem.
\end{proof}

For example, employing the maximum-flow computation algorithm by \rev{Cheriyan et} al.~\cite{Cheriyan96},
we can compute a minimum $s$--$t$ cut of $(U,A,w_{\beta^{(i)}})$ in $O(|U|^3/\log|U|)=O(n^3/\log n)$ time.
Therefore, in that case, the time complexity of Algorithm~\ref{alg:flow} becomes $O(n^3)$.

In the context of maximum-flow-based exact algorithms for DSP, we remark that \rev{Gallo et} al.~\cite{gallo1989fast} showed how a single parametric maximum flow computation provides an optimal solution to DSP.
To the best of our knowledge, the fastest algorithm for parametric maximum flow computation is the one given by Hochbaum~\cite{hochbaum2008pseudoflow} and solves DSP in $\bigO\left(mn \log n\right)$ time, which is better than $O(n^3)$ above, when the input graph is sparse, i.e., $m=O(n)$.



\subsubsection{Charikar's LP-based algorithm}
The algorithm first solves an LP and then constructs a vertex subset, using the information of the optimal solution to the LP,
which is guaranteed to be an optimal solution to DSP.
Let us introduce a variable $x_e$ for each $e\in E$ and a variable $y_v$ for each $v\in V$.
The LP used in the algorithm is as follows:
\begin{alignat*}{4}
&\text{maximize} &\quad &\sum_{e\in E}x_e \\
&\text{subject to} &    &x_e\leq y_u,\ x_e\leq y_v  &\quad &(\forall e=\{u,v\}\in E),\\
&                  &    &\sum_{v\in V}y_v=1, \\
&                  &    &x_e\geq 0, \ y_v\geq 0     &      &(\forall e\in E,\, \forall v\in V).
\end{alignat*}
Roughly speaking, the first constraints stipulate that if we take edge $e=\{u,v\}$, we have to take both of the endpoints $u$ and $v$.
The second constraint just standardizes the objective function of DSP.
The following lemma implies that the LP is a continuous relaxation of DSP.
\begin{lemma}\label{lem:OPT_LP}
Let $\mathrm{OPT}_\mathrm{LP}$ be the optimal value of the LP.
For any $S\subseteq V$, it holds that
$\mathrm{OPT}_\mathrm{LP}\geq d(S).$
In particular, letting $S^*\subseteq V$ be an optimal solution to DSP, we have $\mathrm{OPT}_\mathrm{LP}\geq d(S^*)$.
\end{lemma}
\begin{proof}
Take an arbitrary $S\subseteq V$. Construct a solution $(\bm{x},\bm{y})$ of the LP as follows:
\begin{align*}
x_e=
\begin{cases}
\frac{1}{|S|}  &\text{if } e\in E[S],\\
0              &\text{otherwise},
\end{cases}
\quad
y_v=
\begin{cases}
\frac{1}{|S|}  &\text{if } v\in S,\\
0              &\text{otherwise}. 
\end{cases}
\end{align*}
Then it is easy to see that this solution is feasible for the LP.
Moreover, the objective value of $(\bm{x},\bm{y})$ can be evaluated as
\begin{align*}
\sum_{e\in E}x_e=\sum_{e\in E[S]}\frac{1}{|S|}=\frac{e[S]}{|S|}=d(S).
\end{align*}
Therefore, we have $\text{OPT}_\text{LP}\geq d(S)$.
As we took $S\subseteq V$ arbitrarily, this inequality holds even for $S=S^*$.
\end{proof}

The algorithm first solves the LP to obtain an optimal solution $(\bm{x}^*,\bm{y}^*)$.
For $(\bm{x}^*,\bm{y}^*)$ and $r\geq 0$, let $S(r)=\{v\in V\mid y^*_v\geq r\}$.
Also, let $y^*_\text{max} =\max_{v\in V}y^*_v$.
Then the algorithm computes $r^*\in \argmax\{d(S(r))\mid r\in [0,\,y^*_\text{max}]\}$ and just outputs $S(r^*)$.
To find such $r^*$, it suffices to check the value of $d(S(r))$ at $r=y^*_v$ for every $v\in V$,
because $S(r)$ may change only at those points.
The entire procedure is summarized in Algorithm~\ref{alg:LP}.
Clearly, the algorithm runs in polynomial time.
\begin{algorithm}[t]
\caption{LP-based algorithm}\label{alg:LP}
\SetKwInOut{Input}{Input}
\SetKwInOut{Output}{Output}
\Input{\ $G=(V,E)$}
\Output{\ $S\subseteq V$}
Compute an optimal solution $(\bm{x}^*,\bm{y}^*)$ to the LP\;
$r^*\in \argmax\{d(S(r))\mid r\in [0,\,y^*_\text{max}]\}$\;
\Return{$S(r^*)$}
\end{algorithm}
\begin{theorem}\label{thm:LP}
Algorithm~\ref{alg:LP} is a polynomial-time exact algorithm for DSP.
\end{theorem}
\begin{proof}
Let $S^*\subseteq V$ be an optimal solution to DSP.
By Lemma~\ref{lem:OPT_LP}, we know that $\text{OPT}_\text{LP}\geq d(S^*)$.
Noting the choice of parameter $r^*$ in Algorithm~\ref{alg:LP},
it suffices to show that there exits $r\in [0,\,y^*_\text{max}]$ such that
\begin{align*}
d(S(r))=\frac{e[S(r)]}{|S(r)|}\geq \text{OPT}_\text{LP}.
\end{align*}

Suppose for contradiction that for any $r\in [0,\,y^*_\text{max}]$, it holds that
\begin{align*}
d(S(r))=\frac{e[S(r)]}{|S(r)|}< \text{OPT}_\text{LP}.
\end{align*}
Then we have
\begin{align}\label{ineq:dense_suppose_bar}
\int_0^{y^*_\text{max}}e[S(r)]\,\text{d}r<\text{OPT}_\text{LP}\int_0^{y^*_\text{max}}|S(r)|\,\text{d}r.
\end{align}
Define indicator functions $X_e\colon [0,\,y^*_\text{max}]\rightarrow \{0,1\}$ ($e=\{u,v\}\in E$)
and $Y_v\colon [0,\,y^*_\text{max}]\rightarrow \{0,1\}$ ($v\in V$) as follows:
\begin{align*}
X_e(r)=
\begin{cases}
1  &\text{if } r\leq y^*_u \text{ and } r\leq y^*_v,\\
0  &\text{otherwise},
\end{cases}\quad
Y_v(r)=
\begin{cases}
1  &\text{if } r\leq y^*_v,\\
0  &\text{otherwise}.
\end{cases}
\end{align*}
From the optimality of $(\bm{x}^*,\bm{y}^*)$, we have $x^*_e=\min\{y^*_u,\, y^*_v\}$ for every $e=\{u,v\}\in E$,
and therefore
\begin{align}\label{eq:dense_indicator_x}
\int_0^{y^*_\text{max}}e[S(r)]\,\text{d}r&=\int_0^{y^*_\text{max}}\left(\sum_{e\in E}X_e(r)\right)\text{d}r\nonumber \\
&=\sum_{e\in E}\int_0^{y^*_\text{max}} X_e(r)\,\text{d}r
=\sum_{e\in E}\min\{y^*_u,\,y^*_v\}
=\sum_{e\in E}x^*_e
=\text{OPT}_\text{LP}.
\end{align}
Similarly we have
\begin{align}\label{eq:dense_indicator_y}
\int_0^{y^*_\text{max}}|S(r)|\,\text{d}r&=\int_0^{y^*_\text{max}}\left(\sum_{v\in V}Y_v(r)\right)\text{d}r
=\sum_{v\in V}\int_0^{y^*_\text{max}} Y_v(r)\,\text{d}r
=\sum_{v\in V}y^*_v=1.
\end{align}
By Inequality~\eqref{ineq:dense_suppose_bar} and Equalities~\eqref{eq:dense_indicator_x} and \eqref{eq:dense_indicator_y}, we obtain
\begin{align*}
\text{OPT}_\text{LP}=
\int_0^{y^*_\text{max}}e[S(r)]\,\text{d}r
<\text{OPT}_\text{LP}\int_0^{y^*_\text{max}}|S(r)|\,\text{d}r
=\text{OPT}_\text{LP},
\end{align*}
a contradiction. This concludes the proof.
\end{proof}

It is worth noting that Balalau et al.~\cite{balalau2015topkoverlapping} proved that $\{v\in V\mid y^*_v >0\}$ is a densest subgraph, based on a more sophisticated analysis. This simplifies the above rounding procedure and reduces its time complexity from $O(m+n\log n)$ to $O(n)$, while the time complexity of Algorithm~\ref{alg:LP} is dominated by that for solving the LP.


\subsubsection{An exact algorithm based on submodular function minimization}
The objective function of DSP  has a strong connection with submodularity property of set functions.
Based on this observation, an exact algorithm for DSP can be designed using submodular function minimization. Let $V$ be a finite set.
A function $f\colon 2^V\rightarrow \mathbb{R}$ is said to be submodular
if $f(X)+f(Y)\geq f(X\cup Y)+f(X\cap Y)$ holds for any $X,Y\subseteq V$.
There is a well-known equivalent definition of the submodularity:
a function $f$ is submodular
if and only if $f$ satisfies $f(X\cup \{v\})-f(X)\geq f(Y\cup \{v\})-f(Y)$ for any $X\subseteq Y$ and $v\in V\setminus Y$,
which is called the diminishing marginal return property.
A function $f\colon 2^V\rightarrow \mathbb{R}$ is said to be supermodular if $-f$ is submodular.
A function $f\colon 2^V\rightarrow \mathbb{R}$ is said to be modular if $f$ is submodular and supermodular.
In the submodular function minimization problem (without any constraint),
given a finite set $V$ and a submodular function $f\colon 2^V\rightarrow \mathbb{R}$,
we are asked to find $S\subseteq V$ that minimizes $f(S)$.
Note that the function $f$ is given as a value oracle, which returns $f(S)$ for a given $S\subseteq V$.
The submodular function minimization problem can be solved exactly in polynomial time~\cite{fujishige2005submodular}.
Strictly speaking, there exist algorithms that solve the problem, using the polynomial number of calls of the value oracle in terms of the size of $V$. 
Indeed, Gr\"otschel et al.~\cite{groetschel1981ellipsoid} gave the first polynomial-time algorithm based on the ellipsoid method.
Later, Iwata et al.~\cite{iwata2001combinatorial} \rev{and Schrijver~\cite{Schrijver00}} designed a combinatorial, strongly-polynomial-time algorithm.
Today some faster algorithms are known, e.g., \cite{orlin2009faster,lee2015faster}.

Let us consider the following problem:
$\text{minimize}\ \ \beta |S| - e[S] \, \text{subject to}\ \ S\subseteq V,$
where $\beta \geq 0$ is a constant.
As the function $\beta |S|$ is modular and the function $-e[S]$ is submodular~\cite{fujishige2005submodular}, the function $\beta |S| - e[S]$ is submodular.
Therefore, this problem is a special case of the submodular function minimization problem.
Let us recall the $i$th iteration of the while-loop of Algorithm~\ref{alg:flow},
where either the upper bound $\beta_\text{ub}^{(i)}$ or the lower bound $\beta_\text{lb}^{(i)}$ is updated
using a minimum $s$--$t$ cut of the edge-weighted directed graph $(U,A,w_\beta^{(i)})$,
where $\beta^{(i)}=(\beta_\text{ub}^{(i)}+\beta_\text{ub}^{(i)})/2$.
We can see that the update can be done through solving the above problem.
Let $S_\text{out}$ be an exact solution to the problem with $\beta=\beta^{(i)}$.
If $S_\text{out}\neq \emptyset$ holds,
then $S_\text{out}$ is a vertex subset that satisfies $\beta^{(i)}|S_\text{out}|-e[S_\text{out}]\leq 0$, i.e., $e[S_\text{out}]/|S_\text{out}|\geq \beta^{(i)}$;
hence, the lower bound $\beta_\text{lb}^{(i)}$ can be replaced by $\beta^{(i)}$.
On the other hand, if $S_\text{out}=\emptyset$, then for any $S\subseteq V$,
$e[S]/|S|\leq \beta^{(i)}$ holds; hence, $\beta_\text{ub}^{(i)}$ can be replaced by $\beta^{(i)}$.


\subsection{Approximation algorithms for DSP}\label{subsec:approx}
Despite the polynomial-time solvability of DSP,
there exists a wide literature studying faster approximation algorithms for DSP.
The first approximation algorithm was identified by Kortsarz and Peleg~\cite{Kortsarz-Peleg94}, \rev{based on the concept of $k$-core}.
\rev{For $G=(V,E)$ and $k\in \mathbb{Z}_{+}$, the $k$-core is the (unique) maximal subgraph in which every vertex has degree at least $k$ \cite{malliaros2020core}.
The authors proved that the $k$-core with the maximum $k$ is a $2$-approximate solution for DSP (straightforward after proving Theorem~\ref{thm:peeling} about the greedy peeling algorithm, which is given below).}
Charikar~\cite{Charikar2000} then studied a greedy algorithm for DSP, later to be known as the greedy peeling algorithm,
and proved that the algorithm is also a $2$-approximation algorithm but runs in $O(m+n\log n)$ time.
It is now widely known that the algorithm can be implemented to run in linear time, \rev{i.e., $O(m+n)$ time,} for unweighted graphs.
About 20 years later, Boob et al.~\cite{boob2020flowless} designed an iterative version of the greedy peeling algorithm,
inspired by the multiplicative weights update method~\cite{arora2012multiplicative}.
Later, Chekuri et al.~\cite{Chekuri2022supermod} proved that the iterative greedy peeling algorithm converges to an optimal solution to DSP.
\rev{In the rest of this section, we review these fundamental results.}

\subsubsection{Greedy peeling algorithm}

Here we review the $2$-approximation algorithm for DSP, presented by Charikar~\cite{Charikar2000}.
\rev{The algorithm works by repeatedly removing from the graph the vertex with the smallest degree, and saving the remaining vertices as a potential solution. The process continues until only one node remains in the graph. Finally, it outputs the subset of vertices with the highest density among the recorded candidate solutions.}
The pseudocode is given in Algorithm~\ref{alg:peeling}, where for $S\subseteq V$ and $v\in S$, $\deg_S(v)$ denotes the degree of $v$ in $G[S]$.
\begin{algorithm}[t]
\caption{Greedy peeling algorithm}\label{alg:peeling}
\SetKwInOut{Input}{Input}
\SetKwInOut{Output}{Output}
\Input{\ $G=(V,E)$}
\Output{\ $S\subseteq V$}
$S_{n}\leftarrow V$, $i\leftarrow n$\;
\While{$i>1$}{
  $v_\text{min}\in \argmin\{\text{deg}_{S_i}(v)\mid v\in S_i\}$\;
  $S_{i-1}\leftarrow S_i\setminus \{v_\text{min}\}$\;
  $i\leftarrow i-1$\;
}
$S_\mathrm{max} \in \argmax\{d(S)\mid S\in \{S_1,\dots, S_{n}\}\}$\;
\Return{$S_\mathrm{max}$}
\end{algorithm}

\begin{theorem}\label{thm:peeling}
Algorithm~\ref{alg:peeling} is a $2$-approximation algorithm for DSP. Moreover, the algorithm can be implemented to run in linear time.
\end{theorem}
\begin{proof}
Let $S^*\subseteq V$ be an optimal solution to DSP.
From the optimality of $S^*$, for any $v\in S^*$, we have
\begin{align*}
d(S^*)=\frac{e[S^*]}{|S^*|}\geq \frac{e[S^*\setminus \{v\}]}{|S^*|-1}=d(S^*\setminus \{v\}).
\end{align*}
Transforming the above inequality using $e[S^*\setminus \{v\}]=e[S^*]-\deg_{S^*}(v)$, we have that for any $v\in S^*$,
\begin{align}\label{ineq:dense_optimality}
\deg_{S^*}(v)\geq d(S^*).
\end{align}

Let $v^*$ be the vertex that is contained in $S^*$ and removed first by Algorithm~\ref{alg:peeling}.
Let $S'\subseteq V$ be the vertex subset kept just before removing $v^*$ in Algorithm~\ref{alg:peeling}.
Then the density of $S'$ can be evaluated as
\begin{align}\label{ineq:peeling_LB}
d(S')=\frac{\frac{1}{2}\sum_{v\in S'}\deg_{S'}(v)}{|S'|}
\geq \frac{\frac{1}{2}|S'|\deg_{S'}(v^*)}{|S'|}
\geq \frac{1}{2}\deg_{S^*}(v^*)
\geq \frac{1}{2}d(S^*),
\end{align}
where the first inequality follows from the greedy choice of $v^*\in S'$,
the second inequality follows from $S'\supseteq S^*$,
and the last inequality follows from Inequality~\eqref{ineq:dense_optimality}.
Noticing that $S'$ is one of the candidate subsets of the output, we have the approximation ratio of $2$, as desired.

Next we show that the algorithm can be implemented to run in linear time.
For each (possible) degree $d=0,\dots, n-1$, the vertices having degree $d$ are kept in a doubly-linked list.
In the very first iteration, the algorithm scans the lists with increasing order of $d$ until it finds a vertex.
Once a vertex is found, the algorithm removes the vertex from the list, and moves each neighbor of the vertex to the one lower list.
Owing to the structure of doubly-linked lists, this entire operation can be done in $O(\deg(v))$ time.
Moving to the next iteration, it suffices to go back to the one lower list because no vertex can exist in lower levels than that,
implying that in the entire algorithm, the number of moves from one list to another is bounded by $O(n)$.
Therefore, the above operations can be conducted in $O(m+n)$ time.
Note that we do not need to compute the density of a vertex subset from scratch in each iteration.
It suffices to compute $d(V)$ in the very first iteration and then compute the density of a current vertex subset based on the difference:
in each iteration, the numerator decreases by the degree of the removed vertex in the graph at hand and the denominator decreases by 1.
This concludes the proof.
\end{proof}


Here we demonstrate that the $k$-core with the maximum $k$ is also a $2$-approximate solution for DSP. 
This can be shown easily using Inequality~\eqref{ineq:dense_optimality}.
Let $k^*$ be the maximum value among $k$'s such that there exists a nonempty $k$-core. 
Let $S^*_\text{core}$ be (the vertex subset of) the $k^*$-core.
By the definition of $k^*$-core, we have $\min_{v\in S^*_\text{core}}\deg_{S^*_\text{core}}(v)=k^* \geq \deg_{S^*}(v^*)$,
where $v^*$ is that defined in the proof of Theorem~\ref{thm:peeling}.
Then, using Inequality~\eqref{ineq:dense_optimality}, we have
\begin{align*}
d(S^*_\text{core})
\geq \frac{1}{2}k^*
\geq \frac{1}{2}\deg_{S^*}(v^*)
\geq \frac{1}{2}d(S^*),
\end{align*}
as desired. 


In practice, Algorithm~\ref{alg:peeling} rarely outputs a vertex subset that has an objective value of almost half the optimal value,
which means that the algorithm has a better empirical approximation ratio.
However, Gudapati et al.~\cite{gudapati2021greedy} proved that the approximation ratio of $2$ is tight,
that is, there exists no constant $\alpha < 2$ such that the algorithm is an $\alpha$-approximation algorithm.

It should be remarked that Algorithm~\ref{alg:peeling} can be seen as a primal-dual algorithm for DSP, which also gives a proof of $2$-approximation.
Let us consider the dual of the LP (used in Algorithm~\ref{alg:LP}):
\begin{alignat*}{4}
&\text{minimize}   &\quad &t \\
&\text{subject to} &    &t\geq \sum_{e=\{u,v\}\in E} z_{e,v}      &\quad      &(\forall v\in V),\\
&                  &    &z_{e,u} + z_{e,v} \geq 1 &      &(\forall e=\{u,v\}\in E),\\
&                  &    &z_{e,u}, z_{e,v}\geq 0    &      &(\forall e=\{u,v\}\in E).
\end{alignat*}
The above dual LP can be understood as follows:
each edge $e=\{u,v\}$ has cost $1$ and we need to distribute it to the endpoints $u,v$ so as to minimize the maximum cost over vertices.
Recall that in each iteration of Algorithm~\ref{alg:peeling}, we specify the minimum degree vertex $v_\text{min}$ in the graph at hand and then remove it.
If we assign all the costs of the incident edges of $v_\text{min}$ to $v_\text{min}$ when removing it, we can get a feasible solution for the dual LP.
Let $t^*$ be the objective value of the solution (in terms of the dual LP).
Let us focus on the iteration where the assignment of the maximum cost $t^*$ is carried out.
Then the density of the vertex subset kept just before the assignment is lower bounded by $t^*/2$, because all vertices have degree at least $t^*$.
By the LP's (weak) duality theorem, $t^*$ is lower bounded by the optimal value of the primal LP, thus by the optimal value of DSP.
Therefore, we again see that Algorithm~\ref{alg:peeling} is a $2$-approximation algorithm for DSP.





\subsubsection{Iterative greedy peeling algorithm and beyond}\label{subsubsec:iterative_peeling}

Boob et al.~\cite{boob2020flowless} proposed {\sc Greedy++} for DSP,
an algorithm inspired by the multiplicative weights update~\cite{arora2012multiplicative}.
Let $T\in \mathbb{Z}_{+}$.
The algorithm runs Charikar's greedy peeling algorithm for $T$ times
while updating the priority of each vertex using the information of the past iterations.
The pseudocode is given in \refalg{peeling++}.
Note that each iteration of {\sc Greedy++} requires $\bigO(m+n\log n)$ time. 
The authors empirically showed that the algorithm tends to converge to the optimum
and conjectured its convergence to optimality.
\begin{algorithm}[t]
\caption{\textsf{Greedy++}}\label{alg:peeling++}
\SetKwInOut{Input}{Input}
\SetKwInOut{Output}{Output}
\Input{\ $G=(V,E)$, $T\in \mathbb{Z}_{>0}$}
\Output{\ $S\subseteq V$}
\lForEach{$v \in V$} {$\ell_v \leftarrow 0$}
\For{$t =1,2,\dots,T$}{
  $S^t_{n}\leftarrow V$, $i\leftarrow n$\;
  \While{$i> 1$}{
    $v_\text{min}\in \argmin\{\text{deg}_{S^t_i}(v) + \ell_v \mid v\in S^t_i\}$\;
    $\ell_{v_\text{min}} \leftarrow \ell_{v_\text{min}} + \text{deg}_{S^t_i}(v_\text{min})$\;
    $S^t_{i-1}\leftarrow S^t_i\setminus \{v_\text{min}\}$\;
    $i\leftarrow i-1$\;
  }
  $t\leftarrow t+1$\;
}
$S_\mathrm{max} \in \argmax\{d(S)\mid S\in \{S^1_n,\dots, S^1_1,S^2_n,\dots, S^2_1,\dots, S^T_n,\dots, S^T_1\}\}$\;
\Return{$S_\mathrm{max}$}
\end{algorithm}

Chekuri et al.~\cite{Chekuri2022supermod} proved the conjecture of Boob et al.~\cite{boob2020flowless},
showing that {\sc Greedy++} converges to a solution with an approximation ratio arbitrarily close to $1$
and that it naturally extends to a broad class of supermodular functions (i.e., normalized, non-negative, and monotone supermodular functions) in the numerator of the objective function.
More in detail, the authors proved that for any $\epsilon >0$,
{\sc Greedy++} provides a
$(1+\epsilon)$-approximate solution for DSP
after
$\bigO\left(\frac{\Delta \log n}{\text{OPT} \epsilon^2} \right)$ iterations,
where $\Delta$ is the maximum degree of a vertex in the graph and $\text{OPT}$ is the optimal value of DSP. 
\rev{Harb et al.~\cite{harb2023convergence} recently established the convergence of {\sc Greedy++} to the so-called optimal dense decomposition vector.}
Fazzone et al.~\cite{Fazzone2022} modified {\sc Greedy++} to have a quantitative certificate of the solution quality provided by the algorithm at each iteration.
Thanks to this, the authors equipped {\sc Greedy++} with a practical device
that allows termination whenever a solution with a user-specified approximation ratio is found, making the algorithm suited for practical purposes.
Prior to Chekuri et al.~\cite{Chekuri2022supermod},
Boob et al.~\cite{boob2019faster} provided a $(1+\epsilon)$-approximation algorithm for DSP
by reducing DSP to solving $\bigO\left(\log n\right)$ instances of the mixed packing and covering problem.
This algorithm runs in $\tilde\bigO\left(\frac{m \Delta}{\epsilon} \right)$ time\footnote{$\poly\log n$ factors are hidden in the $\tilde\bigO$ notation.}.
Chekuri et al.~\cite{Chekuri2022supermod} designed another $(1+\epsilon)$-approximation algorithm running in
$\tilde\bigO\left(\frac{m}{\epsilon} \right)$ time by approximating maximum flow.
Very recently, Harb et al.~\cite{harb2022faster} proposed a new iterative algorithm,
which provides an $\epsilon$-additive approximate solution for DSP in
$\bigO\left(\frac{\sqrt{m \Delta}}{\epsilon} \right)$ iterations,
where each iteration requires $\bigO\left(m\right)$ time and admits some level of parallelization.
The authors also provided a different peeling technique called fractional peeling, with theoretical guarantees and good empirical performance.





\section{Adding Constraints}
\label{sec:cons}
\section{Conclusions}

In this work, we present a hybrid ANN-SNN fine-tuning scheme. Our
approach is fairly general and can potentially be applied to many
convolutional networks implemented using the soft LIF neurons or the
rectified linear neurons. We take object localization and image
segmentation as testbed applications, and the effectiveness of our
approach is well demonstrated.  Exploring more network applications,
as well as developing new co-training solutions are our next steps.




\section{Changing the Objective Function}
\label{sec:variants}
The objective function of the densest subgraph problem, i.e., the degree density,
has been generalized to various forms for extracting a more sophisticated structure in a graph.
Section~\ref{subsec:numerator} covers variants that generalize the numerator $e[S]$ of the density,
while Section~\ref{subsec:denominator} discusses variants that generalize the denominator $|S|$ of the density.
Section~\ref{subsec:generalization_others} reviews the other generalizations that do not fall in the above categorization.

\subsection{Generalizing the numerator}\label{subsec:numerator}

Tsourakakis~\cite{Tsourakakis15} generalized the notion of density to the $k$-clique density.
For $G=(V,E)$ and $S\subseteq V$, the $k$-clique density for some fixed positive integer $k$ is defined as
\begin{align*}
h_k(S)=c_k(S)/|S|,
\end{align*}
where $c_k(S)$ is the number of $k$-cliques contained in $G[S]$.
Obviously, when $k=2$, it reduces to the original density.
In the $k$-clique densest subgraph problem ($k$-clique DSP),
given an undirected graph $G=(V,E)$, we are asked to find $S\subseteq V$ that maximizes $h_k(S)$.
In particular, when $k=3$, the problem is referred to as the triangle densest subgraph problem (triangle DSP).
Tsourakakis~\cite{Tsourakakis15} proved that unlike many optimization problems for detecting a large near-clique,
the $k$-clique DSP is polynomial-time solvable when $k$ is constant.
Indeed, the author designed a maximum-flow-based exact algorithm and a supermodular-function-maximization-based exact algorithm
for the problem with constant $k$.
The author also demonstrated that a generalization of the greedy peeling algorithm,
which in each iteration removes a vertex participating in the minimum number of $k$-cliques, attains $k$-approximation.
Furthermore, the author presented a MapReduce implementation of the above greedy peeling algorithm to address large-scale graphs.
The results of computational experiments show that even the triangle densest subgraph is much closer to large near-cliques compared with the densest subgraph, as desired.

Later Mitzenmacher et al.~\cite{mitzenmacher2015scalable} conducted a follow-up work.
Their work is motivated by the fact that it is prohibitive to compute an exact or even well-approximate solution
to the $k$-clique DSP for reasonably large $k$ (e.g., $k>3$) on large graphs, due to the expensive cost of counting $k$-cliques.
To overcome this issue, they presented a sampling scheme called the densest subgraph sparsifier,
yielding a randomized algorithm that produces a well-approximate solution to the $k$-clique DSP
while providing significantly reduced time and space complexities.
Specifically, the sampling scheme samples each $k$-clique independently with an appropriate probability,
which can be incorporated as a preprocessing in any exact algorithm for the problem.
In addition to the sampling scheme, they also devised two simpler exact algorithms for the $k$-clique DSP.
Finally, the authors extended the $k$-clique DSP to the bipartite graph setting.
For an undirected bipartite graph $G=(L\cup R, E)$, positive integers $p,q$, and $S\subseteq L\cup R$, they defined the $(p,q)$-biclique density as $b_{p,q}(S)=c_{p,q}(S)/|S|$,
where $c_{p,q}(S)$ is the number of $(p,q)$-cliques contained in $G[S]$.
In the $(p,q)$-biclique densest subgraph problem ($(p,q)$-biclique DSP), given an undirected bipartite graph $G=(L\cup R, E)$,
we seek $S\subseteq L\cup R$ that maximizes $b_{p,q}(S)$.
They showed that all the above results for the $k$-clique densest subgraph problem can be extended to the $(p,q)$-biclique DSP.
Computational experiments demonstrate that the proposed sampling-based algorithms output near-optimal solutions to the problems,
and such solutions tend to be close to near-cliques and near-bicliques as the parameters $k$ and $p,q$, respectively, become large.

Fang et al.~\cite{Fang2019Efficient} devised more efficient exact and approximation algorithms for the $k$-clique DSP.
To this end, they introduced a generalization of the $k$-core called the $(k,\Psi)$-core.
%Note that the value $k$ of the $k$-clique DSP and the value $(k,\Psi)$-core are not
For a positive integer $k$ and an $h$-clique $\Psi$, a $(k,\Psi)$-core is a maximal subgraph
in which every vertex participates in at least $k$ $h$-cliques.
Therefore, if we take a 2-clique (i.e., an edge) as $\Psi$, the concept reduces to the ordinary $k$-core.
Note that the concept of $(k,\Psi)$-core is a special case of $k$-$(r,s)$ nucleus
introduced by Sar\i{}y\"{u}ce et al.~\cite{Sariyuce2015nucleus,Sariyuce2017nucleus}.
Using the concept, their exact algorithm for the $k$-clique DSP runs as follows:
It computes lower and upper bounds on the $k$-clique density value for each $(\ell,\Psi)$-core computed,
and based on those bounds, it derives lower and upper bounds on the optimal value of the problem.
Then the algorithm specifies some $(\ell,\Psi)$-cores that may contain an optimal solution to the problem,
and solve the problem on them.
A useful fact here is that such $(\ell,\Psi)$-cores tend to be much smaller than the original graph,
enabling us to compute an optimal solution in much shorter time in practice.
On the other hand, their approximation algorithm is based on the fact
that the $(\ell,\Psi)$-core with the maximum value of $\ell$ is a good approximation to an optimal solution.
The algorithm computes the structure without conducting core decomposition from scratch.

Recently, Gao et al.~\cite{gao2022colorful} designed a graph reduction technique to accelerate approximation algorithms for the $k$-clique DSP.
To this end, they introduced the novel concept called the colorful $h$-star.
Assume that the vertices of a graph are colored so that any pair of vertices having an edge receives different colors.
Then, for a positive integer $h$, a colorful $h$-star is a star contained in the graph as a (not necessarily induced) subgraph
in which all vertices have different colors.
Note that the colorful $h$-star is a relaxed concept of $h$-clique; indeed, every $h$-clique is a colorful $h$-star.
They showed that unlike $k$-cliques, the colorful $h$-stars can be counted efficiently using a newly devised dynamic programming method, and designed an efficient colorful $h$-star core decomposition algorithm.
Based on this, they designed a graph reduction technique to accelerate any approximation algorithm for the $k$-clique DSP.
Moreover, they showed that the colorful $h$-star core itself can be a good heuristic solution for the $k$-clique DSP.

Konar and Sidiropoulos~\cite{konar2022triangle} studied the triangle densest $k$-subgraph problem (TD$k$S).
The problem is a variant of D$k$S, where given an undirected graph $G=(V,E)$ and a positive integer $k$,
we are asked to find $S\subseteq V$ that maximizes the $3$-clique density $h_3(S)=c_3(S)/|S|$ (or simply $c_3(S)$) subject to $|S|=k$.
They showed that TD$k$S is \NP-hard, and as a counterpart of the authors' algorithm for D$k$S (reviewed in Section~\ref{subsubsec:DkS}),
they presented a heuristic algorithm based on a mirror descent algorithm for a convex relaxation derived by the Lov\'asz extension,
followed by a simple rounding procedure.
The proposed algorithm is shown to be empirically effective in terms of TD$k$S,
and moreover, it sometimes obtains a better solution even in terms of D$k$S than state-of-the-art algorithms for D$k$S.

Bonchi et al. \cite{BonchiKS19} generalized the notion of density by considering the $h$-degree of a node, i.e., the number of other nodes at distance no more than $h$ from the node. Based on this they defined the following problem:

\begin{problem}[Distance-$h$ densest subgraph]\label{prob:densestsubgraph}
Given a graph $G=(V,E)$ and a distance threshold $h \in \mathbb{N}^+$,  find a subset $S^* \subseteq V$ with the maximum average $h$-degree.
$$
 S^*=\underset{S\subseteq V}{\argmax}\frac{\sum_{v\in S} deg^h_{G[S]}(v)}{|S|}
$$
\end{problem}

It is easy to see that for $h = 1$, Problem~\ref{prob:densestsubgraph} corresponds to the traditional DSP. As the focus of their work was mostly on
distance generalized core decomposition, Bonchi et al. \cite{BonchiKS19} showed that, analogously to the $h = 1$ case, the inner-most core of the core decomposition, i.e., the $(k,h)$-core such that there is no non-empty $(j,h)$-core with $j>k$, provides an approximation to the distance-$h$ densest subgraph.


Miyauchi and Kakimura~\cite{Miyauchi-Kakimura18} aims to find a community,
i.e., a dense subgraph that is only sparsely connected to the rest of the graph, based on DSP.
They generalized the density as follows:
\begin{align*}
d_\alpha(S)=\frac{e[S]-\alpha\cdot e[S,\overline{S}]}{|S|}\ \text{($\alpha \in [0,\infty)$)},
\end{align*}
where $\alpha$ is a nonnegative parameter and $e[S,\overline{S}]$ is the cut size of $S$, i.e., the number of edges between $S$ and $V\setminus S$.
That is, this quality function penalizes the connection between $S$ and $V\setminus S$, resulting in a preferential treatment for community structure.
The authors studied the problem of maximizing this quality function,
and designed an LP-based exact algorithm and a maximum-flow-based exact algorithm.
Moreover, they presented a linear-time algorithm with some quality guarantee.
Computational experiments demonstrate that the proposed algorithms are highly effective in finding community structure in a graph.
For example, for the well-known Web graph called Web-Google, their linear-time approximation algorithm finds a vertex subset with more than 99.1\% density and less than 3.1\% cut size,
compared with an approximate densest subgraph obtained by the greedy peeling algorithm.

Recently, Chekuri, Quanrud, and Torres~\cite{Chekuri2022supermod} have introduced the densest supermodular subset problem (DSS),
where given a finite set $V$ and a nonnegative supermodular function $f:2^V\rightarrow \mathbb{R}_+$, we are asked to find $S\subseteq V$ that maximizes $f(S)/|S|$.
As $e[S]$ is a supermodular function over $V$ given $G=(V,E)$, this problem is a generalization of DSP.
For the above generalized problem, they presented a natural generalization of the iterative greedy peeling algorithm for DSP (reviewed in Section~\ref{subsubsec:iterative_peeling}).
Their significant contribution is the proof of the fact that the generalized algorithm outputs a $(1+\epsilon)$-approximate solution for the generalized problem,
after $T=O\left(\frac{\Delta_f \log n}{\lambda^* \epsilon^2}\right)$ iterations,
where $\Delta_f=\max_{v\in V}(f(V)-f(V\setminus \{v\}))$ and $\lambda^*$ is the optimal value of the problem.
This result affirmatively answers the conjecture of Boob et al.~\cite{boob2020flowless} that the iterative peeling algorithm converges to an optimal solution to DSP.
The proof is based on a consideration of an LP that is derived via the Lov\'asz extension of a supermodular function.


\subsection{Generalizing the denominator}\label{subsec:denominator}

Kawase and Miyauchi~\cite{Kawase-Miyauchi18} addressed the size issue of DSP.
The size issue means that when we solve DSP,
it may happen that the obtained subset is too large or too small in comparison with the size desired in the application at hand.
As mentioned in Section~\ref{sec:cons}, there are size-constrained variants of the densest subgraph problem, e.g., D$k$S, Dal$k$S, and Dam$k$S,
which explicitly specify the size range.
Unlike these variants, Kawase and Miyauchi~\cite{Kawase-Miyauchi18} generalized the density without putting any constraint.
Specifically, they introduced the $f$-density of $S\subseteq V$, which defined as $e[S]/f(|S|)$,
where $f\colon \mathbb{Z}_+\rightarrow \mathbb{R}_+$ is a monotonically non-decreasing function.
Note that earlier than this, Yanagisawa and Hara~\cite{yanagisawa2018discounted} introduced an intermediate generalization
called the discounted average degree, i.e., $e[S]/|S|^\alpha$ for $\alpha \in [1,2]$.
In the $f$-densest subgraph problem ($f$-DS), we are asked to find $S\subseteq V$ that maximizes the $f$-density $e[S]/f(|S|)$.
Although the $f$-DS does not explicitly specify the size of vertex subsets, the above size issue can be handled using convex or concave function $f$ appropriately.
Indeed, the authors showed that any optimal solution to $f$-DS with convex/concave function $f$ has a size smaller/larger than or equal to that of a densest subgraph.
Here a function $f:\mathbb{Z}_+\rightarrow \mathbb{R}_+$ is said to be convex (resp. concave) if $f(x)-2f(x+1)+f(x+2)\geq\, (\text{resp.} \leq)\, 0$ holds for any $x\in \mathbb{Z}_+$.
For the $f$-DS with convex $f$, they proved the \NP-hardness with some concrete $f$,
and designed a polynomial-time
$\min\left\{\frac{f(2)/2}{f(S^*)/|S^*|^2},\, \frac{2f(n)/n}{f(|S|^*)-f(|S^*|-1)}\right\}$-approximation algorithm,
where $S^*\subseteq V$ is an optimal solution to the $f$-DS.
The approximation ratio looks a bit complicated but it reduces to a simpler form by considering some concrete $f$, e.g., $f(x)=x^\alpha$ ($\alpha \in [1,2]$) and $f(x)=\lambda x + (1-\lambda)x^2$ ($\lambda \in [0,1]$).
For the $f$-DS with concave $f$, they designed an LP-based exact algorithm and a maximum-flow-based exact algorithm.
In particular, the LP-based exact algorithm computes not only an optimal solution but also vertex subsets corresponding to dense frontier points, as explained in Section~\ref{sec:cons}.
Finally, they presented linear-time $3$-approximation algorithm based on the greedy peeling algorithm.

\subsection{Other variants}\label{subsec:generalization_others}
Tsourakakis et al.~\cite{tsourakakis2013denser} defined the Optimal Quasi-Clique (OQC) problem of finding the subgraph $S$ that maximizes
$$
e[S] - \alpha \binom{|S|}{2}
$$
thus trying to find a subgraph which is denser in terms of the edge density $e[S]/{|S|\choose 2}$, instead of the degree density adopted by DSP.

Feng et al. \cite{feng2021specgreedy} proposed a generalized framework for addressing DSP and related problems (e.g., \cite{hooi2016fraudar, Miyauchi-Kakimura18, tsourakakis2019novel}) by introducing SpecGreedy, an algorithm that uses graph spectral properties and a greedy peeling strategy to solve the generalized problem. Their framework is a generalization of DSP, which accounts for node weights and a second input graph defined on the same set of nodes $V$. Specifically, the objective is to maximize a function that prioritizes edge-density in the first input graph and node weights while discarding nodes that are dense even in the second input graph.

Recently, Veldt, Benson, and Kleinberg~\cite{veldt2021meandensest} generalized the density to the single-parameter family of quality functions.
Specifically, they introduced the $p$-density for $S\subseteq V$, based on the concept of generalized mean (also called power mean or $p$-mean) of real values, as follows:
\begin{align*}
M_p(S)=\left(\frac{1}{|S|}\sum_{v\in S}\deg_S(v)^p\right)^{1/p}\quad \text{($p\in [-\infty,\infty]$)},
\end{align*}
where for $p\in \{-\infty, 0, \infty\}$, $M_p(S)$ is defined as its limit,
i.e., $M_{-\infty}(S)=\lim_{p\rightarrow -\infty}M_p(S)=\min_{v\in S}\deg_S(v)$, $M_{0}(S)=\lim_{p\rightarrow 0}M_p(S)=\prod_{v\in S}\deg_S(v)$,
and $M_{\infty}(S)=\lim_{p\rightarrow \infty}M_p(S)=\max_{v\in S}\deg_S(v)$.
When $p=1$, the $p$-density reduces to the original density.
The generalized mean densest subgraph problem asks for finding $S\subseteq V$ that maximizes $M_p(S)$.
It is worth mentioning that the generalized mean densest subgraph problem deals with the densest subgraph problem and the problem of finding $k$-core with maximum $k$ in a unified manner ($p=1$ and $p=0$, respectively).
They first proved that when $p\geq 1$, the generalized mean densest subgraph problem can be solved exactly in polynomial time by repeatedly solving submodular function minimization.
They then designed a faster $(p+1)^{1/p}$-approximation algorithm based on the greedy peeling algorithm.
They specified a class of graphs for which this approximation ratio is tight, and showed that as $p\rightarrow \infty$, the approximation ratio converges to $1$.
They also proved that for any $p >1$, the greedy peeling algorithm for DSP outputs an arbitrarily bad solution to the problem, on some graph classes.
Computational experiments demonstrate that the proposed algorithm obtains an extremely good approximate solution (even for the original densest subgraph problem),
scales to large graphs, and highlights a range of different meaningful notions of density.

Balalau et al. \cite{balalau2015topkoverlapping} defined the $(k,\alpha)$-Dense Subgraph with Limited Overlap problem ($(k,\alpha)$-DSLO): given an integer $k >
0$ as well as a real number $\alpha \in [0, 1]$, find at most $k$
subgraphs that maximize the total aggregate density,
i.e., the sum of the average degree of each subgraph,
under the constraint that the maximum pairwise
Jaccard coefficient between the set of nodes in the subgraphs be at most $\alpha$. They proved that the problem is \NP-hard even when $\alpha = 0$ (disjoint subgraphs) and showed that the simple heuristic that greedily finds one densest subgraph in the current graph, remove all its vertices and edges, and iterate until $k$
subgraphs are found or the current graph is empty, can produce arbitrarily bad solutions. Balalau et al. thus presented an efficient algorithm for $(k,\alpha)$-DSLO which comes with provable guarantees in some cases of
interest.


\section{Richer Graphs}
\label{sec:graphs}
In this section, we survey the literature on \rev{DSP} for different types of graphs. We discuss the extension of the problem to labeled graphs, negatively weighted graphs, directed graphs, multilayer graphs, temporal graphs, uncertain graphs, \rev{hypergraphs, and general metric spaces}.
\subsection{Edge- and vertex-labeled graphs}\label{subsec:labeled}
An edge-labeled (respectively vertex-labeled) graph is a type of graph where each edge (respectively vertex) is assigned one or more specific labels or attributes. These types of graphs usually represent real-world scenarios where there is additional information associated with edges or vertices, which can be exploited for different tasks. Despite the simplicity of the input, little attention has been devoted to dense subgraph mining in presence of information brought by the labels.

The simplest example of an edge- and/or vertex-labeled graph is a graph where each edge and/or each vertex is associated with a scalar attribute, i.e. a weight. The problem of densest subgraph on weighted graphs was already studied by Goldberg in his seminal 1984 paper~\cite{goldberg1984finding}, and was recently dusted-off by Fazzone et al.~\cite{Fazzone2022}, that dubbed it as Heavy and Dense Subgraph Problem (HDSP).

\begin{problem}[Heavy and Dense Subgraph Problem (HDSP)]
\label{prb:HDSG}
Given an undirected graph $(G, V, E, w_V, w_E)$ without self-loops, where $w_V: V \rightarrow \mathbb{R^+}$ and $w_E: E \rightarrow \mathbb{R^+}$,
find $S^* \subseteq V$ such that
\[
S^* = \argmax_{S \subseteq V} \frac{e(S) + \sum_{s \in S} w_V(s) }{|S|},
\]
where $e(S) = \sum_{e \in E(S)} w_E(e)$.
\end{problem}

The HDSP problem is in $\mathbb{P}$, Goldberg in~\cite{goldberg1984finding} gave an exact polynomial-time algorithm based on a reduction to the $s$-$t$ MaxFlow / $s$-$t$ MinCut problem.
Recently, Fazzone et al.~\cite{Fazzone2022} studied the approximation guarantees of Charikar's greedy peeling algorithm (Section\ref{subsec:approx}) when applied to HDSP, then adapted the recent iterative greedy approach of Chekuri, Quanrud, and Torres~\cite{Chekuri2022supermod} and Boob et al.~\cite{boob2020flowless}, to HDSP problem.


Anagnostopoulos et al.~\cite{anagnostopoulos2020fair} tackled the algorithmic fairness issue for the densest subgraph problem by defining the fair densest subgraph problem (FDSP) as a constrained version of the original problem, taking into account a binary labelling of the graph vertices. % $l : V \rightarrow \{-1, +1\}$, where the labelling corresponds to an attribute associated with the vertex. % problem definition
According to this setting, the goal is to compute a set of vertices
$S \subseteq V$
of maximum density while ensuring that
$S$
contains an equal number of representatives of either label, guaranteeing that the binary-protected attribute is not disparately impacted.% inappx and appx results
 The authors proved that the FDSP problem is \NP-hard,
and also that approximating the densest fair subgraph with a polynomial time algorithm is at least as hard as Dam$k$S, for which no constant approximation algorithms are known (see Section~\ref{sec:size}).
More precisely, any $\alpha$-approximation (with $\alpha \geq 1$) to Dam$k$S is a $2\alpha$-approximation to the fair densest subgraph problem (FDSP). % combinatorial
 In the case in which the underlying graph is itself fair, the authors define a polynomial time $2$-approximation algorithm for the problem and prove the tightness of this approximation factor under the small set expansion hypothesis~\cite{raghavendra2010expansion}.
The authors also defined an algorithm, based on a spectral embedding, able to retrieve an approximate solution with theoretical guarantees in polynomial time for the case where the input graph is an expander that contains an almost-regular dense subgraph (see Section 2.1 in~\cite{anagnostopoulos2020fair} for more details).

Tsourakakis et al.~\cite{tsourakakis2019novel} considered to solve DSP in edge-(multi)labeled networks with exclusion queries: given a network $G = (V,E)$ in which any edge has associated a subset of labels in the label universe $L$, and an input set $l \subseteq L$, find the densest subgraph that contains only edges whose label is contained in $l$. To solve this problem, they resort to an heuristic based on the Densest Subgraph with Negative Weights problem (see \refsec{negative_weights}).

Rozenshtein et al.\cite{rozenshtein2020mining} studied the problem of finding a dense edge-induced subgraph in an edge-labeled graph whose edges are similar to each other based on a given similarity function of the labels. The authors model the problem setting with an objective function that is the sum of a density term and a similarity term, and design a Lagrangian relaxation optimization problem for its maximization, for which they propose an algorithm based on parametric min-cut, that requires $\bigO(|E|^3 \log |E|)$ time and $\bigO(|E|^2)$ space.

\subsection{Negatively weighted graphs}
\label{sec:negative_weights}
\rev{HDSP, reviewed in the previous section}, was defined on positive weights.
Recently some specific applications (e.g., networks derived from correlation matrices) are requiring to extract dense subgraphs from networks whose edges can be negatively weighted. When we allow negative weights, most of the algorithmic results for DSP no longer \rev{hold}. Cadena et al. \cite{cadena2016dense} were the first to analyze \rev{dense structures} in graphs with negative weights on edges. Building on top of \cite{tsourakakis2013denser} they defined the Generalized Optimal Quasi Clique (GOQC) problem as follows: given a \rev{graph} $G=(V,E)$, a weight function $w: E \rightarrow \mathbb{R}$ and a penalty function $\bar{\alpha}: E \rightarrow \mathbb{R}$, the goal is to find a subset of nodes $S$ that maximizes $f_{\bar{\alpha}}(S) = \sum_{\{u,v\} \in E[S]}(w(u,v) - \bar{\alpha}(u,v))$.
The authors showed that the problem is \NP-hard to approximate within a factor of $\bigO(n^{1/2-\epsilon})$, and proposed an algorithm composed by solving SDP \rev{by refining} the local search algorithm by \cite{tsourakakis2013denser}.

Tsourakakis et al. \cite{tsourakakis2019novel} tackled the Densest Subgraph with Negative Weights \rev{problem (DSNW)} defined as follows: given a graph $G=(V,E)$, a weight function $w: E \rightarrow \mathbb{R}$, the goal is to find a subset of nodes $S$ that maximizes $\sum_{\{u,v\} \in E[S]}w(u,v)/|S|$.
The authors proved that the problem is \NP-hard via a reduction based on the fact that the max-cut problem on graphs with possibly negative edges is \NP-hard.
In order to efficiently solve this problem, they analyzed the performance of the greedy peeling \rev{algorithm} for DSP, proving that for this problem it achieves an approximation of $\frac{\rho^*}{2}-\frac{\Delta}{2}$, where $\rho^*$ is the optimal value and $\Delta$ is the \rev{largest absolute value of the negative degrees}.


\subsection{Directed graphs}\label{subsec:directed}

Directed graphs are graphs in which for any edge it is also taken into account the directionality, meaning that any edge is defined by a source vertex $s$ and a target vertex $t$. Therefore, for two vertices $u$ and $v$, there can exist either the edge $(u,v)$ and $(v,u)$. In this context, the classical notion of density of a subgraph become meaningless, since it does not take into account the directionality. The first notion of density for directed graphs was proposed by Kannan and Vinay \cite{Kannan}: given two sets of nodes $S \subseteq V$ and $T \subseteq V$, we have $f(S,T) = \frac{|E(S,T)|}{\sqrt[]{|S|\cdot|T|}}$, where $E(S,T)$ is the set of edges having its source in $S$ and its target vertex in $T$. With this density notion it is given value to those pair of sets of nodes with a massive number of connections from one to the other; the square root at the denominator ensures that the output is not a single edge, that would take maximum value without it.

\begin{problem}[Directed Densest Subgraph Problem (DDS) \cite{Kannan}]\label{prb:dds}
Given a directed network $G=(V,E)$, the goal is to find two set of nodes $S$ and $T$ that maximize $f(S,T) = \frac{|E(S,T)|}{\sqrt[]{|S|\cdot|T|}}$.
\end{problem}

Kannan and Vinay, besides the notion of density, proposed an $\bigO(\log n)$-approximation algorithm for DDS, by relating the objective function to the singular value of the adjacency matrix, and applying Monte Carlo algorithm for its computation. Charikar \cite{Charikar2000} obtained both an exact and a 2-approximation algorithm based on the solution of $\bigO(n^2)$ linear programs for all the possible values of the quantity $|S| \cdot |T|$. The latter resembles the greedy peeling for DSP, and runs in $\bigO(n+m)$, yielding to a global time complexity of $\bigO(n^2(n+m))$. For both algorithms it was observed that with $\bigO(\frac{\log n}{\epsilon})$ execution of linear programs, it is possible to output respectively a $(1+\epsilon)$- and  $(2+\epsilon)$-approximation.

Khuller and Saha \cite{Khuller2009Dense} gave the first max-flow based polynomial time algorithm for solving DDS; likewise for DSP, this algorithm requires $\bigO(\log n)$ executions of max flow instances, or $\bigO(1)$ executions of parametrized maximum flow problem. In the same work they claimed to have obtained a 2-approximation algorithm with time complexity $\bigO(n+m)$, that was recently disproved by Ma et al. \cite{ma2021directed} by a counter-example; in the same work the authors reported an alternative 2-approximation algorithm provided by Saha, whose time complexity is $\bigO(n(n+m))$.

Bahmani et al. \cite{bahmani} developed a $2(1 + \epsilon)$-approximation algorithm, that ends in $\bigO(\log_{1+\epsilon}n)$ passes over the input graph.

Sawlani and Wang \cite{sawlani2020dynamic} reduced DDS to an instance of HDSP, relying on the knowledge of the ratio between sizes of $|S|$ and $|T|$ for the optimal solution. Since this quantity is unknown, they prove that with $O(\log n/\epsilon)$ instances of a $(1 + \frac{\epsilon}{2})$-approximation algorithm for HDSP is guaranteed a $(1+\epsilon)$-approximation solution for DDS. Furthermore, in their work the authors introduced different proposals for densest subgraph algorithms in fully dynamic setting for undirected graphs and vertex-weighted undirected graphs; therefore, their reduction of DDS to HDSP led to the first algorithm for DDS in fully dynamic setting, that maintains a $(1+\epsilon)$-approximation solution with worst-case time $poly(\log n, \epsilon^-1)$ per update.

Chekuri, Quanrud and Torres \cite{Chekuri2022supermod} very recently followed this reduction to adapt their DSG flow-based algorithm for DDS, that output a $(1+\epsilon)$-approximation in $\tilde \bigO(\frac{m}{\epsilon^2})$.

Ma et al. \cite{ma2021directed} obtained different results in this context. Introducing the notion of $[x,y]$-core, the directed counterpart of the $k$-core notion for undirected graphs, they prove that the directed densest subgraph can be located through it with theoretical guarantees.
As a first direct consequence, these results enable the maximum-flow based exact algorithm to be executed only on a reduced version of the graph, composed by some $[x,y]$-cores, making it faster. Furthermore, they provide a divide-and-conquer strategy to carefully select the optimal value of $\frac{|S|}{|T|}$, that reduces the possible values from $n^2$ to $k$, with n$k << n^2$.
Furthermore, they prove that a particular instance of $[x,y]$-core is a 2-approximation of the optimal solution, thus propose an algorithm to find it that runs in $\bigO(\sqrt{m}(n+m))$.
They introduce the weighted DDS problem, and extend the aforementioned findings for this problem.
Finally, they design algorithms able to handle the edge insertion/deletion in a dynamic setting, both maintaining a 2-approximation solution in $\bigO(\sqrt{m}(n+m))$ for the worst-case, although they practically showed that their algorithms perform faster than the iterative execution of their 2-approximation algorithm for the static setting.

 Ma et al. \cite{ma2022convex} designed a new LP formulation for DDS, with a Frank-Wolfe based algorithm to optimize the associate dual. With this new LP, they were able to design a new algorithmic framework to reduce the number of LP instances to solve. More precisely, they provided a $(1+\epsilon)$-approximation to the optimal solution with $\bigO(\log_{1+\epsilon}n \cdot t_{FW})$, where $t_{fW}$ is the time-complexity for solving a single instance of their LP.


\subsection{Multilayer networks}\label{subsec:multilayer}

Multilayer networks are a generalization of the ordinary (i.e., single-layer) graphs.
For positive integer $\ell$, let $[\ell]=\{1,2,\dots, \ell\}$.
Mathematically, a multilayer network is defined as a tuple $(V,(E_i)_{i\in [\ell]})$,
where $V$ is the set of vertices and each $E_i$ ($i=1,2,\dots, \ell$) is a set of edges on $V$.
That is, a multilayer network has a number of edge sets (called layers),
which may encode different types of connections and/or time-dependent connections over the same set of vertices.
Similarly to other graph mining primitives, dense subgraph discovery, particularly DSP, has been extended to multilayer networks.
As the density value of $S\subseteq V$ varies layer by layer,
there would be several ways to define the objective function of multilayer-network counterparts of DSP.
For $S\subseteq V$ and $i\in [\ell]$, we denote by $d_i(S)$ the density of $S$ in terms of the layer $i$.

Jethava and Beerenwinkel~\cite{jethava2015relational} introduced the first optimization problem for dense subgraph discovery in multilayer networks,
which they referred to as the densest common subgraph problem.
In the problem, given a multilayer network $G=(V,(E_i)_{i\in [\ell]})$, we seek a vertex subset $S\subseteq V$ that maximizes
the minimum density over layers, i.e., $\min_{i\in [\ell]} d_i(S)$.
For the problem, Jethava and Beerenwinkel~\cite{jethava2015relational} devised an LP-based polynomial-time heuristic and a $2\ell$-approximation algorithm based on the greedy peeling.
Reinthal et al.~\cite{reinthal2016finding} studied which algorithm, the simplex method or the interior-point method,
is more suitable for the use in the above LP-based heuristic, and observed that employing the interior-point method can shorten the computation time in practice.
Later, Charikar, Naamad, and Wu~\cite{charikar2018common} designed two combinatorial polynomial-time algorithms with approximation ratios
$O(\sqrt{n\log \ell})$ and $O(n^{2/3})$ (irrespective of $\ell$), respectively.
Moreover, they showed some strong inapproximability results for the problem, based on some computational complexity assumptions.
Specifically, they showed that the densest common subgraph problem is at least as hard to approximate as \textsc{MinRep}, a well-studied minimization version of \textsc{Label Cover}, which implies that the problem cannot be approximated to within a factor of $2^{\log^{1-\epsilon}n}$, unless $\NP \subseteq \text{DTIME}(n^{\textsf{polylog}(n)})$.
They also showed that if the planted dense subgraph conjecture is true, the problem cannot be approximated to within a factor of $n^{1/4-\epsilon}$ and even for $\ell=2$, the problem cannot be approximated to within $n^{1/8-\epsilon}$.

Later, Galimberti et al.~\cite{galimberti2017core,galimberti2020core} introduced a generalization of the densest common subgraph problem,
which they refer to as the multilayer densest subgraph problem.
This problem exploits a trade-off between the minimum density value over layers and the number of layers having such a density value.
Specifically, in the problem, given a multilayer network $G_i=(V,(E_i)_{i\in [\ell]})$ and $\beta \geq 0$,
we are asked to find $S\subseteq V$ that maximizes $\max_{I\subseteq [\ell]}\min_{i\in I}\frac{|E_i[S]|}{|S|}|I|^\beta$.
For this problem, they proposed an exponential-time $O(2\ell^\beta)$-approximation algorithm using a core decomposition technique for multilayer networks.


Recently, Hashemi, Behrouz, and Lakshmanan~\cite{hashemi2022firmcore} designed a sophisticated core decomposition algorithm for multilayer networks,
which they call the FirmCore decomposition algorithm.
For $k\in \mathbb{Z}_+$ and $\lambda\in [\ell]$, a subgraph $H=(S,(E_i[S])_{i\in [\ell]})$ is called a $(k,\lambda)$-FirmCore
if it is a maximal subgraph in which every vertex has degree no less than $k$ in the subgraph for at least $\lambda$ layers.
They devised a polynomial-time algorithm for finding the set of $(k,\lambda)$-FirmCores for all possible $k$ and $\lambda$.
They proved that the FirmCore decomposition unfolds an approximate solution to the multilayer densest subgraph problem,
with a better approximation ratio than that obtained by Galimberti, Bonchi, and Gullo~\cite{galimberti2017core} for many instances.

Semertzidis et al.~\cite{semertzidis2019finding} introduced another generalization of the densest common subgraph problem, called the Best Friends Forever (BFF) problem,
in the context of evolving graphs with a number of snapshots.
The BFF problem is a series of optimization problems that maximize an \emph{aggregate density} over snapshots,
where the aggregate density is set to be the average/minimum value of the average/minimum degree of vertices over layers.
Similarly to the multilayer densest subgraph problem,
they also considered the variant called the On--Off BFF ($\text{O}^2$BFF) problem, which only asks the output to be dense for a part of snapshots.
They investigated the computational complexity of the problems and designed some approximation or heuristic algorithms.

Very recently, Kawase, Miyauchi, and Sumita~\cite{kawase2023stochastic} studied stochastic solutions to dense subgraph discovery in multilayer networks.
Their novel optimization problem asks to find a stochastic solution, i.e., a probability distribution over the family of vertex subsets, rather than a single vertex subset,
whereas it can also be used for obtaining a single vertex subset.
The quality of stochastic solutions is measured using the expectation of the following three metrics, the density, the robust ratio, and the regret,
on the layer selected by the adversary.
Therefore, their optimization problem can be seen as (a generalization of) the stochastic version of the densest common subgraph problem by Jethava and Beerenwinkel~\cite{jethava2015relational}.
Unlike the densest common subgraph problem, their optimization problem can be solved exactly in polynomial time;
indeed, they designed an LP-based polynomial-time exact algorithm.
They proved that the output of the proposed algorithm has a useful structure;
the family of vertex subsets with positive probabilities has a hierarchical structure.
This leads to several practical benefits, e.g., the largest size subset contains all the other subsets and the optimal solution obtained by the algorithm has support size at most $n$.
Moreover, they also demonstrated that the support size of the output can be upper bounded by $\ell$.
For the practical use of the above exact algorithm, they then devised a simple, scalable preprocessing algorithm,
which often reduces the size of the input networks significantly and results in a substantial speed-up.

Finally, we take a look at a very special case of multilayer networks called dual networks, i.e., the case of $\ell=2$ in multilayer networks.
Wu et al.~\cite{wu2015dual,Wu+16} introduced an optimization problem of detecting a dense and connected subgraph in dual networks:
Given a dual network $G=(V,(E_1,E_2))$, we are asked to find a vertex subset $S\subseteq V$ that maximizes $d_1(S)$, i.e., the density on the first layer,
under the constraint that $(S,E_2[S])$ is connected.
They proved that the problem is \NP-hard and designed a scalable heuristic.
Later, Chen et al.~\cite{Chen+22} considered a variant of the above problem,
where $k\in \mathbb{Z}_+$ is given as an additional input, and we seek $S\subseteq V$ that maximizes the minimum degree of vertices on the first layer,
under the constraint that $(S,E_2[S])$ is $k$-edge-connected.
Therefore, this problem enables us to control the strength of connectivity on the second layer.
Owing to the use of the minimum degree, this problem can be solved exactly in polynomial time, unlike the above problem by Wu et al.~\cite{wu2015dual,Wu+16}.

A recent line of research in dual networks is that of contrast subgraphs, i.e. subgraphs that maximize the difference in density between the respective induced subgraphs in the first and second network.
Yang et al. \cite{yang2018mining} proposed the density contrast subgraph problem: given two networks, find the subset of nodes that maximizes the difference in terms of average degree between the 2 networks. They reduce the problem to an instance of DSP in a graph with either positive or negative edge weights, proving that the problem is \NP-hard and cannot be approximated within $O(n^{1-\epsilon})$ for any $\epsilon > 0$. They solve the problem via a variant of the classical greedy peeling algorithm, providing a $O(n)$-approximation guarantee for the final solution.
Lanciano, Bonchi and Gionis \cite{lanciano2020contrast} proposed a variant of this problem, by maximizing the difference in terms of number of edges, subject to an input penalty term that controls the output of the solution. They showed that this problem can be mapped to an instance of the Generalized Optimal Quasi Clique problem defined in \cite{cadena2016dense}, then providing an updated version of their algorithm.
Finally, Feng et al. \cite{feng2021specgreedy} included in their generalized framework for DSP the possibility to include in the denominator the density of a second graph, that makes the solution less attractive if it is dense in both the input graph.




\begin{table}[h]
     \tablestyle{2pt}{1.05}
    
    \centering
    %\resizebox{1\columnwidth}{!}{
    \begin{tabular}{@{}l|ccccc}
    	\toprule
    	%\multicolumn{4}{c}{} &\multicolumn{2}{c}{MiningYoutube}  \\ 
    	%\cmidrule(lr){5-6} 
    	Method   & Backbone &Data & Super. & IoU & IoD \\ 
    	\midrule
    	Mining: MLP \cite{miech2020end} & TSM & MiningYT & Weak & 9.80 & 19.20    \\
             CoMMA* \cite{tan2021look} & S3D-word2vec & HT250K & Self & 2.05 & 5.63    \\
    	MIL-NCE \cite{miech2020end} & S3D-word2vec & HT100M & Self & 18.69 & 26.74    \\
    	Ours                       & S3D-word2vec & HT200K & Self  & 19.18 & 27.65   \\
    	%Ours                       & VAT& S3D-g  & 19.40 & 28.48   \\
            Ours                       & CLIP & HT200K & Self &  \textbf{19.88} & \textbf{28.50}   \\
             %\midrule
            % MCN \cite{chen2021multimodal}      &VAT& R152+RX101   & 23.10 & 32.04    \\
    	\bottomrule
    \end{tabular}
    \vspace{-0.3cm}
    \caption{\textbf{Temporal Grounding on MiningYoutube.} %Spatial-focused model CoMMA is not trained for temporal detection, which results in lower performance, while the proposed model combines global and local representation resulting in better temporal localization than one alone. %\bc{we should include setting without knowing the order}
    %\vspace{-0.5cm}
    \label{tab:temporal}
%    \vspace{-0.4cm}
    }
    %}
\end{table}
\subsection{Uncertain graphs}

Zou~\cite{zou2013polynomial} studied the densest subgraph problem on uncertain graphs.
An uncertain graph is a pair of $G=(V,E)$ and $p\colon E\rightarrow [0,1]$,
where $e\in E$ is present with probability $p(e)$ whereas $e\in E$ is absent with probability $1-p(e)$ \cite{PotamiasBGK10,BonchiGKV14}.
In the problem, given an uncertain graph $G=(V,E)$ with $p$, we seek $S\subseteq V$ that maximizes the expected density.
Zou~\cite{zou2013polynomial} showed that this problem can be reduced to DSP on edge-weighted graphs,
and designed a polynomial-time exact algorithm based on the reduction.

Miyauchi and Takeda~\cite{miyauchi2018robust} considered the uncertainty of edge weights rather than the existence of edges.
To model that, they assumed that there is an edge-weight space $I=\times_{e\in E}[l_e,r_e]\subseteq \times_{e\in E}[0,\infty)$
that contains the unknown true edge weight $w$.
To evaluate the performance of $S\subseteq V$ without any concrete edge weight,
they employed a well-known measure in the field of robust optimization, called the robust ratio.
In their scenario, the robust ratio of $S\subseteq V$ under $I$ is defined as the multiplicative gap between the density of $S$ in terms of edge weight $w'$ and the density of $S^*_{w'}$ in terms of edge weight $w'$ under the worst-case edge weight $w'\in I$, where $S^*_{w'}$ is a densest subgraph on $G$ with $w'$.
Intuitively, $S\subseteq V$ with a large robust ratio has a density close to the optimal value even on $G$ with the edge weight selected adversarially from $I$.
Using the robust ratio, they introduced the robust densest subgraph problem:
Given an undirected graph $G=(V,E)$ and an edge-weight space $I=\times_{e\in E}[l_e,r_e]\subseteq \times_{e\in E}[0,\infty)$,
we are asked to find $S\subseteq V$ that maximizes the robust ratio under $I$.
They designed an algorithm that returns $S\subseteq V$ with a robust ratio at least $\frac{1}{\max_{e\in E}\frac{r_e}{l_e}}$
under some mild condition.
Moreover, they proved that the lower bound on the robust ratio achieved by the above algorithm is the best possible except for the constant factor.
In addition, they also introduced the robust densest subgraph problem with sampling oracle,
where we have access to an oracle that accepts $e\in E$ and outputs a value drawn from a distribution on $[l_e, r_e]$
in which the expected value is equal to the unknown true edge weight.
For this problem, they designed a pseudo-polynomial-time algorithm with a strong quality guarantee.

Tsourakakis et al.~\cite{tsourakakis2019novel} introduced an optimization problem
called the risk-averse dense subgraph discovery problem.
The problem adopts a more general form of uncertain graphs introduced by Tsourakakis et al.~\cite{tsourakakis2018risk}.
An uncertain graph here is defined as a pair of $G=(V,E)$ and $(g_e(\theta_e))_{e\in E}$,
where the weight $w(e)$ of each edge $e\in E$ is drawn independently from the rest
according to some probability distribution $g_e$ with parameter $\theta_e$.
Each probability distribution $g_e$ is assumed to have a finite mean $\mu_e$ and a finite variance $\theta^2_e$.
Roughly speaking, their risk-averse variant aims to find $S\subseteq V$
that maximizes $\frac{\sum_{e\in E[S]}\mu_e}{|S|}$ but minimizes $\frac{\sum_{e\in E[S]}\sigma^2_e}{|S|}$.
Specifically, in the risk-averse dense subgraph discovery problem,
given an uncertain graph $G=(V,E)$ and $(g_e(\theta_e))_{e\in E}$,
we are asked to find $S\subseteq V$ that maximizes
$f(S)=\frac{\sum_{e\in E[S]}\mu_e+\lambda_1|S|}{\sum_{e\in E[S]}\sigma^2_e+\lambda_2|S|}$,
where $\lambda_1, \lambda_2 \geq 0$ are positive parameters, controlling the size of outputs.
Tsourakakis et al.~\cite{tsourakakis2019novel} showed that this problem reduces to DSP on negatively-weighted graphs,
and designed an efficient approximation algorithm based on the reduction.

Recently, Kuroki et al.~\cite{Kuroki+20} pointed out that
the sampling procedure used in Miyauchi and Takeda~\cite{miyauchi2018robust},
where all edges are repeatedly queried by a sampling oracle that returns an individual edge weight,
is often quite costly or sometimes impossible.
To overcome this issue, they introduced a novel framework called the densest subgraph bandits (DS bandits),
by incorporating the concept of stochastic combinatorial bandits~\cite{chen2013combinatorial,chen2014combinatorial} into DSP.
In DS bandits, a learner is given an undirected graph $G=(V,E)$,
whose edge-weights are associated with unknown probability distributions.
During the exploration period, the learner chooses a subset of edges (rather than only single edge, unlike Miyauchi and Takeda~\cite{miyauchi2018robust}) to sample,
and observes the sum of noisy edge weights in a queried subset.
They investigate DS bandits with the objective of best arm identification;
that is, the learner must report one subgraph that (s)he believes to be optimal after the exploration period.
Their first algorithm is designed based on the technique of linear bandits,
for which they presented an upper bound on the number of samples required to identify a $(1+\epsilon)$-approximate solution
with probability at least $1-\delta$ for $\epsilon>0$ and $\delta \in (0,1)$.
Their second algorithm is scalable and parameter free,
designed by combining the successive reject strategy~\cite{audibert2010best} for the multi-armed bandits
and the greedy peeling algorithm~\cite{Charikar2000} for DSP.
They proved that an upper bound on the probability that DS-SR outputs a solution whose density is less than half of the optimal value.

Sun et al. \cite{sun2021efficient} applied their probabilistic truss indexing framework to perform for the first time the triangle densest subgraph detection in uncertain graphs.

The most recent contribution in this context is due to Saha et al. \cite{Saha2022most}, that defined DSP problem in uncertain graphs with an alternative notion of density, called Densest Subgraph Probability (DSPr).
DSPr of a node set $U \subseteq V$ is the summation of the probabilities of all instances in which $U$ represent the densest subgraph (with any density's notion).
They design the Most Probable Densest Subgraph (MPDS) problem considering edge density, clique density and pattern density, and for each of them, their top-$k$ variants.
Their principal contribution is the edge density-based MPDS algorithm, that is built on independent sampling of possible worlds (e.g., via Monte-Carlo sampling) and, in each of them, efficient enumeration of all edge-densest subgraphs.



\subsection{Hypergraphs}\label{subsec:hyper}
Hypergraphs are a generalization of graphs, consisting of a set $V$ of vertices and a set $E$ of \rev{hyperedges}, that are composed by an arbitrary number of vertices. Huang and Kahng \cite{Huang-Kahng95} formally introduced the Densest Subhypergraph problem (DSH), that given \rev{a} hypergraph $G=(V,E)$, requires to find $S\subseteq V$ that maximizes \rev{$d(S) = e[S]/|S|$, where $e[S]=|\{e\in E\mid e\subseteq S\}|$}, and proposed a \rev{polynomial-time} flow-based \rev{exact} algorithm.


\rev{
Very recently, Huang et al.~\cite{Huang+23} generalized DSH to the model with a seed set, 
inspired by the aforementioned ADS (see Section~\ref{subsec:seed}). 
Their model, called the Anchored Densest Subhypergraph (ADSH), is a common generalization of DSH and ADS, 
while the anchored node set is limited to the emptyset. 
Therefore, the model seeks a densest subhypergraph that is close to the given reference node set. 
Its noteworthy feature is the consideration of the so-called locality parameter, 
specifying how the output should be close to the given reference node set. 
The authors presented a flow-based exact algorithm for ADSH, 
based on that for DSP and the recent development of computing a minimum $s$--$t$ cut on hypergraphs~\cite{Veldt+22}. 
This result remains valid even for some generalizations of ADSH, including HDSP (see Section~\ref{subsec:labeled}), 
on hypergraphs with positive and even negative node weights. 
The main algorithmic contribution is to develop a strongly-local exact algorithm for ADSH, 
i.e., an algorithm outputting an optimal solution in time depending only on the parameters of the reference node set. 
The algorithm runs in an iterative fashion: in the first iteration, the algorithm deals with the subhypergraph consisting of the reference nodes and their neighbors, and the later iterations, it computes the minimum $s$--$t$ cut in a hypergraph constructed from the current subhypergraph and expands it using the information of the cut computed. 
}

\rev{Miyauchi et al. \cite{Miyauchi+15} \rev{studied} two general optimization models, in the context of advertising budget allocation, i.e., 
the maximum general-thresholds coverage problem, in which the densest $k$-subhypergraph 
(i.e., the variant of DSH in which the output size is constrained to be equal to $k$) falls, 
and its cost-effective counterpart, in which DSH itself falls. 
For the first model they proposed two different greedy algorithms, while for the second problem they designed a scalable approximation algorithm.
}

Hu et al. \cite{hu2017dynamicsub} \rev{generalized the results by Tsourakakis~\cite{Tsourakakis15}} to hypergraphs, 
by designing \rev{LP-based and flow-based algorithms}. 
Furthermore, they addressed the densest subhypergraph maintenance in \rev{a} dynamic setting, 
by providing two algorithms that maintain a $r(1+\epsilon)$-approximation in the case where there are only edge insertions, 
and a $r^2(1+\epsilon)$-approximation in \rev{a} fully dynamic setting, where $r$ is the rank of the hypergraph, i.e., $r=\max_{e\in E} |e|$. 
Chlamtac et al. \cite{Chlamtac+18} performed a theoretical analysis on \rev{the} densest $k$-subhypergraph problem and provided bounds over 3-uniform hypergraphs; 
Corinzia et al.~\cite{corinzia2022statistical} considered recovering the planted densest $k$-subhypergraph in a $d$-uniform hypergraph (i.e., \rev{a} hypergraph in which for any $e \in E$, $|e| = d$ holds) \rev{and provided} tight statistical bounds on recovering quality and algorithmic bounds based on approximate message passing algorithms.

\rev{The \rev{recent results} by Chekuri et al.~\cite{Chekuri2022supermod} also have generalization to hypergraphs, e.g., 
a fast $(1+\epsilon)$-approximation algorithm based on max flow.
Furthermore, a natural generalization of {\sc Greedy++} (in Section~\ref{subsubsec:iterative_peeling}) to hypergraphs provides a
$(1+\epsilon)$-approximation for DSH.}
The iterative algorithm proposed by Harb et al.~\cite{harb2022faster} \rev{also has a generalization to hypergraphs}; it provides an $\epsilon$-additive approximate solution for DSH in
$\bigO(\sqrt{pr\Delta} / \epsilon)$ iterations,
where each iteration requires $O(p\log r)$ time and admits some level of parallelization. Here $\Delta=\max_{v \in V} |\{e\in E : v \in e\}|$ and $p = \sum_{e \in E} |e|$.

Zhou et al. \cite{zhou2022extracting} \rev{introduced a generalization of DSH by considering the fact that} there might be \rev{some} hyperedges only partially included in the solution, \rev{which} were not counted in the objective function of DSH but might be relevant in \rev{some} specific applications. Therefore, they \rev{defined} a weighting scheme according to the number of vertices of any hyperedge included in the solution, and \rev{a} maximization problem based on it. For this problem they \rev{proposed} exact and $r$-approximation algorithms, where $r$ is the rank of the hypergraph.

Finally, Bera et al. \cite{bera2022dynamicsub} very recently designed a new algorithm for the dynamic setting. They improved the approximation ratio to $(1+\epsilon)$, making it independent \rev{of} the hypergraph rank, with a similar update time to that required in \cite{hu2017dynamicsub}. 



%\input{sec6/private_graphs}
\subsection{Metric spaces}\label{subsec:metricscpaces}
A metric space is a pair $(X, d)$, where $X$ is a set and $d: X \times X \rightarrow \mathbb{R}_+$ is a function called a metric, that assigns a non-negative real number, denoted by $d(x,y)$, to any two elements $x, y \in X$, satisfying the following properties:
$d(x, y) = 0 \text{ if and only if } x = y$ (identity), $d(x, y) = d(y, x)$ (symmetry), and $d(x, y) + d(y, z) \geq d(x, z)$ (triangle inequality). We can define a weighted graph in a metric space, where the vertices are the points of the space, and each pair of point is connected by an edge, weighted according to the distance function $d$. As the resulting graph would be complete, sparsification might be needed.
Esfandiari and Mitzenmacher \cite{esfandiari2018metric} designed a novel sampling approach for the edges of a graph defined in a metric space, using a sublinear number of queries, succeeding with probability at least $1-\bigO(1/n)$.
\rev{This} result implies that for DSP in $\lambda$-metric graphs\footnote{In a $\lambda$-metric graph the triangle inequality that holds is $d(x, y) + d(y, z) \geq \lambda d(x, z)$.}, they can provide a $(2+\epsilon)$-approximation algorithm requiring $\tilde{\bigO}(n / (\lambda \epsilon^2))$ time.




\section{Streaming and Distributed}
\label{sec:computational}
In this section,  we derive tractable conditions for the verification of dissipativity properties of LPV systems based on data. To this end, we can observe that verifying~\eqref{eq:thm_LPV_diss} can be seen as a robust optimization problem~\cite{ben2009robust}. Hence, we will solve the data-driven analysis problem of Theorem~\ref{thm:LPV_diss} by applying Finsler's Lemma, which translates it into a feasibility check of a scheduling-dependent definiteness condition for all $\bar p_{[1,L]}\in\ms{P}_{[1,L]}$. While this feasibility check involves verifying infinitely many definiteness conditions (one for every trajectory in $\ms{P}_{[1,L]}$), we will show that it can be reduced to a finite-dimensional LMI feasibility problem by taking structural assumptions on $\ms{P}_{[1,L]}$ and extending the analysis results from the model-based setting~\cite{apkarian2000parameterized, HoffmannWerner2014, parrilo2000structured, oliveira2007parameter}.

\subsection{Main concept}

{To derive a computable form of the data-driven analysis problem of Theorem~\ref{thm:LPV_diss},} %
we introduce the \emph{non-strict version} of Finsler's lemma~\cite{meijer2024general}
{that will prove to be an essential building block of our proposed approaches.}
\begin{lemma}[{Non-strict Finsler's Lemma \cite{meijer2024general}}]\label{lem:Finsler}
Let $A\in\mathbb{R}^{n\times n}$, $B\in\mathbb{R}^{q\times n}$.
Then, the following statements: %
\begin{enumerate}[label={(\roman*)}]
\item $x^\top Ax\geq0$ for all $x$ satisfying $Bx=0$,

\item $(B^\perp)^\top AB^\perp\succeq0$,

\item There exists a $\mu\in\mathbb{R}$ such that $A+\mu B^\top B\succeq0$,

\item There is an $X\in\mathbb{R}^{n\times q}$ such that $XB+B^\top X^\top - A \preceq 0$,
\end{enumerate}
satisfy that ``(i) $\Leftrightarrow$ (ii)'', ``(iii) $\Rightarrow$ (ii)'', and ``(iv) $\Rightarrow$ (ii)''.
\end{lemma}
According to Theorem~\ref{thm:LPV_diss}, analyzing dissipativity amounts to verifying Condition~\eqref{eq:thm_LPV_diss} for all {admissible} scheduling signals $\bar{p}_{[1,L]}\in\ms{P}_{[1,L]}$. Based on ``(i) $\Leftrightarrow$ (ii)'' in Lemma~\ref{lem:Finsler}, we infer that~\eqref{eq:thm_LPV_diss} holds if and only if
\begin{align}\label{eq:dd_Ldiss_Finsler}
\big(F(\bar{p}_{[1,L]})^\perp\big)^\top \Pi_H F(\bar{p}_{[1,L]})^\perp\succeq0.
\end{align}
The same idea has been been employed in the LTI case in~\cite{romer2019one}, where this reformulation 
provides a simple and elegant condition to verify dissipativity, only requiring to check positive semi-definiteness of a given data-dependent matrix.
For LPV systems of the form \eqref{eq:sys}, condition~\eqref{eq:dd_Ldiss_Finsler} allows us to check whether the dissipation inequality holds for a \emph{fixed} scheduling trajectory $\bar{p}_{[1,L]}$. However, we need to verify dissipativity as a \emph{system} property for the whole $\ms{P}_{[1,L]}$, which makes, together with the fact that
$F(\bar{p}_{[1,L]})^\perp$ depends nonlinearly on $\bar{p}_{[1,L]}$, the verification of~\eqref{eq:dd_Ldiss_Finsler} intractable.

Therefore, in the remainder of this section, we exploit items~(iii) and~(iv) of Lemma~\ref{lem:Finsler} in order to derive computational procedures for the verification of~\eqref{eq:thm_LPV_diss}, which are tailored to the assumed {definition of $\ms{P}_{[1,L]}$}.
First, in Section~\ref{subsec:computational_polytopic}, we employ item (iv) to address {the case when} $\mathbb{P}$ {is a convex, polytopic set, by describing} $\ms{P}_{[1,L]}$ based on a convex hull argument.
Next, in Section~\ref{subsec:computational_S_procedure}, we use the S-procedure to handle admissible scheduling sets $\ms{P}_{[1,L]}$ defined by quadratic inequalities.

\subsection{Dissipativity verification via a convex hull argument}\label{subsec:computational_polytopic}
\subsubsection{Core concept} We now provide a computational procedure to verify~\eqref{eq:thm_LPV_diss} for any $\bar{p}_{[1,L]}\in\ms{P}_{[1,L]}$, where $\ms{P}_{[1,L]}:=\mb{P}^L$ and $\mathbb{P}$ is a %
convex polytope {with finite many vertices}.


\begin{assumption}[{Polytopic description of $\mb{P}$}]\label{ass:polytopic}
The set $\mathbb{P}\subset\mb{R}^{\dnp}$ is a convex polytope, {generated by a finite {number} of vertices, i.e.,   %
$\mathbb{P}=\mathrm{co}(\{\msf{p}_i\}_{i=1}^{n_\mathrm{v}})$, where $\mathrm{co}$ denotes the convex hull.} \hfill$\square$
\end{assumption}
As the scheduling variable varies in $\mb{P}$, i.e., {$\forall k \in\mathbb{Z}:\bar{p}_k\in\mb{P}$}, every trajectory {$\bar{p}_{[1,L]}$} is confined in the space 
\[\ms{P}_{[1,L]}=\mb{P}\times\dots\times\mb{P}=\mb{P}^L,\]
which is generated with $n_\mathrm{v}^L$ vertices {that} are all permutations of the original vertices of $\mb{P}$, i.e., $\ms{P}_{[1,L]}= \mathrm{co}( \mathrm{perm}_L(\{\msf{p}_i\}_{i=1}^{n_\mathrm{v}}) )$.
We now apply ``(iv) $\Rightarrow$ (i)'' in Lemma~\ref{lem:Finsler} to derive a tailored condition for verifying~\eqref{eq:thm_LPV_diss} via a convex hull argument, i.e., by means of Assumption~\ref{ass:polytopic}. {For this, we will explicitly write $F(\bar{p}_{[1,L]})$ in its affine form, i.e., with $\bar{\vect{p}}_{[1,L]}\in\mb{R}^{L\dnp}$,
\begin{equation}\label{eq:affineF}
    F(\bar{p}_{[1,L]}) = \hat{F}_0 + \sum_{i=1}^{L\dnp} \bar{\vect{p}}_{i}\hat{F}_i. %
\end{equation}
}

\begin{prop}[$L$-dissipativity via convex hull argument]\label{prop:LPV_diss_polytopic}
Suppose Assumption~\ref{ass:polytopic} holds and let $\{\bar{\msf{p}}_\nu\}_{\nu=1}^{n_\mathrm{v}^L}$ denote the vertices of $\mb{P}^L$. 
Then,~\eqref{eq:thm_LPV_diss} holds {if} %
there exist matrices $X_i\in\mathbb{R}^{(N-L+1)\times(N-L+1)}$, {$i \in \{ 0,\dots,L\dnp\}$},
\begin{subequations}\label{eq:prop_LPV_diss_polytopic}
satisfying{
\begin{align}
    \sum_{i=0}^{L\dnp}\sum_{j=0}^{L\dnp} \bar{\vect{p}}_{i}\bar{\vect{p}}_{j}\left(X_i \hat{F}_j + \hat{F}_j^\top X_i^\top\right) -\Pi_H\preceq0, \label{eq:prop_LPV_diss_polytopic:a}\\
    X_i \hat{F}_i + \hat{F}_i^\top X_i^\top \succeq 0, \quad {\forall i \in \{ 0,\dots,L\dnp \}},\label{eq:prop_LPV_diss_polytopic:b}
\end{align}}%
\end{subequations}
{for all vertices {$\{ \bar{\msf{p}}_\nu\}_{\nu=1}^{n_\mathrm{v}^L}$}.}
\end{prop}
\begin{proof}
For any given scheduling trajectory $\bar{p}_{[1,L]}\in\ms{P}_{[1,L]}$, Lemma~\ref{lem:Finsler} implies that~\eqref{eq:thm_LPV_diss} holds if there exists {a} matrix function $X(\bar{p}_{[1,L]}):\ms{P}_{[1,L]} \to \mathbb{R}^{(N-L+1)\times(N-L+1)}$ associated with $\bar{p}_{[1,L]}$ such that
\begin{align}\label{eq:dd_Ldiss_Finsler1}
	X(\bar{p}_{[1,L]})F(\bar{p}_{[1,L]})+F(\bar{p}_{[1,L]})^\top X(\bar{p}_{[1,L]})^\top\!\! -\Pi_H \preceq 0.
\end{align}
{By considering $X$ to have affine dependency on $\bar{p}_{[1,L]}$, i.e., $X(\bar{p}_{[1,L]})=X_0+\sum_{i=1}^{L\dnp} \bar{\vect{p}}_{i}X_i$,  \eqref{eq:dd_Ldiss_Finsler1} under~\eqref{eq:affineF} reads as
\begin{equation}\label{eq:pf:nonconvex}
    \sum_{i=0}^{L\dnp}\sum_{j=0}^{L\dnp} \bar{\vect{p}}_{i}\bar{\vect{p}}_{j}\left(X_i \hat{F}_j + \hat{F}_j^\top X_i^\top\right) -\Pi_H\preceq0, \ \forall \bar{p}_{[1,L]}\in\ms{P}_{[1,L]},
\end{equation}
which {is quadratic, but not necessarily convex in $\bar{p}_{[1,L]}$}}. %
Based on the application of the multi-convexity relaxation from~\cite{apkarian2000parameterized, gahinet1996affine}, condition~\eqref{eq:prop_LPV_diss_polytopic:b} enforces convexity of~\eqref{eq:pf:nonconvex}, and hence if~\eqref{eq:prop_LPV_diss_polytopic:a} and~\eqref{eq:prop_LPV_diss_polytopic:b} hold for all vertices $\{ \bar{\msf{p}}_\nu\}_{\nu=1}^{n_\mathrm{v}^L}$ of $\mathbb{P}^L$, then~\eqref{eq:pf:nonconvex} holds for all $\bar{p}_{[1,L]}\in\ms{P}_{[1,L]}$.
\end{proof}


Proposition~\ref{prop:LPV_diss_polytopic} provides a computational approach to verify condition~\eqref{eq:thm_LPV_diss} in case the scheduling variable is varying in a convex polytope (Assumption~\ref{ass:polytopic}).
Thus, we have reduced the {condition of} data-driven dissipativity %
in Theorem~\ref{thm:LPV_diss}, {which needs to be checked for all $\bar{p}_{[1,L]}\in\ms{P}_{[1,L]}$,}  %
to the existence of matrices $X_i$ such that~\eqref{eq:prop_LPV_diss_polytopic} holds, i.e., to an SDP subject to LMI constraints.  %
 
\subsubsection{{Including rate bounds on $\mathit{p}$}}\label{sss:conservatism}
{We now discuss the inclusion of additional system properties into the analysis problem in terms of incorporating rate bounds on the scheduling signal.}
The analysis in Proposition~\ref{prop:LPV_diss_polytopic} allows for maximal \emph{variation} of the scheduling variable, i.e., $p_k-p_{k-1}$ is only limited to remain inside $\mb{P}$. This can result in conservative conclusions, e.g., on the $\ell_2$-gain of the LPV system, as the analysis also considers (possibly non-existent) fast variations of the scheduling signal. We can reduce this %
by {incorporating} %
rate bounds on $p$, {i.e.,} %
defining the {admissible scheduling set as %
\begin{equation}
    {\ms{P}_{[1,L]}} %
    = \{ p_{[1,L]}\in\mb{P}^L \mid p_k-p_{k-1}\in\mb{D}, \ \forall k = 2, \dots, L \},
\end{equation}
where $\mb{P}$ satisfies Assumption~\ref{ass:polytopic}} and $\mb{D}$ is a convex polytope that defines the rate bounds on $p$. {Note that {$\ms{P}_{[1,L]}$}  is again a convex polytope.} By verifying \eqref{eq:prop_LPV_diss_polytopic} on the vertices of {$\ms{P}_{[1,L]}$}, %
{the analysis yields a %
less conservative
conclusion on the dissipativity property of the system.}
\begin{remark}[Computational complexity vs. conservatism]\label{rem:manylmis}
	Verifying dissipativity via \eqref{eq:prop_LPV_diss_polytopic} \emph{without} incorporation of the rate bounds on $p$ requires solving an SDP with {$(1+L\dnp)n_\mathrm{v}^L$} LMI constraints. This number further increases when including rate bounds, which increases the computational complexity exponentially for larger $L$. Explosion of the computational load for large $L$ can be mitigated by replacing~\eqref{eq:prop_LPV_diss_polytopic} with
\[ XF(\bar{\msf{p}}_\nu) + F(\bar{\msf{p}}_\nu)^\top X^\top  -\Pi_H \preceq0, \quad \nu=1,\dots, n_\mr{v}^L.\]
Introducing a constant $X$ can introduce conservatism, however, it also allows to alleviate the computational load of the dissipativity test.
\end{remark}
{We will see in Section~\ref{sec:examples} that Proposition~\ref{prop:LPV_diss_polytopic} is readily applicable to small-scale LPV systems and it provides tight conditions for verifying $L$-dissipativity from data.}

\subsection{Dissipativity verification via the S-procedure}\label{subsec:computational_S_procedure}
In this section, we assume that $\bar{p}_{[1,L]}$ is varying in a space that is described by a quadratic inequality instead of a convex polytope, i.e., we assume that the scheduling signals in $\ms{P}_{[1,L]}$ are described as follows:
\begin{assumption}[{Quadratic description of $\ms{P}_{[1,L]}$}]\label{ass:ellipsoidal}
The scheduling signal $\bar{p}_{[1,L]}$ varies around a nominal scheduling trajectory $\breve{p}_{[1,L]}$, such that $\bar{p}_{[1,L]}$ satisfies
\begin{align}\label{eq:p_norm_bound}
\lVert \bar{p}_k-\breve{p}_{k}\rVert_W\leq p_{\max}
\end{align}
for all $k=1,\dots,L$ with some $p_{\max}>0$. The set $\ms{P}_{[1,L]}$ can then be trivially described by  
\begin{equation}\label{eq:ass_ellipsoidal}
	{\ms{P}_{[1,L]}}=\left\{\bar{p}_{[1,L]}\in\mb{P}^L \Bigm| \vphantom{ \begin{bmatrix} \hat{\mc{P}}^{\dnu+\dny} \\ I \end{bmatrix}^\top} %
	\begin{bmatrix} \bar{\mc{P}}^{\dnu,\dny} \\ I \end{bmatrix}^{\!\top}\!\! M_P \begin{bmatrix} \bar{\mc{P}}^{\dnu,\dny} \\ I \end{bmatrix}\succeq 0 \right\}, %
\end{equation}
with $\bar{\mc{P}}^{\dnu,\dny}=\begin{bmatrix} \bar{\mathcal{P}}^{\dnu} & 0 \\ 0 & \bar{\mathcal{P}}^{\dny}\end{bmatrix}$, $\bar{\mathcal{P}}^{n_\bullet}=\bar{p}_{[1,L]}\bkron I_{n_\bullet}$.
\hfill$\square$
\end{assumption}
{Note that for $W=2$ in~\eqref{eq:p_norm_bound},} %
we have that $M_P$ in~\eqref{eq:ass_ellipsoidal} is
\begin{align}\label{eq:def_of_MP_ellipsiodal}
	M_P = \begin{bmatrix}
		-I & \breve{\mathcal{P}}^{\dnu,\dny} \\
		(\breve{\mathcal{P}}^{\dnu,\dny})^\top & p_{\max}^2 I -(\breve{\mathcal{P}}^{\dnu,\dny})^\top\breve{\mathcal{P}}^{\dnu,\dny}
	\end{bmatrix},
\end{align}
with $\breve{\mc{P}}^{\dnu,\dny} =\begin{bmatrix}\breve{\mc{P}}^{\dnu}&0\\0&\breve{\mc{P}}^{\dny}\end{bmatrix}$, $\breve{\mc{P}}^{n_\bullet}=\breve{p}_{[1,L]}\bkron I_{n_\bullet}$. Note that with Assumption~\ref{ass:ellipsoidal}, we can describe types of $\ms{P}_{[1,L]}$ that have a spherical or {hyper-rectangular} form. %
Next, we employ the S-procedure (see, e.g.,~\cite{scherer1997full, scherer2001lpv}) to derive a tractable reformulation of~\eqref{eq:thm_LPV_diss} for scheduling sets $\ms{P}_{[1,L]}$ that satisfy Assumption~\ref{ass:ellipsoidal}.
To this end, let us write $F(\bar{p}_{[1,L]})$ in~\eqref{eq:Pi_H_F_p_def} as
\begin{equation*}
\hspace{-0.5mm}	F(\bar{p}_{[1,L]})\!  =\!\!  %
	\underbrace{\begin{bmatrix} F_1 \\ \mc{H}_L(u_{[1,N]}^{\mt{p}}) \\ \mc{H}_L(y_{[1,N]}^{\mt{p}}) \end{bmatrix}}_{F_3}\!  -\! 	\underbrace{\begin{bmatrix} 0 & 0 \\ I & 0 \\ 0 & I \end{bmatrix}}_{F_4}\!\! \begin{bmatrix} \bar{\mathcal{P}}^{\dnu}\!\!\! & 0 \\ 0 & \!\!\!\bar{\mathcal{P}}^{\dny}\end{bmatrix} \!\! \underbrace{\begin{bmatrix} \mc{H}_L(u_{[1,N]}) \\ \mc{H}_L(y_{[1,N]}) \end{bmatrix}}_{F_5}\!.
\end{equation*}
\begin{prop}[$L$-dissipativity via the S-procedure]\label{prop:LPV_diss_ellipsoidal}
Suppose Assumption~\ref{ass:ellipsoidal} holds.
Then,~\eqref{eq:thm_LPV_diss} holds if there exist a $\mu\in\mathbb{R}$ and a $\lambda\geq0$ such that
\begin{align}\label{eq:prop_LPV_diss_ellipsoidal}
\begin{bmatrix}\mu F_4^\top F_4&\mu F_4^\top F_3\\
\mu F_3^\top F_4&\mu F_3^\top F_3+\Pi_H\end{bmatrix}-\lambda \begin{bmatrix}I&0\\0&F_5\end{bmatrix}^{\!\top} \!\!
M_P
\begin{bmatrix}I&0\\0&F_5\end{bmatrix}\succeq0.
\end{align}
\end{prop}


\begin{proof}
{Using}
the implication of (i) by (iii) in Lemma~\ref{lem:Finsler},~\eqref{eq:thm_LPV_diss} holds if for each $\bar{p}_{[1,L]}\in\ms{P}_{[1,L]}$ there exists a $\mu\in\mathbb{R}$ satisfying
\begin{align}\label{eq:prop_LPV_diss_ellipsoidal_proof1}
	\Pi_H+\mu F^\top\!(\bar{p}_{[1,L]}) F(\bar{p}_{[1,L]})\succeq0.
\end{align}
 {Note that {in terms of this condition,} $\mu$ {can be different for each} $\bar{p}_{[1,L]}$. To obtain a convex problem, we {enforce} $\mu$ to be {the same}\footnote{{See Remark~\ref{rem:PVmulti} for a discussion on a scheduling dependent $\mu$.}} for all $\bar{p}_{[1,L]}\in\ms{P}_{[1,L]}$}, {which inherently introduces conservatism.}
{Then, we can rewrite}~\eqref{eq:prop_LPV_diss_ellipsoidal_proof1} {as} %
\begin{align}\label{eq:prop_LPV_diss_ellipsoidal_proof2}
\begin{bmatrix}\bar{\mathcal{P}}^{\dnu,\dny}F_5\\I\end{bmatrix}^\top
\begin{bmatrix}\mu F_4^\top F_4&\mu F_4^\top F_3\\
\mu F_3^\top F_4&\mu F_3^\top F_3+\Pi_H\end{bmatrix}
\begin{bmatrix}\bar{\mathcal{P}}^{\dnu,\dny}F_5\\I\end{bmatrix}\succeq0.
\end{align}
{By} left and right {multiplication of {the}} inequality in~\eqref{eq:ass_ellipsoidal} {with} $F_5^\top$ and $F_5$, respectively, we infer that $\bar{p}_{[1,L]}\in\ms{P}_{[1,L]}$ {if and only if} %
\begin{align}\label{eq:prop_LPV_diss_ellipsoidal_proof3}
	\begin{bmatrix} \bar{\mathcal{P}}^{\dnu,\dny}F_5 \\ I \end{bmatrix}^\top
	\begin{bmatrix} I & 0 \\ 0 & F_5 \end{bmatrix}^\top
	M_P
	\begin{bmatrix} I & 0 \\ 0 & F_5 \end{bmatrix}
	\begin{bmatrix} \bar{\mathcal{P}}^{\dnu,\dny}F_5 \\ I \end{bmatrix}\succeq 0.
\end{align}
Multiplying~\eqref{eq:prop_LPV_diss_ellipsoidal} with $\begin{bmatrix}(\bar{\mathcal{P}}^{\dnu,\dny}F_5)^\top & I \end{bmatrix}$ from the left and with $\begin{bmatrix}(\bar{\mathcal{P}}^{\dnu,\dny}F_5)^\top & I \end{bmatrix}^\top$ from the right and using~\eqref{eq:prop_LPV_diss_ellipsoidal_proof3}, we {get that if \eqref{eq:prop_LPV_diss_ellipsoidal} holds, then}~\eqref{eq:prop_LPV_diss_ellipsoidal_proof2} is implied, which concludes the proof.
\end{proof}

Proposition~\ref{prop:LPV_diss_ellipsoidal} provides a sufficient condition for~\eqref{eq:thm_LPV_diss} and hence, by Theorem~\ref{thm:LPV_diss}, for $(L-\tau)$-dissipativity of the underlying LPV system.
Verifying feasibility of~\eqref{eq:prop_LPV_diss_ellipsoidal} is an SDP with only two decision variables $\mu$ and $\lambda$, and thus it can be {efficiently} implemented from a {computational} perspective for {moderate and even large choices of $L$ and $N$} %
data lengths (see Section~\ref{sec:examples} for a numerical example).

Note that, {similar to} Proposition~\ref{prop:LPV_diss_polytopic}, Proposition~\ref{prop:LPV_diss_ellipsoidal} only provides a \emph{sufficient} condition for~\eqref{eq:thm_LPV_diss}.
This is because the multiplier $\mu\in\mathbb{R}$ %
is chosen {to be the same for all} %
scheduling {trajectories} $\bar{p}_{[1,L]}$.
Therefore, if, e.g., a norm bound similar to~\eqref{eq:p_norm_bound} is available, then translating this bound into a convex polytope as in Assumption~\ref{ass:polytopic} and applying Proposition~\ref{prop:LPV_diss_polytopic} will %
generally lead to less conservative results than considering the quadratic description in Assumption~\ref{ass:ellipsoidal} and applying Proposition~\ref{prop:LPV_diss_ellipsoidal}, even for a constant $X$.
On the other hand, as we will also demonstrate with a numerical example in Section~\ref{sec:examples}, Proposition~\ref{prop:LPV_diss_ellipsoidal} is significantly more efficient  from a computational perspective than Proposition~\ref{prop:LPV_diss_polytopic}.

{Finally, we want to highlight that for all the introduced computational methods, we assume that the `true' admissible scheduling set $\ms{P}_{[1,L]}$ is equivalent with the assumed descriptions (Assumptions~\ref{ass:polytopic}~and~\ref{ass:ellipsoidal}). If the assumed descriptions are in fact \emph{over approximating} $\ms{P}_{[1,L]}$, which can happen when the LPV representation is in fact a \emph{surrogate} model for a nonlinear system, then this again introduces a source of conservatism in the analysis. This issue of conservatism and minimizing it is actively studied in %
constructing LPV surrogate models of nonlinear systems \cite{SADEGHZADEH20204737}.}


\begin{remark}[{Sampling-based $L$-dissipativity analysis}]\label{rem:PVmulti}
The described conservatism of Proposition~\ref{prop:LPV_diss_ellipsoidal} can be alleviated by considering \emph{parameter-dependent} multipliers for the application of Finsler's Lemma (Lemma~\ref{lem:Finsler}), similar to Proposition~\ref{prop:LPV_diss_polytopic}.
To be precise,~\eqref{eq:prop_LPV_diss_ellipsoidal_proof1} can be replaced by
\begin{align}\label{eq:prop_LPV_diss_ellipsoidal_pv}
	\Pi_H+\mu(\bar{p}_{[1,L]}) F(\bar{p}_{[1,L]})^\top F(\bar{p}_{[1,L]})\succeq0,
\end{align}
for some $\mu:\ms{P}_{[1,L]}\to\mathbb{R}$.
In fact, by applying Finsler's Lemma point-wise for any parameter trajectory $\bar{p}_{[1,L]}\in\ms{P}_{[1,L]}$, the existence of a mapping $\mu(\cdot)$ such that~\eqref{eq:prop_LPV_diss_ellipsoidal_pv} holds is a less conservative\footnote{In fact, the provided condition is also necessary, and therefore non-conservative, for \emph{strict} $L$-dissipativity.\label{footnote}} condition for~\eqref{eq:thm_LPV_diss} than~\eqref{eq:prop_LPV_diss_ellipsoidal}.
In practice, inequality~\eqref{eq:prop_LPV_diss_ellipsoidal_pv} can be checked by drawing $N_\mathrm{s}$ (e.g., uniformly distributed) samples $\hat{p}_{[1,L]}^i$ from $\ms{P}_{[1,L]}$ and verifying the existence of $\mu_i\in\mathbb{R}$, $i=1,\dots,N_\mathrm{s}$ such that
\begin{align}
	\Pi_H+\mu_i F(\hat{p}_{[1,L]}^i)^\top F(\hat{p}_{[1,L]}^i)\succeq0
\end{align}
for all $i=1,\dots,N_\mathrm{s}$.
For any fixed number $N_\mathrm{s}$, this yields a less conservative\tss{\ref{footnote}} condition for~\eqref{eq:thm_LPV_diss} and, hence, for $L$-dissipativity compared to Proposition~\ref{prop:LPV_diss_ellipsoidal}.
The resulting dissipativity test is \emph{tight} in the limit $N_\mathrm{s}\to\infty$.
While choosing a dense grid over $\ms{P}_{[1,L]}$, i.e., a large value of $N_\mathrm{s}$, can be prohibitive due to the curse of dimensionality, good results can often be achieved in practice for reasonable values of $N_\mathrm{s}$, see the numerical example in Section~\ref{sec:examples}.
\end{remark}
\begin{remark}[Reducing conservatism with matrix-valued multipliers]
    Another option to reduce the conservatism of Proposition~\ref{prop:LPV_diss_ellipsoidal} is to consider a matrix-valued~$\lambda$. As \eqref{eq:p_norm_bound} constitutes~$L$ inequalities, we can %
    {introduce} at least~$L$ degrees of freedom in~$\lambda$ using %
    full-block multipliers~\cite{scherer1997full}. {Using full-block multipliers} does, on the other hand, significantly increase the complexity of the problem. Exploring such a formulation of the analysis is interesting for further research.
\end{remark}

\section{Applications}
\label{sec:apps}
In this section we provide a brief and non-exhaustive, yet representative, coverage of
application domains in which DSP paved the way to interesting solutions for real-world problems.

\spara{Web and Social Networks.} Densest Subgraph has been usefully exploited in developing solutions for several problems related to the web and to social networks. 
\rev{Gibson et al.~\cite{gibson} and Dourisboure et al.~\cite{Dourisboure+07,DourisboureGP09} demonstrated that dense subgraph discovery algorithms are useful for extracting communities in the web graphs.}
\rev{Gajewar and Das Sarma~\cite{Gajewar+12} considered the problem of identifying a team of skilled individuals for collaboration, in which the goal is to maximize the collaborative compatibility of the team, and formulated it as a variant of DSP. 
}
Rangapuram et al. \cite{rangapuram2013team} \rev{also} addressed the team formation problem, in which there is need of selecting a set of employees to complete a certain task, under the constraint \rev{on} a cost function.
Angel et al.~\cite{Angel2013} studied real-time story identification on Twitter via maintenance of \rev{the} densest subgraph in the fully dynamic setting.
Similarly, Bonchi et al. \cite{BonchiBGS16,BonchiBGS19} adopted anomalously dense subgraphs in temporal networks as a way to identify buzzing stories in social media.
Hooi et al. \cite{hooi2016fraudar} approached fraudulent reviews detection via dense subgraph discovery in user-product bipartite graphs.
Kawase et al.~\cite{kawase2019crowd} improved the extraction of reliable experts for answer aggregation in the crowdsourcing framework.
Yikun et al.~\cite{yikun2019no} devised an algorithm for detecting fraudulent entities in tensors based on densest subgraphs.
Kim et al.~\cite{kim2020densely} modeled the community search problem taking into account spatial locations.
Tan et al.~\cite{tan2020scaling} proposed to employ the densest subgraph for candidate committers selection in open source communities.
Fazzone et al. \cite{Fazzone2022} adopted HDSP (see Section \ref{subsec:labeled}) to detect polarized niches in social networks, i.e., set of users that are far from authoritative sources of information and at the same time close to misinformation spreaders.


\spara{Biology.} In biology and in particular in ``omics'' disciplines, there are plenty of settings in which the data is represented with graphs \cite{koutrouli2020guide}. For instance, gene co-expression networks or Protein-Protein Interaction \rev{(PPI)} networks, are graphs built from correlation matrices: in such networks a dense subgraph can represent a set of genes/proteins that are regulating the same process.
Hu et al. \cite{hu2005coherent} defined algorithms for dense and coherent subgraphs across networks, in order to detect recurrent patterns across multiple networks to discover biological modules.
Fratkin et al. \cite{fratkin} exploited dense subgraphs to find regulatory motifs in genomic DNA, by creating a graph where vertices correspond to $k$-mers (sequence of $k$ DNA bases) and edges to $k$-mers that differ in few positions.
Everett et al. \cite{Everett} proposed to extract dense structures in networks composed by transcription factors, their putative target genes, and the tissues in which the target genes are differentially expressed; in this context, they defined a dense subgraph as a transcriptional module.
Saha et al. \cite{SahaHKRZ10} studied the problem of finding complex annotation patterns in gene annotation graphs. Given a distance metric between any pair of nodes, the densest subgraph respecting a specific minimum distance threshold is produced.
Feng et al. \cite{feng} combined the information brought by \rev{PPI} data and microarray gene expression profiles, and provided a densest-subgraph-based algorithm to identify protein complexes.
Li et al. \cite{Li2022densestbio} recently tested different algorithms for DSP (and related problems) for detecting hot spots in \rev{PPI} networks.
Lanciano et al. \cite{lanciano2022biocontrast} modeled the differential co-expression analysis problem with the detection of contrastive subgraphs in co-expression networks of different subtype of breast cancer.
Martini et al.~\cite{martini2022network} developed a dense subgraph searching method for jointly prioritizing putative causal genes for disease and selecting one biologically similar potential causal gene at each genetic risk locus.

\spara{Finance.} Another domain in which \rev{it} is natural to model the data with graphs is finance. Boginski et al.~\cite{BBP} exploited dense subgraphs to predict the behavior of financial instruments, through the lens of the maximum clique. In fact, correlating the stock trends and representing these with a graph, is possible to define a dense structure as a set of stocks whose trend is similar, and viceversa for an independent set.
Li et al. \cite{li2020money} proposed to detect money laundering, modeling the transactions with a multipartite directed graph.
Ren et al.~\cite{ren2021ensemfdet} employed densest subgraphs to design an ensemble method for fraud detection in e-commerce.
Jiang et al.~\cite{Jiang2022spade} developed a real-time fraud detection framework called Spade, which can detect fraudulent communities in hundreds of microseconds on million-scale evolving graphs by incrementally maintaining dense subgraphs.
Chen and Tsourakakis \cite{chen2022anti} approached fraud detection in financial networks by mining dense subgraphs deviating significantly from Benford's law, which describes the distribution of the first digit of numbers appearing in a wide variety of numerical data, and has been used to raise ``red flags'' about potential anomalies in the data such as tax evasion.
Ji et al. \cite{cash_out_2022} proposed to identify cash-out \rev{behaviors}, i.e., withdrawal of cash from a credit card by illegitimate payments with merchants, with densest subgraphs subject to the optimization of a class of suspiciousness metrics.
Xie et al.~\cite{xie2022orion} applied DSP algorithms inside their algorithmic proposal for zero-knowledge proof, a powerful cryptographic primitive that has found various applications.

\spara{Miscellanea.}
\rev{Chen and Saad~\cite{Chen+12} employed a dense subgraph discovery algorithm in the scenario of community detection, where the number of communities is unknown and some vertices may not belong to any of them.} 
Moro et al.~\cite{Moro2014entity} devised an algorithm for entity linking and word sense disambiguation, using densest subgraphs.
Rozenshtein et al.~\cite{rozenshtein2014event} detected interesting events in activity networks, by maximizing the density minus the distance between all the nodes included in the final solution. 
Different works also modeled the reachability and distance query indexing problem via the densest subgraph framework~\cite{CohenHKZ02,JinXRF09}. 
\rev{Shin et al.~\cite{Shin+16,Shin+17,Shin+17_2} developed efficient algorithms for detecting dense subtensors based on densest subgraphs.}
Kamara and Moataz~\cite{kamara2019computationally} implemented structured encryption schemes with computationally-secure leakage, based on the hardness results of the planted densest subgraph problem.
Lanciano et al.~\cite{lanciano2020contrast} proposed to detect the most contrastive subgraph in terms of density for different groups of brain networks.
Wu et al.~\cite{wu2021extracting} extracted densest subgraphs in brain networks, proposing a likelihood-based objective function, to identify brain regions associated to schizophrenia disorder.
Majbouri et al.~\cite{majbouri2020prediction} leveraged DSP to boost the prediction of information diffusion paths in social networks, while Bhadra and Bandyopadhyay~\cite{bhadra2021supervised} exploited it to perform a better feature selection.
Yan et al.~\cite{yan2021anomaly} performed anomaly detection of network streams via densest subgraphs.
DSP on vertex-weighted graphs is used for efficient distribution of quantum circuits by Sundaram et al.~\cite{sundaram2021efficient}.
Lusk et al.~\cite{lusk2021clipper} designed a generalized version of the maximum clique problem to perform robust data association.
Konar and Sidiropoulos~\cite{konar2022triangle} showed that the generalization TD$k$S of D$k$S is useful for unsupervised document summarization.
Sukeda \rev{et al.}~\cite{sukeda2022study} accelerated a column generation algorithm for a clustering problem called the modularity density maximization problem, using the greedy peeling algorithm for a variant of DSP.
Recently Chen et al.~\cite{chen2022algorithmic} introduced a novel framework for motif detection, whose algorithmic proposal for testing the statistical significance of a single motif is based on the greedy peeling for DSP. 
\rev{
Very recently, Ding and Du~\cite{Ding+23} addressed the problem of detecting the edge correlation between a pair of Erd\H{o}s--R\'enyi graphs, based on the observation that the detection problem is related to DSP. 
}



\section{Conclusions and Open Problems}
\label{sec:conclusions}
\section{Conclusions}
We consider the phase-extraction problem, and we showed that, given a unitary $U = e^{i\pi H}$ and its inverse $U^{\dag}$, we could implement a block-encoding of $\phi(H)$ for some smooth function $\phi(x)$. The word `smooth' here means existence and continuity of the derivatives: the higher the number of continuous derivatives that a function has, the faster its Fourier sum (and thus the Laurent polynomial on the eigenphases) uniformly converges to that function. We are confident this can have many more applications beyond what is shown in this work. It is also worth remarking that Jackson showed that the convergence rate of a Fourier series is almost-optimal, in the sense that no trigonometric (or, equivalently, complex exponential) series can approximate the desired function faster, up to that $\log d$ factor~\cite[p.\ 21]{jacksonTheoryApproximation1930a}. Also remember that `smoothing' a function, i.e., replacing its derivative with a continuous function, does not give faster convergence for free in general, as its derivative will become steep in the points where we smooth out discontinuities, and this translates to a high Lipschitz constant: a~clear example is given by Eq.~\ref{eq:lipschitz-constant-recurrence-solution}, but in that case, fortunately, nothing depends on the size of the input $N$, and thus does not influence the asymptotic query complexity of Algorithm~\ref{alg:prop-sampling-qsp}, although the constant factor can become large even for $p = 20$. From a theoretical point of view, this work shows that, for any $\eta > 0$, there is an algorithm with query complexity 
$$\Tilde{\bigO}\left(\frac{1}{\bar{c}^{\frac{1}{2} + \eta}} \frac{1}{\epsilon^\eta} \right)$$
solving the proportional-sampling problem. This statement seems to suggest there exists an algorithm which directly solves the problem with $\eta = 0$, and an open question would be to find such algorithm.


It is also interesting to remark that Theorems~\ref{thm:haah-construction},~\ref{thm:haah-completion} indeed allow the construction for any $\phi$, even complex-valued, provided that its absolute value is reciprocal.

One could think that, in Section~\ref{sec:prop-sampling}, instead of using the linear function in the phase-extraction subroutine, we could approximate the square root and then apply the transformation directly on $e^{i \pi c(x)}$. However, in the case of proportional sampling this would be inconvenient, as the derivative of the square root function has a discontinuity with an infinite jump around 0, and we could not choose a constant $\delta$ if we had values of the oracle that are too close to $0$.

\bibliographystyle{acm}
\bibliography{biblio}
\end{document}
