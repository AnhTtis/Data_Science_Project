In this section we provide a brief and non-exhaustive, yet representative, coverage of
application domains in which DSP paved the way to interesting solutions for real-world problems.

\spara{Web and Social Networks.} Densest Subgraph has been usefully exploited in developing solutions for several problems related to the web and to social networks. 
\rev{Gibson et al.~\cite{gibson} and Dourisboure et al.~\cite{Dourisboure+07,DourisboureGP09} demonstrated that dense subgraph discovery algorithms are useful for extracting communities in the web graphs.}
\rev{Gajewar and Das Sarma~\cite{Gajewar+12} considered the problem of identifying a team of skilled individuals for collaboration, in which the goal is to maximize the collaborative compatibility of the team, and formulated it as a variant of DSP. 
}
Rangapuram et al. \cite{rangapuram2013team} \rev{also} addressed the team formation problem, in which there is need of selecting a set of employees to complete a certain task, under the constraint \rev{on} a cost function.
Angel et al.~\cite{Angel2013} studied real-time story identification on Twitter via maintenance of \rev{the} densest subgraph in the fully dynamic setting.
Similarly, Bonchi et al. \cite{BonchiBGS16,BonchiBGS19} adopted anomalously dense subgraphs in temporal networks as a way to identify buzzing stories in social media.
Hooi et al. \cite{hooi2016fraudar} approached fraudulent reviews detection via dense subgraph discovery in user-product bipartite graphs.
Kawase et al.~\cite{kawase2019crowd} improved the extraction of reliable experts for answer aggregation in the crowdsourcing framework.
Yikun et al.~\cite{yikun2019no} devised an algorithm for detecting fraudulent entities in tensors based on densest subgraphs.
Kim et al.~\cite{kim2020densely} modeled the community search problem taking into account spatial locations.
Tan et al.~\cite{tan2020scaling} proposed to employ the densest subgraph for candidate committers selection in open source communities.
Fazzone et al. \cite{Fazzone2022} adopted HDSP (see Section \ref{subsec:labeled}) to detect polarized niches in social networks, i.e., set of users that are far from authoritative sources of information and at the same time close to misinformation spreaders.


\spara{Biology.} In biology and in particular in ``omics'' disciplines, there are plenty of settings in which the data is represented with graphs \cite{koutrouli2020guide}. For instance, gene co-expression networks or Protein-Protein Interaction \rev{(PPI)} networks, are graphs built from correlation matrices: in such networks a dense subgraph can represent a set of genes/proteins that are regulating the same process.
Hu et al. \cite{hu2005coherent} defined algorithms for dense and coherent subgraphs across networks, in order to detect recurrent patterns across multiple networks to discover biological modules.
Fratkin et al. \cite{fratkin} exploited dense subgraphs to find regulatory motifs in genomic DNA, by creating a graph where vertices correspond to $k$-mers (sequence of $k$ DNA bases) and edges to $k$-mers that differ in few positions.
Everett et al. \cite{Everett} proposed to extract dense structures in networks composed by transcription factors, their putative target genes, and the tissues in which the target genes are differentially expressed; in this context, they defined a dense subgraph as a transcriptional module.
Saha et al. \cite{SahaHKRZ10} studied the problem of finding complex annotation patterns in gene annotation graphs. Given a distance metric between any pair of nodes, the densest subgraph respecting a specific minimum distance threshold is produced.
Feng et al. \cite{feng} combined the information brought by \rev{PPI} data and microarray gene expression profiles, and provided a densest-subgraph-based algorithm to identify protein complexes.
Li et al. \cite{Li2022densestbio} recently tested different algorithms for DSP (and related problems) for detecting hot spots in \rev{PPI} networks.
Lanciano et al. \cite{lanciano2022biocontrast} modeled the differential co-expression analysis problem with the detection of contrastive subgraphs in co-expression networks of different subtype of breast cancer.
Martini et al.~\cite{martini2022network} developed a dense subgraph searching method for jointly prioritizing putative causal genes for disease and selecting one biologically similar potential causal gene at each genetic risk locus.

\spara{Finance.} Another domain in which \rev{it} is natural to model the data with graphs is finance. Boginski et al.~\cite{BBP} exploited dense subgraphs to predict the behavior of financial instruments, through the lens of the maximum clique. In fact, correlating the stock trends and representing these with a graph, is possible to define a dense structure as a set of stocks whose trend is similar, and viceversa for an independent set.
Li et al. \cite{li2020money} proposed to detect money laundering, modeling the transactions with a multipartite directed graph.
Ren et al.~\cite{ren2021ensemfdet} employed densest subgraphs to design an ensemble method for fraud detection in e-commerce.
Jiang et al.~\cite{Jiang2022spade} developed a real-time fraud detection framework called Spade, which can detect fraudulent communities in hundreds of microseconds on million-scale evolving graphs by incrementally maintaining dense subgraphs.
Chen and Tsourakakis \cite{chen2022anti} approached fraud detection in financial networks by mining dense subgraphs deviating significantly from Benford's law, which describes the distribution of the first digit of numbers appearing in a wide variety of numerical data, and has been used to raise ``red flags'' about potential anomalies in the data such as tax evasion.
Ji et al. \cite{cash_out_2022} proposed to identify cash-out \rev{behaviors}, i.e., withdrawal of cash from a credit card by illegitimate payments with merchants, with densest subgraphs subject to the optimization of a class of suspiciousness metrics.
Xie et al.~\cite{xie2022orion} applied DSP algorithms inside their algorithmic proposal for zero-knowledge proof, a powerful cryptographic primitive that has found various applications.

\spara{Miscellanea.}
\rev{Chen and Saad~\cite{Chen+12} employed a dense subgraph discovery algorithm in the scenario of community detection, where the number of communities is unknown and some vertices may not belong to any of them.} 
Moro et al.~\cite{Moro2014entity} devised an algorithm for entity linking and word sense disambiguation, using densest subgraphs.
Rozenshtein et al.~\cite{rozenshtein2014event} detected interesting events in activity networks, by maximizing the density minus the distance between all the nodes included in the final solution. 
Different works also modeled the reachability and distance query indexing problem via the densest subgraph framework~\cite{CohenHKZ02,JinXRF09}. 
\rev{Shin et al.~\cite{Shin+16,Shin+17,Shin+17_2} developed efficient algorithms for detecting dense subtensors based on densest subgraphs.}
Kamara and Moataz~\cite{kamara2019computationally} implemented structured encryption schemes with computationally-secure leakage, based on the hardness results of the planted densest subgraph problem.
Lanciano et al.~\cite{lanciano2020contrast} proposed to detect the most contrastive subgraph in terms of density for different groups of brain networks.
Wu et al.~\cite{wu2021extracting} extracted densest subgraphs in brain networks, proposing a likelihood-based objective function, to identify brain regions associated to schizophrenia disorder.
Majbouri et al.~\cite{majbouri2020prediction} leveraged DSP to boost the prediction of information diffusion paths in social networks, while Bhadra and Bandyopadhyay~\cite{bhadra2021supervised} exploited it to perform a better feature selection.
Yan et al.~\cite{yan2021anomaly} performed anomaly detection of network streams via densest subgraphs.
DSP on vertex-weighted graphs is used for efficient distribution of quantum circuits by Sundaram et al.~\cite{sundaram2021efficient}.
Lusk et al.~\cite{lusk2021clipper} designed a generalized version of the maximum clique problem to perform robust data association.
Konar and Sidiropoulos~\cite{konar2022triangle} showed that the generalization TD$k$S of D$k$S is useful for unsupervised document summarization.
Sukeda \rev{et al.}~\cite{sukeda2022study} accelerated a column generation algorithm for a clustering problem called the modularity density maximization problem, using the greedy peeling algorithm for a variant of DSP.
Recently Chen et al.~\cite{chen2022algorithmic} introduced a novel framework for motif detection, whose algorithmic proposal for testing the statistical significance of a single motif is based on the greedy peeling for DSP. 
\rev{
Very recently, Ding and Du~\cite{Ding+23} addressed the problem of detecting the edge correlation between a pair of Erd\H{o}s--R\'enyi graphs, based on the observation that the detection problem is related to DSP. 
}

