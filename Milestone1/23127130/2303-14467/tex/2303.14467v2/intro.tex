Extracting a dense subgraph from a given input network is a key primitive~\cite{aggarwal,gionis2015dense} in graph mining which, while being investigated since the seventies, keeps attracting a lot of algorithmic research attention nowadays.
Depending on the semantics of the given network, a dense subgraph can represent different interesting patterns: for instance, it can represent a community of closely connected individuals in social networks, a regulatory motif in genomic DNA, or a set of actors coordinating a financial fraudulent activity in a transaction network (see Section \ref{sec:apps} for applications).

\rev{While the literature about extracting dense substructures from graphs is much wider, the focus of the present survey is exclusively on the classic \emph{Densest Subgraph Problem} (DSP) and its many variants. Here, we acknowledge the existence of the literature on related notions, which are not the focus of this survey.}
For instance, the substantial algorithmic effort has been devoted to enumerating all existing dense structures, such as \emph{maximal cliques}, \emph{quasi-cliques}, \rev{\emph{paracliques}, $s$-\emph{cliques}}, $k$-\emph{plex}, $k$-\emph{club}, etc.,~\cite{aggarwal,WU2015693,Chesler+06,chang2018cohesive,farago2019survey,fang2022cohesive,Shahinpour+13}. As we will see more in details later, also the notion of \emph{core decomposition} (see a survey in \cite{malliaros2020core}) is strongly related.
Finding ``clusters'' in graphs is also typically operationalized as finding groups of vertices which are densely connected inside and sparsely connected with the outside. In this regards, \emph{graph clustering}, \emph{graph partitioning}, \rev{\emph{cluster editing},} \emph{spectral clustering} \cite{schaeffer2007graph,malliaros2013clustering,nascimento2011spectral,bulucc2016recent,Bocker+13} are all related notions, as it is the popular topic of \emph{community detection} \cite{fortunato2010community}, or the notion of \emph{correlation clustering} \cite{bonchi_cc}.


Departing from the literature, in this survey we focus on the fundamental problem of extracting \emph{one and only one subgraph}, that maximizes a measure of density.
In the more general setting, we are given a simple graph $G=(V,E)$ \rev{with $n=|V|$ vertices and $m=|E|$ edges. Given} a subset of vertices $S\subseteq V$,  let $G[S]=(S,E[S])$ be the subgraph induced by $S$, and let $e[S]$ be the size of $E[S]$.
The most straightforward notion of density is the so-called \emph{edge density}, defined for a set of vertices $S$ as $\delta(S) = e[S] / {|S| \choose 2}$. However, it is easy to see that finding a set of vertices $S \subseteq V$ that maximizes $\delta(S)$ is trivial and not interesting, as any pair of vertices connected by an edge forms an optimal solution for this objective function.
For this reason, the formulation known as the \emph{Densest Subgraph Problem} (DSP) aims to find a subset of vertices $S \subseteq V$ that maximizes the \emph{degree density} of $S$, defined as $d(S)= e[S]/|S|$.
Note that this objective function is equivalent to half of the average degree of the vertices within the subgraph (since each edge contributes 2 to the sum of degrees, $e[S]$ is half of the sum of the degrees of the vertices in $S$).
This specific problem has received \rev{a} great deal of attention in the algorithmic literature over the last five decades, with many variants and many applications built on top of this basic definition. To have an idea of the span of research interest
just consider that, while the last couple of years have seen the publication of some groundbreaking results and several interesting contributions, e.g., \cite{harb2022faster,MaCLH22,Luo2023,fang2022densest,boob2020flowless,Chekuri2022supermod,bera2022dynamicsub,liu2022stochastic,Fazzone2022,ma2022convex,veldt2021meandensest,gudapati2021greedy,bonchi2021finding,ma2021directed,ma2021efficient,chekuri2023generalized},
the work by Picard and Queyranne \cite{Picard82}, introducing an exact algorithm for DSP, dates back to 1979 (technical report, although the final paper was officially published in 1982). This survey aims to (1) discuss the fundamental results for DSP in depth, and (2) provide an exhaustive coverage of the many variants proposed in the literature, with a special emphasis on the most recent results.

The survey is organized as follows.
In Section \ref{sec:fundamental} we review the fundamental algorithmic results for DSP, distinguishing between exact algorithms and approximate ones and highlighting connections between DSP and other related algorithmic problems, such as \emph{submodular function minimization} or the computation of the \emph{core decomposition}. Section \ref{sec:cons} covers constrained variants of DSP, including, e.g., constraints on the admissible size of the solution, constraints on set of vertices to be included in the solution, or constraints on the level of connectivity of the solution. Section \ref{sec:variants} instead covers variants that tackle \rev{different but related} objective functions. In Section \ref{sec:graphs} we summarize the massive literature on extracting dense subgraphs from  information-richer graphs such as directed, signed, weighted, probabilistic (or uncertain), multilayer, temporal graphs, etc.
Section \ref{sec:computational} covers \rev{variants of DSP in} different computational settings, such as streaming, distributed, parallel, and MapReduce.
In Section~\ref{sec:apps} we survey real-world applications. Finally, in Section \ref{sec:conclusions} we discuss open problems for future investigation. 

