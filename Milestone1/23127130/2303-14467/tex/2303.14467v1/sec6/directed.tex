\subsection{Directed graphs}\label{subsec:directed}

Directed graphs are graphs in which for any edge it is also taken into account the directionality, meaning that any edge is defined by a source vertex $s$ and a target vertex $t$. Therefore, for two vertices $u$ and $v$, there can exist either the edge $(u,v)$ and $(v,u)$. In this context, the classical notion of density of a subgraph become meaningless, since it does not take into account the directionality. The first notion of density for directed graphs was proposed by Kannan and Vinay \cite{Kannan}: given two sets of nodes $S \subseteq V$ and $T \subseteq V$, we have $f(S,T) = \frac{|E(S,T)|}{\sqrt[]{|S|\cdot|T|}}$, where $E(S,T)$ is the set of edges having its source in $S$ and its target vertex in $T$. With this density notion it is given value to those pair of sets of nodes with a massive number of connections from one to the other; the square root at the denominator ensures that the output is not a single edge, that would take maximum value without it.

\begin{problem}[Directed Densest Subgraph Problem (DDS) \cite{Kannan}]\label{prb:dds}
Given a directed network $G=(V,E)$, the goal is to find two set of nodes $S$ and $T$ that maximize $f(S,T) = \frac{|E(S,T)|}{\sqrt[]{|S|\cdot|T|}}$.
\end{problem}

Kannan and Vinay, besides the notion of density, proposed an $\bigO(\log n)$-approximation algorithm for DDS, by relating the objective function to the singular value of the adjacency matrix, and applying Monte Carlo algorithm for its computation. Charikar \cite{Charikar2000} obtained both an exact and a 2-approximation algorithm based on the solution of $\bigO(n^2)$ linear programs for all the possible values of the quantity $|S| \cdot |T|$. The latter resembles the greedy peeling for DSP, and runs in $\bigO(n+m)$, yielding to a global time complexity of $\bigO(n^2(n+m))$. For both algorithms it was observed that with $\bigO(\frac{\log n}{\epsilon})$ execution of linear programs, it is possible to output respectively a $(1+\epsilon)$- and  $(2+\epsilon)$-approximation.

Khuller and Saha \cite{Khuller2009Dense} gave the first max-flow based polynomial time algorithm for solving DDS; likewise for DSP, this algorithm requires $\bigO(\log n)$ executions of max flow instances, or $\bigO(1)$ executions of parametrized maximum flow problem. In the same work they claimed to have obtained a 2-approximation algorithm with time complexity $\bigO(n+m)$, that was recently disproved by Ma et al. \cite{ma2021directed} by a counter-example; in the same work the authors reported an alternative 2-approximation algorithm provided by Saha, whose time complexity is $\bigO(n(n+m))$.

Bahmani et al. \cite{bahmani} developed a $2(1 + \epsilon)$-approximation algorithm, that ends in $\bigO(\log_{1+\epsilon}n)$ passes over the input graph.

Sawlani and Wang \cite{sawlani2020dynamic} reduced DDS to an instance of HDSP, relying on the knowledge of the ratio between sizes of $|S|$ and $|T|$ for the optimal solution. Since this quantity is unknown, they prove that with $O(\log n/\epsilon)$ instances of a $(1 + \frac{\epsilon}{2})$-approximation algorithm for HDSP is guaranteed a $(1+\epsilon)$-approximation solution for DDS. Furthermore, in their work the authors introduced different proposals for densest subgraph algorithms in fully dynamic setting for undirected graphs and vertex-weighted undirected graphs; therefore, their reduction of DDS to HDSP led to the first algorithm for DDS in fully dynamic setting, that maintains a $(1+\epsilon)$-approximation solution with worst-case time $poly(\log n, \epsilon^-1)$ per update.

Chekuri, Quanrud and Torres \cite{Chekuri2022supermod} very recently followed this reduction to adapt their DSG flow-based algorithm for DDS, that output a $(1+\epsilon)$-approximation in $\tilde \bigO(\frac{m}{\epsilon^2})$.

Ma et al. \cite{ma2021directed} obtained different results in this context. Introducing the notion of $[x,y]$-core, the directed counterpart of the $k$-core notion for undirected graphs, they prove that the directed densest subgraph can be located through it with theoretical guarantees.
As a first direct consequence, these results enable the maximum-flow based exact algorithm to be executed only on a reduced version of the graph, composed by some $[x,y]$-cores, making it faster. Furthermore, they provide a divide-and-conquer strategy to carefully select the optimal value of $\frac{|S|}{|T|}$, that reduces the possible values from $n^2$ to $k$, with n$k << n^2$.
Furthermore, they prove that a particular instance of $[x,y]$-core is a 2-approximation of the optimal solution, thus propose an algorithm to find it that runs in $\bigO(\sqrt{m}(n+m))$.
They introduce the weighted DDS problem, and extend the aforementioned findings for this problem.
Finally, they design algorithms able to handle the edge insertion/deletion in a dynamic setting, both maintaining a 2-approximation solution in $\bigO(\sqrt{m}(n+m))$ for the worst-case, although they practically showed that their algorithms perform faster than the iterative execution of their 2-approximation algorithm for the static setting.

 Ma et al. \cite{ma2022convex} designed a new LP formulation for DDS, with a Frank-Wolfe based algorithm to optimize the associate dual. With this new LP, they were able to design a new algorithmic framework to reduce the number of LP instances to solve. More precisely, they provided a $(1+\epsilon)$-approximation to the optimal solution with $\bigO(\log_{1+\epsilon}n \cdot t_{FW})$, where $t_{fW}$ is the time-complexity for solving a single instance of their LP.

