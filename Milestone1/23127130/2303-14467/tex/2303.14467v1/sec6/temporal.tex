\subsection{Temporal networks}
Temporal networks are a special case of multi-layer networks, where the different layers corresponds to different time slices, so that the order of the layers is important. Another way of defining a temporal network is a graph, where each edge has associated a timestamp from a discrete temporal domain.\footnote{When there is a continuous temporal domain information, usually a discrete number of time-intervals are defined, such to create a finite collection of networks.}
More formally we can define a temporal network as a graph $G = (V, E, T)$ where $V$ denotes the set of vertices, $T = [0,1,..., t_{max}] \subset \mathbb{N}$ is the time domain, and $E \subseteq V \times V \times T$.
Therefore, mining the densest subgraph in such networks can be seen in various ways. Among the various, ideally the most common is that of finding a set of nodes that is a dense subgraph along a fixed time interval given in input.

Bodganov et al. \cite{bogdanov2011heavy} were the first to address a similar problem in temporal networks. Allowing the presence of negatively weighted edges, they defined the problem of finding the Heaviest Temporal Subgraph, i.e. a set of nodes and a time-interval maximizing the edge's weights summation, showing that it is \NP-hard even with edge weights restricted in $\{-1, +1\}$.
The same problem has been recently tackled again by Ma et al. \cite{ma2020temporal}, that proposed a new heuristic to solve it.


Another common formulation is the following: given a time-interval $i = [i_1, i_2]$ with $i_1, i_2 \in T$, consider the following static representation of the network $G = (V, E[i])$, where $E = \{(u,v) | (u,v,t) \in E \land t \in i\}$. This static representation allows to employ the classical density notion for DSP and all of its mathematical properties in its maximization, and letting the focus to be more oriented toward the optimal time-interval detection.

For instance, Rozenshtein, Tatti and Gionis \cite{rozenshtein2017dynamic} proposed to find the densest temporal subgraph in such representation of their temporal network, imposing a constraint on the maximum number of timestamp to be included in the time-interval, and another on the maximum length of the span of the time-interval.

Lately, Rozenshtein et al. \cite{Rozenshtein2019segmentation} introduced the $k$-Densest-Episodes problem: defined an episode as a pair $S = \{I,H\}$, where $I = [i_1, i_2]$ is the time-interval and $H \subseteq V$ is a set of nodes, find the $k$ episodes that are densest in $k$ disjoint time-intervals. It is easy to see that for $k=1$ the problem reduces to DSP executed over the whole static network representation of the temporal graph given in input.
The same problem has been recently tackled by Dondi and Hosseinzadeh \cite{Dondi2021}, that proposed an heuristic that runs in $\bigO{|T| + (k + \tau)\cdot t_{DSP}}$, where $\tau$ is the maximum number of iterations given in input for their heuristic, and $t_{DSP}$ is the required time for a single computation of DSP.

Angel et al.~\cite{Angel2013} tackled real-time story identification on Twitter via maintenance of densest subgraph in the fully dynamic setting.
Similarly, Bonchi et al. \cite{BonchiBGS16,BonchiBGS19} adopted anomalously dense subgraphs in temporal networks as a way to identify buzzing stories in social media. 
More specifically, to recognize
a story as buzzing, it needs to have high density in the interactions (or co-occurrences) temporal graph, among
all objects (terms or entities) therein and for all time instants in a temporal window.
Thus they define the density of a subgraph in a given time interval as the minimum degree, among all vertices of the subgraph and all timestamps of the temporal window.
They show that the problem of finding the densest subgraph, for a given time interval, can be solved exactly by the peeling algorithm to construct the core decomposition, and by returning the innermost core. Then they define the problem of extracting the top-$k$ disjoint subgraph, with bounded size, that maximize the global density. The problem is shown to be \NP-hard and heuristics solutions are proposed.
Chu et al. \cite{chu2019online} defined the Density Bursting Subgraph as a subgraph that accumulates its density at the fastest speed in a temporal network, according to their definition of the burstiness of a subgraph.

Zhu et al. \cite{Zhu2022} advanced the static representation model by weighting each edge according to the number of time-stamps in which an edge exist. Finally, they design the problem statement taking into account the classical DSP density divided by the number of time-stamps in the relative time-interval, and imposing a constraint on the minimum number of time-stamps to take into account. For this problem, Zhu et al. designed one exact algorithm and an heuristic.

The most recent contribution in this context is by Qin et al. \cite{qin2023periodic}, which defined a $\sigma$-periodic subgraph to be a subgraph whose occurrences are exactly $\sigma$ in the temporal graph, and the time difference between any pair of consequent occurrences is the same.
They proposed efficient strategies to prune the search space in the temporal graphs, in order to run the maximum flow algorithm in an instance of reduced size.

