\subsection{Hypergraphs}\label{subsec:hyper}

Hypergraphs are a generalization of the concept of graphs. An hypergraph consists of a set $V$ of vertices and a set $E$ of hyper-edges, that are composed by an arbitrary number of vertices (n.b., graphs are hypergraphs in which any hyper-edge has only 2 vertices). All of the other definitions for this particular instance trivially follow from the classical graph notions. We only formalize the notion of $r$-uniform hypergraph, that is an hypergraph in which for any $e \in E$ holds that $|e| = r$, and we define the rank of the hypergraph as $\max_{e \in E}|e|$. In this context, we can define the Densest Subhypergraph problem (DSH).

\begin{problem}[Densest Subhypergraph Problem (DSH) \cite{Huang-Kahng95}]\label{prb:dsh}

Given an hypergraph $G=(V,E)$, the goal is to find a set of nodes $S$ that maximizes $f(S) = \frac{|E(S)|}{|S|}$.

\end{problem}

Huang and Kahng \cite{Huang-Kahng95} formally introduced the problem, proposing a flow-based algorithm that solves the problem in polynomial time.

Miyauchi et al. \cite{Miyauchi+15} proposed two generalized problems for the allocation of advertising budget: the maximum general-thresholds coverage problem, for which the densest $k$-subhypergraph (i.e. the variant of DSH in which the output size is constrained to be equal to $k$,) falls, and the cost-effectiveness maximization problem, the one in which DSH falls. For the first model they propose two different greedy algorithms, while in the context of the last class of problems they designed an almost-linear time approximation algorithm.

Hu, Wu, and Hubert Chan \cite{hu2017dynamicsub} provided a generalization for any weighted hypergraph of the results of \cite{Tsourakakis15}, designing either a linear program and a flow-based algorithm. Furthermore, they are the first to address the Densest Subhypergraph maintenance in the dynamic setting, providing two algorithms that maintain a $r(1+\epsilon)$-approximation in the case where there are only edge insertions, and $r^2(1+\epsilon)$-approximation in the fully dynamic setting.

Chlamtac et al. \cite{Chlamtac+18} performed a theoretical analysis on densest $k$-subhypergraph problem, providing bounds over 3-uniform hypergraphs.

Corinzia et al.~\cite{corinzia2022statistical} consider the recovering of the planted densest $k$-subhypergraph in a $d$-uniform hypergraph, providing tight statistical bounds on recovering's quality and algorithmic bounds based on approximate message passing algorithms.

The already mentioned recent results from Chekuri, Quanrud and Torres \cite{Chekuri2022supermod} also generalize to hypergraphs, yielding to a $(1+\epsilon)$-approximation algorithm based on max flow.
Furthermore, a natural generalization to hypergraphs of the {\sc Greedy++} iterative greedy peeling algorithm for DSP (reviewed in Section~\ref{subsubsec:iterative_peeling}) provides a
$(1+\epsilon)$-approximation for DSH.
The number of required iteration is
$O\left(\frac{\Delta \log n}{\lambda^* \epsilon^2}\right)$
, where $\Delta=\max_{v \in V} |\{e : v \in e\}|$, and $\lambda^*$ is the optimal value of the problem.

The iterative algorithm proposed by Harb, Quanrud, and Chekuri~\cite{harb2022faster} also generalizes to hypergraphs; their algorithm provides an $\epsilon$-additive approximate solution for DSH in
$\bigO\left(\frac{\sqrt{p\,r\,\Delta}}{\epsilon} \right)$ iterations,
where each iteration requires $\bigO\left(p\, \log r\right)$ time and admits some level of parallelization. Here $\Delta=\max_{v \in V} |\{e : v \in e\}|$, $r$ refers to the rank of the hypergraph, and $p = \sum_{e \in E} |e|$.

Zhou et al. \cite{zhou2022extracting} generalized DSH, considering that in the final solution there might be hyperedges only partially included in the solution, that are not counted in the objective function but that might be relevant in any case for specific applications. Therefore they define a weighting scheme according to the number of vertices of any hyperedge included in the final solution, and the relative maximization problem based on it. For this problem they propose either an exact and a $r$-approximation algorithm, where $r$ is the rank of the hypergraph.

Finally, Bera et al. \cite{bera2022dynamicsub} very recently designed a new algorithm for the dynamic setting. They improved the approximation ratio to $(1+\epsilon)$, making it independent from the hypergraph rank, with a similar update time to that required in \cite{hu2017dynamicsub}. Furthermore, their algorithm applied also for weighted hypergraphs.



