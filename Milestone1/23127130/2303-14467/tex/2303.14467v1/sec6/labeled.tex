\subsection{Edge- and vertex-labeled graphs}\label{subsec:labeled}
An edge-labeled (respectively vertex-labeled) graph is a type of graph where each edge (respectively vertex) is assigned one or more specific labels or attributes. These types of graphs usually represent real-world scenarios where there is additional information associated with edges or vertices, which can be exploited for different tasks. Despite the simplicity of the input, little attention has been devoted to dense subgraph mining in presence of information brought by the labels.

The simplest example of an edge- and/or vertex-labeled graph is a graph where each edge and/or each vertex is associated with a scalar attribute, i.e. a weight. The problem of densest subgraph on weighted graphs was already studied by Goldberg in his seminal 1984 paper~\cite{goldberg1984finding}, and was recently dusted-off by Fazzone et al.~\cite{Fazzone2022}, that dubbed it as Heavy and Dense Subgraph Problem (HDSP).

\begin{problem}[Heavy and Dense Subgraph Problem (HDSP)]
\label{prb:HDSG}
Given an undirected graph $(G, V, E, w_V, w_E)$ without self-loops, where $w_V: V \rightarrow \mathbb{R^+}$ and $w_E: E \rightarrow \mathbb{R^+}$,
find $S^* \subseteq V$ such that
\[
S^* = \argmax_{S \subseteq V} \frac{e(S) + \sum_{s \in S} w_V(s) }{|S|},
\]
where $e(S) = \sum_{e \in E(S)} w_E(e)$.
\end{problem}

The HDSP problem is in $\mathbb{P}$, Goldberg in~\cite{goldberg1984finding} gave an exact polynomial-time algorithm based on a reduction to the $s$-$t$ MaxFlow / $s$-$t$ MinCut problem.
Recently, Fazzone et al.~\cite{Fazzone2022} studied the approximation guarantees of Charikar's greedy peeling algorithm (Section\ref{subsec:approx}) when applied to HDSP, then adapted the recent iterative greedy approach of Chekuri, Quanrud, and Torres~\cite{Chekuri2022supermod} and Boob et al.~\cite{boob2020flowless}, to HDSP problem.


Anagnostopoulos et al.~\cite{anagnostopoulos2020fair} tackled the algorithmic fairness issue for the densest subgraph problem by defining the fair densest subgraph problem (FDSP) as a constrained version of the original problem, taking into account a binary labelling of the graph vertices. % $l : V \rightarrow \{-1, +1\}$, where the labelling corresponds to an attribute associated with the vertex. % problem definition
According to this setting, the goal is to compute a set of vertices
$S \subseteq V$
of maximum density while ensuring that
$S$
contains an equal number of representatives of either label, guaranteeing that the binary-protected attribute is not disparately impacted.% inappx and appx results
 The authors proved that the FDSP problem is \NP-hard,
and also that approximating the densest fair subgraph with a polynomial time algorithm is at least as hard as Dam$k$S, for which no constant approximation algorithms are known (see Section~\ref{sec:size}).
More precisely, any $\alpha$-approximation (with $\alpha \geq 1$) to Dam$k$S is a $2\alpha$-approximation to the fair densest subgraph problem (FDSP). % combinatorial
 In the case in which the underlying graph is itself fair, the authors define a polynomial time $2$-approximation algorithm for the problem and prove the tightness of this approximation factor under the small set expansion hypothesis~\cite{raghavendra2010expansion}.
The authors also defined an algorithm, based on a spectral embedding, able to retrieve an approximate solution with theoretical guarantees in polynomial time for the case where the input graph is an expander that contains an almost-regular dense subgraph (see Section 2.1 in~\cite{anagnostopoulos2020fair} for more details).

Tsourakakis et al.~\cite{tsourakakis2019novel} considered to solve DSP in edge-(multi)labeled networks with exclusion queries: given a network $G = (V,E)$ in which any edge has associated a subset of labels in the label universe $L$, and an input set $l \subseteq L$, find the densest subgraph that contains only edges whose label is contained in $l$. To solve this problem, they resort to an heuristic based on the Densest Subgraph with Negative Weights problem (see \refsec{negative_weights}).

Rozenshtein et al.\cite{rozenshtein2020mining} studied the problem of finding a dense edge-induced subgraph in an edge-labeled graph whose edges are similar to each other based on a given similarity function of the labels. The authors model the problem setting with an objective function that is the sum of a density term and a similarity term, and design a Lagrangian relaxation optimization problem for its maximization, for which they propose an algorithm based on parametric min-cut, that requires $\bigO(|E|^3 \log |E|)$ time and $\bigO(|E|^2)$ space.
