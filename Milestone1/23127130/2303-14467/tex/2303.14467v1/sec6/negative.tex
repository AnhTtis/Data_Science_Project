\subsection{Negatively weighted graphs}
\label{sec:negative_weights}
The HDSP we reviewed in previous section, was defined on positive weights.
Recently some specific applications (e.g., networks derived from correlation matrices) are requiring to extract dense subgraphs from networks whose edges can be negatively weighted. When we allow negative weights, most of the algorithmic results for DSP no longer holds.

Cadena, Vullikanti and Aggarwal \cite{cadena2016dense} were the first to analyze the dense-structure mining in graphs with negative weights on edges. They defined, on top of the work of Tsourakakis et al. \cite{tsourakakis2013denser}, the Generalized Optimal Quasi Clique (GOQC) problem.

\begin{problem}[Generalized Optimal Quasi Clique (GOQC) \cite{cadena2016dense}]\label{prb:goqc}
Given a network $G=(V,E)$, a weight function $w(\cdot): E \rightarrow \mathbb{R}$ and a penalty function $\bar{\alpha}(\cdot): E \rightarrow \mathbb{R}$, the goal is to find a subset of nodes $S$ that maximizes $f_{\bar{\alpha}}(S) = \sum_{u,v \in S}w(u,v) - \bar{\alpha}(u,v)$.
\end{problem}

Problem~\ref{prb:goqc} aims to find a set of nodes that maximizes the summation of edges' weight, subject to a penalty term $\bar{\alpha}(u,v)$ for any edge that prevent the uncontrolled growth of the first term. The authors showed that the problem is \NP-hard to approximate within a factor of $\bigO(n^{\frac{1}{2}-\epsilon})$, and proposed an algorithm composed by the solving of an SDP refined by the local search algorithm of \cite{tsourakakis2013denser}.They prove that under specific conditions, the returned solution is a $\bigO(\log n)$-approximation.

In 2019, Tsourakakis et al. \cite{tsourakakis2019novel} tackled the DSP problem in presence of negative weights on edges.

\begin{problem}[Densest Subgraph with Negative Weights (DSNW) \cite{tsourakakis2019novel}]\label{prb:dsnw}
Given a network $G=(V,E)$, a weight function $w(\cdot): E \rightarrow \mathbb{R}$, the goal is to find a subset of nodes $S$ that maximizes $f(S) = \frac{\sum_{u,v \in S}w(u,v)}{|S|}$.
\end{problem}

The authors proved that Problem~\ref{prb:dsnw} is \NP-hard via a reduction based on the fact that the max-cut problem on graphs with possibly negative edges is \NP-hard too.
In order to efficiently solve this problem, they analyzed the performance of the classical greedy peeling for DSP, proving that for this problem it achieves an approximation of $\frac{\rho^*}{2}-\frac{\Delta}{2}$, where $\rho^*$ is the optimal value, and $\Delta$ is the absolute value of the largest negative degree (i.e. the summation of the negative weights for a single node).

Dense subgraph discovery on a negatively weighted graph turns out to be the key component of extractive contrastive subgraphs, as done in \cite{yang2018mining,lanciano2020contrast} (discussed later in Section \ref{subsec:multilayer}).
