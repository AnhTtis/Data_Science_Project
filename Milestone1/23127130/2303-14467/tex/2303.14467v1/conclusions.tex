
The Densest Subgraph Problem is a classic and fundamental problem in graph theory that has received a great deal of attention. In the last couple of years we have witnessed a renewed interest in the problem due to its relevance in emerging applications, which has led to several algorithmic breakthroughs. This revival of interest motivates the present survey, which offers a comprehensive overview of the Densest Subgraph Problem and its many variants, covering a long and rich literature spanning over five decades.

We have summarized the different techniques and algorithms used to solve the problem in its classical setup, including exact and approximation algorithms, and presented a detailed collection of natural variants that have emerged over the years, either with respect to the definition of the objective function, the introduction of different constraints or the typology of the input graph.
Derived from the classical definition of DSP, these variants possess various algorithmic properties that are advantageous, and they also share the elegant and intuitive building foundations of the original problem.
Our analysis has identified the strengths and weaknesses of various approaches, providing a valuable resource for researchers seeking to solve related problems to the Densest Subgraph.
We hope that our survey will inspire new and innovative applications of the Densest Subgraph Problem, as well as serve as a useful reference for researchers in the field. While we acknowledge that gaps may exist in our coverage, we believe that the comprehensive nature of our review will be a valuable asset for future research in this area.


\spara{Open Problems.} Recent advancements have significantly enhanced the efficiency of Densest Subgraph extraction in its classical form, bringing the problem close to being fully resolved. However, several important open problems still require attention, and it is crucial that the scientific community continues to work towards addressing them.

An essential milestone for future research in Densest Subgraph mining is to establish the exact convergence of the solution provided by the iterative peeling method introduced in \cite{boob2020flowless} (Section \ref{subsubsec:iterative_peeling}). Defining this convergence would confirm the existence of a highly efficient method with strong guarantees for computing Densest Subgraphs in any existing graph.
The iterative peeling method has already shown tremendous promise in delivering excellent results and fast computation times. However, a clear understanding of its exact convergence properties is necessary to assess its performance and potential limitations fully.

There is still room for enhancing constrained versions of DSP, including D$k$S and Dal$k$S, which currently have approximation ratios of $O(n^{1/4+\epsilon})$ and $2$, respectively (Section \ref{sec:cons}). Improving these ratios would represent a significant advancement in the field, making it possible to solve these complex problems more accurately and efficiently.

Distance-based generalization of degree opens the door to many interesting problems. The approximation of the distance-$h$ densest subgraph (Section \ref{subsec:numerator}) by means of distance-generalized core decomposition, is only a first step which leaves wide open space for improvements.

While significant advancements have been made in negatively weighted graphs (Section \ref{sec:negative_weights}), the current approximation ratio is unsatisfactory as it depends on the input graphs. Improving this ratio would greatly benefit various applications that require this type of input, such as graphs derived from correlation matrices, and related problems like exclusion queries \cite{tsourakakis2019novel}.

Ensuring fairness is a critical aspect that researchers are increasingly addressing, as it is fundamental to achieving equitable outcomes and combating algorithmic bias. However, the current results in this area are limited to a single work \cite{anagnostopoulos2020fair}, underscoring the urgent need for the research community to focus more attention on this issue and contribute to improving upon these findings.

The literature on graphs defined on metric spaces has received relatively little attention (Section \ref{subsec:metricscpaces}), which leaves room for further improvement. These types of graphs are important in applications that rely on spatial data, such as image and video processing, transportation networks, and wireless sensor networks. Algorithms designed to work with these graphs can be crucial for advancing research in these fields.
