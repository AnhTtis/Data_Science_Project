

\section{Future Roadmap: \\Revolutionizing FPGA Design Flow}\label{sec:results}

DL has been increasingly applied in various stages of the FPGA design flow to enhance performance and efficiency. One of the ideal applications of DL is in \textbf{generating bitstreams directly from high-level descriptions}, eliminating the need for traditional synthesis and place-and-route tools. With this approach, designers can quickly evaluate and optimize designs at the high-level, significantly reducing design iterations and accelerating the overall design process. This approach has the potential to revolutionize the FPGA design flow, making it faster and more accessible to designers with a wide range of expertise. By leveraging the power of DL in the FPGA CAD flow, designers can unlock new possibilities and achieve higher performance and lower power consumption for their designs.

\subsection{Potential DL Models}
There are several types of DL models that could be suitable for this future roadmap of generating bitstreams directly from high-level descriptions. One approach is to use DL models, such as convolutional neural networks (CNNs), recurrent neural networks (RNNs), or transformers, which have shown success in natural language processing and image recognition tasks. These models can be adapted to process high-level descriptions of FPGA designs and generate corresponding bitstreams. 
Another approach is to use reinforcement learning (RL) models, which learn from trial-and-error interactions with the FPGA design environment to generate optimal bitstreams. RL models can potentially adapt to different design objectives and constraints, such as performance, power consumption, and area, and optimize the design accordingly. 
Generating bitstreams directly from high-level descriptions is a challenging task, as it requires a high level of abstraction. This is where large DL models can be useful. Large DL models are capable of learning complex patterns and relationships from vast amounts of data, making them well-suited for generating bitstreams directly from high-level descriptions. 
In addition, Federated Learning (FL) is an approach for collaborative learning across multiple devices that can be applied to the process of generating FPGA bitstreams directly from high-level descriptions. In this approach, individual devices generate candidate bitstreams and share their results with a central server to generate a final optimized FPGA bitstream. However, implementing FL for this application poses some challenges, including ensuring the privacy and security of the shared data between the devices and the central server.

\subsection{Challenges}
Developing a DL-based FPGA CAD flow for generating bitstreams directly from high-level descriptions presents several challenges that need to be addressed. One significant challenge is \textit{the need for large amounts of high-quality training data} to effectively train the machine learning models. The quality and diversity of the training data can directly impact the performance and accuracy of the models. \textit{Data availability} is also a significant challenge in developing DL models for FPGA CAD flow, as collecting and curating this data can be a time-consuming and expensive process.
Ensuring the \textit{interpretability} of the DL models is another concern, as designers need to understand how the models generate the bitstreams and ensure they meet the design objectives and constraints. Finally, there is a need for \textit{collaboration and knowledge-sharing among researchers and industry practitioners} to develop and evaluate the effectiveness DL-based FPGA CAD flows. Addressing these challenges will be crucial in realizing the potential benefits of this future roadmap for FPGA design. 










