\section{Conclusion}\label{sec:concl}
Artificial intelligence in FPGA EDA design is expected to be the next major trend, with major tool vendors and semiconductor companies already making efforts to utilize this technology. The integration of deep machine learning (DL) techniques at various stages of the FPGA design flow has the potential to significantly enhance the efficiency, performance, and accessibility of FPGA design.
Researchers have explored the use of DL models to improve the FPGA design flow, from HDL elaboration to bitstream generation. These models can assist in optimizing design parameters, and improving the accuracy and speed of placement and routing.

% \cite{hu2022machine}
% \cite{martin2021machine}
% \cite{maarouf2018machine}

% \cite{al-hyari2019novel}

% \cite{wang2022learning}
% \cite{cong1994flowmap}
% \cite{wang2022learning}
% \cite{roorda2022fpga}
% \cite{Yanghua2016}
% \cite{Ustun2020}
% \cite{Farooq2021}
% \cite{Martin2021}
% \cite{Maarouf2018}
% \cite{Pui2017}
% \cite{farooq2021efficient}