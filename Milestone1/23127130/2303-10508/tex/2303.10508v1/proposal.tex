\section{Traditional FPGA CAD Flow}
\label{sec:leap}
The main steps of the FPGA CAD flow include:

\begin{itemize}
    \item \textit{Design Entry:} This step involves capturing the design in a HDL such as Verilog or VHDL, or in a high-level representation like C.
    
    \item \textit{Synthesis and Optimization:} In this step, the HDL design is converted into a netlist of gates and registers. The synthesis tool analyzes the design and generates a logic circuit that performs the same function. The synthesized circuit is optimized to improve its performance, reduce its area, and minimize its power consumption. Optimization can be performed at different levels of abstraction, such as gate-level optimization or high-level synthesis.

    \item \textit{Packing and Placement:} The packing step involves grouping the placed logic elements into more compact groups called logic clusters. These logic clusters are then mapped onto physical regions of the FPGA called logic blocks or configurable logic blocks (CLBs).  Placement is the process of assigning the logical elements of the design to physical locations on the FPGA. The placement tool ensures that the placement of the elements satisfies the timing constraints of the design.

    \item \textit{Routing:} Routing is the process of establishing connections between the logical elements on the FPGA by using routing resources such as switching boxes and routing channels. The routing tool determines the best path for the connections and ensures that they meet the timing requirements of the design.
    
    \item \textit{Bitstream Generation:} The final step is to generate a bitstream that can be programmed onto the FPGA. 

\end{itemize}

\subsection{Enhancing FPGA CAD Flow with DL Models: Required Steps}
DL uses neural networks with multiple layers to learn patterns and relationships within data, can be utilized in FPGA CAD flow. DL algorithms can identify complex patterns and relationships within vast amounts of data, potentially resulting in better-performing designs and reduced design cycle times. 
To enhance the FPGA CAD flow using DL models, the following steps are required:

\begin{itemize}
    \item \textbf{Dataset creation:} A large dataset of FPGA designs and their corresponding placement and routing results must be created. This dataset can be used to train and validate the DL models.
    \item \textbf{Feature extraction:} Relevant features must be extracted from the FPGA designs, such as the location of logic blocks, routing resources, timing constraints, and power constraints.
    \item \textbf{Model selection:} Various DL models can be evaluated and compared on the dataset to identify the best-performing model for the FPGA CAD flow.
    \item \textbf{Model training:} The selected DL model must be trained on the dataset using appropriate training and validation techniques, such as cross-validation or early stopping.
    \item \textbf{Model integration:} The trained DL model must be integrated into the FPGA CAD flow and evaluated on a test dataset to assess its accuracy and speed.
\end{itemize}

The integration of DL models into the FPGA CAD flow has the potential to significantly improve the efficiency and quality of FPGA design, making it an exciting area of research for the FPGA community.