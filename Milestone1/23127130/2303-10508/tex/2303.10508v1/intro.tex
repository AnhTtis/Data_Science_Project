\section{Introduction}\label{sec:intro}
%*Aging

Field-Programmable Gate Arrays (FPGAs) have become an integral component of modern digital systems, including healthcare devices, autonomous vehicles, and datacenters. The design process of these systems is a complex and time-consuming task that involves a significant investment of time and resources. Computer-Aided Design (CAD) tools play a crucial role in ensuring the quality and efficiency of the resulting FPGA-based systems.

CAD tools for FPGA design, including high-level synthesis, logic synthesis, placement, and routing algorithms, are used to convert a high-level hardware description into a bitstream representation. The quality of these algorithms significantly impacts the performance and power of the resulting digital systems. However, designing high-quality CAD tools for FPGA design is challenging due to the complexity of the problem and the large number of design variables.

Deep Machine learning (ML) techniques have shown great potential to enhance the efficiency and effectiveness of FPGA CAD algorithms. ML algorithms can optimize design parameters, predict design outcomes, and accelerate the design process. The integration of DL techniques into FPGA CAD flow design has the potential to revolutionize the way FPGA-based systems are designed and implemented.

This paper offers an overview of the latest deep machine learning (DL)-oriented efforts in various FPGA CAD design steps, including high-level and logic synthesis, placement, and routing. The focus is on machine learning-based CAD tasks, and we identify crucial research areas that require more attention in CAD design. Specifically, we emphasize the need for developing open-source benchmarks optimized for an end-to-end machine learning experience. As a result, this paper serves as a valuable resource for researchers and industry professionals interested in comprehending the benefits and challenges of integrating machine learning in FPGA CAD flow design.