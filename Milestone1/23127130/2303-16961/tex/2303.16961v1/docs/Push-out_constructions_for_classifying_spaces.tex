\section{Preliminaries on the Mapping Class Group}

\subsection{Definition and notation}
\begin{definition}
Let $S$ be a compact surface. We define the mapping class group $\mcg(S)$ as the isotopy class of orientation  preserving difeomorphisms of $S$  that restrict to the identity on the boundary $\partial S$.
\end{definition}


\subsection{Cutting and capping homomorphisms}

\subsection{Reduction systems for abelian subgroups}



\begin{definition}
An essential curve in $S$ is a simple closed curve of $S$  that is not homotopic to a point, a puncture  or  a boundary component of $S$.   
\end{definition}

\begin{definition}(Complex of curves)
Let $S$ be  a compact, orientable surface. The complex of curves $C(S)$ is the abstract simplex that has as vertices $V(S)$ the set of essential curves, and has a $k$-simplex for each  $(k+1)$-tuple of vertices, where each pair of corresponding isotopy classes
have disjoint representants.
\end{definition}

\begin{definition}
A subgroup $H$ of $\mcg(S)$ is reducible if there  is a nonempty simplex $\sigma$ of the complex of curves $C(S)$ such that $h\sigma=\sigma$ for all $h$ in $H$. In otherwise $H$ is irreducible. We call $\sigma$ a reduction system for $H$.  \end{definition}

\textcolor{red}{Falta definir $\mcg(S)[m]$.}

\begin{definition}(Canonical reduction system of a subgroup of $\mcg(S)$)
Let $H$ be a subgroup of $\mcg(S)$ consisting of pure elements. An isotopy class $\alpha \in V(S)$ is called an essential reduction class for $H$ if the following two conditions are satisfied: \begin{enumerate}[a)]
    \item  $f(\alpha)=\alpha$ for all $f\in H$;
    \item If $\beta\in V(S)$ and $i(\alpha,\beta)\neq 0$, then $g(\beta)\neq \beta$ for some $g\in H$.
\end{enumerate} 
We denote the set of essential reduction classes for $H$ by $\sigma(H)$. We call $\sigma(H)$ a cononical reduction system for $H$. 
Let  $K$ be  an arbitrary subgroup of $\mcg(S)$  and $L$ a normal subgroup of $K$ that has finite index in $K$ and consist of pure elements. We define a canonical reduction system for $K$ as $\sigma(K)=\sigma(L)$. This definition not depend of the subgroup $L$ see \cite[Section 7.4]{Ivanov:subgroups}
\end{definition}

\textcolor{red}{Hay que mencionar que después de cortar $S$ a través del sistema canónico de reducción, hay componentes donde $H$ actúa trivialmente y otras donde actúa reduciblemente. Las acciones reducibles a su vez, pueden ser finitas o contener un elemento pseudoAnosov. Finalmente, pasando a un subgroup de índice finito de $H$, podemos suponer que la acción en las componentes es trivial o cíclica infinita generada por un pseudoAnosov.}

\begin{lemma}\label{lem:H:has:reduction:system}
 Let $m\ge 3$. Let $H$ be  a free abelian  subgroup of $\mcg(S)[m]$ of rank  at least $2$. Then $H$ is reducible and has a nonempty canonical reduction system.
\end{lemma}
\begin{proof}
Since $H$ is free-torsion group, we have by Nielsen-Thurston classification of elements of $\mcg(S)$ that every element of $H$ either reducible or pseudo-Anosov. But by\textcolor{red}{reference pending} the centralizer of a pseudo-Anosov element is virtually cyclic, since $H$ is a free abelian of rank at least 2 we have that the centralizer of any element of $H$ is not virtually cyclic.  It follow that $H$ has only reducible elements. By before and  \cite[Corollary 7.14]{Ivanov:subgroups} $H$ is reducible, now by \cite[Corollary 7.17]{Ivanov:subgroups} $H$ has a nonempty cononical reduction system  $\sigma=\{\alpha_1, \cdots, \alpha_n\}$.
\end{proof}
 


\section{Geometric dimensions for families}

\subsection{Definitions}

\subsection{Some useful tools}

\begin{proposition}\cite[Lemma 2.4]{tomasz}\label{lemma:key:1}
 Let $G$ be a countable group and $\calF$ a family of subgroups of $G$ such that any $H\in \calF$ is contained is some finitely generated subgroup of $G$. Suppose that there is an integer $d>0$ such that for every finitely generated subgroup $K$ of $G$ we have $\gd_{\calF\cap K}K\leq d$. Then $\gd_{\calF}G\leq d+1$.
 \end{proposition}
 \begin{proposition}\cite[Proposition 5.1 (i)]{LW12}\label{pro:key:2}
 Let $G$ be a group and $\calF$ and $\calG$ be two families of subgroups such that $\calF\subset \calG$. Suppose for every $H\in \calG$ we have $\gd_{\calF\cap H}H\leq d$. Then $\gd_{\calF}G\leq\gd_{\calG}G+d$.
 \end{proposition}
 
 
 
\begin{proposition}\cite[Lemma 5.7]{Nucinkis:Petrosyan}\label{vcd:s.e.s:1}
Let $1\to \Z^n\to G\to Q\to 1$ be an extension of groups where $Q$ is finitely generated with $\vcd Q = K < \infty$. Then $\vcd G = n+k$. 
\end{proposition}
 
 
  
\begin{proposition}\cite[Corollary 5.8]{Nucinkis:Petrosyan}\label{pro:key:3}
Let $1\to \Z^n\to G\to Q\to 1$ be a extension of groups where $Q$ has a cocompact model for $\underline{E}Q$ of dimension $\vcd Q =k< \infty$. Then $G$ has a cocompact model for $\underline{E}G$ of dimension $\underline{\gd}G=\vcd G=n+k$. 
\end{proposition}

\subsection{Some results on the Mapping Class Group}

 \section{Push-out constructions for classifying spaces}
 
 \subsection{The Lück and Weiermann construction}
 \textcolor{red}{Tal vez convenga enunciar directamente Luck-Weiermann para nuestro caso, es decir, con la relación de conmensuración definida en $\calF_n-\calF_{n-1}$. De esta manera $N_G[H]$ siempre denotará al conmensurador de $H$ sin ambiguedad.}
 
 \begin{definition}
 Let $\calF\subset\calG$ be two families of subgroups of $G$. Let $\sim$ be  an equivalence relation in $\calG-\calF$. We say that $\sim$ is strong  if the following it satisfied 
 
\begin{enumerate}[a)]
    \item If  $H, K \in \calG-\calF$ with $H\subseteq K$, then $H\sim K$;
    \item If $H, K \in \calG-\calF$ and $g\in G$, then $H\sim K$ if and only if $gHg^{-1} \sim gKg^{-1}$.
\end{enumerate}
 \end{definition}
 
 \textcolor{red}{Falta definir $N_G[H]$. Hay que darle una revisada a todas estas definiciones.}
 
Let $\sim$ be  a strong equivalence in $\calG-\calF$. Let $[H]$ be a class of $(\calG-\calF)/\sim$, we define $\calG[H]=\{K\subseteq N_G[H]| K\in  \calG-\calF, [K]=[H]\}\cup (\calF\cap N_G[H])$.
 \begin{theorem}\cite[Theorem 2.3]{LW12}  Let $\calF\subset\calG$ be two families of subgroups of $G$. Let $\sim$ be  a strong equivalence  in $\calG-\calF$. Let $I$ a complete representatives class of conjugation in $(\calG-\calF)/\sim$. Choose arbitrary $N_G[H]$-CW-models for $E_{\mathcal{F}\cap N_{G}[H]}N_{G}[H]$ and $E_{ \calG[H]}N_{G}[H]$, and an arbitrary model for  $E_{\mathcal{F}}G$. Consider the  following $G$-push-out 
  \[
\begin{tikzpicture}
  \matrix (m) [matrix of math nodes,row sep=3em,column sep=4em,minimum width=2em]
  {
     \displaystyle\bigsqcup_{[H]\in I} G\times_{N_{G}[H]}E_{\mathcal{F}\cap N_{G}[H]}N_{G}[H] & E_{\mathcal{F}}G \\
      \displaystyle\bigsqcup_{[H]\in I} G\times_{N_{G}[H]}E_{ \calG[H]}N_{G}[H] & X \\};
  \path[-stealth]
    (m-1-1) edge node [left] {$\displaystyle\bigsqcup_{[H]\in I}id_{G}\times_{N_G}f_{[H]}$} (m-2-1) (m-1-1.east|-m-1-2) edge  node [above] {$i$} (m-1-2)
    (m-2-1.east|-m-2-2) edge node [below] {} (m-2-2)
    (m-1-2) edge node [right] {} (m-2-2);
\end{tikzpicture}
\]
such that $f_{[H]}$ is cellular $N_{G}[H]$-map for every $[H]\in I$ and $i$ is an inclusion of $G$-CW-complexes, or such that every map $f_{[H]}$ is an inclusion of $N_{G}[H]$-CW-complexes for every $[H]\in I$ and $i$ is a cellular $G$-map. Then $X$ is a model for $E_{G}G$.
 \end{theorem}
 
 \subsection{The push-out of a union of families}
 \textcolor{red}{Falta enunciar el pushout de una unión de familias.}
 
 
 \section{Commensurators of virtually abelian subgroups}
 
 \textcolor{red}{Hablar del contexto de estos resultados. Es decir, lo que se sabe del caso cíclico infinito de los trabajos de Juan-Pineda--Trujillo y Nucinkis--Petrosyan.}
 
 The following definition is borrowed from \cite[Definition~2.8]{tomasz}.
 
 \begin{definition}(Condition C) Let $n$ be a natural number.
We say that a group $G$ satisfies condition $\mathrm{C}_n$ if for every $H\in \calF_n$, and for all $K\subset N_{G}[H]$ finitely generated, there is $H^{\prime}\in \calF_n$ with $H^{\prime}$ commensurable with $H$ such that $\langle H, K\rangle \subset N_{G}(H^{\prime})$. Whenever $G$ satisfies condition  $\mathrm{C}_n$, for all $n$, we say $G$ satisfies condition $\mathrm{C}$. 
\end{definition}


\begin{proposition}\label{condition:c} Let $S$ be a closed surface possibly with punctures.
The mapping class group $\mcg(S)$ satisfies condition $C$.
\end{proposition}

\begin{proof}
Let $H\in \calF_n$ and let $K\subset N_{G}[H]$ finitely generated, we can assume that $H$ is free abelian by replacing $H$ with a subgroup of finite index if necessary. In   \cite[Theorem 1.1.]{MR2271769} it is proved that every solvable subgroup of $\mcg(S)$ is separable, hence $H$ is separable in $\mcg(S)$. Now by \cite[Corollary 9]{MR4138929} there is a finite index subgroup $H^{\prime}$ of $H$ that is normal in $\langle H, K\rangle$. 
\end{proof}


 \begin{proposition}\label{Normalizer:commensurator:free:abelian:subgroup}
Let $m\ge 3$. Let $H$ be  a free abelian  subgroup of $\mcg(S)[m]$ of rank  at least $2$, and let be its canonical reduction system $\sigma=\{\alpha_1, \cdots, \alpha_n\}$. Then the following statements hold.

\begin{enumerate}[(a)]
    \item There is a short exact sequence 
$$1\to \Z^k \to  N_{\mcg(S)}(H)\xrightarrow[]{\rho_{\sigma}} Q\to 1$$
where $k\leq n$, $Q\subseteq \mcg(\displaystyle\bigsqcup_{i=1}^{a}S_i)\times A$, and $A$ is virtually abelian group of rank almost $l-a$ \textcolor{red}{hay que definir $l$}.

\item $N_{\mcg(S)}[H]$ is finitely generated.
\end{enumerate}
\end{proposition}

\begin{proof} We star with the proof of the item $(a)$. We claim that $N_{\mcg(S)}[H]\subseteq \mcg(S)_{\sigma}$. Let $g\in N_{\mcg(S)}[H]$, this means that $gHg^{-1}$ is commensurable with $H$. Since   the canonical reduction system of a subgroup of $\mcg(S)$ is invariant under subgroups of finite index  we have  $\sigma(H)=\sigma(gHg^{-1}\cap H)=\sigma(gHg^{-1})$. On the other hand, we have by \cite[p. 64]{Ivanov:subgroups} that  $\sigma(gHg^{-1})=g\sigma(H)$, it follows that $g\sigma(H)=\sigma(H)$, this finishes the proof of the claim.

We consider the cutting homomorphism $\rho_{\sigma}\colon \mcg(S)_{\sigma}\to \mcg(\hat{S}_{\sigma})$ that has $\ker(\rho_{\sigma})=\Z^n$. Note that $N_{\mcg(S)}(H)\subseteq N_{\mcg(S)}[H]$ then, using the claim in the previous paragraph, we can restrict $\rho_\sigma$ to obtain the following short exact sequence 
$$1\to \Z^n \cap N_{\mcg(S)}(H) \to  N_{\mcg(S)}(H)\xrightarrow[]{\rho_{\sigma}} Q\to 1$$
where $Q=\rho_{\sigma}(N_{\mcg(S)}(H))$. \textcolor{red}{Passing to a finite index subgroup if its necessary we have by \cite[Theorem 7.11]{Ivanov:subgroups} that $\rho_{\sigma}(H)|_{\mcg(S_i)}$ is trivial or irreducible (tal vez pondremos esto como hipótesis).} By re-indexing if needed we can write $\hat{S}_{\sigma}=\bigsqcup_{i=1}^{a}S_i \bigsqcup_{j=a+1}^{l}S_i$ where $\rho_{\sigma}(H)|_{\mcg(S_i)}$ is trivial for all $1\le i\le a$ and $\rho_{\sigma}(H)|_{\mcg(S_i)}$ is irreducible  for all $a+1\le i\le l$.

We claim that $N_{\mcg(\hat{S}_{\sigma})}(\rho_{\sigma}(H))\subseteq \mcg(\bigsqcup_{i=1}^{a}S_i)\times \mcg(\bigsqcup_{j=a+1}^{l}S_i)$. Suppose that this is not the case, then there exists $g\in N_{\mcg(\hat{S}_{\sigma})}(\rho_{\sigma}(H))$ that sends $S_i$ for some $1\leq i \leq a$ to $S_j$ for some $a+1\leq j \leq l$. Then for all $x\in S_i$, and for all $h\in H$, we have $\rho_{\sigma}(h)g(x)=g\rho_{\sigma}(x)=g(x)$. Therefore $\rho_{\sigma}(H)$ act trivially in $S_j$ and this is a contradiction.

Note that $\rho_{\sigma}(H)\subseteq \{ 1 \}\times  \mcg(\bigsqcup_{j=a+1}^{l}S_i)$ it follows that 
\[N_{\mcg(\hat{S}_{\sigma})}(\rho_{\sigma}(H))\subseteq \mcg(\bigsqcup_{i=1}^{a}S_i)\times N_{\mcg(\bigsqcup_{j=a+1}^{l}S_i)}(\rho_{\sigma}(H)).\] We proof that $N_{\mcg(\bigsqcup_{j=a+1}^{l}S_i)}(\rho_{\sigma}(H))$ is virtually abelian, for this is sufficient to show that $N_{\mcg(\bigsqcup_{j=a+1}^{l}S_i)^{0}}(\rho_{\sigma}(H))$ is virtually abelian. Note that we have projections $\mcg(\bigsqcup_{j=a+1}^{l}S_i)^{0}= \prod_{i=a+1}^{l}\mcg(S_i) \xrightarrow[]{\varphi_{i}} \mcg(S_i)$, we have the following diagram 

 \[
\begin{tikzpicture}
  \matrix (m) [matrix of math nodes,row sep=3em,column sep=4em,minimum width=2em]
  {
     N_{\mcg(\bigsqcup_{j=a+1}^{l}S_i)^{0}}(\rho_{\sigma}(H)) & \prod_{i=a+1}^{l}\mcg(S_i) \\
      N_{\mcg(S_i)}(\varphi_i(\rho_{\sigma}(H)))& \mcg(S_i) \\};
  \path[-stealth]
    (m-1-1) edge node [left] {} (m-2-1) (m-1-1.east|-m-1-2) edge  node [above] {$i$} (m-1-2)
    (m-2-1.east|-m-2-2) edge node [below] {$i$} (m-2-2)
    (m-1-2) edge node [right] {} (m-2-2);
\end{tikzpicture}
\]
Note that $ V_i:=N_{\mcg(S_i)}(\varphi_i(\rho_{\sigma}(H)))$ is virtually cyclic and $N_{\mcg(\bigsqcup_{j=a+1}^{l}S_i)^{0}}(\rho_{\sigma}(H)) \subseteq \prod_{i=a+1}^{l}V_i$, it follows that $N_{\mcg(\bigsqcup_{j=a+1}^{l}S_i)^{0}}(\rho_{\sigma}(H))$ is virtually abelian.\\

Now we proof item $(b)$. In the item $(a)$ we showed that $N_{\mcg(S)}[H]\subseteq \mcg(S)_{\sigma}$, then also we have a short exact sequence 
 $$1\to \Z^n \cap N_{\mcg(S)}[H] \to  N_{\mcg(S)}[H]\xrightarrow[]{\rho_{\sigma}} K\to 1$$
 where $K=\rho_{\sigma}(N_{\mcg(S)}[H])$. It is easy to proof that  $\rho_{\sigma}(N_{\mcg(S)}[H])\subseteq N_{\mcg(\hat{S}_\sigma)}([\rho_{\sigma}(H)])$.  Using the same decomposition of  $\hat{S}_{\sigma}=\bigsqcup_{i=1}^{a}S_i \bigsqcup_{j=a+1}^{l}S_i$ as in the item $(a)$ we can proof that $N_{\mcg(\hat{S}_\sigma)}([\rho_{\sigma}(H)])\subseteq \mcg(\bigsqcup_{i=1}^{a}S_i)\times \mcg(\bigsqcup_{j=a+1}^{l}S_i)$, since $\rho_{\sigma}(H)\subseteq \mcg(\bigsqcup_{j=a+1}^{l}S_i)$  it follows that  $N_{\mcg(\hat{S}_\sigma)}([\rho_{\sigma}(H)])=\mcg(\bigsqcup_{i=1}^{a}S_i)\times N_{\mcg(\bigsqcup_{j=a+1}^{l}S_i)}[\rho_{\sigma}(H)]$. The little variation of the  argument  given in the item $(a)$ show that  $N_{\mcg(\bigsqcup_{j=a+1}^{l}S_i)}[\rho_{\sigma}(H)]$ is virtually abelian. Hence $N_{\mcg(\hat{S}_\sigma)}([\rho_{\sigma}(H)])=\mcg(\bigsqcup_{i=1}^{a}S_i)\times N_{\mcg(\bigsqcup_{j=a+1}^{l}S_i)}[\rho_{\sigma}(H)]$ is finite generated.
 \end{proof}
 
 \begin{proposition}\label{norrmalizer:commensurator:equal}
Let $S$ be a connected, closed, oriented  and possibly with a finite number of punctures. Let $H$ be  a free abelian  subgroup of $\mcg(S)$ of rank $k\ge 2$. Then there is a subgroup $L\leq H$ of finite index such that $N_{\mcg(S)}[L]=N_{\mcg(S)}(L)$.
\end{proposition}
\begin{proof}
  By \cref{Normalizer:commensurator:free:abelian:subgroup} $N_{\mcg(S)}[H]$ is finite generated, then by \cref{condition:c} there is a subgroup $L\leq H$ of finite index such that $\langle H, N_{\mcg(S)}[H] \rangle\subseteq N_{\mcg(S)}(L)$, but $\langle H, N_{\mcg(S)}[H] \rangle=N_{\mcg(S)}[H]= N_{\mcg(S)}[L]$, hence $N_{\mcg(S)}[L]\subseteq N_{\mcg(S)}(L)$. The another contention $N_{\mcg(S)}(L) \subseteq N_{\mcg(S)}[L]$ is always true, this finish the proof. 
\end{proof}

 