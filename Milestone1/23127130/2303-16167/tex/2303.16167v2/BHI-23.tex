\documentclass[leqno]{amsart}
%\documentclass[11pt,reqno]{amsproc}
\usepackage[margin=1in]{geometry}
\usepackage{amsmath, amsthm, amssymb}
\usepackage{hyperref}
\usepackage{multicol}
%\usepackage[latin1]{inputenc}
\hypersetup{colorlinks=true, pdfstartview=FitV, linkcolor=blue, citecolor=blue, urlcolor=blue}
%\usepackage[abbrev,lite,nobysame]{amsrefs}
%\usepackage{showkeys} 
\usepackage{times}
\usepackage{comment}
\usepackage[usenames,dvipsnames]{color}
\usepackage{bbm}
\usepackage{subcaption}
\usepackage{mathtools}
\usepackage{pdfpages}
\usepackage{bm}
\usepackage{verbatim}
\usepackage{mathabx}
\usepackage{geometry}
\usepackage{enumerate} % Fancy enumeration lists 
\geometry{a4paper} % heightrounded, %\usepackage{cancel}

%\usepackage{mathpazo}
%%%%%%%%%%%%%%%%%%%%%%%%%%%%%%%%%%%%%%%%%%%%%%%%

%%%%%%%%%%%%%%%%% theorems %%%%%%%%%%%%%%%%%%%%%
%\theoremstyle{definition}
%\newtheorem*{rmk}{Remark}%[section]
%\theoremstyle{plain}
\newtheorem{thm}{Theorem}[section]%[chapter]%[section]%!!the counter really doesn't follow the chapters!!
\newtheorem{Prop}[thm]{Proposition}
\newtheorem{lem}[thm]{Lemma}
\newtheorem{Cor}[thm]{Corollary}
\newtheorem{defi}[thm]{Definition}%[section]
 \newtheorem{Thm}{Theorem}[section]
 \newtheorem{Rmk}[thm]{Remark}
% \newtheorem{Prop}[thm]{Proposition}
% \newtheorem{Cor}[thm]{Corollary}
 \newtheorem{Lem}[thm]{Lemma}
  \newtheorem{Def}[thm]{Definition}
\newtheorem{boot}[thm]{Bootstrap hypothesis}
   
% \newtheorem{Lemma}{Lemma}
% \newtheorem{Def}[thm]{Definition}
%%%%%%%%%%%%%%%%%%%%%%%%%%%%%%%%%%%%%%%%%%%%%%%%
%\newenvironment{proof}{\noindent{\it Proof.--}\begin{quotation}\noindent }{\end{quotation}\hfill$\square $}
%%%%%%%%%%%%%%%% macros Laure %%%%%%%%%%%%%%%%%%
%\def\epsqdefa{\buildrel\hbox{\footnotesize def}\over =}
%\def\sumetage#1#2{
%\sum_{\scriptstyle {#1}\atop\scriptstyle {#2}} }
%\def\prodetage#1#2{
%\prod_{\scriptstyle {#1}\atop\scriptstyle {#2}} }


\def\epsqdefa{\buildrel\hbox{\footnotesize def}\over =}
\def\sumetage#1#2{
\sum_{\scriptstyle {#1}\atop\scriptstyle {#2}} }
\def\prodetage#1#2{
\prod_{\scriptstyle {#1}\atop\scriptstyle {#2}} }


\def\bS {\mathbb{S}}
\def\bC {\mathbb{C}}
\def\bD {\mathbb {D}}
 \def\N {\mathbb{N}}
\def\R {\mathbb{R}}
\def\D {\mathbb{D}}
\def\Q {\mathbb{Q}}
\def\T {\mathbb{T}}
\def\Z {\mathbb{Z}}
 \def\bP {\mathbb{P}}
\def\J{\bold{J}}
\def\p{\bold{p}}
\def\fH {\mathfrak{H}}
\def\fS {\mathfrak{S}}
\def\fR {\mathfrak{R}}
\def\fa {\mathfrak{a}}
\def\bb {\mathfrak{b}}
\def\cA {\mathcal{A}}
\def\cB {\mathcal{B}}
\def\cC {\mathcal{C}}
\def\DD {\mathcal{D}}
\def\cD {\mathcal{D}}
\def\cK {\mathcal{K}}
\def\cE {\mathcal{E}}
\def\cF {\mathcal{F}}
\def\cH {\mathcal{H}}
\def\cG {\mathcal{G}}
\def\cL {\mathcal{L}}
\def\cF {\mathcal{F}}
\def\cM {\mathcal{M}}
\def\cN {\mathcal{N}}
\def\cQ {\mathcal{Q}}
\def\cP {\mathcal{P}}
\def\cz {\mathcal{z}}
\def\cS {\mathcal{S}}
\def\cZ {\mathcal{Z}}
%\def\hE {\mathcal{R}}
\def\cT {\mathcal{T}}
\def\cW {\mathcal{W}}
\def\wL {{w-L}}
\def\bx {{\textbf{x}}}
\def\aa {\mathrm{axi}}

\newcommand{\asigma}{a_i}




\def\a {{\alpha}}
\def\b {{\beta}}
\def\g {{\gamma}}
\def\de {{\partial}}
\def\eps {{\varepsilon}}
\def\e {{\varepsilon}}
\def\th {{\rho}}
\def\k {{\kappa}}
\def\l {{\lambda}}
\def\si {{\sigma}}
\def\Ups {{\Upsilon}}
\def\tUps {{\tilde\Upsilon}}
\def\om {{\omega}}
\def\Om {{\Omega}}
\def\L {{\Lambda}}
\newcommand{\eul}{\mathrm{e}}

\def\rstr {{\big |}}
\def\indc {{\bf 1}}


\def\la {\langle}
\def\ra {\rangle}
\def \La {\bigg\langle}
\def \Ra {\bigg\rangle}
\def \lA {\big\langle \! \! \big\langle}
\def \rA {\big\rangle \! \! \big\rangle}
\def \LA {\bigg\langle \! \! \! \! \! \,  \bigg\langle}
\def \RA {\bigg\rangle \! \! \! \! \! \,  \bigg\rangle}

\def\bFc {{\bF^{\hbox{conv}}_\eps}}
\def\bFcn{{\bF^{\hbox{conv}}_{\eps_n}}}
\def\bFd {{\bF^{\hbox{diff}}_\eps}}
\def\bFdn{{\bF^{\hbox{diff}}_{\eps_n}}}

\def\d {{\partial}}
\def\grad {{\nabla}}
\def\Dlt {{\Delta}}

\newcommand{\Div}{\operatorname{div}}
\newcommand{\rot}{\operatorname{rot}}
\newcommand{\Diam}{\operatorname{diam}}
\newcommand{\Dom}{\operatorname{Do\def\ff {\mathbf{f}}m}}
\newcommand{\sgn}{\operatorname{sign}}
\newcommand{\Span}{\operatorname{span}}
\newcommand{\supp}{\operatorname{supp}}
\newcommand{\Det}{\operatorname{det}}
\newcommand{\Tr}{\operatorname{trace}}
\newcommand{\Codim}{\operatorname{codim}}
\newcommand{\Dist}{\operatorname{dist}}
\newcommand{\Ker}{\operatorname{Ker}}
%\newcommand{\Diag}{\operatorname{diag}}
\newcommand{\ba}{\begin{aligned}}
\newcommand{\ea}{\end{aligned}}
\newcommand{\be}{\begin{equation}}
\newcommand{\ee}{\end{equation}}
\newcommand{\bu}{\textbf{u}}
\newcommand{\ubl}{u^{BL}}
\newcommand{\pbl}{p^{BL}}
\newcommand{\Pbl}{P^{BL}}
\newcommand{\Ubl}{U^{BL}}
\newcommand{\uint}{u^{int}}
\newcommand{\vint}{v^{int}}
\newcommand{\ustat}{u^{stat}}
\newcommand{\pstat}{p^{stat}}
\newcommand{\lb}{\label}
\newcommand{\Ph}{\mathbb P_{2D}}
\newcommand{\bt}{\bar \rho}
\newcommand{\tbl}{\rho^{BL}}
\newcommand{\tapp}{\rho^{app}}
\newcommand{\petit}{\xi}
\newcommand{\wt}{\widetilde}
\newcommand{\brown}{\Xi}
%\def\i{\bar\imath }
\let\dsp=\displaystyle
\newcommand{\C}{\mathbb C}
\newcommand{\red}{\textcolor{red}}
%%%%% MACROS
\newcommand{\Ps}{\Psi_{\mathrm{app}}}
\newcommand{\Psa}{\Psi_{2}}
\newcommand{\hE}{\Psi_{\mathrm{err}}}
\newcommand{\x}{\xi_{\mathrm{app}}}
\newcommand{\Ome}{\Omega_{\mathrm{app}}}
\newcommand{\et}{\eta_{\mathrm{app}}}
\newcommand{\fe}{f_{\mathrm{app}}}
\newcommand{\csi}{\xi_{\mathrm{app}}}
%%%%%%%%%%%%%%%%%%%%%%%%%%%%%%%%%%%%%%%%%%%%%%%%

%%%%%%%%%%%%%%%% macros Diogo %%%%%%%%%%%%%%%%%%
%\newcommand{\set}[2]{\left\{#1\,:\,#2\right\}}
%\newcommand{\ip}[1]{\left\langle#1\right\rangle}
%\newcommand{\iip}[1]{\left\langle\!\!\left\langle#1\right\rangle\!\!\right\rangle}
%%%%%%%%%%%%%%%%%%%%%%%%%%%%%%%%%%%%%%%%%%%%%%%%

%%%%%%%%%%%%% numbering equations %%%%%%%%%%%%%%
\numberwithin{equation}{section}
%%%%%%%%%%%%%%%%%%%%%%%%%%%%%%%%%%%%%%%%%%%%%%%%

%%%%%%%%%%%%%%%%%%%%%%%%%%%%%%%%%%%%%%%%%%%%%%%%
\begin{document}%%%%%%%%%%%%%%%%%%%%%%%%%%%%%%%%
%%%%%%%%%%%%%%%%%%%%%%%%%%%%%%%%%%%%%%%%%%%%%%%%

% Vous pouvez mettre dans la prochain ligne la rubrique choisie
% (si vous la connaissez) - meme deux, format : Rubrique1/Rubrique2

\title[Strong ill-posedness in $W^{1, \infty}$ of 2d Boussinesq and 3d axisymmetric Euler]{Strong ill-posedness in $W^{1, \infty}$ of the 2d stably stratified Boussinesq equations and application to the 3d axisymmetric Euler Equations}
\author[R. Bianchini]{Roberta Bianchini}
\address{IAC, Consiglio Nazionale delle Ricerche, 00185 Rome, Italy.}
\email{roberta.bianchini@cnr.it}
\author[L.E. Hientzsch]{Lars Eric Hientzsch}
\address{Fakult\"at f\"ur Mathematik, Universit\"at Bielefeld, 33501 Bielefeld, Germany.}
\email{lhientzsch@math.uni-bielefeld.de}
\author[F. Iandoli]{Felice Iandoli}
\address{DEMACS, Università della Calabria, 87035 Rende, Italy.}
\email{felice.iandoli@unical.it}
%\date\today
 
%\null
\maketitle 
\begin{abstract}
We prove the strong ill-posedness in the sense of Hadamard of the two-dimensional Boussinesq equations in $W^{1, \infty}(\R^2)$ without boundary, extending to the case of systems the method that Shikh Khalil \& Elgindi arXiv:2207.04556v1 developed for scalar equations. We provide a large class of initial data with velocity and density of small $W^{1, \infty}(\R^2)$ size, for which the horizontal density gradient has a strong $L^{\infty}(\R^2)$-norm inflation in infinitesimal time, while the vorticity and the vertical density gradient remain bounded. Furthermore, exploiting the three-dimensional version of Elgindi's decomposition of the Biot-Savart law, we apply the method to the three-dimensional axisymmetric Euler equations with swirl and away from the vertical axis, showing that a large class of initial data with velocity field uniformly bounded in $W^{1, \infty}(\R^2)$ provides a solution whose swirl component has a strong $W^{1, \infty}(\R^2)$-norm inflation in infinitesimal time, while the potential vorticity remains bounded at least for small times. Finally, the $W^{1,\infty}$-norm inflation of the swirl (and the $L^{\infty}$-norm inflation of the vorticity field) is quantified from below by an explicit lower bound which depends on time, the size of the data and it is valid for small times.
\end{abstract}
%\tableofcontents


\section{Introduction and Result}
The two-dimensional Boussinesq equations in $(x,y) \in \R^2$, in vorticity $\omega(t,x,y): [0, \infty) \times \R^2 \to \R $ and buoyancy $\rho(t,x,y): [0, \infty) \times \R^2 \to  \R_+$ formulation, read as follows
%%%%%
\begin{align}\label{eq:2dbouss}
\de_t \omega + \bu \cdot \nabla \omega & = \de_x \rho, \notag\\
\de_t \rho + \bu \cdot \nabla \rho&= 0, 
\end{align}
%%%%%
where $\bu (t,x,y)=(u_1 (t,x,y), u_2 (t,x,y)) \in \R^2$ is a divergence-free velocity field that can be expressed in terms of a stream-function as $\bu=\nabla^\perp \psi$, where $\nabla^\perp=(-\de_y, \de_x)^T$ and 
%
\be
-\Delta \psi = \omega. 
\ee
%
This is obtained from the two-dimensional incompressible Euler equations for non-homogeneous fluids, after applying the so-called \emph{Boussinesq approximation} (\cite{long1965}), which neglects density variations everywhere but in gravity terms, see \cite{lannes, bianchini} for further details. Besides being very commonly used for applications in oceanography and atmospheric sciences \cite{rieutord}, the 2d Boussinesq system \eqref{eq:2dbouss} attracted the interest of the mathematical community since several decades and whether finite-time blow-up of solutions emanated from finite-energy, smooth initial data can take place is an outstanding open question (see \cite{yudovich}). 
In the presence of boundary, several impressive contributions have been very recently made in this direction, among which Chen \& Hou \cite{chen2022} showed the blow-up of self similar solutions by means of computer assisted proofs and physics-informed neural networks have been used to construct approximated self-similar blow-up profiles in \cite{wang2022}. The result in \cite{chen2022} also includes the finite-time blow-up for smooth data for the 3d axisymmetric Euler equations which is based on an analogy between such equations in potential vorticity $\frac{\omega_\theta}{r}$ formulation where $\bu^\aa=(u_r^\aa,u_z^\aa)^T$ and the 2d Boussinesq system in vorticity $\omega$ and density gradient $\nabla \rho$, which is very well described for instance in \cite[Section 1.2]{jeong}. We report here for completeness (for explanations on the 3d axisymmetric Euler equations see Section \ref{sec:euler}) :
%
\vspace{0.8cm}
\begin{multicols}{2}
\noindent
\be\label{eq:2dbouss-origianal-gradient}
\ba
D_t \omega &= \de_x \rho, \\
D_t (\de_x\rho) &= - \de_x \bu \cdot \nabla_{x,y} \rho, \\
D_t (\de_y \rho)&=  - \de_y \bu \cdot \nabla_{x,y} \rho,\\
D_t:&=\de_t + \bu \cdot \nabla_{x,y},
\ea
\ee
\be\label{eq:3daxisymm-intro}
\ba
\wt D_t \left({\omega_\theta^\aa}/{r}\right)  &= - {r^{-4}} \de_z ((r u_\theta^\aa)^2), \notag\\
\wt D_t \de_r (r u_\theta^\aa) &=- \de_r \bu^\aa \cdot \nabla_{r, z} (r u_\theta^\aa),  \notag\\
\wt D_t  \de_z(r  u_\theta^\aa) &=- \de_z \bu^\aa \cdot \nabla_{r, z} (r u_\theta^\aa), \\
\wt D_t :&= \de_t + \bu^\aa \cdot \nabla_{r, z}.
\ea\ee
\end{multicols}
%
Very far from being exhaustive, regarding the outstanding open question on finite-time blow up of smooth solutions to the 3d Euler equations we only mention the first rigorous result for $C^{1, \alpha}$ solutions by Elgindi in \cite{tarek1}, the analogous result with boundary by Chen \& Hou \cite{chen1} and the aforementioned work \cite{chen2022}. 
Besides the blow-up scenario, another important question concerning the 2d Boussinesq equations is the small-scale formation and the growth of Sobolev norms in infinite time for large classes of initial data, see the work \cite{kiselev2022} and references therein for recent developments in this direction. 

The 2d Boussinesq system \eqref{eq:2dbouss} admits the so-called \emph{stratified} steady solutions $(\bar \omega_{\text{eq}}, \bar \rho_{\text{eq}})=(0, \bar \rho_{\text{eq}}(y))$, where $\bar \rho_{\text{eq}}(y)$ is a continuous function of the vertical coordinate $y$ only, which are particularly interesting for application in oceanography, see for instance \cite{DY1999, rieutord}. In that context, it is customary to consider a background stratification $\bar \rho_{\text{eq}}(y)$ that decreases with height  $\bar \rho_{\text{eq}}'(y)<0$, so that the steady state is called \emph{stably stratified} because the linear operator associated with the linearization of Eqs. \eqref{eq:2dbouss} around such steady state is spectrally stable (see for instance \cite{gallay, lannes}).
Let us consider the perturbations, modeled by system \eqref{eq:2dbouss}, of the simplest stably stratified density profile, namely $\bar \rho_{\text{eq}}(y) = \bar \rho_c - y$, where $\bar \rho_c >0$ is a constant averaged density. Being \eqref{eq:2dbouss} invariant by translation of $\rho$, we consider for convenience $\bar \rho_{\text{eq}}(y) = -y$. We then introduce the perturbed variables $\wt \omega$ and $\wt \rho=  y+ \rho $ and write the equations satisfied by them below :
%%%%%
\begin{align}\label{eq:2dbouss-stable}
\de_t \wt\omega + \wt \bu \cdot  \nabla \wt \omega & = \de_x \wt \rho, \notag\\
\de_t \wt \rho + \wt \bu \cdot  \nabla \wt\rho&= \wt u_2,\\
\wt \bu &=(\wt u_1, \wt u_2)= (-\Delta)^{-1} \nabla^\perp \wt \omega.\notag
\end{align}
 %%%%%
Within the framework of a spectrally stable regime (under the assumption $\bar \rho_{\text{eq}}'(y)<0$), it could be expected that solutions to \eqref{eq:2dbouss-stable} behave better than the ones of the original 2d Boussinesq equations \eqref{eq:2dbouss}. 
However, even in such a spectrally stable situation, the question of finite-time blow-up Vs global regularity of smooth solutions remains open. To our knowledge, the best available result in this context is due to Elgindi \& Widmayer \cite{klaus1} and provides existence of Sobolev solutions to system \eqref{eq:2dbouss-stable} for initial data of size $\eps$ up to times of order $t \sim \eps^{-4/3}.$ The method leading to the recent global existence result in \cite{klaus2}, namely exploiting the anisotropy and the null structure of the nonlinear term, should allow one to enlarge the existence time-scale for the 2d Boussinesq system \eqref{eq:2dbouss-stable}. If a linearization of \eqref{eq:2dbouss} around a the stably stratified state $\bar \rho_{\mathrm{eq}}(y)=-y$ and the Couette flow $\bar \bu_{\mathrm{eq}}=(y,0)^T$, rather then the trivial flow above, then (infinite regularity) $\eps$-perturbations of the velocity and density have been proved to decay up to times of order $t \sim \eps^{-2}$ although vorticity and density gradient display an algebraic growth in time (see \cite{bianchini1, bianchini2}).
%
We will show that the spectral stability of the Boussinesq system around stable stratification \eqref{eq:2dbouss-stable} is insufficient to rule out norm-inflation at $W^{1, \infty}$-regularity for the density $\rho$ and the velocity $\bu$, (resp. $L^\infty$ for vorticity $\omega$). More precisely, we prove that the 2d Boussinesq equations around the \emph{stably} stratified steady state $\bar \rho_{\text{eq}}(y) = - y$ are \emph{strongly ill-posed} in $W^{1, \infty}(\R^2)$ ($L^\infty(\R^2)$ for the vorticity). As consequence, we infer the same ill-posedness result for the original 2d Boussinesq equations \eqref{eq:2dbouss}. Before describing the results of this paper, we write system \eqref{eq:2dbouss} in a more convenient form. 
Dropping the \emph{tilde} for lightening the notation and taking the gradient $\nabla=(\de_x, \de_y)^T$ of the equation for $\rho$ in \eqref{eq:2dbouss-stable}, we write the system as follows:
%%%%%
\be\label{eq:2dbouss-grad}
\ba\de_t \omega+\bu \cdot \nabla \omega &= \de_x \rho, \\
\de_t (\de_x\rho) + \bu \cdot \nabla (\de_x \rho) &= \de_x u_2 - \de_x u_1 \de_x \rho - \de_x u_2 \de_y \rho, \\
\de_t (\de_y \rho) + \bu \cdot \nabla (\de_y \rho) &= \de_y u_2 - \de_y u_1 \de_x \rho - \de_y u_2 \de_y \rho.
\ea
\ee
%%%%%
As in \cite{tarek3}, our proof of strong ill-posedness relies on an approximation of \eqref{eq:2dbouss-grad}, under a certain scaling and in terms of a small parameter $0<\alpha \ll 1$ (see \eqref{eq:polar-coord}), from which one can extract a Leading Order Model (in terms of the approximation parameter $\alpha$) that drives the ill-posedness mechanism. Interestingly, our Leading Order Model (see \eqref{eq:leading1}) discards the ``stabilizing'' contribution $u_2$ in the right-hand side of the equation for $\wt \rho$ in \eqref{eq:2dbouss-stable}, being indeed $u_2$ a lower order term in $\alpha$ (see Remark \ref{rmk:originalB}). 
However, the stable steady state $\bar \rho_{\mathrm{eq}}(y)$ around which \eqref{eq:2dbouss-stable} is linearized is not transparent in our Leading Order Model. In fact, although being lower order in $\alpha$, the equation for $\de_y \rho$ in \eqref{eq:2dbouss-grad} cannot be completely discarded from our Leading Order Model because of the linear term $\de_y u_2$ in the right-hand side. Therefore the stable steady state $\bar \rho_{\mathrm{eq}}$ marks a difference between the original Boussinesq equations \eqref{eq:2dbouss} and their (spectrally) stable version \eqref{eq:2dbouss-stable}. However, the ill-posedness of \eqref{eq:2dbouss-stable} for small data ($L^\infty(\R^2)$ for the vorticity and $W^{1, \infty}$ for the density)  in Theorem \ref{thm:main} can be essentially reformulated (see Theorem \ref{thm:main2}) as a result of strong ill-posedness of the original 2d Boussinesq equations \eqref{eq:2dbouss} for data which are $W^{1,\infty}$-close to the stable steady state $\bar \rho_{\mathrm{eq}}$.
A key feature of the ill-posedness mechanism is that the main term leading to the $W^{1, \infty}$ norm inflation is $-\de_x u_1 \de_x \rho$ in the right-hand side of system \eqref{eq:2dbouss-grad}. Recalling that $\de_x u_1 = - \de_{xy} (-\Delta)^{-1} \omega = - R_1 R_2 \omega $, where $(R_1, R_2)$ is the two-dimensional Riesz Transform (see for instance \cite{grafakos}), our work confirms that the unboundedness of the Riesz Transform as an operator on $L^\infty$ constitutes the mechanism leading to strong ill-posedness. Note that this fact was already exploited by Elgindi \& Masmoudi \cite{tarek3} to prove the \emph{mild } ill-posedness of the stably stratified Boussinesq system \eqref{eq:2dbouss-stable}
in $(W^{1, \infty})$ (with ``mild'' ill-posedness we mean that, starting with data $\|f_0\|_{W^{1, \infty}} \lesssim \alpha \ll 1,$ then $\| f(t) \|_{W^{1, \infty}} > c$ for some fixed constant $c>0$ and some small time $t=o(1)$ in $\alpha$, see \cite{tarek3} for a more detailed definition). Exploiting the structure of the nonlinearity of system \eqref{eq:2dbouss-stable}, we prove that such ill-posedness in $W^{1, \infty}$ is actually \emph{strong}, namely that, starting again with small data  $\|f_0\|_{W^{1, \infty}} \lesssim \delta \ll 1,$ with $\delta>0$ arbitrarily small, there exist infinitesimal times $t=o(1)$ in $\delta$ such that $\|f(t)\|_{W^{1, \infty}} \ge \delta^{-1}$. 
We point out that the $W^{1, \infty}$ norm inflation in the mild ill-posedness result by Elgindi \& Masmoudi in \cite{tarek3}, which essentially exploits the linear part of system \eqref{eq:2dbouss-stable}, is of order $1/\alpha$, while our norm inflation is of order $\log |\log \alpha | $ (see Theorem \ref{thm:main}-\ref{thm:main2}). This means that the nonlinear term of system \eqref{eq:2dbouss-stable} weakens the effect of the linear part. Such an observation was already made by Elgindi \& Shikh Khalil \cite{tarek2} in the case of scalar equations with a Riesz-type linear operator. Here, we prove this fact for the 2d Boussinesq system providing to the best of our knowledge such a result for a system of equations for the first time. Moreover, exploiting the analogy between the 2d Boussinesq system and the 3d axisymmetric Euler equations in potential vorticity formulation \eqref{eq:2dbouss-origianal-gradient}, we apply our method to the latter. More specifically, we provide in this paper a result of $W^{1, \infty}$ strong ill-posedness of the 3d axisymmetric Euler equations with swirl (Theorem \ref{thm:main3}). 
While our results for the 2d Boussinesq equations (Theorem \ref{thm:main}, Theorem \ref{thm:main2}) are original, the strong ill-posedness of the 3d Euler equations that we prove in Theorem \ref{thm:main3} provides a new proof of a result that is in some sense close to \cite[Proposition 1.10]{tarek3}. However, we emphasize that the authors of \cite{tarek3} consider the so called $2\frac12$-dimensional solutions to the 3d Euler equations while we are concerned with the axisymmetric 3d Euler equations. Beyond the different strategies, we provide a large class (one example of datum is given in \cite{tarek3}) of initial data (see Section \ref{sec:euler} and Section \ref{sec:initialdata}) for which the swirl component of the velocity field displays strong $W^{1,\infty}$-norm inflation at infinitesimal times, while the angular component of the vorticity field remains bounded. In other words, the term $-\de_r u_r^\aa \de_r (r u_\theta^\aa)$ in the right-hand side of 3d Euler in \eqref{eq:2dbouss-origianal-gradient} plays the same role as $-\de_x u_1 \de_x \rho$ for 2d Boussinesq.
Note that the mechanisms of Theorem \ref{thm:main3} and of \cite[Proposition 1.10]{tarek3} are completely different from the analogous Lipschitz-ill-posdeness of the 3d Euler equations by Bourgain \& Li \cite{bourgain2015}, where the ill-posedness comes from the 3d axisymmetric Euler equations \emph{without} swirl.
%%%%%
It is important to underline that several key steps of our proof, such as the $\alpha$-scaling of the radial coordinate $r=\sqrt{x^2+y^2}=R^{1/\alpha}$ and the decomposition of the Biot-Savart law (Theorem \ref{thm:tarek}), follow Elgindi's ideas \cite{tarek1, tarek2}. A detailed comparison with the inspiring works on ill-posedness for the 2d Boussinesq system by Elgindi et al \cite{tarek2, tarek3} and finite-time blow-up of solutions to the 2d Boussinesq system with boundary by Chen \& Hou \cite{chen2022} is provided below, as well as further comments on the recent ill-posedness results for the 3d Euler equations by Bourgain \& Li \cite{bourgain2015} and Elgindi \& Masmoudi \cite{tarek3}.

We prove our results on the 2d Boussinesq equations, while the application to the 3d Euler equations is confined to Section \ref{sec:euler}.
Hereafter, we  work in $\R^2$ in polar coordinates, with the following radial scaling : 
\be\label{eq:polar-coord}
r=\sqrt{x^2+y^2}, \qquad \beta=\arctan (x/y), \qquad R=r^\alpha, \quad 0 <\alpha \le \alpha_0\ll 1,
\ee
where $\alpha \ll 1$ plays the role of the small parameter of the approximation.
 We further define the new
 variables depending on $(R, \beta) \in [0, \infty)\times [0, 2\pi]$ as follows 
\be\label{eq:new-var}\ba
\Omega=\Omega (R, \beta):=\omega(x,y), \qquad P=P(R, \beta):= \rho(x,y). 
\ea\ee

%%%%%
Before going further, we introduce our notation and convention below.
\subsection*{Notation and convention}\label{sec:notation}
\begin{itemize}
\item For the 2d Boussinesq equations, namely up to Section \ref{sec:last}, we work in the polar coordinates $(R,\beta)$ where $R=r^{\alpha}$ with $r=\sqrt{x^2+y^2}$, $\alpha>0$ small and $\beta=\arctan(x/y)$. In Section \ref{sec:euler}, we introduce new coordinates tailored to the 3d setting with axial symmetry which are used in the respective section of the paper only.
\item For $1\leq p \leq \infty$, the $L^p$ space with respect to the variables $(R,\rho)$, namely the space $L^p([0,\infty)\times [0,2\pi])$, is defined with respect to the measure $\mathrm{d}R\mathrm{d}\beta$ - unlike the common definition of $L^2(\R^2)$ in polar coordinates. If not stated otherwise, $L^p$ indicates  $L^p([0,\infty)\times [0,2\pi])$.  
\item The $L^2$-based Sobolev space $H^k=H^k([0,\infty)\times [0,2\pi])$ with $k\in \N_{0}$ and norm
\begin{equation*}
    \|f\|_{H^k}=\sum_{0\leq|\gamma|\leq k}\|\partial^{\gamma}u\|_{L^2},
\end{equation*}
where $\gamma$ indicates the $2d$-multi-index. We shall use the notation $\partial^n$ to indicate $\partial^\gamma$ with $|\gamma|=n$.
\item Following \cite{tarek2}, we introduce the weighted (non)-homogeneous Sobolev spaces $\dot{\cH}^k([0,\infty)\times [0,2\pi])$ and $\cH^k([0,\infty)\times [0,2\pi])$ respectively with norms
%
\begin{equation}\label{def:Hk}
\|f\|_{\dot{\cH}^m}=\sum_{i=0}^{m}\left(\|\partial_R^{i}\partial_\beta^{m-i}f\|_{L^2}+\|R^i\partial_R^{i}\partial_\beta^{m-i}f\|_{L^2}\right), \qquad \|f\|_{\cH^k}=\sum_{m=0}^k\|f\|_{\dot{\cH}^m}.
\end{equation}
%
Similarly, we denote for $k\in \N_{0}$ the spaces $\dot{\cW}^{m,\infty}$ and $\cW^{k,\infty}$ respectively with norms
%
\begin{equation}\label{def:Wk}      \|f\|_{\dot{\cW}^m}=\sum_{i=0}^{m}\left(\|\partial_R^{i}\partial_\beta^{m-i}f\|_{L^\infty}+\|R^i\partial_R^{i}\partial_\beta^{m-i}f\|_{L^\infty}\right), \qquad \|f\|_{\cW^k}=\sum_{m=0}^k\|f\|_{\dot{\cW}^m}.
\end{equation}
%
\item We adopt the notation $C_k=C_k ( \|\omega_0\|_{\cH^k(\R^2)}, \|\rho_0\|_{\cH^k(\R^2)})$ to denote a general constant, which may change from line to line, depending only on the $\cH^k(\R^2)$ norm of the initial data $(\omega_0, \rho_0)$ and on the space dimension $d$. 
\item The symbol $\lesssim $ (resp. $\gtrsim $) denotes $\le C $ (resp. $ \ge C$) for some constant $C>0$ and independent of the small parameter $\alpha$.
\end{itemize}
%%%%%
\subsection*{Main Results}

Our results read as follows.
\begin{Thm}[$W^{1, \infty}$ strong ill-posedness for the stably stratified Boussinesq system \eqref{eq:2dbouss-stable}]
\label{thm:main}
Let $N \ge 3$.
There exist $0<\alpha_0 \ll 1$ and $\alpha_0 \ll \delta \ll 1$ such that, for any $0<\alpha \le \alpha_0 $, there exist initial data $(\omega_0^{\alpha, \delta}(x,y), \rho_0^{\alpha, \delta}(x, y))=(\Omega_0^{\alpha, \delta}(R, \beta), P_0^{\alpha, \delta}(R, \beta))$ of the following form 
\begin{align}
    \Omega_0^{\alpha, \delta} (R, \beta) = \bar g_0^{\alpha, \delta} (R) \sin (2\beta), \qquad P_0^{\alpha, \delta} (R, \beta)= R^{1/\alpha} \bar \eta_0^{\alpha, \delta} (R) \cos \beta, 
\end{align}
where $\bar g_0^{\alpha, \delta} (R), \bar \eta_0^{\alpha, \delta} (R) \in C_c^\infty([1, \infty))$ with 
$$\|(\bar g_0^{\alpha, \delta}, \bar \eta_0^{\alpha, \delta})\|_{\cH^N} \sim C_N, $$
and 
\be\ba
\|(\omega_0^{\alpha, \delta}, \rho_0^{\alpha, \delta})\|_{L^\infty(\R^2)} &\sim  \delta, \qquad \|(\de_x \rho_0^{\alpha, \delta}, \de_y \rho_0^{\alpha, \delta})\|_{L^\infty(\R^2)} \sim  \delta,\\
\|(\omega_0^{\alpha, \delta}, \rho_0^{\alpha, \delta})\|_{\cH^N(\R^2)} &\sim   C_N, \qquad \|(\de_x \rho_0^{\alpha, \delta}, \de_y \rho_0^{\alpha, \delta})\|_{\cH^N(\R^2)} \sim   C_N + \alpha C_{N+1},
\ea \ee
such that the corresponding unique solution $\omega^{\alpha, \delta} (t,x,y), \rho^{\alpha, \delta}(t, x, y)$ to the Cauchy problem associated with the stably stratified Boussinesq Eqs. \eqref{eq:2dbouss-grad} satisfies
%%%%
\be\ba
\sup\limits_{t \in [0, T^*(\alpha)]} \|\de_x \rho ^{\alpha, \delta}(t) \|_{L^\infty(\R^2)}  &\gtrsim  \|\de_x \rho_0^{\alpha, \delta}\|_{L^\infty (\R^2)} \left(1+\frac{\log |\log \alpha|}{C_{N+1}}\right)^{\frac{1}{c_2}},
\ea\ee
for $c_2>0$ independent of $\alpha$, for $T^*(\alpha)\sim  \frac{\alpha}{C_{N+1}}\log |\log (\alpha)| \, \to \, 0$ as $\alpha \to 0$ and such that $\frac{\log |\log \alpha|}{C_{N+1}} \gtrsim \log |\log \alpha|^\mu$ for some $0<\mu <1$. 
\end{Thm}
%%%%%
\begin{Thm}[$W^{1, \infty}$ strong ill-posedness for the original Boussinesq system \eqref{eq:2dbouss}]\label{thm:main2}
Let $N \ge 3$.
There exist $0<\alpha_0 \ll 1$ and $\alpha_0 \ll \delta \ll 1$ such that, for any $0<\alpha \le \alpha_0 $, there exist initial data $(\omega_0^{\alpha, \delta}(x,y), \rho_0^{\alpha, \delta}(x, y))=(\Omega_0^{\alpha, \delta}(R, \beta), P_0^{\alpha, \delta}(R, \beta))$ of the following form 
\begin{align}
    \Omega_0^{\alpha, \delta} (R, \beta) = \bar g_0^{\alpha, \delta} (R) \sin (2\beta), \qquad P_0^{\alpha, \delta} (R, \beta)= \bar \rho_{\mathrm{eq}}(R, \beta) +  R^{1/\alpha} \bar \eta_0^{\alpha, \delta} (R) \cos \beta, 
\end{align}
where $\bar \rho_{\mathrm{eq}}(R, \beta)=- R^{1/\alpha} \sin \beta = - y$, 
and $\bar g_0^{\alpha, \delta} (R), \bar \eta_0^{\alpha, \delta} (R) \in C_c^\infty([1, \infty))$ with
$$\|(\bar g_0^{\alpha, \delta}, \bar \eta_0^{\alpha, \delta})\|_{\cH^N} \sim C_N, $$
and 
\be\ba
\|(\omega_0^{\alpha, \delta}, \rho_0^{\alpha, \delta}- \bar \rho_{\mathrm{eq}})\|_{L^\infty(\R^2)} &\sim  \delta, \qquad \|(\de_x \rho_0^{\alpha, \delta}, \de_y \rho_0^{\alpha, \delta}- \de_y \bar \rho_{\mathrm{eq}})\|_{L^\infty(\R^2)} \sim  \delta,\\
\|(\omega_0^{\alpha, \delta}, \rho_0^{\alpha, \delta})\|_{\cH^N(\R^2)} &\sim   C_N, \qquad \|(\de_x \rho_0^{\alpha, \delta}, \de_y \bar \rho_0^{\alpha, \delta}- \de_y \bar \rho_{\mathrm{eq}})\|_{\cH^N(\R^2)} \sim   C_N + \alpha C_{N+1},
\ea \ee
such that the corresponding unique solution $\omega^{\alpha, \delta} (t,x,y), \rho^{\alpha, \delta}(t, x, y)$ to the Cauchy problem associated with the stably stratified Boussinesq Eqs. \eqref{eq:2dbouss-grad} satisfies
%%%%
\be\ba
\sup\limits_{t \in [0, T^*(\alpha)]} \|\de_x \rho ^{\alpha, \delta}(t) \|_{L^\infty(\R^2)}  &\gtrsim  \|\de_x \rho_0^{\alpha, \delta}\|_{L^\infty (\R^2)} \left(1+\frac{\log |\log \alpha|}{C_{N+1}}\right)^{\frac{1}{c_2}},
\ea\ee
for $c_2>0$ independent of $\alpha$, for 
 $T^*(\alpha)\sim  \frac{\alpha}{C_{N+1}}\log |\log (\alpha)| \, \to \, 0$ as $\alpha \to 0$ and such that $\frac{\log |\log \alpha|}{C_{N+1}} \gtrsim \log |\log \alpha|^\mu$ for some $0<\mu <1$. 
\end{Thm}
%%%%%
Since it requires introducing a different notation and it is settled in $\R^3$, our result on the $W^{1, \infty}$ ill-posedness of the 3d axisymmetric Euler equations is stated and proved in Section \ref{sec:euler}. Here we provide only a very rough idea of its content.
\begin{Thm}[$W^{1, \infty}$ strong ill-posedness for 3d axisymmetric Euler equations with swirl, see Theorem \ref{thm:main3sec3}] 
\label{thm:main3}
There exists a sequence of $C_c^\infty (\R^3) $  initial data with size in $W^{1, \infty}$ of order $\delta=o(1)$ in $\alpha $ for all $\alpha$ small enough, such that there exist infinitesimal times $t = o(1)$ in $\alpha$ such that $$\left\|\frac{u_\theta^\aa}{r}+{\de_r (u_\theta^\aa)^2}\right\|_{L^\infty} \gtrsim  \delta^{-1},$$
while $\omega_\theta^\aa$ remains uniformly bounded,
where $u_\theta^\aa = u_\theta^\aa (t, r, z)$ is the swirl component of $\bu^\aa$ and $\omega_\theta^\aa=\omega_\theta^\aa (t, r, z)$ is the angular vorticity of the solution to the 3d axisymmetric Euler equations with swirl \eqref{eq:3daxisymm-intro}.
\end{Thm}
%%%%%
\subsection*{Comparison with the recent literature on ill-posedness results}
The first result of \emph{mild} ill-posedness in $W^{1, \infty}$ for the 2d Boussinesq equation around a stably stratified steady state with $\bar \rho_{\mathrm{eq}}=-y$ as in \eqref{eq:2dbouss-stable} is due to Elgindi \& Masmoudi \cite{tarek3}. In \cite{tarek3}, the authors prove that for smooth initial data of $O(\alpha)$ in $W^{1, \infty}$, the solution to \eqref{eq:2dbouss-stable} satisfies $\|\omega (t)\|_{L^\infty} > C$ with $C>0$ a constant value independent of $\alpha$, for some $t < \alpha\ll 1$. In this paper we prove that such $W^{1, \infty}$ ill-posedness of \eqref{eq:2dbouss-stable} is actually \emph{strong}, namely we provide infinitesimal initial data in $W^{1, \infty}$ such that $\|\de_x \rho (t)\|_{L^\infty(\R^2)} > \log |\log \alpha|^\mu$ with $\mu>0$ at some infinitesimal time is arbitrarily large. Introducing a stronger notion of ill-posedness, the method of this paper differs substantially from the one of \cite{tarek3}, which is based on the fact that there exist functions $f(x,y) \in L^\infty$ such that the Riesz Transform $R f (x, y) \notin L^\infty $. Our strategy is inspired by the work of Elgindi \& Shikh Khalil \cite{tarek2}, where a result of strong ill-posedness for \emph{scalar} nonlinear transport equations with a Riesz-type linear operator in two space dimensions is provided. More precisely, we use the same scaling of the polar coordinates, leading to the crucial Elgindi's decomposition of the Biot-Savart law \cite{tarek1}. As in \cite{tarek2}, we derive a Leading Order Model (see \eqref{eq:leading1}) yielding the strong ill-posedness result for a suitable choice of the initial data. We point out, however, that there are several key differences between this paper and \cite{tarek2}. In particular, while \cite{tarek2} deals with scalar equations, our Leading Order Model \eqref{eq:leading1} is a system of coupled equations. Moreover, while \cite{tarek2} proves the blow-up of $\omega (t)$ in $L^\infty(\R^2)$, our choice of the initial data (an explicit example of them is given in Section \ref{sec:initialdata}) leads to the discontinuous dependence of the (horizontal) gradient $\de_x \rho (t)$ on the initial data in $L^\infty (\R^2)$. We also point out that our result holds for a large class of initial data $\rho (0,\cdot )= \bar \eta_0 (r) r\cos \beta, \omega (0, \cdot) = \bar g_0(r) \sin (2\beta)$ where $(r, \beta) \in [0, \infty) \times [-\pi/2, \pi/2]$ are the polar coordinates ($x=r \cos \beta, y=r \sin \beta$) and $\bar \eta_0 (r), \bar g_0 (r) \in C_c^\infty ([1, \infty)])$. \\
Concerning the 3d incompressible Euler equations, the $W^{1, \infty}$-ill-posedness has been recently proved by Elgindi \& Masmoudi \cite{tarek3} and Bourgain \& Li \cite{bourgain2015}. Theorem \ref{thm:main3} of the present paper is again a result of $W^{1, \infty}$ ill-posedness of the 3d axisymmetric Euler equations, but the are considerable differences in the approaches. More specifically, the mechanisms that generate the ill-posedness (there is no swirl in \cite{bourgain2015}, while the swirl generates norm inflation in Theorem \ref{thm:main3}), the general form of the admissible initial data and the explicit quantification of the norm inflation by means of a lower bound - point-wise in time - of the $L^\infty$-norm of the swirl (see Theorem \ref{thm:main3}). To the best of our knowledge, Theorem \ref{thm:main3} constitutes the first ill-posedness result achieved by norm inflation of the swirl component. For a more detailed comparison, we refer the reader to Remark \ref{rmk:comparison}.
Regarding recent results on the strong ill-posedness for hydrodynamic equations, we mention for completeness the works \cite{bourgain, jeong3, cordoba1, cordoba2}, which are based on different strategies.
%%%%%
\subsection*{Comparison with the recent literature on finite-time blow-up}
For $C^{1,\alpha}$ initial data, the local well-posedness of the solutions to \eqref{eq:2dbouss} is given in \cite{chae1999}, together with a Beale-Kato-Majda-type criterion for blow-up. Next, finite-time blow-up of the local smooth solutions in domains with corners and data $(\nabla \rho (0, \cdot), \omega (0, \cdot)) \in {\mathring {C}}^{0, \alpha}$ was proved by Elgindi \& Jeong in \cite{jeong} for the 2d Boussinesq equations \eqref{eq:2dbouss-grad} and in \cite{jeong2} for the 3d axisymmetric Euler equations. Furthermore, the finite-time blow-up of finite-energy $C^{1, \alpha}$ solutions to the 2d Boussinesq and 3d axisymmetric Euler equations in a domain with boundary (but without any corner) is due to Chen \& Hou \cite{chen1}. Since \cite{chen1} has been, together with \cite{tarek2}, a source of inspiration for our result, we comment further on similarities and differences. In fact, our strategy and the one of \cite{chen1} share some similarity - even though \cite{chen1} constitutes a result of finite-time blow-up in a regularity class where the well-posedness of \eqref{eq:2dbouss} is well established, while we prove the strong ill-posedness of \eqref{eq:2dbouss} and \eqref{eq:2dbouss-stable} in the lower regularity setting of $W^{1, \infty}$. First, our Theorem \ref{thm:main} yields not only the strong ill-posedness of the original 2d Boussinesq equations \eqref{eq:2dbouss-grad}, but also of the 2d Boussinesq equations \eqref{eq:2dbouss-stable} around a \emph{stably stratified density profile}. Note that the ill-posedness of the stably stratified 2d Boussinesq system \eqref{eq:2dbouss-stable} could be less expected, since the linearized Boussinesq equations around $\bar \rho_{\mathrm{eq}}(y)$ with $\bar \rho_{\mathrm{eq}}'(y)<0$ and $\bar {\bold u}_{\mathrm{eq}}=0$ are spectrally stable (see for instance \cite{gallay}). For that reason, although both our argument and the method in \cite{chen1} are based on the construction of a leading order model for \eqref{eq:2dbouss-grad}, our Leading Order Model for \eqref{eq:2dbouss-stable} in the stable setting is different. In particular, even though, as in \cite{chen1}, $\de_y \rho (t, \cdot)$ has a lower order than $\omega (t, \cdot), \de_x \rho (t, \cdot)$ in terms of the small scaling parameter $0<\alpha \ll 1$, we cannot discard the equation for $\de_y \rho (t, \cdot)$ in our Leading Order Model. The main obstruction is due to the linear term $\de_y u_2$ in the right-hand side of the equation for $\de_y \rho $ in \eqref{eq:2dbouss-stable}, which we include in the Leading Order Model \eqref{eq:leading1} in order to be able to discard it in the remainder estimates of Section \ref{sec:rem}. There are therefore substantial differences between our Leading Order Model and the one of \cite{chen1}. It is also worth pointing out that, in contrast to \cite{chen1}, our Theorem \ref{thm:main} and Theorem \ref{thm:main2} hold in the full $\R^2$ space without any boundary. %\textcolor{red}{Technically, we consider some boundary in the latter case... Dovremmo dire pero' che non abbiamo un dominio periodico come loro? Da loro il boundary ha un ruolo cruciale mentre da noi e' piu' che altro tecnico.}
Note also that a result of finite-time blow-up up of finite-energy smooth solutions to the 2d Boussinesq and the 3d axisymmetric Euler equations with boundaries is the recent work \cite{chen2022}. 
%%%%%
\subsection*{Plan of the paper}
The paper is organized as follows. The derivation of the approximated system leading to the $W^{1, \infty}$ ill-posedness of \eqref{eq:2dbouss-stable} is given in Section \ref{sec:derivation}. Next, Section \ref{sec:estimateLOM} provides the $L^\infty$ ill-posedness of the approximated system (Proposition \ref{prop:expl}) as well as $\cH^k$ a priori estimates for the approximated system. Section \ref{sec:rem} provides quantitative estimates of the difference in $\cH^k$ between the approximated and the true Boussinesq system \eqref{eq:2dbouss-grad}. The proofs of Theorem \ref{thm:main} and Theorem \ref{thm:main2} are achieved in Section \ref{sec:last}. Finally, the application to the 3d axisymmetric Euler equations (proof of Theorem \ref{thm:main3}) is given in Section \ref{sec:euler}.
%%%%%



\section{Derivation of the Leading Order Model and a priori estimates}\label{sec:derivation}
Consider $\Omega (R, \beta)$ as in \eqref{eq:new-var}, and further introduce 
\be\label{eq:psi-polar}
\Psi(R, \beta)=r^{-2} \psi (x, y).\ee 
We recall that
%%%%%
\begin{equation}\label{eq:spatial-der} 
\begin{aligned}
\de_x &\;\Rightarrow \; R^{-\frac 1 \alpha} ((\cos \beta)\alpha R \de_R - (\sin \beta) \de_\beta);\\
\de_y &\,\Rightarrow \; R^{-\frac 1 \alpha} ((\sin \beta) \alpha R \de_R + (\cos \beta) \de_\beta).
\end{aligned}
\end{equation}
 We can then express the velocity $\bu=(u_1, u_2)$ in terms of $\Psi$ as follows
\begin{align}
u_1&=-\de_y (r^2 \Psi (R, \beta))=-R^\frac 1 \alpha (2 \sin \beta \Psi + \alpha \sin \beta R \de_R \Psi + \cos \beta \de_\beta \Psi); \label{eq:u1}\\
u_2&=\de_x (r^2 \Psi (R, \beta))=R^\frac 1 \alpha (2 \cos \beta \Psi + \alpha \cos \beta R \de_R \Psi -\sin  \beta \de_\beta \Psi); \label{eq:u2}.
\end{align}
%%%%%
Now, we use Elgindi's decomposition of the Biot-Savart law.
\begin{Thm}[\cite{tarek1}, \cite{drivas}]\label{thm:tarek}
The unique regular solution to 
\begin{equation}\label{eq:ellittica}
4\Psi + \alpha^2 R^2 \de_{RR} \Psi + \de_{\beta \beta} \Psi + (4\alpha + \alpha^2) R \de_R \Psi = -\Omega (R, \beta)\end{equation}
with boundary conditions $\Psi(R, 0)=\Psi(R, \frac \pi 2)=0$
is given by
\be\label{eq:psi-main}
\Psi=\Psi(\Omega)(R, \beta) =\Psa + \hE, 
\ee 
where
\be \label{eq:psi2}
\Psa=\Psa(\Omega)(R, \beta):=  \Ps(\Omega)(R) \sin (2\beta) + \mathcal{R}^\alpha (\Omega)(R)\sin (2\beta),
\ee
satisfying the equation 
\begin{align}\label{eq:ell2}
    \alpha^2 R^2 \de_{RR} \Psi_2  + (4\alpha + \alpha^2) R \de_R \Psi_2 &= -\Omega_2 (R, \beta):= - \frac{\sin (2\beta)}{\pi} \int_0^{2\pi} \Omega (R, \beta) \sin (2\beta) \, d\beta,
\end{align}
and 
\be \label{eq:psiapp}
\Ps=\Ps(\Omega)(R, \beta):=  \frac{\cL (\Omega)(R)}{4 \alpha}\sin (2\beta),
\ee
satisfying the equation
\be
4\Ps+\de_{\beta \beta }\Ps=0,
\ee
with 
\be\label{def:L}
\cL (\Omega)(R):=\frac 1 \pi\int_R^\infty \int_0^{2\pi} \frac{\Omega(s, \beta) \sin (2\beta)}{s}\, d\beta \, d s,
\ee
and
\begin{align}
    \label{def:R}
    \mathcal{R}^\alpha (\Omega)(R)= \frac{R^{-\frac{4}{\alpha}}}{4\alpha\pi}\int_0^R \int_0^{2\pi} s^\frac{4}{\alpha} \frac{\Omega(s, \beta) \sin (2\beta)}{s}\, d\beta \, d s.
\end{align}
%
Moreover, the error term 
$\mathcal{R}^\alpha(\Omega)(R)$ satisfies the inequality
\begin{align}\label{eq:hardy}
\|\mathcal{R}^\alpha(\Omega)\|_{H^k} \lesssim  \| \Omega\|_{H^k}.
\end{align}
%
\end{Thm}
\begin{proof}
    The proof is given by Theorem 4.23 and Theorem 4.23 in \cite{drivas}. Notice that the last inequality is uniform in $\alpha$ and its simple but refined proof based on the proof of the Hardy inequality is given in \cite[Lemma 4.25]{drivas}.
\end{proof}
%%%%%
\begin{comment}
\begin{Rmk}
Consider the projection on $\sin(2\beta)$ for functions in $L^2(0,2\pi)$ through the $L^2$-scalar product, namely 
\begin{equation*}
   \left\langle \cdot,\sin(2\beta)\right\rangle:= \frac{1}{\pi}\int_0^{2\pi}\cdot \sin(2\beta)d \beta.
\end{equation*}
If $\Omega(R,\beta)=f(R)\sin(2\beta)$, then $\left\langle \Omega,\sin(2\beta)\right\rangle=f(R)$. Note that it follows for such $\Omega$ that 
\begin{equation*}
    \cL (\Omega)(R)=\int_R^{\infty}\frac{f(s)}{s}d s,
\end{equation*}
where $\cL$ as defined in \eqref{def:L}. In \cite[Theorem 2]{tarek2}, the authors state the decomposition \eqref{eq:psi-main} for $\Psi$ for data of that structure.
\textcolor{red}{modificare o cancellare questo remark: aggiungere discorso sul periodo delle funzioni e le loro estensioni?}
\end{Rmk}
\end{comment}
%%%%%

\noindent To construct our leading order model, we plug the formula for $\Psi$  \eqref{eq:psi-main} into the expressions of $u_1, u_2$ in \eqref{eq:u1}-\eqref{eq:u2}  and we extract the following main order terms :
\be\label{eq:approx-vel}
\ba
u_1 = - \frac{2 R^\frac 1 \alpha \cos \beta}{4 \alpha} \cL (\Omega) +\text{l.o.t.}, \qquad u_2 =  \frac{2 R^\frac 1 \alpha \sin \beta}{4 \alpha} \cL(\Omega) +\text{l.o.t.}.
\ea
\ee
Turning to the derivatives,
\be\label{eq:approx-der-vel}
\de_x u_1= - \de_y u_2 = - \frac{2 \cos ^2 \beta}{4 \alpha} \cL(\Omega) -  \frac{2\sin ^2 \beta}{4 \alpha} \cL(\Omega) +\text{l.o.t.}=  - \frac{2\cL(\Omega)}{4 \alpha} +\text{l.o.t.},
\ee
while
\be
\de_x u_2 = \text{l.o.t}, \qquad \de_y u_1 = \text{l.o.t}. 
\ee
In the right-hand side of the equation for $\de_x \rho$ in \eqref{eq:2dbouss-grad}, $\de_x u_2 (1-\de_y \rho)$ is $\text{l.o.t}$ and therefore negligible, as well as  $-\de_y u_1 (\de_y \rho)$ in the right-hand side of the equation for $\de_y \rho$.
%
Introducing the variables 
%
\be \label{eq:variables}
\eta (t, R, \beta)= \de_x \rho (t, x, y), \quad \xi (t, R, \beta)=\de_y \rho (t, x,y)
\ee
%
such approximation yields the following leading order model :
%%%%%
\be\label{eq:model-new-coord}
\ba
\de_t \Omega + \left( - (\alpha R \de_\beta \Psi ) \de_R  + (2\Psi + \alpha R \de_R \Psi) \de_\beta \right) \Omega & = \eta, \\
\de_t \eta + \left( - (\alpha R \de_\beta \Psi ) \de_R + (2\Psi + \alpha R \de_R \Psi) \de_\beta \right) \eta  & = - (\de_x u_1) \eta, \\
\de_t \xi + \left( - (\alpha R \de_\beta \Psi ) \de_R + (2\Psi + \alpha R \de_R \Psi) \de_\beta \right) \xi  & = \de_x u_1 (\xi -1). \\
\ea
\ee
%%%%%
We further reduce the model by plugging the expression of $\Psi$ in \eqref{eq:psi-main} into the transport terms and keeping only 
the leading order terms in $\alpha$. This yields the following approximate Leading Order Model 
\be\label{eq:leading1}\tag{LOM}
\ba
\de_t \Ome  +  \frac{\cL (\Ome)}{2\alpha} \sin (2\beta)   \de_\beta \Ome &= \et, \\
\de_t \et +  \frac{\cL (\Ome)}{2\alpha} \sin (2\beta)  \de_\beta \et &= \left(\frac{\cL(\Ome)}{2 \alpha} \right)\et, \\
\de_t \csi +  \frac{\cL (\Ome)}{2\alpha} \sin (2\beta)  \de_\beta \csi &= \left(\frac{\cL(\Ome)}{2 \alpha} \right)(1-\csi). 
\ea
\ee
%%%%%
Notice that the equations for $\et$ and $\csi$ are \emph{linear}. We then solve the equation for $\et$. The strategy is to select a suitable (radial) initial datum $\bar \eta_0^{\alpha, \delta}(R)$ such that, when plugging the explicit expression of the evolution $\et(t, R, \beta)$ in (the right-hand side of) the equation for $\Ome$ in \eqref{eq:leading1}, we recover the leading order model of the scalar equation for which Elgindi and Shikh Khalil \cite{tarek2} established ill-posedness in $L^\infty$. 
To this end, we let $\et(0, R, \beta)= \bar \eta_0^{\alpha, \delta} (R)$ be a radial function. Then the transport equation
\begin{equation}\label{eq:eta}
\de_t \et + \frac{\cL (\Ome)}{2\alpha} \sin (2\beta)  \de_\beta \et =  \left(\frac{\cL(\Ome)}{2 \alpha} \right)\et, \qquad \et(0, R, \beta)= \bar \eta_0^{\alpha, \delta} (R)
\end{equation}
admits the following solution:
\be\label{eq:eta-sol}\et(t, R, \beta)= \bar \eta_0^{\alpha, \delta} (R) \exp\left(  \frac{1}{2\alpha} \int_0^t {\cL(\Ome (\tau))}\, d\tau\right).\ee
Let us now plug the above formula inside the equation for $\Omega_{\mathrm{app}}$ in \eqref{eq:leading1}, yielding
\be\label{eq:omega}
\de_t \Ome + \frac{\cL (\Ome)}{2\alpha} \sin (2\beta)   \de_\beta \Ome=\bar \eta_0^{\alpha, \delta} (R) \exp\left( \frac{1}{2\alpha} \int_0^t {\cL(\Ome (\tau))} \, d\tau\right).
\ee
%%%%%
\begin{Lem}
Let $\Ome (t, R, \beta)$ be a solution to \eqref{eq:omega} with initial datum $\Omega_{0, \mathrm{app}}^{\alpha, \delta}(R, \beta)=\bar g_0^{\alpha, \delta}(R) \sin (2\beta)$. If we set
\be\label{eq:def-g}
\Ome (t)= g(t) + \bar \eta_0^{\alpha, \delta}(R) \int_0^t \exp\left( \frac{1}{2\alpha} \int_0^\tau {\cL(\Ome (\mu))}\, d\mu \right)\, d\tau,
\ee
then $g(t, R, \beta)$ solves the following transport equation:
\be\label{eq:g}
\de_t g +  \frac{\cL (g)}{2\alpha} \sin (2\beta)   \de_\beta g=0, \qquad g(0)= \bar g_0^{\alpha, \delta} (R).
\ee
\end{Lem}
%
\begin{proof}
The elementary but key observation is that, by definition of the integral operators $\cL(\cdot), \mathcal{R}(\cdot)$ in \eqref{def:L}-\eqref{def:R}, since $\bar \eta_0^{\alpha, \delta}(R) \int_0^t \exp\left( \frac{1}{2\alpha} \int_0^\tau {\cL(\Ome (\mu))}\, d\mu \right)\, d\tau$ is a radial function (independent of $\beta$), it holds that
\begin{equation}\label{eq:Lom-Lg}\cL (\Ome) = \cL \left(g +  \bar \eta_0^{\alpha, \delta}(R) \int_0^t \exp\left( \frac{1}{2\alpha} \int_0^\tau {\cL(\Ome (\mu))}\, d\mu \right)\, d\tau\right)= \cL(g).%=-\cL(g).
\end{equation}
\end{proof}
%%%%%
The key fact now is that Eq. \eqref{eq:g} has been widely studied in \cite{tarek2} and we can rely on the following set of known facts.  
%%%%%
\begin{Lem}[List of known facts, from {Lemma 3.1 and Proposition 3.3}, \cite{tarek2}]\label{prop:LOM}
Let us consider the Cauchy problem for \eqref{eq:g} with initial datum $g(0)=\bar g_0^{\alpha, \delta}(R) \sin(2\beta)$. The following hold. 
\begin{enumerate}
\item The solution $g(t)$ reads as follows:
\be\label{eq:g-formula}
g(t)=g(t)(t, R, \beta)= \bar g_0^{\alpha, \delta}(R)\frac{2(\tan \beta) \exp\left(-\frac 1 \alpha \int_0^t \cL (g(\tau)) \, d \tau \right)}{1+(\tan^2\beta) \exp\left(-\frac 2 \alpha \int_0^t \cL (g(\tau)) \, d \tau \right)}.
\ee
\\
\item For some $c_1>0, c_2>0$ independent of $\alpha$, the lower and upper bound below are fulfilled
\be\label{eq:Lg-bounds}
c_1 \int_R^\infty \frac{\bar g_0^{\alpha, \delta}(s)}{s} \,  \exp\left(-\frac 1 \alpha \int_0^t \cL(g(\tau)) \, d\tau\right)\, ds \le \cL(g(t))(R) \le c_2 \int_R^\infty \frac{\bar g_0^{\alpha, \delta}(s)}{s} \,  \exp\left(-\frac 1 \alpha \int_0^t \cL(g(\tau)) \, d\tau\right)\, ds,
\ee
\\
\item the following estimates hold
\be\label{eq:g-upperandlower}
\frac{2\alpha}{c_2} \log \left(1+\frac{c_2}{2\alpha} t \cG_0(R)\right) \le \int_0^t \cL(g(\tau))(R) \, d\tau  \le \frac{2\alpha}{c_1} \log \left(1+\frac{c_1}{2\alpha} t \cG_0(R) \right),
\ee
where
$$ \cG_0(R) = \cG_0(\bar g_0^{\alpha, \delta})(R) = \int_R^\infty \frac{\bar g_0^{\alpha, \delta}(s)}{s} \, ds.$$\\
\end{enumerate}
\end{Lem}
\begin{proof}
Item (1) is readily verified by plugging formula \eqref{eq:g-formula} into equation \eqref{eq:g}. Let us consider item (2). By definition \eqref{def:L} and formula \eqref{eq:g-formula},
\be\ba
\cL(g(t))&= \frac 4 \pi \int_R^\infty  \frac{\bar g_0^{\alpha, \delta}(s)}{s} \int_0^{2\pi} \frac{(\sin^2 \beta) \exp\left(-\frac 1 \alpha \int_0^t \cL(g(\tau)) \, d\tau\right)}{1+(\tan^2 \beta) \exp\left(-\frac 2 \alpha \int_0^t \cL(g(\tau)) \, d\tau\right)} \, d\beta\, ds \\
& =  4 \int_R^\infty  \frac{\bar g_0^{\alpha, \delta}(s)}{s} \frac{\exp\left(-\frac 1 \alpha \int_0^t \cL(g(\tau)) \, d\tau\right)}{\left(1+\exp\left(-\frac 1 \alpha \int_0^t \cL(g(\tau)) \, d\tau\right)\right)^2} \, ds.
\ea\ee
From the last line, one notices in particular that $\cL(g(t)) \ge 0$ if $\bar g_0^{\alpha, \delta}\ge 0$, from which the estimates of (2) follow. Item (3) is proved in [Lemma 3.2, \cite{tarek2}]. The idea is to verify that the operator 
\be
\cG_t (g(t) ) :=  \int_R^\infty \frac{\bar g_0^{\alpha, \delta}(s)}{s} \,  \exp\left(-\frac 1 \alpha \int_0^t \cG_\tau (g(\tau)) \, d\tau\right)\, ds
\ee
satisfies the equation
$$\de_t  \cG_t (g(t) )  = - \frac{(\cG_t (g(t) ))^2}{\alpha} \quad \Rightarrow \quad \cG_t (g(t) )=\frac{ \cG_0 (\bar g_0^{\alpha, \delta})(R)}{1+\frac t \alpha   \cG_0 (\bar g_0^{\alpha, \delta})(R)},
$$
from which one has the desired estimates after integrating in time.
\end{proof}
%%%%%
\subsection{A concrete example of a sequence of initial data}\label{sec:initialdata}
In our Leading Order Model \eqref{eq:leading1}, and in particular in the equation for $\et$ in \eqref{eq:eta}, we set up the initial datum $\et (0, R, \beta)=\bar \eta_0^{\alpha, \delta}(R)$ to be a radial function. Recalling from \eqref{eq:variables} that $\eta (t, R, \beta)=\de_x \rho (t, x, y)$, here we provide an example of suitable choice of initial data for the \eqref{eq:leading1}. Let us set 
\be\label{eq:initial-rho}
\rho (0, x, y)= \rho_0^{\alpha, \delta}(x,y)= R^{1/\alpha} \bar \eta_0^{\alpha, \delta} (R) \cos \beta,
\ee
where $\bar \eta_0^{\alpha, \delta} (R)$ is the ($\alpha, \delta $-sequence of) initial data for $\eta(t, R, \beta)$. With this choice, as 
$$\de_x \; \Rightarrow \; R^{-\frac 1 \alpha} (\alpha (\cos \beta) R \de_R - (\sin \beta) \de_\beta ),$$
one has that
\be
\de_x  \rho_0^{\alpha, \delta} (x, y) = \bar \eta_0^{\alpha, \delta}(R) + \alpha  R \de_R\bar \eta_0^{\alpha, \delta}(R) \cos^2(\beta). 
\ee
Concerning the initial data for the Leading Order Model \eqref{eq:leading1}, the choice in \eqref{eq:eta} consists in taking $$\de_x \rho_{\mathrm{app}} (0, \cdot)= \de_x \rho_{0,\mathrm{app}}^{\alpha, \delta}  (\cdot) = \bar \eta_0^{\alpha, \delta} (R),$$
which, in other words, means that we make an error at the initial time that is
\be
\de_x \rho_r (0, \cdot):=\de_x \rho_0^{\alpha, \delta} (\cdot)- \de_x \rho_{0, \mathrm{app}}^{\alpha, \delta} (\cdot)=\alpha  (\de_R \bar  \eta_0^{\alpha, \delta})(R) \cos^2 \beta. 
\ee
We estimate the size of such error
\be\label{eq:data eta rem}
\|\de_x \rho_r (0)\|_{\cH^k} = \|\et (0)\|_{\cH^k} \le \alpha \|\bar \eta_0^{\alpha, \delta}\|_{\cH^{k+1}}.
\ee
The (sequence of) initial data $\bar \eta_0 ^{\alpha, \delta}(R)$ is chosen such that there exists $\delta>0$ independent of $\alpha$ for which
%
\be
\|\bar \eta_0^{\alpha, \delta}\|_{L^\infty} \lesssim \delta,
\ee
but, for $k \ge 3$
\be
\|\bar \eta_0^{\alpha, \delta}\|_{H^k} \sim \delta |\log|\log \alpha||^{\mu_k}, \quad \|\bar \eta_0^{\alpha, \delta}\|_{H^{k+1}} \sim  \delta |\log|\log \alpha||^{\mu_{k+1}}
\ee
for some $0< \mu_k, \mu_{(k+1)} <  1$.
For instance, one could set
\be
\bar \eta_0^{\alpha, \delta} (R)= {\delta}\left(\phi(R)+{|\log |\log \alpha||^{\frac{1-k}{4}}} \phi((R-1)|\log |\log \alpha||^\frac 14) \right),
\ee
where $\phi (R) \in C_c^\infty ([1,3])$ and $0 \le \phi, \phi', \cdots, \phi^{(k)} \le 1$.
This way, 
\begin{align*}
    \supp \bar \eta_0^{\alpha, \delta}(R) \subset \{R : \,  1+|\log|\log \alpha||^{-\frac 1 4} \le R \le 1+3|\log|\log \alpha||^{-\frac 1 4}\},
\end{align*}
where $\phi (R) \in C_c^\infty ([1,3])$ and $0 \le \phi, \phi' \le 1$.
Taking the derivatives, as $\alpha \to 0$,
\begin{equation*}
    \|\bar \eta_0^{\alpha, \delta}\|_{H^k} \sim  \delta |\log|\log \alpha||^{\frac 14 }, \quad  \|\bar \eta_0^{\alpha, \delta}\|_{H^{k+1 }} \sim \delta |\log|\log \alpha||^{\frac 12}.
\end{equation*}
Hereafter, we will use the following notation 
\begin{align}\label{eq:size-initial}
    C_k:= C \delta |\log|\log \alpha||^{\frac 14 }; \qquad C_{k+1}:=C \delta |\log|\log \alpha||^{\frac 12 },
\end{align}
which, up to a change of the (uniform) constant $C>0$, encodes the size of the initial data. 
Finally, let us look at the initial datum for $\xi=\de_y \rho$. As
$$\de_y \; \Rightarrow \; R^{-\frac 1 \alpha} (\alpha (\sin \beta) R \de_ R + (\cos \beta) \de_\beta),$$
the choice \eqref{eq:initial-rho} provides a cancellation of the main order term, leading to
\be
\de_y \rho (0, \cdot) = \alpha R(\de_R \bar \eta_0^{\alpha, \delta}(R)) \sin \beta \cos \beta.
\ee
For the Leading Order Model \eqref{eq:leading1}, we then set 
%
\be\label{eq:xi-initial}
\csi (0, R, \beta)= \de_y \rho (0, \cdot),
\ee
so that the error at initial time $\de_y \rho (0, \cdot) = \xi_r(0, \cdot)=0$, and
%
\be
\|\csi (0)\|_{\cH^k}=\|\de_y \rho (0, \cdot)\|_{\cH^k} \sim \alpha \| \bar \eta_0^{\alpha, \delta}\|_{\cH^{k+1}} \sim \delta \alpha |\log|\log \alpha||^{\frac 12 }.
\ee
%%%%%
\section{Estimates on the Leading Order Model}\label{sec:estimateLOM}
The following result yields $L^\infty(\R^2)$ ill-posedness of our Leading Order Model \eqref{eq:leading1}. Afterwards, this section provides $\cH^k$ estimates on the Leading Order Model \eqref{eq:leading1} for initial data $(\Omega_{0, \mathrm{app}}, \eta_{0, \mathrm{app}}, \xi_{0, \mathrm{app}})$ where $\eta_{0, \mathrm{app}}=\bar \eta_{0}^{\alpha, \delta}(R)$ is a radial function, for instance as in Section \ref{sec:initialdata}. First, the ill-posedness of the \eqref{eq:leading1} is stated below.
%%%%%
\begin{Prop}\label{prop:expl}
There exist $0<\alpha_0 \ll 1$ and $\alpha_0 \ll \delta \ll 1$ such that, for any $0<\alpha \le \alpha_0 $, there exist initial data $(\Omega_{0, \mathrm{app}}^{\alpha, \delta}(R, \beta), \eta_{0, \mathrm{app}}^{\alpha, \delta}(R, \beta), \xi_{0, \mathrm{app}}^{\alpha, \delta}(R, \beta))$ of the following form 
\begin{align}
 \Omega_{0,\mathrm{app}}^{\alpha, \delta} (R, \beta) = \bar g_0^{\alpha, \delta} (R) \sin (2\beta), \qquad \eta_{0, \mathrm{app}}^{\alpha, \delta} (R, \beta)= \bar \eta_0^{\alpha, \delta}(R), \qquad  \xi^{\alpha, \delta}_{0, \mathrm{app}}(R, \beta)= \frac{\alpha}{2} R  (\de_R\bar \eta_0^{\alpha, \delta}(R)) \sin (2\beta),
\end{align}
where $(\bar g_{0, \mathrm{app}}^{\alpha, \delta} (R), \bar \eta_{0, \mathrm{app}}^{\alpha, \delta} (R) ) \in C_c^\infty([1, \infty))$ with
$$\|(\Omega_{0, \mathrm{app}}^{\alpha, \delta}, \eta_{0, \mathrm{app}}^{\alpha, \delta}, \xi_{0, \mathrm{app}}^{\alpha, \delta})\|_{L^\infty(\R^2)} \sim  \delta,$$
such that the corresponding solution $(\Ome^{\alpha, \delta}(t), \eta_\mathrm{app}^{\alpha, \delta}(t), \xi_\mathrm{app}^{\alpha, \delta}(t))$ to the Cauchy problem associated with \eqref{eq:leading1} satisfies
%
\be\ba
\|\et^{\alpha, \delta}(t) \|_{L^\infty(\R^2)}  &\ge \|\eta_{0, \mathrm{app}}^{\alpha, \delta}\|_{L^\infty (\R^2)} \left(1+\frac{c_2 t}{2\alpha}C_0\right)^{\frac{1}{c_2}},
\ea\ee
where $C_0=\sup\limits_{R \in \mathrm{supp}(\bar g_0^{\alpha, \delta}(R))} \int_R^\infty \frac{\bar g_0^{\alpha, \delta}(s)}{s} \, ds$ and $c_2>0$ is independent of $\alpha$. In particular, this yields that 
\be\ba
\sup\limits_{t \in [0, T^*(\alpha)]} \|\et^{\alpha, \delta}(t) \|_{L^\infty(\R^2)}  &\ge  \|\eta_{0, \mathrm{app}}^{\alpha, \delta}\|_{L^\infty (\R^2)} \left(1+\frac{\log |\log \alpha|}{C_{N+1}}\right)^{\frac{1}{c_2}},
\ea\ee
for $T^*(\alpha)\sim  \frac{\alpha}{C_{N+1}}\log |\log (\alpha)| \, \to \, 0$ as $\alpha \to 0$ and such that $\frac{\log |\log \alpha|}{C_{N+1}} \gtrsim \log |\log \alpha|^\mu$ for some $0<\mu <1$. 
\end{Prop}
%%%%%
\begin{comment}
\textcolor{red}{If needed, one could take $\xi_{0, \mathrm{app}}^{\alpha, \delta}=0$ and $\xi_{0,r}^{\alpha, \delta}=\xi_0$ given that $\xi_0$ is small (to check!). This would allow for an explicit formula for $\csi$.}
\end{comment}
\begin{proof}[Proof of Proposition \ref{prop:expl}]
The lower bound for $\eta$ is a direct consequence of the explicit formula \eqref{eq:eta-sol}, namely
\be
\eta(t, R, \beta)= \bar \eta_0^{\alpha, \delta} (R) \exp\left(  \frac{1}{2\alpha} \int_0^t {\cL(\Omega (\tau))}\, d\tau\right),
\ee
and the lower bounds \eqref{eq:g-upperandlower} for $g(t)$ satisfying \eqref{eq:g}. Then the growth in $\alpha$ simply follows from the choice of $T^*(\alpha)$. The proof is concluded.
\end{proof}
%%%%
\begin{Rmk}[$\Omega$ does not blow up]
    One has from the explicit for $g(t)$ in \eqref{eq:g-formula} and the estimates of Proposition \ref{prop:LOM} that
\begin{align}\label{est:g-inf}
   \|\bar g_0^{\alpha, \delta}\|_{L^\infty} \lesssim \|g(t)\|_{L^\infty} \lesssim \|\bar g_0^{\alpha, \delta}\|_{L^\infty}.
\end{align}
Now, notice that \eqref{eq:def-g} and the estimates of Proposition \ref{prop:LOM} yield 
\begin{align*}
\Ome(t)&=g(t)+\int_0^t\et(\tau)d\tau\le g(t)+ \int_0^t\left(1+\frac{c_2 \tau}{2\alpha}C_0\right)^{\frac{1}{c_2}}d\tau\\
&= g(t)+\frac{2\alpha}{C_0(1+c_2)}\left(\left(1+\frac{t}{2\alpha}C_0c_2\right)^{\frac{c_2+1}{c_2}}-1\right).
\end{align*}
Therefore $\Ome (t)$ does not blow up and, in the time interval $[0, T^*(\alpha)=\frac{\alpha}{C_{N+1}}\log |\log \alpha|]$, the vorticity $\omega (t, \cdot)$ (solving \eqref{eq:2dbouss-grad}) does not blow up as well (it may explode for later times).
\end{Rmk}
%%%%%
\begin{Rmk}[An explicit formula for $\csi$] \label{rmk:explicitcsi}
At the initial time for the \eqref{eq:leading1}, one could replace  \eqref{eq:xi-initial} with  $\xi_{0, \mathrm{app}}^{\alpha, \delta}=0$, so that the error at the initial time would be $\xi_{0,r}^{\alpha, \delta}= \de_y \rho_0^{\alpha, \delta}$ of size $\|\xi_{0,r}^{\alpha, \delta}\| \sim \delta \alpha \log |\log \alpha|^\frac 12$. The advantage of taking $\xi_{0, \mathrm{app}}^{\alpha, \delta}=0$ is that one can easily solve for $\csi$ in \eqref{eq:leading1}, so obtaining the explicit formula 
\be\label{eq:csi-explicit}
\csi (t, \cdot) =  \left(1-\exp\left(-\frac{1}{2\alpha} \int_0^t {\cL(g(\tau))}\, d\tau\right)\right). 
\ee
In turn, using \eqref{eq:g-upperandlower}, the above formula allows to get an upper bound in $L^\infty (\R^2)$, 
\be
\|\csi(t)\|_{L^\infty (\R^2)} = \| \de_y \rho_{\mathrm{app}}(t)\|_{L^\infty} \le 1+ \left(1+ \frac{c_2 t}{2\alpha} c_0 \right)^{-\frac{1}{c_2}}, 
\ee
with $c_2$ as in Proposition \ref{prop:LOM} and $c_0=\inf\limits_{R \in \mathrm{supp}(\bar g_0^{\alpha, \delta}(R))} \int_R^\infty \frac{\bar g_0^{\alpha, \delta}(s)}{s} \, ds$.
In the time interval $[0, T^*(\alpha)]$ as in Proposition \ref{prop:expl}, one has that
\be
\|\csi(t)\|_{L^\infty (\R^2)} = \| \de_y \rho_{\mathrm{app}}(t)\|_{L^\infty} \le 1+ \left(1+ \frac{\log |\log \alpha|}{C_{N+1}}  \right)^{-\frac{1}{c_2}}  < 3,
\ee
so that $\csi (t, \cdot)$ does not blow up in $L^\infty (\R^2)$ and $\de_y \rho (t, \cdot )$ (solving \eqref{eq:2dbouss-grad}) does not blow up in the time interval $[0, T^*(\alpha)]$. 
\end{Rmk}

%%%%%
 The result below provides $\cH^k$ estimates of the Leading Order Model \eqref{eq:leading1}.
%%%%%
 \begin{Prop}[Estimates on the \eqref{eq:leading1}]\label{lem:35} 
For any $k \ge 3$, let $(\Ome^{\alpha, \delta}(t), \eta_\mathrm{app}^{\alpha, \delta}(t), \xi_\mathrm{app}^{\alpha, \delta}(t))$ be a solution to \eqref{eq:leading1} with initial data $(\Omega_{0, \mathrm{app}}^{\alpha, \delta}, \eta_{0, \mathrm{app}}^{\alpha, \delta}, \xi^{\alpha, \delta}_{0, \mathrm{app}}) \in \cH^k$, where 
%%%%%
\begin{align}
\Omega^{\alpha, \delta}_{0, \mathrm{app}}=\bar{g}_0^{\alpha, \delta} (R)\sin(2\beta),\quad 
\eta^{\alpha, \delta}_{0, \mathrm{app}}= \bar \eta_0^{\alpha, \delta}(R),\quad  \xi^{\alpha, \delta}_{0, \mathrm{app}}=\frac{\alpha}{2} R (\de_R \bar \eta_0^{\alpha, \delta}(R)) \sin (2\beta),
\end{align}
%%%%%
and where $(\bar{g}_0^{\alpha, \delta} (R), \bar{\eta}^{\alpha, \delta}_0(R)) \in C_c^\infty ([1, \infty))$
with $\bar g_0^{\alpha, \delta}(R)\ge 0$. Then, there exist constants $0<C_k=C_k(\|\bar g_0^{\alpha, \delta}\|_{\cH^k}, \|\bar \eta_0^{\alpha, \delta}\|_{\cH^k})$ as defined in \eqref{eq:size-initial} such that the following estimates hold 
\be\ba\label{eq:est-LOM}
\|\Ome(t)\|_{\cH^k}&\lesssim  (\|\bar g_0^{\alpha, \delta}\|_{\cH^k}+\alpha e^{\frac{C_k}{\alpha} t} ) \lesssim C_{k} + \alpha e^{\frac{C_k}{\alpha} t},\\
\|\et(t)\|_{\cH^k} &\lesssim  \|\bar \eta_0^{\alpha, \delta}\|_{\cH^k} e^{\frac{C_k}{\alpha} t} \lesssim C_{k}e^{\frac{C_k}{\alpha} t},\\
\|\csi(t)\|_{\cH^k} &\lesssim  (1+\alpha \|\bar \eta_0^{\alpha, \delta}\|_{\cH^{k+1}}  ) e^{\frac{C_k}{\alpha} t} \lesssim (1+\alpha C_k) e^{\frac{C_k}{\alpha} t}.
\ea
\ee
Moreover, the function $\Ps(\Ome)(R)=\frac{1}{4 \alpha}\cL (\Ome)(R)\sin (2\beta)$, as defined in \eqref{eq:psiapp},
satisfies the estimates below
\be\label{eq:est-psi2}
\|\Ps (\Ome)\|_{\cW^{k+1,\infty}} \le   \frac{C_{k+1}}{\alpha}, \qquad \|\Ps (\Ome)\|_{\cH^{k+1}} \le   \frac{C_{k+1}}{\alpha}.
\ee
\end{Prop}
%
Before providing a proof of Proposition \ref{lem:35}, we state and prove a preliminary result.
%
\begin{Lem}\label{StimeLg}
For any $ k\ge 3$, if $g(t)=g(t)(t, R, \beta)$ fulfills formula \eqref{eq:g-formula} with $g(t)|_{t=0}=\bar g_0^{\alpha, \delta}(R) \in \cH^{k+1}_R$ such that $0<\bar g_0^{\alpha, \delta} (R) \in C_c^\infty ([1, \infty))$, then there exists $0<\alpha_0 \ll 1$ small enough such that the following estimates hold for all $\alpha \le \alpha_0 $
\begin{align}
\|\mathcal{L}(g(t))\|_{L^{\infty}_R}&\leq C_{1} ,\quad \|\mathcal{L}(g(t))\|_{L^{2}_R}\leq C_{0},\label{leggere10}\\ 
\|(1+R)\partial_R\mathcal{L}(g(t))\|_{L^{\infty}_R}&\leq C_{2},\quad \| (1+R)\partial_R\mathcal{L}(g(t))\|_{L^{2}_R}\leq C_{1} \label{pesate10},\end{align}
and, for $k \ge 2$, 
\begin{align}
\|\partial_R^{k+1}\mathcal{L}(g(t))\|_{L^{\infty}_R}&\leq C_{k+1},\quad \|\partial_R^{k+1}\mathcal{L}(g)\|_{L^{2}_R}\leq C_{k},\label{leggere1}\\ 
\|R^{k+1}\partial_R^{k+1}\mathcal{L}(g(t))\|_{L^{\infty}_R}&\leq C_{k+1},\quad \| R^{k+1}\partial_R^{k+1}\mathcal{L}(g(t))\|_{L^{2}_R}\leq C_{k}
\label{pesate1},\end{align}
and the function $g(t)$ satisfies
\be\label{eq:est-g}
\|g(t)\|_{\cH^k(\R^2)} \le C_k,
\ee
where the constants $C_k=C_k(\|\bar g_0^{\alpha, \delta}\|_{\cH^k_R})$ depend only (up to a uniform constant in $\alpha$) on the initial data as defined in \eqref{eq:size-initial}.
\end{Lem}
\begin{proof}
By \eqref{eq:Lg-bounds}, it follows that
\be
\cL(g(t)) (R) \le  \int_R^\infty \frac{\bar g_0^{\alpha, \delta}(s)}{s} \exp\left(-\frac 1 \alpha \int_0^t \cL(g(\tau)) \, d\tau \right) \, ds,
\ee
and recalling from \eqref{eq:g-upperandlower} that $\cL(g(t)) \ge 0$ (being $\bar g_0^{\alpha, \delta}(R) >0$), one has, using  the embedding $H^1_R \hookrightarrow L^\infty_R$, that
\be \|\cL(g(t))\|_{L^2_R} \le \left\|\int_R^\infty \frac{\bar g_0^{\alpha, \delta}(s)}{s} \, ds\right\|_{L^2_R} \le  \|\bar g_0^{\alpha, \delta}\|_{L^2_R} \le   C_0,
\ee
\be
\|\cL(g(t))\|_{L^\infty_R} \le  \left\|\int_R^\infty \frac{\bar g_0^{\alpha, \delta}(s)}{s} \, ds\right\|_{L^\infty_R} \le  \|\bar g_0^{\alpha, \delta}\|_{H^1_R} \le    C_1.
\ee
For the derivative $\de_R$, we plug the expression of $g(t)$ in \eqref{eq:g-formula} inside the definition \eqref{def:L}, yielding
%
\be\ba\label{eq:Lg-formula}
\cL(g(t))&= \frac 1 \pi \int_R^\infty \int_0^{2\pi} \frac{4 (\sin^2\beta)  \exp\left(-\frac 1 \alpha \int_0^t \cL (g(\tau)) \, d \tau \right)}{1+(\tan^2\beta) \exp\left(-\frac 2 \alpha \int_0^t \cL (g(\tau)) \, d \tau \right)} \, \frac{\bar g_0^{\alpha, \delta}(s)}{s} \, ds \, d\beta\notag\\
&= 4 \int_R^\infty  \frac{\exp\left(-\frac 1 \alpha \int_0^t \cL(g(\tau)) \, d\tau\right)}{\left(1+ \exp\left(-\frac 1 \alpha \int_0^t \cL(g(\tau)) \, d\tau\right)\right)^2}\frac{\bar g_0^{\alpha, \delta}(s)}{s} \, ds.
\ea\ee
Applying $\de_R$ yields 
\be\label{eq:der-Lg}
\de_R \cL(g(t))= - \frac{4 \bar g_0^{\alpha, \delta}(R)}{R} \frac{\exp\left(-\frac 1 \alpha \int_0^t \cL(g(\tau)) \, d\tau\right)}{\left(1+ \exp\left(-\frac 1 \alpha \int_0^t \cL(g(\tau)) \, d\tau\right)\right)^2},
\ee
from which one has, using again the embedding $H^1_R \hookrightarrow L^\infty_R$, 
%
\be
\|\de_R \cL(g(t))\|_{L^\infty_R} \le  4\|\bar g_0^{\alpha, \delta}\|_{H^1_R} \le C_1.
\ee
%
Next, let us consider
\be
\ba
\|\de_R^2 \cL(g(t))\|_{L^\infty_R} & \lesssim \left\| \frac{\exp\left(-\frac 1 \alpha \int_0^t \cL(g(\tau)) \, d\tau\right)}{\left(1+ \exp\left(-\frac 1 \alpha \int_0^t \cL(g(\tau)) \, d\tau\right)\right)^2}\right\|_{L^\infty_R} \\
& \qquad \times \left(\left\| \de_R\left(\frac{\bar g_0^{\alpha, \delta}(R)}{R} \right)\right\|_{L^\infty}+\frac 1 \alpha  \left\|\frac{\bar g_0^{\alpha, \delta}(R)}{R} \, \left(\int_0^t \de_R \cL(g(s)) \, ds \right)\right\|_{L^\infty_R} \right)\\
& \lesssim \left\|\exp\left(-\frac 1 \alpha \int_0^t \cL (g(\tau)) \, d \tau \right)\right\|_{L^\infty_R}\\ &\qquad \times \left(\left\| \de_R\left(\frac{\bar g_0^{\alpha, \delta}(R)}{R} \right)\right\|_{L^\infty}+\frac 1 \alpha  \left\|\frac{\bar g_0^{\alpha, \delta}(R)}{R} \, \left(\int_0^t \de_R \cL(g(s)) \, ds \right)\right\|_{L^\infty_R} \right)\\
& \lesssim  \left\|\exp\left(-\frac 1 \alpha \int_0^t \cL (g(\tau)) \, d \tau \right)\right\|_{L^\infty_R}\left(\|\bar g_0^{\alpha, \delta}\|_{\cH^2_R} + \frac 1 \alpha  \|\bar g_0^{\alpha, \delta}\|_{\cH^1_R} \left\|\left(\int_0^t \de_R \cL(g(s)) \, ds \right)\right\|_{L^\infty_R}\right) \\
&\le  \left\|\exp\left(-\frac 1 \alpha \int_0^t \cL (g(\tau)) \, d \tau \right)\right\|_{L^\infty_R}\left( C_2 + \frac{C_1}{\alpha} \left\|\left(\int_0^t \de_R \cL(g(s)) \, ds \right)\right\|_{L^\infty_R}\right).
\ea\notag
\ee
Plugging \eqref{eq:der-Lg} into the above and using \eqref{eq:g-upperandlower}, for $0 < \alpha \ll \alpha_0 $ small enough one has 
\be
\ba
\|\de_R^2 \cL(g(t))\|_{L^\infty_R} & \le  \frac{ C_1}{\alpha} \left\| \int_0^t \exp\left(-\frac 1 \alpha \int_0^\tau \cL(g(s)) \, ds \right)\right\|_{L^\infty_R}\le  \frac{C_1}{\alpha},
\ea
\ee
thanks to the (positive) sign of $\cL (g(s))$ due to the fact that at the initial time $\bar g_0^{\alpha, \delta}(R)\ge 0$, which is ensured by the estimates \eqref{eq:Lg-bounds}.

Furthermore, using Fa\`a di Bruno formula, yields that
\be
\ba
\|\de_R^{k+1} \cL(g(t))\|_{L^\infty_R}= 4\left\|  \sum_{\ell=0}^{k} \de^{k-\ell}_R \left(\frac{\bar g_0^{\alpha, \delta}(R)}{R}\right) \de_R^\ell \left(\frac{\exp\left(-\frac 1 \alpha \int_0^t \cL(g(\tau)) \, d\tau\right)}{\left(1+ \exp\left(-\frac 1 \alpha \int_0^t \cL(g(\tau)) \, d\tau\right)\right)^2}\right) \right\|_{L^\infty_R}\\
\le  4 \sum_{\ell=0}^{k}\left\|\partial_R^{k-\ell}\left(\frac{\bar{g}_0(R)}{R}\right)\exp\left(-\frac{1}{2\alpha}\int_0^t\mathcal{L}(g(\tau))d\tau\right)\sum_{\nu=1}^\ell \frac{(2\alpha)^{-\nu}}{\nu!}\sum_{h_1+\cdots h_{\nu}=\ell}\prod_{j=1}^\nu\int_0^t \partial_R^{h_j}\mathcal{L}(g(\tau))d\tau\right\|_{L^\infty_R}
\le C_{k+1},
\ea
\ee
by a finite number of recursive substitutions and the positive sign of $\cL (g (\tau))$. The $L^2$-based estimates are obtained by the same procedure, replacing the constant $C_{k+1}$ by $C_{k}=C_{k}(\|\bar g_0^{\alpha, \delta}\|_{\cH^{k}_R})$ in \eqref{eq:size-initial}. In fact, notice that $L^\infty$-based estimates lose one additional derivative with respect to the $L^2$-based estimates because of the embedding $H^1_R \hookrightarrow L^\infty_R$.
Finally, the proof of estimate \eqref{eq:est-g} follows the same lines using the explicit formula \eqref{eq:g-formula}.
%%%
\begin{comment}
From \eqref{def:L}, \eqref{eq:Lom-Lg}, one has that
$$\partial_R^j\mathcal{L}(g)=-\frac{1}{\pi}\int_0^{2\pi}\partial_R^{j-1}\Big(\frac{g(R,\beta)}{R}\Big)\sin(2\beta) d\beta.$$
We prove the $L^{\infty}$ bound in \eqref{pesate}
\begin{align*}
\|R^j\partial_R^j \cL\|_{L^{\infty}}&\leq  C_j\sum_{i=1}^{j-1}\left\|\int_0^{2\pi}R^i\partial_R^ig(R,\beta)\sin2\beta d\beta\right \|_{L^{\infty}}\leq C_j\sum_{i=0}^{j-1}\|\|\sin{2\beta}\|_{L^2_{\beta}}\|R^{i}\partial_R^ig(R,\beta)\|_{L^2_{\beta}}\|_{L^{\infty}_R}\\
&\leq C_j\sum_{i=0}^{j-1}\|\|R^i\partial_R^i g(R,\beta)\|_{L^2_{\beta}}\|_{H^1_{R}}\leq C_j \|g\|_{\cH^j},
\end{align*}
where $C_{j}$ is an harmless constants changing from line to line and where in the penultimate inequality we used the embedding $H^1_R\hookrightarrow L^{\infty}$.
Concerning the $L^{\infty}$ bound in  \eqref{leggere}, we proceed analogously: using the Sobolev embedding $L^{\infty}_R\hookrightarrow H^1_R$, $R\geq 1$ on the support of $g$,  we obtain
\begin{equation*}
\|\partial_R^j\mathcal{L}(g)\|_{L^{\infty}}\leq C \sum_{i=0}^{j-1}\left \|\int_0^{2\pi}\partial_R^ig(R,\beta)\sin(2\beta)d\beta\right\|_{L^2_R}+\left\|\int_0^{2\pi}\partial_R^{i+1}g(R,\beta)\sin(2\beta)d\beta\right\|_{L^2_R},
\end{equation*}
one concludes using Cauchy-Schwartz inequality, as done before for the proof of \eqref{pesate}.
\end{comment}
\end{proof}




\begin{proof}[Proof of Proposition \ref{lem:35}]
%We omit the subscript $ `` \text{app}''$ and the apex $\alpha, \delta$ in the course of the proof. 
 First, recalling that $\bar g_0^{\alpha, \delta}(R)\ge 0,$ the proof of the $\cH^k$ estimate \eqref{eq:est-psi2} on $\Ps$ follows directly from the upper bound on $\cL(\cdot)$ in \eqref{eq:Lg-bounds} and Lemma \ref{StimeLg}.
Next, we start with the estimate for $\et$. Taking the scalar product of the equation for $\et$ against $\et$, one gets
\be\ba
 \frac{d}{dt} \|\et\|_{L^2}^2 \le 2 \|\Ps\|_{L^\infty} \|\et\|_{L^2}^2 \le \frac{C_k}{\alpha}  \|\et\|_{L^2}^2,
\ea\ee
where we used the estimate of $\Ps$ in \eqref{eq:est-psi2}. About the $\cH^k$ estimate, notice by Leibniz formula that the worst term is given by
\begin{align*}
    \langle \de_t \partial^k \et,  \partial^k \et\rangle_{L^2} + \langle \de^k \Ps \de_\beta \et, \de^k \et \rangle_{L^2}= \alpha^{-1} \langle \de^k \cL(\Ome) \et, \de^k \et\rangle_{L^2}.
\end{align*}
Again, by means of the $\cH^k$ estimate of $\Ps$ in \eqref{eq:est-psi2}, one has that
\be
\frac{d}{dt} \|\et\|_{\cH^k}^2 \le 2 \|\Ps\|_{\cH^k} \|\et\|_{\cH^k}^2 \le \frac{C_k}{\alpha}  \|\et\|_{\cH^k}^2,
\ee
yielding the desired estimate for $\et$. 
%%%%%
\begin{comment}
Then, we focus on $\eta$. By \eqref{eq:eta-sol}  and \eqref{def:L} we note that $\eta$ is a radial function. By using  \eqref{eq:Lom-Lg}, Leibniz rule and the formula of Fa\`a di Bruno, we obtain
\begin{equation}\label{eq:conto}
\begin{aligned}
\|\partial_R^k\eta\|_{L^2}&\leq C_k \sum_{i=0}^k\left\|\partial_R^{k-i}\bar{\eta}_0(R)\partial_R^i \exp\left(\frac{1}{2\alpha}\int_0^t\mathcal{L}(g(\tau))d\tau\right)\right\|_{L^2}\\
&\leq C_k \sum_{i=0}^k\left\|\partial_R^{k-i}\bar{\eta}_0(R)\exp\left(\frac{1}{2\alpha}\int_0^t\mathcal{L}(g(\tau))d\tau\right)\sum_{\nu=1}^i \frac{(2\alpha)^{-\nu}}{\nu!}\sum_{h_1+\cdots h_{\nu}=i}\prod_{j=1}^\nu\int_0^t \partial_R^{h_j}\mathcal{L}(g(\tau))d\tau\right\|_{L^2}
\end{aligned}
\end{equation}
with the convention that the sum $\sum_{\nu=1}^i\dots=1$ if $i=0$.
We note that \eqref{eq:g-upperandlower} yields that
\begin{equation*}
    \left\|\exp\left(\frac{1}{2\alpha}\int_0^t\mathcal{L}(g(\tau))d\tau\right)\right\|_{L^{\infty}}\leq \left\|\left(1+\frac{c_1}{2\alpha}t \cG_0(R)\right)^{\frac{1}{c_1}}\right\|_{L^{\infty}}\leq e^{\frac{C_1}{\alpha} t},
\end{equation*}
where $C_1=C_1(\|\bar g_0^{\alpha, \delta}\|_{\cH^k})$ as in \eqref{eq:size-initial} and where we used that $\|\cG_0(R)\|_{L^{\infty}}\leq \|\bar g_0^{\alpha, \delta}\|_{H^1}$.
In order to estimate $\|\partial_R^k\eta\|_{L^2}$, we shall distinguish two cases. Let $i=0$ in \eqref{eq:conto}, we just have 
\begin{equation*}
    \left\|\partial_R^k\bar{\eta}_0\exp\left(\frac{1}{2\alpha}\int_0^t\mathcal{L}(g(\tau))d\tau\right)\right\|_{L^2}\leq\|\partial_R^k\bar{\eta}_0\|_{L^2}\left\|\left(1+\frac{c_1}{2\alpha}t \cG_0(R)\right)^{\frac{1}{c_1}}\right\|_{L^{\infty}}\leq \|\bar{\eta}_0\|_{\cH^k}e^{\frac{C_1}{\alpha} t}.
\end{equation*}
When $i\geq 1$, we obtain
\begin{align*}
&\sum_{i=1}^k\left\|\partial_R^{k-i}\bar{\eta}_0(R)\exp\left(\frac{1}{2\alpha}\int_0^t\mathcal{L}(g(\tau))d\tau\right)\sum_{\nu=1}^i \frac{(2\alpha)^{-\nu}}{\nu!}\sum_{h_1+\cdots h_{\nu}=i}\prod_{j=1}^\nu\int_0^t \partial_R^{h_j}\mathcal{L}(g(\tau))d\tau\right\|_{L^2}\\&\quad \leq
e^{\frac{C_1}{\alpha} t} \sum_{i=1}^k\left\|\partial_R^{k-i}\bar{\eta}_0(R)\|_{L^{\infty}}\sum_{\nu=1}^i \frac{(2\alpha)^{-\nu}}{\nu!}\int_0^t\|\partial_R^{h_1}\mathcal{L}(g(\tau))\right\|_{L^2}d\tau\prod_{j=2}^\nu\int_0^t \|\partial_R^{h_j}\mathcal{L}(g(\tau))\|_{L^{\infty}}d\tau.
\end{align*}
At this point, we may use the Sobolev embedding $H^1_R\hookrightarrow L^{\infty}_R$ and the last estimates of Lemma \ref{StimeLg} to obtain
$$\|\bar{\eta}_0(R)\|_{H^k} \sum_{i=1}^k\sum_{\nu=1}^i\frac{(2\alpha)^{-\nu}}{\nu!}(t\|g_0\|)^{\nu}_{H^i}\exp\big({\nu \tilde{c}_i \frac{t}{\alpha}}\big)e^{\frac{C_1}{\alpha} t}\le  C_k e^{\frac{C_k}{\alpha} t},$$
where again $C_k$ depends on the initial data as in \eqref{eq:size-initial}.
Similarly one proves the same kind of estimate for $\|R^k\partial_R^k\eta\|_{L^2}$, hence the bound on the $\cH^k$-norms of $\eta$ follow.
\end{comment}
%%%%%
Next, we deal with the estimates for $\csi$. Similarly to what has been done before, one has
\begin{align*}
    \frac{d}{dt}\|\csi\|^2_{\cH^k} \le 2\|  \Ps\|_{\cH^k}\|\csi\|_{\cH^k} (1+\|\csi\|_{\cH^k})\le \frac{C_{k}}{\alpha} \|\csi\|_{\cH^k} (1+\|\csi\|_{\cH^k}),
\end{align*}
where we used the $\cH_k$ estimate \eqref{eq:est-psi2} on $\Ps$.
This yields (simplifying the squares by a classical approximation procedure and Gronwall lemma)
%
\begin{comment}
\textcolor{green}{io scriverei direttamente la cosa integrale, magari citando il lemma di approssimazione che ci aveva mostrato roberta, perché la derivata della norma non esiste}
\be
 \frac{d}{dt}\|\csi\|_{\cH^k}\le \frac{C_{k}}{\alpha} (1+\|\csi\|_{\cH^k}),
\ee
which rewrites as 
\be
 \frac{d}{dt}(1+\|\csi\|_{\cH^k})\le \frac{C_{k}}{\alpha} (1+\|\csi\|_{\cH^k}),
\ee
and therefore
\end{comment}
%
the desired estimate for $\csi$. 
Finally, we deal with $\Ome$.
Recall from \eqref{eq:def-g} that $$\Ome=g + \bar \eta_0^{\alpha, \delta}(R) \int_0^t \exp \left(\frac{1}{2\alpha} \int_0^\tau \cL(g) \, d\mu \right) \, d\tau=g+\int_0^{t}\et(\tau)\, d\tau,$$ where $g$ satisfies \eqref{eq:g}, which can be written as 
$$\de_t g + \frac{\cL(\Ome)}{2\alpha} \sin (2\beta) \de_\beta g=0.$$ 
First, recall the estimate $\|g(t)\|_{\cH^k} \le C_k $ in \eqref{eq:est-g}. By using that $\|\et\|_{\cH^k} \le C_k e^{\frac{C_k}{\alpha}t}$, one has that
\be\ba
\|\Ome (t) \|_{\cH^k} & \le \|g (t)\|_{\cH^k} + \int_0^t \|\et (\tau) \|_{\cH^k} \, d\tau \le C_k + \int_0^t C_k e^{\frac{C_k}{\alpha} \tau} \, d \tau  \le C_k + \alpha e^{\frac{C_k}{\alpha} t}.
\ea\ee
%%%%%
The proof is concluded.
\end{proof}
%%%%%
\subsection{Elliptic estimates}
\noindent This section provides estimates of $\Psi$ solving the elliptic equation \eqref{eq:ellittica}, and more precisely of its decomposition $\Ps, \Psi_2, \Psi_\mathrm{err}$, due to Elgindi \cite{tarek1}, in Theorem \ref{thm:tarek}.
%
\begin{Prop}[Elliptic estimates for $\Ps(\Omega), \Psa(\Omega)$, $\hE(\Omega)$ and $\mathcal{R}(\Omega)$]\label{lam:Rest}
The following estimates hold: 
\begin{align}
\label{eq:elliptic1-rem}
\|\de_{\beta \beta} \hE \|_{H^k}+\alpha \|R \de_{R\beta} \hE\|_{H^k} + \alpha^2 \|R^2 \de_{RR} \hE \|_{H^k} &\lesssim  \|\Omega\|_{\mathcal{H}^{k}},\\
\|\de_{\beta \beta} (R^k \de_R^k\hE) \|_{L^2}+\alpha \|R \de_{R\beta} (R^k \de_R^k \hE)\|_{L^2} + \alpha^2 \|R^2 \de_{RR} (R^k \de_R^k \hE) \|_{L^2} &\lesssim \|R^k \de_R^k \Omega\|_{L^2},
\label{eq:elliptic2-rem}
\end{align}
%
and
\begin{align}\label{eq:crucial}
{\alpha}\|\de_{\beta \beta} \Psa \|_{H^k}+\alpha \|R \de_{R\beta} \Psa\|_{H^k} + {\alpha^2} \|R^2 \de_{RR} \Psa\|_{H^k}  &\lesssim\|\Omega\|_{\mathcal{H}^{k}},\\
{\alpha}\|\de_{\beta \beta} (R^k \de_R^k\Psa) \|_{L^2}+\alpha\|R \de_{R\beta} (R^k \de_R^k \Psa)\|_{L^2}+{\alpha^2}\|R^2 \de_{RR}(R^k \de_R^k \Psa)\|_{L^2} &\lesssim \|R^k \de_R^k \Omega\|_{L^2}.
\end{align}
Finally, denoting $\mathcal{R}(\Omega)=\mathcal{R}^\alpha(\Omega)\sin(2\beta)$ it holds that
\begin{align}\label{eq:est-rem}
\|\de_{\beta \beta} \mathcal{R} \|_{H^k}+{\alpha} \|R \de_{R\beta} \mathcal{R}\|_{H^k} + {\alpha^2}\|R^2 \de_{RR} \mathcal{R} \|_{H^k}  & \lesssim\|\Omega\|_{\mathcal{H}^{k}},\\
\|\de_{\beta \beta} (R^k \de_R^k\mathcal{R} ) \|_{L^2}+{\alpha} \|R \de_{R\beta} (R^k \de_R^k \mathcal{R} )\|_{L^2}+{\alpha^2}\|R^2 \de_{RR}(R^k \de_R^k \mathcal{R} )\|_{L^2} & \lesssim \|R^k \de_R^k \Omega\|_{L^2},
\end{align}
\end{Prop}
\begin{proof}

 First, observe that $\hE$ satisfies
\begin{align}\label{eq:ell-err}
    4\hE + \de_{\beta \beta} \hE + \alpha^2 R^2 \de_{RR}\hE + (4\alpha+\alpha^2) R \de_R \hE = \Omega- \Omega_2.
\end{align}
 As in the proof of \cite[Theorem 4.23]{drivas}, recalling that we have homogeneous boundary conditions, we consider the series expansion
 \begin{align}
     \hE(R, \beta)= \sum_{n=0}^\infty \hE^n(R) \sin (n \beta).
 \end{align}
 %
As the source term $\Omega_\text{err}= \Omega - \Omega_2$ of the elliptic equation for $\hE$ is orthogonal to the 2-modes, then the above sum is actually over $n \ge 3$.
%
Taking now the scalar product of \eqref{eq:ell-err} against $\de_{\beta \beta}\hE(R, \beta)$ in $L^2([0, 2\pi ]\times [0, \infty))$ integrating by parts as in the proof of \cite[Theorem 4.23]{drivas}, and using that $n\ge 3$ then
$$|\de_{\beta} \hE|^2 \le \frac{1}{9} |\de_{\beta \beta} \hE|^2$$
one gets the first two desired estimates. 
%
We now deal with the crucial estimates \eqref{eq:crucial}. From the explicit formula for $\Psa (\Omega)(R, \beta)$ in \eqref{eq:psi2}, exploiting a key cancellation, we obtain that
\begin{align*}
    \de_R \Psa (\Omega)(R, \beta) = -\frac{1}{\pi\alpha^2} \sin (2\beta) R^{-\frac 4 \alpha -1} \int_0^R \int_0^{2\pi} s^\frac{4}{\alpha} \frac{\Omega (s, \beta) \sin (2\beta)}{s} \, d\beta \, ds. 
\end{align*}
This way
\begin{align}
     R\de_{\beta R} \Psa (\Omega)(R, \beta) = -\frac{2}{\pi\alpha^2} \cos (2\beta) R^{-\frac 4 \alpha -1} \int_0^R \int_0^{2\pi} s^\frac{4}{\alpha} \frac{\Omega (s, \beta) \sin (2\beta)}{s} \, d\beta \, ds = - \frac{2}{\alpha} \cos(2\beta) \mathcal{R}^\alpha (\Omega). 
\end{align}
From \eqref{eq:hardy}, we obtain that $\alpha \|R\de_{\beta R} \Psa\|_{L^2} \lesssim  \|\Omega\|_{L^2}.$
%
Next
\begin{align*}
    \de_{RR} \Psa (\Omega)(R, \beta)&= \frac{1}{\pi \alpha^2} \left(\frac{4}{\alpha}+1\right) \sin (2\beta) R^{-\frac{4}{\alpha}-2} \int_0^R \int_0^{2\pi} s^\frac{4}{\alpha} \frac{\Omega (s, \beta) \sin (2\beta)}{s} \, d\beta \, ds \\
    &\quad - \frac{1}{\pi \alpha^2} \sin (2\beta) R^{-2} \int_0^{2\pi} \Omega (s, \beta) \sin (2\beta) \, d\beta,
\end{align*}
yielding 
\begin{align*}
    R^2\de_{RR} \Psa (\Omega)(R, \beta)&= \left(\frac{4}{\alpha}+1\right) \frac{1}{\alpha} \sin (2\beta) \mathcal{R}^\alpha (\Omega) - \frac{1}{\pi \alpha^2} \sin (2\beta) \int_0^{2\pi} \Omega (s, \beta) \sin (2\beta) \, d\beta,
\end{align*}
from which, using \eqref{eq:hardy} again,
\begin{align*}
    \alpha^2\|\de_{RR}\Psa \|_{L^2} \lesssim \|\Omega\|_{L^2}. 
\end{align*}
Altogether, we obtain \eqref{eq:crucial}. The proof of the estimates for $\mathcal{R}^\alpha$ in \eqref{eq:est-rem} follows the same lines and therefore we omit them. The proof is concluded.
\end{proof}

%%%%%

\section{Remainder estimates}\label{sec:rem}

In this section, we derive suitable estimates for the remainder terms 
\begin{equation}\label{def:omega-r}
    \Omega_r:=\Omega-\Ome, \quad \eta_r:=\eta-\et, \quad \xi_r:=\xi-\csi,
\end{equation}
where 
$$(\Omega, \eta, \xi)=(\Omega (t, R, \beta),\eta (t, R, \beta),\xi (t, R, \beta))=(\omega (t, x, y), \de_x \rho (t, x, y), \de_y \rho (t, x, y))
$$ 
denotes a solution to the full system \eqref{eq:2dbouss-grad} with initial data $(\omega_0^{\alpha, \delta}, \de_x \rho_0^{\alpha, \delta}, \de_y\rho_0^{\alpha, \delta})$ as in Theorem \ref{thm:main}
and $(\Ome, \et, \csi)$ denotes the solution to \eqref{eq:leading1} with the initial data $(\Omega_{0, \mathrm{app}}^{\alpha, \delta},\eta_{0, \mathrm{app}}^{\alpha, \delta},\xi_{0, \mathrm{app}}^{\alpha, \delta})$ as in Proposition \ref{prop:expl}.
The main result of this section is the smallness of the remainders in terms of the $L^{\infty}$ norm.

\begin{Prop}\label{prop:rem}
  Let $N\geq 3$ and 
    \begin{equation*}
        F(t)=\|\Omega_{r}(t)\|_{\cH^N}+\|\eta_{r}(t)\|_{\cH^N}+\|\xi_{r}(t)\|_{\cH^N}.
    \end{equation*}
    Then 
    \begin{equation*}
        F(t)\lesssim \sqrt{\alpha}
        %C_{N+1}\sqrt{\alpha},
    \end{equation*}
    for all $0\leq t\leq T$ with $T=\frac{\alpha}{4C_{N+1}}\log|\log\alpha|$ where $C_{N+1}$ only depends on the $\cH^{N+1}$-norm of the initial data $(\Omega_0,\eta_0,\xi_0)$. In particular,
    \begin{equation}\label{eq:est-rim-Linfty}
        \|\Omega_r(t)\|_{L^{\infty}}+ \|\eta_r(t)\|_{L^{\infty}}+ \|\xi_r(t)\|_{L^{\infty}}\lesssim \sqrt{\alpha}
        %\leq C_{N+1}\sqrt{\alpha}
    \end{equation}
    for all $0\leq t\leq T$ with $T=\frac{\alpha}{4C_{N+1}}\log|\log\alpha|$.
\end{Prop}

While the system of equations satisfied by $(\Omega_r,\eta_r,\xi_r)$ is derived in Section \ref{sec:system rem}, the proof of Proposition \ref{prop:rem} is achieved by means of energy estimates in the Sobolev spaces $\cH^N$ in Section \ref{sec:est rem}.

%We adopt the notation $\Omega_r:=\Omega-\Omega_{\text{app}}$, where $\Omega$ solves the full equation and $\Omega_{\text{app}}$ is the solution to the \eqref{eq:leading1}. Similarly, we introduce $\eta_r:=\eta-\eta_{\text{app}}$ and $\xi:=\xi-\xi_{\text{app}}$. 
For the stream function $\Psi$ as in \eqref{eq:ellittica}, we set the notation
%
\begin{align}\label{def:psi-r}
\Psi_r(\Omega):&=\Psi(\Omega)-\Ps(\Ome) = \hE(\Omega)+\mathcal{R}(\Omega) + \Ps(\Omega_r)
\end{align}
where $$\mathcal{R}(\Omega)=\mathcal{R}^{\alpha}(\Omega)\sin(2\beta)$$ and $\Ps(\Omega), \mathcal{R}^\alpha(\Omega)$ are defined in \eqref{eq:psiapp}-\eqref{eq:psi2}-\eqref{def:R} respectively, while 
$$\hE(\Omega):=\Psi(\Omega)-\Ps(\Omega)-\mathcal{R} (\Omega).$$
%
% The decomposition \eqref{def:psi-r} leads to the following estimates.
% \begin{Lem}[Elliptic estimates for $\Psi_r$]\label{lem: estimates psir}
% The following estimates for $\Psi_r$ hold:
% \begin{align}\label{eq:crucial-r}
%     \alpha\|\de_{\beta \beta} \Psi_r \|_{H^k}+\alpha \|R \de_{R\beta} \Psi_r\|_{H^k} + \alpha^2 \|R^2 \de_{RR} \Psi_r\|_{H^k}  &\le \bar{C}(\|\Omega\|_{\mathcal{H}^{k}}+\|\Ome\|_{\cH^k}),\\
% {\alpha}\|\de_{\beta \beta} (R^k \de_R^k\Psi_r) \|_{L^2}+{\alpha} \|R \de_{R\beta} (R^k \de_R^k \Psi_r)\|_{L^2}+{\alpha^2}\|R^2 \de_{RR}(R^k \de_R^k \Psi_r)\|_{L^2} &\le \bar{C}( \|R^k \de_R^k \Omega\|_{L^2} +\|R^k \de_R^k \Ome\|_{L^2}),
% \end{align}
% where the constant value $\bar C >0$ is independent of $\alpha>0$. 
% \end{Lem}
% \textcolor{red}{Norma $L^2$ di $\Ome$ in (4.2)?}

% \begin{proof}
%     The proof is a direct application of Lemma \ref{lam:Rest}.
% \end{proof}


%
\subsection{System of remainders}\label{sec:system rem}

We derive the system satisfied by $(\Omega_r, \eta_r, \xi_r)$ from \eqref{eq:2dbouss-grad} and 
\eqref{eq:leading1}, namely
%
\be
\ba
\de_t \Omega_r & + (-\alpha R \de_\beta (\Psi_r+\Psi_{\text{app}})) \de_R (\Omega_r+\Omega_{\text{app}}) + (2\Psi_{\text{app}} \de_\beta \Omega_r + 2 \Psi_r \de_\beta \Omega_r + 2 \Psi_r \de_\beta \Omega_{\text{app}}) \\
& + (\alpha R(\de_R \Psi_r + \de_R \Psi_{\text{app}})) (\de_\beta \Omega_{\text{app}} + \de_\beta \Omega_r)=\eta_r, \\\\
\de_t \eta_r  & + (-\alpha R \de_\beta (\Psi_r+\Psi_{\text{app}})) \de_R (\eta_r+\eta_{\text{app}}) + (2\Psi_{\text{app}} \de_\beta \eta_r + 2 \Psi_r \de_\beta \eta_r + 2 \Psi_r \de_\beta \eta_{\text{app}}) \\
& + (\alpha R(\de_R \Psi_r + \de_R \Psi_{\text{app}})) (\de_\beta \eta_{\text{app}} + \de_\beta \eta_r)\\
&=\de_x u_2 (1-\csi-\xi_r) - \de_x u_1(\et +\eta_r) - \frac{ \cL(\Omega_{\text{app}}) }{2\alpha}\eta_{\text{app}}=: \text{(RHS)}_\eta , \\\\
\de_t \xi_r  & + (-\alpha R \de_\beta (\Psi_r+\Psi_{\text{app}})) \de_R (\csi+\xi_r) + (2\Psi_{\text{app}} \de_\beta \xi_r + 2 \Psi_r \de_\beta \xi_r + 2 \Psi_r \de_\beta\csi \\
& + (\alpha R(\de_R \Psi_r + \de_R \Psi_{\text{app}})) \de_\beta(\csi+ \xi_r)=\de_x u_1 (\csi-1+\xi_r) - (\de_y u_1)(\et+\eta_r)\\
&-\frac{ \cL(\Omega_{\text{app}}) }{2\alpha}(1-\csi)=: \text{(RHS)}_\xi.
\ea
\ee
%
From now on, the shortened notation $\Ps$ stands for $\Ps (\Ome)$, unless differently specified. 

We need to write more explicitly the right-hand side of the last two equations. We shall repeatedly use the trigonometric identities $\sin(2\beta)=2\sin(\beta)\cos(\beta)$ and $\cos(2\beta)=1-2\sin^2(\beta)$. 
To this end, notice from \eqref{eq:spatial-der} and the equations for $u_1, u_2$ below that
\be
\ba
-(\de_x u_1)\eta - \frac{ \cL(\Omega_{\text{app}}) }{2\alpha}\eta_{\text{app}} & = [(\alpha R\sin(2\beta) + \frac{\alpha^2 R}{2} \sin(2\beta))(\de_R \Psi_{\text{app}} + \de_R \Psi_r) + \overbrace{\cos(2\beta) \de_\beta \Psi_r} \\
&\quad + \alpha R \cos (2\beta) (\de_{R\beta} \Psi_{\text{app}} + \de_{R\beta} \Psi_r) + \frac{\alpha^2 R^2}{2} \sin(2\beta) (\de_{RR}\Psi_{\text{app}}+\de_{RR}\Psi_r)\\
&\quad \overbrace{-\frac{1}{2}\sin(2\beta) \de_{\beta \beta} \Psi_r} ](\eta_r+\eta_{\text{app}})\\
& \quad + \underbrace{[\cos(2\beta) \de_\beta \Psi_{\text{app}} - \frac{1}{2}\sin(2\beta) \de_{\beta \beta} \Psi_{\text{app}}](\eta_{\text{app}}}+\eta_r) \underbrace{-\frac{\cL(\Omega_{\text{app}}) }{2\alpha}\eta_{\text{app}}}.
\ea
\ee
%
Now, plugging the expression of $\Psi_{\text{app}}$ from \eqref{eq:psi-main} inside the first \emph{underbrace terms}:
\be
\ba
(\eta_r+\eta_{\text{app}}) [\cos(2\beta) \de_\beta \Psi_{\text{app}} - \frac 12 \sin(2\beta) \de_{\beta \beta} \Psi_{\text{app}}]
= (\eta_r+\eta_{\text{app}})\left[\cos^2(2\beta)\frac{\cL(\Omega_{\text{app}}) }{2\alpha}+ \sin^2(2\beta)\frac{\cL(\Omega_{\text{app}}) }{2\alpha} \right]\\
= (\eta_r+\eta_{\text{app}}) \frac{\cL(\Omega_{\text{app}}) }{2\alpha}.
\ea
\ee
%
Subtracting the two underbrace terms, we obtain
%
\be
\ba
-(\de_x u_1)\eta - \frac{ \cL(\Omega_{\text{app}}) }{2\alpha}\eta_{\text{app}} & = [(\alpha R \sin(2\beta) + \frac{\alpha^2 R}{2} \sin(2\beta))(\de_R \Psi_{\text{app}} + \de_R \Psi_r) + \cos(2\beta) \de_\beta \Psi_r \\
&\quad + \alpha R \cos (2\beta) (\de_{R\beta} \Psi_{\text{app}} + \de_{R\beta} \Psi_r) + \frac{\alpha^2 R^2}{2}\sin(2\beta)(\de_{RR}\Psi_{\text{app}}+\de_{RR}\Psi_r)\\
&\quad -\frac{1}{2}\sin(2\beta) \de_{\beta \beta} \Psi_r ](\eta_r+\eta_{\text{app}})+  \frac{\cL(\Omega_{\text{app}}) }{2\alpha} \eta_r.
\ea
\ee
%
Next, we write
\be\label{eq:dexu2}
\ba
\de_x u_2&= \overbrace{2 \Psi} + \alpha (2+\alpha) R (\cos^2\beta) \de_R \Psi + \alpha R \de_R \Psi  \overbrace{-2 \sin \beta \cos \beta \de_\beta \Psi} - 2\alpha R \sin \beta \cos \beta \de_{R\beta}\Psi \\
&\quad + (\alpha R)^2 \cos^2\beta \de_{RR}\Psi \overbrace{+ \sin^2\beta \de_{\beta \beta}\Psi}. 
\ea
\ee
%
Let us look at the above \emph{overbrace terms}. Using the decomposition $\Psi=\Psi_2+\hE(\Omega)$ and noting that $2\Psi_2-\sin(2\beta)\de_\beta\Psi_2+\sin^2(\beta) \de_{\beta \beta}\Psi_2=0$
% Using
% the decomposition $\Psi=\Psa+\Psi_r$ with $\Psi_r$ defined in \eqref{def:psi-r}
%of Theorem \ref{thm:tarek}, which gives $\psi=\psi_{\text{app}}+\psi_{\text{err}}$
% \be\label{eq:psipsi}
% \Psi=\Psi (R, \beta)= - \frac{\cL(\Omega)}{4\alpha} \sin (2\beta) + \hE(\Omega)(R, \beta),
% \ee
and plugging it inside the above overbrace terms, we have 
%
\be\ba
2 \Psi -\sin(2\beta) \beta \de_\beta \Psi+ \sin^2\beta \de_{\beta \beta}\Psi&= \frac{\cL(\Omega)}{2\alpha}\sin(2\beta)[1-\cos(2\beta)-2\sin^2(\beta)] \\
&\quad + (2 \hE(\Omega) - \sin(2\beta) \de_\beta \hE(\Omega) + (\sin^2\beta) \de_{\beta \beta} \hE(\Omega)) \\
&= 2 \hE(\Omega) - \sin(2\beta) \de_\beta \hE(\Omega) + (\sin^2\beta) \de_{\beta \beta} \hE(\Omega). 
\ea\ee
%
Altogether, we have
\be\ba
\text{(RHS)}_\eta&=\frac{\cL(\Omega_{\text{app}}) }{2\alpha} \eta_r + [\cos(2\beta) \de_\beta \Psi_r - \frac 1 2 \sin(2 \beta) \de_{\beta \beta} \Psi_r](\eta_r+\eta_{\text{app}}) + O(\alpha) + F(\Psi_r),
\ea\ee
where $F(\Psi_r)$ is a function of the remainder $\Psi_r$ 
defined in \eqref{def:psi-r}.
%of the elliptic equation in Theorem \ref{thm:tarek}.  \textcolor{red}{to change.}
%
More explicitly,
\be
\ba
\text{(RHS)}_\eta&=\frac{\cL(\Omega_{\text{app}}) }{\alpha} \eta_r + \left[\cos(2\beta) \de_\beta \Psi_r - \frac 1 2 \sin(2 \beta) \de_{\beta \beta} \Psi_r\right](\eta_r+\eta_{\text{app}})\\
& \quad +  [(\alpha R \sin(2\beta)+ \frac{\alpha^2 R}{2} \sin(2\beta))(\de_R \Psi_{\text{app}} + \de_R \Psi_r) \\
&\quad + \alpha R \cos (2\beta) (\de_{R\beta} \Psi_{\text{app}} + \de_{R\beta} \Psi_r) + \frac{\alpha^2 R^2}{2} \sin(2\beta)(\de_{RR}\Psi_{\text{app}}+{\de_{RR}\Psi_r})](\eta_r+\eta_{\text{app}})\\
&\quad + [2 \hE(\Omega) - \sin(2\beta) \de_\beta \hE(\Omega) + (\sin^2\beta) \de_{\beta \beta} \hE(\Omega)](1-\xi). 
\ea
\ee
%%%%%
It remains to rewrite $\text{(RHS)}_\xi$. First,
\be
\ba
\de_x u_1 (\xi-1)&= -  [(\alpha R \sin(2\beta)+ \frac{\alpha^2 R}{2} \sin(2\beta))(\de_R \Psi_{\text{app}} + \de_R \Psi_r) + \overbrace{\cos(2\beta) \de_\beta \Psi_r} \\
&\quad + \alpha R \cos (2\beta) (\de_{R\beta} \Psi_{\text{app}} + \de_{R\beta} \Psi_r) + \frac{\alpha^2 R^2}{2} \sin (2\beta)(\de_{RR}\Psi_{\text{app}}+\de_{RR}\Psi_r)\\
&\quad \overbrace{-\frac{1}{2}\sin(2\beta)\de_{\beta \beta} \Psi_r} ](\xi-1)\\
& \quad - \underbrace{[\cos(2\beta) \de_\beta \Psi_{\text{app}} - \frac{1}{2}\sin(2\beta)\de_{\beta \beta} \Psi_{\text{app}}](\xi-1}).
\ea 
\ee
Similarly to the previous term, the under-brace terms read
\be
\ba
-(\xi-1) [\cos(2\beta) \de_\beta \Psi_{\text{app}} - \frac{1}{2}\sin(2\beta)\de_{\beta \beta} \Psi_{\text{app}}]  \\
\quad = -(\xi-1) \left[\cos^2(2\beta)\frac{\cL(\Omega_{\text{app}}) }{2\alpha}+ \sin^2(2\beta)\frac{\cL(\Ome)) }{\alpha} \right]
=-(\xi-1) \frac{\cL(\Omega_{\text{app}}) }{2\alpha}.
\ea
\ee
%
Thus, we see that
%
\be
\ba
\de_x u_1 (\xi-1)&=  -\frac{\cL(\Omega_{\text{app}}) }{2\alpha}(\xi-1)\\
&\quad  -  [(\alpha R \sin(2\beta)+ \frac{\alpha^2 R}{2} \sin(2\beta))(\de_R \Psi_{\text{app}} + \de_R \Psi_r) + {\cos(2\beta) \de_\beta \Psi_r} \\
&\quad + \alpha R \cos (2\beta) (\de_{R\beta} \Psi_{\text{app}} + \de_{R\beta} \Psi_r) + \frac{\alpha^2 R^2}{2} \sin(2\beta)(\de_{RR}\Psi_{\text{app}}+\de_{RR}\Psi_r)\\
&\quad {-\frac{1}{2}\sin(2\beta) \de_{\beta \beta} \Psi_r} ](\xi-1).
\ea 
\ee
%
Finally, we need the expression of $\de_y u_1$. 
\begin{comment}
\textcolor{blue}{In blue come alternativa per l'equazione $\de_yu_1=\de_xu_2\pm \Delta \psi$:
 A straightforward computations yields
 \begin{equation*}
     \de_y u_1=-(2\Psi+\sin(2\beta)\de_{\beta}\Psi+\cos^2(\beta)\de_{\beta\beta}\Psi)-\alpha R(1+2\sin^2(\beta))\de_R\Psi-\alpha^2\sin^2(\beta)R\de_R\Psi-\alpha R\sin(2\beta)\de_{R\beta}\Psi-(\alpha R)^2\de_{RR}\Psi.
 \end{equation*}
 }
 \end{comment}
 %
Recalling that $\omega=\de_y u_1-\de_x u_2$, we have that
$$\de_y u_1= \de_x u_2 - \omega = \de_x u_2 - \Delta \psi.$$
%
In polar coordinates $(R, \beta)$, from the elliptic equation for $\Psi$ in \eqref{eq:psi-main} and from  \eqref{eq:dexu2}, we have
%
\be
\ba
\de_y u_1 & = \underbrace{-4 \Psi - (\alpha R)^2 \de_{RR}\Psi - \de_{\beta \beta} \Psi - \alpha (\alpha + 4) R \de_R \Psi}_{=-\Omega} \\
&\quad +{2 \Psi} + \alpha (2+\alpha) R (\cos^2\beta) \de_R \Psi + \alpha R \de_R \Psi  {-2 \sin \beta \cos \beta \de_\beta \Psi} - 2\alpha R \sin \beta \cos \beta \de_{R\beta}\Psi \\
&\quad + (\alpha R)^2 \cos^2\beta \de_{RR}\Psi {+ \sin^2\beta \de_{\beta \beta}\Psi}\\
&= \overbrace{-2 \Psi} + \alpha (2+\alpha) R (\cos^2\beta) \de_R \Psi - \alpha (\alpha + 3) R \de_R \Psi \overbrace{- \sin(2\beta) \de_\beta \Psi} - \alpha R  \sin(2\beta)  \de_{R \beta} \Psi \\
&\quad - (\alpha R)^2 (\sin^2\beta) \de_{RR} \Psi \overbrace{- (\cos^2\beta) \de_{\beta \beta} \Psi}.
\ea
\ee
%
Using the decomposition $\Psi=\Psi_2+\Psi_r$ and that $2 \Psi_2+\sin(2\beta) \de_\beta \Psi_2+ (\cos^2\beta) \de_{\beta \beta} \Psi_2=0$ the above overbrace terms yield
\be 
\ba
-2 \Psi- \sin(2\beta) \de_\beta \Psi- (\cos^2\beta) \de_{\beta \beta} \Psi&= - 2 \hE(\Omega) - \sin(2\beta) \de_\beta \hE(\Omega) - (\cos^2\beta) \de_{\beta \beta} \hE(\Omega).
\ea
\ee
%
Therefore, we have
\be
\ba
\text{(RHS)}_\xi&=  -\frac{\cL(\Omega_{\text{app}}) }{2\alpha}\xi_r - \left[(1-2\sin^2\beta) \de_\beta \Psi_r-\frac 12 \sin (2\beta) \de_{\beta \beta} \Psi_r\right] (\xi_r+\csi-1)\\
&\quad -  [(\alpha R \sin(2\beta)+ \frac{\alpha^2 R}{2} \sin(2\beta) )(\de_R \Psi_{\text{app}} + \de_R \Psi_r)  \\
&\quad + \alpha R \cos (2\beta) (\de_{R\beta} \Psi_{\text{app}} + \de_{R\beta} \Psi_r) + \frac{\alpha^2 R^2}{2} \sin(2\beta) (\de_{RR}\Psi_{\text{app}}+\de_{RR}\Psi_r) ](\xi_r+\csi-1)\\
&\quad + [2 \hE(\Omega) + \sin(2\beta) \de_\beta \hE(\Omega) + (\cos^2\beta) \de_{\beta \beta} \hE(\Omega)](\eta_{\text{app}}+\eta_r).
\ea
\ee
%
We can finally write down the system for the remainder terms as follows:
\be
\ba
\de_t \Omega_r & + (-\alpha R \de_\beta (\Psi_r+\Psi_{\text{app}})) \de_R (\Omega_r+\Omega_{\text{app}}) + (2\Psi_{\text{app}} \de_\beta \Omega_r + 2 \Psi_r \de_\beta \Omega_r + 2 \Psi_r \de_\beta \Omega_{\text{app}}) \\
& + (\alpha R(\de_R \Psi_r + \de_R \Psi_{\text{app}})) (\de_\beta \Omega_{\text{app}} + \de_\beta \Omega_r)=\eta_r; \\\\
\de_t \eta_r  & + (-\alpha R \de_\beta (\Psi_r+\Psi_{\text{app}})) \de_R (\eta_r+\eta_{\text{app}}) + (2\Psi_{\text{app}} \de_\beta \eta_r + 2 \Psi_r \de_\beta \eta_r + 2 \Psi_r \de_\beta \eta_{\text{app}}) \\
& + (\alpha R(\de_R \Psi_r + \de_R \Psi_{\text{app}})) (\de_\beta \eta_{\text{app}} + \de_\beta \eta_r)\\&= \text{(RHS)}_\eta =\frac{\cL(\Omega_{\text{app}}) }{2\alpha} \eta_r + \left[\cos(2\beta) \de_\beta \Psi_r - \frac 1 2 \sin(2 \beta) \de_{\beta \beta} \Psi_r\right](\eta_r+\eta_{\text{app}})\\
& \quad +  [(\alpha R \sin(2\beta)+ \frac{\alpha^2 R}{2} \sin(2\beta) )(\de_R \Psi_{\text{app}} + \de_R \Psi_r) \\
&\quad + \alpha R \cos (2\beta) (\de_{R\beta} \Psi_{\text{app}} + \de_{R\beta} \Psi_r) + \frac{\alpha^2 R^2}{2} \sin(2\beta) (\de_{RR}\Psi_{\text{app}}+\de_{RR}\Psi_r)](\eta_r+\eta_{\text{app}})\\
&\quad + [2 \hE(\Omega) - \sin(2\beta) \de_\beta \hE(\Omega) + (\sin^2\beta) \de_{\beta \beta} \hE(\Omega)](1-\xi_r-\csi); \\\\
\de_t \xi_r  & + (-\alpha R \de_\beta (\Psi_r+\Psi_{\text{app}})) \de_R (\xi_r+\csi)  + (2\Psi_{\text{app}} \de_\beta \xi_r + 2 \Psi_r \de_\beta \xi_r+2 \Psi_r \de_\beta \csi) \\
& + (\alpha R(\de_R \Psi_r + \de_R \Psi_{\text{app}})) \de_\beta (\xi_r+\csi)\\&=\text{(RHS)}_\xi=-\frac{\cL(\Omega_{\text{app}}) }{2\alpha}\xi_r- \left[\cos (2\beta) \de_\beta \Psi_r-\frac 12 \sin (2\beta) \de_{\beta \beta} \Psi_r \right](\xi_r+\csi-1)\\
&\quad -  [(\alpha R \sin(2\beta)+ \frac{\alpha^2 R}{2} \sin(2\beta) )(\de_R \Psi_{\text{app}} + \de_R \Psi_r)  \\
&\quad + \alpha R \cos (2\beta) (\de_{R\beta} \Psi_{\text{app}} + \de_{R\beta} \Psi_r) + \frac{\alpha^2 R^2}{2} \sin(2\beta) (\de_{RR}\Psi_{\text{app}}+\de_{RR}\Psi_r) ](\xi_r+\csi-1)\\
&\quad + [2 \hE(\Omega) + \sin(2\beta) \de_\beta \hE(\Omega) + (\cos^2\beta) \de_{\beta \beta} \hE(\Omega)](\eta_{\text{app}}+\eta_r).
\ea
\ee
%
In order to lighten the notation, we introduce the transport operator $\cT$ acting on a function $f(R, \beta)=f=f_r+\fe$ as follows 
%
\be\label{eq:transport terms}
\ba
\cT f(R, \beta) := (-\alpha R \de_\beta (\Psi_r+\Psi_{\text{app}})) \de_R (f_r+\fe) &+ (2\Psi_{\text{app}} \de_\beta f_r + 2 \Psi_r \de_\beta f_r + 2 \Psi_r \de_\beta \fe) \\
& + (\alpha R(\de_R \Psi_r + \de_R \Psi_{\text{app}})) (\de_\beta \fe + \de_\beta f_r),
\ea
\ee
and
\be
\cT_\xi f(R, \beta) := (-\alpha R \de_\beta (\Psi_r+\Psi_{\text{app}})) \de_R f) + (2\Psi_{\text{app}} \de_\beta f + 2 \Psi_r \de_\beta f)  + (\alpha R(\de_R \Psi_r + \de_R \Psi_{\text{app}}))\de_\beta f.
\ee
%
Relying on the notation (in Section 6) of \cite{tarek2}, the system for the remainder terms can be written in the compact form below
\begin{align}
\de_t \Omega_r + \cT \Omega &=\eta_r; \label{eq:omegar} \tag{$\Omega_r$} \\\notag\\
\de_t \eta_r   + \cT \eta&=\frac{\cL(\Omega_{\text{app}}) }{2\alpha} \eta_r +\left[ \cos(2\beta) \de_\beta \Psi_r - \frac 1 2 \sin(2 \beta) \de_{\beta \beta} \Psi_r\right](\eta_r+\eta_{\text{app}}) \qquad [=:\text{(RHS)}_\eta^1]\label{eq:rhseta1}\\
& \quad + [\mathrm{I}_4 + \mathrm{I}_6 + \mathrm{I}_7] (\eta_r+\eta_{\text{app}}) \qquad [=:\text{(RHS)}_\eta^2]\label{eq:rhseta2} \\
&\quad + [2 \hE(\Omega) - \sin(2\beta) \de_\beta \hE(\Omega) + (\sin^2\beta) \de_{\beta \beta} \hE(\Omega)](1-\xi);  \qquad [=:\text{(RHS)}_\eta^3]  \label{eq:rhs-etar1}\\\notag\\
\de_t \xi_r   + \cT_\xi \xi &=-\frac{\cL(\Omega_{\text{app}}) }{2\alpha}\xi_r-  \left[\cos (2\beta) \de_\beta \Psi_r-\frac 12 \sin (2\beta) \de_{\beta \beta} \Psi_r \right] (\xi_r+\xi_\mathrm{app}-1)\qquad [=:\text{(RHS)}_\xi^1]\label{eq:rhsxi1}\\
&\quad -  [\mathrm{I}_4 + \mathrm{I}_6 + \mathrm{I}_7](\xi_r+\xi_\mathrm{app}-1)\qquad [=:\text{(RHS)}_\xi^2]\label{eq:rhsxi2}\\
&\quad + [2 \hE(\Omega) + \sin(2\beta) \de_\beta \hE(\Omega) + (\cos^2\beta) \de_{\beta \beta} \hE(\Omega)](\eta_{\text{app}}+\eta_r), \qquad [=:\text{(RHS)}_\xi^3]\label{eq:rhs-xir1},
\end{align}
%%%%%
where 
\be
\ba
\mathrm{I}_4&:= (\alpha R \sin(2\beta)+\frac{\alpha^2 R}{2} \sin(2\beta))(\de_R \Psi_{\text{app}} + \de_R \Psi_r)=:\mathrm{I}_{4,1}+\mathrm{I}_{4,2}, \\
\mathrm{I}_6&:=\alpha R \cos (2\beta) (\de_{R\beta} \Psi_{\text{app}} + \de_{R\beta} \Psi_r)=: \mathrm{I}_{6,1}+\mathrm{I}_{6,2}, \\
 \mathrm{I}_7&:=\frac{\alpha^2 R^2}{2}\sin(2\beta) (\de_{RR}\Psi_{\text{app}}+\de_{RR}\Psi_r)=: \mathrm{I}_{7,1}+\mathrm{I}_{7,2}.
\ea
\ee

\subsection{Estimates for remainder terms}\label{sec:est rem}
% \textcolor{blue}{Comment: There are two main kinds of remainder terms, namely \textbf{transport terms} $\cT$ and \textbf{source terms} $\cS$. The transport terms are the following:
% \be 
% \cT= \mathrm{I}_1+\mathrm{I}_2+\mathrm{I}_3,
% \ee
% where $\mathrm{I}_1, \mathrm{I}_2, \mathrm{I}_3$ are exactly the ones in [Section 6, \cite{tarek2}].
% }
% %

In this section, it is shown that the remainders $(\Omega_r,\eta_r,\xi_r)$ are sufficiently small in $\cH^N$ on suitable time scales. To that end, we adapt and extend the strategy in \cite[Section 6]{tarek2} to the present setting. Note that here we are concerned with a system while a scalar equation is considered in \cite{tarek2}. Furthermore, additional estimates for $\Psi_r$ are required in view of the decomposition \eqref{def:psi-r} differing from the one available in \cite{tarek2}.

The remainder estimates are proven in the Sobolev spaces $\cH^N$ and $\cW^N$ with weights tailored for that purpose as defined in \eqref{def:Hk} and \eqref{def:Wk}.
We recall the following embedding property for $\cH^N$ proven in \cite[Lemma 5.1]{tarek2}.
\begin{lem}[\cite{tarek2}]\label{lem:embedding}
    Let $N\in \N$, then there exists $C_N>0$ such that
    \begin{equation*}
        \|R^k\de_R^k\de_\beta^mf\|_{L^{\infty}}\leq C_N\|f\|_{\cH^{k+m+2}}
    \end{equation*}
    for all $k+m+2\leq N$ and $f\in \cH^N$.
\end{lem}
Exploiting the elliptic estimates of Lemma \ref{lam:Rest}, we provide estimates on the stream-function $\Psi_r$.
\begin{lem}\label{lem:Psir}
Let $k,m, N\in \N$. With the definition of $\Psi_r$ in \eqref{def:psi-r}, it follows that
\begin{align}\label{eq:est-Psir-N}
    \|\Psi_r\|_{\cH^{k}}&\lesssim \left(C_k+\alpha e^{t\frac{C_k}{\alpha}} + \frac{\|\Omega_r\|_{\cH^k}}{\alpha}\right). %\label{eq:psi-r-quad}.
\end{align}
Moreover, 
\begin{equation}\label{eq:est-Psir-beta}
   \|\de_{\beta\beta}\Psi_r\|_{\cH^k}+\|\de_\beta\Psi_r\|_{\cH^k}\lesssim \left(C_k+\alpha\eul^{\frac{C_k}{\alpha}t}+\frac{\|\Omega_r\|_{\cH^k}}{\alpha}\right),
\end{equation}
and 
\begin{equation}\label{eq:est-Psir-R}
    \alpha\|R\de_R\Psi_r\|_{\cH^k}\lesssim \left(C_k+\alpha\eul^{\frac{C_k}{\alpha}t}+\|\Omega_r\|_{\cH^k}\right).
\end{equation}
The following $L^{\infty}$-bounds hold 
\begin{align}\label{eq:est-Psir-reg-infty}
\left\|\de_R^k\de_\beta^{m}\hE(\Omega)\right\|_{L^{\infty}}+\left\|R^k\de_R^k\de_\beta^{m}\hE(\Omega)\right\|_{L^{\infty}}+\left\|\de_R^k\de_\beta^{m}\mathcal{R}^{\alpha}(\Omega)\right\|_{L^{\infty}}+\left\|R^k\de_R^k\de_\beta^{m}\mathcal{R}^{\alpha}(\Omega)\right\|_{L^{\infty}}\\\lesssim \left(C_N+\alpha\eul^{\frac{C_{N}}{\alpha}t}+\|\Omega_r\|_{\cH^{k+m+1}}\right),
\end{align}
as well as 
\begin{equation}\label{eq:est-Psir-sing-infty}
\left\|\de_R^k\de_\beta^{m}\Ps(\Omega_r)\right\|_{L^{\infty}}+\left\|R^k\de_R^k\de_\beta^{m}\Ps(\Omega_r)\right\|_{L^{\infty}}\leq \frac{C_{k,m}}{\alpha}\|\Omega_r\|_{\cH^{k+m+1}}
\end{equation}
for all $k,m\in \N$ with $k+m+1\leq N$.
\end{lem}
Note that the elliptic estimates of Lemma \ref{lam:Rest} do not allow for an estimate of the type $\|\Psi(\Omega)\|_{\cH^{k}}\leq C_{k}\|\Omega\|_{\cH^{k-1}}$ as the estimates involving $R\de_R$ are not uniform in $\alpha$, even for the remainder terms $\mathcal{R}^{\alpha}(\Omega), \hE(\Omega)$. 
The case $m=0$ is not included in the above $L^{\infty}$-estimate, such an estimate is not required for our analysis.  

\begin{proof}
 Recall from \eqref{def:psi-r} that $\Psi_r=(\hE+\mathcal{R}^\alpha)(\Omega)+\Ps (\Omega_r)$. We shall repeatedly use that $\hE(\Omega)$ admits a series expansion of the type
 \begin{equation*}
     \hE(\Omega)=\sum_{n\geq 3}\hE^n(R)\sin(n\beta)
 \end{equation*}
 and in particular is orthogonal to $\sin(n\beta)$ in $L_{\beta}^2$ for $n=0,1,2$, see the proof of Lemma \ref{lam:Rest} for details. Hence, one has
 \begin{equation*}
     \|\hE(\Omega)\|_{L^2}\leq \frac{1}{3}\|\de_\beta\hE(\Omega)\|_{L^2}\leq \frac{1}{9}\|\de_{\beta\beta}\hE(\Omega)\|_{L^2}.
 \end{equation*}
Similarly, we observe that $\Ps, \mathcal{R}^{\alpha}$ involve $L_{\beta}^2$-projections onto $\sin(2\beta)$. Therefore, the same estimates hold for $\Ps, \mathcal{R}^{\alpha}$ up to changing the constants. 
First, we prove \eqref{eq:est-Psir-N}. It follows from \eqref{eq:hardy} and \eqref{eq:elliptic1-rem}-\eqref{eq:est-rem} as well as \eqref{eq:est-LOM} that
 \begin{equation*}
     \|\hE(\Omega)\|_{\cH^k}+\|\mathcal{R}^{\alpha}(\Omega)\|_{\cH^k}\lesssim \|\Omega\|_{\cH^k}%\lesssim  \left(C_k+\alpha\eul^{\frac{C_k}{\alpha}t}+\|\Omega_r\|_{\cH^k}\right).
 \end{equation*}
Further, upon observing that $\Ps$ is orthogonal to $\sin(n\beta)$ in $L^2$ for all $n\neq 2$ and \eqref{eq:crucial} exploiting, we infer
\begin{equation*}
    \|\Ps(\Omega_r)\|_{\cH^k}\leq C \|\de_{\beta\beta}\Ps(\Omega_r)\|_{\cH^k}\leq C \|\de_{\beta\beta}\Psa(\Omega_r)\|_{\cH^k}+\|\de_{\beta\beta}\mathcal{R}^{\alpha}(\Omega_r)\|_{\cH^k}\leq \frac{C}{\alpha}\|\Omega_r\|_{\cH^k}.
\end{equation*}
Note that in contrast to \eqref{eq:elliptic1-rem}, \eqref{eq:elliptic2-rem} the inequality \eqref{eq:crucial} does not provide a bound on $\de_{\beta\beta}\Ps$ which is uniform in $\alpha$. Hence, the loss in $\alpha$.
Next, we prove the $L^2$-based estimates \eqref{eq:est-Psir-beta}. Exploiting \eqref{eq:elliptic1-rem} and the orthogonality of $\hE(\Omega)$ with respect to $\sin(n\beta)$ with $n=0,1,2$ in $L^2$ we obtain that 
\begin{align*}
\left\|\de_R^k\de_\beta^{m}\hE(\Omega)\right\|_{L^{2}}+\left\|R^k\de_R^k\de_\beta^{m}\hE(\Omega)\right\|_{L^{2}}
&\leq C \left\|\de_R^k\de_\beta^{m-2}(\Omega-\Omega_2)\right\|_{L^{2}}+\left\|R^k\de_R^k\de_\beta^{m-2}(\Omega-\Omega_2)\right\|_{L^{2}}\\
&\leq C\|\Omega-\Omega_2\|_{\cH^{k+m-2}}
\end{align*}
for $k,m\in \N$ and $m\geq 2$. For $m=1$, we recover
\begin{equation*}
\left\|\de_R^k\de_\beta\hE(\Omega)\right\|_{L^{2}}+\left\|R^k\de_R^k\de_\beta\hE(\Omega)\right\|_{L^{2}}
\leq C \left\|\de_R^k(\Omega-\Omega_2)\right\|_{L^{2}}+\left\|R^k\de_R^k(\Omega-\Omega_2)\right\|_{L^{2}}
\leq C_{k,m}\|\Omega-\Omega_2\|_{\cH^{k}}
\end{equation*}

Similarly, it follows from \eqref{eq:elliptic2-rem} that
\begin{equation*}
\left\|R^k\de_R^k\de_\beta^{m}\hE(\Omega)\right\|_{L^{2}}
\leq  \frac{C_{k,m}}{\alpha} \left\|R^{k-1}\de_R^{k-1}\de_\beta^{m}(\Omega-\Omega_2)\right\|_{L^{2}}
\leq \frac{C_{k,m}}{\alpha}\|\Omega-\Omega_2\|_{\cH^{k+m-1}}
\end{equation*}
for all $k\in \N$ and $m\in \N_0$ with $k+m-1\leq N$. The same estimates hold true for $\mathcal{R}^{\alpha}$ upon using \eqref{eq:est-rem} and the $L_\beta^2$-orthogonality to $\sin(n\beta)$ for all $n\neq 2$. Further, the same argument and \eqref{eq:crucial} yield
\begin{equation*}
    \left\|\de_R^k\de_\beta^m\Ps(\Omega_r)\right\|_{L^2}+ \left\|R^k\de_R^k\de_\beta^m\Ps(\Omega_r)\right\|_{L^2}\leq \frac{C_{k,m}}{\alpha}\|\Omega_r\|_{\cH^{k+m-1}}
\end{equation*}
for all $k\in \N$, $m\in \N_0$ and $m+k-1\leq N$.

Following the proof of \cite[Lemma 5.2, 5.3]{tarek2} and exploiting that $\hE(\Omega)$ is orthogonal to $\sin(n\beta)$ for $n=0,1,2$ we infer from \eqref{eq:elliptic1-rem} and Sobolev embedding that
\begin{equation*}
\left\|\de_R^k\de_\beta^{m}\hE(\Omega)\right\|_{L^{\infty}}+\left\|R^k\de_R^k\de_\beta^{m}\hE(\Omega)\right\|_{L^{\infty}}\leq C_{k,m}\|\Omega-\Omega_2\|_{\cH^{k+m+1}}
\end{equation*}
for all $k,m\in \N$ with $k+m+1\leq N$. Note that we crucially exploit that the estimates involving $\de_{\beta\beta}$ in \eqref{eq:elliptic1-rem} are uniform in $\alpha$. Similarly, one infers from \eqref{eq:est-rem} that 
\begin{equation*}
\left\|\de_R^k\de_\beta^{m}\mathcal{R}^{\alpha}(\Omega)\right\|_{L^{\infty}}+\left\|R^k\de_R^k\de_\beta^{m}\mathcal{R}^{\alpha}(\Omega)\right\|_{L^{\infty}}\leq C_{k,m}\|\Omega_2\|_{\cH^{k+m+1}}
\end{equation*}
for all $k,m\in \N$ with $k+m+1\leq N$,
where we used that $\mathcal{R}^\alpha(\Omega)$ is orthogonal to $\sin(n\beta)$ for any $n\neq 2$. Finally, we provide a bound for $\Ps(\Omega_r)=\Psa(\Omega_r)-\mathcal{R}^{\alpha}(\Omega_r)$. Since the estimate \eqref{eq:crucial} is not uniform in $\alpha$, the same argument as for the previous terms leads to
\begin{equation*}
\left\|\de_R^k\de_\beta^{m}\Ps(\Omega_r)\right\|_{L^{\infty}}+\left\|R^k\de_R^k\de_\beta^{m}\Ps(\Omega_r)\right\|_{L^{\infty}}\leq \frac{C_{k,m}}{\alpha}\|\Omega_r\|_{\cH^{k+m+1}}
\end{equation*}
all $k,m\in \N$ with $k+m+1\leq N$. The final estimates follow from \eqref{eq:est-LOM}, namely that
\begin{equation*}
    \|\Omega\|_{\cH^k}\lesssim  \left(C_k+\alpha\eul^{\frac{C_k}{\alpha}t}+\|\Omega_r\|_{\cH^k}\right).
\end{equation*}
\end{proof}

In addition, we repeatedly rely on the bounds for solutions to the leading order model.
More precisely, \eqref{eq:est-LOM} and \eqref{eq:est-psi2} respectively yield for $f=\et, \csi$ that 
\begin{equation}\label{eq:est-fapp}
     \|\Ome\|_{\cH^k}\leq C_k+ \alpha\eul^{\frac{C_k}{\alpha}t}, \quad \|\fe\|_{\cH^k}\leq C_k \eul^{\frac{C_k}{\alpha}t}, \qquad \|\Ps (\Ome)\|_{\cW^{k,\infty}} \le \frac{C_{k}}{\alpha}, \qquad \|\Ps (\Ome)\|_{\cH^{k}} \le \frac{C_{k}}{\alpha},
\end{equation}
for $k\in \N$ where $C_k=C_k(\|\overline{\Omega}_0\|_{H^k}, \|\overline{\eta}_0\|_{H^k}, \|\overline{\xi}_0\|_{H^k})>0$ are only dependent on the initial data and independent of $\alpha$. We further observe that in view of the decomposition $\Ps(\Ome)=\Psa(\Ome)+\mathcal{R}(\Ome)$ and the elliptic estimates of Proposition \ref{lam:Rest}  one has 
\begin{equation}\label{est:Psiapp ell}
     \|\alpha R\de_R\de_\beta\Ps(\Ome)\|_{\cH^N}+\|\alpha R\de_R\Ps(\Ome)\|_{\cH^N}\lesssim \|\Ome\|_{\cH^k}\leq C_{N}+\alpha\eul^{\frac{C_N}{\alpha}t}.
\end{equation}


Next, we provide energy estimates for $(\Omega_r,\eta_r,\xi_r)$. 
To that end, we consider the various contributions separately and start by deriving the estimates for the transport terms $\mathcal{T}$ defined in \eqref{eq:transport terms}. 
\begin{lem}\label{lem:remT}
Let $N\geq3$. Then, for $f=\Omega,\eta,\xi$ with the usual decomposition $f=\fe+f_r$ it holds
\begin{align*}
     \left|\left\langle \mathcal{T}(f), f_r \right\rangle_{\cH^N}\right|
     &\leq C_{N+1}\eul^{\frac{C_{N+1}}{\alpha}t}\left(C_{N+1}+\alpha\eul^{\frac{C_{N}}{\alpha}t}+\frac{\|\Omega_r\|_{\cH^N}}{\alpha}\right)\|f_r\|_{\cH^N}\\
     &\quad +\left(\frac{C_{N+1}}{\alpha}+\alpha\eul^{\frac{C_{N}}{\alpha}t}+\frac{\|\Omega_r\|_{\cH^N}}{\alpha}\right)\|f_r\|_{\cH^N}^2.
\end{align*}
% \begin{align*}
%    \left|\left\langle \mathcal{T}(f), f_r \right\rangle_{\cH^N}\right|
%     & \leq C_{N+1}\eul^{\frac{C_{N+1}}{\alpha}t}\left(\eul^{\frac{C_{N+1}}{\alpha}t}+\frac{\|\Omega_r\|_{\cH^N}}{\alpha}\right)\|f_r\|_{\cH^N}+C_{N+1}\left(\frac{1}{\alpha}\eul^{\frac{C_{N+1}}{\alpha}t}+\frac{\|\Omega_r\|_{\cH^N}}{\alpha}\right)\|f_r\|_{\cH^N}^2, \\
%      \left|\left\langle \mathcal{T}_\xi(\xi), \xi \right\rangle_{\cH^N}\right|
%      & \leq C_{N+1}\eul^{\frac{C_{N+1}}{\alpha}t}\left(\eul^{\frac{C_{N+1}}{\alpha}t}+\frac{\|\Omega_r\|_{\cH^N}}{\alpha}\right)\|\xi\|_{\cH^N}+C_{N+1}\left(\frac{1}{\alpha}\eul^{\frac{C_{N+1}}{\alpha}t}+\frac{\|\Omega_r\|_{\cH^N}}{\alpha}\right)\|\xi\|_{\cH^N}^2.
% \end{align*}
\end{lem}
We emphasize that the term $\frac{C_{N+1}}{\alpha}\eul^{\frac{C_{N+1}}{\alpha}t}\|\Omega_r\|_{\cH^N}\|f_r\|_{\cH^N}$ leads to the largest contribution in the final estimate of the remainder terms \eqref{eq:boot} and determines the choice of the time scale in Proposition \ref{prop:rem}. This contribution stems from $2\Psi_r\de_\beta \fe$ in the transport terms \eqref{eq:transport terms}.

\begin{proof}
Recalling \eqref{eq:transport terms}, we denote 
\begin{equation*}
\ba
\cT f(R, \beta) &:= (-\alpha R \de_\beta (\Psi_r+\Ps)) \de_R (f_r+\fe)+ (2\Ps \de_\beta f_r + 2 \Psi_r \de_\beta f_r + 2 \Psi_r \de_\beta \fe) \\
&\quad + (\alpha R(\de_R \Psi_r + \de_R \Ps)) (\de_\beta \fe + \de_\beta f_r)\\&=:I_1+I_2+I_3.
\ea
\end{equation*}

In order to derive $\cH^N$-bounds, we first derive $H^N$-bounds and then consider the contribution of the weighted norms. 
We further split $I_1$ as 
\begin{equation*}
    I_1=-\alpha \de_\beta \Ps R\de_R \fe-\alpha \de_\beta \Ps R\de_R f_{r}-\alpha \de_\beta \Psi_r R\de_R \fe-\alpha \de_\beta \Psi_r R\de_R f_{r}:=I_{1,1}+I_{1,2}+I_{1,3}+I_{1,4}
\end{equation*}
and treat these terms separately. For the first contribution, one has 
\begin{equation*}
     \left|-\alpha \left\langle \de_\beta\Psi_{\text{app}}R\de_R \fe, f_r\right\rangle_{H^N}\right|
\leq C\alpha\|\de_\beta\Ps\|_{W^{N,\infty}}\|R\de_R\fe\|_{\cH^{N}}\|f_r\|_{\cH^{N}}\leq C_{N+1}^2\eul^{\frac{C_{N+1}}{\alpha}t}\|f_r\|_{\cH^N},
\end{equation*}
where we used \eqref{eq:est-fapp} to bound $\fe$ and $\Ps$. In the following, we only consider the contribution to the $H^N$ scalar-product with derivatives of order $N$ as the bounds for terms with lower order derivatives follow along the same lines. For $I_{1,2}$, we have    
\begin{equation*}
    \left\langle -\alpha\de^N\left( R\de_\beta \Ps\de_R f_{r}\right), \de^N f_r\right\rangle_{L^2}=\left\langle -\alpha \sum_{i=0}^N\de^i\de_\beta \Ps \de^{N-i}(R\de_R f_{r}), \de^N f_r\right\rangle_{L^2}.
\end{equation*}
If $i=0$, an integration by parts in $\mathrm{d}R$ yields
\begin{align*}
    \left|-\alpha\int\int\de_\beta\Ps \frac{R}{2}\de_R(\de^Nf_r)^2\right| & =\left|\alpha \int\int(\de_\beta\Ps+R\de_R\de_\beta \Ps)\frac{(\de^Nf_r)^2}{2}\right|\leq C\alpha \|\Ps\|_{\cW^{2,\infty}}\|f_r\|_{\cH^N}^2\\
    &\leq C_{2}\|f_r\|_{\cH^N}^2
\end{align*}
upon using \eqref{eq:est-fapp}. For $1\leq i\leq N$, it follows
\begin{equation*}
   \left|\left\langle -\alpha \sum_{i=1}^N\de^i\de_\beta \Ps \de^{N-i}(R\de_R f_{r}), \de^N f_r\right\rangle_{L^2}\right|\lesssim \alpha \|\de_\beta\Ps\|_{\cW^{N,\infty}}\|f_r\|_{\cH^N}^2\leq C_{N+1}\|f_r\|_{\cH^N}^2.
\end{equation*}
Hence
\begin{equation*}
    \left|\left\langle I_{1,2}, f_r\right\rangle_{H^N}\right| \leq C_{N+1}\|f_r\|_{\cH^N}^2. 
\end{equation*}
For $I_{1,3}$, the contribution involving derivatives of order $N$ amounts to
\begin{equation*}
     \left|\left\langle\de^N\left(-\alpha \de_\beta \Psi_r R\de_R \fe\right),\de^N f_r\right\rangle_{L^2}\right|\leq \left\|\sum_{i=0}^N\de^i\left(\alpha\de_\beta\Psi_r\right)\de^{N-i}(R\de_R\fe)\right\|_{L^2}\|f_r\|_{H^N}.
\end{equation*}
If $0\leq i\leq \frac{N}{2}$, then \eqref{eq:est-Psir-reg-infty} and \eqref{eq:est-Psir-sing-infty} yield
\begin{equation*}
    \left\|\de^i\left(\alpha\de_\beta\Psi_r\right)\de^{N-i}(R\de_R\fe)\right\|_{L^2}\leq \|\alpha\de^i\de_\beta\Psi_r\|_{L^{\infty}}\|\fe\|_{\cH^{N+1}}\leq C \left(\alpha\|\Ome\|_{\cH^N}+\|\Omega_{r}\|_{\cH^N}\right)\|\fe\|_{H^{N+1}}
\end{equation*}
Similarly for $\frac{N}{2}< i \leq N$, it follows again from \eqref{eq:est-Psir-reg-infty} and \eqref{eq:est-Psir-sing-infty} and the Sobolev embedding of Lemma \ref{lem:embedding} that
\begin{align*}
    \left\|\de^i\left(\alpha\de_\beta\Psi_r\right)\de^{N-i}(R\de_R\fe)\right\|_{L^2}
    &\leq \|\de^i\left(\alpha\de_\beta\Psi_r\right)\|_{L^2}\|\de^{N-i}(R\de_R\fe)\|_{L^{\infty}}\\&\leq C \left(\alpha\|\Ome\|_{\cH^N}+\|\Omega_{r}\|_{\cH^N}\right) \|\fe\|_{\cH^{N+1}}.
\end{align*}
Combining these estimates, we conclude that 
\begin{align*}
    \left|\left\langle I_{1,3},f_r\right\rangle_{H^n}\right|\leq C_{N+1}\eul^{\frac{C_{N+1}}{\alpha}t}\left(C_N\alpha+\alpha^2\eul^{\frac{C_{N}}{\alpha}t}+\|\Omega_r\|_{\cH^N}\right)\|f_r\|_{\cH^N}.
\end{align*}
%\red{*** Above: it is maybe $C_{N+1} \eul^{\frac{C_{N}}{\alpha}t}$} \textcolor{blue}{No, it is not due to the Sobolev embedding.}\\
The highest order term of $I_{1,4}$ is controlled in the same spirit as $I_{1,3}$. More precisely, we consider separately the cases $i=0$, $1\leq i\leq \frac{N}{2}$ and $\frac{N}{2}< i\leq N$ which leads to
\begin{multline*}
\left\langle-\alpha \de^N\left(\de_\beta \Psi_r R\de_R f_{r}\right), \de^N f_r \right\rangle_{L^2}
=-\left\langle\alpha\left(\de_\beta \Psi_r \de^N(R\de_R f_{r})\right), \de^N f_r \right\rangle_{L^2}\\-\left\langle\sum_{i=1}^{\lfloor \frac{N}{2}\rfloor}\de^i\left(\alpha\de_\beta \Psi_r \right)\de^{N-i}(R\de_R f_{r}), \de^N f_r \right\rangle_{L^2}-\left\langle\sum_{i=\lceil\frac{N}{2}\rceil}^{N}\de^i\left(\alpha\de_\beta \Psi_r \right)\de^{N-i}(R\de_R f_{r}), \de^N f_r \right\rangle_{L^2}.
\end{multline*}
Arguing for each contribution similarly to the respective cases for $I_{1,3}$, namely integrating by parts in $\mathrm{d}R$ for $i=0$ and using the $L^{\infty}$-estimate for $\de_\beta\Psi_r$ for $i\leq \frac{N}{2}$ and respectively the $L^2$-estimate for $i>\frac{N}{2}$ provided by Lemma \ref{lem:Psir}, one has
\begin{equation*}
    \left|\left\langle-\alpha \de^N\left(\de_\beta \Psi_r R\de_R f_{r}\right), \de^N f_r \right\rangle_{L^2}\right|\leq \left(\alpha C_N+\alpha^2\eul^{\frac{C_N}{\alpha}t}+\|\Omega_r\|_{\cH^N}\right)\|f_r\|_{\cH^N}^2
\end{equation*}
and hence the bound for $I_{1,4}$ follows along the same lines. For the contributions involving weighted norms, the strategy in \cite[Section 6]{tarek2} can be adapted in the same spirit as for the estimates detailed above.
Finally,
\begin{align*}
    \left|\left\langle I_1, f_r\right\rangle_{\cH^N}\right|&\leq C_{N+1}\eul^{\frac{C_{N+1}}{\alpha}t}\left(C_{N+1}+\alpha C_N\eul^{\frac{C_N}{\alpha}t}+\|\Omega_r\|_{\cH^N}\right)\|f_r\|_{\cH^N}\\&\quad+\left(C_{N+1}+\alpha C_N\eul^{\frac{C_N}{\alpha}t}+\|\Omega_r\|_{\cH^N}\right)\|f_r\|_{\cH^N}^2.
\end{align*}
Next, we consider
\begin{equation*}
    I_2=2\Ps \de_\beta f_r + 2 \Psi_r \de_\beta f_r + 2 \Psi_r \de_\beta \fe.
\end{equation*}
The explicit estimates for $\fe$ and $\Ps$, see \eqref{eq:est-fapp} and \eqref{est:Psiapp ell} one to bound
\begin{equation*}
    \left|\left\langle 2\Ps \de_\beta f_r,f_r \right\rangle_{\cH^N
    }\right|\leq \frac{C_{N+1}}{\alpha}\|f_r\|_{\cH^N}^2
\end{equation*}
by arguing as in the proof of $I_{1,2}$, namely considering the cases $i=0$, $i\leq \frac{N}{2}$ and $\frac{N}{2}<i\leq N$ and using the respective estimates from Lemma \ref{lem:Psir}. In the same spirit, we obtain
\begin{equation*}
     \left|\left\langle2 \Psi_r \de_\beta f_r,f_r\right\rangle_{\cH^N
    }\right|\leq C \|\de_\beta\Psi_r\|_{\cH^{N}}\|f_r\|_{\cH^N}^2\leq \left(C_N+\alpha\eul^{\frac{C_N}{\alpha}t}+\frac{\|\Omega_r\|_{\cH^N}}{\alpha}\right)\|f_r\|_{\cH^N}^2.
\end{equation*}
The last term of $I_2$ is bounded by
\begin{equation*}
    \left|\left\langle2 \Psi_r \de_\beta \fe,f_r\right\rangle_{\cH^N}\right|\leq C_{N+1}\eul^{\frac{C_{N+1}}{\alpha}t}\|\Psi_r\|_{
    \cH^N}\|f_r\|_{\cH^N}\leq C_{N+1}\eul^{\frac{C_{N+1}}{\alpha}t}\left(C_N+\alpha\eul^{\frac{C_{N}}{\alpha}t}+\frac{\|\Omega_r\|_{\cH^N}}{\alpha}\right)\|f_r\|_{\cH^N}.
\end{equation*}
Therefore, we conclude
\begin{align*}
    \left|\left\langle I_2, f_r\right\rangle_{\cH^N}\right|&\leq C_{N+1}\eul^{\frac{C_{N+1}}{\alpha}t}\left(C_N+\alpha\eul^{\frac{C_{N}}{\alpha}t}+\frac{\|\Omega_r\|_{\cH^N}}{\alpha}\right)\|f_r\|_{\cH^N}\\&+\quad \left(\frac{C_{N+1}}{\alpha}+C_N+\alpha\eul^{\frac{C_{N}}{\alpha}t}+\frac{\|\Omega_r\|_{\cH^N}}{\alpha}\right)\|f_r\|_{\cH^N}^2,
\end{align*}
where we have omitted the estimates on the weighted norms which can be recovered by combining the method detailed above and \cite[Section 6]{tarek2}. Finally, it remains to control 
\begin{equation*}
    I_3=\alpha R(\de_R \Psi_r + \de_R \Ps) (\de_\beta \fe + \de_\beta f_r).
\end{equation*}
Following the same strategy as for the previous terms and using the estimates \eqref{eq:est-Psir-R}-\eqref{eq:est-Psir-sing-infty} we infer 
\begin{equation*}
    \left|\left\langle \de^N\left(\alpha R\de_R\Psi_r\de_\beta\fe\right), \de^N f_r\right\rangle_{L^2}\right|
    %&\leq \alpha\left(\|R\de_R\Psi_r\|_{\cW^{\frac{N}{2},\infty}}+\|\de^{N}R\de_R\Psi_r\|_{L^{2}}\right)\|\de_\beta\fe\|_{\cH^{N}}\|f_r\|_{H^N}\\
    \leq C_{N+1}\eul^{\frac{C_{N+1}}{\alpha}t}\left(C_N+\alpha\eul^{\frac{C_{N}}{\alpha}t}+\|\Omega_r\|_{\cH^N}\right)\|f_r\|_{\cH^N}.
\end{equation*}
Moreover, proceeding as for $I_{1,4}$ by separating the cases $i=0$, $1\leq i\leq N/2$ and $N/2< i\leq N$ and from the estimates of Lemma \ref{lem:embedding} as well as Lemma \ref{lem:Psir} we have
\begin{equation*}
     \left|\left\langle \de^N\left(\alpha R\de_R\Psi_r\de_\beta f_r\right), \de^N f_r\right\rangle_{L^2}\right|\leq \alpha\|R\de_R\de_\beta\Psi_r\|_{\cH^N}\|f_r\|_{\cH^N}^2\leq C\left(C_N+\alpha\eul^{\frac{C_N}{\alpha}t}+\|\Omega_r\|_{\cH^N}\right) \|f_r\|_{\cH^N}^2
\end{equation*}
Finally, we prove along the same lines and upon using \eqref{est:Psiapp ell}  that
\begin{equation*}
     \left|\left\langle \de^N\left(\alpha R\de_R\Ps\de_\beta\fe\right), \de^N f_r\right\rangle_{L^2}\right|\leq C_{N+1}\eul^{\frac{C_{N+1}}{\alpha}t}\left(C_{N}+\alpha\eul^{\frac{C_{N}}{\alpha}t}\right)\|f_r\|_{\cH^N},
\end{equation*}
and 
\begin{equation*}
    \left|\left\langle \de^N\left(\alpha R\de_R\Ps\de_\beta f_r\right), \de^N f_r\right\rangle_{L^2}\right|\leq \left(C_{N}+\alpha\eul^{\frac{C_n}{\alpha}t}\right)\|f_r\|_{\cH^N}^2.
\end{equation*}
We conclude that
\begin{equation*}
    \left|\left\langle I_3, f_r\right\rangle_{\cH^N}\right|\leq C_{N+1}\eul^{\frac{C_{N+1}}{\alpha}t}\left(C_N+\alpha\eul^{\frac{C_{N}}{\alpha}t}+\|\Omega_r\|_{H^N}\right)\|f_r\|_{\cH^N}+\left(C_N+\alpha\eul^{\frac{C_N}{\alpha}t}+\|\Omega\|_{\cH^N}\right)\|f_r\|_{\cH^N}^2, 
\end{equation*}
where we have omitted the estimates on the weighted norms which can be recovered by combining the method detailed above and \cite[Section 6]{tarek2}.
Combing the estimates for $I_i$ with $i=1,2,3$ and up to modifying $C_{N+1}$ we conclude that 
\begin{align*}
     \left|\left\langle \mathcal{T}(f), f_r \right\rangle_{\cH^N}\right|&
     \leq C_{N+1}\eul^{\frac{C_{N+1}}{\alpha}t}\left(C_{N+1}+C_N+\alpha\eul^{\frac{C_{N}}{\alpha}t}+\frac{\|\Omega_r\|_{\cH^N}}{\alpha}\right)\|f_r\|_{\cH^N}\\
     &\quad +\left(\frac{C_{N+1}}{\alpha}+C_N+\alpha\eul^{\frac{C_{N}}{\alpha}t}+\frac{\|\Omega_r\|_{\cH^N}}{\alpha}\right)\|f_r\|_{\cH^N}^2.
\end{align*}
Observing that $C_N\leq C_{N+1}\leq \alpha^{-1}C_{N+1}$ yields the desired estimate. This completes the proof.
\end{proof}

Next we estimate the contributions stemming from $\mathrm{(RHS)}_\eta^1$ and $\mathrm{(RHS)}_\xi^1$ respectively. 

\begin{lem}\label{lem:RHS1}
    The following hold true for $\mathrm{(RHS)}_\eta^1$ and $ \mathrm{(RHS)}_\xi^1$ as defined in \eqref{eq:rhseta1} and \eqref{eq:rhsxi1} respectively,  
    \begin{align}
        \left\langle \mathrm{(RHS)}_\eta^1, \eta_r\right\rangle_{\cH^N}&\lesssim  \left(C_N+\alpha e^{\frac{C_N}{\alpha} t}+\frac{\|\Omega_r\|_{\cH^N}}{\alpha}\right)C_N e^{\frac{C_N}{\alpha} t}{{\|\eta_r\|_{\cH^N}}}\\
        &\quad +\left(\frac{C_N}{\alpha}+C_N+\alpha \eul^{\frac{C_N}{\alpha} t}+\frac{\|\Omega_r\|_{\cH^N}}{\alpha}\right)\|\eta_r\|^2_{\cH^N}\label{rhs_eta1},\\
    \left\langle \mathrm{(RHS)}_\xi^1, \xi_r\right\rangle_{\cH^N}&\lesssim \left(C_N+\alpha e^{\frac{C_N}{\alpha} t}+\frac{\|\Omega_r\|_{\cH^N}}{\alpha}\right)\left(C_N\eul^{\frac{C_N}{\alpha} t} +1\right){{\|\xi_r\|_{\cH^N}}}\\
    &\quad+\left(\frac{C_N}{\alpha}+C_N+\alpha \eul^{\frac{C_N}{\alpha} t}+\frac{\|\Omega_r\|_{\cH^N}}{\alpha}\right)\|\xi_r\|^2_{\cH^N}
    \label{rhs_xi1}.
    \end{align}
    % \begin{align}
    %     \left\langle \mathrm{(RHS)}_\eta^1, \eta_r\right\rangle_{\cH^N}&\lesssim  \left(C_N e^{\frac{C_N}{\alpha} t}+\frac{\|\Omega_r\|_{\cH^N}}{\alpha}\right)\left(\|\eta_r\|_{\cH^N}+\|\bar{\eta}_0\|_{\cH^N}e^{\frac{C_N}{\alpha} t} \right){{\|\eta_r\|_{\cH^N}}}+\frac{C_N}{\alpha}\|\eta_r\|^2_{\cH^N}\label{rhs_eta1},\\
    % \left\langle \mathrm{(RHS)}_\xi^1, \xi_r\right\rangle_{\cH^N}&\lesssim \left(C_Ne^{\frac{C_N}{\alpha} t}+\frac{\|\Omega_r\|_{\cH^N}}{\alpha}\right)\left(\|\xi_r\|_{\cH^N}+\|{\xi}_0^{\alpha, \delta}\|_{\cH^N}e^{\frac{C_N}{\alpha} t} +1\right){{\|\xi_r\|_{\cH^N}}}+\frac{C_N}{\alpha}\|\xi_r\|^2_{\cH^N}
    % \label{rhs_xi1}.
    % \end{align}
\end{lem}
\begin{proof}
    We prove \eqref{rhs_eta1}, the other one follows similarly. By using Cauchy-Schwartz inequality we have
    \begin{align*}
        \left\langle\frac{\cL(\Ome)}{2\alpha}\eta_r,\eta_r\right\rangle_{\cH^N}&\leq  \alpha^{-1}\|\cL(\Ome)\eta_r\|_{\cH^N}\|\eta_r\|_{\cH^N}.
    \end{align*}
    We want to estimate the first factor in the r.h.s. of the above inequality. We recall that $\cL(\cdot)$ is a radial function, so that expanding the norms we obtain
    \begin{equation}\label{zucca}
\|\cL(\Ome)\eta_r\|_{\cH^N}\lesssim\sum_{m=0}^N\sum_{i=0}^m\sum_{j=0}^i\|\partial_R^j\cL(\Ome)\partial_R^{i-j}\partial_{\beta}^{m-i}\eta_r\|_{L^2}+\|R^j\partial_R^j\cL(\Ome)R^{i-j}\partial_R^{i-j}\partial_{\beta}^{m-i}\eta_r\|_{L^2}.
    \end{equation}
    When $j<N$ we can put $\partial^j_R\cL(\Ome)$ and $R^j\partial^j_R\cL(\Ome)$ in $L^{\infty}$, in such a way we bound from above \eqref{zucca} by $\|\eta_r\|_{\cH^N}\|\mathcal{L}(\Ome)\|_{\cH^N}$, after using the one dimensional (since $\mathcal{L}$ is radial) Sobolev embedding. The case $j=N$ reduces to $\|\partial_R^N\cL(\Ome)\eta_r\|_{L^2}+\|R^N\partial_R^N\cL(\Ome)\eta_r\|_{L^2}$, in this case we may put $\eta_r$ in $L^{\infty}$ and conclude again by (two dimensional) Sobolev embedding. By means of \eqref{pesate1}, we may therefore conclude that
    \begin{equation*}
    \left\langle\frac{\cL(\Ome)}{2\alpha}\eta_r,\eta_r\right\rangle_{\cH^N}\lesssim\alpha^{-1}\|\cL(\Ome)\|_{\cH^N}\|\eta_r\|_{\cH^N}^2\lesssim \frac{C_N}{\alpha}\|\eta_r\|^2_{\cH^N},
    \end{equation*}
where we used again that $\cL(\Ome)=\cL(g)$ for which one has the bounds of Lemma \ref{StimeLg}. 
  We now want to estimate 
$$\langle\sin(2\beta)(\partial_{\beta\beta}\Psi_r)\eta_r,\eta_{r}\rangle_{\cH^N}.$$ It is enough to estimate
    $\|\sin(2\beta)(\partial_{\beta\beta}\Psi_r)\eta_r\|_{\cH^N}$, we then conclude by Cauchy-Schwartz. The factor $\sin(2\beta)$ is harmless, in fact when using Leibniz rule we can always put the derivatives falling on $\sin(2\beta)$ in $L^{\infty}$. We focus on $\|\partial^i(\partial_{\beta\beta}\Psi_r)\partial^{N-i}\eta_r\|_{L^2}$. When $i\geq 2$, we can put $\partial^{N-i}\eta_r$ in $L^{\infty}$ and then use Sobolev embedding of $H^{2}$ in $L^{\infty}$. In such a way we obtain
    \begin{equation}\label{margherita}
        \sum_{i=2}^N\|\partial^i(\partial_{\beta\beta}\Psi_r)\partial^{N-i}\eta_r\|_{L^2}\leq \|\partial_{\beta\beta}\Psi_r\|_{\cH^N}\|\eta_r\|_{\cH^N}\leq  \|\eta_r\|_{\cH^N}\left(C_N+\alpha \eul^{\frac{C_N}{\alpha} t}+\alpha^{-1}\|\Omega_r\|_{\cH^N}\right),
    \end{equation}
    where we used in the last step \eqref{eq:est-Psir-beta}. In the cases $i=0,1$ we can put $\partial^i\partial_{\beta\beta}\psi$ in $L^{\infty}$, use Sobolev embedding and bound it by the same quantity in the r.h.s. of \eqref{margherita}, again using \eqref{eq:est-Psir-beta} at the level $k=2,3$. Similarly one proves that 
    \begin{equation*}
\|\sin(2\beta)\partial_{\beta\beta}\Psi_r\eta_{\rm app}\|_{\cH^N}\lesssim \|\partial_{\beta\beta}\Psi_r\|_{\cH^N}\|\eta_{\rm app}\|_{\cH^N}\lesssim
\left(C_N+\alpha e^{\frac{C_N}{\alpha} t}+\alpha^{-1}\|\Omega_r\|_{\cH^N}\right) C_Ne^{\frac{C_N}{\alpha} t},
    \end{equation*}
    where in the last step we used \eqref{eq:est-LOM}.
    In view of \eqref{eq:est-Psir-beta}, the estimate for the term $\langle\cos(2\beta)\partial_{\beta}\Psi_r(\eta_r+\eta_{\rm app}),\eta_{r}\rangle_{\cH^N}$, may be obtained following word by word what done before.
%
The inequality for $\langle\mathrm{(RHS)}_\xi^1, \xi_r \rangle $ can be obtained in the same manner and therefore we omit it.
%\textcolor{red}{revise this last part}
% 
    %$$\cos(2\beta)\de_\beta\Ps(\Omega_r)-\frac{1}{2}\sin(2\beta)\de_{\beta\beta}\Ps(\Omega_r)=\frac{\mathcal{L}(\Omega_r)}{2\alpha}$$}
\end{proof}

We proceed to provide bounds for the terms involving $\text{(RHS)}_\eta^2, \text{(RHS)}_\xi^2$ appearing in the (RHS) of the equations for $\eta_r$ and $\xi_r$ respectively. 

\begin{lem}\label{lem:RHS2}
        The following hold true for $\mathrm{(RHS)}_\eta^2$ and $ \mathrm{(RHS)}_\xi^2$ as defined in \eqref{eq:rhs-etar1} and \eqref{eq:rhs-xir1} respectively,  
    \begin{align*}
        \left\langle \mathrm{(RHS)}_\eta^2, \eta_r\right\rangle_{\cH^N}& \lesssim \left(C_N+\alpha\eul^{\frac{C_N}{\alpha}t}\right)\left(C_N\eul^{\frac{C_N}{\alpha}t}+\|\Omega_r\|_{\cH^N}\right)\|\eta_r\|_{\cH^N}+\left(C_N+\alpha\eul^{\frac{C_N}{\alpha}t}+\|\Omega_r\|_{\cH^N}\right)\|\eta_r\|_{\cH^N}^2,\\
        \left\langle \mathrm{(RHS)}_\xi^2, \xi_r\right\rangle_{\cH^N}&\lesssim \left(C_N+\alpha\eul^{\frac{C_N}{\alpha}t}\right)\left(C_N\eul^{\frac{C_N}{\alpha}t}+\|\Omega_r\|_{\cH^N}\right)\|\xi_r\|_{\cH^N}+\left(C_N+\alpha\eul^{\frac{C_N}{\alpha}t}+\|\Omega_r\|_{\cH^N}\right)\|\xi_r\|_{\cH^N}^2.
    \end{align*}
\end{lem}
The proof follows the same strategy as in Lemma \ref{lem:remT} for the transport terms, more specifically the one for $I_1$ and $I_3$ defined in the respective proof.

The terms $\mathrm{I}_j, \, j \in \{4, 6, 7\}$ are exactly the same terms (with the same notation) as in [Section 6, \cite{tarek2}], we adapt the proof to our setting. %  it is easy to provide estimates of $\text{(RHS)}_\eta^2, \text{(RHS)}_\xi^2$.

\begin{proof}
 We sketch the idea of the proof. The main point is to notice that, for any $j \in \{4,6,7\}$, the term $\mathrm{I}_j= \mathrm{I}_{j,1}+\mathrm{I}_{j,2}$ can be decomposed in the sum of $\mathrm{I}_{j,1}=\mathcal{F}^2_j(\Psi_{\text{app}})$, depending only on $\Psi_{\text{app}}$ in \eqref{eq:psiapp},
and $\mathrm{I}_{j,2}=\mathcal{F}^r_j(\Psi_r)$, depending only on the difference $\Psi_r$. We emphasize that as the decomposition \eqref{def:psi-r} of $\Psi_r$ differs from the one in \cite{tarek2}, the estimates for $\mathrm{I}_{j,2}$ need to be adapted. Appealing to Proposition \ref{lem:35} to bound $\Ps$ and the estimate for $\eta_{\text{app}}$ given by Proposition \ref{lem:35}, one has that
\be
\ba
\left|(\mathrm{I}_{j,1}(\eta_r+\eta_{\text{app}}), \eta_r)_{\cH^N} \right|& \lesssim  \left(\|\alpha R\de_R\Psi_{\text{app}}+\alpha R\de_{\beta R}\Psi_{\text{app}}+\alpha^2 R^2\de_{RR}\Psi_{\text{app}}\|_{\cH^{N}}\right) (\|\eta_r \|_{\cH^N} +\|\eta_{\text{app}}\|_{\cH^N})\|\eta_r \|_{\cH^N}\\
& \lesssim  \|\Ome\|_{\cH^N}(\|\eta_r\|_{\cH^N} +C_N\eul^{\frac{C_N}{\alpha} t})\|\eta_r\|_{\cH^N}, 
\ea
\ee
where we proceeded like for $I_{1,2}$ in the proof of Lemma \ref{lem:remT} and in particular exploited \eqref{est:Psiapp ell}. 
Similarly,
\be
\ba
\left|(\mathrm{I}_{j,1}(\csi+\xi_r-1), \xi_r)_{\cH^N}\right| &  \lesssim  \|\Ome\|_{\cH^N}\left(1+\|\xi_r\|_{\cH^N} +C_N \eul^{\frac{C_N}{\alpha} t}\right)\|\xi_r\|_{\cH^N}. 
\ea
\ee
To control the contributions coming from $I_{j,2}$, we rely on the elliptic estimates of Lemma \ref{lem:Psir}. For the sake of a short exposition, we only consider the contribution of the $\cH^N$ scalar-product involving derivatives of $\de^N$. The lower order derivatives contributions are dealt with similarly. 
\be
\ba
(\de^N (\mathrm{I}_{j,2}(\eta_r+\eta_{\text{app}})), \de^N \eta_r)_{L^2} & = \sum_{i=0}^N (\de^{i} (\mathrm{I}_{j,2})\de^{N-i}(\eta_r+\eta_{\text{app}})), \de^N \eta_r)_{L^2} \\
&= \sum_{i=0}^{\frac{N}{2}} (\de^{i} (\mathrm{I}_{j,2}) \de^{N-i} (\eta_r+\eta_{\text{app}}), \de^N \eta_r)_{L^2} =:\text{(A)}\\
&\quad + \sum_{k=\frac{N}{2}}^N (\de^{N-i} (\mathrm{I}_{j,2}) \de^k (\eta_r+\eta_{\text{app}}), \de^N \eta_r)_{L^2}=:\text{(B)},
\ea
\ee
where
\be\ba
\text{(A)} & \le \sum_{i=0}^{\frac{N}{2}}  \|\de^{i} (\mathrm{I}_{j,2})\|_{L^\infty} \|\de^{N-i} (\eta_r+\eta_{\text{app}})\|_{L^2} \|\eta_r\|_{H^N}  \le C \|\mathrm{I}_{j,2}\|_{H^N} \|\eta_r+\eta_{\text{app}}\|_{H^N} \|\eta_r\|_{H^N},
\ea\ee
where we used the Sobolev embedding $\cH^N \hookrightarrow \cW^{1, \infty}$ for {$N \ge 3$}, see Lemma \ref{lem:embedding}. Similarly
\be\ba
\text{(B)} & \le  \sum_{i=\frac{N}{2}}^N \|\de^{i} (\mathrm{I}_{j,2})\|_{L^2} \| \de^{N-i} (\eta_r+\eta_{\text{app}})\|_{L^{\infty}}  \|\eta_r\|_{H^N} \le C  \|\mathrm{I}_{j,2}\|_{H^{N}} \| \eta_r+\eta_{\text{app}}\|_{H^N}  \|\eta_r\|_{H^N}.
\ea\ee
The contributions including weights are proving along the same lines. Upon observing that Lemma \ref{lem:Psir} allows one to infer that 
\begin{equation*}
    \|\mathrm{I}_{j,2}\|_{\cH^{N}}\lesssim \|\Ome\|_{\cH^N}+\|\Omega_r\|_{\cH^N},
\end{equation*}
and applying Proposition \ref{lem:35} to bound $\et,\csi$ we conclude that
\be
\ba
\left|(\mathrm{I}_{j,2}(\eta_r+\eta_{\text{app}}), \eta_r)_{\cH^N} \right|& \leq \left(\|\Ome\|_{\cH^N}+\|\Omega_r\|_{\cH^N}\right)\left(\|\eta_r\|_{\cH^N}+C_N\eul^{\frac{C_N}{\alpha}t}\right)\|\eta_r\|_{\cH^N}\\
\left|(\mathrm{I}_{j,2}(\xi_r+\csi-1), \xi_r)_{\cH^N}\right| &  \le \left(\|\Ome\|_{\cH^N}+\|\Omega_r\|_{\cH^N}\right)\left(1+\|\xi_r\|_{\cH^N}+C_N\eul^{\frac{C_N}{\alpha}t}\right)\|\xi_r\|_{\cH^N}.
\ea
\ee
The final estimates follow upon applying \eqref{eq:est-LOM} for $\|\Ome\|_{\cH^N}$.
\end{proof}
%%%%%
 
%%%%%
Next, we deal with the terms $\text{(RHS)}_\eta^2, \text{(RHS)}_\xi^2$. 

For the remaining terms, we derive the following estimates by exploiting the elliptic regularity of $\hE(\Omega)$.
\begin{lem}\label{lem:RHS3}
    The following hold true for $\mathrm{(RHS)}_\eta^3$ and $ \mathrm{(RHS)}_\xi^3$ as defined in \eqref{eq:rhs-etar1} and \eqref{eq:rhs-xir1} respectively,  
    \begin{align*}
        \left\langle \mathrm{(RHS)}_\eta^3, \eta_r\right\rangle_{\cH^N}&\lesssim \left(C_N \eul^{\frac{C_N}{\alpha}t}+C_N \eul^{\frac{C_N}{\alpha}t}(\|\xi\|_{\cH^N}+\|\Omega_r\|_{\cH^N})+\|\Omega_r\|_{\cH^N}\|\xi\|_{\cH^N}\right)\|\eta_r\|_{\cH^N},\\
        \left\langle \mathrm{(RHS)}_\xi^3, \xi_r\right\rangle_{\cH^N}&\lesssim  \left(C_N\eul^{\frac{C_N}{\alpha}t}+C_N \eul^{\frac{C_N}{\alpha}t}(\|\eta_r\|_{\cH^N}+\|\Omega_r\|_{\cH^N})+\|\Omega_r\|_{\cH^N}\|\eta_r\|_{\cH^N}\right)\|\xi_r\|_{\cH^N}.
    \end{align*}
\end{lem}

\begin{proof}
The key observation to prove the statement is that the terms to be controlled only depend on  $\hE(\Omega)$ being orthogonal to $\sin(2\beta)$ and hence its derivative in the angular direction admit  uniform (in $\alpha$) $\cH^N$-bounds. More precisely, it follows from Lemma \ref{lam:Rest} that
\begin{equation*}
    \|\de_\beta\hE(\Omega)\|_{\cH^k}\leq C  \|\de_{\beta\beta}\hE(\Omega)\|_{\cH^k}\leq C\|\Omega\|_{\cH^k}
\end{equation*}
upon exploiting the series expansion of $\hE(\Omega)$ as in the proof of Lemma \ref{lam:Rest}. By consequence,
\begin{equation*}
    \left\|2 \hE(\Omega) - \sin(2\beta) \de_\beta \hE(\Omega) + (\sin^2\beta) \de_{\beta \beta} \hE(\Omega)\right\|_{\cH^N}\lesssim \|\Omega\|_{\cH^N}\lesssim\left(C_N\eul^{\frac{C_N}{\alpha}t}+\|\Omega_r\|_{\cH^N}\right).
\end{equation*}
It then follows arguing as in the proof of Lemma \ref{lem:remT} and relying on \eqref{eq:est-Psir-N}, \eqref{eq:est-Psir-R} and \eqref{eq:est-Psir-reg-infty} that  
\begin{align*}
        \left|\left\langle \text{(RHS)}_\eta^3, \eta_r\right\rangle_{\cH^N}\right|&= \left|\left\langle [2 \hE(\Omega) - \sin(2\beta) \de_\beta \hE(\Omega) + (\sin^2\beta) \de_{\beta \beta} \hE(\Omega)](1-\csi-\xi_r), \eta_r\right\rangle_{\cH^N}\right|
        \\
        &\lesssim \|\Omega\|_{\cH^N}\left(1+\|\csi\|_{\cH^N}+\|\xi_r\|_{\cH^N}\right)\|\eta_r\|_{\cH^N}\\&
        \lesssim  (C_N \eul^{\frac{C_N}{\alpha}t}+\|\Omega_{r}\|_{\cH^N})\left({(1+\alpha C_{N+1})\eul^{\frac{C_N}{\alpha}t}} + \|\xi_r\|_{\cH^N}\right)\|\eta_r\|_{\cH^N}\\
        &\lesssim C_N\left({(1+\alpha C_{N+1})\eul^{\frac{2C_N}{\alpha}t}}+C_N\eul^{\frac{C_N}{\alpha}t}(\|\xi_r\|_{\cH^N}+\|\Omega_r\|_{\cH^N})+\|\Omega_r\|_{\cH^N}\|\xi_r\|_{\cH^N}\right)\|\eta_r\|_{\cH^N}.
    \end{align*}
Upon exploiting that $\alpha C_{N+1}\leq C_N$, the final estimate follows.
The estimate for the term involving $\xi$ follows along the same lines. 
\end{proof}

We are now ready to proof the desired estimates on the remainder terms $(\Omega_r,\eta_r,\xi)$.



\begin{proof}[Proof of Proposition \ref{prop:rem}]
    Combining the statements from Lemma \ref{lem:remT}-\ref{lem:RHS3}, we conclude that
\begin{align*}
     \frac{\mathrm{d}}{\mathrm{d}t}F(t)^2&\leq C_{N+1}\eul^{\frac{C_{N+1}}{\alpha}t}\left(C_{N+1}+\alpha\eul^{\frac{C_N}{\alpha}t}\right)
     F(t)+C_{N+1}\eul^{\frac{C_{N+1}}{\alpha}t}\frac{\|\Omega_r\|_{\cH^N}}{\alpha}F(t)\\
     &\quad +\left(\frac{C_{N+1}}{\alpha}+C_N\eul^{\frac{C_N}{\alpha}t}\right)F(t)^2
      +\frac{\|\Omega_r\|_{\cH^N}}{\alpha}F(t)^2+F(t)^3\\
     &\leq 
     C_{N+1}\eul^{\frac{C_{N+1}}{\alpha}t}\left(C_{N+1}+\alpha\eul^{\frac{C_N}{\alpha}t}\right)
     F(t)
     +\left(\frac{C_{N+1}}{\alpha}+\frac{C_{N+1}}{\alpha}\eul^{\frac{C_{N+1}}{\alpha}t}\right)F(t)^2+\frac{1}{\alpha} F(t)^3.
\end{align*}
By means of a classical approximation argument one obtains
\begin{equation}\label{eq:boot}
F(t)\leq F(0)+C_{N+1}\int_0^tC_{N+1}\eul^{\frac{C_{N+1}}{\alpha}\tau}+\alpha \eul^{\frac{2C_{N+1}}{\alpha}t}d\tau +\int_0^{t}\left(\frac{C_{N+1}}{\alpha}+\frac{C_{N+1}}{\alpha}\eul^{\frac{C_{N+1}}{\alpha}\tau}\right)F(\tau)d\tau+
\int_0^{t}\frac{1}{\alpha} F(\tau)^2d\tau.
\end{equation}
We recall that the initial condition satisfies $F(0)\lesssim C_{N+1}\alpha \leq \alpha^{1-\eps}$ for any $\eps>0$ for $\alpha$ sufficiently small. Owing to the bootstrap argument provided by the following Lemma we conclude that, for $\alpha$ small enough, $F(t)\leq 3 \sqrt{\alpha}$ for $\eps>0$ and over a time interval $[0,T]$, with $T\geq \frac{\alpha \log|\log(\alpha)|}{4C_{C_{N+1}}}$.
\end{proof}
\begin{Lem}\label{eq:bootstrap}
    There exists $\alpha_0>0$ such that for any $0<\alpha<\alpha_0$ if 
   \begin{equation}\label{hyp:boot}
        F(0)\leq \alpha^{1-\eps}, \quad \sup_{t\in[0,T]}F(t)\leq 4 \sqrt{\alpha}, \quad T\leq \frac{\alpha\log |\log(\alpha)|}{4C_{N+1}},
    \end{equation}
    then actually $\displaystyle\sup_{t\in[0,T]}F(t)\leq 3 \sqrt{\alpha}$.
    \end{Lem}
\begin{proof}
  Because  of inequality \eqref{eq:boot} and assumptions \eqref{hyp:boot}, we have
  \begin{align*}
      F(t)&\leq \alpha^{1-\eps}+C_{N+1}\int_0^tC_{N+1}\eul^{\frac{C_{N+1}}{\alpha}\tau}+\alpha \eul^{\frac{2C_{N+1}}{\alpha}t}d\tau
      +\frac{C_{N+1}}{\alpha}\int_0^{t}\left(1+\eul^{\frac{C_{N+1}}{\alpha}\tau}\right)F(\tau)d\tau+
      16\alpha\int_0^{t}\frac{1}{\alpha}d\tau\\
      &\leq 
      \alpha^{1-\eps}+C_{N+1}\alpha |\log\alpha|^{\frac14}+\alpha^2|\log\alpha|^{\frac12}
      +\frac{C_{N+1}}{\alpha}\int_0^{t}\left(1+\eul^{\frac{C_{N+1}}{\alpha}\tau}\right)F(\tau)d\tau+
      \frac{\alpha\log|\log\alpha|}{4C_{N+1}}\\
      &\leq 3\alpha^{1-\varepsilon}+\frac{C_{N+1}}{\alpha}\int_0^{t}\left(1+\eul^{\frac{C_{N+1}}{\alpha}\tau}\right)F(\tau)d\tau,
  \end{align*}
  where in the last inequality we have chosen $\alpha$ small enough. By means of the integral form of Gronwall lemma we obtain
  \begin{equation*}
F(t)\leq 3\alpha^{1-\varepsilon}\exp\left(\frac{C_{N+1}}{\alpha}\int_0^{t}\left(1+\eul^{\frac{C_{N+1}}{\alpha}\tau}\right)d\tau\right)\leq 3\alpha^{1-\varepsilon}\exp \left(\frac{\log|\log\alpha|}{4}+|\log\alpha|^{1/4}\right)\leq 3 \sqrt{\alpha},
  \end{equation*}
  where, again, we have chosen $\alpha$ small enough.
\end{proof}







\section{Proof of Theorem \ref{thm:main} and Theorem \ref{thm:main2} }\label{sec:last}
We finally have all the ingredients to prove our result. 
We first initialize the approximating Leading Order Model \eqref{eq:leading1} with the following data 
\be
\Omega_{0, \mathrm{app}}^{\alpha, \delta}(R, \beta)= \Omega_0^{\alpha, \delta}(R, \beta), \quad \eta_{0, \mathrm{app}}^{\alpha, \delta}(R, \beta)= \bar \eta_0^{\alpha, \delta}(R), \quad \xi_{0, \mathrm{app}}^{\alpha, \delta}(R, \beta)= \de_y \rho_0^{\alpha, \delta}(x, y). 
\ee
We can then use Proposition \ref{prop:expl}, yielding that the approximated (horizontal) density gradient blows up as 
\be\ba
\sup\limits_{t \in [0, T^*(\alpha)]} \|\de_x \rho_{\mathrm{app}}(t) \|_{L^\infty(\R^2)}  &\ge  \|\bar \eta_{0}^{\alpha, \delta}\|_{L^\infty (\R^2)} \left(1+\frac{\log |\log \alpha|}{C_{N+1}}\right)^{\frac{1}{c_2}},
\ea\ee
for $T^*(\alpha)\sim  \frac{\alpha}{C_{N+1}}\log |\log (\alpha)| \, \to \, 0$.
Now, for $\de_x \rho (t, x, y)$ solving \eqref{eq:2dbouss-grad} with initial datum $\de_x \rho_0^{\alpha, \delta}$, using Proposition \ref{prop:rem} 
%
\be
\ba
\|\de_x \rho (t) \|_{L^\infty(\R^2)}& \ge  \| \de_x \rho_{\mathrm{app}} (t)\|_{L^\infty} -  \| \de_x \rho_{\mathrm{app}} (t) - \de_x \rho (t)\|_{L^\infty} \gtrsim \| \de_x \rho_{\mathrm{app}} (t)\|_{L^\infty} -  \| \de_x \rho_{\mathrm{app}} (t) - \de_x \rho (t)\|_{\cH^k}\\
& \gtrsim \| \de_y \rho_{\mathrm{app}} (t)\|_{L^\infty} -  \| \eta_r (t)\|_{\cH^k}
 \gtrsim \| \de_x \rho_{\mathrm{app}} (t)\|_{L^\infty} -  F(t) \gtrsim \| \de_x \rho_{\mathrm{app}} (t)\|_{L^\infty} - \sqrt \alpha, 
\ea
\ee
for $t \in [0, T^*(\alpha)]$.
%
Taking $0 < \alpha \le \alpha_0 \ll 1$ small enough, this implies that
\be
\|\de_x \rho (t) \|_{L^\infty(\R^2)} \ge \frac 12 \| \de_x \rho_{\mathrm{app}} (t)\|_{L^\infty}, \quad t \in [0, T^*(\alpha)],
\ee
so providing the result and concluding the proof of Theorem \ref{thm:main}. Up to a simple change of variable ($\wt \rho = \rho - \bar \rho_{\mathrm{eq}}$), the proof of Theorem \ref{thm:main2} is identical and therefore omitted.
\begin{Rmk}[On the $W^{1, \infty}$ strong ill-posedness of the original 2d Boussinesq system without background stratification]\label{rmk:originalB}
A final remark is that Theorem \ref{thm:main2}, which states the $W^{1, \infty}$ strong ill-posedness of the original 2d Boussinesq system \eqref{eq:2dbouss} around the steady state $\bar\rho_{\mathrm{eq}}=-y, \bar \bu_{\mathrm{eq}}=0$, can be also proved for the case $\bar\rho_{\mathrm{eq}}=0$ (without background stratification). The strategy of the proof and the related computations are exactly the same. This is immediate once one realizes that the equations for $\Ome$ and $\et$ in the Leading Order Model for the original 2d Boussinesq system \eqref{eq:2dbouss} are exactly the ones of \eqref{eq:leading1}. The only difference would be the equation for $\csi$ in \eqref{eq:leading1}, where the term $\frac{\cL(\Ome)}{2\alpha}$ in the right-hand side should be discarded.
\end{Rmk}
%%%%%
\section{Strong $W^{1, \infty}$ ill-posedness of 3d incompressible Euler equations with swirl}\label{sec:euler}
Consider the 3d incompressible Euler equations posed on $\R^3$ 
\be\label{eq:3DEuler}
\ba
\de_t \bu + (\bu \cdot \nabla) \bu + \nabla P &=0, \\
\nabla \cdot \bu&=0, 
\ea
\ee
where $\bu=\bu(t, x, y, z)=(u_1(t, x, y, z), u_2(t, x, y, z), u_3(t, x, y, z))^T: [0, \infty) \times \R^3 \to \R^3$ is the velocity field and $P=P(t, x, y, z) : [0, \infty) \times \R^3 \to \R$ is the incompressible pressure. Our approach relies on the analogy of \eqref{eq:3DEuler} and \eqref{eq:2dbouss} which is well-established provided that the solutions under consideration are supported away from the $z$-axis, see \cite{drivas, jeong2}. To that end, the compact support of the initial data chosen, see \eqref{eq:eta-initial}, is contained in 
$\{(x, y, z) \in \R^3 \, : \, 1 \le x^2+y^2\}$.
%
A bootstrap argument will ensure that this property holds on a sufficiently long time scale. The vorticity field 
\be
\omega := \nabla \times \bu,
\ee
satisfies the following equation
\be\label{eq:3dvorticity}
\de_t \omega + (\bu \cdot \nabla) \omega = (\omega \cdot \nabla) \bu,
\ee
where the right-hand side is referred to as \emph{vortex stretching}.
Passing to cylindrical coordinates 
\be
r:= \sqrt{x^2+y^2}, \qquad \theta=\arctan\left(\frac{y}{x}\right), \qquad z,
%\rho=\sqrt{r^2+z^2}, \qquad 
%\beta:= \arctan (z/r),
\ee
 we consider the special case of the \emph{axisymmetric} 3d Euler equations, which are symmetric with respect to the vertical $z$-axis and therefore the velocity $\bu $ (resp. the vorticity $\omega$) and the pressure $P$ are independent of the angle $\theta$, namely
\be\ba
(u_r^\aa (t, r, z), u_\theta^\aa (t, r, z), u_z^\aa (t, r, z))&=: \bu (t, x, y, z)^T, \\
(\omega_r^\aa (t, r, z), \omega_\theta^\aa (t, r, z), \omega_z^\aa (t, r, z))&=: \omega  (t, x, y, z)^T, \\
P^\aa (t, r, z)&=: P(t, x, y, z).
\ea\ee
Note that the 3d incompressibility condition $\nabla \cdot \bu=0$ in cylindrical coordinates reduces to 
\begin{equation*}
   \de_r u_r^\aa+\de_z u_z^\aa+\frac{1}{r}u_r^\aa=0.
\end{equation*}
Provided that $r>0$, this yields the weighted incompressibility condition $\Div_{r,z}(r(u_r^\aa, u_z^\aa))=0$ and thus
\begin{equation*}
    (u_r^\aa,u_z^\aa)^{T}=-\frac{1}{r}\nabla_{(r,z)}^{\perp}\wt \psi^\aa,
\end{equation*}
for some stream-function $\wt \psi^\aa$ where we recall that $\nabla_{(r,z)}^\perp = (-\de_z, \de_r)^T$. The identity  $\omega_\theta^\aa=\de_zu_r^\aa-\de_ru_z^\aa$ motivates the introduction of 
the \emph{potential vorticity} 
%
\begin{equation}\label{eq:potential-vorticity}
    \frac{\omega_\theta^\aa}{r}
\end{equation}
%
such that
%
\begin{equation}\label{eq:elliptic3d1}
    \frac{\de_{zz} \wt \psi^\aa}{r^2} + \frac 1 r \de_r \left(\frac{\de_r \wt \psi^\aa}{r} \right)=- \left(\frac{\omega_\theta^\aa}{r}\right).
\end{equation}
The potential vorticity then satisfies the transport equation with source term below
\begin{equation*}
  \de_t \left(\frac{\omega_\theta^\aa}{r}\right) + u_r^\aa \de_r \left(\frac{\omega_\theta^\aa}{r}\right)  + u_z^\aa \de_z \left(\frac{\omega_\theta^\aa}{r}\right)  = - \frac{1}{r^4} \de_z (r u_\theta^\aa)^2,  
\end{equation*}
where the right-hand side accounts for the \emph{swirl}. The velocity field $(u_r^\aa,u_z^\aa)$ is uniquely determined by the potential vorticity $r^{-1}\omega_{\theta}^\aa$ through the elliptic problem \eqref{eq:elliptic3d1}, while the swirl component $u_\theta^\aa$ of the velocity field satisfies
\begin{equation}\label{eq: trans swirl}
    \de_t (r u_\theta^\aa) +  u_r^\aa \de_r (r u_\theta^\aa) + u_z^\aa \de_z (r u_\theta^\aa)=0,
\end{equation}
and furthermore exploiting the regularity properties of the solutions \cite{kato72, constantin, bertozzi} under consideration gives 
\begin{align}\label{eq:utheta2}
        \de_t (r u_\theta^\aa)^2 +  u_r^\aa \de_r (r u_\theta^\aa)^2 + u_z^\aa \de_z (r u_\theta^\aa)^2=0.
\end{align}

The analogy between the 3d axisymmetric Euler equations with \emph{swirl} and the 2d Boussinesq equations becomes evident at this stage: the couple potential vorticity $r^{-1}\omega_\theta^\aa$ and $(ru_\theta^\aa)^2$ plays the role of vorticity $\omega$ and buoyancy $\rho$ in the Boussinesq equations \eqref{eq:2dbouss} respectively. Following this analogy, we derive the respective version of \eqref{eq:2dbouss-grad} for the 3d axisymmetric Euler equations with \emph{swirl} by applying the gradient $\nabla_{(r,z)}$ to \eqref{eq:utheta2}. We are led to 
\be\label{eq:3daxisymm}\ba
\de_t \left(\frac{\omega_\theta^\aa}{r}\right) + u_r^\aa \de_r \left(\frac{\omega_\theta^\aa}{r}\right)  + u_z^\aa \de_z \left(\frac{\omega_\theta^\aa}{r}\right)  &= - \frac{1}{r^4} \de_z (r u_\theta^\aa)^2,\\
\de_t \de_r (r u_\theta^\aa)^2 +  u_r^\aa \de_{rr} (r u_\theta^\aa)^2 + u_z^\aa \de_{zr} (r u_\theta^\aa)^2&=- (\de_r u_r^\aa )\de_r  (r u_\theta^\aa)^2 - (\de_r u_z^\aa) \de_{z} (r u_\theta^\aa)^2 ,\\
 \de_t\de_z(r u_\theta^\aa)^2 +  u_r^\aa \de_{rz} (r u_\theta^\aa)^2 + u_z^\aa \de_{zz} (r u_\theta^\aa)^2&=-(\de_zu_r^\aa)\de_r(ru_\theta^\aa)^2-(\de_zu_z^\aa) \de_z(ru_\theta^\aa)^2,
% \de_t  (r \de_\beta u_\beta^\aa) +  u_r^\aa \de_{r} (r \de_\beta u_\beta^\aa) + u_3^\aa \de_{z} (r \de_\beta u_\beta^\aa)&=- (\de_\beta u_r^\aa )\de_r  (r u_\beta^\aa) - (\de_\beta u_3^\aa) \de_{z} (r u_\beta^\aa),
\ea\ee
together with the elliptic equation 
%
\be
-{\de_{zz}  \psi^\aa}-{\de_{rr}  \psi^\aa}  - \frac 1 r {\de_r  \psi^\aa} + \frac{\psi^\aa}{r^2}={\omega_\theta^\aa},  \label{eq:elliptic3d}
\ee
%
where we set $\psi^\aa=r^{-1}\wt\psi^\aa$ and where we impose  an odd symmetry in $z$ with conditions
%
\be\label{eq:3daxiboundarycond}
\psi^\aa (r, 0)= \psi^\aa (0, z)=0.
\ee
%%%%
Now, since the support of our initial data will be far from the vertical axis $r=0$, we set 
\begin{equation}\label{eq:changeofvariablesalpha}
    r= 1+\zeta \qquad \Rightarrow \, \de_r \, \to \, \de_\zeta.
\end{equation}
By the change of coordinates in the $(r,z)$- plane to polar coordinates $(\rho, \beta)$ centered in $(r_0,z_0)=(1,0)$ with 
\begin{equation*}
    \rho=\sqrt{(r-1)^2+z^2}=\sqrt{\zeta^2+z^2}, \quad \beta=\arctan\left(\frac{z}{\zeta}\right),
\end{equation*}
the radial scaling $R=\rho^\alpha$ for $0 < \alpha \ll 1$ as in \eqref{eq:polar-coord}, together with the new variables defined as in \eqref{eq:new-var} and \eqref{eq:psi-polar} by
\be\label{eq:psibig}
\Omega (\cdot, R, \beta):= {\omega^\aa_\theta (\cdot, r , z)}, \qquad \Psi (\cdot, R, \beta):= \rho^{-2} \psi^\aa (\cdot, r, z), 
\ee
one finally has that
\be\label{eq:der-3d}
\de_r \; \Rightarrow \; R^{-\frac{1}{\alpha}}\left(\alpha\cos \beta R \de_R - \sin \beta \de_\beta \right), \qquad \de_z  \; \Rightarrow \;   R^{-\frac{1}{\alpha}}\left(\alpha\sin \beta R \de_R +\cos \beta \de_\beta\right)
\ee
which we note to be similar to \eqref{eq:spatial-der}.
%
This change of coordinates yields
\be\label{eq:uru3}\ba
u_r^\aa &=\de_z\psi^\aa= \rho (2 (\sin \beta) \Psi + (\cos \beta) \de_\beta \Psi + \alpha (\sin \beta) R \de_R \Psi), \\
u_z^\aa&=-\de_r\psi^\aa-\frac{1}{1+\zeta}\psi^\aa= \rho \left(- \frac{1}{\cos \beta} \Psi - 2 (\cos \beta) \Psi + (\sin \beta) \de_\beta \Psi - \alpha (\cos \beta) R \de_R \Psi\right),
\ea\ee
which is analogous to \eqref{eq:u1}-\eqref{eq:u2} (up to a switching of signs).
The elliptic equation \eqref{eq:elliptic3d} reads
\begin{align}
-\alpha^2 R^2 \de_{RR} \Psi - \alpha (\alpha + 5) R \de_R \Psi - \de_{\beta \beta} \Psi + \de_\beta ((\tan \beta) \Psi)-6 \Psi&=\Omega, \label{eq:elliptic3dnew}
\end{align}
with 
\be
\Psi (R, 0)= \Psi \left(R, \frac \pi 2\right)=0.
\ee
%
We refer for instance to \cite[Section 2]{tarek1} and \cite[Section 1.2]{jeong} and references therein for further details on the derivation and properties of \eqref{eq:elliptic3dnew}.
As in \cite{chen1}, we will rely on elliptic estimates in the weighted space
\begin{align}\label{def:weights3d}
    \|f\|_{\cH^k(\rho_i)}:= \sum_{i=0}^k \|R^i\de_R^i f \rho_1^{1/2}\|_{L^2}+ \|R^i\de_R^i \de_\beta^{k-i} f \rho_2^{1/2}\|_{L^2}, \; \rho_i= \frac{(1+R)^4}{R^4} \sin (2\beta)^{-\sigma_i}, \; \sigma_1=\frac{99}{100}, \; \sigma_2=1+\frac{\alpha}{10}.
\end{align}
Throughout this section, we use the notation $\cH^k:=\cH^k(\rho_i)$ and shall prove the following. 
\begin{Thm}[$W^{1, \infty}$ ill-posedness of the 3d axisymmetric Euler equations with swirl]\label{thm:main3sec3}
Let $N \ge 4$. There exist $0<\alpha_0 \ll 1$ and $\alpha_0 \ll \delta \ll 1$ such that, for any $0<\alpha \le \alpha_0 $, there exist initial data 
%
\be\label{eq:initialdata3d}
\omega^\aa_{\theta,0}(r, z)=-\Omega_0^{\alpha, \delta}(R, \beta), \qquad (r u_{\theta,0}^\aa)^2(r,z)=U_0^{\alpha, \delta}(R, \beta)
\ee
%
satisfying the following bounds
%
\be\ba
\left\|({\omega^\aa_{\theta,0}}, (r u_{\theta,0}^\aa)^2)\right\|_{L^\infty(\R^2)} &\sim  \delta, \qquad \|(\de_r (ru_{\theta,0}^\aa)^2, \de_z (r u_{\theta,0}^\aa)^2)\|_{L^\infty(\R^2)} \sim  \delta,\\
\left\|({\omega^\aa_{\theta,0}}, (r u_{\theta,0}^\aa)^2)\right\|_{\cH^N(\R^2)} &\sim   C_N, \qquad \|(\de_r(r u_{\theta,0}^\aa)^2, \de_z (ru_{\theta,0}^\aa)^2)\|_{\cH^N(\R^2)} \sim C_N + \alpha C_{N+1}.
\ea \ee
%
so that one of the following holds:
\begin{enumerate}[(i)]
\item If the initial datum is taken of the following form 
\begin{align}\label{eq:3deuler-initialU}
    \Omega_0^{\alpha, \delta}(R, \beta)=  \bar g_0^{\alpha, \delta} (R) \sin (2\beta)\cos \beta, \qquad U_0^{\alpha, \delta}(R, \beta)= R^{1/\alpha} \bar \eta_0^{\alpha, \delta} (R) \cos \beta, 
\end{align}
where $\bar g_0^{\alpha, \delta} (R), \bar \eta_0^{\alpha, \delta} (R) \in C_c^\infty([1/10, \infty))$ with
$$\bar g_0^{\alpha, \delta}(R)\ge 0, \quad \|(\bar g_0^{\alpha, \delta}, \bar \eta_0^{\alpha, \delta})\|_{\cH^N} \sim C_N,$$ then the corresponding solution $\omega_\theta^\aa (t,r,z), u_\theta^\aa (t, r, z)$ to the Cauchy problem associated with the 3d axi\-symmetric Euler equations with swirl \eqref{eq:3daxisymm}-\eqref{eq:elliptic3d} satisfies
%%%%
\be\label{eq:crucial-inequality}\ba
\sup\limits_{t \in [0, T^*(\alpha)]} \left\|\left(\frac{u_\theta^\aa}{r}+\de_r (u_\theta^\aa)^2\right) (t) \right\|_{L^\infty}  &\gtrsim  \|\de_r(u_{\theta,0}^\aa)^2\|_{L^\infty} \left(1+\frac{\log |\log \alpha|}{C_{N+1}}\right)^{\frac{1}{c_2}},
\ea\ee
for $c_2>0$ independent of $\alpha$;
\item if the initial datum is taken of the following form 
\begin{align}
    \Omega_0^{\alpha, \delta}(R, \beta)=  \bar g_0^{\alpha, \delta} (R) \sin (2\beta)\cos \beta, \qquad U_0^{\alpha, \delta}(R, \beta)= R^{1/\alpha} \bar \eta_0^{\alpha, \delta} (R) \sin \beta, 
\end{align}
where $\bar g_0^{\alpha, \delta} (R), \bar \eta_0^{\alpha, \delta} (R) \in C_c^\infty([1/10, \infty))$ with
$$\bar g_0^{\alpha, \delta}(R)\le 0, \quad \|(\bar g_0^{\alpha, \delta}, \bar \eta_0^{\alpha, \delta})\|_{\cH^N} \sim C_N,$$
 then 
\be\label{eq:omegablow-upthm}
\ba
\sup\limits_{t \in [0, T^*(\alpha)]} \|\omega_\theta (t) \|_{L^\infty } &\gtrsim \|\omega_{\theta, 0}^\aa\|_{L^\infty} \left(1+\frac{\log |\log \alpha|}{C_{N+1}}\right)^{\frac{1}{c_2}},\\
\sup\limits_{t \in [0, T^*(\alpha)]} \|\de_z(u_\theta^\aa)^2 (t) \|_{L^\infty } &\gtrsim \|\de_z (u_{\theta, 0}^\aa)^2 \|_{L^\infty} \left(1+\frac{\log |\log \alpha|}{C_{N+1}}\right)^{\frac{1}{c_2}},
\ea\ee
\end{enumerate}
for $c_2>0$ independent of $\alpha$; \\
and where the time-scale for (i), (ii) is given by
\be\label{def:talpha-star}
T^*(\alpha)\sim  \frac{\alpha}{C_{N+1}}\log |\log (\alpha)| \, \to \, 0
\quad \text{as} \quad \alpha \to 0,\ee
and such that $\frac{\log |\log \alpha|}{C_{N+1}} \gtrsim \log |\log \alpha|^\mu$ for some $0<\mu <1$.
\end{Thm}
\begin{Rmk}[Comparison with the results of Elgindi \& Masmoudi \cite{tarek3} and  Bourgain \& Li \cite{bourgain2015}]\label{rmk:comparison}
As pointed out in the introduction, we recall that the first results of $W^{1, \infty}$ ill-posedness of the 3d incompressible Euler equations are due to Elgindi \& Masmoudi \cite{tarek3} and  Bourgain \& Li \cite{bourgain2015}. The mechanism of ill-posedness of Theorem \ref{thm:main3sec3} is completely different from \cite{bourgain2015}, and in particular from \cite[Theorem 1.6]{bourgain2015}, which is about the 3d axisymmetric Euler equations without swirl. Notice indeed that in the first point of our Theorem \ref{thm:main3sec3}, the $L^{ \infty}$-norm inflation of the full vorticity field when $\bar g_0^{\alpha, \delta}\ge 0$ is a consequence of the norm inflation of the swirl $u_\theta^\aa$, which is exactly zero in \cite{bourgain2015}. Hence, it is more interesting to compare Theorem \ref{thm:main3sec3} with \cite[Proposition 10.1]{tarek3}, where instead of considering the 3d axisymmetric Euler equations with velocity field depending only on $(r=\sqrt{x^2+y^2}, z)$ (and symmetric with respect to the vertical axis), the authors consider the situation where the velocity field depends only on the plane $(x, y)$. Though the framework is different, they provide an example of a datum for which the third component of the velocity field exhibits a norm inflation, while the horizontal vorticity (which may be compared to our potential vorticity) remains bounded.
\end{Rmk}
%
%
The proof of Theorem \ref{thm:main3sec3} follows. While the method of the proof is very similar to the one of Theorem \ref{thm:main}, the main difference consists in an argument ensuring the validity of the analogy of the Boussinesq \eqref{eq:2dbouss-grad} and axisymmetric Euler \eqref{eq:3daxisymm}-\eqref{eq:elliptic3d} for solutions supported away from the symmetry axis. We outline the proof of Theorem \ref{thm:main3sec3},  highlighting the important differences only.
\begin{enumerate}
    \item Derivation of the Leading Order Model \eqref{eq:LOM3d1} and growth rates for a suitable class of initial data supported away from $z=0$ on $[0,T^*(\alpha)]$.
   % \item Estimates on the growth rates of the solutions \eqref{eq:LOM3d1} in 
    \item Localized (in space) elliptic estimates on $[0,T']$, where $T'$ is the supremum over all times for which the support of the solution remains bounded away from $r=0$, see \eqref{eq:T'}.
    \item Remainder estimates in $[0,T']$, especially for the contribution of the swirl in the first equation of \eqref{eq:3daxisymm}.
    \item A bootstrap argument yielding that $T'\geq T^*(\alpha)$.
    \item Conclusion of the proof of Theorem \ref{thm:main3sec3}.
\end{enumerate}
Each step is presented in a dedicated sub-section. 
\subsection{Derivation of the Leading Order model (LOM)}
As in Section \ref{sec:derivation} for the Bousinesq equations, the first step towards the LOM consists in a suitable expansion of the stream-function $\Psi$. We recall the 3d version of Elgindi's decomposition \cite{tarek1} of the Biot-Savart law.
\begin{Thm}[Proposition 7.1 in \cite{tarek1}, \cite{drivas}]\label{thm:tarek3d}
The unique regular solution to 
\begin{equation}\label{eq:ellittica3d}
-\alpha^2 R^2 \de_{RR} \Psi - \alpha (\alpha + 5) R \de_R \Psi - \de_{\beta \beta} \Psi + \de_\beta ((\tan \beta) \Psi)-6 \Psi= \Omega,
\end{equation}
with boundary conditions $\Psi(R, 0)=\Psi(R, \frac \pi 2)=0$
is given by
\be\label{eq:psi-main3d}
\Psi=\Psi(\Omega)(R, \beta) =\Psa + \hE, 
\ee 
where
\be \label{eq:psi23d}
\Psa=\Psa( \Omega)(R, \beta):=  \Ps( \Omega)(R) \sin (2\beta) + \mathcal{R}^\alpha ( \Omega)(R)\sin (2\beta)
\ee
%satisfies the equation 
%\begin{align}\label{eq:ell23d}
   % -\alpha^2 R^2 \de_{RR} \Psi_2 - \alpha (\alpha + 5) R \de_R \Psi_2 = \frac{\sin 2\beta }{\pi} \int_0^{\pi/2} \Omega (R, \beta) 
   % \sin (2\beta)   \cos \beta  \, d\beta,
%\end{align}
and 
\be \label{eq:psiapp3d}
\Ps=\Ps( \Omega)(R, \beta):=  \frac{\cL_{12} ( \Omega)(R)}{4 \alpha}\sin (2\beta),\qquad \cL_{12} ( \Omega)(R):= \frac{3}{8 \pi} \int_R^\infty \int_0^{2\pi} \frac{ \Omega(s, \beta) \sin (2\beta)\cos \beta}{s}\, d\beta \, d s,
\ee
satisfies the equation
\be\label{eq:L0}
6\Ps-\de_{\beta \beta }\Ps+ \de_\beta ((\tan \beta) \Ps) =0
\ee
and the error term 
$\mathcal{R}^\alpha( \Omega)(R)$ satisfies the inequality
\begin{align}\label{eq:hardy3d}
\|\mathcal{R}^\alpha( \Omega)\|_{H^k} \lesssim \| \Omega\|_{H^k}.
\end{align}
Moreover, $\Ps (\Omega), \Psi_2(\Omega), \hE(\Omega)$ satisfy the following elliptic estimates
\be\ba
\alpha \left\| \de_\beta \left(\frac{\Ps}{\cos \beta}\right) \right\|_{L^2}+ \alpha \|\de_{\beta \beta} \Ps\|_{L^2}+\alpha^2 \|R^2 \de_{RR}\Ps\|_{L^2} & \lesssim \|  \Omega\|_{L^2},\\
\alpha \left\| \de_\beta \left(\frac{\Psi_2}{\cos \beta}\right) \right\|_{L^2}+ \alpha \|\de_{\beta \beta} \Psi_2\|_{L^2}+\alpha^2 \|R^2 \de_{RR}\Psi_2\|_{L^2} & \lesssim \| \Omega\|_{L^2},\\
\left\| \de_\beta \left(\frac{\hE}{\cos \beta}\right) \right\|_{L^2}+ \|\de_{\beta \beta} \hE\|_{L^2}+\alpha^2 \|R^2 \de_{RR}\hE\|_{L^2} & \lesssim \|  \Omega\|_{L^2}.
\ea\ee
%
\end{Thm}
\begin{Rmk}
Note, as remarked in \cite[Remark 7.2]{tarek1}, that the mixed derivative $\alpha \|R\de_{R\beta} \Psi_2\| \lesssim \|\Omega\|_{L^2}$ is obtained by interpolation (same thing for $\Ps, \hE$). 
\end{Rmk}
Let us derive our Leading Order Model arguing as in Section \ref{sec:derivation}. From \eqref{eq:der-3d}-\eqref{eq:uru3}, as previously done to derive \eqref{eq:approx-vel}, we obtain that 
%
\be\label{eq:3dexpansions}\ba
\de_r (r u_\theta^\aa)^2 & = - \frac{\sin \beta}{R^\frac 1 \alpha} \de_\beta  (r u^\aa_\theta){^2} + \text{l.o.t}, \qquad \de_z (r u_\theta^\aa)^2  =  \frac{\cos \beta}{R^\frac 1 \alpha} \de_\beta (r u^\aa_\theta){^2} + \text{l.o.t}, \\
u_r^\aa&= {R^\frac 1 \alpha} \frac{\cL_{12}( \Omega)}{2\alpha} \cos \beta + \text{l.o.t},\qquad 
u_z^\aa= - {R^\frac 1 \alpha} \frac{\cL_{12}( \Omega)}{\alpha} \sin \beta + \text{l.o.t}, \\
\de_z u_r^\aa&=\text{l.o.t},\qquad 
\de_z u_z^\aa= - \frac{\cL_{12}( \Omega)}{\alpha}  + \text{l.o.t}, \\
\de_r u_r^\aa&=  \frac{\cL_{12}( \Omega)}{2\alpha}+ \text{l.o.t},\qquad 
\de_r u_z^\aa=  \text{l.o.t},
\ea\ee
%
and the transport term
%
\be\label{eq:transport3d}
u_r^\aa \de_r + u_z^\aa \de_z = - \frac{3 \cL_{12}( \Omega)}{2\alpha} \sin (2\beta) \de_\beta + \text{l.o.t.}= - 6 \Ps \de_\beta + \text{l.o.t.}.
\ee
%
Plugging now all the expansions \eqref{eq:3dexpansions} into \eqref{eq:3daxisymm}, keeping only the main order terms and introducing the notation
\be
\eta (\cdot, R, \beta) = \de_r (r u_\theta^\aa)^2, \qquad \xi(\cdot, R, \beta) = \de_z (r u_\theta^\aa)^2,
\ee
% auxiliary for us
% \begin{align*}
%     \de_t \Omega+\left(u_r^\aa \de_r + u_z^\aa \de_z\right)\Omega&=-\frac{2}{r^4}(ru_\theta)\xi\\
%     \de_t \eta+\left(u_r^\aa \de_r + u_z^\aa \de_z\right)\eta&=-(\de_ru_r^\aa)\eta-\de_ru_z^\aa \xi\\
%     \de_t \xi+\left(u_r^\aa \de_r + u_z^\aa \de_z\right)\xi&=-(\de_zu_r^\aa)\eta-\de_zu_z^\aa \xi\\
% \end{align*}
we derive the following Leading Order Model
\be\label{eq:LOM3d2}\ba
\de_t \wt \Omega_{\mathrm{app}} - 6 \Ps ( \wt \Omega_{\mathrm{app}}) \de_\beta { \wt \Omega_{\mathrm{app}}}&= \frac{\cL_{12}( \wt \Omega_{\mathrm{app}})}{2\alpha}  \wt \Omega_{\mathrm{app}} - {\frac{\csi }{ (1+\zeta)^3}}, \\
\de_t \et - 6 \Ps ( \wt \Omega_{\mathrm{app}}) \de_\beta \et &= - \frac{\cL_{12}( \wt \Omega_{\mathrm{app}})}{2\alpha} \et, \\
\de_t \csi - 6 \Ps ( \wt \Omega_{\mathrm{app}}) \de_\beta \csi & = \frac{\cL_{12}( \wt \Omega_{\mathrm{app}})}{\alpha} \csi.
\ea\ee
%to be compared with \eqref{LOM} where the only difference is the absence of the terms stemming from the linearization around the stratified equilibrium.
\begin{comment}
\textcolor{red}{$\csi$ dovrebbe essere nella RHS dell'eq. per $\Ome$?.} Now we change $\Ome \to - \Ome $, which yields 
\be\label{eq:LOM3d}\tag{3dLOM}\ba
\de_t \Ome + 3 \Ps (\Ome) \de_\beta \Ome &= \frac{1}{\sqrt R}  (\cos \beta)\left(\int_0^R \et (\cdot, s, \beta) \, ds \right) \et, \\
\de_t \et + 3 \Ps (\Ome) \de_\beta \et &= \frac{\cL_{12}(\Ome)}{2\alpha} \et, \\
\de_t \csi + 3 \Ps (\Ome) \de_\beta \csi & = -\frac{\cL_{12}(\Ome)}{2\alpha} (\sin \beta) \csi - \frac{\cL_{12}(\Ome)}{\alpha} \frac{(\cos^2 \beta)}{R^\frac 1\alpha} \et.
\ea\ee
\end{comment}
We will prove in the following that the second addend in the right-hand side of the equation for $\wt \Omega_\text{app}$ in \eqref{eq:LOM3d2} is negligible at main order, so that, changing $ \wt \Omega_{\mathrm{app}} \, \to \, -\Ome$, \eqref{eq:LOM3d2} rewrites as 
%
\be\label{eq:LOM3d1}\tag{LOM3d}\ba
\de_t {\Ome} + 6 \Ps (\Ome) \de_\beta {\Ome}&= - \frac{\cL_{12}(\Ome)}{2\alpha} \Ome, \\
\de_t \et + 6 \Ps (\Ome) \de_\beta \et &=  \frac{\cL_{12}(\Ome)}{2\alpha} \et, \\
\de_t \csi + 6 \Ps (\Ome) \de_\beta \csi & = - \frac{\cL_{12}(\Ome)}{\alpha} \csi.
\ea\ee
%
\begin{comment}
Note that if $\Ome$ solves the first equation of \eqref{eq:LOM3d1}, the potential vorticity \be
\frac{\Ome}{r}=\frac{\Ome}{1+\zeta}=\frac{\Ome}{R^\frac 1\alpha \cos \beta}\ee
satisfies the following equation
%
\be
\de_t \left(\frac{\Ome}{1+\zeta}\right) +6 \Ps (\Ome) \de_\beta  \left(\frac{\Ome}{1+\zeta}\right)=\frac{\cL_{12}(\Ome)}{\alpha}(2\sin^2\beta - \cos^2\beta)\left(\frac{\Ome}{1+\zeta}\right).
\ee
\end{comment}
%
Let us solve \eqref{eq:LOM3d1} for a class of initial data.
Similarly to the 2d Boussinesq equations, we choose 
\be\label{eq:ID-OA-3d}
{\Ome (0, \cdot)} = \bar g_0^{\alpha, \delta} (R) \sin (2\beta) \cos \beta,
\ee
where 
\be
\bar g_0^{\alpha, \delta}(R)\ge 0; \qquad   
\et (0, \cdot)=\bar \eta_0^{\alpha, \delta}(R)
\ee
are radial functions with compact support. As in Section \ref{sec:initialdata}, the initial datum for the full system \eqref{eq:3daxisymm} is the following :
\be\label{eq:ID-u-3D}
(r u_\theta^\aa)^2 (0, \cdot)= R^\frac 1 \alpha \bar \eta_0^{\alpha, \delta}(R)\cos \beta,
\ee
where for example $\bar \eta_0^{\alpha, \delta}(R)$ could be similar to the one given in Section \ref{sec:initialdata}. For instance, one could take 
\begin{equation}\label{eq:eta-initial}
    \bar \eta_0^{\alpha, \delta}(R)= {\delta}|\log|\log \alpha||^{\frac{1-k}{4}} \phi (|\log|\log \alpha||^{\frac{1}{4}}(R-1/8)), \qquad k \ge 4,
\end{equation}
with $\phi (R) \in C_c^\infty ([1,3])$. With this choice, denoting 
\begin{align}
    \eps:= |\log|\log \alpha||^{-\frac 1 4},
\end{align}
we take for instance $\bar g_0^{\alpha, \delta}(R)=\delta\widetilde\phi(R-1/8-\eps)$ where $\widetilde\phi$ is a smooth bump-function supported in $C_c^{\infty}([0,1/60])$ such that $\mathbf{1}_{[1/180,1/90]}\leq \widetilde\phi \leq \mathbf{1}_{[0,1/60]}$. We introduce, following \cite{chen1}, the \emph{support size} %$\supp (\Ome, \et, \csi)(t, \cdot)$ 
of $\supp\{(\omega_\theta^\aa,u_\theta^\aa)(t, \cdot)\}$ 
for $t \in [0, T^*(\alpha)]$, i.e. 
\begin{align}\label{eq:supp-eta}
    \cS(t):= \text{ess inf} \{\tilde{\rho} \, : \, \omega_\theta^\aa(t, r, z)=u_\theta^\aa(t, r, z)=0 \quad  \text{for} \quad (r-1)^2+z^2\ge \tilde{\rho}^2\}, 
    \end{align}
    where we recall that $\rho=0$ for $(r,z)=(1,0)$.
Note that as $R=\rho^\alpha$, the initial datum specified in \eqref{eq:ID-OA-3d}, \eqref{eq:ID-u-3D} satisfies
%
\be\label{eq:S0}
   (1/8+\eps)^\frac 1\alpha  \le  \cS(0) \le (1/7 + \eps)^\frac 1 \alpha. 
\ee
%
A central step of the proof of Theorem \ref{thm:main3sec3} will be to control the support size $\cS(t)$ up to the time for which $L^{\infty}$-norm inflation of  $\Omega (t, \cdot), \eta (t, \cdot), \xi (t, \cdot)$ occurs, see Section \ref{sec:supp-size}. For our choice of the initial data, see \eqref{eq:3deuler-initialU} (Case (i) of Theorem \ref{thm:main3sec3}) where $(r u^\aa_{\theta, 0})^2=R^{1/\alpha} \bar \eta_0^{\alpha, \delta} (R) \cos \beta$, we see that
\be
\de_r (r u_\theta^\aa)^2 (0, \cdot)= \bar \eta_0^{\alpha, \delta}(R)+\alpha R (\de_R \bar \eta_0^{\alpha, \delta}(R))  \cos^2 \beta, \qquad \de_z (r u_\theta^\aa)^2 (0, \cdot)= \alpha R (\de_R \bar \eta_0^{\alpha, \delta}(R)) \sin \beta \cos \beta.
\ee
Therefore, taking $\et (0, \cdot)=\bar \eta_0^{\alpha, \delta}$,
\be
\| \eta_r (0, \cdot)\|_{\cH^N}= \|\de_r (r u_\theta^\aa)^2 (0, \cdot)- \et (0, \cdot)\|_{\cH^N} \sim \alpha C_{N+1},
\ee
where $C_N=C_N(\|\bar g_0^{\alpha, \delta}\|_{\cH^N}, \|\bar \eta_0^{\alpha, \delta}\|_{\cH^N})$.
%
About $\csi(0, \cdot)$, we can either choose (as in the statement of Theorem \ref{thm:main3sec3}) $\csi(0, \cdot)=\de_z (ru_\theta^\aa)^2(0, \cdot)$, so that the initial error $\xi_r(0, \cdot)=0$, or $\csi(0, \cdot)=0$, yielding
\begin{align}
    \| \xi_r (0, \cdot)\|_{\cH^N}= \|\de_z (r u_\theta^\aa)^2 (0, \cdot)- \csi (0, \cdot)\|_{\cH^N} \lesssim \alpha C_{N+1}.
\end{align}
%
With the notation $g(t):=\Ome$, one has 
from \eqref{eq:LOM3d1} that
%
\be
\de_t g + \frac{3}{\alpha } \gamma \cL_{12}(g) \de_\gamma g=-\frac{\cL_{12}(g)}{2\alpha }   g,\qquad \gamma:=\tan \beta.
\ee
Then, setting 
\begin{equation}
    g(t)=: \gamma^{-\frac{1}{6}} \wt g (t),
\end{equation}
one has that $\wt g$ satisfies
\be\label{eq:omega-formula3d}
\de_t \wt g + \frac{3}{\alpha} \gamma \cL_{12}(\gamma^{-\frac 16} \wt g) \de_\gamma \wt g=0.
\ee
%
Now, the flow map is 
%
\begin{align}\label{eq:tildeg}
\phi_{\gamma} (t)= \gamma e^{\frac{3}{\alpha} \int_0^t \cL_{12} (\wt g(\tau) \gamma^{-1/6}) \, d\tau} \, \Rightarrow \; (\phi_{\gamma} (t))^{-1}= (\tan \beta) e^{-\frac{3}{\alpha} \int_0^t \cL_{12} (\wt g(\tau) \gamma^{-1/6}) \, d\tau}
\end{align}
%
so that the solution to \eqref{eq:omega-formula3d} with initial datum  $\wt g(0)=\bar g_0^{\alpha, \delta}(R) \sin (2\beta)\cos \beta (\tan \beta)^{\frac 16}$ such that $$\Omega_{0, \mathrm{app}}=g(0)=\bar g_0^{\alpha, \delta}(R) \sin (2\beta) \cos \beta,$$
reads
%
\begin{align}
    \wt g(t)=\bar g_0^{\alpha, \delta}(R) \frac{2\gamma^\frac 76}{(1+\gamma^2)^\frac 32}= \bar g_0^{\alpha, \delta}(R)\frac{2 (\tan \beta)^\frac 76 \left(e^{-\frac{3}{\alpha} \int_0^t \cL_{12} (\gamma^{- 1/6} \wt g(\tau)) \, d\tau}\right)^\frac 76}{\left(1+(\tan \beta)^2 e^{-\frac{6}{\alpha} \int_0^t \cL_{12} (\gamma^{- 1/6} \wt g(\tau)) \, d\tau}\right)^\frac 32}.
\end{align}
%
We then deduce that
\begin{align}\label{eq: g explicit}
    g(t) & = (\tan \beta)^{-1/6}  \wt g(t) =\bar g_0^{\alpha, \delta}(R)\frac{2 (\tan \beta) \left(e^{-\frac{7}{2\alpha} \int_0^t \cL_{12} (g(\tau)) \, d\tau}\right)}{\left(1+(\tan \beta)^2 e^{-\frac{6}{\alpha} \int_0^t \cL_{12} ( g(\tau)) \, d\tau}\right)^\frac 32}.
\end{align}
%
Such expression of $g(t)$ is completely analogous to \eqref{eq:g-formula}, and the respective upper and lower bounds can be obtained as the ones of Lemma \ref{prop:LOM} . Once again, we observe that the sign of $\cL_{12}(g(t))$ is determined by the initial datum $\bar g_0^{\alpha, \delta}(R)$. In particular $\|g(t)\|_{L^\infty} \lesssim \delta$ if $\bar g_0^{\alpha, \delta}(R)\ge 0$ (Case (i) of Theorem \ref{thm:main3sec3}), while it can display strong norm inflation if $\bar g_0^{\alpha, \delta}(R)\le 0$ (Case (ii) of Theorem \ref{thm:main3sec3}).
As done in Section \ref{sec:derivation}, we can now solve \eqref{eq:LOM3d1} for $\et$, yielding
%
\be\label{eq:eta-sol3d}\et(t, R, \beta)= \bar \eta_0^{\alpha, \delta} (R) \exp\left(  \frac{3}{\alpha} \int_0^t {\cL_{12}\left({\Ome (\tau)}\right)}\, d\tau\right)=\bar \eta_0^{\alpha, \delta} (R) \exp\left(  \frac{3}{\alpha} \int_0^t {\cL_{12}\left({ g (\tau)}\right)}\, d\tau\right).\ee
%
We adapt Lemma \ref{prop:LOM} and apply it to see that, choosing $\bar g_0^{\alpha, \delta}(R)\ge 0$, the inequalities \eqref{eq:Lg-bounds}-\eqref{eq:g-upperandlower} hold, so that
%
\begin{equation}\label{eq:bound L12}
\frac{1}{\alpha} \int_0^t \cL_{12}(g (\tau)) \, d\tau \ge \frac{1}{ c_2} \log \left( 1+ \frac{c_2  t}{2\alpha} \int_R^\infty \frac{ \bar g_0^{\alpha, \delta}(s)}{s} \, ds  \right),
\end{equation}
for some constant $c_2>0$, independent of $\alpha$.
 %Note that the second term in the expression of $\Ome$ in \eqref{eq:omega-formula3d} is exactly the solution of the fundamental model in \cite[page 669]{tarek1}. 
%Finally, the explicit formula for $\csi (t, R, \beta)$ is exactly given by \eqref{eq:csi-explicit}.
We thus obtain the analogous of Proposition \ref{prop:expl} for \eqref{eq:LOM3d1}.
\begin{Prop}\label{prop:expl3d}
There exist $0<\alpha_0 \ll 1$ and $\alpha_0 \ll \delta \ll 1$ such that, for any $0<\alpha \le \alpha_0 $, there exist initial data $(\Omega_{0, \mathrm{app}}^{\alpha, \delta}(R, \beta), \xi_{0, \mathrm{app}}^{\alpha, \delta}(R, \beta), \eta_{0, \mathrm{app}}^{\alpha, \delta}(R, \beta))$ with
$$\|(\Omega_{0, \mathrm{app}}^{\alpha, \delta}, \eta_{0, \mathrm{app}}^{\alpha, \delta}, \xi_{0, \mathrm{app}}^{\alpha, \delta})\|_{L^\infty(\R^2)} \sim  \delta,$$
such that either (i) or (ii) of the following are satisfied
\begin{enumerate}
\item[(i)] if the initial datum has the following form  \begin{align}
 \Omega_{0,\mathrm{app}}^{\alpha, \delta} (R, \beta) = \bar g_0^{\alpha, \delta} (R) \sin (2\beta)\cos \beta, \qquad \eta_{0, \mathrm{app}}^{\alpha, \delta} (R, \beta)= \bar \eta_0^{\alpha, \delta}(R), \qquad  \xi^{\alpha, \delta}_{0, \mathrm{app}}(R, \beta)=\frac{\alpha}{2} R  (\de_R\bar \eta_0^{\alpha, \delta}(R)) \sin (2\beta),
\end{align}
where $(\bar g_{0}^{\alpha, \delta} (R), \bar \eta_{0}^{\alpha, \delta} (R) ) \in C_c^\infty([1, \infty))$
and $\bar g_0^{\alpha, \delta} (R)\ge 0$, then the  solution $(\Ome^{\alpha, \delta}(t), \eta_\mathrm{app}^{\alpha, \delta}(t), \xi_\mathrm{app}^{\alpha, \delta}(t))$ to the Cauchy problem associated with \eqref{eq:LOM3d1} satisfies
%
\be\ba
\|\et^{\alpha, \delta}(t) \|_{L^\infty(\R^2)}  &\ge \|\eta_{0, \mathrm{app}}^{\alpha, \delta}\|_{L^\infty (\R^2)} \left(1+\frac{c_2 t}{2\alpha }C_0\right)^{\frac{1}{c_2}},
\ea\ee
where $C_0=\sup\limits_{R \in \mathrm{supp}(\bar g_0^{\alpha, \delta}(R))} \int_R^\infty \frac{\bar g_0^{\alpha, \delta}(s)}{s} \, ds$ and $c_2>0$ is independent of $\alpha$. In particular, this yields that 
\be\label{eq:etablow-up}\ba
\sup\limits_{t \in [0, T^*(\alpha)]} \|\et^{\alpha, \delta}(t) \|_{L^\infty}  &\ge  \|\eta_{0, \mathrm{app}}^{\alpha, \delta}\|_{L^\infty} \left(1+\frac{\log |\log \alpha|}{C_{N+1}}\right)^{\frac{1}{c_2}};
\ea\ee
\item[(ii)] if the initial datum has the following form \begin{align}
 \Omega_{0,\mathrm{app}}^{\alpha, \delta} (R, \beta) = \bar g_0^{\alpha, \delta} (R) \sin (2\beta)\sin \beta, \qquad \eta_{0, \mathrm{app}}^{\alpha, \delta} (R, \beta)= \frac{\alpha}{2} R  (\de_R\bar \eta_0^{\alpha, \delta}(R)) \sin (2\beta), \qquad  \xi^{\alpha, \delta}_{0, \mathrm{app}}(R, \beta)=\bar \eta_0^{\alpha, \delta}(R),
\end{align}
where $(\bar g_{0}^{\alpha, \delta} (R), \bar \eta_{0}^{\alpha, \delta} (R) ) \in C_c^\infty([1, \infty))$
and 
$\bar g_0^{\alpha, \delta} (R)\le 0$, then 
\be\label{eq:omegablow-up}\ba
\sup\limits_{t \in [0, T^*(\alpha)]} \|\Ome^{\alpha, \delta}(t) \|_{L^\infty}  &\ge  \|\Omega_{0, \mathrm{app}}^{\alpha, \delta}\|_{L^\infty} \left(1+\frac{\log |\log \alpha|}{C_{N+1}}\right)^{\frac{1}{c_2}},
\ea\ee
\end{enumerate}
for $T^*(\alpha)\sim  \frac{\alpha}{C_{N+1}}\log |\log (\alpha)| \, \to \, 0$ as $\alpha \to 0$ and such that $\frac{\log |\log \alpha|}{C_{N+1}} \gtrsim \log |\log \alpha|^\mu$ for some $0<\mu <1$. 
\end{Prop}
%
\subsubsection{Further estimates on the solutions to the \eqref{eq:LOM3d1}}
Applying Lemma \ref{prop:LOM} to equation \eqref{eq: g explicit} yields
%
\begin{itemize}
    \item Case (i): \eqref{eq:etablow-up} holds, while ${\Ome}$ does not blow up
%
\be
\left\|{\Ome}\right\|_{L^\infty} \lesssim \|\bar g_0^{\alpha, \delta}\|_{L^\infty} \sim \delta, \qquad  \left
\|{\Ome} \right\|_{\cH^N} \lesssim C_N;
\ee
%
\item Case (ii): \eqref{eq:omegablow-up} holds and
%
\be\label{eq:xiblow-up}\ba
\sup\limits_{t \in [0, T^*(\alpha)]} \|\csi^{\alpha, \delta}(t) \|_{L^\infty}  &\ge  \|\xi_{0, \mathrm{app}}^{\alpha, \delta}\|_{L^\infty} \left(1+\frac{\log |\log \alpha|}{C_{N+1}}\right)^{\frac{1}{c_2}},
\ea\ee
%
while $\et$ does not blow up
%
\be
\left\|{\et}\right\|_{L^\infty} \lesssim \|\bar \eta_0^{\alpha, \delta}\|_{L^\infty} \sim \delta.
\ee
%
\end{itemize}
Notice that in Case (i) and Case (ii) of Proposition \ref{prop:expl3d}, the roles of $\et^{\alpha, \delta}, \csi^{\alpha, \delta}$ are simply switched, therefore most of the proofs will only treat Case (i).  
In Case (i), exactly as in Proposition \ref{lem:35}, one gets that
\begin{align}\label{eq:est-app}
   \|\Ome (t) \|_{\cH^k} \lesssim C_k  e^{\frac{C_k}{\alpha} t}, \qquad \|\et (t) \|_{\cH^k} \lesssim C_k  e^{\frac{C_k}{\alpha} t}, \qquad  \|\csi (t) \|_{\cH^k} \lesssim \alpha C_{k+1} e^{\frac{C_k}{\alpha} t}.
\end{align}


%%%%%

\begin{Rmk}[Comparison with the fundamental model in \cite{tarek1}]
    The fundamental model for the 3d axisymmetric Euler equations without swirl in \cite[page 668]{tarek1} reads
    \be
    \de_t \Ome = \frac{\cL_{12}(\Ome)}{2\alpha}\Ome,
    \ee
    which is solved by
\be\Ome=\frac{\Omega_{0, \mathrm{app}}^{\alpha, \delta}}{\left(1- \frac{t}{2\alpha}\cL_{12}(\Omega_0^{\alpha, \delta})\right)^2}.
    \ee
    Passing from \eqref{eq:LOM3d2} to \eqref{eq:LOM3d1}, we performed the change of variable $\Ome \, \to \, -\Ome$, so that, applying it again, the above equation is
     \be
    \de_t \Ome = -\frac{\cL_{12}(\Ome)}{2\alpha}\Ome,
    \ee
    whose solution reads 
    \be
    \Ome=\frac{\Omega_{0, \mathrm{app}}^{\alpha, \delta}}{\left(1+ \frac{t}{2\alpha}\cL_{12}(\Omega_{0, \mathrm{app}}^{\alpha, \delta})\right)^2},
    \ee
    with $\Omega_{0, \mathrm{app}}^{\alpha, \delta}= R^{-\frac 1 \alpha} \bar g_0^{\alpha, \delta}(R) \sin (2\beta) \cos \beta$, such that
    %
    \begin{itemize}
        \item if $\bar g_0^{\alpha, \delta}(R)\ge 0$, then $\cL_{12}(\Omega_{0, \mathrm{app}}^{\alpha, \delta})=\frac{3}{16} \int_R^\infty \bar g_0^{\alpha, \delta}(s)(s) \, ds >0$ and $\Ome$ does not blow up;
        \item if $\bar g_0^{\alpha, \delta}(R)\le 0$, then $\cL_{12}(\Omega_{0, \mathrm{app}}^{\alpha, \delta})<0$, then $\Ome (t)$ blows up at algebraic rate $x^{-2}$ for $x \sim 0$ for times $t \sim \alpha$.
    \end{itemize}
    %
    From inequality \eqref{eq:omegablow-upthm}, we see that the inflation rate of the true solution $\omega_\theta^\aa (t) $ is actually $\log |\log x|$ as $x \sim 0$ for a longer time-scale $t \sim \alpha \log |\log \alpha|$.
\end{Rmk}
%
\subsection{Localized elliptic estimates}\label{sec:supp-size}
We aim to develop elliptic estimates for the stream function $\Psi$ as in \eqref{eq:psibig} which are localized in space to avoid the singularity occurring in $r=0$.
For that purpose, we introduce
\begin{equation}\label{eq:T'}
    T'=\sup\left\{ T\in [0,T^*(\alpha)] \,\, : (1/10)^\frac{1}{\alpha}\leq \cS(t)\leq (1/6)^\frac{1}{\alpha} \, \text{for all} \, t\in [0,T]\right\}, \quad \text{with} \quad T^*(\alpha) \; \text{in} \, \eqref{def:talpha-star},
\end{equation}
and prove the required estimates for all $t\in [0,T']$. Note that in view of the local well-posedness of \eqref{eq:3daxisymm}-\eqref{eq:elliptic3d} one has $T'>0$ by continuity.
First, we provide an estimate for $\Psi$ away from the support of $\omega_\theta^\aa$ the proof of which follows the same lines as \cite{chen1}[Lemma 9.4].
\begin{Lem}\label{lem:outside}
    Let $t\in [0,T']$, $\psi^\aa$ be the solution to \eqref{eq:elliptic3d}-\eqref{eq:3daxiboundarycond} and $\cS(t)$ be defined in \eqref{eq:supp-eta}. For any $(2\cS(t))^\alpha<\lambda<\frac{1}{2}$    
%    $\lambda> (2 \cS(t))^\alpha$
, recalling from \eqref{eq:psibig} that $\Psi(\cdot, R, \beta)=\rho^{-2}\psi^\aa (\cdot, r, z)$, it holds that 
    \begin{align}
        \|\Psi \mathbf{1}_{\lambda \le R \le 2\lambda} \|_{L^2} \lesssim (1+|\log (\lambda^\frac 1 \alpha)|) \cS(t) \lambda^{-\frac 1\alpha}\|\Omega\|_{L^2}.
    \end{align}
\end{Lem}
The $L^2$-estimate is proven along the same lines as the respective version in \cite[Lemma 9.4]{chen1}. However, the proof given here is simplified as opposite to \cite{chen1} no dynamic scaling is used and the elliptic problem is posed on the whole space for which an explicit Biot-Savart formula is known, e.g. \cite{marchioro}.

\begin{proof}
    Exploiting the explicit representation formula for the Biot-Savart law of the axi-symmetric Euler eq., namely for \eqref{eq:elliptic3d}, see \cite[Section 2]{marchioro}, one has that 
    \begin{equation*}
        \left|\psi^\aa(r,z)\right|\lesssim  \left|\int_{-\pi}^\pi \int_0^{\infty} \int_{-\infty}^{\infty}\frac{\cos\gamma}{4\pi\sqrt{\frac{(z-z')^2+(r-r')^2}{rr'}+2(1-\cos\gamma)}}\omega_\theta^\aa (r',z')\sqrt{\frac{r'}{r}}d\gamma dr'dz' \right|.
    \end{equation*}
    Taking into account that $(r,z)$ is chosen away from $\mathrm{supp}(\omega_\theta^\aa)$, integrating in $\gamma$ and using the estimate \cite[(2.27)-(2.28)]{marchioro}, one concludes that
    \begin{equation*}
        \left|\psi^\aa(r,z)\right|\lesssim  \int_0^{\infty}\int_{-\infty}^{\infty} |\omega_\theta^{\aa}(r',z')|\left(1+\left|\log\left(\frac{(r-r')^2+(z-z')^2}{rr'}\right)\right|\right)\sqrt{\frac{r'}{r}}dr'dz'.
    \end{equation*}
Note that in this setting, there exist two positive $r_0>0, r_0'>0$ such that $r>r_0, r'> r_0'$. We conclude that
\begin{equation*}
    \left|\Psi(R,\beta)\right|\lesssim R^{-\frac{2}{\alpha}}\left(1+\left|\log(R^{\frac{2}{\alpha}})\right|\right)\int |\omega_\theta^\aa(r',z')|\sqrt{\frac{r'}{r}}dr' dz'\lesssim R^{-\frac{2}{\alpha}}\left(1+\left|\log(R^{\frac{2}{\alpha}})\right|\right)\frac{S(t)^{2-\frac{2}{\alpha}}}{\sqrt{\alpha}}\|\Omega\|_{L^2}
\end{equation*}
provided  $\lambda\leq R \leq 2\lambda$. The estimate then follows by computing the localized $L^2$-norm.
\end{proof}

\subsubsection{Localized elliptic estimates}
Here and below we use the notation $D_R:=R\de_R$. Given $\chi(R) \in C_c^\infty([0,\infty)) $ such that 
\begin{equation*}
    \mathbf{1}_{[0,1]}(R)\leq \chi(R)\leq \mathbf{1}_{[0,2)}(R), \qquad D_R\chi(R)\lesssim \chi(R)
\end{equation*}
for all $R\geq 0$, let us introduce the cut-off function $\chi_\lambda (R):=\chi(R/\lambda)$, where $\lambda>0$ is a parameter to be chosen. 
We first develop some localized elliptic estimates that will be used to estimate the support size of our unknowns.

%
\begin{Prop}\label{lem:localized-ell}
Suppose that $t\in [0,T']$ with $T'$ in \eqref{eq:T'}, then there exists $(2\cS(t))^{\alpha}<\lambda<1/2$ such that the analogous decomposition of Theorem \ref{thm:tarek3d} holds, i.e.
%
\begin{align}
    \Psi_\lambda= \Psa^\lambda+\hE^\lambda, \quad \Psa^\lambda=\Psa(\Omega \chi_\lambda+Z_2)=\Ps(\Omega)+\widetilde{\mathcal{R}}^\alpha,
\end{align}
%
where $\Psa(\Omega), \Ps(\Omega)$ are given in \eqref{eq:psi23d} and \begin{align}
Z_2:=2\alpha^2 R^2 \frac{\chi'_\lambda}{\lambda}\de_R \Psi+
 \alpha^2R^2\frac{\chi_\lambda''}{\lambda^2}+\alpha(\alpha+5)R \frac{\chi_\lambda'}{\lambda}\Psi.
\end{align}
For $k \ge 3$, the following estimates hold:
\begin{align}\label{eq:first-loc}
    \alpha^2\|R \de_R \Psi_\lambda\|_{L^2}^2+\alpha\|\Psi_\lambda\|_{L^2}^2 + \alpha\|\de_\beta \Psi_\lambda\|_{L^2}^2 &\lesssim \alpha^{-1}\|\Omega\|_{L^2}^2, \\
    \alpha^2\|R \de_{RR} \Psi_\lambda\|_{\cH^k(\rho_i)} + \alpha \|R \de_{R\beta}\Psi_\lambda\|_{\cH^k(\rho_i)} + \|\de_{\beta \beta}(\Psi_\lambda-\Ps (\Omega))\|_{\cH^k(\rho_i)} & \lesssim \|\Omega\|_{\cH^k(\rho_i)}.
\end{align}
\end{Prop}
%
\begin{proof}
Note that with the bound of $\cS(t)$ provided by \eqref{eq:T'} and for $(2S(t))^{\alpha}<\lambda<1/2$ one has that 
$\lambda^{-\frac{1}{\alpha}}\cS(t)<  \frac{S(t)}{2\cS(t)}=\frac{1}{2}$. However, choosing for instance $\lambda=\frac{4}{9}$ yields 
$\lambda^{-\frac 1 \alpha} \cS(t) \lesssim \alpha.$
%
The localized stream function
$\Psi_\lambda$ satisfies
%
\begin{align*}
\alpha^2 R^2 \de_{RR} \Psi_\lambda + \alpha (\alpha + 5) R \de_R \Psi_\lambda + \de_{\beta \beta} \Psi_\lambda - \de_\beta ((\tan \beta) \Psi_\lambda)+6 \Psi_\lambda&=-\Omega \chi_\lambda+Z_2.
\end{align*}
%
Estimating the left-hand side follows the same lines of the proof of \cite{tarek1}[Proposition 7.8], multiplying by $\Psi_\lambda$ yields
%
\begin{align*}
    \alpha^2 \|R \de_R \Psi_\lambda\|_{L^2}^2  + \frac{5\alpha -\alpha^2}{2}\|\Psi_\lambda\|_{L^2}^2+\|\de_\beta \Psi_\lambda\|_{L^2}^2 + \frac{1}{2} \|(\sec \beta )\Psi_\lambda\|_{L^2}^2 - 6 \|\Psi_\lambda\|_{L^2}^2
     \lesssim | \langle Z_2, \Psi_\lambda \rangle|+ \|\Omega \chi_\lambda\|_{L^2} \|\Psi_\lambda\|_{L^2}.
\end{align*}
%
Following the proof \cite{chen1}[Lemma 9.6], we integrate by parts the first term of the right-hand side and by the property of the cut-off $(R \de_R \chi_\lambda)^2\lesssim \chi_\lambda, \, |D_R^k \chi_\lambda| \lesssim \mathbf{1}_{\lambda \le R \le 2\lambda}$, noticing that $|\log (\lambda^\frac 1 \alpha)| \lesssim \alpha^{-1}$ gives
\begin{align*}
    \langle Z_2, \Psi_\lambda \rangle \lesssim \alpha  \|\Psi \mathbf{1}_{\lambda \le R \le 2 \lambda}\|_{L^2}^2 \lesssim \alpha^{-1}  \|\Omega \|_{L^2}^2,
\end{align*}
where we used Lemma \ref{lem:outside}.
Using the Fourier-series expansion $\{\sin(2n \beta)\}_{n \ge 1}$, note that for $n=1$, namely $\Psi_{2, \lambda}=\hat{\Psi}_2 \sin (2\beta),$ the term $\de_{\beta \beta}\Psi_{2, \lambda}-\de_\beta ((\tan \beta) \Psi_{2, \lambda}) + 6 \Psi_{\lambda, 2}=0$. In fact, the desired estimate for $\Psi_{2, \lambda}$ is obtained using its explicit formula.  
For the modes $n \ge 2$ it holds that
\begin{align*}
    \|\de_\beta \Psi_\lambda\|_{L^2}^2 \ge 8 \|\Psi_\lambda\|_{L^2}^2. 
\end{align*}
%
One can then rearrange some of the terms in the right-hand side of the above inequality, yielding
\begin{align*}
    -(6-\alpha) \|\Psi_\lambda\|_{L^2}^2 + \|\de_\beta \Psi_\lambda\|_{L^2}^2 \ge  \left(-\frac{(6-\alpha)}{8}+1\right) \|\de_\beta \Psi_\lambda\|_{L^2}^2\ge \frac 1 4 \|\de_\beta \Psi_\lambda\|_{L^2}^2.
\end{align*}
%
Finally, using that $\|\de_\beta \Psi_\lambda\|_{L^2} \ge 2 \|\Psi_\lambda\|_{L^2}$ and the Young  inequality yields 
\begin{align*}
\|\Omega\chi_\lambda\|_{L^2}\|\Psi_\lambda\|_{L^2} \le \frac 12 \|\Omega\|_{L^2} \|\de_\beta \Psi_\lambda\|_{L^2} \le \frac{1}{4\alpha}\|\Omega\|^2_{L^2}+\frac{\alpha}{4}\|\de_\beta \Psi_\lambda\|_{L^2}^2,
\end{align*}
so that the latter is absorbed by the left-hand side.
This concludes the proof of the first estimate. \\
Let us deal with the second estimate, whose proof, which is an adaptation of \cite{chen1}[Proposition 9.9], follows by a weighted $L^2(\rho_i)$ estimate where the weights $\rho_i$ are defined in \eqref{def:weights3d}. More precisely, the estimate
%
\begin{align*}
   \alpha^2\|R^2 \de_{RR}\Psi_\lambda\|_{L^2(\rho_i)} + \alpha \|R \de_{R\beta}\Psi_\lambda\|_{L^2(\rho_i)} + \|\de_{\beta \beta}(\Psi_\lambda-\Ps (\Omega+Z_2))\|_{L^2(\rho_i)} & \lesssim \|\Omega \chi_\lambda+Z_2\|_{L^2(\rho_i)}
\end{align*}
%
is given by the proof of \cite{chen1}[Proposition 9.9]. We only have to show that
%
\begin{align*}
    \|\Ps(Z_2)\|_{L^2(\rho_i)} \lesssim \|\Omega\|_{L^2(\rho_i)}, \quad \|Z_2\|_{L^2(\rho_i)} \lesssim \|\Omega\|_{L^2(\rho_i)}.
\end{align*}
%
Consider the last addend of $Z_2$, for which one has
\begin{align*}
    \frac \alpha \lambda (\alpha +5)\| R \chi_\lambda' \Psi \rho_1^{1/2}\|_{L^2} &\lesssim\frac \alpha \lambda (\alpha +5)\|R \chi_\lambda' (\sin 2\beta)^{-\frac{99}{100}}\Psi\|_{L^2} \lesssim \frac \alpha \lambda (\alpha +5)\|R \chi_\lambda' \de_\beta \Psi\|_{L^2}\\& = \alpha (\alpha+5) \|D_R \chi_\lambda \de_\beta \Psi\|_{L^2} \lesssim  \alpha (\alpha+5)\alpha^{-1} \|\Omega\|_{L^2} \lesssim \|\Omega\|_{L^2(\rho_1)},
\end{align*}
where we used that $|D_R \chi_\lambda| \lesssim \mathbf{1}_{\lambda/2 \le R \le \lambda}$ and the estimate of Lemma \ref{lem:outside}. Estimating the other addends of $Z_2$ is analogous.  \\
%
Now, by Lemma \ref{lem:outside} and owing to  $\lambda^{-\frac 1 \alpha} \cS(t) \lesssim \alpha$ (which is stronger than $\lambda> (2\cS(t))^\alpha$ in Lemma \ref{lem:outside}), we have 
%
\begin{align}\label{eq:est-fuori}
    \|\Psi\mathbf{1}_{\lambda \le R \le 2\lambda}\|_{L^2} \lesssim \|\Omega\|_{L^2}.
\end{align}
%
From that and from the presence of the factor $\alpha$ in front of all the terms of $Z_2$, the estimate $\|\Ps(Z_2)\|_{L^2(\rho_i)} \lesssim \|\Omega\|_{L^2(\rho_i)}$ follows as well. To obtain weighted and localized estimates for the derivatives, the rest of the proof of \cite[Proposition 9.9]{chen2022} is based on a finite induction where at the step $j$, one considers $j$-th derivatives and $\lambda=\lambda_j$ decreasing in $j \in \{0, \cdots, J\}$ with $J \in \N$. Here we choose $\lambda_j$ so that $\lambda_0=\frac{4}{9}+2^{-5}$ and $\lambda_J=\frac{4}{9}+2^{-5-J}$.
The rest is identical to \cite[Proposition 9.9]{chen2022} and therefore it is omitted.
%
\end{proof}
%

\subsection{Remainder estimates %\ref{thm:main3sec3}.
 }\label{sec:remainder3d}
The only substantial difference between \eqref{eq:leading1} for the 2d Boussinesq equations and \eqref{eq:LOM3d1} for the 3d axisymmetric Euler equations is the equation for $\Ome$. More precisely, for the remainders due to the transport term, namely $\mathcal{T}f(R, \beta)$ in \eqref{eq:transport terms}, one can prove the analogous of Lemma \ref{lem:remT} with $\cH^N$ in \eqref{def:weights3d} and taking into account the localization of Proposition \ref{lem:localized-ell}. In particular, for $\Psi_\lambda = \Psi \chi_\lambda$ as in Proposition \ref{lem:localized-ell}, one notices that
\begin{align*}
    \Psi=\Psi_\lambda+\Psi \mathbf{1}_{R \ge 2\lambda}. 
\end{align*}
The first addend is estimated by Proposition \ref{lem:localized-ell}, while the latter is a lower-order term, with similar estimates to (actually better than) \eqref{eq:est-fuori}.
Therefore, to prove the analogous of Proposition \ref{prop:rem}, we only need to control the error due to the second addend in the right-hand side of the equation for $\Omega_\text{app}$ in \eqref{eq:LOM3d1}, namely
%
\be\ba\label{eq:resto3d}
\frac{\xi}{r^3}=\frac{\xi \chi_\lambda + \xi  (1-\chi_\lambda)}{(1+R^\frac 1\alpha \cos \beta)^3 }.
\ea\ee
%
\begin{Lem}\label{lem:1rem3d} There exists $\alpha_0 \ll 1$ small enough such that, if $\alpha \le \alpha_0$, for all $t\in [0,T']$ with $T'$ defined in \eqref{eq:T'}, taking $N \ge 4$ it holds that
\be
\left\langle \frac{\xi}{r^3}, \Omega_r \right \rangle_{\cH^N}
\le \left(C_N e^{\frac{C_N}{\alpha}t}+\|\xi_r\|_{\cH^N}\right) \|\Omega_r\|_{\cH^N}.
\ee
More precisely, there exists $\lambda>0$ such that 
\be
\left\langle \frac{\xi \chi_\lambda}{r^3}, \Omega_r \right \rangle_{\cH^N}
\le \left(C_N e^{\frac{C_N}{\alpha}t}+\|\xi_r \chi_\lambda\|_{\cH^N}\right) \|\Omega_r\|_{\cH^N}.
\ee
\end{Lem}
\begin{proof}
The considered remainder term is divided into two pieces as in \eqref{eq:resto3d}. We note that for $t\in [0,T']$ it holds that $ \supp(\xi)\subset \{(R,\beta) : \frac{1}{10}\leq R \leq \frac16\}$ by definition. Hence, for $\lambda=\frac49$ yields that 
\begin{equation*}
    (\xi(1-\chi_\lambda))(R)=0
\end{equation*}
for all $R\geq 0$. On the other hand, there exists $\alpha_0>0$ small enough such that, for all $0<\alpha\le \alpha_0$,
    \begin{align*}
        \frac{\chi_\lambda(R)}{(1+R^\frac{1}{\alpha}\cos\beta)^3} \le 2.
    \end{align*}
The desired $L^2(\rho_i)$ estimate is then a consequence of Cauchy-Schwarz inequality and the bound of $\|\csi\|_{\cH^k}$ in \eqref{eq:est-app}. To bound the $\cH^N$-norm, it is sufficient to observe that the most singular term is the one involving derivatives of $(1+R^\frac{1}{\alpha}\cos\beta)^{-3}$ which can be bounded by exploiting
\begin{equation*}
    \alpha^{-N} R^{\frac{1}{\alpha}}\lesssim 1,
\end{equation*}
for $\frac{1}{10}<R<\frac{1}{6}$ and $\alpha$ small enough.

% the most singular term is the one involving
% \begin{equation*}
%     \left\|\chi_{\lambda}(R)\xi\de_R^N\left(1+R^{\frac{1}{\alpha}}\cos(\beta)\right)^{-3}\right\|_{L^2}\lesssim \|\xi\chi_{\lambda}(R)\|_{L^{\infty}}\left(\int_{?}\frac{1}{\alpha^{2N}}R^{\frac{2N}{\alpha}-2}d R\right)^{\frac{1}{2}}
% \end{equation*}
\end{proof}
%
Note that the choice of the regularity index $N\ge 4$ is due to the fact that in $\cH^N, \, N\ge 4$ the remainder estimates below are simpler as the embedding theorems are better, as observed in \cite[Section 8]{tarek1}.
Arguing as for the proof of Proposition \ref{prop:rem} and relying on the Sobolev-embeddings for $\cH^k$ proven in \cite[Corollary 8.3]{tarek1}, we conclude that the remainders are small in $\cH^N$-norm up to time $T'$. 
\begin{Cor}\label{cor:rem3D}
 Let $N\geq 4$, then 
    \begin{equation*}
        F(t)=\|\Omega_{r}(t)\|_{\cH^N}+\|\eta_{r}(t)\|_{\cH^N}+\|\xi_{r}(t)\|_{\cH^N}.
    \end{equation*}
    Then 
    \begin{equation*}
        F(t)\lesssim \sqrt{\alpha}
        %C_{N+1}\sqrt{\alpha},
    \end{equation*}
    for all $0\leq t\leq T'$ with $T'$ defined in \eqref{eq:T'} and where $C_{N+1}$ only depends on the $\cH^{N+1}$-norm of the initial data $(\Omega_0,\eta_0,\xi_0)$. In particular,
    \begin{equation}%\label{eq:est-rim-Linfty}
        \|\Omega_r(t)\|_{L^{\infty}}+ \|\eta_r(t)\|_{L^{\infty}}+ \|\xi_r(t)\|_{L^{\infty}}\lesssim \sqrt{\alpha}
        %\leq C_{N+1}\sqrt{\alpha}
    \end{equation}
    for all $0\leq t\leq T'$.
    % with $T=\frac{\alpha}{4C_{N+1}}\log|\log\alpha|$.
\end{Cor}

\subsection{Support size}\label{sec:support}
The objective of this section is to show that the support of $\omega_\theta^\aa, (ru_\theta^\aa)^2$ remains sufficiently localized and especially away from the symmetric axis up to the time for which norm-inflation of \eqref{eq:LOM3d1} occurs. More precisely, we show in particular that $T'\geq T^*(\alpha)$ with $T'$ and $T^*(\alpha)$ defined in \eqref{def:talpha-star} and \eqref{eq:T'} respectively.
We exploit that the transport type equation with inelastic constraint
\begin{equation*}
    \de_t f+ u\cdot\nabla_{r,z}f=0, \qquad \Div_{r,z}(r u)=0, 
\end{equation*}
satisfied by $(ru_\theta^\aa)^2$, see \eqref{eq:utheta2},
can equivalently be described through the unique associate flow
\begin{equation}\label{eq:flow}
    \dot{X}(t,x_0)=u(t,X(t,x_0)), \qquad X(0)=x_0.
\end{equation}
This is well-known in the cases of (nearly) incompressible vector fields $u$, see e.g. \cite{bertozzi}. Note that $\Div(u)=-r^{-1}\nabla_{r,z}r\cdot u$ is in general not bounded. Existence and uniqueness of the associated flow are proven in \cite[Proposition 2.2]{hientzsch} for the lake equations which represents a generalization of the axi-symmetric Euler eq. (without swirl). 

%
\begin{Lem}\label{lem.local-supp}
    Suppose, as in \eqref{eq:S0}, that $$(1/8+\eps)^\frac 1 \alpha \le \cS(0) \le (1/7+\eps )^\frac 1 \alpha.$$ Then, for all $t \in [0, T^*(\alpha)]$ it holds that 
    $$ (1/8)^\frac 1 \alpha \le \cS(t) \le (1/7+2\eps )^\frac 1 \alpha.$$
    In particular, it holds that $T'=T^*(\alpha)$ for $T'$ defined in \eqref{eq:T'} and there exists $\lambda>0$ such that
    \begin{align}
        \lambda^{-\frac 1 \alpha} \cS(t) \lesssim \alpha \qquad \text{for all} \,\, t \in [0, T^*(\alpha)],
    \end{align}
    and for $\alpha \le \alpha_0 \ll 1 $ small enough.
\end{Lem}
%
As an immediate consequence, we note that the assumptions of Proposition \ref{lem:localized-ell} are satisfied until time $T^*(\alpha)$, so that the remainder terms can be controlled within that time interval.

\begin{proof}
We represent the flow defined in \eqref{eq:flow} in terms of the polar coordinates $X(t)=(R(t),\beta(t))^t$. Consider the equation for $(r u_\theta^\aa)^2$ in \eqref{eq:utheta2}. From \eqref{eq:der-3d}-\eqref{eq:uru3}, writing more carefully the transport term \eqref{eq:transport3d}, one obtains 
\begin{align}\label{eq:transport-polar}
    u_r \de_r + u_z \de_z&= (-\alpha R \de_\beta \Psi)\de_R +(2\Psi+\alpha R \de_R \Psi)\de_\beta.
\end{align}
We want to rely on Lemma \ref{lem:localized-ell} by means of a bootstrap argument. To this end, we work in the time interval $[0,T']$, where $T'$ is defined in \eqref{eq:T'}. Within such times, 
one has in particular that $\lambda^{-\frac 1 \alpha} \cS(t)\lesssim \alpha $ for some $\lambda > 0$ and $\alpha \le \alpha_0 \ll 1 $ small enough. From  \eqref{eq:transport-polar} (one can compare with \cite[(9.66) p. 64]{chen2022}), it follows that
\begin{align*}
    \dot R(t) &= -\alpha R \de_\beta \Psi (\Omega)(R(t), \beta(t)). 
\end{align*}
Then, we can apply Theorem \ref{thm:tarek3d}, yielding
\begin{align}
    \dot R(t) &= -\alpha R \de_\beta \Psi =-R\left(\frac{\cL_{12}(\Omega)}{2}+2\alpha \mathcal{R}^\alpha (\Omega)\right) \cos (2\beta)-\alpha R\de_\beta \hE(\Omega)\notag\\
    &=-R\left(\frac{\cL_{12}(\Ome)}{2}+2\alpha \mathcal{R}^\alpha (\Ome)\right) \cos (2\beta) -R\left(\frac{\cL_{12}(\Omega_r)}{2}+2\alpha \mathcal{R}^\alpha (\Omega_r)\right) \cos (2\beta)\\
    &\quad - \alpha R  \de_\beta \hE(\Ome)- \alpha R  \de_\beta \hE(\Omega_r).
\end{align}
%
This gives 
\begin{align}
    \log \left(\frac{R(t)}{R(0)}\right) &=- \int_0^t  \frac{\cL_{12}(\Ome(\tau))}{2} \cos(2\beta) \, d\tau \\&\quad - \int_0^t  \frac{\cL_{12}(\Omega_r(\tau))}{2} \cos(2\beta) \, d\tau - \alpha \int_0^t (2\mathcal{R}(\Ome+\Omega_r)\cos (2\beta) +\de_\beta \hE(\Ome+\Omega_r))(\tau) \, d\tau.\notag
\end{align}
Appealing now to the explicit formula of $\Ome$ in \eqref{eq: g explicit} and estimate \eqref{eq:g-upperandlower} with $g(t)=\Ome(t)$, for $ t \in [0, T^*(\alpha)]$, yields
\begin{align}
     \log \left(\frac{R(t)}{R(0)}\right)
    &\lesssim \alpha\log \left(1+\frac{c}{2\alpha}t\right) + \sqrt \alpha + \left|\int_0^t (\cL_{12}+\alpha \mathcal{R}^\alpha+\de_\beta \hE)(\Omega_r(\tau))\, d\tau\right|, \label{eq:est-R}
\end{align}
where we used Theorem \ref{thm:tarek3d}, Proposition \ref{lem:localized-ell} and \eqref{eq:est-app}, from which, in particular, it follows that
\begin{align*}
    \alpha \left| \int_0^t (\mathcal{R}^\alpha +\de_\beta \hE)(\Ome(\tau)) \, d\tau\right|\lesssim \alpha t \|\Ome\|_{\cH^N}  \lesssim \alpha C_N e^{\frac{C_N}{\alpha}t} \lesssim \sqrt \alpha, \quad t \in [0, T^*(\alpha)].
\end{align*}
To handle the last term of the right-hand side of \eqref{eq:est-R}, we use Corollary \ref{cor:rem3D}, in particular
\begin{align}
    \|\Omega_r(t)\|_{L^\infty} \le 4\sqrt \alpha.
\end{align}
Applying the same reasoning as before this yields, for $\alpha $ small enough, that 
\begin{align*}
    R(0) + \sqrt \alpha \le \sup_{t \in [0, T^*(\alpha)]} R(t)\le R(0)+2\sqrt \alpha \quad \Rightarrow \quad \left(\frac{1}{8}\right)^\frac 1 \alpha  \le  \cS(t)  \le \left(\frac 17 + 2\eps\right)^\frac 1 \alpha,
\end{align*}
since $\eps \sim |\log|\log|\alpha||^{-\frac 14} \gg \sqrt \alpha$.
Now we turn to $\left(\frac{\omega_\theta^\aa}{r}\right)$ in \eqref{eq:3daxisymm}, satisfying the equation below:
\begin{align*}
    (\de_t+u_r^\aa\de_r + u_z^\aa \de_z)\left(\frac{\omega_\theta^\aa}{r}\right)=-\frac{1}{r^4}\de_z(ru_\theta^\aa)^2. 
\end{align*}
The transport part is the same as before, the difference is the presence of the right-hand side.
We can rewrite the equation for $\omega_\theta^\aa$ as follows:
\begin{align}
    \frac{d}{dt}\omega_\theta^\aa(t, X(t, x_0)) = - \frac{1}{r^3} \de_z (r u_\theta^\aa (t, X(t, x_0)) )^2.
\end{align}
We already know from the above discussion that  $\left(\frac 18\right)^\frac 1 \alpha \lesssim \supp_{\rho} (ru_\theta^\aa)^2 \lesssim  \left(\frac 17\right)^\frac 1 \alpha$. Reasoning as before and solving the Liouville equation ensures that $\left(\frac 18\right)^\frac 1 \alpha \lesssim \cS(t) \lesssim  \left(\frac 17\right)^\frac 1 \alpha$. The proof is concluded.
\end{proof}
%


\subsection{Proof of Theorem \ref{thm:main3sec3}}
With all these ingredients, the proof of Theorem \ref{thm:main3sec3} is easily deduced. By definition of $\eta(\cdot, R, \beta)=\de_r (ru_\theta^\aa)^2$, it follows that
\begin{align*}
    \frac{\eta}{2r^2}=\frac{\de_r(r u_\theta^\aa)^2}{r^2}=\frac{ u_\theta^\aa}{r}+\de_r (u_\theta^\aa)^2.
\end{align*}
%
Consider now the approximate solution $(\Ome (t), \et (t), \csi (t))$. Using the remainder estimates in Section \ref{sec:remainder3d} and the localization of the support in Section \ref{sec:support}, in Case (i) we have that 
%
\begin{align*}
    \sup_{t \in [0, T^*(\alpha)]} \left\| \frac{u_\theta^\aa}{r}+\de_r (u_\theta^\aa)^2\right\|_{L^\infty}& \gtrsim \left\| \et \right\|_{L^\infty}-\sqrt \alpha \gtrsim \|\eta_{0, \text{app}}^{\alpha, \delta}\|_{L^\infty} \left(1+\frac{\log |\log \alpha|}{C_{N+1}}\right)^{\frac{1}{c_2}},
\end{align*}
%
from which \eqref{eq:crucial-inequality} follows. Similarly, in Case (ii) one obtains 
%
\begin{align*}
    \sup_{t \in [0, T^*(\alpha)]} \left\| \de_z (u_\theta^\aa)^2\right\|_{L^\infty}&  \gtrsim \|\xi_{0, \text{app}}^{\alpha, \delta}\|_{L^\infty} \left(1+\frac{\log |\log \alpha|}{C_{N+1}}\right)^{\frac{1}{c_2}}.
\end{align*}
The proof is concluded.
%%%%%
\section*{Acknowledgment}
RB thanks Tarek M. Elgindi for several helpful discussions and suggestions, especially regarding the application to the 3d axisymmetric Euler equations. RB is partially supported by the GNAMPA group of INdAM and from the PRIN project 2020 entitled ``PDEs, fluid dynamics and transport equation''. \\
LEH is funded by the Deutsche Forschungsgemeinschaft (DFG, German Research Foundation) – Project-ID 317210226 – SFB 1283.\\
FI is partially supported by  PRIN project 2017JPCAPN entitled ``Qualitative and quantitative aspects of nonlinear PDEs."
%%%%%%%%%%%%%%%%%%%%%%%%%%%%% Appendix %%%%%%%%%%%%%%5

% \appendix

% \section{Collezione di stime utili}

% The following lemma is useful. 
% \begin{Lem}\label{StimeLg1}
% The function $\mathcal{L}(g)$ defined in \eqref{def:L} satisfies the following estimates
% \begin{align}&\|\partial_R^j\mathcal{L}(g)\|_{L^{\infty}_R}\leq C\|g\|_{\cH^j},\quad \|\partial_R^j\mathcal{L}(g)\|_{L^{2}_R}\leq C\|g\|_{\cH^{j-1}},\label{leggere}\\ 
% &\|R^j\partial_R^j\mathcal{L}(g)\|_{L^{\infty}_R}\leq C\|g\|_{\cH^j},\quad \| R^j\partial_R^j\mathcal{L}(g)\|_{L^{2}_R}\leq C\|g\|_{\cH^{j-1}}\label{pesate},\end{align}
% for any $j\geq 1$.
% \end{Lem}
% \textcolor{red}{The proof uses that $0\notin \mathrm{supp(g)}$! Finally, it seems that we do not neet this lemma.}
% \begin{proof}
% We prove the $L^{\infty}$ bounds, the other ones are similar. Using the formula \eqref{eq:g-formula}, since $g_0(R)$ is supported for $R\geq 1$, we have that the solution $g$ itself is supported for $R\geq 1$.
% Recalling \eqref{def:L}, \eqref{eq:Lom-Lg},
% $$\partial_R^j\mathcal{L}=-\frac{1}{\pi}\int_0^{2\pi}\partial_R^{j-1}\Big(\frac{g(R,\beta)}{R}\Big)\sin(2\beta) d\beta.$$
% We prove the $L^{\infty}$ bound in \eqref{pesate}
% \begin{align*}
% \|R^j\partial_R^j \cL\|_{L^{\infty}}&\leq  C_j\sum_{i=1}^{j-1}\|\int_0^{2\pi}R^i\partial_R^ig(R,\beta)\sin2\beta d\beta\|_{L^{\infty}}\leq C_j\sum_{i=0}^{j-1}\|\|\sin{2\beta}\|_{L^2_{\beta}}\|R^{i}\partial_R^ig(R,\beta)\|_{L^2_{\beta}}\|_{L^{\infty}_R}\\
% &\leq C_j\sum_{i=0}^{j-1}\|\|R^i\partial_R^i g(R,\beta)\|_{L^2_{\beta}}\|_{H^1_{R}}\leq C_j \|g\|_{\cH^j},
% \end{align*}
% where $C_{j}$ is an harmless constants changing from line to line and where in the penultimate inequality we used the embedding $H^1_R\hookrightarrow L^{\infty}$.
% Concerning the $L^{\infty}$ bound in  \eqref{leggere}, we proceed analogously: using the Sobolev embedding $L^{\infty}_R\hookrightarrow H^1_R$, $R\geq 1$ on the support of $g$,  we obtain
% \begin{equation*}
% \|\partial_R^j\mathcal{L}(g)\|_{L^{\infty}}\leq C \sum_{i=0}^{j-1}\|\int_0^{2\pi}\partial_R^ig(R,\beta)\sin(2\beta)d\beta\|_{L^2_R}+\|\int_0^{2\pi}\partial_R^{i+1}g(R,\beta)\sin(2\beta)d\beta\|_{L^2_R},
% \end{equation*}
% one concludes using Cauchy-Schwartz inequality, as done before for the proof of \eqref{pesate}.
% \end{proof}


\bibliographystyle{siam}
\bibliography{biblio} 



% \begin{thebibliography}{99}
% \bibitem{marchioro} Benedetto, D.; Caglioti, E.; Marchioro, C. On the motion of a vortex ring with a sharply concentrated vorticity. Math. Methods Appl. Sci. 23 (2000), no. 2, 147--168.
% \bibitem{bianchini2}  Bedrossian J., Bianchini R., Zelati M. C., and Dolce M. Nonlinear inviscid damping and shear-buoyancy
% instability in the two-dimensional Boussinesq equations. arXiv preprint arXiv:2103.13713 (2021)
% \bibitem{bianchini} Bianchini, Roberta; Dalibard, Anne-Laure; Saint-Raymond, Laure. Near-critical reflection of internal waves. Anal. PDE 14 (2021), no. 1, 205--249. MR4229203
% \bibitem{bianchini1} Bianchini, Roberta; Coti Zelati, Michele; Dolce, Michele. Linear inviscid damping for shear flows near Couette in the 2D stably stratified regime. Indiana Univ. Math. J. 71 (2022), no. 4, 1467--1504. MR4481091
% \bibitem{bourgain} Bourgain, Jean; Li, Dong. Strong ill-posedness of the incompressible Euler equation in borderline Sobolev spaces. Invent. Math. 201 (2015), no. 1, 97--157. MR3359050
% \bibitem{bourgain2015} Bourgain, Jean; Li, Dong. Strong illposedness of the incompressible Euler equation in integer $C^m$ spaces. Geom. Funct. Anal. 25 (2015), no. 1, 1--86. MR3320889
% \bibitem{chae1999} Chae, Dongho; Kim, Sung-Ki; Nam, Hee-Seok. Local existence and blow-up criterion of Hölder continuous solutions of the Boussinesq equations. Nagoya Math. J. 155 (1999), 55--80. MR1711383 
% \bibitem{chen1} Chen, Jiajie; Hou, Thomas Y. Finite time blow-up of 2D Boussinesq and 3D Euler equations with $C^{1,\alpha}$ velocity and boundary. Comm. Math. Phys. 383 (2021), no. 3, 1559--1667. MR4244260 
% \bibitem{chen2022} Chen, Jiajie; Hou, Thomas Y. Stable nearly self-similar blow-up of the 2D Boussinesq and 3D Euler equations with smooth data. arXiv:2210.07191 (2022).
% \bibitem{constantin} Constantin, Peter. On the Euler equations of incompressible fluids. Bull. Amer. Math. Soc. (N.S.) 44 (2007), no. 4, 603--621. MR2338368
% %\bibitem{chen2022} Chen, Jiajie; Hou, Thomas Y. Stable nearly self-similar blow-up of the 2D Boussinesq and 3D Euler equations with smooth data. arXiv:2210.07191 (2022).
% \bibitem{cordoba1} C\'ordoba D.,  Mart\'{\i}nez-Zoroa, L., Non existence and strong ill-posedness in $C^k$ and Sobolev spaces for SQG, arXiv:2107.07463 (2021).
% \bibitem{cordoba2} C\'ordoba D.,  Mart\'{\i}nez-Zoroa, L., Non-existence and strong ill-posedness in $C^{k, \beta}$ for the generalized Surface Quasi-geostrophic equation, arXiv:2207.14385 (2022).
% \bibitem{DY1999}
% {Dauxois T., Young W. R., }
% {Near-critical reflection of internal waves}, 
% J. Fluid Mech. {390}  (1999), 271--295.
% \bibitem{drivas} Drivas T. D., Elgindi T. M., Singularity formation in the incompressible Euler equation in finite and infinite time, arXiv:2203.17221v2 (2022).
% \bibitem{tarek1} Elgindi T. M., Finite-Time Singularity Formation for $C^{1, \alpha}$ Solutions to the Incompressible Euler Equations on $\R^2$, Annals of Mathematics 194.3 (2021): 647-727.
% \bibitem{jeong} Elgindi, Tarek M.; Jeong, In-Jee. Finite-time singularity formation for strong solutions to the Boussinesq system. Ann. PDE 6 (2020), no. 1, Paper No. 5, 50 pp. MR4098032 
% \bibitem{jeong2} Elgindi, Tarek M.; Jeong, In-Jee. Finite-time singularity formation for strong solutions to the axi-symmetric 3D Euler equations. Ann. PDE 5 (2019), no. 2, Paper No. 16, 51 pp. MR4029562
% \bibitem{tarek3} Elgindi, Tarek M.; Masmoudi, Nader. $L^\infty$ ill-posedness for a class of equations arising in hydrodynamics. Arch. Ration. Mech. Anal. 235 (2020), no. 3, 1979--2025. MR4065655 
% \bibitem{tarek2} Elgindi Tarek M., Shikh Khalil Karim R., Strong Ill-Posedness in $L^\infty$ for the Riesz Transform Problem, arXiv:2207.04556v1 (2022).
% \bibitem{klaus1} Elgindi, Tarek M.; Widmayer, Klaus. Sharp decay estimates for an anisotropic linear semigroup and applications to the surface quasi-geostrophic and inviscid Boussinesq systems. SIAM J. Math. Anal. 47 (2015), no. 6, 4672--4684. MR3431131
% \bibitem{gallay} Gallay, Thierry. Stability of vortices in ideal fluids: the legacy of Kelvin and Rayleigh. Hyperbolic problems: theory, numerics, applications, 42--59, AIMS Ser. Appl. Math., 10, Am. Inst. Math. Sci. (AIMS), Springfield, MO, [2020], ©2020. MR4362504
% \bibitem{grafakos} Grafakos, Loukas. Classical Fourier analysis. Third edition. Graduate Texts in Mathematics, 249. Springer, New York, 2014. xviii+638 pp. ISBN: 978-1-4939-1193-6; 978-1-4939-1194-3 MR3243734
% \bibitem{klaus2} Guo, Yan; Pausader, Benoit; Widmayer, Klaus. Global axisymmetric Euler flows with rotation. Invent. Math. 231 (2023), no. 1, 169--262. MR4526823
% \bibitem{hientzsch} Hientzsch L.E., Lacave C., Miot E. Dynamics of several point vortices for the lake equations, arXiv:2207.14680 (2022).
% \bibitem{jeong3} Jeong, In-Jee, Kim Junha, A simple proof of ill-posedness for incompressible Euler equations in critical Sobolev spaces, Journal of Functional Analysis
% Volume 283 (2022).
% \bibitem{kato72} Kato, Tosio. Nonstationary flows of viscous and ideal fluids in ${\bf R}\sp{3}$. J. Functional Analysis 9 (1972), 296--305. MR0481652 
% \bibitem{kiselev2022} Kiselev A., Park J., Yao Yao. Small scale formation for the 2D Boussinesq equation, arXiv:2211.05070 (2022). 
% \bibitem{lannes} Desjardins, Benoît; Lannes, David; Saut, Jean-Claude. Normal mode decomposition and dispersive and nonlinear mixing in stratified fluids. Water Waves 3 (2021), no. 1, 153--192. MR4246392
% \bibitem{long1965} 
% Long R. R. On the Boussinesq approximation and its role in the theory of internal waves, Tellus 17
% (1965), pp. 46–52.
% \bibitem{bertozzi} Majda, Andrew J.; Bertozzi, Andrea L. Vorticity and incompressible flow, Cambridge Texts in Applied Mathematics 27, Cambridge University Press, Cambridge 2002.
% \bibitem{rieutord} 
% Rieutord M.,
% {Fluid Dynamics: An Introduction}, Graduate Texts in Physics, Springer International Publishing
% (2015), XVI+508.
% \bibitem{yudovich} Yudovich, V. I. Eleven great problems of mathematical hydrodynamics. Dedicated to Vladimir I. Arnold on the occasion of his 65th birthday. Mosc. Math. J. 3 (2003), no. 2, 711--737, 746. MR2025281
% \bibitem{wang2022} Wang Y., Lai C.-Y.,  Gomez-Serrano J. , and Buckmaster T. Self-similar blow-up profile
% for the Boussinesq equations via a physics-informed neural network. arXiv:2201.06780v1, 2022.
% \end{thebibliography}
\end{document}
