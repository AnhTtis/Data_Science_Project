We now apply the machinery reviewed in the previous section to study deformation theoretic
questions about coalgebras in the category of spectra. We first prove that the functors which
assign to a connective $\E_\infty$-ring $R$ the categories of connective $R$-modules and $p$-complete
connective $R$-modules respectively, are cohesive an nilcomplete and that this implies the same for
coalgebras in those categories.
We then introduce \textit{formally \'etale coalgebras} in an arbitrary cohesive moduli problem
\[\cC_\blank: \CAlg^{\cn} \to \CAlg(\rm{Pr^L})\]
and show that the space of lifts of a formally
\'etale coalgebra $A\in \cCAlg(\cC_R)$ along any square zero extension $R^\eta \to R$ is contractible.
Moreover, we show that the assignment of $A$ to its essentially unique
lift $A^{\eta} \in \rm{cCAlg}(\cC_{R^{\eta}})$ refines to a fully faithful
functor $\rm{cCAlg}(\cC_R)^{\rm{f\acute{e}t}} \to \rm{cCAlg}(\cC_{R^{\eta}})$.

\subsection{Moduli of coalgebras}

We now prove that moduli of coalgebras are cohesive and nilcomplete,
and hence we can use the tangent complex machinery discussed in Section~\ref{sect22}. We deduce
this from the fact that the categories of connective modules over a connective $\bb{E}_\infty$-ring
commute with the corresponding limits and the equivalences are strong monoidal.

\begin{lemma}\label{edescent}
Let $\cl{C}_{\blank}: \rm{CAlg}^{\rm{cn}} \to \CAlg(\rm{Pr^L})$
be cohesive or nilcomplete. Then the functor
$\rm{cCAlg}(\cl{C}_\blank): \rm{CAlg}^{\rm{cn}} \to \rm{Pr^L}$ is
is also cohesive or nilcomplete, respectively.
\end{lemma}
\begin{proof}
Clear, since by Proposition~\ref{calg} and Lemma~\ref{limits} the functor
$\cCAlg(\blank)$ commutes with limits.
\end{proof}

\begin{theorem}[Lurie]
  Suppose we have a pullback of connective $\bb{E}_{\infty}$-rings
  \[\begin{tikzcd}
      {R^\prime} & {S^\prime} \\
      R & S \arrow[from=1-1, to=1-2] \arrow[from=1-2, to=2-2]
      \arrow[from=1-1, to=2-1] \arrow[from=2-1, to=2-2]
    \end{tikzcd}\]
  such that one of the maps $\pi_{0}R \to \pi_{0}S$, $\pi_{0}S\p \to \pi_{0}S$ is surjective.
  Then the natural map
  \[ \rm{Mod}^{\rm{cn}}_{R\p} \to \rm{Mod}_{R}^{\rm{cn}} \times_{\rm{Mod}_{S}^{\rm{cn}}} \rm{Mod}_{S\p}^{\rm{cn}}\]
  is an equivalences of categories with inverse taking a point in the pullback $(M,N,h)$, consisting of
  $M \in\rm{Mod}_{R}^{\rm{cn}}, N \in \rm{Mod}_{S\p}^{\rm{cn}}$ and a homotopy $h: M\otimes_{R}S\simeq N \otimes_{S\p}S$, to
  $M \times_{M \otimes_{R}S} N$ with the induced $R\p$-module structure.
\end{theorem}
\begin{proof}
\cite[][Theorem 16.2.0.2.]{sag}
\end{proof}

\begin{corollary}\label{Mod}
    The functor
    \[ \CAlg^{\rm{cn}} \to \CAlg(\rm{Pr^L}) \quad R \mapsto \Mod_R^{\rm{cn}} \]
    is cohesive.
\end{corollary}

This shows that moduli of coalgebras are cohesive in the sense of Definition~\ref{cohesive}.
To prove that they are also nilcomplete we need the following technical Lemma.

\begin{lemma}\label{conn}
  Let $\dots \to  E_{2} \to E_{1} \to E_{0}$ be a diagram of spectra. Suppose we are given $L \ge 0$ such
  that for all $\ell\p\ge\ell \ge L$ the map $E_{\ell\p}\to E_{\ell}$ is $m$-connective. Then for any $\ell \ge L$ the map
  \[ \flim_{n} E_{n}\to E_{\ell}\]
  is $m-1$-connective.
\end{lemma}

\begin{proof}
  Writing $F_{\ell\p,\ell}= \rm{fib}(E_{\ell\p}\to E_{\ell})$ and $F_{\ell}= \rm{fib}(\lim_{n}E_{n}\to E_{\ell})$ we want
  to show that $F_{\ell}$ is $m$-connective. Indeed, since limits are exact, we have that
  \[F_{\ell} \simeq \lim_{\ell\p >\ell}F_{\ell\p,\ell} \simeq \rm{fib}\left( \prod_{\ell\p >\ell} F_{\ell\p,\ell}\to \prod_{\ell\p>\ell} F_{\ell\p -1, \ell}\right ).\]
  Thus, since $\rm{Sp}_{\geq m}$ is closed under products and the fiber of a map of $m$-connective spectra
  is $(m-1)$-connective, the claim follows.
\end{proof}

\begin{proposition}\label{nilmod}
The functor
\[ \CAlg^{\rm{cn}} \to \CAlg(\rm{Pr^L}) \quad R \mapsto \rm{Mod}_R^{\rm{cn}}\]
is nilcomplete.
\end{proposition}
\begin{proof}
  Let $R$ be a connective $\bb{E}_{\infty}$-ring. We need to show that the functor
  \[\rm{Mod}_{R}^{\rm{cn}} \to \flim_{n} \rm{Mod}_{\tau_{\le n}R}^{\rm{cn}} \quad M \mapsto M \otimes_{R} \tau_{\le n}R\]
  is an equivalence of categories.
  Write $R_{n}:= \tau_{\leq n}R$. The functor admits a right adjoint which
  takes $(M_{n})\in \flim_{n} \rm{Mod}_{R_{n}}^{\rm{cn}}$ to the limit $\lim_{n} M_{n}$ which inherits
  a natural action by $\lim_{n} R_{n}\simeq R$. Now let $N\in \rm{Mod}_{R}$. Since taking limits is exact,
  the counit of the adjunction sits in a fiber sequence
  \[\lim_{n} \rm{fib}(N \to N \otimes_{R} R_{n}) \to N \rar{\eta} \lim_{n} (N \otimes_{R} R_{n}).\]
  where we compute for the left hand term that
  \[ \rm{fib}(N \to N \otimes_{R} R_{n}) \simeq \rm{fib}(N \otimes_{R} R \to N \otimes_{R} R_{n}) \simeq N \otimes_{R} \rm{fib}(R \to R_{n}).\]
  Now, since $R_{n}= \tau_{\leq n} R$, the connectivity of $\rm{fib}(R\to R_{n})$ increases with $n$. Hence,
  since $R$ and $N$ are connective, so does the connectivity of the tensor product
  $N \otimes_{R} \rm{fib}(R \to R_{n})$ which implies that $\flim_{n}N \otimes_{R} \rm{fib}(R \to R_{n}) \simeq 0$. Thus,
  the counit $N \to \lim_{n}(N \otimes_{R} R_{n})$ is an equivalence. \\
  Now let $(M_{n}) \in \flim_{n}\rm{Mod}_{R_{n}}^{\rm{cn}}$ and write $M= \lim_{n} M_{n}$.
  We need to show that the natural map
  \[ \eps_{k}:M \otimes_{R} R_{k} \to R_{k}\]
  is an equivalence for each $k$. We do this by showing that it is $m$-connective for any $m\ge 0$.
  Indeed, for any such $m$ there exists an integer $L$ such that for all
  $\ell \geq \ell\p > L$ the natural map $R_{\ell} \to R_{\ell\p}$ is $m$-connective. Since
  $M_{\ell\p}\simeq M_{\ell}\otimes_{R_{\ell}} R_{\ell\p}$  we have a fiber sequence
  \[ M_{\ell}\otimes_{R_{\ell}} (\rm{fib}(R_{\ell}\to R_{\ell\p})) \to M_{\ell} \to M_{\ell\p}.\]
  Hence, since $\rm{fib}(R_{\ell}\to R_{\ell\p})$ is $m$-connective and $R_{\ell}$ and $M_{\ell}$ are connective, the tensor
  product $M_{\ell}\otimes_{R_{\ell}} \rm{fib}(R_{\ell}\to R_{\ell\p})$ is $m$-connective as well. Thus, for fixed $m$ and $k$ we
  may apply Lemma~\ref{conn} to obtain  $\ell>k$ such that the maps $M \to M_{\ell}$ and $R\to R_{\ell}$ are
  both $m$-connective. Finally, the map
  \[ \eps_{k}: M\otimes_{R}R_{k} \to M_{\ell}\otimes_{R_{\ell}}R_{k} \simeq M_{k}\]
  is given by the colimit of the induced map between the bar resolutions
\[\begin{tikzcd}
	\vdots & \vdots \\
	{M\otimes R \otimes R_k} & {M_\ell\otimes R_\ell \otimes R_k} \\
	{M \otimes R_k} & {M _\ell \otimes R_k}
	\arrow[from=1-1, to=2-1]
	\arrow[from=1-2, to=2-2]
	\arrow[shift left=2, from=2-1, to=3-1]
	\arrow[shift left=2, from=2-2, to=3-2]
	\arrow[from=3-1, to=3-2]
	\arrow[from=2-1, to=2-2]
	\arrow[shift right=3, from=1-1, to=2-1]
	\arrow[shift left=3, from=1-1, to=2-1]
	\arrow[shift right=3, from=1-2, to=2-2]
	\arrow[shift left=3, from=1-2, to=2-2]
	\arrow[shift right=2, from=2-1, to=3-1]
	\arrow[shift right=2, from=2-2, to=3-2]
\end{tikzcd}.\]
Denote by $F_{n}$ the fiber of the map $M \otimes R^{\otimes n}\otimes R_{k} \to M_{\ell}\otimes R_{\ell}^{\otimes n}\otimes R_{k}$. Since the tensor product
of $m$-connective maps is again $m$-connective, the fiber $F_{n}$ is $m$-connective. Thus, by exactness
of colimits, we obtain a fiber sequence
\[\colim F_{n} \to M \otimes_{R} R_{k} \rar{\eps_{k}} M_{k}\]
and finally, since taking colimits preserves connectivity, this shows that the map
$\eps_{k}$ is $m$-connective. Since $m$ was arbitrary, the map $\eps_{k}$ is in fact an equivalence
which completes the proof.
\end{proof}

\begin{corollary}\label{cohesive}
The functor $\cCAlg(\Mod_{\blank}^{\rm{cn}}):\CAlg^{\rm{cn}} \to \rm{Pr^L}$ is cohesive and nilcomplete.
\end{corollary}

\begin{proof}
Apply Lemma~\ref{edescent} to Proposition~\ref{nilmod} and Corollary~\ref{Mod}.
\end{proof}

\subsection{$p$-complete moduli}

Throughout this section fix a prime $p$.
Let $R$ be an $\bb{E}_{\infty}$-ring. Recall that a module $M \in \rm{Mod}_{R}$ is called
$p$-\textit{complete} if the limit
\[ \lim \left(\dots \rar{\cdot p} M \rar{\cdot p}M \right)\]
vanishes. We denote the full subcategory spanned by the $p$-complete modules 
by $\rm{Mod}_{R}^{\wedge}$.
The inclusion $\rm{Mod}_{R}^{\wedge} \to \rm{Mod_{R}}$ admits a left adjoint which 
takes a module $M$ to its \textit{$p$-completion} given by the limit
\[ M^\wedge_p:=\lim \left( \dots \to M/p^{2} \to M/p \right).\]
In fact, $M$ is $p$-complete if and only if the natural map $M \to \lim M/p^{n}$ is an equivalence.
The spectrum $M^\wedge_p$ inherits a natural $R^{\wedge}_{p}$-module structure,
and $p$-completion induces an equivalence of categories
\[\rm{Mod}_{R}^{\wedge} \simeq \rm{Mod}_{R^{\wedge}_{p}}^{\wedge}.\]
The tensor product of $p$-complete modules is in general not $p$-complete. However, the
category $(\rm{Mod}_{R})_{p}^{\wedge}$ admits a presentably symmetric monoidal structure
given by the formula
 \[ M \otimes_{(\rm{Mod}_{R})_{p}^{\wedge}} N := ( M \otimes N )^{\wedge}_{p}.\]
 With this monoidal structure the $p$-completion functor $\rm{Mod}_{R}\to (\rm{Mod}_{R})_{p}^{\wedge}$
 is monoidal, while the inclusion is only lax monoidal. 

\begin{lemma}
    Let $R_{\blank}:I \to \CAlg^{\rm{cn}}$ be a diagram of connective $\bb{E}_\infty$ rings with limit $R$, such
    that the natural functor
    \[ F:\Mod_R \to \flim_{i\in I} \Mod_{R_i}\]
    is an equivalence of categories. Then the induced functor
    \[ F^\wedge:\Mod_R^\wedge \to \flim_{i \in I} \Mod_{R_i}^\wedge\]
    is also an equivalence.
\end{lemma}
\begin{proof}
    This follows since, for a point $(M_i) \in \flim_i \Mod_{R_i}$, the underlying spectrum
    of $F^{-1}(M)$ is computed as the limit of spectra $\lim_i M \otimes R_i$ and $p$-completion
    commutes with limits and is monoidal.
\end{proof}

 \begin{corollary}\label{pnil}
 The functor 
 \[ \CAlg^{\rm{cn}} \to \rm{CAlg}(\rm{Pr^L}) \quad R \mapsto \Mod_R^{\cn,\wedge}\]
 is cohesive and nilcomplete.
 \end{corollary}


\begin{proposition}\label{pcomp}
 Let $k$ be a perfect $\F_p$-algebra and denote by $W_0(k)$ the $p$-typical Witt-vectors of $k$. 
 The functor
\[ \rm{Mod}_{W_0(k)}^\wedge \to \lim_n \rm{Mod}_{W_0(k)/p^n} \quad N \mapsto N 
\otimes_{W_0(k)} W_0(k)/p^n \]
is a strong monoidal equivalence.
\end{proposition}
\begin{proof}
  Let us first reduce to the case $k= \F_p$ and $W_0(R)= \Z_p$.
  Indeed, suppose we have proven that case, then we have equivalences
  \begin{align*}
      \rm{Mod}_{W_0(k)}^\wedge &\simeq \rm{Mod}_{W_0(k)}(\rm{Mod}_{\Z_p}^\wedge)\\
      &\simeq \rm{Mod}_{W_0(k)}(\lim_n \rm{Mod}_{\Z/p^n})\\
      &\simeq \lim_n \rm{Mod}_{W_0(k)/p^n}(\rm{Mod}_{\Z/p^n})\\
      &\simeq \lim_n \rm{Mod}_{W_0(k)/p^n},
  \end{align*}
  and so we are done. Thus, we assume $k= \F_p$ in the following.
  The functor admits a right adjoint which takes $(M_{n})\in \flim_{n}\rm{Mod}_{\Z/p^{n}}$ to the limit
  $\lim_{n}M_{n}$ taken in the category of $\Z_{p}$-modules. Since $p$-complete modules are closed under
  limits, the essential image of this functor is contained in $\rm{Mod}_{\Z_{p}}^{\wedge}$. Moreover,
  if $M\in \rm{Mod}_{\Z_{p}}^{\wedge}$, then we have that
  \[ \flim_{n}(M \otimes_{\Z_{p}} \Z/p^{n}) \simeq \flim_{n} M/p^{n} \simeq M^{\wedge}_{p}\simeq M.\]
  Hence, the counit of the adjunction is an equivalence on $p$-complete modules.
  Conversely, given $(N_{k})\in \flim_{k}\rm{Mod}_{\Z/p^{k}}$ write $N= \lim_{k}N$. We want
  to show that, for every $n$ the natural map
  \[ N \otimes_{\Z_{p}} \Z/p^{n}\rar{\sim}N_{n}\]
  is an equivalence. Since $N \otimes_{\Z_{p}}Z/p^{n}\simeq N/p^{n}$ and limits are exact, we have an equivalence
  \[N \otimes_{\Z_{p}}\Z/p^{n}\simeq \lim_{k >n}(N_{k}\otimes_{\Z_{p}}\Z/p^{n}).\]
  Thus, the unit of the adjunction may be written as
  \[ \lim_{k>n}(N_{k} \otimes_{\Z_{p}}\Z/p^{n}) \to \lim_{k>n}(N_{k}\otimes_{\Z/p^{k}}\Z/p^{n})\simeq N_{n}\]
  and so has fiber given by
  \[ F_{n}:=\lim_{k>n}\left(N_{k}\otimes_{\Z/p^{k}}\rm{fib}(\Z/p^{k}\otimes_{\Z_{p}}\Z/p^{n}\to \Z/p^{n}) \right).\]
  Now we compute the fiber of $\Z/p^{k}\otimes_{\Z_{p}}\Z/p^{n}\to \Z/p^{n}$ as the module
  \[ \rm{Tor}^{\Z_{p}}(\Z/p^{k}, \Z/p^{n})[1]\simeq \Z/p^{n}[1].\]
  The reduction map $\Z/p^{k}\to \Z/p^{k-1}$ is induced by the map of projective resolutions
\[\begin{tikzcd}
	{\Z_p} & {\Z_p} \\
	{\Z_p} & {\Z_p}
	\arrow["{\cdot p^k}", from=1-1, to=1-2]
	\arrow["\id", from=1-2, to=2-2]
	\arrow["{\cdot p}"', from=1-1, to=2-1]
	\arrow["{\cdot p^{k-1}}"', from=2-1, to=2-2],
\end{tikzcd}\]
hence, on Tor it induces the multiplication by $p$ map
\[ \Z/p^{n}=\rm{Tor}^{\Z_{p}}(\Z/p^{k}, \Z/p^{n})\rar{\cdot p} \rm{Tor}^{\Z_{p}}(\Z/p^{k-1}, \Z/p^{n}) =\Z/p^{n}.\]
Thus, if we have $k\p > k > n$ such that $k\p -k > n$, the transition map
\[ F_{k\p}=N_{k\p} \otimes \rm{Tor}^{\Z_{p}}(\Z/p^{k}, \Z/p^{n})\to N_{k} \otimes \rm{Tor}^{\Z_{p}}(\Z/p^{k-1}, \Z/p^{n})= F_{k}\]
vanishes since the Tor-groups are $p^{n}$-torsion. Choosing a cofinal subset $S\subseteq \bb{N}_{>n}$ such that
$\abs{k\p -k}> n$ for any distinct $k\p,k\in S$, we see that
\[ \lim_{k>n} F_{k}\simeq \lim_{k\in S} F_{k} \simeq 0 \]
vanishes. Thus, since limits are exact, the map $N \otimes_{\Z_{p}} \Z/p^{n}\rar{\sim}N_{n}$ is an equivalence.\\
To see that the functor $\rm{Mod}_{\Z_{p}}^{\wedge} \to \flim_n \rm{Mod}_{\Z/p^{n}}$ is strong monoidal,
we observe that since cofibers and limits are exact, we have for each $n$ equivalences
\begin{align*}
  (M \otimes_{\Z_{p}} N)^{\wedge}_{p} \otimes_{\Z_{p}}\Z/p^{n} &\simeq \lim_{k}(M/p^{k} \otimes_{\Z_{p}}N/p^{k})/p^{n}\\
                                              &\simeq \lim_{k}\left((M/p^{n} \otimes_{\Z_{p}} N/p^{n})\otimes_{Z_{p}}\Z/p^{k}\right) \\
  &\simeq ((N\otimes_{\Z_{p}}\Z/p^{n}) \otimes_{\Z_{p}} (M \otimes_{\Z_{p}}\Z/p^{n}))^{\wedge}_{p}.
\end{align*}
This proves the claim.
\end{proof}

\begin{corollary}\label{pcomp1}
  We have an equivalence of categories
  \[ \rm{cCAlg}(\Mod_{W_0(k)}^{\wedge} \rar{\sim} \flim_{n} \rm{cCAlg}(\Mod_{W_0(k)/p^{n}})
  \quad A \mapsto (A\otimes_{W(R)} W(R)/p^{n})\]
  with inverse taking a system of coalgebras $(B_{n})$ to the limit $\lim_{n}B_{n}$ computed in the
  category of $p$-complete $W_0(k)$-modules.
\end{corollary}
\begin{proof}
This follows from Proposition~\ref{pcomp}, arguing as in the proof of Proposition~\ref{Mod}.
\end{proof}

We also observe that the tangent complex is not affected by $p$-completion of coalgebras.

  \begin{lemma}\label{pcomparison}
    Write $\cl{X}(\blank)=\rm{cCAlg}(\Mod^{\rm{cn}}_{\blank})$ and $\cl{Y}(\blank)=
    \rm{cCAlg}(\Mod^{\rm{cn}~\wedge}_{\blank})$. Then the $p$-completion map $f:\cl{X}\to \cl{Y}$
    induces an equivalence
    \[ T^{M}_{(\cl{X}^{\Delta^{n}})_{\xi}} \to  T^{M}_{(\cl{Y}^{\Delta^{n}})_{f(\xi)}}\]
        for every $\F_{p}$-module $M$, $n\in \bb{N}$ and $\xi \in \cl{X}(\F_{p})^{\Delta^{n}}$.
  \end{lemma}
  \begin{proof}
    Since $\F_{p}$-algebra $R$ is $p$-complete, the $p$-completion functor gives an equivalence
    $\rm{Mod}_{R}\rar{\sim} \rm{Mod}_{R}^{\wedge}$, since multiplication by some power of $p$
    is nullhomotopic over $\F_{p}$. In particular, this applies to the split square zero
    extension $\F_{p}\oplus M$ for any $M \in \rm{Mod}_{\F_{p}}$ and so the natural map
    $\cl{X}(\F_{p}\oplus M) \to \cl{Y}(\F_{p}\oplus M)$ is an equivalence as well.
    Consequently, we also obtain natural equivalences between the fibers
    \[ (\cl{X}^{\Delta^{n}})_{\xi}^{\F_{p}\oplus M} \to  (\cl{Y}^{\Delta^{n}})_{f(\xi_)}^{\F_{p}\oplus M},\]
    which induces the equivalence of spectra
    \[ T^{M}_{(\cl{X}^{\Delta^{n}})_{\xi}} \to  T^{M}_{(\cl{Y}^{\Delta^{n}})_{f(\xi)}}\]
      as claimed.
  \end{proof}

\subsection{Formally \'etale coalgebras}

Throughout this section, we fix a cohesive functor
\[ \cl{C}_{\blank}: \rm{CAlg}^{\rm{cn}}\to \CAlg(\rm{Pr^L}).\]
Then by Lemma~\ref{limits} the functor
\[ \cCAlg(\cl{C}_\blank): \rm{CAlg}^{\rm{cn}} \to \rm{Pr^L}\]
is also cohesive and for any map $f: R \to S$ of connective $\bb{E}_\infty$ rings we have a coalgebraic adjunction
\[ f^\pt: \cCAlg(\cl{C}_{R}) \leftrightarrows \cCAlg(\cl{C}_S):f_\pt.\]
We refer to $f^\pt$ as base change and to $f_\pt$ as Weil restriction along $f$.


\begin{construction}\label{univdef}
Let $R$ be a connective $\bb{E}_\infty$-ring and $M$ a connective $R$-module. Denote by $e: R \to R\oplus M$
the 0-section of the split square zero extension.
    For any $A\in \rm{cCAlg}(\cl{C}_R)$ we define the \textit{universal $M$-deformation coalgebra}
    of $A$ as the Weil restriction
    \[ \Omega^{\infty}_{A}M:= e_{\pt}e^{\pt} A \in \rm{cCAlg}(\cl{C}_R).\]
    This naturally receives a unit map $\varepsilon:A \to \Omega_A^\infty M$.
\end{construction}



\begin{construction}\label{pi}
Suppose we are given an adjunction
\[\begin{tikzcd}
	{f^\pt:\cC} & {\cD:f_\pt}
	\arrow[""{name=0, anchor=center, inner sep=0}, shift left=2, from=1-1, to=1-2]
	\arrow[""{name=1, anchor=center, inner sep=0}, shift left=2, from=1-2, to=1-1]
	\arrow["\dashv"{anchor=center, rotate=-90}, draw=none, from=0, to=1]
\end{tikzcd}\]
 such that $f^\pt$ admits a retract $g^{\pt}:\cl{D}\to \cl{C}$. Consider the natural transformation
  \[\pi: f_{\pt}f^{\pt} \rar{\sim} g^{\pt} f^{\pt}f_{\pt}f^{\pt}
  \to  g^{\pt}f^{\pt} \rar{\sim} \id_{\cl{C}}\]
  defined as the whiskering of the counit $\eps:f^{\pt}f_{\pt} \to \id$ as in the diagram
\[\begin{tikzcd}
	{\cl{C}} & {\cl{D}} & {\cl{D}} & {\cl{C}}
	\arrow[""{name=0, anchor=center, inner sep=0}, "{f^\pt f_\pt}", curve={height=-12pt}, from=1-2, to=1-3]
	\arrow[""{name=1, anchor=center, inner sep=0}, "\id"', curve={height=12pt}, from=1-2, to=1-3]
	\arrow["{f^\pt}", from=1-1, to=1-2]
	\arrow["{g^{\pt}}", from=1-3, to=1-4]
	\arrow[shorten <=3pt, shorten >=3pt, Rightarrow, from=0, to=1].
\end{tikzcd}\]
Unraveling the definition we see that for each $B,A \in \cl{C}$ the composition
\[ \rm{Map}_{\cl{C}}(B, f_{\pt}f^{\pt} A) \rar{\sim}\rm{Map}_{\cl{D}}(f^{\pt}B, f^{\pt} A)
 \rar{g^{\pt}} \rm{Map}_{\cl{C}}(B, A)\]
takes $\psi: B \to f_{\pt}f^{\pt}A$ to the composite $\pi_{A} \circ \psi$. Thus, for each $\varphi:B \to A$ we have an
equivalence between the fiber
\[ \rm{fib}_{\varphi}\left(\rm{Map}_{\cl{D}}(f^{\pt}B, f^{\pt}A) \rar{g^{\pt}}
    \rm{Map}_{\cl{C}}(B, A) \right)\]
and the mapping space
\[ \rm{Map}_{\cl{C}_{/A}}((B\rar{\varphi} A), (f_{\pt}f^{\pt}A \rar{\eta_{A}} A)).\]
\end{construction}


\begin{lemma}\label{prlspectra}
Let $\cC =\lim_i\cC_i$ be a limit diagram in $\CAlg(\rm{Pr^L})$ and suppose we are
given a map $f: \cD\to \cC$ with projections $f_i: \cD \to \cC_i$. Then 
for any $A\in \cCAlg(\cl{\cD})$ we have a natural equivalence
\[ f_\pt f^\pt A \simeq \lim_i (f_i)_\pt (f_i)^\pt A \in \cCAlg(\cD). \]
\end{lemma}
\begin{proof}
Unravelling the adjunction and using the Yoneda lemma, this is equivalent to 
the claim that for any $B\in \cCAlg(\cD)$ the natural map
\[ \Map_{\cCAlg(\cC)}(f^\pt B, f^\pt A) \to \lim_i \Map_{\cCAlg(\cC_i)}(f_i^\pt B, f_i^\pt A) \]
is an equivalence, which is precisely the formula for mapping spaces
in the limit $\cC=\lim_i\cC_i$.
\end{proof}

\begin{proposition}\label{specref}
    Let $R$ be a connective $\bb{E}_\infty$-ring and $M\in \Mod_R^{\cn}$. Then for any 
    $A\in \cCAlg(\cC_R)$ we have a natural pullback diagram
    \[\begin{tikzcd}
	{\Omega^\infty_AM} & A \\
	A & {\Omega^\infty_A M[1]}
	\arrow[from=1-1, to=1-2]
	\arrow[from=1-1, to=2-1]
	\arrow[from=1-2, to=2-2]
	\arrow[from=2-1, to=2-2].
\end{tikzcd}\]
\end{proposition}
\begin{proof}
Indeed, since $\cCAlg(\cC_{\blank})$ is cohesive we have a pullback diagram in $\rm{Pr^L}$
of the form
    \[\begin{tikzcd}
	{\cCAlg(\cC_{R\oplus M})} & {\cCAlg(\cC_{R})} \\
	{\cCAlg(\cC_R)} & {\cCAlg(\cC_{R\oplus M[1]})}
	\arrow[from=1-1, to=1-2]
	\arrow[from=1-1, to=2-1]
	\arrow[from=1-2, to=2-2]
	\arrow[from=2-1, to=2-2],
\end{tikzcd}\]
where the maps are base change along the projection $p: R\oplus M \to R$ and 0-section
$s:R\to R\oplus \Sigma M$ respectively. Base change along the 0-section $e: R\to R\oplus M$
gives a map into the pullback $e^\pt: \cCAlg(\cC_R) \to \cCAlg(\cC_{R\oplus M})$. Unwrapping
the definitions and using that $p^\pt e^\pt =\id$, the claim follows by applying
Lemma~\ref{prlspectra}.
\end{proof}

\begin{definition}\label{formalet}
    Let $R$ be a connective $\bb{E}_\infty$-ring and $A\in \cCAlg(\cC_R)$. 
    By Proposition~\ref{specref}, the assignment
    $M \mapsto \Omega^\infty_A M $ canonically refines to a functor
    \[ L_A: \Mod_R \to \Sp(\cCAlg(\cC_R)_{/A}),\]
    such that $\Omega^\infty L_A(M)\simeq \Omega^\infty_A M$. We call $A$ \textit{formally \'etale} 
    if $L_A\simeq 0$ and denote by $\cCAlg(\cC_R)^{\fet}$ the full subcategory spanned
    by the formally \'etale coalgebras.
\end{definition}

\begin{remark}
    By construction of the functor $L_A$, a coalgebra $A \in \cCAlg(\cC_R)$ 
    is formally \'etale if and only if for any $M\in \Mod_{R}^{\cn}$ either of the natural maps
    \[ A \rar{\eps} \Omega^\infty_A M \rar{\pi} A\]
   is an equivalence. In fact, since the composition $\pi \circ \eps$ is always homotopic to the 
   identity, it suffices to show that there exists some isomorphism $\Omega^\infty_A M \simeq A$.
   This is how we verify the condition in practice. 
\end{remark}

\begin{remark}
    We do not know whether the functor $L_A$ commutes with limits or colimits. Hence, all we may
    deduce about the closure properties of formally \'etale coalgebras is the following
    \begin{enumerate}
        \item Since the product of coalgebras is given by the underlying tensor product and base
        change is symmetric monoidal, finite products of formally \'etale coalgebras are formally 
        \'etale.
        \item Let $\kappa$ be a regular cardinal such that $\cCAlg(\cC_R)$ is $\kappa$-presentable.
              Then $\kappa$-filtered colimits of formally \'etale coalgebras are formally \'etale.
    \end{enumerate}
    One may contemplate the a priori weaker condition where we only ask that $L_A(\Sigma^nR)=0$ for
    all $n\geq 0$. This admits stronger closure properties, for example it is closed under
    limits since $R\oplus \Sigma^nR$ is dualizable as an $R$-module. Moreover, one 
    can combine Proposition~\ref{structure} with Proposition~\ref{etalchar} to see that this is
    equivalent to $L_A(M)$ vanishing for all dualizable $M$. With appropriate modification, the results
    of this section also apply to this notion of formally \'etale coalgebras. However, working
    with this notion makes lifting along iterated square zero extensions more annoying and we do not 
    know of an example of a coalgebra which satisfies this condition but not the stronger one of
    Definition~\ref{formalet}.
\end{remark}


\begin{definition}\label{derivations}
Let $R$ be a connective $\bb{E}_\infty$-ring, $A\in \rm{cCAlg}(\cl{C}_R)$ 
and $M \in \Mod_R^{\rm{cn}}$. For any other $B\in \cCAlg(\cl{C}_R)$ and a 
map $\varphi : B\to A$ we define the \textit{spectrum of derivations} from 
$B$ to $M$ as the mapping spectrum
  \[ \rm{der}_{\varphi}(B, M):= 
  \rm{map}_{\Sp(\rm{cCAlg}(\cl{C}_{R})_{/A})}(\Sigma^\infty_+B, L_A(M)).\]
  We also write 
  \[ \rm{Der}_{\varphi}(B, M) := \Omega^\infty \rm{der}_\varphi(B,M) \]
  for the underlying space.
\end{definition}


\begin{proposition}\label{maplifts}
  Let $R$ be a connective $\bb{E}_\infty$-ring, $M \in \Mod_R^{\rm{cn}}$
  and let $\cl{X}(\blank)=\rm{cCAlg}(\cl{C}_{\blank})$.
  Moreover, let $\varphi: B \to A$ a map in $\cCAlg(\cl{C}_{R})$
  i.e.~a point $\varphi \in \cl{X}^{\Delta^1}(R)$.
  We have natural equivalences
  \[ T^M_{\cl{X}^{\Delta^1}_\varphi} \simeq \rm{der}_\varphi (B, M)\]
  \[ T^M_{\cl{X}^{\Delta^0}_A} \simeq \rm{der}_{\id} (A, M[1]).\]
\end{proposition}
\begin{proof}
The first equivalence is clear from Construction~\ref{pi}. For the second, let $e:R \to R\oplus M[1]$
denote the 0-section. Observe that, since $\cl{X}^{\Delta^0}$ is cohesive we have natural equivalences
\begin{align*}
 (\cl{X}^{\Delta^0})_{A}^{R\oplus M} &\simeq \Omega (\cl{X}^{\Delta^0})_{A}^{R\oplus M[1]}\\
  &\simeq  \rm{fib}_{\id_{A}}(\rm{Map}_{\rm{cCAlg}(\cl{C}_{R\oplus M[1]})}(e^{\pt}A, e^{\pt}A )
  \to  \rm{Map}_{\rm{cCAlg}(\cl{C}_R)}(A, A))\\
  &\simeq \rm{Der}_{\id} (A, M[1]),
\end{align*}
as claimed.
\end{proof}

\begin{corollary}\label{defobject}
  Let $A\in \rm{cCAlg}(\cl{C}_{R})$ be formally \'etale and $R^{\eta} \to R$ a square zero extension 
  with fiber $M$. Then the space
  \[\rm{fib}_{A}\left(\rm{cCAlg}(\cl{C}_{R^{\eta}}) \to \rm{cCAlg}(\cl{C}_{R}) \right)\]
  is contractible, i.e.~$A$ admits an essentially unique lift to a coalgebra in $\cl{C}_{R^\eta}$
\end{corollary}
\begin{proof}
Indeed, since $\cl{X}=\cCAlg(\cC_{\blank})^{\Delta^0}$ is cohesive and $L_A\simeq 0$  Proposition~\ref{maplifts} implies that
\[ T^M_{\cl{X}^{\Delta^0}_A} \simeq \rm{der}_{\id} (A, M[1]) 
= \rm{Map}_{\Sp(\rm{cCAlg}(\cl{C}_{R})_{/A})}(\Sigma^\infty_+B, L_A(M[1])) \simeq 0.\]
Hence, the claim follows from Proposition~\ref{def}.
\end{proof}

For dualizable coalgebras, our notions of formally \'etale and derivations are compatible with
the usual notions from higher algebra.

 \begin{proposition}\label{cotangentder}
   Let $R$ be a connective $\bb{E}_{\infty}$-ring and assume that $B,A\in \rm{cCAlg}(\Mod^{\cn}_{R})$
   with $A$ dualizable. For every map $\varphi:B \to A$ with $R$-linear dual
   $\varphi^\vee: A^\vee \to B^\vee$ and every $M \in \Mod^{\cn}_R$ we have
  \[ \rm{Der}_{\varphi}(B, M) 
  \simeq \rm{Map}_{\rm{Mod}_{A^\vee}}(L_{A^\vee/R}, \varphi{\pt}\rm{map}_{R}(B, M)).\]
 \end{proposition}
\begin{proof}
Write $R\p = R \oplus M$ and denote by $e: R \to R\p$ the 0-section.
  By construction, the space $\rm{Der}_{\varphi}(B, M)$ is equivalent to the fiber
  \[ F_{\varphi}:=\rm{fib}_{\varphi}\left(\rm{Map}_{\rm{cCAlg}(\Mod_{R\p})}(e^\pt B,e^\pt A^\vee)
      \to \rm{Map}_{\rm{cCAlg}(\Mod_{R})}(B, A^\vee) \right)\]
  Since $A$ is dualizable, so is $e^\pt A$ with dual given by 
  $(e^\pt A)^{\vee}\simeq A^{\vee}\otimes_{R} R\p$.
  Thus, by Corollary~\ref{dualad}, applying $(\blank)^{\vee}$ yields an equivalence
  \begin{align*}
    \rm{Map}_{\rm{cCAlg}(\Mod_{R\p})}(e^\pt B, e^\pt A) &\simeq 
    \rm{Map}_{\rm{CAlg}(\Mod_{R\p})}(A^{\vee}\otimes_{R}R\p, \rm{map}_{R\p}(e^\pt B, R\p))\\
    &\simeq \rm{Map}_{\rm{CAlg}(\Mod_{R})}(A^{\vee}, \rm{map}_{R}(B, R\p))
  \end{align*}
  and similarly
  \begin{align*}
    \rm{Map}_{\rm{cCAlg}(\Mod_{R})}(B,A)\simeq \rm{Map}_{\rm{CAlg}(\Mod_{R})}(A^{\vee}, B^{\vee}).
  \end{align*}
  Thus, $F_{\varphi}$ is given by
  \begin{align*}
    F_{\varphi}&\simeq 
    \rm{fib}_{\varphi^{\vee}}\left(\rm{Map}_{\rm{CAlg}(\Mod_{R})}(A^{\vee}, \rm{map}_{R}(B, R\p))
    \to \rm{Map}_{\rm{CAlg}(\Mod_{R})}(A^{\vee}, B^{\vee})\right)\\
    &\simeq \rm{Map}_{(\rm{CAlg}(\Mod_{R}))_{/B^\vee}}(A^{\vee}, \rm{map}_{R}(B, R\p)),
  \end{align*}
  i.e.~the space of lifts in the diagram
\[\begin{tikzcd}
	& {\rm{map}_R(B,R\p)} \\
	{A^\vee} & {B^\vee}
	\arrow[from=2-1, to=2-2]
	\arrow[from=1-2, to=2-2]
	\arrow[dashed, from=2-1, to=1-2].
\end{tikzcd}\]
Since $R\p\to R$ is a split square zero extension with fiber $M$, the map
$\rm{map}_{R}(B, R\p) \to B^{\vee}$ is a square zero extension as well with fiber
$\rm{map}_{R}(B, M)$. Hence, we have that
\begin{align*}
F_{\varphi}\simeq \rm{Map}_{A^{\vee}}(L_{A^{\vee}_{R}/R}, \varphi^{\vee}_{\pt}\rm{map}_{R}(B,M))
\end{align*}
as claimed.
\end{proof}

This provides us with the most accessible examples of formally \'etale coalgebras.

\begin{corollary}\label{dualetal}
  Let $R$ be a connective $\bb{E}_{\infty}$-ring and $A\in \rm{cCAlg}(\Mod^{\cn}_{R})$ be dualizable
  such that the relative cotangent complex $L_{A^{\vee}/R}$ vanishes. Then $A$ is formally \'etale.
\end{corollary}

\begin{example}
    For any $X\in \cl{S}^\omega$ the $\F_p$-homology $\F_p[X]$ is compact and hence dualizable
    in $\Mod_{\F_p}$ with dual given by the cohomology $\F_p^X$. By~\cite[][Proposition 2.4.12.]{dag8}
    we have $L_{\F_p^X/\F_p} \simeq 0$. Hence, by Corollary~\ref{dualetal} we see that $\F_p[X]$
    is a formally \'etale coalgebra. Crucially, this reasoning does not apply to Eilenberg-MacLane
    spaces, for which we give a more direct proof in \ref{homocoalg}, which allows
    us to drop the finiteness assumption on the space $X$.
\end{example}

\begin{remark}
 It is unclear whether the converse of Corollary~\ref{dualetal} holds.
 From Proposition~\ref{cotangentder} we can only deduce that
\[ \rm{Map}_{A^{\vee}}(L_{A^{\vee}/R}, \varphi^{\vee}_{\pt}\rm{map}_{R}(B, M)) \simeq 0\]
for each coalgebra $B$, $R$-module $M$ and morphism of algebras $\varphi:A^{\vee}\to B^{\vee}$.
\end{remark}



It is clear in the split case that maps into formally \'etale coalgebras also admit unique lifts.
To make precise what happens in the non-split case, let us first describe the space of lifts
in the general case.

\begin{proposition}\label{fibpb}
  Let $X(\blank)=(\rm{cCAlg}(\cl{C}_{\blank})^{\Delta^{1}}$, $R$ an $\bb{E}_{\infty}$-ring
  and $(A\rar{\varphi} B) \in X(R)$. Then for every connective $R$-module $M$ the fiber
  $X^{R\oplus M}_{\varphi}$ can be computed as the pullback
  \[ X^{R\oplus M}_{\varphi} \simeq \rm{Der}_{\rm{id}}(B, M[1]) \times_{\rm{Der}_{\varphi}(B, M[1])}
    \rm{Der}_{\rm{id}}(A, M[1]).\]
\end{proposition}
\begin{proof}
Since $X$ is cohesive, the space $X_{\varphi}^{R\oplus M}$ fits into a pullback diagram
\[\begin{tikzcd}
	{X_{\varphi}^{R\oplus M}} & \pt \\
	\pt & {X_{\varphi}^{R\oplus M[1]}}
	\arrow[from=1-1, to=1-2]
	\arrow[from=1-1, to=2-1]
	\arrow[from=2-1, to=2-2]
	\arrow[from=1-2, to=2-2]
\end{tikzcd}\]
where the maps $\pt \to X_{\varphi}^{R\oplus M[1]}$ are given by the base changed
morphisms $\varphi \otimes_{R} (R \oplus M[1])$. \\
If $\cl{C}$ is any category and $(\psi:A\to B)\in \cl{C}^{\Delta^{1}}$, then $\Omega_{\psi}\cl{C}^{\Delta^{1}}$ is given
by the pullback
\[\begin{tikzcd}
	{\Omega_\psi\cl{C}^{\Delta^1}} & {\rm{Aut}_{\cl{C}}(A)} \\
	{\rm{Aut}_{\cl{C}}(B)} & {\rm{Map}_{\cl{C}}(A,B)}
	\arrow["{\blank \circ \psi}"', from=2-1, to=2-2]
	\arrow["{\psi \circ \blank}", from=1-2, to=2-2]
	\arrow[from=1-1, to=2-1]
	\arrow[from=1-1, to=1-2]
\end{tikzcd}.\]
Thus, we can compute the loop space $\Omega X_{\varphi}^{R\oplus M[1]}$ as the pullback
\[\begin{tikzcd}
	{\Omega X_{\varphi}^{R\oplus M[1]}} & {\rm{Der}_{\id}(A, M[1])} \\
	{\rm{Der}_{\id}(B, M[1])} & {\rm{Der}_{\id}(B, M[1])}
	\arrow[from=2-1, to=2-2]
	\arrow[from=1-2, to=2-2]
	\arrow[from=1-1, to=2-1]
	\arrow[from=1-1, to=1-2]
\end{tikzcd}\]
as claimed.
\end{proof}

In particular, if $A$ is formally \'etale this means that the space of lifts of $\varphi$
is equivalent to the space of lifts of $B$ to an $R\oplus M$-coalgebra so we get the following.

\begin{corollary}\label{defmaps}
  Let $R$ be a connective $\bb{E}_\infty$-ring, $A,B\in \rm{cCAlg}(\cl{C}_{R})$ with $A$ formally \'etale
  and let $q:R^{\eta} \to R$ be a square zero extension with fiber $M$. Suppose we are given lifts
  $A\p$ and $B\p$ of $A$ and $B$ respectively to
  $\rm{cCAlg}(\cl{C}_{R^{\eta}})$. Then the natural map
  \[q^\pt:\rm{Map}_{\rm{cCAlg}_{R^{\eta}}}(B\p, A\p) \to \rm{Map}_{\rm{cCAlg}_{R}}(B,A)\]
  is a homotopy equivalence.
\end{corollary}
\begin{proof}
Let $X(\blank)= \rm{cCAlg}(\cl{C}_{\blank})^{\Delta^1}$. Proposition~\ref{fibpb} implies that,
since $A$ is formally \'etale, we have $T_{X_\varphi}^M \simeq 0$ for any $\varphi: B\to A$. Thus,
each fiber of the map $q^\pt$ is contractible and the claim follows.
\end{proof}

We can also characterize formally \'etale coalgebras in this way, as the following proposition shows.

\begin{proposition}\label{etalchar}
    Let $R$ be a connective $\bb{E}_{\infty}$-ring and $A\in \rm{cCAlg}(\cC_R)$. We write
    $\cl{X}(R) = \rm{cCAlg}(\cC_R)$. Then $A$ is formally \'etale if and only if for every
    $B \in \rm{cCAlg}(\cC_R), M \in \rm{Mod}_R^{cn}$ and every morphism
    $\varphi:B\to A$, the map
    \[\rm{ev}_0:T_{(\cl{X}^{\Delta^1})_\varphi}^{M} \to T_{(\cl{X}^{\Delta^0})_{B}}^{M}\]
    induced by the evaluation at the domain is an equivalence.
\end{proposition}
\begin{proof}
    Write
  \[ F_{B\p}^M:=\rm{fib}_{B\p}\left((\cl{X}^{\Delta^{1}})_{\varphi}^{R\oplus M} 
  \to (\cl{X}^{\Delta^{0}})_{B}^{R\oplus M}\right)\]
  for the fiber over some point $B\p\in (\cl{X}^{\Delta^{0}})_{B}^{R\oplus M}$. 
  Then by definition of the tangent complex, the map $T_{(\cl{X}^{\Delta^1})_\varphi}^{M} \to T_{(\cl{X}^{\Delta^0})_{B}}^{M}$ is an equivalence
  if and only if we have $F_{B\p}^{M} \simeq 0$ for all $M\in \rm{Mod}_{R}^{\rm{cn}}, B\p \in (\cl{X}^{\Delta^{0}})_{B}^{R\oplus M}$.
  The two pasted pullback squares
\[\begin{tikzcd}
	F_{B\p}^{M} & {(\cl{X}^{\Delta^1})_{\varphi}^{R \oplus M}} & {\rm{Der}_{\id}(A, C_A(M[1]))} \\
	\ast & {\rm{Der}_{\id}(B,C_B(M[1]))} & {\rm{Der}_{\varphi}(B, C_A(M[1]))}
	\arrow[from=1-1, to=2-1]
	\arrow[from=2-1, to=2-2]
	\arrow[from=1-2, to=2-2]
	\arrow[from=1-1, to=1-2]
	\arrow[from=1-2, to=1-3]
	\arrow[""{name=0, anchor=center, inner sep=0}, from=2-2, to=2-3]
	\arrow[from=1-3, to=2-3]
	\arrow["\lrcorner"{anchor=center, pos=0.125}, draw=none, from=1-1, to=2-2]
	\arrow["\lrcorner"{anchor=center, pos=0.125}, draw=none, from=1-2, to=0]
\end{tikzcd}\]
yield a fiber sequence
\[F_{M} \to \rm{Der}_{\id}(A, C_{A}(M[1])) \rar{\blank \circ \varphi }  \rm{Der}_{\varphi}(B, C_{A}(M[1])).\]
Hence, the ``only if'' direction holds. Moreover, we see that if $F_{B\p}^{M} \simeq 0$ for every
$M\in \rm{Mod}_{R}^{\rm{cn}}, B\p \in (\cl{X}^{\Delta^{0}})_{B}^{R\oplus M}$ and any morphism $B\rar{\varphi} A$, we obtain
a zigzag of equivalences
\[ \rm{Der}_{\varphi}(B, C_{A}(M[1])) \xleftarrow{\sim} \rm{Der}_{\id}(A, C_{A}(M[1]))
  \rar{\sim} \rm{Der}_{0}(0, C_{A}(M[1])) \simeq \pt,\]
where $0 \in \rm{cCAlg}_{k}$ denotes the initial coalgebra. Thus, we have that
\[\rm{Der}_{\varphi}(B, C_{A}(M))\simeq \Omega \rm{Der}_{\varphi}(B, C_{A}(M[1]))\simeq \pt\]
as claimed.
\end{proof}

\begin{corollary}\label{etallift}
  Let $R$ be a connective ring spectrum and $A\in \rm{cCAlg}(\cl{C}_{\blank})$ be formally \'etale.
  For a square zero extension $R^{\eta} \to R$ with fiber $M$ denote by $A^{\eta}$ the essentially
  unique lift of $A$ to $\cCAlg(\cC_{R^\eta})$. Then $A^{\eta}$ is also formally \'etale.
\end{corollary}
\begin{proof}
  Let $B \to A^{\eta}$ be any map of $R^{\eta}$ coalgebras and write 
  $\cl{X}(\blank)= \rm{cCAlg}^{\rm{cn}}_{\blank}$.
  Then for any $N\in \rm{Mod}_{R^{\eta}}^{\rm{cn}}$ we need to show that the induced map
  \[ T^{N}_{\cl{X}^{\Delta^{1}}_{\varphi}} \rar{\sim} T^{N}_{\cl{X}^{\Delta^{0}}_{B}}\]
  is an equivalence. Arguing as in the proof of Proposition~\ref{cofib}, we see that $N$
  sits in a cofiber sequence
  \[ M\otimes_{R}(R \otimes_{R^{\eta}} N) \to N \to R \otimes_{R^{\eta}}N, \]
  where the $R^{\eta}$-action on the outer terms factors through $R$.
  Thus, writing $B\p \simeq B\otimes_{R^{\eta}} R$ and $\varphi\p = \varphi_{R^{\eta}}: B\p \to A$
  and using that the tangent complex functors are excisive, we obtain a commutative diagram
\[\begin{tikzcd}
	{T_{\cl{X}^{\Delta^1}_{\varphi\p}}^{M\otimes_{R}(R\otimes_{R^\eta} N)}} & {T^{M\otimes_{R}(R\otimes_{R^\eta} N)}_{\cl{X}^{\Delta^1}_\varphi}} & {T^N_{\cl{X}^{\Delta^1}_\varphi}} & {T^{R\otimes_{R^\eta} N}_{\cl{X}^{\Delta^1}_\varphi}} & {T_{\cl{X}^{\Delta^1}_{\varphi\p}}^{R\otimes_{R^\eta} N}} \\
	{T_{\cl{X}^{\Delta^0}_{B\p}}^{M\otimes_{R}(R\otimes_{R^\eta} N)}} & {T_{\cl{X}^{\Delta^0}_B}^{M\otimes_{R}(R\otimes_{R^\eta} N)}} & {T^N_{\cl{X}^{\Delta^0}_B}} & {T^{R\otimes_{R^\eta} N}_{\cl{X}^{\Delta^0}_B}} & {T_{\cl{X}^{\Delta^0}_{B\p}}^{R\otimes_{R^\eta} N}}
	\arrow[from=1-2, to=1-3]
	\arrow[from=1-3, to=1-4]
	\arrow[from=1-2, to=2-2]
	\arrow[from=2-2, to=2-3]
	\arrow[from=2-3, to=2-4]
	\arrow[from=1-4, to=2-4]
	\arrow[from=1-3, to=2-3]
	\arrow["\sim", from=1-4, to=1-5]
	\arrow["\sim", from=2-4, to=2-5]
	\arrow["\sim", from=1-5, to=2-5]
	\arrow["\sim"', from=1-2, to=1-1]
	\arrow["\sim", from=1-1, to=2-1]
	\arrow["\sim"', from=2-2, to=2-1]
\end{tikzcd}\]
where the inner two horizontal maps in each row form a cofiber sequence. 
The outer horizontal maps are the base change equivalences from Proposition~\ref{bc} and the outer
vertical maps are equivalences since by assumption $A= A^{\eta}\otimes_{R^{\eta}}R$
is formally \'etale. Thus, the middle map 
$T^{N}_{{\cl{X}^{\Delta^{1}}_{\varphi}}}\to T^{N}_{\cl{X}^{\Delta^{0}}_{B}}$
is an equivalence as well, so by Proposition~\ref{etalchar} the $R^{\eta}$-coalgebra $A^{\eta}$
is formally \'etale.
\end{proof}

We can neatly organize the results of this section into the following proposition.

\begin{proposition}\label{weillift}
    Let $R$ be a connective $\bb{E}_\infty$-ring and $q: R^\eta \to R$ be a square zero extension. 
    Weil-restriction along $q$ induces a fully faithful functor
    \[ q_\pt : \cCAlg(\cl{C}_R)^{\rm{f\acute{e}t}} \to \cCAlg(\cl{C}_{R^\eta})^{\rm{f\acute{e}t}}\]
    Moreover, for any $A\in \cCAlg(\cl{C}_R)^{\rm{f\acute{e}t}}$ the coalgebra $q_\pt A$ is, up to contractible
    choice, the unique lift of $A$ to $\cCAlg(\cl{C}_{R^\eta})$.
\end{proposition}
\begin{proof}
By Corollary~\ref{defobject}, there exists a unique lift $A\p \in \cCAlg(\cl{C}_{R^\eta})$ which is formally 
\'etale by Corollary~\ref{etallift}. Moreover, by Corollary~\ref{defmaps}, we have a natural equivalence 
\[ \Map_{\cCAlg(\cl{C}_{R^\eta})}(B, A\p) \rar{\sim} \Map_{\cCAlg(\cl{C}_R)}(p^\pt A, A)\]
and so $A\p = q_\pt A$. Since $A\p$ was a lift of $A$, the counit $q^\pt q_\pt A \rar{\sim} A$ is an equivalence
which implies that the restriction of $q_\pt$ is fully faithful as claimed.
\end{proof}

We can also iterate lifting along square zero extensions into a limit. This will be crucial in
the next section for going from $\F_p$ to the $p$-completed sphere $\S_p^\wedge$.

\begin{proposition}\label{limlift}
    Suppose we are given a diagram 
    \[ \dots \to R_2 \rar{q^{21}} R_1 \rar{q^{10}} R_0\]
    in $\CAlg^{\cn}$ with limit $R= \lim_n R$ such that each map $R_{n+1} \to R_{n}$ is a square
    zero extension and the functor $\cl{C}_R \to \lim_n \cl{C}_{R_n}$ is an equivalence. Write
    $q:R\to R_0$ for the natural map. Weil restriction along $q$ restricts to a fully faithful functor
    \[ q_\pt: \cCAlg(\cl{C}_{R_0})^{\fet} \to \cCAlg(\cl{C}_{R}),\]
    such that for any $A\in \cCAlg(\cl{C}_{R_0})^{\fet}$ the coalgebra $q_\pt A$ is, up to contractible choice,
    the unique lift of $A$ to an object of $\cCAlg(\cl{C}_{R})$.
\end{proposition}
\begin{proof}
First observe that by Lemma~\ref{limits} the map
\[ \cCAlg(\cl{C}_R) \to \lim \cCAlg(\cl{C}_{R_n})\]
is also an equivalence. Denote by $q^n$ the map $R_n \to R$. 
Then $q_\pt: \cCAlg(\cl{C}_{R_0})\to \cCAlg(\cl{C}_{R})$ is given
   by the limit of the functors
   \[ q^n_\pt:\cCAlg(\cl{C}_{R_0}) \to \cCAlg(\cl{C}_{R_n}). \]
   We have that $q^n_\pt \simeq q^{10}_\pt \circ q^{21}_\pt \circ \dots \circ q^{n(n-1)}_\pt$ and for each 
   $k$ the restriction
   \[ q^{k(k-1)}_\pt :\cCAlg(\cl{C}_{R_{k-1}})^{\fet} \to \cCAlg(\cl{C}_{R_k})^{\fet}\]
   is fully faithful by Proposition~\ref{weillift}. 
   Thus, $q^n_\pt$ is fully faithful when restricted to the formally \'etale coalgebras. 
   Moreover, since mapping spaces in a limit of categories are given by the limit of mapping spaces, a 
   limit of fully faithful functors is fully faithful and so $q_\pt$ is fully faithful when restricted
   to formally \'etale coalgebras. The fact that $q_\pt A$ is the unique lift of $A$ is immediate
   by Proposition~\ref{weillift} and the fact that fibers commute with limits.
\end{proof}