\subsection{Lifting to the spherical Witt vectors}
Fix a prime $p$. We now collect the fruits of our labor and apply the machinery of formally 
\'etale coalgebras to the moduli problem
\[ \Mod_\blank^\wedge: \CAlg^{\cn} \to \CAlg(\rm{Pr^L}),\]
where $\Mod_R^\wedge$ denotes the category of $p$-complete $R$-modules. This is a legal move by 
Corollary~\ref{pnil}. Throughout this section we fix a perfect $\F_p$-algebra $k$ and 
denote the spherical Witt vectors by $\bb{W}(k)$ and the $p$-typical Witt Vectors as
$\bb{W}_0(k)= \pi_0\bb{W}(k)$.
Before we can state our main theorems, we require one more technical result about $p$-adic lifts
of formally \'etale coalgebras.

\begin{proposition}\label{etalliftzp}
 Let $A \in \cCAlg(\Mod_{\W_0(k)}^{\cn,\wedge})$ 
  Then $A$ is formally \'etale if $A_{1}:=A \otimes k \in \rm{cCAlg}(\Mod_k^{\cn})$
  is formally \'etale.
\end{proposition}
\begin{proof}
   Write $\cl{X}= (\rm{cCAlg}_{\blank}^{\rm{cn}})^{\wedge}_{p}$
   and let $\varphi: B \to A$ be a point in $\cl{X}(\W_0(k))^{\Delta^{1}}$.
   Denote for each $n\in \Z_{\geq 1}$ by $\varphi_{n}: B_{n}\to A_{n}$ the base change of 
   $\varphi$ to $\W_0(k)/p^{n}$. By Proposition~\ref{etalchar} it suffices to show that for
   every $\W_0(k)_{p}$-module $M$ the natural map
   \[T_{(\cl{X}^{\Delta^1})_\varphi}^{M} \to T_{(\cl{X}^{\Delta^0})_{B}}^{M}\]
  is an equivalence. This map is induced by the maps of spaces
  \[ (\cl{X}^{\Delta^{1}})_{\varphi}^{\W_0(k)\oplus M [k]} 
  \to (\cl{X}^{\Delta^{0}})_{B}^{\W_0(R) \oplus M[k]}\]
  and hence it suffices to show that these are equivalences. Since fibers commute with limits
  we have for each $i \in \Z_{>0}$, $\eta \in \cl{X}^{\Delta^{i}}(\W_0(R)$ with base change
  $\eta_{n}$ to $\W_0(R)/p^{n}$ and each $\W_0(R)$-module $N$ an equivalence
  \[(\cl{X}^{\Delta^{i}})_{\eta}^{\W_0(R)\oplus N} 
  \simeq \lim_{n} (\cl{X}^{\Delta^{i}})_{\eta_{n}}^{\W_0(R)/p^{n}\oplus N/p^{n}}.\]
  As each $A_{n}$ is a lift of $A_{1}$ to $\W_0(R)/p^{n}$ and hence by Corollary~\ref{etallift}
  formally \'etale, the map
  \[ (\cl{X}^{\Delta^{1}})_{\varphi_{n}}^{\W_0(R)/p^{n} \oplus M/p^{n}} 
  \to (\cl{X}^{\Delta^{1}})_{B_{n}}^{\W_0(R)/p^{n} \oplus M/p^{n}}\]
  is an equivalence for all $n$ and the claim follows.
\end{proof}

We now summarize our results in our first main theorem, namely that the right adjoint to base change
defines a $p$-typical Witt vector style functor for formally \'etale connective coalgebras over $\F_p$.

\begin{theorem}\label{wittzp}
    Let $q: \W_0(k) \to k$ be the augmentation. Connective Weil-restriction along $q$ induces a
    fully faithful functor
    \[\cl{W}_0: \rm{cCAlg}(\Mod_k^{\cn})^{\fet} \to \rm{cCAlg}(\Mod_{\W_0(k)}^{\cn,\wedge})^{\fet}.\]
    Moreover, for any $A \in \rm{cCAlg}(\Mod_{k}^{\rm{cn}})^{\fet}$ the coalgebra $\cl{W}_0(A)$
    is, up to contractible choice, the unique lift of $A$ to an object
    of $\rm{cCAlg}(\Mod_{\W_0(k)}^{\wedge})$.
\end{theorem}
\begin{proof}
  By Proposition~\ref{pcomp} we may apply Proposition~\ref{limlift} to the functor 
  \[\CAlg^{\cn}\to \CAlg(\rm{Pr^L}) \quad R \mapsto \Mod_R^{\cn,\wedge}\]
  and the $p$-completion tower 
 \[ \dots \to \W_0(k)/p^2 \to \W_0(k)/p = k\]
  to get fully faithfulness and the uniqueness of the lifts.
  Then Proposition~\ref{etalliftzp} tells us that the lift is again formally \'etale and we are done.
\end{proof}

\begin{proposition}
    Let $A \in (\rm{cCAlg}_{W(k)}^{\rm{cn}})^\wedge_p$, then $A$ is formally \'etale if
    $A\otimes W_0(k)$ is.
\end{proposition}
\begin{proof}
    Argue exactly as in Proposition~\ref{etalliftzp}.
\end{proof}

With this in hand, we obtain a spherical lift of Theorem~\ref{wittzp}.

\begin{theorem}\label{wittsp}
    Let $q: \W(k) \to k$ be the augmentation. Connective Weil-restriction along $q$ induces a fully
    faithful functor
    \[\cl{W}: \rm{cCAlg}(\Mod_k^{\cn})^{\fet} \to \rm{cCAlg}(\Mod_{\W(k)}^{\cn,\wedge})^{\fet}.\]
    Moreover, for any $A \in \rm{cCAlg}(\Mod_k^{\rm{cn}})^{\fet}$ 
    the coalgebra $\cl{W}(A)$ is, up to contractible choice, the unique lift of $A$ to an object 
    of $\rm{cCAlg}(\Mod_{\W(k)}^{\wedge})$.
\end{theorem}
\begin{proof}
By Theorem~\ref{wittzp} it suffices to show that Weil restriction along
$\W(k)\to \W_0(k)=\pi_0\W(k)$ is fully faithful on formally \'etale coalgebras. By Lemma~\ref{pnil} we can again apply Proposition~\ref{limlift} to the
functor 
  \[\CAlg^{\cn}\to \CAlg(\rm{Pr^L}) \quad R \mapsto \Mod_R^{\cn,\wedge}\]
  and the Postnikov tower
  \[ \dots \to \tau_{\leq 2} \W(k) \to \tau_{\leq 1}\W(k) \to \tau_{\leq 0}\W(k) = \W_0(k)\]
  and win.
\end{proof}

\subsection{Homology coalgebras}\label{homocoalg}

As observed in Example~\ref{homology}, for every space $X$ and every $\bb{E}_{\infty}$-ring $R$, the
$R$-homology $R[X]$ carries a natural $R$-coalgebra structure induced from the diagonal
map $X\to X\times X$. The main result of this section is Theorem~\ref{sepet}, namely that if
$R$ is a connective $\F_p$-algebra and $X$ is a connected space, the coalgebra $R[X]$ is
formally \'etale. Our proof relies on a recent computation of Bachmann\textendash Burklund in \cite{bb}.
Moreover, combining Theorem~\ref{sepet} and Theorem~\ref{wittsp} with the main result of
loc.cit.~yields a further embedding of $p$-complete nilpotent spaces into coalgebras in 
$p$-complete $\bb{W}(\overline{\F}_p)$-modules.
In the following, let $k$ denote a field of characteristic $p$ unless stated otherwise.\\

We begin by quickly recalling some facts about pro-objects, see loc.cit.~Section 5 for a more 
in-depth discussion. Let $\cl{C}$ be a category which admits finite limits. Recall that,
the category $\rm{Pro}(\cl{C})$ is defined as the full subcategory of 
$\rm{Fun}(\cl{C}, \cl{S})\op$ spanned by those functors which preserve finite limits.
We think of objects of $\Pro(\cC)$ as formal cofiltered limits of objects of $\cC$. 
If $\cC$ admits all small limits, the restricted Yoneda embedding gives an adjunction
\[\begin{tikzcd}
	{c: \cC} & {\Pro(\cC): M}
	\arrow[""{name=0, anchor=center, inner sep=0}, shift left=2, from=1-1, to=1-2]
	\arrow[""{name=1, anchor=center, inner sep=0}, shift left=2, from=1-2, to=1-1]
	\arrow["\dashv"{anchor=center, rotate=-90}, draw=none, from=0, to=1].
\end{tikzcd}\]
Here we think of $c$ (aka restricted Yoneda) as taking $X\in \cC$ to the constant diagram on $X$.
The functor $M$ is called materialization and takes a formal limit $\flim_\lambda X_\lambda$ to 
the actual limit computed in $\cC$. An object $ \flim X_\lambda \in \rm{Pro}(\Mod_R)$ is called \textit{pro-truncated} if each $X_\lambda$ is
bounded above. The inclusion of pro-truncated objects admits a left adjoint denoted $\tau_{<\infty}$,
which can explicitly described by the formula
\[ \tau_{<\infty} \lim_{\lambda} X_\lambda = \lim_{n, \lambda} \tau_{\leq n} X_{\lambda}. \]
An object $E \in \rm{Pro}(\Mod_R)$ is called \textit{pro-constant up to pro-truncation} if the natural
map
\[ \tau_{<\infty} c M(E) \to \tau_{<\infty} E\]
is an equivalence in $\rm{Pro}(\Mod_R)$. If $\cC$ is symmetric monoidal, then $\rm{Pro}(\cC)$
inherits a natural symmetric monoidal structure via the pointwise tensor product such that the
inclusion $\cC \to \rm{Pro}(\cC)$ is monoidal. We denote the partially defined right adjoint
to $c: \cCAlg(\cC) \to \cCAlg(\Pro(\cC))$ by $M^{cA}$ and refer to it as
\textit{coalgebraic materialization}. Moreover, we write 
\[ \widehat{C}: \Pro(\cC) \to \cCAlg(\Pro(\cC))\]
for the pro-version of the cofree coalgebra functor. Observe that, for any $X\in \cC$ 
the coalgebraic materialization $M^{cA}\widehat{C}(cX)\in \cCAlg(\cC)$ exists and is given by
$C(X)$. The first key insight is that, although $\Pro(\cC)$ is usually not presentable even if $\cC$ is, 
the category $\cCAlg(\Pro(\cC))$ is better behaved with respect to limits.

\begin{lemma}\label{probc1}
    Let $f:R\to S$ be a map of $\bb{E}_\infty$-rings. The base change 
    \[ f^\pt:\rm{cCAlg}(\rm{Pro}(\Mod_R)) \to \cCAlg(\rm{Pro}(\Mod_S)) \]
    preserves all limits, cofree coalgebras and constant objects.
\end{lemma}
\begin{proof}
    The base change $\Mod_R \to \Mod_S$ commutes with finite limits and is accessible. Thus,
    the induced functor $f^\pt:\rm{Pro}(\Mod_R) \to \rm{Pro}(\Mod_S)$ commutes with all limits. 
    In particular for any $ M\in \rm{Pro}(\Mod_R)$ the cofree coalgebra on $M$ is computed 
    by the formula
    \[ \hat{C}_R(M) \simeq \prod_n (M^{\otimes n})^{h\Sigma_n} \in \cCAlg(\Pro(\Mod_R)),\]
    and thus, since $f^\pt$ commutes with tensor products and limits we get
    \[ f^\pt \hat{C}_R(M) = \hat{C}_S(f^\pt M)\]
    as claimed. Finally, the fact that $f^\pt$ preserves constant objects is clear,
    since it is induced by $\rm{Pro}(\blank)$.
\end{proof}

\begin{lemma}\label{probc2}
   Let $f:R\to S$ be a map of connective $\bb{E}_\infty$-rings and let $E$ be a connective
   $R$-module. Then the base change $f^\pt(\tau_{<\infty}cE) \in \rm{Pro}(\Mod_S)$ is
   pro-constant up to pro-truncation.
\end{lemma}
\begin{proof}
We need to show that the natural map
\[ \tau_{<\infty} c M(f^\pt(\tau_{<\infty}cE )) \to \tau_{<\infty}(f^\pt\tau_{<\infty}cE)\]
is an equivalence. Unraveling the definitions, we see that
 \[\tau_{<\infty} c M(f^\pt(\tau_{<\infty}cE ))
 = \tau_{<\infty}c(\lim_n ((\tau_{\leq n} M) \otimes_R S))
 \simeq \tau_{<\infty} c(E\otimes_R S) \simeq \flim_i c(\tau_{\leq i}(E\otimes_R S)),\]
 where we have used that, since everything is connective, the natural map
 \[  E \otimes_R S \rar{\sim} \lim_n\tau_{\leq n} E \otimes_R S  \]
 is an equivalence. Conversely, we have that 
 \[\tau_{<\infty}(f^\pt\tau_{<\infty}cE)
 \simeq \flim_{k,n} c(\tau_{\leq k}((\tau_{\leq n} E)\otimes_R S)).\]
 Again using the connectivity of everything in sight, we can find a cofinal system 
 of pairs $(n,k)$ with $n>>k$ such that
 \[ \tau_{\leq k}((\tau_{\leq n}E) \otimes_R S) \simeq \tau_{\leq k} (E \otimes_R S),\]
and so the claim follows. 
\end{proof}

The second key insight of is the \textit{Arint-Schreier map} on the cofree coalgebra. We quickly
recall its construction and generalize it slightly.

\begin{construction}
The functor
\[ (\blank)^\vee=\rm{map}_k(\blank , k): \cCAlg(\Mod_k) \to \Mod_k\]
is represented by the spectrum object $\{C_k(\Sigma^n k)\}$. Hence, on any $A\in \cCAlg(\Mod_k)$
the power operation 
\[ Q_0 - 1: A^\vee \to A^\vee\]
induces a map
\[ F-1: C_k(\Sigma^n k) \to C_k(\Sigma^nk)\]
called the \textit{Artin\textendash Schreier} map. In \cite[][Construction 7.4]{bb},
for every pointed set $X$ and any $n\geq 0$ a pro Artin\textendash Schreier map
\[ \widehat{F}-1 :\widehat{C}( c \Sigma^n k\{X\}) \to \widehat{C}(c\Sigma^n k\{X\})\]
in $\cCAlg(\rm{Pro}(\Mod_k))$ is defined, which is natural in inert maps of pointed sets and 
recovers the ordinary Artin\textendash Schreier map upon coalgebraic materialization. 
Now let $S$ be a $k$-algebra.
Since pro-base change commutes with cofree coalgebras, we can base change
$\widehat{F}-1$ along any map of $\E_\infty$-rings $k \to R$ to obtain a map
\[ \widehat{F}_S-1: \widehat{C}_S(c \Sigma^n S\{X\}) \to \widehat{C}_S(c\Sigma^n S\{X\}).\]
We define the ($S$-linear) Artin\textendash Schreier map 
\[ F_S-1: C_S(\Sigma^n S\{X\}) \to C_S(\Sigma^n S\{X\})\]
as the materialization of $\widehat{F}_S-1$. 
\end{construction}

\begin{remark}\label{frobres}
Let $f:R\to S$ be a map of $k$-algebras. We can Weil restrict along $f$ to obtain a modified
Artin\textendash Schreier map 
\[ f_!(F_S-1): C_R(\Sigma^n S\{X\}) \to C_R(\Sigma^n S\{X\}).\]
Tracing through the adjuncions, we see that for $X=\pt$, this represents the operation $Q_0-1$
acting on $\rm{map}_k(A,S) \in \CAlg(\Mod_k)$.
\end{remark}


\begin{proposition}\label{bbpb}
    Let $k$ be a field of characteristic $p$ and $R \in \rm{CAlg}(\rm{Mod}_k)^\rm{cn}$.
    For every $X\in \rm{Set}_\pt$ and $n\geq 0$, the Artin\textendash Schreier map $F_R$ induces a pullback
    diagram in $\rm{cCAlg}(\rm{Mod}_R)$ of the form
   \[\begin{tikzcd}
	{R[\Omega^\infty\Sigma^n \F_p\{X\}]} & {C_R(\Sigma^n R\{X\})} \\
	R & {C_R(\Sigma^n R\{X\})}
	\arrow[from=1-1, to=1-2]
	\arrow["{F_{R}-1}", from=1-2, to=2-2]
	\arrow[from=1-1, to=2-1]
	\arrow[', from=2-1, to=2-2].
\end{tikzcd}\] 
\end{proposition}
\begin{proof}
The proof of~\cite[][Proposition 7.10]{bb} implies that we have a pullback diagram in 
$\cCAlg(\rm{Pro}(\Mod_k))$ of the form
\[\begin{tikzcd}
	{\tau_{<\infty}ck[\Omega^\infty\Sigma^n\F_p\{X\}]} & {\widehat{C}_k(c\Sigma^n k\{X\})} \\
	ck & {\widehat{C}_k(c\Sigma^n k\{X\})}
	\arrow[from=1-1, to=1-2]
	\arrow[from=1-1, to=2-1]
	\arrow[from=2-1, to=2-2]
	\arrow["{\widehat{F}-1}",from=1-2, to=2-2].
\end{tikzcd}\]
By Lemma~\ref{probc1} base change along the unit $k \to R$
yields a pullback square in $\cCAlg(\rm{Pro}(\Mod_R))$ of the form
\begin{equation}\label{pullback}
\begin{tikzcd}
	{\tau_{<\infty}ck[\Omega^\infty\Sigma^n\F_p\{X\}] \otimes_k R}
    & {\widehat{C}_R(c\Sigma^n R \{X\})} \\
	cR & {\widehat{C}_R(c\Sigma^n R\{X\})}
	\arrow[from=1-1, to=1-2]
	\arrow[from=1-1, to=2-1]
	\arrow[from=2-1, to=2-2]
	\arrow["{\widehat{F}_R-1}",from=1-2, to=2-2].
\end{tikzcd}
\end{equation}
Since $k$ and $R$ are connective, Lemma~\ref{probc2} implies that 
\[\tau_{<\infty} ck[\Omega^\infty\Sigma^n\F_p\{X\}] \otimes_k R \]
is pro-constant up to pro-truncation.
Thus, by~\cite[][Lemma 5.10]{bb} the coalgebraic materialization agrees with
the underlying materialization and we see that
\[ M^{\rm{cA}}(\tau_{<\infty} (ck[\Omega^\infty \Sigma^n \F_p\{X\}] \otimes_k R)
\simeq  \lim_n ( (\tau_{\leq n}k[\Omega^\infty \Sigma^n\F_p\{X\}]) \otimes_k R)
\simeq R[\Omega^\infty \Sigma^n \F_p\{X\}],\]
where the limit is computed underlying by~\cite[][Lemma 5.11]{bb} and we again
use that all terms are connective. Since $M^{\rm{cA}}$
is a right adjoint it preserves limits and it is easy to verify from the universal property
that $M^{\rm{cA}}(\widehat{C}_R(c\Sigma^n R\{X\})$ exists and is given by $ C_R(\Sigma^n R\{X\})$.
Thus, applying $M^{\rm{cA}}$ to the pullback~\eqref{pullback} we get our claim.
\end{proof}


\begin{remark}\label{Q0}
Let $R \in \CAlg(\Mod_k), M \in \Mod_R $ and $p:R\oplus M \to R$ be the split square
zero extension of $R$ with fiber $M$. Denote by $e: R \to R \oplus M$ the 0-section.
As observed in Remark~\ref{cohsq0}, the multiplication map
\[
(M^{\otimes p})_{h\Sigma_p} \to M
\]
is nullhomotopic. Hence, the power operation $Q_0:R\oplus M \to R\oplus M$ is nullhomotopic
when restricted to $M$ and hence factors as
\[ R\oplus M \rar{p} R \rar{Q_0} R \rar{e} R\oplus M.\]
This enables us to understand the Weil-restricted Artin\textendash Schreier map of Remark~\ref{frobres}.
\end{remark}

\begin{proposition}\label{split}
Let $R\in \rm{CAlg}(\Mod_k^{\rm{cn}}), M \in \Mod_R^{\rm{cn}}$, $X\in \rm{Set}_\pt$
and denote by $e:R \to R \oplus M$ the 0-section of the split square zero extension.
The (non-connective) Weil-restriction of the map
$F_{R\oplus M}: C_{R\oplus M}(e^\pt R\{X\})\to C_{R\oplus M}(e^\pt R\{X\})$ 
along $e$ yields a map 
    \[ e_!(F_{R\oplus M}-1): C_R(e_\pt e^\pt R\{X\})\rar{}
    C_R(e_\pt e^\pt R\{X\}).\]
which naturally splits in $\cCAlg(\Mod_R)$ as
    \[ (F_R-1) \otimes_R \id: C_R( R\{X\} ) \otimes_R C_R(M\{X\}) 
    \rar{} C_R(R\{X\}) \otimes_R C_R( M\{X\}).\]
\end{proposition}
\begin{proof}
First note that, since $e_\pt e^\pt R\{X\} \simeq R\{X\} \times M \{X\}$ and the cofree 
coalgebra preserves limits, we obtain
\[ C_R(e_\pt e^\pt R\{X\}) \simeq C_R( R\{X\} ) \otimes_R C_R(M\{X\}).\]
To show that $e_!(F_R -1)$ is given by the product $(F_R-1) \otimes \id $, it suffices to check 
this after applying $\Map_{\cCAlg(\Mod_R)}(A, \blank)$ for some arbitrary $A \in \cCAlg(\Mod_R)$.
Since the functor $\rm{Map}_{\Mod_R}(A, \blank)$ commutes with limits, we see that for each
$n\geq 0$ we have 
\[ \Map_{\Mod_R}(A, (R\oplus M)\{S^n\}) \simeq
\Omega^{\infty-n} ( R^A \oplus M^A)\]
where $R^A \oplus M^A$ is the split square zero extension of $R^A =A^\vee$ with fiber the $R$-linear
mapping spectrum $M^A = \rm{map}_R(A, M)$. Thus, for $X=\pt$  the claim is 
immediate from Remark~\ref{Q0}. Observe that, by naturality of the Artin\textendash Schreier map, the map
\[(R\oplus M)\{X\} \to \prod_X(R\oplus M)\] 
induces a commutative diagram
\[\begin{tikzcd}
	{\Map_{\Mod_R}(A, (R\oplus M)\{X\})}  & {\prod_X \Map_{\Mod_R}(A, R\oplus M) }  \\
	{\Map_{\Mod_R}(A, (R\oplus M)\{X\})} & {\prod_X \Map_{\Mod_R}(A, R\oplus M)}
	\arrow[from=1-1, to=2-1]
	\arrow[from=1-1, to=1-2]
	\arrow["{Q_0-\id}", from=1-2, to=2-2]
	\arrow[from=2-1, to=2-2],
\end{tikzcd}\]
which we may further factor through
\[\begin{tikzcd}
	{\Map_{\Mod_R}(A, (R\oplus M)\{X\})} & {\Map_{\Mod_R}(A,(\prod_Xk)\otimes_k (R\oplus M))} \\
	{\Map_{\Mod_R}(A, (R\oplus M)\{X\})} & {\Map_{\Mod_R}(A, (\prod_X k) \otimes_K (R\oplus M))}
	\arrow[from=1-1, to=1-2]
	\arrow[from=1-1, to=2-1]
	\arrow["{Q_0-\id}", from=1-2, to=2-2]
	\arrow[from=2-1, to=2-2].
\end{tikzcd}\]
The right hand vertical map splits as desired by Lemma~\ref{Q0} since 
$(\prod_X k)\otimes_k(R\oplus M)$ is naturally a ring. Moreover, since $k$ is a field, 
we can choose a retraction $\prod_X k \to k\{X\}$, which exhibits the left hand vertical map
as a retract of the right hand one. Hence, the left hand map splits as well, proving the claim.
\end{proof}

\begin{proposition}\label{emfet}
    Let $k$ be a field of characteristic $p$, $R\in \CAlg(\Mod_k^{\rm{cn}})$ 
    and let $V \in \Mod_{k}^{\heartsuit}$ be a discrete $k$-vector space. The $R$-homology coalgebra
    $R[\Omega^\infty \Sigma^n V] \in \rm{cCAlg}(\rm{Mod}^{\cn}_k)$ is formally \'etale.
\end{proposition}
\begin{proof}
Denote by $e:R\to \Sigma^n R$ the zero section of the split square zero extension for some 
$n\geq 0$. Upon choosing
a basis for $V$, i.e.~picking a set $X$ and an isomorphism $k\{X\}\rar{\sim}V$, applying the 
(non-connective) Weil restriction $e_!$ to the pullback of Proposition~\ref{bbpb} yields a pullback
diagram in $\cCAlg(\Mod_R)$
\[\begin{tikzcd}
	{e_{!}e^{\pt}R[\Omega^\infty\Sigma^n V]} & {C_R( e_\pt e^\pt R \otimes \Sigma^n V)} \\
	R & {C_R(e_\pt e^\pt R \otimes \Sigma^nV)} 
	\arrow[from=1-1, to=1-2]
	\arrow[from=1-1, to=2-1]
	\arrow["{{e_!(F_{R\oplus M}-1)}}", from=1-2, to=2-2]
	\arrow[from=2-1, to=2-2].
 \end{tikzcd}\]
By Corollary~\ref{split} the square zero terms splits off on the right hand vertical map and we obtain a pullback
\[\begin{tikzcd}
	{e_{!}e^{\pt}R[\Omega^\infty\Sigma^nV]} & {C_R( R\otimes \Sigma^nV)} \\
	R & {C_R( R \otimes \Sigma^nV)}
	\arrow[from=1-1, to=1-2]
	\arrow[from=1-1, to=2-1]
	\arrow["{F_R-1}", from=1-2, to=2-2]
	\arrow[from=2-1, to=2-2].
\end{tikzcd}\]
However, by Proposition~\ref{bbpb} this pullback is given by $R[\Omega^\infty \Sigma^n V]$. Since
$e_!e^\pt R[\Sigma^n V]= R[\Sigma^n V]$ is connective, it follows from our observations 
in Construction~\ref{radj} that in fact
\[ R[\Sigma^n V] \simeq e_\pt e^\pt R[\Sigma^n V] \simeq e_! e^\pt R[\Sigma^n V].\]
In particular, the natural maps
\[ R[\Omega^\infty\Sigma^n V] \rar{\varepsilon} e_\pt e^\pt R[\Omega^\infty\Sigma^n V]
= \Omega^\infty_{R[\Omega^\infty\Sigma^n V]}M \rar{\pi} R[\Omega^\infty\Sigma^n V] \]
are equivalences and so $R[\Omega^\infty \Sigma^n V]$ is formally \'etale.
\end{proof}

We now expand from Eilenberg\textendash MacLane spaces to arbitrary connected spaces. 
Let us recall the following notions in unstable homotopy theory.

\begin{definition}
 A space $X$ is called \textit{$p$-complete} if it is local with respect to the functor
$\F_p[\blank]:\cl{S}\to \Mod_{\F_p}$. Denote by $\cl{S}_p\subseteq \cl{S}$ the full 
subcategory category of $p$-complete spaces. Moreover, call a space \textit{nilpotent}
if is connected and $\pi_1$ is a nilpotent group which acts nilpotently on the higher homotopy
groups. Write $\cl{S}^{\rm{nilp}}$ for the category of complete nilpotent spaces and
$\cl{S}_p^{\rm{nilp}}= \cl{S}_p \cap \cl{S}^\rm{nilp}$ for the category of $p$-complete
nilpotent spaces.
\end{definition}

\begin{lemma}\label{genem}
    Let $X$ be a connected space and $R$ a connective $\F_p$-algebra.
    Then $R[\Omega^\infty \F_p\{X\}] \in \cCAlg(\Mod_R^{\cn})$ is given by a limit
    $\lim R[X_i]$ where each $X_i$ is a finite product of Eilenberg\textendash MacLane spaces.
\end{lemma}
\begin{proof}
Since any $\F_p$-module is free, we can write 
\[
Y:=\Omega^\infty \F_p\{X\} \simeq \prod_{i\geq 0} \Omega^\infty \Sigma^i V_i
\]
where $V_i = \pi_i\F_p\{X\} \in \Mod_{\F_p}^{\heartsuit}$. Setting
\[
Y_n = \tau_{\leq n}Y = \prod_{i\leq n} \Omega^\infty \Sigma^i V_i 
\]
we have $Y \simeq \flim Y_n$. Since $R[\blank]$ commutes with Postnikov towers we get an equivalence
\[
cR[Y] \simeq \flim R[Y_n] \in \cCAlg(\rm{Pro}(\Mod_R)).
\]
moreover, since $\tau_{\leq k } R[Y] \simeq \tau_\leq R[Y_k]$ this diagram is pro-constant up 
to pro-truncation and hence we get 
\[
R[Y] \simeq M^{cA}(\flim R[Y_n]) = \flim R[Y_n] \in \cCAlg(\Mod_R)
\]
as claimed.
\end{proof}


\begin{lemma}\label{emhom}
    Let $X$ be a space and $R$ an $\bb{E}_\infty$-ring. There exists a cofiltered diagram
    of nilpotent spaces $\{X_{\alpha}\}$, such that the natural map
    \[ R[X] \to \flim_{\alpha} R[X_{\alpha}]\]
    is an equivalence in $\cCAlg(\Mod_R)$.
\end{lemma}
\begin{proof}
   Since finite limits of nilpotent spaces are nilpotent,~\cite[][Remark 3.1.7]{dag8}
   implies that the inclusion $\iota: \cl{S}^{\rm{nilp}} \to \cl{S}$ induces an adjunction
   \[\begin{tikzcd}
	{(\widehat{\blank}): \rm{Pro}(\cl{S})} & {\rm{Pro}(\cl{S}^{\rm{nilp}}):{\iota}}
	\arrow[""{name=0, anchor=center, inner sep=0}, shift left=2, from=1-2, to=1-1]
	\arrow[""{name=1, anchor=center, inner sep=0}, shift left=2, from=1-1, to=1-2]
	\arrow["\dashv"{anchor=center, rotate=-90}, draw=none, from=1, to=0].
\end{tikzcd}\]
  The left adjoint is given by taking a space $X$ to $ \widehat{X}=\flim_{X_{\alpha} \to X} X_{\alpha}$
  where the cofiltered limit runs over all maps $X_{\alpha} \to X$ where $X_{\alpha}$ is nilpotent. 
 Moreover, the homology $R[\blank]\colon \cl{S}\to \Mod_R$ induces a functor 
\[ R[\blank]\colon \rm{Pro}(\cl{S})\to \rm{Pro}(\Mod_R) \quad \flim_i X_i \mapsto \flim_i R[X_i].\]
  Since the infinite loop space of any spectrum is nilpotent, applying $R[\blank]$ to the reflection
  $X\to \flim_\alpha X_{\alpha}$ into $\rm{Pro}(\cl{S}^{\rm{nilp}})$ induces an equivalence
  \[ cR[X] \to \flim_{\alpha} R[X_\alpha]\]
  in $\rm{Pro}(\Mod_R)$. Since the forgetful functor 
  $\cCAlg(\rm{Pro}(\Mod_R)) \to \rm{Pro}(\Mod_R)$ commutes with cofiltered limits, the right 
  hand limit  can also be computed in $\cCAlg(\rm{Pro}(\Mod_R))$. Since the coalgebraic 
  materialization of constant objects exists and commutes with limits, we obtain that
  \[ R[X] \simeq M^{\rm{cA}}(cR[X]) \simeq M^{\rm{cA}}(\flim_\alpha R[X_{\alpha}])
  \simeq \flim_{\alpha} R[X_\alpha ],\]
 as claimed. 
\end{proof}

\begin{theorem}\label{sepet}
    Let $X$ be a connected space and $R$ a connective $\F_p$-algebra.
    Then the $R$-homology $R[X]\in \cCAlg(\Mod_R^{\rm{cn}})$ is formally \'etale. 
\end{theorem}
\begin{proof}
    Since $\Omega^\infty N$ is $p$-complete for any $N \in \Mod_{\F_p}$, the natural map
    \[ R[X] \to R[X^\wedge_p]\]
    is an equivalence, so we may assume that $X$ is $p$-complete. Let $M \in \Mod_R^{\rm{cn}}$ and
    denote by $e:R \to R\oplus M$ be the 0-section.
    By Lemma~\ref{emhom} we again have that
    \[ e^\pt R[X] = \flim e^\pt R[X_\alpha] \]
    where each $X_\alpha$ is nilpotent. Thus, since Weil-restriction again commutes with limits 
    we can assume that $X$ is nilpotent. Write 
    \[
    X^n := (\Omega^\infty \F \{ \blank\})^{\circ n} \circ X,
    \]
    which defines a co-augmented simplicial diagram $X^\bullet \to X$, which admits an additional
    degeneracy by choosing base-points. Then, since $X$ is nilpotent and $p$-complete we have
    by~\cite[][Proposition VI.6.2]{bk} that $X\simeq \lim X^n$. Moreover, the additional degeneracy
    tells us that this is a universal limit diagram, and hence
    \[ e^\pt R[X] \simeq \flim e^\pt R[X^\bullet] \in \cCAlg(\Mod_R).\]
    So we can assume that $X$ is of the form $\Omega^\infty \F_p\{Y\}$. Now, by Lemma~\ref{genem}
    we see that $e^\pt R[X] \simeq \lim_i e^\pt R[X_i]$ where each $X_i$ is a finite 
    product of Eilenberg\textendash MacLane spaces. Hence, the claim follows from Proposition~\ref{emfet}
    since $R[\blank]$ takes products of spaces to products of coalgebras.
\end{proof}


Thus, we have answered our initial question about recovering spherical chains from characteristic
$p$-chains, which we may summarize as follows.

\begin{corollary}
  Let $X\in \cl{S}$ be either connected or compact. The space of lifts of
  $k[X] \in \cCAlg(\Mod_k)$ to a coalgebra in $p$-complete, connective $\bb{W}(k)$-modules 
  is contractible and the unique poiint is given by $\bb{W}(k)[X]^\wedge$. 
  Moreover, for any $A \in \rm{cCAlg}(\Mod_{\bb{W}(k)}^{\wedge,\rm{cn}})$
  the base change along the map $\bb{W}(k)\to k$ induces an equivalence
  \[\rm{Map}_{\rm{cCAlg}(\rm{Mod}_{\bb{W}(k)}^{\wedge})}(A, \bb{W}(k)[X]^\wedge_p)
    \rar{\sim} \rm{Map}_{\rm{cCAlg}(\rm{Mod}_{k})}(A\otimes k, k[X]). \]
\end{corollary}
\begin{proof}
    Combine Theorem~\ref{sepet} with Theorem~\ref{wittsp}.
\end{proof}

Finally, we observe that this also yields a spherical and coalgebraic version
of Mandells embedding in~\cite{mandell}.

\begin{corollary}
 Let $k$ be a perfect, separably closed field of characteristic $p$ with spherical Witt vectors
 $\W(k)$. The $p$-complete $\bb{W}(k)$-homology functor
 \[ \cl{S}_p^{\rm{nilp}} \to \rm{cCAlg}(\Mod^{\wedge}_{\bb{W}(k)})
 \quad X \mapsto \bb{W}(k)[X]^\wedge_p\]
 is fully faithful. In particular, for any nilpotent space $X$ we have a natural equivalence
 \[ X^\wedge_p\simeq \rm{Map}_{\rm{cCAlg}(\Mod_{\bb{W}(k)}^\wedge})(\bb{W}(k), \bb{W}(k)[X]^\wedge_p).\]
\end{corollary}
\begin{proof}
By~\cite[][Theorem 1.2.]{bb} the functor
\[ k[\blank]: \cl{S}_p^{\rm{nilp}} \to \cCAlg(\Mod_{k})\]
is fully faithful. By Theorem~\ref{sepet} it factors through $\cCAlg(\Mod_k^{\rm{cn}})^{\fet}$ and
hence by Theorem~\ref{wittsp} we can compose with $\cl{W}$ and get the claim.
\end{proof}