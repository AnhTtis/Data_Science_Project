In this section, we recall the definition of an $\bb{E}_{\infty}$-coalgebra
in a symmetric monoidal category $\cl{C}$ and collect some facts about it.
Colimits of coalgebras are computed underlying and if $\cl{C}$ is presentable and the tensor
product commutes with colimits in each variable separately, then $\rm{cCAlg}(\cl{C})$ is again
presentable. For a map of connective $\bb{E}_\infty$-rings $f:R \to S$ this lets us introduce the adjunction
\[\begin{tikzcd}
	{\rm{cCAlg}(\rm{Mod}_R^{\rm{cn}})} & {\rm{cCAlg}(\rm{Mod}_S^{\rm{cn}})}
	\arrow[""{name=0, anchor=center, inner sep=0}, "{{f_\pt}}", shift left=2, from=1-2, to=1-1]
	\arrow[""{name=1, anchor=center, inner sep=0}, "{{f^\ast}}", shift left=2, from=1-1, to=1-2]
	\arrow["\dashv"{anchor=center, rotate=-90}, draw=none, from=1, to=0]
\end{tikzcd}\]
where $f^\pt$ is the base change and $f_\pt$ is conjured up by the adjoint functor theorem. Crucially,
$f_\pt$ is \textit{not} induced from the restriction of scalars on the level of module categories. 
We then consider the coalgebra structure on the $R$-chains of a space $X$ and the algebra structure on
the dual of a coalgebra in a presentably monoidal category. 

\subsection{Generalities}

\begin{definition}
  Let $\cl{C}$ be a symmetric monoidal category. We denote by $\rm{CAlg}(\cl{C})$ the category
  of $\bb{E}_{\infty}$-algebras in $\cl{C}$. For $\cl{C}= \rm{Sp}$ we write $\rm{CAlg}(\rm{Sp})= \rm{CAlg}$
  and refer to objects of $\rm{CAlg}$ simply as $\bb{E}_{\infty}$-rings.
\end{definition}

\begin{proposition}[Lurie]\label{calg}
  Let $(\cl{C}, \otimes)$ be a symmetric monoidal category, then the following statements hold:
  \begin{enumerate}
    \item The forgetful functor $U:\rm{CAlg}(\cl{C}) \to \cl{C}$ is conservative and
          commutes with limits.
    \item The coproduct of two algebras $R,S \in \rm{CAlg}(\cl{C})$ is given by the tensor
          product $R \otimes S$.
    \item If $\cl{C}$ is presentable and $\blank \otimes \blank$ commutes with colimits in both variables
          separately, then $\rm{CAlg}(\cl{C})$ is presentable as well.
  \end{enumerate}
\end{proposition}

\begin{proof}
  The first claim is a combination of~\cite[][Lemma 3.2.2.6]{ha} and~\cite[][Corollary 3.2.2.5]{ha}.
  The second is shown in~\cite[][Corollary 3.2.4.7]{ha} and the third is~\cite[][Corollary 3.2.3.5.]{ha}.
\end{proof}

\begin{definition}
  Let $\cl{C}$ be a symmetric monoidal category. The opposite category $\cl{C}\op$ inherits 
  a natural symmetric monoidal structure and we define the category of $\bb{E}_{\infty}$-coalgebras
  in $\cl{C}$ as
  \[\rm{cCAlg}(\cl{C}):= \rm{CAlg}(\cl{C}^{\rm{op}})^{\rm{op}}.\]
\end{definition}

\begin{remark}\label{colimits}
  In particular, a coalgebra $A \in \rm{cCAlg}(\cl{C})$ comes equipped with the datum of a 
  ``coherently commutative'' comultiplication and counit maps
  \begin{align*}
    \Delta_{A}: A &\to (A \otimes A)^{h\Sigma_2} \\
    \eta: A &\to 1_{\cl{C}},
  \end{align*}
  where $1_{\cl{C}}$ denotes the unit of $\cl{C}$. Note that, in general, there is no way
  to describe $\rm{cCAlg}(\cl{C})$ as a category of algebras in some suitable category $\cl{D}$.
  Nonetheless, Proposition~\ref{calg} immediately implies the following:
  \begin{enumerate}
    \item The forgetful functor $U:\rm{cCAlg}(\cl{C}) \to \cl{C}$ is conservative and commutes with colimits.
    \item The product of two coalgebras $R,S \in \rm{cCAlg}(\cl{C})$ is given by $R \otimes S$.
  \end{enumerate}
However, we cannot deduce presentability this way since the opposite of a presentable category is 
almost never presentable. 
\end{remark}

\begin{proposition}[Lurie]\label{present}
  Let $\cl{C}$ be a symmetric monoidal category such that $\blank \otimes \blank$ commutes with
  colimits in each variable separately and $\cl{C}$ is presentable. Then, the category
  $\rm{cCAlg}({\cl{C}})$ is presentable.
\end{proposition}
\begin{proof}
  This is~\cite[][Corollary 3.1.4.]{ellI}.
\end{proof}

This can be seen as an analogue of the classical theorem by Sweedler that every coalgebra
in the 1-category of vector spaces over a field is a filtered colimit of its finite dimensional
sub-coalgebras, see~\cite{sweedler1969hopf}. However, unlike in the classical situation, if $\cl{C}$
is $\kappa$-presentable, we only deduce that $\rm{cCAlg}(\cl{C})$ is $\tau$-presentable for some 
$\tau \geq \kappa$. This is one of the main defects the category of coalgebras has over that
of algebras. 

\begin{definition}
    Let $\cl{C}$ be as in Proposition~\ref{present}. Since by Remark~\ref{colimits} the
    forgetful functor $U:\cCAlg(\cl{C})\to \cl{C}$ commutes with colimits, it admits
    a right adjoint 
    \[C:\cl{C}\to \cCAlg(\cl{C}).\]
    For any $M\in \cl{C}$ we call $C(M)$ the \textit{cofree coalgebra} on $M$.
\end{definition}

\begin{lemma}\label{limits}
  Let $\rm{Cat}_{\infty}^{\otimes}$ denote the (large) category of symmetric monoidal 
  categories and monoidal functors. Then the functors
  \[\rm{Cat}_{\infty}^{\otimes} \to \rm{Cat}_{\infty} \quad \cl{C}\mapsto \rm{CAlg}(\cl{C}),\]
  \[\rm{Cat}_{\infty}^{\otimes} \to \rm{Cat}_{\infty} \quad \cl{C} \mapsto \rm{cCAlg}(\cl{C})\]
  commute with limits.
\end{lemma}

\begin{proof}
  By Proposition~\ref{calg} the functor $\rm{CAlg}(\blank)$ factors through the (large) category 
  $\rm{Cat}_{\infty}^{\amalg}$ of categories which admit finite coproducts and functors which
  preserve finite coproducts. As such it admits a left adjoint which equips $\cl{D} \in \rm{Cat}_{\infty}^{\amalg}$ with the
  cocartesian monoidal structure. Moreover, the inclusion $\rm{Cat}_{\infty}^{\amalg} \rari{} \rm{Cat}_{\infty}$
  admits a left adjoint which takes a category $\cl{C}$ to the free finite coproduct completion, namely
  the full subcategory of $\rm{P}(\cl{C})$ spanned by finite coproducts of representables.
  Thus, both functors commute with limits, and so the composition $\CAlg(\blank)$ does as well. \\
  For the second functor, we simply observe that it is given by the composition 
  \[ \rm{Cat}_{\infty}^{\otimes} \rar{(\blank)\op} \rm{Cat}_{\infty}^{\otimes} \rar{\rm{CAlg(\blank)}} \rm{Cat}_{\infty}
  \rar{(\blank)\op}\rm{Cat}_{\infty},\]
which immediately implies the claim, since taking the opposite category is an involution, i.e.~an
equivalence of categories and thus commutes with limits.
\end{proof}


\subsection{Coalgebraic right adjoints and duality}


\begin{definition}[Lurie tensor product]
  Let $\rm{Pr^{L}}$ denote the (large) category of presentable categories
  with maps given by functors which commute with colimits. 
  By~\cite[][Proposition 4.8.1.15.]{ha} $\rm{Pr^{L}}$ inherits a natural
  symmetric monoidal structure, such that we have a map $\cl{C}\times \cl{D}\to \cl{C}\otimes \cl{D}$
  exhibiting $\rm{Fun}^{\rm{L}}(\cl{C}\otimes \cl{D},\cl{E})$ as the category of functors
  $\cl{C}\times \cl{D}\to \cl{E}$ which commute with colimits in each variable separately.
  We call $\rm{CAlg}(\rm{Pr^{L}})$ the category of \textit{presentably symmetric monoidal}
  categories.
\end{definition}

\begin{example}
  If $R$ is an $\bb{E}_{\infty}$-ring then the category of $R$-modules in spectra $\rm{Mod}_{R}$
  and of connective $R$-modules $\Mod_R^{\cn}$ are both presentable. 
\end{example}

\begin{definition}
By Proposition~\ref{present} the assignment $ \cl{C} \mapsto \cCAlg(\cl{C})$ defines a functor
\[ \rm{cCAlg}(\blank): \CAlg(\rm{Pr^L})\to \rm{Pr^L}.\]
 In particular, for any map $f:\cl{C} \to \cl{D}$  in $\CAlg(\rm{Pr^L})$ we get an adjunction 
\[\begin{tikzcd}
	{\rm{cCAlg}(\cl{C})} & {\rm{cCAlg}(\cl{D})}
	\arrow["{f^\pt}", shift left, from=1-1, to=1-2]
	\arrow["{f_\pt}", shift left, from=1-2, to=1-1].
\end{tikzcd}\]
Let $g:\cl{D}\to \cl{C}$ denote the right adjoint of $f$. The functor $f^\pt$ is induced by $f$, 
that is for any $A\in \cCAlg(\cl{C})$ the underlying object of $f^\pt(A)$
is given by $f(A)$. However, it is in general \textit{not true} that on underlying objects $f_\pt$ agrees with $g$.
Indeed, $g$ need only be \textit{lax symmetric monoidal} but $f_\pt$ preserves products of coalgebras which are given
by the monoidal products of $\cl{C}$ and $\cl{D}$. We call $f_\pt$ the \textit{coalgebraic right adjoint} of $f$.
\end{definition}


\begin{construction}\label{radj}
    Let $\cl{C},\cl{D} \in \rm{CAlg}(\rm{Pr^L})$ be stable and equipped with a compatible $t$-structure.
    Then $\cl{C}_{\geq 0}$ is closed under the tensor product and thus the inclusion
    \[ \cl{C}_{\geq 0} \rari{} \cl{C}\]
    is a map in $\CAlg(\rm{Pr^L})$. Thus it admits a coalgebraic right adjoint 
    \[ \tau_{\geq 0}^{c}: \rm{cCAlg}(\cl{C}) \to \cCAlg(\cl{C}_{\geq 0}).\]
    Since connective coalgebras form a full subcategory, it is true that $\tau_{\geq 0}(A) \simeq A$ whenever
    $A$ is connective but in general, $\tau_{\geq 0}^c(A)$ does not agree with the underlying connective cover of $A$.
    Moreover, if we have a map $f:\cl{C}\to \cl{D}$ in $\CAlg(\rm{Pr}^L)$ which preserves connective objects, then
    we get a commuting diagram of coalgebraic right adjoints
   \[\begin{tikzcd}
	{\rm{cCAlg}(\cl{C})} & {\rm{cCAlg}(\cl{D})} \\
	{\rm{cCAlg}(\cl{C}_{\geq 0})} & {\rm{cCAlg}(\cl{D}_{\geq 0})}
	\arrow["{f_!}"', from=1-2, to=1-1]
	\arrow["{\tau_{\geq 0}^c}", from=1-2, to=2-2]
	\arrow["{\tau^c_{\geq 0}}"', from=1-1, to=2-1]
	\arrow["{f_\ast}", from=2-2, to=2-1].
\end{tikzcd}\] 
Note that, for $A \in \cCAlg(\cl{D}_{\geq 0})$, whenever $f_!(A)$ happens to be connective,
it necessarily agrees with $f_\pt (A)$. 
\end{construction}

\begin{example}[Weil restriction]\label{adjoint}
    Let $f:R \to S$ be a map of connective $\E_\infty$-rings. The base
    change functor $f^\pt: \rm{Mod}_R \to \rm{Mod}_S$ is symmetric monoidal,
    commutes with colimits and preserves connective objects. Thus the restriction to connective modules
    induces a coalgebraic adjunctions
   \[\begin{tikzcd}
	{f^\pt:\rm{cCAlg}(\Mod_R)} & {\rm{cCAlg}(\Mod_S):f_!}
	\arrow[""{name=0, anchor=center, inner sep=0}, shift left=2, from=1-1, to=1-2]
	\arrow[""{name=1, anchor=center, inner sep=0}, shift left=2, from=1-2, to=1-1]
	\arrow["\dashv"{anchor=center, rotate=-90}, draw=none, from=0, to=1]
\end{tikzcd}\] 
   \[\begin{tikzcd}
	{f^\pt:\rm{cCAlg}(\Mod_R^{\rm{cn}})} & {\rm{cCAlg}(\Mod_S^{\rm{cn}}):f_\pt}
	\arrow[""{name=0, anchor=center, inner sep=0}, shift left=2, from=1-1, to=1-2]
	\arrow[""{name=1, anchor=center, inner sep=0}, shift left=2, from=1-2, to=1-1]
	\arrow["\dashv"{anchor=center, rotate=-90}, draw=none, from=0, to=1].
\end{tikzcd}\] 
For $A\in \cCAlg(\Mod_R)$ we refer to $f_! A$ as the \textit{Weil restriction} and to $f_\pt \tau_{\geq 0}^c A$ 
as the \textit{connective Weil restriction} of $A$ along $f$. Observe that we have a commutative diagram
in $\rm{Pr^L}$
\[\begin{tikzcd}
	{\cCAlg(\Mod_R)} & {\cCAlg(\Mod_S)} \\
	{\Mod_R} & {\Mod_S}
	\arrow["{f^\pt}", from=1-1, to=1-2]
	\arrow[from=1-1, to=2-1]
	\arrow[from=1-2, to=2-2]
	\arrow["{f^\pt}"', from=2-1, to=2-2].
\end{tikzcd}\]
Thus, taking right adjoints gives a commutative diagram
\[\begin{tikzcd}
	{\cCAlg(\Mod_R)} & {\cCAlg(\Mod_S)} \\
	{\Mod_R} & {\Mod_S}
	\arrow["{f_!}"', from=1-2, to=1-1]
	\arrow["{C_R}", from=2-1, to=1-1]
	\arrow["{C_S}"', from=2-2, to=1-2]
	\arrow["{f_\pt}", from=2-2, to=2-1],
\end{tikzcd}\]
which tells us that Weil restriction commutes with the cofree coalgebra functors in the sense that
\[ f_!C_S(M) \simeq C_R(f_\pt M)\]
for any $M\in \Mod_S$ where $f_\pt M$ denotes the usual restriction of scalars. Note that we can draw the same
diagram where we replace all module categories by connective modules. Even though we care mainly about the
connective Weil restriction functor, the coalgebraic connective cover of the cofree coalgebra is even less
approachable than the cofree coalgebra itself, so we shall not utilize this. 
Instead, in Section~\ref{homocoalg} we will compute with the non-connective Weil-restriction 
and see that the result is connective.
\end{example}


\begin{example}[Duality]\label{predual}
  For any $\cl{C}\in \CAlg(\rm{Pr^L})$ and $X\in \cl{C}$,
  the functor $X\otimes \blank :\cl{C} \to \cl{C}$ commutes with colimits and thus admits a right adjoint
  $\rm{map}_{\cl{C}}(X, \blank)$.
  This assignment defines a functor $\cl{C}\op \to \rm{Fun}(\cl{C}, \cl{C})$, and we denote
  its adjoint as
  \[ \rm{map}_{\cl{C}}(\blank, \blank): \cl{C}^{\op} \times \cl{C} \to \cl{C}.\]
  The target of $(\blank)^\vee:=\rm{map}_\cl{C}(\blank, \mathds{1}_\cl{C}): \cl{C} \to \cl{C}\op$ 
  is not presentable, so this is not a map in $\CAlg(\rm{Pr^L})$. However, it still admits a coalgebraic 
  right adjoint via the following construction. The functor $\rm{map}_\cl{C}(\blank, \blank) $ is lax monoidal with 
  respect to the monoidal structure on $\cl{C}\op \times \cl{C}$, defined by
\[ (A,B)\otimes_{\cl{C}\op \times \cl{C}} (C,D):= (A \otimes C, B \otimes D).\]
Thus, it refines to a functor
\[ \rm{map}_{\cl{C}}(\blank, \blank): \rm{CAlg}(\cl{C}\op\times \cl{C})
  \simeq \rm{cCAlg}(\cl{C})\op \times \rm{CAlg}(\cl{C})
  \to \rm{CAlg}(\cl{C}).\]
In particular, for each $R \in \rm{CAlg}(\cl{C})$ we get a functor
\begin{align*}
  \rm{map}_{\cl{C}}(\blank, R):\rm{cCAlg}(\cl{C}) = \rm{CAlg}(\cl{C}\op)\op \to \rm{CAlg}(\cl{C})\op.
\end{align*}
Since colimits of coalgebras and limits of algebras are computed underlying, the assignment
\[A \mapsto A^\vee:=\rm{map}_{\cl{C}}(A, \mathds{1}_{\cl{C}}) \]
takes colimits of coalgebras to limits of algebras. By the adjoint functor theorem, we get an adjunction.
\[\begin{tikzcd}
	{(\blank)^\vee:\cCAlg(\cl{C})} & {\rm{CAlg}(\cl{C}):(\blank)^\circ}
	\arrow[""{name=0, anchor=center, inner sep=0}, shift left=2, from=1-1, to=1-2]
	\arrow[""{name=1, anchor=center, inner sep=0}, shift left=2, from=1-2, to=1-1]
	\arrow["\dashv"{anchor=center, rotate=-90}, draw=none, from=0, to=1].
\end{tikzcd}\]
For $A \in \rm{cCAlg}(\cl{C})$ and $R \in \rm{CAlg}(\cl{C})$. We call $A^\vee$ the \textit{dual algebra}
of $A$ and $R^\circ$ the \textit{pre-dual coalgebra} of $R$.
\end{example}

If the underlying object of $R\in \rm{CAlg}(\cl{C})$ is dualizable, then the dual $R^\vee$ also inherits a natural
coalgebra structure. In fact this construction agrees with $R^\circ$ as we will see in Corollary~\ref{dualad}.

\begin{proposition}\label{duality}
Let $\cl{C}$ be a symmetric monoidal category and denote by $\cl{C}^{\rm{dual}}$ the full subcategory
spanned by the dualizable objects. The functor
  \[ \rm{cCAlg}(\cl{C}^{\rm{dual}})\op \rar{\sim} \rm{CAlg}(\cl{C}^{\rm{dual}})
  \quad A \mapsto A^{\vee}\]
  is an equivalence of categories with inverse taking $R\in \rm{CAlg}(\cl{C}^{\rm{dual}})$
  to the dual $R^{\vee}$ with the induced coalgebra structure.
\end{proposition}
\begin{proof}
This is immediate from \cite[][Proposition 3.2.4]{ellI}.
\end{proof}

For the next lemma we employ the following terminology:\\
For a functor $F:\cl{C}\to \cl{D}$ we say that $A \in \cl{C}$ is $F$-local, if the natural
transformation $\rm{Map}_{\cl{C}}(A, \blank) \to \rm{Map}_{\cl{D}}(F(A), F(\blank))$
is an equivalence.
\begin{lemma}\label{laxlocal}
  Let $\cl{C}, \cl{D}$ be symmetric monoidal $\infty$-categories and $F:\cl{C}\to \cl{D}$ be a lax symmetric
  monoidal functor. Suppose we have $R\in \rm{CAlg}(\cl{C})$ such that each tensor power
   $R^{\otimes n}$ considered as an object in $\cl{C}$ is $F$-local. Then the map
  \[ \rm{Map}_{\rm{CAlg(\cl{C})}}(R, S)\rar{F} \rm{Map}_{\rm{CAlg}(\cl{D})}(F(R), F(S)) \]
  is an equivalence.
\end{lemma}
\begin{proof}
  Since $F$ is lax symmetric monoidal, it induces a map of $\infty$-operads
  \[\begin{tikzcd}
	{\cl{C}^\otimes} && {\cl{D}^\otimes} \\
	& {\rm{Fin}_\pt}
	\arrow["p"', from=1-1, to=2-2]
	\arrow["q", from=1-3, to=2-2]
	\arrow["f", from=1-1, to=1-3]
\end{tikzcd}\]
which takes any $(A_{1}, \dots, A_{n})\in \cl{C}^{\otimes}_{\langle n \rangle}$ to the point
$(F(A_{1}), \dots, F(A_{n}))\in \cl{D}^{\otimes}_{\langle n \rangle}$. A commutative algebra structure on an object
$R\in \cl{C}$ is precisely given by a a section $s_{R}$ of $p$ which takes $\langle n \rangle $ to
$(R, \dots, R) \in \cl{C}_{\langle n \rangle }$ and maps inert morphisms to inert morphisms. Let
$\varphi: \langle n \rangle \to \langle m \rangle $ be a morphism in $\rm{Fin}_{\pt}$ and denote by $p_{i}: \langle n \rangle \to \langle 1 \rangle$ the unique inert
map which sends $i \mapsto 1$. For each $i$ we have a factorization
\[ \langle n \rangle \rar{\psi_{i}} \langle k_{i}\rangle \rar{\pi_{i}} \langle 1 \rangle\]
of $p_{i}\circ \varphi$ into an inert map $\psi_{i}$ and an active map $\pi_{i}$.
Then for each $B= (B_{1}, \dots B_{m})\in \cl{C}^{\otimes}$ we get equivalences
\begin{align*}
  \rm{Map}^{\varphi}_{\cl{C}^{\otimes}}(s_{R}(\langle n \rangle), B)
  &\simeq \prod_{i =1, \dots, m}\rm{Map}^{ p_{i}\circ \varphi}_{\cl{C}^{\otimes}}((R, \dots, R), B_{i})\\
  &\simeq \prod_{i =1, \dots, m} \rm{Map}_{\cl{C}}(R^{\otimes k_{i}}, B_{i})\\
  &\simeq \prod_{i =1 ,\dots, m} \rm{Map}_{\cl{D}}(F(R)^{\otimes k_{i}}, F(B_{i}))\\
  &\simeq \prod_{i =1 , \dots m } \rm{Map}^{p_{i} \circ \varphi}_{\cl{D}^{\otimes}}((f \circ s_{R})(\langle n \rangle), B_{i})\\
  &\simeq \rm{Map}_{\cl{D}^{\otimes}}^{\varphi}((f \circ s_{R})(\langle n \rangle), B),
\end{align*}
hence each value of $s_{R}: \rm{Fin}_{\pt}\to \cl{C}^{\otimes}$ is $f$-local, and thus $s_{R}$
is local with respect to the functor $f_{\pt}: \Gamma(p) \to \Gamma(q)$. Since $\rm{CAlg}(\cl{C})$
and $\rm{CAlg}(\cl{D})$ are full subcategories of $\Gamma(p)$ and $\Gamma(q)$ respectively, the
claim follows.
\end{proof}

\begin{corollary}\label{dualad}
Let $\cl{C}$ be a symmetric monoidal category and $A,B \in \rm{cCAlg}(\cl{C})$ with $A$ dualizable.
The natural map
\[ \rm{Map}_{\rm{cCAlg}(\cl{C})}(B,A)\to\rm{Map}_{\rm{CAlg}(\cl{C})}(A^{\vee}, B^{\vee})\]
is an equivalence. In particular, we have that $(A^\vee)^\circ \simeq A$.
\end{corollary}
\begin{proof}
 This is immediate by applying Lemma~\ref{laxlocal} to the duality functor 
 $(\blank)^{\vee}:\cl{C}\op \to \cl{C}$ since dualizable objects are closed under tensor products.
\end{proof}

We now review why the homology of a space carries an $\bb{E}_{\infty}$-coalgebra structure. The following
lemma in particular shows that every $X\in \cl{S}$ admits a unique coalgebra structure with respect 
to the cartesian product induced from the diagonal map $X\rar{\Delta} X\times X$.

\begin{lemma}\label{trivcalg}
  Let $\cl{C}$ be a category which admits finite products equipped with the cartesian monoidal structure.
  Then the forgetful functor $\rm{cCAlg}(\cl{C}) \to \cl{C}$ is an equivalence.
\end{lemma}
\begin{proof}
  By~\cite[][Corollary 2.4.4.10.]{ha} the map $\rm{CAlg}(\cl{C}\op) \to \cl{C}\op $ is an equivalence, hence
  the claim follows by applying $(\blank)\op$.
\end{proof}

The following generalization of the Eilenberg \textendash Zilber Theorem equips every ``generalized suspension''
of a space with the structure of an $\bb{E}_{\infty}$-coalgebra.

\begin{proposition}\label{chains}
  Let $\cl{C}$ a presentably symmetric monoidal category. The functor
  \begin{align*}
    \cl{S} \to \cl{C} \qquad X \mapsto 1_{\cl{C}}[X]
  \end{align*}
  which sends a space $X$ to the colimit over the constant diagram $X \to \pt \rar{1_{\cl{C}}} \cl{C}$
  is symmetric monoidal with respect to the cartesian monoidal structure on $\cl{S}$.
  In particular, this induces a functor
  \[ \cl{S} \simeq \rm{cCAlg}(\cl{S}) \to \rm{cCAlg}(\cl{C}) \quad X \mapsto 1_{\cl{C}}[X]\]
\end{proposition}

\begin{proof}
  Since $\blank \otimes \blank$ commutes with colimits in both variables
  separably by assumption, we have that:
  \begin{align*}
    (\colim_X 1_{\cl{C}})\otimes (\colim_Y 1_{\cl{C}})
    \simeq \colim_X \colim_Y (\underbrace{1_{\cl{C}}\otimes 1_{\cl{C}}}_{\simeq 1_{\cl{C}}}) \simeq \colim_{X \times Y} 1_{\cl{C}}.
  \end{align*}
  The second statement then follows from Lemma~\ref{trivcalg}.
\end{proof}

\begin{example}\label{homology}
By Proposition~\ref{chains}, for each $\bb{E}_{\infty}$-ring $R$, the $R$-homology functor
\begin{align*}
  \cl{S} \to \rm{Mod}_{R} \quad X \mapsto R[X]
\end{align*}
is symmetric monoidal and thus refines to a functor
\[ R[\blank]: \cl{S}\simeq \rm{cCAlg}(\cl{S})\to \rm{cCAlg}(\Mod_{R}). \]
Hence, the $R$-homology of a space $X$ carries a natural coalgebra structure. Moreover,the $R$-cohomology of $X$
  \[ R[X]^\vee = \rm{map}_{\Mod_R}(R[X], R) \simeq \lim_{X}R = R^{X}\]
  inherits a natural $R$-algebra structure. If $X$ is finite, then $R[X]$ is dualizable and so by 
  Corollary~\ref{dualad} we have $(R^X)^\circ \simeq R[X]$.
\end{example}
