In this chapter we review the theory of square zero extensions and the deformation theory of functors
$X:\rm{CAlg}^{\rm{cn}} \to \cl{S}$, as developed in~\cite{ha} and~\cite{dag14}.
Let $R^\eta \to R$ be a square zero extensions of connective $\E_\infty$-rings with fiber
$M \in \rm{Mod}_{R}^{\rm{cn}}$. The goal is to understand the fiber
$\rm{fib}_{A}(X(R^\eta)\to X(R))$ over some point $A\in X(R)$.
The key insight is that $R^\eta\to R$ is classified by a map into
the split square zero extension $R \oplus M[1]\in \rm{cCAlg}_{/R}$. 
For well behaved functors $X:\rm{CAlg}^{\rm{cn}} \to \cl{S}$
the space $X(R^\eta)$ can then be obtained by pulling back along the induced map
$ X(R) \to X(R \oplus M[1])$. In this case the fiber $\rm{fib}_{A}(X(\widetilde{R})\to X(R))$ is given a path space in $X(R \oplus M[1])$. Moreover, for each $A\in X(R)$ writing
\[ X_{A}^{R\oplus M[n]}:=\rm{fib}_{A}(X(R\oplus M[n])\to X(R))\]
the sequence $\{X_{A}^{R\oplus M[n]}\}_{n\in \bb{N}}$ defines a spectrum
$T^{M}_{X_{A}}$ which collects the deformation theoretic data into one spectrum 
called the \textit{tangent complex} of $X$ at the point $A$. 

\subsection{Square zero extensions and deformations}
\begin{proposition}\label{loops}
  For every $\bb{E}_{\infty}$-ring $R$ there is an equivalence of categories
  \[\rm{Mod}_{R} \rar{\sim} \rm{Sp}(\rm{CAlg}_{/R})\]
  such that for each $n \geq 0$ the functor
  \[ \rm{Mod}_{R} \rar{\sim} \rm{Sp}(\rm{CAlg}_{/R}) \rar{\Omega^{\infty -n}} \rm{CAlg}_{/R} \]
  sends a module $M$ to an augmented $R$-algebra whose underlying $R$-module is given by
  the direct sum $R \oplus M[n]$.
\end{proposition}
\begin{proof}
This is~\cite[][Theorem 7.3.4.13]{ha}.
\end{proof}

\begin{remark}
  For a connective $\bb{E}_{\infty}$-ring $R$ and a connective $R$-module $M$ we call
  $ \Omega^{\infty} M =R \oplus M$ with the $R$-algebra structure described above the
  \textit{split square zero extension} of $R$ along $M$. If $R$ and $M$ are discrete, the
  multiplication is explicitly given by
  \[(a+m)(b +n) := ab + an +mb ,\]
  i.e.~ we have that $R \oplus R \simeq R[x]/x^{2}$.
\end{remark}

\begin{definition}
  For a connective $\bb{E}_{\infty}$-ring $R$ and a connective $R$-module $M$ we say that a map $R^{\eta} \to R$ is
  a \textit{square zero extension} of $R$ along $M$ if it fits into
  a pullback diagram
  \[\begin{tikzcd}
	{R^\eta} & R \\
	R & {R \oplus M[1]}
	\arrow[from=1-1, to=1-2]
	\arrow["{(\id, 0)}", from=1-2, to=2-2]
	\arrow[from=1-1, to=2-1]
	\arrow["{(\id,\eta)}"', from=2-1, to=2-2].
\end{tikzcd}\]
Moreover, we call the mapping space $\rm{Map}_{\rm{CAlg}_{/R}}(R, R \oplus M)$ the
space of \textit{derivations} $\eta:R \to M$.
\end{definition}
This definition make sense without any connectivity assumptions. However, since our interest
lies primarily in the case where everything is connective, we have chosen to include them
in the definition to avoid awkward terminology.

\begin{remark}\label{cohsq0}
  Note that, the split square zero extension is precisely the one classified by the zero derivation
  \[R \rar{(\id,0)} R \oplus M[1].\]
   Moreover, if $p:R^{\eta} \to R$ is any square zero extension classified by a derivation 
   $\eta:R \to M[1]$, then by taking fibers we get a commutative diagram with exact columns
  \[\begin{tikzcd}
	{M} & {M} \\
	{R^\eta} & R \\
	R & {R \oplus M[1]}
	\arrow[from=1-1, to=2-1]
	\arrow[from=2-1, to=3-1]
	\arrow["{(\id,\eta)}", from=3-1, to=3-2]
	\arrow["{(\id,0)}", from=2-2, to=3-2]
	\arrow["0",from=1-2, to=2-2]
	\arrow["\sim", from=1-1, to=1-2]
	\arrow[from=2-1, to=2-2].
\end{tikzcd}\]
In particular, the fiber of $f$ inherits a natural $R$-module structure, such that the multiplication
map factors as
\[ M \otimes_{R^\eta} M \to M\otimes_R M \to M\]
where the second map is induced from the 0-map $M \to R$ by applying $\blank_{R}\otimes M$.
Since any map of the form $f_1 \otimes \cdots \otimes f_{n-1} \otimes 0$ admits a $\Sigma_n$-equivariant
nullhomotopy, we deduce that all coherent multiplication maps
\[ (M^{\otimes n})_{h\Sigma_n} \to M\]
are nullhomotopic. In this sense, square zero extensions of $\E_\infty$-rings are "coherently" square
zero and admit no nontrivial power operations on the fiber.
\end{remark}

\begin{remark}\label{torsor}
  For any connective $R$-module $M$ the augmented $R$-algebra $\Omega^{\infty}M = R \oplus M$ inherits the
  structure of an $\bb{E}_{\infty}$-monoid in $\rm{CAlg}_{/R}$ with delooping given by $R \oplus M[1]$.
  Thus, we can think of $R \oplus M[1]$ as the classifying object for square zero extensions with
  fiber $M$ the ``universal'' derivation being given by the trivial section $R \to R \oplus M[1]$.
  From this perspective, a square zero extension is precisely a $R\oplus M$-torsor in $\rm{CAlg}_{/R}$.
\end{remark}

To make this definition work for us we require a way to check in practice whether a given map
of $\bb{E}_{\infty}$-rings is a square zero extension. This is provided by the following proposition, which
in particular implies that for discrete rings our notion agrees with the classical definition of a
square zero extension.

\begin{proposition}
  Let $R\p\to R$ be a map of connective $\bb{E}_{\infty}$-rings with fiber $M$ such that
  $M \in \rm{Sp}_{\geq n} \cap \rm{Sp}_{\leq 2n}$ and the multiplication map $M \otimes_{R\p}M \to M$ is nullhomotopic.
  Then $R\p\to R$ is a square zero extension.
\end{proposition}
\begin{proof}
  This is immediate from~\cite[][Theorem 7.4.1.23.]{ha}.
\end{proof}

\begin{example}
  If $R\p \to R$ is a surjective map of ordinary commutative rings with kernel $M\subseteq R\p$,
  then $R\p \to R$ is a square zero extension if and only if $M^{2}=0$.
  In particular, for every $n$ the map $\Z/p^{n}\to \Z/p^{n-1}$ is square zero with kernel $\F_{p}$.
\end{example}

\begin{example}\label{sq0ex}
  If $R$ is any connective $\bb{E}_{\infty}$-ring, then the map $\tau_{\leq n}R \to \tau_{\leq n-1}R $ is a square zero
  extension with fiber $\pi_{n}R[n]$.
\end{example}

For a functor $X:\rm{CAlg}^{\rm{cn}} \to \cl{S}$ and a square zero extension $R^{\eta} \to R$
we want to study the fibers of the map $X(R^{\eta}) \to X(R)$, i.e.~given $A\in X(R)$ we wish
to understand the space of deformations of $A$ to an object $\widetilde{A}\in X(R^{\eta})$.
Notice that, since $R$ and $M$ are connective, the derivation
\[R \rar{(\rm{id}, \eta)} R \oplus M[1]\]
is surjective on $\pi_{0}$, indeed an isomorphism. This motivates the following definition for
a class of functors which are ``well behaved'' with regard to square zero extensions.


\begin{definition}
Let $\cl{C}$ be a category. A functor $X:\rm{CAlg}^{\rm{cn}}\to \cl{C}$ is called:
\begin{enumerate}
\item  \textit{Cohesive} if for any pullback diagram of
  connective $\bb{E}_{\infty}$-rings
  \[\begin{tikzcd}
      {R^\prime} & {S^\prime} \\
      R & S \arrow[from=1-2, to=2-2] \arrow[from=2-1, to=2-2] \arrow[from=1-1, to=2-1]
      \arrow[from=1-1, to=1-2]
    \end{tikzcd}\]
  which induces surjections $\pi_{0}R \to \pi_{0}S$ and $\pi_{0}S\p \to \pi_{0}S$, the diagram
  \[\begin{tikzcd}
	{X(R\p)} & {X(S\p)} \\
	{X(R)} & {X(S)}
	\arrow[from=1-1, to=1-2]
	\arrow[from=1-2, to=2-2]
	\arrow[from=1-1, to=2-1]
	\arrow[from=2-1, to=2-2]
\end{tikzcd}\]
is a pullback. We refer to such pullbacks of $\bb{E}_{\infty}$-rings as \textit{small pullbacks}.
   \item \textit{Nilcomplete} if for every connective $\bb{E}_{\infty}$-ring $R$ the natural map
         \[ X(R)\to \flim_{n} X(\tau_{\le n} R)\]
         is an equivalence.
  \end{enumerate}
\end{definition}

\begin{example}
  Let $R$ be a connective $\bb{E}_{\infty}$-ring, then the functor
  \[\rm{Spec}(R): \rm{CAlg}^{\rm{cn}} \to \cl{S} \qquad S \mapsto \rm{Map}_{\rm{CAlg}}(R, S) \]
  commutes with \textit{all} limits, and so in particular it is cohesive and nilcomplete.
\end{example}

\begin{construction}
  Let $X : \rm{CAlg}^{\rm{cn}} \to \cl{S}$ be a cohesive functor, $R \in \rm{CAlg}^{\rm{cn}}$
  and $A \in X(R)$. Then we define a functor
  \[ \rm{X}_{A}^{\blank} :\rm{CAlg}^{\rm{cn}}_{/R} \to \cl{S} \qquad (R\p \to R) \mapsto \rm{fib}_{A}(X(R\p)\to X(R)).\]
  We call $\rm{X}_{A}^{R\p}$ the \textit{space of deformations} of $A$ along $R\p \to R$.
\end{construction}

\begin{remark}
  If $X$ is cohesive and nilcomplete, then for every connective $\bb{E}_{\infty}$-ring $R$
  and every $A\in X(\tau_{\le 0}R)$ we have an equivalence
  \[ X_{A}^{R} \simeq \flim_{n} X_{A}^{\tau_{\le n}R},\]
  i.e.~we can construct lifts to $R$ by inductively lifting against the square zero extensions
  $\tau_{\le n}R \to \tau_{\le n-1}R$. This will be very useful for applications, however we will not need
  nilcompleteness for the theoretical groundwork that makes up the  rest of this chapter.
\end{remark}

\subsection{The tangent complex}\label{sect22}

\begin{definition}
  Let $\cl{C},\cl{D}$ be $\infty$-categories, then a functor $F: \cl{C}\to \cl{D}$ is called:
  \begin{enumerate}
          \item \textit{Reduced} if  it preserves the terminal object.
          \item \textit{Excisive} if it takes pushouts to pullbacks.
  \end{enumerate}
\end{definition}

\begin{proposition}\label{redex}
  Let $X: \rm{CAlg}^{\rm{cn}} \to \cl{S}$ be cohesive and $A \in X(R)$ be an $R$-valued point.
  Then the functor given by the composition
  \[ \rm{Mod}_{R}^{\rm{cn}}\rar{\Omega^{\infty}|_{\rm{Mod}_{R}^{\rm{cn}}}}
    \rm{CAlg}_{/R}^{\rm{cn}} \rar{\rm{X}^{\blank}_{A}} \cl{S} \]
  is reduced and excisive.
\end{proposition}
\begin{proof}
  Clearly, we have $\rm{X}^{R}_{A} \simeq \pt$, so the functor is reduced.
  Since $\rm{X}$ is cohesive and taking fibers commutes with limits,
  the functor $\rm{X}^{\blank}_{A}$ takes small pullbacks to pullbacks.
  Hence, the claim follows from~\cite[][Proposition 1.4.2.13]{ha} by observing that
  $\Omega^{\infty}|_{\rm{Mod}_{R}^{\rm{cn}}}$ sends
  \[\begin{tikzcd}
	M & 0 \\
	0 & {M[1]}
	\arrow[from=1-1, to=1-2]
	\arrow[from=1-2, to=2-2]
	\arrow[from=1-1, to=2-1]
	\arrow[from=2-1, to=2-2]
\end{tikzcd}\]
to the small pullback
\[\begin{tikzcd}
	{R\oplus M} & R \\
	R & {R\oplus M[1]}
	\arrow[from=1-1, to=1-2]
	\arrow["{(\id,0)}", from=1-2, to=2-2]
	\arrow[from=1-1, to=2-1]
	\arrow["{(\id,0)}"', from=2-1, to=2-2],
\end{tikzcd}\]
i.e.~we have $\Omega\rm{X}^{R \oplus M[1]}_{A} \simeq \rm{X}_{A}^{R \oplus M}$ for any $M \in \rm{Mod}_{R}^{\rm{cn}}$.
\end{proof}

\begin{construction}\label{tangent}
  Let $X:\rm{CAlg}^{\rm{cn}} \to \cl{S}$ be cohesive and $A \in X(R)$.
  Then by~\cite[][Proposition 1.4.2.22]{ha} we obtain an essentially unique factorization
  \[\begin{tikzcd}
	& {\rm{Sp}} \\
	{\rm{Mod}_R^{\rm{cn}}} & {\cl{S}}
	\arrow["{X_{A}^{R \oplus \blank}}"', from=2-1, to=2-2]
	\arrow["{\Omega^\infty}", from=1-2, to=2-2]
	\arrow["{T_{X_A}^{\blank}}", dashed, from=2-1, to=1-2]
\end{tikzcd}\]
  where for $M\in \rm{Mod}_{R}^{\rm{cn}}$ the  spectrum $T_{X_{A}}^{M}$ is given by
  the sequence of spaces $\{\rm{X}_{A}^{R \oplus M[n]}\}_{n}$.
 For $M=R$ we call $T^{R}_{X_{A}} =: T_{X_{A}}$ the \textit{tangent complex} of $A$.
\end{construction}

\begin{warning}
  The name tangent complex is a historical convention and somewhat misleading.
  In general, the spectrum $T_{X_{A}}$ is not contained in the full subcategory
  $D(\Z) \subseteq \rm{Sp}$ i.e.~cannot be modeled by a chain complex of abelian groups.
\end{warning}

\begin{lemma}\label{acn}
  For a connective $\bb{E}_{\infty}$-ring $R$ denote by $\rm{Mod}_{R}^{\rm{acn}} \subseteq \rm{Mod}_{R}$ the full
  subcategory spanned by those $R$-modules which are contained in $(\rm{Mod}_{R})_{\geq n}$ for some $n$ and let
  $F: \rm{Mod}_{R}^{\rm{cn}} \to \cl{S}$ be an excisive functor. Then $F$ admits an extension to an excisive
  functor $\rm{Mod}_{R}^{\rm{acn}} \to \cl{S}$ which is unique up to contractible choice.
\end{lemma}
\begin{proof}
  This is~\cite[][Lemma 1.3.2]{dag14}.
\end{proof}

\begin{proposition}\label{structure}
  Let $R$ be a connective $\bb{E}_{\infty}$-ring, $X: \rm{CAlg}^{\rm{cn}} \to \cl{S}$ be cohesive and
  $A \in X(R)$. Then $T_{X_{A}}$ inherits a natural
  $R$-module structure such that for any perfect connective $R$-module $M$ we have
  a natural equivalence $T^{M}_{X_{A}}\simeq T_{X_{A}}\otimes_{R} M$.
\end{proposition}
\begin{proof}
  By Lemma~\ref{acn} we can extend the functor $F=X^{R \oplus \blank}_{X_{A}}: \rm{Mod}_{R}^{\rm{cn}}\to \cl{S}$
  uniquely to an excisive functor $F\p:\rm{Mod}_{R}^{\rm{acn}}\to \cl{S}$. Since $\rm{Mod}_{R}^{\rm{acn}}$
  is stable, $F\p$ is an exact functor. Thus, the restriction $F\p |_{\rm{Mod}_{R}^{\rm{perf}}}$ is also
  exact. Hence, since $\rm{Sp}$ is stable,~\cite[][Proposition 1.4.2.22]{ha} implies that we
  get an essentially unique lift to an exact functor
  $\widetilde{F}: \rm{Mod}_{R}^{\rm{perf}} \to \rm{Sp}$. Finally, applying~\cite[][Proposition 5.5.1.9]{htt}
  we see that $\widetilde{F}$ induces a colimit preserving functor
  $\rm{Ind}(\rm{Mod}_{R}^{\rm{perf}}) \simeq \rm{Mod}_{R}\to \rm{Sp}$, which under the equivalence
  \[ \rm{Fun}^{\rm{L}}(\rm{Mod}_{R}, \rm{Sp})\simeq \rm{Mod}_{R} \qquad G\mapsto G(R)\]
  yields a $R$-module whose underlying spectrum is given by $T_{X_{A}}$.
\end{proof}

\begin{proposition}\label{def}
  Let $X: \rm{CAlg}^{\rm{cn}} \to \cl{S}$ be a cohesive functor and $R^{\eta} \to R$ a square zero extension
  classified by a derivation $R \rar{\eta} M[1]$. Then for each $A \in X(R)$ the space of deformations
  $\rm{X}_{A}^{R^{\eta}}$ is either empty or a torsor under the grouplike $\E_{\infty}$-monoid $\Omega^{\infty}T^{M}_{X_{A}}$.
  Moreover, $\eta$ determines an obstruction class in $\pi_{-1}T^{M}_{X_{A}}$, which vanishes if and only
  if $\rm{X}_{A}^{R^{\eta}}$ is non-empty.
\end{proposition}
\begin{proof}
  Since $X$ is cohesive, applying $\rm{X}_{A}^{\blank}$ to the pullback diagram defining $R^{\eta}$
\[\begin{tikzcd}
	{R^\eta} & R \\
	R & {R\oplus M[1]}
	\arrow[from=1-1, to=1-2]
	\arrow["0", from=1-2, to=2-2]
	\arrow[from=1-1, to=2-1]
	\arrow["{(0,\eta)}"', from=2-1, to=2-2],
\end{tikzcd}\]
we get a pullback of spaces
\[\begin{tikzcd}
	{\rm{X}_A^{R^\eta}} & \pt \\
	\pt & {\rm{X}_A^{R \oplus M[1]}}
	\arrow[from=1-1, to=1-2]
	\arrow["A^{0}",from=1-2, to=2-2]
	\arrow[from=1-1, to=2-1]
	\arrow["A^{\eta}"',from=2-1, to=2-2],
\end{tikzcd}\]
exhibiting $\rm{X}_{A}^{R^{\eta}}$ as the space of paths in $\rm{X}^{R \oplus M[1]}_{A}$ between the points $A^{0}$
and $A^{\eta}$. Hence, it is non-empty if and only if the homotopy class determined by the map
\[ \pt \rar{A^{\eta}} \rm{X}_{A}^{R\oplus M[1]} \simeq \Omega^{\infty}T_{X_{A}}^{M}[1]\]
vanishes. Moreover, in this case $\rm{X}_{A}^{R^{\eta}}$ is a torsor under the loop space based at $A^{0}$,
which is given by
\[ \Omega \rm{X}_{A}^{R\oplus M[1]}\simeq \rm{X}_{A}^{R \oplus M} \simeq \Omega^{\infty}T^{M}_{X_{A}}.\]
\end{proof}

\begin{proposition}\label{bc}
  Let $X: \rm{CAlg}^{\rm{cn}} \to \cl{S}$ be cohesive and $R \to R\p$ a map of connective $\bb{E}_{\infty}$-rings.
  Moreover, let $A \in X(R)$ be a $R$-valued point and denote by $A\p$ the image of $A$
  under the induced map $X(R)\to X(R\p)$.
  Then for every $M \in \rm{Mod}^{\rm{cn}}_{R\p}$ we have a natural map
  \[ T_{X_{A}}^{M} \to T_{X_{A\p}}^{M}\]
  which is an equivalence if the map $\pi_{0}R \to \pi_{0}R\p$ is surjective.
\end{proposition}
\begin{proof}
  Applying $X$ to the pullback of connective $\bb{E}_{\infty}$-rings
\[\begin{tikzcd}
	{R\oplus M} & {R\p \oplus M} \\
	R & R\p
	\arrow[from=2-1, to=2-2]
	\arrow[from=1-2, to=2-2]
	\arrow[from=1-1, to=2-1]
	\arrow[from=1-1, to=1-2]
\end{tikzcd}\]
and taking the fibers over the points $A\in X(R)$ and $A\p\in X(R\p)$ gives a commutative diagram
\[\begin{tikzcd}
	{X^{R\oplus M}_A} & {X_{A\p}^{R\p\oplus M}} \\
	{X(R\oplus M)} & {X(R\p \oplus M)} \\
	{X(R)} & {X(R\p)}
	\arrow[from=3-1, to=3-2]
	\arrow[from=2-2, to=3-2]
	\arrow[from=2-1, to=3-1]
	\arrow[from=2-1, to=2-2]
	\arrow[from=1-1, to=2-1]
	\arrow[from=1-2, to=2-2]
	\arrow[from=1-1, to=1-2]
\end{tikzcd}.\]
The map $X_{A}^{R\oplus M} \to X_{A\p}^{R\p \oplus M}$ is natural in $M$ and thus gives a map of spectra
$T^{M}_{X_{A}}\to T^{M}_{X_{A\p}}$ as claimed. Moreover, if $R \to R\p$ is surjective on $\pi_{0}$, then
the pullback of $\bb{E}_{\infty}$-rings is small. Hence, since $X$ is cohesive, the map $X_{A}^{R\oplus M}\to X_{A\p}^{R\p\oplus M}$
is an equivalence and thus the induced map $T^{M}_{X_{A}}\to T^{M}_{X_{A\p}}$ is as well.
\end{proof}
Notice that, if $R^{\eta} \to R$ is a square zero extension of connective $\bb{E}_{\infty}$-rings
the map $\pi_{0}R^{\eta} \to \pi_{0}R$ is necessarily surjective. Thus, if we are given $A\in X(R)$
and a lift $A^{\eta}\in X(R^{\eta})$, we know that if we have $M\in \rm{Mod}^{\rm{cn}}_{R^{\eta}}$ such that the
$R^{\eta}$-action factors through $R$, then $T^{M}_{X_{A^{\eta}}}$ agrees with $T^{M}_{X_{A}}$. The following
Proposition shows that we can compute the value of $T^{\blank}_{X_{A^{\eta}}}$ on arbitrary connective
$R^{\eta}$-modules in terms of $T^{\blank}_{X_{A}}$.

\begin{proposition}\label{cofib}
  Let $X:\rm{CAlg}^{\rm{cn}} \to \cl{S}$ be cohesive, $R\in \rm{CAlg}^{\rm{cn}}$ and $A\p \in X(R)$. Let
  $R^{\eta} \to R$ be a square zero extension with fiber $M$ such that $A\p$ admits a lift $A\in X(R^{\eta})$.
  Then for any $N\in \rm{Mod}_{R^{\eta}}^{\rm{cn}}$ if we have that
  $T^{M \otimes_{R}(R \otimes_{R^{\eta}}N)}_{X_{A\p}} \simeq T^{N \otimes_{R^{\eta}}R}_{X_{A\p}} \simeq 0$ it follows that
  $T^{N}_{X_{A}} \simeq 0$.
\end{proposition}

\begin{proof}
 Applying the functor $\blank \otimes_{R^{\eta}} N$ to the extension
  \[ M \to R^{\eta} \to R\]
  yields a cofiber sequence
  \[ M \otimes_{R^{\eta}} N \to N \to N \otimes_{R^{\eta}} R\]
  of connective $R^{\eta}$-modules. Now the functor $T^{\blank}_{X_{A}}: \rm{Mod}_{R^{\eta}}^{\rm{cn}} \to \rm{Sp}$
  is excisive, hence we get a fiber sequence of spectra
  \[T^{M\otimes_{R^{\eta}}N}_{X_{A}} \to T^{N}_{X_{A}}\to T^{N \otimes_{R^{\eta}}R}_{X_{A}}.\]
  Since the extension $R^{\eta} \to R$ is square zero,
  the action of $R^{\eta}$ on $M$ factors through $R$, i.e.~we have that
  \[ M \otimes_{R^{\eta}}N \simeq (M \otimes_{R}R) \otimes_{R^{\eta}} N 
  \simeq M \otimes_{R} (R \otimes_{R^{\eta}} N).\]
  Applying Proposition~\ref{bc} we see that
  \[ T^{M \otimes_{R}(R \otimes_{R^{\eta}}N)}_{X_{A}}\simeq  T^{M \otimes_{R}(R \otimes_{R^{\eta}}N)}_{X_{A\p}} \simeq 0\]
  and similarly
  \[ T^{N \otimes_{R^{\eta}}R}_{X_{A}}\simeq T^{N \otimes_{R^{\eta}}R}_{X_{A\p}} \simeq 0,\]
  which proves the claim.
\end{proof}

\begin{remark}
  Note that in the setting of Proposition~\ref{cofib}, if $M$ and $N \otimes_{R^{\eta}}R$ are perfect $R$-modules,
  Proposition~\ref{structure} implies that it suffices to assume that $T_{X_{A\p}} \simeq 0$.
  Moreover, as part of the proof we have seen that every connective $R^{\eta}$-Module $N$ sits
  in a cofiber sequence
  \[ M \otimes_{R} (R \otimes_{R^{\eta}} N) \to N \to R \otimes_{R^{\eta}} N.\]
  If we think of $N$ as a lift of $R \otimes_{R^{\eta}} N$ along the square zero extension $R^{\eta}\to R$,
  this is part of a description of the deformation theory of connective modules. The complete
  description may be deduced from Proposition~\ref{Mod}.
\end{remark}

% \Begin{definition}\label{defcotan}
%   Let $X:\rm{CAlg}^{\rm{cn}} \to \cl{S}$ be cohesive, we say that $X$ \textit{admits a cotangent complex} at a
%   point $\varphi \in X(R)$ if there exists a (not necessarily connective) $R$-module $L_{X_{\varphi}}$,
%   together with for every $M \in \rm{Mod}_{R}^{\rm{cn}}$ an equivalence
%   \[ \rm{Map}_{R}(L_{X_{\varphi}}, M) \simeq \rm{fib}_{\varphi}(X(R \oplus M)\to X(R)) = X_{\varphi}^{R\oplus M},\]
%   which is natural in $M$.
% \end{definition}

% \Begin{remark}\label{dualcotangent}
%   Observe that, if $X: \rm{CAlg}^{\rm{cn}} \to \cl{S}$ admits a cotangent complex at
%   a point $\varphi \in X(R)$, then for every connective $R$-module $M$, considering the composition
% \[\begin{tikzcd}
% 	{\cl{S}^{\rm{fin}}_{\ast}} & {\rm{Mod}_R} & {\cl{S}}
% 	\arrow["M", from=1-1, to=1-2]
% 	\arrow[""{name=0, anchor=center, inner sep=0}, "{X^{R\oplus \blank}_\varphi}", shift left=2, from=1-2, to=1-3]
% 	\arrow[""{name=1, anchor=center, inner sep=0}, "{\rm{Map}_{R}(L_{X_\varphi}, \blank)}"', shift right=2, from=1-2, to=1-3]
% 	\arrow[shorten <=1pt, shorten >=1pt, Rightarrow, from=0, to=1]
% \end{tikzcd}\]
% gives a natural equivalence of $R$-modules
%   \[ \rm{map}_{R}(L_{X_{\varphi}}, M) \simeq T^{M}_{X_{\varphi}}.\]
%   In this sense, the tangent complex is the dual of the cotangent complex. The cotangent complex
%   is a much more useful invariant, since it does not depend on the module $M$. However, as we
%   will see in Example~\ref{counterex}, not every cohesive functor admits a cotangent complex.
% \end{remark}

As an example we now summarize the well-known deformation theory of $\bb{E}_\infty$-rings in our language.
Later we will use this description to analyze the deformation theory of dualizable coalgebras.

\begin{example}~\label{spec}
  Let $R$ be any $\bb{E}_{\infty}$-ring.
  The composition
  \[ \rm{Mod}_{R}\rar{\Omega^{\infty}} \rm{CAlg}_{/R} \rar{\rm{Map}(R, \blank)} \cl{S}\]
  is accessible and commutes with limits. Thus, since $\rm{Mod}_{R}$ is presentable, the adjoint functor
  theorem implies that it is corepresented by an $R$-Module $L_{R}$ called the \textit{cotangent complex}
  of $R$. Now the functor
  \[ X=\rm{Spec}(R): \rm{CAlg}^{\rm{cn}}\to \cl{S} \qquad S \mapsto \rm{Map}_{\rm{CAlg}}(R, S)\]
  is clearly cohesive.
  Moreover, for any $(\varphi:R \to S)\in \rm{Spec}(R)(S)$ and $M \in \rm{Mod}_{S}^{\rm{cn}}$ we get that
  \begin{align*}
    &\rm{fib}_{\varphi}(\rm{Map}_{\rm{CAlg}}(R, S \oplus M) \to \rm{Map}_{\rm{CAlg}}(R,S))\\
    \simeq ~ &\rm{Map}_{\rm{CAlg}_{/S}}(R, S \oplus M)\\
    \simeq ~ &\rm{Map}_{\rm{CAlg}_{/R}}(R, R \oplus \varphi_{\pt}M)\\
    \simeq  ~ &\rm{Map}_{R}(L_{R}, \varphi_{\pt}M)\\
    \simeq ~ &\rm{Map}_{S}(\varphi^{\pt}L_{R}, M).
  \end{align*}
  Hence, for each $M$ we have an equivalence
  \[ \rm{map}_{S}(\varphi^{\pt}L_{R}, M) \simeq T_{X_{\varphi}}^{M}.\]
Explicitly, this tells us that
  the space of lifts in the diagram
  \[\begin{tikzcd}
	&&& {S\oplus M} \\
	{} && R & S
	\arrow["\varphi", from=2-3, to=2-4]
	\arrow[from=1-4, to=2-4]
	\arrow[dashed, from=2-3, to=1-4]
\end{tikzcd}\]
is naturally identified with
$\Omega^{\infty}T_{X_{\varphi}}^{M}=\rm{Map}_{S}(\varphi^{\pt}L_{R}, M)$. Moreover, if $L_{R} \simeq 0$, then $R$ admits
unique lifts against \textit{arbitrary} square zero extensions. In the case $S=R$ and $\varphi=\id$
$\rm{Map}_{\rm{CAlg}_{/R}}(R, R \oplus M) = \rm{Map}_{R}(L_{R}, M)$ is also called the space of
\textit{derivations} $R\to M$, which in the discrete case can be explicitly described via
additive maps satisfying the Leibniz rule. Although the existence of a cotangent complex
for coalgebras remains unclear, we will show that there is a coalgebraic notion
of derivations which play a similar role in the deformation theory.
\end{example}

\begin{example}\label{counterex}
  Proposition~\ref{Mod} implies that the functor
  \[ X:\rm{CAlg}^{\rm{cn}} \to \cl{S} \qquad R \mapsto (\rm{CAlg}_{R}^{\rm{cn}})^{\Delta^{0}}\]
  is cohesive. Moreover, it follows from~\cite[][Proposition 7.4.2.5]{ha} that for every $S \in X(R)$
  there exists an $S$-module $L_{S/R}$ called the \textit{relative cotangent complex}, together with,
  for every connective $S$-module $M$ a natural equivalence
  \[ \rm{Map}_{S}(L_{S/R}, M[1] \otimes_R S) \rar{\sim} \rm{fib}_{S}(X(R\oplus M) \to X(R)).\]
  Thus, the problem of lifting $S$ to a $R\oplus M$-algebra $\widetilde{R}$ is equivalent to
  finding a map of $R$-algebras fitting into the diagram
  \[\begin{tikzcd}
	& {S \oplus (S \otimes_R M[1])} \\
	S & S
	\arrow["\id", from=2-1, to=2-2]
	\arrow[from=1-2, to=2-2]
	\arrow[dashed, from=2-1, to=1-2].
\end{tikzcd}\]
Meaning we get an equivalence
\[ \rm{map}_{S}(L_{S/R}, M[1]\otimes_{R} S) \simeq T^{M}_{X_{S}}\]
\end{example}

