In this paper, we systematically study deformation problems of $\bb{E}_\infty$-coalgebras. 
The key insight is that there is a well behaved class of coalgebras for which such deformation
problems are uniquely solvable, which we call \textit{formally \'etale} coalgebras. 
Moreover, we show that there is a Witt vector style functor which is defined for all coalgebras
which produces unique lifts when restricted to formally \'etale coalgebras.

\addtocontents{toc}{\protect\setcounter{tocdepth}{1}}
\subsection{Overview}

Let $X$ be a space and throughout fix a prime $p$. We want to compare two different invariants
associated to $X$, namely the  $\F_p$-homology $\F_p[X]$ and the $p$-completed suspension
spectrum $\Sigma^\infty_+X^\wedge_p = \S[X]^\wedge_p$. The $\F_p$-chains can be recovered from
the spherical chains via the base change
\[ \S[X]^\wedge_p \otimes \F_p \simeq \F_p[X].\]
However, on the level of spectra there is in general no canonical way to go back.
For example, since any module over a field is free, we have that 
$\F_2[\bb{R}\rm{P}^2] \simeq \F_2 \oplus \Sigma^2\F_2$, but the suspension spectrum
$\S[\bb{R}\rm{P}^2]^\wedge_2 \simeq (\Sigma\S/2)^\wedge_2$ is \textit{not} free as module over 
$\S^\wedge_2$. Thus, we have two different lifts of $\F_2[\bb{R}\rm{P}^2]$ to $\S_2^\wedge$,
namely $\S[\bb{R}\rm{P}^2]^\wedge_2$s and $\S_2^\wedge \oplus \Sigma^2\S_2^\wedge$. 
We can rigidify this situation by considering additional structure on the homology of $X$.
For any $\bb{E}_\infty$-ring $R$, the assignment $X \mapsto R[X]$ canonically refines to a functor 
$\cl{S} \to \rm{cCAlg}(\rm{Mod}_R)$ where the right hand side is the category of
$\bb{E}_\infty$-coalgebras in $\rm{Mod}_R$ and the coalgebra structure on $R[X]$ is induced
by the diagonal map $X \to X \times X$. Our main result is that, if $X$ is connected, we can recover
the coalgebra $\S[X]^\wedge_p$ from the coalgebra $\F_p[X]$ in a functorial way.
To state this in the correct generality, let us fix some notation.
Let $k$ be a perfect $\F_p$-algebra and denote by $\bb{W}(k)$ the spherical
Witt vectors of $k$ as in~\cite{ellII}. For an $\bb{E}_\infty$-ring $R$ denote by $\rm{Mod}_R^\wedge$
the category of $p$-complete $R$-modules equipped with the symmetric monoidal structure given by 
the $p$-completed tensor product. 

\newcounter{tmp}
\begingroup
\setcounter{tmp}{\value{theorem}}% store current value of theorem counter
\setcounter{theorem}{0} %assign desired value to theorem counter
\renewcommand\thetheorem{\Alph{theorem}}% locally redefine the representation of the theorem counter

\begin{theorem}\label{result}
  Let $X\in \cl{S}$ be either connected or compact. The space of lifts of
  $k[X] \in \cCAlg(\Mod_k)$ to a coalgebra in $p$-complete, connective $\bb{W}(k)$-modules 
  is contractible and the unique point is given by $\bb{W}(k)[X]^\wedge_p$. 
  Moreover, for any $A \in \rm{cCAlg}(\Mod_{\bb{W}(k)}^{\wedge,\rm{cn}})$
  the base change along the map $\bb{W}(k)\to k$ induces an equivalence
  \[\rm{Map}_{\rm{cCAlg}(\rm{Mod}_{\bb{W}(k)}^{\wedge})}(A, \bb{W}(k)[X]^\wedge_p)
    \rar{\sim} \rm{Map}_{\rm{cCAlg}(\rm{Mod}_{k})}(A\otimes k, k[X]). \]
\end{theorem}

Since $\bb{W}(k)$ is a connective $\bb{E}_\infty$-ring, and thus a limit of square zero extensions of
$\pi_0\bb{W}(k)$, which is in turn a limit of square zero extensions of $\pi_{0}\bb{W}(k)/p\simeq k$,
we approach Theorem~\ref{result} by first understanding how to lift coalgebras along square zero
extensions of connective $\bb{E}_\infty$-rings. The deformation theory of $\bb{E}_\infty$-coalgebras
has thus far mainly been investigated by Lurie in~\cite{ellII} by embedding the category of 
\textit{flat} coalgebras fully faithfully into the category of sheaves on $\CAlg(\Sp)^{\cn}$. 
This provides no insight towards approaching Theorem~\ref{result},
as the $k$-homology of a space is never flat over $k$, unless the space is discrete. 
Our approach is instead the following: Suppose we have any map of connective $\bb{E}_\infty$-rings
$q: R\p \to R$. Colimits of coalgebras are computed underlying and the base change $q^\pt$
commutes with colimits and preserves connective modules. Moreover, the category $\cCAlg(\cl{C})$ 
is presentable whenever $\cl{C}$ is a presentably symmetric monoidal category. 
Thus, the adjoint functor theorem provides us with an adjunction

\begin{equation*}\label{adj}
\begin{tikzcd}
	{q^\ast:\rm{cCAlg}(\rm{Mod}_{R\p}^{\rm{cn}})} & {\mathrm{cCAlg}(\mathrm{Mod}_R^{\mathrm{cn}}):q_\ast}
	\arrow[""{name=0, anchor=center, inner sep=0}, shift left=2, from=1-1, to=1-2]
	\arrow[""{name=1, anchor=center, inner sep=0}, shift left=2, from=1-2, to=1-1]
	\arrow["\dashv"{anchor=center, rotate=-90}, draw=none, from=0, to=1].
\end{tikzcd}
\end{equation*}

We call $q_\pt$ the (connective) \textit{Weil restriction} along the map $q:R\p \to R$
\footnote{Such an adjunction exists without the connectivity assumptions, which we call
\textit{non-connective} Weil restriction. See Example~\ref{adjoint} for a more in-depth discussion.}.
Note that, since restriction of scalars is only \textit{lax} monoidal, an $R$-coalgebra $A$ 
has no underlying $R\p$-coalgebra, and so the "restriction" of $A$ along $q$ via the right 
adjoint $q_\pt$ is something of interest. These right adjoints are hard to describe in
general, but turn out to be essential for the deformation theory of coalgebras.
More precisely, in Definition~\ref{etale} we describe a full subcategory of
$\rm{cCAlg}(\Mod_R)^\rm{cn}$ on which the unit map $q^\pt q_\pt \to \id$ is an
equivalence for any square zero extension $R\p \to R$.
We call these coalgebras \textit{formally \'etale} and denote the subcategory 
with the superscript $(\blank)^{\rm{f\acute{e}t}}$. We show that Weil restriction 
of formally \'etale coalgebras along $\bb{W}(k)\to k$ behaves like a coalgebraic version 
of the spherical Witt vector construction.

\begin{theorem}\label{mainthm}
    Denote by $q:\bb{W}(k)\to k$ the natural augmentation. 
    Weil restriction along $q$ induces a fully faithful functor
    \[\cl{W}:\rm{cCAlg}(\Mod_{k}^{\rm{cn}})^{\rm{\fet}}
    \to \rm{cCAlg}(\Mod_{\bb{W}(k)}^{\cn, \wedge})^{\fet}.\] 
    Moreover, $\cl{W}(A)$ is, up to contractible choice, the unique coalgebra in connective
    $p$-complete $\bb{W}(k)$-modules with an equivalence $\cl{W}(A)\otimes k \simeq A$.
\end{theorem}

This is proven inductively, by first lifting along the $p$-completion tower from $k$ to 
$\pi_0\W(k)$ and then along the Postnikov tower to $\W(k)$.
Theorem~\ref{result} is immediate from Theorem~\ref{mainthm} once we know that
the $k$-homology of a space is formally \'etale. Let us make precise the notion of formally 
\'etale coalgebra before we proceed.

\endgroup

\setcounter{theorem}{\thetmp} % restore value of theorem counter

\begin{construction}
Let $R$ be a connective $\E_\infty$-ring, $M$ a connective $R$-module and denote by 
$e:R\to R\oplus M$ the 0-section of the split square zero extension. For any 
$A\in \cCAlg(\Mod_R^{\cn})$ we define the \textit{universal $M$-deformation coalgebra} of $A$
as the Weil restriction
\[ \Omega^\infty_A M := e_\pt e^\pt A \in \cCAlg(\Mod_R^{\cn}).\]
From the adjunction we get a natural unit map $\eps: A \to \Omega^\infty_A M$. Moreover, the map
$R\oplus M \to R$ induces a section $\pi: \Omega^\infty_A M \to A$.
\end{construction}

We show that deformation problems involving $A$ are equivalent to giving lifts of the form
\[\begin{tikzcd}
      & {\Omega^\infty_A M} \\
    B & A
    \arrow[from=2-1, to=2-2]
    \arrow["{\pi}", from=1-2, to=2-2]
    \arrow[dashed, from=2-1, to=1-2]
\end{tikzcd}\]
  If we ask that these be uniquely solvable for all $M$ and $B$, the Yoneda lemma leads us to write
  down our main definition.
  
\begin{definition}\label{etale}
A coalgebra $A\in \rm{cCAlg}(\Mod_R^{\cn})$ is called \textit{formally \'etale} if the unit map
$\varepsilon:A \to \Omega^{\infty}_{A}M$ is an equivalence for all $M \in \Mod_R^{\cn}$.
We denote by $\rm{cCAlg}(\Mod_R^{\cn})^{\fet}$ the full subcategory 
spanned by the formally \'etale coalgebras.
\end{definition}


Note that for an algebra $S\in \CAlg(\Mod_R^{\cn})$, trying to impose the analogous
criterion would be asking that the inclusion $S \to S \oplus S\otimes_R M$ is an equivalence,
which is only true if $S=0$. Hence, this may not seem like a reasonable definition at first glance.
To reassure the reader that we are not only talking about the trivial coalgebra, we prove the
following general claim.

\begin{proposition}
    Let $A \in \cCAlg(\Mod_R^{\cn})$ be dualizable such that for the $R$-linear dual 
    $A^\vee\in \CAlg(\Mod_R)$ the relative cotangent complex $L_{A^\vee/R}$ vanishes. 
    Then $A$ is formally \'etale.
\end{proposition}

This allows us to prove Theorem~\ref{result} for compact spaces. To access arbitrary connected
spaces, the crucial step is to show that the homology of Eilenberg-MacLane spaces is formally \'etale.
To this end, we utilize recent work of Bachmann\textendash Burklund. In~\cite{bb} they show that for any field
$k$ of characteristic $p$ and any $V\in \Mod_{\F_p}^{\heartsuit}$, we have for $n\geq 0$ pullback
squares in $\cCAlg(\Mod_k)$ of the form

   \[\begin{tikzcd}
	{k[\Omega^\infty\Sigma^n V]} & {C_k(\Sigma^n k \otimes V)} \\
	k & {C_k(\Sigma^n k \otimes V)}
	\arrow[from=1-1, to=1-2]
	\arrow["{F-1}", from=1-2, to=2-2]
	\arrow[from=1-1, to=2-1]
	\arrow[', from=2-1, to=2-2].
\end{tikzcd}\] 

Here $C_k$ denote the cofree coalgebra and $F-1$ is the Artin \textendash Schreier map.
This is the dual picture to the pushout square on cohomology of Mandell in~\cite{mandell}, which
was utilized by Lurie in~\cite{dag8} to prove that the $\F_p$-cohomology of an arbitrary space 
is formally \'etale. The pullback diagram tells us that homology of Eilenberg \textendash MacLane spaces 
is cofree up to one "co-relation". Since Weil restriction preserves cofree coalgebras,
we show that the co-relation $F=1$ forces the coalgebra $k[\Omega^\infty \Sigma^n V]$ to
be formally \'etale, which allows us to prove our third theorem.

\begingroup
\setcounter{tmp}{\value{theorem}}% store current value of theorem counter
\setcounter{theorem}{2} %assign desired value to theorem counter
\renewcommand\thetheorem{\Alph{theorem}}% locally redefine the representation of the theorem counter

\begin{theorem}\label{etresult}
Let $X \in \cl{S}$ be either connected or compact. Then for any $\F_p$-algebra $R$, the $R$-homology 
$R[X] \in \cCAlg(\Mod_R^{\cn})$ is formally \'etale.
\end{theorem}


\endgroup
\setcounter{theorem}{\thetmp} % restore value of theorem counter
We strongly believe, but are unable to prove at this time, that the connectivity assumption
can be dropped for arbitrary spaces, not just compact ones. 
Recall that a space is called \textit{$p$-complete} if it is local with respect to the functor
$\F_p[\blank]:\cl{S}\to \Mod_{\F_p}$ and denote by $\cl{S}_p\subseteq \cl{S}$ the full 
subcategory spanned by the $p$-complete spaces. Moreover, call a space \textit{nilpotent}
if is connected and $\pi_1$ is a nilpotent group which acts nilpotently on the higher homotopy
groups. Write $\cl{S}_p^{\rm{nilp}}$ for the category of $p$-complete nilpotent spaces.
Combining Theorem~\ref{result} with~\cite[][Theorem 1.2]{bb}
we also deduce the following spherical lift of Mandells theorem.

\begin{corollary}\label{emb}
 Let $k$ be a perfect, separably closed field of characteristic $p$ with spherical
 Witt vectors $\W(k)$. The $p$-complete $\W(k)$-homology functor
 \[ \cl{S}_p^{\rm{nilp}} \to \rm{cCAlg}(\Mod^{\wedge}_{\bb{W}(k)})
 \quad X \mapsto \bb{W}(k)[X]^\wedge_p\]
 is fully faithful. In particular, for any nilpotent space $X$ we have a natural equivalence
 \[ X^\wedge_p\simeq \rm{Map}_{\rm{cCAlg}_{\bb{W}(k)}^\wedge}(\bb{W}(k), \bb{W}(k)[X]^\wedge_p).\]
\end{corollary}

It would be pleasant if we could replace $\W(k)$ by $\S_p^\wedge$ in Corollary~\ref{emb}.
On the nose this not possible as we need to keep track of descent data. In upcoming work, 
we plan to identify this descent datum with the \textit{coalgebra Frobenius}, a dual notion to the
Tate Frobenius of~\cite{tch} and construct and embedding of $\cl{S}^{\rm{nilp}}_p$ into coalgebras
$\cCAlg(\Sp^\wedge_p)$ equipped with a trivialization of the Frobenius action. This would dualize
work of Yuan in~\cite{yuan} and allow us to drop more finiteness assumptions on the spaces we input.


% \subsection{Technical methods}
% \todo{Rewrite Rewrite}
%   To make Definition~\ref{etale} work for us we utilize the setup of deformation theory developed
%   by Lurie in~\cite{dag14}~and~\cite{ha}. In~\cite{dag14} a class of functors
%   $\rm{CAlg}^{\rm{cn}}\to \cl{S}$ is introduced which are well behaved with respect
%   to deformation problems.

% \begin{definition}[Lurie]
% A functor $X:\rm{CAlg}^{\rm{cn}}\to \cl{S}$ is called
% \textit{cohesive} if for any pullback diagram of
%   connective $\bb{E}_{\infty}$-rings
%   \[\begin{tikzcd}
%       {R^\prime} & {S^\prime} \\
%       R & S \arrow[from=1-2, to=2-2] \arrow[from=2-1, to=2-2] \arrow[from=1-1, to=2-1]
%       \arrow[from=1-1, to=1-2]
%     \end{tikzcd}\]
%   which induces surjections $\pi_{0}R \to \pi_{0}S$ and $\pi_{0}S\p \to \pi_{0}S$, the diagram
%   \[\begin{tikzcd}
% 	{X(R\p)} & {X(S\p)} \\
% 	{X(R)} & {X(S)}
% 	\arrow[from=1-1, to=1-2]
% 	\arrow[from=1-2, to=2-2]
% 	\arrow[from=1-1, to=2-1]
% 	\arrow[from=2-1, to=2-2]
% \end{tikzcd}\]
% is a pullback of spaces.
% \end{definition}

% Given a cohesive functor $X$ for each $R$-valued point $A\in X(R)$
% there exists a spectrum $T^{M}_{X_{A}}$ which controls deformations of $A$ along square
% zero extensions of $R$ with fiber $M$.

% \begin{theorem}[Lurie]
%   Let $X: \rm{CAlg}^{\rm{cn}} \to \cl{S}$ be a cohesive functor and $R^{\eta} \to R$ 
%   a square zero extension with fiber $M \in \rm{Mod}_{R}^{\rm{cn}}$.
%   Then for each $A \in X(R)$ there exists a spectrum $T^{M}_{X_{A}}$
%   called the \textit{Tangent Complex} of $X$ at $A$ such that the space of deformations
%   $\rm{X}_{A}^{R^{\eta}} = \rm{fib}_{A}(X(R^{\eta})\to X(R))$ is either
%   empty or a torsor under the grouplike $\E_{\infty}$-monoid $\Omega^{\infty}T^{M}_{X_{A}}$. Moreover,
%   we have an obstruction class in $\pi_{-1}T^{M}_{X_{A}}$ which vanishes if and only
%   if $\rm{X}_{A}^{R^{\eta}}$ is non-empty.
% \end{theorem}

% We use an descent theorem for modules also due to Lurie~\cite[][Theorem 16.2.0.2.]{sag}
% to show that this machinery can be applied to coalgebras.

% \begin{proposition}
%   For any $n\in \bb{N}$ the functor $\rm{CAlg}^{\rm{cn}} \to \cl{S}$ which takes a connective $\bb{E}_{\infty}$-ring
%   $R$ to the space
%   $(\rm{cCAlg}_{R}^{\rm{cn}})^{\Delta^{n}}= \rm{Map}_{\rm{Cat_{\infty}}}(\Delta^{n}, \rm{cCAlg}_{R}^{\rm{cn}})$
%   is cohesive.
% \end{proposition}

% Thus, the moduli of connective coalgebras admit tangent complexes. 
% We characterize formally \'etale coalgebras in terms of the tangent complex as follows.

% \begin{proposition}\label{etalchar}
%     Let $R$ be a connective $\bb{E}_{\infty}$-ring and $A\in \rm{cCAlg}_{R}^{\rm{cn}}$ and write
%     $\cl{X}(R) = \rm{cCAlg}_{R}^{\rm{cn}}$. Then $A$ is formally \'etale if and only if for every
%     $B \in \rm{cCAlg}^{\rm{cn}}_{R}, M \in \rm{Mod}^{cn}$ and every morphism
%     $\varphi:B\to A$, the map $T_{(\cl{X}^{\Delta^1})_\varphi}^{M} \to T_{(\cl{X}^{\Delta^0})_{B}}^{M}$ induced by the evaluation
%     $\cl{X}^{\Delta^1}\rar{\rm{ev}_0} \cl{X}^{\Delta^0}$ is an equivalence.
% \end{proposition}

% This means that giving a lift of a map into a formally \'etale coalgebra is equivalent
% to giving a lift of the source. This formalizes the idea that lifts of formally \'etale
% coalgebras are unique and functorial.

% To get from this to Theorem~\ref{mainthm}, we prove two different completeness
% results for coalgebras. Concretely, given a perfect $\F_p$-algebra $k$ and it's spherical Witt vectors
% $W(k)$, the ring $\pi_0W(k)$ is a $p$-complete lift of $k$ and hence given as the limit
% \[  \pi_0W(k)= \lim \left( \cdots \to \pi_0W(k)/p^{3}
% \to \pi_0W(k)/p^{2} \to \pi_0W(k)/p = k \right)\]
% where each map is a square zero extension. Hence, if we can lift against square zero extensions we
% can lift inductively to the limit over $\rm{cCAlg}_{\pi_0W(k)/p^n}$. However, the natural map
% \[ \rm{cCAlg}_{\pi_0W(k)}\to \flim \rm{cCAlg}_{\pi_0W(k)/p^n}\]
% is \textit{not} an equivalence. This is where \textit{$p$-complete} coalgebras come in to play.

%  \begin{definition}
%    Let $R$ be an $\bb{E}_{\infty}$-ring. We define the $\infty$-category of $p$-complete
%    $R$-coalgebras as
%    \[ {(\rm{cCAlg}_{R})}^{\wedge}_{p}:= \rm{cCAlg}({(\rm{Mod}_{R})}^{\wedge}_{p}).\]
%  \end{definition}

%  Here, $(\rm{Mod}_{R})^{\wedge}_{p}$ denotes the $\infty$-category of $p$-complete 
%  $R$-modules equipped with the symmetric monoidal structure given by the $p$-completed
%  tensor product. We prove that $p$-complete coalgebras are suitable for deformation
%  theoretic questions and in fact the correct notion if we want
%  to pass from characteristic $p$ to something $p$-adic inductively.

%  \begin{proposition}
%    For every $n\in \bb{N}$ the functor
%    \[ \rm{CAlg}^{\rm{cn}} \to \cl{S} \qquad R\mapsto [{(\rm{cCAlg}_{R})}^{\wedge}_{p}]^{\Delta^{n}}\]
%    is cohesive. Moreover, the assignment $A \mapsto A\otimes_{\pi_0W(k)}\pi_0W(k)/p^{n}$ induces an
%    equivalence of $\infty$-categories
%     \[(\rm{cCAlg}_{\pi_0W(k)})_{p}^{\wedge} \rar{\sim} \flim \rm{cCAlg}_{\pi_0W(k)/p^{n}}. \]
%  \end{proposition}

% Similarly, $W(k)$ is given by the limit of the Postnikov-tower
% \[W(k)= \lim \left( \dots \to \tau_{\leq2}W(k) \to \tau_{\leq 1} W(k)
% \to \tau_{\leq0}W(k) = \pi_0W(k)\right),\]
%   where each map $\tau_{\leq n+1}W(k) \to \tau_{\leq n} W(k)$ is a square zero extension with fiber
%   $\pi_{n+1}W(k)[n+1]$. Thus, to be able to lift inductively from $\pi_0W(k)$ to $W(k)$
%   we prove the following.

%   \begin{proposition}
%     For every connective $\bb{E}_{\infty}$-ring $R$, the truncation functors
%     $\rm{Mod}_{R}\to \rm{Mod}_{\tau_{\leq n} R}$ induce equivalences of categories
%     \[ \rm{cCAlg}_{R}^{\rm{cn}} \rar{\sim} \flim \rm{cCAlg}^{\rm{cn}}_{\tau_{\leq n}R}\]
%     \[ (\rm{cCAlg}_{R}^{\rm{cn}})^{\wedge}_{p} 
%     \to\flim_{n} (\rm{cCAlg}_{\tau_{\le n}R}^{\rm{cn}})^{\wedge}_{p}.\]
%   \end{proposition}
%   Having shown this, we use Proposition~\ref{etalchar} inductively to prove Theorem~\ref{mainthm}.
  
\subsection*{Outline}

We proceed along the following structure:
In Section 1 we recall the definition and basic properties of $\bb{E}_{\infty}$-coalgebras
and discuss some relevant coalgebraic right adjoints.
In Section 2 we review the setup of deformation theory developed by Lurie
in~\cite{dag14} and~\cite{ha}. We recall the notions of square zero extensions of
$\bb{E}_{\infty}$-rings and discuss cohesive and nilcomplete functors.
We then define the tangent complex of a cohesive functor and give proofs of some important facts
about its behavior.
In Section 3 we prove that the moduli of coalgebras are cohesive and thus we can analyze them using
the machinery discussed in the previous section. We then introduce formally \'etale coalgebras
and prove that they can be lifted uniquely and functorially against square zero extensions.
In Section 4 we define spherical and $p$-typical Witt vector style functors and prove that the
homology of any connected space is formally \'etale.

\subsection*{Conventions}
 Throughout the text, we use the following conventions:
\begin{itemize}
\item[(1)]  By category we always mean $(\infty,1)$-category and refer to $(1,1)$-categories as 
            $1$-categories. The text is \textit{model agnostic}, that is we make
            no reference to any specific model for the theory of $(\infty, 1)$-categories.
\item [(2)] We choose three nested Grothendieck universes $\cl{U} \subseteq \cl{V} \subseteq \cl{W}$,
            and refer to categories built from them as small categories, categories and
            large categories, respectively. We denote by $\rm{Cat}_{\infty}$ the large category
            of categories and disregard size issues from here on out.
\item[(3)] We denote the category of spaces by $\cl{S}$ and the category of spectra by $\rm{Sp}$.
\item[(4)] If $A, B$ are objects in some category $\cl{C}$, we use the words map and morphism
           $A\to B$ interchangeably to mean a point in the mapping space
           $\rm{Map}_{\cl{C}}(A,B) \in \cl{S}$. If moreover $\cl{C}$ is a stable category,
           we regard it as enriched over the category of spectra and write
           $\rm{map}_{\cl{C}}(A,B) \in \rm{Sp}$ for the mapping spectrum.
\item[(6)] By (co)algebra we always mean $\bb{E}_{\infty}$-(co)algebra. If we want to refer to
           a 1-categorical version of some gadget we call them \textit{discrete}.
\end{itemize}

\subsection*{Acknowledgments}
 This paper is based on my Master's thesis which was advised by Thomas Nikolaus and Achim Krause. I would
 like to thank them for guiding this project and being wonderful teachers throughout the years. I want
 to especially thank Achim Krause for many insightful discussions and his tremendous
 patience with my questions. The proof of the final results added in this version emerged from
 discussions with Robert Burklund and Tom Bachmann and utilizes their recent work on coalgebras.
 I am grateful for their input and fearless treatment of pro-objects. I was supported by the Danish
 National Research Foundation through the Copenhagen Center for Geometry and Topology (DNRF 151) for
 the later revisions on this work.
\addtocontents{toc}{\protect\setcounter{tocdepth}{2}}