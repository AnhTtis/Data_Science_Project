We now apply the machinery reviewed in the previous section to study various deformation theoretic
questions about coalgebras in the category of spectra. To this end, we first prove that the functor
$\rm{CAlg}^{\rm{cn}} \to \rm{Cat}_{\infty}$ which takes a connective $\bb{E}_{\infty}$-ring spectrum $R$
to the category $\rm{cCAlg}_{R}^{\rm{cn}}$ commutes with small pullbacks and limits of Postnikov
towers. We then show that our deformation problems against square zero extensions with fiber
$M$ are equivalent to lifting problems against a certain map of coalgebras $\Omega^{\infty}_{A}(M)\to A$.
We define formally \'etale coalgebras as those for which this map is an equivalence and
show that, if $A$ is dualizable and $A^{\vee}$ is formally \'etale, then $A$ is formally \'etale
as well. We then discuss the main deformation problem of constructing lifts along the base change
$\rm{cCAlg}_{R^{\eta}}^{\rm{cn}} \to \rm{cCAlg}_{R}^{\rm{cn}}$ for some square zero extension $R^{\eta}\to R$ with
fiber $M$. We show that if $A\in \rm{cCAlg}_{R}^{\rm{cn}}$ is formally \'etale then this problem
admits a contractible space of solutions and that these are again formally \'etale.
Moreover, we show that the assignment of $A$ to its essentially unique
lift $A^{\eta} \in \rm{cCAlg}_{R^{\eta}}^{\rm{cn}}$ refines to a fully faithful
functor $\rm{cCAlg}_{R}^{\rm{cn}, \rm{f\acute{e}t}} \to \rm{cCAlg}_{R^{\eta}}^{\rm{cn}}$.

\subsection{Descent and completion}

Our entire discussion rests on the following descent result for modules.

\begin{theorem}[Lurie]\label{Mod}
  Suppose we have a pullback of connective $\bb{E}_{\infty}$-rings
  \[\begin{tikzcd}
      {R^\prime} & {S^\prime} \\
      R & S \arrow[from=1-1, to=1-2] \arrow[from=1-2, to=2-2]
      \arrow[from=1-1, to=2-1] \arrow[from=2-1, to=2-2]
    \end{tikzcd}\]
  such that one of the maps $\pi_{0}R \to \pi_{0}S$, $\pi_{0}S\p \to \pi_{0}S$ is surjective.
  Then the natural map
  \[ \rm{Mod}^{\rm{cn}}_{R\p} \to \rm{Mod}_{R}^{\rm{cn}} \times_{\rm{Mod}_{S}^{\rm{cn}}} \rm{Mod}_{S\p}^{\rm{cn}}\]
  is an equivalences of categories with inverse taking a point in the pullback $(M,N,h)$, consisting of
  $M \in\rm{Mod}_{R}^{\rm{cn}}, N \in \rm{Mod}_{S\p}^{\rm{cn}}$ and a homotopy $h: M\otimes_{R}S\simeq N \otimes_{S\p}S$, to
  $M \times_{M \otimes_{R}S} N$ with the induced $R\p$-module structure.
\end{theorem}
\begin{proof}
\cite[][Theorem 16.2.0.2.]{sag}
\end{proof}

\begin{corollary}\label{edescent}
In the setting of Theorem~\ref{Mod}, the map
  \[ \rm{Mod}^{\rm{cn}}_{R\p} \to \rm{Mod}_{R}^{\rm{cn}} \times_{\rm{Mod}_{S}^{\rm{cn}}} \rm{Mod}_{S\p}^{\rm{cn}}\]
  induces equivalences
  \[ \rm{cCAlg}_{R\p}^{\rm{cn}} \rar{\sim} \rm{cCAlg}^{\rm{cn}}_{R} \times_{\rm{cCAlg}^{\rm{cn}}_{S}}
    \rm{cCAlg}^{\rm{cn}}_{S\p},\]
  \[ \rm{CAlg}_{R\p}^{\rm{cn}} \rar{\sim} \rm{CAlg}^{\rm{cn}}_{R} \times_{\rm{CAlg}^{\rm{cn}}_{S}}
    \rm{CAlg}^{\rm{cn}}_{S\p}.\]
\end{corollary}
\begin{proof}
Recall that by Proposition~\ref{calg} the forgetful functor
\[ \rm{CAlg}(\rm{Cat}_{\infty}) \to \rm{Cat}_{\infty} \]
  commutes with limits. In particular, a pullback of presentably monoidal
  categories and strong monoidal functors
\[\begin{tikzcd}
	{\cl{C}\times_{\cl{D}} \cl{E}} & {\cl{E}} \\
	{\cl{C}} & {\cl{D}}
	\arrow[from=2-1, to=2-2]
	\arrow[from=1-2, to=2-2]
	\arrow[from=1-1, to=2-1]
	\arrow[from=1-1, to=1-2]
\end{tikzcd}\]
inherits a natural symmetric monoidal structure. Moreover, by Lemma~\ref{limits} we have
an equivalence
\[\rm{cCAlg}(\cl{C}\times_{\cl{D}} \cl{E}) \simeq \rm{cCAlg}(\cl{C})\times_{\rm{cCAlg}(\cl{D})}\rm{cCAlg}(\cl{E}),\]
so the first claim follows. The argument for $\rm{CAlg}^{\rm{cn}}$ is exactly the same.
\end{proof}

This tells us that for a connective $R$-coalgebra $A$, a connective $S\p$-coalgebra $B$
and an equivalence $A\otimes_{R} S \simeq B \otimes_{S\p} S$ the pullback of spectra $A \times_{A \otimes_{R}S}B$
inherits a natural $R\p$-coalgebra structure and in fact every connective $R\p$-coalgebra is of this form.
We can immediately apply this to the deformation theory of coalgebras.
Let $R^{\eta} \to R$ be a square zero extension classified by a derivation $\eta:R \to M[1]$
and let $A \in \rm{cCAlg}_{R}^{\rm{cn}}$ be a coalgebra. Then by Theorem~\ref{Mod}, the pullback
of ring spectra
\[\begin{tikzcd}
	{R^\eta} & R \\
	R & {R\oplus M[1]}
	\arrow[from=1-1, to=1-2]
	\arrow[from=1-1, to=2-1]
	\arrow["{(\id,\eta)}"', from=2-1, to=2-2]
	\arrow["{(\id,0)}", from=1-2, to=2-2]
\end{tikzcd}\]
induces a pullback of $\infty$-categories
\[\begin{tikzcd}
	{\rm{cCAlg}^{\rm{cn}}_{R^\eta}} & {\rm{cCAlg}^{\rm{cn}}_R} \\
	{\rm{cCAlg}^{\rm{cn}}_R} & {\rm{cCAlg}^{\rm{cn}}_{R\oplus M[1]}}
	\arrow[from=1-1, to=1-2]
	\arrow[from=1-2, to=2-2]
	\arrow[from=1-1, to=2-1]
	\arrow[from=2-1, to=2-2]
\end{tikzcd}.\]
Thus, a $R^{\eta}$-coalgebra is given by a $R$-coalgebra $A$ together with an equivalence of the two base changes
$A \otimes_{0} (R\oplus M[1]) \simeq A \otimes_{\eta} (R\oplus M[1])$. If such an equivalence exists, the $R^{\eta}$-coalgebra is
classified by a map $d:A \to A \otimes M[1]$ in the sense that its underlying spectrum can be
computed as the pullback
\[\begin{tikzcd}
	{A^d} & A \\
	A & {A \oplus A\otimes M[1]}
	\arrow[from=1-1, to=1-2]
	\arrow[from=1-1, to=2-1]
	\arrow["{(\id, d)}"', from=2-1, to=2-2]
	\arrow["{(\id, 0)}", from=1-2, to=2-2].
\end{tikzcd}\]
However, the module $A \oplus A \otimes M[1]$ does not admit a natural coalgebra structure, exhibiting the map
$A \rar{(\id,0)} A \oplus A\otimes M[1]$ as a coalgebra morphism. Even if this was the case, this would not recover
the coalgebra structure on $A^{d}$ since limits of coalgebras are not formed underlying.
This is one of the main defects of the deformation theory of coalgebras compared the one of algebras.
For $S \in \rm{CAlg}_{R}^{\rm{cn}}$, we can forget the $R\oplus M[1]$-algebra structure
on $S\otimes_{R}M[1]$ and exhibit the map
$S \rar{(\id, 0)} S \oplus S \otimes_{R}M[1] $ as a morphism of underlying $R$-algebras. This fact is
exploited by Lurie in~\cite[][Proposition 7.4.2.5]{ha} to give a full classification of connective
$\bb{E}_{\infty}$-algebras over a square zero extension in terms of algebras over the ground ring. However for
coalgebras, there is no forgetful functor which gives us the ``underlying'' $R$-coalgebra, only the
mysterious right adjoint $\rm{cCAlg}_{R\oplus M[1]} \to \rm{cCAlg}_{R}$. The goal of this section is to understand
how big of a problem this poses and which parts of the story carry over regardless.

\begin{lemma}\label{conn}
  Let $\dots \to  E_{2} \to E_{1} \to E_{0}$ be a diagram of spectra. Suppose we are given $L \ge 0$ such
  that for all $\ell\p\ge\ell \ge L$ the map $E_{\ell\p}\to E_{\ell}$ is $m$-connective. Then for any $\ell \ge L$ the map
  \[ \flim_{n} E_{n}\to E_{\ell}\]
  is $m-1$-connective.
\end{lemma}

\begin{proof}
  Writing $F_{\ell\p,\ell}= \rm{fib}(E_{\ell\p}\to E_{\ell})$ and $F_{\ell}= \rm{fib}(\lim_{n}E_{n}\to E_{\ell})$ we want
  to show that $F_{\ell}$ is $m$-connective. Indeed, since limits are exact, we have that
  \[F_{\ell} \simeq \lim_{\ell\p >\ell}F_{\ell\p,\ell} \simeq \rm{fib}\left( \prod_{\ell\p >\ell} F_{\ell\p,\ell}\to \prod_{\ell\p>\ell} F_{\ell\p -1, \ell}\right ).\]
  Thus, since $\rm{Sp}_{\geq m}$ is closed under products and the fiber of a map of $m$-connective spectra
  is $m-1$-connective, the claim follows.
\end{proof}

\begin{proposition}
  Let $R$ be a connective $\bb{E}_{\infty}$-ring. Then the natural map
  \[\rm{Mod}_{R}^{\rm{cn}} \to \flim_{n} \rm{Mod}_{\tau_{\le n}R}^{\rm{cn}} \quad M \mapsto M \otimes_{R} \tau_{\le n}R\]
  is an equivalence of categories.
\end{proposition}
\begin{proof}
  We write $R_{n}:= \tau_{\le n}R$. The functor admits a right adjoint which
  takes $(M_{n})\in \flim_{n} \rm{Mod}_{R_{n}}^{\rm{cn}}$ to the limit $\lim_{n} M_{n}$ which inherits
  a natural action by $\lim_{n} R_{n}\simeq R$. Now let $N\in \rm{Mod}_{R}$. Since taking limits is exact,
  the counit of the adjunction sits in a fiber sequence
  \[\lim_{n} \rm{fib}(N \to N \otimes_{R} R_{n}) \to N \rar{\eta} \lim_{n} (N \otimes_{R} R_{n}).\]
  where we compute for the left hand term that
  \[ \rm{fib}(N \to N \otimes_{R} R_{n}) \simeq \rm{fib}(N \otimes_{R} R \to N \otimes_{R} R_{n}) \simeq N \otimes_{R} \rm{fib}(R \to R_{n}).\]
  Now, since $R_{n}= \tau_{\leq n} R$, the connectivity of $\rm{fib}(R\to R_{n})$ increases with $n$. Hence,
  since $R$ and $N$ are connective, so does the connectivity of the tensor product
  $N \otimes_{R} \rm{fib}(R \to R_{n})$ which implies that $\flim_{n}N \otimes_{R} \rm{fib}(R \to R_{n}) \simeq 0$. Thus,
  the counit $N \to \lim_{n}(N \otimes_{R} R_{n})$ is an equivalence. \\
  Now let $(M_{n}) \in \flim_{n}\rm{Mod}_{R_{n}}^{\rm{cn}}$ and write $M= \lim_{n} M_{n}$.
  We need to show that the natural map
  \[ \eps_{k}:M \otimes_{R} R_{k} \to R_{k}\]
  is an equivalence for each $k$. We do this by showing that it is $m$-connective for any $m\ge 0$.
  Indeed, for any such $m$ there exists an integer $L$ such that for all
  $\ell \geq \ell\p > L$ the natural map $R_{\ell} \to R_{\ell\p}$ is $m$-connective. Since
  $M_{\ell\p}\simeq M_{\ell}\otimes_{R_{\ell}} R_{\ell\p}$  we have a fiber sequence
  \[ M_{\ell}\otimes_{R_{\ell}} (\rm{fib}(R_{\ell}\to R_{\ell\p})) \to M_{\ell} \to M_{\ell\p}.\]
  Hence, since $\rm{fib}(R_{\ell}\to R_{\ell\p})$ is $m$-connective and $R_{\ell}$ and $M_{\ell}$ are connective, the tensor
  product $M_{\ell}\otimes_{R_{\ell}} \rm{fib}(R_{\ell}\to R_{\ell\p})$ is $m$-connective as well. Thus, for fixed $m$ and $k$ we
  may apply Lemma~\ref{conn} to obtain  $\ell>k$ such that the maps $M \to M_{\ell}$ and $R\to R_{\ell}$ are
  both $m$-connective. Finally, the map
  \[ \eps_{k}: M\otimes_{R}R_{k} \to M_{\ell}\otimes_{R_{\ell}}R_{k} \simeq M_{k}\]
  is given by the colimit of the induced map between the bar resolutions
\[\begin{tikzcd}
	\vdots & \vdots \\
	{M\otimes R \otimes R_k} & {M_\ell\otimes R_\ell \otimes R_k} \\
	{M \otimes R_k} & {M _\ell \otimes R_k}
	\arrow[from=1-1, to=2-1]
	\arrow[from=1-2, to=2-2]
	\arrow[shift left=2, from=2-1, to=3-1]
	\arrow[shift left=2, from=2-2, to=3-2]
	\arrow[from=3-1, to=3-2]
	\arrow[from=2-1, to=2-2]
	\arrow[shift right=3, from=1-1, to=2-1]
	\arrow[shift left=3, from=1-1, to=2-1]
	\arrow[shift right=3, from=1-2, to=2-2]
	\arrow[shift left=3, from=1-2, to=2-2]
	\arrow[shift right=2, from=2-1, to=3-1]
	\arrow[shift right=2, from=2-2, to=3-2]
\end{tikzcd}.\]
Denote by $F_{n}$ the fiber of the map $M \otimes R^{\otimes n}\otimes R_{k} \to M_{\ell}\otimes R_{\ell}^{\otimes n}\otimes R_{k}$. Since the tensor product
of $m$-connective maps is again $m$-connective, the fiber $F_{n}$ is $m$-connective. Thus, by exactness
of colimits, we obtain a fiber sequence
\[\colim F_{n} \to M \otimes_{R} R_{k} \rar{\eps_{k}} M_{k}\]
and finally, since taking colimits preserves connectivity, this shows that the map
$\eps_{k}$ is $m$-connective. Since $m$ was arbitrary, the map $\eps_{k}$ is in fact an equivalence
which completes the proof.
\end{proof}

\begin{corollary}\label{nilcomplete}
  Let $R$ be a connective $\bb{E}_{\infty}$-ring, then the natural maps
  \[ \rm{cCAlg}_{R}^{\rm{cn}} \to \flim_{n} \rm{cCAlg}_{\tau_{\le n}R}^{\rm{cn}}\]
  \[ \rm{CAlg}_{R}^{\rm{cn}} \to \flim_{n} \rm{CAlg}_{\tau_{\le n}R}^{\rm{cn}}\]
  are equivalences.
\end{corollary}

\begin{proof}
  Analogously to Proposition~\ref{Mod}, this follows by observing that the map
  \[\rm{Mod}_{R}^{\rm{cn}} \to \flim_{n} \rm{Mod}_{\tau_{\le n}R}^{\rm{cn}} \quad M \mapsto M \otimes_{R} \tau_{\le n}R\]
  from Proposition~\ref{cohesive} is strong monoidal.
\end{proof}

\begin{corollary}\label{cohesive}
  For any $n\in \bb{N}$ the functor $\rm{CAlg}^{\rm{cn}} \to \cl{S}$ which takes a connective $\bb{E}_{\infty}$-ring
  $R$ to the space
  $(\rm{cCAlg}_{R}^{\rm{cn}})^{\Delta^{n}}:= \rm{Map}_{\rm{Cat_{\infty}}}(\Delta^{n}, \rm{cCAlg}_{R}^{\rm{cn}})$
  is cohesive and nilcomplete.
\end{corollary}

\begin{proof}
  This is clear since the functors $\rm{Map}_{\rm{Cat}_{\infty}}(\Delta^{n}, \blank): \rm{Cat}_{\infty} \to \cl{S}$
  commute with limits.
\end{proof}
In particular, we immediately obtain the existence of a tangent complex which controls lifts of
coalgebras.

\begin{corollary}\label{deformations}
  Let $A \in X(R)=(\rm{cCAlg}^{\rm{cn}}_{R})^{\Delta^{0}}$ and let $R^{\eta} \to R$ be a square zero extension
  classified by a derivation $R \rar{\eta} M[1]$. Then the sequence $\{X_{A}^{R\oplus M[n]}\}_{n \in \bb{N}}$ defines
  a spectrum $T^{M}_{X_A}$, yielding an obstruction class $A^{\eta} \in \pi_{-1}T^{M}_{X_{A}}$ such that,
  the space of deformations of $A$ to a $R^{\eta}$-coalgebra is non-empty if and only if $A^{\eta}$
  vanishes and, in this case, is a torsor under $\Omega^{\infty}T^{M}_{X_{A}} = X^{R\oplus M}_{A}$.
\end{corollary}
\begin{proof}
  This is exactly Proposition~\ref{def}.
\end{proof}

\subsection{Formally \'etale coalgebras and derivations}

We now want to introduce a class of coalgebras for which all these lifting problems have a contractible
space of solutions and find an alternate description of the Tangent Complex via a notion of derivation.
We begin with the following general observation.

\begin{construction}\label{pi}
  Let $\cl{C},\cl{D}$ be $\infty$-categories and $f^{\pt}: \cl{C}\to \cl{D}$ with right adjoint
  $f_{!}: \cl{D}\to \cl{C}$. Moreover, suppose we have $g^{\pt}:\cl{D}\to \cl{C}$ such
  that $g^{\pt}f^{\pt} \simeq \id_{\cl{C}}$ and consider the natural transformation
  \[\pi: f_{!}f^{\pt} \rar{\sim} g^{\pt} f^{\pt}f_{!}f^{\pt}\to  g^{\pt}f^{\pt} \rar{\sim} \id_{\cl{C}}\]
  defined as the whiskering of the counit $\eps:f^{\pt}f_{!} \to \id$ as in the diagram
\[\begin{tikzcd}
	{\cl{C}} & {\cl{D}} & {\cl{D}} & {\cl{C}}
	\arrow[""{name=0, anchor=center, inner sep=0}, "{f^\pt f_!}", curve={height=-12pt}, from=1-2, to=1-3]
	\arrow[""{name=1, anchor=center, inner sep=0}, "\id"', curve={height=12pt}, from=1-2, to=1-3]
	\arrow["{f^\pt}", from=1-1, to=1-2]
	\arrow["{g^{\pt}}", from=1-3, to=1-4]
	\arrow[shorten <=3pt, shorten >=3pt, Rightarrow, from=0, to=1].
\end{tikzcd}\]
Unraveling the definition we see that for each $B,A \in \cl{C}$ the composition
\[ \rm{Map}_{\cl{C}}(B, f_{!}f^{\pt} A) \rar{\sim}\rm{Map}_{\cl{D}}(f^{\pt}B, f^{\pt} A)
 \rar{g^{\pt}} \rm{Map}_{\cl{C}}(B, A)\]
takes $\psi: B \to f_{!}f^{\pt}A$ to the composite $\pi_{A} \circ \psi$. Thus, for each $\varphi:B \to A$ we have an
equivalence between the fiber
\[ \rm{fib}_{\varphi}\left(\rm{Map}_{\cl{D}}(f^{\pt}B, f^{\pt}A) \rar{g^{\pt}}
    \rm{Map}_{\cl{C}}(B, A) \right)\]
and the mapping space
\[ \rm{Map}_{\cl{C}_{/A}}((B\rar{\varphi} A), (f_{!}f^{\pt}A \rar{\eta_{A}} A)).\]
\end{construction}

\begin{example}\label{maplifts}
  Let $R \in \rm{CAlg}$ and let $X(\blank)=\rm{cCAlg}_{\blank}$. Moreover, let $A,B \in \rm{cCAlg}_{R}$,
  $M \in \rm{Mod}_{R}$ and suppose we are given a map of coalgebras $ \varphi : B \to A$.
  Denote the natural inclusion as $f_{M}: R \to R\p$ together with it's section $g_{M}: R \oplus M \to R$.
  Recall the adjunction
\[\begin{tikzcd}
	{\rm{cCAlg}_R} & {\rm{cCAlg}_{R\oplus M}}
	\arrow[""{name=0, anchor=center, inner sep=0}, "{f_M^\pt}", shift left=2, from=1-1, to=1-2]
	\arrow[""{name=1, anchor=center, inner sep=0}, "{f_{M,!}}", shift left=2, from=1-2, to=1-1]
	\arrow["\dashv"{anchor=center, rotate=-90}, draw=none, from=0, to=1]
\end{tikzcd}\]
  described in Remark~\ref{adjoint} and  set
  \[\Omega^{\infty}_{A}(M):= f_{M,!}f^{\pt}_{M}A \in \rm{cCAlg}_{R}.\]
  We want to analyze the space of lifts of $\phi$ to a map $f^{\pt}_{M} B \to f^{\pt}_{M} A$.
  Since $g_{M}f_{M} = \id$, we also have $g_{M}^{\pt} f_{M}^{\pt}=\id$.
  Thus, Construction~\ref{pi} gives a natural map of $R$-coalgebras
        \[ \pi_{A}:\Omega_{A}(M) \to A \]
  exhibiting the fiber
  \[\rm{fib}_{\varphi}\left(\rm{Map}_{\rm{cCAlg}_{R\oplus M}}(f^{\pt}_{M} B, f^{\pt}_{M} A )
      \to \rm{Map}_{\rm{cCAlg}_{R}}(B, A)\right)\]
  as the space of lifts in the diagram
\[\begin{tikzcd}
	& {\Omega^{\infty}_{A}(M)} \\
	B & A
	\arrow[from=2-1, to=2-2]
	\arrow[dashed, from=2-1, to=1-2]
	\arrow["{\pi_{A}}", from=1-2, to=2-2].
\end{tikzcd}\]
Moreover, if we assume that $A$ is connective, the problem of constructing a lift of the coalgebra
$A$ itself can now be described as follows: Let $X= (\rm{cCAlg}^{\rm{cn}}_{\blank})^{\Delta^{1}}$ and
$A \in X(R)$. Then since $X$ is cohesive by Corollary~\ref{cohesive} we have an equivalence
\[ X_{A}^{R\oplus M} \simeq \Omega X_{A}^{R\oplus M[1]}
  =  \rm{fib}_{\id_{A}}(\rm{Map}_{\rm{cCAlg}_{\rm{R\oplus M[1]}}}( f^{\pt}_{M}A, f^{\pt}_{M}A )
  \to  \rm{Map}_{\rm{cCAlg}}(A, A)). \]
Hence, $X_{A}^{R\oplus M}$ is given by the space of lifts in the diagram
\[\begin{tikzcd}
	& {\Omega_{A}(M[1])} \\
	A & A
	\arrow["\id", from=2-1, to=2-2]
	\arrow["\pi_{A}",from=1-2, to=2-2]
	\arrow[dashed, from=2-1, to=1-2].
\end{tikzcd}\]
\end{example}

Suppose we have an $R$-coalgebra $A$ for which the lifting problems above admit a contractible
space of solutions. Then, for every $B\in \rm{cCAlg}_{R}$ composing with $\pi_{A}$ gives an equivalence
  \[\rm{Map}_{\rm{cCAlg}_{R}}(B, \Omega^{\infty}_{A}(M)) \rar{\sim} \rm{Map}(B, A).\]
  Hence, by the Yoneda Lemma the map $\Omega^{\infty}_{A}(M)\rar{\pi_{A}} A$ is an equivalence for all
  $M\in \rm{Mod}_{R}$.  Moreover, since the composition
  \[ A \rar{\eps} \Omega^{\infty}_{A}M \rar{\pi_{A}} A\]
  is homotopic to  the identity, the counit $A \rar{\eps} \Omega^{\infty}_{A}M$ is an equivalence if and only if $\pi_{A}$ is.
 This motivates the following definition:

\begin{definition}
Let $R$ be a connective $\bb{E}_{\infty}$-ring, then for each $R$-coalgebra $A$ we have a functor
\[ \rm{Mod}_{R}^{\rm{cn}}\to \rm{Mod}_{R} \qquad M \mapsto C_{A}(M):=\rm{cofib}( A \rar{\eps} \Omega^{\infty}_{A}(M) ).\]
We say that $A \in \rm{cCAlg}_{R}$ is \textit{formally \'etale} if $C_{A}(M)\simeq 0$ for all $M$ and denote
the full subcategory spanned by the formally \'etale coalgebras by $\rm{cCAlg}_{R}^{\rm{f\acute{e}t}}
\subseteq \rm{cCAlg}_{R}$. Moreover, given a map $B\rar{\varphi} A$, we define the \textit{space of derivations}
$B\to C_{A}(M)$ as the mapping space
  \[ \rm{Der}_{\varphi}(B, C_{A}(M)):= \rm{Map}_{\rm{cCAlg}_{R/A}}(B, \Omega^{\infty}_{A}M).\]
\end{definition}

\begin{warning}
  The notation $\Omega^{\infty}_{A}M$ is at this point merely suggestive of the dual story in the algebra world.
  There we had seen that, given a map $\varphi:R\to S$ we have equivalences
  \begin{align*}
    \rm{Map}_{\rm{CAlg}_{/S}}(R, S \oplus M)\simeq \rm{Map}_{\rm{CAlg}_{/R}}(R, R\oplus \varphi_{\pt}M)\simeq
    \rm{Map}_{\rm{CAlg}_{/R}}(R, \Omega^{\infty}({\varphi_{\pt}M}))
  \end{align*}
  Where $\Omega^{\infty}$ denotes the infinite loop space map
  \[\rm{Mod}_{R}\simeq \rm{Sp}(\rm{CAlg}_{/R}) \rar{\Omega^{\infty}}  \rm{CAlg}_{/R}\]
  from Proposition~\ref{loops}. However it is at present unclear whether the functor
  \[ \rm{Mod}_{R}\to \rm{cCAlg}_{/A} \qquad M \mapsto \Omega^{\infty}_{A}M\]
  admits a similar description. This is the main obstacle one would need to overcome to deduce the
  existence of a cotangent complex for a coalgebra $A$. We spend some more time on this issue in
  section 6.1.
\end{warning}

A coalgebra $A \in \rm{cCAlg}_{R}$ is formally \'etale if and only $C_{A}(M)$ admits only trivial derivations,
which is the case if and only if the map $\Omega^{\infty}_{A}M \rar{\pi_{A}} A$ is an equivalence for all
$M\in \rm{Mod}_{R}$. Thus, we can think of derivations into $C_{A}(M)$ as a measuring how far $A$ is from
being formally \'etale. To see that this property is reasonable, i.e.~one that is satisfied
by a nontrivial class of coalgebras, we now consider the case where $A$ is dualizable.

 \begin{proposition}\label{cotangentder}
   Let $R$ be an $\bb{E}_{\infty}$-ring, $A,B\in \rm{cCAlg}_{R}$ with $A$ dualizable and $M \in \rm{Mod}_{R}$.
   Write $R\p = R \oplus M$ and $A_{R\p}$ respectively $B_{R\p}$ for the basechange along the inclusion
   $R \to R\p$. Then for every $\varphi:B \to A$ there is a natural equivalence
  \[ \rm{Der}_{\phi}(B, C_{A}(M)) \simeq \rm{Map}_{A^{\vee}}(L_{A^{\vee}/R}, \varphi^{\vee}_{\pt}\rm{map}_{R}(B, M)).\]
 \end{proposition}
\begin{proof}
  We know from Example~\ref{maplifts} that the space of derivations $B \to C_{A}(M)$ is equivalent
  to the fiber
  \[ F_{\varphi}:=\rm{fib}_{\varphi}\left(\rm{Map}_{\rm{cCAlg}_{R\p}}(B_{R\p}, A_{R\p})
      \to \rm{Map}_{\rm{cCAlg}_{R}}(B, A) \right)\]
  Since $A$ is dualizable, so is $A_{R\p}$ with dual given by $A_{R\p}^{\vee}\simeq A^{\vee}\otimes_{R} R\p$.
  Thus, applying $(\blank)^{\vee}$ we get an equivalence
  \begin{align*}
    \rm{Map}_{\rm{cCAlg}_{R\p}}(B_{R\p}, A_{R\p})&\simeq \rm{Map}_{\rm{CAlg}_{R\p}}(A^{\vee}\otimes_{R}R\p,
    \rm{map}_{R\p}(B_{R\p}, R\p))\\
    &\simeq \rm{Map}_{\rm{CAlg}_{R}}(A^{\vee}, \rm{map}_{R}(B, R\p))
  \end{align*}
  and similarly
  \begin{align*}
    \rm{Map}_{\rm{cCAlg}_{R}}(B,A)\simeq \rm{Map}_{\rm{CAlg}_{R}}(A^{\vee}, B^{\vee}).
  \end{align*}
  Thus, $F_{\varphi}$ is given by
  \begin{align*}
    F_{\varphi}&\simeq \rm{fib}_{\varphi^{\vee}}\left(\rm{Map}_{\rm{CAlg}_{R}}(A^{\vee}, \rm{map}_{R}(B, R\p)) \to
    \rm{Map}_{\rm{CAlg}_{R}}(A^{\vee}, B^{\vee})\right)\\
    &\simeq \rm{Map}_{(\rm{CAlg}_{R})_{/B^\vee}}(A^{\vee}, \rm{map}_{R}(B, R\p)),
  \end{align*}
  i.e.~the space of lifts in the diagram
\[\begin{tikzcd}
	& {\rm{map}_R(B_R,R\p)} \\
	{A^\vee} & {B^\vee}
	\arrow[from=2-1, to=2-2]
	\arrow[from=1-2, to=2-2]
	\arrow[dashed, from=2-1, to=1-2].
\end{tikzcd}\]
Since $R\p\to R$ is a split square zero extension with fiber $M$, the map
$\rm{map}_{R}(B, R\p) \to B^{\vee}$ is a square zero extension as well with fiber given
by $\rm{map}_{R}(B, M)$. Hence, we have that
\begin{align*}
F_{\varphi}\simeq \rm{Map}_{A^{\vee}}(L_{A^{\vee}_{R}/R}, \varphi^{\vee}_{\pt}\rm{map}_{R}(B,M))
\end{align*}
as claimed.
\end{proof}

\begin{corollary}\label{dualetal}
  Let $R$ be an $\bb{E}_{\infty}$-ring and $A\in \rm{cCAlg}_{R}$ be dualizable such that $L_{A^{\vee}/R} \simeq 0$.
  Then $A$ is formally \'etale.
\end{corollary}

\begin{remark}
 It is unclear whether the converse of Corollary~\ref{dualetal} holds. From Proposition~\ref{cotangentder}
 we can only deduce that
\[ \rm{Map}_{A^{\vee}}(L_{A^{\vee}/R}, \varphi_{\pt}\rm{map}_{R}(B, M)) \simeq 0\]
for each coalgebra $B$, $R$-module $M$ and morphism of algebras $\varphi:A^{\vee}\to B^{\vee}$.
\end{remark}

\begin{construction}\label{dersp}
  Let $R$ be a connective $\bb{E}_{\infty}$-ring and $B\rar{\varphi} A$ a morphism of connective
  $R$-coalgebras. Then the functor
  \[ \rm{Mod}_{R}^{\rm{cn}} \to \cl{S} \quad M \mapsto \rm{Der}_{\varphi}(B, C_{A}(M))\]
  is reduced and excisive. Indeed, denoting $X(R) = \rm{cCAlg}_{R}^{\rm{cn}}$, we may write
  \[\rm{Der}_{\varphi}(B, C_{A}(M))\simeq \rm{fib}_{(A_{R\oplus M}, B_{R\oplus M})}\left((X^{\Delta^{1}})^{R\oplus M}_{\varphi} \rar{\ev_{0} ,\ev_{1}}
   (X^{\Delta^{0}})^{R\oplus M}_{A}\times (X^{\Delta^{0}})^{R\oplus M}_{B}\right),\]
  so both properties follow from Proposition~\ref{redex} since $X^{\Delta^{n}}$ is cohesive for every $n$.
  Thus, as in Construction~\ref{tangent}, we can associate to the functor $\rm{Der}_{\varphi}(B, C_{A}(M))$
  a spectrum which we denote $\rm{der}_{\varphi}(B, C_{A}(M))$.
\end{construction}

\begin{lemma}\label{shift}
  Let $X(R)= (\rm{cCAlg}_{R}^{\rm{cn}})^{\Delta^{0}}$ and $M$ be a connective $R$-module. Then we have
  a natural equivalence $T^{M}_{X_{A}}\simeq \rm{der}(A, C_{A}(M[1]))$.
\end{lemma}
\begin{proof}
  The space $\rm{X}_{A}^{R \oplus M[1]}$ is pointed by the coalgebra $A\p:=A \otimes_{R} (R\oplus M[1])$, and since
  $X$ is cohesive, we have that $\rm{X}_{A}^{R\oplus M} \simeq \Omega_{A\p}X_{A}^{R\oplus M[1]}$.
  This loop space is then given by
  \begin{align*}
    \Omega_{A\p}X_{A}^{R\oplus M[1]} & \simeq \rm{fib}_{\id_{A}} \left(\rm{Map}_{\rm{cCAlg}_{R\oplus M[1]}}(A\p, A\p )
                             \to \rm{Map}_{\rm{cCAlg}_{R}}(A, A) \right)\\
    &\simeq \rm{Der}_{\id}(A, C_{A}(M[1])).
  \end{align*}
  This equivalence is natural in $M$ and thus induces an equivalence of the associated spectra
  $T_{X_{A}}^{R\oplus M}\simeq \rm{der}(A, C_{A}(M[1]))$, as claimed.
\end{proof}

\begin{corollary}\label{defobject}
  Let $A\in \rm{cCAlg}_{R}^{\rm{cn}}$ be formally \'etale and $R^{\eta} \to R$ a square zero extension.
  Then the fiber
  \[\rm{fib}_{A}(\rm{cCAlg}^{\rm{cn}}_{R^{\eta}} \to \rm{cCAlg}^{\rm{cn}}_{R})\]
  is contractible, i.e.~$A$ admits an essentially unique lift to a $R^{\eta}$-coalgebra.
\end{corollary}

We now discuss how to lift maps of coalgebras, with the goal of making lifts of formally
\'etale coalgebras functorial. To this end, we first compute the fibers of $\rm{cCAlg}_{(\blank)}^{\Delta^{1}}$
in terms of the spaces of derivations introduced above.

\begin{proposition}\label{fibpb}
  Let $X(\blank)=(\rm{cCAlg}_{\blank}^{\rm{cn}})^{\Delta^{1}}$, $R$ an $\bb{E}_{\infty}$-ring
  and $(A\rar{\varphi} B) \in X(R)$. Then for every connective $R$-module $M$ the fiber
  $X^{R\oplus M}_{\varphi}$ can be computed as the pullback
  \[ X^{R\oplus M}_{\varphi} \simeq \rm{Der}_{\rm{id}}(B, C_{B}(M[1])) \times_{\rm{Der}_{\varphi}(B, C_{A}(M[1]))}
    \rm{Der}_{\rm{id}}(A, C_{A}(M[1])).\]
\end{proposition}
\begin{proof}
Since $X^{\Delta^{1}}$ is cohesive, the space $(X^{\Delta^{1}})_{\varphi}^{R\oplus M}$ fits into a pullback diagram
\[\begin{tikzcd}
	{X_{\varphi}^{R\oplus M}} & \pt \\
	\pt & {X_{\varphi}^{R\oplus M[1]}}
	\arrow[from=1-1, to=1-2]
	\arrow[from=1-1, to=2-1]
	\arrow[from=2-1, to=2-2]
	\arrow[from=1-2, to=2-2]
\end{tikzcd}\]
where the maps $\pt \to X_{\varphi}^{R\oplus M[1]}$ are given by the base changed
morphisms $\varphi \otimes_{R} (R \oplus M[1])$. \\
If $\cl{C}$ is any category and $(\psi:A\to B)\in \cl{C}^{\Delta^{1}}$, then $\Omega_{\psi}\cl{C}^{\Delta^{1}}$ is given
by the pullback
\[\begin{tikzcd}
	{\Omega_\psi\cl{C}^{\Delta^1}} & {\rm{Aut}_{\cl{C}}(A)} \\
	{\rm{Aut}_{\cl{C}}(B)} & {\rm{Map}_{\cl{C}}(A,B)}
	\arrow["{\blank \circ \psi}"', from=2-1, to=2-2]
	\arrow["{\psi \circ \blank}", from=1-2, to=2-2]
	\arrow[from=1-1, to=2-1]
	\arrow[from=1-1, to=1-2]
\end{tikzcd}.\]
Thus, we can compute the loop space $\Omega X_{\varphi}^{R\oplus M[1]}$ as the pullback
\[\begin{tikzcd}
	{\Omega X_{\varphi}^{R\oplus M[1]}} & {\rm{Der}_{\id}(A, C_{A}(M[1]))} \\
	{\rm{Der}_{\id}(B, C_{B}(M[1]))} & {\rm{Der}_{\id}(B, C_{A}(M[1]))}
	\arrow[from=2-1, to=2-2]
	\arrow[from=1-2, to=2-2]
	\arrow[from=1-1, to=2-1]
	\arrow[from=1-1, to=1-2]
\end{tikzcd}\]
as claimed.
\end{proof}
Hence, we can think of a lift $\psi\in X_{\varphi}^{R \oplus M}$ as being given by two derivations
$\mu: B\to \Omega^{\infty}_{B}M[1]$ and $\nu: A \to \Omega^{\infty}_{A}M[1]$ together with a homotopy
\[\begin{tikzcd}
	B & {\Omega^\infty_BM[1]} \\
	A & {\Omega^\infty_AM[1]}
	\arrow["\nu"', from=2-1, to=2-2]
	\arrow["\mu", from=1-1, to=1-2]
	\arrow["\varphi", from=2-1, to=1-1]
	\arrow[from=2-2, to=1-2]
	\arrow[shorten <=7pt, shorten >=4pt, Rightarrow, from=2-1, to=1-2]
\end{tikzcd}\]
lying over the diagram in $\rm{cCAlg_{R}}$
\[\begin{tikzcd}
	A & A \\
	B & B
	\arrow["\id"', from=2-1, to=2-2]
	\arrow["\varphi"', from=2-2, to=1-2]
	\arrow["\varphi", from=2-1, to=1-1]
	\arrow["\id", from=1-1, to=1-2]
	\arrow["\id"', shorten <=4pt, shorten >=4pt, Rightarrow, from=2-1, to=1-2]
\end{tikzcd}.\]
In particular, if $A$ is formally \'etale this means that the space of lifts of $\varphi$
is equivalent to the space of lifts of $B$ to an $R\oplus M$-coalgebra. In fact, this property
characterizes formally \'etale coalgebras, as we will see in the following proposition.

\begin{proposition}\label{etalchar}
    Let $R$ be a connective $\bb{E}_{\infty}$-ring and $A\in \rm{cCAlg}_{R}^{\rm{cn}}$ and write
    $\cl{X}(R) = \rm{cCAlg}_{R}^{\rm{cn}}$. Then $A$ is formally \'etale if and only if for every
    $B \in \rm{cCAlg}^{\rm{cn}}_{R}, M \in \rm{Mod}^{cn}$ and every morphism
    $\varphi:B\to A$, the map $T_{(\cl{X}^{\Delta^1})_\varphi}^{M} \to T_{(\cl{X}^{\Delta^0})_{B}}^{M}$ induced by the evaluation
    $\cl{X}^{\Delta^1}\rar{\rm{ev}_0} \cl{X}^{\Delta^0}$ is an equivalence.
\end{proposition}
\begin{proof}
  Denote the fiber of the map
  \[ (\cl{X}^{\Delta^{1}})_{\varphi}^{R\oplus M} \to (\cl{X}^{\Delta^{0}})_{B}^{R\oplus M}\simeq \rm{Der}_{\id}(B, C_{B}M)\]
  over some point $B\p\in (\cl{X}^{\Delta^{0}})_{B}^{R\oplus M}$ by $F_{B\p}^{M}$. Then by definition
  of the tangent complex the map $T_{(\cl{X}^{\Delta^1})_\varphi}^{M} \to T_{(\cl{X}^{\Delta^0})_{B}}^{M}$ is an equivalence
  if and only if we have $F_{B\p}^{m} \simeq 0$ for all $M\in \rm{Mod}_{R}^{\rm{cn}}, B\p \in (\cl{X}^{\Delta^{0}})_{B}^{R\oplus M}$.
  The two pasted pullback squares
\[\begin{tikzcd}
	F_{B\p}^{M} & {(\cl{X}^{\Delta^1})_{\varphi}^{R \oplus M}} & {\rm{Der}_{\id}(A, C_A(M[1]))} \\
	\ast & {\rm{Der}_{\id}(B,C_B(M[1]))} & {\rm{Der}_{\varphi}(B, C_A(M[1]))}
	\arrow[from=1-1, to=2-1]
	\arrow[from=2-1, to=2-2]
	\arrow[from=1-2, to=2-2]
	\arrow[from=1-1, to=1-2]
	\arrow[from=1-2, to=1-3]
	\arrow[""{name=0, anchor=center, inner sep=0}, from=2-2, to=2-3]
	\arrow[from=1-3, to=2-3]
	\arrow["\lrcorner"{anchor=center, pos=0.125}, draw=none, from=1-1, to=2-2]
	\arrow["\lrcorner"{anchor=center, pos=0.125}, draw=none, from=1-2, to=0]
\end{tikzcd}\]
yield a fiber sequence
\[F_{M} \to \rm{Der}_{\id}(A, C_{A}(M[1])) \rar{\blank \circ \varphi }  \rm{Der}_{\varphi}(B, C_{A}(M[1])).\]
Hence, the ``only if'' direction holds. Moreover, we see that if $F_{B\p}^{M} \simeq 0$ for every
$M\in \rm{Mod}_{R}^{\rm{cn}}, B\p \in (\cl{X}^{\Delta^{0}})_{B}^{R\oplus M}$ and any morphism $B\rar{\varphi} A$, we obtain
a zigzag of equivalences
\[ \rm{Der}_{\varphi}(B, C_{A}(M[1])) \xleftarrow{\sim} \rm{Der}_{\id}(A, C_{A}(M[1]))
  \rar{\sim} \rm{Der}_{0}(0, C_{A}(M[1])) \simeq \pt,\]
where $0 \in \rm{cCAlg}_{k}$ denotes the initial coalgebra. Thus, we have that
\[\rm{Der}_{\varphi}(B, C_{A}(M))\simeq \Omega \rm{Der}_{\varphi}(B, C_{A}(M[1]))\simeq \pt\]
as claimed.
\end{proof}

\begin{corollary}\label{etallift}
  Let $R$ be a connective ring spectrum and $A\in \rm{cCAlg}_{R}^{\rm{cn}}$ be formally \'etale. For
  a square zero extension $R^{\eta} \to R$ with fiber $M$ denote by $A^{\eta}$ the essentially unique lift
  of $A$ to a connective $R^{\eta}$-coalgebra. Then $A^{\eta}$ is also formally \'etale.
\end{corollary}
\begin{proof}
  Let $B \to A^{\eta}$ be any map of $R^{\eta}$ coalgebras and write $\cl{X}(\blank)= \rm{cCAlg}^{\rm{cn}}_{\blank}$.
  Then for any $N\in \rm{Mod}_{R^{\eta}}^{\rm{cn}}$ we need to show that the induced map
  \[ T^{M}_{\cl{X}^{\Delta^{1}}_{\varphi}} \rar{\sim} T^{M}_{\cl{X}^{\Delta^{0}}_{B}}\]
  is an equivalence. Arguing as in the proof of Proposition~\ref{cofib}, we see that $N$ sits in a cofiber
  sequence
  \[ M\otimes_{R}(R \otimes_{R^{\eta}} N) \to N \to R \otimes_{R^{\eta}}N, \]
  where the $R^{\eta}$-action on the outer terms factors through $R$.
  Thus, writing $B\p \simeq B\otimes_{R^{\eta}} R$ and $\varphi\p = \varphi_{R^{\eta}}: B\p \to A$ and using that the tangent complex
  functors are excisive, we obtain a commutative diagram
\[\begin{tikzcd}
	{T_{\cl{X}^{\Delta^1}_{\varphi\p}}^{M\otimes_{R}(R\otimes_{R^\eta} N)}} & {T^{M\otimes_{R}(R\otimes_{R^\eta} N)}_{\cl{X}^{\Delta^1}_\varphi}} & {T^N_{\cl{X}^{\Delta^1}_\varphi}} & {T^{R\otimes_{R^\eta} N}_{\cl{X}^{\Delta^1}_\varphi}} & {T_{\cl{X}^{\Delta^1}_{\varphi\p}}^{R\otimes_{R^\eta} N}} \\
	{T_{\cl{X}^{\Delta^0}_{B\p}}^{M\otimes_{R}(R\otimes_{R^\eta} N)}} & {T_{\cl{X}^{\Delta^0}_B}^{M\otimes_{R}(R\otimes_{R^\eta} N)}} & {T^N_{\cl{X}^{\Delta^0}_B}} & {T^{R\otimes_{R^\eta} N}_{\cl{X}^{\Delta^0}_B}} & {T_{\cl{X}^{\Delta^0}_{B\p}}^{R\otimes_{R^\eta} N}}
	\arrow[from=1-2, to=1-3]
	\arrow[from=1-3, to=1-4]
	\arrow[from=1-2, to=2-2]
	\arrow[from=2-2, to=2-3]
	\arrow[from=2-3, to=2-4]
	\arrow[from=1-4, to=2-4]
	\arrow[from=1-3, to=2-3]
	\arrow["\sim", from=1-4, to=1-5]
	\arrow["\sim", from=2-4, to=2-5]
	\arrow["\sim", from=1-5, to=2-5]
	\arrow["\sim"', from=1-2, to=1-1]
	\arrow["\sim", from=1-1, to=2-1]
	\arrow["\sim"', from=2-2, to=2-1]
\end{tikzcd}\]
where the inner two horizontal maps in each row form a cofiber sequence. The outer horizontal maps are the base change equivalences from Proposition~\ref{bc} and the outer
vertical maps are equivalences since by assumption $A= A^{\eta}\otimes_{R^{\eta}}R$ is formally \'etale. Thus,
the middle map $T^{N}_{{\cl{X}^{\Delta^{1}}_{\varphi}}}\to T^{N}_{\cl{X}^{\Delta^{0}}_{B}}$ is an equivalence as well, so by
Proposition~\ref{etalchar} the $R^{\eta}$-coalgebra $A^{\eta}$ is formally \'etale.
\end{proof}

\begin{corollary}\label{defmaps}
  Let $A,B\in \rm{cCAlg}_{R}^{\rm{cn}}$ with $A$ formally \'etale and let $R^{\eta} \to R$ be a square zero
  extension. Suppose we are given lifts $A\p$ and $B\p$ of $A$ and $B$ respectively to
  $\rm{cCAlg}_{R^{\eta}}^{\rm{cn}}$. Then  the natural map
  \[\rm{Map}_{\rm{cCAlg}_{R^{\eta}}}(B\p, A\p) \to \rm{Map}_{\rm{cCAlg}_{R}}(B,A)\]
  is a homotopy equivalence.
\end{corollary}
\begin{proof}
This is immediate from Proposition~\ref{fibpb}.
\end{proof}

\begin{proposition}\label{witt}
  Let $R$ be a connective $\bb{E}_{\infty}$-ring, $R^{\eta} \to R$ a square zero extension and denote
  by $\cl{C}\subseteq \rm{cCAlg}_{R^{\eta}}^{\rm{cn}}$ the full subcategory spanned by those coalgebras $A$ such that
  $A\otimes_{R^{\eta}} R$ is formally \'etale. Then the functor
  \[ \cl{C} \to \rm{cCAlg}_{R}^{\rm{cn},\rm{f\acute{e}t}} \qquad A \mapsto A \otimes_{R^{\eta}}R\]
  is fully faithful and essentially surjective.
\end{proposition}
\begin{proof}
Combine Corollary~\ref{defobject} and Corollary~\ref{defmaps}.
\end{proof}
This means that we can lift \'etale coalgebras not just uniquely, but functorially against square zero
extensions. Since these lifts are again formally \'etale by Corollary~\ref{etallift},
we see that we can iterate this process. This will be the main theme of the next section.
