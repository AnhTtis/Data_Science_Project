We begin by explaining how deformation problems of $\bb{E}_{\infty}$-coalgebras appear
in $p$-adic homotopy theory. The reader who is already aware of such problems and
primarily interested in our results should feel free to skip to Section 1.2.\\
\subsection{Background and motivation}
One of the fundamental goals of algebraic topology is to construct algebraic models for homotopy types,
that is to give complete homotopy invariants for topological spaces. While this is unrealistic for all
spaces simultaneously, a divide-and-conquer approach borrowed from arithmetic has been immensely successful.
Namely, we can approximate a space $X$ by its rationalization $X_{\Q}$ and for each prime $p$ its
$p$-completion $X^{\wedge}_{p}$. The rational story has been well understood for many years due to work of
Sullivan~\cite{SullivanQ} and Quillen~\cite{QuillenRatl}. We are interested in the $p$-adic setting, where
a theorem of Mandell shows that we can model a full subcategory of
the $\infty$-category of $p$-complete spaces $\cl{S}^{\wedge}_{p}$ using $\bb{E}_{\infty}$-algebras.
\begin{theorem}[Mandell~\cite{mandell}]\label{thmandell}
  The assignment $X \mapsto C^{\pt}(X;\overline{\F}_{p})= X^{\overline{\F}_{p}}$ determines a fully faithful
  contravariant functor from the full subcategory of $\cl{S}^{\wedge}_{p}$ spanned by the simply connected
  $p$-complete spaces of finite type to the $\infty$-category $\rm{CAlg}_{\overline{\F}_{p}}$ of
  $\bb{E}_{\infty}$-algebras over $\overline{\F}_{p}$.
\end{theorem}
In particular, this tells us that the space $X$ can be recovered as the space of $\bb{E}_{\infty}$-ring
maps $\overline{\F}_{p}^{X} \to \overline{\F}_{p}$, hence the $\overline{\F}_{p}$-cohomology remembers
everything there is to know about the space $X$. However, this theorem raises some questions.
Firstly, the $\bb{E}_{\infty}$-algebra structure on
$X^{\overline{\F}_{p}}$ is actually induced via dualizing by the $\bb{E}_{\infty}$-coalgebra structure on the
$\overline{\F}_{p}$-homology $C_{\pt}(X; \overline{\F}_{p})=\overline{\F}_{p}[X]$.
Since the duality functor
\[ \rm{cCAlg}_{\overline{\F}_{p}}\rar{(\blank)^{\vee}} \rm{CAlg}_{\overline{\F}_{p}}\op\]
is not fully faithful, it is natural to ask whether we can exhibit Mandell's Theorem as a shadow
of an even better coalgebraic model for $p$-complete spaces.
We can also wonder why the ring $\overline{\F}_{p}$ in Theorem~\ref{thmandell}
appears instead of $\F_{p}$. Indeed, the $\F_{p}$-cohomology functor $X \mapsto \F_{p}^{X}$ is
\textit{not} fully faithful because we are forgetting the
Frobenius action on the coefficients. More precisely, Mandell shows that
\begin{align*}
  \rm{Map}_{\rm{CAlg}_{\F_{p}}}(\F_{p}^{X}, \F_{p})
  \simeq \rm{Map}_{\rm{CAlg}_{\overline{\F}_{p}}}(\overline{\F}_{p}^{X}, \overline{\F}_{p})^{h\Z}
  \simeq X^{h\Z}\simeq \cl{L}X
\end{align*}
holds, where $\Z$ acts via the Frobenius on $\overline{\F}_{p}$. This tells us that any
algebraic model of $p$-complete spaces needs to take a Frobenius into account. For the $\bb{E}_{\infty}$-algebra
model we can summarize this story informally as follows.

\begin{slogan}
  The datum of a simply connected $p$-complete space is equivalent to that of an $\bb{E}_{\infty}$-ring together
  with a trivialization of the Frobenius via the cohomology functor.
\end{slogan}

To formalize this one would like to use a notion of Frobenius that is intrinsic to the category of
$\E_{\infty}$-algebras and not only defined via the coefficients. Such a notion is provided by the Tate Frobenius map
\[ \varphi_{p}: R \to R^{tC_{p}}\]
introduced by Nikolaus and Scholze in~\cite{tch} for every $\bb{E}_{\infty}$-ring $R$,
which is a generalization of the $p$-th power map $R\to R/p$ instead taking values in the \textit{Tate homology}
of $R$. Unlike the ordinary $p$-th power map, it is not an endomorphism over $\F_{p}$ since $\F_{p}^{tC_{p}}$
is not equivalent to $\F_{p}$. However, it is much better behaved over the $p$-completed sphere since
$(\S_{p}^{\wedge})^{tC_{p}} \simeq \S_{p}^{\wedge}$ does hold by the Segal Conjecture,
first proved in this form for $p=2$ by Lin in~\cite{Lin} and for odd $p$ by Gunawardena in~\cite{Gunawardena}.
This suggests instead considering ${\S^{\wedge}_{p}}^{X}$ as a model for a
$p$-complete homotopy type $X$. These ideas were recently investigated by Yuan in~\cite{yuan}
to give a model for $p$-complete finite spaces in terms of so called \textit{$p$-Frobenius fixed
$\bb{E}_{\infty}$-rings}, see~\cite[][Theorem 7.1.]{yuan}. This is a natural extension of the
$\F_{p}$-model as a consequence of the following statement.

\begin{proposition}[Mandell, Lurie]\label{manlur}
  For any space $X$ the $\F_{p}$-cohomology $\F_{p}^{X}$ is a formally \'etale $\F_{p}$-algebra.
\end{proposition}

Thus, one can use deformation theoretic arguments to show that for every finite space
$X$ the base change map

\begin{equation*}
 \rm{Map}_{\rm{CAlg}_{\S_{p}^{\wedge}}}({\S^{\wedge}_{p}}^{X}, \S_{p}^{\wedge})\to
\rm{Map}_{\rm{CAlg}_{\F_{p}}}(\F_{p}^{X}, \F_{p} )
\end{equation*}
is a homotopy equivalence, compare the argument in~\cite[][Corollary 7.6.1.]{yuan}.
This crucially uses that ${\S_{p}^{\wedge}}^{X}\otimes_{\S_{p}^{\wedge}}\F_{p} \simeq \F_{p}^{X}$ for finite $X$.
The finiteness assumption is also necessary for the model to ensure that the Frobenius defines
an automorphism on ${\S_{p}^{\wedge}}^{X}$. Thus, if we want to improve on this result and fully realize
our slogan, we need to find better behaved Frobenius map.
As it turns out, this is available to us if we are willing to work with $p$-complete
$\bb{E}_{\infty}$-\textit{coalgebras} instead as is the content of the following
unpublished result due to Nikolaus.

\begin{theorem}[Nikolaus]
  Let $\cl{C} = (\rm{cCAlg}^{\rm{cn}}_{\S^{\wedge}_{p}})^{\wedge}_{p}$ denote the $\infty$-category $\bb{E}_{\infty}$-coalgebras
  in the category of $p$-complete spectra equipped with the symmetric monoidal structure given by the
  $p$-completed tensor product. Then there exists a natural transformation
  $\psi_{p}:\id_{\cl{C}}\to \id_{\cl{C}}$ which on an object $A\in \cl{C}$ is given by the composition
  \[ \psi_{p}: A \rar{\Delta_{A}^{\otimes p}} (A^{\otimes p})^{hC_{p}} \rar{\rm{can}} (A^{\otimes p})^{tC_{p}} \rar{\sim} A,\]
    where the right hand map is the inverse of the \textit{Tate diagonal}, see~\cite[][Theorem III.1.7]{tch}.
    \end{theorem}

    This means that for $p$-complete $\S_{p}^{\wedge}$-coalgebra $A$ we always have access to a Frobenius
    \textit{endomorphism} $\psi_{p}:A \to A$.  This suggests instead considering the $\bb{E}_{\infty}$-coalgebra
    $\S[X]^{\wedge}_{p}=(\Sigma_{+}^{\infty}X)^{\wedge}_{p}$ together with the action of $\psi_{p}$ as a more natural model for
    $p$-complete spaces $X$. In fact, the homology of a space is also better behaved with respect
    to base change, as we have $\S_{p}[X]^{\wedge}_{p}\otimes_{\S_{p}^{\wedge}}\F_{p} \simeq \F_{p}[X] = C_{\pt}(X;\F_{p})$
    for \textit{any} space $X$. These ideas, as well as the approach we will now describe are all due to Nikolaus.
    To obtain a coalgebraic version of Theorem~\ref{thmandell} from such a model we would have to show
    that the base change map
    \[ \rm{Map}_{(\rm{cCAlg}_{\S_{p}^{\wedge}})^{\wedge}_{p}}(\S^{\wedge}_{p}, \S[X]_{p}^{\wedge}) \to
    \rm{Map}_{\rm{cCAlg}_{\F_{p}}}(\F_{p}, \F_{p}[X])\]
    is an equivalence, which is the problem we will primarily be investigating.
    Put differently, we can divide the problem of understanding the $\F_{p}$-homology
    as a model for $p$-complete spaces into two parts by factoring the functor $X \mapsto \F_{p}[X]$ as follows
    \[ \cl{S} \rar{\cl{S}[\blank]^{\wedge}_{p}} (\rm{cCAlg}_{\S_{p}^{\wedge}})^{\wedge}_{p}
    \rar{\blank \otimes_{\S_{p}^{\wedge}}\F_{p}}  \rm{cCAlg}_{\F_{p}}.\]
    The goal of this paper is to understand the right hand functor.
    More concretely we ask the following:

    \begin{question}\label{q1}
    Can we describe a full subcategory of $(\rm{cCAlg}_{\S_{p}^{\wedge}})^{\wedge}_{p}$
    which contains $\S[X]^{\wedge}_{p}$ for an arbitrary space $X$ and on which the
    base change to $\F_{p}$ is fully faithful?
    \end{question}

    Since the base change to $\F_{p}$ can further be factored as the composition
    \[ (\rm{cCAlg}_{\S_{p}^{\wedge}})^{\wedge}_{p} \rar{\blank \otimes_{\S^{\wedge}_{p}}\Z_{p}} (\rm{cCAlg}_{\Z_{p}})^{\wedge}_{p}
    \rar{\blank \otimes_{\Z_{p}} \F_{p}} \rm{cCAlg}_{\F_{p}}\]
    this question can be phrased in terms of \textit{deformation theory}, which is the approach
    taken by this paper. More precisely, since $\Z_{p}$ is an iterated square zero extension of $\F_{p}$
    and in turn $\S^{\wedge}_{p}$ is an iterated square zero extension of $\Z_{p}$, it is natural to first
    ask how to lift $\bb{E}_{\infty}$-coalgebras along general square zero extensions. This is also called
    a \textit{deformation problem}.

    \begin{question}\label{q2}
    Let $R$ be an $\bb{E}_{\infty}$-ring, $R^{\eta} \to R$ be a square zero extension and $A \in \rm{cCAlg}_{R}$.
    How can we describe the space of $R^{\eta}$-coalgebras $A\p$ equipped with an equivalence
    $A\p\otimes_{R^{\eta}}R \simeq A$? For which coalgebras is it contractible?
    \end{question}

    \subsection{Summary of results}

    To answer Questions~\ref{q1} and~\ref{q2} we introduce a novel
    and somewhat surprising notion of \textit{formally \'etale} coalgebras.
    Let $R$ be an $\bb{E}_{\infty}$-ring and $M \in \rm{Mod}_{R}$. Denote by $f_{M}:R \to R\oplus M$ the section
    of the split square zero extension and write $f_{M}^{\pt}:\rm{cCAlg}_{R}\to \rm{cCAlg}_{R\oplus M}$
    for the base change functor. This functor admits a right adjoint by the adjoint functor theorem
    which we denote $f_{M,!}$.

    \begin{definition}\label{etale}
    A coalgebra $A\in \rm{cCAlg}_{R}$ is called \textit{formally \'etale} if the counit of the adjunction
    $\eta_{A}:A \to f_{M,!}f_{M}^{\pt} A$ is an equivalence for every $M \in \rm{Mod}_{R}$.
    We denote by $\rm{cCAlg}_{R}^{\rm{f\acute{e}t}}$ the full subcategory spanned by the formally
    \'etale coalgebras.

    \end{definition}

    Our main result is that connective, formally \'etale $\F_{p}$-coalgebras admit essentially unique, functorial
    lifts to the $p$-completed sphere.

    \begin{theorem}\label{mainthm}
    Denote by $\cl{C}\subseteq (\rm{cCAlg}_{\S_{p}^{\wedge}}^{\rm{cn}})^{\wedge}_{p} $ the full subcategory spanned by those
    coalgebras $A$ such that $A\otimes_{\S_{p}^{\wedge}}\F_{p}$ is formally \'etale. Then the base change functor
    \[ \cl{C} \to \rm{cCAlg}_{\F_{p}}^{\rm{cn}, \rm{f\acute{e}t}} \qquad A \mapsto A\otimes_{\S_{p}^{\wedge}}\F_{p}\]
    is fully faithful and essentially surjective.
  \end{theorem}

   In particular, taking the quasi-inverse yields a spherical Witt vector style functor
    \[ W_{\S_{p}^{\wedge}}: \rm{cCAlg}_{\F_{p}}^{\rm{cn}, \rm{f\acute{e}t}}
        \to (\rm{cCAlg}_{\S_{p}^{\wedge}}^{\rm{cn}})^{\wedge}_{p}\]
      which is fully faithful and satisfies $W_{\S_{p}^{\wedge}}(A)\otimes_{\S_{p}^{\wedge}}\F_{p}\simeq A$
      for any connective, formally \'etale $\F_{p}$-coalgebra $A$.\\
      To prove Theorem~\ref{mainthm}, we first unravel what Definition~\ref{etale} has to do with deformation
      problems.
    For an $R$-coalgebra $A$ the counit  $\eta_{A}$ admits a natural splitting
    $\pi_{A}:f_{M,!}f_{M}^{\pt}A\to A$, hence $\eta_{A}$ is an equivalence if and only if $\pi_{A}$ is.
    We show that all relevant deformation problems are equivalent to lifting problems
    of the form
    \[\begin{tikzcd}
        & {f_{M,!}f_M^\pt A} \\
        B & A
        \arrow[from=2-1, to=2-2]
        \arrow["{\pi_A}", from=1-2, to=2-2]
        \arrow[dashed, from=2-1, to=1-2]
    \end{tikzcd}\]
    in the category of $R$-coalgebras. If we ask that these can be solved uniquely for every $B \in \rm{cCAlg}_{R}$,
    the Yoneda Lemma implies that $\pi_{A}$ and thus also $\eta_{A}$ must be an equivalence. This may seem strange and
    unreasonable at first glance. However, notice that since any $\bb{E}_{\infty}$-ring is the terminal coalgebra over
    itself we have
    \[f_{M,!}f_{M}^{\pt}(R)\simeq f_{M,!}(R\oplus M) \simeq R \]
    as $f_{M,!}$ is a right adjoint and hence preserves terminal objects. More generally we show that the following
    holds.

    \begin{proposition}\label{etale2}
    Let $A\in\rm{cCAlg}_{R}$ be dualizable such that its dual $A^{\vee}$ is a formally \'etale $R$-algebra.
    Then $A$ is formally \'etale in the sense of Definition~\ref{etale}.
    \end{proposition}

    This shows that Definition~\ref{etale} is actually \textit{reasonable} i.e.~it is satisfied by
    a nontrivial class of coalgebras. We also prove that it is \textit{powerful},
    namely that formally \'etale coalgebras can be lifted uniquely and functorially along square zero extensions.

    \begin{proposition}\label{subthm}
    Let $R$ be a connective $\bb{E}_{\infty}$-ring, $R^{\eta} \to R$ a square zero extension and denote
    by $\cl{C}\subseteq \rm{cCAlg}_{R^{\eta}}^{\rm{cn}}$ the full subcategory spanned by those coalgebras $A$ such that
    $A\otimes_{R^{\eta}} R$ is formally \'etale. Then the base change functor
    \[ \cl{C} \to \rm{cCAlg}_{\F_{p}}^{\rm{cn},\rm{f\acute{e}t}} \qquad A \mapsto A \otimes_{R^{\eta}}R\]
    is fully faithful and essentially surjective.
    \end{proposition}

We can apply Theorem~\ref{mainthm} to obtain a partial answer to our questions:
For a finite space $X$ the coalgebra $\F_{p}[X]$ is dualizable with dual given by $\F_{p}^{X}$.
Thus, since $\F_{p}^{X}$ is formally \'etale by Proposition~\ref{manlur} we can combine
Theorem~\ref{mainthm} and Proposition~\ref{etale2} to prove the following.

\begin{corollary}
For a finite space $X$ the coalgebra $\S[X]^{\wedge}_{p}$
is the essentially unique lift of $\F_{p}[X]$ to a $p$-complete, connective $\S_{p}^{\wedge}$-coalgebra.
Moreover, for any finite spaces $X,Y$ the base change map
\[ \rm{Map}_{(\rm{cCAlg}_{\S_{p}^{\wedge}})^{\wedge}_{p}}(\S[X]^{\wedge}_{p}, (\S[Y]^{\wedge}_{p}) \rar{\sim}
  \rm{Map}_{\rm{cCAlg}_{\F_{p}}}(\F_{p}[X], \F_{p}[Y])\]
is a homotopy equivalence.
\end{corollary}

This statement can of course be deduced directly from Proposition~\ref{manlur} and the well known deformation
theory of $\bb{E}_{\infty}$-algebras since everything is dualizable. Our main motivation to develop this more
systematic approach was to understand how to deal with the non-dualizable situation, i.e.~the case where
our spaces are not necessarily finite.
We have thus far been unable to give a proof in this generality primarily for the following reason:
Since our notion of formally \'etale is defined using the mysterious right adjoint $f_{M,!}$, what is
crucially still missing is a sufficient condition for a non-dualizable coalgebra to be formally \'etale that can
be checked in practice. We conjecture that this is provided by the coalgebra Frobenius discussed above.

\begin{conjecture}
  Let $A \in (\rm{cCAlg}^{\rm{cn}}_{\S^{\wedge}_{p}})^{\wedge}_{p}$ and write $A\p= A\otimes_{\S^{\wedge}_{p}}\F_{p}$. Then
  for any $M \in \rm{Mod}_{\F_{p}}^{\rm{cn}}$ the coalgebra Frobenius $\psi_p:A\to A$ induces the zero map
  on the $R$-module $C_{A\p}(M) = \rm{cofib}(A\p \rar{\eta_{A\p}} \Omega^{\infty}_{A\p}(M))$.
\end{conjecture}

This would immediately imply that, if $A$ is a $p$-complete $\S_{p}^{\wedge}$-coalgebra such that the Frobenius
$\psi:A \rar{} A$ is a homotopy equivalence, then $A\otimes_{\S_{p}^{\wedge}}\F_{p}$ is a formally \'etale $\F_{p}$-coalgebra.
Moreover, since by naturality the coalgebra Frobenius is given by the identity on $\S[X]^{\wedge}_{p}$,
it would allow us to apply our theorem to the chains of spaces that are not-necessarily finite
and supply us with examples of formally \'etale coalgebras which are not dualizable.

\subsection{Technical methods}
  We attack these problems using the machinery of deformation theory developed
  by Lurie in~\cite{dag14}~and~\cite{ha}. In~\cite{dag14} a class of functors
  $\rm{CAlg}^{\rm{cn}}\to \cl{S}$ is introduced which are well behaved with respect
  to deformation problems.

\begin{definition}[Lurie]
A functor $X:\rm{CAlg}^{\rm{cn}}\to \cl{S}$ is called
\textit{cohesive} if for any pullback diagram of
  connective $\bb{E}_{\infty}$-rings
  \[\begin{tikzcd}
      {R^\prime} & {S^\prime} \\
      R & S \arrow[from=1-2, to=2-2] \arrow[from=2-1, to=2-2] \arrow[from=1-1, to=2-1]
      \arrow[from=1-1, to=1-2]
    \end{tikzcd}\]
  which induces surjections $\pi_{0}R \to \pi_{0}S$ and $\pi_{0}S\p \to \pi_{0}S$, the diagram
  \[\begin{tikzcd}
	{X(R\p)} & {X(S\p)} \\
	{X(R)} & {X(S)}
	\arrow[from=1-1, to=1-2]
	\arrow[from=1-2, to=2-2]
	\arrow[from=1-1, to=2-1]
	\arrow[from=2-1, to=2-2]
\end{tikzcd}\]
is a pullback of spaces.
\end{definition}

Given a cohesive functor $X$ for each $R$-valued point $A\in X(R)$
there exists a spectrum $T^{M}_{X_{A}}$ which controls deformations of $A$ along square
zero extensions of $R$ with fiber $M$.

\begin{theorem}[Lurie]
  Let $X: \rm{CAlg}^{\rm{cn}} \to \cl{S}$ be a cohesive functor and $R^{\eta} \to R$ a square zero extension
  with fiber $M \in \rm{Mod}_{R}^{\rm{cn}}$. Then for each $A \in X(R)$ there exists a spectrum $T^{M}_{X_{A}}$
  called the \textit{Tangent Complex} of $X$ at $A$ such that the space of deformations
  $\rm{X}_{A}^{R^{\eta}} = \rm{fib}_{A}(X(R^{\eta})\to X(R))$ is either
  empty or a torsor under the grouplike $\E_{\infty}$-monoid $\Omega^{\infty}T^{M}_{X_{A}}$. Moreover,
  we have an obstruction class in $\pi_{-1}T^{M}_{X_{A}}$ which vanishes if and only
  if $\rm{X}_{A}^{R^{\eta}}$ is non-empty.
\end{theorem}

In this sense, the tangent complex, if it exists, is the precise answer to Question~\ref{q2}.
We use an \'etale descent theorem for modules also due to Lurie \cite[][Theorem 16.2.0.2.]{sag}
to show that this machinery can be applied to coalgebras.

\begin{proposition}
  For any $n\in \bb{N}$ the functor $\rm{CAlg}^{\rm{cn}} \to \cl{S}$ which takes a connective $\bb{E}_{\infty}$-ring
  $R$ to the space
  $(\rm{cCAlg}_{R}^{\rm{cn}})^{\Delta^{n}}= \rm{Map}_{\rm{Cat_{\infty}}}(\Delta^{n}, \rm{cCAlg}_{R}^{\rm{cn}})$
  is cohesive.
\end{proposition}

This allows us to make sense of the definition of formally \'etale coalgebras and prove
Proposition~\ref{subthm}. To get from this to Theorem~\ref{mainthm}, we prove two different completeness
results for coalgebras. Concretely, we can write the ring $\Z_{p}$ as the limit
\[ \Z_{p}= \lim \left( \dots \to \Z/p^{3} \to \Z/p^{2} \to \Z/p =\F_{p} \right)\]
where each map is a square zero extension with fiber $\Z/p$.
This poses a problem, as the natural map
\[ \rm{cCAlg}_{\Z_{p}}\to \flim \rm{cCAlg}_{\Z/p^{n}}\]
is \textit{not} an equivalence. This is where \textit{$p$-complete} coalgebras come in to play.

 \begin{definition}
   Let $R$ be an $\bb{E}_{\infty}$-ring. We define the $\infty$-category of $p$-complete
   $R$-coalgebras as
   \[ {(\rm{cCAlg}_{R})}^{\wedge}_{p}:= \rm{cCAlg}({(\rm{Mod}_{R})}^{\wedge}_{p}).\]
 \end{definition}

 Here, $(\rm{Mod}_{R})^{\wedge}_{p}$ denotes the $\infty$-category of $p$-complete $R$-modules equipped with
 the symmetric monoidal structure given by the $p$-completed tensor product. We prove that $p$-complete
 coalgebras are suitable for deformation theoretic questions and in fact the correct notion if we want
 to pass from $\F_{p}$ to the $p$-adics inductively.

 \begin{proposition}
   For every $n\in \bb{N}$ the functor
   \[ \rm{CAlg}^{\rm{cn}} \to \cl{S} \qquad R\mapsto [{(\rm{cCAlg}_{R})}^{\wedge}_{p}]^{\Delta^{n}}\]
   is cohesive. Moreover, the assignment $A \mapsto A\otimes_{\Z_{p}}\Z/p^{n}$ induces an
   equivalence of $\infty$-categories
    \[(\rm{cCAlg}_{\Z_{p}})_{p}^{\wedge} \rar{\sim} \flim \rm{cCAlg}_{\Z/p^{n}}. \]
 \end{proposition}

Similarly, $\S_{p}^{\wedge}$ is given by the limit of the Postnikov-Tower
\[ \S_{p}^{\wedge}= \lim \left( \dots \to \tau_{\leq2}\S_{p}^{\wedge} \to \tau_{\leq 1} \S_{p}^{\wedge} \to \tau_{\leq0}\S^{\wedge}_{p} = \Z_{p} \right),\]
  where each map $\tau_{\leq n+1}\S_{p}^{\wedge} \to \tau_{\leq n} \S_{p}^{\wedge}$ is a square zero extension with fiber
  $\pi_{n+1}\S_{p}^{\wedge}[n+1]$. Thus, to be able to lift inductively from $\Z_{p}$ to $\S_{p}^{\wedge}$
  we prove the following.

  \begin{proposition}
    For every connective $\bb{E}_{\infty}$-ring $R$, the truncation functors
    $\tau_{\leq n}: \rm{Mod}_{R}\to \rm{Mod}_{\tau_{\leq n} R}$ induce equivalences of categories
    \[ \rm{cCAlg}_{R}^{\rm{cn}} \rar{\sim} \flim \rm{cCAlg}^{\rm{cn}}_{\tau_{\leq n}R}\]
    \[ (\rm{cCAlg}_{R}^{\rm{cn}})^{\wedge}_{p} \to\flim_{n} (\rm{cCAlg}_{\tau_{\le n}R}^{\rm{cn}})^{\wedge}_{p}.\]
  \end{proposition}
  With these requirements in place, Theorem~\ref{mainthm} amounts to a series of  tangent complex
  computations which work for an arbitrary cohesive and nilcomplete functors.

  \subsection{Outline}
We proceed along the following structure:
In Section 1 we define our basic objects of study and collect some facts about coalgebras and duality.
In Section 2 we introduce and review the setup of deformation theory developed by Lurie
in~\cite{dag14} and~\cite{ha}. We recall the notions of square zero extensions of $\bb{E}_{\infty}$-rings
and discuss cohesive and nilcomplete functors. We then define the (co)tangent complex of a cohesive functor
and prove some facts about its behavior.
In Section 3 we investigate how to lift coalgebras and maps of coalgebras
against square zero extensions. We apply the tangent complex formalism to deformations of maps
of coalgebras and show that these deformation problems can be reformulated as a lifting
problem against certain maps of coalgebras and use this to introduce our notion of formally \'etale coalgebras.
In Section 4 we discuss how to lift coalgebras and morphisms of coalgebras from
$\F_p$ to $\S_{p}^{\wedge}$. We construct a spherical Witt vector style functor for formally \'etale
$\F_{p}$-coalgebras and apply our results to $\F_{p}[X]$ for a finite space $X$.
In Section 5 we give a brief overview of some unanswered questions and sketch a possible way to proceed
with the program envisioned by Nikolaus.
 \subsection{Conventions}
 Throughout the text, we use the following conventions:
 \begin{itemize}
   \item We use the words category, $\infty$-category and $(\infty,1)$-category interchangeably to mean
         $(\infty, 1)$-category. Moreover, the text is \textit{model agnostic}, that is we make
         no reference to any specific model for the theory of $(\infty, 1)$-categories.
         If pressed, we fall back on the Weak Kan Complex model as described in~\cite{htt}.
   \item If $A,B$ are objects in some $\infty$-category $\cl{C}$, we use the words map and morphism
         $A\to B$ interchangeably to mean a point in the mapping space $\rm{Map}_{\cl{C}}(A,B)$.
   \item We denote the $\infty$-category of spaces by $\cl{S}$ and the $\infty$-category of spectra by $\rm{Sp}$.
         Moreover, we write $\rm{Sp}_{\geq n} \subseteq \rm{Sp}$ for the full subcategory spanned by the
         $n$-connective spectra and denote the right adjoint to the inclusion by
         $\tau_{\geq n}:\rm{Sp}\to \rm{Sp}_{\geq n}$.
   \item We choose a Grothendieck Universe $\cl{U}$, denote by $\rm{Cat}_{\infty}$ the large $\infty$-category
         of small $\infty$-categories and disregard all size issues from here on out.
   \item By (co)algebra we always mean $\bb{E}_{\infty}$-(co)algebra.
  \item  All tensor products are derived unless stated otherwise.
 \end{itemize}

 \subsection{Acknowledgments}
 This paper is based on my Master's Thesis which was advised by Thomas Nikolaus and Achim Krause. I would
 like to thank them for guiding this project and being wonderful teachers throughout the years. I want
 to especially thank Achim Krause for countless hours of insightful discussions and his tremendous
 patience with my questions. I also want to thank Marin Janssen for being a mathematical mentor
 and an inspiration to me for many years. Lastly I want to thank Anika Laschewski for dilligently
 proofreading the entire thing.
