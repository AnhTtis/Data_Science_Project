In this section, we collect some general definitions and facts which we will use throughout the text.
We define $\bb{E}_{\infty}$-coalgebras in an arbitrary symmetric monoidal category $\cl{C}$ and discuss
how well behaved this notion is with respect to colimits and presentability. We then specialize to
the case $\cl{C}=\rm{Mod}_{R}$ and introduce the right adjoint to base change in the coalgebra setting.
The latter will play a crucial and annoying role in the considerations of deformation theory that follow.
We then move on to coalgebras in presentably symmetric monoidal $\infty$-categories, introducing the coalgebra
structure on the $R$-chains of a space $X$ and the algebra structure on
$A^{\vee}=\rm{map}_{\cl{C}}(A, 1_{\cl{C}})$ for $A\in \rm{cCAlg}(\cl{C})$ and $\cl{C}$ a presentably monoidal
$\infty$-category.
We begin by recalling basic facts about $\bb{E}_{\infty}$-algebras in an arbitrary symmetric monoidal
$\infty$-category, before moving on to the coalgebra picture.
\subsection{Generalities}
\begin{definition}
  Let $\cl{C}$ be a symmetric monoidal $\infty$-category, then we denote by $\rm{CAlg}(\cl{C})$ the category
  of $\bb{E}_{\infty}$-algebras in $\cl{C}$. For $\cl{C}= \rm{Sp}$ we write $\rm{CAlg}(\rm{Sp})= \rm{CAlg}$
  and refer to objects of $\rm{CAlg}$ simply as $\bb{E}_{\infty}$-rings.
\end{definition}

\begin{proposition}[Lurie]\label{calg}
  Let $(\cl{C}, \otimes)$ be a symmetric monoidal $\infty$-category, then the following statements hold:
  \begin{enumerate}
    \item The forgetful functor $U:\rm{CAlg}(\cl{C}) \to \cl{C}$ is conservative and commutes with limits.
    \item The coproduct of two algebras $R,S \in \rm{CAlg}(\cl{C})$ is given by the tensor product $R \otimes S$.
    \item If $\cl{C}$ is presentable and $\blank \otimes \blank$ commutes with colimits in both variables
          separately, then $\rm{CAlg}(\cl{C})$ is presentable as well.
  \end{enumerate}
\end{proposition}

\begin{proof}
  The first claim is a combination of~\cite[][Lemma 3.2.2.6]{ha} and~\cite[][Corollary 3.2.2.5]{ha}.
  The second is shown in~\cite[][Corollary 3.2.4.7]{ha} and the third is~\cite[][Corollary 3.2.3.5.]{ha}.
\end{proof}

\begin{definition}
  Let $\cl{C}$ be a symmetric monoidal $\infty$-category and denote by $p:\cl{C}^{\otimes} \to \rm{Fin}_{\pt}$
  the associated $\infty$-operad. Composing the straightening of $p$ with the opposite category functor
  $(\blank)\op:\rm{Cat}_{\infty} \to \rm{Cat}_{\infty}$ and taking the unstraightening yields
  a new $\infty$-operad $\rm{Un}(p\op): \cl{D}^{\otimes}\to \rm{Fin}_{\pt}$
  with fiber over $\langle 1 \rangle \in \rm{Fin}_{\pt}$ given by $\cl{C}\op$.
  This equips $\cl{C}\op$ with a natural symmetric monoidal structure and we
  define the category of $\bb{E}_{\infty}$-coalgebras in $\cl{C}$ as
  \[\rm{cCAlg}(\cl{C}):= \rm{CAlg}(\cl{C}^{\rm{op}})^{\rm{op}}.\]
\end{definition}

\begin{remark}
  In particular, any coalgebra $A \in \rm{cCAlg}(\cl{C})$ comes equipped with a datum of
  ``coherently commutative'' multiplication and counit maps
  \begin{align*}
    \Delta_{A}: A &\to A \otimes A \\
    \eta: A &\to 1_{\cl{C}},
  \end{align*}
  where $1_{\cl{C}}$ denotes the tensor unit of $\cl{C}$. Note that, in general, there is no way
  to describe $\rm{cCAlg}_{\cl{C}}$ as a category of algebras in some suitable category $\cl{D}$.
  Thus, coalgebras behave very differently from algebras, although they are still ``well behaved''
  in many ways. For once, Proposition~\ref{calg} immediately yields that for a symmetric monoidal
  $\infty$-category $(\cl{C}, \otimes)$ we have:
  \begin{enumerate}
    \item The forgetful functor $U:\rm{cCAlg}(\cl{C}) \to \cl{C}$ is conservative and commutes with colimits.
    \item The product of two coalgebras $R,S \in \rm{cCAlg}(\cl{C})$ is given by $R \otimes S$.
  \end{enumerate}
However, we cannot deduce presentability this way since the opposite of a presentable category is almost
never presentable.
\end{remark}

\begin{proposition}[Lurie]\label{present}
  Let $(\cl{C},\otimes)$ be a symmetric monoidal $\infty$-category such that $\blank \otimes \blank$ commutes with
  colimits in each variable separately and $\cl{C}$ is presentable. Then $\rm{cCAlg}({\cl{C}})$ is
  presentable.
\end{proposition}
\begin{proof}
  This is~\cite[][Corollary 3.1.4.]{ellI}.
\end{proof}
 This can be seen as an analogue of the classical theorem by Sweedler that every coalgebra
in the 1-category of vector spaces over a field is a filtered colimit of its finite dimensional
sub-coalgebras, see~\cite{sweedler1969hopf}. However, unlike in the classical situation, if $\cl{C}$
is $\kappa$-presentable, we only deduce that $\rm{cCAlg}(\cl{C})$ is $\tau$-presentable for some unknown $\tau \geq \kappa$.
This is one of the main defects the category of coalgebras has over that of algebras. Otherwise they are similarly well behaved.
\begin{lemma}\label{limits}
  Let $\rm{Cat}_{\infty}^{\otimes}$ denote the $\infty$-category of symmetric monoidal $\infty$-categories and strong monoidal
  functors. Then the functors
  \[\rm{Cat}_{\infty}^{\otimes} \to \rm{Cat}_{\infty} \quad \cl{C}\mapsto \rm{CAlg}(\cl{C}),\]
  \[\rm{Cat}_{\infty}^{\otimes} \to \rm{Cat}_{\infty} \quad \cl{C} \mapsto \rm{cCAlg}(\cl{C})\]
  commute with limits.
\end{lemma}

\begin{proof}
  By Proposition~\ref{calg} the functor $\rm{CAlg}(\blank)$ factors through the category $\rm{Cat}_{\infty}^{\amalg}$ of
  categories which admit finite coproducts and functors which preserve finite coproducts. As such it admits
  a left adjoint which equips $\cl{D} \in \rm{Cat}_{\infty}^{\amalg}$ with the
  cocartesian monoidal structure. Moreover, the inclusion $\rm{Cat}_{\infty}^{\amalg} \rari{} \rm{Cat}_{\infty}$
  admits a left adjoint which takes an $\infty$-category $\cl{C}$ to the free finite-coproduct completion, namely
  the full subcategory of $\rm{Psh}(\cl{C})$ spanned by finite coproducts of representables.
  Thus, both functors commute with limits, and so the composition does as well. \\
  For the second functor, we simply observe that it is given by the composition
  \[ \rm{Cat}_{\infty}^{\otimes} \rar{(\blank)\op} \rm{Cat}_{\infty}^{\otimes} \rar{\rm{CAlg(\blank)}} \rm{Cat}_{\infty}
  \rar{(\blank)\op}\rm{Cat}_{\infty},\]
which immediately implies the claim, since taking the opposite category is an involution, i.e.~an
equivalence of categories, and so commutes with limits.
\end{proof}

\begin{definition}
  For an $\bb{E}_{\infty}$-ring $R$ we refer to
  \[ \rm{CAlg}_{R}:=\rm{CAlg}(\rm{Mod}_{R}),\]
  \[ \rm{cCAlg}_{R}:= \rm{cCAlg}(\rm{Mod}_{R})\]
  as the category of $R$-algebras and $R$-coalgebras, respectively.
  Moreover, we write $\rm{CAlg}_{R}^{\rm{cn}}$ and $\rm{cCAlg}_{R}^{\rm{cn}}$ for the full
  subcategory spanned by the (co)-algebras whose underlying $R$-module is connective.
\end{definition}


\begin{remark}\label{adjoint}
  Let $f:R\to R\p$ be a morphism of commutative ring spectra, then we have the well known
  adjunction between base change and restriction of scalars
\[\begin{tikzcd}
	{\rm{Mod}_R} & {\rm{Mod}_{R\p}}
	\arrow[""{name=0, anchor=center, inner sep=0}, "{f^\pt}"', shift right=2, from=1-1, to=1-2]
	\arrow[""{name=1, anchor=center, inner sep=0}, "{f_\pt}"', shift right=2, from=1-2, to=1-1]
	\arrow["\dashv"{anchor=center, rotate=90}, draw=none, from=0, to=1]
\end{tikzcd}.\]
The functor $f^{\pt}$ is strong symmetric monoidal, while $f_{\pt}$ is lax symmetric monoidal. Hence,
they induce an adjunction fitting in the commutative diagram
\[\begin{tikzcd}
	{\rm{CAlg}_R} & {\rm{CAlg}_{R\p}} \\
	{\rm{Mod}_R} & {\rm{Mod}_{R\p}}
	\arrow[""{name=0, anchor=center, inner sep=0}, "{f^\pt}"', shift right=2, from=1-1, to=1-2]
	\arrow[""{name=1, anchor=center, inner sep=0}, "{f_\pt}"', shift right=2, from=1-2, to=1-1]
	\arrow[shift left=2, from=1-1, to=2-1]
	\arrow[shift left=2, from=1-2, to=2-2]
	\arrow[""{name=2, anchor=center, inner sep=0}, "{f^\pt}"', shift right=2, from=2-1, to=2-2]
	\arrow[""{name=3, anchor=center, inner sep=0}, "{f_\pt}"', shift right=2, from=2-2, to=2-1]
	\arrow["\dashv"{anchor=center, rotate=90}, draw=none, from=0, to=1]
	\arrow["\dashv"{anchor=center, rotate=90}, draw=none, from=2, to=3]
\end{tikzcd}.\]
The functor $f_{\pt}$ is however \textit{not oplax symmetric monoidal} and so does not induce
a functor on coalgebras. Nonetheless, since colimits of coalgebras are formed underlying and
$f^{\pt}: \rm{Mod}_{R}\to \rm{Mod}_{R\p}$ commutes with colimits, so does
$f^{\pt}: \rm{cCAlg}_{R} \to \rm{cCAlg}_{R\p}$. Hence, by the adjoint functor theorem, the base
change functor on coalgebras does admit a right adjoint, which we denote $f_{!}$. Notice that
the diagram
\[\begin{tikzcd}
	{\rm{cCAlg}_R} & {\rm{cCAlg}_{R\p}} \\
	{\rm{Mod}_R} & {\rm{Mod}_{R\p}}
	\arrow["{f_!}"', from=1-2, to=1-1]
	\arrow[shift left=2, from=1-1, to=2-1]
	\arrow[shift left=2, from=1-2, to=2-2]
	\arrow["{f_\pt}"', from=2-2, to=2-1]
\end{tikzcd}\]
does \textit{not commute}. Indeed, since $R\p$ and $R$ are the terminal objects in $\rm{cCAlg}_{R\p}$
and $\rm{cCAlg}_{R}$ respectively, we get that $f_{!}(R\p)= R$ as $f_{!}$ preserves limits. We do not
know of a general formula for $f_{!}$, however we will see that it plays a central role in formulating
the deformation theory of coalgebras.
\end{remark}
\subsection{Presentably symmetric monoidal $\infty$-categories and duality}
We now restrict our attention to coalgebras in \textit{presentably symmetric monoidal} $\infty$-categories.
We have already seen in Proposition~\ref{present} that such coalgebra categories are again presentable,
which is essential for the existence of the right adjoint $f_{!}:\rm{cCAlg}_{R\p} \to \rm{cCAlg}_{R}$
discussed in Remark~\ref{adjoint}. In a presentably symmetric monoidal $\infty$-category we also have a
well behaved notion of \textit{duality} between coalgebras and algebras, which we wil discuss next.
\begin{definition}
  Let $\cl{P}\rm{r^{L}}$ denote the $\infty$-category of presentable $\infty$-categories and functors
  which commute with colimits. By~\cite[][Proposition 4.8.1.15.]{ha} $\cl{P}\rm{r^{L}}$ inherits a natural
  symmetric monoidal structure, such that we have a map $\cl{C}\times \cl{D}\to \cl{C}\otimes \cl{D}$
  exhibiting $\rm{Fun}^{\rm{L}}(\cl{C}\otimes \cl{D},\cl{E})$ as the category of functors
  $\cl{C}\times \cl{D}\to \cl{E}$ which commute with colimits in each variable separately.
  We call $\rm{CAlg}(\cl{P}\rm{r^{L}})$ the category of \textit{presentably symmetric monoidal}
  $\infty$-categories.
\end{definition}

\begin{example}
  If $R$ is an $\bb{E}_{\infty}$-ring then the category of $R$-module spectra $\rm{Mod}_{R}$
  is presentably symmetric monoidal.
\end{example}

\begin{remark}
  If $\cl{C}$ is presentably symmetric monoidal, the functor $\blank \otimes \blank$ commutes with colimits
  in both variables separately. Thus, since $\cl{C}$ is presentable, for every $X\in \cl{C}$
  the functor $X\otimes \blank :\cl{C} \to \cl{C}$ admits a right adjoint, i.e.~$\cl{C}$ is also
  closed monoidal. We denote this right adjoint by $\rm{map}_{\cl{C}}(X, \blank)$.
  This assignment defines a functor $\cl{C}\op \to \rm{Fun}(\cl{C}, \cl{C})$, and we denote
  its adjoint as
  \[ \rm{map}_{\cl{C}}(\blank, \blank): \cl{C}^{\op} \times \cl{C} \to \cl{C}.\]
  Notice that, since $\blank \otimes \blank$ is symmetric, we have that
  \begin{align*}
    \rm{Map}_{\cl{C}}(Y, \rm{map}_{\cl{C}}(X,Z)) &\simeq\rm{Map}_{\cl{C}}(X \otimes Y, Z)\\
                                                 &\simeq  \rm{Map}_{\cl{C}}(X, \rm{map}_{\cl{C}}(Y,Z))\\
    &\simeq \rm{Map}_{\cl{C}\op}(\rm{map}_{\cl{C}}(Y,Z), X)
  \end{align*}
  for all $X,Y,Z\in \cl{C}$. Thus, for each $Z\in \cl{C}$, the functor $\rm{map}_{\cl{C}}(\blank, Z)$ is
adjoint to itself.
\end{remark}

For a presentably symmetric monoidal $\infty$-category $\cl{C}$ there is also a ``generalized suspension
coalgebra'' functor $\cl{S} \to \rm{cCAlg}(\cl{C})$ which we now construct. In particular this gives
the coalgebra structure on the $R$-module $C_{\pt}(X; R)$ for $R$ any $\bb{E}_{\infty}$-ring and $X$ a space.

\begin{proposition}\label{chains}
  Let $\cl{C}$ a presentably symmetric monoidal $\infty$-category. Then the functor
  \begin{align*}
    \cl{S} \to \cl{C} \qquad X \mapsto 1_{\cl{C}}[X]
  \end{align*}
  which sends a space $X$ to the colimit over the constant diagram $X \to \pt \rar{1_{\cl{C}}} \cl{C}$
  is symmetric monoidal with respect to the cartesian monoidal structure on $\cl{S}$.
\end{proposition}

\begin{proof}
  Since $\blank \otimes \blank$ commutes with colimits in both variables
  separably by assumption, we have that:
  \begin{align*}
    (\colim_X 1_{\cl{C}})\otimes (\colim_Y 1_{\cl{C}})
    \simeq \colim_X \colim_Y (\underbrace{1_{\cl{C}}\otimes 1_{\cl{C}}}_{\simeq 1_{\cl{C}}}) \simeq \colim_{X \times Y} 1_{\cl{C}}.
  \end{align*}
\end{proof}

\begin{lemma}
  Suppose $\cl{C}$ is an $\infty$-category which admits finite products and  equip it
  with the cartesian monoidal structure.
  Then the forgetful functor $\rm{cCAlg}(\cl{C}) \to \cl{C}$ is an equivalence,
  with inverse given by equipping an object $X \in \cl{C}$ with the comultiplication
  given by the diagonal map $X \to X \times X$ and counit given by the terminal map $X \to \pt_{\cl{C}}$.
\end{lemma}
\begin{proof}
  By~\cite[][Corollary 2.4.4.10.]{ha} the map $\rm{CAlg}(\cl{C}\op) \to \cl{C}\op $ is an equivalence, hence
  the claim follows by applying $(\blank)\op$.
\end{proof}

\begin{example}\label{homology}
By Proposition~\ref{chains}, for each $\bb{E}_{\infty}$-ring $R$ the singular chains functor
\begin{align*}
  \cl{S} &\to \rm{Mod}_{R}\\
  X &\mapsto R[X]
\end{align*}
is strong symmetric monoidal and thus induces a functor
\[ R[\blank]: \cl{S}\simeq \rm{cCAlg}(\cl{S})\to \rm{cCAlg}_{R}. \]
Hence, the $R$-homology of a space $X$ carries a natural coalgebra structure.
\end{example}


\begin{construction}\label{dual}
Suppose $\cl{C}$ is a closed symmetric monoidal category, then we have a natural map
\[ \rm{map}_{\cl{C}}(A,B)\otimes \rm{map}_{\cl{C}}(A,B) \to \rm{map}_{\cl{C}}(A\otimes A, B \otimes B)\]
which is adjoint to the double evaluation
\[ \left( A\otimes \rm{map}_{\cl{C}}(A,B)\right)\otimes \left( A \otimes \rm{map}_{\cl{C}}(A,B)\right)\rar{\rm{ev}\otimes \rm{ev}}B \otimes B. \]
This can be phrased elegantly by saying that the functor
\[ \rm{map}_{\cl{C}}(\blank, \blank): \cl{C}\op\times\cl{C}\to \cl{C}\]
is lax monoidal with respect to the monoidal structure on $\cl{C}\op \times \cl{C}$, defined by
\[ (A,B)\otimes_{\cl{C}\op \times \cl{C}} (C,D):= (A \otimes C, B \otimes D).\]
Thus, it induces a functor
\[ \rm{map}_{\cl{C}}(\blank, \blank): \rm{CAlg}(\cl{C}\op\times \cl{C})
  \simeq \rm{cCAlg}(\cl{C})\op \times \rm{CAlg}(\cl{C})
  \to \rm{CAlg}(\cl{C}),\]
so in particular, for each $R \in \rm{CAlg}(\cl{C})$ we get a functor
\begin{align*}
  \rm{map}_{\cl{C}}(\blank, R):\rm{cCAlg}(\cl{C})\op = \rm{CAlg}(\cl{C}\op) \to \rm{CAlg}(\cl{C}).
\end{align*}
For a coalgebra $A \in \rm{cCAlg}(\cl{C})$ we call the algebra $\rm{map}_{\cl{C}}(A, 1_{\cl{C}})$ the
\textit{dual algebra} of $A$.
\end{construction}

\begin{example}
  As a special case, we see that for every $\bb{E}_{\infty}$-ring $R$ and every space $X$
  the $R$-cohomology of $X$
  \[\rm{map}_{\rm{Sp}}(\S[X],R) \simeq  \rm{map}_{R}(R[X], R) \simeq \lim_{X}R = R^{X}\]
  inherits a ring structure from the coalgebra structure on the $R$-homology $R[X]$.
\end{example}

\begin{proposition}
  Let $\cl{C}$ be a symmetric monoidal $\infty$-category and denote by
  $\cl{C}^{\rm{dual}}$ the full subcategory of dualizable objects. Then the
  assignment $X\mapsto X^{\vee}$ induces a strong monoidal equivalence
  $(\cl{C}^{\rm{dual}})\op \simeq \cl{C}^{\rm{dual}}$. Moreover, if $\cl{C}$ is
  closed monoidal, then the dual is given by  $X^{\vee}\simeq \rm{map}_{\cl{C}}(X, 1_{\cl{C}})$.
\end{proposition}

\begin{proof}
This is~\cite[][Proposition 3.2.4]{ellI}.
\end{proof}

\begin{corollary}\label{duality}
 For every symmetric monoidal $\infty$-category $\cl{C}$ the functor
  \[ \rm{cCAlg}(\cl{C}^{\rm{perf}})\op \rar{\sim} \rm{CAlg}(\cl{C}^{\rm{perf}})\]
  \[ A \mapsto A^{\vee}\]
  is an equivalence of categories with inverse taking $R\in \rm{CAlg}(\cl{C}^{\rm{perf}})$
  to the dual $R^{\vee}$ with the induced coalgebra structure.
\end{corollary}

\begin{remark}
  Observe that, since the tensor product of two dualizable objects $X,Y \in \cl{C}$ is again
  dualizable with $(X \otimes Y)^{\vee}\simeq X^{\vee}\otimes Y^{\vee}$, the inclusion functor $\cl{C}^{\rm{perf}} \rari{} \cl{C}$
  exhibits $\rm{cCAlg}(\cl{C}^{\rm{perf}})$ as the full subcategory of $\rm{cCAlg}(\cl{C})$ spanned
  by those coalgebras whose underlying object is dualizable.
  We call a coalgebra $A\in \rm{cCAlg}(\cl{C})$ \textit{dualizable} if it belongs to this category.
\end{remark}
For a symmetric monoidal $\infty$-category $\cl{C}$ and $X,Y\in \cl{C}$ with $Y$, we have that by definition
the space of maps $X \to Y$ is equivalent to the space of maps $X^{\vee}\to Y^{\vee}$. The corresponding statement for maps of coalgebras holds as well, however in the derived setting this is not immediately clear.
To this end we prove a lemma, during which employ the following terminology:\\
For a functor $F:\cl{C}\to \cl{D}$ we say that $A \in \cl{C}$ is $F$-local, if the natural
transformation $\rm{Map}_{\cl{C}}(A, \blank) \to \rm{Map}_{\cl{D}}(F(A), F(\blank))$
is an equivalence.
\begin{lemma}\label{laxlocal}
  Let $\cl{C}, \cl{D}$ be symmetric monoidal $\infty$-categories and $F:\cl{C}\to \cl{D}$ be a lax symmetric
  monoidal functor. Suppose we have $R\in \rm{CAlg}(\cl{C})$ such that each tensor power
   $R^{\otimes n}$ considered as an object in $\cl{C}$ is $F$-local. Then the map
  \[ \rm{Map}_{\rm{CAlg(\cl{C})}}(R, S)\rar{F} \rm{Map}_{\rm{CAlg}(\cl{D})}(F(R), F(S)) \]
  is an equivalence.
\end{lemma}
\begin{proof}
  Since $F$ is lax symmetric monoidal, it induces a map of $\infty$-operads
  \[\begin{tikzcd}
	{\cl{C}^\otimes} && {\cl{D}^\otimes} \\
	& {\rm{Fin}_\pt}
	\arrow["p"', from=1-1, to=2-2]
	\arrow["q", from=1-3, to=2-2]
	\arrow["f", from=1-1, to=1-3]
\end{tikzcd}\]
which takes any $(A_{1}, \dots, A_{n})\in \cl{C}^{\otimes}_{\langle n \rangle}$ to the point
$(F(A_{1}), \dots, F(A_{n}))\in \cl{D}^{\otimes}_{\langle n \rangle}$. A commutative algebra structure on an object
$R\in \cl{C}$ is precisely given by a a section $s_{R}$ of $p$ which takes $\langle n \rangle $ to
$(R, \dots, R) \in \cl{C}_{\langle n \rangle }$ and maps inert morphisms to inert morphisms. Let
$\varphi: \langle n \rangle \to \langle m \rangle $ be a morphism in $\rm{Fin}_{\pt}$ and denote by $p_{i}: \langle n \rangle \to \langle 1 \rangle$ the unique inert
map which sends $i \mapsto 1$. For each $i$ we have a factorization
\[ \langle n \rangle \rar{\psi_{i}} \langle k_{i}\rangle \rar{\pi_{i}} \langle 1 \rangle\]
of $p_{i}\circ \varphi$ into an inert map $\psi_{i}$ and an active map $\pi_{i}$.
Then for each $B= (B_{1}, \dots B_{m})\in \cl{C}^{\otimes}$ we get equivalences
\begin{align*}
  \rm{Map}^{\varphi}_{\cl{C}^{\otimes}}(s_{R}(\langle n \rangle), B)
  &\simeq \prod_{i =1, \dots, m}\rm{Map}^{ p_{i}\circ \varphi}_{\cl{C}^{\otimes}}((R, \dots, R), B_{i})\\
  &\simeq \prod_{i =1, \dots, m} \rm{Map}_{\cl{C}}(R^{\otimes k_{i}}, B_{i})\\
  &\simeq \prod_{i =1 ,\dots, m} \rm{Map}_{\cl{D}}(F(R)^{\otimes k_{i}}, F(B_{i}))\\
  &\simeq \prod_{i =1 , \dots m } \rm{Map}^{p_{i} \circ \varphi}_{\cl{D}^{\otimes}}((f \circ s_{R})(\langle n \rangle), B_{i})\\
  &\simeq \rm{Map}_{\cl{D}^{\otimes}}^{\varphi}((f \circ s_{R})(\langle n \rangle), B),
\end{align*}
hence each value of $s_{R}: \rm{Fin}_{\pt}\to \cl{C}^{\otimes}$ is $f$-local, and thus $s_{R}$
is local with respect to the functor $f_{\pt}: \Gamma(p) \to \Gamma(q)$. Since $\rm{CAlg}(\cl{C})$
and $\rm{CAlg}(\cl{D})$ are full subcategories of $\Gamma(p)$ and $\Gamma(q)$ respectively, the
claim follows.
\end{proof}

\begin{proposition}
Let $\cl{C}$ be a symmetric monoidal $\infty$-category and $A,B \in \rm{cCAlg}(\cl{C})$ with $A$ dualizable.
Then the natural map
\[ \rm{Map}_{\rm{cCAlg}(\cl{C})}(B,A)\to\rm{Map}_{\rm{CAlg}(\cl{C})}(A^{\vee}, B^{\vee})\]
is a homotopy equivalence.
\end{proposition}
\begin{proof}
  Apply Lemma~\ref{laxlocal} to the duality functor $ (\blank)^{\vee}:\cl{C}\op \to \cl{C}$.
\end{proof}
Because algebras are much better understood than coalgebras, especially in $\infty$-land, we will rely
on this proposition to deduce as much as possible about the coalgebraic setting from already
established results.
