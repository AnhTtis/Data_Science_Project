\section{Edit Localization} \label{sec:localization}

As mentioned above, localizing the edit is especially challenging when changing the object's shape. To this end, we present two 
techniques that assist in localizing the edit from two different aspects. 
These localization techniques are crucial for successfully generating object shape variations. As we shall show, other known editing methods can also benefit from integrating them, leading to better localized manipulations.

\begin{figure}
    \includegraphics[width=\linewidth]{images/Attention-Based_Shape_Localization_figure.pdf}
    \caption{Attention-based shape localization. Refer to Section~\ref{sec:attn-based-loc} for more details.}
    \vspace{-12pt}
    \label{fig:attn_localization}
\end{figure}




\subsection{Attention-Based Shape Localization} \label{sec:attn-based-loc}

To preserve the shapes of objects in the image, 
we introduce a shape localization technique based on injecting information from the self-attention maps of the source image into the self-attention maps of the generated image.
In the object variations pipeline, we apply this technique to objects that we aim to preserve. %
Injecting the full self-attention maps, even for a few steps, accurately preserves the structure of the original image, but at the same time prevents noticeable shape changes in the object we aim to change.

Our technique, depicted in Figure~\ref{fig:attn_localization}, revolves around a selective injection of self-attention maps. 
Consider a specific self-attention layer $l$ in the denoising network, which receives features of dimension $N\times N$, and the attention map formed by this layer, $S_t^{(l)}$, whose dimensions are $N^2 \times N^2$. The value $S_t^{(l)}[i, j]$ in the map indicates the extent to which pixel $j$ affects pixel $i$. In other words, row $i$ of the map shows the degree to which each pixel impacts pixel $i$, while column $j$ displays the degree to which pixel $j$ impacts all other pixels in the image. To preserve the shape of an object, we inject the rows and columns of the self-attention map that correspond to the pixels containing the object of interest.
Specifically, for a given denoising timestep $t$, the self-attention layer $l$,  and the self-attention map $S_t^{(l)}$, we define a corresponding mask $M_t^{(l)}$ by:
\vspace{-2pt}
\begin{equation}
    M_t^{(l)}[i, j] = 
    \begin{cases}
           1 &  i\in O_t^{(l)} \enspace\text{or}\enspace j \in O_t^{(l)}\\
           0 &\text{otherwise}, \\ 
         \end{cases}
\vspace{-2pt}
\end{equation}
where $O_t^{(l)}$ is the set of pixels corresponding to the object we aim to preserve. We explain later how we find $O_t^{(l)}$. After defining the mask $M_t^{(l)}$, the self-attention map in the newly generated image is changed to be: %
\vspace{-2pt}
\begin{equation}
    S_t^{*(l)} \leftarrow M_t^{(l)} \cdot S_t^{(l)} + (1 - M_t^{(l)}) S_t^{*(l)},
\end{equation}
where $S_t^{(l)}$ and $S_t^{*(l)}$ are the self-attention maps of the original and the newly generated images, respectively.
Additional mask controls are presented in the supplementary.

To find the pixels in which an object is located (\ie defining the set of pixels $O_t^{(l)}$), we leverage the cross-attention maps. These maps model the relations between each pixel in the image and each of the prompt's tokens. For an object we aim to preserve, we consider the cross-attention map of the corresponding token in the prompt. We then define the set $O_t^{(l)}$ of the object's pixels to be pixels with high activation in the cross-attention map by setting a fixed threshold.


\subsection{Controllable Background Preservation}

As shown in previous works~\cite{pnpDiffusion2022, hertz2022prompt}, self-attention injection preserves mainly structures.  
Therefore, we introduce a \emph{controllable background preservation} technique which preserves the appearance of the background and possibly some user-defined objects, specified by their corresponding nous in the input prompt. We give the user control to set the user-defined objects to be preserved since different images may require different configurations. 
For example, in Figure~\ref{fig:overview}, a user may want to preserve the oranges if the basket's size fits, while in other cases where the size of the basket is changed, it is desirable to change the oranges to properly fill a basket with a modified size.

To preserve the appearance of the desired regions, at $t=T_1$ we blend the original and the generated images, taking the changed regions (\eg, the object of interest) from the generated image and the unchanged regions (\eg, background) from the original image.
Next, we present our novel segmentation approach and describe the blending.

\vspace{-14pt}
\paragraph{Self-segmentation}
We perform the segmentation on noised latent codes and, as such, off-the-shelf semantic segmentation methods cannot be applied. Hence, we introduce a segmentation method that segments the image based on self-attention maps, and labels each segment by considering cross-attention maps. The method is based on the premise that internal features of a generative model encode the information needed for segmentation~\cite{Collins20, zhang21}.

At $t=T_1$, we average the $32^2 \times 32^2$ self-attention maps from the entire denoising process. We obtain an attention map of size $32^2 \times 32^2$, reshape it to $32 \times 32 \times 1024$, and cluster the deep pixels with the K-Means algorithm, where each pixel is represented by the $1024$ channels of the aggregated self-attention maps. Each resulting cluster corresponds to a semantic segment of the generated image. Several segmentation results are illustrated in Figure~\ref{fig:segmentation}.

The object instance segmentation task is to identify and delineate distinct objects stored in containers in a warehouse. In the context of robotic object manipulation, instance segmentation is used to inform downstream robotic processes such as grasp generation, motion planning, and placement. Accuracy of instance segmentation can have an impact on picking success, object identification, and defects introduced in the process. For example, under-segmentation can result in picking multiple objects at a time, while over-segmentation can result in a bad choice of grasp leading to damage or dropping of objects. Fig.\ \ref{fig:segmentation_subsets}(a) shows manually annotated object segments on the pick-image. Presence of deformable and transparent objects in clutter makes the task challenging.

Our object instance segmentation dataset contains 50K+ images of objects stored in containers in a warehouse with 500K+ annotations. The annotations include instance-level segmentation masks and bounding box for two classes (object and container). Technicians with task-specific training generated high-quality annotations for object boundaries and object class which are verified by two additional quality assurance technicians.

%The verified images and annotations are added to the dataset.

% Object instance segmentation 
%Instance segmentation is used in object manipulation to identify and outline distinct objects presented in storage units. The resulting object segments are useful for subsequent item understanding, grasp detection, and manipulation planning.
% Good instance segmentation can increase picking eligibility, improve identification success, as well as reduce pick defects and item damage.

%Accurate segmentation is one of the critical enablers for scaling high-performing robotic manipulation at Amazon.
%Reliable segmentation at the Amazon scale is very challenging. The segmentation algorithm needs to generalize to millions of unique items in Amazon warehouses, many of which are unseen. The algorithm must be robust in heavy clutter and occlusion, as objects are packed tightly in containers. Finally, the algorithm needs to transfer successfully to changing containers, lighting conditions, and ... 

%Meanwhile, academic researchers in robotic manipulation are facing similar challenges. Robots working in an unstructured environment need to handle a wide variety of seen and novel items, presented in totes, drawers, shelves, and on tabletops. To operate reliably with this diversity, the robot needs to learn the object concepts from a wide distribution of object data to generalize to unseen items. 

%We present the large-scale Amazon warehouse instance segmentation dataset. The dataset contains over 50k images of Amazon products placed in containers and has over 500k instance annotations. 
%The dataset aims to inspire research into object segmentation in clutter and to provide the robotic community with a wide distribution of real-world objects for learning and benchmarking. 
%The annotations include instance-level segmentation masks, bounding boxes, and class labels. Amazon's internal ML labeling team hand labeled the pixel-wise instance masks and class labels. The associates undergo task-specific training and auditing to produce high-quality class labels and segmentation with exact boundaries. Meanwhile, all labeled images are examined by two verifiers to detect defects such as incomplete and missing segments, in-precise boundaries, and wrong classifications. Images and labels are only used when both verifiers vote no issue. 

% We divide the object segmentation dataset into three subsets: 1) \textit{mix-object-tote} which comprises images of objects in yellow and blue totes with region of interest cropped to the boundary of the tote. 2) \textit{zoomed-out-tote-transfer-set} which comprises images of objects in a yellow tote with the tote centered in the image and covering 50\% of the image, and 3) \textit{same-object-transfer-set} which comprises multiple instances of the same object stored in packaging containers (Fig.\ \ref{fig:segmentation_subsets}). The three subsets in the dataset enable us to understand the impact of variation in background, container, and object distribution. The mix-object-tote subset comprises 44,253 images and 467,225 annotations. It has the highest degree of clutter with 10.5 object instances per image. The zoomed-out-tote-transfer-set subset comprises 5,837 images and 43,401 annotations with an average of 7.5 instances per image. The same-object-transfer-set subset comprises 3,323 images and 12,664 annotations with an average of 3.8 instances per image. 

We divide the object segmentation dataset into three subsets. The primary set, \textit{mix-object-tote}, comprises 44,253 images and 467,225 annotations of objects in yellow and blue storage totes. The totes contain a heterogeneous clutter of objects with an average of 10.5 object segments (ranging from 1 to 50 segments) in each image. The other two subsets, namely \textit{zoomed-out-tote-transfer-set} and \textit{same-object-transfer-set} (Fig.\ \ref{fig:segmentation_subsets}(b) and (c)) enable us to understand the impact of variation in data distribution. The \textit{zoomed-out-tote-transfer-set} subset with 5,837 images and 43,401 annotations captures images of containers from a different warehouse. It poses a transfer learning challenge due to significant differences in background, scale, and object distribution. The \textit{same-object-transfer-set} subset contains 3,323 images and 12,664 annotations. It captures a common and visually challenging scenario in warehouses where multiple instances of the same object are tightly packed in a container. 

%We divide the dataset into three subsets to highlight the challenge of transferring learned knowledge across tasks. The first and largest subset is called the mix-object-tote. It comprises images of mixed objects in yellow or blue totes. The second subset is named the zoomed-out-tote-transfer-set. It contains mixed objects placed in a yellow tote but captured with sensors placed further away from the tote and under different lighting. The third subset is named the same-object-transfer-set. It contains multiple instances of the same object placed tightly packed in different storage containers (Fig.~\ref{fig:segmentation_items}).

\begin{figure}
	\centering
	\includegraphics[width = 0.5\textwidth]{images/three-itemsets.jpg}
	\caption{(a) Segmentation annotation overlaid on an image from {\it mix-object-tote}. Each identifiable item is segmented regardless of its size and occlusion. Multiple objects in the same package are considered as one object and is delineated by the boundary of the package. In particular, items wrapped in transparent packaging are segmented by the peripheral of the package, although other products may be seen through them. (b-c) Example images from {\it zoomed-out-tote-transfer-set} and {\it same-object-transfer-set} subsets representing variations in background, scale, and clutter.}
	\label{fig:segmentation_subsets}
	\vspace{-0.2in}
\end{figure}



%The mix-object-tote is the largest dataset among the three subsets, containing 44,253 images and 467,225 annotations. On average, it has the highest clutter level of 10.5 instances per tote. The zoomed-out-tote-transfer-set contains 5,837 images and 43,401 annotations, averaging 7.5 objects per tote. The same-object-transfer-set dataset has 3,323 images and 12,664 annotations, averaging 3.8 objects per scene. We adopt a class-agnostic labeling scheme where each pick scene image is labeled with two classes: Tote and Object. Examples of labeled images are shown in Fig.~\ref{fig:segmentation_rules}:

% Add (b) to Fig. 3(a)
%\begin{figure}
%	\centering
%	\includegraphics[width = 0.5\textwidth]{images/segmentation_masks.png}
%	\caption{Each identifiable distinct item is segmented regardless of its size and occlusion. Multiple objects in the same package are considered as one object and is delineated by the boundary of the package. In particular, items wrapped in transparent packaging (e.g., plastic bags) are segmented by the peripheral of the packages, although other products may be seen through them.}
%	\label{fig:segmentation_rules}
%\end{figure}

%Accuracy and speed are both critical metrics used to evaluate segmentation algorithms for warehouse picking. 
To establish a performance baseline, we trained Matterport's implementation of Mask R-CNN~\cite{matterport_maskrcnn_2017, DBLP:journals/corr/HeGDG17} with ResNet-50 backbone~\cite{DBLP:journals/corr/HeZRS15} on the {\it mix-object-tote} dataset. Default training schedule (for MS-COCO) and hyper-parameters were used along with a train-valid-test split of 0.7:0.15:0.15. % and the images are split by timestamps, so similar images from the same order do not exist across splits. . 
Table~\ref{table:segmentation_results_inference} shows the results for our baseline experiment. Mean average precision ($mAP$) for a threshold of 0.5 ($mAP_{50}$) and 0.75 ($mAP_{75}$) are used to evaluate the performance of the baseline model on test set. 
%For computation time we cite the 5 fps as reported in ~\cite{DBLP:journals/corr/HeGDG17}, and we test the segmentation accuracy on the testing split of all three subsets, and the evaluation results ($mAP_{50}$ and $mAP_{75}$) are reported in :


We observe that applying model weights trained on {\it mix-object-tote} to the {\it zoomed-out-tote-transfer-set} ($mAP_{50}=0.25$) and {\it same-object-transfer-set} subsets ($mAP_{50}=0.11$) yields poor results. While techniques like transfer learning can improve performance on a new scenario when a reasonable amount of domain-specific labeled data is available, labeling specifically for each variation is time-consuming, if feasible at all. The ultimate goal is to readily transfer segmentation to new scenarios with minimal additional annotations.

%It should be noted that directly applying weight trained on mix-object-tote to the two untrained subsets yields poor results. While techniques like transfer learning can significantly improve performance on a new task when a reasonable amount of task-specific labeled data is available, labeling specifically for each variation is time-consuming, if feasible at all. The ultimate goal is to readily transfer segmentation to new tasks with minimal additional annotations. 

\setlength{\tabcolsep}{4pt}
\begin{table}
\centering
\caption{Mask R-CNN performance for object segmentation task. The model was trained on \textit{mix-object-tote} dataset}
\label{table:segmentation_results_inference}
\begin{tabular}{@{}rccc@{}}
    \hline
 & mix-object-tote & zoomed-out-tote- & same-object- \\ 
 & & transfer-set & transfer-set \\ \hline
$mAP_{50}$  & 0.72 & 0.25 & 0.11 \\ \hline
$mAP_{75}$  & 0.61 & 0.19 & 0.10\\ \hline
\end{tabular}
\end{table}
\setlength{\tabcolsep}{1.4pt}

We observe that segmentation performance for our baseline model has a strong correlation to the level of clutter. Fig.~\ref{fig:segmentation_clutter} shows that the performance drops significantly as the number of ground-truth object instances increases in the image. The $mAP_{50}$ score drops sharply from 0.95 when the tote has one to five object instances to a low of 0.38 when there are more than 26 object instances in the image. This motivates developing algorithms that are robust against clutter and occlusion to further improve object segmentation performance.

\begin{figure}[h]
	\centering
	\includegraphics[width = 0.45\textwidth]{images/segmentation_performance}
	\caption{Performance on {\it mix-object-tote} with varying degree of clutter.}
	\label{fig:segmentation_clutter}
	\vspace{-0.15in}
\end{figure}
















\begin{figure*}
    \centering
    \setlength{\tabcolsep}{0pt}
    {\scriptsize
    \begin{tabular}{cc ccc cc cc ccc}
        Original (Synthetic) &&
        \multicolumn{3}{c}{$\longleftarrow$ Object level variations $\longrightarrow$} &&
        Original (Real) &&
        \multicolumn{3}{c}{$\longleftarrow$ Object level variations $\longrightarrow$} \\
        \includegraphics[width=0.12\textwidth]{images/our_results/cactus/cactus.jpg} &
        { } &
        \includegraphics[width=0.12\textwidth]{images/our_results/cactus/plant.jpg} & 
        \includegraphics[width=0.12\textwidth]{images/our_results/cactus/dirt.jpg} & 
        \includegraphics[width=0.12\textwidth]{images/our_results/cactus/desert.jpg} &
        { } &
        \includegraphics[width=0.12\textwidth]{images/real_images/a_bottle_on_the_table/real.jpg} &
        { } &
        \includegraphics[width=0.12\textwidth]{images/real_images/a_bottle_on_the_table/rum.jpg} & 
        \includegraphics[width=0.12\textwidth]{images/real_images/a_bottle_on_the_table/schol.jpg} & 
        \includegraphics[width=0.12\textwidth]{images/real_images/a_bottle_on_the_table/ale.jpg} 
        \\
        && \multicolumn{3}{c}{``A \emph{cactus} and a ball in the desert''} & &
        && \multicolumn{3}{c}{``A \emph{bottle} on the table''} \\
        \includegraphics[width=0.12\textwidth]{images/our_results/sponge/sponge.jpg} &
        { } &
        \includegraphics[width=0.12\textwidth]{images/our_results/sponge/bread.jpg} & 
        \includegraphics[width=0.12\textwidth]{images/our_results/sponge/dough.jpg} & 
        \includegraphics[width=0.12\textwidth]{images/our_results/sponge/rock_sponge.jpg} &
        { } &
        \includegraphics[width=0.12\textwidth]{images/real_images/a_bowl_of_rice_on_the_table/real.jpg} &
        { } &
        \includegraphics[width=0.12\textwidth]{images/real_images/a_bowl_of_rice_on_the_table/pott.jpg} & 
        \includegraphics[width=0.12\textwidth]{images/real_images/a_bowl_of_rice_on_the_table/food.jpg} & 
        \includegraphics[width=0.12\textwidth]{images/real_images/a_bowl_of_rice_on_the_table/tray.jpg} 
        \\
        && \multicolumn{3}{c}{``A \emph{sponge} and sink''} & &
        && \multicolumn{3}{c}{``A \emph{bowl} of rice on the table''} \\
        \includegraphics[width=0.12\textwidth]{images/our_results/hamster/hamster_eating_watermelon_on_the_beach.jpg} &
        { } &
        \includegraphics[width=0.12\textwidth]{images/our_results/hamster/hamster_eating_milk_on_the_beach.jpg} & 
        \includegraphics[width=0.12\textwidth]{images/our_results/hamster/hamster_eating_pizza_on_the_beach.jpg} & 
        \includegraphics[width=0.12\textwidth]{images/our_results/hamster/hamster_eating_mango_on_the_beach.jpg} &
        { } &
        \includegraphics[width=0.12\textwidth]{images/real_images/a_vase_with_flowers_on_the_table/real.jpg} &
        { } &
        \includegraphics[width=0.12\textwidth]{images/real_images/a_vase_with_flowers_on_the_table/bowl2.jpg} & 
        \includegraphics[width=0.12\textwidth]{images/real_images/a_vase_with_flowers_on_the_table/flower2.jpg} & 
        \includegraphics[width=0.12\textwidth]{images/real_images/a_vase_with_flowers_on_the_table/jar.jpg} 
        \\
        && \multicolumn{3}{c}{``A hamster eating \emph{watermelon} on the beach'' } & &
        && \multicolumn{3}{c}{``A \emph{vase} with flowers on the table''} \\
    \end{tabular}
    }
    \vspace{5pt}
    \caption{Object-level variations for various scenes, synthetic and real (inverted~\cite{mokady2022null}). For each scene, the leftmost image is the original one. The \emph{emphasized} word corresponds to the modified object. As observed, our method generates various shape variations for each object.}
    \vspace{-10pt}
    \label{fig:our-results}
\end{figure*} 




Having extracted the semantic segments, we match each segment with a noun in the input prompt. For each segment, we consider the normalized aggregated cross-attention maps of the prompt's nouns, and match each segment to a noun as follows. 
For segment $i$ that corresponds to a binary mask $M_i$, and for a noun $n$ in the prompt that corresponds to a normalized aggregated cross-attention mask $A_n$, we calculate a score $s(i, n) = \sum(M_i \cdot A_n) / \sum(M_i) $. We label segment $i$ with $\argmax_n {s(i, n)}$ if $\max_n {s(i, n) > \sigma}$ and label it as background otherwise. The threshold value $\sigma$ is fixed across all our experiments.

\vspace{-12pt}
\paragraph{Blending the original and generated images}
We use the segmentation map and the corresponding labels to overwrite relevant regions in the newly generated image. We retain pixels from the original image only if they are labeled as background or as a user-defined object in both the original and new images. 
This approach helps to overcome shape modifications in the object of interest, as illustrated by the example of the basket in Figure~\ref{fig:overview}, where the handle region is taken from the newly generated image. After blending the latent images, we proceed with the denoising process.
