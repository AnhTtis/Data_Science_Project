Cardiovascular disease remains a significant problem in modern society. Among non-invasive techniques, the electrocardiogram (ECG) is one of the most reliable methods for detecting abnormalities in cardiac activities. However, ECG interpretation requires expert knowledge and it is time-consuming. Developing a novel method to detect the disease early could prevent death and complication. The paper presents novel various approaches for classifying cardiac diseases from ECG recordings. The first approach suggests the Poincaré representation of ECG signal and deep-learning-based image classifiers (ResNet50 and DenseNet121 were learned over Poincaré diagrams), which showed decent performance in predicting AF (atrial fibrillation) but not other types of arrhythmia. XGBoost, a gradient-boosting model, showed an acceptable performance in long-term data but had a long inference time due to highly-consuming calculation wihtin the pre-processing phase. Finally, the 1D convolutional model, specifically the 1D ResNet, showed the best results in both studied CinC 2017 and CinC 2020 datasets, reaching the F1 score of 85\% and 71\%, respectively, and that was superior to the first-ranking solution of each challenge. The paper also investigated efficiency metrics such as power consumption and equivalent CO2 emissions, with one-dimensional models like 1D CNN and 1D ResNet being the most energy efficient. Model interpretation analysis showed that the DenseNet detected AF using heart rate variability while the 1DResNet assessed AF pattern in raw ECG signals. 
