Many previous works on ECG classification focus on detecting heart disease, including atrial fibrillation, tachycardia, bradycardia, arrhythmia, and other problems \cite{Hong2020, Ribeiro2020, Zhang2021}. Some other studies try to predict mortality rates or demographic characteristics \cite{Attia2019, Lima2021}. The input of these models is usually raw ECG signal in single or multiple leads; however, sometimes only ECG images were used \cite{Jun2018}. % Jun2018 was discussed deeply below

The data could be transformed into good features before feeding to a machine learning model or be processed automatically to produce a high dimensional representation by a deep learning model. In the feature-based method, four groups of descriptors can be extracted from the ECG signal: time-domain features, nonlinear-domain features, distance-based features, and time-series features (\cite{Bertsimas2021}). In the next step, the classifier, such as logistics regression, support vector machine, or boosting algorithms, gives the prediction based on these features. Besides that, the deep learning-based approaches perform feature extracting and predicting simultaneously. These models take advantage of multiple layer perceptron \cite{Ribeiro2020}, CNN model \cite{Hong2020}, or LSTM model \cite{Corradi2019} for extracting the high-level features of ECG signal.

% ECG Classification problem

The work by Jun et al. \cite{Jun2018} focused on predicting arrhythmia diseases based on ECG beat. The author used the data from the MIT-BIH arrhythmia database. In the pre-processing phase, the ECG signal was partitioned and centered based on the Q-wave peak time before being plotted to generate a 128 x 128 grayscale image as the input for learning. Data augmentation was performed by cropping and resizing the training data images. The CNN model tried to classify eight labels from these plotting. In that paper, the AlexNet, VGGNet, and a customized CNN architecture were used to optimize the performance. The proposed model reached 0.989 AUC and over 99\% accuracy. The data augmentation showed the benefit of raising the sensitivity of the models. Compared to our approach on the Poincaré diagram, we used the scatter plot on heart rate variation instead of the raw signal. The used CNN architectures were the novel models, including ResNet and DenseNet. The cropping and resizing augmentation were not applied because this technique could modify the dispersion of the Poincaré plot, which leads to the wrong prediction. Instead, the limited random erasing on the input image was used to regularize the CNN model.

There was a concerted effort by Shenda Hong et al. \cite{Hong2020} to develop an ensemble system to process the waveform data. Hong’s architecture includes three key parts named the model zoo, the ensemble composer, and the real-time serving system. The model zoo takes the responsibility for training the data with several collections of hyperparameters, while the ensemble composer would figure out the best set of models under the constraints of validation performance and latency. The serving system takes the output of the ensemble composer and then deploys a system that can handle massive input data as well as queries in real time. The authors performed many experiments on signal leads with several types of deep learning models such as CNN, ResNet, ResNeXt, and RegNet. The experiments showed that this system could reach an accuracy of 95\% and a latency of under a second on a 64-bed simulation.

In the study by Ribeiro et al. \cite{Ribeiro2020}, the authors investigate how to train a one-dimensional convolutional neural network to predict cardiac diseases. The trained data includes more than two million 12-lead ECG signals that are between 7 to 10 seconds. The dataset was annotated semi-automatically that combines algorithms and human verification. The chosen model was based on the ordinary 1D ResNet architecture that has 4 Residual Blocks with kernel size increasing from 128 to 320. Finally, this pipeline surpassed human performance with an F1 score of over 80\%.

The work by Zhang’s team \cite{Zhang2021} proved the dominant results of deep learning compared to the feature-based machine learning model. In this study, 1D CNN was trained on 12-lead ECG recordings from CPSC 2018 database. The model could predict 9 subtypes of arrhythmias with an F1 score of over 80\%. The impressive idea of this work is to use the SHAP value to explain the model output at the individual level as well as the population level. At the individual level, the model could show the characteristics of ECG that support the model decision such as the abnormal QRS pattern following the P waves in AF, or the prolongated PR distance in IAVB. At the population level, the authors examined the contribution of each lead to model output, so that the lead II, aVR, V1, V2, V5, and V6 are the most important leads in their model.

% Beat classification
Besides the signal classification problem, the beat-level annotation is also investigated. Corradi et al. \cite{Corradi2019} took advantage of recurrent neural networks to annotate the ECG with over 90\% accuracy in many public datasets. Their model was also efficient enough to deploy in the wearable device. In 2020, Teplitzky presented BeatLogic.~\cite{Teplitzky2020} This was a comprehensive system that could detect and classify the cardiac beat and rhythm simultaneously by using the 1D-CNN-based model. As a result, BeatLogic outperformed other methods on every mentioned task.  

Our proposed method also takes advantage of the Poincaré plot, a standard procedure for studying heart rate variability. Early work by \cite{Park2009} used point coordinates in the Poincaré plot to calculate the interval and variability of interbeat before using a support vector machine model to classify AF and non-AF patients. \cite{Lian2011} combined the RR interval and the difference of RR intervals to make a robust AF classifier with few heartbeats. \cite{Zhang2015} deploys the ensemble of neural networks to extract five geometric patterns in the Poincaré plot, including comet, torpedo, fan, double side lobe, and multiple side lobes, before using them to classify the major cardiac arrhythmias. \cite{Bashar2021} proposed a modified Poincaré plot from heart rate difference. The features extracted from these plots by image processing help diagnose AF from Premature Atrial/Ventricular Contraction.