Cardiovascular disease is a serious public health problem that affects millions of people worldwide and is also a leading cause of death~\cite{Tsao2022}. The expense of healthcare, lost productivity, and a diminished quality of life due to heart illness has a significant economic and social impact on individuals, families, and society as a whole~\cite{mensah2019global}. This emphasizes the value of early disease identification. While the electrocardiogram (ECG) is considered as the most crucial method for detecting and diagnosing cardiac problem~\cite{Kligfield2007}, it takes time and requires trained professionals with specialized skills to interpret ECGs. The ECG analysis task includes beat annotation and signal classification. While the former deals with aligning the signal segment to the heart contraction, the latter tries to predict the disease from the signal data.

In the domain of ECG classification, there are a number of methods ranging from feature-based models to deep-learning based ones. The feature-based approach takes advantage of a feature extraction technique and a machine learning model. The methods for feature extraction are very diverse, however, the domain-dependent features, statistical descriptors, morphological characteristics, and frequency-based features are widely chosen \cite{Selcan2018}.

Challenges (with annotated datasets provided) of ECG classification such as The PhysioNet/Computing in Cardiology Challenge (CinC) 2017 and 2020~\cite{cinc2017, cinc2020} are aimed to provide opportunities for data science community to develop a novel method for automatic detection. While CinC 2017 focused on arrhythmia disease, CinC 2020 contained ECG signals in a wide range of cardiac abnormalities. Although there are many efforts to apply machine learning/deep learning approaches to reach the highest performance, the results were still modest, especially in CinC 2020.

Additionally, there are state-of-the-art methods that improve the performance of signal classification, especially in deep-learning-based methods. Besides the improvement of accuracy, the architecture of models becomes more complicated, so they require more energy to train and have a long inference time. This problem limits the application of the method, especially in handheld and wearable devices.

In this study, we focused on enhancing ESG classification approaches in terms of performance, numerical complexity, inference time, and its interpretability. 


\noindent \textbf{Contribution}. The contribution of our paper is threefold: 

\begin{itemize}
\item First, we introduce a pipeline for heart disease classifier evaluation in terms of performance and numerical complexity. 
\item Second, we achieved a state-of-the-art level (regards to CinC 2017 and CinC 2020 challenges) performance with 1D ResNet model for both CinC 2017 and CinC 2020 benchmarks.
\item Third, we provided interpretation techniques for DenseNet121 and 1D ResNet models.

\end{itemize}
