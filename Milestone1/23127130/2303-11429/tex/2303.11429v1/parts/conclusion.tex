In the paper, we have presented novel various approaches to classify cardiac diseases from ECG recordings. The first approach took advantage of the Poincaré diagram and deep-learning-based image classifiers. ResNet50 and DenseNet121 architecture were chosen to process the graph. The experimental results figured out that these methods are decent for atrial fibrillation but not good at predicting other types of arrhythmia. In particular, the Poincaré-based methods have adequate performance in the CinC~2017 data but not good in the CinC~2020 data. However, RR or NN intervals, and therefore Poincaré diagrams, are much more accessible and can be obtained without the relatively complicated and expensive ECG procedure. Thus, it is still worth studying further in this approach. 
The XGBoost's performance is more impressive in the subset of long-term than the short-term data. This gradient-boosting model has a long inference time because of the expensive calculation in the preprocessing step. 
The one-dimensional convolutional model showed the best results in both studied datasets. Especially the 1D ResNet was superior to the first-ranking solution of each challenge. The residual connection showed its advantages in transferring information while keeping the model not too deep.

We have also investigated the efficiency metrics while training the models, including power consumption and equivalent CO2 emissions. Because of the high workload when processing 2D images, the 2D ResNet and DenseNet are at the top in power-consuming rankings. The XGBoost is energy efficient for the short term, but the power requirement is multiplied many times when training on long-term signals. Since the 1D convolution operator is optimized in the calculation, the unidimensional models like 1D CNN and 1D ResNet are the most energy efficient among the studied methods.

In the aspect of model interpretation, three models (DenseNet, 1D ResNet, and XGBoost) were analyzed to figure out how they discriminate the normal and AF data. The DenseNet detected AF using the heart rate variability, which was measured by the spreading of the data cloud and the presence of data in the upper-left and lower-right in the Poincaré diagram. On the other hand, the 1D ResNet assessed the AF pattern in raw ECG signal similar to a medical expert: this model focused on the area around the QRS complex, which is also the location of P and T waves.
