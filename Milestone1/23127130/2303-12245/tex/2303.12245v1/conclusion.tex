\section{Concluding Remarks}
\label{Conclusion}

In the present paper we have considered the approximation of a class of dynamic PDEs of second order in time by physics-informed neural networks (PINN). We provide an analysis of the convergence and the error of PINN for approximating the wave equation, the Sine-Gordon equation, and the linear elastodynamic equation. Our analyses show that, with feed-forward neural networks having two hidden layers and the $\tanh$ activation function for all the hidden nodes, the PINN approximation errors for the solution field, its time derivative and its gradient can be bounded by the PINN training loss and the number of training data points (quadrature points). 

Our theoretical analyses further suggest new forms for the PINN training loss function, which contain certain residuals that are crucial to the error estimate but would be absent from the canonical PINN formulation of the loss function. These typically include the gradient of the equation residual, the gradient of the initial-condition residual, and the time derivative of the boundary-condition residual. In addition, depending on the type of boundary conditions involved in the problem, our analyses suggest that a norm other than the commonly-used $L^2$ norm may be more appropriate for the boundary residuals in the loss function. 
Adopting these new forms of the loss function suggested by the theoretical analyses leads to a variant PINN algorithm. We have implemented the new algorithm and presented a number of numerical experiments on the wave equation, the Sine-Gordon equation and the linear elastodynamic equation. The simulation results demonstrate that the method can capture the solution field well for these PDEs. The numerical data corroborate the theoretical analyses.  

% what are the implications of these results?
% what is the future plan?
% what is the weakness?
% what are the un-answered questions?


\section*{Declarations}
The authors declare that they have no known competing financial interests or personal relationships that could have appeared to influence the work reported in this paper. 

\section*{Availability of data/code and material }
Data will be made available on reasonable request.


\section*{Acknowledgements}
The work was partially supported by the China Postdoctoral Science Foundation (No.2021M702747), Natural Science Foundation of Hunan Province (No.2022JJ40422), NSF of China (No.12101495), General Special Project of Education Department of Shaanxi Provincial Government (No.21JK0943), and the US National Science Foundation (DMS-2012415).
