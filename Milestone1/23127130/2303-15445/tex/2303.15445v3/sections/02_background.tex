We start with a short introduction to the main types of figurative language \cite{LakoffandJohnson1980,Paul-1970,philip2011colouring}.

 A {\bf metaphor} is a comparison between concepts that makes us think of the target concept in terms of the source concept. For example, in ``You’re a peach!'', a person is equated with a peach, suggesting that they are pleasing or delightful.
 
 A {\bf simile} also compares two things, often introduced by ``like'' or ``as''. A simile is open when the shared properties are not explicitly revealed (``Her heart is like a stone''), and closed when they are (``Her heart is hard as stone''). 
 
 An {\bf idiom} is a group of words with a non-literal meaning that can not be understood by looking at its individual words. E.g.,  ``We're on the same page'' means ``Agreeing about something''. 
 
 Understanding figurative language requires commonsense, general knowledge, and the ability to map between domains. Understanding idioms, in particular, requires profound language and cultural knowledge \citep{Paul-1970, philip2011colouring}. 
 %\dnote{did you come up with those definitions? where did you get them from?}\rnote{Done}