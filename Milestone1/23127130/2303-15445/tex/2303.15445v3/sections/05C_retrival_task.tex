% The {\bf Multimodal Figurative Retrieval Task} examines VL-PTMs' preference for figurative images. The task is to rank the figurative and partially literal images using the model-matching score and calculate the precision at $k$, where $k$ is the number of figurative images. Figure~\ref{fig:second-task-idiom-figurative-vs-caption} shows an example of the task for the idiom ``ruffle someone's feathers''.
\begin{figure}[t!]
\includegraphics[width=0.45\textwidth,height=\textheight,keepaspectratio]{figures/05-ranking.jpg}
\caption{Example of multimodal figurative language retrieval task for the idiom  ``ruffle someone's feathers'' (to unease, cause discomfort to someone). 
%The first row demonstrates one possible expected retrieval order, where Figurative (F) are ranked before Partial Literal (P). The second row shows CLIP-VIT-L/14 retrieval order. Green indicates correct position, and red indicates incorrect position. The retrieval order 
The task is to rank the figurative images above the partial literal ones, based on the images' matching score with the idiom. %All models received a $0$ precision score in this example.
}
 % Green indicates correct retrieval order, and red indicates incorrect retrieval order. 



\label{fig:second-task-idiom-figurative-vs-caption}
\end{figure}
The {\bf Multimodal Figurative Retrieval Task} examines VL-PTMs' preference for figurative images. 
Given a set of figurative and partially literal images, the task is to rank the images using the model-matching score such that the figurative images are ranked higher, and calculate the precision at $k$, where $k$ is the number of figurative images in the input. 

Figure~\ref{fig:second-task-idiom-figurative-vs-caption} shows an example of the task for the idiom ``ruffle someone's feathers''.
We wish to have images of people causing discomfort ranked higher than pictures of birds and feathers. 
%Our objective is to introduce a task to tackle VL-PTMs' preference for partially literal images (As seen in \S\ref{sec:results_and_model_analysis}). 
%The primary component of this task involves using the model-matching score or certainty score to rank both figurative and partially literal images. 
%For instance, when presented with the idiom ``let the cat out of the bag'' (meaning ``to disclose a secret; to let a secret be known''), an image of someone revealing a secret should receive a higher matching score than just an image of a bag. 
This task provides an opportunity for a deeper probe into how the model comprehends figurative language in terms of its preferences.

In this task, we use the same baselines and training methods mentioned in the previous task. We train the supervised model on $3802$ images, making sure the training data does not contain any of the images or idioms that appear in the task.

%OLD - To tackle vision and language models' strong preference toward partially literal images over figurative images, we introduce the preference task. The preference task is to rank the Figurative images higher than partially literal distractors based on the model matching score. First, we rank the figurative phrase images by their matching score from higher to lower, then we define two classes, $|F|$ which consists of the Figurative images, and $|P|$ which consists of the partially literal images. The model then predicts the first $|F|$ images as Figurative and the last $|P|$ images as partially literal images, the $F_1$ score of the model predictions is the preference task score. For instance, when presented with the idiom ``let the cat out of the bag'' (meaning ``to disclose a secret; to let a secret be known''), an image of someone revealing a secret should receive a higher matching score than just an image of a bag. This element of the task allows a further examination of the model understanding of figurative language from a viewpoint of its preference. 
