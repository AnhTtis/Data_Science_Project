% We collected $35$ textual metaphors and $142$ textual similes along with their definitions from online sources. Next, we used the metaphors and similes definitions as search queries to search for figurative and literal images. We manually annotated the resulting images into ``Figurative'' and ``Partial Literal'' categories. In total, we obtained $1107$ figurative images and $1816$ partial literal images for similes, and $333$ figurative images and $729$ literal images for metaphors. 
%\dnote{why a separate pipeline}\rnote{Now its good I think}\\
We collected $35$ textual metaphors and $142$ textual similes, compiled from online lists. Generating search queries from definitions (to find figurative images) is a central part of our pipeline for idioms (Section \ref{sec:idioms-collection}). However, idioms are fixed expressions, but metaphors and similes are much more flexible, as the number of possible comparisons between two things is vast. 

For this reason, we had to adapt our pipeline. For metaphors, we asked three expert annotators to agree upon definitions.
%
%\dnote{situation demonstrating}\rnote{Heart of gold -> A volunteer who distributes food in a soup kitchen}
%
%
For similes, we use the simile itself and the target concept with the shared property (``fast'') as search queries to find figurative images. For literal images that serve as distractors, we use the source and target without the shared property. In some cases, the target concept images are inadequate literal distractors (an image of a car might still be considered figurative for the simile "The car is as fast as a cheetah"). To solve this problem, we include the \emph{antonym} of the property (``A slow car'').
% Previous % In our approach to similes, we used both the simile itself and the target concept with the compared property as search queries to find figurative images. For literal images, we used the source concept, the target concept without the compared property, and the target concept with the antonym of the compared property. We specifically employed antonym images as distractors when a literal image of the target concept proved insufficient. For instance, an image of a gallon of milk doesn't serve as an effective distractor for the simile 'The milk is fresh as a daisy'.

%We collected $35$ textual metaphors and $142$ textual similes from online sources. Next, we used the metaphors and similes search queries to find figurative and literal images. 
\xhdr{Annotation} As the number of images was relatively small, we had two experts from our team manually annotate images. We obtained $1107$ figurative and $1816$ partial literal images for similes, $333$ figurative and $729$ partial literal for metaphors (the other categories were less relevant for the specific data generated by our pipeline). 