The {\bf  Multimodal Figurative Language Detection Task} evaluates VL-PTMs’ ability to choose the image that best visualizes the meaning of a figurative expression. Figure~\ref{fig:first-task-idiom-figurative} shows an example of the task for an idiom, a metaphor, and a simile.

Our goal was to create a difficult and diverse task representing the richness of our dataset (Section \ref{sec:the_irfl_dataset}).
We constructed $810$ {``mixed''} task instances for idioms, metaphors, and similes. Each {``mixed''} instance contains four candidates: one is the correct answer, partially literal distractors, and random images. 

Idiom instances have 1-2 partially literal distractors. Simile  instances contain two literal distractors, one of the target concept without the compared property or with its antonym visualized, and one of the source concept. Metaphor {``mixed''} instances consist of between 1-3 partially literal distractors. 

%In $65\%$ of the idioms task instances, the correct answer is ``Figurative''; in the other $35\%$, the correct answer is ``Figurative  Literal''. 
%Figure~\ref{fig:first-task-idiom-figurative} shows an example of the {``mixed''} multimodal figurative language detection task for metaphor, simile, and idioms. 