\begin{table*}[tp!]
\begin{center}
\small
\begin{tabular}{ m{8em} m{8em} m{8em} m{8em} m{8em}}
\toprule
\multicolumn{5}{c}{\textbf{Idiom:} Touch wood} \\
\multicolumn{5}{c}{\begin{tabular}[x]{@{}c@{}} \textbf{Definitions:} 1) Hopefully 2) Said while touching something wooden, \\ to avert  superstitious bad luck from what has just been said \end{tabular}} \\ 

\midrule
\includegraphics[width=0.18\textwidth,height=5.5em]{figures/touch_wood_caption.jpeg} & 
\includegraphics[width=0.18\textwidth ,height=5.5em]{figures/touch_wood_figurative.jpeg} &
\includegraphics[width=0.18\textwidth ,height=5.5em]{figures/touch_wood_partial_objects.jpeg} & 
\includegraphics[width=0.18\textwidth ,height=5.5em]{figures/touch_wood_figurative_literal.jpg} &  \includegraphics[width=0.18\textwidth ,height=5.5em]{figures/touch_wood_none.jpeg} \\ \midrule

 Literal & Figurative & Partial Literal & Figurative+Literal & None  \\ \midrule

 The image illustrates the phrase literally & 
 The image conveys one or more \emph{definitions} of the idiom &
 Some objects/ entities of the phrase are visualized  (here, wood) & 
 Fits the ``Figurative'' definition and also ``Literal''/``Partial Literal''  &
 The image does not fit any of the other categories  \\ \bottomrule
\end{tabular}


% \small
% \begin{tabular}{ c c c c c c}
% \toprule
% \adjustbox{valign=c}{\includegraphics[width=0.18\textwidth ,height=\textheight,keepaspectratio]{figures/touch_wood_figurative_literal.jpg}} &
% \adjustbox{valign=c}{\includegraphics[width=0.18\textwidth ,height=\textheight,keepaspectratio]{figures/touch_wood_figurative_literal.jpg}} &
% \adjustbox{valign=c}{\includegraphics[width=0.18\textwidth ,height=\textheight,keepaspectratio]{figures/touch_wood_figurative_literal.jpg}} &
% \adjustbox{valign=c}{\includegraphics[width=0.18\textwidth ,height=\textheight,keepaspectratio]{figures/touch_wood_figurative_literal.jpg}} &
% \adjustbox{valign=c}{\includegraphics[width=0.18\textwidth ,height=\textheight,keepaspectratio]{figures/touch_wood_figurative_literal.jpg}} & \midrule

%  Figurative+Literal & The image conveys one or more definitions of the idiom to some extent,
% and it literally illustrates the phrase or visualizes the phrase objects/entities & \adjustbox{valign=c}{\includegraphics[width=0.18\textwidth ,height=\textheight,keepaspectratio]{figures/touch_wood_figurative_literal.jpg}} &\\  \midrule
%  Figurative & The image conveys one or more definitions of the idiom to some extent & \adjustbox{valign=c}{\includegraphics[width=0.18\textwidth ,height=\textheight,keepaspectratio]{figures/touch_wood_figurative_literal.jpg}}&\\ \midrule
%  Caption &  The image illustrates the phrase literally & \adjustbox{valign=c}{\includegraphics[width=0.18\textwidth ,height=\textheight,keepaspectratio]{figures/touch_wood_figurative_literal.jpg}} &\\ \midrule
%  Partial Objects &  The objects/entities of the phrase are visualized in the image & \adjustbox{valign=c}{\includegraphics[width=0.18\textwidth ,height=\textheight,keepaspectratio]{figures/touch_wood_figurative_literal.jpg}} &\\ \midrule
%  None &  The image does not fit any of the categories & \adjustbox{valign=c}{\includegraphics[width=0.18\textwidth ,height=\textheight,keepaspectratio]{figures/touch_wood_figurative_literal.jpg}} &\\  \bottomrule 
% \end{tabular}


% \small
% \begin{tabular}{ m{3.5em} m{5.3cm} c c}
% \toprule
%  Figurative+Literal & The image conveys one or more definitions of the idiom to some extent,
% and it literally illustrates the phrase or visualizes the phrase objects/entities & \adjustbox{valign=c}{\includegraphics[width=0.18\textwidth ,height=\textheight,keepaspectratio]{figures/touch_wood_figurative_literal.jpg}} &\\  \midrule
%  Figurative & The image conveys one or more definitions of the idiom to some extent & &\\ \midrule
%  X &  The image illustrates the phrase literally &\\ \midrule
%  x &  The objects/entities of the phrase are visualized in the image & &\\ \midrule
%  None &  The image does not fit any of the categories & &\\  \bottomrule 
% \end{tabular}


\end{center}
\caption{The table shows the different categories of the relation between an image and a phrase, along with matching images for the idiom "Touch wood". Workers were guided to choose the most suitable relation category by a scheme tree that illustrates the correct thinking process (Figure~\ref{fig:image-task-tree}, Appendix~\ref{sec:annotation_ui}).}
\label{tab:relation-categories}
\end{table*}
