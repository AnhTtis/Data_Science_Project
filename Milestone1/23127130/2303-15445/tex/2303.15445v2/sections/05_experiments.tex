%We evaluate the ability of humans and state-of-the-art image recognition models to detect multimodal figurative language (\S\ref{sec:understanding_task}). Building upon the strong literal preference identified (\S\ref{sec:understanding_task}), we introduce a task focused specifically on multimodal figurative language preference. This task is designed to isolate and examine the preference aspect of multimodal figurative understanding.

% \rnote{4.1 Detection task - explain the task}
% \rnote{4.1.1 baselines}
% \rnote{4.1.2 supervision}
% \rnote{4.1.3 Results and analysis}
% \rnote{4.2 Retrieval Task - Talk about literal preference seen in 4.1, and explain the motivation. Mention we use the same baselines and supervision methods.}
% \rnote{Remove the "We show...." and think on concatenating the idioms and similes and metaphors tables.}
% We evaluate humans and state-of-the-art image recognition models ability to understand figurative language (Section \ref{sec:understanding_task}). We show that IRFL tasks are easy for humans (97\% accuracy) and challenging for models (<27\%). Additionally, we provide a detailed analysis per figure of speech, experiments with idioms and their definitions as input, and with different candidate types. We find that models fail the IRFL task due to their preference for partially literal images over figurative images and introduce a preference task to tackle this problem (Section \ref{sec:ranking Task Analysis}). In addition, we examine the ability of generative models such as Dall-E and Stable Diffusion to generate figurative images for idioms (Section \ref{sec:genearive_models_analysis}). We find that they are unable to generate figurative images given idiomatic phrases. Given the definitions of an idiom, generative models can generate figurative images.

%\yonatan{missing more experiments (What we've discussed from the ``Why is Winoground hard paper'') - Iterate each one of the chapters we discussed and see if you can repeat the experiments in this study}\ron{we can do it only with metaphors and similes, it will not work with idioms. I am in favor on adding it to the next version}
\subsection{Multimodal Figurative Language Detection Task}
\label{sec:understanding_task}
The {\bf  Multimodal Figurative Language Detection Task} evaluates VL-PTMs’ ability to choose the image that best visualizes the meaning of a figurative expression. Figure~\ref{fig:first-task-idiom-figurative} shows an example of the task for an idiom, a metaphor, and a simile.

Our goal was to create a difficult and diverse task representing the richness of our dataset (Section \ref{sec:the_irfl_dataset}).
We constructed $810$ {``mixed''} task instances for idioms, metaphors, and similes. Each {``mixed''} instance contains four candidates: one is the correct answer, partially literal distractors, and random images. 

Idiom instances have 1-2 partially literal distractors. Simile  instances contain two literal distractors, one of the target concept without the compared property or with its antonym visualized, and one of the source concept. Metaphor {``mixed''} instances consist of between 1-3 partially literal distractors. 

%In $65\%$ of the idioms task instances, the correct answer is ``Figurative''; in the other $35\%$, the correct answer is ``Figurative  Literal''. 
%Figure~\ref{fig:first-task-idiom-figurative} shows an example of the {``mixed''} multimodal figurative language detection task for metaphor, simile, and idioms. 

\xhdr{Zero-Shot Baselines}
\label{sec:zero_shot_evaluation}
We evaluate several state-of-the-art vision-and-language models. We use four versions of CLIP models \cite{radford2021learning}: RN50, ViT-B/32, ViT-L/14, and RN50x64/14 with 100M, 150M, 430M, and 620M parameters, respectively. We use the official implementations of ViLT \citep{kim2021vilt}, BLIP \citep{li2022blip}, CoCa ViT-L-14 \citep{yu2022coca}, and BLIP2 \citep{li2023blip2}. We evaluate all models with their default hyper-parameters, except ViLT on idioms, due to its maximum sequence length of 40 tokens. 

The models encode the figurative phrase and the image, producing a matching score for each pair. We choose the image with the highest score as the one best matches the expression.

We also experimented with multimodal chatbot models, including LLaVA \cite{liu2023visual}, InstructBLIP \cite{dai2023instructblip}, OpenFlamingo \cite{awadalla2023openflamingo}, and Otter \cite{li2023otter}. We found that the first two do not support our setting, as they can not handle questions about multiple images; the latter two do support the setting, but did not seem to understand the task, returning mostly nonsense answers. 

% \begin{enumerate}
%     \item CLIP \citep{radford2021learning} is pre-trained with a contrastive objective that can be used without directly optimizing for the task. We use four versions of models with different amounts of parameters: RN50, ViT-B/32, ViT-L/14 and RN50x64/14 with 100M, 150M, 430M and 620M parameters, respectively (RN50 was used during data collection).
%     \item CLIP-ViL \citep{shen2021much}, with 290M parameters, is a pre-trained vision-and-language model that uses CLIP as a visual backbone, rather than CNN based visual encoders that are trained on a small set of manually annotated data.
%     \item ViLT \citep{kim2021vilt}, with 111M parameters, incorporates text embeddings into a Vision Transformer (ViT).
    
% \end{enumerate} 

\xhdr{Supervised Models}
\label{sec:supervised_models_evaluation}
% We join a line of benchmarks that introduce a test set without predefined train splits \citep{thrush2022winoground,rudinger2018gender,emelin-sennrich-2021-wino}, \citep{Bitton2022WinoGAViLGA}. 
We train a supervised model for figurative classification of idioms. We add a binary classifier on top of pre-trained embeddings to classify whether a given image is figurative. We use CLIP (VIT-B/32) model, concatenating the textual idiom embedding to the visual image embedding, followed by a classifier that produces a matching score. A  score above $0.5$ is labeled ‘‘Figurative’’. We use the Adam optimizer \citep{Kingma2014} with a learning rate of $0.001$, batch size of $12$, and train for $7$ epochs. We run the fine-tuned model on the multimodal figurative language detection (\S\ref{sec:understanding_task}) task using the model's matching score. We train the binary classifier on $4790$ images, making sure the training data does not contain any of the images or idioms that appear in the task. We repeat five experiments with different random seeds for each task and take the mean score and std. 

%Unlike understanding idioms, that requires language and cultural knowledge, metaphors and similes require abstraction and mapping between domains \citep{mitchell2021abstraction}. %It should be able to solve unseen cases without extensive training . Given these distinct requirements,
%Thus, supervised experiments on metaphors and similes are outside the scope of this paper, and we leave it to future work.

%We believe that in order to understand metaphors and similes, a machine must be able to abstract and map between domains. It should be able to solve unseen cases without extensive training \citep{mitchell2021abstraction}. Contrary to metaphors and similes, understanding idioms requires language and cultural knowledge that can be learned through extensive training. 
\subsubsection{Human Evaluation}
\label{sec:human_evaluation}
We asked annotators that did not work on previous IRFL tasks to solve the multimodal figurative language detection task. Each instance of the ``mixed'' multimodal detection task was annotated by $5$ annotators, and the correct answer was chosen by the majority. We find that human performance on the data sampled for all figures of speech ranges between $90\%-100\%$ (Table~\ref{tab:mixed-single-choice-results}). Additionally, in $82\%-99\%$ of the instances, there was an agreement between at least four annotators compared to a random chance of $6\%$. Samples from the validation process are presented in Appendix \ref{sec:task_samples}.



\subsubsection{Results and Model Analysis}
\label{sec:results_and_model_analysis}
% \yonatan{add paragraphs with the main title findings of each main finding}
Zero-shot results on the ``mixed'' multimodal figurative language detection task are presented in Table \ref{tab:mixed-single-choice-results}. The best model achieved $22\%$, $30\%$, and $66\%$ accuracy on the idioms\footnote{Idioms were passed along with their definitions as input.}, metaphors, and similes tasks compared to a random chance of $25\%$. These results suggest that \textbf{models do not understand the connection between a figurative phrase and an image as humans do.} We next conduct a fine-grained analysis to examine if models failed because they do not see any connection to the figurative images or rather because they prioritize literal connections over figurative ones.

\begin{table}[tp]
\centering
\scalebox{0.65}{
\begin{tabular}{@{}lcccccc@{}}
\toprule
Categories   & \multicolumn{2}{c}{Idioms}         & Metaphors                  & Similes\\ \midrule
             & Figurative    & Figurative Literal &                    & \\ \midrule
Humans          &  97\%          & 90\%           & 99.7\%             & 100\% \\ \midrule
CLIP-VIT-L/14   &  17\%          & \textbf{56\%}  & 25\%               & \textbf{52\%} \\
CLIP-VIT-B/32   &  16\%          & 44\%           & 23\%               & 45\% \\
CLIP-RN50       &  14\%          & 37\%           & 27\%               & 47\% \\
CLIP-RN50x64/14 &  22\%          & \textbf{56\%}  & \textbf{30\%}      & 52\%\\
LiT             &  \textbf{27\%} & 31\%           & 21\%               & 19\%\\ \midrule
ViLT            &   -            &  -             & 23\%               & 40\% \\ \midrule
\# Unique Phrases  & 48       & 30             & 35                 &  142\\ \midrule
\# Tasks     &  135           & 65             & 333                & 277 \\\bottomrule

\end{tabular}}
\caption{Zero-shot models performance on the IRFL "mixed" understanding task by figurative type. There are two columns for idioms, the first column represents the score for the "Figurative" images, and the second for the "Figurative Literal" images. In the idioms tasks the model received both the Idioms and their definitions as input. Numbers are the percentage of instances annotated correctly. Bold numbers indicate the best model performances.}
\label{tab:mixed-single-choice-results}
\end{table}
%\ron{Remove human performance from this tab and put it only on mix}
%\ron{Create very diverse task and start with it}
%\ron{Fine-grained analysis - distractors affect}
%\ron{to better understand the behavior of these models on the dataset, we examine other categories}






\textbf{Models prefer partially literal images over figurative ones.}
We analyze the models' choices on the ``mixed'' multimodal figurative language detection task and found that in all models, a partially literal distractor was selected in $92\%-100\%$ of the instances where the models failed across all figures of speech (idioms, metaphors, and similes). This shows that models prefer partially literal images over figurative ones. { We find the case of idioms to be particularly interesting. Models receive a relatively long prompt (idiom+definitions), and often choose an image that is a literal interpretation of only 1-2 words from the prompt.}

\begin{table}[tp]
\centering
\small
\begin{tabular}{@{}lccccccccc@{}}
\toprule
Categories                                                  & \multicolumn{6}{c}{Fig.}      & \multicolumn{2}{c}{\begin{tabular}[c]{@{}l@{}}Fig. \\ Lit.\end{tabular}}\\ \midrule
Candidates                                                  &     \multicolumn{2}{c}{2}            &    \multicolumn{2}{c}{4}       &    \multicolumn{2}{c}{6}   & \multicolumn{2}{c}{4} \\ \midrule
Random                                                      &     \multicolumn{2}{c}{50}             &    \multicolumn{2}{c}{25}    &    \multicolumn{2}{c}{16.6}   & \multicolumn{2}{c}{25} \\ \midrule
\begin{tabular}[c]{@{}l@{}}CLIP- \\ VIT-L/14\end{tabular}   &  64          & \textbf{87}   &  \textbf{46} & \textbf{71} & \textbf{33} & 53            & 76          & 86 \\
\begin{tabular}[c]{@{}l@{}}CLIP- \\ VIT-B/32\end{tabular}   &  61          & 84            &  38          & 67          & 30          & 53            & 65          & 82\\
CLIP-RN50                                                   &  56          & 75            &  30          & 60          & 24          & 46            & \textbf{78} & 86 \\
\begin{tabular}[c]{@{}l@{}}CLIP- \\ RN50x64\end{tabular}    &  \textbf{67} & 79            &  38          & 67          & 27          & 51            & 69          & 85\\ 
BLIP                                                        &  57          & 79            &  30          & 62          & 19          & 51            & 72           & 88\\
BLIP2                                                       &  58          & 75            &  25          & 58          & 14          & 40            & 75           & 82\\
\begin{tabular}[c]{@{}l@{}}COCA \\ ViT-L-14\end{tabular}    &  62          & 82            &  39          & \textbf{71} & 32          & \textbf{60}   & 68           & \textbf{91}\\ \bottomrule
% LiT                                                         &  51          & 48            &  22          & 25          & 12          & 15            & 21          & 24\\ \bottomrule

\end{tabular}
\caption{Zero-shot models performance on different configurations of the multimodal figurative language detection task, idioms with random candidates. Numbers are \% instances annotated correctly.  The left column of each pair shows the score for the idiom alone as input, and the right column shows the score for the idiom and definitions. Models %achieve higher scores on idioms with random candidates but still 
fail to reach human performance.}
\label{tab:random-idiom-single-choice-results}
\end{table}
% \ron{Remove human performance from this tab and put it only on mix}
% \ron{Create a very diverse task and start with it}
% \ron{Fine-grained analysis - distractors affect}
% \ron{to better understand the behavior of these models on the dataset, we examine other categories}






\textbf{Models partially understand the figurative connection between idioms and images.}
To examine whether models can comprehend a figurative connection between an image and an idiom, we experiment with random candidates and several configurations of the multimodal figurative language detection task (Table~\ref{tab:random-idiom-single-choice-results}). When provided with an idiom and its definitions as input, the accuracy on the Figurative category ranges between $75\%-87\%$ with $2$ candidates and $58\%-71\%$ with $4$ candidates. These results are above chance level but still below human performance on the ``mixed'' task. 

When given the idiom alone as input, the accuracy ranges between $56\%-67\%$ with $2$ candidates and $25\%-46\%$ with $4$ candidates. These results suggest that models partially understand the figurative connection between idioms and images. We see a significant performance drop with all models when the number of candidates increases.

In the Figurative+Literal category, with only the idiom as input, models registered an accuracy of $65\%-78\%$ with $4$ candidates. This performance significantly exceeds the accuracy recorded on the Figurative category with $2$ and $4$ candidates. The performance increase can be explained by the fact that Figurative+Literal images have both a literal and figurative connection to the phrase.

\begin{table}[h!]
\centering
\scalebox{0.8}{
\begin{tabular}{@{}lcccccc@{}}
\toprule
Categories   & \multicolumn{2}{c}{Metaphors} & \multicolumn{2}{c}{Similes}\\ \midrule
Candidates  &      2             &        4            &            2        &        4 \\ \midrule
CLIP-VIT-L/14   &  87\%          &      72\%           &  \textbf{99\%}      & \textbf{97\%} \\
CLIP-VIT-B/32   &  86\%          &      73\%           &  \textbf{99\%}      & \textbf{97\%} \\
CLIP-RN50       &  83\%          &      66\%           &  \textbf{99\%}      & \textbf{97\%} \\
CLIP-RN50x64/14 &  \textbf{88\%} & \textbf{76\%}       &  98\%               & 96 \\ 
LiT             &  47\%          & 27\%                &      49\%           &  24\% \\ \midrule
ViLT            &  72\%          & 53\%                &      96\%           &  91\%\\ \bottomrule

\end{tabular}}
\caption{Zero-shot models performance on the metaphors and similes understanding task with random candidates. Numbers are the percentage of instances annotated correctly.}
\label{tab:random-similes-metaphors-single-choice-results}
\end{table}










\textbf{Models understand metaphors but fail to reach human performance.}   
Table~\ref{tab:random-similes-metaphors-single-choice-results} shows the models' performance on metaphors with random candidates. The accuracy of all models on the Figurative category with $2$ candidates is $72\%-88\%$, and $53\%-76\%$ with $4$ candidates. We see a significant performance drop with all models when the number of candidates increases. The results suggest that models can understand metaphors but fail to reach human performance.

\textbf{Models understand similes well.}
Table~\ref{tab:random-similes-metaphors-single-choice-results} shows the models' performance on the similes with random candidates. The accuracy of all models on the Figurative category with $2$ candidates is $95\%-99\%$, and $88\%-98\%$ with $4$ candidates. Models' performance is competitive with that of humans, and the models maintain their performance when increasing the number of candidates. 
In contrast to the multimodal figurative language detection task with random images, the ``mixed'' task shows a performance gap between closed and open similes due to open similes concealing the compared property, making it harder for the model to choose the figurative image. Analyzing the ``mixed'' task results on closed similes, we found that figurative images scored higher than source concept images in $52\%-74\%$ of cases across all models. 

Additionally, source concept images scored higher than target concept distractor images in $51\%-70\%$ of cases. This pattern suggests a model prioritization for simile images: firstly, target concept images with the compared property, then source concept images, and finally, target concept images lacking the compared property.

%OLD -  As we analyzed the ``mixed'' X results on closed similes in more depth, we found that across all models excluding LiT, $55\%-61\%$ of the figurative images received a higher matching score than the source concept images. In addition, $50\%-66\%$ of the source concept images received a higher matching score than the target concept distractor image. These suggest that models prioritize simile images in the following order: 1) images of the target concept with the compared property, 2) images of the source concept, 3) images of the target concept without the compared property.
\begin{table}[tp]
\begin{center}

\scalebox{0.8}{
\begin{tabular}{@{}lcccc@{}}
\toprule
 Categories & Figurative & Figurative Literal &\\  \midrule
 Zero-Shot & $16\%$ & $41\%$ &\\ 
 Supervised & $58\%\pm4.2$ & $49\%\pm2.6$  & \\ \bottomrule 
\end{tabular}} 


% \begin{tabularx}{230pt}{| c c c X|}
%  \hline
%  Categories & Figurative & Figurative Literal &\\ 
%  \hline
%  Zero-Shot & $13\%$ & $41\%$ &\\ 
%  \hline
%  Supervised & $51.5\%\pm3.9$ & $51\%\pm3.1$  &\\
%  \hline
% \end{tabularx}}

\end{center}
\caption{Supervised models performance. Results are the mean and standard deviation of the accuracy
of five experiments.}
\label{tab:understanding_task_supervision}
\end{table}


\textbf{Fine-tuning improves figurative understanding and reduces literal preference.} The supervised model results are presented in Table~\ref{tab:understanding_task_supervision}. Previously we did not display the models' performance on the ``mixed'' task when taking the idiom alone as input due to their poor performance ($5\%-7\%$ accuracy). However, when training on idioms alone,  the supervised model scored a mean accuracy of $46.2\%$, $9\times$ the zero-shot score of $5\%$. This large performance increase might suggest that VL-PTMs representation of an idiom encodes its definitions. 

Training and testing with the idiom and its definitions as input resulted in a mean accuracy of $58\%$ compared to $16\%$ in the Zero-shot configuration. After analyzing the supervised model results, we found that its literal preference has improved significantly. In $41\%\pm4.3$ of the instances where the model failed, a partially literal distractor was selected compared to $96\%$ in the zero-shot configuration. Along with this improvement in literal preference, Figurative+Literal category accuracy raised from $41\%$ in zero-shot to $49\%$. These results show that models can improve their preference for partially literal images and recognize idiomatic figurative connections better via training. Moreover, the results suggest that the data is a useful training signal for our task.

We have discovered that VL-PTMs tend to prefer partially literal images. In the next section, we design a task to tackle this issue. 



\subsection{Multimodal Figurative Language Retrieval Task}
\label{sec:ranking Task Analysis}
% The {\bf Multimodal Figurative Retrieval Task} examines VL-PTMs' preference for figurative images. The task is to rank the figurative and partially literal images using the model-matching score and calculate the precision at $k$, where $k$ is the number of figurative images. Figure~\ref{fig:second-task-idiom-figurative-vs-caption} shows an example of the task for the idiom ``ruffle someone's feathers''.
\begin{figure}[t!]
\includegraphics[width=0.45\textwidth,height=\textheight,keepaspectratio]{figures/05-ranking.jpg}
\caption{Example of multimodal figurative language retrieval task for the idiom  ``ruffle someone's feathers'' (to unease, cause discomfort to someone). 
%The first row demonstrates one possible expected retrieval order, where Figurative (F) are ranked before Partial Literal (P). The second row shows CLIP-VIT-L/14 retrieval order. Green indicates correct position, and red indicates incorrect position. The retrieval order 
The task is to rank the figurative images above the partial literal ones, based on the images' matching score with the idiom. %All models received a $0$ precision score in this example.
}
 % Green indicates correct retrieval order, and red indicates incorrect retrieval order. 



\label{fig:second-task-idiom-figurative-vs-caption}
\end{figure}
The {\bf Multimodal Figurative Retrieval Task} examines VL-PTMs' preference for figurative images. 
Given a set of figurative and partially literal images, the task is to rank the images using the model-matching score such that the figurative images are ranked higher, and calculate the precision at $k$, where $k$ is the number of figurative images in the input. 

Figure~\ref{fig:second-task-idiom-figurative-vs-caption} shows an example of the task for the idiom ``ruffle someone's feathers''.
We wish to have images of people causing discomfort ranked higher than pictures of birds and feathers. 
%Our objective is to introduce a task to tackle VL-PTMs' preference for partially literal images (As seen in \S\ref{sec:results_and_model_analysis}). 
%The primary component of this task involves using the model-matching score or certainty score to rank both figurative and partially literal images. 
%For instance, when presented with the idiom ``let the cat out of the bag'' (meaning ``to disclose a secret; to let a secret be known''), an image of someone revealing a secret should receive a higher matching score than just an image of a bag. 
This task provides an opportunity for a deeper probe into how the model comprehends figurative language in terms of its preferences.

In this task, we use the same baselines and training methods mentioned in the previous task. We train the supervised model on $3802$ images, making sure the training data does not contain any of the images or idioms that appear in the task.

%OLD - To tackle vision and language models' strong preference toward partially literal images over figurative images, we introduce the preference task. The preference task is to rank the Figurative images higher than partially literal distractors based on the model matching score. First, we rank the figurative phrase images by their matching score from higher to lower, then we define two classes, $|F|$ which consists of the Figurative images, and $|P|$ which consists of the partially literal images. The model then predicts the first $|F|$ images as Figurative and the last $|P|$ images as partially literal images, the $F_1$ score of the model predictions is the preference task score. For instance, when presented with the idiom ``let the cat out of the bag'' (meaning ``to disclose a secret; to let a secret be known''), an image of someone revealing a secret should receive a higher matching score than just an image of a bag. This element of the task allows a further examination of the model understanding of figurative language from a viewpoint of its preference. 

\subsubsection{Results and Model Analysis}
\label{sec:ranking_task_results_and_model_analysis}
Zero-shot results are presented in Table \ref{tab:ranking-task}. We evaluate all figurative phrases that have both Figurative and Partial Literal images.
\begin{table}[h!]
\centering
\scalebox{0.65}{
\begin{tabular}{@{}lcccc@{}}
\toprule
Ranking  &     \multicolumn{2}{c}{Idioms}                   &  Metaphors         &   Similes \\ \midrule
                & Figurative Literal  & Figurative &                                        & \\ \midrule
CLIP-VIT-L/14   &       57            &    37                 & 26               & \textbf{44} \\
CLIP-VIT-B/32   &       54            &    36                 & 22               & 38 \\
CLIP-RN50       &       54            &    37                 & 25               & 38 \\
CLIP-RN50x64/14 &  \textbf{61}        &    39                 & \textbf{29}      & 43 \\
LiT             &  54                 & \textbf{56}           & 25               & 25 \\ \midrule
ViLT            &        -            & -                     & 23               & 34 \\ \midrule
\# of phrases   & 94                 & 149                    & 35               & 142  \\ \bottomrule
\end{tabular}}
\caption{The preference task performance, the scoring metric is $F_1$. The Idiom category is double-columned. The left column shows the score for Figurative Literal images, and the right column shows the score for Figurative images.}
\label{tab:ranking-task}
\end{table}

Models' scores on the preference task are low (<$61\%$). We expect models with proper figurative preference to achieve better results. Models' success in the Figurative+Literal category can be attributed to the literal connections of the Figurative+Literal images.

The supervised model achieved a score of $68\pm3.8$ in the Figurative category, almost double the zero-shot score of CLIP-ViT-B/32 ($36$). Additionally, the score in the Figurative+Literal category was improved by $10\pm2.25$ points. These results align well with the observation that the multimodal figurative language detection task supervised model, which was trained using the same method on a different training set, also showed substantially moderate literal preference. Table~\ref{tab:preference_task_supervision} shows the fine-tuned model results. 
\begin{table}[h!]
\begin{center}
\scalebox{0.8}{

\begin{tabular}{@{}lcccc@{}}
\toprule
 Categories & Figurative & Figurative Literal &\\  \midrule
 Zero-Shot & $36$ & $54$ &\\ 
 Supervised & $68\pm3.8$ & $64\pm2.25$  &\\ \bottomrule
\end{tabular}} 


% \begin{tabularx}{230pt}{| c c c X|}
%  \hline
%  Categories & Figurative & Figurative Literal& \\ 
%  \hline
%  Zero-Shot & $36$ & $54$ &\\ 
%  \hline
%  Supervised & $73\pm1.8$ & $70\pm0.9$  &\\
%  \hline
% \end{tabularx}}

\end{center}
\caption{Supervised models performance. Results are the mean and standard deviation of the $F_1$ score
of five experiments.}
\label{tab:preference_task_supervision}
\end{table}


\subsection{Generative Models Analysis}
\label{sec:genearive_models_analysis}
In our work so far, we focused on finding existing images matching a figurative expression. We now explore the question of whether generative models can \emph{generate} figurative images. We sampled $15$ idioms from the IRFL dataset and experimented with the idioms and their definitions as input to Dall-E \cite{ramesh2021zero} and Stable Diffusion \cite{rombach2022high}. We annotated $345$ generated images and found that generative models failed to generate figurative images for given idioms, generating literal images instead. When provided with the definitions as input, the models had some more success in creating figurative images. Statistics on the generated images can be seen in Table \ref{tab:generative-models-statistics}. We also included the percentage of images from each category found by our pipeline.

\begin{table}[h]
\centering
\small
\begin{tabular}{@{}lccccccc@{}}
\toprule
Categories          & \multicolumn{2}{c}{Dall-E} & \multicolumn{2}{c}{\begin{tabular}[c]{@{}l@{}}Stable \\ Diffusion\end{tabular}} & \multicolumn{2}{c}{IRFL} & \\ \midrule
Figurative          & 0  & 42.5              & 0 & 11                                                                    & 4 & 46                  \\
Figurative+Literal  & 0 & 10                 & 5 & 1                                                                     & 20 & 6                     \\
Literal             & 31 & 0                 & 17 & 0                                                                    & 35 & 0                 \\
Partial Literal     & 48 & 2                 & 42 & 2.5                                                                  & 23 & 1.5              \\
None                & 19  & 44               & 27 & 85                                                                   & 4 & 43             \\ \midrule
Number              & 48 & 120               & 59 & 118                                                                  & 69 & 126      \\ \bottomrule 
\end{tabular}
\caption{The table is double-columned, the first column describes the percentage of images generated by idioms, and the second column describes the percentage of images generated by the idioms' definitions. The results show that our pipeline extracted more Figurative, Figurative+Literal, and Literal images and fewer None images than the generative models.}
\label{tab:generative-models-statistics}
\end{table}


The results show that our pipeline extracted more Figurative, Figurative+Literal, and Literal images and fewer None images than the generative models managed to generate. %Future work might focus on the quality of generative models' figurative images and the emotions they evoke. 
% \ron{@Dafna, should we delete table 9?}
%\input{tables/generative_models_baseline_vs_irfl.tex}
