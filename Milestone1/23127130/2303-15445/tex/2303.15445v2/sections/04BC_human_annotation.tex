\begin{table}[t!]
%\centering
\small
% \resizebox{\columnwidth}{!}{
\begin{tabular}{@{}lcccccc@{}}
\toprule
   & Fig. & \begin{tabular}[x]{@{}c@{}}Fig.\\ Lit.\end{tabular} & Lit. & \begin{tabular}[x]{@{}c@{}}Part.\\ Lit.\end{tabular}  & None & \\ \midrule
\#             & 1970  &  751   &   434  &  487  & 2638 & 6697 \\ \midrule
3-maj & 100\% & 100\%  & 100\%  & 100\% & 100\% &  94\% \\
4-maj & 75.5\%  & 63\%   & 68\%   & 63\%  & 80\% &  70\% \\
5-maj & 45\%  & 33\%   & 35\%   & 38\%  & 53\% &  43\% \\ \midrule
Mean & 3.1   & 1.2    & 0.7    & 0.8   & 4    & - \\
Median  & 2     & 0      & 0      & 0     & 4    & - \\ \bottomrule
\end{tabular}
% }
\caption{IRFL statistics on 628 idioms. The majority of the images are related to the figurative phrase, most images are Figurative. (k-maj means k-majority)}
\label{tab:dataset-statistics}
\end{table}



\remove{
\begin{table}[t!]
%\centering
\small
% \resizebox{\columnwidth}{!}{
\begin{tabular}{@{}lcccccc@{}}
\toprule
Categories   & Fig. & \begin{tabular}[x]{@{}c@{}}Fig.\\ Literal\end{tabular} & Literal & \begin{tabular}[x]{@{}c@{}}Partial\\ Literal\end{tabular}  & None & Total\\ \midrule
Number             & 1970  &  751   &   434  &  487  & 2638 & 6697 \\ \midrule
3 majority & 100\% & 100\%  & 100\%  & 100\% & 100\% &  94\% \\
4 majority & 75.5\%  & 63\%   & 68\%   & 63\%  & 80\% &  70\% \\
5 majority & 45\%  & 33\%   & 35\%   & 38\%  & 53\% &  43\% \\ \midrule
Average & 3.1   & 1.2    & 0.7    & 0.8   & 4    & - \\
Median  & 2     & 0      & 0      & 0     & 4    & - \\ \bottomrule
\end{tabular}
% }
\caption{IRFL statistics on 628 idioms. The majority of the images have some relation to the figurative phrase. Most of the relations are Figurative.}
\label{tab:dataset-statistics}
\end{table}

}
We hired Amazon Mechanical Turk (AMT) workers to annotate the relation between each idiom and its candidate images using the user interface seen in Appendix~\ref{sec:annotation_ui} (Figure~\ref{fig:image-task-ui}). Five workers annotated each image in batches of five images per sample. They received a payment of \$$0.15$ per sample, which resulted in an average hourly wage of \$$15$. We created a qualification test\footnote{https://irfl-dataset.github.io/mturk/image/qualification} to select quality annotators and provided them with an interactive training platform\footnote{https://irfl-dataset.github.io/mturk/image/train} to understand the task and the different categories better. 

We split the annotation process into batches with an average size of $60$ idioms per batch. After each batch, we provided each worker with a personal profile page (Appendix~\ref{sec:annotation_ui}, Figure~\ref{fig:profile-page-ui}) to view their statistics and some examples where their choice was different from the majority of workers. 
%We also set up a leaderboard (Figure~\ref{fig:image-task-leaderboard}, Appendix~\ref{sec:annotation_ui})  that was updated after each batch to improve their competitiveness. 

Full annotation results and statistics are presented in Table \ref{tab:dataset-statistics}. 
%The nature of this task is very subjective. 
%We further discuss this in (Appendix~\ref{sec:annotation_task_discussion}).
% \dnote{why do you have discussion in the appendix? this doesn't make much sense}
Despite the subjective nature of the task, in $94\%$ of the instances, there was a majority of $3$ workers or more out of $5$ compared to a random chance of $29\%$. %This shows that different people can see the same connection most of the time. 







