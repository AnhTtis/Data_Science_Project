Figures of speech such as metaphors, similes, and idioms are integral parts of human communication. They are ubiquitous in many forms of discourse, allowing people to convey complex, abstract ideas, compare situations, provke thought and evoke emotion \citep{LakoffandJohnson1980, Hoffman1987WhatCR, why-do-people-use-figurative-language, SusanAndMallie1994}. Figurative language research often focuses on text alone; however, figurative language is often conveyed through \emph{multiple} modalities (usually text and images) -- for example, in areas such as social media, advertising, and news.  

Figure \ref{fig:figurative-language-social-media} shows two social media posts that require multimodal figurative understanding.  In the left image, the caption reads ``Jumped off the sinking ship just in time'', and the image shows a soccer player who has just left his struggling club. The right image, captioned ``A performing clown'', shows a famous YouTuber losing a boxing match to a professional boxer.
%Figure~\ref{fig:multimodal} presents an image of a luxury car posted on social media with the simile ``As fast as a cheetah'' and an advertisement for Toyota Corolla with the idiom ``get your hands on''. \\
%\yonatan{so far you have a table and an image for explaining existing concepts. I wonder whether it should be in the Appendix rather than taking important space at the start of the paper, because it's not something new you present in the work.}\ron{@Dafna}
% Multimodal metaphors are metaphors conveyed through multiple modes ``whose target and source are each represented exclusively or predominantly in different modes'' \citep{Forceville2016}. Multimodal information from different modes, such as language and vision, can contribute to metaphorical conceptualization and comprehension \citep{PerspectivesMultimodalityBook, multimodalMetaphorBook}. The comprehension of multimodal metaphors takes cognitive efforts like decoding metaphorical messages and understanding the relationships between domains, analyzing the emotion metaphors convey, and interpreting authorial intent [5–7] \citep{conceptual-integration-and-metaphor, YANG2013312, FAUCONNIER1998133}. Since similes are often referred to as metaphors in the cognitive linguistic and rhetorical literature \citep{culler1981pursuit, lou2021multimodal}, the topic of multimodal similes has not been thoroughly investigated. Recently, Adrian Lou \citep{lou2021multimodal} proposed a novel and robust framework for the analysis of both verbal and multimodal similes. Multimodal idioms have yet to be studied cognitively, in part due to a lack of multimodal datasets.  \citep{LakoffandJohnson1980, Hoffman1987WhatCR}. 
% \begin{figure}[h]
% \includegraphics[width=0.48\textwidth,height=12cm,keepaspectratio]{figures/Figure 1 - multimodal example.JPG}
% \caption{An advertisement for Toyota Corolla from the 60s with the idiom "get your hands on" and an image posted on a social media platform with the caption ``As fast as a cheetah''.}
% \label{fig:multimodal}
% \end{figure}
% There has been an increase in the use of text and vision modes over the past few years due to the growing usage of social media and mass media. The increased use of multimodality has created new challenges, which sometimes expand existing challenges from monomodality to multimodality.

%\dnote{do not ever use newline noindent unless you have a very good reason}\rnote{Done}
Due to its integral part in human communication, detecting and understanding multimodal figurative language could play an important role in various multimodal challenges, such as hate-speech detection in memes \citep{detecting-hate-speech-in-multi-modal-memes}, fact checking \citep{multimodal-fact-checking}, sentiment analysis \citep{soleymani-20173}, humor recognition \citep{reyes-20121, shahaf-2015, detecting-sarcasm-in-multimodal-social-platforms}, and tasks focusing on the mental state of social media users \citep{yadav-etal-2020-identifying, multimodal-time-aware-attention-networks}. 

  % In the task of sentiment analysis, the sentiment of the right post should be classified as ``negative'', but this cannot be achieved solely by text or image. Vision and Language Pre-Trained Models’ (VL-PTMs) understanding of figurative language combined with vision has not been thoroughly examined, if at all, partly due to the absence of large-scale datasets with ground truth labels of multimodal smilies, idioms, metaphors, etc.\\ % In light of this rapidly growing trend, figurative language studies must be expanded from monomodality to multimodality.

%\begin{table}[ht]
\begin{center}
\scalebox{0.6}{
\begin{tabularx}{380pt}{| c X X |}
 \hline
 Figure & Example & Explanation \\ [0.5ex] 
 \hline\hline
 Metaphor & "You're a peach!" & The person being addressed is being equated with a peach, with the suggestion that the person is pleasing or delightful. The target concept is "person" and the source concept is "peach".\\ 
 \hline
 Open Simile & "Her heart is like stone" &  Inflexible and unfriendly or unkind disposition. The shared properties of "her heart" and "stone" are not explicitly revealed. The target concept is "her heart" and the source concept is "stone".  \\
 \hline
 Closed Simile & "The old man walk as slow as a snail" &  The old man's movement is compared to that of a snail, the shared property ("slow") is explicitly revealed. The target concept is "old man" and the source concept is "snail".\\
 \hline
 Idiom & "We're on the same page" &  Agreeing about something (such as how things should be done).  \\
 \hline
\end{tabularx}}

\end{center}
\caption{\ron{@Dafna Should we delete this/move it to appendix? If not, I think merging it with Figure 2 could be a good idea}Examples of simile, metaphor and idiom with a corresponding explanation.}
\label{table:figures-of-speech}
\end{table}%
\begin{figure}[t!]
%\begin{center}
\includegraphics[width=0.482\textwidth,height=12cm,keepaspectratio]{figures/figure_-_figurative_language_in_social_media_-_bigger_bigger_bigger.jpg}
%\end{center}
\caption{Two social media posts that require multimodal figurative understanding to comprehend. The left photo depicts a soccer player who left his struggling club. The right photo shows an amateur boxer who lost a boxing match to a professional boxer. 
}
\label{fig:figurative-language-social-media}
\end{figure}

Vision and Language Pre-Trained Models’ (VL-PTMs) understanding of figurative language combined with images has not been thoroughly explored,  partly due to the lack of large-scale datasets.
In this work, we introduce the {\bf IRFL dataset} (Image Recognition of Figurative Language) of idioms, metaphors, and similes with matching images -- both figurative and literal. We developed a pipeline to collect candidate figurative and literal images and annotated them via crowdsourcing. 
%\dnote{you forgot ``language'' in the name of the tasks}\rnote{Done}

Next, we used the dataset to design two novel tasks, {\bf multimodal figurative language detection} and {\bf multimodal figurative language retrieval}, to assess the figurative-language capabilities of VL-PTMs. 
The detection task is to choose the image that best visualizes the figurative phrase. See Figure~\ref{fig:first-task-idiom-figurative} for an example for an idiom, a metaphor, and a simile, with the correct answers highlighted. 

As we noticed that VL-PTMs tend to select images containing \emph{objects} that appear in the figurative phrase, we designed a second task targeting this behavior. In the multimodal figurative language retrieval task, the goal is to rank figurative images higher than images with objects from the phrase. 

%-- cho rank figurative and partially literal images and calculate the precision at $k$, where $k$ is the number of figurative images. 
%\dnote{your description of the rand partially literal images and calculate the precision at $k$, where $k$ is the number of figurative imagesking task is completely confusing. it doesn't sound like the main thing about it is ranking -- it sounds like yet another classification task over a different set of distractors. why is the ranking part important?}\rnote{Done} 

We experiment with several VL-PTMs and find that the best model ($22\%$ accuracy) fails to reach human performance ($97\%$). We also find that generative models have difficulties generating figurative images for idiomatic phrases. 
%

  
We hope our dataset and benchmarks will drive the development of multimodal models that can better understand figurative language, closing the gap with human performance. More broadly, metaphorical reasoning is strongly tied to problem-solving and creativity; we believe such models, able to see analogies between situations that share very little on the surface, could greatly advance the field.


%\dnote{more than what? is it good? this punchline is quite weak}\rnote{Done}
% OLD  The X evaluates VL-PTMs' ability to understand the relation between an image and a figurative phrase. 
% Prefrence task The preference task examines VL-PTMs' preference for figurative images. In this task, the model needs to

%find that they are unable to generate figurative images given idiomatic phrases.

%\ron{Should the result be in intro?}%
% \yonatan{change quotation marks to ``THIS''} - Did not understand this
% \yonatan{figures in seperate files?} - Didn't understand this
% The figurative X evaluates VL-PTMs' ability to understand the relation between an image and a figurative phrase. The task is to choose the image that best visualizes the figurative phrase out of X candidates.  Figure~\ref{fig:first-task-idiom-figurative} shows an example of the task for idiom, metaphor, and simile.
% The figurative X evaluates VL-PTMs' ability to understand the relation between an image and a figurative phrase. The task is to choose the image that best visualizes the figurative phrase out of X candidates. Figure~\ref{fig:first-task-idiom-figurative} shows an example for the idiom ``Let the cat out of the bag'' - to disclose a secret, and examples with difficult distractors for the metaphor "Sea of bees" and the simile "The child is as proud as a peacock". Using partially literal images as distractors, the best model ($22\%$) fails to match human performance ($92\%$). This task goes beyond object detection and scene understanding, it requires a profound and rich language and cultural knowledge (idioms) in addition to commonsense, abstraction, general knowledge and the ability to decode the domains of figurative phrases and understand their relationships (metaphors and similes).
% \newline \noindent The preference task examines VL-PTMs' preference of figurative images over partially literal images. In this task, the model needs to rank figurative phrase images of different categories correctly. We suggest that Vision and language models should prioritize figurative images over partial literal images. Meaning that an image of someone disclosing a secret should receive a higher matching score for the idiom  ``let the cat out of the bag'' than an image of a bag. Figure~\ref{fig:second-task-idiom-figurative-vs-caption} shows the expected order versus the actual order of the idiom ``ruffle someone's feathers'' images based on the model scores. Assuming that the model understands the idiom and sees a figurative connection to an image, the task purpose is to measure how well the model comprehends it. This is done by comparing the figurative images' matching score to other images with a weaker relationship (partially literal). We find that the best model receives a $F_1$ score of $35-50$ depending on the figure of speech.
% \noindent Finally, we experiment with generative models such as Dall-E and Stable Diffusion to examine their ability to generate figurative images for idioms. We provide these models with idioms and their definitions as prompts and compare the results to our automatic pipeline. Our findings show that Stable Diffusion fails to generate figurative images even when given the definitions as input, while Dall-E succeeds in generating figurative images for the idioms' definitions.





%Figure~\ref{fig:first-task-idiom-figurative} shows examples of idiom, metaphor, and simile along an image conveying the figurative message (orange). \\
