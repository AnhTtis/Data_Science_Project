% \yonatan{is it correct to say that it's an automatic generation followed by human ratings? If so I'll start by that. Also I suggest to say something similar to what we said in VASR: this is a process that contains several heuristics and implementation decisions. We evaluate the end2end dataset generation later on, and the fact that human achieve high agreement helps to verify the correctness of the end2end process. }
%\yonatan{is it correct to say that it's an automatic generation followed by human ratings? If so I'll start by that} \ron{We didn't started like this in VASR, why is it important here?}
\begin{figure}[t!]
\includegraphics[width=0.44\textwidth ,height=\textheight,keepaspectratio]{figures/up_a_tree_bigger.JPG}
\includegraphics[width=0.44\textwidth ,height=\textheight,keepaspectratio]{figures/blanket_of_snow_bigger.jpg}
\includegraphics[width=0.44\textwidth ,height=\textheight,keepaspectratio]{figures/car_cheetah_bigger.JPG}
\caption{Examples of the multimodal figurative language detection task for idiom, metaphor, and simile. The input is a figurative phrase and four candidate images (for idiom, we also show the definition). The correct answer is marked with an orange square.}
\label{fig:first-task-idiom-figurative}
\end{figure}

\begin{table*}[tp!]
\begin{center}
\small
\begin{tabular}{ m{8em} m{8em} m{8em} m{8em} m{8em}}
\toprule
\multicolumn{5}{c}{\textbf{Idiom:} Touch wood} \\
\multicolumn{5}{c}{\begin{tabular}[x]{@{}c@{}} \textbf{Definitions:} 1) Hopefully 2) Said while touching something wooden, \\ to avert  superstitious bad luck from what has just been said \end{tabular}} \\ 

\midrule
\includegraphics[width=0.18\textwidth,height=5.5em]{figures/touch_wood_caption.jpeg} & 
\includegraphics[width=0.18\textwidth ,height=5.5em]{figures/touch_wood_figurative.jpeg} &
\includegraphics[width=0.18\textwidth ,height=5.5em]{figures/touch_wood_partial_objects.jpeg} & 
\includegraphics[width=0.18\textwidth ,height=5.5em]{figures/touch_wood_figurative_literal.jpg} &  \includegraphics[width=0.18\textwidth ,height=5.5em]{figures/touch_wood_none.jpeg} \\ \midrule

 Literal & Figurative & Partial Literal & Figurative+Literal & None  \\ \midrule

 The image illustrates the phrase literally & 
 The image conveys one or more \emph{definitions} of the idiom &
 Some objects/ entities of the phrase are visualized  (here, wood) & 
 Fits the ``Figurative'' definition and also ``Literal''/``Partial Literal''  &
 The image does not fit any of the other categories  \\ \bottomrule
\end{tabular}


% \small
% \begin{tabular}{ c c c c c c}
% \toprule
% \adjustbox{valign=c}{\includegraphics[width=0.18\textwidth ,height=\textheight,keepaspectratio]{figures/touch_wood_figurative_literal.jpg}} &
% \adjustbox{valign=c}{\includegraphics[width=0.18\textwidth ,height=\textheight,keepaspectratio]{figures/touch_wood_figurative_literal.jpg}} &
% \adjustbox{valign=c}{\includegraphics[width=0.18\textwidth ,height=\textheight,keepaspectratio]{figures/touch_wood_figurative_literal.jpg}} &
% \adjustbox{valign=c}{\includegraphics[width=0.18\textwidth ,height=\textheight,keepaspectratio]{figures/touch_wood_figurative_literal.jpg}} &
% \adjustbox{valign=c}{\includegraphics[width=0.18\textwidth ,height=\textheight,keepaspectratio]{figures/touch_wood_figurative_literal.jpg}} & \midrule

%  Figurative+Literal & The image conveys one or more definitions of the idiom to some extent,
% and it literally illustrates the phrase or visualizes the phrase objects/entities & \adjustbox{valign=c}{\includegraphics[width=0.18\textwidth ,height=\textheight,keepaspectratio]{figures/touch_wood_figurative_literal.jpg}} &\\  \midrule
%  Figurative & The image conveys one or more definitions of the idiom to some extent & \adjustbox{valign=c}{\includegraphics[width=0.18\textwidth ,height=\textheight,keepaspectratio]{figures/touch_wood_figurative_literal.jpg}}&\\ \midrule
%  Caption &  The image illustrates the phrase literally & \adjustbox{valign=c}{\includegraphics[width=0.18\textwidth ,height=\textheight,keepaspectratio]{figures/touch_wood_figurative_literal.jpg}} &\\ \midrule
%  Partial Objects &  The objects/entities of the phrase are visualized in the image & \adjustbox{valign=c}{\includegraphics[width=0.18\textwidth ,height=\textheight,keepaspectratio]{figures/touch_wood_figurative_literal.jpg}} &\\ \midrule
%  None &  The image does not fit any of the categories & \adjustbox{valign=c}{\includegraphics[width=0.18\textwidth ,height=\textheight,keepaspectratio]{figures/touch_wood_figurative_literal.jpg}} &\\  \bottomrule 
% \end{tabular}


% \small
% \begin{tabular}{ m{3.5em} m{5.3cm} c c}
% \toprule
%  Figurative+Literal & The image conveys one or more definitions of the idiom to some extent,
% and it literally illustrates the phrase or visualizes the phrase objects/entities & \adjustbox{valign=c}{\includegraphics[width=0.18\textwidth ,height=\textheight,keepaspectratio]{figures/touch_wood_figurative_literal.jpg}} &\\  \midrule
%  Figurative & The image conveys one or more definitions of the idiom to some extent & &\\ \midrule
%  X &  The image illustrates the phrase literally &\\ \midrule
%  x &  The objects/entities of the phrase are visualized in the image & &\\ \midrule
%  None &  The image does not fit any of the categories & &\\  \bottomrule 
% \end{tabular}


\end{center}
\caption{The table shows the different categories of the relation between an image and a phrase, along with matching images for the idiom "Touch wood". Workers were guided to choose the most suitable relation category by a scheme tree that illustrates the correct thinking process (Figure~\ref{fig:image-task-tree}, Appendix~\ref{sec:annotation_ui}).}
\label{tab:relation-categories}
\end{table*}

Our goal is to create a dataset with idioms, metaphors, and similes paired with figurative and literal images. This dataset can then serve as a benchmark to evaluate Vision and Language models on multimodal figurative language. 

\xhdr{Labels} Initially, we intended to have our annotators label images ``literal'' or ``figurative''. However, after initial experimentation with the data generated by our pipeline, we realized the necessity of a more nuanced classification system. Hence, we introduced two additional categories.

The first new category, ``Figurative+Literal,'' encompasses images that express the figurative meaning of an expression while also maintaining some aspects of the literal interpretation. The second, ``Partial Literal,'' includes images that visualize some (literal) elements or objects from the expression. 

 Table~\ref{tab:relation-categories} illustrates our categories for the expression ``Touch wood''. For example, an image of someone literally touching wood while crossing his fingers for luck is classified as Figurative+Literal.   
This distinction also allows us to later perform a richer analysis of model performance. 

%To create multimodal idioms, we gathered idiomatic expressions from the MAGPIE corpus \citep{haagsma-etal-2020-magpie}. We then utilized a semi-automatic pipeline we developed to find figurative and literal images (\S\ref{sec:idioms-collection}). For multimodal metaphors, we collected similes and metaphors from various online sources. Subsequently, we manually collected and annotated the corresponding figurative and literal images (\S\ref{sec:metaphors_and_similes}).


\subsection{Pipeline: Idioms}
\label{sec:idioms-collection}
We collected $628$ idioms from the MAGPIE corpus \citep{haagsma-etal-2020-magpie} of idiomatic expressions. The MAGPIE corpus contains $56,622$ crowdsourced potentially idiomatic expressions, covering $1,756$ unique idioms that appear in at least two of the following dictionaries: Wiktionary, Oxford Dictionary of English Idioms, and UsingEnglish.com. After collecting the idioms, we feed them into our pipeline. 

Our pipeline consists of four main steps, illustrated in Figure~\ref{fig:figurative-pipeline}. Given an idiom, we first get its definitions from online dictionaries and parse them into search queries (\S\ref{sec:enriching_similes_and_idioms}). Second, we search for candidate images using the search queries. Third, we filter the images and select the best $k$ literal and figurative candidates for annotation (\S\ref{sec:choosing_images}). Lastly, we annotate the different images via crowdworkers (\S\ref{sec:human_annotation}). 


\subsubsection{Searching for Images}
\label{sec:enriching_similes_and_idioms}
\begin{figure}[b!]
%\begin{center}
\includegraphics[width=0.48\textwidth,keepaspectratio]{figures/pipeline-figure.JPG}
%\end{center}
\caption{The flow of our idiom pipeline: getting definitions, looking for image candidates using the idiom and its definitions, filtering an selecting candidate images. In the human annotation stage, blue represents Literal, Green -- Figurative, and red -- None.}
\label{fig:figurative-pipeline}
\end{figure}
Our goal is to find literal and figurative images for each idiom from the MAGPIE dataset. Searching for an idiom using image search often results in literal images. To find figurative images, we need to understand the \emph{meaning} of the idiom; however, the MAGPIE dataset does not contain idiom definitions, so we crawl them from online dictionaries (Wiktionary definitions tagged with `figurative'' or ``idiomatic''\footnote{We also construct search queries from untagged definitions. Even though untagged definitions are rare (<3\%), they are typically idiomatic.}; if no such definitions exist, we try the Oxford Dictionary).

For example, in Figure \ref{fig:figurative-pipeline}, the idiom ``white hat'' nd is defined as ``A good person; a hero'' (tagged with ``idiomatic''), and also as ``a sailor'' and ``A well-meaning hacker'' (tagged with ``slang"). 

%To accomplish this, we use the idioms and their definitions as search queries for an image search engine.  For each idiom, we search Wiktionary for definitions, and if no definitions are found, we search the Oxford Dictionary. Wiktionary definitions are usually accompanied by tags that indicate the context in which they appear. For example, the idiom ``white hat'' and its definition ``A good person; a hero'' is tagged with "figurative" and "idiomatic" tags, and the definitions ``a sailor'' and ``A well-meaning hacker'' are tagged with a "slang" tag. Using this data, we filter idioms with no ``figurative'' or ``idiomatic'' tags and construct search queries from these definitions \footnote{We also construct search queries from untagged definitions. Even though untagged definitions are rare (<3\%), they are typically idiomatic.}. 
We split concatenated definitions (e.g., ``A good person; a hero'' is split into two definitions). %concatenated into one. For example, we split the definition ``A good person; a hero'' into two search queries ``A good person'' and ``A hero''. 
In some rare cases, a definition may be another idiom, and then we replace that idiom with its definitions.

We then searched Google images for the idioms and their definitions, taking up to $20$ images per search query. Images were searched with ``SafeSearch'' flag ``on'', and in ``United States'' region.
\subsubsection{Image Filtering}
\label{sec:choosing_images}
We noted that many of the retrieved images contained the search query in textual form. We used optical character recognition (OCR) tool EasyOCR to extract text from the images, and TextBlob to correct spelling errors the OCR made. We then  filtered images that contained objects or entities from the idiom  or its definitions in textual form (50\% of the images). Such images are problematic because they may cause the model to select an image solely based on its textual signal. Following this filter, 15\% of the resulting images contained mostly text. To tackle this problem, we used OCR (See Appendix~\ref{sec:documents_filter}) to remove images with more than a couple of words, as well as images with more than 30\% of their space containing text. 

For the remaining images, we calculated the matching score of each image with its phrase and search query using ViLT. Top-$k$ images with a high ``phrase-image'' score (that passed a threshold, see Appendix~\ref{sec:literal_threshold}) were tagged as potentially literal.  We chose the top $k$ images with the highest ``definition-image'' score as Figurative candidates. 

%AMT workers then annotated the relation between the figurative phrase and its Figurative and Literal candidate images using the user interface (UI) seen in Figure~\ref{fig:image-task-ui},  Appendix~\ref{sec:annotation_ui}.
% Old_v2
% To find figurative images for our search queries, we searched Google images \footnote{Images were searched with ``SafeSearch'' flag ``on'', and in ``United States'' region.}, taking up to $20$ images per search query. About 50\% of the resulting images contained part of the idiom they were derived from or its definitions. Such images are problematic because they may cause the model to select an image solely based on its textual signal. We filtered out these images using the optical character recognition (OCR) tool EasyOCR and TextBlob library to correct any spelling errors the OCR had. Following this filter, 15\% of the resulting images were ``garbage'' images, mainly containing text, including artworks, letters, postcards, newspapers, and articles. To tackle this problem, we used OCR and an additional method (See Appendix~\ref{sec:documents_filter}) to remove images with more than a couple of words and images with text size that exceeds more than 30\% of the image size. For the remaining images, we calculated the matching score of each image with its phrase and search query using ViLT. Images with a ``phrase-image'' score that passed a certain literal threshold (Appendix~\ref{sec:literal_threshold}) were tagged as ``literal'', and from these images, we chose the top $K$ images as literal candidates. From the non ``literal'' images, we chose the top $K$ images with the highest ``search query-image'' score as Figurative candidates. AMT workers then annotated the relation between the figurative phrase and its Figurative and Literal candidate images using the user interface (UI) seen in Figure~\ref{fig:image-task-ui},  Appendix~\ref{sec:annotation_ui}.

% Old
% To find figurative images for our search queries, we searched Google images \footnote{Images were searched with ``SafeSearch'' flag ``on'', and in ``United States'' region.}, taking up to $20$ images per search query. The resulting images often included problematic images with part of the idiom they were derived from or its definitions were written. These images were problematic because a model may see a connection between an idiom and an image solely based on the textual signal that appears in it. Such images were filtered out by using the optical character recognition (OCR) tool EasyOCR and TextBlob library to correct any spelling errors the OCR had. Many ``garbage'' images with mostly text also appeared in the search results, including letters, postcards, newspapers, and articles. To tackle this problem, we used OCR to remove images with more than a couple of words and images with a text size bigger than 30\%. In addition, we removed images of documents that the OCR failed to detect using ViLT model (Appendix~\ref{sec:documents_filter}). Next, we calculated the matching score of each image that passed these filterers with its phrase and search query using ViLT. Images with a ``phrase-image'' score that passed a certain literal threshold (Appendix~\ref{sec:literal_threshold}) were tagged as ``literal'', and from these images, we chose the top K images as literal candidates. From the non ``literal'' images, we chose the top K images with the highest ``search query-image'' score as Figurative candidates. AMT workers then annotated the relation between the figurative phrase and its Figurative and Literal candidates using the user interface (UI) seen in Figure~\ref{fig:image-task-ui},  Appendix~\ref{sec:annotation_ui}.

\subsubsection{Human Annotation}
\label{sec:human_annotation}
\begin{table}[t!]
%\centering
\small
% \resizebox{\columnwidth}{!}{
\begin{tabular}{@{}lcccccc@{}}
\toprule
   & Fig. & \begin{tabular}[x]{@{}c@{}}Fig.\\ Lit.\end{tabular} & Lit. & \begin{tabular}[x]{@{}c@{}}Part.\\ Lit.\end{tabular}  & None & \\ \midrule
\#             & 1970  &  751   &   434  &  487  & 2638 & 6697 \\ \midrule
3-maj & 100\% & 100\%  & 100\%  & 100\% & 100\% &  94\% \\
4-maj & 75.5\%  & 63\%   & 68\%   & 63\%  & 80\% &  70\% \\
5-maj & 45\%  & 33\%   & 35\%   & 38\%  & 53\% &  43\% \\ \midrule
Mean & 3.1   & 1.2    & 0.7    & 0.8   & 4    & - \\
Median  & 2     & 0      & 0      & 0     & 4    & - \\ \bottomrule
\end{tabular}
% }
\caption{IRFL statistics on 628 idioms. The majority of the images are related to the figurative phrase, most images are Figurative. (k-maj means k-majority)}
\label{tab:dataset-statistics}
\end{table}



\remove{
\begin{table}[t!]
%\centering
\small
% \resizebox{\columnwidth}{!}{
\begin{tabular}{@{}lcccccc@{}}
\toprule
Categories   & Fig. & \begin{tabular}[x]{@{}c@{}}Fig.\\ Literal\end{tabular} & Literal & \begin{tabular}[x]{@{}c@{}}Partial\\ Literal\end{tabular}  & None & Total\\ \midrule
Number             & 1970  &  751   &   434  &  487  & 2638 & 6697 \\ \midrule
3 majority & 100\% & 100\%  & 100\%  & 100\% & 100\% &  94\% \\
4 majority & 75.5\%  & 63\%   & 68\%   & 63\%  & 80\% &  70\% \\
5 majority & 45\%  & 33\%   & 35\%   & 38\%  & 53\% &  43\% \\ \midrule
Average & 3.1   & 1.2    & 0.7    & 0.8   & 4    & - \\
Median  & 2     & 0      & 0      & 0     & 4    & - \\ \bottomrule
\end{tabular}
% }
\caption{IRFL statistics on 628 idioms. The majority of the images have some relation to the figurative phrase. Most of the relations are Figurative.}
\label{tab:dataset-statistics}
\end{table}

}
We hired Amazon Mechanical Turk (AMT) workers to annotate the relation between each idiom and its candidate images using the user interface seen in Appendix~\ref{sec:annotation_ui} (Figure~\ref{fig:image-task-ui}). Five workers annotated each image in batches of five images per sample. They received a payment of \$$0.15$ per sample, which resulted in an average hourly wage of \$$15$. We created a qualification test\footnote{https://irfl-dataset.github.io/mturk/image/qualification} to select quality annotators and provided them with an interactive training platform\footnote{https://irfl-dataset.github.io/mturk/image/train} to understand the task and the different categories better. 

We split the annotation process into batches with an average size of $60$ idioms per batch. After each batch, we provided each worker with a personal profile page (Appendix~\ref{sec:annotation_ui}, Figure~\ref{fig:profile-page-ui}) to view their statistics and some examples where their choice was different from the majority of workers. 
%We also set up a leaderboard (Figure~\ref{fig:image-task-leaderboard}, Appendix~\ref{sec:annotation_ui})  that was updated after each batch to improve their competitiveness. 

Full annotation results and statistics are presented in Table \ref{tab:dataset-statistics}. 
%The nature of this task is very subjective. 
%We further discuss this in (Appendix~\ref{sec:annotation_task_discussion}).
% \dnote{why do you have discussion in the appendix? this doesn't make much sense}
Despite the subjective nature of the task, in $94\%$ of the instances, there was a majority of $3$ workers or more out of $5$ compared to a random chance of $29\%$. %This shows that different people can see the same connection most of the time. 









\subsection{Pipeline: Metaphors and Similes}
\label{sec:metaphors_and_similes}
% We collected $35$ textual metaphors and $142$ textual similes along with their definitions from online sources. Next, we used the metaphors and similes definitions as search queries to search for figurative and literal images. We manually annotated the resulting images into ``Figurative'' and ``Partial Literal'' categories. In total, we obtained $1107$ figurative images and $1816$ partial literal images for similes, and $333$ figurative images and $729$ literal images for metaphors. 
%\dnote{why a separate pipeline}\rnote{Now its good I think}\\
We collected $35$ textual metaphors and $142$ textual similes, compiled from online lists. Generating search queries from definitions (to find figurative images) is a central part of our pipeline for idioms (Section \ref{sec:idioms-collection}). However, idioms are fixed expressions, but metaphors and similes are much more flexible, as the number of possible comparisons between two things is vast. 

For this reason, we had to adapt our pipeline. For metaphors, we asked three expert annotators to agree upon definitions.
%
%\dnote{situation demonstrating}\rnote{Heart of gold -> A volunteer who distributes food in a soup kitchen}
%
%
For similes, we use the simile itself and the target concept with the shared property (``fast'') as search queries to find figurative images. For literal images that serve as distractors, we use the source and target without the shared property. In some cases, the target concept images are inadequate literal distractors (an image of a car might still be considered figurative for the simile "The car is as fast as a cheetah"). To solve this problem, we include the \emph{antonym} of the property (``A slow car'').
% Previous % In our approach to similes, we used both the simile itself and the target concept with the compared property as search queries to find figurative images. For literal images, we used the source concept, the target concept without the compared property, and the target concept with the antonym of the compared property. We specifically employed antonym images as distractors when a literal image of the target concept proved insufficient. For instance, an image of a gallon of milk doesn't serve as an effective distractor for the simile 'The milk is fresh as a daisy'.

%We collected $35$ textual metaphors and $142$ textual similes from online sources. Next, we used the metaphors and similes search queries to find figurative and literal images. 
\xhdr{Annotation} As the number of images was relatively small, we had two experts from our team manually annotate images. We obtained $1107$ figurative and $1816$ partial literal images for similes, $333$ figurative and $729$ partial literal for metaphors (the other categories were less relevant for the specific data generated by our pipeline). 

% We collected $35$ textual metaphors and $142$ textual similes along with their definitions from online sources. Next, we used the metaphors and similes definitions as search queries to search for figurative and literal images. We manually annotated the resulting images into ``Figurative'' and ``Partial Literal'' categories. In total, we obtained $1107$ figurative images and $1816$ partial literal images for similes, and $333$ figurative images and $729$ literal images for metaphors. 








% We verify the correctness of our dataset on different tasks in the human evaluation section \ref{sec:human_evaluation}. 

% We collected $628$ idioms from the MAGPIE corpus \citep{haagsma-etal-2020-magpie} of idiomatic expressions. The MAGPIE corpus contains $56,622$ crowdsourced potentially idiomatic expressions, covering $1,756$ unique idioms that appear in at least two of the following dictionaries: Wiktionary, Oxford Dictionary of English Idioms, and UsingEnglish.com. After collecting the idioms, we feed them into the pipeline as input. The first step is to collect the idioms' definitions from Wiktionary and Oxford dictionaries and construct search queries to find literal and figurative candidate images (\S\ref{sec:enriching_similes_and_idioms}). The next step is to select the best literal and figurative candidates for annotation using various heuristics and implementation decisions elaborated at (\S\ref{sec:choosing_images}). After collecting the best figurative and literal candidate images for the idioms, AMT workers annotated the different relations (Table~\ref{tab:relation-categories}) between each idiom and its candidate image, thus creating the IRFL dataset (\S\ref{sec:human_annotation}).

% We evaluate the end-to-end dataset generation, and the fact that humans achieve high agreement helps verify the end-to-end process's correctness. \\\\
% To collect metaphors and similes images, we collected $35$ textual metaphors and $142$ textual similes along with their definitions from the internet. Next, we used the metaphors and similes definitions as search queries and adapted the method used in (\S\ref{sec:choosing_images}) to search for figurative and literal images. We manually annotated the resulting images into ``Figurative'' and ``Literal'' categories. In total, we obtained $1107$ figurative images and $1816$ literal images for similes, and $333$ figurative images and $729$ literal images for metaphors. We verify the correctness of our dataset on different tasks in the human evaluation section \ref{sec:human_evaluation}. 



% Older
% Our goal is to generate the IRFL dataset of idioms, metaphors, and similes with matching figurative and literal images and evaluate the figurative understanding and preference of Vision and Language models. To collect figurative and literal images for idioms, we developed an automatic pipeline that takes a list of idioms as input and outputs figurative and literal candidate images. We collected $628$ idioms from the MAGPIE corpus \citep{haagsma-etal-2020-magpie} of idiomatic expressions. The MAGPIE corpus contains $56,622$ crowdsourced potentially idiomatic expressions, covering $1,756$ unique idioms that appear in at least two of the following dictionaries: Wiktionary, Oxford Dictionary of English Idioms, and UsingEnglish.com. After collecting the idioms, we then feed them into the pipeline as input. First, we collect the definitions of these idioms from Wiktionary and Oxford dictionaries and construct search queries to find possible literal and figurative images (\S\ref{sec:enriching_similes_and_idioms}). Next, we select the best literal and figurative candidates for annotation using various heuristics and implementation decisions elaborated at (\S\ref{sec:choosing_images}). AMT workers annotated the different relations between each idiom and its candidate images, creating the IRFL dataset (\S\ref{sec:human_annotation}). We evaluate the end-to-end dataset generation, and the fact that humans achieve high agreement helps to verify the correctness of the end-to-end process. The relation categories can be seen with corresponding explanations and images in Table~\ref{tab:relation-categories}.

% To collect metaphors and similes' images, we collected $35$ textual metaphors and $142$ textual similes from the internet. First, we collected metaphors and similes definitions and used them as search queries and adapted the method used to search images in (\ref{sec:choosing_images}). Next, we annotated these images into ``Figurative'' and ``Literal'' categories. In total, we obtained $1107$ figurative images and $1816$ literal images for similes, and $333$ figurative images and $729$ literal images for metaphors. We verify the correctness of our dataset on different tasks in the human evaluation section \ref{sec:human_evaluation}. 





%\subsection{Dataset Analysis}
%\label{sec:dataset_statistics}
%\input{sections/04E_dataset_statistics}





