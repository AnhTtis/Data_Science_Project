% This must be in the first 5 lines to tell arXiv to use pdfLaTeX, which is strongly recommended.
\pdfoutput=1
% In particular, the hyperref package requires pdfLaTeX in order to break URLs across lines.

\documentclass[11pt]{article}

% Remove the "review" option to generate the final version.
\usepackage{ACL2023}

% Standard package includes
\usepackage{times}
\usepackage{latexsym}

% For proper rendering and hyphenation of words containing Latin characters (including in bib files)
\usepackage[T1]{fontenc}
% For Vietnamese characters
% \usepackage[T5]{fontenc}
% See https://www.latex-project.org/help/documentation/encguide.pdf for other character sets

% This assumes your files are encoded as UTF8
\usepackage[utf8]{inputenc}

% This is not strictly necessary, and may be commented out.
% However, it will improve the layout of the manuscript,
% and will typically save some space.
\usepackage{microtype}

% This is also not strictly necessary, and may be commented out.
% However, it will improve the aesthetics of text in
% the typewriter font.
\usepackage{inconsolata}

\usepackage[utf8]{inputenc}
\usepackage{float}
\usepackage{xcolor}         % colors
\usepackage{graphicx}
\usepackage{comment}
\usepackage{url}
\usepackage[T1]{fontenc}
\usepackage[utf8]{inputenc}
\usepackage{babel}
\usepackage{color}
\usepackage{tabularx}
\usepackage{booktabs}
\usepackage{multirow}
\usepackage{wrapfig}
\usepackage{natbib}
\newcommand{\commenthuji}[3]{{\small{\textcolor{#3}{[#1 #2]}}}}
% \renewcommand{\commenthuji}[3]{}  % uncomment for submission

\newcommand{\com}[1]{}
\newcommand{\resolved}[1]{}

\newcommand{\marker}[1]{#1:}

\newcommand{\dnote}[1]{\commenthuji{\marker{Dafna}}{#1}{blue}}
\newcommand{\yonatan}[1]{\commenthuji{\marker{YONATAN}}{#1}{red}}
\newcommand{\yonatanst}[1]{\yonatan{\sout{#1}}}
\newcommand{\yonatanrep}[2]{\yonatan{\sout{#1} #2}}

\newcommand{\ron}[1]{\commenthuji{\marker{RON}}{#1}{yellow}}
\newcommand{\ronout}[1]{\textcolor{green}{\sout{#1}}}
\newcommand{\ronnew}[1]{\textcolor{green}{#1}}
\newcommand{\ronreplace}[2]{\ronout{#1} \ronnew{#2}}

% If the title and author information does not fit in the area allocated, uncomment the following
%
%\setlength\titlebox{<dim>}
%
% and set <dim> to something 5cm or larger.

\title{IRFL: Image Recognition of Figurative Language}

% Author information can be set in various styles:
% For several authors from the same institution:
% \author{Author 1 \and ... \and Author n \\
%         Address line \\ ... \\ Address line}
% if the names do not fit well on one line use
%         Author 1 \\ {\bf Author 2} \\ ... \\ {\bf Author n} \\
% For authors from different institutions:
% \author{Author 1 \\ Address line \\  ... \\ Address line
%         \And  ... \And
%         Author n \\ Address line \\ ... \\ Address line}
% To start a seperate ``row'' of authors use \AND, as in
% \author{Author 1 \\ Address line \\  ... \\ Address line
%         \AND
%         Author 2 \\ Address line \\ ... \\ Address line \And
%         Author 3 \\ Address line \\ ... \\ Address line}



\author{Ron Yosef, Yonatan Bitton, Dafna Shahaf\\
        The Hebrew University of Jerusalem\\
        \texttt{\{ron.yosef, yonatan.bitton,dafna.shahaf\}@mail.huji.ac.il}\\
        }

% \author{Ron Yosef \And Yonatan Bitton \And Dafna Shahaf\\
%   The Hebrew University of Jerusalem\\ The Hebrew University of Jerusalem\\ The Hebrew University of Jerusalem\\
%   \texttt{{ron.yosef, yonatan.bitton,dafna.shahaf}@mail.huji.ac.il} \\}
  

\begin{document}

\maketitle


\begin{abstract}
%%%%%%%%% ABSTRACT



\begin{abstract}

% Version 1:
% Modern depth sensors such as LiDAR operate by sweeping laser-beams across the scene, resulting in a point cloud with notable 1D curve-like structures. However, most existing point cloud backbones  discard the rich, 1D traversal patterns and rely mainly on Euclidean operations.
% In this work, we present a novel point cloud processing scheme and backbone, \textbf{CurveCloudNet}, that exploits the curve-like structure of modern depth sensors. Concretely, %instead of treating each point independently, 
% we parameterize the point cloud as a collection of polylines and thus establish a local surface-level ordering on the points. 
% We then devise curve-specific operations to process the ``curve clouds:'' (1) a \textit{symmetrical 1D convolution}, 2) a \textit{ball grouping} operation for merging points along curves, and (3) an efficient \textit{1D furthest-point-sampling} algorithm on curves. \textbf{CurveCloudNet} combines these curve operations with existing point-based operations, resulting in an efficient, scalable, and expressive backbone that uses little GPU memory. We evaluate \textbf{CurveCloudNet} on the ShapeNet, Kortx, Audi Driving, and nuScenes datasets, showcasing state-of-the-art segmentation and classification performance across {\em both} object-level and large outdoor scene datasets, the first reported 3D point backbones to do so. 

% Version 2:
% In this work we introduce a new point cloud processing scheme and backbone, called CurveCloudNet, which takes advantage of the curve-like structure inherent in modern depth sensors such as LiDAR. While traditional point cloud backbones discard the rich, 1D laser-traversal patterns and rely on Euclidean operations, CurveCloudNet parameterizes the point cloud as a collection of polylines. This parameterization establishes a local surface-level ordering on the points. Our method applies curve-specific operations to process the ``curve clouds," including symmetrical 1D convolution, ball grouping for merging points along curves, and an efficient 1D furthest-point-sampling algorithm on curves. Combining these curve operations with existing point-based operations results in an efficient, scalable, and expressive backbone that uses little GPU memory. We evaluate CurveCloudNet on several datasets, including ShapeNet, Kortx, Audi Driving, and nuScenes, and report state-of-the-art segmentation and classification performance across \textbf{both} object-level and large outdoor scene datasets, making CurveCloudNet the first 3D point backbone to achieve such results.
% \vspace{-1em}

% Version 3
Modern depth sensors such as LiDAR operate by sweeping laser-beams across the scene, resulting in a point cloud with notable 1D curve-like structures. In this work, we introduce a new point cloud processing scheme and backbone, called \arch, which takes advantage of the curve-like structure inherent to these sensors. While existing backbones discard the rich 1D traversal patterns and rely on generic 3D operations, \arch parameterizes the point cloud as a collection of polylines (dubbed a ``curve cloud”), establishing a local surface-aware ordering on the points. By reasoning along curves, \arch captures lightweight curve-aware priors to efficiently and accurately reason in several \textbf{diverse} 3D environments. 
% , including a symmetric 1D convolution, a ball grouping for merging points along curves, and an efficient 1D farthest point sampling algorithm on curves.
We evaluate \arch on multiple synthetic and real datasets that exhibit distinct 3D size and structure.
%, including: ShapeNet, Audi Driving, nuScenes, Kitti, and a new dataset we name KortX.
We demonstrate that \arch outperforms both point-based and sparse-voxel backbones in various segmentation settings, notably scaling to large scenes better than point-based alternatives while exhibiting improved single-object performance over sparse-voxel alternatives.
In all, \arch is an efficient and accurate backbone that can handle a larger variety of 3D environments than past works. 
%In all, \arch is an off-the-shelf trainable and performant backbone that is ready for the diverse environments faced in open-world applications such as robotics. 

% in various segmentation settings, notably scaling better to large scenes than point-based alternatives while exhibiting better single object performance than sparse-voxel alternatives. 

% CurveCloudNet applies a mix of curve-specific operations and Euclidean point-based operations, resulting in an efficient and accurate backbone that can flexibly reason on \textit{many} different types of 3D scenes. 
% , including a symmetric 1D convolution, a ball grouping for merging points along curves, and an efficient 1D farthest point sampling algorithm on curves.
% By combining these curve operations with existing point-based operations, CurveCloudNet is an efficient and accurate backbone that can flexibly reason on \textit{many} different types of 3D scenes. 
% CurveCloudNet achieves state-of-the-art segmentation performance on the ShapeNet, Kortx, Audi Autonomous Driving, and nuScenes datsets, which include both individual objects and large outdoor scenes captured with various sensor scanning patterns. These evaluations demonstrate that \arch scales to large scenes better than existing point-based backbones while improving object-level semantic segmentation compared to sparse-voxel backbones.
% We evaluate semantic segmentation on four datasets - two common (ShapeNet and nuScenes) and two less common (KortX and Audi Driving). Taken together, these datasets patterns -
% We evaluate semantic segmentation the ShapeNet, Kortx, Audi Driving, and nuScenes datasets. 

% demonstrate that \arch outperforms both point-based and sparse-voxel backbones in various segmentation settings, notably scaling better to large scenes than point-based alternatives while exhibiting better single object performance than sparse-voxel alternatives. 

% Evaluations on ShapeNet, Kortx, Audi Driving, and nuScenes demonstrate that \arch outperforms point-based methods on both individual objects and large-scale scenes, outperforms sparse-voxel backbones on individual objects, and closes the gap between point-based and sparse-voxel backbones on large-scale scenes while requiring significantly less GPU memory.

% CurveCloudNet is evaluated on several datasets that include both individual objects and large
% outdoor scenes captured with various sensor scanning patterns. These evaluations demonstrate that our model can
% outperform point-based and sparse-voxel backbones at both
% object and scene level, achieving state-of-the-art performance on segmentation tasks.
\vspace{-1em}
\end{abstract}


\end{abstract}

\section{Introduction}




Figures of speech include metaphors, similes, and idioms that allow language to be expressive, to convey abstract ideas that might otherwise be difficult to visualize, and to evoke emotion \cite{why-do-people-use-figurative-language, SusanAndMallie1994}. A metaphor is a comparison between two unrelated concepts that enable us to think of the target concept in terms of the source concept. For example, in the sentence ``You’re a peach!'', the person being addressed is equated with a peach, with the suggestion that the person is pleasing or delightful. A simile is a figure of speech that compares two things and is often introduced by ``like'' or ``as'' \cite{Paul-1970}. A simile is called ``open'' when the shared properties are not explicitly revealed, like ``Her heart is like a stone'', and ``closed'' when they are explicitly revealed, like ``Her heart is hard as stone''. An idiom is a group of words with a figurative, non-literal meaning that can not be interpreted by looking at its individual words. For example, the idiom ``We're on the same page'' means ``Agreeing about something (such as how things should be done)''.
\begin{figure}[tp]
\includegraphics[width=0.45\textwidth ,height=\textheight,keepaspectratio]{figures/let_the_cat_out_of_the_bag_big.jpg}
\includegraphics[width=0.45\textwidth ,height=\textheight,keepaspectratio]{figures/blanket_of_snow_bigger.jpg}
\includegraphics[width=0.45\textwidth ,height=\textheight,keepaspectratio]{figures/car_cheetah_bigger.JPG}
\caption{Examples of the figurative understanding task for idiom, metaphor, and simile in corresponding order. The figurative phrase is displayed in the top section, and the bottom section displays four candidates from which the correct answer (orange) has been selected. Idiom tasks also display the idiom definitions below the idiom.}
\label{fig:first-task-idiom-figurative}
\end{figure}
Understanding metaphors and similes require the cognitive ability to map between domains, and depending on the source and target concept, it can require commonsense, association abilities, and general knowledge. Understanding idioms requires profound language, and cultural knowledge 
\cite{Paul-1970, philip2011colouring}. Humans intuitively understand these figures and employ them in everyday communication \cite{LakoffandJohnson1980, Hoffman1987WhatCR}. These figurative forms are often conveyed through multiple modes, such as text and images, and frequently appear in advertising, news, social media, etc. \\
%Figure~\ref{fig:multimodal} presents an image of a luxury car posted on social media with the simile ``As fast as a cheetah'' and an advertisement for Toyota Corolla with the idiom ``get your hands on''. \\
%\yonatan{so far you have a table and an image for explaining existing concepts. I wonder whether it should be in the Appendix rather than taking important space at the start of the paper, because it's not something new you present in the work.}\ron{@Dafna}
% Multimodal metaphors are metaphors conveyed through multiple modes ``whose target and source are each represented exclusively or predominantly in different modes'' \cite{Forceville2016}. Multimodal information from different modes, such as language and vision, can contribute to metaphorical conceptualization and comprehension \cite{PerspectivesMultimodalityBook, multimodalMetaphorBook}. The comprehension of multimodal metaphors takes cognitive efforts like decoding metaphorical messages and understanding the relationships between domains, analyzing the emotion metaphors convey, and interpreting authorial intent [5–7] \cite{conceptual-integration-and-metaphor, YANG2013312, FAUCONNIER1998133}. Since similes are often referred to as metaphors in the cognitive linguistic and rhetorical literature \cite{culler1981pursuit, lou2021multimodal}, the topic of multimodal similes has not been thoroughly investigated. Recently, Adrian Lou \cite{lou2021multimodal} proposed a novel and robust framework for the analysis of both verbal and multimodal similes. Multimodal idioms have yet to be studied cognitively, in part due to a lack of multimodal datasets.  \cite{LakoffandJohnson1980, Hoffman1987WhatCR}. 
% \begin{figure}[h]
% \includegraphics[width=0.48\textwidth,height=12cm,keepaspectratio]{figures/Figure 1 - multimodal example.JPG}
% \caption{An advertisement for Toyota Corolla from the 60s with the idiom "get your hands on" and an image posted on a social media platform with the caption ``As fast as a cheetah''.}
% \label{fig:multimodal}
% \end{figure}
% There has been an increase in the use of text and vision modes over the past few years due to the growing usage of social media and mass media. The increased use of multimodality has created new challenges, which sometimes expand existing challenges from monomodality to multimodality.
\newline\noindent Due to its integral part in human communication, the detection and comprehension of multimodal figurative language is an important aspect of various multimodal challenges. Among these challenges are hate speech detection in memes \cite{detecting-hate-speech-in-multi-modal-memes}, fact-checking \cite{multimodal-fact-checking}, sentiment analysis \cite{SOLEYMANI20173}, humor recognition \cite{REYES20121, detecting-sarcasm-in-multimodal-social-platforms}, and identifying depression in social media posts \cite{yadav-etal-2020-identifying, multimodal-time-aware-attention-networks}. Figure \ref{fig:figurative-language-social-media} shows two photos posted on social media with metaphoric captions. In the left image, the caption reads, ``Jumped off the sinking ship just in time'', as this player left Chelsea - ``the sinking ship'', which is having a bad year, to join the leading team of the premier league, Arsenal. The right image was posted with the caption ``A performing clown'', as the person who is getting hit is a famous YouTuber who lost in a boxing match against a professional boxer. Multimodal figurative understanding is required to comprehend the metaphorical message being conveyed in these two posts. % In the task of sentiment analysis, the sentiment of the right post should be classified as ``negative'', but this cannot be achieved solely by text or image.
% In light of this rapidly growing trend, figurative language studies must be expanded from monomodality to multimodality.
Vision and Language Pre-Trained Models’ (VL-PTMs) understanding of figurative language combined with vision has not been thoroughly examined, if at all, partly due to the absence of large-scale datasets with ground truth labels of multimodal smilies, idioms, metaphors, etc.\\
%\begin{table}[ht]
\begin{center}
\scalebox{0.6}{
\begin{tabularx}{380pt}{| c X X |}
 \hline
 Figure & Example & Explanation \\ [0.5ex] 
 \hline\hline
 Metaphor & "You're a peach!" & The person being addressed is being equated with a peach, with the suggestion that the person is pleasing or delightful. The target concept is "person" and the source concept is "peach".\\ 
 \hline
 Open Simile & "Her heart is like stone" &  Inflexible and unfriendly or unkind disposition. The shared properties of "her heart" and "stone" are not explicitly revealed. The target concept is "her heart" and the source concept is "stone".  \\
 \hline
 Closed Simile & "The old man walk as slow as a snail" &  The old man's movement is compared to that of a snail, the shared property ("slow") is explicitly revealed. The target concept is "old man" and the source concept is "snail".\\
 \hline
 Idiom & "We're on the same page" &  Agreeing about something (such as how things should be done).  \\
 \hline
\end{tabularx}}

\end{center}
\caption{\ron{@Dafna Should we delete this/move it to appendix? If not, I think merging it with Figure 2 could be a good idea}Examples of simile, metaphor and idiom with a corresponding explanation.}
\label{table:figures-of-speech}
\end{table}%
\begin{figure}[tp]
\begin{center}
\includegraphics[width=0.5\textwidth,height=12cm,keepaspectratio]{figures/figure-figurative_language_in_social_media-bigger.jpg}
\end{center}
\caption{Two photos posted on social media. The left photo depicts football player Jorge Luiz Frello Filho Cavaliere wearing an Arsenal football club uniform. The right photo shows famous YouTuber Jake Paul taking a hit from professional boxer Tommy Fury during their boxing fight.}
\label{fig:figurative-language-social-media}
\end{figure}
\newline \noindent In this work, we introduce the IRFL dataset of idioms, metaphors, and similes with matching figurative and literal images. We leveraged a textual dataset of idioms and an extensive pipeline we developed to find possible figurative and literal idiom images. We annotated these images via Amazon Mechanical Turk using the UI seen in (Appendix~\ref{sec:annotation_ui}) to create a large-scale dataset of idioms' figurative and literal images. In addition, we collected metaphors and similes' figurative and literal images. We then used the IRFL dataset to create two novel tasks of figurative understanding and figurative preference to examine the figurative understanding of Vision and Language models. The figurative understanding task evaluates VL-PTMs' ability to understand the relation between an image and a figurative phrase. The task is to choose the image that best visualizes the figurative phrase out of X candidates.  Figure~\ref{fig:first-task-idiom-figurative} shows an example of the task for idiom, metaphor, and simile. The preference task examines VL-PTMs' preference for figurative images. In this task, the model needs to rank figurative images of different categories correctly. Figure~\ref{fig:second-task-idiom-figurative-vs-caption} shows the expected order versus the actual order of the idiom ``ruffle someone's feathers'' images based on the model scores. Finally, we experiment with generative models such as Dall-E and Stable Diffusion to examine their ability to generate figurative images for idioms. %\ron{Should the result be in intro?}%
% \yonatan{change quotation marks to ``THIS''} - Did not understand this
% \yonatan{figures in seperate files?} - Didn't understand this
% The figurative understanding task evaluates VL-PTMs' ability to understand the relation between an image and a figurative phrase. The task is to choose the image that best visualizes the figurative phrase out of X candidates.  Figure~\ref{fig:first-task-idiom-figurative} shows an example of the task for idiom, metaphor, and simile.
% The figurative understanding task evaluates VL-PTMs' ability to understand the relation between an image and a figurative phrase. The task is to choose the image that best visualizes the figurative phrase out of X candidates. Figure~\ref{fig:first-task-idiom-figurative} shows an example for the idiom ``Let the cat out of the bag'' - to disclose a secret, and examples with difficult distractors for the metaphor "Sea of bees" and the simile "The child is as proud as a peacock". Using partially literal images as distractors, the best model ($22\%$) fails to match human performance ($92\%$). This task goes beyond object detection and scene understanding, it requires a profound and rich language and cultural knowledge (idioms) in addition to commonsense, abstraction, general knowledge and the ability to decode the domains of figurative phrases and understand their relationships (metaphors and similes).
% \newline \noindent The preference task examines VL-PTMs' preference of figurative images over partially literal images. In this task, the model needs to rank figurative phrase images of different categories correctly. We suggest that Vision and language models should prioritize figurative images over partial literal images. Meaning that an image of someone disclosing a secret should receive a higher matching score for the idiom  ``let the cat out of the bag'' than an image of a bag. Figure~\ref{fig:second-task-idiom-figurative-vs-caption} shows the expected order versus the actual order of the idiom ``ruffle someone's feathers'' images based on the model scores. Assuming that the model understands the idiom and sees a figurative connection to an image, the task purpose is to measure how well the model comprehends it. This is done by comparing the figurative images' matching score to other images with a weaker relationship (partially literal). We find that the best model receives a $F_1$ score of $35-50$ depending on the figure of speech.
% \noindent Finally, we experiment with generative models such as Dall-E and Stable Diffusion to examine their ability to generate figurative images for idioms. We provide these models with idioms and their definitions as prompts and compare the results to our automatic pipeline. Our findings show that Stable Diffusion fails to generate figurative images even when given the definitions as input, while Dall-E succeeds in generating figurative images for the idioms' definitions.
\begin{figure}[h]
\includegraphics[width=0.45\textwidth,height=\textheight,keepaspectratio]{figures/Figure_5_ranking_task.JPG}
\caption{The Figurative and Partial Objects images of the idiom ``ruffle someone's feathers'' - ``To unease, cause discomfort to someone'' sorted from right to left by the CLIP-VIT-L/14 score. The images with the letter F are figurative, and the images with the letter P are partially literal. Green indicates correct rank, red indicates incorrect rank. The first row shows the ideal ranking order, while the second row shows the actual one. Figurative images with a green letter will appear in the first \#F image from the right, and Partial Objects images with a green letter will appear in the last \#P image. In this example, all of the models except LiT received 0 $F_1$ score.}
\label{fig:second-task-idiom-figurative-vs-caption}
\end{figure}



\section{The IRFL Dataset}
% \yonatan{is it correct to say that it's an automatic generation followed by human ratings? If so I'll start by that. Also I suggest to say something similar to what we said in VASR: this is a process that contains several heuristics and implementation decisions. We evaluate the end2end dataset generation later on, and the fact that human achieve high agreement helps to verify the correctness of the end2end process. }
%\yonatan{is it correct to say that it's an automatic generation followed by human ratings? If so I'll start by that} \ron{We didn't started like this in VASR, why is it important here?}
\begin{figure}[t!]
\includegraphics[width=0.44\textwidth ,height=\textheight,keepaspectratio]{figures/up_a_tree_bigger.JPG}
\includegraphics[width=0.44\textwidth ,height=\textheight,keepaspectratio]{figures/blanket_of_snow_bigger.jpg}
\includegraphics[width=0.44\textwidth ,height=\textheight,keepaspectratio]{figures/car_cheetah_bigger.JPG}
\caption{Examples of the multimodal figurative language detection task for idiom, metaphor, and simile. The input is a figurative phrase and four candidate images (for idiom, we also show the definition). The correct answer is marked with an orange square.}
\label{fig:first-task-idiom-figurative}
\end{figure}

\begin{table*}[tp!]
\begin{center}
\small
\begin{tabular}{ m{8em} m{8em} m{8em} m{8em} m{8em}}
\toprule
\multicolumn{5}{c}{\textbf{Idiom:} Touch wood} \\
\multicolumn{5}{c}{\begin{tabular}[x]{@{}c@{}} \textbf{Definitions:} 1) Hopefully 2) Said while touching something wooden, \\ to avert  superstitious bad luck from what has just been said \end{tabular}} \\ 

\midrule
\includegraphics[width=0.18\textwidth,height=5.5em]{figures/touch_wood_caption.jpeg} & 
\includegraphics[width=0.18\textwidth ,height=5.5em]{figures/touch_wood_figurative.jpeg} &
\includegraphics[width=0.18\textwidth ,height=5.5em]{figures/touch_wood_partial_objects.jpeg} & 
\includegraphics[width=0.18\textwidth ,height=5.5em]{figures/touch_wood_figurative_literal.jpg} &  \includegraphics[width=0.18\textwidth ,height=5.5em]{figures/touch_wood_none.jpeg} \\ \midrule

 Literal & Figurative & Partial Literal & Figurative+Literal & None  \\ \midrule

 The image illustrates the phrase literally & 
 The image conveys one or more \emph{definitions} of the idiom &
 Some objects/ entities of the phrase are visualized  (here, wood) & 
 Fits the ``Figurative'' definition and also ``Literal''/``Partial Literal''  &
 The image does not fit any of the other categories  \\ \bottomrule
\end{tabular}


% \small
% \begin{tabular}{ c c c c c c}
% \toprule
% \adjustbox{valign=c}{\includegraphics[width=0.18\textwidth ,height=\textheight,keepaspectratio]{figures/touch_wood_figurative_literal.jpg}} &
% \adjustbox{valign=c}{\includegraphics[width=0.18\textwidth ,height=\textheight,keepaspectratio]{figures/touch_wood_figurative_literal.jpg}} &
% \adjustbox{valign=c}{\includegraphics[width=0.18\textwidth ,height=\textheight,keepaspectratio]{figures/touch_wood_figurative_literal.jpg}} &
% \adjustbox{valign=c}{\includegraphics[width=0.18\textwidth ,height=\textheight,keepaspectratio]{figures/touch_wood_figurative_literal.jpg}} &
% \adjustbox{valign=c}{\includegraphics[width=0.18\textwidth ,height=\textheight,keepaspectratio]{figures/touch_wood_figurative_literal.jpg}} & \midrule

%  Figurative+Literal & The image conveys one or more definitions of the idiom to some extent,
% and it literally illustrates the phrase or visualizes the phrase objects/entities & \adjustbox{valign=c}{\includegraphics[width=0.18\textwidth ,height=\textheight,keepaspectratio]{figures/touch_wood_figurative_literal.jpg}} &\\  \midrule
%  Figurative & The image conveys one or more definitions of the idiom to some extent & \adjustbox{valign=c}{\includegraphics[width=0.18\textwidth ,height=\textheight,keepaspectratio]{figures/touch_wood_figurative_literal.jpg}}&\\ \midrule
%  Caption &  The image illustrates the phrase literally & \adjustbox{valign=c}{\includegraphics[width=0.18\textwidth ,height=\textheight,keepaspectratio]{figures/touch_wood_figurative_literal.jpg}} &\\ \midrule
%  Partial Objects &  The objects/entities of the phrase are visualized in the image & \adjustbox{valign=c}{\includegraphics[width=0.18\textwidth ,height=\textheight,keepaspectratio]{figures/touch_wood_figurative_literal.jpg}} &\\ \midrule
%  None &  The image does not fit any of the categories & \adjustbox{valign=c}{\includegraphics[width=0.18\textwidth ,height=\textheight,keepaspectratio]{figures/touch_wood_figurative_literal.jpg}} &\\  \bottomrule 
% \end{tabular}


% \small
% \begin{tabular}{ m{3.5em} m{5.3cm} c c}
% \toprule
%  Figurative+Literal & The image conveys one or more definitions of the idiom to some extent,
% and it literally illustrates the phrase or visualizes the phrase objects/entities & \adjustbox{valign=c}{\includegraphics[width=0.18\textwidth ,height=\textheight,keepaspectratio]{figures/touch_wood_figurative_literal.jpg}} &\\  \midrule
%  Figurative & The image conveys one or more definitions of the idiom to some extent & &\\ \midrule
%  X &  The image illustrates the phrase literally &\\ \midrule
%  x &  The objects/entities of the phrase are visualized in the image & &\\ \midrule
%  None &  The image does not fit any of the categories & &\\  \bottomrule 
% \end{tabular}


\end{center}
\caption{The table shows the different categories of the relation between an image and a phrase, along with matching images for the idiom "Touch wood". Workers were guided to choose the most suitable relation category by a scheme tree that illustrates the correct thinking process (Figure~\ref{fig:image-task-tree}, Appendix~\ref{sec:annotation_ui}).}
\label{tab:relation-categories}
\end{table*}

Our goal is to create a dataset with idioms, metaphors, and similes paired with figurative and literal images. This dataset can then serve as a benchmark to evaluate Vision and Language models on multimodal figurative language. 

\xhdr{Labels} Initially, we intended to have our annotators label images ``literal'' or ``figurative''. However, after initial experimentation with the data generated by our pipeline, we realized the necessity of a more nuanced classification system. Hence, we introduced two additional categories.

The first new category, ``Figurative+Literal,'' encompasses images that express the figurative meaning of an expression while also maintaining some aspects of the literal interpretation. The second, ``Partial Literal,'' includes images that visualize some (literal) elements or objects from the expression. 

 Table~\ref{tab:relation-categories} illustrates our categories for the expression ``Touch wood''. For example, an image of someone literally touching wood while crossing his fingers for luck is classified as Figurative+Literal.   
This distinction also allows us to later perform a richer analysis of model performance. 

%To create multimodal idioms, we gathered idiomatic expressions from the MAGPIE corpus \citep{haagsma-etal-2020-magpie}. We then utilized a semi-automatic pipeline we developed to find figurative and literal images (\S\ref{sec:idioms-collection}). For multimodal metaphors, we collected similes and metaphors from various online sources. Subsequently, we manually collected and annotated the corresponding figurative and literal images (\S\ref{sec:metaphors_and_similes}).


\subsection{Pipeline: Idioms}
\label{sec:idioms-collection}
We collected $628$ idioms from the MAGPIE corpus \citep{haagsma-etal-2020-magpie} of idiomatic expressions. The MAGPIE corpus contains $56,622$ crowdsourced potentially idiomatic expressions, covering $1,756$ unique idioms that appear in at least two of the following dictionaries: Wiktionary, Oxford Dictionary of English Idioms, and UsingEnglish.com. After collecting the idioms, we feed them into our pipeline. 

Our pipeline consists of four main steps, illustrated in Figure~\ref{fig:figurative-pipeline}. Given an idiom, we first get its definitions from online dictionaries and parse them into search queries (\S\ref{sec:enriching_similes_and_idioms}). Second, we search for candidate images using the search queries. Third, we filter the images and select the best $k$ literal and figurative candidates for annotation (\S\ref{sec:choosing_images}). Lastly, we annotate the different images via crowdworkers (\S\ref{sec:human_annotation}). 


\subsubsection{Searching for Images}
\label{sec:enriching_similes_and_idioms}
\begin{figure}[b!]
%\begin{center}
\includegraphics[width=0.48\textwidth,keepaspectratio]{figures/pipeline-figure.JPG}
%\end{center}
\caption{The flow of our idiom pipeline: getting definitions, looking for image candidates using the idiom and its definitions, filtering an selecting candidate images. In the human annotation stage, blue represents Literal, Green -- Figurative, and red -- None.}
\label{fig:figurative-pipeline}
\end{figure}
Our goal is to find literal and figurative images for each idiom from the MAGPIE dataset. Searching for an idiom using image search often results in literal images. To find figurative images, we need to understand the \emph{meaning} of the idiom; however, the MAGPIE dataset does not contain idiom definitions, so we crawl them from online dictionaries (Wiktionary definitions tagged with `figurative'' or ``idiomatic''\footnote{We also construct search queries from untagged definitions. Even though untagged definitions are rare (<3\%), they are typically idiomatic.}; if no such definitions exist, we try the Oxford Dictionary).

For example, in Figure \ref{fig:figurative-pipeline}, the idiom ``white hat'' nd is defined as ``A good person; a hero'' (tagged with ``idiomatic''), and also as ``a sailor'' and ``A well-meaning hacker'' (tagged with ``slang"). 

%To accomplish this, we use the idioms and their definitions as search queries for an image search engine.  For each idiom, we search Wiktionary for definitions, and if no definitions are found, we search the Oxford Dictionary. Wiktionary definitions are usually accompanied by tags that indicate the context in which they appear. For example, the idiom ``white hat'' and its definition ``A good person; a hero'' is tagged with "figurative" and "idiomatic" tags, and the definitions ``a sailor'' and ``A well-meaning hacker'' are tagged with a "slang" tag. Using this data, we filter idioms with no ``figurative'' or ``idiomatic'' tags and construct search queries from these definitions \footnote{We also construct search queries from untagged definitions. Even though untagged definitions are rare (<3\%), they are typically idiomatic.}. 
We split concatenated definitions (e.g., ``A good person; a hero'' is split into two definitions). %concatenated into one. For example, we split the definition ``A good person; a hero'' into two search queries ``A good person'' and ``A hero''. 
In some rare cases, a definition may be another idiom, and then we replace that idiom with its definitions.

We then searched Google images for the idioms and their definitions, taking up to $20$ images per search query. Images were searched with ``SafeSearch'' flag ``on'', and in ``United States'' region.
\subsubsection{Image Filtering}
\label{sec:choosing_images}
We noted that many of the retrieved images contained the search query in textual form. We used optical character recognition (OCR) tool EasyOCR to extract text from the images, and TextBlob to correct spelling errors the OCR made. We then  filtered images that contained objects or entities from the idiom  or its definitions in textual form (50\% of the images). Such images are problematic because they may cause the model to select an image solely based on its textual signal. Following this filter, 15\% of the resulting images contained mostly text. To tackle this problem, we used OCR (See Appendix~\ref{sec:documents_filter}) to remove images with more than a couple of words, as well as images with more than 30\% of their space containing text. 

For the remaining images, we calculated the matching score of each image with its phrase and search query using ViLT. Top-$k$ images with a high ``phrase-image'' score (that passed a threshold, see Appendix~\ref{sec:literal_threshold}) were tagged as potentially literal.  We chose the top $k$ images with the highest ``definition-image'' score as Figurative candidates. 

%AMT workers then annotated the relation between the figurative phrase and its Figurative and Literal candidate images using the user interface (UI) seen in Figure~\ref{fig:image-task-ui},  Appendix~\ref{sec:annotation_ui}.
% Old_v2
% To find figurative images for our search queries, we searched Google images \footnote{Images were searched with ``SafeSearch'' flag ``on'', and in ``United States'' region.}, taking up to $20$ images per search query. About 50\% of the resulting images contained part of the idiom they were derived from or its definitions. Such images are problematic because they may cause the model to select an image solely based on its textual signal. We filtered out these images using the optical character recognition (OCR) tool EasyOCR and TextBlob library to correct any spelling errors the OCR had. Following this filter, 15\% of the resulting images were ``garbage'' images, mainly containing text, including artworks, letters, postcards, newspapers, and articles. To tackle this problem, we used OCR and an additional method (See Appendix~\ref{sec:documents_filter}) to remove images with more than a couple of words and images with text size that exceeds more than 30\% of the image size. For the remaining images, we calculated the matching score of each image with its phrase and search query using ViLT. Images with a ``phrase-image'' score that passed a certain literal threshold (Appendix~\ref{sec:literal_threshold}) were tagged as ``literal'', and from these images, we chose the top $K$ images as literal candidates. From the non ``literal'' images, we chose the top $K$ images with the highest ``search query-image'' score as Figurative candidates. AMT workers then annotated the relation between the figurative phrase and its Figurative and Literal candidate images using the user interface (UI) seen in Figure~\ref{fig:image-task-ui},  Appendix~\ref{sec:annotation_ui}.

% Old
% To find figurative images for our search queries, we searched Google images \footnote{Images were searched with ``SafeSearch'' flag ``on'', and in ``United States'' region.}, taking up to $20$ images per search query. The resulting images often included problematic images with part of the idiom they were derived from or its definitions were written. These images were problematic because a model may see a connection between an idiom and an image solely based on the textual signal that appears in it. Such images were filtered out by using the optical character recognition (OCR) tool EasyOCR and TextBlob library to correct any spelling errors the OCR had. Many ``garbage'' images with mostly text also appeared in the search results, including letters, postcards, newspapers, and articles. To tackle this problem, we used OCR to remove images with more than a couple of words and images with a text size bigger than 30\%. In addition, we removed images of documents that the OCR failed to detect using ViLT model (Appendix~\ref{sec:documents_filter}). Next, we calculated the matching score of each image that passed these filterers with its phrase and search query using ViLT. Images with a ``phrase-image'' score that passed a certain literal threshold (Appendix~\ref{sec:literal_threshold}) were tagged as ``literal'', and from these images, we chose the top K images as literal candidates. From the non ``literal'' images, we chose the top K images with the highest ``search query-image'' score as Figurative candidates. AMT workers then annotated the relation between the figurative phrase and its Figurative and Literal candidates using the user interface (UI) seen in Figure~\ref{fig:image-task-ui},  Appendix~\ref{sec:annotation_ui}.

\subsubsection{Human Annotation}
\label{sec:human_annotation}
\begin{table}[t!]
%\centering
\small
% \resizebox{\columnwidth}{!}{
\begin{tabular}{@{}lcccccc@{}}
\toprule
   & Fig. & \begin{tabular}[x]{@{}c@{}}Fig.\\ Lit.\end{tabular} & Lit. & \begin{tabular}[x]{@{}c@{}}Part.\\ Lit.\end{tabular}  & None & \\ \midrule
\#             & 1970  &  751   &   434  &  487  & 2638 & 6697 \\ \midrule
3-maj & 100\% & 100\%  & 100\%  & 100\% & 100\% &  94\% \\
4-maj & 75.5\%  & 63\%   & 68\%   & 63\%  & 80\% &  70\% \\
5-maj & 45\%  & 33\%   & 35\%   & 38\%  & 53\% &  43\% \\ \midrule
Mean & 3.1   & 1.2    & 0.7    & 0.8   & 4    & - \\
Median  & 2     & 0      & 0      & 0     & 4    & - \\ \bottomrule
\end{tabular}
% }
\caption{IRFL statistics on 628 idioms. The majority of the images are related to the figurative phrase, most images are Figurative. (k-maj means k-majority)}
\label{tab:dataset-statistics}
\end{table}



\remove{
\begin{table}[t!]
%\centering
\small
% \resizebox{\columnwidth}{!}{
\begin{tabular}{@{}lcccccc@{}}
\toprule
Categories   & Fig. & \begin{tabular}[x]{@{}c@{}}Fig.\\ Literal\end{tabular} & Literal & \begin{tabular}[x]{@{}c@{}}Partial\\ Literal\end{tabular}  & None & Total\\ \midrule
Number             & 1970  &  751   &   434  &  487  & 2638 & 6697 \\ \midrule
3 majority & 100\% & 100\%  & 100\%  & 100\% & 100\% &  94\% \\
4 majority & 75.5\%  & 63\%   & 68\%   & 63\%  & 80\% &  70\% \\
5 majority & 45\%  & 33\%   & 35\%   & 38\%  & 53\% &  43\% \\ \midrule
Average & 3.1   & 1.2    & 0.7    & 0.8   & 4    & - \\
Median  & 2     & 0      & 0      & 0     & 4    & - \\ \bottomrule
\end{tabular}
% }
\caption{IRFL statistics on 628 idioms. The majority of the images have some relation to the figurative phrase. Most of the relations are Figurative.}
\label{tab:dataset-statistics}
\end{table}

}
We hired Amazon Mechanical Turk (AMT) workers to annotate the relation between each idiom and its candidate images using the user interface seen in Appendix~\ref{sec:annotation_ui} (Figure~\ref{fig:image-task-ui}). Five workers annotated each image in batches of five images per sample. They received a payment of \$$0.15$ per sample, which resulted in an average hourly wage of \$$15$. We created a qualification test\footnote{https://irfl-dataset.github.io/mturk/image/qualification} to select quality annotators and provided them with an interactive training platform\footnote{https://irfl-dataset.github.io/mturk/image/train} to understand the task and the different categories better. 

We split the annotation process into batches with an average size of $60$ idioms per batch. After each batch, we provided each worker with a personal profile page (Appendix~\ref{sec:annotation_ui}, Figure~\ref{fig:profile-page-ui}) to view their statistics and some examples where their choice was different from the majority of workers. 
%We also set up a leaderboard (Figure~\ref{fig:image-task-leaderboard}, Appendix~\ref{sec:annotation_ui})  that was updated after each batch to improve their competitiveness. 

Full annotation results and statistics are presented in Table \ref{tab:dataset-statistics}. 
%The nature of this task is very subjective. 
%We further discuss this in (Appendix~\ref{sec:annotation_task_discussion}).
% \dnote{why do you have discussion in the appendix? this doesn't make much sense}
Despite the subjective nature of the task, in $94\%$ of the instances, there was a majority of $3$ workers or more out of $5$ compared to a random chance of $29\%$. %This shows that different people can see the same connection most of the time. 









\subsection{Pipeline: Metaphors and Similes}
\label{sec:metaphors_and_similes}
% We collected $35$ textual metaphors and $142$ textual similes along with their definitions from online sources. Next, we used the metaphors and similes definitions as search queries to search for figurative and literal images. We manually annotated the resulting images into ``Figurative'' and ``Partial Literal'' categories. In total, we obtained $1107$ figurative images and $1816$ partial literal images for similes, and $333$ figurative images and $729$ literal images for metaphors. 
%\dnote{why a separate pipeline}\rnote{Now its good I think}\\
We collected $35$ textual metaphors and $142$ textual similes, compiled from online lists. Generating search queries from definitions (to find figurative images) is a central part of our pipeline for idioms (Section \ref{sec:idioms-collection}). However, idioms are fixed expressions, but metaphors and similes are much more flexible, as the number of possible comparisons between two things is vast. 

For this reason, we had to adapt our pipeline. For metaphors, we asked three expert annotators to agree upon definitions.
%
%\dnote{situation demonstrating}\rnote{Heart of gold -> A volunteer who distributes food in a soup kitchen}
%
%
For similes, we use the simile itself and the target concept with the shared property (``fast'') as search queries to find figurative images. For literal images that serve as distractors, we use the source and target without the shared property. In some cases, the target concept images are inadequate literal distractors (an image of a car might still be considered figurative for the simile "The car is as fast as a cheetah"). To solve this problem, we include the \emph{antonym} of the property (``A slow car'').
% Previous % In our approach to similes, we used both the simile itself and the target concept with the compared property as search queries to find figurative images. For literal images, we used the source concept, the target concept without the compared property, and the target concept with the antonym of the compared property. We specifically employed antonym images as distractors when a literal image of the target concept proved insufficient. For instance, an image of a gallon of milk doesn't serve as an effective distractor for the simile 'The milk is fresh as a daisy'.

%We collected $35$ textual metaphors and $142$ textual similes from online sources. Next, we used the metaphors and similes search queries to find figurative and literal images. 
\xhdr{Annotation} As the number of images was relatively small, we had two experts from our team manually annotate images. We obtained $1107$ figurative and $1816$ partial literal images for similes, $333$ figurative and $729$ partial literal for metaphors (the other categories were less relevant for the specific data generated by our pipeline). 

% We collected $35$ textual metaphors and $142$ textual similes along with their definitions from online sources. Next, we used the metaphors and similes definitions as search queries to search for figurative and literal images. We manually annotated the resulting images into ``Figurative'' and ``Partial Literal'' categories. In total, we obtained $1107$ figurative images and $1816$ partial literal images for similes, and $333$ figurative images and $729$ literal images for metaphors. 








% We verify the correctness of our dataset on different tasks in the human evaluation section \ref{sec:human_evaluation}. 

% We collected $628$ idioms from the MAGPIE corpus \citep{haagsma-etal-2020-magpie} of idiomatic expressions. The MAGPIE corpus contains $56,622$ crowdsourced potentially idiomatic expressions, covering $1,756$ unique idioms that appear in at least two of the following dictionaries: Wiktionary, Oxford Dictionary of English Idioms, and UsingEnglish.com. After collecting the idioms, we feed them into the pipeline as input. The first step is to collect the idioms' definitions from Wiktionary and Oxford dictionaries and construct search queries to find literal and figurative candidate images (\S\ref{sec:enriching_similes_and_idioms}). The next step is to select the best literal and figurative candidates for annotation using various heuristics and implementation decisions elaborated at (\S\ref{sec:choosing_images}). After collecting the best figurative and literal candidate images for the idioms, AMT workers annotated the different relations (Table~\ref{tab:relation-categories}) between each idiom and its candidate image, thus creating the IRFL dataset (\S\ref{sec:human_annotation}).

% We evaluate the end-to-end dataset generation, and the fact that humans achieve high agreement helps verify the end-to-end process's correctness. \\\\
% To collect metaphors and similes images, we collected $35$ textual metaphors and $142$ textual similes along with their definitions from the internet. Next, we used the metaphors and similes definitions as search queries and adapted the method used in (\S\ref{sec:choosing_images}) to search for figurative and literal images. We manually annotated the resulting images into ``Figurative'' and ``Literal'' categories. In total, we obtained $1107$ figurative images and $1816$ literal images for similes, and $333$ figurative images and $729$ literal images for metaphors. We verify the correctness of our dataset on different tasks in the human evaluation section \ref{sec:human_evaluation}. 



% Older
% Our goal is to generate the IRFL dataset of idioms, metaphors, and similes with matching figurative and literal images and evaluate the figurative understanding and preference of Vision and Language models. To collect figurative and literal images for idioms, we developed an automatic pipeline that takes a list of idioms as input and outputs figurative and literal candidate images. We collected $628$ idioms from the MAGPIE corpus \citep{haagsma-etal-2020-magpie} of idiomatic expressions. The MAGPIE corpus contains $56,622$ crowdsourced potentially idiomatic expressions, covering $1,756$ unique idioms that appear in at least two of the following dictionaries: Wiktionary, Oxford Dictionary of English Idioms, and UsingEnglish.com. After collecting the idioms, we then feed them into the pipeline as input. First, we collect the definitions of these idioms from Wiktionary and Oxford dictionaries and construct search queries to find possible literal and figurative images (\S\ref{sec:enriching_similes_and_idioms}). Next, we select the best literal and figurative candidates for annotation using various heuristics and implementation decisions elaborated at (\S\ref{sec:choosing_images}). AMT workers annotated the different relations between each idiom and its candidate images, creating the IRFL dataset (\S\ref{sec:human_annotation}). We evaluate the end-to-end dataset generation, and the fact that humans achieve high agreement helps to verify the correctness of the end-to-end process. The relation categories can be seen with corresponding explanations and images in Table~\ref{tab:relation-categories}.

% To collect metaphors and similes' images, we collected $35$ textual metaphors and $142$ textual similes from the internet. First, we collected metaphors and similes definitions and used them as search queries and adapted the method used to search images in (\ref{sec:choosing_images}). Next, we annotated these images into ``Figurative'' and ``Literal'' categories. In total, we obtained $1107$ figurative images and $1816$ literal images for similes, and $333$ figurative images and $729$ literal images for metaphors. We verify the correctness of our dataset on different tasks in the human evaluation section \ref{sec:human_evaluation}. 





%\subsection{Dataset Analysis}
%\label{sec:dataset_statistics}
%\input{sections/04E_dataset_statistics}







\section{Experiments}
\section{Experiments}
\label{sec:experiments}
\subsection{Experimental details}
\paragraph{Datasets} We use three benchmark datasets in CZSL problem, namely Clothing16K~\cite{zhang2022learning}, UT-Zappos50K~\cite{yu2014fine}, and C-GQA~\cite{naeem2021learning}. Clothing16K~\cite{zhang2022learning} contains different types of clothing (\eg, shirt, pants) with color attributes (\eg, white, black). UT-Zappos50K~\cite{yu2014fine} is a fine-grained dataset consisting of different kinds of shoes (\eg, sneakers, sandals) with texture attributes (\eg, leather, canvas). C-GQA~\cite{naeem2021learning} is a split built on top of Stanford GQA dataset~\cite{hudson2019gqa}, composed of extensive common attribute concepts (\eg, old, wet) and object concepts (\eg, dog, bus) in real life. We follow the common data splits of these three datasets 
(see~\cref{tab:data-splits}). 

\begin{table}[h]
    \centering
    \scalebox{0.54}{
    \begin{tabular}{cccccccccc}
        \toprule
         & \multicolumn{3}{c}{Composition} & \multicolumn{2}{c}{Train} & \multicolumn{2}{c}{Val} & \multicolumn{2}{c}{Test}  \\
         \cmidrule(lr){2-4} \cmidrule(lr){5-6} \cmidrule(lr){7-8} \cmidrule(lr){9-10}
         Datasets & $|\mathcal{A}|$ & $|\mathcal{O}|$ & $|\mathcal{A}|\times|\mathcal{O}|$ & $|\mathcal{C}_{s}|$ & $|\mathcal{X}|$ & $|\mathcal{C}_{s}|$ / $|\mathcal{C}_{u}|$ & $|\mathcal{X}|$ & $|\mathcal{C}_{s}|$ / $|\mathcal{C}_{u}|$ & $|\mathcal{X}|$
         \\ \midrule
        Clothing16K~\cite{zhang2022learning} & 9 & 8 & 72 & 18 & 7242 & 10 / 10 & 5515 & 9 / 8 & 3413\\
        UT-Zappos50K~\cite{yu2014fine} & 16 & 12 & 192 & 83 & 22998 & 15 / 15 & 3214 & 18 / 18 & 2914 \\
        C-GQA~\cite{naeem2021learning} & 413 & 674 & 278362 & 5592 & 26920 & 1252 / 1040 & 7280 & 888 / 923 & 5098 \\
        \bottomrule
    \end{tabular}}
    \caption{Summary of data split statistics.}
    \label{tab:data-splits}
    \vspace{-10pt}
\end{table}

\paragraph{Open-world setting} In addition to the standard closed-world setting, we also evaluate our model on the open-world setting~\cite{mancini2021open}, which is neglected by most previous works. The open-world setting considers all possible compositions, which requires a much larger testing space than the closed-world setting during inference. Taking UT-Zappos50K as an example (see~\cref{tab:data-splits}), the closed world only considers 36 compositions in the testing set while the open world considers total 192 compositions, in which $\sim$81\% are ignored under the standard closed-world setting.

\paragraph{Evaluation metrics} Since CZSL models have an inherent bias for seen compositions, we follow the generalized CZSL evaluation protocol~\cite{purushwalkam2019task}. To overcome the negative bias for seen compositions, we apply different calibration terms to unseen compositions and compute the corresponding top-1 accuracy of seen and unseen compositions, where a larger bias makes higher unseen accuracy and lower seen accuracy, and vice versa.  We treat seen accuracy as $x$-axis and unseen accuracy as $y$-axis to derive an unseen-seen accuracy curve. We can then compute the area under curve (AUC), the best harmonic mean, the best seen accuracy, and the best unseen accuracy from the curve. In our experiments, we report these four metrics for evaluation, among which AUC is the most representative and stable metric for measuring CZSL model performance. \hsz{Note that the attribute accuracy or the object accuracy alone does not reflect CZSL performance, because the individual accuracy on attribute or object does not necessarily decide the accuracy of their composition.}

\paragraph{Implementation details}
We use a frozen ViT-B-16~\cite{dosovitskiy2020vit} backbone pretrained with DINO~\cite{caron2021emerging} on ImageNet~\cite{deng2009imagenet} in a self-supervised manner as our visual feature extractor. The ViT-B-16 outputs contain 197 tokens (1 \texttt{[CLS]} and 196 patch tokens) of 768 dimensions. For three attention disentangler modules, we implement one-layer multi-head attention framework following~\cite{vaswani2017attention}, changing the single input to paired inputs for cross-attentions. The embedders $\pi_a$, $\pi_c$, $\pi_o$ are the two-layer MLPs following the previous works~\cite{mancini2021open, zhang2022learning}, projecting the 768-dimension visual features to 300-dimension word embedding space. The word embedding prototypes are initialized with word2vec~\cite{mikolov2013distributed} for all datasets and learnable during training. The composition function $\psi$ is one linear layer. We train our model using Adam optimizer~\cite{kingma2015adam} with a learning rate of $5\times 10^{-6}$ for UT-Zappos50K and Clothing16K, and $5\times 10^{-5}$ for C-GQA. All models are trained with 128 batch size for 300 epochs.

\begin{table*}[t]
    \centering
    \scalebox{0.75}{
    \begin{tabular}{l>{\columncolor{tabcolor}}cccccc>{\columncolor{tabcolor}}cccccc>{\columncolor{tabcolor}}cccccc}
        \toprule
         Closed-world & \multicolumn{6}{c}{Clothing16K} & \multicolumn{6}{c}{UT-Zappos50K} & \multicolumn{6}{c}{C-GQA} \\
         \cmidrule(lr){2-7} \cmidrule(lr){8-13} \cmidrule(lr){14-19}
         Models & AUC & HM & Seen & Unseen & Attr & Obj & AUC & HM & Seen & Unseen & Attr & Obj & AUC & HM & Seen & Unseen & Attr & Obj \\
         \midrule
         SymNet~\cite{li2020symmetry} & 78.8 & 79.3 & 98.0 & 85.1 & 75.6 & 84.1 & 32.6 & 45.6 & 60.6 & 68.6 & 48.2 & 77.0 & 3.1 & 13.5 & 30.9 & 13.3 & 11.4 & 34.6 \\
         CompCos~\cite{mancini2021open} & 90.3 & 87.2 & 98.5 & 96.8 & \textbf{90.2} & 91.8 & 31.8 & 48.1 & 58.8 & 63.8 & 45.5 & 72.4 & 2.9 & 12.8 & 30.7 & 12.2 & 10.4 & 33.9 \\
         GraphEmb~\cite{naeem2021learning} & 89.2 & 84.2 & 98.0 & 97.4 & 90.0 & 93.1 & 34.5 & 48.5 & 61.6 & \textbf{70.0} & \textbf{50.8} & \textbf{77.1} & 3.8 & 15.0 & 32.3 & 14.9 & 13.8 & 33.2 \\
         Co-CGE~\cite{mancini2022learning} & 88.3 & 87.9 & 98.5 & 94.7 & 87.4 & 91.4 & 30.8 & 44.6 & 60.9 & 62.6 & 46.0 & 73.5 & 3.6 & 14.7 & 31.6 & 14.3 & 12.6 & 34.6 \\
         SCEN~\cite{li2022siamese} & 78.8 & 78.5 & 98.0 & 89.6 & 81.2 & 85.4 & 30.9 & 46.7 & \textbf{65.7} & 62.9 & 44.0 & 74.4 & 3.5 & 14.6 & 31.7 & 13.4 & 10.7 & 31.4 \\ 
         IVR~\cite{zhang2022learning} & 90.6 & 86.6 & \textbf{99.0} & 97.0 & 89.3 & \textbf{93.6} & 34.3 & 49.2 & 61.5 & 68.1 & 48.4 & 74.6 & 2.2 & 10.9 & 27.3 & 10.0 & 10.3 & \textbf{37.5} \\
         OADis~\cite{Saini_2022_CVPR} & 88.4 & 86.1 & 97.7 & 94.2 & 84.9 & 93.1 & 32.6 & 46.9 & 60.7 & 68.8 & 49.3 & 76.9 & 3.8 & 14.7 & 33.4 & 14.3 & 8.9 & 36.3 \\
         \midrule
         \framework (ours) & \textbf{92.4} & \textbf{88.7} & 98.2 & \textbf{97.7} & \textbf{90.2} & \textbf{93.6} & \textbf{35.1} & \textbf{51.1} & 63.0 & 64.3 & 46.3 & 74.0 & \textbf{5.2} & \textbf{18.0} & \textbf{35.0} & \textbf{17.7} & \textbf{16.8} & 32.3\\ 
        \bottomrule
    \end{tabular}}
    \caption{Closed-world results on three datasets. We report the area under curve (AUC), the best harmonic mean (HM), the best seen accuracy (Seen), the best unseen accuracy (Unseen), the attribute accuracy (Attr), and the object accuracy (Obj) of the unseen-seen accuracy curve under the closed world-setting. AUC is the core CZSL metric. All models use the same DINO ViT-B-16 backbone.} 
    \label{tab:cw-results}
\end{table*}

\begin{table*}[t]
    \centering
    \scalebox{0.75}{
    \begin{tabular}{l>{\columncolor{tabcolor}}cccccc>{\columncolor{tabcolor}}cccccc>{\columncolor{tabcolor}}cccccc}
        \toprule
         Open-world & \multicolumn{6}{c}{Clothing16K} & \multicolumn{6}{c}{UT-Zappos50K} & \multicolumn{6}{c}{C-GQA} \\
         \cmidrule(lr){2-7} \cmidrule(lr){8-13} \cmidrule(lr){14-19}
         Models & AUC & HM & Seen & Unseen & Attr & Obj  & AUC & HM & Seen & Unseen & Attr & Obj & AUC & HM & Seen & Unseen & Attr & Obj  \\
         \midrule
         SymNet~\cite{li2020symmetry} & 57.4 & 68.3 & 98.2 & 60.7 & 57.6 & 81.2 & 25.0 & 40.6 & 60.4 & 51.0 & 38.2 & \textbf{75.0} & 0.77 & 4.9 & 30.1 & 3.2 & 18.4 & 37.5 \\
         CompCos~\cite{mancini2021open} & 64.1 & 70.8 & 98.2 & 69.8 & 71.7 & 83.7 & 20.7 & 36.0 & 58.1 & 46.0 & 36.4 & 71.1 & 0.72 & 4.3 & 32.8 & 2.8 & 15.1 & 37.8 \\
         GraphEmb~\cite{naeem2021learning} & 62.0 & 68.3 & 98.5 & 69.7 & 71.8 & 82.4 & 23.5 & 40.0 & 60.6 & 47.0 & 37.1 & 69.3 & 0.81 & 4.8 & 32.7 & 3.2 & 17.2 & 36.7 \\
         Co-CGE~\cite{mancini2022learning} & 59.3 & 69.2 & 98.7 & 63.8 & 68.5 & 76.2 & 22.0 & 40.3 & 57.7 & 43.4 & 33.9 & 67.2 & 0.48 & 3.3 & 31.1 & 2.1 & 15.5 & 35.7 \\
         SCEN~\cite{li2022siamese}& 53.7 & 61.5 & 96.7 & 62.3 & 63.6 & 79.1 & 22.5 & 38.0 & \textbf{64.8} & 47.5 & 34.9 & 73.3 & 0.34 & 2.5 & 29.5 & 1.5 & 14.8 & 32.3 \\ 
         IVR~\cite{zhang2022learning} & 63.6 & 72.0 & 98.7 & 69.0 & 70.3 & 84.8 & 25.3 & 42.3 & 60.7 & 50.0 & 38.4 & 71.4 & 0.94 & 5.7 & 30.6 & 4.0 & 16.9 & 36.5 \\
         OADis~\cite{Saini_2022_CVPR} & 53.4 & 63.2 & 98.0 & 58.6 & 57.3 & \textbf{85.4} & 25.3 & 41.6 & 58.7 & \textbf{53.9} & \textbf{40.3} & 74.7 & 0.71 & 4.2 & 33.0 & 2.6 & 14.6 & \textbf{39.7} \\
         \midrule
         \framework (ours) & \textbf{68.0} & \textbf{74.2} & \textbf{99.0} & \textbf{73.1} & \textbf{75.0} & 84.5 & \textbf{27.1} & \textbf{44.8} & 62.4 & 50.7 & 39.9 & 71.4 & \textbf{1.42} & \textbf{7.6} & \textbf{35.1} & \textbf{4.8} & \textbf{22.4} & 35.6 \\ 
        \bottomrule
    \end{tabular}}
    \caption{Open-world results on three datasets. Different from~\cref{tab:cw-results}, open-world setting considers all possible compositions in testing.} 
    \label{tab:ow-results}
\end{table*}

\subsection{Comparison}
\hsznew{To ensure a fair comparison and demonstrate that our improvement over baseline models is not merely by ViT, we adopt ViT backbone to state-of-the-art CZSL models and \emph{re-train} all models.} We compare our method with them: \hsznew{(1)~OADis~\cite{Saini_2022_CVPR} disentangles attribute and object features from spatial convolutional maps;} (2)~SymNet~\cite{li2020symmetry} introduces the symmetry principle of attribute-object transformation and group theory as training objectives; (3)~CompCos~\cite{mancini2021open} extends CZSL to an open-world setting considering all possible compositions during inference, proposing a feasibility score based on data statistics to remove unfeasible compositions; (4)~GraphEmb~\cite{naeem2021learning} and Co-CGE~\cite{mancini2022learning} propose to use graph convolutional networks (GCN) to represent attribute-object relationships and compositions; (5)~SCEN~\cite{li2022siamese} projects visual features to a Siamese contrastive space to capture concept prototypes, and introduces complex state transition module to produce virtual compositions; (6)~IVR~\cite{zhang2022learning} proposes to disentangle visual features into concept-invariant domains from a perspective of domain generalization, by masking specific channels of visual features. 

\paragraph{Closed-world evaluation} In~\cref{tab:cw-results}, we compare our \framework model with the state-of-the-art methods. \framework consistently outperforms others by a significant margin. \framework increases the core metric AUC by 1.8 on Clothing16K, 0.6 on UT-Zappos50K, and 1.4 on C-GQA ($\sim$37\% relatively). Similarly, \framework increases the best harmonic mean (HM) by 0.8\% on Clothing16K, 1.9\% on UT-Zappos50K, and 3.0\% on C-GQA. We notice that SymNet~\cite{li2020symmetry} and SCEN~\cite{li2022siamese} perform badly on Clothing16K. The reason might be that not learning concept prototypes harms the word embedding expressivity on small-scale concepts. We also notice that IVR~\cite{zhang2022learning} performs very well on curated datasets Clothing16K and UT-Zappos50K but badly on larger-scale real-world dataset C-GQA. We hypothesize ideal concept-invariant domains might be difficult to learn from natural images and large-scale concepts of C-GQA. In contrast, our \framework model achieves state-of-the-art performance on all datasets.

\paragraph{Open-world evaluation} In~\cref{tab:ow-results}, we consider the open-world setting to compare our \framework with other methods. Likewise, \framework also performs the best among all methods under open-world setting. \framework increases AUC by 3.9 on Clothing16K, 1.8 on UT-Zappos50K, and 0.48 on C-GQA ($\sim$51\% relatively). \framework also increases the best harmonic mean (HM) by 2.2\% on Clothing16K, 2.5\% on UT-Zappos50K, and 1.9\% on C-GQA ($\sim$33\% relatively). From the above results, \framework surpasses others by a larger margin on open-world AUC and HM than closed-world ones, indicating \framework maintains utmost efficiency when turning from the closed world to the open world. It is worth mentioning that \framework does not apply any special operations (\eg, feasibility masking~\cite{mancini2021open}) for the open world and deals with the two settings in exactly the same way. 
IVR~\cite{zhang2022learning} keeps its performance to a great extent but still lags behind our method significantly.

\begin{table*}[t]
\begin{minipage}[t]{0.44\linewidth}
    \centering
    \scalebox{0.72}{
    \begin{tabular}{lcccc>{\columncolor{tabcolor}}cccc}
        \toprule
         & CA & AA & OA & Reg & AUC & HM & Seen & Unseen\\
         \midrule
         (0) & \xmark & \xmark & \xmark & \xmark & 23.8 & 41.1 & 59.0 & 48.9 \\
         (1) & self & \xmark & \xmark & \xmark & 25.3 & 42.3 & 61.1 & 49.9 \\
         (2) & self & self & self & \xmark & 26.7 & 44.6 & 61.9 & 49.8 \\
         (3) & self & cross & cross & \xmark & 26.9 & 44.5 & \textbf{63.4} &48.7 \\
         (4) & self & cross & cross & \cmark &  \textbf{27.1} & \textbf{44.8} & 62.4 & \textbf{50.7}\\
         
        \bottomrule
    \end{tabular}}
    \caption{Ablate the components in \framework on open-world UT-Zappos50K. CA, AA, and OA denote composition, attribute, and object attention. Reg denotes the regularization term. We test self- or cross-attention for AA and OA.} 
    \label{tab:model-ab}
\end{minipage}
\hspace{2mm}
\begin{minipage}[t]{0.54\textwidth}
\centering
    \scalebox{0.68}{
    \begin{tabular}{ll>{\columncolor{tabcolor}}cccc>{\columncolor{tabcolor}}cccc}
        \toprule
        & & \multicolumn{4}{c}{C-GQA} & \multicolumn{4}{c}{Clothing16K} \\
        \cmidrule(lr){3-6} \cmidrule(lr){7-10}
        & Inference formulation  & AUC & HM & Seen & Unseen & AUC & HM & Seen & Unseen\\
        \midrule
        (0) & $p(c)$ & 4.6 & 16.8 & \textbf{35.1} & 16.0 & \textbf{92.4} & \textbf{88.8} & \textbf{98.2} & \textbf{97.7} \\
        (1) & $p(a) \cdot p(o)$ &  4.0 & 15.8 & 31.4 & 15.1 & 57.3 & 66.3 & 96.7 & 63.1\\
        (2) & $p(c) + p(a) \cdot p(o)$ & \textbf{5.2} & \textbf{18.0} & 35.0 & \textbf{17.7} & 90.4 & 85.9 & 98.2 & 97.0\\
        (3) & $p(c) + \beta \cdot p(a) \cdot p(o)$ & \textbf{5.2} & \textbf{18.0} & 35.0 & \textbf{17.7} & \textbf{92.4} & 88.7 & \textbf{98.2} & \textbf{97.7} \\
        \bottomrule
    \end{tabular}}
    \caption{Results on closed-world Clothing16K and C-GQA using different inference formulations. Rows (0)-(2) respectively represents the cases when $\beta=0.0$, $\beta=+\infty$, and $\beta=1.0$. Row (3) is our inference formulation, which applies an $\beta$ optimized on the validation set.} 
    \label{tab:eval-ab}
\end{minipage}
\end{table*}

\subsection{Ablation study}
\paragraph{Backbone: ResNet \textit{vs} ViT}
\hsznew{Our work leverages ViT as the default backbone to excavate more high-level sub-space information, while ResNet18 is the most common backbone in previous works. 
In~\Cref{tab:backbone}, we compare our \framework to OADis~\cite{Saini_2022_CVPR}  with both backbones. Our \framework performs similarly to OADis with ResNet18, but outperforms it significantly with ViT. Additionally, we present an ablation study on different components of our method with the ResNet18 backbone in the Appendix. These experiments indicate that our model benefits from ViT and all components of our method are effective regardless of the backbone.}

\begin{table}[h]
    \centering
    \scalebox{0.75}{
    \begin{tabular}{ll>{\columncolor{tabcolor}}cc>{\columncolor{tabcolor}}cc>{\columncolor{tabcolor}}cc}
        \toprule
          \multicolumn{2}{l}{Closed-world} & \multicolumn{2}{c}{Clothing16K} & \multicolumn{2}{c}{UT-Zappos50K} & \multicolumn{2}{c}{C-GQA} \\
         \cmidrule(lr){3-4} \cmidrule(lr){5-6} \cmidrule(lr){7-8}
         Backbone & Models & AUC & HM & AUC & HM & AUC & HM \\
         \midrule
         \multirow{2}{*}{ResNet18} & OADis~\cite{Saini_2022_CVPR} & 85.5 & 84.7 & \textbf{30.0} & 44.4 & \textbf{3.1} & 13.6 \\
          & \framework (ours) & \textbf{87.2} & \textbf{85.1}  & 29.5 & \textbf{47.0} & \textbf{3.1} & \textbf{13.7} \\
         \midrule
         \multirow{2}{*}{ViT-B-16} & OADis~\cite{Saini_2022_CVPR} & 88.4 & 86.1  & 32.6 & 46.9 & 3.8 & 14.7 \\
          & \framework (ours) & \textbf{92.4} & \textbf{88.7}  & \textbf{35.1} & \textbf{51.1} & \textbf{5.2} & \textbf{18.0} \\
        \bottomrule
    \end{tabular}}
    \caption{Compare \framework and OADis~\cite{Saini_2022_CVPR} with ResNet18 and ViT.} 
    \label{tab:backbone}
    \vspace{-5pt}
\end{table}

\paragraph{Different parts of \framework} We evaluate the effectiveness of attention disentanglers (composition, attribute, and object attention) and the regularization term in our model. We report the ablation study results on the open-world UT-Zappos50K in~\cref{tab:model-ab}. Rows~(0)-(2) show attention disentanglers can significantly improve the performance. Rows~(2)-(3) show that cross-attention learns disentangled concepts better than self-attention for AA and OA. Rows~(3)-(4) show the regularization term can further benefit the visual disentanglement, improving the unseen accuracy and overall AUC.

\paragraph{Inference formulation} We also investigate the effect of our inference formulation $p(c) + \beta \cdot p(a) \cdot p(o)$ in~\cref{tab:eval-ab}. We report the results with extreme values of $\beta$, \ie, $\beta=0.0$ and $\beta=1.0$. Note that $\beta=0.0$ means only using composition probability for prediction. In addition, we also test the performance only using the product of attribute and object probabilities $p(a) \cdot p(o)$. We can observe that the best fixed $\beta$ value is unfixed among datasets. For example, $\beta=1.0$ gives the highest AUC for C-GQA in row (2) while $\beta=0.0$ for Clothing16K in row (0). In contrast, our validated $\beta$ consistently gives the best inference results for both datasets. Another observation on C-GQA is that $p(a) \cdot p(o)$ alone is not a good prediction, but adding it to $p(c)$ can increase the unseen accuracy. This indicates that the disentangled attribute prediction $p(a)$ and object prediction $p(o)$ indeed enhance the unseen generalization for CZSL problem.

\paragraph{Effect of regularization term} We propose an EMD-adapted regularization term at the attention level to force attentions to disentangle the concept of interest. We also investigate the effect of applying the regularization term at the feature level. Specifically, 
we compare our EMD-based distance to the cosine and euclidean feature distances. The results on open-world UT-Zappos50K are shown in~\cref{tab:reg-ab}. Our EMD-based regularization outperforms other distance forms, because our attention-level EMD distance considers token-wise similarity capturing the specific concept-related attention responses.
\begin{table}[h]
   \centering
   \setlength{\tabcolsep}{20pt}
    \scalebox{0.7}{
    \begin{tabular}{l>{\columncolor{tabcolor}}cccc}
        \toprule
        Reg & AUC & HM & Seen & Unseen \\
        \midrule
        Cosine & 26.8 & 44.7 & \textbf{63.0} & 48.6 \\
        Euclidean & 26.2 & 44.3 & 62.6 & 47.5 \\
        Ours (EMD) & \textbf{27.1} & \textbf{44.8} & 62.4 & \textbf{50.7}\\
        \bottomrule
    \end{tabular}}
    \caption{Comparison of different regularization terms on open-world UT-Zappos50K.} 
    \label{tab:reg-ab} 
    \vspace{-5pt}
\end{table}

\subsection{Qualitative analysis}
Visual disentanglement in feature space is hard to visualize~\cite{Saini_2022_CVPR}. Inspired by previous work attempts~\cite{Saini_2022_CVPR,zhang2022learning,li2020symmetry,nagarajan2018attributes}, we conduct qualitative analysis of image and text retrieval to show how our \framework model correlates the visual image and the concept composition. In addition, to further validate \framework is efficient to disentangle visual concepts, we conduct unseen-to-seen image retrieval based on their visual concept features extracted by attribute and object attentions.

\begin{figure*}[t]
     \centering
     \begin{subfigure}[b]{0.32\textwidth}
         \centering
         \includegraphics[width=0.92\linewidth]{images/txt2img.pdf}
    \caption{Top-5 text-to-image retrieval.}
    \label{fig:wrd2img}
     \end{subfigure}
     \hfill
     \begin{subfigure}[b]{0.35\textwidth}
    \centering
    \includegraphics[width=\linewidth]{images/img2wrd.pdf}
    \caption{Top-5 image-to-text retrieval.}
    \label{fig:img2wrd}
     \end{subfigure}
     \hfill
     \begin{subfigure}[b]{0.32\textwidth}
    \centering
    \includegraphics[width=0.9\linewidth]{images/retrieve_visual.pdf}
    \caption{Top-5 visual concept retrieval.}
    \label{fig:concept-retrieve}
     \end{subfigure}
    \caption{\hsznew{Qualitative analysis. (a) In the last row of ``suede sandals", the wrong image (red box) is ``fake leather sandals". (b) Each image has the ground-truth label (black text) and 5 retrieval results (colored text), in which the green text is the correct prediction. (c) We retrieve images sharing the same visual concepts by their visual concept features for unseen images of ``yellow skirt" and ``pink pants".}}
    \label{fig:qualitative}
    \vspace{-4pt}
\end{figure*}

\paragraph{Image and text retrieval}
We first consider text-to-image retrieval. Given a text composition, \eg, ``leather heels", we embed it and retrieve the top-5 closest visual features based on the feature distance. We display four text compositions of the different objects sharing the same attributes and vice versa in~\cref{fig:wrd2img}. We can observe that the retrieved images are correct in most of the cases. One exception is when retrieving ``suede sandals", the third closest image is ``fake leather sandals". Although ``suede sandals" and ``fake leather sandals" are not the same composition, they are quite visually similar. We then consider image-to-text retrieval, shown in~\cref{fig:img2wrd}. Given an image, \eg, the image of a ``brown zebra", we extract its visual feature and retrieve the top-5 closest text composition embeddings. It is difficult to retrieve the ground-truth label in the top-1 closest text composition, but all top-5 results are all semantically related to the image. We take the image of ``blond person" (row 3, col 2) as an example. Although the text composition ``blond person" is not retrieved in the top-5 matches, the retrieved results ``white shirts", ``white outfit", ``white shorts", ``white pants", and ``young girl" are all reasonable and actually present in the image. Image and text retrieval experiments validate that our \framework efficiently projects visual features and word embeddings into a uniform space.
\vspace{-2pt}

\paragraph{Visual concept retrieval}
Because the attribute and the object are visually coupled in an image to a high degree of entanglement, it is challenging to visualize the disentanglement in feature space~\cite{Saini_2022_CVPR}. Saini \etal~\cite{Saini_2022_CVPR} retrieve single attribute or object text from test images. However, this process is the same as multi-label classification and insufficient to validate that disentangled visual concepts are learned from images. \hsz{Based on the disentanglement ability of ADE, we construct a visual concept retrieval experiment to investigate the distances between visual concept features, \ie, the embedded attribute feature $\pi_a(v_a)$ and the embedded object feature $\pi_o(v_o)$, extracted from different images. Prior to our work, no existing models can do so, because none of them extracts concept-exclusive features like ADE.} The results are shown in~\cref{fig:concept-retrieve}. We first extract attribute features and object features from all seen images. Given an unseen image, we retrieve the top-5 closest images by measuring the feature distance between the attribute feature of the given image and that of all seen images, and the same goes for the object feature. For the image of ``yellow skirt", all retrieval results for the visual concept ``yellow" are all  ``yellow \texttt{[OBJ]}", and all retrieval results for ``skirt" are ``\texttt{[ATTR]} skirt". For the ``pink pants" image, our model also perfectly retrieves the visual concepts, \ie, the attribute ``pink" and the object ``pants". Our experimental results demonstrate that our \framework model is effective to disentangle visual concepts from seen compositions and combine learned concept knowledge into unseen compositions.

\section{Related Work}
\begin{figure*}
    \centering
    \includegraphics[clip, trim = 0in 8.5in 0in 0in, width=\textwidth]{figures/ForwardModel.pdf}
    \caption{\textbf{Hologram optimization framework.}
    This figure illustrates the three key components of the simultaneous color optimization framework: an SLM model, a propagation model, and a perceptual loss function. The SLM model maps voltage values to a complex field using a learned cross-talk kernel and a linear lookup table. The complex wavefront from the SLM is then propagated to the sensor plane using a modified version of the model proposed by \citet{Gopakumar2021UnfilteredDisplays}, which separates the zeroth and first diffraction orders and combines them through a U-Net. The output is then fed into the perceptual loss function, and gradients are calculated using Pytorch's autograd implementation. The SLM voltages are then updated using these gradients. Rubik's cube source image by Iwan Gabovitch (CC BY 2.0).}  
    \label{fig:ForwardModel}
\end{figure*}

\section{Related Works}
\label{RelatedWorks}

\paragraph{Field Sequential Color}
The vast majority of color holographic displays use field sequential color in which the SLM is sequentially illuminated by red, green, and blue sources while the SLM pattern is updated accordingly \cite{maimone2017holographic, jang2018holographic, Chakravarthula2019WirtingerDisplays, Chakravarthula2020LearnedDisplays, chakravarthula2022pupil,  Peng2020NeuralTraining, Peng2021Speckle-freeCalibration,Choi2021NeuralDisplays, choi2021optimizing, shi2021towards, yang2022diffraction, li2016holographic}. Field sequential color is effective at producing full color holograms but reduces framerate by a factor of $3\times$. This is a challenge for LCoS SLMs where framerate is severely limited by the LC response time \cite{zhang2014fundamentals}. Although, SLMs based on MEMS technology can run at high framerates in the kilohertz range \cite{MEMS}, so far these modulators are maximum 4-bit displays, with most being binary \cite{Choi2022Time-multiplexedModulators, kim2022accommodative, lee2022high}. Even with high framerate modulators, it may be worthwhile to maintain the full temporal bandwidth, since the extra bandwidth can be used to address other holography limitations. For example, speckle can be reduced through temporal averaging \cite{Choi2022Time-multiplexedModulators, kim2022accommodative, lee2022high}, and limited etendue can be mitigated through pupil scanning \cite{jang2018holographic, kim2022holographic}.

\paragraph{Spatial Multiplexing} An alternate approach is spatial multiplexing, which maintains the native SLM framerate by using different regions of the SLM for each color. Most prior works in this area use three separate SLMs and an array of optics to combine the wavefronts~\cite{Yaras2009Real-timeIllumination, Shiraki2009SimplifiedLinks, Nakayama2010Real-timePanels}. 
Although this method produces high quality holograms, the resulting systems are  bulky, expensive, and require precise alignment, making them poorly suited for near-eye displays. Spatial multiplexing can also be implemented with a single SLM split into sub-regions~\cite{Makowski2011SimpleColor, Makowski2009ExperimentalDisplay}; while less expensive, this approach still requires bulky combining optics and sacrifices space-bandwidth product (SBP), also known as etendue. Etendue is already a limiting factor in holographic displays \cite{kuo2020high}, and further reduction is undesirable as it limits the range of viewing angles or display field-of-view.

\paragraph{Frequency Multiplexing} Rather than split the physical extent of the SLM into regions, frequency multiplexing assigns each color a different region in the frequency domain, and the colors are separated with a physical color filter at the Fourier plane of a 4$f$ system \cite{Makowski2010ColorHolograms, Lin17, Lin19}. A variation on this idea uses different angles of illumination for each color so that the physical filter in Fourier space is not color-specific \cite{Xue:14}. Frequency multiplexing can also be implemented with white light illumination, which reduces speckle noise at the cost of resolution \cite{Kozacki16, yang2019full}. However, all of these techniques involve filtering in Fourier space, which sacrifices system etendue and requires a bulky 4$f$ system.

\paragraph{Depth Division and Bit Division for Simultaneous Color} The prior methods most closely related to our work also use simultaneous RGB illumination over the SLM, maintain the full SLM etendue, and don't require a bulky 4$f$ system \cite{Pi2022ReviewDisplay}. We refer to the first method as \textit{depth division multiplexing} which takes advantage of the ambiguity between color and propagation distance (explained in detail in Sec. \ref{sec:color-depth-ambiguity}) and assigns each color a different depth~\cite{Makowski2008ColorfulHologram, Makowski2010ColorHolograms}. After optimizing with a single color for the correct multiplane image, the authors show they can form a full color 2D hologram when illuminating in RGB. However,  this approach does not account for wavelength dependence of the SLM response, and since it explicitly defines content at multiple planes, it translates poorly to 3D.

Another similar approach is \textit{bit division multiplexing}, which takes advantage of the extended phase range of LCoS SLMs \cite{Jesacher2014ColourRange}. The authors calibrate an SLM lookup-table consisting of phase-value triplets (for RGB) as a function of digital SLM input, and they note that SLMs with extended phase range (up to $10\pi$) can create substantial diversity in the calibrated phase triplets. After pre-optimizing a phase pattern for each color separately, the lookup-table is used on a per-pixel basis to find the digital input that best matches the desired phase for all colors. In our approach, we also use an extended SLM phase range for the same reason, but rather than using a two-step process, we directly optimize the output hologram. This flexible framework also allows us to incorporate a perceptual loss function to further improve perceived image quality.

\paragraph{Algorithms for Hologram Generation}
Our work builds on a body of literature applying iterative optimization algorithms to holographic displays. Perhaps most popular is the Gerchberg-Saxton (GS) method \cite{gerchberg1972practical}, which is effective and easy to implement, but does not have an explicitly defined loss function, making it challenging to adapt to specific applications. \citet{zhang20173d} and \citet{Chakravarthula2019WirtingerDisplays} were the first to explicitly formulate the hologram generation problem in an optimization framework. This framework has been very powerful, enabling custom loss functions \cite{Choi2022Time-multiplexedModulators} and flexible adaptation to new optical configurations \cite{choi2021optimizing, Gopakumar2021UnfilteredDisplays}. In particular, perceptual loss functions can improve the perceived image by taking aspects of human vision into account, such as human visual acuity \cite{kuo2020high}, foveated vision \cite{walton2022metameric}, and sensitivity to spatial frequencies during accommodation \cite{kim2022accommodative}. Like these prior works, we use an optimization-based framework which we adapt to account for the wavelength-dependence of the SLM; this also enables our  new perceptual loss function for color, which is based on visual acuity difference between chrominance and luminance channels.


\paragraph{Camera-Calibration of Holographic Displays}
When the model used for hologram generation does not match the physical system, deviations cause artifacts in the experimental holograms. Recently, several papers have addressed this issue using measurements from a camera in the system for calibration.  \citet{Peng2020NeuralTraining} proposed using feedback from the camera to update the SLM pattern for a particular image; although a single image can be improved, it does not extend to new content.
A more flexible approach uses pairs of SLM patterns and camera captures to estimate the learnable parameters in a model, which is then used for offline hologram generation. Learnable parameters can be physically-based \cite{Peng2020NeuralTraining, kavakli2022learned, Chakrabarti2016LearningBack-propagation}, black box CNNs \cite{Choi2021NeuralDisplays}, or a combination of both \cite{Choi2022Time-multiplexedModulators}. The choice of learnable parameters effects the ability of the model to match the physical system; we introduce a new parameter for modeling SLM cross talk and tailor the CNN architecture for higher diffraction orders from the SLM.  

\section{Limitations and Conclusions}
We introduced IRFL, a dataset of Figurative and Literal images for idioms, metaphors, and similes. We introduced two novel tasks as a benchmark for figurative understanding. Our tasks are easy for humans and challenging for state-of-the-art models. We provided an extensive evaluation of the dataset. \\\\Our pipeline has several limitations. Future work can focus on improving the pipeline, in particular, improving the quality of figurative candidates for idioms, and increasing the automation for metaphors and similes. We hope that the IRFL dataset and benchmark will drive the development of models that will better understand figurative language.

\section{Acknowledgements}
\section{\new{Acknowledgements}}
Animesh and Niloy were partially funded by the European Union’s Horizon 2020 research and innovation programme under the Marie Skłodowska-Curie grant agreement No.~956585. This research has been partly supported by MetaAI and the UCL AI Centre. Finally, Animesh thanks  \href{https://ajolicoeur.wordpress.com/about/}{Alexia Jolicoeur-Martineau} for the helpful and insightful guidance on diffusion  models.

% Entries for the entire Anthology, followed by custom entries
\bibliography{anthology,custom}
\bibliographystyle{acl_natbib}

\appendix
\section{Appendix}
% \subsection{Dataset Supplementary Materials}
% \label{sec:dataset_supplementary_materials}
% \begin{enumerate}
  \item Dataset documentation, metadata, and download instructions are available at \url{https://irfl-dataset.github.io/download}.
  \item Intended uses: We hope researchers will use our benchmarks to evaluate Vision and Language models. We also hope that our pipeline and dataset will inspire future work on creating extensive multimodal datasets of other figures of speech.
  \item Author statement: We bear all responsibility in case of violation of rights in using our benchmark. 
  \item Licenses: Code is licensed under the MIT license \url{https://opensource.org/licenses/MIT}. Dataset is licensed under CC-BY 4.0 license \url{https://creativecommons.org/licenses/by/4.0/legalcode}.
  \item Hosting \& preservation: our website is deployed, and all data is accessible and available. We encourage researchers to send us model predictions on the created test sets. We will update a model and players leaderboard with these results periodically. 
  \item Code repository: \url{https://github.com/irfl-dataset/irfl}
\end{enumerate}

% \subsection{OCR filters}
% \label{sec:OCR_filters}
% \input{sections/100C_OCR_filter.tex}
\subsection{Annotation UI}
\label{sec:annotation_ui}
\begin{figure}[ht]
\begin{center}
\includegraphics[width=7.5cm,height=8cm, keepaspectratio]{figures/steps-tree.png}
\end{center}
\caption{The scheme tree that was provided to annotators to illustrate the correct thinking process.}
\label{fig:image-task-tree}
\end{figure}

\begin{figure}[h!]
\begin{center}
\includegraphics[width=6.5cm,height=6.5cm]{figures/Figure_6_image_task_ui.JPG}
\end{center}
\caption{The UI used to annotate the automatic pipeline candidate images. Annotators need to choose the category that best describes the relationship between the idiom and the image.}
\label{fig:image-task-ui}
\end{figure}


\begin{figure}[!h]
\begin{center}
\includegraphics[width=7cm, keepaspectratio]{figures/profile-page-example.PNG}
\end{center}
\caption{An example of the profile page includes the worker's statistics and some handily picked examples where his choice was distant from a majority of four workers.}
\label{fig:profile-page-ui}
\end{figure}


\begin{figure}[!h]
\begin{center}
\includegraphics[width=7cm, keepaspectratio]{figures/leaderboard-example.PNG}
\end{center}
\caption{An example of the profile page includes the worker's statistics and some handily picked examples where his choice was distant from a majority of four workers.}
\label{fig:image-task-leaderboard}
\end{figure}

\newpage
\subsection{Documents Filter}
\label{sec:documents_filter}
In an effort to minimize the number of images dominated by text, we filtered out images containing more than a few words, which accounted for $15\%$ of the total. Despite this, certain images like documents, books, and contracts managed to bypass our OCR-based filters, representing $2\%$ of the total images. To address this issue, we developed a filter using the ViLT model \citep{kim2021vilt}. This filter calculates an image's matching score with the prompts "a document", "a page of a book", or "a contract" and removes it if the total score surpasses a set "document" threshold. To find this threshold, we conducted a grid search on $20$ sampled images at each point in the distribution of $-30,-25,-20,-15,-10,-5,0,5,10,15,20,25\\,30$ categorizing each as a "document" or "non-document". The $(20, 15)$ range showed the best results, so we conducted a more dense grid search within this range and found the best threshold to be $18.77$ with a TPR of $100\%$ and an FPR of $1\%$.
\subsection{Literal Threshold}
\label{sec:literal_threshold}
To find a literal threshold, we conducted two grid searches on images that passed the OCR filters and had a "phrase-image" score higher than the "search-query" score. We sampled $20$ images from each point in the distribution of $-10,-8,-6,-4,-2,0,2,4,6,8,10$, and annotated them as "literal" or "non-literal". This distribution aligns with the normal distribution of the images that stand the two criteria mentioned above (Figure~\ref{fig:idiom_phrase_image_distribution}). We found the range of $(-2, 2)$ to result in the best thresholds, and so we conducted a more dense grid search in this range. We sampled $30$ images from each point in the distribution of $-5,-4,-2,-1,0,1,2,4,5$, and annotated them as "literal" or "non-literal". We chose the threshold of $1.150353$ with a TPR of $86\%$ and FPR of $18\%$. \\\\
\begin{figure}[h]
    \centering
    \includegraphics[width=0.48\textwidth,height=12cm,keepaspectratio]{figures/Figure_102_idiom_images_distribution}
    \caption{The distribution of the images that passed the OCR filters and had a "phrase-image" score higher than the "search-query" score.}
    \label{fig:idiom_phrase_image_distribution}
\end{figure}
\noindent We observed that when the "phrase-image" score is high, we can say that the image is literal with a high probability. However, the reverse is not true, there can be multiple “literal” images with a very low literal score (Figure~\ref{fig:literal_idiom_images}).
\begin{figure}[h]
    \centering
    \includegraphics[width=0.48\textwidth,height=12cm,keepaspectratio]{figures/Figure_101_literal_idiom_images.JPG}
    \caption{Literal images of the idiom "Foam at the mouth" and the idiom "Take the bull by the horns". Both images have a "phrase-image" score of $-9$.}
    \label{fig:literal_idiom_images}
\end{figure}

% \subsection{Annotation Task Discussion}
% \label{sec:annotation_task_discussion}
% The nature of the image annotation task is very subjective, and often the relation worker A sees between an idiom and an image differs from the relation worker B sees. The connection a worker sees between an image and an idiom can vary based on his understanding of the image scene and his interpretation of that scene. For example, Figure~\ref{fig:tiget_by_the_tail} shows a "Caption" image of a person holding a tiger by the tail in a non-dangerous or difficult situation in which one should not remain. The person is smiling as it seems like he is playing with a very young tiger (small in size). The majority of workers agreed with this explanation, except one who disagreed and chose the "Figurative Literal" category. This worker's interpretation arose from her viewpoint as a mother, and she said, "as a mother, I would still say that it's dangerous and the person is being foolish".\\\\
\begin{figure}[H]
    \centering
    \includegraphics[width=0.48\textwidth,height=12cm,keepaspectratio]{figures/Figure_104_have_a_tiger_by_the_tail.JPG}
    \caption{A candidate image from the training platform.}
    \label{fig:tiget_by_the_tail}
\end{figure}
\noindent Another example of different interpretations is the image seen in Figure~\ref{fig:cowboy_bunny}, which shows an image of a cowboy bunny drawing with the idiom - "quick on the draw". The annotations of this image were very diverse as $5$ workers chose $4$ different categories. Two workers chose "Figurative" as they saw a connection to the idiom definitions. One worker chose "Figurative Literal" as he saw a connection to the idiom definitions and a literal connection to the idiom. Another worker chose "Caption" because he did not find the image to be "Figurative" and saw the idiom literally as illustrating the image. The last worker selected "None" as he did not find a clear literal connection and didn't see the image as "Figurative" as it lacked an indication that the bunny was "quick to act" or "characterized by rapid response".
\begin{figure}[h!]
    \centering
    \includegraphics[width=0.48\textwidth,height=12cm,keepaspectratio]{figures/Figure_103_cowboy_bunny.JPG}
    \caption{A figurative candidate of the idiom "quick on the draw".}
    \label{fig:cowboy_bunny}
\end{figure}


\subsection{GenBench Evaluation Card}
\label{sec:evaluation_card}
\begin{table}[h!]
% Set tabular size
\newcommand{\tabularwidth}{\columnwidth}

% Set symbols for experiments
\newcommand{\expone}{$\square$}
\newcommand{\exptwo}{$\bigtriangleup$}
        
% Create table         
\renewcommand{\arraystretch}{1.1}         
\setlength{\tabcolsep}{0mm}         
\begin{tabular}{|p{\tabularwidth}<{\centering}|}         
\hline
               
% Record the experiments' motivations               
\rowcolor{gray!60}               
\textbf{Motivation} \\               
\footnotesize
\begin{tabular}{p{0.25\tabularwidth}<{\centering} p{0.25\tabularwidth}<{\centering} p{0.25\tabularwidth}<{\centering} p{0.25\tabularwidth}<{\centering}}                        
\textit{Practical} & \textit{Cognitive} & \textit{Intrinsic} & \textit{Fairness}\\
		% practical
& 		% cognitive
& \expone\hspace{0.8mm}		% intrinsic
& 		% fairness_inclusivity

\vspace{2mm} \\
\end{tabular}\\
               
% Record the experiments' generalisation type               
\rowcolor{gray!60}               
\textbf{Generalisation type} \\               
\footnotesize
\begin{tabular}{m{0.17\tabularwidth}<{\centering} m{0.20\tabularwidth}<{\centering} m{0.14\tabularwidth}<{\centering} m{0.17\tabularwidth}<{\centering} m{0.18\tabularwidth}<{\centering} m{0.14\tabularwidth}<{\centering}}                   
\textit{Compo- sitional} & \textit{Structural} & \textit{Cross Task} & \textit{Cross Language} & \textit{Cross Domain} & \textit{Robust- ness}\\
\expone\hspace{0.8mm}		% compositional
& 		% structural
& 		% across_task
& 		% across_language
& 		% across_domain
& \vspace{2.8mm}\hspace{4mm}\expone		% robustness

\vspace{2mm} \\
\end{tabular}\\
             
% Record the experiments' shift type             
\rowcolor{gray!60}             
\textbf{Shift type} \\             
\footnotesize
\begin{tabular}{p{0.25\tabularwidth}<{\centering} p{0.25\tabularwidth}<{\centering} p{0.25\tabularwidth}<{\centering} p{0.25\tabularwidth}<{\centering}}                        
\textit{Covariate} & \textit{Label} & \textit{Full} & \textit{Assumed}\\  
		% covariate
& 		% label
& \expone\hspace{0.8mm}		% full
& 		% assumed

\vspace{2mm} \\
\end{tabular}\\
             
% Record the experiments' shift source             
\rowcolor{gray!60}             
\textbf{Shift source} \\             
\footnotesize
\begin{tabular}{p{0.25\tabularwidth}<{\centering} p{0.25\tabularwidth}<{\centering} p{0.25\tabularwidth}<{\centering} p{0.25\tabularwidth}<{\centering}}                          
\textit{Naturally occuring} & \textit{Partitioned natural} & \textit{Generated shift} & \textit{Fully generated}\\
		% naturally_occurring
& \expone\hspace{0.8mm}		% partitioned_natural_data
& 		% generated_shifts
& 		% fully_generated_data

\vspace{2mm} \\
\end{tabular}\\
             
% Record the experiments' shift locus             
\rowcolor{gray!60}             
\textbf{Shift locus}\\             
\footnotesize
\begin{tabular}{p{0.25\tabularwidth}<{\centering} p{0.25\tabularwidth}<{\centering} p{0.25\tabularwidth}<{\centering} p{0.25\tabularwidth}<{\centering}}                         
\textit{Train--test} & \textit{Finetune train--test} & \textit{Pretrain--train} & \textit{Pretrain--test}\\
\expone\hspace{0.8mm}		% train-test
& \expone\hspace{0.8mm}		% finetune_train-test
& 		% pretrain-train
& 		% pretrain-test

\vspace{2mm} \\
\end{tabular}\\

\hline
\end{tabular}

\caption{The GenBench evaluation card \cite{hupkes2023stateoftheart} for the IRFL Multimodal Figurative Language Detection Task and the Multimodal Figurative Language Retrieval Task.}
\label{tab:gen-bench-card}
\end{table}
\newpage
\subsection{Understanding Task Samples}
\label{sec:task_samples}
\begin{figure}[h]
    \centering
    \includegraphics[width=0.40\textwidth,height=12cm,keepaspectratio]{figures/Figure_105_idiom_task_example_one.JPG}
    \label{fig:idiom_task_example_one}
\end{figure}
\begin{figure}[hp]
    \centering
    \includegraphics[width=0.40\textwidth,height=12cm,keepaspectratio]{figures/Figure_106_idiom_task_example_one.JPG}
    \label{fig:idiom_task_example_two}
\end{figure}
\begin{figure}[hp]
    \centering
    \includegraphics[width=0.40\textwidth,height=12cm,keepaspectratio]{figures/Figure_107_idiom_task_example_one.JPG}
    \label{fig:idiom_task_example_three}
\end{figure}
\begin{figure}[hp]
    \centering
    \includegraphics[width=0.40\textwidth,height=12cm,keepaspectratio]{figures/Figure_108_idiom_task_example_one.JPG}
    \label{fig:idiom_task_example_four}
\end{figure}
% Metaphors
\begin{figure}[hp]
    \centering
    \includegraphics[width=0.42\textwidth,height=12cm,keepaspectratio]{figures/Figure_109_metaphor_task_example_one.JPG}
    \label{fig:metaphor_task_example_one}
\end{figure}
\begin{figure}[hp]
    \centering
    \includegraphics[width=0.42\textwidth,height=12cm,keepaspectratio]{figures/Figure_110_metaphor_task_example_one.JPG}
    \label{fig:metaphor_task_example_two}
\end{figure}
\begin{figure}[hp]
    \centering
    \includegraphics[width=0.42\textwidth,height=12cm,keepaspectratio]{figures/Figure_111_metaphor_task_example_one.JPG}
    \label{fig:metaphor_task_example_three}
\end{figure}
\begin{figure}[hp]
    \centering
    \includegraphics[width=0.42\textwidth,height=12cm,keepaspectratio]{figures/Figure_112_metaphor_task_example_one.JPG}
    \label{fig:metaphor_task_example_four}
\end{figure}
% Similes
\begin{figure}[hp]
    \centering
    \includegraphics[width=0.42\textwidth,height=12cm,keepaspectratio]{figures/Figure_113_simile_task_example_one.JPG}
    \label{fig:simile_task_example_one}
\end{figure}
\begin{figure}[hp]
    \centering
    \includegraphics[width=0.40\textwidth,height=12cm,keepaspectratio]{figures/Figure_114_simile_task_example_one.JPG}
    \label{fig:simile_task_example_two}
\end{figure}
\begin{figure}[hp]
    \centering
    \includegraphics[width=0.40\textwidth,height=12cm,keepaspectratio]{figures/Figure_115_simile_task_example_one.JPG}
    \label{fig:simile_task_example_three}
\end{figure}
\begin{figure}[hp]
    \centering
    \includegraphics[width=0.40\textwidth,height=12cm,keepaspectratio]{figures/Figure_116_simile_task_example_one.JPG}
    \label{fig:simile_task_example_four}
\end{figure}

% \label{sec:instructions} 
\newpage
\subsection{Instructions}
\begin{center}
\begin{figure}
    \centering
    \includegraphics[width=12cm,height=30cm,keepaspectratio]{instructions/instructions_for_paper-1.jpg}
    \caption{Page 1 out of 8}
    \label{fig:instructions_first_page}
\end{figure}
\end{center}
\newpage
\begin{center}
\begin{figure}
    \centering
    \includegraphics[width=12cm,height=30cm,keepaspectratio]{instructions/instructions_for_paper-2.jpg}
    \caption{Page 2 out of 8}
    \label{fig:instructions_second_page}
\end{figure}
\end{center}
\newpage
\begin{center}
\begin{figure}
    \centering
    \includegraphics[width=12cm,height=30cm,keepaspectratio]{instructions/instructions_for_paper-3.jpg}
    \caption{Page 3 out of 8}
    \label{fig:instructions_three_page}
\end{figure}
\end{center}
\newpage
\begin{center}
\begin{figure}
    \centering
    \includegraphics[width=12cm,height=30cm,keepaspectratio]{instructions/instructions_for_paper-4.jpg}
    \caption{Page 4 out of 8}
    \label{fig:instructions_four_page}
\end{figure}
\end{center}
\newpage
\begin{center}
\begin{figure}
    \centering
    \includegraphics[width=12cm,height=30cm,keepaspectratio]{instructions/instructions_for_paper-5.jpg}
    \caption{Page 5 out of 8}
    \label{fig:instructions_five_page}
\end{figure}
\end{center}
\newpage
\begin{center}
\begin{figure}
    \centering
    \includegraphics[width=12cm,height=30cm,keepaspectratio]{instructions/instructions_for_paper-6.jpg}
    \caption{Page 6 out of 8}
    \label{fig:instructions_six_page}
\end{figure}
\end{center}
\newpage
\begin{center}
\begin{figure}
    \centering
    \includegraphics[width=12cm,height=30cm,keepaspectratio]{instructions/instructions_for_paper-7.jpg}
    \caption{Page 7 out of 8}
    \label{fig:instructions_seven_page}
\end{figure}
\end{center}
\newpage
\begin{center}
\begin{figure}
    \centering
    \includegraphics[width=12cm,height=30cm,keepaspectratio]{instructions/instructions_for_paper-8.jpg}
    \caption{Page 8 out of 8}
    \label{fig:instructions_eight_page}
\end{figure}
\end{center}
\newpage \subsection{Qualification} \label{sec:qualification} \input{sections/100D_qualification.tex} \subsection{Profile Page} \label{sec:profile_page} \input{sections/100E_profile_page.tex}






\end{document}
