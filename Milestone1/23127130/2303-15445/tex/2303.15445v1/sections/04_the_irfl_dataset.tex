% \yonatan{is it correct to say that it's an automatic generation followed by human ratings? If so I'll start by that. Also I suggest to say something similar to what we said in VASR: this is a process that contains several heuristics and implementation decisions. We evaluate the end2end dataset generation later on, and the fact that human achieve high agreement helps to verify the correctness of the end2end process. }
%\yonatan{is it correct to say that it's an automatic generation followed by human ratings? If so I'll start by that} \ron{We didn't started like this in VASR, why is it important here?}
Our goal is to introduce the IRFL dataset of idioms, metaphors, and similes with matching figurative and literal images and evaluate the figurative understanding and preference of Vision and Language models. To collect figurative and literal images for idioms, we developed an automatic pipeline that takes a list of idioms as input and outputs figurative and literal candidate images. We collected idioms from the MAGPIE corpus \cite{haagsma-etal-2020-magpie} of idiomatic expressions collected from Wiktionary, Oxford Dictionary of English Idioms, and UsingEnglish.com. The MAGPIE corpus contains 56,622 crowdsourced potentially idiomatic expressions, covering 1,756 unique idioms that appear in at least two of the dictionaries mentioned above. After collecting the idioms, we then feed them into the pipeline as input. First, we collect the definitions of the idioms from Wiktionary and Oxford dictionaries and construct search queries to find possible literal and figurative images (\ref{sec:enriching_similes_and_idioms}). The process of choosing figurative and literal candidates involves several heuristics and implementation decisions elaborated at (\ref{sec:choosing_images}). %At this point, the IRFL dataset contains figurative candidate images with a $41\%$ probability of being figurative. 
We annotate the different relations between each idiom and its candidate images, thus creating the IRFL dataset (\ref{sec:human_annotation}). We evaluate the end2end dataset generation, and the fact that humans achieve high agreement helps to verify the correctness of the end2end process. The relation categories can be seen with a corresponding explanation in Table~\ref{tab:relation-categories}.
\\\\
To collect metaphors and similes' images, we collected $35$ textual metaphors and $142$ textual similes from the internet. First, we constructed manual search queries and adapted the method used to search images in (\ref{sec:choosing_images}). Next, we annotated these images into ``Figurative'' and ``Literal'' categories. In total, we obtained $1107$ figurative images and $1816$ literal images for similes, and $333$ figurative images and $729$ literal images for metaphors. We verify the correctness of our dataset on different tasks in human evaluation section \ref{sec:human_evaluation}. 
\begin{table*}[tp!]
\begin{center}
\small
\begin{tabular}{ m{8em} m{8em} m{8em} m{8em} m{8em}}
\toprule
\multicolumn{5}{c}{\textbf{Idiom:} Touch wood} \\
\multicolumn{5}{c}{\begin{tabular}[x]{@{}c@{}} \textbf{Definitions:} 1) Hopefully 2) Said while touching something wooden, \\ to avert  superstitious bad luck from what has just been said \end{tabular}} \\ 

\midrule
\includegraphics[width=0.18\textwidth,height=5.5em]{figures/touch_wood_caption.jpeg} & 
\includegraphics[width=0.18\textwidth ,height=5.5em]{figures/touch_wood_figurative.jpeg} &
\includegraphics[width=0.18\textwidth ,height=5.5em]{figures/touch_wood_partial_objects.jpeg} & 
\includegraphics[width=0.18\textwidth ,height=5.5em]{figures/touch_wood_figurative_literal.jpg} &  \includegraphics[width=0.18\textwidth ,height=5.5em]{figures/touch_wood_none.jpeg} \\ \midrule

 Literal & Figurative & Partial Literal & Figurative+Literal & None  \\ \midrule

 The image illustrates the phrase literally & 
 The image conveys one or more \emph{definitions} of the idiom &
 Some objects/ entities of the phrase are visualized  (here, wood) & 
 Fits the ``Figurative'' definition and also ``Literal''/``Partial Literal''  &
 The image does not fit any of the other categories  \\ \bottomrule
\end{tabular}


% \small
% \begin{tabular}{ c c c c c c}
% \toprule
% \adjustbox{valign=c}{\includegraphics[width=0.18\textwidth ,height=\textheight,keepaspectratio]{figures/touch_wood_figurative_literal.jpg}} &
% \adjustbox{valign=c}{\includegraphics[width=0.18\textwidth ,height=\textheight,keepaspectratio]{figures/touch_wood_figurative_literal.jpg}} &
% \adjustbox{valign=c}{\includegraphics[width=0.18\textwidth ,height=\textheight,keepaspectratio]{figures/touch_wood_figurative_literal.jpg}} &
% \adjustbox{valign=c}{\includegraphics[width=0.18\textwidth ,height=\textheight,keepaspectratio]{figures/touch_wood_figurative_literal.jpg}} &
% \adjustbox{valign=c}{\includegraphics[width=0.18\textwidth ,height=\textheight,keepaspectratio]{figures/touch_wood_figurative_literal.jpg}} & \midrule

%  Figurative+Literal & The image conveys one or more definitions of the idiom to some extent,
% and it literally illustrates the phrase or visualizes the phrase objects/entities & \adjustbox{valign=c}{\includegraphics[width=0.18\textwidth ,height=\textheight,keepaspectratio]{figures/touch_wood_figurative_literal.jpg}} &\\  \midrule
%  Figurative & The image conveys one or more definitions of the idiom to some extent & \adjustbox{valign=c}{\includegraphics[width=0.18\textwidth ,height=\textheight,keepaspectratio]{figures/touch_wood_figurative_literal.jpg}}&\\ \midrule
%  Caption &  The image illustrates the phrase literally & \adjustbox{valign=c}{\includegraphics[width=0.18\textwidth ,height=\textheight,keepaspectratio]{figures/touch_wood_figurative_literal.jpg}} &\\ \midrule
%  Partial Objects &  The objects/entities of the phrase are visualized in the image & \adjustbox{valign=c}{\includegraphics[width=0.18\textwidth ,height=\textheight,keepaspectratio]{figures/touch_wood_figurative_literal.jpg}} &\\ \midrule
%  None &  The image does not fit any of the categories & \adjustbox{valign=c}{\includegraphics[width=0.18\textwidth ,height=\textheight,keepaspectratio]{figures/touch_wood_figurative_literal.jpg}} &\\  \bottomrule 
% \end{tabular}


% \small
% \begin{tabular}{ m{3.5em} m{5.3cm} c c}
% \toprule
%  Figurative+Literal & The image conveys one or more definitions of the idiom to some extent,
% and it literally illustrates the phrase or visualizes the phrase objects/entities & \adjustbox{valign=c}{\includegraphics[width=0.18\textwidth ,height=\textheight,keepaspectratio]{figures/touch_wood_figurative_literal.jpg}} &\\  \midrule
%  Figurative & The image conveys one or more definitions of the idiom to some extent & &\\ \midrule
%  X &  The image illustrates the phrase literally &\\ \midrule
%  x &  The objects/entities of the phrase are visualized in the image & &\\ \midrule
%  None &  The image does not fit any of the categories & &\\  \bottomrule 
% \end{tabular}


\end{center}
\caption{The table shows the different categories of the relation between an image and a phrase, along with matching images for the idiom "Touch wood". Workers were guided to choose the most suitable relation category by a scheme tree that illustrates the correct thinking process (Figure~\ref{fig:image-task-tree}, Appendix~\ref{sec:annotation_ui}).}
\label{tab:relation-categories}
\end{table*}



\subsection{Search Queries}
\label{sec:enriching_similes_and_idioms}
We want to find literal and figurative images for each idiom we collected from the MAGPIE dataset. For that, we collect the idioms' definitions from online dictionaries and parse them into ``search queries''. We first search Wiktionary for each idiom's definition and scrape the data using a web crawler. In case no definitions are found, we search the Oxford Dictionary and collect definitions using a similar method. The definitions in Wiktionary are usually tagged with the context in which they appear. For example, the idiom ``white hat'' has the ``figurative'' definition of ``A good person; a hero'', and the ``slang'' definitions of ``a sailor'' and ``A well-meaning hacker''. We collect this data and filter idioms with no ``figurative'' or ``idiomatic'' definitions. We then construct search queries by parsing the ``figurative'' and ``idiomatic'' definitions of idioms \footnote{We also construct search queries from untagged definitions. Even though untagged definitions are rare (1-2\% of all definitions), they are typically idiomatic.}. The parsing process separates definitions that are in fact several definitions concatenated into one. For example, we split the definition ``A good person; a hero'' into two search queries ``A good person'' and ``A hero''. In some rare cases, a definition may be an idiom, and to tackle such cases, we replace the idiom with its definitions.
\subsection{Choosing Images}
\label{sec:choosing_images}
To find figurative images for our search queries, we searched Google images \footnote{Images were searched with ``SafeSearch'' flag ``on'', and in ``United States'' region.}, taking up to 20 images per search query. The resulting images included a lot of ``garbage'' and problematic images with specific characteristics, such as images in which the idiom they were derived from and its definitions are written. These images were problematic because a model may see a connection between an idiom and a figurative image solely based on the textual signal that appears in it. Such images were filtered out by using OCR and a spelling tool to correct any spelling errors the OCR had. A large number of ``garbage'' images were found in the search results, including letters, postcards, newspapers, and images with mostly text in them. To tackle this problem, we used OCR to remove images with more than a couple of words and images with a text size bigger than 30\%. In addition, we removed images that looked like documents that the OCR failed to detect. Images that passed these filters were literal, figurative, or had no connection to the phrase they originated from. Next, we calculated the matching score of each image with its phrase and search query. Images with a ``phrase-image'' score that passed a certain literal threshold (Appendix~\ref{sec:literal_threshold}) were tagged as ``literal'', and from these images, we chose the top K images as literal candidates. From the non ``literal'' images, we chose the top K images with the highest ``search query-image'' score as Figurative candidates. We then annotated the relation between the figurative phrase and its Figurative and Literal candidates using the UI seen in Figure~\ref{fig:image-task-ui}.

\subsection{Human Annotation}
\label{sec:human_annotation}
We hired Amazon Mechanical Turk workers to annotate the relation between each idiom and its candidate images. Five workers annotated each image, the images were annotated in batches of five for the reward of \$0.15 for batch. We created a difficult qualification test \footnote{https://irfl-dataset.github.io/mturk/image/qualification} to select quality annotators and provided them with an interactive training platform \footnote{https://irfl-dataset.github.io/mturk/image/train} to understand the task and the different categories better. We split the annotation process into batches with an average size of 60 idioms per batch. After each batch, We provided each worker with a personal profile page \footnote{https://irfl-dataset.github.io/profile/example} to view its statistics and some handily picked examples where his choice was distant from a majority of four workers. We also provided workers with a leaderboard \footnote{https://irfl-dataset.github.io/mturk/leaderboard} that was updated after each batch to improve their competitiveness. Full annotation results and statistics are presented in Table \ref{tab:dataset-statistics}.
\\\\
The nature of this task is very subjective, and often the relation worker A sees between an idiom, and an image differs from the relation worker B see. We provide further discussion about this aspect of the task in (Appendix~\ref{sec:annotation_task_discussion}). Despite the subjective aspect of the task and its complexity in distinguishing between the various categories, in 94\% of the instances, there was a majority of 3 workers or more compared to a random chance of 29\%. This shows that different people can see the same connection most of the time. 

\begin{table}[t!]
%\centering
\small
% \resizebox{\columnwidth}{!}{
\begin{tabular}{@{}lcccccc@{}}
\toprule
   & Fig. & \begin{tabular}[x]{@{}c@{}}Fig.\\ Lit.\end{tabular} & Lit. & \begin{tabular}[x]{@{}c@{}}Part.\\ Lit.\end{tabular}  & None & \\ \midrule
\#             & 1970  &  751   &   434  &  487  & 2638 & 6697 \\ \midrule
3-maj & 100\% & 100\%  & 100\%  & 100\% & 100\% &  94\% \\
4-maj & 75.5\%  & 63\%   & 68\%   & 63\%  & 80\% &  70\% \\
5-maj & 45\%  & 33\%   & 35\%   & 38\%  & 53\% &  43\% \\ \midrule
Mean & 3.1   & 1.2    & 0.7    & 0.8   & 4    & - \\
Median  & 2     & 0      & 0      & 0     & 4    & - \\ \bottomrule
\end{tabular}
% }
\caption{IRFL statistics on 628 idioms. The majority of the images are related to the figurative phrase, most images are Figurative. (k-maj means k-majority)}
\label{tab:dataset-statistics}
\end{table}



\remove{
\begin{table}[t!]
%\centering
\small
% \resizebox{\columnwidth}{!}{
\begin{tabular}{@{}lcccccc@{}}
\toprule
Categories   & Fig. & \begin{tabular}[x]{@{}c@{}}Fig.\\ Literal\end{tabular} & Literal & \begin{tabular}[x]{@{}c@{}}Partial\\ Literal\end{tabular}  & None & Total\\ \midrule
Number             & 1970  &  751   &   434  &  487  & 2638 & 6697 \\ \midrule
3 majority & 100\% & 100\%  & 100\%  & 100\% & 100\% &  94\% \\
4 majority & 75.5\%  & 63\%   & 68\%   & 63\%  & 80\% &  70\% \\
5 majority & 45\%  & 33\%   & 35\%   & 38\%  & 53\% &  43\% \\ \midrule
Average & 3.1   & 1.2    & 0.7    & 0.8   & 4    & - \\
Median  & 2     & 0      & 0      & 0     & 4    & - \\ \bottomrule
\end{tabular}
% }
\caption{IRFL statistics on 628 idioms. The majority of the images have some relation to the figurative phrase. Most of the relations are Figurative.}
\label{tab:dataset-statistics}
\end{table}

}





%\subsection{Dataset Analysis}
%\label{sec:dataset_statistics}
%\input{sections/04E_dataset_statistics}





