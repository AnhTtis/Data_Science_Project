% The task is discriminative for humans, the annotators were able to distinguish between the different relations of image and the idiomatic phrase without coordination.


We evaluate humans and state-of-the-art image recognition models ability to understand figurative language (Section \ref{sec:understanding_task}). We show that IRFL tasks are easy for humans (97\% accuracy) and challenging for models (<27\%). Additionally, we provide a detailed analysis per figure of speech, experiments with idioms and their definitions as input, and with different candidate types. We find that models fail the IRFL task due to their preference for partially literal images over figurative images and introduce a preference task to tackle this problem (Section \ref{sec:ranking Task Analysis}). In addition, we examine the ability of generative models such as Dall-E and Stable Diffusion to generate figurative images for idioms (Section \ref{sec:genearive_models_analysis}). We find that they are unable to generate figurative images given idiomatic phrases. Given the definitions of an idiom, generative models can generate figurative images.

%\yonatan{missing more experiments (What we've discussed from the ``Why is Winoground hard paper'') - Iterate each one of the chapters we discussed and see if you can repeat the experiments in this study}\ron{we can do it only with metaphors and similes, it will not work with idioms. I am in favor on adding it to the next version}
\subsection{Zero-Shot Baselines}
\label{sec:zero_shot_evaluation}
We evaluate several diverse state-of-the-art vision-and-language models. Due to ViLT's maximum sequence length of 40, we do not evaluate it on idioms. In all cases described below (except CLIP-ViL), the model encodes the figurative phrase and the image and produces a matching score for each pair. We chose the image that results the highest matching score as the image that best matches the figurative expression. 
\begin{enumerate}
    \item CLIP \cite{radford2021learning} is pre-trained with a contrastive objective that can be used without directly optimizing for the task. We use four versions of models with different amounts of parameters: RN50, ViT-B/32, ViT-L/14 and RN50x64/14 with 100M, 150M, 430M and 620M parameters respectively (RN50 was used during data collection).
    \item CLIP-ViL \cite{shen2021much}, with 290M parameters, is a pre-trained vision-and-language model that uses CLIP as a visual backbone, rather than CNN based visual encoders that are trained on a small set of manually annotated data.
    \item ViLT \cite{kim2021vilt}, with 111M parameters, incorporates text embeddings into a Vision Transformer (ViT).
    
\end{enumerate} 
\subsection{Supervised Models}
\label{sec:supervised_models_evaluation}
We join a line of benchmarks that introduce a test set without predefined train splits \cite{thrush2022winoground,rudinger2018gender,emelin-sennrich-2021-wino}, \cite{Bitton2022WinoGAViLGA}. We believe that in order to understand metaphors and similes, a machine must be able to abstract and map between domains. It should be able to solve unseen cases without extensive training \cite{mitchell2021abstraction}. Contrary to metaphors and similes, understanding idioms requires language and cultural knowledge that can be learned through extensive training. We train a supervised model for figurative classification of idioms. We add a binary classifier on top of the pre-trained embeddings to classify whether a given image is figurative or not. We use CLIP (VIT-B/32) model, concatenate the textual idiom embedding to the visual image embedding, followed by a classifier that produces a matching score, where a matching score above 0.5 is labeled ‘Figurative’. We use the Adam optimizer \cite{Kingma2014} with a learning rate of 0.001, batch size of 12, and train for 7 epochs. We run the fine-tuned model on the understanding and preference task using the model's matching score. We train the binary classifier on 4790 images for the understanding task and 3802 images for the preference task\footnote{Training data does not contain any of the images or idioms that appear in the task.}. We repeat five experiments with different random seeds for each task and take the mean score along with the standard deviation. 

\subsection{Understanding Task}
\label{sec:understanding_task}
The figurative understanding task evaluates VL-PTMs’ ability to understand the relation between an image and a figurative phrase. The task is to choose the image that best visualizes the figurative phrase out of X candidates. Our goal was to create an understanding task that consists of ``mixed'' candidate types and represents the richness of our dataset. The ``mixed'' tasks provide a holistic image of the figurative understanding of vision and language models. We constructed and crowdsourced $810$ ``mixed'' figurative understanding task instances for idioms, metaphors, and similes. \\\\
\noindent The basic structure of all ``mixed'' instances is the same. Each instance contains four candidates, of which one is the correct answer, and $1-3$ candidates are partially literal distractors. The ``mixed'' idiom instances have one to two partially literal distractors and one to two random images. The simile ``mixed'' instances contain a distractor image of the target concept without the compared property or with its antonym visualized, a distractor image of the source concept, and one random image. The metaphor ``mixed'' instances consist of between one to three partially literal distractors, and the remaining candidates are random images. In $65\%$ of the idioms understanding task instances, the correct answer is ``Figurative'', and in the other $35\%$, the correct answer is ``Figurative  Literal''. Figure~\ref{fig:first-task-idiom-figurative} shows two examples of the ``mixed'' figurative understanding task for metaphor and simile. 
\subsubsection{Human Evaluation}
\label{sec:human_evaluation}
We asked annotators that did not work on previous IRFL tasks to solve the figurative understanding task. Each instance of the ``mixed'' understanding task was annotated by $5$ annotators, and the correct answer is chosen by the majority. We find that human performance on the data sampled for all figures of speech ranges between $90\%-100\%$. Additionally, in $83\%-99\%$ of the instances, there was an agreement between at least four annotators compared to a random chance of $6\%$. Samples from the validation process are presented in Appendix \ref{sec:task_samples}.



\subsubsection{Results and Model Analysis}
\label{sec:results_and_model_analysis}
% \yonatan{add paragraphs with the main title findings of each main finding}
Zero-shot results on the ``mixed'' figurative understanding task are presented in Table \ref{tab:mixed-single-choice-results}. The best model achieved $27\%$, $30\%$, and $52\%$ accuracy on the idioms\footnote{Idioms were passed along with their definitions as input.}, metaphors, and similes tasks compared to a random chance of $25\%$. \textbf{These results suggest that models do not understand the connection between a figurative phrase and an image like humans do.} We conduct a fine-grained analysis to examine if models failed the ``mixed'' understanding task because they do not see any connection to the figurative images or rather because they prioritize ``weak'' literal connections over figurative ones. \\
\begin{table}[tp]
\centering
\scalebox{0.65}{
\begin{tabular}{@{}lcccccc@{}}
\toprule
Categories   & \multicolumn{2}{c}{Idioms}         & Metaphors                  & Similes\\ \midrule
             & Figurative    & Figurative Literal &                    & \\ \midrule
Humans          &  97\%          & 90\%           & 99.7\%             & 100\% \\ \midrule
CLIP-VIT-L/14   &  17\%          & \textbf{56\%}  & 25\%               & \textbf{52\%} \\
CLIP-VIT-B/32   &  16\%          & 44\%           & 23\%               & 45\% \\
CLIP-RN50       &  14\%          & 37\%           & 27\%               & 47\% \\
CLIP-RN50x64/14 &  22\%          & \textbf{56\%}  & \textbf{30\%}      & 52\%\\
LiT             &  \textbf{27\%} & 31\%           & 21\%               & 19\%\\ \midrule
ViLT            &   -            &  -             & 23\%               & 40\% \\ \midrule
\# Unique Phrases  & 48       & 30             & 35                 &  142\\ \midrule
\# Tasks     &  135           & 65             & 333                & 277 \\\bottomrule

\end{tabular}}
\caption{Zero-shot models performance on the IRFL "mixed" understanding task by figurative type. There are two columns for idioms, the first column represents the score for the "Figurative" images, and the second for the "Figurative Literal" images. In the idioms tasks the model received both the Idioms and their definitions as input. Numbers are the percentage of instances annotated correctly. Bold numbers indicate the best model performances.}
\label{tab:mixed-single-choice-results}
\end{table}
%\ron{Remove human performance from this tab and put it only on mix}
%\ron{Create very diverse task and start with it}
%\ron{Fine-grained analysis - distractors affect}
%\ron{to better understand the behavior of these models on the dataset, we examine other categories}





\noindent \textbf{Models prefer partially literal images over figurative ones.}
We analyzed the models' choices on the ``mixed'' figurative understanding task and found that in all models (excluding LiT on idioms and similes), a partially literal distractor was selected in $92\%-100\%$ of the instances where the models failed across all figures of speech (idioms, metaphors, and similes) . This shows that models prefer partially literal images over figurative ones.\textbf{ We find the case of idioms to be particularly interesting in this regard. Models receive a relatively long prompt containing both the idiom and its definitions as input. Instead of picking an image that fits the prompt semantically, they choose an image that is literal to one or two words.}\\
%\ron{@Dafna, should I add here a full table of this data? maybe in Appendix? (It will be 6X4)}% 
\begin{table}[tp]
\centering
\small
\begin{tabular}{@{}lccccccccc@{}}
\toprule
Categories                                                  & \multicolumn{6}{c}{Fig.}      & \multicolumn{2}{c}{\begin{tabular}[c]{@{}l@{}}Fig. \\ Lit.\end{tabular}}\\ \midrule
Candidates                                                  &     \multicolumn{2}{c}{2}            &    \multicolumn{2}{c}{4}       &    \multicolumn{2}{c}{6}   & \multicolumn{2}{c}{4} \\ \midrule
Random                                                      &     \multicolumn{2}{c}{50}             &    \multicolumn{2}{c}{25}    &    \multicolumn{2}{c}{16.6}   & \multicolumn{2}{c}{25} \\ \midrule
\begin{tabular}[c]{@{}l@{}}CLIP- \\ VIT-L/14\end{tabular}   &  64          & \textbf{87}   &  \textbf{46} & \textbf{71} & \textbf{33} & 53            & 76          & 86 \\
\begin{tabular}[c]{@{}l@{}}CLIP- \\ VIT-B/32\end{tabular}   &  61          & 84            &  38          & 67          & 30          & 53            & 65          & 82\\
CLIP-RN50                                                   &  56          & 75            &  30          & 60          & 24          & 46            & \textbf{78} & 86 \\
\begin{tabular}[c]{@{}l@{}}CLIP- \\ RN50x64\end{tabular}    &  \textbf{67} & 79            &  38          & 67          & 27          & 51            & 69          & 85\\ 
BLIP                                                        &  57          & 79            &  30          & 62          & 19          & 51            & 72           & 88\\
BLIP2                                                       &  58          & 75            &  25          & 58          & 14          & 40            & 75           & 82\\
\begin{tabular}[c]{@{}l@{}}COCA \\ ViT-L-14\end{tabular}    &  62          & 82            &  39          & \textbf{71} & 32          & \textbf{60}   & 68           & \textbf{91}\\ \bottomrule
% LiT                                                         &  51          & 48            &  22          & 25          & 12          & 15            & 21          & 24\\ \bottomrule

\end{tabular}
\caption{Zero-shot models performance on different configurations of the multimodal figurative language detection task, idioms with random candidates. Numbers are \% instances annotated correctly.  The left column of each pair shows the score for the idiom alone as input, and the right column shows the score for the idiom and definitions. Models %achieve higher scores on idioms with random candidates but still 
fail to reach human performance.}
\label{tab:random-idiom-single-choice-results}
\end{table}
% \ron{Remove human performance from this tab and put it only on mix}
% \ron{Create a very diverse task and start with it}
% \ron{Fine-grained analysis - distractors affect}
% \ron{to better understand the behavior of these models on the dataset, we examine other categories}





\noindent \textbf{Models partially understand the figurative connection between idioms and images.}
To examine whether models can comprehend a figurative connection between an image and an idiom, we experiment with random candidates and several configurations of the understanding task (Table~\ref{tab:random-idiom-single-choice-results}). The accuracy score on the Figurative category with $2$ candidates is $61\%-87\%$, and $22\%-71\%$ with $4$ candidates. These results are marginally above random chance but still below human performance on the ``mixed'' task. When given the idiom alone as input, most models achieved $80\%-84\%$ with $2$ candidates and $30\%-46\%$ with $4$ candidates compared to random chance of $50\%$ and $25\%$. These results suggest that models partially understand the figurative connection between idioms and images. Moreover, we see a significant performance drop with all models when increasing the number of candidates.\\\\
In the Figurative Literal category, models achieve a $65\%-78\%$ accuracy score with 4 candidates, significantly higher than the performance in the Figurative category with $2$ and $4$ candidates. These results can be explained by the fact that Figurative Literal images possess a literal connection to the phrase in addition to a figurative one. \\
\begin{table}[h!]
\centering
\scalebox{0.8}{
\begin{tabular}{@{}lcccccc@{}}
\toprule
Categories   & \multicolumn{2}{c}{Metaphors} & \multicolumn{2}{c}{Similes}\\ \midrule
Candidates  &      2             &        4            &            2        &        4 \\ \midrule
CLIP-VIT-L/14   &  87\%          &      72\%           &  \textbf{99\%}      & \textbf{97\%} \\
CLIP-VIT-B/32   &  86\%          &      73\%           &  \textbf{99\%}      & \textbf{97\%} \\
CLIP-RN50       &  83\%          &      66\%           &  \textbf{99\%}      & \textbf{97\%} \\
CLIP-RN50x64/14 &  \textbf{88\%} & \textbf{76\%}       &  98\%               & 96 \\ 
LiT             &  47\%          & 27\%                &      49\%           &  24\% \\ \midrule
ViLT            &  72\%          & 53\%                &      96\%           &  91\%\\ \bottomrule

\end{tabular}}
\caption{Zero-shot models performance on the metaphors and similes understanding task with random candidates. Numbers are the percentage of instances annotated correctly.}
\label{tab:random-similes-metaphors-single-choice-results}
\end{table}










\noindent \textbf{Models understand metaphors, but fail to reach human performance.}   
Table~\ref{tab:random-similes-metaphors-single-choice-results} shows the models' performance on the metaphors figurative understanding task with random candidates. The accuracy score of all models, excluding LiT, on the Figurative category with $2$ candidates is $72\%-88\%$, and $53\%-76\%$ with $4$ candidates. We see a significant performance drop with all models when increasing the number of candidates. The results suggest that models understand metaphors but fail to reach human performance. \\\\
\noindent \textbf{Models understand similes as well as humans.}
Table~\ref{tab:random-similes-metaphors-single-choice-results} shows the models' performance on the similes figurative understanding task with random candidates. The accuracy score of all models, excluding LiT, on the Figurative category with $2$ candidates is $96\%-99\%$, and $91\%-97\%$ with $4$ candidates. Models' performance is competitive with that of humans, and the models maintain their performance when increasing the number of candidates. We note that we experiment with open similes where the compared property is explicitly mentioned in the simile. Thus the Figurative images can be seen as Figurative Literal. As we analyzed the ``mixed'' understanding task results in more depth, we found that across all models excluding LiT, $55\%-61\%$ of the figurative images received a higher matching score than the source concept images. In addition, $50\%-66\%$ of the source concept images received a higher matching score than the target concept distractor image. These suggest that models prioritize simile images in the following order: 1) images of the target concept with the compared property, 2) images of the source concept, 3) images of the target concept without the compared property.\\\\
\noindent \textbf{Fine-tuning improves figurative understanding and reduces partially literal preference.} Fine-tuning results are presented in Table~\ref{tab:understanding_task_supervision}. The mean Figurative category accuracy is $58\%$ compared to $13\%$ in the Zero-shot configuration. We analyzed the fine-tuned model results and compared them to the zero-shot configuration and found that in $41\%\pm4.3$ of the instances where the model failed, a partially literal distractor was selected compared to $96\%$ in the zero-shot configuration. Along with this improvement in literal preference, Figurative Literal category accuracy raised from $41\%$ in zero-shot to $49\%$. These results show that models can moderate their preference for partially literal images and recognize idiomatic figurative connections better, using extensive training. Moreover, the results suggest that the data is a valuable training signal for this task.
\begin{table}[tp]
\begin{center}

\scalebox{0.8}{
\begin{tabular}{@{}lcccc@{}}
\toprule
 Categories & Figurative & Figurative Literal &\\  \midrule
 Zero-Shot & $16\%$ & $41\%$ &\\ 
 Supervised & $58\%\pm4.2$ & $49\%\pm2.6$  & \\ \bottomrule 
\end{tabular}} 


% \begin{tabularx}{230pt}{| c c c X|}
%  \hline
%  Categories & Figurative & Figurative Literal &\\ 
%  \hline
%  Zero-Shot & $13\%$ & $41\%$ &\\ 
%  \hline
%  Supervised & $51.5\%\pm3.9$ & $51\%\pm3.1$  &\\
%  \hline
% \end{tabularx}}

\end{center}
\caption{Supervised models performance. Results are the mean and standard deviation of the accuracy
of five experiments.}
\label{tab:understanding_task_supervision}
\end{table}





\subsection{Ranking Task Analysis}
\label{sec:ranking Task Analysis}
To tackle vision and language models' strong preference toward partially literal images over figurative images, we introduce the preference task. The preference task is to rank the Figurative images higher than partially literal distractors based on the model matching score. First, we rank the figurative phrase images by their matching score from higher to lower, then we define two classes, $\#F$ which consists of the Figurative images, and $\#P$ which consists of the partially literal images. The model then predicts the first $\#F$ images as Figurative and the last $\#P$ images as partially literal images, the $F_1$ score of the model predictions is the preference task score. The results of the preference task are presented in Table \ref{tab:ranking-task}. We evaluate all figurative phrases that have images from both of the categories.
\begin{table}[h!]
\centering
\scalebox{0.65}{
\begin{tabular}{@{}lcccc@{}}
\toprule
Ranking  &     \multicolumn{2}{c}{Idioms}                   &  Metaphors         &   Similes \\ \midrule
                & Figurative Literal  & Figurative &                                        & \\ \midrule
CLIP-VIT-L/14   &       57            &    37                 & 26               & \textbf{44} \\
CLIP-VIT-B/32   &       54            &    36                 & 22               & 38 \\
CLIP-RN50       &       54            &    37                 & 25               & 38 \\
CLIP-RN50x64/14 &  \textbf{61}        &    39                 & \textbf{29}      & 43 \\
LiT             &  54                 & \textbf{56}           & 25               & 25 \\ \midrule
ViLT            &        -            & -                     & 23               & 34 \\ \midrule
\# of phrases   & 94                 & 149                    & 35               & 142  \\ \bottomrule
\end{tabular}}
\caption{The preference task performance, the scoring metric is $F_1$. The Idiom category is double-columned. The left column shows the score for Figurative Literal images, and the right column shows the score for Figurative images.}
\label{tab:ranking-task}
\end{table}

\noindent Models' scores on the preference task are low (<$61\%$). We expect models with proper figurative preference to achieve better results. Models' success in the Figurative Literal category can be attributed to the literal connections of the Figurative Literal images.
\begin{table}[h!]
\begin{center}
\scalebox{0.8}{

\begin{tabular}{@{}lcccc@{}}
\toprule
 Categories & Figurative & Figurative Literal &\\  \midrule
 Zero-Shot & $36$ & $54$ &\\ 
 Supervised & $68\pm3.8$ & $64\pm2.25$  &\\ \bottomrule
\end{tabular}} 


% \begin{tabularx}{230pt}{| c c c X|}
%  \hline
%  Categories & Figurative & Figurative Literal& \\ 
%  \hline
%  Zero-Shot & $36$ & $54$ &\\ 
%  \hline
%  Supervised & $73\pm1.8$ & $70\pm0.9$  &\\
%  \hline
% \end{tabularx}}

\end{center}
\caption{Supervised models performance. Results are the mean and standard deviation of the $F_1$ score
of five experiments.}
\label{tab:preference_task_supervision}
\end{table}

\\\noindent The supervised model, after fine-tuning, achieved a $68\pm3.8$ $F_1$ score on the Figurative category, almost double the zero-shot score of CLIP-ViT-B/32 ($36$). Additionally, the score in the Figurative Literal category was improved by $10\pm2.25$ points. These results align well with the observation that the fined-tuned understanding task model showed substantially moderate literal preference. Table~\ref{tab:preference_task_supervision} shows the fine-tuned model results. 

\subsection{Generative Models Analysis}
\label{sec:genearive_models_analysis}
To examine whether generative models can generate figurative images, we sampled 15 idioms from the IRFL dataset and experimented with the idioms and their definitions as input to Dall E and Stable Diffusion. We annotated $345$ generated images and found that generative models failed to generate figurative images for given idioms but instead generated literal images. When provided with the definitions as input, the models succeeded in creating figurative images to some extent. Statistics on the generated images and the matching IRFL images can be seen in Table \ref{tab:generative-models-statistics}.\\
\begin{table}[h]
\centering
\small
\begin{tabular}{@{}lccccccc@{}}
\toprule
Categories          & \multicolumn{2}{c}{Dall-E} & \multicolumn{2}{c}{\begin{tabular}[c]{@{}l@{}}Stable \\ Diffusion\end{tabular}} & \multicolumn{2}{c}{IRFL} & \\ \midrule
Figurative          & 0  & 42.5              & 0 & 11                                                                    & 4 & 46                  \\
Figurative+Literal  & 0 & 10                 & 5 & 1                                                                     & 20 & 6                     \\
Literal             & 31 & 0                 & 17 & 0                                                                    & 35 & 0                 \\
Partial Literal     & 48 & 2                 & 42 & 2.5                                                                  & 23 & 1.5              \\
None                & 19  & 44               & 27 & 85                                                                   & 4 & 43             \\ \midrule
Number              & 48 & 120               & 59 & 118                                                                  & 69 & 126      \\ \bottomrule 
\end{tabular}
\caption{The table is double-columned, the first column describes the percentage of images generated by idioms, and the second column describes the percentage of images generated by the idioms' definitions. The results show that our pipeline extracted more Figurative, Figurative+Literal, and Literal images and fewer None images than the generative models.}
\label{tab:generative-models-statistics}
\end{table}

\newline The experiment results show that our pipeline extracted more Figurative, Figurative literal, and Caption images and fewer None images than the generative models. Future work might focus on the quality of generative models' figurative images and the emotions they evoke. 
% \ron{@Dafna, should we delete table 9?}
%\input{tables/generative_models_baseline_vs_irfl.tex}
