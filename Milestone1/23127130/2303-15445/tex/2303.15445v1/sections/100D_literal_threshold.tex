To find a literal threshold, we conducted two grid searches on images that passed the OCR filters and had a "phrase-image" score higher than the "search-query" score. We sampled $20$ images from each point in the distribution of $-10,-8,-6,-4,-2,0,2,4,6,8,10$, and annotated them as "literal" or "non-literal". This distribution aligns with the normal distribution of the images that stand the two criteria mentioned above (Figure~\ref{fig:idiom_phrase_image_distribution}). We found the range of $(-2, 2)$ to result in the best thresholds, and so we conducted a more dense grid search in this range. We sampled $30$ images from each point in the distribution of $-5,-4,-2,-1,0,1,2,4,5$, and annotated them as "literal" or "non-literal". We chose the threshold of $1.150353$ with a TPR of $86\%$ and FPR of $18\%$. \\\\
\begin{figure}[h]
    \centering
    \includegraphics[width=0.48\textwidth,height=12cm,keepaspectratio]{figures/Figure_102_idiom_images_distribution}
    \caption{The distribution of the images that passed the OCR filters and had a "phrase-image" score higher than the "search-query" score.}
    \label{fig:idiom_phrase_image_distribution}
\end{figure}
\noindent We observed that when the "phrase-image" score is high, we can say that the image is literal with a high probability. However, the reverse is not true, there can be multiple “literal” images with a very low literal score (Figure~\ref{fig:literal_idiom_images}).
\begin{figure}[h]
    \centering
    \includegraphics[width=0.48\textwidth,height=12cm,keepaspectratio]{figures/Figure_101_literal_idiom_images.JPG}
    \caption{Literal images of the idiom "Foam at the mouth" and the idiom "Take the bull by the horns". Both images have a "phrase-image" score of $-9$.}
    \label{fig:literal_idiom_images}
\end{figure}
