



Figures of speech include metaphors, similes, and idioms that allow language to be expressive, to convey abstract ideas that might otherwise be difficult to visualize, and to evoke emotion \cite{why-do-people-use-figurative-language, SusanAndMallie1994}. A metaphor is a comparison between two unrelated concepts that enable us to think of the target concept in terms of the source concept. For example, in the sentence ``You’re a peach!'', the person being addressed is equated with a peach, with the suggestion that the person is pleasing or delightful. A simile is a figure of speech that compares two things and is often introduced by ``like'' or ``as'' \cite{Paul-1970}. A simile is called ``open'' when the shared properties are not explicitly revealed, like ``Her heart is like a stone'', and ``closed'' when they are explicitly revealed, like ``Her heart is hard as stone''. An idiom is a group of words with a figurative, non-literal meaning that can not be interpreted by looking at its individual words. For example, the idiom ``We're on the same page'' means ``Agreeing about something (such as how things should be done)''.
\begin{figure}[tp]
\includegraphics[width=0.45\textwidth ,height=\textheight,keepaspectratio]{figures/let_the_cat_out_of_the_bag_big.jpg}
\includegraphics[width=0.45\textwidth ,height=\textheight,keepaspectratio]{figures/blanket_of_snow_bigger.jpg}
\includegraphics[width=0.45\textwidth ,height=\textheight,keepaspectratio]{figures/car_cheetah_bigger.JPG}
\caption{Examples of the figurative understanding task for idiom, metaphor, and simile in corresponding order. The figurative phrase is displayed in the top section, and the bottom section displays four candidates from which the correct answer (orange) has been selected. Idiom tasks also display the idiom definitions below the idiom.}
\label{fig:first-task-idiom-figurative}
\end{figure}
Understanding metaphors and similes require the cognitive ability to map between domains, and depending on the source and target concept, it can require commonsense, association abilities, and general knowledge. Understanding idioms requires profound language, and cultural knowledge 
\cite{Paul-1970, philip2011colouring}. Humans intuitively understand these figures and employ them in everyday communication \cite{LakoffandJohnson1980, Hoffman1987WhatCR}. These figurative forms are often conveyed through multiple modes, such as text and images, and frequently appear in advertising, news, social media, etc. \\
%Figure~\ref{fig:multimodal} presents an image of a luxury car posted on social media with the simile ``As fast as a cheetah'' and an advertisement for Toyota Corolla with the idiom ``get your hands on''. \\
%\yonatan{so far you have a table and an image for explaining existing concepts. I wonder whether it should be in the Appendix rather than taking important space at the start of the paper, because it's not something new you present in the work.}\ron{@Dafna}
% Multimodal metaphors are metaphors conveyed through multiple modes ``whose target and source are each represented exclusively or predominantly in different modes'' \cite{Forceville2016}. Multimodal information from different modes, such as language and vision, can contribute to metaphorical conceptualization and comprehension \cite{PerspectivesMultimodalityBook, multimodalMetaphorBook}. The comprehension of multimodal metaphors takes cognitive efforts like decoding metaphorical messages and understanding the relationships between domains, analyzing the emotion metaphors convey, and interpreting authorial intent [5–7] \cite{conceptual-integration-and-metaphor, YANG2013312, FAUCONNIER1998133}. Since similes are often referred to as metaphors in the cognitive linguistic and rhetorical literature \cite{culler1981pursuit, lou2021multimodal}, the topic of multimodal similes has not been thoroughly investigated. Recently, Adrian Lou \cite{lou2021multimodal} proposed a novel and robust framework for the analysis of both verbal and multimodal similes. Multimodal idioms have yet to be studied cognitively, in part due to a lack of multimodal datasets.  \cite{LakoffandJohnson1980, Hoffman1987WhatCR}. 
% \begin{figure}[h]
% \includegraphics[width=0.48\textwidth,height=12cm,keepaspectratio]{figures/Figure 1 - multimodal example.JPG}
% \caption{An advertisement for Toyota Corolla from the 60s with the idiom "get your hands on" and an image posted on a social media platform with the caption ``As fast as a cheetah''.}
% \label{fig:multimodal}
% \end{figure}
% There has been an increase in the use of text and vision modes over the past few years due to the growing usage of social media and mass media. The increased use of multimodality has created new challenges, which sometimes expand existing challenges from monomodality to multimodality.
\newline\noindent Due to its integral part in human communication, the detection and comprehension of multimodal figurative language is an important aspect of various multimodal challenges. Among these challenges are hate speech detection in memes \cite{detecting-hate-speech-in-multi-modal-memes}, fact-checking \cite{multimodal-fact-checking}, sentiment analysis \cite{SOLEYMANI20173}, humor recognition \cite{REYES20121, detecting-sarcasm-in-multimodal-social-platforms}, and identifying depression in social media posts \cite{yadav-etal-2020-identifying, multimodal-time-aware-attention-networks}. Figure \ref{fig:figurative-language-social-media} shows two photos posted on social media with metaphoric captions. In the left image, the caption reads, ``Jumped off the sinking ship just in time'', as this player left Chelsea - ``the sinking ship'', which is having a bad year, to join the leading team of the premier league, Arsenal. The right image was posted with the caption ``A performing clown'', as the person who is getting hit is a famous YouTuber who lost in a boxing match against a professional boxer. Multimodal figurative understanding is required to comprehend the metaphorical message being conveyed in these two posts. % In the task of sentiment analysis, the sentiment of the right post should be classified as ``negative'', but this cannot be achieved solely by text or image.
% In light of this rapidly growing trend, figurative language studies must be expanded from monomodality to multimodality.
Vision and Language Pre-Trained Models’ (VL-PTMs) understanding of figurative language combined with vision has not been thoroughly examined, if at all, partly due to the absence of large-scale datasets with ground truth labels of multimodal smilies, idioms, metaphors, etc.\\
%\begin{table}[ht]
\begin{center}
\scalebox{0.6}{
\begin{tabularx}{380pt}{| c X X |}
 \hline
 Figure & Example & Explanation \\ [0.5ex] 
 \hline\hline
 Metaphor & "You're a peach!" & The person being addressed is being equated with a peach, with the suggestion that the person is pleasing or delightful. The target concept is "person" and the source concept is "peach".\\ 
 \hline
 Open Simile & "Her heart is like stone" &  Inflexible and unfriendly or unkind disposition. The shared properties of "her heart" and "stone" are not explicitly revealed. The target concept is "her heart" and the source concept is "stone".  \\
 \hline
 Closed Simile & "The old man walk as slow as a snail" &  The old man's movement is compared to that of a snail, the shared property ("slow") is explicitly revealed. The target concept is "old man" and the source concept is "snail".\\
 \hline
 Idiom & "We're on the same page" &  Agreeing about something (such as how things should be done).  \\
 \hline
\end{tabularx}}

\end{center}
\caption{\ron{@Dafna Should we delete this/move it to appendix? If not, I think merging it with Figure 2 could be a good idea}Examples of simile, metaphor and idiom with a corresponding explanation.}
\label{table:figures-of-speech}
\end{table}%
\begin{figure}[tp]
\begin{center}
\includegraphics[width=0.5\textwidth,height=12cm,keepaspectratio]{figures/figure-figurative_language_in_social_media-bigger.jpg}
\end{center}
\caption{Two photos posted on social media. The left photo depicts football player Jorge Luiz Frello Filho Cavaliere wearing an Arsenal football club uniform. The right photo shows famous YouTuber Jake Paul taking a hit from professional boxer Tommy Fury during their boxing fight.}
\label{fig:figurative-language-social-media}
\end{figure}
\newline \noindent In this work, we introduce the IRFL dataset of idioms, metaphors, and similes with matching figurative and literal images. We leveraged a textual dataset of idioms and an extensive pipeline we developed to find possible figurative and literal idiom images. We annotated these images via Amazon Mechanical Turk using the UI seen in (Appendix~\ref{sec:annotation_ui}) to create a large-scale dataset of idioms' figurative and literal images. In addition, we collected metaphors and similes' figurative and literal images. We then used the IRFL dataset to create two novel tasks of figurative understanding and figurative preference to examine the figurative understanding of Vision and Language models. The figurative understanding task evaluates VL-PTMs' ability to understand the relation between an image and a figurative phrase. The task is to choose the image that best visualizes the figurative phrase out of X candidates.  Figure~\ref{fig:first-task-idiom-figurative} shows an example of the task for idiom, metaphor, and simile. The preference task examines VL-PTMs' preference for figurative images. In this task, the model needs to rank figurative images of different categories correctly. Figure~\ref{fig:second-task-idiom-figurative-vs-caption} shows the expected order versus the actual order of the idiom ``ruffle someone's feathers'' images based on the model scores. Finally, we experiment with generative models such as Dall-E and Stable Diffusion to examine their ability to generate figurative images for idioms. %\ron{Should the result be in intro?}%
% \yonatan{change quotation marks to ``THIS''} - Did not understand this
% \yonatan{figures in seperate files?} - Didn't understand this
% The figurative understanding task evaluates VL-PTMs' ability to understand the relation between an image and a figurative phrase. The task is to choose the image that best visualizes the figurative phrase out of X candidates.  Figure~\ref{fig:first-task-idiom-figurative} shows an example of the task for idiom, metaphor, and simile.
% The figurative understanding task evaluates VL-PTMs' ability to understand the relation between an image and a figurative phrase. The task is to choose the image that best visualizes the figurative phrase out of X candidates. Figure~\ref{fig:first-task-idiom-figurative} shows an example for the idiom ``Let the cat out of the bag'' - to disclose a secret, and examples with difficult distractors for the metaphor "Sea of bees" and the simile "The child is as proud as a peacock". Using partially literal images as distractors, the best model ($22\%$) fails to match human performance ($92\%$). This task goes beyond object detection and scene understanding, it requires a profound and rich language and cultural knowledge (idioms) in addition to commonsense, abstraction, general knowledge and the ability to decode the domains of figurative phrases and understand their relationships (metaphors and similes).
% \newline \noindent The preference task examines VL-PTMs' preference of figurative images over partially literal images. In this task, the model needs to rank figurative phrase images of different categories correctly. We suggest that Vision and language models should prioritize figurative images over partial literal images. Meaning that an image of someone disclosing a secret should receive a higher matching score for the idiom  ``let the cat out of the bag'' than an image of a bag. Figure~\ref{fig:second-task-idiom-figurative-vs-caption} shows the expected order versus the actual order of the idiom ``ruffle someone's feathers'' images based on the model scores. Assuming that the model understands the idiom and sees a figurative connection to an image, the task purpose is to measure how well the model comprehends it. This is done by comparing the figurative images' matching score to other images with a weaker relationship (partially literal). We find that the best model receives a $F_1$ score of $35-50$ depending on the figure of speech.
% \noindent Finally, we experiment with generative models such as Dall-E and Stable Diffusion to examine their ability to generate figurative images for idioms. We provide these models with idioms and their definitions as prompts and compare the results to our automatic pipeline. Our findings show that Stable Diffusion fails to generate figurative images even when given the definitions as input, while Dall-E succeeds in generating figurative images for the idioms' definitions.
\begin{figure}[h]
\includegraphics[width=0.45\textwidth,height=\textheight,keepaspectratio]{figures/Figure_5_ranking_task.JPG}
\caption{The Figurative and Partial Objects images of the idiom ``ruffle someone's feathers'' - ``To unease, cause discomfort to someone'' sorted from right to left by the CLIP-VIT-L/14 score. The images with the letter F are figurative, and the images with the letter P are partially literal. Green indicates correct rank, red indicates incorrect rank. The first row shows the ideal ranking order, while the second row shows the actual one. Figurative images with a green letter will appear in the first \#F image from the right, and Partial Objects images with a green letter will appear in the last \#P image. In this example, all of the models except LiT received 0 $F_1$ score.}
\label{fig:second-task-idiom-figurative-vs-caption}
\end{figure}

