Common sense is a topic of increasing interest [36]. Many commonsense reasoning tasks have been proposed, both in NLP \cite{zellers2019hellaswag,sap2019atomic,forbes2019neural,saha-etal-2021-explagraphs}, and computer vision \cite{Marino2019OKVqa, zellers2019recognition, Park2020VisualCOMETRA, bitton2023breaking} ranging from physical context \cite{bisk2020piqa} to social interactions \cite{Sap2019}. A particularly relevant line of work are abstractions \cite{Ji2022Kilogram}, associations \cite{Bitton2022WinoGAViLGA}, and analogies \cite{Bitton2022VASRVA}: understanding metaphors and similes often require association, abstraction, and general knowledge depending on the target and source concepts and the metaphorical message. For example, understanding the simile ``as stubborn as a mule'' requires the common sense knowledge that mules are stubborn (where in fact, they are not). The metaphor ``John is a fox'' uses the association of foxes with slyness.
% Aesop's ``The Fox and the Crow,'' painted the fox as a very crafty and cunning hunter. 