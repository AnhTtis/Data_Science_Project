To find figurative images for our search queries, we searched Google images \footnote{Images were searched with ``SafeSearch'' flag ``on'', and in ``United States'' region.}, taking up to 20 images per search query. The resulting images included a lot of ``garbage'' and problematic images with specific characteristics, such as images in which the idiom they were derived from and its definitions are written. These images were problematic because a model may see a connection between an idiom and a figurative image solely based on the textual signal that appears in it. Such images were filtered out by using OCR and a spelling tool to correct any spelling errors the OCR had. A large number of ``garbage'' images were found in the search results, including letters, postcards, newspapers, and images with mostly text in them. To tackle this problem, we used OCR to remove images with more than a couple of words and images with a text size bigger than 30\%. In addition, we removed images that looked like documents that the OCR failed to detect. Images that passed these filters were literal, figurative, or had no connection to the phrase they originated from. Next, we calculated the matching score of each image with its phrase and search query. Images with a ``phrase-image'' score that passed a certain literal threshold (Appendix~\ref{sec:literal_threshold}) were tagged as ``literal'', and from these images, we chose the top K images as literal candidates. From the non ``literal'' images, we chose the top K images with the highest ``search query-image'' score as Figurative candidates. We then annotated the relation between the figurative phrase and its Figurative and Literal candidates using the UI seen in Figure~\ref{fig:image-task-ui}.
