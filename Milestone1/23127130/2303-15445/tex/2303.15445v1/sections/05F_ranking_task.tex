To tackle vision and language models' strong preference toward partially literal images over figurative images, we introduce the preference task. The preference task is to rank the Figurative images higher than partially literal distractors based on the model matching score. First, we rank the figurative phrase images by their matching score from higher to lower, then we define two classes, $\#F$ which consists of the Figurative images, and $\#P$ which consists of the partially literal images. The model then predicts the first $\#F$ images as Figurative and the last $\#P$ images as partially literal images, the $F_1$ score of the model predictions is the preference task score. The results of the preference task are presented in Table \ref{tab:ranking-task}. We evaluate all figurative phrases that have images from both of the categories.
\begin{table}[h!]
\centering
\scalebox{0.65}{
\begin{tabular}{@{}lcccc@{}}
\toprule
Ranking  &     \multicolumn{2}{c}{Idioms}                   &  Metaphors         &   Similes \\ \midrule
                & Figurative Literal  & Figurative &                                        & \\ \midrule
CLIP-VIT-L/14   &       57            &    37                 & 26               & \textbf{44} \\
CLIP-VIT-B/32   &       54            &    36                 & 22               & 38 \\
CLIP-RN50       &       54            &    37                 & 25               & 38 \\
CLIP-RN50x64/14 &  \textbf{61}        &    39                 & \textbf{29}      & 43 \\
LiT             &  54                 & \textbf{56}           & 25               & 25 \\ \midrule
ViLT            &        -            & -                     & 23               & 34 \\ \midrule
\# of phrases   & 94                 & 149                    & 35               & 142  \\ \bottomrule
\end{tabular}}
\caption{The preference task performance, the scoring metric is $F_1$. The Idiom category is double-columned. The left column shows the score for Figurative Literal images, and the right column shows the score for Figurative images.}
\label{tab:ranking-task}
\end{table}

\noindent Models' scores on the preference task are low (<$61\%$). We expect models with proper figurative preference to achieve better results. Models' success in the Figurative Literal category can be attributed to the literal connections of the Figurative Literal images.
\begin{table}[h!]
\begin{center}
\scalebox{0.8}{

\begin{tabular}{@{}lcccc@{}}
\toprule
 Categories & Figurative & Figurative Literal &\\  \midrule
 Zero-Shot & $36$ & $54$ &\\ 
 Supervised & $68\pm3.8$ & $64\pm2.25$  &\\ \bottomrule
\end{tabular}} 


% \begin{tabularx}{230pt}{| c c c X|}
%  \hline
%  Categories & Figurative & Figurative Literal& \\ 
%  \hline
%  Zero-Shot & $36$ & $54$ &\\ 
%  \hline
%  Supervised & $73\pm1.8$ & $70\pm0.9$  &\\
%  \hline
% \end{tabularx}}

\end{center}
\caption{Supervised models performance. Results are the mean and standard deviation of the $F_1$ score
of five experiments.}
\label{tab:preference_task_supervision}
\end{table}

\\\noindent The supervised model, after fine-tuning, achieved a $68\pm3.8$ $F_1$ score on the Figurative category, almost double the zero-shot score of CLIP-ViT-B/32 ($36$). Additionally, the score in the Figurative Literal category was improved by $10\pm2.25$ points. These results align well with the observation that the fined-tuned understanding task model showed substantially moderate literal preference. Table~\ref{tab:preference_task_supervision} shows the fine-tuned model results. 
