\begin{table}[tp]
\centering
\scalebox{0.6}{
\begin{tabular}{@{}lccccccccc@{}}
\toprule
Categories   & \multicolumn{6}{c}{Figurative}      & \multicolumn{2}{c}{\begin{tabular}[c]{@{}l@{}}Figurative \\ Literal\end{tabular}}\\ \midrule
Candidates  &     \multicolumn{2}{c}{2}            &    \multicolumn{2}{c}{4}       &    \multicolumn{2}{c}{6}   & \multicolumn{2}{c}{4} \\ \midrule
Random  &     \multicolumn{2}{c}{50\%}             &    \multicolumn{2}{c}{25\%}    &    \multicolumn{2}{c}{16.6\%}   & \multicolumn{2}{c}{25\%} \\ \midrule
CLIP-VIT-L/14   &  64\%          & \textbf{87\%}   &  \textbf{46\%} & \textbf{71\%} & \textbf{33\%} & \textbf{53\%}   & 76\%          & \textbf{86\%} \\
CLIP-VIT-B/32   &  61\%          & 84\%            &  38\%          & 67\%          & 30\%          & \textbf{53\%}   & 65\%          & 82\%\\
CLIP-RN50       &  56\%          & 75\%            &  30\%          & 60\%          & 24\%          & 46\%            & \textbf{78\%} & \textbf{86\%} \\
CLIP-RN50x64/14 &  \textbf{67\%} & 79\%            &  38\%          & 67\%          & 27\%          & 51\%            & 69\%          & 85\%\\ 
LiT             &  57\%          & 61\%            &  22\%          & 22\%          & 19\%          & 18\%            & 17\%          & 24\%\\ \bottomrule

\end{tabular}}
\caption{Zero-shot models performance on the different configurations of the idiom understanding task with random candidates. Numbers are the percentage of instances annotated correctly. Bold numbers indicate the best model performance. There are three configurations for the Figurative Category with 2, 4, and 6 candidates. The table is double-columned. The left column shows the score for the phrase alone as input, and the right column shows the score for the phrase and its definitions as input.}
\label{tab:random-idiom-single-choice-results}
\end{table}
% \ron{Remove human performance from this tab and put it only on mix}
% \ron{Create very diverse task and start with it}
% \ron{Fine-grained analysis - distractors affect}
% \ron{to better understand the behavior of these models on the dataset, we examine other categories}




