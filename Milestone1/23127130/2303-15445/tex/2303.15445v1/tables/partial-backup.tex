\begin{table}[h]
\centering
\scalebox{0.7}{
\begin{tabular}{@{}lcccccc@{}}
\toprule
Categories   & \multicolumn{2}{c}{Figurative}\\ \midrule
Candidates  &                      2                           &                      4 \\ \midrule
CLIP-VIT-L/14   &  \begin{tabular}[c]{@{}l@{}}8\% 18\%\end{tabular} & \begin{tabular}[c]{@{}l@{}}1\% 7\%\end{tabular} \\
CLIP-VIT-B/32   &  \begin{tabular}[c]{@{}l@{}}6\% 20\%\end{tabular} & \begin{tabular}[c]{@{}l@{}}0\% 7\%\end{tabular} \\
CLIP-RN50       &  \begin{tabular}[c]{@{}l@{}}6\% 18\%\end{tabular} & \begin{tabular}[c]{@{}l@{}}1\% 8\%\end{tabular} \\
CLIP-RN50x64/14 &  \begin{tabular}[c]{@{}l@{}}7\% 23\%\end{tabular} & \begin{tabular}[c]{@{}l@{}}0\% 11\%\end{tabular} \\
LiT             &  \begin{tabular}[c]{@{}l@{}}\textbf{48\%} \textbf{50\%}\end{tabular} & \begin{tabular}[c]{@{}l@{}}\textbf{23\%} \textbf{26\%}\end{tabular} \\ \bottomrule

\end{tabular}}
\caption{Zero-shot models performance on the idioms understanding task with partially literal distractors. Numbers are the percentage of instances annotated correctly. There are two configurations with 2 and 4 candidates. The table is double-columned. The left column shows the score for the phrase alone as input, and the right column shows the score for the phrase and its definitions as input.}
\label{tab:partial-objects-idiom-single-choice-results}
\end{table}




