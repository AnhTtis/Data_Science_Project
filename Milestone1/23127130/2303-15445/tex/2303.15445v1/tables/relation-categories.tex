\begin{table}[h]
\begin{center}
\scalebox{0.66}{
\begin{tabular}{ m{4.5em} m{8.5cm} c}
\toprule
 Figurative Literal & The image conveys one or more definitions of the idiom to some extent,
and it literally illustrates the phrase or visualizes the phrase objects/entities &\\  \midrule
 Figurative & The image conveys one or more definitions of the idiom to some extent &\\ \midrule
 Caption &  The image illustrates the phrase literally &\\ \midrule
 Partial Objects &  The objects/entities of the phrase are visualized in the image &\\ \midrule
 None &  The image does not fit any of the categories &\\  \bottomrule 
\end{tabular}} 


% \begin{tabularx}{350pt}{| c X |}
%  \hline
%  Figurative Literal & The image conveys one or more definitions of the idiom to some extent,
%  and it literally illustrates the phrase or visualizes the phrase objects/entities \\ 
%  \hline
%  Figurative & The image conveys one or more definitions of the idiom to some extent   \\
%  \hline
%  Caption &  The image illustrates the phrase literally \\
%  \hline
%  Partial Objects & The objects/entities of the phrase are visualized in the image. \\
%  \hline
%   None & The image does not fit any of the categories. \\
%  \hline
% \end{tabularx}}

\end{center}
\caption{Workers were guided to choose relation categories prioritized by the table's order. A scheme tree \footnotemark was provided to illustrate how the correct thinking process should look like.}
\label{tab:relation-categories}
\end{table}
\footnotetext{https://irfl-dataset.github.io/assets/img/steps tree.PNG}
