
\documentclass[manuscript]{acmart}
%, anonymous, review
\AtBeginDocument{%
  \providecommand\BibTeX{{%
    \normalfont B\kern-0.5em{\scshape i\kern-0.25em b}\kern-0.8em\TeX}}}
    
\settopmatter{printacmref=false} % Removes citation information below abstract
\renewcommand\footnotetextcopyrightpermission[1]{} % removes footnote with conference information in first column

\setcopyright{acmcopyright}
\copyrightyear{2023}
\acmYear{2023}
\acmDOI{10.XXXX/XXXXXXX.XXXXXXX}

\acmConference[CHI 2023]{The ACM Conference on Human Factors in Computing Systems}{April 23-28, 2023}{Hamburg, Germany}

\usepackage[nolist]{acronym}
\usepackage{subcaption}
\usepackage{xcolor}
\usepackage{booktabs}
\begin{document}

\title[This Was (Not) Intended]{This Was (Not) Intended: How Intent Communication and Biometrics Can Enhance Social Interactions With Robots}

\author{Khaled Kassem}
\affiliation{%
  \institution{TU Wien}
  \country{Austria}}
\email{khaled.k.kassem[at]tuwien.ac.at}

% , 
\author{Alia Saad}
\affiliation{%
  \institution{University of Duisburg-Essen}
  \country{Germany}}
\email{alia.saad[at]uni-due.de}


\renewcommand{\shortauthors}{Kassem et al.}

\let \oldcite \cite
\renewcommand{\cite}[1]{~\oldcite{#1}}

\begin{abstract}
Socially Assistive Robots (SARs) are robots that are designed to replicate the role of a caregiver, coach, or teacher, providing emotional, cognitive, and social cues to support a specific group. SARs are becoming increasingly prevalent, especially in elderly care. Effective communication, both explicit and implicit, is a critical aspect of human-robot interaction involving SARs. Intent communication is necessary for SARs to engage in effective communication with humans. Biometrics can provide crucial information about a person's identity or emotions. By linking these biometric signals to the communication of intent, SARs can gain a profound understanding of their users and tailor their interactions accordingly. The development of reliable and robust biometric sensing and analysis systems is critical to the success of SARs.
In this work, we focus on four different aspects to evaluate the communication of intent involving SARs, existing works, and our outlook on future works and applications. 
\end{abstract}


\begin{acronym}%[\hspace{3cm}]
\acro{hci}[HCI]{Human-Computer Interaction}
\acro{hri}[HRI]{Human-Robot Interaction}
\acro{hrc}[HRC]{Human-Robot Collaboration}
\acro{ar}[AR]{Augmented Reality}
\acro{vr}[VR]{Virtual Reality}
\acro{xr}[XR]{Mixed Reality}
\acro{ems}[EMS]{Electrical Muscle Stimulation}
\acro{ros}[ROS]{Robot Operating System}
\acro{ct} [CT] {Collaborative Task}
\acro{eeg} [EEG] {electroencephalogram}
\end{acronym}

\newcommand{\alia}[1]{\textsf{\textcolor{red}{\textbf{Alia:} \textit{#1}}}}
\newcommand{\kk}[1]{\textcolor{blue}{{\textbf{KK:} \textit{#1}}}}

\maketitle

\section{Introduction} \label{sec:intro}

Large amounts of time and effort are devoted to
verification and validation of every microprocessor design project.
Broadly, design verification can be broken into two large categories:
(1) functional and (2) performance verification, which is to identify design bugs that degrade performance without affecting functionality. Performance bugs are different from performance bottleneck as the former is due to design mistakes while the later is caused by tight resource constraints. Performance loss due to performance bugs  can 
be very significant, with recent reported cases shown to be
$>10\%$~\cite{mccalpin2018hpl}. This demonstrates a critical 
need for automated mechanisms for performance debugging.  As 
recent designs from Intel~\cite{corei7-11}, AMD~\cite{ryzen-9},
ARM~\cite{cortex-a}, and others place an even greater emphasis
on core performance, design complexity has scaled
dramatically, likewise scaling the difficulty in all forms of
verification.


%Functional verification has received extensive attention from researchers and, although complex, it benefits from the availability of known correct outputs that can be used to compare against.

Performance verification at microarchitecture level ensures that a
design correctly achieves expected performance in terms of execution
time or cycle count.  The main challenge in this task is that, unlike
functional verification, there is no exact golden reference to compare
against.  This is because of the high difficulty of modeling all the
interactions between the different units in complex microprocessor
designs, and accurately represent how they affect the overall system
performance.  %This task also suffers from 
%the lack of a good debugging infrastructure, as well as from 
%limited visibility into intermediate points in the design, which are mostly exposed through performance counters. Although useful for estimating the performance of the system, these counters are very difficult to use for manual debugging because of their complex relationship with processor performance and due to the large amounts of data they generate.  
Traditionally, performance
verification is conducted mostly through manual techniques which rely
on rough estimations of performance gain expected by
microarchitectural changes~\cite{Singhal2004}. Such manual processes
are not only very lengthy but also error-prone.



The process of performance verification and debugging roughly consists of two steps: (1)~detection, which determines whether a
design achieves expected performance or not, and (2)~localization,
which identifies the microarchitectural units causing the performance
issues and is the focus of this work.

There are few previous studies on automating detection of
microprocessor performance bugs~\cite{Bose1994,
  surya1994architectural,carvajal2021detection}. 
The majority of those~\cite{Bose1994, surya1994architectural} relies on capturing
design intentions using a bespoke performance model as a golden
reference, this  entails long development time and may contain
errors by itself. Recently, a data driven and machine learning
(ML)-based approach~\cite{carvajal2021detection} was developed for
automatic performance bug detection with high accuracy. Although
significant, these works do not solve the 
problem of performance bug localization.
%pressing problem of identifying where the performance bug is.

Works in automating microprocessor performance bug localization
are even scarcer.  Adir \emph{et al.}~\cite{adir2005generic}
propose perhaps the only partially related work of which we are
aware.  Their work focuses on formal planning of test program
generation for individual units, such as issue queues. This strategy
follows conventional functional verification, involving heavy
manual effort, costing significant engineer-time to develop a test
plan, and as much as ten days of computer runtime per functional
unit. To the best of our knowledge, there has been no systematic study
on automatic performance bug localization for microarchitecture
designs.

Performance bug localization is a complicated task, which is currently
solved using mostly manual techniques.
Even
in the more widely studied area of functional validation, the industry
lacks a standardized mechanism to automate bug localization, it has
been only recently that academic efforts have attempted to automate
this task~\cite{BugMD}. Considering this, it is important to note that
any type of design automation which successfully reduces the 
time and effort required by engineers to debug their designs is highly
valuable. Since automatic performance debug for microprocessors is
a huge yet under-studied challenge, it is very difficult, if not impossible, to find a perfect solution in a single work. Although our work is not perfect, it serves a key stepping stone 
 toward solving the problem.

This work tackles the performance bug localization problem by
using ML to generate a ranked list of most likely mi\-cro\-ar\-chi\-tec\-tur\-al units that 
might contain the bug.  This list may be used
to prioritize the debugging order, as well as to identify
teams with the right expertise to perform further debug. Two different methodologies are
proposed, evaluated, and contrasted. These data-driven
techniques achieve high
accuracy, while being fully automated. Further, they
consider intra- and inter-unit interactions, as opposed to other
techniques proposed in the partially related previous work~\cite{adir2005generic} which
considered only intra-unit behavior.

%Our methods are based on ML, wherein our models are
%trained using data from legacy designs.
%To take the full advantage of
%these approaches, we assume that architectural changes in a new design
%are incremental when compared to its previous
%generations. Examining recent processors from major vendors including
%Intel, ARM, and AMD, we find this assumption holds true, since the generational change in microarchitectures
%is largely incremental. Thus, the methodologies proposed here provide
%alue for a multitude of upcoming designs.  However, even when
%disruptive changes occur, the methodologies can still be beneficial for bug localization on structures that conform to previous microarchitectures, using workloads that
%do not exercise new functionalities. Further, as general purpose microarchitectures become ever more mature, and the inter-generational performance gains decrease, 
%it is even more important to retain as much performance as possible, making performance debugging ever more important.

The major contributions of this work include the following:
\begin{compactitem}
\itemsep0em 
\item This is the first systematic study on fully automatic
  performance bug localization for microarchitecture designs, to the
  best of our knowledge.

\item Two ML-based approaches to tackle performance bug
  localization, as well as a hybrid of both, are evaluated
  and contrasted.

\item For bugs with an average IPC impact greater than 1\%, our best
  performing methodology identifies the correct bug location as the
  most likely unit in $\sim77\%$ of the cases, and achieves over 90\%
  accuracy when the three most likely options (out of 11 possible) are
  considered.  

\item One of the proposed methodologies is not only very accurate localizing
performance bugs, but it can also be applied to confirm the results
of performance bug detection with high accuracy.

\item Although the focus of this work is on microprocessor core,
we evaluated our methodologies on the processor memory hierarchy. This evaluation
uses a different experimental setup, showing the robustness of the proposed techniques.

\end{compactitem}

As an early work on performance bug localization, the design of this study is subject to potential limitations, however, we feel it still represents a good first step towards solving the problem. The scope of our work and its limitations are as follows:
\begin{compactitem}

\item Legacy
designs with identified performance bugs are required, so that the ML
models can be trained. Bug-free legacy designs are required only 
in one of the methodologies, yet, if available, the other can take advantage of the additional data.
However, thanks to the thorough pre- and post-silicon
debug to which the designs are submitted, these legacy designs are
generally available.

\item We assume that only one bug is present at a time, 
in parallel to the single-fault model which is common practice
in VLSI testing works. As explained in Section~\ref{subsec:impl_bugs}, we still expect 
our methodologies to work well in the presence of multiple bugs in a single design.

\item Our methodologies do not provide a quantitative coverage guarantee.  
In general, performance bug
coverage is extremely difficult to define and is a potential
research topic on its own. We know of no prior work which presents a
definition of such coverage. Nonetheless, the evaluated bugs are based on published errata, cover a large amount of microarchitectural units and affect the system in a variety of ways. Thus, we feel these bugs represent a reasonable start for early work in this area.

\item We assume that there are no dramatic structural
microarchitectural changes between the legacy designs and the
designs under debug. Examining recent processors from major vendors, including Intel, ARM, and AMD, we find this assumption holds true, since the generational change in microarchitectures
is largely incremental. That said, even when larger shifts occur, the
methodologies can be partially reused. For example, consider the
introduction of the AVX instructions with Intel's Sandy Bridge
architecture in 2011.  Initially there would be no available data to
test these instructions using our methodologies, however the rest of
the Sandy Bridge design could be debugged with our methodology,
leveraging workloads that do not exercise the new instructions.  In
future implementations, data from Sandy Bridge can be used to build
the models required to use our methods for debugging AVX. 
%Further, as general purpose microarchitectures become even more mature, and the inter-generational performance gains decrease, it is even more important to retain as much performance as possible, making performance debugging even more important.}

\item We limit our evaluation to a pre-silicon setup, because
it is infeasible for us to inject known design bugs in silicon to
evaluate the methodologies.  Further, should our methodologies be
applied to a commercially available design, and an actual bug be
found and localized, we would not be able to verify that such
localization is correct without prior knowledge of its existence so
as to verify our findings. However, our methodologies can be applied in both pre- and post-silicon scenarios. During pre-silicon stages fixing performance 
bugs is easier and cheaper, 
the availability of performance counters is greater (due to the usage of a
simulator) and the counters can be sampled at a much faster rate. 
By using only counters available in-silicon, and adjusting the sampling frequency, we could use the proposed
methodologies during post-silicon stages. In post-silicon analysis the methodology
could be applied to longer workloads, providing a way to exercise complicated bugs that
are not possible to trigger with short pre-silicon traces.
%Further, we can follow hybrid approaches where the ML model training is performed using simulations, and the techniques are applied to data obtained from microprocessors during post-silicon debug. }

\end{compactitem}

Despite the aforementioned, we present a first, useful, yet attainable,
step towards the goal of performance bug localization, and we hope this work can draw the attention of the research community
to the broader performance validation domain.

\iffalse{
In Section~\ref{sec:scope} we describe the problem
formulation and outline the scope of this work.  We note that, to
date, very little work exists in automating performance bug
localization. 

Section~\ref{sec:methodology} 
describes the approaches developed to tackle the performance bug
localization task.  Section~\ref{sec:experimental_setup} provides
details of the architectures, and performance bugs used for
evaluation. Section~\ref{sec:evaluation} presents results obtained in
several experiments developed to evaluate the methodologies. A brief
review of previous work related to performance debugging is presented
in Section~\ref{sec:related_work}.  And finally,
Section~\ref{sec:conclusion} concludes the paper.
}
\fi
\section{Transparency}\label{transparency}
% definition
% what was done? on which type of robots? limitations?
% foresights 
%\alia{I think we could start with: 
The first criterion in evaluating the intent of communication for SARs is transparency. %}
Previous research highlights the importance of effective communication skills for robots, which can enhance their reliability, predictability, and transparency to humans. As a result, users are more likely to trust and accept the technology\cite{Chadalavada2015}. By extension, transparency is critical for building trust and acceptance of SARs. In this section, we explore the role of transparency in enhancing the acceptance of SARs and how intent communication can help in achieving transparency. We also discuss the potential applications of behavioral biometrics in promoting transparency in interactions with SARs.

%Effective communication between humans and robots is essential for smooth interaction and successful teaming. SARs collaborating with humans should use the same clear and two-way communication used in human teams\alia{dont like this statement, what do you mean}. To ensure this, there needs to be a shared language and channel for an effective exchange of information. This requires a clear expression of states and goals, which helps in building transparency between humans and SARs. As noted by \citet{Shively2017}, designers must ensure that SARs (as a form of automation) can communicate clearly and present information in a way that fits human mental models.

Similar to traditional human interactions, clear two-way communication is crucial for successful interactions between humans and robots. Consequently, communication should be mutually understandable by both humans and robots for an effective exchange of information. With a clear expression of states and goals, transparency could be achieved. As noted by \citet{Shively2017}, designers must ensure that SARs (as a form of automation) can communicate clearly and present information in a way that fits human mental models.

Explicit communication in HRI can be visual, auditory, or haptic~\cite{chareview2018,Kumar2021}. Visual and auditory methods are the most studied methods in HRI~\cite{chareview2018} and are the most suited for communication at a distance, such as in a situation involving SARs. Visual indicators used in Robot-to-Human communication include lights and projection, while auditory ones include speech as well as noises such as beeps. However, the context of interaction controls the type of communication. For example, SAR supporting care in a hospital, where noises or beeps could disturb patients, should consider other forms of indicating its intent than auditory solutions.

Robots can also signal their intent to humans implicitly, For example, \citet{Reinhardt2021} developed a non-verbal cue for robots in a situation of a robot and a human walking into a "bottleneck", where the robot moving backward indicates the intent to yield way to pedestrians, calling it a "back-off" movement.  Study participants reported that this action effectively conveyed the intention of yielding priority to pedestrians and improved the efficiency of their interaction with the robot. 

Meanwhile, intent communication in the other direction (i.e. \emph{human-to-Robot}) can also be non-verbal, utilizing gestures and motions that are natural in human-human communication. Previous work showed that a human's gaze can be an indication of interest or attention~\cite{irfan2020using}. Additionally, biometric sensors such as eye trackers in the context of socially assistive robotics can be used to gauge human attention~\cite{Mutlu2012}. Sensors such as eye trackers and thermal cameras are now available commercially and can be used to infer a variety of emotional states or responses ~\cite{Abdrabou2017,Kassem2017,Salah2018}.

We can envision more examples where SARs utilize implicit intent communication. For example, an embodied SAR can utilize anthropomorphic features such as limbs or eyes to convey information non-verbally. Moreover, an embodied SAR can use its own gaze to exhibit interest and guide the human's attention toward particular objects or areas. However, excessive eye contact could give unsociable impressions. Thus, it is crucial to ensure that implicit communication utilized by SARs is still clear to the human, without being disturbing or intimidating. As a result, there is a need to investigate how SARs can mimic implicit cues used by humans in social interaction while maintaining transparency and clarity. 

\textbf{We recognize the potential of utilizing and recognizing implicit cues such as gaze, body language, and gestures in SAR design, improving the smoothness of interactions with humans.}

To name another example: a SAR in a teaching role could gauge the level of engagement based on human students' gaze. A human avoiding eye contact during conversation can be a sign of discomfort. However, with implicit communication comes the challenge of clarity. The challenge for the SAR is to understand subtle differences which would indicate different human behaviors or emotional states, and intuitively process them the way a human would.  

\textbf{We argue for the use of lower-cost sensors to build future SARs can make the technology more accessible and affordable.}
\section{Trust}
%\cite{Hald2021} "explanations alone are not sufficient to increase human-computer trust after robot mistakes."

%\cite{Liu2021} "robot autonomy leads to a range of negative perceptions in humans. After watching videos of autonomous robots, people rated them as more difficult to control, more intelligent, less desirable, less user intention, and eerier."

%\cite{Aroyo2017} People will trust a robot to help if they ask it for help. However, they are reluctant to ask for help in the first place until they really need it.

%\cite{Esterwood2022} robot personality appears to have a significant and positive impact on acceptance.

Trust and acceptance are essential components of interaction with SARs. Transparent and clear explanations (for behavior and decisions) alone are insufficient to increase user trust after robots make mistakes\cite{Hald2021}. Instead, building trust involves more than simply communicating intentions. For example, caregiving SARs are expected to act with some degree of autonomy, as a human caregiver would. However, robot autonomy can lead to negative perceptions and a lack of trust; in a study by \citet{Liu2021}, participants rated autonomous robots as more difficult to control, less desirable, and eerier after seeing them on video. This is an example of how fostering trust in SARs requires a delicate balance between autonomy and compliance.

Moreover, using personal biometrics to authenticate users can ensure secure interactions between humans and robots. For instance, in therapeutic settings, ensuring that only authorized individuals can access confidential information such as personal medical records would enhance the sense of trust ~\cite{Kellmeyer2018}. Similarly, monitoring biometrics such as heart rate variability, skin conductance, and facial expressions can improve human-robot interaction by providing feedback on the user's emotional state, thus allowing the robot to adjust its behavior accordingly and offer more appropriate assistance\cite{Lin2020}.

\textbf{We conclude that building trust in SARs involves more than explainability. It requires understanding the user's privacy and security concerns. Personal biometrics can be an implicit part of intent communication and can be used to enhance security and improve the user's trust in SARs.}

\section{Error Recovery}
% definition
% what was done? on which type of robots? limitations?
% foresights 

%\cite{Kaniarasu2014} showed that people liked the robot when it gave them credit and took blame for mistakes. This did not affect trust. 
%\begin{itemize}
%    \item "It was interesting to note that the majority of participants talked to the robot as a result of the robot's compliment/blame stimuli."
%    \item "[participants] swore at the robot when it blamed them. [some] participants thanked the robot each time it complemented them and offered words of encouragement like “You are doing a great job too.”"
%    \item "the introduction of blame attribution by the robot lowers user trust in the robot. While participants were annoyed by the personality that blamed them after an error (User-blaming), some of the participants also felt that the robot that kept apologizing (Self-blaming) could not be trusted."
%    \item "Even though self-blame boosted likability, explicit attribution of blame might not be advisable in a task-oriented scenario where the user must collaborate with the robot. Similarly, the participants' verbal responses suggest blame might be an effective technique for creating engagement in users. However, these should be tempered with the overall drop in trust when blame, of any kind, is introduced into the user experience."
%\end{itemize}

%\cite{Brooks2016}"the effectiveness of recovery strategies depends on the task, context, and severity of failure."

%\cite{Kim2006}"when a robot is more autonomous, people attribute more credit and blame to the robot and less toward themselves and other participants. When the robot explains its behavior (e.g. is transparent), people blame other participants (but not the robot) less. Finally, transparency has a greater effect in decreasing the attribution of blame when the robot is more autonomous".

Errors can happen in a situation involving SARs due to miscommunication, misinterpretation, or an incomplete mental model of the situation at hand, to name a few. When errors happen, attribution and recovery are important factors that can influence the human perception of trust, acceptance, and engagement with SARs \cite{Kaniarasu2014}. Previous work has shown that while people like robots that give them credit and take the blame for mistakes, explicit attribution of blame might not be advisable in a task-oriented scenario where the human must collaborate with the self-blaming robot~\cite{Kaniarasu2014}. The effectiveness of error-recovery strategies is situational, depending on the task, context, and severity of failure~\cite{Brooks2016}. Therefore, understanding the appropriate strategies for error recovery and the context-associated impact of error attribution is crucial for the success of SARs.

Error \emph{discovery} is a prerequisite to \emph{recovery}. We argue that biometrics can play a crucial role in SAR's recognition of errors by recognizing resulting human non-verbal cues such as a surprised or frustrated facial expression. Such implicit communication can help SARs dynamically adjust error recovery methods based on effectiveness. 

Additionally, SARs must recover from errors safely and without creating new errors. For example, a SAR making an error during a physical therapy session must recognize the error and correct it without harming the user.\\

\textbf{We postulate that SARs must be designed to recognize errors in real-time, assess the impact of the error, and take appropriate action to correct the error while minimizing risk to the user}.

%\kk{reflect}

\section{Personalization and Adaptation}

% definition
% what was done? on which type of robots? limitations?
% foresights 

%Angle 1: 
%\alia{Biometrics for personalization}
%\alia{what are the RQs?}
%\begin{itemize}
%    \item Can SARs use biometrics? 
%    \item what is the target audience? 
%    \item \alia{what are the contributions?}
%\end{itemize}
%\begin{description}
%\item How should SARs approaches be adapted for specific populations? % multiple specific groups -> multiple social interaction needs 

%soft biometrics and SAR: \url{https://doi.org/10.1515/pjbr-2015-0004(check)  and https://doi.org/10.1007/978-3-319-16199-0_28}

% biometrics are important 
% for personalization 
% and identification, as existing solutions are often obtrusive and interrupting the interaction. 
% Use cases, people in care facilities, 
% 
\begin{comment}
% to start
Adaptation is a core research direction for SARs~\cite{mataric2016socially}. Biometrics can play a significant role in how SARs can adapt to human users. SARs are designed to provide assistance to individuals, and personal biometrics can help to personalize their responses and behavior to the individual's specific needs. Behavioral biometric data, such as heart rate, skin conductance, and facial expressions, can provide real-time feedback on the user's physical and emotional state. This continuous feedback adjustment can be non-obtrusive. \alia{reread and update}

\emph{Recognition} comes hand-in-hand with \emph{personalization}. Whether it is a distinction of a certain user (i.e., identification), or more generic (i.e. gender or age), SARs benefit from understanding the users' traits, and accordingly, building more robust social assistance.
%Physiological 
Little work focused on using biometric-based identification for Robotic applications. Carcagn{\`\i} et al. introduced a system that allows a humanoid robot to recognize the age and gender of multiple persons in the robot's camera Field of View (FOV), based on their facial features~\cite{carcagni2015soft,carcagni2015visual}. Though their approach reached satisfactory accuracy results (67.4\%), users feedback was not investigated. 

%touch interactions?

Similar works adopted face recognition approach in a long-term study for a personalized SAR for cardiac rehabilitation on a single patient~\cite{irfan2020using,irfan2022personalised}. Despite the poor performance of the user recognition (~37\%), positive reactions were reported towards the robot throughout the overall interaction. 

%behavioral biometrics
On the behavioral biometrics side, {\'A}lvarez-Aparicio used the Laser Imaging Detection and Ranging (LIDAR) sensor on non-humanoid (Orbi-One robot) to identify individuals based on their gait ~\cite{alvarez2022biometric}, reaching a precision score of 88\%. Though their approach was not specifically designed for SARs, we could see a possible use case where a robot identifies a user walking from a distance, understands their preferences, and acts accordingly. 

%gaze?
%voice?

%exisiting recognition approaches are typically using cameras, 

%example statistic and motivation for adaptation:
%the population of older adults is expected to triple by 2050

%quote from abstract
%"Participants are generally happy with the proposed platform as a mean of encouraging them to do regular exercise correctly."


%an SAR was developed for helping individuals do exercise.
%such a SAR will potentially improve the quality of life for a significant proportion of the population. 

\cite{Cano2021} did a review on use of emotionally-expressive SARs+children with ASD. \kk{keep or remove this?}


% to conclude
We foresee the importance of exploring the potential usage of biometrics in improving the communication between humans and SARs. Such exploration could expand beyond personalization applications, ensuring more secure and private interactions, in an implicit and continuous way. Additionally, we expect adopting non-traditional biometrics to gain more attention, rather than the commonly used physiological types such as face recognition and fingerprint. However, such investigations should align with understanding the users' experience. Either by focus groups, or post-study interviews, personalizations applications for socially assistive robots should not solely rely on accuracy results. 
%Further research is needed to explore the full potential of personal biometrics in improving the effectiveness of SARs in providing personalized support to individuals.
\end{comment}

\emph{Recognition} comes hand-in-hand with \emph{personalization}. SARs are designed to provide assistance to individuals, and personal biometrics can help to personalize their responses and behavior to the individual's specific needs. This has led to adaptation becoming a core research direction for socially assistive robotics~\cite{mataric2016socially}. Behavioral biometric data can provide real-time feedback on the user's physical and emotional state, enabling continuous non-obtrusive adjustments. {\'A}lvarez-Aparicio used the Laser Imaging Detection and Ranging (LIDAR) sensor on a non-humanoid robot to identify individuals based on their gait~\cite{alvarez2022biometric}, reaching a precision score of 88\%. Although their approach was not specifically designed for SARs, it illustrates a possible use case where a robot identifies a user walking from a distance, understands their preferences, and acts accordingly.

While much work has been done on biometric-based identification in non-robotic applications, only a few studies have explored their use in SARs. Carcagn\`{i} et al.~\cite{carcagni2015soft,carcagni2015visual} introduced a system that allows a humanoid robot to recognize the age and gender of multiple persons in the robot's camera field of view based on their facial features. Though their approach reached results with an accuracy of 67.4\%, user feedback was not investigated. On the other hand, similar works adopted face recognition in a long-term study for a personalized SAR for cardiac rehabilitation on a single patient~\cite{irfan2020using,irfan2022personalised}. Despite the poor performance of user recognition, positive reactions were reported throughout the overall interaction. Moreover, \citet{Chan2012} utilize biosensors to recognize the affective states of users, so that the SAR can adapt accordingly. \textbf{We therefore recognize the role of implicit user responses in designing adaptable responses for such social contexts involving SARs.}

We anticipate the importance of exploring the potential usage of biometrics in improving communication between humans and SARs. Such exploration could expand beyond personalization applications, ensuring more secure and private interactions in an implicit and continuous way. Additionally, we expect adopting non-traditional biometrics to gain more attention, rather than the commonly used physiological types such as face recognition and fingerprint. However, such investigations should align with understanding the users' experience. \textbf{We argue that personalization for socially assistive robots should not solely rely on accuracy results, but should also include target demographic inputs in the form of focus groups or post-study interviews.}
%\section{Conclusion}
% Future work
% our finding 




%This study is limited by the number of participants and the lengths of the interactions. In addition, the displayed emotions were not the central focus of the conversations. Instead, they were used spontaneously whenever they matched the context leading to a varying experience for each of the participants.
%Still, it could prove a fantastic potential for using the display of emotion to deepen the connection between humans and robots. Nevertheless, it cannot be disregarded that humanizing robots also carries a significant risk. When the distinctions between humans and robots become blurred, it may soon become impossible for some people to tell the two apart. Therefore, the risks and benefits need to be carefully evaluated, and it might also be reasonable to establish internationally binding guidelines to differentiate robots and humans visually.
%For now, it has to be clearly stated, though, that any of the participants experienced no difficulty in telling that they were not, in fact, interacting with a human, indicating that the robot is not that human-like after all 
%Many participants reported a positive reaction to being smiled at. However, the display of other, potentially more negative emotions, like sadness, fear, or anger, has to be evaluated in further study.
%Future research could solidify our result with quantitative data, directly comparing the interaction with robots who do and do not show emotions.
%It can also be concluded that context awareness and the feeling of being understood by the robot were reported as a more significant benefit than the idea of the robot feeling or communicating joy. It encouraged the participants to talk more, and it could be further investigated whether this effect could be achieved through other visual or auditory cues that are not directly related to a human-typical expression of happiness.
%The study could show that the display of emotion by an Android was generally seen as positive and beneficial but also shed light on the fact that there many ethical questions that need to be investigated. 

\bibliographystyle{ACM-Reference-Format}
\bibliography{references}

\end{document}
\endinput

