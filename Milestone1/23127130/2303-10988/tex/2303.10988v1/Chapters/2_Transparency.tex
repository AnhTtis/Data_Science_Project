\section{Transparency}\label{transparency}
% definition
% what was done? on which type of robots? limitations?
% foresights 
%\alia{I think we could start with: 
The first criterion in evaluating the intent of communication for SARs is transparency. %}
Previous research highlights the importance of effective communication skills for robots, which can enhance their reliability, predictability, and transparency to humans. As a result, users are more likely to trust and accept the technology\cite{Chadalavada2015}. By extension, transparency is critical for building trust and acceptance of SARs. In this section, we explore the role of transparency in enhancing the acceptance of SARs and how intent communication can help in achieving transparency. We also discuss the potential applications of behavioral biometrics in promoting transparency in interactions with SARs.

%Effective communication between humans and robots is essential for smooth interaction and successful teaming. SARs collaborating with humans should use the same clear and two-way communication used in human teams\alia{dont like this statement, what do you mean}. To ensure this, there needs to be a shared language and channel for an effective exchange of information. This requires a clear expression of states and goals, which helps in building transparency between humans and SARs. As noted by \citet{Shively2017}, designers must ensure that SARs (as a form of automation) can communicate clearly and present information in a way that fits human mental models.

Similar to traditional human interactions, clear two-way communication is crucial for successful interactions between humans and robots. Consequently, communication should be mutually understandable by both humans and robots for an effective exchange of information. With a clear expression of states and goals, transparency could be achieved. As noted by \citet{Shively2017}, designers must ensure that SARs (as a form of automation) can communicate clearly and present information in a way that fits human mental models.

Explicit communication in HRI can be visual, auditory, or haptic~\cite{chareview2018,Kumar2021}. Visual and auditory methods are the most studied methods in HRI~\cite{chareview2018} and are the most suited for communication at a distance, such as in a situation involving SARs. Visual indicators used in Robot-to-Human communication include lights and projection, while auditory ones include speech as well as noises such as beeps. However, the context of interaction controls the type of communication. For example, SAR supporting care in a hospital, where noises or beeps could disturb patients, should consider other forms of indicating its intent than auditory solutions.

Robots can also signal their intent to humans implicitly, For example, \citet{Reinhardt2021} developed a non-verbal cue for robots in a situation of a robot and a human walking into a "bottleneck", where the robot moving backward indicates the intent to yield way to pedestrians, calling it a "back-off" movement.  Study participants reported that this action effectively conveyed the intention of yielding priority to pedestrians and improved the efficiency of their interaction with the robot. 

Meanwhile, intent communication in the other direction (i.e. \emph{human-to-Robot}) can also be non-verbal, utilizing gestures and motions that are natural in human-human communication. Previous work showed that a human's gaze can be an indication of interest or attention~\cite{irfan2020using}. Additionally, biometric sensors such as eye trackers in the context of socially assistive robotics can be used to gauge human attention~\cite{Mutlu2012}. Sensors such as eye trackers and thermal cameras are now available commercially and can be used to infer a variety of emotional states or responses ~\cite{Abdrabou2017,Kassem2017,Salah2018}.

We can envision more examples where SARs utilize implicit intent communication. For example, an embodied SAR can utilize anthropomorphic features such as limbs or eyes to convey information non-verbally. Moreover, an embodied SAR can use its own gaze to exhibit interest and guide the human's attention toward particular objects or areas. However, excessive eye contact could give unsociable impressions. Thus, it is crucial to ensure that implicit communication utilized by SARs is still clear to the human, without being disturbing or intimidating. As a result, there is a need to investigate how SARs can mimic implicit cues used by humans in social interaction while maintaining transparency and clarity. 

\textbf{We recognize the potential of utilizing and recognizing implicit cues such as gaze, body language, and gestures in SAR design, improving the smoothness of interactions with humans.}

To name another example: a SAR in a teaching role could gauge the level of engagement based on human students' gaze. A human avoiding eye contact during conversation can be a sign of discomfort. However, with implicit communication comes the challenge of clarity. The challenge for the SAR is to understand subtle differences which would indicate different human behaviors or emotional states, and intuitively process them the way a human would.  

\textbf{We argue for the use of lower-cost sensors to build future SARs can make the technology more accessible and affordable.}