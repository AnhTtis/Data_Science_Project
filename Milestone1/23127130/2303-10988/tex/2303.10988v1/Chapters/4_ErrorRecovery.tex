\section{Error Recovery}
% definition
% what was done? on which type of robots? limitations?
% foresights 

%\cite{Kaniarasu2014} showed that people liked the robot when it gave them credit and took blame for mistakes. This did not affect trust. 
%\begin{itemize}
%    \item "It was interesting to note that the majority of participants talked to the robot as a result of the robot's compliment/blame stimuli."
%    \item "[participants] swore at the robot when it blamed them. [some] participants thanked the robot each time it complemented them and offered words of encouragement like “You are doing a great job too.”"
%    \item "the introduction of blame attribution by the robot lowers user trust in the robot. While participants were annoyed by the personality that blamed them after an error (User-blaming), some of the participants also felt that the robot that kept apologizing (Self-blaming) could not be trusted."
%    \item "Even though self-blame boosted likability, explicit attribution of blame might not be advisable in a task-oriented scenario where the user must collaborate with the robot. Similarly, the participants' verbal responses suggest blame might be an effective technique for creating engagement in users. However, these should be tempered with the overall drop in trust when blame, of any kind, is introduced into the user experience."
%\end{itemize}

%\cite{Brooks2016}"the effectiveness of recovery strategies depends on the task, context, and severity of failure."

%\cite{Kim2006}"when a robot is more autonomous, people attribute more credit and blame to the robot and less toward themselves and other participants. When the robot explains its behavior (e.g. is transparent), people blame other participants (but not the robot) less. Finally, transparency has a greater effect in decreasing the attribution of blame when the robot is more autonomous".

Errors can happen in a situation involving SARs due to miscommunication, misinterpretation, or an incomplete mental model of the situation at hand, to name a few. When errors happen, attribution and recovery are important factors that can influence the human perception of trust, acceptance, and engagement with SARs \cite{Kaniarasu2014}. Previous work has shown that while people like robots that give them credit and take the blame for mistakes, explicit attribution of blame might not be advisable in a task-oriented scenario where the human must collaborate with the self-blaming robot~\cite{Kaniarasu2014}. The effectiveness of error-recovery strategies is situational, depending on the task, context, and severity of failure~\cite{Brooks2016}. Therefore, understanding the appropriate strategies for error recovery and the context-associated impact of error attribution is crucial for the success of SARs.

Error \emph{discovery} is a prerequisite to \emph{recovery}. We argue that biometrics can play a crucial role in SAR's recognition of errors by recognizing resulting human non-verbal cues such as a surprised or frustrated facial expression. Such implicit communication can help SARs dynamically adjust error recovery methods based on effectiveness. 

Additionally, SARs must recover from errors safely and without creating new errors. For example, a SAR making an error during a physical therapy session must recognize the error and correct it without harming the user.\\

\textbf{We postulate that SARs must be designed to recognize errors in real-time, assess the impact of the error, and take appropriate action to correct the error while minimizing risk to the user}.

%\kk{reflect}
