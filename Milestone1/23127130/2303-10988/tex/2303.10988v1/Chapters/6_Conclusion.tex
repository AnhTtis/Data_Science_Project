\section{Future Outlook: A Hospital Scenario}
Consider a scenario in which a SAR is utilized in a pediatric hospital to provide support for children during medical procedures. We examine how the preceding sections can guide the SAR's design:
\begin{enumerate}
\item \textbf{Transparency}: The SAR can offer clear explanations of its actions to both children and their caregivers to facilitate understanding of the robot's decisions. It uses implicit cues such as gaze and tone of voice to make the interaction smoother.
\item \textbf{Trust}: Incorporating a friendly demeanor into the SAR's design can establish trust by providing words of encouragement and reassurance during procedures. To maintain patient confidentiality, biometrics are employed, and sensitive or personal information is accessible only to caregivers.
\item \textbf{Error-recovery}: The SAR recognizes negative reactions to its suggestions and adapts its behavior accordingly.
\item \textbf{Adaptation}: Monitoring biometric data, such as facial expressions, allows the SAR to tailor its behavior to better suit a child's emotional state and enhance their overall experience. The SAR utilizes age-appropriate language when communicating with younger patients, as opposed to adult caregivers.
\end{enumerate}

\section{Conclusion}

This paper examines the intersection of research in personal biometrics and intent communication, and explores how findings in these fields can be leveraged to improve the usability and acceptability of SARs. Based on our reflection on previous work, we identify several areas for future research, including \textbf{developing guidelines for transparent communication in SARs that can be applied universally, adapting implicit cues for unique situations such as those involving cultural differences, creating biometric privacy and security guidelines for SARs, incorporating commercial biosensors in SAR design for accessibility and affordability}, and \textbf{allowing target user groups to influence SAR design through, for example, focus groups}. We believe that research in these directions can produce results that enhance the acceptability of SARs.





\begin{comment}
In this paper, we reflected on prominent fields of research in HCI and HRI, and how different findings can be applied to enhance the usability and acceptability of SARs. We highlight some areas for future research work in SARs including:
\begin{itemize}
    \item Balancing implicit communication with transparency to avoid discomfort or misunderstandings for humans.
    \item Developing guidelines for transparent communication in SARs that can be universally applied.
    \item Adapting interpretation of implicit cues for unique situations, e.g., across cultural differences.
    \item Creating biometric privacy and security guidelines for SARs.
    \item Incorporating commercial biosensors in SAR design for accessibility and affordability.
    \item Allowing target user groups to influence SAR design, e.g. through focus groups.
\end{itemize}
\end{comment}