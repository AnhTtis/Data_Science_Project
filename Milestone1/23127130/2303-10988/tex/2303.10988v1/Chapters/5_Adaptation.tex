\section{Personalization and Adaptation}

% definition
% what was done? on which type of robots? limitations?
% foresights 

%Angle 1: 
%\alia{Biometrics for personalization}
%\alia{what are the RQs?}
%\begin{itemize}
%    \item Can SARs use biometrics? 
%    \item what is the target audience? 
%    \item \alia{what are the contributions?}
%\end{itemize}
%\begin{description}
%\item How should SARs approaches be adapted for specific populations? % multiple specific groups -> multiple social interaction needs 

%soft biometrics and SAR: \url{https://doi.org/10.1515/pjbr-2015-0004(check)  and https://doi.org/10.1007/978-3-319-16199-0_28}

% biometrics are important 
% for personalization 
% and identification, as existing solutions are often obtrusive and interrupting the interaction. 
% Use cases, people in care facilities, 
% 
\begin{comment}
% to start
Adaptation is a core research direction for SARs~\cite{mataric2016socially}. Biometrics can play a significant role in how SARs can adapt to human users. SARs are designed to provide assistance to individuals, and personal biometrics can help to personalize their responses and behavior to the individual's specific needs. Behavioral biometric data, such as heart rate, skin conductance, and facial expressions, can provide real-time feedback on the user's physical and emotional state. This continuous feedback adjustment can be non-obtrusive. \alia{reread and update}

\emph{Recognition} comes hand-in-hand with \emph{personalization}. Whether it is a distinction of a certain user (i.e., identification), or more generic (i.e. gender or age), SARs benefit from understanding the users' traits, and accordingly, building more robust social assistance.
%Physiological 
Little work focused on using biometric-based identification for Robotic applications. Carcagn{\`\i} et al. introduced a system that allows a humanoid robot to recognize the age and gender of multiple persons in the robot's camera Field of View (FOV), based on their facial features~\cite{carcagni2015soft,carcagni2015visual}. Though their approach reached satisfactory accuracy results (67.4\%), users feedback was not investigated. 

%touch interactions?

Similar works adopted face recognition approach in a long-term study for a personalized SAR for cardiac rehabilitation on a single patient~\cite{irfan2020using,irfan2022personalised}. Despite the poor performance of the user recognition (~37\%), positive reactions were reported towards the robot throughout the overall interaction. 

%behavioral biometrics
On the behavioral biometrics side, {\'A}lvarez-Aparicio used the Laser Imaging Detection and Ranging (LIDAR) sensor on non-humanoid (Orbi-One robot) to identify individuals based on their gait ~\cite{alvarez2022biometric}, reaching a precision score of 88\%. Though their approach was not specifically designed for SARs, we could see a possible use case where a robot identifies a user walking from a distance, understands their preferences, and acts accordingly. 

%gaze?
%voice?

%exisiting recognition approaches are typically using cameras, 

%example statistic and motivation for adaptation:
%the population of older adults is expected to triple by 2050

%quote from abstract
%"Participants are generally happy with the proposed platform as a mean of encouraging them to do regular exercise correctly."


%an SAR was developed for helping individuals do exercise.
%such a SAR will potentially improve the quality of life for a significant proportion of the population. 

\cite{Cano2021} did a review on use of emotionally-expressive SARs+children with ASD. \kk{keep or remove this?}


% to conclude
We foresee the importance of exploring the potential usage of biometrics in improving the communication between humans and SARs. Such exploration could expand beyond personalization applications, ensuring more secure and private interactions, in an implicit and continuous way. Additionally, we expect adopting non-traditional biometrics to gain more attention, rather than the commonly used physiological types such as face recognition and fingerprint. However, such investigations should align with understanding the users' experience. Either by focus groups, or post-study interviews, personalizations applications for socially assistive robots should not solely rely on accuracy results. 
%Further research is needed to explore the full potential of personal biometrics in improving the effectiveness of SARs in providing personalized support to individuals.
\end{comment}

\emph{Recognition} comes hand-in-hand with \emph{personalization}. SARs are designed to provide assistance to individuals, and personal biometrics can help to personalize their responses and behavior to the individual's specific needs. This has led to adaptation becoming a core research direction for socially assistive robotics~\cite{mataric2016socially}. Behavioral biometric data can provide real-time feedback on the user's physical and emotional state, enabling continuous non-obtrusive adjustments. {\'A}lvarez-Aparicio used the Laser Imaging Detection and Ranging (LIDAR) sensor on a non-humanoid robot to identify individuals based on their gait~\cite{alvarez2022biometric}, reaching a precision score of 88\%. Although their approach was not specifically designed for SARs, it illustrates a possible use case where a robot identifies a user walking from a distance, understands their preferences, and acts accordingly.

While much work has been done on biometric-based identification in non-robotic applications, only a few studies have explored their use in SARs. Carcagn\`{i} et al.~\cite{carcagni2015soft,carcagni2015visual} introduced a system that allows a humanoid robot to recognize the age and gender of multiple persons in the robot's camera field of view based on their facial features. Though their approach reached results with an accuracy of 67.4\%, user feedback was not investigated. On the other hand, similar works adopted face recognition in a long-term study for a personalized SAR for cardiac rehabilitation on a single patient~\cite{irfan2020using,irfan2022personalised}. Despite the poor performance of user recognition, positive reactions were reported throughout the overall interaction. Moreover, \citet{Chan2012} utilize biosensors to recognize the affective states of users, so that the SAR can adapt accordingly. \textbf{We therefore recognize the role of implicit user responses in designing adaptable responses for such social contexts involving SARs.}

We anticipate the importance of exploring the potential usage of biometrics in improving communication between humans and SARs. Such exploration could expand beyond personalization applications, ensuring more secure and private interactions in an implicit and continuous way. Additionally, we expect adopting non-traditional biometrics to gain more attention, rather than the commonly used physiological types such as face recognition and fingerprint. However, such investigations should align with understanding the users' experience. \textbf{We argue that personalization for socially assistive robots should not solely rely on accuracy results, but should also include target demographic inputs in the form of focus groups or post-study interviews.}