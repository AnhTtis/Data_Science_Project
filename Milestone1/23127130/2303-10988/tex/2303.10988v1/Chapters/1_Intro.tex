\section{Introduction}

Socially Assistive Robots (SARs) can be deployed in situations where they can replicate the role of a caregiver or coach. SARs can support, motivate, and provide emotional support without physical contact, to encourage learning, development, or therapy~\cite{mataric2016socially}. Previous findings showed the promising results of SARs in education and elderly care~\cite{Robinson2022}.

% explicit communication
For assistive robots to be accepted as trustworthy and effective, they must be able to engage in effective communication with humans, who may be individuals receiving support from SARs, or human "peers" in a similar position of care, e.g. a nurse or teacher. Effective communication is crucial in human-robot interaction (HRI) involving SARs, as SARs should both communicate their intent and comprehend human intent, especially in a collaborative setting~\cite{KalpagamGanesan2018}.

%implicit communication
In addition to explicit communication, implicit communication has a vital role in social interaction~\cite{Matari2017}. Biometrics such as facial expressions, body posture, and tone of voice are also used by humans for implicit communication, and can continuously provide crucial information about a person's emotional state, personality traits, and level of engagement. By using these biometric signals as implicit modalities for intent communication, SARs can gain a more complete understanding of their users and tailor their interactions accordingly. Moreover, SARs being able to emulate human non-verbal cues in social situations makes interaction smoother\cite{mataric2016socially}. As such, the development of reliable and robust biometric sensing and analysis systems is critical to the success of SARs. 

%tie a bow around it
In this paper, we reflect on the role of implicit and explicit bidirectional intent communication in establishing connections and fostering trust in SARs. Additionally, we look at how findings from different contexts (e.g. industry) can be applied to the more social settings of SARs. We focus on four aspects of HRI with SARs, namely: \textbf{transparency}, \textbf{trust}, \textbf{error-recovery}, and \textbf{adaptation}.
\begin{comment}
We summarize the research directions in form of questions as follows:

\begin{itemize}
    \item How can socially assistive robots adapt to communicate transparently and recover from errors during interaction?
    \item In what ways can socially assistive robots be adapted to better serve diverse user populations?
\end{itemize}

    \item \alia{How should SARs transparently communicate when reasoning is based on complex processes?} 
    \item \alia{How do SARs recover from unintended/incorrect actions through communication with the user?}  
    \item \alia{How should SARs approaches be adapted for specific populations?} % multiple specific groups -> multiple social interaction needs   
\end{comment}




