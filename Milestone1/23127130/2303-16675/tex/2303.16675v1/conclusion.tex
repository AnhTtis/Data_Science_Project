This paper presents a dependency delivery system based on CVMFS to provide complex software stacks on sequestered computing resources such as worker nodes of supercomputers not having external connectivity.

After introducing CVMFS (section \ref{section:11}), a critical tool - especially for LHC communities - to supply workloads with complex dependencies on Grid Sites, we have described the context of this study (section \ref{section:12}): several virtual organizations are exporting their workflow from WLCG to supercomputers, which have more restrictive policies than grid sites and generally do not allow to mount CVMFS on the worker nodes.

We have highlighted several solutions aiming to overcome the issue such as collaborating with the system administrators and using tools such as \emph{Parrot} and \emph{cvmfsexec}.
Nevertheless, these approaches do not work when worker nodes have no external connectivity.
Then, we have emphasized different ways to export parts of CVMFS to supercomputers with no external connectivity: \emph{uncvmfs} and \emph{cvmfs\_shrinkwrap}.
These solutions require several manual steps and therefore we have proposed a utility to assist communities in this process.

We have explained the different steps of the utility in detail (section \ref{section:22}). It traces - captures the system calls of - applications of interest, builds a subset with the required files, tests the subset and deploys it to a remote computing resource.
We also described the structure of the solution (section \ref{section:23}), which is composed of two layers: a first one, generic with simple components, and a second one more complex, adapted to communities needs that can be fine-tuned.

Finally, we have provided a use case based on Gauss, a Monte-Carlo simulation application reproducing events occurring in the LHCb detector (section \ref{section:31}).
Gauss is highly configurable and can be coupled with different packages, extra packages, options, data and versions.
It represents a complex bundle of dependencies, which makes it ideal to test our utility.
We have proposed a method to encapsulate Gauss and its dependencies in a subset, which represents 12.4 Gb of space on the GPFS of the Mare Nostrum supercomputer (section \ref{section:33}).
The solution produced represents 0.24\% of the full LHCb repository and, thus, is easier to update.
We have successfully tested the solution with different Gauss workloads.
Future work could focus on encapsulating further applications from different domains using this utility, and analyzing its performances to deploy subsets on various supercomputers.

