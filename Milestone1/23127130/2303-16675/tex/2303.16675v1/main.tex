\documentclass[a4paper,12pt]{article}
%
\usepackage[margin=0.9in]{geometry}
\usepackage[dvipsnames,svgnames]{xcolor}
\usepackage{graphicx}
\usepackage[edges]{forest}
\usepackage[frozencache,cachedir=.]{minted}
\usepackage[T1]{fontenc}
%\usepackage[libertine]{newtxmath}
\usepackage[bitstream-charter]{mathdesign}
%
\title{A Subset of the CERN Virtual Machine File System: Fast Delivering of Complex Software Stacks for Supercomputing Resources}
\author{Alexandre F. Boyer$^{1,2}$, Christophe Haen$^2$, Federico Stagni$^2$, David R.C. Hill$^1$\\
\normalsize{\{alexandre.franck.boyer, christophe.haen, federico.stagni\}@cern.ch, david.hill@uca.fr}\\
\normalsize{1 Université Clermont Auvergne, Clermont Auvergne INP,}\\\normalsize{CNRS, Mines Saint-Etienne, LIMOS, 63000 Clermont-Ferrand, France}\\
\normalsize{2 European Organization for Nuclear Research, Meyrin, Switzerland}}
\date{}

%
% First names are abbreviated in the running head.
% If there are more than two authors, 'et al.' is used.
%
\begin{document}

\maketitle              % typeset the header of the contribution
%
\begin{abstract}
Delivering a reproducible environment along with complex and up-to-date software stacks on thousands of distributed and heterogeneous worker nodes is a critical task. The CernVM-File System (CVMFS) has been designed to help various communities to deploy software on worldwide distributed computing infrastructures by decoupling the software from the Operating System. However, the installation of this file system depends on a collaboration with system administrators of the remote resources and an HTTP connectivity to fetch dependencies from external sources. Supercomputers, which offer tremendous computing power, generally have more restrictive policies than grid sites and do not easily provide the mandatory conditions to exploit CVMFS. Different solutions have been developed to tackle the issue, but they are often specific to a scientific community and do not deal with the problem in its globality. In this paper, we provide a generic utility to assist any community in the installation of complex software dependencies on supercomputers with no external connectivity. The approach consists in capturing dependencies of applications of interests, building a subset of dependencies, testing it in a given environment, and deploying it to a remote computing resource. We experiment this proposal with a real use case by exporting Gauss - a Monte-Carlo simulation program from the LHCb experiment - on Mare Nostrum, one of the top supercomputers of the world. We provide steps to encapsulate the minimum required files and deliver a light and easy-to-update subset of CVMFS: 12.4 Gigabytes instead of 5.2 Terabytes for the whole LHCb repository.
\end{abstract}
%
%
%
\section{Introduction}
\section{Introduction}
\label{sec:introduction}
% \begin{itemize}
%     % Diffusion of FL
%     \item {\st{Diffusion of FL}}
%     % Security threats to FL
%     \item {\st{Security threats to FL with particular focus on model poisoning}}
%     % Limitations of existing countermeasures
%     \item {\st{Current countermeasures (e.g., KRUM) and their limitations}}
%     % Proposed method and its advantages
%     \item {\st{Intuitive description of the proposed method and its difference (i.e., advantages) w.r.t. state of the art}}
%     % Main contributions
%     \item {\st{Summary of the main contributions of this work}}
%     % Paper's structure and organization
%     \item {\st{Paper's structure and organization}}
% \end{itemize}

% Diffusion of FL
Recently, {\em federated learning} (FL) has emerged as the leading paradigm for training distributed, large-scale, and privacy-preserving machine learning (ML) systems~\cite{mcmahan2017googleai,mcmahan2017aistats}. 
The core idea of FL is to allow multiple edge clients to collaboratively train a shared, global model without disclosing their local private training data.
%Specifically, an FL system consists of a central server and many edge clients; 
A typical FL round involves the following steps: {\em(i)} the server randomly picks some clients and sends them the current, global model; {\em(ii)} each selected client locally trains its model with its own private data; then, it sends the resulting local model to the server;\footnote{Whenever we refer to global/local model, we mean global/local model {\em parameters}.} {\em(iii)} the server updates the global model by computing an \emph{aggregation function}, usually the average (FedAvg), on the local models received from clients.
% \begin{enumerate}
%     \item[{\em(i)}] the server sends the current, global model to the clients and appoints some of them for training;
%     \item[{\em(ii)}] each selected client locally trains its copy of the global model with its own private data; then, it sends the resulting local model back to the server;\footnote{Whenever we refer to global/local model, we mean global/local model {\em parameters}.}
%     \item[{\em(iii)}] the server updates the global model by computing an \emph{aggregation function} on the local models received from clients (by default, the average, also referred to as FedAvg~\cite{mcmahan2017aistats}).
% \end{enumerate}
This process goes on until the global model converges. %(e.g., after a certain number of rounds or other similar stopping criteria).
%\\
% The advantages of FL over the traditional, centralized learning paradigm are undoubtedly clear in terms of flexibility/scalability (clients can join/disconnect from the FL network dynamically), network communications (only model weights\footnote{We will use \textit{parameters} and \textit{weights} interchangeably.} are exchanged between clients and server), and privacy (each client's private training data is kept local at the client's end and not uploaded to the server).
\\
% Security threats to FL
%However, the growing adoption of FL also raises security concerns~\cite{costa2022covert}, particularly about its confidentiality, integrity, and availability.
Although its advantages over standard ML, FL also raises security concerns~\cite{costa2022covert}. %, particularly about its confidentiality, integrity, and availability~\cite{costa2022covert}.
% OLD, LONG VERSION
% Indeed, some work deals with privacy leakage that may expose the local data of some clients~\cite{melis2019sp}. 
% A large body of work, instead, investigates attacks that usually aim to detriment the predictive accuracy of the learned global model. For instance, \emph{data poisoning} attacks achieve this goal by letting an adversary pollute the training set of some corrupt FL clients with maliciously crafted examples~\cite{jagielski2018sp}.
% Similarly, in \emph{model poisoning} the attacker attempts to tweak the global model weights~\cite{bhagoji2019pmlr} by directly perturbing the local model's weights of some infected FL clients before these are sent to the central server for aggregation, usually via so-called Byzantine attacks. 
% It turns out that Byzantine model poisoning attacks severely impact standard FedAvg; therefore, more robust aggregation functions must be designed to make FL systems secure.
Here, we focus on \emph{untargeted model poisoning} attacks~\cite{bhagoji2019pmlr}, where an adversary attempts to tweak the global model weights %\footnote{We will use the terms \textit{parameters} and \textit{weights} interchangeably.} 
by directly perturbing the local model's parameters of some infected clients before these are sent to the central server for aggregation.
In doing so, the adversary aims to jeopardize the global model \textit{indiscriminately} at inference time.
Such model poisoning attacks severely impact standard FedAvg; therefore, more robust aggregation functions must be designed to secure FL systems.
\\
% In this paper, we focus on designing a novel robust aggregation scheme at the server's end to contrast the effect of Byzantine model poisoning attacks.
%
% Current countermeasures and their limitations
%Several countermeasures have been proposed in the literature to combat model poisoning attacks on FL systems.
% Some methods use simple statistics more robust than plain average to smooth the impact of malicious updates (e.g., Trimmed Mean and FedMedian~\cite{yin2018icml}). 
% Other defenses implement outlier detection techniques to discard malicious updates from the aggregation performed at the server's end. Those are either based on heuristics (e.g., Krum/Multi-Krum~\cite{blanchard2017nips} and Bulyan~\cite{mhamdi2018pmlr}) or data-driven approaches (e.g., K-means clustering~\cite{shen2016acm} or DnC via spectral analysis~\cite{shejwalkar2021ndss}). 
% Finally, some strategies rely on a centralized ``source of trust'' to spot potential malicious updates (e.g., FLTrust~\cite{cao2020fltrust}).
% Several countermeasures have been proposed in the literature to combat model poisoning attacks on FL systems, i.e., to discard possible malicious local updates from the aggregation performed at the server's end. 
% These techniques range from simple statistics more robust than plain average (e.g., Trimmed Mean and FedMedian~\cite{yin2018icml}) to outlier detection heuristics (e.g., Krum/Multi-Krum~\cite{blanchard2017nips} and Bulyan~\cite{mhamdi2018pmlr}) or data-driven approaches (e.g., spectral analysis via K-means clustering~\cite{shen2016acm} or spectral analysis), or methods based on ``source of trust'' (e.g., FLTrust~\cite{cao2020fltrust}).
% OLD, LONG VERSION
%Several countermeasures have been proposed in the literature to combat Byzantine model poisoning attacks on FL systems.
% Descriptive statistics
% For example, Trimmed Mean and FedMedian aggregate local model updates using more robust statistics than standard average~\cite{yin2018icml}.
%
% % Heuristics for outlier detection
% Many existing Byzantine-resilient strategies implement some outlier detection heuristics to discard the model updates sent by potentially malicious clients from the input of the aggregation function.
% One of the most popular heuristics is Krum~\cite{blanchard2017nips}.
% This strategy tries to mitigate the impact of Byzantine attacks by selecting as a global model the local model with the smallest sum of Euclidean distances to {\em all} the other local models.
% Although powerful, Krum requires the server to know (or, at least, estimate) the number of malicious FL clients upfront, which is generally impossible in a realistic attack scenario. %
% Moreover, Krum may become ineffective for complex, high-dimensional model parameter spaces due to the curse of dimensionality.
% Bulyan~\cite{mhamdi2018pmlr} tries to overcome this issue by combining Krum with a variant of Trimmed Mean.
% % Data-driven outlier detection
% Other strategies use data-driven outlier detection techniques -- e.g., via K-means clustering~\cite{shen2016acm} -- to spot potential malicious local model updates. 
% %For instance, Shen et al. propose to cluster local model updates with K-means and thus identify outliers.
%
% % Other techniques
% As far as the server is concerned, any local model received can be from a potential malicious client. 
% FLTrust~\cite{cao2020fltrust} assumes the server acts as a client, i.e., trains a local model on an additional {\em trustworthy} dataset at the server's end and compares it against all the local models from other clients. 
% This way, the server can rely on some ``source of trust'' when discarding potentially malicious clients.
%\\
% Limitations of existing Byzantine-resilient strategies
Unfortunately, existing defense mechanisms either rely on simple heuristics (e.g., Trimmed Mean and FedMedian by~\cite{yin2018icml}) or need strong and unrealistic assumptions to work effectively (e.g., foreknowledge or estimation of the number of malicious clients in the FL system, as for Krum/Multi-Krum~\cite{blanchard2017nips} and Bulyan~\cite{mhamdi2018pmlr}, which, however, cannot exceed a fixed threshold).
Furthermore, outlier detection methods using K-means clustering~\cite{shen2016acm} or spectral analysis like DnC~\cite{shejwalkar2021ndss} do not directly consider the temporal evolution of local model updates received.
Finally, strategies like FLTrust~\cite{cao2020fltrust} require the server to collect its own dataset and act as a proper client, thereby altering the standard FL protocol.
\\
% OLD, LONG VERSION
% Overall, existing Byzantine-resilient strategies are either simple heuristics (e.g., FedMedian) or, if they are more complex, they rely on strong and unrealistic assumptions to work effectively (e.g., knowing the number of malicious clients in the FL system in advance, as for Krum and alike).
% Furthermore, data-driven outlier detection methods do not consider the temporary evolution of local model updates received (e.g., K-means clustering). 
% Finally, strategies like FLTrust requires the server to collect its own dataset and act as a proper client, thereby altering the standard FL protocol.
%
% Description of the proposed method
This work introduces a novel pre-aggregation \textit{filter} robust to untargeted model poisoning attacks. Notably, this filter $(i)$ operates without requiring prior knowledge or constraints on the number of malicious clients and $(ii)$ inherently integrates temporal dependencies. 
The FL server can employ this filter as a preprocessing step before applying \textit{any} aggregation function, be it standard like FedAvg or robust like Krum or Bulyan.
Specifically, we formulate the problem of identifying corrupted updates as a multidimensional (i.e., matrix-valued) time series anomaly detection task. 
The key idea is that legitimate local updates, resulting from well-calibrated iterative procedures like stochastic gradient descent (SGD) with an appropriate learning rate, show \textit{higher predictability} compared to malicious updates. This hypothesis stems from the fact that the sequence of gradients (thus, model parameters) observed during legitimate training exhibit regular patterns, as validated in Section~\ref{subsec:intuition}. %until convergence. 
%This regularity may be more pronounced for smooth convex loss functions, but it can still be captured within an appropriate time window, even for more complex and convoluted loss surfaces. 
%We provide evidence of this claim in Appendix~B, where we show that the average mutual information (i.e., ``predictability''), calculated over pairs of legitimate model updates sent at different FL rounds, is significantly higher than the corresponding computation for a malicious client.
\\
Inspired by the matrix autoregressive (MAR) framework for multidimensional time series forecasting~\cite{chen2021je}, we propose the FLANDERS ({\em \textbf{F}ederated \textbf{L}earning meets \textbf{AN}omaly \textbf{DE}tection for a \textbf{R}obust and \textbf{S}ecure}) filter.
The main advantages of FLANDERS over existing strategies like FLDetector~\cite{zhao2020multivariate} are its resilience to large-scale attacks, where $50\%$ or more FL participants are hostile, and the capability of working under realistic non-iid scenarios.
We attribute such a capability to two key factors: $(i)$ FLANDERS works without knowing a priori the ratio of corrupted clients, and $(ii)$ it embodies temporal dependencies between intra- and inter-client updates, quickly recognizing local model drifts caused by evil players. Below, we summarize our main contributions:

\begin{itemize}
\item[{\em(i)}]
We provide empirical evidence that the sequence of models sent by legitimate clients is more predictable than those of malicious participants performing untargeted model poisoning attacks.
\\
\item[{\em(ii)}] 
We introduce FLANDERS, the first pre-aggregation filter for FL robust to untargeted model poisoning based on multidimensional time series anomaly detection.
\\
\item[{\em(iii)}] 
We integrate FLANDERS into Flower,\footnote{\scriptsize{\url{https://flower.dev/}}} a popular FL simulation framework for reproducibility.
\\
\item[{\em(iv)}] 
We show that FLANDERS improves the robustness of the existing aggregation methods under multiple settings: different datasets, client's data distribution (non-iid), models, and attack scenarios.
\\
\item[{\em(v)}] 
We publicly release all the implementation code of FLANDERS along with our experiments.\footnote{\scriptsize{\url{https://anonymous.4open.science/r/flanders_exp-7EEB}}}
\end{itemize}

% Paper's structure and organization
The remainder of the paper is structured as follows. %some related work and the current state-of-the-art solutions to security issues that FL entails. 
Section~\ref{sec:background} covers background and preliminaries. 
In Section~\ref{sec:related}, we discuss related work.
Section~\ref{sec:problem} and Section~\ref{sec:method} describe the problem formulation and the method proposed. % to tackle it. 
Section~\ref{sec:experiments} gathers experimental results. %, and Section~\ref{sec:limitations} discusses some limitations of this work.
Finally, we conclude in Section~\ref{sec:conclusion}.
 %discusses the limitations of this work and draws future research directions.
%reports conclusions and draws perspectives for future research directions.

%%%%%%% OLD %%%%%%%
%to overcome the resilience of Byzantine failures in distributed Stochastic Gradient Descent computations. 
% The strength of Krum is its time complexity, which is linear in the gradient dimension. 
% However, the robustness of the approach is guaranteed for gradient-based learning applications only when the majority of the clients are not compromised. 
% Besides, the aggregation mechanism of Krum, as well as that of similar methods, is robust from a coarse-grained perspective and does not provide solutions to errors and perturbations that may occur at inference time.
%A related approach to~\cite{blanchard2017nips} is the work of Su et al.~\cite{su2016dc}. Here, the authors propose an iterated approximate agreement to tackle a multi-layer scenario attacked by Byzantine agents. 
%However, the method works efficiently on the sole discrete context and it is inapplicable to continuous state environments.
%\gabri{Maybe, we should just talk about the main limitations of existing countermeasures without digging into their details (or, we can just mention Krum as this is the most popular one). I will move the description of all these methods to the Related Work section.}

\section{Context}\label{section:1}
\subsection{CVMFS to distribute software on grid resources}\label{section:11}

At the beginning of 2021, CVMFS was managing about 1 billion files delivered to more than 100,000 computing nodes by (i) 10 public data mirror servers - called \emph{Stratum1}s - located in Europe, Asia and the United States and (ii) 400 site-local cache servers \cite{CVMFS_2021}.

To keep the file system consistent and scalable, developers conceived CVMFS as a read-only file system.
Release managers - or continuous integration workers - aiming to publish a software release has to log in to a dedicated machine - named \emph{Stratum0} - with an attached storage volume providing an authoritative and editable copy of a given repository \cite{Blomer_2019}.
Changes are written into a staging area until they are committed as a consistent changeset: new and modified files are transformed into a content-addressed object providing file-based deduplication and versioning.
In 2019, Popescu et al. \cite{Popescu_2019} introduced a gateway component, a web service in front of the authoritative storage, allowing release managers to perform concurrent operations on the same repository and make CVMFS more responsive (Figure \ref{fig:cvmfsGrid}.1.b and \ref{fig:cvmfsGrid}.2.b).

The transfer of files is then done lazily via HTTP connections initiated by the CVMFS clients \cite{Popescu_2019} (Figure \ref{fig:cvmfsGrid}.3.b).
Clients request updates based on their Time-to-Live (TTL) value, which is generally about a few minutes. Once the TTL value expires, clients download the latest version of a manifest - a text file located in the top-level directory of a given repository composed of the current root hash, metadata and the revision number of this repository - and make the updated content available.
Dykstra et al. \cite{Dykstra_2014} provide additional details about data integrity and authenticity mechanisms of CVMFS to ensure that data received matches data initially sent by a trusted server.
This pull-based approach has been proven to be robust and efficient, according to Popescu et al. \cite{Popescu_2019}, and has been widely used to distribute up-to-date software on grid sites for many years (Figure \ref{fig:cvmfsGrid}.2.a). Figure \ref{fig:cvmfsGrid} presents a simplified schema summarizing the software distribution process on grid sites via CVMFS.

\begin{figure}[ht]
    \centering
    \includegraphics[width=0.7\textwidth]{CVMFSOnGrid.pdf}
    \caption{Schema of the CVMFS workflow on Grid Sites: (a) the steps to get software dependencies from the job; (b) the steps to publish a release of a software in CVMFS.}
    \label{fig:cvmfsGrid}
\end{figure}

Users may need to use various versions of software on heterogeneous computing resources implying different OS and architectures.
To provide a convenient environment for the users, release managers generally provide software along with build files related to many architectures, OS and compilers.
Framework for building and installing scientific software on heterogeneous systems can be used to supply CVMFS with build files.
Easybuild \cite{easybuild}, Spack \cite{spack}, Nix \cite{nix} or Gentoo \cite{gentoo} are popular choices in this area \cite{Vokl_2021,Benda_2020,Burr_2019}. 

\subsection{Software delivery on supercomputers}\label{section:12}

Communities working around the Large Hadron Collider (LHC) \cite{LHC} have extensively used WLCG and CVMFS to process a growing amount of data.
This approach was reliable during LHC Run1 but has demonstrated its limit.
According to the analysis of Stagni et al. \cite{Stagni_McNab_Luzzi_Krzemien_Consortium_2017} on the use of CPU cycles in 2016, all the LHC experiments have consumed more CPU-hours than those officially pledged to them by the WLCG: they found ways to exploit opportunistic and not officially supported resources.
Moreover, in the High-Luminosity Large Hadron Collider (HL-LHC) \cite{osti_1365580} era, experiments are expected to produce up to an order of magnitude more data compared to the current phase (LHC Run2).
To keep up with the computing needs, experiments have started to use supercomputers. 
They offer a significant amount of computing power and would potentially offer a more cost-effective data processing infrastructure compared to dedicated resources in the form of commodity clusters, as Sciacca emphasizes \cite{Sciacca_Weber_2020}.
Nevertheless, supercomputers have more restrictive security policies than Grid Sites: they do not allow CVMFS to be mounted on the nodes by default and many of them have limited or even no external connectivity.
The LHC communities have developed different solutions and strategies to cope with the lack of CVMFS, which is a critical component to run their workflows.

Stagni et al. \cite{Stagni_Valassi_Romanovskiy_2020} rely on a close collaboration with some supercomputer centers - Cineca in Italy and CSCS in Switzerland -  to get CVMFS mounted on the worker nodes.
Nevertheless, their strategy is limited to a few supercomputers and their approach would be difficult to reproduce on a large number of supercomputers: most of them do not allow such collaboration.

To deal with the lack of CVMFS on supercomputers with outbound connectivity, Filipčič et al. studied two solutions: \emph{rsync} and \emph{Parrot}  \cite{Filipcic_Haug_Hostettler_Walker_Weber_2015}.
The first solution consisted in copying the CVMFS software repository in the shared file system using \emph{rsync}: a utility aiming to transfer and synchronize files and directories between two different systems.
\emph{rsync} added a significant load on the shared file system of the supercomputers and required changes in the repository absolute paths.
The second solution was based on Parrot: a utility copied on the shared file system of the supercomputer, usable without any user privileges. 
Parrot is a wrapper using \emph{ptrace} attached to a process that intercepts system calls that access the file system and can simulate the presence of arbitrary file system mounts, CVMFS in this case.
Nevertheless, the solution was "unreliable in a multi-threaded environment" \cite{Filipcic_Haug_Hostettler_Walker_Weber_2015} because it was unable to handle race conditions. 
These methods did not constitute a production-level solution but contributed to further and future advanced solutions.

In recent years, developments in the Fuse user space libraries and the Linux kernel have lifted restrictions for mounting Fuse file systems such as CVMFS.
Developers of CVMFS have integrated these changes and designed a package called \emph{cvmfsexec} \cite{cvmfsexec}, which allows mounting the file system as an unprivileged user. 
The program needs a specific environment to work correctly: (i) external connectivity; (ii) the \emph{fusermount} library or unprivileged namespace mount points or a setuid installation of \emph{Singularity} (efficient High-Performance Computing container technology).
Blomer et al. provide additional details about the package \cite{Blomer_2020}.

Communities exploiting supercomputers that do not provide outbound connectivity cannot directly benefit from \emph{cvmfsexec}: the package still needs to pull updated data via HTTP, which is not available in such context.
We can distinguish two cases: (i) supercomputers that grant outside network or specific service access to a limited number of nodes and (ii) supercomputers that do not provide nodes with any external connectivity at all.

Tovar et al. recently worked on the first case \cite{Tovar_2021}.
They managed to build a virtual private network (VPN) client and server to redirect network traffic from the workloads running on the worker nodes to external services such as CVMFS.
In this configuration, the VPN client runs on a worker node along with the job, while the VPN server is hosted on one of the specific nodes of the supercomputer and can interact with external services.
Communities working on supercomputers from the second case cannot leverage the solution developed by Tovar et al.

O'Brien et al., one of the first teams to work with supercomputers in the LHC context, address the lack of external network access by copying part of it to the shared Lustre file system accessible by the WNs \cite{OBrien_Walker_Washbrook_2014}.
The approach (i) worked because the environment of the supercomputer was similar to a grid site one, (ii) required changes in the CVMFS files and (iii) degraded the performance of the software as Angius et al. described \cite{Angius_Oleynik_Panitkin_Turilli_De_Klimentov_Oral_Wells_Jha_2017}. 
To tackle the latter issue on the Titan supercomputer, Angius et al. moved the software to a read-only NFS server \cite{Angius_Oleynik_Panitkin_Turilli_De_Klimentov_Oral_Wells_Jha_2017}: this eliminated the problem of metadata contention and improved metadata read performance.

Similarly, on the Chinese HPC CNGrid, Filipčič regularly packed a part of CVMFS in a tarball.
Filipčič provided a deployment script to install the software and fix the path relocation on the shared file system to the local system administrators: they were then responsible for getting and updating the CVMFS tarball on the network when requested \cite{Filipcic_2017}.

To help communities to unpack a CVMFS repository in a file system, a team of developers designed \emph{uncvmfs} \cite{uncvmfs}. The utility deduplicates files of a software stack: it populates a given directory with the CVMFS files that are then hard-linked into it, if possible.
The program was used, in combination with Shifter \cite{Gerhardt_Bhimji_Canon_Fasel_Jacobsen_Mustafa_Porter_Tsulaia_2017}, a container technology providing a reproducible environment, in the context of the integration of the ALICE and CMS experiments workflows on the NERSC High-Performance Computing resources \cite{Fasel_2016,Hufnagel_2017}.
As a proof of concept, Gerhardt et al. used \emph{uncvmfs} to deduplicate the ATLAS repository and copy it into an ext4 image  - about 3.5Tb of data containing 50 million files and directories -, compressed into a 300Gb squashfs image; and Shifter to provide a software-compatible environment to run the jobs \cite{Gerhardt_Bhimji_Canon_Fasel_Jacobsen_Mustafa_Porter_Tsulaia_2017}.
Despite encapsulating the files in a container reduced the startup time of the applications, the solution generated large images, long to update and deliver on time.

To cope with large images, Teuber and the CVMFS developers conceived \emph{cvmfs\_shrinkwrap} \cite{Teuber_2019}.
The tool supports \emph{uncvmfs} features with certain optimizations and delivers additional features: \emph{cvmfs\_shrinkwrap} can extract specific files and directories based on specification files, deduplicate them, making them easy to export in various formats such as squashfs or tarball.
In this way, the following operations remain on behalf of the user communities: (i) trace their applications - meaning, in this context,  "capturing all their dependencies and their locations in the file system" -, (ii) call \emph{cvmfs\_shrinkwrap} to get a subset of CVMFS composed of the minimum required files, and (iii) export this subset in a certain format and deploy it on sequestered computing resources to run their jobs.

Douglas et al. already described such a project in an article \cite{Douglas_2019}, but the work remains specific to the ATLAS experiment.
They use \emph{uncvmfs} to produce a large image that has to be filtered afterward.
In this paper, we aim at assisting various user communities in this process by providing an open-source utility that would take applications of interest in input and would output - with the help of \emph{cvmfs\_shrinkwrap} - a subset of CVMFS with the minimum required files to run the given applications, in combination with a container image if needed.
To our knowledge, no paper has already covered the subject.

\section{Design of the CVMFS Subset Builder}\label{section:2}
\subsection{Input and output data}\label{section:21}

The utility takes a directory as input that should contain: (i) a list of applications of interest (\texttt{apps}): a command along with its input data in a separate sub-directory for each application to trace; and/or (ii) a list of files composed of paths to include in the subset of CVMFS (\texttt{namelists}).
Additionally, user communities can embed a (iii) container image compatible with Singularity to get a specific environment to trace and test the applications; (iv) and a configuration file to fine-tune the utility with variables related to the deployment process, or information about repositories.
A schema of the inputs is available in Figure \ref{fig:inputTree}.

\begin{figure}[ht]
    \centering
    \input{inputTree}
    \caption{Schema of the input structure given to the utility.}
    \label{fig:inputTree}
\end{figure}

The expected output can take different forms depending on the utility configuration:
\begin{itemize}
    \item The subset of CVMFS, generated as a standalone.
    In this case, administrators representing their user communities need to provide the right environment by themselves, which might also involve discussions with the system administrators.
    \item The subset of CVMFS embedded within the given Singularity container image.
    The utility merges both elements and submits the resulting image, which can be long to generate and deploy but may limit manual operations on the remote location. 
\end{itemize}

\subsection{Features}\label{section:22}

We break down the process into four main steps, namely:
\begin{itemize}
    \item \emph{Trace}: consists in running applications contained in \texttt{apps} and trapping their system calls at runtime, using \emph{Parrot}, to identify and extract the paths of their dependencies.
    Applications can run in a Singularity container when provided, which delivers further software dependencies and a reproducible environment. 
    Dependencies are then saved in a specific file \texttt{namelist.txt}. 
    In this context, \emph{Parrot} is only used to capture system calls and, thus, is not impacted by the issues mentioned in section \ref{section:12}.
    If the step detects an error during the execution of an application, then the program is stopped.
    The step is particularly helpful for users of the utility having no technical knowledge of the applications of interest.
    \item \emph{Build}: builds a subset of CVMFS based on the paths coming from \emph{Trace} and the \texttt{namelists} directory. First, the step merges the namelist files to remove duplicated or non-existent path references, and then separates the paths in different specification files related to repositories. Finally, the step calls \emph{cvmfs\_shrinkwrap} to generate the subset of CVMFS.
    Figures \ref{fig:traceProcess} and \ref{fig:pipeline}.3 illustrate an example.
    The utility deduplicates the files, and hard-link data to populate a directory, ready to be exported in various formats as explained in Section \ref{section:12} and shown in Figure \ref{fig:pipeline}.3.

    \begin{figure}[ht]
    \centering
    \begin{minted}{yaml}
    in namelist1.txt: 
    /cvmfs/repoA/path/to/file
    /cvmfs/repoB/path/to/another/file
    in namelist2.txt:
    /cvmfs/repoA/path/to/file
    /cvmfs/repoB/path/to/yet/another/file
    
    in repoA.spec:
    /path/to/file
    in repoB.spec:
    /path/to/another/file
    /path/to/yet/another/file
    \end{minted}
    \caption{Transformation process occurring during the \emph{Trace} step: CVMFS dependencies are extracted  from \texttt{namelist.txt} and moved to specification files.}
    \label{fig:traceProcess}
    \end{figure}

    \item \emph{Test}: consists in testing certain applications - in the given Singularity container environment when provided - using the subset of CVMFS obtained during the \emph{Build} step (see Figure \ref{fig:pipeline}.4). 
    By default, applications from \texttt{apps} are used but further tests can also be provided by modifying the utility configuration.
    All the applications have to complete their execution to go to the next step.
    \item \emph{Deploy}: deploys the subset of CVMFS (Figure \ref{fig:pipeline}.5) embedded or not within the container image depending on the configuration options. If such is the case, then the utility (i) generates a new container definition file that includes the files with the container image, (ii) executes it to produce a new read-only container image. The utility supports ssh deployment via \emph{rsync}, provided the right credentials in the configuration. 
\end{itemize}

\begin{figure}[ht]
    \centering
    \includegraphics[width=0.74\textwidth]{PipelineSteps.pdf}
    \caption{Schema of the utility workflow: from getting an application to trace to a subset of CVMFS on the Data Transfer Node of a High-Performance Computing cluster.}
    \label{fig:pipeline}
\end{figure}

\subsection{Implementation}\label{section:23}

The utility is built as a 2-layer system.
The first layer, \emph{subcvmfs-builder}\cite{subcvmfs-builder}, is the core of the system and is self-contained.
It takes the form of a Python package, which embeds the steps described in section \ref{section:22}, and provides a command-line interface to call and execute steps independently from each other.
The first layer is, and should remain, simple and generic to be easily managed by developers and used by various communities. 

The second layer is the glue code: it consists of a workflow executing - all, or some of - the steps of the first layer.
It contains the complexity required to generate and deliver a subset of dependencies according to the needs of its users.
Unlike the first layer, the second one can take several forms and each community can tailor it for its software stack.

We propose a first, simple and generic layer-2 implementation calling each step one after the other: \emph{subcvmfs-builder-pipeline}\cite{subcvmfs-builder-pipeline}.
This layer-2 implementation is executed from a GitLab CI/CD \cite{gitlabci}, which provides a runner and a docker executor bound to a CVMFS client to execute the code (see Figure \ref{fig:subcvmfsbuilderpipeline}) 
GitLab includes features such as log preservation to help debug the implementation and integrates a pipeline scheduling mechanism to regularly update a subset of dependencies. 
Even though this layer-2 solution is adapted for basic examples - implying a few commands to trace and test, having a small number of dependencies -, it might require further fine-tuning for more advanced use cases.

Indeed, this generic layer-2 implementation is not scalable as it (i) is a single-threaded and single-process program, and (ii) requires manual operations to insert additional inputs in the process.
This is not adapted to communities having to trace and test hundreds of various applications to generate large subsets of CVMFS.
Two possibilities for such communities: building a new layer-2 implementation - able to automatically fetch applications and trace/test them in parallel - based on \emph{subcvmfs-builder-pipeline} or creating one from scratch.

In the next section, we are going to study how the LHCb experiment \cite{LHCb_2008} leverages \emph{subcvmfs-builder} and \emph{subcvmfs-builder-pipeline} to deliver Gauss \cite{Clemencic_2011}, a Monte-Carlo simulation program, on the worker nodes of Mare Nostrum \cite{David_2010}, a supercomputer with no external connectivity based in Barcelona, Spain.

\begin{figure}[ht]
    \centering
    \includegraphics[width=0.8\textwidth]{SubCVMFSBuilder.pdf}
    \caption{Schema of a layer-2 implementation within GitLab CI.}
    \label{fig:subcvmfsbuilderpipeline}
\end{figure}

% talk about cvmfs_shrinkwrap that can be kept and easily updated afterwards? Is this really true? 




\section{A Practical Use Case}\label{section:3}
\subsection{Gauss}\label{section:31}

To better understand experimental conditions and performances, the LHCb collaboration has developed Gauss, a Monte-Carlo simulation application - based on the Gaudi framework \cite{Barrand_2001} - that reproduces events occurring in the LHCb detector.
The application consists of two independent phases executed sequentially, namely the generation of the events \cite{Belyaev_2011} relying on Pythia \cite{Sj_strand_2001} by default; the tracking of the particles through the simulated detector depending on Geant4 \cite{geant_2003}.

In 2021, Gauss represents about 70\% of the distributed computing activities of the LHCb collaboration and 150 million events are simulated per day.
The application has originally been tailored for WLCG grid sites: Gauss is a compute-intensive single-process (SP), single-threaded (ST) application, only supporting x86 architectures and CERN-CentOS-compatible environments \cite{LinuxWebCERN}.
Gauss and most of its dependencies are delivered via CVMFS.

Gauss takes a certain number of events to process as inputs, as well as a "run number" and an "event number".
The combination of both numbers forms a seed, which ensures repeatability during the generation and simulation phases.
It mainly relies on packages such as Python, Boost and gcc to produce histograms and \emph{ntuples} under the form of a ROOT \cite{root} file.

Gauss is modular and highly configurable and constitutes a complex use-case: it can integrate extra packages such as various event generators and decay tools. 
Depending on LHCb production needs and the computing environments available, different versions of Gauss and its attached packages can be used.
A plethora of option files can also be passed as inputs to the extra packages.
Figure \ref{fig:gauss} describes the inputs, outputs and dependencies of Gauss as well as its interactions with some extra packages and their options.

\begin{figure}[ht]
    \centering
    \includegraphics[width=0.8\textwidth]{GaussDependencies.pdf}
    \caption{Example of a Gauss instance, its dependencies and some interactions with extra packages and their options.}
    \label{fig:gauss}
\end{figure}

\subsection{Mare Nostrum}\label{section:32}

To start integrating their workflows on High-Performance computing resources, LHC experiments can benefit from a collaboration with PRACE \cite{prace} and GÉANT \cite{geant,collabhpc}.
This collaboration gives them access to several European supercomputers such as Marconi in Italy and Mare Nostrum in Spain.

Managed by the Barcelona Supercomputing Center (BSC), MareNostrum is the most powerful and emblematic supercomputer in Spain \cite{marenostrum}.
MareNostrum was built in 2004 (MareNostrum 1), has been updated 3 times since then (Mare Nostrum 2,3 and 4) and was ranked 63rd in the June 2021 Top500 list \cite{top500}.
Each node composing the general-purpose block is equipped with two Intel Xeon Platinum 8160 24 cores at 2.1 GHz chips, and at least 2GB of RAM: this configuration matches with Gauss requirements.
Nevertheless, Mare Nostrum is more restrictive than a traditional Grid Site on WLCG: (i) no external connectivity at all; (ii) no service can be installed on the edge node; (iii) no CVMFS, and thus, no Gauss and its dependencies available.

% Lustre share file system?

\subsection{Running Gauss on Mare Nostrum}\label{section:33}

Running embarrassingly parallel applications such as Gauss on a supercomputer can be seen as counterproductive.
While it is true that the interconnect of the supercomputer partitions has not been designed for millions of small Monte-Carlo runs, it is better to use available, otherwise unused, cycles in agreement with the management of the supercomputer sites.
In the meantime, developers are adapting software \cite{Siddi_Muller_2019,Mazurek_2021}, but it remains a long process, requiring deep and technical software inputs.

To deliver Gauss on Mare Nostrum, LHCb can rely on (i) \emph{subcvmfs-builder} to produce a subset of CVMFS containing the required files; (ii) a CernVM Singularity container to provide a Gauss-compatible environment and to mount the subset of CVMFS as if it was a CVMFS client.

Nevertheless, as we explained in \ref{section:31}, a Gauss execution can involve different packages, extra packages, options, data and versions. Encapsulating its ecosystem requires a good understanding of the application and/or a large amount of storage to encapsulate the right dependencies.
Therefore, different options are available:
\begin{itemize}
    \item Include the whole LHCb CVMFS repository: would not require any specific knowledge about Gauss and would involve all the necessary files to run any Gauss instance. However, this option would imply a tremendous quantity of storage - the full LHCb repository needs 5.2 Terabytes -, long periods to update the subset and many unnecessary files.
    \item Include the dependencies of various Gauss runs: as the first option, would not need any specific knowledge about Gauss and would include a few gigabytes of data. Nevertheless, such an option would not guarantee the presence of all needed files and would require a tremendous amount of computing resources to trace Gauss workloads continuously.
    \item Include all the known dependencies of Gauss: would require a deep understanding of Gauss and its dependencies to include all the required files in a subset of CVMFS. While this option would not involve many computing or storage resources, it would include human resources to update the content of the subset of CVMFS according to the releases of Gauss and its extra packages. 
\end{itemize}

As the default storage quota on Mare Nostrum is smaller than the LHCb repository, we decided to reject the first option.
LHCb has access to tremendous computing power: it interacts with hundreds of WLCG Sites to run Gauss workloads and could theoretically trace them and extract their requirements.
In practice, tracing Gauss workloads in production could slow down the applications and their execution, which is not an option.
Similarly, LHCb does not have human resources to update the subset of CVMFS according to the changes done.
Thus, we chose to combine the second and the third options to propose a light and easy to update and maintain solution.
The process consists in getting insights into the structure of the Gauss dependencies by running and tracing a small set of Gauss workloads and analyzing the system calls before including the structure in \emph{subcvmfs-builder-pipeline}.

After analyzing 500 commands calling Gauss from the LHCb production environment and tracing 3 Gauss applications using \emph{subcvmfs-builder} \cite{gauss_analysis}, we noticed that:
\begin{itemize}
    \item 97\% of the workloads studied were running the same Gauss versions (v49r20) with the same extra packages and versions.
    The versions of Gauss and its extra packages seem related to the underlying architecture.
    \item 846 Mb of files were needed to run 3 Gauss (v49r20) workloads.
    About 95\% of the size is related to the Gauss version and the underlying architecture, and is common to the Gauss workloads traced, while the 5\% left is bound to the options and Geant4 data used that are specific to a given Gauss workload.
    \item Integrating all the options and Geant4 data related to Gauss v49r20 would correspond to 1.8 Gb of files.
\end{itemize}

Based on these assumptions, we created a \texttt{namelist} file containing (i) the files shared by the 3 Gauss applications that we traced and (ii) all the options and Geant4 data in order to generate a subset of CVMFS able to run any Gauss workload targeting the v49r20 version.
We used \emph{subcvmfs-builder-pipeline} to build the subset of CVMFS, to successfully test it with 5 Gauss workloads - different from the ones we used previously - and to deploy it to Mare Nostrum.
We fine-tuned the utility to disable the \emph{trace} step and to deploy the subset separately from the container.
Indeed, CernVM - the container that we use to provide a reproducible environment to the workload - does not need regular updates and merging it with the subset of CVMFS is a time-consuming operation.

This resulted in a CernVM singularity container occupying 6.4 Gb on the General Parallel File System (GPFS) of Mare Nostrum combined with a subset of CVMFS covering 6 Gb: dependencies occupies 3.2 Gb of space while 2.8 Gb are required for the \emph{cvmfs\_shrinkwrap} metadata.
Thus, 12.4 Gb of space on the GPFS of Mare Nostrum is currently sufficient to run 97\% of the Gauss workloads analyzed: 0.24\% of the LHCb repository.

% does it produce the same results? reproducibility
Even though this approach provides a light, easy and fast-to-update solution, LHCb developers need to keep it up to date to integrate new versions or structure changes.
One way to proceed would consist in automating and repeating the analysis work regularly. 
One could also integrate the \emph{trace} command of \emph{subcvmfs-builder} within the LHCb production test phase, which consists in running a few events of upcoming Gauss workloads on a given Grid Site.
LHCb developers could trace some of them during the process and store the traces in a database.
An LHCb-specific \emph{subcvmfs-pipeline-builder} could then periodically fetch the content of the database to build, test and deploy a new subset of dependencies to Mare Nostrum.

% container: passing from cern-vm to centos because a lot of unnecessary files


\section{Conclusion}
\section{Conclusion}\label{sec:conclusion}
In this work, we focus on addressing the fundamental challenge of OOD detection tasks, which is how to fully understand the semantic discrepancy between the ID/OOD samples. We reveal that the key to success in the realistic SCOOD task is to allocate as many ID samples in the unlabeled set correctly as possible. To this end, we propose a novel uncertainty-aware optimal transport scheme that introduces class-specific energy scores as guidance for effective label assignment. Experimental results show that our method achieves better performance than previous state-of-the-art methods on SCOOD benchmarks.

\textbf{Limitations.} In addition to temperature scaling, other techniques such as feature clipping applied in ReAct~\cite{sun2021react} also enhance the performance of energy score, so how to obtain an OOD score that best fits the SCOOD task can be further explored. Moreover, a setting highly related to SCOOD has been proposed in \cite{katz2022training} and formulated as a constrained optimization problem. We will also theoretically analyze these practical OOD settings in our feature work.

% \section*{Acknowledgments}
\textbf{Acknowledgments.} 
This work is supported by National Key R\&D Program of China under Grant 2020AAA0105701, National Natural Science Foundation of China (NSFC) under Grants 61872327, Major Special Science and Technology Project of Anhui, National Natural Science Foundation of China (62033012) and Ant Group through Ant Research Intern Program.

%
% ---- Bibliography ----
%
% BibTeX users should specify bibliography style 'splncs04'.
% References will then be sorted and formatted in the correct style.
%
\bibliographystyle{plain}
\bibliography{summary}
\end{document}
