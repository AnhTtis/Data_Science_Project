\subsection{Reconstructions}

Our forward solvers for \eqref{Helmholtz}, \eqref{imped} (in the special cases \eqref{Helmholtz_obj},\eqref{Helmholtz_pts}, of \eqref{Helmholtz}) rely on the fact that with the fundamental solution to the Helmhotz equation
$\mathcal{G}(x)=\frac{\imath}{4} H_0^1(\kaptil|x|)$ in two space dimensions, the solution to
\[
\triangle u^{\mathbb{R}^2} +\kaptil^2 u^{\mathbb{R}^2} = f \quad\mbox{ in }\mathbb{R}^2
\] 
can be determined by convolution $u^{\mathbb{R}^2}=\mathcal{G}*f$.
It thus remains to solve the homogeneous boundary value problem   
\[
\triangle u^{\textup{d}} +\kaptil^2 u^{\textup{d}} = 0 \quad\mbox{ in }\Omega\,, \quad
\partial_\nu u^{\textup{d}} +\imath\kaptil u^{\textup{d}} = g
\] 
with $g=-\partial_\nu u^{\mathbb{R}^2} -\imath\kaptil u^{\mathbb{R}^2}$, which we do by the integral equation approach described in \cite[Sections 3.1, 3.4]{ColtonKress:2013}, that easily extends to the case of impedance boundary conditions.
The solution to \eqref{Helmholtz}, \eqref{imped} is then obtained as $u=u^{\mathbb{R}^2}+u^{\textup{d}}$.
We point to the fact that solving the Helmholtz equation with large wave numbers is a challenging task and a highly active field of research, see, e.g., \cite{LafontaineSpenceWunsch:2022,MelenkSauterTorres:2020,PeterseimVerfuerth:2020} and the references therein.
Since our emphasis lies on a proof of concept for parameter identification, we did not implement any of these high frequency solvers here.

In all our reconstructions it is apparent that the point source reconstruction algorithm from \cite{BrediesPikkarainen:2013,PTTW:2020} combined with the equivelant discs approximation -- that is, steps (1.) and (2.) in Algorithm 1 -- provides an extremely good initial guess of the curves to be recovered. This is essential for the convergence of Newton's method in view of the high nonlinearity of the shape identification problem.

\medskip

\paragraph{Using the third harmonic $M=3$:}

\begin{figure}[htbp]
\begin{center}
\includegraphics[width=0.19\textwidth]{polpl_3obj_Tmeas1_Pts.eps}
\includegraphics[width=0.19\textwidth]{polpl_3obj_Tmeas1_Disks.eps}
\includegraphics[width=0.19\textwidth]{polpl_3obj_Tmeas1_Newton10.eps}
\includegraphics[width=0.19\textwidth]{polpl_3obj_Tmeas1_Newton15.eps}
\includegraphics[width=0.19\textwidth]{polpl_3obj_Tmeas1_Newtonsim.eps}
\\[0.5ex]
\includegraphics[width=0.19\textwidth]{polpl_2obj_Tmeas1_Pts.eps}
\includegraphics[width=0.19\textwidth]{polpl_2obj_Tmeas1_Disks.eps}
\includegraphics[width=0.19\textwidth]{polpl_2obj_Tmeas1_Newton10.eps}
\includegraphics[width=0.19\textwidth]{polpl_2obj_Tmeas1_Newton15.eps}
\includegraphics[width=0.19\textwidth]{polpl_2obj_Tmeas1_Newtonsim.eps}
\\[0.5ex]
\hspace*{0.08\textwidth}(a)\hspace*{0.165\textwidth}(b)\hspace*{0.165\textwidth}(c)\hspace*{0.165\textwidth}(d)\hspace*{0.165\textwidth}(e)\hspace*{0.08\textwidth}
\caption{Reconstruction of three (top row) or two (bottom row) inclusions from full data: (a) point sources step (1.) of Algorithm 1; (b) equivalent disks step (2.) of Algorithm 1; (c) Newton with second harmonic; (d) Newton with third harmonic; (e) Newton with second and third harmonic
\label{fig:3obj}}
\end{center}
\end{figure}

The reconstructions in Figure~\ref{fig:3obj} are obtained by following the steps of Algorithm 1 at wave number $\kaptil=10$ and then carrying out another Newton step with data from the third harmonic at $\kaptil=15$ either: (d) sequentially, using the result from  $\kaptil=10$ as a starting value or, (e) applying Newton's method simultaneously to $\kaptil=10$ and $\kaptil=15$.

The numerical results indicate that the additional information obtained from the next ($m=3$) harmonic does not yield much improvement. This is due to the lower -- by two to three orders of magnitude -- intensity of the signal at that higher frequency and seems to confirm the experimental evidence and common practice of skipping higher than second harmonics.  

\bigskip

\begin{figure}[htbp]
\begin{center}
\includegraphics[width=0.19\textwidth]{polpl_3obj_Tmeas1_Disks.eps}
\includegraphics[width=0.19\textwidth]{polpl_3obj_TmeasDot75_Disks.eps}
\includegraphics[width=0.19\textwidth]{polpl_3obj_TmeasDot5_Disks.eps}
\includegraphics[width=0.19\textwidth]{polpl_3obj_TmeasDot4_Disks.eps}
\includegraphics[width=0.19\textwidth]{polpl_3obj_TmeasDot3_Disks.eps}
\\[2ex]
\includegraphics[width=0.19\textwidth]{polpl_3obj_Tmeas1_Newton10.eps}
\includegraphics[width=0.19\textwidth]{polpl_3obj_TmeasDot75_Newton10.eps}
\includegraphics[width=0.19\textwidth]{polpl_3obj_TmeasDot5_Newton10.eps}
\includegraphics[width=0.19\textwidth]{polpl_3obj_TmeasDot4_Newton10.eps}
\includegraphics[width=0.19\textwidth]{polpl_3obj_TmeasDot3_Newton10.eps}
\\[2ex]
(a) $\frac{\alpha}{2\pi}=1$ \hspace*{0.05\textwidth}
(b) $\frac{\alpha}{2\pi}=0.75$ \hspace*{0.05\textwidth}
(c) $\frac{\alpha}{2\pi}=0.5$ \hspace*{0.05\textwidth}
(d) $\frac{\alpha}{2\pi}=0.4$ \hspace*{0.05\textwidth}
(e) $\frac{\alpha}{2\pi}=0.3$ \hspace*{-0.01\textwidth}
\caption{Reconstruction of three inclusions from partial data; 
top row: equivalent point sources and disks; bottom row: boundary curves from Newton's method
\label{fig:3obj_partial}}
\end{center}
\end{figure}

\begin{figure}[htbp]
\begin{center}
\includegraphics[width=0.19\textwidth]{polpl_2obj_Tmeas1_Disks.eps}
\includegraphics[width=0.19\textwidth]{polpl_2obj_TmeasDot75_Disks.eps}
\includegraphics[width=0.19\textwidth]{polpl_2obj_TmeasDot5_Disks.eps}
\includegraphics[width=0.19\textwidth]{polpl_2obj_TmeasDot4_Disks.eps}
\includegraphics[width=0.19\textwidth]{polpl_2obj_TmeasDot3_Disks.eps}
\\[2ex]
\includegraphics[width=0.19\textwidth]{polpl_2obj_Tmeas1_Newton10.eps}
\includegraphics[width=0.19\textwidth]{polpl_2obj_TmeasDot75_Newton10.eps}
\includegraphics[width=0.19\textwidth]{polpl_2obj_TmeasDot5_Newton10.eps}
\includegraphics[width=0.19\textwidth]{polpl_2obj_TmeasDot4_Newton10.eps}
\includegraphics[width=0.19\textwidth]{polpl_2obj_TmeasDot3_Newton10.eps}
\\[2ex]
(a) $\frac{\alpha}{2\pi}=1$ \hspace*{0.05\textwidth}
(b) $\frac{\alpha}{2\pi}=0.75$ \hspace*{0.05\textwidth}
(c) $\frac{\alpha}{2\pi}=0.5$ \hspace*{0.05\textwidth}
(d) $\frac{\alpha}{2\pi}=0.4$ \hspace*{0.05\textwidth}
(e) $\frac{\alpha}{2\pi}=0.3$ \hspace*{-0.01\textwidth}
\caption{Reconstruction of two inclusions from partial data;
top row: equivalent point sources and disks; bottom row: boundary curves from Newton's method
\label{fig:2obj_partial}}
\end{center}
\end{figure}

\paragraph{Reconstructions from partial data:} 

In Figures~\ref{fig:3obj_partial}, \ref{fig:2obj_partial} we show reconstructions from partial data.
The quality appears to decrease only slightly with decreasing amount of data, until at a certain point (between 30 and 40 per cent of the full angle) the algorithm partially breaks down and fails to find one of the objects completely.
The ability of an inclusion to stay reconstructible from a low amount of data is related to its weight according to the associated weight $\lambda_k$ according to \eqref{meanvalue_Helmholtz} (using the object's average radius). 
In Figures~\ref{fig:3obj_partial} and \ref{fig:2obj_partial} these weights are: 0.0725 for the circle, 0.0692 for the cardiod and 0.0515 for the ellipse.
Also, the position relative to the measurement boundary clearly plays a role.


It may seem that simple completion
of data from the measurement subarc to the entire boundary should give
similar results by for example using the Fourier series expansion.
%
However, this analytic continuation step comes at a price.
If we have $N$ Fourier modes over an arc of length $\alpha$
then this analytic continuation results from solving a system with a matrix $P(N,\alpha)$
the conditioning of which can be computed analytically.
Of course the condition number will increase with both $N$ and decreasing
values of $\alpha$, $0<\alpha<2\pi$.
In fact this is a well-understood problem, see \cite{Slepian:1983} where it has been shown that the condition number of $P(N,\alpha)$ is asymptotic (for large $N$) to
\begin{equation}\label{eqn:condSlepian}
c_N\sim e^{\gamma(\alpha) N}\textup{ where } 
\gamma(\alpha) = \log\Bigl(\frac{\sqrt{2}+\sqrt{1+\cos\alpha}}{\sqrt{2}-\sqrt{1+\cos\alpha}}\Bigr).
\end{equation}
This has been used in several inverse problems, see, e.g., \cite{HR:1997,Louis:1986}.

However, in our situation the reconstructions are performing much better than
the above pessismistic estimate would suggest.
% in our case, with M=4 we have N=9 and alpha corrsponds to Tmeas/T
%\todo{runs of force-h-imp-pts-partialdata; plot condJmat and cnd-Slepian for different values of Tmeas/T between 0 and 1 for a single object}
%The results are much better than to be expected from classical results about extending limited data on $\Sigma\subset\partial\Omega$ to the rest of the boundary \cite{Slepian:1983}.
This is due to the fact that our reconstruction does not rely on extending the boundary data but rather on  directly applying our method to the restricted flux $g=\partial_\nu \hat{p}\vert_\Sigma$. The additional information that the PDE model provides clearly contributes to this inprovement, which is also reflected in the condition number of the Jacobian in Newton's method versus the theoretical prediction for data completion from \cite{Slepian:1983}. This can be seen in Table~\ref{tab:conds}. 
%The underlying number of basis functions is $N=9$.
\begin{table}[htbp]
\begin{center}
\begin{tabular}{|c|c|c|}\hline
$\frac{\alpha}{2\pi}$ & cond(J) & $c_N$ \cite{Slepian:1983}\\ \hline
0.75 & 29.6 & 2.8e+2\\
0.5 & 64.9 & 2.3e+5\\
0.4& 73.7 & 1.8e+07\\
0.3& 1733.8& 2.6e+08\\ \hline
\end{tabular}
\end{center}
\caption{Condition numbers of Jacobian in Newton's method for a single inclusion using $9$ basis functions versus condition number formula \eqref{eqn:condSlepian} for data completion with $N=9$
\label{tab:conds}}
\end{table}
%fraction-angle =   0.7500:
%condJmat =  29.5862
%cndSlepian =  282.8159

%fraction-angle =   0.5000:
%condJmat =  64.8617
%cndSlepian =  2.2849e+05

%fraction-angle =   0.3750:
%condJmat =  73.7328
%cndSlepian =  1.7900e+07

%fraction-angle =   0.3125:
%condJmat =  1.7338e+03
%cndSlepian =  2.6122e+08

\bigskip

\begin{figure}[htbp]
\begin{center}
\includegraphics[width=0.19\textwidth]{polpl_2obj_distDot3_Disks.eps}
\includegraphics[width=0.19\textwidth]{polpl_2obj_distDot2_Disks.eps}
\includegraphics[width=0.19\textwidth]{polpl_2obj_distDot1_Disks.eps}
\includegraphics[width=0.19\textwidth]{polpl_2obj_distDot09_Disks.eps}
\\[2ex]
\includegraphics[width=0.19\textwidth]{polpl_2obj_distDot3_Newton10.eps}
\includegraphics[width=0.19\textwidth]{polpl_2obj_distDot2_Newton10.eps}
\includegraphics[width=0.19\textwidth]{polpl_2obj_distDot1_Newton10.eps}
\includegraphics[width=0.19\textwidth]{polpl_2obj_distDot09_Newton10.eps}
\\[2ex]
(a) $\frac{\theta}{2\pi}=0.3$ \hspace*{0.05\textwidth}
(b) $\frac{\theta}{2\pi}=0.2$ \hspace*{0.05\textwidth}
(c) $\frac{\theta}{2\pi}=0.1$ \hspace*{0.05\textwidth}
(d) $\frac{\theta}{2\pi}=0.09$ \hspace*{-0.01\textwidth}
\caption{Reconstruction of two inclusions at different distances;
top row: equivalent point sources and disks; bottom row: boundary curves from Newton's method
\label{fig:2obj_dist}}
\end{center}
\end{figure}

\paragraph{Varying distance between objects:} 
Figure~\ref{fig:2obj_dist} shows reconstructions of two inclusions at several distance, given by the difference $\theta$ in the phase of the centroid (in polar coordinates).
%The actual weights are 0.0725 for the left and 0.0515 for the right object.
%The relative error after application of Newton's method at $\kaptil=10$ was (a) 0.1430 (b) 0.1385 (c) 0.2247 (d) (for the single found object only) 0.1896.
The given data appears to allow distinction of objects very well, as long as they do not overlap. However, decreasing distance between them compromises the quality of reconstructions. 

\bigskip

\begin{figure}[htbp]
\begin{center}
\includegraphics[width=0.19\textwidth]{polpl_1obj_distbndyPos1_Disks.eps}
\includegraphics[width=0.19\textwidth]{polpl_1obj_distbndyPos2_Disks.eps}
\includegraphics[width=0.19\textwidth]{polpl_1obj_distbndyPos3_Disks.eps}
\\[2ex]
\includegraphics[width=0.19\textwidth]{polpl_1obj_distbndyPos1_Newton10.eps}
\includegraphics[width=0.19\textwidth]{polpl_1obj_distbndyPos2_Newton10.eps}
\includegraphics[width=0.19\textwidth]{polpl_1obj_distbndyPos3_Newton10.eps}
\\[2ex]
(a) \hspace*{0.15\textwidth}
(b) \hspace*{0.15\textwidth}
(c) 
\caption{Reconstruction of one inclusion at different distances from the boundary;
top row: equivalent point sources and disks; bottom row: boundary curves from Newton's method
\label{fig:2obj_dist}}
\end{center}
\end{figure}
\paragraph{Varying distance to boundary:} 
Figure~\ref{fig:2obj_dist} shows reconstructions of one inclusion at several distances from the boundary. The relative error after application of Newton's method at $\kaptil=10$ was (a) 0.2963 (b) 0.1931 (c) 0.1434.
Also visually, it is obvious that closeness to the observation surface significantly improves the reconstruction quality.

\begin{figure}[htbp]
\begin{center}
\includegraphics[width=0.19\textwidth]{polpl_3obj_deltaDot02_Disks.eps}
\includegraphics[width=0.19\textwidth]{polpl_3obj_deltaDot03_Disks.eps}
\includegraphics[width=0.19\textwidth]{polpl_3obj_deltaDot02partial_Disks.eps}
\includegraphics[width=0.19\textwidth]{polpl_3obj_deltaDot03partial_Disks.eps}
\\[2ex]
\includegraphics[width=0.19\textwidth]{polpl_3obj_deltaDot02_Newton10.eps}
\includegraphics[width=0.19\textwidth]{polpl_3obj_deltaDot03_Newton10.eps}
\includegraphics[width=0.19\textwidth]{polpl_3obj_deltaDot02partial_Newton10.eps}
%\includegraphics[width=0.19\textwidth]{polpl_3obj_deltaDot03partial_Newton10.eps}
\hspace*{0.19\textwidth}
\\[2ex]
(a) $\delta=2$ \% \hspace*{0.05\textwidth}
(b) $\delta=3$ \% \hspace*{0.05\textwidth}
(c) $\delta=2$ \% \hspace*{0.05\textwidth}
(d) $\delta=3$ \% \hspace*{-0.01\textwidth}\\
\hspace*{0.42\textwidth}
$\frac{\alpha}{2\pi}=0.5$ \hspace*{0.09\textwidth}
$\frac{\alpha}{2\pi}=0.5$\\
\caption{Reconstruction of three inclusions from noisy data;
top row: equivalent point sources and disks; bottom row: boundary curves from Newton's method
\label{fig:3obj_noise}}
\end{center}
\end{figure}
\paragraph{Reconstruction from noisy data:} 
Finally we study the impact of noise in the measurements on the reconstruction quality, see Figure~\ref{fig:3obj_noise} for the case of three objects. Regularisation is mainly achieved by the sparsity prior incorporated via the PDAP point source identification and this actually makes the process very stable with respect to perturbations in the measurements up to noise levels of about three per cent. 
Using partial data clearly impacts this robustness and thus only works with noise levels of two per cent or less.
 
%\begin{itemize}
%\item two inclusions from full data with third harmoinc
%\item three inclusions from full data with third harmoinc
%\item two inclusions from partial data with third harmoinc
%\item \todo{one object from partial data and plotting condition numbers}
%\item \todo{vary distance between objects and try different frequencies}
%\item \todo{vary size of (single) object} NO
%\item \todo{vary distance of (single) object from observation boundary}
%\end{itemize}

