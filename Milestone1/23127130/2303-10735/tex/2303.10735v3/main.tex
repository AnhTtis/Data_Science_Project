 \documentclass[10pt,twocolumn,letterpaper]{article}

\usepackage{iccv}
\usepackage{times}
\usepackage{epsfig}
\usepackage{graphicx}
\usepackage{caption}
\usepackage{amsmath}
\usepackage{amssymb}
\usepackage{booktabs}  
\usepackage{float}
\usepackage{dsfont}
\usepackage[section]{placeins}
\usepackage{overpic}
\usepackage{wrapfig}
\usepackage[dvipsnames]{xcolor}
%\usepackage{color}
\definecolor{green}{rgb}{0, 0.5, 0}
\definecolor{orange}{rgb}{0.8, 0.6, 0.2}
\definecolor{red}{rgb}{1.0, 0.0, 0.0}
\definecolor{teal}{rgb}{0.0, 0.4, 0.4}
\definecolor{purple}{rgb}{0.65,0,0.65}
\definecolor{saffron}{rgb}{0.95,0.75,0.2}
\definecolor{turquoise}{rgb}{0.0,0.5,0.5}
\definecolor{black}{rgb}{0.0, 0.0, 0.0}
\definecolor{gray}{rgb}{0.5, 0.5, 0.5}
\definecolor{blue}{rgb}{0.0, 0.0, 1.0}

\newcommand{\fg}[1]{{\color{orange}#1}}
\newcommand{\fgc}[1]{{\color{gray}[FG: #1]}}

\newcommand{\rqw}[1]{{\color{black}#1}}
\newcommand{\rqc}[1]{{\color{teal}[RQ: #1]}}

\newcommand{\agp}[1]{{\color{black}#1}}
\newcommand{\rz}[1]{{\color{black}#1}}
\newcommand{\rzz}[1]{{\color{black}#1}}
\newcommand{\chk}[1]{{\color{black}#1}}

\newcommand{\xeable}{{\color{black}moveable~}}
\newcommand{\Xeable}{{\color{black}Moveable~}}

% Include other packages here, before hyperref.


% If you comment hyperref and then uncomment it, you should delete
% egpaper.aux before re-running latex.  (Or just hit 'q' on the first latex
% run, let it finish, and you should be clear).
\usepackage[pagebackref=true,breaklinks=true,letterpaper=true,colorlinks,bookmarks=false]{hyperref}

 \iccvfinalcopy % *** Uncomment this line for the final submission

\def\iccvPaperID{1206} % *** Enter the ICCV Paper ID here
\def\httilde{\mbox{\tt\raisebox{-.5ex}{\symbol{126}}}}

% Pages are numbered in submission mode, and unnumbered in camera-ready
\ificcvfinal\pagestyle{empty}\fi

\begin{document}


%%%%%%%%% TITLE
\title{SKED: Sketch-guided Text-based 3D Editing}

\author{Aryan Mikaeili$^{1}$ \hspace{6mm}
Or Perel$^{2}$ \hspace{6mm}
Mehdi Safaee$^{1}$ \hspace{6mm}
Daniel Cohen-Or$^{3}$\hspace{6mm}
Ali Mahdavi-Amiri$^{1}$\\\\
$^1$Simon Fraser University \hspace{0.25cm}
$^2$NVIDIA \hspace{0.25cm}
$^3$Tel Aviv University
\vspace{4mm}
}

\twocolumn[{%
\maketitle
\vspace{-14mm}
\begin{figure}[H]
\hsize=\textwidth
\centering
\includegraphics[width=\textwidth]{figs/teaser_4.png}
\captionsetup{font=small}
\caption{Examples of our \textbf{Sketch-guided}, \textbf{Text-based} 3D editing method. Taking a pretrained Neural Radiance Field as input, multiview sketches determining the coarse region of edit and a text-prompt, our method is able to generate a localized, meaningful edit.
}
\vspace{0.5cm}
\label{fig:teaser}
\end{figure}
}]


% Remove page # from the first page of camera-ready.
\ificcvfinal\thispagestyle{empty}\fi




\begin{abstract}
% \vspace{-1em}
The diffusion-based generative models have achieved remarkable success in text-based image generation. However, since it contains enormous randomness in generation progress, it is still challenging to apply such models for real-world visual content editing, especially in videos. 
In this paper, we propose \texttt{FateZero}, a zero-shot text-based editing method on real-world videos without per-prompt training or use-specific mask. 
\RM{Specifically, different from a pipeline of two independent inversion and then generation stages, we find the intermediate attention maps during inversions store better structure and motion information. We thus reform them to temporally casual attention and replace them in the generation progress. To further reduce the unnecessary semantic leakage of source video and enhance the editing quality, we then remix the temporally casual attentions via the cross-attention features of the source prompt as the mask.}
To edit videos consistently, we propose several techniques based on the pre-trained models. Firstly, in contrast to the straightforward DDIM inversion technique, our approach captures intermediate attention maps during inversion, which effectively retain both structural and motion information. These maps are directly fused in the editing process rather than generated during denoising. To further minimize semantic leakage of the source video, we then fuse self-attentions with a blending mask obtained by cross-attention features from the source prompt. Furthermore, we have implemented a reform of the self-attention mechanism in denoising UNet by introducing spatial-temporal attention to ensure frame consistency.
Yet succinct, our method is the first one to show the ability of zero-shot text-driven video style and local attribute editing from the trained text-to-image model. We also have a better zero-shot shape-aware editing ability based on the text-to-video model~\cite{tuneavideo}. \RM{Besides video, our unified method also achieves state-of-the-art performance in zero-shot image editing.\chenyang{Need exp or remove the zero-shot image}} Extensive experiments demonstrate our superior temporal consistency and editing capability than previous works.
% The code will be released.
% \chenyang{emphasize: our observation at inversion time} \xiaodong{replacing the bold part to the actual pipeline: \textbf{Specifically, we work on replacing and mixing the attention maps between the inversion and generation since the self-attention map keeps the structure of the original natural image and the cross-attention is semantic-related, after remixing, we replace them in the corresponding generation steps for denoising.}}
% \footnote{Since there is no general video diffusion model is publicly available, we use one-shot video generation method~(Tune-A-Video~\cite{tuneavideo}) as the pretrained video diffusion model for zero-shot video editing\xiaodong{can be removed if we actually zero-shot on video}.}.
\end{abstract}
\section{Introduction}

\label{sec:intro}

% \textit{"Drawing and colour are not separate at all; in so far as you paint, you draw. The more the colour harmonizes, the more exact the drawing becomes."} - Paul Cezanne.

Art is a reflection of the figments of human imagination. 
While many are limited in their practical creative capabilities, by pushing the boundaries of digital media, new ways can be created for casual artists and experts alike to express their ideas. At the same time, current neural generative art takes away much of the control from humans. In this work, we attempt to take a step towards restoring some of that control, enabling neural networks to complement users and naturally extend their skills rather than taking hold over the generative process.

% \orr{TBD - make the opening colorful : 1. Add quote:  2. Elaborate: art is a rendering of figments of imaginations of humans. Most people are limited in their drawing capabilities, and by pushing the boundaries we allow new ways for casual artists and experts alike in expressing ideas. At the same time, neural generative art takes a lot of the control away. Here, we want to give back some of this control to humans, such that neural networks complement them and compensate their lack of skills, rather than replacing them.}

% The field of image synthesis has been significantly propelled by neural generative models, particularly by the latest text-to-image models that predominantly rely on large language-image models ~\cite{balaji2022eDiff-I, ramesh2022dalle, rombach2021highresolution, imagen2022saharia}. These models have revolutionized the field of computer vision as they can produce astonishing visual outcomes from text prompts only.

The field of image synthesis has been significantly propelled by neural generative models, particularly by the latest text-to-image models that predominantly rely on large language-image models ~\cite{balaji2022eDiff-I, ramesh2022dalle, rombach2021highresolution, imagen2022saharia}. These models have revolutionized the field of computer vision, as they can produce astonishing visual outcomes from text prompts alone.

The ability of text-to-image models has sparked a wave of editing methods that utilize these models. Many of these techniques rely on prompt editing ~\cite{ fu2022shapecrafter, hertz2022prompt, kawar2022imagic,lin2022magic3d,mokady2022null, poole2022dreamfusion}. Nevertheless, simplifying the interface to text alone means users lack the necessary level of granularity to produce their exact desired outcomes.
% which is} insufficient for effectively editing local content. 
% editing and manipulating visual content, as users lack the necessary level of control to achieve their desired outcomes
Sketch-guided editing, on the other hand, provides intuitive control that aligns with user's conventional drawing and painting skills. By incorporating user-guided sketches into text-to-image models, powerful editing systems can be created, offering a high degree of flexibility and fine-grained control for manipulating visual content~\cite{zhang2023controlnet, voynov2022sketch}.

Although sketch-guided and text-driven methods have proven successful in generating and manipulating 2D images \cite{meng2022sdedit, voynov2022sketch, cheng2023wacv}, it immediately raises the intriguing question of whether a similar approach could be developed to edit 3D shapes. 
Since direct text-to-3D models require an abundance of data to scale, state-of-the-art 3D generative models, such as DreamFusion~\cite{poole2022dreamfusion} and Magic3D~\cite{lin2022magic3d}, which build on the capabilities of text-to-image models, may be considered as an alternative.
% Due to the difficulty of scaling general direct text-to-3D models, incorporating conditions into a text-to-3D model is not straightforward. Thus, state of the art 3D generative models, such as DreamFusion~\cite{poole2022dreamfusion} \orrc{and Magic3D~\cite{lin2022magic3d}}, which build on the capabilities of text-to-image models, may be considered as an alternative.
However, maintaining control via conditioning with such models remains a challenging task, as these generative pipelines optimize a Neural Radiance Field (NeRF) \cite{mildenhall2020nerf} by amortizing gradients from a multitude of 2D views. In particular, providing consistent sketches across all possible views presents a hurdle for users. Instead, a plausible user interface should act with guidance from as few views as possible, e.g. up to two or three.


In this paper, we present \textbf{SKED}, a \textbf{SK}etch-guided 3D \textbf{ED}iting technique. Our method acts on reconstructed or generated NeRF models. We assume a text prompt and a minimum of two sketches as input and provide output edits over the neural field faithful to the input conditions.
Meeting all input requirements can be challenging as the text prompt may not match the sketch's semantics, and sketch views may lack coherence.
To undertake this complex task, we conceptually break it down into two subtasks that are easier to handle: one that depends on pure geometric reasoning and the other that exploits the rich semantic knowledge of the generative model. These two subtasks work together, with the former providing a coarse estimate of location and boundary, and the latter adding and refining geometric and texture details through fine-grained operations.


Our experiments highlight the effectiveness of our approach for editing various pretrained NeRF instances. We introduce assorted accessories, objects, and artifacts, which are generated and blended into the original neural field seamlessly.
Finally, we validate our method through quantitative evaluations and ablation studies to assert the contribution of individual components in our method. 
% By presenting examples in the paper, we illustrate that our method can generate realistic 3D artifacts with accurate texture and geometry using only a few basic sketches.



% Due to the absence of a direct text-to-3D model, incorporating conditions into a text-to-3D model is not straightforward. Thus, 3D generative models, such as DreamFusion~\cite{poole2022dreamfusion}, which build on the capabilities of text-to-image models, may be considered as an alternative.
% However, this is a challenging task since DreamFusion generates a NeRF by integrating many different 2D views. It is very hard to provide consistent sketches across all possible views. The challenge is to use sketches as a guide on only a few views (e.g., two or three) and generate 3D edit of the existing NeRF that is subject to being edited. 

% In this paper, we present \textbf{SKED}, a \textbf{SK}etch-guided 3D \textbf{ED}iting technique, that takes as input a text prompt and a few (two or more) sketches and edits a 3D given object represented as a NeRF in a geometrically plausible and controlled way. 
% We acknowledge the difficulty of this task, as there are no existing text-to-3D generative models available for manipulating the geometry of the existing object based on a text prompt. 
% To undertake this complex task, we conceptually break it down into two simpler subtasks that are easier to handle: one that depends on pure geometric reasoning and the other that exploits the rich semantic knowledge of generative model. These two subtasks work together, with the former providing a coarse estimate of location and the latter adding and refining geometric and texture details through fine-grained operations.

% Our experiments showcase the effectiveness of our approach in performing sketch-guided text-based edits on different base nerfs by introducing various accessories, objects, and artifacts. We also conduct ablation studies and experiments to evaluate the performance of individual components in our method. By presenting examples in the paper, we illustrate that our method can generate realistic 3D artifacts with accurate texture and geometry using only a few basic sketches.

%\dcc{Add here the traditional paragraph that tell about what we achieved and evaluated}
\section{Related work}
\label{sec:related}

\mypara{Learning fonts}
There are numerous techniques to study, design, and stylize fonts. 
%Kautz~\cite{campbell2014learning} utilized GP-LVM algorithm to learn a font manifold from  the polyline representation of glyph outlines.
Campbell and Kautz~\cite{campbell2014learning} utilized an algorithm to learn a font manifold from  the polyline representation of glyph outlines. By exploring this manifold, new fonts can be obtained, or existing fonts can be interpolated to achieve a desired effect. 
Balashova et al.~\cite{balashova2019learning} proposed an approach that uses a stroke-based geometric model for glyphs and a fitting procedure to reparametrize arbitrary fonts to the representation, which is again estimated through a manifold learning technique that estimates a low-dimensional font space.
More recently, Wang et al.~\cite{wang2021deepvecfont} proposed a dual-modality learning scheme to synthesize vector fonts, which are refined with glyph images using differentiable rasterization.

\mypara{Font stylization}
Efforts have also been made to stylize fonts to enhance their artistic and aesthetic appeal.
For instance, Azadi et al.~\cite{azadi2018multi} proposed a conditional-GAN~\cite{mirza2014conditional,isola2017image} to generate glyph images with different font and texture styles that match a given template.
Berio et al. ~\cite{berio2022strokestyles} proposed to segment a font’s glyphs into a set of overlapping and intersecting strokes to generate artistic stylizations.
To accomplish context-aware text image stylization and synthesis, Yang et al.~\cite{yang2017awesome,yang2018context1,yang2018context2} proposed a style transfer method with the ability to preserve legibility.
Nonetheless, artistic typography surpasses mere manipulation of fonts and often entails imbuing letters with a semantic meaning.

\mypara{Semantic and artistic typography}
Semantic typography involves adding certain elements to a text to emphasize certain aspects, communicate a message, or highlight a property. There have been several endeavors to integrate multiple elements in the creation of a logo, text, or other design elements. One example of this is through the use of collage-based techniques to fill a letter by incorporating semantic elements within it ~\cite{kwan2016pyramid,saputra2019improved,chen2019manufacturable,zhang2022creating}.
The concept of legible calligrams~\cite{zou2016legible} focuses on solving an inverse problem by placing a word or group of letters into a semantic shape. When our approach is applied to an entire word as the input (as shown in Figure~\ref{fig:res_qual_multi_letter}), it may yield results that resemble legible calligrams.
However, our problem statement is entirely distinct, and our approach to solving it is significantly dissimilar because our method is learning-based and not limited to a predetermined template shape that dictates the letter placement and deformation.

The works that are most closely related to ours are those that attempt to alter a letter or a set of letters to achieve a specific semantic meaning, whether through replacement, deformation, or texturization.
For instance, Zhang et al. \cite{zhang2017synthesizing} propose a semi-manual technique that involves manually dividing a letter into sections, fitting semantically related shapes to those sections, and performing post-processing to eliminate artifacts. However, the effectiveness of this method is heavily reliant on the accuracy of manual letter segmentation and the use of predefined shapes, which can impact the quality of the results.
Trick or treat~\cite{tendulkar2019trick} is an attempt to replace a letter with an icon by identifying an icon that closely matches the letter from a joint embedding of letters and icons. The chosen icon is then slightly deformed to better represent the letter. However, this method requires the existence of an icon that closely resembles the letter in order to produce satisfactory results.
Instead, we argue that a more effective solution would involve learning how to \emph{generate} the desired semantic typography by creating letters or words that convey a particular semantic or aesthetic feature in a subtle yet effective manner, as depicted in Figure~\ref{fig:res_qual_single_letter_composed}.

\mypara{Text-based generative design}
Large Language Models such as BERT~\cite{devlin2018bert} significantly advance the understanding of human language, which makes text-based generation tasks much easier.
With the emergence of powerful models that can establish connections between natural language and images, such as CLIP~\cite{radford2021learning}, several downstream tasks have benefited from these models, including mesh and image editing, stylization, and generation~\cite{michel2022text2mesh,jain2022zero,wang2022clip,mildenhall2020nerf}. Of particular relevance to our work, CLIPDraw~\cite{frans2021clipdraw} aims to produce SVG-format drawings by first rasterizing them using DiffVG~\cite{li2020differentiable}, and then utilizing CLIP for evaluation.
Recently, there has been a surge in popularity of diffusion models~\cite{sohl2015diffusion, ho2020ddpm, song2020denoising, song2020improvedsd}, including stable diffusion~\cite{rombach2022high}, which has produced impressive text-to-image results. Our approach utilizes latent diffusion~\cite{rombach2022high} to encode and decode glyph and style images, and employs BERT to condition the denoising process (Figure~\ref{fig:method_pipeline}). To ensure both glyph and style images are respected, we utilize a discriminator to combine adversarial learning and diffusion, making it one of the first of its kind. In contrast to Diffusion-GAN~\cite{wang2022diffusion} which uses a discriminator to distinguish a diffused real image from a diffused fake image at all steps, our discriminator is designed to preserve glyph structures in stylized images as one component of our optimization objectives.

\mypara{Semantic Typography}

In concurrent work, Word-as-Image Semantic Typography (ST) \cite{iluz2023word2img} stylizes a letter through a semantic-aware {\em font deformation\/}; see Figure~\ref{fig:semantic_comp}. 
To guide the deformation, ST uses a pre-trained Stable Diffusion model along with losses to preserve the font structure. 
Our approach differs from ST in multiple ways. We focus on extracting salient features of a style and applying them to a glyph shape and ensure legibility using a discriminator rather than deforming a font. This allows us to incorporate multiple relevant colors to semantic and stylistic attributes in raster form, while the results produced by ST remain single color in vector form. In addition, we can stylize multiple letters together as a single shape (Figure~\ref{fig:res_qual_multi_letter}) while ST deforms each letter individually.



%provide users with multiple outputs for each style + glyph combination, and
%Our goal is to inspire artistic creation and explore the possibilities of adapting text-to-image generators for personalized user experiences.
 % (1) using DiffVG to end2end fine-tune SVG outlines (2) structure and style preserving losses via a low pass filter
%It is an end-to-end network that fine-tunes SVG images using DiffVG \cite{Li:2020:DVG}, a differentiable rasterizer for vector images. 

%They present structure and style preserving losses for a stylized output where font structure is maintained.
%Unlike ST we do not aim to deform a font, rather our aim is to extract underlying salient or representative features of a style and apply them onto a glyph shape. Therefore, we are not bound by the topology of the font. Furthermore, we have the option to provide users multiple outputs for each style + glyph combination, unlike ST which generates one stylistic output per training. We also have the flexibility to incorporate \am{multiple relevant colors to a semantic and} stylistic attributes like Picasso, pixel, sculpture, etc. Most importantly, our work is aimed at artistic creation and inspiring new ideas. We aim to fully explore the possibilities of adapting powerful text-to-image generators for personalized user experience. }


% (1) Unlike ST we are not aiming to deform a specific font. Rather our aim is to abstractly incorporate elements of a style by extracting its underlying salient or representative features onto a glyph. 
% (2) ST is bound by the topology of their single input letter. In one-font mode we are able to follow the shape of the font, but we are not bound by the topology.
% (3) We can 'theoretically' generate infinitely multiple results for each style + glyph combination, while ST works to fine-tune to one output for each combination. In this way, we give end users options to choose from.
% (4) Building on point.4 , we give users the option to include attributes like Picasso, silhouette, pixel, etc. 
% (5) Finally, and most importantly perhaps, our work is aimed at artistic creation, to inspire new ideas. By using the stable diffusion model's powerful generator, we aim to fully explore the possibilities of how it can be adapted to a personalized user experience.





%and BERT~\cite{devlin2018bert}
%Diffusion-GAN~\cite{wang2022diffusion} introduced a discriminator learning to distinguish a diffused real image from a diffused fake image at all steps. 
%The discriminator in our method, however, aims to preserve the glyph structures in stylized images, and it is only one part of our optimization goals.








%\mypara{Combining GANs with Diffusion}
%
%Diffusion-GAN~\cite{wang2022diffusion} introduced a discriminator learning to distinguish a diffused real image from a diffused fake image at all diffusion steps. Different from Diffusion-GAN, the discriminator in our method aims to preserve the glyph structures in stylized images, and it is only one part of our optimization goals.
%building connections between images from two different domains.
%in our method is built on top of the output of Diffusion Models. It 
%The discriminator in Diffusion-GAN is still the same as vanilla GAN, which moves the distributions of fake images to real (target) images.






%introduced a fully automatic method, following the steps of path generation, alignment and deformation, to generate legible compact calligrams.
%To handle this task, Kwan et al.~\cite{kwan2016pyramid} presented pyramid of arclength descriptor (PAD) for partial-shape matching to obtain compact results.
%Saputra et al.~\cite{saputra2019improved} presented RepulsionPak, a deformation-driven packing technique, to handle the situations when elements typically do not fit perfectly with each other or the target container.
%When the target shape is a boundary, Chen et al.~\cite{chen2019manufacturable} simplified the process into heuristic ellipse packing, coarse-level user interaction, and fine-level optimization of the element placement.

%Nagata and Imahori~\cite{nagata2021escherization} handles the Escherization problem which finds the most similar shape to a given goal figure that can tile the plane.
%TReAT~\cite{tendulkar2019trick} aimed to retrieve similar icons with given glyphs by measuring their distance in a learned latent space.
%Text2mesh~\cite{michel2022text2mesh}, Dream Fields~\cite{jain2022zero} and Clip-nerf~\cite{wang2022clip} tried to generate meshes or neural radiance fields~\cite{mildenhall2020nerf} of objects with the guidance of CLIP~\cite{}.
\begin{figure*}[t!]
\includegraphics[width=1.0\linewidth, trim={0 0.3cm 0 0.1cm}, clip]{figures/architecture/architecture.pdf}
\vspace{-15pt}
\caption{
\textbf{Point2Vec pre-training.}
Our model divides the input point cloud into %
point patches using farthest point sampling (FPS) and $k$-NN aggregation.
We obtain patch embeddings by applying a mini-PointNet\,\colorsquare{m_pointnet} to each point patch (\emph{right}).
The teacher Transformer encoder\,\colorsquare{m_green} infers a contextualized %
representation for all patch embeddings which, after normalization and averaging over the last $K$ Transformer layers, serve as training targets.
The student's input is a masked view on the input data, \ie we randomly mask out a ratio of patch embeddings and only pass the remaining embeddings into the student Transformer encoder\,\colorsquare{m_blue}.
After applying a shallow decoder\,\colorsquare{m_red} on the outputs of the student, padded with learned mask embeddings\,\protect\maskembedding{}, we train the student and decoder to predict the latent teacher representation of the patch embeddings.
\vspace{-10pt}
}
\label{fig:model}
\end{figure*}
\section{Method}

The aim of this work is to unlock the full potential of data2vec-like\,\cite{baevski2022data2vec} pre-training on point clouds by addressing point cloud specific challenges.
To achieve this, we first summarize the technical concepts of data2vec (\refsec{method_d2v}) and show how to learn rich representations on point clouds using data2vec pre-training (\refsec{method_d2v_pcl}).
Finally, we propose \name{}, which accounts for the point cloud specific limitations of data2vec (\refsec{method_p2v}).

\subsection{Data2vec}\label{sec:method_d2v}
Data2vec\,\cite{baevski2022data2vec} is designed to pre-train Transformer-based models, which involve a feature encoder that maps the input data to a sequence of embeddings.
These embeddings are subsequently passed to a standard Transformer encoder to generate the final latent representations.
During pre-training, two versions of the Transformer encoder are kept: a \emph{student} and a \emph{teacher}.
The teacher is a momentum encoder, \ie its parameters $\Delta$ track the student's parameters $\theta$ by being updated after each training step according to an exponential moving average (EMA) rule\,\cite{caron2021dino, baevski2022data2vec, grill2020BYOL, he2020moco}: $\Delta \leftarrow \tau \Delta + (1-\tau)\theta$,
where $\tau \in [0,1]$ is the EMA decay rate.
The teacher provides the training targets, which the student predicts given a corrupted version of the same input.

In a first step, the teacher encodes the uncorrupted input sequence.
The training targets are then constructed by averaging the outputs of the last $K$ blocks of the teacher, which are normalized beforehand to prevent a single block from dominating the sum.
Due to the self-attention layers, these targets are \emph{contextualized}, \ie they incorporate global information from the whole input sequence.
This is an important difference to other masked-prediction methods such as BERT\,\cite{devlin2018bert} and MAE\,\cite{he2022mae}, where the targets only comprise local information, \eg a word or an image patch. %

The student is given a masked version of the same input, where some of the embeddings in the input sequence are substituted by a special learned \emph{mask embedding}. %
The student's task is to predict the targets corresponding to the masked parts of the input.
The model is trained by optimizing a Smooth L1 loss on the regressed targets. %







\subsection{Data2vec for Point Clouds}\label{sec:method_d2v_pcl}

To apply data2vec to point clouds, we utilize the same underlying model as Point-BERT\,\cite{yu2021pointbert} and Point-MAE\,\cite{pang2022pointmae}.
This model is well suited for data2vec pre-training: it extracts a sequence of patch embeddings from the input point cloud and feeds it to a standard Transformer encoder.
For downstream tasks, we append a task-specific head to the Transformer encoder (\refsec{experiments}).
Next, we describe the point cloud embedding and the Transformer in detail and conclude with a summary of data2vec for point clouds.


\parag{Point Cloud Embedding.}
First, we sample $n$ center points from the input point cloud using farthest point sampling (FPS)\,\cite{qi2017pointnetplusplus}.
Grouping the center points' $k$-nearest neighbors ($k$-NN) in the point cloud yields $n$ contiguous \emph{point patches}, \ie sub-clouds of $k$ elements.
Next, we normalize the point patches by subtracting the corresponding center point from the patch's points.
This untangles the positional and the structural information.
To account for the permutation-invariant property of point clouds, we employ a mini-PointNet\,\cite{qi2016pointnet} (\reffig{model}, \emph{right}) that maps each normalized point patch to a \emph{patch embedding}.

The mini-PointNet involves the following steps:
First, we map each point of a patch to a feature vector using a shared MLP.
Then, we concatenate max-pooled features to each feature vector.
The resulting feature vectors are then passed through a second shared MLP and a final max-pooling layer to obtain the patch embedding.

\paragraph{Transformer Encoder.}
The central component of the model is a standard Transformer encoder.
The patch embeddings form the input sequence to the Transformer encoder.
Since the point patches are normalized, the patch embeddings carry no positional information;
therefore, a two-layer MLP maps each center point to a position embedding, which is then added to the corresponding patch embedding.
Due to the special importance of positional information in point clouds, the position embeddings are added again before each subsequent Transformer block to ensure that the positional information is incorporated at every step of the encoding process.

\paragraph{\emakefirstuc{\datavec{}}.}

To establish a baseline, we apply the unmodified data2vec approach to the previously described underlying model of Point-BERT and Point-MAE.
Going forward, we will refer to this approach as \datavec{}.


\subsection{\emakefirstuc{\name{}}}\label{sec:method_p2v}
In \reffig{model}, we present the complete pipeline of our \name{} model.
Directly applying data2vec to point cloud data without modifications is not optimal, as the position embeddings are also added to the mask embeddings, revealing the overall shape of the point cloud to the student.
As positions are the only features for point clouds, this makes the masking far less effective, as noted by Pang \etal \cite{pang2022pointmae} in the context of masked autoencoders.

To solve this issue, we adopt an approach inspired by MAE\,\cite{he2022mae}, where we only feed the non-masked embeddings to the student\,\colorsquare{m_blue}.
A separate decoder\,\colorsquare{m_red}, implemented as a shallow Transformer encoder, takes the output of the student and the previously held-back masked embeddings\,\maskembedding{} as input and predicts the training targets.
In contrast to \datavec{}, this approach does not suffer from leaking positional information from the masked-out point patches to the student.
Moreover, utilizing an MAE-inspired setup provides additional benefits:
First, the student is more computationally efficient, as it only needs to process the non-masked embeddings.
Second, the model's inputs during fine-tuning are more similar to those during pre-training because the inputs during pre-training are no longer dominated by masked embeddings which are absent during fine-tuning.
This likely makes the learned representations more transferable to downstream tasks.

\section{Results}\label{sec:Background}

And, so, since 2016, researchers have been probing the submitted methods, and in 2022 NIST published the final 10: ASCON, Elephant, GIFT-COFB, Grain128-AEAD, ISAP, Photon-Beetle, Romulus, Sparkle, TinyJambu, and Xoodyak. A particular focus is on the security of the methods, along with their performance on low-cost FPGAs/embedded processes and their robustness against side-channel attacks.

The current set of benchmarks includes running on an Arduino Uno R3 (AVR ARmega 328P), Arduino Nano Every (AVR ARmega 4809), Arduino MKR Zero (ARM Cortex M10+) and Arduino Nano 33 BLE (ARM Cortex M4F). These are just 8-bit processors and fit into an Arduino board. Along with their processing limitations, they are also limited in their memory footprint (to run code and also to store it). The lightweight cryptography method must thus overcome these limitations, and still, be secure and provide a good performance level. Running AES in block modes on these devices is often not possible, as there is not enough resources. Overall we use a benchmark for encryption — with AEAD (Authenticated Encryption with Additional Data) and for hashing. With AEAD we add extra information — such as the session ID — into the encryption process. This type of method can bind the encryption to a specific stream.



\subsection{ARM Cortex M3}

In Table \ref{tab:table01} [1], we see a sample run using an Arduino Due with an ARM Cortex M3 running at 84MHz. The tests are taken in comparison with the ChaCha20 stream cipher and defined for AEAD, and where the higher the value the better the performance. We can see that Sparkle, Xoodyak and ASCON are the fastest of all. Sparkle has a 100\% improvement, and Xoodyak gives a 60\% increase in speed over ChaCha20. Elephant, ISAP and PHOTON-Beetle have the worst performance for encryption (with around 1/20th of the speed of ChaCha20).

\begin{table*}
\caption{\label{tab:table01} Arduino Due with an ARM Cortex M3 running at 84MHz for encryption against ChaCha20 \cite{light01}}
\centering
\begin{tabular}{|l|l|l|l|l|l|l|l|l|}
\hline
Algorithm&Key Bits&Nonce Bits&Tag Bits&Encrypt 128~B&Decrypt 128~B&Encrypt 16~B &Decrypt 16~B&Aver
\\ \hline \hline
Schwaemm128-128 (SPARKLE)	&128	&128&	128	&1.6	&1.58	&2.84	&2.39	&2.01\\
Xoodyak 	&128	&128	&128	&1.66	&1.51	&1.73	&1.6	&1.62\\
ASCON-128	&128	&128&	128	&1.54	&1.44	&1.78	&1.68	&1.61\\
TinyJAMBU-128 	&128	&96	&64	&0.93	&0.95	&1.63	&1.61	&1.21\\
GIFT-COFB	&128	&128	&128	&1.01	&1.01	&1.16	&1.15	&1.08\\
Grain-128AEAD	&128	&96	&64	&0.26	&0.26	&0.56	&0.56	&0.37\\
Romulus-M1	&128	&128	&128	&0.1	&0.11	&0.15	&0.16	&0.13\\
PHOTON-Beetle-AEAD-ENC-128	&128	&128	&128	&0.06	&0.07	&0.11	&0.12	&0.08\\
ISAP-A-128	&128	&128	&128	&0.08	&0.08	&0.03	&0.04	&0.05\\
Delirium (Elephant)	&128	&96	&128	&0.04	&0.05	&0.06	&0.07	&0.05\\
\hline
\end{tabular}
\end{table*}

Not all of the finalists can do hash functions. Table \ref{tab:table02} outlines these.

\begin{table*}
\caption{\label{tab:table02} Arduino Due with an ARM Cortex M3 running at 84MHz for hashing against BLAKE2s \cite{NISTgov}}
\centering
\begin{tabular}{|l|l|l|l|l|l|}
\hline
Algorithm	& Hash Bits	& 1024 bytes	& 128 bytes	& 16 bytes	& Average\\
\hline\hline
Esch256 (SPARKLE) 	&256	&0.89	&0.78	&1.5	&1.06\\
Xoodyak 	&256	&0.71	&0.65	&1.43	&0.93\\
GIMLI-24-HASH	&256	&0.54	&0.47	&0.86	&0.62\\
ASCON-HASH 	&256	&0.51	&0.41	&0.63	&0.52\\
PHOTON-Beetle-HASH	&256	&0.01	&0.01	&0.05	&0.02\\
\hline
\end{tabular}
\end{table*}


Again, we see Sparkle and Xoodyak in the lead, with Sparkle actually faster in the test than BLAKE2s, and Xoodyak just a little bit slower. ASCON has a weaker performance, and PHOTON-Beetle is relatively slow. For all the tests, the ranking for authenticated encryption is (and where the higher the rank, the better):

14 SPARKLE
12 Xoodyak
12 ASCON
10 TinyJAMBU
9 GIFT-COFB, Gimli
4 Grain-128AEAD,KNOT
0 Elephant, ISAP, PHOTON-Beetle

and for hashing SPARKLE and Xoodyak are ranked the same:

7 SPARKLE, Xoodyak 5 Gimli 3 ASCON 0 PHOTON-Beetle

\subsection{Uno Nano performance}

For AEAD on Uno Nano Every [2], the benchmark is against AES GCM. We can see in \ref{tab:table03} , that SPARKLE is 4.7 times faster than AES GCM for 128-bit data sizes, and Xoodyak comes in second with a 3.3 times improvement over AES GCM. When it comes to 8-bit data sizes TinyJambu actually is the fastest, but where Sparkle and Xoodyak still perform well. PHOTON-Beetle, Grain128 and ISAP do not do well, and only slightly improve on AES GCM. In fact, Grain128 and ISAP are actually slower than AES GCM.




\begin{table*}
\caption{\label{tab:table03} Uno Nano for AEAD against AES GCM and showing cycles (showing fastest of the method)}
\centering
\begin{tabular}{|l|l|l|l|l|l|l|l|l|l|l|}
\hline
Algorithm&Impl.&Primary&Flag&Size&Enc(0:8)&Dec(0:8)&Enc(128:129)&Dec(128:128)&Bench.(128)&Bench.(8)
\\ \hline \hline
sparkle       &rhys	          &yes&	   O3	&12290	&1276	&1316	&4648    &5072  &4.7  &3.3\\
Xoodyak       &XKCP-AVR8	  &yes&    O3	&4560	&2596	&2608	&7184    &7128  &3.3  &1.6\\
knot	      &$avr8_speed$   &no&	   Os	&1664	&2124	&2140	&8144    &8160  &2.9  &2\\
ascon 	      &rhys	          &no&     O3	&5180	&1240	&1284	&8056    &8488  &2.8  &3.3\\
GIFT-COFB     &rhys	          &yes&    O1	&23312	&1852	&1892	&8220    &8776  &2.7  &2.2\\
saeaes	      &ref	          &no&     O3	&17062	&1208	&1212	&8992    &9004  &2.6  &3.4\\
hyena	      &rhys           &yes&    O3	&293860	&1912	&1964	&8960    &9396  &2.5  &2.2\\
elephant      &rhys           &no&     O3	&13106	&1924	&1948	&9260    &9796  &2.4  &2.2\\
estate	      &ref            &yes&    O3	&9434 	&1424	&1448	&10276   &10292 &2.3  &2.9\\
romulus	      &rhys           &no&     O3	&19346 	&1632	&1676	&10152   &10568 &2.2  &2.5\\
spook	      &rhys           &no&     O3	&12942 	&2984	&2968	&10272   &10708 &2.2  &1.4\\
tinyjambu     &rhys           &yes&    O3	&9174 	&1232	&1288	&10364   &10888 &2.2  &3.4\\
subterranean  &rhys           &yes&    Os	&6042 	&3372	&3460	&10288   &10944 &2.2  &1.2\\
orange        &rhys           &yes&    O3	&12140 	&2500	&2536	&11200   &11620 &2    &1.7\\
gimli         &rhys           &yes&    O3	&21272 	&1920	&1956	&11944   &12360 &1.9  &2.2\\
skinny        &rhys           &no&     O1	&12452 	&1604	&1644	&12960   &14372 &1.7  &2.6\\
photon-beetle & $avr8_speed$  &yes&    Os	&3536 	&2444	&2472	&20076   &20092 &1.2  &1.7\\
{\bf reference}&rhys          &yes&    O2	&7874 	&4152	&4156	&23812   &23764 &1    &1\\
grain128aead  &rhys           &yes&    O2	&9532 	&3992	&3980	&30396   &30124 &0.8  &1\\
isap          &rhys           &no&     O2	&3824 	&20212	&20256	&42936   &43372 &0.5  &0.2\\
\hline
\end{tabular}
\end{table*}

And so for AEAD  (performance) the ordering is

1. Sparkle
2. Xoodyak
3. Ascon
4. GIFT-COFB.
5. Elephant.
6. Romulus.
7. Tiny Jambu.
8. PHOTON-Beetle.
9. Grain128
10. ISAP.

For hashing on an Uno Nano Every, Table \ref{tab:table04} shows a similar performance level as to the ARM Cortex M3 assessment. In this case, the benchmark hash is SHA-256, and we can see that it takes Sparkle twice as many cycles for a 128-bit hash, and 2.9 times for Xoodyak. PHOTON-Beetle is way behind with a 128-bit hash and which is 17.4 times slower than SHA-256. That said, though, PHOTON-Beetle could be more focused on reducing power consumption rather than speed. GIMLI and SKINNY are included to show a comparison with well-designed methods in lightweight hashing. It can be seen that every method beats SKINNY, but only SPARKLE and Xoodyak beat GIMLI.


\begin{table*}
\caption{\label{tab:table04}  Uno Nano for hashing against SHA-256 and showing cycles (showing fastest of the method for hashing)}
\centering
\begin{tabular}{|l|l|l|l|l|l|l|l|l|l|l|}
\hline
Algorithm&Impl.&Primary&Flag&Size&h(8)&h(16)&h(32)&h(64)&h(128)&Benchmark
\\ \hline \hline
{\bf reference}&$nacl_ref$    &yes&    O3	&18774 	&768	&768	&772     &1364  &1968  &1\\
sparkle       &rhys	          &yes&	   O1	&7912	&1036	&1036	&1468    &2272  &3884  &2\\
Xoodyak       &XKCP-AVR8	  &yes&    O3	&2604	&1284	&1288	&1924    &3192  &5732  &2.9\\
gimli         &rhys           &yes&    O3	&19554 	&1284	&1920	&2544    &3804  &6312  &3.2\\
ascon 	      &rhys	          &yes&    O3	&2178	&2972	&3552	&4736    &7088  &11784 &6\\
drygascon     &rhys           &no&	   O3	&15500	&4604	&4600	&6540    &10360 &17912 &9.1\\
photon-beetle & $avr8_speed$  &yes&    O3	&2948 	&2372	&2364	&6940    &16084 &34172 &17.4\\
skinny        &rhys           &yes&    O2	&9784 	&7048	&10556	&13976   &20952 &34896 &17.7\\
\hline
\end{tabular}
\end{table*}


And so for hashing (performance) the ordering is:
\begin{enumerate}
    \item Sparkle.
    \item Xoodyak.
    \item Ascon
    \item PHOTON-Beetle. 
\end{enumerate}
    
\section{Conclusion, Limitations, and Future Work}
\label{sec:future}
We presented \ours, a NeRF editing method conditioned on text and sketch. Using novel loss functions, our framework allows for local editing of neural fields.
\begin{wrapfigure}{r}{0.2\textwidth} 
\vspace{-10pt}
  \begin{center}
    \includegraphics[width=0.2\textwidth]{figs/failures_Ali.jpg}
  \end{center}
    \vspace{-15pt}
 \vspace{1pt}
\end{wrapfigure} 
Similar to previous works \cite{poole2022dreamfusion, lin2022magic3d, metzer2022latent}, our approach utilizes the SDS Loss and may be vulnerable to the well-known "multiface issue" (inset figure) depending on the choice of diffusion model and prompt. Our method supports a single set of prompt and sketch views at a time. A simple workaround is to apply our method multiple times progressively (Fig.~\ref{fig:progressive}). 
Our results rely on the publicly available Stable-Diffusion model \cite{rombach2021highresolution}, which is less amenable to directional text prompts and produces lower quality 3D generated outputs compared to commercial diffusion models used by previous works~\cite{poole2022dreamfusion, lin2022magic3d}. In Fig~\ref{fig:diff_diff} we show that it is possible to get better results by using the Deepfloyd-IF model \cite{deepfloyd}.


Future directions may expand our method to better support for non-opaque materials, or condition on other modalities, possibly through the diffusion model. More research may further extend the usage of sketch scribbles for animation, similar to \cite{dvoro2020monstermash}. 



% \orrc{In addition, the interface of our method may further close the gap with non data-driven methods, through allowing inflated single sketch views or other primitive based sketch interfaces. Mention also we didn't explore half-transparent objects enough

% \orr{
% Limitations: 1. Janus effect / multiface problem (cat with santa hat), 2. sketching multiple disjoint regions at once. 3. mention that quality presented in this work depend on the diffusion model used? (we can't compete with the larger IMAGINE / e-diffi).

% Notes: remember thanking people: Andrey for SGMT code and mention mesh sources. (the cat, the plate, the horse)
% }

{\small
\bibliographystyle{ieee_fullname}
\bibliography{egbib}
}

\appendix

\renewcommand\thefigure{\arabic{figure}}
\setcounter{figure}{10}
\renewcommand\thetable{\arabic{table}}
\setcounter{table}{3}

% \textbf{Acknowledgements:} We thank Rinon Gal, Masha Shugrina, Roy Bar-On and Janick Martinez Esturo for helpful discussions and proofreading our paper. We also thank Andrey Voynov for providing access to their sketch-based diffusion work.

\section{Background}

In the following, we include an extended background chapter cut off from the main paper for brevity.

\subsection{Latent diffusion models (LDMs)}
LDMs 
\cite{rombach2021highresolution}
are a class of diffusion models that operate on a latent space instead of directly sampling high-resolution color images.
These models have two main components: a variational autoencoder consisting of an encoder $\mathcal{E}(x)$ and a decoder $\mathcal{D}(z)$, pretrained on the training data, and a denoising diffusion probabilistic model (DDPM) trained on the latent space of the autoencoder. Specifically, let $Z$ be the latent space learned by the autoencoder.\
% They usually consist of an encoder $\mathcal{E}(x)$ and a decoder $\mathcal{D}(z)$, pretrained on the training data, and a denoising diffusion probabilistic model (DDPM) trained on the latent space of the autoencoder.
The objective of the DDPM is to minimize the following expectation:
\begin{equation}
   \mathbb{E}_{z_0 \sim Z, \epsilon \sim \mathcal{N}(0, I), t}[||\epsilon_\phi(z_t, t) - \epsilon || ^ 2],
\end{equation}
where $t$ is the time-step of the diffusion process, $z_t~=~ \sqrt{\alpha_t}z_0 + \sqrt{1 - \alpha_t}\epsilon$ is the input latent image with noise added to it, and $\epsilon_\phi$ is the denoising model, often constructed as a U-Net~\cite{ho2020ddpm}.
Once trained, it is possible to sample from the latent space $Z$ by starting from a random standard Gaussian noise and running the backward diffusion process as described by Ho et al.~\cite{ho2020ddpm}. 
The sampled latent image then can be fed to $D(z)$ to get a final high-resolution image.

\subsection{Score distillation sampling (SDS)}
\label{subsec:prelim}
%\vspace{-10pt}
%\paragraph{Latent Diffusion Models (LDMs):} LDMs \cite{rombach2021highresolution} are a class of diffusion models that operate on a latent space.
%instead of directly sampling high-resolution images.
%These models have two main components: a variational autoencoder consisting of an encoder $\mathcal{E}(x)$ and a decoder $\mathcal{D}(z)$, pretrained on the training data, and a denoising diffusion probabilistic model (DDPM) trained on the latent space of the autoencoder. Specifically, let $Z$ be the latent space learned by the autoencoder.\
%\am{They usually consist of an encoder $\mathcal{E}(x)$ and a decoder $\mathcal{D}(z)$, pretrained on the training data, and a denoising diffusion probabilistic model (DDPM) trained on the latent space of the autoencoder.}
%The objective of the DDPM is to minimize the following expectation:
%\begin{equation}
%    \mathbb{E}_{z_0 \sim Z, \epsilon \sim \mathcal{N}(0, I), t}[||\epsilon_\phi(z_t, t) - \epsilon || ^ 2],
%\end{equation}
%where $t$ is the time-step of the diffusion process, $z_t~=~ \sqrt{\alpha_t}z_0 + \sqrt{1 - \alpha_t}\epsilon$ is the input latent image %with noise added to it, and $\epsilon_\phi$ is the denoising model, often constructed as a U-Net~\cite{ho2020ddpm}. %Once trained, it is possible to sample from the latent space $Z$ by starting from a random standard Gaussian noise and running the backward diffusion process as described by Ho et al.~\cite{ho2020ddpm}. The sampled latent image then can be fed to $D(z)$ to get a final high-resolution image.

%\vspace{-15pt}

First introduced by DreamFusion~\cite{poole2022dreamfusion}, SDS is a method of generating gradients from a pretrained diffusion model, by using its \textit{Score Function} to push the outputs of a parameterized image model towards the mode of the diffusion model distribution. More formally, let $I_\theta$ be an image model with parameters $\theta$. In the case of our application, $I_\theta$ is a neural renderer such as NeRF ~\cite{mildenhall2020nerf} or Instant-NGP ~\cite{mueller2022instant}. We can use a pretrained diffusion model with denoiser $\epsilon_\phi(z_t, t)$, to optimize the following:
\begin{equation}
    \min_{\theta}\mathbb{E}_{\epsilon \sim \mathcal{N}(0, I), t}[||\epsilon_\phi(\sqrt{\alpha_t}I_\theta + \sqrt{1 - \alpha_t}\epsilon, t) - \epsilon|| ^ 2],
\end{equation}
where $t$ is the time-step of the diffusion process, and $\alpha_t$ is a constant scheduling the diffusion forward and backward processes. The Jacobian of the denoiser can be omitted in the gradient of the above expression, to get:
\begin{equation}
    \mathbb{E}_{\epsilon \sim \mathcal{N}(0, I), t}[(\epsilon_\phi(\sqrt{\alpha_t}I_\theta + \sqrt{1 - \alpha_t}\epsilon, t) - \epsilon)\frac{\partial I_\theta}{\partial \theta}].
\end{equation}
%We can push the outputs of the image model to the mode of the distribution of a pretrained %diffusion model with denoiser $\epsilon_\phi(x_t, t)$ by solving:
%Additionally, assume that we have a pretrained text-to-image LDM with denoiser $\epsilon_\phi(z_t, t)$. Then we can sample from the distribution learned by the diffusion model by solving the following optimization problem:
The advantage of SDS is that one can apply constraints directly on the image model making this framework suitable for our application of sketch-guided 3D generation.
%To get a more stable optimization and also avoid the time-consuming backpropagation through the diffusion U-Net, the jacobian term that appears when differentiating the above expression can be omitted. Therefore one can compute the gradient of the above expression as:

%To get a faster, more stable optimization process, the gradient of the above expression is simplified by omitting the diffusion U-Net Jacobian and can be written as:

% \begin{equation}
%     \mathbb{E}_{\epsilon \sim \mathcal{N}(0, I), t}[(\epsilon_\phi(\sqrt{\alpha_t}I_\theta + \sqrt{1 - \alpha_t}\epsilon, t) - \epsilon)\frac{\partial I_\theta}{\partial \theta}].
% \end{equation}
%Applying this gradient to the image model parameters for multiple steps would yield an image model which adheres to the learned probability distribution of the diffusion model. 
%The advantage of SDS is that one could apply constraints directly on the image model making this %framework suitable for our application of sketch-guided 3D generation.
%instead of applying them to the noisy outputs of the diffusion backward process, 

\section{Additional Evaluations}

\subsection{Quantitative Comparisons}
\begin{table*}
\centering
% \setlength{\tabcolsep}{4pt}

\captionsetup{font=small}
\caption{
Fidelity of base field. We measure the \textbf{Structural Similarity (SSIM $\uparrow$)} of the method's output against renderings from the base model. \ours  \enspace \textit{(no-preserve)} refers to a variant of our method which doesn't apply $\mathcal{L}_{pres}$. Text-Only refers to a public re-implementation of Latent-NeRF~\cite{metzer2022latent}. Latent-NeRF uses the setting from Section~\ref{sec:supp_qualitative}.
}
\label{tab:ssim}

% \footnotesize
\begin{tabular}{l \tsm \tsm c \ts c \tsm c \ts c \tsm c \ts c \tsm c \ts c \tsm c \ts c \tsm c}
\toprule

{Method} & 
\multicolumn{2}{c \tsm  \ts}{{Cat}} &
\multicolumn{2}{c \tsm}{{Cupcake}} &
\multicolumn{2}{c \tsm}{{Horse}} &
\multicolumn{2}{c \tsm}{{Sundae}} &
\multicolumn{2}{c \tsm}{{Plant}} &
{{Mean}} \\

& 
\multicolumn{2}{c \tsm  \ts}{{\textit{+chef hat}}} &
\multicolumn{2}{c \tsm}{{\textit{+candle}}} &
\multicolumn{2}{c \tsm}{{\textit{+horn}}} &
\multicolumn{2}{c \tsm}{{\textit{+cherry}}} &
\multicolumn{2}{c \tsm}{{\textit{+flower}}} &
\\


& A & B
& A & B
& A & B
& A & B
& A & B
&
\\
\midrule

\ours  & 
\textbf{0.978} & \textbf{0.990} & 
\textbf{0.964} & \textbf{0.973} & 
\textbf{0.990} & \textbf{0.986} & 
\textbf{0.963} & \textbf{0.962} & 
0.927 & \textbf{0.938} & 
\textbf{0.967}
\\

\ours \enspace \textit{(no-preserve)}  & 
0.867 & 0.890 & 
0.944 & 0.948 & 
0.950 & 0.934 & 
0.913 & 0.921 & 
0.803 & 0.801 & 
0.897
\\


Text-Only~\cite{stable-dreamfusion}  & 
0.875 & 0.918 & 
0.937 & 0.943 & 
0.933 & 0.908 & 
0.947 & 0.951 & 
0.891 & 0.883 & 
0.919
\\

Latent-NeRF~\cite{metzer2022latent} & 
0.915 & 0.948 & 
0.950 & 0.956 & 
0.947 & 0.927 & 
0.904 & 0.906 & 
\textbf{0.930} & 0.925 & 
0.930
\\

\bottomrule

\end{tabular}
\end{table*}

% ? & ? & 
% ? & ? & 
% ? & ? & 
% ? & ? & 
% ? & ? & 
% ?
\begin{table*}
\centering
% \setlength{\tabcolsep}{4pt}

\captionsetup{font=small}
\caption{
Fidelity of base field. We measure the \textbf{Perceptual Image Patch Similarity (LPIPS $\downarrow$)} of the method's output against renderings from the base model. We use VGG \cite{simon2015vgg} as the learned perceptual encoder. \ours  \enspace \textit{(no-preserv)} refers to a variant of our method which doesn't apply $\mathcal{L}_{pres}$. Text-Only refers to a public re-implementation of DreamFusion~\cite{poole2022dreamfusion}. Latent-NeRF uses the setting from Section~\ref{sec:supp_qualitative}.
}
\label{tab:lpips}

% \footnotesize
\begin{tabular}{l \tsm \tsm c \ts c \tsm c \ts c \tsm c \ts c \tsm c \ts c \tsm c \ts c \tsm c}
\toprule

{Method} & 
\multicolumn{2}{c \tsm  \ts}{{Cat}} &
\multicolumn{2}{c \tsm}{{Cupcake}} &
\multicolumn{2}{c \tsm}{{Horse}} &
\multicolumn{2}{c \tsm}{{Sundae}} &
\multicolumn{2}{c \tsm}{{Plant}} &
{{Mean}} \\

& 
\multicolumn{2}{c \tsm  \ts}{{\textit{+chef hat}}} &
\multicolumn{2}{c \tsm}{{\textit{+candle}}} &
\multicolumn{2}{c \tsm}{{\textit{+horn}}} &
\multicolumn{2}{c \tsm}{{\textit{+cherry}}} &
\multicolumn{2}{c \tsm}{{\textit{+flower}}} &
\\


& A & B
& A & B
& A & B
& A & B
& A & B
&
\\
\midrule

\ours  & 
\textbf{0.070} & \textbf{0.069} & 
0.069 & \textbf{0.061} & 
\textbf{0.028} & \textbf{0.032} & 
0.086 & 0.094 & 
0.158 & 0.128 & 
\textbf{0.079}
\\

\ours \enspace \textit{(no-preserv)}  & 
0.290 & 0.250 & 
0.091 & 0.093 & 
0.089 & 0.098 & 
0.169 & 0.154 & 
0.291 & 0.309 & 
0.183
\\


Text-Only~\cite{stable-dreamfusion}  & 
0.150 & 0.137 & 
0.076 & 0.076 & 
0.115 & 0.134 & 
\textbf{0.081} & \textbf{0.079} & 
0.170 & 0.180 & 
0.120
\\


Latent-NeRF~\cite{metzer2022latent}  & 
0.102 & 0.101 & 
\textbf{0.066} & 0.065 & 
0.081 & 0.100 & 
0.139 & 0.141 & 
\textbf{0.108} & \textbf{0.113} & 
0.101
\\


\bottomrule

\end{tabular}
\end{table*}

 \begin{table*}
\vspace{10mm}
\centering
\captionsetup{font=small}
\caption{
Fidelity of base field. Following the experiments in section~\ref{subsec:quantitative}, we measure the PSNR of the base objects on additional examples provided in Fig.~\ref{fig:additional} and Fig.~\ref{fig:diff_diff}.
}
\label{tab:psnr_supp}

\begin{tabular}{@{}llrlrlrlrlr@{}}
\toprule
& \multicolumn{2}{c}{Tree to Cactus} & \multicolumn{2}{c}{Anime+Skirt} & \multicolumn{2}{c}{Pancake+Cream }  & \multicolumn{2}{c}{Gift on Table} & \multicolumn{2}{c}{Mean}  \\ 
\cmidrule(l){2-3} \cmidrule(l){4-5} \cmidrule(l){6-7} \cmidrule(l){8-9} \cmidrule(l){10-10}
Method & view 1 & view 2 & view 1 & view 2 & view 1 & view 2 & view 1 & view 2 &    \\ \midrule
SKED & \textbf{29.15} & \textbf{27.47} & \textbf{39.67} & \textbf{37.40} & \textbf{27.48} & \textbf{26.64} & \textbf{34.16} & \textbf{31.52}  & \textbf{31.68}\\
 Text-Only & 23.12 & 24.40 & 22.61 & 21.95 & 16.97 & 15.35 & 19.05 & 20.70 & 20.51 \\
 %Latent-Nerf & 15.37 & 16.28 & 21.97 & 21.02 & 15.70 & 14.98 & 18.30 & 18.70 & 17.97\\

\bottomrule
\end{tabular}

% \caption{Ablation of input view number.}
\label{tab:reb:ablation}
\vspace{20mm}
\end{table*}

\textbf{Base Model Fidelity.} In Table~\ref{tab:ssim}, We include the SSIM metric to further quantify our method's capability to preserve the base model.

\subsection{Qualitative Comparisons}
\label{sec:supp_qualitative}

%\orrc{TODO}
%Optional:
%* SSIM / LPIPS in addition to PSNR
% * CLIP-R quantitative comparison in addition to CLIP-sim

% \subsection{Comparisons to Latent-NeRF} %~\cite{metzer2022latent}} 
\begin{figure*}
    \centering
    \includegraphics[width=0.8\linewidth]{figs/comparison.jpg}
    \captionsetup{font=small}
    \caption{Examples from the modified version of the sketch shape pipeline of Latent-NeRF~\cite{metzer2022latent}}
    \label{fig:comparison}
\end{figure*}
%\begin{table}
%\centering
% \setlength{\tabcolsep}{4pt}

%\captionsetup{font=small}
%\caption{
%Comparison to 
%Latent-NeRF~\cite{metzer2022latent}.
%To assess our method's ability to preserve the original content compared to Latent-NeRF
%~\cite{metzer2022latent}
%we compare the average PSNR computed on outputs of both methods on the same five examples tested for the previous metrics.
%~\ref{tab:psnr}. 
%In addition, we include the average runtime of each method over the aforementioned examples (units shown in minutes).
%}
%\label{tab:compare}

% \footnotesize
%\begin{tabular}{l \tsm \tsm c \ts c \tsm c \ts c \tsm c \ts c \tsm c \ts c \tsm c \ts c \tsm c}
%\toprule

%{Method} & 

%{{Mean PSNR}} & & {{Mean Runtime }}\\

%\midrule

%\ours  & 
%\textbf{27.53} & & \textbf{38} & 

%\\

%Latent-NeRF~\cite{metzer2022latent} & 
%18.59 & & 64 & 

%\\

%\bottomrule

%\end{tabular}
%\end{table}
\begin{table*}
\centering
\setlength{\tabcolsep}{4pt}

\captionsetup{font=small}
\caption{
To compare our method's ability to preserve the base with the baseline derived from Latent-NeRF~\cite{metzer2022latent}, we measure the PSNR of both method's outputs against renderings from the base model. Additionally, we report the average runtime of our method compared to the baseline.
}
\label{tab:compare}

% \footnotesize
\begin{tabular}{l \tsm \tsm c \ts c \tsm c \ts c \tsm c \ts c \tsm c \ts c \tsm c \ts c \tsm c \tsm c}
\toprule

{Method} & 
\multicolumn{2}{c \tsm  \ts}{{Cat}} &
\multicolumn{2}{c \tsm}{{Cupcake}} &
\multicolumn{2}{c \tsm}{{Horse}} &
\multicolumn{2}{c \tsm}{{Sundae}} &
\multicolumn{2}{c \tsm}{{Plant}} &
{{PSNR}} &
{{Runtime (minutes)}}\\

& 
\multicolumn{2}{c \tsm  \ts}{{\textit{+chef hat}}} &
\multicolumn{2}{c \tsm}{{\textit{+candle}}} &
\multicolumn{2}{c \tsm}{{\textit{+horn}}} &
\multicolumn{2}{c \tsm}{{\textit{+cherry}}} &
\multicolumn{2}{c \tsm}{{\textit{+flower}}} &
Mean &
Mean
\\


& A & B
& A & B
& A & B
& A & B
& A & B
&
\\
\midrule

\ours  & 
\textbf{31.05} & \textbf{34.13} & 
\textbf{23.73} & \textbf{25.98} & 
\textbf{32.45} & \textbf{31.46} & 
\textbf{26.47} & \textbf{25.99} & 
\textbf{21.71} & \textbf{22.31} & 
\textbf{27.53} &
\textbf{38}
\\

Latent-NeRF~\cite{metzer2022latent}  & 
21.15 & 22.62& 
21.99 & 21.20 & 
17.00 & 15.97 & 
16.07 & 15.47 & 
17.66 & 16.78 & 
18.59 &
64
\\


\bottomrule

\end{tabular}
\end{table*}


\textbf{Comparison to Latent-NeRF~\cite{metzer2022latent}.} To the best of our knowledge, we are the first work to employ 2D sketch-based editing of NeRFs. Given that prior works are not directly comparable with our editing setting, we attempt to create a close comparison instead, faithful to the original compared method and fair to evaluate our editing setting. As baseline, we use the method from Latent-NeRF's
~\cite{metzer2022latent} 
3D sketch shape pipeline. We initialize a NeRF with the base object weights, and create a \textit{3D sketch shape}, a mesh, by intersecting the bounding boxes of our 2D sketches in the 3D space. Note that we could also intersect the sketch masks, however, due to view inconsistencies, we found that the results are far inferior. After initializing the NeRF and creating the sketch shape, we proceed to use the sketch shape loss from the paper to preserve the geometry, while editing the NeRF according to the input text. In Fig.~\ref{fig:comparison}, we establish that while this baseline is able to perform meaningful edits, it suffers from two apparent issues: (i) the baseline severely changes the base NeRF, and (ii) the edited region is bound to the coarse geometry of the intersected bounding boxes. To alleviate the latter, one could resort to modeling 3D assets as a sketch shape. However, we show that by using simple multiview sketches, it is possible to perform local editing without going through the effort of modeling accurate 3D masks. Finally, we include a quantitative summary of the preservation ability and the performance of the two methods in Table~\ref{tab:compare}.

%\textbf{Ablation Study} In table \orrc{TODO}, we demonstrate some of the assets %used to conduct the ablation study in Section 4.3 of the main paper.


%https://zju3dv.github.io/neumesh/ - cannot generate new content
%http://editnerf.csail.mit.edu/ - limited by 3d train data
%https://cassiepython.github.io/nerfart/ - discuss limits of stylization
%ClipMesh - cannot generate examples with genus > 0
%Original Dreamfusion - compare to the progressive squirrel example from their paper (show they affect the base)

%\subsection{Intersection over Sketch}
%\orrc{TODO}
%* Discuss if we want to include the low iou results also
%\subsection{Generation from scratch}
%\orrc{TODO}
% (currently failed maybe needs pretraining to zero density)

\begin{figure*}[t]
   \centering
    \begin{overpic}[width=0.8\linewidth,tics=10, trim=0 0 0 0,clip]{figs/ui.jpg}
   \end{overpic}
   \captionsetup{font=small}
    \caption{The interactive UI allows users to sketch over a pretrained NeRF. \textbf{Top row}: The user draws scribbles from two different views using "Sketch Mode". \textbf{Bottom left}: After pressing "Add sketch", the scribbles are filled to generate masks, ready to be used with our pipeline. \textbf{Bottom right}: The bounding box marks the sketches intersection region, where the edit takes place.}
    \label{fig:ui}
 \end{figure*}

 \begin{figure*}[h]
    \centering
    \hspace{0.25in}
    \begin{overpic}[width=0.8\linewidth,tics=10, trim=0 0 0 0,clip]{figs/geometry_nc.jpg}
    % \put(-20, 12){\large{Depth map}}
    % \put(-20, 40){\large{Normal map}}
    % \put(-15, 65){\large{RGB}}
    \end{overpic}
    %\includegraphics[width=0.8\linewidth]{figs/geometry.jpg}
    \captionsetup{font=small}
    \caption{From top to bottom: color, normal and depth maps of outputs generated by our method.}
    \label{fig:geometry}
\end{figure*}


\section{Implementation Details}%\vspace{-3\baselineskip}
\vspace{-10pt}
This section contains additional implementation details omitted from the manuscript.
\vspace{-35pt}
\subsection{Text Prompts}
\vspace{-10pt}
In the following, we include the full list of prompts that were used to generate the examples within the paper.
\vspace{-40pt}
\begin{itemize}
\setlength{\parskip}{0pt}
  \setlength{\itemsep}{0pt}
    \item "A cat wearing a chef hat"
    \item "A cherry on top of a sundae"
    \item "A red flower stem rising from a potted plant"
    \item "A teddy bear wearing sunglasses"
    \item "A candle on top of a cupcake"
    \item "An anime girl wearing a brown bag"
    \item "An apple on a plate"
    \item "A Nutella jar on a plate"
    \item "A globe on a plate"
    \item "A tennis ball on a plate"
    \item "A cat wearing a red tie"
    \item "A cat wearing red tie wearing a chef hat"
    \item "A 3D model of a unicorn head"
\end{itemize}
\vspace{-40pt}

Additionally, similar to 
DreamFusion\cite{poole2022dreamfusion} 
we use directional prompts, where based on the rendering view, we modify prompt $\textbf{T}$ as follows:
\vspace{-40pt}
\begin{itemize}
\setlength{\parskip}{0pt}
  \setlength{\itemsep}{0pt}
    \item "$\textbf{T}$, overhead view"
    \item "$\textbf{T}$, side view"
    \item "$\textbf{T}$, back view"
    \item "$\textbf{T}$, bottom view"
    \item "$\textbf{T}$, front view"
\end{itemize}
\vspace{-60pt}
%\subsection{Extended Settings}
%* Discuss details of Instant-NGP and preventing gradient flow to original base model (if time permits, add an experiment)

%* Discuss Albedo v.s. Lambertian shading (didn't have it in the framework at the time of conducting experiments)

%\vspace{-3\baselineskip}

\subsection{Interactive UI}
\vspace{-\baselineskip}

Since our method requires user interaction, we include an interactive user interface with our implementation (Fig.~\ref{fig:ui}). The user interface allows users to optimize newly reconstructed base NeRF models, or load pretrained ones. To perform edits, users can position the camera on the desired sketch view, and draw scribbles to guide SKED. By pressing "Add Sketch", scribbles are filled and converted to masked sketch inputs, ready to be used with our method.
\vspace{-30pt}
\subsection{Quality Notes}

Our implementation uses an early version of Stable-DreamFusion
~\cite{stable-dreamfusion}
which does not include the optimizations very recently suggested by
Magic3D~\cite{lin2022magic3d}. 
In contrast to DreamFusion
~\cite{poole2022dreamfusion} 
and Magic3D
~\cite{lin2022magic3d},
which use commercial diffusion models with larger language models 
\cite{imagen2022saharia, balaji2022eDiff-I},
we rely on Stable Diffusion
\cite{rombach2021highresolution}, 
which is less sensitive to directional prompts. Our results are therefore not comparable in visual quality to these previous works.


% In addition, different to DreamFusion, though possible, we do not leverage the Lambertian lighting model in our architecture. The reason behind this is technical: we found this feature produces inferior results with the backbone and implementation we use.





\section{Additional Assets}
%\subsection{Ablation Assets}
%The images Mehdi used to compute IoS
%\orr{Aryan, Mehdi: please complete}

\subsection{Geometry and Depth}
In addition to RGB images, we share examples highlighting the geometry of our method's outputs. In Fig.~\ref{fig:geometry} we include the normal maps and depth maps of two output samples.


% \begin{figure*}
%    \centering
%     \begin{overpic}[width=0.8\linewidth,tics=10, trim=0 0 0 0,clip]{figs/ui.jpg}
%    \end{overpic}
%    \captionsetup{font=small}
%     \caption{The interactive UI allows users to sketch over a pretrained NeRF. \textbf{Top row}: the user draws scribbles from two different views using "Sketch Mode". \textbf{Bottom left}: after pressing "Add sketch", the scribbles are filled to generate masks, ready to be used with our pipeline. \textbf{Bottom right}: the bounding box marks the sketches intersection region, where the edit takes place.}
%     \label{fig:ui}
%  \end{figure*}

%  \begin{figure*}
%     \centering
%     \includegraphics[width=0.8\linewidth]{figs/geometry.jpg}
%     \captionsetup{font=small}
%     \caption{Normal and depth maps generated by our method.}
%     \label{fig:geometry}
% \end{figure*}


\end{document}