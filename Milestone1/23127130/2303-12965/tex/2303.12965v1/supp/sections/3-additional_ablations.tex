\section{Additional Ablations}

\label{supp:additional_ablations}

\noindent \textbf{Number of Training Views}.
\label{supp:additional_ablations:num_views}
Table~\ref{table:abla:view} and Figure~\ref{fig:abla:view} show that giving one camera of view degrades the overall reconstruction quality, and multi-view consistency improves the final results.
The model can aggregate multi-view information for better shading optimization, thus leading to clearer surface materials.

\begin{table}[htbp]
\centering
\caption{\textbf{Ablation results of training views on the ZJU-MoCap 313 subject}.}
\label{table:abla:view}
\begin{tabular}{ccccc}
\toprule
\multicolumn{1}{l|}{}    & \multicolumn{2}{c|}{Training pose}                         & \multicolumn{2}{c}{Novel pose}     \\ \hline
\multicolumn{1}{l|}{ZJU-MoCap 313}              & \multicolumn{1}{c|}{PSNR$\uparrow$}  & \multicolumn{1}{c|}{SSIM$\uparrow$}  & \multicolumn{1}{c|}{PSNR$\uparrow$}  & \multicolumn{1}{c}{SSIM$\uparrow$} \\ \midrule
\multicolumn{1}{l|}{1 view}   & \multicolumn{1}{l|}{24.39} & \multicolumn{1}{l|}{0.913} & \multicolumn{1}{l|}{21.45} & 0.869 \\
\multicolumn{1}{l|}{2 views}   & \multicolumn{1}{l|}{28.06} & \multicolumn{1}{l|}{0.945} & \multicolumn{1}{l|}{22.81} & 0.888 \\
\multicolumn{1}{l|}{3 views}   & \multicolumn{1}{l|}{28.50} & \multicolumn{1}{l|}{0.956} & \multicolumn{1}{l|}{23.17} & 0.894 \\
\multicolumn{1}{l|}{4 views}   & \multicolumn{1}{l|}{\textbf{29.04}} & \multicolumn{1}{l|}{\textbf{0.961}} & \multicolumn{1}{l|}{\textbf{23.20}} & \textbf{0.896} \\
\bottomrule
\end{tabular}
\end{table}

\begin{figure}[tbph]
\centering
\includegraphics[width=\linewidth]{supp/figs/ablation-view.png}
\vspace{-1em}
\caption{\textbf{Ablation study of training views on the ZJU-MoCap 313 subject.}}
\label{fig:abla:view}
\end{figure}






\noindent \textbf{The effect of Skinning Module Design}
\label{supp:additional_ablations:skinning}
Table~\ref{table:ablation:skinning-h36m}-\ref{table:ablation:skinning-zju} and Figure~\ref{fig:abla:h36m}-\ref{fig:abla:mesh:zju} reveal that the initialization with pre-trained skinning net and the regularization on surface skinning improve the overall reconstruction quality.
The initialization provides skinning prior which helps to speed up geometry convergence. 
From Figure~\ref{fig:abla:h36m}-\ref{fig:abla:zju}, the geometry details improve with the initialization under the same training time.

The regularization on surface skinning prevents geometry degradation. 
Figure~\ref{fig:abla:mesh:zju} indicates that
our model can not learn correct canonical geometry without the initialization and the regularization.
The mesh distortion is reduced with the regularization.

\begin{table}[htbp]
\centering
\caption{
\textbf{The ablation on skinning module of H36M S9 dataset}.}
\label{table:ablation:skinning-h36m}
\resizebox{\columnwidth}{!}{%
\begin{tabular}{l|cc|cc}
\toprule
              & \multicolumn{2}{c|}{Training Pose} & \multicolumn{2}{c}{Novel Pose}     \\ \hline
H36M S9       & \multicolumn{1}{c|}{PSNR}  & SSIM  & \multicolumn{1}{c|}{PSNR}  & SSIM  \\ \midrule
w/o skinning init. \& reg.   & \multicolumn{1}{c|}{24.88} & 0.905 & \multicolumn{1}{c|}{21.97} & 0.851 \\
w/ skinning intialization & \multicolumn{1}{c|}{26.28} & 0.926 & \multicolumn{1}{c|}{24.47} & 0.897 \\
w/ skinning regularization       & \multicolumn{1}{c|}{26.24} & 0.925 & \multicolumn{1}{c|}{24.34} & 0.896 \\ \midrule
Full          & \multicolumn{1}{c|}{\textbf{26.29}} & \textbf{0.926} & \multicolumn{1}{c|}{\textbf{24.53}} & \textbf{0.899} \\
\bottomrule
\end{tabular}%
}
\end{table}


\begin{table}[htbp]
\centering
\caption{
\textbf{The ablation on skinning module of ZJU-MoCap 313 dataset}.}
\label{table:ablation:skinning-zju}
\resizebox{\columnwidth}{!}{%
\begin{tabular}{l|cc|cc}
\toprule
              & \multicolumn{2}{c|}{Training Pose} & \multicolumn{2}{c}{Novel Pose}     \\ \hline
ZJU-MoCap 313 & \multicolumn{1}{c|}{PSNR}  & SSIM  & \multicolumn{1}{c|}{PSNR}  & SSIM  \\ \midrule
w/o skinning init. \& reg.   & \multicolumn{1}{c|}{27.46} & 0.949 & \multicolumn{1}{c|}{20.31} & 0.831 \\
w/ skinning intialization & \multicolumn{1}{c|}{28.82} & 0.958 & \multicolumn{1}{c|}{23.08} & 0.893 \\
w/ skinning regularization       & \multicolumn{1}{c|}{28.80} & 0.959 & \multicolumn{1}{c|}{23.14} & 0.895 \\ \midrule
Full          & \multicolumn{1}{c|}{\textbf{29.05}} & \textbf{0.961} & \multicolumn{1}{c|}{\textbf{23.27}} & \textbf{0.897} \\
\bottomrule
\end{tabular}%
}
\end{table}




\begin{figure}[tbph]
\centering
\includegraphics[width=\linewidth]{supp/figs/ablation-skinning-h36m.png}
\vspace{-1em}
\caption{\textbf{Ablation study of the skinning module on the H36M S9 subject.}}
\label{fig:abla:h36m}
\end{figure}

\begin{figure}[tbph]
\centering
\includegraphics[width=\linewidth]{supp/figs/ablation-skinning-zju.png}
\vspace{-1em}
\caption{\textbf{Ablation study of the skinning module on the ZJU-MoCap 313 subject.}}
\label{fig:abla:zju}
\end{figure}

\begin{figure}[tbph]
\centering
\includegraphics[width=\linewidth]{supp/figs/ablation-mesh-zju.png}
\vspace{-1em}
\caption{\textbf{Ablation study of the skinning module on the ZJU-MoCap 313 subject.}}
\label{fig:abla:mesh:zju}
\end{figure}



\noindent \textbf{Effect of SDF Network}
\label{supp:additional_ablations:sdf}
The MLP parametrization of the SDF field keeps our surface both water-tight and smooth, as shown in Figure~\ref{fig:abla:sdf}.

\begin{figure}[tbph]
\centering
\includegraphics[width=\linewidth]{supp/figs/ablation-sdf.png}
\vspace{-1em}
\caption{\textbf{Ablation study of SDF field parametrization.}}
\label{fig:abla:sdf}
\end{figure}