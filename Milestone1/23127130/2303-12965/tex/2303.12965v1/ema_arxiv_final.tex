\documentclass[10pt,twocolumn,letterpaper]{article}

\usepackage{iccv}
\usepackage{times}
\usepackage{epsfig}
\usepackage{graphicx}
\usepackage{amsmath}
\usepackage{amssymb}

% Include other packages here, before hyperref.
\usepackage{graphicx}
\usepackage{amsmath}
\usepackage{amssymb}
\usepackage{booktabs}
\usepackage{bbding}

\usepackage{pifont}
\newcommand{\cmark}{\ding{51}}%
\newcommand{\xmark}{\ding{55}}%
\usepackage{tabularx}
\usepackage{tablefootnote}
% \usepackage{xcolor}
\usepackage[table,xcdraw]{xcolor}

\usepackage{times}
\usepackage{epsfig}
% \usepackage{tabulary}
\usepackage{multirow}
\usepackage{silence}
\WarningFilter{caption}{Unsupported}
\WarningFilter{caption}{The option}
\usepackage{caption}
\usepackage{wrapfig}

% If you comment hyperref and then uncomment it, you should delete
% egpaper.aux before re-running latex.  (Or just hit 'q' on the first latex
% run, let it finish, and you should be clear).
\usepackage[breaklinks=true,bookmarks=false]{hyperref}

\iccvfinalcopy % *** Uncomment this line for the final submission

\def\iccvPaperID{****} % *** Enter the ICCV Paper ID here
\def\httilde{\mbox{\tt\raisebox{-.5ex}{\symbol{126}}}}

% Pages are numbered in submission mode, and unnumbered in camera-ready
\ificcvfinal\pagestyle{empty}\fi

\begin{document}

%%%%%%%%% TITLE
\title{Efficient Meshy Neural Fields for Animatable Human Avatars}

\author{
Xiaoke Huang\textsuperscript{1}
\quad
Yiji Cheng\textsuperscript{1}
\quad 
Yansong Tang\textsuperscript{1}$^*$
\quad 
Xiu Li\textsuperscript{1}
\quad 
Jie Zhou\textsuperscript{2}
\quad 
Jiwen Lu\textsuperscript{2}
\\[1.5mm]
\{\textsuperscript{1}Shenzhen International Graduate School, \textsuperscript{2}Department of Automation\}, Tsinghua University
}

% \maketitle
% Remove page # from the first page of camera-ready.
\ificcvfinal\thispagestyle{empty}\fi


% \iffalse
\twocolumn[{
    \renewcommand\twocolumn[1][]{#1}
    \maketitle
    \begin{center}
        \includegraphics[width=1.0\linewidth]{figs/teaser.fig-1.3.png}
        \vspace{-1.5em}
        \captionof{figure}{\textbf{EMA} efficiently and jointly learns canonical shapes, materials, and motions via differentiable inverse rendering in an end-to-end manner. The method does not require any predefined templates or riggings. The derived avatars are animatable and can be directly applied to the graphics renderer and downstream tasks. All figures are best viewed in color.}
        \label{fig:teaser}
    \end{center}
}]
% \fi

\let\thefootnote\relax\footnotetext{$^*$Corresponding author.}

%%%%%%%%% ABSTRACT
\begin{abstract}
As models continue to grow in size, the development of memory optimization methods (MOMs) has emerged as a solution to address the memory bottleneck encountered when training large models. To comprehensively examine the practical value of various MOMs, we have conducted a thorough analysis of existing literature from a systems perspective. 
% Furthermore, we have evaluated the most widely adopted MOMs employed in mainstream frameworks for both vision and language models.
Our analysis has revealed a notable challenge within the research community: the absence of standardized metrics for effectively evaluating the efficacy of MOMs. The scarcity of informative evaluation metrics hinders the ability of researchers and practitioners to compare and benchmark different approaches reliably. Consequently, drawing definitive conclusions and making informed decisions regarding the selection and application of MOMs becomes a challenging endeavor.
To address the challenge, this paper summarizes the scenarios in which MOMs prove advantageous for model training. We propose the use of distinct evaluation metrics under different scenarios. By employing these metrics, we evaluate the prevailing MOMs and find that their benefits are not universal. We present insights derived from experiments and discuss the circumstances in which they can be advantageous.

\end{abstract}

%%%%%%%%% BODY TEXT
\section{Introduction}


Recent years have witnessed the rise of human digitization~\cite{habermannDeepCapMonocularHuman2020,alexanderCREATINGPHOTOREALDIGITAL,pengNeuralBodyImplicit2021,alldieckDetailedHumanAvatars2018, rajANRArticulatedNeural2020}. This technology greatly impacts the entertainment, education, design, and engineering industry.
There is a well-developed industry solution for this task.
High-fidelity reconstruction of humans can be achieved either with full-body laser scans~\cite{saitoSCANimateWeaklySupervised2021}, dense synchronized multi-view cameras~\cite{xiangModelingClothingSeparate2021a,xiangDressingAvatarsDeep2022a}, or light stages~\cite{alexanderCREATINGPHOTOREALDIGITAL}.
However, these settings are expensive and tedious to deploy and consist of a complex processing pipeline, preventing the technology's democratization.

Another solution is to view the problem as inverse rendering and learn digital humans directly from custom-collected data.
Traditional approaches directly optimize explicit mesh representation~\cite{loperSMPLSkinnedMultiperson2015, fangRMPERegionalMultiperson2018, pavlakosExpressiveBodyCapture2019} which suffers from the problems of smooth geometry and coarse textures~\cite{prokudinSMPLpixNeuralAvatars2020,alldieckVideoBasedReconstruction2018}. Besides, they require professional artists to design human templates, rigging, and unwrapped UV coordinates.
Recently, with the help of volumetric-based implicit representations~\cite{mildenhallNeRFRepresentingScenes2020, parkDeepSDFLearningContinuous2019, meschederOccupancyNetworksLearning2019} and neural rendering~\cite{laineModularPrimitivesHighPerformance2020, liuSoftRasterizerDifferentiable2019, thiesDeferredNeuralRendering2019}, 
one can easily digitize a quality-plausible human avatar from video footage~\cite{jiangNeuManNeuralHuman2022,wengHumanNeRFFreeviewpointRendering}.
Particularly, volumetric-based implicit representations~\cite{mildenhallNeRFRepresentingScenes2020, pengNeuralBodyImplicit2021} can reconstruct scenes or objects with much higher fidelity against previous neural renderer~\cite{thiesDeferredNeuralRendering2019,prokudinSMPLpixNeuralAvatars2020}, and is more user-friendly as it does not need any human templates, pre-set rigging, or UV coordinates.
Captured visual footage and corresponding skeleton tracking are enough for training.
However, better reconstructions and more friendly usability are at the expense of the following factors.
1) \textbf{Inefficiency:}
They require longer optimization times (typically tens of hours or days) and inference slowly.
Volume rendering~\cite{mildenhallNeRFRepresentingScenes2020,lombardiNeuralVolumesLearning2019} formulates images by querying the densities and colors of millions of spatial coordinates. 
In the training stage, due to memory constraints, only a small fraction of points are sampled which leads to slow convergence speed.
2) \textbf{Entangled representations}:
The geometry, materials, and motion dynamics are entangled in the neural networks. 
Due to the implicit nature of neural nets, one can hardly edit one property without touching the others~\cite{yuanNeRFEditingGeometryEditing2022a,liuEditingConditionalRadiance2021}.
3) \textbf{Graphics incompatibility}:
Volume rendering is incompatible with the current popular graphic pipeline,
which renders triangular/quadrilateral meshes efficiently with the rasterization technique.
Many downstream applications require mesh rasterization in their workflow (\eg, editing~\cite{foundationBlenderOrgHome}, simulation~\cite{benderPositionBasedSimulationMethods2015}, real-time rendering~\cite{akenine2019real}, ray-tracing~\cite{waldRTXRayTracing}).
Although there are approaches~\cite{lorensenMarchingCubesHigh,labelleIsosurfaceStuffingFast2007} can convert volumetric fields into meshes, the gaps from discrete sampling degrade the output quality in terms of both meshes and textures.


To address these issues, we present \textbf{EMA}, a method based on \textbf{E}fficient \textbf{M}eshy neural fields to reconstruct animatable human \textbf{A}vatars.
Our method enjoys flexibility from implicit representations and efficiency from explicit meshes, yet still maintains high-fidelity reconstruction quality.
Given video sequences and the corresponding pose tracking, our method digitizes humans in terms of canonical triangular meshes, physically-based rendering (PBR) materials, and skinning weights \textit{w.r.t.} skeletons.
We jointly learn the above components via inverse rendering~\cite{laineModularPrimitivesHighPerformance2020,chenDIBRLearningPredict2021,chenLearningPredict3D2019} in an end-to-end manner.
Each of them is derived from a separate neural field, which relaxes the requirements of a preset human template, rigging, or UV coordinates.
Specifically, we predict a canonical mesh out of a signed distance field (SDF) by differentiable marching tetrahedra~\cite{shenDeepMarchingTetrahedra2021,gaoGET3DGenerativeModel,gaoLearningDeformableTetrahedral2020,munkbergExtractingTriangular3D2022}, then we extend the marching tetrahedra~\cite{shenDeepMarchingTetrahedra2021} for spatial-varying materials by utilizing a neural field to predict PBR materials \textit{on the mesh surfaces} after rasterization~\cite{munkbergExtractingTriangular3D2022,hasselgrenShapeLightMaterial2022,laineModularPrimitivesHighPerformance2020}.
To make the canonical mesh animatable, we take another neural field to model the forward linear blend skinning for the meshes. 
Given a posed skeleton, the canonical mesh is then transformed into the corresponding poses.
Finally, we shade the mesh with a rasterization-based differentiable renderer~\cite{laineModularPrimitivesHighPerformance2020} and train our models with a photo-metric loss.
After training, we export the mesh with materials and discard the neural fields.

\looseness=-1
There are several merits of our method design.
1) \textbf{Efficiency}:
Powered by efficient mesh rendering, our method can render in real-time.
Besides, the training speed is boosted as well, 
since we compute loss holistically on the whole image and the gradients only flow on the mesh surface. In contrast, volume rendering takes limited pixels for loss computation and back-propagates the gradients in the whole space.
Our method only needs about an hour of training and minutes of optimization are enough for plausible avatar reconstruction.
2) \textbf{Disentangled representations}:
Our shape, materials, and motion modules are disentangled naturally by design, which facilitates editing. 
Besides, Canonical meshes with forward skinning modeling handle the out-of-distribution poses better.
3) \textbf{Graphics compatibility}:
Our derived mesh representation is compatible with 
the prominent graphic pipeline, which leads to instant downstream applications (\eg, the shape and materials can be edited directly in design software~\cite{foundationBlenderOrgHome}).
To further improve reconstruction quality, we additionally optimize image-based environment lights and non-rigid motions.


We conduct extensive experiments on standards benchmarks H36M~\cite{ionescuHuman36MLarge2014b} and ZJU-MoCap~\cite{pengNeuralBodyImplicit2021}.
Our method achieves very competitive performance for novel view synthesis, generalizes better for novel poses, 
and significantly improves both training time and inference speed against previous arts.
Our research-oriented code reaches real-time inference speed (100+ FPS for rendering $512\times512$ images).
We in addition showcase applications including novel pose synthesis, material editing, and relighting.
\section{Related Work}
\textbf{Unanswerable Questions. } Unanswerable questions in MRC draw much attention from the research community after the publication of SQuAD 2.0 \cite{rajpurkar-etal-2018-know}. Following the guidelines proposed by \citet{rajpurkar-etal-2018-know}, unanswerable questions in MRC are introduced in MRC of other languages such as French in FQuAD 2.0 \cite{fquad20} and Vietnamese in UIT-ViQuAD 2.0 \cite{viquad20}. The research community commonly refers to unanswerable questions in SQuAD, FQuAD, and UIT-ViQuAD as "artificial unanswerable questions" because annotators are instructed to intentionally create questions that cannot be answered using the information provided in the given context. On the other hand, unanswerable questions that naturally arise are also introduced recently in Natural Questions \cite{kwiatkowski-etal-2019-natural} and TyDi QA \cite{clark-etal-2020-tydi}, in which the evidence documents are provided after the questions are written by annotators.\\
\textbf{Multilingual versus Monolingual Models. } \citet{vulic-etal-2020-probing} probe an empirical analysis on monolingual BERTs and mBERT across six languages and five different lexical tasks. They show that Monolingual BERT encodes significantly more lexical information than mBERT.

Besides, \citet{rust-etal-2021-good} compare pre-trained multilingual language models with monolingual counterparts regarding their monolingual task performances in nine languages and five tasks to reveal the reason for the gap between the performances of monolingual models and multilingual models. This comprehensive analysis later reveals that while pre-training data size played a vital role in the performances of language models on downstream tasks, the monolingual tokenizers designed by native speakers are also an important reason for the high performances of monolingual models in single-language settings. Results from this analysis show that \citet{nguyen-tuan-nguyen-2020-phobert} significantly contributed to the development of Vietnamese language models with a high-quality tokenizer that is suitable for the unique linguistic features of Vietnamese.
\begin{table*}[h]
\centering
\begin{tabular}{@{}llllcc@{}}
\toprule
 &          & EM(\%)    & F1(\%)   & Recall\textsubscript{unanswerable}(\%)  & Recall\textsubscript{answerable}(\%)  \\ \cmidrule(l){2-6} 
\multirow{2}{*}{monolingual}  & WikiBERT & 46.51 & 55.84 & 50.68 & 74.37 \\
 & PhoBERT     & 63.52          & 75.87          & 73.37          & \textbf{89.21} \\ \cmidrule(l){2-6} 
\multirow{3}{*}{multilingual} & mBERT\textsubscript{our}    & 57.66 & 66.84 & 65.84 & 80.47        \\
& mBERT\textsubscript{VLSP}    & 53.55 & 63.03 & - & -        \\
 & XLM-RoBERTa & \textbf{67.84} & \textbf{78.15} & \textbf{75.86} & 88.81 \\ \bottomrule
\end{tabular}
\caption{Performance of models on the UIT-ViQuAD 2.0 Development set}
\label{overall-performane}
\end{table*}

\vspace{-0.3em}
\section{Method}
\vspace{-0.3em}

Our sensitivity-aware visual parameter-efficient fine-tuning consists of two stages. In the first stage, SPT measures the task-specific sensitivity for the pre-trained parameters (Section~\ref{subsec:sensitivity}). Based on the parameter sensitivity and a given parameter budget, SPT then adaptively allocates trainable parameters to task-specific important positions (Section~\ref{subsec:SPT}).

\vspace{-0.3em}
\subsection{Task-specific Parameter Sensitivity}
\label{subsec:sensitivity}
\vspace{-0.3em}

Recent research has observed that pre-trained backbone parameters exhibit varying feature patterns~\cite{raghu2021vision,naseer2021intriguing} and criticality~\cite{zhang2019all,chatterji2019intriguing} at distinct positions. 
Moreover, when transferred to downstream tasks, their efficacy varies depending on how much pre-trained features are reused and how well they adapt to the specific domain gap~\cite{yosinski2014transferable,kumar2022finetuning,neyshabur2020being}. Motivated by these observations, we argue that not all parameters contribute equally to the performance across different tasks in PEFT and propose a new criterion to measure the sensitivity of the parameters in the pre-trained backbone for a given task.

Specifically, given the training dataset $\gD_t$ for the $t$-th task and the pre-trained model weights $\vw=\left\{w_1, w_2, \ldots, w_N\right\}\in \sR^N$ where $N$ is the total number of parameters, the objective for the task is to minimize the empirical risk: $\min_{\vw} E(\gD_t, \vw)$.
We denote the parameter sensitivity \bohan{set} as $\gS=\{s_1, \ldots, s_N\}$ and the sensitivity $s_n$ for parameter $w_n$ is measured by the empirical risk difference when tuning it:
\begin{equation}
\vspace{-0.3em}
    \begin{aligned}
        s_n = E(\gD_t, \vw)-E(\gD_t, \vw\mid w_n=w_n^*),
    \end{aligned}
\label{eq:sensitivity}
\end{equation}
where $w_n^*=\underset{w_n}{\rm argmin}(E(\gD_t, \vw))$. We can reparameterize the tuned parameters as  $w_n^*=w_n+\Delta_{w_n}$, where $\Delta_{w_n}$ denotes the update for $w_n$ after tuning. Here we individually measure the sensitivity of each parameter, which is reasonable given that most of the parameters are frozen during fine-tuning in PEFT. However, it is still computationally intensive to compute Eq.~(\ref{eq:sensitivity}) for two reasons. Firstly, getting the empirical risk for $N$ parameters requires forwarding the entire network $N$ times, which is time-consuming. Secondly, it is challenging to derive $\Delta_{w_n}$, as we have to tune each individual $w_n$ until convergence.

{\begin{algorithm}[t!]
\caption{\label{alg:tps} Computing task-specific parameter sensitivities}
\begin{algorithmic}
    \STATE \textbf{Input:} Pre-trained model with network parameters $\vw$, training set $\gD_t$ for the $t$-th task, and number of training samples $C$ used to calculate the parameter sensitivities
    \STATE \textbf{Output:} Sensitivity set $\gS=\{s_1, \ldots, s_N\}$
    \STATE Initialize $\gS=\{0\}^N$
    \FOR{$i\in\{1,\ldots,C\}$}
        \STATE Get the $i$-th training sample of $\gD_t$
	    \STATE Compute loss $E$
		\STATE Compute gradients $\vg$
		\FOR{$n\in\{1,\ldots,N\}$}
                \STATE Update sensitivity for the $n$-th parameter: $s_{n} = s_{n} + g_n^2$
		    \ENDFOR
    \ENDFOR
\end{algorithmic}
\end{algorithm}}


\begin{figure*}[t]
\begin{center}
    \includegraphics[width=\linewidth]{main_figure.pdf}
\end{center}\vspace{-2em}
\caption{Overview of our trainable parameter allocation strategy. With the parameter sensitivity \bohan{set} $\gS$, we first get the top-$\tau$ sensitive parameters. Instead of directly tuning these sensitive parameters, we also boost the representational capability by replacing unstructured tuning with structured tuning at sensitive weight matrices that have a large number of sensitive parameters, which can be implemented by an existing structured tuning method, \eg, LoRA~\cite{hu2022lora} and Adapter~\cite{houlsby2019parameter}. Red lines and blocks represent trainable parameters and modules, while blue lines represent frozen parameters.}
\label{fig:main}
\vspace{-1.5em}
\end{figure*}


To overcome the first barrier, we simplify the empirical loss by approximating $s_n$ in the vicinity of $\vw$ by its first-order Taylor expansion
\vspace{-0.3em}
\begin{equation}
\vspace{-0.5em}
    \begin{aligned}
        s_n^{(1)} = -g_n\Delta_{w_n},
    \end{aligned}
\label{eq:first-order}
\end{equation}
where the gradients $\vg=\partial E/\partial\vw$, and $g_n$ is the gradient of the $n$-th element of $\vg$. 
To address the second barrier, following~\cite{liu2018darts,cai2018proxylessnas}, we take the one-step unrolled weight as the surrogate for $w_n^*$ and approximate $\Delta_{w_n}$ in Eq.~(\ref{eq:first-order}) with a single step of gradient descent. We can accordingly get $s_n^{(1)} \approx g_n^2\epsilon$,
where $\epsilon$ is the learning rate. Since $\epsilon$ is the same for all parameters, we can eliminate it when comparing the sensitivity with the other parameters and finally get 
\vspace{-0.5em}
\begin{equation}
\vspace{-0.3em}
    \begin{aligned}
        s_n^{(1)} \approx g_n^2.
    \end{aligned}
\label{eq:first-order-simp}
\end{equation}
Therefore, the sensitivity of a parameter can be efficiently measured by its potential to reduce the loss on the target domain. Note that although our criterion draws inspiration from pruning work~\cite{molchanov2019importance}, it is distinct from it. \cite{molchanov2019importance} measures the parameter importance by the squared change in loss when removing them, \ie, $\left( E(\gD_t, \vw)-E(\gD_t, \vw\mid w_n=0) \right)^2$ and finally derives the parameter importance by $\left( g_n w_n \right)^2$, which is different from our formulations in Eqs.~(\ref{eq:sensitivity}) and~(\ref{eq:first-order-simp}).

In practice, we accumulate $\gS$ from a total number of $C$ training samples ahead of fine-tuning to generate accurate sensitivity as shown in Algorithm~\ref{alg:tps}, where $C$ is a pre-defined hyper-parameter. In Section~\ref{subsec:abl}, we show that employing only 400 training samples is sufficient for getting reasonable parameter sensitivity, which requires only 5.5 seconds with a single GPU for any VTAB-1k dataset with ViT-B/16 backbone~\cite{vit}.

\vspace{-0.3em}
\subsection{Adaptive Trainable Parameters Allocation}
\label{subsec:SPT}
\vspace{-0.2em}

Our next step is to allocate trainable parameters based on the obtained parameter sensitivity set $\gS$ and a desired parameter budget $\tau$. A straightforward solution is to directly tune the top-$\tau$ most sensitive unstructured connections (parameters) \rev{while keeping the rest frozen}, which we name unstructured tuning. Specifically, we select the top-$\tau$ most sensitive weight connections in $\gS$ to form the sensitive weight connection set $\gT$. Then, for \rev{a} weight matrix $\mW\in \sR^{d_{\rm in}\times d_{\rm out}}$, we can get a binary mask $\mM\in \sR^{d_{\rm in}\times d_{\rm out}}$ computed by
\vspace{-0.5em}
\begin{equation}
\vspace{-0.5em}
    {\begin{array}{ll}
    \small
    \begin{aligned}
    \mM^j =
    \left\{\begin{array}{ll} 
    1 ~~~~~ \mW^j \in \gT \\
    0 ~~~~~ \mW^j \notin \gT
    \end{array}\right.
    \end{aligned},
    \small
    \end{array}}
\label{eq:mask}
\end{equation}
where $\mW^j$ and $\mM^j$ are the $j$-th element in $\mW$ and $\mM$, respectively. Accordingly, we can train the sensitive parameters by gradient descent and the updated weight matrix can be formulated as $\mW'\leftarrow \mW - \epsilon\vg_{\mW}\odot\mM$, where $\vg_{\mW}$ is the gradient for $\mW$.

However, considering PEFT approaches generally limit the proportion of trainable parameters to less than 1\%, tuning only a small number of unstructured weight connections might not have enough representational capability to handle the downstream datasets with large domain gaps from the source pre-training data. Therefore, to improve the representational capability, we propose to replace unstructured tuning with structured tuning at the sensitive weight matrices that have a high number of sensitive parameters. To preserve the parameter budget, we can implement structured tuning with an existing efficient structured tuning PEFT method~\cite{hu2022lora,chen2022adaptformer,houlsby2019parameter,jie2022convolutional} that learns to directly adjust \rev{all hidden dimensions at once}. We depict an overview of our trainable parameter allocation strategy in Figure~\ref{fig:main}. For example, we can employ the low-rank reparameterization trick LoRA~\cite{hu2022lora} to the sensitive weight matrices \rev{and the one-step update for $\mW$ can be formulated as}
\vspace{-0.4em}
\begin{equation}
\vspace{-0.4em}
    {\begin{array}{ll}
    \small
    \begin{aligned}
    \mW' = \left\{\begin{array}{ll} 
    \mW + \mW_{\rm down}\mW_{\rm up} & ~~ \text { if } ~~ \sum_{j=0}^{d_{\rm in}\times d_{\rm out}} \mM^j \geq \sigma_{\rm opt} \\
    \mW - \epsilon\vg_{\mW}\odot\mM & ~~ {\rm otherwise}
    \end{array}\right.
    \end{aligned},
    \small
    \end{array}}
\label{eq:weight_updat}
\end{equation}
where $\mW_{\rm down}\in \sR^{d_{\rm in}\times r}$ and $\mW_{\rm up}\in \sR^{r\times d_{\rm out}}$ are two learnable low-rank matrices to approximate the update of $\mW$ and rank $r$ is a hyper-parameter where $r \ll {\rm min}(d_{\rm in},d_{\rm out})$. In this way, we perform structured tuning on $\mW$ when its number of sensitive parameters exceeds $\sigma_{\rm opt}$, whose value depends on the pre-defined type of structured tuning method. For example, since implementing structured tuning with LoRA requires $2\times d_{\rm in} \times d_{\rm out} \times r$ trainable parameters for each sensitive weight matrix, we set $\sigma_{\rm LoRA} \leftarrow 2\times d_{\rm in} \times d_{\rm out} \times r$ to ensure that the number of trainable parameters introduced by structured tuning is always equal to or lower than the number of sensitive parameters.

In this way, our SPT adaptively incorporates both structured and unstructured tuning granularities to enable higher flexibility and stronger representational power, simultaneously. In Section~\ref{subsec:abl}, we show that structured tuning is important for the downstream tasks with larger domain gaps and both unstructured and structured tuning contribute clearly to the superior performance of our SPT. 

\begin{table*}[t]
\centering
\caption{\textbf{Quantitative results}. On the marker-based H36M, our method achieves SOTA performance in all optimization durations. While on the markerless ZJU-MoCap, our method is comparable with previous arts.  ``T.F.'' means template-free; ``Rep.'' means representation; ``T.T'' means the training time; $\ast$ denotes the evaluation on a subset of validation splits.}
\label{tab:main-table}
\resizebox{0.98\textwidth}{!}{%
\begin{tabular}{llll|cccc|cccc}
\toprule
\multicolumn{4}{l|}{}                                                                                                                                & \multicolumn{4}{c|}{H36M}                                                                                                                                                                                                                              & \multicolumn{4}{c}{ZJUMOCAP}                                                                                                                       \\ \hline
\multicolumn{1}{l|}{}                   &                        & \multicolumn{1}{l|}{}                       &                                     & \multicolumn{2}{c}{Training pose}                                                    & \multicolumn{2}{c|}{Novel pose}                                                                             & \multicolumn{2}{c}{Training pose}                             & \multicolumn{2}{c}{Novel pose}                                                    \\ \cline{5-8} \cline{9-12} 
\multicolumn{1}{l|}{\multirow{-2}{*}{}} & \multirow{-2}{*}{T.F.} & \multicolumn{1}{l|}{\multirow{-2}{*}{Rep.}} & \multirow{-2}{*}{T.T.}              & PSNR$\uparrow$                                       & SSIM$\uparrow$                & PSNR$\uparrow$                                       & SSIM$\uparrow$                                       & PSNR$\uparrow$                & SSIM$\uparrow$                & PSNR$\uparrow$                & \multicolumn{1}{c}{SSIM$\uparrow$}                \\ \midrule
\multicolumn{1}{l|}{NB~\cite{pengNeuralBodyImplicit2021}}                 &                        & \multicolumn{1}{l|}{NV}                     & $\sim$10 h                          & 23.31                                                & 0.902                         & {\color[HTML]{1F2329} 22.59}                         & \multicolumn{1}{l|}{{\color[HTML]{1F2329} 0.882}}                         & 28.10                         & 0.944                         & 23.49                         & \multicolumn{1}{l}{0.885}                         \\
\multicolumn{1}{l|}{SA-NeRF~\cite{xuSurfaceAlignedNeuralRadiance2022a}}            &                        & \multicolumn{1}{l|}{NV}                     & $\sim$30 h                          & 24.28                                                & 0.909                         & {\color[HTML]{1F2329} 23.25}                         & \multicolumn{1}{l|}{{\color[HTML]{1F2329} 0.892}}                         & \cellcolor[HTML]{FFFC9E}28.27 & \cellcolor[HTML]{FFFC9E}0.945 & \cellcolor[HTML]{FFFC9E}24.42 & \multicolumn{1}{l}{\cellcolor[HTML]{FFFC9E}0.902} \\
\multicolumn{1}{l|}{Ani-NeRF~\cite{pengAnimatableNeuralRadiance2021}}           & $\checkmark$           & \multicolumn{1}{l|}{NV}                     & $\sim$10 h                          & 23.00                                                & 0.890                         & {\color[HTML]{1F2329} 22.55}                         & \multicolumn{1}{l|}{{\color[HTML]{1F2329} 0.880}}                         & 26.19                         & 0.921                         & 23.38                         & \multicolumn{1}{l}{0.892}                         \\
\multicolumn{1}{l|}{ARAH~\cite{wangARAHAnimatableVolume}}               & $\checkmark$           & \multicolumn{1}{l|}{NV}                     & $\sim$48 h                          & \cellcolor[HTML]{FFCCC9}{\color[HTML]{000000} 24.79} & \cellcolor[HTML]{FFCCC9}0.918 & \cellcolor[HTML]{FFFC9E}{\color[HTML]{1F2329} 23.42} & \multicolumn{1}{l|}{\cellcolor[HTML]{FFFC9E}{\color[HTML]{1F2329} 0.896}} & \cellcolor[HTML]{FFCCC9}28.51 & \cellcolor[HTML]{FFCCC9}0.948 & \cellcolor[HTML]{FFCCC9}24.63 & \multicolumn{1}{l}{\cellcolor[HTML]{FFCCC9}0.911} \\
\multicolumn{1}{l|}{Ours}               & $\checkmark$           & \multicolumn{1}{l|}{Hybr}                   & $\sim$1 h                           & \cellcolor[HTML]{FFFC9E}24.72                        & \cellcolor[HTML]{FFFC9E}0.916 & \cellcolor[HTML]{FFCCC9}23.64                        & \multicolumn{1}{l|}{\cellcolor[HTML]{FFCCC9}0.899}                        & 26.57                         & 0.901                         & 24.38                         & \multicolumn{1}{l}{0.875}                         \\ \midrule
\multicolumn{1}{l|}{NB}                 &                        & \multicolumn{1}{l|}{NV}                     &                                     & 20.58                                                & 0.879                         & 20.27                                                & \multicolumn{1}{l|}{0.867}                                                & \cellcolor[HTML]{FFCCC9}26.87 & \cellcolor[HTML]{FFFC9E}0.922 & 23.67                         & \multicolumn{1}{l}{\cellcolor[HTML]{FFFC9E}0.885}                         \\
\multicolumn{1}{l|}{SA-NeRF}            &                        & \multicolumn{1}{l|}{NV}                     &                                     & 21.03                                                & 0.878                         & 20.71                                                & \multicolumn{1}{l|}{0.869}                                                & 24.92                         & 0.882                         & 23.38                         & \multicolumn{1}{l}{0.869} \\
\multicolumn{1}{l|}{Ani-NeRF}           & $\checkmark$           & \multicolumn{1}{l|}{NV}                     &                                     & 22.54                                                & 0.872                         & 21.79                                                & \multicolumn{1}{l|}{0.856}                                                & 21.23                         & 0.659                         & 20.65                         & \multicolumn{1}{l}{0.652}                         \\
\multicolumn{1}{l|}{ARAH}               & $\checkmark$           & \multicolumn{1}{l|}{NV}                     &                                     & \cellcolor[HTML]{FFFC9E}24.25                        & \cellcolor[HTML]{FFFC9E}0.904 & \cellcolor[HTML]{FFFC9E}23.61                        & \multicolumn{1}{l|}{\cellcolor[HTML]{FFFC9E}0.892}                        & 26.33 & \cellcolor[HTML]{FFCCC9}0.924 & \cellcolor[HTML]{FFCCC9}24.67 & \multicolumn{1}{l}{\cellcolor[HTML]{FFCCC9}0.911} \\
\multicolumn{1}{l|}{Ours}               & $\checkmark$           & \multicolumn{1}{l|}{Hybr}                   & \multirow{-5}{*}{$\sim$1 h$^\ast$}  & \cellcolor[HTML]{FFCCC9}24.83                        & \cellcolor[HTML]{FFCCC9}0.917 & \cellcolor[HTML]{FFCCC9}23.64                        & \multicolumn{1}{l|}{\cellcolor[HTML]{FFCCC9}0.899}                        & \cellcolor[HTML]{FFFC9E}26.66                         & 0.901                         & \cellcolor[HTML]{FFFC9E}24.64 & \multicolumn{1}{l}{0.880} \\ \midrule
\multicolumn{1}{l|}{NB}                 &                        & \multicolumn{1}{l|}{NV}                     &                                     & 20.54                                                & 0.863                         & 20.15                                                & \multicolumn{1}{l|}{0.853}                                                & \cellcolor[HTML]{FFFC9E}25.37 & \cellcolor[HTML]{FFFC9E}0.894                         & 23.54                         & \multicolumn{1}{l}{0.873}                         \\
\multicolumn{1}{l|}{SA-NeRF}            &                        & \multicolumn{1}{l|}{NV}                     &                                     & 20.81                                                & 0.848                         & 20.49                                                & \multicolumn{1}{l|}{0.841}                                                & 24.48                         & 0.878                         & 23.75                         & \multicolumn{1}{l}{0.872}                         \\
\multicolumn{1}{l|}{Ani-NeRF}           & $\checkmark$           & \multicolumn{1}{l|}{NV}                     &                                     & 20.57                                                & 0.822                         & 20.22                                                & \multicolumn{1}{l|}{0.806}                                                & 21.17                         & 0.652                         & 21.16                         & \multicolumn{1}{l}{0.656}                         \\
\multicolumn{1}{l|}{ARAH}               & $\checkmark$           & \multicolumn{1}{l|}{NV}                     &                                     & \cellcolor[HTML]{FFFC9E}23.83                        & \cellcolor[HTML]{FFFC9E}0.895 & \cellcolor[HTML]{FFFC9E}23.13                        & \multicolumn{1}{l|}{\cellcolor[HTML]{FFFC9E}0.884}                        & 25.09 & \cellcolor[HTML]{FFCCC9}0.906 & \cellcolor[HTML]{FFFC9E}24.21 & \multicolumn{1}{l}{\cellcolor[HTML]{FFCCC9}0.898} \\
\multicolumn{1}{l|}{Ours}               & $\checkmark$           & \multicolumn{1}{l|}{Hybr}                   & \multirow{-5}{*}{$\sim$10 m$^\ast$} & \cellcolor[HTML]{FFCCC9}24.27                        & \cellcolor[HTML]{FFCCC9}0.909 & \cellcolor[HTML]{FFCCC9}23.37                        & \multicolumn{1}{l|}{\cellcolor[HTML]{FFCCC9}0.897}                        & \cellcolor[HTML]{FFCCC9}25.51                         & 0.888 & \cellcolor[HTML]{FFCCC9}24.42 & \multicolumn{1}{l}{\cellcolor[HTML]{FFFC9E}0.878} \\
\bottomrule
\end{tabular}%
}
\vspace{-1em}
\end{table*}

\section{Experiments}

\subsection{Dataset and Metrics}


\noindent \textbf{H36M} consists of 4 multi-view cameras and uses \textbf{marker-based} motion capture to collect human poses.
Each video contains a single subject performing a complex action.
We follow~\cite{pengAnimatableNeuralRadiance2021} data protocol which includes subject S1, S5-S9, and S11.
The videos are split into two parts: ``training poses'' for novel view synthesis and ``Unseen poses'' for novel pose synthesis.
Among the video frames, 3 views are used for training, and the rest views are for evaluation.
The novel view and novel pose metrics are computed on rest views.
We use the same data preprocessing as~\cite{pengAnimatableNeuralRadiance2021}.

\noindent \textbf{ZJU-MoCap} records 9 subjects performing complex actions with 23 cameras.
The human poses are obtained with a markerless motion capture system.
Thus the pose tracking is rather noisier compared with H36M.
Likewise, there are two sets of video frames, ``training poses'' for novel view synthesis and ``Unseen poses'' for novel pose synthesis.
4 evenly distributed camera views are chosen for training, and the rest 19 views are for evaluation.
Again, the evaluation metrics are computed on rest views.
The same data protocol and processing approaches are adopted following~\cite{pengNeuralBodyImplicit2021, pengAnimatableNeuralRadiance2021}.

\noindent \textbf{Metrics}. We follow the typical protocol in ~\cite{pengNeuralBodyImplicit2021, pengAnimatableNeuralRadiance2021} using
two metrics to measure image quality: peak signal-to-noise ratio (PSNR) and structural similarity index (SSIM).

\subsection{Evaluation and Comparison}





\noindent \textbf{Baselines}. We compare our method with template-based methods~\cite{pengNeuralBodyImplicit2021, xuSurfaceAlignedNeuralRadiance2022a} and template-free methods~\cite{pengAnimatableNeuralRadiance2021, wangARAHAnimatableVolume}. Here we list the average metric values with different training times to illustrate our very competitive performance and significant speed boost. 1) Tempelate-based methods. 
Neural Body (NB)~\cite{pengNeuralBodyImplicit2021} learns a set of latent codes anchored to a deformable template mesh to provide geometry guidance.
Surface-Aligned NeRF (SA-NeRF)~\cite{xuSurfaceAlignedNeuralRadiance2022a} proposes projecting a point onto a mesh surface to align surface points and signed height to the surface. 
2) Template-free methods.
Animatable NeRF (Ani-NeRF)~\cite{pengAnimatableNeuralRadiance2021} introduces neural blend weight fields to produce the deformation fields instead of explicit template control.
ARAH~\cite{wangARAHAnimatableVolume} combines an articulated implicit surface representation with volume rendering and proposes a novel joint root-finding algorithm.


\looseness=-1
\noindent \textbf{Comparisons with state-of-the-arts}.
Table~\ref{tab:main-table} illustrates the quantitative comparisons with previous arts.
Notably, our method achieves very competitive performance within much less training time.
The previous volume rendering-based counterparts spend tens of hours of optimization time, while our method only takes an hour of training (for previous SOTA method ARAH~\cite{wangARAHAnimatableVolume}, it takes about 2 days of training).
On the marker-based H36M dataset, our method reaches the SOTA performance in terms of novel view synthesis on training poses and outperforms previous SOTA (ARAH~\cite{wangARAHAnimatableVolume}) for novel view synthesis on novel poses, which indicates that our method can generalize better on novel poses.
The significant boost in training speeds lies in, on the one hand, the core mesh representation which can be rendered efficiently with the current graphic pipeline~\cite{laineModularPrimitivesHighPerformance2020}.
On the other hand, the triangular renderer uses less memory.
Thus we can compute losses over the whole image to learn the representations holistically.
In contrast, previous methods are limited to much fewer sampled pixels in each optimization step.

On the markerless ZJU-Mocap dataset, our method falls behind for training poses novel view synthesis and ranks 3rd place in terms of unseen poses novel view synthesis among the competitors.
We argue that the quality of pose tracking results in the performance gaps between the two datasets.
The markerless pose tracking data \textbf{are much noisier} than the marker-based ones (\eg, the tracked skeleton sequence is jittering, and the naked human~\cite{loperSMPLSkinnedMultiperson2015} rendering is misaligned with human parsings), which makes our performance saturated by harming the multi-view consistency.
The problem is even amplified with the holistic loss computation over the whole pixels.
We conduct additional ablation on pose tracking quality on H36M in Sec.~\ref{exp:data:pose_tracking_type}.
Besides, our non-rigid modeling is only over the surface (no topology change), which is less powerful than the volume rendering-based ones (with topology change).

 We further each method under \textbf{the same optimization duration} in Table~\ref{tab:main-table}.
For the extremely low inference speed of our competitor, we only evaluate at most 10 frames in each subject, and for ZJU MoCap we only choose another 4 evenly spaced cameras as the evaluation views.
For both 1 hour and 10 minutes optimization time, our method outperforms other methods for both training poses and unseen poses novel view synthesis on the marker-based H36M dataset.
On the markerless ZJU-Mocap dataset, our method is comparable with previous SOTA in terms of PSNR and SSIM for both evaluation splits.

\looseness=-1
Figure~\ref{fig:h36m} illustrates the qualitative comparisons between our method and previous arts under the same optimization duration.
It is worth noting that on both H36M and ZJU-MoCap datasets, our method can synthesize clearer and more fine-grained images against competitors, which raises the misalignment of the quantitative metrics for measuring image similarity.


\setlength{\columnsep}{6pt}
\begin{wrapfigure}{r}{0.625\linewidth}
\centering
\vspace{-1em}
\includegraphics[width=1\linewidth]{figs/inference.png}
\vspace{-2em}
\caption{\textbf{Rendering Efficiency}}
\vspace{-1em}
\label{fig:inference}
\end{wrapfigure}


\noindent \textbf{Rendering Efficiency}:
We provide the rendering speed of our method against previous methods in Figure~\ref{fig:inference}. Our method reaches real-time inference speed (100+ FPS for rendering 512×512 images), which is hundreds of times faster than the previous ones. 
Our method takes considerably less memory than the previous ones.


\subsection{Ablation Studies}


\begin{table}[htbp]
\centering
\caption{
\textbf{The ablation on each module from our method}.
The mesh tends to be noisy and poor for rendering novel poses without MLP parametrization for the geometry module;
Removing the non-rigid module harms the convergence of our model due to the disability to solve multi-view inconsistency;
PBR materials improve the overall shading quality by joint modeling both decomposed materials and lighting.
}
\label{table:ablation:method}
\begin{tabular}{l|cc|cc}
\toprule
              & \multicolumn{2}{c|}{Training Pose} & \multicolumn{2}{c}{Novel Pose}     \\ \hline
              & \multicolumn{1}{c|}{PSNR}  & SSIM  & \multicolumn{1}{c|}{PSNR}  & SSIM  \\ \midrule
w/o SDF MLP   & \multicolumn{1}{c|}{25.17} & 0.913 & \multicolumn{1}{c|}{23.37} & 0.874 \\
w/o Non-rigid & \multicolumn{1}{c|}{25.03} & 0.909 & \multicolumn{1}{c|}{23.45} & 0.877 \\
w/o PBR       & \multicolumn{1}{c|}{25.10} & 0.914 & \multicolumn{1}{c|}{23.44} & 0.878 \\
w/o Specular  & \multicolumn{1}{c|}{25.24} & 0.915 & \multicolumn{1}{c|}{\textbf{23.58}} & \textbf{0.879} \\ \midrule
Full          & \multicolumn{1}{c|}{\textbf{25.26}} & \textbf{0.916} & \multicolumn{1}{c|}{23.52} & \textbf{0.879} \\
\bottomrule
\end{tabular}
\end{table}

\begin{figure}[t]
\centering
\includegraphics[width=1\linewidth]{figs/ablate-method.s9.1x6}
\caption{\textbf{Qualitative ablation on each module}. The SDF MLP improves the mesh smoothness; non-rigid modeling proves the texture quality by solving the multi-view consistency of cloth dynamics; The PBR materials have a larger capacity for modeling complex materials and lighting against the only-RGB and the no-specular counterparts, which further facilitates both mesh and material learning.}
\label{fig:ablation:method}
\end{figure}

We conduct ablation studies on the H36M S9 subject.


\noindent \textbf{The parametrization type for SDF field.}
The SDF fields can either be parameterized as either MLPs or value fields.
Table~\ref{table:ablation:method} and Figure~\ref{fig:ablation:method} show that using MLP to predict SDF values results in a smoother mesh surface that is watertight.
MLP offers extrapolation ability to predict invisible parts and keep the mesh watertight.
While directly optimizing SDF value fields leads to a jiggling mesh surface and holes in invisible parts during training (\eg, underarm).


\noindent \textbf{The shading model type in geometry module.}
We compare PBR shading models with directly predicting RGB colors and PBR without shading specular.
Table~\ref{table:ablation:method} and Figure~\ref{fig:ablation:method} show that PBR shading models lead to higher metrics against RGB predications,
which indicates that PBR materials can better model complex textures and lights for dynamic humans.
Removing the specular term in PBR does not affect the performance much. 
We conjecture that there is less specularity in human skin and clothes materials.


\noindent \textbf{The impact of the non-rigid net in motion module.}
As shown in Table~\ref{table:ablation:method} and Figures~\ref{fig:ablation:method}, modeling pose-dependent non-rigid dynamics of clothes improves the overall reconstruction quality. It facilitates the aggregation of shading information for multi-view inputs during training.

\begin{figure}[t]
\centering
\includegraphics[width=1\linewidth]{figs/pose_track_type.pdf}
\vspace{-1em}
\caption{\textbf{Qualitative results of models trained on poses} from marker-less and marker-based systems.}
\label{fig:ablation:data:pose_track_type}
\vspace{-1em}
\end{figure}

\begin{figure}[t]
\centering
\includegraphics[width=1\linewidth]{figs/num_views.pdf}
\vspace{-1em}
\caption{\textbf{Comparison of models trained with different numbers of camera views} on the subject ``S9".}
\label{fig:ablation:data:num_views}
\vspace{-1em}
\end{figure}


\begin{table}[htbp]
\centering
\caption{\textbf{The ablations results on data quality and quantity} on H36M S9 subject,  in terms of PSNR and SSIM (higher is better). The better the data quality, the better the reconstruction results.}
\label{table:ablation:data}
\begin{tabular}{ccccc}
\toprule
\multicolumn{1}{l|}{}    & \multicolumn{2}{c|}{Training pose}                         & \multicolumn{2}{c}{Novel pose}     \\ \hline
\multicolumn{1}{l|}{}              & \multicolumn{1}{c|}{PSNR$\uparrow$}  & \multicolumn{1}{c|}{SSIM$\uparrow$}  & \multicolumn{1}{c|}{PSNR$\uparrow$}  & \multicolumn{1}{c}{SSIM$\uparrow$} \\ \midrule
\multicolumn{5}{l}{\textbf{(a)} type of pose tracking}                                                                                     \\ \midrule
\multicolumn{1}{l|}{w/o marker}  & \multicolumn{1}{l|}{24.73} & \multicolumn{1}{l|}{0.893} & \multicolumn{1}{l|}{22.60} & 0.853                    \\
\multicolumn{1}{l|}{w/ marker} & \multicolumn{1}{l|}{\textbf{25.53}} & \multicolumn{1}{l|}{\textbf{0.911}} & \multicolumn{1}{l|}{\textbf{23.80}} & \textbf{0.879}                    \\ \midrule
\multicolumn{5}{l}{\textbf{(b)} number of training views}                                                                                     \\ \midrule
\multicolumn{1}{l|}{1 view}   & \multicolumn{1}{l|}{25.09} & \multicolumn{1}{l|}{0.906} & \multicolumn{1}{l|}{22.97} & 0.866 \\
\multicolumn{1}{l|}{2 views}   & \multicolumn{1}{l|}{25.56} & \multicolumn{1}{l|}{\textbf{0.911}} & \multicolumn{1}{l|}{\textbf{23.76}} & \textbf{0.878} \\
\multicolumn{1}{l|}{3 views}   & \multicolumn{1}{l|}{\textbf{25.57}} & \multicolumn{1}{l|}{\textbf{0.911}} & \multicolumn{1}{l|}{23.67} & 0.876 \\ \midrule
\multicolumn{5}{l}{\textbf{(c)} number of training frames}                                                                                    \\ \midrule
\multicolumn{1}{l|}{1 frame}   & \multicolumn{1}{l|}{20.93} & \multicolumn{1}{l|}{0.817} & \multicolumn{1}{l|}{19.58} & 0.785 \\
\multicolumn{1}{l|}{100 frames} & \multicolumn{1}{l|}{23.99} & \multicolumn{1}{l|}{0.882} & \multicolumn{1}{l|}{22.49} & 0.856 \\
\multicolumn{1}{l|}{200 frames} & \multicolumn{1}{l|}{\textbf{25.27}} & \multicolumn{1}{l|}{\textbf{0.905}} & \multicolumn{1}{l|}{\textbf{23.32}} & \textbf{0.873} \\
\multicolumn{1}{l|}{800 frames} & \multicolumn{1}{l|}{24.89} & \multicolumn{1}{l|}{0.900} & \multicolumn{1}{l|}{23.16} & \textbf{0.873} \\ 
\bottomrule
\end{tabular}
\end{table}




\noindent \textbf{The impact of human tracking quality.}
\label{exp:data:pose_tracking_type}
Table~\ref{table:ablation:data}~(a) and Figure~\ref{fig:ablation:data:pose_track_type} show that using marker-based pose-tracking data can give better results. 
The same phenomenon has been stated in~\cite{pengAnimatableNeuralRadiance2021}.
Noisy marker-less pose-tracking harms the optimization process by damaging the multi-view consistency and the exact pose for shading optimization,
which leads to blurry textures.

\noindent \textbf{The impact of training view amount.}
Table~\ref{table:ablation:data}~(b) and Figure~\ref{fig:ablation:data:num_views} reveal that giving one camera of view degrades the overall reconstruction quality, and multi-view consistency improves the final results.
The model can aggregate multi-view information for better shading optimization, thus leading to clearer surface materials.

\noindent \textbf{The impact of training frame amount.}
As the number of training frames increases, the rendering quality on novel view and novel pose increases as well (Table~\ref{table:ablation:data}~(c) and Figure~\ref{fig:ablation:data:num_frames}). Notice that the reconstruction quality saturated after using a certain amount of training frames, the same results can be observed in~\cite{pengAnimatableNeuralRadiance2021} as well.

\begin{figure}[t]
\centering
\includegraphics[width=1\linewidth]{figs/num_frames.pdf}
\vspace{-1em}
\caption{\textbf{Comparison of models trained with different numbers of video frames} on the subject ``S9".}
\label{fig:ablation:data:num_frames}
\vspace{-1em}
\end{figure}


\subsection{Applications}
\label{exp:application}
After training, we can export mesh representations, which enables instant downstream applications.
We showcase two examples of novel pose synthesis, material editing, and human relighting in Figure~\ref{fig:teaser}. For more examples, please refer to our supplementary.


\section{Conclusions}

In this paper, we propose HGNAS, the first hardware-aware framework to explore efficient graph neural architecture for edge devices.
HGNAS can automatically search for optimal GNN architectures that maximize both task accuracy and computation efficiency.
HGNAS leverages the multi-stage hierarchical search strategy and GNN hardware performance predictor to efficiently explore the fine-grained GNN design space.
Extensive experiments show that GNN models generated by HGNAS consistently outperform SOTA GNNs, achieving about $10.6\times$ speedup and $88.2\%$ peak memory reduction across various edge platforms.
We believe that HGNAS has made pivotal progress in bringing GNNs to real-life edge applications.


%%%%%%%%% REFERENCES
{\small
\bibliographystyle{ieee_fullname}
\bibliography{ema_bib}
}

\clearpage

\section*{Appendix}
Thank you for reading our supplementary materials!
Here we provide depth descriptions of our method, including details about loss functions (Sec.~\ref{supp:loss_functions}), image-based lighting (Sec.~\ref{supp:light_integration}), and image-based lighting (Sec.~\ref{supp:implementation_details}).
Then we present additional ablations about training views (Sec.~\ref{supp:additional_ablations:num_views}) and skinning module (Sec.~\ref{supp:additional_ablations:skinning}).
Additional experimental results are illustrated in Sec.~\ref{supp:more_comparisons} and Sec.~\ref{supp:mesh_evaluation}.
We showcase application examples in Sec.~\ref{supp:applications}.
In the end, we discuss limitations and social impact in Sec.~\ref{supp:limitation_discussions}.
We strongly encourage our readers to view the supplemental video for a more comprehensive visual perception.

\section{Loss Functions}

\label{supp:loss_functions}

Our loss function $L = L_\mathrm{img} + L_\mathrm{mask} + L_\mathrm{reg}$ is composed of three parts: an image loss $L_\mathit{img}$ using $\ell_1$ norm on tone mapped colors, and mask loss $L_\mathrm{mask}$ using squared $\ell_2$, and regularization losses $L_\mathrm{reg}$ to improve the quality of canonical geometry, materials, lights, and motion.

\noindent \textbf{Image loss}: 
our renderer utilizes physically-based shading to produce high-dynamic range (HDR) images.
Then the complex materials and environmental lights are elaborately optimized.
Thus our loss function requires a full range of floating point values.
We follow~\cite{hasselgrenAppearanceDrivenAutomatic3D2021,munkbergExtractingTriangular3D2022,hasselgrenShapeLightMaterial2022} to compute $\ell_1$ norm on tone mapped colors.
Specifically, we first transform linear radiance values $i$ according to a tone-mapping operator $T(i)=\Gamma(\log (i+1))$, in which $\Gamma(i)$ is a linear RGB to sRGB transformation function~\cite{stokesStandardDefaultColor1996}:
\begin{equation}
\begin{aligned}
\Gamma(i) & = \begin{cases}12.92 i & i \leq 0.0031308 \\
(1+a) i^{1 / 2.4}-a & i>0.0031308\end{cases} \\
a & =0.055,
\end{aligned}
\end{equation}


\noindent \textbf{Mask loss}:
The renderer~\cite{laineModularPrimitivesHighPerformance2020} renders both the shaded images and the corresponding rasterization masks in a differentiable manner.
Therefore, we compute the $\ell_2$ norm between the masks and the preprocessed mattings (in both ZJU-MoCap and H36M benchmarks, we use the provided preprocessed subject masks from~\cite{pengNeuralBodyImplicit2021,gongInstanceLevelHumanParsing2018}), akin to the traditional shape-from-silhouette~\cite{maInvitation3DVision2004} technique.
The mask loss is parallel with the image loss, yet facilitates the course of shading optimization by making shape convergence super fast in about a hundred training steps.

\noindent \textbf{Regularizers}:
We need various priors to encourage the optimization to converge at a place where the geometry, materials, and lighting are well separated and smooth enough~\cite{munkbergExtractingTriangular3D2022,hasselgrenShapeLightMaterial2022}.
Therefore, we choose to minimize regularization during training.

\noindent We introduce smoothness to PBR materials in terms of albedo $\mathbf{k}_d$, specular parameters $\mathbf{k}_\mathrm{orm}$, and surface geometry nomral $\mathbf{n}$ as following:
\begin{equation}
L_{\mathbf{k}}=\frac{1}{\left|\mathbf{x}_{\text {surf }}\right|} \sum_{\mathbf{x}_{\text {surf }}}\left|\mathbf{k}\left(\mathbf{x}_{\text {surf }}\right)-\mathbf{k}\left(\mathbf{x}_{\text {surf }}+\mathbf{\epsilon}\right)\right|,
\end{equation}

\noindent where ${\left|\mathbf{x}_{\text {surf }}\right|}$ is a surface point on the surface in canonical space and $\mathbf{\epsilon} \sim \mathcal{N}(0,\sigma\!\!=\!\!0.01)$ is a small random offset.
We regularize the geometry normal on the surface of the canonical mesh derived from the SDF field for a seek of a smoother surface and avoidance of holes in the surface.

\noindent We regularize light by assuming the neutral spectrum in the real world.
Specifically, given the per-channel average radiance densities $\Bar{c_i}$, we penalize the color shifts as:
\begin{equation}
L_{\text {light }}=\frac{1}{3} \sum_{i=0}^3 \left| {c_i}-\frac{1}{3} \sum_{i=0}^3 {c_i} \right|,
\end{equation}

To encourage a watertight surface and reduce floating meshes both inside and outside the subject, 
we impose regularizations on the SDF field as:
\begin{equation}
\begin{aligned}
L_{\mathrm{sdf}}=\sum_{i, j \in \mathrm{S}_e} &H\left(\sigma\left(s_i\right), \operatorname{sign}\left(s_j\right)\right) \\
& +H\left(\sigma\left(s_j\right), \operatorname{sign}\left(s_i\right)\right),
\end{aligned}
\end{equation}

\noindent where $\mathrm{S}_e$ is the set of all vertex along their edges in which the signs of the SDF values are different (\ie, $\mathrm{sign}(s_i) \neq \mathrm{sign}(s_j)$).
To remove the floating meshes outside the surface, we impose an additional loss.
For a triangle surface $f$ extracted by marching tetrahedra, if $f$ is invisible, we encourage its SDF values to be positive as:
\begin{equation}
L_{\mathrm{invis}}=\sum_{i \in \mathrm{S}_{\mathrm{invis}}} H(\sigma\left(s_i\right), 1).
\end{equation}

We weigh the above terms and use the loss for all our experiments:
\begin{equation}
\begin{aligned}
	L &= L_\text{image} + L_\text{mask} \\
	&+ \underbrace{\lambda_{\mathbf{k}_{\mathrm{d}}}}_{=0.03} L_{\mathbf{k}_{\mathrm{d}}} 
	+ \underbrace{\lambda_{\mathbf{k}_{\mathrm{orm}}}}_{=0.05} L_{\mathbf{k}_{\mathrm{orm}}}
	+ \underbrace{\lambda_{\mathbf{n}}}_{=0.025} L_{\mathbf{n}} \\
	&+ \underbrace{\lambda_\text{light}}_{=0.005} L_\text{light}
 	+ \underbrace{\lambda_\text{sdf}}_{=0.02} L_\text{sdf}
  	+ \underbrace{\lambda_\text{invis}}_{=0.01} L_\text{invis}.
\end{aligned}
\end{equation}
\section{Image-based Lighting}

\label{supp:light_integration}
The split sum shading model is widely used in real-time rendering~\cite{akenine2019real}, giving both realism and efficiency against spherical Gaussians (SG) and spherical harmonics (SH)~\cite{bossNeRDNeuralReflectance2021a,chenLearningPredict3D2019,zhangPhySGInverseRendering2021}.
We use a differentiable split sum~\cite{karisRealShadingUnreal2013} shading model to approximate rendering equation~\cite{kajiyaRENDERINGEQUATION1986} for image-based environment light learning as~\cite{munkbergExtractingTriangular3D2022}:
\begin{equation}
\begin{split}
L\left(\omega_o\right) & \approx \int_{\Omega} f\left(\omega_i, \omega_o\right)\left(\omega_i \cdot \mathbf{n}\right) d \omega_i \\
&\quad \int_{\Omega} L_i\left(\omega_i\right) D\left(\omega_i, \omega_o\right)\left(\omega_i \cdot \mathbf{n}\right) d \omega_i.
\label{supp:euq:render}
\end{split}
\end{equation}

\noindent where $D$ is the GGX normal distribution function (NDF)~\cite{walterMicrofacetModelsRefraction} in a Cook-Torrance microfacet specular shading model~\cite{cookReflectanceModelComputer}.
The first term contributes to the specular BSDF \textit{wrt.} a solid white environment light, which depends solely on the roughness $r$ of the BSDF and the light-surface angles $\cos \theta = \omega_i \cdot \mathbf{n}$.
The second term contributes to the integral of the incoming radiance with the GGX normal distribution function, $D$.
Both terms can be pre-computed and filtered to reduce computation~\cite{karisRealShadingUnreal2013}.

The training parameters are texels of a cube light map whose resolution is $6 \times 512 \times 512$.
The pre-integrated lighting for the least roughness values is derived from the base level, and multiple smaller mip levels are constructed from it~\cite{karisRealShadingUnreal2013}.
Each mip-map is filtered by average-pooling the base level of the current resolution and is convolved with the GGX normal distribution function.
The per mip-level filter bounds are pre-computed as well.
We leverage a PyTorch implementation with CUDA extensions from~\cite{munkbergExtractingTriangular3D2022}.
Moreover, a cube map is created to represent the diffuse lighting in a low resolution, akin to the filtered specular probe.
It shares the same optimizable parameters and is average-pooled to the mip level with $r=1$ roughness.
The pre-filtering only involves the first term in Eq.~\ref{supp:euq:render}.

\section{Implementation Details}

\label{supp:implementation_details}



\noindent \textbf{SDF network}. We parametrize the SDF field with an MLP to increase surface water-tight and smoothness.
We choose the MLP architecture from \cite{mildenhallNeRFRepresentingScenes2020}, which consists of 6 frequency bands for positional encoding, and 8 linear layers, each having 256 neurons, followed by ReLU activations.
We implicitly regularize the smoothness by increasing the Lipschitz property in the SDF field\cite{liuLearningSmoothNeural2022a}.

\noindent \textbf{Material network}. The material model is a small MLP with hash-encoding~\cite{mullerInstantNeuralGraphics2022} as the materials query is computationally extensive.
The MLP consists of two linear layers, each having 32 neurons, followed by ReLU activations.
The hash-encoding has a spatial resolution of 4096 and the rest configures are the same as~\cite{munkbergExtractingTriangular3D2022}.
To reduce computation, we predict all material channels at once with one backbone network.
Besides, we introduce inductive bias of materials of clothed humans in the real world,
by providing minimum and maximum values for each materials channel.
We follow~\cite{Zhang2021NeRFactorNF} to limit the albedo $\mathbf{k}_d \in [0.03, 0.8]$, and the roughness $\mathbf{k}_r \in [0.08, 1]$.
The texels in the environment light are randomly initialized between $[0.25, 0.75]$.

\noindent \textbf{Motion networks}. For the motion module, we use the same MLP architecture as~\cite{chenSNARFDifferentiableForward2021,wangARAHAnimatableVolume}, which is similar to our SDF MLP.
To resolve the problem where the training pose variation is too limited for skinning field learning (\eg, self-rotation video without any limbs movements), 
we initialize the MLP with the pre-trained skinning model provided by~\cite{wangARAHAnimatableVolume},
and impose $\ell_2$ norm for the skinning weights logits between our predictions and the ground truth from SMPL~\cite{loperSMPLSkinnedMultiperson2015}.
We ablate the design choices in Sec.~\ref{supp:additional_ablations:skinning}.
For the non-rigid modeling, we use another 4-layer ReLU MLP with a 4-frequency-band positional encoding.
We also progressively anneal its encoding for 5k iterations as~\cite{parkNerfiesDeformableNeural2021}.
The weights of the last layer are initialized with a uniform distribution $\mathcal{U}(-10^{-5}, 10^{-5})$, \ie initializing the non-rigid offsets to be close to zero and not interfering with the major optimizations of geometry and materials.

\noindent \textbf{Optimization}. 
We use Adam~\cite{kingmaAdamMethodStochastic2017} as our default optimizer.
We optimize the subject for 5k steps for 1024$\times$1024 images or 10k steps 512$\times$512 images.
We disable the perturbed normal map during optimization as it leads to SDF collapsing abruptly at a certain step (\ie, all SDF values are positive or negative where marching tetrahedra fails).
The optimization process takes about an hour on a single NVIDIA GTX3090 GPU.
The indicative results with plausible quality appear after a few minutes, which is quite faster than our counterparts~\cite{pengNeuralBodyImplicit2021,pengAnimatableNeuralRadiance2021,wangARAHAnimatableVolume,xuSurfaceAlignedNeuralRadiance2022a}.
Such superior efficiency could largely accelerate downstream applications.
The training visualization is presented in the supplemental video.

\noindent \textbf{Tetrahedra grids}. We start with a tetrahedra grid with $128 \times 128$ resolution, including 192k tetrahedra and 37k vertices.
Each tetrahedron can produce at most 2 triangles by marching tetrahedra algorithm~\cite{munkbergExtractingTriangular3D2022,shenDeepMarchingTetrahedra2021,gaoLearningDeformableTetrahedral2020}.
To increase the resolution of the tetrahedra grid, we subdivide the grid at the 500th step.
To avoid the out-of-memory problem caused by the vast amount of floating meshes in the void space at the beginning of training, we pre-train the SDF network to \textbf{match a visual hull} of humans in canonical space.
The hull could be derived from either the skeleton capsules or the SMPL~\cite{loperSMPLSkinnedMultiperson2015} mesh.
Note that we only pre-train for 500 iterations, which leads to \textbf{a very coarse shape} akin to the visual hull rather than the given ground truth mesh.
The initialized mesh is presented in the training visualization part of the supplemental video.
\section{Additional Ablations}

\label{supp:additional_ablations}

\noindent \textbf{Number of Training Views}.
\label{supp:additional_ablations:num_views}
Table~\ref{table:abla:view} and Figure~\ref{fig:abla:view} show that giving one camera of view degrades the overall reconstruction quality, and multi-view consistency improves the final results.
The model can aggregate multi-view information for better shading optimization, thus leading to clearer surface materials.

\begin{table}[htbp]
\centering
\caption{\textbf{Ablation results of training views on the ZJU-MoCap 313 subject}.}
\label{table:abla:view}
\begin{tabular}{ccccc}
\toprule
\multicolumn{1}{l|}{}    & \multicolumn{2}{c|}{Training pose}                         & \multicolumn{2}{c}{Novel pose}     \\ \hline
\multicolumn{1}{l|}{ZJU-MoCap 313}              & \multicolumn{1}{c|}{PSNR$\uparrow$}  & \multicolumn{1}{c|}{SSIM$\uparrow$}  & \multicolumn{1}{c|}{PSNR$\uparrow$}  & \multicolumn{1}{c}{SSIM$\uparrow$} \\ \midrule
\multicolumn{1}{l|}{1 view}   & \multicolumn{1}{l|}{24.39} & \multicolumn{1}{l|}{0.913} & \multicolumn{1}{l|}{21.45} & 0.869 \\
\multicolumn{1}{l|}{2 views}   & \multicolumn{1}{l|}{28.06} & \multicolumn{1}{l|}{0.945} & \multicolumn{1}{l|}{22.81} & 0.888 \\
\multicolumn{1}{l|}{3 views}   & \multicolumn{1}{l|}{28.50} & \multicolumn{1}{l|}{0.956} & \multicolumn{1}{l|}{23.17} & 0.894 \\
\multicolumn{1}{l|}{4 views}   & \multicolumn{1}{l|}{\textbf{29.04}} & \multicolumn{1}{l|}{\textbf{0.961}} & \multicolumn{1}{l|}{\textbf{23.20}} & \textbf{0.896} \\
\bottomrule
\end{tabular}
\end{table}

\begin{figure}[tbph]
\centering
\includegraphics[width=\linewidth]{supp/figs/ablation-view.png}
\vspace{-1em}
\caption{\textbf{Ablation study of training views on the ZJU-MoCap 313 subject.}}
\label{fig:abla:view}
\end{figure}






\noindent \textbf{The effect of Skinning Module Design}
\label{supp:additional_ablations:skinning}
Table~\ref{table:ablation:skinning-h36m}-\ref{table:ablation:skinning-zju} and Figure~\ref{fig:abla:h36m}-\ref{fig:abla:mesh:zju} reveal that the initialization with pre-trained skinning net and the regularization on surface skinning improve the overall reconstruction quality.
The initialization provides skinning prior which helps to speed up geometry convergence. 
From Figure~\ref{fig:abla:h36m}-\ref{fig:abla:zju}, the geometry details improve with the initialization under the same training time.

The regularization on surface skinning prevents geometry degradation. 
Figure~\ref{fig:abla:mesh:zju} indicates that
our model can not learn correct canonical geometry without the initialization and the regularization.
The mesh distortion is reduced with the regularization.

\begin{table}[htbp]
\centering
\caption{
\textbf{The ablation on skinning module of H36M S9 dataset}.}
\label{table:ablation:skinning-h36m}
\resizebox{\columnwidth}{!}{%
\begin{tabular}{l|cc|cc}
\toprule
              & \multicolumn{2}{c|}{Training Pose} & \multicolumn{2}{c}{Novel Pose}     \\ \hline
H36M S9       & \multicolumn{1}{c|}{PSNR}  & SSIM  & \multicolumn{1}{c|}{PSNR}  & SSIM  \\ \midrule
w/o skinning init. \& reg.   & \multicolumn{1}{c|}{24.88} & 0.905 & \multicolumn{1}{c|}{21.97} & 0.851 \\
w/ skinning intialization & \multicolumn{1}{c|}{26.28} & 0.926 & \multicolumn{1}{c|}{24.47} & 0.897 \\
w/ skinning regularization       & \multicolumn{1}{c|}{26.24} & 0.925 & \multicolumn{1}{c|}{24.34} & 0.896 \\ \midrule
Full          & \multicolumn{1}{c|}{\textbf{26.29}} & \textbf{0.926} & \multicolumn{1}{c|}{\textbf{24.53}} & \textbf{0.899} \\
\bottomrule
\end{tabular}%
}
\end{table}


\begin{table}[htbp]
\centering
\caption{
\textbf{The ablation on skinning module of ZJU-MoCap 313 dataset}.}
\label{table:ablation:skinning-zju}
\resizebox{\columnwidth}{!}{%
\begin{tabular}{l|cc|cc}
\toprule
              & \multicolumn{2}{c|}{Training Pose} & \multicolumn{2}{c}{Novel Pose}     \\ \hline
ZJU-MoCap 313 & \multicolumn{1}{c|}{PSNR}  & SSIM  & \multicolumn{1}{c|}{PSNR}  & SSIM  \\ \midrule
w/o skinning init. \& reg.   & \multicolumn{1}{c|}{27.46} & 0.949 & \multicolumn{1}{c|}{20.31} & 0.831 \\
w/ skinning intialization & \multicolumn{1}{c|}{28.82} & 0.958 & \multicolumn{1}{c|}{23.08} & 0.893 \\
w/ skinning regularization       & \multicolumn{1}{c|}{28.80} & 0.959 & \multicolumn{1}{c|}{23.14} & 0.895 \\ \midrule
Full          & \multicolumn{1}{c|}{\textbf{29.05}} & \textbf{0.961} & \multicolumn{1}{c|}{\textbf{23.27}} & \textbf{0.897} \\
\bottomrule
\end{tabular}%
}
\end{table}




\begin{figure}[tbph]
\centering
\includegraphics[width=\linewidth]{supp/figs/ablation-skinning-h36m.png}
\vspace{-1em}
\caption{\textbf{Ablation study of the skinning module on the H36M S9 subject.}}
\label{fig:abla:h36m}
\end{figure}

\begin{figure}[tbph]
\centering
\includegraphics[width=\linewidth]{supp/figs/ablation-skinning-zju.png}
\vspace{-1em}
\caption{\textbf{Ablation study of the skinning module on the ZJU-MoCap 313 subject.}}
\label{fig:abla:zju}
\end{figure}

\begin{figure}[tbph]
\centering
\includegraphics[width=\linewidth]{supp/figs/ablation-mesh-zju.png}
\vspace{-1em}
\caption{\textbf{Ablation study of the skinning module on the ZJU-MoCap 313 subject.}}
\label{fig:abla:mesh:zju}
\end{figure}



\noindent \textbf{Effect of SDF Network}
\label{supp:additional_ablations:sdf}
The MLP parametrization of the SDF field keeps our surface both water-tight and smooth, as shown in Figure~\ref{fig:abla:sdf}.

\begin{figure}[tbph]
\centering
\includegraphics[width=\linewidth]{supp/figs/ablation-sdf.png}
\vspace{-1em}
\caption{\textbf{Ablation study of SDF field parametrization.}}
\label{fig:abla:sdf}
\end{figure}
\section{More Comparisons}

\label{supp:more_comparisons}
We present full quantitative comparisons in Table~\ref{tab:full:zju:trainingpose}, Table~\ref{tab:full:h36m:trainingpose}, Table~\ref{tab:full:zju:unseenpose}, and Table~\ref{tab:full:h36m:unseenpose}.
Meanwhile, more qualitative comparisons are illustrated in Figure~\ref{fig:supp:h36m_qua}, Figure~\ref{fig:supp:zju_qua}, and Figure~\ref{fig:supp:full}.

\begin{figure*}[tbph]
\centering
\includegraphics[width=0.83\linewidth]{supp/figs/h36m.supp.png}
\vspace{-1em}
\caption{\textbf{Qualitative results of novel pose synthesis on H36M dataset.} Zoom in for a better view.}
\label{fig:supp:h36m_qua}
\end{figure*}

\begin{figure*}[tbph]
\centering
\includegraphics[width=0.825\linewidth]{supp/figs/zju.supp.png}
\vspace{-1em}
\caption{\textbf{Qualitative results of novel pose synthesis on ZJU-MoCap dataset.} ``N/A'' denotes nothing to render due to no convergence. Zoom in for a better view.}
\label{fig:supp:zju_qua}
\end{figure*}

\begin{figure*}[tbph]
\centering
\includegraphics[width=0.8\linewidth]{supp/figs/supp.full.png}
\vspace{-1em}
\caption{\textbf{Qualitative results of novel pose synthesis on H36M and ZJU-MoCap datasets with the full models.} Zoom in for a better view.}
\label{fig:supp:full}
\end{figure*}

\begin{table*}[thbp]
\centering
\caption{\textbf{Quantitative results of training pose novel view synthesis of H36M dataset}.}
\label{tab:full:h36m:trainingpose}
\resizebox{\textwidth}{!}{%
\begin{tabular}{l|cccccccccc}
\toprule
\multicolumn{1}{c|}{\multirow{3}{*}{}} & \multicolumn{10}{c}{Training pose} \\ \cline{2-11} 
\multicolumn{1}{c|}{} & \multicolumn{5}{c|}{PSNR} & \multicolumn{5}{c}{SSIM} \\ \cline{2-11} 
\multicolumn{1}{c|}{} & NB & SA-NeRF & Ani-NeRF & ARAH & \multicolumn{1}{c|}{Ours} & NB & SA-NeRF & Ani-NeRF & ARAH & Ours \\ \midrule
S1 & 22.87 & 23.71 & 22.05 & 24.45 & \multicolumn{1}{c|}{24.56} & 0.897 & 0.915 & 0.888 & 0.919 & 0.919 \\
S5 & 24.60 & 24.78 & 23.27 & 24.54 & \multicolumn{1}{c|}{24.51} & 0.917 & 0.909 & 0.892 & 0.918 & 0.920 \\
S6 & 22.82 & 23.22 & 21.13 & 24.61 & \multicolumn{1}{c|}{24.55} & 0.888 & 0.881 & 0.854 & 0.903 & 0.902 \\
S7 & 23.17 & 22.59 & 22.50 & 24.31 & \multicolumn{1}{c|}{24.05} & 0.914 & 0.905 & 0.890 & 0.919 & 0.916 \\
S8 & 21.72 & 24.55 & 22.75 & 24.02 & \multicolumn{1}{c|}{23.94} & 0.894 & 0.922 & 0.898 & 0.921 & 0.920 \\
S9 & 24.28 & 25.31 & 24.72 & 26.20 & \multicolumn{1}{c|}{25.99} & 0.910 & 0.913 & 0.908 & 0.924 & 0.919 \\
S11 & 23.70 & 25.83 & 24.55 & 25.43 & \multicolumn{1}{c|}{25.48} & 0.896 & 0.917 & 0.902 & 0.921 & 0.915 \\ \midrule
Average & 23.31 & 24.28 & 23.00 & 24.79 & \multicolumn{1}{c|}{24.72} & 0.902 & 0.909 & 0.890 & 0.918 & 0.916 \\ \bottomrule
\end{tabular}%
}
\end{table*}




\begin{table*}[thbp]
\centering
\caption{\textbf{Quantitative results of unseen pose novel view synthesis of H36M dataset}.}
\label{tab:full:h36m:unseenpose}
\resizebox{\textwidth}{!}{%
\begin{tabular}{l|cccccccccc}
\toprule
\multicolumn{1}{c|}{\multirow{3}{*}{}} & \multicolumn{10}{c}{Unseen pose} \\ \cline{2-11} 
\multicolumn{1}{c|}{} & \multicolumn{5}{c|}{PSNR} & \multicolumn{5}{c}{SSIM} \\ \cline{2-11} 
\multicolumn{1}{c|}{} & NB & SA-NeRF & Ani-NeRF & ARAH & \multicolumn{1}{c|}{Ours} & NB & SA-NeRF & Ani-NeRF & ARAH & Ours \\ \midrule
S1 & 21.93 & 22.67 & 19.96 & 23.08 & \multicolumn{1}{c|}{23.72} & 0.873 & 0.890 & 0.855 & 0.899 & 0.904 \\
S5 & 23.33 & 23.27 & 20.02 & 22.79 & \multicolumn{1}{c|}{23.13} & 0.893 & 0.881 & 0.840 & 0.890 & 0.898 \\
S6 & 23.26 & 23.23 & 23.64 & 24.04 & \multicolumn{1}{c|}{24.17} & 0.888 & 0.888 & 0.882 & 0.900 & 0.903 \\
S7 & 22.40 & 22.51 & 21.76 & 22.58 & \multicolumn{1}{c|}{22.72} & 0.888 & 0.898 & 0.869 & 0.891 & 0.889 \\
S8 & 20.78 & 23.06 & 21.63 & 22.34 & \multicolumn{1}{c|}{22.71} & 0.872 & 0.904 & 0.877 & 0.896 & 0.902 \\
S9 & 22.87 & 23.84 & 21.95 & 24.36 & \multicolumn{1}{c|}{24.54} & 0.880 & 0.889 & 0.871 & 0.894 & 0.895 \\
S11 & 23.54 & 24.19 & 22.55 & 24.78 & \multicolumn{1}{c|}{24.47} & 0.879 & 0.891 & 0.875 & 0.902 & 0.900 \\ \midrule
Average & 22.59 & 23.25 & 21.64 & 23.42 & \multicolumn{1}{c|}{23.64} & 0.882 & 0.892 & 0.867& 0.896 & 0.899 \\ \bottomrule
\end{tabular}%
}
\end{table*}


\begin{table*}[thbp]
\centering
\caption{\textbf{Quantitative results of training pose novel view synthesis of ZJU-MoCap dataset}.}
\label{tab:full:zju:trainingpose}
\resizebox{\textwidth}{!}{%
\begin{tabular}{l|cccccccccc}
\toprule
\multicolumn{1}{c|}{\multirow{3}{*}{}} & \multicolumn{10}{c}{Training pose} \\ \cline{2-11} 
\multicolumn{1}{c|}{} & \multicolumn{5}{c|}{PSNR} & \multicolumn{5}{c}{SSIM} \\ \cline{2-11} 
\multicolumn{1}{c|}{} & NB & SA-NeRF & Ani-NeRF & ARAH & \multicolumn{1}{c|}{Ours} & NB & SA-NeRF & Ani-NeRF & ARAH & Ours \\ \midrule
Twirl(313) & 30.56 & 31.32 & 29.80 & 31.60 & \multicolumn{1}{c|}{29.67} & 0.971 & 0.974 & 0.963 & 0.973 & 0.947 \\
Taichi(315) & 27.24 & 27.25 & 23.10 & 27.00 & \multicolumn{1}{c|}{24.21} & 0.962 & 0.962 & 0.917 & 0.965 & 0.919 \\
Swing1(392) & 29.44 & 29.29 & 28.00 & 29.50 & \multicolumn{1}{c|}{27.58} & 0.946 & 0.946 & 0.931 & 0.948 & 0.899 \\
Swing2(393) & 28.44 & 28.76 & 26.10 & 27.70 & \multicolumn{1}{c|}{25.91} & 0.940 & 0.941 & 0.916 & 0.940 & 0.890 \\
Swing3(394) & 27.58 & 27.50 & 27.50 & 28.90 & \multicolumn{1}{c|}{27.67} & 0.939 & 0.938 & 0.924 & 0.945 & 0.902 \\
Warmup(377) & 27.64 & 27.67 & 24.20 & 27.80 & \multicolumn{1}{c|}{26.69} & 0.951 & 0.954 & 0.925 & 0.956 & 0.926 \\
Punch1(386) & 28.60 & 28.81 & 25.60 & 29.20 & \multicolumn{1}{c|}{27.65} & 0.931 & 0.931 & 0.878 & 0.934 & 0.881 \\
Punch2(387) & 25.79 & 26.08 & 25.40 & 27.00 & \multicolumn{1}{c|}{25.68} & 0.928 & 0.929 & 0.926 & 0.945 & 0.908 \\
Kick(390) & 27.59 & 27.77 & 26.00 & 27.90 & \multicolumn{1}{c|}{24.08} & 0.926 & 0.927 & 0.912 & 0.929 & 0.840 \\ \midrule
Average & 28.10 & 26.19 & 28.27 & 28.51 & \multicolumn{1}{c|}{26.57} & 0.944 & 0.945 & 0.921 & 0.948 & 0.901 \\ \bottomrule
\end{tabular}%
}
\end{table*}



\begin{table*}[thbp]
\centering
\caption{\textbf{Quantitative results of unseen pose novel view synthesis of ZJU-MoCap dataset}.}
\label{tab:full:zju:unseenpose}
\resizebox{\textwidth}{!}{%
\begin{tabular}{l|cccccccccc}
\toprule
\multicolumn{1}{c|}{\multirow{3}{*}{}} & \multicolumn{10}{c}{Unseen pose} \\ \cline{2-11} 
\multicolumn{1}{c|}{} & \multicolumn{5}{c|}{PSNR} & \multicolumn{5}{c}{SSIM} \\ \cline{2-11} 
\multicolumn{1}{c|}{} & NB & SA-NeRF & Ani-NeRF & ARAH & \multicolumn{1}{c|}{Ours} & NB & SA-NeRF & Ani-NeRF & ARAH & Ours \\ \midrule
Twirl(313) & 23.95 & 24.33 & 22.80 & 24.40 & \multicolumn{1}{c|}{23.63} & 0.905 & 0.908 & 0.863 & 0.914 & 0.878 \\
Taichi(315) & 19.56 & 19.87 & 18.47 & 20.00 & \multicolumn{1}{c|}{20.42} & 0.852 & 0.863 & 0.795 & 0.881 & 0.850 \\
Swing1(392) & 25.76 & 26.27 & 18.44 & 26.20 & \multicolumn{1}{c|}{25.49} & 0.909 & 0.927 & 0.670 & 0.927 & 0.883 \\
Swing2(393) & 23.80 & 24.96 & 21.87 & 24.40 & \multicolumn{1}{c|}{24.31} & 0.878 & 0.900 & 0.836 & 0.915 & 0.883 \\
Swing3(394) & 23.25 & 24.24 & 17.69 & 25.20 & \multicolumn{1}{c|}{24.72} & 0.893 & 0.908 & 0.792 & 0.908 & 0.870 \\
Warmup(377) & 23.91 & 25.34 & 23.28 & 25.50 & \multicolumn{1}{c|}{24.80} & 0.909 & 0.928 & 0.901 & 0.933 & 0.894 \\
Punch1(386) & 25.68 & 27.30 & 25.55 & 27.00 & \multicolumn{1}{c|}{26.24} & 0.881 & 0.905 & 0.872 & 0.910 & 0.853 \\
Punch2(387) & 21.60 & 23.08 & 21.92 & 24.20 & \multicolumn{1}{c|}{24.06} & 0.870 & 0.890 & 0.838 & 0.917 & 0.889 \\
Kick(390) & 23.90 & 24.43 & 23.90 & 24.80 & \multicolumn{1}{c|}{25.79} & 0.870 & 0.889 & 0.887 & 0.896 & 0.873 \\ \midrule
Average & 23.49 & 24.42 & 21.55 & 24.63 & \multicolumn{1}{c|}{24.38} & 0.885 & 0.902 & 0.828 & 0.911 & 0.875 \\ \bottomrule
\end{tabular}%
}
\end{table*}



\section{Applications}
\label{supp:applications}
We showcase \textbf{relighting}, \textbf{texture editing}, and \textbf{novel poses synthesis} on AIST dataset~\cite{Li2021AICM} in Figure~\ref{fig:supp:relight}, Figure~\ref{fig:supp:texture-edited}, and Figure~\ref{fig:supp:aist} separately. 
All the above applications are presented in the supplemental video.

\begin{figure*}[tbph]
\centering
\includegraphics[width=1.0\linewidth]{supp/figs/supp.relight.pdf}
\caption{\textbf{Relighting visualization.} Zoom in for a better view. We strongly encourage our readers to view the supplemental video for a more comprehensive visual perception.}
\label{fig:supp:relight}
\end{figure*}

\begin{figure*}[tbph]
\centering
\includegraphics[width=1.0\linewidth]{supp/figs/texture-edited.pdf}
\caption{\textbf{Texture editing visualization.} Zoom in for a better view. We strongly encourage our readers to view the supplemental video for a more comprehensive visual perception.}
\label{fig:supp:texture-edited}
\end{figure*}

\begin{figure*}[tbph]
\centering
\includegraphics[width=0.9\linewidth]{supp/figs/aist.pdf}
\caption{\textbf{Extreme pose visualization.} Zoom in for a better view. We strongly encourage our readers to view the supplemental video for a more comprehensive visual perception.}
\label{fig:supp:aist}
\end{figure*}




\section{Mesh Visualizations}

\label{supp:mesh_evaluation}

We visualize the canonical mesh and present the number of faces of each mesh in Figure~\ref{fig:supp:mesh:h36m} and Figure~\ref{fig:abla:mesh:zju}.
Note that the number of faces for each mesh is quite small. 
Though increasing the resolution of tetrahedra grids may improve the details of both geometry and materials,
we do not conduct this experiment for it is orthogonal to our technical contributions.

\begin{figure*}
\centering
\includegraphics[width=0.7\linewidth]{supp/figs/mesh-h36m.png}
\caption{\textbf{Mesh visualization on the H36M dataset.} Zoom in for a better view.}
\label{fig:supp:mesh:h36m}
\end{figure*}

\begin{figure*}
\centering
\includegraphics[width=0.7\linewidth]{supp/figs/mesh-zju.png}
\caption{\textbf{Mesh visualization on the ZJU-MoCap dataset.} Zoom in for a better view.}
\label{fig:supp:mesh:zju}
\end{figure*}

\section{Limitations and Further Discussions}

\label{supp:limitation_discussions}

Our method is biased for shape-material ambiguity~\cite{wangARAHAnimatableVolume,liTAVATemplatefreeAnimatable2022,suANeRFArticulatedNeural2021,noguchiNeuralArticulatedRadiance2021,pengAnimatableNeuralRadiance2021,wangNeuSLearningNeural2021,zhangNeRFAnalyzingImproving2020}.
Taking subject 315 from ZJU-MoCap as an example, the strips in the T-shirt are modeled as ravines on the surface.
The high contrast color in the cloth surface makes our model biased for shape modeling.
That might be resolved by introducing additional surface regularizers or pre-defined parameters for the materials.

Need foreground mask to enable the mesh optimization, akin to shape-from-silhouette.
One future direction might be equipping our method with the ability to separate foreground and background automatically~\cite{jiangNeuManNeuralHuman2022,guoVid2Avatar3DAvatar2023b}.
It is also promising to model the background simultaneously during foreground subject optimization~\cite{jiangNeuManNeuralHuman2022,guoVid2Avatar3DAvatar2023b},
which eliminates the requirement of foreground mask processing.

Our method can digitize humans from visual footage, which may involve avatar misuse without the permission of the owners. 
Methods like implicit adversarial watermarks~\cite{chenFAWAFastAdversarial2021,liWatermarkingbasedDefenseAdversarial2021} that disable the neural nets inference could assist the video creation to protect their portrait rights.
Another concern is the deep fake misuse~\cite{nguyenDeepLearningDeepfakes2022}, which corrupts the identity in the visual footage rendered by our model.
Methods like deep fake detection~\cite{panDeepfakeDetectionDeep2020} could help to discover and prevent deep fake creations.
Besides, our method involves training with GPUs, which leads to carbon emissions and increasing global warming~\cite{pattersonCarbonEmissionsLarge2021}.

\end{document}
