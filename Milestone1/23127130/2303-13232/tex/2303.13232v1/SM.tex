% CVPR 2023 Paper Template
% based on the CVPR template provided by Ming-Ming Cheng (https://github.com/MCG-NKU/CVPR_Template)
% modified and extended by Stefan Roth (stefan.roth@NOSPAMtu-darmstadt.de)

\documentclass[10pt,twocolumn,letterpaper]{article}

%%%%%%%%% PAPER TYPE  - PLEASE UPDATE FOR FINAL VERSION
\usepackage{cvpr}      % To produce the REVIEW version
%\usepackage{cvpr}              % To produce the CAMERA-READY version
%\usepackage[pagenumbers]{cvpr} % To force page numbers, e.g. for an arXiv version

% Include other packages here, before hyperref.
\usepackage{graphicx}
\usepackage{amsmath}
\usepackage{amssymb}
\usepackage{booktabs}
\usepackage[ruled,vlined]{algorithm2e}
\usepackage{color}
\usepackage{amsthm}
\newtheorem{theorem}{Theorem}
\newtheorem{proposition}[theorem]{Proposition}
\newtheorem{lmm}{Theorem}
\newtheorem{rmk}{Theorem}
\newtheorem{lemma}[lmm]{Lemma}
\newtheorem{remark}[rmk]{Remark}
\setlength\parskip{0pt}
\newcommand{\PyComment}[1]{\ttfamily\textcolor{commentcolor}{\# #1}}  % add a "#" before the input text "#1"
\newcommand{\PyCode}[1]{\ttfamily\textcolor{black}{#1}} % \ttfamily is the code font
\definecolor{commentcolor}{RGB}{110,154,155}
% It is strongly recommended to use hyperref, especially for the review version.
% hyperref with option pagebackref eases the reviewers' job.
% Please disable hyperref *only* if you encounter grave issues, e.g. with the
% file validation for the camera-ready version.
%
% If you comment hyperref and then uncomment it, you should delete
% ReviewTempalte.aux before re-running LaTeX.
% (Or just hit 'q' on the first LaTeX run, let it finish, and you
%  should be clear).
\usepackage[pagebackref,breaklinks,colorlinks]{hyperref}


% Support for easy cross-referencing
\usepackage[capitalize]{cleveref}
\crefname{section}{Sec.}{Secs.}
\Crefname{section}{Section}{Sections}
\Crefname{table}{Table}{Tables}
\crefname{table}{Tab.}{Tabs.}


%%%%%%%%% PAPER ID  - PLEASE UPDATE
\def\cvprPaperID{9799} % *** Enter the CVPR Paper ID here
\def\confName{CVPR}
\def\confYear{2023}


\begin{document}
	
	%%%%%%%%% TITLE - PLEASE UPDATE
	\title{Transforming  Radiance Field with Lipschitz Network for 
		
		Photorealistic 3D Scene Stylization
  
  \textemdash\textemdash CVPR 2023 Supplementary Material}
	
	 \author{%
 	Zicheng Zhang\textsuperscript{1}%\thanks{Work done during an internship at JD AI Research.}
	\quad
	Yinglu Liu\textsuperscript{2}
	\quad
	Congying Han\textsuperscript{1}
	\quad
        Yingwei Pan\textsuperscript{2}
        \quad
	Tiande Guo\textsuperscript{1}
	\quad
	Ting Yao\textsuperscript{2}
	\quad
	\\ 
	\textsuperscript{1}University of Chinese Academy of Sciences
	\quad
	\textsuperscript{2}JD AI Research
	\quad
	\vspace{.5em} 
	\\
	\tt\small zhangzicheng19@mails.ucas.ac.cn
	\quad
	liuyinglu1@jd.com
	\quad
	hancy@ucas.ac.cn
	\\
	\tt\small
     panyw.ustc@gmail.com
	\quad
	tdguo@ucas.ac.cn
	\quad
	tingyao.ustc@gmail.com
}
	\maketitle
	\appendix

	\section{Proofs}
    \begin{proposition}\label{proposition 1}
		Considering  $f(\boldsymbol{c}) = \boldsymbol{A}\boldsymbol{c}+\boldsymbol{b}$, $\boldsymbol{A}\in \mathbb{R}^{3\times3}$, $\boldsymbol{b}\in \mathbb{R}^{3\times1}$, if $\mathbf{F}_{app}' = f\circ\mathbf{F}_{app}$, $\sum_{i=1}^{T}w_{i} = 1$ and  $vrr(\mathbf{r}_1,\mathbf{r}_2; \mathbf{F})<\epsilon$, we have  $vrr(\mathbf{r}_1,\mathbf{r}_2; \mathbf{F}')<K\epsilon$, where  $K =\left\| \boldsymbol{A} \right\|_2$ is the Lipschitz constant of $f$. 
	\end{proposition}
	
	\begin{proof} 
	\begin{small}
	\begin{equation}\label{eq: vrr}
		\begin{aligned}
			vrr(\mathbf{r}_1,\mathbf{r}_2; \mathbf{F}') &= \left \| {C}(\mathbf{r}_1;\mathbf{F}') -{C}(\mathbf{r}_2;\mathbf{F}')  \right \| \\ 
			&=\left \| \sum\limits_{i=1}^{T} w^{\mathbf{r}_1}_{i}f(\boldsymbol{c}^{\mathbf{r}_1}_i) -\sum\limits_{i=1}^{T} w^{\mathbf{r}_2}_{i}f(\boldsymbol{c}^{\mathbf{r}_2}_i)  \right \|\\
			&= \left \| \sum\limits_{i=1}^{T} w^{\mathbf{r}_1}_{i}(\boldsymbol{A}\boldsymbol{c}^{\mathbf{r}_1}_i+\boldsymbol{b}) -\sum\limits_{i=1}^{T} w^{\mathbf{r}_2}_{i}(\boldsymbol{A}\boldsymbol{c}^{\mathbf{r}_2}_i+\boldsymbol{b})  \right \| \\
			&= \left \| \sum\limits_{i=1}^{T} w^{\mathbf{r}_1}_{i}\boldsymbol{A}\boldsymbol{c}^{\mathbf{r}_1}_i -\sum\limits_{i=1}^{T} w^{\mathbf{r}_2}_{i}\boldsymbol{A}\boldsymbol{c}^{\mathbf{r}_2}_i  \right \| \\
			&= \left \| \boldsymbol{A} \left( \sum\limits_{i=1}^{T} w^{\mathbf{r}_1}_{i}\boldsymbol{c}^{\mathbf{r}_1}_i -\sum\limits_{i=1}^{T} w^{\mathbf{r}_2}_{i}\boldsymbol{c}^{\mathbf{r}_2}_i \right) \right \| \\
			&\leq \left\| \boldsymbol{A} \right\| \left \|  \sum\limits_{i=1}^{T} w^{\mathbf{r}_1}_{i}\boldsymbol{c}^{\mathbf{r}_1}_i -\sum\limits_{i=1}^{T} w^{\mathbf{r}_2}_{i}\boldsymbol{c}^{\mathbf{r}_2}_i \right \|
			\\
			&=\left\| \boldsymbol{A} \right\| vrr(\mathbf{r}_1,\mathbf{r}_2; \mathbf{F}) \\
			&< K \epsilon \notag
		\end{aligned}
	\end{equation}	
	\end{small}
	\end{proof}

	
	
	\begin{lemma}\label{lemma 1}
	Given $f=f_l \circ\cdots \circ f_1$,  $f_j(x) = \boldsymbol{A}_jx+\boldsymbol{b}$ if $j = l$ and  $\sigma(\boldsymbol{A}_{j}x)$ otherwise, where $\sigma$ is a $1$-Lipschitz function. Then $K = \Pi^l_{j=1}\left\| \boldsymbol{A}_{j} \right\|_{2} $ is the Lipschitz constant of $f$.
	\end{lemma}
	\begin{proof}
		Suppose that inputs $x$, $y$ belong to the domain of $f_{j}$, 
		\begin{equation}
			\begin{aligned}
			\left\| f_j(x) - f_{j}(y) \right\| &\leq  \left\| \sigma(\boldsymbol{A}_{j}x) - \sigma(\boldsymbol{A}_{j}y) \right\|  \\	
			 &\leq \left\|  \boldsymbol{A}_jx  - \boldsymbol{A}_jy \right\| \\
				&\leq \left\| \boldsymbol{A}_j \right\| \left\|  x  -  y \right\|.
			\end{aligned}
		\end{equation}
		When $l = 2$, the claim is clearly valid. The remaining cases can be easily proved by induction.
%		\begin{equation}
%			\begin{aligned}
%			\left\| f(x) - f(y) \right\| &\leq \left\| \boldsymbol{A}_l \right\| \left\| f^{l-1}(x) - f^{l-1}(y) \right\| \\
%			 &\leq 	\left\| \boldsymbol{A}_l \right\| \left\| f_{i}(x) - f_{i}(y) \right\| \\ 
%			 &\leq 	\left\| \boldsymbol{A}_l \right\| \left\| f_{i}(x) - f_{i}(y) \right\| \\
%%				&\leq \left\| \boldsymbol{A}_2\sigma(\boldsymbol{A}_1\boldsymbol{x}) - 		\boldsymbol{A}_2\sigma(\boldsymbol{A}_1\boldsymbol{y}) \right\| \\
%%				&\leq \left\| \boldsymbol{A}_2 \right\| \left\| \sigma(\boldsymbol{A}_1\boldsymbol{x})  - 		\sigma(\boldsymbol{A}_1\boldsymbol{y}) \right\| \\
%%				&\leq \left\| \boldsymbol{A}_2 \right\| \left\|  \boldsymbol{A}_1\boldsymbol{x}  - 	\boldsymbol{A}_1\boldsymbol{y} \right\|
%			\end{aligned}
%		\end{equation}
	\end{proof}

	\begin{proposition}\label{proposition 2}
		Considering $f=f_l \circ\cdots \circ f_1$,  $f_j(x) = \boldsymbol{A}_jx+\boldsymbol{b}$ if $j = l$ and  $\sigma(\boldsymbol{A}_{j}x)$ otherwise, where $\sigma=\max(0,x)$. If $\mathbf{F}_{app}' = f\circ\mathbf{F}_{app}$, $\sum_{i=1}^{T}w_{i} = 1$ and $\max_{i=1,\dots,T}\left\|  w^{\mathbf{r}_1}_{i}\boldsymbol{c}^{\mathbf{r}_1}_i - w^{\mathbf{r}_2}_{i}\boldsymbol{c}^{\mathbf{r}_2}_i\right\| < \epsilon/T$,  we have $vrr(\mathbf{r}_1,\mathbf{r}_2; \mathbf{F}')<K\epsilon$, where $K = \Pi^l_{j=1}\left\| \boldsymbol{A}_{j} \right\|_{2} $ is the Lipschitz constant of $f$. 
	\end{proposition}
	\begin{proof}
    \begin{small}
    Note that $\forall a\in \mathbb{R}^{+}$ and $1\leq j < l$, $af_j(x)=a\sigma(\boldsymbol{A}_{j}x)=\sigma(a\boldsymbol{A}_{j}x)=f_j(ax)$. Denoting $f^{j}=f_j \circ\cdots \circ f_1$, we can get the following derivation: 
    \begin{equation}
        \begin{aligned}
            af^{j}(x)&=a\sigma(\boldsymbol{A}_{j}f^{j-1}(x)) = \sigma(a\boldsymbol{A}_{j}f^{j-1}(x)) \\
            &=  \sigma(a\boldsymbol{A}_{j}\sigma(\boldsymbol{A}_{j-1}f^{j-2}(x))) \\
            &=  \sigma(\boldsymbol{A}_{j}\sigma(a\boldsymbol{A}_{j-1}f^{j-2}(x))) \\
            &\cdots \\
            & = f^{j}(ax).
        \end{aligned}
    \end{equation}
    Because the weights are always non-negative in the volume rendering integral,
    we further have
		\begin{equation}\label{eq: vrr}
		\begin{aligned}
			vrr(\mathbf{r}_1,\mathbf{r}_2; \mathbf{F}') &= \left \| {C}(\mathbf{r}_1;\mathbf{F}') -{C}(\mathbf{r}_2;\mathbf{F}')  \right \| \\ 
			&=\left \| \sum\limits_{i=1}^{T} w^{\mathbf{r}_1}_{i}f(\boldsymbol{c}^{\mathbf{r}_1}_i) -\sum\limits_{i=1}^{T} w^{\mathbf{r}_2}_{i}f(\boldsymbol{c}^{\mathbf{r}_2}_i)  \right \|\\
			&=\left \| \sum\limits_{i=1}^{T} w^{\mathbf{r}_1}_{i}\boldsymbol{A}_{l}f^{l-1}(\boldsymbol{c}^{\mathbf{r}_1}_i) -\sum\limits_{i=1}^{T} w^{\mathbf{r}_2}_{i}\boldsymbol{A}_{l}f^{l-1}(\boldsymbol{c}^{\mathbf{r}_2}_i)  \right \|\\
    		&=\left \| \sum\limits_{i=1}^{T} \boldsymbol{A}_{l}f^{l-1}(w^{\mathbf{r}_1}_{i}\boldsymbol{c}^{\mathbf{r}_1}_i) -\sum\limits_{i=1}^{T} \boldsymbol{A}_{l}f^{l-1}(w^{\mathbf{r}_2}_{i}\boldsymbol{c}^{\mathbf{r}_2}_i)  \right \|\\
			&\leq \sum\limits_{i=1}^{T}\left \|  \boldsymbol{A}_{l}f^{l-1}(w^{\mathbf{r}_1}_{i}\boldsymbol{c}^{\mathbf{r}_1}_i) - \boldsymbol{A}_{l}f^{l-1}(w^{\mathbf{r}_2}_{i}\boldsymbol{c}^{\mathbf{r}_2}_i)  \right \|.
			\\
		\end{aligned}
	\end{equation}	
    Based on above inequality and Lemma \ref{lemma 1}, we have
\begin{equation}\label{eq: vrr}
		\begin{aligned}
			&\ \ \ \ \left\|  \boldsymbol{A}_{l}f^{l-1}(w^{\mathbf{r}_1}_{i}\boldsymbol{c}^{\mathbf{r}_1}_i) - \boldsymbol{A}_{l}f^{l-1}(w^{\mathbf{r}_2}_{i}\boldsymbol{c}^{\mathbf{r}_2}_i) \right\|  \\
		&\leq \left\| \boldsymbol{A}_{l} \right\| \left\|  f^{l-1}(w^{\mathbf{r}_1}_{i}\boldsymbol{c}^{\mathbf{r}_1}_i) - f^{l-1}(w^{\mathbf{r}_2}_{i}\boldsymbol{c}^{\mathbf{r}_2}_i) \right\| \\
			&\leq \prod^{l}_{j=1} \left\| \boldsymbol{A}_{i} \right\| \ \left\|  w^{\mathbf{r}_1}_{i}\boldsymbol{c}^{\mathbf{r}_1}_i - w^{\mathbf{r}_2}_{i}\boldsymbol{c}^{\mathbf{r}_2}_i \right\|\\
			&= K \ \left\|  w^{\mathbf{r}_1}_{i}\boldsymbol{c}^{\mathbf{r}_1}_i - w^{\mathbf{r}_2}_{i}\boldsymbol{c}^{\mathbf{r}_2}_i \right\|.
		\end{aligned} 
	\end{equation}
	Therefore, 
	\begin{equation}\label{eq: vrr}
		\begin{aligned}
			vrr(\mathbf{r}_1,\mathbf{r}_2; \mathbf{F}') &\leq \sum\limits_{i=1}^{T} K \left\|  w^{\mathbf{r}_1}_{i}\boldsymbol{c}^{\mathbf{r}_1}_i - w^{\mathbf{r}_2}_{i}\boldsymbol{c}^{\mathbf{r}_2}_i \right\| \\
			&< K \sum\limits_{i=1}^{T} \epsilon/T = K\epsilon \\
		\end{aligned} 
	\end{equation}	
    \end{small}
	\end{proof}

		\begin{lemma} \label{lemma 2}
		$\mathbf{F}_{app}({\boldsymbol{x}, \boldsymbol{d}})$ = $\mathbf{F}_{sh}(\boldsymbol{x})\Gamma(\boldsymbol{d})+\boldsymbol{v}$, where $\Gamma(\boldsymbol{d}):\mathbb{R}^{2\times1} \rightarrow \mathbb{R}^{\ell\times1}$ is the spherical harmonic basis function, $\mathbf{F}_{sh}(\boldsymbol{x}) :\mathbb{R}^{3\times1} \rightarrow \mathbb{R}^{3\times \ell}$ is the coefficient function, and $\boldsymbol{v} \in \mathbb{R}^{3\times1}$. Given $\boldsymbol{A} \in \mathbb{R}^{3\times3}$, $\boldsymbol{b} \in \mathbb{R}^{3\times1}$, then
		$\boldsymbol{A}\mathbf{F}_{app}({\boldsymbol{x}, \boldsymbol{d}})+\boldsymbol{b} \Leftrightarrow 
		\boldsymbol{A}\mathbf{F}_{sh}({\boldsymbol{x}})+2\sqrt{\pi}[\boldsymbol{A}\boldsymbol{v}+\boldsymbol{b}-\boldsymbol{v},\boldsymbol{0}]$.
	\end{lemma}
	\begin{proof}
		\begin{equation}\label{eq: sh}
			\begin{aligned}
				\boldsymbol{A}\mathbf{F}_{app}({\boldsymbol{x}, \boldsymbol{d}})+\boldsymbol{b} &= 
				\boldsymbol{A}(\mathbf{F}_{sh}(\boldsymbol{x})\Gamma(\boldsymbol{d})+\boldsymbol{v}) +\boldsymbol{b} \\
				&= \boldsymbol{A}\mathbf{F}_{sh}(\boldsymbol{x})\Gamma(\boldsymbol{d})+\boldsymbol{A}\boldsymbol{v} +\boldsymbol{b} \\
				&= \boldsymbol{A}\mathbf{F}_{sh}(\boldsymbol{x})\Gamma(\boldsymbol{d})+\boldsymbol{A}\boldsymbol{v} +\boldsymbol{b}. \\
			\end{aligned}
		\end{equation}
		Because the first component of the spherical harmonic basis function outputs a constant value $\frac{1}{2\sqrt{\pi}}$,  we have
		\begin{equation}
			\begin{aligned}
				&(\boldsymbol{A}\mathbf{F}_{sh}({\boldsymbol{x}})+2\sqrt{\pi}[\boldsymbol{A}\boldsymbol{v}+\boldsymbol{b}-\boldsymbol{v},\boldsymbol{0}])\Gamma(\boldsymbol{d})+\boldsymbol{v}\\
				& = \boldsymbol{A}\mathbf{F}_{sh}\Gamma(\boldsymbol{d}) + \boldsymbol{A}\boldsymbol{v}+\boldsymbol{b}-\boldsymbol{v} + \boldsymbol{v} \\
				& =  \boldsymbol{A}\mathbf{F}_{sh}\Gamma(\boldsymbol{d}) + \boldsymbol{A}\boldsymbol{v}+\boldsymbol{b}
			\end{aligned}	
		\end{equation}
	\end{proof}
	\noindent\textbf{Remark of Lemma \ref{lemma 2}}. Similarly, it  can prove $\boldsymbol{A}\mathbf{F}_{sh}({\boldsymbol{x}})+[\boldsymbol{b},\boldsymbol{0}] \Leftrightarrow 
	 \boldsymbol{A}\mathbf{F}_{app}({\boldsymbol{x}, \boldsymbol{d}})+\frac{\boldsymbol{b}}{2\sqrt{\pi}} + \boldsymbol{v}-\boldsymbol{A}\boldsymbol{v}$.
	 
	 \begin{proposition}\label{proposition 3}
	 	Considering  $f(x) = \boldsymbol{A}x+\boldsymbol{b}$, $\boldsymbol{A}\in \mathbb{R}^{3\times \ell}$, $\boldsymbol{b}\in \mathbb{R}^{3\times \ell}$, if $\mathbf{F}_{sh}' = f\circ\mathbf{F}_{sh}$, $\sum_{i=1}^{T}w_{i} = 1$, $vrr(\mathbf{r}_1,\mathbf{r}_2; \mathbf{F})<\epsilon_1$ and $\left\| \Gamma(\boldsymbol{d^{\textbf{r}_{1}}}) -\Gamma(\boldsymbol{d^{\textbf{r}_{2}}}) \right\|<\epsilon_2$, we have  $vrr(\mathbf{r}_1,\mathbf{r}_2; \mathbf{F}')<K_1\epsilon_1$ + $K_2\epsilon_2$, where  $K_1 =\left\| \boldsymbol{A} \right\|_2$ and $K_2 =\left\| \boldsymbol{b} \right\|_2$. Moreover, if $\boldsymbol{b}$ vanishes except for the first column (\textit{i.e.}, the form in above remark), $vrr(\mathbf{r}_1,\mathbf{r}_2; \mathbf{F}')<K_1\epsilon_1$.
	 \end{proposition}
	
		\begin{proof} 
		\begin{small}
			\begin{equation}\label{eq: vrr}
				\begin{aligned}
					&vrr(\mathbf{r}_1,\mathbf{r}_2; \mathbf{F}')\\
					 &= \left \| {C}(\mathbf{r}_1;\mathbf{F}') -{C}(\mathbf{r}_2;\mathbf{F}')  \right \| \\ 
					&= \left \| \sum\limits_{i=1}^{T} w^{\mathbf{r}_1}_{i}\textbf{F}'(\boldsymbol{x}^{\textbf{r}_{1}}_{i})\Gamma(\boldsymbol{d^{\textbf{r}_{1}}}) -\sum\limits_{i=1}^{T} w^{\mathbf{r}_2}_{i}\textbf{F}'(\boldsymbol{x}^{\textbf{r}_{2}}_{i})\Gamma(\boldsymbol{d^{\textbf{r}_{2}}})  \right \| \\
%					&\ \ \ \ + \left \| \sum\limits_{i=1}^{T} w^{\mathbf{r}_2}_{i}\textbf{F}'(\boldsymbol{x}^{\textbf{r}_{2}}_{i})(\Gamma(\boldsymbol{d^{\textbf{r}_{2}}})- \Gamma(\boldsymbol{d^{\textbf{r}_{1}}}))  \right \| \\
					&\leq \left \| \sum\limits_{i=1}^{T} w^{\mathbf{r}_1}_{i}\boldsymbol{A}\textbf{F}(\boldsymbol{x}^{\textbf{r}_{1}}_{i})\Gamma(\boldsymbol{d^{\textbf{r}_{1}}}) -\sum\limits_{i=1}^{T} w^{\mathbf{r}_2}_{i}\boldsymbol{A}\textbf{F}(\boldsymbol{x}^{\textbf{r}_{2}}_{i})\Gamma(\boldsymbol{d^{\textbf{r}_{2}}}) \right \|  \\
					&\ \ \ \ +  \left \| \boldsymbol{b}\Gamma(\boldsymbol{d^{\textbf{r}_{1}}}) -\boldsymbol{b}\Gamma(\boldsymbol{d^{\textbf{r}_{2}}}) \right \| \\
%					&\leq  \left\| \boldsymbol{A} \right\| vrr(\mathbf{r}_1,\mathbf{r}_2; \mathbf{F})  +  \left \| \boldsymbol{b}\Gamma(\boldsymbol{d^{\textbf{r}_{1}}}) -\boldsymbol{b}\Gamma(\boldsymbol{d^{\textbf{r}_{2}}}) \right \|\\
					&\leq  \left\| \boldsymbol{A} \right\| vrr(\mathbf{r}_1,\mathbf{r}_2; \mathbf{F})  +  \left \| \boldsymbol{b}\right \| \left\| \Gamma(\boldsymbol{d^{\textbf{r}_{1}}}) -\Gamma(\boldsymbol{d^{\textbf{r}_{2}}})  \right\|\\
					&<  K_1\epsilon_1 + K_2\epsilon_2.\\
				\end{aligned}
			\end{equation}	
		If $\boldsymbol{b}$ vanishes except for the first column, $ \left \| \boldsymbol{b}\Gamma(\boldsymbol{d^{\textbf{r}_{1}}}) -\boldsymbol{b}\Gamma(\boldsymbol{d^{\textbf{r}_{2}}}) \right \| =0$, thus   $vrr(\mathbf{r}_1,\mathbf{r}_2; \mathbf{F}')<K_1\epsilon_1$.
	\end{small}
	\end{proof}

\noindent\textbf{Remarks}. Prop.~\ref{proposition 3} extends Lipschitz-constrained linear mapping in Prop.~\ref{proposition 1} from appearance representation to spherical harmonics. To prove the bound of Lipschitz MLP applied to spherical harmonics, some fussy assumptions are further required, and the proof will be trivial to repeat the above proving processes. We believe the three propositions have exhibited the intuition and importance of Lipschitz transformations for this task.
\section{More results}
For comprehensive analysis and evaluation, we have supplied a video\footnote{\url{https://www.youtube.com/watch?v=1ft8Mev3RmE}} in the supplementary materials,  which contains the continuous novel views of multiple scenes stylized with various references.  It can be observed that, both $\text{WCT}^2$ and CCPL create noises and disharmony to affect the photorealism of video. In specific, $\text{WCT}^2$ is likely to sharpen the edges excessively that produces artificial boundaries around edges (\textit{e.g.}, the trex and room scenes). It also generates noticeable noises in some stylized scenes. The results of CCPL usually have richer colors that enhances the visual effects. However, the variegated colors acceptable in a still image may be harmful to 3D scenes. For example, in the trex and fortress scenes, the interframe variations of colors results in artifacts and unconsistency of videos. In the flower scene, due to the unconsistency, the colorful leaves and flowers seem to be unrealistic and flickering. In contrast, LipRF can alleviate these downsides to generate more consistent and photorealistic stylized novel views while transferring the color style. The videos of LipRF are more like camera shots to meet the requirement of photorealistic 3D scene stylization. 


%	{\small
%		\bibliographystyle{ieee_fullname}
%		\bibliography{citation}
%	}
	

	
\end{document}
