

\begin{tikzpicture}[
        theNode/.style={
            draw,
	        minimum width=2cm,
	        minimum height=0.5cm,
	        align=center
        },
        font=\small,
        thick
    ]
    
	\node[theNode] (block1) {traffic flow models};
	\node[theNode, below  left=of block1 ] (block9)  {model-driven};
	\node[theNode, below  left=of block9 ] (block2)  {differential};
	\node[theNode, below right=of block9 ] (block10) {nondifferential};
    \node[theNode, below right=of block1 ] (block3)  {data-driven};
	\node[theNode, below  left=of block2 ] (block4)  {micro};
	\node[theNode, below      =of block2 ] (block5)  {meso};
	\node[theNode, below right=of block2 ] (block6)  {macro};
   % \node[theNode, below      =of block10] (block11) {cellular\\automata};
	\node[theNode,       right=of block10] (block7)  {parametric};
	\node[theNode, below right=of block3 ] (block8)  {nonparametric};

	\draw[-latex] (block1) -- (block9);
	\draw[-latex] (block9) -- (block2);
	\draw[-latex] (block1) -- (block3);
	\draw[-latex] (block2) -- (block4);
	\draw[-latex] (block2) -- (block5);
	\draw[-latex] (block2) -- (block6);
	\draw[-latex] (block3) -- (block7);
	\draw[-latex] (block3) -- (block8);
	\draw[-latex] (block9) -- (block10);
%	\draw[-latex] (block10) -- (block11);
	%\draw[-latex] (block2) -| (block3) node[pos=0.75,fill=white,inner sep=2]{yes};
	%\draw[-latex] (block3) edge node[]{}(block7);

\end{tikzpicture}         
%https://latexdraw.com/draw-flowcharts-latex-tutorial/
%\caption{Structure of pedestrian modeling research lines.}
%\label{fig:diagram}
