\appendix

\section{Transferring results from causal team semantics}\label{Appendix: transferring results}

In this section we provide tools for transferring results from causal team semantics to causal multiteam semantics. We formulate the results for $\CO$, although the methods extend to other logics considered in the literature (e.g. extensions of $\CO$ with dependence atoms and/or with the global disjunction).
 
%We write $\CO$ for the usual language (including the \emph{lax} tensor!) and $\COU, \COD, \COD_\sqcup$ for the various extensions introduced in the first part of the paper. \commf{We might want to remove references to the extended languages at some point. They don't play a crucial role in this second paper.}
%\corrf{Let us introduce some notations. We will follow the conventions of \cite{BarYan2022} and treat a causal (multi)team $T$ as a pair $(T^-,\F)$, where the function component $\F$ also encodes information about the parent sets of the endogenous variables. That is, $\F$ associates to each endogenous variable $V\in En(\F)$ both a set of parent variables $\PA_V^\F$ and a function $\F_V : \ran(\PA_V^\F) \rightarrow \ran(V)$.}
For most purposes, we can think of a causal team (of signature $\sigma$) as a pair $(T^-,\F)$, where $T^-$ is a team instead of a multiteam (i.e., a set of assignments on $Dom$ instead of $Dom \cup \{Key\}$), satisfying the conditions given in Definition \ref{def: causal multiteam}. In previous papers, the definition of causal teams differs in the sense that one can have distinct function components which differ from each other just in the sense that some of their functions have different sets of dummy argument. Such function components were dubbed to be \emph{equivalent} in \cite{BarYan2022}, and a notion of causal team equivalence was derived from it. In our framework, equivalence just coincides with equality. Lemma \ref{lemma: transfer} below will show that these differences of detail do not constitute a serious obstacle for transferring results from the framework of \cite{BarYan2022} to our causal multiteam semantics (in particular, for deriving Theorem \ref{thm:COchar}) from \cite[Theorem 4.4]{BarYan2022}. %I have heavily reformulated this because I don't see what exactly means for two semantics to coincide. Lemma \ref{lemma: transfer} is more like an embedding result, and the situation is somewhat subtle -- we have an equivalence (of function components) on the side of causal teams, but the fact that we look at $\CO$ formulas also induces an equivalence (of support) on the side of causal multiteams.
%Hence, for obtaining the result of Theorem \ref{thm:COchar} from \cite[Theorem 4.4]{BarYan2022}, it suffices to prove that for $\CO$ causal multiteam semantics and causal team semantics coincide (that is Lemma \ref{lemma: transfer} below).
%In our framework, such distinctions are trivial, and we will need to remove references to them in transferring results between the two frameworks.

We write $\models$ for the satisfaction relation over causal multiteams, and $\models^{ct}$ for the satisfaction relation over causal teams (when there is a need to make the distinction). The constructions $T^\alpha$ and $T_{\SET X = \SET x}$ are defined analogously as in the multiteam case.
The satisfaction clauses of the causal team semantics for language $\CO$, as given in previous literature, are formally identical to those of causal multiteam semantics, with the exception that the symbol $\lor$ is interpreted a \emph{lax} tensor, that is:

\begin{center}
$T\models^{ct} \psi\lor \chi$ if there are two causal subteams $T_1,T_2$ of $T$ such that $T_1^-\cup T_2^- = T^-$, $T_1\models^{ct} \psi$ and $T_2\models^{ct} \chi$
\end{center}
i.e. the causal subteams $T_1,T_2$ are not required to be disjoint.


%We write $\models$ for the satisfaction relation over causal multiteams, and $\models^{ct}$ for the satisfaction relation over causal teams (when there is need to make the distinction). The satisfaction clauses for causal team semantics are formally identical 
%The two notions differ only for what regards the clause for tensor disjunction, which is strict for the former, lax for the latter:

%\begin{center}
%$T\models \psi\lor \chi$ if there are two \emph{disjoint} causal sub-multiteams $T_1,T_2$ of $T$ such that $T_1^-\cup T_2^- = T^-$, $T_1\models \psi$ and $T_2\models \chi$.
%\end{center}

%\begin{center}
%$T\models^{ct} \psi\lor \chi$ if there are two causal sub-multiteams $T_1,T_2$ of $T$ such that $T_1^-\cup T_2^- = T^-$, $T_1\models^{ct} \psi$ and $T_2\models^{ct} \chi$.
%\end{center}


Remember that, given a multiteam $T^-$ of signature $\sigma=(\dom,\ran)$, there is a corresponding team $\Team(T^-):= \{s_{\upharpoonright \dom} \mid s\in T^-\}$ of signature $\sigma$. More generally, given a causal multiteam  $T = (T^-,\F)$ of signature $\sigma$, there is a corresponding causal team $\Team(T) = (\Team(T^-), \F)$.

The following key lemma allows to translate results between causal team semantics and causal \emph{multi}team semantics. %, and the present section is devoted to its proof.

\begin{lemma}\label{lemma: transfer}
Let $T$ be a causal multiteam of signature $\sigma$, and $\varphi\in\CO$. Then:
\[
T\models\varphi \iff \mathrm{\Team}(T)\models^{ct} \varphi
\]
\end{lemma}


%\commf{What follows in this section is boring and should be postponed to an appendix. However, we learn something interesting from it.}
\noindent Towards the proof of this result, we introduce three further lemmas.

\begin{lemma}\label{lemma: cm equals ct on singletons}
Let $\sigma=(\dom,\ran)$ be a signature. Let $s\in\B_\sigma$. % $s:\dom\rightarrow \bigcup_{X\in \dom}\ran(X)$ be such that, for all $X\in \dom$, $s(X)\in \ran(X)$.
 Then for all $n\in\mathbb N$, %$G$ graph over $\dom$,
 $\F$ function component over $\dom$ and %$\varphi\in \CODU_{\sigma}$
 $\varphi\in \CO_{\sigma}$ we have:
\[
(\{s(n/Key)\},  \F)\models \varphi \iff (\{s\},  \F)\models^{ct} \varphi.
\]
\end{lemma}

\begin{proof}
A straightforward proof by induction on $\varphi$, hinging on the fact that the variable $Key$ is not used in $\CO_{\sigma}$. In the case for $\lor$, we use the fact that, over a singleton causal (multi)team $T$, $T\models \psi_1\lor\psi_2\iff T\models \psi_1$ or $T\models \psi_2$. 
\end{proof}


\begin{lemma}\label{lemma: transfer alpha}
Let $T$ be a causal multiteam of signature $\sigma$, and $\alpha\in\CO_{\sigma}$. Then $\Team(T^\alpha)= \Team(T)^\alpha$.
\end{lemma}

\begin{proof} 
Write $T=(T^-,\F)$. 
Then
\begin{align*}
\Team(T^\alpha) &= \Team((\{s\in T^- \mid (\{s\},\F)\models\alpha\},\F))\\
& = (\{s_{\upharpoonright \dom} \mid s \in T^- \text{ and } (\{s\},\F)\models\alpha\},\F)\\
& = (\{s_{\upharpoonright \dom} \mid s \in T^- \text{ and } (\{s_{\upharpoonright \dom}\},\F)\models^{ct}\alpha\},\F)\\
& = (\{s_{\upharpoonright \dom} \in T^-\},\F)^\alpha\\
& = \Team((T^-,\F))^\alpha\\
& = \Team(T)^\alpha,
\end{align*}
where in the third equality we used Lemma \ref{lemma: cm equals ct on singletons}.
%\corrf{Then $T^\alpha = ((T^\alpha)^-,\F)$, $\Team(T^\alpha) = (\Team((T^\alpha)^-),\F)$, $\Team(T)=(\Team(T^-),\F)$ and $\Team(T)^\alpha=(\Team(T^-)^\alpha,\F)$. We then just need to prove that $\Team((T^\alpha)^-) = \Team(T^-)^\alpha$.}
%\commf{This first paragraph is very pedantic, but I have noticed that it's easy to misuse these indexes and notations. Indeed there were many notational errors in this and the next lemma.}
%Let $s\in \Team((T^\alpha)^-)$. Then there is an $n\in\mathbb N$ such that $s(n/Key)\in (T^\alpha)^-$. So, $s(n/Key)\in T^-$ and $(\{s(n/Key)\},\F)\models\alpha$. The former entails that $s\in \Team(T^-)$; the latter, by Lemma \ref{lemma: cm equals ct on singletons}, that $(\{s\},\F)\models\alpha$. So $s\in \Team(T^-)^\alpha$.
%In the opposite direction, let $s\in \Team(T^-)^\alpha$. So $(\{s\},\F)\models\alpha$, and $s(n/Key)\in T^-$ for some $n\in\mathbb N$. By Lemma \ref{lemma: cm equals ct on singletons} we have  $(\{s(n/Key)\},\F)\models\alpha$. So  $s(n/Key)\in (T^\alpha)^-$; thus $s\in \Team((T^\alpha)^-)$. 
\end{proof}


\begin{lemma}\label{lemma: transfer X x}
Let $T$ be a causal multiteam of signature $\sigma=(\dom,\ran)$, $\SET X\subseteq \dom$ and $\SET x\in \ran(\SET X)$. Then $\Team(T_{\SET X = \SET x}) = \Team(T)_{\SET X = \SET x}$.
\end{lemma}

\begin{proof}
Write $T=(T^-,\F)$. 
Then 
\begin{align*}
\Team(T_{\SET X = \SET x}) 
&= (\{t_{\upharpoonright \dom} \mid t\in T_{\SET X = \SET x}^-\}, \F)\\
&= (\{t_{\upharpoonright \dom} \mid \exists s\in T^- \text{ such that } t = s_{\SET X = \SET x}^\F\}, \F)\\
&= (\{(s_{\upharpoonright \dom})_{\SET X = \SET x}^\F \mid s\in T^-\}, \F)\\
&= (\{s_{\upharpoonright \dom} \mid s\in T^-\}, \F)_{\SET X = \SET x}\\
&= \Team(T)_{\SET X = \SET x},
\end{align*}
where in the third equality we used the fact that $(s_{\SET X = \SET x}^\F)_{\upharpoonright \dom} = (s_{\upharpoonright \dom})_{\SET X = \SET x}^\F$.
%Write $T=(T^-,\F)$. \corrf{Then we have $T_{\SET X = \SET x} =((T_{\SET X = \SET x})^-,\F_{\SET X = \SET x})$, $\Team(T_{\SET X = \SET x})=(\Team((T_{\SET X = \SET x})^-),\F_{\SET X = \SET x})$, $\Team(T)=(\Team(T^-),\F)$ and finally $\Team(T)_{\SET X = \SET x}=(\Team(T^-)_{\SET X = \SET x},\F_{\SET X = \SET x})$. Thus, we only have to show that  $\Team((T_{\SET X = \SET x})^-) = \Team(T^-)_{\SET X = \SET x}$.}
%Let $s\in \Team((T_{\SET X = \SET x})^-)$.  Then there is an $n\in\mathbb N$ such that $s(n/Key)\in (T_{\SET X = \SET x})^-$; but then there is a $t\in T^-$ such that $t^{\F}_{\SET X = \SET x}= s(n/Key)$. Then $t_{\upharpoonright \dom}\in \Team(T^-)$ and $(t_{\upharpoonright \dom})_{\SET X = \SET x}=s$.  So $s\in \Team(T^-)_{\SET X = \SET x}$.
%In the other direction, let $s\in \Team(T^-)_{\SET X = \SET x}$. Then there is a $t\in \Team(T^-)$ such that $t^\F_{\SET X = \SET x}=s$. Since $t\in \Team(T)^-$, there is an $n\in\mathbb N$ such that $t(n/Key)\in T^-$. So $t(n/Key)^\F_{\SET X = \SET x}\in (T_{\SET X = \SET x})^-$. Notice furthermore that $t(n/Key)^\F_{\SET X = \SET x}= t^\F_{\SET X = \SET x}(n/Key)$; so $t^\F_{\SET X = \SET x}\in \Team((T_{\SET X = \SET x})^-)$, i.e., $s\in \Team((T_{\SET X = \SET x})^-)$.
\end{proof}

\noindent We can now prove the key lemma. % main result of this section.




\begin{proof}[Proof of Lemma \ref{lemma: transfer}]
By induction on $\varphi$. The case for $\land$ % and $\sqcup$ are 
is straightforward.

\begin{itemize}
\item Base case: $\varphi$ is $X=x$. (The case for $X\neq x$ is completely analogous.)

$\Rightarrow$) Suppose $T\models X=x$. Then, for every $s\in T^-$, $s(X)=x$; therefore $s_{\upharpoonright \dom}(X)=x$. Since every assignment in $\Team(T^-)$ is of the form $s_{\upharpoonright \dom}$ for some $s\in T^-$, this amounts to saying that, for all $t\in \Team(T^-), t(X)=x$; i.e., $\Team(T)\models^{ct} X=x$.

$\Leftarrow$) Suppose $\Team(T)\models^{ct} X=x$, i.e., for all $t\in \Team(T)^-, t(X)=x$. For each $n\in \mathbb N$, write $s^n_t$ for the assignment $t(n/Key)$. Then $s^n_t(X)=x$. Now $T^-\subseteq \{s^n_t \mid t\in \Team(T)^-, n \in \mathbb N\}$; therefore $T\models X=x$.

%\item Base case: $\varphi$ is $X\neq x$. Completely analogous to the previous case.

%\item Base case: $\varphi$ is $\dep{\SET X}{Y}$.

%Assume $T\models \dep{\SET X}{Y}$. Then, for all $s,s'\in T^-$, $s(X)=s'(X)$ implies $s(Y)=s'(Y)$. Thus the same holds for all pairs $s_{\upharpoonright \dom},s'_{\upharpoonright \dom}$, i.e., for all $t,t'\in \Team(T)^-$. So $\Team(T)\models^{ct} \dep{\SET X}{Y}$.

%The opposite direction is analogous. %, using the fact that, if $s(n/Key)(X)=s'(n/Key)(X)$ implies $s(n/Key)(Y)=s'(n/Key)(Y)$, then the same holds for all other $m\in \mathbb N$.   


\item Case $\varphi$ is $\psi_1\lor \psi_2$.

$\Rightarrow$) If $T=(T^-,\F)\models\psi_1\lor \psi_2$ then there are disjoint multiteams $S_1,S_2\subseteq T^-$ such that %$S_1\cap S_2= \emptyset$,
 $S_1 \cup S_2 = T^-$ and $(S_i,\F)\models \psi_i$. Now define teams $S_i^*:=\{ s_{\upharpoonright \dom} \mid s\in S_i\}= \Team(S_i)$. By the inductive hypothesis, $(S_i^*,\F)\models^{ct} \psi_i$. Furthermore, if $t\in \Team(T^-)$, then $t=s_{\upharpoonright \dom}$ for an $s$ that belongs to either $S_1$ or $S_2$; so $t\in S_1^*$ or  $t\in S_2^*$. That is, $S_1^* \cup S_2^* = \Team(T^-)$. Thus $\Team(T)\models^{ct} \psi_1\lor \psi_2$. %\commf{Let me point out that it would have been impossible to prove that $S_1^*\cap S_2^*=\emptyset$; it is false in general. Thus, oddly enough, to have a straightforward correspondence between multiteams and teams, we need the \emph{lax} clause for tensor in the case of teams. We could have done also with the strict clause, by defining $S_2^*:= \Team(T)^-\setminus S_1^*$ and then using downward closure.}

$\Leftarrow$) Assume $\Team(T)\models^{ct} \psi_1\lor \psi_2$. Then there are teams $S_1^*,S_2^*\subseteq \Team(T^-)$ such that $S_1^* \cup S_2^* = \Team(T^-)$ and $S_i^*\models^{ct} \psi_i$. Now define the disjoint multiteams $S_1:=\{s\in T^- \mid s_{\upharpoonright \dom}\in S_1^*\}$ and 
$S_2:=T^-\setminus S_1$.
%\corrf{$S_2:=\{s\in T^- \mid s_{\upharpoonright \dom}\in S_2^*\}$}.
 Obviously then %$S_1\cap S_2= \emptyset$,
  $S_1 \cup S_2 = T^-$ and $S_1^* = \Team(S_1)$, and by the inductive hypothesis $S_1\models\psi_1$. %***From the latter we have, by inductive hypothesis, that $S_i\models \psi_i$. Thus, $T\models \psi_1 \lor \psi_2$.  ****

%******

Let us show that $S_2 \subseteq \{s\in T^- \mid s_{\upharpoonright \dom}\in S_2^*\}$.  If $s\in S_2$, then  $s\notin S_1$; thus, $s_{\upharpoonright \dom}\notin S_1^*$. But then, since $t_{\upharpoonright \dom}\in \Team(T^-)$, $s_{\upharpoonright \dom}\in S_2^*$, as claimed. 

Now, since $S_2^*\models \psi_2$, by inductive hypothesis $\{s\in T^- \mid s_{\upharpoonright \dom}\in S_2^*\}\models \psi_2$; and since $S_2 \subseteq \{s\in T^- \mid s_{\upharpoonright \dom}\in S_2^*\}$, by the downward closure of $\CO$ we have $S_2\models \psi_2$. Thus in conclusion $T\models \psi_1 \lor \psi_2$.


%******

%Let us show that $S_2^* \subseteq \Team(S_2)$. If $t\in S_2$, then  $t\notin S_1$; thus, for all $n\in \mathbb N$, $t(n/Key)\notin S_1$. However, since $t \in \Team(T)^-$, there must be an $n\in\mathbb N$ such that $t(n/Key)\in T^-$. But then $t(n/Key)\in S_2$. Thus $t\in \Team(S_2)$.

%By the inductive hypothesis we have $\Team(S_2)\models \psi_2$; since $S_2^* \subseteq \Team(S_2)$, by downward closure of $\CO$ we have $S_2^*\models\psi_2$. \commf{Notice the use of downward closure.} Putting everything together, $\Team(T)\models\psi_1\lor\psi_2$.

\item Case $\varphi$ is $\alpha\supset \psi$.

 $T\models \alpha\supset \psi \iff T^\alpha\models\psi \iff$ (by the inductive hypothesis) $\Team(T^\alpha)\models^{ct} \psi \iff$ (by Lemma \ref{lemma: transfer alpha}) $\Team(T)^\alpha\models^{ct} \psi \iff \Team(T)\models^{ct} \alpha\supset\psi$.


\item Case $\varphi$ is $\SET X = \SET x \cf \psi$.

$T\models\SET X = \SET x \cf \psi \iff T_{\SET X = \SET x}\models  \psi \iff$ (by inductive hypothesis) $\Team(T_{\SET X = \SET x})\models^{ct}  \psi\iff$ (by Lemma \ref{lemma: transfer X x}) $\Team(T)_{\SET X = \SET x}\models^{ct}  \psi\iff \Team(T)\models^{ct} \SET X = \SET x \cf \psi$.
\end{itemize}
\end{proof}


\noindent This result can be easily extended to formulas with $\sqcup$ of $\dep{\SET X}{Y}$. Notice also that the case for $\lor$ may fail in languages that are not downward closed. %(such as $\PCO$).
%%%%%%%%%%%%%%%%%%%%%%%%%%%%%%%%%%%%%%%%%%%%
%\section{Expressive power of non-probabilistic languages} \label{Appendix: expressivity of CO}
%We begin in this section an analysis the expressivity of various languages a evaluated in causal multiteam semantics. We will consider first the case of non-probabilistic languages, such as $\PCO$.
%One advantage of team semantics and its variants, witnessed by a vast literature (e.g. \cite{KonVaa2011}, \cite{Gal2012},\cite{GalHel2013}) is the possibility to find exact characterizations of the expressive power of languages in terms of closure properties of the definable classes of teams. Such kinds of results often lead to the separation of languages or help in proving the undefinability of logical operators. 
We will give two applications of lemma \ref{lemma: transfer}, which were already mentioned in the main text. First, we will show that the formulas $\Phi^\F$ characterize the property of ``having function component $\F$'' in causal multiteam semantics. Secondly, we will semantically characterize $\CO$ over causal multiteams.

Let us begin considering the issue of the $\Phi^\F$ formulas (as given in section \ref{subs: PO and PC}). The paper \cite{BarYan2022} used a slightly different semantics, in which there may exist causal functions $F_V, G_V$ that only differ for their set of dummy arguments. %\commf{Maybe now this discussion of dummy arguments is redundant, we have something in the main text.} *****
 For example, the functions $F_V(U,X):= U +X -X$ and $G_V(U,Z):= U +Z -Z$ have different argument variables, but they produce the same values of $V$ for each given value of $U$; $X$ is a dummy argument for $F_X$ and $Z$ is a dummy argument for $\G_X$. In such a case, $F_V$ and $G_V$ coincide over $\PA_V^F \cap \PA_V^G$, while the variables $\PA_V^F \setminus \PA_V^G$ are dummy arguments for $F_V$, and $\PA_V^G \setminus \PA_V^F$ are dummy arguments for $G_V$. When this happens we say that $F_V$ and $G_V$ are \textbf{similar}, and we write $F_V\sim G_V$. Notice furthermore that  each such function $F_V$ (possibly with dummy argument)  is similar to a (unique) \textbf{minimal canonical representative} $f_V$ - a function with no dummy arguments; and also similar to a (unique) \textbf{maximal canonical representative} $\F_V$ - a function whose arguments are all the variables in $\dom\setminus\{V\}$. The latter are just the kinds of functions we defined earlier in this text.

The notion of similarity is then extended to function components as follows. We write $F,G$ for  function components in the sense of \cite{BarYan2022}. 
%Write $End(F)$ for the set of endogenous variables of $F$ and $Cn(F)$ for the set of endogenous  variables $V$ of $F$ such that $F_V$ is a constant function.
%We say that $\F$ and $\G$ are similar ($\F\sim\G$) if $En(\F)\setminus Cn(\F) = En(\G)\setminus Cn(\G)$ and, for all $V\in En(\F)\setminus Cn(\F)$, $\F_V\sim\G_V$. %\commf{These definitions look unneededly complicated, but with our particular formalization of causal teams, it seems necessary to formulate them like this.}
Write $End(F)$ for the set of endogenous variables of $F$.\footnote{By the conventions of this paper, if a variable is in $End(F)$ then it is generated by a \emph{non-constant} causal function.}.
We say that $F$ and $G$ are similar ($F\sim G$) if $End(F) = End(G)$ and, for all $V\in End(F)$, $F_V\sim G_V$.\footnote{This definition looks simpler than that in \cite{BarYan2022} due to our convention that causal functions must be non-constant.} %\commf{These definitions look unneededly complicated, but with our particular formalization of causal teams, it seems necessary to formulate them like this.}
 Finally, we say that two causal (multi)teams $S=(S^-,F), T=(T^-,G)$ are \textbf{equivalent} ($S\approx T$) iff $S^-=T^-$ and $F\sim G$.

We can now show the correctness of the $\Phi^\F$ formulas.

\begin{theorem}\label{thm: phif}
Let $T=(T^-,\G)$ be a \emph{nonempty} causal multiteam. Then,
\[
T\models \Phi^\F \iff \G=\F.
\]
\end{theorem}

\begin{proof}
In \cite{BarYan2022}, Theorem 3.4, it it proved that if $S = (S^-,G)$ is a causal team of signature $\sigma$, then $S\models \Phi^F  \iff G \sim F$. In our case, $T\models \Phi^\F$ iff $Team(T)\models^{ct} \Phi^\F$ (by lemma \ref{lemma: transfer}), thus, by the theorem in \cite{BarYan2022}, iff $\G\sim\F$. But since $\F$ and $\G$ are both maximal canonical representatives, it must be $\F=\G$. Vice versa, trivially $\F=\G$ implies $\G\sim\F$, and then we can use the same equivalences as before, in the opposite direction.
%
%the same statement is proved for causal teams; this yields that $Team(T) \models^{ct} \Phi^\F$ iff $\F\sim\G$, where $\models^{ct}$ denotes the semantic relation over causal teams. On the other hand, lemma \ref{lemma: transfer} tells us that $T\models \Phi^\F$ iff $Team(T) \models^{ct} \Phi^\F$.
\end{proof}

As mentioned above, the language $\CO$ and its extensions including $\sqcup$ and dependence atoms received semantic characterizations in causal team semantics \cite{BarYan2022}. Lemma \ref{lemma: transfer}  allows us to convert these results (for non-probabilistic languages) into characterizations in causal multiteam semantics. We consider here only the case of language $\CO$, whose expressive power will be seen to be characterized by the property of flatness alone (Theorem \ref{thm:COchar}). 



%As we know, the formulas of $\CO_\sigma$ are flat. We can also state flatness as a property of classes of causal multiteams, as follows:

%\begin{itemize}
%\item $\K$ is \textbf{flat} if $T= (T^-,\F)\in\K$ iff for every $s\in T^-$, $(\{s\},\F)\in \K$.\footnote{For teams and for the generalized causal teams of \cite{BarYan2022}, this property is equivalent to the conjunction of empty multiteam property, downward closure and union closure. For causal (multi)teams the situation is more subtle, since the union of two causal (multi)teams is often not well defined. See \cite{BarYan2022} for details.}
%\end{itemize}

In general, a key property of non-probabilistic languages %$\CO$ formulas, which essentially states that they are non-probabilistic in nature, 
 is support-closedness:  %. We can express it as a property of classes of causal \emph{multi}teams only:
\begin{itemize}
\item $\K$ is \textbf{support-closed} if, whenever $T\in\K$ and $\Team(T)=\Team(S)$, then $S\in \K$.
\end{itemize}
 
\noindent A moment of thought shows that support-closedness logically follows from flatness, and thus we did not need to mention the former in the semantic characterization of language $\CO$. It would naturally appear in characterization theorems for more general languages. %It turns out that these two conditions suffice to characterize the expressive power of $\CO$ formulas over causal multiteams, in the following sense.
%\begin{theorem}[Characterization of $\CO$]\label{thm: characterizing CO}
%Let $\sigma$ be a finite signature, and $\K$ a class of causal multiteams of signature $\sigma$. Then $\K$ is definable by a $\CO_{\sigma}$ formula (resp. a set of $\CO_{\sigma}$ formulas) if and only if $\K$ is flat.
%\end{theorem}
%By the same methods, 
For example, it can be proved that the language that extends $\CO$ with dependence atoms (denoted $\COD$ in previous literature) is characterized by the nonempty multiteam property, downward closure and support closure; we omit the similar proof.\footnote{In this case, support closure is not entailed by the other two closure conditions.}


%We now use the tools from appendix \ref{Appendix: transferring results} in order to 
%Here we provide a characterization of the expressive power of $\CO$ over causal multiteams.
Let us work towards a proof of Theorem \ref{thm:COchar}. %The characterization is derived from an analogous result for causal team semantics (Theorem 4.4 from \cite{BarYan2022}) using the tools developed in appendix \ref{Appendix: transferring results}. %\commf{I had also analogous, unsurprising results for languages with dependence atoms and global disjunction. I commented them out. We can restore them if you prefer.}


% want to characterize the classes of causal multiteams that are definable by the various non-probabilistic languages. These classes will be characterized by a few closure properties. One of these closure properties is specific of our definition of causal multiteams, and it concerns the possible presence of dummy arguments for functions; we need to introduce some notation to discuss this point.



%\commf{Commented out a list of closure properties.}
%We are now ready to discuss some closure properties for classes of causal (multi)teams. Let $\K$ be a class of causal multiteams of a signature $\sigma$.

%\begin{itemize}
%\item $\K$ is \textbf{closed under equivalence} if $T\in\K$ and $T\approx S$ imply $S\in\K$.

%\item $\K$ is \textbf{(causally) downward closed} if, whenever $T\in\K$ and $S$ is a causal sub(multi)team of $T$, then $S\in\K$.

%\item $\K$ is \textbf{flat} if $T= (T^-,\F)\in\K$ iff for every $s\in T^-$, $(\{s\},\F)\in \K$.

%\end{itemize}

%\commf{I omit a discussion of union closure because it is complex and not strictly necessary for what follows. See section 4.1 of my paper with Fan Yang, if you want.}

%We point out that, if $\K$ is flat, then it is also 1) downward closed, 2) it contains all causal multiteams (of signature $\sigma$) with empty team component, and 3) it is closed under unions of causal teams, whenever these unions are well-defined (see \cite{BarYan2022}).


%The next closure property is for causal \emph{multi}teams only:

%\begin{itemize}
%\item $\K$ is \textbf{support-closed} if, whenever $T\in\K$ and $\Team(T)=\Team(S)$, then $S\in \K$.
%\end{itemize}

%All the non-probabilistic languages considered here are 

%\noindent The language $\CO_\sigma$ is support-closed.


Given a class $\K$ of causal multiteams, let $\Team(\K):= \{ \Team(T) \mid T\in \K\}$; and given a class $\K$ of causal multiteams of a common signature $\sigma$, let $\Team_\sigma^\approx(\K) = \{S \text{ of signature $\sigma$ } \mid \exists T\in\K : S\approx \Team(T)\}$. Let us see what kind of closure properties are preserved when passing from $\K$ to %$\Team(\K)$ or
$\Team_\sigma^\approx(\K)$.

\begin{lemma}\label{lemma: transferable closure properties}
%Let $\mathbb P$ be either %closure under equivalence,
% downward closure or flatness. %\commf{commented out case for closure under equivalence.} 
% Let $\K$ be a \corrf{support-closed} 
%  class of causal multiteams of a common signature $\sigma$. If $\K$ has property $\mathbb P$, then $\Team_\sigma^\approx(\K)$ has the corresponding property for classes of causal teams. 
Let $\K$ be a class of causal multiteams of a common signature $\sigma$.

1) If $K$ is flat, then $\Team_\sigma^\approx(\K)$ is flat.

2) If $K$ is downward closed and support closed, then  $\Team_\sigma^\approx(\K)$ is downward closed.  
  %The assumption of support-closure is not needed for flatness.
% \commf{Having only two properties, we can reformulate this statement...}
 
% If $\K$ is a support-closed class of causal multiteams of a common signature $\sigma$, then   the same statements hold for $\Team_\sigma^\approx(\K)$. % of causal multiteams of signature $\sigma$ that are equivalent to teams from $\Team(\K)$ (in the sense of differing only for the presence of dummy arguments in the function components).
\end{lemma}

\begin{proof}
%\begin{itemize}
%\item Case $\mathbb{P}$ is flatness. 
% Now assume $T_*=(T_*^-,\F)\in \Team(\K)$, and let $s\in T_*^-$. There is a $T=(T^-,\F)\in\K$ such that $T_*=\Team(T) $. But then there is an $n\in\mathbb N$ such that $s(n/Key)\in T^-$. By flatness of $\K$, then, $(\{s(n/Key)\},  \F)\in\K$. But then $\Team((\{s(n/Key)\},  \F))= (\{s\},  \F)\in \Team(\K)$, as required.

%In the opposite direction, suppose that, for all $s\in T_*^-$, $(\{s\},  \F)\in \Team(\K)$. Then there is a sequence $(n_s)_{s\in T_*^-}$ of natural numbers such that $(\{s(n_s/Key)\},  \F)\in \K$ for each $s\in T_*^-$. Take $T^-:= \{s(n_s/Key) \mid s\in T_*^-\}$ and let $T:=(T^-,\F)$. Since $\K$ is flat, $T\in \K$. Then $\Team(T)\in \Team(\K)$. But $\Team(T^-)= \{s\in T_*^-\}=T_*^-$; so  $\Team(T)=T_*$, and thus $T_*\in \Team(\K)$.   
%\end{itemize}

%\vspace{5pt}

%\noindent Now again let $T_*\in \Team(\K)$. Then there is a $T\in\K$ such that $T_*=\Team(T)$.

%\begin{itemize}
%\item Case $\mathbb{P}$ is closure under equivalence. Let $S_*=(S_*^-,\F)\approx T_*=(S_*^-,\G)$ be causal teams; we will show that $S_*\in\K$. There is an $S$ such that $S_*=\Team(S)$. Write $S=(S^-,\F)$, and $S'=(S^-,\G)$; clearly $S\approx S'$. 

%We want to show that $\Team(S')=\Team(T)$; it suffices to show that $\Team(S^-)=\Team(T^-)$. But this holds, since we know that both are equal to $S_*^-$.

%Now since $\Team(S')=\Team(T)$ and $T\in\K$, and since $\K$ is support closed, we have $S'\in\K$. Since $S\approx S'$, and since $\K$ is closed under equivalence, we have $S\in\K$. Thus $\Team(S)=S_*\in \Team(\K)$, as required.

%\item
% Case $\mathbb{P}$ is downward closure. Suppose $S_*=(S_*^-,\F)\subseteq T_*=(T_*^-,\F)\in \Team(\K)$. Since $T^*\in \Team(\K)$, there is a $T\in\K$ such that $\Team(T)=T^*$. Furthermore,  let $S = (S^-,\F)$ be such that $\Team(S)=S_*$. %, and let $T\in\K$ such that $\Team(T)=T_*$.
% Now $\Team(S^-) \subseteq \Team(T^-)$, but $S^-$ is not necessarily included in $T^-$. It is instead included in $S^-\cup T^-$; so, define $T':=(S^-\cup T^-,\F)$, and notice that $\Team((T')^-)= \Team(S^-\cup T^-)= \Team(S^-) \cup \Team(T^-) =   \Team(T^-)$. Since 
%  $\Team(T')= \Team(T), T\in\K$ and $\K$ is support-closed, we have $T'\in \K$. But $S$ is a causal submultiteam of $T'$; since $\K$ is downward closed, then, $S\in \K$. Thus $S^*=\Team(S)\in \Team(\K)$.
%\end{itemize}

%ADD ARGUMENT FOR $\Team_\sigma^+(\K)$.

%Finally, we see how to extend the arguments above to $\Team_\sigma^\approx(\K)$. 
Let us consider first the case of downward closure. Let $T=(T^-,G)\in \Team_\sigma^\approx(\K)$ and $S$ a causal submultiteam of it. Since $T\in \Team_\sigma^\approx(\K)$, there is a $T'=((T')^-,\G)\in \K$ such that $T\approx \Team(T')$; then  $G\sim\G$. By the support-closure of $\K$, we can assume without loss of generality that there is a bijection between $T^-$ and $(T')^-$, i.e. each assignment in $(T')^-$ is of the form $t_s = s(n/Key)$ (for distinct values of $n$), where $s\in T^-$.
Define $S' = ((S')^-,\G)$, where $(S')^- = \{t_s \mid s\in S^-\}$. Then $S' < T'$; since $T'\in \K$ and $\K$ is downward closed, $S'\in\K$. However, clearly $S\approx \Team(S')$. Thus $S\in  \Team_\sigma^\approx(\K)$.


Now flatness. %One direction immediately follows from the downward closure case. Let consider the opposite direction.
 Let $T = (T^-,F)$ be a team of signature $\sigma$, and assume that, for all $s\in T^-$, $(\{s\}, F) \in \Team_\sigma^\approx(\K)$. But then, using also the support-closure of $\K$ (which follows from flatness), there is a (unique!) $\F$, made of the maximal canonical representatives of the functions given by $F$, such that $(\{s\}, \F)\in\K$ for each $s\in T^-$. By the flatness of $\K$, $(T^-,\F)\in \K$. But clearly $T \approx \Team((T^-,\F))$; thus, $T\in \Team_\sigma^\approx(\K)$. 
 The opposite direction immediately follows from the downward closure case.
%
%In the opposite direction, assume $T = (T^-,F)\in \Team_\sigma^\approx(\K)$. Let $s\in T^-$. There is a $T'= ((T')^-,\F)\in \K$ such that $T\approx \Team(T')$ (thus $F\sim \F$). Since $s\in T^-$, there is at least one $n\in\mathbb N$ such that $s(n/Key)\in (T')^-$. By flatness of $\K$,
%$(\{s(n/Key)\},\F)\in \K$. Since $(s,F)\approx \Team((s,\F))$, we have $(s,F) \in  \Team_\sigma^\approx(\K)$.
%\corrf{Maybe we only need the statement for $\Team_\sigma^\approx(\K)$...}
\end{proof}


%\item $\K$ is \textbf{union closed} if, whenever $T_i\in \K$ for $i\in I$, then $\bigcup_{i\in I}T_i\in\K$.
%\item $\K$ is \textbf{closed under finite unions} if $T,S\in K$ implies $T\cup$
%\item $\K$ is \textbf{}
%\item $\K$ is \textbf{}
%\item $\K$ is \textbf{}

%\end{itemize}

%A support-closed class has the following important property:


%\begin{lemma}[Support-closure]\label{lemma: support closure}
%Let $\varphi\in\CO_\sigma$ 
%and let $T,S$ be causal multiteams of signature $\sigma$. If $T\models\varphi$ and $\Team(T)=\Team(S)$, then $S\models\varphi$. 
%\end{lemma}

%\begin{proof}
%By two applications of Lemma \ref{lemma: transfer}, $T\models\varphi \iff \Team(T)=\Team(S)\models \varphi \iff S\models \varphi$.
%\end{proof}


%\commf{Commented out closure under equivalence lemma.}
%Concerning equivalence, we have: 

%\begin{lemma}\label{lemma: closure under equivalence}
%Let $S\approx T$ be causal multiteams of signature $\sigma$, $\varphi\in\PCO_{\sigma}$, and $S\models \varphi$. Then $T\models \varphi$.
%\end{lemma}

%\begin{proof}
%The proof is identical to that of \cite{BarYan2020}, Theorem 3.3.
%\end{proof}




We can now prove the correctness of the characterization of language $\CO_{\sigma}$.% (for $\sigma$ a \emph{finite} signature).

%\begin{theorem}
%Let $\sigma$ be a finite signature, and $\K$ a class of causal multiteams of signature $\sigma$. Then $\K$ is definable by a $\CO_{\sigma}$ formula (resp. a set of $\CO_{\sigma}$ formulas) if and only if:
%\begin{enumerate}
%\item $\K$ is support-closed
%\item $\K$ is closed under equivalence
%\item $\K$ is flat.
%\end{enumerate}
%\end{theorem}

\begin{proof}[Proof of Theorem \ref{thm:COchar}]
$\Rightarrow$) %1. immediately follows from the fact that $\CO$ formulas are flat. %Similarly, 2. follows from the fact that $\CO$ formulas cannot tell apart equivalent causal multiteams \corrf{(lemma \ref{lemma: closure under equivalence})}. %(see \cite{BarYan2020}, Theorem 3.3    for a proof in the causal team case). 
%2. follows from Lemma \ref{lemma: support closure}.
This is just Theorem \ref{thm: CO flatness}.

$\Leftarrow$) Assume $\K$ satisfies flatness; as observed before, then, $\K$ is support closed.  Let $\Team_\sigma^\approx(\K) = \{S \text{ of signature $\sigma$ } \mid \exists T\in\K : S\approx \Team(T)\}$ as before. % Define $\Team(\K):= \{\Team(T) \mid T\in\K\}$. 
By Lemma \ref{lemma: transferable closure properties},   $\Team_\sigma^\approx(\K)$ satisfies the causal team version of flatness. Furthermore, by definition $\Team_\sigma^\approx(\K)$ is closed under the equivalence relation $\approx$. But then, by the characterization of $\CO_{\sigma}$ over causal teams (Theorem 4.4 of \cite{BarYan2022}), there is a formula $\varphi\in\CO_{\sigma}$ which defines $\Team_\sigma^\approx(\K)$ over causal teams of signature $\sigma$. We show that the same formula defines $\K$ over causal multiteams of signature $\sigma$.

Let $T\in\K$. Then $\Team(T)\in \Team_\sigma^\approx(\K)$. So $\Team(T)\models^{ct} \varphi$. By Lemma \ref{lemma: transfer}, then, $T\models \varphi$.

In the opposite direction, let $T$ be a causal multiteam such that $T\models \varphi$. By Lemma \ref{lemma: transfer}, $\Team(T)\models^{ct} \varphi$. But then $\Team(T)\in \Team_\sigma^\approx(\K)$. Since $\Team(T)\in \Team_\sigma^\approx(\K)$, there is an $S\in \K$ such that $\Team(S) \approx \Team(T)$. This entails, in particular, that the function components of $\Team(S)$ and $\Team(T)$ are similar. But since they are also the function components of $S$, resp. $T$, the fact that they are similar just means that they are identical. Thus $\Team(S) = \Team(T)$.  %\in \Team(\K))$.
 Then, by the support-closure of $\K$, $T\in\K$. 
\end{proof}


%We now move to characterizing the expressive power of the various extensions of $\CO$ (which are all equiexpressive):

%\begin{theorem}
%Let $\sigma$ be a finite signature, and $\K$ a class of causal multiteams of signature $\sigma$. Then $\K$ is definable by a (set of) $\COU_{\sigma}$ (resp. $\COD_{\sigma},\CODU_{\sigma}$) formula(s)  if and only if:
%\begin{enumerate}
%\item $\K$ is support-closed
%\item $\K$ is closed under equivalence
%\item $\K$ is downward closed.
%\end{enumerate}
%\end{theorem}

%\begin{proof}
%The proof is analogous to the previous one.
%\end{proof}

%As a second application of the methods developed in this appendix, we show that the formulas $\Phi^\F$ (introduced in section \ref{subs: PO and PC}) still work correctly in causal multiteam semantics. 

