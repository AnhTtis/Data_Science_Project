\documentclass[12pt]{elsarticle}
%\documentclass[12pt]{article}
\usepackage{amssymb,latexsym,amsmath}
\usepackage{graphicx,times}
\usepackage{amsthm}
\usepackage{dcolumn}
\usepackage[margin=1.0in]{geometry}
\setlength{\textheight}{21cm}
\setlength{\textwidth}{16cm}
\newtheorem{thm}{Theorem}
\newtheorem{lem}[thm]{Lemma}
\newtheorem{cor}[thm]{Corollary}
\newtheorem{Def}{Definition}[section]
\newtheorem{Ex}{Example}[section]
\numberwithin{equation}{section}
\baselineskip 20 truept
%\usepackage{lineno,hyperref}
%\modulolinenumbers[5]

\journal{Journal of \LaTeX\ Templates}

%%%%%%%%%%%%%%%%%%%%%%%
%% Elsevier bibliography styles
%%%%%%%%%%%%%%%%%%%%%%%
%% To change the style, put a % in front of the second line of the current style and
%% remove the % from the second line of the style you would like to use.
%%%%%%%%%%%%%%%%%%%%%%%

%% Numbered
%\bibliographystyle{model1-num-names}

%% Numbered without titles
%\bibliographystyle{model1a-num-names}

%% Harvard
%\bibliographystyle{model2-names.bst}\biboptions{authoryear}

%% Vancouver numbered
%\usepackage{numcompress}\bibliographystyle{model3-num-names}

%% Vancouver name/year
%\usepackage{numcompress}\bibliographystyle{model4-names}\biboptions{authoryear}

%% APA style
%\bibliographystyle{model5-names}\biboptions{authoryear}

%% AMA style
%\usepackage{numcompress}\bibliographystyle{model6-num-names}

%% `Elsevier LaTeX' style
\bibliographystyle{elsarticle-num}
%%%%%%%%%%%%%%%%%%%%%%%

\begin{document}

\begin{frontmatter}

\title{ Existence and Uniqueness Theorems for Differential Equations with Proportional Delay}
%\tnotetext[mytitlenote]{Fully documented templates are available in the elsarticle package on \href{http://www.ctan.org/tex-archive/macros/latex/contrib/elsarticle}{CTAN}.}

%% Group authors per affiliation:

\author{Prajakta Rajmane$^{a*}$ }
\ead{prajaktarajmane@gmail.com}
\author{Jayvant Patade$^{b}$ }
\ead{dr.jayvantpatade@gmail.com}
\author{M. T. Gophane$^{a}$ }
\ead{mtg\_math@unishivaji.ac.in}
\address{$^{a}$Department of Mathematics, Shivaji University, Kolhapur - 416004, India.\\ $^{b}$Department of Mathematics, Jaysingpur College, Jaysingpur - 416101, India.}%\\ Email: $^1$dr.jayvantpatade@gmail.com,  $^2$sbb\_maths@unishivaji.ac.in 
%\fntext[myfootnote]{Since 1880.}

%% or include affiliations in footnotes:
%\author[mymainaddress,mysecondaryaddress]{Elsevier Inc}
%\ead[url]{www.elsevier.com}

%\author[mysecondaryaddress]{Global Customer Service\corref{mycorrespondingauthor}}
\cortext[mycorrespondingauthor]{Corresponding author}
%\address[mymainaddress]{1600 John F Kennedy Boulevard, Philadelphia}
%\address[mysecondaryaddress]{360 Park Avenue South, New York}
\begin{abstract}
The differential equation (DE) with proportional delay is a particular case of the time-dependent delay differential equation (DDE). In this paper, we solve non-linear  DEs with proportional delay using the successive approximation method (SAM). We prove the existence, uniqueness of theorems, and stability for DEs with proportional delay using SAM. We derive convergence results for these equations by using the Lipschitz condition. We generalize these results to the fractional differential equations (FDEs) and system of FDEs containing Caputo fractional derivative. Further, we obtain the series solution of the pantograph equation and Ambartsumian equation in the form of a power series which are convergent for all reals. Finally, we illustrate the efficacy of the SAM  by example. The results obtained by SAM are compared with exact solutions and other iterative methods. It is observed that SAM is simpler compared to other methods and the solutions obtained using SAM are consistent with the exact solution.

\end{abstract}
\begin{keyword}
Successive approximation method; Lipschitz condition; Caputo derivative; Existence-uniqueness; proportional delay; pantograph equation; Ambartsumian equation.
\MSC[2020] 26A33; 34A08; 34K06; 34K20. 
\end{keyword}
\end{frontmatter}

%\linenumbers
\section{Introduction}
The delay differential equations (DDE) contain the state variable term at a past time $t-\tau$. The inclusion of the delay $\tau$ makes the DDE an infinite dimensional dynamical system. Even if it is very difficult to analyze and solve such equations, this branch is popular among the applied scientists due to the applications in various fields. 
\par On the other hand, if the order of the derivative in a differential equation is any arbitrary number (instead of a positive integer) then the equation is called as the fractional differential equation (FDE).  Even though there are several inequivalent definitions of fractional derivative operator, one can select the derivative which is appropriate for the model under consideration. This flexibility is a key feature behind the popularity of fractional calculus.
% Review of FDDE %
\par Daftardar-Gejji and coworkers proposed numerical schemes \cite{bhalekar2011predictor, daftardar2015solving} for solving fractional order delay differential equations (FDDE).  Modified Laguerre wavelets method \cite{iqbal2015modified}, spectral collocation method \cite{ali2019chebyshev},  fractional-order fibonacci-hybrid functions \cite{sabermahani2020fractional} are few other methods for solving FDDEs.  Stability analysis of FDDEs is proposed in \cite{bhalekar2013stability,bhalekar2016stability,kaslik2012analytical,bhalekar2019analysis}. Applications of FDDE are presented in \cite{bhalekar2011fractional,rihan2020dynamics,latha2017fractional,rihan2021fractional}.

% Motivation, description of work %
\par In general, the delay $\tau$ in the DDE ${x'(t)}=f(t,x(t),x(t-\tau))$ is not constant. The analysis becomes more difficult when $\tau$ depends on time or state. The proportional delay differential equation ${x'(t)}=f(t,x(t),x(qt))$ or a pantograph equation is a particular case of time-dependent DDE with $\tau(t)=(1-q)t$. 
% review of proportional dde
These equations are proposed by Ockendon and Tayler  in the seminal work \cite{ockendon1971dynamics} to model the motion of an overhead trolley wire. Few other applications of these equations are discussed in \cite{buhmann1993stability,bhalekar2017series}. The Daftardar-Gejji and Jafari method (DJM) is applied in \cite{bhalekar2016analytical} to find analytical solutions of pantograph equation. Further, the authors presented the various relations of the solution series with the existing special functions.  Patade and Bhalekar  proposed the power series solution Ambartsumian equation \cite{patade2017analytical} by using DJM. The analytical solution of pantograph equation are discussed in \cite{patade2017analytical1}. 
%%%%%
 In this paper, we generalize the results in \cite{coddington2012introduction} on the existence-uniqueness of ordinary differential equations (ODE) to differential equations with proportional delay,  FDEs proportional delay and system of FDEs with proportional delay. We use the successive approximation method (SAM) to prove our results.
 
  The paper is organized as follows. In Section 2, we
   give definitions and notations of fractional derivatives and integrals. The differential equations with proportional delay are described in Section 3. 
   Successive approximate solutions and existence theorem are discussed in Section 4. The stability analysis is presented in Section 5. The series solution of the pantograph equation and Ambartsumian equation are described in Section 6.   
     The generalization of these results to  FDEs and the system of FDEs are derived in Section 7 and Section 8. Section 9 deals with illustrative
     example and the conclusions are summarized in Section 10. 
\section{Preliminaries and Notations} \label{Preliminaries}
\begin{Def}\cite{kilbas2006theory}
	The  Riemann-Liouville fractional integral of order $\alpha> 0 $ of $f\in C[0, \infty) $ is defined as
	\begin{equation}
	I^\alpha f(t)=\frac{1}{\Gamma(\alpha)} \int_{0}^t (t-\zeta)^{\alpha-1}f(\zeta)d\zeta,\quad t>0.
	\end{equation}
\end{Def}
\begin{Def}\cite{kilbas2006theory}
	The (left sided) Caputo fractional derivative of $f, f \in C_{-1}^m, m\in\mathbb{N}\cup\{0\}$, is defined as:
	\begin{eqnarray}
	D^\alpha f(t)&=&\frac{d^m}{ dt^m} f(t),\quad \alpha = m \nonumber\\
	&=& I^{m-\alpha}\frac{d^m}{ dt^m} f(t),\quad {m-1} <\alpha <m,\quad m\in \mathbb{N}.
	\end{eqnarray}
\end{Def}
Note that for $0\le m-1 < \alpha \le m$ and $\beta>-1$
\begin{eqnarray}
I^\alpha x^\beta &=&\frac{\Gamma{(\beta+1)}}{ \Gamma{(\beta+\alpha+1)}} x^{\beta+\alpha},\nonumber\\
\left(I^\alpha D^\alpha f\right)(t)&=& f(t)-\sum_{k=0}^{m-1} f^{(k)}(0)\frac{t^k}{k!}.
\end{eqnarray}
\begin{Def}\cite{kilbas2006theory}
	The  Mittag-Leffler function is defined as  
	\begin{equation}
	E_\alpha (t)=\sum_{n=0}^\infty\frac{t^n}{\Gamma{(\alpha n+1)}}, \quad \alpha>0.
	\end{equation}
\end{Def}

\begin{Def}\cite{kilbas2006theory}
 The multi-parameter Mittag-Leffler function is defined as:
	\begin{equation*}
	E_{(\alpha_{1},\cdots,\alpha_{n}), \beta}(z_1, z_2,\cdots,z_{n})=\sum^\infty_{k=0}\mathop{\sum_{l_1+\cdots+l_{n } =k}}_{l_j\geq 0}(k;l_1,\cdots,l_{n})\left[\frac{\displaystyle\prod_{j=1}^{n }z^{l_j}_{j}}{\Gamma(\beta +\displaystyle\sum^{n}_{j=1}\alpha_{j}l_j)}\right].
	\end{equation*}
	where, $\left(k;l_1,l_2,\cdots,l_n\right)$ is the multinomial coefficient defined as
	\begin{equation}
	\left(k;l_1,l_2,\cdots,l_n\right)=\frac{k!}{l_1!l_2!\cdots l_n!}.
	\end{equation}
\end{Def}

\section{Differential Equations with Proportional delay}
Consider the differential equations with proportional delay
\begin{equation}
y'(t)=f(t, {y}(t), {y}(qt)),y(0)=y_0, 0<q<1, \label{1}
\end{equation}
where  $f$ is a continuous function defined on some rectangle
\begin{equation}
R=\{|t|\leq a,|y(t)-y_0|\leq b,|y(qt)-y_0|\leq b, a>0,\, b>0\}\nonumber.
\end{equation}

\begin{thm}
A function $\phi$ is a solution of the IVP (\ref{1}) on an interval $I$ if and only if it is a solution of the integral equation 
\begin{equation}
y(t)=y_0+\int_{0}^{t} f(x, {y}(x), {y}(qx))dx \quad \text{on}\quad I \label{2.1}
\end{equation}
\end{thm}
\begin{proof}
Let $\phi$ is a solution of the IVP (\ref{1}) on an interval $I$. Then\\
\begin{equation}
\phi'(t)=f(t, {\phi}(t), {\phi}(qt)),\phi(0)=y_0, 0<q<1 \label{3.1}
\end{equation}
The equivalent integral equation (\ref{3.1}) is
\begin{equation}
\phi(t)=\phi(0)+\int_{0}^{t} f(x, {\phi}(x), {\phi}(qx))dx.\label{4.1}
\end{equation}
and $ \phi(0)=y_0 $. Thus $\phi$ is a solution of the IVP (\ref{2.1}).\\
Conversely, suppose equation (\ref{4.1}) hold. Differentiate equation (\ref{4.1}) w.r.t. $t$, we get
\begin{equation}
\phi'(t)=f(t, {\phi}(t), {\phi}(qt)), 0<q<1 \quad \forall t\in I\nonumber
\end{equation}
From equation (\ref{2.1}) $\phi(0)=y_0$.\\
Hence $\phi$ is a solution of the IVP (\ref{1}).
\end{proof}

\section{Successive Approximate Solution for Differential Equations with Proportional Delay}

Let $\phi_0(t)=y_0$ be the first approximate solution of the IVP (\ref{1}). Then
\begin{eqnarray}
\phi_1(t)&=&y_0+\int_{0}^{t} f(x, {\phi_0}(x), {\phi_0}(qx))dx.\nonumber\\
\phi_2(t)&=&y_0+\int_{0}^{t} f(x, {\phi_1}(x), {\phi_1}(qx))dx.\nonumber
\end{eqnarray}
Continuing in this way, we obtain
\begin{equation}
\phi_{k+1}(t)=y_0+\int_{0}^{t} f(x, {\phi_k}(x), {\phi_k}(qx))dx.\quad k=0,1,2,\cdots.\label{22}
\end{equation}

\begin{thm}\label{thm2}
Let f is continuous and $|f|\leq M$ on $R$. The successive approximation (\ref{22}) exist and continuous on the interval $I=[-\zeta,\zeta]$, where $\zeta=\text{min}\left\{a,\frac{b}{M}\right\}$.  
	If $t\in I$  then $\left(t,y(t),y(qt)\right)\in R$ and $|\phi_{k}(t)-y_0|\leq M|t|$, $|\phi_{k}(qt)-y_0|\leq M|t|$.
\end{thm}
\begin{proof}
We prove the result by mathematical induction.\\
(i) Clearly  $\phi(0)=y_0$ is continuous on $I$. Thus, theorem is true for $k=0$.\\
(ii) For $k=1$, we have
\begin{eqnarray}
\phi_1(t)&=& y_0+\int_{0}^{t} f(x, {\phi_0}(x), {\phi_0}(qx))dx.\nonumber\\
\phi_1(t)&=& y_0+\int_{0}^{t} f(x, y_0, y_0)dx.\nonumber
\end{eqnarray}
Since $f$ is continuous and hence,   $\phi_1(t)$ exist.
 \begin{eqnarray}
|\phi_1(t)-y_0| &=&|\int_{0}^{t} f(x, {\phi_0}(x), {\phi_0}(qx))dx|.\nonumber\\
 &\leq&\int_{0}^{t}| f(x, {\phi_0}(x), {\phi_0}(qx))|dx.\nonumber\\
 &\leq& M|t| \nonumber\\
 &\leq& b,  \quad t\in I\nonumber\\
\text{and}\quad |\phi_1(qt)-y_0| &\leq& M|qt| \nonumber\\
&\leq& M|t|, \quad 0<q<1 \nonumber\\
 &\leq& b,  \quad t\in I\nonumber
\end{eqnarray}
Thus, for $t\in I$, $\left(t,y(t),y(qt)\right)\in R$ and $|\phi_{1}(t)-y_0|\leq M|t|$, $|\phi_{1}(qt)-y_0|\leq M|t|$.\\
The theorem is true for $k=1$\\
(iii)  Assume that theorem is true for $k=n$.\\
i.e. For $t\in I$, $\left(t,y(t), y(qt)\right)\in R$ and $|\phi_{n}(t)-y_0|\leq M|t|$, $|\phi_{n}(qt)-y_0|\leq M|t|$.\\
(iv) To prove the theorem for $k=n+1$.\\
If $t\in I$, then 
 \begin{eqnarray}
\phi_{n+1}(t)&=&y_0+\int_{0}^{t} f(x, {\phi_n}(x), {\phi_n}(qx))dx.\nonumber
\end{eqnarray}
Since $f$ is continuous and hence,   $\phi_{n+1}(t)$ exist on $I$.
\begin{eqnarray}
|\phi_{n+1}(t)-y_0|  &\leq& M|t| \nonumber\\
 &\leq& b,  \quad t\in I\nonumber\\
\text{and}\quad |\phi_{n+1}(qt)-y_0| &\leq& M|qt| \nonumber\\
&\leq& M|t|, \quad 0<q<1 \nonumber\\
 &\leq& b,  \quad t\in I\nonumber
\end{eqnarray}
Thus, if  $t\in I$, $\left(t,y(t),y(qt)\right)\in R$ and $|\phi_{n+1}(t)-y_0|\leq M|t|$, $|\phi_{n+1}(qt)-y_0|\leq M|t|$.\\
Hence by mathematical induction, the result is true for all positive integer $n$.
\end{proof}

\begin{thm}(Existence Theorem)
Let f is continuous and $|f|\leq M$ on the rectangle
\begin{equation}
R=\{|t|\leq a,|y(t)-y_0|\leq b,|y(qt)-y_0|\leq b, a>0,\, b>0\}\nonumber.
\end{equation}
Suppose $f$ satisfies Lipschitz condition in second and third variable with  Lipschitz constants $L_1$ and $L_2$ such that 
\begin{equation}
|f(t, {y_1}(t), {y_1}(qt))-f(t, {y_2}(t), {y_2}(qt))|\leq L_1|y_1(t)-y_2(t)|+L_1|y_1(qt)-y_2(qt)|\nonumber.
\end{equation}
Then the successive approximations (\ref{22}) converges on the interval  $I=[-\zeta,\zeta]$, where $\zeta=\text{min}\left\{a,\frac{b}{M}\right\}$ to a solution  $\phi$ of the IVP  (\ref{1}) on $I$.
\end{thm}
\begin{proof}
We have 
\begin{equation}
\phi_k(t)=\phi_0(t)+\sum_{n=1}^{k}[\phi_n(t)-\phi_{n-1}(t)]\nonumber.
\end{equation}
To prove the sequence $\{\phi_k\}$ converges, it is enough to prove the series 
\begin{equation}
\phi_0(t)+\sum_{n=1}^{\infty}[\phi_n(t)-\phi_{n-1}(t)]\label{6.1}
\end{equation}
is convergent.\\
By theorem (\ref{thm2}) the function $\phi_k$ all exist and continuous on $I$.\\
Also, $|\phi_{1}(t)-\phi_{0}(t)|\leq M|t|$ and $|\phi_{1}(qt)-\phi_{0}(qt)|\leq M|t|$ for $t\in I$.\\
Now,
\begin{eqnarray}
\phi_{2}(t)-\phi_{1} (t)&=&\int_{0}^{t}[ f(x, {\phi_1}(x), {\phi_1}(qx))- f(x, {\phi_0}(x), {\phi_0}(qx))]dx\nonumber\\
\therefore |\phi_{2}(t)-\phi_{1} (t)|&\leq& \int_{0}^{t}| f(x, {\phi_1}(x), {\phi_1}(qx))- f(x, {\phi_0}(x), {\phi_0}(qx))|dx\nonumber\\
&\leq& \int_{0}^{t}[ L_1 |{\phi_1}(x)-{\phi_0}(x)|+L_2 |{\phi_1}(qx)-  {\phi_0}(qx)|]dx\nonumber\\
&\leq& M(L_1+L_2)\frac{|t|^2}{2}\nonumber.
\end{eqnarray}
We shall prove by mathematical induction
\begin{equation}
|\phi_{n}(t)-\phi_{n-1} (t)|\leq M(L_1+L_2)^{n-1}\frac{|t|^n}{n!}\label{7}
\end{equation}
We have prove that equation (\ref{7}) true for $n=1,2.$\\
Assume that (\ref{7}) true for $n=m.$\\
We have 
\begin{eqnarray}
\phi_{m+1}(t)-\phi_{m} (t)&=&\int_{0}^{t}[ f(x, {\phi_m}(x), {\phi_m}(qx))- f(x, {\phi_{m-1}}(x), {\phi_{m-1}}(qx))]dx\nonumber\\
\therefore |\phi_{m+1}(t)-\phi_{m} (t)|&\leq& \int_{0}^{t}| f(x, {\phi_m}(x), {\phi_m}(qx))- f(x, {\phi_{m-1}}(x), {\phi_{m-1}}(qx)|dx\nonumber\\
&\leq& \int_{0}^{t}[ L_1 | {\phi_m}(x)-\phi_{m-1}(x)|+L_2 |{\phi_m}(qx)-  {\phi_{m-1}}(qx)|]dx\nonumber\\
&\leq& M(L_1+L_2)^{m}\frac{|t|^{m+1}}{(m+1)!}\label{6}.\nonumber
%&\leq& \frac{M}{(L_1+L_2)}\frac{[(L_1+L_2)|t|]^{m+1}}{(m+1)!} 
\end{eqnarray}
Thus, the result is true for $ n=m+1 $.\\
Hence, by the mathematical induction result is true for all $ n=1,2,\cdots.$\\
Therefore, the infinite series  (\ref{7}) is absolutely convergent on $I$.
This shows that the $n^{th}$ term of the series $|\phi_0(t)|+\sum_{n=1}^{\infty}|\phi_n(t)-\phi_{n-1}(t)|$ is less than $\frac{M}{(L_1+L_2)}$ times the $n^{th}$ term of the power series $e^{(L_1+L_2)|t|}$.
Hence The series (\ref{7}) is convergent.
\end{proof}

\section{Stability Analysis}
 The differential equations with proportional delay
\begin{equation}
y'(t) = f\left(t, y(t), y(qt)\right),
\end{equation}
is a special case of the time-dependent delay differential equation (DDE)
\begin{equation}
y'(t) = f\left(t, y(t), y\left(t-\tau(t)\right)\right)\quad \textrm{with}\quad\tau(t)=(1-q)t,  \nonumber
\end{equation}

\begin{Def}\cite{deng2006stability}
Consider the  DDE,
\begin{equation}
y'(t) = f(y(t), y(t-\tau(t))),\label{13}
\end{equation}
where $f:R\times R\rightarrow R$.
The flow $\phi_t(t_0)$ is a solution $y(t)$ of ({\ref{13}}) with initial condition $y(t) = t_0,\, t\leq 0$. The point $y^*$ is called equilibrium solution of  ({\ref{13}}) if $f(y^*,y^*)=0$.\\
\textbf{(a)} If, for any  $\epsilon > 0$, there exist  $\delta > 0$ such that $ |t_0-y^*|< \delta \Rightarrow |\phi_t(t_0)-y^*| < \epsilon,$ then the system ({\ref{13}}) is stable (in the Lyapunov sense) at the equilibrium  $y^*$.\\
\textbf{(b)} If the system ({\ref{13}}) is stable at   $y^*$ and moreover,$\lim\limits_{t\rightarrow\infty}|\phi_t(t_0)-y^*|=0$ then the system  ({\ref{13}}) is said to be asymptotically stable at $y^*$. 
\end{Def}
The following results are similar to those in \cite{deng2006stability}

\begin{thm}
Suppose that the  equilibrium solution $y^*$ of the equation
\begin{equation}
y' = f(y(t),y(t-\tau^*)),\quad \tau^*= \tau(t_0)
\end{equation}
is stable and 
$\|f(y(t),y(t-\tau(t))) - f(y(t),y(t-\tau(t_1)))\|<\epsilon_1 |t-t_1|$,
for some $\epsilon_1>0$ and $t,t_1\in [t_0, t_0+c),$
c is a positive constant, then there exists $\bar{t}> 0$ such that
the equilibrium solution $y^*$ of Eq. ({\ref{13}}) is stable on finite time interval $[t_0,\bar{t})$.
\end{thm}
\begin{cor}
If the real parts of all roots of $ \lambda -a-be^{-\lambda\tau^*}=0$ are negative, where $a =\partial_1f, b = \partial_2f$ evaluated at equilibrium. Then there exist $\epsilon_c, \bar{t}(>t_0) $, such that when $\epsilon_1 <\epsilon_c$, the solution  $y^*=0$ of Eq. ({\ref{13}}) is stable on finite time interval  $[t_0,\bar{t})$.
\end{cor}

\section{Series Solution of Pantograph Equation}
A pantograph is a device used in electric trains to collect current from overloaded lines. The pantograph equation was formulated by Ockendon and Taylor in 1971 and originates in electrodynamics.\\
Consider the pantograph equation,

\begin{equation}
y'(t) = ay(t)+ by(q t),\quad y(0)= 1, \label{14}
\end{equation}
where $0<q<1$, $a, b \in {R}$.

Integrating (\ref{14}), we get
\begin{equation}
y(t)= 1 + \int_{0}^{t}\left(ay(x)+ by(qx)\right)dt\label{15}
\end{equation}
Suppose $\phi_k(t)$ be the $k^{th}$ approximate solution, where the initial approximate solution is
taken as
\begin{equation}
\phi_0(t)=1.
\end{equation}

For $k \ge 1$, the recurrent formula as below:
\begin{equation}
\phi_k (t) =1 + \int_{0}^{t}\left(a\phi_{k-1}(x)+b\phi_{k-1}(qx)\right)dx\label{16}.
\end{equation}
From the recurrent formula, we have 
\begin{eqnarray}
\phi_1(t)&=& 1+\int_{0}^{t}\left( a\phi_0(x)+ by_0(q x)\right)dx\nonumber\\
&=& 1+ (a+b)\frac{t}{1!},\nonumber\\
\phi_2(t)&=& 1+ \int_{0}^{t}\left( a\phi_1(x)+ b\phi_1(q x)\right)dt\nonumber\\
&=& 1+(a+b)\frac{t}{1!}+(a+b)(a+bq)\frac{t^2}{2!},\nonumber\\
\phi_3(t)&=& 1 +\int_{0}^{t}\left(a\phi_2(t)+ b\phi_2(q t)\right)dt\nonumber\\
&=& 1+(a+b)\frac{t}{1!}+(a+b)(a+bq)\frac{t^2}{2!} (a+b)(a+bq)(a+bq^2)\frac{t^3}{3!},\nonumber\\
&\vdots& \nonumber\\
\phi_k(x)&=& 1+\frac{t^k}{k!}\prod_{j=0}^{k-1}\left(a+bq^j\right), \quad k=1,2,3\cdots. \nonumber\\
&&\text{As} \quad k\rightarrow\infty,\quad \phi_k (t)\rightarrow y(t)\nonumber\\
y(t) &=&  1 + \sum_{m=1}^\infty \frac{t^m}{m!}\prod_{j=0}^{m-1}\left(a+bq^j\right).\nonumber
\end{eqnarray}
If we define $\prod_{j=0}^{m-1}\left(a+bq^j\right)=1$, for $m=0$,
then
\begin{equation}
		y(t) = \sum_{m=0}^\infty \frac{t^m}{m!}\prod_{j=0}^{m-1}\left(a+bq^j\right).\label{17}
\end{equation}

\begin{thm}
	For $0<q<1$, the  power series (\ref{17})	is  convergent for $t\in{R}$.
\end{thm}

\begin{cor}
	The power series (\ref{17}) is  absolutely convergent for all $t$ and hence it is uniformly convergent on any compact interval on ${R}$.
\end{cor}

\begin{thm}
If $0<q<1$, $a, b\ge 0$, then
\begin{equation}
 e^{at}\le y(t) =\sum_{m=0}^\infty \frac{t^m}{m!}\prod_{j=0}^{m-1}\left(a+bq^j\right)\le e^{(a+b+c)t},\quad 0 \le t < \infty.\nonumber
\end{equation}
\end{thm}


\begin{thm}
If $(a+b)< 0$ then zero solution of ({\ref{14}}) is asymptotically stable.
\end{thm}
\begin{proof}
 \begin{eqnarray}
 \text{Define}\qquad u(t)&=&\max_{0\leq x \leq t}y^2(t)\nonumber\\
\therefore\frac{1}{2}u'(t)&=& \frac{1}{2} \frac{d}{dt}(y^2(t))\nonumber\\
&=& y(t)y'(t)\nonumber\\
&=& y(t)(ay(t)+ by(q t))\nonumber\\
&=& a y^2(t) + b y(t) y(qt)\nonumber\\
&\leq& (a+b)u(t)\nonumber\\
\Rightarrow u(t)&\leq& u(0)e^{2(a+b)t}\nonumber\\
\therefore \lim\limits_{x\rightarrow\infty}y(t)&=& 0, \quad \textrm{if} \quad(a+b)< 0.\nonumber
\end{eqnarray}
\end{proof}

\subsection{Series Solution of Ambartsumian Equation}
In \cite{ambartsumian1944fluctuation} Ambartsumian derived a delay differential equation describing the fluctuations of the surface brightness in a milky way. The equation is described as:
\begin{equation}
y'(t) = -y(t)+\frac{1}{q}y\left(\frac{t}{q}\right)\label{a1}
\end{equation}
where $q>1$ and is constant for the given model.\\
The Eq.(\ref{a1}) with initial condition $y(0)=\lambda$ can be written equivalently as
\begin{equation}
y(t)= \lambda + \int_{0}^{t}\left(\frac{1}{q}y\left(\frac{x}{q}\right)-y(x)\right)dx.\label{a2}
\end{equation}
Suppose $\phi_k(t)$ be the $k^{th}$ approximate solution, where the initial approximate solution is
taken as
\begin{equation}
\phi_0(t)=\lambda.
\end{equation}
For $k \ge 1$, the recurrent formula as below:
\begin{equation}
\phi_k (t) =\lambda + \int_{0}^{t}\left(\frac{1}{q}\phi_{k-1}\left(\frac{x}{q}\right)-\phi_{k-1}(x)\right)dx\label{a4}.
\end{equation}
From the recurrent formula, we have 
\begin{eqnarray}
\phi_1 (t) &=& \lambda + \int_{0}^{t}\left(\frac{1}{q}\phi_{0}\left(\frac{x}{q}\right)-\phi_{0}(x)\right)dx\nonumber\\
&=&\lambda +\int_{0}^{t}\left(\frac{\lambda}{q}-\lambda\right)dx\nonumber\\
&=&\lambda +\left(\frac{\lambda}{q}-\lambda\right)\frac{t}{1!}\nonumber\\
&=&\left(1+\left(\frac{1}{q}-1\right)\frac{t}{1!}\right)\lambda, \nonumber\\
\phi_2 (t)&=&\lambda + \int_{0}^{t}\left(\frac{1}{q}\phi_{1}\left(\frac{x}{q}\right)-\phi_{1}(x)\right)dx\nonumber\\
&=&  \left(1+\left(\frac{1}{q}-1\right)\frac{t}{1!}+\left(\frac{1}{q}-1\right)\left(\frac{1}{q^2}-1\right)\frac{t^2}{2!}\right)\lambda,\nonumber\\
&&\vdots\nonumber\\
\phi_k (t)&=&  \left( 1 + \sum_{m=1}^k \frac{t^m}{m!}\prod_{j=1}^m\left(\frac{1}{q^j}-1\right)\right)\lambda.\nonumber\\	
&&\text{As} \quad k\rightarrow\infty,\quad \phi_k (t)\rightarrow y(t)\nonumber\\
y(t)&=& \left( 1 + \sum_{m=1}^\infty \frac{t^m}{m!}\prod_{j=1}^m\left(\frac{1}{q^j}-1\right)\right)\lambda.\nonumber
\end{eqnarray}
If we define $\prod_{j=1}^m\left(\frac{1}{q^j}-1\right)=1$, for $m=0$,
then

\begin{equation}
y(t)= \left( \sum_{m=0}^\infty \frac{t^m}{m!}\prod_{j=1}^m\left(\frac{1}{q^j}-1\right)\right)\lambda.\label{a5}
\end{equation}
\begin{thm}
	For $q>1$, the  power series (\ref{a5})	is  convergent for $t\in{R}$.
\end{thm}

\begin{cor}
	The power series (\ref{a5}) is  absolutely convergent for all $t$ and hence it is uniformly convergent on any compact interval on ${R}$.
\end{cor}
\begin{thm}
The zero solution of ({\ref{a1}}) is asymptotically stable.
\end{thm}
\section{Fractional order differential equations with proportional delay}
Consider the initial value problem (IVP)
\begin{eqnarray}
D^{\alpha}y(t)&=&f(t, {y}(t), {y}(qt)), 0<\alpha\leq 1, 0<q<1\nonumber\\
y(0)&=&y_0,\label{8}
\end{eqnarray}
where $D^{\alpha}$ denotes Caputo fractional derivative and $f$ is a continuous function defined on the   rectangle
\begin{equation}
R=\{|t|\leq a,|y(t)-y_0|\leq b,|y(qt)-y_0|\leq b, a>0,\, b>0\}\nonumber.
\end{equation}
\begin{thm}
A function $\phi$ is a solution of the IVP (\ref{8}) on an interval $I$ if and only if it is a solution of the integral equation 
\begin{equation}
y(t)=y_0+\int_{0}^{t} \frac{(t-x)^{\alpha-1}}{\Gamma(\alpha)}f(x, {y}(x), {y}(qx))dx \quad \text{on}\quad I. \label{9}
\end{equation}
\end{thm}

\begin{thm}\label{thm5}
Let f is continuous and $|f|\leq M$ on $R$. The successive approximation 
\begin{eqnarray}
\phi_{k+1}(t)&=&y_0\nonumber\\
\phi_{k+1}(t)&=&y_0+\int_{0}^{t} \frac{(t-x)^{\alpha-1}}{\Gamma(\alpha)}f(x, {\phi_k}(x), {\phi_k}(qx))dx.\quad k=0,1,2,\cdots.\label{10}
\end{eqnarray}

 exist and continuous on the interval $I=[-\zeta,\zeta]$, where $\zeta=\text{min}\left\{a,(\frac{\Gamma(\alpha+1)b}{M})^\frac{1}{\alpha}\right\}$.  
	If $t\in I$  then $\left(t,y(t),y(qt)\right)\in R$ and $|\phi_{k}(t)-y_0|\leq M\frac{|t|^\alpha}{\Gamma(\alpha+1)}$, $|\phi_{k}(qt)-y_0|\leq M\frac{|t|^\alpha}{\Gamma(\alpha+1)}$.
\end{thm}
\begin{thm}(Existence Theorem)
Let f is continuous and $|f|\leq M$ on the rectangle
\begin{equation}
R=\{|t|\leq a,|y(t)-y_0|\leq b,|y(qt)-y_0|\leq b, a>0,\, b>0\}\nonumber.
\end{equation}
Suppose $f$ satisfies Lipschitz condition in second and third variable with  Lipschitz constants $L_1$ and $L_2$ such that 
\begin{equation}
|f(t, {y_1}(t), {y_1}(qt))-f(t, {y_2}(t), {y_2}(qt))|\leq L_1|y_1(t)-y_2(t)|+L_1|y_1(qt)-y_2(qt)|\nonumber.
\end{equation}
Then the successive approximations (\ref{10}) converges on the interval  $I=[-\zeta,\zeta]$, where $\zeta=\text{min}\left\{a,(\frac{\Gamma(\alpha+1)b}{M})^\frac{1}{\alpha}\right\}$ to a solution $\phi$ of the IVP  (\ref{8}) on $I$.
\end{thm}

\subsection{Series Solution of Fractional Order Pantograph Equation}	
Consider the  fractional order pantograph equation as :

	\begin{equation}
	D^\alpha y(t) = ay(t)+ by(q t),\quad y(0)= 1, \label{19}
	\end{equation}
	where  $0<\alpha \leq 1$, $ 0<q<1$, $a, b \in{R}$.\\
	
	The solution of (\ref{19}) using successive approximation is
	
	\begin{equation}
	y(t) =  \sum_{m=0}^\infty \frac{t^{\alpha m}}{\Gamma{(\alpha m + 1)}}\prod_{j=0}^{m-1}\left(a+bq^{\alpha j}\right).\label{20}
	\end{equation}
	\begin{thm}
	If $0<q<1$, then the power series (\ref{20}) is convergent for all finite values of $t$.
	\end{thm}
\begin{thm}
If $ 0<q<1$, $a,b\ge 0$, then 
\begin{equation*}
 E_\alpha{(at^\alpha)}\le y(t) =  \sum_{m=0}^\infty \frac{t^{\alpha m}}{\Gamma{(\alpha m + 1)}}\prod_{j=0}^{m-1}\left(a+bq^{\alpha j}\right)\le E_\alpha{((a+b)t^\alpha)},\quad 0 \le t < \infty.
\end{equation*}
\end{thm}	

\subsection{Series Solution of Fractional Order Ambartsumian Equation}
Consider the  fractional order Ambartsumian equation as:
\begin{equation}
	D^\alpha y(t) = -y(t)+\frac{1}{q}y\left(\frac{t}{q}\right),\quad y(0)= 1\label{24}
\end{equation}
where $q>1$ and is constant for the given model.

	The solution of (\ref{24}) using successive approximation is
	
	\begin{equation}
	y(t) =  \sum_{m=0}^\infty \frac{t^{\alpha m}}{\Gamma{(\alpha m + 1)}}\prod_{j=0}^{m-1}\left(\frac{1}{q^{1+\alpha j}}-1\right).\label{25}
	\end{equation}
	\begin{thm}
	If $q>1$, then the power series (\ref{25}) is convergent for all finite values of $t$.
	\end{thm}


\section{System of fractional order differential equations with proportional delay}
Consider the initial value problem (IVP)
\begin{eqnarray}
D^{\alpha_i}y_i(t)&=&f_i(t, \bar{y}(t), \bar{y}(qt)), 0<\alpha_i\leq 1, 0<q<1\nonumber\\
y_i(0)&=&^iy_0,\quad 1\leq i \leq n,\label{11}
\end{eqnarray}
where $D^{\alpha_i}$ denotes Caputo fractional derivative, $\bar{y}(t)=(y_1(t),y_2(t)\cdots,y_n(t))$,\\ $\bar{y}(qt)=(y_1(qt),y_2(qt)\cdots,y_n(qt))$ and $f=(f_1,f_2\cdots,f_n)$ is a continuous function defined on the rectangle
\begin{equation}
R=\{|t|\leq a,|y_i(t)-^iy_0|\leq b_i,|y_i(qt)-^iy_0|\leq b_i, a>0,\, b_i>0, 1\leq i \leq n\}\nonumber.
\end{equation}

\begin{thm}
A function $\bar\phi$ is a solution of the IVP (\ref{11}) on an interval $I$ if and only if it is a solution of the integral equation 
\begin{equation}
y_i(t)=^iy_0+\int_{0}^{t} \frac{(t-x)^{\alpha_i-1}}{\Gamma(\alpha_i)}f(x, \bar{y}(x), \bar{y}(qx))dx \quad \text{on}\quad I, 
\end{equation}
where $\bar{\phi}_m=\left(^1\phi_m, ^2\phi_m,\cdots, ^n\phi_m\right)$
\end{thm}
\begin{thm}
	Let $||f||=M$ on rectangle R. The successive approximation 
	\begin{eqnarray}
	^i\phi_{0}(t)&=&^iy_0, \quad i=0,1,2,\cdots.\nonumber\\
	^i\phi_{k+1}(t)&=&y_0+\int_{0}^{t} \frac{(t-x)^{\alpha_i-1}}{\Gamma(\alpha_i)}f(x, {\bar\phi_k}(x), {\bar\phi_k}(qx))dx.\quad k=0,1,2,\cdots.\label{12}
	\end{eqnarray}
	 exist and continuous on the interval $I=[-\zeta,\zeta]$, where 
	\begin{equation*}
	\zeta=\text{min}\left\{a, \left(\frac{\Gamma(\alpha_1+1)b_1}{M}\right)^{\frac{1}{\alpha_1}},\cdots, \left(\frac{\Gamma(\alpha_n+1)b_n}{M}\right)^{\frac{1}{\alpha_n}}\right\}.
	\end{equation*}
	If $t$ is in interval $I$ then $\left(t,\bar{y}_{m}(t),\bar{y}_{m}(qt)\right)$ is in rectangle R and $||\bar{y}_{m}(t)-\bar{y}(0)||\leq M\sum_{i=1}^{m}\frac{|t|^{\alpha_i}}{\Gamma(\alpha_i+1)}$, $||\bar{y}_{m}(qt)-\bar{y}(0)||\leq M\sum_{i=1}^{m}\frac{|t|^{\alpha_i}}{\Gamma(\alpha_i+1)}\, \forall m$.
\end{thm}
\begin{thm}
	Let $f$ be a continuous function defined on the rectangle 
	\begin{equation}
R=\{|t|\leq a,|y_i(t)-^iy_0|\leq b_i,|y_i(qt)-^iy_0|\leq b_i, a>0,\, b_i>0, 1\leq i \leq n\}\nonumber.
	\end{equation}
	Suppose $f$ satisfies Lipschitz condition in second and third variable with  Lipschitz constants $L_1$ and $L_2$ such that 
$
	|f(t, \bar{y}(t), \bar{y}(qt))-f(t, \bar{y}(t), \bar{y}(qt))|\leq L_1|\bar y_1(t)-\bar y_2(t)|+L_1|\bar y_1(qt)-\bar y_2(qt)|\nonumber.
	$
	Then the successive approximations (\ref{12}) converges on the interval  $I=[-\zeta,\zeta]$, where $\zeta=\text{min}\left\{a, \left(\frac{\Gamma(\alpha_1+1)b_1}{M}\right)^{\frac{1}{\alpha_1}},\cdots, \left(\frac{\Gamma(\alpha_n+1)b_n}{M}\right)^{\frac{1}{\alpha_n}}\right\}
		$ to a solution of the $\phi$ of the IVP  (\ref{11}) on $I$.
	\end{thm}
	
\subsection{System of Fractional Order Pantograph Equation}
Consider the system of fractional order pantograph equation
\begin{equation}
D^\alpha y(t) = Ay(t)+ By(q t),\quad y(0)= y_0,  \quad 0< \alpha \le 1\label{26}
\end{equation}
where $0<q<1$, $A=\left(a_{ij}\right)_{n\times n}$, $B=\left(b_{ij}\right)_{n\times n}$ and $y=[y_1,y_2,\cdots,y_n]^T$
	The solution of (\ref{16}) using successive approximation is
\begin{eqnarray}
 y(t)&=& \left[\sum_{k=0}^\infty\prod_{j=1}^{k}(A+Bq^{-(k-j)\alpha})\frac{ t^{k\alpha}}{\Gamma (k\alpha+1)}\right]\lambda.\label{27}
\end{eqnarray}
\begin{thm}
	 For $0<q<1$, the  power series (\ref{27})  is  convergent for $t\in{R}$.
\end{thm}


\subsection{ System of Fractional Order Ambartsumian Equations}
In this section, we generalize the Ambartsumian equation (\ref{1}) to the system of fractional order Ambartsumian equations \cite{patade2020series} as: 

\begin{equation}
D^\alpha y(t) = -Iy(t)+By\left(\frac{t}{q}\right),  \quad y(0)=\lambda, \quad 0< \alpha \le 1, \label{2}
\end{equation}
where  $D^\alpha$ denotes Caputo fractional derivative, $I$ is the identity matrix of order $n$,  $1<q$, \\
$y=$$
\begin{bmatrix}
y_1\\
y_2\\
\vdots\\
y_n
\end{bmatrix}
$, 
$\lambda=$
$\begin{bmatrix}
\lambda_1\\
\lambda_2\\
\vdots\\
\lambda_n
\end{bmatrix}$
and 
 $B=$$
\begin{bmatrix}
\frac{1}{q}& 0 & 0 & \cdots & 0\\
0& \frac{1}{q} & 0 & \cdots & 0\\
\vdots& \vdots & \vdots & \ddots & 0\\
0& 0 & 0 & \cdots & \frac{1}{q}
\end{bmatrix}_{n\times n}.
$\\

 Applying SAM to the initial value problem (\ref{2}), we have
 \begin{equation}
  y(t) = y(0) -I J^\alpha y(t)+ B J^\alpha y\left(\frac{t}{q}\right)\label{3}.
 \end{equation}
 Suppose $\phi_k(t)$ be the $k$th approximate solution, where the initial approximate solution is
 taken as
\begin{equation}
\phi_0(t)=\lambda.
\end{equation}

For $k \ge 1$, the recurrent formula as below:
 \begin{equation}
\phi_k (t) =\lambda -I J^\alpha \phi_{k-1} (t)+ B J^\alpha  \phi_{k-1}\left(\frac{t}{q}\right)\label{4}.
\end{equation}
From the recurrent formula, we have 
\begin{eqnarray}
	\phi_1 (t) &=& \lambda -I J^\alpha \phi_{0} (t)+ B J^\alpha  \phi_{0}\left(\frac{t}{q}\right)\nonumber\\
	&=&\lambda -I \frac{\lambda t^\alpha}{\Gamma (\alpha+1)}+ B \frac{\lambda t^\alpha}{\Gamma (\alpha+1)}\nonumber\\
	&=&\left(I +(-I +B)\frac{ t^\alpha}{\Gamma (\alpha+1)}\right)\lambda,\nonumber\\
\phi_2 (t) &=& \lambda -I J^\alpha \phi_{1} (t)+ B J^\alpha  \phi_{1}\left(\frac{t}{q}\right)\nonumber\\
&=&\lambda -I J^\alpha \left[\left(I +(-I +B)\frac{t^\alpha}{\Gamma (\alpha+1)}\right)\lambda\right]+ B J^\alpha  \left(I +(-I +B)\frac{ q^{-\alpha}t^\alpha}{\Gamma (\alpha+1)}\right)\lambda\nonumber\\
&=&\lambda -I \left[\frac{\lambda t^\alpha}{\Gamma (\alpha+1)} +(-I +B)\frac{\lambda t^{2\alpha}}{\Gamma (2\alpha+1)}\right]+ B   \left[\frac{\lambda t^\alpha}{\Gamma (\alpha+1)} +(-I +B)\frac{\lambda q^{-\alpha} t^{2\alpha}}{\Gamma (2\alpha+1)}\right]\nonumber\\
&=&\left[I +(-I +B)\frac{ t^\alpha}{\Gamma (\alpha+1)} +  (-I +Bq^{-\alpha})(-I +B)\frac{ t^{2\alpha}}{\Gamma (2\alpha+1)}\right]\lambda,\nonumber
\end{eqnarray}

\begin{eqnarray}
\phi_3 (t)&=&\left[I +(-I +B)\frac{ t^\alpha}{\Gamma (\alpha+1)} +  (-I +Bq^{-\alpha})(-I +B)\frac{ t^{2\alpha}}{\Gamma (2\alpha+1)}\right.\nonumber\\
&&\left.+(-I +Bq^{-2\alpha})(-I +Bq^{-\alpha})(-I +B)\frac{ t^{3\alpha}}{\Gamma (3\alpha+1)}\right]\lambda,\nonumber\\
&&\cdots,\nonumber\\
\phi_k (t)&=& \left[I + \sum_{m=1}^k\prod_{j=1}^{m}(-I+Bq^{-(m-j)\alpha})\frac{ t^{m\alpha}}{\Gamma (m\alpha+1)}\right]\lambda\nonumber
\end{eqnarray}
As $k\rightarrow\infty$,$\quad \phi_k (t)\rightarrow y(t)$
\begin{eqnarray}
y(t)&=& \left[I + \sum_{k=1}^\infty\prod_{j=1}^{k}(-I+Bq^{-(k-j)\alpha})\frac{ t^{k\alpha}}{\Gamma (k\alpha+1)}\right]\lambda.\nonumber
\end{eqnarray}
If we set $\prod_{j=1}^{k}(-I+Bq^{(k-j)\alpha})=I$, for $k=0$,
then

\begin{eqnarray}
 y(t)&=& \left[\sum_{k=0}^\infty\prod_{j=1}^{k}(-I+Bq^{-(k-j)\alpha})\frac{ t^{k\alpha}}{\Gamma (k\alpha+1)}\right]\lambda.\label{5}
\end{eqnarray}

\begin{thm}
	 For $q>1$, the  power series 
	\begin{eqnarray}
	y(t)&=& \left[\sum_{k=0}^\infty\prod_{j=1}^{k}(-I+Bq^{-(k-j)\alpha})\frac{ t^{k\alpha}}{\Gamma (k\alpha+1)}\right]\lambda\nonumber
	\end{eqnarray}
 is  convergent for $t\in{R}$.
\end{thm}
\textit{\textbf{Proof:}}$\quad$
Result follows immediately by ratio test \cite{apostol1958mathematical}.
\section{Illustrative Examples}\label{exampl}
\textbf{Example 1.} Consider the  non-linear differential equations with proportional delay \cite{evans2005adomian,rangkuti2012exact,alomari2009solution,ratib2013optimal}
\begin{equation}
\frac{dy(t)}{dt} = 1 - 2 y^2\left(\frac{t}{2}\right),\quad y(0) = 0. \label{21}
\end{equation}

The corresponding integral equation is
\begin{equation}
y(t)= \int_{0}^{t}\left(1-2u^2\left(\frac{t}{2}\right)\right) dx.
\end{equation}
By using successive approximation method  (\ref{22}), we obtain
\begin{eqnarray}
\phi_0(t) &=& 0,\nonumber\\
\phi_1(t) &=&t,\nonumber\\
\phi_2(t) &=& t-\frac{t^3}{6},\nonumber\\
\phi_3(t) &=& t-\frac{t^3}{6}+\frac{t^5}{120}-\frac{t^7}{8064},\nonumber\\
\phi_4(t) &=& t-\frac{t^3}{6}+\frac{t^5}{120}-\frac{t^7}{5040}+\frac{61 t^9}{23224320}-\frac{67 t^{11}}{3406233600}+\frac{t^{13}}{12881756160}-\frac{t^{15}}{7990652436480},\nonumber\\
\phi_5(t) &=& t-\frac{t^3}{6}+\frac{t^5}{120}-\frac{t^7}{5040}+\frac{61 t^9}{23224320}-\cdots-\frac{t^{31}}{1062664199886151693758358595882188800},\nonumber
\end{eqnarray}
and so on.\\
The exact solution of Eq.(\ref{21}) is \, $y(t) = \sin t$.\\
The 5-term solutions of Eq.(\ref{21}) using Adomian decomposition method (ADM) \cite{evans2005adomian}, variational iteration method (VIM) \cite{rangkuti2012exact}, homotopy analysis method (HAM) \cite{alomari2009solution}, optimal homotopy asymptotic method (OHAM) \cite{ratib2013optimal}   are same and is given by
\begin{eqnarray}
y(t) &=& t-\frac{t^3}{6}+\frac{t^5}{120}-\frac{t^7}{5040}+\frac{t^9}{362880}-\frac{t^{11}}{39916800}+\frac{t^{13}}{6227020800}\nonumber\\
&&-\frac{t^{15}}{1307674368000}+\frac{t^{17}}{355687428096000}.\label{n1.4}
\end{eqnarray}
The 4-term OHAM solution \cite{ratib2013optimal} of Eq.(\ref{21}) is
\begin{eqnarray}
y(t) &=& t-0.166665 t^3+0.00832857 t^5 - 0.000192105 t^7.
\end{eqnarray}
We compare $5^{th}$ approximation solution (SAM) and 5-term solutions ( ADM, VIM, HAM) with  exact solution   in Fig.(1) and $4^{th}$ approximation solution (SAM) with 4-term solution (OHAM) in Fig.(2). The absolute errors in computation are shown in Figs.(3)-(4). It can be observed that SAM solution is better than the solution obtained by using other  methods.

\begin{tabular}{c}
\includegraphics[scale=0.5]{fig1.pdf}\\
Fig.1:Comparison of SAM, ADM/VIM/HAM solutions with exact solution of Eq.(\ref{21}).\\
\end{tabular}

\begin{tabular}{c}
\includegraphics[scale=0.5]{fig2.pdf}\\
Fig.2:Comparison of SAM, OHAM solutions with exact solution of Eq.(\ref{21}).\\
\end{tabular}


Remark: The ADM/VIM/HAM and OHAM solutions  given in \cite{evans2005adomian,rangkuti2012exact,alomari2009solution,ratib2013optimal} are considered only for the interval $[0,1]$ . Here we have successfully extended the solution using SAM in intervals $[0,8]$. 


\begin{tabular}{c}
\includegraphics[scale=0.5]{fig3.pdf}\\
Fig.3:Comparison of absolute errors in SAM and ADM/VIM/HAM solutions.\\
\end{tabular}

\begin{tabular}{c}
\includegraphics[scale=0.5]{fig4.pdf}\\
Fig.4:Comparison of absolute errors in SAM and OHAM solutions.\\
\end{tabular}

\section{Conclusions}\label{concl}
  In this paper, we solved non-linear differential equations with proportional delay using the successive approximation method (SAM). The existence, uniqueness, and stability theorems for differential equations with proportional delay are presented. The convergence results are derived by using the Lipschitz condition. A generalization of fractional order and a system of fractional order cases are also presented. The series solution of the pantograph equation and Ambersumian equation is obtained using SAM. Finally, we illustrated the effectiveness of SAM through an example.

\section{References}
\begin{thebibliography}{99}

\bibitem{bhalekar2011predictor}
SACHIN Bhalekar and VARSHA Daftardar-Gejji.
\newblock A predictor-corrector scheme for solving nonlinear delay differential
  equations of fractional order.
\newblock {\em Journal of Fractional Calculus and Applications}, 1(5):1--9,
  2011.

\bibitem{daftardar2015solving}
Varsha Daftardar-Gejji, Yogita Sukale, and Sachin Bhalekar.
\newblock Solving fractional delay differential equations: a new approach.
\newblock {\em Fractional Calculus and Applied Analysis}, 18(2):400--418, 2015.  
  
\bibitem{iqbal2015modified}
Muhammad~Asad Iqbal, Umer Saeed, and Syed~Tauseef Mohyud-Din.
\newblock Modified laguerre wavelets method for delay differential equations of
  fractional-order.
\newblock {\em Egyptian journal of basic and applied sciences}, 2(1):50--54,
  2015.  
  
\bibitem{ali2019chebyshev}
Khalid~K Ali, Mohammad A~Abd El~Salam, and Emad~M Mohamed.
\newblock Chebyshev operational matrix for solving fractional order
  delay-differential equations using spectral collocation method.
\newblock {\em Arab Journal of Basic and Applied Sciences}, 26(1):342--353,
  2019.

\bibitem{sabermahani2020fractional}
Sedigheh Sabermahani, Yadollah Ordokhani, and Sohrab-Ali Yousefi.
\newblock Fractional-order fibonacci-hybrid functions approach for solving
  fractional delay differential equations.
\newblock {\em Engineering with Computers}, 36(2):795--806, 2020.

\bibitem{bhalekar2013stability}
Sachin~B Bhalekar.
\newblock Stability analysis of a class of fractional delay differential
  equations.
\newblock {\em Pramana}, 81(2):215--224, 2013.

\bibitem{bhalekar2016stability}
Sachin Bhalekar.
\newblock Stability and bifurcation analysis of a generalized scalar delay
  differential equation.
\newblock {\em Chaos: An Interdisciplinary Journal of Nonlinear Science},
  26(8):084306, 2016.
  
\bibitem{kaslik2012analytical}
Eva Kaslik and Seenith Sivasundaram.
\newblock Analytical and numerical methods for the stability analysis of linear
  fractional delay differential equations.
\newblock {\em Journal of Computational and Applied Mathematics},
  236(16):4027--4041, 2012.


\bibitem{bhalekar2019analysis}
Sachin Bhalekar.
\newblock Analysis of 2-term fractional-order delay differential equations.
\newblock In {\em Fractional Calculus and Fractional Differential Equations},
  pages 59--75. Springer, 2019.


\bibitem{bhalekar2011fractional}
Sachin Bhalekar, Varsha Daftardar-Gejji, Dumitru Baleanu, and Richard Magin.
\newblock Fractional bloch equation with delay.
\newblock {\em Computers \& Mathematics with Applications}, 61(5):1355--1365,
  2011.

\bibitem{rihan2020dynamics}
FA~Rihan and G~Velmurugan.
\newblock Dynamics of fractional-order delay differential model for
  tumor-immune system.
\newblock {\em Chaos, Solitons \& Fractals}, 132:109592, 2020.

\bibitem{latha2017fractional}
V~Preethi Latha, Fathalla~A Rihan, R~Rakkiyappan, and G~Velmurugan.
\newblock A fractional-order delay differential model for ebola infection and
  cd8+ t-cells response: stability analysis and hopf bifurcation.
\newblock {\em International Journal of Biomathematics}, 10(08):1750111, 2017.

\bibitem{rihan2021fractional}
Fathalla~A Rihan.
\newblock Fractional-order delay differential equations with predator-prey
  systems.
\newblock In {\em Delay Differential Equations and Applications to Biology},
  pages 211--232. Springer, 2021.
  
  
\bibitem{ockendon1971dynamics}
John~Richard Ockendon and Alan~B Tayler.
\newblock The dynamics of a current collection system for an electric
  locomotive.
\newblock {\em Proceedings of the Royal Society of London. A. Mathematical and
  Physical Sciences}, 322(1551):447--468, 1971.  
  
\bibitem{buhmann1993stability}
Martin Buhmann and Arieh Iserles.
\newblock Stability of the discretized pantograph differential equation.
\newblock {\em Mathematics of Computation}, 60(202):575--589, 1993.

\bibitem{bhalekar2017series}
Sachin Bhalekar and Jayvant Patade.
\newblock Series solution of the pantograph equation and its properties.
\newblock {\em Fractal and Fractional}, 1(1):16, 2017.


\bibitem{bhalekar2016analytical}
Sachin Bhalekar and Jayvant Patade.
\newblock Analytical solutions of nonlinear equations with proportional delays.
\newblock {\em Appl. Comput. Math}, 15(3):331--345, 2016.


\bibitem{patade2017analytical}
Jayvant Patade and Sachin Bhalekar.
\newblock On analytical solution of ambartsumian equation.
\newblock {\em National Academy Science Letters}, 40:291--293, 2017.

\bibitem{patade2017analytical1}
Jayvant Patade and Sachin Bhalekar.
\newblock Analytical solution of pantograph equation with incommensurate delay.
\newblock {\em Physical Sciences Reviews}, 2(9), 2017.

\bibitem{coddington2012introduction}
Earl~A Coddington.
\newblock {\em An introduction to ordinary differential equations}.
\newblock Courier Corporation, 2012.

\bibitem{kilbas2006theory}
Anatoli{\u\i} Kilbas.
\newblock {\em Theory and applications of fractional differential equations} \newblock elsevier, 2006.

\bibitem{deng2006stability}
Weihua Deng, Yujiang Wu, and Changpin Li.
\newblock Stability analysis of differential equations with time-dependent
  delay.
\newblock {\em International Journal of Bifurcation and Chaos},
  16(02):465--472, 2006.

\bibitem{ambartsumian1944fluctuation}
VA~Ambartsumian.
\newblock On the fluctuation of the brightness of the milky way.
\newblock In {\em Dokl. Akad. Nauk SSSR}, volume~44, pages 244--247, 1944.

\bibitem{patade2020series}
Jayvant Patade.
\newblock Series solution of system of fractional order ambartsumian equations:
  Application in astronomy.
\newblock {\em arXiv preprint arXiv:2008.04904}, 2020.

\bibitem{apostol1958mathematical}
Tom~M Apostol and CM~Ablow.
\newblock Mathematical analysis.
\newblock {\em Physics Today}, 11(7):32, 1958.

\bibitem{evans2005adomian}
David~J Evans and KR~Raslan.
\newblock The adomian decomposition method for solving delay differential
  equation.
\newblock {\em International Journal of Computer Mathematics}, 82(1):49--54,
  2005.

\bibitem{rangkuti2012exact}
Yulita~Molliq Rangkuti and MSM Noorani.
\newblock The exact solution of delay differential equations using coupling
  variational iteration with taylor series and small term.
\newblock {\em Bulletin of Mathematics}, 4(01):1--15, 2012.

\bibitem{alomari2009solution}
AK~Alomari, Mohd Salmi~Md Noorani, and R~Nazar.
\newblock Solution of delay differential equation by means of homotopy analysis
  method.
\newblock {\em Acta Applicandae Mathematicae}, 108:395--412, 2009.


\bibitem{ratib2013optimal}
N~Ratib~Anakira, AK~Alomari, and Ishak Hashim.
\newblock Optimal homotopy asymptotic method for solving delay differential
  equations.
\newblock {\em Mathematical Problems in Engineering}, 2013, 2013.



%\bibitem{daftardar2006iterative}
%Varsha Daftardar-Gejji and Hossein Jafari.
%\newblock An iterative method for solving nonlinear functional equations.
%\newblock {\em Journal of Mathematical Analysis and Applications},
%  316(2):753--763, 2006.

%\bibitem{rabiei2019solving}
%Kobra Rabiei and Yadollah Ordokhani.
%\newblock Solving fractional pantograph delay differential equations via
 % fractional-order boubaker polynomials.
%\newblock {\em Engineering with Computers}, 35(4):1431--1441, 2019.

\end{thebibliography}

\end{document}

