\UseRawInputEncoding


\documentclass[journal=nalefd,email=True]{achemso}


\usepackage{graphicx}
\usepackage{latexsym}
\usepackage{amsmath}
\usepackage{amssymb}
\usepackage{amsfonts}
\usepackage{color}
\usepackage{bm}% bold math
\usepackage{verbatim}
\usepackage{pagecolor,lipsum}
\usepackage{float}
\usepackage{hyperref}
\usepackage[normalem]{ulem}
\hypersetup{colorlinks=true,urlcolor=blue,linkcolor=blue,citecolor=blue}
\usepackage{comment}
\usepackage{soul,xcolor}
\usepackage{mhchem}

\usepackage{stfloats}


\SectionNumbersOn
\renewcommand{\thefigure}{S\arabic{figure}}

\renewcommand{\thesection}{\Roman{section}}

\captionsetup[figure]{labelfont=bf, labelsep=period}


\newcommand*\mycommand[1]{\texttt{\emph{#1}}}
\usepackage{xcolor}

\definecolor{MS-color}{RGB}{128,0,128}
\newcommand{\MS}[1]{\textcolor[named]{MS-color}{{\bf MS: #1 }}}
\newcommand{\MSedit}[1]{{\color{MS-color}#1}}

\setstcolor{red}



\author{Yao Jungxiang}
\author{Remko Fermin}
\author{Kaveh Lahabi}
\author{Jan Aarts}
\affiliation{Huygens-Kamerlingh Onnes Laboratory, Leiden University, P.O. Box 9504, 2300 RA Leiden, The Netherlands.}
%
\email{aarts@physics.leidenuniv.nl}

\raggedbottom


\title[Triplet Supercurrents in half-metallic ferromagnets]{Triplet supercurrents in lateral Josephson junctions with a half-metallic ferromagnet}


\begin{document}


\section{Characterization of the epitaxy of LSMO films}

LSMO films with a thickness of 40~nm were grown on LSAT substrates in an off-axis sputtering system. We employed different methods to examine and characterize the epitaxy of LSMO. Atomic force microscopy (AFM) was used to map the morphology of the films. As shown in Fig.\ref{disk-figs1}a, clear atomic terraces were observed. Furthermore, x-ray diffraction (XRD) was used to verify the epitaxial growth of LSMO free of crystalline defects (Fig.\ref{disk-figs1}b). Also, we measured the temperature-dependent resistivity of LSMO films using a Van de Pauw method. By calculating the temperature derivative of resistivity, The Curie temperature (T$_c$) was determined to be $\sim$ 362~K (inset in Fig.\ref{disk-figs1}c), consistent with the bulk value. The magnetic properties were measured by vibrating sample magnetometry (Fig.\ref{disk-figs1}d). The coercive field was about 2.5~mTe, and the saturation magnetization was calculated to be $\sim$ 3.8~$\mu_B / f.u.$, in agreement with the theoretical value. Therefore, we conclude that the LSMO films used inthis work are of high quality.     

\begin{figure}[H]
\centering
\includegraphics[width=\textwidth]{FigS1.pdf}
\caption{Property of epitaxial LSMO film. (a) AFM image of morphology of LSMO film. (b) XRD analysis at low-angel region. (c) Temperature-dependent resistivity characteristics. The inset shows the temperature derivative of resistivity to determine the Curie temperature T$_c$ $\sim$ 362~K. (d) Magnetization $vs$ field curve obtained at 50~K. The inset is the full-range measurement with a field up to 3 T.}\label{disk-figs1}
\end{figure}


\newpage
\section{ Junctions with shallow and deep trenches}

\begin{figure}[H]
\centering
\includegraphics[width=\textwidth]{FigS3.pdf}
\caption{Characterizing NbTi/LSMO junctions with shallow and deep trenches. Front-view of SEM images of junctions with shallow (a) and deep (d) trenches. (b) and (e) corresponding RT curves. The inset in (e) is the magnification of the low-temperature region to clarify the second transition temperature. At 5.5~K, the measured SQI pattern of the junction with a shallow trench is Fraunhofer-like, while the junction with a deep trench exhibits a SQUID-like SQI pattern at 2.5~K, indicating rim supercurrents only appear in a disk-shaped ferromagnetic weak link. Note that a shallow trench means the weak link is NbTi.}\label{disk-figs2}
\end{figure}

\newpage\section{Analyzing SQI patterns}
\begin{figure}[H]
\centering
\includegraphics[width=\textwidth]{FigS4.pdf}
\caption{Plot of the extracted critical current with a resistance criterion for the disk-shaped (a), square-shaped (b), and bar-shaped (c) NbTi/LSMO junctions, corresponding to Fig.\textcolor{blue}{3} in the main text.}\label{disk-figs3}
\end{figure}

\begin{figure}[H]
\centering
\includegraphics[width=\textwidth]{FigS5.pdf}
\caption{Measured SQI patterns with constant IP field 10~mT (a) and 100~mT (b) along the trench at 4.1~K. No change in both the amplitude and period of the SQI patterns is observed. The shift of the central peak with respect to the zero field is due to the misalignment between the sample and IP fields. (c) and (d) are the simulated magnetization states, correspondingly.}\label{disk-figs4}
\end{figure}

\begin{figure}[H]
\centering
\includegraphics[width=\textwidth]{FigS6.pdf}
\caption{(a) Top-view SEM image of an irregular disk. (b) SQI pattern measured at 4.1~K. The red curve is the plot of the critical current that is extracted with a resistance criterion of 0.2~$\Omega$. The inset shows the RT curve of the junction. (c) SQI pattern with sweeping current positively and negatively. (d) Simulated critical current as a function of magnetic flux quanta. (e) Fourier analysis on the SQI pattern (b) with a resistance criterion. (f) SQI pattern in the presence of IP 200~mT field.}\label{disk-figs5}
\end{figure}

\section{A disk-shaped NbTi/LSMO junction with a flat side}

Having seen no effect of the IP fields on the triplet supercurrents (Fig.\textcolor{blue}{4} in the main text), we reshape the LSMO disk by cutting off one side and thus acquire an irregular geometry of the LSMO-based junction (Fig.\ref{disk-figs5}). Consequently, the pure magnetic vortex will not be the ground state, and local stray fields can occur. In Fig.\ref{disk-figs5}b, the obtained SQI pattern becomes quite abnormal with non-zero minima. By determining both the positive and the negative critical current, we see that the SQI pattern is  asymmetric (Fig.\ref{disk-figs5}c).  Moreover, the ratio between the period of the central peak and that of the first lobe has increased significantly, to $\sim$ 1.59. B�rcs�k $et~al.$\cite{2019Borcsok} modeled the complex Fraunhofer patterns arising from a non-homogeneous magnetization inside the barrier, in particular when it is larger at the edges of a magnetic Josephson junction. Using Eq9 in Ref\cite{2019Borcsok}, we show a qualitative calculation in Fig.\ref{disk-figs5}d with $p$ = 0.2, $Q$ = -1, and $q$ = 3, though a quantitative fitting is not obtained due to the irregular geometry of this junction. The simulated curve is analogous to the measured critical current upon the out-of-plane field (red curve in Fig.\ref{disk-figs5}b). Interestingly, we still see rim supercurrents on one side in this case. On the other side, the rim supercurrents are largely suppressed, according to the Fourier analysis(Fig.\ref{disk-figs5}e). We again apply IP fields ($\sim$ 200~mT) to saturate the magnetization of this junction. As shown in Fig.\ref{disk-figs5}f, the measured SQI pattern does not change, indicating the transport of triplet supercurrents is intrinsic in the variously shaped NbTi/LSMO junctions, regardless of the magnetization states.

\section{Determining $I_c$ and calculating $E_{Th}$}

At high temperatures, all $IV$ curves have pronounced rounding features at $I \sim I_c$ due to phase slippage, leading to ambiguity in determining $I_c$.\cite{2021Blom} Following Ambegaokar and Halperin,\cite{1969AH} we calculate the IV curves analytically giving

\begin{equation}\label{disk-eq1}
    V = \frac{2I_cR_N}{\gamma_0}\frac{e^{\pi\gamma_0i}-1}{e^{\pi\gamma_0i}}\left \{ \int_0 ^{2\pi}e^{-\pi\gamma_0\varphi/2}I_0 (\gamma_0sin\frac{\varphi}{2})d\varphi\right\}^{-1}
\end{equation}
where $\gamma_0 = \Phi_0I_c/\pi k_BT$, $i = I/I_c$. $I_c$ is the critical current, $R_N$ is the normal resistance, $I_0$ represents a modified Bessel function. The simulated results are shown in Fig.\ref{disk-figs6}b. As $\gamma_0$ becomes large enough, meaning the Josephson coupling energy is comparable to the thermal energy, the rounding feature is significantly suppressed. Shifting the baselines of the $IV$ curves to zero by subtracting the minimum of each $IV$ curve individually, we then fit the measured $IV$ to Eq\ref{disk-eq1} and determine $I_c$ analytically (Fig.\ref{disk-figs6}c). We obtain an average $R_N~\approx 0.8~\Omega$. The fitted $I_c$ is slightly larger than the extracted $I_c$ with a resistance criterion. 

\begin{figure}[ht!]
\centering
\includegraphics[width=\textwidth]{FigS7.pdf}
\caption{Determining $I_c$ and calculating $E_{Th}$. (a) Raw $IV$ curves. (b) Simulated IV curves based on the AH theory, Eq.~\ref{disk-eq1}, for different values of $\gamma_0 = \Phi_o I_c / (\pi k_B T)$. (c) Measured $IV$ curves at high temperatures and the corresponding fit (black dashed line) using Eq\ref{disk-eq1}. (d) A plot of the fitted $I_c$ as a function of temperature and fits using Eq.~\ref{disk-eq2} (cyan curve) and Eq.\ref{disk-eq3} (magenta curve).}\label{disk-figs6}
\end{figure}


Next, we discuss the $I_c$ versus temperature. In the main text, we argued that a description that assumes the diffusive and long regime of a mesoscopic junction, such as used in Refs.~\cite{2010Anwar,2001Dubos,2022Sanchez}, is not valid. Here, a diffusive and long junction means $\ell_H < d$ and $\Delta > E_{Th}$. Unlike the YBCO/LSMO system, in which $k_{B}T \gg E_{Th}$,\cite{2022Sanchez} the second transition temperature of the NbTi/LSMO junction is quite low $\sim 5.2~K$. First we therefore fit the temperature dependence of $I_c$ to Eq.~\ref{disk-eq1} in the low-temperature limit ($k_{B}T \ll E_{Th}$), according to Ref\cite{2001Dubos}, 

\begin{equation}\label{disk-eq2}
    \frac{eR_NI_c}{E_{Th}} = a(1-be^{-aE_{Th}/3.2k_BT})
\end{equation}
where the coefficients a and b are 10.82 and 1.30, respectively. $k_B$ is the Boltzmann constant, and $T$ represents temperature. The result of the fit is shown in Fig.~\ref{disk-figs6}d (cyan line) and at first sight looks good. However, the resulting $E_{Th} \approx34.6~\mu$eV, is much smaller than $k_B T$ ($\sim$ 353~$\mu$eV at 4.1~K). The fitted $R_N$ is 0.37~$\Omega$ and unreasonable, in view of both the residual resistance in the RT curve (Fig.\textcolor{blue}{2} in the main text) and the fitted data with the AH theory.  

Therefore, we consider the high-temperature case (k$_BT \gg E_{Th}$) and fit the temperature-dependent $I_c$ to  

\begin{equation}\label{disk-eq3}
    eR_NI_c = 64\pi k_BT\sum_{n = 0}^{\infty}\frac{L}{L_{\omega_n}}\frac{\Delta^2\exp^{-L/L_{\omega_n}}}{[\omega_n+\Omega_n+\sqrt{2(\Omega_n^2+\omega_n \Omega_n)}]^2}
\end{equation}
where $L$ is the length of the junction ($d$ in the main text), $L_{\omega_n} = \sqrt{\hbar D/2\omega_n}$, $\omega_n = (2n+1)\pi k_BT$ is the Matsubara frequency, $\Omega_n = \sqrt{\Delta^2 + \omega_n^2}$. The fitting curve (magenta) is plotted in Fig.~\ref{disk-figs6}d. From this fit, we find $E_{Th}$ to be about 112~$\mu eV$, the fitted $\Delta$ is 788~$\mu eV$, and $T_c$ is 5.27~K. The ratio $E_{Th}/\Delta(0)$ is therefore 0.14. The fitted R$_N$ is about 0.87~$\Omega$, coincident with the fitting results with the AH theory. According to $\Delta(0) = 1.764k_BT_c$,\cite{1957BCStheory} we obtain $\Delta(0) \approx 790~\mu eV$, which is in agreement with the fitted $\Delta$. In order to show better what the high-temperature looks like, we simulate the temperature-dependent characteristic $I_c R_N$ voltage based on Eq.\ref{disk-eq3} and plot $I_c$ (in appropriate units) versus $T/T_c$ for various values of $E_{Th}/\Delta(0)$ in Fig.~\ref{disk-figs7}. The case of our fit is the green line (a ratio of 0.15), clearly starting off quadratically. The plateau is reached at $T/T_c \approx0.6$, corresponding to $\sim 300$~mK in the experiment. \\

We still argue, however, that this description is unphysical. In the diffusive regime we have $\xi_F = \sqrt{\hbar D/2\pi k_B T_c} \approx$ 19~nm with $T_c$ = 5.5~K. Therefore, the junction length $d$ nearly equals $\xi_F$. Then, $E_{Th} = \hbar D/L^2 \approx$ 2.6~meV, which is clearly larger than $\Delta \approx$ 0.9~meV. In other words, there is a strong discrepancy between the calculated (diffusive) value of $E_{Th}$ and the fitted result. We conclude, not surprisingly, that the long diffusive junction limit cannot be valid. Instead, given $\ell_H < d$ and $E_{Th} > \Delta$, we are in the short regime, where our $I_c$ data are (well) well described by $(1-T/T_c)^2$, as seen in Fig.\textcolor{blue}{5} in the main text. Further quantitative analysis of $I_c(T)$ of half-metallic Josephson junction may need rigorous theoretical study.


\begin{figure*}[ht]
\centering
\includegraphics[width=\textwidth]{FigS8.pdf}
\caption{Simulated temperature dependence of the product of $eI_c R_N$ at various ratios of $E_{Th}/\Delta(0)$ based on Eq.~\ref{disk-eq3}. The inset shows the universal $\Delta(T)/\Delta(0)$ as a function of $T/T_c$, in which $\Delta(0) = 1.764 k_B T_c$.\cite{1957BCStheory}}\label{disk-figs7}
\end{figure*}


\providecommand{\latin}[1]{#1}
\makeatletter
\providecommand{\doi}
  {\begingroup\let\do\@makeother\dospecials
  \catcode`\{=1 \catcode`\}=2 \doi@aux}
\providecommand{\doi@aux}[1]{\endgroup\texttt{#1}}
\makeatother
\providecommand*\mcitethebibliography{\thebibliography}
\csname @ifundefined\endcsname{endmcitethebibliography}
  {\let\endmcitethebibliography\endthebibliography}{}
\begin{mcitethebibliography}{8}
\providecommand*\natexlab[1]{#1}
\providecommand*\mciteSetBstSublistMode[1]{}
\providecommand*\mciteSetBstMaxWidthForm[2]{}
\providecommand*\mciteBstWouldAddEndPuncttrue
  {\def\EndOfBibitem{\unskip.}}
\providecommand*\mciteBstWouldAddEndPunctfalse
  {\let\EndOfBibitem\relax}
\providecommand*\mciteSetBstMidEndSepPunct[3]{}
\providecommand*\mciteSetBstSublistLabelBeginEnd[3]{}
\providecommand*\EndOfBibitem{}
\mciteSetBstSublistMode{f}
\mciteSetBstMaxWidthForm{subitem}{(\alph{mcitesubitemcount})}
\mciteSetBstSublistLabelBeginEnd
  {\mcitemaxwidthsubitemform\space}
  {\relax}
  {\relax}

\bibitem[B\"{o}rcs\"{o}k \latin{et~al.}(2019)B\"{o}rcs\"{o}k, Komori, Buzdin,
  and Robinson]{2019Borcsok}
B\"{o}rcs\"{o}k,~B.; Komori,~S.; Buzdin,~A.~I.; Robinson,~J. W.~A. Fraunhofer
  patterns in magnetic Josephson junctions with non-uniform magnetic
  susceptibility. \emph{Sci Rep} \textbf{2019}, \emph{9}, 5616\relax
\mciteBstWouldAddEndPuncttrue
\mciteSetBstMidEndSepPunct{\mcitedefaultmidpunct}
{\mcitedefaultendpunct}{\mcitedefaultseppunct}\relax
\EndOfBibitem
\bibitem[Blom \latin{et~al.}(2021)Blom, Mechielsen, Fermin, Hesselberth, Aarts,
  and Lahabi]{2021Blom}
Blom,~T.~J.; Mechielsen,~T.~W.; Fermin,~R.; Hesselberth,~M. B.~S.; Aarts,~J.;
  Lahabi,~K. Direct-Write Printing of Josephson Junctions in a Scanning
  Electron Microscope. \emph{ACS Nano} \textbf{2021}, \emph{15}, 322--329\relax
\mciteBstWouldAddEndPuncttrue
\mciteSetBstMidEndSepPunct{\mcitedefaultmidpunct}
{\mcitedefaultendpunct}{\mcitedefaultseppunct}\relax
\EndOfBibitem
\bibitem[Ambegaokar and Halperin(1969)Ambegaokar, and Halperin]{1969AH}
Ambegaokar,~V.; Halperin,~B. Voltage due to thermal noise in the dc Josephson
  effect. \emph{Physical Review Letters} \textbf{1969}, \emph{22}, 1364\relax
\mciteBstWouldAddEndPuncttrue
\mciteSetBstMidEndSepPunct{\mcitedefaultmidpunct}
{\mcitedefaultendpunct}{\mcitedefaultseppunct}\relax
\EndOfBibitem
\bibitem[Anwar \latin{et~al.}(2010)Anwar, Czeschka, Hesselberth, Porcu, and
  Aarts]{2010Anwar}
Anwar,~M.~S.; Czeschka,~F.; Hesselberth,~M.; Porcu,~M.; Aarts,~J. Long-range
  supercurrents through half-metallic ferromagneticCrO2. \emph{Physical Review
  B} \textbf{2010}, \emph{82}\relax
\mciteBstWouldAddEndPuncttrue
\mciteSetBstMidEndSepPunct{\mcitedefaultmidpunct}
{\mcitedefaultendpunct}{\mcitedefaultseppunct}\relax
\EndOfBibitem
\bibitem[Dubos \latin{et~al.}(2001)Dubos, Courtois, Pannetier, Wilhelm, Zaikin,
  and Sch�n]{2001Dubos}
Dubos,~P.; Courtois,~H.; Pannetier,~B.; Wilhelm,~F.~K.; Zaikin,~A.~D.;
  Sch�n,~G. Josephson critical current in a long mesoscopic S-N-S junction.
  \emph{Physical Review B} \textbf{2001}, \emph{63}\relax
\mciteBstWouldAddEndPuncttrue
\mciteSetBstMidEndSepPunct{\mcitedefaultmidpunct}
{\mcitedefaultendpunct}{\mcitedefaultseppunct}\relax
\EndOfBibitem
\bibitem[Sanchez-Manzano \latin{et~al.}(2022)Sanchez-Manzano, Mesoraca,
  Cuellar, Cabero, Rouco, Orfila, Palermo, Balan, Marcano, Sander, Rocci,
  Garcia-Barriocanal, Gallego, Tornos, Rivera, Mompean, Garcia-Hernandez,
  Gonzalez-Calbet, Leon, Valencia, Feuillet-Palma, Bergeal, Buzdin, Lesueur,
  Villegas, and Santamaria]{2022Sanchez}
Sanchez-Manzano,~D. \latin{et~al.}  Extremely long-range, high-temperature
  Josephson coupling across a half-metallic ferromagnet. \emph{Nat Mater}
  \textbf{2022}, \emph{21}, 188--194\relax
\mciteBstWouldAddEndPuncttrue
\mciteSetBstMidEndSepPunct{\mcitedefaultmidpunct}
{\mcitedefaultendpunct}{\mcitedefaultseppunct}\relax
\EndOfBibitem
\bibitem[Bardeen \latin{et~al.}(1957)Bardeen, Cooper, and
  Schrieffer]{1957BCStheory}
Bardeen,~J.; Cooper,~L.~N.; Schrieffer,~J.~R. Theory of superconductivity.
  \emph{Physical review} \textbf{1957}, \emph{108}, 1175\relax
\mciteBstWouldAddEndPuncttrue
\mciteSetBstMidEndSepPunct{\mcitedefaultmidpunct}
{\mcitedefaultendpunct}{\mcitedefaultseppunct}\relax
\EndOfBibitem
\end{mcitethebibliography}



\end{document}


