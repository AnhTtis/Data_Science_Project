\UseRawInputEncoding


\documentclass[journal=nalefd,manuscript=letter,email=True,layout=twocolumn]{achemso}


\usepackage{graphicx}
\usepackage{latexsym}
\usepackage{amsmath}
\usepackage{amssymb}
\usepackage{amsfonts}
\usepackage{color}
\usepackage{bm}% bold math
\usepackage{verbatim}
\usepackage{pagecolor,lipsum}
\usepackage{float}
\usepackage{hyperref}
\usepackage[normalem]{ulem}
\hypersetup{colorlinks=true,urlcolor=blue,linkcolor=blue,citecolor=blue}
\usepackage{comment}
\usepackage{soul,xcolor}
\usepackage{mhchem}

\usepackage{stfloats}


\newcommand*\mycommand[1]{\texttt{\emph{#1}}}
\usepackage{xcolor}

\definecolor{MS-color}{RGB}{128,0,128}
\newcommand{\MS}[1]{\textcolor[named]{MS-color}{{\bf MS: #1 }}}
\newcommand{\MSedit}[1]{{\color{MS-color}#1}}

\setstcolor{red}




\author{Yao Jungxiang}
\author{Remko Fermin}
\author{Kaveh Lahabi}
\author{Jan Aarts}
\affiliation{Huygens-Kamerlingh Onnes Laboratory, Leiden University, P.O. Box 9504, 2300 RA Leiden, The Netherlands.}
%
\email{aarts@physics.leidenuniv.nl}


\title[Triplet Supercurrents in half-metallic ferromagnets]{Triplet supercurrents in lateral Josephson junctions with a half-metallic ferromagnet}

\keywords{Superconductivity, Ferromagnetism, Half metal, Magnetic texture, Triplet Cooper pairs}



\raggedbottom


\begin{document}

\begin{center}
\today
\end{center}


 
\begin{abstract}
Spin triplet supercurrents in half-metallic ferromagnets (HMFs) would be a valuable addition to the toolbox of superconducting spintronics since they promise to be long-range, fully spin-polarized, and with high current density. However,  experimental studies on the subject remain scarce because there are only few HMFs available. Here we report on the generation of triplet supercurrents in mesoscopic lateral Josephson junctions, fabricated with the conventional superconductor NbTi and the HMF La$_{0.7}$Sr$_{0.3}$MnO$_3$ (LSMO), having a typical junction length of 20~nm. We use the electrode geometry as a tool to examine the triplet generator in the HMF and study junctions shaped as bar, square, and disk. In all junctions, we find strong supercurrents and high current densities, that are independent of the junction geometry and are insensitive to the application of in-plane fields, ruling out the magnetic vorticity in the disk geometry as the generator. Analysis of the critical current interference pattern upon application of a perpendicular magnetic field shows that the triplet supercurrent is highly constricted to the rims of the disk devices, yet uniformly distributed in the bar-shaped devices. The appearance of these \textit{rim supercurrents} is tied to the combination of triplet transport and the disk geometry but is not central to the generator mechanism. Next, we analyze the temperature dependence of the critical current and find in its quadratic dependence a strong indication for the triplet nature of the supercurrents. Finally, we find the supercurrent to persist in junctions fabricated from a NbTi/Ag/LSMO trilayer. The results strongly indicate that the magnetic inhomogeneity needed for triplet generation resides in the LSMO layer adjacent to the NbTi/LSMO interface.
\end{abstract}



\section{Introduction}\label{sec1}

Arguably the most promising ferromagnetic (F) materials for superspintronics applications\cite{2015Eschrig,2015Linder,Robinson2021} are the half-metallic ferromagnets (HMF), such as CrO$_2$ and the metallic oxide perovskites La$_{0.7}$X$_{0.3}$MnO$_3$ (X = Sr (LSMO); or Ca (LCMO)). Despite their fully spin-polarized nature, both in CrO$_2$ and in the metallic oxide perovskites long-range triplet (LRT) proximity effect was found, showing high critical currents, and superconducting proximity over extremely long length scales~\cite{2006Keizer,Anwar2010,2012Visani,2022Sanchez}: in CrO$_2$ over a distance of almost 0.5~{\textmu}m~\cite{2006Keizer} and in LSMO even bridging 1~{\textmu}m~\cite{2022Sanchez}. However, the mechanism that converts singlets in the superconductor (S) to triplets in the HMF - regardless whether it concerns CrO$_2$ or the manganite oxides - is still poorly understood. This is contrary to LRT generation in conventional S/F systems, where a spin-active interface is engineered at which spin mixing and spin rotation processes conspire to the formation of equal-spin triplet correlations\cite{2007Houzet,2008Eschrig}. Stack of magnets with non-collinear magnetizations are generally used for the magnetic inhomogeneity\cite{2010Khaire,2010Robinson,2016Singh}. More recently, magnetic vortices\cite{Kalenkov2011,Silaev2009,2017Lahabi,2022Fermin}, domain walls~\cite{Bergeret2001a,Fominov2007,Kalcheim2011,Aikebaier2019} and spin-orbit coupling~\cite{Niu2012,Bergeret2013_New,Bergeret2014a,Alidoust2015,jacobsen2015,Bujnowski2019,Eskilt2019,Silaev2020} were also identified as triplet generators, but they do not explain the results on HMFs. The results on CrO$_2$ without engineered magnetic non-collinearity are hypothesized to result from intrinsic strain-induced magnetic inhomogeneity or grain boundary disorder~\cite{Anwar2011}, but there are no established sources of LRT correlations in the recent experiments on LSMO junctions\cite{2022Sanchez}. Therefore, the intrinsic mechanism for triplet generation in LSMO remains a major open question.\\

\begin{figure}[ht]
\centering
\includegraphics[width=6cm]{L-disk_fig1.pdf}
\caption{Schematic illustration of NbTi/LSMO junctions of different geometry. (a) A disk-shaped junction consisting of NbTi (orange) and LSMO (blue). Due to the shape anisotropy, a stable magnetic vortex forms in the LSMO layer, as indicated. (b) A bar-shaped junction with a uniform magnetic texture. The $x(z)$-axis is perpendicular (parallel) to the trench (both in the sample plane). The out-of-plane direction is the $y$-axis.}\label{disk-fig1}
\end{figure}

Here we study the generation of LRT correlations in junctions of LSMO and NbTi, which is a conventional $s$-wave superconductor with a relatively low transition temperature ($\sim$ 7.5~K in this work) and qualitatively different from the high-$T_c$ and $d$-wave superconductor YBCO used in previous experiments. Inspired by  previous work\cite{2022Fermin} where we showed that a magnetic vortex is capable of generating spin-polarized supercurrents, we use geometry as a tool to examine the role of spin texture in LRT generation (see Figure \ref{disk-fig1}). We fabricated lateral junctions with disk, square, and bar shapes. We find high critical currents in our devices, which we identify as triplet supercurrents, by examining their temperature dependence. Unexpectedly, the triplet supercurrents occur irrespective of the device geometry, a result that rules out spin texture as the LRT generator. However, the current distribution in both devices is quite different. Analysis of the critical current interference patterns upon application of a perpendicular magnetic field shows that in the disk, the supercurrent is highly constricted to its rims, similar to results we obtained on disk junctions made of Nb/Co~\cite{2022Fermin}. For the bars we find a homogeneous distribution over its cross-section. The results show that the triplet generator mechanism and the origin of these \textit{rim supercurrents} in our HMF/NbTi disks are not the same. Finally, we find the supercurrents unaltered in samples fabricated from a NbTi/Ag/LSMO trilayer, thereby ruling out any LRT sources resulting from an emerging interfacial magnetism due to exchange interactions across the NbTi/LSMO interface. We conclude that there is a generator at play due to an intrinsic magnetic inhomogeneity in the LSMO at, or close to, the LSMO/NbTi interface.

 

\section{Results}\label{sec2}

Josephson junction devices were fabricated from bilayers of 60~nm NbTi deposited on 40~nm LSMO grown on an (LaAlO$_3$)$_{0.3}$(Sr$_2$TaAlO$_6$)$_{0.7}$ (LSAT) substrate.  Samples were structured by Focused Ion Beam (FIB). We fabricated disk, square, and bar geometries, the latter with two different aspect ratios (length-between-contacts to width) of 3:1 and 5:1. The typical dimension of all structures was 1~{\textmu}m. Also by FIB, a trench was made in the middle of the devices. The trench width corresponds to the junction length. All junctions in this work have an identical length of d $\sim$ 20~nm. More details on the device fabrication can be found in the Methods section and in Supporting Information (SI) Section \textcolor{blue}{I, II}. \\

\begin{figure*}[t]
\centering
\includegraphics[width=\textwidth]{L-disk_fig2.pdf}
\caption{Electrical transport. (a) Resistance versus temperature curves for disk-shaped (magenta), square-shaped (black), bar-shaped 3:1 (blue), and 5:1 (orange) devices. The grey dashed line is a guide to the eye at 2~$\Omega$. (b) A plot of current (I) - voltage (V) curves taken at different temperatures on the disk-shaped junction. (c) Measured IV curves on the bar-shaped (3:1) junction at different temperatures. }\label{disk-fig2}
\end{figure*}

The resistance $R$ of the differently shaped NbTi/LSMO junctions was measured as a function of temperature $T$, as shown in Fig.\ref{disk-fig2}a. The normal state resistance for all devices is around 20~$\Omega$, except for the square one which is 75~$\Omega$, due to a different contact geometry.  The traces show two transitions with decreasing temperature. The first one corresponds to the superconducting transition of NbTi at $\sim$ 7.5~K, and is followed by a plateau, which corresponds to the resistance of the LSMO weak link. 
As the temperature further decreases, the resistances fully go to zero, with broad tails, indicating that the weak links become superconducting in all junctions, regardless of their geometry and aspect ratio. Since LSMO is a half-metal, the proximity effect must be carried by LRT correlations. \\
%
In Fig.\ref{disk-fig2} we also show current $I$ versus voltage $V$ (IV-characteristics) for different temperatures of the disk-shaped and the (3:1) bar-shaped junctions, measured by sweeping the current back and forth. The data clearly show the onset of voltage at the critical current I$_c$, without visible hysteresis. The multiple steps in the IV curves of the disk junction at higher temperatures result from destroying superconductivity in nanostructured regions in the junction. 

As mentioned, previous work on Nb/Co devices found proximity supercurrents in disk junctions, resulting from the spin texture of the device. Nb/Co-based bar-shaped junctions were never superconducting due to their uniform magnetization. In the case of LSMO-based junctions, we find a temperature-dependent $I_c$ for all junctions, regardless of the device geometry. 
%


\subsection{Spatial supercurrent distribution}\label{subsec3}

To study the relative distribution of critical currents along the junction, we examined the current-phase relation, $i.e$. we recorded I$_c$(B$_{\perp}$)-patterns of the differently shaped junctions at 4.1~K (disk and square) and 3.5 K (3:1 bar). The results (see Fig.\ref{disk-fig3}d-f) show a clear dependence on the sample geometry. For the disk, the I$_c$(B$_{\perp}$)-pattern is two-channel-like, where the side lobes have a period comparable to the central peak, and the I$_c$(B$_{\perp}$) oscillations decay gradually and less fast than the 1/B$_{\perp}$ behavior expected for a single junction. This points to the existence of two superconducting channels in the disk-shaped NbTi/LSMO junction.
%
Contrarily, for the bar (aspect ratio 3:1), we find a typical single-junction Fraunhofer-like pattern. The square-shaped junction is somewhat in between. We extracted the periodicity for each pattern by using a voltage criterion (see (SI) Section \textcolor{blue}{III}, Fig.\textcolor{blue}{S3}). In the case of the disk-shaped junction, the width of the central peak is about 5.91~mT, and that of the first lobe is about 5.04 mT, yielding a ratio of $\sim$ 1.17. For the square-shaped and bar-shaped junctions, we obtain ratios of $\sim$ 1.37 and $\sim$ 1.45, respectively. The difference may be intrinsic in these junctions or correlated with the magnetic structures, as we will discuss later. \\

To convert the I$_c$(B$_{\perp}$)-pattern into the spatial distribution of the critical supercurrent, we employ a Fourier method and reconstruct the supercurrent density along the $z$ axis (along the trench)\cite{1971Dynes,2022Fermin,2023Fermin}. The analysis requires the effective junction length $L_{eff}$, which is often taken as $L_{eff}$ = $d$+2$\lambda_L$. Since the thickness of NbTi electrodes (60 nm) is smaller than their London penetration depth ($\lambda_L$; of order 0.5~{\textmu}m), we adopt the universal limits \( L_{eff} = (1/3.8)*L, (1/2.5)*L, (1/1.842)*L\) for the disk-shaped, square-shaped and bar-shaped junctions, respectively, where $L \approx 1.3~\mu m$ is the junction width\cite{2023Fermin}. We find the supercurrents to be largely constricted to the rims of the device in the disk-shaped LSMO junction (Fig.\ref{disk-fig3}h). This is similar to the disk-shaped Nb/Co junctions, although the width of the channels is larger (measured by the full width at half maximum $\sim$200~nm). However, as the geometry of the junction changes from disk to square to bar (Fig.\ref{disk-fig3}i and j), the density of supercurrent in the middle regions becomes more and more pronounced (Fig.\ref{disk-fig3}h-j). Notably, the control experiment on disk-shaped MoGe/Ag junction also proves singlet supercurrents are distributed homogeneously across the trench, rather than being localized at the rim.\cite{2022Fermin} 

\begin{figure*}[t]%
\centering
\includegraphics[width=\textwidth]{L-disk_fig3.pdf}
\caption{Spatial distribution of supercurrents versus geometry. (a-c) Top-view false-colored scanning electron micrographs of disk-shaped, square-shaped, and bar-shaped (3:1) devices. Here, the color indicates the NbTi/LSMO bilayer and all scale bars are equal to 500 nm. (d-f) Measured I$_c$(B$_{\perp}$)-patterns using a $B_{\perp}$ sweep. The color scale gives the differential resistance. (h-j) The results of Fourier analysis. The red-dashed lines indicate the size of the weak link. }\label{disk-fig3}
\end{figure*}

\subsection{The effect of in-plane fields on the supercurrent distribution}\label{subsec4}

In Nb/Co disk junctions, we found that the proximity effect, and in particular the rim supercurrents, were a consequence of the spin vortex texture. Therefore, the transport characteristics of the device are very sensitive to the exact spin texture of the weak link, for example by moving the vortex core using an in-plane (IP) field. In order to explore the similarity, we apply IP fields on the NbTi/LSMO disk junction to change the magnetic ground state, which is also a magnetic vortex (according to micromagnetic simulations; details in Methods). The results are shown in Fig.\ref{disk-fig4}. Surprisingly, we find no change in the critical current while applying an IP field. Moreover, the I$_c$(B$_{\perp}$)-patterns persist, both when IP fields are applied perpendicular or parallel to the trench. Even in a 200~mT field, for which micromagnetic simulations demonstrate that the disk is fully magnetized, we find hardly any change in the two-channel I$_c$(B$_{\perp}$)-patterns (Fig.\ref{disk-fig4}b,c). Note that we see small shifts in the central peaks with respect to zero out-of-plane field, but these are due to the misalignment between the sample plane and field direction.

We conclude from our IP field experiments that spin texture is not responsible for the proximity effect in our LSMO-based junctions. Specifically, the observed phenomena are very different from the observations in the Nb/Co disk junctions, where changes in spin texture lead to changes in $I_c$ with fields of mT's, and where $I_c$ is fully quenched when the disk becomes homogeneously magnetized. Here, $I_c(0)$ does not even decrease, emphasizing the robustness of the LRT supercurrents and indicating that the spin texture is not the LRT generator. Yet, the occurrence of the rim currents appears to be coupled to the geometry. In order to test this further, we reshaped a disk junction by cutting off one side. Interestingly, rim supercurrents remain on the curved side, but are absent on the flat side (See SI Section \textcolor{blue}{IV}).

\begin{figure*}[ht]%
\centering
\includegraphics[width=\textwidth]{L-disk_fig4.pdf}
\caption{I$_c$(B$_{\perp}$)-patterns recorded on a disk-shaped bilayer junction under simultaneous application of IP fields (a-c) and I$_c$(B$_{\perp}$)-patterns recorded on trilayer junctions (d-f). (a) I$_c$(B$_{\perp}$)-pattern measured at zero IP field. The micromagnetic simulation in the inset demonstrates the LSMO disk has a ground state of a magnetic vortex. I$_c$(B$_{\perp}$)-patterns in the presence of 200~mT perpendicular (b) or parallel (c) to the trench. Accordingly, the insets display the fully-magnetized state in both cases. The color bar represents $z$ components in the micromagnetic simulation. (d) RT curves of disk-shaped (blue) and bar-shaped (4:1) (red) NbTi/Ag/LSMO trilayer junctions. The corresponding I$_c$(B$_{\perp}$)-patterns, displayed in (e) and (f), are SQUID-like and Fraunhofer-like (f) at 4~K.}\label{disk-fig4}
\end{figure*}

Since we ruled out spin texture, we conclude that there is a generator at play at, or close to, the LSMO/NbTi interface. In order to further investigate this, we fabricated junctions with an altered interface, by inserting a thin Ag layer ($\sim$ 10~nm). We find these junctions to retain their Josephson coupling and high critical current (see Fig.\ref{disk-fig4}d-f). Besides, the NbTi/Ag/LSMO disk-junctions exhibit a two-channel-like I$_c$(B$_{\perp}$)-pattern, while the bar-shaped ones show a Fraunhofer pattern, as shown in Fig.\ref{disk-fig4}e and f, respectively.

Finally, we paid special attention to small fields ($\sim$ 10~mT) when small motions of the vortex core are expected. We found no sign of a $0-\pi$ transition as observed in the Nb/Co system. This agrees with the predicted implication that such a transition cannot exist in a HMF\cite{2008Eschrig}. More details are given in SI, Fig~\textcolor{blue}{S4}. \\

\subsection{Dependence of I$_c$ on temperature}\label{subsec5}

One of the promises of supercurrents in HMFs is that the density can be very high, and therefore we measured $I_c(T)$ down to 1.5~K. Although $I_c$ is well-defined at low temperatures, near the critical temperature the $IV$ characteristics become rounded around $I_c$, which results in a slight ambiguity in determining $I_c$. The latter is believed to result from phase slips\cite{2021Blom}. In this regime, we find $I_c$ by fitting the $IV$ curves to a model proposed by Ambegaokar and Halperin (AH)\cite{1969AH}. The values for $I_c$ obtained this way are slightly different from the values extracted with a voltage criterion. More details are given in SI Section \textcolor{blue}{V}; the relevant $IV$ characteristics are given in the SI, Fig.S6. 
Also, we note that the fitting yields an average value for $R_N$ of 0.8~$\Omega$. This is significantly lower than the plateau in the $R(T)$ curve for this sample, which is around 2~$\Omega$, indicating that the interface resistance is not small, and the interface is not very transparent.\\

The results for $I_c(T)$ are summarized Fig.\ref{disk-fig5}. They show a parabolic dependence, and a current density $J_c = I_c/(d*L)$ of about $1.7\times10^{10}$~A/m$^2$ at 1.5~K. If we consider that the currents mostly are confined to the rim of the disk, over a distance of $\sim$~200~nm, $J_c$ is $\sim 1.1\times10^{11}$~A/m$^2$. Such high-density spin-polarized supercurrent holds promise for practical applications in superconducting spintronics.\\

Previously, $I_c(T)$ of HMF junctions were analyzed by an analogy to long diffusive SNS junctions\cite{2010Anwar,2001Dubos}. Here this analogy cannot be made, since the diffusive coherence length is roughly the size of the weak link. To show this, we estimate a mean free path $\ell_H$ of 6.5~nm\cite{2001Nadgorny} using the measured resistivity of LSMO (40~$\mu\Omega$cm). Next, using a Fermi velocity value $v_F \sim 7.4\times10^5$~m/s this leads to a diffusion constant $D = v_F*\ell_H$/3, and using $T_c$~= 5.5~K, we find the diffusive coherence length $\xi_{F} = \sqrt{\hbar D/(2\pi k_B T_c)}$ to be $\approx$~19~nm, which is roughly the size of the weak link.  The same conclusion follows from considering the Thouless energy $E_{Th} = \hbar D/ L^2$, which is about 2.6~meV, clearly larger than the gap $\Delta$, which is 0.9~meV.\\


\begin{figure}[t]%
%\centering
\includegraphics[width=8cm]{L-disk_fig5.pdf}
\caption{$I_c$ as a function of temperature $T$. The inset shows how $I_c$ values are obtained by fitting the IV curve (yellow) to the AH theory (black-dashed line) at 4.41~K, instead of being extracted with a voltage criterion. The extracted values of $I_c$ are plotted by blue dots and well fitted by (1-$t$)$^2$ with $t$ = $T/T_c$ (red line), with $T_c$ = 5.5~K} \label{disk-fig5}
\end{figure}

A theoretical framework for describing HMF-based junctions, which is more suited than the analogy to diffusive SNS junctions, was given in Refs.\cite{2003Eschrig,2008Eschrig}. For the clean case, the strength of $I_c$ is determined by the two spin scattering channels $(\uparrow\downarrow - \downarrow\uparrow)$ and $(\uparrow\uparrow)$ from the superconductor to the half-metallic spin-up band. Close to $T_c$, $I_c(T)$ shows a (1-$t$)$^2$ dependence (with $t=T/T_c$), and a maximum at lower $T$, typically around $t=0.2$. The maximum is robust, and is also predicted for long diffusive junctions\cite{2008Eschrig}, but has never been observed. In the current experiment, the data are well described by (1-$t$)$^2$, using $T_c$ = 5.5~K, see Fig.\ref{disk-fig5}. The experimentally lowest accessible temperature is 1.5~K, or $t = 0.3$, so the issue of the peak in $I_c$ remains open.


\section{Discussion}\label{sec3}

Above, we demonstrated the occurrence of a superconducting proximity effect in LSMO-based junctions of various shapes. In this section, we discuss the possible origin of the effect. 

First note that the superconducting transport is carried by triplet Cooper pairs, as is indicated by the quite pure $(1-t)^2$ dependence of $I_c(t)$. For short junctions with a normal metal interlayer, the expected behavior is $1-t$\cite{1979Likharev}. A quadratic dependence, or an upward curved behavior of $I_c(T)$, can in principle be found in S/N/S structures\cite{1964deGennes}, but that would concern long junctions, whereas our junctions are in the short junction limit. We further discuss this point in SI Section \textcolor{blue}{V}. \\ 

Next, we find that the supercurrent generation is independent of the magnetic texture of the HMF. By applying in-plane fields, we observe the supercurrents to persist, even though the spin texture has been quenched. This suggests that another generator is at play for the triplets than the vortex magnetization pattern. A similar indication comes from the fact that homogeneously magnetized bar-shaped junctions show equally strong supercurrents.\\

Rejecting spin texture as the generator suggests that LRT correlations are instead generated at the NbTi/LSMO interface. This hypothesis is fully in line with research on the YBCO/LCMO and YBCO/LSMO systems. Visani $et~al.$ reported supercurrents through 30~nm of LCMO between two layers of YBCO\cite{2012Visani}. Very recently, Sanchez-Manzano $et~al.$ found LRP effects in micrometer-long lateral junctions of LSMO with YBCO contacts. No engineered spin texture to generate triplets was utilized in either case. A possible source of LRT correlations could result from interfacial magnetism, possibly inhomogeneous in nature. Such mechanism actually exists at the YBCO/LCMO interface, due to exchange interactions over the Cu-O-Mn chains across the interface that lead to magnetic moment on the Cu\cite{2006Chakhalian}. However, since the samples fabricated from a NbTi/Ag/LSMO trilayer retain the same transport characteristics as the bilayer samples, the triplet generation in our NbTi/LSMO system is clearly not due to any orbital hybridization between NbTi and LSMO. Alternatively, this suggests that an intrinsic magnetic inhomogeneity in the LSMO top layer may be the generator. This could be formed in the growth procedure, if, for instance, the oxygen content in the top layer is reduced. That would lead to an increase in the amount of Mn$^{3+}$ and shift the magnetism to a canted ferromagnetic insulating state\cite{1998vasi}. Such a mechanism would also provide an explanation for LRT correlations in the LSMO/YBCO system.\\

Finally, we discuss the relation of the LRT generator to the highly confined triplet supercurrents at the rims of the disk-shaped devices. In the Nb/Co disk junctions studied before, these rim supercurrents were believed to be a direct result of the triplet nature of the supercurrents. Specifically, in the Nb/Co system the generator was identified as resulting from the synergistic effects of an effective SOI (provided by the vortex magnetization) and sample boundaries. This entails that LRT transport in the Nb/Co system can only emerge at the rims of the device, and consequently, rim supercurrents are very sensitive to in-plane fields. Contrarily, in the NbTi/LSMO system, triplets generation is independent of spin texture. Moreover, by modifying the geometry of the LSMO-based junctions, we see rim supercurrents appear in the disk and become weakened in the square and bar. Instead, the above results suggest rim supercurrents result from the combination between exotic triplet transport and the disk-shaped junction geometry, yet the driving mechanism remains to be further studied. We suspect that the difference between the Nb/Co and NbTi/LSMO systems is connected to the high spin polarization of the LSMO, where only one spin state can exist, but also this remains an open question.


\section{Conclusion}\label{sec4}
In summary, we have unambiguously shown spin-polarized supercurrents and Josephson coupling in lateral NbTi/LSMO junctions with different geometries, and examined their origin. Surprisingly, these currents remain robust against large in-plane fields that are able to erase the spin texture, indicating that the spin texture is not relevant for producing such currents. From the unchanged results on devices fabricated from a NbTi/Ag/LSMO trilayer, we conclude that singlet to triplet conversion is not due to some interface coupling. Instead, intrinsic magnetic inhomogeneity in the LSMO top layer may be the key, which would also explain triplet generation in the LSMO/YBCO system. Finally, by performing Fourier analysis on the various I$_c$(B$_{\perp}$)-patterns, we find rim supercurrents in the disk-shaped junctions. The origin of these rim supercurrents is tied to the combination of triplet transport and disk-shaped junction geometry. However, their exact origin is currently left unexplained.

\bigskip
\noindent

%\backmatter

\section{Methods}\label{sec5}

\textbf{Device fabrication:} LSMO (40~nm) was deposited on a (001)-oriented \\
(LaAlO$_3$)$_{0.3}$(Sr$_2$TaAlO$_6$)$_{0.7}$ (LSAT) crystal substrate at 700 $^{o}$C in an off-axis sputtering system. The growth pressure was 0.7~mbar with Ar:O (3:2) mixing atmosphere. Information on the characterization of the LSMO films is given in (SI) Section \textcolor{blue}{I}. After cooling down to room temperature at a rate of 10 $^{o}$C/min, the NbTi layer (60~nm) was deposited on LSMO $in~situ$. The bilayer NbTi/LSMO was patterned through ebeam lithography and argon etching. Next, a focused ion beam ($\sim 30$~pA) was utilized to structure the bilayer nanopattern. Especially, the trench with width $\sim 20$~nm was opened by using a small beam current ($\sim 1.5$~pA). The depth of the trench was defined by the milling time. In the control experiments, we reduced and increased the milling time to get the NbTi and LSMO weak links (Fig.S\textcolor{blue}{2}), respectively. Note the trilayer NbTi/Ag/LSMO devices were fabricated using the same recipe except for adjusting the milling time for making the trench.  \\

\noindent
\textbf{ Transport measurements:} In a four-probe configuration, both the electrical transport and magnetoresistance were measured by using a lock-in setup. The frequency was set to 77.3~Hz for all the measurements. The resistance-temperature characteristics were taken with a 10~$\mu$A current in a cryostat (Oxford vector magnet). Direction-varying fields and a wide range of temperatures (300~K to 1.5~K) can be achieved in this cryostat. At the setpoint of temperature, the AC current was set to 1~$\mu$A as a background. The current $vs$ voltage measurements were performed by sweeping a DC current superimposed on the background and reading the corresponding voltages, with a field sweep (yielding SQI patterns) or a temperature sweep (yielding $I_c(T)$). \\

\noindent
\textbf{Micromagnetic simulation:} The micromagnetic simulations were conducted using a GPU-accelerated mumax3.\cite{2014mumax} The magnetization was set to 5.75$\times10^{5}$~A/m, and exchange stiffness was 1.7$\times10^{-12}$~J/m, yielding an exchange length of $\ell_{ex}$ $\approx$ 2.86~nm~($\ell_{ex} = \sqrt{2A_{ex}/\mu_0 M_s^2}$). According to Ref\cite{2011Boschker}, we consider a biaxial anisotropy in the LSMO(40~nm)/LSAT system. Therefore, the constant of biaxial anisotropy was set to 600~J/m$^3$. The damping constant was set to 0.5 artificially to get a high efficiency of convergence. To mimic the real situation, the arms were included in the simulation design.


\section{Author information}
All authors contributed to designing the experiment. YJ and RF fabricated the devices, YJ performed the measurements and the data analysis, with input from RF and JA. YJ and RF wrote the manuscript, with input from all authors

\section{Acknowledgments}
This work was supported by the project �Spin texture Josephson junctions� (project number 680-91-128) and by the Frontiers of Nanoscience (NanoFront) program, which are both (partly)  financed by the Dutch Research Council (NWO). YJ is funded by the China Scholarship Council (No. 201808440424). The work was further supported by EU Cost actions CA16218 (NANOCOHYBRI) and CA21144 (SUPERQMAP). It benefitted from access to the Netherlands Centre for Electron Nanoscopy (NeCEN) at Leiden University.



\providecommand{\latin}[1]{#1}
\makeatletter
\providecommand{\doi}
  {\begingroup\let\do\@makeother\dospecials
  \catcode`\{=1 \catcode`\}=2 \doi@aux}
\providecommand{\doi@aux}[1]{\endgroup\texttt{#1}}
\makeatother
\providecommand*\mcitethebibliography{\thebibliography}
\csname @ifundefined\endcsname{endmcitethebibliography}
  {\let\endmcitethebibliography\endthebibliography}{}
\begin{mcitethebibliography}{44}
\providecommand*\natexlab[1]{#1}
\providecommand*\mciteSetBstSublistMode[1]{}
\providecommand*\mciteSetBstMaxWidthForm[2]{}
\providecommand*\mciteBstWouldAddEndPuncttrue
  {\def\EndOfBibitem{\unskip.}}
\providecommand*\mciteBstWouldAddEndPunctfalse
  {\let\EndOfBibitem\relax}
\providecommand*\mciteSetBstMidEndSepPunct[3]{}
\providecommand*\mciteSetBstSublistLabelBeginEnd[3]{}
\providecommand*\EndOfBibitem{}
\mciteSetBstSublistMode{f}
\mciteSetBstMaxWidthForm{subitem}{(\alph{mcitesubitemcount})}
\mciteSetBstSublistLabelBeginEnd
  {\mcitemaxwidthsubitemform\space}
  {\relax}
  {\relax}

\bibitem[Eschrig(2015)]{2015Eschrig}
Eschrig,~M. Spin-polarized supercurrents for spintronics: a review of current
  progress. \emph{Rep Prog Phys} \textbf{2015}, \emph{78}, 104501\relax
\mciteBstWouldAddEndPuncttrue
\mciteSetBstMidEndSepPunct{\mcitedefaultmidpunct}
{\mcitedefaultendpunct}{\mcitedefaultseppunct}\relax
\EndOfBibitem
\bibitem[Linder and Robinson(2015)Linder, and Robinson]{2015Linder}
Linder,~J.; Robinson,~J. W.~A. Superconducting spintronics. \emph{Nature
  Physics} \textbf{2015}, \emph{11}, 307--315\relax
\mciteBstWouldAddEndPuncttrue
\mciteSetBstMidEndSepPunct{\mcitedefaultmidpunct}
{\mcitedefaultendpunct}{\mcitedefaultseppunct}\relax
\EndOfBibitem
\bibitem[Yang \latin{et~al.}(2021)Yang, Ciccarelli, and Robinson]{Robinson2021}
Yang,~G.; Ciccarelli,~C.; Robinson,~J. W.~A. {Boosting spintronics with
  superconductivity}. \emph{APL Mater.} \textbf{2021}, \emph{9}, 050703\relax
\mciteBstWouldAddEndPuncttrue
\mciteSetBstMidEndSepPunct{\mcitedefaultmidpunct}
{\mcitedefaultendpunct}{\mcitedefaultseppunct}\relax
\EndOfBibitem
\bibitem[Keizer \latin{et~al.}(2006)Keizer, Goennenwein, Klapwijk, Miao, Xiao,
  and Gupta]{2006Keizer}
Keizer,~R.~S.; Goennenwein,~S.~T.; Klapwijk,~T.~M.; Miao,~G.; Xiao,~G.;
  Gupta,~A. A spin triplet supercurrent through the half-metallic ferromagnet
  CrO2. \emph{Nature} \textbf{2006}, \emph{439}, 825--7\relax
\mciteBstWouldAddEndPuncttrue
\mciteSetBstMidEndSepPunct{\mcitedefaultmidpunct}
{\mcitedefaultendpunct}{\mcitedefaultseppunct}\relax
\EndOfBibitem
\bibitem[Anwar \latin{et~al.}(2010)Anwar, Czeschka, Hesselberth, Porcu, and
  Aarts]{Anwar2010}
Anwar,~M.~S.; Czeschka,~F.; Hesselberth,~M.; Porcu,~M.; Aarts,~J. {Long-range
  supercurrents through half-metallic ferromagnetic CrO$_2$}. \emph{Phys. Rev.
  B} \textbf{2010}, \emph{82}, 100501\relax
\mciteBstWouldAddEndPuncttrue
\mciteSetBstMidEndSepPunct{\mcitedefaultmidpunct}
{\mcitedefaultendpunct}{\mcitedefaultseppunct}\relax
\EndOfBibitem
\bibitem[Visani \latin{et~al.}(2012)Visani, Sefrioui, Tornos, Leon, Briatico,
  Bibes, Barth�l�my, Santamar�a, and Villegas]{2012Visani}
Visani,~C.; Sefrioui,~Z.; Tornos,~J.; Leon,~C.; Briatico,~J.; Bibes,~M.;
  Barth�l�my,~A.; Santamar�a,~J.; Villegas,~J.~E. Equal-spin Andreev
  reflection and long-range coherent transport in high-temperature
  superconductor/half-metallic ferromagnet junctions. \emph{Nature Physics}
  \textbf{2012}, \emph{8}, 539--543\relax
\mciteBstWouldAddEndPuncttrue
\mciteSetBstMidEndSepPunct{\mcitedefaultmidpunct}
{\mcitedefaultendpunct}{\mcitedefaultseppunct}\relax
\EndOfBibitem
\bibitem[Sanchez-Manzano \latin{et~al.}(2022)Sanchez-Manzano, Mesoraca,
  Cuellar, Cabero, Rouco, Orfila, Palermo, Balan, Marcano, Sander, Rocci,
  Garcia-Barriocanal, Gallego, Tornos, Rivera, Mompean, Garcia-Hernandez,
  Gonzalez-Calbet, Leon, Valencia, Feuillet-Palma, Bergeal, Buzdin, Lesueur,
  Villegas, and Santamaria]{2022Sanchez}
Sanchez-Manzano,~D. \latin{et~al.}  Extremely long-range, high-temperature
  Josephson coupling across a half-metallic ferromagnet. \emph{Nat Mater}
  \textbf{2022}, \emph{21}, 188--194\relax
\mciteBstWouldAddEndPuncttrue
\mciteSetBstMidEndSepPunct{\mcitedefaultmidpunct}
{\mcitedefaultendpunct}{\mcitedefaultseppunct}\relax
\EndOfBibitem
\bibitem[Houzet and Buzdin(2007)Houzet, and Buzdin]{2007Houzet}
Houzet,~M.; Buzdin,~A.~I. Long range triplet Josephson effect through a
  ferromagnetic trilayer. \emph{Physical Review B} \textbf{2007}, \emph{76},
  060504\relax
\mciteBstWouldAddEndPuncttrue
\mciteSetBstMidEndSepPunct{\mcitedefaultmidpunct}
{\mcitedefaultendpunct}{\mcitedefaultseppunct}\relax
\EndOfBibitem
\bibitem[Eschrig and L�fwander(2008)Eschrig, and L�fwander]{2008Eschrig}
Eschrig,~M.; L�fwander,~T. Triplet supercurrents in clean and disordered
  half-metallic ferromagnets. \emph{Nature Physics} \textbf{2008}, \emph{4},
  138--143\relax
\mciteBstWouldAddEndPuncttrue
\mciteSetBstMidEndSepPunct{\mcitedefaultmidpunct}
{\mcitedefaultendpunct}{\mcitedefaultseppunct}\relax
\EndOfBibitem
\bibitem[Khaire \latin{et~al.}(2010)Khaire, Khasawneh, Pratt, and
  Birge]{2010Khaire}
Khaire,~T.~S.; Khasawneh,~M.~A.; Pratt,~J.,~W.~P.; Birge,~N.~O. Observation of
  spin-triplet superconductivity in Co-based Josephson junctions. \emph{Phys
  Rev Lett} \textbf{2010}, \emph{104}, 137002\relax
\mciteBstWouldAddEndPuncttrue
\mciteSetBstMidEndSepPunct{\mcitedefaultmidpunct}
{\mcitedefaultendpunct}{\mcitedefaultseppunct}\relax
\EndOfBibitem
\bibitem[Robinson \latin{et~al.}(2010)Robinson, Witt, and
  Blamire]{2010Robinson}
Robinson,~J.~W.; Witt,~J.~D.; Blamire,~M.~G. Controlled injection of
  spin-triplet supercurrents into a strong ferromagnet. \emph{Science}
  \textbf{2010}, \emph{329}, 59--61\relax
\mciteBstWouldAddEndPuncttrue
\mciteSetBstMidEndSepPunct{\mcitedefaultmidpunct}
{\mcitedefaultendpunct}{\mcitedefaultseppunct}\relax
\EndOfBibitem
\bibitem[Singh \latin{et~al.}(2016)Singh, Jansen, Lahabi, and Aarts]{2016Singh}
Singh,~A.; Jansen,~C.; Lahabi,~K.; Aarts,~J. High-Quality \ce{CrO2} Nanowires
  for Dissipation-less Spintronics. \emph{Physical Review X} \textbf{2016},
  \emph{6}\relax
\mciteBstWouldAddEndPuncttrue
\mciteSetBstMidEndSepPunct{\mcitedefaultmidpunct}
{\mcitedefaultendpunct}{\mcitedefaultseppunct}\relax
\EndOfBibitem
\bibitem[Kalenkov \latin{et~al.}(2011)Kalenkov, Zaikin, and
  Petrashov]{Kalenkov2011}
Kalenkov,~M.~S.; Zaikin,~A.~D.; Petrashov,~V.~T. Triplet Superconductivity in a
  Ferromagnetic Vortex. \emph{Phys. Rev. Lett.} \textbf{2011}, \emph{107},
  087003\relax
\mciteBstWouldAddEndPuncttrue
\mciteSetBstMidEndSepPunct{\mcitedefaultmidpunct}
{\mcitedefaultendpunct}{\mcitedefaultseppunct}\relax
\EndOfBibitem
\bibitem[Silaev(2009)]{Silaev2009}
Silaev,~M.~A. Possibility of a long-range proximity effect in a ferromagnetic
  nanoparticle. \emph{Phys. Rev. B} \textbf{2009}, \emph{79}, 184505\relax
\mciteBstWouldAddEndPuncttrue
\mciteSetBstMidEndSepPunct{\mcitedefaultmidpunct}
{\mcitedefaultendpunct}{\mcitedefaultseppunct}\relax
\EndOfBibitem
\bibitem[Lahabi \latin{et~al.}(2017)Lahabi, Amundsen, Ouassou, Beukers,
  Pleijster, Linder, Alkemade, and Aarts]{2017Lahabi}
Lahabi,~K.; Amundsen,~M.; Ouassou,~J.~A.; Beukers,~E.; Pleijster,~M.;
  Linder,~J.; Alkemade,~P.; Aarts,~J. Controlling supercurrents and their
  spatial distribution in ferromagnets. \emph{Nat Commun} \textbf{2017},
  \emph{8}, 2056\relax
\mciteBstWouldAddEndPuncttrue
\mciteSetBstMidEndSepPunct{\mcitedefaultmidpunct}
{\mcitedefaultendpunct}{\mcitedefaultseppunct}\relax
\EndOfBibitem
\bibitem[Fermin \latin{et~al.}(2022)Fermin, van Dinter, Hubert, Woltjes,
  Silaev, Aarts, and Lahabi]{2022Fermin}
Fermin,~R.; van Dinter,~D.; Hubert,~M.; Woltjes,~B.; Silaev,~M.; Aarts,~J.;
  Lahabi,~K. Superconducting Triplet Rim Currents in a Spin-Textured
  Ferromagnetic Disk. \emph{Nano Lett} \textbf{2022}, \emph{22},
  2209--2216\relax
\mciteBstWouldAddEndPuncttrue
\mciteSetBstMidEndSepPunct{\mcitedefaultmidpunct}
{\mcitedefaultendpunct}{\mcitedefaultseppunct}\relax
\EndOfBibitem
\bibitem[Bergeret \latin{et~al.}(2001)Bergeret, Volkov, and
  Efetov]{Bergeret2001a}
Bergeret,~F.~S.; Volkov,~A.~F.; Efetov,~K.~B. Long-Range Proximity Effects in
  Superconductor-Ferromagnet Structures. \emph{Phys. Rev. Lett.} \textbf{2001},
  \emph{86}, 4096\relax
\mciteBstWouldAddEndPuncttrue
\mciteSetBstMidEndSepPunct{\mcitedefaultmidpunct}
{\mcitedefaultendpunct}{\mcitedefaultseppunct}\relax
\EndOfBibitem
\bibitem[Fominov \latin{et~al.}(2007)Fominov, Volkov, and Efetov]{Fominov2007}
Fominov,~Y.~V.; Volkov,~A.~F.; Efetov,~K.~B. {Josephson effect due to the
  long-range odd-frequency triplet superconductivity in SFS junctions with
  N\'eel domain walls}. \emph{Phys. Rev. B} \textbf{2007}, \emph{75},
  104509\relax
\mciteBstWouldAddEndPuncttrue
\mciteSetBstMidEndSepPunct{\mcitedefaultmidpunct}
{\mcitedefaultendpunct}{\mcitedefaultseppunct}\relax
\EndOfBibitem
\bibitem[Kalcheim \latin{et~al.}(2011)Kalcheim, Kirzhner, Koren, and
  Millo]{Kalcheim2011}
Kalcheim,~Y.; Kirzhner,~T.; Koren,~G.; Millo,~O. {Long-range proximity effect
  in La$_{2/3}$Ca$_{1/3}$MnO$_{3}$/(100)YBa$_2$Cu$_3$O$_{7-\delta}$
  ferromagnet/superconductor bilayers: Evidence for induced triplet
  superconductivity in the ferromagnet}. \emph{Phys. Rev. B} \textbf{2011},
  \emph{83}, 2--7\relax
\mciteBstWouldAddEndPuncttrue
\mciteSetBstMidEndSepPunct{\mcitedefaultmidpunct}
{\mcitedefaultendpunct}{\mcitedefaultseppunct}\relax
\EndOfBibitem
\bibitem[Aikebaier \latin{et~al.}(2019)Aikebaier, Virtanen, and
  Heikkil{\"{a}}]{Aikebaier2019}
Aikebaier,~F.; Virtanen,~P.; Heikkil{\"{a}},~T. {Superconductivity near a
  magnetic domain wall}. \emph{Phys. Rev. B} \textbf{2019}, \emph{99},
  1--11\relax
\mciteBstWouldAddEndPuncttrue
\mciteSetBstMidEndSepPunct{\mcitedefaultmidpunct}
{\mcitedefaultendpunct}{\mcitedefaultseppunct}\relax
\EndOfBibitem
\bibitem[Niu(2012)]{Niu2012}
Niu,~Z.~P. {A spin triplet supercurrent in half metal
  ferromagnet/superconductor junctions with the interfacial Rashba spin-orbit
  coupling}. \emph{Appl. Phys. Lett.} \textbf{2012}, \emph{101}, 062601\relax
\mciteBstWouldAddEndPuncttrue
\mciteSetBstMidEndSepPunct{\mcitedefaultmidpunct}
{\mcitedefaultendpunct}{\mcitedefaultseppunct}\relax
\EndOfBibitem
\bibitem[Bergeret and Tokatly(2013)Bergeret, and Tokatly]{Bergeret2013_New}
Bergeret,~F.~S.; Tokatly,~I.~V. Singlet-Triplet Conversion and the Long-Range
  Proximity Effect in Superconductor-Ferromagnet Structures with Generic Spin
  Dependent Fields. \emph{Phys. Rev. Lett.} \textbf{2013}, \emph{110},
  117003\relax
\mciteBstWouldAddEndPuncttrue
\mciteSetBstMidEndSepPunct{\mcitedefaultmidpunct}
{\mcitedefaultendpunct}{\mcitedefaultseppunct}\relax
\EndOfBibitem
\bibitem[Bergeret and Tokatly(2014)Bergeret, and Tokatly]{Bergeret2014a}
Bergeret,~F.~S.; Tokatly,~I.~V. Spin-orbit coupling as a source of long-range
  triplet proximity effect in superconductor-ferromagnet hybrid structures.
  \emph{Phys. Rev. B} \textbf{2014}, \emph{89}, 134517\relax
\mciteBstWouldAddEndPuncttrue
\mciteSetBstMidEndSepPunct{\mcitedefaultmidpunct}
{\mcitedefaultendpunct}{\mcitedefaultseppunct}\relax
\EndOfBibitem
\bibitem[Alidoust and Halterman(2015)Alidoust, and Halterman]{Alidoust2015}
Alidoust,~M.; Halterman,~K. Proximity induced vortices and long-range triplet
  supercurrents in ferromagnetic Josephson junctions and spin valves.
  \emph{Journal of Applied Physics} \textbf{2015}, \emph{117}, 123906\relax
\mciteBstWouldAddEndPuncttrue
\mciteSetBstMidEndSepPunct{\mcitedefaultmidpunct}
{\mcitedefaultendpunct}{\mcitedefaultseppunct}\relax
\EndOfBibitem
\bibitem[Jacobsen \latin{et~al.}(2015)Jacobsen, Ouassou, and
  Linder]{jacobsen2015}
Jacobsen,~S.~H.; Ouassou,~J.~A.; Linder,~J. {Critical temperature and tunneling
  spectroscopy of superconductor-ferromagnet hybrids with intrinsic
  Rashba-Dresselhaus spin-orbit coupling}. \emph{Phys. Rev. B} \textbf{2015},
  \emph{92}, 024510\relax
\mciteBstWouldAddEndPuncttrue
\mciteSetBstMidEndSepPunct{\mcitedefaultmidpunct}
{\mcitedefaultendpunct}{\mcitedefaultseppunct}\relax
\EndOfBibitem
\bibitem[Bujnowski \latin{et~al.}(2019)Bujnowski, Biele, and
  Bergeret]{Bujnowski2019}
Bujnowski,~B.; Biele,~R.; Bergeret,~F.~S. {Switchable Josephson current in
  junctions with spin-orbit coupling}. \emph{Phys. Rev. B} \textbf{2019},
  \emph{100}, 1--9\relax
\mciteBstWouldAddEndPuncttrue
\mciteSetBstMidEndSepPunct{\mcitedefaultmidpunct}
{\mcitedefaultendpunct}{\mcitedefaultseppunct}\relax
\EndOfBibitem
\bibitem[Eskilt \latin{et~al.}(2019)Eskilt, Amundsen, Banerjee, and
  Linder]{Eskilt2019}
Eskilt,~J.~R.; Amundsen,~M.; Banerjee,~N.; Linder,~J. Long-ranged triplet
  supercurrent in a single in-plane ferromagnet with spin-orbit coupled
  contacts to superconductors. \emph{Phys. Rev. B} \textbf{2019}, \emph{100},
  224519\relax
\mciteBstWouldAddEndPuncttrue
\mciteSetBstMidEndSepPunct{\mcitedefaultmidpunct}
{\mcitedefaultendpunct}{\mcitedefaultseppunct}\relax
\EndOfBibitem
\bibitem[Silaev \latin{et~al.}(2020)Silaev, Bobkova, and Bobkov]{Silaev2020}
Silaev,~M.~A.; Bobkova,~I.~V.; Bobkov,~A.~M. Odd triplet superconductivity
  induced by a moving condensate. \emph{Phys. Rev. B} \textbf{2020},
  \emph{102}, 100507\relax
\mciteBstWouldAddEndPuncttrue
\mciteSetBstMidEndSepPunct{\mcitedefaultmidpunct}
{\mcitedefaultendpunct}{\mcitedefaultseppunct}\relax
\EndOfBibitem
\bibitem[Anwar and Aarts(2011)Anwar, and Aarts]{Anwar2011}
Anwar,~M.~S.; Aarts,~J. Inducing supercurrents in thin films of ferromagnetic
  \ce{CrO2}. \emph{Su. Sci. Tech.} \textbf{2011}, \emph{24}, 024016\relax
\mciteBstWouldAddEndPuncttrue
\mciteSetBstMidEndSepPunct{\mcitedefaultmidpunct}
{\mcitedefaultendpunct}{\mcitedefaultseppunct}\relax
\EndOfBibitem
\bibitem[Dynes and Fulton(1971)Dynes, and Fulton]{1971Dynes}
Dynes,~R.~C.; Fulton,~T.~A. Supercurrent Density Distribution in Josephson
  Junctions. \emph{Physical Review B} \textbf{1971}, \emph{3}, 3015\relax
\mciteBstWouldAddEndPuncttrue
\mciteSetBstMidEndSepPunct{\mcitedefaultmidpunct}
{\mcitedefaultendpunct}{\mcitedefaultseppunct}\relax
\EndOfBibitem
\bibitem[R. \latin{et~al.}(2023)R., de~Wit, and Aarts]{2023Fermin}
R.,~F.; de~Wit,~B.; Aarts,~J. Beyond the effective length: How to analyse
  magnetic interference patterns of thin film planar Josephson junctions with
  finite lateral dimensions. \emph{Physical Review B} \textbf{2023},
  \emph{107}, 064502\relax
\mciteBstWouldAddEndPuncttrue
\mciteSetBstMidEndSepPunct{\mcitedefaultmidpunct}
{\mcitedefaultendpunct}{\mcitedefaultseppunct}\relax
\EndOfBibitem
\bibitem[Blom \latin{et~al.}(2021)Blom, Mechielsen, Fermin, Hesselberth, Aarts,
  and Lahabi]{2021Blom}
Blom,~T.~J.; Mechielsen,~T.~W.; Fermin,~R.; Hesselberth,~M. B.~S.; Aarts,~J.;
  Lahabi,~K. Direct-Write Printing of Josephson Junctions in a Scanning
  Electron Microscope. \emph{ACS Nano} \textbf{2021}, \emph{15}, 322--329\relax
\mciteBstWouldAddEndPuncttrue
\mciteSetBstMidEndSepPunct{\mcitedefaultmidpunct}
{\mcitedefaultendpunct}{\mcitedefaultseppunct}\relax
\EndOfBibitem
\bibitem[Ambegaokar and Halperin(1969)Ambegaokar, and Halperin]{1969AH}
Ambegaokar,~V.; Halperin,~B. Voltage due to thermal noise in the dc Josephson
  effect. \emph{Physical Review Letters} \textbf{1969}, \emph{22}, 1364\relax
\mciteBstWouldAddEndPuncttrue
\mciteSetBstMidEndSepPunct{\mcitedefaultmidpunct}
{\mcitedefaultendpunct}{\mcitedefaultseppunct}\relax
\EndOfBibitem
\bibitem[Anwar \latin{et~al.}(2010)Anwar, Czeschka, Hesselberth, Porcu, and
  Aarts]{2010Anwar}
Anwar,~M.~S.; Czeschka,~F.; Hesselberth,~M.; Porcu,~M.; Aarts,~J. Long-range
  supercurrents through half-metallic ferromagneticCrO2. \emph{Physical Review
  B} \textbf{2010}, \emph{82}\relax
\mciteBstWouldAddEndPuncttrue
\mciteSetBstMidEndSepPunct{\mcitedefaultmidpunct}
{\mcitedefaultendpunct}{\mcitedefaultseppunct}\relax
\EndOfBibitem
\bibitem[Dubos \latin{et~al.}(2001)Dubos, Courtois, Pannetier, Wilhelm, Zaikin,
  and Sch�n]{2001Dubos}
Dubos,~P.; Courtois,~H.; Pannetier,~B.; Wilhelm,~F.~K.; Zaikin,~A.~D.;
  Sch�n,~G. Josephson critical current in a long mesoscopic S-N-S junction.
  \emph{Physical Review B} \textbf{2001}, \emph{63}\relax
\mciteBstWouldAddEndPuncttrue
\mciteSetBstMidEndSepPunct{\mcitedefaultmidpunct}
{\mcitedefaultendpunct}{\mcitedefaultseppunct}\relax
\EndOfBibitem
\bibitem[Nadgorny \latin{et~al.}(2001)Nadgorny, Mazin, Osofsky, Soulen,
  Broussard, Stroud, Singh, Harris, Arsenov, and Mukovskii]{2001Nadgorny}
Nadgorny,~B.; Mazin,~I.~I.; Osofsky,~M.; Soulen,~R.~J.; Broussard,~P.;
  Stroud,~R.~M.; Singh,~D.~J.; Harris,~V.~G.; Arsenov,~A.; Mukovskii,~Y. Origin
  of high transport spin polarization inLa0.7Sr0.3MnO3:Direct evidence for
  minority spin states. \emph{Physical Review B} \textbf{2001}, \emph{63}\relax
\mciteBstWouldAddEndPuncttrue
\mciteSetBstMidEndSepPunct{\mcitedefaultmidpunct}
{\mcitedefaultendpunct}{\mcitedefaultseppunct}\relax
\EndOfBibitem
\bibitem[Eschrig \latin{et~al.}(2003)Eschrig, Kopu, Cuevas, and
  Sch\"{o}n]{2003Eschrig}
Eschrig,~M.; Kopu,~J.; Cuevas,~J.~C.; Sch\"{o}n,~G. Theory of
  Half-Metal/Superconductor Heterostructures. \emph{Phys Rev Lett}
  \textbf{2003}, \emph{90}, 137003\relax
\mciteBstWouldAddEndPuncttrue
\mciteSetBstMidEndSepPunct{\mcitedefaultmidpunct}
{\mcitedefaultendpunct}{\mcitedefaultseppunct}\relax
\EndOfBibitem
\bibitem[Likharev(1979)]{1979Likharev}
Likharev,~K.~K. Superconducting weak links. \emph{Rev. Mod. Phys.}
  \textbf{1979}, \emph{51}, 101\relax
\mciteBstWouldAddEndPuncttrue
\mciteSetBstMidEndSepPunct{\mcitedefaultmidpunct}
{\mcitedefaultendpunct}{\mcitedefaultseppunct}\relax
\EndOfBibitem
\bibitem[De~Gennes(1964)]{1964deGennes}
De~Gennes,~P.~G. Boundary effects in Superconductors. \emph{Rev. Mod. Phys.}
  \textbf{1964}, \emph{36}, 225\relax
\mciteBstWouldAddEndPuncttrue
\mciteSetBstMidEndSepPunct{\mcitedefaultmidpunct}
{\mcitedefaultendpunct}{\mcitedefaultseppunct}\relax
\EndOfBibitem
\bibitem[Chakhalian \latin{et~al.}(2006)Chakhalian, Freeland, Srajer,
  Strempfer, Khaliullin, Cezar, Charlton, Dalgliesh, Bernhard, Cristiani,
  Habermeier, and Keimer]{2006Chakhalian}
Chakhalian,~J.; Freeland,~J.~W.; Srajer,~G.; Strempfer,~J.; Khaliullin,~G.;
  Cezar,~J.~C.; Charlton,~T.; Dalgliesh,~R.; Bernhard,~C.; Cristiani,~G.;
  Habermeier,~H.~U.; Keimer,~B. Magnetism at the interface between
  ferromagnetic and superconducting oxides. \emph{Nature Physics}
  \textbf{2006}, \emph{2}, 244--248\relax
\mciteBstWouldAddEndPuncttrue
\mciteSetBstMidEndSepPunct{\mcitedefaultmidpunct}
{\mcitedefaultendpunct}{\mcitedefaultseppunct}\relax
\EndOfBibitem
\bibitem[Vasiliu-Doloc \latin{et~al.}(1998)Vasiliu-Doloc, Lynn, Moudden,
  Leon-Guevara, and Revcolevschi]{1998vasi}
Vasiliu-Doloc,~L.; Lynn,~J.~W.; Moudden,~A.~H.; Leon-Guevara,~A. M.~d.;
  Revcolevschi,~A. Structure and spin dynamics of
  La$_{0.85}$Sr$_{0.15}$MnO$_3$. \emph{Physical Review B} \textbf{1998},
  \emph{58}, 14913\relax
\mciteBstWouldAddEndPuncttrue
\mciteSetBstMidEndSepPunct{\mcitedefaultmidpunct}
{\mcitedefaultendpunct}{\mcitedefaultseppunct}\relax
\EndOfBibitem
\bibitem[Vansteenkiste \latin{et~al.}(2014)Vansteenkiste, Leliaert, Dvornik,
  Helsen, Garcia-Sanchez, and Van~Waeyenberge]{2014mumax}
Vansteenkiste,~A.; Leliaert,~J.; Dvornik,~M.; Helsen,~M.; Garcia-Sanchez,~F.;
  Van~Waeyenberge,~B. The design and verification of MuMax3. \emph{AIP
  Advances} \textbf{2014}, \emph{4}\relax
\mciteBstWouldAddEndPuncttrue
\mciteSetBstMidEndSepPunct{\mcitedefaultmidpunct}
{\mcitedefaultendpunct}{\mcitedefaultseppunct}\relax
\EndOfBibitem
\bibitem[Boschker \latin{et~al.}(2011)Boschker, Mathews, Brinks, Houwman,
  Vailionis, Koster, Blank, and Rijnders]{2011Boschker}
Boschker,~H.; Mathews,~M.; Brinks,~P.; Houwman,~E.; Vailionis,~A.; Koster,~G.;
  Blank,~D. H.~A.; Rijnders,~G. Uniaxial contribution to the magnetic
  anisotropy of La0.67Sr0.33MnO3 thin films induced by orthorhombic crystal
  structure. \emph{Journal of Magnetism and Magnetic Materials} \textbf{2011},
  \emph{323}, 2632--2638\relax
\mciteBstWouldAddEndPuncttrue
\mciteSetBstMidEndSepPunct{\mcitedefaultmidpunct}
{\mcitedefaultendpunct}{\mcitedefaultseppunct}\relax
\EndOfBibitem
\end{mcitethebibliography}


\end{document}

