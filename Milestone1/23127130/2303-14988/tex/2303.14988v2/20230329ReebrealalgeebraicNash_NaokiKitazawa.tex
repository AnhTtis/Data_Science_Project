\documentclass{amsart}
%\setlength{\textheight}{43pc}
%\setlength{\textwidth}{28pc}
\usepackage{amsfonts}
\usepackage{amsmath,amssymb}
\usepackage{amsthm}
\usepackage{amscd}
\usepackage{graphics}
\usepackage{graphicx}

%\usepackage[pagewise]{lineno}\linenumbers
\theoremstyle{remark}{
\newtheorem{Def}{{\rm Definition}}
\newtheorem{Ex}{{\rm Example}}
\newtheorem{Rem}{{\rm Remark}}
\newtheorem{Prob}{{\rm Problem}}
\newtheorem*{MainProb}{Main Problem}
}
\theoremstyle{plain}
{
\newtheorem{Cor}{Corollary}
\newtheorem{Prop}{Proposition}
\newtheorem{Thm}{Theorem}
\newtheorem{MainThm}{Main Theorem}
\newtheorem*{MainCor}{Main Corollary}
\newtheorem{Lem}{Lemma}
\newtheorem{Fact}{Fact}
}
\renewcommand*{\urladdrname}{\itshape Webpage}
\begin{document}
\title[Real algebraic functions which may have non-compact preimages]{Explicit real algebraic functions which may have both compact and non-compact preimages}
\author{Naoki kitazawa}
\keywords{Real algebraic functions and maps. Nash functions and maps. Real algebraic sets. Semi-algebraic sets. Real algebraic hypersurfaces. \\
\indent {\it \textup{2020} Mathematics Subject Classification}: Primary~14P05, 14P10, 14P20, 14P25, 57R45, 58C05. Secondary~57R19.}
\address{Institute of Mathematics for Industry, Kyushu University, 744 Motooka, Nishi-ku Fukuoka 819-0395, Japan\\
 TEL (Office): +81-92-802-4402 \\
 FAX (Office): +81-92-802-4405 \\
}
\email{n-kitazawa@imi.kyushu-u.ac.jp}
\urladdr{https://naokikitazawa.github.io/NaokiKitazawa.html}
\maketitle
\begin{abstract}
As a pioneering work we construct explicit real algebraic functions which may have both compact and non-compact preimages.

The author has obtained explicit real algebraic functions with preimages satisfying some nice conditions. More precisely, we have given answers to a considerably revised version of Sharko's question. Sharko originally asked whether we can have nice smooth functions whose {\it Reeb graphs} are as desired. The {\it Reeb graph} of a smooth function of a certain nice class is a natural graph whose underlying space is the space of all connected components. Such graphs can have some important topological information of the manifolds.

Our answers are new in having real algebraic functions on non-compact manifolds with no boundaries. We also avoid using so-called existence theory and approximation theory whereas we also avoid in our previous studies.  

 


\end{abstract}
%【REVISE】 combinatoric ~ is → combinatorial object. It is .
%【REVISE】  such that a point is a vertex if and only if the corresponding connected component of the level set contains some singular points → whose vertex set is the set of all points containing some singular points in the corresponding connected component of the level set .
%【REVISE】 We delete "extending the result before".
\section{Introduction.}
\label{sec:1}
\subsection{A short presentation on real algebraic manifolds and maps and Nash ones and backgrounds of our studies.}
Smooth real algebraic manifolds and smooth real algebraic maps between such manifolds have been studied as important topics in real algebraic geometry. Smooth Nash ones are slight extensions and they are also real analytic. As a viewpoint, real algebraic cases are based on the zero sets of real polynomials, which are so-called {\it real algebraic} sets {\it defined by the polynomials}. Similarly, Nash ones are based on the sets defined by equations or inequalities on real polynomials, which are so-called {\it semi-algebraic} sets. In our paper, fundamental notions from related theory and our presentation on related history are based on \cite{akbulutking, bochnakcosteroy, kollar, kucharz, nash, shiota, tognoli} for example.

It is important to know existence of such manifolds and maps.
Thanks to various studies, for example, Nash and Tognoli have shown that smooth closed manifolds are regarded as the real algebraic sets defined by some real polynomials. Akbulut and King with various researchers on real algebraic geometry have contributed to studies on problems asking whether subsets in Euclidean spaces are real algebraic sets, real algebraic manifolds, or approximated by suitable sets of such classes, for example. Approximations of smooth maps by smooth real algebraic maps are also important and some affirmative answers have been known and some are still open and important problems.

As studies on natural wider classes, for example, Shiota's systematic studies on Nash manifolds and maps are important and related fundamental theory has been established.

Our interest is on explicit construction of explicit smooth real algebraic functions or maps of non-positive codimensions or Nash ones. Natural projections of spheres embedded naturally in the one-dimensional higher Euclidean spaces are simplest examples on closed manifolds. As functions regarded as generalized cases, some functions on so-called {\it symmetric spaces} are well-known. See \cite{maciasvirgospereirasaez, ramanujam, takeuchi} for example.
They are given by suitable real polynomials. However, their global structures are hard to understand. For example, can we known about preimages? It is in general very difficult.

As pioneering results, the author has given explicit examples in \cite{kitazawa3, kitazawa4} for example. This is based on a problem asking whether we can construct nice smooth functions with mild singularities and prescribed preimages. This is based on a revised version of Sharko's problem, first considered by the author essentially as \cite{kitazawa1, kitazawa2}. Sharko had asked whether for a graph we can construct a nice smooth function whose Reeb graph is the given graph in \cite{sharko}. Related to those previous studies of us, \cite{masumotosaeki, michalak} are also important for example. The {\it Reeb graph} of a smooth function of a certain nice class such as one in \cite{saeki} is a natural graph whose underlying space is the space of all connected components of preimages and the canonically defined quotient space of the manifold.

 \cite{reeb} is one of classical pioneering paper on Reeb graphs. Reeb graphs are important tools in knowing some important topological information of the manifolds, which we do not explain about precisely.

\subsection{Manifolds and maps.}
Let $X$ be a topological space having the structure of some cell complex of a finite dimension. We can define the dimension $\dim X$ as a unique integer. A topological manifold is well-known to be homeomorphic to some CW complex and a smooth manifold is well-known to be homeomorphic to some polyhedron. We can also define the structure of a certain polyhedron for a smooth manifold in a canonical way. We call the polyhedron a PL manifold. We do not need to understand such theory precisely in our paper.
%A topological space homeomorphic to a polyhedron whose dimension is at most $2$ is well known to have the structure of a polyhedron and it is unique. For a topological manifold, such a nice fact holds in the case where the dimension is at most $3$. Check the celebrated theory \cite{moise} for example. 
  
Let ${\mathbb{R}}^k$ denote the $k$-dimensional Euclidean space. This is a simplest $k$-dimensional smooth manifold. This is also regarded as the Riemannian manifold endowed with the standard Euclidean metric. Let $\mathbb{R}:={\mathbb{R}}^1$ and $\mathbb{Z} \subset \mathbb{R}$ be the set of all integers. Let $\mathbb{N} \subset \mathbb{Z}$ the set of all integers.
For each point $x \in {\mathbb{R}}^k$, $||x|| \geq 0$ can be defined as the distance between $x$ and the origin $0$ under the metric.
This is also naturally a smooth real algebraic manifold, smooth Nash manifold and a real analytic manifold. For example, the real algebraic manifold is called the k-dimensional real affine space. $S^k:=\{x \in {\mathbb{R}}^{k+1} \mid ||x||=1\}$ denotes the $k$-dimensional unit sphere, which is a $k$-dimensional smooth compact submanifold of ${\mathbb{R}}^{k+1}$ and has no boundary. It is connected for any positive integer $k \geq 1$. It is a discrete set with exactly two points for $k=0$. It is a smooth real algebraic set, which is the zero set of the real polynomial $||x||^2={\Sigma}_{j=1}^{k+1} {x_j}^2$ with  $x:=(x_1,\cdots,x_{k+1})$. 
%It is also a smooth real algebraic submanifold defined by the zero set of the real polynomial $||x||-1={\Sigma}_{j=1}^{k+1} {x_j}^2 -1$ where $x:=(x_1,\cdots,x_{k+1})$.
$D^k:=\{x \in {\mathbb{R}}^{k} \mid ||x|| \leq 1\}$ is the $k$-dimensional unit disk, which is a $k$-dimensional smooth compact and connected submanifold of ${\mathbb{R}}^{k}$ for any non-negative integer $k \geq 0$. 
This is also a semi-algebraic set.
%We can also easily understand the topologies of these fundamental spaces.

Let $c:X \rightarrow Y$ be a differentiable map between two differentiable manifolds $X$ and $Y$. $x \in X$ is a {\it singular} point of the map if the rank of the differential at $x$ is smaller than the minimum between the dimensions $\dim X$ and $\dim Y$. We call $c(x)$ a {\it singular value} of $c$.
Let $S(c)$ denote the {\it singular set} of $c$. It is the set of all singular points of $c$.
In our paper, for differentiable maps, we consider smooth maps, which are maps of the class $C^{\infty}$, unless otherwise stated.

A canonical projection of a Euclidean space ${\mathbb{R}}^k$ is defined as the smooth surjective map mapping 
each point $x=(x_1,x_2) \in {\mathbb{R}}^{k_1} \times {\mathbb{R}}^{k_2}={\mathbb{R}}^k$ to the first component $x_1 \in {\mathbb{R}}^{k_1}$ where the conditions on the dimensions are given by $k_1, k_2>0$ and $k=k_1+k_2$. Let ${\pi}_{k,k_1}:{\mathbb{R}}^{k} \rightarrow {\mathbb{R}}^{k_1}$ denote the map. We can define a canonical projection of the unit sphere $S^{k-1}$ as the restriction of this canonical projection.

%\subsection{Graphs and Reeb graphs.} 
%(Reeb) graphs are our fundamental tools. 

%A graph can be defined as a $1$-dimensional CW complex with the {\it vertex set} and the {\it edge set}. They are the set of all $0$-dimensional cells and the set of all $1$-dimensional cells, respectively. A vertex is an element of the vertex set. An edge is an element of the edge set. 
%The closure of an edge homeomorphic to a circle is a {\it loop}. We do not consider graphs with having loops. In other words, a graph is always a $1$-dimensional polyhedron. Furthermore, it is {\it finite}. In a word, its vertex set and edge set are finite. On the other hand, a graph may be a so-called multi-graph, a graph with more than one edge connecting some distinct vertices. An {\it isomorphism} from a graph $K_1$ onto another graph $K_2$ means a PL or a piecewise smooth homeomorphism mapping the edge set and the vertex set of $K_1$ onto those of $K_2$. This defines a natural equivalence relation on the family of all graphs. Two graphs $K_1$ and $K_2$ are {\it isomorphic} if an isomorphism from $K_1 $ onto $K_2$ exists. For a smooth function $c:X \rightarrow \mathbb{R}$, we can define an equivalence relation by the rule that two points $x_1$ and $x_2$ in $X$ are equivalent if and only if they are in some same connected component of some preimage $c^{-1}(y)$. Let $W_c$ denote the quotient space.

%\begin{Def}
%If the quotient space has the structure of a graph satisfying the rule that a point $p$ is a vertex if ${q_c}^{-1}(p)$ contains some singular points, then $W_c$ is the {\it Reeb graph} of $c$.
%\end{Def}
%We explain about a main result of \cite{saeki2}.
%For a smooth function on a compact manifold having finitely many singular values, the quotient space $W_c$ has been shown to be homeomorphic to a graph. If the manifold of the domain is closed, then we can define the Reeb graph $W_c$ of $c$. {\rm Morse}({\rm -Bott}) functions and functions of some considerably wide classes satisfy such conditions.

%The Reeb graph of a smooth function has been already defined in \cite{reeb} for example. The Reeb graphs of nice smooth functions have been important tools and objects in the theory on singularities and applications to geometry. These graphs have some information on the manifolds.
%For a smooth function $c$, we can also define the quotient map $q_c:x \rightarrow W_c$ and another continuous map $\bar{c}:W_c \rightarrow \mathbb{R}$ enjoying the relation $c=\bar{c} \circ q_c$ uniquely.

 
\subsection{Our main results.}
% \cite{masumotosaeki} generalizes the pioneering work of \cite{sharko}. \cite{sharko} constructs nice smooth functions on closed surfaces and \cite{masumotosaeki} extends this to arbitrary finite graphs.
%Later, for example, \cite{martinezalfaromezasarmientooliveira, michalak} have set explicit problems and solved. Before the study \cite{kitazawa1} of the author, functions are, ones on closed surfaces or ones preimages containing no singular points of which are disjoint unions of spheres essentially.

%However we can apply some important cases. For example \cite{saeki2} considers very general cases and we cannot use analytic functions for construction there.

%Hereafter, let ${\mathbb{N}}_a$ be the set of all elements of $\mathbb{N}$ smaller than or equal to a given real number $a$.
Let $X \subset \mathbb{R}$, $a \in X$ and let $X_{a}$ denote the set of all elements of $X$ which are smaller than or equal to $a$.
\begin{MainThm}
	\label{mthm:1}

	Let $l>1$ be an integer, $\{t_j\}_{j=1}^l$ an increasing sequence of real numbers, and $l_{{\mathbb{N}}_{l-1}}:{\mathbb{N}}_{l-1} \rightarrow \{0,1\}$ be a map. 	Let $m$ be a sufficiently large integer. Then we have a suitable $m$-dimensional smooth connected Nash manifold $M$ with no boundary and a smooth Nash function $f:M \rightarrow \mathbb{R}$ enjoying the following properties.
	\begin{enumerate}
		\item The image $f(M)=[t_1,t_{l}]$.
		\item The image $f(S(f))$ of the singular set of $f$ is $\{t_j\}_{j=1}^l$.
		\item The preimage $f^{-1}(t)$ is closed and connected if $t \in (t_{j},t_{j+1})$ and $l_{{\mathbb{N}}_{l-1}}(j)=0$. The preimage $f^{-1}(t)$ is connected and non-compact and it has no boundary if $t \in (t_{j},t_{j+1})$ and $l_{{\mathbb{N}}_{l-1}}(j)=1$. The preimage $f^{-1}(t)$ is connected and homeomorphic to a CW complex whose dimension is at most $m-1$ for any $t \in f(M)$. 
	\end{enumerate}
	\end{MainThm}
For this, especially, a smooth case \cite{kitazawa2}, in which we have constructed smooth functions with prescribed preimages on non-compact manifolds with no boundaries, has motivated. We construct smooth functions which are not real-analytic there.
 Note that \cite{kitazawa3, kitazawa4} are cases of our present study where manifolds are smooth, real algebraic and closed.
%This is a variant or an extension of main results of the author, presented in \cite{kitazawa3, kitazawa4}. So is the following.


The next section explains about this including the proof. 
For example, Main Theorem \ref{mthm:2} is also presented with its proof. This shows that the function of Main Theorem \ref{mthm:1} can be constructed explicitly as the composition of a nice explicitly constructed map into ${\mathbb{R}}^3$ with the canonical projection. We can also know that our functions and maps are constructed without using existence theory and approximations for example.
\\
\ \\
\noindent {\bf Conflict of interest.} \\
The author was a member of the project JSPS Grant Number JP17H06128 and he is
also a member of the project JSPS KAKENHI Grant Number JP22K18267 "Visualizing twists in data through monodromy" (Principal Investigator: Osamu Saeki). Under their support we do this study. \\
\ \\
{\bf Data availability.} \\
Data essentially supporting our present study are all in the
 paper.
\section{On Main Theorems.}
Hereafter, essentially, for smooth real algebraic manifolds or smooth Nash ones, we only consider connected components of real algebraic sets or connected semi-algebraic sets. 


\subsection{Important subsets in the real affine space ${\mathbb{R}}^2$ or ${\mathbb{R}}^3$.}
We introduce several subsets in ${\mathbb{R}}^2$. They are important in proving Main Theorems.
\subsubsection{$P_{[a_1,a_2],b}$.}  Let $a_1<a_2$ and $b$ be real numbers.
We consider the subset in ${\mathbb{R}}^2$ represented
 as $\{(x_1,x_2) \in {\mathbb{R}}^2 \mid a_1 \leq x_1 \leq a_2, x_2 \geq b\}$. Let it be denoted by $P_{[a_1,a_2],b}$. This is a semi-algebraic set and $2$-dimensional and this is also connected. 

\subsubsection{$P_{(a_1,a_2),(b_1,b_2)}$.} Let $a_1$ and $a_2$ be real numbers. Let $b_1<0$ and $b_2>0$ be real numbers.
We consider the subset in ${\mathbb{R}}^2$ represented
as $\{(x_1,x_2) \in {\mathbb{R}}^2 \mid x_1 \leq a_1, x_2 \geq b_1(x_1-a_1)+a_2\} \bigcup \{(x_1,x_2) \in {\mathbb{R}}^2 \mid x_1 \geq a_1, x_2 \geq b_2(x_1-a_1)+a_2\}=\{(x_1,x_2) \in {\mathbb{R}}^2 \mid x_2 \geq b_1(x_1-a_1)+a_2\} \bigcap \{(x_1,x_2) \in {\mathbb{R}}^2 \mid x_2 \geq b_2(x_1-a_1)+a_2\}$. Let it be denoted by $P_{(a_1,a_2),(b_1,b_2)}$. This is a semi-algebraic set and $2$-dimensional and this is also connected. 
\subsubsection{$R_{[a_1,a_2],[b_1,b_2]}$.}
Let $a_1<a_2$ and $b_1<b_2$ be real numbers.
We consider the subset in ${\mathbb{R}}^2$ represented
as $\{(x_1,x_2) \in {\mathbb{R}}^2 \mid a_1 \leq x_1 \leq a_2, b_1 \leq x_2 \leq b_2\}$. Let it be denoted by $R_{[a_1,a_2],[b_1,b_2]}$. This is a semi-algebraic set and $2$-dimensional and this is also connected.
\subsubsection{$R_{a,b,c_1,c_2}$.}
Let $a$ and $b>0$ be real numbers. Let $c_1<c_2$ be real numbers.
We consider the subset in ${\mathbb{R}}^2$ represented
as $P_{(a,c_1),(-b,b)} \bigcap \{(x_1,x_2) \mid (x_1,-x_2) \in P_{(a,-c_2),(-b,b)}\}$. Let it be denoted by $R_{a,b,c_1,c_2}$. This is a semi-algebraic set and $2$-dimensional and this is also connected.
%\subsubsection{$P_{[a,\infty),b}$.}  Let $a$ and $b$ be real numbers.
% We consider the subset in ${\mathbb{R}}^2$ represented
% as $\{(x_1,x_2) \in {\mathbb{R}}^2 \mid x_1 \geq a, x_2 \geq b\}$. Let it be denoted by $P_{[a,\infty),b}$.
%\subsubsection{$P_{(-\infty,a],b}$.}  Let $a$ and $b$ be real numbers.
%We consider the subset in ${\mathbb{R}}^2$ represented
%as $\{(x_1,x_2) \in {\mathbb{R}}^2 \mid x_1 \leq a, x_2 \geq b\}$. Let it be denoted by $P_{(-\infty,a],b}$.
%\subsubsection{$C_{{\rm P},(-\infty,a],b,c}$.}  Let $a$, $b$ and $c>0$ be real numbers.
%We consider the subset in ${\mathbb{R}}^2$ represented
%5as $\{(x_1,x_2) \in {\mathbb{R}}^2 \mid (x_1-a)-c{(x_2-b)}^2=0\}$. Let it be denoted by $C_{{\rm P},(-\infty,a],b,c}$. Different from the previous six cases, which are for $2$-dimensional connected semi-algebraic sets, this is a $1$-dimensional smooth real algebraic curve. This is connected. This is for a parabola.  
\subsubsection{$C_{{\rm P},[a,\infty),b,c}$.}  Let $a$, $b$ and $c>0$ be real numbers.
We consider the subset in ${\mathbb{R}}^2$ represented
as $\{(x_1,x_2) \in {\mathbb{R}}^2 \mid (x_1-a)+c{(x_2-b)}^2=0\}$. Let it be denoted by $C_{{\rm P},[a,\infty),b,c}$. This is a $1$-dimensional smooth real algebraic manifold and a curve and has no boundary. This is connected. This is also for a parabola.  
%\subsubsection{$C_{{\rm H},(-\infty,a],b,c}$.}  Let $a$, $b$ and $c>0$ be real numbers.
%We consider the subset in ${\mathbb{R}}^2$ represented
%as $\{(x_1,x_2) \in {\mathbb{R}}^2 \mid (x_1-a)(x_2-b)=c\}$. Let it be denoted by $C_{{\rm H},(-\infty,a],b,c}$. 
%Different from the previous cases, this is not connected and this consists of exactly two connected components. These connected components are denoted by $C_{{\rm H}_{+},(-\infty,a],b,c}:=\{(x_1,x_2) \in C_{{\rm H},(-\infty,a],b,c} \mid x_1>a, x_2>b\}$ and $C_{{\rm H}_{-},(-\infty,a],b,c}:=\{(x_1,x_2) \in C_{{\rm H},(-\infty,a],b,c} \mid x_1<a, x_2<b\}$, respectively. They are $1$-dimensional smooth real algebraic curves. This is for a hyperbola.
\subsubsection{$C_{{\rm H},[a,\infty),b,c}$.}  Let $a$, $b$ and $c<0$ be real numbers.
We consider the subset in ${\mathbb{R}}^2$ represented
as $\{(x_1,x_2) \in {\mathbb{R}}^2 \mid (x_1-a)(x_2-b)=c\}$. Let it be denoted by $C_{{\rm H},[a,\infty),b,c}$. 
This is a $1$-dimensional smooth real algebraic manifold and a curve and has no boundary. This is also for a hyperbola.
This also consists of exactly two connected components and one of the two connected components is denoted by $C_{{\rm H}_{+},[a,\infty),b,c}:=\{(x_1,x_2) \in C_{{\rm H},[a,\infty),b,c} \mid x_1<a, x_2>b\}$. The connected component defined here appears later.
% and $C_{{\rm H}_{-},[a,\infty),b,c}:=\{(x_1,x_2) \in C_{{\rm H},[a,\infty),b,c} \mid x_1>a, x_2<b\}$ respectively. They are $1$-dimensional smooth real algebraic curves. This is also for a hyperbola.
\subsubsection{$C_{{\rm E},a,b,c,d_1,d_2,d_3}$.}  Let $a$, $b$ and $c$ be real numbers. Let $d_1, d_2, d_3>0$ be positive numbers.
We consider the subset in ${\mathbb{R}}^3$ represented
as $\{(x_1,x_2,x_3) \in {\mathbb{R}}^3 \mid \frac{{(x_1-a)}^2}{d_1}+\frac{{(x_2-b)}^2}{d_2}+\frac{{(x_3-c)}^2}{d_3}=1\}$. Let it be denoted by $C_{{\rm E},a,b,c,d_1,d_2,d_3}$. This is a $2$-dimensional smooth real algebraic manifold and it has no boundary. This is connected. This is for a $2$-dimensisonal ellipse or the (boundary of) a (2-dimensional) ellipsoid.
\subsection{A proposition for our proof of Main Theorems.}
Our main ingredient is, in short, explicitly reviewing construction of smooth real algebraic maps first presented in \cite{kitazawa3}, followed by \cite{kitazawa4} in a generalized way.
We prepare the following for our proof of Main Theorems. For a finite set $X$, $|X| \in \mathbb{N} \subset \{0\}$ denotes its size.
\begin{Def}
\label{def:1}
Let $D$ be a connected open set in ${\mathbb{R}}^n$ enjoying the following properties. 
\begin{enumerate}
\item There exists a suitable positive integer $l>0$.
\item We have a family $\{f_j\}_{j=1}^l$ of $l$ real polynomials with $n$ variables. 
\item $D:={\bigcap}_{j=1}^l \{x \in {\mathbb{R}}^n \mid f_j(x)>0\}$ holds and it is also a semi-algebraic set.  
\item For the closure $\overline{D}$ of $D$, $\overline{D}={\bigcap}_{j=1}^l \{x \in {\mathbb{R}}^n \mid f_j(x) \geq 0\}$ holds and it is also a semi-algebraic set.
\item $S_j$ is a connected component of the real algebraic set $\{x \in {\mathbb{R}}^n \mid f_j(x)=0\}$ and a smooth connected real algebraic hypersurface with no boundary.
\item Distinct hypersurfaces in $\{S_j\}$ intersect satisfying the condition on the "transversality". In other words, the following conditions are satisfied.
\begin{itemize}
\item Each intersection ${\bigcap}_{j \in \Lambda} S_j$ ($\Lambda \subset {\mathbb{N}}_{l}$) is empty or a smooth real algebraic hypersurface with no boundary and of dimension $n-|\Lambda|$ (where $\Lambda$ is not empty).
\item Let $e_j:S_j \rightarrow {\mathbb{R}}^n$ denote the canonically defined smooth (real algebraic) embedding. For each intersection ${\bigcap}_{j \in \Lambda} S_j$ before and each point $p$ there, the image of the differential of the canonically defined embedding of ${\bigcap}_{j \in \Lambda} S_j$ at the point $p$ and the intersection of the images of all differentials of the embeddings in ${\{e_j\}}_{j \in \Lambda}$ at the point $p$ coincide. 
\end{itemize}
\end{enumerate}
Then $D$ is said to be a {\it normal and convenient} semi-algebraic set. We also call it an {\it NC} semi-algebraic set. 
\end{Def}
Note that "normal and convenient" or "NC" is identical to our paper, as we know. 

%\subsection{Additional several terminologies and notions.}
%A {\it diffeomorphism} is a smooth homeomorphism with no singular points. A {\it diffeomorphism on a smooth manifold} is a diffeomorphism from the manifold to itself.
%The {\it diffeomorphism type} of a smooth manifold is defined as the equivalence class under the natural equivalence relation. 
%This is defined on the family of all smooth manifolds defined by the existence of diffeomorphisms. Two manifolds whose diffeomorphism types are same are said to be {\it diffeomorphic}.

%The {\it diffeomorphism group} of a smooth manifold is the group of all diffeomorphisms there. This is topologized with the so-called {\it Whitney $C^{\infty}$ topology} and a so-called topological group. More generally, Whitney $C^{\infty}$ topologies on the set of all smooth maps between two given smooth manifolds and subspaces of the space are important in the (singularity) theory of smooth maps for example. See \cite{golubitskyguillemin} for example. This is a book for elementary and some advanced theory of singularity theory of differentiable maps.

%A {\it smooth} bundle is a bundle whose fiber is a smooth manifold and whose structure group is regarded as (some subgroup) of the diffeomorphism group of the fiber.

%We introduce {\it fold} maps. For the Whitney $C^{\infty}$ topology and fold maps for example, see also \cite{golubitskyguillemin} as a book for elementary singularity theory of differentiable maps
%for example.

%\begin{Def}
%	Let $X$ and $Y$ be smooth manifolds with no boundaries satisfying $\dim X \geq \dim Y$.
%	A {\it fold} map $c:X \rightarrow Y$ is a smooth map such that at each singular point $p$, we have suitable integer $0 \leq i(p) \leq \frac{\dim X-\dim Y+1}{2}$ and local coordinates around $p$ and $c(p)$ and that there we have a local form $c(x_1,\cdots,x_{\dim X})=(x_1,\cdots,x_{\dim Y-1},{\Sigma}_{j=1}^{\dim X-\dim Y-i(p)+1} {x_{\dim Y-1+j}}^2-{\Sigma}_{j=1}^{i(p)} {x_{\dim X-i(p)+j}}^2)$ around the singular point $p$ and the singular value $c(p)$.
%\end{Def}
%5\begin{Prop}
%	In the previous definition, $i(p)$ is chosen uniquely and defined as the {\rm index} of $p$. The set of all singular points of $c$ of a fixed index is a smooth regular submanifold of $c$ and of dimension $\dim Y-1$. If $X$ is closed, then the submanifold is compact and with no boundary. The restriction to the submanifold is a smooth immersion. 
%\end{Prop}
%5\begin{Def}
%	If in the definition of the fold map, $i(p)=0$ always holds, then this is called a {\it special generic} map.
%\end{Def}
%A Morse function is of course a fold map. 
%In short, fold maps are locally projections or the product map of a Morse function and the identity map on some disk. For special generic maps, the Morse function is chosen as a so-called {\it height function} of a unit disk. A {\it height function} $h$ of a unit sphere is a function of the form $h(x)=\pm ||x||^2+c$ where $c$ is some real number.

%We introduce very fundamental and explicit special generic maps. They are also keys in our main results.

%\begin{Ex}
%\label{ex:1}
%	\begin{enumerate}
	%	\item Canonical projections of unit spheres are special generic. The restrictions to the singular sets, which are also regarded as unit spheres, are embeddings. The images are regarded as the unit disks whose dimensions are same as those of the Euclidean spaces of the targets.
	%	\item Let $m>n$ be positive integers. 
	%	Let $M$ be an $m$-dimensional smooth manifold represented as a connected sum of $l> 0$ manifolds diffeomorphic to $S^{k_j} \times S^{m-k_j}$ for each integer $1 \leq j \leq l$ and some integer $1 \leq k_j \leq n-1$ where the connected sum is taken in the smooth category. We easily have a special generic map $f:M \rightarrow {\mathbb{R}}^n$ such that the restriction to the singular set $S(f)$ is an embedding and that the image is a smoothly embedded submanifold diffeomorphic to one represented as a boundary connected sum of $l> 0$ manifolds diffeomorphic to $S^{k_j} \times D^{n-k_j}$ for each integer $1 \leq j \leq l$. The boundary connected sum is, as before, taken in the smooth category. 
%		\end{enumerate}
%For these maps, we also have the following two trivial smooth bundles. We consider a case where $m$ and $n$ are the dimensions of the manifolds of the domain and the target.
%\begin{itemize}
%	\item We have some small collar neighborhood of the boundary of the image and the composition of the restriction of the map to the preimage with the canonical projection to the boundary gives a trivial smooth bundle whose fiber is diffeomorphic to a unit disk $D^{m-n+1}$. 
%	\item On the complementary set of the interior of the collar neighborhood, the restriction of the map gives a trivial smooth bundle whose fiber is diffeomorphic to a unit sphere $S^{m-n}$.
%\end{itemize}
%Moreover, they are glued by the product map of the diffeomorphism for the natural identification between the base spaces and the identity map on the fiber. Note that fibers are identified in some canonical way. For such special generic maps, see also the preprint \cite{kitazawa4, kitazawa7} of the author.
%\end{Ex}



%A {\it Morse-Bott} function is a smooth function on a manifold with no boundary at each singular point of which it is represented as the composition of some projection with a Morse function for suitable local coordinates. See \cite{bott}.

%As our main work, we construct real algebraic functions such that for each singular point, the singularity is as one of these cases (at least) topologically. We do not investigate these singularities in our main theorems. This presents another fundamental, important and difficult problem on singularities of polynomial maps or more generally, smooth maps.

%To simplify our arguments, let us assume the following where $l \geq 0$ is an integer.
%\begin{itemize}
%\item For each hypersurface $S_j$ in the family $\{S_j\}_{j=1}^l$, a real polynomial $f_{{\rm P}, S_j}$ is given so that the zero set and $S_j$ coincide and that the polynomial function $f_{{\rm P}, S_j}:S_j \rightarrow \mathbb{R}$ defined canonically has no singular points on $S_j$.
%\item $D$ is assumed to be the intersection ${\bigcap}_{j=1}^l  \{x \in {\mathbb{R}}^k \mid  f_{{\rm P}, S_j}(x)>0 \}$.
%\end{itemize}

%For example, the interior ${\rm Int}\ D^k$ of $D^k:=\{x \in {\mathbb{R}}^{k} \mid ||x|| \leq 1 \}$ is a simplest example and $D^k$ is the $k$-dimensional unit disk. This is also a $k$-dimensional smooth, compact and connected submanifold.
%Note that $||x||={\Sigma}_{j=1}^k {x_j}^2$ where $x:=(x_1,\cdots,x_k)$.

%A {\it Poincar\'e-Reeb graph} is defined for a pair of an algebraic domain $D$ of the real affine space of dimension $k>1$ and a canonical projection ${\pi}_{k,1}$ mapping $(x_1,x_2) \in {\mathbb{R}}^{k}$ to $x_1 \in {\mathbb{R}}$. This can be presented in a more general manner.
%Hereafter, we mainly respect the preprint \cite{bodinpopescupampusorea} and there such cases are discussed. Note that terminologies and situations are different in considerable cases and that here we can argue in a self-contained way.
%\begin{Def}
%\label{def:1}
%A {\it Poincar\'e-Reeb graph for the pair $(D,{\pi}_{k,1})$} is a graph in the real affine space embedded by a piecewise smooth embedding with the following conditions.
%\begin{itemize}
%\item Each edge $e$ intersects each preimage of the projection ${\pi}_{k,1}$ in a so-called {\it generic} way or satisfying the "transversality". In other words, each edge is embedded smoothly and for each point $p_e$ in each edge $e$, the image of the differential at the point and the tangent vector space at the value $v(p_e)$ in the preimage ${{\pi}_{k,1}}^{-1}(p)$ of a suitable {\rm (}unique{\rm )} point $p$ by the projection ${\pi}_{k,1}$ generate the tangent vector space at the point $v(p_e) \in {\mathbb{R}}^k$. 
%\item Two points in the closure $\overline{D}$ of $D$ can be defined to be equivalent if and only if they are in a same connected component of the preimage $\overline{D} \bigcap {{\pi}_{k,1}}^{-1}(p)$ for some point $p \in {\mathbb{R}}$ and the map obtained by the restriction of the projection to the closure $\overline{D}$. Let ${\pi}_{D}$ denote the restriction to  the closure $\overline{D}$.
%Our Poincar\'e-Reeb graph for the pair can be also defined as the quotient space obtained by this equivalence relation. This is isomorphic to the Reeb graph of ${\pi}_{D}$. Furthermore, an isomorphism is defined as the canonically obtained correspondence. 
%\%item The vertex set of our Poincar\'e-Reeb graph for the pair is the union of the set of all singular points of the restrictions of the projection ${\pi}_{k,1}$ or ${\pi}_D$ to all connected components of the boundary $\partial \overline{D} \subset \overline{D}$. This set is also finite. 
%\end{itemize}
%%\end{Def}
%See also \cite{sorea1, sorea2} for related theory for example. We present our main result. In the following section we prove this and present related comments as our main content.
%\begin{MainThm}
%5\label{mthm:1}
%Consider a Poincar\'e-Reeb graph $K$ for the pair in Definition \ref{def:1} such that the closure $\overline{D}$ is compact. Take an arbitrary integer $k_0>k+1$. Then we can construct a real algebraic function on some {\rm (}$k_0-1${\rm )}-dimensional smooth closed manifold regarded as a smooth real algebraic manifold whose Reeb graph is isomorphic to the graph $K$ as a graph.
%\end{MainThm}


%\cite{kitazawa3} presents most of our key tools. We review some important ingredients. We apply them. We also apply them in suitably improved ways in several scenes. 
%At present we do not find essential errors in the (accepted) version on which a positive report for publication has been announced to have been sent. However we will not publish the paper revising the accepted version essentially.

%Note that the work is motivated by \cite{bodinpopescupampusorea} with \cite{sorea1, sorea2}. \cite{bodinpopescupampusorea} studies {\it algebraic domains} collapsing to some graphs. An {\it algebraic domain} means an open set in an real affine space the boundary of whose closure is surrounded by smooth real algebraic hypersurfaces. 
%Originally our main theorem of \cite{kitazawa3} respects these studies.
%The work essentially related to such work is regarded as one of important future problems. However, we do not investigate related problems in our paper. For the related studies, it seems that we need more sophisticated or advanced knowledge and arguments on real algebraic geometry.
The following reviews main ingredients of \cite{kitazawa3, kitazawa4}.
\begin{Prop}
	\label{prop:1}
For an NC semi-algebraic set $D$ in ${\mathbb{R}}^n$ and an arbitrary sufficiently large integer $m$, we have a suitable $m$-dimensional smooth and connected manifold $M$ which is also a connected component of the real algebraic set defined by some $l$ real polynomials and regarded as a semi-algebraic set and a smooth real algebraic map $f:M \rightarrow \mathbb{R}$ enjoying the following properties.
\begin{enumerate}
\item The image $f(M)$ is the closure $\overline{D}$.
\item $f(S(f))=\overline{D}-D$.
\item The preimage $f^{-1}(p)$ of each point $p$ is at most {\rm (}$m-n${\rm )}-dimensional and it is seen as a smooth submanifold diffeomorphic to the product of finitely many copies of unit spheres if $p \in D$.  
\end{enumerate} 
\end{Prop}

\begin{proof}

This essentially reviews a main result or Main Theorem 1 of \cite{kitazawa4} explicitly.

We abuse the notation from Definition \ref{def:1} for example.

For local coordinates and points, we use the notation like $x:=(x_1,\cdots,x_k)$ for an arbitrary integer $k>0$.

We define a set in $S \subset {\mathbb{R}}^{m+1}$ by
$S:= \{(x,y) \in \overline{D} \times {\mathbb{R}}^{m-n+1} \subset {\mathbb{R}}^n \times {\mathbb{R}}^{m-n+1}={\mathbb{R}}^{m+1} \mid f_j(x_1,\cdots,x_n)-||y_j||^2=0, j \in {\mathbb{N}_l}\}$ and we investigate this. 
First this is a semi-algebraic set in ${\mathbb{R}}^{m+1}$.
Here $y_j:=(y_{j,1},\cdots,y_{j,d_j}) \in {\mathbb{R}}^{d_j}$ and $y:=(y_1,\cdots,y_l)$ where $d_j>0$ is a suitable positive integer.
%Here we may replace $||y||^2$ by a general polynomial, represented in the form ${\Sigma}_{j=1}^{n^{\prime}-n} a_j {y_{j}}^{b_j}$ where $a_j$ and $b_j$ are a positive integer and a positive even integer, respectively.  


%We show this is also a desired algebraic domain. 

%By the definition, we have (\ref{thm:1.1}). The closure $\overline{D^{\prime}}$ is easily known to be compact.
 
We consider the partial derivative of the function defined canonically from the real polynomial $f_j(x_1,\cdots,x_n)-{\Sigma}_{j^{\prime}=1}^{d_j} {y_{j,j^{\prime}}}^2$ for variants $x_j$ and $y_{j,j^{\prime}}$. 

We take a point $(x_0,y_0) \in S$ such that $y_0$ is not the origin.
By the assumption $f_{j}(x_0)>0$ for each $j$. 
Here we consider the partial derivative of the function for each variant $y_{j,j^{\prime}}$.
We introduce the notation on $y_0$ by $y_0:=(y_{0,1},\cdots y_{0,l})$ and $y_{0,j}:=(y_{0,j,1},\cdots,y_{0,j,d_j}) \in {\mathbb{R}}^{d_j}$. 
As a result, we have $2{y_{j,j_0}}=2y_{0,j,j_0} \neq 0$ for some variant $y_{j,j_0}$ with $j^{\prime}=j_0$. 

The differential of the restriction of the function defined canonically from the real polynomial $f_j(x)-{||y_j||}^2$ at $(x_0,y_0)$ is not of rank $0$. This is not a singular point of this function defined from $f_j(x_1,\cdots,x_n)-{\Sigma}_{j^{\prime}=1}^{d_j} {y_{j,j^{\prime}}}^2$. We also have the fact that the map into ${\mathbb{R}}^l$ obtained canonically from the $l$ functions is of rank $l$.

We take a point $(x_{\rm O},y_{\rm O}) \in S$ such that $y_{\rm O}$ is the origin. Let $x_{\rm O} \in S_{j}$ for $j \in J \subset {\mathbb{N}}_l$ and $x_{\rm O} \notin S_{j}$ for $j \notin J$ where $J$ is a non-empty subset.

%We use the notation like $x_0:=(x_{0,1},\cdots,x_{0,k})$ and $y_0:=(y_{0,1},\cdots,y_{0,k})$ before. 
By the assumption, for each real polynomial $f_{j_1}(x)$ with $j_1 \notin J$, $f_{j_1}(x_{\rm O})> 0$. The function defined canonically from the real polynomial $f_{j_0}$ with $j_0 \in J$ has no singular points on the hypersurfaces $S_{j_0}$ by the assumption that $S_{j}$ is smooth for all $j$. Consider the partial derivative of the function defined from the real polynomial $f_{j_0}(x)-{||y_{j_0}||}^2$ with $j_0 \in J$ for each variant $x_{j}$. For the map into ${\mathbb{R}}^{|J|}$ obtained canonically from the $|J|$ real polynomials, the rank of the differential at the point is $|J|$. This comes from the assumption on the "transversality" for the hypersurfaces $S_j$ in Definition \ref{def:1}. 

By considering the canonically obtained map into ${\mathbb{R}}^{l-|J|}$ similarly respecting the real polynomials, the rank of the differential at the point is $n-|J|$. We can see this by considering the partial derivative by some variant $y_{j_1,j^{\prime}}$ with $j_1 \notin J$. We have a result as in the case $(x_0,y_0)$.

By the assumption and the definition, for each point $x^{\prime}$ in ${\mathbb{R}}^n-D$ sufficiently close to $D$, the set $S_{x^{\prime}}:= \{(x^{\prime},y) \in {\mathbb{R}}^n \times {\mathbb{R}}^{m-n+1} \subset {\mathbb{R}}^n \times {\mathbb{R}}^{m-n+1}={\mathbb{R}}^{m+1} \mid f_j(x_1,\cdots,x_n)-||y_j||^2=0, j \in {\mathbb{N}_l}\}$ is empty. From this with the implicit function theorem, the original set $S$ is a smooth closed and connected manifold and it is also a connected component of the real algebraic set defined by the $l$ real polynomials. The set $S$ is also regarded as a semi-algebraic set by the definition and the assumption on $D$. 

By the structures of the sets, we can set $M:=S$ and define $f:M \rightarrow {\mathbb{R}}^n$ as the canonical projection to ${\mathbb{R}}^n$.

This completes the proof.
	
\end{proof}

 \subsection{A proof of Main Theorems.}

\begin{MainThm}
\label{mthm:2}	
In Main Theorem \ref{mthm:1}, we can have the function as the composition of a suitable smooth Nash map into ${\mathbb{R}}^3$ enjoying the following properties with the canonical projection ${\pi}_{3,1}:{\mathbb{R}}^3 \rightarrow \mathbb{R}$ mapping $(x_1,x_2,x_3)$ to $x_1$.
\begin{enumerate}
\item We consider the case where the map is on a closed manifold. The map is a smooth real algebraic map.
        \item We consider the case where the map is on a non-compact manifold. The map is represented as the restriction to some connected semi-algebraic set of some real algebraic map.
	\item The image of the map is the disjoint union of some NC semi-algebraic $D \subset {\mathbb{R}}^3$ and some connected components of $\overline{D}-D$ where we abuse the notation used in Definition \ref{def:1} for example.
	\item $\overline{D}-D$ is a union of finitely many $2$-dimensional smooth connected submanifolds of connected components of the real algebraic sets defined by single real polynomials of degrees at most $2$.
\end{enumerate}
\end{MainThm}

\begin{proof}[A proof of Main Theorems]
As our main ingredients, we obtain a suitable semi-algebraic set $D_0 \subset {\mathbb{R}}^3$ and apply Proposition \ref{prop:1} with additional arguments to have a desired subset $D \subset {\mathbb{R}}^3$. We need to consider several cases. \\
\ \\	
	Case 1. The case $l=2$. \\
	\indent We consider $l_{{\mathbb{N}}_1}:\{1\} \rightarrow \{0,1\}$. In the case the value is $0$, we take $D_0:=R_{[t_1,t_2],[b_1,b_2]} \times [-R,R]$ where $b_1<b_2$ and $R>0$ are real numbers. In the case the value is $1$, we take $D_0$ as the interior of $P_{[t_1,t_2],b} \times [-R,R]$ where $b$ and $R>0$ are real numbers.
	
	We can apply Proposition \ref{prop:1} by putting $D:=D_0$ to complete the proof in this case.\\
	\ \\
	\noindent Case 2. The case $l=3$ with the function $l_{{\mathbb{N}}_2}(1)=0$ and $l_{{\mathbb{N}}_2}(2)=0$. \\
	\indent We take $D_0$ as the interior of ${\rm R}_{t_2,b,c_1,c_2} \times [-R,R]$ where $b>0$, $c_1<c_2$ and $R>0$ are real numbers. We also need to choose the numbers so that the relations $-b(t_1-t_2)+c_1=b(t_1-t_2)+c_2$ and $b(t_3-t_2)+c_1=-b(t_3-t_2)+c_2$ are enjoyed.
	
		We can apply Proposition \ref{prop:1} by putting $D:=D_0$ to complete the proof in this case. \\
	\ \\
	\noindent Case 3. The case $l=3$ with the function $l_{{\mathbb{N}}_2}(1)=0$ and $l_{{\mathbb{N}}_2}(2)=1$. \\
	\indent 
	 We choose real numbers $b_1<b_2$, $c_1>0$, $c_2<0$, and $R>0$. 
        We choose $c_1>0$ sufficiently large so that $b_1+\frac{\sqrt{t_2-t_1}}{c_1}<b_2$. We choose $c_2<0$ sufficiently small so that $b_2+\frac{c_2}{(t_3-t_2)}<b_1$. 
	We define the connected component of the complementary set ${\mathbb{R}}^2-C_{{\rm P},[t_2,\infty),b_1,c_1}$ containing points $(p_1,p_2)$ satisfying $p_1 \geq t_2$ by $R_{{\rm P},[t_2,\infty),b_1,c_1}$.
	We define the connected component of the complementary set ${\mathbb{R}}^2-C_{{\rm H}_{+},[t_2,\infty),b_2,c_2}$ containing points $(p_1,p_2)$ satisfying $p_1 \geq t_2$ by $R_{{\rm H}_{+},[t_2,\infty),b_2,c_2}$.
	We take $D_0$ as the interior of  $(P_{[t_1,t_3],b_1} \bigcap R_{{\rm P},[t_2,\infty),b_1,c_1} \bigcap R_{{\rm H}_{+},[t_2,\infty),b_2,c_2}) \times [-R,R]$. The
 relations $b_1+\frac{\sqrt{t_2-t_1}}{c_1}<b_2$ and $b_2+\frac{c_2}{(t_3-t_2)}<b_1$ are important
 in the shape of this subset. For this, See also FIGURE \ref{fig:1}.

\begin{figure}

	
	\includegraphics[height=75mm, width=100mm]{20230327sets1.eps}
\begin{centering}
	\caption{Some important (sub)sets of ${\mathbb{R}}^2$ in Case 3.}
	\label{fig:1}
\end{centering}	
\end{figure}
	
	 We can apply Proposition \ref{prop:1} by putting $D:=D_0$ to complete the proof in this case. Here the relation $b_2+\frac{c_2}{(t_3-t_2)}<b_1$ and the definitions of $C_{{\rm H}_{+},[t_2,\infty),b_2,c_2}$ and $C_{{\rm H},[t_2,\infty),b_2,c_2}$ are important. \\ 
	 \ \\
	 	\noindent Case 4. The case $l=3$ with the function $l_{{\mathbb{N}}_2}(1)=1$ and $l_{{\mathbb{N}}_2}(2)=0$. \\
	 	\indent We can consider as Case 3 by the symmetry. \\
	 \ \\	
	\noindent Case 5. The case $l=3$ with the function $l_{{\mathbb{N}}_2}(1)=1$ and $l_{{\mathbb{N}}_2}(2)=1$. \\
	\indent We take $D_0$ as the interior of $(P_{[t_1,t_3],a_1} \bigcap P_{(t_1,a_2),(b_1,b_2)}) \times [-R,R]$ where we choose two real numbers satisfying $a_1<a_2$ and another additional real numbers $b_1<0$ and $b_2>0$.
	
	
	We can apply Proposition \ref{prop:1} by putting $D:=D_0$ to complete the proof in this case. \\
	\ \\
%		\noindent Case 6. The case $l=3$ with the function $l_{{\mathbb{N}}_2}(1)=1$ and $l_{{\mathbb{N}}_2}(2)=1$. \\
%	\indent We take $D_0:=({\rm P}_{[t_1,t_3],a_1} \bigcap {\rm P}_{(t_1,a_2),(b_1,b_2)}) \times [-R,R]$ where we choose two real numbers satisfying $a_1<a_2$ and another additional real numbers $b_1<0$ and $b_2>0$. We can apply Proposition \ref{prop:1} by putting $D:=D_0$ to complete the proof.  \\
%	\ \\
	\noindent Case 6. The case $l=4$ with the function $l_{{\mathbb{N}}_3}(1)=1$, $l_{{\mathbb{N}}_2}(2)=0$ and $l_{{\mathbb{N}}_2}(3)=1$. \\
	\indent In case 3, we replace "$t_2$" by $t_3$ and "$t_3$" by $t_4$. We first construct $D_0$ similarly.
Second, we consider $R_{{\rm P},[t_2,\infty),b_1+\sqrt{\frac{t_3-t_2}{c_1}},c_1}$, defined as in Case 3 and define $D_1$ as the intersection of $D_0$ and this.  This set $D_1$ is an open subset. We apply Proposition \ref{prop:1} by defining "$D$" there  as $D_1$.
We remove a subset and its preimage from the obtained image and map to have a desired map into ${\mathbb{R}}^3$. The subset is a closed set and chosen as the product of the following two.

\begin{itemize}
\item The intersection of the original set, which is a closed set in ${\mathbb{R}}^3$, and the complementary set of the open subset $R_{{\rm H}_{+},[t_2+a,\infty),b_3,c_2} \subset {\mathbb{R}}^2$, defined as in Case 3 in the same way for a suitably chosen positive number $a>0$ and real number $b_3$. In addition, we can choose the new numbers satisfying the following relations.
\begin{itemize}

\item \quad  $b_1+\sqrt{\frac{t_3-t_1}{c_1}} < b_1+\sqrt{\frac{t_3-t_2}{c_1}}+\sqrt{\frac{t_2-t_1}{c_1}}
<b_3<b_2$. For this, it is also essential that $c_1>0$ is sufficiently large.
\item \quad $t_2+a<t_3$.
\item \quad $(t_2-t_3)(b_3+\frac{c_2}{t_2-(t_2+a)}-b_2)=c_2$. It is simplified as $(t_3-t_2)(\frac{c_2}{a}+b_2-b_3)=c_2$. This means that the point $(t_2,b_3+\frac{c_2}{t_2-(t_2+a)})=(t_2,b_3-\frac{c_2}{a})$ is a point in the curves $C_{{\rm H}_{+},[t_3,\infty),b_2,c_2}$ and $C_{{\rm H}_{+},[t_2+a,\infty),b_3,c_2}$.
\item \quad $b_3+\frac{c_2}{t_4-t_2-a}<b_1$. This respects the definitions of $C_{{\rm H},[t_2+a,\infty),b_3,c_2}$ and $C_{{\rm H}_{+},[t_2+a,\infty),b_3,c_2}$. 
Remember also that $c_2<0$ is chosen as a sufficiently small number. For example, we can regard that another connected component of $C_{{\rm H},[t_2+a,\infty),b_3,c_2}$ and the connected component $C_{{\rm H},[t_3,\infty),b_2,c_2}-C_{{\rm H}_{+},[t_3,\infty),b_2,c_2}$ are outside the closure of the image of our desired map into ${\mathbb{R}}^3$. This is also for Proposition \ref{prop:1}.
\end{itemize}
We choose them satisfying the relations. The set is also a closed set.
\item $[-R,R]$.
\end{itemize}

For subsets related to this case, see FIGURE \ref{fig:2}.

\begin{figure}
	
	\includegraphics[height=75mm, width=100mm]{20230327sets2.eps}

	\caption{Some important (sub)sets of ${\mathbb{R}}^2$ in Case 6: another component of $C_{{\rm H},[t_2+a,\infty),b_3,c_2}$ and that of $C_{{\rm H},[t_3,\infty),b_3,c_2}$ are omitted and are apart from the closure of the image of our desired real algebraic (Nash) map into ${\mathbb{R}}^3$. Big black dots show some important points whose preimages have singular
 points of both the real algebraic (Nash) map into ${\mathbb{R}}^3$ and the function obtained by the composition.}
	\label{fig:2}
	
\end{figure}

We remove this and have a desired open set $D$ by considering the interior to complete the proof in this case.

\ \\
\ \\
\noindent Case 7. A general case $l \geq 4$ with the condition $l_{{\mathbb{N}}_{l-1}}(1)=l_{{\mathbb{N}}_{l-1}}(l-1)=0$ being satisfied. \\
\indent We apply Case 1 by replacing "$t_2$" there by "$t_l$" with "$l_{{\mathbb{N}}_{1}}(1)=0$" there or Case 2 by replacing "$t_2$" and "$t_3$" there by "$t_{l-1}$" and "$t_l$", respectively. 
For each integer $2 \leq j \leq l-2$ in Cases 1 and 2 with $l \geq 4$, we choose $C_{{\rm E},\frac{t_j+t_{j+1}}{2},b_{j,t_j},c_{j,t_j},\frac{(t_{j+1}-t_{j})^2}{4},d_{j,t_j,1},d_{j,t_j,2}} \subset {\mathbb{R}}^3$ for suitable real numbers $b_{j,t_j}$ and $c_{j,t_j}$ and sufficiently small numbers $d_{j,t_j,j^{\prime}}>0$ ($j^{\prime}=1,2$) making the set as a sufficiently small subset contained in the interior of the set "$D_0$" in these cases. We remove the interiors of the subsets to have a new set $D_1$. By defining $D$ as the interior of $D_1$ and applying Proposition \ref{prop:1}, this completes the case in the case $l_{{\mathbb{N}}_{l-1}}(j)=0$ for any $j$. To prove in the general case, for some $j$, we need to choose another set $C_{{\rm E},\frac{t_j+t_{j+1}}{2},{b_{j,t_j}}^{\prime},{c_{j,t_j}}^{\prime},\frac{(t_{j+1}-t_{j})^2}{4},{d_{j,t_j,1}}^{\prime},{d_{j,t_j,2}}^{\prime}} \subset {\mathbb{R}}^3$ for suitable real numbers ${b_{j,t_j}}^{\prime}$ and ${c_{j,t_j}}^{\prime}$ and sufficiently small numbers ${d_{j,t_j,j^{\prime}}}^{\prime}>0$ ($j^{\prime}=1,2$) making the set as a sufficiently small subset contained in the interior of the set $D_1$ here. We also need to remove this closed set for each $j$ to make the situation $l_{{\mathbb{N}}_{l-1}}(j)=1$. We have an open subset $D_2 \subset {\mathbb{R}}^3$ and we set $D:=D_2$ to complete the proof in this case. \\
\ \\
Case 8. A general case with $l \geq 4$ with the condition $l_{{\mathbb{N}}_{l-1}}(1)=l_{{\mathbb{N}}_{l-1}}(l-1)=0$ being not satisfied. \\
\indent We first apply Case 1, Case 3, Case 4, or Case 6. 

In applying Case 1,  we replace "$t_2$" there by "$t_l$" with "$l_{{\mathbb{N}}_{1}}(1)=1$" there.
We can argue as Case 7 to show the case $l \geq 4$ with the condition $l_{{\mathbb{N}}_{l-1}}(j)=1$ being satisfied for any $j$. 
%We also note that we only need to remove the closed sets.

%we replace "$t_2$" and "$t_3$" by "$t_{l-1}$" and "$t_{l}$" respectively.
% or Case 2 by replacing "$t_2$", "$t_3$" and "$t_4$" by "$t_{l-2}$", "$t_{l-1}$" and "$t_l$", respectively. 

In applying Case 3, we replace "$t_2$" and "$t_3$" there by "$t_{l-1}$" and "$t_l$" respectively. We can argue as before for each integer $2 \leq j \leq l-2$ to show the case $l \geq 4$ with the condition $l_{{\mathbb{N}}_{l-1}}(1)=0$ and $l_{{\mathbb{N}}_{l-1}}(l-1)=1$. For Case 4, we can do similarly to show the case $l \geq 4$ with the condition $l_{{\mathbb{N}}_{l-1}}(1)=1$ and $l_{{\mathbb{N}}_{l-1}}(l-1)=0$ being satisfied.

In applying Case 6, we replace "$t_j$" there by $t_{j+l-4}$ for $j=3,4$ and we do not change the role of "$t_2$" there.
We can argue as before for each integer $2 \leq j \leq l-2$ to show the case $l \geq 4$ with the condition $l_{{\mathbb{N}}_{l-1}}(1)=1$ and $l_{{\mathbb{N}}_2}(l-1)=1$ being satisfied.

\ \\
\indent We have a desired Nash map for Main Theorem \ref{mthm:2}
and we can also see that the function represented as the composition of this map with the canonical projection ${\pi}_{3,1}(x_1,x_2,x_3)=x_1$ enjoying all presented properties in Main Theorem \ref{mthm:1}.

We give several remark on singular points of the function represented as the composition of the map with the canonical projection ${\pi}_{3,1}(x1,x_2,x_3)=x_1$.

 First, the pair of the points regarded as "poles" for each subset $C_{{\rm E},\frac{t_j+t_{j+1}}{2},b_{j,t_j},c_{j,t_j},\frac{(t_{j+1}-t_{j})^2}{4},d_{j,t_j,1},d_{j,t_j,2}} \subset {\mathbb{R}}^3$ correspond to singular points of the resulting function. More precisely, the preimages contain some singular points of it.
Vertical lines such as ones in FIGURE 1 and FIGURE 2 play same roles. Local extrema for parabolas such as ones in these figures also play same roles. Intersections of hypersurfaces in $\{S_j\}$ as in Definition \ref{def:1} (and Proposition \ref{prop:1}) also play important roles. Points represented by the black dots in FIGURE \ref{fig:2} show explicit cases. Except such cases, preimages contain no singular points of the function.

Thus we also have Main Theorem \ref{mthm:1}. 
This completes the proof. 

	\end{proof}

\begin{thebibliography}{25}
	\bibitem{akbulutking} S. Akbulut and H. King, \textsl{Topology of real algebraic sets}, MSRI Pub, 25. Springer-Verlag, Nwe York (1992).
	\bibitem{bochnakcosteroy} J. Bochnak, M. Coste and M.-F. Roy, \textsl{Real algebraic geometry}, Ergebnisse der Mathematik und ihrer Grenzgebiete (3) [Results in Mathematics and Related Areas (3)], vol. 36, Springer-Verlag, Berlin, 1998. Translated from the 1987 French original; Revised by the authors.
%	\bibitem{bodinpopescupampusorea} A. Bodin, P. Popescu-Pampu and M. S. Sorea, \textsl{Poincar\'e-Reeb graphs of real algebraic domains}, arXiv:2207.06871.
%\bibitem{bott} R. Bott, \textsl{Nondegenerate critical manifolds}, Ann. of Math. 60 (1954), 248--261.
%\bibitem{costantino}  F. Costantino, \textsl{A short introduction to shadows of $4$-manifolds}, Fundamenta Mathematicae 251 no. 2 (2005), 427--442.
%\bibitem{costantinothurston} F. Costantino, D. Thurston, \textsl{$3$-manifolds efficiently bound $4$-manifolds}, J. Topol. 1 (2008),
%703--745.
%\bibitem{ehresmann} C. Ehresmann, \textsl{Les connexions infinitesimales dans un espace fibre differentiable}, Colloque de Topologie, Bruxelles (1950), 29--55.
%\bibitem{fujitakitabeppumitsuishi} H. Fujita, Y Kitabeppu and A. Mitsuishi, \textsl{Distance functions and convex bodies and symplectic toric manifolds}, arXiv:2003.02293.
%\bibitem{gelbukh} I. Gelbukh, \textsl{Loops in Reeb graphs of $n$-manifolds}, diskrete \& Computational Geometry, 59 (4) (2018), 843--863. 
%%\bibitem{gelbukh2} I. Gelbukh, \textsl{Approximation of Metric Spaces by Reeb Graphs: Cycle Rank of a Reeb Graph, the Co-rank of the Fundamental Group, and Large Components of Level Sets on Riemannian Manifolds}, Filomat (in press), arxiv:1903.00777.
%\bibitem{gelbukh} I. Gelbukh, \textsl{A finite graph is homeomorphic to the Reeb graph of a Morse-Bott function}, Mathematica Slovaca, 71 (3), 757--772, 2021; doi: 10.1515/ms-2021-0018. 
%%\bibitem{gelbukh2} I. Gelbukh, \textsl{Morse-Bott functions with two critical values on a surface}, Czechoslovak Mathematical Journal, 71 (3), 865--880, 2021; doi: 10.21136/CMJ.2021.0125-20. 
%\bibitem{golubitskyguillemin} M. Golubitsky and V. Guillemin, \textsl{Stable Mappings and Their Singularities}, Graduate Texts in Mathematics (14), Springer-Verlag(1974).
%\bibitem{hempel} J. Hempel, \textsl{3- Manifolds}, AMS Chelsea Publishing, 2004. 
%\bibitem{hiratukasaeki} J. T. Hiratuka and O. Saeki, \textsl{Triangulating Stein factorizations of generic maps and Euler Characteristic formulas}, RIMS Kokyuroku Bessatsu B38 (2013), 61--89. 
%\bibitem{hiratukasaeki2} J. T. Hiratuka and O. Saeki, \textsl{Connected components of regular fibers of differentiable maps}, in "Topics on Real and Complex Singularities", Proceedings of the 4th Japanese-Australian Workshop (JARCS4), Kobe 2011,  World Scientific, 2014, 61--73. 
%\bibitem{ishikawakoda} M. Ishikawa and Y. Koda, \textsl{Stable maps and branched shadows of $3$-manifolds}, Mathematische Annalen 367 (2017), no. 3, 1819--1863, arXiv:1403.0596.
%\bibitem{kitazawa1} N. Kitazawa, \textsl{On round fold maps} (in Japanese), RIMS Kokyuroku Bessatsu B38 (2013), 45--59.
%\bibitem{kitazawa2} N. Kitazawa, \textsl{On manifolds admitting fold maps with singular value sets of concentric spheres}, Doctoral Dissertation, Tokyo Institute of Technology (2014).
%\bibitem{kitazawa3} N. Kitazawa, \textsl{Fold maps with singular value sets of concentric spheres}, Hokkaido Mathematical Journal Vol.43, No.3 (2014), 327--359.
\bibitem{kitazawa1} N. Kitazawa, \textsl{On Reeb graphs induced from smooth functions on $3$-dimensional closed orientable manifolds with finitely many singular values}, Topol. Methods in Nonlinear Anal. Vol. 59 No. 2B, 897--912, arXiv:1902.08841.
\bibitem{kitazawa2} N. Kitazawa, \textsl{On Reeb graphs induced from smooth functions on closed or open surfaces}, Methods of Functional Analysis and Topology Vol. 28 No. 2 (2022), 127--143, arXiv:1908.04340.
\bibitem{kitazawa3} N. Kitazawa, \textsl{Real algebraic functions on closed manifolds whose Reeb graphs are given graphs}, a positive report for publication has been announced to have been sent and this will be published in Methods of Functional Analysis and Topology, arXiv:2302.02339.
\bibitem{kitazawa4} N. Kitazawa, \textsl{Real algebraic maps pf non-positive codimensions of certain general types}, submitted to a refereed journal, arXiv:2303.10723.
%\bibitem{kitazawa6} N. Kitazawa, \textsl{Round fold maps and the topologies and the differentiable structures of manifolds admitting explicit ones}, submitted to a refereed journal, arXiv:1304.0618.
%\bibitem{kitazawa0.5} N. Kitazawa, \textsl{Constructing fold maps by surgery operations and homological information of their Reeb spaces}, submitted to a refereed journal, arxiv:1508.05630.
%\bibitem{kitazawa0.6} N. Kitazawa, \textsl{Notes on fold maps obtained by surgery operations and algebraic information of their Reeb spaces}, arxiv:1811.04080.

%\bibitem{kitazawa5} N. Kitazawa, \textsl{Notes on explicit special generic maps into Euclidean spaces whose dimensions are greater than $4$}, a revised version is submitted based on positive comments (major revision) by referees and editors after the first submission to a refereed journal, arxiv:2010.10078.
%\bibitem{kitazawa6} N. Kitazawa, \textsl{On Reeb graphs induced from smooth functions on $3$-dimensional closed manifolds which may not be orientable}, a revised version is submitted to a refereed journal after based on positive comments by editors and referees after the second submission to a refreed journal, arXiv:2108.01300.
%\bibitem{kitazawa7} N. Kitazawa, \textsl{Realization problems of graphs as Reeb graphs of Morse functions with prescribed preimages}, submitted to a refereed journal, arXiv:2108.06913.
%\bibitem{kitazawa10} N. Kitazawa,\textsl{Round fold maps on $3$-dimensional manifolds and their integral and rational cohomology rings}, arXiv:2301.07008.
%\bibitem{kitazawa8} N. Kitazawa, \textsl{A note on cohomological structures of special generic maps}, a revised version is submitted based on positive comments by referees and editors after the second submission to a refereed journal.
%\bibitem{kitazawasaeki1} N. Kitazawa and O. Saeki, \textsl{Round fold maps on $3$-manifolds}, accepted for publication after a refereeing process and to appear in Algebraic \& Geometric Topology, arXiv:2105.00974.
		%	\bibitem{kitazawasaeki2} N. Kitazawa and O. Saeki, \textsl{Round fold maps of $n$-dimensional manifolds into ${\mathbb{R}}^{n-1}$}, submitted to a refereed journal, arXiv:2111.13510.
%\bibitem{ishikawakoda} M. Ishikawa, Y. Koda, \textsl{Stable maps and branched shadows of $3$-manifolds}, arXiv:1403.0596.
%\bibitem{kobayashisaeki} M. Kobayashi and O. Saeki, \textsl{Simplifying stable mappings into the plane from a global viewpoint}, Trans. Amer. Math. Soc. 348 (1996), 2607--2636.
\bibitem{kollar} J. Koll\'ar, \textsl{Nash's work in algebraic geometry}, Bulletin (New Series) of the American Matematical Society (2) 54, 2017, 307--324.
\bibitem{kucharz} W. Kucharz, \textsl{Some open questions in real algebraic geometry}, Proyecciones Journal of Mathematics, Vol. 41 No. 2 (2022), Universidad Cat\'olica del Norte Antofagasta, Chile, 437--448.
%\bibitem{martinezalfaromezasarmientooliveira} J. Martinez-Alfaro, I. S. Meza-Sarmiento and R. Oliveira, \textsl{Topological  classification of simple Morse Bott functions on surfaces}, Contemp. Math. 675 (2016), 165--179.%
\bibitem{masumotosaeki} Y. Masumoto and O. Saeki, \textsl{A smooth function on a manifold with given Reeb graph}, Kyushu J. Math. 65 (2011), 75--84.
\bibitem{maciasvirgospereirasaez} E. Mac\'ias-Virg\'os and M. J. Pereira-S\'aez, Height functions on compact symmetric spaces, Monatshefte f\"ur Mathematik 177 (2015), 119--140. 
\bibitem{michalak} L. P. Michalak, \textsl{Realization of a graph as the Reeb graph of a Morse function on a manifold}. Topol. Methods in Nonlinear Anal. 52 (2) (2018), 749--762, arXiv:1805.06727.
%\bibitem{michalak2} L. P. Michalak, \textsl{Combinatorial modifications of Reeb graphs and the realization problem}, arxiv:1811.08031.
%\bibitem{milnor} J. Milnor, \textsl{Singular points of complex hypersurfacs}, Annals of Mathematics Studies, No. 61, Princeton University Press, Princeton, N. J.; University of Tokyo Press, Tokyo, 1968.
%\bibitem{milnor} J. Milnor, \textsl{Lectures on the h-cobordism theorem}, Math. Notes, Princeton Univ. Press, Princeton, N.J. 1965.
%\bibitem{moise} E. E. Moise, \textsl{Affine Structures in $3$-Manifold{\rm :} V. The Triangulation Theorem and Hauptvermutung}, Ann. of Math., Second Series, Vol. 56, No. 1 (1952), 96--114.
%\bibitem{morin} B. Morin, \textsl{Formes canoniques des singulariti\'{e}s d\'{}une application diff\'{e}rentiable}, C. E. Acad. Sci. Paris 260 (1965), 5662--5665, 6503--6506.
\bibitem{nash} J. Nash, \textsl{Real algbraic manifolds}, Ann. of Math. (2) 56 (1952), 405--421.
%\bibitem{ranicki} A. Ranicki, \textsl{Algebraic and geometric surgery}, https://www.maths.ed.ac.uk/~v1ranick/books/surgery.pdf, 2002.
\bibitem{ramanujam} S. Ramanujam, \textsl{Morse theory of certain symmetric spaces}, J. Diff. Geom. 3 (1969), 213--229.
\bibitem{reeb} G. Reeb, \textsl{Sur les points singuliers d\'{}une forme de Pfaff compl\'{e}tement int\`{e}grable ou d\'{}une fonction num\'{e}rique}, Comptes Rendus
 Hebdomadaires des S\'{e}ances de I\'{}Acad\'{e}mie des Sciences 222 (1946), 847--849.
%\bibitem{saeki1} O. Saeki, \textsl{Notes on the topology of folds}, J. Math. Soc. Japan Volume 44, Number 3 (1992), 551--566.
%\bibitem{saeki1} O. Saeki, \textsl{Topology of special generic maps of manifolds into Euclidean spaces}, Topology Appl. 49 (1993), 265--293.
%\bibitem{saeki0.2} O. Saeki, \textsl{Topology of singular fibers of differentiable maps}, Lecture Notes in Math., Vol. 1854, Springer-Verlag, 2004. 
%\bibitem{saeki4} O. Saeki, \textsl{Morse functions with sphere fibers}, Hiroshima Math. J. Volume 36, Number 1 (2006),  141--170.
\bibitem{saeki} O. Saeki, \textsl{Reeb spaces of smooth functions on manifolds}, International Mathematics Research Notices, maa301, Volume 2022, Issue 11, June 2022, 8740--8768, https://doi.org/10.1093/imrn/maa301, arXiv:2006.01689.
%\bibitem{saekitakase} O. Saeki and M. Takase, \textsl{Desingularizing special generic maps}, Journal of G\"okova Geometry Topology (2013), 1--24.
%\bibitem{sakurai} S. Sakurai, Master Thesis, Kyushu. Univ..
% \bibitem{saekitakase} O. Saeki and M. Takase, \textsl{Desingularizing special generic maps}, Journal of Gokova Geometry Topology 7 (2013), 1--24.
%\bibitem{saeki2} O. Saeki, \textsl{Topology of special generic maps of manifolds into Euclidean spaces}, Topology Appl. 49 (1993), 265--293.
%\bibitem{saeki4} O. Saeki, \textsl{Singular fibers and $4$-dimensional cobordism group}, Pacific J. Math. 248 (2010), 233--256.
%\bibitem{saekisakuma} O. Saeki and K. Sakuma, \textsl{On special generic maps into ${\mathbb{R}}^3$}, Pacific J. Math. 184 (1998), 175--193.
%\bibitem{saekisuzuoka} O. Saeki and K. Suzuoka, \textsl{Generic smooth maps with sphere fibers} J. Math. Soc. Japan Volume 57, Number 3 (2005), 881--902.
\bibitem{sharko} V. Sharko, \textsl{About Kronrod-Reeb graph of a function on a manifold}, Methods of Functional Analysis and
 Topology 12 (2006), 389--396.
 \bibitem{shiota} M. Shiota, \textsl{Nash Manifolds}, Lecture Notes in Mathematics 1269 (1987), Edited by A. Dold and B. Eckmann, Springer-Verlag. 
%\bibitem{shiota} M. Shiota, \textsl{Thom's conjecture on triangulations of maps}, Topology 39 (2000), 383--399. 
%\bibitem{sorea1} M. S. Sorea, \textsl{The shapes of level curves of real polynomials near strict local maxima},  Ph. D. Thesis, Universit\'e de Lille, Laboratoire Paul Painlev\'e, 2018.
%\bibitem{sorea2} M. S. Sorea, \textsl{Measuring the local non-convexity of real algebraic curves}, J. Symbolic Compute. 109 (2022), 482--509.
%\bibitem{stong} R. E. Stong, \textsl{Notes on cobordsm theory}, Princeton Universty Press, 1968.
\bibitem{takeuchi} M. Takeuchi, \textsl{Nice functions on symmetric spaces}, Osaka. J. Mat. (2) Vol. 6 (1969), 283--289.
%\bibitem{thom} R. Thom, \textsl{Les singularites des applications differentiables}, Ann. Inst. Fourier (Grenoble) 6 (1955-56), 43--87.
\bibitem{tognoli} A. Tognoli, \textsl{Su una congettura di Nash}, Ann. Scuola Norm. Sup. Pisa (3) 27 (1973), 167--185.
%\bibitem{turaev} Vladimir G. Turaev, \textsl{Topology of shadows}, Preprint, 1991.
%\bibitem{wall} C. T. C Wall, \textsl{Classification problems in differential topology -- {\rm I:} Classificationon handlebodies}, Topology 2 (1963), 253--261.
%\bibitem{wall2} C. T. C. Wall \textsl{Classification problems in differential topology -- {\rm II:} Diffeomorphismsof handlebodies}, Topology 2 (1963), 263--272.
%\bibitem{wall3} C. T. C. Wall, \textsl{Classification problems in differential topology -- {\rm Q:} Quadratic forms on finite groups and related topics}, Topology 2 (1963), 281--298.
%\bibitem{wall4} C. T. C. Wall, \textsl{Classification problems in differential topology -- {\rm III:} Applications to special cases}, Topology 3 (1965), 291--304.
%%\bibitem{wall5} C. T. C. Wall, \textsl{Classification problems in differential topology -- {\rm IV:} Thickenings}, Topology 5 (1966), 73--94.
%\bibitem{wall6} C. T. C. Wall, \textsl{Classification problems in differential topology -- {\rm VI:} Classification of |{\rm (}$s-1${\rm )}-connected {\rm (}$2s+1${\rm )}-manifolds}, Topology 6 (3) (1967), 273--296.
%\bibitem{whitney} H.  Whitney,  \textsl{On singularities of mappings of Euclidean spaces: I,  mappings of the plane into the plane},  Ann.  of Math.  62 (1955),  374--410. 
\end{thebibliography}
\end{document} 
%\bibitem{wrazidlo} D. Wrazidlo, \textsl{Bordism of constrained Morse functions}, arxiv:1803.11177.&
      % points 
      %  Thom's 
      % An answer will be presented in a forthcoming paper.
%STEP 1
% We delete the exposition via handle attachments to surfaces and decided to avoidusing the figure ''3hd.eps''. We respect the last version. First we start with Michalak'ws argument and generalize.  
%Step 2 Case 1 A_i → E_i
%We corrected some other minor phrases without changing the argument.
%Case 2 We changed the exposision of fold maps: we construct the fold map by using a defromation of smooth functions on an closed and connected surface of genus $0$ with holes (for the functions ''20201125func.eps'' is added). 
%In Problem 3, we adopt this way to present the answer more clearly and we added an exposition on smooth functions and a deformation of these functionsn we need. For STEP 1 and STEP 2 Case 1 in this scene, we introduce functions ${t^{\prime}}_{i,s_1,s_0,s_2}$ and ${t^{\prime}}_{d,s_1,s_0,s_2}$.
%The exposition of an answer to Problem 2 is revised.  