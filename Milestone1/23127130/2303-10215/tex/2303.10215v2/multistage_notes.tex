A the misclassification model first introduced in Section \ref{model} can be extended to a multistage framework. As before, we let $Y = j$ denote an observation's true outcome status, taking values $j \in \{1, 2\}$ and we are interested in the relationship between $Y$ and a set of predictors $X$. Instead of obtaining just one potentially misclassified measurement of $Y$, we now have $a$ sequential imperfect measurements of $Y$. $Y^{*(a)$ denotes the observed outcome from stage $a$ of the data generating process, taking values $k^{(a)} \in \{ 1, 2 \}$. Let $Z^(a)$ denote a set of predictors related to the misclassification of $Y^{*(a)$. The mechanism that generates the observed outcome, $Y^{*(a)$, given the true outcome, $Y$, and all earlier-stage observed outcomes, $Y^{*(a - 1) \dots Y^{*(1)$, is called the $a^{th}$\texit{-stage observation mechanism}. Figure INSERT FIGURE HERE displays the conceptual model for a two-stage misclassification model. EQUATION expresses the conceptual process mathematically.