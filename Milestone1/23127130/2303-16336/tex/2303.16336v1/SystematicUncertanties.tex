\section{Systematic uncertainties}\label{sec:SystematicUncertanties}
\autoref{fig:Geometries} shows that there is very little difference between the on- and off-plane regions in the local reference frame of the detector. The data sets from the two regions also consist of events measured with the same instrumentation in the same location, and have been reconstructed with the same methods. Because systematic uncertainties are derived from measurement effects in the local frame, these similarities between the on- and off-plane samples mean that the majority of uncertainty sources outlined in~\cite{Aab:2014kda} will apply equally to both regions and thereby cancel in a comparison. Furthermore, from the acceptance, resolution, and bias studies in \autoref{sec:ADEffects}, the two regions are also free from selection and reconstruction biases.

\begin{table}[!htb]
    \vspace{2mm}
    \centering
    \caption{Systematic uncertainties on the difference in the means, \Dxmaxmu{}, and widths, \Dxmaxsigma{}, of the on- and off-plane \xmax{} distributions.}\label{tab:SysUncertainties}
    \vspace{-1mm}
    \renewcommand*{\arraystretch}{1.1}
    \setlength{\tabcolsep}{0.5em}
        \begin{tabular}{l|cc}
            \multirow{2}{*}{Source} & \multicolumn{2}{c}{ Uncertainty [\gcm{}] of}\\
            & \Dxmaxmu{} & \Dxmaxsigma{}\\\hline\hline
            $A$ correction  & $^{+1.14}_{-0.71}$ & $^{+2.37}_{-1.61}$ \\
            $B$ correction  & \small{$\pm 0.36$} & \small{$\pm 0.01$} \\
            $R$ correction  & 0                  & $^{+1.78}_{-0.24}$ \\
            Seasonal        & $^{+1.00}_{-1.53}$ & $^{+1.19}_{-1.23}$ \\
            Instrumentation & \small{$\pm 1.41 $}& \small{$\pm 1.41 $}\\ \hline
            %\xmax{} Normalization   & \small{$\pm0.02$} & \small{$\pm0.02$} \\ \hline
            Sum in Quadrature         & $^{+2.10}_\mathbf{-2.23}$ & $^{+3.49}_\mathbf{-2.48}$ \\
        \end{tabular}
    \vspace{-1mm}
\end{table} 

To test for potential systematic effects derived from uncertainties in the $A$, $B$, and $R$ corrections, possible seasonal effects, and small differences between the instrumentation at different fluorescence telescope sites, \textit{FD-sites}, several studies were performed. All permutations of the uncertainties in the $A$, $B$, and $R$ corrections were evaluated and the maximum changes in \Dxmaxmunorm{} and \Dxmaxsigmanorm{} were recorded. To look for on-/off-plane biases associated with instrumentation differences, events seen by two or more FD-sites were used to compare the \xmax{} reconstructions of each site. No significant biases were found, and the instrumentation-derived systematic uncertainties from~\cite{Aab:2014kda} were adopted. To check for systematics derived from a combination of the seasonal dependence of the FD exposure and normal yearly variation of the atmospheric quality, \xmaxmunorm{} and \xmaxsigmanorm{} for the on- and off-plane regions were tracked over the course of the year, and is shown in \autoref{fig:seasonal}. The maximum exposure-weighted difference between the two regions at any point of the year has been adopted as the systematic uncertainty due to seasonal effects. These uncertainties and the total systematic uncertainty on \Dxmaxmunorm{} and \Dxmaxsigmanorm{} are listed in \autoref{tab:SysUncertainties}.
\begin{figure*}[!htb]
            % \centering
            \includegraphics[width=\columnwidth]{Figures/WeightedMeanVariationOnOff.pdf}\hspace{2mm}
            \includegraphics[width=\columnwidth]{Figures/WeightedRMSVariationOnOff.pdf}
            \caption{Seasonal fluctuation of the first (left) and second (right) \xmax{} moments for the on-plane (blue) and off-plane (red) samples. The  double black line shows the difference between these curves. The maximum and minimum values of these differences are taken as the systematic uncertainty on \Dxmaxmunorm{} and \Dxmaxsigmanorm{}.}\label{fig:seasonal}
\end{figure*}


% \begin{table}[!htb]
%     \vspace{-1mm}
%     \centering
%     \renewcommand*{\arraystretch}{1.1}
%     \setlength{\tabcolsep}{0.5em}
%     \begin{tabular}{lc|cc}
%         \multirow{2}{*}{FD site} &\multirow{2}{*}{N events} &
%         \multicolumn{2}{c}{Difference $\text{off}-\text{on}$ [\gcm{}]}\\
%         & & \xmaxmu{} & \xmaxsigma{}\\\hline\hline
%         LL & 167 & $-0.8 \pm 3.7$ & $-3.2 \pm 2.5$\\
%         LM & 181 & $-1.1 \pm 3.7$ & $-1.0 \pm 2.5$\\
%         LA & 198 & $-0.1 \pm 3.2$ & $+0.7 \pm 2.2$\\
%         CO & 230 & $3.0  \pm 3.1$ & $-2.5 \pm 2.1$\\
%     \end{tabular}
%     \caption*{Comparisons of on- and off-plane \xmax{} reconstructions between FD-sites using stereo events.}
% \end{table}

\myparagraph{Confidence level considering systematic uncertainties}
The observed \Dxmaxmunorm{} of $9.1\pm1.6$\,\gcm{} is 4.1 times larger than the 2.2\,\gcm{} systematic uncertainty listed in Table \ref{tab:SysUncertainties}. The observed \Dxmaxsigmanorm{} of $5.9\pm2.9$\,\gcm{} is 2.4 times larger than its 2.5\,\gcm{} systematic uncertainty. This means it is unlikely that the result could be entirely due to systematic effects. However, the systematic uncertainties in \Dxmaxmunorm{} and \Dxmaxsigmanorm{} may increase the likelihood of an extreme result occurring in data. To quantify the result significance taking possible systematic effects into account, a two step approach is taken. First, the on-/off-plane difference is reduced by adding a value sampled from a Gaussian distribution with $\mu = 2.2$\,\gcm{} and $\sigma = 2.5$\,\gcm{} to the on-plane sample. Then, the AD-test is applied to the resulting on- and off-plane distributions. Repeating this process 1 million times results in a mean TS of $11.3\pm0.5$
\footnote{Treating the other side of the systematic errors in the same way results in a $TS$ of $31.8\pm1.1$ ($6.3\,\sigma$).}. 
If these values are converted to significances using the data from \autoref{fig:TStoSignificanceConversionNew}, this corresponds to at least 3.3\,$\sigma{}$. Conservatively, to include systematic effects, this lower bound of 3.3\,$\sigma{}$ is adopted as the confidence level of the result.

\vspace{-.1cm}
\myparagraph{Cross-check: Results by zenith angle and FD-site}
If astrophysical in nature, the difference in composition of UHECR arriving from the on- and off-plane regions should be independently observed by each FD-site and in all zenith angle ($\theta$) ranges. \autoref{fig:EyeZenith} shows that the difference in \xmax{} on and off the plane is indeed present in all zenith angles bins and is also observed by all FD-sites independently. Even more stringently, when the response of each FD-site is split in $\cos^2\theta$ bins, it appears in 22 out of  28 bins. This independent observation at all sites and zeniths is a strong confirmation of the stereo study described above, showing that detector systematics can not play a large role in the result. Furthermore, because the FD-sites have FoVs differing by $90^\circ$ on average, each sees the galactic plane at a different local geometry and time during the year, making it unlikely that some unidentified detector, reconstruction, or atmospheric effect is causing the observed anisotropy. 

\begin{figure}[!htb]
    % \vspace{-.3cm}
    \centering
    \includegraphics[width=.45\textwidth]{Figures/ZenithDiffPlot.pdf}
    \vspace{-3mm}
    \caption{\Dxmaxmunorm{} by FD-site and zenith bin.}\label{fig:EyeZenith}
    % \vspace{-.9cm}
\end{figure} 