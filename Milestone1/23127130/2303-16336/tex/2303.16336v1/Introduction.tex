\section{Introduction}
\vspace{-1mm}
The Pierre Auger Observatory is currently the largest observatory dedicated to studying cosmic rays with energies in the EeV range, so-called ultra-high-energy cosmic rays, \textit{UHECR}~\cite{PierreAuger:2015eyc}. To do this, it uses both an array of particle detectors on the Earth's surface, the Surface Detector, \textit{SD}~\cite{PierreAuger:2007kus}, and an array of fluorescence telescopes monitoring the atmosphere above the SD, the Fluorescence Detector, \textit{FD}~\cite{PierreAuger:2009esk}. The highest quality data set of the Observatory is made up of UHECR events which have simultaneously been measured by both the FD and SD, so-called \textit{hybrid events}. In a hybrid event reconstruction, the geometry of the shower axis is highly constrained by combining the triggered pixel geometry/timing from the FDs and the high confidence core location/timing provided by the SDs. This results in an angular resolution for the pointing direction of the shower axis of better than $0.5^\circ$~\cite{Bonifazi:2009ma}, and a resolution on the location of the shower core of $50$\,m~\cite{Mostafa:2006id}. 

The low uncertainty geometric reconstruction provided by the hybrid method allows the evolution of the intensity of UV fluorescence light measured by the FD to be inverted to model the \textit{shower profile}, which is the number of charged particles in the air shower as a function of the amount of matter it has traversed, the \text{slant depth}, $X$. From the shower profile, the slant depth at which the maximum development of the shower occurs, \xmax{}, can be extracted. \xmax{} is closely related to the mass of the primary cosmic ray which induced the air-shower, but is subject to large fluctuations meaning that it can not be used on a shower-by-shower basis to determine primary mass. However, if collected with sufficient statistics, the first and second moments of distributions of \xmax{}, \xmaxmu{} and \xmaxsigma{} respectively, can be used to make high certainty estimations of the mean mass of the UHECR events used to form that particular \xmax{} distribution~\cite{Aab:2014kda}.

Up until recently~\cite{PierreAuger:2021jlg}, \xmax{} derived from hybrid measurements has been used to study the average composition of the cosmic-ray sky as whole, rather than being used to compare the mean compositions of different subsets of the sky. This choice was likely driven by the relatively sparse statistics available in hybrid studies due to the upper-limit of exposure available to them being set by the relatively low 14\,\% up-time of FDs.
%, which is mostly driven by the dark, moonless, and clear conditions required by the FD technique. 
The possibility of splitting the data set into subsamples was therefore limited by the need to maintain sufficient statistics to say something useful about primary composition. However, the Pierre Auger Observatory has now collected more than 14-years of FD data, and tens of thousands of high-quality hybrid measurements. With this quantity of data, the sky can be split into different regions and the mean mass of UHECRs arriving from them can be studied. This new reality then prompts two questions:
\begin{enumerate}
    \item Are the systematic uncertainties which trend with event arrival direction for hybrid reconstruction low enough to allow the mean mass arriving from different regions of the sky to be meaningfully compared?
    \item Is there a reason to expect that different regions of the sky may display differing compositions due to astrophysical causes?
\end{enumerate}

Question 1) will be explored in \autoref{sec:SystematicUncertanties}. For question 2) it is clear that the opportunity exists. The flux above the ankle at ${\sim}5$\,EeV~\cite{PierreAuger:2020kuy} is mixed in composition and has long been thought (now confirmed) to be extragalactic in origin~\cite{Linsley:1963bk}. Furthermore, it definitively displays anisotropy above 8\,EeV~\cite{Aab:2017tyv}. Additionally, as was nicely put by Alan Watson in 1990:
\begin{myquote}{3mm}
    \textit{``... the Larmor radius of a proton of $10^{18}$\,eV in a 3\,$\mu$G field is about 400\,pc, comparable to the thickness of the galactic disk. It follows, therefore, that, if the bulk of cosmic rays are protons, anisotropies associated with the magnetic field structure of the galactic disk might appear as the energy increases.''} - \cite{Watson:1990fj}
\end{myquote}
Indeed, there were hints of such a spectral feature starting somewhere around $10^{18.5}$\,eV for directions within $30^\circ$ of the galactic plane ~\cite{Szabelski:1986rx,Watson:1990fj}. Unfortunately, such an excess so far does not appear to be significant in the data of current experiments and therefore has not been given much attention since those initial publications. 

Now it is known that the flux above 1\,EeV is best described as an evolving mix of light-, intermediate-, and high-mass primaries~\cite{pierre2014aab, PierreAuger:2021mmt}. Due to the galactic magnetic field, \textit{GMF}, the different mass components present at any given energy will be deflected to different degrees as they travel from their extragalactic sources to Earth. This mass dependent deflection suggests that an anisotropy associated with the structure of the GMF would kick in for increasingly heavier components as energies climb. It is therefore distinctly possible that an anisotropy associated with the galactic plane could arise in the higher mass components of the flux at some energy in the EeV range.  

What follows below are specific tests for such a mass-dependent anisotropy associated with the galactic plane using hybrid data of the Pierre Auger Observatory collected between 2004 and the end of 2018. To avoid repeating the contents of the ICRC 2021 proceedings on this result~\cite{PierreAuger:2021jlg} in its entirety, the contents of this proceeding will aim to include components of the analysis that could not be fit in the eight pages allotted in that publication. Therefore, while this proceeding will cover the details of the analysis, increased space will be given to the cross checks and studies of the systematics uncertainties of the analysis. Additionally, a new test on an independent hybrid data set, recovered from the quality cuts, will also be discussed.