\section{Independent Test of the Result}

To cross-check the validity of the previous results, an independent data set is needed. The selection presented in \autoref{sec:2} aims to select only high-quality events. In particular, the \textit{FidFoV} cut is needed to ensure that the distribution of selected events covers the expected range of \xmax{} values with an unbiased acceptance. The events rejected by the FidFoV cut are still high-quality and perfectly well reconstructed. In this section, a second data set is built from the events removed in this FidFoV cut to form an \textit{out-FidFoV} data set which is 82.9\,\% as large as the data set already used in this analysis, the \textit{in-FidFoV} data set.

\begin{figure}[!hbt] \centering
    \includegraphics[width=0.42\textwidth]{Figures/Acceptance_out_FidFoV}
    \vspace{-2mm}
    \caption{Top panel: acceptance for out-\textit{FidFoV} data set split in 0.1\,\lge{} energy bins. The energy in the legend is the lower energy edge of the bin, excepting 19.7 which corresponds to $\log(E/\text{eV}) \geq 19.7$. Lower panel: the \xmax{} distribution of events for the full energy range.}
    \label{fig:acceptance}
\end{figure}

\autoref{fig:acceptance} represents the \xmax{} acceptance of this out-FidFoV data set in different energy ranges.  For all energies, the efficiency maximized for low \xmax{} values and decreases as \xmax{} increases. Indeed, without the FidFoV selection, deep showers tend to be under-represented \citep{Aab:2014kda}. Thus, the new data set is made out of shallower \xmax{} events. As consequence, the \xmaxmu{} distribution of the new data set is on average $10$ g/cm$^2$ shallower. Despite this bias, if the difference between on- and off-plane is astrophysical, it should also appear in the out-FidFoV data set.

To test this hypothesis, this new data set is divided into on- and off-plane samples using the $E_{\rm th} = 10^{18.7}$\,eV, $b_{\rm split}=30^\circ$ splitting determined by the scan. The resulting on/off distributions show a somewhat smaller \Dxmaxmu{} of ${\sim}5$g/cm$^2$, which is only 55\,\% of what was obtained with the in-FidFoV data set. The AD-test returns a $TS = 1.8$. The post-trial significance of this test is re-evaluated with the method in \autoref{sec:TestingAnisotropy}, using two million randomized MC trials generated from the out-FidFoV data set. The corresponding $TS$ distribution is represented by the orange histogram in \autoref{fig:FF_TS}. The red dashed line depicts the AD-test obtained for the data set. The corresponding significance is ${\sim}2.2\sigma$, which gives a probability of 0.03 that this would result from an isotropic sky. The significance seen in this sample is, therefore, much lower than with the in-FidFoV data set. 

\begin{figure}[hbtp] \centering
    \includegraphics[width=0.4\textwidth]{Figures/out_FidFoV_Anderson-Darling_Test_forward_folding_Xmax_galactic}
    \vspace{-2mm}
    \caption{Anderson-Darling test from randomized skies (orange) and from forward folding (blue) for the out-\textit{FidFoV} data set. The red dashed line indicates the value obtained for data in the on/off test.}
    \label{fig:FF_TS}
\end{figure}

The lower significance of the out-FidFoV sample begs the question, why is there such a large difference between the in-FidFoV and out-FidFoV data sets? The bottom panel of \autoref{fig:acceptance} shows the distribution of \xmax{} values from events. Clearly the bulk of the distribution lies in a region where the acceptance is decreasing quickly. It is possible that this could reduce the sensitivity to a difference in composition. To test this hypothesis, two million mock data sets have been generated from the on- and off-plane \xmax{} distributions of the in-FidFoV data set. The non-flat acceptance and lower \xmax{} resolution of the out-FidFoV data set has then been forward folded onto these samples to create out-FidFoV mock data sets which assume the difference in-FidFoV is real. The AD-test is then computed for each mock data set. The blue histogram represents the corresponding $TS$ distribution. It peaks exactly at the value observed with the out-FidFoV data set, showing that the out-FidFoV is indeed less sensitive to the tested anisotropy. Overall, the independent test with the out-FidFoV shows that the on/off separation is present in both data sets, and that the difference seen in the out-FidFoV sample is consistent with the magnitude of the difference seen in the in-FidFoV sample. 