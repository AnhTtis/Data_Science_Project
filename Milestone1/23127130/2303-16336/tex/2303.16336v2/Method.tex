\section{Testing for anisotropy}\label{sec:TestingAnisotropy}
The specific hypothesis to be tested is whether, above some energy threshold, $E_{\rm th}$, the mean composition of UHECRs coming from directions near to the galactic plane is significantly higher in mass than those arriving further from it. This is to be tested using \xmax{} as a mass sensitive parameter. Typically, \xmax{} based composition analyses leverage the first two moments of \xmax{} distributions binned in energy, to comment on primary mass. This approach, however, does not lend itself well to quantifying the significance of a result testing the above statement. Instead, a test statistic, $TS$, which quantifies the degree of dissimilarity between the \xmax{} distributions in the two regions in a single value is preferred. For this, the returned value from the Anderson-Darling two-sample homogeneity test \cite{andersondarling}, \textit{AD-test}, has been selected as it scales with the dissimilarity of the tested distributions. The AD-test has good sensitivity to the full width of a distribution \cite{scholz1987k}, and has more power than the Kolmorogov-Smirnov test while remaining robust against false positives \cite{engmann2011comparing}.

To use the AD-test and \xmax{} for this purpose, two modifications are required. First, a single $TS$ comparing all events in each region above $E_{\rm th}$ is desired. So, all events with $E\geq E_{\rm th}$ in the on- and off-plane samples separately need to be collected into a common on-plane distribution and a common off-plane distribution. To do this, the natural evolution of \xmax{} with energy needs to be removed so that spectral features in the flux do not influence the result. Therefore, we define an energy-normalized \xmax{} value
\begin{equation}\label{eq:XmaxNorm}
X_{\text{max}}^{'} =  X_{\text{max}} -  \underbrace{\left(649 + 63.1 \, Z_{18} + 1.97 \, Z_{18}^{2}\right)}_\text{EPOS-LHC elongation rate for iron},
\end{equation}
where $Z_{18}=\log_{10}\left(E_\text{rec}/\,\text{EeV}\right)$. The last term in \autoref{eq:XmaxNorm} is the natural energy evolution of mean \xmax{} for iron primaries as predicted by EPOS-LHC~\cite{Pierog:2013ria}\footnote{Choice of hadronic interaction model varies result by $\sim0.02$\,\gcm{}.}. Second, the \xmaxnorm{} distribution of an on-plane sample populated with primaries which are on average heavier than those in the off-plane sample will display a lower mean and a narrower width than that of the off-plane \xmaxnorm{} distribution. Since the null hypothesis is that there is either no composition difference or a heavier off-plane sample, a $TS$ sensitive to the ordering of the \xmaxnorm{} distributions is required\footnote{Modifying the test to also require $\sigma( X_{\text{max}}^\prime)^{\rm on} < \sigma( X_{\text{max}}^\prime)^{\rm off}$ would be more restrictive, but conservatively has not been applied.}. The AD-test is insensitive to ordering, so it is modified to
\begin{equation}
TS =
\begin{cases}
    AD: \langle X_{\text{max}}^\prime \rangle^{\rm on} < \langle X_{\text{max}}^\prime \rangle^{\rm off} \\
    -3\hspace{1mm}: \text{else}
\end{cases},
\end{equation}
where $AD$ is the result of the AD-test comparing the on- and off-plane distributions, and $-3$ is selected as it is well below the minimum of the AD-test.

\vspace{-.1cm}
\myparagraph{Scan for energy and galactic latitude thresholds}
\vspace{-.1cm}
A scan has been used to select the optimal on/off splitting latitude, $b_{\rm split}$, and minimum energy, $E_{\rm th}$, as uncertainties in GMF models and source distributions make other approaches impractical. In this scan, each trial [$E_{\rm th}$, $b_{\rm split}$] pair is used to form on- and off-plane subsets and the $TS$ is extracted. To preserve the statistical strength of the sparse FD data set, a coarse scan of $5^\circ$ steps in $\abs{\,b\,}$ from $20^\circ$ to $35^\circ$ and 0.1\,\lge{} steps in energy from $18.4$ to $19.4$\,\lge{} is used. The scan is performed on the data set from~\cite{Aab:2014kda}, which includes events through Dec 31\textsuperscript{st} 2012. At the time of writing, this \textit{scan data set} represents $54\,\%$ of the analyzed events. The remaining $46\,\%$ of events, the \textit{post-scan data set}, is reserved as blind. 

\begin{figure}[!htb]
    % \vspace{-.4cm}
    \centering
    \includegraphics[width=0.45\textwidth]{Figures/ScanResults.pdf}
    \vspace{-2mm}
    \caption{Parameter scan over 54\% of the data.}\label{fig:PRDScan}
    % \vspace{-7mm}
\end{figure}

Interestingly, as shown in \autoref{fig:PRDScan}, all tested pairs result in $\langle X_{\text{max}}^\prime \rangle^{\rm on} < \langle X_{\text{max}}^\prime \rangle^{\rm off}$. An optimal [$E_{\rm th}$, $b_{\rm split}$] of [$10^{18.7}$\,eV,$30^\circ$] was found with a $TS = 8.4$. The selected [$E_{\rm th}$, $b_{\rm split}$] is applied as a prescription to the post-scan data set, which independently confirms the result with a $TS = 12.6$, for a total $TS=21.0$ for the full data set. 

\vspace{-.1cm}
\myparagraph{Statistical significance}
\vspace{-.1cm}
The chance probability of the observed TS occurring with in an isotropic sky is tested using Monte Carlo methods on randomized skies derived from the real data. To form each randomized sky, the arrival direction is first decoupled from the energy and \xmaxnorm{} values of each event. These are then randomly re-paired to create a new sky which maintains the real \xmax{}, energy, and sky exposure distributions, but has a scrambled arrival direction/composition pairing. The above analysis is then used to extract a $TS$ from each sky which is compared to the result in data. Skies which display more extreme on-/off-plane differences than those observed in data are tallied and used to calculate the probability of an isotropic sky generating the observed $TS$. The results of this procedure are shown in  \autoref{fig:TStoSignificanceConversionNew}.

\begin{figure}[!htb]
    \centering
    \includegraphics[width=.8\columnwidth]{Figures/MCADSig.pdf}
    % \vspace{-3mm}
    \caption{The Monte Carlo determination of the post-scan (red) and all-data (blue) significance with 1 and 10 billion randomized skies, respectively.}\label{fig:TStoSignificanceConversionNew}
    % \vspace{-3mm}
\end{figure}

For the blind, post-scan data set, the prescribed [$E_{\rm th},b_{\rm split}$] pair is used to split each randomized sky into on- and off-plane samples and a $TS$ is extracted. In one billion random skies, only 5865 resulted in a more extreme $TS$ than the 12.6 observed in data. This indicates a chance probability of $5.87\times10^{-6}$ which corresponds to 4.4\,$\sigma$. 

To calculate the significance of the result when the scan- and post-scan data sets are combined, the entire analysis chain, including the scan, is duplicated. In each random sky, 54\,\% of the data is used to scan for the [$E_{\rm th},b_{\rm split}$] pair which results in the most extreme result, fully penalizing for the scan. These values are then used to split all data in the random sky into on- and off-plane subsamples and the $TS$ for the sky is extracted. From 10 billion random skies, only 5964 resulted in a more extreme $TS$ than the 21.0 observed in data. This indicates to a chance probability of $5.96\times10^{-7}$ which corresponds to 4.9\,$\sigma$. The strong penalization of the scanned data is evident as the additional 54\,\% of the data (with \Dxmaxmunorm{} $= 8.5$\,\gcm{}) only resulted in an 11\,\% increase of the significance of the observation. 

\myparagraph{\xmax{} moments and trends}

To illustrate the difference in composition on and off the plane, the first two moments of the \xmax{} distribution in each 0.1\,\lge{} energy bin has been plotted in \autoref{fig:CompositionPlots} for both regions. Above $10^{18.7}$\,eV there is a clear separation in \xmaxmu{} for all energy bins. Most energy bins also display a separation in \xmaxsigma{}. Heavier primaries are expected to, on average, have a shallower \xmax{} and lower shower-to-shower fluctuations. Therefore the correlated difference seen here indicates that, for this data sample, primaries from the on-plane region have a higher mean mass than that of the off-plane region above $10^{18.7}$\,eV.

To evaluate the degree to which fluctuation plays a role in the observed result, the growth of the $TS$ over time has been plotted in \autoref{fig:TimeEvolution}. The time evolution of the signal is consistent with linear growth at a rate of 1.3\,$TS$\,yr$^{-1}$. This behavior is in line with expectations for a real difference in mean mass between the subsamples. The shaded region of \autoref{fig:TimeEvolution} shows preliminary data from 2019. These reconstructions were not subject to a validated reconstruction chain and may change. Still, when added, a 3.7/4.4\,$\sigma$ (post-scan/all data) statistical significance is expected. The best fit rate of growth of 1.3\,$TS$\,yr$^{-1}$ remains unchanged.

\begin{figure*}[!hbt]
\centering
    \begin{minipage}{.63\textwidth}
      \centering
      \captionsetup{width=.9\linewidth}
      \includegraphics[width=.49\textwidth,valign=t]{Figures/Mean-crop.pdf}
      \includegraphics[width=.49\textwidth,valign=t]{Figures/Sigma-crop.pdf}
      \vspace{-1mm}
      \caption{The first (left) and second (right) moments of the \xmax{} distributions from on- and off-plane regions.}
      \label{fig:CompositionPlots} 
    \end{minipage}%
    \hfill
    \begin{minipage}{.35\textwidth}
      \centering
      \vspace{-1mm}
      \includegraphics[width=\textwidth,valign=t]{Figures/TimePredict.pdf}
      \vspace{1.5mm}
      \captionof{figure}{The time evolution of the TS with significance indicated on the right. The shaded region is preliminary data.}
      \label{fig:TimeEvolution}
    \end{minipage}
\end{figure*}