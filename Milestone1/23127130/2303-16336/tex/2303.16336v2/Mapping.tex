\section{Mapping the UHECR sky in composition}

\vspace{-.1cm}
To aid interpretation of the latitude-dependent difference in composition, a test statistic quantifying the relative difference in \xmax{} between different parts of the sky is mapped in \autoref{fig:Map}, for UHECR primaries with $E \geq 10^{18.7}$\,eV. To produce this map from the in-FidFoV data set, first the requirement $E \geq 10^{18.7}$\,eV is imposed. Then,  because small portions of the sky are analysed, in contrast to the on/off study, each event has its $B$, $R$, and $A$ corrected for based on its arriving declination instead of its arriving galactic latitude. This is because local geometry has a time-independent relationship with arrival declination. This means $B$, $R$, and $A$ can be corrected equally well for each direction in the sky\footnote[1]{Using declination-dependent corrections changes the on/off comparison only by $+0.1$\,\gcm{} and increases systematic uncertainties.}. 

At this point, a top-hat sampling is used to collect all events with arrival directions within $30^\circ$ of a point $(\ell,b)$ into an \textit{in-hat} sample. All other events are placed in an \textit{out-hat} sample. The distributions of \xmaxnorm{} for the in-hat and out-hat samples are then compared using Welch's t-test~\cite{welch1938significance}:
\begin{equation}
    TS = \frac{\langle X_{\text{max}}^{\prime\,\text{in}} \rangle - \langle X_{\text{max}}^{\prime\,\text{out}} \rangle}{\sqrt{\left(\sigma\left( X_{\text{max}}^{\prime\,\text{in}} \right)/\sqrt{N^{in}}\right)^2 + \left(\sigma\left( X_{\text{max}}^{\prime\,\text{out}} \right)/\sqrt{N^{out}}\right)^2}},
\end{equation}
where $N^{in}$ and $N^{out}$ are the event counts for the in- and out-hat samples respectively\footnote{Because Welch's t-test considers event statistics, the FD arrival direction-dependent exposure is naturally treated through its use.}. This procedure is repeated for top-hats centered on each point in a $5^\circ$ by $5^\circ$ galactic latitude and longitude grid. The result is shown in \autoref{fig:Map}, which illustrates the relative composition of UHECRs with $E\geq 10^{18.7}$\,eV arriving from each point in the sky.

In \autoref{fig:Map}, positive $TS$ values (red) indicate that events within $30^\circ$ of that point have a lighter mean mass than the rest of the sky. Negative values (blue) indicate that events within $30^\circ$ of that point have a heavier mean mass than the rest of the sky. An excess of heavy particles within $30^\circ$ of the galactic plane is visible. This can not be  due to detector systematics as they would be declination dependent and appear as radial patterns centered on $\ell = -57^{\circ}, b =-27^{\circ}$ due to the geographic location of the Observatory. 
\begin{figure*}[!htb]
    \centering
    \includegraphics[width=.75\textwidth]{Figures/CompMap_FDFidFoV_E187to202_Z0to90_R30_Tophat.pdf}\vspace{-2mm}
    \caption{Sky map of comic ray composition for $E \geq 10^{18.7}$ eV}
    \label{fig:Map}
    \vspace{-1mm}
\end{figure*}