\title{Update on the indication of a mass-dependent anisotropy above 10\textsuperscript{18.7}\,eV in the hybrid data of the Pierre Auger Observatory }

\author{
    \firstname{Eric} \lastname{Mayotte}\inst{1,2}\fnsep\thanks{\email{emayotte@mines.edu}}
    \and
    \firstname{Thomas} \lastname{Fitoussi}\inst{3}\
    for the \lastname{Pierre Auger Collaboration}\inst{4}\fnsep\thanks{\email{spokespersons@auger.org}}
}

\institute{
    Colorado School of Mines, Department of Physics, Golden, CO, USA
    \and
    Bergische Universit\"at Wuppertal, Department of Physics, Wuppertal, Germany
    \and
    Karlsruhe Institute of Technology (KIT), Institute for Astroparticle Physics, Karlsruhe, Germany
    \and
    Observatorio Pierre Auger, Av.\ San Mart\'in Norte 304, 5613, Malarg\"ue, Argentina.\\ Full author list: \href{https://www.auger.org/archive/authors_2022_10.html}{https://www.auger.org/archive/authors\_2022\_10.html}
}

\abstract{%
  We test for an anisotropy in the mass of arriving cosmic-ray primaries associated with the galactic plane. The sensitivity to primary mass is obtained through the depth of shower maximum, \xmax{}, extracted from hybrid events measured over a 14-year period at the Pierre Auger Observatory. The sky is split into distinct on- and off-plane regions using the galactic latitude of each arriving cosmic ray to form two distributions of \xmax{}, which are compared using an Anderson-Darling 2-samples test. A scan over roughly half of the data is used to select a lower threshold energy of $10^{18.7}$\,eV and a galactic latitude splitting at $|b| = 30^\circ$, which are set as a prescription for the remaining data. With these thresholds, the distribution of \xmax{} from the on-plane region is found to have a $9.1 \pm 1.6^{+2.1}_{-2.2}$\,\gcm{} shallower mean and a $5.9\pm2.1^{+3.5}_{-2.5}$\,\gcm{} narrower width than that of the off-plane region and is observed in all telescope sites independently. These differences indicate that the mean mass of primary particles arriving from the on-plane region is greater than that of those from the off-plane region. Monte Carlo studies yield a $5.9\times10^{-6}$ random chance probability for the result in the independent data, lowering to a $6.0\times10^{-7}$ post-penalization random chance probability when the scanned data is included. Accounting for systematic uncertainties leads to an indication for anisotropy in mass composition above $10^{18.7}$\,eV with a $3.3\,\sigma$ significance. Furthermore, the result has been newly tested using additional FD data recovered from the selection process. This test independently disfavors the on- and off-plane regions being uniform in composition at the $2.2\,\sigma$ level, which is in good agreement with the expected sensitivity of the dataset used for this test. 
}