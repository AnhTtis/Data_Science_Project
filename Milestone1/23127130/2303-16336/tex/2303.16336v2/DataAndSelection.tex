\vspace{-1mm}
\section{Reconstruction and selection}\label{sec:2}
\vspace{-1mm}
The reconstruction methods, selection cuts, and core analysis of \xmax{} distributions used here are largely the same as those described in~\cite{Yushkov:2020nhr}. A rigorous description of these methods can be found in~\cite{Aab:2014kda}. Other than the fiducial field-of-view cut, \textit{FidFoV}, which is treated below, a detailed description of the methods will not be provided here. 

The important differences from~\cite{Yushkov:2020nhr} are that the minimum energy for inclusion in the data set has been raised to $E > 10^{18.4}$\,eV, and that, as described below in \autoref{sec:ADEffects}, the event acceptance, reconstruction bias, and \xmax{} resolution are now treated based on the arrival direction of each event. The lower limit of $10^{18.4}$\,eV has been chosen as above this energy, the composition is well mixed and expected to be primarily of extragalactic origin \cite{Aab:2016zth}. The period over which data has been collected has also been slightly expanded to span events observed between 2004-12-01 and 2018-12-31, yielding 7572 high-quality events. A further subdivision of the data is necessary to test for the hypothesized anisotropy. Following the results of the scan shown in \autoref{fig:PRDScan}, the data is split into the events with a galactic latitude, $\abs{\,b\,} \leq 30^\circ$, the \textit{on-plane} sample, and, $\abs{\,b\,} > 30^\circ$, the \textit{off-plane} sample.

\vspace{-1mm}
\paragraph{Fiducial field-of-view selection}
\vspace{-1mm}
To ensure a good reconstruction of \xmax{} with the FD, \xmax{} itself should be directly observed. The telescopes of the FD have a field-of-view, \textit{FoV}, which is vertically constrained. Additionally, some showers in more vertical geometries will not reach \xmax{} before impacting the ground. These factors together lead to a geometric and \xmax{} dependence for what events end up in the analyzed data set. If unaccounted for, this \textit{\xmax{} acceptance}, will inevitably bias a composition study based on FD \xmax{} data. At the Observatory, this \xmax{} acceptance is addressed primarily through mitigation of the effect with the fiducial field-of-view, \textit{FidFoV}, cut. The FidFoV cut constrains the FD detector volume to only event geometries where the expected range of \xmax{} values would be visible in the FD FoV. As can be seen in \autoref{fig:FidAcc}, this changes the natural FD \xmax{} acceptance (gray) to one which is unbiased from ${\sim}600$ to ${\sim}900$\,\gcm{}, which spans the typical range of observed event \xmax{} values. Remaining effects on rare events outside this range are corrected using parameterizations as a function of \xmax{} and primary energy.

\begin{figure}[!htb]
    % \vspace{-1mm}
    \centering
    \includegraphics[width=\columnwidth]{Figures/AcceptanceExample.pdf}
    % \vspace{-6mm}
    \caption{FD \xmax{} acceptance before (gray) and after (black) FidFoV cuts. A 4-variable parameterization of the post-FidFoV acceptance is shown in red. The range of \xmax{} values with unbiased sampling is shown in blue.}
    \label{fig:FidAcc}
    % \vspace{-10mm}
\end{figure}

\subsection{Distributions of \xmaxbf{} and arrival direction}\label{sec:ADEffects}
After measurement, reconstruction, and selection, the observed \xmax{} distribution does not quite represent the true \xmax{} distribution of all cosmic rays landing within the Observatory. This is due to energy dependent biases on the reconstruction of \xmax{} ($B$), the resolution on \xmax{} of the hybrid reconstruction method ($R$), and the residual effects of the \xmax{} acceptance ($A$). Since the location of an event and its inclination with respect to the observing fluorescence telescope plays a role in the magnitude of $A$, $R$, and $B$, they have an inherent geometric dependence. If unaccounted for, this could bias this study. Distributions of the key geometric relationships between events and the FD can be seen in \autoref{fig:Geometries} for on- and off-plane regions. It is clear that there is little difference between the two regions, so $A$, $R$, and $B$ are expected to also be similar. 

\begin{figure}[!htb]
    \centering
    \vspace{2mm}
    \includegraphics[width=0.47\columnwidth]{Figures/ZenithOnOff.pdf}
    \includegraphics[width=0.47\columnwidth]{Figures/CoreDistOnOff.pdf}\\
    \includegraphics[width=0.47\columnwidth]{Figures/VAODOnOff.pdf}
    \includegraphics[width=0.47\columnwidth]{Figures/Chi0OnOff.pdf}
    \vspace{-1mm}
    \caption{Geometries of the on- and off-plane samples.}\label{fig:Geometries}
    \vspace{3mm}
\end{figure}

To explicitly verify $A$, $R$, and $B$ similarity on and off the galactic plane, CONEX \cite{Bergmann:2006yz} is used to generate showers with Sibyll-2.3c~\cite{Riehn:2015oba}. The showers are then isotropically thrown into detector simulations which include the time-dependent state of the FD and SD from 2004 through 2018. These simulations, therefore, mimic the measurement conditions, trigger efficiency, and up-time of the real data, accurately modelling the exposure and geometries of events arriving from all parts of the sky~\cite{Abreu:2010aa}. Two sets of these simulations are produced, one formed from showers generated with a flat sampling of \xmax{} between 300 and 1500\,\gcm{}, the \textit{flat-MC}, and one formed from an equal number of proton, helium, nitrogen, and iron simulations which are then weighted to their abundances observed in data as reported in~\cite{Bellido:2017cgf}, the \textit{mixed-MC}. These simulated event sets are then subjected to the same reconstruction and selection techniques used on the real data, so that $A$, $R$, and $B$ are accurately included in them.

The flat-MC is split into on- and off-plane subsamples which are then used to extract the functional form of $A$ in 0.1\,\lge{} energy bins using the method illustrated in \autoref{fig:FidAcc}. The form of $A$ is extracted by leveraging the flatly sampled \xmax{} generation of the Monte Carlo, as, once the plateau of the distribution is normalized to one, the height of each bin represents the acceptance of events with \xmax{} values in that range. The acceptance is then fit with the 4-component parameterization illustrated in \autoref{fig:FidAcc}. The energy evolution of $x_1,\lambda_1,x_2$, and $\lambda_2$ is then parameterized separately with 2D polynomials for the on- and off-plane subsamples, resulting in \autoref{fig:OnOffAccParam}. From \autoref{fig:OnOffAccParam}, it is clear that above $10^{18.4}$\,eV there is no statistically significant difference in \xmax{} acceptance between the on- and off-plane regions. Even so, the region-specific parameterizations of $A$ are used to correct the 1.4\,\% of events with partial \xmax{} acceptance using the up-weighting technique outlined in~\cite{Aab:2014kda}. Uncertainties in these parameterizations result in the systematic uncertainties on the first and second moments specified in \autoref{tab:SysUncertainties}.

\begin{figure}
    \includegraphics[width=\columnwidth]{Figures/OnOff_Fid_ParCompare.pdf}
    \vspace{-6mm}
    \caption{The \xmax{} acceptance parameterizations for the on- and off-plane sky regions from Monte Carlo.}
    \label{fig:OnOffAccParam}
    % \vspace{-4mm}
\end{figure}

\begin{figure}
    % \vspace{6mm}
    \includegraphics[width=\columnwidth]{Figures/OnOff_WithFid_Bias.pdf}
    \vspace{-6mm}
    \caption{The \xmax{} reconstruction bias and resolution parameterizations for the on- and off-plane sky regions from Monte Carlo. Note: only the detectors and reconstruction \xmax{} resolution is shown. Other effects lowering the resolution are included as specified in~\cite{Aab:2014kda}.}
    \label{fig:OnOffBiasAndResParam}
    % \vspace{-4mm}
\end{figure}

The mixed-MC is likewise split into on- and off-plane subsamples which are then used to extract the energy evolution of $B$ and $R$ for each region. $B$ and $R$ are extracted by forming a distribution of the difference between the FD reconstructed value of \xmax{} and the Monte Carlo truth value in 0.1\,\lge{} energy bins. From these, the mean reconstruction bias, $B=\langle X_{\rm max}^{\rm FD} - X_{\rm max}^{\rm MC} \rangle$, and the \xmax{} resolution, $R=\sigma \left(X_{\rm max}^{\rm FD} - X_{\rm max}^{\rm MC} \right)$, are extracted in each energy bin. Again, the evolution of each is parameterized with a 2D polynomial. \autoref{fig:OnOffBiasAndResParam} shows that $B$ and $R$ for the two regions are found to agree within errors. These however are also corrected for separately. Fit uncertainties again result in the systematic uncertainties outlined in \autoref{tab:SysUncertainties}. At this point, the \xmax{} distributions and moments from the on- and off-plane regions in each energy bin can be compared without bias from selection and reconstruction.

