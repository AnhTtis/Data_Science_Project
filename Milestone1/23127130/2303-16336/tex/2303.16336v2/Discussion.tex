\section{Conclusions and Outlook}
There is an apparent difference in the mean mass of primaries with energies greater than $10^{18.7}$\,eV that arrive from within $30^\circ$ of the galactic plane. This has been observed at least at the $3.3\,\sigma$ level in the standard hybrid data set used for $X_{\text{max}}$-based composition analyses. It has now been independently confirmed with an additional $2.1\,\sigma$ significance in a second hybrid data set formed from high quality events cut by a selection aimed at reducing the bias caused by the $X_{\text{max}}$-dependent event acceptance. The combined significance of these two results has not yet been evaluated. Further tests of the on-/off-plane difference are being planned using analyses of data from the SD and will be reported elsewhere. 

Currently, this result should be considered to primarily provide a new verification of a mixed composition above the ankle as it is clear no such difference could be observed in a flux with a  single mass component. Though the analysis provides a possible indication that the galactic magnetic field may have an observable impact on mass-dependent anisotropies, the result found in this analysis does not necessarily support a causal relationship with galactic structures. The differing horizons of different nuclear species at a given energy could also result in composition-dependent anisotropic patterns~\cite{Ding:2021emg}. It is important, however, to note there is significant tension with models~\cite{Allard:2021ioh}. Alternative scenarios are being explored along these lines of thought.