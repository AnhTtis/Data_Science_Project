\documentclass[11pt]{extarticle}
\usepackage{latexsym}
\usepackage{theorem}
\usepackage{graphicx}
\usepackage{amsmath,color}
\usepackage{amsfonts}
%\usepackage{endfloat}
%\usepackage{harvard}
\usepackage{natbib}
\usepackage{soul}
\usepackage{enumerate}


\usepackage{tikz}
\usepackage{caption}
\usepackage{subcaption}
\usepackage{comment}
\newcommand{\red}[1]{{\color{red}#1}}

\headsep 0pt
\headheight 0pt
\topmargin 0pt
\oddsidemargin 0pt
\evensidemargin 0pt
\textwidth 6.5in 
\textheight 8.75in

\theorembodyfont{\rmfamily}
\newtheorem{theorem}{Theorem}
\newtheorem{conjecture}[theorem]{Conjecture}
\newtheorem{lemma}[theorem]{Lemma}
\newtheorem{proposition}[theorem]{Proposition}
\newtheorem{corollary}[theorem]{Corollary}
\newtheorem{definition}[theorem]{Definition}

\theoremstyle{break}
\newtheorem{remark}[theorem]{Remark}
\newtheorem{example}[theorem]{Example}
\newcommand{\om}{\omega}
\newcommand{\la}{\lambda}
\newcommand{\al}{\alpha}
\newcommand{\be}{\beta}
\newcommand{\ep}{\epsilon}
\newcommand{\si}{\sigma}
\renewcommand{\baselinestretch}{1.37}
\newcommand{\mbf}[1]{\mbox{\boldmath$#1$}}
\newcommand{\smbf}[1]{\mbox{\scriptsize \boldmath$#1$}}

\newcommand{\blue}[1]{{\color{blue}#1}}


\usepackage{mathtools}
\DeclarePairedDelimiter\ceil{\lceil}{\rceil}
\DeclarePairedDelimiter\floor{\lfloor}{\rfloor}

\newenvironment{proof}{\paragraph{Proof.}}{\hfill$\square$}



\title{Searching a Tree with Signals: Routing Mobile Sensors for Targets
Emitting Radiation, Chemicals or Scents}

\date{}
\author{Steve Alpern\thanks{Warwick Business School, University of Warwick, Coventry CV4 7AL, UK, steve.alpern@wbs.ac.uk} \and Thomas Lidbetter\thanks{Department of Engineering Systems and Environment, University of Virginia, VA 22903, USA, tlidbetter@virginia.edu} \thanks{Rutgers Business School, Newark, NJ 07102, USA, tlidbetter@business.rutgers.edu}}


%\linespread{1.25}

\providecommand{\keywords}[1]{\textbf{\textbf{Keywords:}} #1}

\linespread{1.5}

\begin{document}
	
\maketitle

\begin{abstract}
Adversarial search of a network for an immobile Hider (or target) was
introduced and solved for rooted trees by \cite{Gal79}. In this zero-sum game, a Hider picks a point to hide on the tree and a Searcher picks a unit speed trajectory starting at the root. The
payoff (to the Hider) is the search time. In Gal's model (and many
subsequent investigations), the Searcher receives no additional information
after the Hider chooses his location. In reality, the Searcher will often receive such locational information. For
homeland security, mobile sensors on vehicles have been used to locate
radioactive material stashed in an urban environment. 
In a military setting, mobile sensors can detect chemical signatures from
land mines. In predator-prey search, the predator
often has specially attuned senses (hearing for wolves, vision for eagles,
smell for dogs, sonar for bats, pressure sensors for sharks) that may help
it locate the prey. How can such noisy locational information be used by the
Searcher to modify her route? We model such information as signals which
indicate which of two branches of a binary tree should be searched first,
where the signal has a known accuracy $p<1$. Our solution calculates which
branch (at every branch node) is \textit{favored}, meaning it should always
be searched first when the signal is in that direction. When the signal is
in the other direction, we calculate the probability the signal should be
followed. Compared to the optimal Hider strategy in the classic search game of Gal, the Hider's optimal distribution for this model is more skewed towards leaf nodes that are further from the root.
\end{abstract}

\keywords{game theory, signals, zero-sum games, networks}

\newpage

\section{Introduction}

The game where a Searcher wishes to find a stationary adversarial Hider (or
target), starting from a designated location in a search region $S$, was
introduced in the classic book on Differential Games of \cite{Isaacs}.
Later, \cite{Gal79} solved the game where $S$ is a tree, considering the
Searcher starting point as the root. It is optimal for the Hider to locate at a leaf node, as
other locations are dominated. The optimal hiding distribution is the Equal
Branch Density (EBD) distribution, which locates in each branch at a branch node
with a probability proportional to the length of that branch. It is optimal for the Searcher,
when reaching a branch node for the first time, to choose to fully
search each branch equiprobably, and when returning to the branch node to
search the other branch. This description assumes a binary tree, but any tree can be transformed into a binary tree with the addition of some arcs of length zero. 
The Searcher thus searches in a depth-first manner and traces out a minimal length tour, or {\em Chinese
Postman Tour}. The value of the game is the total length $\mu $ of the tree.

A particular assumption made by Isaacs and Gal is that the Searcher receives
no additional information (on the Hider's location) during the search. In
reality, this assumption is often unwarranted, as the Hider might emit
signals (odor, radiation, sound) that the Searcher might be able to detect.
The aim of this paper is to see how such information changes the optimal
strategies of Searcher and Hider. If the signals given out by the Hider were
perfect, the solution is simple: the Hider locates at the furthest point
from the Searcher starting point and the Searcher proceeds directly to the
Hider's location. However in practice the Hider's location is a noisy signal
at best. Our model gives the Searcher a signal as to which branch at her
current branch node contains the Hider -- it is correct with known
probability $p$. When $p$ is $1/2$ (for a binary tree) the signal is useless
and the game reduces to that solved by Gal. When $p>1/2,$ we identify a
favored branch at each branch node: a signal that the Hider is in that
branch is always followed, while we determine the probability that the alternative signal should be followed. We also
determine the optimal hiding distribution, as a function of $p$. Examples
illustrating the solution are given in Section~\ref{sec:main}, before our formal
analysis. 

The motivation for giving the Searcher additional information in the form of
signals comes from real world examples and prior academic work which we put
into a game theoretic context for the first time here. In the field of
Homeland Security, \cite{Hochbaum14} analyzed the use of mobile sensors
in cities to locate cached nuclear material through the emitted radiation.
In a military setting, \cite{JA15} considering the detection of
landmines and improvised explosive devices. Sometimes specially trained dogs
can use their sense of smell to the same end \cite[see][]{Evans22}. The optimal
routing problem for mobile sensors has been explored by \cite{Paley16}. But
none of these investigations have been carried out in a game theoretic
context, where the target is hidden adversarially. 

In the field of predator search for prey (stationary or mobile), sensory
cues are important and explain in part the highly developed senses of many
predators. \cite{HM13} observe that ``Most motile organisms use
sensory cues when searching for resources, mates, or prey{\ldots}Yet,
classical models of species encounter rates assume that searchers move
independently of their targets.'' This failure also appears in some classical
hide-seek game models which we attempt to remedy here. Further work on
detection of targets during search is mentioned in our literature review
(Section~\ref{sec:lit}).

\section{Literature Review} \label{sec:lit}


We now give a short overview of work on network search games since \cite{Gal79} as well as some further examples of sensory detection of targets in a non game theoretic context. 

The pioneering work of \cite{Gal79} on tree search, described in our opening paragraph, has been extended in many ways. More general networks were analyzed by \cite{RP93} and \cite{Gal2001}. 
Computational methods for determining optimal strategies were given by \cite{AA}. 
The requirement for a designated Searcher starting point was removed by \cite{DG08}. \cite{AL13, AL14} studied search games
on windy networks and by expanding regions rather than paths. \cite{Alpern17} restricted the search paths on general networks to ``combinatorial'' ones consisting of sequences of edges. 
Other investigations considered search at nodes of a lattice \citep{Zoroa13} and costs for searching at nodes \citep{BK15}. The requirement to bring the target back to the root after capture (find-and-fetch) was considered by \cite{Alpern11}. \cite{Angel20} considered the {\em linear search problem} in the setting where the Searcher has a ``hint'' about the location of the target. For general discussions of search games, see \cite{AG03}, \cite{Garnaev} and \cite{Hohzaki16}.

There has been considerable work on the detection of targets by sensors (machines) or senses (animals). An abstract computer science approach is given in \cite{Patan12}. A search technique which combines vehicles and drones has been analyzed by \cite{GF19}. In the field of ecology, \cite{NON20} show how the ladybird beetle uses olfcatory signals in their walks on plants to find aphids. Females made more use of these signals than males. The location signals can sometimes come in multiple forms, as in \cite{Catania08}, where movement, shape and smell of prey can all be detected by water shrews. 


\section{Illustration of Main Results} \label{sec:main}

Our main result is that for our game $G(Q,O,p)$, every branch node $j$ has a {\em favored branch}. With some probability $\beta=\beta(j)$ the Searcher searches the favored branch first, whatever the signal is. We call $\beta$ the {\em favoring bias}, and it is calculated according to Equation~({\ref{beta}) in Theorem~\ref{thm:value}.
With probability $1-\beta$ the Searcher follows the signal, whichever direction it points.
That is, she searches the branch the signal implies the Hider is in. 
When she arrives back at the branch node, she searches the other branch. 
In particular, the Searcher never searches (first) the unfavored branch when the signal indicates the favored branch. 
For simple two-arc trees, or penultimate nodes (whose two branches are both leaf arcs), the favored branch is the longer one. 
The hiding strategy is different that the one found by Gal for the no-signal case, but is also given by a recursive algorithm.

To illustrate the nature of the optimal Searcher strategy we describe the solution for the simplest tree which has more than two arcs. 

\begin{example} \label{ex1}
Consider the tree shown in Figure~\ref{fig:network}. The quantitative recursive calculations of the favoring biases will be given in Section~\ref{sec:example}, but for now we qualitatively describe the optimal Searcher strategy.  

\begin{figure}[h]
\center
\includegraphics[width=8cm]{network}
\caption{A tree with root $O$.}
\label{fig:network}
\end{figure}

The tree of Figure~\ref{fig:network} has two branch points, $O$ (Searcher starting point) and $A$.
The Hider can choose one of the two leaf nodes on the left ($L_1$ and $L_2$) or the leaf node $R$ on the right, as all other points are clearly dominated by these.
Suppose that $p = 2/3$. 
The optimal Searcher strategy is given below in Figure~\ref{fig:network-sol}.

\begin{figure}[h]
\center
\includegraphics[width=8cm]{network-sol}
\caption{A tree with root $O$, favored branches thickened.}
\label{fig:network-sol}
\end{figure}

We have to specify what the Searcher does, probabilistically, at each of her branch nodes $A$ and $O$. 
At node $A$, the favored branch is the longer arc $AL_2$ (indicated by the thickened line) and the favoring bias is given by $\beta =1/16$, which is calculated according to (\ref{beta}), as described later in Theorem~\ref{thm:value}.
So when the Searcher reaches $A$ she goes to $L_{2}$ before or without looking at the signal with probability $1/16$ and otherwise simply follows the signal. 
At $O$ the favored branch is the arc $OR$ and the favoring bias is $\beta =1/112$.
So at the start of the game, the Searcher goes to $R$ with probability $1/112$, otherwise she follows her signal.

\end{example}

\section{Formal Description of the Model}
\label{sec:model}

Here we give a more detailed description of the game  than in the Introduction. We define a game $G=G(Q,O,p)$, where  $Q$ is a tree, given as a metric space (in that arcs have lengths), $O$ is a point of $Q$ and $1/2<p\le1$. 
The Hider chooses a leaf node (other hiding strategies are dominated) and then the Searcher conducts some unit speed minimum length tour of the tree (that is, a depth-first search).
We assume that $Q$ is a binary tree: that is, each node has degree either three or one with the possible exception of the root $O$. (Any tree that is not binary can be converted to a binary tree by adding some arcs of length $0$.) We call every node of degree three a {\em branch node}, and if $O$ has degree two, we also call $O$ a branch node. At each branch node $j$, we call the set of points at least as far away from $O$ as $j$ the {\em subtree at $j$}. Removing $j$ from the subtree at $j$ partitions this subtree into two connected components we refer to as the {\em branches} at $j$.

When arriving at any branch node $j$, the Searcher receives a signal as to which branch, $Q_1$ or $Q_2$ the Hider belongs to. This signal is correct with the fixed and known probability $p$, and wrong with probability $q=1-p$. 
If the Hider lies in neither branch, any signal distribution may be used, as in this case the Searcher will return to node $j$ again after time equal to twice the length of the branches, regardless of her search method. As $p \rightarrow 1/2$, the signal becomes useless and the solution of the game reverts back to that of the game without any signals, as solved in \cite{Gal79}. 

%We choose not to let $p$ depend on $j$ but this is possible for future models. 
The Searcher picks which branch to search with knowledge of the signal, but she does not have to
follow the signal (in fact it is almost never optimal to always follow the signal). 
The payoff is the time~$T$ for the Searcher to reach the Hider. A Searcher mixed strategy can
be given by specifying her choice of branch on her first
arrival to each branch node, dependent on the signal. It is a distribution
over four strategies, specifying which of two ways to go based on the binary signal:
choose $Q_{1}$ with any signal, choose $Q_{2}$ with any signal, follow the
signal, oppose the signal. The last unintuitive choice will indeed be shown
to be dominated.





The solution for the Searcher will have the following structure. At every branch node $j$ there is a favored branch $Q_1$ and a positive probability $\beta$ (the {\em favoring bias}) for it to be chosen before looking at the signal. 
With the remaining probability $1-\beta$ the search follows the signal. 
So in particular the Searcher will never choose the unfavored arc (branch) when the signal is for the favored one. The use of biased depth-first Searcher strategies (random choices at every branch node) of the Searcher was introduced in another context in \cite{Alpern10} and \cite{AL14}, but those distributions are not the same is in the present context.

The optimal Hider distribution over the leaf nodes can be found by a similar stochastic process in which the Hider starts at the root $O$ and at each branch node chooses a branch to enter according to a certain distribution. Of course, this is merely a mental calculation for the Hider, who is stationary in this game. 

Some qualitative findings are that for all penultimate nodes (where both branches are single arcs), the longer arc is favored, and that it is (almost) never optimal for the Searcher to simply follow her signal.

A final, fairly obvious, observation is that having signals cannot hinder the Searcher, because she could always ignore them. 
That is, the value of the game with signals cannot be larger than that of the classic search game without signals. 
Since the value of this classic search game on a tree is equal to its total length $\mu$ \citep{Gal79}, this must be an upper bound for the value of the search game with signals on a tree. 
For later purposes, we state the following.

\begin{lemma}[$V<\mu$] \label{lem:value}
The game $G(Q,O,p)$ has a value not larger than the total length $\mu$ of $Q$.
\end{lemma}

Later we can show that $V=\mu $ {\em implies} that the tree is a single arc (with $O$ at one end).

For the sake of completeness, we could establish Lemma~\ref{lem:value} directly by repeating Gal's idea of having the Searcher play an equiprobable mixture of a Chinese Postman Tour of the tree $Q$ and its time reversed tour. 
Such a mixed strategy reaches every point $H$ in $Q$ in expected time not more than the length $\mu $ of $Q$ because such a tour has length $2\mu$.

\section{Solution of the Game}
\label{sec:solution}

We first give some basic definitions that will enable us to state our main results on the value of the game $G(Q,O,p)$.

\begin{definition}
We define the {\em mean depth} $D$ of a rooted tree, with respect to a given probability measure $\lambda$ on its leaf nodes, as the mean distance from the root to the set $\mathcal{L}$ of leaf nodes, weighted with respect to $\lambda$. More precisely, we define
\[
D = D_Q = D(Q,O,\lambda) = \sum_{v \in \mathcal{L}}  \lambda(v) ~ d(O,v),
\]
\end{definition}

In this paper we will take $\lambda$ to be the optimal Hider strategy $\bar{\lambda}$. We will show that $\bar{\lambda}$ is optimal in Theorem~\ref{thm:value}. Clearly, $D \le D_{\max} \equiv \max_{v \in \mathcal{L}} d(O,v)$, where $D_{\max}$ is usually called the {\em depth} of the tree.


\begin{definition}[Optimal Hider distribution $\bar{\lambda}$.] The optimal Hider distribution $\bar{\lambda}=\bar{\lambda}_Q$ on a binary tree $Q$ with root $O$ is concentrated on the leaf nodes of $Q$ and will be defined recursively. 
If $Q$ has only one arc $OA$, then $\bar{\lambda}_Q(A)=1$.

Suppose now that $\bar{\lambda}_Q$ has been defined for all trees with at most $m$ arcs. Let $Q$ have $m+1$ arcs. If $O$ has degree $1$ with an arc leading to a branch $Q'$, then $Q$ and $Q'$ have the same leaf nodes, and the Hider distribution on these is the same.

Now suppose that $O$ is a branch node with branches $Q_1$ and $Q_2$ rooted at $O$ whose optimal hiding distributions are $\bar{\lambda}^1$ and $\bar{\lambda}^2$, respectively. Let $\mu_1$ and $\mu_2$ denote their respective total lengths. Without loss of generality, suppose that $Q_1$ has greater mean depth than $Q_2$, that is $D(Q_1,\bar{\lambda}^1) \ge D(Q_2,\bar{\lambda}^2)$.

In this case, we will call $Q_1$ the {\em favored branch}. We now define the optimal Hider distribution on $Q$ (on its leaf nodes) by the formula
\begin{align}
\bar{\lambda}(v) = 
\begin{cases}
\frac{p \mu_1}{p \mu_1 + q \mu_2} \bar{\lambda}^1(v) \text{ if $v$ is a leaf node of $Q_1$ and} \\
\frac{q \mu_2}{p \mu_1 + q \mu_2} \bar{\lambda}^2(v) \text{ if $v$ is a leaf node of $Q_2$.}
\end{cases}
\label{eq:lambda-bar}
\end{align}
\end{definition}

We illustration the optimal Hider strategy with the following simple example

\begin{example}
Suppose $Q$ is a tree with two arcs, $OA$ of length $3$ and $OB$ of length $5$. With no signals ($p=1/2$), the EBD distribution of \cite{Gal79} says that the optimal probabilities of hiding at $A$ and $B$ are proportional to their lengths: that is, $3/8$ and $5/8$, respectively. Since $D_{OB}=5>3=D_{OA}$, clearly $OB$ is the favored branch, so on $Q$ we have
\[
\bar{\lambda}(B) = \frac{p \cdot 5}{p \cdot 5 + q \cdot 3}(1) = \frac{5p}{2p+3} > \frac{5}{8},
\]
where the inequality follows from our assumption $1/2 < p <1$. Thus we see that when there are signals the weights on leaf nodes are no longer proportional, but skewed further towards the longer branches. So for this tree we have that the mean depth is given by
\[
D=\bar{\lambda}(A) \cdot 3 + \bar{\lambda}(B) \cdot 5 =  \frac{3 - 3p}{2p+3} (3) + \frac{5p}{2p+3}(5) = \frac{16p+9}{2p+3}.
\]



We observe that as $p$ goes to $1$ and $q$ to $0$, the distribution of $\bar{\lambda}$ becomes concentrated on the leaf node at greatest distance from $O$, and $D$ converges to that distance. As $p$ goes to $1/2$, the distribution of $\bar{\lambda}$ converges to the EBD distribution.

\end{example}

Before stating our main theorem, we make an observation.

\begin{proposition} \label{prop:Delta}
Suppose $Q$ is a tree with root $O$ and first suppose that $O$ has degree $1$. Let $A$ be the neighbor of $O$ and let $Q'$ be the subtree rooted at $A$. Let $\ell$ denote the length of the arc $OA$. Then 
\[
D_Q=\ell + D_Q'.
\]
Now suppose $O$ has degree 2 and let $Q_1$ and $Q_2$ be the branches at $O$ with lengths $\mu_1$ and $\mu_2$, respectively and suppose $Q_1$ is the favored branch (${D_{Q_1} \ge D_{Q_2}}$). Then
\[
D_Q = \frac{p \mu_1 D_{Q_1} + q\mu_2 D_{Q_2}}{p\mu_1+q\mu_2}.
\]
\end{proposition}
\begin{proof}
For the first part of the proposition, we calculate
\begin{align*}
D_Q &=  \sum_{v \in \mathcal{L}}  \bar{\lambda}(v) ~ d(O,v) \\
&=  \sum_{v \in \mathcal{L}}  \bar{\lambda}(v) (\ell + d(A,v)) \\
&= \ell + \sum_{v \in \mathcal{L}}  \bar{\lambda}(v) ~ d(A,v) \\
&= \ell+ D_Q',
\end{align*}
where the penultimate equality follows from the fact that $\bar{\lambda}$ is a probability distribution on $\mathcal{L}$ and the final equality follows from the fact that the leaf nodes of $Q$ and $Q'$ are the same.

For the second part of the proposition, let $\mathcal{L}_1$ and $\mathcal{L}_2$ be the leaf nodes of $Q_1$ and $Q_2$, respectively. Also, let $\bar{\lambda}^1$ and $\bar{\lambda}^2$ be the optimal Hider strategies on $Q_1$ and $Q_2$, respectively.  Then
\begin{align*}
D_Q &=  \sum_{v \in \mathcal{L}_1}  \bar{\lambda}(v) ~ d(O,v) +  \sum_{v \in \mathcal{L}_2}  \bar{\lambda}(v) ~ d(O,v) \\
&= \left( \frac{p\mu_1}{p\mu_1 + q\mu_2} \right) \sum_{v \in \mathcal{L}_1}  \bar{\lambda}^1(v) ~ d(O,v) +   \left(\frac{q\mu_2}{p\mu_1 + q\mu_2} \right) \sum_{v \in \mathcal{L}_2}  \bar{\lambda}^2(v) ~ d(O,v) \\ 
&= \frac{p \mu_1 D_{Q_1} + q\mu_2 D_{Q_2}}{p\mu_1+q\mu_2},
\end{align*}
where the penultimate equality follows from the definition of $\bar{\lambda}$ and the final equality follows from the definition of $D_{Q_1}$ and $D_{Q_2}$.
\end{proof}

We can now state and prove our main theorem, which includes an expression for the value of the game. We describe the optimal strategy for the Searcher by giving the favoring bias $\beta$ of searching the favored branch first (without needing to observe the signal) when at a branch node.
\begin{theorem} \label{thm:value}
Let $(Q,O)$ be a rooted tree with length $\mu = \mu(Q)$. 
\begin{enumerate}[(i)]
\item The value $V=V(Q)$ of the  game $G(Q,O,p)$ is
\begin{equation}
V(Q)=2q  \mu + (p-q)D_Q.
\label{V}
\end{equation}
\item The hiding distribution $\bar{\lambda}$ is optimal for the Hider. 
\item When at a branch node with branches $Q_1$ and $Q_2$ of lengths $\mu_1$ and $\mu_2$ it is optimal for the Searcher to search the favored branch first with probability 
\begin{equation}
\beta=\frac{(p-q)(D_{Q_1}-D_{Q_2})}{2(p \mu_1 + q \mu_2)},
\label{beta}
\end{equation}
 where, without loss of generality, $Q_1$ is the favored branch. With the complementary probability $1-\beta$ the Searcher follows the signal, whichever direction it points.
\end{enumerate}
\end{theorem}

\begin{proof} 
The proof is by induction on the number of arcs of $Q$. If $Q$ has only one arc of length $\mu$, then $D(Q)=\mu$ and the right-hand side of~(\ref{V}) simplifies to $\mu$, which is trivially equal to the value of the game. Each player only has one strategy.

Now suppose the theorem is true for all trees with up to $m$ arcs, and let $Q$ have $m+1$ arcs. First suppose $O$ has degree one, and let $A$ be its neighbor. Let $Q'$ be the subtree rooted at $A$ and denote its length by $\mu'$. Let $\ell$ be the length of the arc $OA$. Then it is easy to see that $V(Q)=\ell+V(Q')$, so by the induction hypothesis,
\[
V(Q)=\ell+ 2q  \mu' + (p-q)D_{Q'} = \ell + 2q(\mu-\ell) +(p-q)D_{Q'}.
\]
By Proposition~\ref{prop:Delta}, we have $D_{Q'}=D_Q-\ell$, so
\[
V(Q) = \ell + 2q(\mu-\ell) + (p-q)(D_Q - \ell) = 2q \mu + (p-q)D_Q.
\]
Thus, part (i) is proven, and parts (ii) and (iii) are trivially true.

Finally, suppose that $O$ is a branch node with branches $Q_1$ and $Q_2$, where $Q_1$ is the favored branch. Let $\mu_1$ and $\mu_2$ be the lengths of $Q_1$ and $Q_2$, respectively.
Since we assume the Searcher chooses a depth-first search, the Hider has two strategy classes. He can play the optimal mixed strategy on the leaves of $Q_1$ or the optimal mixed strategy on the leaves of $Q_2$. 
For the Searcher, she has two possible signals ($1$ or $2$) and two searches (search $Q_{1}$ first or search $Q_{2}$ first). 
This gives her four strategies $\left[ X,Y\right] $ where $X$ is the index of the subtree she searches first with signal $1$ and $Y$ is the index of the subtree she searches first with signal $2$. 
We call the strategy $\left[ 1,2\right]$ ``follow'' (go with the signal) and the strategy $\left[ 2,1\right]$ ``opposite''. 
It is easy to see that the resulting matrix game is given by Table~\ref{tab:matrix}, where $V_1$ is the value of the game played on $Q_1$ and $V_2$ is the value of the game played on $Q_2$.
\begin{table}[h!]
\caption{\label{tab:matrix} Payoff matrix.} 
\center
\begin{tabular}{l|l|l|l|l}
Hider \textbackslash Searcher & $[1,1]$ & $[2,2] $ & $[1,2] =$ ``follow'' & $[2,1]=$ ``opposite''
  \\ \hline
1 & $V_1$ & $2\mu_2+V_1$ & $q\cdot 2\mu_2 +V_1 $ & $p \cdot 2\mu_2 + V_1 $   \\ 
2 & $2\mu_1+V_2$ & $V_2$ & $q \cdot 2\mu_1 +V_2 $ & $p \cdot 2\mu_1 + V_2$  
\end{tabular}
\end{table}

We explain only one entry, that of row 2 and column 3 (``follow''). The Hider is in $Q_{2}$ so with probability $p$ the signal is $2$, in which case the Searcher goes to $Q_{2}$ first and finds the Hider in expected time $V_2$. 
With probability $q$ the Searcher wastes time $2 \mu_1$ searching $Q_{1}$, covering every arc twice, before finding the Hider in
additional expected time $V_2$. 
We now show that ``follow'' dominates ``opposite'' for any $p>1/2$ by the simply taking the last column of the matrix from the third column.
\[
\left( 
\begin{array}{c}
p\cdot 2\mu_2 +V_1   \\ 
p \cdot 2\mu_1 +V_2 
\end{array}
\right) -\left( 
\begin{array}{c}
q \cdot 2\mu_2+ V_1 \\ 
q \cdot 2\mu_1 + V_2
\end{array}
\right)  \\
=\left( 
\begin{array}{c}
2\mu_2(2p-1) \\ 
2\mu_1(2p-1) 
\end{array}
\right).
\]
Since $p>1/2$, it follows that $2p-1>0$. So ``opposite'' is dominated by ``follow''.

If the Hider chooses row 1 (hide optimally in $Q_1)$ with
probability $x$, the payoffs (expected capture times) $T_{\left[ 1,1%
\right] },T_{\left[ 2,2\right] }$ and $T_{\left[ 1,2\right] }$
corresponding to the Searcher's three undominated columns are given as
linear functions of $x$, as 
\begin{align*}
T_{[1,1]}(x)  &=  xV_1 +(1-x)(2\mu_1+V_2) \\
T_{[2,2]}(x)  &= x(2\mu_2+V_1) + (1-x)V_2 \\
T_{[1,2]}(x) &= x (q \cdot2\mu_2 +V_1) +(1-x)(q \cdot 2\mu_1+V_2).
\end{align*}
The payoff functions of the two no-signal strategies $T_{[1,1]}$ and $T_{[2,2]}$ intersect at the value
\[
\bar{x}=\frac{\mu_1}{\mu_1+\mu_2},\text{ with }T_{[1,1]}\left( \bar{x}\right) =T_{[2,2]}\left( 
\bar{x}\right) =\frac{\mu_1 V_1+\mu_2V_2 + 2\mu_1 \mu_2}{\mu_1+\mu_2}.
\]
Note that $T_{[1,1]}$ is decreasing, because its slope is%
\[
V_1-V_2-2\mu_1 \leq \mu_1-V_2-2\mu_1 = -(V_2 +\mu_1) <0,
\]
where the first inequality follows from Lemma~\ref{lem:value}. 

Similarly, $T_{[2,2]}$ is increasing because its slope is%
\[
V_1-V_2+2\mu_2 \ge V_1-\mu_2+2\mu_2 = V_1+\mu_2 >0,
\]
again applying Lemma~\ref{lem:value}. 

Finally, $T_{[1,2]}$ is decreasing, since its slope is
\[
q \cdot 2\mu_2 +V_1 - q \cdot 2\mu_1  -V_2 = (p-q)(D_{Q_1}-D_{Q_2}) > 0 
\]
where the  equality follows from the induction hypothesis and the inequality follows from the fact that $Q_1$ is the favored branch and $p > q$.

At the point $\bar{x},$ the line $T_{\left[ 1,2\right] }\left( x\right) $ lies below the intersection of the lines $T_{[1,1]}(x)$ and $T_{[2,2]}(x)$.
This can be seen by calculating
\[
T_{[1,2] }( \bar{x})  = \frac{\mu_1V_1+\mu_2 V_2+4q\mu_1 \mu_2}{\mu_1+\mu_2}\\
\]
and observing that
\begin{align*}
T_{[2,2]}( \bar{x}) -T_{[ 1,2] }( \bar{x})  &=\frac{\mu_1V_1+\mu_2 V_2+2\mu_1 \mu_2}{\mu_1+\mu_1}-\frac{\mu_1 V_1+\mu_2 V_2+4q \mu_1 \mu_2}{\mu_1+\mu_2}\\
&=\frac{2\mu_1 \mu_2( 1-2q) }{\mu_1+\mu_2}>0,
\end{align*}
because $q<1/2$. The full picture is sketched out in Figure~\ref{fig:graph}.
\begin{figure}[h]
\center
\includegraphics[width=8cm]{sketch}
\caption{A sketch of the functions $T_{[1,1]}(x)$, $T_{[2,2]}(x)$ and $T_{[1,2]}(x)$.}
\label{fig:graph}
\end{figure}

It is clear that the minimum of $T_{[1,1]}(x)$, $T_{[2,2]}(x)$ and $T_{[1,2]}(x)$ has a unique maximum at the intersection of $T_{[1,1]}(x)$ and $T_{[1,2]}(x)$, which occurs at the point
\[
x^* = \frac{p\mu_1}{p\mu_1+q\mu_2} = \bar{\lambda}(Q_1).
\]
This is therefore the optimal choice of $x$ for the Hider, and we have proven part (iii) of the theorem. 

Applying the induction hypothesis, the value of the game is
\begin{align*}
V = T_{[1,1]}(x^*)  &= \frac{p\mu_1V_1}{p\mu_1+q\mu_2} +\frac{q\mu_2(2\mu_1+V_2)}{p\mu_1+q\mu_2} \\
&= \frac{p\mu_1(2q\mu_1 + (p-q)D_{Q_1})}{p\mu_1+q\mu_2} +\frac{q\mu_2(2\mu_1+2q \mu_2 + (p-q)D_{Q_2})}{p\mu_1+q\mu_2} \\
& = 2q \mu + (p-q)D_Q.
\end{align*}
by Proposition~\ref{prop:Delta}. We have proven part (i) of the theorem.

For part (ii), the Searcher's optimal strategy must mix between her two best responses to the Hider strategy calculated above: that is between strategies 1 and [1,2] (``follow''). Using the principle of indifference we calculate the probability $\beta$ of searching the favored branch $Q_1$ first which makes the Hider indifferent between hiding in the two subtrees $Q_1$ and $Q_2$.
\begin{align*}
\beta V_1+(1-\beta) (q\cdot 2 \mu_2 +V_1) & = \beta(2\mu_1+V_2) + (1-\beta)(q \cdot 2\mu_1 +V_2), \text{ so} \\
\beta( 2p\mu_1  + 2q\mu_2 ) &=V_1 - V_2  -2q(\mu_1-\mu_2)\text{ and}\\
\beta &= \frac{ (p-q)(D_{Q_1}-D_{Q_2})}{2(p\mu_1  + q\mu_2) },
\end{align*}
by the induction hypothesis. Thus, we have established (\ref{beta}).
\end{proof}

Note that as the signal becomes more certain, so that $p \rightarrow 1$ and $q \rightarrow 0$, the value (\ref{V}) goes to $D_{Q}$, which goes to the distance of the furthest leaf node from $O$. As previously remarked, the distribution of $\bar{\lambda}$ becomes concentrated on the leaf node at furthest distance from $O$ (assuming the generic case where there is a unique such node). The favored branch will always be the one containing this leaf node, and the signal will always be accurate, so the Searcher will always take this branch.

As the signal becomes more uncertain, so that $p ,q \rightarrow 1/2$, the value (\ref{V}) goes to $\mu$, the hiding distribution $\bar{\lambda}$ goes to the EBD distribution, and the probability $\beta$ goes to 0, so that the Searcher follows the signal (which is determined by the toss of a fair coin) with probability 1.

We also note that if the value of the game is $\mu$, then $\mu=2q\mu+(p-q)D_Q$, so that
\begin{align}
(p-q) \mu = (p-q) D_Q \label{eq:value-mu}
\end{align}
As we remarked earlier, $D_Q \le D_{\max}$ and also $D_{\max} \le \mu$. Since $p > q$, we must have $D_{Q}=D_{\max}=\mu$ for (\ref{eq:value-mu}) to hold, so $Q$ must be a single arc with $O$ at one end.

\begin{corollary} \label{cor:pen-node}
Suppose a penultimate node has branches (leaf arcs) of length $\ell,s$, with $s<\ell$.
Then the favoring bias $\beta$ to the favored long arc is given by 
\begin{align}
\beta &=\frac{(p-q)(\ell-s)}{2(p\ell+qs)},
\label{fpen}
\end{align}
the optimal probability of hiding in the long arc is given by
\begin{align}
x^*&=\frac{p \ell }{p \ell+qs }  \label{xf pen}
\end{align}
and the value of the game by 
\begin{align}
V&= 2q(\ell+s) + \frac{(p-q) (p \ell^2 + qs^2)}{p\ell +qs}. \label{Vpen}
\end{align}
\end{corollary}

\begin{proof}
For leaf arcs the game values are simply the arc lengths and the favored arc $Q_1$ is the arc of length $\ell$, so we have $V(Q_1)=\ell$ and $V(Q_2)=s$. 
So the formula (\ref{beta}) for the favoring bias $\beta$ becomes (\ref{fpen}). 
Observe that as $p,q \rightarrow 1/2$, we have $\beta \rightarrow 0$ so that the Searcher just follows the equiprobable signal, randomly choosing which branch to search first. 
The optimal probability of hiding in the long (favorite) branch, given by the Weighted Branch Density distribution, becomes (\ref{xf pen}), which in the no-signal $\left(p,q \rightarrow 1/2\right) $ model reduces to $\ell/( \ell+s) $ (hiding in each arc with a probability proportional to its length). 
Finally, the formula (\ref{V}) reduces to (\ref{Vpen}) which simplifies to the total length $\ell+s$ in the no signal case of $p,q \rightarrow 1/2$.
\end{proof}

We end this section by showing that the value of the game is non-increasing in $p$. This is intuitively obvious, since the higher $p$ is, the more reliable the signal is, so one would expect the search time to go down.

\begin{proposition} \label{prop:non-inc}
The value of the game is non-increasing in $p$.
\end{proposition}
\begin{proof}
Suppose $1 \ge p > p' > 1/2$, and let $V$ and $V'$ be the respective values of the games $G(Q,O,p)$ and $G(Q,O,p')$.  We will show that we can generate a signal which is correct with probability $p'$ by using a signal that is correct with probability $p$. In this way, we can use the Searcher strategy for $p'$ in the game $G(Q,O,p)$, thereby ensuring that we can find the Hider in expected time $V'$ in this game.
 
When the Searcher is at a branch node in game $G(Q,O,p)$ and receives a signal, we create a new signal which is equal to the received signal with probability $x$ and chosen uniformly at random with probability $1-x$.  Then the probability this signal is correct is $xp+(1-x)/2$.  Choose $x$ so that $xp+(1-x)/2 = p'$, and we obtain a signal that is correct with probability $p'$.  The precise value of $x$ is $(p'-1/2)/(p-1/2)$.

Thus the Searcher can ensure an expected search time of at most $V'$ in $G(Q,O,p)$ so that $V \le V'$.
\end{proof}

Note that if $Q$ is a single arc with the root at one of its ends, the value of the game is $\mu$ for any value of $p$, so we cannot say that the game is strictly decreasing in $p$ in general.

\section{Application of the Recursion Techniques}
\label{sec:example}

We now show how the recursion techniques of the previous section can be used to solve the game on the tree of Example~\ref{ex1} (with $p=2/3$), obtaining the optimal Searcher strategies of Figure~\ref{fig:network-sol}.
The rooted tree of Figure~\ref{fig:network} has two branch nodes, $A$ and $O$. 
The recursion works backwards from penultimate nodes to the root, so we start with the subtree at $A$: the arcs $A L_{1}$ and $AL_{2}$.
Since $A$ is a penultimate node, the two branches have values equal to their
lengths $\ell=3$ and $s=2$. Equation~(\ref{fpen}) of Corollary~\ref{cor:pen-node} says that the long arc of length $\ell=3$ is favored and the favoring bias is
\[
\beta=\frac{(p-q)(\ell-s)}{2(p\ell+qs)} = \frac{ (2/3-1/3)(3-2)}{2(2/3(3)+1/3(2))}=1/16,
\]
as indicated in Figure 2. 
Thus if the Searcher arrives at branch node $A$, she goes to leaf node $L_{2}$ immediately with probability $1/16$ without looking at her signal. 
Otherwise she follows the signal. 

The optimal Hider distribution, $\bar{\lambda}$ for the subtree $Q_1$ with arcs $AL_1$ and $AL_2$ is to choose $L_2$ with probability proportional to $p \cdot 3 = 2$ and $L_1$ with probability proportional to $q \cdot 2 = 2/3$, by Equation~(\ref{eq:lambda-bar}). That is, $L_2$ is chosen with probability $3/4$ and $L_1$ is chosen with probability $1/4$. Hence, 
\[
D_{Q_1}= 3/4 \cdot 3 + 1/4 \cdot 2 = 11/4.
\]
Now consider the original game starting at node $O$.
The left branch $Q_2$ at $O$ has 
\[
D_{Q_2}=1+D_{Q_1}=15/4.
\]
The right branch $Q_3$, which is simply the arc $OR$ has $D_{Q_3}=4 > D_{Q_2}$, so it is the favored branch, as indicated in Figure~\ref{fig:network-sol}.
The favoring bias $\beta(O)$, given by (\ref{beta}), is 
\[
\beta(O)=\frac{(p-q)(D_{Q_3}-D_{Q_2})}{2(p \mu(Q_3)+q \mu(Q_2))} =  \frac{(2/3-1/3)(4-15/4)}{2(2/3 \cdot 4+1/3 \cdot 6)} = 1/112.
\]
It is optimal for the Hider to choose $Q_3$ and $Q_2$ with probability proportional to $p \cdot 4=8/3$ and $q \cdot 6 = 2$, respectively. That is, $Q_3$ is chosen with probability $4/7$ and $Q_2$ is chosen with probability $3/7$. Putting this together with the optimal Hider strategy on $Q_1$, we get the optimal hiding probabilities
\[
\bar{\lambda}(R)=4/7, \quad \bar{\lambda}(L_1)= 3/7 \cdot 1/4 = 3/28, \quad \bar{\lambda}(L_2)=3/7 \cdot 3/4 = 9/28.
\]
Also, we have
\[
D_Q=4/7 \cdot 4 + 3/28 \cdot 3 + 9/28 \cdot 4= 109/28.
\]
Hence, by~(\ref{V}), the value of the game is
\[
V = 2q \mu + (p-q)D_Q = 2/3 \cdot 10 + (2/3-1/3)\cdot109/28 = 223/28 \approx 7.96.
\]

\section{Trees of Constant Depth}

We say that a tree has {\em constant depth} if there is an $r$ such that for all leaf nodes $v$, we have $d(O,v)=r$. If $Q$ has more than one arc, the root $O$ will be the unique center of $Q$ and $r$ will be the radius. For trees of constant depth $r$, the mean depth $D(Q,O,\lambda)$ is obviously equal to~$r$. The next corollary follows immediately from Equation~(\ref{V}).
\begin{corollary}
If $Q$ is a rooted tree with constant depth $r$, the value $V$ of the game $G(Q,O,p)$ is given by
\[
V= 2q \mu + (p-q)r.
\]
\end{corollary}
So as $p$ varies from the no-signal value $p=1/2$ to the full information value $p=1$, the value of the game varies from $\mu$ to $r$.

To consider the effect of signals, consider a family of {\em perfect} binary trees $B_n$. The tree $B_1$ consists of two arcs meeting at the root $O$. Assuming $B_n$ has been defined, $B_{n+1}$ consists of two  arcs meeting at the root $O$, along with a copy of $B_n$ attached (at its root) to the other end of each of these arcs. The tree $B_3$ is depicted in Figure~\ref{fig:B3}.

\begin{figure}[h]
\center
\includegraphics[width=8cm]{B3}
\caption{The tree $B_3$.}
\label{fig:B3}
\end{figure}


The tree $B_n$ has $2^n$ leaf nodes and $2^{n+1}-2$ arcs. Suppose we take each arc to have length $\ell=1/(2^{n+1}-2)$, so that the total length $\mu$ of $B_n$ is $1$. Without signals, they are equally easy to search, and the value of the game is $\mu=1$. Now fix any $p>1/2$. The tree $B_n$ has constant depth $n\ell=n/(2^{n+1}-2)$, which is decreasing in $n$ and converges to $0$ as $n\rightarrow \infty$. By Theorem~\ref{thm:value}, we have
\begin{align*}
V & = 2q\mu  + (p-q)D \\
&= 2q + (p-q)n\ell \\
&\rightarrow 2q,
\end{align*}
as $n \rightarrow \infty$. So the value is decreasing in $n$ and converges to $2q$. For $p=2/3$, we have $V=2/3+(1/3)n/(2^{n+1}-2)$, converging to $2/3$, as plotted in Figure~\ref{fig:graph} for $n \ge 2$.

\begin{figure}[h]
\center
\includegraphics[width=8cm]{desmos-graph}
\caption{Value of the game on $B_n$ for $p = 2/3$}
\label{fig:graph}
\end{figure}


\section{Conclusion} 

This paper has addressed the question of how to optimally search for an adversarially hidden target on a tree network in the presence of unreliable signals. We have found optimal solutions for both the Searcher and the Hider that can be calculated recursively, and a closed form expression for the value of the game. Future work might consider a variation of the game we consider here in which the time to traverse an arc depends on the direction of travel, as in the variable speed networks studied in \cite{AL14}. 






\begin{comment}

\section{Variable Speed Networks}

We now expand the model to encompass variable speed networks, introduced in \cite{Alpern10}. In these networks, the time taken for the Searcher to traverse an arc depends on the direction traveled. We denote the time to tour a variable speed tree network $Q$ by $\tau(Q)$, which is equal to the sum of the forward and backward travel times of each arc.

The search game with signals on a variable speed tree network is identical to the game analyzed so far in this paper, except that the travel times depend on the direction of travel. 

We repeat the analysis below for variable speed networks.

\begin{definition}[Weighted Branch Density distribution.] We define the {\em Weighted Branch Density Distribution} $x_Q$ on a tree $Q$ with root $O$ inductively. If $Q$ has only one arc, $x_Q$ is concentrated on the unique neighbor of $O$. Suppose now that $x_Q$ has been defined for all trees with at most $n$ arcs for some integer $n \ge 1$. For such trees $Q$ with leaf nodes $\mathcal{L}=\mathcal{L}(Q)$ we define  $\Lambda(Q)$ of $Q$ as 
\[
\Lambda(Q)= \sum_{j \in \mathcal{L}} x_Q(j) \cdot (p d(O,j) + qd(j,O)).
\]

Suppose $Q,O$ is a rooted tree with more than $n$ arcs, and first suppose $O$ has degree 1. Let $O'$ be the neighbor of $O$ and let $Q'$ be the subtree rooted at $O'$. Then we set $x_{Q'}=x_Q$. 

Now suppose $O$ has degree two, and let $Q_1$ and $Q_2$ be the two subtrees rooted at $O$, and let their tour times be $\tau_1$ and $\tau_2$, respectively. Without loss of generality suppose $\Lambda(Q_1) \ge \Lambda(Q_2)$. Then we call $Q_1$ the {\em favored branch} and set
\[
x_Q(Q_1)=\frac{p \tau_1}{p \tau_1 + q\tau_2} x_{Q_1}, \quad x_Q(Q_2)=\frac{q \tau_2}{p \tau_1 + q\tau_2} x_{Q_2}.
\]
\end{definition}

We observe that as $p$ goes to $1$ and $q$ to $0$, the distribution of $x_Q$ becomes concentrated on the leaf node at greatest distance from $O$, and $\Lambda(Q)$ converges to that distance. As $p$ goes to $1/2$, the distribution of $x_Q$ converges to the Equal Branch Density Distribution for variable speed trees and $\Lambda(Q)$ converges to $0$.

Before stating our main theorem, we make an observation.

\begin{proposition} \label{prop:Delta}
Suppose $Q$ is a tree with root $O$ and first suppose that $O$ has degree $1$. Let $O'$ be the neighbor of $O$ and let $Q'$ be the subtree rooted at $O'$. Let $a$ and $b$ denote the forward and backward travel times of the arc $OO'$. Then 
\[
\Lambda(Q)=pa-qb + \Lambda(Q').
\]
Now suppose $Q$ has degree 2 and let $Q_1$ and $Q_2$ be the branches at $O$ with tour times $\tau_1$ and $\tau_2$, respectively and suppose $Q_1$ is the favored branch. Then
\[
\Lambda(Q) = \frac{p \tau_1 \Lambda(Q_1) + q\tau_2 \Lambda(Q_2)}{p\tau_1+q\tau_2}.
\]
\end{proposition}

We can now state our main theorem, which includes an expression for the value of the game.
\begin{theorem}
Let $(Q,O)$ be a rooted tree with tour time $\tau = \tau(Q)$. The value $V=V(Q)$ of the search game with signals played on $(Q,O)$ is
\begin{equation}
V(Q)=q  \tau + \Lambda(Q).
\label{V}
\end{equation}
The Weighted Branch Density distribution $x_Q$ is optimal for the Hider. We describe the optimal strategy for the Searcher by giving the favoring bias $\beta$ of searching the favored branch first (without needing to observe the signal) when at a branch node $j$ with branches $Q_1$ and $Q_2$ with tour times $\tau_1$ and $\tau_2$. Without loss of generality, suppose $Q_1$ is the favored branch. Then
\begin{equation}
\beta=\frac{\Lambda(Q_1)-\Lambda(Q_2)}{p \tau_1 + q \tau_2}.
\label{beta}
\end{equation}
With the complementary probability $1-\beta$ the Searcher follows the signal.
\end{theorem}

Note that as $p,q \rightarrow 1/2$, so that the signal is useless, the value converges to $\tau/2+\Lambda$, and $\Lambda$ converges to
\[
\frac{1}{2} \sum_{j \in \mathcal{L}} x_Q(j)(d(O,j)-d(j,O)),
\]
which, in the language of \cite{AL14}, is half the {\em incline} $\Delta$ of the tree $Q$. So the value converges to $(\tau+\Delta)/2$, as found in \cite{AL14}.

\end{comment}

% Appendix here
% Options are (1) APPENDIX (with or without general title) or
%             (2) APPENDICES (if it has more than one unrelated sections)
% Outcomment the appropriate case if necessary
%
% \begin{APPENDIX}{<Title of the Appendix>}
% \end{APPENDIX}
%
%   or
%
% \begin{APPENDICES}
% \section{<Title of Section A>}
% \section{<Title of Section B>}
% etc
% \end{APPENDICES}

%%
%\theendnotes

% Acknowledgments here



% References here (outcomment the appropriate case)

% CASE 1: BiBTeX used to constantly update the references
%   (while the paper is being written).
%\bibliographystyle{informs2014} % outcomment this and next line in Case 1
%\bibliography{<your bib file(s)>} % if more than one, comma separated

% CASE 2: BiBTeX used to generate mypaper.bbl (to be further fine tuned)
%\documentclass[review]{elsarticle}

\usepackage{hyperref}
\usepackage{amsmath,amssymb,amsfonts}
\usepackage{algorithmic}
\usepackage{ntheorem}
\usepackage{caption}
\usepackage{graphicx}
\usepackage{textcomp}
\usepackage{indentfirst}
\usepackage{float}
\usepackage{bm}
\usepackage{tagging}
\usepackage{amsfonts,amssymb}
\usepackage{amsmath}
\usepackage{tablists}
\usepackage{subfigure}
\usepackage{ragged2e} 
\usepackage{booktabs,makecell, multirow, tabularx}
\usepackage{stfloats}
\usepackage{color}
\usepackage{appendix}
\DeclareMathAlphabet\mathbfcal{OMS}{cmsy}{b}{n}

\journal{Journal of \LaTeX\ Templates}

%%%%%%%%%%%%%%%%%%%%%%%
%% Elsevier bibliography styles
%%%%%%%%%%%%%%%%%%%%%%%
%% To change the style, put a % in front of the second line of the current style and
%% remove the % from the second line of the style you would like to use.
%%%%%%%%%%%%%%%%%%%%%%%

%% Numbered
%\bibliographystyle{model1-num-names}

%% Numbered without titles
%\bibliographystyle{model1a-num-names}

%% Harvard
%\bibliographystyle{model2-names.bst}\biboptions{authoryear}

%% Vancouver numbered
%\usepackage{numcompress}\bibliographystyle{model3-num-names}

%% Vancouver name/year
%\usepackage{numcompress}\bibliographystyle{model4-names}\biboptions{authoryear}

%% APA style
%\bibliographystyle{model5-names}\biboptions{authoryear}

%% AMA style
%\usepackage{numcompress}\bibliographystyle{model6-num-names}

%% `Elsevier LaTeX' style
\bibliographystyle{elsarticle-num}
%%%%%%%%%%%%%%%%%%%%%%%

\begin{document}
\captionsetup[figure]{labelfont={bf},labelformat={default},labelsep=period,name={Fig.}}

\begin{frontmatter}

\title{The Graph feature fusion technique for speaker recognition based on wav2vec2.0 framework \tnoteref{mytitlenote} }
\tnotetext[mytitlenote]{This work has been supported by National Natural Science Foundations of China (No.62071242)}

%% Group authors per affiliation:
\author{Zirui~Ge, Haiyan~Guo, Tingting~Wang, Zhen~Yang\corref{mycorrespondingauthor} }
\address{School of Communication and Information Engineering, Nanjing University of Posts and Telecommunications, Nanjing 2100023, China}
\fntext[myfootnote]{E-mail address: yangz@njupt.edu.cn (Z. Yang)}
%%1019010430@njupt.edu.cn (Z. Ge), guohy@njupt.edu.cn (H.Guo), 2018010215@njupt.edu.cn (T. Wang), 

%% or include affiliations in footnotes:
%\author[mymainaddress,mysecondaryaddress]{Elsevier Inc}
%\ead[url]{www.elsevier.com}
%%ead{ }
%\author[mysecondaryaddress]{Global Customer Service\corref{mycorrespondingauthor}}
\cortext[mycorrespondingauthor]{Corresponding author}
%\ead{ This work has been supported by National Natural Science Foundations of China (No.61671252, No.61271335, No.61901229), the Natural Science Research of}

%\address[mymainaddress]{1600 John F Kennedy Boulevard, Philadelphia}
%\address[mysecondaryaddress]{360 Park Avenue South, New York}

\begin{abstract}
Pre-trained wav2vec2.0 model has been proved its effectiveness for speaker recognition. However, current feature processing methods are focusing on classical pooling on the output features of the pre-trained wav2vec2.0 model, such as mean pooling, max pooling etc. That methods take the features as the independent and irrelevant units, ignoring the inter-relationship among all the features, and do not take the features as an overall representation of a speaker. Gated Recurrent Unit (GRU), as a feature fusion method, can also be considered as a complicated pooling technique, mainly focuses on the temporal information, which may show poor performance in some situations that the main information is not on the temporal dimension. In this paper, we investigate the graph neural network (GNN) as a backend processing module based on wav2vec2.0 framework to provide a solution for the mentioned matters. The GNN takes all the output features as the graph signal data and extracts the related graph structure information of features for speaker recognition. Specifically, we first give a simple proof that the GNN feature fusion method can outperform than the mean, max, random pooling methods and so on theoretically. Then, we model the output features of wav2vec2.0 as the vertices of a graph, and construct the graph adjacency matrix by graph attention network (GAT). Finally, we follow the message passing neural network (MPNN) to design our message function, vertex update function and readout function to transform the speaker features into the graph features. The experiments show our performance can provide a relative improvement compared to the baseline methods. Code is available at xxx.
\end{abstract}

\begin{keyword}
speaker recognition, wav2vec2.0, graph neural network, pooling, feature fusion
\end{keyword}
\end{frontmatter}


\section{Introduction}
\par Automatic speaker recognition (ASR) is the task of authenticating the claimed identity using the speaker’s voiceprint. As a means of using bio-metrics, ASR has attracted considerable attention from many researchers due to its accessibility and uniqueness. With the development of deep neural network, speaker recognition models based on deep neural networks are being more complicated, and need more larger quantities of labeled training data.
\par However, producing large and high quality labeled data is hard and expensive, and only learning from the labeled samples also seems to be inconsistent with the process of language acquisition of the infants, i.e., self-learning from listening and watching, supervise learning from training and testing with instructors. Self-supervision learning on unlabeled data and fine-tuning on the pre-trained models is similar to the mentioned two stages and have been proved successful for natural language processing such as BERT \cite{Ref1}, GPT-3 \cite{Ref2} \emph{etc}.
\par In the field of speech signal processing, wav2vec2.0 \cite{Ref3} also applies to the two stages learning process. Wav2vec2.0 shows an excellent performance on speech recognition, and it first learns the speech representation from the unlabeled speech audio dataset and fine-tune the pre-trained weights on the labeled data. There are mainly four modules in wav2vec2.0 framework, i.e., a multi-layer convolution feature encoder, a Transformer group module, the quantization module and contrastive loss. More specifically, wav2vec2.0 first encodes the raw audio signal into latent speech representations via the multi-layer convolution feature encoder. Then, the masked latent speech representations are fed into the Transformer module group to capture the contextualized representations from the entire sequence. Meanwhile, the quantization module converts the unmasked latent speech representation into its discrete version via product quantization. Finally, the discrete representation of quantization module and the output of the Transformer group are put into the contrastive loss to identify the true quantized latent speech representation. Wav2vec2.0 framework has achieved 1.8/3.3 word error rate (WER) on the clean/other test sets using all labeled data of Librispeech dataset. When using more less labeled data, wav2vec2.0 still outperforms the state of the art at that time. The excellent performance shows that the phonemic constructions are well learned during the pre-training and the downstream modules can finish their tasks via fine-tuning the pre-training weights. 
\par Pre-trained wav2vec2.0 as the upstream model also works well in speaker recognition. The authors in \cite{Ref4} first applied the wav2vec2.0 to multi-task learning, i.e., speaker recognition and language identification task, and investigated wav2vec2.0 as the audio encoder to extract the speaker and language features \cite{Ref4}. The work of multi-task learning in \cite{Ref4} first demonstrated the effectiveness of wav2vec2.0 on the speaker recognition and language identification task. At the same time, the authors in \cite{Ref5} took pre-trained wav2vec2.0 models to implement the speech emotion recognition task, and propose to weight the output of several layers from the pre-trained model using trainable weights which are learned jointly with the downstream model. The authors in \cite{Ref6} applied wav2vec2.0 framework to speaker recognition task, and investigated effectiveness of different pooling methods. \cite{Ref6} further proposed the first\&cls pooling method that inserts a "start token" (all values are +1) in the input sequence of encoder, and selected the first output token as the speaker embedding. These literatures completed different speech related tasks based on wav2vec2.0 framework via using mean, max, mean\&std pooling methods etc. Though these classical methods have been proven their effectiveness, they only focus on some simple information of features, for example, the mean and mean\&max pooling mainly consider the distribution of features, and max pooling mainly focuses on the "texture information". Besides, these methods consider the output features as many independent elements in the regular (Euclidean) space. They do not view these output features as an entirety feature of a speaker identity and do not consider more complicate relationship among these features. Though GRU as a complicate feature fusion method can focus the temporal information, it may be uncompetitive when meet the features lacking the temporal information. For example, these features have been processed by some other models that can also extract the temporal information, such as the Transformer module \cite{Ref7}. Therefore, when features require to be viewed as an entirety or not merely extracted the temporal information, a new data structure or signal processing technique require to be introduced.
\par Some literatures \cite{Ref8,Ref9,Ref10} have shown that the speech signals can be reformulated as a graph signal and processed using graph signal processing theory (GSP) \cite{Ref25,Ref26} in the irregular space, i.e., graph domain, and a better performance can be obtained compared to regular space. In above literatures, speech features are considered as an entirety, i.e., a graph signal, not merely a set of different independent features. Graph neural networks (GNN) as a nonlinear form of GSP that corporate the advantages of the graph structure and deep neural networks \cite{Ref11,Ref12} have shown the excellent performance on image classification \cite{Ref13,Ref14,Ref15} which is mainly processed in the regular (Euclidean) space previously. Thus, exploring the combination of GNN and speaker recognition is well-founded.
\par The authors in \cite{Ref16} first proposed the concept of GNN and extended the neural networks for processing the data resided in graph domains. With the development of GNN, there are some major variants of GNN in the world including the message passing neural network (MPNN) \cite{Ref17}, graph attention network (GAT) etc. MPNN contains two phases, a message passing phase and a readout phase. The message passing phase mainly include hidden states updating and graph vertex aggregating, and the readout phase is aim to obtain the whole graph representation using the readout function. GAT incorporates the self-attention mechanism \cite{Ref7} into the propagation step and obtains new node features via weighting neighborhood node features using attention coefficients.
\par \cite{Ref19,Ref20} first applied GNN as the backend feature fusion method of ResNet and RawNet2 model to speaker recognition task. \cite{Ref20} uses the GAT mechanism to design an undirected graph with asymmetric weight matrix to depict the relationship between different graph vertex pairs, and take the graph U-Net architecture to obtain the final graph representation. The GNN related framework in [19] takes a pair of utterances as the enrollment and the test utterance to train their similarity score. \cite{Ref21} also took the same GNN architecture in \cite{Ref19} to implement speaker anti-spoofing task. The both GNN backend models obtain the improved performance. However, that mentioned utterance-pair classification format is inferior to the single-utterance classification for speaker recognition based on wav2vec2.0 framework \cite{Ref6}. Besides, both works do not talk about the relationship between the GNN fusion method and classical fusion method such as mean, max, random, etc., and why the GNN based feature fusion models can outperform the compared GRU model.
\par Under this background, we first discuss the relationship between GNN or GSP and classical pooling methods, such as mean, max, random, etc., and show that why GNN based feature fusion models can outperform the mean, max, random pooling methods theoretically. Then we propose GNN as the downstream processing framework to explore the output features of wav2vec2.0 framework in non-Euclidean space. Specifically, we model the set of output features as a graph signal, i.e., each feature is considered as a graph vertex’s feature, and the number of graph vertices is equal to the number of output features. Then we use the graph attention mechanism to construct the symmetric weight matrix to capture the similarity between different vertex pairs. The attention weight matrix is applied to our designed message passing graph neural network to obtain the graph embedding which is also the final speaker embedding. When the downstream processing framework is constructed, we take the wav2vec2.0 as the upstream audio feature extractor and fine-tune its pre-training weights. We implement our experiments on the VoxCeleb datasets \cite{Ref20, Ref21}, and the experiments show that proposed graph pooling method can obtain better performance than the classical feature pooling methods. We also explain why GNN can obtain the top performance compared to other feature fusion methods. To the best of our knowledge, this study is the first to apply a GNN to wav2vec2.0 framework for the speaker recognition task.

\section{Related Work}

\par In this section, we first review the pre-training of the wav2vec 2.0 and how to apply the pre-training to downstream. Then, we introduce graph message passing neural network.
\subsection{Wav2vec 2.0 pretraining and fine-tuning}
\par The main body of the model consists a CNN based feature encoder, a Transformer-based context network, a quantization module and a contrastive loss. The feature encoder consists of 7 blocks and each block contains a temporal convolution with 512 channels with respective kernel sizes of (10, 3, 3, 3, 3, 2, 2) and stride (5, 2, 2, 2, 2, 2, 2) followed by a layer normalization and a GELU activation function \cite{Ref21}. As the Figure.1 left depicted, the CNN feature extractor takes as input raw audio $X$ and and outputs latent speech representations $Z$. Then, the latent speech representations are projected into a new dimension, before fed to the following modules.
\par The context network contains 12 Transformer blocks and a residual 2-layer feed forward network with 3072 and 768 units. The relative positional embeddings instead of fixed positional embeddings are first added to the masked speech representations, before the masked representations are input to the context network. Transformer then contextualizes the masked representations and finally generates context representations $C$. The outputs of 12 Transformer blocks and projected speech representation are considered as hidden features.
\par The quantization module is to discretize the output of the feature encoder to a finite set of speech representations via product quantization, where the product quantization is aim to choosing quantized representations from multiple codebooks and concatenating them. The number of codebooks is equal to 2 and there are 320 elements and respective size is 128 in each codebook. The gumbel softmax function \cite{Ref24} is also used to enable choosing discrete codebook entries in a fully differentiable way.
\par The objective function is the weighted sum of the contrastive loss and diversity loss. The contrastive loss requires to identify the true quantized latent speech representation for a masked time step within a set of distractors. The diversity loss is designed to encourage using the codebook entries equally. 
\par Fig.1 right shows the fine-tuning stage, the models are identical to the training stage, except the quantization modules and extra output layers. We take a speech audio with 48000 samples as the input, and show the output shape of different modules. In the fine-tuning stage, how to select the hidden feature has an important effect on downstream tasks \cite{Ref5}, therefore we consider two approaches to select hidden feature, one way is taking the output of last Transformer block, the other is weighting all the hidden features.

\begin{figure}[t]
	{	\centering
		\includegraphics[width=5.2in]{finetuning2.eps}\\}
	\caption{An overview of the pre-training and fine-tuning. The right subfigure shows the output shape of different model for a given input.}\label{fig1}
\end{figure}

\subsection{Message Passing Neural Networks }
\par Let $G=(V,E,A)$ be a graph and $X\in \mathbb{R} ^{F\times N}$ be input vertex features, where $V=\left\{ v_1,...,v_N \right\} $ is the set of $N=|V|$ vertices, $F$ is the dimension of one input feature. $\mathcal{E} =\left\{ e_{i,j} \right\} _{i,j\in \mathcal{V}}$ is the set of edges between vertices, such that $e_{i,j}=1$ if there is a link from node $j$ to node $i$, otherwise $e_{i,j}=0$. $A$ is the $N\times N$ weight adjacency matrix and its entries are the edge weights $a_{i,j}$, for  $i, j=1,...,N$.
\par The MPNN contains two phases, a message passing phase and a readout phase. The message phase runs for $T$ time steps and contains two functions that are message function $M_t$ and vertex update function $U_t$. The hidden state $h_t$ of each vertex at the $t$ time step can be written as:
$$
m_{v}^{t+1}=\sum_{w\in N(v)}{M_t}\left( h_{v}^{t},h_{w}^{t},e_{vw} \right),\eqno(1) \label{eq1}
$$
$$
h_{v}^{t+1}=U_t\left( h_{v}^{t},m_{v}^{t+1} \right) ,\eqno(2) \label{eq2}
$$
where $N(v)$ denotes the neighbors of vertex $v$ in graph $G$. The readout phase computes the final graph feature vector via the following readout function $R$
$$
\hat{y}=R\left( \left\{ h_{v}^{T}\mid v\in G \right\} \right) ,\eqno(3) \label{eq3}
$$
The message function $M_t$, vertex update function $U_t$ and readout function $R$ are all learned differential functions. 

\section{METHODOLOGY}
\label{}
\par In this section, we first show that some classical pooling methods can be represented by GNN or GSP, and introduce the graph neural network as the feature pooling method. The proposed module is located after the context network and takes as input the hidden states of wav2vec2.0. There are three components in our proposed model: 1) the graph attention layer; 2) the message and vertex update function; and 3) the readout function. Specifically, we first show how to obtain the graph weight adjacency matrix from the graph attention layer. Then we show how to aggregate neighbor vertices information in term of one vertex, and update the status of every vertex. Finally, we propose a readout function to obtain the whole graph feature embedding. The overall scheme is illustrated in Figure 2.
\subsection{Reformulating some classical pooling methods with GNN}
\par We reformulate some pooling methods using GSP and GNN. Specifically, we use the GSP to reformulate the linear pooling, including mean, random, first, middle, last pooling method, and use the GNN to reformulate the nonlinear pooling, i.e., the max pooling method.
\par In the graph signal processing theory, the weight adjacency matrix $A$ is also is also named as the shift operator, and the notion of graph shift operator is defined as a local operation that replaces a graph signal feature with the linear combination of features at the neighbors of that vertex \cite{Ref27}. The graph shift operation can be expressed as 
$$
Y=AX^T, \eqno(4) \label{eq4}
$$
where $X\in \mathbb{R} ^{D\times N}$.
\par We take the above graph shift operator to reformulate the mean and random pooling methods. When 
$$
A=\left[\begin{array}{ccc}
1 / N & \cdots & 1 / N \\
\vdots & \ddots & \vdots \\
1 / N & \cdots & 1 / N
\end{array}\right], \eqno(5) \label{eq5}
$$

the expression $A^T$ can be considered as the mean pooling method, and for 

$$
A=\left[\begin{array}{ccc}
0 & 1 & 0 \\
\vdots &\vdots & \vdots \\
0 & 1 & 0
\end{array}\right], \eqno(6) \label{eq6}
$$
the expression $A^T$ can be considered as the random pooling method. When the random pooling method selects the $i$th feature, the entries of the $i$th column in $A$ in are all 1. When the $i$ in random pooling is always 1, $N$, or $\lfloor {{N} /{2}} \rfloor $, the pooling method can be considered as the first, last, and middle pooling.
\par The max pooling method as a nonlinear operation is impossible to be reformulated by the linear operation. We take the MLP as an approximation for the max pooling method, thanks to the universal approximation theorem \cite{Ref28}, and it can be expressed as 
$$
Y=MLP\left( AX^T \right)  , \eqno(7) \label{eq7}
$$
\par Therefore, some classical pooling methods can be reformulated with GSP or GNN, and we may obtain a better performance theoretically, if we directly use GSP or GNN as our pooling methods.
\subsection{Graph attention layer}
\par We first formulate a graph using the output features of wav2vec2.0 framework. Specially, each output feature is considered as a vertex of a graph. Due to these feature does not have an obvious graph structure, we need to specify the edge set and weight adjacency matrix. For this model, we argue that every vertex pair has an edge to link each other, such that the entries $e_{i,j}$ in edge set $\mathcal{E} $ are all 1. Hence set $\mathcal{E} $ means that the entire output features are interacted each other. Next, we show how to obtain the weight adjacency matrix from the graph attention mechanism. Specially, we consider the weights in the adjacency matrix as the degree of similarity between pairs of vertices. There are two main approaches to graph attention mechanism. One approach leverages the explicit attention mechanism to obtain the attention weights such as the cosine similarity between different vertex pairs \cite{Ref29}. The other approach does not rely on any prior information and leverages complete parameter learning to gain attention weight \cite{Ref30}. In this study, we take the first approach as our graph attention layer. Compared to \cite{Ref30} used in \cite{Ref20}, the symmetric weight adjacency matrix learned in \cite{Ref29} can save half the computation in the graph attention layer. The GAT process can be described as:
$$
b_i=Wx_i, \eqno(8) \label{eq8}
$$
$$
a\left( i,j \right) =\frac{\exp \left( \beta \cos \left( b_i,b_j \right) \right)}{\sum\nolimits_{k=1}^N{\exp \left( \beta \cos \left( b_i,b_j \right) \right)}}, \eqno(9) \label{eq9}
$$
where $x_i\in \mathbb{R} ^F$ is the vertex feature, and $W\in \mathbb{R} ^{F'\times F}$ is the projection matrix, $\beta $ is the learnable parameter. In the GAT layer, the vertex feature is first projected into $F'$ dimensional space via multiplying $W$. Then the attention score is obtained by equation (9). 
\subsection{Message and vertex update function}
\par In this study, we formulate the message function as 
$$
M_t=AH_{t-1}^{\top}, \eqno(10) \label{eq10}
$$
where $H_t\in \mathbb{R} ^{F'\times N}$ is the set of the hidden states $h$, $M_t$ is the aggregated states set, and the subscript $t$ is the $t$th massage aggregation, $\top $ is the transpose symbol. (10) can also be considered as the one order graph shift. When $t=1$, $H_{t-1}$ is the projected graph vertex features set $B=\left\{ b_i \right\} ,i=1,...,N$. 
\par The vertex update function is defined as the nonlinear transformation of aggregated representation:
$$
H_t=\sigma \left( LN\left( MLP_t\left( M_t \right) \right) \right) , \eqno(11) \label{eq11}
$$
where $\sigma \left( \cdot \right) $ is the non-linear activation, $LN\left( \cdot \right) $ is the layer normalization function, $MLP\left( \cdot \right) $ is the multi-layer perceptron (MLP) with a set of learnable parameters.
\subsection{Readout function }
When all the hidden states are sufficiently updated, they are aggregated to a graph-level representation for the speaker voice print feature, based on which the final prediction is produced. We define the readout function as:
$$
H^T=MLP_{\theta}\left( H^{T-1} \right) \odot sigmoid\left( MLP_{\varphi}\left( H^{T-1} \right) \right) , \eqno(12) \label{eq12}
$$
$$
h_{\mathcal{G}}=\frac{1}{|\mathcal{V} |}\sum_{i=0}^T{\sum_{v\in \mathcal{V}}{h_{v}^{T}}}+\,\,\mathrm{Maxpooling} \left( H^T \right) , \eqno(13) \label{eq13}
$$
where $\odot $ is the element-wise multiplication. In the equation (13), $\frac{1}{|\mathcal{V} |}\sum_{i=0}^{T-1}{\sum_{v\in \mathcal{V}}{h_{v}^{T}}}$ denotes the residual connection in GNNs. In the residual connection, we extract the distribution information of early representation as the supplement for the final representation. The MPNN framework is shown in Figure 2.

\begin{figure}[t]
	{	\centering
		\includegraphics[width=5.2in]{graph-aggregation2.eps}\\}
	\caption{The MPNN framework.}\label{fig2}
\end{figure}




\section{EXPERIMENTS}

\begin{figure}[t]
	{	\centering
		\includegraphics[width=5.2in]{feature_weights.eps}\\}
	\caption{The weights of different hidden features.}\label{fig2}
\end{figure}
\begin{figure}[t]
	{	\centering
		\includegraphics[width=4.3in]{weight2.eps}\\}
	\caption{The weight matrices of different speakers.}\label{fig3}
\end{figure}
\subsection{Dataset}
All experiments are conducted on the VoxCeleb1\&2 datasets \cite{Ref22, Ref23}. VoxCeleb2 development set contains over a million utterances from 5994 celebrates from the YouTobe, and the average duration of a signal speaker is about 7.2 seconds. The reported performance in terms of equal error rates (EERs) is evaluated on extended (vox1-o, vox1-e, vox1-h) test sets from the VoxCeleb12. The pretrained weights1 used in the experiments with the wav2vec2.0 framework are released on Hugging-Face \cite{Ref31}.
\subsection{Model description and implementation details}
\par We conduct all experiments using the PyTorch framework, on a 3090 GPU, and take the cosine similarity as the back-end performance evaluation tool.
\par \textbf{Baselines.} In the experiments, we take some released works \cite{Ref4,Ref6,Ref32} for speaker recognition based on wav2vec2.0 framework and feature fusion methods as our baselines. The authors in \cite{Ref6} considered different pooling methods, and they are mean, max, mean\&std, quantile, first\&cls, middle, last, first, random. In this study, we use new symbols to represent them, i.e., w2v2-mean, w2v2-max, w2v2-mean\&std, w2v2-quantile, w2v2-first\&cls, w2v2-middle, w2v2-last, w2v2-first, w2v2-random. The mean pooling is also used in \cite{Ref4} for speaker recognition task, which is restamp as w2v2-mean. Graph U-Net and GRU are also considered as the baseline which are used as a pooling method for speaker recognition in \cite{Ref20,Ref32}. 
\par \textbf{Proposed method.} In the graph attention layer, we set $F'=F$, do not change the dimension of input features, and just project the input features into another space. In the vertex update and readout function, we set $T=2$, the number of hidden layer of ${MLP_t}$, $t=1,2$, $MLP_{\theta}$ and $MLP_{\varphi}$ is 1, and the dimension of hidden layer is 1024 for all the MLPs. 
\par \textbf{Implementation details.} In each experiment, we set batch size is 48, and every sample’s duration is 3 seconds sampling from the audio files. In Table II, we take the pretraining weights of w2v2-mean as the initial weights of wav2vec2.0 for the other models to accelerate the convergence process of the models. We train 30 epochs for w2v2-mean, and train 5 epochs for other models. The optimizer is Adam \cite{Ref33} with a OneCycle learning rate schedule [34], and the loss function is angular additive softmax (AAM) loss function \cite{Ref35,Ref36}. We also propose a thin version of our model, i.e., the   is removed in this version.
\par Similar to the [5], we also weight all the hidden features as:
$$
x=\frac{\sum_{i=1}^{13}{w_i}x_i}{\sum_{i=1}^{13}{w_i}}, \eqno(14) \label{eq14}
$$
where the trainable weights $w_i$ are initialized with 1.0, and the $x_i$ is the output feature of different hidden modules, where $x_1$ s the projected speaker representation, and $x_2$, $\cdots$, $x_{13}$ is the 12 Transformer modules’ output features successively.


\subsection{Results}

\par Table I shows the paraments of different module in each model. From the Table I. we can obtain that our method can reduce the paraments compared to the GRU, but still possesses more paraments than graph U-Net pooling method.
% Please add the following required packages to your document preamble:
% \usepackage{multirow}
\begin{table}[]
\caption{The number of parameter in different pooling methods.}
\begin{tabular}{llll}
            & wav2vec2 & pooling & loss\_fn \\
our         & 94.4 M   & 6.9 M   & 4.6 M    \\
our/thin    & 94.4 M   & 5.3 M   & 4.6 M    \\
Graph U-Net & 94.4 M   & 3.7 M   & 4.6 M    \\
GRU         & 94.4 M   & 7.9 M   & 6.1 M   
\end{tabular}
\end{table}
\begin{table}[]
\caption{The performance of different pooling methods.}
\begin{tabular}{lllll}

                             &                & voxceleb1 & voxceleb-e & voxceleb-h \\
\multirow{2}{*}{our}         & all\_features  & 1.79      & 1.75       & 3.2        \\
                             & final\_feature & 2.29      & 2.33       & 3.9        \\
\multirow{2}{*}{our/thin}    & all\_features  & 1.83      & 1.88       & 3.22       \\
                             & final\_feature & 1.96      & 1.84       & 3.36       \\
\multirow{2}{*}{Graph U-Net} & all\_features  & 2.07      & 2.2        & 3.99       \\
                             & final\_feature & 1.9       & 1.84       & 3.36       \\
\multirow{2}{*}{GRU}         & all\_features  & 2.1       & 2.07       & 3.92       \\
                             & final\_feature & 2.2       & 2.17       & 4.05       \\
\multirow{2}{*}{max}         & all\_features  & 2.09      & 2.03       & 3.65       \\
                             & final\_feature & 2.04      & 2.03       & 3.65       \\
\multirow{2}{*}{mean}        & all\_features  & 2.14      & 1.9        & 3.89       \\
                             & final\_feature & 1.89      & 1.85       & 3.42      

\end{tabular}
\end{table}

 
\par Table II shows the performance for different pooling methods based on the wav2vec2.0 framework so that the benefits of proposed methods are assessed. From the Table II, we can obtain that our method can provide a better performance compared to other fusion methods. Our thin model provides a comparable performance and reduce about 23\% parameters compared to the original version. 
\par We also notice that if the only final feature is used as the output feature, the performance of our models and GRU will be degraded, especially in our original model. However, these phenomena are not obvious in other models, and some models even show the opposite results, such as mean, max, graph U-net model. These models all has less parameters than our models and GRU. The number of parameters can determine the upper limit of the expressive power of a model. We believe that the additional information of weighted features beyond the capacity of these models, the extra information may confuse the backend classifier and degrade the performance.
\par Figure 2 shows different weights of each hidden feature, we notice that the final output feature possesses the largest weight compared to other output features in all the models, which is different from the \cite{Ref5}. In the \cite{Ref5}, the larger weights are clustered in the middle features, which means that the emotion features are different from the identity features of a speaker and that is consistent with common sense.
\par It’s worth noting that the GRU based backend classifier does not show the competitive performance compared to other methods. That’s an interesting phenomenon. GRU fusion technique has shown an excellent performance in many models, including speech recognition \cite{Ref37,Ref38,Ref39}, speech emotion recognition \cite{Ref40,Ref41}, speech enhancement \cite{Ref42,Ref43}, speaker recognition \cite{Ref32,Ref44}, etc. We notice that most of these models put the GRU modules right behind convolution modules. That’s reasonable and effective, the convolution modules extract latent speech representation and GRU modules further fused the information in temporal dimension. However, in the wav2vec2.0 framework, the extracted latent representations from the convolution modules have been fed to the Transformer modules, which can also focus the temporal information and has been proved to be excellent at this \cite{Ref7}. Therefore, it is difficult for the GRU modules to extract extra temporal information from the output features of Transformer modules. That is the main reason for GRU possesses the largest number of parameters, but do not show the competitive performance.
\par Mean and max fusion methods mainly focus on the statistical information of output features and the representative features, respectively. These two methods both provide extra information in non-temporal dimension. That is the reason why these two methods can outperform than GRU.
\par GNN uses graph message passing and vertex updating mechanism to obtain the graph structure information in the non-Euclidean space. In Figure 4, each subfigure shows the graph adjacency matrix in different utterances. From the Figure 4, we can obtain that each center feature has high weights with its adjacent features, which means that our GNN can extract the temporal information. Besides, it’s worth noting that the high similarity has no obvious distance property, i.e., the high weights also appear in some features that are far away from the center features. That means that GNN can not only fuse its adjacency features, but also some important features even in a long distance.  Compared to mean, middle, first and random fusion method that allocate equal weight to each feature and the maximum weight to a random feature, our method can treat different features with different weights. In subsection 3.1, we have proved that the mean, max and random fusion method are the special case of GNN and Figure 3 further concretes this experimentally. 

\section{CONCLUSION}
\par In this paper, we mainly take the wav2vec2.0 framework as the speaker feature extractor to apply to speaker recognition task and then investigate the graph neural network as the backend processing tool to aggregate the speaker features. Specifically, we first show that our motivation is reasonable by proving some classical pooling methods can be expressed in the GSP form or GNN form. Then, we obtain the graph structure by GAT and we follow the MPNN framework to design our GNN including message function, vertex update function and readout function. Finally, we evaluate our model on Voxceleb1 dataset, the experiments show the GNN can obtain better performance than the classical pooling method, GRU feature fusion method and the other GNN applied to the speaker recognition task. In experiment results, we show why our proposed method can obtain the best performance compared to other methods.


\bibliography{mybibfile}

\end{document}
 % outcomment this line in Case 2

%If you don't use BiBTex, you can manually itemize references as shown below.

\begin{thebibliography}{}

\bibitem[Alpern(2010)]{Alpern10} Alpern S (2010) Search games on trees with asymmetric travel times. {\em SIAM J. Control Optim.} 48(8):5547--5563.

\bibitem[Alpern(2011)]{Alpern11} Alpern S (2011) Find-and-fetch search on a tree. {\em Oper. Res.} 59(5):1258--1268.

\bibitem[Alpern(2017)]{Alpern17} Alpern S (2017) Hide-and-seek games on a network, using combinatorial search paths. {\em Oper. Res.} 65(5):1207--1214.

\bibitem[Alpern and Gal(2003)]{AG03} Alpern S, Gal S (2003) {\em The Theory of Search Games and Rendezvous} Kluwer International Series in Operations Research and Management Sciences (Kluwer, Boston), 319.

\bibitem[Alpern and Lidbetter(2013)]{AL13} Alpern S, Lidbetter T (2013) Mining coal or finding terrorists: The expanding search paradigm. {\em Oper. Res.}61(2):265--279.

\bibitem[Alpern and Lidbetter(2014)]{AL14} Alpern S, Lidbetter T (2014) Searching a variable speed network. {\em Mathematics of Operations research} 39(3):697--711.

\bibitem[Anderson and Aramendia(1990)]{AA} Anderson EJ, Aramendia MA (1990) The search game on a network with immobile hider, {\em Networks} 20(7):817--844.

\bibitem[Angelopoulos(2020)]{Angel20} Angelopoulos S (2020) Online search with a hint. arXiv preprint arXiv:2008.13729.

\bibitem[Baston and Kikuta(2015)]{BK15} Baston V, Kikuta K (2015) Search games on a network with travelling and search costs. {\em Internat. J. Game Theory} 44(2):347--365.

\bibitem[Catania {\em et al.}(2008)]{Catania08} Catania KC, Hare JF, Campbell KL (2008) Water shrews detect movement, shape, and smell to find prey underwater. {\em P. Natl. Acad. Sci} 105(2):571--576.

\bibitem[Dagan and Gal(2008)]{DG08} Dagan A, Gal S (2008) Network search games, with arbitrary searcher starting point. {\em Networks} 52(3):156--161.

\bibitem[Evans(2022)]{Evans22} Evans R (2022) A brief history of mine detection dogs. {\em The Journal of Conventional Weapons Destruction} 26(1):5.

\bibitem[Gal(1979)]{Gal79} Gal S (1979) Search games with mobile and immobile Hider. {\em SIAM J. Control Optim.} 17(1):99--122.

\bibitem[Gal(2001)]{Gal2001} Gal S (2001) On the optimality of a simple strategy for searching graphs. {\em Internat. J. Game Theory} 29:533--542.

\bibitem[Garcia-Fernandez {\em et al.}(2019)]{GF19} Garcia-Fernandez M, Morgenthaler A, Alvarez-Lopez Y, Las Heras F, Rappaport C (2019) Bistatic landmine and IED detection combining vehicle and drone mounted GPR sensors. {\em Remote Sensing} 11(19):	2299.

\bibitem[Garnaev(2000)]{Garnaev} Garnaev A (2000) {\em Search Games and Other Applications of Game Theory}, Lecture Notes in Economics and Mathematical Systems, Vol. 485 (Springer-Verlag, Berlin).

\bibitem[Hein and McKinley(2013)]{HM13} Hein AM, McKinley SA (2013). Sensory information and encounter rates of interacting species. {\em PLoS Computational Biology}, 9(8):e1003178.

\bibitem[Hochbaum(2014)]{Hochbaum14} Hochbaum DS, Lyu C, Ord\'{o}\~{n}ez F (2014) Security routing games with multivehicle Chinese postman problem. {\em Networks} 64(3):181--191.

\bibitem[Hohzaki(2016)]{Hohzaki16} Hohzaki R (2016) Search games: Literature and survey. {\em J. Oper. Res. Soc. Japan} 59(1):1--34.

\bibitem[Isaacs(1965)]{Isaacs} Isaacs R (1965) {\em Differential Games} (John Wiley \& Sons, New York).

\bibitem[Johnson and Ali(2015)]{JA15} Johnson D, Ali A (2015) Modeling and simulation of landmine and improvised explosive device detection with multiple loops. {\em J. Def. Model. Simul.} 12(3):257--271.

\bibitem[Norkute {\em et al.}(2020)]{NON20} Norkute M, Olsson U, Ninkovic V (2020) Aphids?induced plant volatiles affect diel foraging behavior of a ladybird beetle Coccinella septempunctata. {\em Insect Science} 27(6):1266--1275.

\bibitem[Paley(2016)]{Paley16} Paley D (2016) {\em Optimized Routing of Intelligent, Mobile Sensors for Dynamic, Data-Driven Sampling}. University of Maryland College Park United States.

\bibitem[Patan(2012)]{Patan12} Patan M (2012) Resource Aware Mobile Sensor Routing. {\em Optimal Sensor
Networks Scheduling in Identification of Distributed Parameter Systems}, 97--134.

\bibitem[Reijnierse and Potters(1993)]{RP93} Reijnierse JH, Potters JA. (1993) Search games with immobile hider. {\em Internat. J. Game Theory} 21:385--394.

\bibitem[Zoroa {\em et al.}(2013)]{Zoroa13} Zoroa N, Fern\'{a}ndez-S\'{a}ez MJ and Zoroa P (2013) Tools to manage search
games on lattices. Alpern S, Fokkink R, G\c{a}sieniec L, Lindelauf R, Subrahmanian VS, eds. {\em Search Theory: A Game Theoretic Perspective} (Springer, New York), 29--58.


\end{thebibliography}

%%%%%%%%%%%%%%%%%
\end{document}
%%%%%%%%%%%%%%%%%
