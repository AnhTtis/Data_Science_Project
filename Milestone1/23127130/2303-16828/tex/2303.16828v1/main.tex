%%
%% This is file `sample-manuscript.tex',
%% generated with the docstrip utility.
%%
%% The original source files were:
%%
%% samples.dtx  (with options: `manuscript')
%%
%% IMPORTANT NOTICE:
%%
%% For the copyright see the source file.
%%
%% Any modified versions of this file must be renamed
%% with new filenames distinct from sample-manuscript.tex.
%%
%% For distribution of the original source see the terms
%% for copying and modification in the file samples.dtx.
%%
%% This generated file may be distributed as long as the
%% original source files, as listed above, are part of the
%% same distribution. (The sources need not necessarily be
%% in the same archive or directory.)
%%
%%
%% Commands for TeXCount
%TC:macro \cite [option:text,text]
%TC:macro \citep [option:text,text]
%TC:macro \citet [option:text,text]
%TC:envir table 0 1
%TC:envir table* 0 1
%TC:envir tabular [ignore] word
%TC:envir displaymath 0 word
%TC:envir math 0 word
%TC:envir comment 0 0
%%
%%
%% The first command in your LaTeX source must be the \documentclass command.
\documentclass[sigchi]{acmart}

\usepackage{multirow}
\usepackage{graphicx}
\usepackage{enumerate}
%%
%% \BibTeX command to typeset BibTeX logo in the docs
\AtBeginDocument{%
  \providecommand\BibTeX{{%
    \normalfont B\kern-0.5em{\scshape i\kern-0.25em b}\kern-0.8em\TeX}}}

%% Rights management information.  This information is sent to you
%% when you complete the rights form.  These commands have SAMPLE
%% values in them; it is your responsibility as an author to replace
%% the commands and values with those provided to you when you
%% complete the rights form.
% \setcopyright{acmcopyright}
% \copyrightyear{2022}
% \acmYear{2022}
% \acmDOI{}

% %% These commands are for a PROCEEDINGS abstract or paper.
% \acmConference[ICTD '22]{12th International Conference on Information & Communication Technologies and Development}{June 27-29, 2022}{Washington, USA}
% \acmISBN{}


%%
%% Submission ID.
%% Use this when submitting an article to a sponsored event. You'll
%% receive a unique submission ID from the organizers
%% of the event, and this ID should be used as the parameter to this command.
%%\acmSubmissionID{123-A56-BU3}

%%
%% The majority of ACM publications use numbered citations and
%% references.  The command \citestyle{authoryear} switches to the
%% "author year" style.
%%
%% If you are preparing content for an event
%% sponsored by ACM SIGGRAPH, you must use the "author year" style of
%% citations and references.
%% Uncommenting
%% the next command will enable that style.
%%\citestyle{acmauthoryear}

%%
%% end of the preamble, start of the body of the document source.
\copyrightyear{2022}
\acmYear{2022}
\setcopyright{rightsretained}
\acmConference[ICTD2022]{International Conference on Information \& Communication Technologies and Development 2022}{June 27--29, 2022}{Seattle, WA, USA}
\acmBooktitle{International Conference on Information \& Communication Technologies and Development 2022 (ICTD2022), June 27--29, 2022, Seattle, WA, USA}
\acmDOI{10.1145/3572334.3572372}
\acmISBN{978-1-4503-9787-2/22/06}
\begin{document}

%%
%% The "title" command has an optional parameter,
%% allowing the author to define a "short title" to be used in page headers.
\title{Tackling Hate Speech in Low-resource Languages \\with Context Experts}

%%
%% The "author" command and its associated commands are used to define
%% the authors and their affiliations.
%% Of note is the shared affiliation of the first two authors, and the
%% "authornote" and "authornotemark" commands
%% used to denote shared contribution to the research.
\author{Daniel Nkemelu}
\email{dnkemelu@gatech.edu}
\affiliation{%
  \institution{Georgia Institute of Technology}
  \city{Atlanta}
  \state{Georgia}
  \country{USA}
}

\author{Harshil Shah}
\email{harshil18@gatech.edu}
\affiliation{%
  \institution{Georgia Institute of Technology}
  \city{Atlanta}
  \state{Georgia}
  \country{USA}
}

\author{Irfan Essa}
\email{irfan@gatech.edu}
\affiliation{%
  \institution{Georgia Institute of Technology}
  \city{Atlanta}
  \state{Georgia}
  \country{USA}
}

\author{Michael L. Best}
\email{mikeb@gatech.edu}
\affiliation{%
  \institution{Georgia Institute of Technology}
  \city{Atlanta}
  \state{Georgia}
  \country{USA}
}

%%
%% By default, the full list of authors will be used in the page
%% headers. Often, this list is too long, and will overlap
%% other information printed in the page headers. This command allows
%% the author to define a more concise list
%% of authors' names for this purpose.
\renewcommand{\shortauthors}{Nkemelu et al.}

%%
%% The abstract is a short summary of the work to be presented in the
%% article.
\begin{abstract}
  Given Myanmar's historical and socio-political context, hate speech spread on social media have escalated into offline unrest and violence. This paper presents findings from our remote study on the automatic detection of hate speech online in Myanmar. We argue that effectively addressing this problem will require community-based approaches that combine the knowledge of context experts with machine learning tools that can analyze the vast amount of data produced. To this end, we develop a systematic process to facilitate this collaboration covering key aspects of data collection, annotation, and model validation strategies. We highlight challenges in this area stemming from small and imbalanced datasets, the need to balance non-glamorous data work and stakeholder priorities, and closed data-sharing practices. Stemming from these findings, we discuss avenues for further work in developing and deploying hate speech detection systems for low-resource languages.
\end{abstract}

%%
%% The code below is generated by the tool at http://dl.acm.org/ccs.cfm.
%% Please copy and paste the code instead of the example below.
%%
\begin{CCSXML}
<ccs2012>
   <concept>
       <concept_id>10003120.10003130.10003233.10010519</concept_id>
       <concept_desc>Human-centered computing~Social networking sites</concept_desc>
       <concept_significance>500</concept_significance>
       </concept>
   <concept>
       <concept_id>10003120.10003130.10011762</concept_id>
       <concept_desc>Human-centered computing~Empirical studies in collaborative and social computing</concept_desc>
       <concept_significance>500</concept_significance>
       </concept>
 </ccs2012>
\end{CCSXML}
\ccsdesc[500]{Human-centered computing~Social networking sites}
\ccsdesc[500]{Human-centered computing~Empirical studies in collaborative and social computing}

%%
%% Keywords. The author(s) should pick words that accurately describe
%% the work being presented. Separate the keywords with commas.
\keywords{hate speech, context experts, digital threats, democracy, low-resource text classification}


%%
%% This command processes the author and affiliation and title
%% information and builds the first part of the formatted document.
\maketitle

\section{Introduction}
\section{Introduction}
\label{sec:introduction}
% \begin{itemize}
%     % Diffusion of FL
%     \item {\st{Diffusion of FL}}
%     % Security threats to FL
%     \item {\st{Security threats to FL with particular focus on model poisoning}}
%     % Limitations of existing countermeasures
%     \item {\st{Current countermeasures (e.g., KRUM) and their limitations}}
%     % Proposed method and its advantages
%     \item {\st{Intuitive description of the proposed method and its difference (i.e., advantages) w.r.t. state of the art}}
%     % Main contributions
%     \item {\st{Summary of the main contributions of this work}}
%     % Paper's structure and organization
%     \item {\st{Paper's structure and organization}}
% \end{itemize}

% Diffusion of FL
Recently, {\em federated learning} (FL) has emerged as the leading paradigm for training distributed, large-scale, and privacy-preserving machine learning (ML) systems~\cite{mcmahan2017googleai,mcmahan2017aistats}. 
The core idea of FL is to allow multiple edge clients to collaboratively train a shared, global model without disclosing their local private training data.
%Specifically, an FL system consists of a central server and many edge clients; 
A typical FL round involves the following steps: {\em(i)} the server randomly picks some clients and sends them the current, global model; {\em(ii)} each selected client locally trains its model with its own private data; then, it sends the resulting local model to the server;\footnote{Whenever we refer to global/local model, we mean global/local model {\em parameters}.} {\em(iii)} the server updates the global model by computing an \emph{aggregation function}, usually the average (FedAvg), on the local models received from clients.
% \begin{enumerate}
%     \item[{\em(i)}] the server sends the current, global model to the clients and appoints some of them for training;
%     \item[{\em(ii)}] each selected client locally trains its copy of the global model with its own private data; then, it sends the resulting local model back to the server;\footnote{Whenever we refer to global/local model, we mean global/local model {\em parameters}.}
%     \item[{\em(iii)}] the server updates the global model by computing an \emph{aggregation function} on the local models received from clients (by default, the average, also referred to as FedAvg~\cite{mcmahan2017aistats}).
% \end{enumerate}
This process goes on until the global model converges. %(e.g., after a certain number of rounds or other similar stopping criteria).
%\\
% The advantages of FL over the traditional, centralized learning paradigm are undoubtedly clear in terms of flexibility/scalability (clients can join/disconnect from the FL network dynamically), network communications (only model weights\footnote{We will use \textit{parameters} and \textit{weights} interchangeably.} are exchanged between clients and server), and privacy (each client's private training data is kept local at the client's end and not uploaded to the server).
\\
% Security threats to FL
%However, the growing adoption of FL also raises security concerns~\cite{costa2022covert}, particularly about its confidentiality, integrity, and availability.
Although its advantages over standard ML, FL also raises security concerns~\cite{costa2022covert}. %, particularly about its confidentiality, integrity, and availability~\cite{costa2022covert}.
% OLD, LONG VERSION
% Indeed, some work deals with privacy leakage that may expose the local data of some clients~\cite{melis2019sp}. 
% A large body of work, instead, investigates attacks that usually aim to detriment the predictive accuracy of the learned global model. For instance, \emph{data poisoning} attacks achieve this goal by letting an adversary pollute the training set of some corrupt FL clients with maliciously crafted examples~\cite{jagielski2018sp}.
% Similarly, in \emph{model poisoning} the attacker attempts to tweak the global model weights~\cite{bhagoji2019pmlr} by directly perturbing the local model's weights of some infected FL clients before these are sent to the central server for aggregation, usually via so-called Byzantine attacks. 
% It turns out that Byzantine model poisoning attacks severely impact standard FedAvg; therefore, more robust aggregation functions must be designed to make FL systems secure.
Here, we focus on \emph{untargeted model poisoning} attacks~\cite{bhagoji2019pmlr}, where an adversary attempts to tweak the global model weights %\footnote{We will use the terms \textit{parameters} and \textit{weights} interchangeably.} 
by directly perturbing the local model's parameters of some infected clients before these are sent to the central server for aggregation.
In doing so, the adversary aims to jeopardize the global model \textit{indiscriminately} at inference time.
Such model poisoning attacks severely impact standard FedAvg; therefore, more robust aggregation functions must be designed to secure FL systems.
\\
% In this paper, we focus on designing a novel robust aggregation scheme at the server's end to contrast the effect of Byzantine model poisoning attacks.
%
% Current countermeasures and their limitations
%Several countermeasures have been proposed in the literature to combat model poisoning attacks on FL systems.
% Some methods use simple statistics more robust than plain average to smooth the impact of malicious updates (e.g., Trimmed Mean and FedMedian~\cite{yin2018icml}). 
% Other defenses implement outlier detection techniques to discard malicious updates from the aggregation performed at the server's end. Those are either based on heuristics (e.g., Krum/Multi-Krum~\cite{blanchard2017nips} and Bulyan~\cite{mhamdi2018pmlr}) or data-driven approaches (e.g., K-means clustering~\cite{shen2016acm} or DnC via spectral analysis~\cite{shejwalkar2021ndss}). 
% Finally, some strategies rely on a centralized ``source of trust'' to spot potential malicious updates (e.g., FLTrust~\cite{cao2020fltrust}).
% Several countermeasures have been proposed in the literature to combat model poisoning attacks on FL systems, i.e., to discard possible malicious local updates from the aggregation performed at the server's end. 
% These techniques range from simple statistics more robust than plain average (e.g., Trimmed Mean and FedMedian~\cite{yin2018icml}) to outlier detection heuristics (e.g., Krum/Multi-Krum~\cite{blanchard2017nips} and Bulyan~\cite{mhamdi2018pmlr}) or data-driven approaches (e.g., spectral analysis via K-means clustering~\cite{shen2016acm} or spectral analysis), or methods based on ``source of trust'' (e.g., FLTrust~\cite{cao2020fltrust}).
% OLD, LONG VERSION
%Several countermeasures have been proposed in the literature to combat Byzantine model poisoning attacks on FL systems.
% Descriptive statistics
% For example, Trimmed Mean and FedMedian aggregate local model updates using more robust statistics than standard average~\cite{yin2018icml}.
%
% % Heuristics for outlier detection
% Many existing Byzantine-resilient strategies implement some outlier detection heuristics to discard the model updates sent by potentially malicious clients from the input of the aggregation function.
% One of the most popular heuristics is Krum~\cite{blanchard2017nips}.
% This strategy tries to mitigate the impact of Byzantine attacks by selecting as a global model the local model with the smallest sum of Euclidean distances to {\em all} the other local models.
% Although powerful, Krum requires the server to know (or, at least, estimate) the number of malicious FL clients upfront, which is generally impossible in a realistic attack scenario. %
% Moreover, Krum may become ineffective for complex, high-dimensional model parameter spaces due to the curse of dimensionality.
% Bulyan~\cite{mhamdi2018pmlr} tries to overcome this issue by combining Krum with a variant of Trimmed Mean.
% % Data-driven outlier detection
% Other strategies use data-driven outlier detection techniques -- e.g., via K-means clustering~\cite{shen2016acm} -- to spot potential malicious local model updates. 
% %For instance, Shen et al. propose to cluster local model updates with K-means and thus identify outliers.
%
% % Other techniques
% As far as the server is concerned, any local model received can be from a potential malicious client. 
% FLTrust~\cite{cao2020fltrust} assumes the server acts as a client, i.e., trains a local model on an additional {\em trustworthy} dataset at the server's end and compares it against all the local models from other clients. 
% This way, the server can rely on some ``source of trust'' when discarding potentially malicious clients.
%\\
% Limitations of existing Byzantine-resilient strategies
Unfortunately, existing defense mechanisms either rely on simple heuristics (e.g., Trimmed Mean and FedMedian by~\cite{yin2018icml}) or need strong and unrealistic assumptions to work effectively (e.g., foreknowledge or estimation of the number of malicious clients in the FL system, as for Krum/Multi-Krum~\cite{blanchard2017nips} and Bulyan~\cite{mhamdi2018pmlr}, which, however, cannot exceed a fixed threshold).
Furthermore, outlier detection methods using K-means clustering~\cite{shen2016acm} or spectral analysis like DnC~\cite{shejwalkar2021ndss} do not directly consider the temporal evolution of local model updates received.
Finally, strategies like FLTrust~\cite{cao2020fltrust} require the server to collect its own dataset and act as a proper client, thereby altering the standard FL protocol.
\\
% OLD, LONG VERSION
% Overall, existing Byzantine-resilient strategies are either simple heuristics (e.g., FedMedian) or, if they are more complex, they rely on strong and unrealistic assumptions to work effectively (e.g., knowing the number of malicious clients in the FL system in advance, as for Krum and alike).
% Furthermore, data-driven outlier detection methods do not consider the temporary evolution of local model updates received (e.g., K-means clustering). 
% Finally, strategies like FLTrust requires the server to collect its own dataset and act as a proper client, thereby altering the standard FL protocol.
%
% Description of the proposed method
This work introduces a novel pre-aggregation \textit{filter} robust to untargeted model poisoning attacks. Notably, this filter $(i)$ operates without requiring prior knowledge or constraints on the number of malicious clients and $(ii)$ inherently integrates temporal dependencies. 
The FL server can employ this filter as a preprocessing step before applying \textit{any} aggregation function, be it standard like FedAvg or robust like Krum or Bulyan.
Specifically, we formulate the problem of identifying corrupted updates as a multidimensional (i.e., matrix-valued) time series anomaly detection task. 
The key idea is that legitimate local updates, resulting from well-calibrated iterative procedures like stochastic gradient descent (SGD) with an appropriate learning rate, show \textit{higher predictability} compared to malicious updates. This hypothesis stems from the fact that the sequence of gradients (thus, model parameters) observed during legitimate training exhibit regular patterns, as validated in Section~\ref{subsec:intuition}. %until convergence. 
%This regularity may be more pronounced for smooth convex loss functions, but it can still be captured within an appropriate time window, even for more complex and convoluted loss surfaces. 
%We provide evidence of this claim in Appendix~B, where we show that the average mutual information (i.e., ``predictability''), calculated over pairs of legitimate model updates sent at different FL rounds, is significantly higher than the corresponding computation for a malicious client.
\\
Inspired by the matrix autoregressive (MAR) framework for multidimensional time series forecasting~\cite{chen2021je}, we propose the FLANDERS ({\em \textbf{F}ederated \textbf{L}earning meets \textbf{AN}omaly \textbf{DE}tection for a \textbf{R}obust and \textbf{S}ecure}) filter.
The main advantages of FLANDERS over existing strategies like FLDetector~\cite{zhao2020multivariate} are its resilience to large-scale attacks, where $50\%$ or more FL participants are hostile, and the capability of working under realistic non-iid scenarios.
We attribute such a capability to two key factors: $(i)$ FLANDERS works without knowing a priori the ratio of corrupted clients, and $(ii)$ it embodies temporal dependencies between intra- and inter-client updates, quickly recognizing local model drifts caused by evil players. Below, we summarize our main contributions:

\begin{itemize}
\item[{\em(i)}]
We provide empirical evidence that the sequence of models sent by legitimate clients is more predictable than those of malicious participants performing untargeted model poisoning attacks.
\\
\item[{\em(ii)}] 
We introduce FLANDERS, the first pre-aggregation filter for FL robust to untargeted model poisoning based on multidimensional time series anomaly detection.
\\
\item[{\em(iii)}] 
We integrate FLANDERS into Flower,\footnote{\scriptsize{\url{https://flower.dev/}}} a popular FL simulation framework for reproducibility.
\\
\item[{\em(iv)}] 
We show that FLANDERS improves the robustness of the existing aggregation methods under multiple settings: different datasets, client's data distribution (non-iid), models, and attack scenarios.
\\
\item[{\em(v)}] 
We publicly release all the implementation code of FLANDERS along with our experiments.\footnote{\scriptsize{\url{https://anonymous.4open.science/r/flanders_exp-7EEB}}}
\end{itemize}

% Paper's structure and organization
The remainder of the paper is structured as follows. %some related work and the current state-of-the-art solutions to security issues that FL entails. 
Section~\ref{sec:background} covers background and preliminaries. 
In Section~\ref{sec:related}, we discuss related work.
Section~\ref{sec:problem} and Section~\ref{sec:method} describe the problem formulation and the method proposed. % to tackle it. 
Section~\ref{sec:experiments} gathers experimental results. %, and Section~\ref{sec:limitations} discusses some limitations of this work.
Finally, we conclude in Section~\ref{sec:conclusion}.
 %discusses the limitations of this work and draws future research directions.
%reports conclusions and draws perspectives for future research directions.

%%%%%%% OLD %%%%%%%
%to overcome the resilience of Byzantine failures in distributed Stochastic Gradient Descent computations. 
% The strength of Krum is its time complexity, which is linear in the gradient dimension. 
% However, the robustness of the approach is guaranteed for gradient-based learning applications only when the majority of the clients are not compromised. 
% Besides, the aggregation mechanism of Krum, as well as that of similar methods, is robust from a coarse-grained perspective and does not provide solutions to errors and perturbations that may occur at inference time.
%A related approach to~\cite{blanchard2017nips} is the work of Su et al.~\cite{su2016dc}. Here, the authors propose an iterated approximate agreement to tackle a multi-layer scenario attacked by Byzantine agents. 
%However, the method works efficiently on the sole discrete context and it is inapplicable to continuous state environments.
%\gabri{Maybe, we should just talk about the main limitations of existing countermeasures without digging into their details (or, we can just mention Krum as this is the most popular one). I will move the description of all these methods to the Related Work section.}

\section{Background}
\section{Related Work}
\label{sec:relatedwork}

%%%%%%%%%%%%%%%%%%%%%%%%%% Outline %%%%%%%%%%%%%%%%%%%%%%%%%%%%%%%%%%%%%
%(1) Evasion Attacks
%(1.1) Surveys on evasion attacks and their relation to data properties - Michael
%(1.2) Individual papers that study non-data related reasons behind evasion attacks - Michael
%(1.3) Techniques related to evasion attacks and defenses (new) - Gabby
%(2) Non-Evasion Attacks (new), and - ???
%(3) Effects of training data on standard generalization - done 
%
%
%
%(1) Evasion Attacks
%(1.1) A number of surveys review literature on evasion attacks. - Michael
%Most of them do not focus specifically on properties of data but also discuss attack and defense mechanisms, non-data-related reasons for adversarial vulnarability, and  more. ~\jr{cite 4}.
%Yet, they these surveys mention data and its relation to evasion attacks. Specifically \jr{what they say about data.}
%The most close to ours is concurrent work by XXX + concrete facts that we have and they don't.
%
%(1.2) individual papers that study non-data related reasons behind evasion attacks, - Michael
%Literature identifies multiple reasons for adversarial vulnerability, in particular, for evasion attacks. 
%These include data-related properties extensively discussed in this survey, as well as reasons related to the models 		   themselves, computations resources, and feature representations. We discuss these below. 
%
%\jr{the rest is from the paper (non-data related reasons for adversarial vulnerability), with sections potentially renamed.}
%
%{\bf Model.}
%
%{\bf Computational Resources.}
%
%{\bf Robustness of Features.}
%
%(1.3) Techniques Related to Evasion Attacks and Defenses (new) - Gabby
%A number of works focus on techniques for generating evasion attacks, countermeasures against these attacks, 
%and defining the notion of the attack itself.   
%
%{\bf Attacks and Defense.}
%Here are the 5 remaining surveys + 1 additional paper for the reviewer.
%
%{\bf Adversarial Examples.}
%2 surveys lines 13 and 14 + 1 additional paper for the reviewer.
%
%(2) Non-Evasion Attacks (new) 
%Need to say that there are other type of attacks, define them, cite surveys (Bo's survey, maybe something else). 
%Only one work explicitly focus on effects of data. 
%
%
%(3) Effects of training data on standard generalization (done)

%%%%%%%%%%%%%%%%%%%%%%%%% Outline %%%%%%%%%%%%%%%%%%%%%%%%%%%%%%%%%%%%%


\revreplace{
We divide related work into three categories:
(1) surveys on adversarial robustness and its relation to data properties,
(2) surveys that discuss the influence of data properties on standard generalization, and
(3) individual papers that study non-data-related reasons for adversarial vulnerability.\\
}
{
This survey investigates properties of training data in the context of model robustness under evasion attacks. 
We start the discussion of related work by reviewing other surveys that focus on evasion attacks and 
include some discussion about data (Section~\ref{sec:relatedwork-surveys-data}).  
We then discuss non-data related reasons behind evasion attacks (Section~\ref{sec:relatedwork-not-data}),
as well as techniques related to evasion attacks and defenses (Section~\ref{sec:relatedwork-attacks}). 
Finally, we discuss data-related concerns for non-evasion attacks (Section~\ref{sec:relatedwork-poisoning}) and
the effects of training data on standard generalization (Section~\ref{sec:relatedwork-standard}).
}

%\vspace{-0.1in}
\subsection{Surveys on Evasion Attacks that Discuss Data}
\label{sec:relatedwork-surveys-data}
Numerous existing surveys 
\revreplace{focus on attack and defense techniques for adversarial robustness. 
%~\cite{Biggio:Roli:PR:2018,
%Rosenberg:Shabtai:Elovici:Rokach:CSUR:2021,
%Li:Li:Ye:Xu:CSUR:2021,
%Maiorca:Biggio:Giorgio:CSUR:2019,
%Demetrio:Coull:Biggio:Lagorio:Armando:Roli:ACMTPS:2021,
%Liu:Tantithamthavorn:Li:Liu:CSUR:2022,
%Liu:Nogueria:Fernandes:Kantarci:IEEECST:2022,
%Akhtar:Mian:IEEEAccess:2018,
%Akhtar:Mian:Kardan:Shah:IEEEAccess:2021,
%Serban:Poll:Visser:CSUR:2020,
%Machado:Silva:Goldschmidt:CSUR:2021,
%Zhang:Sheng:Alhazmi:Li:ACMTIST:2020}.
Only a few of these works mention the relationship between adversarial robustness and properties of the underlying data.} 
{review the literature on evasion attacks.
Most of these works do not focus specifically on properties of data but discuss attack and defense mechanisms, non-data-related reasons for adversarial vulnerability, 
and the different threat models. 
Only a few of these works mention data-related reasons for the existence of adversarial examples~\cite{Serban:Poll:Visser:CSUR:2020, Machado:Silva:Goldschmidt:CSUR:2021, Akhtar:Mian:Kardan:Shah:IEEEAccess:2021, Akhtar:Mian:IEEEAccess:2018}.
}
Specifically, Serban et al.~\cite{Serban:Poll:Visser:CSUR:2020} observe that adversarial vulnerability can be caused by an insufficient training sample size %~\cite{Schmidt:Santurkar:Tsipras:Talwar:Madry:NeurIPS:2018}
and high data dimensionality. %~\cite{Gilmer:Metz:Faghri:Schoenholz:Raghu:Wattenberg:Goodfellow:ICLR:2018}.
Similarly, Machado et al.~\cite{Machado:Silva:Goldschmidt:CSUR:2021} mention that the lack of sufficient training data, high dimensionality, 
and high concentration contribute to adversarial vulnerability.
\revadd{
Akhtar et al.~\cite{Akhtar:Mian:IEEEAccess:2018, Akhtar:Mian:Kardan:Shah:IEEEAccess:2021} also mention high dimensionality, along with other non-data-related reasons, 
as a source of adversarial examples.}

\revadd{A concurrent work by Han et al.~\cite{Han:Lin:Shen:Wang:Guan:CSUR:2023} (published at the end of April 2023) 
studies the origins of adversarial vulnerability in deep learning w.r.t. the model, data, and other perspectives.
The authors mention high dimensionality, distributions with high concentration, a small number of output classes, data imbalance, and the perceptual difference in image frequencies as potential sources of adversarial examples.
However, as (a) the focus of that survey is not on data-related properties in particular, 
(b) its paper search was conducted in 2021, and 
(c) it focuses on deep learning models only, 
our work was able to identify more than 50 additional relevant papers which focus on other types of models, 
e.g., non-parametric and linear classifiers, 
and/or discuss additional types of data-related properties, 
such as, types of distribution, class density, separation, and label quality.}
\revreplace{Yet, none of these surveys explicitly collect and analyze work that focuses on the effects of data properties
on adversarial robustness.}
{In summary, by explicitly focusing on the effects of data properties on evasion attacks in our survey, 
we are able to provide a more complete and detailed discussion on this topic, not covered in prior surveys.}

\vspace{-0.05in}
\subsection{Non-data-related Reasons Behind Evasion Attacks}
\label{sec:relatedwork-not-data}

%\vspace{-0.1in}
%\subsection{Non-data Related Reasons for Adversarial Vulnerability}

There has been a variety of hypotheses regarding the reasons behind adversarial vulnerability of ML systems, particularly for evasion attacks.
%\revreplace{
%In addition to the data used for training,  adversarial robustness could also depend on the choice of the model architecture,
%the training procedure, and the interplay between data and the learning algorithm, i.e., correspondence between the complexity of a model to that of the data.
%This section summarizes the key hypotheses regarding these aspects.
%%The hypotheses reviewed in this section are complementary to the potential influence from the data.
%}
These include data-related properties extensively discussed in this survey, as well as reasons related to the models themselves, 
computational resources, and feature learning procedures. We discuss these below.

%\jr{there is a lot of undefined terminology and jargon in this section.}

\vspace{0.02in}
\noindent
\textbf{Model.}
When Szegedy et al.~\cite{Szegedy:Zaremba:Sutskever:Bruna:Erhan:Goodfellow:Fergus:ICLR:2014} first discovered adversarial examples for visual models, they suspected that the high non-linearity of DNNs resulted in low probability `pockets' of adversarial examples in the learned representation manifold.
They hypothesize that while these pockets can be found through attack algorithms, the samples residing in these pockets have different distributions compared to normal samples and are thus subsequently harder to find when randomly sampling from the input space.
Instead, Goodfellow et al.~\cite{Goodfellow:Shlens:Szegedy:ICLR:2015} hypothesize that
the linearity from activation functions, like ReLU and sigmoid found in high-dimensional neural networks, induce vulnerability towards adversarial perturbations.
To support their claim, they present the attack method FGSM that exploits the linearity of the target classifier.
Fawzi et al.~\cite{Fawzi:Fawzi:Frossard:ICMLWorkshop:2015} also argue against the hypothesis of high non-linearity as the cause for adversarial examples.
They show that all classifiers are susceptible to adversarial attacks and claim that it is the low flexibility of the classifier compared to the complexity of the classification task that results in vulnerability.
The lack of consensus on the primary causes of model vulnerability invites more studies on this topic.

Singla et al.~\cite{Singla:Ge:Basri:Jacobs:NeurIPS:2021} show that enforcing invariance to circular shifts (e.g., rotation) in neural networks induces decision boundaries with a smaller margin than normal, fully connected networks,
which, in turn, reduces the adversarial robustness of the model.
Moosavi{-}Dezfooli et al.~\cite{Moosavi-Dezfooli:Fawzi:Fawzi:Frossard:Soatto:ICLR:2018} introduce universal,
input-agnostic perturbations to mislead the classifier and hypothesize that the vulnerability of a multi-class classifier to such perturbations is related to the shape of its decision boundaries, e.g.,
linear classifiers with decision boundaries that are parallel to each other and
nonlinear classifier with decision boundaries that are curved in a similar way
tend to be less robust as
perturbations in one direction can change the prediction label for a different class.

Tanay and Griffin~\cite{Tanay:Griffin:ArXiv:2016} conjecture that the decision boundary learned by the classifier being too close to (or `tilted towards') the data manifold instead of being perpendicular to it,
results in small perturbations being sufficient to move samples across the decision boundary for misclassification.
%data manifold refers to the underlying structure that the data exhibit

\vspace{0.02in}
\noindent
\textbf{Computational Resources.}
Bubeck et al.~\cite{Bubeck:Lee:Price:Razenshteyn:ICML:2019} use computational hardness theory to show that the time complexity for learning a robust model is exponential to the size of input data and thus is computationally intractable.
Hence, they attribute adversarial vulnerability to computational limitations of current learning algorithms.
Degwekar et al.~\cite{Degwekar:Nakkiran:Vaikuntanathan:COLT:2019} further extend this work and also show the impossibility of efficiently training robust classifiers.

%\subsubsection{Ineffective Learning Perspective}
\vspace{0.02in}
\noindent
\textbf{Feature Learning.}
Ilyas et al.~\cite{Ilyas:Santurkar:Tsipras:Engstrom:Tran:Madry:NeurIPS:2019} show that adversarial vulnerability can be a consequence of a model exploiting well-generalizing but non-robust features,
i.e., features that are spurious and sometimes incomprehensible to humans;
when constraining the model to use robust features, the adversarial robustness increases together with the
interpretability of the learned features.
However, Tsipras et al.~\cite{Tsipras:Santurkar:Engstrom:Turner:Madry:ICLR:2019} note that, as the features for achieving high accuracy may be different from the ones for achieving high robustness, robustness may be at odds with standard accuracy.
%
%\jr{why is it called Ineffective learning when it is about features.}\gx{I put it under ineffective learning as in this case, the model learns/decides the features for generalization, and when given the correct objective, the model in fact, can learn more robust features, so I think the underlying reason is objective we gave for the model didn't guide the model to learn the right features}
%
Instead of seeing adversarial vulnerability as a product of classifiers being overly sensitive to changes in spurious features, Jacobsen et al.~\cite{Jacobsen:Behrmann:Zemel:Bethge:ICLR:2019} hypothesize that classifiers can rather be
overly insensitive to relevant semantic information, e.g., images with drastically different content can share similar latent representations.
The authors introduce a new type of adversarial examples that exploit such insensitivity, where the content of images is altered without changing the resulting prediction label.
%As both insensitivity to semantic content and sensitivity to spurious changes can simultaneously exist in models,
%more investigation into how to define proper objectives for models to effectively distinguish the relevant information is needed.

While all these works propose possible reasons for adversarial vulnerabilities, they are orthogonal to our survey, which focuses particularly on the influence of training data.

\vspace{-0.05in}
\revadd{
\subsection{Evasion Attacks and Defenses}
\label{sec:relatedwork-attacks}
A number of works focus on techniques for generating evasion attacks, countermeasures against these attacks, 
and defining the notion of the attack itself.

%\jr{need to include~\cite{Biggio:Roli:PR:2018,
%Rosenberg:Shabtai:Elovici:Rokach:CSUR:2021,
%Li:Li:Ye:Xu:CSUR:2021,
%Maiorca:Biggio:Giorgio:CSUR:2019,
%Demetrio:Coull:Biggio:Lagorio:Armando:Roli:ACMTPS:2021,
%Liu:Tantithamthavorn:Li:Liu:CSUR:2022,
%Liu:Nogueria:Fernandes:Kantarci:IEEECST:2022,
%Zhang:Sheng:Alhazmi:Li:ACMTIST:2020} x and one more survey.}
%\js{\cite{Biggio:Roli:PR:2018, Rosenberg:Shabtai:Elovici:Rokach:CSUR:2021} moved to Adversarial Examples.
%\cite{Rosenberg:Shabtai:Elovici:Rokach:CSUR:2021,
%Li:Li:Ye:Xu:CSUR:2021,
%Maiorca:Biggio:Giorgio:CSUR:2019, Liu:Tantithamthavorn:Li:Liu:CSUR:2022,
%Liu:Nogueria:Fernandes:Kantarci:IEEECST:2022,
%Zhang:Sheng:Alhazmi:Li:ACMTIST:2020, Demetrio:Coull:Biggio:Lagorio:Armando:Roli:ACMTPS:2021} in Attacks and Defense. \cite{Sun:Dou:Yang:Zhang:Wang:Philip:He:Li:TKDE:2022} was the "one more survey" and is also in Attacks and Defenses.}

\vspace{0.02in}
\noindent
{\bf Attacks and Defense.}
Several works~\cite{Liu:Tantithamthavorn:Li:Liu:CSUR:2022,Liu:Nogueria:Fernandes:Kantarci:IEEECST:2022,Sun:Dou:Yang:Zhang:Wang:Philip:He:Li:TKDE:2022, Demetrio:Coull:Biggio:Lagorio:Armando:Roli:ACMTPS:2021} survey adversarial attacks and defenses, observing that most work focuses on computer vision and NLP domains. 
Zhang et al.~\cite{Zhang:Sheng:Alhazmi:Li:ACMTIST:2020}, 
Rosenberg et al.~\cite{Rosenberg:Shabtai:Elovici:Rokach:CSUR:2021},
Li et al.~\cite{Li:Li:Ye:Xu:CSUR:2021}, and 
Maiorca et al.~\cite{Maiorca:Biggio:Giorgio:CSUR:2019}, 
survey attacks and defenses in the NLP domain, cybersecurity domain for networks, Android malware, and PDF malware, respectively. 
These works identify a similar trend of new attacks constantly bypassing defenses, which gives rise to new defenses being proposed, only to be broken again (a.k.a. the `cat and mouse race' or the `arms race'). 
They also observe that research in this field studies attacks / defenses at a feature-level, which restricts 
the practicality of the developed techniques by the feasibility of perturbing the corresponding features in real life. 

%practical attacks are quite difficult and require some basic knowledge about the model or training data such as the feature set or model architecture. 
%Zhang et al.~\cite{Zhang:Sheng:Alhazmi:Li:ACMTIST:2020}, who study adversarial attacks and defenses in the NLP domain,  
%also find that there are obstacles to generating attacks in real-time. 
%For instance, methods that iteratively use gradients to create adversarial examples can be time-consuming, while one-time approaches may fail to produce potent adversarial examples.
%Several works~\cite{Liu:Tantithamthavorn:Li:Liu:CSUR:2022,Liu:Nogueria:Fernandes:Kantarci:IEEECST:2022,Sun:Dou:Yang:Zhang:Wang:Philip:He:Li:TKDE:2022, Demetrio:Coull:Biggio:Lagorio:Armando:Roli:ACMTPS:2021} 
%discuss how most new attacks and defenses are explored in computer vision and NLP, prior to other fields.


%our survey finds the state of the art w.r.t. data properties
%our survey finds that dimensionality is bad ...
%
%%%Here are the 5 remaining surveys + 1 additional paper for the reviewer.
%Numerous surveys have explored the landscape of adversarial evasion attacks and defenses. 
%For instance, Akhtar et al.~\cite{Akhtar:Mian:IEEEAccess:2018, Akhtar:Mian:Kardan:Shah:IEEEAccess:2021} survey the literature on adversarial robustness of deep learning models from Computer Vision field.
%They review popular attacks on visual models, and provided a categorization of existing defense techniques based on the components it modify in the visual model system \gx{Check}.
%
%Rosenberg et al.~\cite{Rosenberg:Shabtai:Elovici:Rokach:ACMComputingSurvey:2021}, Li et al. ~\cite{Li:Li:Ye:Xu:ACMComputingSurvey:2021} and Demetrio et al.~\cite{Demetrio:Coull:Biggio:Lagorio:Armando:Roli:ACMTPS:2021} review the literature on evasion attacks for cyber-security fields. 
%Li et al. proposed a partial order scheme to compare key attacks and defenses techniques for malware detection in Windows, Android, and PDF domains. 
%
%Zhang et al.~\cite{Zhang:Sheng:Alhazmi:Li:ACMTIST:2020} review the literature on adversarial attacks on deep-learning models for textual classification.
%They pointed out the intrinsic differences between Computer Vision and Natural Language Processing fields that pose challenges to directly apply attacks proposed for Visual models to NLP models and identified the strategies proposed that overcomes the barriers.
%The challenges they identified for creating realistic attacks in NLP fields are from a domain characteristics perspective (e.g., definition of imperceptible perturbations, measurement of the semantic changes),  we differ from them by trying to understand the adversarial robustness of machine learning from the characteristics of underlying data. 
%
%Attack and Defenses for wireless and Mobile systems~\cite{Liu:Nogueria:Fernandes:Kantarci:IEEECST:2022}
%
%

More recent research, not included in the surveys above, has also started investigating the 
susceptibility of newer models to adversarial evasion attacks. 
For example, several studies~\cite{Wang:Pan:Hu:Duan:Pan:IJSWIS:2022,Yin:Lin:Sun:Wei:Chen:TIFS:2023, 
Shi:Han:Tan:Kuang:NeurIPS:2022, Wang:Xie:Microsoft:ChatGPT:ArXiv:2023} proposed attack techniques against contemporary models, 
such as Graph Neural Networks, Generative Pre-training Transformers (GPT), and Vision Transformers. 
These studies showed that adversarial examples persist even for the newer models, some of which are 
trained with large volumes of data. 
As all these works focus on attack and defense mechanisms rather than 
the effects of data on adversarial robustness, our work extends and complements this research.
}

\revadd{
\vspace{0.02in}
\noindent
{\bf Adversarial Examples.}
%2 surveys lines 13 and 14 + 1 additional paper for the reviewer.
Adversarial examples are inputs constructed by perturbing a correctly classified sample in a way that makes the change imperceptible to a human. % but causes the model to misclassify the sample.
However, as `imperceptible to a human' is hard to define, existing research on adversarial examples approximates imperceptibility with a small perturbation measured through $L_p$ norms.
A line of research~\cite{Gilmer:Adams:Goodfellow:Anderson:Dahl:ArXiv:2018,Sharif:Bauer:Reiter:CVPRW:2018,Fezza:Bakhti:Hamidouche:Deforges:QoMEX:2019, Mezher:Deng:Karam:EUVIP:2022} 
investigates the validity of this assumption. 
This work shows that perturbations generated by $L_p$ norms do not entirely align with human perceptions, 
i.e., some changes with a small $L_p$ norm can be apparent to humans. 
In addition, adversarial examples with the minimum $L_p$ perturbation may be less effective and transferable than 
higher perturbation~\cite{Biggio:Roli:PR:2018,Rosenberg:Shabtai:Elovici:Rokach:CSUR:2021}. 
Hence, a number of approaches explore metrics for imperceptibility 
in computer vision and NLP domains~\cite{Fezza:Bakhti:Hamidouche:Deforges:QoMEX:2019,Mezher:Deng:Karam:EUVIP:2022, Zhang:Sheng:Alhazmi:Li:ACMTIST:2020}. 
Yet another issue with $L_p$ norms is that they cannot be used reliably in domains other than images. 
For example, in the case of software/malware, simply generating adversarial examples with $L_p$ norms 
may result in feature representations that are not possible in 
the problem space~\cite{Rosenberg:Shabtai:Elovici:Rokach:CSUR:2021,Pierazzi:Pendlebury:Cortellazz:Cavallaro:2020}. 

While all these works focus on the properties of adversarial examples, 
they are orthogonal to the topic of our survey, as we rather focus on how properties of the training data 
affect the success of adversarial examples.
}

%Gilmer et al.~\cite{Gilmer:Adams:Goodfellow:Anderson:Dahl:ArXiv:2018} argue that, while constraining the perturbations by sufficiently small $L_p$ norms can generate indistinguishable samples for most inputs, the actual imperceptibility of the changes depends on the input sample. 
%Several individual studies~\cite{Sharif:Bauer:Reiter:CVPRW:2018,Fezza:Bakhti:Hamidouche:Deforges:QoMEX:2019, Mezher:Deng:Karam:EUVIP:2022} find faults with using $L_p$ norms to generate adversarial examples. They show that the changes measured by $L_p$ norm, does not entirely align with human perceptions, i.e., some changes with a small $L_p$ norm appear apparent to humans. 
%In some domains adversarial examples do not need to be imperceptible but rather semantically preserving. 
%For example, in the case of Android malware~\cite{Rosenberg:Shabtai:Elovici:Rokach:CSUR:2021}, adversarial examples are small perturbations which fool a model while preserving the semantics of the sample, 
%i.e., a malware stays malicious even after the perturbation. 
%This highlights another problem with $L_p$ norm based adversarial examples as Dong et al.~\cite{Dong:Liu:Shang:NeurIPS:2022} show that the semantics of a sample change during adversarial training. 
%Hence, there is a need for metrics to measure the size of perturbations that is imperceptible or semantically preserving.
%Fezza et al.~\cite{Fezza:Bakhti:Hamidouche:Deforges:QoMEX:2019} and Mezher et al.~\cite{Mezher:Deng:Karam:EUVIP:2022} propose to use objective metrics for image quality to approximate the imperceptibility in the computer vision domain.
%Zhang et al.~\cite{Zhang:Sheng:Alhazmi:Li:ACMTIST:2020}, focusing on providing such a metric for Natural Language Processing.
%Vadillo et al.~\cite{Vadillo:Santana:CS:2022} also highlight conducted subject studies to evaluate the noticeability of audio adversarial examples.

%Even in computer vision, adversarial examples are not always imperceptible. For example, Machado et al.~\cite{Machado:Silva:Goldschmidt:CSUR:2021} find that visible perturbations such as adversarial patch~\cite{Brown:Mane:Roy:Abadi:Gilmer:ArXiv:2017}, and graffiti on stop signs~\cite{Eykholt:Evtimov:Fernandes:Li:Rahmati:Xiao:Prakash:Kohno:Song:CVPR:2018} are also considered adversarial examples in research.

%The aforementioned research examines the work on defining and creating adversarial examples, demonstrating the insufficiency of using conventional $L_p$ norms to evaluate the imperceptibility and semantics between clean and adversarial examples. 

\vspace{-0.1in}
\revadd{
\subsection{Non-Evasion Attacks}
\label{sec:relatedwork-poisoning}
Similar to evasion attacks, data poisoning and backdoor attacks aim to compromise model accuracy. 
However, they achieve it by tampering the training data to create deceptive model decision boundaries. 
%Data poisoning attacks involve modifying the training data to create deceptive decision boundaries, either to manipulate the prediction outcomes of a specific input or the entire model.
%Meanwhile, Backdoor attacks are a form of poisoning attacks where the attacker inject tempered training data with triggers 
% and then activates the attack by showing the trigger pattern at inference time.
In addition, backdoor attacks also require perturbing the test instance to result in a misclassification. 
This is achieved by introducing manipulated training data with triggers that can be activated during the testing phase.

Goldblum et al.~\cite{Goldblum:Tsipras:Xie:Chen:Schwarzchild:song:Madry:Li:Goldstein:TPAMI:2022} and Cinà et al.~\cite{Cina:Grosse:Demontis:Sebastiano:Zellinger:Moser:Oprea:Biggio:Pelillo:Roli:CSUR:2023} 
review recent literature on attack methodologies and countermeasures for both poisoning and backdoor attacks.
Both of these surveys found that existing research made overly-optimistic assumptions when designing / validating attack techniques, e.g., assuming the knowledge of a large portion of training data. 
They advocate for researchers to test proposed methods in more realistic situations to better assess the potential threats. 
Furthermore, they encourage exploration of the relationship between poisoning attacks and evasion attacks. 
This could lead to the creation of attacks that produce less noticeable poisoning examples, 
or defensive strategies that can safeguard models against both backdoor and evasion attacks.
%Their survey catalogs and systematizes the threats in the dataset creation process, and discuss the open problems that benefits the understanding of dataset security. 

In addition to undermining model accuracy, 
adversarial attacks also aim at breaching the privacy and confidentiality of training data. 
In particular, membership inference attacks~\cite{Shokri:Stronati:Song:Shmatikov:SP:2017} attempt to determine whether a specific data point was part of the training set used to train the model.
Hu et al.~\cite{Hu:Salcic:Sun:Dobbie:Yu:Zhang:CSUR:2022} present a comprehensive survey of existing research efforts on membership inference attacks. 
They find that, similar to evasion attacks, the membership inference attack success rate decreases as 
%the training data better represents the whole data distribution, i.e., 
the number of training samples increases.
%and model stealing attacks~\cite{Oliynyk:Mayer:Rauber:CSUR:2023} are designed to breach the privacy of training data and machine learning models. 
However, all these attacks are orthogonal to our survey, as we focus on adversarial evasion attacks.

%Li et al. ~\cite{Li:Jiang:Li:Xia:TNNLS:2022} 
%provide the first survey that focuses on backdoor attacks and identified common scenarios in which backdoor attack happen in real life. 
%Furthermore, they proposed a systematic taxonomy for backdoor attacks and defenses for researchers and practitioners to identify the characteristics and limitations of each method. 

%Wang et al.~\cite{Wang:Ma:Wang:Hu:Qin:Ren:CSUR:2022} and Tian et al.~\cite{Tian:Cui:Liang:Yu:CSUR:2022} argue federated learning~\cite{McMahan:Moore:Ramage:Hampson:Arcas:AISTATS:2017} 
%creates new venue for poisoning attack, and survey recent literature on poisoning attacks for both standard and federated learning scenarios. 
%They present a unified framework to categorize both data poisoning and model poisoning attacks, and compared the defense techniques proposed for each of the learning framework, analyzed their advantages and disadvantages.
}

\vspace{-0.1in}
\subsection{Effects of Training Data on Standard Generalization}
\label{sec:relatedwork-standard}
A number of surveys investigate the influence of data properties on standard
rather than robust generalization.
One of the earliest is probably the work of Raudys and Jain~\cite{Raudys:Jain:TPAMI:1991},
who review studies related to the influence of sample size on binary classifiers, showing that
a limited sample size usually leads to sub-optimal generalization.
%With the development of deep learning and the ever-increasing need for larger training datasets,
%a variety of data augmentation techniques have been proposed.
Bansal et al.~\cite{Bansal:Sharma:Kathuria:CSUR:2021} and
Bayer et al.~\cite{Bayer:Kaufhold:Reuter:CSUR:2022} also survey papers addressing the data scarcity problem,
focusing in particular on the recent advancements in data augmentation techniques in the fields of computer vision, security, and text classification.
Their results show that augmentation techniques %exist for various application domain and
can help improve a model's generalization by reducing the problem of model overfitting.
%They evaluate the effectiveness of such techniques in improving the accuracy of machine learning models.

%Limited sample size is also one of the culprit behind poor robust generalization~\cite{Schmidt:Santurkar:Tsipras:Talwar:Madry:NeurIPS:2018}, we collected a number of researches characterize the sample complexity for robust generalization or propose data augmentation techniques to fill in the sample complexity gap.

Label noise is another aspect of data that influences both standard and robust generalization.
Most works on this topic find that the presence of noisy labels increases the need for a greater number of training samples and may result in unnecessarily complex decision boundaries~\cite{Frenay:Verleysen:TNNLS:2014,Song:Kim:Park:Shin:Lee:TNNLS:2022}.
For example, Fr\'{e}nay and Verleysen~\cite{Frenay:Verleysen:TNNLS:2014} show
that overfitting to label noise greatly degrades a model's standard generalization;
the same effect has been observed in the case of robust generalization~\cite{Sanyal:Dokania:Kanade:Torr:ICLR:2021}.
Song et al.~\cite{Song:Kim:Park:Shin:Lee:TNNLS:2022} survey the impact of label noise in deep learning, arguing
that the presence of noisy labels is a more serious concern for deep models as they contain a larger number of parameters which makes them prone to overfitting to the noise in training data.
%They also point out the connection between adversarial poisoning attacks and noisy labels as
%the countermeasures for both share the goal of learning noise-resilient representations.
They mention that adversarial defense techniques, e.g., adversarial training, are effective against label noise~\cite{Zhu:Zhang:Han:Liu:Niu:Yang:Kankanhalli:Sugiyama:ArXiv:2021, Fatras:Damodaran:Lobry:Flamary:Tuia:Courty:TPAMI:2022}
but do not discuss how label noise influences a deep learning model's robustness under attacks.

Lorena et al.~\cite{Lorena:Garcia:Lehmann:Souto:Ho:CSUR:2020} identify a collection of 26 quantitative metrics that measure data complexity with respect to
(1) ambiguity of classes, i.e., whether the classes can be clearly distinguished with the given features,
(2) sparsity and dimensionality of data, 
%i.e., whether enough information are provided to learn confident decision boundaries, and
(3) complexity of boundary separating the classes, i.e., whether more intricate functions are required to describe the decision boundaries.
The authors also discuss how these metrics help estimate the difficulty of performing classification on a given dataset.
Similar to our survey, the authors show that high dimensionality and small separation between classes hinder standard generalization.
However, the relationship of some of the metrics reviewed by these authors, e.g.,
%faction of borderline points (i.e., a measure for the complexity of the required decision boundary) and
%the fraction of hyperspheres covering data (i.e.,
the number of non-intersecting spheres needed to enclose all data points of a class,
to robust generalization is not studied, according to our survey.

%Moreover, the effect of XXX on standard generalization needs future investigation as well (that is if we found something they do not have).

%Knowing the characteristics of a dataset according to these perspectives can assist researchers and practitioners to select optimal learning algorithms~\cite{Ho:Basu:TPAMI:2002}.

He and Garcia~\cite{He:Garcia:TKDE:2009} focus on the imbalance learning problem. %~--
%the disproportion in the number of samples belonging to each class in a given dataset.
The authors found that most standard algorithms %are designed with the assumption of a balanced class distribution.
%These algorithms
fail to reliably represent the characteristics of the imbalanced data and result in unfavorable performance across classes.
Furthermore, L\'{o}pez et al.~\cite{Lopez:Fernandez:Garcia:Palade:Herrera:InfSci:2013} discuss six intrinsic data characteristics that potentially complicate learning from imbalanced data:
low density, sample overlap between classes, noisy data, borderline instances,
dataset shift between training and testing distributions, and
small disjuncts, i.e., disperse small clusters of samples from a single class.
Their analysis concludes that while all these ``unfavorable'' data characteristics further complicate the data imbalance
issues, data overlap between classes is probably one of the most harmful.
To follow up on this point, Santos et al.~\cite{Santos:Henriques:Pedro:Japkowicz:Fernandez:Soares:Wilk:Santos:AIR:2022}
focus on the joint effect of data imbalance and class overlap on model generalization.
The negative impact of data imbalance, low separation, and noisy data on robust generalization was also discussed in our survey.
Yet, the compounding effect of these factors, as well as the effect of other properties,
on robust generalization needs future investigation.

Recently, Yang et al.~\cite{Yang:Jiang:Song:Guo:IJCV:2022} summarized relevant studies focusing on
long-tailed distributions in the field of Computer Vision.
% and categorize the main methods for alleviating the issues caused by long-tailed distribution.
%They present quantitative metrics for measuring data imbalance and .
This survey also includes work on the influence of long-tail distributions on a model's adversarial robustness~\cite{Wu:Liu:Huang:Wang:Lin:CVPR:2021}, which is covered in our survey.
%which is included in our survey,
The authors advocate for more research on adapting long-tailed-based approaches for standard generalization to improve robust generalization.

Finally, Moreno-Torres et al.~\cite{MorenoTorres:Raeder:Rodrigues:Chawla:Herrera:PR:2012} present a unifying framework to categorize existing definitions of dataset shift~-- the case where the joint distribution of inputs and outputs differs between training and testing data.
While ML models are normally trained under the premise that testing data has a similar distribution to the training data,
in reality, the observed data distribution may be different from the historical data that the model is trained on.
Such difference can substantially compromise the quality of model predictions.
The authors analyze the possible causes for dataset shift, e.g., malicious software that evolves over time, and
review the techniques dealing with dataset shift.
They characterize adversarial attacks as one form of dataset shift, where adversaries adaptively
change test instances to create a distribution that differs from training data.
%All works discussed in our survey assumed similar distribution on training and testing data, treating adversarial attacks as the only dataset shift in the problem setup.
%However, in real applications, the underlying data distribution itself can be non-stationary, and the characterize the influence of the dataset shift between training and testing data on the adversarial robustness is yet to be investigated.

\revadd{Overall, despite the similarities with our work, literature discussed in this section focuses on standard generalization while our survey discusses 
the effect of data on robust generalization.}

%More works use the connection between adversarial attacks and distributional shift to analyze the effect of adversaries on generalization performance~\cite{Tu:Zhang:Tao:NeurIPS:2019}.
%However, we do not discuss them in detail, as they focus more on models instead of data.
%\jr{How is that relevant to data properties section?} \gx{This can be removed, as it an individual work we filtered}

\vspace{-0.1in}
\subsection{Summary}
\revadd{
Our survey is the first to explicitly focus on properties of training data in the context of model robustness under evasion attacks.
Numerous other surveys on evasion attacks discuss attack and defense mechanisms, non-data-related reasons for adversarial vulnerability, and the different threat models. 
We identified only five surveys that considered data-related reasons for evasion attacks. 
However, as these surveys are older and do not focus on data in particular, our work provides a more extensive
and comprehensive view on this topic. 
By including more than 50 papers not covered in prior work, we were able to 
identify additional relevant properties, practical suggestions, and future research directions in this area. 

Additional work studies non-data-related reasons for evasion attacks, as well as non-evasion attacks, 
such as poisoning and backdoor. 
Yet another body of literature examines how data properties affect standard generalization. These works show that 
some of the properties discussed in our survey, such as 
the number of samples, dimensionality, and label quality, also affect clean accuracy. 
There are also additional data properties that are covered exclusively by these or by our work. 
Studying the interplay between data properties for clean and robust accuracy is an interesting research direction, 
which could be facilitated by our work. 
However, all these current works are orthogonal and complementary to ours.
}

%\ad{
%The related work of our survey can be categorized into four key topics: 
%The first topic examines data for other adversarial attacks, this include the research that investigates the link between the data characteristics and model's resilience against poisoning attacks as well as the studies that explore data poisoning and backdoor attacks and their countermeasures. \jr{same issues as before: this is meta-summary, we need a concrete summary.}
%These studies complement our survey as they highlight the threats directly aimed at data, thus emphasizing the importance of secure data collection. 
%The second topic focuses on the relationship between various properties of training data and model's standard generalization ability. 
%This body of work suggests that data traits such as number of samples, dimensionality, label quality also influence model's ability to generalize in standard classification. \jr{this looks more concrete!}
%
%The third strand of research concerns adversarial evasion attacks. 
%The work in this area encompasses the research frontier in evasion attacks and the countermeasures. 
%Due to the large volume of work in this area, there are numerous surveys that gives more detail on the advancement. 
%\jr{meta-summary again}
%In addition to attacks and defenses, one relevant line of work investigates the alignment of the conventional similarity metrics used for adversarial examples and human perception, showing the need for supplementary metrics. \jr{why important?}
%These studies \jr{which "these studies"?} collectively present an extensive overview of other types of work conducted on adversarial robustness.
%The last category of work proposes alternative explanations for model vulnerability to adversarial examples.
%These studies presented hypothesis showing the characteristics of machine learning models, e.g., nonlinearity, invariance to rotational shift etc, induces susceptibility to attacks, as well as limited computational resources and non-robust feature representations. \jr{all text based on previous related work looks somewhat concrete; the new additions should be at least at the same level, or better.}
%These studies supplement our work, offering a broader perspective of potential factors affecting model's robust generalization ability. }
%



\section{Automatic Hate Speech Detection with Context Experts}
\section{Method}
\label{s:method}

We consider the 3D euclidean space $\Real^3$ with points $p=(x,y,z)\in\Real^3$. We discretize the unit cube $\gC=[0,1]^3$ with a 3D voxel grid $\gG=\set{p_I}$, with nodes $p_I$ indexed by $I=(i,j,k)$, $i,j,k\in [n]=\set{1,\ldots,n}$, \ie, $p_I=(x_{ijk},y_{ijk},z_{ijk})$. We denote by $h=n^{-1}$, and by $N=n^3$ the total number of nodes.   
We represent our reconstructed surface as a zero level of a scalar function $f$ defined over the cube $\gC$. $f$ is defined by prescribing its values at the grid's nodes $f_I\in\Real$ and trilinear interpolating in each voxel. We will denote by $f(p)$ the interpolated value at point $p$. 

Given an input point cloud consisting of $m$ points $q_k\in\Real^3$ with or without (unit norm) normals $n_k\in \Real^3$, $k\in [m]$, our goal is to compute $f$ so that its zero level set approximates the unknown surface, \ie, 
\begin{equation}
    \gS = \set{p\in\gC \ \vert \ f(p)=0}.
\end{equation}
Our approach to compute $f$ is to minimize a loss function of the form
\begin{equation}
    \gL = \gL_{\text{data}} + \gL_{\text{prior}}
\end{equation}
where 
\begin{equation}\label{e:loss_data}
    \gL_{\text{data}} = \frac{\lambda_{\text{p}}}{m}\sum_{k=1}^m \abs{f(q_k)}^2 + \frac{\lambda_{\text{n}}}{m}\sum_{k=1}^m \norm{\nabla f(q_k) - n_k}^2
\end{equation}
where $\norm{\cdot}$ is the standard euclidean norm in $\Real^3$, $\nabla f(p) \in \Real^3$ is the gradient of $f$ sampled at point $p$. Note that $\nabla f$ is defined in interior of voxels, which is generically where the input points $q_k$ resides. $\gL_{\text{data}}$ is the standard data loss encouraging the zero level to pass through the input points $q_k$, and its normals (defined by gradients of $f$) to coincide with input normals $n_k$. 

The prior, $\gL_{\text{prior}}$, is the main contribution of this work, where we combine two novel losses,
\begin{equation}
    \gL_{\text{prior}} = \lambda_{\text{v}} \gL_{\text{viscosity}} + \lambda_{\text{c}} \gL_{\text{coarea}}
\end{equation}
Intuitively, the viscosity loss optimizes for a smooth Signed Distance Function (SDF) solutions, avoiding auxiliary bad minima of the Eikonal equation, while the coarea loss strives to minimize the area of the zero level surface. Our loss has $4$ hyper-parameters $\lambda_{\text{p}},\lambda_{\text{n}},\lambda_{\text{v}},\lambda_{\text{c}}$. We provide more details on these priors next. 


\subsection{Viscosity Loss}\label{ss:viscosity_loss}
The goal of the viscosity loss is to make $f$ approximate an SDF over $\gC$. Given boundary conditions asking $f$ to vanish on some closed compact surface $\gS$, the SDF solves the Eikonal equation PDE, \ie, $\norm{\nabla f(p)}=1$, in a certain well defined sense (viscosity). This motivated some previous work to directly optimize the Eikonal loss \citep{gropp2020implicit,sitzmann2020implicit}
\begin{equation}\label{e:loss_eikonal}
    \gL_{\text{eikonal}} = \int_\gC \Big (\norm{\nabla f(p)}-1\Big )^2 dp
\end{equation}
\begin{wrapfigure}[14]{r}{0.28\textwidth}\vspace{-15pt}
  \begin{center}
    \includegraphics[width=0.25\textwidth]{figs/illustrations/eikonl_1d.png}
  \end{center}
  \caption{Two global minimizers of the Eikonal loss over a grid in 1D. Top solution is not an SDF. }\label{fig:eikonal_1d}
\end{wrapfigure}
Unfortunately, the Eikonal loss has many undesirable minima which are not SDFs. Figure \ref{fig:eikonal_1d} shows a 1D example: both depicted solutions (denoted $f$) vanish at the input points $q_1,q_2$ (black points) and globally minimize the Eikonal loss over the grid (grid points are shown in blue). The INR works mentioned above use neural networks for representing $f$ which injects an inductive bias avoiding these bad minima, however on grids, minimizing \eqref{e:loss_eikonal} cannot avoid these solutions. See, \eg, middle column in Figure \ref{fig:teaser}. 

More classical Eikonal solvers do work with grids however use mostly fast marching or sweeping methods \citep{osher1988fronts,sethian1996fast,zhao2005fast,chacon2012fast}. Namely, use a special discretization of the Eikonal equation favoring the viscosity  solution of the Eikonal \cite{rouy1992viscosity}, and update node values according to a moving front \cite{sethian1996fast}. Since this discretization is up-wind (will only propagate values in one direction) and requires choosing the maximal among its solution, its success in adaptation to a loss is not clear. 

We use a different approach to build a loss favoring SDF solutions over grids motivated by the vanishing viscosity method \cite{crandall1983viscosity}. Namely, adding to the Eikonal PDE a small perturbation of the Laplacian of $f$ (denoted by $\Delta f$), \ie, $\norm{\nabla f(p)}-1 - \eps\Delta f(p)=0$, makes the PDE semi-linear elliptic \citep{calder2018lecture}, and hence with suitable boundary conditions it is uniquely solvable inside $\gS$ with a smooth solution, approaching the viscosity positive distance function to the boundary as $\eps\too 0$. Similarly, for $1-\norm{\nabla f(p)} - \eps \Delta f(p)=0$ the solution approaches the negative distance function inside the domain. 
Motivated by the vanishing viscosity principle we suggest the following viscosity loss:
\begin{equation}\label{e:loss_viscosity_eikonal}
\gL_{\text{viscosity}} = \int_\gC \Big((\norm{\nabla f (p)}-1)\mathrm{sign}(f(p)) - \eps \Delta f(p)\Big)^2 dp
\end{equation}
We discretize this loss over the grid $\gG$ by replacing the first order derivatives and second order derivatives with symmetric finite  differences, \ie,
\begin{align*}
    D_x f_I=D_x f_{i,j,k} = \frac{f_{i+1,j,k}-f_{i-1,j,k}}{2h}, \quad D^2_x f_I = D^2_x f_{i,j,k}=\frac{f_{i+1,j,k}-2f_{i,j,k}+f_{i-1,j,k}}{h^2}
\end{align*}
and similarly for $D_y$ and $D_z$. We use these discrete operators to approximate the gradient $\widehat{\nabla} f(p_I) = (D_x f_I, D_y f_I, D_z f_I)$ and Laplacian $\widehat{\Delta}f(p_I) = D_x^2f_I + D_y^2 f_I + D_z^2 f_I$. The discretized viscosity loss now takes the form
\begin{equation}
    \widehat{\gL}_{\text{viscosity}} = \frac{1}{N}\sum_{I} \Big((\|\widehat{\nabla} f (p_I)\|-1)\mathrm{sign}(f(p_I)) - \eps \widehat{\Delta} f(p_I)\Big)^2
\end{equation}



\subsection{Coarea loss}\label{ss:coarea_loss}
The coarea loss is approximating the area of the zero level set, and therefore incorporating it in the optimization pushes the reconstructed surface to be economic in area. 

First, similarly to  \citep{yariv2021volume} we use the centered Laplace CDF
\begin{equation}
   \Psi\beta(s)= \begin{cases}
   \frac{1}{2}\exp\parr{\frac{s}{\beta}} & s\leq 0 \\ 1-\frac{1}{2}\exp\parr{-\frac{s}{\beta}} & s\geq  0
   \end{cases}
\end{equation} to transform the SDF $f$ to a smooth approximation of the indicator function:
\begin{equation}
    \chi_\beta(p)=\Psi\beta (-f(p))
\end{equation}
As $\beta\too 0$, $\chi_\beta$ converges to an indicator function leading to $1$ inside $\gS$ and $0$ outside. The coarea loss is defined as 
\begin{equation}
    \gL_{\text{coarea}} = \int_\gC \norm{\nabla \chi_\beta (p)} dp
\end{equation}
To understand why this loss approximates the area of $\gS$ we can use the coarea formula \citep{rindler2018calculus}:
\begin{equation}\label{e:coarea}
    \int_\gC \norm{\nabla \chi_\beta(p)}dp = \int_{-\infty}^{\infty} \mathrm{area}(\chi_\beta^{-1}(s))ds,
\end{equation}
where $\chi_\beta^{-1}(s)=\set{p\ \vert \ \chi_\beta(p)=s}$ is the preimage of the value $s$. Since $\chi_x(p)\in [0,1]$ the r.h.s.~integral can be restricted to the interval $[0,1]$, and therefore the coarea loss averages the area of the level sets of $\chi_\beta$. Next,  $$\chi_\beta^{-1}(s)= \set{p\ \vert \ \Psi\beta (-f(p)) = s } = \{p\ \vert \ f(p) = -\Psi\beta^{-1} (s) \} = f^{-1}(-\Psi\beta^{-1} (s)),$$
\begin{wrapfigure}[11]{r}{0.28\textwidth}\vspace{-20pt}
  \begin{center}
  \includegraphics[width=0.25\textwidth]{figs/semi.png}
  \end{center}
  \caption{Reconstruction of a semisphere point cloud (white dots) without (left) and with (right) coarea loss. }\label{fig:coarea_semisphere}
\end{wrapfigure}

which shows that the level set $s\in (0,1)$ of $\chi_\beta$ is the level set $-\Psi\beta^{-1}(s)$ of the SDF $f$. As $\beta\too 0$, $-\Psi\beta^{-1}(s)\too 0$ for all $s\in (0,1)$ (and uniformly in $(\eps,1-\eps)$ for fixed $\eps>0$). Therefore the average of the level set area (\ie, the r.h.s.~of \eqref{e:coarea}) converges to the area of $f^{-1}(0)=\gS$. Figure \ref{fig:teaser} (right) shows how removing the coarea loss introduces an extraneous zero level set, and hence results in an undesired surface part. Figure \ref{fig:coarea_semisphere} shows a comparison of a reconstruction of semisphere with and without coarea. In the experiments section we provide more ablation tests with the coarea and viscosity losses.

To discretize the coarea loss we let $w_I$ denote the centers of grid's voxels, and note that $\nabla \chi_\beta(w_I) = \Phi_\beta(-f(w_I))\nabla f(w_I)$, where 
\begin{equation*}
    \Phi_\beta(s) = \frac{1}{2\beta}\exp\parr{\frac{\abs{s}}{\beta}}
\end{equation*}
is the PDF of the Laplace distribution, and $\nabla f(w_I)$ is computed as a linear combination of the voxel's corner values $f_{I_1},\ldots,f_{I_8}$, see more details in the Appendix. We end up with the discretized loss:
\begin{equation}
    \widehat{\gL}_{\text{coarea}} = \frac{1}{N}\sum_{I}\Phi_\beta(-f(w_I))\norm{\nabla f(w_I)}
\end{equation}
This loss is usually incorporated with a small hyper-parameter $\lambda_{\text{c}}$ with the purpose of eliminating redundant surface parts.



\section{Study findings \& Discussion}
In this section, we discuss our findings from the collaborative model with context experts, which we have described in the previous section. While we drew most of these observations from our process in Myanmar, they demonstrate similar issues in other low-resource language settings where machine learning may be applied. An appreciation of these findings is crucial for designing efficient machine learning systems. We have identified three central issues: (i) exploring avenues to boost ground truth data, (ii) incentivizing data work for machine learning, and (iii) engendering open data-sharing practices.

\subsection{Exploring avenues to boost ground truth data}
A major challenge with hate speech detection tasks in low-resource language settings is limited training data in target languages. This limitation is often due to the rarity of hate speech in proportion to the amount of available social media data and the infeasibility of labeling the entire dataset on a given platform~\cite{madukwe2020data}. Much of the progress in the field of machine learning has been driven by the availability of benchmark datasets such as ImageNet \cite{deng2009imagenet} for computer vision, and GLUE \cite{wang2018glue} for natural language processing. While such datasets have sprung up for hate speech detection in the English language \cite{waseem2016hateful, mathew2020hatexplain}, researchers often filter out data not in English, leaving more work to do for very low-resource languages. This lack of training data can hamper machine learning research within these contexts. The ML4D community can adopt the process described in this work to bootstrap the development of hate speech training datasets in diverse contexts.

A caveat to note is that engaging in elaborate data collection and labeling efforts do not always guarantee significant outcomes. We found that only less than 4\% of the entire Facebook posts in our dataset were labeled by annotators as hate speech. To boost training data, practitioners have relied on curated hate speech lexicons in the target languages. One example is the PeaceTech Lab Lexicons, a series of hate speech terms explaining inflammatory social media keywords and offering counter-speech suggestions to combat the spread of hate speech~\cite{peacetechlab}. The PeaceTech Lab has curated hate lexicons for languages in conflict-affected countries such as the Democratic Republic of the Congo, Sudan, and Lebanon. Though keyword approaches are ineffective when used alone, they can help researchers select a subset of data to work with. Organizing civil society workshops or other avenues for crowdsourcing hate terms can be beneficial for supporting hate speech detection work.

Our findings reveal how quickly new hate terms emerge on social media. This dynamism is primarily due to changes in political and social concerns. For instance, online actors may derive new hate terms as social media discourse changes from talking about a local election to focusing on a pandemic~\cite{ziems2020racism}. Context experts can help identify when social media topics drift or when users find creative ways to guise hate speech. 

\subsection{Incentivizing data work for machine learning}
Data work is ``any human activity related to creating, collecting, managing, curating, analyzing, interpreting, and communicating data''~\cite{bossen2019data}. This fundamental component of machine learning is often undervalued in research and practice~\cite{wagstaff2012machine} and can lead to negative outcomes because data is critical for effective machine learning systems~\cite{polyzotis2018data}. Sambasivan et al.~\cite{sambasivan2021everyone} discuss the challenges with data cascades in high-stakes AI, which they define as compounding events over time that result from the undervaluing of data quality by researchers and practitioners. These data cascades often have significant effects in low-resource language settings due to the lack of existing ground data and supporting infrastructure for data collection and processing. We identified some of the cascade factors identified by Sambasivan et al.~\cite{sambasivan2021everyone} in our work, namely, incentives in AI and data education. First, we observed that while stakeholders and partners might identify with and value the role of AI in addressing the problem of low-resource hate speech, they often did not place similar consideration on the invisible and challenging data work. Second, context experts and partners lack experience in creating the kinds of complex and quality datasets that hate speech research requires. Stakeholders ought to see the value in data work to appreciate the time spent on training context experts.

At the onset of communications with sponsors and partners, it is crucial to emphasize the importance of data work for machine learning and the vital role that context experts will play. This step may imply a more costly engagement with context experts, both in time and financial compensation. We learned that a comprehensive data training plan is helpful at the start of the project to address the data education challenge. 

\subsection{Supporting open data sharing practices}
Social media researchers in low-resource contexts do not often have the freedom to choose what platforms to work on. This decision is mainly driven by which platform is widely used and most likely to have relevant data within a given context. With over 21 million active users in 2019~\cite{kemp19}, our work in Myanmar primarily focused on Facebook. Like any other platform, Facebook offers unique affordances and constraints for data access and sharing. For example, the company grants privileged data access to researchers via its CrowdTangle API service. Yet, its terms of use do not allow researchers to share any data with people outside CrowdTangle. Such practices make it challenging to freely collaborate with other teams who may also have access to data but are bound by the platform's terms.

These limitations posed by platforms motivate community-based data-sharing workflows led by context experts and built on trust, ethics, and shared values. Data sharing has been discussed as an effective means for scientific progress, especially within developing countries \cite{chawinga2019global, sebake2012assessing}. However, there are lots of structural, organizational, cultural, and ethical complexities that undermine data sharing~\cite{anane2018you}. Exploring data-sharing practices that actively address power imbalances, understands potential benefits and risks, and comply with local norms and cultures~\cite{abebe2021narratives} will help produce a critical mass of relevant data that can be useful for tackling low-resource hate speech.

Furthermore, significant aspects of a country's history are often undocumented digitally, only existing as paper documents within locked-out cabinets or as informal knowledge passed on from one generation to another. In some cases, only a fraction of the context experts know the historical and cultural issues that underlie hateful speech posted online. Our annotation model addresses this knowledge imbalance with the peer feedback model discussed as part of our process. 

\section{Conclusion}
\section{Conclusion}\label{sec:conclusion}
In this work, we focus on addressing the fundamental challenge of OOD detection tasks, which is how to fully understand the semantic discrepancy between the ID/OOD samples. We reveal that the key to success in the realistic SCOOD task is to allocate as many ID samples in the unlabeled set correctly as possible. To this end, we propose a novel uncertainty-aware optimal transport scheme that introduces class-specific energy scores as guidance for effective label assignment. Experimental results show that our method achieves better performance than previous state-of-the-art methods on SCOOD benchmarks.

\textbf{Limitations.} In addition to temperature scaling, other techniques such as feature clipping applied in ReAct~\cite{sun2021react} also enhance the performance of energy score, so how to obtain an OOD score that best fits the SCOOD task can be further explored. Moreover, a setting highly related to SCOOD has been proposed in \cite{katz2022training} and formulated as a constrained optimization problem. We will also theoretically analyze these practical OOD settings in our feature work.

% \section*{Acknowledgments}
\textbf{Acknowledgments.} 
This work is supported by National Key R\&D Program of China under Grant 2020AAA0105701, National Natural Science Foundation of China (NSFC) under Grants 61872327, Major Special Science and Technology Project of Anhui, National Natural Science Foundation of China (62033012) and Ant Group through Ant Research Intern Program.


\section{Future Work}

\section{Conclusion and Future Work}
\label{sec:future}
% Whenever a new solution concept is defined, one of the most intuitive questions is how it can be related to other properties? 
We presented a practical solution to the problem of leximin optimization when only an approximate single-objective solver is available. 
The algorithm is guaranteed to terminate in polynomial time, and its approximation ratio degrades gracefully as a function of the approximation ratio of the single-objective solver.

Currently, our algorithm handles two main settings. First, when inaccuracies in the single-objective solver stem from numeric errors.
Second, when the problem is convex and satisfy several assumptions.
It may be interesting to study more settings in which the inaccuracies stem from computational hardness of the single-objective problem.
% Currently, our algorithm handles settings in which the inaccuracies in the single-objective solver stem from numeric errors.
% It may be interesting to study settings in which the inaccuracies stem from computational hardness of the single-objective problem.
%
%

% identify problems in which an appropriate approximate solver can be designed. 
In particular, to approximate the egalitarian welfare, it is common to model the problem as an integer program or as an exponential sized linear program (e.g., \cite{bansal2006santa, kawase_max-min_2020}) and then approximate the program using different techniques.
% rounding techniques or methods for convex optimization (such as the ellipsoid method).
Can these algorithms be generalized to consider the additional constraints described in Section \ref{sec:algo-short}? This will allow approximating leximin using the approach in this paper.
% In particular, in the problem of stochastic allocations (in Section \ref{sec:app}), to extend the approximation algorithm for the egalitarian welfare, we had to change some steps within.
% What if an algorithm for egalitarian welfare is provided as a black box --- could it be used to design the appropriate procedure to approximate leximin?

% In the context of fair division, this study assumes that there is an access to the true valuations of the agents involved. 
% In reality, people may lie about their valuations.
% Can our definition of approximate-leximin be related to some approximate version of truthfulness?

Another question is whether it is possible to obtain a better approximation factor for leximin, given an $(\multApprox, \additiveApprox)$-approximation algorithm for the single-objective problem.
Specifically, can an $(\multApprox, \additiveApprox)$-approximation to leximin can be obtained in polynomial time? 
If not, what would be the best possible approximation in this case?
% \erel{Mention the tightness of our results}


\iffalse % EREL: removed for the submission. To clarify later
Further, the algorithm suggested in Section \ref{sec:algo-short} tend to work very well if the single-objective optimization problems are convex, and in particular if they are linear programs. 
\erel{Why? the algorithm of \textcite{Ogryczak2004TelecommunicationsND} works even for non-convex programs.}
Can we find an algorithm that works with general approximation algorithms? For example, naive algorithms such as \emph{Next-fit}?
\fi

% \begin{itemize}
%     \item Meaning of approximately leximin in the context of other characteristics like truthfulness.
    
%     \item Solving more problems.
    
%     \item Is it possible to obtained a better approximation factor for leximin maximization in polynomial time? given that $(1-\beta)$ is the best possible for the egalitarian maximization, is it possible to obtain a $(1-\beta)$ approximately leximin optimal solution? what is the best possible approximation  in this case?
    
%     \item The algorithm works very well if the single-objective problem is convex, in particular if it is a linear programs, but can it be modify to work with general approximation algorithms? such as algorithms for makespan minimization?
% \end{itemize}

\begin{acks}
Special thanks to our partners at The Carter Center for collaborating with us in Myanmar and establishing our partnership with the New Myanmar Foundation (NMF); to the NMF for working with us and serving as context experts despite challenges with the COVID-19 pandemic; and to Koe Koe Tech for their review and feedback on an initial version of the hate speech annotation guideline.
\end{acks}

%%
%% The next two lines define the bibliography style to be used, and
%% the bibliography file.
\bibliographystyle{ACM-Reference-Format}
\bibliography{main}

%%
%% If your work has an appendix, this is the place to put it.
\appendix
\section{Appendix}
This Github link contains data on the Burmese hate lexicons (Hatebase.org and Phandeeyar), adopted protected characteristics, annotation guideline and plan: \url{https://github.com/TID-Lab/myanmarhsc}.

\end{document}
\endinput
%%
%% End of file `main.tex'.
