\section{Construction}

We carry out the proof of Theorem \ref{THM:A} in this section.

\subsection{Weakly transverse submanifolds}
Let $\F$ be a singular holomorphic foliation on a complex manifold $W$. We will say that
a submanifold $Z\subset W$ is weakly transverse to  $\F$ if
\begin{enumerate}
    \item the singular set of $\F$ does not intersect $Z$; and
    \item for every point $z \in Z$, the tangent space of $Z$ at $z$ intersects the tangent space of $\F$ at $z$
    only at $0$.
\end{enumerate}


The relevance of the concept to our discussion is put in evidence by our next result.


\begin{prop}\label{P:leaf space}
    Let $\F$ be a singular holomorphic foliation on a complex manifold $W$.
    If $Z \subset W$ is a compact submanifold weakly transverse to $\F$ then there exist
    a neighborhood $U$ of $Z$; a complex manifold $X$ of dimension equal to $\dim W- \dim \F$; and
    a holomorphic submersion $\pi : U \to X$ such that
    \begin{enumerate}
        \item the leaves of  $\restr{\F}{U}$ coincide with the fibers of $\pi$; and
        \item the morphism $\pi$ maps  $Z$ isomorphically to a submanifold $Y$ of $X$  with normal bundle isomorphic to
        the quotient of $N_{Z/W}$ by the image of $\restr{T_{\F}}{Z}$ inside it; and
        \item the field of germs of meromorphic functions $\mathbb C(X,Y)$ is mapped by $\pi^*$ isomorphically
        onto the field of germs of meromorphic first integrals of the germ of foliation $\restr{\F}{(W,Z)}$.
    \end{enumerate}
\end{prop}
\begin{proof}
    Let $U$  be equal to a sufficiently small tubular neighborhood of $Z$ in $W$.  The weakly transversality between $\F$ and $Z$ implies that the leaves of $\restr{\F}{U}$ are closed subvarieties of $U$. Therefore the leaf space $X=U/\F$ of $\restr{\F}{U}$ is a Hausdorff complex manifold. The quotient morphism $\pi: U \to X$ has the sought properties.
\end{proof}

\subsection{Formal meromorphic functions versus rational first integrals} The field of meromorphic functions
of the leaf space constructed in Proposition \ref{P:leaf space} is described by the lemma below.

\begin{lemma}\label{L:field of functions}
    Notation and the assumptions as in Proposition \ref{P:leaf space}. Further assume that $W = \mathbb P^m$. If $Y(\infty) \subset X$ is the formal completion of $Y= \pi(Z)$ in $X$ then $\mathbb C(Y(\infty))$ is isomorphic to the field of rational first integrals of the foliation $\F$.
\end{lemma}
\begin{proof}
    Direct consequence of Hironaka-Matsumura result discussed in Subsection \ref{SS:field}. Indeed, the pull-back of $\mathbb C(Y(\infty))$
    under $\pi$ is the subfield of $\mathbb C(Z(\infty))$ formed by the formal meromorphic first integrals of $\restr{\F}{Z(\infty)}$. By Hartshorne result, $\mathbb C(Z(\infty))$ coincides with the field  $\mathbb C(W) = \mathbb C(\mathbb P^m)$ the field of rational functions on $W$. Consequently, we get an isomorphism between $\mathbb C(Y(\infty))$ and the field of rational first integrals of $\F$.
\end{proof}

\begin{remark}\label{R:super}
    Likewise, when $\F$ is a global foliation on $\mathbb P^n$,  foliations/webs living in the leaf space of $\restr{\F}{U}$ correspond to  global foliations/global webs invariant by $\F$. This again is a consequence of Hartshorne's result.
\end{remark}

The situation for subvarieties of the leaf space is slightly different, and it is not true that their pre-iamges under the quotient morphism  can always be globalized.

\begin{lemma}\label{L:subvarieties}
    Notation and the assumptions as in Proposition \ref{P:leaf space}. Further assume that $W = \mathbb P^m$. If $S \subset X = U/\F$ is a subvariety intersecting $Y= \pi(Z)$ such that $\dim S + \dim Y > \dim X$ then $\pi^{-1}(S)$ is contained in a $\F$-invariant subvariety $\overline S$ of $\mathbb P^m$ with $\dim \overline S = \dim S + \dim \F$.
\end{lemma}
\begin{proof}
    Direct consequence of  Theorem \ref{T:Rossi}.
\end{proof}


\subsection{Existence of weakly transverse submanifolds} As shown below, the existence of weakly transverse submanifolds, under suitable numerical assumptions, 
is easy consequence of Kleiman's transversality of a general translate.

\begin{lemma}\label{L:existence}
    Let $\F$ be a foliation by curves on $\mathbb P^{m+1}$ with isolated singularities.
    Let $Y \subset \mathbb P^{m+1}$ be a projective submanifold of dimension $n$.
    If $m\ge 2n$ then $Y$ is weakly transverse to $g^* \F$ for any general $g \in \Aut(\mathbb P^{m+1})$.
\end{lemma}
\begin{proof}
    The proof is a simple application of Kleiman's transversality of a general translate, \cite[Theorem 2]{MR360616}.
    Indeed, we can identify the projectivization of the tangent bundle of $Y$, $\mathbb P(T_Y)$ (space of lines on $T_Y$),
    with a submanifold $\tau(Y)$ of $\mathbb P T_{\mathbb P^{m+1}}$ of dimension $2n-1$. Likewise, we consider $\tau(\F) \subset \mathbb P(T_{\mathbb P^m+1})$ as the Zariski closure of the tangent lines of $\restr{\F}{\mathbb P^{m+1} - \sing(\F)}$. As such $\tau(\F) \subset \mathbb P (T_{\mathbb P^{m+1}})$ is a subvariety of dimension $m+1$. By assumption
    \[
        \dim \tau(Y) + \dim \tau(\F) < \dim \mathbb P(T_{\mathbb P^n}) \, .
    \]
    Since the natural action of $\Aut(\mathbb P^{m+1})$ on $\mathbb P (T_{\mathbb P^{m+1}})$ is transitive, we can apply
    \cite[Theorem 2]{MR360616} to guarantee that, for a general $g \in \Aut(\mathbb P^{m+1})$, $g^*\tau(\F)=\tau(g^*\F)$ is disjoint from $\tau(y)$. The lemma follows.
\end{proof}

\subsection{Field of rational first integrals of foliations on projective manifolds}

Let $\F$ be a foliation on a projective manifold $X$. According to \cite{MR2223484}, there exists a unique foliation by algebraic leaves $\G$ containing $\F$ and such that $\mathbb C(\F) = \mathbb C(\G)$.

\begin{ex}\label{Ex:linear}
    Let $\lambda_1, \ldots, \lambda_{m+1}$ be complex numbers and consider the following vector field on $\mathbb C^{m+1}$
    \[
        v = \sum_{i=1}^{m+1} \lambda_i x_i \frac{\partial}{\partial x_i} \, .
    \]
    The foliation $\F$ on $\mathbb P^{m+1}$ defined by $v$ has Zariski closure $\G$ of dimension equal to
    the dimension of $\mathbb Q$-vector subspace of $\mathbb C$ generated by $\lambda_1, \ldots ,\lambda_{m+1}$.
    In particular, choosing appropriately the complex numbers $\lambda$ we have one dimensional foliations on $\mathbb P^{m+1}$
    having field of rational first integrals of any transcendence degree between $0$ and $m$.
\end{ex}



\subsection{Proof of Theorem \ref{THM:A}}
Let $\F$ be a one-dimensional foliation $\mathbb P^{m+1}$ with field of rational first integrals of transcendence degree $\ell$ over $\C$. For instance, we can take a foliation $\F$ as in Example \ref{Ex:linear}.
Embed $Y$ in $\mathbb P^{m+1}$. According to Lemma \ref{L:existence} we can assume that $Y$ is weakly transverse to $\F$. Proposition \ref{P:leaf space} implies the existence of an Euclidean open subset $U \subset \mathbb P^{m+1}$ containing $Y$ such that the leaf space $X = U / \restr{\F}{U}$
has $\C(X)$ equal to $\C(\restr{\F}{U})$. Lemma \ref{L:field of functions} implies that $\C(X)$ equals to $\C(\F)$. This shows that the leaf space $X = U / \restr{\F}{U}$ is a complex manifold of dimension $m$ containing $Y$ and with $\trdeg \mathbb C(X,Y) = \ell$ as claimed.

The last claim concerning the existence of manifolds with $\ell =0$ follows from Remark \ref{R:super} and Lemma \ref{L:subvarieties} combined with \cite[Theorem 1]{MR2862041} which guarantees that the very general one-dimensional foliation on $\mathbb P^{m+1}$ of degree at least two is not tangent to any other foliation or web and does not leave invariant any algebraic subvariety.
\qed







