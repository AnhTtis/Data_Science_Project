\section{Algebraicity}\label{S:Van de Ven}

In this section we will present a proof of Theorem \ref{T:Van de Ven} from the Introduction.
Actually Theorem \ref{T:Van de Ven} is the combination of Theorems \ref{T:transverse} and \ref{T:weakly transverse}
below.

\subsection{Submanifolds transverse to foliations} We start by treating submanifolds transverse to foliations. 

\begin{thm}\label{T:transverse}
    Let $\F$ be a codimension $q$ foliation on a projective space $\mathbb P^n$.  If $Z \subset \mathbb P^n$ is a submanifold
    disjoint from $\sing(\F)$ and transverse to $\F$ then $\dim Z = q$,  $Z$ is a linear subspace, and the foliation
    $\F$ has degree zero.
\end{thm}
\begin{proof}
    Let $i: Z \to \mathbb P^n$ be the natural inclusion and let $\omega \in H^0(\mathbb P^n, \Omega^q_{\mathbb P^n} \otimes \det N_{\F})$  be a $q$-form
    defining $\F$. Observe that $\dim Z \ge q = \codim \F$ by the definition of transversality.
    Moreover, $i^* \omega \in H^0(Z, \Omega^q_{Z} \otimes \restr{\det N_{\F}}{Z})$ is an everywhere non-zero
    twisted $q$-form on $Z$. If $\dim Z \ge q+1$ then $i^*\omega$ defines a smooth foliation on $Z$ of positive dimension. Bott's vanishing
    theorem implies that $(\restr{\det N_{\F}}{Z})^{q+1} = 0$. But $\restr{\det N_{\F}}{Z}$ is an ample line-bundle and  $q+1\le \dim Z$, hence
    $(\restr{\det N_{\F}}{Z})^{q+1} \neq 0$. We obtain in this way  a contradiction that establishes that $\dim Z=q$.
    Therefore the non-vanishing of $i^*\omega$ implies that the canonical sheaf $\Omega^q_Z$ is isomorphic to $\restr{\det N^*_{\F}}{Z} = \mathcal O_Z(- \deg(\F) - q - 1)$.
    Kobayashi-Ochiai Theorem \cite[Corollary of Theorem 1.1]{MR316745} implies that $Z = \mathbb P^{q}$ , $\deg(\F) =0$ and $\deg (Z) = 1$.
\end{proof}


\begin{remark}
    Let $\F$ be a codimension $q$ foliation on a projective manifold $X$. If $Z \subset X$ is a submanifold with ample normal bundle, disjoint from $\sing(\F)$, and transverse to $\F$ then proof of Theorem \ref{T:transverse} shows that $\dim Z=q$. Hence $\restr{T_{\F}}{Z} \simeq N_Z$ is ample and we can apply  \cite[Theorem 3.5]{MR1863738} or \cite{MR3644242} to guarantee that every leaf of $\F$ passing through the points of $Z$ is algebraic. Consequently, by dimension reasons, every leaf of $\F$ is algebraic and $\F$ is algebraically integrable.
\end{remark}


We present below a proof of a classical result by Van de Ven.

\begin{cor}
    Let $X$ be a submanifold of $\mathbb P^n$. If the normal sequence
    \[
        0 \to T_X \to \restr{T_{\mathbb P^n}}{X} \to N_X \to 0
    \]
    splits then $X$ is a linear subspace of $\mathbb P^n$.
\end{cor}
\begin{proof}
    Let $q$ be the codimension of $X$.
    The existence of a splitting $\varphi: N_X \to \restr{T_{\mathbb P^n}}{X}$ gives us for every $x \in X$, a unique linear $\mathbb P^q$
    through $x$ with tangent space at $x$ equal to $\varphi(N_X(x))$. If $U$ is a sufficiently nieghborhood of $X$ then this family of $\mathbb P^q$'s will not intersect at $U$, and will thus define a codimension $q$ foliation $\F_U$ on $U$. Theorem \ref{T:Hartshorne} implies that $\F_U$ extends to a foliation $\F$ on $\mathbb P^n$. By construction $\F$ is transverse to $X$. Theorem \ref{T:transverse} implies the result.
\end{proof}

\subsection{Submanifolds weakly trasnverse to foliations} Our last result treats the case of submanifolds weakly transverse to foliations. 

\begin{thm}\label{T:weakly transverse}
    Let $\F$ be a codimension $q$ foliation on $\mathbb P^n$. If $Z$ is a submanifold
    weakly transverse to $\F$ and $q <2\dim Z$ then $\F$ is algebraically integrable. Moreover, if $\dim \F =1$ then $\deg(\F)=0$.
\end{thm}
\begin{proof}
    Let $U \subset \mathbb P^n$ be a sufficiently small Euclidean neighborhood of $Z$.
    Let $\pi : U \to U/\F$ be the quotient of $U$ by $\restr{\F}{U}$.  Observe that $V=\pi^{-1}(\pi(Z))$ is a
    closed subvariety of $U$ of dimension
    \begin{equation} \label{E:dimensao V}
        \dim \F + \dim Z.
    \end{equation}
    Hence our assumptions imply that $\dim Z + \dim V = 2 \dim Z + \dim \F > n$. Theorem \ref{T:Rossi} implies that $\overline{V}$, the Zariski closure of $V$, and $V$ have  the same dimension. Since $V$ is invariant by $\restr{\F}{U}$, $\overline{V}$ is also invariant by $\F$.


    Let $\G$ be the unique foliation by algebraic leaves containing $\F$ and with $\mathbb C(\F) = \mathbb C(\G)$ given by \cite{MR2223484}.
    After replacing $Z$ by $gZ$ for a sufficiently general automorphism $g \in \Aut(\mathbb P^n)$, we can assume that the Zariski closure of a  leaf of $\F$ through a general point $z$ of $Z$ coincides  with a leaf of $\G$. Therefore
    \begin{equation}\label{E:dimensao V barra}
        \dim \overline{V} = \dim \G + \dim Z.
    \end{equation}
    Since $\dim V = \dim \overline{V}$, we can combine  Equations (\ref{E:dimensao V}) and (\ref{E:dimensao V barra}) to establish the algebraicity of the leaves of $\F$, in other terms $\F = \G$.

    In the one dimensional case, weakly transversality implies the existence of a nowhere zero section of $N_Z\otimes \omega_{\F}|_Z$, thus $c_{\mathrm{cod}(Z)}(N_Z\otimes \omega_{\F})=0$.

    Since $N_Z$ is an ample vector bundle, and the Chern classes of ample vector bundles are strictly positive according to \cite[Theorem 2.5]{MR297773}, we deduce that
    $\omega_{\F}$ is not nef. Since $\omega_{\F}|_Z=\mathcal{O}_Z(\deg(\F) - 1)$ we necessarily have $\deg(\F)=0$ as claimed.
\end{proof} 


\begin{remark}
    We believe that a similar statement should hold true if $\mathbb P^n$ is replaced by an arbitrary projective manifold and $Z$ is replaced by a submanifold with ample normal bundle.  
\end{remark}
