\section{Extension}


\subsection{Conventions}
Although we are mainly interested on submanifolds of complex manifolds, we will need to
consider the more general notion of complex analytic spaces in the sense of Grauert,
see for instance \cite[Appendix B]{MR0463157}. To wit,  subvarieties are closed complex analytic
subspaces without nilpotents in their structural sheaves, and submanifolds are smooth subvarieties.





\subsection{Infinitesimal neighborhoods}
Let $X$ be a complex manifold and let $Y\subset X$ be a closed and connected complex analytic subspace.
If $\mathcal I$ is the defining ideal sheaf of $Y$ then $n$-th infinitesimal neighborhood
of $Y$ is the complex analytic space $Y(n)$ with underlying topological space equal to $Y$ and structural sheaf equal to
\[
    \frac{\mathcal O_X}{\mathcal I^{n+1}} \, .
\]

There are natural inclusions $Y(n) \hookrightarrow Y(n+1)$ between infinitesimal neighborhoods of $Y$ induced by the natural
restriction morphisms
\[
    \frac{\mathcal O_X}{\mathcal I^{n+2}} \to \frac{\mathcal O_X}{\mathcal I^{n+1}} \, .
\]
The formal completion of $X$ along $Y$, denoted here by $Y(\infty)$, is the formal analytic space
obtained as the direct limit of $Y(n)$ when $n$ goes to $\infty$, \ie
\[
    Y(\infty) = \varinjlim Y(n) \, .
\]
In other words, $Y(\infty)$ is the ringed space with underlying topological space equal to $Y$ and with structural sheaf equal to the inverse limit
\[
    \mathcal O_{Y(\infty)} = \varprojlim \frac{\mathcal O_X}{\mathcal I^{n+1}} = \varprojlim \mathcal O_{Y(n)} \, .
\]

\subsection{The field of formal meromorphic functions}\label{SS:field}
Following \cite{MR251043}, we define the sheaf $\mathcal M_{Y(\infty)}$ of (formal) meromorphic functions on $Y(\infty)$ as the
sheaf associated to the presheaf  with sections over an open subset $U\subset Y$ equal to the total ring of fractions of $\mathcal O_{Y(\infty)}(U)$. We will denote the global sections of $\mathcal M_{Y(\infty)}$ by $\mathbb C(Y(\infty))$.

As we are assuming that $Y$ is connected $\mathbb C(Y(\infty))$ is a field, see \cite[Proposition 9.2]{MR2103516}.  Moreover,
we have  injective restriction morphisms
\begin{equation}\label{E:inclusions}
    \mathbb C(X) \hookrightarrow \mathbb C(X,Y) \hookrightarrow \mathbb C(Y(\infty)).
\end{equation}
When $Y$ is a point on a $n$-dimensional projective manifold $X$, then $\mathbb C(X,Y)$ is the quotient field of the ring
of convergent power series $\C\{x_1, \ldots, x_n\}$  while $\mathbb C(Y(\infty))$ is the quotient field of the ring of formal power series $\C[[x_1, \ldots, x_n]]$. Therefore, in general, both inclusions can have infinite transcendence degree.

In sharp contrast, Hironaka and Matsumura \cite[Theorem 3.3]{MR251043} proved the following result for subvarieties of projective spaces.

\begin{thm}\label{T:HironakaMatsumura}
    Let $Y$ be a closed and connected complex analytic subspace of $\mathbb P^n$, $n \ge 2$. If $\dim Y \ge 1$ then $\C(Y(\infty)) = \C(X,Y) = \C(\mathbb P^n)$.
\end{thm}

For an analytic version of Theorem \ref{T:HironakaMatsumura}, establishing the equality $\C(X,Y) = \C(\mathbb P^n)$ through Hartog's theorem and without the use of formal geometry,  see \cite[Theorem 3]{MR244516}.

When $\dim Y>0$ and the normal bundle of $Y$ is ample,  we can rephrase a result by Hartshorne \cite[Theorem 6.7 and Corollary 6.8]{MR232780}
to our setting as follows.

\begin{thm}\label{T:Hartshorne}
    Let $Y$ be a connected and compact complex analytic subspace of a complex manifold $X$. If $Y$ is locally a complete intersection of dimension at least one and the normal bundle $N_{Y/X}$ is ample then the transcendence degree over $\C$ of $\C(Y(\infty))$ is bounded by the dimension of $X$. Moreover, if $\trdeg \C(Y(\infty)) = \dim X$ then $\C(Y(\infty))$ is a finitely generated extension of $\C$.
\end{thm}


Although Theorem \ref{T:Hartshorne} was originally stated in the algebraic category, Hartshorne's proof works, as it is, in the context of complex manifolds considered here.



When the ambient manifold $X$ is projective (or more generally Moishezon) and $Y$ has ample normal bundle, Theorem \ref{T:Hartshorne} implies that $\C(Y(\infty))$ is a finite algebraic extension of $\mathbb C(X)$. Consequently, in this case, both inclusions in (\ref{E:inclusions}) are finite algebraic extensions. This fact combined with the proposition below,  due to Badescu and Schneider (see  \cite[Proposition 3.5]{MR1954055} or \cite[Proposition 10.17]{MR2103516}), imply the convergence of formal meromorphic functions defined on formal neighborhoods of subvarieties with ample normal bundle on projective manifolds.

\begin{prop}\label{P:Badescu Schneider}
    Let $X$ be a projective manifold  and $Y \subset X$ a connected subvariety. Then the algebraic closure of $\C(X)$ inside of $\C(Y(\infty))$
    is contained in $\mathbb C(X,Y)$.
\end{prop}


\begin{cor}\label{C:converge}
    Let $X$ be a projective manifold  and $Y \subset X$ a connected subvariety with ample normal bundle. Then   $\C(X,Y)=\C(Y(\infty))$.
\end{cor}



\begin{question}\label{Q:converge?}
    If the ambient space is not projective, just a small euclidean neighborhood of $Y$, is it true that  the ampleness of $N_Y$ implies
    the convergence of the formal meromorphic functions ?
\end{question}

Rossi's theorem (\cite{MR0176106}) gives a positive answer when $Y$ is a hypersurface of dimension at least two.

\subsection{Continuation of analytic subvarieties} The next result was proved by Rossi in \cite[Theorem 3.2]{MR244516} under the additional assumption $\dim (V \cap Y) + n =  \dim Y + \dim V$. The improvement below is due to Chow \cite[Corollary of Theorem 4]{MR257074}.  Unaware of Chow's improvement on Rossi's result, one of the authors of the present paper obtained an alternative proof of Rossi-Chow Theorem   in \cite[Theorem C]{MR3999055}.

\begin{thm}\label{T:Rossi}
    Let $Y \subset \mathbb P^n$ be an irreducible subvariety and  let $U\subset \mathbb P^n$ be a Euclidean neighborhood of $Y$.
    If $V\subset U$ is an irreducible subvariety of $U$ with  $\dim V + \dim Y > n$ and non-empty intersection with $Y$ then there exists a projective subvariety $\overline V \subset \mathbb P^n$ such that $V$ is an irreducible component of $\overline V \cap U$.
\end{thm}

Bost \cite[Theorem 3.5]{MR1863738}, Bogomolov-McQuillan \cite{MR3644242},  Campana-Paun \cite[Theorem 1.1]{MR3949026}, Druel \cite[Proposition 8.4]{MR3742759}, and others, used variants of Hartshorne's argument to establish algebraicity criteria for leaves of foliations on projective varieties. See also the work of Chen \cite[Theorem A]{MR2957623} for an algebraicity criterion for formal subschemes of projective schemes. Here we point out that we can deduce from Theorem  \ref{T:Hartshorne} a variant of Theorem \ref{T:Rossi}.

\begin{prop}\label{P:Rossi variant}
    Let $X$ be a projective manifold, let $Y \subset X$ be an irreducible locally complete intersection analytic subspace/subscheme with ample normal bundle, and let $U \subset X$ be a small Euclidean neighborhood of $Y$. If $V \subset U$ is a closed, irreducible, smooth subvariety of $U$ such that
    \begin{enumerate}
        \item\label{I:Rossi variant 1} $\dim V + \dim Y > \dim X$ and
        \item\label{I:Rossi variant 2} $\dim (V \cap Y) + \dim X =  \dim Y + \dim V$
    \end{enumerate}
    then there exists a projective subvariety $\overline V \subset X$ such that $V$ is an irreducible component of $\overline V \cap U$.
\end{prop}
\begin{proof}
    Condition (\ref{I:Rossi variant 1}) implies that $\dim V \cap Y >0$.
    Condition (\ref{I:Rossi variant 2}) implies that the codimension of $V\cap Y$ in $V$ equals the codimension of $Y$ in $X$.
    Therefore $V\cap Y$ is a locally complete intersection analytic subspace of $V$ with ample normal bundle and positive dimension. Consider the Zariski closure $\overline V \subset X$ of $V$ and notice that 
    $\trdeg \C(\overline V)=\trdeg \C(V\cap Y(\infty))$. On the other hand, we can apply Theorem \ref{T:Hartshorne}
    to $V\cap Y\subset V$ and get that $\trdeg \C(V\cap Y(\infty))\le \dim(V)$, and conclude.
\end{proof}


