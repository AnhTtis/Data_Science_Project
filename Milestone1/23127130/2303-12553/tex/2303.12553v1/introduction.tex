\section{Introduction}

Let $X$ be a (not necessarily compact) complex manifold and let $Y\subset X$ be a compact and connected submanifold.
We are interested in the field $\mathbb C(X,Y)$ of germs of meromorphic functions along $Y$, \ie equivalence classes of meromorphic functions defined on arbitrarily small open subsets of $X$ containing $Y$.

When the normal bundle of $Y$ in $X$ is ample, classical results by Andreotti and Hartshorne imply that the transcendence degree of $\mathbb C(X,Y)$  over $\mathbb C$ is bounded by the dimension of $X$. Recently, in \cite{MR4403217}, the first two of the authors of the present paper proved the existence of germs of smooth surfaces $S$
along a smooth compact curve $C$ isomorphic to $\mathbb P^1$ with arbitrary positive self-intersection without non-constant  meromorphic functions.

Our first main result is a simpler, yet more general, construction of germs of complex manifolds without non-constant meromorphic functions that generalizes the main result of \cite{MR4403217}.

\begin{THM}\label{THM:A}
    For any complex projective manifold $Y$ of dimension $n$ and any pair of natural numbers $(\ell, m)$ such that $m \ge 2n$ and $\ell \le m$, there exists a germ of complex manifold $X$ of dimension $m$ and containing $Y$  such that the normal bundle of $Y$ in $X$ is ample and the transcendence degree of $\mathbb C(X,Y)$ over $\mathbb C$ is equal to $\ell$. Moreover, when $\ell =0$ we can further guarantee that $X$ carries no irreducible subvarieties of dimension at least $n+1$ neither foliations/webs of arbitrary non-zero dimension/codimension.
\end{THM}

More informally, if a submanifold $Y\subset X$ with ample normal bundle has dimension smaller than its codimension then there are no restrictions on the transcendence degree
of $\C(X,Y)$ over $\C$ besides the upper bound $\dim X$.

In sharp contrast, when $Y$ is a smooth hypersurface with ample normal bundle on a complex manifold $X$ and $\dim Y \ge 2$, Rossi proved in \cite[Section 5, Theorem 3]{MR0176106} the
existence of a projective manifold $\overline X$ and an open subset $U \subset X$ containing $Y$ that can be identified with an open subset of $\overline X$.

\begin{question}\label{Q:wild guess}
    Let $X$ be a complex manifold and $Y \subset X$ a compact submanifold with ample normal bundle.
    Does $\dim X < 2 \dim Y$ (\ie $\codim Y< \dim Y)$  implies $\trdeg \mathbb C(X,Y) = \dim X$ ?
\end{question}

Theorem \ref{THM:A} and Question \ref{Q:wild guess} should be compared with a conjecture by Peternell \cite{MR3049294} which predicts that field $\C(X)$ of meromorphic functions on  a compact Kähler manifold $X$ admitting a subvariety $Z$ with ample normal bundle satisfies $\trdeg \mathbb C(X) \ge \dim Z + 1$. We do not know if we can construct examples of pairs $(X,Y)$ with $X$ compact Kähler with arbitrary $\trdeg \mathbb C(X,Y)$. Peternell's conjecture predicts that this is not the case.

One of the main ingredients of our proof of Theorem \ref{THM:A} is an old trick attributed to Haefliger and explored by Loeb-Nicolau, López de Medrano-Verjovsky, and Meersseman, see \cite[Introduction]{MR1760670} and references therein. In order to construct complex manifolds, it suffices to construct smooth differentiable manifolds transverse to
holomorphic foliations. Another key ingredients used in the proof of Theorem \ref{THM:A} are the very same results by Andreotti and Hartshorne mentioned above that served as motivation for this work.

Although we are not able to provide an answer to Question \ref{Q:wild guess}, our second main result shows that the method of proof of Theorem \ref{THM:A}
cannot be applied to produce counter-examples to it.


\begin{THM}\label{T:Van de Ven}
    Let $\F$ be a codimension $q$ foliation on $\mathbb P^n$ and let $Z \subset \mathbb P^n$ be a submanifold disjoint from $\sing(\F)$.
    If $q < 2 \dim Z$ and,  for every $z \in Z$, the intersection of the tangent spaces of $\F$ and $Z$ at $z$ has the expected
    dimension $\max \{ 0, \dim Z + \dim \F -n\}$ then $\F$ is algebraically integrable.
\end{THM}

As we will explain in Section \ref{S:Van de Ven}, Theorem \ref{T:Van de Ven} is strictly related to a classical result of Van de Ven \cite{MR0116361} characterizing linear subspaces of projective spaces. We also point out, that it is unclear what the optimal statement should be. It is conceivable that under the assumptions of Theorem \ref{T:Van de Ven} the foliation
$\F$ must defined by a linear projection (\ie $\F$ has degree zero). We actually prove this when $\F$ is a foliation by curves, see Theorem \ref{T:weakly transverse}.



\subsection{Acknowledgments} The authors thank Brazilian-French Network in Mathematics and CAPES/COFECUB Project Ma 932/19 “Feuilletages holomorphes et intéractions avec la géométrie”. 
Falla Luza acknowledges support from CNPq (Grant number 402936/2021-3).
Loray is supported by CNRS and  Centre Henri Lebesgue, program ANR-11-LABX-0020-0.
Pereira acknowledges support from CNPq (Grant number 301683/2019-0), and FAPERJ (Grant number E-26/200.550/2023). 