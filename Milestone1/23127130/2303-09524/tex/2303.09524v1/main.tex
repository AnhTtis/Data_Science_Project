\documentclass[11pt, letterpaper]{article}


\usepackage[margin=1in]{geometry}
\usepackage{charter}

\usepackage{amsfonts, amsmath, amssymb, amsthm}
\usepackage{bm}
\usepackage{verbatim}
\usepackage{color}
\usepackage{euscript}
\usepackage{graphicx}
\usepackage[usenames,dvipsnames]{xcolor}
%\usepackage{paralist}
\usepackage{subcaption}

\usepackage{xspace}
\usepackage{wrapfig}




\def\figcapup{\vspace{-0mm}}
\def\figcapdown{\vspace{-0mm}}
\def\extraspacing{\vspace{2mm} \noindent}
\def\minilen{0.95\linewidth}
\def\vgap{\vspace{0.1 in}}




\renewcommand{\arraystretch}{1.3}

\newtheorem{theorem}{Theorem}
\newtheorem{lemma}{Lemma}
\newtheorem{corollary}{Corollary}
\newtheorem{observation}{Observation}
\newtheorem{proposition}{Proposition}
\newtheorem{definition}{Definition}
\newtheorem{problem}{Problem}
\newtheorem{claim}{Claim}
%-------------- 'template' ?

\DeclareMathOperator{\sign}{sign}
%\graphicspath{{../img/}}
\usepackage[ruled,vlined,linesnumbered]{algorithm2e}
\usepackage{todonotes}
\usepackage{multirow}
\usepackage{thmtools}
\usepackage{thm-restate}
\usepackage{bbm}
\usepackage{cleveref}
\usepackage{authblk}
%\usepackage{subfigure}

\title{Online and Dynamic Algorithms for Geometric Set Cover and Hitting Set}
%for Hyperrectangles

% \author{Arindam Khan\affil{1}, Aditya Lonkar\affil{1}, Saladi Rahul\affil{1}, Aditya Subramanian\affil{1}, Andreas Wiese\affil{2}}

% \affiliation{1}{Indian Institute of Science, Bengaluru, India}
% \affiliation{2}{Technical University of Munich, Germany}

\author[1]{Arindam Khan\thanks{supported by IUSSTF virtual center on ``Polynomials as an Algorithmic Paradigm', Pratiksha Trust Young Investigator Award,  Google India Research Award, Google ExploreCS Award, and SERB Core Research Grant     (CRG/2022/001176) on ``Optimization under Intractability and Uncertainty''.}}
\author[1]{Aditya Lonkar}
\author[1]{Saladi Rahul}
\author[1]{Aditya Subramanian\thanks{supported in part by Kotak IISc AI-ML Centre (KIAC) PhD Fellowship.}}
\author[2]{Andreas Wiese}
\affil[1]{Indian Institute of Science, Bengaluru, India}
\affil[2]{Technical University of Munich, Germany}


% \authorrunning{A. Khan, A. Lonkar,  A. Subramanian, S. Rahul, and A. Wiese}

% \Copyright{Arindam Khan, Aditya Lonkar, Saladi Rahul, Aditya Subramanian and Andreas Wiese}

% \keywords{Geometric Set Cover, Hitting Set, Rectangles, Squares, Hyperrectangles, Online Algorithms, Dynamic Data Structures} %TODO mandatory; please add comma-separated list of keywords
% \category{} %optional, e.g. invited paper
% \relatedversion{} %optional, e.g. full version hosted on arXiv, HAL, or other respository/website
%\funding{(Optional) general funding statement \dots}%optional, to capture a funding statement, which applies to all authors. Please enter author specific funding statements as fifth argument of the \author macro.

% \acknowledgements{Arindam Khan gratefully acknowledges the generous support due to IUSSTF virtual center on ``Polynomials as an Algorithmic Paradigm'' Pratiksha Trust Young Investigator Award,  Google India Research Award, Google ExploreCS Award, and SERB Core Research Grant 	(CRG/2022/001176) on ``Optimization under Intractability and Uncertaint''. Aditya Subramanian was supported in part by Kotak IISc AI-ML Centre (KIAC) PhD Fellowship.}

%\nolinenumbers

% %Editor-only macros:: begin (do not touch as author)%%%%%%%%%%%%%%%%%%%%%%%%%%%%%%%%%%
% \EventEditors{John Q. Open and Joan R. Access}
% \EventNoEds{2}
% \EventLongTitle{42nd Conference on Very Important Topics (CVIT 2016)}
% \EventShortTitle{CVIT 2016}
% \EventAcronym{CVIT}
% \EventYear{2016}
% \EventDate{December 24--27, 2016}
% \EventLocation{Little Whinging, United Kingdom}
% \EventLogo{}
% \SeriesVolume{42}
% \ArticleNo{23}
% %%%%%%%%%%%%%%%%%%%%%%%%%%%%%%%%%%%%%%%%%%%%%%%%%%%%%%

\newcommand{\secc}{supremal edge-covering}


% \newcommand{\awr}[1]{\todo[color=orange!100!black!50]{\small AW: #1}}
% \newcommand{\asr}[1]{\todo[color=blue!100!black!50]{\small AS: #1}}
% \newcommand{\akr}[1]{\todo[color=red!100!black!50]{\small AK: #1}}
% \newcommand{\alr}[1]{\todo[color=orange!100!black!50]{\small AL: #1}}
% \newcommand{\rsr}[1]{\todo[color=blue!100!black!50]{\small RS: #1}}

% \newcommand{\aw}[1]{\textcolor{orange}{#1}}
% \newcommand{\as}[1]{\textcolor{blue}{#1}}
% \newcommand{\ak}[1]{\textcolor{red}{#1}}
% \newcommand{\al}[1]{\textcolor{green!70!black!70}{#1}}
% \newcommand{\rs}[1]{\textcolor{yellow}{#1}}

%Added by Rahul
\newcommand{\IR}{\mathbb{R}}
\newcommand{\tcr}[1]{\textcolor{red}{#1}}
\newcommand{\mT}{\mathbb{T}}



\newcommand{\awr}[1]{}
\newcommand{\asr}[1]{}
\newcommand{\akr}[1]{}
\newcommand{\alr}[1]{}
\newcommand{\rsr}[1]{}

\newcommand{\aw}[1]{{#1}}
\newcommand{\as}[1]{{#1}}
\newcommand{\ak}[1]{{#1}}
\newcommand{\al}[1]{{#1}}
\newcommand{\rs}[1]{{#1}}


\begin{document}

\maketitle

\begin{abstract}
Set cover and hitting set are fundamental problems in combinatorial
optimization which are well-studied in the offline, online, and dynamic
settings. We study the geometric versions of these problems and present
new online and dynamic algorithms for them. In the online version
of set cover (resp. hitting set), $m$ sets (resp.~$n$ points) are given
%offline
 and $n$ points (resp.~$m$ sets) arrive online, one-by-one. In the dynamic
versions, points (resp. sets) can arrive as well as depart. Our goal
is to maintain a set cover (resp. hitting set), minimizing the size
of the computed solution.

For online set cover for (axis-parallel) squares of arbitrary sizes,
we present a tight $O(\log n)$-competitive algorithm. In the same
setting for hitting set, we provide a tight $O(\log N)$-competitive
algorithm, assuming that all points have integral coordinates in $[0,N)^{2}$.
No online algorithm had been known for either of these settings, not
even for unit squares (apart from the known online algorithms for
arbitrary set systems).

For both dynamic set cover and hitting set with $d$-dimensional hyperrectangles,
we obtain $(\log m)^{O(d)}$-approximation algorithms with $(\log m)^{O(d)}$
worst-case update time. This partially answers an open question
posed by Chan et al. {[}SODA'22{]}. Previously, no dynamic algorithms
with polylogarithmic update time were known even in the setting of squares (for either of these problems). %Our approximation ratio improves to $O(\log n)$ for the case of squares (of arbitrary sizes).
 Our main technical contributions are an \emph{extended quad-tree
}approach and a \emph{frequency reduction} technique that reduces
geometric set cover instances to instances of general set cover with
bounded frequency. 
\end{abstract}
\global\long\def\OPT{\mathsf{OPT}}%
\global\long\def\P{\mathcal{P}}%
\global\long\def\NP{\mathsf{NP}}%
\global\long\def\ALG{\mathsf{ALG}}%
\global\long\def\RR{\mathbb{R}}%
\global\long\def\R{\mathbb{R}}%
\global\long\def\N{\mathbb{N}}%
\global\long\def\Z{\mathbb{Z}}%
\global\long\def\X{\mathcal{X}}%
\global\long\def\C{\mathcal{C}}%
\global\long\def\conv{\mathrm{conv}}%
\global\long\def\Prob{\mathrm{Prob}}%
\global\long\def\F{\mathcal{F}}%
\global\long\def\E{\mathbb{E}}%
\global\long\def\MST{\mathsf{MST}}%
\global\long\def\DP{\mathsf{DP}}%
\global\long\def\YES{\mathrm{YES}}%
\global\long\def\NO{\mathrm{NO}}%
\global\long\def\SIZE{\mathrm{SIZE}}%
\global\long\def\FF{\mathrm{FF}}%
\global\long\def\true{\mathsf{true}}%
\global\long\def\false{\mathsf{false}}%
\global\long\def\S{\mathcal{S}}%
\global\long\def\A{\mathcal{A}}%


\section{Introduction}

%The set cover problem is fundamental in combinatorial optimization. We are given a universe $P$ of $n$ elements and a family
%$\S$ of $m$ subsets of these elements. Our goal is to select a collection
%$\S'\subseteq\S$ of these subsets that contain (i.e., cover) all
%elements in $P$, minimizing the cardinality of the sets in $\S'$.
%In the offline, online, and dynamic setting the problem is very well
%understood: there are $O(\min\{\log n,f\})$-approximation algorithms
%\cite{slavik1996tight,hochbaum1982approximation,bhattacharya2021dynamic,assadi2021fully}
%known in the offline and in the dynamic setting (\ak{with $(f\log n)^{O(1)}$-update
%time}), where $f$ denotes the maximum number of sets any element
%is contained in (also known as the \emph{frequency}), and there are
%corresponding lower bounds ~\cite{alon2006algorithmic,dinur2014analytical,khot2008vertex}
%\al{under \aw{certain} complexity theoretic assumptions}. In the
%online case where the items arrive over time, there is an $O(\log n\log m)$-approximation
%which is asymptotically tight~\cite{alon2003online}. Set cover can
%be shown to be equivalent to the hitting set problem in which instead
%we seek to select a subset $P'\subseteq P$ of the elements such that
%each set contains at least one element in $P'$. Hence, we obtain
%the corresponding (essentially tight) results for hitting set.
Geometric set cover is a fundamental and well-studied problem in computational geometry
%classical variant of the set cover problem that arises in the geometric setting 
\cite{clarkson2005improved, chan2012weighted, varadarajan2010weighted, har2012weighted, mustafa2014settling}.
%An interesting \aw{setting} arises in the %geometric \aw{case}: 
Here, we are given a universe $P$ of $n$ points in $\R^{d}$, and a family $\S$ of $m$ sets, where each set $S\in\S$ is a geometric object (we assume $S$ to be a {\em closed} set in $\R^{d}$ and $S$ {\em covers} all points in $P\cap S$),
e.g., a hyperrectangle.
Our goal is to select a collection
$\S'\subseteq\S$ of these sets that contain (i.e., cover) all elements in $P$, minimizing the cardinality of $\S'$ (see \Cref{fig:problem-def} for an illustration). The {\em frequency} $f$ of the set system $(P,\S)$ is defined as the maximum number of sets that contain an element in $P$.
%\aw{we assume that} the elements in $P$ are points in $\R^{d}$
%for some $d\in\N$ and that each set $S\in\S$ is a geometric object \ak{(which
%\aw{we assume to be a closed set in $\R^{d}$)}}, %e.g.,
%a square or a hyperrectangle (which hence covers all points in $P\cap S$).

In the offline setting of
some cases of geometric set cover,  
better approximation ratios are known 
than those for the general set cover, e.g., there is
a polynomial-time approximation scheme (PTAS) for (axis-parallel)
squares \cite{MustafaR09}.
%an $O(1)$-approximation (\al{in particular, a PTAS})
%algorithm~\cite{MustafaR09} and for (axis-parallel) rectangles \alr{this part is not true. Best known is $O(\log \OPT)$. In fact, the $\varepsilon$-net approach fails for rectangle set cover~\cite{pach2013tight}}{there is an
%$O(\log\log\OPT)$-approximation algorithm {[}?{]}}. 
However, much less is understood in the online and in the dynamic
variants of geometric set cover. In the online setting, the sets are given offline and the points arrive one-by-one, and for an uncovered point, we have to select a (covering) set in an immediate and irrevocable manner. To the best of our knowledge, even for 2-D unit squares, there is no known online algorithm with an asymptotically improved competitive ratio compared to the $O(\log n\log m)$-competitive 
algorithm for general online set cover~\cite{alon2003online,BuchbinderN09}.
%\alr{Can mention as $2$-factor trivially but not published I think. For the weighted case, $O(\log m)$ was known.}{The only exception is a $O(\log n)$-competitive
%algorithm for $d=1$, i.e., for intervals {[}??{]}}. 
In the dynamic case, the sets are again given offline and at each time step a point is inserted or deleted.
Here, we are interested in algorithms that update the
current solution quickly when the input changes.  In particular, it is desirable to have algorithms
whose update times are polylogarithmic. Unfortunately, hardly any
such algorithm is known for geometric set cover. 
Agarwal et al.~\cite{AgarwalCSXX20} initiated the study of dynamic geometric
set cover for intervals and 2-D unit squares and presented $(1+\varepsilon)$-
and $O(1)$-approximation algorithms with polylogarithmic update times,
respectively. 
%For squares of arbitrary sizes, the fastest known algorithm
%is due to Chan et al. \cite{chan2022dynamic} and has an update time
%of $n^{1/2+\varepsilon}$ (achieving an $O(1)$-approximation).
%Note
%that all these algorithms store their solutions \emph{implicitly,
%}rather than storing them explicitly in a certain space in memory.}\awr{check whether true...}
\begin{comment}
Chan and He \cite{chan2021more} extended it to set cover with arbitrary
squares. Finally, very recently, Chan et al. \cite{chan2022dynamic}
gave $(1+\varepsilon)$-approximation for the special case of intervals
in $O(\log^{3}n/\varepsilon^{3})$-amortized update time, where $\varepsilon$
is an arbitrarily small constant. They also gave $O(1)$-approximation
for dynamic set cover for unit squares, arbitrary squares, and weighted
intervals in amortized update time of $2^{O(\sqrt{\log n})},n^{1/2+\varepsilon}$,
and $2^{O(\sqrt{\log n\log\log n})}$, respectively.
\end{comment} 
%; only very recently the first such
%algorithm was found, again for the special case of $d=1$, i.e., intervals~\cite{chan2022dynamic}. Even for unit squares,
%the best update time is $2^{O(\sqrt{\log n})}$ and for arbitrary
%squares even only $n^{1/2+\delta}$ (both yielding an $O(1)$-approximation)~\cite{chan2022dynamic}. 
To the best of our knowledge, for more general objects, e.g., rectangles,
three-dimensional cubes, or hyperrectangles in higher
dimensions, no such dynamic algorithms are known. Note that in dynamic
geometric set cover, the inserted points are represented by their
coordinates, which is more compact than for  general (dynamic) set cover
(where for each new point $p$ we are given a list of the sets that
contain $p$, and hence, already to read this input we might need $\Omega(f)$
time).

\begin{figure}[!htb]
\centering
\includegraphics[page=11, scale=0.8]{02SetCovHitSet.pdf}
\caption{(a) A set of squares $\S$ and a set of points $P$, (b) A set cover (in green) $\S'\subseteq \S$ covering $P$, (c) A hitting set (green points) $P'\subseteq P$ for $\S$.}
\label{fig:problem-def}
\end{figure}

\iffalse{
\begin{figure}
\centering %  \begin{subfigure}[b]{0.3\textwidth}
 %      \centering
 %      \includegraphics[page=1,width=0.9\textwidth]{../img/02SetCovHitSet}
 %      \caption{Problem instance.}
 %  \end{subfigure}
 %  \hfill
 %  \begin{subfigure}[b]{0.3\textwidth}
 %      \centering
 %      \includegraphics[page=6,width=0.9\textwidth]{../img/02SetCovHitSet.pdf}
 %      %caption{Set Cover.}
 %  \end{subfigure}
 %  \hfill
 %  \begin{subfigure}[b]{0.3\textwidth}
 %      \centering
 %      \includegraphics[page=7,width=0.9\textwidth]{../img/02SetCovHitSet.pdf}
 %      %caption{Hitting set.}
 %  \end{subfigure}

\includegraphics[scale=0.8, page=11]{02SetCovHitSet.pdf}
\caption{a)Problem instance \hspace{20pt} b) Set cover (in green) \hspace{20pt} c)
Hitting set (in green).}
% \caption{Example of set cover and hitting set (highlighted in green).}
 \label{fig:problem-def} 
\end{figure}
}
\fi

Related to set cover is the hitting set problem (see \Cref{fig:problem-def} for an illustration) where, given a set of points $P$ and a collection of sets $\S$, we seek to select the minimum number of points $P' \subseteq P$ that hit each set $S\in \S$, i.e., such that $P' \cap S \ne \emptyset$. 
%The situation 
%for geometric hitting set is similar as for set cover:
Again, in the offline geometric case, there are better approximation ratios known
than for the general case, e.g., a PTAS for squares~\cite{MustafaR09}, 
and an $O(\log\log\OPT)$-approximation  for rectangles~\cite{aronov2010small}.
However, in the online and the dynamic cases, only few results are
known that improve on the results for the general case. In the online
setting, there is an $O(\log n)$-competitive algorithm for $d=1$,
i.e., intervals, and an $O(\log n)$-competitive algorithm for unit
disks~\cite{EvenS11}. In the dynamic case, the only known algorithms
are for intervals and unit squares (and thus\al{,} also for quadrants),
yielding approximation ratios of $(1+\varepsilon)$ and $O(1)$, respectively~\cite{AgarwalCSXX20}.



\subsection{Our results}
In this paper, we study online algorithms for geometric set cover
and hitting set for squares \emph{of arbitrary sizes}, while previously no improved results were known even for unit squares.
%could handle only objects of identical sizes,
%i.e., unit squares \cite{AgarwalCSXX20} and unit disks \cite{EvenS11}. 
Also, we present
dynamic algorithms for these problems for hyperrectangles of constant dimension $d$ (also called $d$-boxes or orthotopes) which are far more general geometric objects than those
which were previously studied, e.g., intervals \cite{AgarwalCSXX20} or (2-D)
squares \cite{chan2022dynamic}. 
\paragraph*{Online set cover for squares}
In \Cref{sec:Set-cover-squares} we study online set cover for axis-parallel
squares of arbitrary sizes and provide an online $O(\log n)$-competitive
algorithm. %Thus,
%up to constant factors we achieve the same competitive ratio as the
%\alr{$2$-factor for intervals}{known $O(\log n)$-competitive algorithm for the much easier case
%of intervals}. Also, 
We also match (asymptotically) the lower bound of $\Omega(\log n)$,
and hence, our competitive ratio is tight. In our online model (as in \cite{alon2003online}), we
assume that the sets (squares) are given initially and the elements (points) arrive
online. 

Our online algorithm is based on a new offline algorithm that is \emph{monotone},
i.e., it has the property that if we add a new point $p$ to $P$,
the algorithm outputs a superset of the squares that it outputs given
only $P$ without $p$. The algorithm is based on a quad-tree decomposition.
It traverses the tree from the root to the leaves, and for each cell
$C$ in which points are still uncovered, it considers each edge $e$
of $C$ and selects the ``most useful'' squares containing $e$, i.e.,
the squares with the largest intersection with $C$. We assume (throughout
this paper) that all points and all vertices of the squares 
have integral coordinates in $[0,N)^{2}$ for a given $N$, and we
obtain a competitive ratio of $O(\log N)$. If we know that all the
inserted points come from an initially given set of $n$ candidate
points $P_{0}$ (as in, e.g., Alon et al.~\cite{alon2003online}), we improve our competitive ratio to $O(\log n)$. For this case,
we use the BBD-tree data structure due to Arya et al.~\cite{arya1998optimal}
which uses a more intricate decomposition into cells than a standard
quad-tree, and adapt our algorithm to it in a non-trivial manner.
Due to the monotonicity of our offline algorithm, we immediately obtain
an $O(\log n)$-competitive online algorithm.%
\begin{comment}
In the online setting, it is common to assume that we are given a
set of $n$ candidate elements (points in our case) $P_{0}$ such
that the adversary may introduce only elements from $P_{0}$, e.g.,
see Alon et al.~\cite{alon2003online} who initiated the study of
online set cover. For this case, we use the BBD-tree data structure
due to Arya et al.~\cite{arya1998optimal} which uses a more intricate
decomposition into cells than a standard quad-tree. We adapt our algorithm
to the BBD-tree, obtaining an approximation of $O(\log n)$.
\end{comment}


\begin{comment}
    

In the dynamic case, we can show that only $O(\log n)$ sets change
after each update (where a point is inserted or deleted) and we can update our solution accordingly
in polylogarithmic update time. We require this algorithm and all
our other dynamic algorithms presented in this paper to store the
computed solution \emph{explicitly} in memory after each update. This
is a harder requirement than storing a solution only \emph{implicitly}
as, e.g., in~\cite{chan2021more}. Note that if squares were
also allowed to be inserted and deleted, it would be impossible
to maintain solutions with any non-trivial approximation ratio explicitly
in polylogarithmic time, since then $\Omega(m+n)$ sets may need to
change after each update (even in an amortized sense).
\end{comment}

\paragraph*{Online hitting set for squares}
In \Cref{sec:hit-set-squares} we present an $O(\log N)$-competitive
algorithm for online hitting set for squares of arbitrary sizes, where the points are  given initially 
and the squares arrive online. This matches the best-known $O(\log N)$-competitive  algorithm for the much simpler case of intervals~\cite{EvenS11}. 
Also, there is a matching lower bound of $\Omega(\log N)$, even 
for intervals. 



\iffalse{
\begin{table}
    \begin{centering}
        \begin{tabular}{|c|c|c|c|}
            \hline
            Problem & Objects & Approximation ratio & Update time \\
            \hline
            \hline
            \multirow{4}{*}{Set cover} & intervals & $(1+\varepsilon)$~\cite{AgarwalCSXX20} & $(n^{\alpha}/\varepsilon)(\log n)^{O(1)}$ \\ \cline{2-4}
                      & unit squares & $O(1)$~\cite{AgarwalCSXX20} & $(n^{1/2+\alpha})(\log n)^{O(1)}$ \\ \cline{2-4}
                      & squares & $O(\log n)$~[Thm \ref{thm:dynSCSq}] & $(\log n)^{O(1)}$ \\ \cline{2-4}
                      & hyperrectangles & $\ensuremath{O_{d}(\log^{4d-1}m)\log W}$~[Thm \ref{thm:WtDynSetCov}] & $O_{d}(\log^{2d}m)\log^{3}(Wm)$ \\ \hline
            \multirow{3}{*}{Hitting set} & intervals & $(1+\varepsilon)$~\cite{AgarwalCSXX20} & $(1/\varepsilon)(\log n)^{O(1)}$ \\ \cline{2-4}
                        & unit squares & $O(1)$~\cite{AgarwalCSXX20} & $(\log n)^{O(1)}$ \\\cline{2-4}  
                        %& squares & $O(\log N)$ & $(\log N)^{O(1)}$ \\\cline{2-4} 
                        & hyperrectangles & $\ensuremath{O_{d}(\log^{4d-1}n)\log W}$~[Thm \ref{thm:WtDynHitSet}] & $O_{d}(\log^{2d-1}n)\log^{3}(Wn)$\\\hline
        \end{tabular}
    \par\end{centering}
    \caption{Dynamic algorithms for geometric set cover. Update times in \cite{AgarwalCSXX20} are amortized. Our results are for worst-case update times. Here $d$ is the dimension of hyperrectangles and $\alpha>0$ is a small constant. The notation $O_d(\cdot)$ means that the implicit constant hidden by the big-O notation can depend on $d$.}
\end{table}
}
\fi

In a nutshell, if a new square $S$ is inserted by the adversary,
we identify $O(\log N)$ quad-tree cells for which $S$ contains one of its edges. Then, we pick the most useful points in these cells
to hit such squares: those are the points closest to the four edges
of the cell. We say that this \emph{activates} the cell. In
our analysis, we turn this around: we show that for each point $p\in\OPT$
there are only $O(\log N)$ cells that can possibly get activated
if a square $S$ is inserted that is hit by $p$. This yields a competitive
ratio of $O(\log N)$. 
%For similar reasons as above, it is impossible
%to guarantee a competitive ratio which is $o(n)$ or $o(m)$ if also points
%can be inserted by the adversary.

\paragraph*{Dynamic set cover and hitting set for $d$-D hyperrectangles}
Then, in \Cref{sec:set-cover-hyperrectangles} and \ref{sec:hit-set-hyperrectangles}
we present our dynamic algorithms for set cover and hitting set for
hyperrectangles in $d$ dimensions. 
%\akr{commented  fully dynamic part}
\begin{comment}
\al{For the dynamic setting of set cover foreven 2-D squares where set insertions and deletions are allowed, it is unlikely that one can maintain a set cover with polylogarithmic approximation guarantee, while having polylogarithmic update time as well.
Consider the example where $P$ consists of points denoted by $p_{ij}\in \mathbb{R}^2$ for all $i,j$ such that $i\in\{0,1,...,N-1\}$, $j\in \{0,1,...,N-1\}$ and its coordinates are $(i+0.5,j+0.5)$ for an integer $N$. The set collection $\mathcal{S}$ consists of a square $S'$ defined by $[0,0]\times [N,N]$ and the other squares denoted by $S_{ij}\in\mathcal{S}$ are defined by $[i,i+1]\times [j,j+1]$ for all $i,j$ such that $i\in\{0,1,...,N-1\}$, $j\in \{0,1,...,N-1\}$. Now, in the case when the points to be covered are all the points in $P$, the sets which have been introduced are all the squares in $\mathcal{S}\setminus \{S'\}$. Then, when the adversary introduces $S'$ in the current iteration, $\OPT$ becomes $1$ and when it is deleted from the instance by the adversary, $\OPT$ becomes $|\S|-1$. %Thus, it is unlikely that one can maintain an approximate solution with ``good" enough competitive ratio (polylogarithmic or better) while having polylogarithmic update time.}.
} \rsr{Why do we need to write this? There is no reason to believe it 
cannot be done.}
\end{comment}
Note that no dynamic algorithm
with polylogarithmic update time and polylogarithmic approximation
ratio is known even for set cover for rectangles and it was asked
explicitly by Chan et al.~\cite{chan2022dynamic} whether such an
algorithm exists. Thus, we answer this question in the affirmative
for the case when only points are inserted and deleted. Note that
this is the relevant case when we seek to store our solution explicitly,
as discussed above. Even though our considered objects are very
general, our algorithms need only polylogarithmic worst-case update time. In contrast, Abboud et al.~\cite{abboud2019dynamic} showed that under Strong Exponential Time Hypothesis any general (dynamic) set cover algorithm with an amortized update time of $O(f^{1-\varepsilon})$ must have an approximation ratio of $\Omega(n^{\alpha})$ for some constant $\alpha>0$, and $f$ can be as large as $\Theta(m)$.

%\akr{Rahul, please give a pass}
We first discuss our algorithm for set cover. We start with
reducing the case of hyperrectangles in $d$ dimensions to $2d$-dimensional hypercubes
with integral corners in $[0,4m]^{2d}$. Then, a natural approach would be to adapt our
algorithm for squares from above to $2d$-dimensional hypercubes.
A canonical generalization would be to build a quad-tree, traverse
it from the root to the leaves, and to select for each cell $C$ and
for each facet $F$ of $C$ the most useful hypercube $S$ containing
$F$, i.e., the hypercube $S$ with maximal intersection with $C$.
%(which is  well-defined up to tie-breaking since $S$ is required to contain the facet $F$).
Unfortunately, this is no longer sufficient, not even in three dimensions:
it might be that there is a cell $C$ for which it is necessary that
we select cubes that contain only an edge of $C$ but not a facet
of $C$ (see \Cref{fig:CounterEx3D}). Here, we introduce a crucial
new idea: for each cell $C$ of the (standard) quad-tree and 
for each dimension $i\in [2d]$, consider the hypercubes which are ``edge-covering'' 
$C$ along dimension $i$. Based on these hypercubes a 
$(2d{-}1)$-dimensional recursive secondary structure is built on all the 
dimensions except the $i$-th dimension (see \Cref{fig:extended-quadtree}).
%We interpret each such
%cell $C'$ as the root of a new tree, and we apply the construction
%recursively, i.e., we first build a standard quad-tree rooted in $C'$
%and then add new children to each cell $C''$, corresponding to reducing
%each dimension of $C''$.

\begin{figure}[ht]
    \center
    \includegraphics[scale=0.3]{04CounterEx3D}
    \caption{The red cube is the only cube that covers a facet of the (uncolored) cell. The green cube (from $\OPT$) only covers an edge of the cell. Note that there is no corner of a cube from $\OPT$ in the cell.
    Picking the red cube does not cover the
    the intersection of the green cube  with the cell. }
    \label{fig:CounterEx3D}
\end{figure}

We call the resulting tree the \emph{extended quad-tree. }Even though
it is much larger than the standard quad-tree, we show that each
point is contained in only $(\log m)^{O(d)}$ cells. Furthermore,
we use it for our second crucial idea to \emph{reduce the frequency}
of the set cover instance: we build an auxiliary instance of general
set cover with bounded frequency. It has the same points as the given
instance of geometric set cover, but different sets: for each node 
corresponding  to a one-dimensional cell $C$ of the extended quadtree
% cell 
% at the last level of the extended quadtree with a corresponding 
% cell $C$
, 
we consider each of its endpoints $p$
% facets $F$,
and introduce a set that corresponds to the ``most useful'' hypercube
covering $p$,
% $F$, 
i.e., the hypercube covering 
$p$
% $F$ 
with maximal intersection
with $C$. Since each point is contained in only $(\log m)^{O(d)}$
cells, the resulting frequency is bounded by $(\log m)^{O(d)}$. Also,
we show that our auxiliary set
cover instance admits a solution with at most $\OPT\cdot(\log m)^{O(d)}$
sets. Then we use a dynamic algorithm from \cite{bhattacharya2021dynamic} for \emph{general }set
cover to maintain an approximate solution for our auxiliary instance,
which yields a dynamic $(\log m)^{O(d)}$-approximation algorithm.
%\akr{change and make this one para}

We further adapt our dynamic set cover algorithm mentioned above to hitting set for $d$-dimensional hyperrectangles with an approximation ratio of $(\log n)^{O(d)}$.
%Our dynamic algorithm for hitting set works similarly: we use again
%the extended quad-tree, but instead of picking the maximal facet-covering
%hypercubes for each cell $C$, for each facet $F$ of $C$ we pick
%the points in $C$ that are closest to $F$. Since each point $p\in\OPT$
%is contained in at most $(\log m)^{O(d)}$ cells, we obtain an approximation
%ratio of $(\log m)^{O(d)}$.
Finally, we extend our algorithms for set cover and hitting set for
$d$-dimensional hyperrectangles even to the weighted case, at the
expense of only an extra factor of $(\log W)^{O(1)}$ in the update time and approximation ratio,
assuming that all sets/points in the input have weights in $[1,W]$.
%\tcr{Rahul: remove this line?}Due to space limitations, many proofs are moved to the appendix. 
See the following tables for a summary of our results. 

\begin{table}[!ht]
    \begin{centering}
        \begin{tabular}{|c|c|c|c|}
            \hline
            Problem & Objects & Competitive ratio & Lower bound \\
            \hline
            \hline
            \multirow{2}{*}{Set cover}  & intervals & 2 [Thm \ref{thm:onlineintervalupper}] & 2 [Thm \ref{thm:onlineintervallower}] \\ \cline{2-4}
                                        & 2-D squares & $O(\log n)$~[Thm \ref{squaressetcover_1}] & $\Omega(\log n)$~[Thm \ref{lb:unitsquares}] \\ \hline
            \multirow{2}{*}{Hitting set} & intervals & $O(\log N)$ \cite{EvenS11} & $\Omega(\log N)$ \cite{EvenS11} \\ \cline{2-4}
                                       % & unit disks & $O(\log n)$ \cite{EvenS11} & $\Omega(\log N)$ \cite{EvenS11} \\ \cline{2-4}
                                        & 2-D squares & $O(\log N)$~[Thm \ref{lem:squareshittingset_2}] & $\Omega(\log N)$\cite{EvenS11}  \\ \hline
        \end{tabular}
    \par\end{centering}
    \caption{Online algorithms for geometric set cover and hitting set.}
\end{table}

\begin{table}[!ht]
    \begin{centering}
        \begin{tabular}{|c|c|c|c|}
            \hline
            Problem & Objects & Approximation ratio & Update time \\
            \hline
            \hline
            \multirow{2}{*}{Set cover} %& intervals & $(1+\varepsilon)$~\cite{AgarwalCSXX20} & $(n^{\alpha}/\varepsilon)(\log n)^{O(1)}$ \\ \cline{2-4}
                      & $2$-D unit squares & $O(1)$~\cite{AgarwalCSXX20} & $(\log n)^{O(1)}$ \\ \cline{2-4}
                     % & $2$-D squares & $O(\log n)$~[Thm \ref{thm:dynSCSq}] & $(\log n)^{O(1)}$ \\ \cline{2-4}
                      & $d$-D hyperrectangles & $O(\log^{4d-1}m)\log W$~[Thm \ref{thm:WtDynSetCov}] & $O(\log^{2d}m)\log^{3}(Wn)$ \\ \hline
            \multirow{2}{*}{Hitting set} %& intervals & $(1+\varepsilon)$~\cite{AgarwalCSXX20} & $(1/\varepsilon)(\log n)^{O(1)}$ \\ \cline{2-4}
                        & unit squares & $O(1)$~\cite{AgarwalCSXX20} & $(\log n)^{O(1)}$ \\\cline{2-4}  
                        %& squares & $O(\log N)$ & $(\log N)^{O(1)}$ \\\cline{2-4} 
                        & $d$-D hyperrectangles & $\ensuremath{O(\log^{4d-1}n)\log W}$~[Thm \ref{thm:WtDynHitSet}] & $O(\log^{2d-1}n)\log^{3}(Wm)$\\\hline
        \end{tabular}
    \par\end{centering}
    \caption{Dynamic algorithms for geometric set cover and hitting set. Update times in \cite{AgarwalCSXX20} are amortized and for the unweighted case. Our results are for worst-case update times. 
    %The hyperrectangles lie in a constant dimension $d$.
    }
\end{table}


\subsection{Other related work}
The general set cover is well-studied in both online and dynamic settings. 
Several variants and generalizations of online set cover have been considered, e.g., online submodular cover \cite{gupta2020online}, online set cover under random-order arrival \cite{gupta2022random}, online set cover with recourse \cite{gupta2017online}, etc. 

For dynamic setting, Gupta et al.~\cite{gupta2017online} initiated the study and provided $O(\log n)$-approximation algorithm with $O(f \log n)$-amortized update time, even in the weighted setting. Similar to our model, in their model sets are given offline and only elements can appear or depart. After this, there has been a series of works~\cite{abboud2019dynamic, bhattacharya2018deterministic, bhattacharya2019new, bhattacharya2018dynamic, bhattacharya2021dynamic, gupta2017online, gupta2020fully, assadi2021fully}.%\alr{maybe mention a few words here like `obtain both approx and update times of the form $(f\log n)^{O(1)}$'}. 

Bhattacharya et al.~\cite{bhattacharya2021dynamic} have given deterministic $(1+\varepsilon)f$-approximation in \\ $O\left((f^2/\varepsilon^3) + (f/\varepsilon^2) \log (W)\right)$-amortized update time and $O(f\log^2(Wn)/\varepsilon^3)$-worst-case update time, where $W$ denotes the ratio of the weights of the highest and lowest weight sets. 
%Abboud et al.~\cite{abboud2019dynamic} show that under Strong Exponential Time Hypothesis any dynamic set cover algorithm with an amortized update time of $O(f^{1-\varepsilon})$ must have an approximation
%ratio of $\Omega(n^{\alpha})$ for some constant $\alpha>0$.
Assadi and Solomon \cite{assadi2021fully} have given a randomized $f$-approximation algorithm with $O(f^2)$-amortized update time. %even in the weighted setting. 

Agarwal et al.~\cite{AgarwalCSXX20} studied another dynamic setting for geometric set cover, where both points and sets can arrive or depart,  and presented $(1+\varepsilon)$-
and $O(1)$-approximation with sublinear update time for intervals and unit squares,
respectively.
Chan and He \cite{chan2021more} extended it to set cover with arbitrary
squares. Recently, Chan et al.~\cite{chan2022dynamic}
gave $(1+\varepsilon)$-approximation for the special case of intervals
in $O(\log^{3}n/\varepsilon^{3})$-amortized update time. They also gave $O(1)$-approximation
for dynamic set cover for unit squares, arbitrary squares, and weighted
intervals in amortized update time of $2^{O(\sqrt{\log n})},n^{1/2+\varepsilon}$,
and $2^{O(\sqrt{\log n\log\log n})}$, respectively.

Dynamic algorithms are also well-studied for other geometric problems such as maximum independent set of intervals and hyperrectangles \cite{Henzinger0W20, bhore2020dynamic, cardinal2021worst}, and geometric measure  \cite{dallant2021conditional}. 
%In the (offline) Set Cover problem, we are given a universe $U$ of $n$ elements and a family
%$\mathcal{F}$ of $m$ sets (with
%an associated non-negative weight for each set in $\mathcal{F}$, in the weighted setting). The objective is to find a minimum
%cardinality (resp. weight) collection of sets $\mathcal{F}_0\subseteq\mathcal{F}$ such that this
%collection covers all elements of $U$. Set Cover is $\mathsf{NP}$-hard~\cite{lewis1983michael} and a trivial polynomial time greedy algorithm~\cite {williamson2011design} achieves an
%approximation ratio of $O(\log n)$ for this problem.
%On the hardness side, \cite{lund1994hardness}
%showed an approximation hardness of $\approx 0.72\ln n$ under the assumption that $\mathsf{NP}\notin
%\mathsf{QP}$. \cite{feige1998threshold} improved this to an approximation hardness of $(1-o(1))\ln
%n$ under the same assumption.
%Finally,
%The work of Alon et al.\cite {alon2006algorithmic} and Dinur et al.\cite{dinur2014analytical}
%showed an approximation hardness of $(1-o (1))\ln n$ assuming $\mathsf{P}\neq
%\mathsf{NP}$.
%under the weaker assumption of $\mathsf{P}\neq
%\mathsf{NP}$.


%Alon et al.~\cite{alon2003online} initiated the study of the online set cover problem, where we are given a
%set of ranges (sets), offline. At each time step a new point (element) arrives. The objective is to
%maintain a valid set cover at each step (without recourse). Online set cover can be understood as a
%game between an algorithm and an adversary, where the set family $\mathcal{F}$ is revealed at the
%beginning of the game by the adversary as well as the point set $U$ (set of elements). At each time
%step, the adversary introduces points from $U$ and the algorithm has to make a choice to pick sets
%and maintain a set cover. The adversary can stop at any point in time (i.e, not necessary that all the points from $U$ arrrive). Here, the performance guarantee of an algorithm is
%measured by the competitive ratio, i.e., at the time step when the game ends, the competitive ratio
%is the ratio between the cardinality (resp. weight) of the set cover produced by the algorithm against the minimum
%cardinality (resp. weight) offline set cover corresponding to those set of points. Our objective is to produce an
%efficient algorithm that obtains a \textit{good} competitive ratio.
%Unlike the offline version, where we are interested in the computational complexity of the problem,
%here we are more interested in its information theoretic complexity. Our setting involves an
%adaptive adversary who has access to the decisions of the algorithm at each time step.
%For the
%general online set cover problem, the primal-dual approach is known to result in a randomized algorithm with $O (\log
%n\log m)$ competitive ratio~\cite{BuchbinderN09}. Also, Alon et al.~\cite{alon2003online} presented a
%multiplicative weight updates based deterministic algorithm which gives a $O(\log n\log m)$
%competitive ratio.
%They maintain a fractional set cover and in each iteration when the adversary
%introduces an element from the universe, they update the weights %of every set that covers that
%element. Further, they pick at most $4\log n$ sets in each %iteration in the set cover so that the
%value of a carefully designed potential function decreases as %compared to the last iteration. Also,
%the fact that their algorithm stops in $\OPT\log m$ iterations %helps to show the desired competitive
%ratio bound (Here OPT refers to the optimum set cover size %finally).
%They also present an almost
%tight lower bound of $\Omega(\frac{\log n\log m}{\log \log n+\log \log m})$ for any deterministic
%online algorithm. We study the problem where the given ranges are geometric objects.

%Gupta et al.~\cite{gupta2022random} studied online set cover in the random order model and gave a
%$O(\log mn)$-competitive multiplicative weight updates algorithm with matching lower bounds. As
%opposed to the algorithms of ~\cite{alon2003online} and ~\cite {BuchbinderN09}, they maintain a
%non-monotone fractional set cover solution (which is not even necessarily feasible) and round it
%online. Note that their algorithm is a randomized one. They show a lower bound of $\Omega(\log n)$
%for even the fractional variant of random order set cover, which implies that their algorithm
%achieves right guarantees as long as $m=poly(n)$.
%Gupta et al.~\cite {gupta2017online} studied the set cover
%problem in the fully dynamic model. In this model, the set of elements to be covered can change as
%elements can arrive as well as depart. The objective as before, is to maintain a set cover at all
%time steps. The difference between their model as compared to ours is that they allow recourse in
%the online setting, i.e., they want to minimize the number of updates the algorithm makes to the set
%cover solution. They show that their algorithm achieves a $O(\min\{\log n,f\})$-competitive ratio
%with $O(1)$ recourse, where $f$ is the maximum frequency of an element in the universe. \al{Abboud
 %   et al.~\cite{abboud2019dynamic} show that under Strong Exponential Time Hypothesis any dynamic set
  %  cover algorithm with amortized update time of $O(f^{1-\varepsilon})$ must have an approximation
%    ratio of $\Omega(n^{\alpha})$ for some constant $\alpha>0$. Bhattacharya et
 %   al.~\cite{bhattacharya2021dynamic} gave the first worst case update time bounds for dynamic set
  %  cover with an approximation guarantee of $(1+\varepsilon)f$. Their worst case update time is
%$O(f\log^2(Cn)/\varepsilon^3)$}
% 
% \begin{table}[!h]
%     \renewcommand{\arraystretch}{1.5}
%     \centering
%     \begin{tabular}{|p{1.5cm}| p{1.8cm}|p{2cm}|p{3.5cm}|p{2cm}|  }
%         \hline
%         %\multicolumn{3}{|c|}{Country List} \\
%         %\hline
%         Authors & Randomized&Approximation guarantee &Amortized update time & Weighted \\
%         \hline
%         \cite{bhattacharya2018dynamic}& No & $O(f^2)$ & $O(f\log (m+n))$& Yes\\
%         \hline
%         \cite{gupta2017online}& No & $O(\log n)$ & $O(f\log n)$& Yes\\
%         \hline
%         \cite{abboud2019dynamic}& Yes& $(1+\varepsilon)f$ &$O(f^2\log n/\varepsilon^5)$  & No\\
%         \hline
%         \cite{bhattacharya2019new}& No& $(1+\varepsilon)f$ &$O(f\log (Wn)/\varepsilon^2)$&Yes \\
%         \hline
%         \cite{bhattacharya2021dynamic}&No & $(1+\varepsilon)f$ &$O(f^2/\varepsilon^3+f/\varepsilon^2\log (W))$ &\iffalse{$O(f\log^2 (Wn)/\varepsilon^3)$}\fi Yes \\
%         \hline
%         \cite{assadi2021fully}& Yes& $f$  & $O(f^2)$ &Yes \\
%         \hline
%     \end{tabular}
%     \caption{General dynamic set cover results in points insertion/deletion setting. Here, $W$ denotes the ratio of the weights of the highest and the lowest weight sets.}
%     \label{tab:Results1}
% \end{table}
% 
% 
% 
% 
% 
% \begin{table}[!h]
%     \renewcommand{\arraystretch}{1.5}
%     \centering
%     \begin{tabular}{|p{3cm} |p{3cm}|p{3cm}|p{3cm}|  }
%         \hline
%         %\multicolumn{3}{|c|}{Country List} \\
%         %\hline
%         Authors & Approximation guarantee &Amortized update time & Worst Case update time \\
%         \hline
%         \cite{AgarwalCSXX20} &  & & \\
%         \hline
%         \cite{chan2021more} &  & & \\
%         \hline
%         \cite{chan2022dynamic} &  & & \\
%         \hline
%     \end{tabular}
%     \caption{dynamic set cover/hitting results in the fully dynamic setting}
%     \label{tab:Results2}
% \end{table}

%\subsection{Preliminaries}


\section{\label{sec:Set-cover-squares} Set cover for squares}

In this section we present our online and dynamic algorithms for set cover for squares. 
We are given a set of $m$ squares $\S$ %:=\{S_1, S_2, \dots,S_m\}$ 
such that each square $S\in\S$ has integral corners in $[0,N)^{2}$.  
 W.l.o.g.~assume that $N$ is a power of 2.
We first describe an offline $O(\log N)$-approximate algorithm. Then we 
construct an online algorithm and a dynamic algorithm based on it, such that both of them have
approximation ratios of $O(\log N)$ as well. For our offline algorithm, we assume that in addition
to $\S$ and $N$, we are given a set of points $P$ 
that we need to
cover, such that $P\subseteq[0,N)^{2}$ and each point $p\in P$ has integral coordinates.

\paragraph*{Quad-tree}
We start with the definition of a quad-tree $T=(V,E)$, similarly as in, e.g., \cite{arora1998polynomial, berg1997computational}. 
In $T$
each node $v\in V$ corresponds to a square cell $C_{v}\subseteq[0,N)^{2}$ whose vertices have integral coordinates. The
root $r\in V$ of $T$ corresponds to the cell $C_{r}:=[0,N)^{2}$. Recursively, consider a node $v\in
V$, corresponding to a cell $C_{v}$ and assume that
$C_{v}=[x_{1}^{(1)},x_{2}^{(1)})\times[x_{1}^{(2)},x_{2}^{(2)})$.  If $C_{v}$ is a unit square,
i.e., $|x_{2}^{(1)}-x_{1}^{(1)}|=|x_{2}^{(2)}-x_{1}^{(2)}|=1$, then we define that $v$ is a leaf.
Otherwise, we define that $v$ has four children $v_{1},v_{2},v_{3},v_{4}$ that correspond to the
four cells that we obtain if we {\em partition} $C_{v}$ into four equal sized smaller cells, i.e., define
$x_{\text{\ensuremath{\mathrm{mid}}}}^{(1)}:=(x_{2}^{(1)}-x_{1}^{(1)})/2$ and
$x_{\text{\ensuremath{\mathrm{mid}}}}^{(2)}:=(x_{2}^{(2)}-x_{1}^{(2)})/2$ and
$C_{v_{1}}=[x_{1}^{(1)},x_{\text{\ensuremath{\mathrm{mid}}}}^{(1)})\times[x_{1}^{(2)},x_{\text{\ensuremath{\mathrm{mid}}}}^{(2)})$,
$C_{v_{2}}=[x_{1}^{(1)},x_{\text{\ensuremath{\mathrm{mid}}}}^{(1)})\times[x_{\text{\ensuremath{\mathrm{mid}}}}^{(2)},x_{2}^{(2)})$,
$C_{v_{3}}=[x_{\text{\ensuremath{\mathrm{mid}}}}^{(1)},x_{2}^{(1)})\times[x_{1}^{(2)},x_{\text{\ensuremath{\mathrm{mid}}}}^{(2)})$,
and
$C_{v_{4}}=[x_{\text{\ensuremath{\mathrm{mid}}}}^{(1)},x_{2}^{(1)})\times[x_{\text{\ensuremath{\mathrm{mid}}}}^{(2)},x_{2}^{(2)})$.
Note that the
depth of this tree is $\log N$, where depth of a node in the tree is its distance from
the root of $T$, and depth of $T$ is the maximum depth of any node in $T$.
By the construction, each leaf node contains at most one point and it will lie on the bottom-left corner of the corresponding cell.
%\akr{we can remove some details of quad-tree if needed}

\paragraph*{Offline algorithm} 
%We first define an 
%offline algorithm $\mathcal {A}_\text{off}$, based on $T$. 
In the offline algorithm $\mathcal {A}_\text{off}$, we
traverse $T$ in a breadth-first-order, i.e., we order the nodes in $V$ by their distances to the
root $r$ and consider them in this order (breaking ties arbitrarily but in a fixed manner). Suppose that in one
iteration we consider a node $v\in V$, corresponding to a cell $C_{v}$. We check whether the
squares selected in the ancestors of $v$ cover all points in $P\cap C_{v}$. If this is the case,
we do not select any squares from $\S$ in this iteration (corresponding to $v$). Observe that hence
we also do not select any squares in the iterations corresponding to the descendants of $v$ in
$T$ (so we might as well skip the whole subtree rooted at $v$).

Suppose now that the squares selected in the ancestors of $v$ do \emph{not }cover all points in
$P\cap C_{v}$. We call such a node to be {\em explored} by our algorithm.
Let $e$ be an edge of $C_{v}$. 
We say that a square containing $e$ is \emph{edge-covering for $e$}.
%We call a square containing $e$ to be {\em edge-covering} for $C_{v}$.
We select a square from $\S$ that is edge-covering for $e$ 
 and that has the largest intersection with
$C_{v}$ among all such squares in $\S$ (we call such a square {\em maximum area-covering}
for $C_v$ for edge $e$). We break ties in an arbitrary but fixed way, e.g., by
selecting the square with smallest index according to an arbitrary ordering of $\S$. If there is no
square in $\S$ that is edge-covering for $e$ then we do not select a square corresponding to $e$. We do this
 for each of the four edges of $C_{v}$. See \Cref{fig:offline-set-cov}. 
 If we reach a leaf node, and if there is an uncovered point (note that it must be on the bottom-left corner of the cell), then we select any arbitrary square that covers the point (the existence of such a square is guaranteed as some square in $\OPT$ covers it). See \Cref{fig:offline-edge}.
 %\al{Otherwise, if $v$ is a leaf which contains at least one point, such that it is not covered by the squares picked for the ancestors of $v$ in $T$, $\A_{\text{off}}$ picks $O(1)$ squares for this leaf as we will describe now. First, for each leaf $v'\in T$, we associate with it at most $4$ fixed squares to cover its closed corners (if possible). Denote such squares as \textit{corner-covering} for $C_v$. Then, if $v$ is such an aforementioned leaf, $\A_{\text{off}}$ picks at most $4$ such corner-covering squares to cover the corners of $C_v$ (we ignore the rest of the area, but the corner points, covered by these squares in $C_v$).}



% \begin{figure}[ht]
%     \centering
%     \includegraphics[page=3,scale=0.4]{../img/02SetCovHitSet.pdf}
%     \caption{Squares picked for covering an input point.}
%     \label{fig:offline-set-cov}
% \end{figure}

\begin{figure}
     \centering
    %  \begin{subfigure}[b]{0.45\textwidth}
    %      \centering
    %      \includegraphics[page=8,width=\textwidth, scale=0.9]{../img/02SetCovHitSet.pdf}
    %      \caption{Maximum area-covering square for a cell.}
    %      \label{fig:offline-set-cov-b}
    %  \end{subfigure}
    %  \hfill
    %  \begin{subfigure}[b]{0.45\textwidth}
    %      \centering
    %      \includegraphics[page=3,width=\textwidth, scale=0.9]{../img/02SetCovHitSet.pdf}
    %      \caption{Squares picked for cells of various levels.}
    %      \label{fig:offline-set-cov-a}
    %  \end{subfigure}
     \includegraphics[page=10,width=\textwidth, scale=0.7]{02SetCovHitSet.pdf}
    
    \caption{
    % a)Maximum area-covering square for a cell. \hspace{15pt} b)Squares picked for cells of various levels.
    % Squares picked by $\mathcal{A}_\text{off}$.
    Left figure shows a quad-tree cell in purple. The maximum area-covering square (solid black) is picked, while the other edge-covering squares (dashed) are not.
    %Example of how the Algorithm picks squares.
%The figure on the right shows a cell in the quad-tree in purple, and edge-covering squares for it in black.
%The maximum area-covering square (solid black) is picked, while the other squares (dashed) are not.
Right figure shows the quad-tree cells (level-wise color-coded)
containing an uncovered point. In increasing order of depth of these cells, at most $4$ maximum-area covering squares (solid black) are picked together per cell, till the point gets covered.
    }
    \label{fig:offline-set-cov}
\end{figure}

\begin{figure}
\begin{center}
\includegraphics[page=14,scale=0.8]{02SetCovHitSet.pdf}
\end{center}
\caption{Point $p$ lies in a leaf cell $C$ (which may not even have any edge-covering squares). In this case, we pick an arbitrary square $S$ to cover the point (since one such square always exists).}
\label{fig:offline-edge}
\end{figure}

\begin{restatable}{lemma}{offfeasible}
    \label{lem:offfeasible}
    $\mathcal{A}_\text{off}$ outputs a feasible set cover for the points in $P$.
\end{restatable}
    %See \Cref{subsec:offfeasible}.
    \begin{proof} Assume for contradiction that no square in ALG covered some point $p \in P$.  Since
    $\OPT$ is a feasible set cover, there is a square $S\in\OPT$ which covered $p$. There are two cases to consider here: either $p$ is exactly at one of the corners of $S$, or not. In the latter case, note that $S$
    is edge-covering for at least one  quad-tree cell containing $p$. Let $C_v$ be such a cell
    (which contains $p$ and its edge $e$ is contained in $S$) with minimum depth. Now the algorithm
    will traverse $T$ till we reach the node $v$ (corresponding to cell $C_v$)  containing $p$. As the
    squares selected by the algorithm for the ancestors of $v$ do not cover $p$, we will select the
    maximum area-covering square $S'$ (for $e$) in ALG. As $(S \cap C_v) \subseteq (S'\cap C_v)$, $S'$
    will cover $p$. This is a contradiction. Now in the first case, i.e., where $p$ is at one of the corners of $S\in\OPT$, either there is a leaf $v\in T$ which contains it and $S$ is edge-covering for $C_v$, or for such a leaf $v$, $S$ is corner-covering. In both the cases, $\A_{\text{off}}$ will pick a square for $v$ or one of its ancestors such that this square covers $p$.
\end{proof}

%Example of how the Algorithm picks squares.
%The figure on the left shows a cell in the quad-tree in purple, and edge-covering squares for it in black.
%The maximum area-covering square (solid black) is picked, while the other squares (dashed) are not.
%The figure on the right shows the cells in the quad-tree (color coded according to level)
%containing an uncovered point. In increasing order of depth of these cells, maximum-area covering squares (solid black squares) are picked, till the point gets covered.



\paragraph*{Approximation ratio}
Let $\ALG\subseteq\S$ denote the selected set of squares and let $\OPT$ denote the optimal solution.
To prove the $O(\log N)$-approximation guarantee, the main idea is the following:
consider a node $v \in V$ and suppose that we selected at least one square in the iteration corresponding
to $v$. If $C_{v}$ contains a corner of a square $S\in\OPT$, then we charge the (at most four)
squares selected for $v$ to $S$. Otherwise, we argue that the squares selected for $v$ cover at
least as much of $C_{v}$ as the squares in $\OPT$, and that they cover all the remaining uncovered
points in $P\cap C_{v}$.  In particular, we do not select any further squares in the descendants of
$v$. The squares selected for $v$ are charged to the parent of $v$ (which contains a corner of a square $S \in \OPT$). Since each corner of each square
$S\in\OPT$ is contained in $O(\log N)$ cells, we show that each square $S\in\OPT$ receives
a total charge of $O(\log N)$.
Thus, we obtain the following lemma.
\begin{restatable}{lemma}{sqoffline}
    \label{lem:sqoffline}
    We have that $|\ALG|= O(\log N)\cdot|\OPT|$.
\end{restatable}
%\begin{proof}
%    See \Cref{subsec:sqoffline}.
%\end{proof}
\begin{proof} We will charge each square picked in $\ALG$ to some square in $\OPT$. A cell $C_v$
    with its corresponding node $v$, can either contain (at least) a corner of some square in $\OPT$,
    or be edge-covered by (at least) a square in $\OPT$, or not intersect any square from $\OPT$ at
    all.
    \begin{itemize}
        \item $C_v$ contains a corner of $S\in\OPT$: In this case,  $\mathcal{A}_\text{off}$ picks at
            most four squares for the cell, and we charge these squares to a corner of $S$ in the cell. If
            there are multiple squares from $\OPT$ with a corner in the cell, pick one arbitrarily. This claim is true even when $C_v$ corresponds to a leaf node.
        \item Some square $S\in\OPT$ is edge-covering for $C_v$ (and $C_v$ has no corner of a square in $\OPT$): If
            $\mathcal{A}_\text{off}$ picks no edge-covering squares for such a cell, then we are fine.
            Otherwise, if $\mathcal{A}_\text{off}$ picks squares for such a cell, we claim that it covers
            all points in the cell%\alr{may not cover all points in the cell but covers all points in $P\cap S$ in the cell}
            . This is due to the fact that any point in this cell is covered by a
            square in $\OPT$ that is edge-covering for $C_v$, due to the absence of corners of squares of
            $\OPT$. So when $\mathcal{A}_\text{off}$ picks edge-covering squares with the largest
            intersection with the cell, the intersection of any square $S'\in\OPT$ with $C_v$ will also
            get covered). So, no child node of $v$ will be further explored by the algorithm. %\alr{may get explored by virtue of other squares in OPT. Either need to remove this line or say that will get explored because of other squares in OPT}.
            This also means that the parent $v'$ of $v$ in the tree will contain a corner of $S$
            (because $C_{v'}$ intersects $S$, but cannot be edge-covered by it). We charge any squares
            picked by $\mathcal{A}_\text{off}$ at $C_v$ (at most four times) to this particular corner in
            the parent node. If there are multiple such corners, pick one arbitrarily.

        \item No squares from $\OPT$ intersect $C_v$: In this case, $C_v$ does not contain any points
            in $P$. Thus, $\mathcal{A}_\text{off}$ will not pick any squares for such a cell.
    \end{itemize}

    \begin{figure}[ht]
        \centering
        \includegraphics[page=4,scale=0.7]{02SetCovHitSet.pdf}
        \caption{Charging picked (red) edge-covering squares to the corner of a (cyan) square in $\OPT$.
        In the image on the left, the (yellow) cell contains a corner of the square from $\OPT$, and in the image on the right, the parent of the cell contains such a corner.}
        \label{fig:2Dsetcov-charging}
    \end{figure}

    Now we note that a corner of any square in $\OPT$, will lie in at most $\log N$ cells of the
    quad-tree. For each of these cells, a corner is charged at most four times for the squares picked
    at the cell, and at most four times for each of its four child nodes. This amounts to a total
    charge of at most 20 per corner per cell. So each square in $\OPT$ is charged at most $20 \text{
    (per corner, per cell)}\times4\text{ (corners)}\times\log N\text{ (cells)}=80\log N$ times.
    Therefore, there are at most $80\log N\cdot|\OPT|$ squares in $\ALG$.
\end{proof}

%\begin{proof}
% If $C_{v}$ does not contain the vertex of any square $S\in\OPT$, then any square $S\in\OPT$
% intersecting with $C_{v}$ must contain an edge $e$ of $C_{v}$. Hence, when we considered $v$, we
% selected a square $S'\in\S$ that covers at least as much of $C_{v}$ as $S$, i.e., $S\cap
% C_{v}\subseteq S'\cap C_{v}$. Thus, our squares selected for $v$ cover all remaining uncovered
% points in $P\cap C_{v}$, and we do not select any further squares in any of the descendants of $v$.
% Therefore, for the parent $v'$ of $v$ there must have been a square $S\in\OPT$ such that one of its
% vertices is contained in $C_{v'}$. We charge the squares selected for $v$ to $S$.
%\end{proof}

% ---------- ALTERNATE PROOF BY DECOMPOSITION ---------- %

% We first show that for any square $S\in\OPT$, we can find its decomposition
% $\mathcal{C}_S$ into at most  $16\log N$ cells such that the union of these cells covers
% $S$ and this decomposition consists of cells for (some edge of) which $S$ is edge-covering, or
% cells containing a corner of $S$. Note here that for any point $p$ covered by $S$, there will be
% cell in $C\in\mathcal{C}_S$, such that $C$ covers $p$, and $S$ is edge-covering for $C$. We
% compute the decomposition as follows: Initialize  $\mathcal{C}_S$ to be  $\emptyset$. Traverse $T$ starting from the root in a BFS fashion and
% consider only cells which have some intersection with $S$.  If for a certain cell $C\in T$, $S\cap C$ is edge-covering for it, add
% this cell $C$ to $\mathcal{C}_S$ and ignore the rest of the subtree of $C$ in the BFS. Also if a
% cell $C$ contains at least one corner of $S$, add it to $\mathcal{C}_S$ and explore the subtree of
% $C$ in a BFS fashion. It is easy to see that if a cell $C$ is such that  $S$ is edge-covering for
% $C$ but not for its parent, then $C$ along with all of its ancestors are included in $\mathcal{C}_S$
% (since the cells containing corners of $S$ are also in $\mathcal{C}_S$).
% Now we prove the following two lemmas.

% \begin{lemma}
% \label{lem:Sqdecom}
% For any square $S\in \S$, $|\mathcal{C}_S|\leq 16\log N$.
% \end{lemma}
% \begin{proof}
% See \Cref{subsec:Sqdecom} for the proof.
% \end{proof}


% % Now, consider the points (introduced by the adversary) covered by a square $S\in\OPT$. Denote this
% set by $P_S$. For any point $p\in P_S$, it is contained in a cell $C$ in $\mathcal{C}_S$ for which
% $S$ was maximum area-covering. Then, we prove that $\mathcal{A}_\text{off}$ outputs a feasible set cover.
% Then, in the run of the algorithm, we claim that $p$ was surely covered when the algorithm scanned
% this cell of $T$ since it picked edge-covering maximal squares for this cell.
% $\mathcal{A}_\text{off}$ might scan cells in $T$ for which $S$ had a corner in it. But such cells
% are also included in $\mathcal{C}_S$. Thus, the squares which $\mathcal{A}_\text{off}$ picked while
% % traversing exactly those cells in $\mathcal{C}_S$ can be mapped to $S$ and these squares picked up
% % by ALG covered all of the points that $S$ was covering.  {\color{red}para a little unclear}

% \begin{lemma}
% \label{lem:offfeasible}
%     $\mathcal{A}_\text{off}$ outputs a feasible set cover for the points in $P$.
% \end{lemma}
% \begin{proof}
% See \Cref{subsec:offfeasible}.
% \end{proof}

% % We now define a mapping $f:\OPT\rightarrow2^{\ALG}$. For any square $S\in \OPT$, denote by
% $\mathcal{C}_S'\subseteq \mathcal{C}_S$ the cells in $\mathcal{C}_S$ that the algorithm traversed
% in its run. Now, define $f(S)$ to be the set of squares that algorithm picked while traversing the
% cells in $\mathcal{C}_S'$. Since the algorithm picks at most $4$ squares per cell of $\mathcal
% {C}_S$, we have the following claim.

% \begin{lemma}
% \label{lem:sqOnlmap1}
%     To any square $S\in\OPT$, $f$ maps at most $64\log N$ squares of $\ALG$, such that their union
%     covers all points covered by $S$.
% \end{lemma}
% \begin{proof}
%   See \Cref{subsec:sqOnlmap1}.
% \end{proof}


% Finally, we need to prove that according to the previously defined mapping $f$, every square in ALG
% was mapped back to at least one square in OPT.
% \begin{lemma}
% \label{lem:sqOnlmap2}
%     Every square in $\ALG$ is mapped under $f$, from some square in $\OPT$.
% \end{lemma}
% \begin{proof}
%     See \Cref{subsec:sqOnlmap2}.
% \end{proof}
% Lemmas \Cref{lem:sqOnlmap1} and \Cref{lem:sqOnlmap2} together completes the proof.

\subsection{Online set cover for squares}

\subsubsection{$O(\log N)$-approximate online algorithm}

We want to turn our offline algorithm $\mathcal {A}_\text{off}$ into an online algorithm $\mathcal{A}_\text{on}$, assuming that in each {\em round} a new point is 
introduced by the adversary.
 The key insight for
this is that the algorithm above is \emph{monotone}, i.e., if we add a point to $P$, then it
outputs a superset of the squares from $\S$ that it had output before (when running it on $P$
only).  For a given set of points $P$, let $\ALG(P)\subseteq\S$ denote the set of squares that our
(offline) algorithm outputs.
\begin{restatable}{lemma}{monotone}
    \label{lem:monotone}
    Consider a set of points $P$ and a point $p$. Then $\ALG(P)\subseteq\ALG(P\cup\{p\})$.
\end{restatable}
%\begin{proof}
 %   See \Cref{subsec:monoproof} for the proof.
%\end{proof}
\begin{proof} Assume towards contradiction that there exists some square $S$ in $\ALG(P)$ which did
    not belong to $\ALG(P\cup\{p\})$. According to the description of $\mathcal {A}_\text{off}$, we
    can infer that $S$ was picked by the algorithm in some iteration because it was maximum
    area-covering for some cell $C_v$ (corresponding to node $v$ in $T$) that contained a point $p' \in P$
    introduced by the adversary. Also, $\mathcal{A}_\text{off}$ in its run must have explored all the
    ancestors of $v$ in $T$.  Note that any such point $p'$ could be
    covered in a run of the algorithm only when it traverses cells that contain $p'$. This is due to
    the fact that
    %in Step 7 of $\mathcal{A}_\text{off}$,
    once we pick some squares associated with a cell in the quad-tree, we only account for the area
    inside this cell that the squares cover. In light of this fact, %\al{wherever the adversary
    %introduces a new point $p$, $\mathcal{A}_\text{on}$ uses $\mathcal{A}_\text{off}$ as subroutine
    %on the points $P\cup\{p\}$ and }%\alr{remove this statement as $\mathcal{A}_\text{on}$ is not needed here}
    if $\mathcal{A}_\text{off}$ did not explore $C_v$ in this time
    step, then it also would not have explored the children of $v$ in $T$. Hence, the point $p'$
    would not have been covered which is a contradiction.
\end{proof}


Hence, it is easy now to derive an online algorithm for set cover for squares. Initially,
$P=\emptyset$. If a point $p$ is introduced by the adversary, then we compute $\ALG(P)$ (where $P$ denotes the set of previous points, i.e., \emph{without} $p$) and $\ALG(P\cup\{p\})$
and we add the squares in $\ALG(P\cup\{p\})\setminus\ALG(P)$ to our solution.
Therefore,
due to \Cref{lem:sqoffline} and \Cref{lem:monotone} we obtain
an $O(\log N)$-competitive online algorithm. 

%\akr{commented the bbd tree intuition}
%%%%%%%%%%%%%%%
\begin{comment}
Suppose now, that a set of $n$ points $P_0$ is given, such that the adversary can insert only points from $P_0$. We want to get a competitive ratio of $O(\log n)$ in this case. 
If $N=n^{O(1)}$ then this is immediate.  Otherwise, we extend our
algorithm such that it uses the BBD-tree data structure from \cite{arya1998optimal}
instead of the quad-tree. We build a BBD-tree of depth $O(\log n)$
for the $n$ points in $P_{0}$. Each node of that tree corresponds
to a cell (like our quad-tree); however, such a cell might not be
a square but it can be a rectangle with an aspect ratio of up to 3,
possibly with a hole which is another rectangle of aspect ratio
at most 3. Thus the cells are more complicated and,
therefore, our procedure for selecting appropriate squares is more technically 
involved. 
E.g., if a cell does not contain any corner of an intersecting square, we can not say that the square is edge-covering, as the square could be {\em crossing} or have (one or two) vertices in the hole (see \Cref{fig:bbd-opt-intersection}). 
% However, with nontrivial adaptations, given such a cell $C_{v}$, we show how to select at most $O(1)$ squares
% for which we can still argue that if $C_{v}$ does not contain a vertex
% of a square in $\OPT$, then we cover all the remaining uncovered points
% in $P\cap C_{v}$. See \Cref{sec:bbdscon} for a detailed description.
However, with non-trivial adaptations, we show how to pick $O(1)$ squares such that
a required property for the proof is satisfied. The property being that, if $C_{v}$ does not contain a corner
of a square in $\OPT$, then we cover all the remaining uncovered points
in $P\cap C_{v}$. %See \Cref{sec:bbdscon} for a detailed description.
We describe the details in the next subsection.
\end{comment}
%%%%%%%%%%%%%

\subsubsection{$O(\log n)$-approximate online set cover for squares \label{sec:bbdscon}}

We assume now that we are given a set $\tilde{P}\subseteq\R^{2}$ with $|\tilde{P}|=n$ such that in
each round a point from $\tilde{P}$ is inserted to $P$,  %\al{or deleted from $P$}, 
i.e., $P\subseteq\tilde
{P}$ after each round. 
We want to get a competitive ratio of $O(\log n)$ in this case. 
If $N=n^{O(1)}$ then this is immediate.  Otherwise, we extend our
algorithm such that it uses the balanced box-decomposition tree (or BBD-tree) data structure due to Arya et al.~\cite{arya1998optimal}, 
instead of the quad-tree. 
Before the first round, $P=\emptyset$ and we initialize the BBD-tree which 
yields a tree $\tilde{T}=(\tilde{V},\tilde{E})$ with the following properties:
\begin{itemize}
    \item each node $v\in\tilde{V}$ corresponds to a cell $\tilde{C}_{v}\subseteq[0,N)^{2}$ which is
        described by an outer box $b_{O}\subseteq[0,N)^{2}$ and an inner box $b_{I}\subseteq b_{O}$; both
        of them are axis-parallel rectangles and $\tilde{C}_{v}=b_{O}\setminus b_{I}$
        (Note that $b_I$ could be the empty set).
    \item the aspect ratio of $b_{O}$, i.e., the ratio between the length of the longest edge to the
        length of the shortest edge of $b_{O}$, is bounded by 3.
    \item if $b_{I}\ne\emptyset$, then $b_{I}$ is \emph{sticky }which intuitively means that in each
        dimension, the distance of $b_{I}$ to the boundary of $b_{O}$ is either 0 or at least the width of
        $b_{I}$. Formally, assume that $b_{O}=[x_{O}^{(1)},x_{O}^{(2)}]\times[y_{O}^{(1)},y_{O}^{
        (2)}]$ and $b_{I}=[x_{I}^{(1)},x_{I}^{(2)}]\times[y_{I}^{(1)},y_{I}^{(2)}]$. Then $x_{O}^{(1)}=x_
        {I}^{(1)}$ or $x_{I}^{(1)}-x_{O}^{(1)}\ge x_{I}^{(2)}-x_{I}^{(1)}$. Also $x_{O}^{(2)}=x_{I}^{
        (2)}$ or $x_{O}^{(2)}-x_{I}^{(2)}\ge x_{I}^{(2)}-x_{I}^{(1)}$. Analogous conditions also hold for
        the $y$-coordinates.
    \item each node $v\in\tilde{V}$ is a leaf or it has two children $v_{1},v_{2}\in\tilde{V}$; in the
        latter case $\tilde{C}_{v}=\tilde{C}_{v_{1}}\dot{\cup} \tilde{C}_{v_{2}}$.
    \item the depth of $\tilde{T}$ is $O(\log n)$ and each point $q\in[0,N)^{2}$ is contained in 
        $O(\log n)$ cells.
    \item each leaf node $v\in\tilde{V}$ contains at most one point in $\tilde{P}$.
\end{itemize}

In the construction of the BBD-tree, we make the cells at the same depth disjoint so that a point $p$ may be contained in exactly one cell at a certain depth. 
Hence, for a cell $\tilde{C}_{v}=b_{O}\setminus b_{I}$ we assume both $b_{O}$ and $b_{I}$ to be {\em closed} set, i.e., the boundary of the outer box $b_{O}$ is part of the cell and the boundary of the inner box $b_{I}$ is {\em not} part of the cell. 
%Hence, we assume for every inner child $v_I$ and the corresponding outer child $v_O$, their boundary is contained only inside $v_O$.}
We now describe an adjustment of our offline algorithm from \Cref{sec:Set-cover-squares}, working
with $\tilde{T}$ instead of $T$. Similarly, as before, we traverse $\tilde{T}$ in a
breadth-first-order. Suppose that in one iteration we consider a node $v\in\tilde{V}$ corresponding
to a cell $\tilde{C}_{v}$. We check whether the squares selected in the ancestors of $v$ cover all
points in $P\cap\tilde{C}_{v}$. If this is the case, we do not select any squares from $\S$ in this
iteration corresponding to $v$.

Suppose now that the squares selected in the ancestors of $v$ do
\emph{not }cover all points in $P\cap\tilde{C}_{v}$. 
%Let $b_{O}$ and $b_{I}$ denote the two
%axis-parallel rectangles such that $\tilde{C}_{v}=b_{O}\setminus b_{I}$. 
Similar to \Cref{sec:Set-cover-squares}, we want to select $O(1)$ squares for $\tilde{C}_{v}$
such that if $\tilde{C}_{v}$ contains no corner of a square $S\in\OPT$, then the squares we selected for $\tilde{C}_{v}$ should cover all points in $P\cap\tilde{C}_{v}$. %\alr{Every point may not be covered}. 
Similarly as before, for each edge $e$ of $b_{O}$ we select a square from $\S$
that contains $e$ and that has the largest intersection with $b_{O}$ among all such squares in
$\S$. We break ties in an arbitrary but fixed way.
However, as $\tilde{C}_{v}$ may not be a square and can have holes (due to $b_I$), apart from the edge-covering squares, we need to consider two additional types of squares in $\OPT$  with nonempty overlap with $\tilde{C}_{v}$:
(a) crossing $\tilde{C}_{v}$, i.e., squares that intersect two parallel edges of $b_O$; (b) has one or two corners inside $b_I$.

%\akr{chnaged this. please check.}
The following \textit{greedy subroutine} $\mathcal{G}$ will be useful in our algorithm to handle such problematic cases.
Let $R$ be a  box of width $w$ and height $h$ such that $w/h \le B$, for some constant $B \in \mathbb{N}$; and $P_R$ be a set of points inside $R$ that can be covered by a collection of vertically-crossing (i.e., they intersect both horizontal edges of $R$) squares $\S'$. 
Then, the set of squares picked according to $\mathcal{G}$ covers $P_R$ in the following way:
\begin{itemize}
    % \item Consider the leftmost point $p\in P_R$ that is not already covered by a square that this greedy routine has already selected.
    % \item We select the vertically-crossing square intersecting $p$ (by assumption, such a square exists) with rightmost right edge.
    % \item We repeat this procedure until there is no uncovered point $p'\in P_R$ left.
    \item While there is an uncovered point $p'\in P_R$:
    \begin{itemize}
        \item Consider the leftmost such uncovered point $p\in P_R$.
        \item Select the vertically-crossing square intersecting $p$ (by assumption, such a square exists) with the rightmost edge.
    \end{itemize}
\end{itemize}
(The above subroutine is for finding vertically-crossing squares. For finding horizontally-crossing squares, we can appropriately rotate the input $90^{\circ}$ anti-clockwise, and apply the same subroutine.)
Then, we have the following claim about the aforementioned subroutine.
\begin{claim}
\label{cl:crossbbd}
Let $R$ be a  box of width $w$ and height $h$ such that $w/h \le B$, for some constant $B \in \mathbb{N}$; and $P_R$ be a set of points inside $R$ that can be covered by a collection of vertically-crossing (i.e., they intersect both horizontal edges of $R$) squares $\S'$. Then we can find at most $B+1$
% an $O(1)$ number of 
squares from $\S'$ that can cover all points inside $R$. 

We have an analogous claim for horizontally-crossing squares when $h/w \le B$.
\end{claim}
\begin{proof}

%  %suppose w.l.o.g.~that the two longer
% %edges of $R$ are horizontal. 
% Consider the leftmost point $p\in P_R$ that is not already
% covered 
% %by a previously selected square or 
% by a square that this greedy routine has already selected. We select
% the vertically-crossing square intersecting $p$ (by assumption, such a square exists) with rightmost right edge.
% We repeat this procedure until %we find an uncovered point $p'\in P_R$ that is not contained in
% %a crossing square or if 
% there is no uncovered point $p'\in P_R$ left.
% %Since
% %the aspect ratios of $b_{O}$ and $b_{I}$ are at most $B$ and $b_{I}$ is sticky, we can show that
% %also the aspect ratio of $R$ is bounded by $B$ as well. Using this, we can show that our greedy
% %routine selects at most $B+1$ crossing squares.

Consider each iteration of the greedy subroutine $\mathcal{G}$. We call {\em pivot} to be  the leftmost point $p\in P_R$ that is not already
covered 
%by a previously selected square or  
by a square selected by $\mathcal{G}$ so far. Then all selected vertically-crossing squares for $R$ % by the greedy subroutine
will contain exactly one point that was identified as a pivot point at some point during the execution of the algorithm. As the aspect ratio is bounded by $B$ and the squares are vertically-crossing (i.e., their vertical length  is more than the vertical length of $R$), there can be at most $B+1$ pivot points. Hence, we select at most $B+1$ crossing squares  due to $R$. 
This produces a feasible set cover. 
\end{proof}

%\akr{added more explanation below. please check.}
Now we describe our algorithm. 
First, we take care of the squares that can cross $b_O$. So, we apply the greedy subroutine $\mathcal{G}$ on $b_O$. As $b_O$ has bounded aspect ratio of 3, from Claim~\ref{cl:crossbbd}, we obtain at most $(3+1)+(1+1)=6$ squares that can cross $C_v$ vertically or horizontally. 
If $b_I=\emptyset$, we do not select any more squares. 
%Now we consider the squares which can have one or two corner points inside $b_I$. 
Otherwise, we need to take care of the squares that can have one or two corners inside $b_I$. 
Let $\ell_
{1},\ell_{2},\ell_{3},\ell_{4}$ denote the four lines that contain the four edges of $b_
{I}$. Observe that $\ell_{1},\ell_{2},\ell_{3},\ell_{4}$ partition $b_{O}$ into up to nine
rectangular regions, one being identical to $b_{I}$.
For each such rectangular region $R$, if it is sharing a horizontal edge with $b_I$, we again use $\mathcal{G}$ to select vertically-crossing squares. Otherwise, if $R$ is sharing a vertical edge with $b_I$, we use the subroutine $\mathcal{G}$ appropriately to select horizontally-crossing squares.
This takes care of squares having two corners inside $b_I$. 
Otherwise, if the rectangular region $R$ does not share an edge with $b_I$, then we check if there is a square  $S \in \S$ with a corner within $b_I$ that completely contains $R$.   We add $S$ to our solution too. This finally takes care of the case when a square has a single  corner inside $b_I$.  

Finally, to complete our algorithm, before its execution, we do the following: for every leaf $v$ for which $C_v$ contains at most one point $p\in \tilde{P}$, we associate a fixed square which covers $p$. Then, if our algorithm reaches a leaf $v$ while traversing that has an uncovered point $p$, we pick the associated square with this leaf that covers it. This condition in our algorithm guarantees feasibility.

\begin{figure}[ht]
    \centering
    \includegraphics[page=12,scale=0.6]{02SetCovHitSet.pdf}
    \caption{Outer box $b_O$ being partitioned into at most 9 rectangles due to inner box $b_I$.}
    \label{fig:bbd-boxes}
\end{figure}

\begin{figure}[ht]
    \centering
    \includegraphics[page=13,scale=0.6]{02SetCovHitSet.pdf}
    \caption{Possible intersections of a (cyan) square from $\OPT$ with a cell, such that no corner of the square is in the cell. The left image shows edge-covering, and crossing squares. The right image shows squares with one of two corners  inside $b_I$.}
    \label{fig:bbd-opt-intersection}
\end{figure}

%%%%%%%%%%%%
\begin{comment}
First, assume that $b_
{I}=\emptyset$.
Similarly as before, for each edge $e$ of $b_{O}$ we select a square from $\S$
that contains $e$ and that has the largest intersection with $b_{O}$ among all such squares in
$\S$. We break ties in an arbitrary but fixed way. Assume now that $b_{I}\ne\emptyset$. Let $\ell_
{1},\ell_{2},\ell_{3},\ell_{4}$ denote the four lines that contain the four edges of $b_
{I}$. Observe that $\ell_{1},\ell_{2},\ell_{3},\ell_{4}$ partition $b_{O}$ into up to nine
rectangles, one being identical to $b_{I}$. For each such rectangle $R$ that is \emph
{not }identical to $b_{I}$, we first do a similar operation as we did for $b_{O}$ for the case
that $b_{I}=\emptyset$: for each edge $e$ of $R$ we select a square from $\S$ that contains $e$
and that has the largest intersection with $R$ among all such squares in $\S$ (if such a square
exists, breaking ties arbitrarily). In addition, if $R$ is not a square, consider the input
squares that intersect both of the two longer edges of $R$. We call these squares the \emph
{crossing squares}. Our goal is to check whether we can cover all remaining uncovered points in
$R$ with crossing squares. We do this with a greedy routine: suppose w.l.o.g.~that the two longer
edges of $R$ are horizontal. Consider the leftmost point $p\in P\cap R$ that is not already
covered by a previously selected square or by a square that this greedy routine selects. We select
the crossing square intersecting $p$ (if such a crossing square exists) with rightmost right edge.
We repeat this procedure until we find an uncovered point $p'\in P\cap R$ that is not contained in
a crossing square or if there is no uncovered point $p'\in P\cap R$ left.
% In the former case, we
% stop the procedure and do not add any crossing square to our solution that was selected by this
% greedy routine. In the latter case, we add all selected crossing squares to our solution.
Since
the aspect ratios of $b_{O}$ and $b_{I}$ are at most 3 and $b_{I}$ is sticky, we can show that
also the aspect ratio of $R$ is bounded by 3 as well. Using this, we can show that our greedy
routine selects at most 4 crossing squares.
As in \Cref{sec:Set-cover-squares}, the algorithm produces a feasible set cover. 
\end{comment}
%%%%%%%

\begin{restatable}{lemma}{bbdstruct}
\label{lem:bbdstruct}
Let $\tilde{C}_{v}$ be a cell such that the squares selected in the ancestors of $v$
    do \emph{not }cover all points in $P\cap\tilde{C}_{v}$. Then
    \begin{enumerate}
        \item[(a)] we select at most $O(1)$ squares for $\tilde{C}_{v}$ and
        \item[(b)] if $\tilde{C}_{v}$ contains no corner of a square $S\in\OPT$, then the squares we selected for
            $\tilde{C}_{v}$ cover all points in $P\cap\tilde{C}_{v}$.
    \end{enumerate}
\end{restatable}
\begin{proof}
First, we prove part (a). 
If $\tilde{C}_{v}$ corresponds to a  leaf node, we select at most one square. 
Otherwise, We select at most 4 edge-covering squares for $\tilde{C}_{v}$.
From Claim~\ref{cl:crossbbd}, we select $O(1)$ number squares for $\tilde{C}_{v}$ that are horizontally or vertically-crossing $b_O$. 
We select no more squares if  $b_{I}= \emptyset$.

So consider the other case:  $b_{I}\neq \emptyset$. Let $R$ be one of the (at most) four rectangular regions obtained from partitioning of $b_{O}$  (by $\ell_{1},\ell_{2},\ell_{3},\ell_{4}$) that share an edge with $b_I$.
Let $w,h$ be width and height of $R$, respectively. 
W.l.o.g.~assume $R$ shares a horizontal edge with $b_I$.
As  $b_{I}$ is sticky, and $b_{I}$ and $b_{O}$ have a bounded aspect ratio of 3, it can be seen that $R$ also has a $w/h \le 3$ (similarly, if $R$ shared a vertical edge with $b_I$, then $h/w \le 3$).
Again using Claim~\ref{cl:crossbbd}, we select $O(1)$ vertically-crossing squares for $R$.
We do a similar operation for other such regions. 
Now consider the remaining (at most four) regions obtained from partitioning of $b_{O}$  (by $\ell_{1},\ell_{2},\ell_{3},\ell_{4}$) that do not share an edge with $b_I$.
We select at most one square for each of them. 
Thus, in total, we select at most $O(1)$ squares for $\tilde{C}_{v}$. 

Now we prove part (b). 
If $\tilde{C}_{v}$ contains no corner of a square $S\in\OPT$, then all the squares in $\OPT$ that intersect $\tilde{C}_{v}$ are either edge-covering $b_O$, or crossing $\tilde{C}_{v}$, or contain one or two corners inside $b_I$. 
However, as we have picked maximal edge-covering squares for $b_O$, they contain all points in $\tilde{C}_{v}$ that are covered by the edge-covering squares from $\OPT$.

Similarly, by our selected  squares (that vertically or horizontally crosses $b_O$) in the greedy subroutine $\mathcal{G}$, we have covered all points that can be covered by such crossing squares in $\OPT$ that crosses $b_O$.
% AS: I think the above line can be given more explanation later.

For squares that have (at least) a corner inside $b_I$, note that they have to cross one of the rectangular regions  that came from partitioning of $b_O$ and shares an edge with $b_I$. 
In fact, for such a square $S$ with a corner inside $b_I$, there is a rectangular region $R$ (say, with width $w$ and height $h$) among these four rectangular regions such that either $S$ is vertically-crossing for $R$ and $R$ shared a horizontal edge with $b_I$ (then $w/h\le 3$) or $S$ is horizontally-crossing for $R$ and $R$ shared a vertical edge with $b_I$ (then $h/w\le 3$). 
But then using Claim~\ref{cl:crossbbd}, we cover all points covered by such squares.
Additionally, a square that has exactly one corner inside $b_I$ may completely contain another rectangular region from partitioning of $b_O$ (that do not share an edge with $b_I$). For them, again we have covered them by selecting one square, if it exists.  
\end{proof}


%In this way, we achieve the following properties for each cell $\tilde{C}_{v}$:
%\begin{itemize}
%\item we select at most $O(1)$ squares
%\item if $\tilde{C}_{v}$ contains no vertex of a square $S\in\OPT$, then
%the squares we selected for $\tilde{C}_{v}$ cover all points in $P\cap\tilde{C}_{v}.$
%\end{itemize}

%\al{Currently, our algorithm does not output a feasible solution in general because it is not guaranteed that if a square $S\in\OPT$ covers a point $p$, then it is crossing for at least one cell $C$ in the BBD tree which contains this point. For fixing, we need to add the following statement: If for a leaf, after executing our current algorithm, there remains an uncovered point, then cover it using a fixed square (fixed in the sense that it is pre-decided for the cell). We can guarantee such a fixed square to cover this potentially uncovered point because the point set is fixed.}


Thus, we can establish a similar charging scheme as in \Cref{sec:Set-cover-squares}. To pay for our
solution, we charge each corner $q$ of a square $S\in\OPT$ at most $O(\log n)$ times. Hence, our
approximation ratio is $O(\log n)$. Similarly as in \Cref{sec:Set-cover-squares}, we can modify the
above offline algorithm to an online algorithm with an approximation
ratio of $O(\log n)$ each.

%\al{To prove the charging argument we need a claim of the form: If a square $S$ is crossing/edge-covering for say, more than $16$ cells at the same depth in the BBD tree, then it is crossing/edge-covering for some ancestor of at least one of these cells.}


\begin{restatable}{theorem}{squaresetcoverone} \label{squaressetcover_1}
There is a deterministic $O(\log n)$-competitive online algorithm
for set cover for axis-parallel squares of arbitrary sizes.
\end{restatable}

%\akr{we can probably remove the folowing two lines}
%Note that if any point with integral coordinates in $[0,N)^2$ might be inserted, then we can define $P_0$ to be all these points, and obtain a  competitive ratio of $O(\log N)$.

\begin{comment}
    Therefore, from \Cref{lem:sqoffline} and \Cref{lem:monotone}, we obtain the following theorem.
    \begin{theorem}
        There is an $O(\log N)$-competitive online algorithm for set cover for squares.
    \end{theorem}
    For simplicity, we assume $N=m^{O(1)}$. However, we can remove this assumption by using BBD-tree \cite{arya1998optimal} instead of the quad-tree.

    \begin{theorem}
        There is an $O(\log m)$-competitive online algorithm for set cover for squares.
    \end{theorem}
    \begin{proof}
        See \Cref{sec:bbdscon}.
    \end{proof}
\end{comment}

\subsection{Lower bounds}


It is a natural question whether  algorithms having a competitive factor better than $O(\log n)$ are possible for online set cover for squares. We answer this question in the negative, even for the case of unit squares and even for randomized algorithms. We also remark here that there exists a tight $2$-competitive online algorithm for set cover for intervals (see Section~\ref{sec:interval_upper}).
%with fractional set cover variant.


\subsubsection{Unit squares and quadrants}
%\begin{definition}[Quadrant]
 %   A rectangle with one of its corners as the origin is called a quadrant.
%\end{definition}

Given a set cover instance $(P, \F)$, for each $F\in\F$ define a variable $x_F$ which takes values
$\in[0,1]$. For a point $p\in P$, let $\F_p\subseteq\F$ be the sets that cover it. In the
{\em fractional set cover} problem, the aim is to assign values to the variables $x_F$ such that
for all points $p\in P$, $\sum_{F\in\F_p} x_F\ge1$.


\begin{figure}[ht]
    \center
    \includegraphics{01QuadHalfLogHard}
    \caption{$\Omega(\log m)$ hard instances for quadrants and unit squares}
    \label{fig:LogHard}
\end{figure}

\label{subsec:sqLower}
\begin{restatable}{lemma}{quadlower} \label{lem:quad_lower}
    There is an instance of the online fractional covering problem on $m$ quadrants (quadrant is a  rectangle with one of its corners as the origin) such that any
    online deterministic algorithm is $\Omega(\log m)$-competitive on this instance.
\end{restatable}
\begin{proof} Consider the set of $m$ quadrants (See \Cref{fig:LogHard}) with their top-right
    corners as follows: $Q_1=(1, m), Q_2=(2, m-1), \dots, Q_k=(k, m-k+1), \dots, Q_m=(m, 1)$ (and let
    the corresponding variables for the set cover instance be $x_1, x_2, \dots, x_m$,  respectively).
    Now consider a point $p_{i,j}=(i-0.5, m-j+0.5)$. We claim that this point intersects exactly the
    $i^{th}$ to $j^{th}$ indexed quadrants. Since $p(x)>i-1$, $p_{i,j}$ does not intersect any of the
    first $(i-1)$ quadrants. Additionally since $p(y)=m-j+0.5<m-i-1$ it does intersect quadrant $i$
    and this holds true up to quadrant $j$ (Since, $p(y)=m-j+0.5<m-i-1$). Also $p_{i,j}$ does not
    intersect any of the last  $m-j+1$ quadrants because $p(y)>m-j$.

    Now consider an adversary that introduces points as follows: $P_1=p_{1,m}$. If the algorithm
    assigns values to the variables such that $\sum_{i=1}^{m/2}x_i\ge\sum_{i=m/2+1}^{i=m} x_i$, then the
    point $P_2=p_{m/2+1,m}$ is given, otherwise the point $P_2=p_{1,m/2}$ is given. The adversary
    repeats this process of halving the set of quadrants intersected, and puts the next point in
    the range with the lower sum, till only one quadrant, say quadrant $j$ is left.

    The optimal solution would have been to assign only $x_j$ to 1 and the remaining variables to
    zero. But any online algorithm can only halve the set of the potential optimal solution in each
    step, while assigning at least $1/2$ ``cost'' to the non-optimal quadrants. Hence, the cost of any
    online algorithm is $\Omega(\log m)$.
\end{proof}
\begin{corollary}
\label{lb:unitsquares}
    There is an instance of the online fractional covering problem on $m$ unit squares
    such that any deterministic online algorithm is $\Omega(\log m)$-competitive on this instance.
\end{corollary}
\begin{proof}
    We can appropriately extend the quadrants from \Cref{lem:quad_lower} in the bottom-left
    direction to obtain a similar instance on squares with side-length of $m$. More precisely,
    let the bottom left corner of the square $i$ corresponding to quadrant $i$ be $(i-m,1-i)$.
    The points introduced by the adversary are the same as in the quadrants instance.

    Now scale down this instance on squares appropriately, by a factor of $m$, to get the required
    unit square instance.
\end{proof}

%\akr{added this lemma}
Using standard techniques, as in \cite{bienkowski2020unbounded}, we can extend the lower bound for deterministic algorithms for the fractional variant to the lower bound for randomized algorithms for the integral variant. 

\begin{corollary}
    There is an instance of the online (integral) set cover problem on $m$ unit squares
    such that any randomized online algorithm is $\Omega(\log m)$-competitive on this instance.
\end{corollary}

Since in our lower bound construction, $n=\Theta(m^2)$, $\log n=\Theta(\log m)$ and hence, we have the theorem as stated below.

\begin{theorem}
    Any deterministic or randomized online algorithm for set cover for unit squares has a competitive ratio of $\Omega(\log n)$,
    even if all squares contain the origin and all points are contained in the same quadrant.
\end{theorem}
%\begin{proof}
    %See \Cref{subsec:sqLower} for the proof.
%\end{proof}





\section{Online hitting set for squares \label{sec:hit-set-squares}}

We present our online algorithm for hitting set for squares. We assume that we are given a fixed set
of points $P\subseteq[0,N)^{2}$ with integral coordinates. We maintain a set $P'$ of selected points
such that initially $P':=\emptyset$. In each round, we are given a square $S\subseteq[0,N)^{2}$ whose
corners have integral coordinates. 
%If $S\cap P'=\emptyset$ then we need to add at least one point $p\in P\cap S$ to $P'$.

We assume w.l.o.g.~that $N$ is a power of 2. Let $Q$ be all points with integral coordinates in
$[0,N)^{2}$, i.e., $P\subseteq Q$. 
%Note that not necessarily $Q\subseteq P$. 
For each point $q\in Q$ we say that
$q=(q_{x},q_{y})$ \emph{is of level $\ell$ }if both $q_{x}$ and $q_{y}$ are integral multiples of
$N/2^{\ell}$. We build the same quad-tree as in \Cref{sec:Set-cover-squares}.
%with the corresponding cells. 
We say that a cell $C_{v}$ \emph{is of level }$\ell$ if its height and width equal
$N/2^{\ell}$.

We present our algorithm now. Suppose that in some round a new square $S$ is given. If $S\cap
P'\ne\emptyset$ then we do not add any point to $P'$. Suppose now that $S\cap P'=\emptyset$. Let $q$
be a point of smallest level among all points in $Q\cap S$ (if there are many 
such points, then we select an arbitrary point in $Q\cap S$ of smallest level).
Intuitively, we interpret $q$ as if it were the origin and partition the plane into four {\em quadrants}. We define $O_{TR}:=\{(p_{x},p_{y})\mid p_{x}\ge q_{x}\, ,\, p_{y}\ge q_{y}\}$, and
$S_{TR}:=O_{TR}\cap S$, and define similarly $O_{TL},O_{BR},O_{BL}$,  and $S_{TL},S_{BR},S_{BL}$.
Consider $O_{TR}$ and $S_{TR}$. For each level $\ell=0,1,\dots,\log N$, we do the following. Consider
each cell $C$ of level $\ell$ in some fixed order  such that $C\subseteq O_{TR}$ and $S_{TR}$ is edge-covering
for some edge $e$ of $C$. Then, for each edge identify the point $p_b$
($p_t, p_l, p_r$, resp.) in $P\cap C$ that is closest to its bottom (top, left, and right, resp.)
edge. We add these (at most 4) points to our solution if at least one of $p_b,p_t,p_l,p_r$ is contained in $S_{TR}$ (see \Cref{fig:HitSet2D}).
% Since $S_{TR}$ contains the bottom-left corner of $O_{TR}$, the edge $e$
% must be the bottom edge or the left edge of $C$. If there is such a cell $C$ of level $\ell$ then we
% identify the point $p_{B}\in P\cap C\cap S_{TR}$ that is closest to the bottom edge of $C$ and add
% it to $P'$, assuming that such a point exists. Similarly, we identify the point $p_{L}\in P\cap
% C\cap S_{TR}$ that is closest to the left edge of $C$ and add it to $P'$, again assuming that such a
% point exists.
% We do this also for the right edge and the top edge of $C$ and pick at most $4$ points.
If we add at least one such  point $p$ of the cell $C$ to $P'$ in this way, we say that $C$ gets
\emph{activated}. Note that we add possibly all of the points $p_b,p_t,p_l,p_r$ to $P'$ even though only one may be contained in $S_{TR}$. This is to ensure that $C$ gets activated at most once during a run of the online algorithm. This will be proved in Claim~\ref{claim:activate_once}, which will ultimately help us prove that our algorithm is $O(\log N)$-competitive. If for the current level $\ell$ we activate at least one cell $C$ of level
$\ell$, then we stop the loop and do not consider the other levels $\ell+1,\dots,\log N$. Otherwise,
we continue with level $\ell+1$.  We do a symmetric operation for the pairs ($O_{TL},S_{TL}$),
($O_{BR},S_{BR}$), and ($O_{BL},S_{BL}$).
% \awr{a figure would be nice I think}
We now prove  the correctness of the algorithm and that its competitive ratio is $O(\log N)$.

\begin{figure}[!ht]
    \center
    \includegraphics[scale=0.6, page=9]{02SetCovHitSet.pdf}
    \caption{In cell $C$ lying in $O_\text{TR}$ the red points are chosen by the algorithm.}
    \label{fig:HitSet2D}
\end{figure}
%\akr{fig and alg is different}
%\alr{Might be a better idea to move this figure to the first page in hitting set section}
 %The proofs for this section can be found in \Cref{subsec:sqhittingset}. 
%We
%first prove that it outputs a feasible solution.
\begin{restatable}{lemma}{sqhittingsetone}
%\Cref{lem:sq}
\label{lem:sqhittingset_1}
After each round, the set $P'$ is a hitting set for the squares that have been added so far.
\end{restatable}
\begin{proof}
    We will prove that in case $S\cap P'=\emptyset$ when a square $S$ is inserted, at least one cell $C$ gets activated. Note that there exists a point $p$ in an optimum hitting set such that $p\in S$. Assume w.l.o.g.~that $p$ belongs to $S_{TR}$. Then, consider the set of cells $\C_p'$ that contain the point $p$. Since $S$ has side-length at least $1$ (it has integral coordinates for the corners) there exists a cell $C'\in\C_p'$ such that $S_{TR}$ covers an edge $e$ of $C'$. Hence, there will exist one level $\ell$ in $\{0,1,...,\log N\}$ such that a cell $C''\subseteq O_{TR}$  of level $\ell$ exists for which $S$ covered its edge (say, the bottom edge $e'$) and $p\in C''\cap S$. Then, our algorithm picked $p_b$ (point closest to the bottom edge) for $C''$, such that $p_b\in C''\cap S$. 
\end{proof}
%Next, we want to bound its competitive ratio. 
Now we show that in each round $O(1)$
points are added to $P'$.
\begin{restatable}{lemma}{sqhittingsettwo} \label{lem:round-few-points}
\label{lem:sqhittingset_2}
In each round we add $O(1)$ points to $P'$.
\end{restatable}
\begin{proof}
We show that given a square $S$ such that $S\cap P'=\emptyset$ ($P'$ is the hitting set maintained by our algorithm), our algorithm activates at most $4$ cells. For this, we just observe that in each of the four quadrants $O_{TR}, O_{BR}, O_{TL}, O_{BL}$, we activate at most $1$ cell. For each of these cells, we pick at most $4$ points and hence, we add at most $16$ points in any round.
\end{proof}
%
Denote by $\OPT$ the optimal solution after the last round of inserting a square.
\begin{restatable}{lemma}{fewrounds} \label{lem:few-rounds}
    Let $p\in\OPT$. Then there are $O(\log N)$ rounds in which a square $S$ with $p\in S$ was
    inserted, such that at the beginning of the round  $P'\cap S=\emptyset$.
\end{restatable}
\begin{proof}
    First, define horizontal distance between two cells of level $\ell$ to be the distance between the 
    $x$-coordinates of their left edges. Analogously, define the vertical distance.  Now, we define a set of cells $\C_{p}$ corresponding to the point $p$,
    initialized to $\emptyset$. We will show later that in a certain round if a square $S$ introduced by the adversary contains $p$, and our algorithm activates at least one cell, then one cell in $\C_p$ is also activated. For each level $\ell\in\{0,1,\dots,\log N\}$, include in the set $\C_{p}$: 
    the cell $C$ of level $\ell$ containing $p$ and the other cells of level $\ell$ if they exist which have the same
    parent as $C$. We call these cells to be {\em primary} cells. Further, for level $\ell\in\{0,1,\dots,\log N\}$ consider the cell $C$ of level
    $\ell$ which contains $p$. Then, consider all the cells of level $\ell$ which are at a distance of
    $N/2^{\ell}$ horizontally but at a distance of $0$ vertically; also, consider cells which are at a distance of
    $N/2^{\ell}$ vertically but at a distance of $0$ horizontally from $C$.
    %That is, consider cell $C'$ of level $\ell$ if it is obtained by translating $C$ by a
    %horizontal/vertical distance of $\pm N/2^{\ell}$.
    There can be at most $4$ such cells. For each such cell, include all its children in $\C_p$ (which
    are $4$ in number). We call these cells to be {\em secondary} cells. Hence, per level we select at most $4+4\times4=20$ cells. Therefore, $|\C_{p}|=O(\log N)$.

    \begin{figure}
        \centering
        \includegraphics[page=16,scale=0.75]{02SetCovHitSet.pdf}
        \caption{Cell $C$ contains point $p$. The primary cells are highlighted in green,
        while the secondary cells are highlighted in pink.}
        \label{fig:prim-sec-cells}
    \end{figure}

    We first observe that once a cell $\hat{C}$ gets activated, it does not get activated again.

    \begin{claim}
        A cell $\hat{C}\in T$ does not get activated more than once by our algorithm.
        \label{claim:activate_once}
    \end{claim}
    \begin{claim}
        Assume for contradiction that $\hat{C}$ was already activated in a previous round for a square $S'$.
        In the current round,   the square $S$ was introduced by the adversary such that $S\cap
        P'=\emptyset$. If $\hat{C}$ gets activated again, $S$ covered an edge of $\hat{C}$ (assume this is
        the bottom edge w.l.o.g.). If $\hat{C}$ was already activated in a previous round, then the
        algorithm must have picked points closest to the bottom, top, left, and right edges of $\hat{C}$ and
        at least one hit the square. Among these points, denote the point picked closest to the bottom edge
        by $p'$. If $\hat{C}$ was activated again for $S$, it clearly contained at least one point that hit
        $S$ by definition. Then, $p'$ would have hit $S$ since it had the lowest $y$-coordinate in
        $\hat{C}$.
        %Since $S$ covered the bottom edge of $\hat{C}$, $p'$ had a lesser (or same) $y$-coordinate as
        %compared to $p''$ since it was a point in $\hat{C}$ closest to the bottom edge. Hence, it was
        %contained in $S$ inside $\hat{C}$ which is a contradiction to the fact that $S\cap P'$ was empty
        %when $S$ was introduced by the adversary.
    \end{claim}

    We now want to show that if a square $S$ is inserted in some round where $S\cap P'=\emptyset$ but $p\in
    S_{TR}$, then one cell in $\C_{p}$ gets activated
    but no cell $\hat{C}\notin\C_{p}$ with $\hat{C}\subseteq O_{TR}$ gets activated. 
    %The same claim analogously holds even if $p$ was in some other quadrant w.r.t. $S$.  
    Once we prove this, observe
    that in each such round we activate one cell in $\C_p$. Then, by using the above claim that no cell in $\C_p$
    gets activated again in such a round, we are guaranteed that after $|\C_p|$ rounds, every square
    $S'$ introduced by the adversary which contained $p$ was hit by at least one point in the hitting
    set that the algorithm maintained.
    
    Then we do a symmetric argumentation for the cases that $p\in S_{TL}$, $p\in S_{BR}$, and $p\in S_
    {BL}$, each of them yielding the fact that if a square $S$ is added
    with $S\cap P'=\emptyset$ but $p\in S$, some cell among the cells in $\C_p$ gets activated. Thus, there can be only $|\C_p|=O(\log N)$ such rounds.
    Therefore, finally it remains to prove the claim.
    \begin{claim}
    \label{claim:few-rounds}
    If a square $S$ is inserted in some round where $S\cap P'=\emptyset$ but $p\in
    S_{TR}$, then one cell in $\C_{p}$ gets activated
    but no cell $\hat{C}\notin\C_{p}$ with $\hat{C}\subseteq O_{TR}$ gets activated.
    \end{claim}
    \begin{claim}
    Denote the level of $q$ which was one of the points at the smallest level  among points in $Q\cap S$ to be $\ell'$.
    Assume by contradiction that a cell $\hat{C}\notin\C_{p}$ with $\hat{C}\subseteq O_{TR}$ gets
    activated and $S$ w.l.o.g. covered its bottom edge. Let $\ell$ be the level of $\hat{C}$. Let $C$ be
    the cell of level $\ell$ containing $p$ and let $C'$ be its parent (which is at level $\ell-1$).  By the construction, we
    know that $\hat{C}\cap C'=\emptyset$ and hence, the parent of $\hat{C}$ is not $C'$. Then, the
    parent of $\hat{C}$ (denote by $C''$) and $C'$ are level $\ell-1$ cells at a distance  at least
    $N/2^{\ell-1}$, either horizontally or vertically. Assume w.l.o.g. that $C'$ is to the right side of
    $C''$. 
    %Let us say the $x$-coordinate of the left edge of $C'$ is $(j_1 \cdot
    %N)/2^{\ell-1}$ for some integer $j_1$. Denote the $x$-coordinate of the right edge of $\hat{C}$ by
    %$(j_2 \cdot N)/2^{\ell}$ for some integer $j_2$. 
    %This line ($x=(j_2 \cdot N)/2^{\ell}$) 
    By our assumption, right edge of $\hat{C}$ does not lie to the
    right side of the left edge of $C'$ (could coincide). 
    %By our assumption since $\hat{C}$ was in
    %$O_{TR}$, got activated and the fact that $\hat{C}\cap C'=\emptyset$, $S_{TR}$ covered the bottom
    %edge of $\hat{C}$. 
    Any of the corners of $\hat{C}$ are points at level at most $\ell$. Then, we know
    that the level of $q$, which was $\ell'$ is at most $\ell$. 
    %Denote the $x$-coordinate of $q$ by
    %$j_3\cdot N/2^{ \ell'}$ for some integer $j_3$. 
    Now there are two cases.
    
    In the first case, the horizontal distance between $q$
    and the left edge of $C'$ is at least $N/2^{\ell-1}$ (see \Cref{fig:HittingSet-proof}(a)). Then $\ell'\leq \ell-1$ since then $S_{TR}$ covers
    the left bottom corner of $C'$. In this case, by our assumption $S_{TR}$ covers the bottom edge of
    $C''$ which also contains at least one point that hits it. This is a contradiction on the level of
    the activated cell in this round since $C''$ has level $\ell-1$. 

    In the other case,  the horizontal distance
    between $q$ and the left edge of $C'$ is strictly less than $N/2^{\ell-1}$ (see \Cref{fig:HittingSet-proof}(b)). In this case, $q$ is again at
    level exactly $\ell'=\ell-1$. Then, the right edge of $\hat{C}$ coincides with left edge of $C'$.
    Therefore, $C''$ is at a distance of exactly $N/2^{\ell-1}$ to the left of $C'$ and should have been added as a secondary cell. Hence,
    $\hat{C}\in\mathcal{C}_p$. This is a contradiction. 
    \end{claim}
    This completes the proof of the lemma. 
\end{proof}
%\akr{Shortened this proof. Check.}
%\alr{Will need a figure here for both cases of the proof of \Cref{lem:few-rounds}.}
%\begin{proof}
\begin{comment}
    

\begin{figure}[!htb]
\centering
\captionsetup[subfigure]{justification=centering}
\begin{subfigure}[b]{0.45\linewidth}
\centering
\includegraphics[scale=0.3, page=1]{HittingSet-proof-fig-re.pdf}
\subcaption{}
\label{subfig_vert_nice}
\end{subfigure}
\hspace{1mm}
\begin{subfigure}[b]{0.45\linewidth}
\centering
\includegraphics[scale=0.3, page=2]{HittingSet-proof-fig-re.pdf}
\subcaption{}
\label{subfig_hor_nice}
\end{subfigure}
\caption{Proof of \Cref{claim:few-rounds}. (a) Either, the distance between $C''$ and $C'$ is large, in which case $S$ is edge-covering for $C''$, or (b) $\hat{C}\in\C_p$.}
\label{fig:HittingSet-proof}
\end{figure}
\end{comment}

\begin{figure}[ht]
 \centering
 \includegraphics[page=15,scale=.7]{02SetCovHitSet.pdf}
 \caption{Proof of Claim~\ref{claim:few-rounds}. (a) Either, the distance between $C''$ and $C'$ is large, in which case $S$ is edge-covering for $C''$, or (b) $\hat{C}\in\C_p$.}
\label{fig:HittingSet-proof}
 \end{figure}


%\end{proof}

%
Hence, \Cref{lem:round-few-points} and \Cref{lem:few-rounds} imply that for each point $p\in\OPT$ we
add $O(\log N)$ points to $P'$. Thus, our competitive ratio is $O(\log N)$.
\begin{restatable}{theorem}{squarehittingsettwo}
\label{lem:squareshittingset_2}
    There is an $O(\log N)$-competitive deterministic online algorithm for hitting set for axis-parallel squares of arbitrary sizes. 
\end{restatable}

This is tight, as even for intervals, Even et al.~\cite{EvenS11} have shown an $\Omega(\log
N)$ lower bound.  %\awr{does this lower bound also hold for randomized algorithms?}

\section{Dynamic set cover for $d$-dimensional hyperrectangles}
\label{sec:set-cover-hyperrectangles} 

In this section, we will design an algorithm to dynamically maintain 
an approximate set cover for $d$-dimensional hyperrectangles.%, for the case when $d$ is a constant.
 The main result we prove in 
this section is the following.

\begin{theorem}\label{thm:dyn-set-cover}
After performing a pre-processing step which takes 
$O(m\log^{2d}m)$ time, there is an algorithm for dynamic set cover for $d$-dimensional hyperrectangles 
with an approximation factor of $O(\log^{4d-1}m)$
 and an update time 
of $O(\log^{2d+2}m)$. 
\end{theorem}

Our goal is to adapt the quad-tree based algorithms designed 
in the previous sections of the paper. As a first step towards that, 
we  transform the problem such that 
the points and hyperrectangles in $\IR^d$ get transformed to points and {\em hypercubes}  
in $\IR^{2d}$, 
and the new problem is to cover the points in $\IR^{2d}$ with these hypercubes.
As discussed in the introduction, a simple $2d$-dimensional quad-tree on the 
hypercubes does not suffice for our purpose. We augment the 
quad-tree in two
ways: (a) at each node, we collect 
the hypercubes which are edge-covering w.r.t. that node and ``ignore'' 
that dimension in which they are edge-covering, and (b) recursively 
construct a $(2d{-}1)$-dimensional quad-tree on these hypercubes 
based on the remaining $2d{-}1$ dimensions. We call this new 
structure an {\em extended quad-tree}. The nice feature we obtain 
is that  any point in $\IR^{2d}$ will belong to 
only $O(\log^{2d}m)$ cells in the extended quad-tree. 
Furthermore, at the $1$-dimensional cells of the extended quad-tree, for each cell we
will %pick 
identify
$O(1)$ ``most useful'' hypercubes.
This ensures that 
any point belongs to only $O(\log^{2d}m)$ most useful hypercubes. As a result, a ``bounded 
frequency'' set system can be constructed 
with the most useful hypercubes. The
 dynamic algorithm from Bhattacharya {\em et al.}~\cite{bhattacharya2021dynamic} (for general set cover) works efficiently 
 on bounded frequency set systems and 
 applying it in our setting leads to
 an $O(\log^{4d-1}m)$-approximation 
algorithm.

 





\subsection{Transformation to hypercubes in $\IR^{2d}$.} Recall that the input is a set $P$ of points and $\S$ is a collection 
of hyperrectangles in $\IR^d$. The first step of the 
algorithm is to transform the hyperrectangles in $\S$ to hypercubes in $\IR^{2d}$. 
Consider a hyperrectangle $S \in \S$ with $a=(a_1,\ldots,a_d)$ 
%\awr{I have never seen the notation $a(a_1,\ldots,a_d)$. What does it mean? Maybe better use standard notation?}
and 
$b=(b_1,\ldots, b_d)$ being the ``lower-left'' and the ``upper-right'' corners of $S$, 
respectively.
Let $\Delta=\max_{j=1}^d (b_j-a_j)$.
Then $S$ is transformed to a hypercube $S'$ in $\IR^{2d}$ with side-length 
$\Delta$ and ``top-right'' corner $(-a_1,-a_2,\ldots,-a_d,b_1,b_2,\ldots,b_d)$.
Let $\S'$ be the collection of these $m$ transformed hypercubes.
Let $P'$ be the set of $n$ points in $\IR^{2d}$ obtained by transforming 
each point $p=(p_1,\ldots,p_d)\in P$ to $p'=(-p_1,\ldots,-p_d,p_1,\ldots,p_d)$.

\begin{figure}[h]
 \begin{center}
      \includegraphics[scale=1]{set-cover-dual.pdf}
 \end{center}
 \caption{(a) A point $p$ in 1-D lying inside an interval $S=[a_1,b_1]$, and 
 (b) the transformation of $p$ into a point $p'=(-p_1,p_1)$, 
 and the transformation of $S$ into a 
 square $S'$ in 2-D.}
 \label{fig:set-cover-dual}
\end{figure}

\begin{observation}
A point $p=(p_1,\ldots,p_d)$ lies inside $S$ if and only if 
$p'=(-p_1,\ldots,-p_d,p_1,\ldots,p_d)$ lies inside $S'$.
\end{observation}
\begin{proof}
Assume that $p$ lies inside $S$.
Consider the $i$-th coordinate of $p'$ with $i\leq d$. 
Since $b_i -a_i \leq \Delta$ implies that $ -a_i -\Delta \leq -b_i$, 
we observe that $-a_i - \Delta \leq -b_i \leq -p_i  \leq -a_i$.
Therefore, for all $1\leq i \leq d$, we have $-a_i-\Delta \leq -p_i \leq -a_i$.
 
Now consider the $i$-th coordinate of $p'$ with $i >d$. 
Since $b_i -a_i \leq \Delta$ implies that $ b_i -\Delta \leq a_i$, 
we observe that $b_i - \Delta \leq a_i \leq p_i \leq b_i$.
Therefore, for all $d+1\leq i \leq 2d$, we have $b_i-\Delta \leq p_i \leq b_i$.
Thus, we claim that if a point $p$ lies inside $S$, then $p'$ will lie inside 
$S'$. 

It is easy to prove the other direction. If $p$ lies outside $S$, then there is 
at least one coordinate (say $i$) in which either $p_i <a_i$ or $p_i >b_i$. 
If $p_i < a_i$, then $-p_i > -a_i$ and hence, $p'$ lies outside $S'$. 
On the other hand, if $p_i>b_i$, then again $p'$ lies outside $S'$.
\end{proof}


By a standard rank-space reduction we can assume that the corners of the hyperrectangles 
in $\S$ lie on the grid $[0,2m]^{d}$. After applying the above transformation, 
we note that the coordinates of the corners of the hypercubes in $\S'$ will 
lie on the grid $[-4m,0]^{d}\times [-2m,2m]^d$: 
trivially, $\Delta + a_i \leq 4m$, and hence $-4m \leq -a_i -\Delta 
\leq -a_i \leq 0$. Also, $-2m \leq b_i-\Delta \leq b_i\leq 2m$. 
After performing a suitable shifting of the grid, we will assume that all corners 
of the hypercubes in $\S'$ will lie on the grid $[0,4m]^{2d}$. 

\subsection{Constructing a bounded frequency set system.} 
We will now present a technique to select  a set $\hat{\S} \subseteq \S'$
with the following properties:
 \begin{enumerate}
 \item ({\em Bounded frequency}) Any point in $P'$ lies inside $O(\log^{2d}m)$ hypercubes in $\hat{\S}$.
 \item An $\alpha$-approximation dynamic set cover algorithm for $(P',\hat{\S})$ 
 implies an $O(\alpha\log^{2d-1}m)$-approximation dynamic set cover algorithm 
 for  $(P', {\cal S}')$.
 \item The time taken to update 
 the solution for  the set system $(P',\hat{\S})$ is
$O(\log^{2d}m\cdot\log^2n)$. 
   \item   The time taken to construct the set $\hat{\S}$ is $O(m\log^{2d}m)$.
 \end{enumerate}
 
 \subsubsection{Extended quad-tree for $2$-dimensional squares.}
 \label{subsec:ext_quad_2D}
 Given a set of squares $\S'$, construct a 2-dimensional quad-tree $\mT$ (as defined in Section~\ref{sec:Set-cover-squares}), such that its
 root contains all the squares in $\S'$.
We assume for simplicity that no two input squares in $\S'$ share a corner. Then, we can perturb the input points slightly so that no point $p\in P'$ lies on any of the grid points of the quad-tree and each square still contains the same set of points as before.
 %\al{We will assume for simplicity that none of the points in $P'$ lie at a corner of some square in $\S'$. This assumption can be easily removed by consider corner-covering squares w.r.t. the quadtree as in \Cref{sec:Set-cover-squares}, which will not affect the asymptotic performance of our algorithm.}
 %\awr{maybe refer to previous section for formal definition of quad-tree?}
 Consider a node $v\in \mT$ and a square $S\in \S'$. 
 %A square $S\in \S'$ is defined 
 %to {\em $i$-span} node $v$ if and only if $S$ is ``$i$-long'' at $v$ but not $i$-long 
 %at the parent node of $v$. 
 Let $C$ and $par(C)$ 
 be the cell corresponding to node $v$ and the parent node of $v$, respectively. 
 Let $\mathsf{proj}_i(C)$, $\mathsf{proj}_i(par(C))$ and $\mathsf{proj}_i(S)$ be the projection of $C$, $par(C)$ and $S$, respectively, on to the $i$-th dimension. Then $S$ is {\em $i$-long} at $v$ if and only if $\mathsf{proj}_i(C) \subseteq \mathsf{proj}_i(S)$ but 
 $\mathsf{proj}_i(par(C)) \not\subseteq \mathsf{proj}_i(S)$. See Figure~\ref{fig:long-assigned}(a).
 For all $u\in \mT$, let $\S(u,i) \subseteq \S'$ be 
 the squares which are $i$-long at
 node $u$. Intuitively, these are squares that cover the edge of $C$ in the $i$-th dimension but do not cover any edge of $par(C)$ in the $i$-th dimension. Now, at each node of $\mT$ we will construct 
 two {\em secondary structures} as follows: the first structure is a $1$-dimensional 
 quad-tree built on the projection of the squares in 
 $\S(u,1)$ on to the second dimension
 %\awr{shall we say that intersect the squares with the cell of $u$, before projecting them? It might be clearer for the reader then what we will do with them}
 , and the second structure is a $1$-dimensional quad-tree built on the projection of the 
 squares in $\S(u,2)$ on to the first dimension. 


 In each secondary structure, 
 an interval $I$ (corresponding to a square $S\in\S'$) is {\em assigned} to a node $u$ 
  if and only if $u$ is the node 
 % \awr{do we mean that $u$ is *some* node with smallest depth... ?}
 with the smallest depth (the root is at depth zero) where $I$ intersects 
 either the left endpoint or the right endpoint of the cell $C_u$. See 
 Figure~\ref{fig:long-assigned}(b). By this definition, any interval will be assigned to 
 at most two nodes in the secondary structure.
 
 
Now we will use $\mT$ to construct the geometric collection $\hat{\S}$. 
 Let $V_{\text{sec}}$ be the set of nodes in all the secondary structures of $\mT$. 
 For any node $u\in V_{\text{sec}}$, among its assigned intervals which 
 intersect the left (resp., right) endpoint of the cell $C_u$,  
 identify the {\em maximal} interval 
 $I_{\ell}$ (resp., $I_r$) , i.e., the interval which has maximum overlap with $C_u$.
 See Figure~\ref{fig:long-assigned}(c). We then do the following set of operations over all the nodes in $V_{\text{sec}}$: For a node $u\in V_{\text{sec}}$, denote by $S'$ and $S''$ the corresponding squares for the assigned intervals $I_{\ell}$ and $I_r$, respectively. Further, let $w$ be the node in $\mT$, on which the secondary structure of $u$ was constructed. Then, we include in $\hat{\S}$ the rectangles $S_1\cap C_{w}$ and $S_2\cap C_{w}$. 
 %the set $\hat{\S}$ adds the  in $\S'$ corresponding 
% to $I_{\ell}$ and $I_r$.

 
     
 \iffalse{
%  In each secondary structure, 
%  an interval $I$ (corresponding to a square $S\in\S'$) is {\em assigned} to a node $u$ 
%   if and only if $u$ is the node 
%  % \awr{do we mean that $u$ is *some* node with smallest depth... ?}
%  with the smallest depth (the root is at depth zero) where $I$ intersects 
%  either the left endpoint or the right endpoint of the cell $C_u$. See 
%  Figure~\ref{fig:long-assigned}(b). By this definition, any interval will be assigned to 
%  at most two nodes in the secondary structure.
 
% Now we will use $\mT$ to construct the  collection of sets $\hat{\S}$ \al{for which we will proceed in two steps: First, we will define a routine $\mathsf{FreqRed}(\S',2)$ for the set collection $\S'$ of 2-D squares. Once, we add sets to $\hat{\S}$ by $\mathsf{FreqRed}(\S',2)$, we will define a second routine $\mathsf{feas}(\S',2)$ using which we may possibly add singleton sets to $\hat{\S}$ to ensure that every point $p\in P'$ is contained in at least one set in the final set collection $\hat{\S}$.}

% \al{We define $\mathsf{FreqRed}(\S',2)$ first:} Let $V_{\text{sec}}$ be the set of nodes in all the secondary structures of $\mT$. 
%  For any node $u\in V_{\text{sec}}$, among its assigned intervals which 
%  intersect the left (resp., right) endpoint of the cell $C_u$,  
%  identify the {\em maximal} interval 
%  $I_{\ell}$ (resp., $I_r$) , i.e., the interval which has maximum overlap with $C_u$.
%  See Figure~\ref{fig:long-assigned}(c). We then do the following set of operations over all the nodes in $V_{\text{sec}}$: For a node $u\in V_{\text{sec}}$, denote by $S'$ and $S''$ the corresponding squares for the assigned intervals $I_{\ell}$ and $I_r$, respectively. Further, let $w$ be the node in $\mT$, on which the secondary structure of $u$ was constructed. Then, for $\hat{\S}$ we pick the rectangles $S_1\cap C_{w}$ and $S_2\cap C_{w}$. 

%  \al{Now we define $\mathsf{feas}(\S',2)$ to complete the description of construction of $\hat{\S}$: Once we have constructed $\hat{S}$, we iterate over all the points $p\in P'$ and check whether $p$ is contained in at least one set in $\hat{\S}$. If that is not the case, consider any set $S\in\S'$ which covers it and add a corresponding $\hat{S}=\{p\}$ to the set collection $\hat{\S}$. By a similar argumentation as in \Cref{sec:Set-cover-squares}, we claim that such points $p$ need necessarily be at the top-right corner of a square $S\in\S'$.}
%  %the set $\hat{\S}$ adds the  in $\S'$ corresponding 
% % to $I_{\ell}$ and $I_r$.
}
\fi
 %\alr{Here, we are adding sets in $\S'$ w.r.t. the quad-tree to $\hat{\S}$. Need to define which points it covers since it is not a geometric set anymore.}

%\alr{The above para to define $\hat{\S}$ is not as we desire. Redefining in the next para.}
% \al{Now we will use $\mT$ to construct the new set system $(P',\hat{\S}$). For each square $S\in \S'$, define a corresponding empty set in $\hat{\S}$ as $\hat{\S}[S]$. Let $V_{\text{sec}}$
% \awr{$V$ looks like the set of all vertices. Maybe use different notation?}
% be the set of {root} nodes in all the secondary structures (\as{here, of dimension $d-1=2-1=1$})  of $\mT$. \as{For every $u\in V_{\text{sec}}$ and every square $S\in\S'$, we define sets $\hat{\S}_u[S]$ that we initialize to be empty, and to which we will add points later}.
% %\awr{maybe better: "We define sets $\hat{\S}_u[S]$ that we initialize to be empty, and to which we will add points later"?}
% For any node $u\in V_{\text{sec}}$, among its assigned intervals which 
%  intersect the left (resp., right) endpoint of the cell $C_u$,  
%  consider the {\em maximal} interval 
%  $I_{\ell}$ (resp., $I_r$) , i.e., the interval which has maximum overlap with $C_u$.
%  See Figure~\ref{fig:long-assigned}(c). \as{Let the square corresponding to this interval be $M$}.
%  Then, for all the points $p\in P'$, whose corresponding projection in $C_u$ is covered by $I_{\ell}$, add $p$ to the set \as{$\hat{\S}_u[M]$. Repeat this operation for $I_r$. With this we can obtain the collection of sets $\hat{S}$ as follows:
%  For each square $S\in\S'$,
% \[\hat{\S}[S] \leftarrow \bigcup_{u\in V_{\text{sec}}} \hat{\S}_u[S]. \] }}
 %For all $u\in V$, the set $\hat{\S}$ adds the squares in $\S'$ corresponding 
 %to $I_{\ell}$ and $I_r$.}


 \begin{figure}[h]
 \begin{center}
      \includegraphics[scale=1]{re-long-assinged.pdf}
 \end{center}

 \caption{(a) A square $S$ which is $1$-long at node $v$ (corrs.  
 cell $C$ is highlighted in darker orange), (b) $I$ is assigned to the two children of $v$, and (c) the maximal intervals 
 $I_{\ell}$ and $I_r$ at $C_v$.}
 \label{fig:long-assigned}
\end{figure}
  %\awr{Let us not forget to remove the kind of comments like the ones in the caption (and not just make them black)}
  
 \subsubsection{Extended quad-tree for $2d$-dimensional hypercubes.} 
 
 In this section, we need a generalization of the quad-tree defined in \Cref{sec:Set-cover-squares}. For $d'>2$, a $d'$-dimensional quad-tree is defined
analogously to the the quad-tree defined in \Cref{sec:Set-cover-squares}, where instead of four, each internal node will now have $2^{d'}$ children.
 % Via induction, 
 % assume that we can build
 Assume by induction that we have defined how to construct the extended quad-tree for all dimensions less than or 
 equal to $2d{-}1$.
 % the extended quad-tree for all dimensions less than or 
 % equal to $2d{-}1$. 
 (The base case is the extended quad-tree built for $2$-dimensional squares).
We define now how to construct the structure for $2d$-dimensional hypercubes. 
First construct the regular $2d$-dimensional quad-tree $\mT$ for the 
set of hypercubes $\S'$. 
%\awr{Shouldn't we define our $2d$-dimensional quad-tree?}
 Consider any node $v\in \mT$. Generalizing the previous 
definition, for any $1\leq i \leq 2d$, a hypercube $S\in \S'$ is defined 
 to be {\em $i$-long} at node $v$ if and only if $\mathsf{proj}_i(C) \subseteq \mathsf{proj}_i(S)$, but 
 $\mathsf{proj}_i(par(C)) \not\subseteq \mathsf{proj}_i(S)$.  For all $v\in \mT$, let $\S(v,i) \subseteq \S'$ 
 be the hypercubes which are $i$-long at 
 node $v$.  Now, at each node of $\mT$ we will construct 
 $2d$ secondary structures as follows: for all $1\leq i \leq 2d$, the $i$-th 
 secondary structure is
 a $(2d{-}1)$-dimensional extended quad-tree built on $\S(v,i)$ and all its $2d$ dimensions 
 except the $i$-th dimension. Specifically, any hypercube $S \in \S(v,i)$ of the form 
 $\ell_1 \times \dots \times \ell_i \times \dots \times \ell_{2d}$ is projected to a $(2d{-}1)$-dimensional
 hypercube $\ell_1 \times\dots\times \ell_{i-1} \times \ell_{i+1}\times\dots \times \ell_{2d}$. 
Let  $\hat{\S}_v$ be the collection of the $(2d{-}1)$-dimensional hyperrectangles that are inductively picked for the secondary structure constructed at $v\in\mT$ using the routine $\mathsf{}$. Define the function $g$ which maps a $(2d{-}1)$-dimensional hyperrectangle picked as part of the collection $\hat{\S}_v$ (for a $v\in\mT$) to its corresponding $2d$-dimensional hypercube $S\in\S'$. We now define the collection of sets $\hat{\S}$ consisting of $2d$-dimensional hyperrectangles:
 %which are added to $\hat{\S}$ by the subtrees of the $2d$ secondary structures at node $v \in \mT$. 
 %In other words, 
 \[\hat{\S} \leftarrow \bigcup_{v\in \mT}\left(\bigcup_{S'\in \hat{\S}_v} (g(S')\cap C_v)\right). \]
 

 %\al{Similarly as mentioned for the case of 2-D squares, we ensure that every point $p$ will be covered by at least one set in $\hat{\S}$, by possibly adding some new singleton sets.% For that, once we have constructed $\hat{S}$, we iterate over all the points $p\in P'$ and check whether $p$ is contained in at least one set in $\hat{\S}$. If that is not the case, consider any set $S\in\S'$ which covers it and add a corresponding $\hat{S}=\{p\}$ to the set collection $\hat{\S}$. By a similar argumentation as in \Cref{sec:Set-cover-squares}, we claim that such points $p$ need necessarily be at the top-right corner of a square $S\in\S'$.
 
 % \as{Let $\hat{\S}_v$ be the collection of sets inductively defined corresponding to the
 % secondary structure (of dimension $2d-1$) for each node $v \in \mT$ (note that $\hat{\S}[S]$ below, is a subset of points in $P'$ contained in $S$, that we consider for the new set system $\hat{\S}$ that we are constructing).
 % Then we have,
 % \[\hat{\S}[S] \leftarrow \bigcup_{v\in \mT} \hat{\S}_v[S]. \]}
% \alr{Need to define that $\hat{\S}[S]$ is the subset of points in $P'$ contained in $S$, and similarly define for $\hat{\S}_v[S]$}

\begin{figure}[!ht]
\centering \includegraphics[scale=0.5]{03extendedQTree}
\caption{Extended quad-tree with a $2\times2\times2$ cube as the root.}
\label{fig:extended-quadtree} 
\end{figure}

%\alr{Need to add the old figure here}


\begin{claim}
(Feasibility) Any point $p\in P'$ is covered by at least one set in the collection $\hat{\S}$.
 \end{claim}
\begin{proof}
We will prove the claim first for the case of 2-D squares. For any point $p\in P'$, we know that at least one square $S\in \S'$ covers it. By our assumption, $p$ is not on any of the grid points of the quad-tree. Then, we claim that $S$ is edge-covering for a cell $C_v$ such that $v$ is a leaf node and $C_v$ contains $p$. Then, by the definition of $i$-long, there exists an ancestor $u$ of $v$ in $\mT$ (or possibly $v$ itself) such that $S$ is $i$-long for $C_u$, for some $i\in [2]$. This implies that $\hat{\S}$ consists of a (maximal) square $S'$ such that $S'\in \S(u,i)$ and $S'\cap C_u\supseteq S\cap C_u$. Then, by our construction of the extended quad-tree, $S'\cup C_u$ is part of the collection $\hat{\S}$. Hence, the claim holds for 2-D squares. Generalizing this idea, feasibility can be guaranteed for the case of $2d$-dimensional hypercubes for $d\geq 1$.   
\end{proof}

 
 \begin{lemma}
 (Bounded frequency) Any point in $P'$ lies inside $O(\log^{2d}m)$ sets in $\hat{\S}$.
 \end{lemma}
 \begin{proof}
For the extended quad-tree for $2$-dimensional squares, let $\tau(2)$ be the maximum number of sets 
 in $\hat{\S}$ which contain a point $p=(p_x,p_y) \in P'$. By 
 properties of standard quad-tree, the number of nodes 
 in the $2$-dimensional quad-tree whose corresponding 
 cells contain $p$ is $O(\log m)$. At any such node, 
 each of the secondary structures will have $O(\log m)$ 
 nodes whose corresponding cells contain the projection 
 of $p$ (either $p_x$ or $p_y$). 
 At each cell in the secondary structure which contains $p$,
 we select at most two (maximal) hyperrectangles  into  $\hat{\S}$. Therefore, 
 $\tau(2)=O(\log^2m)$. 
 
 In general, for an extended quad-tree 
 for $d'$-dimensional hypercubes, let $\tau(d')$ be the maximum number of sets in $\hat{\S}$ that contain any given $d'$-dimensional point. Since we construct $2d$ 
 secondary structures at each node, we obtain the following 
 recurrence:
 
 \[\tau(2d) = O((2d)\log m) \cdot \tau(2d{-}1) =O(\log^{2d}m), \]     
 where the constant hidden by the big-$O$-notation depends on $d$. 
 % d dim quadtree, each 2d children that are (2d-1)-dim
 % constant hides the dependency on $d$.
 \end{proof}

%\awr{Shouldn't we prove somewhere that each point is contained in one of our sets?}
 
 \begin{lemma}\label{lem:alpha-approx}
 If there is an $\alpha$-approximation dynamic set cover algorithm for $(P',\hat{\S})$ 
 then there is an $O(\alpha\log^{2d-1}m)$-approximation dynamic set cover algorithm 
 for $(P', {\cal S}')$.
 %\awr{I find "implies" imprecise. Better something like "If there is a ... -approximation for X, then there is a ... -approximation for Y"}
 \end{lemma}
 \begin{proof}
Let $\OPT$ and $\widehat{\OPT}$ be the size of the optimal set cover 
for $(P', \S')$ and $(P',\hat{\S})$. For any hypercube $S$ in $\S'$ we will 
show  that there exists 
$O(\log^{2d-1}m)$ hyperrectangles in $\hat{\S}$ whose union will completely 
cover $S$. Therefore, $\widehat{\OPT}= \OPT\cdot O(\log^{2d-1}m)$.
Note that for every set $S'\in\hat{\S}$, there exists a corresponding hypercube $S\in \S'$ which covers at least the set of points in $P'$ that $S'$ covers. %Hence, any $\alpha$-approximate 
%set cover solution of size $b$ for 
%$(P',\hat{\S})$ will be a feasible solution for $(P', \S')$}.
For $(P', \S')$,
an $\alpha$-approximation set cover for $(P',\hat{\S})$
implies an approximation factor of:
\[\frac{b}{\OPT} = \frac{b\cdot O(\log^{2d-1}m)}{\widehat{\OPT}} = O(\alpha\log^{2d-1}m),\]
where the last equation follows from $b\leq \alpha \cdot\widehat{\OPT}$. 

 
 Finally, we  establish the covering property.
 We will prove it via induction on the dimension size.
 As a base case, for squares in 2-D, let $\mu(2)$ be the number 
 of sets needed in $\hat{\S}$ such that 
 their union completely covers a square $S\in \S'$. 
 For any $S\in \S'$, 
  let $\mathsf{long}(S)$ be the set of nodes in 
$\mT$ where $S$ is $1$-long. By standard properties of a quad-tree, 
we have (a) $|\mathsf{long}(S)|=O(\log m)$, and (b) $S\leftarrow \bigcup_{v\in \mathsf{long}(S)} 
(S\cap C_v)$, where $C_v$ is the cell corresponding to $v$.
Now consider any node $v\in \mathsf{long}(S)$. Let $I$ be the interval corresponding 
to $S$ in the secondary structure of $v$ built on $\S(v,1)$. 
Via our selection of maximal intervals at the secondary nodes, it is clear that 
there exist two maximal intervals which cover $I$. %\al{Also, there are at most $\OPT$ points which are ``top-right corners'' of hypercubes in an optimum set cover. We may need at most $\OPT$ sets in $\hat{\S}$ to cover them}. 
Therefore, 
$\mu(2) \leq 2\cdot|\mathsf{long}(S)|=O(\log m)$. 
 
 In general, let $\mu(2d)$ be the number of sets needed in $\hat{\S}$ such that 
 their union completely covers a hypercube $S\in \S'$. Then we claim that 
 \[ \mu(2d) = O(2^d\log m)\times \mu(2d{-}1),\]
 where $O(2^d\log m)$ is the number of nodes in $\mT$ where $S$ is $1$-long.
Solving the recurrence, we obtain $\mu(2d) =O(\log^{2d-1}m)$.
\end{proof}

\subsection{The final algorithm}
For the general dynamic set cover problem, Bhattacharya {\em et al.}~\cite{bhattacharya2021dynamic} gave an $O(f)$-approximation 
algorithm with a worst-case update time of $O(f\log^2n)$. Recall 
that $f$ is the frequency of the set system. We will use their algorithm 
as a blackbox on the set system $(P',\hat{\S})$. Let $\ALG$ be the sets reported 
by their algorithm. Then our algorithm will also report $\ALG$ as a set cover for 
$(P',\S')$. The solution is feasible
since each set in $\hat{\S}$ belongs to $\S'$ as well. %\alr{Or possibly write that if a point $p$ is covered by a set $S$ in the new set system, it is also covered in the old set system by definition}.

\begin{lemma}
The approximation factor of our algorithm is $O(\log^{4d-1}m)$.
\end{lemma}
\begin{proof}
For the set system $(P',\hat{\S})$ the frequency $f=O(\log^{2d}m)$, and hence,
 the algorithm of \cite{bhattacharya2021dynamic} leads to
an $O(\log^{2d}m)$ approximation for this set system. Using Lemma~\ref{lem:alpha-approx}, this 
implies an approximation factor of $O(\log^{2d}m) \cdot O(\log^{2d-1}m)=
O(\log^{4d-1}m)$ 
for the set system $(P',\S')$.
\end{proof}
\begin{lemma} 
 The update time  is $O(\log^{2d}m\cdot \log^2n)=O(\log^{2d+2}m)$.
 \end{lemma}
\begin{proof}
When a point is inserted or deleted, the $O(\log^{2d}m)$ 
sets in $\hat{\S}$ containing the point can be found in 
$O(\log^{2d}m)$ time by traversing the tree $\mT$. 
The algorithm of \cite{bhattacharya2021dynamic}
has an update time of $O(f\log^2n)=O(\log^{2d}m\cdot\log^2n)= O(\log^{2d+2}m)$. This is true since $d$-dimensional hypercubes have dual $VC$-dimension of $O(d)$ \cite{sauer1972density,shelah1972combinatorial} and hence, $O(\log n)=O(\log m)$.
\end{proof}

\begin{lemma}
  The time taken to construct the set $\hat{\S}$ is $O(m\log^{2d}m)$.
  \end{lemma}
  
  \begin{proof}
  Let $T(m,d)$ be the time taken to build the extended 
  quad-tree on $m$ hypercubes in $d$ dimensions.
  As a base case, we first compute $T(m,1)$. 
  Constructing the skeleton structure of the 
  1-dimensional quad-tree takes $O(m)$ time, 
  since the endpoints of the intervals lie on the integer grid $[0,4m]$.
  Then ``assigning'' each interval to a node in this 
  quad-tree takes $O(m\log m)$ time. For a node $v$ 
  which is assigned $m_v$ intervals, finding the two
  maximal intervals takes $O(m_v)$ time.
  Therefore, $T(m,1)= O(m\log m) + \sum_{v} O(m_v)=O(m\log m)$.
  
  Now consider the extended quad-tree on $m$ hypercubes in $2d$ dimensions.
  Again constructing the skeleton of the quad-tree takes only $O(m)$ time.
  For any $1\leq i\leq 2d$, finding the nodes in $\mT$ where a hypercube is 
  ``$i$-long'' takes $O(2^{2d}\log m)=O(\log m)$ time. Therefore,
  \begin{align*}
  T(m, 2d) &= O(m\log m) +  \sum_{v} T(m_v, 2d-1) \\
                &= O(m\log m) + \sum_{v} m_v\log^{2d-1}m_v \qquad\qquad \text{(by induction)}\\
                &= O(m\log m) + \sum_{v} m_v\log^{2d-1}m\\
                &= O(m\log^{2d}m), \qquad \qquad \text{since, $\sum_{v}m_v =O(m\log m)$.}
                \qedhere
  \end{align*}
 \end{proof} 

\subsection{Weighted setting}
We present an easy extension of our algorithm to the setting 
where each hyperrectangle $S \in \S$ has a weight 
$w_S \in [1, W]$. First, we {\em round} the weight of each 
set $S$ to the smallest power of two greater than or equal to
$w_S$. This leads to $O(\log W)$ different weight classes.
Next, for each weight class, 
we will build an extended quad-tree based on the hypercubes 
of that weight class after the reduction to the case of $2d$-dimensional hypercubes from $d$-dimensional hyperrectangles as shown in the previous section. Finally, let $\hat{\S}$ be the collection of 
(maximal) hypercubes obtained from all the $O(\log W)$ 
extended quad-trees. Run the dynamic set cover algorithm of 
Bhattacharya {\em et al.}~\cite{bhattacharya2021dynamic} 
on $(P',\hat{\S})$.




\begin{lemma}
The approximation factor of the algorithm is $O(\log^{4d-1}m\cdot \log W)$.
\end{lemma}

\begin{proof}
Consider the optimal solution for $(P',\S')$ and let $\OPT$ be 
the optimal weight. By rounding the 
weight of each set in the optimal solution, their total weight 
becomes at most $2\cdot \OPT$. Therefore, the weight of 
the optimal solution in the set system after rounding is 
at most $2\cdot \OPT$. Compared to the unweighted setting, 
now the frequency of the set system $(P',\hat{\S})$ increases 
by a $O(\log W)$ factor. As a result, we obtain an approximation 
factor of $O(\log^{4d-1}m\cdot \log W)$.
\end{proof}

\begin{lemma}
The update time of the algorithm is $O(\log^{2d}m\log^{3}(Wm))$. 
\end{lemma}
\begin{proof}
When a point is inserted or deleted, the $O(\log^{2d}m\log W)$ sets 
in $\hat{\S}$ containing the point can be found in 
$O(\log^{2d}m\log W)$ time by traversing the $O(\log W)$ extended 
quad-trees. The algorithm of \cite{bhattacharya2021dynamic} has an 
update time of $O(f\log^{2} Wn)=O(\log^{2d}m\log W\log^{2}Wn)=
O(\log^{2d}m\log^{3}(Wm))$. 
\end{proof}

\begin{theorem}\label{thm:WtDynSetCov}
There is an algorithm for weighted dynamic set cover for $d$-dimensional hyperrectangles 
with an approximation factor of $O(\log^{4d-1}m\cdot\log W)$
 and an update time 
of $O(\log^{2d}m\cdot\log^3(Wm))$. 
\end{theorem}
We also have the following corollary for dynamic set cover for $2d$-dimensional hypercubes where all the corners are integral and bounded in $[0,cm]^{2d}$ for some constant $c>0$.

\begin{corollary}\label{Cor:WtDynSetCov_cube}
There is an algorithm for weighted dynamic set cover for $2d$-dimensional hypercubes 
with an approximation factor of $O(\log^{4d-1}m\cdot\log W)$
 and an update time 
of $O(\log^{2d}m\cdot\log^3(Wm))$, when all of their corners are integral and bounded in $[0,cm]^{2d}$ for a fixed $c>0$. 
\end{corollary}


\section{Dynamic hitting set for $d$-dimensional hyperrectangles}
\label{sec:hit-set-hyperrectangles}
In this section we present a dynamic algorithm for hitting set for 
$d$-dimensional hyperrectangles. %\al{assuming $d$ is a constant}.  
We will reduce the problem to an instance of dynamic set cover in $2d$-dimensional space and use the algorithm designed in the previous 
section (Theorem~\ref{Cor:WtDynSetCov_cube}).
Recall that $P$ is the set of 
points and $\S$ is the set of hyperrectangles.
%Via suitable shifting we can assume that all the points of $P$ 
%have coordinate value greater than zero in all the $d$-dimensions.
Assume that all
the points of $P$ and the hyperrectangles in $\S$ lie in the box $[0, N]^d$. For all $p\in P$, we transform 
$p=(p_1,\ldots, p_d)$ to a $2d$-dimensional hypercube  $\S(p)$ 
of side length $N$ and 
``lower-left'' corner $p'=(-p_1,\ldots,-p_d, p_1,\ldots,p_d)$. 
Let $\S(P)$ be the  transformed hypercubes.
Next, we transform each hyperrectangle, say  $S \in \S$, with ``lower-left'' corner 
$a=(a_1,\ldots,a_d)$ and ``top-right'' corner $b=(b_1,\ldots,b_d)$ into a 
$2d$-dimensional point $P(S)=(-a_1,\ldots,-a_d,b_1,\ldots,b_d)$. 
Let $P(\S)$ be the  transformed points.
See Figure~\ref{fig:hit-duality}.

\begin{figure}[h]
 \begin{center}
      \includegraphics[scale=1]{re-hit-duality.pdf}
 \end{center}
 \caption{(a) A point $p$ in 1-D lying inside an interval $S=[a_1,b_1]$, and 
 (b) the transformation of $p$ into a square $\S(p)$, and transformation of $S$ into a 
 point $(-a_1,b_1)$ in 2-D.}
 \label{fig:hit-duality}
\end{figure}

\begin{lemma} \label{lem:duality}
In $\IR^d$ a point $p$ lies inside a hyperrectangle $S$, if and only if, 
in $\IR^{2d}$ the hypercube $\S(p)$ contains the point $P(S)$.
\end{lemma}
\begin{proof}
 We define a point $q(q_1,\ldots,q_{2d})$ 
to {\em dominate} another point $q'(q'_1,\ldots,q'_{2d})$ if and only if 
$q_i \geq q'_i$, for all $1\leq i\leq 2d$. 
Assume that $p$ lies inside $S$. Then we have 
$p_i \geq a_i$ and $p_i \leq b_i$, for all $1\leq i\leq d$.
This implies that $P(S)=(-a_1,\ldots,-a_d,b_1,\ldots, b_d)$ dominates 
the point $p'(-p_1,\ldots,-p_d, p_1,\ldots,p_d)$, which is the lower-left 
corner of $\S(p)$. 


The coordinates of the top-right corner of $\S(p)$ is 
$(-p_1+N,\ldots,-p_d+N, p_1+N,\ldots,p_d+N)$.
For all $1\leq i \leq d$, since $N\geq p_i-a_i$, it implies that 
$-p_i + N \geq -a_i$. For all $1\leq i \leq d$, since 
$p_i + N \geq N \geq b_i$, it implies that $ p_i + N \geq b_i$.
This finally implies that 
that the top-right corner of $\S(p)$ dominates $P(S)$. 
Therefore, $\S(p)$ contains $P(S)$.

Assume that $p$ lies outside $S$. Then there exists a 
dimension $i$ such that $p_i < a_i$ or $p_i > b_i$. 
If $p_i < a_i$, then $-p_i > -a_i$, which implies that 
$P(S)$ cannot dominate the lower-left corner of $\S(p)$.
If $p_i > b_i$, again $P(S)$ cannot dominate the 
lower-left corner of $\S(p)$. This implies that 
$P(S)$ lies outside $\S(p)$.
\end{proof}

We will use the above reduction 
to transform the points in $P$
into  hypercubes in $\IR^{2d}$ and transform the hyperrectangles 
in $\S$ into points in $\IR^{2d}$. 
Therefore, the hitting set problem in 
$\IR^d$ on hyperrectangles has been 
reduced to the set cover problem in $\IR^{2d}$
on hypercubes. And with a similar rank-space reduction as mentioned in the previous section, all the points as well as the corners of the hypercubes in this instance have integral coordinates.
Then, the set cover instance is answered using ~\Cref{Cor:WtDynSetCov_cube}. The correctness follows from \Cref{lem:duality}. Noting that for $2d$-dimensional hypercubes, $O(\log n)=O(\log m)$, the performance of the algorithm is summarized 
below.
 

\begin{theorem}\label{thm:WtDynHitSet}
After performing a pre-processing step which takes 
$O(n\log^{2d}n)$ time, there is an algorithm for 
hitting set for $d$-dimensional hyperrectangles 
with an approximation factor of $O(\log^{4d-1}n)$ and 
an update time of $O(\log^{2d+2}n)$.
In the weighted setting, the approximation factor 
is $O(\log^{4d-1}n\cdot\log W)$ 
and the update time is $O(\log^{2d}n\log^3(Wn))$.
\end{theorem}



\section{Future work}

In the first part of this work, we have studied online geometric set cover 
and hitting set for 2-D squares. This opens up an interesting line of work 
for the future. We state a few open problems: 
\begin{enumerate}
    \item As a natural extension of 2-D squares, is it possible to 
    design a $o(\log n \log m)$-competitive algorithm for 3-D cubes? %\awr{do we mean "cubes" instead of "cuboids" here?}
    The techniques used in this paper for 2-D squares do not seem to 
    extend to 3-D cubes. Another setting of interest here is when the geometric objects are 2-D disks. Can we obtain a $o(\log n \log m)$-competitive online set cover algorithm for them?
    \item Design an online algorithm for set cover (resp., hitting set) for rectangles with competitive ratio $o(\log^2 n)$ or show an almost matching lower bound of $\Omega\left(\frac{\log^2 n}{\log\log n}\right)$ (which holds for the general case of online set cover~\cite{alon2003online})?
    \item As a generalization of the above question, is it possible to obtain online algorithms for set cover and hitting set with competitive ratio $o(\log^2 n)$ for set systems with ``constant'' VC-dimension.
    \item For the weighted case of online set cover, even in unit squares, can we obtain algorithms with competitive ratio $o(\log n\log m)$?
    \item Design an online algorithm for hitting set for squares with competitive ratio $O(\log n)$, and hence, improving our algorithm's competitive ratio of $O(\log N)$ (where the corners of the squares are integral and contained in $[0,N)^2$).
\end{enumerate}

In the second part of our work, we studied dynamic geometric set cover 
and hitting set for $d$-dimensional hyperrectangles. This line of work 
nicely brings together data structures, computational geometry, and approximation 
algorithms. We finish with a few open problems in the dynamic setting:
\begin{enumerate}
    \item Improve the approximation factor for dynamic set cover for the case of  2-D rectangles. Specifically, is it possible to obtain an $O(\log n)$ approximation with polylogarithmic update time? In this setting the 
    rectangles are fixed, but the points are dynamic.
    \item For weighted dynamic set cover for the case of 2-D rectangles, is it possible to obtain approximation and update bounds independent of $W$ (where $W$ is the ratio of the weight of the highest weight rectangle to the lowest weight rectangle in the input)?
    \item For the (fully) dynamic case of set cover studied in~\cite{agarwal2020dynamic, chan2021more, chan2022dynamic}, can we obtain algorithms with sublinear update time and polylogarithmic approximation when the sets are rectangles (as {originally asked in~\cite{chan2022dynamic}})?
\end{enumerate}

\iffalse{
\section{Omitted Proofs from \Cref{sec:hit-set-squares}}
\label{subsec:sqhittingset}

\subsection{Proof of \Cref{lem:sqhittingset_1}}
\sqhittingsetone*
\begin{proof}
       
\end{proof}

\subsection{Proof of \Cref{lem:sqhittingset_2}}
\sqhittingsettwo*



\subsection{Proof of \Cref{lem:few-rounds}} \label{subsec:hit-approx}
\fewrounds*
}
\fi






\iffalse{
\akr{complete "??" in prev section, shift following section to appendix}
\section{Hitting Set for $d$-dimensional Hyperrectangles}
We present now our dynamic algorithm for hitting set for $d$-dimensional hyperrectangles. We
assume that we are given a set of points $P$ and that hyperrectangles are inserted and deleted
dynamically. Like for set cover, we first reduce the problem to hypercubes, by increasing the
dimension by a factor of 2. Note that this reduction is different from our reduction for set cover
from \Cref{lem:reduce-to-cubes} since now the points $P$ are fixed, rather than the hyperrectangles.
\begin{restatable}{lemma}{reducetocubesHS} \label{lem:reduce-to-cubes-HS}
    If we are given a dynamic $\alpha$-approximation
    algorithm for hitting set in $2d'$-dimensional hypercubes (For $d'\in\Z_+$) with integer coordinates
    having $f(m,n,N)$ update
    time and $g(m,n,N)$ initialization time, then we can construct a dynamic $\alpha$-approximation
    algorithm for hitting set in $d'$-dimensional hyperrectangles, with update time $O_d(f(m,n,m)+\log N)$ and initialization time $O_d(g(m,n,m)+m\log N')$.
\end{restatable}
%
Suppose now that we want to solve hitting set dynamically for $2d$-dimensional hypercubes for a given
set of points $P\subseteq\R^{d}$. Also, suppose that $\S$ is the current set of hypercubes. We build
the same extended quad-tree $T'=(V',E')$ as in \Cref{sec:set-cover-hyperrectangles} and assign sets
(and their respective projections) and points (and their respective projections) to the cells of
$T'$ in exactly the same way.  Hence, \Cref{lem:few-sets} still holds. And in particular, for any
set $S\in\S$, the set $V_S'$ is defined in the exact same way. For any point $p\in P$, the set
$V_p''$ is defined in the exact same way. But, we do not designate any covering sets for the
respective covering nodes. Rather, if a node $v\in V_S'$ was a covered node and there existed a
non-empty set of points $P'\subseteq P$ which were assigned to it, we arbitrarily designate one of
them as a \textit{hitting point} for $v$.

% We organize the points $P$ in a range-counting data structure {[}?{]} that allows us to query in
% time $???$ .... <\textcompwordmark < add the needed query here, in particular, the one needed for
% \Cref{lem:construct-hat-S-fast}>\textcompwordmark > ... .

\paragraph*{Auxiliary hitting set instance}
Our goal is to construct an auxiliary instance $(\hat{P},\hat{\S})$ of general hitting set with a
family of sets $\hat{\S}$ and a set of points $\hat{P}$ such that each set $\hat{S}\in\hat{\S}$
contains at most $(\log N)^{O(d)}$ points from $\hat{P}$.
The point set $\hat{P}$ is such that for each point $p\in P$, there is a corresponding point $\hat{p}\in\hat{P}$.
The set set memberships for this set are described next.

First, include every set $S\in \S$ in the collection of sets $\hat{\S}$ but each set is empty to
begin with (and we denote this set by $\hat{S}$ corresponding to $S$). Based on the old set system
$(P,\S)$, we will include a subset of $P$ to define the points that $S$ covers according to
$(\hat{P},\hat{\S})$.

For each set $S\in \hat{\S}$, consider each node $v$ in $V_S'$. If $v$ was a covered node by the
projection of $S$ in it, then we include the hitting point for $v$ (if it exists) in  $\hat{S}$.
Else if $v$ was not a covered leaf by the projection of $S$ in it, then we consider that point in
$P$ whose projection in $v$ was the leftmost (if it exists). Similarly, find such a point in $P$
whose projection in $p$ was the rightmost (if it exists). Include them (if they exist) in $\hat{S}$
based on whether their projections in $v$ hit the projection of $S$ in $v$. 
%For each node $v\in V'$ and each face $F$
%of $C_{v}$, we identify the point $p\in C_{v}$ that is closest to
%$F$ (breaking ties in an arbitrary fixed way). We add a corresponding
%point $(v,p)$ to $\hat{P}$. Intuitively, we pick $p$ since it is
%the most useful point in $P_{v}$ if we want to hit a set $S$ that
%contains $F$, since among all points in $P_{v}$ the point $p$ hits
%the maximum number of such sets $S$.
%Next, we define the sets $\hat{\S}$. Let $S\in\S$. We add a set
%$\hat{S}$ to $\hat{\S}$ which \emph{corresponds} to $S$. To construct
%$\hat{S}$, we do the following procedure for each $v\in V'$ such
%that $S$ contains a face $F$ of $C_{v}$ but $S$ does not contain
%a face $F'$ of a cell $C_{v'}$ of an ancestor $v'$ of $v$ in $T'$.
%We identify the point $p\in C_{v}$ that is closest to $F$ (breaking
%ties in the same way as above). If $p\in S$ then we add $(v,p)$
%to $\hat{S}$. We can show that for $S$ there are at most $(\log N)^{O(d)}$
%such vertices $v\in V'$ which proves the following lemma.
\begin{restatable}{theorem}{frequencyHS} \label{lem:frequency-HS}
    For all $\hat{S}\in\hat{\S}$. We have that $|\hat{S}|=O_d(\log^{2d-1} N)$.
\end{restatable}
\begin{proof}
    We have shown in \Cref{sec:set-cover-hyperrectangles} that $|V_S'|=O_d(\log^{2d-1}N)$.
    Hence, $|\hat{S}|=O_d(\log^{2d-1}N)$.
\end{proof}

\begin{restatable}{lemma}{rectangledynhittingseteq}
    \label{lem:rectangledynhittingset_eq}
    %Any point $p\in U_2$ is covered by a hypercube $h\in \mathcal{F}_2$ if and only if $h$ appears as one of the square pieces in some 2d  instance $I$ in $\mathcal{I}_C$, where this square piece covered the equivalent projection of $p$ in the 2d instance ($C$ is the cell in $\mathcal{F}_2$ which contained $p$ and for which $h$ was \secc in the original instance).
    A hypercube $S\in\mathcal{S}$ is hit by a point $p\in P$ if and only if there exists a node $v$ in
    $V_S'$ such that $p$ is assigned to $v$ and the projection of $S$ in $v$ is hit by the projection of
    $p$ in it.
\end{restatable}
\begin{proof}
    The claim follows directly from \Cref{lem:rectangleDynSetCov_covering}.
\end{proof}



%The second property intuitively says that covering a projection of point in some appropriate cell in $T'$ by projection of a hypercube is equivalent to covering the point in the actual instance $(P,\S)$.
We can show that $(\hat{P},\hat{\S})$ admits a solution with at most $\OPT\cdot(\log^{2d} N)$
points, where $\OPT$ is the optimal solution to $(P,\S)$.
%First, we show that for any set $S\in\S$, $|V_S'|=O_d(\log^{2d-1}N)$.
We will show that for $p\in P$ in $\OPT$, we can pick a subcollection $P'$ of points of size
$O_d(\log^{2d}N)$ such that if $S$ was hit by $p$ in $(P,\S)$, in the new set system
$(\hat{P},\hat{\S})$, a point $p'\in P'$ was contained in ${\hat{S}}$.  Consider the set $ V_p''$
and note that any node $v\in V_p''$ has $p$ assigned to it. Then, if $v$ was a covered node, we
include the corresponding hitting point in $P'$.  Else, %and $p$ such that the projection of $p$ was
covered by the projection of $S$ in this cell for $v$, we show that we can map at most one set
$S'\in\hat{\S}$ such that the projection of $S'$ necessarily covers the projection of $p$ in this
cell.
%That is, for any such node $v\in V_S'$ which is a covered node, we pick the covering set for
%it. This set covers $p$ in the new set system by definition. Else,
$v$ is necessarily an uncovered leaf and we add at most $2$ points to $P'$ for this cell whose
respective projections are the leftmost and the rightmost inside $v$, which is an interval. Note
that $P'$ has size $O_d(\log^{2d}N)$ from the fact that $|V_p''|=O_d(\log^{2d}N)$. Then, we claim
that if a set $S$ was hit by $p$ in $(P,\S)$, in the new set system, a point $p'\in P'$ (also
$p'\in\hat{P}$) was contained in ${\hat{S}}$.

If $S$ was hit by $p\in P$, by \Cref{lem:rectangledynhittingset_eq} there exists a node $v'\in V_S'$
such that $p$ is assigned to $v'$ and the projection of $S$ in $v'$ was hit by the projection of $p$
in $v'$. If this was a covered node, $P'$ contained a point $p'$ which was a hitting point for it.
Then, clearly the projection of $p'$ hit the projection of $S$ in this node and again by
\Cref{lem:rectangledynhittingset_eq} this implies that $S$ was hit by it according to $(P,\S)$. But
also note that $p'\in\hat{S}$ which implies the lemma for this case. If $v$ was an uncovered leaf,
then the projection of $p$ in $v$ hit the projection of $S$ in it by
\Cref{lem:rectangledynhittingset_eq}. The set $P'$ contained at most two points whose projections
were extremal in $v$. Then, at least one of these projected points hit the projection of $S$ in $v$.
Also, at least one of them was included in $\hat{S}$. Then, by \Cref{lem:rectangledynhittingset_eq},
the lemma statement follows even for this case.

%This hitting point hits the projection of $S$ for $v$ if $S$ was assigned to $v$ and its projection covered it entirely (and hence, $v\in V_S'$).
%In this case as well, if $v\in V_S'$, the projection of  of this mentioned set inside the cell. Denote this set by $S'$ in either case and note that $S'$ covers $p$. Then, by \Cref{lem:rectangleDynSetCov_covering}, we are able to cover $S$ entirely in the original instance of set cover $(P,\S)$ by a subcollection of $O_d(\log^{2d-1}N)$ sets. Applying this argument over all sets in $\OPT$ yields the following lemma.



%To show this, we consider each point $p\in\OPT$ and
%each of the at most $(\log N)^{O(d)}$ cells $C_{v}$ for a node
%$v\in V'$ such that a projection of $p$ is contained in $P_{v}$.
%For each face $F$ of $C_{v}$ we select the point $(v,p')$ such
%that $p'$ is the point in $P_{v}$ closest to $F$. This yields $\OPT\cdot(\log N)^{O(d)}$
%points in total.
\begin{restatable}{lemma}
    The instance $(\hat{P},\hat{\S})$ has a solution with at most $\OPT\cdot(\log^{2d} N)$
    points.
\end{restatable}
On the other hand, we can translate each solution to $(\hat{P},\hat{\S})$
to a solution to $(P,\S)$ with the same cardinality.
\begin{restatable}{lemma}{correspondingsolutionHS} \label{lem:corresponding-solution-HS}
For any solution $\hat{\A}\subseteq\hat{P}$
    to $(\hat{P},\hat{\S})$ there is a solution $\A$ to $(P,\S)$ with
    $|\A|\le|\hat{\A}|$. For each $(v,p)\in\hat{\A}$ there is a corresponding
    point $\tilde{p}\in P$ and given $(v,p)$, we can identify $\tilde{p}$
    in time $O(\log m)$.
\end{restatable}

We claim that we can maintain the instance $(\hat{P},\hat{\S})$ dynamically when sets are inserted
to $\S$ or removed from $\S$. If a set $S\in\S$ is removed then we simply remove the corresponding
set $\hat{S}\in\hat{S}$. If a set $S$ is inserted to $\S$ then we can construct the corresponding
set $\hat{S}$ quickly via the following lemma. Intuitively, we calculate the nodes $v\in T'$ for
which we might add a point $(v,p)$ to $\hat{S}$. For each such $v$ and each facet $F$ of $C_{v}$ we
find the closest point $p\in P\cap C_{v}$ quickly using our range-counting data structure for $P$.
\begin{restatable}{lemma}{constructhatSfast}
    \label{lem:construct-hat-S-fast}Let $S\subseteq[0,N)^{d}$ be a hypercube.
    In time $(\log^{2d-1} N)$ we can construct a set $\hat{S}\subseteq\hat{P}$
    that corresponds to $S$.
\end{restatable}
We maintain an approximate solution $\hat{\A}\subseteq\hat{P}$ to
$(\hat{P},\hat{\S})$ dynamically using the data structure introduced by \cite{bhattacharya2021dynamic}
for arbitrary instances of set cover. It guarantees an approximation
ratio of $(1+\varepsilon)f$ (for any $\varepsilon>0$) and an update time of $O(f\log^2(Wm)/\varepsilon^3)$, since if we translate
our instance of hitting set to set cover, due to \Cref{lem:frequency-HS}
each point is contained in at most $(\log^{2d-1} N)$ sets.

With similar ideas as in \Cref{subsec:set-cover-weighted} we can extend the above algorithm to the
weighted case, by increasing the update time by a factor of $\log W$, assuming that for each point
$p\in P$ its weight $w_{p}$ is in the interval $[1,W]$ for a fixed value $W$ (see Appendix~?? for
details).

\begin{restatable}{theorem}
    There is a dynamic algorithm for hitting set for $d$-dimensional
    hyperrectangles with worst-case update time of $O_d(\log^{2d-1} m)\log^2 (Wm)\cdot\log W$ when a rectangle
    is added or deleted.
\end{restatable}

}
\fi




\begin{comment}
    %\section{Conclusion and Open Problems}
    \textcolor{red}{Arindam: We may omit this section.}
    Our results provide a general framework for tackling the problem of dynamic set cover/hitting set in
    geometric settings: To transform the original set system to a new one by decomposing every set $S$
    into a polylogarithmic number of parts. Also, with the condition that $w(\OPT_{\text{new}})$ is at
    most a polylog away from  $w(\OPT_{\text {old}})$. We believe that this framework can be extended to
    general set systems with bounded VC-dimension to obtain polylogarithmic approximation and
    polylogarithmic update time.

    Many questions remain open in this direction of work. Some of them are as follows:
    \begin{enumerate}
        \item Whether our techniques can be extended to other objects in $\mathbb{R}^d$ like halfspaces,
            spheres, etc.
            %  \item Whether in the most general case of set insertions/deletions for set cover in rectangles,
            Can one show a lower bound of the form $\Omega(n^{\alpha})$ on the update time if a
            polylogarithmic approximate set cover is to be maintained for some constant $\alpha>0$?
        \item Whether our results can be further improved in terms of update time and approximation
            factor both. That is, whether for set cover in hyperrectangles in $\mathbb{R}^d$ it is
            possible to maintain an $O(\log^{c} m)$ approximate solution with polylogarithmic update time
            (exponent may depend on $d$), where $c$ is independent of the ambient dimension $d$. In particular, in the classical online setting is it possible to obtain an $O(\log m)$-competitive algorithm for set cover in hyperrectangles for a fixed dimension $d$?
    \end{enumerate}
\end{comment}

\bibliographystyle{plainurl}
\bibliography{ref_dyn}

\appendix

\section{Online algorithms for interval set cover}
\label{sec:interval_upper}
In this section, we present a tight 2-competitive algorithm for the case of interval set cover.

\begin{comment}
\begin{algorithm}
    \caption{Online Algorithm for Intervals}
    \label{alg:Int2Online}
    \KwData{offline: a set of intervals; online: set of points}
    \KwResult{a set of intervals that covers all the input points seen till now}
    $cover \leftarrow \emptyset$\\
    \While{new point $p$ arrives} {
        \If{$p$ is not covered} {
            Find $R$, the interval with rightmost right end-point covering  $p$\\
            Find $L$, the interval with leftmost left end-point covering  $p$\\
            $cover \leftarrow cover \cup \{R, L\}$
        }
    }
\end{algorithm}
\end{comment}
In the algorithm, we start with an empty set cover. 
In each iteration, when a new point $p$ arrives, if it is covered then we do nothing.
Otherwise, we select among the intervals covering  $p$, the one with the rightmost right end-point and the one with the leftmost left end-point.

The correctness of the algorithm follows trivially, since for every new uncovered point we pick an
interval covering it. We do not remove intervals from our solution at any later steps in the algorithm, and
hence, all points are covered when the algorithm terminates.
% TODO : consider implementation details, and also look at improving runtime using nice data
% structures.

\begin{restatable}{theorem}{onlineintervalupper}
\label{thm:onlineintervalupper}
    There exists a 2-competitive algorithm for the  online interval set cover problem.
\end{restatable}
\begin{proof}
    Consider an interval $I$ in the optimum solution $\OPT$. When the first uncovered point covered by
    it, arrives in the input, our algorithm picks two intervals and  ensures that these two intervals cover all of  $I$. Hence, for each interval in $\OPT$,
    we pick at most 2 intervals in our solution, giving us a 2-competitive solution.
\end{proof}

\begin{restatable}{theorem}{onlineintervallower}
\label{thm:onlineintervallower}
    There is an instance of the set cover problem on intervals such that any online algorithm
    (without recourse) can at best be 2-competitive on this instance.
\end{restatable}
\begin{proof}
    Consider the given set of intervals to be $A:=[0,1],B:=[1,2],C:=[2,3],D:=[3,4]$. The first point to
    arrive is $p_1:=2$. 
    If the algorithm picks two or more sets, then we are done as $\OPT$ is of size 1. 
    Otherwise, to cover $p$, an algorithm can pick either interval  $B$ or  $C$.  In the
    former case, the second point should be  $p_2:=3$; and in the latter  $p_3:=1$.  We see that in both cases $\OPT$ is of size one, but an online algorithm is forced to pick two intervals.
\end{proof}






\iffalse{
\section{Omitted Proofs from \Cref{sec:set-cover-hyperrectangles} }
\subsection{Hyperrectangles to Hypercubes reduction (Proof of \Cref{lem:reduce-to-cubes}) \label{subsec:hypercube-reduction}}
% We will define hyperrectangles and orthants in $d$-dimensions first for $d\in\mathbb{Z}^{+}$. High-dimensional hyperrectangles are generalizations of rectangles, which are hyperrectangles of dimension $2$.

% \begin{definition}[$d$-dimensional Hyperrectangle]
%  A $d$-dimensional hyperrectangle is the convex set defined by the cartesian product $[b_1,b_1+v_1]\times \dots\times [b_d,b_d+v_d]$, where $b=(b_1,\dots,b_d)$ is a $d$-dimensional vector and $b_i>0$ for $i\in[d]$ and $b$ is the \textit{bottom-justified-corner} of the hyperrectangle. Here, we have defined the bottom-justified-corner for the sign vector (of $v$) $\{+\}^{d}$. It can as well be defined for some arbitrary sign vector $v$ where $\sign(v_i)\in\{+,-\}$.
% \end{definition}
% Note that a $d$-dimensional hyperrectangle has $2^d$ corner vertices and any two of its faces, which are hyperplanes, are either parallel or perpendicular to each other.
% Any hyperrectangle can be represented by the bottom-justified-corner $b\in\mathbb{R}^d$ and a vector $v\in \mathbb{R}^d$ such that $\sign ({v_i})\in\{+,-\}$ for each $i\in[d]$. The absolute value of the coordinates of the $v$ vector specify the length of the edges of the hyperrectangle in each dimension. Depending on the sign values of $v$, we have different orthant orientations.

% \begin{definition}[$d$-dimensional orthants]
% For $d\in\mathbb{Z}^{+}$, $d$-dimensional orthants are hyperrectangles which share a common \textit{bottom-justified-corner}.
% \end{definition}

% We can assume by an origin shift that the bottom-justified-corner is $\{0\}^d$ for $d$-dimensional orthants. The length vector for the orthants is defined as the \textit{apex point}. We will define a contraction operation for $d$-dimensional hyperrectangles in some direction $x_j$ for $j\in[d]$

% \begin{definition}[Contraction of a hyperrectangle]
% For a hyperrectangle $h\in\mathbb{R}^d$ ($d\geq 2$) and given direction $x_j$ ($j\in[d]$), define the contraction $h^j$ of $h$ in direction $x_j$ to be the $(d-1)$-dimensional hyperrectangle such that it has bottom-justified-corner $b^j$ and length vector $v^j$, where $b_i^j=b_i$ for $i\in [d]$; $v_i^j=v_i$ for $i\in[d]\setminus \{j\}$ and $v_j^j=0$.
% \end{definition}
% Technically, the contraction defined above is in fact, a $d$-dimensional hyperrectangle but we want to essentially ignore one of its dimensions, the one along which it was contracted. So that, in the Euclidean space defined only by the vectors in directions $x_i$ for $i\in[d]\setminus \{j\}$, $h^j$ is a $(d-1)$-dimensional hyperrectangle. We then will refer to it as $h^j$ was obtained by contracting $h$ along direction $x_j$.

In this section we describe a reduction, that transforms a given set of points and hyperrectangles in
$d$ dimensions, first to a set of points and orthants in $2d$-dimension, and then a set of points and hypercubes
in $2d$-dimensions. We claim that this reduction \textit{preserves set intersections} (will be described
more formally later) and hence, we can run the given set cover algorithm on the new instance, and obtain a
solution for the original instance, with the same approximation ratio.

\begin{lemma}[Folklore] \label{lemma:orthantmap_1}
    Set intersections of orthants in $2d$-dimensions generalize set intersections of hyperrectangles
    in $d$-dimensions, for $d\in\mathbb{Z}^{+}$.
    % Set cover in orthants in dimension $2d$ generalizes set cover in hyperrectangles in dimension
    % $d$ for $d\in\mathbb{Z}^{+}$
\end{lemma}
\begin{proof}
    Given any $m$ hyperrectangles in $d$-dimensions (define a set $\mathcal{H}=\{h_1,\dots,h_m\}$ which
    contains exactly the given hyperrectangles), we will show a mapping $f$ such that any given
    hyperrectangle $h_i$ is mapped by $f$ to a $2d$-dimensional orthant $o_i$, i.e., $f(h_i)=o_i$ for
    $i\in[m]$. This mapping will preserve the set intersections which are present in the given instance
    of the $m$ hyperrectangles. In other words, consider a point $p\in\mathbb{R}^d$ such that the
    indices of the hyperrectangles which cover this point are specified by set $I$. That is, $p\in h_i$
    for $i\in I$ and $p\notin h_i$ for $i\notin I$. Then, there exists another point
    $p'\in\mathbb{R}^{2d}$ such that $p'\in o_i$ for $i\in I$ and $p'\notin o_i$ for $i\notin I$. First,
    we will define some notations: For a hyperrectangle $h_j\in H$, it is represented by $b_j$ (its
    bottom-justified-corner) and $v_j$ (its length vector). Note that we are consider all sets to be
    closed. In other words, boundary points for such sets are considered to be contained inside.

    We define our mapping now: For any dimension $i\in[d]$, consider the projection $\mathcal{H}_i$ of
    the hyperrectangles in $H$ in dimension $i$. That is, the projection of $h_j\in \mathcal{H}$ in
    dimension $i$ given by $h_j^i$ is an interval with left endpoint $b_j(i)$ and right endpoint
    $b_j(i)+v_j(i)$. Now, we sort the left endpoints of these intervals in non-increasing order (assume
    for simplicity that no two endpoints have the same coordinates) and the right endpoints of the
    intervals in a non-decreasing order. Based on this, we have two ordered lists $L_i$ and $R_i$, where
    $L_i$ contains the hyperrectangles such that the corresponding left endpoints of the intervals
    ($h_j^i$) as mentioned before are in a non-increasing order.  $R_i$ contains the hyperrectangles
    such that the corresponding right endpoints of the intervals ($h_j^i$) as mentioned before are in a
    non-decreasing order. In case two of the intervals had the exact same left coordinate, then they can
    be appropriately assigned the same rank in the list $L_i$ (similarly for the analogous case of
    $R_i$). Let the hyperrectangle $h_j$ have rank $L_i(j)$ in the list $L_i$ and rank $R_i(j)$ in the
    list $R_i$. Once we have the lists $L_i$ and $R_i$ for every dimension $i\in[d]$, for a
    hyperrectangle $h_j\in \mathcal{H}$, we can now describe the coordinates of the apex point for $o_j$
    (denote the apex point of $o_j$ by $a_j$) such that $f(h_j)=o_j$. For $i\in[d]$ and $j\in[m]$,
    $a_j(2i-1)=R_i(j)$ and $a_j(2i)=L_i(j)$.

    Having described the mapping, we will prove that the set intersections are preserved in it. Thus,
    consider a point $p\in\mathbb{R}^d$ and denote by the set $I$ the indices of the hyperrectangles in
    $H$ which cover this point. We will show that a point $p'\in\mathbb{R}^{2d}$ exists such that if
    $I'$ denotes the set of indices of the orthants which cover this point, then $I=I'$. Hence, we will
    first define the coordinates of $p'$. For any dimension $i\in[d]$, consider the two lists $L_i$ and
    $R_i$. Now, again consider $\mathcal{H}_i$ which is the set containing the projections of the
    hyperrectangles in dimension $i$ (the projections are just intervals).  When projected in dimension
    $i$, consider the hyperrectangle with the lowest rank in $R_i$ which covers $p(i)$ (projection of
    $p$ in dimension $i$) and denote its rank in $R_i$ by $R(p,i)$. Similarly, consider the
    hyperrectangle with the lowest rank in $L_i$ which covers $p(i)$ (projection of $p$ in dimension
    $i$) and denote its rank in $L_i$ by $L(p,i)$. Then, $p'(2i-1)=R(p,i)$ and $p'(2i)=L(p,i)$ for
    $i\in[d]$.

    Now, we prove that hyperrectangle $h_j\in \mathcal{H}$ covers $p$ if and only if $o_j$ covers $p'$.
    A hyperrectangle $h_j$ covers $p$ if and only if the projection of $p$ in every dimension $i\in[d]$
    is covered by the corresponding projection of the hyperrectangle $h_j$ in that dimension. Hence, if
    projection of $h_j$ in every dimension $i$ covers the projection of $p$ in that dimension, we will
    prove that $o_j$ covers $p'$. That is, we will prove that the apex point of $o_j$ dominates $p'$.
    Assume then for the sake of contradiction that there exists $g\in[2d]$ such that $p'(g)>a_j(g)$.
    Assume w.l.o.g. that $g$ is odd and $g=2u-1$ for some $u\in[d]$ (we can assume w.l.o.g. since a
    symmetric idea will work for $g$ if it was even). Hence, when we consider the projection of $p'$ in
    dimension $g$, we have that $p'(g)>a_j(g)$. But $a_j(2u-1)=R_u(j)$ and $p'(2u-1)=R(p,u)$. Now, it
    cannot be that $h_j$ covers $p$ and rank of $h_j$ in the list $R_u$ was less than $R(p,u)$, since
    $R(p,u)$ is the lowest rank of the hyperrectangle in that list which covers $p$ in dimension $u$.
    Hence, we arrive at a contradiction.

    Now, we prove the other direction which is that if $h_j$ in some dimension $u$ does not cover the
    projection of $p$ in that dimension, then $o_j$ will not cover $p'$. Denote the projection of $h_j$
    in dimension $u$ by $I_{u,j}$. Assume w.l.o.g. that its right endpoint lies to the left of $p'(u)$
    (The other case where the left endpoint of $I_{u,j}$ lies to the right of $p'(u)$ works out with a
    similar reasoning). Then, $R_u(j)<R(p,u)$ by definition and hence, $a_j(2u-1)<p'(2u-1)$, which
    implies that $o_j$ does not cover $p'$.
   % Note that for a mapped point $p'\in\mathbb{R}^{2d}$ as described above, we may perturb it with the
   % vector $v_{\text{pert}}\in\mathbb{R}^{2d}$ such that $v_{\text{pert}}(i)=-0.5$ for every $i\in[2d]$.
    %That is, $p'_{\text{new}}=p'+v_{\text{pert}}$, where $p'_{\text{new}}$ denotes the perturbed point.
   % This shifting operation does not alter the collection of orthants that $p'$ was contained in due to
   % the fact that the apex points of the orthants have positive integer coordinates. 
\end{proof}

\begin{figure}[ht]
    \centering
    \includegraphics[page=5,scale=0.7]{02SetCovHitSet.pdf}
    \caption{Converting an instance on intervals to an equivalent one on quadrants.}
    \label{fig:interval-to-quadrant}
\end{figure}

\begin{lemma} \label{claim:rectangledynsetcover_1}
    Any given instance $\mathcal{Q}_1$ of $m$ hyperrectangles in dimension $d$ can be reduced to an instance $\mathcal{Q}_2$ of $m$
    hypercubes in dimension $2d$ with all of their corners having integer coordinates lying in the
    region $[0,2m]^{2d}$ such that the set intersections in $\mathcal{Q}_1$ are preserved in
    $\mathcal{Q}_2$.
\end{lemma}
\begin{proof}
    It is shown in Lemma~\ref{lemma:orthantmap_1} that every hyperrectangle $h_i$
    in the given instance $\mathcal{Q}_1$, can be mapped to an orthant $o_i$ in $2d$ dimensional
    space in another instance $\mathcal{Q}_2$, with the condition that set intersections are preserved in
    $\mathcal{Q}_2$ from $\mathcal{Q}_1$. That is, once the adversary introduces a point $p$ in
    $\mathcal{Q}_1$, we can map $p$ to another point $p'$ in $\mathcal{Q}_2$ such that the set of
    indices of the hyperrectangles which cover $p$ in $\mathcal{Q}_1$ is the same as the set of indices
    of the orthants in $\mathcal{Q}_2$ which cover $p'$. Also, in this reduction since the intercepts of
    any orthant in $\mathcal{Q}_2$ are integers in $[m]$, we can essentially extend the sides of the
    orthant sufficiently in order to get hypercubes. More specifically, for any orthant
    $o_i\in\mathcal{Q}_2$, let its apex point be denoted by the vector $(a_1,a_2,\dots,a_{2d})$. From
    Lemma~\ref{lemma:orthantmap_1} we have that all of the $a_j$s are integers in $[m]$. Then, for $o_j$
    define $a'=\max_{j\in[m]}a_j$. Then, we do the following: for every $j\in[2d]$, define $y_j=a_j-a'$.
    We define the hypercube $H_i=[y_1,a_1]\times [y_2,a_2]\times \dots\times [y_{2d},a_{2d}]$. Because of
    the choice of the $y_j$s, all the sides of $H_i$ are equal. Note also that $y_j\leq 0$ for every
    $j\in[2d]$. Hence, extending the sides of the orthants in the negative direction does not change
    anything as far as the set intersections are concerned (for the points introduced by the adversary).
    Now, by a coordinate shift of $\{-m\}^{2d}$, we can assume that $\mathcal{Q}_2$ is an instance with
    $m$ $2d$-dimension hypercubes (with corners having integer coordinates) contained completely in the
    region $[0,2m]^{2d}$.
\end{proof}


\begin{lemma}
    An $\alpha$-approximate set cover algorithm on hypercubes in $\R^{2d}$
    provides an $\alpha$-approximate set cover algorithm on hyperrectangles in $\R^{d}$.
\end{lemma}
\begin{proof} Let $\mathcal{I}_1=(U_1, \mathcal{Q}_1)$ be a given set of set cover on
 hyperrectangles in $\R^d$. Using \Cref{claim:rectangledynsetcover_1} we can construct a set of
 hypercubes $\mathcal{Q}_2$ in $\R^{2d}$. Now, every time the adversary introduces any point
 $p\in\mathbb{R}^d$ in $\mathcal{Q}_1$; using \Cref{lemma:orthantmap_1} and a coordinate shift of $\{-m\}^{2d}$ (as used in previous Lemma), we can map it to another point $p'\in\mathbb{R}^
 {2d}$ such that set intersections remain the same for $p'$ as they were for $p$ in $\mathcal
 {Q}_1$. That is, the set of indices of the hyperrectangles which cover $p$ in $\mathcal{Q}_1$ is
 the same as the set of indices of the hypercubes which cover $p'$ in $\mathcal{Q}_2$. Denote by
 $U_2$ the set of mapped points in $\mathcal{Q}_2$ from $U$ by the aforementioned mapping to get
 set cover instance $\mathcal{I}_2=(U_2, \mathcal{Q}_2)$. Then our mapping (for sets and points)
 implies that a hyperrectangle $h_i\in\mathcal{Q}_1$ covers the set of points in $U_1$ indexed by
 $I$ if and only if the mapped hypercube $o_i\in\mathcal{Q}_2$ covers the exact set of points
 indexed by $I$ in $U_2$. Hence, by solving the set cover instance on hypercubes in $2d$ dimensions
 in $\mathcal{Q}_2$, and taking the corresponding hyperrectangles in $\mathcal{Q}_1$, we obtain a
 valid set cover in the original instance (of the same size).

Also note that this means that the optimal solution $\OPT_1$ to $\mathcal{I}_1$, is a valid set cover
for $\mathcal{I}_2$, and similarly the optimal solution $\OPT_2$ to $\mathcal{I}_2$, is a valid set
cover for $\mathcal{I}_1$. So, we have $|\OPT_1|\ge|\OPT_2|$ and $|\OPT_2|\ge|\OPT_1|$, giving us
that $|\OPT_1|=|\OPT_2|$. This means that an $\alpha$-approximate solution for set cover on
$\mathcal{I}_2$ will also correspond to an $\alpha$-approximate solution for set cover on $\mathcal
{I}_1$.
\end{proof}

We note here that the reduction described in \Cref{lemma:orthantmap_1} projects the given
hyperrectangles on to each dimension, and sorts their (projected interval's) left and right end
points to obtain the required coordinates for the orthants. This will take a total of $O_d
(m\log m)$ time. Further we only do a constant time extension of each orthant to get a
hypercube, and then translate it accordingly. This takes $O(m)$ time. Note further that
reading the each input hyperrectangle itself takes $O(\log N')$ time, giving us an overall pre-processing
time of $O_d(m(\log m + \log N'))$ to construct the set of hypercubes.

Now on arrival of a point, we need to binary search in the ordered list of left and right end points
of the projected hyperrectangles (computed earlier), to find an appropriate projection covering the point.
With this information we can compute the corresponding point in the new instance in constant time.
Accounting for the time to read the input point, the update operation takes $O_d(\log m+\log N')$ time.


% Thus, to be able to implement the aforementioned mapping of any point from $\mathcal{Q}_1$ to
% $\mathcal{Q}_2$, we consider the set $\mathcal{H}_i$ for every $i\in[d]$. $\mathcal{H}_i$ is the set
% of the projections of the hyperrectangles in dimension $i$, which is just a set of intervals. Denote
% the set of endpoints in $\mathcal{H}_i$ by $E_i$, where $E_i$ is a sorted list according to
% increasing coordinate values of the points. Then, every element $e$ in this set, which is one of the
% endpoints of the projection of some hyperrectangle $h\in \mathcal{H}$ in dimension $i$, we maintain
% \begin{enumerate}
%     \item  the hyperrectangle whose projection in $i$ with the lowest rank in $R_i$ and covers the point with coordinate $e$ in 1-dimension (denote this hyperrectangle by $R_e^i$);
%     \item the hyperrectangle whose projection in $i$ with the lowest rank in $L_i$ and covers the point with coordinate $e$ in 1-dimension (denote this hyperrectangle by $L_e^i$).
% \end{enumerate}

% When a point $p$ is introduced by the adversary in $\mathcal{Q}_1$, we consider $p(i)$ for
% $i\in[d]$. Then, we search for the largest element $e_L\in E_i$ which is at most $p(i)$. We also
% search for the smallest element $e_S$ which is at least as much as $p(i)$. Assuming none of the
% intervals in $H_i$ are just single points for simplicity, we map $p$ to $p'\in\mathbb{R}^{2d}$ where
% $p'(2i-1)=R_{e_L}^i+v$ and $p'(2i)=L_{e_S}^i+v$ (here, $v\in\mathbb{R}^{2d}$ such that $v_k=-0.5$
% for $k\in[2d]$). Hence, from now we will assume that the adversary introduces points in
% $\mathcal{Q}_2$ according to our mapping.







\subsection{Proof of \Cref{lem:few-sets}\label{subsec:few-sets}}
%\fewsets*
%\akr{mixing $N$ and $m$, be uniform.}
\begin{proof}
     For any point $p\in P$, consider the $1$-dimensional nodes $v$ in $T'$ such that a projection of $p$ is included in $P_v$. We will prove that there are $O_d({\log^{2d-1}N})$ such nodes (denote this set of leaves by $V_p''$). We will prove  this claim by induction on the dimension $d'$.    The induction hypothesis on the dimension $d'$ is as
    follows: If a projection of $p$ is included in $S_v$ for a $d'$-dimensional node $v$ in $T'$,  then
    the number of leaves in the subtree of $v$ such that $C_v$ contains a projection of $p$ is at most $2^{d'-1}(d'!)({\log^{d'-1}N})$.

    For the base case $d'=1$, observe that the number of leaves is exactly $1$. Then, the induction hypothesis trivially holds for the base case. Now, consider the
    general case of a node $v$ in $T'$ of dimension $d'>1$ where  $C_v$ contains a projection of $p$
    (denote the projection by $p'$). Consider first only the $d'$-dimensional descendants of $C_v$ which contain $p'$ (denote by $\C_{v,p}$). These are at most $2\log N$ in number. Since there are no other $d'$-dimensional nodes whose corresponding cells which contain $p'$,    we consider now the $(d'-1)$-dimensional cells which contain a projection of $p$. Any such cell has to be a projection child of one of the cells in $\C_{v,p}$. Then,
    consider one of these cells, say $C_{v'}$ and note that it has $d'$ projection children. Each of these cells (denote one of them by $C_{v''}$) correspond to $(d'-1)$-dimensional nodes and they all contain a projection of $p$ by definition of the extended quad-tree $T'$. We
    apply our induction hypothesis on these cells. That is, we get that the number of leaves from the set $V_p''$ in the subtree of ${v''}$ is at most $2^{d'-2}((d'-1)!)({\log^{d'-2}N})$. Note that there are $d'$ such children $v''$ and adding over all of them we get that the subtree of ${v'}$ has at most $2^{d'-2}(d'!)({\log^{d'-2}N})$ leaves which are included in $V_p''$. Finally, noting that $\C_{v,p}$ has at most $2\log N$ members, the induction step holds for the node $v$. Then, we can claim that the number of leaf nodes in the subtree, which are also members of $V_p''$ is at most $4(d!)({\log^{2d-1}N})$.
    %\akr{Show the calculation details for the induction. Use $(d'-1)!{\log^{d'-1}N}$}
    
    Note that  $V_p'$ is the set of all nodes $v$ for
   which a projection of $p$ is contained in $C_v$. Now, we observe the fact that if a projection of $p$ is contained in $C_v$ for some node $v$ which is not a leaf, then there exists at least one subdivision child $v'$ of $v$ for which $p\in\S_{v'}$. Using this claim and  the fact that the depth of $T'$ is $O(2d+\log N)$, we get that $|V_p'|=O((2d+\log N)|V_p''|)$. 
    %The induction hypothesis: For any instance of $m$ hypercubes in $\mathbb{R}^d$, the number of sets over instances in $\mathcal{I}_C$ over cells $C\in T$ that cover a point $p\in U_2$ is $O_d(\log^{d-1}m)$. Note that the base case holds for $d=2$ from Lemma~\Cref{lem:squaresetcover_1} since there are $O(\log m)$ cells which contain the given point $p$ and there are at most $4$ squares per cell which may cover it. Thus, we just have to observe that in the original instance in $\mathbb{R}^{2d}$, any point $p$ is contained in $O(\log m)$ cells. For any of these cells $C$, consider over all non-empty sets $s\in[2d]$, the appropriate projections of the instance w.r.t the directions denoted by the elements of $s$. Applying the induction hypothesis for any dimension instance $d'\leq 2d-1$, we get that for the cell $C$, the number of sets containing the point $p$ is at most $4^d\cdot O_d(\log^{2d-2}m)$. Then, summing over all those $O(\log m)$ cells, we prove the induction step.
\end{proof}

\subsection{Proofs of \Cref{lem:rectangleDynSetCov_covering} and  \Cref{lem:cheap-solution} \label{subsec:rectangleDynSetCov_covering}}
We will prove  \Cref{lem:rectangleDynSetCov_covering} and  \Cref{lem:cheap-solution} in this subsection. But, first we need a few definitions.

%\subsection{Definition of edge-covering hypercubes and Proof of \Cref{clm:secc_1}\label{subsec:secc_1}}
%\begin{claim} %We show by a  corner charging argument that $\mathcal{C}_S$ contains at most
   % $16^d$ cells at each level. For $S$, consider any of its corners $u\in\mathbb{R}^{2d}$. Then, we
   % know that $u$ is present in $O(\log m)$ cells (the root included). Consider the lowest level cell
   % (denote by $C$) among these cells such that \ak{at least one of its children ($C'$) through
    %subdivision edges in the extended quad-tree is edge-covered by $S$}. Then, by definition $S$
    %is \secc~ for $C'$. The number of such children $C'$ (which are the same level) is at most $2^
   % {2d}$. Applying this idea for all of the corners of $S$, we get that at most $2^{2d}\times 2^
  %  {2d}=16^d$ cells are included in $\mathcal{C}_S$. To prove that by this argument no cell $\tilde
   % {C}$ was missed for which $S$ was \secc, observe that if $S$ was indeed \secc~ for $\tilde
   % {C}$, then it was not edge-covering for its parent $\tilde{C}_{\text{par}}$. This can only happen
  %  if one of the corners of $S$ was inside $\tilde{C}_{\text{par}}$, which completes the argument.
%\end{claim}

\begin{definition}
A hypercube $S\in\S$ is edge-covering for a cell $C_v\in V'$ (assuming $S\in\S_v$) if there exists a dimension $i\in[2d]$ such that the (non-empty) projection of $S\cap C_v$ to dimension $i$ equals the projection of $C_v$ to dimension $i$.
\end{definition}

Now, consider a hypercube $S$ in the instance and a $d'$-dimensional node $v$ such that $S\in\S_v$. Denote by $\C_v$ the $d'$-dimensional cells in the subtree of the node $v$. Among the cells in $\C_v$, denote by $\C_{v,S}$ the cells for which $S$ is edge-covering but is not edge-covering for its parent. Then, we have the following claim.

\begin{claim}
\label{clm:hypercubecrossing}
Consider a node $v$ such that $S\in\S_v$. Denote by $v'$ a subdivision child of $v$. Then, if $S$ is edge-covering for $C_{v'}$ but not for $C_v$, then there exists at least one corner $u$ of $S$ such that a projection of $u$ is contained in $C_v$.        
\end{claim}
\begin{claim}
Assume that there existed no such corner $u$ of $S$ and also that $S$ was not edge-covering for $C_v$. Then, $S$ projected to the set of dimensions of $C_v$ is a crossing hypercube for $C_v$ which is not possible.
\end{claim}
Then, we have the following claim.

%
%\begin{definition}
%A hypercube $S$ is supremal edge-covering for a cell $C_v\in T$ if the projection of $S$ in $C_v$ is edge-covering for $C_v$. However, for the parent of $C$ connected to it by a subdivision edge (denote by $C_v'$), the projection of $S$ in $C_v'$ is not edge-covering.
%\end{definition}

%In $T'$, for a hypercube $S$ define the set $\mathcal{C}_S$ to be the collection of $2d$-dimensional
%cells for which it is supremal edge-covering.  We will mention a few observations here.
\begin{claim} \label{clm:secc_1}
Then, $|\C_{v,S}|= 4^{d'}\cdot O(\log N)$.
\end{claim}
%\begin{claim}
 %   Deferred to \Cref{subsec:secc_1}
%\end{claim}
%\subsection{Proof of \Cref{clm:secc_2}\label{subsec:secc_2}}
\begin{claim} 
We show by a  corner charging argument that $\mathcal{C}_{v,S}$ contains at most
$4^{d'}$ cells of the same side-length. For a hypercube $S$, consider any of the corners $u\in\mathbb{R}^{d'}$ of $S\cap C_v$. Then, we
know that $u$ is present in $O(\log N)$ cells in $\C_v$. Consider the lowest side-length cell
(denote by $C_v$) among the cells in $\C_v$ such that at least one of its subdivision children ($C_{v'}$) is edge-covered by $S$. Let the side-length of $C_v$ be $s$. Then, by definition $C_{v'}$
is contained in $\C_{v,S}$. The number of such children $C_{v'}$ (which have side-length $s$) is at most $2^
{d'}$. Applying this idea for all the $2^{d'}$ corners of $S$, we get that at most $2^{d'}\cdot 2^
{d'}=4^{d'}$ cells of side-length $s$ in $T'$ are included in $\mathcal{C}_{v,S}$. 
We will prove  by contradiction that no cell $
{C_{v_0}}$ of side-length $s$ was excluded from $\C_{v,S}$ for which $S$ was edge-covering but not for its parent. Assume then that $S$ was edge-covering for $C_{v_0}$ but it was not edge-covering for its parent $C_{v_0'}$. This can only happen
if one of the corners of $S$ was inside $C_{v_0'}$. To finish the argument, for each corner $u$ of $S\cap C_v$, we recursively find the cells to be included in $\C_{v',S}$, where $v'$ is the node whose corresponding cell has side-length $s/2$ and contains the corner $u$.

Note that we can compute the nodes $v$ for which $C_v\in\C_{v,S}$ in $O_d(\log N)$ time. We will mention the respective running times of all of the operations needed while computing $\C_{v,S}$. We will follow the idea in the above paragraph to compute $\C_{v,S}$. The operation of finding whether the corner of a hypercube lies in a particular cell can be performed in $O_d(1)$ time. For a hypercube, all of its children can be found in $O_d(1)$ time. Also, projection of a hypercube to any cell in $V'$ can be done in $O_d(1)$ time. Finding out whether $S\cap C_v$ is edge-covering for it can can be done in $O_d(1)$ time. The above operations show that we can find all of the cells of a fixed side-length $s$, which are members in $\C_{v,S}$ in $O_d(1)$ time. Then, applying this idea over all of the possible side-lengths of the cells in $T'$ implies that we can compute all the cells in $\C_{v,S}$ in $O_d(\log N)$ time.
\end{claim}


\paragraph*{Definition of $V_S'$}
We define a set of nodes $V_S'\subseteq V'$ (used in proving \Cref{lem:cheap-solution}) for $S\in\S$ recursively. For each $S\in\S$ and for a $d'$-dimensional node $v\in V'$, we will  define the nodes in the subtree of $v$ that we include in $V_S'$.  Hence, finally $V_S'$ is defined by considering all the nodes that we include for the subtree of the root $r$. In case $S\notin \S_v$, we do not add anything from this subtree to $V_S'$. Else if $S$ was edge-covering for $C_v$ but also for its parent, we do not add anything from its subtree to $V_S'$. Else,  we consider the set $\C_{v,S}$ (defined in \Cref{subsec:rectangleDynSetCov_covering})  and process the nodes in this set in a BFS-like manner. For a node $v'\in \C_{v,S}$, in case $S\cap C_{v'}$ is facet-covering for ${v'}$, we add $v'\in V_S'$. Else, for a node $v'\in\C_v'$ we consider all of its projection children $v''$ such that $S\in\S_{v''}$. Such a child $v''$ exists since $S$ was edge-covering for $C_{v'}$. Then, recursively add nodes in the subtree of $v''$ to $V_S'$ (do this over all the children $v''$). We will show %in \Cref{lem:cheap-solution} 
now that $|V_S'|=O_d(\log^{2d-1}N)$.as

\begin{claim}
\label{proofofvsprime}
For a hypercube $S\in \S$, $|V_S'|=O_d(\log^{2d-1}N)$.
\end{claim}
\begin{claim}
 We show that $|V_S'|=O_d(\log^{2d-1}N)$ by induction on the dimension $d'$ for any set $S\in S$. The induction
    hypothesis is as follows: For a given $d'$-dimensional node $v$ such that $S\in\S_v$, the number of nodes in its subtree which are contained in $V_S'$
    is $2^{d'-1}(d'!)\cdot 4^{\frac{(d')(d'+1)}{2}}({\log^{d'-1}N})$.

    The base case follows trivially when $d'=1$ since it is a leaf. Now, consider the general case of a
    cell $C_v$ in $T'$ of dimension $d'>1$ such that $S\in \S_v$. Then, as per the definition of $V_S'$, if $S\cap C_v$ was edge-covering for $C_v$ but also for its parent, then we do not add any node from the subtree of $v$ to $V_S'$. Hence, we consider the set $\C_{v,S}$ which is of size at most $4^{d'}\cdot (\log N)$. Then, we add all those nodes $v_0$ from this set to $V_S'$ for which $S\cap v_0$ was facet-covering. For any other node $v'$, there are at most $d'$ projection children $v''$ for which $S\in\S_{v''}$. Then, for each such $v''$ which is $(d'-1)$-dimensional, we can use induction hypothesis to get that at most $4((d'-1)!)\cdot 4^{\frac{(d'-1)(d')}{2}}({\log^{d'-2}N})$ are included in $V_S'$ from this subtree. Since the number of such children $v''$ for each $v'$ is at most $d'$ and the number of such nodes $v'$ in the subtree of $v$ is at most $4^{d'}\cdot 2(\log N)$, we get that the number of nodes in $V_S'$ in the subtree of $v$ is at most
    \[2^{d'-2}((d'-1)!)\cdot 4^{\frac{(d'-1)(d')}{2}}({\log^{d'-2}N})\cdot( 4^{d'}\cdot d'\cdot 2\log N)+4^{d'}\cdot (\log N)\leq 2^{d'-1}(d'!)\cdot 4^{\frac{(d')(d'+1)}{2}}({\log^{d'-1}N})\]
    which implies that the induction holds. 
\end{claim}


We will now prove the following lemma, which is of a similar nature as \Cref{lem:rectangleDynSetCov_covering}. The arguments for proving \Cref{lem:rectangleDynSetCov_covering} will follow in a similar vein.

%\asr{cannot link to lem14 cuz statement is subtly different.}
% \rectangleDynSetCovcovering*
\begin{lemma}
\label{lem:rectangleDynSetCov_covering1}
A hypercube $S\in\mathcal{S}$ covers a point $p\in P$ if and only if there exists a node $v$ in $V'_S$ such that a projection of $p$  is included in $P_v$ and this projection is covered by $S\cap C_v$ such that $S\cap C_v$ is facet-covering for $v$. 
\end{lemma}
\begin{proof}
    %For $d=2$, we know that the claim holds from Lemma~\Cref{lem:squaresetcover_1}. For a general dimension $d'=2d$, if $h$ covers a point $p\in U_2$, then consider the cell $C$ in which $h$ covers $p$ for which it was \secc (such a cell exists by definition). Then, when we project to get another instance $I'$ based on the set $s$ (the set containing the indices of the directions for which $h$ completely covered those edges), the projection of $h$ in this new instance continues to cover the projection of $p$ due to the fact that shrinking along the directions for which $h$ was edge-covering for $C$ does not alter containment of $p$ in it. On this new instance, we apply our induction hypothesis to get the claim in one direction. For the other direction, the claim follows similarly in the sense that if $h$ did not cover $p$ in the original instance, even after projecting in any of the cells for which it was \secc, $h$ will not continue to cover $p$ in any of the resulting instances in any of those cells.
   % Let's say $S\in\S$ covers $p\in P$ in the root cell (which is $2d$-dimensional) of $T'$. Then, we
    %consider the decomposition of $S$ in the cells $\mathcal{C}_{S}$ as according to
    %Claim~\Cref{claim:rectangledynsetcover_3}.  know that according to the definition of set assignment
   % in $T'$, there exists a $2d$-dimensional cell $C$ such that $C$ contains $p$ and $C\in\mathcal{C}_S$
   % or equivalently, $S$ was \secc~ for $C$ by an argument similar to the one in the proof of
   % \Cref{claim:hyperrectangle_secc}. Hence, a projection of $S$ is assigned to $C$. The point $p$ is
    %also assigned to $C$. %Then, there exists an equivalent $2d$-dimensional node in $T_{\text{ex}}$
    %corresponding to $C$ to which both $H$ and $p$ both were assigned and the assigned part of $H$
    %inside $C$ covered $p$.  If $C\in V_S'$, we are done. Else, since $S$ was \secc~ for $C$, there
   % exists some child (by virtue of a projection edge) of $C$, say $C'$ in which the projection (by
   % contracting along the specific direction for which $S$ was edge-covering for $C$) of $S$ covered the
    %appropriate projection of $p$. Note that $C'$ is a $(2d-1)$-dimensional node in $T'$ and we can
   % essentially keep using the argument until we eventually reach a node in $V_S'$ (which can be either
   % an uncovered leaf or a covered node for which the projection of $S$ covered it entirely).
Consider the forward direction first. Let's say $S\in\S$ covers $p\in P$ inside the root cell $C_r$. Then, we consider the $2d$-dimensional nodes in $V'$ for which $S$ is edge-covering but not for the parent (denoted by $\C_{r,S}$). There exists at least one such node in this set since the side-length of $S$ is at least $1$ (which is the side-length of the smallest $2d$-dimensional cell in $T'$). Denote this node by $v$ and if $S\cap C_v$ is facet-covering for $v$, the claim statement holds since $v\in V_S'$. Else, $S\cap C_v$ is edge-covering for it but not for its parent. Then, there exists a projection child $v'$ of $v$ such that $S\in\S_{v'}$ and $S\cap C_{v'}$ covers the projection of $p$ in $C_{v'}$. According to the definition of $V_S'$, we will add nodes from the subtree of $v'$ to $V_S'$ recursively. Note that $v'$ is a $(2d-1)$-dimensional node and we can apply our argument iteratively until we eventually reach a node $v_0$ such that $v_0\in V_S'$ and $S$ and $S\cap C_{v_0}$ covers the projection of $p$ in $v_0$.  


For the reverse direction, consider a node $v\in V_S'$ such that  $S\cap C_v$ covered
the projection of $p$ in it. If $C_v$ is not trivially the root of $T'$, we need to show that $S$ covers $p$
inside the root node. Then, let's say this cell $C$ is a $d'$-dimensional node for $d'\leq 2d$. If
    $d'=2d$ and $C_v$ is not the root node, this means that $S$ was facet-covering for the $2d$-dimensional node $v$
     and $S\cap C_v$ covered the projection of $p$ in $C_v$. But since $v$ is $2d$-dimensional, this trivially implies our claim. %Since the root is connected to $C$ by subdivision edges , $H$
    %was \secc for $C$ and in fact, $H\cap C=C$.  This means trivially that $S$ covered $p$ inside the
    %root node.  
    If $d'<2d$, we will show that there indeed exists another node $C_{v_0}$ which is
    $(d'+1)$-dimensional in $T'$ such that $S\cap C_{v_0}$ covered the projection of $p$ in $C_{v_0}$. Then, we can just apply the argument iteratively until we reach the root of $T'$, which will imply the reverse direction of the claim. %such
    %that the projection/part of $H$ inside $C$ was 
    Denote the parent of $v$ by $v'$.
    Now, observe that by definition, if $v$ was a subdivision child of $v'$, then
    $S\cap C_{v'}$ covered the projection of $p$ in $v'$. This holds because of the fact that $(S\cap C_v)\subseteq (S\cap C_{v'})$. Else if $v$ was a projection child of $v'$, this again implies that $S\cap C_{v'}$ covered the projection of $p$ in $v'$ since if $S\cap C_{v'}$ did not cover the projection of $p$ in $C_{v'}$, projection of $S\cap C_{v'}$ along any set of directions will not cover the corresponding projection of $p$.
    %$p\in P_{v}$, then $p\in P_{v'}$. If $v$ was a subdivision child of $v'$, the projection of $S$ in $C'$
    %covered the projection of $p$ in $C'$. If the connecting edge was a projection edge then the
    %projection of $S$ in $C'$ was \secc~ for it. Hence, the projection of $S$ covered the projection of
    %$p$ in $C'$. 
    Now, observe that since $C_v$ was $d'$-dimensional for $d'<2d$, we can keep on applying the previous argument until we reach a $(d'+1)$-dimensional node $v''$ (ancestor of $v$) in which $S\cap C_{v''}$ covers the projection of $p$ in $v''$.
    We can keep on applying the argument because the graph $T'$ is connected and has only a finite number of nodes. Hence, in some iteration of this argument we will reach a node $\tilde{v}$ which is a projection child for its parent $v''$ and then, we know that $v''$ was $(d+1)$-dimensional.  
    %Hence, either $C$ is an immediate-subdivision root or there exists an immediate-subdivision root
    %for $C$ (denote by $C''$) for which $H$ is assigned by Claim~\Cref{clm:rectangleDynSetCov_4}. Also,
    %since there exists a path of subdivision edges from $C$ to $C''$, $p$ was also assigned to $C''$ by
    %definition. In either case, the parent of $C''$ is connected to it by a projection edge in $T'$.
    %Hence, the parent of $C''$ (denote by $C_{\text{par}}''$) is a $(d'+1)$-dimensional node and
    %necessarily has $p$ assigned to it as well as the projection of $S$ in it. Then, we infer from
    %Claim~\Cref{claim:rectangledynsetcover_5} that the projection/part of $H$ assigned to
    %$C_{\text{par}}''$ covers the projection of $p$ in it.
\end{proof}


%\akr{this claim should be moved to appendix or under the time complexity analysis of dynamic algorithm}


% \paragraph*{Proof of \Cref{lem:cheap-solution}\label{subsec:cheap-solution}}
%\cheapsolution*
\begin{proof}
We know that for any set $S\in\S$, $|V_S'|=O_d(\log^{2d-1}N)$ from \Cref{proofofvsprime}. Then, for $S\in\OPT$, we will define a subcollection $\S'\in\hat{\S}$ of size $O_d(\log^{2d-1}N)$. This subcollection has the property such that if $p\in P$  was covered by $S$, then $p$ was covered by at least one set in $\S'$. 
First, for any node $v\in
V_S'$  we map at most $4d$ sets in $\hat{\S}$ to it. Denote this subcollection  of size at most $4d$ by $\S'_v$. %$\S'_v$ is such that for any $p\in P$ such that the projection of $p$ was covered by the projection of $S$ in the cell for
%$v$, one of the sets in it 
%necessarily covers $p$ in $(P,\hat{\S})$.  
For any node $v\in V_S'$, for $S'\in\S$, include $(v,S')$ in $\S'$ if $(v,S')$ was defined according to our auxiliary set cover instance definition. Since $S\cap C_v$ was facet-covering for $v$, there existed a maximal facet-covering hypercube $S_0\in\S$ such that $S_0\cap C_v\supseteq S\cap C_v$. 
Then, $(v,S_0)$ was included in $\S_v'$. Further, applying~\Cref{lem:rectangleDynSetCov_covering}, we get that for any $p\in P$ such that the projection of $p$ was covered by $S\cap C_v$, $S_0$ 
necessarily covered $p$ in $(P,{\S})$. Since, $(v,S_0)$ is defined, we have that $p$ was covered by a set in $\S'_v$ in $(P,\hat{\S})$. 
Define another subcollection $\S'=\bigcup_{v\in V_S'}\S'_v$. Then, we have that $|\S'|=O_d(\log^{2d-1}N)$. Observe that if a hypercube $S$ covered a point $p\in P$ in the original set system, then by \Cref{lem:rectangleDynSetCov_covering1} there exists a node  $v'\in V_S'$ such that the projection of $p$ in $C_{v'}$ was covered by $S\cap C_{v'}$. By definition, $S\cap C_{v'}$ was facet-covering for $v'$. Note that $v'\in  V_S'$ and hence, applying the same argument as before, there exists a set $S_0$ such that $(v',S_0)\in \S'$ and $(v',S_0)$ covered $p$ in $(P,\hat{\S})$. Also, $S_0$ covered $p$ in $(P,\S)$.
    \iffalse{We will prove that for any $S\in\S$, $|V_S'|=O_d(\log^{2d-1}N)$. But assuming this is true, we show
    how the lemma follows. We show that for any hypercube $S\in\OPT$, we can pick a subcollection of
    sets of size $O_d(\log^{2d-1}N)$ such that any point $p$ that $S$ covered in $(P,\S)$ is covered by
    one of these sets as per the new set system $(P,\hat{\S})$. This subcollection is defined in the
    following manner: For every cell $C\in V_S'$, if $C$ is a covered node, then pick the covering set
    in the subcollection. Else, it is an uncovered leaf in which case pick the two maximal intervals for
    it. Observe that in either case, the projection of $S$ in $C$ gets covered by the set we pick in the
    subcollection for $C$. More specifically, by \Cref{lem:hyperrectangle_cover}, we have that the
    projection of point $p$ is assigned to some cell in $V_S'$.  Then, the set picked in the
    subcollection for $C$ is present in $\hat{\S}_p$. This implies the main claim.}\fi
    %consider the nodes in $T'$ in
    %the subtree of $C$ which are $d'$-dimensional and have a projection of $S$ assigned to them but this
    %projection of $S$ is not edge-covering. 
%No node in the subtree of such nodes can be contained in
 %   $V_S'$. For the other $d'$-dimensional nodes in the subtree of $C$ for which the projection of $S$
  %  is \secc, by definition of $T'$, there can be  $O_d(\log N)$ such nodes from
   % \Cref{claim:hyperrectangle_secc} (denote this set by $\mathcal{C}'$). Denote one such node in
    %$\mathcal{C}'$ by $C'$. Now, either $C'\in V_S'$ in which the subtree of $C'$ contains just $1$ node
    %from $V_S'$ or we need to consider the children of $C'$ connected by projection edges. Hence, there
    %exists a child of $C'$ (say $C''$) connected via a projection edge with $C'$ for which a projection
    %of $S$ is assigned (since the projection of $S$ in $C$ was edge-covering). In fact, $S$ may be
%    assigned to at most $d-1$ children of $C'$ depending on the edges of $C$ for which it was \secc.
 %   Note now that $C''$ is a $(d'-1)$-dimensional node to which a projection of $S$ is assigned. Then,
  %  such a child may exist for each node in $\mathcal{C}'$. Applying the induction hypothesis for all
   % such child nodes validates the induction step.
\end{proof}
    %Hence, consider the decomposition $\mathcal{C}_S$ of $H$ with $C$ as the reference root (according to Claim~\Cref{claim:rectangledynsetcover_3}). Note that $\mathcal{C}_S$ has only $d'$-dimensional nodes and in the subtree of $C$, the only $d'$-dimensional nodes that $S$ is assigned to are exactly the ones in this set (apart from maybe $C$). For each of these corresponding cells, the projection of $S$ contained in it is \secc for some edge of the cell. Then, having considered all the $d'$-dimensional nodes that $S$ was assigned connected to $C$ by subdivision edges, we consider the other children in the extended quad-tree to which $S$ may be assigned. These are precisely the ones connected by projection edges. For each such cell in $\mathcal{C}_S$, $S$ may be assigned to at most $d'$ of its children (connected by projection edges), where each such child node is a $(d'-1)$-dimensional immediate-subdivision-root. Here, we apply the induction hypothesis to justify the induction step.

    %Note that once we prove the aforementioned claims, consider a hypercube $H$ and its set of nodes/cells defined by ${T_{\text{ex}}}_H$. Then, consider a node in ${T_{\text{ex}}}_H$, say $C$. Suppose there exists a minimal depth ancestor of $C\in T_{\text{ex}}$ denoted by $C'$ which is a covered node by some other hypercube $H'$. Then, using the third claim in the list, we can infer that  $H'$ covers all the points in $U_2$ that were assigned to $C$. In turn, it means that all the points in $C$ whose projections were covered by the appropriate projection of $H$ in this cell will be covered by $H'$. Observe that, $H'$ covered all of these points even in the new set system $(U_3,\mathcal{F}_3)$. If $C$ was not an uncovered leaf node, then there definitely exists such a minimal depth ancestor for it, since $H$ is a candidate. Else, $C$ was an uncovered leaf node in which case, the projection of $H$ in this cell was an interval. Then, the two maximal intervals (one for each direction if they existed) covered $H$.

    %Now, observe that if $H\in\OPT$, consider the nodes/cells in ${T_{\text{ex}}}_H$. These are  $O_d(\log^{2d-1}m)$ many cells in $T_{\text{ex}}$. Consider now any instance of set cover in the older set system defined by $(U_2',\mathcal{F}_2)$, where $U_2'\subseteq U_3$ denotes the set of points to be covered. Consider now the set of points that $H$ covers in $(U_2',\mathcal{F}_2)$. Then, using the aforementioned ideas
    %we have a collection of $O_d(\log^{2d-1}m)$ hypercubes whose union completely covers those points according to $(U_3',\mathcal{F}_3)$, where $U_3'\subseteq U_3$ refers to the equivalent set of points as $U_2'$.

    %Denote the optimum set cover for $(U_2',\mathcal{F}_2)$ by $\OPT_{\text{old}}$. Denote the optimum set cover for $(U_3',\mathcal{F}_3)$ by $\OPT_{\text{new}}$. Then, we claim that $\OPT_{\text{new}}=O_d(\log^{2d-1}m)\cdot\OPT_{\text{old}}$.


%\subsection{Proof of \Cref{clm:rectangleDynSetCov_access}\label{subsec:rectangleDynSetCov_access}}
%\begin{claim}
 
%\end{claim}
    %For the weighted case, the only difference is that we can first round the weights to nearest higher powers of $2$ and then, in the resulting 2d instances according to the processing step, need to consider only $\log W$ weight classes for the square pieces which are maximal in every direction. This adds another $O(\log W)$ factor in the claim for the weighted case.

    %For proving the second part of the claim statement, the induction hypothesis is as follows: For any instance of $m$ hypercubes in $\mathbb{R}^{2d}$, the number of pieces over instances in $\mathcal{I}_C$ over cells $C\in T$ that a hypercube $h$ in the original instance is split into is $O_d(\log^{2d-1}m)$. Note that the base case holds for $d=2$ from Lemma~\Cref{lem:squaresetcover_1} since we can decompose any square $S\in\mathbb{R}^2$ into $O(\log m)$ cells for which it is \secc. Thus, using Claim~\Cref{claim:rectangledynsetcover_3} we can infer that any hypercube $h\in\mathbb{R}^{2d}$ in our original instance can be decomposed in at most $16^d\log m$ cells $\mathcal{C}_h=C_1,\dots,C_t$ such that $\bigcup_{i=1}^{i=t}C_i=h$ and $h$ is \secc for every cell in $\mathcal{C}_h$. Further, applying the projection step over all cells in $\mathcal{C}_h$ over all non-empty subsets $s\in[2d]$ along with the induction hypothesis implies the induction step since there are $O_d(\log m)$ cells in $\mathcal{C}_h$.

%\end{claim}

%\subsection{Proof of \Cref{lem:corresponding-solution}\label{subsec:corresponding-solution}}
\subsection{Proof of \Cref{lem:update-fast} \label{subsec:update-fast}}
Before proving \Cref{lem:update-fast}, we prove another lemma about fast access to the new set system $(P,\hat{\S})$.

\begin{lemma} \label{clm:rectangleDynSetCov_access}
    %Any given instance $(U_3,\mathcal{F}_3)$ of weighted set cover mapped according to our processing from the original instance $(U_2,\mathcal{F}_2)$ is such that $f=O_d(\log^{2d-1}m\cdot \log W)$, where $f$ denotes the maximum frequency of any element according to the set system $(U_3,\mathcal{F}_3)$. Also, a hypercube $h\in\mathcal{F}_2$ is split into $O_d(\log^{2d-1}m)$ square pieces over 2d instances in $\mathcal{I}_C$ according to our processing algorithm.
    \begin{enumerate}
        % \item The frequency $f$ of $(U_3,\mathcal{F}_3)$ is $O_d(\log^{2d-1}m)$;
        %\item $|{T_{\text{ex}}}_H|=O_d(\log^{2d-1}m)$ for any $H\in\mathcal{F}_2$;
        \item For any point $p\in P$, the set $V_p'$ can be computed in $O_d(\log^{2d}N)$ time. Further, the entire collection of sets in $\hat{\S}$ that contain $p$ can be found in  $O_d(\log^{2d}N)$ time;
        \item For any hypercube $S\in\S$, the set $V_S'$ can be computed in $O_d(\log^{2d-1}N)$ time.
        \item Given a point $p\in P$ and a set $(v,S)\in\hat{\S}$, we can find out whether $p\in (v,S)$  for some $v\in V'$ in $O_d(1)$ time. 
    \end{enumerate}
\end{lemma}
\begin{proof}
    %Proof deferred to \Cref{subsec:rectangleDynSetCov_access}
We will prove the first subclaim. Any point $p\in P$ is contained in the root cell $C_r$. It is also contained in $O(\log N)$ cells $\mathcal{C}'$ which are $2d$-dimensional in $T'$. One of these cells, say $C_w$ is a subdivision child of the root for which we just need to search its $2^{2d}$ children. Having found the node $w$, $p$ has to contained in one of its children again. By using this approach, we can find all of these $2d$-dimensional cells that contain $p$ in $O_d(\log N)$ time. Further, a projection of $p$ may only be found in nodes of dimension at most $2d-1$. Any node that obeys this property is necessarily a projection child of one of the cells in $\C'$. Hence, for any node $v\in \C'$, we can find its corresponding projections in each of the $2d$-projection children of time in $O_d(1)$ time. We can show by induction here that in fact, we can find the set $V_p'$ in $O_d(\log^{2d}N)$ time similar to the proof of \Cref{lem:few-sets}.

To actually find the sets in $\hat{\S}$ that cover $p$, we do the following: we scan the tree in a BFS-like manner exactly the same as the procedure mentioned above for identifying the nodes in  $V_p'$. For any such node $d'$-dimensional node $v$ in $V_p'$ (for $d'\leq 2d$), when we scan it, we consider all the maximal facet-covering hypercubes for it. For any such hypercube $S$ such that $S\cap C_v$ was maximal facet-covering for $v$, we just check if $S\cap C_v$ covers the projection of $p$ in $v$ in $O(1)$ time. Note that if $S\cap C_v$ is maximal facet-covering for $v$ and covers the facet $F$, we just store the interval defined by the projection of $S\cap C_v$ in the dimension orthogonal to $F$ (done in pre-processing). Note that since $F$ is a facet of $C_v$, there is only one such dimension orthogonal to $F$ w.r.t. the cell $C_v$. 

Now, we prove the second subclaim keeping in mind the proof of \Cref{lem:cheap-solution}. We show this run time using induction. The induction hypothesis is as follows: For a given $d'$-dimensional node $v$ such that $S\in\S_v$, the nodes in its subtree which are contained in $V_S'$
can be found in  $O_{d'}(\log^{d'-1}N)$ (we avoid the actual expression here for simplicity). The base case for $d'=1$ is trivial. Now, consider the general case of a
cell $C_v$ in $T'$ of dimension $d'>1$ such that $S\in \S_v$. 
Then, as per the definition of $V_S'$, if $S\cap C_v$ was edge-covering for $C_v$ but also for its parent, then we do not add any node from the subtree of $v$ to $V_S'$. Hence, we consider the set $\C_{v,S}$ which is of size at most $4^{d'}\cdot (\log N)$. We claim that $\C_{v,S}$ can  be found in $O_{d'}(\log N)$ time. To observe this, we reference the proof of \Cref{clm:secc_1}. Once, we have $\C_{v,S}$, 
 we add all those nodes $v_0$ from this set to $V_S'$ for which $S\cap v_0$ was facet-covering. For any other node $v'$, there are at most $d'$ projection children $v''$ for which $S\in\S_{v''}$ and which are all $(d'-1)$-dimensional nodes on which we can apply our induction. The task of finding $S\cap C_{v''}$ can be done in $O_d(1)$ time over all such children $v''$ of $v'$. The proof that the induction holds follows from \Cref{lem:cheap-solution}.
 %Then, for each such $v''$ which is $(d'-1)$-dimensional, we can use induction hypothesis to get that at most $4((d'-1)!)\cdot 4^{\frac{(d'-1)(d')}{2}}({\log^{d'-2}N})$ are included in $V_S'$ from this subtree. Since the number of such children $v''$ for each $v'$ is at most $d'$ and the number of such nodes $v'$ in the subtree of $v$ is at most $4^{d'}\cdot 2(\log N)$, we get that the number of nodes in $V_S'$ in the subtree of $v$ is at most


For the third subclaim, observe that for a set defined as $(v,S)$ in $(P,\hat{\S})$ we first find $S\cap C_v$ in $O_d(1)$ time and further find the projection of $p$ in $C_v$ in $O_d(1)$ time. The final check whether this projection of $p$ is contained in $C_v$ can be done in $O_d(1)$ time as well.
\end{proof}

% \paragraph*{Proof of \Cref{lem:update-fast}}
%\updatefast*
\begin{proof}
 From \Cref{lem:point-few-sets}, we established that the frequency $f$ of the set system $(P,\hat{\S})$ is $O_d(\log^{2d}N)$. Further, we make use of the dynamic set cover algorithm by Bhattacharya et al.~\cite{bhattacharya2021dynamic} and our query time to $(P,\hat{\S})$ is given by the subclaims in \Cref{clm:rectangleDynSetCov_access}. Subclaim $1$ in the same Lemma essentially guarantees that we can find out sets in $\hat{\S}$ that contain a point $p$ in $(P,\hat{\S})$ in $O_d(f)$ time. Also, we may query whether any point in the set system $(P,\hat{\S})$ lies is contained in a certain set in $O_d(1)$ time by subclaim $3$ in \Cref{clm:rectangleDynSetCov_access}. 
\end{proof}

%\al{ We need the following claim regarding fast access.}

\subsection{%Proofs of \Cref{lem:point-few-sets-weighted}, \Cref{lem:2approx} and \Cref{thm:WtDynSetCov}}
Weighted Case}
\label{subsec:dynwtsetcover_2}
% \paragraph*{Proof of \Cref{lem:2approx}\label{subsec:2approx}}



%\twoapprox*
\begin{proof}
We are assuming that all the weights for the sets lie in $[1,W]$ ($w$ is the given weight function for the set system). Then, we assign new weights to the sets in our system by defining a weight function $w'$. For every set $S\in\S$, we find an integer $k$ such that $2^{k}<w(S)\leq 2^{k+1}$ and then define $w'(S):=2^{k+1}$. Note that for any set $S\in\S$, $w'(S)/w(S)\leq 2$. Now, for any given instance of set cover $\mathcal{I}_1$ with weight function $w$ denote  an optimum set cover by $\OPT_1$ and the corresponding weight by $W_1$ and. Similarly, for the same instance of set cover but with weight function $w'$, denote the instance by $\mathcal{I}_2$, the weight of an optimum set cover by $W_2$ and an optimum set cover by $\OPT_2$. Then, we can claim that $w'(\OPT_2)\leq 2w(\OPT_1)$ (weight of a set cover is the sum of the weights of the sets in it). This is because $w'(\OPT_1)\leq 2w(\OPT_1)$.  
\end{proof}

Hence, our hyperrectangles have $O(\log W)$ different weights now and each hyperrectangle belongs to one of these weight classes.  We build the extended quad-tree
$T'$ like above. In the definition of the auxiliary instance $(P,\hat{\S})$, we slightly adjust the
sets that contain a point $p\in P$. For each node $v\in V'$ and for each weight class,   we consider a maximally facet-covering hypercube $S\in\S_v$ for $v$ belonging to this weight class and introduce a corresponding set in $\hat{\S}$; we denote it by $(v,S)$. We define that $(v,S)$ contains all points $p\in P$ such that there is a projection $p'$ of $p$ in $C_v$ with the property that $p'\in S\cap C_v$.   %We define $V_p'$ and $V_S'$ just as we defined in the previous subsection. 
%The definition of $V_p'$ and $V_S'$ (for a point $p\in P$ and a set $S\in\S$, respectively,) remains the same as before.  }
%Only for the uncovered leaf vertices in $V_p''$, instead of only including the sets representing the
%two maximal intervals in $\hat{\S}_p$, we iterate over all the $O(\log W)$ weight classes and for
%each weight class, separately consider the sets whose projections were two maximal intervals in each
%direction. Include them in $\hat{\S}_p$ if and only if they cover the projection of $p$ therein. For
%the covered nodes, we again assign the covering sets, one for each of the weight classes, if they
%exist.

%In the definition
%of the auxiliary instance $(P,\hat{\S})$, we slightly adjust the
%definition of maximally face-covering. We define for each each vertex
%$v\in V'$ and each hypercube $S\in\S_{v}$ that $S$ is \emph{maximally
%face-covering for }$v$ if there is a face $F$ of $C_{v}$ such that
%$F\subseteq S$ and $S\cap C_{v}$ is maximal among all intersections
%$S'\cap C_{v}$ for all $S'\in\S_{v}$ \emph{with $w_{S}=w_{S'}$
%}containing $F$. The reason for the adjustment is that maybe $w_{S}>w_{S'}$
%but $S'\cap C_{v}\subsetneq S\cap C_{v}$ and then $S$ does not dominate
%$S'$ (since $w_{S}>w_{S'}$) but on the other hand $S'$ does not
%dominate $S$ either (since $S'\cap C_{v}\subsetneq S\cap C_{v}$).
Note that if all hyperrectangles in $\S$ have the same weight, our adjusted definition coincides
with our definition from the unweighted case. Similarly as before, in case that there are two
hypercubes $S,S'\in\S_{v}$ with $S\cap C_{v}=S'\cap C_{v}$ and $w_{S}=w_{S'}$, we break ties in an
arbitrary fixed way.

%Also, as before, for each vertex $v\in V'$
%and each maximally face-covering hypercube $S\in\S_{v}$ we introduce
%a corresponding set $(v,S)$ in $\hat{\S}$ that contains all points
%$p\in P$ such that there is a projection $p'$ of $p$ such that
%$p'\in S\cap C_{v}$.

We can still bound the number of sets that each point $p\in P$ is contained in. In fact, the number of sets containing a point may blow up by a factor of $O(\log W)$ for all the weight classes. Hence, $f=O_d(\log^{2d}N)\log W$ now.

%\pointfewsetsweighted*
\begin{proof}
We have $O(\log W)$ weight classes now and correspondingly have $O(\log W)$ different set collections. Applying \Cref{lem:point-few-sets} to each of these set collections implies that the frequency is $O_d(\log^{2d}N)\log W$.  
\end{proof}
%\al{After an update, our data structure for $(P,\hat{\S})$ might update its solution.\alr{The following sentence has to be adapted according to an $O_d(\log N)$ overhead. We may update $O_d(\log^{2d} N)\log^2
%(n)$ sets but the update time may have an $O_d(\log N)$ overhead.} Since its update
%time is $O_d(\log^{2d+1} N)\log^2 (n)$ in the worst case, we know that at most $O_d(\log^{2d} N)\log^2
%(n)$ sets in the solution are changed. Using \Cref{lem:corresponding-solution} and
%\Cref{clm:rectangleDynSetCov_access} we can hence update our solution to $(P,\S)$ in time
%$O_d(\log^{2d} N)\log^2 (n)$.}

%\subsection{Proof of \Cref{thm:DynSetCov}\label{subsec:DynSetCov}}



% \paragraph*{Proof of \Cref{thm:WtDynSetCov}}
%\WtDynSetCov*
\begin{proof}
 We know from \cite{toth2017handbook} that the VC-dimension of $2d$-dimensional hypercubes is $O(d)$. Then, using \textit{Sauer-Shelah Lemma}~\cite{sauer1972density,shelah1972combinatorial}, we claim that $\log n=O_d(\log m)$. Using \Cref{lem:reduce-to-cubes} we have that $N=2m$ in all of our lemma statements. Finally, the proof follows from \Cref{thm:DynSetCov} and \Cref{lem:point-few-sets-weighted}.     
\end{proof}
%\section{Hitting Set}

\subsection{Pre-processing for dynamic set cover algorithm in $d$-dimensional hyperrectangles}
\label{subsec:preprocessandquerysc}
\paragraph*{Pre-processing for the sets}
In this subsection, we pre-process the given sets so as to store the maximal facet-covering hypercubes
for nodes in the extended quad-tree. Essentially, if a node $v\in V_S'$ for some $S\in\S$, we want to find the maximal facet-covering sets for it. That is, after pre-processing all the sets in $\S$, we want to maintain the defined sets $(v,S)$ for each $v\in V'$. Note that for each set $S\in \S$, we want to consider only those nodes $v$ such that $v\in V_S'$. We do not want to consider other nodes $v'\in V'$, $v'\notin V_S'$ w.r.t. $S$ even if $S\in \S_{v'}$ and $S\cap C_{v'}$ was facet-covering for $v'$. Hence, we will not even need to store that $S\in\S_{v'}$. This pre-processing is different from the auxiliary set cover instance definition described in \Cref{sec:set-cover-hyperrectangles}.  But, we will show that this pre-processing algorithm is enough for our purposes and can be completed in $O_d(m\log^{2d-1}N)$ time.  We do this as follows:

\begin{enumerate}
    \item In the given set of $d$-dimension hyperrectangles,  using the ideas in \Cref{subsec:hypercube-reduction}, we map them to hypercubes in $2d$-dimensions in $O_d(m\log m)$ time plus some additional time for reading the entire input.
    \item Then, we process the hypercubes in an arbitrary fixed order; for each hypercube $S\in\S$, we compute the nodes in $V_S'$ in $O_d(\log^{2d-1}N)$ time by \Cref{clm:rectangleDynSetCov_access} and for each node $v\in V_S'$, we maintain the maximal-facet covering hypercube for each facet $F$ of $C_v$. Then, for a facet $F$, we check whether $S\cap C_v$ is at least as good as the currently stored maximal-facet covering set $S'$ by checking whether $S\cap C_v\supseteq S'\cap C_v$. If that is the case, we store $S$ as the current best maximal facet-covering hypercube for facet $F$ of $C_v$ and define $(v,S)$ w.r.t. facet $F$ and node $v$. Also, update $(v,S')$ appropriately. Note that even with this extra overhead, this operation over all nodes in $V_S'$ can be performed in $O_d(\log^{2d-1}N)$ time. 
    \item Iterating over all the sets $S\in\S$, the above procedure can be executed in $O_d(m\log^{2d-1}N)$ time.
\end{enumerate}
\iffalse{
For each node  $v$
in the quad-tree, and for each facet $F$ of its corresponding cell $C_v$, define and initialize a variable
$M(v,F)$ to $\emptyset$.

For each hypercube in the instance, we do a BFS-like traversal of the extended quad-tree. Start with
the BFS queue storing just the root:
\begin{enumerate}
    \item\label{step:pop1} Remove the top element $v$ from the bfs queue. Let $S'$ be the projection of $S$ to the
        appropriate dimensions of $C_v$.
    \item Add to the queue all child nodes $v'$ of $v$, such that $C_{v'}$ intersects with an appropriate
        projection of $S$ (i.e., there is some projection of $S$ in $\mathcal{S}_{v'}$). Note that
        we can go over all children of $v$ in $O_d(1)$ time to do this.
    \item If $S'$ is facet-covering for a facet $F$ of $C_v$, then update $M(v,F)$ as follows: If
        $S'\cap C_v$ covers $M(v,F)$, set $M(v,F)$ to $S'\cap C_v$.
    \item Repeat from \Cref{step:pop1} till the queue is empty.
\end{enumerate}

We know that some projection of $S$ is included in $\S_v$ for at most $O_d(\log^{2d-1}N)$ nodes
$v$ in the tree (from \Cref{proofofvsprime}). Since the traversal described above only visits such nodes, it takes only
$O_d(\log^{2d-1}N)$ time.

We repeat this process for every set in $\mathcal{S}$, which takes a total of $O_d(m\log^{2d-1}N)$ time. This
ensures that the value stored for any $M(v,F)$ has iterated over all relevant sets, and hence, stores
the maximal facet-covering set.}
\fi

\paragraph*{On arrival of a new input point $p$}
We first map point $p$ to an equivalent point $p'\in\mathbb{R}^{2d}$ by \Cref{subsec:hypercube-reduction}.
We need to find the collection of sets in the new set
system $(P,\hat{\S})$ that contain the given point $p'$. Having done the pre-processing of the sets in $\S$, we achieve this as follows: 
\begin{enumerate}
    \item Given a point $p$ in the original hyperrectangle set system, we map it to a $2d$-dimensional point $p'$ w.r.t. the set system $(P,\S)$ by the ideas in \Cref{subsec:hypercube-reduction}. 
    \item In $(P,\S)$, we compute the sets in $\hat{\S}$ that it is contained in by \Cref{clm:rectangleDynSetCov_access} in $O_d(\log^{2d}N)$ time.
\end{enumerate}


\iffalse{We again achieve this by a BFS like traversal of the extended
quad-tree. We use it to identify the nodes containing $p$ or its projection of. Once we have these nodes,
we can refer to the previously described data structure to choose all sets which cover $p$ and are
maximal facet-covering for the corresponding cell (i.e., are included in the new instance $\hat{\mathcal{S}}$).

We now describe the method in some more detail. Initialize the BFS queue to the root node of the extended quad-tree:
\begin{enumerate}
    \item\label{step:pop2} Remove the top element $v$ from the BFS queue. Let $p'$ be the projection of $p$ to the
        appropriate dimensions of $C_v$.
    \item Add to the queue all child nodes $v'$ of $v$, such that $C_{v'}$ contains $p$ or a
        projection of $p$ (i.e., there is some projection of $p$ in $\mathcal{P}_{v'}$). Note that
        we can go over all children of $v$ in $O_d(1)$ time to do this.
    \item For each facet $F$ of $C_v$, if $M(v,F)$ contains a projection of  $p$, then pick the set
        $M(v,f)$.
    \item Repeat from \Cref{step:pop2} till the queue is empty.
\end{enumerate}

Again, we know from \Cref{lem:point-few-sets} that there are only $O_d(\log^{2d}N)$ nodes in the extended-quad-tree that
contain some projection of point $p$. And since the above traversal only visits such nodes,
we can compute the required collection of sets in $O_d(\log^{2d}N)$ time.
}
\fi

\section{Hitting Set for $d$-dimensional Hyperrectangles}
\label{sec:hypercube_hs}
We present now our dynamic algorithm for hitting set for $d$-dimensional hyperrectangles. We
assume that we are given a set of points $P$ and that hyperrectangles are inserted and deleted
dynamically. Like for set cover, we first reduce the problem to hypercubes, by increasing the
dimension by a factor of 2. Note that this reduction is different from our reduction for set cover
from \Cref{lem:reduce-to-cubes} since now the points $P$ are fixed, rather than the hyperrectangles.
\begin{restatable}{lemma}{reducetocubesHS} \label{lem:reduce-to-cubes-HS}
Let $d\in\mathbb{Z}_{+}$ and let's say all the given points have integral coordinates in $[0,N')^{d'}$.
If we are given a dynamic $\alpha$-approximation
algorithm for hitting set in $2d'$-dimensional hypercubes  with integral corner coordinates
having $f(d',m,n,N')$ update
time and $g(d',m,n,N')$ initialization time, then we can construct a dynamic $\alpha$-approximation
algorithm for hitting set in $d'$-dimensional hyperrectangles with update time $O_{d'}(f(d',m,n,n)+\log n+\log N')$ and initialization time $O_{d'}(g(d',m,n,n)+n(\log n+\log N'))$.
\end{restatable}
%
For the remainder of this section suppose that we are given a set
of $2d$-dimensional points $P$  having integral corner
coordinates, contained in $[0,N)^{2d}$ for some constant $d\in\Z_+$,
where $N$ is a power of 2 with $N=n^{{O_d}(1)}$. %For the remainder of this section suppose that we are given a set of $2d$-dimensional hypercubes
%$\S$ contained in $[0,N)^{2d}$ for some constant $d\in\N$. Assume w.l.o.g.~that $N$ is a power of
%2.
Also, suppose that any hypercube $S$ introduced by the adversary has integral coordinates for each of its corners in $[0,N)^{2d}$.
%(\ak{with integral coordinates})

\paragraph*{Mapping $d$-dimensional hyperrectangles to $2d$-dimensional hypercubes}(Proof of \Cref{lem:reduce-to-cubes-HS}:)
In the dynamic setting of hitting set, we are initially given only the set of points $P'\in\R^d$, and the
hyperrectangles are inserted and deleted as the algorithm progresses. We note here that any potential input hyperrectangle can be
appropriately shrunk such that its vertices coincide with the coordinates of some given point (and
the same new hyperrectangle covers the same set of points as before). So we essentially have only
$O(n^{2d})$ combinatorially relevant hyperrectangles.

We apply \Cref{claim:rectangledynsetcover_1} assuming that each of these $O(n^{2d})$ hyperrectangles is
present in the instance, to get a map from these hyperrectangles to a set of hypercubes with integer
coordinates in  $[0,2n]^{2d}$.  (this is done only to aid in mapping the given set of points. The
hyperrectangles themselves are considered only when they are added/removed at some time step).  Now
if we similarly (to the set cover case) map the given set $P'$ to a set of points $P$ (using
\Cref{lemma:orthantmap_1} and a coordinate shift), we ensure that on arrival/deletion of any
hyperrectangle, the set intersections are preserved in the new instance.

\begin{claim}
An $\alpha$-approximate hitting set algorithm on hypercubes in $\R^{2d}$
provides an $\alpha$-approximate hitting set algorithm on hyperrectangles in $\R^{d}$.
\end{claim}
\begin{claim} Note that since the above mapping preserves set intersections, a solution to the
 hitting set instance $\mathcal{I}_1$ on hyperrectangles, can be mapped to a hitting set (of the
 same size) of the newly constructed instance $\mathcal{I}_2$ on hypercubes. This means that the
 optimal solution $\OPT_1$ to $\mathcal{I}_1$, is a valid hitting set for $\mathcal{I}_2$, and
 similarly the optimal solution $\OPT_2$ to $\mathcal{I}_2$, is a valid hitting set for $\mathcal
 {I}_1$. So, we have $|\OPT_1|\ge|\OPT_2|$ and $|\OPT_2|\ge|\OPT_1|$, giving us that
 $|\OPT_1|=|\OPT_2|$. This means that an $\alpha$-approximate solution for hitting set on $\mathcal
 {I}_2$ will also correspond to an $\alpha$-approximate solution for hitting set on $\mathcal
 {I}_1$.
\end{claim}

Now let us look at the running time for these operations:
\Cref{lemma:orthantmap_1} sorts the list of end-points of projections of hyperrectangles to
each dimension to do the reduction. This is because in the case of dynamic hitting set, the hyperrectangles are not given explicitly. However, we know that any hyperrectangle that is introduced can be equivalently assumed to have its corners coinciding with points in $P$. This sorting takes $O_d(n\log n)$ time.
It takes $O(\log N')$ additional time to read each point in the input; hence, amounting to a total
of $O_d(n(\log n+\log N'))$ time in the pre-processing.

Any time a hyperrectangle is introduced we find can compute its corresponding hypercube in $O(d)$
time from the already available sorted lists of coordinates (from \Cref{lemma:orthantmap_1}). Taking into
account the time to read the input hyperrectangle, we get a total of $O_d(\log N')$ time during any update operation.
In addition to the above, we also need to account for a $O(\log N')$ overhead for reading the input
in both of the above operations.



 %For any point $p\in P$, the set
%$V_p''$ is defined in the exact same way. 
%But, we do not designate any covering sets for the
%respective covering nodes. 

%Now, we define the auxiliary hitting set instance. Rather, if a node $v\in V_S'$ was a covered node and there existed a
%non-empty set of points $P'\subseteq P$ which were assigned to it, we arbitrarily designate one of
%them as a \textit{hitting point} for $v$.

% We organize the points $P$ in a range-counting data structure {[}?{]} that allows us to query in
% time $???$ .... <\textcompwordmark < add the needed query here, in particular, the one needed for
% \Cref{lem:construct-hat-S-fast}>\textcompwordmark > ... .

\paragraph*{Auxiliary hitting set instance}
Suppose now that we want to solve hitting set dynamically for $2d$-dimensional hypercubes for a given
set of points $P\subseteq\R^{2d}$. Also, suppose that $\S$ is the current set of hypercubes. We build
the same extended quad-tree $T'=(V',E')$ as in \Cref{sec:set-cover-hyperrectangles} and for any node $v\in V'$,
define the sets $\S_v$ and $P_v$  in exactly the same way.  Hence, the definition of $V_p'$ for a point $p\in P$ remains unchanged and \Cref{lem:few-sets} still holds. Also, for any
set $S\in\S$, the set $V_S'$ is defined in the exact same way (see \Cref{subsec:rectangleDynSetCov_covering} for the definition of $V_S'$).


Our goal is to construct an auxiliary instance $(\hat{P},\hat{\S})$ of general hitting set with a
family of sets $\hat{\S}$ and a set of points $\hat{P}$ such that each set $\hat{S}\in\hat{\S}$
contains at most $(\log N)^{O(d)}$ points from $\hat{P}$.
%The point set $\hat{P}$ is such that for each point $p\in P$, there is a corresponding point $\hat{p}\in\hat{P}$.
The set memberships for this set system are described next.

For each each node $v\in V'$, consider all the points $p$ in $P$ such that $p\in P_v$. Then, among the points in $P_v$ define the maximal facet point $p$ for a facet $F$ of $C_v$ if the projection of $p$ in $C_v$ is the closest to $F$ among the projections of the points in $P_v$. Closest here is defined w.r.t. the distance of the projection of $p$ in $C_v$ from $F$. For such a maximal facet point $p$ w.r.t. facet $F$ of $C_v$, define an element $(p,v)\in\hat{P}$ (which just represents the projection of $p$ in $C_v$).

To define the sets in $\hat{\S}$, for every set $S\in \S$ in the collection of sets $\hat{\S}$ contains an empty set $\hat{S}$ to being with. Based on the old set system
$(P,\S)$, we will include a subset of $\hat{P}$ to define the points that $\hat{S}$ covers according to
$(\hat{P},\hat{\S})$. We now define for a set $\hat{S}\in\hat{\S}$ the elements in $\hat{P}$ that it contains.

     %each hypercube $S\in\S_{v}$ we
%say that $S$ is \emph{maximally facet-covering for }$v$ if there
%is a facet $F$ of $C_{v}$ such that $F\subseteq S$ and $S\cap C_{v}$
%is maximal among all intersections $S'\cap C_{v}$ for all $S'\in\S_{v}$
%containing $F$. In case that there are two hypercubes $S,S'\in\S_{v}$
%with $S\cap C_{v}=S'\cap C_{v}$, we break ties in an arbitrary fixed
%way.

% and each maximally facet-covering hypercube
%$S\in\S_{v}$ for $v$ we introduce a corresponding set in $\hat{\S}$; we
%denote this set by $(v,S)$. We define that $(v,S)$ contains all
%points $p\in P$ such that there is a projection $p'$ of $p$ in $C_v$ with the property that $p'\in S\cap C_{v}$.

For each set $\hat{S}\in \hat{\S}$, consider the set $V_S'$ for the corresponding set $S\in\S$. For each $v\in V_S'$, we consider all the elements $(v,p)$ for some $p\in P$ and include $(v,p)$ in $\hat{S}$ if $S\cap C_v$ covers the projection of $p$ in $C_v$. This completes the definition of $(\hat{P},\hat{\S})$. Then, observe that for any set $\hat{S}\in\hat{\S}$, $|\hat{S}\leq 4d(|V_S'|)|$ since the total number of defined elements $(v,p)$ for $v\in V'$ is at most $4d$ corresponding to the number of facets of $C_v$. Then, we have the following lemma. % $O_d(\log^{2d-1}N)$ by \Cref{} 
 

%each node $v$ in $V_S'$. If $v$ was a covered node by the
%projection of $S$ in it, then we include the hitting point for $v$ (if it exists) in  $\hat{S}$.
%Else if $v$ was not a covered leaf by the projection of $S$ in it, then we consider that point in
%$P$ whose projection in $v$ was the leftmost (if it exists). Similarly, find such a point in $P$
%whose projection in $p$ was the rightmost (if it exists). Include them (if they exist) in $\hat{S}$
%based on whether their projections in $v$ hit the projection of $S$ in $v$. 
%For each node $v\in V'$ and each face $F$
%of $C_{v}$, we identify the point $p\in C_{v}$ that is closest to
%$F$ (breaking ties in an arbitrary fixed way). We add a corresponding
%point $(v,p)$ to $\hat{P}$. Intuitively, we pick $p$ since it is
%the most useful point in $P_{v}$ if we want to hit a set $S$ that
%contains $F$, since among all points in $P_{v}$ the point $p$ hits
%the maximum number of such sets $S$.
%Next, we define the sets $\hat{\S}$. Let $S\in\S$. We add a set
%$\hat{S}$ to $\hat{\S}$ which \emph{corresponds} to $S$. To construct
%$\hat{S}$, we do the following procedure for each $v\in V'$ such
%that $S$ contains a face $F$ of $C_{v}$ but $S$ does not contain
%a face $F'$ of a cell $C_{v'}$ of an ancestor $v'$ of $v$ in $T'$.
%We identify the point $p\in C_{v}$ that is closest to $F$ (breaking
%ties in the same way as above). If $p\in S$ then we add $(v,p)$
%to $\hat{S}$. We can show that for $S$ there are at most $(\log N)^{O(d)}$
%such vertices $v\in V'$ which proves the following lemma.
\begin{theorem} \label{lem:frequency-HS}
    For all $\hat{S}\in\hat{\S}$. We have that $|\hat{S}|=O_d(\log^{2d-1} N)$.
\end{theorem}
\begin{proof}
    We showed in \Cref{proofofvsprime} that $|V_S'|=O_d(\log^{2d-1}N)$. Hence, $|\hat{S}|=O_d(\log^{2d-1}N)$.\end{proof}

\begin{lemma}
    \label{lem:rectangledynhittingset_eq}
    %Any point $p\in U_2$ is covered by a hypercube $h\in \mathcal{F}_2$ if and only if $h$ appears as one of the square pieces in some 2d  instance $I$ in $\mathcal{I}_C$, where this square piece covered the equivalent projection of $p$ in the 2d instance ($C$ is the cell in $\mathcal{F}_2$ which contained $p$ and for which $h$ was \secc in the original instance).
    A hypercube $S\in\mathcal{S}$ is hit by a point $p\in P$ if and only if there exists a node $v$ in
    $V_S'$ such that $p\in P_v$ and $S\cap C_v$ is hit by the projection of
    $p$ in $C_v$.
\end{lemma}
\begin{proof}
    The proof follows directly from \Cref{lem:rectangleDynSetCov_covering}.
\end{proof}

Now, we prove that if there exists an optimum hitting set $\OPT$ for $(P,S)$, then there exists a solution to $(\hat{P},\hat{\S})$ that contains at most $\OPT\cdot O_d(\log^{2d} N)$ points.

We know that for any point $p\in P$, $|V_p'|=O_d(\log^{2d}N)$ from \Cref{lem:few-sets}. Then, for $p\in\OPT$, we will define a subcollection $P'\in\hat{P}$ of size $O_d(\log^{2d}N)$. This subcollection has the property that if $p\in P$  was hitting a set $S\in\S$, then there exists one point $p'\in P'$ with the following property: There exists one node $v\in V'$ for which $(v,p')$ was defined and $(v,p')\in \hat{S}$. 
First, for any node $v\in
V_p'$  we map at most $4d$ points in $\hat{P}$ to it. Denote this subcollection  of size at most $4d$ by $\hat{P}'_v$. %$\S'_v$ is such that for any $p\in P$ such that the projection of $p$ was covered by the projection of $S$ in the cell for
%$v$, one of the sets in it 
%necessarily covers $p$ in $(P,\hat{\S})$.  
For any node $v\in V_p'$, for $p'\in P$, include $(v,p')$ in $\hat{P}'$ if $(v,p')$ was defined according to our auxiliary hitting set instance definition. Now, consider a set $S\in\S$ which covered $p$ in $(P,\S)$. Then, from \Cref{lem:rectangledynhittingset_eq} we know that there exists some node $v$ such that $v\in V_S'$ and $S\cap C_v$ covered the projection of $p$ in $C_v$ and $S\cap C_v$ was facet-covering for $v$. Then, there exists a maximal facet point $p'\in P$ w.r.t. $C_v$ such that the projection of $p'$ hit $S\cap C_v$. Then, $(v,p')\in \hat{P}'_v$ and $(v,p')$ hit $\hat{S}$ in $(\hat{P},\hat{\S})$.  %      Since the projection of $S$ in $C_v$ was facet-covering for it, there existed a maximal facet-covering hypercube $S_0\in\S$ such that $S_0\cap C_v\supseteq S\cap C_v$. 
%Then, $(v,S_0)$ was included in $\S'$. 
Further, applying~\Cref{lem:rectangledynhittingset_eq}, %we get that for any $p\in P$ such that the projection of $p$ was covered by the projection of $S$ in the cell for
%$v$, $S_0$ 
%necessarily covered $p$ in $(P,{\S})$. 
we get that $p'$ hit $S$ in $(P,\S)$. Define another set $\hat{P}'=\bigcup_{v\in V_p'}\hat{P}'_v$. Then, we have that $|\hat{P}'|=O_d(\log^{2d}N)$. Observe that if a hypercube $S$ covered a point $p\in P$ in the original set system, then by \Cref{lem:rectangledynhittingset_eq} there exists a node  $v'\in V_S'$ such that the projection of $p$ in $C_{v'}$ was covered by $S\cap C_{v'}$. By definition, $S\cap C_{v'}$ was facet-covering for $v'$. Then, we apply the same argumentation as before w.r.t. node $v'$ for which there exists a set $\hat{P}'_v\subseteq \hat{P}'$. This implies that there existed an element $(v,p')\in \hat{P}'$ such that $(v,p')$ was covered by $\hat{S}$ and also, $p'$ was covered by $S$ as per the set system $(P,\S)$. %Note that $v'\in  V_S'$ and hence, applying the same argument as before, there exists a set $S_0$ such that $(v',S_0)\in \S'$ and $(v',S_0)$ covered $p$ in $(P,\hat{\S})$. Also, $S_0$ covered $p$ in $(P,\S)$.

\iffalse{
Since, $(v,S_0)$ is defined, we have that $p$ was covered by a set in $\S'_v$. 
Define another subcollection $\S'=\bigcup_{v\in V_S'}\S'_v$. Then, we have that $|\S'|=O_d(\log^{2d-1}N)$. Observe that if a hypercube $S$ covered a point $p\in P$ in the original set system, then by \Cref{lem:rectangledynhittingset_eq} there exists a node  $v'\in V_S'$ such that the projection of $p$ in $C_{v'}$ was covered by the projection of $S$ in it. By definition, $S\cap C_{v'}$ was facet-covering for $v'$. Note that $v'\in  V_S'$ and hence, applying the same argument as before, there exists a set $S_0$ such that $(v',S_0)\in \S'$ and $(v',S_0)$ covered $p$ in $(P,\hat{\S})$.


%The second property intuitively says that covering a projection of point in some appropriate cell in $T'$ by projection of a hypercube is equivalent to covering the point in the actual instance $(P,\S)$.
We know that for any set $S\in\S$, $|V_S'|=O_d(\log^{2d-1}N)$ from \Cref{proofofvsprime}. Then, for $S\in\OPT$, we will define a subcollection $\S'\in\hat{\S}$ of size $O_d(\log^{2d-1}N)$. This subcollection has the property such that if $p\in P$  was covered by $S$, then $p$ was covered by at least one set in $\S'$. 
First, for any node $v\in
V_S'$  we map at most $4d$ sets in $\hat{\S}$ to it. Denote this subcollection  of size at most $4d$ by $\S'_v$. %$\S'_v$ is such that for any $p\in P$ such that the projection of $p$ was covered by the projection of $S$ in the cell for
%$v$, one of the sets in it 
%necessarily covers $p$ in $(P,\hat{\S})$.  
For any node $v\in V_S'$, for $S'\in\S$, include $(v,S')$ in $\S'$ if $(v,S')$ was defined according to our auxiliary set cover instance definition. Since $S\cap C_v$ was facet-covering for $v$, there existed a maximal facet-covering hypercube $S_0\in\S$ such that $S_0\cap C_v\supseteq S\cap C_v$. 
Then, $(v,S_0)$ was included in $\S_v'$. Further, applying~\Cref{lem:rectangleDynSetCov_covering}, we get that for any $p\in P$ such that the projection of $p$ was covered by $S\cap C_v$, $S_0$ 
necessarily covered $p$ in $(P,{\S})$. Since, $(v,S_0)$ is defined, we have that $p$ was covered by a set in $\S'_v$ in $(P,\hat{\S})$. 

Define another subcollection $\S'=\bigcup_{v\in V_S'}\S'_v$. Then, we have that $|\S'|=O_d(\log^{2d-1}N)$. Observe that if a hypercube $S$ covered a point $p\in P$ in the original set system, then by \Cref{lem:rectangleDynSetCov_covering1} there exists a node  $v'\in V_S'$ such that the projection of $p$ in $C_{v'}$ was covered by $S\cap C_{v'}$. By definition, $S\cap C_{v'}$ was facet-covering for $v'$. Note that $v'\in  V_S'$ and hence, applying the same argument as before, there exists a set $S_0$ such that $(v',S_0)\in \S'$ and $(v',S_0)$ covered $p$ in $(P,\hat{\S})$. Also, $S_0$ covered $p$ in $(P,\S)$.}
\fi

%This hitting point hits the projection of $S$ for $v$ if $S$ was assigned to $v$ and its projection covered it entirely (and hence, $v\in V_S'$).
%In this case as well, if $v\in V_S'$, the projection of  of this mentioned set inside the cell. Denote this set by $S'$ in either case and note that $S'$ covers $p$. Then, by \Cref{lem:rectangleDynSetCov_covering}, we are able to cover $S$ entirely in the original instance of set cover $(P,\S)$ by a subcollection of $O_d(\log^{2d-1}N)$ sets. Applying this argument over all sets in $\OPT$ yields the following lemma.



%To show this, we consider each point $p\in\OPT$ and
%each of the at most $(\log N)^{O(d)}$ cells $C_{v}$ for a node
%$v\in V'$ such that a projection of $p$ is contained in $P_{v}$.
%For each face $F$ of $C_{v}$ we select the point $(v,p')$ such
%that $p'$ is the point in $P_{v}$ closest to $F$. This yields $\OPT\cdot(\log N)^{O(d)}$
%points in total.
\begin{lemma}
\label{lem:cheap-solution_hs}
  The instance $(\hat{P},\hat{\S})$ has a solution with at most $\OPT\cdot(\log^{2d} N)$
points which respectively yields a solution of at most that size for $(P,\S)$.
\end{lemma}
\begin{proof}
 Proof similar to that of \Cref{lem:corresponding-solution}.
\end{proof}
On the other hand, we can translate each solution to $(\hat{P},\hat{\S})$
to a solution to $(P,\S)$ with the same cardinality.
\begin{lemma}
    \label{lem:corresponding-solution-HS}For any solution $\hat{\A}\subseteq\hat{P}$
    to $(\hat{P},\hat{\S})$ there is a solution $\A$ to $(P,\S)$ with
    $|\A|\le|\hat{\A}|$. For each $(v,p)\in\hat{\A}$ there is a corresponding
    point $\tilde{p}\in P$ and given $(v,p)$, we can identify $\tilde{p}$
    in time $O_d(\log N)$.
\end{lemma}

We claim that we can maintain the instance $(\hat{P},\hat{\S})$ dynamically when sets are inserted
in $\S$ or removed from $\S$. If a set $S\in\S$ is removed then we simply remove the corresponding
set $\hat{S}\in\hat{S}$. If a set $S$ is inserted to $\S$ then we can construct the corresponding
set $\hat{S}$ quickly via the following lemma. For this, we compute the nodes $v\in V_S'$ for
which we might add a point $(v,p)$ to $\hat{S}$. For each such $v$ and each facet $F$ of $C_{v}$ we
find the closest point $p\in P\cap C_{v}$ quickly using our range-counting data structure for $P$.
\begin{lemma}
    \label{lem:construct-hat-S-fast}Let $S\subseteq[0,N)^{d}$ be a hypercube.
    In time $O_d(\log^{2d-1} N)$ we can construct a set $\hat{S}\subseteq\hat{P}$
    that corresponds to $S$. Also, given $(v,p)\in \hat{P}$ and $\hat{S}\in \hat{\S}$, we can compute whether $(v,p)\in \hat{S}$ in $O_d(1)$ time.
\end{lemma}
\begin{proof}
We use \Cref{lem:update-fast} to compute the nodes in $V_S'$ in $O_d(\log^{2d-1}N)$ time. Further, we consider the maximal facet points for $C_v$ and find their respective projections in $C_v$ in $O_d(1)$ time. Finally, we can check if their projections are contained in $S\cap C_v$ in $O_d(1)$ time. This implies the first part of the lemma statement.

For the second part of the lemma statement, we first check whether $(v,p)$ is defined in $O_d(1)$ time since we explicitly store it in our pre-processing step. Then, we just compute the projection of $p$ in $C_v$ and check whether it is contained in $S\cap C_v$ in $O_d(1)$ time. 
\end{proof}
We maintain an approximate solution $\hat{\A}\subseteq\hat{P}$ to
$(\hat{P},\hat{\S})$ dynamically using the data structure introduced by \cite{bhattacharya2021dynamic}
for arbitrary instances of set cover. It guarantees an approximation
ratio of $O(f)$ and an update time of $O(f\log^2(Wm))$, since if we translate
our instance of hitting set to set cover, due to \Cref{lem:frequency-HS}
each point is contained in at most $(\log^{2d-1} N)$ sets.

With similar ideas as in \Cref{subsec:set-cover-weighted} we can extend the above algorithm to the
weighted case, by increasing the update time by a factor of $\log W$, assuming that for each point
$p\in P$ its weight $w_{p}$ is in the interval $[1,W]$ for a fixed value $W$.

For this, we round the
weights of the points so that we get $O(\log W)$ weight classes.
\begin{lemma} \label{lem:2approx_hs}
    By losing a factor of 2 in the approximation ratio, we can assume
    that for each $p\in P$ the weight $w_{p}$ is a power of 2.
\end{lemma}
\begin{proof}
 Proof follows similarly as the proof of \Cref{lem:2approx}.
\end{proof}

\iffalse{
\al{Hence, our hyperrectangles have $O(\log W)$ different weights now and each hyperrectangle belongs to one of these weight classes.  We build the extended quad-tree
$T'$ like above. In the definition of the auxiliary instance $(P,\hat{\S})$, we slightly adjust the
sets that contain a point $p\in P$. For each node $v\in V'$ and for each weight class,   we consider a maximally facet-covering hypercube $S\in\S_v$ for $v$ belonging to this weight class and introduce a corresponding set in $\hat{\S}$; we denote it by $(v,S)$. We define that $(v,S)$ contains all points $p\in P$ such that there is a projection $p'$ of $p$ in $C_v$ with the property that $p'\in S\cap C_v$.}   %We define $V_p'$ and $V_S'$ just as we defined in the previous subsection. 
%The definition of $V_p'$ and $V_S'$ (for a point $p\in P$ and a set $S\in\S$, respectively,) remains the same as before.  }
%Only for the uncovered leaf vertices in $V_p''$, instead of only including the sets representing the
%two maximal intervals in $\hat{\S}_p$, we iterate over all the $O(\log W)$ weight classes and for
%each weight class, separately consider the sets whose projections were two maximal intervals in each
%direction. Include them in $\hat{\S}_p$ if and only if they cover the projection of $p$ therein. For
%the covered nodes, we again assign the covering sets, one for each of the weight classes, if they
%exist.

%In the definition
%of the auxiliary instance $(P,\hat{\S})$, we slightly adjust the
%definition of maximally face-covering. We define for each each vertex
%$v\in V'$ and each hypercube $S\in\S_{v}$ that $S$ is \emph{maximally
%face-covering for }$v$ if there is a face $F$ of $C_{v}$ such that
%$F\subseteq S$ and $S\cap C_{v}$ is maximal among all intersections
%$S'\cap C_{v}$ for all $S'\in\S_{v}$ \emph{with $w_{S}=w_{S'}$
%}containing $F$. The reason for the adjustment is that maybe $w_{S}>w_{S'}$
%but $S'\cap C_{v}\subsetneq S\cap C_{v}$ and then $S$ does not dominate
%$S'$ (since $w_{S}>w_{S'}$) but on the other hand $S'$ does not
%dominate $S$ either (since $S'\cap C_{v}\subsetneq S\cap C_{v}$).
Note that if all hyperrectangles in $\S$ have the same weight, our adjusted definition coincides
with our definition from the unweighted case. Similarly as before, in case that there are two
hypercubes $S,S'\in\S_{v}$ with $S\cap C_{v}=S'\cap C_{v}$ and $w_{S}=w_{S'}$, we break ties in an
arbitrary fixed way.

%Also, as before, for each vertex $v\in V'$
%and each maximally face-covering hypercube $S\in\S_{v}$ we introduce
%a corresponding set $(v,S)$ in $\hat{\S}$ that contains all points
%$p\in P$ such that there is a projection $p'$ of $p$ such that
%$p'\in S\cap C_{v}$.

\al{We can still bound the number of sets that each point $p\in P$ is contained in. In fact, the number of sets containing a point may blow up by a factor of $O(\log W)$ for all the weight classes. }
}
\fi
\begin{lemma} \label{lem:sets-few-points-weighted}
Each set  $\hat{S}\in \hat{\S}$ is hit by at most $O_d(\log^{2d} N)\log W$ elements in $\hat{P}$.
\end{lemma}
\begin{proof}
 Proof follows similarly as the proof of \Cref{lem:point-few-sets-weighted}.
\end{proof}
We define the auxiliary set system in this case by taking into account each weight class separately for the points in $P$. Hence, the frequency may blow up by a factor of $O(\log W)$ (which in this case the number of elements in $(\hat{P},\hat{\S})$) that are contained in some set $\hat{S}\in\hat{\S}$.
Therefore, we have that $f=O_d(\log^{2d} N)\log W$. Similarly as in \Cref{lem:cheap-solution_hs} we can show
that there exists a solution with weight at most $\OPT\cdot O_d(\log^{2d} N)$.  Also, like in
\Cref{lem:update-fast}, we can update our solution in time $O(f\log^2 (Wm))=O_d(\log^{2d} N)\log^2(W
m)\log W$.
%\Cref{thm:WtDynHitSet}
%\begin{theorem} \label{thm:WtDynSetCov}
   % There is an $O_d(\log^{4d-1}m)\log W$-approximate dynamic algorithm for weighted geometric set cover for $d$-dimensional
  %  hyperrectangles with worst-case update time of $O_d(\log^{2d} m)\log^3(W m)$ when a point is
  %  added or deleted.%\alr{Should we not add approximation ratio here?}
%\end{theorem}
%\begin{proof}
 %   Proof deferred to \Cref{subsec:update-fast}.
%\end{proof}
% \paragraph*{Proof of \Cref{thm:WtDynHitSet}}
%\WtDynHitSet*
\begin{proof}
 We know from \cite{toth2017handbook} that the VC-dimension of the dual system of $2d$-dimensional hypercubes is $O(d)$. Then, using \textit{Sauer-Shelah Lemma}~\cite{sauer1972density,shelah1972combinatorial}, we claim that $\log m=O_d(\log n)$. Using \Cref{lem:reduce-to-cubes-HS} we have that $N=2n$ in all of our lemma statements. Also $f=O_d(\log^{2d} N)\log W$. Finally, the proof follows from using the algorithm in ~\cite{bhattacharya2021dynamic} along with \Cref{lem:construct-hat-S-fast} and \Cref{lem:corresponding-solution-HS}.     
\end{proof}


\subsection{Pre-processing and query for dynamic hitting set algorithm in $d$-dimensional hyperrectangles}





\paragraph*{Pre-processing for the points}

In this subsection, we pre-process the given points so as to store the maximal facet points
for nodes in the extended quad-tree. First, we map all the points in $\mathbb{R}^d$ in the hyperrectangle set system to an equivalent $2d$-dimensional set system $(P,\S)$. %Essentially, if a node $v\in V_p'$ for some $S\in\S$, we want to find the maximal facet-covering sets for it. 
After pre-processing, we want to maintain the defined elements $(v,p)$ for each $v\in V'$ (for some $p\in P$) as per the auxiliary hitting set instance. We will show that our pre-processing algorithm runs in $O_d(n\log^{2d}N)$ time.  We do this as follows:
\begin{enumerate}
    \item In the given set of $d$-dimension points,  using the ideas in \Cref{sec:set-cover-hyperrectangles}, we map them to equivalent points in $2d$-dimensions (the set $P$) in $O_d(n\log n)$ time plus some additional time for reading the entire input.
    \item Then, we process the points in $P$ in an arbitrary fixed order; for each point $p\in P$, we compute the nodes in $V_p'$ in $O_d(\log^{2d}N)$ time by \Cref{clm:rectangleDynSetCov_access} and for each node $v\in V_p'$, we maintain the maximal-facet points for each facet $F$ of $C_v$. Then, for a facet $F$, we check whether the projection of $p$ in $C_v$ is at least as close to $F$ as the projection of the currently stored maximal-facet point $p'$ by distance computation from the facet $F$. If that is the case, we store $p$ as the current best maximal facet point facet $F$ of $C_v$ and define $(v,p)$ w.r.t. facet $F$ and node $v$. Also, update $(v,p')$ appropriately. Note that even with this extra overhead, this operation over all nodes in $V_p'$ can be performed in $O_d(\log^{2d}N)$ time. 
    \item Iterating over all the points $p\in P$, the above procedure can be executed in $O_d(n\log^{2d}N)$ time.
\end{enumerate}
\iffalse{
For each node  $v$
in the quad-tree, and for each facet $F$ of its corresponding cell $C_v$, define and initialize a variable
$M(v,F)$ to $\emptyset$.

For each hypercube in the instance, we do a BFS-like traversal of the extended quad-tree. Start with
the bfs queue storing just the root:
\begin{enumerate}
    \item\label{step:pop1} Remove the top element $v$ from the bfs queue. Let $S'$ be the projection of $S$ to the
        appropriate dimensions of $C_v$.
    \item Add to the queue all child nodes $v'$ of $v$, such that $C_{v'}$ intersects with an appropriate
        projection of $S$ (i.e., there is some projection of $S$ in $\mathcal{S}_{v'}$). Note that
        we can go over all children of $v$ in $O_d(1)$ time to do this.
    \item If $S'$ is facet-covering for a facet $F$ of $C_v$, then update $M(v,F)$ as follows: If
        $S'\cap C_v$ covers $M(v,F)$, set $M(v,F)$ to $S'\cap C_v$.
    \item Repeat from \Cref{step:pop1} till the queue is empty.
\end{enumerate}

We know that some projection of $S$ is included in $\S_v$ for at most $O_d(\log^{2d-1}N)$ nodes
$v$ in the tree (from \Cref{proofofvsprime}). Since the traversal described above only visits such nodes, it takes only
$O_d(\log^{2d-1}N)$ time.

We repeat this process for every set in $\mathcal{S}$, which takes a total of $O_d(m\log^{2d-1}N)$ time. This
ensures that the value stored for any $M(v,F)$ has iterated over all relevant sets, and hence, stores
the maximal facet-covering set.}
\fi

\paragraph*{On arrival of a new hyperrectangle}
First, when a new hyperrectangle arrives, we map it to an equivalent $2d$-dimensional hypercube by using the mapping in \Cref{sec:hypercube_hs}.
Hence, we assume that an equivalent new hypercube $S$ arrives; we need to find the collection of elements in the new set
system ($\hat{P}$,$\hat{\S}$) that are covered by $\hat{S}$ (the corresponding set in $\hat{\S}$). Having done the pre-processing of the points in $P$, we achieve this as follows: 
\begin{enumerate}
    \item Given a set in the original hyperrectangle set system, we map it to a $2d$-dimensional hypercube $S$ w.r.t. the set system $(P,\S)$ by the ideas in \Cref{sec:hypercube_hs}. 
    \item In $(P,\S)$, we compute the set of nodes $V_S'$ in $O_d(\log^{2d-1}N)$ time. Further for each node $v\in V_S'$, we access all of the maximal facet points and check if their respective projections in $C_v$ are contained in $S\cap C_v$. That is, there are at most $4d$ maximal facet points, which are elements in $(\hat{P},\hat{\S})$ of the form $(v,p)$. Then, we just check for each such defined $(v,p)$ if the projection of $p$ in $C_v$ is contained in $S\cap C_v$. We can achieve this in $O_d(\log^{2d-1}N)$ time by \Cref{clm:rectangleDynSetCov_access}.
\end{enumerate}






\iffalse{In this subsection, we pre-process the given set of
points $P$ to compute for each cell, the points that are closest to each of its facets.
For each node $v$ in the quad-tree, and for each facet $f$ of its corresponding cell $C_v$, define
and initializing a variable $N(v,f)$ to $\emptyset$ (Here we assume for consistency that the orthogonal
distance of $\emptyset$ to any facet, of any cell, is infinity).

For each point in the instance, we do a bfs-like traversal of the extended quad-tree. Start with
the bfs queue storing just the root:
\begin{enumerate}
    \item\label{step:pop3} Remove the top element $v$ from the bfs queue. Let $p'$ be the projection
     of $p$ to the appropriate dimensions of $C_v$.
    \item Add to the queue all child nodes $v'$ of $v$, such that $C_{v'}$ contains $p$ or its appropriate
        projection (i.e., there is some projection of $p$ in $\mathcal{P}_{v'}$). Note that
        we can go over all children of $v$ in $O_d(1)$ time to do this.
    \item For each facet $f$ of $C_v$, if $p'$ is closer to the facet than $N(v,f)$, then update
        $N(v,f)$ to $p'$.
    \item Repeat from \Cref{step:pop3} till the queue is empty.
\end{enumerate}

We know from \Cref{lem:point-few-sets} that there are only $O_d(\log^{2d}N)$ nodes in the extended-quad-tree that
contain some projection of point $p$. And since the above traversal only visits such nodes,
we can compute the required collection of sets in $O_d(\log^{2d}N)$ time.

We repeat this process for every point in $P$, which takes a total of $O_d(m\log^{2d}N)$ time.
This ensures that the value stored for any $N(v,f)$ has iterated over all relevant
points in the cell, and hence, stores the closest such point.

\paragraph*{On arrival of a new input set $S$} We need to find the collection of points in the new
 instance that hit this set.
 \asr{this might change depending on the aux set instance.}

 We again achieve this by a bfs like traversal of the extended
 quad-tree. We use it to identify the nodes intersecting $S$ or containing a projection of $S$. Once we have these
 nodes, we can refer to the previously described data structure to choose a subset of points in
 the cell that hit the projection of $S$.
 
We now describe the method in some more detail. Initialize the bfs queue to the root node of the extended quad-tree:
\begin{enumerate}
    \item\label{step:pop4} Remove the top element $v$ from the bfs queue. Let $S'$ be the projection
     of $S$ to the appropriate dimensions of $C_v$.
    \item Add to the queue all child nodes $v'$ of $v$, such that $C_{v'}$ intersects with an
     appropriate projection of $S$ (i.e., there is some projection of $S$ in $\mathcal{S}_{v'}$). Note that we can go over all children of $v$ in $O_d(1)$ time to do this.
    \item For each facet $f$ of $C_v$, pick $N(v,f)$ if it intersects with $S'$.
    \item Repeat from \Cref{step:pop4} till the queue is empty.
\end{enumerate}

We know that some projection of $S$ is included in $S_v$ for at most $O_d(\log^{2d-1}N)$ nodes
$v$ in the tree (from \Cref{proofofvsprime}). Since the traversal described above only visits such nodes, it takes only
$O_d(\log^{2d-1}N)$ time.}
\fi


\iffalse{
\section{pre-processing and Query}
\label{sec:pre-processing and Query}
%\Cref{pre-processing and Query}

\subsection{Set Cover in $d$-dimensional hyperrectangles}
\paragraph*{Pre-processing for the sets}
In this subsection, we pre-process the given sets so as to find the maximal facet-covering hypercubes
for each node in the extended quad-tree. For each node  $v$
in the quad-tree, and for each facet $f$ of its corresponding cell $C_v$, define and initializing a variable
$M(v,f)$ to $\emptyset$.

For each hypercube in the instance, we do a bfs-like traversal of the extended quad-tree. Start with
the bfs queue storing just the root:
\begin{enumerate}
    \item\label{step:pop1} Remove the top element $v$ from the bfs queue. Let $S'$ be the projection of $S$ to the
        appropriate dimensions of $C_v$.
    \item Add to the queue all child nodes $v'$ of $v$, such that $C_{v'}$ intersects with an appropriate
        projection of $S$ (i.e., there is some projection of $S$ in $\mathcal{S}_{v'}$). Note that
        we can go over all children of $v$ in $O_d(1)$ time to do this.
    \item If $S'$ is facet-covering for a facet $f$ of $C_v$, then update $M(v,f)$ as follows: If
        $S'\cap C_v$ covers $M(v,f)$, set $M(v,f)$ to $S'\cap C_v$.
    \item Repeat from \Cref{step:pop1} till the queue is empty.
\end{enumerate}

We know that some projection of $S$ is included in $S_v$ for at most $O_d(\log^{2d-1}N)$ nodes
$v$ in the tree (from \Cref{?}). Since the traversal described above only visits such nodes, it takes only
$O_d(\log^{2d-1}N)$ time.

We repeat this process for every set in $\mathcal{S}$, which takes a total of $O_d(m\log^{2d-1}N)$ time. This
ensures that the value stored for any $M(v,f)$ has iterated over all relevant sets, and hence, stores
the maximal facet-covering set.

\paragraph*{On arrival of a new input point $p$} We need to find the collection of sets in the new set
system that contain this point. We again achieve this by a bfs like traversal of the extended
quad-tree. We use it to identify the nodes containing $p$ or its projection of. Once we have these nodes,
we can refer to the previously described data structure to choose all sets which cover $p$ and are
maximal facet-covering for the corresponding cell (i.e., are included in the new instance $\hat{\mathcal{S}}$).

We now describe the method in some more detail. Initialize the bfs queue to the root node of the extended quad-tree:
\begin{enumerate}
    \item\label{step:pop2} Remove the top element $v$ from the bfs queue. Let $p'$ be the projection of $p$ to the
        appropriate dimensions of $C_v$.
    \item Add to the queue all child nodes $v'$ of $v$, such that $C_{v'}$ contains $p$ or a
        projection of $p$ (i.e., there is some projection of $p$ in $\mathcal{P}_{v'}$). Note that
        we can go over all children of $v$ in $O_d(1)$ time to do this.
    \item For each facet $f$ of $C_v$, if $M(v,f)$ contains a projection of  $p$, then pick the set
        $M(v,f)$.
    \item Repeat from \Cref{step:pop2} till the queue is empty.
\end{enumerate}

Again, we know from \Cref{?} that there are only $O_d(\log^{2d}N)$ nodes in the extended-quad-tree that
contain some projection of point $p$. And since the above traversal only visits such nodes,
we can compute the required collection of sets in $O_d(\log^{2d}N)$ time.





\subsection{Hitting Set in $d$-dimensional hyperrectangles}
\paragraph*{Mapping $d$-dimensional hyperrectangles to $2d$-dimensional hypercubes}
In the dynamic setting of hitting set, we are initially given only the set of points $P\in\R^d$, and the
hyperrectangles are inserted and deleted as the algorithm progresses. We note here that any potential input hyperrectangle, can be
appropriately shrunk such that its vertices coincide with the coordinates of some given point (and
the same new hyperrectangle covers the same set of points as before). So we essentially have only
$n^d$ combinatorially relevant hyperrectangles.

We apply \Cref{claim:rectangledynsetcover_1} assuming that each of these $n^d$ hyperrectangles is
present in the instance, to get a map from these hyperrectangles to a set of hypercubes with integer
coordinates in  $[0,2m]^{2d}$.  (this is done only to aid in mapping the given set of points. The
hyperrectangles themselves are considered only when they are added/removed at some time step).  Now
if we similarly (to the set cover case) map the given set $P$ to a set of points $P'$ (using
\Cref{lemma:orthantmap_1} and a coordinate shift), we ensure that on arrival/deletion of any
hyperrectangle, the set intersections are preserved in the new instance.

\begin{claim}
    An $\alpha$-approximate hitting set algorithm on hypercubes in $\R^{2d}$
    provides an $\alpha$-approximate hitting set algorithm on hyperrectangles in $\R^{d}$.
\end{claim}
\begin{claim} Note that since the above mapping preserves set intersections, a solution to the
 hitting set instance $\mathcal{I}_1$ on hyperrectangles, can be mapped to a hitting set (of the
 same size) of the newly constructed instance $\mathcal{I}_2$ on hypercubes. This means that the
 optimal solution $\OPT_1$ to $\mathcal{I}_1$, is a valid hitting set for $\mathcal{I}_2$, and
 similarly the optimal solution $\OPT_2$to $\mathcal{I}_2$, is a valid hitting set for $\mathcal
 {I}_1$. So, we have $|\OPT_1|\ge|\OPT_2|$ and $|\OPT_2|\ge|\OPT_1|$, giving us that
 $|\OPT_1|=|\OPT_2|$. This means that an $\alpha$-approximate solution for hitting set on $\mathcal
 {I}_2$ will also correspond to an $\alpha$-approximate solution for hitting set on $\mathcal
 {I}_1$.
\end{claim}

Now let us look at the running time for these operations:
\Cref{lemma:orthantmap_1} sorts the list of end-points (of projections of hyperrectangles to
each dimension) to do the reduction. This sorting takes $O_d(n\log n)$ time.
It takes $\log N'$ additional time to read each point in the input; hence, amounting to a total
of $O_d(n(\log n+\log N'))$ time in the pre-processing.

Any time a hyperrectangle is introduced we find can compute its corresponding hypercube in $O(d)$
time from the already available sorted lists of coordinates (from \Cref{lemma:orthantmap_1}). Taking into
account the time to read the input hyperrectangle, we get a total of $O_d(\log N')$ time during any update operation.

In addition to the above, we also need to account for a $O(\log N')$ overhead for reading the input
in both of the above operations.




\paragraph*{Pre-processing for the points} In this subsection, we pre-process the given set of
points $P$ to compute for each cell, the points that are closest to each of its facets.
For each node $v$ in the quad-tree, and for each facet $f$ of its corresponding cell $C_v$, define
and initializing a variable $N(v,f)$ to $\emptyset$ (Here we assume for consistency that the orthogonal
distance of $\emptyset$ to any facet, of any cell, is infinity).

For each point in the instance, we do a bfs-like traversal of the extended quad-tree. Start with
the bfs queue storing just the root:
\begin{enumerate}
    \item\label{step:pop3} Remove the top element $v$ from the bfs queue. Let $p'$ be the projection
     of $p$ to the appropriate dimensions of $C_v$.
    \item Add to the queue all child nodes $v'$ of $v$, such that $C_{v'}$ contains $p$ or its appropriate
        projection (i.e., there is some projection of $p$ in $\mathcal{P}_{v'}$). Note that
        we can go over all children of $v$ in $O_d(1)$ time to do this.
    \item For each facet $f$ of $C_v$, if $p'$ is closer to the facet than $N(v,f)$, then update
        $N(v,f)$ to $p'$.
    \item Repeat from \Cref{step:pop3} till the queue is empty.
\end{enumerate}

We know from \Cref{?} that there are only $O_d(\log^{2d}N)$ nodes in the extended-quad-tree that
contain some projection of point $p$. And since the above traversal only visits such nodes,
we can compute the required collection of sets in $O_d(\log^{2d}N)$ time.

We repeat this process for every point in $P$, which takes a total of $O_d(m\log^{2d}N)$ time.
This ensures that the value stored for any $N(v,f)$ has iterated over all relevant
points in the cell, and hence, stores the closest such point.

\paragraph*{On arrival of a new input set $S$} We need to find the collection of points in the new
 instance that hit this set.
 \asr{this might change depending on the aux set instance.}

 We again achieve this by a bfs like traversal of the extended
 quad-tree. We use it to identify the nodes intersecting $S$ or containing a projection of $S$. Once we have these
 nodes, we can refer to the previously described data structure to choose a subset of points in
 the cell that hit the projection of $S$.
 
We now describe the method in some more detail. Initialize the bfs queue to the root node of the extended quad-tree:
\begin{enumerate}
    \item\label{step:pop4} Remove the top element $v$ from the bfs queue. Let $S'$ be the projection
     of $S$ to the appropriate dimensions of $C_v$.
    \item Add to the queue all child nodes $v'$ of $v$, such that $C_{v'}$ intersects with an
     appropriate projection of $S$ (i.e., there is some projection of $S$ in $\mathcal{S}_{v'}$). Note that we can go over all children of $v$ in $O_d(1)$ time to do this.
    \item For each facet $f$ of $C_v$, pick $N(v,f)$ if it intersects with $S'$.
    \item Repeat from \Cref{step:pop4} till the queue is empty.
\end{enumerate}

We know that some projection of $S$ is included in $S_v$ for at most $O_d(\log^{2d-1}N)$ nodes
$v$ in the tree (from \Cref{?}). Since the traversal described above only visits such nodes, it takes only
$O_d(\log^{2d-1}N)$ time.
}
\fi

\section{Set Cover for $d$-dimensional Hyperrectangles\label{sec:set-cover-hyperrectangles}}
We present our dynamic $O_{d}(\log^{4d-1}m)$-approximation algorithm
for set cover for $d$-dimensional hyperrectangles:  first, for the unweighted case;  and then we extend it to the
weighted case.

First, we reduce our given instance on hyperrectangles in $\R^d$, to an
equivalent instance on hypercubes in $[0,2m]^{2d}$  such that all the vertices of hypercubes  have integral 
coordinates\footnote{This is a standard (folklore) reduction. See \Cref{lemma:orthantmap_1} for details.}. We further go on to claim that by solving the (dynamic) set cover
instance on hypercubes with some approximation factor $\alpha$, we can also
compute an $\alpha$-approximate solution in the original instance, with some
additional overhead in the initialization and update times.
%\akr{added footnote and some details in red. and changed lemma stmt}

\iffalse{
First, we reduce our given instance to an equivalent instance in which
all coordinates of the given $m$ hyperrectangles are integers in
$\{1,\dots,2m\}$: in each dimension, simply sort the coordinates
of the squares non-decreasingly, and then redefine them to $\{1,\dots,2m\}$
while keeping the order. Note that if a point is inserted according
to the original coordinates, we can easily compute a corresponding
point according to the new coordinates. Next, we reduce our problem
to hypercubes by increasing the dimension to $2d$.}
\fi
\begin{restatable}{lemma}{reducetocubes}\label{lem:reduce-to-cubes}
 Let $d'\in\Z_+$, and we are given hyperrectangles with integral vertices in $[0,N')^{d'}$. 
If we have a dynamic $\alpha$-approximation
algorithm for set cover in $2d'$-dimensional hypercubes in $[0,N]^{2d'}$ with integral vertices 
having $f(d',m,n,N)$ update
time and $g(d',m,n,N)$ initialization time, then we can construct a dynamic $\alpha$-approximation
algorithm for set cover in $d'$-dimensional hyperrectangles with update time $O_{d'}(f(d',m,n,2m)+\log m+\log N')$ and initialization time $O_{d'}(g(d',m,n,2m)+m(\log m+\log N'))$
\end{restatable}
\begin{proof}See \Cref{subsec:hypercube-reduction} and \Cref{subsec:preprocessandquerysc} for the proof. \end{proof}

For the remainder of this section, we suppose that we are given a set
of $2d$-dimensional hypercubes $\S$ with integral vertices, contained in $[0,N)^{2d}$, %for some constant $d\in\Z_+$,
where $N$ is a power of 2 with $N=m^{{O_d}(1)}$. %For the remainder of this section suppose that we are given a set of $2d$-dimensional hypercubes
%$\S$ contained in $[0,N)^{2d}$ for some constant $d\in\N$. Assume w.l.o.g.~that $N$ is a power of
%2.
Also, suppose that $P\subseteq[0,N)^{2d}$ is the current set of points
%(\ak{with integral coordinates})
to cover.
% We will first define a generalization of the
% quad-tree that we have used in \Cref{sec:Set-cover-squares}.
We first explain
why a straight-forward generalization of our algorithm from the two-dimensional
case does not work even for the three dimensional case. Further, we will extend our quad-tree
and define an auxiliary instance of (not necessarily geometric) set
cover in which each element is contained in  $O_{d}(\log^{2d}N)$
sets. We will maintain an $O(f)$-approximate solution for it (using
the data structures introduced by Bhattacharya et al.~\cite{bhattacharya2021dynamic})
in $O_{d}(f\log^{2}n)$ worst-case update time.  Finally, we will argue
that this yields an $O_{d}(\log^{4d-1}N)$-approximate solution for our instance. 
%for our instance of geometric set cover for hyperrectangles.

\begin{figure}[ht]
    \center
    \includegraphics[scale=0.6]{04CounterEx3D}
    \caption{Picking the red facet-covering square does not cover the
    intersection of the green square (from $\OPT$) with the (uncolored) cell.
    Note that there is no corner of a set from $\OPT$ in the cell.}
    \label{fig:CounterEx3D}
\end{figure}

\paragraph*{Quad-tree.}

Like in \Cref{sec:Set-cover-squares}, we can define a quad-tree $T=(V,E)$
to guide our algorithm. The definition of $T$ is the $2d$-dimensional
analog to the definition of $T$ in \Cref{sec:Set-cover-squares}, where instead of four, each internal node will now have $2^{2d}$ children.
\begin{comment}
The definition of $T$ is the $2d$-dimensional
analog to the definition of $T$ in \Cref{sec:Set-cover-squares}.
Each node $v\in V$ corresponds to a cell $C_{v}\subseteq[0,N)^{2d}$.
The root $r$ corresponds to $C_{r}:=[0,N)^{2d}$. A node $v\in V$
is a leaf of $T$ if it corresponds to a hypercube $C_{v}$ in which
each edge has length 1. Otherwise, we define that $v$ has $\ak{2^{2d}}$
children $v_{1},\dots,v_{2^{2d}}$ corresponding to the $2^{2d}$
hypercubes we obtain if we partition $C_{v}$ into hypercubes whose
edges have half the length of the edges of $C_{v}$. Formally, assume
that $C_{v}=[x_{1}^{(1)},x_{2}^{(1)})\times\dots\times[x_{1}^{(2d)},x_{2}^{(2d)})$.
Each child $v_{i}$ corresponds to a cell $C_{v_{i}}$ such that in
each dimension $j\in[2d]$ the projection of cell $C_{v_{i}}$ to
the $j$-axis is $[x_{1}^{(j)},x_{\mathrm{mid}}^{(j)})$ or $[x_{\mathrm{mid}}^{(j)},x_{2}^{(j)})$
with $x_{\mathrm{mid}}^{(j)}:=(x_{2}^{(j)}-x_{1}^{(j)})/2$, and there
is one child $v_{i}$ for each of these $2^{2d}$ combinations. 
For
each $v\in V$, let $P_{v}:=C_{v}\cap P$ and let $\S_{v}$ denote
the squares in $\S$ that intersect $C_{v}$. Note that each point
$\ak{p\in[0,N)^{2d}}$ is contained in $O(\log N)$ cells $C_{v}$,
and if $p\in P$ then it contained in $O(\log N)$ sets $P_{v}$.
\end{comment}
Unfortunately, our approach from \Cref{sec:Set-cover-squares} no
longer works even for hypercubes of dimension 3.
%any dimension $d'>2$. Suppose that $d'=3$. 
A natural generalization of our approach would be to iterate
through the nodes of $T$, and for each node $v$ to consider
the facets of $C_{v}$. For each such facet $F$ we could select a cube
$S\in\S$ that contains $F$ and that has maximal intersection with
$C_{v}$ among all such cubes. The problem is that maybe $C_{v}$
does not contain a vertex of a cube $S\in\OPT$, but still the selected
cubes do not cover all remaining uncovered points of $P\cap C_{v}$
(see \Cref{fig:CounterEx3D}).
Then, we cannot charge the selected cubes to a cube in $\OPT$. Also,
we cannot argue that we do not select any further cubes in the descendants
of $v$. Another possible generalization would be to consider each
edge $e$ of $C_{v}$ and try to select cubes in $\S$ containing
$e$. However, 
%unlike the approach via the facets, 
two
such cubes $S,S'\in\S$ might be incomparable, i.e.,
%in the sense that 
$S\cap C_v \not\subseteq S'\cap C_v$
and $S'\cap C_v\not\subseteq S\cap C_v$ and we cannot afford to select all 
such %pairwise incomparable 
cubes.

\paragraph*{Extended quad-tree.}

Therefore, in contrast to \Cref{sec:Set-cover-squares}, we extend
now $T$ to a tree $T'=(V',E')$ that we call the \emph{extended quad-tree}.
We %start with an empty graph and show how to
build $T'$ recursively. First, we define the root node of $T'$ to
be the same as the root node of $T$, i.e., corresponding to the cell
$[0,N)^{2d}$. 

Now, we describe the recursive process to define children of a node.
For each node $v\in V'$ whose children have not been defined, if
$v$ corresponds to a $d'$-dimensional hypercube $C_{v}$ then we
add $(2^{d'}+d')$ number of children. The first $2^{d'}$ children
are formed in exactly the same way as in the usual quad-tree (if the
side-length of $C_{v}$ is at least $2$, otherwise we do not add
these children). We refer to these children as the \emph{subdivision
children} of $v$. Then, we add $d'$ new children $v_{1},\dots,v_{d'}$
to $v$, which we call the \emph{projection children}. For each $i\in[d']$,
the cell $C_{v_{i}}$ is a ($d'-1$)-dimensional cell corresponding to $v_{i}$ is the projection
of $C_{v}$ to the dimensions $\{1,...,i-1,i+1,...,d'\}$ (see \Cref{fig:extended-quadtree}).
Note that $T'$ contains the usual quad-tree $T=(V,E)$, but it is
extended by the projection children of each node $v\in V$ and their
descendants. A node $v\in V$ has projection children as long as its dimension is at least $2$. Once a node has dimension $1$, we do not add either subdivision or projection children to it in $T'$.

For each node $v\in V'$, we define a set of points $P_{v}$ and a
set of hypercubes $\S_{v}$. The set $P_{v}$ is essentially the subset
of $P$  (or their projections) that is contained in $C_{v}$. 
Formally,
for the root node $r$ we initialize $P_{r}=P$ and $\S_{r}=\S$. Let $v\in V'$ and
let $v'\in V'$ be a child of $v$. If $v'$ is a subdivision child
then we define $P_{v'}=P_{v}\cap C_{v'}$, if $v'$ is a projection
child then we define $P_{v'}$ to be the projection of the points
in $P_{v}$ to $C_{v'}$. 
%In the definition of the sets $\S_{v}$, for the root node $r$ we
%again initialize $\S_{r}=\S$. %\al{We say that for a hypercube $S\in\S$ and a node $v\in V'$, $C_v\cap S$ is the projection of $S$ in $v$}. 
%Let $v\in V'$ and let $v'\in V'$
%be a child of $v$. 
To define $\S_{v'}$,  if $v'$ is a subdivision child then we define
that $\S_{v'}$ contains all the hypercubes in $\S_{v}$ that have non-empty
intersection with $C_{v'}$. Suppose now that $v'$ is a projection
child such that $C_{v'}$ is the projection of $C_{v}$ to the dimensions
$\{1,\dots,i-1,i+1,\dots,d'\}$. Let $J$ denote the projection of $C_{v}$
to the $i$-th dimension (hence, $J$ is an interval). Consider
each $S\in\S_{v}$ \emph{for which the projection of $S\cap C_v$ to the $i$-th dimension
equals }$J$. We then include $S\in \S_{v'}$. %\alr{add $S$ to $\S_{v'}$ instead of $S'$ maybe. Better to store the actual hypercubes} %$S'$ to $\S_{v'}$ where $S'$
%is the projection of $S$ to the dimensions $\{1,\dots,i-1,i+1,\dots,d'\}$.
%Note that we are storing the actual hypercube $S\in\S$ as well as its corresponding projection in $C_v'$ as members in the set $\S_{v'}$. Hence, we will use the terminology of a hypercube or its projection interchangeably w.r.t. a particular cell.}
%\alr{Should we make this a claim}{
Observe that $S\in\S_{v}$ contains a point $p\in P_{v}$
if and only if  $S'$  contains the corresponding projection of
point $p$ in $C_{v'}$. 
%This is why we required that the projection of $S\cap C_v$ to
%the $i$-th dimension equals $J$. 


%Note that we make the notion of projections transitive: we say that if a point
%$p''$  is a projection of a point $p'$ which is a projection of a point $p$, then $p''$ is also a projection of $p$. We do this similarly for a hypercube and its corresponding projections.
%Also, for convenience, we define that each point $p$ and each hypercube
%$S$ is a projection of itself. 
%\akr{commented above para in source, may be put in appendix}

%\al{ 
For a hypercube $S\in\S$ and a node $v\in V'$, we consider $S\cap C_v$ only if $S\in\S_v$.
For $p\in P$, let $V_p'$ be the set containing all nodes $v$ such that a projection $p'$ of $p$ is included in $P_v$.

\begin{figure}[ht]
\centering \includegraphics[scale=0.5]{03extendedQTree}
\caption{Extended quad-tree on a $2\times2\times2$ cube, with green edges for
subdivision children, and red edges for projection children.}
\label{fig:extended-quadtree} 
\end{figure}

\begin{restatable}{lemma}{fewsets} \label{lem:few-sets}
    For each point $p\in P$, $|V_p'|=O_d(\log^{2d}N)$.
\end{restatable}\begin{proof}
Deferred to \Cref{subsec:few-sets} \end{proof}

%\paragraph*{Supremal edge-covering}


\paragraph*{Auxiliary set cover instance.}

Based on $T'$, we define now an instance of set cover for the set
of elements $P$ and a family of sets $\hat{\S}$, where the sets
in $\hat{\S}$ will be arbitrary subsets of $P$.
%(so not necessarily
%geometric objects like hypercubes etc.). 
We will maintain an $O(f)$-approximate
solution to $(P,\hat{\S})$, and argue that it will yield an $O_{d}(\log^{4d-1}N)$-approximate
solution to $(P,\S)$, where $f$ refers to the frequency of this
new set system.

For each each node $v\in V'$ and each hypercube $S\in\S_{v}$ we
say that $S$ is \emph{maximally facet-covering for }$v$ if there
is a facet $F$ of $C_{v}$ such that $F\subseteq S$ and $S\cap C_{v}$
is maximal among all intersections $S'\cap C_{v}$ for all $S'\in\S_{v}$
containing $F$. In case that there are two hypercubes $S,S'\in\S_{v}$
with $S\cap C_{v}=S'\cap C_{v}$, we break ties in an arbitrary fixed
way.
For each node $v\in V'$ and each maximally facet-covering hypercube
$S\in\S_{v}$ for $v$ we introduce a corresponding set in $\hat{\S}$; we
denote this set by $(v,S)$. We define that $(v,S)$ contains all
points $p\in P$ such that there is a projection $p'$ of $p$ in $C_v$ with the property that $p'\in S\cap C_{v}$. %\al{For a set $S\in\S$, we will define a set of cells $V_S'$ of }

%\akr{$\log m$ vs $\log N$ in claim statements}

%This claim is generalized as follows: Consider a  $d'$-dimensional cell $C$ to which $S$ was
%assigned. Denote the projection of $S$ in $C$ by $S'$. In the subtree of $C$, consider only those
%cells which are connected to $C$ by subdivision edges and denote this set by $\mathcal{C}'$. These
%cells are $d'$-dimensional by definition. Define the set $\mathcal{C}_{S'}$ to be the collection of
%cells in $\mathcal{C}'$ for which $S$ is supremal edge-covering. Then, we have the following claim.

%\begin{claim} \label{clm:secc_2}
 %   For a $d'$-dimensional hypercube $S$ in the instance, $|\mathcal{C}_{S'}|\leq 4^{d'} \cdot O(\log m)$.
%\end{claim}
%The proof of this claim follows similarly as the proof of \Cref{clm:secc_1}.
%\begin{claim}
 %   Deferred to \Cref{subsec:secc_2}
%\end{claim}



%efine the set $V_S'$ to include all the nodes $v\in V'$ such that the projection of $S$ in $C_v$ is facet-covering for it but not for its parent in $T'$.

\paragraph*{Properties of set cover instance.}

We want to argue that our auxiliary set cover instance $(P,\hat{\S})$ has three important properties:
\begin{enumerate}
    \item Let $f$ denote the maximum number of sets in $\hat{\S}$ that a point $p\in P$ is contained in;
        we will show that $f=O_d(\log^{2d} N)$,
    \item  A hypercube $S\in\mathcal{S}$ covers a point $p\in P$ if and only if there exists a node $v$ in $V'$ such that %a projection of 
   $p\in P_v$, $S\in \S_v$  %is included in 
    and the projection of $p$ in $C_v$ is covered by %a projection $S'$ of 
    $S\cap C_v$ such that $S\cap C_v$ is facet-covering for $v$,
    \item Any $\alpha$-approximate solution for $(P,\hat{\S})$ yields an $O_d(\alpha\log^{2d-1} N)$-approximate solution for our given instance $(P,\S)$ of geometric set cover.
\end{enumerate}

We start by proving the first property.  By \Cref{lem:few-sets}, for each point $p\in P$ there are
only $O_d(\log^{2d} N)$  nodes $v\in V'$ whose sets $P_{v}$ contain $p$ or its projections. Then,
% For each
%such $v$ there are at most $2d$ maximally face-covering hypercubes.
by definition, for each such node $v$,  we may include $p$ in at most $4d$ sets in $\hat{\S}$ (per facet of the cell $C_v$). Thus, $p$ is
contained in  $O_d(\log^{2d} N)$ sets in $\hat{\S}$.
\begin{restatable}{lemma}{pointfewsets} \label{lem:point-few-sets}
    Each point $p\in P$ is contained in $O_d(\log^{2d} N)$ sets in $\hat{\S}$.
\end{restatable}

The second property, informally, says that covering a projection of point in some appropriate cell in
$T'$ by projection of a hypercube is equivalent to covering the point in the actual instance
$(P,\S)$.

\begin{restatable}{lemma}{rectangleDynSetCovcovering} \label{lem:rectangleDynSetCov_covering}
    %Any point $p\in U_2$ is covered by a hypercube $h\in \mathcal{F}_2$ if and only if $h$ appears as one of the square pieces in some 2d  instance $I$ in $\mathcal{I}_C$, where this square piece covered the equivalent projection of $p$ in the 2d instance ($C$ is the cell in $\mathcal{F}_2$ which contained $p$ and for which $h$ was \secc in the original instance).
    %A hypercube $S\in\mathcal{S}$ covers a point $p\in P$ if and only if there exists a node $v$ in
    %$V_S'$ such that the projection of $p$ is included in $P_v$ and the projection of $S$ in $v$ covers the projection of
   % $p$ in it.
   A hypercube $S\in\mathcal{S}$ covers a point $p\in P$ if and only if there exists a node $v$ in $V'$ such that %a projection of 
   $p\in P_v$, $S\in \S_v$  %is included in 
    and the projection of $p$ in $C_v$ is covered by %a projection $S'$ of 
    $S\cap C_v$ such that $S\cap C_v$ is facet-covering for $v$.
\end{restatable}
\begin{proof}
    Proof deferred to \Cref{subsec:rectangleDynSetCov_covering}.
\end{proof}

Now we prove the third property. 
Let $\OPT\subseteq\S$ be an optimal
solution to $(P,\S)$ (we will use $\OPT$ to refer to the set and its size/weight both, which will  be clear from the context). We construct a relatively small solution 
$\hat{\S}'\subseteq\hat{\S}$ to $(P,\hat{\S})$. 
% by doing the following procedure for each $S\in\OPT$.
Intuitively, for each $S\in\OPT$ we identify each cell $C_{v}$ for which $S$ or a projection
of $S$ are facet-covering for a facet $F$ of $C_{v}$. Then $\hat{\S}$
contains a set $(v,S')$ where $S'$ is maximally facet-covering for
$v$ and $F\subseteq S'$. If we select all such sets $(v,S')$ then
we cover all points from $P$ that $S$ covers in $\OPT$. In order to achieve the latter, we show intuitively that it is sufficient
to consider only nodes $v$ such that for no ancestor $v'$ of $v$ the
set $S$ or a projection of $S$  are facet-covering for $v'$.  In
this way, we select at most $O_d(\log^{2d-1}N)$ sets corresponding
to $S$.
%Let $\OPT\subseteq\S$ be an optimal solution to $(P,\S)$. Then we
%can construct a solution $\hat{\S}'\subseteq\hat{\S}$ to $(P,\hat{\S})$ by doing the following
%procedure for each $S\in\OPT$.  Intuitively, we identify all cells $C_{v}$ for which $S$ (or a
%projection $S'$ of $S$) are face-covering, and for those cells we select the set $(v,S)\in\hat{\S}$
%or the set $(v,S')$, respectively). In this way, our selected sets from $\hat{\S}$ cover all points
%in $P$ that are contained in $S$. In fact, in order to achieve the latter, it is sufficient that we
%do this only for cells $C_{v}$ such that $S$ is \emph{not} face-covering for any cell $C_{v'}$ of an
%ancestor $v'\in V'$ of $v$ in $T'$. In this way, we select at most $O_d(\log^{2d-1} N)$ sets
%corresponding to $S$.
\begin{comment}
\al{First, we show that for any set $S\in\S$, $|V_S'|=O_d(\log^{2d-1}N)$. After this, for $S\in\OPT$, we will define a subcollection $\S'\in\hat{\S}$ of size $O_d(\log^{2d-1}N)$. This subcollection has the property such that if $p\in P$  was covered by $S$, then $p$ was covered by at least one set in $\S'$. }
\al{First, for any node $v\in
V_S'$  we map at most $4d$ sets in $\hat{\S}$ to it. Denote this subcollection  of size at most $4d$ by $\S'_v$. %$\S'_v$ is such that for any $p\in P$ such that the projection of $p$ was covered by the projection of $S$ in the cell for
%$v$, one of the sets in it 
%necessarily covers $p$ in $(P,\hat{\S})$.  
For any node $v\in V_S'$, for $S\in\S$, include $(v,S)$ in $\S'$ if $(v,S)$ was defined according to our auxiliary set cover instance definition. Since the projection of $S$ in $C_v$ was facet-covering for $v$, there existed a maximal facet-covering hypercube $S_0\in\S$ such that $S_0\cap C_v\supseteq S\cap C_v$. 
Then, $(v,S_0)$ was included in $\S'$. Further, applying~\Cref{lem:rectangleDynSetCov_covering}, we get that fot any $p\in P$ such that the projection of $p$ was covered by the projection of $S$ in the cell for
$v$, $S_0$ 
necessarily covered $p$ in $(P,{\S})$. Since, $(v,S_0)$ is defined, we have that $p$ was covered by a set in $\S'_v$. 
Define another subcollection $\S'=\bigcup_{v\in V_S'}\S'_v$. Then, we have that $|\S'|=O_d(\log^{2d-1}N)$. Observe that if a hypercube $S$ covered a point $p\in P$ in the original set system, then by\Cref{lem:rectangleDynSetCov_covering} there exists a node  $v'\in V_S'$ such that the projection of $p$ in $C_{v'}$ was covered by the projection of $S$ in it. By definition, $S\cap C_{v'}$ was facet-covering for $v'$. Note that $v'\in  V_S'$ and hence, applying the same argument as before, there exists a set $S_0$ such that $(v',S_0)\in \S'$ and $(v',S_0)$ covered $p$ in $(P,\hat{\S})$.}
% Else,  $v$ is necessarily an uncovered leaf and we pick the set whose
%respective projection covers the projection of $S$ in $v$. We know this exists because the
%projection of $S$ in $v$ is an interval covering either the left or the right endpoint of the cell
%$v$ (which is an interval). In this case as well, the projection of $p$ is covered by the projection
%of this mentioned set inside the cell. %Denote this set by $S'$ in either case and note that $S'$
%covers $p$. 
\end{comment}
%Hence, %by \Cref{lem:rectangleDynSetCov_covering}, 
%we are able to cover $S$ entirely in
%the original instance of set cover $(P,\S)$ by a subcollection of $O_d(\log^{2d-1}N)$ sets. 
Applying this argument to all sets $S\in \OPT$ yields the following lemma.

\begin{restatable}{lemma}{cheapsolution} \label{lem:cheap-solution}
    There is a solution to $(P,\hat{\S})$ that contains at most $\OPT\cdot O_d(\log^{2d-1} N)$ sets.
\end{restatable}
\begin{proof}
    Proof deferred to \Cref{subsec:rectangleDynSetCov_covering}.
\end{proof}
%\begin{proof}
%Let $S\in\OPT$. We identify all vertices $v'\in V'$ such that
% \begin{itemize}
% \item $\S_{v'}$ contains a set $S'\in\S_{v'}$ that is a projection of $S$ and that is
%   face-covering for some face $F$ of $C_{v'}$ and
% \item for no ancestor $v''\in V'$ of $v'$ the corresponding set $\S_{v''}$ contains a set
%  $S'\in\S_{v'}$ that is a projection of $S$ and that is face-covering for some face $F$
%  of $C_{v'}$.
% \end{itemize}
%For each such vertex $v'\in V'$, let $S''\in\S_{v'}$ denote the maximally face-covering
%hypercube in $\S_{v'}$ that contains $F$.  Let $\hat{S}''$ be the set in $\hat{\S}$ that corresponds to $S''$.  We select $\hat{S}''$. We do this procedure for each $S\in\S'$.

%It remains to prove that we select at most $\OPT\cdot O_d(\log^{2d-1} N)$ sets. \dots todo ...
%\end{proof}

%On the other hand, we want to argue that we can 
\noindent To turn any solution $\hat{\A}$ for
$(P,\hat{\S})$ into a solution for $(P,\S)$, 
%. This is simple: 
for each set $(v,S)\in\hat{\A}$ we
identify the hypercube $\tilde{S}\in\S$ s.t. $S$ is a projection of $\tilde{S}$, and we select
$\tilde{S}$ for our solution to $(P,\S)$.



\begin{restatable}{lemma}{correspondingsolution} \label{lem:corresponding-solution}
    For any solution $\hat{\A}\subseteq\hat{\S}$
    to $(P,\hat{\S})$ there is a solution $\A$ to $(P,\S)$ with $|\A|\le|\hat{\A}|$.
    For each $(v,S)\in\hat{\A}$ there is a corresponding hypercube $\tilde{S}\in\A$
    and given $(v,S)$, we can identify $\tilde{S}$ in $O_d(\log N)$ time.
\end{restatable}

\paragraph*{Dynamic algorithm.}

We maintain an approximate solution $\hat{\A}\subseteq\hat{\S}$ to $(P,\hat{\S})$ dynamically using
the data structure for arbitrary instances
of set cover in \cite{bhattacharya2021dynamic}. It guarantees an approximation ratio of $O(f)$ and worst-case update time of
$O(f\log^2 (Wn))$ even for the weighted case. If a point $p$ is inserted to $P$ or removed from $P$,
then by \Cref{lem:few-sets} we know that we need to update at most $(\log N)^{O(d)}$ sets from
$\hat{\S}$, i.e., insert $p$ in all these sets or remove $p$ from all these sets.

%\al{We may have an $O_d(\log N)$ overhead on the update time bound as compared to the dynamic algorithm of \cite{bhattacharya2021dynamic} since subclaim $3$ in \Cref{clm:rectangleDynSetCov_access} requires $O_d(\log N)$ time.}
\begin{restatable}{lemma}{updatefast}\label{lem:update-fast}
    If a point $p$ is inserted into $P$ or removed from $P$, then we can update the sets $\hat{\S}$
    in worst-case time $O_d(\log^{2d} N)\log^2 n$.
\end{restatable}
Proof of the above lemma can be found in \Cref{subsec:update-fast}. %After an update, our data structure for $(P,\hat{\S})$ might update its solution. Since %its update
%time is $O_d(\log^{2d+1} N)\log^2 (n)$ in the worst case, 
We know that at most $O_d(\log^{2d} N)\log^2
(n)$ sets in the solution are changed after an update. Hence, using \Cref{lem:corresponding-solution} and
\Cref{lem:update-fast} we can update our solution to $(P,\S)$ in time
$O_d(\log^{2d} N)\log^2 n$.
\begin{restatable}{theorem}{DynSetCov} \label{thm:DynSetCov}
    There is a dynamic algorithm for unweighted geometric set cover for $d$-dimensional
    hyperrectangles with a worst-case update time of $O_d(\log^{2d}N\cdot\log^2 n)$ when a point is added
    or deleted.%\alr{Should we not add approximation ratio here?}
\end{restatable}

%\subsection{\label{subsec:set-cover-weighted}Weighted case}
%We can extend our algorithm to the weighted case by rounding the weights of the  to $O(\log W)$ weight classes
%and then appropriately defining the auxiliary instance.
%\begin{theorem} \label{thm:WtDynSetCov}
 %   There is an $O_d(\log^{4d-1}m)\log W$-approximate dynamic algorithm for weighted geometric set cover for $d$-dimensional
  %  hyperrectangles with worst-case update time of $O_d(\log^{2d} m)\log^3(W m)$ when a point is
   % added or deleted.%\alr{Should we not add approximation ratio here?}
%\end{theorem}
%\begin{proof}
  %  Proof deferred to \Cref{subsec:update-fast}.
%\end{proof}

\subsection{\label{subsec:set-cover-weighted}Weighted case}
We now extend our algorithm  to the weighted case, where each given
hyperrectangle $S\in\S$ has a weight $w_{S}\in[1,W]$. First, we round the
weights of the hyperrectangles.
\begin{restatable}{lemma}{twoapprox} \label{lem:2approx}
    By losing a factor of 2 in the approximation ratio, we can assume
    that for each $S\in\S$ the weight $w_{S}$ is a power of 2.
\end{restatable}
\iffalse{
\al{Hence, our hyperrectangles have $O(\log W)$ different weights now and each hyperrectangle belongs to one of these weight classes.  We build the extended quad-tree
$T'$ like above. In the definition of the auxiliary instance $(P,\hat{\S})$, we slightly adjust the
sets that contain a point $p\in P$. For each node $v\in V'$ and for each weight class,   we consider a maximally facet-covering hypercube $S\in\S_v$ for $v$ belonging to this weight class and introduce a corresponding set in $\hat{\S}$; we denote it by $(v,S)$. We define that $(v,S)$ contains all points $p\in P$ such that there is a projection $p'$ of $p$ in $C_v$ with the property that $p'\in S\cap C_v$.}   %We define $V_p'$ and $V_S'$ just as we defined in the previous subsection. 
%The definition of $V_p'$ and $V_S'$ (for a point $p\in P$ and a set $S\in\S$, respectively,) remains the same as before.  }
%Only for the uncovered leaf vertices in $V_p''$, instead of only including the sets representing the
%two maximal intervals in $\hat{\S}_p$, we iterate over all the $O(\log W)$ weight classes and for
%each weight class, separately consider the sets whose projections were two maximal intervals in each
%direction. Include them in $\hat{\S}_p$ if and only if they cover the projection of $p$ therein. For
%the covered nodes, we again assign the covering sets, one for each of the weight classes, if they
%exist.

%In the definition
%of the auxiliary instance $(P,\hat{\S})$, we slightly adjust the
%definition of maximally face-covering. We define for each each vertex
%$v\in V'$ and each hypercube $S\in\S_{v}$ that $S$ is \emph{maximally
%face-covering for }$v$ if there is a face $F$ of $C_{v}$ such that
%$F\subseteq S$ and $S\cap C_{v}$ is maximal among all intersections
%$S'\cap C_{v}$ for all $S'\in\S_{v}$ \emph{with $w_{S}=w_{S'}$
%}containing $F$. The reason for the adjustment is that maybe $w_{S}>w_{S'}$
%but $S'\cap C_{v}\subsetneq S\cap C_{v}$ and then $S$ does not dominate
%$S'$ (since $w_{S}>w_{S'}$) but on the other hand $S'$ does not
%dominate $S$ either (since $S'\cap C_{v}\subsetneq S\cap C_{v}$).
Note that if all hyperrectangles in $\S$ have the same weight, our adjusted definition coincides
with our definition from the unweighted case. Similarly as before, in case that there are two
hypercubes $S,S'\in\S_{v}$ with $S\cap C_{v}=S'\cap C_{v}$ and $w_{S}=w_{S'}$, we break ties in an
arbitrary fixed way.

%Also, as before, for each vertex $v\in V'$
%and each maximally face-covering hypercube $S\in\S_{v}$ we introduce
%a corresponding set $(v,S)$ in $\hat{\S}$ that contains all points
%$p\in P$ such that there is a projection $p'$ of $p$ such that
%$p'\in S\cap C_{v}$.

\al{We can still bound the number of sets that each point $p\in P$ is contained in. In fact, the number of sets containing a point may blow up by a factor of $O(\log W)$ for all the weight classes. }
}
\fi
\begin{restatable}{lemma}{pointfewsetsweighted} \label{lem:point-few-sets-weighted}
Each point $p\in P$ is contained in at most $O_d(\log^{2d} N)\log W$ sets in $\hat{\S}$.
\end{restatable}
Hence, we define $f:=O_d(\log^{2d} N)\log W$. Similarly as in \Cref{lem:cheap-solution} we can show
that there exists a solution with weight at most $\OPT\cdot O_d(\log^{2d-1} N)$.  Also, like in
\Cref{lem:update-fast}, we can update our solution in time $O(f\log^2 (Wn))=O_d(\log^{2d} N)\log^2(W
n)\log W$.
\begin{restatable}{theorem}{WtDynSetCov} \label{thm:WtDynSetCov}
    There is an $O_d(\log^{4d-1}m)\log W$-approximate dynamic algorithm for weighted geometric set cover for $d$-dimensional
    hyperrectangles with worst-case update time of \\$O_d(\log^{2d} m)\log^3(W m)$ when a point is
    added or deleted.%\alr{Should we not add approximation ratio here?}
\end{restatable}
The details of this subsection can be found in \Cref{subsec:dynwtsetcover_2}




% \iffalse{
% We present our dynamic $O_d(\log^{4d-2} m)\log W$-approximation algorithm for set cover for $d$-dimensional
% hyperrectangles. We first present it for the unweighted case and explain later how to extend it to
% the weighted case.

% First, we show that we can reduce our problem to hypercubes by increasing the dimension by a factor
% of~2.
% \begin{lemma}\label{lem:reduce-to-cubes}
% Let $d'\in\N$. Suppose that there is
% a dynamic $\alpha$-approximation algorithm for set cover for $2d'$-dimensional
% hypercubes contained in $[0,N)^{2d'}$ with initialization time $f(m,N)$
% and update time $g(n,m,N)$ for functions $f,g$. Then there is a
% dynamic $\alpha$-approximation algorithm for set cover for $d'$-dimensional
% hyperrectangles contained in $[0,N)^{d'}$ with initialization time
% $O(md'\log m)+f(m,m)$ and update time $g(n,m,m)$.
% \end{lemma}

% For the remainder of this section suppose that we are given a set of $d$-dimensional hypercubes $\S$
% contained in $[0,N)^{d}$ for some constant $d\in\N$. Assume w.l.o.g.~that $N$ is a power of 2. Also,
% suppose that $P\subseteq[0,N)^{d}$ is the current set of points that we want to cover. We will first
% define a $d$-dimensional generalization of the two-dimensional quad-tree that we have used in
% \Cref{sec:Set-cover-squares}. Then, we will argue why a straight-forward generalization of
% our algorithm from the two-dimensional case does not work if $d>2$. Further, we will extend our
% quad-tree and define an auxiliary instance of (not necessarily geometric) set cover in which each
% element is contained in at most $O_d(\log^{2d-1}N)$ sets. We will maintain an $O(\log
% n)$-approximate solution for it (using the data structures in Bhattacharya et al.~\cite{bhattacharya2021dynamic} in $O_d(\log^{2d-1} N\cdot \log^2 n)$
% worst case update time. Finally, we will argue that such a solution yields an $O_d(\log^{2d-1}N)$-approximate
% solution for our instance of geometric set cover for hyperrectangles.

% \paragraph*{Quad-tree.}

% Like in \Cref{sec:Set-cover-squares}, we define a quad-tree $T=(V,E)$ that will guide our
% algorithm. The definition of $T$ is the $d$-dimensional analog to the definition of $T$ in
% \Cref{sec:Set-cover-squares}.  Each vertex $v\in V$ corresponds to a cell
% $C_{v}\subseteq[0,N)^{d}$.  The root $r$ corresponds to $C_{r}:=[0,N)^{d}$. A vertex $v\in V$ is a
% leaf of $T$ if it corresponds to a hypercube $C_{v}$ in which each edge has length 1. Otherwise, we
% define that $v$ has $2^{d}$ children $v_{1},\dots,v_{2^{d}}$ corresponding to the $2^{d}$ hypercubes
% we obtain if we partition $C_{v}$ into hypercubes whose edges have half the length of the edges of
% $C_{v}$. Formally, assume that
% $C_{v}=[x_{1}^{(1)},x_{2}^{(1)})\times\dots\times[x_{1}^{(d)},x_{2}^{(d)})$.  Each child $v_{i}$
% corresponds to a cell $C_{v_{i}}$ such that in each dimension $j\in[d]$ the projection of cell
% $C_{v_{i}}$ to the $j$-axis is $[x_{1}^{(j)},x_{\mathrm{mid}}^{(j)})$ or
% $[x_{\mathrm{mid}}^{(j)},x_{2}^{(j)})$ with $x_{\mathrm{mid}}^{(j)}:=(x_{2}^{(j)}-x_{1}^{(j)})/2$,
% and there is one child $v_{i}$ for each of these $2^{d}$ combinations. For each $v\in V$, let
% $P_{v}:=C_{v}\cap P$ and let $\S_{v}$ denote the squares in $\S$ that intersect $C_{v}$. Note that
% each point $p\in[0,N)^{d}$ is contained in $O(\log N)$ cells $C_{v}$ and sets $P_{v}$.

% Unfortunately, our approach from \Cref{sec:Set-cover-squares} no longer works if $d>2$.
% Suppose that $d=3$. A natural generalization would be to iterate through the vertices of $T$, and
% for each vertex $v$ to consider the faces of $C_{v}$. For each such face $F$ we could select a cube
% $S\in\S$ that contains $F$ and that has maximal intersection with $C_{v}$ among all such cubes. The
% problem is that maybe $C_{v}$ does not contain a vertex of a hypercube $S\in\OPT$, but still the
% selected cubes do not cover all remaining uncovered points of $P\cap C_{v}$. Then, we cannot charge
% the selected cubes to a cube in $\OPT$. Also, we cannot argue that we do not select any further
% cubes in the descendants of $v$. Another possible generalization would be to consider each edge
% $e$ of $C_{v}$ and try to select cubes in $\S$ containing $e$. However, unlike the approach via the
% faces of $C_{v}$, two such cubes $S,S'\in\S$ might be incomparable in the sense that $S\not\subseteq
% S'$ and $S'\not\subseteq S$ and we cannot afford to select all pairwise incomparable cubes.

% \paragraph*{Extended quad-tree.}

% Therefore, in contrast to \Cref{sec:Set-cover-squares}, we extend now $T$ to a tree
% $T'=(V',E')$ that we call the \emph{extended quad-tree}. For each node $v\in V$ we do the following.
% We add $d$ new children $v_{1},\dots,v_{d}$ to $v$. Each of them intuitively corresponds to reducing
% $C_{v}$ to a $(d-1)$-dimensional hypercube along one of the $d$ dimensions (see Figure~??).
% Formally, for each $i\in[d]$, we define that $C_{v_{i}}$ is the projection of $C_{v}$ to the
% dimensions $\{1,\dots,i-1,i+1,\dots,d\}$. We define a set of points $P_{v_{i}}$ and a set of hypercubes
% $\S_{v_{i}}$ corresponding to $v_{i}$. For each $p\in P_{v}$ we add to $P_{v_{i}}$ the point $p'$
% that we obtain if we project $p$ to the dimensions $\{1,\dots,i-1,i+1,\dots,d\}$.  To define
% $\S_{v_{i}}$, let $J$ denote the projection of $C_{v}$ to the $i$-th dimension (and hence $J$ is
% an interval). Consider each $S\in\S_{v}$ \emph{for which the projection to the $i$-th dimension
% equals }$J$. We add a hypercube $S'$ to $\S_{v_{i}}$ where $S'$ is the projection of $S$ to the
% dimensions $\{1,\dots,i-1,i+1,\dots,d\}$.  Observe that $S'\in\S_{v_{i}}$ contains a point $p'\in
% P_{v_{i}}$ if and only if the corresponding set $S$ contains the corresponding point $p$. This is
% why we required that the projection of $S$ to the $i$-th dimension equals $J$. In the above
% definition, we say that $p'$ is a \emph{projection} \emph{of} $p$ and that $S'$ is a
% \emph{projection of} $S$.

% Recursively, for each $d'=d-1,\dots,3$ we take each vertex $v'\in V'$ that corresponds to a
% $d'$-dimensional hypercube $C_{v'}$ and apply the same procedure to add $d'$ children to $v'$, each
% of them corresponding to $(d'-1)$-dimensional hypercubes. For convenience, we define that each point
% $p$ and each hypercube $S$ is a projection of itself.  Also, we make the notion of projections
% transitive: we say that if a point $p''$ (or a hypercube $S''$) is a projection of a point $p'$ (or
% a hypercube $S'$) which is a projection of a point $p$ (or a hypercube $S$), then $p''$ (or $S''$)
% is also a projection
% of $p$ (or $S$).

% \begin{lemma} \label{lem:few-sets}
%     For each point $p\in P$ there are at most $O_d(\log^{2d-1} N)$ vertices $v'\in V'$ such that a
%     projection $p'$ of $p$ is contained in $P_{v'}$.
% \end{lemma}

% \paragraph*{Auxiliary set cover instance.}

% Based on $T'$, we define now an instance of set cover for the set of elements $P$ and a family of
% sets $\hat{\S}$, where the sets in $\hat{\S}$ might be arbitrary subsets of $P$ (so not necessarily
% geometric objects like hypercubes etc.). We will maintain $O(\log n)$-approximate solutions to
% $(P,\hat{\S})$, and argue that they will yield $O_d(\log^{2d}m)$-approximate solutions to $(P,\S)$.

% For each each vertex $v\in V'$ and each hypercube $S\in\S_{v}$ we say that $S$ is \emph{maximally
% face-covering for }$v$ if there is a face $F$ of $C_{v}$ such that $F\subseteq S$ and $S\cap C_{v}$
% is maximal among all intersections $S'\cap C_{v}$ for all $S'\in\S_{v}$ containing $F$. In case that
% there are two hypercubes $S,S'\in\S_{v}$ with $S\cap C_{v}=S'\cap C_{v}$, we break ties in an
% arbitrary fixed way.

% For each vertex $v\in V'$ and each maximally face-covering hypercube $S\in\S_{v}$ we introduce a
% corresponding set in $\hat{\S}$; we denote this set by $(v,S)$. We define that $(v,S)$ contains all
% points $p\in P$ such that there is a projection $p'$ of $p$ such that $p'\in S\cap C_{v}$.


% \paragraph*{Properties of set cover instance.}

% We want to argue that our auxiliary set cover instance $(P,\hat{\S})$ has two important properties:
% \begin{enumerate}
%     \item let $f$ denote the maximum number of sets in $\hat{\S}$ that a point $p\in P$ is contained in;
%         we will show that $f=O_d(\log^{2d-1} N)$
%     \item any $\alpha$-approximate solution for $(P,\hat{\S})$ yields an $O_d(\log^{2d-1} N)$-approximate solution for our given instance $(P,\S)$ of geometric set cover.
% \end{enumerate}
% We start proving property 1. By \Cref{lem:few-sets}, for each point $p\in P$ there are only
% $O_d(\log^{2d-1} N)$ vertices $v\in V'$ whose sets $P_{v}$ contain $p$ or its projections. For each
% such $v$ there are at most $2d$ maximally face-covering hypercubes. Thus, $p$ is contained in at
% most $O_d(\log^{2d-1} N)$ sets in $\hat{\S}$.
% \begin{lemma}
%     \label{lem:point-few-sets}Each point $p\in P$ is contained in at
%     most $O_d(\log^{2d-1} N)$ sets in $\hat{\S}$.
% \end{lemma}
% %
% We want to prove now property 2. Let $\OPT\subseteq\S$ be an optimal solution to $(P,\S)$. Then we
% can construct a solution $\hat{\S}'\subseteq\hat{\S}$ to $(P,\hat{\S})$ by doing the following
% procedure for each $S\in\OPT$.  Intuitively, we identify all cells $C_{v}$ for which $S$ (or a
% projection $S'$ of $S$) are face-covering, and for those cells we select the set $(v,S)\in\hat{\S}$
% (or the set $(v,S')$, respectively). In this way, our selected sets from $\hat{\S}$ cover all points
% in $P$ that are contained in $S$. In fact, in order to achieve the latter, it is sufficient that we
% do this only for cells $C_{v}$ such that $S$ is \emph{not} face-covering for any cell $C_{v'}$ of an
% ancestor $v'\in V'$ of $v$ in $T'$. In this way, we select at most $O_d(\log^{2d-1} N)$ sets
% corresponding to $S$.

% \begin{lemma}\label{lem:cheap-solution}
%     There is a solution to $(P,\hat{\S})$ that contains at most $\OPT\cdot O_d(\log^{2d-1} N)$ sets.
% \end{lemma}
% \begin{proof}
%     Let $S\in\OPT$. We identify all vertices $v'\in V'$ such that
%     \begin{itemize}
%         \item $\S_{v'}$ contains a set $S'\in\S_{v'}$ that is a projection of $S$ and that is
%             face-covering for some face $F$ of $C_{v'}$ and
%         \item for no ancestor $v''\in V'$ of $v'$ the corresponding set $\S_{v''}$ contains a set
%             $S'\in\S_{v'}$ that is a projection of $S$ and that is face-covering for some face $F$
%             of $C_{v'}$.
%     \end{itemize}
%     For each such vertex $v'\in V'$, let $S''\in\S_{v'}$ denote the maximally face-covering
%     hypercube in $\S_{v'}$ that contains $F$.  Let $\hat{S}''$ be the set in $\hat{\S}$ that
%     correpsonds to $S''$.  We select $\hat{S}''$. We do this procedure for each $S\in\S'$.

%     It remains to prove that we select at most $\OPT\cdot O_d(\log^{2d-1} N)$ sets. ... todo ...
% \end{proof}

% On the other hand, we want to argue that we can turn any solution $\hat{\A}\subseteq\hat{\S}$ to
% $(P,\hat{\S})$ into a solution to $(P,\S)$. This is simple: for each set $(v,S)\in\hat{\A}$ we
% identify the hypercube $\tilde{S}\in\S$ such that $S$ is a projection of $\tilde{S}$, and we select
% $\tilde{S}$ for our solution to $(P,\S)$.

% \begin{lemma}
%     \label{lem:corresponding-solution}For any solution $\hat{\A}\subseteq\hat{\S}$
%     to $(P,\hat{\S})$ there is a solution $\A$ to $(P,\S)$ with $|\A|\le|\hat{\A}|$.
%     For each $(v,S)\in\hat{\A}$ there is a corresponding hypercube $\tilde{S}\in\A$
%     and given $(v,S)$, we can identify $\tilde{S}$ in time $O(1)$.
% \end{lemma}

% \paragraph*{Dynamic algorithm.}

% We maintain an approximate solution $\hat{\A}\subseteq\hat{\S}$ to $(P,\hat{\S})$ dynamically using
% the data structure from Bhattacharya et al.~\cite{bhattacharya2021dynamic} for arbitrary instances of set cover. It guarantees an approximation
% ratio of $(1+\varepsilon)f$ and an amortized update time of $f\log^2 (Wn)/\varepsilon^3$ even for the weighted case for any $\varepsilon>0$. If a point $p$ is inserted to $P$ or removed from
% $P$, then by \Cref{lem:few-sets} we know that we need to update at most $O_d(\log^{2d-1} N)\log^2 (n)$ sets
% from $\hat{\S}$, i.e., insert $p$ all these sets or remove $p$ from all these sets.
% \begin{lemma}\label{lem:update-fast}
%     If a point $p$ is inserted into $P$ or removed from $P$, then we can update the sets $\hat{\S}$
%     in worst case time $O_d(\log^{2d-1} N)\log^2 (n)$.
% \end{lemma}
% After an update, our data structure for $(P,\hat{\S})$ might update its solution. Since its update
% time is $O_d(\log^{2d-1} N)\log^2 (n)$ in the worst case, we know that at most $O_d(\log^{2d-1} N)\log^2 (n)$ sets in the solution are changed. Using
% \Cref{lem:corresponding-solution} we can hence update our solution to $(P,\S)$ in time $O_d(\log^{2d-1} N)\log^2 (n)$.
% \begin{theorem}
%     There is a fully-dynamic algorithm for unweighted geometric set cover for $d$-dimensional hyperrectangles with a worst-case update time of $O_d(\log^{2d-1} N)\log^2 (n)$ when a point is added or deleted.
% \end{theorem}


% \subsection{\label{subsec:set-cover-weighted}Weighted Case}
% We extend our algorithm above now to the weighted case, where we assume
% that each given hyperrectangle $S\in\S$ has a weight $w_{S}\in[1,W]$
% for a fixed value $W$. First, we round the weights of the hyperrectangles.
% \begin{lemma}
% By losing a factor of 2 in the approximation ratio, we can assume
% that for each $S\in\S$ the weight $w_{S}$ is a power of 2.
% \end{lemma}
% Hence, our hyperrectangles have $O(\log W)$ different weights now.
% We build the extended quad-tree $T'$ like above. In the definition
% of the auxiliary instance $(P,\hat{\S})$, we slightly adjust the
% definition of maximally face-covering. We define for each each vertex
% $v\in V'$ and each hypercube $S\in\S_{v}$ that $S$ is \emph{maximally
% face-covering for }$v$ if there is a face $F$ of $C_{v}$ such that
% $F\subseteq S$ and $S\cap C_{v}$ is maximal among all intersections
% $S'\cap C_{v}$ for all $S'\in\S_{v}$ \emph{with $w_{S}=w_{S'}$
% }containing $F$. The reason for the adjustment is that maybe $w_{S}>w_{S'}$
% but $S'\cap C_{v}\subsetneq S\cap C_{v}$ and then $S$ does not dominate
% $S'$ (since $w_{S}>w_{S'}$) but on the other hand $S'$ does not
% dominate $S$ either (since $S'\cap C_{v}\subsetneq S\cap C_{v}$).
% Note that if all hyperrectangles in $\S$ have the same weight, our
% adjusted definition coincides with our definition from the unweighted
% case. Similarly as before, in case that there are two hypercubes $S,S'\in\S_{v}$
% with $S\cap C_{v}=S'\cap C_{v}$ and $w_{S}=w_{S'}$, we break ties
% in an arbitrary fixed way. Also, as before, for each vertex $v\in V'$
% and each maximally face-covering hypercube $S\in\S_{v}$ we introduce
% a corresponding set $(v,S)$ in $\hat{\S}$ that contains all points
% $p\in P$ such that there is a projection $p'$ of $p$ such that
% $p'\in S\cap C_{v}$.

% We can still bound the number of sets that each point $p\in P$ is
% contained in.
% \begin{lemma}
% \label{lem:point-few-sets-weighted}Each point $p\in P$ is contained
% in at most $O_d(\log^{2d-1} N)\log W$ sets in $\hat{\S}$.
% \end{lemma}
% Hence, we define $f:=O_d(\log^{2d-1} N)\log W$. Similarly as in \Cref{lem:cheap-solution}
% we can show that there exists a solution with weight at most $\OPT\cdot O_d(\log^{2d-1} N)$.
% Also, like in \Cref{lem:update-fast}, we can update our solution
% in time $O(f\log^2(Wn)/\varepsilon^3)=O_d(\log^{2d-1} N)\log^2 (Wn)\cdot\log W$ for some fixed constant $\varepsilon>0$.
% \begin{theorem}
% There is a fully-dynamic algorithm for weighted geometric set cover
% for $d$-dimensional hyperrectangles with worst case update time
% of $O_d(\log^{2d-1} m)\log^2 (Wm)\cdot\log W$ when a point is added or deleted.
% \end{theorem}
% }
% \fi

\section{Hitting Set for $d$-dimensional Hyperrectangles}
\label{sec:hit-set-hyperrectangles}

We adapt our algorithm \al{in \Cref{sec:set-cover-hyperrectangles}} to geometric hitting set for $d$-dimensional
hyperrectangles. Like above, we reduce the problem to $2d$-dimensional hypercubes
and define the extended quad-tree $T'=(V',E')$ based on them. Then,
we define an auxiliary instance of (not necessarily geometric)
hitting set $(\hat{P},\hat{\S})$. For each $S\in\S$ there is a corresponding
set $\hat{S}\in\hat{\S}$. For each node $v\in V'$ corresponding
to a $d'$-dimensional cell $C_{v}$, and for each point with weight~$w$
(out of $O(\log W)$ many weights), we add to $\hat{P}$  upto $2d'$ points from $P_{v}$ of weight $w$ that are closest to
one of the $2d'$ facets of $C_{v}$. Intuitively, those are the ``most
useful'' points of weight $w$ to select if we want to hit a set
$S\in\S_{v}$ that contains a facet of $C_{v}$. %\alr{Need to change this to somehow reference $V_S'$. Current definition is incorrect for the new set system}
Essentially, we define
that a set $\hat{S}\in\hat{\S}$ (corresponding to a hypercube $S\in\S$)
contains a point $\hat{p}\in\hat{P}$ (corresponding to a point $p\in P$)
by identifying only certain nodes $v\in V'$ such that $S$ contains a facet of $C_{v}$,
$p$ hits $S$ inside $C_{v}$, but there is no ancestor $v'\in V'$
of $v$ such that $S$ contains a facet of $C_{v'}$. In this way,
we can guarantee that the frequency of the resulting instance $(\hat{P},\hat{\S})$
(i.e., the maximum number of points in $\hat{P}$ that hit any set
in $\hat{\S}$) is bounded by $O_d(\log^{2d-1}N)\log W$, and that there is a solution
with total cost of at most $\OPT\cdot O_{d}(\log^{2d}N)$. Similarly
as above, we use our dynamic algorithm for general set cover for the dual set system of  $(\hat{P},\hat{\S})$. That is, we now take the points in $\hat{P}$ to be the sets and the sets in $\hat{\S}$ to be the points to be covered.

\begin{restatable}{theorem}{WtDynHitSet}
\label{thm:WtDynHitSet}
There is an $O_d(\log^{4d-1}n)\log W$-approximate dynamic algorithm for weighted hitting
set for $d$-dimensional hyperrectangles with worst-case update time
of $O_{d}(\log^{2d-1}n)\log^{3}(Wn)$ when a hyperrectangle
is added or deleted.%\alr{Should we not add approximation ratio here?} 
\end{restatable}
All the details of this section can be found in \Cref{sec:hypercube_hs}.
}
\fi
\end{document}
