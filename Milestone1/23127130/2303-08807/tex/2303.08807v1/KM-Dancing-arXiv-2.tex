\documentclass[11pt]{amsart} 
                
              
\usepackage[marginratio=1:1,height=660pt,width=480pt,tmargin=70pt]{geometry}

\usepackage{amssymb,latexsym,amsmath,amsthm,amscd}
\usepackage{mathrsfs}   
\usepackage{fancyhdr}
 \usepackage{ifthen} 
 \usepackage{xstring}
 
  
  
  
\pagestyle{fancy} 
\fancyhf{}
\renewcommand{\headrulewidth}{0pt}

\usepackage{etoolbox}
 
%----------------------------interior product

% \DeclareFontFamily{U}{MnSymbolC}{}
% \DeclareSymbolFont{MnSyC}{U}{MnSymbolC}{m}{n}
% \DeclareFontShape{U}{MnSymbolC}{m}{n}{
%     <-6>  MnSymbolC5
%    <6-7>  MnSymbolC6
%    <7-8>  MnSymbolC7
%    <8-9>  MnSymbolC8
%    <9-10> MnSymbolC9
%   <10-12> MnSymbolC10
%   <12->   MnSymbolC12}{}
% \DeclareMathSymbol{\im}{\mathbin}{MnSyC}{'270}

\newcommand{\im}{\lrcorner\,}
%----------------------------------------------


\appto\appendix{\addtocontents{toc}{\protect\setcounter{tocdepth}{1}}}
\appto\listoffigures{\addtocontents{lof}{\protect\setcounter{tocdepth}{1}}}
\appto\listoftables{\addtocontents{lot}{\protect\setcounter{tocdepth}{1}}}
\theoremstyle{plain}
\newtheorem{theorem}{Theorem}[section]
\newtheorem{corollary}[theorem]{Corollary}
\newtheorem*{main}{Main~Theorem}
\newtheorem{lemma}[theorem]{Lemma}
\newtheorem{proposition}[theorem]{Proposition}
\newtheorem{definition}[theorem]{Definition}


  
\theoremstyle{remark}
\newtheorem*{assumption}{Assumptions}
\newtheorem{quesionn}[theorem]{Question}

\newtheorem{remark}[theorem]{Remark}

 
\newtheorem{example}[theorem]{Example}
\newtheorem*{example*}{Example}
\newtheorem{notation}[theorem]{Notation}


\numberwithin{equation}{section}

\renewcommand{\baselinestretch}{1.}

\usepackage{enumitem,array,rotating}  
\usepackage{color}
\usepackage{xcolor}
\definecolor{carmine}{rgb}{0.59, 0.0, 0.09}
\definecolor{mediumpersianblue}{rgb}{0.0, 0.4, 0.65}
\definecolor{persianplum}{rgb}{0.44, 0.11, 0.11}
\usepackage[colorlinks=true,
            linkcolor=persianplum, 
            urlcolor=olive,
            citecolor=mediumpersianblue, 
            backref=page]{hyperref}
\usepackage{amsmath}
\usepackage{amsfonts}
\usepackage{amssymb}
 
\urlstyle{same}
 \def\mathbi#1{\textbf{\em #1}}


%-----------mathserif---------
\newcommand{\sA}{\mathsf{A}}
\newcommand{\sB}{\mathsf{B}}
\newcommand{\sM}{\mathsf{M}}
\newcommand{\sN}{\mathsf{N}}
\newcommand{\sQ}{\mathsf{Q}}
\newcommand{\sh}{\mathsf{h}}
\newcommand{\sq}{\mathsf{q}}
\newcommand{\sP}{\mathsf{P}}
\newcommand{\sF}{\mathsf{F}}
\newcommand{\sR}{\mathsf{R}}
\newcommand{\sC}{\mathsf{C}}
\newcommand{\sE}{\mathsf{E}}
\newcommand{\sK}{\mathsf{K}}
\newcommand{\sL}{\mathsf{L}}
\newcommand{\sW}{\mathsf{W}}
\newcommand{\sV}{\mathsf{V}}
\newcommand{\sx}{\mathsf{x}}
\newcommand{\sbb}{\mathsf{b}}
\newcommand{\scc}{\mathsf{c}}
\newcommand{\sw}{\mathsf{w}}
\newcommand{\sff}{\mathsf{f}}


%---------mathcal---------------
\newcommand{\cC}{\mathcal{C}}
\newcommand{\cE}{\mathcal{E}}
\newcommand{\cG}{\mathcal{G}}
\newcommand{\cH}{\mathcal{H}}
\newcommand{\cV}{\mathcal{V}}
\newcommand{\cD}{\mathcal{D}}
\newcommand{\cX}{\mathcal{X}}
\newcommand{\cS}{\mathcal{S}}
\newcommand{\cP}{\mathcal{P}}
\newcommand{\cF}{\mathcal{F}}
\newcommand{\cJ}{\mathcal{J}}
\newcommand{\cW}{\mathcal{W}}
\newcommand{\cT}{\mathcal{T}}
\newcommand{\cQ}{\mathcal{Q}}
\newcommand{\cR}{\mathcal{R}}
\newcommand{\cI}{\mathcal{I}}
\newcommand{\cN}{\mathcal{N}}
\newcommand{\cK}{\mathcal{K}}



\newcommand{\X}{\mathcal{X}}
\newcommand{\D}{\mathcal{D}}
%-----------------boldsymbol
\newcommand{\balpha}{\boldsymbol\alpha}
\newcommand{\bbeta}{\boldsymbol\beta}

%---------mathscr---------
\newcommand{\scX}{\mathscr{X}}
\newcommand{\scV}{\mathscr{V}}
\newcommand{\scD}{\mathscr{D}}
\newcommand{\scE}{\mathscr{E}}
\newcommand{\scK}{\mathscr{K}}
\newcommand{\scB}{\mathscr{B}}
\newcommand{\scC}{\mathscr{C}}
\newcommand{\scH}{\mathscr{H}}

%--------mathbb------------


%-------------mathfrak---------
\newcommand{\fo}{\mathfrak{o}}
\newcommand{\fg}{\mathfrak{g}}
\newcommand{\fp}{\mathfrak{p}}
\newcommand{\ff}{\mathfrak{f}}
\newcommand{\fe}{\mathfrak{e}}
\newcommand{\ft}{\mathfrak{t}}
\newcommand{\fs}{\mathfrak{s}}
%-------------mathrm---------
\renewcommand{\i}{\mathrm{i}}
\renewcommand{\d}{\mathrm{d}}
\newcommand{\const}{\mathrm{const}}
\newcommand{\rT}{\mathrm{T}}
\newcommand{\rO}{\mathrm{O}}
\newcommand{\GLr}{\mathrm{GL}_{n+1}(\mathbb{R}}
\newcommand{\rG}{\mathrm{G}}

% --------mathbb------------
\newcommand{\R}{\mathbb{R}}
\newcommand{\RR}{\mathbb{R}}
\newcommand{\II}{\mathbb{I}}
\newcommand{\NN}{\mathbb{N}}
\newcommand{\CC}{\mathbb{C}}
\newcommand{\ZZ}{\mathbb{Z}}
\newcommand{\PP}{\mathbb{P}}

%---------mathbffont---------
\newcommand{\ba}{\mathbf{a}}
\newcommand{\bb}{\mathbf{b}}
\newcommand{\bc}{\mathbf{c}}
\newcommand{\be}{\mathbf{e}}
\newcommand{\bh}{\mathbf{h}}
\newcommand{\bg}{\mathbf{g}}
\newcommand{\bx}{\mathbf{x}}
\newcommand{\bu}{\mathbf{u}}
\newcommand{\bs}{\mathbf{s}}
\newcommand{\bw}{\mathbf{w}}


\newcommand{\bA}{\mathbf{A}}
\newcommand{\bB}{\mathbf{B}}
\newcommand{\bC}{\mathbf{C}}
\newcommand{\bD}{\mathbf{D}}
\newcommand{\bE}{\mathbf{E}}
\newcommand{\bF}{\mathbf{F}}
\newcommand{\bG}{\mathbf{G}}
\newcommand{\bH}{\mathbf{H}}
\newcommand{\bI}{\mathbf{I}}
\newcommand{\bJ}{\mathbf{J}}
\newcommand{\bK}{\mathbf{K}}
\newcommand{\bL}{\mathbf{L}}
\newcommand{\bM}{\mathbf{M}}
\newcommand{\bN}{\mathbf{N}}
\newcommand{\bO}{\mathbf{O}}
\newcommand{\bP}{\mathbf{P}}
\newcommand{\bQ}{\mathbf{Q}}
\newcommand{\bR}{\mathbf{R}}
\newcommand{\bS}{\mathbf{S}}
\newcommand{\bT}{\mathbf{T}}
\newcommand{\bU}{\mathbf{U}}
\newcommand{\bV}{\mathbf{V}}
\newcommand{\bW}{\mathbf{W}}
\newcommand{\bX}{\mathbf{X}}
\newcommand{\bY}{\mathbf{Y}}
\newcommand{\bZ}{\mathbf{Z}}

\newcommand{\bq}{\mathbf{q}}


\newcommand{\tx}{\tilde x}
\newcommand{\ty}{\tilde y}
\newcommand{\tp}{\tilde p}

\newcommand{\w}{{\,{\wedge}\;}}
\newcommand{\uo}{\underline 1}
\newcommand{\ud}{\underline 2}
\newcommand{\ut}{\underline 3}
\newcommand{\uz}{\underline 0}
\newcommand{\exd}{\mathrm{d}}
\newcommand{\ts}{\textstyle}
\newcommand{\ve}{\varepsilon}

\newcommand{\spn}{\operatorname{span}}
\newcommand{\tr}{\operatorname{tr}}
\newcommand{\rk}{\operatorname{rk}}
\newcommand{\ad}{\operatorname{ad}}
\newcommand{\Id}{\operatorname{Id}}
\newcommand{\la}{\lambda}
\newcommand{\norm}[1]{\left\lVert #1 \right\rVert}
\newcommand{\sgn}{\mathrm{sign}}
\newcommand{\Vect}{\mathrm{Vect}}
\newcommand{\wsf}{\textsc{Wsf }}


\newcommand{\half}{\textstyle{\frac 12}}
\newcommand{\Dt}{\textstyle{\frac{\mathrm D}{\exd t}}} 
\newcommand{\Ker}{\mathrm{Ker}}
\renewcommand{\span}{\mathrm{span}}
\newcommand{\Ann}{\mathrm{Ann}}

%%%%%%%%%%%%%%%%%%%%%%%%  Section  #######################
\renewcommand{\sectionautorefname}{\S}
\makeatletter
\renewcommand*{\p@section}{\S\,}
\renewcommand*{\p@subsection}{\S\,}
\renewcommand*{\p@subsubsection}{\S\,}
 
\makeatother
%%%%%%%%%%%%%%%%%%%%%%%%%%%%%%%%%
\usepackage{float}

\newcommand{\note}[1]{
 \marginpar{\scriptsize\itshape\color{red}#1}}
%*************
%\usepackage[small,nohug]{diagrams}        

%\diagramstyle[labelstyle=\scriptstyle]

\fancyhead[CE]{ Kry\'nski and  Makhmali}
\fancyhead[CO]{Chains and dancing in the path geometry of surfaces}
\fancyhead[RO,RE]{\thepage }
 
 

 \begin{document}

\author{Wojciech Kry\'nski}\author{Omid Makhmali}

 \address{\newline Wojciech Kry\'nski\\\newline
   Institute of Mathematics, Polish Academy of Sciences, ul.~\'Sniadeckich 8, 00-656 Warszawa, Poland\\\newline
   \textit{Email address: }{\href{mailto:krynski@impan.pl}{\texttt{krynski@impan.pl}}}\\\newline\newline
 Omid Makhmali\\\newline
Center for Theoretical Physics, PAS, Al. Lotnik\'ow 32\slash46, 02-668 Warszawa, Poland \\\newline
    \textit{Email address: }{\href{mailto:omakhmali@cft.edu.pl}{\texttt{omakhmali@cft.edu.pl}.}}
 }


 

\title[]
          {Two constructions in the path geometry of surfaces:\\ chains and dancing} 
\date{\today}


\begin{abstract}
  Given a path geometry on a surface $M$, there are two constructions via which the 3-dimensional projectivized tangent bundle $\PP TM$ can be endowed with a (generalized) path geometry. One of these constructions is given by the well-known class of chains. The other construction is referred to as the dancing construction. In this article we provide necessary and sufficient conditions that determine whether a path geometry in dimension three arises via one of these two constructions.
\end{abstract}


\subjclass{Primary: 53A40, 53B15,  53C15; Secondary: 53A20, 53A55, 34A26, 34A55}
\keywords{path geometry, chains, dancing construction, Cartan connection, Cartan reduction, second order ODEs, freestyling}


\maketitle
  
\vspace{-.5 cm}

\setcounter{tocdepth}{2} 
\tableofcontents


 
\section{Introduction}
\label{sec:introduction}

Geometric structures on smooth manifolds can give rise to distinguished set of curves  whose behaviour are of interest in a variety problems. For example,  geodesics in (pseudo-)Riemannian geometry or in projective structures,  conformal geodesics in conformal geometry, null-geodesics in strictly pseudo-Riemannian conformal geometry, and chains in CR structures or, more generally, in  contact parabolic geometries are some of the most well-known cases of such distinguished curves.



In this article we are interested in a geometric structure that is referred to as a \emph{path geometry} and we restrict ourselves to  lowest dimensional cases which are path geometries on surfaces and 3-manifolds.  Roughly speaking, a classical path geometry is defined in terms of a set of paths on a manifold with the property that along each direction in the tangent space of each point of the manifold, there passes a unique path in that family that is tangent to that direction. For instance, geodesics of an affine connection on a manifold define  a path geometry. Path geometries are a generalization of projective structures in which the curves may not be geodesics of any affine connection or satisfy any variational property. More generally, they may not be defined along every direction in each tangent space and may only be defined for an open set of directions at each point.

We characterize two constructions via which two   3-dimensional path geometries are canonically associated  to a path geometry on a surface,  referred to as \emph{chains} and \emph{dancing} constructions. We give a set of invariant conditions for 3-dimensional path geometries which are satisfied if and only if the  3-dimensional path geometry arises from  one of these two constructions. Except for one condition in the dancing construction, all other conditions are computationally verifiable, i.e. if a 3-dimensional path geometry is presented either in terms of abstract structure equations or as a pair of second order ODEs, these conditions can be computed by manipulations that involve  roots of a binary quadric or a binary quartic, finding eigenspaces of $2\times 2$ matrices and differentiation.

\subsection{Outline of the article and main results}
\label{sec:outline-article-main}



In  \ref{sec:path-geom-review} we give a review of path geometry in dimension 2 and 3. The definition of (generalized) path geometries on a surface $M$ is given in terms of (an open set of) the 3-dimensional projectivized tangent bundle $\nu\colon\PP TM\to M$. 
Abstractly, a 2-dimensional path geometry is defined via a contact 3-manifolds $N$  with a choice of splitting $\scC=\scD_1\oplus\scD_2$ of the contact distribution $\scC\subset TN$ and, therefore, is denoted by a triple $(N,\scD_1,\scD_2)$. In sufficiently small open sets $U\subset N,$ one can identify $U$ with an open subset of  $\PP TM$ where $M$ is the leaf space of the integral curves of $\scD_2.$ We recall local realizability of path geometries in any dimension in terms of second order ODEs and then discuss the so-called twistor correspondence for path geometries in dimensions  2 and 3. Lastly, we recall the solution of the equivalence problem for these geometric structures. An important feature of 3-dimensional path geometries for us is that their fundamental invariants can be represented as a binary quadric $\bT,$ referred to as the torsion, and a binary quartic $\bC,$ referred to as the curvature.  As will be discussed in \ref{sec:path-geom-defin}, a generalized 3-dimensional path geometry on a 3-manifold $N$ is denoted by a triple $(Q,\scX,\scV)$ where $Q$ is 5-dimensional, $\scX,\scV\subset TQ$ have rank 1 and 2, respectively,  their Lie bracket spans $T Q$ everywhere,  $\scV$ is integrable, and the local 3-dimensional leaf space of the foliation  induced by $\scV$ on $Q$ is an open set of $N.$ 


Path geometry on surfaces, defined in terms of $(N,\scD_1,\scD_2)$ as above, is considered the lowest dimensional case of the so-called  \emph{integrable Legendrian contact structures} which have many similarities to non-degenerate CR structures. They are sometimes referred to as integrable Lagrangean structures or para-CR structures as they can be defined similar to non-degenerate CR structures where the complex structure on the contact distribution is replaced by a  \emph{para-complex structure}.  Following the construction of chains in CR geometry, as done by \cite{CM-CR}, such contact 3-manifolds are equipped with a distinguished set of unparametrized curves defined along directions that are transversal to the contact distribution and are referred to as chains, e.g. see \cite{CZ-CR}.  Consequently, the chains of a 2-dimensional path geometry $(N,\scD_1,\scD_2)$  define a (generalized) 3-dimensional path geometry on $N,$ i.e. a triple $(Q,\scX,\scV).$


In \ref{sec:an-overview-chains} we review the definition of chains and their properties in 2-dimensional  path geometries. In \ref{sec:path-geom-aris-1} we provide  our invariant characterization of chains in  Corollary \ref{cor:3d-path-geometries-chain-characterization} which involves the algebraic type of the binary quartic $\bC$ and the existence of a \emph{quasi-symplectic} 2-form on the corresponding 5-manifold $Q.$ It follows that $Q$ is an open subset of $\PP (TN\backslash\scC)$ where $\scC$ is a contact distribution on a 3-manifold $N$. Recall that a quasi-symplectic 2-form is the odd dimensional analogue of a symplectic 2-form, i.e. a closed 2-form of maximal rank. 
\newtheorem*{cor1}{\bf Corollary \ref{cor:3d-path-geometries-chain-characterization}}
\begin{cor1}
  \textit{
      A 3D path geometry $(Q,\scX,\scV)$ arises from chains of a 2D path geometry $(N,\scD_1,\scD_2)$ if and only if 
  \begin{enumerate}
  \item The binary quartic $\bC$ has two distinct real roots of multiplicity 2. 
  \item The 5-manifold $Q$ has a compatible quasi-symplectic structure, i.e. there exists $\rho\in\Omega^2(Q)$ such that  $\rho\w\rho\neq 0$ and $\exd\rho=0,$ for which the paths are  characteristic curves, the fibers of $Q\to N$ are isotropic, where $N:=Q\slash\scV$, and $\rho$ is preserved by the fiber action of the principal bundle $\iota\colon\cG_{D_r}\hookrightarrow\cG$ in Proposition \ref{prop:type-D-curv-chains}.
  \item  The entries of the  binary quadric $\iota^*\bT$  in \eqref{eq:quadric-quartic} have no dependency on the fibers of  $Q\to N.$
  \end{enumerate}
  }
\end{cor1}
We express the quasi-symplectic 2-form $\rho$ in terms of the Cartan connection of the 3D path geometry on the reduced structure bundle $\cG_{D_r}\to Q.$ All 3 conditions above are in principle straightforward to check for any 3D path geometry expressed either as a pair of second order ODEs or in terms of abstract structure equations. 
Moreover, Corollary \ref{cor:3d-path-geometries-chain-characterization} identifies chains  as a sub-class of a  larger class of 3-dimensional path geometries introduced in Theorem \ref{thm:2d-path-geometries-generalized-chains}. We point out in Remark \ref{rmk:chains-3d-path-geometries-exclusive} that the characterization  of path geometry of chains for 2D path geometries is significantly different from the one for higher dimensional (para-)CR structures. 


In \ref{sec:an-overview-dancing} we describe the dancing construction which canonically associates a 3-dimensional path geometry to a 2-dimensional path geometry. Remark \ref{rmk:dancing} justifies the terminology of this construction by showing that these  3-dimensional path geometries,  via the twistorial construction, correspond to what is called the dancing construction  in \cite{Bor,D}. In \ref{sec:characterization-dancing} we give a characterization of dancing path geometries in dimension 3 in the form of Theorem \ref{thm1}. As a by-product of Theorem \ref{thm1}, in \ref{sec:liberal-dancing-as} we turn our attention to  a natural set of 3-dimensional path geometries  that contains dancing path geometries as a proper sub-class, which we  refer to  as freestyling path geometries.  In \ref{sec:path-geom-aris},  we provide a characterization of freestyling path geometries in Theorem \ref{thm2}. Subsequently, our necessary and sufficient conditions for  the dancing construction, as a special case of freestyling,  is the content of Corollary \ref{cor:3d-path-geometries-dancing}. Since by Proposition \ref{prop:torsionfree-freestyling} freestyling path geometries with vanishing torsion are uniquely determined as the path geometry of chains for the flat 2D path geometry, we consider dancing path geometries with non-zero torsion. 
\newtheorem*{cor2}{\bf Corollary \ref{cor:3d-path-geometries-dancing}}
\begin{cor2}
  \textit{
      A 3D path geometry $(Q,\scX,\scV)$ with non-zero torsion, i.e. $\bT\neq 0,$ arises from a dancing construction applied to a 2D path geometry $(N,\scD_1,\scD_2)$ if and only if 
  \begin{enumerate}
  \item The binary quadric $\bT$  has two distinct real roots and the binary quartic $\bC$ has  at least two distinct real roots and no repeated roots.
  \item  Using a reduction $\iota\colon\cG_{T_{G_r}}\to\cG,$ characterized  in Proposition \ref{prop:torsion-curv-type-freestyling-reduction}, and the  splitting $\scV=\scV_1\oplus\scV_2$ induced by the two real roots of $\bT,$   four rank 2   Pfaffian systems $\iota^*\cI_1,\iota^*\cI_2,\iota^*\cI_3,\iota^*\cI_4$ on $\cG_{T_{G_r}}$ are integrable whose respective leaves project to $Q$ as integral manifolds of rank 3 distributions
    \[
        \langle\scX,\scV_2,[\scX,\scV_2]\rangle,\quad \langle\scX,\scV_1,[\scX,\scV_1]\rangle,\quad \langle\scV,[\scX,\scV_2]\rangle,\quad \langle\scV,[\scX,\scV_1]\rangle.
\]
  \item   Defining $\tilde T,\tilde M,M,T$ to be the local leaf spaces of $\cI_1,\cI_2,\cI_3,\cI_4,$ respectively,  
    they are equipped with 2D path geometries corresponding to $(N,\scD_1,\scD_2),$ $(N,\tilde\scD_1,\scD_2)$ and $(N,\scD_1,\tilde \scD_2)$  forming double fibrations
    \[M\leftarrow N\rightarrow T,\qquad M \leftarrow  N\rightarrow \tilde T,\qquad \tilde M\leftarrow   N\rightarrow T,\]
    respectively. The triple of 2D path geometries above, defined on $N,$ are equivalent. 
\end{enumerate}
  }
\end{cor2}
In Corollary \ref{cor:3d-path-geometries-dancing-chains} we show that the flat 2-dimensional path geometry is the only 2D path geometry where the chains define a freestyling which is, in fact, the original dancing construction defined in \cite{Bor}.
In Remark \ref{rmk:integrability-soln-space-freestyling} we comment on an interpretation of freestyling as certain intagrability conditions for the induced \emph{half-flat causal structure} on the 4-dimensional space of paths.  Remark \ref{rmk:higher-dimensional-dancing} addresses the dancing construction for higher dimensional integrable Legendrian contact structures.

In Appendix  we give the Bianchi identities for the fundamental invariants of 3D path geometries in \eqref{eq:W-A-curvature-torsion-Bianchies}, which will be used in the proof of Proposition \ref{prop:type-D-curv-chains}. Moreover, we give the structure equations for the reduced 3D path geometries arising from the dancing construction  in \eqref{eq:freestyling-streqs} which is discussed  in Proposition \ref{prop:torsion-curv-type-freestyling-reduction}.





\subsection{Conventions}
\label{sec:conventions}
Our consideration in this paper will be over smooth and real manifold.  Throughout the article we always consider path geometry in the generalized sense and therefore will not use the term ``generalized'' when talking about path geometries. When defining a leaf space of a foliation we always restrict to sufficiently small open sets where the leaf space is smooth and Hausdorff.  Consequently, given an integrable distribution $\scD$ on a manifold $N,$ by abuse of notation, we denote the leaf space of its induced foliation by $N\slash\scD.$

When dealing with differential forms, the algebraic ideal generated by 1-forms $\alpha^1,\cdots,\alpha^k$ is denoted as $\{\alpha^1,\cdots,\alpha^k\}.$ 
Given a principal bundle $\cG\to Q$ with respect to which the 1-forms $\alpha^0,\cdots,\alpha^n,\beta^1,\cdots,\beta^n$ are semi-basic,  we define the \emph{coframe derivatives} of a function $f\colon\cG\to \RR$ as 
\begin{equation*}
  f_{;i}=\tfrac{\partial}{\partial\alpha^i}\im\exd f,\quad   f_{;ij}=\tfrac{\partial}{\partial\alpha^j}\im\exd f_{;i},\quad   f_{;\underline{a}}=\tfrac{\partial}{\partial\beta^a}\im\exd f,\quad  f_{;\underline{ab}}=\tfrac{\partial}{\partial\beta^b}\im\exd f_{;\underline a},\quad  f_{;\underline{a}i}=\tfrac{\partial}{\partial\alpha^i}\im\exd f_{;\underline a},\quad f_{;i\underline{a}}=\tfrac{\partial}{\partial\beta^a}\im\exd f_{;i} 
\end{equation*}
and similarly for higher orders, where $0\leq i,j\leq n$ and $1\leq a,b\leq n.$ Note that in case we reduce the structure bundle of a geometric structure to a proper principal sub-bundle, by abuse of notation, we suppress the pull-back and use the same notation as above for the coframe derivatives on the reduced bundle. 

Lastly, given two distributions $\scD_1$ and $\scD_2,$ we denote by $[\scD_1,\scD_2]$ the distribution whose sheaf of sections is  $\Gamma([\scD_1,\scD_2])=\Gamma(\scD_1)+\Gamma(\scD_2)+[\Gamma(\scD_1),\Gamma(\scD_2)].$ The sheaf of sections of  $\bigwedge^k(T^* M)$ is denoted by $\Omega^k(M).$
\section{A review of path geometries}  
\label{sec:path-geom-review} 
In this section we  recall some of the well-known facts about path geometries in dimensions two and three including the definition of  path geometries, the notion of duality between path geometries of surfaces and a solution for their equivalence problem.

\subsection{Definition and realizability as systems of 2nd order ODEs}
\label{sec:path-geom-defin}


As mentioned previously, a path geometry  on an $(n+1)$-dimensional manifold $M$ is classically defined as a $2n$-parameter family of paths with the property that along each direction at any point of $M,$ there passes a unique path in that family. Consequently, the natural lift of the paths of a path geometry to the projectivized tangent bundle, $\PP TM,$ results in a foliation by curves which are transversal to the fibers $\PP T_xM.$ It is well-known that a path geometry on an $(n+1)$-dimensional manifold can be locally defined in terms of  a system of $n$ second order ODEs 
\begin{equation}\label{systemODE}
   (z^i)''=F^i(t,z,z'),\quad t\in\R,\ \ z=(z^1,\dots,z^n),\quad 1\leq i\leq n,
\end{equation}
defined up to \emph{point transformations},  i.e.  
\[t\mapsto \tilde t=\tilde t(t,z^1,\ldots,z^{n}),\quad z^i\mapsto \tilde z^i=\tilde z^i(t,z^1,\ldots,z^{n}),\quad 1\leq i\leq n.\]
 Given a path geometry, its straightforward to see how it defines a  system of $n$ ODEs: let $(z^0,\dots,z^n)$ be local coordinates on $V\subset M$. In a sufficiently small open set $U\subset \PP TM,$ the family of paths can be parametrized as $s\mapsto \gamma(s)=(z^0(s),\dots,z^n(s))$ for $s\in (a,b)\subset\RR.$ Each path is determined by the value of  $\gamma(s),$ and $\gamma'(s)$ at $s=s_0\in U,$ thus, taking another derivative, a system of $n+1$ 2nd order ODEs of the form $\gamma''= G(\gamma,\gamma')$ is obtained for a function   $G\colon \RR^{2n+2}\to\RR^{n+1}.$ If $U\subset \PP TM$ is sufficiently small, without loss of generality one can assume $\frac{\exd z^0}{\exd s}\neq 0$ in $U.$  Since the paths are given up to reparametrization, one is able to eliminate $s$ from the system $\gamma''=G(\gamma,\gamma').$ As a result, one arrives at the system of ODEs \eqref{systemODE} where $t:=z^0.$


Conversely, starting with an  equivalence class of system of $n$ 2nd order  ODEs up to point transformations  \eqref{systemODE}, the system defines  a codimension $n$ submanifold $\cE\subset J^2(\RR,\RR^n)$ of the second  jet space.  By pulling-back  the natural contact system on $J^2(\RR,\RR^n)$ to $\cE$ one can identify $\cE$ with $J^1(\RR,\RR^n)$ which is additionally foliated by contact curves, i.e. the solution curves of the ODE system.  Locally $J^1(\RR,\RR^n)\to J^0(\RR,\RR^n)\cong\RR^{n+1}$ can be identified as an open subset of the projectivized tangent bundle $\PP TM\to M$ for an $(n+1)$-dimensional manifold $M$ and the solution curves project to a $2n$-parameter family of paths on $M.$ However, for arbitrary ODE systems it may happen that more than a single path are tangent to certain directions at some points or no path is tangent to some directions at some points of $M.$ 



Similarly,  there are many instances of  path geometries that do not fit the classical description since the paths are only defined along  an open set of directions. Thus, in order to study  path geometries one is led to work with a generalized notion of such structures as defined below, which is sometimes referred to as \emph{generalized path geometry}. However, in this article since path geometries for us are always defined in this generalized sense we will not use the terms generalized.
\begin{definition}
\label{def:generalized-path-geom}
  An $(n+1)$-dimensional   path geometry   is a triple $(Q,\scX,\scV)$   where $Q$ is a $(2n+1)$-dimensional manifold  equipped with a  pair of integrable and transversal distributions $(\scX,\scV)$ of rank 1 and $n,$ respectively,  that satisfy $[\scX,\scV]=T Q$. The $(n+1)$-dimensional local leaf space of the foliation induced by $\scV$ is said to be equipped with a path geometry given by triple $(Q,\scX,\scV)$. 
\end{definition}
 The rank $(n+1)$  distribution $\scC$ spanned by $\scX$ and $\scV$  induces a    \emph{multi-contact} structure on $Q,$ i.e., one can write  $\scC=\ker\{\alpha^1,\ldots,\alpha^n\}$ for some 1-forms $\alpha^0,\alpha^a,\beta^b,$ 
 such that  $\scX=\ker\{\alpha^a,\beta^b\}_{a,b=1}^n,$ $\scV=\ker\{\alpha^0,\ldots,\alpha^n\}$  and 
\[\exd \alpha^i\equiv \alpha^0\w\beta^i\ \ \mathrm{mod\ \ }\{\alpha^1,\ldots,\alpha^n\},\]
for all $1\leq i\leq n$.  Path geometry of surfaces corresponds to  $n=1$ in which case $\scC$ is a contact distribution on a 3-manifold $Q.$ In that sense the path geometry in dimension 2 has certain unique properties that do not appear in higher dimensions.  

One can easily check that Definition \ref{def:generalized-path-geom} is satisfied for a classical path geometry on an $(n+1)$-dimensional manifold by letting $Q=\PP TM$ and $\scV$ be the vertical tangent space of the fibration $\PP TM\to M$  and $\scX$ be the line field tangent to the natural lift of the path on $M$ to $\PP TM.$  Explicitly, for a system \eqref{systemODE} one has
\[
\scX=\spn\left\{\tfrac{\mathrm{D}}{\exd t}\right\}\quad \text{where}\quad \tfrac{\mathrm{D}}{\exd t}:=\partial_t+\sum_{i=1}^n p^i\partial_{z^i}+\sum_{i=1}^n F^i\partial_{p^i},
\]
and
\[
\scV=\spn\{\partial_{p^1},\ldots,\partial_{p^n}\},
\]
where $p^i$'s are fiber coordinates on the tangent bundle $TM$. The vector field $\tfrac{\mathrm{D}}{\exd t}$ spanning $\scX$ is often called the total derivative vector field.

As was mentioned before, the converse is not necessarily true, i.e. not all geometries arising from Definition \ref{def:generalized-path-geom} are classical path geometries. However, restricting to  sufficiently small open sets   $U\subset Q$ in Definition \ref{def:generalized-path-geom},  $U$ can be realized as an open set of $\PP TM$ for the $(n+1)$-dimensional manifold $M$ which is the the leaf space of $\scV.$ Consequently,  $\scX$ foliates $U\subset\PP TM$ by curves that are transversal to the fibers of $\PP TM\rightarrow M.$ We refer the reader to \cite[Section 2]{Bryant-ProjFlat} for more detail on the path geometry on surfaces, most of which hold in higher dimensions.



\subsection{Duality in path geometry of surfaces}\label{sec:2D-review}
Let $(Q,\scX,\scV)$ denote a path geometry on  the surface $M:=Q\slash\scV$.  In this case both $\scX$ and $\scV$ are of rank 1, and, as we shall explain below, the roles of the two line bundles  can be interchanged to define another path geometry, i.e. a dual path geometry. For this reason, in this case we shall denote $\scD_1:=\scX,$ $\scD_2:=\scV,$ and $N:=Q$ to distinguish path geometry on surfaces. Define
\[
M=N/\scD_2
\]
to be the leaf space of the foliation induced by $\scD_1.$ The paths of the induced path geometry on $M$ are projections of integral curves of $\scD_1$ via the quotient map $\pi_M\colon N\to M$. As a result,  the paths are parameterized by points of the 2-dimensional  leaf space 
\[
T=N/\scD_1, 
\]
defined by the quotient map $\pi_T\colon N\to T$. The surface $T$ is usually referred to as the twistor space (or the mini-twistor space) of the path geometry on $M$. Thus, one has the double fibration 
\[
M\longleftarrow N\longrightarrow T.
\]
Similary a 2D path geometry is induced on $T$ for which the paths are the projection of the integral curves of $\scD_2$. As a result, we get a correspondence between points in $M$ and paths in $T$, and also between  paths in $M$ and points in $T$. Furthermore, any point in $N$ is uniquely determined as an intersection point of two integral curves, one tangent to $\scD_1$ and the other tangent to $\scD_2$. As a result,  any point of $N$ is determined by a path in $M$ with a fixed point on it, or, equivalently, by a path in $T$ with a fixed point on it. This extends the flat model of the 2-dimensional path geometry where $M$ and $T$ are given by 2-dimensional projective spaces $\RR\PP^2$ and $(\RR\PP^2)^*$ and $N$ is the 3-dimensional space of flags of a line inside a plane in $\RR^3$ which is equivalent to the space of marked projective lines in $\PP^2.$ The 3-manifold $N$ is usually referred to as the correspondence space and can be viewed via the inclusion
\[
N\subset M\times T.
\]
On the level of ODEs the correspondence gives rise to the Cartan duality for second order ODEs, which from the viewpoint of the geometry on $N$ simply interchange the role of $\scD_1$ and $\scD_2$. Namely, picking coordinates $(t,z)$ on $M$ and $(\tilde t, \tilde z)$ on $T$ one gets a second order ODE on $M$
\begin{equation}\label{eq:sode}
\frac{\exd^2z}{\exd t^2}=F\left(t,z,\frac{\exd z}{\exd t}\right)
\end{equation}
whose solutions are paths on $M$, and, similarly, a second order ODE on $T$
\begin{equation}\label{eq:dual}
\frac{\exd^2\tilde z}{\exd\tilde t^2}=\tilde F\left(\tilde t,\tilde z,\frac{\exd\tilde z}{\exd\tilde t}\right),
\end{equation}
whose solutions are paths on $T$. The two scalar second order ODEs \eqref{eq:sode} and \eqref{eq:dual}, or more precisely, their point equivalence classes, and the corresponding path geometries,  are called dual to each other. Notice that $T$ is the solutions space for \eqref{eq:sode} and $M$ is the solution space for \eqref{eq:dual}.

More generally one can write solutions in the following implicit form
\[
\Phi(t,z,\tilde t,\tilde z)=0
\]
where $\Phi$ is a function on $M\times T$. Fixing $\tilde t=\const$, $\tilde z=\const$ we get a curve in $M$, and, conversely, fixing $t=\const$, $z=\const$ we get a curve in $T$. Then, the aforementioned inclusion $N\subset M\times T$ is realized as
\[
N=\{(t,z,\tilde t, \tilde z)\in M\times T \ |\ \Phi(t,z,\tilde t, \tilde z)=0\}.
\]


Using the notion of duality of 2-dimensional path geometries,  we give the following definition.
\begin{definition}\label{def:proj-co-proj}
  A path geometry on $M$ is called a \emph{projective structure} if its paths coincide with the geodesics of  a linear connection $\nabla$ on $M$ as unparametrized curves. A path geometry on $M$ is called a \emph{co-projective structure} if its dual path geometry  is a projective structure.  
\end{definition}
\subsection{Solution of the equivalence problem}\label{sec:solut-equiv-probl}
Two path geometries $(Q_i,\scX_i,\scV_i)$ are called equivalent if there exists a diffeomorphism $f\colon Q_1\to Q_2$ such that $f_*(\scX_1)=\scX_2$ and $f_*(\scV_1)=\scV_2.$ We point out that a 2-dimensional path geometry is not in general  equivalent to its dual.

To determine when two path geometries are locally equivalent, we provide a solution to the equivalence problem of path geometries using the notion of a Cartan geometry and Cartan connection as defined below.  
\begin{definition}
  Let $G$ be a Lie group and $P\subset G$ a Lie subgroup  with Lie algebras $\fg$ and $\fp.$ A Cartan geometry on $Q,$ denoted as $(\cG\to Q,\psi),$ is a right principal $P$-bundle $\tau\colon\cG\to Q$ equipped with a Cartan connection $\psi\in\Omega^1(\cG,\fg),$ i.e. a $\fg$-valued 1-form on $\cG$ satisfying
  \begin{enumerate}
  \item $\psi$ is $P$-equivariant, i.e. $r_g^*\psi=\mathrm{Ad}_{g^{-1}}\circ\psi$ for all $g\in P.$
  \item $\psi$ maps fundamental vector fields to their generators, i.e. $\psi(\zeta_X)=X$ for any $X\in\fp.$
  \item $\psi$ is an isomorphism $\psi\colon T_u\cG\to \fg$ for all $u\in \cG.$
    The 2-form $\Psi\in\Omega^2(\cG,\fg)$ defined as
    \[\Psi(u,v)=\exd\psi(u,v)+[\psi(X),\psi(Y)]\quad \text{for\ \ }X,Y\in \Gamma(T\cG),\]
is called the Cartan curvature and is $P$-equivariant and semi-basic with respect to the fibration $\cG\to Q.$
  \end{enumerate}
\end{definition}
The following solution of the equivalence problem for path geometries is due to Cartan \cite{Cartan-Proj} for surfaces and due to Grossman and Fels  \cite{Grossman-Thesis,Fels} in higher dimensions. 
\begin{theorem}\label{thm:path-geom-cartan-conn}
Every path geometry $(Q^{2n+1},\scX,\scV),$  as in Definition \ref{def:generalized-path-geom},   defines Cartan geometry  $(\cG,\Sigma,\psi)$ of type $(\mathrm{SL}_{n+2}(\RR),P_{12})$ where $P_{12}\subset \mathrm{SL}_{n+2}(\RR)$ is the parabolic subgroup preserving the flag of a line and a 2-plane in $\RR^{n+3}.$ Assume that  the distributions $\scX$ and $\scV$  are the projection of the distributions  $\ker\{\alpha^i,\beta^i\}_{i=1}^n$ and $\ker\{\alpha^i\}_{i=0}^n$ respectively.   When $n\geq 2$, the Cartan connection and its curvature can be  expressed as
  \begin{equation}
  \label{eq:path-geom-cartan-conn-3D}  
  \psi=
  \def\arraystretch{1.3}
\begin{pmatrix}    
 -\psi^i_i-\psi^0_0 &\mu_0& \mu_j\\
\alpha^0 &\psi^0_0&\nu_j\\
\alpha^i&  \beta^i& \psi^i_j\\
\end{pmatrix}    
\quad \mathrm{and}\quad \Psi:=\exd\psi+\psi\w\psi=
  \def\arraystretch{1.3}
\begin{pmatrix}    
0 &M_0& M_j\\
0 &\Psi^0_0&V_j\\
0 &   B^i& \Psi^i_j\\
\end{pmatrix}    
    \end{equation}
where $1\leq i,j\leq n$ and 
\[B^i=T^i_j\alpha^0\w\alpha^j+\half T^i_{jk}\alpha^j\w\alpha^k,\qquad \mathrm{and}\quad \Psi^i_j=C^i_{jkl}\alpha^k\w\beta^l+\half T^i_{jkl}\alpha^k\w\alpha^l +T^i_{j0k}\alpha^0\w\alpha^k. \]
The fundamental invariants of a path geometry  are the \emph{torsion}, $\mathbf{T}=(T^i_j)_{1\leq i,j\leq n},$ and the \emph{curvature}, $\mathbf{C}=(C^i_{jkl})_{1\leq i,j,k,l\leq n}$, satisfying
\begin{equation}
  \label{eq:fels-invariants-path}
  T^i_i=0,\qquad C^i_{jkl}=C^i_{(jkl)},\qquad C^i_{ijk}=0.
\end{equation}
\end{theorem}
\begin{remark}\label{rmk:1-form-defined-on-G-or-Q}
  We point out that unlike in \ref{sec:path-geom-defin}, the 1-forms in Theorem \ref{thm:path-geom-cartan-conn} are defined on the principal bundle $\cG$ rather than the manifold $Q.$ By taking a section $s\colon Q\to\cG,$ the entries of  $s^*\psi$ become 1-forms on $Q.$ We will not make the distinction between 1-forms defined on $\cG$ or on $Q$ explicit when it is clear from the context.
  \end{remark}



If $(t,z^i)$ and $(t,z^i,p^i)$ are  the natural local coordinates for $\RR\times \RR^{n}$ and $J^1(\RR,\RR^n)$ then in terms of the system of ODEs \eqref{systemODE}, one obtains 
\begin{equation}
  \label{eq:fels-invariants-explicit}
  T^i_j=F^i_j-\textstyle{\frac{1}{2}\delta^i_jF^k_k},\qquad\qquad C^i_{jkl}=F^i_{jkl}-\textstyle{\frac{3}{4}}F^r_{r(jk}\delta^i_{l)}
\end{equation}
where $1\leq i,j,k,l\leq n$ and 
\begin{equation}
  \label{eq:fels-torsion-explicit-components}
  F^i_{j}=\textstyle{- F^i_{z^j}+\frac{1}{2}\Dt (F^i_{p^j})-\frac{1}{4} F^i_{p^k} F^k_{p^j},}\quad F^i_{jkl}=F^i_{p^jp^kp^l}  ,\quad \Dt=\partial_t+p^i\partial_{z^i}+F^i\partial_{p^i}.
\end{equation}

Cartan's solution of the equivalent problem for path geometries on surfaces is as follows.
\begin{theorem}\label{thm:2D-path-geome}
 Given a 2-dimensional path geometry  $(N,\scD_1,\scD_2)$ it defines a Cartan geometry $(\cP \to N,\phi)$ of type $(\mathrm{SL}(3,\RR),P_{12})$ where $P_{12}$ is the subgroup of upper diagonal matrices and
  \begin{equation}
  \label{eq:2D-path-geom-cartan-conn}  
  \phi=
  \def\arraystretch{1.3}
\begin{pmatrix}    
 -\phi^i_i & \theta_2 &\theta_0\\
 \omega^1 &\phi^1_1&\theta_1\\
\omega^0&  \omega^2& \phi^2_2\\
\end{pmatrix}
\end{equation}
The Cartan curvature of $\phi$ is given by
     \begin{equation}
            \label{eq:path-geom-curv}
  \begin{pmatrix}
    0 & T_1 \omega^0 \w\omega^1 & T_2\omega^0 \w\omega^1+C_2\omega^0\w\omega^2\\
    0 & 0 & C_1\omega^0\w\omega^2\\
0 & 0 & 0
  \end{pmatrix}
\end{equation}
for some functions $T_1,T_2,C_1,C_2$ on $\cP$ where $\bT=T_1$ and $\bC=C_1$ are the \emph{fundamental invariants}.
 \end{theorem}
 More explicitly, since 2D path geometries correspond to point equivalence classes of scalar 2nd order ODEs, given a scalar ODE
 \[z''=F(t,z,z'),\]
 one obtains
 \[T_1=\tfrac{\mathrm{D^2}}{\exd t^2}F_{z'z'}-4\tfrac{\mathrm{D}}{\exd t}F_{zz'}-F_{z'}\tfrac{\mathrm{D}}{\exd t}F_{z'z'}+4F_{z'}F_{zz'}-3F_zF_{z'z'}+6F_{zz},\quad C_1=F_{z'z'z'z'}\]



Here we are interested in path geometries on 3-dimensional manifolds, i.e.  $Q$ in Definition \ref{def:generalized-path-geom} is 5-dimensional, which correspond to point equivalence classes of pairs of 2nd order ODEs. The following proposition will be important for us whose proof is straightforward and will be skipped.






\begin{proposition}\label{prop:3D-path-geom}
In three-dimensional path geometries the fundamental invariants $\bT$ and $\bC,$ as $\mathrm{GL}_2(\RR)$-modules  can be represented as a  quadric and a quartic on $Q$, respectively, given by
\begin{equation}
  \label{eq:quadric-quartic}
  \begin{aligned}
 \bT&= s^*(A_0(\beta^1)^2+2A_1\beta^1\beta^2+A_2(\beta^2)^2)\otimes V\otimes E^{-2},\\
\bC&=s^*(W_0(\beta^1)^4+4W_1(\beta^1)^3(\beta^2)+6W_2(\beta^1)^2(\beta^2)^2+4W_3(\beta^1)(\beta^2)^3+W_4(\beta^2)^4)\otimes V\otimes E^{-1} \\
\end{aligned}
\end{equation}
where $s\colon Q\to\cG$ is a section, $\bT\in \mathrm{Sym}^2(\scV)\otimes\scV\otimes\scX,$ $\bC\in \mathrm{Sym}^4(\scV)\otimes\scV\otimes\scX,$ $E:=\frac{\partial}{\partial s^*\alpha^0},$  $V:=\frac{\partial}{\partial s^*\beta^1}\w\frac{\partial}{\partial s^*\beta^2},$ 
\[
\begin{gathered}
A_0=T^2_1,\quad A_1=T^2_2,\quad A_2=-T^1_2,\\
  W_0=C^2_{111},\quad W_1=C^2_{211},\quad W_2=C^2_{221},\quad W_3=C^2_{222},\quad W_4=-C^1_{222},
\end{gathered}
\]
and $\tfrac{\partial}{\partial s^*\alpha^i},\tfrac{\partial}{\partial s^*\beta^a}$ denote the vector fields dual to the coframing $s^*(\alpha^0,\alpha^1,\alpha^2,\beta^1,\beta^2).$
Moreover, $Q$ is equipped with a degenerate conformal structure given by  $[s^*h]$ where
\begin{equation}
  \label{eq:degenerate-conformal-str}
  h:=\alpha^1\beta^2-\alpha^2\beta^1\in \mathrm{Sym}^2(T^*\cG).
  \end{equation}
\end{proposition}
Using Cartan's method of equivalence, two path geometries $(Q_i,\scX,\scV)$ in any dimension are equivalent if and only if for their respective Cartan geometries $(\cG_i\to Q_i,\psi_i)$ there is a diffeomorphism $f\colon\cG_1\to\cG_2$ such that $f^*\psi_2=\psi_1.$


In Theorems \ref{thm:path-geom-cartan-conn} and \ref{thm:2D-path-geome}  if the fundamental invariants vanish, i.e. $\bC=\bT=0,$ then the Cartan curvature is zero and the Cartan connection coincides with the Maurer-Cartan forms on $\mathrm{SL}_{n+2}(\RR)$ and the path geometry is locally equivalent to the canonical one on $\RR\PP^{n+1}$ whose paths are projective lines.  If $\bC=0,$ then the path geometry defines a projective structure on the $(n+1)$-dimensional leaf space of $\scV,$ denoted by $M.$ If $\bT=0,$ then  the $2n$-dimensional leaf space of $\scX$ is endowed with a so-called $\beta$-integrable Segr\'e structure if $n\geq 2$ and a projective structure if $n=1$.

Note that when $n=2$ and $\bT=0,$ then the conformal class of the bilinear form defined by $h$ in Proposition \ref{prop:3D-path-geom} descends to a self-dual  conformal structure of neutral signature on the leaf space of $\scX,$ i.e. the 4-dimensional solution space of the corresponding pair of ODEs. In this case $\bC$ coincides with the quartic representation of the self-dual Weyl curvature  of the induced conformal structure.

Lastly we recall the notion of orientability for  path geometries on  surfaces, following \cite[Section 2]{Bryant-ProjFlat}. A path geometry on a surface  given by $(N,\scD_1,\scD_2)$ is said to be oriented if the leaves of $\scD_1$ and $\scD_2$ can be endowed with a continuous choice of orientation. This is equivalent to the existence of two non-vanishing vector fields $v_i\in\Gamma(\scD_i)$ on $N.$  A  1-form $\alpha$ is called $\scD_i$-positive if its pull-back to each leaf of $\scD_i$ is positive with respect to a  pre-assigned orientation. As a result, by the naturally induced co-orientation on the contact distribution due to the relation $\exd\omega^0\equiv\omega^1\w\omega^2$ mod $\{\omega^0\},$  having  $\scD_i$-positive 1-forms  $\omega^1$ and $\omega^2$ determines an orientation on $N.$ It is straightforward to see that an oriented path geometry determines an orientation on the leaf spaces of $\scD_1$ and $\scD_2.$  Similarly, one can define oriented path geometries $(Q,\scX,\scV)$ in higher dimensions by assigning a continuous choice of orientation to the leaves of $\scX$ and $\scV$ which imply the existence of a nonvanishing vector field spanning $\scX$ and a nonvanishing $n$-form in $\Omega^n(\scV^*).$ As in the case of surfaces, via the multi-contact structure, one obtains a orientation on the leaf space of the foliation induced by $\scV.$  
  

\section{Chains}
\label{sec:chain-construction}
In this section we will discuss chains in 2-dimensional path geometries. We first review known facts about chains and derive necessary conditions for a 3-dimensional path geometry that is defined by chains of a 2D path geometry. We finish by describing our characterization of chains as a subclass of  a larger set of 3-dimensional path geometries.

\subsection{Chains in 2D path geometry}
\label{sec:an-overview-chains}


As was discussed in \ref{sec:2D-review}, path geometries on surfaces are defined in terms of a contact 3-manifold with the property that the contact distribution has a splitting. As a result, 2D path geometries can be considered as the lowest dimensional case  of integrable Legendrian contact structures also known as integrable para-CR structures (see \cite{DMT} for more information.) Such Cartan geometries are among contact parabolic geometries which include CR structures as well.

Similar to the case of CR structures, integrable Legendrian contact structures are equipped with a set of distinguished curves known as chains, which in the case of CR structures were introduced in \cite{CM-CR}. We refer the reader to \cite{CZ-CR} for a detailed account of chains in integrable Legendrian contact structures from a  Cartan geometric viewpoint.

From now on we restrict ourselves to the case of 2-dimensional path geometries. Recall that the Lie algebra $\mathfrak{sl}(3,\RR)$ in which, by Theorem \ref{thm:2D-path-geome}, the Cartan connection  $\psi$ of a 2D path geometry takes value has a contact grading
\[\mathfrak{sl}(3,\RR)=\fg_{-2}\oplus\fg_{-1}\oplus\fg_{0}\oplus\fg_{1}\oplus\fg_{2},\]
where $\fg_{-2},\fg_{2}$ have rank 1 and  $\fg_{-1},\fg_{1},\fg_0$ have rank  2. This point of view can be used to define chains as follows.
\begin{definition}
Let $(N,\scD_1,\scD_2)$ be a path geometry on the surface $M=N\slash\scD_2$ with Cartan geometric data $(\cP\to N,\phi).$   Fixing a contact grading of $\mathfrak{sl}(3,\RR)$ and taking any $X\in\fg_{-2},$ a chain on $N$ is a curve that is the projection of an integral curve of the vector field $\phi^{-1}(X)\in \Gamma(T\cP)$ via the projection $\tau\colon\cP\to N.$  
\end{definition}
As a result of the definition, it can be shown, e.g. see \cite{CZ-CR}, that  chains of a 2D path geometry $(N,\scD_1,\scD_2)$ define a 3-dimensional path geometry $(Q,\scX,\scV)$ on the 3-manifold $N=Q\slash\scV.$ The paths of the induced path geometry on $N$ are defined for all directions that  are transversal to the contact distribution $\scC=\scD_1\oplus\scD_2\subset TN.$  In other words, the 5-manifold  $Q=\PP(TN\backslash\scC)$ is foliated by the natural lift of chains.
 

The relation between the geometry of integrable Legendrian contact structures and the path geometry of their chains has been studied via the so-called \emph{extension functor} in \cite{CZ-CR}. In particular, if $(\cG\to Q,\psi)$ denotes the Cartan geometry of a 3D path geometry and $(\cP\to N,\phi)$ is the Cartan geometry of a 2D path geometry, then one has the inclusion
\begin{equation}
  \label{eq:inclusion-P-G}  
  \iota\colon\cP\hookrightarrow\cG,
\end{equation}
and if $\phi$ is given as \eqref{eq:2D-path-geom-cartan-conn}, the pull-back of $\psi$ to $\cP$ is
  \[ \def\arraystretch{1.3}
\iota^*\psi=
    \begin{pmatrix}
      -\phi^2_2-\half \phi^1_1 & \theta_0 & \half \theta^2 & -\half \theta^1\\
      \omega^0 & \phi^2_2+\half\phi^1_1 & \half\omega^2 & -\half\omega^1\\
      \omega^1 & \theta^1 & \tfrac{3}{2}\phi^1_1 &  0\\
      \omega^2 & \theta^2 & 0 & -\tfrac{3}{2}\phi^1_1
    \end{pmatrix}.
  \]
It is a straightforward computation to show that the torsion and curvature of the 3D path geometry, as given in \eqref{eq:quadric-quartic},  can be expressed in terms of the fundamental invariants of the 2D path geometry, as given in \eqref{eq:path-geom-curv}. More precisely, one has
\begin{equation}
  \label{eq:TC-3D-to-2D}
  \iota^*\bT=(T_1(\theta^1)^2+C_1(\theta^2)^2)\otimes V\otimes E^{-2} ,\quad \iota^*\bC=6(\theta^1)^2(\theta^2)^2\otimes V\otimes E^{-1}
  \end{equation}
where on the right hand side we have suppressed the pull-back by a section $s\colon Q\to\cG$ and by the inclusion \eqref{eq:inclusion-P-G}.   In other words, if a 3D path geometry arises as the geometry of chains of a 2D path geometry, then there is a distinguished coframe in which the torsion and curvature can always be put in the form \eqref{eq:TC-3D-to-2D}. Moreover, one can check that  $\iota^*\exd\rho=0$ where
\begin{equation}
  \label{eq:quasi-symp-chains}
  \rho=\omega^1\w\theta^2+\omega^2\w\theta^1.  
\end{equation}
This implies that the 5-manifold $Q$ carries a \emph{quasi-symplectic structure}, i.e. $\rho\in\Omega^2(Q)$ satisfying $\rho\w\rho\neq 0$ and $\exd\rho=0.$

Lastly, we point out that chains can be defined as the projection of the null geodesics of the corresponding \emph{Fefferman conformal structure} of a 2D path geometry which has signature (2,2). We refer the reader to \cite{BW-chains} for a detailed account of this alternative definition. Using this point of view, the authors started from a 2D path geometry,  realized by a scalar 2nd order ODEs
\begin{equation}
  \label{eq:chain-scalar-ode}
  z''=F(t,z,z'),
  \end{equation}
for a function $F:=F(x,z,p)$ and showed that the pair of 2nd order ODEs that corresponds to the 3D path geometry of its chains is given by
\begin{equation}
  \label{eq:chain-pair-ODEs}
      \begin{aligned}
      z'' =& F + F_p\Delta+\half F_{pp}\Delta^2 +\tfrac{1}{6} F_{ppp}\Delta^3\\
      p''=&\tfrac{2}{\Delta}(p'-F)^2+F_p(3p' - 2F ) + F_t + pF_z + [F_{pp}(p' - F ) + 2F_z]\Delta\\
      & +\tfrac{1}{6}[F_{ppp} (p' - 2F) - F_{tpp} + 4F_{zp} - pF_{zpp})]\Delta^2
    \end{aligned}
  \end{equation}
where   $\Delta=z'-p.$ 

\subsection{3D path geometries arising from chains}
\label{sec:path-geom-aris-1}
 In this section we present a way of determining whether a 3D path geometry arises as the chains of a 2D path geometry. We first note that by \eqref{eq:TC-3D-to-2D}, a necessary condition for such 3D path geometries is that the curvature $\bC$ has two distinct real roots of multiplicity 2. We use the following proposition in order to describe our characterization.
\begin{proposition}\label{prop:type-D-curv-chains}
  Given a 3D path geometry $(Q,\scX,\scV)$ with associated Cartan geometry $(\cG\to Q,\psi),$ where $\psi$ is given as \eqref{eq:path-geom-cartan-conn-3D}, if the curvature $\bC$ has 2 distinct real roots of multiplicity 2, then there is a principal $B$-subbundle $\iota\colon\cG_{D_r}\hookrightarrow\cG,$  where  $B\subset\mathrm{GL}(2,\RR)$ is the Borel subgroup, over which 
  \[\iota^* W_0=\iota^*W_1=\iota^*W_3=\iota^* W_4=0,\quad \iota^*W_2=1,\]
  and the components of $\iota^*\psi$, as given in \eqref{eq:path-geom-cartan-conn-3D}, satisfy
  \[
    \begin{aligned}
      \iota^*\psi^1_2\equiv& 0&&\mod \{\iota^*\alpha^0,\iota^*\alpha^2,\iota^*\beta^2\},\\
      \iota^*\psi^2_1\equiv& 0&&\mod \{\iota^*\alpha^0,\iota^*\alpha^1,\iota^*\beta^1\},\\
      \iota^*\psi^2_2\equiv& -\iota^*\psi^1_1&&\mod \{\iota^*\alpha^0,\iota^*\alpha^1,\iota^*\alpha^2,\iota^*\beta^1,\iota^*\beta^2\}\\
      \iota^*\nu_1,\iota^*\mu_1,\iota^*\nu_2,\iota^*\mu_2\equiv & 0&&\mod  \{\iota^*\alpha^0,\iota^*\alpha^1,\iota^*\alpha^2,\iota^*\beta^1,\iota^*\beta^2\}
    \end{aligned}
  \]
\end{proposition}
\begin{proof}
  The proof is done via a standard application of Cartan's reduction procedure in the following way. Recall that a 3D path geometry is a Cartan geometry of type $(\mathrm{SL}(4,\RR),P)$ where $P=P_0\ltimes P_+$ is the stabilizer of a flag of a line inside a plane in $\RR^4$,  $P_0=\RR^*\times\mathrm{GL}(2,\RR)$ is referred to as the structure group and $P_+$ is the nilpotent part of $P$. As was mentioned before, the curvature of a 3D path geometry, $\bC,$ can be presented as the quartic \eqref{eq:quadric-quartic} with an induced  $\mathrm{GL}(2,\RR)$-action by the structure group. In a trivialization $\cG\cong Q\times P,$ the structure group $P_0$ is a  block-diagonal matrix
  \begin{equation}
    \label{eq:P-0-3Dpath}
    P_0=\left\{A\in\mathrm{SL}(4,\RR)\ \vline \ A=\mathrm{diag}(\tfrac{1}{a_{00}\det(\bH)},a_{00},\bH),\bH=
    \begin{pmatrix}
      a_{11} & a_{12}\\
      a_{21} & a_{22}
    \end{pmatrix}
  \right\}
\end{equation}
and
\begin{equation}
  \label{eq:P-p-3Dpath}
P_+=\left\{B\in\mathrm{SL}(4,\RR)\ \vline\  B=
    \begin{pmatrix}
      \hspace{-.3cm}1 & \hspace{-.3cm} p_0 & p_1+\half p_0q_1 & p_2+\half p_0q_2 & \\
      \hspace{-.3cm}0 & \hspace{-.3cm} 1 & q_1 & q_2 &\\
      \ \ 0_{2\times 1} & \ \ 0_{2\times 1} & &\hspace{-1.9cm} \mathrm{Id}_{2\times 2}&\\
    \end{pmatrix}
\right\}.\end{equation}
Since it is assumed that $\bC$ has two distinct real roots, using the action of $\mathrm{GL}(2,\RR),$ it is possible to translate the real roots to  $0$ and $\infty.$ More explicitly, for a choice of trivialization of $\cG,$ let
$\bC(u)=\Sigma_{i=0}^4
{\tiny\begin{pmatrix}
  4\\ i
\end{pmatrix}}
W_i(u)(\beta^1)^{4-i}(\beta^2)^i$
be the curvature at $u\in\cG.$ Using the right action  of the fibers, for  $g\in P$  and $u\in\cG$ the equivariant transformation of the Cartan connection implies the gauge transformation $\psi(u)\to \psi(r_gu)=g^{-1}\psi(u)g+g^{-1}\exd g$ and the Cartan curvature transforms as $\Psi(u)\to\Psi(r_gu)= g^{-1}\Psi(u)g$. Consequently,    using the group parameters in  \eqref{eq:P-0-3Dpath} and \eqref{eq:P-p-3Dpath} to express $g\in P$, it is straightforward to obtain 
\begin{equation}
  \label{eq:action-W0-W4}
    \begin{aligned}
      W_0(g^{-1}u)=&W_0(u)a_{11}^4+4W_1(u)a_{11}^3a_{21}+6W_2(u)a_{11}^2a_{21}^2+4W_3(u)a_{11}a_{21}^3+W_4(u)a_{21}^4,\\
      W_4(g^{-1}u)=&W_0(u)a_{12}^4+4W_1(u)a_{12}^3a_{22}+6W_2(u)a_{12}^2a_{22}^2+4W_3(u)a_{12}a_{22}^3+W_4(u)a_{22}^4.
  \end{aligned}
  \end{equation}
  Take $g\in P$ such that $a_{11}=a_{22}=1.$ Since $\bC(u)$ has two distinct real roots of multiplicity two, one can find $a_{12}$ and $a_{21}$ so that at $g^{-1}u\in\cG$ one has $W_0=W_1=W_3=W_4=0.$ As a result, one can consider the a sub-bundle $\iota_1\colon\cG^{(1)}\hookrightarrow\cG$ characterized by 
  \begin{equation}
    \label{eq:G1-first-reduction-curvature-chains)}
    \cG^{(1)}=\left\{u\in\cG\ \vline \ W_0(u)=W_1(u)=W_3(u)=W_4(u)=0\right\}.
      \end{equation}
      By our discussion above, the bundle  $\cG^{(1)}\to Q$ is a principal $P^{(1)}$-bundle where $P^{(1)}=(\RR^*)^3\ltimes P^+$ and $(\RR^*)^3\subset P_0$ is the Cartan subgroup, given by setting $a_{12}=a_{21}=0$ in \eqref{eq:P-0-3Dpath}. As a result, one obtains that $\iota^*_1\bC=6W_2(\beta^1)^2(\beta^2)^2,$ where by abuse of notation we have suppressed $\iota_1^*$ on the right hand side.  Moreover, the pull-back of   the Bianchi identities for $\exd W_3$ and $\exd W_1,$ given in \eqref{eq:W-A-curvature-torsion-Bianchies}, to $\cG^{(1)}$ gives $\iota_1^*\psi^2_1\equiv 0$ and $\iota_1^*\psi^1_2\equiv 0$ modulo $\{\iota^*_1\alpha^0,\iota^*_1\alpha^1,\iota^*_1\alpha^2,\iota^*_1\beta^1,\iota^*_1\beta^2\}.$ Suppressing  $\iota^*_1,$ one can write
      \begin{equation}
        \label{eq:psi12-psi21-chains}
        \psi^1_2=A^1_{2i}\alpha^i+B^1_{2a}\beta^a,\qquad \psi^2_1=A^2_{1i}\alpha^i+B^2_{1a}\beta^a,
      \end{equation}
      for some functions $A^a_{bi}$ and $B^a_{bc}$ on $\cG^{(1)}.$

      Since the curvature $\bC$ has to have two distinct real roots, it follows that $W_2\neq 0.$ The action of $P^{(1)}$ on $W_2$ is given by
      \[W_2(g^{-1}u)=a_{11}^2a_{22}^2W_2(u).\]
      Thus, depending on the sign of $W_2,$ one can normalize it to $\pm 1.$   From now on we assume   $W_2> 0,$ although the case $W_2<0$ can be treated identically.   See Remark \ref{rmk:sign-of-W_2-chains} for the difference of outcome in these two cases.  Define the sub-bundle $\iota_2\colon\cG^{(2)}\hookrightarrow \cG^{(1)}$ as
      \begin{equation}
        \label{eq:P2-2nd-reduction}
        \cG^{(2)}=\left\{u\in\cG^{(1)}\ \vline\ W_2(u)=1\right\}.
      \end{equation}
      It follows that $\cG^{(2)}\to Q$ is a principal $P^{(2)}$-bundle where $P^{(2)}=(\RR^*)^2\ltimes P_+$ and $(\RR^*)^2\subset P_0$ is given by $a_{12}=a_{21}=0$ and $a_{22}=1/a_{11}$ in \eqref{eq:P-0-3Dpath}. Using the Bianchi identities \eqref{eq:W-A-curvature-torsion-Bianchies}, via pull-back to $\cG^{(2)}$ one obtains
      \begin{equation}
        \label{eq:psi22-reduced-chains}
        \psi^2_2=-\psi^1_1+A^2_{2i}\alpha^i+B^2_{2a}\beta^a
      \end{equation}
      for functions $A^2_{2i}$ and $B^2_{2a}$ on $\cG^{(2)}.$

      The pull-back of the Cartan connection $\psi$ to $\cG^{(1)}$ and $\cG^{(2)}$ is no longer equivariant under the action of the fibers $P^{(1)}$ and $P^{(2)}$, respectively. Nevertheless, the pull-back of $\psi$ to $\cG^{(2)}$ defines a so-called \emph{$\{e\}$-structure}. It is straightforward to find the action of the fibers on the quantities $A^a_{bk}$ and $B^a_{bc}.$ In particular, using the parametrization in \eqref{eq:P-p-3Dpath}, an action by $g\in P^{(2)}$ gives
      \begin{equation}\label{eq:AB-group-action-chains}
        \begin{aligned}
          B^1_{21}(g^{-1}u)=&\tfrac{1}{a_{11}a_{00}}B^1_{21}(u)+q_2\\
          B^2_{12}(g^{-1}u)=&\tfrac{a_{11}}{a_{00}}B^2_{12}(u)+q_1\\
          A^1_{21}(g^{-1}u)=&{\tfrac{a_{00}^2}{a_{11}^2}q_1A^1_{20}(u)+\tfrac{a_{00}}{a_{11}}A^1_{21}(u)-\tfrac{1}{a_{11}a_{00}}p_0B^{1}_{21}(u) -\half p_0q_2+p_2}\\
          A^2_{12}(g^{-1}u)=& a_{00}^2a_{11}^2q_2A^2_{10}(u)+a_{00}a_{11}A^2_{12}(u)-\tfrac{a_{11}}{a_{00}}p_0B^2_{12}(u)-\half p_0q_1+p_1
        \end{aligned}
      \end{equation}
      Infinitesimally, these actions correspond to Bianchi identities 
      \begin{equation}
        \label{eq:AB-inf-action-Bianchies-chains}
        \begin{aligned}
          \exd B^1_{21}\equiv& -(\psi^0_0+\psi^1_1)B^1_{21}+\nu_2\\
          \exd B^2_{12}\equiv& -(\psi^0_0-\psi^1_1)B^2_{12}+\nu_1\\
          \exd A^1_{21}\equiv&  \nu_1A^1_{20}+(\psi^0_0-\psi^1_1)A^1_{21}-B^1_{21}\mu_0+\mu_2\\
          \exd A^2_{12}\equiv&  \nu_2A^2_{10}+(\psi^0_0+\psi^1_1)A^2_{12}-B^2_{12}\mu_0+\mu_1
        \end{aligned}
      \end{equation}
      modulo $\{\alpha^0,\alpha^1,\alpha^2,\beta^1,\beta^2\}.$ As a result, the sub-bundle $\iota_3\colon\cG^{(3)}\hookrightarrow \cG^{(2)}$ given by
      \begin{equation}
        \label{eq:G3-reduced-Pp-chains}
        \cG^{(3)}=\{u\in\cG^{(2)}\ \vline\ B^1_{21}(u)=B^2_{12}(u)=A^1_{21}(u)=A^2_{12}(u)=0\}
      \end{equation}
      is well-defined as a principal $B$-bundle where $B\cong(R^*)^2\ltimes\RR\subset \mathrm{GL}(2,\RR)$ is the Borel subgroup. In terms of parametrizations \eqref{eq:P-0-3Dpath} and \eqref{eq:P-p-3Dpath} for $P=P_0\ltimes P_+,$ one can express $B\subset P$ as $a_{22}=1/a_{11}$ and $a_{12}=a_{21}=p_1=p_2=q_1=q_2=0.$

      In the differential relations \eqref{eq:AB-inf-action-Bianchies-chains}, the pull-back for the first two equations to $\cG^{(3)}$ imply $\nu_1,\nu_2$ vanish mod $\{\alpha^i,\beta^a\}.$ Consequently, the last two relations  imply $\mu_1$ and $\mu_2$ vanish mod $\{\alpha^i,\beta^a\}.$ The reduction of $\mu_1,\mu_2,\nu_1,\nu_2$ on $\cG^{(3)},$ together with the pull-back of \eqref{eq:psi12-psi21-chains} and \eqref{eq:psi22-reduced-chains} to $\cG^{(3)}$ finishes the proof, where $\iota:=\iota_1\circ\iota_2\circ\iota_3$ and $\cG_{D_r}:=\cG^{(3)}.$
    \end{proof}
     \begin{remark}\label{rmk:D-r-curvature-type}
   A basic invariant of a binary quartic, such as the curvature $\bC,$ acted on by $\mathrm{GL}(2,\RR),$ is its root type. Motivated by the Petrov classification of the self-dual and anti-self-dual Weyl curvature of a Lorentzian conformal structure, one finds 10 possible algebraic types for the quartic $\bC$ in our setting depending on the multiplicity and reality of the root type. 
Motivated by the symbols used for Petrov types,  when a quartic has two distinct real roots of multiplicity two its algebraic type is denoted by $D_r,$ hence we denote the  reduced 8-dimensional bundle in this case by $\cG_{D_r}.$  
\end{remark}
    \begin{remark}
      In the proof of  Proposition \ref{prop:type-D-curv-chains} it was not necessary to know the  explicit group actions. We provided the explicit form in order to clarify the reduction procedure. In order to carry out such reductions it suffices to have the infinitesimal form of the group action on invariants which, as mentioned above, on $\cG$  are given by the Bianchi identities \eqref{eq:W-A-curvature-torsion-Bianchies}. We refer the reader to \cite{Gardner-Book} for a discussion on the relation between explicit group action and its infinitesimal form and also for the notion of an $\{e\}$-structure in the context of Cartan's method of equivalence, which appeared in the proof above. 
    \end{remark}

    

Now we can state the main theorem of this section which identifies a natural class of 3D path geometries that contains chains as a proper subclass. 
\begin{theorem}\label{thm:2d-path-geometries-generalized-chains}
Let $(\cG\to Q,\psi)$ be the Cartan geometry associated with a 3D path geometry $(Q,\scX,\scV)$ satisfying the following conditions:
  \begin{enumerate}
  \item The quartic $\bC$ has two distinct real roots of multiplicity 2. 
  \item The invariantly defined 2-form $\rho=\alpha^1\w\beta^2+\alpha^2\w\beta^1\in\Omega^2(\cG_{D_r})$ satisfies $\iota^*\exd\rho=0$ for $\iota\colon\cG_{D_r}\hookrightarrow\cG$ from Proposition \ref{prop:type-D-curv-chains}, inducing a quasi-symplectic structure on $Q.$
  \end{enumerate}
  Then the Pfaffian systems $\cI_2:=\{\alpha^0,\alpha^1\},\cI_1=\{\alpha^0,\alpha^2\}$ are integrable and their respective leaf spaces, $M$ and $T,$ are equipped with 2D path geometries that are dual via the fibration $M\leftarrow N\rightarrow T,$ where $N$ is the leaf space of $\{\alpha^0,\alpha^1,\alpha^2\}$.  The projection of each path on $Q$ to $N$ is transversal to the contact distribution  $\ker\alpha^0\subset TN.$ The invariants $T_1$ and $C_1$ of such  2D path geometries  $(N,\scD_1,\scD_2)$ depend on the 4th jet of torsion entries $A_0$ and $A_2$ of the 3D path geometry, respectively. 
\end{theorem}
\begin{proof}
  By Proposition \ref{prop:type-D-curv-chains}, from condition (1) one obtains a sub-bundle $\iota\colon\cG_{D_r}\hookrightarrow\cG$ which is a principal $B$-bundle over $Q.$

  The 2-form $\rho=\alpha^1\w\beta^2+\alpha^2\w\beta^1\in\Omega^2(\cG_{D_r})$ is well-defined  and, as a semi-basic 2-form with respect to $\cG_{D_r}\to Q,$ has maximal rank. Moreover, its characteristic direction,  defined as $v\im\rho=0,$  coincides with  $\scX=\tau_*\langle\tfrac{\partial}{\partial\alpha^0}\rangle,$ i.e. the tangent direction to the paths of the initial 3D path geometry.

  The vanishing of $\iota^*\exd\rho$ is invariantly defined on $\cG_{D_r}$. Suppressing $\iota^*,$ it is straightforward to see that $\exd\rho=0$ implies the vanishing conditions
  \[A_1=A^1_{20}=A^2_{10}=A^2_{20}=A^2_{21}=A^2_{22}=B^2_{21}=B^2_{22}  = 0,\]
for the functions $A^i_{jk}$ and $B^a_{bi}$ in \eqref{eq:psi12-psi21-chains} and \eqref{eq:psi22-reduced-chains}. In particular, by \eqref{eq:psi22-reduced-chains}, one has $\psi^2_{2}=-\psi^1_1.$ Checking the differential consequences of this vanishing condition is a matter of tedious computation which yields 
  \begin{equation}
    \label{eq:quasi-symp-consequences-chains}
    \begin{gathered}
      \psi^1_2=0,\quad \psi^2_1=0,\quad \psi^2_2=-\psi^1_1,\quad \nu_1=\half\alpha^2,\quad \nu_2=-\half\alpha^1,\\ \mu_1=-\half\beta^2-\tfrac{1}{4}A_{0;\underline{22}}\alpha^2-\tfrac{1}{4}A_{0;\underline{2}}\alpha^0,\quad \mu_2=\half\beta^1+\tfrac{1}{4}A_{2;\underline{11}}\alpha^1+\tfrac{1}{4}A_{2;\underline{1}}\alpha^0.
    \end{gathered}
  \end{equation}
  where the pull-back $\iota^*$ is suppressed.  Note that the relations $A_{0;\underline{22}}=-A_{2;\underline{11}}$  and $A_{0;\underline{21}}=A_{2;\underline{21}}=0$ hold.

  Using the relations \eqref{eq:quasi-symp-consequences-chains} to re-compute the Cartan curvature $\Psi$ for the 3D path geometry on $\cG_{D_r}$, it follows that on $\cG_{D_r}$ one has
  \begin{equation}
    \label{eq:general-chain-2D-path}
    \begin{aligned}
      \exd\alpha^0=&-2\psi^0_0\w\alpha^0+\alpha^1\w\alpha^2,\\
      \exd\alpha^1=&(-\psi^0_0-\psi^1_1)\w\alpha^1-\beta^1\w\alpha^0,\\
      \exd\alpha^2=&(-\psi^0_0+\psi^1_1)\w\alpha^2-\beta^2\w\alpha^0.
    \end{aligned}
  \end{equation}
  Thus, the Pfaffian systems $\cI_2:=\{\alpha^0,\alpha^1\}$ and $\cI_1:=\{\alpha^0,\alpha^2\}$ are integrable whose leaf space will be denoted as $M$ and $T,$ respectively. Moreover, by \eqref{eq:general-chain-2D-path}, the leaf space of $\{\alpha^0,\alpha^1,\alpha^2\}$ denoted by $N,$ defines a 2D path geometry on $M$ with contact distribution $\ker\alpha^0=\scD_1\oplus\scD_2$ where $\scD_1:=\ker\cI_1=\langle\tfrac{\partial}{\partial\alpha^1}\rangle$ and $\scD_2:=\ker\cI_2=\langle\tfrac{\partial}{\partial\alpha^2}\rangle.$ Furthermore, using the quotient map $\nu\colon Q\to N,$ it is clear that the tangent line to paths on $Q,$ i.e. $\langle\tfrac{\partial}{\partial\alpha^0}\rangle,$ are mapped to transversal lines to the contact distribution via $\nu_*.$

More precisely,  it follows that  the Cartan geometry for the 2D path geometry on $N$ is given by $(\cG_{D_r}\to N,\phi)$ where
\begin{equation}
  \label{eq:phi-2D-path-general-chain}
  \phi=    \begin{pmatrix} 
      \tfrac{1}{6} A_{0;\underline{22}}\alpha^0-\psi^0_0-\tfrac 13\psi^1_1 & B_{20}\alpha^0+\beta^2 & \mu_0+B_{0i}\alpha^i\\
      \alpha^1 & -\tfrac{1}{12}A_{0;\underline{22}}\alpha^0+\tfrac 23\psi^1_1 & B_{10}\alpha^0+B_{11}\alpha^1+\beta^1  \\
      \alpha^0 & \alpha^2 & -\tfrac{1}{12}A_{0;\underline{22}}\alpha^0+\psi^0_0-\tfrac 13\psi^1_1   
    \end{pmatrix}
    \end{equation}
in which
  \[
    \begin{gathered}
      B_{10}=\tfrac{7}{36}A_{2;\underline{11}2}-\tfrac 19A_{2;\underline{1}2\underline{1}},\quad B_{11}=-\tfrac{1}{4}A_{0;\underline{22}},\quad B_{20}=\tfrac{7}{36}A_{0;\underline{22}1}-\tfrac 19A_{0;\underline{2}1\underline{2}},\\
      B_{02}=\tfrac{5}{36}A_{2;\underline{11}2}+\tfrac{1}{36}A_{2;\underline{1}2\underline{1}},\quad B_{01}=\tfrac{1}{36}A_{0;\underline{2}1\underline{2}}-\tfrac{1}{9}A_{0;\underline{22}1},\quad B_{00}=\tfrac{7}{36}A_{0;\underline{22}12}-\tfrac{1}{9}A_{0;\underline{2}1\underline{2}2}.            
    \end{gathered}
  \]
  Consequently, the invariants $T_1$ and $C_1$ for such path geometries are given by
  \begin{equation}
    \label{eq:TU-general-chain}
    T_1=\tfrac{7}{36}A_{0;\underline{22}11}-\tfrac 19A_{0;\underline{2}1\underline{2}1}+A_0,\qquad C_1=\tfrac{7}{36}A_{2;\underline{11}22}-\tfrac 19A_{2;\underline{1}2\underline{1}2}+A_2
  \end{equation}        
\end{proof}
\begin{remark}\label{rmk:3d-path-geometries-chain-general}
  The 3D path geometry obtained in Theorem \ref{thm:2d-path-geometries-generalized-chains}  is an example of  variational orthopath structures defined in \cite{MS-cone}. This is due to the fact that in the path geometry $(Q,\scX,\scV)$  the conformal class of the bundle metric $[s^*\beta^1\circ s^*\beta^2]\subset \mathrm{Sym}^2(\scV^*)$ is well-defined for any section $s\colon Q\to\cG_{D_r}$  and the 2-form $\rho$ is a closed quasi-symplectic form. It is shown in \cite{MS-cone} that the paths of such structures are the extremal curves of a class of non-degenerate first order Lagrangians. Using Cartan-K\"ahler analysis, it follows that their local generality depends on 3 functions of 3 variables. 
\end{remark}
\begin{remark}\label{rmk:sign-of-W_2-chains}
  The sign of $W_2$ determines the orientation induced on the 2D path geometry $(N,\scD_1,\scD_2)$ from the 3D path geometry $(Q,\scX,\scV)$ where $\scD_i=\tau_*\langle\tfrac{\partial}{\partial\alpha^i}\rangle$. More precisely, before normalizing $W_2$ to $\pm 1,$ one has $\exd\alpha^0\equiv W_2\alpha^1\w\alpha^2$ mod $\{\alpha^0\}.$ Thus, when $W_2>0$ it follows that $\alpha^1$ and $\alpha^2$ are $\scD_1$-positive and $\scD_2$-positive, respectively, and $\alpha^0$ is positive with respect to the induced co-orientation on the contact distribution $\scC=\scD_1\oplus\scD_2$, as we recalled  at the end of  \ref{sec:solut-equiv-probl}. Similarly, it follows that  when $W_2<0$ then  $\exd\alpha^0$ in \eqref{eq:general-chain-2D-path} changes to $\exd\alpha^0\equiv-\alpha^1\w\alpha^2$ modulo $\{\alpha^0\}.$ Thus, the descent from the 3D path geometry to  the 2D path geometry, as described above, induces a negative co-orientation.    
\end{remark}


Note that by \eqref{eq:TU-general-chain}  3D path geometries in Theorem \ref{thm:2d-path-geometries-generalized-chains} do not satisfy the necessary condition \eqref{eq:TC-3D-to-2D} relating the torsion entries of the 3D path geometry to $T_1$ and $C_1.$ It turns out that adding this necessary condition to conditions (1) and (2) in Theorem \ref{thm:2d-path-geometries-generalized-chains} is also sufficient for a 3D path geometry to arise as chains of a 2D path geometry.
\begin{corollary}\label{cor:3d-path-geometries-chain-characterization}
        A 3D path geometry $(Q,\scX,\scV)$ arises from chains of a 2D path geometry $(N,\scD_1,\scD_2)$ if and only if 
  \begin{enumerate}
  \item The binary quartic $\bC$ has two distinct real roots of multiplicity 2. 
  \item The 5-manifold $Q$ has a compatible quasi-symplectic structure, i.e. there exists $\rho\in\Omega^2(Q)$ such that  $\rho\w\rho\neq 0$ and $\exd\rho=0,$ for which the paths are  characteristic curves, the fibers of $Q\to N$ are isotropic, where $N:=Q\slash\scV$, and $\rho$ is preserved by the fiber action of the principal bundle $\iota\colon\cG_{D_r}\hookrightarrow\cG$ in Proposition \ref{prop:type-D-curv-chains}.
  \item  The entries of the  binary quadric $\iota^*\bT$  in \eqref{eq:quadric-quartic} have no dependency on the fibers of  $Q\to N.$
  \end{enumerate}
\end{corollary}
\begin{proof}
As was discussed in \ref{sec:an-overview-chains}, conditions (1) and (2) are necessary conditions  for a 3D path geometry to arise as chains. Furthermore, the necessary condition \eqref{eq:TC-3D-to-2D}  has to hold on $\cG_{D_r}$ as well due to the fact that  the torsion entries $A_0$ and $A_2$ need to be well-defined up to scale on $N,$ i.e.
  \[\exd A_0,\exd A_2\equiv 0\mod\{\alpha^0,\alpha^1,\alpha^2,\psi^0_0,\psi^1_1\}.\]
  Using \eqref{eq:quasi-symp-consequences-chains} and \eqref{eq:phi-2D-path-general-chain}, condition (3) implies 
  \[       \def\arraystretch{1.3}
\iota^*\psi=   \begin{pmatrix}
      -\psi^0_0 & \mu_0 & \half \beta^2 & -\half \beta^1\\
      \alpha^0 & \psi^0_0 & \half\alpha^2 & -\half\alpha^1\\
      \alpha^1 & \beta^1 & \psi^1_1 &  0\\
      \alpha^2 & \beta^2 & 0 & -\psi^1_1
    \end{pmatrix},\qquad   \phi=    \begin{pmatrix} 
      -\psi^0_0-\tfrac 13\psi^1_1 & \beta^2 & \mu_0\\
      \alpha^1 & \tfrac 23\psi^1_1 & \beta^1  \\
      \alpha^0 & \alpha^2 & \psi^0_0-\tfrac 13\psi^1_1   
    \end{pmatrix}.
\]
 where $(\cG_{D_r}\to N,\phi)$ is the Cartan geometry for the 2D path geometry  induced by $\iota^*\psi$ and, by \eqref{eq:TU-general-chain}, it follows that the invariants $T_1$ and $C_1$ are $\iota^*A_0$ and $\iota^*A_2,$ respectively.  We recall that, as was explained in \ref{sec:conventions}, in our notation condition (3) can be expressed as $A_{0;\underline{2}}=A_{2;\underline{1}}=0.$ This is due to the fact that conditions $A_{0;\underline{1}}=A_{2;\underline{2}}=0 $ already follow from conditions (1) and (2) and are satisfied for 3D path geometries in Theorem \ref{thm:2d-path-geometries-generalized-chains}.
 
Furthermore, the resulting Cartan connection $\phi$  uniquely determines $\iota^*\psi$ which coincides with what is obtained via the extension functor from chains of the 2D path geometry $(\cG_{D_r}\to N,\phi)$ as discussed in \ref{sec:an-overview-chains}. Thus,  conditions (1),(2),(3) provide necessary and sufficient conditions for a 3D path geometry to arise as chains of a 2D path geometry.
\end{proof}
\begin{remark}\label{rmk:chains-3d-path-geometries-exclusive}
Given a pair of second order ODEs, checking conditions (1),(2) and (3) only involves finding  roots of a quartic, linear algebra  and differentiation and can be verified straightforwardly.    Note that the line  fields spanned by the vector fields $\tfrac{\partial}{\partial\alpha^1},\tfrac{\partial}{\partial\alpha^2}\tfrac{\partial}{\partial\beta^1},\tfrac{\partial}{\partial\beta^2}$ are well-defined on $\cG_{D_r}$ and, therefore, condition (3) is easy to verify. Lastly, we point out that the curvature of  path geometries in dimensions larger than 3 cannot be represented as a binary polynomial. Nevertheless,  one can formulate an analogue of condition (1) in Corollary \ref{cor:3d-path-geometries-chain-characterization} for the curvature of path geometries defined by chains of higher dimensional (para-)CR structures. However, condition (3)  is never true for chains of   non-flat (para-)CR structures in higher dimensions. 
\end{remark} 






\section{Dancing and freestyling} 
\label{sec:dancing-construction}
In this section we define the class of dancing and freestyling path geometries in dimension three and give necessary and sufficient conditions that characterize them among 3D path geometries. We give examples and show when they arise as chains of a 2D path geometry.


\subsection{Dancing construction for a 2D path geometry}
\label{sec:an-overview-dancing}
Let $(N,\scD_1,\scD_2)$ be a 2D path geometry on the surface $M=N\slash\scD_2$. Then
\[
\scC=\scD_1\oplus \scD_2
\]
is a contact distribution, and
\[\pi_T\colon N\to T=N/\scD_1,\quad \pi_M\colon N\to M=N/\scD_2\] are 2-dimensional leaf spaces of integral curves of $\scD_1$ and $\scD_2$, respectively. We shall now construct a canonical 3D path geometry on $N$. We will see in Remark \ref{rmk:dancing} that it is  a reinterpretation of a construction given in \cite{Bor,D}.

{\bf Step 1.} Firstly we define a pair of 2-parameter family of   surfaces $\{\Sigma^1_x\}_{x\in M}$ and $\{\Sigma^2_y\}_{y\in T}$ in $N$ parameterized by points of $M$ and $T$.  
We define  the surface  $\Sigma^1_x\subset N,x\in M$ to be the union of all curves $\pi^{-1}_T(\hat y)\subset N$ where $\hat y\in T$ such that the curves $\pi^{-1}_T(\hat y)$  and $\pi^{-1}_M(x)$ intersect, and similarly $\Sigma^2_y\subset N,y\in T$ to be union of all curves $\pi_M^{-1}(x)\subset N$ where $\hat x\in M$ such that $\pi_M^{-1}(\hat x)$ intersects $\pi_M^{-1}(y)$. In other words, in order to define $\Sigma^1_x$ we fix an integral curve of $\scD_2$ in $N$, represented by $x\in M$, and from any point of this curve we take the unique integral curve of $\scD_1$ that passes through that point. Similarly, in order to define $\Sigma^2_y$ we fix an integral curve of $\scD_1$ in $N$, represented by $y\in T$, and from any point of this curve we take the unique integral curve of $\scD_2$ that contains that point. As a result, the tangent planes to $\Sigma^1_x$ and $\Sigma^2_y$ coincide with the contact distribution along $\pi^{-1}_M(x)$ and $\pi^{-1}_T(y)$ in sufficiently small neighborhoods along these curves, respectively.   
\begin{center}
\begin{figure}[h]%[!ht]%[h]%[ht]
  \includegraphics[width=.4\linewidth]{sigma1.png} 
    \includegraphics[width=.4\linewidth]{sigma2.png}   
  \caption{ (a) For $x_1,x_2\in M,$ the vertical green lines are open subsets of the fibers $\pi_M^{-1}(x_i)\subset N.$ The blue curves intersecting the fibers $\pi_M^{-1}(x_i)$ are natural lifts of the paths (purple curves) on $M$ passing through $x_i.$ The surfaces $\Sigma^1_{x_i}\subset N$  are ``ruled'' by the blue curves and  its generating curve is the fiber $\pi_M^{-1}(x_i)$. (b) For $y_1,y_2\in T,$ the solid purple curves on $M$ are the paths $\pi_M\circ\pi^{-1}_T(y_i).$ Their lift to $N,$ represented by solid blue curves, are the generating curves of the ruled surfaces $\Sigma^2_{y_i}\subset N$ whose ruling lines are vertical green lines which represent open subsets of the fibers $\pi_M^{-1}(t)\subset N$ where $t\in\pi_M\circ\pi^{-1}_T(y_i).$}
\end{figure}
\end{center}

{\bf Step 2.} We define paths $\gamma_{xy}\subset N$ as intersections of surfaces $\Sigma^1_x,\Sigma^2_y\subset N$ provided that $(x,y)\in M\times T\setminus N$. Recall from \ref{sec:2D-review} that $N$ is identified with a subset of $M\times T$ comprised  of points $(x,y)\in M\times T$ such that $x$ intersects $y$. Hence, any  pair of non-intersecting integral curves of $\scD_1$ and $\scD_2$ define a path $\gamma_{xy}$.

We shall show that the collection of all curves $\gamma_{xy}\subset N$ for $(x,y)\in M\times T\setminus N$ define a path geometry on $N$. More precisely, we  prove that the set of tangent directions to paths $\gamma_{xy}$ defines an open subset  $Q\subset\PP TN$ and therefore defines a (generalized) path geometry on $N$. 
\begin{proposition}\label{prop0}
Let $(N,\scD_1,\scD_2)$ be a 2-dimensional path geometry on $M:=N\slash\scD_2.$  Restricting to a sufficiently small neighborhood $U\subset N$ of any point in $N$ and defining  $M_U=U/\scD_2\subset M$ and $T_U=U/\scD_1\subset T,$ one has the following. 
\begin{itemize}
\item[(a)] Exactly 1-parameter sub-family of surfaces from $\{\Sigma^1_x\}_{x\in M_U}$ and exactly 1-parameter sub-family of surfaces from $\{\Sigma^2_y\}_{y\in T_U}$ pass through any point of $U$.
\item[(b)] For any two $\Sigma^1_{x^1}$, $\Sigma^1_{x^2}$ passing through a fixed point of $U$ their tangent spaces have a common line (which is $\scD_1$), and similarly, for any two $\Sigma^2_{y^1}$, $\Sigma^2_{y^2}$ passing through a fixed point of $U$ their tangent spaces have a common line (which is $\scD_2$). 
\item[(c)] Exactly one surface from the family $\{\Sigma^1_x\}_{x\in M_U}$, is tangent to the contact distribution $\scC=\scD_1\oplus\scD_2$ and exactly one surface from family $\{\Sigma^2_y\}_{y\in T_U}$, is tangent to $\scC$ at each point $p\in U$; any other pair of surfaces from these two families passing through $p$ intersect along a line at $p$.
\end{itemize}
\end{proposition}
\begin{proof}
Identify $U$ as a subset of $M_U\times T_U$ as explained in
\ref{sec:2D-review} and fix $p\in U$. Thus, the point $p\in U$ can be  represented as a
pair $(\hat x,\hat y)\in M_U\times T_U$. Then a surface $\Sigma^1_x$
passes through $p\in U$ if and only if $(x,\hat y)\in U\subset
M_U\times T_U,$ i.e. the integral curve  of $\scD_1$ represented by $x$
intersects the integral curve of $\scD_2$ represented by $\hat y$.
Similarly, $\Sigma^2_y$ passes through $p\in U$ if and only if $(\hat
x, y)\in U\subset M_U\times T_U$, i.e. the integral curve of $\scD_2$
represented by $y$ intersects the integral curve of $\scD_1$
represented by $\hat x$. This proves statement (a). Statement (b)
follows from the construction as all $\Sigma^1_x$ are foliated by
integral curves of $\scD_1$ and  all $\Sigma^2_y$ are foliated by
integral curves of $\scD_2$. Furthermore, in order to prove statement
(c) notice that, $\Sigma^1_x$ is tangent to $\scC$ at $p$ if and only
if $x=\hat x$ and $\Sigma^2_y$ is tangent to $\scC$ at $p$ if and only
if $y=\hat y$. Finally, the tangent spaces of $\Sigma^1_{x^1}$ and
$\Sigma^1_{x^2}$, as well as $\Sigma^2_{y^1}$ and $\Sigma^2_{y^2}$  do
not coincide for sufficiently small $U$ because flows of $\scD_1$ and
$\scD_2$ do not commute since they span is a contact distribution, i.e.
$[\scD_1,\scD_2]=TN$.
\end{proof}
By Proposition \ref{prop0}  it follows that there is an open subset
\[
Q\subset \PP TU
\]
such that  the natural lift of the 4-parameter family of paths $\gamma_{xy}\subset U$, where $x\in M_U$ and  $y\in T_U$, are well-defined curves foliating $Q$ giving rise to  a  3D path geometry on $U$. 
Since our consideration is local, from now on we shall assume $U=N.$ As in \ref{sec:path-geom-defin}, we denote the tangent direction to the natural lift of paths  $\gamma_{xy}$ foliating  $Q$ by $\scX\subset TQ$ and  denote the vertical tangent bundle of the fibration $Q\to N$ by $\scV\subset TQ.$
\begin{definition}
Given a 2D path geometry $(N,\scD_1,\scD_2)$  on  $M=N\slash\scD_2$,  the 3D path geometry  $(Q,\scX,\scV)$ defined via steps 1 and 2 above is called its corresponding dancing path geometry.
\end{definition}


\begin{remark}
If a 2D path geometry is presented as a scalar second order ODE \eqref{eq:sode} then $N=J^1(\RR,\RR)$ and
\[
Q=\PP (TN\setminus \scC),
\]
i.e. the curves in the dancing path geometry are tangent to all directions that are transversal to the contact distribution $\scC$. Moreover, each surface $\Sigma^1_x\subset N$     projects onto $M:=J^0(\RR,\RR)$ and is foliated by the lift of all paths in $M$ passing through $x$. On the other hand, each $\Sigma^2_y\subset N$ is the circle bundle $\PP TM$ restricted to the curve  $\pi_M\circ\pi^{-1}_T(y)\subset M$ where $y\in T.$
\end{remark}
Our next proposition is about the  pair of second order ODEs that corresponds to a dancing construction.
\begin{proposition}\label{prop:ODEs-danc-constr-2d}
  Given a scalar second order ODE
  \begin{equation}
    \label{eq:Sclar-ODE-dancing}
    z''=F(t,z,z'),
  \end{equation}
 it defines a 2D path geometry on $M\cong J^0(\RR,\RR)$ given by the triple  $(N,\scD_1,\scD_2)$ where, locally, $N\cong J^1(\RR,\RR).$ Let $(t,z,p)$ be the induced jet coordinates on $N.$ By dancing construction  $N$ is equipped with a 3D path geometry whose corresponding pair of second order ODEs is given by
 \begin{equation}
   \label{eq:pair-ODEs-dancing}
   z''=F(t,z,z'),\quad p''=G(t,z,p,z',p').
 \end{equation}
 for a uniquely determined function $G\colon J^1(\RR,\RR^2)\to \RR.$
\end{proposition}
\begin{proof}
  Recall that every solution curve of pairs of ODEs for the dancing construction are contained in a surface   $\Sigma^2_y$ for some $y\in M.$ Moreover, the surfaces $\Sigma_y^2$ project to solutions of \eqref{eq:sode} on $M.$  As a result, it follows that for such  choice of coordinates, the original equation \eqref{eq:sode} on $M$ is contained in the system for the dancing path geometry. Hence, the dancing construction extends the original scalar second order ODE to a pair of second order ODEs.
\end{proof}
\begin{remark}
Analogously, one can replace $M$ in Proposition \ref{prop:ODEs-danc-constr-2d} by the twistor space $T:=N\slash\scD_1$ and take a \emph{dual} viewpoint in the sense of \ref{sec:2D-review}. Identifying $T$ as $J^0(\RR,\RR)$  with jet  coordinates $(\tilde t,\tilde v),$ let $(\tilde t,\tilde z,\tilde p)$ be the induced jet coordinates  on $N\cong J^1(\RR,\RR)$. Similar reasoning as above shows that the dual equation \eqref{eq:dual} on $T$ shows up in the system of the dancing path geometry written in coordinates $(\tilde t,\tilde z,\tilde p)$. From this viewpoint, the dancing construction extends the dual ODE to a system as well. However, the two equations (the original one and the dual) do not show up in the dancing construction at the same time. Each of them appear only for the particular choices of coordinates. Nevertheless, the dancing construction can be thought of as a geometric pairing of the original ODE and its dual into one system.
\end{remark}


\begin{remark}\label{rmk:dancing}
  One can obtain relatively explicit formulas for the dancing construction, and, in particular, function $G$ in Proposition \ref{prop:ODEs-danc-constr-2d} in  terms of a general solution function $\Phi\colon M\times T\to\R$ of a scalar second order ODE. Take local coordinate $(t,z)$ on $M$ and $(a, b)$ on $T.$ The coordinates on $T$ can be interpreted as constants of integration for \eqref{eq:sode}, as explained in  \ref{sec:2D-review}. If we fix $\hat x=(\hat t,\hat z)\in M$ and $\hat y=(\hat a,\hat b)\in T$ we get 
\[
\begin{aligned}
&\Sigma^1_{\hat x}=\{(t,z, a,b)\in M\times T\ |\ \Phi(t,z,a,b)=0,\ \Phi(\hat t,\hat z,a,b)=0\}\\
&\Sigma^2_{\hat y}=\{(t,z,a,b)\in M\times T\ |\ \Phi(t,z,a,b)=0,\ \Phi(t, z,\hat a,\hat b)=0\}
\end{aligned}
\]
as subsets of $M\times T$. Hence, the curves of the dancing path geometry are given by
\[
\gamma_{\hat x\hat y}=\{(t,z,a,b)\in M\times T\ |\ \Phi(t,z,a,b)=0,\ \Phi(\hat t,\hat z,a,b)=0,\ \Phi(t, z,\hat a,\hat b)=0\}.
\]
Using this expression for the paths of dancing path geometry, in principle one is able to obtain the  corresponding pair of second order ODEs for them  (c.f. \cite{D}). Indeed, differentiating $ \Phi(t,z,a,b)=0$, $\Phi(\hat t,\hat z,a,b)=0$, and $\Phi(t, z,\hat a,\hat b)=0$ with respect to $t$ and eliminating $\hat t,\hat z,\hat a,\hat b$ as well as the coordinate $a,$ one can express second order derivatives of $z$ and $b$ in terms of their 1st and 0th jets with respect to $t$.  
\end{remark}
\begin{example}[Flat path geometry]
If \eqref{eq:sode} is trivial, i.e. $F=0$, then
\[
\Phi(t,z,a,b)=z-bt-a
\]
and $\gamma_{xy}$ are solutions to
\[
z''=0,\qquad b''=-\frac{2(b')^2}{z'-b}.
\]

\end{example}
\begin{example}[A self-dual path geometry]
Consider the scalar second order  \eqref{eq:sode} where $F=\sqrt{z'}.$ For this scalar ODE the induced 2D path geometry is equivalent to its dual. One obtains
\[
\Phi(t,z,a,b)= z-(\tfrac{1}{12}t^3+\tfrac 14 b t^2+\tfrac 14b^2t+a)
\]
and, therefore, $\gamma_{xy}$ are solutions to the pair of second order ODEs
\[
  \begin{gathered}    
    z''=\sqrt{z'},\\
    b''=\tfrac{1}{-b'(4z'-(t+b)^2)}\left(-2(b')^2(b'+1)^2(t+b)+4\sqrt{z'}(b')^2 +2(b')^3\sqrt{(t+b)^2b'(b'+1)-4z' b'}\right).
  \end{gathered}
\]

\end{example}
\begin{remark}\label{rmk:dancing-vs-chains-ODEs}
  Comparing the relation between the ODEs in Proposition \ref{prop:ODEs-danc-constr-2d}  to the case of chains, in the dancing construction it appears to be much more difficult to give a closed form of the pair of ODEs, unlike the pair \eqref{eq:chain-pair-ODEs} for chains. An underlying reason is that, as is mentioned in Remark \ref{rmk:dancing-not-variational},  3D path geometry of chains is variational in a \emph{canonical} way, unlike the path geometry defined by the dancing construction. The variationality of chains, which  manifests itself in the Fefferman construction (see \cite{MS-cone} for details), is key in the derivation of the pair of ODEs \eqref{eq:chain-pair-ODEs} as done in \cite{BW-chains}.
\end{remark}


\subsection{Characterization of the dancing construction}
\label{sec:characterization-dancing}
In this section we provide a characterization of 3D path geometries obtained via the dancing construction.

\begin{proposition}\label{prop1}
Let $(N,\scD_1,\scD_2)$ be a 2D path geometry on $M:=N\slash\scD_2$ with dual path geometry on $T=N/\scD_1$. Then all surfaces $\Sigma^1_x,\Sigma^2_y\subset N$, $x\in M$ and $y\in T$ are \emph{totally geodesic} in  the dancing construction, i.e. if a path is tangent  to $\Sigma^1_x$ or $\Sigma^2_y$ at a point, then it stays in $\Sigma^1_x$ or $\Sigma^2_y$, respectively.
\end{proposition}
\begin{proof}
This is a direct consequence of the constructions because all paths tangent to $\Sigma^1_x$ are of the form $\Sigma^1_x\cap \Sigma^2_y$, for some $y\in T$. Similarly, all paths tangent to $\Sigma^2_y$ are of the form $\Sigma^2_y\cap \Sigma^1_x$ for some $x\in M$.
\end{proof}

It follows from Proposition \ref{prop1} that surfaces $\Sigma^1_x$ and  $\Sigma^2_y$ for all  $x\in M$ and $y\in T$ are equipped with their own 2D path geometries. Hence, we arrive at the following proposition.

\begin{proposition}\label{prop2}
Let $(N,\scD_1,\scD_2)$ be a 2D path geometry on $M:=N\slash\scD_2$ with dual path geometry on $T:=N/\scD_1.$  For any $x\in M$, the restriction of the projection $\pi_M\colon N\to M$ to $\Sigma^1_x$  establishes the equivalence of  path geometries on $\Sigma^1_x$ and $M$. Similarly, for any $y\in T$ the restriction of the projection $\pi_T\colon N\to T$ to $\Sigma^2_y$  establishes the equivalence of path geometries on $\Sigma^1_x$  and $T$.
\end{proposition}
\begin{proof}
Fix $x\in M$ and the corresponding surface $\Sigma^1_x$. The path geometry on $\Sigma^1_x$ is defined by intersections of $\Sigma^1_x$ with all surfaces $\Sigma^2_y\subset N$, $y\in T$. The image of $\Sigma^2_y$ under the projection $\pi_M$ is a path on $M$ corresponding to $y\in T$. Therefore the dancing path $\gamma_{xy}:=\Sigma^1_x\cap \Sigma^2_y$ is projected to a path for the original path geometry on $M$.
Second part of the proposition is shown analogously.
\end{proof}

Now, we are ready to state our characterization of dancing path geometries. 

\begin{theorem}\label{thm1}
  Let $(Q,\scX,\scV)$  be a  3D path geometry on $N:=Q/\scV.$ Then $(Q,\scX,\scV)$  is a dancing path geometry if and only if there are two 2-parameter families $\{\Sigma^1_x\}_{x\in \tilde M}$ and $\{\Sigma^2_y\}_{y\in \tilde T}$ of totally geodesic surfaces in $N$ with the following properties.
  \begin{itemize}
\item[(a)] Exactly 1-parameter sub-family of surfaces from $\{\Sigma^1_x\}_{x\in \tilde M}$ and exactly 1-parameter sub-family of surfaces from $\{\Sigma^2_y\}_{y\in \tilde T}$ pass through any point of $N$.
\item[(b)] Tangent planes to all surfaces $\Sigma^1_x$ (respectively,  $\Sigma^2_y)$ which pass through a fixed point  of $N$ have a common line denoted as $\scD_1$ (respectively, $\scD_2.$)
\item[(c)] The rank 2 distribution $\scC:=\scD_1\oplus \scD_2\subset T N$ is contact.
\item[(d)] Path geometries on all surfaces $\Sigma^1_x$, $x\in \tilde M$, are equivalent to the 2D path geometry induced on $N\slash\scD_2$ whose paths are  the projection of the integral curves of  $\scD_1.$ Moreover, path geometries on all surfaces $\Sigma^2_y$, $y\in \tilde T$,  are equivalent to the 2D path geometry induced on $N\slash\scD_1$ whose paths are  the projection of $\scD_2.$
\end{itemize}
\end{theorem}
\begin{proof}
If a structure comes from the dancing construction then $\tilde T= T:=N/\scD_1$ and $\tilde M= M:=N/\scD_2$ and the conditions (a)-(c) are satisfied by definition (see Proposition \ref{prop0}). Furthermore, the surfaces $\Sigma^1_x$ and $\Sigma^2_y$ are totally geodesic due to Proposition \ref{prop1} and condition (d) is satisfied by Proposition \ref{prop2}. 

On the other hand, if conditions (a)-(c) are satisfied then the pair
$(\scD_1,\scD_2)$ is a 2D path geometry on $N$ by (b) and (c). The
claim follows if we prove that surfaces $\Sigma^1_x$ and $\Sigma^2_y$
are necessarily obtained via the dancing construction applied to
$(\scD_1,\scD_2)$. First we prove that $\Sigma^1_x\cap \Sigma^2_y$ is
a path of the original 3D path geometry for all $x$ and $y$. Indeed,
since both $\Sigma^1_x$ and $\Sigma^2_y$ are totally geodesic, any
path tangent to $\Sigma^1_x$ and $\Sigma^2_y$ at some point has to
stay simultaneously in both $\Sigma^1_x$ and $\Sigma^2_y$ at all time.
Hence, the path coincides with $\Sigma^1_x\cap \Sigma^2_y$.

Furthermore, by condition (b) any surface $\Sigma^1_x$ is foliated by
integral curves of $\scD_1.$ Hence, the projection of $\Sigma^1_x$ to
$N/\scD_1$ is a curve. By (d), this curve is necessarily a projection
of an integral curve of $\scD_2$, i.e. $\Sigma^1_x$ is defined as the
union of integral curves of $\scD_2$ crossing a choice of integral
curve of $\scD_1$. In this way we identify $\tilde T$ with $N/\scD_1$.
Similarly, we prove that $\tilde M$ can be identified with $N/\scD_2$ and that any $\Sigma^2_y$
is defined as a union of integral curves of $\scD_1$ crossing a
choice of integral curve of $\scD_2$.




\end{proof}


\subsection{Freestyling: a generalization of dancing} 
\label{sec:liberal-dancing-as}
Theorem \ref{thm1} suggests that one can consider a generalization of dancing path geometries  by dropping condition (d). Later on we shall provide  equivalent descriptions.
\begin{definition}\label{def:freestyle}
A  3D path geometry $(Q,\scX,\scV)$ on $N:=Q\slash\scV$ is a freestyling if it admits two 2-parameter families $\{\Sigma^1_x\}_{x\in \tilde M}$ and $\{\Sigma^2_y\}_{y\in \tilde T}$ of totally geodesic surfaces in $N$ with the following properties.
 \begin{itemize}
\item[(a)] Exactly 1-parameter sub-family of surfaces from $\{\Sigma^1_x\}_{x\in \tilde M}$ and exactly 1-parameter sub-family of surfaces from $\{\Sigma^2_y\}_{y\in \tilde T}$ pass through any point of $N$.
\item[(b)] Tangent planes to all surfaces $\Sigma^1_x$ (respectively,  $\Sigma^2_y)$ which pass through a fixed point  of $N$ have a common line denoted as $\scD_1$ (respectively, $\scD_2.$)
\item[(c)] The rank 2 distribution $\scC:=\scD_1\oplus \scD_2\subset T N$ is contact.
\end{itemize}
\end{definition}

Now we have  the following.
\begin{proposition}\label{prop:freestyle}
Any freestyling  on a 3-dimensional  manifold  $N$ with a contact distribution $\scC\subset TN$ uniquely determines, in sufficiently small open sets,  a triple of  2D path geometries  $(N,\scD_1,\scD_2)$, $(N,\tilde\scD_1,\scD_2)$, $(N,\scD_1, \tilde\scD_2)$ such that  $\scC=\scD_1\oplus\scD_2=\tilde\scD_1\oplus\scD_2=\scD_1\oplus\tilde\scD_2.$  
\end{proposition}
\begin{proof}
  By condition (c) in Definition \ref{def:freestyle}, a freestyling path
geometry defines a 2D path geometry $(N,\scD_1, \scD_2)$ on
$M:=N\slash\scD_2.$ Furthermore, $N$ is
equipped with two additional families of surfaces
$\{\Sigma^1_x\}_{x\in \tilde M}$ and $\{\Sigma^2_y\}_{y\in \tilde T}$.
For general freestyling $\tilde M$ and $\tilde T$ cannot be identified
with $M:=N/\scD_2$ and
$T:=N/\scD_1$, respectively. However, by conditions (a) and (b), for
any $x\in \tilde M$,
$\Sigma^1_x$ is foliated by integral curves of $\scD_2$, and for any
$y\in \tilde T$, $\Sigma^2_y$ is foliated by integral curves of
$\scD_1$. It follows that $\pi_T(\Sigma^1_x)$ and $\pi_M(\Sigma^2_y)$
are well-defined curves in the quotient spaces $M=N/\scD_2$ and
$T=N/\scD_2,$ respectively. The 2-parameter family of curves
$\{\pi_T(\Sigma^1_x)\}_{x\in \tilde M}$ defines an additional path
geometry on $T$
and the family $\{\pi_M(\Sigma^2_y)\}_{y\in \tilde T}$ defines an
additional path geometry on
$M$. Note that in this regard $\tilde T$ and $\tilde M$ become the
twistor spaces for the path
geometry on $M$ and $T$, respectively. To show the last part of the proposition, one notes that, in sufficiently small open sets, $N$ can be identified as $\PP TM,$ where $M=N\slash\scD_2,$ with canonical contact distribution $\scC$ splitting as $\scC=\scD_1\oplus\scD_2=\tilde\scD_1\oplus\scD_2.$ Viewing $N$ as  $\PP T(N\slash\scD_1),$ in sufficiently small neighborhoods, it follows that $\scC=\scD_1\oplus\scD_2=\scD_1\oplus\tilde\scD_2.$ 
\end{proof}
\begin{remark}\label{rmk:leaf-space-leaves}
In the case of the usual dancing construction one has $\tilde\scD_1=\scD_1$ and $\tilde\scD_2=\scD_2$.
Now, having three 2D path geometries $(N,\scD_1,\scD_2)$, $(N,\tilde\scD_1,\scD_2)$, $(N,\scD_1, \tilde\scD_2),$ we can repeat what we did in the case of the dancing construction, but use the pair of line fields $(\scD_1, \tilde\scD_2)$ and $(\tilde\scD_1, \scD_2)$  in order to define $\Sigma^1_x$ and $\Sigma^2_y,$ respectively. According to Proposition \ref{prop:freestyle}, this gives rise to a freestyling path geometry. Moreover, it follows that Proposition \ref{prop2} can be extended for freestyling in the sense that the 2D path geometries induced on $\tilde M$ and $\tilde T$ are equivalent to the induced 2D path geometries on any of the surfaces $\Sigma^2_y$ and $\Sigma^1_x,$ respectively. 
\end{remark}


\begin{remark}\label{rmk:freestyle}
Similarly to Remark \ref{rmk:dancing} one can express the pair of second order ODEs that corresponds to a freestyling. In this case the starting point would be three functions $\Phi_1\colon M\times T\to\R$, $\Phi_2\colon M\times \tilde T\to\R$, and $\Phi_3\colon \tilde M\times T\to\R$, where  $\tilde T=N/\tilde\scD_1$ and $\tilde M=N/\tilde\scD_2$ are two additional leaf spaces related to two additional path geometries  $(N,\tilde\scD_1,\scD_2)$ and $(N,\scD_1, \tilde\scD_2)$. Clearly, $N$ is identified simultaneously with subsets $\{\Phi_1=0\}$, $\{\Phi_2=0\}$, and $\{\Phi_3=0\}$ of $M\times T$, $M\times \tilde T$ and $\tilde M\times T$, respectively. 

By an abuse of notation by $(t,z)$ we denote coordinates on $M$ and $\tilde M$ and by $(a, b)$ coordinates on $T$ and $\tilde T$. It follows that  
\[
\begin{aligned}
&\Sigma^1_{\hat x}=\{(t,z, a,b)\in M\times T\ |\ \Phi_1(t,z,a,b)=0,\ \Phi_3(\hat t,\hat z,a,b)=0\},\\
&\Sigma^2_{\hat y}=\{(t,z,a,b)\in M\times T\ |\ \Phi_1(t,z,a,b)=0,\ \Phi_2(t, z,\hat a,\hat b)=0\},
\end{aligned}
\]
where  $\hat x=(\hat t,\hat z)$ is a fixed point in $\tilde M$ and $\hat y=(\hat a,\hat b)$ is a fixed point in $\tilde T$. Consequently, one obtains
\[
\gamma_{\hat x\hat y}=\{(t,z,a,b)\in M\times T\ |\ \Phi_1(t,z,a,b)=0,\ \Phi_3(\hat t,\hat z,a,b)=0,\ \Phi_2(t, z,\hat a,\hat b)=0\}
\]
are paths for the freestyling.
\end{remark}
\begin{center}
\begin{figure}%[!ht]%[h]%[ht]
  \includegraphics[width=.7\linewidth]{dancing-and-freestyling2.png}
  \caption{A visualization of the dancing paths on the left, and of the
freestyling on the right, viewed through the geometry on $M$. 
Points in $N$ are represented by black lines with distinguished points. A dancing path in $N$ is uniquely defined by a
green line and a red point that doesn't lie on this line. The line and
the point are projections of nonintersecting integral curves of
$\scD_1$ (green line) and of $\scD_2$ (a red point). A dancing path
consists of lines passing through the red point with marked points
defined as intersection points with the green line.
In the freestyling a green line is replaced by a curve from the
additional path geometry on $M$. Similarly, instead of the family of
lines passing through the red point we have a 1-parameter family
parameterized by a curve from the additional path geometry on $T$
(whose projection to $M$ is no longer a point in $M$ but a red cusp).}
\end{figure}
\end{center}
\begin{example}
  Consider a freestyling defined by three 2D path geometries such that
two of them: $(N,\scD_1,\scD_2)$ and $(N,\scD_1,\tilde \scD_2)$ are
flat, but the third one $(N,\tilde\scD_1, \scD_2)$ is arbitrary,
defined by a general second order ODE $z''=F(t,z,z')$. We have
\[
\Phi_1(t,z,a,b)=z-at-b,\qquad \Phi_3(\hat t,\hat z,a,b)=\hat z-b\hat t-a.
\]
On the other hand, by definition, the solution function
$\Phi_2=\Phi(t,z,\hat a,\hat b)$ for $(N,\tilde\scD_1, \scD_2)$ is a
function such that differentiating $\Phi_2$ twice with respect to $t$
and expressing the constants of integration $(\hat a,\hat b)$ with
respect to $t,z,z'$ one gets $z''=F(t,z,z')$.
Now, assuming $a$ and $b$ are functions of $t$ and differentiating
$\Phi_3$ with respect to $t$, we express constants $(\hat t,\hat z)$
as
\[
\hat t=-\tfrac{a'}{b'},\qquad\hat z=a-b\tfrac{a'}{b'}
\]
One more differentiation of $\Phi_3$ gives
\begin{equation}\label{eq_btt}
b''=a''\tfrac{a'}{b'}.
\end{equation}
Further, differentiating $\Phi_1$ with respect to $t$ and exploiting
$z''=F(t,z,z')$ we are able to compute $a$, $a'$ and $a''$ in terms of
$t$, $z$ and $b$ and their derivatives with respect to $t$. Indeed, we
get
\begin{equation}\label{eq_atatt}
a'=z'-b't-b,\qquad a''=F(t,z,z')-b''t-2b'.
\end{equation}
Combining \eqref{eq_btt} with \eqref{eq_atatt} we get
$b''=\frac{(F-2b')b'}{z'-b}$. Summarizing, the 3D freestyling path
geometry is described by the system
\[
z''=F(t,z,z'),\qquad b''=\tfrac{b'}{z'-b}\left(F(t,z,z')-2b'\right).
\]
\end{example}


\subsection{3D path geometries arising from dancing and freestyling} 
\label{sec:path-geom-aris}

Let $(Q,\scV,\scX)$  be a 3D  path geometry on $N:=Q\slash\scV$. In this section our objective is to characterize those triples $(Q,\scV,\scX)$ that define a freestyling by exploiting Theorem \ref{thm:path-geom-cartan-conn}. Thus, we need to interpret the properties of Definition \ref{def:freestyle}, expressed at the level of  the 3-dimensional manifold $N$, on the 5-dimensional manifold $Q$. 
\begin{theorem}\label{thm2}
Let $(Q,\scV,\scX)$ be a  3D path geometry on $N=Q/\scV.$ Then $(Q,\scV,\scX)$ is  freestyling if and only if the vertical bundle $\scV$ has a splitting $\scV=\scV_1\oplus \scV_2$ with the following properties. 
\begin{itemize}
\item[(a)] The distributions
\[
\scB_1= \spn\{\scV_1,\scX, [\scV_1,\scX]\},\qquad \scB_2= \spn\{\scV_2,\scX, [\scV_2,\scX]\}
\]
are integrable.
\item[(b)] Rank-3 distributions $\scB_1$ and $\scB_2$ have rank-2 integrable sub-distributions $\scK_1$ and $\scK_2$ containing $\scV_1$ and $\scV_2$, respectively.
\item[(c)] The projections of $\scK_1$ and $\scK_2$ to $N$ span a contact distribution.
\end{itemize}
\end{theorem} 
\begin{proof}
Surfaces $\Sigma^1_x,x\in M$  and $\Sigma^2_y,y\in T$ are the projection  of the leaves of $\scB_2$ and $\scB_1$ to $N$. Then, the line fields $\scV_i\subset TQ$, $i=1,2$, project to $\scD_i$, $i=1,2$ on $N$.  The only nontrivial observation needed here is that condition (a) implies that any leaf of $\scB_1$ or $\scB_2$  project to a totally geodesic surface in $N$. This follows from the fact that the paths of the 3D path geometry are defined as projections of integral curves of $\scX$ to $N$. Since $\scB_1$ and $\scB_2$ are integrable, the paths that are tangent to the surface at a point stay in the surface.
\end{proof}




\begin{remark}
We point out that the three parts of Theorem \ref{thm2} can provide an alternative definition of  freestyling instead of the one given in Definition \ref{def:freestyle}. Moreover, following  \cite[Section 4.2]{KM-Cayley}, one can show that condition (1) in the theorem above implies that $\scV_1$ and $\scV_2$ are eigenspaces of the torsion of the 3D path geometry viewed as a trace-free $2\times 2$-matrix acting on $\scV.$ 
\end{remark}
\begin{proposition}\label{prop:torsion-curv-type-freestyling-reduction}
Any freestyling defines a principal $B$-subbundle $\iota\colon\cG_{T_{G_r}}\hookrightarrow\cG,$  where $B\subset\mathrm{GL}(2,\RR)$ is the Borel subgroup, over which the following invariant conditions are satisfied.
  \[\iota^* A_0=\iota^*A_2=0,\quad \iota^*W_2=1,\]
and the components of $\iota^*\psi$, as given in \eqref{eq:path-geom-cartan-conn-3D}, satisfy
  \[
    \begin{aligned}
      \iota^*\psi^1_2\equiv& 0&&\mod \{\iota^*\alpha^0,\iota^*\alpha^2,\iota^*\beta^2\},\\
      \iota^*\psi^2_1\equiv& 0&&\mod \{\iota^*\alpha^0,\iota^*\alpha^1,\iota^*\beta^1\},\\
      \iota^*\psi^2_2\equiv& -\iota^*\psi^1_1&&\mod \{\iota^*\alpha^0,\iota^*\alpha^1,\iota^*\alpha^2\}\\
      \iota^*\nu_1,\iota^*\mu_1,\iota^*\nu_2,\iota^*\mu_2\equiv & 0&&\mod  \{\iota^*\alpha^0,\iota^*\alpha^1,\iota^*\alpha^2,\iota^*\beta^1,\iota^*\beta^2\}
    \end{aligned}
  \]
    As a result, given a freestyling path geometry $(\cG\to Q,\psi)$, its curvature has at least two distinct real roots whose multiplicities are at most two and  its torsion is either zero or has two distinct real roots. 
\end{proposition}
\begin{proof}
  By Theorem \ref{thm2}, a freestyling gives a splitting  $\scV=\scV_{1}\oplus\scV_{2}.$ Since the structure group of a 3D path geometry is $P_0=\RR^*\times\mathrm{GL}(2,\RR),$ having a splitting reduced $P_0$ to $(\RR^*)^3$ in order for the splitting to be preserved.  Thus, similarly to Proposition \ref{prop:type-D-curv-chains}, such a splitting gives an inclusion $\iota_1\colon\cG^{(1)}\to \cG,$ where $\cG^{(1)}$ is a principal $P^{(1)}$-bundle with $P^{(1)}:=(\RR^*)^3\ltimes P_+$ over which 
  \begin{equation}
    \label{eq:psi12-red-freestyling}
    \iota_1^*\psi^1_2=A^1_{2i}\alpha^i+B^1_{2a}\beta^a,\quad     \iota_1^*\psi^2_1=A^2_{1i}\alpha^i+B^2_{1a}\beta^a 
  \end{equation}
  for some functions $A^a_{bi}$ and $B^a_{bc}$ on $\cG^{(1)}.$   In terms of the Cartan connection \eqref{eq:path-geom-cartan-conn-3D} one has $\scV_a=\span\{\tfrac{\partial}{\partial s^*\beta^a}\}$ and $\scX=\span\{\tfrac{\partial}{\partial s^*\alpha^0}\}$ for any section $s\colon Q\to\cG^{(1)}.$ This implies  that $[\scX,\scV_a]=\span\{\tfrac{\partial}{\partial s^*\alpha^a}\}.$  As a result, for the rank 3 integrable distributions in \ref{thm2} one has $\scB_a=\ker \cI_a$  where $\cI_1:=\{\alpha^2,\beta^2\}$ and  $\cI_2:=\{\alpha^1,\beta^1\}.$ Using the structure equations, one immediately obtains, in case the torsion $\bT$ is nonzero, that the integrability of $\iota_1^*\cI_{1}$ and $\iota_1^*\cI_{2}$  implies
  \begin{equation}
    \label{eq:A02-zero-freestyling}  
    \iota_1^*A_0=0,\quad \iota_1^*A_2=0.
  \end{equation}
Moreover, as a result of  differential relations arising from the integrability of $\iota_1^*\cI_{1}$ and $\iota_1^*\cI_{2},$ it follows that
  \begin{equation}
    \label{eq:W04-zero-freestyling}
    \iota_1^*W_0=0,\quad \iota^*_1W_4=0.
  \end{equation}
By part (b) of Theorem \ref{thm2} the integrability of $\scK_1$ and $\scK_2$ are required for freestyling. In order to impose these integrability conditions, as in Proposition \ref{prop:type-D-curv-chains}, an additional reduction to a principal $P^{(2)}$-bundle $\iota_2\colon\cG^{(2)}\hookrightarrow\cG^{(1)}$ is required  where $P^{(2)}\cong (\RR^*)^3\ltimes\RR$ and 
  \begin{equation}
    \label{eq:P2-2nd-reduction-freestyling}
        \cG^{(2)}=\{u\in\cG^{(1)}\ \vline\ B^1_{21}(u)=B^2_{12}(u)=A^1_{21}(u)=A^2_{12}(u)=0\}.
      \end{equation}
      As in Proposition \ref{prop:type-D-curv-chains}, one has $\iota_2^*\mu_1,\iota_2^*\mu_2,\iota_2^*\nu_1,\iota_2^*\nu_2$ are zero modulo $\{\iota^*_2\alpha^0,\iota^*_2\alpha^1,\iota^*_2\alpha^2,\iota^*_2\beta^1,\iota^*_2\beta^2\}.$
As a result, the integrability of $\cI_3=\mathrm{Ann}\scK_1$ and $\cI_4=\mathrm{Ann}\scK_2$ on $\cG^{(2)}$ is well-defined. It is matter of straightforward computation to show that if $\cI_3$ and $\cI_4$ are integrable  on $\cG^{(2)}$ then
\[\exd\alpha^0\equiv W_2\alpha^1\w\alpha^2\mod\{\alpha^0\},\]
wherein we have suppressed  the pull-backs.  As a result, condition (c) in Theorem \ref{thm2} implies that  for a freestyling
\begin{equation}
  \label{eq:W2-zero-freestyling}
  \iota_2^*W_2\neq 0
  \end{equation}
  which combined with \eqref{eq:A02-zero-freestyling} and \eqref{eq:W04-zero-freestyling}  shows that the quartic $\bC$ has at least two real roots whose multiplicity is at most two.

  Lastly, using \eqref{eq:W2-zero-freestyling}, there is a third reduction of the structure bundle to a principal $P^{(3)}$-bundle $\iota_3\colon\cG^{(3)}\hookrightarrow\cG^{(2)}$   where $P^{(3)}= (\RR^*)^2\ltimes\RR\cong B$ where $B\subset\mathrm{GL}(2,\RR)$ is the Borel subgroup defined as
  \begin{equation}
    \label{eq:P3-3rd-reduction-freestyling}
        \cG^{(3)}=\{u\in\cG^{(2)}\ \vline\ W_2(u)=1\}.
  \end{equation}
  where we have assumed $W_2(u)> 0$ (see Remark \ref{rmk:sign-of-W_2-chains} for a discussion on the sign of $W_2$ which remains equally valid here as well.) On $\cG^{(3)}$ one has $\iota^*_3(\psi^2_2+\psi^1_1)$ is semi-basic with respect to the fibration $\cG^{(3)}\to Q.$  The finish the proof where  $\cG_{T_{G_r}}:=\cG^{(3)}$ and $B=(\RR^*)^2\ltimes\RR$ and $\iota:=\iota_3\circ\iota_2\circ\iota_1.$ The full structure equations for the $\{e\}$-structures $\iota^*\psi$ on $\cG_{T_{G_r}}$ is given in \eqref{eq:freestyling-streqs} for some quantities $x_i$'s on $\cG_{T_{G_r}}$ where
  \[\iota^*\psi^2_2=-\iota^*\psi^1_1+4\iota^*x_{15}\alpha^0+4\iota^*x_{11}\alpha^1-4\iota^*x_{12}\alpha^2.\]
\end{proof}
\begin{remark}
  We point out that the subscript $T_{G_r}$ is chosen to point out that the fact that the algebraic type of the torsion $\bT,$ represented as the quadric \eqref{eq:quadric-quartic}, is \emph{general} with two real roots, in analogy with subscript $D_r$  mentioned in Remark \ref{rmk:D-r-curvature-type}.
Moreover, in comparison with Proposition \ref{prop:type-D-curv-chains}, the same reductions were carried out in the proposition above although in a different order. Since the proof is almost identical to that of  Proposition \ref{prop:type-D-curv-chains}, we skipped some of the  details. 
We point out that, unlike Proposition \ref{prop:type-D-curv-chains},  the principal $B$-bundle $(\cG_{T_{G_r}}\to Q,\iota^*\psi)$ does not  defines a Cartan geometry, as can be seen from the non-horizontal 2-forms in the structure equations for $\exd\mu_0$ in \eqref{eq:freestyling-streqs}. We point out that the principal $\RR^*\times B$-bundle $(\cG^{(2)}\to Q,(\iota_2\circ\iota_1)^*\psi)$ does define a Cartan geometry. However, the third reduction that involves normalizing $W_2$ breaks the equivariancy. It is a matter of straightforward computation to show that if one requires such structures to define a Cartan connection, i.e. $x_{11}=x_{12}=x_{15}=0$ then the freestyling is torsion-free, which is the case treated in Proposition \ref{prop:torsionfree-freestyling}.
\end{remark}

\begin{remark}  \label{rmk:dancing-Not-Variational}
Another outcome of Proposition \ref{prop:torsion-curv-type-freestyling-reduction} is that projection of the paths of any freestyling to $N$ is always transversal to the contact distribution. This is due to the fact that the path are tangent to the line field spanned by $\tfrac{\partial}{\partial\alpha^0}$ and that $\alpha^0$ is a contact form on $N.$ Lastly, in the language of \cite{MS-cone}, $(\cG^{(2)}\to Q,(\iota_2\circ\iota_1)^*\psi)$ defines an \emph{orthopath geometry} since the 2-form $\alpha^1\w\beta^2+\alpha^2\w\beta^1\in\Omega^2(\cG^{(2)})$ is well-defined which also satisfies
  \[\exd\rho=-2(\psi^1_1+\psi^2_2)\w\rho+ A_1\alpha^0\w\alpha^1\w\alpha^2.\]
  As a result, such structures define an \emph{variational} orthopath geometry, i.e. $\exd\rho\equiv 0$ modulo $\{\rho\}$ if they are torsion-free which is treated in the next proposition.
\end{remark}

As mentioned in the two remarks above, torsion-free freestyling is very special.
\begin{proposition}\label{prop:torsionfree-freestyling}
  The following are equivalent.
  \begin{enumerate}
  \item The torsion of a freestyling is zero.
  \item The curvature of a freestyling has two distinct real roots of multiplicity two.    
  \item A freestyling  is locally equivalent to the chains of the flat 2D path geometry.
  \end{enumerate}
\end{proposition}
\begin{proof}
Let us show $(1)\to (2):$  Using the  $\{e\}$-structure obtained in Proposition \ref{prop:torsion-curv-type-freestyling-reduction} and the structure equations \eqref{eq:freestyling-streqs}, it follows that there is only one fundamental invariant one obtains  differential relations
\begin{equation}
  \label{eq:W13-A1}
  W_3=-\tfrac{1}{6}A_{1;\underline{2221}},\quad W_1=\tfrac{1}{6}A_{1;\underline{1112}}.
  \end{equation}
  Thus,  $A_1=0$ implies $W_1=W_3=0.$

To show $(2)\to (1)$ we note that the conditions $W_1=W_3=0$ are invariantly defined on $\cG_{T_{G_r}}.$ Assuming  $W_1=W_3=0$, it is a matter of tedious but straightforward calculations to show that identities arising from $\exd^2=0$ and the resulting differential relations  (in fact three iterations are needed) imply $A_1=0.$  

Showing $(1)\leftrightarrow (3)$ is a straightforward computation as shown in Proposition \ref{prop:type-D-curv-chains} and Theorem \ref{thm:2d-path-geometries-generalized-chains}.  Starting from (1), since it implies the curvature has two distinct real roots of multiplicity 2, one can check all the necessary and sufficient conditions are satisfied and the claim follows. Conversely, starting from the chains of the flat 2D path geometry, by the discussion in  \ref{sec:an-overview-chains} one can easily check that the conditions in Theorem \ref{thm2} are satisfied on $\cG_{D_r}$ and moreover  $\bT=0$ for such freestyling.
\end{proof}
\begin{remark}
  The proposition above generalizes the result in \cite[Theorem 2.3]{D} where it is shown that   a dancing path geometry is torsion-free if and only if it corresponds to the chains of the flat 2D path geometry.  Moreover, from Propositions \ref{prop:torsion-curv-type-freestyling-reduction} and \ref{prop:torsionfree-freestyling} it follows in the $\{e\}$-structure $(\cG_{T_{G_r}}\to Q,\iota^*\psi)$  the fundamental invariant is $A_1$ whose vanishing implies $W_1,W_3$ and all $x_i$'s in structure equations \eqref{eq:freestyling-streqs} are zero and hence one arrives at the  ``flat model''  $(\mathrm{SL}(3,\RR)\to \mathrm{SL}(3,\RR)\slash B)$ where $B\subset \mathrm{GL}(2,\RR)$ is the Borel subgroup, i.e. the chains of the flat 2D path geometry. 
\end{remark}

\begin{corollary}\label{cor:3d-path-geometries-dancing-chains}
  The chains of a 2D path geometry define a freestyling if and only if the 2D path geometry is flat. 
\end{corollary}
\begin{proof}
This follows from Proposition \ref{prop:torsionfree-freestyling} and the fact that by Theorem \ref{cor:3d-path-geometries-chain-characterization} the curvature of 3D path geometry of chains always has two distinct real roots of multiplicity two.
\end{proof}
\begin{remark}\label{rmk:dancing-not-variational}
  As was mentioned in Remark \ref{rmk:dancing-Not-Variational}, a freestyling defines  a canonical orthopath geometry which is variational if and only if the freestyling is torsion-free. As pointed out in  Remark \ref{rmk:3d-path-geometries-chain-general}, chains and more generally the class of path geometries considered in Theorem \ref{thm:2d-path-geometries-generalized-chains} always define variational orthopath geometries.  As a result  freestyling path geometries and the class of path geometries introduced in Theorem \ref{thm:2d-path-geometries-generalized-chains} only have one path geometry in common which is the chains of the flat 2D path geometry. As a result, chains and the dancing construction associate  a canonical 3D path geometry to any 2D path geometry whose paths are transversal to the contact distribution (see \ref{sec:an-overview-chains} for chains and Remark \ref{rmk:dancing-Not-Variational} for dancing and freestyling for the transversality property)  which only coincide for the flat 2D path geometry, but otherwise are never equivalent. 
\end{remark}
Ruling out the special case of torsion-free freestyling, we have the following. 
\begin{proposition}\label{pro:3d-path-geometries-freestyling-inv-char}
  A 3D path geometry arises as freestyling with non-zero torsion if and only if 
  \begin{enumerate}
  \item The torsion $\bT$  has two distinct real roots and the quartic $\bC$ has at least two distinct real roots and at most one real root of multiplicity two.
  \item  There is a reduction $\iota\colon\cG_{T_{G_r}}\to\cG$ characterized  as in Proposition \ref{prop:torsion-curv-type-freestyling-reduction} such that the Pfaffian systems $\iota^*\cI_1,\iota^*\cI_2,\iota^*\cI_3,\iota^*\cI_4$ are integrable where
    \begin{equation}
      \label{eq:I_1234-freestyling}
      \cI_1=\{\alpha^1,\beta^1 \},\quad \cI_2=\{\alpha^2,\beta^2\},\quad \cI_3=\{\alpha^1,\alpha^0 \},\quad \cI_4=\{\alpha^2,\alpha^0\}.
    \end{equation}
  \end{enumerate}
  Defining $\tilde T,\tilde M,M,T$ to be the local leaf spaces of $\cI_1,\cI_2,\cI_3,\cI_4,$ respectively,  as in Proposition \ref{prop:freestyle}, they are equipped with 2D path geometries corresponding to $(N,\scD_1,\scD_2),$ $(N,\tilde\scD_1,\scD_2)$ and $(N,\scD_1,\tilde \scD_2)$  with  respective double fibrations
  \begin{equation}
    \label{eq:N-respective-fibrations}
    M\leftarrow N\rightarrow T,\qquad M \leftarrow  N\rightarrow \tilde T,\qquad \tilde M\leftarrow   N\rightarrow T.
  \end{equation}
  Then the invariants of the path geometry  $(N,\scD_1,\scD_2)$ depend on the 6th jet of $A_1$ and the invariants of the other two path geometries depend on the 5th jet of $A_1.$   
\end{proposition}

\begin{proof}
  The proof of the first part of the corollary directly follows from  Proposition \ref{prop:torsion-curv-type-freestyling-reduction} and  Proposition \ref{prop:torsionfree-freestyling}. By Theorem \ref{thm2} and Proposition \ref{prop:torsion-curv-type-freestyling-reduction} the surfaces $\tilde T,\tilde M,M$ and $T$ defined in Proposition \ref{prop:freestyle} are the leaf spaces of $\cI_1,\cI_2,\cI_3$ and $\cI_4$, respectively. The last part of the corollary is shown by direct computation using  structure equations \eqref{eq:freestyling-streqs} as outlined below.

  Let us define $N$ to be the leaf space of the Pfaffian system  $\{\alpha^0,\alpha^1,\beta^1\}$ with double fibration $M\leftarrow N\rightarrow \tilde T.$ By structure equations \eqref{eq:freestyling-streqs} one has
  \[
    \begin{aligned}
      \exd\alpha^1&\equiv -(\phi^1_1+2\phi^2_2)\w\alpha^1+\alpha^0\w\beta^1\\
      \exd\alpha^0&\equiv -(2\phi^2_2+\phi^1_1)\w\alpha^0\mod\{\alpha^1\}\\
      \exd\beta^1&\equiv -(\phi^2_2-\phi^1_1)\w\beta^1\mod \{\alpha^1\}.
    \end{aligned}
  \]
  where
  \begin{equation}
    \label{eq:psi1-2-UHF-1}
    \phi^1_1=\tfrac 43(x_{11}\alpha^1-x_{12}\alpha^2+x_{15}\alpha^0)+\psi^0_0-\tfrac 13\psi^1_1\qquad \phi^2_2=\tfrac 43(x_{11}\alpha^1-x_{12}\alpha^2+x_{15}\alpha^0)+\tfrac 23\psi^1_1.
      \end{equation}
      Using the structure equations \eqref{eq:freestyling-streqs}, the Cartan connection \eqref{eq:2D-path-geom-cartan-conn} for  the induced 2D path geometry on the 2-dimensional leaf space $M$ of the Pfaffian system $\{\alpha^1,\alpha^0\}$ is given below wherein   $\phi^1_1$ and $\phi^2_2$ are given as in \eqref{eq:psi1-2-UHF-1} and
  \begin{equation}
    \label{eq:UHF-1-2Dpath-Cartan-Conn}
      \def\arraystretch{1.3}
    \phi=
    \begin{pmatrix}
      -\phi^1_1-\phi^2_2 & \theta_2 &\theta_0 \\
      \alpha^0 & \phi^1_1 & \theta_1\\
      \alpha^1 & \beta^1 & \phi^2_2
    \end{pmatrix}\qquad 
    \left\{
    \begin{array}{ll}
          \theta_2=-x_{18}\alpha^2-\tfrac 53x_4\alpha^1-A_1\alpha^0+\mu_0 ,\\
          \theta_1=\tfrac 43 W_1\alpha^1+\alpha^2,\\
          \theta_0=8x_{15}\alpha^2-\tfrac 83 x_{14}\alpha^1-\tfrac{16}{3}x_4\alpha^0-\tfrac 43W_1\beta^1-\beta^2,
        \end{array}\right.
      \end{equation}
      and the fundamental invariants  are
      \begin{equation}
        \label{eq:T1-C1-tM-T}
        T_1=\tfrac 43 A_1A_{1;\underline{121}}-\tfrac 49A_{1;\underline 1}A_{1;\underline{12}}+2A_{1;1}-\tfrac 23A_{1;\underline 1 0},\qquad C_1=\tfrac 29A_{1;\underline{11121}}
      \end{equation}
      which implies $T_1$ and $C_1$ depend on the 3rd and 5th jets of $A_1,$ respectively.
      
      Similarly, the leaf space of $\{\alpha^0,\alpha^2,\beta^2\}$ corresponds to the  2D path geometry on the 2-dimensional leaf space $T$ of the Pfaffian system $\{\alpha^0,\alpha^2\}.$ The Cartan connection \eqref{eq:2D-path-geom-cartan-conn} for this 2D path geometry can be obtained analogously wherein $\omega^0=\alpha^2,$$\omega^1=\alpha^0,$ $\omega^2=\beta^2$ and its fundamental invariants are given as
      \[T_1=-\tfrac 43A_1A_{1;\underline{212}}+\tfrac 23A_{1;\underline{2}}A_{1;\underline{21}} -2A_{1;2}+\tfrac 23 A_{2;\underline 2 0},\qquad C_1=\tfrac 29 A_{1;\underline{22212}}.\]
      Lastly, as was shown previously, the leaf space of the Pfaffian system $\{\alpha^0,\alpha^1,\alpha^2\}$ corresponds to a 2D path geometry on the 2-dimensional leaf space of $\{\alpha^0,\alpha^1\}$ for which one can find the Cartan connection \eqref{eq:2D-path-geom-cartan-conn} wherein $\omega^0=\alpha^0,$  $\omega^1=\alpha^1$ and $\omega^2=\alpha^2.$ The expression for the fundamental invariants of this 2D path geometry is much longer that the other two cases using which one can easily check that both fundamental invariants $T_1$ and $C_1$ depend on the 6th jet of $A_1.$
\end{proof}
\begin{remark}\label{rmk:integrability-soln-space-freestyling}
One can interpret the freestyling construction in terms of an induced geometric structure on the solution space of the corresponding pair of ODEs, i.e. the leaf space of $\{\alpha^1,\alpha^2,\beta^1,\beta^2\}.$ It is shown in \cite[Theorem 4.4.1]{Omid-Thesis}, that 3D path geometries are in one-to-one correspondence with 4-dimensional \emph{half-flat causal structures}. A causal structure is a \emph{Finslerian generalization} of pseudo-Riemannian conformal structures in the sense that  the field of \emph{null cones} of codimension one with Gauss map of maximal rank defined on a 4-manifold are not necessarily quadratic. Half-flatness implies the existence of a 3-parameter family of null surfaces which can also be defined in terms of a Lax pair that is the integrable Pfaffian system $\cJ:=\{\alpha^0,\alpha^1,\alpha^2\}$ in terms of the structure equations \eqref{eq:freestyling-streqs}. However, half-flat causal structures arising from freestyling have the additional property that the Lax pair is extendable to a Lax triple in two distinct ways, i.e. the integrable Pfaffian systems $\{\alpha^0,\alpha^1\}$ and $\{\alpha^0,\alpha^2\}$. In other words, their Lax pair extends to a pair of 3-parameter family of \emph{null hypersurfaces} in the 4-manifold  which means the annihilator of the tangent space of the hypersurface is null with respect to the  conformal structure $[h]$ of degenerate bilinear form  \eqref{eq:degenerate-conformal-str}
 on the 5-dimensional leaf space $Q$ of $\{\alpha^0,\alpha^1,\alpha^2,\beta^1,\beta^2\}.$   Moreover, the integrability of the Pfaffian systems $\cI_1=\{\alpha^1,\beta^1\}$ and $\cI_2=\{\alpha^2,\beta^2\}$ define two distinct null congruences, i.e. surfaces whose tangent planes are null with respect to $[h]$ in \eqref{eq:degenerate-conformal-str}. Borrowing  the terminology used in \cite{Calderbank},   a half-flat causal structure with the property that the leaf spaces of null congruences $\cI_1$ and $\cI_2$ are equipped with a 2D path geometry is referred to in \cite{KM-Cayley}  as being ultra-half-flat causal structures. If the freestyling is torsion-free, i.e. its paths are the chains of flat 2D path geometry, the induced half-flat causal structure descends to a half-flat conformal structure on the 4-dimensional solution space and the conformal structure $[h]$ coincides with the so-called \emph{dancing metric} \cite{Bor,D}.

Lastly we point out that applying Cartan-K\"ahler machinery to the structure equations \eqref{eq:freestyling-streqs}, one obtains that the local generality of freestyling  depends on 3 functions of 3 variables. In comparison with Theorem \ref{thm:2d-path-geometries-generalized-chains}, although the the jet order of the fundamental invariants of the induced 2D path geometries are different, the locally generalities are the same. 
\end{remark}



\begin{remark}
We point out that by Remark \ref{rmk:leaf-space-leaves} the leaf space of the Pfaffian systems $\cI_1=\{\alpha^0,\alpha^1,\beta^1\}$ and $\cI_2=\{\alpha^0,\alpha^2,\beta^2\}$ in the 5-manifold $Q$ can be identified with any of the  leaves of the Pfaffian systems $\cJ_1=\{\alpha^2,\beta^2\}$ and $\cJ_2=\{\alpha^1,\beta^1\},$ respectively,  and the induced path geometries are equivalent. This is clear from the obtained Cartan connection in the proof of Corollary  \ref{cor:3d-path-geometries-dancing}. For instance, setting $\alpha^2=0$ and $\beta^2=0$ amounts to restricting to a leaf of $\cJ_1$ and thus the  the Cartan connection of the induced 2D path geometry for this leaf is given as \eqref{eq:UHF-1-2Dpath-Cartan-Conn} after setting $\alpha^2=0$ and $\beta^2=0.$ 
\end{remark}




As was discussed in Proposition \ref{prop:torsionfree-freestyling}, if the curvature has two distinct real roots of multiplicity two then the freestyling arises as the chains of the flat 2D path geometry. More generally we have the following. 
\begin{proposition}\label{prop:3d-path-geometries-dancing-projective}
Given a freestyling, if  the curvature has a  repeated root of multiplicity 2 then the 2D path geometry induced on $M$  is either a  projective or co-projective structure. 
\end{proposition}
\begin{proof}
Similarly to the proof  $(2)\to (1)$ in Proposition \ref{prop:torsionfree-freestyling}, the proof is  done by direct computation by checking $\exd^2=0$ and its differential consequences after setting either $W_1=0$ or $W_3=0.$ 
\end{proof}
Now we can state an invariant characterization of dancing path geometries.
\begin{corollary}\label{cor:3d-path-geometries-dancing}
  A 3D path geometry  with non-zero torsion arises from a dancing construction if and only if 
  \begin{enumerate}
  \item The quadric $\bT$  has two distinct real roots and the quartic $\bC$ has  at least two distinct real roots and no repeated roots.
  \item  There is a reduction $\iota\colon\cG_{T_{G_r}}\to\cG$ characterized  as in Proposition \ref{prop:torsion-curv-type-freestyling-reduction} such that the Pfaffian systems $\iota^*\cI_1,\iota^*\cI_2,\iota^*\cI_3,\iota^*\cI_4$ given in \eqref{eq:I_1234-freestyling} are integrable 
  \item   Defining $\tilde T,\tilde M,M,T$ to be the local leaf spaces of $\cI_1,\cI_2,\cI_3,\cI_4,$ respectively,  as in Proposition \ref{prop:freestyle}, they are equipped with 2D path geometries corresponding to $(N,\scD_1,\scD_2),$ $(N,\tilde\scD_1,\scD_2)$ and $(N,\scD_1,\tilde \scD_2)$  with respective  double fibrations  \eqref{eq:N-respective-fibrations}.
    The triple 2D path geometries defined by $N$ are equivalent. 
\end{enumerate}
\end{corollary}
\begin{proof}
By our discussion in \ref{sec:liberal-dancing-as}, a dancing path geometry is a freestyling for which, by Theorem \ref{thm1}, the triple path geometries on $N$ are equivalent, i.e. condition (3) holds. Consequently, conditions (1) and (2) in Proposition \ref{pro:3d-path-geometries-freestyling-inv-char} remain valid for dancing path geometries. To finish the proof we need to show that  the quartic $\bC$ cannot have a  repeated root. Using the reduction in Proposition \ref{prop:torsion-curv-type-freestyling-reduction}, a repeated root of the quartic $\bC$ has at most multiplicity two and is either at zero or at infinity.  Assume $W_0=W_1=W_4=0,$ i.e. $\bC$ has a repeated root at zero. Using  relation \eqref{eq:W13-A1}, one obtains that $C_1$ in \eqref{eq:T1-C1-tM-T} vanishes. Thus, by  Theorem \ref{thm:2D-path-geome} and our discussion after Theorem \ref{prop:3D-path-geom} about vanishing of $C_1$, the double fibration $M\leftarrow N\rightarrow \tilde T$ arising from the 2D path geometry $(N,\tilde\scD_1,\scD_2)$ in Proposition \ref{pro:3d-path-geometries-freestyling-inv-char} induces a projective structure on $M.$  Moreover, as in the proof of  Proposition \ref{prop:3d-path-geometries-dancing-projective}, direct computation shows if $W_0=W_1=W_4=0,$ then the double fibration $M\leftarrow N\to T$ arising from the path geometry $(N,\scD_1,\scD_2)$ induces a  co-projective structure on $M$.  However, in a dancing path geometry  the triple of path geometries defined by $N$ are always equivalent.   This implies that the induced path geometry on $M$ is both projective and co-projective and, therefore, has to be locally flat. As a result, by Proposition \ref{prop:torsionfree-freestyling} the dancing path geometry is torsion-free. Hence, if a dancing path geometry has non-zero torsion, then the quartic $\bC$  cannot have a repeated root at zero. Similarly, the case  $W_0=W_3=W_4=0$ is treated which finished the proof.  
\end{proof}
As mentioned in the proof of Corollary \ref{cor:3d-path-geometries-dancing},  the last part of  condition (1), which states that the quartic cannot have any repeated root, follows from condition (3). Nevertheless, we have included it in condition (1) since as a necessary condition it can be checked easily. Also, the condition that the quadric $\bT$ has to distinct real roots follows from condition (2), but we have included it as a necessary condition for the same reason. Note that unlike conditions (1) and (2), in general it is not easy to check condition (3) in Corollary \ref{cor:3d-path-geometries-dancing} for a pair of path geometries.

\begin{remark}\label{rmk:higher-dimensional-dancing}
The definition of the dancing construction, as given in \ref{sec:an-overview-dancing}, is applicable  to any \emph{integrable Legendrian contact structure}, also known as para-CR structure, denoted as $(N,\scD_1,\scD_2),$ where $N$ is a contact manifold of dimension $2n+1$ and $\scD_i$'s are integrable Legendrian subspaces that give a splitting of the contact distribution. The two family of hypersurfaces $\Sigma^1_x,\Sigma^2_y\subset N$ can be defined as ``ruled'' hypersurfaces whose generating curves are an integral manifold of $\scD_2$ and $\scD_1,$ and are foliated by integral manifolds of $\scD_1$ and $\scD_2,$ respectively. The pairwise intersection of sufficiently close hypersurfaces $\Sigma^1_x$ and $\Sigma^2_y$ is non-empty and has codimension 2 in $N.$ If $n=1,$  such intersections define the paths of an induced path geometry. When $n\geq 2,$ such intersections have dimension $2n-2>1$ and their corresponding geometry is unclear. 
\end{remark}
\newpage

\appendix
\setcounter{equation}{0}
\setcounter{subsection}{0}
 \setcounter{theorem}{0} 


\section*{Appendix}
\renewcommand{\theequation}{A.\arabic{equation}}
\renewcommand{\thesection}{A}
 


Bianchi identities for the  curvature entries of a 3D path geometry, as expressed in Proposition \ref{prop:3D-path-geom}, modulo $\{\alpha^0,\alpha^1,\alpha^2,\beta^1,\beta^2\},$ is given by
\begin{equation}
  \label{eq:W-A-curvature-torsion-Bianchies}
  \begin{aligned}
    \exd W_0 \equiv& 4 W_0 \psi^1_1 + 4 W_1 \psi^2_1\\
    \exd W_1 \equiv& W_0 \psi^1_2 + (3  \psi_1^1 +  \psi^2_2)W_1 + 3 W_2 \psi^2_1\\
    \exd W_2 \equiv& 2 W_1 \psi^1_2 + (2 \psi^1_1 + 2  \psi^2_2)W_2 + 2 W_3 \psi^2_1\\
    \exd W_3 \equiv& 3 W_2 \psi^1_2 + (\psi^1_1 + 3 \psi^2_2)W_3 + W_4 \psi^2_1\\
    \exd W_4 \equiv& 4 W_3 \psi^1_2 + 4 W_4 \psi^2_2\\
    \exd A_0 \equiv& (4 \psi^0_0 + 3 \psi^1_1 + \psi^2_2)A_0 + 2 A_1 \psi^2_1\\
    \exd A_1 \equiv& A_0 \psi^1_2 + (4 \psi^0_0 + 2 \psi^1_1 + 2 \psi^2_2)A_1 + A_2 \psi^2_1\\
    \exd A_2 \equiv& 2 A_1 \psi^1_2 + (4 \psi^0_0 + \psi^1_1 + 3  \psi^2_2)A_2.
  \end{aligned}
\end{equation}
The structure equations for freestyling path geometries after carrying out reductions in Proposition \ref{prop:torsion-curv-type-freestyling-reduction} on the 8-dimensional principal $B$-bundle $\cG_{T_{G_r}}\to Q$ is given by 
\begin{equation}
  \label{eq:freestyling-streqs}
  \begin{aligned}
    \exd\alpha^2&=(\psi^1_1-\psi^0_0)\w\alpha^2+\alpha^0\w\beta^2 -8x_{15}\alpha^0\w\alpha^2-8x_{11}\alpha^1\w\alpha^2\\
    \exd\alpha^1&=(\psi^1_1-\psi^0_0)\w\alpha^1+\alpha^0\w\beta^1 -4x_{15}\alpha^0\w\alpha^1- 4x_{12}\alpha^1\w\alpha^2 \\
    \exd\alpha^0& = -2\psi^0_0\w\alpha^0+\alpha^1\w\alpha^2+4x_{11}\alpha^0\w\alpha^1-4x_{12}\alpha^0\w\alpha^2 \\
    \exd\beta^2&=(\psi^0_0+\psi^1_1)\w\beta^2-\alpha^2\w\mu_0+ A_1\alpha^0\w\alpha^2+x_4\alpha^1\w\alpha^2\\
    &\ \  -4x_{15}\alpha^0\w\beta^2 -4x_{11}\alpha^1\w\beta^2+4x_{12}\alpha^2\w\beta^2\\
    \exd\beta^1&= (\psi^0_0-\psi^1_1)\w\beta^1-\alpha^1\w\mu_0-A_1\alpha^0\w\alpha^1+x_{18}\alpha^1\w\alpha^2\\
    \exd\psi^0_0&=-\half \alpha^2\w\beta^1+\half\alpha^1\w\beta^2 -W_1\alpha^1\w\beta^1+W_3\alpha^2\w\beta^2\\
    &\ \ +x_4\alpha^0\w\alpha^1-x_{18}\alpha^0\w\alpha^2-\alpha^0\w\mu_0 +2x_{15}\alpha^1\w\alpha^2\\
    \exd\psi^1_1 &= \tfrac 32 \alpha^1\w\beta^2+\tfrac{3}{2}\alpha^2\w\beta^1 +3W_{1}\alpha^1\w\beta^1
    +W_3\alpha^2\w\beta^2\\
    &\ \ -6x_4\alpha^0\w\alpha^1-2x_{18}\alpha^0\w\alpha^2-6x_{15}\alpha^1\w\alpha^2\\   
    \exd\mu_0& =2\psi^0_0\w\mu_0-\beta^1\w\beta^2 + x_1\alpha^0\w\alpha^1+x_7\alpha^0\w\alpha^2+x_{16}\alpha^1\w\alpha^2 \\
  &\ \  +3x_4\alpha^0\w\beta^1+x_{14}\alpha^1\w\beta^1-5x_{15} \alpha^2\w\beta^1-3x_{18}\alpha^0\w\beta^2+5x_{15}\alpha^1\w\beta^2\\
    &\ \ -x_3\alpha^2\w\beta^2+4x_{15}\alpha^0\w\mu_0-4x_{12}\alpha^2\w\mu_0+4x_{11}\alpha^1\w\mu_0\\
  \end{aligned}
\end{equation}
for some functions $x_1,\cdots,x_{18},A_1,W_1,W_3$ on $\cG_{T_{G_r}}.$

\subsection*{Acknowledgments}


The starting point of this article was several discussions that Maciej Dunajski initiated with the authors about his work on the dancing construction. The authors would like to thank him for his encouragement and sharing his ideas. The authors also thank Anna Snela for suggesting the term freestyling. WK was partially supported by the grant 2019/34/E/ST1/00188 from the National Science Centre, Poland. OM received funding from the Norwegian Financial Mechanism 2014-2021 with project registration number 2019/34/H/ST1/00636. The  EDS calculations  are done using Jeanne Clelland's \texttt{Cartan} package in Maple.
 



%-------------BIBLIOGRAPHY--------
\bibliographystyle{alpha}      
\bibliography{dancing}
\end{document}
%%% Local Variables:
%%% mode: latex
%%% TeX-master: t
%%% End:

