% This must be in the first 5 lines to tell arXiv to use pdfLaTeX, which is strongly recommended.
\pdfoutput=1
% In particular, the hyperref package requires pdfLaTeX in order to break URLs across lines.

\documentclass[11pt]{article}

% Remove the "review" option to generate the final version.
\usepackage{EMNLP2023}

% Standard package includes
\usepackage{times}
\usepackage{latexsym}

% For proper rendering and hyphenation of words containing Latin characters (including in bib files)
\usepackage[T1]{fontenc}
% For Vietnamese characters
% \usepackage[T5]{fontenc}
% See https://www.latex-project.org/help/documentation/encguide.pdf for other character sets

% This assumes your files are encoded as UTF8
\usepackage[utf8]{inputenc}

% This is not strictly necessary and may be commented out.
% However, it will improve the layout of the manuscript,
% and will typically save some space.
\usepackage{microtype}

% This is also not strictly necessary and may be commented out.
% However, it will improve the aesthetics of text in
% the typewriter font.
\usepackage{inconsolata}

\newcommand{\theHalgorithm}{\arabic{algorithm}}
% Recommended, but optional, packages for figures and better typesetting:
\usepackage{graphicx}
\usepackage{algorithm,algorithmic}
\usepackage{booktabs} % for professional tables
 
\usepackage{hyperref}

\usepackage{natbib}
 

% For theorems and such
\usepackage{amsmath}
\usepackage{amssymb}
\usepackage{mathtools}
\usepackage{amsthm}
\allowdisplaybreaks

\usepackage{mathrsfs}

% if you use cleveref..
\usepackage[capitalize,noabbrev]{cleveref}

%%%%%%%%%%%%%%%%%%%%%%%%%%%%%%%%
% THEOREMS
%%%%%%%%%%%%%%%%%%%%%%%%%%%%%%%%
\theoremstyle{plain}
\newtheorem{theorem}{Theorem}[section]
\newtheorem{proposition}[theorem]{Proposition}
\newtheorem{hypothesis}[theorem]{Hypothesis}

\newtheorem{lemma}[theorem]{Lemma}
\newtheorem{corollary}[theorem]{Corollary}
\theoremstyle{definition}
\newtheorem{definition}[theorem]{Definition}
\newtheorem{assumption}[theorem]{Assumption}
\theoremstyle{remark}
\newtheorem{remark}[theorem]{Remark}

% Todonotes is useful during development; simply uncomment the next line
%    and comment out the line below the next line to turn off comments
%\usepackage[disable,textsize=tiny]{todonotes}
\usepackage[textsize=tiny]{todonotes}

%\usepackage[utf8]{inputenc} % allow utf-8 input
%\usepackage[T1]{fontenc}    % use 8-bit T1 fonts
\usepackage{hyperref}       % hyperlinks
\usepackage{cleveref}       % \Cref
\usepackage{url}            % simple URL typesetting
\usepackage{booktabs}       % professional-quality tables
%\usepackage{amsfonts}       % blackboard math symbols
\usepackage{nicefrac}       % compact symbols for 1/2, etc.
\usepackage{microtype}      % microtypography
\usepackage{xcolor}         % colors
%\usepackage{fullpage}

\usepackage{microtype}
\usepackage{graphicx}
\usepackage{makecell}
\usepackage{caption}
\usepackage{subcaption}
\usepackage{multirow}
\usepackage{rotating}

\usepackage{colortbl}
\definecolor{bgcolor}{rgb}{0.66,0.88,1.00}

\usepackage{tablefootnote}
\usepackage{threeparttable}


\usepackage{xspace}

\xspaceaddexceptions{[]\{\}}


\newcommand{\dataset}[1]{{\tt #1}\xspace}

\usepackage{thm-restate}

\newcommand{\squeeze}{\textstyle}

\usepackage{eqparbox}
\renewcommand{\algorithmiccomment}[1]{\hfill\eqparbox{COMMENT}{// #1}}

\usepackage{etoolbox}
\usepackage{enumitem}
\patchcmd{\quote}{\rightmargin}{\leftmargin 1em \rightmargin}{}{}
%============
\newcommand{\algname}[1]{{\sf\small#1}\xspace}
 
\makeatletter
\allowdisplaybreaks


\newcommand{\dotp}[2]{\left\langle {#1}, {#2}, \right\rangle}
\newcommand{\norm}[1]{\left\|{#1}\right\|}
\newcommand{\bbox}{\text{bbox}}
\newcommand{\alphapck}{\alpha_\bbox}
\newcommand{\kcycle}{\text{k-CyPCK}}
\newcommand{\cycle}{\text{-CyPCK}}

\newcommand{\I}{\mathbf{I}}
\newcommand{\Ia}{\I^\text{a}}
\newcommand{\Ib}{\I^\text{b}}
\newcommand{\Iatob}{\I^\text{a $\rightarrow$ b}}
\newcommand{\F}{\mathbf{F}}
\newcommand{\Fa}{\F^\text{a}}
\newcommand{\Fb}{\F^\text{b}}
\newcommand{\f}{\mathbf{f}}
\newcommand{\fa}{\f^\text{a}}
\newcommand{\fb}{\f^\text{b}}
\newcommand{\p}{\mathbf{p}}
\newcommand{\pa}{\p^\text{a}}
\newcommand{\pb}{\p^\text{b}}
\newcommand{\A}{\boldsymbol{\Phi}_\text{align}}
\newcommand{\G}{\mathbf{G}}
\newcommand{\C}{\mathbf{C}}
\newcommand{\Ca}{\C^\text{a}}
\newcommand{\Cb}{\C^\text{b}}
\newcommand{\cc}{\mathbf{c}}
\newcommand{\cca}{\cc^\text{a}}
\newcommand{\ccb}{\cc^\text{b}}
\newcommand{\Irec}{\I_\text{Recon}}
\newcommand{\M}{\mathbf{M}}
\newcommand{\Mrec}{\M_\text{Recon}}
\newcommand{\loss}{\mathcal{L}}
\newcommand{\T}{\mathcal{T}}
\newcommand{\W}{\mathcal{W}}
\newcommand{\Id}{\mathcal{I}}


\newcommand{\haoyu}[1]{{\color{orange}[HZ: #1]}}
\newcommand{\abhishek}[1]{{\color{blue}[AP: #1]}}
\newcommand{\rong}[1]{{\color{green}[RG: #1]}}

\title{Do Transformers Parse while Predicting the Masked Word?}
\author{
Haoyu Zhao\textsuperscript{1,2}\footnotemark[1]\quad Abhishek Panigrahi\textsuperscript{1,2}\footnotemark[1]\quad Rong Ge\textsuperscript{3}\quad Sanjeev Arora\textsuperscript{1,2} \\
\textsuperscript{1}Department of Computer Science, Princeton University \\
\textsuperscript{2} Princeton Language and Intelligence, Princeton University \\
\textsuperscript{3}Department of Computer Science, Duke University \\
\texttt{\{haoyu,ap34,arora\}@cs.princeton.edu, rongge@cs.duke.edu}
%	Haoyu Zhao\thanks{Equal contribution.} \textsuperscript{,}\thanks{Department of Computer Science, Princeton University, Princeton, NJ; Email: \texttt{\{haoyu,ap34,arora\}@cs.princeton.edu}.}
%	\And
%	Abhishek Panigrahi\footnotemark[1] \textsuperscript{,}\footnotemark[2]
%   \And
%   Rong Ge\thanks{Department of Computer Science, Duke University, Durham, NC; Email: \texttt{rongge@cs.duke.edu}.}
%   \And
%   Sanjeev Arora\footnotemark[2]
}
\date{}

\begin{document}

\maketitle

\renewcommand{\thefootnote}{\fnsymbol{footnote}}
\footnotetext[1]{Equal contribution.}
%\footnotetext[2]{Department of Computer Science, Princeton University, Princeton, NJ; Email: \texttt{\{haoyu,ap34,arora\}@cs.princeton.edu}.}
%\footnotetext[3]{Department of Computer Science, Duke University, Durham, NC; Email: \texttt{rongge@cs.duke.edu}.}


\begin{abstract}


Over the past few years, there has been a significant amount of research focused on studying the ReLU activation function, with the aim of achieving neural network convergence through over-parametrization. However, recent developments in the field of Large Language Models (LLMs) have sparked interest in the use of exponential activation functions, specifically in the attention mechanism.

Mathematically, we define the neural function $F: \R^{d \times m} \times  \mathbb{R}^d \rightarrow \mathbb{R}$ using an exponential activation function. Given a set of data points with labels $\{(x_1, y_1), (x_2, y_2), \dots, (x_n, y_n)\} \subset \mathbb{R}^d \times \mathbb{R}$ where $n$ denotes the number of the data. Here $F(W(t),x)$ can be expressed as $F(W(t),x) := \sum_{r=1}^m a_r \exp(\langle w_r, x \rangle)$, where $m$ represents the number of neurons, and $w_r(t)$ are weights at time $t$. It's standard in literature that $a_r$ are the fixed weights and it's never changed during the training. We initialize the weights $W(0) \in \mathbb{R}^{d \times m}$ with random Gaussian distributions, such that $w_r(0) \sim \mathcal{N}(0, I_d)$ and initialize $a_r$ from random sign distribution for each $r \in [m]$.

Using the gradient descent algorithm, we can find a weight $W(T)$ such that $\| F(W(T), X) - y \|_2 \leq \epsilon$ holds with probability $1-\delta$, where $\epsilon \in (0,0.1)$ and $m = \Omega(n^{2+o(1)}\log(n/\delta))$. To optimize the over-parametrization bound $m$, we employ several tight analysis techniques from previous studies [Song and Yang arXiv 2019, Munteanu, Omlor, Song and Woodruff ICML 2022]. 

 

\end{abstract}

\section{Introduction}

The increasing complexity of source code poses a key challenge to the reliability of large-scale software systems. Software bugs in these systems can lead to safety issues~\cite{bug_safety} for users around the world as well as cause non-negligible financial losses~\cite{bug_loss}. As such, developers have to spend a large amount of time and effort on bug fixing. Consequently, \aprfull (\apr), designed to automatically generate patches to fix software bugs, has attracted wide attention from both academia and industry~\cite{long2016prophet, legoues2012genprog, long2015spr, lou2020can, tufano2018empstudy}. 


To achieve \apr, one popular approach is known as Generate-and-Validate (G\&V)~\cite{qi2015gv, ghanbari2019prapr, lou2020can, le2016hdrepair, legoues2012genprog, wen2018capgen, hua2018sketchfix, martinez2016astor, koyuncu2020fixminder, liu2019tbar, liu2019avatar}, which is typically based on the following pipeline: First, fault localization techniques~\cite{wong2016fl, abreu2007ochiai, zhang2013injecting, papadakis2015metallaxis, li2019deepfl, li2017transforming} are applied to determine the suspicious locations in programs where bugs are likely to exist. Then, the buggy locations are used by the \apr tools to generate a list of patches that replace buggy lines with correct lines. Afterward, each patch is validated against the original test suite to identify any \emph{plausible patches} (i.e., passing all tests in the test suite). Finally, to determine the \emph{correct patches}, developers examine the list of plausible patches to see if any of them can correctly fix the bug. 

Traditional \apr tools can mainly be categorized into heuristic-based~\cite{legoues2012genprog, le2016hdrepair, wen2018capgen}, constraint-based~\cite{mechtaev2016angelix, le2017s3, demacro2014nopol, long2015spr} and \template~\cite{ghanbari2019prapr, hua2018sketchfix, martinez2016astor, liu2019tbar, liu2019avatar}. Among these traditional tools, \template \apr tools~\cite{ghanbari2019prapr, liu2019tbar, benton2020effectiveness} have been able to achieve state-of-the-art results. \Template \apr tools typically leverage pre-defined templates (e.g., adding a nullness check) for bug fixing. However, since these fix templates are typically handcrafted, the number and types of bugs they are able to fix can be limited. 



To address the limitations of traditional \apr, researchers have proposed various \learning \apr tools~\cite{li2020dlfix, chen2018sequencer, jiang2021cure, lutellier2020coconut, zhu2021recoder, ye2022rewardrepair} based on the \nmtfull (\nmt) architecture~\cite{sutskever2014mt} where the input is the buggy code snippets and the goal is to translate the buggy code snippets into a fixed version. To accomplish this, \learning \apr tools require supervised training datasets with pairs of both buggy and fixed code snippets in order to learn how to perform this translation step. These training data are usually obtained by mining historical bug fixes using heuristics/keywords~\cite{dallmeier2007benchmark}, which can be imprecise for identifying bug-fixing commits; even the actual bug-fixing commits can include irrelevant code changes, leading to further pollution in the dataset~\cite{xia2022alpharepair}.
% 
Moreover, it can be hard for such \apr tools to generalize and fix bug types unseen during training. 



To better leverage recent advances in \plmfull{s} (\plm{s}), researchers~\cite{xia2022alpharepair, xia2023repairstudy, kolak2022patch, prenner2021codexws} have directly applied \plm{s} to generate patches without bug-fixing datasets. These \llm-based \apr tools work by either directly generating a complete code function~\cite{prenner2021codexws, xia2023repairstudy} or predict/infill the correct code snippet given its surrounding context~\cite{xia2022alpharepair, xia2023repairstudy}. By directly using \llm{s} that are pre-trained on billions of open-source code snippets, \llm-based \apr tools can achieve state-of-the-art performance on many repair datasets~\cite{xia2022alpharepair}. 


% 
%
%

Traditional \apr tools have long used the insight of the \emph{plastic surgery hypothesis}~\cite{barr2014plastic} where it states that the code ingredients to fix a bug already exist within the same project. Traditional \apr tools have manually designed pattern-~\cite{ghanbari2019prapr, saha2017elixir} or heuristic-based~\cite{jiang2018simfix, legoues2012genprog} approaches to finding and using such relevant code ingredients to generate fixes for bugs. However, the plastic surgery hypothesis has been largely ignored in \llm-based \apr. In fact, \llm provides a unique opportunity to fully automate the plastic surgery hypothesis idea via fine-tuning (learning project-specific information via model updates from the buggy project) and prompting (directly providing relevant code ingredients to the model), and make it directly applicable to different languages (since the \llm{s} are typically multi-lingual).%
Moreover, despite the intensive manual efforts involved, traditional \apr tools still cannot fully leverage project-specific information due to large search space for leveraging/composing existing code ingredients. In contrast, the project-specific information can effectively leveraged by \llm{s} due to their power in code understanding/vectorization, e.g., even partial/imprecise information may still guide \llm{s} in correct patch generation!
 To this end, we ask the question: \emph{How useful is the plastic surgery hypothesis in the era of \plm{s}}?








\mypara{Our Work.} To answer the question, we present \ourtech{\xspace} -- a \llm-based approach that automatically utilizes the plastic surgery hypothesis by systematically combining multiple fine-tuning and prompting strategies for \apr. \ourtech fine-tunes \plm{s} using two novel domain-specific training strategies: \textbf{\epfinetune} -- we fine-tune using the original buggy project by aggressively masking out a high percentage of tokens, which allows \plm to learn project-specific code tokens and programming styles; and \textbf{\rofinetune} -- which only masks out a single continuous code sequence per training sample, allowing the model to get used to the final \csapr task of predicting a single continuous code sequence. Furthermore, we directly leverage the ability for \plm{s} to understand natural language instructions and introduce a novel prompting strategy, \textbf{\idprompting}, which uses information retrieval and static analysis to obtain a list of relevant identifiers for the buggy lines. While such relevant identifiers are critical for fixing some difficult bugs, they may not be seen by the \llm during inference due to limited context window size. Through the use of prompting, we directly tell the model to use these extracted identifiers (relevant code ingredients) to generate the correct code. Finally, to perform repair, we combine all four model variants (including the base model, both fine-tuned models and the base model with prompting) for the final repair.





While our insight of leveraging the plastic surgery hypothesis for \llm-based \apr is generalizable across different types of \plm{s}, to implement \ourtech, we choose a recent \plm{\xspace}, \ctfive~\cite{wang2021codet5}, which is pre-trained on millions of open-source code snippets. \ctfive is an encoder-decoder model trained using \mspfull (\msp) objective where a percentage of tokens are masked out and each continuous masked token sequence is referred to as a masked span. Also, although we only extract relevant identifiers from the current buggy project (since this paper focuses on the plastic surgery hypothesis), our work can be easily extended to obtain other code information (such as relevant statements or functions) from other sources, such as  the massive pre-training corpora~\cite{husain2020codesearchnet} or historical bug-fixing datasets~\cite{jiang2019infer}, which can provide more coding knowledge for \llm{s}. Besides, although we mainly focus on using traditional string comparison algorithms for information retrieval in this paper, these techniques can be easily replaced by other frequency-based retrieval~\cite{robertson2009probabilistic} and neural search (or embedding-based search)~\cite{reimers2019sentence}.
  In summary, this paper makes the following contributions:


%


\begin{itemize}[noitemsep, leftmargin=*, topsep=0pt]
    \item \textbf{Dimension.} This paper is the first to revisit the important plastic surgery hypothesis in the era of \llm{s}. It opens up a new dimension for \llm-based \apr to incorporate previously neglected information from the buggy project itself to boost \apr performance. Furthermore, it demonstrates the promising future of retrieval-based prompting for modern \llm-based \apr.
    \item \textbf{Implementation.} We implement \ourtech based on the recent \ctfive model. We augment the model using two novel fine-tuning strategies: \epfinetune and \rofinetune, along with a novel prompting strategy based on information retrieval and static analysis: \idprompting. We combine the patches generated by all four models together and perform patch ranking to speed up \apr.% 
    \item \textbf{Evaluation Study.} We conduct an extensive evaluation against state-of-the-art \apr tools. On the widely studied \dfj 1.2 and 2.0 datasets~\cite{just2014dfj}, \ourtech is able to achieve the new state-of-the-art results of 89 and 44 correct bug fixes (15 and 8 more than best baseline) respectively.  Furthermore, we perform a broad ablation study to justify our design. \ourtech demonstrates for the first time that the plastic surgery hypothesis can substantially boost \llm-based \apr and advance state-of-the-art \apr, while being fully automated and general. Moreover, even partial/imprecise code ingredients may still effectively guide \llm{s} for \apr!
\end{itemize}



%!TEX root = ../main.tex

\section{Inductive Conformal Prediction}
\label{sec:pre:icp}
Given a set $\{ z_i = ( x_i, y_i ) \}_{i=1}^l$ with observation $x_i \in \calX$ and label $y_i \in \calY$ such that each $z_i \in \calZ := \calX \times \calY $ is drawn i.i.d. from an \emph{unknown} distribution on $\calZ$, inductive conformal prediction (ICP) provides 
% a simple yet powerful framework to learn 
a \emph{set prediction} $\Feps(x) \subseteq \calY$, parameterized by an error rate $0 < \epsilon <1$, such that given a new sample $z_{l+1} = (x_{l+1},y_{l+1})$ satisfying an \emph{exchangeability} condition (elaborated in Theorem~\ref{thm:icp-validity}), we have
\bea\label{eq:icpmiscoverage}
\probof{ y_{l+1} \in \Feps(x_{l+1}) } \geq 1-\epsilon, 
\eea
\ie, the prediction set $\Feps$ guarantees to contain the true label $y_{l+1}$ with probability at least $1-\epsilon$. 

% In order to achieve the probabilistic coverage in~\eqref{eq:icpmiscoverage}, ICP performs the following three steps.

{\bf Training}. We start by dividing the dataset into a \emph{proper training set} $\{ z_1,\dots,z_m \}$ and a \emph{calibration set} $\{ z_{m+1},\dots,z_{l} \}$. We shorthand $n = l - m$ as the size of the calibration set.
We learn a prediction function $f: \calX \rightarrow \tcalY$ from the proper training set using \emph{any} architecture, which allows us to fully exploit the power of modern deep learning. The prediction space $\tcalY$ can be the same as the label space $\calY$, or can contain auxiliary information such as a heuristic notion of uncertainty (\eg, softmax scores in classification or a heatmap in the case of keypoint detection). 

{\bf Conformal calibration}. 
% Leveraging the learned $f$, 
We define a \emph{nonconformity} function $S: \calZ^{m} \times \calZ \rightarrow \Real{}$ to measure how well a given sample $z = (x,y)$ \emph{conforms} to the proper training set. A popular instance of $S$ leverages the learned prediction $f$:
\bea \label{eq:nonconformity}
S\parentheses{\cbrace{z_1,\dots,z_m},(x,y)} \stackrel{\eg}{=} r(y,f(x)),
\eea
where $r: \calY \times \tcalY \rightarrow \Real{}$ is a measure of disagreement between the label $y$ and the prediction $f(x)$. For example, consider $\calY = \tcalY = \Real{}$, one can design $r(y,f(x)) = \abs{y - f(x)}$: if $(x,y)$ poorly conforms to the training set, $f$ will incur large errors.   
While the function $S$ can be arbitrary (\eg, a learnable neural network~\cite{stutz22iclr-learnconformal}), \eqref{eq:nonconformity} is a convenient definition since $f$ is implicitly dependent on $\{z_i\}_{i=1}^m$ and $r$ can incorporate domain-specific knowledge.
We then compute the nonconformity scores on the calibration set as $\alpha_i = r(y_i,f(x_i)), i = m+1,\dots,l$,
and sort them in \emph{nonincreasing} order $\alpha_{\pi(1)}\geq\dots \geq \alpha_{\pi(n)}$, where $\pi(i) \in \{m+1,\dots,l\}$ is an index permutation.
 % (offset by $m$).

{\bf Conformal prediction}. Given a new observation $x_{l+1}$ (with an unknown $y_{l+1}$) and a user-specified $\epsilon \in (0,1)$, we compute the inductive conformal prediction (ICP) set as
\bea\label{eq:icpcompute}
\Feps \parentheses{x_{l+1}} = \cbrace{y \in \calY \mid \alpha^y \leq \alpha_{\pi(\floor{(n+1)\epsilon})}},
\eea
where $\alpha^y = r(y,f(x_{l+1}))$
is the nonconformity score of the new sample when fixing the true label to be $y$. In other words, the ICP set~\eqref{eq:icpcompute} outputs the set of all labels that make the nonconformity score of the new sample no greater than $\alpha_{\pi(\floor{(n+1)\epsilon})}$ -- the $\floor{(n+1)\epsilon}$-th largest nonconformity score in the calibration set. 
% By doing so, ICP ensures that there are at least $\floor{(n+1)\epsilon}$ samples in the calibration set that are less conforming than the new sample. 
We have the following result stating the probabilistic coverage of the ICP set~\eqref{eq:icpcompute}.
% provides a valid statistical coverage of the true label $y_{l+1}$.

\begin{theorem}[Validity of ICP Coverage {\cite{vovk05book-conformal,lei18jasa-conformal,vovk12acml-icpconditional}}] \label{thm:icp-validity}
If $z_{m+1},\dots,z_l$, $z_{l+1} = (x_{l+1},y_{l+1})$ are exchangeable, \ie, their distribution is invariant under permutation, then
\bea\label{eq:icpvalidity}
1 - \epsilon \leq \probof{y_{l+1} \in \Feps(x_{l+1})} \leq 1 - \epsilon + 1/(n+1)
\eea
for any $\epsilon \in (0,1)$. Furthermore, when conditioned on the calibration set, calling $h = \floor{(n+1)\epsilon}$, we have
\begin{equation}\label{eq:beta}
\hspace{-4mm}\probof{y_{l+1}\!\in\!\Feps(x_{l+1})\!\mid\!\{z_{m+1},\dots,z_l\}}\!\sim\!\mathrm{Beta}(n+1\!-\!h,h).
\end{equation}
\end{theorem}
A few remarks are in order about Theorem~\ref{thm:icp-validity}.
First, asking $z_{m+1},\dots,z_l,z_{l+1}$ to be exchangeable is weaker than asking them to be independent. However, this assumption typically fails when the calibration set is a single video sequence, where the image frames $\{z_{m+1},\dots,z_l\}$ are temporally correlated~\cite{luo21arxiv-conformalsafety}. Fortunately, as we detail in Section~\ref{sec:experiments}, the way the LineMOD Occlusion dataset~\cite{brachmann14eccv-linemodocc} was collected makes the exchangeability condition easily satisfied, which also suggests best practices to make the exchangeability condition hold in computer vision. 
Second, the lower bound in~\eqref{eq:icpvalidity} can be intuitively proved because under exchangeability, $\alpha_{l+1} := r(y_{l+1},f(x_{l+1}))$ --the nonconformity score of the new sample with the true label-- is \emph{exchangeable} with the nonconformity scores of the calibration samples, and hence \emph{equally likely} to fall in anywhere between the scores $\{ \alpha_{\pi(i)}\}_{i=1}^n$. Consequently, $\probof{y_{l+1} \in \Feps(x_{l+1})} = \probof{\alpha_{l+1} \leq \alpha_{\pi(\floor{(n+1)\epsilon})}} = 1 - \floor{(n+1)\epsilon}/(n+1) \geq 1 - \epsilon$. The upper bound in \eqref{eq:icpvalidity} states that $1-\epsilon$ is not overly conservative (indeed tight if $n$ is large). 
Lastly, the probabilistic guarantee in \eqref{eq:icpvalidity} is \emph{marginal} over the randomness of the calibration set, meaning if one chooses an infinite number of calibration sets,  the \emph{average} empirical coverage will converge to $1-\epsilon$. This, however, implies that the empirical coverage given one calibration set is a random variable that fluctuates as the Beta distribution~\eqref{eq:beta}. Fig.~\ref{fig:beta-distribution} plots the Beta distribution at $\epsilon=0.1$ with different sizes of the calibration set. We observe that as $n$ increases the empirical coverage becomes more concentrated at $1-\epsilon$. Our experiments show that even with a small ($n=200$) calibration set, the empirical coverage is close to, and mostly higher than, $1-\epsilon$.

% \begin{proposition}[Conditional Validity of ICP {\cite{vovk12acml-icpconditional}}] \label{prop:icp-conditional-validity}
% \red{To be filled out}
% \end{proposition}


% Proposition~\ref{prop:icp-validity} states that, if the new observation $z_{l+1}$ is exchangeable with the calibration set (which is a weaker condition than requiring $z_{l+1}$ is jointly i.i.d. with the calibration set), then no matter which prediction function $f$ has been learned from the proper training set, and which function $A$ has been chosen to compute the nonconformity score, we have at least $1-\epsilon$ confidence that the ICP $\Feps$ defined in \eqref{eq:icp} contains the true label. Of course, the caveat here is that the quality of the learned prediction function $f$ and the nonconformity function $A$ will decide the conservativeness of the ICP $\Feps$. For example, if $f$ has poor predictive power, then the set $\Feps$ may be arbitrarily large so that it tells little information about the true label $y$. \red{Fortunately, as we will show in experiments, with modern deep learning architectures for learning $f$, we can obtain ICPs that are both confident and tight.}

%!TEX root = ../main.tex
% \begin{figure}
% \hspace{-4mm}\includegraphics[width=1.1\columnwidth]{icp-overview-half.pdf}
% \caption{Given a learned prediction function and a calibration set of $n$ samples, conformal calibration uses a nonconformity function~\eqref{eq:nonconformity} to compute and sort nonconformity scores $\{ \alpha_{\pi(i)}\}_{i=1}^n$. Given a new observation and an error rate $\epsilon$, conformal prediction~\eqref{eq:icpcompute} outputs a prediction set of all labels under which the nonconformity score of the new sample is no larger than $\alpha_{\pi(\floor{(n+1)\epsilon})}$.
% \label{fig:icp-overview}}
% \end{figure}
\begin{figure}
\vspace{-4mm}
\begin{center}
\includegraphics[width=0.6\columnwidth]{beta.pdf}
\end{center}
\vspace{-6mm}
\caption{Beta distribution of the conditional coverage in~\eqref{eq:beta} with $\epsilon=0.1$ and different $n$. Notice how the conditional probability becomes more concentrated around $1-\epsilon$ when $n$ increases.
\label{fig:beta-distribution}}
\vspace{-7mm}
\end{figure}

\section{Coating in the 3D Hybrid Model}
\label{sec:construction}

In this section, we apply our coating algorithm to the 3D hybrid model.
We first define a triangulation on nodes of $L$ with degree $\Delta \leq 8$, and afterwards construct a virtual graph on which we emulate our algorithm using $2^{2\Delta}$ different types of tiles.
%We first give a surface graph $\tri$ that is a triangulation with degree $4 \leq \Delta \leq 8$ in which the boundary of each node is chordless.
%We show that there is a restricted class of objects for which $C^0$ is coatable w.r.t. $\tri$, i.e., empty nodes have degree at most six.
%To solve the problem on surface graphs of degree $\Delta > 6$, we construct a virtual graph $\tri^*$ of size at most $2 \Delta n$ on which we emulate the algorithm using $2^{2\Delta}$ different types of tiles.
%Each tile type corresponds to a bit-sequence of length $2\Delta$ that we use to encode whether nodes of $\tri^*$ are tiled or empty.

\begin{figure}[t]
    \centering
    \begin{minipage}{.43\textwidth}
        \centering
        \includegraphics[width=\linewidth]{snapshot}
        \caption{Snapshot of $\surf$: the circled nodes are adjacent in $G(L)$ but not in $\surf$.}
        \label{fig:coatingLayerGraph}
    \end{minipage}%
    \hfill
    \begin{minipage}{.55\textwidth}
        \centering
        \includegraphics[width=\linewidth]{emulation}
        \caption{A triangular face $f = \{u_1,u_2,u_3\}$ of $\tri$ and its corresponding virtual edges and nodes in $\tri^*$.}
        \label{fig:emulation}
    \end{minipage}
\end{figure}


\begin{figure}[b]
    \centering
    \hfill
    \foreach \x in {a,...,f}{%
        \begin{subfigure}[c]{0.12\linewidth}
            \includegraphics[width=\linewidth]{boundaries_\x}
            \subcaption{}
            \label{subfig:triangulation_\x}
        \end{subfigure}%
        \hfill
    }%
    \null\hfill
    \\
    \foreach \x in {g,...,m}{%
        \begin{subfigure}[c]{0.12\linewidth}
            \includegraphics[width=\linewidth]{boundaries_\x}
            \subcaption{}
            \label{subfig:triangulation_\x}
        \end{subfigure}%
        \hfill
    }%
    \caption{All possible arrangements of faces in $\surf$ apart from rotation. Dashed edges indicate that the distance between its endpoints is precisely two w.r.t. $G$.}
    \label{fig:triangulation}
\end{figure}

%\subsection{Triangulation of the Surface Graph}
%\label{subsec:smoothObjects}

Recall the definition of graph $G = (V,E)$ and its embedding in $\mathbb{R}^3$ from \cref{sec:model}.
%Any two adjacent nodes $v,w \in V$ have euclidean distance $|\vec{v}-\vec{w}|_2 = \sqrt{2}$, and tiles at $v$ and $w$ share a common face.
%If two nodes $v,w \in V$ are adjacent w.r.t. $G$, then tiles at $v$ and $w$ share a common face.
%Note that in this case $|\vec{v}-\vec{w}|_2 = \sqrt{2}$, where $|\cdot|_2$ is the Euclidean distance.
%We call $v$ and $w$ \emph{vertex adjacent}, if $|\vec{v}-\vec{w}|_2 = 2$, e.g., $v = w + \UNE + \SE$.
%In contrast to adjacent nodes, tiles at vertex adjacent nodes only share a common vertex.
Define $\surf = (L,E')$ as the subgraph of $G(L)$ that contains only those edges $\{v,w\}$ for which $v$ and $w$ share adjacent object neighbors (see \cref{fig:coatingLayerGraph}), i.e., $E' = \{\{v,w\}\mid d_\obj{}(N_1(v), N_1(w)) \leq 1\}$.
%The difference between $G(L)$ and $\surf$ is illustrated in \cref{fig:coatingLayerGraph}.
%The two circled empty nodes are adjacent in $G(L)$, but not in $\surf$;
%an agent that moves in $\surf$ must thereby remain locally connected to the object.
We can view $\surf$ as embedded on the surface of our 3D object.
%It contains no cross edges, and especially, $\surf$ is planar if the object contains no holes.
That embedding contains triangular and tetragonal faces (see \cref{fig:coatingLayerGraph}) where tetragonal faces can occur in one of three orientations: 
(1) $v, v+ \NE, v + \NE + \USE, v + \USE$, (2) $v, v + \NW, v + \NW + \UNE, v + \UNE$, and (3)~$v, v + \N, v + \N + \UW, v+ \UW$.
%Note that any two nodes $v,w$ on the diagonal of a tetragonal face have euclidean distance $|\vec{v}-\vec{w}|_2 = 2$ while adjacent nodes have distance $|\vec{v}-\vec{w}|_2 = \sqrt{2}$.
%While nodes of triangular faces are pairwise adjacent, nodes on the diagonals of tetragonal faces are only vertex adjacent. 
%Using (1) as an example, $v$ is vertex adjacent to $v + \NE + \USE$, and $v + \NE$ is vertex adjacent to $v + \USE$.
Apart from rotation, \cref{fig:triangulation} shows all possible arrangements of faces within $\surf$. % w.r.t.\ a fixed common node $v \in L$ (centering node in the figure).
We define the class of \emph{smooth objects} $\mathcal{S}$ as all objects for which $\surf$ contains only the cases (a)--(f) from \cref{fig:triangulation}.
Let $\tri$ be the triangulation of $\surf$ in which the same diagonal edge is added for each tetragonal face of the same orientation (1)--(3) (since we want the agent to be able to deduce the triangulation).
%Each face of $\tri$ is triangular with pairwise adjacent nodes such that $B(v)$ is chordless w.r.t. $\tri$ for any $v \in L$.
Since $d_L(v,w) = 1$ w.r.t. $\tri$ implies $d_L(v,w) \leq 2$ w.r.t. $\surf$, the agent can emulate moving on $\tri$ with a multiplicative time and memory overhead of at most two.
%Since tetragonal faces of different orientations (1)--(3) cannot contain a common edge, (d), (k) and (m) from \cref{fig:triangulation} are the only cases in which a node is contained in tetragonal faces of different orientations.
%Hence, $\tri$ has degree at most six for any $\obj \in \mathcal{S}$ such that our next theorem follows directly from \cref{thm:algorithm}:
It is easy to see that $B(v)$ is chordless and $v$ has degree at most six for all $v \in L$ within the class $\mathcal{S}$.
Together with \cref{thm:algorithm} follows:

\begin{theorem}
    \label{thm:coatableSingleType}
    A finite-state agent with a single type of passive tiles solves the coating problem on any object $\obj \in \mathcal{S}$ with coating layer $L$ in $\O(n^2)$ steps, where $n = |L|$.
\end{theorem}


\subsection{Emulation of Coatable Surface Graphs}
\label{subsec:emulation}

Consider an arbitrary triangulation $\tri = (L,E)$ of constant degree $\Delta$ and an initially valid configuration $C_0$.
We construct a virtual graph $\tri^* = (L^*,E^*)$ with virtual initial configuration $C^{0*}$ such that $\tri^*$ is coatable w.r.t. $C^{0*}$.
During that construction, we define a partial surjective function $\mathcal{R}: L^* \rightarrow L$ that maps virtual nodes to real nodes.
We show that an agent $r$ operating on $\tri$ w.r.t. $C_0$ with $2^{2\Delta}$ tile types can emulate an agent $r^*$ that executes \cref{alg:algorithm} on $\tri^*$ w.r.t. $C^{0*}$ such that throughout the emulation $\mathcal{R}(p^*) = p$.


\subsubsection{Virtual Graph Construction}

The virtual graph $\tri^*$ is the result of subdividing each face of $\tri$ into nine triangular faces (see \cref{fig:emulation}).
The node set $L^*$ contains a virtual node $v^*_u$ for each node $u \in L$, two virtual nodes $v^*_{u,w}$ and $v^*_{w,u}$ for each edge $\{u,w\} \in E$, and a virtual node $v^*_f$ for each triangular face $f$ of $\tri$.
For each edge $\{u,w\} \in E$ the edge set $E^*$ contains three virtual edges $\{v^*_u,v^*_{u,w}\}$, $\{v^*_{u,w}, v^*_{w,u}\}$ and $\{v^*_{w,u}, v^*_w\}$.
For each triangular face $f = \{u_1,u_2,u_3\}$ of $\tri$ the edge set $E^*$ contains six virtual edges $\{v^*_f,v^*_{u_i,u_j}\}$ and three virtual edges $\{v^*_{u_i,u_j}, v^*_{u_i,u_k}\}$, where $u_i, u_j, u_k \in f$ are pairwise distinct.
We define $\mathcal{R}(v^*_{u,w}) = u$ for any virtual node $v^*_{u,w} \in L^*$. % that corresponds to an edge $\{u,w\}$ of $\tri$.
Consider an arbitrary but fixed order on the vectors $\vec{\X}_1,...,\vec{\X}_m$ that correspond to edges in the embedding of $\tri$.
Let $\pi$ represent that order, i.e., $\pi(\vec{\X}_i) = i$.
For some face $f = \{u_1,u_2,u_3\}$ of $\tri$, we define $\mathcal{R}(v^*_f) = u_i$, where $u_i$ is the node that minimizes $\pi(\vec{u_i} - \vec{u_j})$ for any $u_i,u_j \in f$ with $i \neq j$.
We define the virtual initial configuration $C^{0*}$ such that all $v^*_u$ are tiled, i.e., $\occ^{0*} = \cup_{u \in L} v^*_u$, $p^{0*} = v^*_{p^0}$ and assume a material depot of size at least $|L^*|-|L|$ at $v^*_{p^0}$. 

\begin{lemma}
    \label{lem:virtualGraph}
    $C^{0*}$ is coatable w.r.t. $\tri^*$.
\end{lemma}

\begin{proof}
    Each face of $\tri$ is triangular, and two virtual nodes are added for each edge of $\tri$.
    Hence, $|B(v^*_f)|= 6$ w.r.t. $\tri^*$ for any face $f$ of $\tri$.
    Any $v^*_{u,w}$ is adjacent to $v^*_{f_1}$ and $v^*_{f_2}$, where $f_1,f_2$ are the two faces of $\tri$ that both contain $u$ and $w$, to two nodes $v^*_{u,w_1}, v^*_{u,w_2}$, where $w_1 \in f_1$ and $w_2 \in f_2$, and to $v^*_{w,u}$ and $v^*_u$.
    Hence, $|B(v^*_{u,w})| = 6$ w.r.t. $\tri^*$ for any edge $\{u,w\}$ of $\tri$.
    Any other virtual node is initially tiled, which implies $|B(v^*)| \leq 6$ for any $v^* \in \emp^*$.
    %$B(v^*)$ is chordless since all nodes in $B(v^*)$ correspond to the same or some adjacent face of $\tri$ for any $v^* \in L^*$. 
    By construction, each initially tiled node is isolated, i.e., $d(v^*,w^*) \geq 3$ for any $v^*,w^* \in \occ{}^{0*}$.
    Since $\tri^*$ is connected, it follows that $\emp^{0*}$ is connected and $\be(v^*)$ is connected for any $v^* \in \emp^{0*}$, i.e., $\links^{0*} = \emptyset$.
    Hence, each property of \cref{def:coatability} is satisfied.
\end{proof}

\begin{lemma}
    \label{lem:emulation}
    A finite-state agent can emulate \cref{alg:algorithm} on $\tri^*$ in $\O(\Delta^2n^2)$ steps while moving and placing tiles of at most $2^{2\Delta}$ types on $\tri$.
\end{lemma}

\begin{proof}
    Let $F^* \subset L^*$ be the set of virtual nodes $v^*_f$ that correspond to some face $f$ of $\tri$ in the construction of $\tri^*$.
    Since $\tri^*(L^* \setminus F^*)$ is a subdivision of $\tri$, it can be embedded in the same 3D surface as $\tri$ using vectors that are collinear to vectors in the embedding of $\tri$.
    It follows that we can use the same fixed order $\pi$ from the construction of $\tri^*$.

    In the following, we define for each node $u \in L$ a bit-sequence $x(u) = (x_1,...,x_{2\Delta})$ that encodes the occupation of all nodes $v^* \in L^*$ with $\mathcal{R}(v^*) = u$, where a $0$ encodes an empty, and a $1$ encodes an occupied virtual node.
    By the construction of $\tri^*$, there are at most $2 \Delta$ nodes $v^*$ with $\mathcal{R}(v^*) = u$ such that $2 \Delta$ bits suffice.
    The order of bits in $x(u)$ is uniquely given by $\pi$ where the first $\Delta$ bits encode virtual nodes that correspond to edges of $\tri$, and the following bits encode virtual nodes that correspond to faces of $\tri$.
    There is no bit for the virtual node $v^*_u \in L^*$ since it is initially occupied and remains occupied until termination by following \cref{alg:algorithm}.
    In fact, $\mathcal{R}$ is undefined for $v^*_u \in L^*$.

    Consider an agent $r$ on $\tri$ that utilizes $k = 2^{2\Delta}$ types of passive tiles.
    Each tile type uniquely describes a bit-sequence of length $\log(k) = 2\Delta$ such that $r$ emulates an agent $r^*$ on $\tri^*$ with initial configuration $C^{0*}$ as follows:
    If $r^*$ moves from $v^*$ to $w^*$, then $r$ moves from $\mathcal{R}(v^*)$ to $\mathcal{R}(w^*)$ (if $\mathcal{R}(v^*) \neq \mathcal{R}(w^*)$).
    If $r^*$ places a tile at $v^*$ and $\mathcal{R}(v^*)$ is empty, then $r$ places a tile at $\mathcal{R}(v^*)$ that corresponds to the bit-sequence $x$ in which only $v^*$ is encoded as occupied, otherwise $r$ incorporates the occupation of $v^*$ by changing the tile type.
    If $r^*$ gathers material and $r$ carries no material, then $r$ also gathers material.

    By \cref{thm:algorithm} and \cref{lem:virtualGraph}, $r^*$ solves the coating problem on $\tri^*$.
    Since $\mathcal{R}$ is surjective and any node $\mathcal{R}(v^*) \in L$ is occupied, if $v^*\in L^*$ is occupied, the emulation solves the coating problem on $\tri$ in $\O(|L^*|^2) = \O(\Delta n)$ steps.
\end{proof}

Our final theorem follows from the virtual graph construction on top of our triangulation $\tri$ of $\surf$ (with $\Delta \leq 8$) and the previous lemma:

\begin{theorem}
    \label{thm:coatableManyType}
    A finite-state agent utilizing constantly many tile types can solve the coating problem on arbitrary objects in worst-case optimal $\O(n^2)$ steps.
\end{theorem}

%\section{Omitted Details in \Cref{sec:approx-overview}}\label{sec:approx-detailed}
In \Cref{sec:approx-overview}, we claim that it is possible to approximately execute the Inside-Outside algorithm for PCFG learned on \dataset{PTB} dataset, and can drastically reduce the size of our constructed model with minimal impact on the 1-masking predictions and parsing performance (\Cref{thm:approx-low-rank-informal}) by applying two ingredients: restricting the computations to few non-terminals and utilizing the underlying low-rank structure between the non-terminals. 
%We overview the techniques in \Cref{sec:approx-overview}, and we show the details in this section.
This section is organized as follows: In \Cref{appendix:small_nonterminal_subset}, we show more intuition and experiment results on why we can restrict the computation of the inside-outside algorithm to a small subset of non-terminals. In \Cref{appendix:approx-low-rank-informal}, we add more discussions on the second ingredient (utilizing the low-rank structure). Then in \Cref{sec:approx-few-nt}, we show the details why restricting the computations of few non-terminals can reduce the size of the attention model. In \Cref{sec:approx-low-rank}, we show the detailed proof of \Cref{thm:approx-low-rank-informal}. Finally in \Cref{sec:approx-exp-details}, we show the experiment details in \Cref{sec:approx-overview}.


\subsection{More discussions on computation with few non-terminals} \label{appendix:small_nonterminal_subset}
We hypothesize that we can focus only on a few non-terminals while retaining most of the performance.

\begin{hypothesis}\label{hyp:small_nonterminal_subset}
     For the PCFG $\gG = (\gN, \gI, \gP, n, p)$ learned on the English corpus, there exists $\tilde\gI\subset\gI,\tilde\gP\subset\gP$ with $|\tilde\gI|\ll |\gI|, |\tilde\gP|\ll |\gP|$, such that simulating Inside-Outside algorithm with $\tilde\gI \cup \tilde\gP$ non-terminals introduces \underline{small} error in the 1-mask perplexity and has \underline{minimal} impact on the parsing performance of the Labeled-Recall algorithm.
\end{hypothesis}


To find candidate sets $\tilde\gI,\tilde\gP$ for our hypothesis, we check the frequency of different non-terminals appearing at the head of spans in the parse trees of the \dataset{PTB}~\citep{marcus1993building} training set. We consider the Chomsky-transformed (binarized) parse trees for sentences in the \dataset{PTB} training set, and collect the labeled spans $\{(A, i, j)\}$ from the parse trees of all sentences. For all non-terminals $A$, we compute $\text{freq}(A)$, which denotes the number of times non-terminal $A$ appears at the head of a span. %Formally, 
%\[\text{freq}(A,\ell) := \sum_{(B,j,j')\in \{T_i\}_i}\mathbb I\{A = B, j'-j+1 = \ell\}.\]
\Cref{fig:freq-dist} shows the plot of $\text{freq}(A)$ for in-terminals and pre-terminals, with the order of the non-terminals sorted by the magnitude of $\text{freq}(\cdot)$. We observe that an extremely small subset of non-terminals have high frequency, which allows us to restrict our computation for the inside and outside probabilities to the few top non-terminals sorted by their $\text{freq}$ scores. We select the top frequent non-terminals as possible candidates for forming the set $\tilde\gN$.

%appears with high frequency in a specific length, and thus computing only the top-frequent non-terminals should not affect the computation a lot intuitively.

\begin{figure}[!t]
     \centering
     \iffalse
    \begin{subfigure}[t]{0.4\textwidth}
        \includegraphics[width=\linewidth]{figs/fig-in-global.png}
    \end{subfigure}
    \begin{subfigure}[t]{0.4\textwidth}
        \includegraphics[width=\linewidth]{figs/fig-pre-global.png}
    \end{subfigure}
    \fi
    \includegraphics[width=0.8\linewidth]{figs/fig-nt-global.png}
    \iffalse
    \begin{subfigure}[t]{0.48\textwidth}
        \includegraphics[width=\linewidth]{figs/nts-frequency.pdf}
    \end{subfigure}
    \fi
    
        \caption{Plot for the frequency distribution of in-terminals ($\gI$) and pre-terminals ($\gP$). We compute the number of times a specific non-terminal appears in a span of a parse tree in the \dataset{PTB} training set. We then sort the non-terminals according to their normalized frequency and then show the frequency vs. index plot.}
        \label{fig:freq-dist}
\end{figure}




%To further verify that with the approximated computation, we select the non-terminals $\gN^{(\ell)}$ that will be computed for spans with length $\ell$ greedily from $\text{freq}(A,\ell)$ (i.e., select the non-terminals with the highest frequency). Then we execute the approximated version of the Inside-Outside algorithm and compute the unlabelled F1 score.
\iffalse
\begin{table}[]
    \centering
    \begin{tabular}{|c|c|c|c|}
    \hline
         & Corpus F1 & Sent F1 & TV \\
         \hline
         \makecell{No approx.} & 75.90 & 78.77 & 0 \\
         \hline
         $|\gN^{(\ell)}| = 20$ & 62.49 & 61.17 & 0.054 \\
         $|\gN^{(\ell)}| = 30$ & 70.29 & 70.92 & 0.045 \\
         $|\gN^{(\ell)}| = 40$ & 72.41 & 72.97 & 0.029\\
         \hline
    \end{tabular}
    \caption{Unlabelled F1 scores for approximate Inside-Outside algorithm with very few non-terminals to compute in each layer on \dataset{PTB} development set. $\gN^{(\ell)}$ denotes the set of non-terminals to compute for layer $\ell$ and is selected to be the non-terminals with top frequency for spans with length $\ell$. The PCFG is learned on \dataset{PTB} training dataset. Besides the parsing F1 results, we also show the TV distance between the exact computation and the approximated computation for 1-masking prediction.}
    \label{tab:few-nt-pcfg}
\end{table}
\fi



We verify the effect of restricting our computation to the frequent non-terminals on the 1-mask perplexity and the unlabeled F1 score of the approximate Inside-Outside algorithm in \Cref{tab:few-nt-pcfg-global}. Recall from \Cref{thm:io-optimal-mlm}, the 1-mask probability distribution for a given sentence $w_1, \cdots, w_L$ at any index $i$ is given by \cref{eq:1mask_pcfg}, and thus we can use \cref{eq:1mask_pcfg} to compute the 1-mask perplexity on the corpus. To measure the impact on 1-mask language modeling, we report the perplexity of the original and the approximate Inside-Outside algorithm on 200 sentences generated from PCFG. 

%Besides the F1 score, we also compute the total variation distance between the 1-masking distribution computed by the Inside-Outside algorithm and the approximated version. 
We observe that restricting the computation to the top-$40$ and $45$ frequent in-terminals and pre-terminals leads to $<6.5\%$ increase in average 1-mask perplexity. % with the average TV distance of the masked token predictions between the original and the approximate Inside-Outside algorithm being only $0.05$. 
Furthermore, the Labeled-Recall algorithm observes at most $4.24\%$ drop from the F1 performance of the original PCFG. %and the total variation between distributions is $0.05$. \abhishek{What is TV between distributions here? Please explain} 
If we further restrict the computation to the top-$20$ and $45$ in-terminals and pre-terminals, we can still get $71.91\%$ sentence F1 score, and the increase in average 1-mask perplexity is less than $8.6\%$. However, restricting the computation to $10$ in-terminals leads to at least $15\%$ drop in parsing performance.

Thus combining \Cref{thm:soft_attnt} and \Cref{tab:few-nt-pcfg-global}, we have the following informal theorem.


\begin{theorem}[Informal]\label{thm:approx-few-nt-informal}
    Given the PCFG $\gG = (\gN, \gI, \gP, n, p)$ learned on the English corpus, there exist  subsets $\tilde\gI\subset\gI,\tilde\gP\subset\gP$ with $|\tilde\gI| = 20, |\tilde\gP| = 45$, and an attention model with soft relative attention modules (\ref{eq:soft_attention}) with embeddings of size $275+40L$, $2L+1$ layers, and $20$ attention heads in each layer, that can simulate the Inside-Outside algorithm restricted to $\tilde\gI,\tilde\gP$
    on all sentences of length at most $L$ generated from $\gG$. The restriction introduces a $9.29\%$ increase in average 1-mask perplexity and $8.71\%$ drop in the parsing performance of the Labeled-Recall algorithm. 
    %Parsing using this approximated Inside-Outside algorithm gives $68.41\%$ corpus F1 $71.91\%$ sentence F1 on \dataset{PTB} dataset.
    %There exists an attention model
    %By computing the inside and outside probabilities for only the top-$20$ non-terminals ($|\tilde\gN| = 20$), we can reduce the size of the model in \Cref{thm:soft_attnt} to $20$ attention heads in each layer, $540+40L$ embedding dimension, and $2L$ layers to approximately execute the Inside-Outside algorithm on PCFG learned on English corpus. Parsing using this approximated Inside-Outside algorithm gives us $68.41\%$ corpus F1 $71.91\%$ sentence F1 on \dataset{PTB} dataset.
\end{theorem}

\iffalse
\begin{table}[!t]
    \centering
    \footnotesize
    \begin{tabular}{|c|c|c|c|}
    \hline
         Approximation & Corpus F1 & Sent F1 & ppl. \\
         \hline
         \makecell{No approx.} & 75.90 & 78.77 & 50.80 \\
         \hline
         $|\tilde\gI| = 10,|\tilde\gP|=45$ & 57.14 & 60.32 & 59.57 \\
         $|\tilde\gI| = 20,|\tilde\gP|=45$ & 68.41 & 71.91 & 55.16 \\
         $|\tilde\gI| = 40,|\tilde\gP|=45$ & 72.45 & 75.43 & 54.09 \\
         \hline
    \end{tabular}
    \iffalse
    \begin{tabular}{|c|c|c|c|c|}
    \hline
         & Corpus F1 & Sent F1 & TV & ppl. \\
         \hline
         \makecell{No approx.} & 75.90 & 78.77 & 0 & 50.80 \\
         \hline
         $|\tilde\gN| = 10$ & 57.14 & 60.32 & 0.114 & 58.11 \\
         $|\tilde\gN| = 20$ & 68.41 & 71.91 & 0.073 & 55.16 \\
         $|\tilde\gN| = 40$ & 72.45 & 75.43 & 0.050 & 54.09 \\
         \hline
    \end{tabular}
    \fi
    \iffalse
    \begin{tabular}{|c|c|c|}
    \hline
         & Corpus F1 & Sent F1 \\
         \hline
         \makecell{No approx.} & 75.90 & 78.77 \\
         \hline
         $|\tilde\gN| = 10$ & 57.14 & 60.32  \\
         $|\tilde\gN| = 20$ & 68.41 & 71.91  \\
         $|\tilde\gN| = 40$ & 72.45 & 75.43  \\
         \hline
    \end{tabular}
    \fi
    \caption{Experiment results by approximately computing the Inside-Outside algorithm with very few non-terminals. We show the unlabelled F1 scores on \dataset{PTB} development set as well as the 1-masking perplexity. $\tilde\gI$ ($\tilde\gP$) denotes the set of in(pre)-terminals to compute and are selected to be the in(pre)-terminals with top frequency. The PCFG is learned on \dataset{PTB} training dataset. The ppl. column denote the 1-masking perplexity on 200 sentences generated from the learned PCFG. %Besides the parsing F1 results, we also show the TV distance between the exact computation and the approximated computation for 1-masking prediction.
    }
    \label{tab:few-nt-pcfg-global}
\end{table}
\fi

If we plug in the average length $L\approx 25$ for sentences in \dataset{PTB}, we can get a model with $20$ attention heads, $1275$ hidden dimension, and $51$ layers. Compared with the construction in \Cref{thm:soft_attnt}, the size of the model is much closer to reality. The proof of \Cref{thm:approx-few-nt-informal} is shown in \Cref{sec:approx-few-nt}.
%Besides, this approximation doesn't affect the parsing performance much: compared with parsing using the Inside-Outside algorithm that achieves $75.90\%$ corpus F1 and $78.77\%$ sentence F1 on \dataset{PTB} dataset, the approximated computation shows a drop by $8.71\%$. 


\subsection{More discussions on low-rank approximation} \label{appendix:approx-low-rank-informal}
We hypothesize that we can  find linear transformation matrices $\{\mW^{(\ell)}\}_{\ell\le L}$ that can reduce the computations while retaining most of the performance, and our hypothesis is formalized as follow:
%The following informal theorem show the effectiveness of using this approximation, and please refer to \Cref{sec:approx-low-rank} for more details.

\begin{hypothesis}
     For the PCFG $\gG = (\gN, \gI, \gP, n, p)$ learned on the English corpus, there exists transformation matrices $\mW^{(\ell)}\in\R^{k^{(\ell)}\times |\tilde\gI|}$ for every $\ell \le L$, such that approximately simulating the Inside-Outside algorithm with $\{\mW^{(\ell)}\}_{\ell\le L}$ introduces \underline{small} error in the 1-mask perplexity and has \underline{minimal} impact on the parsing performance of the Labeled-Recall algorithm.
\end{hypothesis}



\Cref{tab:learned-transformation-global} verifies our hypothesis, and lead to \Cref{thm:approx-low-rank-informal}. Compared with the parsing results from \Cref{thm:approx-few-nt-informal}, the corpus and sentence F1 scores are nearly the same, and we further reduce the number of attention heads in each layer from $20$ to $15$. If we only use $10$ attention heads to approximately execute the Inside-Outside algorithm, we can still get $61.72\%$ corpus F1 and $65.31\%$ sentence F1 on \dataset{PTB} dataset, which is still much better than the Right-branching baseline. \Cref{thm:approx-low-rank-informal} shows that attention models with a size much closer to the real models (like BERT or RoBERTa) still have enough capacity to parse decently well (>70\% sentence F1 on \dataset{PTB}).

It is also worth noting that approximately executing the Inside-Outside algorithm using the transformation matrices $\{\mW^{(\ell)}\}_{\ell\le L}$ is very different from reducing the size of the PCFG grammar, since we use different matrix $\mW^{(\ell)}$ when computing the probabilities for spans with different length. If we choose to learn the same transformation matrix $\mW$ for all the layers $\ell$, the performance drops.
%Although the number of layers of our constructed models ($50$ to process average length sentence in \dataset{PTB}) are still relatively large compared to the models in reality ($12$ for BERT and RoBERTa), our approximations in \Cref{thm:approx-few-nt-informathm:approx-low-rank-informal} show strong hint that the PCFG learned on English corpus are ``highly compressible''.

\paragraph{More discussions on the transformation matrix $\mW^{(\ell)}$} We can observe that by introducing the transformation matrix $\mW^{(\ell)}$ generalized the first ingredient that only computes a small set of in-terminals $\tilde\gI$ and pre-terminals $\tilde\gP$, and in theory we can directly learn the transformation matrix $\mW^{(\ell)}$ from the original PCFG without reducing the size at first, i.e., $\mW^{(\ell)}\in\R^{k^{(\ell)}\times |\gI|}$. However empirically, if we directly learn $\mW^{(\ell)}$ from all the in-terminals $\gI$ but not from the top-20 frequent in-terminals $\tilde\gI$, the performance drops. Thus, we choose to learn the matrix $\mW^{(\ell)}$ starting from the most frequent in-terminals $\tilde\gI$. One possible explanation is that the learning procedure is also heuristic, and certainly may not learn the best transformation matrix.

Besides, we use the same transformation matrix $\mW^{(\ell)}$ when computing the inside and outside probabilities, and it is also natural to use different transformation matrices when computing the inside and outside probabilities. Recall that we learn the transformation $\mW^{(\ell)}$ by the Eigenvalue decomposition on matrix $\mX^{(\ell)}$, where $\mX^{(\ell)} = \sum_{s} \mX_s^{(\ell)} / \norm{\mX_s^{(\ell)}}_{\text{F}}$ and $\mX_s^{(\ell)} = \sum_{i,j:j-i=\ell} \vmu_s^{i,j}(\vmu_s^{i,j})^\top$. Then, we can also learn two matrices $\mW^{(\ell)}_{\text{inside}}$ and $\mW^{(\ell)}_{\text{outside}}$ through the Eigenvalue decomposition on matrices $\mX^{(\ell)}_{\text{inside}}$ and $\mX^{(\ell)}_{\text{outside}}$ respectively, where
\begin{align*}
    \mX^{(\ell)}_{\text{inside}} =& \sum_{s} \mX_{s,\text{inside}}^{(\ell)} / \norm{\mX_{s,\text{inside}}^{(\ell)}}_{\text{F}}, \\
    \mX_{s,\text{inside}}^{(\ell)} =& \sum_{i,j:j-i=\ell} \bm\alpha_s^{i,j}(\bm\alpha_s^{i,j})^\top, \\
    \mX^{(\ell)}_{\text{outside}} =& \sum_{s} \mX_{s,\text{outside}}^{(\ell)} / \norm{\mX_{s,\text{outside}}^{(\ell)}}_{\text{F}}, \\
    \mX_{s,\text{outside}}^{(\ell)} =& \sum_{i,j:j-i=\ell} \bm\beta_s^{i,j}(\bm\beta_s^{i,j})^\top.
\end{align*}
However empirically, we also find that the performance drops by using different transformation matrices for inside and outside probabilities computation, which may also be attributed to the non-optimality of our method to learn the transformation matrix.



\subsection{Proof for \cref{thm:approx-few-nt-informal}}\label{sec:approx-few-nt}

Note that in both \Cref{thm:hard_attnt} and \Cref{thm:soft_attnt}, in every layer $1 \le \ell \le L-1$, we use one attention head with parameters $\mK_A^{(\ell)}, \mQ_A^{(\ell)}, \mV_A^{(\ell)}$ to compute all the inside probabilities $\alpha(A, i, j)$ for all spans with length $\ell+1$, i.e. $j-i = \ell$. For layer $L+1 \le \ell \le 2L-1$, the model constructed in \Cref{thm:hard_attnt} uses two attention heads to compute the outside probabilities $\beta(A,i,j)$ for a specific non-terminal $A$ for spans with length $2L - \ell$, and the model constructed in \Cref{thm:soft_attnt} uses one attention heads to compute the outside probabilities $\beta(A,i,j)$ for a specific non-terminal $A$ for spans with length $2L - \ell$. Now to show how restricting the computations to certain non-terminals $\tilde\gI\cup\tilde\gP$ can reduce the size of the constructed models in \Cref{thm:hard_attnt,thm:soft_attnt} we classify the inside and outside probabilities into four categories: (1) the inside probabilities for pre-terminals, $\alpha(A,i,i)$ for $A\in\gP$; (2) the inside probabilities for in-terminals, $\alpha(A,i,j)$ for $A\in\gI$; (3) the outside probabilities for in-terminals, $\beta(A,i,j)$ for $A\in\gI$; and (4) the outside probabilities for pre-terminals, $\beta(A,i,i)$ for $A\in \gP$.

\paragraph{Category (1): the inside probabilities for pre-terminals} Recall that in the constructed model in \Cref{thm:hard_attnt,thm:soft_attnt}, the inside probabilities for pre-terminals $\alpha(A,i,i)$ for $A\in\gP$ is directly initialized from the PCFG rules, and thus do not need attention heads to compute. Thus, we can just use $O(|\gP|)$ dimensions to store all the inside probabilities for pre-terminals $\alpha(A,i,i)$ for $A\in\gP$. Although we can also only initialize the inside probabilities only for the pre-terminals $\tilde\gP$, i.e. initialize $\alpha(A,i,i)$ for $A\in\tilde\gP$ and use less embedding dimensions, empirically the performance will drop and thus we initialize all the probabilities $\alpha(A,i,i)$ for $A\in\gP$. Although we should store the probabilities for pre-terminals larger than the set $\tilde\gP$, there is indeed another technique to reduce the embedding dimensions. Note that since in the future computations, we only compute the probabilities for the in-terminals $\tilde\gI$, and not every pre-terminal $A\in\gP$ can be produced by in-terminals $B\in\tilde\gI$. Thus, we only need to store the pre-terminals $\gP_{\tilde\gI}$ that can be produced from $\tilde\gI$. Empirically, for PCFG learned on \dataset{PTB} dataset, $|\gP| = 720$, but if we choose $|\tilde\gI| = 20$, the number of pre-terminals that can be produced from $\tilde\gI$ drops to $|\gP_{\tilde\gI}| = 268 < 270$.
Specifically for the model in \Cref{thm:soft_attnt}, we need $|\gP_{\tilde\gI}|$ coordinates at each position to store these inside probabilities.

\paragraph{Category (2): the inside probabilities for in-terminals} The computation of the inside probabilities for in-terminals, $\alpha(A,i,j)$ for $A\in\gI$ happens from layer $1$ to layer $L-1$ in the constructed model in \Cref{thm:hard_attnt,thm:soft_attnt}. Note that from layer $1$ to layer $L-1$, the model only computes the probabilities for the in-terminals, since a span with a length larger than 1 cannot be labeled by a pre-terminal. Thus, if we only compute the inside probabilities for in-terminals $|\tilde\gI|$, we can reduce the number of attention heads in layer $1$ to layer $L-1$ from $O(|\gI|)$ to $O(|\tilde\gI|)$ since in \Cref{thm:hard_attnt,thm:soft_attnt} we use a constant number of attention heads to compute the probabilities for a single in-terminal. Specifically for the model in \Cref{thm:soft_attnt}, we only need $|\tilde\gI|$ attention heads from layer $1$ to layer $L-1$; besides, we need $(L-1)|\tilde\gI|$ coordinates at each position to store these inside probabilities.

\paragraph{Category (3): the outside probabilities for in-terminals} The computation of the outside probabilities for in-terminals, $\beta(A,i,j)$ for $A\in\gI$ happens from layer $L$ to layer $L-2$ in the constructed model in \Cref{thm:hard_attnt,thm:soft_attnt}. Note that in layer $L$, we only need to initialize the outside probabilities $\beta(A,1,L)$ for $A\in\gI$, thus do not need attention heads for computation (however we need attention heads to move the inside and outside probabilities in this layer, which cost 2 attention heads). Then from layer $L+1$ to layer $L-2$, the model computes the outside probabilities for the in-terminals $\beta(A,i,j)$ for $A\in\tilde\gI$. Thus if we only compute the outside probabilities for in-terminals $|\tilde\gI|$, we can reduce the number of attention heads in layer $1$ to layer $L-1$ from $O(|\gI|)$ to $O(|\tilde\gI|)$. Specifically for the model in \Cref{thm:soft_attnt}, we only need $|\tilde\gI|$ attention heads from layer $L$ to layer $L-2$; besides, we need $(L-1)|\tilde\gI|$ coordinates at each position to store these outside probabilities for in-terminals $\tilde\gI$.

\paragraph{Category (4): the outside probabilities for pre-terminals} The outside probabilities for pre-terminals $\beta(A,i,i)$ for $A\in\gP$ is only computed in the final layer in \Cref{thm:hard_attnt,thm:soft_attnt}. Thus if we choose to compute the probabilities for only $\tilde\gP$, we can reduce the number of attention heads in layer $2L-1$ from $O(|\gI|)$ to $O(|\tilde\gI|)$. Specifically for the model in \Cref{thm:soft_attnt}, we only need $|\tilde\gP|$ attention heads in layer $L-1$; besides, we need $|\tilde\gP|$ coordinates at each position to store these outside probabilities for in-terminals $\tilde\gP$. Also as mentioned in \Cref{sec:approx-overview}, if $|\tilde\gP| < c|\tilde\gI|$ for some constant $c$, we can also simulate the computations in the last layer with $|\tilde\gP|$ heads by $c$ layers with $|\tilde\gI|$ heads. In particular, if we choose $|\tilde\gP| = 45, |\tilde\gI|=20$, we can use 3 layers with $20$ attention heads in each layer to simulate the last layer with $45$ attention heads in the original construction.

\paragraph{Put everything together: proof of \Cref{thm:approx-few-nt-informal}} We choose $|\tilde\gP| = 45, |\tilde\gI|=20$. We can use $20$ attention heads in each layer, and we now count the number of layers and the embedding dimension we need. The number of layers is easy to compute, since we just need to use $3$ layers with $20$ attention heads to simulate the original $1$ layer with $45$ attention heads, thus the total number of layers is $2L-1 + (3-1) = 2L+1$. As for the embedding dimension, we need
\begin{align*}
    d = & |\gP_{\tilde\gI}| + (L-1)|\tilde\gI| + (L-1)|\tilde\gI| + |\tilde\gP| \\
    \le & 270 + (2L-2)|\tilde\gI| + |\tilde\gP| \\
    =& 275 + 2L|\tilde\gI|\\
    =& 275 + 40L.
\end{align*}

%For layer $L+1 \le \ell \le 2L-1$, the model constructed in \Cref{thm:hard_attnt} uses two attention heads to compute the outside probabilities $\beta(A,i,j)$ for a specific non-terminal $A$ for spans with length $2L - \ell$, and the model constructed in \Cref{thm:soft_attnt} uses one attention heads to compute the outside probabilities $\beta(A,i,j)$ for a specific non-terminal $A$ for spans with length $2L - \ell$. Thus, it is clear that if we only compute the inside and outside probabilities for a small set of non-terminals $\tilde\gN$, the dependency on $|\gN|$ reduces to $|\tilde\gN|$ for the number of attention heads in both constructions, i.e., for the construction in \Cref{thm:hard_attnt}, the number of attention heads drops from $4|\gN|$ to $4|\tilde\gN|$, and for the construction in \Cref{thm:soft_attnt}, the number of attention heads drops from $|\gN|$ to $|\tilde\gN|$.

%Besides reducing the number of attention heads, computing only the probabilities for important non-terminals can also reduce the number of embedding dimensions at each layer. Note that in the construction in \Cref{thm:hard_attnt} and \Cref{thm:soft_attnt}, we need to store all the inside and outside probabilities, and thus the embedding size has the term $|\gN| L$, where $L$ is the length of the sentence. The number of embedding dimensions for different models may have different coefficients before $|\gN| L$. Now if we only compute the probabilities for non-terminals $\tilde\gN$, we don't need to store the probabilities

\subsection{Proof for \cref{thm:approx-low-rank-informal}}\label{sec:approx-low-rank}

In this section, we show the details of how to further reduce the number of attention heads using structures across non-terminals, and add more discussion on how we learn the transformation matrices $\{\mW^{(\ell)}\}_{\ell \le L}$

\paragraph{Reducing the number of attention heads} We focus on reducing the number of attention heads to compute the inside and outside probabilities for the in-terminals $\tilde\gI$, since the computation for the outside probabilities for pre-terminals $\tilde\gP$ only happens in the final layer of the constructed model, and thus can use multiple layers to compute as long as $\tilde\gP$ is not too large.

For simplicity, we only show the details of how to reduce the number of attention heads to compute the inside probabilities for in-terminals $\tilde\gI$ in \Cref{thm:soft_attnt}, and the technique can be easily applied to the computation of outside probabilities for in-terminals $\tilde\gI$ in \Cref{thm:soft_attnt}, and the inside and outside probabilities for $\tilde\gI$ in \Cref{thm:hard_attnt}.

Recall from the proof of \Cref{thm:soft_attnt} that we at each layer $\ell$, we use a single attention head $\mK^{(\ell)}_A, \mQ^{(\ell)}_A$ to compute the inside probabilities $\alpha(A,i,j)$ for spans with length $\ell+1$, i.e., $j-i = \ell$. Specifically, for the attention head $\mK^{(\ell)}_A, \mQ^{(\ell)}_A$ at layer $\ell$, we want to compute and store the probability $\alpha(A, i-\ell, i)$ at position $i$. Thus we construct $\mK^{(\ell)}_A, \mQ^{(\ell)}_A$ such that the attention score $a_{i,j}^{A,(\ell)}$ when the position $i$ attends to position $j$ satisfies

{\small
\begin{align*}
     &a_{i,j}^{A,(\ell)} \\
     =& \text{ReLU}(\mK_{A}^{(\ell)} \ve_j^{(\ell-1)} + p_{j-i} - b_{j-i, \ell})^{\top}  \mQ_{A}^{(\ell)} \ve_i^{(\ell-1)} \\
     =& \sum_{B, C \in \gN} \Pr[A \to B C] \cdot \alpha(B, j+1, i) \cdot  \alpha (C, i-\ell, j),
\end{align*}
}

if $i - \ell \le j \le i-1$ and $0$ otherwise. Then, summing over all locations $j$ gives us $\alpha(A, i-\ell, i)$. Also, a key property of $\mK_{A}^{(\ell)}$ is that this key matrix does not depend on the non-terminal $A$, but only depends on $\ell$. Thus, if we have a set of coefficients $\{\omega_A^{(\ell)}\}_{A\in\gI}$, we can compute the linear combination of the inside probability $\sum_{A\in\tilde\gI} \omega_A^{(\ell)}\alpha(A, i-\ell, i)$ using one attention head, since if we choose
\[\mQ^{(\ell)} = \sum_{A\in\tilde\gI} \omega_A^{(\ell)}\mQ_{A}^{(\ell)}, \quad\mK^{(\ell)} = \mK_A^{(\ell)}, \forall A\in\tilde\gI,\]
we have the attention score

{\small
\begin{align*}
     &a_{i,j}^{(\ell)} \\
     = &\text{ReLU}(\mK^{(\ell)} \ve_j^{(\ell-1)} + p_{j-i} - b_{j-i, \ell})^{\top}  \mQ^{(\ell)} \ve_i^{(\ell-1)} \\
     = &\text{ReLU}(\mK^{(\ell)} \ve_j^{(\ell-1)} + p_{j-i} - b_{j-i, \ell})^{\top} \\
     &\cdot \left(\sum_{A\in\tilde\gI} \omega_A^{(\ell)}\mQ_{A}^{(\ell)}\right) \ve_i^{(\ell-1)} \\
     = &\sum_{A\in\tilde\gI} \omega_A^{(\ell)}\\
     &\cdot \text{ReLU}(\mK^{(\ell)} \ve_j^{(\ell-1)} + p_{j-i} - b_{j-i, \ell})^{\top}  \mQ_{A}^{(\ell)} \ve_i^{(\ell-1)} \\
     = &\sum_{A\in\tilde\gI} \omega_A^{(\ell)}\\
     &\cdot \text{ReLU}(\mK_A^{(\ell)} \ve_j^{(\ell-1)} + p_{j-i} - b_{j-i, \ell})^{\top}  \mQ_{A}^{(\ell)} \ve_i^{(\ell-1)} \\
     = &\sum_{A\in\tilde\gI} \omega_A^{(\ell)}\\
     &\cdot\left(\sum_{B, C \in \gN} \Pr[A \to B C] \cdot \alpha(B, j+1, i) \cdot  \alpha (C, i-\ell, j)\right),
\end{align*}
}
if $i - \ell \le j \le i-1$ and $0$ otherwise. Then, summing over all locations $j$ gives us $\sum_{A\in\tilde\gI} \omega_A^{(\ell)}\alpha(A, i-\ell, i)$. Then if we have a transformation matrix $\mW^{(\ell)}\in\R^{k^{(\ell)}\times |\tilde\gI|}$, we can use $k^{(\ell)}$ attention heads to compute $\mW^{(\ell)}\bm\alpha(i-\ell, i)$, where $\bm\alpha(i-\ell, i)\in\R^{|\tilde\gI|}$ is the vector that contains $\alpha(A, i-\ell, i)$ for all $A\in\tilde\gI$. Then after we use $k^{(\ell)}$ attention heads to compute the probabilities $\mW^{(\ell)}\bm\alpha(i-\ell, i)$ and stored them in position $i$'s embeddings, we can then use linear layer on position $i$ to recover the original probabilities by $\bm{\tilde\alpha}(i-\ell, i) = (\mW^{(\ell)})^{\dagger} \mW^{(\ell)}\bm\alpha(i-\ell, i)$, and use $\tilde\alpha(A, i-\ell, i)$ for $A\in\tilde\gI$ for the future computations.

\paragraph{Put everything together: proof of \Cref{thm:approx-low-rank-informal}} We choose $k^{(\ell)} = 15,|\tilde\gP| = 45, |\tilde\gI|=20$. Note that the embedding dimension doesn't change if we apply the approximation technique, and only the number of attention heads reduces from $20$ to $15$. Thus, the embedding dimension is still
\begin{align*}
    d =& |\gP_{\tilde\gI}| + (L-1)|\tilde\gI| + (L-1)|\tilde\gI| + |\tilde\gP| \\
    \le& 270 + (2L-2)|\tilde\gI| + |\tilde\gP|\\
    =& 275 + 2L|\tilde\gI|\\
    =& 275 + 40L.
\end{align*}
Also note that $|\tilde\gP| = 45 = 3\times 15$, and thus we can compute all the outside probabilities for pre-terminals $\tilde\gP$ by $3$ layers where each layer has $15$ attention heads.

\subsection{Experiment details in \Cref{sec:approx-overview}}\label{sec:approx-exp-details}
In this section, we provide the experiment details in \Cref{sec:approx-overview}. We use and modify the code~\citep{Spectral-Parser} to learn the PCFG from the \dataset{PTB} dataset and conduct the experiments with approximated computations. \citet{Spectral-Parser} implements the spectral learning method to learn PCFG~\citep{cohen2012spectral,cohen2014spectral} and is under MIT licence. We follow all the default hyperparameters in \citet{Spectral-Parser}, and we also follow the split of \dataset{PTB}: using \dataset{PTB} section 02-21 as the training set and \dataset{PTB} section 22 as the development set.

%%%%%%%%%%%%%%%%%%%%%%%%%%%%%%%%%%%%%%%%%%%%%%%%%%%

\section{Probing Masked Language Models for Parsing Information}\label{sec:mlmtoparsing}
%\rong{It's not very clear to me what this title is trying to say; maybe something like ``Probing Masked Language Model for Parsing Information''?}

\looseness=-1 \cref{sec:construction} shows that transformers can execute the Inside-Outside algorithm and contain syntactic information in their intermediate states. These results are existential, and it is unclear if models pre-trained under MLM possess similar information. 
%, such as information about spans in the parsing tree and marginal probabilities computed by the Inside-Outside algorithm. %what remains to be answered are: (1) whether the models trained with masked language modeling loss contain syntactic information, and (2) whether the models learn to ``execute'' the algorithm when trained with masked language modeling loss. 

\looseness=-1 One difficulty in answering this question is that syntactic probes on BERT-like models may leverage semantic cues to parse. To address this concern, we pre-train multiple RoBERTa models on synthetic datasets derived from English PCFG (\Cref{sec:pretrain-pcfg}), which eliminates semantic dependencies. We then probe the models for parse tree construction (\Cref{sec:parse}) and marginal probabilities (\Cref{sec:probe-marginal-probs}) to verify if they capture information computed by the Inside-Outside algorithm.
%Interestingly, we find probing patterns that indicate the existence of these span probabilities inside the contextual embeddings.
%, and thus do well on parsing and MLM (\Cref{thm:hard_attnt,thm:soft_attnt,thm:io-optimal-mlm}). The remaining questions are: Do attention models trained using masked language modeling really contain syntactic information? Do these models contain information computed by the Inside-Outside algorithm as shown by \Cref{thm:io-optimal-mlm}?


\subsection{Pre-training on PCFG}\label{sec:pretrain-pcfg}
%\haoyu{maybe pertaining details can be put into the appendix, and only briefly mention it here?}
\looseness=-1 We pre-train RoBERTa models with varying attention heads and layers on synthetic PCFG data. We denote the models with A$i$L$j$, where $i$ and $j$ indicate the number of attention heads and layers, respectively. Additional pre-training details are available in~\Cref{sec:pretraining-details}. \Cref{tab:pretraining-ppl} shows the perplexity for various models. We find that except for models with too few layers (A12L1) and too few attention heads (A3L12), other models have nearly the same perplexity. Further increasing depth and number of heads does not appear to improve the result.

\begin{table}[!t]
    \centering
    \footnotesize
    \begin{tabular}{|c|c|c|}
        \hline
        Model & Training ppl. & Validation ppl. \\
        \hline
        A12L12 & 106.16 & 106.68 \\
        A12L1 & 111.8 & 110.57 \\
        A12L3 & 108.09 & 105.79 \\
        A12L6 & 105.78 & 104.58 \\
        A3L12 & 120.52 & 117.39 \\
        A24L12 & 106.28 & 104.5 \\
        \hline
    \end{tabular}
    \caption{\looseness=-1 Perplexity of different models trained on synthetic PCFG data. A$i$L$j$  refers to a model with $i$ attention heads and $j$ layers. Except for models with few layers (A12L1) and  few attention heads (A3L12), trained models have nearly the same perplexity.}
    \label{tab:pretraining-ppl}
\end{table}

\begin{figure}[!t]
    \centering
    \includegraphics[width=0.7\linewidth]{figs/probe_parsing.pdf}
    \caption{Comparison between different probes (linear or a 2-layer neural net) under different settings. 2-layer probes achieve better parsing performance, compared to linear probes. %We also report the 
    The large performance gap of the probes on layer 0's embeddings from A12L12 and the best layer shows the existence of meaningful syntactic information in the contextualized embeddings.
    %The probe is set to be linear or a 2-layer neural net, and the input to the probe is layer 0's embedding from A12L12 or the embeddings from the layer that achieves the highest F1 score.
    }
    \label{fig:probe-parsing-comparison}
\end{figure}

\begin{table*}[!th]
    \centering
    \footnotesize
    \begin{tabular}{|c|c|c|ccccccc|}
    \hline
         & & & IO & A12L12 & A12L1 & A12L3 & A12L6 & A3L12 & A24L12 \\
    \hline
        \multirow{6}{*}{\rotatebox[origin=c]{90}{Linear}}& \multirow{2}{*}{\begin{turn}{90} \dataset{PCFG} \end{turn}} & Sent. F1 & 81.61 &  \textbf{71.34} & 63.16 & 69.96 & \textbf{71.23} & 64.71 & \textbf{70.76} \\
        & & Corpus F1 & 71.65 & \textbf{63.01} & 54.24 & 61.54 & \textbf{62.57} & 55.36 & \textbf{62.56} \\
        \cline{2-10}
        & \multirow{2}{*}{\begin{turn}{90} \dataset{PTB} \end{turn}} & Sent. F1 & 78.77 &  \textbf{69.31} & 62.99 & 68.22 & 68.13 & 61.56 & \textbf{68.79} \\
        & & Corpus F1 & 75.90 & \textbf{65.01} & 59.96 & \textbf{65.21} & \textbf{65.01} & 58.31 & \textbf{65.97} \\
        \cline{2-10}
        & \multirow{2}{*}{\begin{turn}{90} OOD \end{turn}} & Sent. F1 & 81.61 & \textbf{64.26} & 57.96 & 63.22 & \textbf{63.89} & 58.00 & \textbf{63.88} \\
        & & Corpus F1 & 71.65 & \textbf{60.98} & 54.29 & 59.79 & \textbf{60.58} & 54.39 & \textbf{60.62} \\
        \hline
        \multirow{6}{*}{\rotatebox[origin=c]{90}{2-layer NN}}& \multirow{2}{*}{\begin{turn}{90} \dataset{PCFG} \end{turn}} & Sent. F1 & 81.61 &  \textbf{73.71} & 64.80 & 72.62 & \textbf{73.60} & 62.55 & \textbf{73.27} \\
        & & Corpus F1 & 71.65 & \textbf{66.18} & 57.16 & \textbf{65.36} & \textbf{66.01} & 53.36 & \textbf{65.92} \\
        \cline{2-10}
        & \multirow{2}{*}{\begin{turn}{90} \dataset{PTB} \end{turn}} & Sent. F1 & 78.77 & \textbf{71.32} & 64.89 & 70.15 & \textbf{70.33} & 63.23 & \textbf{70.59} \\
        & & Corpus F1 & 75.90 & \textbf{68.07} & 62.09 & \textbf{67.25} & \textbf{67.31} & 60.59 & \textbf{67.93} \\
        \cline{2-10}
        & \multirow{2}{*}{\begin{turn}{90} OOD \end{turn}} & Sent. F1 & 81.61 & \textbf{66.99} & 59.89 & \textbf{66.21} & \textbf{66.56} & 57.60 & \textbf{67.18} \\
        & & Corpus F1 & 71.65 & \textbf{63.89} & 56.74 & 63.30 & 63.81 & 54.60 & \textbf{64.54} \\
        \hline
    \end{tabular}
    \caption{Parsing results for different models under different settings using Linear and 2-layer neural net probes, when compared to Inside-Outside algorithm (IO). We report the best F1 score achieved using any of the layer's embeddings. 
    %IO denotes the results using t%A$i$L$j$ denotes the model with $i$ attention heads and $j$ layers. 
    Scores within 1\% of the max (except IO) in each row are highlighted. %The numbers come from a single run since the probe is stable across multiple runs.
    Models except A12L1 and A3L12 give decent parsing F1 scores, and models with more layers or heads tend to get better F1 scores in general.
    }
    \label{tab:parsing-results}
\end{table*}

\begin{table*}
    \centering
    \footnotesize
    \begin{tabular}{|c|cccccc|}
    \hline
         \makecell{Span \\Length} & A12L12 & A12L1 & A12L3 & A12L6 & A3L12 & A24L12 \\
    \hline
        $\ell = 2$ &  .88 / \textbf{.93} & .83 / .88 &  .88 / .91  &  .88 / \textbf{.92}  &  .86 / .88 & .87 / \textbf{.92} \\
        $\ell = 3$ &  .79 / \textbf{.90} & .74 / .84 &  .80 / .88  &  .79 / \textbf{.89}  &  .77 / .84 & .79 / \textbf{.89}  \\
        $\ell = 4$ &  .69 / \textbf{.86} & .65 / .77 &  .69 / .82  &  .69 / .84  &  .66 / .78 & .69 / \textbf{.85}  \\
        $\ell = 5$ &  .62 / .79 & .57 / .70 &  .62 / .77   &  .61 / \textbf{.81} &  .58 / .69 & .62 / .79  \\
        $\ell = 10$ & .51 / \textbf{.77} & .48 / .68 & .51 / .75 & .51 / \textbf{.78} & .51 / .61 & .51 / .73 \\
        \hline
    \end{tabular}
    \caption{Probing for the ``normalized'' marginal probabilities of spans at different lengths on different pre-trained models. We report the Pearson correlation between the predicted probabilities  and the span marginal probabilities computed by the Inside-Outside algorithm on \dataset{PTB} datasets, for both the linear and the 2-linear net probes (separated by /). 
    %Two numbers in each entry denote the correlations with a linear probe and a 2-layer neural net probe respectively on the final layer of the model. 
    The high correlation indicates that the MLM pre-trained models approximately encode the marginal span probabilities of the Inside-Outside algorithm during pre-training. 
    %The numbers come from a single run since the probe is stable across multiple runs.
    }
    \label{tab:probe-probs-ptb}
\end{table*}


\subsection{Probing for constituency parse trees}\label{sec:parse}

\looseness=-1 We probe the language models pre-trained on synthetic PCFG data and show that these models indeed capture the ``syntactic information'', in particular, the structure of the constituency parse trees. % underlying the input sentences.

%In this part, we probe the pre-trained language models trained on PCFG data and show that simple probes can parse decently well, verifying that pre-training on mask language modeling captures the synthetic information.

\looseness=-1 \paragraph{Experiment setup} We mostly follow the probing procedure in \citet{vilares2020parsing} that predicts the relative depth of common ancestors between different token pairs and then constructs the constituency tree. Given a sentence $w_1w_2\dots w_L$ with parse tree $T$, we denote $\text{depth}(i,i+1)$ the depth of the least common ancestor of $w_i,w_{i+1}$ in the parse tree $T$. We want to find a probe $f^{(\ell)}$ to predict the relative depth $\text{tar}(i) = \text{depth}(i,i+1) - \text{depth}(i-1,i)$ for position $i$. In \citet{vilares2020parsing}, the probe $f^{(\ell)}$ is linear, and the input to the probe $f^{(\ell)}$ at position $i$ is the concatenation of the embeddings at position $i$ and the BOS (or EOS) token. Besides the linear probe $f^{(\ell)}$, we also experiment with the probe where $f^{(\ell)}$ is a 2-layer neural network with 16 hidden neurons. We consider three settings for probing: train and test the probe on synthetic PCFG data (\dataset{PCFG}); train and test on \dataset{PTB} dataset (\dataset{PTB}); and train on the synthetic PCFG data while test on \dataset{PTB} (out of distribution, OOD). The OOD setting serves as a baseline for a syntactic probe on \dataset{PTB} since semantic relations do not appear in the pre-trained model or the probe.
\iffalse
In \dataset{PCFG} setting, we train and test the probe $f^{(\ell)}$ on the synthetic PCFG data we generated, and in \dataset{PTB} setting we train on \dataset{PTB} sections 02-21 and test on \dataset{PTB} section 22~\citep{marcus1993building} without removing the punctuations. In OOD setting we train the probe on the synthetic \dataset{PCFG} dataset, but test on the \dataset{PTB} section 22, excluding nearly all the semantic information contained in the probe itself.
\fi

\iffalse
\begin{table*}[]
    \centering
    \scriptsize
    \begin{tabular}{|c|c|c|ccccccc|}
    \hline
         & & & IO & A12L12 & A12L1 & A12L3 & A12L6 & A3L12 & A24L12 \\
    \hline
        \multirow{6}{*}{\rotatebox[origin=c]{90}{Linear}}& \multirow{2}{*}{\begin{turn}{90} \dataset{PCFG} \end{turn}} & Sent. F1 & 81.61 &  30.50 / \textbf{71.34} & 30.69 / 63.16 & 31.04 / 69.96 & 31.01 / \textbf{71.23} & 30.80 / 64.71 & 31.09 / \textbf{70.76} \\
        & & Corpus F1 & 71.65 &  22.27 / \textbf{63.01} &  22.19 / 54.24 & 22.37 / 61.54 & 22.79 / \textbf{62.57} & 22.37 / 55.36 & 22.64 / \textbf{62.56} \\
        \cline{2-10}
        & \multirow{2}{*}{\begin{turn}{90} \dataset{PTB} \end{turn}} & Sent. F1 & 78.77 &  30.54 / \textbf{69.31} & 30.58 / 62.99 & 30.33 / 68.22 & 30.34 / 68.13 & 30.16 / 61.56 & 31.09 / \textbf{68.79} \\
        & & Corpus F1 & 75.90 & 27.01 / \textbf{65.01} &  27.33 / 59.96 & 27.06 / \textbf{65.21} & 27.01 / \textbf{65.01} & 26.94 / 58.31 & 27.70 / \textbf{65.97} \\
        \cline{2-10}
        & \multirow{2}{*}{\begin{turn}{90} OOD \end{turn}} & Sent. F1 & 81.61 & 25.55 / \textbf{64.26} & 25.74 / 57.96 & 25.47 / 63.22 & 25.57 / \textbf{63.89} & 25.33 / 58.00 & 25.44 / \textbf{63.88} \\
        & & Corpus F1 & 71.65 & 21.77 / \textbf{60.98} & 21.91 / 54.29 & 21.69 / 59.79 & 21.73 / \textbf{60.58} & 21.54 / 54.39 & 21.74 / \textbf{60.62} \\
        \hline
        \multirow{6}{*}{\rotatebox[origin=c]{90}{2-layer NN}}& \multirow{2}{*}{\begin{turn}{90} \dataset{PCFG} \end{turn}} & Sent. F1 & 81.61 &  39.06 / \textbf{73.71} & 41.11 / 64.80 & 36.93 / 72.62 & 40.65 / \textbf{73.60} & 32.06 / 62.55 & 40.88 / \textbf{73.27} \\
        & & Corpus F1 & 71.65 & 29.53 / \textbf{66.18}  & 30.89 / 57.16 & 28.06 / \textbf{65.36} & 30.43 / \textbf{66.01} & 24.25 / 53.36 & 30.70 / \textbf{65.92} \\
        \cline{2-10}
        & \multirow{2}{*}{\begin{turn}{90} \dataset{PTB} \end{turn}} & Sent. F1 & 78.77 & 39.31 / \textbf{71.32} & 40.48 / 64.89 & 38.37 / 70.15 & 38.88 / \textbf{70.33} & 38.14 / 63.23 & 38.74 / \textbf{70.59} \\
        & & Corpus F1 & 75.90 & 36.50 / \textbf{68.07} & 37.62 / 62.09 & 35.28 / \textbf{67.25} & 36.08 / \textbf{67.31} & 35.35 / 60.59 & 36.04 / \textbf{67.93} \\
        \cline{2-10}
        & \multirow{2}{*}{\begin{turn}{90} OOD \end{turn}} & Sent. F1 & 81.61 & 33.33 / \textbf{66.99} & 38.19 / 59.89 & 34.00 / \textbf{66.21} & 37.21 / \textbf{66.56} & 33.95 / 57.60 & 29.83 / \textbf{67.18} \\
        & & Corpus F1 & 71.65 & 29.27 / \textbf{63.89} & 33.82 / 56.74 & 30.31 / 63.30 & 32.88 / 63.81 & 29.49 / 54.60 & 26.69 / \textbf{64.54} \\
        \hline
    \end{tabular}
    \caption{The parsing results (unlabelled F1 score) for different models under different settings. Linear and 2-layer NN denote the classifier for the probes respectively. The 2 scores in each entry denote the F1 score using the (un-contextualized) embeddings of the zeroth layer ($\ell = 0$) and the best F1 score achieved using one of the layer's (contextualized) embeddings. The IO column denotes the results for parsing using the Inside-Outside algorithm. A$i$L$j$ denotes the model with $i$ attention heads and $j$ layers. We highlight the scores that are within 1\% to the max in each row.}
    \label{tab:parsing-results}
\end{table*}
\fi


\looseness=-1 \paragraph{Experiment results}  \Cref{fig:probe-parsing-comparison} reveals a substantial difference between the probing outcomes of layer 0 embeddings and those of the best layer in all settings.  Both probing approaches profit greatly from the representations of subsequent layers. %bes are both sensitive to syntactic information we want to test.

\looseness=-1\Cref{tab:parsing-results} shows probing results for different settings (\dataset{PCFG}, \dataset{PTB}, and OOD), different probes (linear or a 2-layer neural net) on different models. Except for A12L1 and A3L12, the linear and neural net probes give decent parsing scores (> 70\% sentence F1 for neural net probes) in both \dataset{PCFG} and \dataset{PTB} settings. As for the OOD setting, the performances achieved by the best layer drop by about 5\% compared with \dataset{PCFG} and \dataset{PTB}, but they are still much better than the performance achieved by the $0$-th layer embeddings. In this setting, there is no semantic information even in the probe itself and thus gives a baseline for the probes on \dataset{PTB} dataset that only uses syntactic information. As a comparison, the naive baseline, Right-branching (RB), reaches $<40\%$ for both sentence and corpus F1 score~\citep{li2020empirical} on \dataset{PTB} dataset, and if we use layer 0's embeddings to probe, the sentence F1 is $<41\%$ in all settings for all models. Our positive results on syntactic parsing support the claim that pre-training language models using MLM loss can indeed capture the structural information of the underlying constituency parse tree.

%\looseness=-1The A12L12 model achieves 69.31\% and 71.32\% unlabelled F1 for linear and 2-layer neural net probes in the \dataset{PTB} setting. These results are lower than the labeled sentence F1 achieved by BERT and RoBERTa models pre-trained on natural language, which are 78.2\% and 82.6\%, respectively~\citep{vilares2020parsing,arps2022probing}. 
%This gap suggests that models pre-trained on natural language contain semantic cues that aid in parsing.

%, since in the \dataset{PTB} setting, the probes may also leverage the semantic information, and the only difference comes from the embeddings used to probe.

%As for the OOD setting, the performances achieved by the best layer drop by about 5\% compared with \dataset{PCFG} and \dataset{PTB}, but they are still much better than the performance achieved by the $0$-th layer embeddings. In this setting, there is no semantic information even in the probe itself, and thus gives a baseline for the probes on \dataset{PTB} dataset that only uses syntactic data. Our positive results on syntactic parsing support the claim that pretraining language models on masked language models can indeed capture syntactic information.

%\haoyu{currently I think the following discussion can be put into the appendix. then in the appendix, i will show more plots and results on the probes with 3 adjacent tokens as input.}

%\haoyu{we may discuss some potential problems for probes here, e.g., one can get high F1 by getting correct on very short spans.}

%\haoyu{third exp: TBA. compare the parsing results using the embeddings from different layers}

\begin{figure}[!th]
\begin{subfigure}[t]{0.49\textwidth}
    \centering
    \includegraphics[width=0.7\linewidth]{figs/probe_prob_comparison.pdf}
    \caption{Compare linear/2-layer NN probes under \dataset{PTB} setting. We observe: (a) 2-layer NN probe has better performance, and (b) the probes give better performance on 12th-layer embeddings.} 
    %the input comes from the $0$-th/$12$-th layer, and the probe is linear/2-layer neural net. A 2-layer neural net is a better probe for marginal probabilities compared with a linear model, while their }
    \label{fig:probe-prob-comparison}
\end{subfigure}
\hfill
\begin{subfigure}[t]{0.49\textwidth}
    \centering
    \includegraphics[width=0.7\linewidth]{figs/probe_prob_setting_comparison.pdf}
    \caption{Performance of 2-layer neural net probe on the $12$-th layer embeddings under different settings. The closer correlation performance of the probe across  settings (including OOD) indicates true marginal probabilities captured by the trained probe. }
    %The correlations results indicate the 12-th layer contains the syntactic information computed by the Inside-Outside algorithm and is not overfitting to the training dataset.}
    \label{fig:probe-prob-setting-comparison}
\end{subfigure}
\caption{Comparison between different probes for marginal probabilities on the A12L12 model. The y-axis denotes correlation between the prediction and the target, and the x-axis denotes probes for different lengths.}
\end{figure}

\subsection{Probing for the marginal probabilities}\label{sec:probe-marginal-probs}
%In the previous part, we show that pre-training using MLM can capture syntactic data. In this part, we want to understand more about how pre-training on MLM can capture syntactic information. We probe the pre-trained language models trained on PCFG data and show that the models also contain syntactic information like the marginal probabilities computed by the Inside-Outside algorithm, thus empirically verifying \Cref{thm:io-optimal-mlm} that pre-training on MLM may implicitly execute some approximated version of Inside-Outside algorithm, and thus can do well on parsing task.

\looseness=-1\Cref{sec:parse} verifies that language models can capture structure information of the parse trees, but we still don't know if the model executes the Inside-Outside algorithm proposed in \Cref{sec:construct-io,sec:mlmandio}.
%\Cref{sec:construct-io,sec:mlmandio} suggest that a possible mechanism is to execute the Inside-Outside algorithm pr
In this subsection, we test if model representations can be used to predict marginal probabilities computed in the Inside-Outside algorithm. 

\looseness=-1\paragraph{Experiment setup} We train a probe to predict the normalized marginal probabilities for spans with a specific length. Fix the span length $\ell$, for each sentence $w_1w_2\dots w_L$, denote $\ve_1, \ve_2,\dots,\ve_L$ the embeddings from the last layer of the pre-trained language model. We want to find a probe $f^{(\ell)}$ such that for each span $[i,i+\ell-1]$ with length $\ell$, the probe $f^{(\ell)}([\ve_i;\ve_{i+\ell-1}])$ predicts the normalized marginal probability of span $[i,i+\ell-1]$, i.e. $\text{tar}(i,i+\ell-1) = s(i,i+\ell-1) / \max_{j,j'}s(j,j')$,
where $s(i,j) = \max_A \mu(A,i,j)$ is the marginal probability of span $[i,j]$ and $\mu(A,i,j)$ is given by eq.~\ref{eq:marginal_probability}.
The input to the probe $[\ve_i;\ve_{i+\ell-1}]\in\R^{2d}$ is the concatenation of $\ve_i$ and $\ve_{i+\ell-1}$. To test the sensitivity of our probe, we also take the embeddings from the $0$-th layer as input to the probe $f^{(\ell)}$.

\looseness=-1We give two options for the probe $f^{(\ell)}$: (1) linear, and (2) a 2-layer neural network with 16 hidden neurons, since the relation between the embeddings and the target may not be a simple linear function. Similar to the \Cref{sec:parse}, we also consider three settings: \dataset{PCFG}, \dataset{PTB}, and OOD.
%(1) \dataset{PCFG}, where we train and test $f^{(\ell)}$ on the synthetic \dataset{PCFG} dataset, (2) \dataset{PTB}, where we train and test $f^{(\ell)}$ on the \dataset{PTB} dataset, and (3) OOD, where we train $f^{(\ell)}$ on the synthetic \dataset{PCFG} dataset and test on the \dataset{PTB} dataset.


\paragraph{Experiment results} 

\looseness=-1\Cref{fig:probe-prob-comparison} reports the correlation between the span marginal probabilities and the predictions of the 4 different probes for A12L12 model. For both linear and 2-layer neural net probes, changing the input from layer 0 to layer 12 drastically increases the predicted correlation, which again suggests that the uncontextualized embeddings don't contain enough information about the marginal probabilities. Besides, the neural net can predict better on layer 12 embeddings, but performs nearly the same on layer 0, suggesting that the neural network is a better probe in this setting. % (more sensitive to the information).

\looseness=-1\Cref{fig:probe-prob-setting-comparison} compares the probing results under three different settings. Surprisingly, we find that the probe can achieve high correlation with the real marginal probabilities under all settings. Furthermore, we observe that there is almost no drop in performance when changing the test dataset from \dataset{PCFG} to \dataset{PTB} (\dataset{PCFG} setting and OOD setting). This result implies that the probe, along with the embeddings, indeed contains the syntactic information computed by the Inside-Outside algorithm and is not overfitting to the training dataset.

\looseness=-1\Cref{tab:probe-probs-ptb} shows the probing results on different pre-trained models. The results show that the neural network probe is highly correlated with the target for most pre-trained models, except for A12L1 and A3L12 models. Surprisingly, even for length $10$ spans, the neural network probe still achieves an F1 score of up to 78\% for the best model. The high correlation suggests that the pre-trained models contain certain syntactic information computed by the Inside-Outside algorithm. Overall, the results indicate that MLM training may incentivize the model to approximate the Inside-Outside algorithm, thus validating our constructions in \Cref{sec:construction}.

\iffalse
\begin{table*}
\begin{subtable}[h]{\textwidth}
    \centering
    \scriptsize
    \begin{tabular}{|c|c|c|c|c|c|c|}
    \hline
    \makecell{Length\\ of span} & \dataset{PTB} & \makecell{\dataset{PTB}\\ sent length $\le 10$} & \makecell{\dataset{PTB}\\ $11 \le$ sent length $\le 20$} & \makecell{\dataset{PTB}\\ $21 \le$ sent length $\le 30$} & \makecell{\dataset{PTB}\\ $31 \le$ sent length $\le 40$} \\
    \hline 
    2 & .86 / .88 / .87 & .88 / .88 / .88 & .87 / .88 / .87 & .86 / .88 / .87 & .87 / .89 / .87 \\
    3 & .77 / .79 / .79 & .84 / .83 / .85 & .78 / .79 / .79 & .78 / .79 / .80 & .78 / .79 / .80 \\
    4 & .66 / .69 / .69 & .77 / .76 / .75 & .68 / .70 / .70 & .67 / .69 / .69 & .67 / .70 / .71 \\
    5 & .58 / .62 / .62 & .77 / .77 / .74 & .60 / .64 / .63 & .58 / .62 / .61 & .60 / .63 / .63 \\
    \hline
    \end{tabular}
    \caption{Linear probing for the ``normalized'' marginal probabilities at different lengths. The three numbers in each entry denote the results of models with 3, 12, and 24 attention heads respectively. All models have 12 layers and 768 embedding dimensions.}
    \label{tab:lp-probs-ptb-attn}
\end{subtable}

\begin{subtable}[h]{\textwidth}
    \centering
    \scriptsize
    \begin{tabular}{|c|c|c|c|c|c|c|}
    \hline
    \makecell{Length\\ of span} & \dataset{PTB} & \makecell{\dataset{PTB}\\ sent length $\le 10$} & \makecell{\dataset{PTB}\\ $11 \le$ sent length $\le 20$} & \makecell{\dataset{PTB}\\ $21 \le$ sent length $\le 30$} & \makecell{\dataset{PTB}\\ $31 \le$ sent length $\le 40$} \\
    \hline 
    2 & .83 / .88 / .88 / .88 & .84 / .89 / .88 / .88 & .84 / .89 / .89 / .88 & .82 / .88 / .88 / .88 & .83 / .89 / .88 / .89 \\
    3 & .74 / .80 / .79 / .79 & .83 / .84 / .83 / .83 & .75 / .80 / .79 / .79 & .75 / .80 / .79 / .79 & .75 / .81 / .79 / .79 \\
    4 & .65 / .69 / .69 / .69 & .73 / .76 / .77 / .76 & .67 / .70 / .70 / .70 & .65 / .69 / .70 / .69 & .65 / .71 / .70 / .70 \\
    5 & .57 / .62 / .61 / .62 & .76 / .76 / .75 / .77 & .60 / .63 / .63 / .64 & .58 / .61 / .62 / .62 & .58 / .63 / .62 / .63 \\
    \hline
    \end{tabular}
    \caption{Linear probing for the ``normalized'' marginal probabilities at different lengths. The four numbers in each entry denote the results of models with 1, 3, 6, and 12 layers respectively. All models have 12 attention heads and 768 embedding dimensions.}
    \label{tab:lp-probs-ptb-layer}
\end{subtable}

\begin{subtable}[h]{\textwidth}
    \centering
    \scriptsize
    \begin{tabular}{|c|c|c|c|c|c|c|}
    \hline
    \makecell{Length\\ of span} & \dataset{PTB} & \makecell{\dataset{PTB}\\ sent length $\le 10$} & \makecell{\dataset{PTB}\\ $11 \le$ sent length $\le 20$} & \makecell{\dataset{PTB}\\ $21 \le$ sent length $\le 30$} & \makecell{\dataset{PTB}\\ $31 \le$ sent length $\le 40$} \\
    \hline 
    2 & .88 / .93 / .92 & .74 / .88 / .88 & .88 / .94 / .93 & .90 / .93 / .93 & .89 / .93 / .93 \\
    3 & .84 / .90 / .89 & .49 / .84 / .86 & .82 / .90 / .90 & .86 / .90 / .90 & .86 / .91 / .90 \\
    4 & .78 / .86 / .85 & .09 / .81 / .83 & .79 / .85 / .82 & .82 / .85 / .85 & .74 / .88 / .86 \\
    5 & .69 / .79 / .79 & .10 / .76 / .89 & .67 / .84 / .80 & .66 / .80 / .77 & .74 / .83 / .78 \\
    \hline
    \end{tabular}
    \caption{Probing for the ``normalized'' marginal probabilities at different lengths with a 2-layer neural net. The three numbers in each entry denote the results of models with 3, 12, and 24 attention heads respectively. All models have 12 layers and 768 embedding dimensions.}
    \label{tab:nnp-probs-ptb-attn}
\end{subtable}

\begin{subtable}[h]{\textwidth}
    \centering
    \scriptsize
    \begin{tabular}{|c|c|c|c|c|c|c|}
    \hline
    \makecell{Length\\ of span} & \dataset{PTB} & \makecell{\dataset{PTB}\\ sent length $\le 10$} & \makecell{\dataset{PTB}\\ $11 \le$ sent length $\le 20$} & \makecell{\dataset{PTB}\\ $21 \le$ sent length $\le 30$} & \makecell{\dataset{PTB}\\ $31 \le$ sent length $\le 40$} \\
    \hline 
    2 & .88 / .91 / .92 / .93 & .89 / .91 / .90 / .88 & .89 / .93 / .93 / .94 & .88 / .91 / .93 / .93 & .89 / .92 / .93 / .93 \\
    3 & .84 / .88 / .89 / .90 & .86 / .88 / .84 / .84 & .84 / .85 / .90 / .90 & .85 / .90 / .90 / .90 & .86 / .88 / .90 / .91 \\
    4 & .77 / .82 / .84 / .86 & .80 / .90 / .84 / .81 & .80 / .86 / .87 / .85 & .80 / .84 / .84 / .85 & .78 / .84 / .84 / .88 \\
    5 & .70 / .77 / .81 / .79 & .87 / .88 / .80 / .76 & .74 / .83 / .83 / .84 & .73 / .80 / .81 / .80 & .73 / .79 / .79 / .83 \\
    \hline
    \end{tabular}
    \caption{Probing for the ``normalized'' marginal probabilities at different lengths with a 2-layer neural net. The four numbers in each entry denote the results of models with 1, 3, 6, and 12 layers respectively. All models have 12 attention heads and 768 embedding dimensions.}
    \label{tab:nnp-probs-ptb-layer}
\end{subtable}
\caption{Probing for the ``normalized'' marginal probabilities at different lengths with a linear model or a 2-layer neural net. We show the Pearson correlation between the predicted probabilities and the probabilities computed by the Inside-Outside algorithm on \dataset{PTB} datasets and its subsets partitioned by different lengths of sentences.}
\label{tab:probe-probs-ptb}
\end{table*}
\fi

\subsection{Control tasks}\label{sec:control-task-main}


\begin{table*}
    \footnotesize
    \centering
    \begin{tabular}{|c|c|ccccccccccccc|}
    \hline
         & & L0 & L1 & L2 & L3 & L4 & L5 & L6 & L7 & L8 & L9 & L10 & L11 & L12 \\
         \hline
        \multirow{3}{*}{\rotatebox[origin=c]{90}{Linear}} & pred. rel. depth & .606 & .760 & .789 & .796 & .800 & .803 & .803 & .803 & .802 & .801 & .800 & .800 & .799 \\
         & control task & .758 & .677 & .645 & .626 & .620 & .610 & .608 & .617 & .599 & .595 & .612 & .606 & .608 \\
         & selectivity & -.152 & .083 & .144 & .170 & .180 & .193 & .195 & .186 & .203 & \textbf{.206} & .188 & .194 & .191 \\
         \hline
        \multirow{3}{*}{\rotatebox[origin=c]{90}{NN}} & pred. rel. depth & .616 & .771 & .804 & .810 & .814 & .807 & .815 & .802 & .795 & .810 & .806 & .803 & .776 \\
         & control task & .861 & .793 & .758 & .667 & .728 & .653 & .653 & .668 & .678 & .693 & .680 & .697 & .687 \\
         & selectivity & -.245 & -.022 & .046 & .143 & .086 & .154 & \textbf{.162} & .134 & .117 & .117 & .126 & .106 & .089 \\ 
         \hline
    \end{tabular}
    \caption{Computing the selectivity of constituency parsing probes with linear and 2-layer NN architectures (see \cref{sec:parse} and \cref{sec:control-task-main}). The ``pred. rel. depth'' rows denote the probing results for the relative depth of common ancestors in the constituency parse tree using different layers' representations of A12L12. We report the predicting accuracy under the \dataset{PTB} setting where the probe is trained and tested on \dataset{PTB} dataset. The ``control task'' rows denote the predicting accuracy for the control task on \dataset{PTB} dataset using different layers' representations of A12L12. The selectivity is the difference between the original task performance and the control task performance. We can observe that for all layers representations, the probe with a linear classifier has a larger selectivity.}
    \label{tab:parsing-control-task}
\end{table*}

\looseness=-1In probing experiments, it is crucial to ensure that the probing performance accurately reflects the presence of the specific information we intend to test. 
Consequently, it is undesirable for the probe to possess excessive power and be capable of learning all aspects (see \cref{sec:preliminary} for further discussions). 
\citet{chen2021probing} utilize ``sensitivity'' to assess the extent to which the probe captures the targeted information. The ``sensitivity'' of a probe is defined as the difference in probing performance between the layer of interest and the 0-th layer.
% (see \cref{sec:parse} and \cref{sec:probe-marginal-probs} for further details). 
Intuitively, a large gap indicates that the probe fails to perform adequately using representations from the 0-th layer but achieves better performance when utilizing representations from a later layer, thus confirming the presence of the targeted information. 
%In situations where there are two probe choices (e.g., a linear classifier or a 2-layer neural network), the option exhibiting greater ``sensitivity'' should be selected as it captures a relatively higher amount of the targeted information.

\looseness=-1\citet{hewitt2019designing} introduced another metric, known as ``selectivity'', to assess the degree to which the probe captures the targeted information. Broadly speaking, \citet{hewitt2019designing} devised a specific task referred to as the ``control task'' to evaluate the probe's capability to align with specific types of random labels. Subsequently, ``selectivity'' is defined as the difference in performance between the probe for the original task, utilizing the layer of interest, and the probe for the control task, also utilizing the layer of interest. Intuitively, a large gap suggests that the probe lacks sufficient expressive power, resulting in the performance boost originating from the representations of the layer being probed. %thus confirming the presence of specific information. 
%Similarly, in scenarios involving two probe choices (e.g., a linear classifier or a 2-layer neural network), the option exhibiting greater ``selectivity'' should be preferred as it captures a relatively higher amount of the targeted information.

\looseness=-1Note that a probe with higher ``sensitivity'' does not necessarily imply larger ``selectivity''. Nevertheless, as demonstrated in the subsequent parts (and appendix), the metrics of ``sensitivity'' and ``selectivity'' align for both the constituency parsing probes and the marginal probability probes (\cref{sec:control-task}). We sketch the control task design and results for the constituency parsing probe, and defer the preliminaries of control tasks in~\citet{hewitt2019designing} and the control tasks experiments for marginal probabilities probe to \cref{sec:control-task}.

\paragraph{Control task for constituency parsing } For the constituency parsing in \cref{sec:parse}, we follow the design of control task for sequence labeling problems~\citep{hewitt2019designing}.
Specifically, we have $y_i = \text{tar}(i) = \text{depth}(i,i+1) - \text{depth}(i-1,i)$ for position $i$. Then for the control task, for each word $w$, we uniformly sample $\phi(w) \in \{-1,0,1\}$, and then define the labels for the control task as $\hat y_{1:T} = [\phi(x_1), \phi(x_2),\dots,\phi(x_T)]$.

\paragraph{\emph{Selectivity} is aligned with \emph{Sensitivity}}

\cref{tab:parsing-control-task} provides a summary of the performance of the constituency parsing probe, employing different architectures (linear classifier and a 2-layer neural network with 16 hidden neurons), on the original task, control task, as well as the selectivity.

\looseness=-1From \cref{tab:parsing-control-task}, the probe with a 2-layer NN achieves slightly higher accuracy in predicting the relative depth of common ancestors, leading to a higher F1 score in parsing. However, its performance on the control task surpasses that of the probe with a linear classifier by a significant margin. This suggests that when using the ``selectivity'' metric, the linear probe outperforms the 2-layer neural network probe in recovering the constituency parse tree, aligning with the conclusions drawn using the ``sensitivity metric'' (see Figure \ref{fig:probe-parsing-comparison}, where the sensitivity of the linear probe is greater than that of the 2-layer NN probe). Experiment results for marginal probability control task (\Cref{sec:control-task}) also support the alignment of \emph{Selectivity} and \emph{Sensitivity}.

\section{Related Works}

%\looseness=-1\paragraph{Probing for transformers}
\paragraph{(Structural) probing}
\looseness=-1Several recent works on probing have aimed to study the encoded information in BERT-like models~\citep{rogers2020primer}. \citet{hewitt2019structural,reif2019visualizing,manning2020emergent,vilares2020parsing,maudslay2020tale,maudslay2021syntactic,chen2021probing,arps2022probing,jawahar2019does} have demonstrated that it is possible to predict various syntactic information present in the input sequence, including parse trees or POS tags, from internal states of BERT. 
%However, in contrast to these existing approaches that typically utilize a pre-trained model as-is, we adopt a close environment approach to understand the relationship between the data distribution, the masked language model objective, and the architecture to its ability to do syntactic parsing. We show for one particular pre-training data distribution, the pre-trained model's representation captures quantities correlated with the quantities of an optimal algorithm. We hope that our work can motivate future work in this direction.
In contrast to existing approaches that commonly employ a model pre-trained on natural language, we pre-train our model under PCFG-generated data to investigate the interplay between the data, the MLM objective, and the architecture's capacity for parsing. 
Besides syntax, probing has also been used to test other linguistic structures like semantics, sentiment, etc.~\citep{belinkov2017neural,reif2019visualizing,kim2020pre,richardson2020probing,vulic2020probing,conia-navigli-2022-probing}.
%e.g. the syntactic (structural) information~\citep{hewitt2019structural,reif2019visualizing,manning2020emergent,vilares2020parsing,maudslay2020tale,maudslay2021syntactic,chen2021probing,arps2022probing,jawahar2019does}. 


%However as mentioned in \citet{maudslay2021syntactic}, the probing success of the previous works on syntax emerged from the model using semantic cues to parse. 
%The probed pre-trained models had been pre-trained on natural language datasets, where the semantic structures are most likely correlated with the syntactic ones. Indeed, \citet{arps2022probing} tried to separate the semantics from syntax by training the probe on a mixture of natural language and manipulated data, however, the authors acknowledged the possibility of semantics still affecting the decision of the probe trained on manipulated data. 
%\paragraph{Other probings} Besides syntax, probing has been used for other linguistic structures like semantics, sentiment, etc.~\citep{belinkov2017neural,reif2019visualizing,kim2020pre,richardson2020probing,vulic2020probing,conia-navigli-2022-probing}.

\paragraph{Expressive power of transformers}
\looseness=-1\citet{Yun2020Are,yun2020n} show that transformers are universal sequence-to-sequence function approximators. Later, \citet{perez2021attention,bhattamishra2020computational} show that attention models can simulate Turing machines, with \citet{wei2022statistically} proposing statistically meaningful approximations of Turing machines. 
%Attention models with bounded size have been shown capable of recognizing deterministic context-free languages such as bounded-depth Dyck-k~\citep{yao2021self}. 
%The size of the constructed models, however, depends on the complexity of the target function and often requires arbitrary precision to encode the target function. 
%, that also exhibit good statistical learnability. 
%\citet{liu2022transformers} constructed (by hand) transformers that can efficiently simulate automata. 
To understand the behavior of moderate-size transformer architectures, many works have investigated specific classes of languages, e.g. bounded-depth Dyck languages~\citep{yao2021self}, modular prefix sums~\citep{anil2022exploring}, adders~\citep{nanda2023progress}, regular languages~\citep{bhattamishra2020ability}, and sparse logical predicates~\citep{edelman2022inductive}. \citet{merrill2022saturated} relate saturated transformers with constant depth threshold circuits, and \citet{liu2022transformers} provide a unified theory on understanding automata within transformers.
%Compared to the existing literature, our expressiveness power results are more related to the real natural language, since the PCFG we study is learned from natural language and the transformer we construct is of moderate size. \haoyu{is previous sentence OK? Or the following sentence?} 
These works study expressive power under a class of synthetic language. Compared to the prior works, our results are more related to the natural language, as we consider not only a class of synthetic language (PCFG), but also a specific PCFG tailored to the natural language.

% for inputs of a small range of lengths. 
 %and empirically showed the existence of such solutions in pre-trained transformers. 
 %However, their short-cut solution is not robust to OOD generalization, while our probes are all OOD generalizable since we pre-train on synthetic PCFG data while probing on the \dataset{PTB} dataset.
 %However, as evident from the name, shortcut solutions aren't robust to OOD generalization.
 %Interestingly, we observe that language models pre-trained on PCFG-generated data encode relevant information from the Inside-Outside algorithm for sentences from both natural language and PCFG-generated data, which suggests an implicit bias toward learning the general algorithm (generalizing to all input lengths). 
 %This may suggest that pre-training on purely synthetic data and data closer to natural language have different regimes. Besides, these results do not shed light on the expressivity of transformers needed to encode relevant syntactic and semantic information in languages. 
 %A careful study is left for future work.
 

 


%\paragraph{Grammar induction}

\section{Conclusion}\label{sec:conclusion}
In this work, we focus on addressing the fundamental challenge of OOD detection tasks, which is how to fully understand the semantic discrepancy between the ID/OOD samples. We reveal that the key to success in the realistic SCOOD task is to allocate as many ID samples in the unlabeled set correctly as possible. To this end, we propose a novel uncertainty-aware optimal transport scheme that introduces class-specific energy scores as guidance for effective label assignment. Experimental results show that our method achieves better performance than previous state-of-the-art methods on SCOOD benchmarks.

\textbf{Limitations.} In addition to temperature scaling, other techniques such as feature clipping applied in ReAct~\cite{sun2021react} also enhance the performance of energy score, so how to obtain an OOD score that best fits the SCOOD task can be further explored. Moreover, a setting highly related to SCOOD has been proposed in \cite{katz2022training} and formulated as a constrained optimization problem. We will also theoretically analyze these practical OOD settings in our feature work.

% \section*{Acknowledgments}
\textbf{Acknowledgments.} 
This work is supported by National Key R\&D Program of China under Grant 2020AAA0105701, National Natural Science Foundation of China (NSFC) under Grants 61872327, Major Special Science and Technology Project of Anhui, National Natural Science Foundation of China (62033012) and Ant Group through Ant Research Intern Program.


\section{Generalization, Limitation and Future Work}
The Matcha framework exhibits a high degree of generalizability thanks to the commonsense knowledge inside LLMs.
Without LLMs, a control algorithm, e.g. one trained with reinforcement learning \cite{Li23InternallyRewarded, Singh20COGConnecting}, may require massive datasets/interactions to learn
the common sense \cite{Singh20COGConnecting} of collaborating different modalities, yet being less efficient and generalizable.

However, interpreting the real world with language can be limited to the complexity of the task and the environment dynamics.
For example, advanced reasoning techniques such as decomposing may be required to deal with a complicated task,
where the task is decomposed into several sub-tasks to tackle separately. 
This automatic operation highlights the flexibility of LLMs but also poses challenges to the static language expression of a complex world
--- The vision-to-language module should be called multiple times with flexible queries.
This brings the requirement of vision-enabled LLMs \cite{Zhu23MiniGPT4Enhancing, Brohan23RT2Visionlanguageaction}, 
built on which the reasoning can be malleable. But multimodal LLMs are yet less controllable and accurate in terms of describing the scene
compared with a templated module.

Despite current limitations, multimodal LLMs gain increasing attention due to their great potential and flexibility.
Future work will explore the multimodal models \cite{Tong22VideoMAEMasked, Brohan23RT2Visionlanguageaction} to leverage unified features.

\paragraph{Acknowledgement}
Haoyu Zhao, Abhishek Panigrahi, and Sanjeev Arora are supported by funding from NSF, ONR, Simons Foundation, DARPA, and SRC. Rong Ge is supported by NSF Award DMS-2031849, CCF-1845171 (CAREER), CCF-1934964 (Tripods), and a Sloan Research Fellowship.

\bibliographystyle{acl_natbib}
\bibliography{ref.bib}

\section{Appendix for Proofs}

\paragraph{Proof of Theorem \ref{thm:main}.}

\begin{proof}
\label{proof:main}
Our proof has two steps. In Step 1, we will show that SimCLR is equivalent to minimizing the cross entropy loss defined in Eqn.~(\ref{eqn:cross-entropy}). 
In Step 2, we will show  that minimizing the cross-entropy loss 
is equivalent to spectral clustering on $\bfpi$. 
Combining the two steps together, we have proved our theorem. 

\textbf{Step 1: } SimCLR is equivalent to minimizing the cross entropy loss.

The cross-entropy loss takes expectation over 
$\bfW_\bfX\sim \mathbb{P}(\cdot ; \bfpi)$, 
which means $\bfW_\bfX$ has exactly one non-zero entry in each row $i$. By Lemma~\ref{lem:multinomial}, we know every row $i$ of $\bfW_\bfX$ is independent of other rows. Moreover, 
$\bfW_{\bfX,i}\sim \mathcal{M}(1, \bfpi_i/\sum_j \bfpi_{i,j})=\mathcal{M}(1, \bfpi_i)$, because $\bfpi_i$ itself is a probability distribution.
Similarly, we know $\bfW_\bfZ$ also has the row-independent property by sampling over $\mathbb{P}(\cdot;\bfK_\bfZ)$.
Therefore, by Lemma~\ref{lem:cross_split}, we know Eqn.~(\ref{eqn:cross-entropy}) is equivalent to:
\[
 -\sum_{i=1}^n \mathbb{E}_{\bfW_{\bfX,i}}[\log \mathbb{P}(\bfW_{\bfZ,i}=\bfW_{\bfX,i};\bfK_\bfZ)],
\]

This expression takes expectation over $\bfW_{\bfX,i}$ for the given row $i$. Notice that 
$\bfW_{\bfX,i}$ has exactly one non-zero entry, which equals $1$ (same for $\bfW_{\bfZ,i}$). 
As a result
we expand the above expression to be:
\begin{equation}
 -\sum_{i=1}^n \sum_{j\neq i} \Pr(\bfW_{\bfX,i,j}=1)\log \Pr(\bfW_{\bfZ,i,j}=1).
\label{eqn:detailed-expansion}    
\end{equation}


By Lemma~\ref{lem:multinomial}, $\Pr(\bfW_{\bfZ,i,j}=1)=\bfK_{\bfZ,i,j}/\|\bfK_{\bfZ,i}\|_1$ for $j\neq i$. Recall that $\bfK_\bfZ=(k(\bfZ_i-\bfZ_j))_{(i,j)\in[n]^2}$, which means 
$\bfK_{\bfZ,i,j}/\|\bfK_{\bfZ,i}\|_1=\frac{\exp(-\|\bfZ_i-\bfZ_j\|^2/{2\tau})}{\sum_{k\neq i}
\exp(-\|\bfZ_i-\bfZ_k\|^2/{2\tau})
}$ for $j\neq i$, when $k$ is the Gaussian kernel with variance $\tau$. 

Notice that $\bfZ_i=f(\bfX_i)$, so we know
\begin{equation}
-\log \Pr(\bfW_{\bfZ,i,j}=1)=
-\log \frac{\exp(-\|f(\bfX_i)-f(\bfX_j)\|^2/{2\tau})}{\sum_{k\neq i}
\exp(-\|f(\bfX_i)-f(\bfX_k)\|^2/{2\tau}),
}
\label{eqn:infonce-equivalence}    
\end{equation}


The right hand side is exactly the InfoNCE loss defined in Eqn.~(\ref{eqn:infonce}).
Inserting Eqn.~(\ref{eqn:infonce-equivalence}) into Eqn.~(\ref{eqn:detailed-expansion}), we get the SimCLR algorithm, which first samples augmentation pairs $(i,j)$ with $\Pr(\bfW_{\bfX,i,j}=1)$ for each row $i$, and then optimize the InfoNCE loss. 

\textbf{Step 2: } minimizing the cross entropy loss 
is equivalent to spectral clustering on $\bfpi$.


By Lemma~\ref{lem:convert_to_spectral}, we may further convert the loss to 
\begin{equation}
\label{eqn:main-theorem-repul-attr}
\min_{\bfZ}
-\sum_{(i,j)\in [n]^2} \mathbf{P}_{i,j}
\log k (\bfZ_i-\bfZ_j)+\log \mathbf{R}(\bfZ).
\end{equation}
Since $k$ is the Gaussian kernel, this reduces to \[
\min_\bfZ \mathrm{tr}(\bfZ^\top \mathbf{L}(\bfpi) \bfZ)
+\log \mathbf{R}(\bfZ),
\]

where we use the fact that $\mathbb{E}_{\bfW_\bfX\sim \mathbb{P}(\cdot; \bfpi)}[\mathbf{L}(\bfW_\bfX)]
=\mathbf{L}(\bfpi)
$, because the Laplacian operator is linear and $
\mathbb{E}_{\bfW_\bfX\sim \mathbb{P}(\cdot; \bfpi)}(\bfW_\bfX)=\bfpi
$.
\end{proof}

\paragraph{Proof of Theorem \ref{thm:clip}.}
\begin{proof}
Since $\bfW_\bfX\sim \mathbb{P}(\cdot;\bfpi_{\mathbf{A}, \mathbf{B}})$, we know 
$\bfW_\bfX$ has exactly one non-zero entry in each row, denoting the pair that got sampled. 
A notable difference compared to the previous proof is we now have $n_\mathcal{A}+n_\mathcal{B}$ objects in our graph. CLIP deals with this by taking a mini-batch of size $2N$, 
such that $n_\mathcal{A}=n_\mathcal{B}=N$, and adding the $2N$ InfoNCE losses together. We label the objects in $\mathcal{A}$ as $[n_\mathcal{A}]$, and the objects in $\mathcal{B}$ as $\{n_\mathcal{A}+1, \cdots, n_\mathcal{A}+n_\mathcal{B}\}$. 

Notice that $\bfpi_{\mathbf{A}, \mathbf{B}}$ is a bipartite graph, so the edges of objects in $\mathcal{A}$ will only connect to object in $\mathcal{B}$ and vice versa. We can define the similarity matrix in $\cZ$ as $\bfK_\bfZ$, 
where $\bfK_\bfZ(i, j+n_\mathcal{A})=\bfK_\bfZ(j+n_\mathcal{A},i)= k(\bfZ_i-\bfZ_j)$ for $i\in [n_\mathcal{A}], j\in [n_\mathcal{B}]$, and otherwise we set $\bfK_\bfZ(i,j)=0$. 
The rest is same as the previous proof. 
\end{proof}

\paragraph{Proof of Theorem \ref{thm:exponential}.}

\begin{proof}
\label{proof:exponential}
Since the objective function consists of a linear term combined with an entropy regularization, which is a strongly concave function, the maximization problem is a convex optimization problem. Owing to the implicit constraints provided by the entropy function, the problem is equivalent to having only the equality constraint. We then introduce the Lagrangian multiplier $\lambda$ and obtain the following relaxed problem:

$$
\widetilde{E}(\boldsymbol{\alpha})=\psi_{1}-\sum_{i=1}^n \alpha_{i} \psi_{i}+\tau \sum_{i=1}^n \alpha_{i}\log \alpha_{i}+\lambda\left(\boldsymbol{\alpha}^{\top} \mathbf{1}_n-1\right).
$$

As the relaxed problem is unconstrained, taking the derivative with respect to $\alpha_{i}$ yields

$$
\frac{\partial \widetilde{E}(\boldsymbol{\alpha})}{\partial \alpha_{i}}=-\psi_{i}+\tau\left(\log \alpha_{i}+\alpha_{i} \frac{1}{\alpha_{i}}\right)+\lambda=0.
$$

Solving the above equation implies that $\alpha_{i}$ takes the form
$
\alpha_{i}=\exp \left(\frac{1}{\tau} \psi_{i}\right) \exp \left(\frac{-\lambda}{\tau}-1\right).
$ Since $\alpha_{i}$ lies on the probability simplex, the optimal $\alpha_{i}$ is explicitly given by
$
\alpha^{*}_{i}=\frac{\exp \left(\frac{1}{\tau} \psi_{i}\right)}{\sum_{i^{\prime}=1}^n \exp \left(\frac{1}{\tau} \psi_{i^{\prime}}\right)} .
$ Substituting the optimal point into the objective function, we obtain
$$
\begin{aligned}
E\left(\boldsymbol{\alpha}^*\right)  &=\psi_1-\sum_{i=1}^n \frac{\exp \left(\frac{1}{\tau} \psi_{i}\right)}{\sum_{i^{\prime}=1}^n \exp \left(\frac{1}{\tau} \psi_{i^{\prime}}\right)} \psi_{i}+\tau \sum_{i=1}^n \frac{\exp \left(\frac{1}{\tau} \psi_{i}\right)}{\sum_{i^{\prime}=1}^n \exp \left(\frac{1}{\tau} \psi_{i^{\prime}}\right)}\log \frac{\exp \left(\frac{1}{\tau} \psi_{i}\right)}{\sum_{i^{\prime}=1}^n \exp \left(\frac{1}{\tau} \psi_{i^{\prime}}\right)} \\
& =\psi_1 - \tau \log \left(\sum_{i=1}^n \exp \left(\frac{1}{\tau} \psi_{i}\right)\right).
\end{aligned}
$$
Thus, the Lagrangian dual function is given by
\begin{equation*}
-E\left(\boldsymbol{\alpha}^*\right)= -\tau \log \frac{\exp \left(\frac{1}{\tau} \psi_{1}\right)}{\sum_{i=1}^n \exp \left(\frac{1}{\tau} \psi_{i}\right)}.\qedhere
\end{equation*}
\end{proof}



\section{More on Experiments} \label{section: experiment_details}

\paragraph{CIFAR-10 and CIFAR-100} CIFAR-10 ~\citep{krizhevsky2009learning} and CIFAR-100 ~\citep{krizhevsky2009learning} are well-known classic image classification datasets. Both CIFAR-10 and CIFAR-100 contain a total of 60k $32 \times 32$ labeled images of different classes, with 50k for training and 10k for testing. CIFAR-10 is similar to CIFAR-100, except there are 10 different classes in CIFAR-10 and 100 classes in CIFAR-100.

\paragraph{TinyImageNet} TinyImageNet ~\citep{le2015tiny} is a subset of ImageNet ~\citep{deng2009imagenet}. There are 200 different object classes in TinyImageNet, with 500 training images, 50 validation images, and 50 test images for each class. All the images in TinyImageNet are colored and labeled with a size of $64 \times 64$.

\textbf{Pseudo-code.} Algorithm \ref{alg:Training Procedure} presents the pseudo-code for our empirical training procedure.

\begin{algorithm}[!htbp]
\caption{Training Procedure}
\label{alg:Training Procedure}
\begin{algorithmic}[1]
\REQUIRE trainable encoder network $f$, batch size $N$, augmentation strategy \textit{aug}, loss function $L$ with hyperparameters \textit{args}
\FOR {sampled minibatch ${x_i}_{i=1}^N$}
\FORALL{$i \in { 1, ..., N }$}
\STATE draw two augmentations $t_i = \textit{aug}\left(x_i\right) $, $t_i' = \textit{aug}\left(x_i\right) $
\STATE $z_i = f\left(t_i\right)$, $z_i' = f\left(t_i'\right)$
\ENDFOR
\STATE compute loss $\mathcal{L} = L(N, z, z', \textit{args})$
\STATE update encoder network $f$ to minimize $\mathcal{L}$
\ENDFOR
\STATE \textbf{Return} encoder network $f$
\end{algorithmic}
\end{algorithm}

We also provide the pseudo-code for our core loss function used in the training procedure in Algorithm \ref{alg:Core loss}. The pseudo-code is almost identical to SimCLR's loss function, with the exception of an extra parameter $\gamma$.

\begin{algorithm}[!htbp]
\caption{Core loss function $\mathcal{C}$}
\label{alg:Core loss}
\begin{algorithmic}[1]
\REQUIRE batch size $N$, two encoded minibatches $z_1, z_2$, $\gamma$, temperature $\tau$
\STATE $z = \textit{concat}\left(z_1, z_2\right)$
\FOR {$i \in {1, ..., 2N }, j \in {1, ..., 2N}$ }
\STATE $s_{i,j} = \Vert z_i - z_j \Vert_2^{\gamma}$
\ENDFOR
\STATE \textbf{define} $l(i, j)$ \textbf{as} $l(i, j) = - \log \frac{exp\left(s_{i,j}/\tau \right)}{\sum_{k=1}^{2N} \mathbf{1}{[k \ne i]} exp\left(s{i, j} / \tau \right)} $
\STATE \textbf{Return} $\frac{1}{2N} \sum_{k=1}^N\left[l(i, i+N) + l(i+N, i)\right]$
\end{algorithmic}
\end{algorithm}

Utilizing the core loss function $\mathcal{C}$, we can define all kernel loss functions used in our experiments in Table \ref{table: loss definition}. For all $z_i \in z$ with even dimensions $n$, we define $z_{L_i} = z_i\left[0:n/2\right]$ and $z_{R_i} = z_i\left[n/2:n\right]$.

\begin{table}[ht]
\centering
\begin{tabular}{{@{}l|l@{}}}
Kernel  &  Loss function \\ \midrule
Laplacian & $\mathcal{C}\left(N, z, z', \gamma=1, \tau\right)$\\ \midrule
Sum       & $\lambda * \mathcal{C}\left(N, z, z', \gamma=1, \tau_1\right) + (1-\lambda) * \mathcal{C}\left(N, z, z', \gamma=2, \tau_2\right)$  \\ \midrule
Concatenation Sum&$\lambda * \mathcal{C}\left(N, z_L, z'_L, \gamma=1, \tau_1\right) + (1-\lambda) * \mathcal{C}\left(N, z_R, z'_R, \gamma=2, \tau_2\right)$\\ \midrule
$\gamma = 0.5$ & $\mathcal{C}\left(N, z, z', \gamma=0.5, \tau\right)$          \\ 

\end{tabular}

\caption{Definition of kernel loss functions in our experiments}
\label {table: loss definition}
\end{table}

\textbf{Baselines.} We reproduce the SimCLR algorithm using PyTorch Lightning~\citep{PytorchLightning}.

\textbf{Encoder details.}
The encoder $f$ consists of a backbone network and a projection network. We employ ResNet50~\citep{ResNet} as the backbone and a 2-layer MLP (connected by a batch normalization~\citep{ioffe2015batch} layer and a ReLU \cite{nair2010rectified} layer) with hidden dimensions 2048 and output dimensions 128 (or 256 in the concatenation kernel case).

\textbf{Encoder hyperparameter tuning.}
For each encoder training case, we randomly sample 500 hyperparameter groups (sample details are shown in Table \ref{table: Hyperparameter sample}) and train these samples simultaneously using Ray Tune ~\citep{RayTune}, with the ASHA scheduler~\citep{li2018massively}. Ultimately, the hyperparameter group that maximizes the online validation accuracy (integrated in PyTorch Lightning) within 5000 validation steps is chosen for the given encoder training case.

\begin{table}[ht]
\centering

\begin{tabular}{@{}l|l|l@{}}
\midrule
Hyperparameter  & Sample Range & Sample Strategy \\ \midrule
start learning rate & $\left[10^{-2}, 10\right]$ & log uniform \\ \midrule
$\lambda$       & $\left[0, 1\right]$ & uniform \\ \midrule
$\tau$, $\tau_1$, $\tau_2$ & $\left[0, 1\right]$ & log uniform \\ \midrule
\end{tabular}

\caption{Hyperparameters sample strategy}
\label {table: Hyperparameter sample}
\end{table}

\textbf{Encoder training.} 
We train each encoder using the LARS optimizer~\citep{LARSOptimizer}, LambdaLR Scheduler in PyTorch, momentum 0.9, weight decay $10^{-6}$, batch size 256, and the aforementioned hyperparameters for 400 epochs on a single A-100 GPU.

\textbf{Image transformation.} The image transformation strategy, including augmentation, is identical to the default transformation strategy provided by PyTorch Lightning.

\textbf{Linear evaluation.}
The linear head is trained using the SGD optimizer with a cosine learning rate scheduler, batch size 64, and weight decay $10^{-6}$ for 100 epochs. The learning rate starts at $0.3$ and ends at $0$.

\textbf{Moco Experiments.} We also tested our method based on MoCo~\citep{he2019moco}. The results are summarized in Table \ref{tab:results-moco}. Here we choose ResNet18~\citep{ResNet} as the backbone and set a temperature of $0.1$ as default. For our simple sum kernel, we set $\lambda=0.8$. The results show that our method outperforms the original MoCo method.

\begin{table}[thb]
\centering
\caption{MoCo Experiment Results on CIFAR-10 and CIFAR-100.}
\label{tab:results-moco}
\resizebox{\textwidth}{!}{%
\begin{tabular}{@{}c|ccc|ccc@{}}
\toprule
\multirow{3}{*}{Method} & \multicolumn{3}{c|}{CIFAR-10} & \multicolumn{3}{c}{CIFAR-100} \\ \cmidrule(lr){2-4} \cmidrule(lr){5-7} 
                        & 200 epochs & 400 epochs    & 1000 epochs   & 200 epochs & 400 epochs & 1000 epochs         \\ \midrule
MoCo (repro.)         & $76.41 \pm 0.12$    & $80.01 \pm 0.15$          & $84.45 \pm 0.08$    & $\mathbf{47.02 \pm 0.11}$ & $52.50 \pm 0.07$ & $57.62 \pm 0.15$            \\
\midrule
Laplacian Kernel        & ${78.09 \pm 0.10}$    & $\mathbf{83.85 \pm 0.09}$          & $\mathbf{88.34 \pm 0.16}$    & $46.12 \pm 0.22$   & $53.44 \pm 0.17$ & $59.10 \pm 0.14$        \\
Simple Sum Kernel & $\mathbf{78.12 \pm 0.15}$   & $83.23 \pm 0.18$ & $87.50 \pm 0.20$ & $46.65 \pm 0.06$ & $\mathbf{53.62 \pm 0.19}$ & $\mathbf{59.83 \pm 0.12}$\\
\bottomrule
\end{tabular}
}
\end{table}



\section{More Experiments on Synthetic Data}


Consider a scenario with $n$ clusters, each containing $k$ vertices. Let the probability of vertices $u$ and $v$ from the same cluster belonging to $\bfpi$ be $p$. Conversely, for vertices $u$ and $v$ from different clusters, let the probability of belonging to $\pi$ be $q$. We generate the graph $\bfpi$ randomly, based on $p$ and $q$. We experiment with values of $k=100$ and $n=6$ for ease of visualization, embedding all points in a two-dimensional space. Each vertex's initial position originates from a normal distribution. In each iteration, we sample a subgraph of $\bfpi$ uniformly, ensuring each vertex has an out-degree of $1$. We then optimize the corresponding vectors using InfoNCE loss with an SGD optimizer and iterate until convergence. Our experimental setup consists of an SGD learning rate of $1$, an InfoNCE loss temperature of $0.5$, and a batch size of $50$. We evaluate two scenarios with different $p$ and $q$ values: $p=1$, $q=0$, and $p=0.75$, $q=0.2$. The results of these experiments are visualized in Figure \ref{fig:vis-spectral-cluster}. The obtained embeddings exhibit the hallmark pattern of spectral clustering of graph $\bfpi$.

\begin{figure}[!tb]
\centering
\subfigure{
\includegraphics[width=1\textwidth]{Figures/cluster_pi.png}
\label{fig:vis-cluster}
}
\subfigure{
\includegraphics[width=1\textwidth]{Figures/noised_cluster_pi.png}
\label{fig:vis-noised-cluster}
}
\caption{Visualizations of the optimization process using InfoNCE Loss on the vectors corresponding to $\bfpi$. Points of identical color belong to the same cluster within $\bfpi$. To showcase the internal structure of $\bfpi$, we randomly select 10 vertices from each cluster to display the edge distribution of $\bfpi$.}
\label{fig:vis-spectral-cluster}
\end{figure}




\end{document}
