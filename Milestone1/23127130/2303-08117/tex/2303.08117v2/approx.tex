\section{Omitted Details in \Cref{sec:approx-overview}}\label{sec:approx-detailed}
In \Cref{sec:approx-overview}, we claim that it is possible to approximately execute the Inside-Outside algorithm for PCFG learned on \dataset{PTB} dataset, and can drastically reduce the size of our constructed model with minimal impact on the 1-masking predictions and parsing performance (\Cref{thm:approx-low-rank-informal}) by applying two ingredients: restricting the computations to few non-terminals and utilizing the underlying low-rank structure between the non-terminals. 
%We overview the techniques in \Cref{sec:approx-overview}, and we show the details in this section.
This section is organized as follows: In \Cref{appendix:small_nonterminal_subset}, we show more intuition and experiment results on why we can restrict the computation of the inside-outside algorithm to a small subset of non-terminals. In \Cref{appendix:approx-low-rank-informal}, we add more discussions on the second ingredient (utilizing the low-rank structure). Then in \Cref{sec:approx-few-nt}, we show the details why restricting the computations of few non-terminals can reduce the size of the attention model. In \Cref{sec:approx-low-rank}, we show the detailed proof of \Cref{thm:approx-low-rank-informal}. Finally in \Cref{sec:approx-exp-details}, we show the experiment details in \Cref{sec:approx-overview}.


\subsection{More discussions on computation with few non-terminals} \label{appendix:small_nonterminal_subset}
We hypothesize that we can focus only on a few non-terminals while retaining most of the performance.

\begin{hypothesis}\label{hyp:small_nonterminal_subset}
     For the PCFG $\gG = (\gN, \gI, \gP, n, p)$ learned on the English corpus, there exists $\tilde\gI\subset\gI,\tilde\gP\subset\gP$ with $|\tilde\gI|\ll |\gI|, |\tilde\gP|\ll |\gP|$, such that simulating Inside-Outside algorithm with $\tilde\gI \cup \tilde\gP$ non-terminals introduces \underline{small} error in the 1-mask perplexity and has \underline{minimal} impact on the parsing performance of the Labeled-Recall algorithm.
\end{hypothesis}


To find candidate sets $\tilde\gI,\tilde\gP$ for our hypothesis, we check the frequency of different non-terminals appearing at the head of spans in the parse trees of the \dataset{PTB}~\citep{marcus1993building} training set. We consider the Chomsky-transformed (binarized) parse trees for sentences in the \dataset{PTB} training set, and collect the labeled spans $\{(A, i, j)\}$ from the parse trees of all sentences. For all non-terminals $A$, we compute $\text{freq}(A)$, which denotes the number of times non-terminal $A$ appears at the head of a span. %Formally, 
%\[\text{freq}(A,\ell) := \sum_{(B,j,j')\in \{T_i\}_i}\mathbb I\{A = B, j'-j+1 = \ell\}.\]
\Cref{fig:freq-dist} shows the plot of $\text{freq}(A)$ for in-terminals and pre-terminals, with the order of the non-terminals sorted by the magnitude of $\text{freq}(\cdot)$. We observe that an extremely small subset of non-terminals have high frequency, which allows us to restrict our computation for the inside and outside probabilities to the few top non-terminals sorted by their $\text{freq}$ scores. We select the top frequent non-terminals as possible candidates for forming the set $\tilde\gN$.

%appears with high frequency in a specific length, and thus computing only the top-frequent non-terminals should not affect the computation a lot intuitively.

\begin{figure}[!t]
     \centering
     \iffalse
    \begin{subfigure}[t]{0.4\textwidth}
        \includegraphics[width=\linewidth]{figs/fig-in-global.png}
    \end{subfigure}
    \begin{subfigure}[t]{0.4\textwidth}
        \includegraphics[width=\linewidth]{figs/fig-pre-global.png}
    \end{subfigure}
    \fi
    \includegraphics[width=0.8\linewidth]{figs/fig-nt-global.png}
    \iffalse
    \begin{subfigure}[t]{0.48\textwidth}
        \includegraphics[width=\linewidth]{figs/nts-frequency.pdf}
    \end{subfigure}
    \fi
    
        \caption{Plot for the frequency distribution of in-terminals ($\gI$) and pre-terminals ($\gP$). We compute the number of times a specific non-terminal appears in a span of a parse tree in the \dataset{PTB} training set. We then sort the non-terminals according to their normalized frequency and then show the frequency vs. index plot.}
        \label{fig:freq-dist}
\end{figure}




%To further verify that with the approximated computation, we select the non-terminals $\gN^{(\ell)}$ that will be computed for spans with length $\ell$ greedily from $\text{freq}(A,\ell)$ (i.e., select the non-terminals with the highest frequency). Then we execute the approximated version of the Inside-Outside algorithm and compute the unlabelled F1 score.
\iffalse
\begin{table}[]
    \centering
    \begin{tabular}{|c|c|c|c|}
    \hline
         & Corpus F1 & Sent F1 & TV \\
         \hline
         \makecell{No approx.} & 75.90 & 78.77 & 0 \\
         \hline
         $|\gN^{(\ell)}| = 20$ & 62.49 & 61.17 & 0.054 \\
         $|\gN^{(\ell)}| = 30$ & 70.29 & 70.92 & 0.045 \\
         $|\gN^{(\ell)}| = 40$ & 72.41 & 72.97 & 0.029\\
         \hline
    \end{tabular}
    \caption{Unlabelled F1 scores for approximate Inside-Outside algorithm with very few non-terminals to compute in each layer on \dataset{PTB} development set. $\gN^{(\ell)}$ denotes the set of non-terminals to compute for layer $\ell$ and is selected to be the non-terminals with top frequency for spans with length $\ell$. The PCFG is learned on \dataset{PTB} training dataset. Besides the parsing F1 results, we also show the TV distance between the exact computation and the approximated computation for 1-masking prediction.}
    \label{tab:few-nt-pcfg}
\end{table}
\fi



We verify the effect of restricting our computation to the frequent non-terminals on the 1-mask perplexity and the unlabeled F1 score of the approximate Inside-Outside algorithm in \Cref{tab:few-nt-pcfg-global}. Recall from \Cref{thm:io-optimal-mlm}, the 1-mask probability distribution for a given sentence $w_1, \cdots, w_L$ at any index $i$ is given by \cref{eq:1mask_pcfg}, and thus we can use \cref{eq:1mask_pcfg} to compute the 1-mask perplexity on the corpus. To measure the impact on 1-mask language modeling, we report the perplexity of the original and the approximate Inside-Outside algorithm on 200 sentences generated from PCFG. 

%Besides the F1 score, we also compute the total variation distance between the 1-masking distribution computed by the Inside-Outside algorithm and the approximated version. 
We observe that restricting the computation to the top-$40$ and $45$ frequent in-terminals and pre-terminals leads to $<6.5\%$ increase in average 1-mask perplexity. % with the average TV distance of the masked token predictions between the original and the approximate Inside-Outside algorithm being only $0.05$. 
Furthermore, the Labeled-Recall algorithm observes at most $4.24\%$ drop from the F1 performance of the original PCFG. %and the total variation between distributions is $0.05$. \abhishek{What is TV between distributions here? Please explain} 
If we further restrict the computation to the top-$20$ and $45$ in-terminals and pre-terminals, we can still get $71.91\%$ sentence F1 score, and the increase in average 1-mask perplexity is less than $8.6\%$. However, restricting the computation to $10$ in-terminals leads to at least $15\%$ drop in parsing performance.

Thus combining \Cref{thm:soft_attnt} and \Cref{tab:few-nt-pcfg-global}, we have the following informal theorem.


\begin{theorem}[Informal]\label{thm:approx-few-nt-informal}
    Given the PCFG $\gG = (\gN, \gI, \gP, n, p)$ learned on the English corpus, there exist  subsets $\tilde\gI\subset\gI,\tilde\gP\subset\gP$ with $|\tilde\gI| = 20, |\tilde\gP| = 45$, and an attention model with soft relative attention modules (\ref{eq:soft_attention}) with embeddings of size $275+40L$, $2L+1$ layers, and $20$ attention heads in each layer, that can simulate the Inside-Outside algorithm restricted to $\tilde\gI,\tilde\gP$
    on all sentences of length at most $L$ generated from $\gG$. The restriction introduces a $9.29\%$ increase in average 1-mask perplexity and $8.71\%$ drop in the parsing performance of the Labeled-Recall algorithm. 
    %Parsing using this approximated Inside-Outside algorithm gives $68.41\%$ corpus F1 $71.91\%$ sentence F1 on \dataset{PTB} dataset.
    %There exists an attention model
    %By computing the inside and outside probabilities for only the top-$20$ non-terminals ($|\tilde\gN| = 20$), we can reduce the size of the model in \Cref{thm:soft_attnt} to $20$ attention heads in each layer, $540+40L$ embedding dimension, and $2L$ layers to approximately execute the Inside-Outside algorithm on PCFG learned on English corpus. Parsing using this approximated Inside-Outside algorithm gives us $68.41\%$ corpus F1 $71.91\%$ sentence F1 on \dataset{PTB} dataset.
\end{theorem}

\iffalse
\begin{table}[!t]
    \centering
    \footnotesize
    \begin{tabular}{|c|c|c|c|}
    \hline
         Approximation & Corpus F1 & Sent F1 & ppl. \\
         \hline
         \makecell{No approx.} & 75.90 & 78.77 & 50.80 \\
         \hline
         $|\tilde\gI| = 10,|\tilde\gP|=45$ & 57.14 & 60.32 & 59.57 \\
         $|\tilde\gI| = 20,|\tilde\gP|=45$ & 68.41 & 71.91 & 55.16 \\
         $|\tilde\gI| = 40,|\tilde\gP|=45$ & 72.45 & 75.43 & 54.09 \\
         \hline
    \end{tabular}
    \iffalse
    \begin{tabular}{|c|c|c|c|c|}
    \hline
         & Corpus F1 & Sent F1 & TV & ppl. \\
         \hline
         \makecell{No approx.} & 75.90 & 78.77 & 0 & 50.80 \\
         \hline
         $|\tilde\gN| = 10$ & 57.14 & 60.32 & 0.114 & 58.11 \\
         $|\tilde\gN| = 20$ & 68.41 & 71.91 & 0.073 & 55.16 \\
         $|\tilde\gN| = 40$ & 72.45 & 75.43 & 0.050 & 54.09 \\
         \hline
    \end{tabular}
    \fi
    \iffalse
    \begin{tabular}{|c|c|c|}
    \hline
         & Corpus F1 & Sent F1 \\
         \hline
         \makecell{No approx.} & 75.90 & 78.77 \\
         \hline
         $|\tilde\gN| = 10$ & 57.14 & 60.32  \\
         $|\tilde\gN| = 20$ & 68.41 & 71.91  \\
         $|\tilde\gN| = 40$ & 72.45 & 75.43  \\
         \hline
    \end{tabular}
    \fi
    \caption{Experiment results by approximately computing the Inside-Outside algorithm with very few non-terminals. We show the unlabelled F1 scores on \dataset{PTB} development set as well as the 1-masking perplexity. $\tilde\gI$ ($\tilde\gP$) denotes the set of in(pre)-terminals to compute and are selected to be the in(pre)-terminals with top frequency. The PCFG is learned on \dataset{PTB} training dataset. The ppl. column denote the 1-masking perplexity on 200 sentences generated from the learned PCFG. %Besides the parsing F1 results, we also show the TV distance between the exact computation and the approximated computation for 1-masking prediction.
    }
    \label{tab:few-nt-pcfg-global}
\end{table}
\fi

If we plug in the average length $L\approx 25$ for sentences in \dataset{PTB}, we can get a model with $20$ attention heads, $1275$ hidden dimension, and $51$ layers. Compared with the construction in \Cref{thm:soft_attnt}, the size of the model is much closer to reality. The proof of \Cref{thm:approx-few-nt-informal} is shown in \Cref{sec:approx-few-nt}.
%Besides, this approximation doesn't affect the parsing performance much: compared with parsing using the Inside-Outside algorithm that achieves $75.90\%$ corpus F1 and $78.77\%$ sentence F1 on \dataset{PTB} dataset, the approximated computation shows a drop by $8.71\%$. 


\subsection{More discussions on low-rank approximation} \label{appendix:approx-low-rank-informal}
We hypothesize that we can  find linear transformation matrices $\{\mW^{(\ell)}\}_{\ell\le L}$ that can reduce the computations while retaining most of the performance, and our hypothesis is formalized as follow:
%The following informal theorem show the effectiveness of using this approximation, and please refer to \Cref{sec:approx-low-rank} for more details.

\begin{hypothesis}
     For the PCFG $\gG = (\gN, \gI, \gP, n, p)$ learned on the English corpus, there exists transformation matrices $\mW^{(\ell)}\in\R^{k^{(\ell)}\times |\tilde\gI|}$ for every $\ell \le L$, such that approximately simulating the Inside-Outside algorithm with $\{\mW^{(\ell)}\}_{\ell\le L}$ introduces \underline{small} error in the 1-mask perplexity and has \underline{minimal} impact on the parsing performance of the Labeled-Recall algorithm.
\end{hypothesis}



\Cref{tab:learned-transformation-global} verifies our hypothesis, and lead to \Cref{thm:approx-low-rank-informal}. Compared with the parsing results from \Cref{thm:approx-few-nt-informal}, the corpus and sentence F1 scores are nearly the same, and we further reduce the number of attention heads in each layer from $20$ to $15$. If we only use $10$ attention heads to approximately execute the Inside-Outside algorithm, we can still get $61.72\%$ corpus F1 and $65.31\%$ sentence F1 on \dataset{PTB} dataset, which is still much better than the Right-branching baseline. \Cref{thm:approx-low-rank-informal} shows that attention models with a size much closer to the real models (like BERT or RoBERTa) still have enough capacity to parse decently well (>70\% sentence F1 on \dataset{PTB}).

It is also worth noting that approximately executing the Inside-Outside algorithm using the transformation matrices $\{\mW^{(\ell)}\}_{\ell\le L}$ is very different from reducing the size of the PCFG grammar, since we use different matrix $\mW^{(\ell)}$ when computing the probabilities for spans with different length. If we choose to learn the same transformation matrix $\mW$ for all the layers $\ell$, the performance drops.
%Although the number of layers of our constructed models ($50$ to process average length sentence in \dataset{PTB}) are still relatively large compared to the models in reality ($12$ for BERT and RoBERTa), our approximations in \Cref{thm:approx-few-nt-informathm:approx-low-rank-informal} show strong hint that the PCFG learned on English corpus are ``highly compressible''.

\paragraph{More discussions on the transformation matrix $\mW^{(\ell)}$} We can observe that by introducing the transformation matrix $\mW^{(\ell)}$ generalized the first ingredient that only computes a small set of in-terminals $\tilde\gI$ and pre-terminals $\tilde\gP$, and in theory we can directly learn the transformation matrix $\mW^{(\ell)}$ from the original PCFG without reducing the size at first, i.e., $\mW^{(\ell)}\in\R^{k^{(\ell)}\times |\gI|}$. However empirically, if we directly learn $\mW^{(\ell)}$ from all the in-terminals $\gI$ but not from the top-20 frequent in-terminals $\tilde\gI$, the performance drops. Thus, we choose to learn the matrix $\mW^{(\ell)}$ starting from the most frequent in-terminals $\tilde\gI$. One possible explanation is that the learning procedure is also heuristic, and certainly may not learn the best transformation matrix.

Besides, we use the same transformation matrix $\mW^{(\ell)}$ when computing the inside and outside probabilities, and it is also natural to use different transformation matrices when computing the inside and outside probabilities. Recall that we learn the transformation $\mW^{(\ell)}$ by the Eigenvalue decomposition on matrix $\mX^{(\ell)}$, where $\mX^{(\ell)} = \sum_{s} \mX_s^{(\ell)} / \norm{\mX_s^{(\ell)}}_{\text{F}}$ and $\mX_s^{(\ell)} = \sum_{i,j:j-i=\ell} \vmu_s^{i,j}(\vmu_s^{i,j})^\top$. Then, we can also learn two matrices $\mW^{(\ell)}_{\text{inside}}$ and $\mW^{(\ell)}_{\text{outside}}$ through the Eigenvalue decomposition on matrices $\mX^{(\ell)}_{\text{inside}}$ and $\mX^{(\ell)}_{\text{outside}}$ respectively, where
\begin{align*}
    \mX^{(\ell)}_{\text{inside}} =& \sum_{s} \mX_{s,\text{inside}}^{(\ell)} / \norm{\mX_{s,\text{inside}}^{(\ell)}}_{\text{F}}, \\
    \mX_{s,\text{inside}}^{(\ell)} =& \sum_{i,j:j-i=\ell} \bm\alpha_s^{i,j}(\bm\alpha_s^{i,j})^\top, \\
    \mX^{(\ell)}_{\text{outside}} =& \sum_{s} \mX_{s,\text{outside}}^{(\ell)} / \norm{\mX_{s,\text{outside}}^{(\ell)}}_{\text{F}}, \\
    \mX_{s,\text{outside}}^{(\ell)} =& \sum_{i,j:j-i=\ell} \bm\beta_s^{i,j}(\bm\beta_s^{i,j})^\top.
\end{align*}
However empirically, we also find that the performance drops by using different transformation matrices for inside and outside probabilities computation, which may also be attributed to the non-optimality of our method to learn the transformation matrix.



\subsection{Proof for \cref{thm:approx-few-nt-informal}}\label{sec:approx-few-nt}

Note that in both \Cref{thm:hard_attnt} and \Cref{thm:soft_attnt}, in every layer $1 \le \ell \le L-1$, we use one attention head with parameters $\mK_A^{(\ell)}, \mQ_A^{(\ell)}, \mV_A^{(\ell)}$ to compute all the inside probabilities $\alpha(A, i, j)$ for all spans with length $\ell+1$, i.e. $j-i = \ell$. For layer $L+1 \le \ell \le 2L-1$, the model constructed in \Cref{thm:hard_attnt} uses two attention heads to compute the outside probabilities $\beta(A,i,j)$ for a specific non-terminal $A$ for spans with length $2L - \ell$, and the model constructed in \Cref{thm:soft_attnt} uses one attention heads to compute the outside probabilities $\beta(A,i,j)$ for a specific non-terminal $A$ for spans with length $2L - \ell$. Now to show how restricting the computations to certain non-terminals $\tilde\gI\cup\tilde\gP$ can reduce the size of the constructed models in \Cref{thm:hard_attnt,thm:soft_attnt} we classify the inside and outside probabilities into four categories: (1) the inside probabilities for pre-terminals, $\alpha(A,i,i)$ for $A\in\gP$; (2) the inside probabilities for in-terminals, $\alpha(A,i,j)$ for $A\in\gI$; (3) the outside probabilities for in-terminals, $\beta(A,i,j)$ for $A\in\gI$; and (4) the outside probabilities for pre-terminals, $\beta(A,i,i)$ for $A\in \gP$.

\paragraph{Category (1): the inside probabilities for pre-terminals} Recall that in the constructed model in \Cref{thm:hard_attnt,thm:soft_attnt}, the inside probabilities for pre-terminals $\alpha(A,i,i)$ for $A\in\gP$ is directly initialized from the PCFG rules, and thus do not need attention heads to compute. Thus, we can just use $O(|\gP|)$ dimensions to store all the inside probabilities for pre-terminals $\alpha(A,i,i)$ for $A\in\gP$. Although we can also only initialize the inside probabilities only for the pre-terminals $\tilde\gP$, i.e. initialize $\alpha(A,i,i)$ for $A\in\tilde\gP$ and use less embedding dimensions, empirically the performance will drop and thus we initialize all the probabilities $\alpha(A,i,i)$ for $A\in\gP$. Although we should store the probabilities for pre-terminals larger than the set $\tilde\gP$, there is indeed another technique to reduce the embedding dimensions. Note that since in the future computations, we only compute the probabilities for the in-terminals $\tilde\gI$, and not every pre-terminal $A\in\gP$ can be produced by in-terminals $B\in\tilde\gI$. Thus, we only need to store the pre-terminals $\gP_{\tilde\gI}$ that can be produced from $\tilde\gI$. Empirically, for PCFG learned on \dataset{PTB} dataset, $|\gP| = 720$, but if we choose $|\tilde\gI| = 20$, the number of pre-terminals that can be produced from $\tilde\gI$ drops to $|\gP_{\tilde\gI}| = 268 < 270$.
Specifically for the model in \Cref{thm:soft_attnt}, we need $|\gP_{\tilde\gI}|$ coordinates at each position to store these inside probabilities.

\paragraph{Category (2): the inside probabilities for in-terminals} The computation of the inside probabilities for in-terminals, $\alpha(A,i,j)$ for $A\in\gI$ happens from layer $1$ to layer $L-1$ in the constructed model in \Cref{thm:hard_attnt,thm:soft_attnt}. Note that from layer $1$ to layer $L-1$, the model only computes the probabilities for the in-terminals, since a span with a length larger than 1 cannot be labeled by a pre-terminal. Thus, if we only compute the inside probabilities for in-terminals $|\tilde\gI|$, we can reduce the number of attention heads in layer $1$ to layer $L-1$ from $O(|\gI|)$ to $O(|\tilde\gI|)$ since in \Cref{thm:hard_attnt,thm:soft_attnt} we use a constant number of attention heads to compute the probabilities for a single in-terminal. Specifically for the model in \Cref{thm:soft_attnt}, we only need $|\tilde\gI|$ attention heads from layer $1$ to layer $L-1$; besides, we need $(L-1)|\tilde\gI|$ coordinates at each position to store these inside probabilities.

\paragraph{Category (3): the outside probabilities for in-terminals} The computation of the outside probabilities for in-terminals, $\beta(A,i,j)$ for $A\in\gI$ happens from layer $L$ to layer $L-2$ in the constructed model in \Cref{thm:hard_attnt,thm:soft_attnt}. Note that in layer $L$, we only need to initialize the outside probabilities $\beta(A,1,L)$ for $A\in\gI$, thus do not need attention heads for computation (however we need attention heads to move the inside and outside probabilities in this layer, which cost 2 attention heads). Then from layer $L+1$ to layer $L-2$, the model computes the outside probabilities for the in-terminals $\beta(A,i,j)$ for $A\in\tilde\gI$. Thus if we only compute the outside probabilities for in-terminals $|\tilde\gI|$, we can reduce the number of attention heads in layer $1$ to layer $L-1$ from $O(|\gI|)$ to $O(|\tilde\gI|)$. Specifically for the model in \Cref{thm:soft_attnt}, we only need $|\tilde\gI|$ attention heads from layer $L$ to layer $L-2$; besides, we need $(L-1)|\tilde\gI|$ coordinates at each position to store these outside probabilities for in-terminals $\tilde\gI$.

\paragraph{Category (4): the outside probabilities for pre-terminals} The outside probabilities for pre-terminals $\beta(A,i,i)$ for $A\in\gP$ is only computed in the final layer in \Cref{thm:hard_attnt,thm:soft_attnt}. Thus if we choose to compute the probabilities for only $\tilde\gP$, we can reduce the number of attention heads in layer $2L-1$ from $O(|\gI|)$ to $O(|\tilde\gI|)$. Specifically for the model in \Cref{thm:soft_attnt}, we only need $|\tilde\gP|$ attention heads in layer $L-1$; besides, we need $|\tilde\gP|$ coordinates at each position to store these outside probabilities for in-terminals $\tilde\gP$. Also as mentioned in \Cref{sec:approx-overview}, if $|\tilde\gP| < c|\tilde\gI|$ for some constant $c$, we can also simulate the computations in the last layer with $|\tilde\gP|$ heads by $c$ layers with $|\tilde\gI|$ heads. In particular, if we choose $|\tilde\gP| = 45, |\tilde\gI|=20$, we can use 3 layers with $20$ attention heads in each layer to simulate the last layer with $45$ attention heads in the original construction.

\paragraph{Put everything together: proof of \Cref{thm:approx-few-nt-informal}} We choose $|\tilde\gP| = 45, |\tilde\gI|=20$. We can use $20$ attention heads in each layer, and we now count the number of layers and the embedding dimension we need. The number of layers is easy to compute, since we just need to use $3$ layers with $20$ attention heads to simulate the original $1$ layer with $45$ attention heads, thus the total number of layers is $2L-1 + (3-1) = 2L+1$. As for the embedding dimension, we need
\begin{align*}
    d = & |\gP_{\tilde\gI}| + (L-1)|\tilde\gI| + (L-1)|\tilde\gI| + |\tilde\gP| \\
    \le & 270 + (2L-2)|\tilde\gI| + |\tilde\gP| \\
    =& 275 + 2L|\tilde\gI|\\
    =& 275 + 40L.
\end{align*}

%For layer $L+1 \le \ell \le 2L-1$, the model constructed in \Cref{thm:hard_attnt} uses two attention heads to compute the outside probabilities $\beta(A,i,j)$ for a specific non-terminal $A$ for spans with length $2L - \ell$, and the model constructed in \Cref{thm:soft_attnt} uses one attention heads to compute the outside probabilities $\beta(A,i,j)$ for a specific non-terminal $A$ for spans with length $2L - \ell$. Thus, it is clear that if we only compute the inside and outside probabilities for a small set of non-terminals $\tilde\gN$, the dependency on $|\gN|$ reduces to $|\tilde\gN|$ for the number of attention heads in both constructions, i.e., for the construction in \Cref{thm:hard_attnt}, the number of attention heads drops from $4|\gN|$ to $4|\tilde\gN|$, and for the construction in \Cref{thm:soft_attnt}, the number of attention heads drops from $|\gN|$ to $|\tilde\gN|$.

%Besides reducing the number of attention heads, computing only the probabilities for important non-terminals can also reduce the number of embedding dimensions at each layer. Note that in the construction in \Cref{thm:hard_attnt} and \Cref{thm:soft_attnt}, we need to store all the inside and outside probabilities, and thus the embedding size has the term $|\gN| L$, where $L$ is the length of the sentence. The number of embedding dimensions for different models may have different coefficients before $|\gN| L$. Now if we only compute the probabilities for non-terminals $\tilde\gN$, we don't need to store the probabilities

\subsection{Proof for \cref{thm:approx-low-rank-informal}}\label{sec:approx-low-rank}

In this section, we show the details of how to further reduce the number of attention heads using structures across non-terminals, and add more discussion on how we learn the transformation matrices $\{\mW^{(\ell)}\}_{\ell \le L}$

\paragraph{Reducing the number of attention heads} We focus on reducing the number of attention heads to compute the inside and outside probabilities for the in-terminals $\tilde\gI$, since the computation for the outside probabilities for pre-terminals $\tilde\gP$ only happens in the final layer of the constructed model, and thus can use multiple layers to compute as long as $\tilde\gP$ is not too large.

For simplicity, we only show the details of how to reduce the number of attention heads to compute the inside probabilities for in-terminals $\tilde\gI$ in \Cref{thm:soft_attnt}, and the technique can be easily applied to the computation of outside probabilities for in-terminals $\tilde\gI$ in \Cref{thm:soft_attnt}, and the inside and outside probabilities for $\tilde\gI$ in \Cref{thm:hard_attnt}.

Recall from the proof of \Cref{thm:soft_attnt} that we at each layer $\ell$, we use a single attention head $\mK^{(\ell)}_A, \mQ^{(\ell)}_A$ to compute the inside probabilities $\alpha(A,i,j)$ for spans with length $\ell+1$, i.e., $j-i = \ell$. Specifically, for the attention head $\mK^{(\ell)}_A, \mQ^{(\ell)}_A$ at layer $\ell$, we want to compute and store the probability $\alpha(A, i-\ell, i)$ at position $i$. Thus we construct $\mK^{(\ell)}_A, \mQ^{(\ell)}_A$ such that the attention score $a_{i,j}^{A,(\ell)}$ when the position $i$ attends to position $j$ satisfies

{\small
\begin{align*}
     &a_{i,j}^{A,(\ell)} \\
     =& \text{ReLU}(\mK_{A}^{(\ell)} \ve_j^{(\ell-1)} + p_{j-i} - b_{j-i, \ell})^{\top}  \mQ_{A}^{(\ell)} \ve_i^{(\ell-1)} \\
     =& \sum_{B, C \in \gN} \Pr[A \to B C] \cdot \alpha(B, j+1, i) \cdot  \alpha (C, i-\ell, j),
\end{align*}
}

if $i - \ell \le j \le i-1$ and $0$ otherwise. Then, summing over all locations $j$ gives us $\alpha(A, i-\ell, i)$. Also, a key property of $\mK_{A}^{(\ell)}$ is that this key matrix does not depend on the non-terminal $A$, but only depends on $\ell$. Thus, if we have a set of coefficients $\{\omega_A^{(\ell)}\}_{A\in\gI}$, we can compute the linear combination of the inside probability $\sum_{A\in\tilde\gI} \omega_A^{(\ell)}\alpha(A, i-\ell, i)$ using one attention head, since if we choose
\[\mQ^{(\ell)} = \sum_{A\in\tilde\gI} \omega_A^{(\ell)}\mQ_{A}^{(\ell)}, \quad\mK^{(\ell)} = \mK_A^{(\ell)}, \forall A\in\tilde\gI,\]
we have the attention score

{\small
\begin{align*}
     &a_{i,j}^{(\ell)} \\
     = &\text{ReLU}(\mK^{(\ell)} \ve_j^{(\ell-1)} + p_{j-i} - b_{j-i, \ell})^{\top}  \mQ^{(\ell)} \ve_i^{(\ell-1)} \\
     = &\text{ReLU}(\mK^{(\ell)} \ve_j^{(\ell-1)} + p_{j-i} - b_{j-i, \ell})^{\top} \\
     &\cdot \left(\sum_{A\in\tilde\gI} \omega_A^{(\ell)}\mQ_{A}^{(\ell)}\right) \ve_i^{(\ell-1)} \\
     = &\sum_{A\in\tilde\gI} \omega_A^{(\ell)}\\
     &\cdot \text{ReLU}(\mK^{(\ell)} \ve_j^{(\ell-1)} + p_{j-i} - b_{j-i, \ell})^{\top}  \mQ_{A}^{(\ell)} \ve_i^{(\ell-1)} \\
     = &\sum_{A\in\tilde\gI} \omega_A^{(\ell)}\\
     &\cdot \text{ReLU}(\mK_A^{(\ell)} \ve_j^{(\ell-1)} + p_{j-i} - b_{j-i, \ell})^{\top}  \mQ_{A}^{(\ell)} \ve_i^{(\ell-1)} \\
     = &\sum_{A\in\tilde\gI} \omega_A^{(\ell)}\\
     &\cdot\left(\sum_{B, C \in \gN} \Pr[A \to B C] \cdot \alpha(B, j+1, i) \cdot  \alpha (C, i-\ell, j)\right),
\end{align*}
}
if $i - \ell \le j \le i-1$ and $0$ otherwise. Then, summing over all locations $j$ gives us $\sum_{A\in\tilde\gI} \omega_A^{(\ell)}\alpha(A, i-\ell, i)$. Then if we have a transformation matrix $\mW^{(\ell)}\in\R^{k^{(\ell)}\times |\tilde\gI|}$, we can use $k^{(\ell)}$ attention heads to compute $\mW^{(\ell)}\bm\alpha(i-\ell, i)$, where $\bm\alpha(i-\ell, i)\in\R^{|\tilde\gI|}$ is the vector that contains $\alpha(A, i-\ell, i)$ for all $A\in\tilde\gI$. Then after we use $k^{(\ell)}$ attention heads to compute the probabilities $\mW^{(\ell)}\bm\alpha(i-\ell, i)$ and stored them in position $i$'s embeddings, we can then use linear layer on position $i$ to recover the original probabilities by $\bm{\tilde\alpha}(i-\ell, i) = (\mW^{(\ell)})^{\dagger} \mW^{(\ell)}\bm\alpha(i-\ell, i)$, and use $\tilde\alpha(A, i-\ell, i)$ for $A\in\tilde\gI$ for the future computations.

\paragraph{Put everything together: proof of \Cref{thm:approx-low-rank-informal}} We choose $k^{(\ell)} = 15,|\tilde\gP| = 45, |\tilde\gI|=20$. Note that the embedding dimension doesn't change if we apply the approximation technique, and only the number of attention heads reduces from $20$ to $15$. Thus, the embedding dimension is still
\begin{align*}
    d =& |\gP_{\tilde\gI}| + (L-1)|\tilde\gI| + (L-1)|\tilde\gI| + |\tilde\gP| \\
    \le& 270 + (2L-2)|\tilde\gI| + |\tilde\gP|\\
    =& 275 + 2L|\tilde\gI|\\
    =& 275 + 40L.
\end{align*}
Also note that $|\tilde\gP| = 45 = 3\times 15$, and thus we can compute all the outside probabilities for pre-terminals $\tilde\gP$ by $3$ layers where each layer has $15$ attention heads.

\subsection{Experiment details in \Cref{sec:approx-overview}}\label{sec:approx-exp-details}
In this section, we provide the experiment details in \Cref{sec:approx-overview}. We use and modify the code~\citep{Spectral-Parser} to learn the PCFG from the \dataset{PTB} dataset and conduct the experiments with approximated computations. \citet{Spectral-Parser} implements the spectral learning method to learn PCFG~\citep{cohen2012spectral,cohen2014spectral} and is under MIT licence. We follow all the default hyperparameters in \citet{Spectral-Parser}, and we also follow the split of \dataset{PTB}: using \dataset{PTB} section 02-21 as the training set and \dataset{PTB} section 22 as the development set.

%%%%%%%%%%%%%%%%%%%%%%%%%%%%%%%%%%%%%%%%%%%%%%%%%%%