%%%%%%%%%%%%%%%%%%%%%%%%%%%%%%%%%%%%%%%%%%%%%%%%%%%%%%%%%%%%%%%%%%%%%%%%%%%%%%%%
%2345678901234567890123456789012345678901234567890123456789012345678901234567890
%        1         2         3         4         5         6         7         8

\documentclass[letterpaper, 10 pt, conference]{ieeeconf}  % Comment this line out
                                                          % if you need a4paper
%\documentclass[a4paper, 10pt, conference]{ieeeconf}      % Use this line for a4
                                                          % paper

\IEEEoverridecommandlockouts                              % This command is only
                                                          % needed if you want to
                                                          % use the \thanks command
\overrideIEEEmargins
% See the \addtolength command later in the file to balance the column lengths
% on the last page of the document



% The following packages can be found on http:\\www.ctan.org
\usepackage{graphics} % for pdf, bitmapped graphics files
\usepackage{epsfig} % for postscript graphics files
\usepackage{mathptmx} % assumes new font selection scheme installed
\usepackage{times} % assumes new font selection scheme installed
\usepackage{amsmath} % assumes amsmath package installed
\usepackage{amssymb}  % assumes amsmath package installed
\usepackage{color}
%\usepackage[numbers]{natbib}

\title{\LARGE \bf
Assessing the impact of Byzantine attacks on coupled phase oscillators
}
%\title{\LARGE \bf
%Simplest framework for sensor attacks in coupled oscillator networks
%}

%\author{ \parbox{3 in}{\centering Huibert Kwakernaak*
%         \thanks{*Use the $\backslash$thanks command to put information here}\\
%         Faculty of Electrical Engineering, Mathematics and Computer Science\\
%         University of Twente\\
%         7500 AE Enschede, The Netherlands\\
%         {\tt\small h.kwakernaak@autsubmit.com}}
%         \hspace*{ 0.5 in}
%         \parbox{3 in}{ \centering Pradeep Misra**
%         \thanks{**The footnote marks may be inserted manually}\\
%        Department of Electrical Engineering \\
%         Wright State University\\
%         Dayton, OH 45435, USA\\
%         {\tt\small pmisra@cs.wright.edu}}
%}

\author{Melvyn Tyloo% <-this % stops a space
\thanks{This work has been supported by the Laboratory Directed Research and Development program of Los Alamos National Laboratory under project numbers 20220797PRD2 and 20220774ER.}
\thanks{M. Tyloo is  with the Theoretical Division and the Center for Nonlinear Studies (CNLS), Los Alamos National Laboratory, Los Alamos, NM 87545, USA.}
\thanks{
{Email: \tt\small mtyloo@lanl.gov}}
}


\begin{document}



\maketitle
\thispagestyle{empty}
\pagestyle{empty}


%%%%%%%%%%%%%%%%%%%%%%%%%%%%%%%%%%%%%%%%%%%%%%%%%%%%%%%%%%%%%%%%%%%%%%%%%%%%%%%%
\begin{abstract}
For many coupled dynamical systems, the interaction is the outcome of the measurement that each unit has of the others or of physical flows e.g. modern inverter-based power grids, autonomous vehicular platoons or swarms of drones. Synchronization among all the components of these systems is of primal importance to avoid failures. The overall operational state of these systems therefore crucially depends on the correct and reliable functioning of the individual elements as well as the information they transmit through the network. Here we investigate the effect of Byzantine attacks where one unit does not behave as expected, but is controlled by an external attacker. For such attacks, we assess the impact on the global collective behavior of nonlinearly coupled phase oscillators. We relate the synchronization error induced by the input signal to the properties of the attacked node. This allows to anticipate the potential of an attacker and identify which network components to secure.


\end{abstract}


%%%%%%%%%%%%%%%%%%%%%%%%%%%%%%%%%%%%%%%%%%%%%%%%%%%%%%%%%%%%%%%%%%%%%%%%%%%%%%%%
\section{Introduction}

Networked systems where the interaction among the individual units is mediated through the perception each units have of its neighbors find numerous realizations both in natural and engineered systems~\cite{Pik03,strogatz2014nonlineardynamics}.  For example, renewable energy sources such as solar panels or wind turbines typically read the voltage frequency on the grid and inject their power at this particular frequency~\cite{pattabiraman2018comparison}, or autonomous cars and drones being part of a larger vehicular platoon, where each component measures the distance between itself and its neighbors~\cite{8778746}. From this interaction, collective macroscopic phenomena emerge such as synchronization or consensus. Such coherent operational states are made possible thanks to the interplay between both the units' internal parameters and the coupling within them. Therefore, alterations or attacks of either the coupling within the interacting units or the units themselves might dramatically impact the whole state of the system and, thus, its good and desired functioning.
Robustness of network-coupled dynamical system can be investigated from various points of view. One may consider the effect of input perturbation signals or noise on the overall stability of the system~\cite{Dev12,bamieh2013price,Pag17,doi:10.1063/5.0129123,Hin18}, or assess the vulnerabilities to line and node failure or removal~\cite{DELABAYS2022270}, and how such attacks may lead to cascades where, eventually, the operational state of the system is disrupted~\cite{crucitti2004model,PhysRevE.95.012315}. Another way to impact the stability of the system is by spoofing some components. Indeed, the collective state characterizing the stable state of coupled dynamical system essentially depends on the relative values of the degrees of freedom of each and every interacting elements~\cite{strogatz2014nonlineardynamics,Str04}. Therefore, if one or a subset of these degrees of freedom can be controlled or spoofed, then one may globally impact the collective behavior of the system and, potentially, disrupt it~\cite{8320800}. In this manuscript, we investigate the latter scenario where an attacker has the ability to control a single degree of freedom, i.e. a single unit, of a system of networked oscillators, which is often referred to as a \textit{Byzantine attack}~\cite{lamport1982byzantine}. Such situation could take place, for example, in the formation of vehicular platoons or swarming robots where some attacker controls some sensors or the shared information. This problem has been previously investigated for pulse-coupled oscillators where Ref.~\cite{8320800} showed that synchronization is still achievable despite Byzantine type attacks. The resilience to such malicious attacks of discrete consensus algorithm including mobile agent has been explored in Ref.~\cite{8013830}. Detection and mitigation of false data injection in controlled networked systems in general, and microgrids has recently attracted an increasing interest~\cite{10.1145/2185505.2185515,8260848,8894512}. Less has been done on the effect of Byzantine attack on the collective behavior of phase coupled oscillators, which are directly related to integrator dynamics describing e.g. vehicular platoon formation~\cite{li2015overview}.

Here we investigate the global impact on synchronization induced by an attacker, who has the full control of a single oscillator. Based on response theory, we develop a framework to investigate the reaction of the system to such attacks and use it to calculate a performance metric that assess how synchronization is affected by an external perturbation. We derive a general expression for arbitrary input signals the attacker would choose and consider then more precisely the case of random signals for which the properties of the attacked node can be directly connected to the disruptive potential of the attack.

The manuscript is organized as follows. Section~\ref{sec1} gives the mathematical notations. Section~\ref{Sec1} introduces the networked oscillator model and defines Byzantine attacks we consider. Section~\ref{sec2} gives the response theory of the system to external attacks and defines the performance metric to assess the potential of the attack. Section~\ref{sec_rand} applies the theory to the specific case of a random input signal. Section~\ref{Num} illustrates numerically the theory developed in the previous sections.
Conclusions are given in Section~\ref{conc}.

\begin{figure}
    \centering
    \includegraphics[scale=0.5]{figure0.pdf}
    \caption{Left: original network before the attack; right: network following the attack where the degree of freedom of oscillator 2 is replaced by an input signal $f(t)$\,. The resulting system is an oscillatory network made of nodes 1, 3, 4, 5 where oscillators 1 and 3 are influenced by the input signal from the attacker, i.e. $1,3\in \mathcal{N}(2)$\,.}
    \label{fig1}
\end{figure}

\section{Mathematical notations}\label{sec1}
We consider a network of $N$ nodes where the $k$-th node is controlled by an attacker. The set of nodes connected to node $k$ is denoted by $\mathcal{N}(k)$\,. 
In the following, we write column vectors ${\bf v}\in \mathbb{R}^N$ as bold lower case letters. Matrices ${\bf M}\in \mathbb{R}^{(N\times N)}$ are denoted with bold upper case letters. The Kroenecker symbol equivalent to the $j$-th component of the unit vector is written as $\hat{e}^i_j=\delta_{ij}$\,. The Dirac delta function is denoted $\delta(t)$\,.

\section{Byzantine attacks on Networked Oscillators}\label{Sec1}
\subsection{Kuramoto oscillators}
As a paradigmatic model for investigating synchronization phenomena, we consider $N$ coupled Kuramoto oscillators whose time-evolution is governed by the following set of coupled differential equations~\cite{Kur75},
\begin{eqnarray}\label{eq1} 
\dot{\theta}_i = \omega_i - \sum_j b_{ij}\sin(\theta_i-\theta_j) \,,i=1,...N\,,
\end{eqnarray}
where the degrees of freedom are defined over a compact set i.e. $\theta_i\in[0,2\pi)$\,, and $\omega_i$ is the natural frequency of the $i$-th oscillator. The coupling between the oscillators is given by the elements of the adjacency matrix $b_{ij}$\,. If the natural frequencies are not too broadly distributed and the network is connected, with the coupling that is strong enough, the oscillators eventually reach a synchronized state (also called \textit{phase-locked state}) such that $\dot{\theta}_i(t\rightarrow\infty)=\Omega=\sum_i N^{-1}\omega_i$ $\forall i$\,. Without loss of generality, one can move to the rotating frame and thus have $\Omega =0$\,. Depending on the coupling topology and the natural frequencies, many stable synchronous states may exist, each of them with their own basin of attraction. Perturbations applied to the system can induce transitions between the different synchronous states or even simply disrupt synchrony~\cite{Dev12,Tyl18c,Hin18}. 
\subsection{Byzantine attack}
The type of attack we consider in this manuscript is inspired by the Byzantine Generals problem~\cite{lamport1982byzantine}, where an individual being part of a larger networked game, decides to pass some corrupted information to their neighbors, instead of behaving reliably as expected. In our framework of coupled oscillators, we model this as an attacker having full control of an oscillator $k$\,, which is formulated as,
\begin{eqnarray}\label{eq2}
    \theta_k(t) = f(t) \,, t>0\,
\end{eqnarray}
where $f(t)$ is the input signal that is chosen by the attacker, which might be e.g. constrained by some budget. The controlled input signal Eq.~(\ref{eq2}) directly affect the neighbors of node $k$ whose time-evolution is then given by,
\begin{eqnarray}\label{eq3} 
\dot{\theta}_{i} = \omega_i - \sum_{j\neq k} b_{ij}\sin(\theta_i-\theta_j) - b_{ik}\sin[\theta_i-f(t)]  \,,
\end{eqnarray}
where $i\in \mathcal{N}(k)$ with $\mathcal{N}(k)$ the set of the neighbors of node $k$\,. An important question is then, how robust is the synchronous state to such attacks. One must notice that Eq.~(\ref{eq3}) effectively changes the coupling network topology by removing node $k$\, as the latter is not influenced by its neighbors anymore. The situation is illustrated on Fig.~\ref{fig1}. In the following, we investigate this question using the synchronization error and the linear response of the system.

\section{Assessment of the system's response}\label{sec2}
\subsection{Linear response}
Before the attack (i.e $t<0$), we assume that the system sits at a synchronous state $\{\theta_i^{(0)}\}$ that satisfies the algebraic equations,
\begin{eqnarray}\label{eq40} 
0 = \omega_i - \sum_j b_{ij}\sin(\theta_i^{(0)}-\theta_j^{(0)}) \,,i=1,...N\,.
\end{eqnarray}
It is clear that many stable fixed point might exist or even none at all, because of the nonlinear coupling function. Here we assume that at least one stable synchronous solution to Eq.~(\ref{eq40}) exists before the attack. Further assuming that the perturbations applied to the system are small enough, the behavior of the oscillators following the attack is well approximated by their linear response. Defining the deviation from the synchronous state $\phi_i(t) = \theta_i(t)-\theta_i^{(0)}$ for $i\neq k$ and $\phi_k(t) = f(t)$ and taking into account the attack Eq.~(\ref{eq2}) one has the dynamics,
\begin{align}\label{eq4} 
\begin{split}
\dot{\phi}_i =
\begin{cases}
- \sum_j \tilde{\mathbb{L}}_{ij}\,\phi_j  \quad \text{for }i\neq k \,,i\notin \mathcal{N}(k)\,,\\
-\sum_{j\neq k} \tilde{\mathbb{L}}_{ij}\,\phi_j - b_{ik}\cos(\theta_i^{(0)}-\theta_k^{(0)})[\phi_i-f(t)]\, \\ \quad \text{for }i\neq k \,,i\in \mathcal{N}(k)\,\\
\dot{f}(t) \quad \text{for } i=k\,,
\end{cases}
\end{split}
\end{align}
where we defined the weighted Laplacian matrix $\tilde{\mathbb{L}}\in \mathbb{R}^{(N-1)\times(N-1)}$ of the network where node $k$ has been removed (see Fig.~\ref{fig1}),
\begin{eqnarray}\label{eq6}
   \tilde{\mathbb{L}}_{ij} = 
   \begin{cases}
           -b_{ij}\cos(\theta_i^{(0)}-\theta_j^{(0)}) \,, i\neq j\,,\\
            \sum_{k}b_{ik}\cos(\theta_i^{(0)}-\theta_k^{(0)}) \,, i=j\,.
   \end{cases}
\end{eqnarray}
This matrix depends on both the coupling network and the synchronous state of the system. Note that, in the rest of this manuscript we denote the effective weights on the edges of the new network $\tilde{b}_{ij}=b_{ij}\cos({\theta_i}^{(0)}-{\theta_j}^{(0)})$\,.
Rather interestingly, the dynamics given by Eq.~(\ref{eq4}) corresponds to the Taylor model that describes the interaction between networked agents where some of them are stubborn i.e. have an additional bias~\cite{doi:10.1177/001872676802100202, Bau20}. 
Restricting Eq.~(\ref{eq4}) to the uncontrolled oscillators by defining the new variable $\bf \varphi$ which is simply $\bf \phi$ where the $k$-th component has been removed, one can write the dynamics in a matrix form as,
\begin{eqnarray}\label{eq7}
    \dot{{\bf \varphi}} = - (\tilde{\mathbb{L}} + {\bf K})\,{\bf \varphi} + {\bf{g}}(t)  \,,
\end{eqnarray}
where we defined the diagonal matrix with non-zero elements given by $K_{ii} = \tilde{b}_{ik}$ for $i\in \mathcal{N}(k)$\,, and the vector $\bf g$ with non-zero components ${g}_i(t) = \tilde{b}_{ik}\,f(t)$ for $i\in\mathcal{N}(k)$\,. Equation~\ref{eq7} is a linear system which can be solved by expanding the deviations over the eigenbasis of the symmetric matrix $(\tilde{\mathbb{L}} + {\bf K})$\,, i.e. $\varphi_i(t)=\sum_{\alpha}c_\alpha(t)\,u_{\alpha,i}$\,. The eigenvalues of $(\tilde{\mathbb{L}} + {\bf K})$ satisfy $\lambda_k>0$ and set the time-scales of the response dynamics. The corresponding eigenvectors are denoted ${\bf u}_{\alpha}$\,. The time-evolution of the system is then given by,
\begin{eqnarray}\label{eq8}
    \varphi_i(t) = \sum_\alpha e^{-\lambda_\alpha t}\int_0^t e^{\lambda_\alpha t } \sum_j g_j(t)u_{\alpha,j}\, {\rm d}t'\,u_{\alpha,i}\,.
\end{eqnarray}
The latter equation gives the response of the system following an attack at node $k$ where its degree of freedom has been replaced by $f(t)$\,. In the following section, we introduce a metric to assess the impact of the attack on the overall synchronization of the system, which can be calculated using Eq.~(\ref{eq8}).

\subsection{Synchronization error}
In order to quantify the global impact of the attack on the synchronization of the system, we consider the so-called \textit{synchronization error} which is a common performance metric for networked systems subjected to perturbations~\cite{bamieh2013price,Pag17,Tyl18a,Bau20}. We define the synchronization error at time $t$ as,
\begin{eqnarray}
    \mathcal{P}(t) = \sum_{i<j}\tilde{b}_{ij}[\varphi_i(t) - \varphi_j(t)]^2 \,{\rm d}t\,.
\end{eqnarray}
It measures the mismatch in the angle deviations between neighboring oscillators, due to the attack. Large values of $\mathcal{P}$ mean that the deviations between connected oscillators were asynchronous while small values correspond to a synchronous response of the system where most of the deviations followed the same direction. Moreover, it can be conveniently written as,
\begin{eqnarray}\label{eq9}
    \mathcal{P}(t) &=& \sum_{i,j}\varphi_i(t)\,\tilde{\mathbb{L}}_{ij}\,\varphi_{j}(t)\\ \nonumber
    &=& \sum_{\alpha}\lambda_\alpha\,c_\alpha^2(t) + \sum_{j\in\mathcal{N}(k)}\varphi_j^2(t)\,,
\end{eqnarray}
where in the second row we used the expansion of the deviations over the eigenbasis and the orthogonality between the eigenvectors ${\bf u}_\alpha$'s. The synchronization error can be obtained for any input signal from the attacker by plugging $f(t)$ into Eq.~(\ref{eq8}) and then substituting it in Eq.~(\ref{eq9}). One can therefore investigate any desired signal restricted or not to some budget, which models the limited capacities of the attacker. Here, to get a first intuitive picture of the potential of the attack, we consider a random input signal.
\begin{figure}
    \centering
    \includegraphics[scale=0.5]{figure00.pdf}
    \caption{Left: original network before the attack; right: network following the attack where the degree of freedom of oscillator k, which is connected to all other nodes, is replaced by an input signal $f(t)$\,.}
    \label{fig2}
\end{figure}

\section{Random input signal}\label{sec_rand}
The simplest input the attacker could choose is that of a random signal around the initial value of the degree of freedom at node $k$\,, i.e. $\langle f(t) \rangle = \theta_k^{(0)}$\,, where $\langle . \rangle$ denotes the statistical average which also corresponds to the time average i.e. $\langle f(t) \rangle = \lim_{T\rightarrow\infty}T^{-1}\int_0^T f(t'){\rm d}t'$\,. If the latter is not correlated in time one has
\begin{eqnarray}\label{eq10}
    \langle f(t) f(t') \rangle = \tau_0 \, \delta({t-t'})\,,
\end{eqnarray}
where $\tau_0$ is a time constant introduced to keep the input signal $f(t)$ without units and which also gives the standard deviation of the signal. Using this two-point correlator, one can then calculate the average of the synchronization error. 
Plugging Eq.~(\ref{eq10}) into Eq.~(\ref{eq9}) yields,
\begin{align}
\begin{split}
    \langle \mathcal{P} \rangle = \lim_{t\rightarrow\infty}\left[\sum_\alpha \lambda_\alpha e^{-2\lambda_\alpha t}\right.&\int \hspace{-0.2cm}\int_0^t e^{\lambda_\alpha (t_1+t_2)}  \hspace{-0.2cm}\sum_{i,j\in\mathcal{N}(k)}\tilde{b}_{ik}\tilde{b}_{jk}u_{\alpha,i}u_{\alpha,j} \\
      &\left.\times \langle f(t_1)f(t_2) \rangle{\rm d}t_1{\rm d}t_2 \right.\\
      + \sum_{\alpha,\beta}  e^{-(\lambda_\alpha + \lambda_\beta) t}\int \hspace{-0.2cm}\int_0^t &\left. e^{\lambda_\alpha t_1}e^{\lambda_\beta t_2} \hspace{-0.2cm} \sum_{l,m\in\mathcal{N}(k)}\tilde{b}_{lk}\tilde{b}_{mk}u_{\alpha,l}u_{\beta,m} \right.\\
   \times \langle f(t_1)f(t_2) \rangle &{\rm d}t_1{\rm d}t_2 \left.\sum_{j\in\mathcal{N}(k)}u_{\alpha,j}u_{\beta,j}  \right]\,,
     \end{split}
\end{align}
which, after some algebra gives
\begin{align}\label{eq11}
\begin{split}
     \langle \mathcal{P} \rangle = \frac{\tau_0}{2}\sum_{j\in\mathcal{N}(k)} \tilde{b}^2_{jk} + \tau_0\sum_{\alpha,\beta}\sum_{i,j,l\in\mathcal{N}(k)}\tilde{b}_{ik}\tilde{b}_{jk}\frac{u_{\alpha,i}u_{\beta,j}u_{\alpha,l}u_{\beta,l}}{\lambda_\alpha+\lambda_\beta}\,.
     \end{split}
\end{align}
\begin{figure}[h]
    \centering
    \includegraphics[scale=0.6]{Sc.pdf}
    \caption{Scaling of $T_1$ and $T_2$ with $\gamma$ [see Eq.~(\ref{numeq})] for an arbitrary Watt-Strogatz network obtained with rewiring probability $p=0.05$\,, 4 initial nearest-neighbors, $N=100$\,, edge weights set to 1, and depicted in Fig.~\ref{fig4}. For $\gamma\lesssim 0.4$ the synchronization error is rather dominated by $T_2$\,, while for $\gamma\gtrsim 0.4$\, it is dominated by $T_1$\,.}
    \label{fig3}
\end{figure}
\begin{figure*}[h!]
    \centering
    \includegraphics[scale=0.38]{WSS.pdf}
    \caption{Each row gives $T_1$ (left panels), $T_2$ (middle panels) and the synchronization error (right panels) for a Watts-Strogatz network with rewiring probability $p=0.05$\,, 4 initial nearest-neighbors and $N=100$\,. The weights on the edges are set to (a) $\gamma=5$\,, (b) $\gamma=0.4$\,, (c) $\gamma=0.05$\,. The synchronization error is given by the sum of $T_1$ and $T_2$\,.}
    \label{fig4}
\end{figure*}
While the first term in the synchronization error Eq.~(\ref{eq11}) is simply the sum of the squared edge weights of the attacked node $k$ in the initial network divided by 2, the second term is more intricate with the eigenmodes and the weights between the attacked node and its neighbors. The latter might becomes large when the nodes connected to $k$ are sitting on the slowest eigenmodes for which $\lambda_\alpha$ is small. The synchronization error for such random input signal is therefore a trade-off between a local property of the system given by the first term in Eq.~(\ref{eq11}) and a more global one given by the second term. Below we illustrate numerically the theoretical predictions.

At this stage, it is instructive to briefly discuss the case where the attacked node $k$ is connected to all the nodes in the network. This would happen in a system where all units are connected to an oscillator possibly modelling a centralized control unit. This is illustrated in Fig.~\ref{fig2}. Then, $\mathcal{N}(k)$ includes all the network nodes, and the synchronization error reduces to,
\begin{eqnarray}\label{eq110}
     \langle \mathcal{P} \rangle &=& \frac{\tau_0}{2}\sum_{j\in\mathcal{N}(k)} \tilde{b}^2_{jk} + \frac{\tau_0}{2}\sum_{i,j\in\mathcal{N}(k)}\tilde{b}_{ik}\tilde{b}_{jk}\sum_{\alpha}\frac{u_{\alpha,i}u_{\alpha,j}}{\lambda_\alpha}\\ \nonumber
     &=& \frac{\tau_0}{2}\sum_{j\in\mathcal{N}(k)} \tilde{b}_{jk}^2 + \frac{\tau_0}{2}\sum_{i,j\in\mathcal{N}(k)}\tilde{b}_{ik}\tilde{b}_{jk}[(\tilde{\mathbb{L}} + {\bf{K}})^{-1}]_{ij}\,,
\end{eqnarray}
where the intricate second term in Eq.~(\ref{eq11}) is now expressed in term of the inverse of $(\tilde{\mathbb{L}} + {\bf{K}})$\,. Further assuming that the weights between node $k$ and the rest of the network are the same i.e. $\tilde{b}_{ik}=b$ $\forall i$\,, one has
\begin{eqnarray}\label{eq12}
     \langle \mathcal{P} \rangle &=& \tau_0\,{b}^2\,(N-1)\,.
\end{eqnarray}
Quite intuitively, attacking the oscillator that is connected to all the other ones in the network results in a poor synchronization error that scales linearly with the system size. 

\section{Numerical simulations}\label{Num}
For simplicity, we consider in this section only systems where all oscillators have the same natural frequency i.e. $\omega_i=0$ $\forall i$\,, which means that one of the synchronous states satisfies $\theta_i^{(0)}=C$ $\forall i$\,, where $C\in\mathbb{R}$ is some constant. 

In the previous section, we derived a closed form expression for the synchronization error Eq.~(\ref{eq11}). Here we illustrate numerically this results and show how both terms in Eq.~(\ref{eq11}) are important to assess the potential of an attack. Let us denote the two terms as,
\begin{eqnarray}\label{numeq}
    T_1 &=& \frac{\tau_0}{2}\sum_{j\in\mathcal{N}(k)} \tilde{b}^2_{jk} \,,\\ 
    T_2 &=& \tau_0\sum_{\alpha,\beta}\sum_{i,j,l\in\mathcal{N}(k)}\tilde{b}_{ik}\tilde{b}_{jk}\frac{u_{\alpha,i}u_{\beta,j}u_{\alpha,l}u_{\beta,l}}{\lambda_\alpha+\lambda_\beta}\,.
\end{eqnarray}
Note that $T_1$ is computationally easier to obtain compared to $T_2$\,.
To clarify when Eq.~(\ref{eq11}) is rather dominated by $T_1$ or $T_2$\,, we consider an arbitrary network with adjacency matrix denoted $\bf A$ and a copy of this network where all the weights have been rescaled by a positive factor $\gamma$\,, for which the adjacency matrix therefore reads $\gamma\,{\bf A}$\,. By inspecting $T_1$ and $T_2$\,, one concludes that their scaling with $\gamma$ are $T_1\propto \gamma^2$\,, $T_2\propto \gamma$\,. Indeed, $\gamma$ appears then in the edge weights as well as the eigenvalues $\lambda_\alpha$'s. Therefore, by tuning $\gamma$\,, one should be able to generate networks where either $T_1$ or $T_2$ is dominating Eq.~(\ref{eq11}). This is shown in Fig.~\ref{fig3} where we tuned $\gamma$ for an arbitrary network.
We now pick three values of $\gamma$\,: one below $0.4$ where $T_2$ is larger $T_1$, one above $0.4$ where $T_1$ is larger than $T_2$\,, and one close to $0.4$ so that both terms contribute similarly to the synchronization error. The results are shown in Fig.~~\ref{fig4}. First one observes that $T_1$\,, which is a local quantity is clearly different from $T_2$ which is a more global quantity as it depends on all the eigenmodes of $(\tilde{\mathbb{L}} + {\bf K})$\,. This is true for all three rows (a), (b), (c).  Second, as expected, the synchronization error is dominated by $T_1$ in row (a); it is dominated by $T_2$ in row (c); and in row (b), the final answer is contrasted between the two terms. The detailed answer given by Eq.~(\ref{eq11}) is therefore necessary to reliably identify the most vulnerable oscillators. This result is particularly interesting as it might appear as counter intuitive. Indeed, one could think that attacking the node with the highest degree is obviously the most efficient strategy. However, while this might be true in specific cases where the degree is widely distributed so that a small fraction of nodes have degrees much larger than all the others (i.e. $T_1\gg T_2$ for such nodes with very high degree), it is not true in general. For some networks, the degree does not vary much among the different nodes e.g. the network in Fig.~\ref{fig4}. Moreover, using the configuration model, one could build many different instances of networks where all the nodes have the same degree yet their connectivities are different~\cite{New18book}. One should therefore use Eq.~(\ref{eq11}) to correctly identify vulnerabilities to a Byzantine attack injecting a random signal.

Finally, the colormaps depicted in the right panels of Fig.~\ref{fig4} can be used to identify the oscillator that is most vulnerable (largest $\langle \mathcal{P} \rangle$) or the most robust (smallest $\langle \mathcal{P} \rangle$) in the network. Indeed, Fig.~\ref{fig5} shows the difference in angle deviations between connected oscillators in the effective network, i.e. $\tilde{\mathbb{L}}\,\varphi$ as a function of time, when the most vulnerable (top panels) and the most robust (bottom panels) nodes are attacked. Trajectories are obtained by numerically time-evolving Eq.~(\ref{eq1}). One clearly sees that when the most vulnerable node is attacked, the difference of angle deviations vary over a wider scale compared to the attack on the most robust one.

\begin{figure}
    \centering
    \includegraphics[scale=0.56]{Fig5.pdf}
    \caption{Difference in angle deviations between connected oscillators in the effective network where the attacked node has been removed, i.e. $\tilde{\mathbb{L}}\,\varphi$ as a function of time. At time $t=0$\,, the attacker replaces the degree of freedom of the most vulnerable (top panel) and the most robust node by a random input signal, as discussed in Section~\ref{sec_rand}. Both nodes are identified using the right panel of Fig.~\ref{fig4}(b). The network used is the same as in Fig.~\ref{fig4}(b).}
    \label{fig5}
\end{figure}

\section{Conclusions}\label{conc}
We defined a framework for the analysis of Byzantine attacks on phase coupled oscillators. We derived closed form expression for the linear response of the system to a general input signal from the attacker. The latter depends on the eigenvectors and eigenvalues of a matrix that is the sum of the Laplacian matrix of the network where the attacked node has been removed, and a diagonal matrix encoding the influence on the neighboring nodes. Interestingly, such systems has been considered in the field of opinion dynamics. Based on this framework, we then assessed the global impact of such attacks on the synchronous state using the synchronization error. One should note that other performance metrics can be computed within this framework. As a simple application of our theory, we considered attacks where the degree of freedom of a single oscillator is replaced by a random input signal uncorrelated in time. In this case, the synchronization error can be expressed in terms of the sum of the squared weights of the attacked node plus a term that depends on the localization of its neighbors on the eigenmodes of the matrix $(\tilde{\mathbb{L}} + {\bf{K}})$\,. This give an efficient method to identify the most vulnerable and most robust oscillators in the system subjected to this type of attack. 

Based on previous results on the robustness of networked oscillators to noise inputs in the natural frequencies, it is expected that the synchronization error will depend on network properties that are more global than $T_1$ if the time-correlation of the attacker's signal is within or slower than the time-scales of the system~\cite{Tyl18a}.
Direct extensions of these results should consider more complicated input signals such as correlated noise, deterministic functions, or copies of some other degrees of freedom in the system, and also include budget constraints for the attacker. One should notice that our theory can easily account for multiple input signals, corresponding to attacks where multiple oscillators are controlled at the same time.
%Using a toy example, we showed that, simple spoofing strategies as replacing one oscillator's degrees of freedom by one of its neighbors can effectively globally affect the synchronous state of networked oscillators. For linear coupling functions, one can bring the system to another synchronization frequency while for nonlinear coupling function, one can even disrupt the synchronous state of the system. In the attacker strategy presented in this manuscript, we considered discontinuous spoofing as suddenly, the degrees of freedom of one oscillator is replaced by a copy from another oscillator with the addition of a constant shift $\delta\omega$\,. In order to render such attacks harder to identify, one may instead apply a continuous spoofing by slowly altering the degrees of freedom of the controlled oscillator, i.e. having $\delta\omega(t)$\,. This will be the subject of further investigation as well as developing methods to identify such attacks.
%Easily extended to the case of multiple controlled oscillators.

\section*{Acknowledgements}
We thank A. Lokhov and M. Vuffray for useful discussions.
% Generated by IEEEtran.bst, version: 1.14 (2015/08/26)
% Generated by IEEEtran.bst, version: 1.14 (2015/08/26)
\begin{thebibliography}{10}
\providecommand{\url}[1]{#1}
\csname url@samestyle\endcsname
\providecommand{\newblock}{\relax}
\providecommand{\bibinfo}[2]{#2}
\providecommand{\BIBentrySTDinterwordspacing}{\spaceskip=0pt\relax}
\providecommand{\BIBentryALTinterwordstretchfactor}{4}
\providecommand{\BIBentryALTinterwordspacing}{\spaceskip=\fontdimen2\font plus
\BIBentryALTinterwordstretchfactor\fontdimen3\font minus
  \fontdimen4\font\relax}
\providecommand{\BIBforeignlanguage}[2]{{%
\expandafter\ifx\csname l@#1\endcsname\relax
\typeout{** WARNING: IEEEtran.bst: No hyphenation pattern has been}%
\typeout{** loaded for the language `#1'. Using the pattern for}%
\typeout{** the default language instead.}%
\else
\language=\csname l@#1\endcsname
\fi
#2}}
\providecommand{\BIBdecl}{\relax}
\BIBdecl

\bibitem{Pik03}
A.~Pikovsky, M.~Rosenblum, and J.~Kurths, \emph{Synchronization: a universal
  concept in nonlinear sciences}.\hskip 1em plus 0.5em minus 0.4em\relax
  Cambridge university press, 2003.

\bibitem{strogatz2014nonlineardynamics}
S.~H. Strogatz, ``Nonlinear dynamics and chaos: With applications to physics,
  biology, chemistry, and engineering.'' \emph{Westview Press, Boulder}, 2014.

\bibitem{pattabiraman2018comparison}
D.~Pattabiraman, R.~Lasseter, and T.~Jahns, ``Comparison of grid following and
  grid forming control for a high inverter penetration power system,'' in
  \emph{2018 IEEE Power \& Energy Society General Meeting (PESGM)}.\hskip 1em
  plus 0.5em minus 0.4em\relax IEEE, 2018, pp. 1--5.

\bibitem{8778746}
T.~Zeng, O.~Semiari, W.~Saad, and M.~Bennis, ``Joint communication and control
  for wireless autonomous vehicular platoon systems,'' \emph{IEEE Transactions
  on Communications}, vol.~67, no.~11, pp. 7907--7922, 2019.

\bibitem{Dev12}
L.~{DeVille}, ``Transitions amongst synchronous solutions in the stochastic
  kuramoto model,'' \emph{Nonlinearity}, vol.~25, no.~5, pp. 1473--1494, 2012.

\bibitem{bamieh2013price}
B.~Bamieh and D.~F. Gayme, ``The price of synchrony: Resistive losses due to
  phase synchronization in power networks,'' in \emph{2013 American Control
  Conference}.\hskip 1em plus 0.5em minus 0.4em\relax IEEE, 2013, pp.
  5815--5820.

\bibitem{Pag17}
F.~Paganini and E.~Mallada, ``Global performance metrics for synchronization of
  heterogeneously rated power systems: The role of machine models and
  inertia,'' \emph{\textit{55th Annual Allerton Conference on Communication,
  Control, and Computing}}, pp. 324--331, 2017.

\bibitem{doi:10.1063/5.0129123}
M.~Tyloo, ``Faster network disruption from layered oscillatory dynamics,''
  \emph{Chaos}, vol.~32,
  no.~12, p. 121102, 2022.

\bibitem{Hin18}
J.~Hindes and I.~B. Schwartz, ``Rare slips in fluctuating synchronized
  oscillator networks,'' \emph{Chaos}, vol.~28, no.~7, p. 071106, 2018.

\bibitem{DELABAYS2022270}
R.~Delabays, L.~Pagnier, and M.~Tyloo, ``Locating fast-varying line
  disturbances with the frequency mismatch,'' \emph{IFAC-PapersOnLine},
  vol.~55, no.~13, pp. 270--275, 2022, 9th IFAC Conference on Networked Systems
  NECSYS 2022.

\bibitem{crucitti2004model}
P.~Crucitti, V.~Latora, and M.~Marchiori, ``Model for cascading failures in
  complex networks,'' \emph{Physical Review E}, vol.~69, no.~4, p. 045104,
  2004.

\bibitem{PhysRevE.95.012315}
T.~Yuan, K.~Aihara, and G.~Tanaka, ``Robustness and fragility in coupled
  oscillator networks under targeted attacks,'' \emph{Physical Review E}, vol.~95,
  p. 012315, 2017.

\bibitem{Str04}
S.~H. Strogatz, \emph{Sync: The Emerging Science of Spontaneous Order}, ser.
  Penguin Press Science Series.\hskip 1em plus 0.5em minus 0.4em\relax Penguin
  Adult, 2004.

\bibitem{8320800}
Z.~Wang and Y.~Wang, ``Attack-resilient pulse-coupled synchronization,''
  \emph{IEEE Transactions on Control of Network Systems}, vol.~6, no.~1, pp.
  338--351, 2019.

\bibitem{lamport1982byzantine}
L.~Lamport, R.~Shostak, and M.~Pease, ``The byzantine generals problem,''
  \emph{ACM Transactions on Programming Languages and Systems}, vol.~4, no.~3,
  pp. 382--401, 1982.

\bibitem{8013830}
C.~Zhao, J.~He, and J.~Chen, ``Resilient consensus with mobile detectors
  against malicious attacks,'' \emph{IEEE Transactions on Signal and
  Information Processing over Networks}, vol.~4, no.~1, pp. 60--69, 2018.

\bibitem{10.1145/2185505.2185515}
A.~Teixeira, D.~P\'{e}rez, H.~Sandberg, and K.~H. Johansson, ``Attack models
  and scenarios for networked control systems,'' in \emph{Proceedings of the
  1st International Conference on High Confidence Networked Systems}, ser.
  HiCoNS '12.\hskip 1em plus 0.5em minus 0.4em\relax New York, NY, USA:
  Association for Computing Machinery, 2012, p. 55–64.

\bibitem{8260848}
H.~Zhang, W.~Meng, J.~Qi, X.~Wang, and W.~X. Zheng, ``Distributed load sharing
  under false data injection attack in an inverter-based microgrid,''
  \emph{IEEE Transactions on Industrial Electronics}, vol.~66, no.~2, pp.
  1543--1551, 2019.

\bibitem{8894512}
A.~Sargolzaei, K.~Yazdani, A.~Abbaspour, C.~D. Crane~III, and W.~E. Dixon,
  ``Detection and mitigation of false data injection attacks in networked
  control systems,'' \emph{IEEE Transactions on Industrial Informatics},
  vol.~16, no.~6, pp. 4281--4292, 2020.

\bibitem{li2015overview}
S.~E. Li, Y.~Zheng, K.~Li, and J.~Wang, ``An overview of vehicular platoon
  control under the four-component framework,'' in \emph{2015 IEEE Intelligent
  Vehicles Symposium (IV)}.\hskip 1em plus 0.5em minus 0.4em\relax IEEE, 2015,
  pp. 286--291.

\bibitem{Kur75}
Y.~Kuramoto, ``Self-entrainment of a population of coupled non-linear
  oscillators,'' in \emph{Lecture Notes in Physics {\bf 39}, International
  Symposium on Mathematical Problems in Theoretical Physics}, H.~Araki,
  Ed.\hskip 1em plus 0.5em minus 0.4em\relax Berlin: Springer, 1975.

\bibitem{Tyl18c}
M.~Tyloo, R.~Delabays, and P.~Jacquod, ``Noise-induced desynchronization and
  stochastic escape from equilibrium in complex networks,'' \emph{Physical Review
  E}, vol.~99, p. 062213, 2019.

\bibitem{doi:10.1177/001872676802100202}
M.~Taylor, ``Towards a mathematical theory of influence and attitude change,''
  \emph{Human Relations}, vol.~21, no.~2, pp. 121--139, 1968.

\bibitem{Bau20}
F.~Baumann, I.~M. Sokolov, and M.~Tyloo, ``A laplacian approach to stubborn
  agents and their role in opinion formation on influence networks,''
  \emph{Physica A: Statistical Mechanics and its Applications}, vol. 557, p.
  124869, 2020.

\bibitem{Tyl18a}
M.~Tyloo, T.~Coletta, and P.~Jacquod, ``Robustness of {Synchrony} in {Complex}
  {Networks} and {Generalized} {Kirchhoff} {Indices},'' \emph{Physical Review
  Letters}, vol. 120, no.~8, p. 084101, 2018.

\bibitem{New18book}
M.~Newman, \emph{Networks}.\hskip 1em plus 0.5em minus 0.4em\relax Oxford
  University Press, 2018.

\end{thebibliography}




\end{document}

