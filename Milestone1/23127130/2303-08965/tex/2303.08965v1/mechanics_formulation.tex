\section{Mechanics of Pivoting}\label{sec:mechanics}
% What We need to say:
% 1. Stability margin Concept. Given fx, fy (u) and length, parameters, we can have safety margin so that we can discuss stability
% exact formulation of safety margin in formulation
% during manipulation, you may not be able to get the precise parameters. Then your manipulation task can fail.
% however, in practice, we don't need to have so much accurate parameters. of course, we can do it better with better estimated parameters but we have uncertainty during the manipulation since inherently friction already compensate for uncertainty as long as it satisfies static equilibrium of force moment and friction cones. For example accordingly friction magnitude can change to some extent. This would not happen to other example. This is due to the property of friction forces. our prime goal in this paper is to somehow utilize this frictional feature so that we can consider the most robust nominal trajectory

\begin{figure}
    \centering    \includegraphics[width=0.45\textwidth]{Figures/pivot22.png} 
    \caption{A schematic showing the free-body diagram of a rigid body during pivoting manipulation. Point $P$ is the contact point with a manipulator.}
    \label{fig:mechanics_pivoting_eq}
\end{figure}


In this section, we explain quasi-static stability of two-point pivoting in a plane. %We use this to present our proposed concept of \textit{frictional stability} which explains how friction can compensate for the inaccuracy of physical parameters during pivoting.
% 
Before explaining the details, we present our assumptions in this work. The following assumptions are used in the model for the pivoting manipulation task presented in this paper:
\begin{enumerate}
\item The object is rigid.
\item We consider static equilibrium of the object.
\item The external contact surfaces are perfectly flat. 
\item The dimensions and pose of the object is perfectly known.
% \item The frictional parameters for the contact between the object and manipulator are perfectly known.
\item The object makes point contacts.
\end{enumerate}
The above assumptions are common in manipulation problems. 
% Regarding the fifth assumption, we should mention that other parameters such as coefficient of friction can be also uncertain. However, uncertainty in coefficient of friction would lead to stochastic complementarity system, which is left as a future work~\cite{yuki2021chance}. 

\subsection{Mechanics of Pivoting with External Contacts}
%Before describing frictional stability, we describe our problem setting. 
We consider pivoting where the object with a length of $l$ and width of $w$ maintains slipping contact with two external surfaces (see \fig{fig:mechanics_pivoting_eq}). A free body diagram showing the static equilibrium of the object is shown in \fig{fig:mechanics_pivoting_eq}. The object experiences four forces corresponding to two friction forces $f_A, f_B$ from external contact points $A$ and $B$, one control input $f_P$ from manipulator at point $P$, and gravity, $mg$ at point $C$ where $m$ is mass of a body. We denote $f_{ni}, f_{ti}$ as a normal force and friction force at point $\forall i, i=\{A, B\}$, respectively, defined in ${\{F_W\}}$. $f_{nP}, f_{tP}$ are normal and friction force at point $P$ defined in ${\{F_B\}}$. Note that we define the $[f_x, f_y]^\top = \mathbf{R} [f_{nP}, f_{tP}]^\top$ where $\mathbf{R}$ is a rotation matrix from ${\{F_B\}}$ to ${\{F_W\}}$. We denote $x, y$ position at point in ${\{F_W\}}$ $\forall i, i=\{A, B, P\}$ as $i_x, i_y$, respectively. We denote $y$ position of point $P$ in ${\{F_B\}}$ as $p_y \in [-\frac{w}{2}, \frac{w}{2}]$.
% 
We define the angle of body with respect to $x$-axis as $\theta$. The coefficient of friction at point $\forall i, i=\{A, B, P\}$ are $\mu_A, \mu_B, \mu_P$, respectively. 
In the later sections, we present trajectory optimization formulation where we consider friction force variables $f_{ni}, f_{ti}$, contact point variables $i_{x}, i_{y}$ $\forall i, i=\{A, B, P\}$, $\theta$, and $p_y$ at each time-step $k$ denoted as $f_{k, ni}, f_{k, ti}, i_{k, x}, i_{k, y}, \theta_k, p_{y, k}$. In this section, we remove $k$ to represent variables for simplicity. 

% By setting $B_x = B_y = 0$, the static equilibrium of force in $x$ and $y$ directions the static equilibrium of and moment along point $B$ can be given by:
The static equilibrium conditions for the object can be represented by the following equations (note we consider the moment at point $B$ by setting $B_x = B_y = 0$):
\begin{subequations}
\begin{flalign}
 f_{nA} + f_{tB} + f_{xP}  =0,\label{forceeq1}\\
f_{tA} + f_{nB} + mg + f_{yP}   = 0,  \label{forceeq2}\\
A_xf_{tA} - A_yf_{nA} + C_xmg + P_xf_{y} - P_y f_x = 0 \label{moment_eq1}
% -\frac{l_\text{com}}{2}c_{\theta - \gamma}f_{tA} + \frac{l_\text{com}}{2}s_{\theta - \gamma}f_{nA} -\frac{l_\text{com}}{2}c_{\theta + \gamma}f_{nb} +\frac{l_\text{com}}{2}s_{\theta + \gamma}f_{tb}
% (A_x-B_x)f_{tA} - (A_y-B_y)f_{nA} + (C_x-B_x)mg + (P_x-B_x)f_{y} - (P_y - B_y) f_x = 0
\end{flalign}
\label{force_eq}
\end{subequations}
We consider Coulomb friction law which results in friction cone constraints as follows:
\begin{equation}
 |f_{tA}|  \leq \mu_A f_{nA}, |f_{tB}|  \leq \mu_B f_{nB}, \quad f_{nA}, f_{nB} \geq 0
% f_{tA} + f_{nB} + mg + f_{yP}   = 0,  \label{forceeq2}\\
% A_xf_{tA} - A_yf_{nA} + C_xmg + P_xf_{y} - P_y f_x = 0
% -\frac{l_\text{com}}{2}c_{\theta - \gamma}f_{tA} + \frac{l_\text{com}}{2}s_{\theta - \gamma}f_{nA} -\frac{l_\text{com}}{2}c_{\theta + \gamma}f_{nb} +\frac{l_\text{com}}{2}s_{\theta + \gamma}f_{tb}
% (A_x-B_x)f_{tA} - (A_y-B_y)f_{nA} + (C_x-B_x)mg + (P_x-B_x)f_{y} - (P_y - B_y) f_x = 0
\label{general_FC}
\end{equation}
To describe sticking-slipping complementarity constraints, we have the following complementarity constraints at point $i = \{A, B\}$:
\begin{subequations}
\begin{flalign}
 0 \leq   \dot{p}_{i+} \perp \mu_i f_{ni}-f_{ti} \geq 0  \\
 0 \leq   \dot{p}_{i-} \perp \mu_i  f_{ni}+f_{ti} \geq 0 
 \end{flalign}
 \label{slippingAB}
\end{subequations}
where the slipping velocity at point $i$ follows $\dot{p}_i=\dot{p}_{i+}-\dot{p}_{i-}$.
$\dot{p}_{i+}, \dot{p}_{i-}$ represent the slipping velocity along positive and negative directions for each axis, respectively.
The notation $0 \leq a \perp b \geq 0$ means the complementarity constraints $a \geq 0, b \geq 0, a b=0$.
Since we consider slipping contact during pivoting, we have "equality" constraints in friction cone constraints at points $A, B$:
\begin{equation}
 f_{tA}  =\mu_A f_{nA}, f_{tB}  =-\mu_B f_{nB}
% f_{tA} + f_{nB} + mg + f_{yP}   = 0,  \label{forceeq2}\\
% A_xf_{tA} - A_yf_{nA} + C_xmg + P_xf_{y} - P_y f_x = 0
% -\frac{l_\text{com}}{2}c_{\theta - \gamma}f_{tA} + \frac{l_\text{com}}{2}s_{\theta - \gamma}f_{nA} -\frac{l_\text{com}}{2}c_{\theta + \gamma}f_{nb} +\frac{l_\text{com}}{2}s_{\theta + \gamma}f_{tb}
% (A_x-B_x)f_{tA} - (A_y-B_y)f_{nA} + (C_x-B_x)mg + (P_x-B_x)f_{y} - (P_y - B_y) f_x = 0
\label{slipping_friction_cone}
\end{equation}
To realize stable pivoting, actively controlling position of point $P$ is important. Thus, we consider the following complementarity constraints that represent the relation between the slipping velocity $\dot{p}_y$  at point $P$ and friction cone constraint at point $P$:
\begin{subequations}
\begin{flalign}
 0 \leq   \dot{p}_{y+} \perp \mu_p f_{nP}-f_{tP} \geq 0  \\
 0 \leq   \dot{p}_{y-} \perp \mu_p  f_{nP}+f_{tP} \geq 0 
 \end{flalign}
 \label{slippingP}
\end{subequations}
where $\dot{p}_y=\dot{p}_{y+}-\dot{p}_{y-}$. 

