%%%%%%%%%%%%%%%%%%%%%%%%%%%%%%%%%%%%%%%%%%%%%%%%%%%%%%%%%%%%%%%%%%%%%%%%%%%%%%%%
%2345678901234567890123456789012345678901234567890123456789012345678901234567890
%        1         2         3         4         5         6         7         8
%\documentclass[sageh,times]{TRR}
\documentclass[letterpaper, 10 pt, journal,twoside]{IEEEtran} % Comment this line out if you need a4paper

%\documentclass[a4paper, 10pt, conference]{ieeeconf}      % Use this line for a4 paper

\IEEEoverridecommandlockouts   % Comment this line out if you need a4paper

%\documentclass[a4paper, 10pt, conference]{ieeeconf}      % Use this line for a4 paper

% \IEEEoverridecommandlockouts                              % This command is only needed if 
%                                                           % you want to use the \thanks command

% \overrideIEEEmargins                                      % Needed to meet printer requirements.

%In case you encounter the following error:
%Error 1010 The PDF file may be corrupt (unable to open PDF file) OR
%Error 1000 An error occurred while parsing a contents stream. Unable to analyze the PDF file.
%This is a known problem with pdfLaTeX conversion filter. The file cannot be opened with acrobat reader
%Please use one of the alternatives below to circumvent this error by uncommenting one or the other
%\pdfobjcompresslevel=0
%\pdfminorversion=4

% See the \addtolength command later in the file to balance the column lengths
% on the last page of the document
\usepackage[colorlinks,bookmarksopen,bookmarksnumbered,citecolor=red,urlcolor=red]{hyperref}

\usepackage{bm}
% The following packages can be found on http:\\www.ctan.org
\usepackage{amsmath} % assumes amsmath package installed
\usepackage{amssymb}  % assumes amsmath package installed
\usepackage{dsfont}
\usepackage{graphicx}

\usepackage{psfrag,graphicx,epsfig}
\usepackage{epstopdf}
\usepackage{xspace}
\usepackage{subcaption}
% \usepackage{caption}
\usepackage{float}
\usepackage{placeins}
\usepackage{multirow}
\usepackage{pgf,tikz}
\usepackage{nowidow}
\usepackage{lineno}
\usepackage{xcolor}
\usepackage{color}
%\usepackage{arydshln} 
% \usepackage{subcaption}
\DeclareMathOperator*{\argmin}{arg\,min}
\DeclareMathOperator*{\argmax}{arg\,max}
\captionsetup{font=footnotesize}
\usepackage{siunitx}
\usepackage{color}
%\usepackage{flushend}
\usepackage{lineno}
\usepackage{algorithmic}% Needed to meet printer requirements.
\allowdisplaybreaks

\allowdisplaybreaks

\newcommand{\R}{\mathbb{R}}
\newcommand{\Z}{\mathbb{Z}}
\newcommand{\subblk}[3]{{#1}_{#2[#3]}}
\newcommand{\setX}{{\cal X}}
\newcommand{\setU}{{\cal U}}
\newcommand{\setY}{{\cal Y}}
\newcommand{\bftheta}{\boldsymbol{\theta}}
\newcommand{\dist}{\text{dist}}

\newcommand{\pyrobocop}{\textsc{PyRoboCOP}}
\newcommand{\adolc}{\textsf{ADOL-C}}
\newcommand{\ipopt}{\textsf{IPOPT}}

\newcommand{\mysub}[2]{[#1]_{#2}}
\newcommand{\setcompl}{{\cal L}}
\newcommand{\fdiff}{f_d}
\newcommand{\falg}{f_a}
\newcommand{\lb}[1]{\underline{#1}}
\newcommand{\ub}[1]{\overline{#1}}
\newcommand{\nfe}{N_e}
\newcommand{\ncoll}{N_{c}}
\newcommand{\setnfe}{{\cal N}_e}
\newcommand{\setncoll}{{\cal N}_c}
\newcommand{\fig}[1]{Fig.~\ref{#1}}
\newcommand{\tab}[1]{Table~\ref{#1}}
\newcommand{\eq}[1]{(\ref{#1})}

\newcommand{\setmodes}{{\cal M}}

\newcommand{\devesh}[1]{\textcolor{magenta}{DKJ: #1}}

%\newtheorem{theorem}{Theorem}
%\newtheorem*{remark}{Remark}

% \setcounter{secnumdepth}{3}
% \begin{document}
% \runninghead{\pyrobocop}

\title{
Robust Pivoting Manipulation using Contact Implicit Bilevel Optimization
}


% \author{Yuki Shirai$^{\dagger}$, Devesh K. Jha$^{\ddagger}$, Arvind U. Raghunathan$^{\ddagger}$ and Diego Romeres$^{\ddagger}$% <-this % stops a space
% % <-this % stops a space
% % \thanks{$^1$All authors are with Mitsubishi Electric Research Laboratories (MERL), Cambridge, MA 02139. Email--{ \tt\small \{raghunathan,jha, romeres\}@merl.com}}
% }

\author{Yuki Shirai$^{\dagger}$, Devesh K. Jha$^{\ddagger}$, and Arvind U. Raghunathan$^{\ddagger}$ %   
\thanks{$^{\dagger}$ Yuki Shirai is with the Department of Mechanical and Aerospace Engineering, University of California, Los Angeles, CA, USA 90095 {\tt\small yukishirai4869@g.ucla.edu}}%
\thanks{$^{\ddagger}$Devesh K. Jha and Arvind U. Raghunathan are with Mitsubishi Electric Research Laboratories (MERL), Cambridge, MA, USA 02139 {\tt\small \{jha,raghunathan\}@merl.com}}}%


\begin{document}



\maketitle

\begin{abstract}
Generalizable manipulation requires that robots be able to interact with novel objects and environment. This requirement makes manipulation extremely challenging as a robot has to reason about complex frictional interactions with uncertainty in physical properties of the object and the environment. In this paper, we study robust optimization for planning of pivoting manipulation in the presence of uncertainties. We present insights about how friction can be exploited to compensate for inaccuracies in the estimates of the physical properties during manipulation. Under certain assumptions, we derive analytical expressions for stability margin provided by friction during pivoting manipulation. This margin is then used in a Contact Implicit Bilevel Optimization (CIBO) framework to optimize a trajectory that maximizes this stability margin to provide robustness against uncertainty in several physical parameters of the object. We present analysis of the stability margin with respect to several parameters involved in the underlying bilevel optimization problem.  We demonstrate our proposed method using a 6 DoF manipulator for manipulating several different objects.

\end{abstract}

\maketitle
% \thispagestyle{plain}
% \pagestyle{plain}


%%%%%%%%%%%%%%%%%%%%%%%%%%%%%%%%%%%%%%%%%%%%%%%%%%%%%%%%%%%%%%%%%%%%%%%%%%%%%%%%

%bilevel trajectory optimization algorithm to design a controller that maximizes this stability margin to provide robustness against uncertainty in physical properties of the object.

\section{Introduction}
\label{sec:introduction}
% \begin{itemize}
%     % Diffusion of FL
%     \item {\st{Diffusion of FL}}
%     % Security threats to FL
%     \item {\st{Security threats to FL with particular focus on model poisoning}}
%     % Limitations of existing countermeasures
%     \item {\st{Current countermeasures (e.g., KRUM) and their limitations}}
%     % Proposed method and its advantages
%     \item {\st{Intuitive description of the proposed method and its difference (i.e., advantages) w.r.t. state of the art}}
%     % Main contributions
%     \item {\st{Summary of the main contributions of this work}}
%     % Paper's structure and organization
%     \item {\st{Paper's structure and organization}}
% \end{itemize}

% Diffusion of FL
Recently, {\em federated learning} (FL) has emerged as the leading paradigm for training distributed, large-scale, and privacy-preserving machine learning (ML) systems~\cite{mcmahan2017googleai,mcmahan2017aistats}. 
The core idea of FL is to allow multiple edge clients to collaboratively train a shared, global model without disclosing their local private training data.
%Specifically, an FL system consists of a central server and many edge clients; 
A typical FL round involves the following steps: {\em(i)} the server randomly picks some clients and sends them the current, global model; {\em(ii)} each selected client locally trains its model with its own private data; then, it sends the resulting local model to the server;\footnote{Whenever we refer to global/local model, we mean global/local model {\em parameters}.} {\em(iii)} the server updates the global model by computing an \emph{aggregation function}, usually the average (FedAvg), on the local models received from clients.
% \begin{enumerate}
%     \item[{\em(i)}] the server sends the current, global model to the clients and appoints some of them for training;
%     \item[{\em(ii)}] each selected client locally trains its copy of the global model with its own private data; then, it sends the resulting local model back to the server;\footnote{Whenever we refer to global/local model, we mean global/local model {\em parameters}.}
%     \item[{\em(iii)}] the server updates the global model by computing an \emph{aggregation function} on the local models received from clients (by default, the average, also referred to as FedAvg~\cite{mcmahan2017aistats}).
% \end{enumerate}
This process goes on until the global model converges. %(e.g., after a certain number of rounds or other similar stopping criteria).
%\\
% The advantages of FL over the traditional, centralized learning paradigm are undoubtedly clear in terms of flexibility/scalability (clients can join/disconnect from the FL network dynamically), network communications (only model weights\footnote{We will use \textit{parameters} and \textit{weights} interchangeably.} are exchanged between clients and server), and privacy (each client's private training data is kept local at the client's end and not uploaded to the server).
\\
% Security threats to FL
%However, the growing adoption of FL also raises security concerns~\cite{costa2022covert}, particularly about its confidentiality, integrity, and availability.
Although its advantages over standard ML, FL also raises security concerns~\cite{costa2022covert}. %, particularly about its confidentiality, integrity, and availability~\cite{costa2022covert}.
% OLD, LONG VERSION
% Indeed, some work deals with privacy leakage that may expose the local data of some clients~\cite{melis2019sp}. 
% A large body of work, instead, investigates attacks that usually aim to detriment the predictive accuracy of the learned global model. For instance, \emph{data poisoning} attacks achieve this goal by letting an adversary pollute the training set of some corrupt FL clients with maliciously crafted examples~\cite{jagielski2018sp}.
% Similarly, in \emph{model poisoning} the attacker attempts to tweak the global model weights~\cite{bhagoji2019pmlr} by directly perturbing the local model's weights of some infected FL clients before these are sent to the central server for aggregation, usually via so-called Byzantine attacks. 
% It turns out that Byzantine model poisoning attacks severely impact standard FedAvg; therefore, more robust aggregation functions must be designed to make FL systems secure.
Here, we focus on \emph{untargeted model poisoning} attacks~\cite{bhagoji2019pmlr}, where an adversary attempts to tweak the global model weights %\footnote{We will use the terms \textit{parameters} and \textit{weights} interchangeably.} 
by directly perturbing the local model's parameters of some infected clients before these are sent to the central server for aggregation.
In doing so, the adversary aims to jeopardize the global model \textit{indiscriminately} at inference time.
Such model poisoning attacks severely impact standard FedAvg; therefore, more robust aggregation functions must be designed to secure FL systems.
\\
% In this paper, we focus on designing a novel robust aggregation scheme at the server's end to contrast the effect of Byzantine model poisoning attacks.
%
% Current countermeasures and their limitations
%Several countermeasures have been proposed in the literature to combat model poisoning attacks on FL systems.
% Some methods use simple statistics more robust than plain average to smooth the impact of malicious updates (e.g., Trimmed Mean and FedMedian~\cite{yin2018icml}). 
% Other defenses implement outlier detection techniques to discard malicious updates from the aggregation performed at the server's end. Those are either based on heuristics (e.g., Krum/Multi-Krum~\cite{blanchard2017nips} and Bulyan~\cite{mhamdi2018pmlr}) or data-driven approaches (e.g., K-means clustering~\cite{shen2016acm} or DnC via spectral analysis~\cite{shejwalkar2021ndss}). 
% Finally, some strategies rely on a centralized ``source of trust'' to spot potential malicious updates (e.g., FLTrust~\cite{cao2020fltrust}).
% Several countermeasures have been proposed in the literature to combat model poisoning attacks on FL systems, i.e., to discard possible malicious local updates from the aggregation performed at the server's end. 
% These techniques range from simple statistics more robust than plain average (e.g., Trimmed Mean and FedMedian~\cite{yin2018icml}) to outlier detection heuristics (e.g., Krum/Multi-Krum~\cite{blanchard2017nips} and Bulyan~\cite{mhamdi2018pmlr}) or data-driven approaches (e.g., spectral analysis via K-means clustering~\cite{shen2016acm} or spectral analysis), or methods based on ``source of trust'' (e.g., FLTrust~\cite{cao2020fltrust}).
% OLD, LONG VERSION
%Several countermeasures have been proposed in the literature to combat Byzantine model poisoning attacks on FL systems.
% Descriptive statistics
% For example, Trimmed Mean and FedMedian aggregate local model updates using more robust statistics than standard average~\cite{yin2018icml}.
%
% % Heuristics for outlier detection
% Many existing Byzantine-resilient strategies implement some outlier detection heuristics to discard the model updates sent by potentially malicious clients from the input of the aggregation function.
% One of the most popular heuristics is Krum~\cite{blanchard2017nips}.
% This strategy tries to mitigate the impact of Byzantine attacks by selecting as a global model the local model with the smallest sum of Euclidean distances to {\em all} the other local models.
% Although powerful, Krum requires the server to know (or, at least, estimate) the number of malicious FL clients upfront, which is generally impossible in a realistic attack scenario. %
% Moreover, Krum may become ineffective for complex, high-dimensional model parameter spaces due to the curse of dimensionality.
% Bulyan~\cite{mhamdi2018pmlr} tries to overcome this issue by combining Krum with a variant of Trimmed Mean.
% % Data-driven outlier detection
% Other strategies use data-driven outlier detection techniques -- e.g., via K-means clustering~\cite{shen2016acm} -- to spot potential malicious local model updates. 
% %For instance, Shen et al. propose to cluster local model updates with K-means and thus identify outliers.
%
% % Other techniques
% As far as the server is concerned, any local model received can be from a potential malicious client. 
% FLTrust~\cite{cao2020fltrust} assumes the server acts as a client, i.e., trains a local model on an additional {\em trustworthy} dataset at the server's end and compares it against all the local models from other clients. 
% This way, the server can rely on some ``source of trust'' when discarding potentially malicious clients.
%\\
% Limitations of existing Byzantine-resilient strategies
Unfortunately, existing defense mechanisms either rely on simple heuristics (e.g., Trimmed Mean and FedMedian by~\cite{yin2018icml}) or need strong and unrealistic assumptions to work effectively (e.g., foreknowledge or estimation of the number of malicious clients in the FL system, as for Krum/Multi-Krum~\cite{blanchard2017nips} and Bulyan~\cite{mhamdi2018pmlr}, which, however, cannot exceed a fixed threshold).
Furthermore, outlier detection methods using K-means clustering~\cite{shen2016acm} or spectral analysis like DnC~\cite{shejwalkar2021ndss} do not directly consider the temporal evolution of local model updates received.
Finally, strategies like FLTrust~\cite{cao2020fltrust} require the server to collect its own dataset and act as a proper client, thereby altering the standard FL protocol.
\\
% OLD, LONG VERSION
% Overall, existing Byzantine-resilient strategies are either simple heuristics (e.g., FedMedian) or, if they are more complex, they rely on strong and unrealistic assumptions to work effectively (e.g., knowing the number of malicious clients in the FL system in advance, as for Krum and alike).
% Furthermore, data-driven outlier detection methods do not consider the temporary evolution of local model updates received (e.g., K-means clustering). 
% Finally, strategies like FLTrust requires the server to collect its own dataset and act as a proper client, thereby altering the standard FL protocol.
%
% Description of the proposed method
This work introduces a novel pre-aggregation \textit{filter} robust to untargeted model poisoning attacks. Notably, this filter $(i)$ operates without requiring prior knowledge or constraints on the number of malicious clients and $(ii)$ inherently integrates temporal dependencies. 
The FL server can employ this filter as a preprocessing step before applying \textit{any} aggregation function, be it standard like FedAvg or robust like Krum or Bulyan.
Specifically, we formulate the problem of identifying corrupted updates as a multidimensional (i.e., matrix-valued) time series anomaly detection task. 
The key idea is that legitimate local updates, resulting from well-calibrated iterative procedures like stochastic gradient descent (SGD) with an appropriate learning rate, show \textit{higher predictability} compared to malicious updates. This hypothesis stems from the fact that the sequence of gradients (thus, model parameters) observed during legitimate training exhibit regular patterns, as validated in Section~\ref{subsec:intuition}. %until convergence. 
%This regularity may be more pronounced for smooth convex loss functions, but it can still be captured within an appropriate time window, even for more complex and convoluted loss surfaces. 
%We provide evidence of this claim in Appendix~B, where we show that the average mutual information (i.e., ``predictability''), calculated over pairs of legitimate model updates sent at different FL rounds, is significantly higher than the corresponding computation for a malicious client.
\\
Inspired by the matrix autoregressive (MAR) framework for multidimensional time series forecasting~\cite{chen2021je}, we propose the FLANDERS ({\em \textbf{F}ederated \textbf{L}earning meets \textbf{AN}omaly \textbf{DE}tection for a \textbf{R}obust and \textbf{S}ecure}) filter.
The main advantages of FLANDERS over existing strategies like FLDetector~\cite{zhao2020multivariate} are its resilience to large-scale attacks, where $50\%$ or more FL participants are hostile, and the capability of working under realistic non-iid scenarios.
We attribute such a capability to two key factors: $(i)$ FLANDERS works without knowing a priori the ratio of corrupted clients, and $(ii)$ it embodies temporal dependencies between intra- and inter-client updates, quickly recognizing local model drifts caused by evil players. Below, we summarize our main contributions:

\begin{itemize}
\item[{\em(i)}]
We provide empirical evidence that the sequence of models sent by legitimate clients is more predictable than those of malicious participants performing untargeted model poisoning attacks.
\\
\item[{\em(ii)}] 
We introduce FLANDERS, the first pre-aggregation filter for FL robust to untargeted model poisoning based on multidimensional time series anomaly detection.
\\
\item[{\em(iii)}] 
We integrate FLANDERS into Flower,\footnote{\scriptsize{\url{https://flower.dev/}}} a popular FL simulation framework for reproducibility.
\\
\item[{\em(iv)}] 
We show that FLANDERS improves the robustness of the existing aggregation methods under multiple settings: different datasets, client's data distribution (non-iid), models, and attack scenarios.
\\
\item[{\em(v)}] 
We publicly release all the implementation code of FLANDERS along with our experiments.\footnote{\scriptsize{\url{https://anonymous.4open.science/r/flanders_exp-7EEB}}}
\end{itemize}

% Paper's structure and organization
The remainder of the paper is structured as follows. %some related work and the current state-of-the-art solutions to security issues that FL entails. 
Section~\ref{sec:background} covers background and preliminaries. 
In Section~\ref{sec:related}, we discuss related work.
Section~\ref{sec:problem} and Section~\ref{sec:method} describe the problem formulation and the method proposed. % to tackle it. 
Section~\ref{sec:experiments} gathers experimental results. %, and Section~\ref{sec:limitations} discusses some limitations of this work.
Finally, we conclude in Section~\ref{sec:conclusion}.
 %discusses the limitations of this work and draws future research directions.
%reports conclusions and draws perspectives for future research directions.

%%%%%%% OLD %%%%%%%
%to overcome the resilience of Byzantine failures in distributed Stochastic Gradient Descent computations. 
% The strength of Krum is its time complexity, which is linear in the gradient dimension. 
% However, the robustness of the approach is guaranteed for gradient-based learning applications only when the majority of the clients are not compromised. 
% Besides, the aggregation mechanism of Krum, as well as that of similar methods, is robust from a coarse-grained perspective and does not provide solutions to errors and perturbations that may occur at inference time.
%A related approach to~\cite{blanchard2017nips} is the work of Su et al.~\cite{su2016dc}. Here, the authors propose an iterated approximate agreement to tackle a multi-layer scenario attacked by Byzantine agents. 
%However, the method works efficiently on the sole discrete context and it is inapplicable to continuous state environments.
%\gabri{Maybe, we should just talk about the main limitations of existing countermeasures without digging into their details (or, we can just mention Krum as this is the most popular one). I will move the description of all these methods to the Related Work section.}
\section{Related work}
% There is extensive recent work on speeding up analytical queries due to the need for consistent execution times in the face of the explosive growth in the volume of available data.
% In this section, we divide existing work into two categories: maintaining data freshness in MVs (\Cref{sec:server_side}) and optimizations for minimizing ad-hoc query latency (\Cref{sec:client_side}).

% \subsection{Maintaining Data Freshness in MVs}
% \label{sec:server_side}
% There exists a variety of data warehousing applications aimed at supporting low-latency analytical queries on fresh data.
% In particular, these applications require efficiency in the propagation of newly ingested data into downstream MVs.
 
\mypara{Efficient MV Refresh}
Incremental view maintenance (IVM) aims to update MVs to reflect newly ingested data, taking advantage of already computed results to perform the update in a manner more efficient than computing from scratch (full refresh)
~\cite{ahmad2012dbtoaster,mcsherry2013differential,armbrust2013generalized,zeng2016iolap, palpanas2002incremental, griffin1995incremental, agiwal2021napa, braun2015analytics}. 
There is an abundance of work in IVM, including incremental updates on duplicate values~\cite{griffin1995incremental}, non-distributive aggregate functions~\cite{palpanas2002incremental}, higher-order views~\cite{ahmad2012dbtoaster}, and sliding windows~\cite{braun2015analytics}. 
More recent works also investigate the scalability aspect of IVM, proposing scale-independent updates~\cite{armbrust2013generalized} and sampled views~\cite{zeng2016iolap}. Since \system is applicable to arbitrary SQL statements, \system is orthogonal to and is fully compatible with existing IVM techniques.

\mypara{MV Refresh Scheduling}
There exist works on scheduling the refresh of a MV set focusing on resolving cyclic dependencies~\cite{folkert2005optimizing}, minimizing weighted average staleness~\cite{golab2009scheduling}, and prioritizing MVs with the highest speedups on predicted future queries~\cite{ahmed2020automated}.
\system's scheduling to speed up the end-to-end refresh of the MV set is not addressed in existing works.

\mypara{DAG Workflow Scheduling}
The execution of workloads consisting of individual jobs with acyclic dependencies is a well-studied topic~\cite{apacheoozie,sparkdag,marchal2018parallel,bathie2020revisiting,baruah2022ilp}; many of these techniques can be applied to MV refresh runs studied in this paper.
Existing workflow scheduling systems such as Apache Oozie~\cite{apacheoozie}, Apache Airflow~\cite{airflow}, and Spark DAG scheduler~\cite{sparkdag} automate the execution of user-defined workflows following a topological order.
There are recent works aimed at finding more optimal execution orders in terms of peak memory usage~\cite{marchal2018parallel, bathie2020revisiting} and execution time on parallel platforms~\cite{baruah2022ilp}.
While \system is designed for use with MV refresh runs/workloads, our technique on joint scheduling and optimization can be reasonably applied to general workloads as a possible future direction.

% \paragraph{Incremental MV indexing}
% Update-optimized indices such as the log-structured merge-trees (LSM)~\cite{o1996log} are used for indexing MVs due to frequent updates induced by data ingestion~\cite{gupta2016mesa,agiwal2021napa}.
% \system is orthogonal to indexing: \system is capable of efficiently performing MV refresh runs regardless of whether the individual MVs are indexed or not.

% \subsection{Ad-hoc Query Latency Reduction}
% \label{sec:client_side}

% The minimization of ad-hoc analytical query response times is a well-studied topic due to latency being negatively correlated with the productivity of a data analyst during a data analysis session~\cite{liu2014effects}.
% Sessions are commonly conducted within visualization systems that contain a variety of optimization techniques to ensure that query response times fall within a certain latency tolerance.

% \mypara{Data prefetching}
% Data is often loaded into memory on a by-need basis in visualization systems to minimize interference with user-issued query computations~\cite{mani2017effective,xin2021enhancing,galakatos2017revisiting, yan2020auto, battle2016dynamic, crotty2016case, jalaparti2018netco}. 
% Query-time data retrieval can be significantly expedited by anticipating the data usage of the user in future queries and pre-loading the data into memory during the downtime between user queries (`think time'). SMART~\cite{mani2017effective} prefetches data for modified versions of current user-issued queries with different filters and dimensions. A-WARE~\cite{crotty2016case} maintains a sample store constantly refined through ingesting data based on speculations of future plots.
% ForeCache~\cite{battle2016dynamic} uses an SVM to predict the user's current analysis phase and accordingly prefetches data tiles partitioned based on different numerical values. NetCo predicts future queries via log analysis, and solves an ILP formulation to prefetch data to maximize the number of SLO-meeting queries~\cite{jalaparti2018netco}.
% In the case of MV refresh workloads, `think time' is nonexistent as individual MVs are refreshed back-to-back, rendering data prefetching techniques non-applicable.

\mypara{Intermediate Data Caching}
Some existing data visualization systems cache user-defined variables to support the typical incremental construction of data visualizations~\cite{zgraggen2016progressive, eichmann2020idebench} during data analysis sessions~\cite{jupyter, rstudio, colab}. 
Recent work proposes a management scheme for these cached variables under a memory constraint that greedily keeps variables with the highest estimated time savings based on predicted future user behavior via neural networks~\cite{xin2021enhancing}.
While useful for data visualization, a greedy approach to memory management fails to achieve satisfactory results compared to \system.

\mypara{Intermediate Result Reuse}

There exist works on storing intermediate results from computations to speedup future computations~\cite{yang2018intermediate, dursun2017revisiting, nagel2013recycling, michiardi2019memory, galakatos2017revisiting}.
Studied topics include the identification of reuse opportunities by finding overlaps in computation graphs of successive jobs~\cite{yang2018intermediate, michiardi2019memory},
selective storage under a space constraint with heuristics such as reuse probability~\cite{dursun2017revisiting}, expected savings~\cite{yang2018intermediate}, and recompute-storage cost difference~\cite{nagel2013recycling},
and rewriting incoming jobs to take advantage of stored intermediates~\cite{galakatos2017revisiting}.
These works share similarity with \system in their selection of items to store under a memory constraint, however, \system's problem setting requires it to uniquely consider the joint (re)ordering of job executions along with the selection of items.

% work that considers both job execution (re)order as well as intermediate result caching with a bounded amount of memory. but notably lack the joint aspect of \system and cannot be used to achieve immediate speedup on an incoming MV refresh run if no intermediates are stored beforehand. 

\mypara{Incremental Query Processing} Incremental processing (IQP) is useful for cases where not all data required for a query is immediately available. Similar to online aggregation~\cite{hellerstein1997online}, initial results of a query are computed on a subset of required data and progressively refined as the rest of the required data arrives in a predictable pattern~\cite{tang2019intermittent,wangtempura}. Tang et al. propose a dynamic programming formulation to pick intermediate states to store in memory given a limited memory budget~\cite{tang2019intermittent}. Tempura rewrites the query plan for more efficient execution based on predicted data arrival patterns~\cite{wangtempura}. While similarities exist between the problem setting of IQP and \system, such as management of bounded memory, \system notably includes additional joint optimization for the order of MV updates.

% \paragraph{Sampling}
% Sampling has seen wide use in visualization systems for reducing the computation time of ad-hoc queries by computing an approximate result over a subset of data as exact results are not always required by the user~\cite{crotty2016case, mani2017effective, zgraggen2014panoramicdata, kraska2021northstar, galakatos2017revisiting, kandula2016quickr}. 
% Commonly studied topics in sampling for ad-hoc queries include complex query sampling~\cite{kandula2016quickr}, rare event aggregation~\cite{kraska2021northstar, galakatos2017revisiting}, and maintaining consistency between related sampled visualizations~\cite{zgraggen2014panoramicdata}.
% Sampling server-side at the MV level compromises the assumptions of downstream applications and is thus not considered in \system.

% \paragraph{Progressive visualization}
% The latency tolerance for time-consuming queries can be circumvented by presenting a partially-computed visualization to the user within the tolerance, which is then incrementally refined until it is fully accurate~\cite{rahman2017ve, zgraggen2016progressive, crotty2015vizdom, kraska2021northstar, kamat2017infiniviz}.
% Example plots which benefit from progressive visualization include bar charts~\cite{kamat2017infiniviz} and heatmaps~\cite{rahman2017ve}.
% Similar to sampling, study on this topic is orthogonal to \system as pushing out partially-updated MVs compromises downstream assumptions.
\section{Mechanics of Pivoting}\label{sec:mechanics}
% What We need to say:
% 1. Stability margin Concept. Given fx, fy (u) and length, parameters, we can have safety margin so that we can discuss stability
% exact formulation of safety margin in formulation
% during manipulation, you may not be able to get the precise parameters. Then your manipulation task can fail.
% however, in practice, we don't need to have so much accurate parameters. of course, we can do it better with better estimated parameters but we have uncertainty during the manipulation since inherently friction already compensate for uncertainty as long as it satisfies static equilibrium of force moment and friction cones. For example accordingly friction magnitude can change to some extent. This would not happen to other example. This is due to the property of friction forces. our prime goal in this paper is to somehow utilize this frictional feature so that we can consider the most robust nominal trajectory

\begin{figure}
    \centering    \includegraphics[width=0.4\textwidth]{Figures/pivot22_new_notation.png} 
    \caption{A schematic showing the free-body diagram of a rigid body during pivoting manipulation \revise{when the relative angle between $F_W$ and $F_S$ is zero.} Point $P$ is the contact point with a manipulator.
    \revise{ The black circle represents the origin of each frame. 
    The object experiences four forces corresponding to two friction forces  from external contact points $A$ and $B$, one control input $f_P$ from the manipulator at point $P$, and gravity at point $C$.
    }
    }
    \label{fig:mechanics_pivoting_eq}
\end{figure}

\begin{figure}
    \centering    \includegraphics[width=0.4\textwidth]{Figures/pivot22_new_notation_frame_explanation.png} 
    \caption{\revise{A schematic showing the frame definition of a rigid body during pivoting manipulation. $F_W$, $F_S$, $F_O$, and $F_B$ are the world frame, slope frame, object frame, and frame at contact location $B$, respectively. 
    Gravity is defined in $F_W$ where the gravity is parallel to $y$-axis of $F_W$. 
    Pivoting manipulation happens with extrinsic contact $A$ and $B$ defined in $F_S$. $F_O$ is fixed with CoM of an object. $F_B$ is in parallel to $F_S$ with offset $B^S_x$ along $x$-axis of $F_S$.    We also show an example of $i_x^\Sigma$ and $i_x^\Sigma$ in \tab{tab:notion}. In this example,  $C_x^B$ and $C_y^B$ are illustrated.}
    }
    \label{fig:frame_def}
\end{figure}

% \begin{figure}
%     \centering    \includegraphics[width=0.45\textwidth]{Figures/slope_box.png} 
%     \caption{\textcolor{blue}{
%     A schematic showing the free-body diagram of a rigid body on an inclined surface during pivoting manipulation. }}
%     \label{fig:mechanics_pivoting_eq_slope}
% \end{figure}


In this section, we explain quasi-static stability of two-point pivoting in a plane. %We use this to present our proposed concept of \textit{frictional stability} which explains how friction can compensate for the inaccuracy of physical parameters during pivoting.
% 
Before explaining the details, we present our assumptions in this work. The following assumptions are used in the model for the pivoting manipulation task presented in this paper:
\begin{enumerate}
\item The object is rigid.
\item We consider \revise{quasi-static} equilibrium of the object.
\item The external contact surfaces are perfectly flat. 
\item The dimensions and pose of the object is perfectly known.
% \item The frictional parameters for the contact between the object and manipulator are perfectly known.
\item The object makes point contacts.
\end{enumerate}
% The above assumptions are common in manipulation problems. 
% Regarding the fifth assumption, we should mention that other parameters such as coefficient of friction can be also uncertain. However, uncertainty in coefficient of friction would lead to stochastic complementarity system, which is left as a future work~\cite{yuki2021chance}. 

\subsection{Mechanics of Pivoting with External Contacts}
\begin{table}[]
    \centering
        \caption{\revise{Notation of variables for analysis of frictional stability margin. In $\Sigma$ column, we indicate the frame of variables. We use the following indices for defining variables in this table: $j \in \{A, B, C, P\} $ for representing the location of frames, $i \in \{A, B, P\} $ for representing contact location, and $\Sigma \in \{W, S, O, B\} $ for representing a frame. }
        }

\begin{tabular}{|c|c|c|c|}
\hline Name & Description & Size & $\Sigma$ \\
\hline
% $j$ & indices for location. $j \in \{A, B, C, P\} $ &   &  \\
 % $i$ & indices for contact location.  $i \in \{A, B, P\} $ &   &  \\
 % % $q$ &  $q \in \{A, B\} $ &   &  \\
 % $\Sigma$ & indices for frame.  $\Sigma \in \{W, S, O, B\} $ &   &  \\
 $F_\Sigma$ & $\Sigma$ frame.   &   &  \\
  $f_{nj}^\Sigma$ & normal force at  $j$ in frame $F_\Sigma$& $\mathbb{R}^1$  & $\Sigma$ \\
    $f_{tj}^\Sigma$ & friction force at  $j$ in frame $F_\Sigma$& $\mathbb{R}^1$  & $\Sigma$ \\
 % $f_{nq}$ & normal force at contact $q$ & $\mathbb{R}^1$  & $S$ \\
 % $f_{tq}$ & friction force at contact $q$ &  
 % $\mathbb{R}^1$  & $S$ \\
 %  $f_{np}$ & normal force at contact $P$ & $\mathbb{R}^1$  & $O$ \\
 % $f_{tp}$ & friction force at contact $P$ & $\mathbb{R}^1$  & $O$ \\
   $f_{xj}^\Sigma$ & force at $j$ along $x$-axis in frame $F_\Sigma$  & $\mathbb{R}^1$  & $\Sigma$ \\
   $f_{yj}^\Sigma$ & force  at $j$ along $y$-axis in frame $F_\Sigma$  & $\mathbb{R}^1$  & $\Sigma$ \\
 $m$ & mass & $\mathbb{R}^1$  &  \\
 $g$ & gravity acceleration & $\mathbb{R}^1$  &  $W$ \\
  $l$ & length of an object & $\mathbb{R}^1$  &   \\
    $w$ & width of an object & $\mathbb{R}^1$  &   \\
  $\mu_i$ & coefficient of friction at $i$ & $\mathbb{R}^1$  & \\
  $i_x^\Sigma$ & contact location at $i$ along $x$-axis in frame $F_\Sigma$ & $\mathbb{R}^1$  & $\Sigma$ \\
    $i_y^\Sigma$ & contact location at $i$ along $y$-axis in frame $F_\Sigma$ & $\mathbb{R}^1$  & $\Sigma$ \\
      $\dot{i}_x^\Sigma$ & slipping velocity at $i$ along $x$-axis in frame $F_\Sigma$ & $\mathbb{R}^1$  & $\Sigma$ \\
      $\dot{i}_y^\Sigma$ & slipping velocity at $i$ along $y$-axis in frame $F_\Sigma$ & $\mathbb{R}^1$  & $\Sigma$ \\
     $\theta$ & angle of an object & $\mathbb{R}^1$  & $S$ \\
          $\phi$ & relative angle of frame from $\{F_W\}$ to $\{F_S\}$ & $\mathbb{R}^1$  & $W$ \\
          % $p_y^O$ & finger location in frame $O$ & $\mathbb{R}^1$  & $O$ \\
          %           $\dot{p}_y^O$ & finger slipping velocity in frame $O$ & $\mathbb{R}^1$  & $O$ \\
\hline
\end{tabular}
    \label{tab:notion}
\end{table}

%Before describing frictional stability, we describe our problem setting. 
We consider pivoting where the object maintains slipping contact with two external surfaces (see \fig{fig:mechanics_pivoting_eq}). A free body diagram showing the \revise{quasi-static} equilibrium of the object is shown in \fig{fig:mechanics_pivoting_eq}. 
\revise{The definitions of frames and  variables are summarized in 
\fig{fig:frame_def} and \tab{tab:notion}, respectively.}
% 
% The object experiences four forces corresponding to two friction forces $f_A, f_B$ from external contact points $A$ and $B$, one control input $f_P$ from manipulator at point $P$, and gravity, $mg$ at point $C$ where $m$ is mass of a body.
% We denote $f_{ni}, f_{ti}$ as a normal force and friction force at point $\forall i, i=\{A, B\}$, respectively, defined in ${\{F_W\}}$. $f_{nP}, f_{tP}$ are normal and friction force at point $P$ defined in ${\{F_B\}}$. Note that we define the $[f_x, f_y]^\top = \mathbf{R} [f_{nP}, f_{tP}]^\top$ where $\mathbf{R}$ is a rotation matrix from ${\{F_B\}}$ to ${\{F_W\}}$. We denote $x, y$ position at point in ${\{F_W\}}$ $\forall i, i=\{A, B, C, P\}$ as $i_x, i_y$, respectively. We denote $y$ position of point $P$ in ${\{F_B\}}$ as $p_y \in [-\frac{w}{2}, \frac{w}{2}]$.
% % 
% We define the angle of body with respect to $x$-axis as $\theta$. The coefficient of friction at point $\forall i, i=\{A, B, P\}$ are $\mu_A, \mu_B, \mu_P$, respectively. 
In the later sections, we present trajectory optimization formulation where we consider
\revise{
decision variables at time step $k$ (e.g., $f_{k, ni}$). In this section, we remove $k$ to represent variables for simplicity.}
% In this section, we remove $k$ to represent variables for simplicity. }
% $f_{ni}, f_{ti}$, location variables $i_{x}, i_{y}$ $\forall i, i=\{A, B, C, P\}$, $\theta$, and $p_y$ at each time-step $k$ denoted as $f_{k, ni}, f_{k, ti}, i_{k, x}, i_{k, y}, \theta_k, p_{y, k}$. 

% By setting $B_x = B_y = 0$, the static equilibrium of force in $x$ and $y$ directions the static equilibrium of and moment along point $B$ can be given by:
The \revise{quasi-static} equilibrium conditions for the object \revise{in $F_B$ when the relative angle between  $F_W$ and $F_S$ is zero (see \fig{fig:mechanics_pivoting_eq})} can be represented by the following equations.
% Note that we consider the moment at point $B$ by setting $B_x = B_y = 0$:
% (note we consider the moment at point $B$ by setting $B_x = B_y = 0$):
\begin{subequations}
\begin{flalign}
 f_{nA}^{\revise{B}} + f_{tB}^{\revise{B}} + f_{xP}^{\revise{B}}   =0,\label{forceeq1}\\
f_{tA}^{\revise{B}} + f_{nB}^{\revise{B}} + mg + f_{yP}^{\revise{B}}   = 0,  \label{forceeq2}\\
A_x^{\revise{B}} f_{tA}^{\revise{B}} - A_y^{\revise{B}}f_{nA}^{\revise{B}} + C_x^{\revise{B}}mg + P_x^{\revise{B}}f_{yP}^{\revise{B}} - P_y^{\revise{B}}f_{xP}^{\revise{B}} = 0 \label{moment_eq1}
% -\frac{l_\text{com}}{2}c_{\theta - \gamma}f_{tA} + \frac{l_\text{com}}{2}s_{\theta - \gamma}f_{nA} -\frac{l_\text{com}}{2}c_{\theta + \gamma}f_{nb} +\frac{l_\text{com}}{2}s_{\theta + \gamma}f_{tb}
% (A_x-B_x)f_{tA} - (A_y-B_y)f_{nA} + (C_x-B_x)mg + (P_x-B_x)f_{y} - (P_y - B_y) f_x = 0
\end{flalign}
\label{force_eq}
\end{subequations}
\revise{Note that because we define $F_B$ as parallel to $F_S$, all force variables in $F_B$ and $F_S$ are the same. }
We consider Coulomb friction law which results in friction cone constraints as follows:
\begin{equation}
 |f_{tA}^{\revise{B}}|  \leq \mu_A f_{nA}^{\revise{B}}, |f_{tB}^{\revise{B}}|  \leq \mu_B f_{nB}^{\revise{B}}, \quad f_{nA}^{\revise{B}}, f_{nB}^{\revise{B}} \geq 0,
% f_{tA} + f_{nB} + mg + f_{yP}   = 0,  \label{forceeq2}\\
% A_xf_{tA} - A_yf_{nA} + C_xmg + P_xf_{y} - P_y f_x = 0
% -\frac{l_\text{com}}{2}c_{\theta - \gamma}f_{tA} + \frac{l_\text{com}}{2}s_{\theta - \gamma}f_{nA} -\frac{l_\text{com}}{2}c_{\theta + \gamma}f_{nb} +\frac{l_\text{com}}{2}s_{\theta + \gamma}f_{tb}
% (A_x-B_x)f_{tA} - (A_y-B_y)f_{nA} + (C_x-B_x)mg + (P_x-B_x)f_{y} - (P_y - B_y) f_x = 0
\label{general_FC}
\end{equation}
To describe sticking-slipping complementarity constraints, we have the following complementarity constraints at point $A, B$:
\begin{subequations}
\begin{flalign}
 0 \leq  \revise{\dot{A}_{y+}^B} \perp \mu_A f_{nA}^{\revise{B}}-f_{tA}^{\revise{B}} \geq 0,  \\
 0 \leq   \revise{\dot{A}_{y-}^B} \perp \mu_A  f_{nA}^{\revise{B}}+f_{tA}^{\revise{B}} \geq 0, \\
 0 \leq  \revise{\dot{B}_{x+}^B} \perp \mu_B f_{nB}^{\revise{B}}-f_{tB}^{\revise{B}} \geq 0,  \\
 0 \leq   \revise{\dot{B}_{x-}^B} \perp \mu_B  f_{nB}^{\revise{B}}+f_{tB}^{\revise{B}} \geq 0
 \end{flalign}
 \label{slippingAB}
\end{subequations}
where the slipping velocities  follows \revise{ $\dot{A}_y^B=\dot{A}_{y+}^B-\dot{A}_{y-}^B, \dot{B}_x^B=\dot{B}_{x+}^B-\dot{B}_{x-}^B$}.
$\dot{A}_{y+}^B, \dot{A}_{y-}^B$ represent the slipping velocity \revise{at $A$} along positive and negative directions for \revise{$y$-axis in $F_B$}, respectively.
\revise{$\dot{B}_{x+}^B, \dot{B}_{x-}^B$ represent the slipping velocity at $B$ along positive and negative directions for $x$-axis in $F_B$}, respectively.
The notation $0 \leq a \perp b \geq 0$ means the complementarity constraints $a \geq 0, b \geq 0, a b=0$.
Since we consider slipping contact during pivoting, we have "equality" constraints in friction cone constraints at points $A, B$:
\begin{equation}
 f_{tA}^{\revise{B}}  =\mu_A f_{nA}^{\revise{B}}, f_{tB}^{\revise{B}}  =-\mu_B f_{nB}^{\revise{B}}
% f_{tA} + f_{nB} + mg + f_{yP}   = 0,  \label{forceeq2}\\
% A_xf_{tA} - A_yf_{nA} + C_xmg + P_xf_{y} - P_y f_x = 0
% -\frac{l_\text{com}}{2}c_{\theta - \gamma}f_{tA} + \frac{l_\text{com}}{2}s_{\theta - \gamma}f_{nA} -\frac{l_\text{com}}{2}c_{\theta + \gamma}f_{nb} +\frac{l_\text{com}}{2}s_{\theta + \gamma}f_{tb}
% (A_x-B_x)f_{tA} - (A_y-B_y)f_{nA} + (C_x-B_x)mg + (P_x-B_x)f_{y} - (P_y - B_y) f_x = 0
\label{slipping_friction_cone}
\end{equation}
To realize stable pivoting, actively controlling position of point $P$ is important. Thus, we consider the following complementarity constraints that represent the relation between the slipping velocity \revise{$\dot{P}_y$}  at point $P$ in \revise{$F_O$} and friction cone constraint at point $P$:
\begin{subequations}
\begin{flalign}
 0 \leq   \dot{P}_{y+}^{\revise{O}} \perp \mu_p f_{nP}^{\revise{O}}-f_{tP}^{\revise{O}} \geq 0  \\
 0 \leq   \dot{P}_{y-}^{\revise{O}} \perp \mu_p  f_{nP}^{\revise{O}}+f_{tP}^{\revise{O}} \geq 0 
 \end{flalign}
 \label{slippingP}
\end{subequations}
where \revise{$\dot{P}_y^O=\dot{P}_{y+}^O-\dot{P}_{y-}^O$}. 


\section{Robust Pivoting Formulation}\label{sec:sec_formulation}
% \subsection{Robust Pivoting Formulation}\label{subsec:robust_pivoting} 
In this section, we present a generic formulation for robust pivoting manipulation. In particular, we use the \revise{quasi-static} equilibrium conditions~\eqref{force_eq} in the presence of disturbances to formulate the robust planning problem. In particular, using sufficiency for stability of the object during manipulation we can estimate the bound of disturbance that can be tolerated during manipulation. Since this bound would depend on the pose of the object, we reason about the margin throughout the manipulation trajectory during the optimization problem formulation. We present the general idea in the following paragraph.

In the most general case, we assume that there is an external force $F_{ext}^{\revise{B}}$ and moment $M_{ext}^{\revise{B}}$ acting on the object during manipulation. Let us assume that the $x$ and $y$ component of the external force \revise{in $F_B$} are represented as $F_{ext,x}^{\revise{B}}$ and $F_{ext,y}^{\revise{B}}$ respectively. Then the \revise{quasi-static} equilibrium conditions~\eqref{force_eq} can be rewritten as follows:
\begin{subequations}
\begin{flalign}
 f_{nA}^{\revise{B}} + f_{tB}^{\revise{B}} + f_{xP}^{\revise{B}}+F_{ext,x}^{\revise{B}}  =0,\label{robust_forceeq1}\\
f_{tA}^{\revise{B}} + f_{nB}^{\revise{B}} + mg + f_{yP}^{\revise{B}}+F_{ext,y}^{\revise{B}}   = 0,  \label{robust_forceeq2}\\
A_x^{\revise{B}}f_{tA}^{\revise{B}} - A_y^{\revise{B}}f_{nA}^{\revise{B}} + C_x^{\revise{B}}mg + P_x^{\revise{B}}f_{yP}^{\revise{B}} 
- P_y^{\revise{B}} f_{xP}^{\revise{B}}  \nonumber \\+M_{ext}^{\revise{B}} = 0 
\label{robust_moment_eq1}
% -\frac{l_\text{com}}{2}c_{\theta - \gamma}f_{tA} + \frac{l_\text{com}}{2}s_{\theta - \gamma}f_{nA} -\frac{l_\text{com}}{2}c_{\theta + \gamma}f_{nb} +\frac{l_\text{com}}{2}s_{\theta + \gamma}f_{tb}
% (A_x-B_x)f_{tA} - (A_y-B_y)f_{nA} + (C_x-B_x)mg + (P_x-B_x)f_{y} - (P_y - B_y) f_x = 0
\end{flalign}
\label{robust_force_eq}
\end{subequations}
Note that $F_{ext}^{\revise{B}}$ and $M_{ext}^{\revise{B}}$ may not be independent of each other. They are related via the the point of application of force $F_{ext}^{\revise{B}}$ in the \revise{quasi-static} equilibrium conditions~\eqref{robust_force_eq}. These equations may not be satisfied for all possible values of $F_{ext}^{\revise{B}}$ and $M_{ext}^{\revise{B}}$. Since the contact forces can be readjusted in~\eqref{robust_force_eq}, the \revise{quasi-static} equilibrium can be satisfied for a certain range of $F_{ext}^{\revise{B}}$ and $M_{ext}^{\revise{B}}$. A generic analysis for estimating this margin or bound for which these disturbances can be compensated by contact forces is a bit involved as such a bound is dependent on the point and angle of application of the external force $F_{ext}^{\revise{B}}$. In the following sections, we present some specific cases which can be analyzed by making some simplifying assumptions on these disturbances. 
\revise{For brevity, we omit superscript $B$ of variables in the following sections because we consider \revise{quasi-static} equilibrium in $F_B$ unless we consider \revise{quasi-static} equilibrium in a different frame (see Sec~\ref{sec:stability_margin_finger_contact_location}). }
% \eq{slippingP} means that $\dot{p}_y$ is non-zero only at the boundary of friction cone.

% Additionally, one can have the following complementarity constraints to discuss the case where the object can lose contact at point B and can incur rotation about point A: 
% \begin{equation}
%  0 \leq   B_y \perp   f_{nB} \geq 0 
%  \label{complementarity_in_air}
% \end{equation}
% f_{tA} + f_{nB} + mg + f_{yP}   = 0,  \label{forceeq2}\\
% A_xf_{tA} - A_yf_{nA} + C_xmg + P_xf_{y} - P_y f_x = 0
% -\frac{l_\text{com}}{2}c_{\theta - \gamma}f_{tA} + \frac{l_\text{com}}{2}s_{\theta - \gamma}f_{nA} -\frac{l_\text{com}}{2}c_{\theta + \gamma}f_{nb} +\frac{l_\text{com}}{2}s_{\theta + \gamma}f_{tb}
% (A_x-B_x)f_{tA} - (A_y-B_y)f_{nA} + (C_x-B_x)mg + (P_x-B_x)f_{y} - (P_y - B_y) f_x = 0

% In this work, we do not consider \eq{complementarity_in_air} during optimization since \eq{complementarity_in_air} leads to stochastic complementarity system.

\subsection{Frictional Stability Margin}


%Otherwise, the manipulation may not be able to be completed due to uncertain parameters. However, in reality, uncertainty always exists and the manipulation can be completed under uncertainty to some extent. This is because friction forces inherently compensate for uncertainty as long as they satisfy static equilibrium of force and moments and friction cone constraints. the body can lift up by sticking at point $A$.  Hence, in this case, we can imagine that 

% In model-based manipulation, it is important to have precise estimate of physical parameters. However, it is desirable that a robot can compensate for uncertainty during manipulation of novel objects. 
% In this paper, we provide some insights about how a robot can use the stability margin in static equilibrium during manipulation to compensate for uncertainty in gravitational forces and moments. \fig{fig:concept} shows our proposed concept of frictional stability and bilevel robust trajectory optimization. The latter is discussed in Sec~\ref{sec:robust_to}. 
The robust quasi-static equilibrium conditions shown in~\eqref{robust_force_eq} can be used to explain the concept of stability margin. The stability margin is given by the magnitude of the external force $F_{ext}^{\revise{B}}$ and moment $M_{ext}^{\revise{B}}$ which can be satisfied in~\eqref{robust_force_eq} in any stable configuration of the object. This margin would depend on the contact force between the object and the environment as well as the control force used by the manipulator during the task. This provides the intuition that one can design a control trajectory such that the stability margin can be maximized.


We briefly provide some physical intuition about frictional stability for a few specific cases. First suppose that uncertainty exists in mass of a body. In the case when the actual mass is lower than  estimated, the friction force at point $A$ would increase while the friction force at point $B$ would decrease, compared to the nominal case. In contrast, suppose if the actual mass of the body is heavier than that of what we estimate, then the body can tumble along point $B$ in the clockwise direction. In this case, we can imagine that the friction force at point $A$ would decrease while the friction force at point $B$ would increase. However, as long as the friction forces are non-zero, the object can stay in contact with the external environment.
Similar arguments could be made for uncertainty in CoM location. The key point to note that the friction forces can re-distribute at the two contact locations and thus provide a margin of stability to compensate for uncertain gravitational forces and moments. We call this margin as \textit{frictional stability}.

In the following sections, we present the mathematical formulation of \textit{frictional stability} for cases when the mass, CoM location, friction coefficients, or \revise{finger contact location} are not known perfectly.
% In \fig{fig:concept},  if the actual CoM location is more left (represented as light-blue) than that of what we estimate (represented as black), then the body can lift up by sticking at point $A$.  Hence, in this case, we can imagine that the friction force at point $A$ would increase while the friction force at point $B$ would decrease. In contrast, if the actual CoM location is more right, then the body can tumble along point $B$ in the clockwise direction. In this case, we can imagine that the friction force at point $A$ would decrease while the friction force at point $B$ would increase. We could provide a similar insight when the body has uncertain CoM location. 

% In the next few subsections, we formalize the intuition for frictional stability margin proposed above and derive sufficient condition for stability of the object during pivoting.

%ly formulate frictional stability and confirm that what we describe above is actually confirmed mathematically. Here we consider the body with uncertain mass.

\subsection{Stability Margin for Uncertain Mass}\label{sec:sec_uncertain_mass}
 For simplicity, we denote $\epsilon$ as uncertain weight with respect to the estimated weight. Also, to emphasize that we consider the system under uncertainty, we put superscript $\epsilon$ for each friction force variable. Thus, the \revise{quasi-static} equilibrium conditions in \eq{force_eq} can be rewritten as:
\begin{subequations}
\begin{flalign}
f_{nA}^\epsilon + f_{tB}^\epsilon + f_{xP}  =0\label{forceeq11},\\
f_{tA}^\epsilon + f_{nB}^\epsilon + (mg + \epsilon) + f_{yP}   = 0,  \label{forceeq21}\\
A_xf_{tA}^\epsilon - A_yf_{nA}^\epsilon + C_x (mg+\epsilon) + P_xf_{yP}  = P_y f_{xP} \label{moment_eq11}
\end{flalign}
\label{force_eq_mass}
\end{subequations}
% equation with super script B
% \begin{subequations}
% \begin{flalign}
% f_{nA}^\epsilon + f_{tB}^\epsilon + f_{xP}^{\revise{B}}  =0\label{forceeq11},\\
% f_{tA}^\epsilon + f_{nB}^\epsilon + (mg + \epsilon) + f_{yP}^{\revise{B}}   = 0,  \label{forceeq21}\\
% A_x^{\revise{B}}f_{tA}^\epsilon - A_y^{\revise{B}}f_{nA}^\epsilon + C_x^{\revise{B}} (mg+\epsilon) + P_x^{\revise{B}}f_{yP}^{\revise{B}}  = P_y^{\revise{B}} f_{xP} \label{moment_eq11}
% % -\frac{l_\text{com}}{2}c_{\theta - \gamma}f_{tA} + \frac{l_\text{com}}{2}s_{\theta - \gamma}f_{nA} -\frac{l_\text{com}}{2}c_{\theta + \gamma}f_{nb} +\frac{l_\text{com}}{2}s_{\theta + \gamma}f_{tb}
% % (A_x-B_x)f_{tA} - (A_y-B_y)f_{nA} + (C_x-B_x)mg + (P_x-B_x)f_{y} - (P_y - B_y) f_x = 0
% %In order to realize the robustness, the more contact points are preferred since the system obtains more frictional stability. Therefore, we aim at ensuring contacts during pivoting (i.e., slipping). 
% \end{flalign}
% \label{force_eq_mass}
% \end{subequations}
Then, using \eq{slipping_friction_cone} and \eq{moment_eq11}, we obtain:
\begin{equation}
f_{nA}^\epsilon = \frac{-{C_x}\left(mg + \epsilon\right) -{P_x}f_{yP} + {P_y}f_{xP}}{\mu_A {A_x} - {A_y}}
\label{fna_111}
\end{equation}
To ensure that the body maintains contact with the external surfaces, we would like to enforce that the body experience non-zero normal forces at the both contacts.
To realize this, we have $f_{nA}^\epsilon \geq 0, f_{nB}^\epsilon \geq 0$ as conditions that the system needs to satisfy. Consequently, by simplifying \eq{fna_111}, we get the following:
\begin{subequations}
\begin{flalign}
\epsilon \geq \frac{P_yf_{xP} - P_xf_{yP} - C_xmg}{C_x}, \text{ if } C_x>0,  \label{fnacond1}\\
\epsilon \leq \frac{P_yf_{xP} - P_xf_{yP} - C_xmg}{C_x}, \text{ if } C_x<0  \label{fnacond2}
% -\frac{l_\text{com}}{2}c_{\theta - \gamma}f_{tA} + \frac{l_\text{com}}{2}s_{\theta - \gamma}f_{nA} -\frac{l_\text{com}}{2}c_{\theta + \gamma}f_{nb} +\frac{l_\text{com}}{2}s_{\theta + \gamma}f_{tb}
% (A_x-B_x)f_{tA} - (A_y-B_y)f_{nA} + (C_x-B_x)mg + (P_x-B_x)f_{y} - (P_y - B_y) f_x = 0
\end{flalign}
\label{fna_cond_mass}
\end{subequations}
Note that the upper-bound of $\epsilon$ means that the friction forces can exist even when we make the mass of the body lighter up to $\frac{\epsilon}{g}$. The lower-bound of $\epsilon$ means that the friction forces can exist even when we make the mass of the body heavier up to $\frac{\epsilon}{g}$. 
\eq{fna_cond_mass} provides some useful insights. \eq{fna_cond_mass} gives either upper- or lower-bound of $\epsilon$ for $f_{nA}^\epsilon$ according to the sign of $C_x$ (the moment arm of gravity). This is because the uncertain mass would generate an additional moment along with point $B$ in the clock-wise direction if $C_x >0$ and in the counter clock-wise direction if $C_x <0$. 
% In other words, the heavier body can make itself tumble and lose contact at point $A$ so it is not preferred if $C_x>0$, and lighter body can make itself again tumble and lose contact at point $A$ so it is not preferred if $C_x<0$  based on $f_{nA}^\epsilon \geq 0$. 
If $C_x = 0$, we have an unbounded range for $\epsilon$, meaning that the body would not lose contact at point $A$ no matter how much uncertainty exists in the mass. 

\eq{fna_cond_mass} can be reformulated as an inequality constraint: 
\begin{equation}
 C_x(\epsilon - \epsilon_A) \geq 0
% f_{tA} + f_{nB} + mg + f_{yP}   = 0,  \label{forceeq2}\\
% A_xf_{tA} - A_yf_{nA} + C_xmg + P_xf_{y} - P_y f_x = 0
% -\frac{l_\text{com}}{2}c_{\theta - \gamma}f_{tA} + \frac{l_\text{com}}{2}s_{\theta - \gamma}f_{nA} -\frac{l_\text{com}}{2}c_{\theta + \gamma}f_{nb} +\frac{l_\text{com}}{2}s_{\theta + \gamma}f_{tb}
% (A_x-B_x)f_{tA} - (A_y-B_y)f_{nA} + (C_x-B_x)mg + (P_x-B_x)f_{y} - (P_y - B_y) f_x = 0
\label{fna_cond_mass_one}
\end{equation}
where $\epsilon_A = \frac{P_yf_{xP} - P_xf_{yP} - C_xmg}{C_x}$.

We can derive condition for $\epsilon$ based on $f_{nB}^\epsilon \geq 0$ from \eq{slipping_friction_cone}, \eq{forceeq11}, and \eq{forceeq21}:
\begin{equation}
 \epsilon \leq \mu_A f_{xP} -f_{yP} -mg
% f_{tA} + f_{nB} + mg + f_{yP}   = 0,  \label{forceeq2}\\
% A_xf_{tA} - A_yf_{nA} + C_xmg + P_xf_{y} - P_y f_x = 0
% -\frac{l_\text{com}}{2}c_{\theta - \gamma}f_{tA} + \frac{l_\text{com}}{2}s_{\theta - \gamma}f_{nA} -\frac{l_\text{com}}{2}c_{\theta + \gamma}f_{nb} +\frac{l_\text{com}}{2}s_{\theta + \gamma}f_{tb}
% (A_x-B_x)f_{tA} - (A_y-B_y)f_{nA} + (C_x-B_x)mg + (P_x-B_x)f_{y} - (P_y - B_y) f_x = 0
\label{fnb_cond_mass}
\end{equation}
We only have upper-bound on $\epsilon$ based on $f_{nB}^\epsilon \geq 0$, meaning that the contact at point $B$ cannot be guaranteed if the actual mass is lighter than $\mu_A f_{xP} -f_{yP} -mg$. 
%Note that the conditions  we derive in this paper are sufficient but not necessary conditions.



\subsection{Stability Margin for Uncertain CoM Location}\label{sec:stability_margin_com_location}
% We consider the case with uncertain CoM location. We have a similar discussion we have in Sec~\ref{sec_uncertain_mass}. 
We denote $\revise{d_x^{{O}}, d_y^{{O}}}$ as residual CoM locations with respect to the estimated CoM location in $F_{\revise{O}}$ coordinate, respectively. Thus, the residual CoM location in $\revise{F_W}$, $\revise{d_x^{{W}}, d_y^{{W}}}$, are represented by $\revise{d_x^{{W}}} = d \cos({\theta + \theta_d}), \revise{d_y^{{W}}} = d \sin({\theta + \theta_d})$, where $d = \sqrt{\revise{\left({d_x^{{O}}}\right)^2 + \left({d_y^{{O}}}\right)^2}}$, $\theta_d = \arctan{\frac{\revise{{d_y^{{O}}}}}{\revise{{d_x^{{O}}}}}}$.  For notation simplicity, we use $r$ to represent $\revise{d_x^{{W}}}$.  In this paper, we put superscript $r$ for each friction force variable. The \revise{quasi-static} equilibrium conditions in \eq{force_eq} can be rewritten as follows:
\begin{subequations}
\begin{flalign}
f_{nA}^r + f_{tB}^r + f_{xP}  =0\label{forceeq12},\\
f_{tA}^r + f_{nB}^r + mg + f_{yP}   = 0,  \label{forceeq22}\\
A_xf_{tA}^r - A_yf_{nA}^r + (C_x + r) mg + P_xf_{yP}  = P_y f_{xP}  \label{moment_eq12}
\end{flalign}
\label{force_eq_location}
\end{subequations}
Then, using \eq{slipping_friction_cone} in \eq{force_eq_location}, we obtain:
\begin{subequations}
\begin{flalign}
r \leq \frac{P_yf_{xP} -P_x f_{yP}}{mg} - C_x \label{fna_fnb_r1},\\
r \geq  - \frac{\frac{\mu_A A_x - A_y}{1 + \mu_A}(-f_{xP}-f_{yP}-mg) -P_yf_{xP} + P_xf_{yP}}{mg}  -C_x  \label{fna_fnb_r2}
% -\frac{l_\text{com}}{2}c_{\theta - \gamma}f_{tA} + \frac{l_\text{com}}{2}s_{\theta - \gamma}f_{nA} -\frac{l_\text{com}}{2}c_{\theta + \gamma}f_{nb} +\frac{l_\text{com}}{2}s_{\theta + \gamma}f_{tb}
% (A_x-B_x)f_{tA} - (A_y-B_y)f_{nA} + (C_x-B_x)mg + (P_x-B_x)f_{y} - (P_y - B_y) f_x = 0
\end{flalign}
\label{fna_fnb_r}
\end{subequations}
where \eq{fna_fnb_r1}, \eq{fna_fnb_r2} are obtained based on $f_{nA}^r \geq 0, f_{nB}^r \geq 0$, respectively. \eq{fna_fnb_r} means that the object would lose contact at $A$ if the actual CoM location is more to the right than our expected CoM location while the object would lose the contact at $B$ if the actual CoM location is more to the left.

\subsection{Stability Margin for Stochastic Friction}\label{subsec:stochasticfriction_planning}
In this section, we present modeling and analysis of pivoting manipulation in the presence of stochastic friction coefficients. In particular, we consider stochastic friction at the two different contact points $A$ and $B$. We do not consider stochastic friction at the contact point between the robot and the manipulator since that leads to stochastic complementarity constraints (please see~\cite{shirai2023covariance, yuki2021chance} for detailed analysis on stochastic complementarity constraints). 
We make the assumption that the friction coefficients at $A$ and $B$ are partially known. In particular, we assume that the friction coefficients for contact at $A$ could be represented as $\mu_A=\hat{\mu}_A+\revise{\tilde{\mu}_A}$
where $\revise{\tilde{\mu}_A}$ is the uncertain stochastic variable.
% $\revise{\tilde{\mu}_A}\sim\mathcal N(0,\sigma_{\mu_A}^2)$. 
Similarly, the friction coefficient at $B$ could be represented as  $\mu_B=\hat{\mu}_B+\revise{\tilde{\mu}_B}$ where
  $\revise{\tilde{\mu}_B}$ is the uncertain stochastic variable. Note that we do not need to need to know any information regarding the \revise{probabilistic} distribution \revise{(e.g., probability density function of Gaussian distribution, beta distribution.)} of the unknown part. 
% $\revise{\tilde{\mu}_B}\sim \mathcal N(0,\sigma_{\mu_B}^2)$. 
We can rewrite~\eqref{robust_force_eq} for this case as follows. We put superscript $\mu$ for each friction variable:
\begin{subequations}
\begin{flalign}
 f_{nA}^\mu + \hat{f}_{tB}^\mu + f_{xP}+\epsilon_B  =0,\label{robust_forceeq_friction_friction}\\
\hat{f}_{tA}^\mu + f_{nB}^\mu + mg + f_{yP}+\epsilon_A   = 0,  \label{robust_forceeq_friction_friction}\\
A_x\hat{f}_{tA}^\mu+A_x\epsilon_A - A_yf_{nA}^\mu + C_xmg  \nonumber \\+ P_xf_{yP} - P_y f_{xP}=  0\label{robust_moment_eq_friction_friction}
\end{flalign}
\label{robust_force_eq_friction_friction}
\end{subequations}
where, $f_{tA}^\mu=\hat{f}_{tA}^\mu+f_{nA}^\mu\revise{\tilde{\mu}_A}$ and $f_{tB}^\mu=\hat{f}_{tB}^\mu+f_{nB}^\mu\revise{\tilde{\mu}_B}$. The above equations are obtained by representing $f_{nA}\revise{\tilde{\mu}_A}$ as $\epsilon_A$ for contact at $A$ and similarly, $\epsilon_B$ for the contact at $B$. Thus, $\epsilon_A$ and $\epsilon_B$ are the uncertain contact forces for the contacts at $A$ and $B$. The robust formulation that we consider in this paper considers the worst-case effect of these uncertainties on the stability of the object during manipulation. Thus, we try to maximize the bound of these variables $\epsilon_A$ and $\epsilon_B$ using our proposed bilevel optimization. It is noted that $\epsilon_A$ and $\epsilon_B$ are the stability margin for this particular case of stochastic friction.

To ensure that the body maintains contact, we impose $f_{nA}^\mu \geq 0, f_{nB}^\mu \geq 0$, so that we get the following inequalities for $\epsilon_A, \epsilon_B$:
\begin{subequations}
\begin{flalign}
-\mu_Af_{xP}  + \epsilon_A + mg + f_{yP} \leq \mu_A\epsilon_B
\\
\epsilon_B \leq -\mu_B (\epsilon_A + mg + f_{yP})- f_{xP}
 \label{moment_eq12_condition}
\end{flalign}
\label{force_eq_location_friction_condition}
\end{subequations}
To ensure slipping contact even in the presence of uncertainties, we need to satisfy friction cone constraints specified earlier in~\eqref{general_FC}, \eq{slipping_friction_cone}. Using these constraints, we can find the upper and lower bound for the variables $\epsilon_A$ and $\epsilon_B$: 
\begin{subequations}
\begin{flalign}
(\hat{\mu}_A+\revise{\tilde{\mu}_A}) f_{nA}^\mu = \hat{f}_{tA}^\mu+\revise{\tilde{\mu}_A}f_{nA}^\mu \\
(\hat{\mu}_B+\revise{\tilde{\mu}_B}) f_{nB}^\mu = -\hat{f}_{tB}^\mu-\revise{\tilde{\mu}_B}f_{nB}^\mu
\end{flalign}
\label{eq:sliping_eq_friction_uncertainty}
\end{subequations}
To get a lower bound for the variables $\epsilon_A$ and $\epsilon_B$, we make a assumption regarding the uncertainty for the friction coefficients at $A$ and $B$. We assume that the unknown part is bounded above by the known part, i.e., $\revise{\tilde{\mu}_i}\leq \hat{\mu}_i$, $\forall i=A,B$. Note that this is not a restrictive assumption. What this implies is that the above parameter has bounded uncertainty. For simplicity, we assume that uncertainty is bounded by the known part of the parameter. For example, if the friction coefficient is modeled as a stochastic random variable, then we assume that we know the mean of the friction parameter and the standard deviation is bounded by some multiple of mean (note that this bound is just for simplification and one can assume any practical bound for uncertainty).
% In practice, the variances of $\revise{\tilde{\mu}_A}$ and $\revise{\tilde{\mu}_B}$ are relatively small. Thus, we can argue that each realization of $\revise{\tilde{\mu}_A}$ and $\revise{\tilde{\mu}_B}$ are much smaller than $\nu_{\mu_A}$ and $\nu_{\mu_B}$, respectively.  
Consequently, we can derive the following relations:
\begin{subequations}
\begin{flalign}
-\hat{\mu}_A f_{nA}^\mu\leq \epsilon_A \leq \hat{\mu}_A f_{nA}^\mu
\\
-\hat{\mu}_B f_{nB}^\mu\leq \epsilon_B \leq \hat{\mu}_B f_{nB}^\mu
\end{flalign}
\label{eq:sliping_eq_friction_uncertainty_simple}
\end{subequations}
Thus, we get constraints~\eqref{force_eq_location_friction_condition} and~\eqref{eq:sliping_eq_friction_uncertainty_simple} for the stability margin by considering the stability and the friction cone constraints in the presence of uncertain friction coefficients. These constraints are used to estimate the stability margin during the proposed bilevel optimization.


\subsection{\textcolor{blue}{}
Stability Margin for Finger Contact Location}\label{sec:stability_margin_finger_contact_location}
\revise{
We consider another case of uncertainty which might arise due to an imperfect robot controller or due to imperfect pose information of the object. For this case, we consider the stability margin $d$ of finger contact location on an object, as illustrated in \fig{fig:mechanics_pivoting_finger_margin}. There could be multiple reasons for this uncertainty. One possible reason could be due to imperfect state information for the object being manipulated which can lead to imprecise information about the finger contact location. Another reason could be imprecise stiffness controller of the robot. It is noted that we use a stiffness controller for a position controlled robot to  implement the computed force trajectory. Due to compliance of the object and the robot, the actual robot trajectory is different from the planned and thus, this could lead to this uncertainty. 
% We consider the stability margin  due to imperfect stiffness controller from robotic manipulators. 
We can formulate the following \revise{quasi-static} equilibrium in $F_O$.  We put superscript $d$ for each extrinsic friction variable:
\begin{subequations}
\begin{flalign}
f_{xA}^{O, d} + f_{xB}^{O, d} + mg\sin{\theta} + f_{nP}^O  =0\label{forceeq12_finger},\\
f_{yA}^{O, d} + f_{yB}^{O, d} + mg\cos{\theta} + f_{tP}^O  =0\label{forceeq22_finger}\\
\sum_{i\in\{A, B\}}\left(
i_x^O f_{yi}^{O, d} - i_y^O f_{xi}^{O, d}
\right)
% A_x^O f_{yA}^{O, d} - A_y^O f_{xA}^{O, d} + B_x^O f_{yB}^{O, d} - B_y^O f_{xB}^{O, d} 
 \nonumber \\+P_x^O f_{tP}^O - (P_y^O + d) f_{nP}^O = 0  \label{moment_eq12_finger}
\end{flalign}
\label{force_eq_location_finger}
\end{subequations}
Note that $-A_x^O = -B_x^O = P_x^O =  \frac{l}{2}, A_y^O = -B_y^O = \frac{w}{2}$. Using this relation, we can simplify \eq{force_eq_location_finger}. In particular, we use $f_{xA}^{O, d} \geq 0,  f_{xB}^{O, d} \geq 0,  f_{nP}^{O} \geq 0$ and thus we can get the following bound for $d$:
\begin{subequations}
\begin{flalign}
 \underline{d} \leq d \leq \bar{d} 
\label{forceeq12_finger_margin},\\
\underline{d} = -A_x\frac{mg\cos{\theta} + 2f_{tP}}{f_{nP}}   - A_y \frac{mg\sin{\theta} + f_{nP}}{f_{nP}}  -P_y^O,\\
\bar{d} = -A_x\frac{mg\cos{\theta} + 2f_{tP}}{f_{nP}}   + A_y \frac{mg\sin{\theta} + f_{nP}}{f_{nP}}  -P_y^O
\label{forceeq22_finger_margin}
\end{flalign}
\label{force_eq_location_finger_margin}
\end{subequations}
}
\revise{
When $f_{nP}^O \rightarrow 0$, the equation suggests that $\bar{d}$ tends to infinity and $\underline{d}$ tends to negative infinity. As $f_{nP}^O = 0$ implies no force at point $P$, the finger's placement becomes inconsequential as it does not affect the \revise{quasi-static} equilibrium of the object.}

\revise{
We can consider that uncertainty in finger contact location and uncertainty in the geometry of an object have a similar influence on the manipulation. This is because the relative pose of the object with respect to the robot changes for both cases, resulting in the potential contact mode changes.  
}
% As 
% Consider the case where $f_{np} \rightarrow
%  0$. In this case, \eq{force_eq_location_finger_margin} indicate that $\bar{d}\rightarrow
% \infty$ and $\underline{d}\rightarrow
% -\infty$. Because $f_{np} = 0$ means there is no force at point $P$, the finger can be placed anywhere since finger position does not matter to the static equilibrium of the object. 



\begin{figure}
    \centering    \includegraphics[width=0.4\textwidth]{Figures/finger_uncertainty.png} 
    \caption{\revise{
    A schematic showing the free-body diagram of a rigid body during pivoting manipulation. We consider the stability margin of finger location due to imperfect control of stiffness controller in a robotic manipulator.}}
    \label{fig:mechanics_pivoting_finger_margin}
\end{figure}

\revise{
\subsection{Stability Margin for Uncertain Mass on a Slope}\label{sec:sec_uncertain_mass_slope}
We consider the case where we tilt the two external walls by the angle of $\phi$. 
% as illustrated in \fig{fig:mechanics_pivoting_eq_slope}. 
Our discussion in Sec.~\ref{sec:sec_uncertain_mass} still holds. The only difference arises from gravity terms. Hence, the \revise{quasi-static} equilibrium conditions in $F_B$ can be rewritten as:
\begin{subequations}
\begin{flalign}
f_{nA}^\epsilon + f_{tB}^\epsilon + f_{xP} + (mg + \epsilon) \sin{\phi}  =0\label{forceeq11_slope},\\
f_{tA}^\epsilon + f_{nB}^\epsilon + f_{yP} + (mg + \epsilon) \cos{\phi}    = 0,  \label{forceeq21_slope}\\
A_xf_{tA}^\epsilon - A_yf_{nA}^\epsilon + \left(C_x \cos{\phi} -  C_y \sin{\phi}\right) (mg+\epsilon) \nonumber \\ + P_xf_{yP} - P_y f_{xP} = 0 \label{moment_eq11_slope}
\end{flalign}
\label{force_eq_mass_slope}
\end{subequations}
Following the same logic in Sec.~\ref{sec:sec_uncertain_mass}, we can get the following bound for the stability margin $\epsilon$ under uncertain mass when the object is on a slope: 
\begin{subequations}
\begin{flalign}
\epsilon \geq \frac{P_yf_{xP} - P_xf_{yP} - (C_x\cos{\phi}-C_y \sin{\phi}) mg}{C_x\cos{\phi}-C_y\sin{\phi}}, \nonumber \\ \text{ if } C_x\cos{\phi}>C_y\sin{\phi} \label{fnacond1_slope}\\
\epsilon \leq \frac{P_yf_{xP} - P_xf_{yP} - (C_x\cos{\phi}-C_y \sin{\phi}) mg}{C_x\cos{\phi}-C_y\sin{\phi}}, 
\nonumber \\ \text{ if } C_x\cos{\phi}<C_y\sin{\phi}  \label{fnacond2_slope}
\end{flalign}
\label{fna_cond_mass_slope}
\end{subequations}
As a result, \eq{fnacond1_slope} and \eq{fnacond2_slope} result in the following inequality constraint:
\begin{equation}
 \left(C_x\cos{\phi}-C_y\sin{\phi}  \right)(\epsilon - \epsilon_A) \geq 0
\label{eq:uncertain_mass_slope}
\end{equation}
where $\epsilon_A = \frac{P_yf_{xP} - P_xf_{yP} - (C_x\cos{\phi}-C_y \sin{\phi}) mg}{C_x\cos{\phi}-C_y\sin{\phi}}$. We also derive the bound on $\epsilon$ using $f_{nB}^\epsilon\geq 0$, \eq{fnacond1_slope}, and \eq{fnacond2_slope}:
\begin{equation}
 \left(\mu_A \sin{\phi} - \cos{\phi}  \right)\epsilon \geq f_{yP} -\mu_A f_{xP} 
\label{eq:uncertain_mass_slope_upper_bound}
\end{equation}
Note that the sign of $\mu_A \sin{\phi} - \cos{\phi}$ can change depending on the angle of slope. In this paper, we choose $\phi$ such that the sign of $\mu_A \sin{\phi} - \cos{\phi}$ does not change during manipulation. 
}

\revise{The discussion in this section for manipulation under uncertain mass on a slope can be easily extended with other uncertain parameters such as CoM location, friction, and finger contact location. }


\subsection{Pivoting with Patch Contact between the object and the manipulator}\label{subsec:Pivoting_manipulation}


\begin{figure}[t]
    \centering
    \includegraphics[width=0.35\textwidth]{Figures/patch.png} % 
    \caption{A schematic showing the free-body diagram of a rigid body during pivoting manipulation with patch contact. We approximate patch contact as two point contacts $P_1$ and $P_2$ with the same force distribution. We assume that $P_1$ always lies on the vertex of the object for this simplistic patch contact model. \revise{$s$ is the distance between point contact $P_1$ and $P_2$ along $y$-axis of $F_O$.}}
    \label{fig:mechanics_pivoting_patch_eq}
\end{figure}

In the previous sections, we considered point contact between the manipulator and the object. This could be potentially restrictive. Moreover, this may not be a realistic assumption when a robot is interacting with objects. 
In this section, we present a slightly modified formulation by considering patch contact between the object and the manipulator. We would like to analyze and understand how patch contact would compare against a point contact model for stability during pivoting manipulation. \fig{fig:mechanics_pivoting_patch_eq} shows the simplest patch contact model during the pivoting task we consider in this paper. Using this model, we can write the following quasi-static equilibrium:
\begin{subequations}
\begin{flalign}
 f_{nA} + f_{tB} + \revise{f_{xP_{1}}+ f_{xP_{2}}}   =0,\label{forceeq1}\\
 f_{tA} + f_{nB} + mg + \revise{f_{yP_{1}}+ f_{yP_{2}}}= 0,\label{forceeq2}\\
 A_xf_{tA}- A_yf_{nA} + C_xmg  \nonumber \\ + \sum_{i=1}^2 \left(P_{{i}_x} f_{yP_{i}} - P_{{i}_y} f_{xP_{i}}\right) = 0
% f_{tA} + f_{nB} + mg + 2f_{yP}= 0,  \label{forceeq2}\\
% A_xf_{tA} - A_yf_{nA} + C_xmg + \sum_{i=1}^2 \left(P_{ix} f_{y} - P_{iy} f_x\right) = 0 \label{moment_eq1}
\end{flalign}
\label{force_balance_patch_contact}
\end{subequations}
where $P_{{i}_x}, P_{{i}_y}$ represent $x$ and $y$ coordinate of $P_1$ and $P_2$ \revise{in $F_O$}, respectively. 
\revise{In this work, we assume that patch contact as two point contacts $P_1$ and $P_2$ as the same force distribution, which indicates that $f_{xP_{1}} = f_{xP_{2}}, f_{yP_{1}} = f_{yP_{2}}$.}  
\revise{$s$} is the distance between point contact $P_1$ and $P_2$ and \revise{$s$} is a decision variable, meaning that location of $P_2$ is a decision variable and can change over time. In this work, we assume that $P_1$ does not move over time, which simplifies the model of patch contact. 

Using the above \revise{quasi-static} equilibrium conditions with $f_{nA}\geq 0, f_{nB}\geq 0$, we can solve and find the upper and the lower bound of stability margin under the various uncertainties described earlier in the previous subsections. We will present some results in the later section using this formulation and compare them against the point contact formulation.

\revise{\textit{Remark 1}: The patch contact discussion in this section can be extended into the patch contact at extrinsic contact with sliding contacts. We can approximate the extrinsic patch contact as two-point contacts with the same force distribution. Then, we can formulate the quasi-static equilibrium and derive the bound of the stability margin.}
% based on the discussion in Sec~\ref{sec_uncertain_mass} and Sec~\ref{sec:stability_margin_com_location}.

% In particular, we consider two different models of the patch contact, and try to answer the following questions:
% \begin{enumerate}
%     \item Does patch contact model allow better feasibility for the trajectory optimization during pivoting compared to the point contact model?
%     \item How does the force trajectory using the patch contact model compare against the point contact model?
% \end{enumerate}





% \subsection{Stability Margin of Pivoting under Uncertain Mass and CoM Location}

% % % In this section, we describe the optimal control problem for stable pivoting. To formulate the problem, we first describe kinematics and dynamics of pivot as illustrated in \fig{fig:mechanics_pivoting_eq}. We have a rigid body with three contact points $A, B, P$ where point $P$ is a contact point where a robot exerts control forces. 
% % % % Also, we use the following notation to represent the position of $i$ contact point: $p_[]$

% % % We use the following notation frequently. $l_\text{com}$ is the length of diagonal of a rectangle, $l_\text{com} = \sqrt{w^2 + l^2}$. Also, for simplicity, we use the following notation: $\sin{\theta} = s_\theta, \cos{\theta} = c_\theta, \sin{(\theta \pm \gamma)} = s_{\theta\pm \gamma}, \cos{(\theta \pm \gamma)} = c_{\theta\pm \gamma}$. Also, we define the origin of coordinate in \fig{fig:mechanics_pivoting_eq} as $x_O, y_O$.

% % % \subsection{Kinematics}
% % % We derive the kinematics equation for \fig{fig:mechanics_pivoting_eq}. Assuming that the location of center of mass is same to the location of center of geometry, which is located at the middle of the object, we use 

% % % \begin{subequations}
% % % \begin{flalign}
% % %  x_\text{com}=\frac{1}{2}l_\text{com}c_{\theta-\gamma} + x_O ,\label{kinematics1}\\
% % % y_\text{com}=\frac{1}{2}l_\text{com}s_{\theta-\gamma} + wc_\theta +y_O \label{kinematics2}
% % % \end{flalign}
% % % \label{kinematics}
% % % \end{subequations}

% % % \subsection{Dynamics}
% % % Here, we derive the quasi-static dynamics equations for pivoting. Since we are interested in working on planar pivoting, we have two static force equilibrium along $x, y$ and one moment equilibrium equation, which is calculated at point B in this paper. 

% % % The static equilibrium equations are: 


% % % $\beta$ and $l_p$ are derived as follows. $l_p =\sqrt{p_x^2 + p_y^2}$. First we can create a triangle BOP and $\alpha  =\arctan \left(\frac{p_y}{p_x}\right)$ and using law of cosines, $c^2=l_p^2 + (ws_\theta)^2 - 2 l_p ws_\theta c_\alpha$. Using law of cosines again, $l_p^2 = (ws_\theta)^2 + c^2 - 2cws_\theta c_{\pi-\beta - \theta}$. By organizing this, we can get $c_{\beta + \theta} = l_p c_\alpha - ws_\theta$. Then, we can get $\beta = \arccos{(l_p c_\alpha - ws_\theta)} - \theta$.

% % % We have three equations but we have the following decision variables: $f_{nA}, f_{tA}, f_{nB}, f_{tB}, f_{xP}, f_{yP}, \theta, p_x, p_y$.

% % % % Discussion point: where do we want to have complementarity constraints? For example, if we consider slipping, like pushing example, we have the complementarity constraints like: 
% % % % \begin{equation}
% % % % \begin{aligned}
% % % %  0 \leq   \dot{p}_{y+} \perp \mu_p f_{\overrightarrow{n}}(t)-f_{\overrightarrow{t}}(t) \geq 0  \\
% % % %  0 \leq   \dot{p}_{y-} \perp \mu_p f_{\overrightarrow{n}}(t)+f_{\overrightarrow{t}}(t) \geq 0 
% % % %  \end{aligned}
% % % %  \label{eqn:compl_pushing}
% % % % \end{equation}

% % % % Another complementarity constraint would be the touching condition. For example, we observe that sometimes only one point (e.g., point A) makes a contact and the other point is in the air. In this case, point A is slipping and following the complementarity constraints based on \eq{eqn:compl_pushing} and point B is in the air which follows the complementarity constraints such as:
% % % % \begin{equation}
% % % % \begin{aligned}
% % % %  0 \leq   f_{nB} \perp \phi_B\geq 0
% % % %  \end{aligned}
% % % %  \label{eqn:compl_force_air}
% % % % \end{equation}
% % % % where $\phi_B$ is the distance from the point B to the object along $y$ axis. Eventually, we have duplicated complementarity constraints (e.g., is point B making contact? if yes, is it slipping to which direction?), which can be computationally challenging but an interesting problem.



% % % % Another discussion point: if one of the coefficient of frictions or normal force is very large, we can have the one point contact where one of the other contact is now in the air. So, for our case, maybe we can prevent it from happening by bounding the $f_{xP}, f_{yP}$.

% % % % Another point: lifting up the object can be easier than lifting down the object. 

% % % % \subsection{Dynamics for General Objects}
% % % % We describe the more general dynamics which does not depend on the shape of the object. 


% % \subsection{Stability Margin Discussion for Uncertain Mass}
% % See \fig{fig:mechanics_pivoting_eq} for the definition of notation. Here, we first introduce static equilibrium of force equation as:
% % \begin{subequations}
% % \begin{flalign}
% %  f_{nA} + f_{tB} + f_{xP}  =0,\label{forceeq10}\\
% % f_{tA} + f_{nB} + mg + f_{yP} + \epsilon   = 0,  \label{forceeq20}
% % \end{flalign}
% % \label{force_eqm}
% % \end{subequations}
% % where $\epsilon$ is the stability margin based on inaccurate mass. 

% % % Also, along point O, we have the following static equilibrium of moment equation. For simplicity, we set $Q_{By} = Q_{Ax}=0$.
% % % \begin{subequations}
% % % \begin{flalign}
% % %  Q_{Bx}f_{nB} - Q_{Ay}f_{nA} + Q_{Cx}(mg+\epsilon) + Q_{Px} f_{Py} - Q_{Py}f_{Px}  =0,\label{momenteq10}
% % % \end{flalign}
% % % \label{moment_eqm}
% % % \end{subequations}
% % % Using slipping condition $f_{tB} = -\mu_B f_{nB}$ and \eq{forceeq10}, we can represent $f_{nA}$ as a function of $\epsilon$ given $mg, f_{Px}, f_{Py}$ as follows:
% % % \begin{equation}
% % % f_{nA} = \frac{-(Q_{Cx}(mg+\epsilon) + Q_{Px}f_{Py} + (\frac{Q_{Bx}}{\mu_B}-Q_{Py})f_{Px})}{\frac{ Q_{Bx}}{\mu_B} - Q_{Ay}},
% % % \label{fna}
% % % \end{equation}
% % % Then, 
% % % \begin{equation}
% % % f_{nB} = \frac{1}{\mu_B} \left(f_{Px} + \frac{-(Q_{Cx}(mg+\epsilon) + Q_{Px}f_{Py} + (\frac{Q_{Bx}}{\mu_B}-Q_{Py})f_{Px})}{\frac{ Q_{Bx}}{\mu_B} - Q_{Ay}}\right),
% % % \label{fna}
% % % \end{equation}

% % In order to have the robustness on the controller, we have $f_{nA} >0, f_{nB} >0$. Thus, from these equations, we can specify condition as:
% % % \begin{subequations}
% % % \begin{flalign}
% % % -Q_{Px}f_{Py} + Q_{Py}f_{Px} < Q_{Cx}(mg + \epsilon),\label{fcondition1}\\
% % % f_{Px} (Q_{Ax}\mu_A - Q_{Ay}) -Q_{Px}f_{Py} + Q_{Py}f_{Px} < Q_{Cx}(mg + \epsilon),  \label{fcondition2}
% % % \end{flalign}
% % % \label{force_condition}
% % % \end{subequations}
% % % Since we cannot specify the sign of $Q_{Cx}$, \eq{fcondition1} cannot be organized more. Also, since $Q_{Ax}\mu_A - Q_{Ay} <0$ and "likely" $f_{Px} <0$, $f_{Px} (Q_{Ax}\mu_A - Q_{Ay}) >0$. Therefore, \eq{fcondition2} is more tight than \eq{fcondition1}. So, if  \eq{fcondition2} is satisfied, \eq{fcondition1} is automatically satisfied (need to discuss)? This may not be true since the sign of $Q_{Cx}$ can change.

% % Also, along point B, we have the following static equilibrium of moment equation. For simplicity, we set $Q_{Bx} = Q_{By}=0$.
% % \begin{equation}
% %  Q_{Ax}f_{tA} - Q_{Ay}f_{nA} + Q_{Cx}(mg+\epsilon) + Q_{Px} f_{Py} - Q_{Py}f_{Px}  =0,\label{momenteq10}
% % \end{equation}
% % Using slipping condition $f_{tA} = \mu_A f_{nA}$, we can represent $f_{nA}$ as a function of $\epsilon$ given $mg, f_{Px}, f_{Py}$ as follows:
% % Here, using $f_{tB} = -\mu_B f_{nB}$ and \eq{forceeq10}, 
% % \begin{equation}
% % f_{nB} = \frac{\mu_Af_{Px} - f_{Py} - mg - \epsilon}{1 + \mu_A \mu_B},
% % \label{fnb}
% % \end{equation}


% % In order to have the robustness on the controller, we have $f_{nA} >0, f_{nB} >0$. Thus, from these equations, we can specify condition as:
% % \begin{subequations}
% % \begin{flalign}
% % -Q_{Px}f_{Py} + Q_{Py}f_{Px} < Q_{Cx}(mg + \epsilon),\label{fcondition1}\\
% % \epsilon < \mu_Af_{Px}-f_{Py} - mg,  \label{fcondition2}
% % \end{flalign}
% % \label{force_condition}
% % \end{subequations}
% % Since we cannot specify the sign of $Q_{Cx}$, \eq{fcondition1} cannot be organized more. Also, since $Q_{Ax}\mu_A - Q_{Ay} <0$ and "likely" $f_{Px} <0$, $f_{Px} (Q_{Ax}\mu_A - Q_{Ay}) >0$. Therefore, \eq{fcondition2} is more tight than \eq{fcondition1}. So, if  \eq{fcondition2} is satisfied, \eq{fcondition1} is automatically satisfied (need to discuss)? This may not be true since the sign of $Q_{Cx}$ can change.


% % Using the slipping condition at Point B $f_{tB} = -\mu_B f_{nB}$ and \eq{forceeq10}, 
% % \begin{equation}
% % f_{nB} = \frac{1}{\mu_B} \left(f_{Px} +  \frac{-Cx\left(mg + \epsilon\right) -Q_{Px}f_{Py} + Q_{Py}f_{Px}}{\mu_A Q_{Ax} - Q_{Ay}}\right),
% % \label{fna}
% % \end{equation}

% % In order to have the robustness on the controller, we have $f_{nA} >0, f_{nB} >0$. Thus, from these equations, we can specify condition as:
% % \begin{subequations}
% % \begin{flalign}
% % -Q_{Px}f_{Py} + Q_{Py}f_{Px} < Q_{Cx}(mg + \epsilon),\label{fcondition1}\\
% % f_{Px} (Q_{Ax}\mu_A - Q_{Ay}) -Q_{Px}f_{Py} + Q_{Py}f_{Px} < Q_{Cx}(mg + \epsilon),  \label{fcondition2}
% % \end{flalign}
% % \label{force_condition}
% % \end{subequations}
% % Since we cannot specify the sign of $Q_{Cx}$, \eq{fcondition1} cannot be organized more. Also, since $Q_{Ax}\mu_A - Q_{Ay} <0$ and "likely" $f_{Px} <0$, $f_{Px} (Q_{Ax}\mu_A - Q_{Ay}) >0$. Therefore, \eq{fcondition2} is more tight than \eq{fcondition1}. So, if  \eq{fcondition2} is satisfied, \eq{fcondition1} is automatically satisfied (need to discuss)? This may not be true since the sign of $Q_{Cx}$ can change.


% Next, let's consider the case where we have uncertainty $r$ as well as $\epsilon$. We use $r$ to represent the uncertainty of CoM location. This uncertain effect from $r$ does not show up in the force equilibrium equation. Then, we consider the moment equation such as: 
% \begin{equation}
% Q_{Ax}f_{tA} - Q_{Ay}f_{nA} + (Q_{Cx} + r)(mg+\epsilon) + Q_{Px} f_{Py} - Q_{Py}f_{Px}  =0,\label{momenteq10}
% \end{equation}
% It is worth pointing out that the term $(Q_{Cx} + r)(mg+\epsilon)$ results in non-convex term $r\epsilon$. Then you get:
% \begin{equation}
% f_{nA} = \frac{-(Q_{Cx}+ r)\left(mg + \epsilon\right) -Q_{Px}f_{Py} + Q_{Py}f_{Px}}{\mu_A Q_{Ax} - Q_{Ay}},
% \label{fna}
% \end{equation}
% Here, using $f_{tB} = -\mu_B f_{nB}$ and \eq{forceeq10}, 
% \begin{equation}
% f_{nB} = \frac{\mu_Af_{Px} - f_{Py} - mg - \epsilon}{1 + \mu_A \mu_B},
% \label{fnb}
% \end{equation}
% And the conditions for $\epsilon, r$ would be:
% \begin{subequations}
% \begin{flalign}
% -Q_{Px}f_{Py} + Q_{Py}f_{Px} < (Q_{Cx} + r)(mg + \epsilon),\label{fcondition1}\\
% \epsilon < \mu_Af_{Px}-f_{Py} - mg,  \label{fcondition2}
% \end{flalign}
% \label{force_condition}
% \end{subequations}


\section{Robust Trajectory Optimization}\label{sec:robust_to}
% also want to say that we denote friction stability as pure function of \epsilon, making inner loop of bilevel optimization linear
%\color[red]{This section can be expanded and re-written to explain everything properly.}\\
% \devesh{This section can be expanded and explained in detail. May be consider and explain the case of patch contact as well as uncertainty due to friction. Also include the formulation for mode-based optimization. The uncertainty due to friction might be a bit different treatment than earlier. The other option to consider is to include recovery from failure using control.}\\
\begin{figure}[t]
    \centering
\includegraphics[width=0.49\textwidth]{Figures/drawing_cropped.pdf} % 
    \caption{Conceptual schematic of our proposed frictional stability and robust trajectory optimization for pivoting. Due to slipping contact, friction forces at points $A, B$ lie on the edge of friction cone. Given the nominal trajectory of state and control inputs, friction forces can account for uncertain physical parameters to satisfy \revise{quasi-static} equilibrium. We define the range of disturbances that can be compensated by contacts as frictional stability. The above figure shows the case of uncertain mass and CoM location.}
    \label{fig:concept}
\end{figure}
% concept.eps
% In our proposed optimization formulation, we maximize the worst frictional stability margin over the entire trajectory where we obtain the maximum frictional stability at each time step given $x, u$, leading to a bilevel optimization problem.
Using the notion of \textit{frictional stability} introduced in the previous section, we describe our proposed contact implicit bilevel optimization (CIBO) method for robust optimization of manipulation trajectories. The proposed method explicitly considers frictional stability under uncertain physical parameters. It is noted that the proposed method considers robustness under slipping contact which results in equality for friction cone constraints (see \fig{fig:concept}).
After describing the formulation for convex objects, we also describe how to extend the proposed CIBO to consider objects with non-convex geometry. Our proposed method is also presented as a schematic in \fig{fig:concept}. As shown in \fig{fig:concept}, the proposed CIBO considers frictional stability margin along the entire trajectory for manipulation and then maximizes the minimum margin in the proposed framework. This is also explained in \fig{fig:cibo}, where we show that we estimate the bound of stability margin in the lower level optimization and maximize the minimum margin in the upper level optimization. Before introducing our proposed bilevel optimization, we present a baseline contact-implicit TO which can be formulated as an MPCC. 

%\textcolor{red}{equality constraints}

%non-convex shape of objects and patch contact in this section.

\subsection{Contact-Implicit Trajectory Optimization for Pivoting}

% Here we first show our general formulation of optimization with uncertainty and show its robust counterpart. we first describe the optimization problem for trajectory generation during pivoting.
The purpose of our optimal control is to 
\revise{find optimal control input sequences under constraints for pivoting manipulation. In particular, we consider the objective function for achieving the minimum motion of objects under kinematics constrains, \revise{quasi-static} equilibrium, friction cone constraints, and sticking-slipping complementarity constraints as follows:}
% The purpose of our optimal control is to regulate the contact state and object state simultaneously given by:
% The purpose of our optimal control is to regulate the contact state and object state simultaneously. Our optimal control problem is:
% 
\begin{subequations}
\begin{flalign}
% \min _{x, u, \lambda}\sum_{k=0}^{N-1} \phi(x_k, u_k, \lambda_k)\\
\min _{x, u, f}  \sum_{k=1}^{N} ({x}_{k} - x_g)^{\top} Q ({x}_{k} - x_g)+\sum_{k=0}^{N-1}u_{k}^{\top} R u_{k} \\
\text{s. t. } i_{k, x}, i_{k, y} \in FK(\theta_k, \revise{P}_{k, y}^{\revise{O}}), \eq{force_eq}, \eq{slipping_friction_cone}, \eq{slippingP},  \label{const2}\\
% \eq{force_eq}, \eq{slipping_friction_cone}, \eq{slippingP}, \label{const3}\\
% \sum_{c=1}^{C} f_{k, c} + mg  = 0\\
% \sum_{c=1}^{C} \left(p_{k, c} - q_{k}\right) \times \lambda_{k, c}  = 0\\
x_{0} = x_s, x_{N} = x_g,
x_{k} \in \mathcal{X}, u_{k} \in \mathcal{U}, 0\leq f_{k, ni} \leq f_{u} \label{bounds_variables}
% f_{k, c, n} \geq 0 ,
%  0 \leq   \dot{p}_{k, c, j+} \perp \mu_c \lambda_{{k,c, n}}-\lambda_{k, c, t} \geq 0  \\
%  0 \leq   \dot{p}_{k, c, j-} \perp \mu_c \lambda_{{k,c, n}}+\lambda_{k, c, t} \geq 0 
\end{flalign}
\label{equation_control}
\end{subequations}
where $x_k = [\theta_k, \revise{P}_{k, y}^{\revise{O}}, \dot{\theta}_k, \revise{\dot{P}}_{k, y}^{\revise{O}}]^\top$, $u_k=[f_{k, nP}, f_{k, tP}]^\top$, $f_k = [f_{k, nA},f_{k, nB}]^\top$, $Q=Q^{\top} \geq 0,R=R^{\top} > 0$.
\revise{The input of \eq{equation_control} consists of physical parameters such as mass, length, and width of the object and the optimization parameters such as $Q$ and $R$. The output of  \eq{equation_control} consists of trajectories of $x_k, u_k, f_k, \forall k \in \{0, 1, \ldots, N\}$. }
We use explicit Euler to discretize the dynamics with sample time $\Delta$. The function $FK$ represents forward kinematics to specify each contact point $i$ and CoM location. $\mathcal{X}$ and $\mathcal{U}$ are convex polytopes, consisting of a finite number of linear inequality constraints.  $f_u$ is an upper-bound of normal force at each contact point. Note that we impose \eq{force_eq}, \eq{slipping_friction_cone} at each time step $k$. $x_s, x_g$ are the states at $k=0,k=N$, respectively.
%  Index $k, c, j$ represent time-step, contact point, appropriate slipping direction, respectively. $x_k$ is the decision variable of states including $q_k$, which is the $x, y, \theta$. $u_k = [\lambda_{k,p,n}, \lambda_{k,p,t}]^\top$ where $\lambda_{k,p,n}, \lambda_{k,p,t}$ are the normal and shear forces at the point P (the point where a robot touches). $\lambda_k$ has the rest of other friction forces at other contact points. 

% In this work, we work on the problem where uncertainty exists in $m, p_{k, c}, \mu_c $. The robust version of \eq{equation_control} can be formulated easily except for equality constraints. 

% \begin{enumerate}
% \item how to realize the control with state and contact force simultaneously? use additional cost?
% \item robust equality constraints
% \end{enumerate}
% The question is eventually what we want to do.

%of Pivoting Considering Frictional Stability
\subsection{Robust CIBO}\label{bilevel_sec}


\begin{figure}
    \centering    \includegraphics[width=0.49\textwidth]{Figures/cibo.png} % 
    \caption{This figure illustrates the idea of the proposed contact implicit bilevel optimization, CIBO. Given the trajectory of $x, u, f$, the stability margin over the trajectory can be computed as shown in lower-level optimization problem. Then, given the computed stability margin over the trajectory $\epsilon$, the upper-level optimization problem maximizes the worst-case stability margin over the trajectory by optimizing the trajectory of $x, u, f$. Our CIBO simultaneously optimizes the lower-level optimization problem and the upper-level optimization problem.
    In the right plot, red and blue arrows represent the stability margin along positive and negative directions, respectively. Our CIBO optimizes the stability margin for each direction. 
    }
    \label{fig:cibo}
\end{figure}


% As described in Sec~\ref{sec:mechanics}, considering frictional stability is critical for robust manipulation. % To robustify the trajectory optimization in \eq{equation_control}, we need to incorporate frictional stability in \eq{equation_control}. 
In this section, we present our formulation where we incorporate frictional stability in trajectory optimization to obtain robustness.
In particular, we first focus on discussing the optimization problem with uncertain mass, CoM location, and \revise{finger contact location}. We later discuss the optimization problem of uncertain coefficient of friction in Sec~\ref{subsec:opt_friction}.

An important point to note is that the optimization problem would be ill-posed if we naively add \eq{force_eq_mass}, \eq{force_eq_location}, and/or \revise{\eq{force_eq_location_finger_margin}} to \eq{equation_control} since there is no $u$ to satisfy all uncertainty realization in equality constraints \cite{yalmip_2018}.  
Therefore, our strategy is that we plan to find an optimal nominal trajectory that can ensure external contacts under uncertain physical parameters. 
% In particular, we here focus on discussing uncertain mass or CoM location uncertainty and we later discuss the optimization problem of uncertain coefficient of friction in Sec~\ref{subsec:opt_friction}.
In other words, we aim at maximizing the worst-case stability margin over the trajectory given the maximal frictional stability at each time-step $k$ (also shown in \fig{fig:concept}). Thus,  we maximize the following objective function:
% 
\begin{equation}
\min _{k} \epsilon_{k, +}^* - \max _{k} -\epsilon_{k, -}^*
\label{bilevel_obj}
\end{equation}
% 
where $\epsilon_{k, +}^*, \epsilon_{k, -}^*$ are non-negative variables. Note that $\epsilon_{k, +}^*, \epsilon_{k, -}^*$ are the largest uncertainty in the positive and negative direction, respectively, at instant $k$ given $x, u, f$, which results in non-zero contact forces (i.e., stability margin, see also \fig{fig:concept}).
% 
% that represents the magnitude of friction force (i.e., frictional stability margin) under $\epsilon_{k, +}^*, \epsilon_{k, -}^*$ along positive and negative directions of uncertainty at instant $k$ given $x, u, f$, respectively (see also Figure~\ref{fig:concept}).
% 
% the maximum stability margin along positive and negative direction of uncertainty at $k$ given $x, u, f$, respectively. 
\eq{bilevel_obj} calculates the smallest stability margin over time-horizons by subtracting the stability margin along the positive direction from that along the negative direction. 
Hence, we formulate a bilevel optimization problem which consists of two lower-level optimization problems as follows (see also \fig{fig:cibo}):
% 
% % we consider the following optimization problem to obtain the stability margin:
% \begin{subequations}
% \begin{flalign}
% \max_{x, u, f, \epsilon^*} \min _{k} \|\epsilon_k^*\|  \ \\
% \text{s. t. } \quad \text{\eq{const2}, \eq{const3}, \eq{bounds_variables}} \\
% \epsilon_k^* \in \argmax_{\epsilon} \{\epsilon_k^2:\text{\eq{fna_cond_mass_one}, \eq{fnb_cond_mass}} \} \label{const_bilevel_1}
% %  \text{s. t. } \eq{fna_cond_mass_one}, \eq{fnb_cond_mass}\label{force_eq_inner}
% %  \sum_{c=1}^{C} p_{k, c} \times \lambda_{k, c} + p_{k, u} \times R\left(x_{k, o}\right) u_k + r_{k, g} \times (\bar{m}g + \epsilon_k)  = 0\\
% % \lambda_{k, c, n} \geq 0 ,\\
% % \mu_c \lambda_{{k,c, n}}-\lambda_{k, c, t} = 0\label{friction_eq_inner}
% % \mu_c \lambda_{{k,c, n}}+\lambda_{k, c, t} = 0 
% %   0 \leq   \dot{p}_{k, c, j+} \perp \mu_c \lambda_{{k,c, n}}-\lambda_{k, c, t} \geq 0  \\
% %  0 \leq   \dot{p}_{k, c, j-} \perp \mu_c \lambda_{{k,c, n}}+\lambda_{k, c, t} \geq 0 
% \end{flalign}
% \label{equation_sm}
% \end{subequations}
% % \begin{subequations}
% % \begin{flalign}
% % \max_{x, u, \lambda} \min _{k=0, \ldots, N-1} \epsilon_k^\top Q \epsilon_k + \phi_k^\top R \phi_k  \ \\
% % \text{s. t. } \sum_{c=1}^{C}\lambda_{k, c} + \bar{m}g + \epsilon_k = 0, \\
% % \sum_{c=1}^{C} \left(p_{k, c} - \left(q_{k} - \phi_k\right)\right) \times \lambda_{k, c}  = 0\\
% % \quad x_{0} = x_s, 
% % x_{k} \in \mathcal{X}, u_{k} \in \mathcal{U}, \lambda_k \leq \lambda_{u}, \\
% % \lambda_{k, c, n} \geq 0 ,\\
% % \mu_c \lambda_{{k,c, n}}-\lambda_{k, c, t} = 0
% % % \mu_c \lambda_{{k,c, n}}+\lambda_{k, c, t} = 0 
% % %   0 \leq   \dot{p}_{k, c, j+} \perp \mu_c \lambda_{{k,c, n}}-\lambda_{k, c, t} \geq 0  \\
% % %  0 \leq   \dot{p}_{k, c, j-} \perp \mu_c \lambda_{{k,c, n}}+\lambda_{k, c, t} \geq 0 
% % \end{flalign}
% % \label{equation_sm}
% % \end{subequations}
% where \eq{fna_cond_mass_one} and \eq{fnb_cond_mass} represent the lower-level constraints and  \eq{const2}, \eq{const3}, \eq{bounds_variables} represent the upper-level constraints. $\epsilon_k^2$, $\min _{k} \epsilon_k^*$ are the lower- and the upper-level objective function, respectively. $\epsilon$ is the lower-level decision variable and $x, u, f, \epsilon^*$ are the upper-level decision variables. \eq{equation_sm} finds an optimal nominal trajectory that maximize the worst frictional stability over time-horizon given the best frictional stability at each time-step $k$.
% 
% Unfortunately, solving the bilevel optimization problem \eq{equation_sm} is difficult since the lower-level optimization problem (i.e., maximization of convex function) is non-convex. Instead, we consider the following bilevel optimization problem which effectively solves two lower-level optimization problems:
\begin{subequations}
\begin{flalign}
\max_{x, u, f, \epsilon_+^*, \epsilon_-^*} (\min _{k} \epsilon_{k, +}^* - \max _{k} -\epsilon_{k, -}^*)  \ \\
% \max_{x, u, f, \epsilon_+^*, \epsilon_-^*} (\min _{k} \epsilon_{k, +}^* + q\min _{k} \epsilon_{k, -}^* )  \ \\
\text{s. t. } \quad \text{\eq{const2}, \eq{bounds_variables}}, \\
\epsilon_{k, +}^* \in \argmax_{\epsilon_{k, +}} \{\epsilon_{k, +}: A_k\epsilon_{k, +} \leq b_k , \epsilon_{k, +} \geq 0 \}, \label{bi-const1} \\
\epsilon_{k, -}^* \in \argmax_{\epsilon_{k, -}} \{\epsilon_{k, -}: -A_k\epsilon_{k, -} \leq b_k , \epsilon_{k, -} \geq 0 \}
%  \text{s. t. } \eq{fna_cond_mass_one}, \eq{fnb_cond_mass}\label{force_eq_inner}
%  \sum_{c=1}^{C} p_{k, c} \times \lambda_{k, c} + p_{k, u} \times R\left(x_{k, o}\right) u_k + r_{k, g} \times (\bar{m}g + \epsilon_k)  = 0\\
% \lambda_{k, c, n} \geq 0 ,\\
% \mu_c \lambda_{{k,c, n}}-\lambda_{k, c, t} = 0\label{friction_eq_inner}
% \mu_c \lambda_{{k,c, n}}+\lambda_{k, c, t} = 0 
%   0 \leq   \dot{p}_{k, c, j+} \perp \mu_c \lambda_{{k,c, n}}-\lambda_{k, c, t} \geq 0  \\
%  0 \leq   \dot{p}_{k, c, j-} \perp \mu_c \lambda_{{k,c, n}}+\lambda_{k, c, t} \geq 0 
\end{flalign}
\label{equation_sm_1}
\end{subequations}
% where \eq{fna_cond_mass_one}, \eq{fnb_cond_mass} represent the lower-level constraints and $\epsilon_k^2$ represent the lower level objective function. $\epsilon_k$ is the lower-level decision variable. The upper-level constraints are \eq{const2}, \eq{const3}, \eq{bounds_variables} and the upper-level objective function is $\min _{k} \epsilon_k^*$. The upper-level 
% where $k$ is the time-step, $x_k \in\mathbb{R}^{n_{x}}$ is the state of the object. For planar pivoting, we consider $x_k = [x_{k, p}, x_{k, o}]^\top, x_{k, p} \in\mathbb{R}^{2}, x_{k, o} \in\mathbb{R}^{1}$, where $x_{k, p}$ is the position of the object and $x_{k, o}$ is the orientation of the object. $R\left(x_{k, o}\right)$ is the rotation matrix from zero to $x_{k, o}$.  $u_k \in\mathbb{R}^{n_{u}}$ is the control, which is the contact force from a robot finger. $\lambda_k \in\mathbb{R}^{n_{\lambda}}$ is the contact forces from external contact points.  $C$ represents the number of external contact points except for a contact point by a robot finger. For example, in \fig{fig:mechanics_pivoting_eq}, $C=2$ because we have external contact points at A and B. $\epsilon_k \in\mathbb{R}^{n_{x}}$ represents the error of gravity.  For our planar pivoting, $\epsilon_k = [0, \epsilon_{k, y}]^\top$. $\mu_c$ is the coefficient of friction at point $c$. $\lambda_{{k,c, n}}, \lambda_{{k,c, t}}$ represent the normal and sheer forces respectively at point $C$ at $k$. $p_{k, c}$ is nonlinear function with respect to $x_{k, o}$ and it represents the distance from the origin to each contact point. $p_{k, u}$ represents the distance from the origin to control contact point. 
% $r_{k, g}$ is the distance from the origin to a gravity vector.
% Note that from equation 11d to 11g, they are constraints for an inner optimization problem.
where $A_k \in\mathbb{R}^{2 \times 1}, b_k \in\mathbb{R}^{2 \times1}$ represent inequality constraints in \eq{fna_cond_mass_one} and \eq{fnb_cond_mass} \revise{or \eq{eq:uncertain_mass_slope} and  \eq{eq:uncertain_mass_slope_upper_bound} if the object is on a slope.} 
 $A_k\epsilon_{k, +} \leq b_k , \epsilon_{k, +} \geq 0,$ and $-A_k\epsilon_{k, -} \leq b_k , \epsilon_{k, -} \geq 0$ represent the lower-level constraints for each lower-level optimization problem while \eq{const2}, \eq{bounds_variables} represent the upper-level constraints. $\epsilon_+, \epsilon_-$ are the lower-level objective functions while $\min _{k} \epsilon_{k, +}^* - \max _{k} -\epsilon_{k, -}^* $ is the upper-level objective function. $\epsilon_{k, +}, \epsilon_{k, -}$ are the lower-level decision variables of each lower-level optimization problem while $x, u, f, \epsilon_+^*, \epsilon_-^*$ are the upper-level decision variables. 

% Note that $A_k, b_k$ are nonlinear function with respect to $x, u, f$.  

\eq{equation_sm_1} considers the largest one-side frictional stability margin along positive and negative direction at $k$. Therefore, by solving these two lower-level optimization problems, we are able to obtain the maximum frictional stability margin along positive and negative direction. 
% We initially plan to solve the lower-level optimization problem which objective function is $\epsilon_k^2$ to obtain the largest magnitude of $\epsilon_k$ but this objective function makes the lower-level optimization non-convex, which is difficult to solve.
% In contrast, 
The advantage of \eq{equation_sm_1} is that since the lower-level optimization problem are formulated as two linear programming problems, we can efficiently solve the entire bilevel optimization problem using the Karush-Kuhn-Tucker (KKT) condition as follows:
\begin{subequations}
\begin{flalign}
 w_{k, +, j}, w_{k, -, j} \geq 0, C_k\epsilon_{k, +} \leq d_k , E_k\epsilon_{k, -} \leq d_k,\\
w_{k, +, j}(C_k\epsilon_{k, +} - d_k)_j = 0, \\
w_{k, -, j}(E_k\epsilon_{k, -} - d_k)_j = 0, \\
% \epsilon_{k, +}^* \in \argmax_{\epsilon_+} \{\epsilon_{k, +}: A_k\epsilon_{k, +} \leq b_k , \epsilon_{k, +} \geq 0 \}, \\
% \epsilon_{k, -}^* \in \argmax_{\epsilon_-} \{\epsilon_{k, -}: -A_k\epsilon_{k, -} \leq b_k , \epsilon_{k, -} \geq 0 \}, \\
\nabla (-\epsilon_{k, +}) + \sum_{j=1}^{3}w_{k, +, j} \nabla (C_k\epsilon_{k, +} - d_k)_j= 0,\\
\nabla (-\epsilon_{k, -}) + \sum_{j=1}^{3}w_{k, -, j} \nabla (E_k\epsilon_{k, -} - d_k)_j= 0
% t_+ \leq \epsilon_{k, +}, t_- \leq \epsilon_{k, -}, \forall k
%  \text{s. t. } \eq{fna_cond_mass_one}, \eq{fnb_cond_mass}\label{force_eq_inner}
%  \sum_{c=1}^{C} p_{k, c} \times \lambda_{k, c} + p_{k, u} \times R\left(x_{k, o}\right) u_k + r_{k, g} \times (\bar{m}g + \epsilon_k)  = 0\\
% \lambda_{k, c, n} \geq 0 ,\\
% \mu_c \lambda_{{k,c, n}}-\lambda_{k, c, t} = 0\label{friction_eq_inner}
% \mu_c \lambda_{{k,c, n}}+\lambda_{k, c, t} = 0 
%   0 \leq   \dot{p}_{k, c, j+} \perp \mu_c \lambda_{{k,c, n}}-\lambda_{k, c, t} \geq 0  \\
%  0 \leq   \dot{p}_{k, c, j-} \perp \mu_c \lambda_{{k,c, n}}+\lambda_{k, c, t} \geq 0 
\end{flalign}
\label{kkt_equations}
\end{subequations}
where $C_k = [A_k^\top, -1]^\top \in\mathbb{R}^{3 \times 1}, d_k = [b_k^\top, 0]^\top \in\mathbb{R}^{3\times 1}, E_k = [-A_k^\top, -1]^\top \in\mathbb{R}^{3 \times 1}$.
$w_{k, +, j}$ is Lagrange multiplier associated with  $(C_k\epsilon_{k, +} \leq d_k)_j$, where $(C_k\epsilon_{k, +} \leq d_k)_j$ represents the $j$-th inequality constraints in $C_k\epsilon_{k, +} \leq d_k$. $w_{k, -, j}$ is Lagrange multiplier associated with  $(E_k\epsilon_{k, -} \leq d_k)_j$. 
% 
Using the KKT condition and epigraph trick, we eventually obtain a single-level large-scale nonlinear programming problem with complementarity constraints: 
% where $\epsilon_+^*, \epsilon_-^*$ are introduced instead of $\epsilon^*$ so \eq{equation_sm_1} is able to solve lins
\begin{subequations}
\begin{flalign}
\max_{x, u, f, \epsilon_+^*, \epsilon_-^*} (t_+ + \alpha t_- ) \label{cost_bilevel}  \ \\
\text{s. t. } \quad \text{\eq{const2}, \eq{bounds_variables}, \eq{kkt_equations}},\\
% w_{k, +, j}, w_{k, -, j} \geq 0\\
% C_k\epsilon_{k, +} \leq d_k , E_k\epsilon_{k, -} \leq d_k,\\
% w_{k, +, j}(C_k\epsilon_{k, +} - d_k)_j = 0, \\
% w_{k, -, j}(E_k\epsilon_{k, -} - d_k)_j = 0, \\
% % \epsilon_{k, +}^* \in \argmax_{\epsilon_+} \{\epsilon_{k, +}: A_k\epsilon_{k, +} \leq b_k , \epsilon_{k, +} \geq 0 \}, \\
% % \epsilon_{k, -}^* \in \argmax_{\epsilon_-} \{\epsilon_{k, -}: -A_k\epsilon_{k, -} \leq b_k , \epsilon_{k, -} \geq 0 \}, \\
% \nabla (-\epsilon_{k, +}) + \sum_{j=1}^{3}w_{k, +, j} \nabla (C_k\epsilon_{k, +} - d_k)_j= 0,\\
% \nabla (-\epsilon_{k, -}) + \sum_{j=1}^{3}w_{k, -, j} \nabla (E_k\epsilon_{k, +} - d_k)_j= 0, \\
t_+ \leq \epsilon_{k, +}, t_- \leq \epsilon_{k, -}, \forall k
%  \text{s. t. } \eq{fna_cond_mass_one}, \eq{fnb_cond_mass}\label{force_eq_inner}
%  \sum_{c=1}^{C} p_{k, c} \times \lambda_{k, c} + p_{k, u} \times R\left(x_{k, o}\right) u_k + r_{k, g} \times (\bar{m}g + \epsilon_k)  = 0\\
% \lambda_{k, c, n} \geq 0 ,\\
% \mu_c \lambda_{{k,c, n}}-\lambda_{k, c, t} = 0\label{friction_eq_inner}
% \mu_c \lambda_{{k,c, n}}+\lambda_{k, c, t} = 0 
%   0 \leq   \dot{p}_{k, c, j+} \perp \mu_c \lambda_{{k,c, n}}-\lambda_{k, c, t} \geq 0  \\
%  0 \leq   \dot{p}_{k, c, j-} \perp \mu_c \lambda_{{k,c, n}}+\lambda_{k, c, t} \geq 0 
\end{flalign}
\label{kkt_convertion}
\end{subequations}
% where $C_k \in\mathbb{R}^{3 \times 1}, d_k \in\mathbb{R}^{3}$ represent inequality constraints associated with $\epsilon_{k, +}$.  $E_k \in\mathbb{R}^{3 \times 1}, d_k \in\mathbb{R}^{3}$ represent inequality constraints associated with $\epsilon_{k, -}$.
where $\alpha$ is a weighting scalar. 
Note that we derive \eq{kkt_convertion} for the case with an uncertain mass parameter but this formulation can be easily converted to the case where uncertainty exists in CoM location by replacing $A_k, b_k$ in \eq{equation_sm_1} with \eq{fna_fnb_r}.
\revise{Similarly, we can consider uncertainty in finger contact location by replacing $A_k, b_k$ in \eq{equation_sm_1} with \eq{force_eq_location_finger_margin}.}
% with $\epsilon_k^* \in \argmax_{\epsilon} \{\epsilon_k^2:\text{\eq{fna_fnb_r}} \} $ in \eq{equation_sm}. 
Therefore, by solving tractable \eq{kkt_convertion}, we can efficiently generate robust trajectories that are robust against uncertain mass, CoM location, \revise{and contact location} parameters. 

% \textit{Remark 1}:
% In practice, we can add $\min _{x, u, f} \sum_{k=0}^{N-1} ({x}_{k} - x_g)^{\top} Q ({x}_{k} - x_g)+u_{k}^{\top} R u_{k}$ to \eq{cost_bilevel} to realize a smooth trajectory and avoid jerky control inputs. 

\textit{Remark \revise{2}}:
If we consider the case where uncertainty exists in both mass and CoM location simultaneously, we would have a nonlinear coupling term $(C_x+r)(mg + \epsilon)$ in \revise{quasi-static} equilibrium of moment. This makes the lower-level optimization non-convex optimization, making it extremely challenging to solve during bilevel optimization.
Once the lower-level optimization becomes a non-convex optimization problem, there is no guarantee that the lower-level optimization finds globally optimal solutions, resulting in finding a very sub-optimal controller.
% 
\revise{Similarly, all of the constraints (e.g., considering sticking-slipping contact at point contact $A$ and $B$ requires complementarity constraints) which results in non-convex constraints cannot be handled in our CIBO.}
% Thus, it is left as a future work.

% \epsilon_k^* \in \argmax_{\epsilon} \{\epsilon_k^2:\text{\eq{fna_cond_mass_one}, \eq{fnb_cond_mass}} \} \label{const_bilevel_1}

% Problem \eq{equation_sm} is the bilevel optimization. Here, the decision variables of the inner optimization is $\lambda, \epsilon$ and all constraints \eq{force_eq_inner}-\eq{friction_eq_inner} are linear constraints with respect to $\lambda, \epsilon$. For simplicity, here we assume that $\epsilon_{k, y} \geq 0$. Here, to simplify the notation, we represent Problem \eq{equation_sm} as:
% \begin{subequations}
% \begin{flalign}
% \max_{x, u, \lambda^*, \epsilon^*} (\min _{k} \epsilon_k^*)   \\
% \text{s. t. } \quad x_{0} = x_s, 
% x_{k} \in \mathcal{X}, u_{k} \in \mathcal{U}, \\
% \lambda_k^*, \epsilon_k^* = \argmax_{\lambda, \epsilon} e_k ^\top z_k \\
%  \text{s. t. } A_k z_k + b_k = 0\label{equality_eq_inner}\\
% C_k z_k + d_k \leq 0\label{inequality_eq_inner}
% % \mu_c \lambda_{{k,c, n}}+\lambda_{k, c, t} = 0 
% %   0 \leq   \dot{p}_{k, c, j+} \perp \mu_c \lambda_{{k,c, n}}-\lambda_{k, c, t} \geq 0  \\
% %  0 \leq   \dot{p}_{k, c, j-} \perp \mu_c \lambda_{{k,c, n}}+\lambda_{k, c, t} \geq 0 
% \end{flalign}
% \label{equation_bilevel}
% \end{subequations}
% where $z_k = [\lambda_k, \epsilon_k]^\top \in\mathbb{R}^{n_{\lambda}+1}$. We use \eq{equality_eq_inner} to represent all equality constraints in the inner optimization problem and \eq{inequality_eq_inner} to represent all inequality constraints in the inner optimization problem. $A_k \in\mathbb{R}^{q \times (n_{\lambda}+1)}, b_k \in\mathbb{R}^{q}, C_k \in\mathbb{R}^{p \times (n_{\lambda}+1)}, d_k \in\mathbb{R}^{p}, e_k \in\mathbb{R}^{n_\lambda + 1}$. Note that $A_k,  C_k$ are nonlinear function with respect to $x, u$. 


% Thus, we can reformulate the inner loop optimization problem using KKT condition and the reformulated optimization would be:
% \begin{subequations}
% \begin{flalign}
% \max_{x, u, \lambda^*, \epsilon^*} \min _{k} \epsilon_k^*  \ \\
% \text{s. t. } \quad x_{0} = x_s, 
% x_{k} \in \mathcal{X}, u_{k} \in \mathcal{U}, \\
% % \lambda_k^*, \epsilon_k^* = \argmax_{\lambda, \epsilon} e_k ^\top z_k \\
%  A_k z_k + b_k = 0\label{test1}\\
% C_k z_k + d_k \leq 0\label{test2}\\
% w_k^\top \geq 0\\
% w_{k, i}(C_k z_k + d_k)_i = 0, i = 1, \ldots, p\\
% \nabla (e_k^\top z_k) +\sum_{i=1}^{p}w_{k, i} \nabla (C_k z_k + d_k)_i + \sum_{i=1}^{q}v_{k, i}\nabla ( A_k z_k + b_k) = 0
% % e_k + \sum_{i=1}^{p}w_{k, i}C_{k, i}^\top + \sum_{i=1}^{q}v_{k, i}A_{k, i}^\top = 0
% % \mu_c \lambda_{{k,c, n}}+\lambda_{k, c, t} = 0 
% %   0 \leq   \dot{p}_{k, c, j+} \perp \mu_c \lambda_{{k,c, n}}-\lambda_{k, c, t} \geq 0  \\
% %  0 \leq   \dot{p}_{k, c, j-} \perp \mu_c \lambda_{{k,c, n}}+\lambda_{k, c, t} \geq 0 
% \end{flalign}
% \label{bilevel_opt1}
% \end{subequations}
% % , $\phi_k \in\mathbb{R}^{n_{x}}$ represents the error of the CoM location. 
% where $w_{k, i} v_{k, i}$ are dual variables associated with equality constraints and inequality constraints, respectively. We calculate deriviatives with respect to $z_k$. We have $p$ inequality constraints and $q$ equality constraints. 

% Using epigraph trick, we can reformulate the problem as follows:
% \begin{subequations}
% \begin{flalign}
% \max_{x, u, \lambda^*, \epsilon^*} t  \ \\
% \text{s. t. } \quad x_{0} = x_s, 
% x_{k} \in \mathcal{X}, u_{k} \in \mathcal{U}, \\
% % \lambda_k^*, \epsilon_k^* = \argmax_{\lambda, \epsilon} e_k ^\top z_k \\
%  A_k z_k + b_k = 0\label{test1}\\
% C_k z_k + d_k \leq 0\label{test2}\\
% w_k^\top \geq 0\\
% w_{k, i}(C_k z_k + d_k)_i = 0, i = 1, \ldots, p\\
% \nabla (e_k^\top z_k) +\sum_{i=1}^{p}w_{k, i} \nabla (C_k z_k + d_k)_i + \sum_{i=1}^{q}v_{k, i}\nabla ( A_k z_k + b_k) = 0\\
% t \leq \epsilon_k^* \forall k
% % e_k + \sum_{i=1}^{p}w_{k, i}C_{k, i}^\top + \sum_{i=1}^{q}v_{k, i}A_{k, i}^\top = 0
% % \mu_c \lambda_{{k,c, n}}+\lambda_{k, c, t} = 0 
% %   0 \leq   \dot{p}_{k, c, j+} \perp \mu_c \lambda_{{k,c, n}}-\lambda_{k, c, t} \geq 0  \\
% %  0 \leq   \dot{p}_{k, c, j-} \perp \mu_c \lambda_{{k,c, n}}+\lambda_{k, c, t} \geq 0 
% \end{flalign}
% \label{bilevel_opt2}
% \end{subequations}
% % We can solve the problem in several ways. First, we can use epigraph trick by introducing a new scalar variable $z$ as follows:
% % \begin{subequations}
% % \begin{flalign}
% % \max_{x, u, \lambda} z  \ \\
% % \text{s. t. } \sum_{c=1}^{C}\lambda_{k, c} + \bar{m}g + \epsilon_k = 0, \\
% % \sum_{c=1}^{C} \left(p_{k, c} - \left(q_{k} - \phi_k\right)\right) \times \lambda_{k, c}  = 0\\
% % \quad x_{0} = x_s, 
% % x_{k} \in \mathcal{X}, u_{k} \in \mathcal{U}, \lambda_k \leq \lambda_{u}, \\
% % \lambda_{k, c, n} \geq 0 ,\\
% % \mu_c \lambda_{{k,c, n}}-\lambda_{k, c, t} = 0. \\
% % z \leq \epsilon_k^\top Q \epsilon_k + \phi_k^\top R \phi_k,  \forall	k
% % % \mu_c \lambda_{{k,c, n}}+\lambda_{k, c, t} = 0 
% % %   0 \leq   \dot{p}_{k, c, j+} \perp \mu_c \lambda_{{k,c, n}}-\lambda_{k, c, t} \geq 0  \\
% % %  0 \leq   \dot{p}_{k, c, j-} \perp \mu_c \lambda_{{k,c, n}}+\lambda_{k, c, t} \geq 0 
% % \end{flalign}
% % \label{equation_sm}
% % \end{subequations}


\subsection{Robust CIBO for Frictional Uncertainty}
\label{subsec:opt_friction}
We consider the case where the system has uncertainty in the friction coefficients at $A$ and $B$ as discussed in Sec~\ref{subsec:stochasticfriction_planning}. In order to design a robust open-loop controller for the system,  we can use the similar formulation presented in Sec~\ref{bilevel_sec}.
The proposed formulation aims at maximizing the stability margin from stochastic friction. In particular, to avoid non-convex optimization as the lower-level optimization problem, we consider the stability margin along positive and negative direction for both $\epsilon_A$ and $\epsilon_B$, as we discuss in Sec~\ref{bilevel_sec}. By borrowing the optimization problem \eq{equation_sm_1},  the proposed formulation can be seen as follows.
For simplicity, we abbreviate subscript $k$. 
% \begin{subequations}
% \begin{flalign}
% \max_{\Gamma} (\min _{k} \epsilon_{A, +}^* - \max _{k} -\epsilon_{A, -}^* + \min _{k} \epsilon_{B, +}^* - \max _{k} -\epsilon_{B, -}^*)  \ \\
% \text{s. t. } \quad \text{\eq{const2}, \eq{bounds_variables}}, \\
% \epsilon_{A, +}^* \in \argmax_{\epsilon_{A, +}} \{\epsilon_{A, +}:  g(x, u, f, \epsilon_{A, +}, \epsilon_{B}^*)\leq 0 , \nonumber \\ \epsilon_{A, +} \geq 0, \epsilon_{B}^* \in [-\epsilon_{B, -}^*, \epsilon_{B, +}^*]  \}, \label{bi-const1-friction1} \\
% \epsilon_{A, -}^* \in \argmax_{\epsilon_{A, -}} \{\epsilon_{A, -}:  g(x, u, f, -\epsilon_{A, -}, \epsilon_{B}^*)\leq 0 , \nonumber \\ \epsilon_{A, -} \geq 0, \epsilon_{B}^* \in [-\epsilon_{B, -}^*, \epsilon_{B, +}^*]  \}, \label{bi-const1-friction2} \\
% \epsilon_{B, +}^* \in \argmax_{\epsilon_{B, +}} \{\epsilon_{B, +}:  g(x, u, f, \epsilon_{B, +}, \epsilon_{A}^*)\leq 0 , \nonumber \\ \epsilon_{B, +} \geq 0, \epsilon_{A}^* \in [-\epsilon_{A, -}^*, \epsilon_{A, +}^*]  \}, \label{bi-const1-friction3} \\
% \epsilon_{B, -}^* \in \argmax_{\epsilon_{B, -}} \{\epsilon_{B, -}:  g(x, u, f, -\epsilon_{B, -}, \epsilon_{A}^*)\leq 0 , \nonumber \\ \epsilon_{B, -} \geq 0, \epsilon_{A}^* \in [-\epsilon_{A, -}^*, \epsilon_{A, +}^*]  \}, \label{bi-const1-friction4} 
% \end{flalign}
% \label{eq:bilvel_friction}
% \end{subequations}
% Actually, we can instead have the following optimization problem by moving the constraints:  
\begin{subequations}
\begin{flalign}
% \max_{\Gamma} (\min _{k} \epsilon_{A, +}^* - \max _{k} -\epsilon_{A, -}^* + \min _{k} \epsilon_{B, +}^* - \max _{k} -\epsilon_{B, -}^*)  \ \\
\max_{x, u, f, \epsilon_{A,+}^*, \epsilon_{A, -}^*, \epsilon_{B,+}^*, \epsilon_{B, -}^*} \sum_{c\in \mathcal{C}} (\min _{k} \epsilon_{c, +}^* - \max _{k} -\epsilon_{c, -}^*)\\
\text{s. t. } \quad \text{\eq{const2}, \eq{bounds_variables}}, \\
% 
\epsilon_{A}^* \in [-\epsilon_{A, -}^*, \epsilon_{A, +}^*], \epsilon_{B}^* \in [-\epsilon_{B, -}^*, \epsilon_{B, +}^*], \label{eq:stochastic_friction_bilevel_const0} \\
% 
\epsilon_{A, +}^* \in \argmax_{\epsilon_{A, +}} \{\epsilon_{A, +}:  g(x, u, f, \epsilon_{A, +},  \epsilon_{B}^*)\leq 0,  \nonumber \\
\epsilon_{A, +} \geq 0 \}, \label{bi-const1-friction12} \\
% 
\epsilon_{A, -}^* \in \argmax_{\epsilon_{A, -}} \{\epsilon_{A, -}:  g(x, u, f, -\epsilon_{A, -}, \epsilon_{B}^*)\leq 0, \nonumber \\
\epsilon_{A, -} \geq 0,   \}, \label{bi-const1-friction22} \\
% 
\epsilon_{B, +}^* \in \argmax_{\epsilon_{B, +}} \{\epsilon_{B, +}:  g(x, u, f, \epsilon_{B, +}, \epsilon_{A}^*)\leq 0, \nonumber \\ \epsilon_{B, +} \geq 0, \}, \label{bi-const1-friction32} \\
% 
\epsilon_{B, -}^* \in \argmax_{\epsilon_{B, -}} \{\epsilon_{B, -}:  g(x, u, f, -\epsilon_{B, -}, \epsilon_{A}^*)\leq 0, \nonumber \\
\epsilon_{B, -} \geq 0  \}, \label{bi-const1-friction42} 
\end{flalign}
\label{eq:bilvel_friction2}
\end{subequations}
where $g$ summarizes the constraints for each lower-level optimization problem and $\mathcal{C} = \{A, B\}$. 
For each lower-level optimization problem, we consider that another uncertain friction is in the range of optimal stability margin.
For instance, \eq{bi-const1-friction12} is one of the four lower-level optimization problems which aims at maximizing the stability margin under stochastic friction forces at $A$, given stochastic friction force at $B$, $\epsilon_{B}^*$. \eq{eq:stochastic_friction_bilevel_const0}  ensures that $\epsilon_{B}^*$ needs to be within the range of stability margin computed from other two lower-level optimization problems \eq{bi-const1-friction32} and \eq{bi-const1-friction42}.


The resulting optimization introduces many complementarity constraints through the KKT condition because of four lower-level optimization problems, but the resulting computation is still tractable. We discuss computational results in Sec~\ref{subsec::computation}.

\revise{\textit{Remark 3}: In practice, the choice of the particular parameter for the which one should use CIBO to obtain robust trajectories depends on the amount of uncertainty in different parameters associated with the manipulation task. For instance, if we have access to the CAD model of the objects, we can have a good guess of mass and CoM location of the object and thus the major source of uncertainty can be from other parameters such as coefficients of friction.}


\subsection{Robust CIBO for Non-Convex Objects}\label{subsec:mode_based_optimization}
\begin{figure*}[t]
    \centering
    \includegraphics[width=0.8\textwidth]{Figures/mode.png}  
    \caption{A schematic of pivoting for a non-convex shape object where contact set changes over time. During mode 1, the peg rotates with contact at $A$ and $B_2$. During mode 2, the peg rotates with contact at $A$ and $B_1$. 
    $\gamma$ represents one of the kinematic features of peg, which is used to discuss the result in Sec~\ref{fig:openloop_result}. 
    }
    \label{fig:mode_concept}
\end{figure*}


% \devesh{Rephrase to use the formulation in previous subsections}\\
% While our proposed framework in \cite{9811812} is able to design the robust open-loop controller under uncertain physical parameters, it is unable to design the controller for non-convex shape objects such as pegs as shown in \fig{fig:pivoting_abstractfig}. 
The method introduced in the previous subsections assumes convex geometry of the object being manipulated and can not be applied to objects with non-convex geometry (such as pegs as shown in \fig{fig:pivoting_abstractfig}). This is because non-convex objects could result in different contact formations between the object and the environment
 and it is not trivial to identify a feasible contact sequence. In \cite{9811812}, the proposed optimization \eq{kkt_convertion} was solved sequentially for pegs with non-convex geometry. As illustrated in \fig{fig:mode_concept},  we first solve the optimization for a particular contact set (i.e., mode 1 in \fig{fig:mode_concept}) and then solve the optimization for another contact set (i.e., mode 2 in \fig{fig:mode_concept}) given the solution obtained from the first optimization. 
While this method works, it requires extensive domain knowledge. We observed that the second stage optimization can result in infeasible solutions given the solution from the first stage optimization. Thus, we had to carefully specify the parameters of optimization and, in particular, the initial state and terminal state constraints. Such a hierarchical approach has difficulty in finding a feasible solution once the object becomes more complicated. 

To overcome these issues, in general, complementarity constraints can be used to model the change of contact. However, introducing complementarity constraints inside the lower-level optimization makes the lower-level optimization non-convex optimization. Hence, the KKT condition is not a necessary and sufficient condition for optimality but rather a necessary condition. Thus, it is not guaranteed to find globally optimal safety margins over the trajectory. 

In this work, we propose another approach to deal with the non-convexity of the object. Inspired by \cite{9812069}, we formulate the optimization that optimizes the trajectory given mode sequences instead of optimizing mode sequences. It is worth noting that our framework still optimizes the trajectory over the time duration of each mode given the sequence of the mode. Our goal is that the optimization has a larger feasible space so that less domain knowledge is required.%, resulting in less parameter tuning. 





Using the formulation presented in~\cite{9812069}, we present a mode-based formulation for non-convex shaped objects. 
See \cite{9812069} for more details regarding mode-based optimization. For simplicity of exposition, we only present the formulation considering two modes. But one can easily extend this to problems with multiple modes.
% We consider the case where the system has two contact sets and we define the mode as $m = \{\left(l, b\left(l\right)| l\in \mathcal{L}, b\left(l\right) \in \{1, 2\}\right)\}$ where $\mathcal{L}$ is a set of pairs of indicies that defines the complementarity relationship. 
% As shown in FIGURE, we use $m_1$ and $m_2$ to represent mode 1 (i.e., contact on $A$ and $B_1$) and mode 2 (i.e., contact on $A$ and $B_2$), respectively. 
% Thus, we can define each mode as 
For each contact mode, the system has the different constraints. For brevity, we abbreviate the subscript $k$:
\begin{subequations}
\begin{flalign}
% \min _{\tilde{x}, u, f} \sum_{k=0}^{N-1} (\tilde{x}_{k} - x_g)^{\top} Q (\tilde{x}_{k} - x_g)+u_{k}^{\top} R u_{k} + \sum_{l=1}^2 T_l \\
% \text{s. t. } 
i_{x}, i_{y} \in FK_{m}(\theta_k, \revise{P}_{k, y}^{\revise{O}}),  \forall i  \in \{A, B_m\}\\
g_m(f_{nA}, f_{tA}, f_{nB_{1}}, f_{tB_{1}}, f_{nP}, f_{tP}, \revise{P}_{y}^{\revise{O}}) \ \text{if} \ m = 1
\\
g_m(f_{nA}, f_{tA}, f_{nB_{2}}, f_{tB_{2}}, f_{nP}, f_{tP}, \revise{P}_{y}^{\revise{O}}) \ \text{if} \ m = 2 \label{eq:mode2_dynamics_const}
\\
 f_{tA}  =\mu_A f_{nA}, f_{tB_{1}}  =-\mu_{B_1} f_{nB_{1}}, f_{tB_{2}}  =-\mu_{B_2} f_{nB_{2}}
 \\
 \eq{slippingP}, x_{k} \in \mathcal{X}, u_{k} \in \mathcal{U}, 0\leq f_{k, ni} \leq f_{u}
\end{flalign}
\label{eq:mode_const}
\end{subequations}
where $m \in \{1, 2\}$ to represent each contact mode. 
$g_m$ represents the quasi-static model of pivoting manipulation for mode $m$.
It is worth noting that since we decompose the optimization problem into the two mode optimization problem, complementarity constraints are encoded for each mode. %\devesh{Yuki : the above equations are not clear. What is g? have we introduced it earlier?}

 What the optimization problem needs to perform is that for each mode, it only considers the associated constraints and does not consider constraints associated with different contact mode. For example, during mode 1, the optimization should consider only constraints associated with mode 1 and should not consider constraints such as \eq{eq:mode2_dynamics_const}. 
 Another thing the optimization needs to consider is that it needs to scale $\dot{\theta}, \revise{\dot{P}}_{y}^{\revise{O}}$ since we would like to optimize over the time duration. To achieve that, we employ the scaled time variables as discussed in \cite{9812069}. As a result, 
 we recast the quasi-static model by introducing a new state variable with a scaled time, $\tilde{x}_k = \left[\theta_k, \revise{P}_{k, y}^{\revise{O}}, \frac{\dot{\theta}_k}{T}, \frac{\revise{\dot{P}}_{k, y}^{\revise{O}}}{T}\right]^\top$ where $T = T_1$ during mode 1 and $T = T_2$ during mode 2.

 For two contact modes, we can remodel our optimization \eq{equation_control} as follows:
\begin{subequations}
\begin{flalign}
\min _{\tilde{x}, u, f} \sum_{k=0}^{N-1} (\tilde{x}_{k} - x_g)^{\top} Q (\tilde{x}_{k} - x_g)+u_{k}^{\top} R u_{k} + \sum_{l=1}^2 T_l \\
\text{s. t. }  \revise{h_1}(\tilde{x}_k, u_k, f_k) \leq 0, \text{for} \ k\Delta \leq 1 \\
\revise{h_2}(\tilde{x}_k, u_k, f_k) \leq 0, \text{for} \ k\Delta > 1
\label{bounds_variables_mode}
\end{flalign}
\label{eq:mode_change}
\end{subequations}
where $\tilde{x}_k = \left[\theta_k, \revise{P}_{k, y}^{\revise{O}}, \frac{\dot{\theta}_k}{T_1}, \frac{\revise{\dot{P}}_{k, y}^{\revise{O}}}{T_1}\right]^\top$ for $k\Delta \leq 1$ and $\tilde{x}_k = \left[\theta_k, \revise{P}_{k, y}^{\revise{O}}, \frac{\dot{\theta}_k}{T_2}, \frac{\revise{\dot{P}}_{k, y}^{\revise{O}}}{T_2}\right]^\top$ for $k\Delta > 1$.
We use $\revise{h_1}$ and $\revise{h_2}$ to represent all constraints for each mode. 
Given \eq{eq:mode_change}, we can obtain bilevel optimization formulation for non-convex shape objects by following the logic in Sec~\ref{bilevel_sec}.


% In this section, we present optimization formulation for objects with non-convex geometry to obtain robust trajectories as presented in the previous section.

% \subsection{Robust Bilevel Contact-Implicit Trajectory Optimization under Frictional Uncertainty}
% \label{subsec:opt_friction}
% We consider the case where the system has uncertainty in the friction coefficients at $A$ and $B$ as discussed in Sec~\ref{subsec:stochasticfriction_planning}. In order to design a robust open-loop controller for the system,  we can use the similar formulation presented in Sec~\ref{bilevel_sec}.
% As discussed in Sec~\ref{subsec:stochasticfriction_planning}, the proposed formulation aims at maximizing the stability margin from stochastic friction. In particular, to avoid non-convex optimization as the lower-level optimization problem, we consider the stability margin along positive and negative direction for both $\epsilon_A$ and $\epsilon_B$, as we discuss in Sec~\ref{bilevel_sec}. By borrowing the optimization problem \eq{equation_sm_1},  the proposed formulation can be seen as follows.
% For simplicity, we abbreviate subscript $k$. 
% % \begin{subequations}
% % \begin{flalign}
% % \max_{\Gamma} (\min _{k} \epsilon_{A, +}^* - \max _{k} -\epsilon_{A, -}^* + \min _{k} \epsilon_{B, +}^* - \max _{k} -\epsilon_{B, -}^*)  \ \\
% % \text{s. t. } \quad \text{\eq{const2}, \eq{bounds_variables}}, \\
% % \epsilon_{A, +}^* \in \argmax_{\epsilon_{A, +}} \{\epsilon_{A, +}:  g(x, u, f, \epsilon_{A, +}, \epsilon_{B}^*)\leq 0 , \nonumber \\ \epsilon_{A, +} \geq 0, \epsilon_{B}^* \in [-\epsilon_{B, -}^*, \epsilon_{B, +}^*]  \}, \label{bi-const1-friction1} \\
% % \epsilon_{A, -}^* \in \argmax_{\epsilon_{A, -}} \{\epsilon_{A, -}:  g(x, u, f, -\epsilon_{A, -}, \epsilon_{B}^*)\leq 0 , \nonumber \\ \epsilon_{A, -} \geq 0, \epsilon_{B}^* \in [-\epsilon_{B, -}^*, \epsilon_{B, +}^*]  \}, \label{bi-const1-friction2} \\
% % \epsilon_{B, +}^* \in \argmax_{\epsilon_{B, +}} \{\epsilon_{B, +}:  g(x, u, f, \epsilon_{B, +}, \epsilon_{A}^*)\leq 0 , \nonumber \\ \epsilon_{B, +} \geq 0, \epsilon_{A}^* \in [-\epsilon_{A, -}^*, \epsilon_{A, +}^*]  \}, \label{bi-const1-friction3} \\
% % \epsilon_{B, -}^* \in \argmax_{\epsilon_{B, -}} \{\epsilon_{B, -}:  g(x, u, f, -\epsilon_{B, -}, \epsilon_{A}^*)\leq 0 , \nonumber \\ \epsilon_{B, -} \geq 0, \epsilon_{A}^* \in [-\epsilon_{A, -}^*, \epsilon_{A, +}^*]  \}, \label{bi-const1-friction4} 
% % \end{flalign}
% % \label{eq:bilvel_friction}
% % \end{subequations}
% % Actually, we can instead have the following optimization problem by moving the constraints:  
% \begin{subequations}
% \begin{flalign}
% % \max_{\Gamma} (\min _{k} \epsilon_{A, +}^* - \max _{k} -\epsilon_{A, -}^* + \min _{k} \epsilon_{B, +}^* - \max _{k} -\epsilon_{B, -}^*)  \ \\
% \max_{x, u, f, \epsilon_{A,+}^*, \epsilon_{A, -}^*, \epsilon_{B,+}^*, \epsilon_{B, -}^*} \sum_{c\in \mathcal{C}} (\min _{k} \epsilon_{c, +}^* - \max _{k} -\epsilon_{c, -}^*)\\
% \text{s. t. } \quad \text{\eq{const2}, \eq{bounds_variables}}, \\
% % 
% \epsilon_{A}^* \in [-\epsilon_{A, -}^*, \epsilon_{A, +}^*], \epsilon_{B}^* \in [-\epsilon_{B, -}^*, \epsilon_{B, +}^*],  \\
% % 
% \epsilon_{A, +}^* \in \argmax_{\epsilon_{A, +}} \{\epsilon_{A, +}:  g(x, u, f, \epsilon_{A, +},  \epsilon_{B}^*)\leq 0,  \nonumber \\
% \epsilon_{A, +} \geq 0 \}, \label{bi-const1-friction12} \\
% % 
% \epsilon_{A, -}^* \in \argmax_{\epsilon_{A, -}} \{\epsilon_{A, -}:  g(x, u, f, -\epsilon_{A, -}, \epsilon_{B}^*)\leq 0, \nonumber \\
% \epsilon_{A, -} \geq 0,   \}, \label{bi-const1-friction22} \\
% % 
% \epsilon_{B, +}^* \in \argmax_{\epsilon_{B, +}} \{\epsilon_{B, +}:  g(x, u, f, \epsilon_{B, +}, \epsilon_{A}^*)\leq 0, \nonumber \\ \epsilon_{B, +} \geq 0, \}, \label{bi-const1-friction32} \\
% % 
% \epsilon_{B, -}^* \in \argmax_{\epsilon_{B, -}} \{\epsilon_{B, -}:  g(x, u, f, -\epsilon_{B, -}, \epsilon_{A}^*)\leq 0, \nonumber \\
% \epsilon_{B, -} \geq 0  \}, \label{bi-const1-friction42} 
% \end{flalign}
% \label{eq:bilvel_friction2}
% \end{subequations}
% where $g$ summarizes the constraints for each lower-level optimization problem and $\mathcal{C} = \{A, B\}$. 
% % 
% The resulting optimization introduces many complementarity constraints through the KKT condition because of four lower-level optimization problems, but the resulting computation is still tractable. We discuss the computation result in Sec~\ref{subsec::computation}.

\subsection{Robust CIBO with Patch Contact}
The formulation for robust CIBO is similar to the point contact case except that the underlying equilibrium conditions are different. The \revise{quasi-static} equilibrium conditions for the patch contact case were earlier presented in~\eqref{force_balance_patch_contact}. Using these equations and the analysis presented in Sections~\ref{sec:sec_uncertain_mass} through~\ref{subsec:stochasticfriction_planning}, it is straightforward to compute the constraints for the corresponding robust CIBO similar to~\eqref{equation_sm_1}. More explicitly, this can be achieved by computing the appropriate constraints of the type $A_k\epsilon_{k, +} \leq b_k$ and $-A_k\epsilon_{k, -} \leq b_k$ using~\eqref{force_balance_patch_contact} and the frictional stability margin discussion in Sec~\ref{subsec:Pivoting_manipulation}.

% Given patch contact model, we can formulate optimization problem in the same fashion using \eq{equation_sm_1}. We simply need to replace  with the constraints discussed in Sec~\ref{subsec:Pivoting_manipulation}.
% \textcolor{red}{where do we say considering complementarity constraints would make optimization intractable? here or in Sec~\ref{subsec:Pivoting_manipulation}?}




% \input{feedbackmanipulation/instability_detection}
\section{Results}
\label{results}

\begin{figure*}[ht]
    \centering
    \includegraphics[scale=0.15,trim={0 2.5cm 0 5cm},clip]{images/aoi-single_burst}
    \caption{The time average peak Age of Information with burst and \gls{soa} loss values against the dynamic reliability logic for different network topologies.}
    \label{fig:aoi_burst}\vspace{-0.4cm}
\end{figure*}


This paper focuses on both transport layer and application layer metrics to determine the feasibility of dynamic reliability. For this, we have selected the session packet volume, as transmitted, retransmitted, lost and backlogged packets as \glspl{kpi} for the transport layer; while focusing on the \gls{aoi} for the application layer. The \gls{aoi} was chosen as a crucial indicator for the freshness of packets in real-time applications. More specifically, this work adopts the time average peak \gls{aoi} equation \cite{aoi_equation} depicted in Eq. \ref{aoi}, where $\Delta(r_{i+1})$ is the $i$th update at the time it was received at the server, for a session time period of $\tau$.

\begin{equation}
    \label{aoi}
    \gls{aoi}_\tau = \frac{1}{n-1}\sum_{i=1}^{n-1} \Delta(r_{i+1})
\end{equation}

We include a comparison between the vanilla QUIC implementation which does not enjoy the dynamic reliability extension, with a number of dynamic reliability policies. The tests were run a number of times for statistical significance, with the mean value of vanilla implementation used as a baseline for comparison. The topology utilised both random loss and bursty loss to explore the bounds of dynamic reliability. The \gls{soa} loss in the figures correspond to the loss values presented in Table. \ref{tab:path_char}, for ease of comparison between bursty and random loss scenarios.

\subsection{Transport-Layer KPIs}

To analyse the performance gain at the transport layer due to dynamic reliability, the volume of transmitted and backlogged packets is examined. The figures are in the form of boxplots, which take the vanilla implementation as a benchmark, depicted as the red dashed line.

As seen in Fig. \ref{fig:sent_burst}, the loss plays a crucial role in the performance of the reliability policies. The policies under random loss did incredibly well for the networks with a larger capacity, namely \gls{mmwave} and Sub-6~GHz, whereas for burst loss, the lower network capacities had a larger packet reduction. With the increase in burst loss, the behaviour of the set split reliable policies became unpredictable, if a reliable assignment happened to coincide with a burst loss, the number of transmitted packets increases, and vice versa. On the other hand, in smarter policies, such as Loss-Aware, the performance lightly matched the vanilla baseline, as the reliable assignment dominated the session to compensate for a higher burst loss. Not only that but, the burst loss also impacted the variance of the transmitted packets for the policies.

Unsurprisingly, the unreliable focused policy, 80-20 split, outperformed other policies for all topologies in random and bursty loss scenarios, with an approximate reduction of 80\%. That being said, the majority of the policies reduced the transmitted packets on the link by approximately 70\% for random loss, while the reduction started at $\approx 15\%$ and decreased as the loss increased for the burst loss scenario.

The retransmitted and lost packets, not shown due to space limitations, followed the same trend as the transmitted packets for the random loss scenarios. However, for the burst loss scenarios, the larger capacity networks had a lower reduction in the retransmitted and lost packets. This can be seen as a favorable outcome since the lower capacity networks are scarce on resources. It is important to note that the Loss-Aware policy mimicked the vanilla approach as the burst loss increased, signifying the overwhelming appointment of reliable packets in adapting to the harsh burst loss conditions.
 
Alternatively, Fig. \ref{fig:backlog_burst} clearly shows a stark comparison between the policies and loss scenario in the reduction of the backlogged packets. The Loss-Aware policy for random loss scenario reduced the backlogged packets by up to 50\%, beating all other policies by approximately 30\%. Furthermore, it is clear that the unreliability focused policies resulted in the lowest backlog for the session. In comparison, we notice that the burst loss and the backlogged frequency have a positive correlation, where the maximum reduction of the backlogged packets for the policies is at most 20\%. Much like the transmitted packets, the probability of a burst loss occurrence plays a vital role in the number of retransmissions sent and by extension the number of backlogged packets. Thus, we can conclude that the stress placed on the buffer is a result of the reliable packets which is tightly coupled with the congestion on the session. Whereas, unreliable focused policies did not encounter such a phenomenon regardless if it was experiencing a burst loss.


\subsection{Application-Layer KPIs}

The feasibility of dynamic reliability for real-time applications can be determined by the \gls{aoi}, with comparison across different topologies and policies. If we take a strict approach and consider anything below $10$~ms is real-time \cite{real-time}, then all the reliability policies passed that requirement, which is attractive for real-time applications, as shown in Fig. \ref{fig:aoi_burst}. Utilising the median as an estimate of the runs, the policies in the WLAN and Sub-6~GHz topology with random loss floated around $4-5$~ms with negligible difference, while the \gls{aoi} for \gls{mmwave} was $\approx 2-3$~ms. It is clear that the \gls{aoi} and the network capacity have a negative correlation, as the network capacity decreases, the \gls{aoi} increases. The same correlation is extended to the bursty loss scenarios, where \gls{mmwave} dominated the other topologies. That being said, it is crucial to note that the \gls{aoi} for the reliability policies is often slightly better than or equal to the \gls{aoi} of the vanilla implementation, proving that dynamic reliability reduces the congestion of the session at no cost to the \gls{aoi}.

We provide some comments on the growth conditions which constituted the majority of our analysis in sections \ref{sec:Hmixing} and \ref{sec:Hsigma}. In the simplest cases of Lemma \ref{lemma:unstableGrowth}, growth was established in an analogous fashion to the old one-step expansion condition (\ref{eq:oldOneStepExpansion}), finding the relevant Jacobians $M_j$ and checking that their expansion factors $K(M_j)$ satisfy
\begin{equation}
    \label{eq:discussionOneStep}
    \sum_j \frac{1}{K(M_j)} <1.
\end{equation}
For the more complicated cases, the inductive method used to establish growth near the accumulation points in Lemma \ref{lemma:unstableGrowth} and the weakened one-step expansion condition (\ref{eq:oneStep}) both address the same fundamental issue: the splitting of unstable curves by singularities into an unbounded number of small components. They circumvent this obstacle in rather different ways, however. While (\ref{eq:oneStep}) generalises (\ref{eq:discussionOneStep}) to ensure an growth of unstable curves `on average' (see \cite{chernov_statistical_2009} for a precise statement), our inductive method is a more direct adaptation of (\ref{eq:discussionOneStep}), using it to generate contradictory geometric conditions which a hypothetical non-growing unstable curve must satisfy. It may be possible to prove Theorem \ref{sec:Hmixing} using (\ref{eq:oneStep}) as the basis for growth. Since we required (\ref{eq:oneStep}) anyway for proving Theorem \ref{thm:HsigmaExp}, this could potentially condense our analysis, but only to a minor extent. A convenience of the method used in section \ref{sec:Hmixing} is that, by way of the `simple intersection' property, it naturally gives geometric information on the images of manifolds, useful for proving the property \textbf{(M)} of Theorem \ref{thm:katok-strelcyn}.

We expect that essentially analogous analysis can be applied to establish mixing properties in a wide class of piecewise linear non-uniformly hyperbolic maps, including those (like the OTM) which sit on the boundary of ergodicity and beyond. While we have relied on the precise partition structure of $H_\sigma$, its fundamental feature (self-similar sequences of elements $A^k$, sharing boundaries with its neighbours $A^{k-1},A^{k+1}$ and accumulating onto some point $p$) is quite typical to return map systems. See, for example, those of various stadium billiards \cite{chernov_chaotic_2006,chernov_improved_2008,chernov_statistical_2009} and LTMs \cite{springham_polynomial_2014}. Indeed, the same method can be used to prove the Bernoulli property for non-monotonic LTMs \cite{myers_hill_mixing_2022}, where monotonicity of the manifold images cannot be assumed and the classical argument \cite{sturman_mathematical_2006} fails. The OTM is the pointwise limit of these maps as the boundary shrinks to null measure. It further has utility in proving growth conditions for maps which are uniformly hyperbolic but possess regions $A_j$ where the hyperbolicity is very weak, signified by $K(M_j) \approx 1$, so that (\ref{eq:discussionOneStep}) fails. Typically this leads to suboptimal bounds on mixing windows, see e.g. \cite{wojtkowski_model_1981,przytycki_ergodicity_1983,myers_hill_family_2022}. The map $H_{(\eta,\eta)}$ for $\eta \approx 1/2$ is another example, possessing weak hyperbolicity over $A_2, A_3$. Letting $\varepsilon = |\eta-1/2|>0$, there is an upper bound $N = N(\varepsilon)$ on escape times from the intersections $A_2\cap \sigma, A_3 \cap \sigma$. The growth lemma then follows by applying the inductive step roughly $N$ times and can be established for arbitrarily small $\varepsilon$, opening the door to establishing optimal mixing windows.

The above gives two examples of piecewise linear perturbations to $H$ where mixing with respect to Lebesgue is preserved and our methods can be applied. Nonlinear perturbations to the shear profiles complicate the analysis in several ways. Firstly as the map's Jacobians takes on a broader range of values, cone invariance becomes an increasingly harder condition to establish. Cones must be widened, giving looser bounds on expansion factors, which may already be weak due to new regions of weaker stretching. This, together with the change from polygonal to curvilinear return time partition elements and nonlinear local manifolds, adds some complexity to showing growth conditions. This does not rule out certain (small) nonlinear perturbations however. There is some leeway in the inequalities which govern cone invariance and growth of local manifolds, the latter of which is not too dissimilar from the piecewise linear setting (see Lemmas \ref{lemma:piecewiseApprox}, \ref{lemma:componentLength}). Certain small perturbations would not alter the \emph{topological} structure of the return time partition, i.e. which elements share boundaries, the key information needed for setting up the induction. Finally while the partition elements would no longer be polygonal, only coarse geometric information is required for verifying each inductive step. Following the above, a potential perturbation could be to replace the linear portions of each shear by a cubic, perturbing the tent profile
\[  f(t) = \begin{cases} 2t & 0 \leq t \leq 1/2, \\ 2(1-t) & 1/2 \leq t \leq 1 ,\end{cases} \]
of the OTM shears to
\[  f_a(t) = \begin{cases} \frac{1}{8} t \left(16 - a + 6at - 8at^{2} \right) & 0 \leq t \leq 1/2, \\ \frac{1}{8}\left(1-t\right)\left( 16 - a + 6a\left(1-t\right) - 8a\left(1-t\right)^{2}\right)  & 1/2 \leq t \leq 1, \end{cases}   \]
for $a>0$. For small enough $a$ the gradient range $f'(t)$ is restricted to small neighbourhoods of $\{ 2, -2\}$ and the escape time partition retains a similar structure. We illustrate this in Figure \ref{fig:perturbations}, showing escapes from the square $S_3$ under the map $G \circ F$, equivalent to escapes from the perturbed $A_3$ under the $G \circ F$, but with a cleaner geometry for comparison. When $a$ is too large the analogy to the OTM breaks down. At $a=16$ the map is twice differentiable everywhere and features a new source of slowed mixing, the Jacobian is the identity at the corner points $x,y \in \{  0, 1/2 \}$ giving locally parabolic behaviour (visible in the escape time partition). 

\begin{figure}
    \centering
    \includegraphics[width=0.24 \linewidth]{0.png}
    \includegraphics[width=0.24 \linewidth]{4.png}
    \includegraphics[width=0.24 \linewidth]{8.png}
    \includegraphics[width=0.24 \linewidth]{16.png}
    \caption{Partition of escape times from $S_3$ under the mapping $F \circ G$ for $a= 0,4,8,16$. }
    \label{fig:perturbations}
\end{figure}

%\appendix{}
%\section{Appendix for Proofs}

\paragraph{Proof of Theorem \ref{thm:main}.}

\begin{proof}
\label{proof:main}
Our proof has two steps. In Step 1, we will show that SimCLR is equivalent to minimizing the cross entropy loss defined in Eqn.~(\ref{eqn:cross-entropy}). 
In Step 2, we will show  that minimizing the cross-entropy loss 
is equivalent to spectral clustering on $\bfpi$. 
Combining the two steps together, we have proved our theorem. 

\textbf{Step 1: } SimCLR is equivalent to minimizing the cross entropy loss.

The cross-entropy loss takes expectation over 
$\bfW_\bfX\sim \mathbb{P}(\cdot ; \bfpi)$, 
which means $\bfW_\bfX$ has exactly one non-zero entry in each row $i$. By Lemma~\ref{lem:multinomial}, we know every row $i$ of $\bfW_\bfX$ is independent of other rows. Moreover, 
$\bfW_{\bfX,i}\sim \mathcal{M}(1, \bfpi_i/\sum_j \bfpi_{i,j})=\mathcal{M}(1, \bfpi_i)$, because $\bfpi_i$ itself is a probability distribution.
Similarly, we know $\bfW_\bfZ$ also has the row-independent property by sampling over $\mathbb{P}(\cdot;\bfK_\bfZ)$.
Therefore, by Lemma~\ref{lem:cross_split}, we know Eqn.~(\ref{eqn:cross-entropy}) is equivalent to:
\[
 -\sum_{i=1}^n \mathbb{E}_{\bfW_{\bfX,i}}[\log \mathbb{P}(\bfW_{\bfZ,i}=\bfW_{\bfX,i};\bfK_\bfZ)],
\]

This expression takes expectation over $\bfW_{\bfX,i}$ for the given row $i$. Notice that 
$\bfW_{\bfX,i}$ has exactly one non-zero entry, which equals $1$ (same for $\bfW_{\bfZ,i}$). 
As a result
we expand the above expression to be:
\begin{equation}
 -\sum_{i=1}^n \sum_{j\neq i} \Pr(\bfW_{\bfX,i,j}=1)\log \Pr(\bfW_{\bfZ,i,j}=1).
\label{eqn:detailed-expansion}    
\end{equation}


By Lemma~\ref{lem:multinomial}, $\Pr(\bfW_{\bfZ,i,j}=1)=\bfK_{\bfZ,i,j}/\|\bfK_{\bfZ,i}\|_1$ for $j\neq i$. Recall that $\bfK_\bfZ=(k(\bfZ_i-\bfZ_j))_{(i,j)\in[n]^2}$, which means 
$\bfK_{\bfZ,i,j}/\|\bfK_{\bfZ,i}\|_1=\frac{\exp(-\|\bfZ_i-\bfZ_j\|^2/{2\tau})}{\sum_{k\neq i}
\exp(-\|\bfZ_i-\bfZ_k\|^2/{2\tau})
}$ for $j\neq i$, when $k$ is the Gaussian kernel with variance $\tau$. 

Notice that $\bfZ_i=f(\bfX_i)$, so we know
\begin{equation}
-\log \Pr(\bfW_{\bfZ,i,j}=1)=
-\log \frac{\exp(-\|f(\bfX_i)-f(\bfX_j)\|^2/{2\tau})}{\sum_{k\neq i}
\exp(-\|f(\bfX_i)-f(\bfX_k)\|^2/{2\tau}),
}
\label{eqn:infonce-equivalence}    
\end{equation}


The right hand side is exactly the InfoNCE loss defined in Eqn.~(\ref{eqn:infonce}).
Inserting Eqn.~(\ref{eqn:infonce-equivalence}) into Eqn.~(\ref{eqn:detailed-expansion}), we get the SimCLR algorithm, which first samples augmentation pairs $(i,j)$ with $\Pr(\bfW_{\bfX,i,j}=1)$ for each row $i$, and then optimize the InfoNCE loss. 

\textbf{Step 2: } minimizing the cross entropy loss 
is equivalent to spectral clustering on $\bfpi$.


By Lemma~\ref{lem:convert_to_spectral}, we may further convert the loss to 
\begin{equation}
\label{eqn:main-theorem-repul-attr}
\min_{\bfZ}
-\sum_{(i,j)\in [n]^2} \mathbf{P}_{i,j}
\log k (\bfZ_i-\bfZ_j)+\log \mathbf{R}(\bfZ).
\end{equation}
Since $k$ is the Gaussian kernel, this reduces to \[
\min_\bfZ \mathrm{tr}(\bfZ^\top \mathbf{L}(\bfpi) \bfZ)
+\log \mathbf{R}(\bfZ),
\]

where we use the fact that $\mathbb{E}_{\bfW_\bfX\sim \mathbb{P}(\cdot; \bfpi)}[\mathbf{L}(\bfW_\bfX)]
=\mathbf{L}(\bfpi)
$, because the Laplacian operator is linear and $
\mathbb{E}_{\bfW_\bfX\sim \mathbb{P}(\cdot; \bfpi)}(\bfW_\bfX)=\bfpi
$.
\end{proof}

\paragraph{Proof of Theorem \ref{thm:clip}.}
\begin{proof}
Since $\bfW_\bfX\sim \mathbb{P}(\cdot;\bfpi_{\mathbf{A}, \mathbf{B}})$, we know 
$\bfW_\bfX$ has exactly one non-zero entry in each row, denoting the pair that got sampled. 
A notable difference compared to the previous proof is we now have $n_\mathcal{A}+n_\mathcal{B}$ objects in our graph. CLIP deals with this by taking a mini-batch of size $2N$, 
such that $n_\mathcal{A}=n_\mathcal{B}=N$, and adding the $2N$ InfoNCE losses together. We label the objects in $\mathcal{A}$ as $[n_\mathcal{A}]$, and the objects in $\mathcal{B}$ as $\{n_\mathcal{A}+1, \cdots, n_\mathcal{A}+n_\mathcal{B}\}$. 

Notice that $\bfpi_{\mathbf{A}, \mathbf{B}}$ is a bipartite graph, so the edges of objects in $\mathcal{A}$ will only connect to object in $\mathcal{B}$ and vice versa. We can define the similarity matrix in $\cZ$ as $\bfK_\bfZ$, 
where $\bfK_\bfZ(i, j+n_\mathcal{A})=\bfK_\bfZ(j+n_\mathcal{A},i)= k(\bfZ_i-\bfZ_j)$ for $i\in [n_\mathcal{A}], j\in [n_\mathcal{B}]$, and otherwise we set $\bfK_\bfZ(i,j)=0$. 
The rest is same as the previous proof. 
\end{proof}

\paragraph{Proof of Theorem \ref{thm:exponential}.}

\begin{proof}
\label{proof:exponential}
Since the objective function consists of a linear term combined with an entropy regularization, which is a strongly concave function, the maximization problem is a convex optimization problem. Owing to the implicit constraints provided by the entropy function, the problem is equivalent to having only the equality constraint. We then introduce the Lagrangian multiplier $\lambda$ and obtain the following relaxed problem:

$$
\widetilde{E}(\boldsymbol{\alpha})=\psi_{1}-\sum_{i=1}^n \alpha_{i} \psi_{i}+\tau \sum_{i=1}^n \alpha_{i}\log \alpha_{i}+\lambda\left(\boldsymbol{\alpha}^{\top} \mathbf{1}_n-1\right).
$$

As the relaxed problem is unconstrained, taking the derivative with respect to $\alpha_{i}$ yields

$$
\frac{\partial \widetilde{E}(\boldsymbol{\alpha})}{\partial \alpha_{i}}=-\psi_{i}+\tau\left(\log \alpha_{i}+\alpha_{i} \frac{1}{\alpha_{i}}\right)+\lambda=0.
$$

Solving the above equation implies that $\alpha_{i}$ takes the form
$
\alpha_{i}=\exp \left(\frac{1}{\tau} \psi_{i}\right) \exp \left(\frac{-\lambda}{\tau}-1\right).
$ Since $\alpha_{i}$ lies on the probability simplex, the optimal $\alpha_{i}$ is explicitly given by
$
\alpha^{*}_{i}=\frac{\exp \left(\frac{1}{\tau} \psi_{i}\right)}{\sum_{i^{\prime}=1}^n \exp \left(\frac{1}{\tau} \psi_{i^{\prime}}\right)} .
$ Substituting the optimal point into the objective function, we obtain
$$
\begin{aligned}
E\left(\boldsymbol{\alpha}^*\right)  &=\psi_1-\sum_{i=1}^n \frac{\exp \left(\frac{1}{\tau} \psi_{i}\right)}{\sum_{i^{\prime}=1}^n \exp \left(\frac{1}{\tau} \psi_{i^{\prime}}\right)} \psi_{i}+\tau \sum_{i=1}^n \frac{\exp \left(\frac{1}{\tau} \psi_{i}\right)}{\sum_{i^{\prime}=1}^n \exp \left(\frac{1}{\tau} \psi_{i^{\prime}}\right)}\log \frac{\exp \left(\frac{1}{\tau} \psi_{i}\right)}{\sum_{i^{\prime}=1}^n \exp \left(\frac{1}{\tau} \psi_{i^{\prime}}\right)} \\
& =\psi_1 - \tau \log \left(\sum_{i=1}^n \exp \left(\frac{1}{\tau} \psi_{i}\right)\right).
\end{aligned}
$$
Thus, the Lagrangian dual function is given by
\begin{equation*}
-E\left(\boldsymbol{\alpha}^*\right)= -\tau \log \frac{\exp \left(\frac{1}{\tau} \psi_{1}\right)}{\sum_{i=1}^n \exp \left(\frac{1}{\tau} \psi_{i}\right)}.\qedhere
\end{equation*}
\end{proof}



\section{More on Experiments} \label{section: experiment_details}

\paragraph{CIFAR-10 and CIFAR-100} CIFAR-10 ~\citep{krizhevsky2009learning} and CIFAR-100 ~\citep{krizhevsky2009learning} are well-known classic image classification datasets. Both CIFAR-10 and CIFAR-100 contain a total of 60k $32 \times 32$ labeled images of different classes, with 50k for training and 10k for testing. CIFAR-10 is similar to CIFAR-100, except there are 10 different classes in CIFAR-10 and 100 classes in CIFAR-100.

\paragraph{TinyImageNet} TinyImageNet ~\citep{le2015tiny} is a subset of ImageNet ~\citep{deng2009imagenet}. There are 200 different object classes in TinyImageNet, with 500 training images, 50 validation images, and 50 test images for each class. All the images in TinyImageNet are colored and labeled with a size of $64 \times 64$.

\textbf{Pseudo-code.} Algorithm \ref{alg:Training Procedure} presents the pseudo-code for our empirical training procedure.

\begin{algorithm}[!htbp]
\caption{Training Procedure}
\label{alg:Training Procedure}
\begin{algorithmic}[1]
\REQUIRE trainable encoder network $f$, batch size $N$, augmentation strategy \textit{aug}, loss function $L$ with hyperparameters \textit{args}
\FOR {sampled minibatch ${x_i}_{i=1}^N$}
\FORALL{$i \in { 1, ..., N }$}
\STATE draw two augmentations $t_i = \textit{aug}\left(x_i\right) $, $t_i' = \textit{aug}\left(x_i\right) $
\STATE $z_i = f\left(t_i\right)$, $z_i' = f\left(t_i'\right)$
\ENDFOR
\STATE compute loss $\mathcal{L} = L(N, z, z', \textit{args})$
\STATE update encoder network $f$ to minimize $\mathcal{L}$
\ENDFOR
\STATE \textbf{Return} encoder network $f$
\end{algorithmic}
\end{algorithm}

We also provide the pseudo-code for our core loss function used in the training procedure in Algorithm \ref{alg:Core loss}. The pseudo-code is almost identical to SimCLR's loss function, with the exception of an extra parameter $\gamma$.

\begin{algorithm}[!htbp]
\caption{Core loss function $\mathcal{C}$}
\label{alg:Core loss}
\begin{algorithmic}[1]
\REQUIRE batch size $N$, two encoded minibatches $z_1, z_2$, $\gamma$, temperature $\tau$
\STATE $z = \textit{concat}\left(z_1, z_2\right)$
\FOR {$i \in {1, ..., 2N }, j \in {1, ..., 2N}$ }
\STATE $s_{i,j} = \Vert z_i - z_j \Vert_2^{\gamma}$
\ENDFOR
\STATE \textbf{define} $l(i, j)$ \textbf{as} $l(i, j) = - \log \frac{exp\left(s_{i,j}/\tau \right)}{\sum_{k=1}^{2N} \mathbf{1}{[k \ne i]} exp\left(s{i, j} / \tau \right)} $
\STATE \textbf{Return} $\frac{1}{2N} \sum_{k=1}^N\left[l(i, i+N) + l(i+N, i)\right]$
\end{algorithmic}
\end{algorithm}

Utilizing the core loss function $\mathcal{C}$, we can define all kernel loss functions used in our experiments in Table \ref{table: loss definition}. For all $z_i \in z$ with even dimensions $n$, we define $z_{L_i} = z_i\left[0:n/2\right]$ and $z_{R_i} = z_i\left[n/2:n\right]$.

\begin{table}[ht]
\centering
\begin{tabular}{{@{}l|l@{}}}
Kernel  &  Loss function \\ \midrule
Laplacian & $\mathcal{C}\left(N, z, z', \gamma=1, \tau\right)$\\ \midrule
Sum       & $\lambda * \mathcal{C}\left(N, z, z', \gamma=1, \tau_1\right) + (1-\lambda) * \mathcal{C}\left(N, z, z', \gamma=2, \tau_2\right)$  \\ \midrule
Concatenation Sum&$\lambda * \mathcal{C}\left(N, z_L, z'_L, \gamma=1, \tau_1\right) + (1-\lambda) * \mathcal{C}\left(N, z_R, z'_R, \gamma=2, \tau_2\right)$\\ \midrule
$\gamma = 0.5$ & $\mathcal{C}\left(N, z, z', \gamma=0.5, \tau\right)$          \\ 

\end{tabular}

\caption{Definition of kernel loss functions in our experiments}
\label {table: loss definition}
\end{table}

\textbf{Baselines.} We reproduce the SimCLR algorithm using PyTorch Lightning~\citep{PytorchLightning}.

\textbf{Encoder details.}
The encoder $f$ consists of a backbone network and a projection network. We employ ResNet50~\citep{ResNet} as the backbone and a 2-layer MLP (connected by a batch normalization~\citep{ioffe2015batch} layer and a ReLU \cite{nair2010rectified} layer) with hidden dimensions 2048 and output dimensions 128 (or 256 in the concatenation kernel case).

\textbf{Encoder hyperparameter tuning.}
For each encoder training case, we randomly sample 500 hyperparameter groups (sample details are shown in Table \ref{table: Hyperparameter sample}) and train these samples simultaneously using Ray Tune ~\citep{RayTune}, with the ASHA scheduler~\citep{li2018massively}. Ultimately, the hyperparameter group that maximizes the online validation accuracy (integrated in PyTorch Lightning) within 5000 validation steps is chosen for the given encoder training case.

\begin{table}[ht]
\centering

\begin{tabular}{@{}l|l|l@{}}
\midrule
Hyperparameter  & Sample Range & Sample Strategy \\ \midrule
start learning rate & $\left[10^{-2}, 10\right]$ & log uniform \\ \midrule
$\lambda$       & $\left[0, 1\right]$ & uniform \\ \midrule
$\tau$, $\tau_1$, $\tau_2$ & $\left[0, 1\right]$ & log uniform \\ \midrule
\end{tabular}

\caption{Hyperparameters sample strategy}
\label {table: Hyperparameter sample}
\end{table}

\textbf{Encoder training.} 
We train each encoder using the LARS optimizer~\citep{LARSOptimizer}, LambdaLR Scheduler in PyTorch, momentum 0.9, weight decay $10^{-6}$, batch size 256, and the aforementioned hyperparameters for 400 epochs on a single A-100 GPU.

\textbf{Image transformation.} The image transformation strategy, including augmentation, is identical to the default transformation strategy provided by PyTorch Lightning.

\textbf{Linear evaluation.}
The linear head is trained using the SGD optimizer with a cosine learning rate scheduler, batch size 64, and weight decay $10^{-6}$ for 100 epochs. The learning rate starts at $0.3$ and ends at $0$.

\textbf{Moco Experiments.} We also tested our method based on MoCo~\citep{he2019moco}. The results are summarized in Table \ref{tab:results-moco}. Here we choose ResNet18~\citep{ResNet} as the backbone and set a temperature of $0.1$ as default. For our simple sum kernel, we set $\lambda=0.8$. The results show that our method outperforms the original MoCo method.

\begin{table}[thb]
\centering
\caption{MoCo Experiment Results on CIFAR-10 and CIFAR-100.}
\label{tab:results-moco}
\resizebox{\textwidth}{!}{%
\begin{tabular}{@{}c|ccc|ccc@{}}
\toprule
\multirow{3}{*}{Method} & \multicolumn{3}{c|}{CIFAR-10} & \multicolumn{3}{c}{CIFAR-100} \\ \cmidrule(lr){2-4} \cmidrule(lr){5-7} 
                        & 200 epochs & 400 epochs    & 1000 epochs   & 200 epochs & 400 epochs & 1000 epochs         \\ \midrule
MoCo (repro.)         & $76.41 \pm 0.12$    & $80.01 \pm 0.15$          & $84.45 \pm 0.08$    & $\mathbf{47.02 \pm 0.11}$ & $52.50 \pm 0.07$ & $57.62 \pm 0.15$            \\
\midrule
Laplacian Kernel        & ${78.09 \pm 0.10}$    & $\mathbf{83.85 \pm 0.09}$          & $\mathbf{88.34 \pm 0.16}$    & $46.12 \pm 0.22$   & $53.44 \pm 0.17$ & $59.10 \pm 0.14$        \\
Simple Sum Kernel & $\mathbf{78.12 \pm 0.15}$   & $83.23 \pm 0.18$ & $87.50 \pm 0.20$ & $46.65 \pm 0.06$ & $\mathbf{53.62 \pm 0.19}$ & $\mathbf{59.83 \pm 0.12}$\\
\bottomrule
\end{tabular}
}
\end{table}



\section{More Experiments on Synthetic Data}


Consider a scenario with $n$ clusters, each containing $k$ vertices. Let the probability of vertices $u$ and $v$ from the same cluster belonging to $\bfpi$ be $p$. Conversely, for vertices $u$ and $v$ from different clusters, let the probability of belonging to $\pi$ be $q$. We generate the graph $\bfpi$ randomly, based on $p$ and $q$. We experiment with values of $k=100$ and $n=6$ for ease of visualization, embedding all points in a two-dimensional space. Each vertex's initial position originates from a normal distribution. In each iteration, we sample a subgraph of $\bfpi$ uniformly, ensuring each vertex has an out-degree of $1$. We then optimize the corresponding vectors using InfoNCE loss with an SGD optimizer and iterate until convergence. Our experimental setup consists of an SGD learning rate of $1$, an InfoNCE loss temperature of $0.5$, and a batch size of $50$. We evaluate two scenarios with different $p$ and $q$ values: $p=1$, $q=0$, and $p=0.75$, $q=0.2$. The results of these experiments are visualized in Figure \ref{fig:vis-spectral-cluster}. The obtained embeddings exhibit the hallmark pattern of spectral clustering of graph $\bfpi$.

\begin{figure}[!tb]
\centering
\subfigure{
\includegraphics[width=1\textwidth]{Figures/cluster_pi.png}
\label{fig:vis-cluster}
}
\subfigure{
\includegraphics[width=1\textwidth]{Figures/noised_cluster_pi.png}
\label{fig:vis-noised-cluster}
}
\caption{Visualizations of the optimization process using InfoNCE Loss on the vectors corresponding to $\bfpi$. Points of identical color belong to the same cluster within $\bfpi$. To showcase the internal structure of $\bfpi$, we randomly select 10 vertices from each cluster to display the edge distribution of $\bfpi$.}
\label{fig:vis-spectral-cluster}
\end{figure}



% \newpage
% \clearpage

\bibliographystyle{IEEEtran}
\bibliography{main}

\end{document}
