\section{Related Work}\label{sec:related_work}


%\textcolor{red}{shouldn't we add more papers?}

% In this section, we present some literature which is closely related to the work we present in this paper. 
Contact modeling has been extensively studied in mechanics as well as robotics literature~\cite{todorov2010implicit, drumwright2011evaluation, 9366782, shirai2022iros, 9561521, 9739950}, \revise{\cite{Yoshida_Regrasp2009, Dafle_gripper2015}}. One of the most common contact models is based on the linear complementarity problem (LCP). LCP-based contact models have been extensively used for performing trajectory optimization in manipulation~\cite{9812069,jin2021trajectory} as well as locomotion~\cite{posa2014direct, 8648229}. More recently, there has also been some work for designing robust manipulation techniques for contact-rich systems using stochastic optimization~\cite{yuki2021chance,drnach2021robust, 9812069, shirai2023covariance}. These problems consider stochastic complementarity systems and consider robust optimization for the underlying stochastic system. However, these problems consider a dynamical model and do not explicitly consider the mechanical stability during planning. Our work is motivated by the concepts of stability under multiple contacts in legged locomotion. \revise{Quasi-static} stability with multiple contacts has been widely studied in legged locomotion~\cite{4598894, 8383993, 8416785, 8358969, 9113247}. These works consider the problem of mechanical stability of the legged robot under multiple contacts by considering the stability polygon defined by the frictional contacts. 
The planning framework for optimizing contact wrench cone margin during locomotion is able to achieve robust locomotion results \cite{dai2016planning, 8358969, 8289420}.
Similar to the concept of these works, we present the idea of frictional stability which defines the extent to which multiple points of contact can compensate for unknown forces and moments in the presence of uncertainty in the mass, CoM location, \revise{contact location,} and frictional parameters. This idea exploits contact forces to ensure stability of the object during the two-point pivoting.  % of the object being manipulated. 
Our work is also related to manipulation by shared grasping~\cite{hou2020manipulation} which discusses mechanics of shared grasping and shows impressive demonstrations. In contrast to the work presented in~\cite{hou2020manipulation}, we present a robust contact-implicit bilevel optimization (CIBO) framework that can be used to find feasible solutions in the presence of uncertainty during the pivoting manipulation and avoids consideration of different modes during planning.

In~\cite{hogan2020tactile}, authors consider stabilization of a table-top manipulation task during online control. They consider a decomposition of the control task in object state control and contact state control. The contact state was detected using vision-based tactile sensors~\cite{donlon2018gelslim,li2020f}. As the task mostly required sticking contact for stability, the tactile feedback was designed to make corrections to push the system away from the boundary of friction cone at the different contact locations. However, the authors did not consider the problem of designing trajectories which can provide robustness to uncertainty. Furthermore, the authors only considered controlled sticking in~\cite{hogan2020tactile} which is, in general, easier than controlled slipping.  
\revise{
Similarly, in \cite{DanicaIn-hand2015}, authors design and validate their sliding controller for in-hand tool pivoting. In \cite{DanicaIn-hand2016}, the authors extend their sliding controller in \cite{DanicaIn-hand2015} such that the sliding controller is able to achieve adaptive control for friction coefficients using visual and force measurements, showing impressive demonstrations. 
Also, authors in \cite{Cruciani_inhand2017} consider pivoting manipulation with a parallel gripper without relying on fast and precise robotic systems.
In contrast to their work in \cite{DanicaIn-hand2015, DanicaIn-hand2016, Cruciani_inhand2017}, we present the pivoting manipulation with extrinsic contacts, which introduces additional complexity of the manipulation, and other uncertain parameters such as mass, CoM location, and robot contact location. 
The work in \cite{Dafle_Extrinsic2014} discusses dexterous in-hand manipulation including extrinsic contact.
% Also, authors in \cite{Cruciani_inhand2017} 
However, the work in \cite{Dafle_Extrinsic2014} does not consider uncertainty in physical parameters.
}
Other previous works that study stable pivoting also consider sticking contact during pivoting using multiple points of contact~\cite{hou2018fast}. The problem in~\cite{hou2018fast} is inherently stable as the object is always in stable grasp. Furthermore, the authors do not consider any uncertainty during planning. Similarly, authors in~\cite{aceituno2020global} present a mixed integer programming formulation to generate contact trajectory given a desired reference trajectory for the object for several manipulation primitives. 
In contrast, this work proposes a bilevel optimization technique which maximizes the minimum margin from instability that the object experiences during an entire trajectory.
Another related work is presented in~\cite{han2020local} where the authors study the feedback control during manipulation of a half-cylinder. The idea there is to design a reference trajectory and then use a local controller by building a funnel around the reference trajectory by linearizing the dynamics. The online control is computed by solving linear programs to locally track the reference trajectory.  

From the above discussion, we can arrive at the following conclusion. In contrast to most of the related work, this proposed work presents a novel formulation for two-point pivoting which requires slipping contact formation between the object and the environment. Furthermore, in comparison to most of the work on contact implicit trajectory optimization, we present a contact implicit bilevel optimization (CIBO) for robust trajectory optimization for manipulation. Even though this method is illustrated on a particular pivoting manipulation problem in this paper, the proposed optimization algorithm could be used for other robust manipulation problems based on the mechanics of the manipulation task. 


% \revise{A two-phase gripper to reorient and
% grasp. Propose the design of the gripper. Maybe just cite. }

% \revise{Extrinsic Dexterity: In-Hand Manipulation with External Forces. Using extrinsic dexterity, the robot can still do very cool motion. No discussion in robustness. Extrinsic contact }

% \revise{In-hand manipulation using
% three-stages open loop pivoting. Use q-learning. But in open loop. no discussion on robustness, extrinsic contact. }


% Vision-based tactile sensors have received a lot of attention in the recent years~\cite{donlon2018gelslim,li2020f}. These vision-based tactile sensors can observe very high resolution contact-patch formed at these sensors when an object is grasped at the fingers by a gripper. These high resolution images can then be used for several different tasks. These sensors have been used to perform various estimation tasks using these visuo-tactile sensors. For example, incipient slip detection for grasped objects is a basic requirement for 
