\section{Robust Pivoting Formulation}\label{sec:sec_formulation}
% \subsection{Robust Pivoting Formulation}\label{subsec:robust_pivoting} 
In this section, we present a generic formulation for robust pivoting manipulation. In particular, we use the \revise{quasi-static} equilibrium conditions~\eqref{force_eq} in the presence of disturbances to formulate the robust planning problem. In particular, using sufficiency for stability of the object during manipulation we can estimate the bound of disturbance that can be tolerated during manipulation. Since this bound would depend on the pose of the object, we reason about the margin throughout the manipulation trajectory during the optimization problem formulation. We present the general idea in the following paragraph.

In the most general case, we assume that there is an external force $F_{ext}^{\revise{B}}$ and moment $M_{ext}^{\revise{B}}$ acting on the object during manipulation. Let us assume that the $x$ and $y$ component of the external force \revise{in $F_B$} are represented as $F_{ext,x}^{\revise{B}}$ and $F_{ext,y}^{\revise{B}}$ respectively. Then the \revise{quasi-static} equilibrium conditions~\eqref{force_eq} can be rewritten as follows:
\begin{subequations}
\begin{flalign}
 f_{nA}^{\revise{B}} + f_{tB}^{\revise{B}} + f_{xP}^{\revise{B}}+F_{ext,x}^{\revise{B}}  =0,\label{robust_forceeq1}\\
f_{tA}^{\revise{B}} + f_{nB}^{\revise{B}} + mg + f_{yP}^{\revise{B}}+F_{ext,y}^{\revise{B}}   = 0,  \label{robust_forceeq2}\\
A_x^{\revise{B}}f_{tA}^{\revise{B}} - A_y^{\revise{B}}f_{nA}^{\revise{B}} + C_x^{\revise{B}}mg + P_x^{\revise{B}}f_{yP}^{\revise{B}} 
- P_y^{\revise{B}} f_{xP}^{\revise{B}}  \nonumber \\+M_{ext}^{\revise{B}} = 0 
\label{robust_moment_eq1}
% -\frac{l_\text{com}}{2}c_{\theta - \gamma}f_{tA} + \frac{l_\text{com}}{2}s_{\theta - \gamma}f_{nA} -\frac{l_\text{com}}{2}c_{\theta + \gamma}f_{nb} +\frac{l_\text{com}}{2}s_{\theta + \gamma}f_{tb}
% (A_x-B_x)f_{tA} - (A_y-B_y)f_{nA} + (C_x-B_x)mg + (P_x-B_x)f_{y} - (P_y - B_y) f_x = 0
\end{flalign}
\label{robust_force_eq}
\end{subequations}
Note that $F_{ext}^{\revise{B}}$ and $M_{ext}^{\revise{B}}$ may not be independent of each other. They are related via the the point of application of force $F_{ext}^{\revise{B}}$ in the \revise{quasi-static} equilibrium conditions~\eqref{robust_force_eq}. These equations may not be satisfied for all possible values of $F_{ext}^{\revise{B}}$ and $M_{ext}^{\revise{B}}$. Since the contact forces can be readjusted in~\eqref{robust_force_eq}, the \revise{quasi-static} equilibrium can be satisfied for a certain range of $F_{ext}^{\revise{B}}$ and $M_{ext}^{\revise{B}}$. A generic analysis for estimating this margin or bound for which these disturbances can be compensated by contact forces is a bit involved as such a bound is dependent on the point and angle of application of the external force $F_{ext}^{\revise{B}}$. In the following sections, we present some specific cases which can be analyzed by making some simplifying assumptions on these disturbances. 
\revise{For brevity, we omit superscript $B$ of variables in the following sections because we consider \revise{quasi-static} equilibrium in $F_B$ unless we consider \revise{quasi-static} equilibrium in a different frame (see Sec~\ref{sec:stability_margin_finger_contact_location}). }
% \eq{slippingP} means that $\dot{p}_y$ is non-zero only at the boundary of friction cone.

% Additionally, one can have the following complementarity constraints to discuss the case where the object can lose contact at point B and can incur rotation about point A: 
% \begin{equation}
%  0 \leq   B_y \perp   f_{nB} \geq 0 
%  \label{complementarity_in_air}
% \end{equation}
% f_{tA} + f_{nB} + mg + f_{yP}   = 0,  \label{forceeq2}\\
% A_xf_{tA} - A_yf_{nA} + C_xmg + P_xf_{y} - P_y f_x = 0
% -\frac{l_\text{com}}{2}c_{\theta - \gamma}f_{tA} + \frac{l_\text{com}}{2}s_{\theta - \gamma}f_{nA} -\frac{l_\text{com}}{2}c_{\theta + \gamma}f_{nb} +\frac{l_\text{com}}{2}s_{\theta + \gamma}f_{tb}
% (A_x-B_x)f_{tA} - (A_y-B_y)f_{nA} + (C_x-B_x)mg + (P_x-B_x)f_{y} - (P_y - B_y) f_x = 0

% In this work, we do not consider \eq{complementarity_in_air} during optimization since \eq{complementarity_in_air} leads to stochastic complementarity system.

\subsection{Frictional Stability Margin}


%Otherwise, the manipulation may not be able to be completed due to uncertain parameters. However, in reality, uncertainty always exists and the manipulation can be completed under uncertainty to some extent. This is because friction forces inherently compensate for uncertainty as long as they satisfy static equilibrium of force and moments and friction cone constraints. the body can lift up by sticking at point $A$.  Hence, in this case, we can imagine that 

% In model-based manipulation, it is important to have precise estimate of physical parameters. However, it is desirable that a robot can compensate for uncertainty during manipulation of novel objects. 
% In this paper, we provide some insights about how a robot can use the stability margin in static equilibrium during manipulation to compensate for uncertainty in gravitational forces and moments. \fig{fig:concept} shows our proposed concept of frictional stability and bilevel robust trajectory optimization. The latter is discussed in Sec~\ref{sec:robust_to}. 
The robust quasi-static equilibrium conditions shown in~\eqref{robust_force_eq} can be used to explain the concept of stability margin. The stability margin is given by the magnitude of the external force $F_{ext}^{\revise{B}}$ and moment $M_{ext}^{\revise{B}}$ which can be satisfied in~\eqref{robust_force_eq} in any stable configuration of the object. This margin would depend on the contact force between the object and the environment as well as the control force used by the manipulator during the task. This provides the intuition that one can design a control trajectory such that the stability margin can be maximized.


We briefly provide some physical intuition about frictional stability for a few specific cases. First suppose that uncertainty exists in mass of a body. In the case when the actual mass is lower than  estimated, the friction force at point $A$ would increase while the friction force at point $B$ would decrease, compared to the nominal case. In contrast, suppose if the actual mass of the body is heavier than that of what we estimate, then the body can tumble along point $B$ in the clockwise direction. In this case, we can imagine that the friction force at point $A$ would decrease while the friction force at point $B$ would increase. However, as long as the friction forces are non-zero, the object can stay in contact with the external environment.
Similar arguments could be made for uncertainty in CoM location. The key point to note that the friction forces can re-distribute at the two contact locations and thus provide a margin of stability to compensate for uncertain gravitational forces and moments. We call this margin as \textit{frictional stability}.

In the following sections, we present the mathematical formulation of \textit{frictional stability} for cases when the mass, CoM location, friction coefficients, or \revise{finger contact location} are not known perfectly.
% In \fig{fig:concept},  if the actual CoM location is more left (represented as light-blue) than that of what we estimate (represented as black), then the body can lift up by sticking at point $A$.  Hence, in this case, we can imagine that the friction force at point $A$ would increase while the friction force at point $B$ would decrease. In contrast, if the actual CoM location is more right, then the body can tumble along point $B$ in the clockwise direction. In this case, we can imagine that the friction force at point $A$ would decrease while the friction force at point $B$ would increase. We could provide a similar insight when the body has uncertain CoM location. 

% In the next few subsections, we formalize the intuition for frictional stability margin proposed above and derive sufficient condition for stability of the object during pivoting.

%ly formulate frictional stability and confirm that what we describe above is actually confirmed mathematically. Here we consider the body with uncertain mass.

\subsection{Stability Margin for Uncertain Mass}\label{sec:sec_uncertain_mass}
 For simplicity, we denote $\epsilon$ as uncertain weight with respect to the estimated weight. Also, to emphasize that we consider the system under uncertainty, we put superscript $\epsilon$ for each friction force variable. Thus, the \revise{quasi-static} equilibrium conditions in \eq{force_eq} can be rewritten as:
\begin{subequations}
\begin{flalign}
f_{nA}^\epsilon + f_{tB}^\epsilon + f_{xP}  =0\label{forceeq11},\\
f_{tA}^\epsilon + f_{nB}^\epsilon + (mg + \epsilon) + f_{yP}   = 0,  \label{forceeq21}\\
A_xf_{tA}^\epsilon - A_yf_{nA}^\epsilon + C_x (mg+\epsilon) + P_xf_{yP}  = P_y f_{xP} \label{moment_eq11}
\end{flalign}
\label{force_eq_mass}
\end{subequations}
% equation with super script B
% \begin{subequations}
% \begin{flalign}
% f_{nA}^\epsilon + f_{tB}^\epsilon + f_{xP}^{\revise{B}}  =0\label{forceeq11},\\
% f_{tA}^\epsilon + f_{nB}^\epsilon + (mg + \epsilon) + f_{yP}^{\revise{B}}   = 0,  \label{forceeq21}\\
% A_x^{\revise{B}}f_{tA}^\epsilon - A_y^{\revise{B}}f_{nA}^\epsilon + C_x^{\revise{B}} (mg+\epsilon) + P_x^{\revise{B}}f_{yP}^{\revise{B}}  = P_y^{\revise{B}} f_{xP} \label{moment_eq11}
% % -\frac{l_\text{com}}{2}c_{\theta - \gamma}f_{tA} + \frac{l_\text{com}}{2}s_{\theta - \gamma}f_{nA} -\frac{l_\text{com}}{2}c_{\theta + \gamma}f_{nb} +\frac{l_\text{com}}{2}s_{\theta + \gamma}f_{tb}
% % (A_x-B_x)f_{tA} - (A_y-B_y)f_{nA} + (C_x-B_x)mg + (P_x-B_x)f_{y} - (P_y - B_y) f_x = 0
% %In order to realize the robustness, the more contact points are preferred since the system obtains more frictional stability. Therefore, we aim at ensuring contacts during pivoting (i.e., slipping). 
% \end{flalign}
% \label{force_eq_mass}
% \end{subequations}
Then, using \eq{slipping_friction_cone} and \eq{moment_eq11}, we obtain:
\begin{equation}
f_{nA}^\epsilon = \frac{-{C_x}\left(mg + \epsilon\right) -{P_x}f_{yP} + {P_y}f_{xP}}{\mu_A {A_x} - {A_y}}
\label{fna_111}
\end{equation}
To ensure that the body maintains contact with the external surfaces, we would like to enforce that the body experience non-zero normal forces at the both contacts.
To realize this, we have $f_{nA}^\epsilon \geq 0, f_{nB}^\epsilon \geq 0$ as conditions that the system needs to satisfy. Consequently, by simplifying \eq{fna_111}, we get the following:
\begin{subequations}
\begin{flalign}
\epsilon \geq \frac{P_yf_{xP} - P_xf_{yP} - C_xmg}{C_x}, \text{ if } C_x>0,  \label{fnacond1}\\
\epsilon \leq \frac{P_yf_{xP} - P_xf_{yP} - C_xmg}{C_x}, \text{ if } C_x<0  \label{fnacond2}
% -\frac{l_\text{com}}{2}c_{\theta - \gamma}f_{tA} + \frac{l_\text{com}}{2}s_{\theta - \gamma}f_{nA} -\frac{l_\text{com}}{2}c_{\theta + \gamma}f_{nb} +\frac{l_\text{com}}{2}s_{\theta + \gamma}f_{tb}
% (A_x-B_x)f_{tA} - (A_y-B_y)f_{nA} + (C_x-B_x)mg + (P_x-B_x)f_{y} - (P_y - B_y) f_x = 0
\end{flalign}
\label{fna_cond_mass}
\end{subequations}
Note that the upper-bound of $\epsilon$ means that the friction forces can exist even when we make the mass of the body lighter up to $\frac{\epsilon}{g}$. The lower-bound of $\epsilon$ means that the friction forces can exist even when we make the mass of the body heavier up to $\frac{\epsilon}{g}$. 
\eq{fna_cond_mass} provides some useful insights. \eq{fna_cond_mass} gives either upper- or lower-bound of $\epsilon$ for $f_{nA}^\epsilon$ according to the sign of $C_x$ (the moment arm of gravity). This is because the uncertain mass would generate an additional moment along with point $B$ in the clock-wise direction if $C_x >0$ and in the counter clock-wise direction if $C_x <0$. 
% In other words, the heavier body can make itself tumble and lose contact at point $A$ so it is not preferred if $C_x>0$, and lighter body can make itself again tumble and lose contact at point $A$ so it is not preferred if $C_x<0$  based on $f_{nA}^\epsilon \geq 0$. 
If $C_x = 0$, we have an unbounded range for $\epsilon$, meaning that the body would not lose contact at point $A$ no matter how much uncertainty exists in the mass. 

\eq{fna_cond_mass} can be reformulated as an inequality constraint: 
\begin{equation}
 C_x(\epsilon - \epsilon_A) \geq 0
% f_{tA} + f_{nB} + mg + f_{yP}   = 0,  \label{forceeq2}\\
% A_xf_{tA} - A_yf_{nA} + C_xmg + P_xf_{y} - P_y f_x = 0
% -\frac{l_\text{com}}{2}c_{\theta - \gamma}f_{tA} + \frac{l_\text{com}}{2}s_{\theta - \gamma}f_{nA} -\frac{l_\text{com}}{2}c_{\theta + \gamma}f_{nb} +\frac{l_\text{com}}{2}s_{\theta + \gamma}f_{tb}
% (A_x-B_x)f_{tA} - (A_y-B_y)f_{nA} + (C_x-B_x)mg + (P_x-B_x)f_{y} - (P_y - B_y) f_x = 0
\label{fna_cond_mass_one}
\end{equation}
where $\epsilon_A = \frac{P_yf_{xP} - P_xf_{yP} - C_xmg}{C_x}$.

We can derive condition for $\epsilon$ based on $f_{nB}^\epsilon \geq 0$ from \eq{slipping_friction_cone}, \eq{forceeq11}, and \eq{forceeq21}:
\begin{equation}
 \epsilon \leq \mu_A f_{xP} -f_{yP} -mg
% f_{tA} + f_{nB} + mg + f_{yP}   = 0,  \label{forceeq2}\\
% A_xf_{tA} - A_yf_{nA} + C_xmg + P_xf_{y} - P_y f_x = 0
% -\frac{l_\text{com}}{2}c_{\theta - \gamma}f_{tA} + \frac{l_\text{com}}{2}s_{\theta - \gamma}f_{nA} -\frac{l_\text{com}}{2}c_{\theta + \gamma}f_{nb} +\frac{l_\text{com}}{2}s_{\theta + \gamma}f_{tb}
% (A_x-B_x)f_{tA} - (A_y-B_y)f_{nA} + (C_x-B_x)mg + (P_x-B_x)f_{y} - (P_y - B_y) f_x = 0
\label{fnb_cond_mass}
\end{equation}
We only have upper-bound on $\epsilon$ based on $f_{nB}^\epsilon \geq 0$, meaning that the contact at point $B$ cannot be guaranteed if the actual mass is lighter than $\mu_A f_{xP} -f_{yP} -mg$. 
%Note that the conditions  we derive in this paper are sufficient but not necessary conditions.



\subsection{Stability Margin for Uncertain CoM Location}\label{sec:stability_margin_com_location}
% We consider the case with uncertain CoM location. We have a similar discussion we have in Sec~\ref{sec_uncertain_mass}. 
We denote $\revise{d_x^{{O}}, d_y^{{O}}}$ as residual CoM locations with respect to the estimated CoM location in $F_{\revise{O}}$ coordinate, respectively. Thus, the residual CoM location in $\revise{F_W}$, $\revise{d_x^{{W}}, d_y^{{W}}}$, are represented by $\revise{d_x^{{W}}} = d \cos({\theta + \theta_d}), \revise{d_y^{{W}}} = d \sin({\theta + \theta_d})$, where $d = \sqrt{\revise{\left({d_x^{{O}}}\right)^2 + \left({d_y^{{O}}}\right)^2}}$, $\theta_d = \arctan{\frac{\revise{{d_y^{{O}}}}}{\revise{{d_x^{{O}}}}}}$.  For notation simplicity, we use $r$ to represent $\revise{d_x^{{W}}}$.  In this paper, we put superscript $r$ for each friction force variable. The \revise{quasi-static} equilibrium conditions in \eq{force_eq} can be rewritten as follows:
\begin{subequations}
\begin{flalign}
f_{nA}^r + f_{tB}^r + f_{xP}  =0\label{forceeq12},\\
f_{tA}^r + f_{nB}^r + mg + f_{yP}   = 0,  \label{forceeq22}\\
A_xf_{tA}^r - A_yf_{nA}^r + (C_x + r) mg + P_xf_{yP}  = P_y f_{xP}  \label{moment_eq12}
\end{flalign}
\label{force_eq_location}
\end{subequations}
Then, using \eq{slipping_friction_cone} in \eq{force_eq_location}, we obtain:
\begin{subequations}
\begin{flalign}
r \leq \frac{P_yf_{xP} -P_x f_{yP}}{mg} - C_x \label{fna_fnb_r1},\\
r \geq  - \frac{\frac{\mu_A A_x - A_y}{1 + \mu_A}(-f_{xP}-f_{yP}-mg) -P_yf_{xP} + P_xf_{yP}}{mg}  -C_x  \label{fna_fnb_r2}
% -\frac{l_\text{com}}{2}c_{\theta - \gamma}f_{tA} + \frac{l_\text{com}}{2}s_{\theta - \gamma}f_{nA} -\frac{l_\text{com}}{2}c_{\theta + \gamma}f_{nb} +\frac{l_\text{com}}{2}s_{\theta + \gamma}f_{tb}
% (A_x-B_x)f_{tA} - (A_y-B_y)f_{nA} + (C_x-B_x)mg + (P_x-B_x)f_{y} - (P_y - B_y) f_x = 0
\end{flalign}
\label{fna_fnb_r}
\end{subequations}
where \eq{fna_fnb_r1}, \eq{fna_fnb_r2} are obtained based on $f_{nA}^r \geq 0, f_{nB}^r \geq 0$, respectively. \eq{fna_fnb_r} means that the object would lose contact at $A$ if the actual CoM location is more to the right than our expected CoM location while the object would lose the contact at $B$ if the actual CoM location is more to the left.

\subsection{Stability Margin for Stochastic Friction}\label{subsec:stochasticfriction_planning}
In this section, we present modeling and analysis of pivoting manipulation in the presence of stochastic friction coefficients. In particular, we consider stochastic friction at the two different contact points $A$ and $B$. We do not consider stochastic friction at the contact point between the robot and the manipulator since that leads to stochastic complementarity constraints (please see~\cite{shirai2023covariance, yuki2021chance} for detailed analysis on stochastic complementarity constraints). 
We make the assumption that the friction coefficients at $A$ and $B$ are partially known. In particular, we assume that the friction coefficients for contact at $A$ could be represented as $\mu_A=\hat{\mu}_A+\revise{\tilde{\mu}_A}$
where $\revise{\tilde{\mu}_A}$ is the uncertain stochastic variable.
% $\revise{\tilde{\mu}_A}\sim\mathcal N(0,\sigma_{\mu_A}^2)$. 
Similarly, the friction coefficient at $B$ could be represented as  $\mu_B=\hat{\mu}_B+\revise{\tilde{\mu}_B}$ where
  $\revise{\tilde{\mu}_B}$ is the uncertain stochastic variable. Note that we do not need to need to know any information regarding the \revise{probabilistic} distribution \revise{(e.g., probability density function of Gaussian distribution, beta distribution.)} of the unknown part. 
% $\revise{\tilde{\mu}_B}\sim \mathcal N(0,\sigma_{\mu_B}^2)$. 
We can rewrite~\eqref{robust_force_eq} for this case as follows. We put superscript $\mu$ for each friction variable:
\begin{subequations}
\begin{flalign}
 f_{nA}^\mu + \hat{f}_{tB}^\mu + f_{xP}+\epsilon_B  =0,\label{robust_forceeq_friction_friction}\\
\hat{f}_{tA}^\mu + f_{nB}^\mu + mg + f_{yP}+\epsilon_A   = 0,  \label{robust_forceeq_friction_friction}\\
A_x\hat{f}_{tA}^\mu+A_x\epsilon_A - A_yf_{nA}^\mu + C_xmg  \nonumber \\+ P_xf_{yP} - P_y f_{xP}=  0\label{robust_moment_eq_friction_friction}
\end{flalign}
\label{robust_force_eq_friction_friction}
\end{subequations}
where, $f_{tA}^\mu=\hat{f}_{tA}^\mu+f_{nA}^\mu\revise{\tilde{\mu}_A}$ and $f_{tB}^\mu=\hat{f}_{tB}^\mu+f_{nB}^\mu\revise{\tilde{\mu}_B}$. The above equations are obtained by representing $f_{nA}\revise{\tilde{\mu}_A}$ as $\epsilon_A$ for contact at $A$ and similarly, $\epsilon_B$ for the contact at $B$. Thus, $\epsilon_A$ and $\epsilon_B$ are the uncertain contact forces for the contacts at $A$ and $B$. The robust formulation that we consider in this paper considers the worst-case effect of these uncertainties on the stability of the object during manipulation. Thus, we try to maximize the bound of these variables $\epsilon_A$ and $\epsilon_B$ using our proposed bilevel optimization. It is noted that $\epsilon_A$ and $\epsilon_B$ are the stability margin for this particular case of stochastic friction.

To ensure that the body maintains contact, we impose $f_{nA}^\mu \geq 0, f_{nB}^\mu \geq 0$, so that we get the following inequalities for $\epsilon_A, \epsilon_B$:
\begin{subequations}
\begin{flalign}
-\mu_Af_{xP}  + \epsilon_A + mg + f_{yP} \leq \mu_A\epsilon_B
\\
\epsilon_B \leq -\mu_B (\epsilon_A + mg + f_{yP})- f_{xP}
 \label{moment_eq12_condition}
\end{flalign}
\label{force_eq_location_friction_condition}
\end{subequations}
To ensure slipping contact even in the presence of uncertainties, we need to satisfy friction cone constraints specified earlier in~\eqref{general_FC}, \eq{slipping_friction_cone}. Using these constraints, we can find the upper and lower bound for the variables $\epsilon_A$ and $\epsilon_B$: 
\begin{subequations}
\begin{flalign}
(\hat{\mu}_A+\revise{\tilde{\mu}_A}) f_{nA}^\mu = \hat{f}_{tA}^\mu+\revise{\tilde{\mu}_A}f_{nA}^\mu \\
(\hat{\mu}_B+\revise{\tilde{\mu}_B}) f_{nB}^\mu = -\hat{f}_{tB}^\mu-\revise{\tilde{\mu}_B}f_{nB}^\mu
\end{flalign}
\label{eq:sliping_eq_friction_uncertainty}
\end{subequations}
To get a lower bound for the variables $\epsilon_A$ and $\epsilon_B$, we make a assumption regarding the uncertainty for the friction coefficients at $A$ and $B$. We assume that the unknown part is bounded above by the known part, i.e., $\revise{\tilde{\mu}_i}\leq \hat{\mu}_i$, $\forall i=A,B$. Note that this is not a restrictive assumption. What this implies is that the above parameter has bounded uncertainty. For simplicity, we assume that uncertainty is bounded by the known part of the parameter. For example, if the friction coefficient is modeled as a stochastic random variable, then we assume that we know the mean of the friction parameter and the standard deviation is bounded by some multiple of mean (note that this bound is just for simplification and one can assume any practical bound for uncertainty).
% In practice, the variances of $\revise{\tilde{\mu}_A}$ and $\revise{\tilde{\mu}_B}$ are relatively small. Thus, we can argue that each realization of $\revise{\tilde{\mu}_A}$ and $\revise{\tilde{\mu}_B}$ are much smaller than $\nu_{\mu_A}$ and $\nu_{\mu_B}$, respectively.  
Consequently, we can derive the following relations:
\begin{subequations}
\begin{flalign}
-\hat{\mu}_A f_{nA}^\mu\leq \epsilon_A \leq \hat{\mu}_A f_{nA}^\mu
\\
-\hat{\mu}_B f_{nB}^\mu\leq \epsilon_B \leq \hat{\mu}_B f_{nB}^\mu
\end{flalign}
\label{eq:sliping_eq_friction_uncertainty_simple}
\end{subequations}
Thus, we get constraints~\eqref{force_eq_location_friction_condition} and~\eqref{eq:sliping_eq_friction_uncertainty_simple} for the stability margin by considering the stability and the friction cone constraints in the presence of uncertain friction coefficients. These constraints are used to estimate the stability margin during the proposed bilevel optimization.


\subsection{\textcolor{blue}{}
Stability Margin for Finger Contact Location}\label{sec:stability_margin_finger_contact_location}
\revise{
We consider another case of uncertainty which might arise due to an imperfect robot controller or due to imperfect pose information of the object. For this case, we consider the stability margin $d$ of finger contact location on an object, as illustrated in \fig{fig:mechanics_pivoting_finger_margin}. There could be multiple reasons for this uncertainty. One possible reason could be due to imperfect state information for the object being manipulated which can lead to imprecise information about the finger contact location. Another reason could be imprecise stiffness controller of the robot. It is noted that we use a stiffness controller for a position controlled robot to  implement the computed force trajectory. Due to compliance of the object and the robot, the actual robot trajectory is different from the planned and thus, this could lead to this uncertainty. 
% We consider the stability margin  due to imperfect stiffness controller from robotic manipulators. 
We can formulate the following \revise{quasi-static} equilibrium in $F_O$.  We put superscript $d$ for each extrinsic friction variable:
\begin{subequations}
\begin{flalign}
f_{xA}^{O, d} + f_{xB}^{O, d} + mg\sin{\theta} + f_{nP}^O  =0\label{forceeq12_finger},\\
f_{yA}^{O, d} + f_{yB}^{O, d} + mg\cos{\theta} + f_{tP}^O  =0\label{forceeq22_finger}\\
\sum_{i\in\{A, B\}}\left(
i_x^O f_{yi}^{O, d} - i_y^O f_{xi}^{O, d}
\right)
% A_x^O f_{yA}^{O, d} - A_y^O f_{xA}^{O, d} + B_x^O f_{yB}^{O, d} - B_y^O f_{xB}^{O, d} 
 \nonumber \\+P_x^O f_{tP}^O - (P_y^O + d) f_{nP}^O = 0  \label{moment_eq12_finger}
\end{flalign}
\label{force_eq_location_finger}
\end{subequations}
Note that $-A_x^O = -B_x^O = P_x^O =  \frac{l}{2}, A_y^O = -B_y^O = \frac{w}{2}$. Using this relation, we can simplify \eq{force_eq_location_finger}. In particular, we use $f_{xA}^{O, d} \geq 0,  f_{xB}^{O, d} \geq 0,  f_{nP}^{O} \geq 0$ and thus we can get the following bound for $d$:
\begin{subequations}
\begin{flalign}
 \underline{d} \leq d \leq \bar{d} 
\label{forceeq12_finger_margin},\\
\underline{d} = -A_x\frac{mg\cos{\theta} + 2f_{tP}}{f_{nP}}   - A_y \frac{mg\sin{\theta} + f_{nP}}{f_{nP}}  -P_y^O,\\
\bar{d} = -A_x\frac{mg\cos{\theta} + 2f_{tP}}{f_{nP}}   + A_y \frac{mg\sin{\theta} + f_{nP}}{f_{nP}}  -P_y^O
\label{forceeq22_finger_margin}
\end{flalign}
\label{force_eq_location_finger_margin}
\end{subequations}
}
\revise{
When $f_{nP}^O \rightarrow 0$, the equation suggests that $\bar{d}$ tends to infinity and $\underline{d}$ tends to negative infinity. As $f_{nP}^O = 0$ implies no force at point $P$, the finger's placement becomes inconsequential as it does not affect the \revise{quasi-static} equilibrium of the object.}

\revise{
We can consider that uncertainty in finger contact location and uncertainty in the geometry of an object have a similar influence on the manipulation. This is because the relative pose of the object with respect to the robot changes for both cases, resulting in the potential contact mode changes.  
}
% As 
% Consider the case where $f_{np} \rightarrow
%  0$. In this case, \eq{force_eq_location_finger_margin} indicate that $\bar{d}\rightarrow
% \infty$ and $\underline{d}\rightarrow
% -\infty$. Because $f_{np} = 0$ means there is no force at point $P$, the finger can be placed anywhere since finger position does not matter to the static equilibrium of the object. 



\begin{figure}
    \centering    \includegraphics[width=0.4\textwidth]{Figures/finger_uncertainty.png} 
    \caption{\revise{
    A schematic showing the free-body diagram of a rigid body during pivoting manipulation. We consider the stability margin of finger location due to imperfect control of stiffness controller in a robotic manipulator.}}
    \label{fig:mechanics_pivoting_finger_margin}
\end{figure}

\revise{
\subsection{Stability Margin for Uncertain Mass on a Slope}\label{sec:sec_uncertain_mass_slope}
We consider the case where we tilt the two external walls by the angle of $\phi$. 
% as illustrated in \fig{fig:mechanics_pivoting_eq_slope}. 
Our discussion in Sec.~\ref{sec:sec_uncertain_mass} still holds. The only difference arises from gravity terms. Hence, the \revise{quasi-static} equilibrium conditions in $F_B$ can be rewritten as:
\begin{subequations}
\begin{flalign}
f_{nA}^\epsilon + f_{tB}^\epsilon + f_{xP} + (mg + \epsilon) \sin{\phi}  =0\label{forceeq11_slope},\\
f_{tA}^\epsilon + f_{nB}^\epsilon + f_{yP} + (mg + \epsilon) \cos{\phi}    = 0,  \label{forceeq21_slope}\\
A_xf_{tA}^\epsilon - A_yf_{nA}^\epsilon + \left(C_x \cos{\phi} -  C_y \sin{\phi}\right) (mg+\epsilon) \nonumber \\ + P_xf_{yP} - P_y f_{xP} = 0 \label{moment_eq11_slope}
\end{flalign}
\label{force_eq_mass_slope}
\end{subequations}
Following the same logic in Sec.~\ref{sec:sec_uncertain_mass}, we can get the following bound for the stability margin $\epsilon$ under uncertain mass when the object is on a slope: 
\begin{subequations}
\begin{flalign}
\epsilon \geq \frac{P_yf_{xP} - P_xf_{yP} - (C_x\cos{\phi}-C_y \sin{\phi}) mg}{C_x\cos{\phi}-C_y\sin{\phi}}, \nonumber \\ \text{ if } C_x\cos{\phi}>C_y\sin{\phi} \label{fnacond1_slope}\\
\epsilon \leq \frac{P_yf_{xP} - P_xf_{yP} - (C_x\cos{\phi}-C_y \sin{\phi}) mg}{C_x\cos{\phi}-C_y\sin{\phi}}, 
\nonumber \\ \text{ if } C_x\cos{\phi}<C_y\sin{\phi}  \label{fnacond2_slope}
\end{flalign}
\label{fna_cond_mass_slope}
\end{subequations}
As a result, \eq{fnacond1_slope} and \eq{fnacond2_slope} result in the following inequality constraint:
\begin{equation}
 \left(C_x\cos{\phi}-C_y\sin{\phi}  \right)(\epsilon - \epsilon_A) \geq 0
\label{eq:uncertain_mass_slope}
\end{equation}
where $\epsilon_A = \frac{P_yf_{xP} - P_xf_{yP} - (C_x\cos{\phi}-C_y \sin{\phi}) mg}{C_x\cos{\phi}-C_y\sin{\phi}}$. We also derive the bound on $\epsilon$ using $f_{nB}^\epsilon\geq 0$, \eq{fnacond1_slope}, and \eq{fnacond2_slope}:
\begin{equation}
 \left(\mu_A \sin{\phi} - \cos{\phi}  \right)\epsilon \geq f_{yP} -\mu_A f_{xP} 
\label{eq:uncertain_mass_slope_upper_bound}
\end{equation}
Note that the sign of $\mu_A \sin{\phi} - \cos{\phi}$ can change depending on the angle of slope. In this paper, we choose $\phi$ such that the sign of $\mu_A \sin{\phi} - \cos{\phi}$ does not change during manipulation. 
}

\revise{The discussion in this section for manipulation under uncertain mass on a slope can be easily extended with other uncertain parameters such as CoM location, friction, and finger contact location. }


\subsection{Pivoting with Patch Contact between the object and the manipulator}\label{subsec:Pivoting_manipulation}


\begin{figure}[t]
    \centering
    \includegraphics[width=0.35\textwidth]{Figures/patch.png} % 
    \caption{A schematic showing the free-body diagram of a rigid body during pivoting manipulation with patch contact. We approximate patch contact as two point contacts $P_1$ and $P_2$ with the same force distribution. We assume that $P_1$ always lies on the vertex of the object for this simplistic patch contact model. \revise{$s$ is the distance between point contact $P_1$ and $P_2$ along $y$-axis of $F_O$.}}
    \label{fig:mechanics_pivoting_patch_eq}
\end{figure}

In the previous sections, we considered point contact between the manipulator and the object. This could be potentially restrictive. Moreover, this may not be a realistic assumption when a robot is interacting with objects. 
In this section, we present a slightly modified formulation by considering patch contact between the object and the manipulator. We would like to analyze and understand how patch contact would compare against a point contact model for stability during pivoting manipulation. \fig{fig:mechanics_pivoting_patch_eq} shows the simplest patch contact model during the pivoting task we consider in this paper. Using this model, we can write the following quasi-static equilibrium:
\begin{subequations}
\begin{flalign}
 f_{nA} + f_{tB} + \revise{f_{xP_{1}}+ f_{xP_{2}}}   =0,\label{forceeq1}\\
 f_{tA} + f_{nB} + mg + \revise{f_{yP_{1}}+ f_{yP_{2}}}= 0,\label{forceeq2}\\
 A_xf_{tA}- A_yf_{nA} + C_xmg  \nonumber \\ + \sum_{i=1}^2 \left(P_{{i}_x} f_{yP_{i}} - P_{{i}_y} f_{xP_{i}}\right) = 0
% f_{tA} + f_{nB} + mg + 2f_{yP}= 0,  \label{forceeq2}\\
% A_xf_{tA} - A_yf_{nA} + C_xmg + \sum_{i=1}^2 \left(P_{ix} f_{y} - P_{iy} f_x\right) = 0 \label{moment_eq1}
\end{flalign}
\label{force_balance_patch_contact}
\end{subequations}
where $P_{{i}_x}, P_{{i}_y}$ represent $x$ and $y$ coordinate of $P_1$ and $P_2$ \revise{in $F_O$}, respectively. 
\revise{In this work, we assume that patch contact as two point contacts $P_1$ and $P_2$ as the same force distribution, which indicates that $f_{xP_{1}} = f_{xP_{2}}, f_{yP_{1}} = f_{yP_{2}}$.}  
\revise{$s$} is the distance between point contact $P_1$ and $P_2$ and \revise{$s$} is a decision variable, meaning that location of $P_2$ is a decision variable and can change over time. In this work, we assume that $P_1$ does not move over time, which simplifies the model of patch contact. 

Using the above \revise{quasi-static} equilibrium conditions with $f_{nA}\geq 0, f_{nB}\geq 0$, we can solve and find the upper and the lower bound of stability margin under the various uncertainties described earlier in the previous subsections. We will present some results in the later section using this formulation and compare them against the point contact formulation.

\revise{\textit{Remark 1}: The patch contact discussion in this section can be extended into the patch contact at extrinsic contact with sliding contacts. We can approximate the extrinsic patch contact as two-point contacts with the same force distribution. Then, we can formulate the quasi-static equilibrium and derive the bound of the stability margin.}
% based on the discussion in Sec~\ref{sec_uncertain_mass} and Sec~\ref{sec:stability_margin_com_location}.

% In particular, we consider two different models of the patch contact, and try to answer the following questions:
% \begin{enumerate}
%     \item Does patch contact model allow better feasibility for the trajectory optimization during pivoting compared to the point contact model?
%     \item How does the force trajectory using the patch contact model compare against the point contact model?
% \end{enumerate}





% \subsection{Stability Margin of Pivoting under Uncertain Mass and CoM Location}

% % % In this section, we describe the optimal control problem for stable pivoting. To formulate the problem, we first describe kinematics and dynamics of pivot as illustrated in \fig{fig:mechanics_pivoting_eq}. We have a rigid body with three contact points $A, B, P$ where point $P$ is a contact point where a robot exerts control forces. 
% % % % Also, we use the following notation to represent the position of $i$ contact point: $p_[]$

% % % We use the following notation frequently. $l_\text{com}$ is the length of diagonal of a rectangle, $l_\text{com} = \sqrt{w^2 + l^2}$. Also, for simplicity, we use the following notation: $\sin{\theta} = s_\theta, \cos{\theta} = c_\theta, \sin{(\theta \pm \gamma)} = s_{\theta\pm \gamma}, \cos{(\theta \pm \gamma)} = c_{\theta\pm \gamma}$. Also, we define the origin of coordinate in \fig{fig:mechanics_pivoting_eq} as $x_O, y_O$.

% % % \subsection{Kinematics}
% % % We derive the kinematics equation for \fig{fig:mechanics_pivoting_eq}. Assuming that the location of center of mass is same to the location of center of geometry, which is located at the middle of the object, we use 

% % % \begin{subequations}
% % % \begin{flalign}
% % %  x_\text{com}=\frac{1}{2}l_\text{com}c_{\theta-\gamma} + x_O ,\label{kinematics1}\\
% % % y_\text{com}=\frac{1}{2}l_\text{com}s_{\theta-\gamma} + wc_\theta +y_O \label{kinematics2}
% % % \end{flalign}
% % % \label{kinematics}
% % % \end{subequations}

% % % \subsection{Dynamics}
% % % Here, we derive the quasi-static dynamics equations for pivoting. Since we are interested in working on planar pivoting, we have two static force equilibrium along $x, y$ and one moment equilibrium equation, which is calculated at point B in this paper. 

% % % The static equilibrium equations are: 


% % % $\beta$ and $l_p$ are derived as follows. $l_p =\sqrt{p_x^2 + p_y^2}$. First we can create a triangle BOP and $\alpha  =\arctan \left(\frac{p_y}{p_x}\right)$ and using law of cosines, $c^2=l_p^2 + (ws_\theta)^2 - 2 l_p ws_\theta c_\alpha$. Using law of cosines again, $l_p^2 = (ws_\theta)^2 + c^2 - 2cws_\theta c_{\pi-\beta - \theta}$. By organizing this, we can get $c_{\beta + \theta} = l_p c_\alpha - ws_\theta$. Then, we can get $\beta = \arccos{(l_p c_\alpha - ws_\theta)} - \theta$.

% % % We have three equations but we have the following decision variables: $f_{nA}, f_{tA}, f_{nB}, f_{tB}, f_{xP}, f_{yP}, \theta, p_x, p_y$.

% % % % Discussion point: where do we want to have complementarity constraints? For example, if we consider slipping, like pushing example, we have the complementarity constraints like: 
% % % % \begin{equation}
% % % % \begin{aligned}
% % % %  0 \leq   \dot{p}_{y+} \perp \mu_p f_{\overrightarrow{n}}(t)-f_{\overrightarrow{t}}(t) \geq 0  \\
% % % %  0 \leq   \dot{p}_{y-} \perp \mu_p f_{\overrightarrow{n}}(t)+f_{\overrightarrow{t}}(t) \geq 0 
% % % %  \end{aligned}
% % % %  \label{eqn:compl_pushing}
% % % % \end{equation}

% % % % Another complementarity constraint would be the touching condition. For example, we observe that sometimes only one point (e.g., point A) makes a contact and the other point is in the air. In this case, point A is slipping and following the complementarity constraints based on \eq{eqn:compl_pushing} and point B is in the air which follows the complementarity constraints such as:
% % % % \begin{equation}
% % % % \begin{aligned}
% % % %  0 \leq   f_{nB} \perp \phi_B\geq 0
% % % %  \end{aligned}
% % % %  \label{eqn:compl_force_air}
% % % % \end{equation}
% % % % where $\phi_B$ is the distance from the point B to the object along $y$ axis. Eventually, we have duplicated complementarity constraints (e.g., is point B making contact? if yes, is it slipping to which direction?), which can be computationally challenging but an interesting problem.



% % % % Another discussion point: if one of the coefficient of frictions or normal force is very large, we can have the one point contact where one of the other contact is now in the air. So, for our case, maybe we can prevent it from happening by bounding the $f_{xP}, f_{yP}$.

% % % % Another point: lifting up the object can be easier than lifting down the object. 

% % % % \subsection{Dynamics for General Objects}
% % % % We describe the more general dynamics which does not depend on the shape of the object. 


% % \subsection{Stability Margin Discussion for Uncertain Mass}
% % See \fig{fig:mechanics_pivoting_eq} for the definition of notation. Here, we first introduce static equilibrium of force equation as:
% % \begin{subequations}
% % \begin{flalign}
% %  f_{nA} + f_{tB} + f_{xP}  =0,\label{forceeq10}\\
% % f_{tA} + f_{nB} + mg + f_{yP} + \epsilon   = 0,  \label{forceeq20}
% % \end{flalign}
% % \label{force_eqm}
% % \end{subequations}
% % where $\epsilon$ is the stability margin based on inaccurate mass. 

% % % Also, along point O, we have the following static equilibrium of moment equation. For simplicity, we set $Q_{By} = Q_{Ax}=0$.
% % % \begin{subequations}
% % % \begin{flalign}
% % %  Q_{Bx}f_{nB} - Q_{Ay}f_{nA} + Q_{Cx}(mg+\epsilon) + Q_{Px} f_{Py} - Q_{Py}f_{Px}  =0,\label{momenteq10}
% % % \end{flalign}
% % % \label{moment_eqm}
% % % \end{subequations}
% % % Using slipping condition $f_{tB} = -\mu_B f_{nB}$ and \eq{forceeq10}, we can represent $f_{nA}$ as a function of $\epsilon$ given $mg, f_{Px}, f_{Py}$ as follows:
% % % \begin{equation}
% % % f_{nA} = \frac{-(Q_{Cx}(mg+\epsilon) + Q_{Px}f_{Py} + (\frac{Q_{Bx}}{\mu_B}-Q_{Py})f_{Px})}{\frac{ Q_{Bx}}{\mu_B} - Q_{Ay}},
% % % \label{fna}
% % % \end{equation}
% % % Then, 
% % % \begin{equation}
% % % f_{nB} = \frac{1}{\mu_B} \left(f_{Px} + \frac{-(Q_{Cx}(mg+\epsilon) + Q_{Px}f_{Py} + (\frac{Q_{Bx}}{\mu_B}-Q_{Py})f_{Px})}{\frac{ Q_{Bx}}{\mu_B} - Q_{Ay}}\right),
% % % \label{fna}
% % % \end{equation}

% % In order to have the robustness on the controller, we have $f_{nA} >0, f_{nB} >0$. Thus, from these equations, we can specify condition as:
% % % \begin{subequations}
% % % \begin{flalign}
% % % -Q_{Px}f_{Py} + Q_{Py}f_{Px} < Q_{Cx}(mg + \epsilon),\label{fcondition1}\\
% % % f_{Px} (Q_{Ax}\mu_A - Q_{Ay}) -Q_{Px}f_{Py} + Q_{Py}f_{Px} < Q_{Cx}(mg + \epsilon),  \label{fcondition2}
% % % \end{flalign}
% % % \label{force_condition}
% % % \end{subequations}
% % % Since we cannot specify the sign of $Q_{Cx}$, \eq{fcondition1} cannot be organized more. Also, since $Q_{Ax}\mu_A - Q_{Ay} <0$ and "likely" $f_{Px} <0$, $f_{Px} (Q_{Ax}\mu_A - Q_{Ay}) >0$. Therefore, \eq{fcondition2} is more tight than \eq{fcondition1}. So, if  \eq{fcondition2} is satisfied, \eq{fcondition1} is automatically satisfied (need to discuss)? This may not be true since the sign of $Q_{Cx}$ can change.

% % Also, along point B, we have the following static equilibrium of moment equation. For simplicity, we set $Q_{Bx} = Q_{By}=0$.
% % \begin{equation}
% %  Q_{Ax}f_{tA} - Q_{Ay}f_{nA} + Q_{Cx}(mg+\epsilon) + Q_{Px} f_{Py} - Q_{Py}f_{Px}  =0,\label{momenteq10}
% % \end{equation}
% % Using slipping condition $f_{tA} = \mu_A f_{nA}$, we can represent $f_{nA}$ as a function of $\epsilon$ given $mg, f_{Px}, f_{Py}$ as follows:
% % Here, using $f_{tB} = -\mu_B f_{nB}$ and \eq{forceeq10}, 
% % \begin{equation}
% % f_{nB} = \frac{\mu_Af_{Px} - f_{Py} - mg - \epsilon}{1 + \mu_A \mu_B},
% % \label{fnb}
% % \end{equation}


% % In order to have the robustness on the controller, we have $f_{nA} >0, f_{nB} >0$. Thus, from these equations, we can specify condition as:
% % \begin{subequations}
% % \begin{flalign}
% % -Q_{Px}f_{Py} + Q_{Py}f_{Px} < Q_{Cx}(mg + \epsilon),\label{fcondition1}\\
% % \epsilon < \mu_Af_{Px}-f_{Py} - mg,  \label{fcondition2}
% % \end{flalign}
% % \label{force_condition}
% % \end{subequations}
% % Since we cannot specify the sign of $Q_{Cx}$, \eq{fcondition1} cannot be organized more. Also, since $Q_{Ax}\mu_A - Q_{Ay} <0$ and "likely" $f_{Px} <0$, $f_{Px} (Q_{Ax}\mu_A - Q_{Ay}) >0$. Therefore, \eq{fcondition2} is more tight than \eq{fcondition1}. So, if  \eq{fcondition2} is satisfied, \eq{fcondition1} is automatically satisfied (need to discuss)? This may not be true since the sign of $Q_{Cx}$ can change.


% % Using the slipping condition at Point B $f_{tB} = -\mu_B f_{nB}$ and \eq{forceeq10}, 
% % \begin{equation}
% % f_{nB} = \frac{1}{\mu_B} \left(f_{Px} +  \frac{-Cx\left(mg + \epsilon\right) -Q_{Px}f_{Py} + Q_{Py}f_{Px}}{\mu_A Q_{Ax} - Q_{Ay}}\right),
% % \label{fna}
% % \end{equation}

% % In order to have the robustness on the controller, we have $f_{nA} >0, f_{nB} >0$. Thus, from these equations, we can specify condition as:
% % \begin{subequations}
% % \begin{flalign}
% % -Q_{Px}f_{Py} + Q_{Py}f_{Px} < Q_{Cx}(mg + \epsilon),\label{fcondition1}\\
% % f_{Px} (Q_{Ax}\mu_A - Q_{Ay}) -Q_{Px}f_{Py} + Q_{Py}f_{Px} < Q_{Cx}(mg + \epsilon),  \label{fcondition2}
% % \end{flalign}
% % \label{force_condition}
% % \end{subequations}
% % Since we cannot specify the sign of $Q_{Cx}$, \eq{fcondition1} cannot be organized more. Also, since $Q_{Ax}\mu_A - Q_{Ay} <0$ and "likely" $f_{Px} <0$, $f_{Px} (Q_{Ax}\mu_A - Q_{Ay}) >0$. Therefore, \eq{fcondition2} is more tight than \eq{fcondition1}. So, if  \eq{fcondition2} is satisfied, \eq{fcondition1} is automatically satisfied (need to discuss)? This may not be true since the sign of $Q_{Cx}$ can change.


% Next, let's consider the case where we have uncertainty $r$ as well as $\epsilon$. We use $r$ to represent the uncertainty of CoM location. This uncertain effect from $r$ does not show up in the force equilibrium equation. Then, we consider the moment equation such as: 
% \begin{equation}
% Q_{Ax}f_{tA} - Q_{Ay}f_{nA} + (Q_{Cx} + r)(mg+\epsilon) + Q_{Px} f_{Py} - Q_{Py}f_{Px}  =0,\label{momenteq10}
% \end{equation}
% It is worth pointing out that the term $(Q_{Cx} + r)(mg+\epsilon)$ results in non-convex term $r\epsilon$. Then you get:
% \begin{equation}
% f_{nA} = \frac{-(Q_{Cx}+ r)\left(mg + \epsilon\right) -Q_{Px}f_{Py} + Q_{Py}f_{Px}}{\mu_A Q_{Ax} - Q_{Ay}},
% \label{fna}
% \end{equation}
% Here, using $f_{tB} = -\mu_B f_{nB}$ and \eq{forceeq10}, 
% \begin{equation}
% f_{nB} = \frac{\mu_Af_{Px} - f_{Py} - mg - \epsilon}{1 + \mu_A \mu_B},
% \label{fnb}
% \end{equation}
% And the conditions for $\epsilon, r$ would be:
% \begin{subequations}
% \begin{flalign}
% -Q_{Px}f_{Py} + Q_{Py}f_{Px} < (Q_{Cx} + r)(mg + \epsilon),\label{fcondition1}\\
% \epsilon < \mu_Af_{Px}-f_{Py} - mg,  \label{fcondition2}
% \end{flalign}
% \label{force_condition}
% \end{subequations}

