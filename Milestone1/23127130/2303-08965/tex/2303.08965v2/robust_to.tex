\section{Robust Trajectory Optimization}\label{sec:robust_to}
% also want to say that we denote friction stability as pure function of \epsilon, making inner loop of bilevel optimization linear
%\color[red]{This section can be expanded and re-written to explain everything properly.}\\
% \devesh{This section can be expanded and explained in detail. May be consider and explain the case of patch contact as well as uncertainty due to friction. Also include the formulation for mode-based optimization. The uncertainty due to friction might be a bit different treatment than earlier. The other option to consider is to include recovery from failure using control.}\\
\begin{figure}[t]
    \centering
\includegraphics[width=0.49\textwidth]{Figures/drawing_cropped.pdf} % 
    \caption{Conceptual schematic of our proposed frictional stability and robust trajectory optimization for pivoting. Due to slipping contact, friction forces at points $A, B$ lie on the edge of friction cone. Given the nominal trajectory of state and control inputs, friction forces can account for uncertain physical parameters to satisfy \revise{quasi-static} equilibrium. We define the range of disturbances that can be compensated by contacts as frictional stability. The above figure shows the case of uncertain mass and CoM location.}
    \label{fig:concept}
\end{figure}
% concept.eps
% In our proposed optimization formulation, we maximize the worst frictional stability margin over the entire trajectory where we obtain the maximum frictional stability at each time step given $x, u$, leading to a bilevel optimization problem.
Using the notion of \textit{frictional stability} introduced in the previous section, we describe our proposed contact implicit bilevel optimization (CIBO) method for robust optimization of manipulation trajectories. The proposed method explicitly considers frictional stability under uncertain physical parameters. It is noted that the proposed method considers robustness under slipping contact which results in equality for friction cone constraints (see \fig{fig:concept}).
After describing the formulation for convex objects, we also describe how to extend the proposed CIBO to consider objects with non-convex geometry. Our proposed method is also presented as a schematic in \fig{fig:concept}. As shown in \fig{fig:concept}, the proposed CIBO considers frictional stability margin along the entire trajectory for manipulation and then maximizes the minimum margin in the proposed framework. This is also explained in \fig{fig:cibo}, where we show that we estimate the bound of stability margin in the lower level optimization and maximize the minimum margin in the upper level optimization. Before introducing our proposed bilevel optimization, we present a baseline contact-implicit TO which can be formulated as an MPCC. 

%\textcolor{red}{equality constraints}

%non-convex shape of objects and patch contact in this section.

\subsection{Contact-Implicit Trajectory Optimization for Pivoting}

% Here we first show our general formulation of optimization with uncertainty and show its robust counterpart. we first describe the optimization problem for trajectory generation during pivoting.
The purpose of our optimal control is to 
\revise{find optimal control input sequences under constraints for pivoting manipulation. In particular, we consider the objective function for achieving the minimum motion of objects under kinematics constrains, \revise{quasi-static} equilibrium, friction cone constraints, and sticking-slipping complementarity constraints as follows:}
% The purpose of our optimal control is to regulate the contact state and object state simultaneously given by:
% The purpose of our optimal control is to regulate the contact state and object state simultaneously. Our optimal control problem is:
% 
\begin{subequations}
\begin{flalign}
% \min _{x, u, \lambda}\sum_{k=0}^{N-1} \phi(x_k, u_k, \lambda_k)\\
\min _{x, u, f}  \sum_{k=1}^{N} ({x}_{k} - x_g)^{\top} Q ({x}_{k} - x_g)+\sum_{k=0}^{N-1}u_{k}^{\top} R u_{k} \\
\text{s. t. } i_{k, x}, i_{k, y} \in FK(\theta_k, \revise{P}_{k, y}^{\revise{O}}), \eq{force_eq}, \eq{slipping_friction_cone}, \eq{slippingP},  \label{const2}\\
% \eq{force_eq}, \eq{slipping_friction_cone}, \eq{slippingP}, \label{const3}\\
% \sum_{c=1}^{C} f_{k, c} + mg  = 0\\
% \sum_{c=1}^{C} \left(p_{k, c} - q_{k}\right) \times \lambda_{k, c}  = 0\\
x_{0} = x_s, x_{N} = x_g,
x_{k} \in \mathcal{X}, u_{k} \in \mathcal{U}, 0\leq f_{k, ni} \leq f_{u} \label{bounds_variables}
% f_{k, c, n} \geq 0 ,
%  0 \leq   \dot{p}_{k, c, j+} \perp \mu_c \lambda_{{k,c, n}}-\lambda_{k, c, t} \geq 0  \\
%  0 \leq   \dot{p}_{k, c, j-} \perp \mu_c \lambda_{{k,c, n}}+\lambda_{k, c, t} \geq 0 
\end{flalign}
\label{equation_control}
\end{subequations}
where $x_k = [\theta_k, \revise{P}_{k, y}^{\revise{O}}, \dot{\theta}_k, \revise{\dot{P}}_{k, y}^{\revise{O}}]^\top$, $u_k=[f_{k, nP}, f_{k, tP}]^\top$, $f_k = [f_{k, nA},f_{k, nB}]^\top$, $Q=Q^{\top} \geq 0,R=R^{\top} > 0$.
\revise{The input of \eq{equation_control} consists of physical parameters such as mass, length, and width of the object and the optimization parameters such as $Q$ and $R$. The output of  \eq{equation_control} consists of trajectories of $x_k, u_k, f_k, \forall k \in \{0, 1, \ldots, N\}$. }
We use explicit Euler to discretize the dynamics with sample time $\Delta$. The function $FK$ represents forward kinematics to specify each contact point $i$ and CoM location. $\mathcal{X}$ and $\mathcal{U}$ are convex polytopes, consisting of a finite number of linear inequality constraints.  $f_u$ is an upper-bound of normal force at each contact point. Note that we impose \eq{force_eq}, \eq{slipping_friction_cone} at each time step $k$. $x_s, x_g$ are the states at $k=0,k=N$, respectively.
%  Index $k, c, j$ represent time-step, contact point, appropriate slipping direction, respectively. $x_k$ is the decision variable of states including $q_k$, which is the $x, y, \theta$. $u_k = [\lambda_{k,p,n}, \lambda_{k,p,t}]^\top$ where $\lambda_{k,p,n}, \lambda_{k,p,t}$ are the normal and shear forces at the point P (the point where a robot touches). $\lambda_k$ has the rest of other friction forces at other contact points. 

% In this work, we work on the problem where uncertainty exists in $m, p_{k, c}, \mu_c $. The robust version of \eq{equation_control} can be formulated easily except for equality constraints. 

% \begin{enumerate}
% \item how to realize the control with state and contact force simultaneously? use additional cost?
% \item robust equality constraints
% \end{enumerate}
% The question is eventually what we want to do.

%of Pivoting Considering Frictional Stability
\subsection{Robust CIBO}\label{bilevel_sec}


\begin{figure}
    \centering    \includegraphics[width=0.49\textwidth]{Figures/cibo.png} % 
    \caption{This figure illustrates the idea of the proposed contact implicit bilevel optimization, CIBO. Given the trajectory of $x, u, f$, the stability margin over the trajectory can be computed as shown in lower-level optimization problem. Then, given the computed stability margin over the trajectory $\epsilon$, the upper-level optimization problem maximizes the worst-case stability margin over the trajectory by optimizing the trajectory of $x, u, f$. Our CIBO simultaneously optimizes the lower-level optimization problem and the upper-level optimization problem.
    In the right plot, red and blue arrows represent the stability margin along positive and negative directions, respectively. Our CIBO optimizes the stability margin for each direction. 
    }
    \label{fig:cibo}
\end{figure}


% As described in Sec~\ref{sec:mechanics}, considering frictional stability is critical for robust manipulation. % To robustify the trajectory optimization in \eq{equation_control}, we need to incorporate frictional stability in \eq{equation_control}. 
In this section, we present our formulation where we incorporate frictional stability in trajectory optimization to obtain robustness.
In particular, we first focus on discussing the optimization problem with uncertain mass, CoM location, and \revise{finger contact location}. We later discuss the optimization problem of uncertain coefficient of friction in Sec~\ref{subsec:opt_friction}.

An important point to note is that the optimization problem would be ill-posed if we naively add \eq{force_eq_mass}, \eq{force_eq_location}, and/or \revise{\eq{force_eq_location_finger_margin}} to \eq{equation_control} since there is no $u$ to satisfy all uncertainty realization in equality constraints \cite{yalmip_2018}.  
Therefore, our strategy is that we plan to find an optimal nominal trajectory that can ensure external contacts under uncertain physical parameters. 
% In particular, we here focus on discussing uncertain mass or CoM location uncertainty and we later discuss the optimization problem of uncertain coefficient of friction in Sec~\ref{subsec:opt_friction}.
In other words, we aim at maximizing the worst-case stability margin over the trajectory given the maximal frictional stability at each time-step $k$ (also shown in \fig{fig:concept}). Thus,  we maximize the following objective function:
% 
\begin{equation}
\min _{k} \epsilon_{k, +}^* - \max _{k} -\epsilon_{k, -}^*
\label{bilevel_obj}
\end{equation}
% 
where $\epsilon_{k, +}^*, \epsilon_{k, -}^*$ are non-negative variables. Note that $\epsilon_{k, +}^*, \epsilon_{k, -}^*$ are the largest uncertainty in the positive and negative direction, respectively, at instant $k$ given $x, u, f$, which results in non-zero contact forces (i.e., stability margin, see also \fig{fig:concept}).
% 
% that represents the magnitude of friction force (i.e., frictional stability margin) under $\epsilon_{k, +}^*, \epsilon_{k, -}^*$ along positive and negative directions of uncertainty at instant $k$ given $x, u, f$, respectively (see also Figure~\ref{fig:concept}).
% 
% the maximum stability margin along positive and negative direction of uncertainty at $k$ given $x, u, f$, respectively. 
\eq{bilevel_obj} calculates the smallest stability margin over time-horizons by subtracting the stability margin along the positive direction from that along the negative direction. 
Hence, we formulate a bilevel optimization problem which consists of two lower-level optimization problems as follows (see also \fig{fig:cibo}):
% 
% % we consider the following optimization problem to obtain the stability margin:
% \begin{subequations}
% \begin{flalign}
% \max_{x, u, f, \epsilon^*} \min _{k} \|\epsilon_k^*\|  \ \\
% \text{s. t. } \quad \text{\eq{const2}, \eq{const3}, \eq{bounds_variables}} \\
% \epsilon_k^* \in \argmax_{\epsilon} \{\epsilon_k^2:\text{\eq{fna_cond_mass_one}, \eq{fnb_cond_mass}} \} \label{const_bilevel_1}
% %  \text{s. t. } \eq{fna_cond_mass_one}, \eq{fnb_cond_mass}\label{force_eq_inner}
% %  \sum_{c=1}^{C} p_{k, c} \times \lambda_{k, c} + p_{k, u} \times R\left(x_{k, o}\right) u_k + r_{k, g} \times (\bar{m}g + \epsilon_k)  = 0\\
% % \lambda_{k, c, n} \geq 0 ,\\
% % \mu_c \lambda_{{k,c, n}}-\lambda_{k, c, t} = 0\label{friction_eq_inner}
% % \mu_c \lambda_{{k,c, n}}+\lambda_{k, c, t} = 0 
% %   0 \leq   \dot{p}_{k, c, j+} \perp \mu_c \lambda_{{k,c, n}}-\lambda_{k, c, t} \geq 0  \\
% %  0 \leq   \dot{p}_{k, c, j-} \perp \mu_c \lambda_{{k,c, n}}+\lambda_{k, c, t} \geq 0 
% \end{flalign}
% \label{equation_sm}
% \end{subequations}
% % \begin{subequations}
% % \begin{flalign}
% % \max_{x, u, \lambda} \min _{k=0, \ldots, N-1} \epsilon_k^\top Q \epsilon_k + \phi_k^\top R \phi_k  \ \\
% % \text{s. t. } \sum_{c=1}^{C}\lambda_{k, c} + \bar{m}g + \epsilon_k = 0, \\
% % \sum_{c=1}^{C} \left(p_{k, c} - \left(q_{k} - \phi_k\right)\right) \times \lambda_{k, c}  = 0\\
% % \quad x_{0} = x_s, 
% % x_{k} \in \mathcal{X}, u_{k} \in \mathcal{U}, \lambda_k \leq \lambda_{u}, \\
% % \lambda_{k, c, n} \geq 0 ,\\
% % \mu_c \lambda_{{k,c, n}}-\lambda_{k, c, t} = 0
% % % \mu_c \lambda_{{k,c, n}}+\lambda_{k, c, t} = 0 
% % %   0 \leq   \dot{p}_{k, c, j+} \perp \mu_c \lambda_{{k,c, n}}-\lambda_{k, c, t} \geq 0  \\
% % %  0 \leq   \dot{p}_{k, c, j-} \perp \mu_c \lambda_{{k,c, n}}+\lambda_{k, c, t} \geq 0 
% % \end{flalign}
% % \label{equation_sm}
% % \end{subequations}
% where \eq{fna_cond_mass_one} and \eq{fnb_cond_mass} represent the lower-level constraints and  \eq{const2}, \eq{const3}, \eq{bounds_variables} represent the upper-level constraints. $\epsilon_k^2$, $\min _{k} \epsilon_k^*$ are the lower- and the upper-level objective function, respectively. $\epsilon$ is the lower-level decision variable and $x, u, f, \epsilon^*$ are the upper-level decision variables. \eq{equation_sm} finds an optimal nominal trajectory that maximize the worst frictional stability over time-horizon given the best frictional stability at each time-step $k$.
% 
% Unfortunately, solving the bilevel optimization problem \eq{equation_sm} is difficult since the lower-level optimization problem (i.e., maximization of convex function) is non-convex. Instead, we consider the following bilevel optimization problem which effectively solves two lower-level optimization problems:
\begin{subequations}
\begin{flalign}
\max_{x, u, f, \epsilon_+^*, \epsilon_-^*} (\min _{k} \epsilon_{k, +}^* - \max _{k} -\epsilon_{k, -}^*)  \ \\
% \max_{x, u, f, \epsilon_+^*, \epsilon_-^*} (\min _{k} \epsilon_{k, +}^* + q\min _{k} \epsilon_{k, -}^* )  \ \\
\text{s. t. } \quad \text{\eq{const2}, \eq{bounds_variables}}, \\
\epsilon_{k, +}^* \in \argmax_{\epsilon_{k, +}} \{\epsilon_{k, +}: A_k\epsilon_{k, +} \leq b_k , \epsilon_{k, +} \geq 0 \}, \label{bi-const1} \\
\epsilon_{k, -}^* \in \argmax_{\epsilon_{k, -}} \{\epsilon_{k, -}: -A_k\epsilon_{k, -} \leq b_k , \epsilon_{k, -} \geq 0 \}
%  \text{s. t. } \eq{fna_cond_mass_one}, \eq{fnb_cond_mass}\label{force_eq_inner}
%  \sum_{c=1}^{C} p_{k, c} \times \lambda_{k, c} + p_{k, u} \times R\left(x_{k, o}\right) u_k + r_{k, g} \times (\bar{m}g + \epsilon_k)  = 0\\
% \lambda_{k, c, n} \geq 0 ,\\
% \mu_c \lambda_{{k,c, n}}-\lambda_{k, c, t} = 0\label{friction_eq_inner}
% \mu_c \lambda_{{k,c, n}}+\lambda_{k, c, t} = 0 
%   0 \leq   \dot{p}_{k, c, j+} \perp \mu_c \lambda_{{k,c, n}}-\lambda_{k, c, t} \geq 0  \\
%  0 \leq   \dot{p}_{k, c, j-} \perp \mu_c \lambda_{{k,c, n}}+\lambda_{k, c, t} \geq 0 
\end{flalign}
\label{equation_sm_1}
\end{subequations}
% where \eq{fna_cond_mass_one}, \eq{fnb_cond_mass} represent the lower-level constraints and $\epsilon_k^2$ represent the lower level objective function. $\epsilon_k$ is the lower-level decision variable. The upper-level constraints are \eq{const2}, \eq{const3}, \eq{bounds_variables} and the upper-level objective function is $\min _{k} \epsilon_k^*$. The upper-level 
% where $k$ is the time-step, $x_k \in\mathbb{R}^{n_{x}}$ is the state of the object. For planar pivoting, we consider $x_k = [x_{k, p}, x_{k, o}]^\top, x_{k, p} \in\mathbb{R}^{2}, x_{k, o} \in\mathbb{R}^{1}$, where $x_{k, p}$ is the position of the object and $x_{k, o}$ is the orientation of the object. $R\left(x_{k, o}\right)$ is the rotation matrix from zero to $x_{k, o}$.  $u_k \in\mathbb{R}^{n_{u}}$ is the control, which is the contact force from a robot finger. $\lambda_k \in\mathbb{R}^{n_{\lambda}}$ is the contact forces from external contact points.  $C$ represents the number of external contact points except for a contact point by a robot finger. For example, in \fig{fig:mechanics_pivoting_eq}, $C=2$ because we have external contact points at A and B. $\epsilon_k \in\mathbb{R}^{n_{x}}$ represents the error of gravity.  For our planar pivoting, $\epsilon_k = [0, \epsilon_{k, y}]^\top$. $\mu_c$ is the coefficient of friction at point $c$. $\lambda_{{k,c, n}}, \lambda_{{k,c, t}}$ represent the normal and sheer forces respectively at point $C$ at $k$. $p_{k, c}$ is nonlinear function with respect to $x_{k, o}$ and it represents the distance from the origin to each contact point. $p_{k, u}$ represents the distance from the origin to control contact point. 
% $r_{k, g}$ is the distance from the origin to a gravity vector.
% Note that from equation 11d to 11g, they are constraints for an inner optimization problem.
where $A_k \in\mathbb{R}^{2 \times 1}, b_k \in\mathbb{R}^{2 \times1}$ represent inequality constraints in \eq{fna_cond_mass_one} and \eq{fnb_cond_mass} \revise{or \eq{eq:uncertain_mass_slope} and  \eq{eq:uncertain_mass_slope_upper_bound} if the object is on a slope.} 
 $A_k\epsilon_{k, +} \leq b_k , \epsilon_{k, +} \geq 0,$ and $-A_k\epsilon_{k, -} \leq b_k , \epsilon_{k, -} \geq 0$ represent the lower-level constraints for each lower-level optimization problem while \eq{const2}, \eq{bounds_variables} represent the upper-level constraints. $\epsilon_+, \epsilon_-$ are the lower-level objective functions while $\min _{k} \epsilon_{k, +}^* - \max _{k} -\epsilon_{k, -}^* $ is the upper-level objective function. $\epsilon_{k, +}, \epsilon_{k, -}$ are the lower-level decision variables of each lower-level optimization problem while $x, u, f, \epsilon_+^*, \epsilon_-^*$ are the upper-level decision variables. 

% Note that $A_k, b_k$ are nonlinear function with respect to $x, u, f$.  

\eq{equation_sm_1} considers the largest one-side frictional stability margin along positive and negative direction at $k$. Therefore, by solving these two lower-level optimization problems, we are able to obtain the maximum frictional stability margin along positive and negative direction. 
% We initially plan to solve the lower-level optimization problem which objective function is $\epsilon_k^2$ to obtain the largest magnitude of $\epsilon_k$ but this objective function makes the lower-level optimization non-convex, which is difficult to solve.
% In contrast, 
The advantage of \eq{equation_sm_1} is that since the lower-level optimization problem are formulated as two linear programming problems, we can efficiently solve the entire bilevel optimization problem using the Karush-Kuhn-Tucker (KKT) condition as follows:
\begin{subequations}
\begin{flalign}
 w_{k, +, j}, w_{k, -, j} \geq 0, C_k\epsilon_{k, +} \leq d_k , E_k\epsilon_{k, -} \leq d_k,\\
w_{k, +, j}(C_k\epsilon_{k, +} - d_k)_j = 0, \\
w_{k, -, j}(E_k\epsilon_{k, -} - d_k)_j = 0, \\
% \epsilon_{k, +}^* \in \argmax_{\epsilon_+} \{\epsilon_{k, +}: A_k\epsilon_{k, +} \leq b_k , \epsilon_{k, +} \geq 0 \}, \\
% \epsilon_{k, -}^* \in \argmax_{\epsilon_-} \{\epsilon_{k, -}: -A_k\epsilon_{k, -} \leq b_k , \epsilon_{k, -} \geq 0 \}, \\
\nabla (-\epsilon_{k, +}) + \sum_{j=1}^{3}w_{k, +, j} \nabla (C_k\epsilon_{k, +} - d_k)_j= 0,\\
\nabla (-\epsilon_{k, -}) + \sum_{j=1}^{3}w_{k, -, j} \nabla (E_k\epsilon_{k, -} - d_k)_j= 0
% t_+ \leq \epsilon_{k, +}, t_- \leq \epsilon_{k, -}, \forall k
%  \text{s. t. } \eq{fna_cond_mass_one}, \eq{fnb_cond_mass}\label{force_eq_inner}
%  \sum_{c=1}^{C} p_{k, c} \times \lambda_{k, c} + p_{k, u} \times R\left(x_{k, o}\right) u_k + r_{k, g} \times (\bar{m}g + \epsilon_k)  = 0\\
% \lambda_{k, c, n} \geq 0 ,\\
% \mu_c \lambda_{{k,c, n}}-\lambda_{k, c, t} = 0\label{friction_eq_inner}
% \mu_c \lambda_{{k,c, n}}+\lambda_{k, c, t} = 0 
%   0 \leq   \dot{p}_{k, c, j+} \perp \mu_c \lambda_{{k,c, n}}-\lambda_{k, c, t} \geq 0  \\
%  0 \leq   \dot{p}_{k, c, j-} \perp \mu_c \lambda_{{k,c, n}}+\lambda_{k, c, t} \geq 0 
\end{flalign}
\label{kkt_equations}
\end{subequations}
where $C_k = [A_k^\top, -1]^\top \in\mathbb{R}^{3 \times 1}, d_k = [b_k^\top, 0]^\top \in\mathbb{R}^{3\times 1}, E_k = [-A_k^\top, -1]^\top \in\mathbb{R}^{3 \times 1}$.
$w_{k, +, j}$ is Lagrange multiplier associated with  $(C_k\epsilon_{k, +} \leq d_k)_j$, where $(C_k\epsilon_{k, +} \leq d_k)_j$ represents the $j$-th inequality constraints in $C_k\epsilon_{k, +} \leq d_k$. $w_{k, -, j}$ is Lagrange multiplier associated with  $(E_k\epsilon_{k, -} \leq d_k)_j$. 
% 
Using the KKT condition and epigraph trick, we eventually obtain a single-level large-scale nonlinear programming problem with complementarity constraints: 
% where $\epsilon_+^*, \epsilon_-^*$ are introduced instead of $\epsilon^*$ so \eq{equation_sm_1} is able to solve lins
\begin{subequations}
\begin{flalign}
\max_{x, u, f, \epsilon_+^*, \epsilon_-^*} (t_+ + \alpha t_- ) \label{cost_bilevel}  \ \\
\text{s. t. } \quad \text{\eq{const2}, \eq{bounds_variables}, \eq{kkt_equations}},\\
% w_{k, +, j}, w_{k, -, j} \geq 0\\
% C_k\epsilon_{k, +} \leq d_k , E_k\epsilon_{k, -} \leq d_k,\\
% w_{k, +, j}(C_k\epsilon_{k, +} - d_k)_j = 0, \\
% w_{k, -, j}(E_k\epsilon_{k, -} - d_k)_j = 0, \\
% % \epsilon_{k, +}^* \in \argmax_{\epsilon_+} \{\epsilon_{k, +}: A_k\epsilon_{k, +} \leq b_k , \epsilon_{k, +} \geq 0 \}, \\
% % \epsilon_{k, -}^* \in \argmax_{\epsilon_-} \{\epsilon_{k, -}: -A_k\epsilon_{k, -} \leq b_k , \epsilon_{k, -} \geq 0 \}, \\
% \nabla (-\epsilon_{k, +}) + \sum_{j=1}^{3}w_{k, +, j} \nabla (C_k\epsilon_{k, +} - d_k)_j= 0,\\
% \nabla (-\epsilon_{k, -}) + \sum_{j=1}^{3}w_{k, -, j} \nabla (E_k\epsilon_{k, +} - d_k)_j= 0, \\
t_+ \leq \epsilon_{k, +}, t_- \leq \epsilon_{k, -}, \forall k
%  \text{s. t. } \eq{fna_cond_mass_one}, \eq{fnb_cond_mass}\label{force_eq_inner}
%  \sum_{c=1}^{C} p_{k, c} \times \lambda_{k, c} + p_{k, u} \times R\left(x_{k, o}\right) u_k + r_{k, g} \times (\bar{m}g + \epsilon_k)  = 0\\
% \lambda_{k, c, n} \geq 0 ,\\
% \mu_c \lambda_{{k,c, n}}-\lambda_{k, c, t} = 0\label{friction_eq_inner}
% \mu_c \lambda_{{k,c, n}}+\lambda_{k, c, t} = 0 
%   0 \leq   \dot{p}_{k, c, j+} \perp \mu_c \lambda_{{k,c, n}}-\lambda_{k, c, t} \geq 0  \\
%  0 \leq   \dot{p}_{k, c, j-} \perp \mu_c \lambda_{{k,c, n}}+\lambda_{k, c, t} \geq 0 
\end{flalign}
\label{kkt_convertion}
\end{subequations}
% where $C_k \in\mathbb{R}^{3 \times 1}, d_k \in\mathbb{R}^{3}$ represent inequality constraints associated with $\epsilon_{k, +}$.  $E_k \in\mathbb{R}^{3 \times 1}, d_k \in\mathbb{R}^{3}$ represent inequality constraints associated with $\epsilon_{k, -}$.
where $\alpha$ is a weighting scalar. 
Note that we derive \eq{kkt_convertion} for the case with an uncertain mass parameter but this formulation can be easily converted to the case where uncertainty exists in CoM location by replacing $A_k, b_k$ in \eq{equation_sm_1} with \eq{fna_fnb_r}.
\revise{Similarly, we can consider uncertainty in finger contact location by replacing $A_k, b_k$ in \eq{equation_sm_1} with \eq{force_eq_location_finger_margin}.}
% with $\epsilon_k^* \in \argmax_{\epsilon} \{\epsilon_k^2:\text{\eq{fna_fnb_r}} \} $ in \eq{equation_sm}. 
Therefore, by solving tractable \eq{kkt_convertion}, we can efficiently generate robust trajectories that are robust against uncertain mass, CoM location, \revise{and contact location} parameters. 

% \textit{Remark 1}:
% In practice, we can add $\min _{x, u, f} \sum_{k=0}^{N-1} ({x}_{k} - x_g)^{\top} Q ({x}_{k} - x_g)+u_{k}^{\top} R u_{k}$ to \eq{cost_bilevel} to realize a smooth trajectory and avoid jerky control inputs. 

\textit{Remark \revise{2}}:
If we consider the case where uncertainty exists in both mass and CoM location simultaneously, we would have a nonlinear coupling term $(C_x+r)(mg + \epsilon)$ in \revise{quasi-static} equilibrium of moment. This makes the lower-level optimization non-convex optimization, making it extremely challenging to solve during bilevel optimization.
Once the lower-level optimization becomes a non-convex optimization problem, there is no guarantee that the lower-level optimization finds globally optimal solutions, resulting in finding a very sub-optimal controller.
% 
\revise{Similarly, all of the constraints (e.g., considering sticking-slipping contact at point contact $A$ and $B$ requires complementarity constraints) which results in non-convex constraints cannot be handled in our CIBO.}
% Thus, it is left as a future work.

% \epsilon_k^* \in \argmax_{\epsilon} \{\epsilon_k^2:\text{\eq{fna_cond_mass_one}, \eq{fnb_cond_mass}} \} \label{const_bilevel_1}

% Problem \eq{equation_sm} is the bilevel optimization. Here, the decision variables of the inner optimization is $\lambda, \epsilon$ and all constraints \eq{force_eq_inner}-\eq{friction_eq_inner} are linear constraints with respect to $\lambda, \epsilon$. For simplicity, here we assume that $\epsilon_{k, y} \geq 0$. Here, to simplify the notation, we represent Problem \eq{equation_sm} as:
% \begin{subequations}
% \begin{flalign}
% \max_{x, u, \lambda^*, \epsilon^*} (\min _{k} \epsilon_k^*)   \\
% \text{s. t. } \quad x_{0} = x_s, 
% x_{k} \in \mathcal{X}, u_{k} \in \mathcal{U}, \\
% \lambda_k^*, \epsilon_k^* = \argmax_{\lambda, \epsilon} e_k ^\top z_k \\
%  \text{s. t. } A_k z_k + b_k = 0\label{equality_eq_inner}\\
% C_k z_k + d_k \leq 0\label{inequality_eq_inner}
% % \mu_c \lambda_{{k,c, n}}+\lambda_{k, c, t} = 0 
% %   0 \leq   \dot{p}_{k, c, j+} \perp \mu_c \lambda_{{k,c, n}}-\lambda_{k, c, t} \geq 0  \\
% %  0 \leq   \dot{p}_{k, c, j-} \perp \mu_c \lambda_{{k,c, n}}+\lambda_{k, c, t} \geq 0 
% \end{flalign}
% \label{equation_bilevel}
% \end{subequations}
% where $z_k = [\lambda_k, \epsilon_k]^\top \in\mathbb{R}^{n_{\lambda}+1}$. We use \eq{equality_eq_inner} to represent all equality constraints in the inner optimization problem and \eq{inequality_eq_inner} to represent all inequality constraints in the inner optimization problem. $A_k \in\mathbb{R}^{q \times (n_{\lambda}+1)}, b_k \in\mathbb{R}^{q}, C_k \in\mathbb{R}^{p \times (n_{\lambda}+1)}, d_k \in\mathbb{R}^{p}, e_k \in\mathbb{R}^{n_\lambda + 1}$. Note that $A_k,  C_k$ are nonlinear function with respect to $x, u$. 


% Thus, we can reformulate the inner loop optimization problem using KKT condition and the reformulated optimization would be:
% \begin{subequations}
% \begin{flalign}
% \max_{x, u, \lambda^*, \epsilon^*} \min _{k} \epsilon_k^*  \ \\
% \text{s. t. } \quad x_{0} = x_s, 
% x_{k} \in \mathcal{X}, u_{k} \in \mathcal{U}, \\
% % \lambda_k^*, \epsilon_k^* = \argmax_{\lambda, \epsilon} e_k ^\top z_k \\
%  A_k z_k + b_k = 0\label{test1}\\
% C_k z_k + d_k \leq 0\label{test2}\\
% w_k^\top \geq 0\\
% w_{k, i}(C_k z_k + d_k)_i = 0, i = 1, \ldots, p\\
% \nabla (e_k^\top z_k) +\sum_{i=1}^{p}w_{k, i} \nabla (C_k z_k + d_k)_i + \sum_{i=1}^{q}v_{k, i}\nabla ( A_k z_k + b_k) = 0
% % e_k + \sum_{i=1}^{p}w_{k, i}C_{k, i}^\top + \sum_{i=1}^{q}v_{k, i}A_{k, i}^\top = 0
% % \mu_c \lambda_{{k,c, n}}+\lambda_{k, c, t} = 0 
% %   0 \leq   \dot{p}_{k, c, j+} \perp \mu_c \lambda_{{k,c, n}}-\lambda_{k, c, t} \geq 0  \\
% %  0 \leq   \dot{p}_{k, c, j-} \perp \mu_c \lambda_{{k,c, n}}+\lambda_{k, c, t} \geq 0 
% \end{flalign}
% \label{bilevel_opt1}
% \end{subequations}
% % , $\phi_k \in\mathbb{R}^{n_{x}}$ represents the error of the CoM location. 
% where $w_{k, i} v_{k, i}$ are dual variables associated with equality constraints and inequality constraints, respectively. We calculate deriviatives with respect to $z_k$. We have $p$ inequality constraints and $q$ equality constraints. 

% Using epigraph trick, we can reformulate the problem as follows:
% \begin{subequations}
% \begin{flalign}
% \max_{x, u, \lambda^*, \epsilon^*} t  \ \\
% \text{s. t. } \quad x_{0} = x_s, 
% x_{k} \in \mathcal{X}, u_{k} \in \mathcal{U}, \\
% % \lambda_k^*, \epsilon_k^* = \argmax_{\lambda, \epsilon} e_k ^\top z_k \\
%  A_k z_k + b_k = 0\label{test1}\\
% C_k z_k + d_k \leq 0\label{test2}\\
% w_k^\top \geq 0\\
% w_{k, i}(C_k z_k + d_k)_i = 0, i = 1, \ldots, p\\
% \nabla (e_k^\top z_k) +\sum_{i=1}^{p}w_{k, i} \nabla (C_k z_k + d_k)_i + \sum_{i=1}^{q}v_{k, i}\nabla ( A_k z_k + b_k) = 0\\
% t \leq \epsilon_k^* \forall k
% % e_k + \sum_{i=1}^{p}w_{k, i}C_{k, i}^\top + \sum_{i=1}^{q}v_{k, i}A_{k, i}^\top = 0
% % \mu_c \lambda_{{k,c, n}}+\lambda_{k, c, t} = 0 
% %   0 \leq   \dot{p}_{k, c, j+} \perp \mu_c \lambda_{{k,c, n}}-\lambda_{k, c, t} \geq 0  \\
% %  0 \leq   \dot{p}_{k, c, j-} \perp \mu_c \lambda_{{k,c, n}}+\lambda_{k, c, t} \geq 0 
% \end{flalign}
% \label{bilevel_opt2}
% \end{subequations}
% % We can solve the problem in several ways. First, we can use epigraph trick by introducing a new scalar variable $z$ as follows:
% % \begin{subequations}
% % \begin{flalign}
% % \max_{x, u, \lambda} z  \ \\
% % \text{s. t. } \sum_{c=1}^{C}\lambda_{k, c} + \bar{m}g + \epsilon_k = 0, \\
% % \sum_{c=1}^{C} \left(p_{k, c} - \left(q_{k} - \phi_k\right)\right) \times \lambda_{k, c}  = 0\\
% % \quad x_{0} = x_s, 
% % x_{k} \in \mathcal{X}, u_{k} \in \mathcal{U}, \lambda_k \leq \lambda_{u}, \\
% % \lambda_{k, c, n} \geq 0 ,\\
% % \mu_c \lambda_{{k,c, n}}-\lambda_{k, c, t} = 0. \\
% % z \leq \epsilon_k^\top Q \epsilon_k + \phi_k^\top R \phi_k,  \forall	k
% % % \mu_c \lambda_{{k,c, n}}+\lambda_{k, c, t} = 0 
% % %   0 \leq   \dot{p}_{k, c, j+} \perp \mu_c \lambda_{{k,c, n}}-\lambda_{k, c, t} \geq 0  \\
% % %  0 \leq   \dot{p}_{k, c, j-} \perp \mu_c \lambda_{{k,c, n}}+\lambda_{k, c, t} \geq 0 
% % \end{flalign}
% % \label{equation_sm}
% % \end{subequations}


\subsection{Robust CIBO for Frictional Uncertainty}
\label{subsec:opt_friction}
We consider the case where the system has uncertainty in the friction coefficients at $A$ and $B$ as discussed in Sec~\ref{subsec:stochasticfriction_planning}. In order to design a robust open-loop controller for the system,  we can use the similar formulation presented in Sec~\ref{bilevel_sec}.
The proposed formulation aims at maximizing the stability margin from stochastic friction. In particular, to avoid non-convex optimization as the lower-level optimization problem, we consider the stability margin along positive and negative direction for both $\epsilon_A$ and $\epsilon_B$, as we discuss in Sec~\ref{bilevel_sec}. By borrowing the optimization problem \eq{equation_sm_1},  the proposed formulation can be seen as follows.
For simplicity, we abbreviate subscript $k$. 
% \begin{subequations}
% \begin{flalign}
% \max_{\Gamma} (\min _{k} \epsilon_{A, +}^* - \max _{k} -\epsilon_{A, -}^* + \min _{k} \epsilon_{B, +}^* - \max _{k} -\epsilon_{B, -}^*)  \ \\
% \text{s. t. } \quad \text{\eq{const2}, \eq{bounds_variables}}, \\
% \epsilon_{A, +}^* \in \argmax_{\epsilon_{A, +}} \{\epsilon_{A, +}:  g(x, u, f, \epsilon_{A, +}, \epsilon_{B}^*)\leq 0 , \nonumber \\ \epsilon_{A, +} \geq 0, \epsilon_{B}^* \in [-\epsilon_{B, -}^*, \epsilon_{B, +}^*]  \}, \label{bi-const1-friction1} \\
% \epsilon_{A, -}^* \in \argmax_{\epsilon_{A, -}} \{\epsilon_{A, -}:  g(x, u, f, -\epsilon_{A, -}, \epsilon_{B}^*)\leq 0 , \nonumber \\ \epsilon_{A, -} \geq 0, \epsilon_{B}^* \in [-\epsilon_{B, -}^*, \epsilon_{B, +}^*]  \}, \label{bi-const1-friction2} \\
% \epsilon_{B, +}^* \in \argmax_{\epsilon_{B, +}} \{\epsilon_{B, +}:  g(x, u, f, \epsilon_{B, +}, \epsilon_{A}^*)\leq 0 , \nonumber \\ \epsilon_{B, +} \geq 0, \epsilon_{A}^* \in [-\epsilon_{A, -}^*, \epsilon_{A, +}^*]  \}, \label{bi-const1-friction3} \\
% \epsilon_{B, -}^* \in \argmax_{\epsilon_{B, -}} \{\epsilon_{B, -}:  g(x, u, f, -\epsilon_{B, -}, \epsilon_{A}^*)\leq 0 , \nonumber \\ \epsilon_{B, -} \geq 0, \epsilon_{A}^* \in [-\epsilon_{A, -}^*, \epsilon_{A, +}^*]  \}, \label{bi-const1-friction4} 
% \end{flalign}
% \label{eq:bilvel_friction}
% \end{subequations}
% Actually, we can instead have the following optimization problem by moving the constraints:  
\begin{subequations}
\begin{flalign}
% \max_{\Gamma} (\min _{k} \epsilon_{A, +}^* - \max _{k} -\epsilon_{A, -}^* + \min _{k} \epsilon_{B, +}^* - \max _{k} -\epsilon_{B, -}^*)  \ \\
\max_{x, u, f, \epsilon_{A,+}^*, \epsilon_{A, -}^*, \epsilon_{B,+}^*, \epsilon_{B, -}^*} \sum_{c\in \mathcal{C}} (\min _{k} \epsilon_{c, +}^* - \max _{k} -\epsilon_{c, -}^*)\\
\text{s. t. } \quad \text{\eq{const2}, \eq{bounds_variables}}, \\
% 
\epsilon_{A}^* \in [-\epsilon_{A, -}^*, \epsilon_{A, +}^*], \epsilon_{B}^* \in [-\epsilon_{B, -}^*, \epsilon_{B, +}^*], \label{eq:stochastic_friction_bilevel_const0} \\
% 
\epsilon_{A, +}^* \in \argmax_{\epsilon_{A, +}} \{\epsilon_{A, +}:  g(x, u, f, \epsilon_{A, +},  \epsilon_{B}^*)\leq 0,  \nonumber \\
\epsilon_{A, +} \geq 0 \}, \label{bi-const1-friction12} \\
% 
\epsilon_{A, -}^* \in \argmax_{\epsilon_{A, -}} \{\epsilon_{A, -}:  g(x, u, f, -\epsilon_{A, -}, \epsilon_{B}^*)\leq 0, \nonumber \\
\epsilon_{A, -} \geq 0,   \}, \label{bi-const1-friction22} \\
% 
\epsilon_{B, +}^* \in \argmax_{\epsilon_{B, +}} \{\epsilon_{B, +}:  g(x, u, f, \epsilon_{B, +}, \epsilon_{A}^*)\leq 0, \nonumber \\ \epsilon_{B, +} \geq 0, \}, \label{bi-const1-friction32} \\
% 
\epsilon_{B, -}^* \in \argmax_{\epsilon_{B, -}} \{\epsilon_{B, -}:  g(x, u, f, -\epsilon_{B, -}, \epsilon_{A}^*)\leq 0, \nonumber \\
\epsilon_{B, -} \geq 0  \}, \label{bi-const1-friction42} 
\end{flalign}
\label{eq:bilvel_friction2}
\end{subequations}
where $g$ summarizes the constraints for each lower-level optimization problem and $\mathcal{C} = \{A, B\}$. 
For each lower-level optimization problem, we consider that another uncertain friction is in the range of optimal stability margin.
For instance, \eq{bi-const1-friction12} is one of the four lower-level optimization problems which aims at maximizing the stability margin under stochastic friction forces at $A$, given stochastic friction force at $B$, $\epsilon_{B}^*$. \eq{eq:stochastic_friction_bilevel_const0}  ensures that $\epsilon_{B}^*$ needs to be within the range of stability margin computed from other two lower-level optimization problems \eq{bi-const1-friction32} and \eq{bi-const1-friction42}.


The resulting optimization introduces many complementarity constraints through the KKT condition because of four lower-level optimization problems, but the resulting computation is still tractable. We discuss computational results in Sec~\ref{subsec::computation}.

\revise{\textit{Remark 3}: In practice, the choice of the particular parameter for the which one should use CIBO to obtain robust trajectories depends on the amount of uncertainty in different parameters associated with the manipulation task. For instance, if we have access to the CAD model of the objects, we can have a good guess of mass and CoM location of the object and thus the major source of uncertainty can be from other parameters such as coefficients of friction.}


\subsection{Robust CIBO for Non-Convex Objects}\label{subsec:mode_based_optimization}
\begin{figure*}[t]
    \centering
    \includegraphics[width=0.8\textwidth]{Figures/mode.png}  
    \caption{A schematic of pivoting for a non-convex shape object where contact set changes over time. During mode 1, the peg rotates with contact at $A$ and $B_2$. During mode 2, the peg rotates with contact at $A$ and $B_1$. 
    $\gamma$ represents one of the kinematic features of peg, which is used to discuss the result in Sec~\ref{fig:openloop_result}. 
    }
    \label{fig:mode_concept}
\end{figure*}


% \devesh{Rephrase to use the formulation in previous subsections}\\
% While our proposed framework in \cite{9811812} is able to design the robust open-loop controller under uncertain physical parameters, it is unable to design the controller for non-convex shape objects such as pegs as shown in \fig{fig:pivoting_abstractfig}. 
The method introduced in the previous subsections assumes convex geometry of the object being manipulated and can not be applied to objects with non-convex geometry (such as pegs as shown in \fig{fig:pivoting_abstractfig}). This is because non-convex objects could result in different contact formations between the object and the environment
 and it is not trivial to identify a feasible contact sequence. In \cite{9811812}, the proposed optimization \eq{kkt_convertion} was solved sequentially for pegs with non-convex geometry. As illustrated in \fig{fig:mode_concept},  we first solve the optimization for a particular contact set (i.e., mode 1 in \fig{fig:mode_concept}) and then solve the optimization for another contact set (i.e., mode 2 in \fig{fig:mode_concept}) given the solution obtained from the first optimization. 
While this method works, it requires extensive domain knowledge. We observed that the second stage optimization can result in infeasible solutions given the solution from the first stage optimization. Thus, we had to carefully specify the parameters of optimization and, in particular, the initial state and terminal state constraints. Such a hierarchical approach has difficulty in finding a feasible solution once the object becomes more complicated. 

To overcome these issues, in general, complementarity constraints can be used to model the change of contact. However, introducing complementarity constraints inside the lower-level optimization makes the lower-level optimization non-convex optimization. Hence, the KKT condition is not a necessary and sufficient condition for optimality but rather a necessary condition. Thus, it is not guaranteed to find globally optimal safety margins over the trajectory. 

In this work, we propose another approach to deal with the non-convexity of the object. Inspired by \cite{9812069}, we formulate the optimization that optimizes the trajectory given mode sequences instead of optimizing mode sequences. It is worth noting that our framework still optimizes the trajectory over the time duration of each mode given the sequence of the mode. Our goal is that the optimization has a larger feasible space so that less domain knowledge is required.%, resulting in less parameter tuning. 





Using the formulation presented in~\cite{9812069}, we present a mode-based formulation for non-convex shaped objects. 
See \cite{9812069} for more details regarding mode-based optimization. For simplicity of exposition, we only present the formulation considering two modes. But one can easily extend this to problems with multiple modes.
% We consider the case where the system has two contact sets and we define the mode as $m = \{\left(l, b\left(l\right)| l\in \mathcal{L}, b\left(l\right) \in \{1, 2\}\right)\}$ where $\mathcal{L}$ is a set of pairs of indicies that defines the complementarity relationship. 
% As shown in FIGURE, we use $m_1$ and $m_2$ to represent mode 1 (i.e., contact on $A$ and $B_1$) and mode 2 (i.e., contact on $A$ and $B_2$), respectively. 
% Thus, we can define each mode as 
For each contact mode, the system has the different constraints. For brevity, we abbreviate the subscript $k$:
\begin{subequations}
\begin{flalign}
% \min _{\tilde{x}, u, f} \sum_{k=0}^{N-1} (\tilde{x}_{k} - x_g)^{\top} Q (\tilde{x}_{k} - x_g)+u_{k}^{\top} R u_{k} + \sum_{l=1}^2 T_l \\
% \text{s. t. } 
i_{x}, i_{y} \in FK_{m}(\theta_k, \revise{P}_{k, y}^{\revise{O}}),  \forall i  \in \{A, B_m\}\\
g_m(f_{nA}, f_{tA}, f_{nB_{1}}, f_{tB_{1}}, f_{nP}, f_{tP}, \revise{P}_{y}^{\revise{O}}) \ \text{if} \ m = 1
\\
g_m(f_{nA}, f_{tA}, f_{nB_{2}}, f_{tB_{2}}, f_{nP}, f_{tP}, \revise{P}_{y}^{\revise{O}}) \ \text{if} \ m = 2 \label{eq:mode2_dynamics_const}
\\
 f_{tA}  =\mu_A f_{nA}, f_{tB_{1}}  =-\mu_{B_1} f_{nB_{1}}, f_{tB_{2}}  =-\mu_{B_2} f_{nB_{2}}
 \\
 \eq{slippingP}, x_{k} \in \mathcal{X}, u_{k} \in \mathcal{U}, 0\leq f_{k, ni} \leq f_{u}
\end{flalign}
\label{eq:mode_const}
\end{subequations}
where $m \in \{1, 2\}$ to represent each contact mode. 
$g_m$ represents the quasi-static model of pivoting manipulation for mode $m$.
It is worth noting that since we decompose the optimization problem into the two mode optimization problem, complementarity constraints are encoded for each mode. %\devesh{Yuki : the above equations are not clear. What is g? have we introduced it earlier?}

 What the optimization problem needs to perform is that for each mode, it only considers the associated constraints and does not consider constraints associated with different contact mode. For example, during mode 1, the optimization should consider only constraints associated with mode 1 and should not consider constraints such as \eq{eq:mode2_dynamics_const}. 
 Another thing the optimization needs to consider is that it needs to scale $\dot{\theta}, \revise{\dot{P}}_{y}^{\revise{O}}$ since we would like to optimize over the time duration. To achieve that, we employ the scaled time variables as discussed in \cite{9812069}. As a result, 
 we recast the quasi-static model by introducing a new state variable with a scaled time, $\tilde{x}_k = \left[\theta_k, \revise{P}_{k, y}^{\revise{O}}, \frac{\dot{\theta}_k}{T}, \frac{\revise{\dot{P}}_{k, y}^{\revise{O}}}{T}\right]^\top$ where $T = T_1$ during mode 1 and $T = T_2$ during mode 2.

 For two contact modes, we can remodel our optimization \eq{equation_control} as follows:
\begin{subequations}
\begin{flalign}
\min _{\tilde{x}, u, f} \sum_{k=0}^{N-1} (\tilde{x}_{k} - x_g)^{\top} Q (\tilde{x}_{k} - x_g)+u_{k}^{\top} R u_{k} + \sum_{l=1}^2 T_l \\
\text{s. t. }  \revise{h_1}(\tilde{x}_k, u_k, f_k) \leq 0, \text{for} \ k\Delta \leq 1 \\
\revise{h_2}(\tilde{x}_k, u_k, f_k) \leq 0, \text{for} \ k\Delta > 1
\label{bounds_variables_mode}
\end{flalign}
\label{eq:mode_change}
\end{subequations}
where $\tilde{x}_k = \left[\theta_k, \revise{P}_{k, y}^{\revise{O}}, \frac{\dot{\theta}_k}{T_1}, \frac{\revise{\dot{P}}_{k, y}^{\revise{O}}}{T_1}\right]^\top$ for $k\Delta \leq 1$ and $\tilde{x}_k = \left[\theta_k, \revise{P}_{k, y}^{\revise{O}}, \frac{\dot{\theta}_k}{T_2}, \frac{\revise{\dot{P}}_{k, y}^{\revise{O}}}{T_2}\right]^\top$ for $k\Delta > 1$.
We use $\revise{h_1}$ and $\revise{h_2}$ to represent all constraints for each mode. 
Given \eq{eq:mode_change}, we can obtain bilevel optimization formulation for non-convex shape objects by following the logic in Sec~\ref{bilevel_sec}.


% In this section, we present optimization formulation for objects with non-convex geometry to obtain robust trajectories as presented in the previous section.

% \subsection{Robust Bilevel Contact-Implicit Trajectory Optimization under Frictional Uncertainty}
% \label{subsec:opt_friction}
% We consider the case where the system has uncertainty in the friction coefficients at $A$ and $B$ as discussed in Sec~\ref{subsec:stochasticfriction_planning}. In order to design a robust open-loop controller for the system,  we can use the similar formulation presented in Sec~\ref{bilevel_sec}.
% As discussed in Sec~\ref{subsec:stochasticfriction_planning}, the proposed formulation aims at maximizing the stability margin from stochastic friction. In particular, to avoid non-convex optimization as the lower-level optimization problem, we consider the stability margin along positive and negative direction for both $\epsilon_A$ and $\epsilon_B$, as we discuss in Sec~\ref{bilevel_sec}. By borrowing the optimization problem \eq{equation_sm_1},  the proposed formulation can be seen as follows.
% For simplicity, we abbreviate subscript $k$. 
% % \begin{subequations}
% % \begin{flalign}
% % \max_{\Gamma} (\min _{k} \epsilon_{A, +}^* - \max _{k} -\epsilon_{A, -}^* + \min _{k} \epsilon_{B, +}^* - \max _{k} -\epsilon_{B, -}^*)  \ \\
% % \text{s. t. } \quad \text{\eq{const2}, \eq{bounds_variables}}, \\
% % \epsilon_{A, +}^* \in \argmax_{\epsilon_{A, +}} \{\epsilon_{A, +}:  g(x, u, f, \epsilon_{A, +}, \epsilon_{B}^*)\leq 0 , \nonumber \\ \epsilon_{A, +} \geq 0, \epsilon_{B}^* \in [-\epsilon_{B, -}^*, \epsilon_{B, +}^*]  \}, \label{bi-const1-friction1} \\
% % \epsilon_{A, -}^* \in \argmax_{\epsilon_{A, -}} \{\epsilon_{A, -}:  g(x, u, f, -\epsilon_{A, -}, \epsilon_{B}^*)\leq 0 , \nonumber \\ \epsilon_{A, -} \geq 0, \epsilon_{B}^* \in [-\epsilon_{B, -}^*, \epsilon_{B, +}^*]  \}, \label{bi-const1-friction2} \\
% % \epsilon_{B, +}^* \in \argmax_{\epsilon_{B, +}} \{\epsilon_{B, +}:  g(x, u, f, \epsilon_{B, +}, \epsilon_{A}^*)\leq 0 , \nonumber \\ \epsilon_{B, +} \geq 0, \epsilon_{A}^* \in [-\epsilon_{A, -}^*, \epsilon_{A, +}^*]  \}, \label{bi-const1-friction3} \\
% % \epsilon_{B, -}^* \in \argmax_{\epsilon_{B, -}} \{\epsilon_{B, -}:  g(x, u, f, -\epsilon_{B, -}, \epsilon_{A}^*)\leq 0 , \nonumber \\ \epsilon_{B, -} \geq 0, \epsilon_{A}^* \in [-\epsilon_{A, -}^*, \epsilon_{A, +}^*]  \}, \label{bi-const1-friction4} 
% % \end{flalign}
% % \label{eq:bilvel_friction}
% % \end{subequations}
% % Actually, we can instead have the following optimization problem by moving the constraints:  
% \begin{subequations}
% \begin{flalign}
% % \max_{\Gamma} (\min _{k} \epsilon_{A, +}^* - \max _{k} -\epsilon_{A, -}^* + \min _{k} \epsilon_{B, +}^* - \max _{k} -\epsilon_{B, -}^*)  \ \\
% \max_{x, u, f, \epsilon_{A,+}^*, \epsilon_{A, -}^*, \epsilon_{B,+}^*, \epsilon_{B, -}^*} \sum_{c\in \mathcal{C}} (\min _{k} \epsilon_{c, +}^* - \max _{k} -\epsilon_{c, -}^*)\\
% \text{s. t. } \quad \text{\eq{const2}, \eq{bounds_variables}}, \\
% % 
% \epsilon_{A}^* \in [-\epsilon_{A, -}^*, \epsilon_{A, +}^*], \epsilon_{B}^* \in [-\epsilon_{B, -}^*, \epsilon_{B, +}^*],  \\
% % 
% \epsilon_{A, +}^* \in \argmax_{\epsilon_{A, +}} \{\epsilon_{A, +}:  g(x, u, f, \epsilon_{A, +},  \epsilon_{B}^*)\leq 0,  \nonumber \\
% \epsilon_{A, +} \geq 0 \}, \label{bi-const1-friction12} \\
% % 
% \epsilon_{A, -}^* \in \argmax_{\epsilon_{A, -}} \{\epsilon_{A, -}:  g(x, u, f, -\epsilon_{A, -}, \epsilon_{B}^*)\leq 0, \nonumber \\
% \epsilon_{A, -} \geq 0,   \}, \label{bi-const1-friction22} \\
% % 
% \epsilon_{B, +}^* \in \argmax_{\epsilon_{B, +}} \{\epsilon_{B, +}:  g(x, u, f, \epsilon_{B, +}, \epsilon_{A}^*)\leq 0, \nonumber \\ \epsilon_{B, +} \geq 0, \}, \label{bi-const1-friction32} \\
% % 
% \epsilon_{B, -}^* \in \argmax_{\epsilon_{B, -}} \{\epsilon_{B, -}:  g(x, u, f, -\epsilon_{B, -}, \epsilon_{A}^*)\leq 0, \nonumber \\
% \epsilon_{B, -} \geq 0  \}, \label{bi-const1-friction42} 
% \end{flalign}
% \label{eq:bilvel_friction2}
% \end{subequations}
% where $g$ summarizes the constraints for each lower-level optimization problem and $\mathcal{C} = \{A, B\}$. 
% % 
% The resulting optimization introduces many complementarity constraints through the KKT condition because of four lower-level optimization problems, but the resulting computation is still tractable. We discuss the computation result in Sec~\ref{subsec::computation}.

\subsection{Robust CIBO with Patch Contact}
The formulation for robust CIBO is similar to the point contact case except that the underlying equilibrium conditions are different. The \revise{quasi-static} equilibrium conditions for the patch contact case were earlier presented in~\eqref{force_balance_patch_contact}. Using these equations and the analysis presented in Sections~\ref{sec:sec_uncertain_mass} through~\ref{subsec:stochasticfriction_planning}, it is straightforward to compute the constraints for the corresponding robust CIBO similar to~\eqref{equation_sm_1}. More explicitly, this can be achieved by computing the appropriate constraints of the type $A_k\epsilon_{k, +} \leq b_k$ and $-A_k\epsilon_{k, -} \leq b_k$ using~\eqref{force_balance_patch_contact} and the frictional stability margin discussion in Sec~\ref{subsec:Pivoting_manipulation}.

% Given patch contact model, we can formulate optimization problem in the same fashion using \eq{equation_sm_1}. We simply need to replace  with the constraints discussed in Sec~\ref{subsec:Pivoting_manipulation}.
% \textcolor{red}{where do we say considering complementarity constraints would make optimization intractable? here or in Sec~\ref{subsec:Pivoting_manipulation}?}



