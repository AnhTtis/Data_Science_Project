\section{Introduction}\label{sec:introduction}
\begin{figure*}
    \centering
    \includegraphics[width=0.95\textwidth]{Figures/pivot_fig1.pdf} % 
    \caption{We consider the problem of reorienting parts for assembly using pivoting manipulation primitive. Such reorientation could possibly be required when the parts being assembled are too big to grasp in the initial pose (such as the gears) or the parts to be inserted during assembly are not in the desired pose (such as the pegs). The figure shows some instances during the implementation of our  controller to reorient a gear and a peg.}
    \label{fig:pivoting_abstractfig}
\end{figure*}

\IEEEPARstart{C}{ontacts} are central to most manipulation tasks as they can provide additional dexterity to robots to interact with their environment~\cite{mason2018toward}. 
% \revise{For example,  
% in \fig{fig:pivoting_abstractfig}, the objective of the task is assembly. \fig{fig:pivoting_abstractfig} shows that the robot is unable to accomplish grasping since the size of the gear is too large and the peg is an undesired pose. To accomplish the task, we propose to use the pivoting manipulation with extrinsic contacts and thus the robot is able to reorient the parts. Therefore, the robot can successfully conduct the grasping, resulting in the successful assembly.} 
It is desirable that a robot should be able to interact with unknown objects in unknown environments during operation and thus achieve generalizable manipulation. 
\revise{However, designing systems which can achieve such behavior is difficult. Such behavior requires that a robot should be able to reason about and generate plans that are robust to uncertainties arising from a variety of different reasons.} Robust planning for frictional interaction with objects with uncertain physical properties is challenging as the mechanical stability of the object depends on these physical properties. Inspired by this problem, we consider the task of robust pivoting manipulation in this paper. The pivoting task considered in this paper requires that the slipping contact be maintained at the two external contact points which presents unique challenges for robust planning. 
% In particular, we consider the problem of re-orienting objects with uncertain mass and Center of Mass (CoM) location using pivoting.
We are interested in ensuring mechanical stability via friction to compensate for uncertainty in the physical properties (e.g., physical parameters, coefficient of friction, \revise{contact location}.) of the objects during manipulation. We present a novel formulation and an optimization technique that can solve robust manipulation trajectories for manipulation problems.% for the proposed pivoting manipulation. 

%We present a bilevel trajectory optimization method that can maximize this frictional stability under some simplifying assumptions. We present analysis showing that friction can provide stability to uncertainty in mass and CoM location of the object. 
Robust planning (and control) for frictional interaction is challenging due to the  hybrid nature of underlying frictional dynamics. Consequently, a lot of classical robust planning and control techniques are not applicable to these systems in the presence of  uncertainties~\cite{drnach2021robust,9812069,yuki2021chance}. While concepts of stability margin or Lyapunov stability have been well studied in the context of nonlinear dynamical system controller design~\cite{vidyasagar2002nonlinear}, such notions have not been explored in contact-rich manipulation problems. This can be mostly attributed to the fact that a controller has to reason about the mechanical stability constraints of the frictional interaction to ensure stability. Mechanical stability closely depends on the contact configuration during manipulation, and thus a planner (or controller) has to ensure that the desired contact configuration is either maintained during the task or it can maintain stability even if the contact sequence is perturbed. Analysis of such systems is difficult in the presence of friction as it leads to differential inclusion system (see~\cite{raghunathan2020stability}) . One of the key insights we present in this paper is that friction provides mechanical stability margin during a contact-rich task. We call the mechanical stability provided by friction as \textit{Frictional Stability}. This \textit{frictional stability} can be exploited during optimization to allow stability of manipulation in the presence of uncertainty. We show the effect of several different parameters on the stability of the manipulation using the proposed approach. In particular, we consider the effect of contact modes and point of contact between the robot \& object on the stability of the manipulation. %We believe that our proposed ideas could also be used for designing feedback controllers to correct contact trajectories based on estimates of contact states.

We study pivoting manipulation where the object being manipulated has to maintain slipping contact with two external surfaces (see \fig{fig:mechanics_pivoting_eq}). A robot can use this manipulation to reorient parts on a planar surface to allow grasping or assist in assembly by manipulating objects to a desired pose (see \fig{fig:pivoting_abstractfig}). Note that this manipulation is challenging as it requires controlled slipping (as opposed to sticking contact~\cite{hou2018fast, hogan2020tactile, shirai2023tactile}). Ensuring robustness for slipping contact is challenging due to the equality constraints for the contact forces compared to inequality constraints for sticking contact. To ensure mechanical stability of the two-point pivoting in the presence of uncertainty, we derive a sufficient condition for stability which allows us to compute a margin of stability. This margin is then used in a bilevel optimization technique, CIBO (Contact Implicit Bilevel Optimization).
\revise{Our proposed CIBO designs an optimal control trajectory while maximizing the worst-case margin along the entire trajectory for manipulation.}
% , to design an optimal control trajectory while maximizing this margin. 
Through numerical simulations as well as physical experiments, we verify that CIBO is able to achieve more robustness compared to the basic trajectory optimization.


% What Yuki wants to describe in this section: 
% \begin{enumerate}
% \item contacts are important
% \item ideally we would like to do manipulation such as pivoting for unknown objects
% \item it is challenging because it is not clear how to formulate the planning/control problem under uncertainty. 
% \item also, slipping control is challenging since it can switch to different mode.
% \item Thus, in this paper, recognizing that friction forces can account for uncertainty, we can generate open-loop trajectories
% \end{enumerate}






% We use vision-based tactile sensors co-located at the fingers of a parallel jaw gripper to detect stability of an object in an unknown grasp. In Figure~\ref{fig:mechanics_placement}, we show a sticking contact configuration between an object and its external environment during a placement attempt. The Figure~\ref{fig:mechanics_placement} shows the force and moments experienced by the object in the shown contact configuration. It can be seen that the object will undergo rotation about the axis of grasp, when the object is pushed against the external surface. 

\textbf{Contributions.} This paper has the following contributions.
\begin{enumerate}
    \item We present analysis of mechanical stability of pivoting manipulation with uncertainty in mass, CoM location, \revise{contact location,} and coefficient of friction.
    \item We present a robust contact-implicit bilevel optimization (CIBO) technique which can be used to optimize the mechanical stability margin to compute robust trajectories for pivoting manipulation. For objects with non-convex shapes, we present a formulation with mode-based optimization. 
     %during packing and assembly-like scenarios.
\end{enumerate}
The proposed method is demonstrated for reorienting parts using a 6 DoF manipulator (see \fig{fig:pivoting_abstractfig}. Please see a video demonstrating hardware experiments at this link\footnote{\url{https://www.youtube.com/watch?v=ojlZDaGytSY}}).
A preliminary version of this work was initially presented at a conference~\cite{9811812}. However, compared to the previous work, this paper has the following major differences:
\begin{enumerate}
    % \item We present a generic formulation for stability during the pivoting manipulation task.
    \item We present analysis of the proposed manipulation considering patch contact, \revise{uncertain mass on a slope, robot finger contact location}, and stochastic friction coefficients at the different points of contact.
    \item \revise{We present a mode-based optimization formulation which can be used for computing robust trajectories for objects with non-convex geometry.}
    \item \revise{We also implement a closed-loop controller with vision feedback which operates in an MPC fashion where we use CIBO for re-computation of controller up on state feedback. We show that we are able to achieve additional robustness for the closed-loop controller.}
    % \item We present formulation and verification for recovery from failure during the proposed pivoting manipulation.
\end{enumerate}

In Section~\ref{sec:related_work}, we present work which is relevant to our proposed work. In Section~\ref{sec:mechanics}, we present the mechanics of pivoting manipulation. Section~\ref{sec:sec_formulation} presents an analysis of frictional stability margin considering different sources of uncertainty. In Section~\ref{sec:robust_to}, we present the proposed contact-implicit bilevel optimization (CIBO) for robust pivoting manipulation. Section~\ref{sec:results} presents numerical results of trajectory optimization as well as experimental evaluation using a manipulator arm and several different objects. Finally, the paper is concluded in Section~\ref{sec:discussion} with some topics for future research.
% We believe that this is the first attempt in literature to analyze and propose a solution to the problem of detecting stability of an object depending on the contact formation between the object and environment. We believe that the proposed work would be useful for several manipulation problems like stable feedback pivoting, part re-orientation during assembly, etc.