\section{Experimental Results}\label{sec:results}
%We verify our proposed optimal control presented in Sec~\ref{sec:robust_to} for pivoting. 
In this section, we verify the performance of our proposed approach for pivoting. Through the experiments we present in this section, we evaluate the efficacy of the proposed planner in several different settings and the computational requirement of the method. We also present results of implementation of the proposed planner on a robotic system using a 6 DoF manipulator arm and compare it against a baseline trajectory optimization method.

% We present experiments to answer the following questions:
% \begin{enumerate}
%     \item How much robustness can the proposed bilevel optimization method provide over the baseline method?
%     \item Can we demonstrate robustness of the proposed optimization during control of pivoting manipulation?
% \end{enumerate}

%We first show the result of our proposed controller \eq{kkt_convertion} compared to a baseline trajectory generated by solving \eq{equation_control}. %We also implement our controller using a physical manipulator arm and evaluate the performance.

\begin{table}[t]
    \caption{{Parameters of objects. $m, l, w$ represent the mass, length, and the width of the object, respectively. For pegs, the first element in $l, w$ are $l_1, w_1$ and the second element in $l, w$ are $l_2, w_2$, respectively, shown in \fig{fig:hardwareresults}.} For pegs, since they are made of the same material and they make contact on the same environment, we can assume $\mu_B = \mu_{B_1} = \mu_{B_2}$. }
    \centering
    \begin{tabular}{c|c|c|c|c}
     & $m$ [g] & $l$ [mm] & $w$ [mm] & $\mu_A, \mu_B, \mu_P$\\
         \hline\hline gear 1 & 140 & 84 & 20 & 0.3, 0.3, 0.8\\
         \hline gear 2 & 100 & 121 & 9.5 & 0.3, 0.3, 0.8\\
         \hline gear 3 & 280 & 84 & 20 & 0.3, 0.3, 0.8\\
         \hline peg 1 & 45 & 36, 40 & 20, 28 & 0.3, 0.3, 0.8\\
         \hline peg 2 & 85 & 28, 40 & 10, 11 & 0.3, 0.3, 0.8
         \\
         \hline peg 3 & 85 & 28, 40 & 10, 27.5 & 0.3, 0.3, 0.8
    \end{tabular}
    \label{parameter_table}
\end{table}

\begin{figure*}
    \centering
    \includegraphics[width=0.9\textwidth]{Figures/fc2_gear_cropped.pdf} % 
    \caption{Trajectory of frictional stability margin. $\epsilon_A, \epsilon_B$ are bounds of $\epsilon$ from \eq{fna_cond_mass_one}, \eq{fnb_cond_mass}. $r_A, r_B$ are bounds of $r$ from \eq{fna_fnb_r}.  $\epsilon_+, \epsilon_-, r_+, r_i$ are solutions obtained from CIBO. (a), (b): Trajectory of frictional stability of gear 1 based on uncertain mass obtained from baseline optimization, our CIBO, respectively. (c), (d): Trajectory of frictional stability of gear 1 based on uncertain CoM location obtained from baseline optimization, CIBO, respectively. (e): Snapshots of pivoting motion for gear 1 obtained from CIBO considering uncertain CoM location.}
    \label{fig:openloop_result}
\end{figure*}

\begin{figure}
    \centering
    \includegraphics[width=0.35\textwidth]{Figures/fc2_peg_cropped.pdf} % 
    \caption{(a), (b): Trajectory of frictional stability margin of peg 1 based on uncertain mass obtained from CIBO, baseline optimization, respectively. Note that here we solve CIBO sequentially for each mode (i.e., hierarchical planning), instead of using the proposed mode-sequence-based optimization.  (c): Snapshots of pivoting motion for peg 1, obtained from CIBO considering uncertain mass.}
    \label{fig:openloop_result_peg1}
\end{figure}


\begin{figure*}
    \centering
    \includegraphics[width=0.98\textwidth]{Figures/benchmark_vs_robust_com2.png} % 
    \caption{We show the time history of object angle, finger position, and contact forces from a manipulator during pivoting of gear 1. The top row shows the result using CIBO \eq{kkt_convertion} considering CoM uncertainty and the bottom one shows the result using \eq{equation_control} (i.e., it does not consider robustness criteria in the formulation explicitly.). The top row results and the bottom row results are used in visualizing the stability margin in \fig{fig:openloop_result} (d), (c), respectively.}
    \label{fig:benchmark_vs_robust_com}
\end{figure*}


% \begin{figure*}
%     \centering
%     \includegraphics[width=1\textwidth]{Figures/fc2_cropped.pdf} % 
%     \caption{Time history of frictional stability margin with time history of control inputs $u$. (a): The trajectory obtained from the baseline optimization is shown with stability margin considering uncertain mass. (b): The trajectory obtained from our proposed optimization is shown with stability margin considering uncertain mass. (c): The trajectory obtained from the baseline optimization is shown with stability margin considering uncertain CoM location. (d): The trajectory obtained from our proposed optimization is shown with stability margin considering uncertain CoM location. (e): Snapshots of motion of pivoting for gear 1 obtained from our proposed optimization considering uncertain CoM location. (e): Snapshots of motion of pivoting for peg 2 obtained from our proposed optimization considering uncertain mass. (g), (h): Time history of stability margin of (f) obtained from our proposed optimization, the benchmark optimization, respectively. Our proposed could generate more robust trajectories than baseline optimization. In (a), at $t=0$ s, $f_{nB}$ is almost zero so that the stability margin obtained from \eq{fnb_cond_mass} would be zero. In contrast, in (b), our proposed optimization could realize non-zero $f_{nB}$ as shown as a red arrow in (b). In (c), at $t=25$ s,  the trajectory has smallest stability margin. In (d), our bilevel optimization is able to increase the stability margin at $t=25$ s. }
%     \label{fig:openloop_result}
% \end{figure*}



\begin{table}[t]
    \caption{{Worst-case stability margin over the control horizon obtained from optimization for gear 1. Note that the stability margin for the solution of the benchmark optimization is analytically calculated.}}
    \centering
    \begin{tabular}{c|c|c}
     & $\epsilon_{+}^*$, $\epsilon_{-}^* $  [N]& $r_{+}^*$, $r_{-}^*$ [mm]\\
         \hline\hline Benchmark optimization \eq{equation_control} & 0.10, 0.66 & 1.5, 0.85\\
         \hline Ours  \eq{kkt_convertion} with mass uncertainty & 0.34, 0.50 & N/A \\
         \hline Ours  \eq{kkt_convertion} with CoM uncertainty & N/A & 3.43, 2.70
    \end{tabular}
    \label{epsilon}
\end{table}

\begin{table}[t]
    \caption{{Obtained worst stability margins over the time horizons from optimization for peg 1. Note that the stability margin for the solution of the benchmark optimization is analytically calculated.}}
    \centering
    \begin{tabular}{c|c|c}
     & $\epsilon_{+}^*$, $\epsilon_{-}^* $  [N]& $r_{+}^*$, $r_{-}^*$ [mm]\\
         \hline\hline Benchmark optimization \eq{equation_control} & 0.035, 0.018 & 31, 0\\
         \hline Ours  \eq{kkt_convertion} with mass uncertainty & 0.050, 0.021 & N/A \\
         \hline Ours  \eq{kkt_convertion} with CoM uncertainty & N/A & 38, 0
    \end{tabular}
    \label{epsilon_peg}
\end{table}


\subsection{Experiment Setup}
We implement our method in Python using IPOPT solver \cite{80fe29bf9dc245ffa5c8bd7b3eee2902} with \pyrobocop\ wrapper \cite{9812069}. 
We use HSL MA86 \cite{hsl2007collection} as a linear solver for IPOPT. 
The optimization problem is implemented on a computer with Intel i7-12800K. 





We demonstrate our algorithm on several different objects, as detailed in \tab{parameter_table}. During optimization, we set 
$Q=\text{diag}(0.1, 0), R=\text{diag}(0.01, 0.01)$. We use $\alpha= 1$ when we run \eq{kkt_convertion}. We set $x_s = [0, \frac{w}{4}]^\top, \theta_g = \frac{\pi}{2}$. Note that we only enforce terminal constraints for convex shape objects. For non-convex shape objects, we do not enforce terminal constraints since the peg cannot achieve $\theta_N = \frac{\pi}{2}$ unless we consider another contact mode (see \fig{fig:mode_concept}).
In \pyrobocop\ wrapper, we did warm-start for the state at $k=0, N$ by setting initial and terminal states as initial guesses. We did not explicitly conduct a warm-start for other decision variables and we set them to 0. 
%Regarding pegs, because the contact points change during pivoting, we decouple one trajectory optimization problem with complementarity constrains expressing the contact-noncontact configuration and solve it as a multi-stage optimization problem where each stage problem solves the case with a certain contact configuration.

%As shown in 

We use a Mitsubishi Electric Factory Automation (MELFA) RV-5AS-D Assista 6 DoF arm (see \fig{fig:pivoting_abstractfig}) for the experiments. The robot has a pose repeatability of $\pm 0.03$mm. The robot is equipped with Mitsubishi Electric F/T sensor 1F-FS001-W200 (see \fig{fig:pivoting_abstractfig}). To implement the computed force trajectory during manipulation, we use the default stiffness controller for the robot. By selecting an appropriate stiffness matrix \cite{stiffness_control}, we design a reference trajectory that would result in the desired interaction force required for manipulation~\cite{9838102, https://doi.org/10.48550/arxiv.2204.10447}. 
\revise{More specifically, we use the following relationship, $x^r_k=x_k+u_k K_s^{-1}$, where, $x_{k}$ and $f_k$ are the configuration trajectory and the force trajectory at time step $k$, respectively, obtained by CIBO as output. $K_s$ is the stiffness matrix for the robot which is appropriately chosen. At each time step $k$, we command $x^r_k$ as a reference trajectory of the robot's internal position controller.}
Note that this trajectory is implemented in open-loop and we do not design a controller to ensure that the computed force trajectory is precisely tracked during execution.

% \textcolor{red}{Explain MPC setting: Devesh could you work on this?}
\revise{For the MPC experiments discussed in Section~\ref{sec:error_recovery}, the object states are tracked using AprilTag \cite{apriltag_2011icra}. The robot states are tracked using the robot's joint encoders. The contact states at contact $A, B, P$ in \fig{fig:mechanics_pivoting_eq} are estimated using the object state, the robot state, and the known geometry of the object. }
% To track the reference force profile in open-loop fashion, we use a stiffness controller which is designed using the default stiffness controller \cite{stiffness_control} of the robot, as illustrated in \fig{fig:control_block}. \devesh{Yuki : please be careful in the choice of words. We are not tracking the force profile. We are simply designing a reference position trajectory so that we can perform indirect force control. Note that the there is no feedback loop tracking the force trajectory.}
% We test our method on several different objects listed in \tab{parameter_table}.
% talk stiffness control, robot specification, using a Mitsubishi F/T sensor $1$F-FS$001$-W$200$




% \begin{figure}
%     \centering
%     \includegraphics[width=0.45\textwidth]{Figures/control_block.png} % 
%     \caption{A schematic showing the implemented control block diagram. }
%     \label{fig:control_block}
% \end{figure}

\subsection{CIBO for Uncertain Mass and CoM Parameters}
% We first show the results of our proposed controller for gear 1 in \fig{fig:openloop_result}. 

\fig{fig:openloop_result} shows the trajectory of frictional stability margin of gear 1 obtained from the proposed robust CIBO considering uncertain mass and uncertain CoM location, and the benchmark optimization. Overall, CIBO could generate more robust trajectories. 
For example, at $t=0$ s, $f_{nB}$ in (a) is almost zero so that the stability margin obtained from \eq{fnb_cond_mass} is almost zero. In contrast,  CIBO could realize non-zero $f_{nB}$ as shown as a red arrow in (b). 
In (d), to increase the stability margin, the finger position $\revise{P}_{y}^{\revise{O}}$ moves on the face of gear 1 so that the controller can increase the stability margin more than the benchmark optimization. This would not happen if we do not consider complementarity constraints \eq{slippingP}. 
% In fact, in this problem setting, we only 
% finds infeasible solutions if we get rid of \eq{lippingP} and ensure that point $P$ is always sticking. 
Also, our obtained $\epsilon_{+}, \epsilon_{-}, r_+, r_-$ follows bounds of stability margin. It means that CIBO can successfully design a controller that maximizes the worst stability margin given the best stability margin for each time-step.  
% 


% In (a), at $t=0$ s, $f_{nB}$ is almost zero so that the stability margin obtained from \eq{fnb_cond_mass} would be zero. In contrast, in (b), our proposed optimization could realize non-zero $f_{nB}$ as shown as a red arrow in (b). In (c), at $t=25$ s,  the trajectory has smallest stability margin. In (d), our bilevel optimization is able to increase the stability margin at $t=25$ s.

% \fig{fig:benchmark_vs_robust_com} also explains why the proposed bilevel optimization achieves the robust solution. 

\fig{fig:benchmark_vs_robust_com} shows that both the benchmark and CIBO actually change the finger position $\revise{P}_{y}^{\revise{O}}$ by considering complementarity constraints \eq{slippingP}. In fact, we observed that at $t=25$ s, $\revise{P}_{y}^{\revise{O}}$ in both results moves to the negative value to maintain the stability of the object.
% 
In practice, we are unable to find any feasible solutions with fixed $\revise{P}_{y}^{\revise{O}}$, instead of using \eq{slippingP}. Thus, \eq{slippingP} is critically important to find a feasible solution. 

Next, we discuss how much CIBO improves the worst-case stability margin.
% Regarding optimality of robustness, 
The trajectories of $f_{nP}$ in \fig{fig:benchmark_vs_robust_com} show that the magnitude of $f_{nP}$ from CIBO increase at $t=25$ s to improve the worst-case stability margin. On the other hand, $f_{nP}$ from the benchmark optimization does not increase at $t=25$ s. Hence, we verify that by increasing normal force, the robot could successfully robustify the pivoting manipulation. 
This result can be also understood in \fig{fig:openloop_result} (c) and (d) where the stability margin in (d) at $t=25$ s is larger than that in (c), as discussed above. 


\tab{epsilon} and \tab{epsilon_peg} summarize the computed stability margin from \fig{fig:openloop_result}. In \tab{epsilon}, for the case where CIBO considers uncertainty of mass, we observe that the value of $\epsilon_-^*$ from CIBO is smaller than that from the benchmark optimization although the sum of the stability margin $\epsilon_+^* + \epsilon_-^* $ from CIBO is greater than that from the benchmark optimization. This result means that CIBO can actually improve the worst-case performance by sacrificing the general performance of the controller. Regarding the case where we consider the uncertain CoM location, CIBO outperforms the benchmark trajectory optimization in both $r_+^*, r_-^*$. For peg 1, the bilevel optimizer without using mode sequence-based optimization (i.e., hierarchical optimization) finds trajectories that have larger stability margins for both uncertain mass and CoM location as shown in \tab{epsilon_peg}. The trajectory of stability margin obtained from CIBO considering mass uncertainty is illustrated in \fig{fig:openloop_result_peg1}.
We discuss the results using CIBO with mode-sequence based optimization in Sec~\ref{sec:mode_seq_result}.
%how much our proposed controller is robust under uncertain parameters. 

\subsection{CIBO with Different Manipulator Initial State}

\begin{figure}
    \centering    \includegraphics[width=0.48\textwidth]{Figures/stability_different_py.pdf} % 
    \caption{Time history of frictional stability margin considering CoM location with different initial manipulator position $\revise{P}_{y}^{\revise{O}}(t=0)$.}
    \label{fig:diiferent_py_t0}
\end{figure}

\begin{table}[t]
    \caption{{Computed worst-case stability margin considering uncertain CoM location with different $\revise{P}_{y}^{\revise{O}}$ at $t=0$ over the control horizon obtained from optimization for gear 1.}}
    \centering
    \begin{tabular}{c|c}
     & $r_{+}^*$, $r_{-}^*$ [mm]\\
         \hline\hline Ours  with $\revise{P}_{y}^{\revise{O}}(t=0) = 0$  & 16.47, 1.36\\
         % \hline Ours  \eq{kkt_convertion} with mass uncertainty & 0.34, 0.50 & N/A \\
         \hline Ours  with $\revise{P}_{y}^{\revise{O}}(t=0) = 0.125w$  & 12.99, 2.98\\
         \hline Ours  with $\revise{P}_{y}^{\revise{O}}(t=0) = 0.25w$  & 10.00, 4.41\\
         \hline Ours  with $\revise{P}_{y}^{\revise{O}}(t=0) = 0.375w$  & 5.94, 5.67\\
         \hline Ours  with $\revise{P}_{y}^{\revise{O}}(t=0) = 0.5w$  & 1.94, 6.77
    \end{tabular}
    \label{py_table}
\end{table}


 %  CIBO is conditioned with states at $t = 0$ (i.e., $x_0$), which is true for other trajectory optimization frameworks.
% We discuss how initial states have an effect on the stability margin during pivoting. 
We believe that the efficiency of the optimization depends on the initial location of the manipulator finger. This is because the stability margin depends on the manipulation finger location, which is partially governed by its location at $t = 0$. Thus we present some results by randomizing over the manipulator finger location at $t = 0$.
We sample initial state of finger position $\revise{P}_{y}^{\revise{O}}(t=0)$ from a discrete uniform distribution with the range of $\revise{P}_{y}^{\revise{O}}(t=0) \in [-0.5w, -0.375w, -0.25w, -0.125w, \ldots, 0.5w]$. 
% As illustrated in \fig{fig:mechanics_pivoting_eq}, $\revise{P}_{y}^{\revise{O}}(t=0)=0.5w$ indicates that the contact location of finger tip is at $C_1$. 
Then we run CIBO considering CoM location uncertainty.

\fig{fig:diiferent_py_t0} illustrates the time history of stability margin with different $\revise{P}_{y}^{\revise{O}}(t=0)$. CIBO is not able to find feasible solutions with $\revise{P}_{y}^{\revise{O}}(t=0) < 0$. It makes sense since there may not be enough moment for the desired motion if $\revise{P}_{y}^{\revise{O}}(t=0) < 0$.

\fig{fig:diiferent_py_t0} shows that different $\revise{P}_{y}^{\revise{O}}(t=0)$ leads to different stability margin over the time horizon. 
% 
\tab{py_table} summarizes the worst-case stability margin over the trajectory obtained from \fig{fig:diiferent_py_t0}. \tab{py_table} also shows that the worst-case stability margin is different with different $\revise{P}_{y}^{\revise{O}}(t=0)$. 
% 
Finding a good $\revise{P}_{y}^{\revise{O}}(t=0)$ is not trivial and it requires domain knowledge. Thus, ideally, we should formulate CIBO where $\revise{P}_{y}^{\revise{O}}(t=0)$ is also a decision variable so that the solver can optimize the trajectory over $\revise{P}_{y}^{\revise{O}}(t=0)$ as well. 

Since CIBO is non-convex optimization, it is still possible that a feasible solution exists for $\revise{P}_{y}^{\revise{O}}(t=0) < 0$. However, we can at least argue that it is much more difficult to find a feasible solution with $\revise{P}_{y}^{\revise{O}}(t=0) < 0$ than that with $\revise{P}_{y}^{\revise{O}}(t=0) \geq 0$.


% Another important limitation regarding the complexity of dynamics is that CIBO is conditioned with states at $t = 0$ (i.e., $x_0$), which is true for other trajectory optimization frameworks. During the implementation of CIBO, we observed that it is not trivial to find "good" $x_0$ and the behavior of CIBO dramatically changes as  $x_0$ changes. For example, for a certain $x_0$, CIBO is able to find a solution while for another $x_0$, CIBO is not able to find a solution. Finding a good $x_0$ is not trivial at all and it requires domain knowledge. Thus, ideally, we should formulate CIBO where $x_0$ is also a decision variable so that the solver can optimize the trajectory over $x_0$ as well. 

\subsection{\revise{CIBO for Uncertain CoM parameters with Different Mass and Friction of Object}}
\revise{We first study how stability margin with uncertain CoM location changes with different mass parameters. We sample the mass of the object from a discrete uniform distribution with range of $m \in [0.1, 0.12, 0.14, 0.16, 0.18, 0.2]$ kg. Then we run CIBO considering CoM location uncertainty.}


\begin{figure}[t]
    \centering
    \begin{subfigure}{0.5\textwidth}
        \centering
        \includegraphics[width=\textwidth]{Figures/g7299.png} % Replace 'image1' with your image file name
        \caption{\revise{Time history of stability margin considering CoM location with different mass. The trajectory with the same color means that the same mass is used in the CIBO. The trajectories where $r>0$ are the trajectories of $r_+$ and the the trajectories where $r<0$ are the trajectories of $r_-$.}}
        \label{fig:sub1_cibo_mass}
    \end{subfigure}
    % \hfill
    % \begin{subfigure}{0.241\textwidth}
    %     \centering
    %     \includegraphics[width=\textwidth]{Figures/q_2_pivot_com_uncertainty_different_mass.pdf} % Replace 'image2' with your image file name
    %     \caption{Subfigure 2}
    %     \label{fig:sub2}
    % \end{subfigure}

        \hfill

    % \vspace{0.5cm} % Adjust vertical space between rows

    \begin{subfigure}{0.241\textwidth}
        \centering
        \includegraphics[width=\textwidth]{Figures/u_0_pivot_com_uncertainty_different_mass.pdf} % Replace 'image3' with your image file name
        \caption{\revise{Time history of $f_{nP}^O$}}
        \label{fig:sub3_cibo_mass}
    \end{subfigure}
    \hfill
    \begin{subfigure}{0.241\textwidth}
        \centering
        \includegraphics[width=\textwidth]{Figures/u_1_pivot_com_uncertainty_different_mass.pdf} % Replace 'image4' with your image file name
        \caption{\revise{Time history of $f_{tP}^O$}}
        \label{fig:sub4_cibo_mass}
    \end{subfigure}
    \caption{\revise{Results of CIBO considering CoM location with different mass.}}
    \label{fig:subplot_cibo_mass}
\end{figure}

\revise{\fig{fig:subplot_cibo_mass} shows the time history of stability margin and contact forces over the time horizon. 
% We observe that at $t \in [0, 15]$ s, the stability margin with different masses shows the same value.
For this analysis, the projection of CoM lies on the contact $B$ (i.e., $C_x^B = 0$.) at $t = 15$ s. At $t \in [0, 15]$  s (i.e., $C_x^B > 0$), the robot has to execute the contact forces to support the object against gravity. In fact, \fig{fig:sub3_cibo_mass} and \fig{fig:sub4_cibo_mass} show that the contact forces increase as mass increases. Since other parameters of the system are the same, the CIBO designs the trajectory whose stability margin is the same with different mass by changing the contact forces from the robot. At $t \in [15, 30]$ s,  the upper-bound of stability margin $r_+$ shows the larger value with the lighter object, and the lower-bound of stability margin $r_-$ also shows the larger value with the lighter mass of the object. This makes sense because as the object becomes lighter, the system allows for a longer moment arm $r$ in \revise{quasi-static} equilibrium.}
% since $C_x^B < 0$,


\begin{figure}[t]
    \centering
    \begin{subfigure}{0.5\textwidth}
        \centering
        \includegraphics[width=\textwidth]{Figures/g11154.png} % Replace 'image1' with your image file name
        \caption{\revise{Time history of stability margin considering CoM location with different friction at $P$. The trajectory with the same color means that the same mass is used in the CIBO. The trajectories where $r>0$ are the trajectories of $r_+$ and the the trajectories where $r<0$ are the trajectories of $r_-$.}}
        \label{fig:sub1_cibo_friction}
    \end{subfigure}
        \hfill

    \begin{subfigure}{0.241\textwidth}
        \centering
        \includegraphics[width=\textwidth]{Figures/q_5_pivot_com_uncertainty_different_friction.pdf} 
        \caption{\revise{Time history of $\revise{P}_{y}^{\revise{O}}$}}
        \label{fig:sub3_cibo_friction}
    \end{subfigure}
    \hfill
    \begin{subfigure}{0.241\textwidth}
        \centering
        \includegraphics[width=\textwidth]{Figures/u_0_pivot_com_uncertainty_different_friction.pdf} 
        \caption{\revise{Time history of $f_{nP}^O$}}
        \label{fig:sub4_cibo_friction}
    \end{subfigure}
    \caption{\revise{Results of CIBO considering CoM location with different friction.}}
    \label{fig:subplot_cibo_friction}
\end{figure}

\revise{Second, we study how stability margin with uncertain CoM location changes with different coefficients of friction between the object and the robot finger (i.e., $\mu_P$ at contact $P$ in \fig{fig:mechanics_pivoting_eq}). We sample the friction of the object from a discrete uniform distribution with a range of $\mu_P \in [0.6, 0.7, 0.8, 0.9, 1.0]$. Then we run CIBO considering CoM location uncertainty.}

\revise{\fig{fig:subplot_cibo_friction} shows the time history of stability margin, finger contact location $\revise{P}_{y}^{\revise{O}}$, and the contact normal force $f_{nP}^O$ over the time horizon. 
We observe that the different friction leads to different trajectories of the stability margin. In particular, we observe that the CIBO considering the lower $\mu_P$ results in a larger $r_+$. As \fig{fig:sub3_cibo_friction}, the finger keeps moving during the manipulation to complete the pivoting. It means that the complementarity constraints at $P$ \eq{slippingP} are always equality constraints like \eq{slipping_friction_cone}, $f_{tP}^O = \mu_P f_{nP}^O$. With the small $\mu_P$, the robot can execute the large $f_{nP}^O$ with the small $f_{tP}^O$, which is beneficial at  $t \in [0, 18]$ s to avoid losing the contact $A$,  before the projection of CoM lies on the contact $B$. 
% while we observe almost the same value of $r_i$ over the time horizon. 
% We observe that at $t \in [0, 15]$ s, the stability margin with different masses shows the same value.
}

\subsection{CIBO for Uncertain Friction Parameters}
\fig{fig:result_friction} shows the time history of frictional stability margin of gear 1 and gear 3 using \eq{eq:bilvel_friction2}. CIBO could successfully design an optimal open-loop trajectory by improving the worst-case performance of stability margin.
%
We observe that \fig{fig:result_friction} (b) shows a larger stability margin compared to  (a). This result makes sense since in (b), we consider gear 3 whose weight is heavier than the weight of gear 1 and thus we get stability margins which are bigger than those obtained for (a). 

% \begin{figure}
% \centering\includegraphics[width=0.49\textwidth]{Figures/friction_result.png} % 
%     \caption{Trajectory of frictional stability margin of (a) gear 1 and (b) gear 3, based on uncertain friction obtained from our proposed bilevel optimization \eq{eq:bilvel_friction2}, respectively. }
%     \label{fig:result_friction}
% \end{figure}



\begin{figure}
     \centering
     \begin{subfigure}[b]{0.241\textwidth}
         \centering
         \includegraphics[width=\textwidth]{Figures/friction_margin_gear1.pdf}
         \caption{}
         \label{fig:y equals x}
     \end{subfigure}
     \hfill
     \begin{subfigure}[b]{0.241\textwidth}
         \centering
         \includegraphics[width=\textwidth]{Figures/friction_margingear3.pdf}
         \caption{}
         \label{fig:three sin x}
     \end{subfigure}
        \caption{Trajectory of frictional stability margin of (a) gear 1 and (b) gear 3, based on uncertain friction obtained from CIBO \eq{eq:bilvel_friction2}, respectively. }
        \label{fig:result_friction}
\end{figure}



\subsection{\revise{CIBO for Uncertain Finger Contact Location}}\label{sec:result_bilevel_opt_uncertain_finger_contact_location}
\revise{In this section, we present results for pivoting manipulation under uncertain finger contact location. \fig{fig:finger_contact_location} shows the time history of the stability margin of gear 2 using \eq{kkt_convertion}. Our CIBO could successfully design a controller for an uncertain contact location. Also, \fig{fig:finger_contact_location} shows that stability margin has a quite large value at $t = 37$ s. At $t = 37$ s, the controller makes the finger move with zero normal force, resulting in a large stability margin as we explain in Sec~\ref{sec:stability_margin_finger_contact_location}. }


% \begin{figure}
%     \centering    \includegraphics[width=0.3\textwidth]{Figures/y_16_pivot_different_noise_finger_contact_location.pdf} % 
%     \caption{Time history of frictional stability margin considering finger contact location.}
%     \label{fig:finger_contact_location}
% \end{figure}

\begin{figure}
     \centering
     \begin{subfigure}[b]{0.241\textwidth}
         \centering
         \includegraphics[width=\textwidth]{Figures/y_16_pivot.pdf}
         \caption{}
         \label{fig:finger_contact_location_margin}
     \end{subfigure}
     \hfill
     \begin{subfigure}[b]{0.241\textwidth}
         \centering
         \includegraphics[width=\textwidth]{Figures/u_0_pivot.pdf}
         \caption{}
         \label{fig:contact_force}
     \end{subfigure}
        \caption{
        \revise{
        We consider CIBO with uncertain finger contact location. (a): Time history of frictional stability margin. (b) Time history of normal force at the finger. }
        }
        \label{fig:finger_contact_location}
\end{figure}

\subsection{CIBO over Mode Sequences for Non-Convex Objects}\label{sec:mode_seq_result}
% what we want to say:
% 1. using mode-based opt, you can don't need to tune parameters anymore
% 2. maybe you can show the better margin
% 3. highlight the change of mode with different shapes of objects 
In this section, we present results for objects with non-convex geometry using the mode-based optimization presented in Section~\ref{subsec:mode_based_optimization}.
\fig{fig:mode_result} shows the time history of states, control inputs, and frictional stability margins for pegs whose geometry are non-convex and the contact sets change over time. First of all, we can observe that CIBO in \eq{eq:mode_change} could successfully optimize the stability margin over trajectory while it optimizes the time duration of each mode. We observe that $\frac{T_1}{T_1 + T_2}$ (i.e., the ratio of mode 1 over the horizon) of peg 2 is much smaller than that of peg 3 since $\gamma$ (see \fig{fig:mode_concept} for the definition of $\gamma$) of peg 2 is smaller than that of peg 3 and thus, it spends less time in mode 1.   \fig{fig:mode_result} shows that $f_{tP}$ of peg 3 dramatically changes at $t = T_1$ s while that of peg 1 does not. 
% This result also makes sense because the shape of peg 3 is quite non-convex (\devesh{Yuki, can you please check--Is it right to compare the convexity of objects? Im not sure if we can use such words. please check}) and thus $f_{tP}$ needs to change accordingly once the contact mode switches from mode 1 to mode 2. 
In contrast, the shape of peg 2 has smaller $\gamma$ (i.e., less non-convex shape) and it can be regarded as a rectangle shape. Thus, the effect of contact mode is less, leading to a smaller change of $f_{tP}$ at $t = T_1$ s. 

In order to show that we can find solutions much more effortlessly using \eq{eq:mode_change} compared to two-stage optimization (that was earlier used in~\cite{9811812}), we sample 20 different $p_{y}$ at $t = 0$ s and count the number of times the benchmark two-stage optimization problem and the proposed optimization problem over the mode sequences \eq{eq:mode_change} can find feasible solution. We observed that the benchmark two-stage optimization problem found feasible solutions only 2 times while the mode-based optimization using \eq{eq:mode_change} was successfully able to find feasible 18 out of 20 times. Therefore, we verify that our proposed optimization problem enables to find solutions much more effortlessly. The benchmark method requires careful selection of parameters to ensure feasibility (as was explained in~\cite{9811812}).

% It is really worth noting that using \eq{eq:mode_change}, we can find solutions much more effortlessly. Before using REFER FIGURE, we had to manually indicate the initial and goal states of for mode 1


\begin{figure*}
    \centering
    \includegraphics[width=0.99\textwidth]{Figures/mode_opt_result3.png} % 
    \caption{We show the time history of object angle, finger position, contact forces from a manipulator, and frictional stability margins. The top row shows the result with peg 2 and the bottom one shows the result with peg 3. The pink and blue shade regions represent that the system follows mode 1 and mode 2, respectively.}
    \label{fig:mode_result}
\end{figure*}



\subsection{\revise{CIBO for Uncertain Mass on a Slope}}
\begin{figure}[t]
     \centering
     \begin{subfigure}[b]{0.241\textwidth}
         \centering
         \includegraphics[width=\textwidth]{Figures/mass_slope_positive.png}
         \caption{}
         \label{fig:mass_slope_uncertainty_positive}
     \end{subfigure}
     \hfill
     \begin{subfigure}[b]{0.241\textwidth}
         \centering
         \includegraphics[width=\textwidth]{Figures/mass_slope_negative.png}
         \caption{}
         \label{fig:mass_slope_uncertainty_negative}
     \end{subfigure}
        \caption{
        \revise{
        We consider CIBO with uncertain mass on varying angles of slope. (a): Time history of stability margin, $\epsilon_+$. (b) Time history of stability margin, $\epsilon_-$. The case where the object is on the slope whose angle of slope is $20\degree$ is illustrated in \fig{fig:mass_slope_uncertainty_negative}.}
        }
        \label{fig:mass_slope_uncertainty}
\end{figure}


\revise{We present results of objects with uncertain mass with varying angles of slope discussed in Sec~\ref{sec:sec_uncertain_mass_slope}. We consider gear 2 with $\phi = [-20\degree, 0\degree, 20\degree]$ as an angle of slope. 
}

\revise{\fig{fig:mass_slope_uncertainty_positive} and \fig{fig:mass_slope_uncertainty_negative} shows the time history of the stability margin $\epsilon_+$ and $\epsilon_-$, respectively. \fig{fig:mass_slope_uncertainty_positive} shows that the smaller $\phi$ is, the larger $\epsilon_+$ is during the manipulation. $\epsilon_+$ under mass uncertainty considers if contact $B$ is losing as we discuss in \eq{fnb_cond_mass}. \fig{fig:mass_slope_uncertainty_positive} means that contact $B$ can more easily lose contact as $phi$ increases. This makes sense because the larger the angle of slope $\phi$ is, the larger the moment which makes the object rotate along the counter-clockwise direction, resulting in the loss of contact at $B$.
Similarly, $\epsilon_-$ under mass uncertainty considers if contact $A$ is losing as we discuss in \eq{fna_cond_mass}. \fig{fig:mass_slope_uncertainty_negative} means that contact $A$ can more easily lose contact as $\phi$ decreases at $t = \in [0, 15]$ s.
This makes sense because the smaller the angle of slope $\phi$ is, the larger the moment which makes the object rotate along the clockwise direction, resulting in the loss of contact at $A$.
% Note that we observe the 
% 
% Based on the above discussion, we argue that our proposed CIBO can consider uncertain mass with different 
}


\subsection{CIBO for Patch Contact}
% what we want to say:
% patch contact has different trajectory
% analyze the reason and why it does not work
% here we talk about
% 1. patch one shows the better result and successfully 
% 2. not so much
% 3. over the trajectory still shows the better although optimization only cares about the worst-case

\tab{patch_table} shows the computed stability margin considering patch contact shows the greater margins for both positive and negative directions. Hence, we verify that our optimization can still work with patch contact and design the robust controller for maximizing the worst-case stability margin. 
Intuitively, this result makes sense since the contact area increases and the pivoting system has a larger physically feasible space, resulting in a greater stability margin. 

\fig{fig:patchcontact_gear2} illustrates the time history of frictional stability margin of gear 2 from CIBO with considering point contact and with considering patch contact.  Both CIBO with point contact and patch contact have the smallest (i.e., worst-case) stability margin at $t = 0$. However, CIBO with patch contact shows a greater margin at $t=0$, as we discuss above. In addition, over the trajectory, CIBO with patch contact shows a greater margin than that with point contact. Thus, we quantitatively verify that using patch contact is beneficial over the trajectory even though the optimization aims at maximizing the worst-case margin, not the stability margin over the trajectory. It is noted that we are not able to obtain better margins using patch contact due to the non-convexity of the underlying optimization problem.

% \textcolor{red}{YS: do we want to say that patch contact does not always give you better results? in general patch shows the better often, but not always, due to non-convexity of optimization}

% \textcolor{blue}{DJ: We can make such a comment. We should also include the generic contact patch model. Let me work on that. }


\begin{table}[t]
    \caption{{Average Solving Time (AST) comparison between benchmark optimization \eq{equation_control} and CIBO under mass uncertainty using \eq{kkt_convertion} with gear 2.}}
    \centering
    \begin{tabular}{c|c|c}
   $N$  &  AST (s) of \eq{equation_control} & AST (s) of \eq{kkt_convertion}\\
         \hline\hline 
       30 & 0.21 & 0.38 \\
       \hline
      60 & 0.50 & 0.68 \\
      \hline
         120 & 1.01 & 1.24  \\
    \end{tabular}
    \label{computation_benchmark_vs_CIBO}
\end{table}


\subsection{Computation Results}\label{subsec::computation}
\tab{computation_benchmark_vs_CIBO} compares the computation time between benchmark optimization \eq{equation_control} and CIBO under mass uncertainty using \eq{kkt_convertion} for gear 2. Overall, \eq{kkt_convertion} is not so computationally demanding compared to \eq{equation_control}. 
% 
However, as you can see in \tab{computation_friction} and \tab{computation_mode}, once the optimization problem has too many complementarity constraints because of the KKT condition, we clearly observe that the computational time increases. 

\tab{computation_friction} and \tab{computation_mode} shows the computational results for CIBO considering frictional uncertainty \eq{eq:bilvel_friction2} and bilevel optimization over mode sequences  \eq{eq:mode_change}, respectively. 

In general, the computational time for CIBO is larger than the benchmark optimization as CIBO has larger number of complementarity constraints. In the future, we will try to work on better warm-starting strategies so that we might be able to accelerate the optimization.

% Although we do not think our optimization can run in receding-horizon fashion like Model Predictive Control (MPC), we might be able to accelerate the optimization process with a better warm-start strategy. 



\begin{table}[t]
    \caption{{NLP specification for CIBO under frictional uncertainty using \eq{eq:bilvel_friction2} with gear 1.}}
    \centering
    \begin{tabular}{c|c|c|c}
   $N$  & $\#$ of Variables & $\#$ of Constraints & Average Solving Time (s)\\
         \hline\hline 
       30 & 2339 & 2280 & 1.9\\
       \hline
      60 & 4679 & 4560 & 10.6\\
      \hline
         120 & 9359 & 9130 & 30.9 \\
    \end{tabular}
    \label{computation_friction}
\end{table}

\begin{table}[t]
    \caption{{NLP specification for CIBO over mode sequences considering uncertain CoM location using \eq{eq:mode_change} with peg 3.}}
    \centering
    \begin{tabular}{c|c|c|c}
   $N$  & $\#$ of Variables & $\#$ of Constraints & Average Solving Time (s)\\
         \hline\hline 
       30 & 1648 & 1590 & 3.68\\
       \hline
      60 & 3298 & 3180 & 61.6\\
      \hline
         120 & 6598 & 6360 & 73.0 \\
    \end{tabular}
    \label{computation_mode}
\end{table}




\begin{table}[t]
    \caption{{Computed worst-case stability margin considering uncertain CoM location over the control horizon obtained from optimization for gear 2.}}
    \centering
    \begin{tabular}{c|c}
     & $r_{+}^*$, $r_{-}^*$ [mm]\\
         \hline\hline Ours  with point contact  & 5.27, 1.31\\
         % \hline Ours  \eq{kkt_convertion} with mass uncertainty & 0.34, 0.50 & N/A \\
         \hline Ours  with patch contact  & 6.81, 8.82
    \end{tabular}
    \label{patch_table}
\end{table}
\begin{figure}[t]
    \centering    \includegraphics[width=0.45\textwidth]{Figures/patch_result2.png} % 
    \caption{Trajectory of frictional stability margin of gear 2 based on uncertain CoM obtained from CIBO using point contact model and patch contact model, respectively. The vertical blue line represents the moment when the projection of CoM lies on the contact $B$.}
    \label{fig:patchcontact_gear2}
\end{figure}


\subsection{Hardware Experiments}
We implement our controller using a 6 DoF manipulator to demonstrate the efficacy of our proposed method. In particular, we perform a set of experiments to compare our method against a baseline method using gear 1. 
% \textcolor{red}{Devesh, could you put this explanation in the experiment setup section in section VI-A?}
% \revise{For all these experiments, we design a reference trajectory using the stiffness matrix of the robot. More specifically, we use the following relationship, $x_r[k]=x_{\text{CIBO}}[k]+f_{\text{CIBO}}[k]K_s^{-1}$, where, $x_{\text{CIBO}}$ is the position trajectory, $f_{\text{CIBO}}$ is the force trajectory obtained by CIBO whereas $K_s$ is the stiffness matrix for the robot which is appropriately chosen.}
% We obtain robust trajectories for the objects in \tab{parameter_table} from  \eq{kkt_convertion} and also obtain the benchmark trajectories from \eq{equation_control} using the same hyper parameters.
%  We evaluate the performance of different algorithms using gear 1.
 To evaluate robustness for objects with unknown mass, we solve the optimization with mass different from the true mass of the object and implement the obtained trajectory on the object. 
 % \revise{The true mass of gear 1 is $140$ g, which is used in the trajectory optimization. Then, }
%  To evaluate robustness of the bilevel technique, we solve optimization problem for gear with mass different from the true mass and implement the obtained trajectories. 
 We implement trajectories obtained from the two different optimization techniques 
 using 4 different mass values, \revise{$m = \{100, 110, 140, 170\}$ g. Then, we implement the obtained trajectory on the object with known mass. Note that the actual mass of gear 1 is 140 g. We test the trajectories over 10 trials for the two different methods.} 
%   We implement trajectories obtained from our proposed bilevel optimization and benchmark optimization using 4 different mass values.
%  Thus, in total we implement 8 different trajectories from the two different optimization methods.
 %Because we aim at generating trajectories of pivoting that would make a body stable even under unknown physical parameters (i.e., here, it is mass),

% \begin{table}[t]
%     \caption{{Number of successful pivoting attempts of gear 1 over 10 trials for the two different methods. To evaluate robustness for objects with unknown mass, we solve the optimization with  mass different from the known object and implement the obtained trajectory on the object with known mass. Note that the actual mass of gear 1 is 140 gm. }}
%     \centering
%     \begin{tabular}{c|c|c}
%  & CIBO & Benchmark Optimization\\
%          \hline\hline $m=100$ g  & 10 / 10 & 0 / 10 \\
%          \hline $m=110$ g  & 10 / 10 & 0 / 10 \\
%          \hline $m=140$ g  & 10 / 10 & 0 / 10 \\
%          \hline $m=170$ g  & 10 / 10 & 0 / 10 
%     \end{tabular}
%     \label{hardware_result}
% \end{table}

% \revise{We observe that }
% \tab{hardware_result} shows the success rate of pivoting for the hardware experiments. 
We observe that our proposed bilevel optimization is able to achieve 100 $\%$ success rates for all $4$ mass values while benchmark optimization cannot realize stable pivoting \revise{for all $4$ mass values over 10 trials}. 
% It means that our proposed bilevel optimization is actually able to enhance the robustness of the trajectory based on the frictional stability. 
Note that the benchmark trajectory optimization also generates trajectories with non-zero frictional stability margin but they failed to pivot the object. The reason would be that the system has a number of uncertainties such as incorrect coefficient of friction, sensor noise in the F/T sensor (for implementing the force controller), etc. which are not considered in the model. We believe that these uncertainties make the objects unstable leading to the failure of pivoting.  In contrast, even though CIBO also does not consider these uncertainties, it generates more robust trajectories and we believe that this additional robustness could account for the unknown uncertainty in the real hardware.  We also observe that the trajectories generated by benchmark optimization can successfully realize pivoting if the manipulator uses patch contact during manipulation (thus getting more stability).
%wrong physical parameters (e.g., coefficient of friction) and sensor noise, which are not modeled in frictional stability yet. 

We perform hardware experiments with additional objects to evaluate the generalization of the proposed planning method. All the objects used in the hardware experiments are shown in \fig{fig:hardware_objects}. 
 \fig{fig:hardwareresults} shows the snapshots of hardware experiments for the 4 objects detailed in \tab{parameter_table}. We observe that our bilevel optimization can successfully pivot all the objects during hardware experiments (see \fig{fig:hardwareresults} and the videos). This shows that we can use the proposed method with objects with different size and shape.

\begin{figure}[t]
    \centering
    \includegraphics[width=0.48\textwidth]{Figures/Picture3.jpg} % 
    \caption{Snapshots of hardware experiments. We show snapshots of the white peg and gear (instead of overlaid images) for clarity.}
    \label{fig:hardwareresults}
\end{figure}
%Our bilevel optimization could successfully generate robust trajectories and we could verify their robustness on different objects in real hardware experiments under mass and CoM location uncertainty by providing the inaccurate parameters with the optimizer.  

\begin{figure}
    \centering
    \includegraphics[width=0.48\textwidth]{Figures/hardware_objects.jpg} % 
    \caption{The different objects used in hardware evaluation of the proposed method. Please check the hardware experiments results in the video at this link \url{https://www.youtube.com/watch?v=ojlZDaGytSY}.}
    \label{fig:hardware_objects}
\end{figure}

\subsection{Recovery from Disturbance during Execution}\label{sec:error_recovery}
\revise{In the next set of hardware experiments, we present the recovery of the proposed controller from disturbances applied on the object during execution. For performing these experiments we use a cuboid object (see \fig{fig:camera_tracking_system} for the experimental setup, $l=110$, $w=55$, and $m=110$ g). During the execution of the trajectory, we apply random disturbances and record the object orientation using a vision-tracking system. The results from $5$ runs of the trajectory are shown in \fig{fig:error_plot_planning}. As can be seen in the figure, we apply external disturbance on every run of the trajectory at $t=30$. It is noted that the disturbance can not be large enough which results in loss of contact. As long as the contact between the object and the robot is maintained, the robust planner can successfully recover from the disturbance applied during execution and can reach the desired goal (see \fig{fig:error_plot_planning}).}

\revise{Furthermore, we also implement the algorithm in an MPC fashion to understand if it implements the algorithm in a closed-loop fashion as well as its performance. We use an initial reference trajectory planned by CIBO to initialize the controller. During online execution, we use a trajectory tracking cost function for CIBO. In particular, the vision system is used to estimate $\theta^W$ of the object. For brevity, we abbreviate the superscript $W$ here. The following cost function is used for CIBO:
\begin{equation}
   % c_{\text{MPC}}[k]=
   \lambda_1\sum_{t=k}^T \left(\hat{\theta}_t-\theta^\star_t\right)^2 +\lambda_2 u^2 \nonumber
\end{equation}
where $\theta^\star$ is the initial planned orientation trajectory of the object obtained by CIBO. We optimize the controller after every $10$ control steps till the object reaches the goal. Results of $5$ such runs are shown in \fig{fig:error_plot_mpc}. We apply random disturbances during execution between $t=20$ and $t=50$ as could be seen in the plot (please see the image inset in \fig{fig:error_plot_mpc}). As we can observe from these plots, the controller is successfully able to recover from these disturbances and thus, the controller can always guide the system from any initial state to the desired goal state. This shows that the proposed controller can be used in closed-loop to perform the desired pivoting manipulation. Note that we can not precisely estimate slip between the robot and the object accurately using the vision system. We believe we can generate more complex recovery behavior using additional slip information. However, designing such an estimator requires new hardware and additional work on tactile estimation~\cite{ota2023tactile} which is left as a future exercise. 
}
\begin{figure}
    \centering
    \includegraphics[width=0.42\textwidth]{Figures/feedback_tracking_system.jpg} % 
    \caption{\revise{The vision-based feedback pivoting system which can observe the state in real-time and adapt to recover from disturbances during execution. The inset image shows the tracking of the object using an Apriltag system.}}
    \label{fig:camera_tracking_system}
\end{figure}

\begin{figure}
    \centering
    \includegraphics[width=0.48\textwidth]{Figures/error_traj_pivoting.png} % 
    \caption{\revise{This plot shows recovery from disturbance applied during execution of the robust trajectories obtained from CIBO. The inset image shows the amount of disturbance applied during execution at $t = 30$. As could be seen in the plots, the robot we could successfully recover in all test runs.}}
    \label{fig:error_plot_planning}
\end{figure}

\begin{figure}
    \centering
    \includegraphics[width=0.48\textwidth]{Figures/mpc_pivoting_traj.png} % 
    \caption{\revise{We run the proposed CIBO in MPC fashion with state feedback using the vision system shown in \fig{fig:camera_tracking_system}. We apply multiple disturbances between $t = 20$ and $t = 50$ during execution which could be seen in the zoomed image. We show that due to the closed-loop nature of the controller, the controller is always able to guide the object to the desired goal. }}
    \label{fig:error_plot_mpc}
\end{figure}


