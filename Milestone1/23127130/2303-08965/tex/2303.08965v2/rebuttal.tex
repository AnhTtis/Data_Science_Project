
\documentclass[10pt]{article}    % Enable this line and disable the 
                                     % preceding line to obtain a two-column 
                                     % document whose style resembles the
                                     % printed Automatica style.


\usepackage{vmargin}
\setpapersize{USletter}
\setmarginsrb{2.25cm}{2.25cm}{2.25cm}{2.25cm}
                 {12pt}{20pt}{12pt}{36pt}
                 
\usepackage{palatino}
\renewcommand{\baselinestretch}{1.05}
\parskip = \medskipamount
                 
\usepackage{amssymb,url,amsmath}
\usepackage[square,sort,comma,numbers]{natbib}

%\usepackage{savesym}
%\savesymbol{AND} % SoderstromFOR AUTOMATICA STYLE (problem with algorithm)

\usepackage{graphicx,multicol}% for pdf, bitmapped graphics files
\graphicspath{{./../figures/}}
\usepackage{epsfig} % for postscript graphics files
\usepackage{mathptmx} % assumes new font selection scheme installed
\usepackage{times} % assumes new font selection scheme installed
\usepackage{euscript}
\usepackage{color}
%\usepackage{algorithmic}
\usepackage{algorithm,algpseudocode}
\usepackage[bookmarks=true]{hyperref}
\usepackage{tabularx}
\usepackage{color}
\usepackage{caption}
\captionsetup{tablename=Tab.}
\usepackage{subcaption}
\usepackage{makecell}
\usepackage{multirow}
\usepackage{array}
\newcommand{\PreserveBackslash}[1]{\let\temp=\\#1\let\\=\temp}
\newcolumntype{C}[1]{>{\PreserveBackslash\centering}p{#1}}
\newcolumntype{R}[1]{>{\PreserveBackslash\raggedleft}p{#1}}
\newcolumntype{L}[1]{>{\PreserveBackslash\raggedright}p{#1}}


\newcommand{\tq}[1]{\textquotedblleft #1\textquotedblright}
\newcommand{\until}[1]{\{1,\dots, #1\}}

\newcommand{\adolc}{\textsf{ADOL-C}}
\newcommand{\setcompl}{{\cal L}}
\newcommand{\ncoll}{N_{c}}
\newcommand{\setnfe}{{\cal N}_e}
\newcommand{\setncoll}{{\cal N}_c}
\newcommand{\mysub}[2]{[#1]_{#2}}
\newcommand{\ipopt}{\textsf{IPOPT}}

\newcommand{\revise}[1]{\textcolor{blue}{#1}}

\usepackage{multirow}

\definecolor{amethyst}{rgb}{0.6, 0.4, 0.8}

% comments
\newcommand{\red}[1]{{\bf\textcolor{red}{#1}}}
\newcommand{\blue}[1]{{\textcolor{blue}{#1}}}

% def
\def\R{{\Bbb R}}
\def\N{{\Bbb N}}
\def\P{{\Bbb P}}
\def\E{{\Bbb E}}

\newtheorem{theorem}{Theorem}
\newtheorem{proposition}[theorem]{Proposition}
\newtheorem{remark}[theorem]{Remark}
\newtheorem{definition}[theorem]{Definition}
\graphicspath{{./figures/}}

\newcommand{\pyrobocop}{\textsc{PyRoboCOP}}

\newcommand{\referee}[1]{\;
  \begin{minipage}[t]{.95\textwidth}
    ``{\small\color{amethyst} \textsc{#1}}''
  \end{minipage}\medskip
  }

\newcommand{\refereebit}[1]{\; ``{\small\color{blue} \textsc{#1}}'' }
\DeclareUnicodeCharacter{2212}{-}

\begin{document}
\pagestyle{myheadings}
\thispagestyle{empty}

\markright{{\sl Robust Pivoting Manipulation using Contact Implicit Bilevel Optimization
}}

\headsep 0.5cm

\bigskip\bigskip

\noindent{Prof. Wolfram Burgard,\\
  Editor-in-Chief, \textbf{IEEE Transactions on Robotics}
 %Australian National University
 }

\bigskip\bigskip

\begin{flushright}
  Devesh Jha\\ 
  Principal Research Scientist, MERL\\
  201 Broadway, Cambridge, MA 02139-1955\\
  jha@merl.com\\
\end{flushright}

\vspace*{2cm}

Cambridge, January 27, 2024%%
\medskip

\noindent 
\textbf{IEEE Transactions on Robotics (T-RO) 23-0270 :} {\sl Robust Pivoting Manipulation using Contact Implicit Bilevel Optimization}

\bigskip

Dear Prof. Wolfram Burgard,

Please find enclosed our revised manuscript IEEE T-RO: 23-0270  with the corresponding statement of revision.

We would like to thank you for the patience and effort in reviewing our paper. We appreciate the thoughtful and constructive comments from the associate editor and the reviewers, which we address in the following pages.

We look forward to hearing from you. 

Sincerely yours,

\vspace*{1.25cm}

\hspace*{5cm}  \begin{flushright}
    Y. Shirai, D. Jha, A.U. Raghunathan
\end{flushright}

\clearpage
\setcounter{page}{1}
\parindent=0pt

\begin{center}
  {\bf \LARGE Statement of Revision}
\end{center}

In the following, we provide a detailed point-by-point answer to all the reviewers' comments and concerns, together with the corresponding changes to the manuscript. In the revised manuscript, all the changes have been highlighted in {\color{blue}blue-colored} fonts. We have structured this statement of revision in separate blocks, corresponding to the remarks and suggestions made by each Reviewer and by the Editor. We would like to thank all for the thoughtful and constructive remarks.



\bigskip
%%
\hspace*{-25pt} \textbf{\large Comments by Editor}
%%
\begin{enumerate}
  \renewcommand{\labelenumi}{[E:\,\arabic{enumi}]}
  
  \item \referee{Three reviews have been collected for the present manuscript.      There is an overall agreement that the approach appears to be mathematically
    correct and of interest for the community active in nonprehensile
    manipulation.} 

    
    We thank the editor and the three anonymous referees for their constructive comments, and positive feedback. We hope our replies satisfactorily address all of the comments. 

    
      \item \referee{The reviewers however lament that the approach is limited to a
horizontal surface and vertical wall, as well as the fact that the
approach is planar and not a 3D one and that the only source of
uncertainty is limited to the mass, center of mass and friction, while
not considering geometric shape uncertainty.  The selected baseline
algorithm, furthermore, seems to be unsuitable for a correct comparison
of the approach, as it does not solve the problem in any circumstance
(cf. Table IX).} 

We appreciate the reviewers' comments and agree with them. Therefore, we introduced additional analysis and results to the paper.

First, we added the analysis of the pivoting manipulation on the slope in Section IV-F, which is the more generalized pivoting manipulation. Second, we added the analysis of the frictional stability margin under uncertain robot finger contact location in Section IV-E, which indirectly considers the geometric uncertainty of the object. 


Regarding the pivoting manipulation in 3D, we agree that it is quite important but we would argue that it is quite difficult. Please check our reply to the comment \ref{r2:3} by the reviewer 9. 


Regarding the baseline algorithm, we removed Table IX based on the comment \ref{r1:8} by the reviewer 3. 
Please check our reply to the comment \ref{r1:8} by the reviewer 3. 
    
  \item \referee{One reviewer points out the lack of novelty. Another one provides a
list of references that could be considered to be included. It is thus
required to clearly dissipate any doubt about the novelty of the work
in the next resubmission and better motivate why pivoting manipulation
is interesting from an application perspective.
} 
We appreciate the reviewers' comments and agree with them.

Regarding the novelty of the paper, we have the following contributions we would like to emphasize
% compared to our original conference paper \cite{robust_pivot_icra22}:
\begin{itemize}
    \item \textbf{Insight of mechanical stability from friction}: Although there are some manipulation works that work well, it is still not clear at least for dexterous manipulation why and how it works well or not. This paper argues that friction forces are the key to introduce stability to manipulation tasks such as the pivoting manipulation under uncertain physical parameters (e.g., mass, friction). Compared to our original conference paper, \cite{robust_pivot_icra22}, we consider uncertainty in coefficients of friction, patch contact, the finger contact location, and mass on a slope. The uncertainty in the finger contact location and mass on a slope are the new results in the revised paper thanks to the comments by the reviewers. In particular, we believe that considering uncertainty in coefficients of friction and the finger contact location improves the value of this paper because they are in general very difficult to estimate during the manipulation. 
    % need to study why it works. losing contact and formally discuss why and how it loses contact under various 
    \item \textbf{Computational tool}: Designing robust open-loop controller is challenging for contact-rich systems because there is no control inputs $u$ to satisfy all realizations of uncertainty with equality constraints. Our novel optimization formulation designs the optimal control inputs for contact-rich systems while improving the worst-case stability margin. We demonstrate our CIBO extensively in the paper thanks to the comments by the reviewers. We even demonstrate our CIBO for objects whose geometry is not convex. We believe that our paper makes some progress toward model-based dexterous manipulation by providing the novel optimization formulation with detailed analysis. 
    % \item \textbf{A number of verification}: 
\end{itemize}


Regarding the references, we added the papers suggested by reviewer 10 as references. Please check the reply to the comment \ref{R10:9} by reviewer 10.

% \revise{Devesh, could you organize introduction with updated figure 1 to motivate why pivoting is important? Update - I'll do that}

  \item \referee{Finally, it is required to better introduce and clarify the notation,
as well as the optimization problem solved: the notation and
mathematical derivation is at times hard to follow. To this end, it
could be considered to introduce the notation and provide clarifying
drawings in one place, at the beginning of the manuscript.
} 
We have introduced notation at the beginning of the manuscript. Please see our reply in the detailed comments.

    

\end{enumerate}


%%%%%%%%%%%%%%%%%%%%%%%%%%%%%%%%%%%%%%%%%%%%%%%%%%%%%%%%%%%%%%%%%%%%%%%%%%%%%%%%%%%%
\clearpage

\bigskip
%%
\hspace*{-25pt} \textbf{\large Comments by Reviewer \# 3 (Review ID 99061)}
%%

\begin{enumerate}
  \renewcommand{\labelenumi}{[R3:\,\arabic{enumi}]}  
  
	\item \referee{This paper presents an algorithmic approach for planning trajectories
to manipulate a planar object by pivoting it between a horizontal floor
and against a vertical wall. The key contribution is a formulation that
accounts for uncertainty in the mass of the object, its center of mass,
or the coefficient of friction between the object and the environment.
This is done through a bilevel optimization process by which first
maximum margins of stability are optimized for a trajectory and then in
a high level loop the trajectory is optimized to increase the current
minimum margin.} 

    We thank the reviewer for a thorough review of our manuscript. We address the comments by the reviewer in the following text.

   \item \referee{The problem is relevant to the community of mechanics-based robotic
manipulation, extending classical notions of robustness, for example in
grasping, to the quasi-static trajectories that come up from pivoting
an object. The proposed algorithm is sound, and is thoroughly evaluated
both against the model and in real experiments. I would like to highlight, however, what I think are a few weaknesses of the paper:}
We appreciate the reviewer's kind comments. And we address the reviewer's comments as follows. 
    

   % \item \referee{I would like to highlight, however, what I think are a few weaknesses of the paper:}

   \item \referee{The description of the technical details in the paper is hard to
follow. The notation is very complex, and I would say not standard
(e.g. notation commonly used for grasp stability analysis). The
notation is also introduced all throughout the paper. Since the paper
is so notation heavy, I suggest to create a section close to the
beginning that includes all the terminology and notation, with diagrams
that clearly describe the reference frames used. For example, as far as
I could see, it is never clearly stated in the paper where  is the
origin of x and y is, and is left for the reader to guess. Similarly it
would be very useful to clearly state the complete optimization problem
that the paper is solving, with input, output, objectives, and
constraints before heading into the many pages of describing the
technical details. 
}\label{R1.4}

We really appreciate and agree with the reviewer's comments. Thus, we introduced Table 1 in Section III which summarizes the notation we use in the paper. We also added
Figure 3, which shows the definition of frames explicitly. Regarding the clarity of the optimization problem, we added additional explanations in the first and second paragraphs of Section V-A, which explain the objective function, the constraints, and the input and output of the optimization problem.


   \item \referee{The paper is narrowly focused on a very specific instantiation of the
pivoting problem. The floor is horizontal, and the wall is vertical.
The object makes single point contact with each of these surfaces, and
both contacts are assumed to slide. While some of these are not likely
necessary to the method, the notation and algebraic expressions in the
paper make use of them. This constraints the problem formulation and
the applicability of the math that is derived in the paper to a narrow
set of conditions. This is heightened by the fact that the technical
derivations of the mechanics and the margins of stability in the paper
are quite lengthy. For such an investment in notation and space, one
would expect a description with slightly more general notation and
algebra, applicable to a more general case..}
We appreciate and agree with the reviewer's comment. Therefore, we added a new analysis where we consider an object with uncertain mass on a slope in Section IV-F, which leads to more general pivoting manipulation. 
Also, regarding point contacts with extrinsic contact surfaces, we can use the same technique we used in Section IV-G where we present patch contact model, which is the new result compared to our conference paper \cite{robust_pivot_icra22}. Thus, we add \textit{remark} in Section IV to explain this.
Regarding the slipping contact assumptions at extrinsic contacts, it is quite challenging to discuss in this paper for the same reason we described in Remark 1 in Section V-B due to the non-convexity of the lower-level optimization, which is out of the scope of this paper. Hence, we added this explanation in Remark 2 (previously Remark 1).
% 
We hope that these new results and explanations will be beneficial to readers.


   
   \item \referee{The paper describes how to compute margins of stability for the mass
of the object, its center of mass, and for friction with the
environment. These, however are determined independently from each
other, and the paper is not clear what the implications are from these
not being independent in practice.}\label{r3:5}
We thank this comment by the reviewer. If we consider some of uncertain parameters simultaneously, the resulting optimization problem becomes non-convex and thus it can be very computationally demanding as explained in Remark 1 in Section V-B. Another reason why we do not consider simultaneous uncertain parameters is that once the lower-level optimization becomes a non-convex optimization problem, there is no guarantee that the lower-level optimization finds globally optimal solutions. It means that the lower-level optimization problem outputs some random values $\epsilon_-$ and $\epsilon_+$ from  $\epsilon_-^*\leq \epsilon_- \leq \epsilon \leq \epsilon_+ \leq \epsilon_+^*$, not the actual stability margins ($\epsilon_+^*$, $\epsilon_-^*$) from (27c) and (27d). Then, the upper-level optimization designs the controller with these fake margins. Therefore, the upper-level optimizer might output non-optimal controller, which we want to avoid. We added this explanation briefly in Remark 2. 

In practice, the choice of the particular parameter for the which one should use CIBO to obtain robust trajectories depends on the amount of uncertainty in different parameters associated with the manipulation task. For instance, if we have access to the CAD model of the objects, we can have a good guess of mass and CoM location of the object and thus the major source of uncertainty can be from other parameters such as coefficients of friction. We added Remark 3 in the section V-C to briefly describe this explanation. 


% Maybe we can say that considering them simultaneously is computationally intractable. We can also talk about non-convexity of the lower-level optimization of our bilevel optimization.  

    
   \item \referee{The paper should be more clear on how the output of the optimization
problem is passed to the robot controller. The paper describes a
stiffness controller, but if the output of the optimizer is a
trajectory of state and forces, how are these combined before being
sent to the stiffness controller? Is there some kind of hybrid
velocity/force controller? Or is the model of the stiffness included in
the optimization? Or in some other heuristic manner?}
We appreciate the reviewer's comment. To implement the computed force trajectory during manipulation, we use the default stiffness controller for the robot. By selecting an appropriate stiffness matrix, we design a reference trajectory that would result in the desired interaction force required for manipulation. 
We added this explanation in paragraph 3 of Section VI-A.


   \item \referee{ Table 9 shows experimental results comparing with a benchmark that
fails 100\% of the time in all different experiments, while the proposed
algorithm works 100\% of the time in all different experiments. This
table probably does not add too much value and speaks to the fact that
the benchmark is rather an algorithm that does not solve the problem.}\label{r1:8}
We appreciate and agree with the reviewer's comment. Therefore, we removed Table 9 and added an explanation in the first and the second paragraph of  Section VI-9 to mention that our proposed algorithm works 100 $\%$ of the time while the benchmark does not work, which justifies that the pivoting manipulation is not easy manipulation task and requires a robust open-loop controller. 



\item\referee{The paper ofter refers to the dynamic regime of the task as static,
in other case uses the term quasi-static. I believe it is more
appropriate to refer to it as quasi-static, but in any case it should
be consistent.}
Thank you so much for your feedback. We agree with your comment and thus we modified the paper such that we use the term quasi-static instead the term static.


\item\referee{Section III.A defines $p_y$ as the length along the side of the object
where point P is located. This notation is confusing since y is used in
the rest of the paper to refer to the vertical axis, and this is noting
a length a long the side of the object which is at an angle.}
We thank and agree with the reviewer's comment. Based on the review comment \ref{R1.4}, we change the notation of the variables such that they are notated with their defined frames. Thus, we use $P_y^O$ to indicate that $P_y^O$ is the contact location at $P$ along $y$-axis of frame $O$.

\item\referee{In this same section when the derivative of $p_i$ is introduced
($\dot{p}_i$) as the subtraction of the slipping velocities along the
positive and negative directions, I do not understand what these
positive and negative directions are.}
We appreciate and agree with the reviewer's comment.
We also apologize for the confusing definition. 
Thus, we explicitly define the frame $O$ in Figure 2. 
Thus, $\dot{P}^O_{y+}$ and $\dot{P}^O_{y-}$ are the positive and negative slipping velocity along $y$-axis of frame $O$.
Also, we added sentences in paragraph 4 of Section III-A for the clear explanation.


% is (3c) considering the distribution of complementarity constraints (i.e., stochastic constraints). 
% In other words, (5c) argues that (3c) is satisfied with a certain probability. 

\item\referee{Section IV.D says "Note that we do not need to know any information
regarding the distribution of the unknown part" I do not understand
what distribution refers to. }
We apologize for the confusion. What we mean is that our goal is to get the boundary of the random variable $\epsilon_{A}$ and $\epsilon_{B}$, which occurs from $\epsilon_{\mu_A}$ and $\epsilon_{\mu_B}$. Then, in order to discuss the boundary of $\epsilon_{A}$ and $\epsilon_{B}$, we do not need to know the underlying probability density function (e.g., Gaussian distribution, beta distribution) of $\epsilon_{\mu_A}$ and $\epsilon_{\mu_B}$. 
% We do not even need to know the 
Our proposed optimization problem figures out the upper-bound and the lower-bound of $\epsilon_{A}$ and $\epsilon_{B}$, which indirectly gives the bounds of $\epsilon_{\mu_A}$ and $\epsilon_{\mu_B}$. To make this statement clear, we added more explanation in the first paragraph of Section IV-D.

% This discussion is similar to the discussion of comparison between robust optimization and stochastic optimization (Please see preface xiii in \cite{robust_textbook} as your reference). 
% $ which are given by our proposed optimization algorithm using \revise{(17)}.

\item\referee{Section IV.D introduces notation for uncertainty in the coefficient
of friction, however the term $\epsilon$ is used both as uncertainty in
coefficient of friction, as well as uncertainty in friction force,
which have different units. It'd be useful to simplify the notation.
Either use different symbols, or just use one or the other one if both
are not necessary.}
We appreciate your comments and feedback. We definitely agree with your comment.
Thus, we updated the notation of the uncertain variable of coefficients of friction as  ${\tilde{\mu}_i}, \forall i=A,B$ instead of $\epsilon_{\mu_i}$.

\item \referee{Equation (18) shows a 2 in front of the force $f_{xP}$. I assume that
$f_{xP}$ is the resultant of the normals of the two forces applied at the
patch contact. In that case, wouldn't the 2 be a typo?}
We apologize for the confusion. What we meant was that we define $f_{xP}$ as the contact force at each contact point, i.e., $f_{xP_1} = f_{xP_2} = f_{xP}$ where $f_{xP_1}$ and $f_{xP_2}$ are the normal force at point contact $P_1$ and $P_2$, respectively. 
Thus, $f_{xP}$ is not the resultant of the normal forces of the two forces applied at the patch contact. 
Since our notation is confusing, we modified the notation such that we explicitly use $f_{xP_1}$ and $f_{xP_2}$, instead of simply using $f_{xP}$. 

\item \referee{The sentence in section V "The proposed CIBO considers frictional
stability margin along the entire trajectory for manipulation and then
maximizes the minimum margin in the proposed framework" is a good
description of the key method and would be useful to have a similar
sentence in the introduction or when describing the overall algorithm.}
We appreciate your kind feedback. We agree with the reviewer's comment. Hence, we added the sentence in the third paragraph of Section I to describe the big picture of our proposed algorithm.
\end{enumerate}

%%%%%%%%%%%%%%%%%%%%%%%%%%%%%%%%%%%%%%%%%%%%%%%%%%%%%%%%%%%%%%%%%%%%%%%%%%%%%%%%%%%%
%\clearpage

\bigskip
%%
\hspace*{-25pt} \textbf{\large Comments by Reviewer \# 9 (Review ID 99421)}
%%


\begin{enumerate}
	  \renewcommand{\labelenumi}{[R9:\,\arabic{enumi}]} 

   \item\referee{The paper under review investigates the impact of friction and
uncertainties in the center of mass (CoM) position on pushing-based
pivoting manipulation. It proposes a bi-level optimization approach to
determine stable margin bounds for the uncertain parameters at the
lower level and maximize the worst margin at the upper level. This
method effectively generates an optimal trajectory for pushing-based
pivoting manipulation. Additionally, the paper explores different
contact modes and applies the bi-level optimization method to sequences
of these modes, enabling the extension of manipulation to non-convex
objects. The manuscript is well-written and easy to comprehend, while
the experiments and examinations provide satisfying evidence supporting
the authors' claims.}
   We thank the reviewer for a thorough review of our manuscript. We address the comments by the reviewer in the following text.


    \item \referee{Despite the positive aspects mentioned above, I have reservations about
the work. Although uncertainty is an intriguing topic in the field of
robotics, the paper's approach of assuming a limited number of
uncertain parameters and optimizing against these assumptions
diminishes its overall appeal. Notably, geometric shape uncertainty,
which is a significant source of uncertainty, cannot be adequately
addressed by solely considering friction and CoM parameters.
Furthermore, dealing with uncertainties in geometric shapes becomes
even more challenging when considering different contact modes.}\label{r2.1}


% uncertainty in shape is challenging (making and breaking contact) but important.
% we consider finger contact location which is also another source of uncertainty. Also, it does kinda relative geometry is incorrect. We still assume that contact does not break contact, which introduce additional discussing such as stochastic complementarity constraints, we cite the paper. T
% To the best of our knowledge, this work considers robust optimization for challenging manipulation 3d pivot. which is not scope of this paper. Our goal is to understand the underlying mechanics and study the fundamental 
We appreciate the reviewer's comment. 
Addressing uncertainties in robotic manipulation, particularly those related to geometric shapes, presents a complex challenge including changes in contact modes such as between making and breaking contact. We understand the limitations in our approach that primarily focus on a restricted set of uncertain parameters, such as friction, mass, and center of mass. We added this explanation in Section VII to emphasize the limitation of this paper. 

Based on the reviewer's comment, we additionally consider the manipulator's finger contact location in Section IV-E. Uncertainty in the finger contact location indirectly means the uncertainty in the shape of the object since the relative pose between the object and the manipulator changes due to the finger contact location uncertainty. Therefore, we believe that considering the finger contact location is a good strategy to indirectly deal with the geometric uncertainty in the object. 

Our paper also considers frictional uncertainty in Section IV-D, which is a new result with respect to our original result in \cite{robust_pivot_icra22}. Uncertainty in friction constants can also change the contact mode between slipping and sliding contact. We believe our new result in this paper will be useful to design and analyze a robust controller for uncertain friction constants, which prevents the system from changing contact modes.


% While our work doesn't delve into robust optimization for pivoting under geometric uncertainties, 
The primary objective of the paper is to delve deeper into the underlying mechanics and study the fundamental aspects of uncertainty in the challenging manipulation task such as pivoting. We acknowledge the reviewer's insightful comments and aim to further explore and expand our understanding of these challenging uncertainties in future research endeavors.


    
    \item \referee{Apart from the limited discussion on uncertainties, I am also concerned
about the lack of novelty in the paper. There are two interesting
problems related to this topic that could have been explored further:
1. Real-time feedback utilizing tactile or vision sensors, and 2. 3D
analysis considering generalized friction cones. Unfortunately, these
problems are merely mentioned as future work in the manuscript, which
leaves me disappointed.}\label{r2:3}
Thank you so much for your feedback. We performed real-time feedback utilizing vision sensors using our proposed optimization framework in receding-horizon-horizon fashion, which is discussed in Section \textcolor{red}{VI.L}. In this section, we present recovery from disturbance of the controller computed using the proposed controller to further reinforce the robustness of CIBO. Furthermore, we also design a closed-loop controller which makes use of a visual tracking system to run the pivoting manipulation in closed loop. However, we found that estimating slip information is difficult using vision sensors. We would need additional tactile sensors to recover slip information between the robot and the object. This can help us perform more complex recovery and online control. However, this would require additional tactile hardware and software for slip detection \& estimation. Designing tactile-based closed-loop controller is left as a future exercise at the moment.

We believe that considering 3D analysis including generalized friction cones is quite interesting but challenging. 
One of the reasons is that it is necessary to consider the Minkowski sum of friction cones for all contact locations, which can be not trivial since the contact mode can change. For example, let's consider the patch contact in 3D. The contact mode can change from patch contact to line contact or even point contact, which makes the design of the open-loop controller quite challenging. 
In fact, many robotic dexterous manipulation works also only consider manipulation in 2D \cite{motion_cone_ijrr, Hou_rss2020, Taylor_2023iros} although we also consider manipulation in 2D for objects whose geometry is non-convex. Although it is quite interesting to work on as we describe in Section VII, it is quite challenging to work under the scope of this paper. 
We believe that our algorithm for non-convex objects can be applied to the pivoting in 3D, which requires another detailed studied and can not be covered in the current framework. 
% We believe that working on generalized friction cones in 3D can be another TRO paper. 
   
    
    \item\referee{On a minor note, it would be worthwhile to assess the performance of
the proposed method using objects with varying gravitational or
frictional properties. For instance, evaluating how the method responds
to different states of bottles (full, half, 1/4, etc.) could yield
interesting insights. Additionally, explicitly discussing the influence
of finger materials or the mutual frictional interaction between the
fingers and objects would be valuable.} 
We appreciate and agree with the reviewer's comment. Therefore, we added additional results of our proposed optimization considering center of mass uncertainty with varying mass and coefficients of friction in Section VI-D. 
 

\item\referee{On page 5, again mainly for thoroughness, perhaps discuss what the original conference paper contributed and how the additions in this paper make for a more useful and significant overall work.
} 
Thank you for your suggestion. We have the sentences summarzing what additional contributions we have in this paper in the paragraph 5 of section I.


\item\referee{In conclusion, I recommend rejecting the paper for now and suggesting
re-submission. While the paper lacks significant novelty, it does have a
satisfactory presentation and well-conducted experiments and analyses.
However, I maintain reservations about its publication in TRO unless
the aforementioned two interesting problems can be adequately
addressed.
}
This paper studies robust manipulation with two external contact points. We analyzed several different uncertainty cases where uncertainty could arise from physical properties like mass, coefficient of friction, CoM, and the contact point between the robot and the object. We also provide results for non-convex geometry which is also not fully understood in manipulation literature. We introduced additional results including real-time feedback utilizing vision sensors demonstrating that we can run the proposed method in closed-loop in an MPC fashion. We performed several experiments to show robustness as well as recovery from external disturbances during execution of the proposed controllers.

While kinematic uncertainty is definitely important and very interesting, kinematic uncertainty can lead to uncertainty in making or breaking of contact. Our current framework can not model such kinds of uncertainty during planning. Once the contact breaks, a re-planning has to be done using an online estimator which can update the uncertain kinematic parameters. This requires a different framework and thus, is left as a future exercise. However, we believe that the current manuscript covers lots of different uncertainty scenarios which hopefully can be useful for various different kinds of non-prehensile manipulation. 



\end{enumerate}

\bigskip
%%
\hspace*{-25pt} \textbf{\large Comments by Reviewer \# 10 (Review ID 99745)}
%%

\begin{enumerate}
	  \renewcommand{\labelenumi}{[R10:\,\arabic{enumi}]}
\item\referee{The paper presents a robust contact-implicit bilevel optimization
(CIBO) framework for planning of pivoting manipulation in the presence
of uncertainties. Friction provides a stability margin during
manipulation ñ analytical expressions for this margin are derived and
maximized in CIBO to provide robustness against uncertainty in several
physical parameters of the object including mass, center of mass, and
coefficient of friction. Since this is an evolved paper, the authors
also consider patch contact (in addition to point contact) and
non-convex objects in their method and evaluation.}

We thank the reviewer for the kind words of appreciation.
\item\referee{The significance of this paper is clearly explained in the
introduction. Robot operation in unknown environments with
uncertainties in object properties requires robust planning, especially
in the presence of contact. The authors identify key physical
parameters like mass, center of mass, and coefficient of friction and
formulate an optimization technique that considers uncertainties in
these parameters for contact-rich manipulation in the real-world. More
specifically, they ensure mechanical stability of the overall system by
exploiting friction.
Very clearly written. There is a clear identification of parts that are
new/evolved from their previous conference submission. Organized in a
way that is easy to follow. Provides sufficient detail and intuitive
understanding even for someone with a cursory knowledge of the field.
The novel concept of "Frictional Stability" and the proposed bilevel
optimization technique are well explained and their significance is
clearly highlighted. The paper does a commendable job of detailing the
mechanics of pivoting, including the assumptions made, the mechanics of
pivoting with external contacts, and the concept of frictional
stability margin.}
We appreciate the reviewer's kind comment. 





\item\referee{Sorry if I missed this in the evaluation, but you only present success
rates for gear 1? Other objects are acknowledged in Figure 15 and 100%
success rate is briefly mentioned in text. But it is also stated in
Discussion and Future Work that
> Without a closed-loop controller, even the robust trajectories need
to be initialized precisely and the system can not recover from a
failure.}\label{r3:100_success_case}
We apologize for the confusion. We only showed the success rate for gear 1 using Table IX since we did not test with other objects over 10 trials. The purpose of hardware experiments for other objects was to show the generalization of the proposed framework. 
What we meant was in Discussion and Future Work was to argue that once uncertainty is too large, robust open-loop controller might not complete its desired task. For example, we consider gear 1 whose weight is 140 g and we consider the mass value such as 100 g as input of CIBO, where the different is 40 g. If we use our CIBO which considers the mass value as 1 kg, since the gap of the mass is 860 g, the robot might not be able to accomplish the pivoting. Although our CIBO is able to design robust trajectories, in order to deal with a larger uncertainty, using feedback control in addition to our proposed CIBO would be beneficial. We also implement a closed-loop controller with vision feedback which operates in an MPC fashion where we use CIBO for re-computation of controller up on state feedback. We show that we are able to achieve additional robustness for the closed-loop controller.

%To make this statement more clear, we modified the paragraph in "Closed Loop Control" in Section VII.


\item\referee{Did you find that your use of an open-loop controller led to some
failures? This should be presented in the Evaluation if so.}
The computed controller was found to be very robust to lots of different kinds of uncertainty. However, we think that our robust open-loop controller might fail once the uncertainty of the system is very large as we describe in R10:\ref{r3:100_success_case}. In general, we found that our controller would always succeed as long as there is no unexpected loss of contact between the robot and the object. This could be related to either feasibility of manipulation in lack of enough friction or some external disturbance leading to loss of contact. We present new results in Section VI.L showing that the proposed controller is robust to external disturbances as long as there is no loss of contact. In the future, we would like to investigate design of closed-loop control with additional slip information between the robot and the object; however, that requires additional tactile sensing and algorithms to detect and estimate slip.

\item\referee{Discussion/Section V.D also states:
> considering these non-convex constraints inside the lower-level
optimization problem is not guaranteed to find globally optimal safety
margins
What is the effect of this on the success rate? Or more generally, does
this limit the performance of the optimizer or present itself as an
experimental failure down the line?}
Thank you so much for your comment. Since the lower-level optimizer does not find globally optimal safety margins, the upper-level optimizer designs a sub-optimal controller. Using this sub-optimal controller, we expect that the robot might not be able to complete the pivoting, meaning that the performance of the controller worsen. 

We did not evaluate it in the hardware experiment since the computational complexity increase dramatically due to the non-convexity in the lower-level optimization problem. 
We believe that designing the controller with non-globally optimal safety margins does not physically make sense. It means that the controller is designed not with margin bound (Please check Fig. 7). The lower-level optimizer returns some random values in the gray box at each time step, not the bound represented as the orange cap at each time step. 
It would be also appreciated if you can check our reply to the comment \ref{r3:5} by the reviewer 3.

\item\referee{Minor: A brief line on how the object state was tracked during the
experiments would be helpful to understand the setup. Also, were the
contact points between the robot and the object and the object and the
support surface tracked?}
We thank the reviewer's comment. We added sentences in Section VI-A to mention that object states and contact states are not tracked for evaluating the CIBO in open-loop during hardware experiments. We added new results using MPC-based control as shown in Section VI-L. For MPC results, we tracked the object state using an AprilTag system described in Section VI-A. The robot states are estimated using the robot's joint encoders.  The contact states between the robot and the object and the object and the support surface are not estimated at the moment and are left as future work. It is extremely difficult to precisely measure the contact states using vision alone and thus, we do not use the vision estimates to track the contact state between the robot and the object.


\item\referee{I see you acknowledge the use of pivoting manipulation in grasping
and/or assembly, but Iíd like to see a little more motivation on why
pivoting manipulation is interesting from an application perspective,
i.e., more examples of compelling use cases or positioning in the
broader context of nonprehensile manipulation. Even better if you can
show in video how your method can be used to supplement grasping or
assist in assembly.}
We appreciate the reviewer's comment. Hence, we added the assembly video in the supplemental multimedia where the robot could successfully accomplish assembly by using non-prehensile manipulation (i.e., pivoting). We have provided a new example of an assembly scenario based on reviewer's request. Please see the attached video between 6:20 to 7:10. In this video, we show how the pivoting manipulation could be used in modular assembly systems where the robot can manipulate the pose of the object to the desired pose during grasping. In this video, we show how the proposed pivoting manipulation can be used during assembly of a functional gear box.

\item\referee{These are some other works I found that refer to pivoting manipulation and likely should be acknowledged in the paper. Some of these may not
be directly related but still worth acknowledging since they are
different forms of pivoting manipulation. Clarifying your own
definition against a backdrop of these could be helpful.
\begin{itemize}
    \item Karayiannidis, Yiannis, et al. \"In-hand manipulation using gravity and
controlled slip.\" 2015 IEEE/RSJ International Conference on Intelligent
Robots and Systems (IROS). IEEE, 2015.
\item Karayiannidis, Yiannis, Christian Smith, and Danica Kragic. "Adaptive
control for pivoting with visual and tactile feedback." 2016 IEEE
International Conference on Robotics and Automation (ICRA). IEEE, 2016. (Handles controlled slipping much like your paper but for in-hand
manipulation. )
\item Chavan-Dafle, Nikhil, et al. "A two-phase gripper to reorient and
grasp." 2015 IEEE International Conference on Automation Science and
Engineering (CASE). IEEE, 2015.
\item Dafle, Nikhil Chavan, et al. "Extrinsic dexterity: In-hand manipulation
with external forces." 2014 IEEE International Conference on Robotics
and Automation (ICRA). IEEE, 2014.
\item Cruciani, Silvia, and Christian Smith. "In-hand manipulation using
three-stages open loop pivoting." 2017 IEEE/RSJ International
Conference on Intelligent Robots and Systems (IROS). IEEE, 2017.
\item Yoshida, Eiichi, et al. "Regrasp planning for pivoting manipulation by
a humanoid robot." 2009 IEEE International Conference on Robotics and
Automation. IEEE, 2009.
\end{itemize}}\label{R10:9}
We appreciate the reviewer's comment. We agree with the comment and thus we added the above papers to our references and discussed them in Section II. 

\item\referee{Overall, the work is very thorough and well-written. Thank you for
making it a breeze to read, despite it being so mathematically dense.}
We really appreciate the reviewer's kind and insightful comments. 
\end{enumerate}

\begin{thebibliography}{10}
\bibitem[]{robust_pivot_icra22} Y. Shirai, D. K. Jha, A. U. Raghunathan and D. Romeres, "Robust Pivoting: Exploiting Frictional Stability Using Bilevel Optimization," 2022 International Conference on Robotics and Automation (ICRA), Philadelphia, PA, USA, 2022, pp. 992-998, doi: 10.1109/ICRA46639.2022.9811812.
\bibitem[]{motion_cone_ijrr}
Chavan-Dafle, Nikhil, Rachel Holladay, and Alberto Rodriguez. "Planar in-hand manipulation via motion cones." The International Journal of Robotics Research 39.2-3 (2020): 163-182.
\bibitem[]{Taylor_2023iros}
O. Taylor, N. Doshi and A. Rodriguez, "Object Manipulation Through Contact Configuration Regulation: Multiple and Intermittent Contacts," 2023 IEEE/RSJ International Conference on Intelligent Robots and Systems (IROS), Detroit, MI, USA, 2023, pp. 8735-8743, doi: 10.1109/IROS55552.2023.10341362.
\bibitem[]{Hou_rss2020}
Hou, Yifan, Zhenzhong Jia, and Matthew Mason. "Manipulation with Shared Grasping." Robotics: Science and Systems. 2020.
\bibitem[]{Olson_icra2021}
E. Olson, "AprilTag: A robust and flexible visual fiducial system," 2011 IEEE International Conference on Robotics and Automation, Shanghai, China, 2011, pp. 3400-3407, doi: 10.1109/ICRA.2011.5979561.
\bibitem[]{robust_textbook}
Ben-Tal, Aharon, Laurent El Ghaoui, and Arkadi Nemirovski. Robust optimization. Vol. 28. Princeton university press, 2009.


\end{thebibliography}

\end{document}

