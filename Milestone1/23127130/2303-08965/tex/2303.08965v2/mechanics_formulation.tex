\section{Mechanics of Pivoting}\label{sec:mechanics}
% What We need to say:
% 1. Stability margin Concept. Given fx, fy (u) and length, parameters, we can have safety margin so that we can discuss stability
% exact formulation of safety margin in formulation
% during manipulation, you may not be able to get the precise parameters. Then your manipulation task can fail.
% however, in practice, we don't need to have so much accurate parameters. of course, we can do it better with better estimated parameters but we have uncertainty during the manipulation since inherently friction already compensate for uncertainty as long as it satisfies static equilibrium of force moment and friction cones. For example accordingly friction magnitude can change to some extent. This would not happen to other example. This is due to the property of friction forces. our prime goal in this paper is to somehow utilize this frictional feature so that we can consider the most robust nominal trajectory

\begin{figure}
    \centering    \includegraphics[width=0.4\textwidth]{Figures/pivot22_new_notation.png} 
    \caption{A schematic showing the free-body diagram of a rigid body during pivoting manipulation \revise{when the relative angle between $F_W$ and $F_S$ is zero.} Point $P$ is the contact point with a manipulator.
    \revise{ The black circle represents the origin of each frame. 
    The object experiences four forces corresponding to two friction forces  from external contact points $A$ and $B$, one control input $f_P$ from the manipulator at point $P$, and gravity at point $C$.
    }
    }
    \label{fig:mechanics_pivoting_eq}
\end{figure}

\begin{figure}
    \centering    \includegraphics[width=0.4\textwidth]{Figures/pivot22_new_notation_frame_explanation.png} 
    \caption{\revise{A schematic showing the frame definition of a rigid body during pivoting manipulation. $F_W$, $F_S$, $F_O$, and $F_B$ are the world frame, slope frame, object frame, and frame at contact location $B$, respectively. 
    Gravity is defined in $F_W$ where the gravity is parallel to $y$-axis of $F_W$. 
    Pivoting manipulation happens with extrinsic contact $A$ and $B$ defined in $F_S$. $F_O$ is fixed with CoM of an object. $F_B$ is in parallel to $F_S$ with offset $B^S_x$ along $x$-axis of $F_S$.    We also show an example of $i_x^\Sigma$ and $i_x^\Sigma$ in \tab{tab:notion}. In this example,  $C_x^B$ and $C_y^B$ are illustrated.}
    }
    \label{fig:frame_def}
\end{figure}

% \begin{figure}
%     \centering    \includegraphics[width=0.45\textwidth]{Figures/slope_box.png} 
%     \caption{\textcolor{blue}{
%     A schematic showing the free-body diagram of a rigid body on an inclined surface during pivoting manipulation. }}
%     \label{fig:mechanics_pivoting_eq_slope}
% \end{figure}


In this section, we explain quasi-static stability of two-point pivoting in a plane. %We use this to present our proposed concept of \textit{frictional stability} which explains how friction can compensate for the inaccuracy of physical parameters during pivoting.
% 
Before explaining the details, we present our assumptions in this work. The following assumptions are used in the model for the pivoting manipulation task presented in this paper:
\begin{enumerate}
\item The object is rigid.
\item We consider \revise{quasi-static} equilibrium of the object.
\item The external contact surfaces are perfectly flat. 
\item The dimensions and pose of the object is perfectly known.
% \item The frictional parameters for the contact between the object and manipulator are perfectly known.
\item The object makes point contacts.
\end{enumerate}
% The above assumptions are common in manipulation problems. 
% Regarding the fifth assumption, we should mention that other parameters such as coefficient of friction can be also uncertain. However, uncertainty in coefficient of friction would lead to stochastic complementarity system, which is left as a future work~\cite{yuki2021chance}. 

\subsection{Mechanics of Pivoting with External Contacts}
\begin{table}[]
    \centering
        \caption{\revise{Notation of variables for analysis of frictional stability margin. In $\Sigma$ column, we indicate the frame of variables. We use the following indices for defining variables in this table: $j \in \{A, B, C, P\} $ for representing the location of frames, $i \in \{A, B, P\} $ for representing contact location, and $\Sigma \in \{W, S, O, B\} $ for representing a frame. }
        }

\begin{tabular}{|c|c|c|c|}
\hline Name & Description & Size & $\Sigma$ \\
\hline
% $j$ & indices for location. $j \in \{A, B, C, P\} $ &   &  \\
 % $i$ & indices for contact location.  $i \in \{A, B, P\} $ &   &  \\
 % % $q$ &  $q \in \{A, B\} $ &   &  \\
 % $\Sigma$ & indices for frame.  $\Sigma \in \{W, S, O, B\} $ &   &  \\
 $F_\Sigma$ & $\Sigma$ frame.   &   &  \\
  $f_{nj}^\Sigma$ & normal force at  $j$ in frame $F_\Sigma$& $\mathbb{R}^1$  & $\Sigma$ \\
    $f_{tj}^\Sigma$ & friction force at  $j$ in frame $F_\Sigma$& $\mathbb{R}^1$  & $\Sigma$ \\
 % $f_{nq}$ & normal force at contact $q$ & $\mathbb{R}^1$  & $S$ \\
 % $f_{tq}$ & friction force at contact $q$ &  
 % $\mathbb{R}^1$  & $S$ \\
 %  $f_{np}$ & normal force at contact $P$ & $\mathbb{R}^1$  & $O$ \\
 % $f_{tp}$ & friction force at contact $P$ & $\mathbb{R}^1$  & $O$ \\
   $f_{xj}^\Sigma$ & force at $j$ along $x$-axis in frame $F_\Sigma$  & $\mathbb{R}^1$  & $\Sigma$ \\
   $f_{yj}^\Sigma$ & force  at $j$ along $y$-axis in frame $F_\Sigma$  & $\mathbb{R}^1$  & $\Sigma$ \\
 $m$ & mass & $\mathbb{R}^1$  &  \\
 $g$ & gravity acceleration & $\mathbb{R}^1$  &  $W$ \\
  $l$ & length of an object & $\mathbb{R}^1$  &   \\
    $w$ & width of an object & $\mathbb{R}^1$  &   \\
  $\mu_i$ & coefficient of friction at $i$ & $\mathbb{R}^1$  & \\
  $i_x^\Sigma$ & contact location at $i$ along $x$-axis in frame $F_\Sigma$ & $\mathbb{R}^1$  & $\Sigma$ \\
    $i_y^\Sigma$ & contact location at $i$ along $y$-axis in frame $F_\Sigma$ & $\mathbb{R}^1$  & $\Sigma$ \\
      $\dot{i}_x^\Sigma$ & slipping velocity at $i$ along $x$-axis in frame $F_\Sigma$ & $\mathbb{R}^1$  & $\Sigma$ \\
      $\dot{i}_y^\Sigma$ & slipping velocity at $i$ along $y$-axis in frame $F_\Sigma$ & $\mathbb{R}^1$  & $\Sigma$ \\
     $\theta$ & angle of an object & $\mathbb{R}^1$  & $S$ \\
          $\phi$ & relative angle of frame from $\{F_W\}$ to $\{F_S\}$ & $\mathbb{R}^1$  & $W$ \\
          % $p_y^O$ & finger location in frame $O$ & $\mathbb{R}^1$  & $O$ \\
          %           $\dot{p}_y^O$ & finger slipping velocity in frame $O$ & $\mathbb{R}^1$  & $O$ \\
\hline
\end{tabular}
    \label{tab:notion}
\end{table}

%Before describing frictional stability, we describe our problem setting. 
We consider pivoting where the object maintains slipping contact with two external surfaces (see \fig{fig:mechanics_pivoting_eq}). A free body diagram showing the \revise{quasi-static} equilibrium of the object is shown in \fig{fig:mechanics_pivoting_eq}. 
\revise{The definitions of frames and  variables are summarized in 
\fig{fig:frame_def} and \tab{tab:notion}, respectively.}
% 
% The object experiences four forces corresponding to two friction forces $f_A, f_B$ from external contact points $A$ and $B$, one control input $f_P$ from manipulator at point $P$, and gravity, $mg$ at point $C$ where $m$ is mass of a body.
% We denote $f_{ni}, f_{ti}$ as a normal force and friction force at point $\forall i, i=\{A, B\}$, respectively, defined in ${\{F_W\}}$. $f_{nP}, f_{tP}$ are normal and friction force at point $P$ defined in ${\{F_B\}}$. Note that we define the $[f_x, f_y]^\top = \mathbf{R} [f_{nP}, f_{tP}]^\top$ where $\mathbf{R}$ is a rotation matrix from ${\{F_B\}}$ to ${\{F_W\}}$. We denote $x, y$ position at point in ${\{F_W\}}$ $\forall i, i=\{A, B, C, P\}$ as $i_x, i_y$, respectively. We denote $y$ position of point $P$ in ${\{F_B\}}$ as $p_y \in [-\frac{w}{2}, \frac{w}{2}]$.
% % 
% We define the angle of body with respect to $x$-axis as $\theta$. The coefficient of friction at point $\forall i, i=\{A, B, P\}$ are $\mu_A, \mu_B, \mu_P$, respectively. 
In the later sections, we present trajectory optimization formulation where we consider
\revise{
decision variables at time step $k$ (e.g., $f_{k, ni}$). In this section, we remove $k$ to represent variables for simplicity.}
% In this section, we remove $k$ to represent variables for simplicity. }
% $f_{ni}, f_{ti}$, location variables $i_{x}, i_{y}$ $\forall i, i=\{A, B, C, P\}$, $\theta$, and $p_y$ at each time-step $k$ denoted as $f_{k, ni}, f_{k, ti}, i_{k, x}, i_{k, y}, \theta_k, p_{y, k}$. 

% By setting $B_x = B_y = 0$, the static equilibrium of force in $x$ and $y$ directions the static equilibrium of and moment along point $B$ can be given by:
The \revise{quasi-static} equilibrium conditions for the object \revise{in $F_B$ when the relative angle between  $F_W$ and $F_S$ is zero (see \fig{fig:mechanics_pivoting_eq})} can be represented by the following equations.
% Note that we consider the moment at point $B$ by setting $B_x = B_y = 0$:
% (note we consider the moment at point $B$ by setting $B_x = B_y = 0$):
\begin{subequations}
\begin{flalign}
 f_{nA}^{\revise{B}} + f_{tB}^{\revise{B}} + f_{xP}^{\revise{B}}   =0,\label{forceeq1}\\
f_{tA}^{\revise{B}} + f_{nB}^{\revise{B}} + mg + f_{yP}^{\revise{B}}   = 0,  \label{forceeq2}\\
A_x^{\revise{B}} f_{tA}^{\revise{B}} - A_y^{\revise{B}}f_{nA}^{\revise{B}} + C_x^{\revise{B}}mg + P_x^{\revise{B}}f_{yP}^{\revise{B}} - P_y^{\revise{B}}f_{xP}^{\revise{B}} = 0 \label{moment_eq1}
% -\frac{l_\text{com}}{2}c_{\theta - \gamma}f_{tA} + \frac{l_\text{com}}{2}s_{\theta - \gamma}f_{nA} -\frac{l_\text{com}}{2}c_{\theta + \gamma}f_{nb} +\frac{l_\text{com}}{2}s_{\theta + \gamma}f_{tb}
% (A_x-B_x)f_{tA} - (A_y-B_y)f_{nA} + (C_x-B_x)mg + (P_x-B_x)f_{y} - (P_y - B_y) f_x = 0
\end{flalign}
\label{force_eq}
\end{subequations}
\revise{Note that because we define $F_B$ as parallel to $F_S$, all force variables in $F_B$ and $F_S$ are the same. }
We consider Coulomb friction law which results in friction cone constraints as follows:
\begin{equation}
 |f_{tA}^{\revise{B}}|  \leq \mu_A f_{nA}^{\revise{B}}, |f_{tB}^{\revise{B}}|  \leq \mu_B f_{nB}^{\revise{B}}, \quad f_{nA}^{\revise{B}}, f_{nB}^{\revise{B}} \geq 0,
% f_{tA} + f_{nB} + mg + f_{yP}   = 0,  \label{forceeq2}\\
% A_xf_{tA} - A_yf_{nA} + C_xmg + P_xf_{y} - P_y f_x = 0
% -\frac{l_\text{com}}{2}c_{\theta - \gamma}f_{tA} + \frac{l_\text{com}}{2}s_{\theta - \gamma}f_{nA} -\frac{l_\text{com}}{2}c_{\theta + \gamma}f_{nb} +\frac{l_\text{com}}{2}s_{\theta + \gamma}f_{tb}
% (A_x-B_x)f_{tA} - (A_y-B_y)f_{nA} + (C_x-B_x)mg + (P_x-B_x)f_{y} - (P_y - B_y) f_x = 0
\label{general_FC}
\end{equation}
To describe sticking-slipping complementarity constraints, we have the following complementarity constraints at point $A, B$:
\begin{subequations}
\begin{flalign}
 0 \leq  \revise{\dot{A}_{y+}^B} \perp \mu_A f_{nA}^{\revise{B}}-f_{tA}^{\revise{B}} \geq 0,  \\
 0 \leq   \revise{\dot{A}_{y-}^B} \perp \mu_A  f_{nA}^{\revise{B}}+f_{tA}^{\revise{B}} \geq 0, \\
 0 \leq  \revise{\dot{B}_{x+}^B} \perp \mu_B f_{nB}^{\revise{B}}-f_{tB}^{\revise{B}} \geq 0,  \\
 0 \leq   \revise{\dot{B}_{x-}^B} \perp \mu_B  f_{nB}^{\revise{B}}+f_{tB}^{\revise{B}} \geq 0
 \end{flalign}
 \label{slippingAB}
\end{subequations}
where the slipping velocities  follows \revise{ $\dot{A}_y^B=\dot{A}_{y+}^B-\dot{A}_{y-}^B, \dot{B}_x^B=\dot{B}_{x+}^B-\dot{B}_{x-}^B$}.
$\dot{A}_{y+}^B, \dot{A}_{y-}^B$ represent the slipping velocity \revise{at $A$} along positive and negative directions for \revise{$y$-axis in $F_B$}, respectively.
\revise{$\dot{B}_{x+}^B, \dot{B}_{x-}^B$ represent the slipping velocity at $B$ along positive and negative directions for $x$-axis in $F_B$}, respectively.
The notation $0 \leq a \perp b \geq 0$ means the complementarity constraints $a \geq 0, b \geq 0, a b=0$.
Since we consider slipping contact during pivoting, we have "equality" constraints in friction cone constraints at points $A, B$:
\begin{equation}
 f_{tA}^{\revise{B}}  =\mu_A f_{nA}^{\revise{B}}, f_{tB}^{\revise{B}}  =-\mu_B f_{nB}^{\revise{B}}
% f_{tA} + f_{nB} + mg + f_{yP}   = 0,  \label{forceeq2}\\
% A_xf_{tA} - A_yf_{nA} + C_xmg + P_xf_{y} - P_y f_x = 0
% -\frac{l_\text{com}}{2}c_{\theta - \gamma}f_{tA} + \frac{l_\text{com}}{2}s_{\theta - \gamma}f_{nA} -\frac{l_\text{com}}{2}c_{\theta + \gamma}f_{nb} +\frac{l_\text{com}}{2}s_{\theta + \gamma}f_{tb}
% (A_x-B_x)f_{tA} - (A_y-B_y)f_{nA} + (C_x-B_x)mg + (P_x-B_x)f_{y} - (P_y - B_y) f_x = 0
\label{slipping_friction_cone}
\end{equation}
To realize stable pivoting, actively controlling position of point $P$ is important. Thus, we consider the following complementarity constraints that represent the relation between the slipping velocity \revise{$\dot{P}_y$}  at point $P$ in \revise{$F_O$} and friction cone constraint at point $P$:
\begin{subequations}
\begin{flalign}
 0 \leq   \dot{P}_{y+}^{\revise{O}} \perp \mu_p f_{nP}^{\revise{O}}-f_{tP}^{\revise{O}} \geq 0  \\
 0 \leq   \dot{P}_{y-}^{\revise{O}} \perp \mu_p  f_{nP}^{\revise{O}}+f_{tP}^{\revise{O}} \geq 0 
 \end{flalign}
 \label{slippingP}
\end{subequations}
where \revise{$\dot{P}_y^O=\dot{P}_{y+}^O-\dot{P}_{y-}^O$}. 

