%%%%%%%%%%%%%%%%%%%% author.tex %%%%%%%%%%%%%%%%%%%%%%%%%%%%%%%%%%%
%
% sample root file for your "contribution" to a contributed volume
%
% Use this file as a template for your own input.
%
%%%%%%%%%%%%%%%% Springer %%%%%%%%%%%%%%%%%%%%%%%%%%%%%%%%%%
\pdfoutput=1

% RECOMMENDED %%%%%%%%%%%%%%%%%%%%%%%%%%%%%%%%%%%%%%%%%%%%%%%%%%%
\documentclass[graybox]{svmult}

% choose options for [] as required from the list
% in the Reference Guide

\usepackage{type1cm}        % activate if the above 3 fonts are
                            % not available on your system
%
\usepackage{makeidx}         % allows index generation
\usepackage{graphicx}        % standard LaTeX graphics tool
                             % when including figure files
\usepackage{multicol}        % used for the two-column index
\usepackage[bottom]{footmisc}% places footnotes at page bottom


\usepackage{newtxtext}       % 
\usepackage{newtxmath}       % selects Times Roman as basic font
\usepackage{natbib}
\usepackage{bm,color}

\graphicspath{{./fig/}{./png/}}

\input macros
\def\red{\textcolor{red}}
\def\blue{\textcolor{blue}}

% see the list of further useful packages
% in the Reference Guide

\makeindex             % used for the subject index
                       % please use the style svind.ist with
                       % your makeindex program

%%%%%%%%%%%%%%%%%%%%%%%%%%%%%%%%%%%%%%%%%%%%%%%%%%%%%%%%%%%%%%%%%%%%%%%%%%%%%%%%%%%%%%%%%

\begin{document}

\title*{Turbulent processes and mean-field dynamo}
\author{Axel Brandenburg, Detlef Elstner, Youhei Masada, and Valery Pipin}
\authorrunning{A. Brandenburg et al.}
\institute{
Axel Brandenburg \at
Nordita, KTH Royal Institute of Technology and Stockholm University, Hannes Alfv\'ens v\"ag 12, 10691 Stockholm, Sweden;
The Oskar Klein Centre, Department of Astronomy, Stockholm University, AlbaNova, 10691 Stockholm, Sweden;
School of Natural Sciences and Medicine, Ilia State University, 0194 Tbilisi, Georgia;
McWilliams Center for Cosmology and Department of Physics, Carnegie Mellon University, Pittsburgh, Pennsylvania 15213, USA,
\email{brandenb@nordita.org}
\and Detlef Elstner
\at Leibniz-Institut f\"ur Astrophysik Potsdam (AIP), An der Sternwarte 16, 14482 Potsdam, Germany,
\email{delstner@aip.de}
\and Youhei Masada
\at Department of Applied Physics, Faculty of Science, Fukuoka University, Fukuoka 814-0180, Japan
\email{ymasada@fukuoka-u.ac.jp}
\and Valery Pipin
\at Institute of Solar-Terrestrial Physics, Russian Academy of Sciences, Irkutsk, 664033, Russia,
\email{pip@iszf.irk.ru}
\hfill\today
}

\maketitle

\abstract{
Mean-field dynamo theory has important applications in solar physics
and galactic magnetism.
We discuss some of the many turbulence effects relevant to the generation
of large-scale magnetic fields in the solar convection zone.
The mean-field description is then used to illustrate the physics of the
$\alpha$ effect, turbulent pumping, turbulent magnetic diffusivity,
and other effects on a modern solar dynamo model.
We also discuss how turbulence transport coefficients are derived from
local simulations of convection and then used in mean-field models.
}

\section{Introduction}

The problem of solar and stellar dynamos is still an open one.
In spite of tremendous progress over recent decades, we still do not
understand with any degree of certainty the reason behind the equatorward
migration of solar activity belts, the dependence of cycle frequency on
rotation frequency, or the level of magnetic activity.
All models of solar and stellar magnetism rely on some assumptions.
Even the most realistic simulations suffer from finite resolution and
the compromises in the physics that are made.
The crucial question is then, when and where we are allowed to make
compromises and when not.
Among those approximations is the second-order correlation approximation
(SOCA), also known as the first-order smoothing approximation.
These are nowadays either replaced by other approximations or by numerical
techniques such as the test-field method, as will be explained later in
this review.

The Sun's magnetic field exhibits a clear mean field with spatio-temporal
order: antisymmetry of radial and toroidal fields about the equator and
the 11-yr cycle.
This mean field can well be described by an azimuthal average.
The radial component of such an azimuthally averaged mean field has a
typical strength of $\pm10\G$.
This is not much compared with the peak strength of $\pm2\kG$ in sunspots,
but much of this is ``lost'' in the process of averaging.
Of course, whatever is lost corresponds to fluctuations, which actually
play crucial parts and correlations between different fluctuations lead
to various mean-field effects.

Mathematically, once an averaging procedure has been defined, we have
the mean field $\meanBB$, indicated by an overbar.
Then, the difference between the actual and the mean field, $\BB$ and
$\meanBB$, gives the fluctuating field as $\bb\equiv\BB-\meanBB$.
The same procedure also applies to all other quantities.
This formal distinction between mean and fluctuating fields, which are
sometimes also called large-scale and small-scale fields, is important
in discussions with observers.
Coronal mass ejections, for example, are superficially reported as being
part of a large-scale field, but this may not be true anymore when we
think of averaging over the full solar circumference.
Thus, paradoxically, even if something is large by some standards,
it may not qualify as large-scale under this formal definition of an
azimuthal averaging.

Azimuthal averaging is not always a good recipe.
Some stars have nonaxisymmetric magnetic fields, and even the Sun
is believed to have what is known as active longitudes -- a weak
nonaxisymmetric magnetic field on top of a predominantly axisymmetric one.
Those nonaxisymmetric fields might best be described through low-order
Fourier mode filtering.
This is probably completely fine, but slightly problematic at the formal
level, because then the average of the product of mean and fluctuating
fields is no longer vanishing, as it would be in the case of an azimuthal
average.
This mathematical property is one of several rules that are called the
Reynolds rules.
However, as alluded to above, the violation of this particular Reynolds
rule this is probably just a technicality that makes mean-field
predictions less accurate.
We refer here to the work of \cite{ZBC18} for a detailed investigation.
There are a number of other limitations in mean-field theories that will
be discussed below.

The purpose of defining mean fields is twofold.
On the one hand, they allow us to quantify large-scale magnetic, velocity,
and other fields that are observed or that are present in a simulation.
On the other hand, they allow us to develop predictive theories for
these averages.
In these theories, mean fields can sometimes emerge because of
instabilities and/or because of suitable boundary conditions.
This is possible because of certain mean-field effects, by which one
usually means the relations between correlations of fluctuations and
various mean fields.
Discussing those effects is an important purpose of this review.
The ultimate goal of mean-field dynamo theory is to understand and model
the Sun and other stars.
We therefore also discuss in this review the status of such attempts.
For a {\em basic} introduction to mean-field theory, which is not the
subject of this review, we refer to standard textbooks \citep{Mof78,
KR80,ZRS83} and other reviews \citep{BS05, KZ08, MT09, Charbonneau10,
Charbonneau14}.

\section{Mean-field theory and avoiding some of its limitations}

We can never expect a mean-field theory to produce an accurate
representation of reality.
One reason is the fact that the underlying turbulence has stochastic
aspects, so each realization with slightly different initial conditions
would result in a somewhat different outcome.
However, there could be other reasons for discrepancies that we discuss
next.
Some of those discrepancies can nowadays be avoided.

{\bf Mean-field electrodynamics.~}
In mean-field theory, one derives evolution equations for the averaged
fields, namely the mean magnetic field $\meanBB$, the mean velocity
$\meanUU$, and the mean thermodynamic variables such as mean specific
entropy $\meanS$ and the mean density $\meanrho$.
Often, one neglects the evolution of $\meanUU$, $\meanS$, and $\meanrho$,
which is then already an important limitation.

If one focuses on the evolution of the mean magnetic field only,
one often talks about the mean-fields electrodynamics or quasi-kinematic
mean-field theory, which can still be nonlinear if the various mean-field
transport coefficients depend on the mean fields.
If they are unaffected, one talks about kinematic mean-field theory, which
is linear.
Of course, once there is a dynamo, we have an exponentially growing
solution, so the magnetic field would grow without limit, i.e., it would
not saturate within kinematic mean-field theory.
Obviously, a correct mean-field theory must be nonlinear, but even within
the realm of linear theory, there are important lessons to be learnt.
Below, we discuss the aspects of nonlocality, which were often omitted
out of ignorance, but nowadays we know that this is often not possible
and this restriction can easily be relaxed.

{\bf Nonlocality.~}
The mean magnetic field is governed by the mean induction equation,
which is sometimes also referred to as the mean-field dynamo equation.
The most important term here is the electromotive force,
\begin{equation}
\meanEMF=\overline{\uu\times\bb},
\end{equation}
i.e., the averaged cross product of velocity and magnetic fluctuations.
In mean-field electrodynamics, it is often expressed as
\begin{equation}
\meanemf_i=\meanemf_{0i}+\alpha_{ij}\meanB_j+\eta_{ijk}\partial\meanB_j/\partial x_k+...,
\label{mult}
\end{equation}
where the ellipsis denotes higher derivative terms, of which there should
be infinitely many, and there should also be time derivatives.
The term $\meanemf_{0i}$ is a contribution that can exist already in
the absence of a mean field; see \cite{BR13} for details and numerical
experiments.
Including only a finite number of derivatives in \Eq{mult} and ignoring
time derivatives is another important approximation.
In fact, it is usually easier to express $\meanEMF$ as a convolution
between an integral kernel and the mean field.
Furthermore, it is instructive to split the integral kernel into two
pieces and write
\begin{equation}
\meanemf_i=\meanemf_{0i}+\hat{\alpha}_{ij}\ast\meanB_j
+\hat{\eta}_{ijk}\ast\partial\meanB_j/\partial x_k,
\label{conv}
\end{equation}
where the asterisks mean a convolution in space and time, and the hats
denote integration kernels.
In principle, the spatial derivative can be absorbed as being part of
the integral kernel, but separating the kernel into $\hat{\alpha}_{ij}$
and $\hat{\eta}_{ijk}$ has conceptual advantages, because they preserve
the similarity to \Eq{mult}.
Note also that, unlike \Eq{mult}, where we allowed for arbitrarily many
derivatives, here, we have no other terms, because all even derivatives
are already absorbed in $\hat{\alpha}_{ij}$ and all odd derivatives are
absorbed in $\hat{\eta}_{ijk}$.
Time derivatives can also absorbed in both of them if the convolution
with the kernels is also over time.

For the benefit of better interpretation, both $\alpha_{ij}$ and
$\eta_{ijk}$ (and analogously also for $\hat{\alpha}_{ij}$ and
$\hat{\eta}_{ijk}$) can be broken down into further pieces.
The $\alpha_{ij}$ tensor can be split into a symmetric and an
antisymmetric tensor.
The latter is characterized by a vector,
$\gamma_i=-\half\epsilon_{ijk}\alpha_{jk}$,
which corresponds to a pumping velocity.
Having in mind that the magnetic gradient tensor can also be split
into symmetric and antisymmetric parts, where the latter is the
mean current density, $\meanJJ$, with $\meanJ_i=-\half\epsilon_{ijk}
\partial\meanB_j/\partial x_k$, we can separate the rank-3 tensor,
$\eta_{ijk}$, into a rank-2 tensor operating only on $\meanJJ$ and the
rest operating on the symmetric part of $\partial\meanB_j/\partial x_k$.

The convolution can only be replaced by a multiplication, as in
\Eq{mult}, if the mean
field is constant in time (which is normally never the case!) and if it
varies at most linearly in space (which is normally also not the case).
We return to this point further below.

{\bf Avoiding SOCA.~}
Another approximation that is often discussed has to do with the correct
calculation of the coefficients or the corresponding $\alpha_{ij}$
and $\eta_{ijk}$ kernels.
It results from the fact that the differential equations for these
expressions are nonlinear and therefore hard to solve analytically.
But this is not really a problem when one can calculate numerical
solutions of the underlying differential equations.
This is done in what is called the test-field method \citep{schrinner+05,
schrinner+07}, which will also be explained below.

In summary, the limitations discussed so far are in principle all
avoidable:
(i) Evolution equations for $\meanUU$, $\meanS$, and $\meanrho$ can (and
have been) included, in addition to that for $\meanBB$, but in practice,
even this is still an approximation in the sense that the full set of
equations is not (or only approximately) known.
(ii) The electromotive force can (and has been) solved as a convolution.
In practice, this is cumbersome, but it is possible to approximate this
by a differential equation for $\meanEMF$ of the form
\begin{equation}
\left(1+\tau\frac{\partial}{\partial t}-\ell^2\nabla^2\right)
\meanemf_i=\alpha_{ij}\meanB_j+\eta_{ijk}\partial\meanB_j/\partial x_k.
\label{evol}
\end{equation}
This has been considered in several papers \citep{RB12,Rhei+14,BC18}.
(iii) Numerical solutions can be employed to have precise expressions
for $\alpha_{ij}$ and $\eta_{ijk}$; see \cite{War+18, War+21} for doing
this for solar simulations using the test-field method.
It often turns out that analytical closure techniques are very useful
as a first orientation and they are often also accurate enough for a
qualitatively useful model.
In special cases, when an accurate solution is required, the answer may
well be obtained numerically using the test-field method.
The problem is then only that numerical solutions themselves are limited
in just the same way as those for a full numerical solution in the solar
and stellar dynamo problems.

\FFig{ppk_both} shows results for $\tilde\alpha(k)$ and $\tilde\eta_{\rm t}(k)$
with $\nu / \eta = 1$.
Both $\tilde\alpha$ and $\tilde\eta_{\rm t}$ decrease monotonously with increasing $|k|$.
The functions $\tilde\alpha (k)$ and $\tilde\eta_{\rm t} (k)$
are well represented by Lorentzian fits of the form
\EQ
\tilde\alpha(k)\approx{\alpha_0\over1+(k/k_{\rm f})^2} \, ,\quad
\tilde\eta_{\rm t}(k)\approx{\eta_{\rm t0}\over1+(k/2k_{\rm f})^2} \, .
\label{KernelsTurb}
\EN

\begin{figure}[t]
\centering\includegraphics[width=\columnwidth]{ppk_both}\caption{
Top: Dependences of the normalized $\tilde\alpha$ and $\tilde\eta_{\rm t}$
on the normalized wavenumber $k/k_{\rm f}$
for isotropic turbulence forced at wavenumbers $k_{\rm f}/k_1=5$ with $\Rm=10$ (squares)
and $k_{\rm f}/k_1=10$ with $\Rm=3.5$ (triangles), all with $\nu/\eta=1$,
using data from \cite{BRS08}.
The solid lines give the Lorentzian fits \eq{KernelsTurb}.
Bottom: Normalized integral kernels $\hat\alpha$ and $\hat\eta_{\rm t}$ versus
$k_{\rm f}\zeta$ for isotropic turbulence forced at wavenumbers $k_{\rm f}/k_1=5$ with $\Rm=10$ (squares)
and $k_{\rm f}/k_1=10$ with $\Rm=3.5$ (triangles), all with $\nu/\eta=1$.
The solid lines are defined by \eq{KernelsTurb2}.
Adapted from \cite{BRS08}.
}\label{ppk_both}\end{figure}

The kernels $\hat\alpha (\zeta)$ and $\hat\eta_{\rm t} (\zeta)$ in
the lower part of \Fig{ppk_both} are obtained numerically.
Also shown are the Fourier transforms of the Lorentzian fits,
\EQ
\hat\alpha(\zeta)\approx\half\alpha_0 k_{\rm f} \exp(-k_{\rm f}|\zeta|) \, ,\quad
\hat\eta_{\rm t}(\zeta)\approx\eta_{\rm t0} k_{\rm f} \exp(-2k_{\rm f}|\zeta|) \, .
\label{KernelsTurb2}
\EN
We see that the profile of $\hat\eta_{\rm t}$ is half as wide as that
of $\hat\alpha$.

{\bf The use of mean-field theory.~}
If mean-field theory cannot reliably be applied to a regime outside
that of the direct numerical simulations (DNS), one must ask what is
then the use of mean-field theory.
The answer lies in the fact that mean-field theory provides us with a
diagnostic ``tool'' for approaching the problem.
Particular features of a solution can usually be attributed to particular
terms in the mean-field equation.
This would then allow as a more informed answer by saying that the main
dynamo mechanism is, for example, of $\alpha\Omega$ type, or of the type
of a shear flow dynamo, for example.
{\em Thus, mean-field theory may be regarded as a convenient tool for
understanding what is going on rather than predicting what might be
going on.}

\section{The catastrophic quenching problem}

Since the 1990s, a problem emerged in that 
numerical dynamo solutions were found to depend on the value of
the microphysical magnetic diffusivity.
Typically, the strengths of the mean-fields then decreases with
increasing magnetic Reynolds number.
This is unusual and does not have any correspondence with ordinary
hydrodynamics where the large-scale dynamics is usually already
captured at moderate fluid Reynolds numbers.
In its original form, the catastrophic quenching problem refers to the
finding that the volume-averaged electromotive force scales with the
microphysical magnetic diffusivity, and thus goes to zero when $\eta\to0$.
To some extent, this is a problem related to the use of periodic boundary
conditions.
However, even for astrophysically more realistic boundary conditions,
numerical simulations reveal that there is still a problem.

\subsection{Mean fields in periodic domains}

Under astrophysical conditions of interest, $\eta$ is so small that
the volume-average electromotive force would be negligibly small.
If this result was actually astrophysically relevant, it would be a
``catastrophe,'' i.e., it would not be possible to understand astrophysical
magnetic fields as mean-field dynamos.
The solution to this particular problem turned out to be that relating
the volume-averaged electromotive force to the volume-averaged mean
magnetic field is only of limited relevance to the problem of $\alpha$
effect dynamos.
Any dynamo would produce a non-uniform field.
Especially in a periodic domain, the mean magnetic flux through any of
the faces of the periodic domain is constant in time, so if it was zero
to begin with, it would always remain zero.
A dynamo problem can therefore not be formulated in that case.

A proper dynamo problem should always allow for the possibility of
the magnetic field to decay to zero if there is sufficient magnetic
diffusivity.
Simple examples of nontrivial mean fields in a periodic domain are
Beltrami fields of the form
\begin{equation}
\meanBB(x)\propto\begin{pmatrix}0\\ \sin kx\\ \cos kx\end{pmatrix},\quad
\meanBB(y)\propto\begin{pmatrix}\cos ky\\ 0\\ \sin ky\end{pmatrix},\quad\mbox{or}\quad
\meanBB(z)\propto\begin{pmatrix}\sin kz\\ \cos kz\\ 0\end{pmatrix},
\label{Belt}
\end{equation}
which can be solutions of the simple $\alpha^2$ dynamo problem,
$\partial\meanBB/\partial t=\alpha\nab\times\meanBB+\etaT\nabla^2\meanBB$.
Nevertheless, there is still a problem of catastrophic nature because
it turned out that the time required to reach the final solution scales
inversely with $\eta$.
This is demonstrated in \Fig{psat}, where we show the evolution of one of
the three planar averages.
In the beginning, all three mean fields grow in a similar fashion,
but at some point, only one of the three reaches a significant amplitude.
Note, however, that the ultimate saturation takes a resistive time,
$\tau_{\rm res}=1/(2\eta k_1^2)$.

\subsection{Quenching phenomenology}

To understand the reason for the catastrophically slow saturation, it
suffices to consider the magnetic helicity equation,
\begin{equation}
\frac{\dd}{\dd t}\bra{\AAA\cdot\BB}=-2\eta\mu_0\bra{\JJ\cdot\BB}
-\nab\cdot\left(\EE\times\AAA+\Phi\BB\right),
\end{equation}
which follows directly from the uncurled induction equation,
\begin{equation}
\frac{\partial\AAA}{\partial t}=\UU\times\BB
-\eta\mu_0\JJ-\nab\Phi.
\end{equation}
For periodic domains, we just have
\begin{equation}
\frac{\dd}{\dd t}\bra{\AAA\cdot\BB}=-2\eta\mu_0\bra{\JJ\cdot\BB}.
\label{dABdt}
\end{equation}
This equation is gauge-independent, because the gauge
transformation $\AAA\to\AAA'+\nab\Lambda$ yields
$\bra{\AAA\cdot\BB}=\bra{\AAA'\cdot\BB}$, with
$\bra{\BB\cdot\nab\Lambda}=\bra{\nab\cdot(\Lambda\BB)}
-\bra{\Lambda\nab\cdot\BB}=0$, because $\nab\cdot\BB=0$ and
the domain is periodic, so the average of a divergence vanishes.

\begin{figure}\begin{center}
\includegraphics[width=\columnwidth]{psat}
\end{center}\caption[]{
Evolution of the normalized $\bra{\meanBB^2}$
and that of $\bra{\meanBB^2}+\tau_{\rm diff}\dd\bra{\meanBB^2}/\dd t$ (dotted),
compared with its average in the interval $1.2\leq t/\tau_{\rm diff}\leq3.5$
(horizontal blue solid line), as well as averages over three subintervals
(horizontal red dashed lines).
The green dashed line corresponds to \Eq{B2vol} with
$t_{\rm sat}/\tau_{\rm diff}=0.54$.
Adapted from \cite{CB13}.
}\label{psat}\end{figure}

For fully helical large-scale and small-scale magnetic fields
of opposite magnetic helicity, \Eq{dABdt} becomes \citep{Bra01}
\EQ
{\dd\over\dd t}\bra{\meanBB^2}=2\eta k_1\kf\Beq^2-2\eta k_1^2\bra{\meanBB^2},
\label{dB2voldt}
\EN
with the solution
\EQ
\bra{\meanBB^2}=\Beq^2{\kf\over k_1}
\left[1-e^{-2\eta k_1^2(t-t_{\rm sat})}\right].
\label{B2vol}
\EN
This agrees with the slow saturation behavior seen first in the simulations
of \cite{Bra01}; see \Fig{psat}.
Here $t_{\rm sat}$ is the time when the slow saturation phase commences;
see the crossing of the green dashed line with the abscissa.
Interestingly, instead of waiting until full saturation is accomplished,
one can obtain the saturation value already much earlier simply by
differentiating the simulation data to compute \citep{CB13}
\EQ
B_{\rm sat}^2\approx
\bra{\meanBB^2}+\tau_{\rm res}{\dd\over\dd t}\bra{\meanBB^2}.
\EN
Since $\tau_{\rm res}$ involves the microphysical magnetic diffusivity,
the quenching is still in that sense catastrophic.

\subsection{The $\alpha$ quenching formula}

A more complete description is in terms of kinetic and magnetic $\alpha$
effects, i.e.,
\begin{equation}
\alpha=\alpK+\alpM\sim\approx\frac{\tau}{3}\left(\overline{\oo\cdot\uu}
-\overline{\jj\cdot\bb}/\meanrho\right),
\label{alphaFOSA}
\end{equation}
and observing the fact that the magnetic helicity evolution of averages
and fluctuations is given by
\begin{equation}
\frac{\dd}{\dd t}\bra{\meanAA\cdot\meanBB}=
+2\bra{\meanEMF\cdot\meanBB}-2\eta\mu_0\bra{\meanJJ\cdot\meanBB},
\label{dAmBmdt}
\end{equation}
\begin{equation}
\frac{\dd}{\dd t}\bra{\aaaa\cdot\bb}=
-2\bra{\meanEMF\cdot\meanBB}-2\eta\mu_0\bra{\jj\cdot\bb}\label{smhev}.
\end{equation}
\EEq{dAmBmdt} allows for the possibility that magnetic helicity
can be produced by the mean electromotive force, because, in general,
$\meanEMF\cdot\meanBB\equiv\overline{\uu\times\BB}\cdot\meanBB\neq0$.
(By contrast, of course, $(\uu\times\meanBB)\cdot\meanBB=0$.)
In particular, if $\meanEMF=\alpha\meanBB-\etat\mu_0\meanJJ$, then,
$\meanEMF\cdot\meanBB=\alpha\meanBB^2-\etat\mu_0\meanJJ\cdot\meanBB$,
which produces positive (negative) magnetic helicity of the mean field
when $\alpha>0$ ($\alpha<0$)

\EEq{smhev} is constructed such that the sum of \Eqs{dAmBmdt}{smhev}
yields \Eq{dABdt}.
Given that $\bra{\aaaa\cdot\bb}$ is related to $\bra{\jj\cdot\bb}$,
which, in turn, is related to a magnetic contribution to the $\alpha$
effect \citep{pouquet+76}, \Eq{smhev} can be rewritten as an evolution equation
for the total $\alpha$ \citep{Bra08AN},
\begin{equation}
\frac{\dd\alpM}{\dd t}=-2\etatz\kf^2 \left(
\frac{\alpha\meanBB^2-\etat\mu_0\meanJJ\cdot\meanBB}{\Beq^2}
+\frac{\alpM}{\Rm}\right),
\end{equation}
which can also be expressed in the form
\begin{equation}
\alpha(\meanBB)=\frac{\alpha_0+\Rm\times\mbox{``extra terms''}}
{1+\Rm\meanBB^2/\Beq^2}
\label{quenching}
\end{equation}
where
\begin{equation}
\mbox{``extra terms''}=
\etat\frac{\mu_0\meanJJ\cdot\meanBB}{\Beq^2}
-\frac{\nab\cdot\meanFFF}{2\kf^2\Beq^2}
-\frac{\partial\alpha/\partial t}{2\kf^2\Beq^2}.
\end{equation}
Note that the last term is here a time derivative.
\EEq{quenching} resembles the catastrophic quenching formula of \cite{VC92},
but it also shows that it need to be extended in several important
ways: when the mean field is no longer defined as a volume average,
extra terms emerge that are of  the same order as those in the denominator.
They can therefore potentially offset the catastrophic quenching.
In practice, this is only partially true, because there are also other
terms, for example the aforementioned time derivative term.
It is responsible for the fact that a strong field state is only reached
after a resistively long time.

\subsection{Analogy with the chiral magnetic effect}

The $\alpha$ effect in mean-field dynamo theory is an effect that
emerges after averaging over the scale of several turbulent eddies.
We know already that turbulent diffusion is somewhat analogous to
microphysical diffusion, which also emerges after averaging,
but here after averaging over atomic scales.
Interestingly, even for the $\alpha$ effect there can be an effect
on atomic and subatomic scales, because fermions, such as electrons,
are chiral.
The spin of an electron emerging from the decay of a neutron is
anti-aligned with its momentum vector, so their dot product is a
negative pseudo-scalar, called the chirality.
Positrons have positive chirality.
In the presence of an ambient magnetic field, the spins align,
but electrons and positrons move in opposite directions, causing
therefore an electric current.
This constitutes a microscopic $\alpha$ effect \citep{Roga+17,Bran+17},
\begin{equation}
\alpha_{\rm micro}\equiv\mu_5\eta=24\alpha_{\rm fine}
(n_{\rm L}-n_{\rm R})(\hbar c/\kB T)^2,
\end{equation}
where $\mu_5$ is the normalized chiral chemical potential (with units
of inverse length), $\eta$ is the microscopic magnetic diffusivity,
$\alpha_{\rm fine}\approx1/137$ is the fine structure constant (quantifying
the strength of electromagnetic interaction between charged particles),
$n_{\rm L}$ and $n_{\rm R}$ are the number densities of left- and
right-handed fermions, $\hbar\approx10^{-27}\erg\s$ is the reduced Planck
constant, $c\approx3\times10^{10}\cm\s^{-1}$ is the speed of light,
$\kB\approx10^{-16}\erg\K^{-1}$ is the Boltzmann constant, and $T$
is the temperature.

The applications of chiral MHD are manifold and range from condensed matter
systems and heavy ion collisions to neutron stars and the early Universe;
see \cite{Khar14} for a review.
Interestingly, because this microscopic $\alpha$ effect produces
helical magnetic fields, and because the total chirality is conserved
\citep{BFR12}, this effect does not last forever, but is being quenched
in a form analogous to the catastrophic quenching formula, which takes
the form \citep{Roga+17}
\begin{equation}
\frac{\partial\mu_5}{\partial t}=-\lambda\eta
\left(\mu_5\meanBB^2-\etat\mu_0\meanJJ\cdot\meanBB\right)
-\Gamma_{\rm flip}\mu_5,
\end{equation}
where $\lambda$ is a coupling constant which, in the catastrophic quenching
formalism, is related to $2\etat\kf^2/\Beq^2$, and $\Gamma_{\rm flip}$
is a spin-flipping parameter, which is related to $2\eta\kf^2$ in the
catastrophic quenching formalism \citep[see, e.g.,][]{FB02,BB02}.

There is a vast range of recent work in this field, which goes well
beyond the scope of the present paper.
We just mention here the paper of \cite{Masada+18}, who studied
chiral magnetohydrodynamic turbulence in core-collapse supernovae.
They found that the inverse cascade related to the chiral effects impacts
the magnetohydrodynamic evolution in the supernova core toward explosion.

\subsection{Magnetic helicity fluxes and helicity reversals}

Magnetic helicity fluxes could in principle remove the catastrophic
quenching problem, but only if preferentially small-scale magnetic
helicity is being removed \citep{Kleeorin+00}.
To see this, let us first consider the problem of an $\alpha^2$
dynamo in insulating boundaries, i.e.,
\begin{equation}
\frac{\dd}{\dd t}\meanAA=\alpha\meanBB-\etaT\mu_0\meanJJ,\quad
\mbox{with $\partial_z\meanA_x=\partial_z\meanA_y=\meanA_z=0$}.
\end{equation}
The boundary condition implies that $\meanB_x=\meanB_y=0$, and is
therefore also referred to as the vertical field condition.
In this 1-D problem, however, this boundary condition is equivalent
to a proper vacuum boundary condition.

The $\alpha^2$ dynamo with this boundary condition was first considered
by \cite{GD94}, who found that the saturation field strength of such
a dynamo decreases with $\Rm$.
This was later confirmed by \cite{BD01}.
In \Fig{pphelflux_2panels_M288h_cont2}, we show the profiles
of magnetic helicity, current helicity, and the magnetic helicity fluxes
for Runs~A of \cite{Brandenburg2018AN} with $\Rm=180$.
For normalization purposes, they defined the reference values
\EQA
C_{\rm f0}=\kf\Beq^2\quad\mbox{and}\quad
F_{\rm m0}=\etatz k_1^2\int_0^{\pi/2}\meanBB^2\,\dd z.
\ENA
They emphasized that the largest contribution to the magnetic helicity
density comes from the large-scale field.
Near the surface ($z=\pi/2$), the (negative) magnetic helicity flux from
small-scale fields is only about $0.02\,F_{\rm m0}$, which explains why
they are not efficient enough to alleviate the catastrophic dependence
of the resulting mean magnetic field \citep{DSordo2013MN, Rincon21}.

\begin{figure}[t!]\begin{center}
\includegraphics[width=\columnwidth]{pphelflux_2panels_M288h_cont2}
\end{center}\caption[]{
Magnetic helicity, current helicity, and magnetic helicity fluxes
for Run~A of \cite{Brandenburg2018AN} with $\Rm=180$.
The kinetic helicity is shown in green and is found to be of similar
magnitude as the current helicity of the small-scale field.
The second panel shows $\overline{\EE\times\AAA}$ near zero.
The green line denotes $\overline{\phi\bb}$, which is seen
to fluctuate around zero.
}\label{pphelflux_2panels_M288h_cont2}\end{figure}

Subsequent simulations with an outside corona indicated that the magnetic
helicity changes sign at or near the outer surface \citep{BCC09}.
This was just a speculation and needs to be reconsidered with the
help of global models of the type considered by \cite{War+11,War+12}
and \cite{BAJ17}.
This is shown in \Fig{ppfft3_pi_times}, where we present the line-of-sight
averaged current helicity density, $\bra{\JJ\cdot\BB}$ in the plane of
the sky using a simulation of \cite{BAJ17}.
The quantity $\bra{\JJ\cdot\BB}$ is a proxy of magnetic helicity at
small scales and shows clearly the reversal of sign between the dynamo
interior and the exterior.

\begin{figure}[t!]\begin{center}
\includegraphics[width=\columnwidth]{ppfft3_pi_times}
\end{center}\caption[]{
Current helicity $\bra{\JJ\cdot\BB}$ in the plane of the observer
at four different times.
Yellow and white shades denote positive values and
blue and black shades denote negative values;
adapted from \cite{BAJ17}.
}\label{ppfft3_pi_times}\end{figure}

\subsection{Radial magnetic helicity reversal in the solar wind}

If the idea of alleviating catastrophic quenching by magnetic helicity
fluxes is to make sense, when would expect to see signs of the expelled
magnetic helicity to see in the solar wind.
The magnetic helicity spectrum can be measured in the solar wind by
determining the parity-odd contribution to the magnetic correlation
tensor, which, in Fourier space, takes the form
\begin{equation}
\bra{\tilde{B}_i(\kk)\tilde{B}^*_j(\kk)}
=\left(\delta_{ij}-\hat{k}_i\hat{k}_j\right)E(k)
-\ii\hat{k}_k\epsilon_{ijk}H(k).
\end{equation}
This would allow one to compute $H(k_z)=\Imag(\tilde{B}_x\tilde{B}_y^*)$
and $E(k_z)=\half(|\tilde{B}_x|^2+|\tilde{B}_y|^2)$, which also obeys
the realizability condition $k_z|H(k_z)|\leq E(k_z)$.

The Ulysses spacecraft was the only one to cover high heliographic
latitudes, where a non-vanishing sign of magnetic helicity can be
expected.
It turned out that $H(k)$ has, as expected from dynamo theory, different
signs in the northern and southern hemispheres.
It also has different signs at small and large wavenumbers.
This, in itself, is also expected from an $\alpha^2$ dynamo, because
the $\alpha$ effect produces no net magnetic helicity, but it separates
magnetic helicity in wavenumber space.
However, the signs are opposite to what is seen at the solar surface,
where the helicity in the north is negative at small length scales.
In the solar wind, however, it is positive in the north and at small
scales.
Of course, the meaning of small is here relative and has to be with
respect to larger scales, where a sign change in $k$ has been seen.
If one just assumed a linear expansion of all scales from the solar
surface (radius $r=700\Mm$, to the location of the Earth at $1\AU$,
we expect a corresponding expansion ratio so that a wavenumber of
$1\Mm^{-1}$ corresponds to $1/700\AU^{-1}$.
In particular, $20\Mm^{-1}$ corresponds to $2/70\AU^{-1}$,
which is close to the wavenumber where we see a sign-change in
\Fig{phelicity_plat4b_Lidingo}.
It is unexpected, however, that at the solar surface
(\Fig{phelicity_plat4b_Lidingo}b), the sign in the northern hemisphere
changes from minus to plus as $k$ increases, while in the solar wind,
it changes from plus to minus.
This apparent mismatch may not just be a measurement error, but it may
actually be a real result and would tell us that the simpleminded picture
of expelling magnetic helicity of one sign all the way to infinity may
not be accurate.

\begin{figure}[t]
\begin{center}
\includegraphics[width=\textwidth]{phelicity_plat4b_Lidingo}
\end{center}
\caption{
Magnetic energy and magnetic helicity spectra for southern latitudes
(a) at the solar surface in active region AR~11158, and
(b) in the solar wind at $\sim1\AU$ distance ($1\AU\approx149,600\Mm$).
Positive (negative) signs are shown as red open (blue filled) symbols.
Positive signs are the solar surface at intermediate and large $k$
correspond to positive values in the solar wind at small $k$.
Note that $1\G=10^{-4}\T=10^5\nT$.
}\label{phelicity_plat4b_Lidingo}
\end{figure}

Looking at the evolution equation for the small scale magnetic helicity,
we have
\begin{equation}
\nab\cdot\meanFFFFf
=\underbrace{{-2\alpha\meanBB^2+2\etat\mu_0\meanJJ\cdot\meanBB}}_{-2\meanEMF\cdot\meanBB}
-2\eta\mu_0\overline{\jj\cdot\bb}.
\end{equation}
In the dynamo interior at the northern hemisphere, $\alpha>0$,
and, assuming $\alpha\BB^2$ to dominate the EMF, we expect
$-2\meanEMF\cdot\meanBB$ to be negative.
However, a negative flux divergence of a negative quantity would
eventually make this quantity positive, which is what has been observed.

Whether or not this is really the right interpretation remains still
an open question.
It would clearly be useful to have an independent assessment of this
interpretation.

\section{Alternative large-scale dynamo effects}

Given the difficulties encountered with $\alpha$ effect dynamos, there
have been various attempts to construct large-scale dynamos that are
not based on the $\alpha$ effect.
A common misconception here is that the idea that catastrophic quenching
would not apply if just because there is no $\alpha$ effect, but it is
not true.
An $\alpM$ term can always emerge regardless of whether they existed
original an $\alpha$ effect or not.
An example is shear--current effect.
It is due to the presence of shear and boundaries that a helicity can
be introduced.
Shear of the form $\meanUU=(0,Sx,0)$ implies a finite vorticity,
$\nab\times\meanUU=(0,0,S)$ and boundaries would lead to a gradient
vector of turbulent intensity near the boundaries.
Thus, while there can be hope that catastrophic quenching may not be as
strong, this may turn out not to be the case.
An example of this was presented in \cite{BS05c}.

\subsection{R\"adler and shear--current effects}

The R\"adler effect is another large-scale dynamo effect \citep{Raedler1969}.
In the simplest representation it leads to an EMF proportional to $\OO\times\meanJJ$.
It is similar to the shear--current effect. In this case it cannot change the
magnetic energy of the mean field.
Indeed, the energy equation for the mean field is given by
\begin{equation}
\frac{\dd}{\dd t}\bra{\meanBB^2/2}=
\underbrace{\meanJJ\cdot(\OO\times\meanJJ)}_{=0}
+\underbrace{\bra{\nab\cdot[(\OO\times\meanJJ)\times\meanBB]}}_{\mbox{$=0$ under periodicity}}\label{engdelt}
\end{equation}
In the general case, the generation effects due to global
rotation and mean currents can be written as follows
(see \citealp{KR80, Kitchatinov1994,Raedler2003a,Pipin2008a}):
\begin{equation}
\meanEMF^{(\delta)}=\delta_{1}\boldsymbol{\Omega}\times\meanJJ+\delta_{2}\boldsymbol{\nabla}\left(\boldsymbol{\Omega}\cdot\meanBB\right)+\delta_{3}\frac{\boldsymbol{\Omega}\left(\boldsymbol{\Omega}\cdot\meanBB\right)}{\boldsymbol{\Omega}^{2}}\boldsymbol{\nabla}\left(\boldsymbol{\Omega}\cdot\meanBB\right),\label{deltas}
\end{equation}
where the coefficients $\delta_{1,2,3}$ depend on the spatial profiles of
the turbulent parameters such as the typical convective turnover time, 
the convective velocity $\urms$, etc.
The last two terms in this equation may lead to an $\delta^2$ dynamo
\citep{Pipin2009}.
For the solar case, the $\delta$ effect can provide an additional
non-helical source of poloidal magnetic field generation.
Interestingly, \cite{Pipin2009} found that for the solar-type dynamos,
i.e., those with equatorward propagation of the dynamo waves, the $\delta$
dynamo effect does not dominate the contributions of the $\alpha$-effect.
We will discuss the available scenario in the next section.

\subsection{Dynamos from negative turbulent magnetic diffusivity}

There are two other effects that are noteworthy, although it is not
clear that either of them can play a role in stellar convection zones.
One is the negative turbulent magnetic diffusivity and the other is
the memory effect in conjunction with a pumping effect.

When modeling a negative turbulent magnetic diffusivity dynamo,
high wavenumbers must not be destabilized at the same time.
\cite{BC20} studied classes of dynamos with a very low critical $\Rm$.
The Willis dynamo \citep{Willis12} has a critical $\Rm$ of2.01, which
is small compared to 6.3 for the Roberts flow and 17.9 for the ABC flow.
In this dynamo, one of the two horizontally averaged field components
grows exponentially, because the total magnetic diffusivity in that
direction is negative \citep{BC20}.
The other component decays and is not coupled to the former one.

\begin{figure}\begin{center}
\includegraphics[width=0.9\columnwidth]{pom0k}
\end{center}\caption[]{
Dependence of $\tilde{\eta}_{xx}$ (red) and $\tilde{\eta}_{yy}$ (blue) on $k$
for the Willis flow in the marginally exited case with $\eta=0.403$.
The dashed line denotes the fit $-0.233+0.11\,k^2$.
Adapted from \cite{BC20}.
}\label{pom0k}\end{figure}

As we see from \Fig{pom0k}, $\etat$ is negative only for $k\la1.5$.
The $k$ dependence of the turbulent magnetic diffusivity can be expanded
up to second order as
\begin{equation}
\tilde{\eta}_{yy}(k)=\tilde{\eta}_{yy}^{(0)}+\tilde{\eta}_{yy}^{(2)}k^2+
\ldots,
\end{equation}
where the tildes indicate Fourier transformed quantities.
In the proximity of $k=1$, which corresponds to the
largest scale in the computational domain of $2\pi$,
we have $\tilde{\eta}_{yy}^{(0)}\approx-0.233$ and
$\tilde{\eta}_{yy}^{(2)}\approx0.11$.
In addition, there is still the microphysical magnetic diffusivity,
which is positive ($\eta=0.403$).
To a first approximation, one can just consider the equation
for $\meanA_{yy}$, which can then be written as
\begin{equation}
{\partial\meanA_{yy}\over\partial t}=
\left[\eta+\tilde{\eta}_{yy}^{(0)}\right]{\partial^2\meanA_{yy}\over\partial z^2}
-\tilde{\eta}_{yy}^{(2)}{\partial^4\meanA_{yy}\over\partial z^4}.
\label{dAmyydt}
\end{equation}
We recall that the minus sign in front of the fourth derivative corresponds
to positive diffusion if $\tilde{\eta}_{yy}^{(2)}$ is positive, and so
does the plus sign in front of the second derivative, unless the term in
squared brackets is negative, which is the case we are considering here.

\subsection{Dynamos from pumping and memory effects}

Pumping effects alone cannot usually lead to interesting dynamo effects,
unless there is also a memory effect.
This effect means that the mean electromotive force depends not just
on the instantaneous mean magnetic field at that time, but also on the
mean magnetic field at earlier times.
It is therefore described as a convolution between a pumping kernel and
the mean magnetic field.
This can lead to dynamo action, as has been demonstrated by \cite{Rhei+14}
for the case of one of four flow fields studied by \cite{Rob72}.

The example of Roberts flow~III may be peculiar, because there is so far
no other known example of a flow where pumping produces a memory effect
that is strong enough to lead to dynamo action.
This is mostly because the computational tools for determining the memory
effect are not broadly used by the community.
Indeed, it was only with the development of the test-field method
\citep{schrinner+05, schrinner+07} that the importance of the memory
effect was noticed \citep{HBO9} and applied to pumping.

The dispersion relation for a problem with turbulent pumping
$\gamma$ and turbulent magnetic diffusion $\etat$ is given by
$\lambda=-\ii k\gamma-\etat k^2$.
Since $\Rey\lambda<0$, the solution can only decay, but it is
oscillating with the frequency $\omega=\Imag\lambda=\gamma$.
In the presence of a memory effect, $\gamma$ is replaced by
$\gamma/(1-\ii\omega\tau)$, where $\tau$ is the memory time.
Then, $\lambda\approx-\ii k\gamma\,(1-\ii\omega\tau)-\etat k^2$,
and $\Rey\lambda$ can be positive if total.
This is the case for the Roberts flow.

We return to nonlocality and memory effects further below in this article
when we discuss concrete solar models; see \cite{Pipin23}.
One of the most obvious consequences of the memory effect is a lowering
of the critical excitation conditions for the dynamo, which was already
reported by \cite{RB12}.
Interestingly, for the nonlocal mean electromotive force, the lowering
of the critical threshold can be accompanied by multiple instabilities
of different dynamo modes that have different frequencies and spatial
localization; see \cite{Pipin23}.

\subsection{Dynamos from cross-helicity}

An alignment of velocity and magnetic field, i.e., cross helicity, plays
a key role in numerous processes and phenomena of astrophysical plasmas.
\cite{KR80} showed that the saturation stage of the turbulent generation
is characterized by an alignment of the turbulent convective velocity
and the magnetic field.
This consideration does not account for the effects of cross-helicity
that take place in the strongly stratified subsurface layers of the
stellar convective envelope.
For example, the direct numerical simulations of \cite{Matthaeus2008}
showed a directional alignment of velocity and magnetic field fluctuations
in the presence of gradients of either pressure or kinetic energy.

The mean electromotive force in this case is along to the mean vorticity,
\begin{equation}
   \meanEMF^{\Upsilon}= \Upsilon\nabla\times\overline{\mathbf{U}}+\dots, \label{crhl}
\end{equation}
where, $\Upsilon=\tau_c\left\langle
\mathbf{u}\cdot\mathbf{b}\right\rangle$ is the cross helicity
pseudoscalar, and $\tau_c$ is the turbulent turnover time.
Dynamo scenarios based on cross helicity have been suggested in a
number of papers \citep{Yoshizawa1993,Yoshizawa2000,Yokoi2013}.
\cite{Yokoi2018} showed that the large-scale dynamo instability does
%not require for to be the global axisymmetric mean.
not require the existence of a global axisymmetric mean.
The mix of axisymmetric and nonaxisymmetric magnetic fields can be
produced even in the case $\overline{\Upsilon=0}$, where the overbar
means the azimuthal averaging.
The surface magnetic field of the Sun and other similar stars tends
to be organized in sunspots, plagues, ephemeral regions, super-granular
magnetic network, etc.
These structures tend to demonstrate the alignment of local
velocity and magnetic fields \cite{Ruediger2011s}.
Therefore, the cross helicity dynamo instability can contribute to
dynamo generation effects that operate near the stellar surface.
Stellar observations, for example the results of \cite{Katsova2021G},
require such dynamo effects to be working in situ at the stellar surface.
The solar analogs show an increase of the spottiness with an increase
of the rotation rate \citep{Berdyugina2005LRSP}.
In this case, cross helicity dynamo effects can be considered as a
relevant addition to the standard turbulent generation by means of
convective helical motions.
Rapidly rotating M-dwarfs show the highest level of the magnetic activity
\citep{Kochukhov2021AAr}.
There is a population of rapidly rotating M-dwarfs that show a rather
strong dipole type magnetic field.
These stars show a rather small level of differential rotation.
For solid body rotation, an $\alpha^2$ dynamo generates nonaxisymmetric
magnetic field \cite{Chabrier2006AA,Elstner2007AN}.
At high rotation rates, the $\alpha$ effect is highly anisotropic
\cite{Ruediger1993AA}.
It cannot employ the component of the large-scale magnetic field along
the rotation axis for the generation of an axial electromotive force.
Results of \cite{Yokoi2018} show that the $\alpha^2\Upsilon^2$ scenario
can produce a strong constant dipole magnetic field.
The model predicts the existence of large-scale cross helicity patterns
occupying the stellar surface.
We hope that this can be tested either in observations or in global
convective simulations.

The nonlinear theory for the cross helicity effect is not yet developed.
\cite{Sur2009MN} showed that the turbulent generation due to $\Upsilon$
is quenched by the large scale vorticity in a way that is similar to
catastrophic quenching given by \Eq{quenching}, i.e.,
\begin{equation}
    \Upsilon \sim \frac{1}{1+\Rm \tau_c^2 (\nab\times\overline{\mathbf{U}})^2} 
\end{equation}
One should remember that for its initialization of the cross-helicity
dynamo instability we have to seed both the cross helicity and the
magnetic field.
The solar type models scenarios based on cross helicity require an
$\alpha$ effect, which produces poloidal magnetic field and cross helicity
at the top of the dynamo domain \citep{Yokoi2016}.

Given that cross helicity is an ideal invariant of the MHD equations,
it is natural to ask whether systems with strong cross helicity exhibit
inverse cascading.
The answer seems to be yes; see \citep{BGJKR14}.
In \Fig{pBz_spec2__pBzm_top_comp_gkf_bern23} we show the gradual
build-up of magnetic fields in the vertical direction when the system
has significant cross helicity owing to the presence of a magnetic field
along the direction of gravity \citep{Ruediger2011s}.

\begin{figure}[t!]\begin{center}
\includegraphics[width=.98\textwidth]{pBz_spec2__pBzm_top_comp_gkf_bern23}
\end{center}\caption[]{
Normalized spectra of $B_z$ from a simulation of MHD turbulence
with strong gravity at turbulent diffusive times
$t\etat/H_\rho^2\approx0.2$, 0.5, 1, and 2.7
with $\kf H_\rho=10$ and $k_1 H_\rho=0.25$.
Adapted from \cite{BGJKR14}.
}\label{pBz_spec2__pBzm_top_comp_gkf_bern23}\end{figure}

\subsection{Origin of sunspots and active regions}\label{sec:sp}

An important goal in solar dynamo theory is to compute synthetic
butterfly diagrams.
The question then emerges from which depth to take the mean toroidal
field, for example.
The usual argument here is to invoke Parker's theory of sunspot
formation and to postulate that the field at some depth translates
directly to one at the surface.
This is critical because the final result depends on the assumed depth.

It is possible that sunspots are not deeply rooted, but are actually
a surface phenomenon.
No successful and self-consistent model of shallow formation of active
regions or sunspots exists as yet.
Noteworthy in this context is the negative effective magnetic pressure
instability (NEMPI), which is a mean-field theory of the Reynolds and
Maxwell stresses.
This theory is extremely successful in that its results agree remarkably
well with direct numerical simulations (DNS).\footnote{DNS means that
viscous and diffusive operators are assumed to be the physical ones, but
with coefficients that are enhanced relative to the physical ones, but
as small as possible.
Large eddy simulations (LES) or implicit LES, by contrast, use just
numerical schemes to keep the code stable.
Such schemes are often too complicated to state them as an explicit
term in the equations, as if they are negligible, but they never are.}
The problem is only that the effect is not strong enough to make real
sunspots or active regions.
Because of this remarkable agreement between theory and simulations,
and because it is an important mean-field process, we shall discuss
here a bit more detail.

The essence of the effect is the contribution of the turbulent
hydromagnetic pressure to the horizontal force balance.
The turbulent pressure is a small-scale effect, but it reacts to
the large-scale magnetic field.
As the magnetic field increases, it suppresses the turbulence locally,
disturbing therefore the horizontal force balance.
Although this large-scale magnetic field itself contributes with its
own magnetic pressure to the horizontal force balance, the effect from
the suppression of the turbulence is often stronger, so the net effect
is a negative one.
This is why the mean-field effect from a large-scale magnetic field is
a negative effective magnetic pressure.
This idea goes back to early work of \cite{KRR89,Kleeorin1996}, who
developed the mean-field theory for this effect.

In the beginning, it was not clear what kind of numerical experiments
one could try to test the a negative effective magnetic pressure effect.
The first mean-field simulations were done with a uniform horizontal
magnetic field \citep{BKR10}.
This led to the development of magnetic flux concentrations near the
surface, but those began to sink downward as time went on.
A similar effect was soon also seen in DNS \citep{brandenburg+11}.
The sinking of such structures was explained by the {\em negative}
effective magnetic pressure: a positive magnetic pressure would lead to
the rise of structures \citep{Parker67} while a negative one leads to
a sinking.
The sinking of magnetic structures had the side effect that the
structures disappeared from the surface and became less prominent.

\begin{figure}[t]
\includegraphics[width=\textwidth]{pslices_V256k30VF_Bz002}
\caption[]{Cuts of $B_z/\Beq(z)$ in the $xy$ plane at the top boundary
($z/H_\rho=\pi$) and the $xz$ plane through the middle of the spot at $y=0$.
In the $xz$ cut, we also show magnetic field lines and flow vectors
obtained by numerically averaging in azimuth around the spot axis.
Adapted from \cite{BKR13}.}
\label{pslices_V256k30VF_Bz002}\end{figure}

Subsequent experiments with a vertical field had a more dramatic effect
on the general appearance of structures.
Because the ambient field was vertical, the downflow had little effect
on the magnetic flux concentrations themselves \citep{BKR13}.
\Fig{pslices_V256k30VF_Bz002} shows the spontaneous development of a
magnetic spot.

Most of the numerical experiments where done with forced turbulence,
where one had explicit control over the degree of scale separation.
This is not the case in convection, where the development of magnetic
structures takes different shapes \citep{SN12,masada+16,KBKKR16}.

\section{Mean-field dynamo models}
\label{MeanFieldModels}

The first mean-field model was constructed by \citet{Parker1955}.
In his scenario the toroidal magnetic field is generated from the
dipole field by the nonuniform rotation. To overcome restrictions
of the Cowling's theorem \citep{Cowling1933}, Parker suggested that
the dipole magnetic field can be regenerated by cyclonic convective
motions which transform emerging toroidal magnetic loops into poloidal
magnetic field. The coalescing loops can amplify in the dipole magnetic
field. Studying the combing action of the differential rotation and
cyclonic motions he found a solution in form of the dynamo wave and
formulated conditions for the equatorward propagation of the dynamo
waves.  \citet{SKR1966} and \citet{StKr1969} constructed the theoretical
basis of the mean-field theory, introduced the notion of the mean
electromotive force (MEMF) of the turbulence [see Eq.~(\ref{mult})]
and showed that the Parker's effect of the cyclonic convective motions
is an equivalent to the effective MEMF along the large-scale field.
The 1970s can be considered the golden years of mean-field dynamo theory.
\cite{Schuessler1983I} stated: ``dynamo theory reached the textbook
state'', mentioning the famous monographs by \cite{Mof78},
\cite{Par79}, \cite{KR80}, and \cite{VZR1980}.

Indeed, the intensive theoretical and observational studies leaded to
establishment of the basic solar dynamo scenarios, identification the
key dynamo parameters and  formation of general paradigm about nature
of the solar and stellar magnetism.

\cite{Schuessler1983I} summarized that the mean-field dynamo models can reproduce the ``physics of solar activity to a great extent'' including:
\begin{itemize}
\item the Hale polarity rule of sunspots groups
\item the time-latitude evolution of the sunspot activity (``butterfly diagram'')
\item reversals of the polar magnetic field
\item the phase relationship between evolution of the poloidal and toroidal
magnetic field and their consistence with the butterfly diagram \citep{Stix1976}
\item rigid rotation of magnetic sector structure and coronal holes \citep{Stix1974,Stix1977}
\item chaotic variations of the dynamo activity as due to the random $\alpha$ effect and the dynamo nonlinearity because of the Lorentz force \citep{Leighton1969,Yoshimura1978,Ruzmaikin1981}
\item first models scenarios of the solar torsional oscillations \citep{Schuessler1981, Yoshimura1981}
\end{itemize}
We have to note that the first and second items are based on assumption
the sunspot groups are formed from the large-scale toroidal magnetic
field.
Already that time it was well realized and acknowledged that the
mean-field models needs to take into account the fibril state of the
magnetic field which we observed on the solar surface.
We return to this point later.

The classical mean-field dynamo models utilize $\alpha\Omega$ scenario
using the differential rotation ($\Omega$ effect) as the source of
the toroidal magnetic flux production and the $\alpha$ effect for  the
poloidal magnetic field generation.
In general, the $\alpha$ effect, as well as any other turbulent
generation effect, including $\delta$ effect \citep{Raedler1969},
shear-current effect \citep{Kleeorin2000} and the cross-helicity effect
\citep{Yokoi2013} can generate both the toroidal and poloidal magnetic
fields.
Therefore there can be a number of possibility for the solar-types
dynamo models \cite{KR80, Yokoi2016, Pipin2018c}.
Some of them, e.g., skip the $\alpha$ effect at all.
For example, \cite{Seehafer2009} studied $\delta^{\Omega}\Omega$ and
$\delta^{W}\Omega$ scenarios, where turbulent generation of the poloidal
magnetic field is due to $\OO\times\meanJJ$ and shear-current effect,
respectively.
These scenarios show oscillating solution and correct time-latitude
diagram of toroidal magnetic field if the meridional circulation is
included.
Similar possibility was mentioned earlier by \cite{KR80} for
$\delta\Omega$ scenario.
However, the given scenarios result to incorrect phase relation
between activity of the toroidal and poloidal magnetic field.
The aim to search for the $\alpha$ effect alternatives pursues double
benefits.
Firstly, the nonhelical source of dynamo generations avoid the
above mentioned catastrophic quenching problem.
This issue is less important currently.
Secondly, and it was already mentioned earlier by \cite{Koehler1973}
as well as \cite{StKr1969} the mixing length estimate of the $\alpha$ effect
for the solar convection zone parameters results in a very strong $\alpha$
effect with a magnitude as strong as the convective velocity rms.
Solar observations of the ratio between the typical strength of the
toroidal and poloidal fields and the solar cycle period, favor an order
of magnitude smaller $\alpha$ effect.
In addition, the turbulent generation sources in the $\alpha\Omega$
scenario help reduce the given constraints.
We must stress that the global convection dynamo simulations of
\cite{Schrinner2011}, \cite{Schrinner2011a}, and \cite{War+21} showed
that the mean-field models need a full spectrum of turbulent effects to
describe DNS.

In the case of the solar-like star, i.e., with the solar-like
stratification, differential rotation, and meridional circulation
profiles, the turbulent sources of the poloidal magnetic field generation
due to $\delta$, shear-current and cross-helicity effects are likely
complimentary to the $\alpha$ effect.

We thus arrive at the conclusion that the $\alpha^2\Omega$ dynamo is,
probably, the simplest scenario for the solar dynamo.
Also, this scenario seems to fit well in observations of stellar activity
of young solar-type stars.

\subsection{Basic model\label{MFMsubsec}}

We discuss some results of the state of art mean-field dynamo
model of the solar dynamo developed recently by \citet{Pipin2019c}.
The magnetic field evolution is governed by the mean-field induction
equation: 
\begin{equation}
\frac{\partial \meanBB}{\partial t} =\mathbf{\nabla}\times
\left(\meanEMF+\meanUU\times\meanBB-\eta\mu_0\meanJJ\right).
\label{eq:mfe}
\end{equation}
The expression for the components of $\meanEMF$ reads as follows, 
\begin{equation}
\overline{\mathcal{E}}_{i}=\left(\alpha_{ij}+\gamma_{ij}\right)\meanB{}_{j}-\eta_{ijk}\nabla_{j}\meanB{}_{k}.\label{eq:Ea}
\end{equation}
Here, $\alpha_{ij}$ describes the turbulent generation by the $\alpha$ effect,
$\gamma_{ij}$ represents turbulent pumping, and $\eta_{ijk}$ is the eddy
magnetic diffusivity tensor.
The $\alpha$ effect tensor includes
the small-scale magnetic helicity density contribution, i.e., the
pseudoscalar $\left\langle \mathbf{a}\cdot\mathbf{b}\right\rangle $,
\begin{eqnarray}
\alpha_{ij} & = & C_{\alpha}\psi_{\alpha}(\beta)\alpha_{ij}^{\rm K}
+\alpha_{ij}^{\rm M}\psi_{\alpha}(\beta)\frac{\left\langle \mathbf{a}\cdot\mathbf{b}\right\rangle \tau_{c}}{4\pi\overline{\rho}\ell_{c}^{2}},\label{alp2d}
\end{eqnarray}
where $C_{\alpha}$ is the dynamo parameter characterizing the magnitude
the of the kinetic $\alpha$ effect, and $\alpha_{ij}^{\rm K}$ and
$\alpha_{ij}^{\rm M}$ are the anisotropic versions of the kinetic and
magnetic $\alpha$ effects, as described in PK19.
The radial profiles of the $\alpha_{ij}^{(H)}$ and
$\alpha_{ij}^{(M)}$ depend on the mean density stratification, profile
of the convective velocity $\urms$ and on the Coriolis number,
\begin{equation}
  {\rm Co} = 2\Omega_0 \tau_c, \label{eq_M8}
\end{equation}
where $\Omega_{0}$ is the global angular
velocity of the star and $\tau_{c}$ is the convective turnover time.
The magnetic quenching function $\psi_{\alpha}(\beta)$ depends on
the parameter $\beta=|\meanBB|/(\sqrt{4\pi\overline{\rho}}\urms)$.
In this model the magnetic helicity is governed by the global conservation
law for the total magnetic helicity,
$\left\langle \mathbf{A}\cdot\mathbf{B}\right\rangle=\left\langle \mathbf{a}\cdot\mathbf{b}\right\rangle +\meanAA\cdot\meanBB$
\citep[see][]{Hubbard2012,Pipin2013c}:
\begin{equation}
\left(\frac{\partial}{\partial t}+\meanUU\cdot\boldsymbol{\nabla}\right)
\left\langle \mathbf{A}\cdot\mathbf{B}\right\rangle
=-\frac{\left\langle \mathbf{a}\cdot\mathbf{b}\right\rangle }{\Rm\tau_{c}}
-2\eta\meanBB\cdot\meanJJ-\mathbf{\nabla\cdot}\meanFFFF,
\label{eq:helcon}
\end{equation}
where we have used $2\eta\mathbf{\left\langle j\cdot b\right\rangle}=\left\langle \mathbf{a}\cdot\mathbf{b}\right\rangle/{\Rm\tau_{c}}$
\citep{Kleeorin1999}.
Also, we have introduced the diffusive
flux of the small-scale magnetic helicity density,
$\mathbf{\mathbf{\mathcal{F}}}^{\chi}=-\eta_{\chi}\mathbf{\nabla}\left\langle
\mathbf{a}\cdot\mathbf{b}\right\rangle $, and $\Rm$ is the magnetic
Reynolds number, we employ $\Rm=10^{6}$.
Following results of \citet{Mitra2010} we put $\eta_{\chi}=\frac{1}{10}\eta_{T}$.  Here, the turbulent fluxes of the magnetic helicity are approximated by the only term which is related to the diffusive flux. 
Besides the diffusive helicity flux, the other turbulent
fluxes of the magnetic helicity can be important for the
nonlinear dynamo regimes and the catastrophic quenching problem
\citep{Kleeorin2000,Vishniac2001,Pipin2008a,Chatterjee2011,BS05,Kleeorin2022,Gopalakr2023}.
The relative importance of the different kind helicity fluxes for the
dynamo should be studied further.

The above ansatz of the helicity evolution differs from that given by
Eq.~(\ref{smhev}); see also papers by \citet{Kleeorin1982,Kleeorin1999}.
\cite{Hubbard2012} had been studying the magnetic helicity evolution
for the shearing dynamos.
They found that employing \Eq{smhev} in the dynamo problem can result
in nonphysical fluxes of magnetic helicity over spatial scales.
For this ansatz given by \Eq{smhev}, the nonlinear dynamo models can
show the sharp magnetic structures inside the dynamo model domain. Such
structures are connected with concentrations of the magnetic helicity;
see, e.g., \cite{Chatterjee2011} and \cite{BC18}.
Even a strong diffusive helicity flux does not seem to correct those
irrelevant features from the numerical solution.
The technical point is that the helicity fluxes, which are involved in
\Eq{smhev}, should be consistent with the turbulent effects involved
in the mean electromotive force, e.g., the rotationally induced anisotropy
of the $\alpha$ effect, the magnetic eddy diffusivity, etc.
Such calculation are currently absent.
Also, we have to take into account the modulation of the magnetic helicity
density by the magnetic activity.
On the other hand, with the magnetic helicity evolution equation
\Eq{eq:helcon}, \citet{Pipin2013c} found that magnetic helicity density
follows the large-scale dynamo wave.
This alleviates the catastrophic quenching of the $\alpha$ effect.
They showed that if we write the Eq.~(\ref{eq:helcon}) in the form of
Eq.~(\ref{smhev}), we get an additional helicity flux due to the global dynamo, 
Rewriting Eq.~(\ref{eq:helcon}) in the form of Eq.~(\ref{smhev}) we get
\begin{equation}
\frac{\partial \left\langle \mathbf{a}\cdot\mathbf{b}\right\rangle  }{\partial t} =  -2\left(\meanEMF\cdot\overline{\bm{B}}\right)
-\frac{\left\langle \mathbf{a}\cdot\mathbf{b}\right\rangle }{\Rm\tau_{c}}
+\boldsymbol{\nabla}\cdot\left(\eta_{\chi}\boldsymbol{\nabla}
\left\langle \mathbf{a}\cdot\mathbf{b}\right\rangle \right)
 -\eta\overline{\mathbf{B}}\cdot\mathbf{\overline{J}}-\boldsymbol{\nabla}\cdot\left(\meanEMF\times\meanAA\right)+\dots,\label{eq:hel-1}
\end{equation}
where $\dots$ includes additional helicity transport terms due to the large-scale flow. The term $\left(\meanEMF\times\overline{\mathbf{A}}\right)$
consists of the counterparts of the sources magnetic helicity, which
are represented by $-2\meanEMF\cdot\overline{\mathbf{B}}$,
and the fluxes which result from pumping of the large-scale magnetic
fields. The sources magnetic helicity in the term $-2\left(\meanEMF\cdot\overline{\bm{B}}\right)$
are partly compensated in \Eq{eq:hel-1} by the counterparts
in $\left(\meanEMF\times\meanAA\right)$.
This results in the spatially homogeneous quenching of the large-scale
magnetic generation and alleviation of the catastrophic quenching problem.
The effect of $\left(\meanEMF\times\meanAA\right)$
was not unambiguously confirmed in DNS because of limited numerical
resolution; see \cite{DSordo2013MN} and \cite{Brandenburg2018AN}.

The turbulent pumping, which is expressed by the antisymmetric tensor
$\gamma_{ij}$. The tuning of $\gamma_{ij}$ for the solar-type mean-field
dynamo model was discussed by \citet{Pipin2018b}. We define it as
follows, 
\begin{eqnarray}
\gamma_{ij} & = & \gamma_{ij}^{(\Lambda\rho)}+\frac{\alpha_{\mathrm{MLT}}\urms}{\gamma}\mathcal{H}\left(\beta\right)\mathrm{\hat{r}_{n}\varepsilon_{inj}},\label{eq:pump0}\\
\gamma_{ij}^{(\Lambda\rho)} & = & 3\nu_{T}f_{1}^{(a)}\left\{ \left(\mathbf{\boldsymbol{\Omega}}\cdot\boldsymbol{\Lambda}^{(\rho)}\right)\frac{\Omega_{n}}{\Omega^{2}}\varepsilon_{\mathrm{inj}}-\frac{\Omega_{j}}{\Omega^{2}}\mathrm{\varepsilon_{inm}\Omega_{n}\Lambda_{m}^{(\rho)}}\right\} \label{eq:pump1}
\end{eqnarray}
where $\mathbf{\boldsymbol{\Lambda}}^{(\rho)}=\boldsymbol{\nabla}\log\overline{\rho}$
, $\mathrm{\alpha_{MLT}}=1.9$ is the mixing-length theory parameter,
$\gamma$ is the adiabatic law constant.
In \Eq{eq:pump0}, the first term takes into
account the mean drift of large-scale field due the gradient of the
mean density, and the second one does the same for the mean-field
magnetic buoyancy effect. The function $\mathcal{H}\left(\beta\right)$
takes into account the effect of the magnetic tensions.
It is $\mathcal{H}\left(\beta\right)\sim\beta^{2}$ for the small $\beta$
and it saturates as $\beta^{-2}$ for $\beta\gg1$; see P22.

We employ an anisotropic diffusion tensor following the formulation
of \citet{Pipin2008a} (hereafter, P08): 
\begin{eqnarray}
\eta_{ijk} & = & 3\eta_{T}\left\{ \left(2f_{1}^{(a)}-f_{2}^{(d)}\right)\varepsilon_{ijk}+2f_{1}^{(a)}\frac{\Omega_{i}\Omega_{n}}{\Omega^{2}}\varepsilon_{jnk}\right\} ,\label{eq:diff}
\end{eqnarray}
where functions $f_{1,2}^{(a,d)}\left(\Omega^{*}\right)$ are determined
in P08.
Analytical calculations of $\meanEMF$ in  the above cited paper includes
effects of the small scale dynamo.
In the above expressions of the $\meanEMF$ we assume an
equipartition condition between kinetic energy of the turbulence and
magnetic fluctuations which stem from the small-scale dynamo.
It was found that for the case of slow rotation ($Co\ll 1$),
the part of $\meanEMF$ that depends on the gradients of $\meanBB$ consists
of an isotropic eddy diffusivity and R\"adler's $\OO\times\meanJJ$
effect due to the small-scale dynamo (see also \citealp{Raedler2003a}).
In the case of rapid rotation, the fluctuating magnetic fields from the
small-scale dynamo contribute both to isotropic and anisotropic parts
of the diffusivity.
The effect appears already in the terms of order $\Omega^2$ in the global
rotation rate \citep{Raedler2003a}.
In particular, the part of emf which corresponds to  Eq(\ref{eq:diff} can be written as follows,
\begin{eqnarray}
\meanEMF^{\eta}&=& -3\eta_{T}\left(2f_{1}^{(a)}-f_{2}^{(d)}\right)\meanJJ+
6\eta_{T}f_{1}^{(a)}\OO\frac{\OO\cdot\meanJJ}{\Omega^2}.\label{eq:difJ}
\end{eqnarray}
It is noteworthy that the full expression of $\meanEMF$ obtained in P08 is complicated and includes different other
contributions due to effects of global rotation $\OO$, mean shear, mean current,
$\meanJJ$, and the magnetic deformation tensor $(\nabla\meanBB)$. 
We skip them in the application to the solar dynamo model.
The analytical results about the relations of the specific  effects of
the $\meanEMF$ and the global rotation rate show a qualitative agreement
with the DNS of \cite{Kaepylae2009AA,Brandenburg2012AA}.
Yet, a detailed comparison of the analytical results and the global
convective simulations is needed; for further discussions, see
\Sec{dnssec}.

We assume that the large-scale flow is axisymmetric. It is decomposed
into sum of the meridional circulation and differential rotation,
$\mathbf{\overline{U}}=\mathbf{\overline{U}}^{m}+r\sin\theta\Omega\left(r,\theta\right)\hat{\mathbf{\boldsymbol{\phi}}}$,
where $r$ is the radial coordinate, $\theta$ is the polar angle, $\hat{\mathbf{\boldsymbol{\phi}}}$ is
is the unit vector in azimuthal direction, and $\Omega\left(r,\theta\right)$
is the angular velocity profile. The angular momentum conservation
and the equation for the azimuthal component of large-scale vorticity,
$\mathrm{\overline{\omega}=(\boldsymbol{\nabla}\times\overline{\mathbf{U}}^{m})_{\phi}}$,
determine distributions of the differential rotation and meridional
circulation: 
\begin{equation}
\frac{\partial}{\partial t}\overline{\rho}r^{2}\sin^{2}\theta\Omega  =  -\boldsymbol{\nabla\cdot}\left[r\sin\theta\overline{\rho}\left(\hat{\mathbf{T}}_{\phi}+r\sin\theta\Omega\mathbf{\overline{U}^{m}}\right)\right]
  + \boldsymbol{\nabla\cdot}\left[r\sin\theta\frac{\meanBB\meanB_{\phi}}{4\pi}\right],\label{eq:angm} 
\end{equation}
\begin{figure}[t]
\centering \includegraphics[width=0.95\columnwidth]{figm}
\caption{\label{fig1}
(a) Streamlines of meridional circulation and the angular velocity
distribution; the magnitude of circulation velocity is of 13 m/s on the
surface at the latitude of $45^\circ$.
(b) Radial profiles of $\etaT+\eta_{||}$, the rotationally
induced part $\eta_{||}$, as well as $\nuT$.
(c) Radial profiles of the $\alpha$ tensor
at $45^\circ$ latitude.
(d) Streamlines of effective drift velocity from magnetically affected
pumping and meridional circulation.
Reproduced by permission from \cite{Pipin2022}.}
\end{figure}

\begin{eqnarray}
\frac{\partial\omega}{\partial t} & = & r\sin\theta\boldsymbol{\nabla}\cdot\left[\frac{\hat{\boldsymbol{\phi}}\times\boldsymbol{\nabla\cdot}\overline{\rho}\hat{\mathbf{T}}}{r\overline{\rho}\sin\theta}-\frac{\mathbf{\overline{U}}^{m}\overline{\omega}}{r\sin\theta}\right]\label{eq:vort}
  +  r\sin\theta\frac{\partial\Omega^{2}}{\partial z}-\frac{g}{c_{p}r}\frac{\partial\overline{s}}{\partial\theta}\nonumber \\
 & + & \frac{1}{4\pi\overline{\rho}}\left(\meanBB\boldsymbol{\cdot\nabla}\right)\left(\boldsymbol{\nabla}\times\meanBB\right)_{\phi}-\frac{1}{4\pi\overline{\rho}}\left[\left(\boldsymbol{\nabla}\times\meanBB\right)\boldsymbol{\cdot\nabla}\right]\meanB{}_{\phi},\nonumber 
\end{eqnarray}
where $\hat{\mathbf{T}}$ is the turbulent stress tensor: 
\begin{equation}
\hat{T}_{ij}=\left\langle u_{i}u_{j}\right\rangle -\frac{1}{4\pi\overline{\rho}}\left(\left\langle b_{i}b_{j}\right\rangle -\frac{1}{2}\delta_{ij}\left\langle \mathbf{b}^{2}\right\rangle \right),
\label{eq:rei}
\end{equation}
(see detailed description in \citealp{Pipin2018c,Pipin2019c}, hereafter PK19). Also, $\overline{\rho}$
is the mean density, $\overline{s}$ is the mean entropy;
$\partial/\partial z=\cos\theta\partial/\partial r-\sin\theta/r\cdot\partial/\partial\theta$
is the gradient along the axis of rotation. The mean heat transport
equation determines the mean entropy variations from the reference
state due to the generation and dissipation of the large-scale magnetic field  and large-scale flows: 
\begin{equation}
\overline{\rho}\overline{T}\left[\frac{\partial\overline{\mathrm{s}}}{\partial t}+\left(\overline{\mathbf{U}}\cdot\boldsymbol{\nabla}\right)\overline{\mathrm{s}}\right]=-\boldsymbol{\nabla}\cdot\left(\mathbf{F}^{c}+\mathbf{F}^{r}\right)-\hat{T}_{ij}\frac{\partial\overline{U}_{i}}{\partial r_{j}}-\meanEMF\cdot\meanJJ,\label{eq:heat}
\end{equation}
where $\overline{T}$ is the mean temperature, $\mathbf{F}^{r}$ is
the radiative heat flux, $\mathbf{F}^{c}$ is the anisotropic convective
flux (see PK19).
The last two terms in \Eq{eq:heat} take into account the convective
energy gain and sink caused by the generation and dissipation of LSMF
and large-scale flows.
The reference profiles of mean thermodynamic parameters, such as entropy,
density, and temperature are determined from the stellar interior model
MESA \citep{Paxton2015}.
The radial profile of the typical convective turnover time, $\tau_{c}$,
is determined from the MESA code, as well. We assume that $\tau_{c}$
does not depend on the magnetic field and global flows. The convective
rms velocity is determined from the mixing-length approximation, 
\begin{equation}
u_{\rm c}=\frac{\ell_{c}}{2}\sqrt{-\frac{g}{2c_{p}}\frac{\partial\overline{s}}{\partial r}},\label{eq:uc-1}
\end{equation}
where $\ell_{c}=\alpha_{\mathrm{MLT}}H_{p}$ is the mixing length,
$\alpha_{\mathrm{MLT}}=1.9$ is the mixing length parameter, and $H_{p}$
is the pressure height scale. Equation~(\ref{eq:uc-1}) determines the
reference profiles for the eddy heat conductivity, $\chi_{T}$, eddy
viscosity, $\nu_{T}$, and eddy diffusivity, $\eta_{T}$, as follows,
\begin{eqnarray}
\chi_{T} & = & \frac{\ell^{2}}{6}\sqrt{-\frac{g}{2c_{p}}\frac{\partial\overline{s}}{\partial r}},\label{eq:ch-1}\\
\nu_{T} & = &  \mathrm{Pr_{\rm T}}\chi_{\rm T},\label{eq:nu-1}\\
\eta_{T} & = & \mathrm{Pm_{\rm T}}\nu_{\rm T}.\label{eq:et-1}
\end{eqnarray}

It should be noted that stellar convection might well have convection zones
with slightly subadiabatic stratification in some layers.
In those cases, the enthalpy flux can no longer be transported entirely
by the mean entropy gradient, but there can be an extra term that is nowadays
called the Deardorff term; see \cite{Deardorff72}.
Such convection can be driven through the rapid cooling in the surface layers
and is therefore sometimes referred to as entropy rain \cite{B16}.
It is useful to stress that the Deardorff term is distinct from the
usual overshoot, because there the enthalpy flux points downward, while
entropy rain still produces an outward enthalpy flux.
It is instead more similar to semiconvection.

\textbf{Boundary conditions.} At the bottom of the tachocline, $r_{\rm i}=0.68\,R$
we put the solid body rotation and the perfect conductor boundary conditions.
Following to the MESA solar interior model we put the bottom of the
convection zone to $r_{\rm b}=0.728\,R$. At this boundary we fix
the total heat flux, $F_{\rm r}^{\rm conv}+F_{\rm r}^{\rm rad}={\displaystyle \frac{L_{\star}\left(r_{b}\right)}{4\pi r_{b}^{2}}}$.
We introduce the decrease factor of $\exp\left(-100\,z/R\right)$ for
all turbulent coefficients (except the eddy viscosity and eddy diffusivity),
where $z$ is the distance from the bottom of the convection zone.
The decrease of the eddy viscosity and eddy diffusivity is confined
by one order of magnitude for the numerical stability. At the top,
$r_{\rm t}=0.99\,R$ we employ the stress free and black body radiating
boundary. Following ideas of \citet{Moss1992} we formulate the top
boundary condition in the form that allows penetration of the toroidal
magnetic field to the surface: 
\begin{eqnarray}
\delta\frac{\eta_{T}}{r_{\mathrm{top}}}B\left(1+\left(\frac{\left|B\right|}{B_{\mathrm{esq}}}\right)\right)+\left(1-\delta\right)\mathcal{E}_{\theta} & = & 0,\label{eq:tor-vac}
\end{eqnarray}

\textbf{Free parameters.} The model employs the number of free parameters,
including $C_{\alpha}$, the turbulent Prandtl numbers $\PrT$ and $\PmT$,
$\delta$, $B_{\mathrm{esq}}$, and the global rotation rate $\Omega_{0}$.
For the solar case we use the period of rotation of solar tachocline determined
from helioseismology, $\Omega_{0}/2\pi$=434 nHz \citep{Kosovichev1997}.
The best agreement of the angular velocity profile with helioseismology
results is found for $\mathrm{Pr}_{T}=3/4$. Also, the dynamo model
reproduces the solar magnetic cycle period, $\sim20$ years, if $\mathrm{Pm}_{T}=10$.
Results of \citet{Pipin2011a} showed that the parameters $\delta$
and $B_{\mathrm{esq}}$ affect the drift of the equatorial drift of
the toroidal magnetic field field in the subsurface shear layer and
magnitude of the surface toroidal magnetic field. The solar observations
show the magnitude of surface toroidal field about 1-2 G \citep{Vidotto2018}.
To reproduce it we use $\delta=0.99$ and $B_{\mathrm{esq}}=50$G.
In what follows we demonstrate results of the solar dynamo model for
the slightly supercritical parameter $C_{\alpha}$ (10\% above the
threshold). Further details of the dynamo model can be found in \citet{Pipin2019c}.

The \Fig{fig1} illustrates profiles of the basic turbulent effects
and large-scale flow distributions for the nonmagnetic case.
The amplitude of the meridional circulation on the surface is about $13\m\s^{-1}$.
In the low part of the convection zone the equatorward flow is about $1\m\s^{-1}$.
The angular velocity distribution is in agreement with the helioseismology data.

Interestingly, the stagnation point of the meridional circulation is near
lower boundary of the subsurface shear layer, i.e., at $r=0.9\,R$.
This is in agreement with observations of \cite{Hathaway2012} and the
helioseismic inversions of \cite{Stej2021}.
The structure of meridional circulation and turbulent pumping promotes
an effective equatorward drift of the toroidal magnetic field below
the subsurface shear layer; see \Fig{fig1}(d).

\begin{figure}[t]
\includegraphics[width=0.86\textwidth]{fig2}
\caption{\label{fig2}a) The surface radial magnetic field evolution (color
image) and the toroidal magnetic field at $r=0.9R$ (contours in range
of $\pm$1kG); b) snapshots of the magnetic field distributions inside
the convection zone for half dynamo cycle, color shows the toroidal
magnetic field and contours show streamlines of the poloidal field;
c) snapshots of the dynamo induced variations of zonal acceleration
(color image) and streamlines of the meridional circulation variations
(contours); d) variations of zonal velocity acceleration at the surface. }
\end{figure}

\subsection{Parker--Yoshimura dynamo waves and extended cycle} 

The dynamo shown in \Fig{fig2} demonstrates the numerical solution of the dynamo
system including Eqs.~(\ref{eq:mfe}) and (\ref{eq:angm})--(\ref{eq:heat}).
The time latitude diagrams of the surface radial
magnetic field and the toroidal magnetic field in the upper part of
the convection zone show agreement with observations of evolution
the large-scale magnetic field of the Sun (\citealp{Hathaway2015,Vidotto2018},
see also the review of Righmire in this volume). The dynamo waves propagate
to the surface equatorward. The radial direction of propagation follows
the Parker--Yoshimura rule because of positive sign of the $\alpha$
effect in the main part of the convection zone and the positive latitudinal
shear.   Noteworthy that at high latitude the model shows another dynamo wave family which propagates poleward along the convection zone boundary.
This family follows the Parker--Yoshimura rule as well. 
Further we will see that the latitudinal shear plays the dominant
role in this dynamo model and perhaps in the solar dynamo as well
(see also \citealp{CS15}). The latitudinal drift of the toroidal
magnetic field in this model results from the turbulent pumping and
meridional circulation; see Fig\ref{fig1}(d).
The global convective dynamo simulations of \citealp{War+18,War+21} show
the crucial role of the turbulent pumping in the solar type dynamo model,
as well.
The extended mode of the dynamo cycle is another feature of the given
model. The toroidal magnetic field dynamo wave starts
at the bottom of the convection zone around 50$^{\circ}$ latitude
(see the mark points in \Fig{fig2}).
It disappears near the solar equator after full dynamo cycle.
On the surface the extended mode
of the solar cycle is seen in the radial magnetic field evolution,
in the torsional oscillations of zonal flow and in variations of the
meridional circulation as well \citep{Getling2021}. The origin of
the extended mode of the dynamo cycle is due the distributed character
of the large-scale dynamo and interaction of the global dynamo modes,
where the low order dynamo modes, e.g., dipole and octupole modes,
are mainly generated in the deep part of the convection zone and the high
order modes are predominantly generated in the near surface level.
The phase difference between the models results into the dynamo mode of
the extended length \citep{Sten1992,Obridko21}.

\subsection{Torsional oscillations}

Solar zonal variations of the angular
velocity (``torsional oscillations'') were discovered by \citet{Howard1980}.
Since that time it was found that torsional oscillations represent
a complicated wave-like pattern which consists of alternating zones
of accelerated and decelerated plasma flows \citep{Snodgrass1985,Altrock2008,Howe2011}.
\citet{Ulrich2001} found two oscillatory modes of these variations
with the periods of 11 and 22 years. Torsional oscillations were linked
to ephemeral active regions that emerge at high latitudes during the
declining phase of solar cycles, but represent magnetic field of the
subsequent cycle \citep{Wilson1988}. Interesting that in original
paper \citet{Howard1980} conjectured that the solar torsional oscillation
can shear magnetic fields and induce the dynamo cycle. This idea was
further elaborated in a number of papers. However the idea looks unreasonable
because of conflicts with the Cowling theorem. Also, the magnitude
of the torsional oscillations of $3$--$6\m\s^{-1}$ is too small in compare to
magnitude of the magnetic field generated by dynamo.
The first papers by \citet{Schuessler1981} and \cite{Yoshimura1981}
suggested that the 11-h year solar torsional oscillation can be explained
by the mechanical effect of the Lorentz force.
The double frequency of the zonal variation
results from the $B^{2}$ modulation of the large-scale flow due to
the dynamo activity. On the base of the flux-tube dynamo model \citet{Schuessler1981}
using the simple estimation of the large-scale Lorentz force found
both 11 and 22 year mode of the torsional oscillations. This results
was elaborated further by \citet{Kleeorin1991}. 
The further development of the mean-field theory of the solar differential
rotation showed that in addition to the large-scale Lorentz force, the
dynamo induced $B^{2}$ modulation of the turbulent angular momentum
fluxes is also an essential source of the torsional oscillations
\citep{Ruediger1990, Kitchatinov1994a, Kleeorin1996, Kueker1996,
Ruediger2012}.
Global convective dynamo simulations \citep[e.g.,][]{Beaudoin2013,
Kapyla2016, Guerrero2016} confirmed the conclusion.
The strength of the solar torsional oscillations is more than two
orders of magnitude less than the differential rotation.
It looks like the theory of the torsional oscillations can
be constructed using the perturbative approximations. The models of
this type (see, e.g., \citealt{Tobias1996,Covas2000,Bushby2007,Pipin2015,Hazra2017})
were inspired by results of \citet{Malkus1975}. Yet, the constructed
models are incomplete because they ignore the Taylor-Proudman balance,
which is the key ingredient of the solar differential rotation theory
(see \citealt{Kitchatinov2013}, also contribution of Hazra
et al, this volume). The complete mean-field dynamo models which take
into accounts the Taylor-Proudman balance (hereafter TPB) were constructed
by \citet{Brandenburg1992}, \citet{Rempel2007ApJ} and \citet{Pipin2019c}
(hereafter PK19). 
Figure \ref{fig2} shows variations of the zonal
acceleration for our mean-field model in following PK19 line of work.
Similar to results of helioseismology \citep{Howe2011,Kosovichev2019}
and results of \citet{Rempel2007ApJ}, snapshots of the model show
that in the main part of the convection zone the acceleration patterns
are elongated along the rotation axis. This is caused by the Taylor-Proudman
balance. Near the convection zone boundaries these patterns deviate
in the radial direction, which is in agreement with the above cited
helioseismology results, as well. The given observation on the role
of TPB show importance of the meridional circulation and the dynamo
induced heat transport perturbation \citep{Spruit2003,Rempel2007ApJ}
in the theory of the torsional oscillations. This fact does not deny
importance of the large-scale Lorentz force and the magnetic modulation
of the turbulent angular momentum transport. Results of Figures \ref{fig2}b
and c show that the positive sign of the zonal acceleration propagates
from the high latitude bottom of the convection zone toward equator
sticking to the equatorial edge of the dynamo wave. The torsional
oscillation wave is accompanied by the corresponded variations of
the meridional circulation. These variations are induced by the magnetic
perturbations of the heat transport (see details in PK19).
We emphasize that the given dynamo models also show overlapping magnetic
cycles; see \Fig{fig2}(b), similarly to what was originally proposed
by \citealp{Schuessler1981}].
In this case $B^{2}$ effect of the dynamo on the heat transport and the
TPB results in about 4 to 5 meridional circulation cells along latitude.
This track transports zonal variations of angular
velocity, which are caused by the mechanical action of the large scale
Lorentz force and magnetic quenching of the turbulent stresses, from
polar regions to the equator. PK19 found that the induced zonal acceleration
is $\sim(2$--$4)\times10^{-8}~$m$\,$s$^{-2}$, which is in agreement
with the observational results of \citet{Kosovichev2019}.
However, the individual forces in the angular momentum balance such that
the large-scale Lorentz force, the variations of the angular momentum transport
due to meridional circulation, the inertial forces, and others are
by more than an order of magnitude stronger than their combined action
and can reach a magnitude of $\sim10^{-6}~$m$\,$s$^{-2}$.
Therefore the resulting pattern of the torsional oscillations forms in nonlinear
balance, which include the forces driving the angular momentum transport,
the TPB and heat perturbations due to magnetic activity in the convection
zone (see details in PK19).

\subsection{Dynamo flux budget}

Following \cite{CS15} (hereafter, CS15, also see the chapter by Cameron \&
Sch\"ussler) we now estimate the budget of the toroidal magnetic flux
in the dynamo region.
Using the Stokes theorem and the induction equation \Eq{eq:mfe},
we define the derivative of the toroidal magnetic field flux in the
northern hemisphere of the Sun as
\begin{equation}
\frac{\partial\Phi_{\mathrm{tor}}^{\mathrm{N}}}{\partial t}=\oint_{\delta\Sigma}\left(\overline{\mathbf{U}}\times\mathbf{\overline{B}}+\boldsymbol{\mathcal{E}}\right)\cdot\mathrm{d\mathbf{l}},\label{eq:St}
\end{equation}
where $\Phi_{\mathrm{tor}}^{\mathrm{N}}=\int_{\Sigma}\overline{B}_{\phi}\mathrm{dS}$,
$\Sigma$ is the meridional cut of the northern hemisphere of the
solar convection zone, $\delta\Sigma$ stands for the contour confining
the cut and the differential $\mathrm{dl}$ is the line element of
$\delta\Sigma$. The same can be written for the southern hemisphere
flux $\Phi_{\mathrm{tor}}^{\mathrm{S}}$.
Similarly to CS15, we use
the boundary conditions, and we estimate the RHS of the Eq.~(\ref{eq:St})
in the coordinate system which is co-rotating with angular velocity
of the solar equator, $\overline{U}_{0\phi}=R\sin\theta\Omega_{0}$,
and $\Omega_{0}$ the surface angular velocity at the equator,
\begin{eqnarray}
\frac{\partial\Phi_{\mathrm{tor}}^{\mathrm{N}}}{\partial t} & = & \int_{0}^{\pi/2}\overset{\mathrm{I_{1}}}{\overbrace{\left(\overline{U}_{\phi}-\overline{U}_{0\phi}\right)\overline{B}_{r}\mathrm{r_{t}}}}\,\mathrm{d}\theta+\int_{\mathrm{r_{i}}}^{\mathrm{r}_{\mathrm{t}}}\overset{\mathrm{I_{2}}}{\overbrace{\left(\overline{U}_{\phi}^{(\mathrm{\frac{\pi}{2}})}-\overline{U}_{0\phi}\right)\overline{B}_{\theta}^{(\mathrm{\frac{\pi}{2}})}}}\mathrm{dr}\label{eq:integ}\\
 & + & \int_{\mathrm{r_{i}}}^{\mathrm{r}_{\mathrm{t}}}\overset{\mathrm{I_{3}}}{\overbrace{\left(\mathcal{E}_{r}^{(\mathrm{0})}-\mathcal{E}_{r}^{(\mathrm{\frac{\pi}{2}})}\right)}}\mathrm{dr}+\int_{0}^{\pi/2}\overset{\mathrm{I_{4}}}{\overbrace{\left(\mathcal{E}_{\theta}^{(\mathrm{t)}}\mathrm{r_{t}}-\mathcal{E}_{\theta}^{(\mathrm{i})}\mathrm{r_{i}}\right)}}\mathrm{d}\theta\nonumber 
\end{eqnarray}
here, $r_{t}=0.99\,R$, $r_{i}=0.67\,R$, are the radial boundaries of the dynamo region.
In compare to CS15 we have
additional contributions in the budget equation. Figure \ref{fig:bdgs} shows profiles of the
kernels $I_{1-4}$ for the period of the magnetic cycle minimum. The
estimations are based on results and parameters of the mean-field
model presented above. Noteworthy, the south hemisphere should show
the profiles of the opposite sign (see CS15). The results for $I_{1,4}$
qualitatively similar to CS15. This is because the mean-field model
show the qualitative agreement with solar observations for the time
latitude evolution of the surface radial magnetic field. The diffusive
decay of the toroidal magnetic flux is captured as well because of
the phase shift between evolution of the poloidal and toroidal magnetic
field in dynamo model and presumably in the solar dynamo as well.
The model show the sharp poleward increase of $I_{1}$. This effect
produces the winding of the toroidal magnetic field from poloidal
component by the latitudinal shear. The effect of the radial shear,
$I_{2}$, has maximum near the bottom of the convection zone, where
its magnitude is less than the $I_{1}$.
\begin{figure}[t]
\includegraphics[width=0.99\textwidth]{bdgs}\caption{
\label{fig:bdgs}Estimation of contributions of the budget equation;
see \Eq{eq:integ}, for the time of cycle minimum.}
\end{figure}

Figure~\ref{fig:bdg}(a) shows the time evolution of the RHS contributions
of Eq.~(\ref{eq:integ}). In our model the We see that $I_{2}$ is about
factor 2 less than $I_{1}$. Winding of the toroidal field by the
latitudinal shear seems to be the main generation effect in our model
and, perhaps, in the solar dynamo, as well. 
The radial shear is less efficient because it is small in the main part  of the convection zone.
Also, it has the opposite sign near the convective zone boundaries.
This justifies applications of simple 1-D Babcock-Leighton dynamos to
the solar observations as argued by CS15.
Together with the fact of the poleward increase of
$I_{1}$ it explain the relative success of correlation of the polar
magnetic field strength and the magnitude of the subsequent magnetic
cycle for the solar cycle prediction \citep{Choudhuri2007}.
\begin{figure}[t]
\includegraphics[width=0.96\textwidth]{bdg}\caption{\label{fig:bdg}
(a) Time evolution of the RHS contributions of Eq.~(\ref{eq:integ});
(b) the dynamo models budget, black line show the standard mean-field
model, green line - the budget which includes only the surface contributions
($I_{1,3}$) , blue line - the run where the radial subsurface shear
(region r=0.9-0.99R) is neglected and the red line shows the model
with accounts of surface spot-like activity effects. Starting point
is marked by black circle.}
\end{figure}

Figure~\ref{fig:bdg}(b) shows the budget of the toroidal flux generation
rate and loss rate for our dynamo model. The parameters of the budget
are larger than those deduced by CS15 from solar observations. The
difference is because of additional generation and loss terms. The
budget which includes only the surface activity contributions (green
line in Fig.\ref{fig:bdg}b) is less than the full case. Also, the
magnitude of the generation rate by the latitudinal shear can be larger
than in the solar observations because of difference in the latitudinal
profiles of the surface radial magnetic field. We guess that in the
dynamo model the radial magnetic field increase poleward steeper than
in observations. This issue have to be studied further. Figure \ref{fig:bdg}b
shows the budget for another two dynamo models. In one case, we neglect
the generation effect of the radial subsurface shear in region r=0.9-0.99R.
In compare to the standard case, this model shows the reduction of
the generation rate, the amplitude of the generated toroidal flux,
and increase of the loss rate. Therefore we conclude the importance
of the subsurface shear layer for our dynamo model.

\begin{figure}[t]
\includegraphics[width=0.95\textwidth]{form}
\caption{\label{fig:bip}Snapshots of magnetic regions in the south hemisphere in ascending phase of the magnetic cycle.
The left column shows  the nonaxisymmetric magnetic
field lines, time is shown in days.
The right column shows the radial magnetic field on the top boundary. 
(reproduced by permission from \citealp{Pipin2023a}).}
\end{figure}

\subsection{Impact of the surface activity on the deep dynamo}

The above analysis shows the importance of surface activity for the
dynamo model and perhaps for the solar dynamo as well.
Sunspot activity in the form of magnetic bipolar regions is one of
the most important aspects of magnetic surface activity.
A consistent approach to include it in dynamo models is at present absent. 
Also, the origin of sunspots and their bipolar magnetic field is
not well known; see \Sec{sec:sp}.
The Babcock-Leighton type and flux-transport dynamo models use a
phenomenological approach.
It is also applicable to mean-field models.
\citet{Pipin2022} studied the effect of surface activity on convection
zone dynamos. 
Here, we briefly discuss some results of the paper.
The emergence of bipolar magnetic regions (BMRs) is modeled using
the mean electromotive force which is represented by the $\alpha$ and
magnetic buoyancy effects acting on the unstable part of the axisymmetric
magnetic field as follows:
\begin{equation}
\mathcal{E}_{i}^{(\mathrm{BMR})}=\alpha_{\beta}\delta_{i\phi}\left\langle B\right\rangle _{\phi}+V_{\beta}\left(\hat{\boldsymbol{r}}\times\left\langle \mathbf{B}\right\rangle \right)_{i},\label{eq:ep}
\end{equation}
where the first term takes into account the BMR's tilt and the second
term models the surface magnetic region in the bipolar form.
To produce the bipolar like regions we have to restrict spatially
$V_{\beta}$ in Eq.~(\ref{eq:ep}) to the small scales that are typical
for the solar BMR; see details in the above cited paper.
Position and emergence time are chosen to be random and
modulated by the large-scale magnetic activity.
The BMR's $\alpha$-effect parameters are random as well; see details in
\citep{Pipin2022,Pipin2023a}. The given approach could be refined further
using the 3D hydrodynamics, effects of the Coriolis force and the theory
of the Joy's law developed recently by \cite{Kleeorin2020m}.
Figure~\ref{fig:bip} illustrates the formation of BMR simulated in the
dynamo model.
It was found that the BMR effects on the dynamo are restricted to the
shallow layer below the surface.
At the surface, the effect of the BMR on the magnetic field generation
is dominant.
Compare to the standard axisymmetric mean-field model discussed in
the subsections above, the nonaxisymmetric dynamo, which includes the
emergence of tilted BMR, can result in additional dynamo generation of
the large-scale poloidal magnetic field and to an increase of the polar
magnetic field.
The red line in \Fig{fig:bdg}(b) shows the budget for this nonaxisymmetric
dynamo model.
We see an increase of the toroidal flux generation rate in the
nonaxisymmetric model because of the surface BMR activity.
Similar to \citet{CS15}, we can conclude that sunspot surface activity
seems to play an important part in the solar large-scale dynamo.

\subsection{Effect of corona on the dynamo}

Usually, dynamo models are limited to the star embedded in a vacuum,
which is described by boundary conditions on the stellar surface. 
However, the boundary conditions have a determining influence on the
global solutions, such as the symmetry about the equator.
With the assumption of an external vacuum, all induction effects in the
corona are neglected.
Since the solar surface rotates differentially, the highly conductive
plasma in the corona also causes induction effects through shear.
Observations of coronal rotation are very scarce. 
There is evidence from extended coronal holes of rigid rotation in latitude \citep{Timothy75, Bagashvili17}.  
Kinematic dynamo models involving the corona with various assumptions 
on its rotation and conductivity give a wide range of solutions \citep{Elstner20}. 
A notable influence of the corona on the dynamo in the convection zone 
was also observed in DNS by \citet{Warnecke16}. 
A too weak density contrast and too strong viscous coupling of the corona to the star 
in their model probably underestimates the effect of the Lorentz force in the corona. 
Considering a dynamical situation with dominant Lorentz force in the corona, 
the solution in the Sun corresponds to that with vacuum boundary condition independent of rotation and conductivity in the corona. 
The magnetic field in the corona varies in time to a nearly force-free solution. 
Further investigations of the star-corona coupling are needed to clarify the exchange of magnetic energy and helicity. 

\section{Stellar cycle periods}

\cite{NWV84} developed an early understanding of the observed stellar
cycle periods, $P_{\rm cyc}$.
In those early years, there where just six stars with measured rotation
and cycle periods.
Remarkably, those values have not changed much with the more accurate
data of \cite{Bal+95}; see \Tab{Tstar1} for a list of the cycle periods
of \cite{NWV84}, compared with those of \cite{Bal+95} and the more
recent data set of \cite{BC22}.
The data of \cite{NWV84} suggested
\begin{equation}
\omega_{\rm cyc}\propto \Omega^{1.25}.
\label{omcyc}
\end{equation}
for the cycle frequency $\omega_{\rm cyc}=2\pi/P_{\rm cyc}$ versus angular
rotation rate $\Omega$.
This dependence is reproduced by considering free dynamo waves
by assuming axisymmetric mean fields $\meanBB=b\pphi+\nab\times a\pphi$
with $(a,b)\propto e^{\ii(ky-\omega t)}$ and writing
$-\ii\omega=-\ii\omega_{\rm cyc}+\lambda$, where both
$\omega_{\rm cyc}$ and $\lambda$ are assumed to be real.
The main field dynamo equations result in traveling wave solutions
with a dispersion relation of the form
\begin{equation}
\lambda=\sqrt{\alpha\Omega' kL/2}-\etaT k^2,
\end{equation}
\begin{equation}
\omega_{\rm cyc}=\sqrt{\alpha\Omega' kL/2}.
\end{equation}
At least up to moderate rotation rates, it is reasonable to assume that $\alpha$
and $\Omega'$ are proportional to $\Omega$.
The crucial assumption in arriving at an approximation that matches
\Eq{omcyc} is to assume that the relevant wavenumber $k_y$ is selected
not by the condition of marginal excitation, but by the assumption that
$\lambda=\lambda(k)$ is maximized.
Thus, $k$ has to obey the $\dd\lambda/\dd k=0$, which yields
$\omega_{\rm cyc}\propto(\alpha\Omega')^{2/3}\propto\Omega^{4/3}$.
By contrast, if the dynamo is quenched to the being marginally excited, then
$\omega_{\rm cyc}\propto(\approx\etaT/L^2$, which would be either independent
of $\Omega$, or perhaps even decreasing with $\Omega$, if $\etaT$ decreases
with increasing $\Omega$ due to quenching.

\begin{table}[t]
\caption{Comparison of cycle periods $P_{\rm cyc}$ (in years) from
\cite{NWV84} (NWV84), \cite{Bal+95} (Bal+95), and \cite{BC22} (BC22).
The last two columns compare the seismic age given by BC22 and the
gyrochronological age as listed by \cite{Brandenburg2017A} (BMM17).
The latter differ significantly, but the determined cycle periods
were remarkably stable over the decades.
}\label{Tstar1}
\centerline{\begin{tabular}{rccccc}
\hline\noalign{\smallskip}
       & \multicolumn{3}{c}{--- $P_{\rm cyc}$ [yr] ---} & \multicolumn{2}{c}{age [Gyr] } \\
   HD  & NWV84 & Bal+95 & BC22  & BC22 & BMM17 \\
\noalign{\smallskip}\svhline\noalign{\smallskip}
  3651 &  10   & 13.8   & 14.70 & ---  & 7.2 \\
  4628 &   8.5 &  8.37  &  8.47 & 3.33 & 5.3 \\
 16160 &  11.5 & 13.2   & 12.68 & ---  & 6.9 \\
160346 &   7   &  7.00  &  7.19 & ---  & 4.4 \\
201091 &   7   &  7.3   &  7.11 & 6.10 & 3.3 \\
201092 &  11   & 11.7   &  ---  & ---  & 3.2 \\
\noalign{\smallskip}\hline\noalign{\smallskip}
\end{tabular}}
\end{table}

Of course, nonlinear dynamos must always be quenched to reach a steady state.
This led \cite{BST98} to suggest that \Eq{omcyc} could be obeyed if both
$\alpha$ and $\etaT$ are {\em antiquenched} in such a way that
$\etaT$ is quenched faster than $\alpha$, so that $\omega_{\rm cyc}$ would
increase with increasing magnetic field strength, and hence $\Omega$, and
would still saturate.
Whether this the only viable solution to this puzzle remained unclear.

\begin{figure}[t]
\begin{center}
\includegraphics[width=0.72\columnwidth]{rp1}
\end{center}
\caption{\label{corp}Dependence of cycle period on stellar rotation rate.
Red and black crosses show the results of \citet{Brandenburg2017A},
green crosses those of  \citet{Lehtinen2020},
orange squares the models of \citet{warnecke18},
and stars are from the models of \citet{Pipin21c};
act/inact marks the active and inactive branches of activity;
`kin' and `nkin' stand for kinematic and nonkinematic models (adapted by permission from \citealp{Pipin21c}).}
\end{figure}

Recently, a number of the numerical dynamo models were applied to
investigate the relation of the cycle period on the stellar rotation
rate in the solar analogs \citep{Pipin2015,Strugarek2017,warnecke18,Hazra2019,Pipin21c,Noraz2022}.
Figure \ref{corp} shows some these results including
the results of observations of \citet{Brandenburg2017A} and survey
of \citet{Lehtinen2020}. Interesting that the saturation branch of  the stellar activity on the young solar analogs with period of rotation  less than 10 days is well reproduced in the very different solar-like dynamo models including the global convective dynamo simulation \citep{Strugarek2017,warnecke18},  flux transport model of \cite{Hazra2019} and mean-field model of \cite{Pipin21c}. In Fig.\ref{corp} this branch is marked by the green line. The mean cycle period in this branch is almost independent of stellar rotation rate. The non-kinematic nonlinear model of \cite{Pipin21c} show multiple periods along this line.  \cite{Pipin21c} found that saturation  of the dynamo activity is accompanied  by depression of the latitudinal shear, concentration of the magnetic activity to the surface and changes the meridional circulations from one-cell to multiple-cell per hemisphere structure.  Following conclusions of the above cited paper, in saturated stated the dynamo waves do not follow the Parker--Yoshimura law.
Their cycle period is determined by the turbulent  diffusion and meridional circulation. That is why predictions of the flux-transport and nonkinematic mean-field dynamo models coincide. The independence of the cycle period from rotation rate can be typical for the dynamo solutions which show concentration of the magnetic activity toward the dynamo region boundaries (see \citealp{Pipin2015,Pipin2016b}).

\begin{figure}[t]
\begin{center}
\includegraphics[width=.9\textwidth]{pKG_BonannoCorsaro}
\end{center}
\caption{$P_{\rm rot}/P_{\rm cyc}$ versus $\log\bra{R_{\rm HK}'}$ for all
stars of \cite{BC22} (small black symbols).
Lowercase (uppercase) letters denote data points of \cite{BC22}
that were also included in the sample of \cite{Brandenburg2017A}.
The dotted lines denote the fits determined by \cite{Brandenburg2017A}
while the upper (lower) solid lines denote fits to the stars of
\cite{BC22} with lowercase  (uppercase) letters.}
\label{pKG_BonannoCorsaro}\end{figure}

The inactive branch of the nonkinematic mean-filed dynamo models shows
fairly strong positive inclination (see \Fig{corp}), which is absent in
the kinematic models.
We see that the dynamo model can reproduce an power law $\sim\Co^{0.5}$
avoiding the {\em antiquenching} concept of \cite{BST98}.
In fact, the nonkinematic dynamo models show the so-called doubling
frequency phenomena for the models in between 10 and 15 days rotation
period (see Figs.~3 and 8 of \citealp{Pipin21c}).
The frequency doubling or the second harmonic generation is known from
nonlinear optics.
It is typical for the waves propagation in the nonlinear media.
In the dynamo waves, the  second harmonics are generated because of
the $B^2$ effects such as the magnetic effects on the large-scale flow,
magnetic helicity conservation and magnetic buoyancy effects.
The second harmonics can be found in the solar activity, as well
\citep{Setal20}.
For the solar case they are subdominant.
However they can become dominant for the fast rotating stars.
This makes the interpretation of the magnetic activity cycles difficult
\citep{Stepanov2020MN}.
Summarizing, we find the Parker--Yoshimura dynamo regime for the solar
analogs rotating with period above 15 days; in interval of the stellar
rotation periods between 10 to 15 days the doubling frequency occurs; for
the lower rotational periods the dynamo transits to a saturation stage,
it can be characterized by the high magnetic activity and multiply dynamo
periods which are independent of the stellar rotation rate.

\begin{table}[t]
\caption{Comparison of stellar cycle properties from the samples of
\cite{BC22} and \cite{Brandenburg2017A} (indicated as ``old'').
The blue italics and red roman letters refer to the stars
discussed in \cite{Brandenburg2017A} and are also indicated in
\Fig{pKG_BonannoCorsaro}.
}\label{Tstar2}
\centerline{\begin{tabular}{rcccccccc}
\hline\noalign{\smallskip}
HD~~ & Sym & $\log\bra{R_{\rm HK}'}$ & $\log\bra{R_{\rm HK}^{'old}}$ & 
$P_{\rm rot}$ [d] & $P_{\rm rot}^{\rm old}$ [d] &
$P_{\rm cyc}$ [yr] & $P_{\rm cyc}^{\rm I}$ [yr] & $P_{\rm cyc}^{\rm A}$ [yr] \\
\hline\noalign{\smallskip}
100180 & \blue{\it h} & $-4.83$ & $-4.92$ & 14.06 & 14.00 &  3.60 &  3.60 & 12.90\\
103095 & \blue{\it i} & $-4.90$ & $-4.90$ & 32.51 & 31.00 &  7.07 &  7.30 &  --- \\
 10476 &      \red{c} & $-4.97$ & $-4.91$ & 35.40 & 35.20 & 10.45 &  9.60 &  --- \\
146233 & \blue{\it l} & $-4.95$ & $-4.93$ & 22.66 & 22.70 & 11.59 &  7.10 &  --- \\
160346 &      \red{m} & $-4.86$ & $-4.79$ & 34.20 & 36.40 &  7.19 &  7.00 &  --- \\
 16160 &      \red{d} & $-4.94$ & $-4.96$ & 48.29 & 48.00 & 12.68 & 13.20 &  --- \\
165341 &      \red{n} & $-4.61$ & $-4.55$ & 19.51 & 19.00 &  5.09 &  5.10 & 15.50\\
166620 &      \red{o} & $-5.00$ & $-4.96$ & 42.25 & 42.40 & 16.81 & 15.80 &  --- \\
219834 &      \red{s} & $-4.93$ & $-4.94$ & 38.89 & 43.00 &  9.48 & 10.00 &  --- \\
 26965 &      \red{f} & $-4.96$ & $-4.87$ & 40.83 & 43.00 & 10.24 & 10.10 &  --- \\
  3651 &      \red{a} & $-5.06$ & $-4.99$ & 40.50 & 44.00 & 14.70 & 13.80 &  --- \\
  4628 &      \red{b} & $-4.95$ & $-4.85$ & 37.82 & 38.50 &  8.47 &  8.60 &  --- \\
 81809 &      \red{g} & $-4.89$ & $-4.92$ & 40.93 & 40.20 &  8.05 &  8.20 &  --- \\
219834 &      \red{r} & $-5.10$ & $-5.07$ & 43.40 & 42.00 & 16.29 & 21.00 &  --- \\
201091 &      \red{p} & $-4.56$ & $-4.76$ & 35.62 & 35.40 &  7.11 &  7.30 &  --- \\
   Sun & \blue{\it a} & $-4.94$ & $-4.90$ & 25.55 & 25.40 & 10.70 & 11.00 & 80.00\\
149661 &      \red{K} & $-4.61$ & $-4.58$ & 20.92 & 21.10 & 12.38 &  4.00 & 17.40\\
152391 & \blue{\it M} & $-4.46$ & $-4.45$ & 11.01 & 11.40 & 11.94 &  ---  & 10.90\\
156026 &      \red{L} & $-4.56$ & $-4.66$ & 18.85 & 21.00 & 19.31 &  ---  & 21.00\\
190406 & \blue{\it N} & $-4.76$ & $-4.80$ & 14.01 & 13.90 & 18.61 &  2.60 & 16.90\\
 76151 & \blue{\it F} & $-4.68$ & $-4.66$ & 14.70 & 15.00 & 16.34 &  2.50 &  --- \\
 78366 & \blue{\it G} & $-4.57$ & $-4.61$ &  9.60 &  9.70 & 14.26 &  5.90 & 12.20\\
114710 & \blue{\it J} & $-4.74$ & $-4.75$ & 11.99 & 12.30 & 14.12 &  9.60 & 16.60\\
 22049 &      \red{E} & $-4.46$ & $-4.46$ & 11.09 & 11.10 & 11.00 &  2.90 & 12.70\\
\noalign{\smallskip}\hline\noalign{\smallskip}
\end{tabular}
}\end{table}

In recent work of \cite{BC22}, new cycle data were collected for
altogether 67 stars.
Their new sample includes stars with less accurate data points, so the
existence of different branches was no longer a pronounced feature.
In addition, many of the new data points are different from the earlier
ones of \cite{Brandenburg2017A}; see \Tab{Tstar2}.
As in their paper, we denote G and F dwarfs by the same blue italic symbols
and K dwarfs by the same red roman symbols.

To see how strong this revision of the data is, we plot in 
\Fig{pKG_BonannoCorsaro} the ratios $P_{\rm rot}/P_{\rm cyc}$
versus $\log\bra{R_{\rm HK}'}$ for all stars of \cite{BC22}
and highlight with lowercase and uppercase letters the stars
that were also included in the sample of \cite{Brandenburg2017A}.
We see that the new data are remarkably consistent with the
old ones.
Out of the eight stars on the branch of active stars,
five where listed by \cite{Brandenburg2017A} as having two periods.
Of the 16 inactive stars, three were listed with two periods, but
the case of the Sun was classified by \cite{Brandenburg2017A}
as somewhat different, because the 80 years Gleissberg cycle
does not fit well on the active branch and, unlike all the
other stars with two cycle periods, which are all younger than
$3.3\Gyr$, the Sun is relatively old.

\section{Mean-field models based on the EMF obtained from DNS\label{dnssec}}

We now review recent studies of mean-field dynamo models constructed
based on the electromotive force (EMF) obtained from direct numerical
simulation (DNS) of rotating stratified convection, especially focusing on
``semi-global'' models.
The properties of solar and stellar convection, and the various methods
for extracting the information of the EMF from DNS are also summarized.

\subsection{Properties of Solar and Stellar Convection}

The convection zones (CZs) of the Sun and stars are in a turbulent state
with huge values of fluid Reynolds number (${\rm Re} \gtrsim 10^{12}$),
magnetic Reynolds number (${\rm Re_M} \gtrsim 10^8$),
Rayleigh number (${\rm Ra} \gtrsim 10^{20}$), and an
extremely low Prandtl number (${\rm Pr} \sim 10^{-4}$--$10^{-7}$);
see, e.g., \cite{ossendrijver03}.
A quantitative physical description of solar and stellar dynamos, which
should be the result of the nonlinear interaction of turbulent flows and
magnetic fields, is a great challenge for us and constitutes a significant
milestone on the long way to a full understanding of turbulence.
Even with state-of-the-art supercomputers, it is impossible to numerically
simulate solar and stellar convection and its interaction with the
magnetic field and to observe/analyze numerical data in detail with
realistic parameters.
Therefore, to say with complete confidence that one has fully understood
the solar and stellar dynamo problem, it should be necessary to find a
universal law of magneto-hydrodynamic (MHD) turbulence, build a reliable
sub-grid scale (SGS) turbulence model, and then reproduce the magnetic
activities of the Sun and stars quantitatively in an integrated framework
by numerical models with incorporating the SGS model.
This is because fluid quantities that may be verified in future
observations should include the meridional distributions of fluid velocity,
vorticity, kinetic helicity, and thus the turbulence model constructed
on the basis of these profiles \citep[e.g.,][]{hanasoge+16}.
Only when the correctness of the turbulence model is observationally
validated should our understanding of the solar and stellar dynamos as
a consequence of the turbulent dynamo process be completed.
In the near future, a very exciting time may come when we will be able
to test and verify various turbulence models under extreme conditions
inside the solar and stellar interiors.

What physical characteristics should be taken into account when
constructing a turbulence model of thermal convection in the Sun
and stars?
Let us summarize some essential features:
\begin{enumerate}
\item {\bf Extremely low dissipation}: turbulent state with ${\rm Re} \gtrsim 10^{12}$, $\Rm \gtrsim10^{8}$,
and a large P\`{e}clet number, ${\rm Pe} \sim 10^6-10^9$ (where ${\rm Pe} = {\rm Re} \cdot {\rm Pr})$.
\item {\bf Huge separation of dissipation scales}: ${\rm Pr} \sim 10^{-4}$--$10^{-7}$, ${\rm Pr_M} \sim 10^{4}$ 
\item {\bf Compressibility}: high Mach number $\mathcal{O}(1)$ in the upper convection zone makes the convective motion compressible.
\item {\bf Anisotropy}: spin of stars (i.e., Coriolis force in a rotating system) makes fluid motions anisotropic. 
\item {\bf Inhomogeneity}: density contrast of $10^6$ between top and bottom CZs results in multi-scale properties of fluid motion. 
\item {\bf Non-locality}: Radiative energy loss at the CZ surface (open system), allowing the growth of cooling-driven downflow. 
\end{enumerate}
In view of these features, it can be seen that the characteristics of
thermal convection operating inside the Sun and stars are quite different
from those of isotropic turbulence.
Those can be considered to some extent in DNS even with the current
computing performance, as listed under 3--6, while the others, 
(items 1 and 2) are unreachable with current grid-based simulations.
It should be emphasized, however, that higher resolution simulations
using state-of-the-art supercomputers is a classical way forward in
turbulence research, and the knowledge obtained from such studies in
unexplored low-dissipation regimes will greatly expand the horizon of
our understanding of turbulence \citep[e.g.,][]{kaneda+03,hotta+21}.
Moreover, if sufficient scale separation between the turbulent and mean
fields is ensured and the inertial range of the turbulent cascade is
captured appropriately, there is the possibility that the evolution
of mean-field components, such as large-scale flow and large-scale
magnetic field, can be approximately reproduced even by simulations with
enhanced dissipation compared to the actual solar and stellar values
\citep[e.g.,][]{ossendrijver03}.
It should be remembered, however, that in spite of the rapid increase
in computing power, some rather basic questions about the solar dynamo
still remain, for example the equatorward migration of the sunspot belts
and the formation of sunspots themselves.

\subsection{Semi-global simulation of rotating stratified convection}

On our way toward a reliable SGS turbulence model for solar and stellar
interiors, numerical models of convection and its dynamo should be
studied, while keeping the characteristic features of solar and stellar
convection, as listed under items 3--6 above, in mind.
It should be noted that the underlying necessity for numerical modeling
is an important component of earlier studies that applied mixing-length
type concepts to the dynamo theory, which never successfully explained the
magnetic activities of the Sun and stars \citep[e.g.,][]{brandenburg+88}.

In recent years, significant progress has been made in global convective
dynamo simulations \citep[e.g.,][]{browning+06, ghizaru+10, kapyla+12,
masada+13, fan+14, augustson+15, hotta+16, warnecke18}, there is also
a growing effort to extract the information of turbulent transport
processes from so-called ``semi-global'' (or local model) MHD convection
simulations with the aim of quantifying the dynamo effect of rotational
stratified convection \citep[e.g.,][]{brandenburg+90, brandenburg+96,
nordlund+92, brummell+98, brummell+02, ossendrijver+01, kapyla+06a,
kapyla+09, masada+14a, masada+14b, masada+16, bushby+18, masada+22}.
A typical numerical setup of the semi-global model is shown in
\Fig{fig_YM1} schematically.
In this setting, the gas is gravitationally stratified in the vertical
direction, while periodicity is assumed in the horizontal directions.
The governing equations (mostly compressible MHD equations) are solved
in a rotating Cartesian frame, and the rotation axis is usually set to
be parallel or anti-parallel to the gravity vector.
Several studies have simulated the model with the tilt of the rotation
axis with respect to the gravity vector, and the latitudinal dependence
of the convection has been investigated \citep[e.g.,][]{ossendrijver+01,
kapyla+04, kapyla+06a}.

\begin{figure}[t]
\includegraphics[width=0.96\textwidth]{f1_YM.pdf}
\caption{Numerical setup typical for semi-global simulation of rotating
stratified convection.
Since the CZs of the Sun and stars are strongly stratified, there is
a large separation of time scales from minutes (upper CZs) to months
(bottom CZs).}
\label{fig_YM1}
\end{figure}

\subsection{Extraction of information of dynamo effects}
In the semi-global studies, four-types of approaches have been used typically to extract the information of dynamo effects veiled in the convective motion.
The starting point of all the four methods is common, the decomposition of the flow field (${\bm U}$) and magnetic field (${\bm B}$) into a spatially large-scale,
slowly-varying mean-component, and a small-scale, rapidly varying fluctuating component, as introduced in \S~1, i.e.,
${\bm U} = \overline{{\bm U}} + {\bm u}$ and ${\bm B} = \overline{{\bm B}} + {\bm b}$,
where the lower-case represent the fluctuating component and the overbars denote the mean component. 
In the case of a semi-global model, a temporal and horizontal average
is often used for deriving the mean component.
Then, the equation of mean-field electrodynamics can be derived
\begin{equation}
  \frac{\partial \overline{\bm B}}{\partial t} = \nabla \times (\overline{{\bm U}} \times \overline{{\bm B}} + \meanEMF - \eta \nabla \times \overline{{\bm B}} ) \;, \label{eq_M2}
\end{equation}
where $\meanEMF = \overline{ {\bm u}^\prime \times {\bm b}^\prime} $ is the mean electromotive force (EMF) due to the fluctuation of the flow and
the magnetic field. The mean EMF can be described as a power series about the large-scale magnetic component and its derivatives as
\begin{equation}
  \meanEMF = \overline{ {\bm u} \times {\bm b}} = {\bm \alpha} \cdot \overline{{\bm B}} + {\bm \gamma}\times \overline{{\bm B}} - {\bm \beta}\cdot (\nabla \times \overline{{\bm B}}) + \cdots \;, \label{eq_M3}
\end{equation}
where ${\bm \alpha}$ represents (tonsorial form of) the $\alpha$-effect, ${\bm \gamma}$ is the turbulent pumping, and ${\bm \beta}$ denotes the turbulent diffusion. 

To obtain the information of the dynamo coefficients, such as
${\bm \alpha}$, ${\bm \gamma}$, and ${\bm \beta}$, from the MHD
convection simulation, there are the following four methods:
\begin{enumerate}
\renewcommand{\labelenumi}{(\roman{enumi})}
\item Method based on first-order smoothing approximation (FOSA) expressions
\item Imposed-field method
\item Test-field method
\item Self-sustained field based method
\end{enumerate}

Method (i) involves the the estimation of dynamo coefficients based on
FOSA (also known as the second-order correlation approximation).
There, the distributions of, for example, the fluctuating components of
the convection velocity (${\bm u}$), vorticity (${\bm \omega} = \nabla
\times {\bm u}$), and the resulting kinetic helicity ($\mathcal{H} =
{\bm \omega}\cdot{\bm u}$), are directly extracted from the simulation
results and used to reconstruct the turbulent $\alpha$ and $\beta$
via their analytic forms, derived under FOSA, such as
\Eq{alphaFOSA} and $\beta =(\tau/3)\overline{ {\bm u}^2}$,
where $\tau$ is the correlation time of the turbulence and
is often replaced by the convective turnover time.
Note that anisotropy effects are often neglected in the expressions above,
but see \cite{BS07}, who included them.

Method (ii) is mainly used in the analysis of the numerical results
without self-sustained magnetic field.
In this method, a uniform external magnetic field is imposed
as the mean component to the computational domain, artificially.
Then, the turbulent $\alpha$-effect or the turbulent magnetic diffusivity
is inferred from $\meanEMF = \overline{ {\bm u} \times {\bm b}}$, which
is directly calculated from simulation data, via the relationship, 
if $\meanEMF = \alpha \overline{{\bm B}} -\beta \mu_0 \overline{{\bm J}}$ with $\overline{{\bm J}} = \nabla \times \overline{{\bm B}}$,
\begin{equation}
\alpha = \meanEMF \cdot \overline{{\bm B}} /\overline{{\bm B}}^2  \;, \label{eq_M5}
\end{equation} 
when assuming $\meanJJ\cdot\meanBB = 0$. Furthermore, 
one might be tempted to compute $\beta=\meanEMF\cdot\meanJJ/(\mu_0 \meanJJ^2$),
but these would assume that $\meanJJ\cdot\meanBB$ is vanishing, which is generally
not the case for $\alpha$-effect dynamos; see \cite{Hubbard+09} for details.

Method (iii) utilizes a so-called test-field, as introduced by
\citet{schrinner+05,schrinner+07} for the spherical case and
\cite{BRS08} for the Cartesian case, allowing for scale dependence.
In this method, the evolution equation of ${\bm b}^\prime_{\rm T}$, the
fluctuating component of the test field ${\bm B}_{\rm T}$, which are passive
to the velocity field taken from the simulation, is solved additionally
to the basic (MHD) equations.
From the linear evolution of the test-field, the mean EMF is evaluated
and then the full set of turbulent transport coefficients can be obtained.
For example, in the case without the large-scale flow,
the test-field equation is, for ${\bm b}_{\rm T}$,
\begin{equation}
\frac{\partial {\bm b}_{\rm T}}{\partial t} = \nabla \times ({\bm u} \times {\bm B}_{\rm T} + {\bm u} \times {\bm b}_{\rm T} - \overline{{\bm u}\times {\bm b}_{\rm T}} - \eta \nabla \times {\bm b}_{\rm T}) \;, \label{eq_M6}
\end{equation} 
with a chosen test field ${\bm B}_{\rm T}$ while taking ${\bm u}$ from the MHD simulation. Note that the test-field method is only valid in the absence of turbulent magnetic
components primarily, that is, if the magnetic fluctuation ${\bm b}$ vanish for $\overline{{\bm B}} = 0$. 

Method (iv) can be used only in the analysis of the numerical results
with self-sustained magnetic fields.
Since the fluctuating and mean components are all known quantities
in such simulations, the mean {\it emf}, $\meanEMF = \overline{ {\bm u} \times {\bm b}}$, and the mean magnetic component, $\overline{{\bm B}}$, can be directly
calculated from the simulation data. Then, the mean profiles of dynamo coefficients are inferred based on a fitting procedure via the relationship,  
\begin{equation}
  \mathcal{E}_i = \alpha_{ij}\overline{B}_j + \epsilon_{ijk}\gamma_j\overline{B}_k 
+\mbox{higher derivative terms}\;. \label{eq_M7}
\end{equation}
Given $\mathcal{E}_i$ and $\overline{B}_i$ which are calculated from
simulation data, and then find $\alpha_{ij}$ and $\gamma_i$ such that
the residual of Eq.~(\ref{eq_M7}) is minimized.
In the equation above, the contributions from the derivatives of the
mean magnetic component to the mean {\it emf} are neglected
\citep[see, e.g.,][for the fitting based analysis of
the dynamo coefficient with including the contribution
from the first-order derivative of the mean magnetic
component]{racine+11,simard+13,simard+16,shimada+22}.
In all cases, however, the first (and often higher) derivative terms are
of the same order as the first term and can therefore not be neglected.
This was already done in the work of \cite{BS02}, who typically found small
diffusion coefficients in the cross-stream direction.
This, however, turned out to be a shortcoming of the method and has not
been borne out by more advanced measurements \citep{Karak+14}.

\subsection{Transport coefficients from semi-global turbulence simulations}

Here, we briefly review the results of previous semi-global simulations,
with a particular focus on the studies that have been dedicated for
extracting information about dynamo coefficients.

\citet{brandenburg+90}, hereafter B90, performed turbulent 3-D magneto-convection simulations under the influence of the rotation for the semi-global model whose depth is
equivalent to about one pressure scale height. They found that, due to the effect of the rotation, a systematic separation of positive and negative values of the kinetic
helicity was developed in the vertical direction of the CZ, i.e., in the upper CZ, negative (positive) helicity in the northern (southern) hemisphere, while positive
(negative) helicity in the northern (southern) hemisphere. Using the imposed field method, they evaluated the magnitude of the turbulent $\alpha$-effect with anisotropic properties
as $\alpha_V/(\tau \mathcal{H}) \sim \mathcal{O}(0.1)$ and $\alpha_H/(\tau \mathcal{H}) \sim \mathcal{O}(0.01)$, where $\mathcal{H} = {\bm \omega}\cdot{\bm u}$ and 
$\meanEMF = \alpha_{\rm H} \overline{\bm B}_{\rm H} + \alpha_{\rm V} \overline{\bm B}_{\rm V}$.
It is interesting to note that these values are about one to two orders
of magnitude smaller than $\alpha \sim \Omega d$, which is the estimation
based on the mixing-length theory.
Additionally, it was also suggested that the magnetic helicity showed
a similar depth variation, but the sign was opposite to that of the
kinetic helicity.

While $\alpha_{\rm H}$ had the expected sign (opposite to that
of the kinetic helicity), $\alpha_{\rm V}$ was found to have the
`wrong' sign (same as that of the kinetic helicity).
Such a result was subsequently also obtained by \cite{Ferriere+93}.
The theoretical possibilities for such effects should be studied further.
For example, \cite{Ruediger2000AA} found that large-scale shear can affect
both the sign of the $\alpha$ effect and kinetic helicity in magnetically
driven compressible turbulence in such a way that they have the same sign,
e.g., for Keplerian accretion disks.
These ideas were also applied to understanding the finding of a
negative $\alpha$ effect in stratified accretion disk simulations
\citep{Brandenbur1998proc}.

\citet{ossendrijver+01} also performed the semi-global simulation with a similar model as B90. They showed that, even in the regime where the condition justifying the
FOSA (or SOCA) is not satisfied, i.e., in the situation where ${\rm St} = u_{\rm rms}\tau /d \gtrsim 1$ and $Re > 1$, the kinetic helicity was clearly separated into positive
and negative values at the lower and upper CZs when taking temporal average of the convective motion over sufficiently long time. Using the imposed field method, they also
measured the magnitude of the turbulent $\alpha$-effect and obtained similar values to B90 in terms of $\alpha_H$ and $\alpha_V$. The rotational dependence of the $\alpha$-effect
was also investigated in this work for the first time. They showed that, in the larger ${\rm Co}$ regime, the $\alpha_V$ underwent a rotational quenching, while the $\alpha_H$
was saturated, where {\rm Co} is the Coriolis number [see \Eq{eq_M8}]. 
The turnover time was defined, in this work, as $\tau = d/u_{\rm rms}$.
While the depth-dependence or rotational dependence
of the $\alpha$, which was obtained from the simulation, agreed, to some extent, with a theoretical model based on the mixing-length theory (R\"udiger \& Kitchatinov 1993), their
amplitudes were one to two orders of magnitude smaller than those predicted from the theoretical model. 
Noteworthy, the critical threshold of the $\alpha$ effect parameter in
mean field dynamo models (see subSection \ref{MFMsubsec}) is about same
magnitude less than the mixing-length models of the solar convection
zone predicts; see \Sec{MeanFieldModels}.

In \citet{kapyla+04,kapyla+06a}, additionally to the rotational dependence, the latitudinal dependence of the turbulent $\alpha$-effect was studied in the semi-global convection
simulations with varying the inclination of the rotation axis with respect to the gravity vector. With the imposed field method, they found that, for slow and moderate rotation
with ${\rm Co} < 4$, the latitudinal dependence of the $\alpha $ followed $\cos\theta $ profile with a peak at the pole \citep[see also,][]{egorov+04}, while, in the rapid rotation
regime with ${\rm Co} \approx 10$, it rather peaked much closer to the equator at $\theta \simeq 30^\circ$. Additionally, the vertical profile of the $\alpha$ directly evaluated from
simulation was found to be qualitatively consistent with analytic expression derived under the FOSA even when changing the latitude. A practical application of these results was 
the development of a kinematic mean-field solar dynamo model in \citet{kapyla+06b}. In it, the rotation profile deduced from the helioseismic observation and the meridional
profiles of the $\alpha$-effect and turbulent pumping obtained with the semi-global simulation of \citet{kapyla+06a} are integrated into the framework of the $\alpha$--$\Omega$
dynamo, and then the solar dynamo mean-field model was constructed. It is interesting that their kinematic dynamo model correctly reproduced many of the general features of the
solar magnetic activity, for example realistic migration patterns and correct phase relation.

The existence of large-scale dynamo, i.e., self-excitation of the mean
magnetic component, in rigidly-rotating convection was demonstrated for
the first time in the semi-global simulation by \citet{kapyla+09}.
By changing the angular velocity, they showed that the large-scale
dynamo could be excited only when the rotation is rapid enough, i.e.,
${\rm Co} \gtrsim 60$, with Eq.~(\ref{eq_M8}) as the definition of
${\rm Co}$ which is same as that used in \citet{ossendrijver+01} and
\citet{kapyla+06a}; see, e.g., \citet{tobias+08} and \citet{cattaneo+06},
and \citet{favier+13}, for unsuccessful large-scale dynamo in
rigidly-rotating convection probably due to slow rotation, and/or short
integration time.
From the measurements of the turbulent $\alpha$-effect and the turbulent
diffusivity by test-field method, they also suggested that while the
magnitude of the $\alpha$-effect stayed approximately constant as a
function of rotation, the turbulent diffusivity decreased monotonically
with increasing the angular velocity, resulting in the excitation of
the large-scale dynamo in the higher ${\rm Co}$.
The reliability of the dynamo coefficients extracted with the test-field
method from the simulation was validated with the one-dimensional
mean-field dynamo model in which the test-field results for $\alpha$
and $\beta$ were used as input parameters by studying the excitation of
the large-scale magnetic field at the linear stage.
Note that the oscillatory properties of the large-scale dynamo
in rigidly-rotating convection and its possible relationship with
$\alpha^2$ dynamo mode with inhomogeneous $\alpha$ profile were also
found in \citet{kapyla+13}; see, e.g., \citet{baryshnikova+87} and \citet{mitra+10} for the oscillatory $\alpha^2$ dynamo.

\subsection{Mean-field dynamo models linked with DNSs}
\subsubsection{Weakly-stratified Model}

Below we review recent mean-field dynamo models linked with semi-global MHD convection simulations, where the large-scale dynamo is successfully operated; see
\citet{masada+14a,masada+14b,masada+16,masada+22} for a series of numerical studies.% with the semi-global models in details.

While \citet{kapyla+09,kapyla+13} were the first to demonstrate that
rigidly-rotating convection can excite the large-scale dynamo as reviewed
above, their simulation model was a so-called ``three-layer polytrope''
consisting of top and bottom stably-stratified layers and the CZ in
between them.
Therefore, it was suspected for a while that the essential factor for
the successful large-scale dynamo observed there might be the presence
of the stably-stratified layer assumed in their model rather than the
rapid rotation \citep[e.g.,][]{favier+13}.
To pin down the key requirement for the large-scale dynamo, the impact of the stably-stratified layers on the
large-scale dynamo was studied in \citet{masada+14a}, hereafter MS14a, in which two-types of semi-global
models with and without stably-stratified layers are compared with the same control parameters and the same grid
spacing. It was found in this study that a large-scale dynamo was successfully operated even in the model without
the stably-stratified layer, and confirmed that the key requirement for it should be a rapid rotation if we
evolved the simulation for a sufficiently long time than the ohmic diffusion time. Note that a relatively weak
density stratification (the density contrast between the top and bottom CZs is about $10$) was assumed in the
simulation model employed in this study as well as \citet{kapyla+09,kapyla+13}.

With these results, \citet{masada+14b}, hereafter MS14b, explored the mechanism of the large-scale dynamo
operated in the rigidly-rotating stratified convection by linking the mean-field (MF) dynamo model with the DNS. In this study, the FOSA based approach was adopted in the MF modeling. The mean
vertical profiles of the kinetic helicity and root-mean square velocity were directly extracted from the
simulation data and then the vertical profiles of the turbulent $\alpha$, turbulent pumping ($\gamma $) and
turbulent diffusivity ($\beta$) were reconstructed according to the analytic expressions of 
\begin{equation}
\alpha (z) = - \tau_c (\overline{u_z\partial_z u_y} + \overline{u_x\partial_y u_z}) \;, \ \ \ \gamma (z) = -\tau_c \partial_z \overline{(u_z)^2} \;, \ \ \ \beta(z) = \tau_c \overline{(u_z)^2} \;, \label{eq_M9}
\end{equation} 
in anisotropic forms of dynamo coefficients under the FOSA \citep[e.g.,][]{kapyla+06a}. 
Although recent numerical studies indicate that the small-scale current helicity, i.e., 
${\bm j} \cdot {\bm b}$, is important for the $\alpha$-effect when the magnetic field is dynamically important \citep[][]{pouquet+76,brandenburg+05b}, its contribution was ignored in this study. 
As the correlation time
$\tau_c$, the convective turnover time defined by $\tau = H_\rho(z)/u_{\rm rms }$ was chosen there ($H_\rho$ is the
density scale-height as a function of the depth).
By solving one-dimensional MF $\alpha^2$ dynamo equation in which these
profiles were used as input parameters, i.e.,
\begin{equation}
  \frac{\partial \overline{\bm B}_h}{\partial t} = \nabla \times ( \meanEMF - \eta \nabla \times \overline{{\bm B}}_h ) \;, \label{eq_M10}
\end{equation}
with 
\begin{equation}
  \meanEMF = \alpha (z)\overline{\bm B}_h + \gamma (z) {\bm e}_z \times \overline{\bm B}_h - \beta (z) \nabla \times \overline{\bm B}_h \;, \label{eq_M11}
\end{equation}
the time-depth diagram for the mean (horizontal) magnetic component
($\overline{\bm B}_h$) was obtained.
Shown in Fig.~\ref{fig_YM2} is $\overline{B_x}(z,t)$ for the MF model
(panels~(b)) and its DNS counterpart (panels~(a)).
Note that, for ensuring the saturation of the magnetic field growth,
the quenching effect was also taken into account.
Since the DNS results were quantitatively reproduced by the MF $\alpha^2$
dynamo, MS14b concluded that the large-scale magnetic field organized
in the rigidly-rotating turbulent convection was a consequence of the
oscillatory $\alpha^2$ dynamo.

\begin{figure}[t]
\includegraphics[width=0.96\textwidth]{f2_YM.pdf}
\caption{Time-depth diagram $\overline{B}_x(z,t)$ for the MF model (panel~(b)) and its DNS counterpart (panel~(a)). 
For DNS result, the horizontal average of the magnetic field is shown. The orange and blue tones represent 
positive and negative $\overline{B}_x$ in units of $B_{\rm cv} \equiv (\overline{\rho {\bm u}^2})^{1/2}$. 
Time is normalized by $\tau_c$. Note that $\overline{B}_y$ shows a similar cyclic behavior with $\overline{B}_x$ yet 
with a phase delay of $\pi/2$; see MS14a,b for details.}
\label{fig_YM2}
\end{figure}

Reproducing the DNS results with mean-field models using coefficients
from the original DNS is an important verification of the whole approach.
This has been done on many occasions in the past; see, for example, the
work by \cite{Gressel10} and \cite{War+21}.

\subsubsection{A strongly stratified model}

In MS14a,b, a weakly-stratified model, in which the density contrast
between top and bottom CZs is about $10$, was adopted. However, the actual Sun has  
a strong stratification with a density contrast of $10^6$ between top and bottom CZs, 
resulting in a large segregation of time scales from minutes to months.
Bearing the application to solar and stellar interiors in mind,
\citet{masada+16}, hereafter MS16, performed a convective dynamo
simulation in a strongly stratified atmosphere with a density contrast
of $700$ in a semi-global setup.
Due to the strong solar- and stellar-like density stratification,
multi-scale convection with a strong up-down asymmetry, i.e.,
slower and broader upflows surrounded by a network of faster and
narrower downflow lanes, was developed in this simulation, as shown
in Fig.~\ref{fig_YM3}(a).
Even in such a situation, the large-scale dynamo was found to operate.
As shown in \Fig{fig_YM3}(b), the mean magnetic field components observed
there showed a time-depth evolution similar to that in the weakly-stratified
model (MS14a,b), indicating that an oscillatory $\alpha^2$ dynamo is
responsible for it.
It was intriguing that, additionally to the mean horizontal component, the
large-scale structures of the vertical magnetic field were spontaneously
organized at the CZ surface in the case of the strongly stratified
atmosphere, as shown in \Fig{fig_YM4}.

A possible physical origin of such surface magnetic structure formation is
the negative magnetic pressure instability (NEMPI; see \S~8 for details).
NEMPI is a mean-field process in the momentum equation, where the Reynolds
and Maxwell stresses attain a component proportional to the square of
the mean magnetic field, which acts effectively like a negative pressure
by suppressing the turbulent pressure.
Since its growth rate becomes larger for stronger density stratification
\citep[e.g.,][]{jabbari+14}, one can imagine that it may play an important
role in organizing sunspot-like large-scale magnetic field structures
in the upper part of the solar CZ.
Although its presence has been confirmed numerically in forced MHD
turbulence \citep[e.g.,][]{brandenburg+11,warnecke+13}, it does not
play a significant role in organizing the surface magnetic structure
seen in MS16 because of their relatively rapid rotation; $\Ro = 0.02$
was assumed there, while, according to \citet{losada+12}, $\Ro \gtrsim 5$ is
required to excite the NEMPI.

The large-scale structure of the vertical magnetic field observed in MS16
is similar to that observed in the large-scale dynamo by forced turbulence
in a strongly stratified atmosphere \citep[][]{mitra+14,jabbari+16}.
This suggests that there may be an as-yet-unknown mechanism for the
self-organization of large-scale magnetic structures, which would be
inherent in a strongly stratified atmosphere.

\begin{figure}[t]
\includegraphics[width=0.96\textwidth]{f3_YM.pdf}
\caption{(a) 3-D view of the strongly stratified convection (for the
progenitor run without rotation).
The black (gray) tone denotes downflows (upflows).
(b) Time-depth diagrams for $\overline{B}_x$ and $\overline{B}_y$.
The normalization is the equipartition field strength,
$B_{\rm eq} \equiv (\overline{\rho {\bm u}^2})^{1/2}$.
In MS16, a one-layer polytrope with a super-adiabaticity of
$\delta \equiv \nabla -\nabla_{\rm ad} = 1.6\times 10^{-3}$ was used; see
MS16 and MS22 for details.}
\label{fig_YM3}
\end{figure}

\begin{figure}[t]
\includegraphics[width=0.96\textwidth]{f4_YM.pdf}
\caption{(a) A snapshot for the horizontal distribution of $B_z$ at the CZ surface. (b) A snapshot for the Fourier filtered
$B_z$ processed from the data shown in panel~(a). Here, the small-scale structures with $k/k_c \gtrsim 8$ are eliminated for
casting light on the large-scale pattern ($k$ is the wavenumber and $k_c = 2\pi/L_h$ with the horizontal box size $L_h$). }
\label{fig_YM4}
\end{figure}

In \citet{masada+22}, hereafter MS22, while varying angular velocity
as a control parameter, they explored the rotational dependence of the
large-scale dynamo in rigidly-rotating convection.
They linked its cause through MF dynamo models with DNSs where a strongly
stratified polytrope was adopted as a model of the convective atmosphere,
as in MS16.

\begin{figure}[t]
\includegraphics[width=0.96\textwidth]{f5_YM.pdf}
\caption{Time-depth diagrams of $\overline{B}_x$ for (A) DNS models and (B) MF models.
(C) Vertical profiles of the turbulent $\alpha$-effect (top)
and turbulent diffusivity $\beta$ (bottom) which are reconstructed with the analytic expressions of \Eq{eq_M9} from the information, such
as kinetic helicity and rms velocity, directly extracted from DNSs.} 
\label{fig_YM5}
\end{figure}

In \Fig{fig_YM5}(a), DNS results are shown where a time-depth diagram
of $\overline{B}_x$ is depicted for models with different values of $\Co$.
While in the slowly rotating model with low $\Co$, the large-scale
magnetic component starts to grow, it gradually stalls as time passes
and finally disappears.
The oscillatory large-scale magnetic field was found to be spontaneously
organized in the rapidly-rotating models with high $\Co$.
It was found from DNS that the large-scale dynamo was excited when
$\Co \gtrsim \Co_{\rm crit}$, where $\Co_{\rm crit}$ is the critical
Coriolis number in the range $25 \lesssim \Co_{\rm crit} \lesssim 80$,
with \Eq{eq_M8} as the definition of the Coriolis number.
It is remarkable that $\Co_{\rm crit}$, which determines the success or
failure of the large-scale dynamo, is almost the same regardless of the
strength of the stratification \citep[see][]{kapyla+09} or the geometry
of the simulation model \citep[see, e.g.,][for $\Co_{\rm crit}$ in the
global simulations]{kapyla+12,warnecke18}; see MS22 for the quantitative
comparison between models.

To explore the underlying physics of the rotational dependence of the
large-scale dynamo, the influence of the rotation on the turbulent dynamo
coefficients was studied with the FOSA-based approach similar to that
of MS14b.
In \Fig{fig_YM4}(c), the vertical profiles of the turbulent $\alpha$
effect and turbulent diffusivity $\beta$ reconstructed with the analytic
expressions of \Eq{eq_M9} were shown.
With increasing the spin rate, the turbulent diffusion weakens while the
$\alpha$ effect remains essentially unchanged over the CZ, providing
an intuition that the rotational dependence of the large-scale dynamo
observed in MS16 and MS22 was mainly due to the change in the magnitude
of the turbulent diffusion.
In fact, this insight was confirmed by the evidence that the MF
dynamo model with incorporating the dynamo coefficients shown in
\Fig{fig_YM4} reproduced quantitatively the result of the DNS; see
\Fig{fig_YM4}(b) for the time-depth diagram of $\overline{B}_x$
obtained in the MF models with utilizing different dynamo coefficients
extracted from the corresponding DNSs with different ${\rm Co}$.
Their conclusion taken from the FOSA-based MF approach was that, with
increasing the angular velocity, the turbulent $\alpha$-effect remains
essentially unchanged over the CZ while the turbulent diffusion weakens,
giving the rotational dependence of the large-scale dynamo, which is
not only the same as the conclusion obtained by \citet{kapyla+09} from
weakly-stratified convective dynamo simulations using the ``test-field
method'', but also the same as that obtained by \citet{shimada+22}
from global solar dynamo simulation with using the ``self-sustained
field method''. Although we don't know whether the independence of the 
$\alpha$-effect on the rotation, seen in these studies, is universal or not, 
it may give an important suggestion not only on the turbulence modeling 
but on the solar dynamo modeling.

\section{Looking forward}

In this review, we have provided some insight into recent developments
in our understanding of the generation of astrophysical large-scale
magnetic fields.
The current development of mean-field theory allows to go beyond some of
the original restrictions that were related to the assumption of large
scale-separation and the inappropriate neglect of nonlinear effects
due to higher order correlations in contributions to the mean turbulent
electromotive force.
A big portion of the progress comes from the development in the DNS
of astrophysical turbulence.
Noteworthy, the classical mean-field theory is based on the fundamental
equations of electrodynamics and has well-known limits.
With the new steps forward, we can take into account results of the DNS,
e.g., the spectral kernels, and treat them as the experimental facts.
The necessity of some phenomenological additions to classical mean-field
theory are motivated both by DNS and observations of the magnetic activity
in astrophysical systems, such as those in our Sun and other stars.
In this way, mean-field models become a valuable tool to understand
the real and virtual worlds of the dynamo in stars and in DNS.

\begin{acknowledgement}
Support through the grant 2019-04234 from the Swedish Research Council (Vetenskapsr{\aa}det) (AB)
is gratefully acknowledged.
We thank for the allocation of computing resources provided by the
Swedish National Infrastructure for Computing (SNIC)
at the PDC Center for High Performance Computing Stockholm and Link\"oping.
VP thanks the financial support of the Ministry of Science
and Higher Education of the Russian Federation (Subsidy No.075-GZ/C3569/278).
\end{acknowledgement}

\bibliographystyle{spbasic}
\bibliography{ref}
\end{document}
