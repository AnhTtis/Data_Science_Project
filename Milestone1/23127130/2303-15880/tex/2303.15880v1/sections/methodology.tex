\section{Methodology}
\label{sec:Methodology}

We jointly learn pose estimation as part of the view synthesis process, using our Gaussian-based renderer, as shown in \cref{fig:Methodology:BlockDiagram}.
From the input image $I_{1}^{*}$, our model encodes two modalities, the three-dimensional pose $P_{1}$ relative to input camera, and the appearance $a_{1}$ of the subject.
The pose $P_{1}$ is transferred to a new viewpoint using camera extrinsic parameters $(R_{1 \rightarrow 2}, t_{1 \rightarrow 2})$.
From the pose $P_{2}$, seen from a novel orientation, we derive the location $\mu_{2}$ and shape $\Sigma_{2}$ of the primitives, which are used, along with their appearance $a_{1}$ and the intrinsic parameters and distortion coefficients of the second camera $(K_{2}, D_{2})$, for the rendering of the subject in a high-dimensional image $J_{2}$.
This feature image $J_{2}$ is enhanced in an image translation module to form the output image $I_{2}$, which closely resembles the target image $I_{2}^{*}$.

\subsection{Extracting Human Pose and Appearance from Images}
\label{sec:Methodology:FeatureExtraction}

We model the human skeleton as a graph $G = (P, E)$ with $N$ vertices and $M$ edges, where $P \in \mathbb{R}^{N \times 3}$ denotes the joint locations and $E = \{(i, j) | i, j \in [1..N]\}$.

\subsubsection{Inferring the Pose}
\label{sec:Methodology:FeatureExtraction:ImageToPose}

We use OpenPose~\cite{cao2017realtime}, an off-the-shelf detector, to infer the 2D pose from the input image $I_{1}^{*} \in \mathbb{R}^{H_{I} \times W_{I} \times 3}$,
\begin{align}
    \label{eq:Methodology:FeatureExtraction:ImageToPose2D}
    I_{1}^{*}
    \rightarrow
    p_{1}, c_{1},
\end{align}
where $p_{1} \in \mathbb{R}^{N \times 2}$ refers to the 2D joints locations and $c_{1} \in \mathbb{R}^{N}$ to their respective confidence values. \\
We use a simple fully-connected network~\cite{martinez2017simple} to infer the 3D pose from the 2D pose,
\begin{align}
    \label{eq:Methodology:FeatureExtraction:Pose2DToPose3D}
    p_{1}, c_{1} \rightarrow \bar{P}_{1},
\end{align}
where $\bar{P}_{1} \in \mathbb{R}^{(N - 1) \times 3}$ refers to the 3D pose relative to the root joint. \\
We compute the root joint location $P_{1, \text{root}}$ in camera coordinates by finding the optimal depth which minimize the error between the re-projected 3D pose $\bar{P}_{1}$ and the 2D pose $p_{1}$ (see \cref{appendix:Depth}).
We obtain the pose in camera coordinates $P_{1} \in \mathbb{R}^{N \times 3}$,
\begin{align}
    P_{1} = [P_{1, \text{root}} | P_{1, \text{root}} + \bar{P}_{1}].
\end{align}

\subsubsection{Estimating the Appearance}
\label{sec:Methodology:FeatureExtraction:ImageToAppearance}

From the input image $I_{1}^{*}$, we use a ResNet-50~\cite{he2016deep} to infer high-dimensional appearance vectors $a_{1}$ used to produce the primitives for rendering,
\begin{align}
    \label{eq:Methodology:FeatureExtraction:ImageToAppearance}
    I_{1}^{*}
    \rightarrow
    a_{1},
\end{align}
where $a_{1} \in \mathbb{R}^{M \times A}$ describe the appearances of each of the edges in the human skeleton as seen from the input view.
These high-dimensional vectors allow the transfer of the subject's appearance from one view to the other without considering pose configuration of the subject, and disentangling pose and appearance.

\begin{figure*}[tbp]
    \centering
    \begin{minipage}{0.5\linewidth}
        \centering
        \includegraphics[width=0.75\linewidth]{images/VisualisationRenders.pdf}
        % \includegraphics[width=\linewidth]{images/VisualisationRenders.png}
        % \includegraphics[width=\linewidth]{images/VisualisationRenders.jpg}
    \end{minipage}%
    \begin{minipage}{0.5\linewidth}
        \centering
        \includegraphics[width=0.75\linewidth]{images/VisualisationImages.pdf}
        % \includegraphics[width=\linewidth]{images/VisualisationImages.png}
        % \includegraphics[width=\linewidth]{images/VisualisationImages.jpg}
    \end{minipage}
    \caption{Visualising the subject from multiple viewpoints. Left: Output of the renderer. Right: Output of the image translation module.}
    \label{fig:Methodology:Visualisation}
\end{figure*}

\subsubsection{Changing the Viewpoint}
\label{sec:Methodology:FeatureExtraction:ViewChange}

To create an image of the world as seen from another viewpoint, we transfer the pose using the extrinsic parameters $(R_{1 \rightarrow 2}, t_{1 \rightarrow 2})$,
\begin{align}
\label{eq:Methodology:FeatureExtraction:ViewChange}
    P_{2} = R_{1 \rightarrow 2} \times P_{1} + t_{1 \rightarrow 2}.
\end{align}

\subsection{Differentiable Rendering of Diffuse Gaussian Primitives}
\label{sec:Methodology:Rendering}
We render a simplified skeletal structure of diffuse primitives, directly obtained from the 3D pose.
The intuition underlying our new renderer is straightforward.
Each primitive can be understood as an anisotropic Gaussian defined by its location $\mu$ and shape $\Sigma$, and the rendering operation is the process of integrating along each ray.
Occlusions are handled by a smooth aggregation step.
Our renderer is differentiable with respect to each input parameter, as the rendering function is itself a composition of differentiable functions.

\subsubsection{From Pose to Primitives}
\label{sec:Methodology:FeatureExtraction:PoseToPrimitives}
From the subject's pose, we compute the location and shape of the primitives,
\begin{align}
\label{eq:Methodology:FeatureExtraction:PoseToPrimitives}
    P_{2}\rightarrow \mu_{2}, \Sigma_{2},
\end{align}
where $\mu_{2} \in \mathbb{R}^{M \times 3}$ refers to the locations, while $\Sigma_{2} \in \mathbb{R}^{M \times 3 \times 3}$ refers to the shapes.
For clarity, we drop the subscripts indicating the viewpoint.

For each edge $(i, j)$, we create a primitive at the midpoint between two joints, with an anisotropic shape aligned with the vector between the two joints,
\begin{align}
\label{eq:Methodology:FeatureExtraction:Primitives}
    \mu_{i j} &= \frac{P_{i} + P_{j}}{2}, \\
    \Sigma_{i j} &= R_{i j} \times \Lambda_{i j} \times R_{i j}^{\intercal},
\end{align}
where,
\begin{align}
\label{eq:Methodology:FeatureExtraction:DetailedPrimitives}
    R_{i j} &= f_{R}(P_{j} - P_{i}, e), \\
    \Lambda_{i j} &= 
    % \begin{pmatrix}
    %     {||P_{j} - P_{i}||}_2 & 0 & 0 \\
    %     0 & w_{i j} & 0 \\
    %     0 & 0 & w_{i j}
    % \end{pmatrix}.
    \text{diag}({||P_{j} - P_{i}||}_2, w_{i j}, w_{i j}).
\end{align}
Here, $f_{R}$ calculates the rotation between two non-zero vectors (see \cref{appendix:Rotation}) and $w_{i j}$ loosely represents the width of the limb.

\subsubsection{Rendering the Primitives}
\label{sec:Methodology:FeatureExtraction:RenderingPrimitives}

Modelling the scene as a collection of diffuse primitives, we render a high-dimensional latent image $J_{2}$,
\begin{align}
\label{eq:Methodology:FeatureExtraction:RenderingPrimitives}
    J_{2} &= \mathcal{R}_{\alpha, \beta}(\mu_{2}, \Sigma_{2}, a_{1}, b, K_{2}, D_{2}) \in \mathbb{R}^{H_{R} \times W_{R} \times A},
\end{align}
where,
$\mathcal{R}$ is the rendering function;
$\alpha > 0$ is a coefficient scaling the magnitude of the shapes of the primitives;
$\beta > 1$ is a background blending coefficient;
$b \in \mathbb{R}^{A}$ describes the appearance of the background;
and $K_{2} \in \mathbb{R}^{3 \times 3}$ and $D_{2} \in \mathbb{R}^{K}$ refers to the intrinsic parameters and distortion coefficients.

To simplify notation, we drop the subscripts indicating the viewpoint, and retain the following letters for subscripts: $i \in [1..H_{R}]$ refers to the height of the image, $j \in [1..W_{R}]$ refers to the width of the image, and $k \in [1..M]$ refers to each of the $M$ primitives.

We define the rays $r_{i j}$ as unit vectors originating from the pinhole, distorted by the lens, and passing through every pixel of the image,
\begin{align}
\label{eq:Methodology:FeatureExtraction:Rays}
    r_{i j} &= \frac{d^{-1}(K^{-1} \times p_{i j}, D)}{{||d^{-1}(K^{-1} \times p_{i j}, D)||}_{2}},
\end{align}
where $d^{-1}$ is a fast fixed-point iterative method~\cite{opencv_library} finding an approximate solution to undistorting the rays, and $p_{i j} = \begin{pmatrix} j & i & 1 \end{pmatrix}^{\intercal}$ is a pixel on the image plane.

We calculate the integral $F_{i j k}$ of the diffusion of a single primitive $(\mu_{k}, \Sigma_{k})$ along a ray $r_{i j}$,
\begin{align}
\label{eq:Methodology:FeatureExtraction:DiffusionIntegral}
    F_{i j k} &= \int_{0}^{+\infty}{e^{-\Delta^{2}(z \cdot r_{i j}, \mu_{k}, \alpha \cdot \Sigma_{k})} dz}.
\end{align}
See \cref{appendix:Rendering_Integral} for the analytical solution.

Objects closer to the camera should occlude those farther from it.
We define a smooth rasterisation coefficient $\lambda_{i j k}$ for each primitive $(\mu_{k}, \Sigma_{k})$, which smoothly favours one primitive, and discounts the others, based on their proximity to the ray $r_{i j}$,
\begin{align}
\label{eq:Methodology:FeatureExtraction:SmoothRasterisation}
    \lambda_{i j k} &=  \frac{1}{1 + {(z_{i j k}^{*})}^{4}},
\end{align}
where,
\begin{align}
    z_{i j k}^{*} &= \argmax_{z} {e^{-\Delta^{2}(z \cdot r_{i j}, \mu_{k}, \alpha \cdot \Sigma_{k})}}.
\end{align}
The optimal depth $z_{i j k}^{*}$, is attained when the squared Mahalanobis distance $\Delta^{2}$ between the point on the ray $z \cdot r_{i j}$ and the primitive $(\mu_{k}, \Sigma_{k})$ is minimal, which in turns maximises the Gaussian density function.
See \cref{appendix:Smooth_Rasterization} for details.

The background is treated as the $(M + 1)^{\text{th}}$ primitive, with unique properties: it is colinear with every ray and located after the furthest primitive, its shape is a constant matrix, and its appearance is also a constant.
Its density $F_{i j M + 1}$ and smooth rasterisation coefficient $\lambda_{i j M + 1}$ are detailed in \cref{appendix:Background_Primitive}.

We derive the weights $\omega_{i j k}$ quantifying the influence of each primitive $(\mu_{k}, \Sigma_{k})$ (including the background) onto each ray $r_{i j}$, such that $\forall k \in [1..{M + 1}]$,
\begin{align}
\label{eq:Methodology:FeatureExtraction:Weights}
    \omega_{i j k} &= \frac{\lambda_{i j k} \cdot F_{i j k}}{\sum\limits_{l=1}^{M + 1}{\lambda_{i j l} \cdot F_{i j l}}}.
\end{align}
Finally, we render the image by combining the weights with their respective appearance,
\begin{align}
\label{eq:Methodology:FeatureExtraction:Render}
    J_{i j} &= \sum\limits_{k=1}^{M + 1}{\omega_{i j k} \cdot a_{k}}.
\end{align}

\subsection{Synthesizing the Output Image}
\label{sec:Methodology:ImageSynthesis}

As shown on the left side of \cref{fig:Methodology:Visualisation}, the intermediate rendered image is feature-based and not photorealistic due to a small number of primitives.
While we could render any image with a sufficient number of primitives, increasing the number not only increases realism, but also the computational cost and the difficulty in optimising the overall problem.
Instead, we render a simplified skeletal structure $(\mu_{2}, \Sigma_{2})$ with a high-dimensional appearance $a_{1}$.

We synthesize the output image of the subject $I_{2} \in \mathbb{R}^{H_{O} \times W_{O} \times 3}$ using the primitive image $J_{2} \in \mathbb{R}^{H_{R} \times W_{R} \times A}$, which is fed to an encoder-decoder network.
Following StyleGAN2~\cite{karras2020analyzing} style mixing and design principles, we use a U-Net encoder-decoder~\cite{ronneberger2015u}.
Where the output resolution is higher than the rendered resolution, the input is upsampled.
To recover high-frequency details, we incorporate the appearance $a_{1}$ as styles at each stage,
\begin{align}
\label{eq:Methodology:ImageSynthesis}
    J_{2}, a_{1} \rightarrow I_{2}.
\end{align}
From the input image $I_{1}^{*}$, we only estimate information about the foreground subject, as seen on the right side of \cref{fig:Methodology:Visualisation}.
As it is impossible to accurately infer a novel view of a static background captured by a static camera, we infer a constant background around the subject, and use a segmentation mask to discard the background information from the groundtruth image $I_{2}^{*}$, as seen on \cref{fig:Experiments:KnownViewpoints}.

\begin{figure}[tbp]
    \centering
    \includegraphics[width=\linewidth]{images/InferReal.pdf}
    % \includegraphics[width=\linewidth]{images/InferReal.png}
    % \includegraphics[width=\linewidth]{images/InferReal.jpg}
    \caption{View synthesis of known subjects in previously unseen poses on Panoptic Studio \textit{(top)} and Human3.6M \textit{(bottom)}. From the input image $(a)$, we extract pose and appearance before rendering the primitives in the novel viewpoint and synthesizing the image $(b)$ which resembles target image $(c)$.}
    \label{fig:Experiments:KnownViewpoints}
\end{figure}


\subsection{Losses}
\label{sec:Methodology:Losses}

\subsubsection{Image Reconstruction}
\label{sec:Methodology:Losses:Reconstruction}

We assess the performance of novel view synthesis by measuring the average pixel-to-pixel distance between the image generated by the model $I_{2}$ and the target image $I_{2}^{*}$ in the pixel space,
\begin{align}
\label{eq:Methodology:Losses:Pixel}
    \mathcal{L}_{I} = \mathbb{E}_{I_{1}^{*}, I_{2}^{*}}\left[{||I_{2}^{*} - I_{2}||}_{1}^{1}\right],
\end{align}
which we complement with either the standard perceptual loss~\cite{johnson2016perceptual}, $\mathcal{L}_{\phi_{\text{VGG}}}$ with a pretrained VGG network~\cite{simonyan2014very}, or the fine-tuned LPIPS loss~\cite{zhang2018unreasonable}, $\mathcal{L}_{\phi_{\text{LPIPS}}}$, to enable the model to synthesize images containing high-frequency detail.

\subsubsection{Adversarial Framework}
\label{sec:Methodology:Losses:Adversarial}

To further enhance the realism of the synthesized images, we fine-tune our novel synthesis model in an adversarial framework,
\begin{align}
\label{eq:Methodology:Losses:Adversarial}
    \mathcal{L}_{A} = \mathbb{E}_{I_{2}^{*}}\left[\log(D(I_{2}^{*}))\right] + \mathbb{E}_{I_{1}^{*}}\left[\log(1 - D(I_{2}))\right].
\end{align}

\subsubsection{Pose Estimation}
\label{sec:Methodology:Losses:Pose}

We require supervision to ensure that the locations of body parts correspond to prespecified keypoints and convey the same semantic meaning,
\begin{align}
\label{eq:Methodology:Losses:Pose}
    \mathcal{L}_{P} = \mathbb{E}_{p_{1}, \bar{P}_{1}^{*}}\left[{c^{*} \cdot ||\bar{P}_{1}^{*} - \bar{P}_{1}||}_{2}^{2}\right],
\end{align}
where $c^{*} \in \mathbb{R}^{N}$ denotes the confidence values of the points.

\subsubsection{Appearance Regularization}
\label{sec:Methodology:Losses:Appearance}

Appearance vectors are an unsupervised intermediate representation. We regularize the squared norm of the appearance vectors,
\begin{align}
\label{eq:Methodology:Losses:Appearance}
    R_{a} = {||a_{1}||}_{2}^{2}.
\end{align}

\subsubsection{Final Objective}
\label{sec:Methodology:Losses:Final}

We obtain our final objective,
\begin{align}
\label{eq:Methodology:Losses:Total}
    \min_{M} \max_{D} \lambda_{I} \mathcal{L}_{I} + \lambda_{\phi} \mathcal{L}_{\phi} + \lambda_{A} \mathcal{L}_{A} + \lambda_{P} \mathcal{L}_{P} + \lambda_{a} R_{a},
\end{align}
with, $\lambda_{I} = \lambda_{\phi} = 10$, $\lambda_{A} = \lambda_{P} = 1$, and $\lambda_{a} = 10^{-3}$.
