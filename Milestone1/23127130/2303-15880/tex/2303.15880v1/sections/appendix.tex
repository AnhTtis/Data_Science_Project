\appendices
\crefalias{section}{appendix}
\crefalias{subsection}{appendix}

\section{Finding the Depth of the Root Joint}
\label{appendix:Depth}
We have $p^{*} \in \mathbb{R}^{J \times 3}$, the 2D pose in ray coordinates, and $\bar{P}^{*} \in \mathbb{R}^{(J - 1) \times 3}$, the 3D pose in the camera view, centered around the root joint. \\
We want to find $Z_{\text{root}}^{*} \in \mathbb{R}^{+}$ to obtain $P \in \mathbb{R}^{J \times 3}$, the 3D pose in the camera coordinate frame,
\begin{align}
    P = [P_{\text{root}} | P_{\text{root}} + \bar{P}]
\end{align}
With,
\begin{align}
    P_{\text{root}} = Z_{\text{root}}^{*} \cdot p_{\text{root}}^{*}
\end{align}
Where,
\begin{align}
    Z_{\text{root}}^{*} = \argmin_{Z_{\text{root}}} L \implies \nabla_{Z_{\text{root}}^{*}}{L} = 0
\end{align}
We develop,
\begin{align}
    L
    =
    \frac{1}{J - 1} \sum_{j}{d_{j}}
    \implies
    \nabla_{Z_{1}}{L}
    =
    \frac{1}{J - 1} \sum_{j}{\nabla_{Z_{1}}{d_{j}}}
\end{align}
And,
\begin{align}
    d_{j}
    =
    {||p_{j} - p_{j}^{*}||}_{2}^{2}
    \implies
    \nabla_{p_{j}}{d_{j}}
    =
    2 \cdot (p_{j} - p_{j}^{*})
\end{align}
And,
\begin{align}
    p_{j}
    =
    \begin{pmatrix}
    x_{j} \\
    y_{j} \\
    1
    \end{pmatrix}
    =
    \begin{pmatrix}
    X_{j} / Z_{j} \\
    Y_{j} / Z_{j} \\
    1
    \end{pmatrix}
    \implies
    \nabla_{P_{j}}{p_{j}}
    =
    \begin{pmatrix}
    \frac{1}{Z_{j}} & 0 & -\frac{X_{j}}{Z_{j}^{2}} \\
    0 & \frac{1}{Z_{j}} & -\frac{Y_{j}}{Z_{j}^{2}} \\
    0 & 0 & 0
    \end{pmatrix}
\end{align}
And,
\begin{align}
    P_{j}
    =
    \begin{pmatrix}
    X_{j} \\
    Y_{j} \\
    Z_{j}
    \end{pmatrix}
    =
    \begin{pmatrix}
    X_{1} + \bar{X}_{j}^{*} \\
    Y_{1} + \bar{Y}_{j}^{*} \\
    Z_{1} + \bar{Z}_{j}^{*}
    \end{pmatrix}
    \implies
    \nabla_{P_{1}}{P_{j}}
    =
    \begin{pmatrix}
    1 & 0 & 0 \\
    0 & 1 & 0 \\
    0 & 0 & 1
    \end{pmatrix}
\end{align}
And,
\begin{align}
    P_{1}
    =
    Z_{1} \cdot p_{1}^{*}
    =
    \begin{pmatrix}
    Z_{1} \cdot x_{1}^{*} \\
    Z_{1} \cdot y_{1}^{*} \\
    Z_{1}
    \end{pmatrix}
    \implies
    \nabla_{Z_{1}}{P_{1}}
    =
    \begin{pmatrix}
        x_{1}^{*} \\
        y_{1}^{*} \\
        1
    \end{pmatrix}
\end{align}
By the chain rule,
\begin{align}
    \nabla_{Z_{1}}{L}
    &=
    \frac{1}{J - 1} \sum_{j = 2}^{J}
    {
    \nabla_{p_{j}}{d_{j}}
    \times
    \nabla_{P_{j}}{p_{j}}
    \times
    \nabla_{P_{1}}{P_{j}}
    \times
    \nabla_{Z_{1}}{P_{1}}
    } \\
    &=
    \frac{1}{J - 1} \sum_{j = 2}^{J}
    {
    \frac{2}{Z_{j}} [(x_{j} - x_{j}^{*})(x_{1}^{*} - x_{j}) + (y_{j} - y_{j}^{*})(y_{1}^{*} - y_{j})]
    } \\
    &= \frac{1}{J - 1} \sum_{j = 2}^{J}
    {
    \frac{2}{Z_{1} + \bar{Z}_{j}^{*}}
    \left[\left(\frac{Z_{1} \cdot x_{1}^{*} + \bar{X}_{j}^{*}}{Z_{1} + \bar{Z}_{j}^{*}} - x_{j}^{*}\right) \cdot \left(x_{1}^{*} - \frac{Z_{1} \cdot x_{1}^{*} + \bar{X}_{j}^{*}}{Z_{1} + \bar{Z}_{j}^{*}}\right)
    + \left(\frac{Z_{1} \cdot y_{1}^{*} + \bar{Y}_{j}^{*}}{Z_{1} + \bar{Z}_{j}^{*}} - y_{j}^{*}\right) \cdot \left(y_{1}^{*} - \frac{Z_{1} \cdot y_{1}^{*} + \bar{Y}_{j}^{*}}{Z_{1} + \bar{Z}_{j}^{*}}\right)\right]
    }
\end{align}
Using a symbolic solver, we obtain,
\begin{align}
    Z_{1}^{*}
    =
    \frac{1}{J - 1} \sum_{j = 2}^{J}
    {
    \frac
    {
    \bar{X}_{j}^{*2} + \bar{Y}_{j}^{*2} +
    [
    (x_{j}^{*} \cdot x_{1}^{*} + y_{j}^{*} \cdot y_{1}^{*}) \cdot \bar{Z}_{j}^{*}
    - (x_{j}^{*} + x_{1}^{*}) \cdot \bar{X}_{j}^{*}
    - (y_{j}^{*} + y_{1}^{*}) \cdot \bar{Y}_{j}^{*}
    ]
    \cdot \bar{Z}_{j}^{*}
    }
    {
    (x_{j}^{*} - x_{1}^{*}) \cdot (\bar{X}_{j}^{*} - x_{1}^{*} \cdot \bar{Z}_{j}^{*})
    + (y_{j}^{*} - y_{1}^{*}) \cdot (\bar{Y}_{j}^{*} - y_{1}^{*} \cdot \bar{Z}_{j}^{*})
    }
    }
\end{align}

\section{Finding the Rotation Between Two Vectors}
\label{appendix:Rotation}
We want to find the rotation matrix between two non-zero vectors, $x, y \in \mathbb{R}^{n} \setminus \{0\}$.
\\
We normalise $x$ to a unit vector $u$,
\begin{align}
    u = \frac{x}{||x||}
\end{align}
We compute the normalised vector rejection $v$ of $y$ on $u$,
\begin{align}
    v = \frac{y - (u^{\intercal} y) u}{||y - (u^{\intercal} y) u||}
\end{align}
We compute the cosinus and sinus of the angle $\theta$ between $x$ and $y$,
\begin{align}
    \cos(\theta)    &=  \frac{x^{\intercal} y}{||x|| \cdot ||y||} \\
    \sin(\theta)    &= \sqrt{1 - \cos^{2}(\theta)}
\end{align}
We compute the projection $Q$ onto the complemented space generated by $x$ and $y$,
\begin{align}
    Q = I - u u^{\intercal} + v v^{\intercal}
\end{align}
Finally, we compute the rotation $R$,
\begin{align}
    R = Q +
    {\begin{bmatrix}
        u \\
        v
    \end{bmatrix}}^{\intercal}
    {\begin{pmatrix}
        \cos(\theta)    &   -\sin(\theta)\\
        \sin(\theta)    &   \cos(\theta)
    \end{pmatrix}}
    {\begin{bmatrix}
        u \\
        v
    \end{bmatrix}}
\end{align}

\section{Differentiable Rendering of Diffuse Primitives}
\label{appendix:DifferentiableRendering}
Modelling a scene as a collection of diffuse primitives, we render a high-dimensional latent image,
\begin{align}
    J &= \mathcal{R}_{\alpha, \beta}(\mu, \Sigma, a, b, K, D) \in \mathbb{R}^{H \times W \times A}
\end{align}
Where,
\begin{itemize}
    \item $\mathcal{R}$ is the rendering function;
    \item $\alpha > 0$ is a coefficient scaling the magnitude of the shapes of the primitives;
    \item $\beta > 1$ is a background blending coefficient;
    \item $\mu = \{\mu_{k} \in \mathbb{R}^{3} | k \in [1..M]\}$, where $\mu_{k}$ refers to the location of the $k^\text{th}$ primitive in camera coordinates;
    \item $\Sigma = \{\Sigma_{k} \in \mathbb{R}^{3 \times 3} | k \in [1..M]\}$, where $\Sigma_{k}$ refers to the ellipsoidal shape of the $k^\text{th}$ primitive, given by its positive definite matrix;
    \item $a = \{a_{k} \in \mathbb{R}^{A} | k \in [1..M]\}$, where $a_{k}$ describes the appearance of the $k^\text{th}$ primitive;
    \item $b \in \mathbb{R}^{A}$, describes the appearance of the background;
    \item $K \in \mathbb{R}^{3 \times 3}$ and $D \in \mathbb{R}^{K}$ refers to the intrinsic parameters and distortion coefficients.
\end{itemize}
To simplify notation, we retain the following letters for subscripts: $i \in [1..H]$ refers to the height of the image, $j \in [1..W]$ refers to the width of the image, and $k \in [1..M]$ refers to each of the $M$ primitives.

We define the rays $r_{i j}$ as unit vectors originating from the pinhole, distorted by the lens, and passing through every pixels of the image,
\begin{align}
    r_{i j} &= \frac{u(K^{-1} \times p_{i j}, D)}{{||u(K^{-1} \times p_{i j}, D)||}_{2}}
\end{align}
Where, $u$ is a fast fixed-point iterative method finding an approximate solution for un-distorting the rays, and $p_{i j} = \begin{pmatrix} j & i & 1 \end{pmatrix}^{\intercal}$ is a pixel on the image plane.

Let $F_{i j k}$ be the density diffused onto a ray $r_{i j}$ by a single primitive $(\mu_{k}, \Sigma_{k})$,
\begin{align}
    F_{i j k} &= \int_{0}^{+\infty}{e^{-\Delta^{2}(z \cdot r_{i j}, \mu_{k}, \alpha \cdot \Sigma_{k})} dz}
\end{align}
See \cref{appendix:Rendering_Integral} for the analytical solution.

For each ray $r_{i j}$, we define a smooth rasterisation coefficient $\lambda_{i j k}$ for each primitive $(\mu_{k}, \Sigma_{k})$.
In a nutshell, this coefficient `smoothly' favours a primitive, and discounts the others, based on their proximity to the ray $r_{i j}$.
See \cref{appendix:Smooth_Rasterization} for details.

The background is treated as the $(M + 1)^{\text{th}}$ primitive, with unique properties.
Its density $F_{i j M + 1}$ and smooth rasterisation coefficient $\lambda_{i j M + 1}$ are detailed in \cref{appendix:Background_Primitive}.

We derive the weights $\omega_{i j k}$ quantifying the influence of each primitive $(\mu_{k}, \Sigma_{k})$ (including the background) onto each ray $r_{i j}$, $\forall k \in [1..{M + 1}]$,
\begin{align}
    \omega_{i j k} &= \frac{\lambda_{i j k} \cdot F_{i j k}}{\sum\limits_{l=1}^{M + 1}{\lambda_{i j l} \cdot F_{i j l}}}
\end{align}
Finally, we render the image by combining the weights with their respective appearance,
\begin{align}
    J_{i j} &= \sum\limits_{k=1}^{M + 1}{\omega_{i j k} \cdot a_{k}}
\end{align}

\subsection{Rendering Integral}
\label{appendix:Rendering_Integral}
We want to measure the density $F_{i j k}$ diffused by a single primitive $(\mu_{k}, \Sigma_{k})$ along a ray $r_{i j}$, by calculating the integral of the diffusion along a ray,
\begin{align}
    F_{i j k} &= \int_{0}^{+\infty}{e^{-\Delta^{2}(z \cdot r_{i j}, \mu_{k}, \alpha \cdot \Sigma_{k})} dz}
\end{align}
To simplify the notation further, we remove the subscript notations. \\
Let,
\begin{align}
    F &= \int_{0}^{+\infty}{e^{-\Delta^{2}(z \cdot r, \mu, \alpha \cdot \Sigma)} dz}
\end{align}
With $\Delta^{2}$, the squared Mahalanobis distance,
\begin{align}
    \Delta^{2}(u, v, A) &= \sbf{\left(u - v\right)}{A}{\left(u - v\right)}
\end{align}
We define $\Sigma'$ to simplify future calculations,
\begin{align}
    \Sigma' &= \alpha \cdot \Sigma
\end{align}
We expand and factorise the quadratic form to isolate $z$,
\begin{align}
    \Delta^{2}(z \cdot r, \mu, \alpha \cdot \Sigma)
    &=
    \Delta^{2}(z \cdot r, \mu, \Sigma') \\
    &=
    \sbf{\left(z \cdot r - \mu\right)}{\Sigma'}{\left(z \cdot r - \mu\right)} \\
    &=
    z^2 \sbf{r}{\Sigma'}{r} - 2 z \sbf{r}{\Sigma'}{\mu} + \sbf{\mu}{\Sigma'}{\mu} \\
    &=
    \sbf{r}{\Sigma'}{r} \left[z^2 - 2 z\frac{\sbf{r}{\Sigma'}{\mu}}{\sbf{r}{\Sigma'}{r}} \right] + \sbf{\mu}{\Sigma'}{\mu} \\
    &=
    \sbf{r}{\Sigma'}{r} \left[z^2 - 2 z\frac{\sbf{r}{\Sigma'}{\mu}}{\sbf{r}{\Sigma'}{r}} + {\left(\frac{\sbf{r}{\Sigma'}{\mu}}{\sbf{r}{\Sigma'}{r}} \right)}^2 - {\left(\frac{\sbf{r}{\Sigma'}{\mu}}{\sbf{r}{\Sigma'}{r}} \right)}^2\right] + \sbf{\mu}{\Sigma'}{\mu} \\
    &=
    \sbf{r}{\Sigma'}{r} \left[ {\left(z - \frac{\sbf{r}{\Sigma'}{\mu}}{\sbf{r}{\Sigma'}{r}} \right)}^2 - {\left(\frac{\sbf{r}{\Sigma'}{\mu}}{\sbf{r}{\Sigma'}{r}} \right)}^2 \right] + \sbf{\mu}{\Sigma'}{\mu} \\
    &=
    \sbf{r}{\Sigma'}{r} {\left(z - \frac{\sbf{r}{\Sigma'}{\mu}}{\sbf{r}{\Sigma'}{r}} \right)}^2 - \frac{{\left(\sbf{r}{\Sigma'}{\mu}\right)}^2}{\sbf{r}{\Sigma'}{r}} + \sbf{\mu}{\Sigma'}{\mu} 
\end{align}
Therefore,
\begin{align}
    F(r, \mu, \alpha \cdot \Sigma)
    &=
    \int_{0}^{+\infty}{e^{-\Delta^{2}(z \cdot r, \mu, \alpha \cdot \Sigma)} dz} \\
    &=
    \int_{0}^{+\infty}{e^{-\Delta^{2}(z \cdot r, \mu, \Sigma')} dz} \\
    &=
    \int_{0}^{+\infty}{e^{- \left[\sbf{r}{\Sigma'}{r} {\left(z - \frac{\sbf{r}{\Sigma'}{\mu}}{\sbf{r}{\Sigma'}{r}} \right)}^2 - \frac{{\left(\sbf{r}{\Sigma'}{\mu}\right)}^2}{\sbf{r}{\Sigma'}{r}} + \sbf{\mu}{\Sigma'}{\mu}\right]} dz} \\
    &=
    e^{- \left[- \frac{{\left(\sbf{r}{\Sigma'}{\mu}\right)}^2}{\sbf{r}{\Sigma'}{r}} + \sbf{\mu}{\Sigma'}{\mu}\right]} \int_{0}^{+\infty}{e^{- \sbf{r}{\Sigma'}{r} {\left(z - \frac{\sbf{r}{\Sigma'}{\mu}}{\sbf{r}{\Sigma'}{r}} \right)}^2} dz}
\end{align}
Let,
\begin{align}
    u &= \sqrt{\sbf{r}{\Sigma'}{r}} {\left(z - \frac{\sbf{r}{\Sigma'}{\mu}}{\sbf{r}{\Sigma'}{r}} \right)}
\end{align}
Therefore,
\begin{align}
    du &= \sqrt{\sbf{r}{\Sigma'}{r}} dz
\end{align}
\begin{align}
    u(0) &= \frac{\sbf{r}{\Sigma'}{\mu}}{\sqrt{\sbf{r}{\Sigma'}{r}}} \\
    \lim_{z \to +\infty} u(z) &= +\infty
\end{align}
By substitution,
\begin{align}
    F(r, \mu, \Sigma')
    &=
    e^{- \left[- \frac{{\left(\sbf{r}{\Sigma'}{\mu}\right)}^2}{\sbf{r}{\Sigma'}{r}} + \sbf{\mu}{\Sigma'}{\mu}\right]} \frac{1}{\sqrt{\sbf{r}{\Sigma'}{r}}} \int_{u(0)}^{+\infty}{e^{- u^{2}} du}
\end{align}
However, we know that,
\begin{align}
    \erfc(x) &= \frac{2}{\sqrt{\pi}} \int_{x}^{+\infty}{e^{- u^{2}} du}
\end{align}
Therefore,
\begin{align}
    F(r, \mu, \Sigma')
    &=
    e^{- \left[\sbf{\mu}{\Sigma'}{\mu} - \frac{{\left(\sbf{r}{\Sigma'}{\mu}\right)}^2}{\sbf{r}{\Sigma'}{r}}\right]} \frac{1}{\sqrt{\sbf{r}{\Sigma'}{r}}} \frac{\sqrt{\pi}}{2} \erfc(u(0)) \\
    &=
    \frac{\sqrt{\pi}}{2} \frac{1}{\sqrt{\sbf{r}{\Sigma'}{r}}} \erfc\left(\frac{\sbf{r}{\Sigma'}{\mu}}{\sqrt{\sbf{r}{\Sigma'}{r}}}\right)  e^{-\left[\sbf{\mu}{\Sigma'}{\mu} - \frac{{\left(\sbf{r}{\Sigma'}{\mu}\right)}^2}{\sbf{r}{\Sigma'}{r}}\right]}
\end{align}
Finally,
\begin{align}
    F(r, \mu, \alpha \cdot \Sigma)
    &=
    \frac{\sqrt{\alpha \pi}}{2 \sqrt{\sbf{r}{\Sigma}{r}}} \erfc\left( \frac{\sbf{r}{\Sigma}{\mu}}{\sqrt{\alpha} \sqrt{\sbf{r}{\Sigma}{r}}}\right)  e^{-\frac{1}{\alpha} \left[\sbf{\mu}{\Sigma}{\mu} - \frac{{\left(\sbf{r}{\Sigma}{\mu}\right)}^2}{\sbf{r}{\Sigma}{r}}\right]}
\end{align}

\subsection{Smooth Rasterization}
\label{appendix:Smooth_Rasterization}

Objects closer to the camera should occlude those farther from it. \\
For each ray $r_{i j}$, we define a smooth rasterisation coefficient $\lambda_{i j k}$ for each primitive $(\mu_{k}, \Sigma_{k})$,
\begin{align}
    \lambda_{i j k} &=  \frac{1}{1 + {(z_{i j k}^{*})}^{4}}
\end{align}
Where,
\begin{align}
    z_{i j k}^{*} &= \argmax_{z} {e^{-\Delta^{2}(z \cdot r_{i j}, \mu_{k}, \alpha \cdot \Sigma_{k})}}
\end{align}
The optimal depth $z_{i j k}^{*}$, is attained when the squared Mahalanobis distance $\Delta^{2}$ between the point on the ray $z \cdot r_{i j}$ and the primitive $(\mu_{k}, \Sigma_{k})$ is minimal, which in turns maximises the Gaussian density function.

Again, to simplify the notation further, we remove the subscript notations. \\
Let,
\begin{align}
    z^{*} &= \argmax_{z} {e^{-\Delta^{2}(z \cdot r, \mu, \alpha \cdot \Sigma)}}
\end{align}
With,
\begin{align}
    \Delta^{2}(z \cdot r, \mu, \alpha \cdot \Sigma)
    &=
    \Delta^{2}(z \cdot r, \mu, \Sigma') \\
    &=
    \sbf{\left(z \cdot r - \mu\right)}{\Sigma'}{\left(z \cdot r - \mu\right)} \\
    &=
    \sbf{r}{\Sigma'}{r} {\left(z - \frac{\sbf{r}{\Sigma'}{\mu}}{\sbf{r}{\Sigma'}{r}} \right)}^2 - \frac{{\left(\sbf{r}{\Sigma'}{\mu}\right)}^2}{\sbf{r}{\Sigma'}{r}} + \sbf{\mu}{\Sigma'}{\mu}
\end{align}
We solve,
\begin{align}
    \frac{\partial e^{-\Delta^{2}}}{\partial z} &= 0
\end{align}
By the chain rule,
\begin{align}
    \frac{\partial e^{-\Delta^{2}}}{\partial z} &= \frac{\partial e^{-\Delta^{2}}}{\partial \Delta^{2}} \frac{\partial \Delta^{2}}{\partial z}
\end{align}
We have,
\begin{align}
    \frac{\partial \Delta^{2}}{\partial z} &= 2 \sbf{r}{\Sigma'}{r} {\left(z - \frac{\sbf{r}{\Sigma'}{\mu}}{\sbf{r}{\Sigma'}{r}} \right)}
\end{align}
And,
\begin{align}
    \frac{\partial e^{-\Delta^{2}}}{\partial \Delta^{2}} &= - e^{-\Delta^{2}} < 0
\end{align}
Therefore,
\begin{align}
    \frac{\partial e^{-\Delta^{2}}}{\partial z} = 0 \iff \frac{\partial \Delta^{2}}{\partial z}   =  0
\end{align}
Which implies,
\begin{align}
    z^{*} &= \frac{\sbf{r}{\Sigma'}{\mu}}{\sbf{r}{\Sigma'}{r}} = \frac{\sbf{r}{\left( \alpha \cdot \Sigma \right)}{\mu}}{\sbf{r}{\left( \alpha \cdot \Sigma \right)}{r}}
\end{align}
Therefore,
\begin{align}
    z^{*} &= \frac{\sbf{r}{\Sigma}{\mu}}{\sbf{r}{\Sigma}{r}}
\end{align}
Finally, we apply the density function to this distance to obtain the smooth rasterization coefficient, in order to model the order in which primitives should appear,
\begin{align}
    \lambda &= \frac{1}{1 + {(z^{*})}^{4}}
\end{align}

\subsection{The Background Primitive}
\label{appendix:Background_Primitive}
We consider the background as an additional primitive, \emph{e.g.} the $(M+1)^\text{th}$, with special properties.
First, it is colinear with every ray and located after the furthest primitive,
\begin{align}
    \mu_{i j {M + 1}} &= z^{*}_{M + 1} \cdot r_{i j}
\end{align}
With,
\begin{align}
    z^{*}_{M + 1} &= \beta \cdot \max_{i j k}{z^{*}_{i j k}}
\end{align}
Where $\beta$ determines how further away the background is assumed to be, we set $\beta = 2$. \\
Second, its shape is a constant, given by the identity matrix,
\begin{align}
    \Sigma_{M + 1} &= \alpha \cdot I
\end{align}
Therefore its density is given by,
\begin{align}
    F_{i j {M + 1}}
    &=
    \int_{0}^{+\infty}{e^{-\Delta^{2}(r_{i j}, \mu_{i j {M + 1}}, \Sigma_{M + 1})} dz}
\end{align}
Which simplifies to,
\begin{align}
    F_{i j {M + 1}}
    &=
    \frac{\sqrt{\alpha \pi}}{2} \erfc\left(\frac{z^{*}_{M + 1}}{\sqrt{\alpha}}\right)
\end{align}
As we can see, its density is not tied to the rays, but to the depth of the primitives only. \\
Third, its rasterization coefficient is given by,
\begin{align}
    \lambda_{M + 1} &= \frac{1}{1 + {(z^{*}_{M + 1})}^{4}}
\end{align}
Last, it has a given constant appearance,
\begin{align}
    a_{M + 1} &= b
\end{align}
