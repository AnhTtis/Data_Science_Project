\documentclass[12pt,a4paper]{article}

%%%%%%%%%%%%%%%%%%%%%%%%%%%%%%%%%%%%%%%%%%%%%%%%%%%%%%%%%%%%%%%%

\usepackage{amsmath, amsthm, amsfonts, amssymb,color,geometry,enumitem}
\usepackage{graphicx}
\usepackage{fancybox}
\usepackage{cases}
\usepackage{fancyhdr}
\usepackage{graphicx}
\usepackage{overpic}

%\renewcommand{\appendixname}{Appendix }




\geometry{margin=1.65cm}
\pagestyle{plain}


\theoremstyle{definition}
\newtheorem{lem}{Lemma}
\newtheorem{thm}[lem]{Theorem}
\newtheorem{prop}[lem]{Proposition}
\newtheorem{cor}[lem]{Corollary}
\newtheorem{conj}[lem]{Conjecture}
\newtheorem{df}[lem]{Definition}
\newtheorem{ex}[lem]{Example}
\newtheorem{rem}[lem]{Remark}



%%%%%%%%%%%%%%%%%%%%%%%%%%%%%%%%%%%%%%%%%%%%%%%%%%%%%%%%%%%%%%%%

\newtheorem{theorem}{Theorem}[section]
\newtheorem{proposition}[theorem]{Proposition}
\newtheorem{lemma}[theorem]{Lemma}
\newtheorem{definition}[theorem]{Definition}
\newtheorem{remark}[theorem]{Remark}
\newtheorem{problem}[theorem]{Problem}
\newtheorem{example}[theorem]{Example}
\newtheorem{conjecture}[theorem]{Conjecture}
\newtheorem{corollary}[theorem]{Corollary}


%%%%%%%%%%%%%%%%%%%%%%%%%%%%%%%%%%%%%%%%%%%%%%%%%%%%%%%%%%%%%%%%

\makeatletter
\renewcommand{\theequation}{%
  \thesection.\arabic{equation}}
\@addtoreset{equation}{section}
\makeatother



%%%%%%%%%%%%%%%%%%%%%%%%%%%%%%%%%%%%%%

\newcommand{\R}{\mathbb{R}}   % Real numbers 
\newcommand{\C}{\mathbb{C}}   % Complex numbers
\newcommand{\N}{\mathbb{N}}   % Natural numbers
\newcommand{\im}{\text{\normalfont Im}}    % Imaginary part
\newcommand{\re}{\text{\normalfont Re}}    % Real part 
\renewcommand{\epsilon}{\varepsilon}    % epsilon 
\newcommand{\ui}{\mathbf{UI}}
\newcommand{\dd}{\mathrm{d}} % for e.g. dx, dt
\newcommand{\Restr}[2]{{#1}{\restriction}_{#2}} % Restriction 
\newcommand\FN{\widetilde{F}^{[N]}}  % Truncated expansion of the reciprocal Cauchy transform



%%%%%%%%%%%%%%%%%%%%%%%%%%%%%%%%%%%%%%%%%%%%%%%%%%%%%%%%%%%%%%%%

\begin{document}



\title{On the free L\'evy measure of the normal distribution}
\author{Takahiro Hasebe and Yuki Ueda}
\maketitle

\begin{abstract} Belinschi et al.\ \cite{BBLS11} proved that the normal distribution is freely infinitely divisible. 
This paper establishes a certain monotonicity, real analyticity and asymptotic behavior of the density of the free L\'evy measure. The monotonicity property strengthens the result in Hasebe et al.\ \cite{HST19} that the normal distribution is freely selfdecomposable. 
\end{abstract}

%%%
{%\large  %for booklet printing

\section{Introduction}\label{sec:intro}
\subsection{Backgrounds}
The role of the normal distribution is played by Wigner's semicircle distribution in free probability. Most notably, the latter appears in the free central limit theorem. Although a role of the normal distribution in free probability is not very obvious, there are still some attempts to understand it.  In \cite{BBLS11} the normal distribution was proven to be freely infinitely divisible, and then, as a stronger result, the normal distribution was proven to be freely selfdecomposable in \cite{HST19}. 
Combinatorial aspects are also investigated in \cite{BBLS11}.  

This paper further analyzes the free infinite divisibility of the normal distribution. Key analytical machineries are the Cauchy transform, its reciprocals and the Voiculescu transform, defined as follows. The \emph{Cauchy transform} of a probability measure $\mu$ on $\R$ is the function
\[
G_\mu(z)=\int_\R \frac{1}{z-x}\mu(\dd x), \qquad z\in \C\setminus \R.
\]
It is easy to see that $G_\mu$ is analytic and maps the complex upper half-plane (denoted $\C^+$) to the lower half-plane (denoted $\C^-$) and also $\C^-$ to $\C^+$. Note that the Cauchy transform is often defined only on $\C^+$ but in this paper the values on $\C^-$ is also useful, see \eqref{eq:CauchyND}. 
Then the \emph{reciprocal Cauchy transform} of $\mu$ is defined to be 
\begin{equation} \label{eq:F}
F_\mu(z)=\frac{1}{G_\mu (z)}, \qquad z\in \C^+, 
\end{equation}
which is an analytic selfmap of $\C^+$. 

According to \cite[Proposition 5.4]{BV93}, for a probability measure $\mu$ on $\R$ and $\lambda>0$ there exist positive numbers $\alpha,\beta$ and $M$ such that $F_\mu$ is univalent on the set $\Gamma_{\alpha,\beta}:=\{z\in \C^+: \im (z)>\beta, |\re(z)|<\alpha \im(z)\}$ and $\Gamma_{\lambda,M}\subset F_\mu(\Gamma_{\alpha,\beta})$. Therefore, the compositional right inverse $F_\mu^{-1}$ is defined on $\Gamma_{\lambda,M}$. The {\it Voiculescu transform} $\varphi_\mu$ is defined by
\[
\varphi_\mu(z):=F_\mu^{-1}(z)-z,\qquad z\in \Gamma_{\lambda,M}.
\]
For probability measures $\mu$ and $\nu$ on $\R$, the {\it free additive convolution} $\mu\boxplus\nu$ is a unique probability measure satisfying that
\[
\varphi_{\mu\boxplus\nu}(z)=\varphi_\mu(z)+\varphi_\nu(z)
\]
on the intersection of the domains where three Voiculescu transforms are defined.

A probability measure $\mu$ on $\R$ is said to be {\it freely infinitely divisible} if for any $n\in \N$ there exists a probability measure $\mu_n$ on $\R$ such that
\[
\mu=\underbrace{\mu_n\boxplus \cdots \boxplus \mu_n}_{n \text{ times}}.
\]
A basic fact is a characterization of freely infinitely divisible distributions  in terms  of the Voiculescu transform. 
\begin{theorem}[{\cite[Theorem 5.10]{BV93}}]
A probability measure $\mu$ on $\R$ is freely infinitely divisible if and only if the Voiculescu transform $\varphi_\mu$ has an analytic extension defined on $\C^+$ with values in $\C^-\cup \R$.
\end{theorem}
For a freely infinitely divisible distribution $\mu$, let $\widetilde{\varphi}_\mu$ denote the analytic extension of its Voiculescu transform as described above. Then the transform $\widetilde{\varphi}_\mu$ has the following Pick--Nevanlinna representation:
\[
\widetilde{\varphi}_\mu(z)=b_\mu+\int_\R \frac{1+xz}{z-x}\tau_\mu(\dd x), \qquad z\in \C^+,
\]
for some $b_\mu\in \R$ and a finite measure $\tau_\mu$ on $\R$. The pair $(b_\mu, \tau_\mu)$ is unique. The measure 
\begin{equation}
\nu_\mu(\dd x) = \frac{1+x^2}{x^2}\mathbf1_{\R\setminus\{0\}}(x) \,\tau_\mu(\dd x)   \label{eq:freeLM}
\end{equation}
is called the \emph{free L\'evy measure} of $\mu$ and the mass of $\tau_\mu$ at zero is called the \emph{semicircular component}. The standard semicircle distribution (i.e.\ with mean 0 and variance 1) corresponds to $(b_\mu,\tau_\mu) = (0,\delta_0)$. 




For many classical distributions including the normal distribution,  its Voiculescu transform cannot be explicitly calculated. In such a case, the following condition has been a useful sufficient (but not necessary, see \cite[Proposition 3.6]{AH13b} for a counterexample) condition for proving the free infinite divisibility. 
\begin{definition}[{\cite[Definition 5.1]{AH13}}]  \label{def:UI}
A probability measure $\mu$ on $\R$ is said to be in class ${\bf UI}$ (denoted by $\mu\in {\bf UI}$) if $F_\mu^{-1}$, defined in some $\Gamma_{\lambda,M}$, analytically extends to a univalent map in $\C^+$, or  equivalently, if there exists a domain $ \C^+ \subset \Omega_\mu \subset \C$ such that $F_\mu$ extends to an analytic bijection from $\Omega_\mu$ onto $\C^+$.
\end{definition}

\begin{proposition}[{\cite[Proposition 5.2]{AH13}}]
If $\mu \in {\bf UI}$, then $\mu$ is freely infinitely divisible.
\end{proposition}

Many classical distributions, despite their unexplicit Voiculescu transforms,  have been proven to be in class {\bf UI}. They include the normal distribution \cite{BBLS11}, some beta distributions and some gamma distributions \cite{Has14} and some HCM distributions \cite{Has16}; see e.g.\ \cite{BBLS11,AH16,BH13,Has14,Has16,MU20} for further examples. On the other hand, little is known about free L\'evy measures of these distributions; most of the former results in the literature were limited to the existence of $\Omega_\mu$. For the normal distribution, it is nonetheless shown in \cite{HST19} that the free L\'evy measure of $N(0,1)$ is of the form
\begin{equation} \label{eq:FSD}
\frac{k(x)}{|x|} \mathbf1_{\R\setminus\{0\}}(x)\,\dd x, 
\end{equation}
where $k$ is nondecreasing on $(0,\infty)$ and is non-increasing on $(-\infty,0)$, i.e., $N(0,1)$ is freely selfdecomposable. The fact that $N(0,1)$ is symmetric also implies that $k$ is an even function. The aim of this paper is to clarify further properties of the function $k$. 




\subsection{Main results and the outline of proofs}\label{sec:main}

The main result of this paper is the following two theorems on the free L\'evy measure of $N(0,1)$. 

\begin{theorem}[Analyticity and monotonicity]\label{thm:freeLevy}
The free L\'{e}vy measure of $N(0,1)$ is of the form 
\begin{equation} \label{eq:FLM_normal}
\frac{1}{\pi x^{2}} h(|x|) \mathbf{1}_{\R\setminus\{0\}}(x)\,\dd x,  
\end{equation}
 where $h\colon (0,\infty) \to (0,\infty)$ is a real analytic function with $h' <0$ on $(0,\infty)$. 
\end{theorem}

This result readily reproduces the known fact \eqref{eq:FSD} because $h'<0$ implies $k'<0$, where $k(x) := x^{-1} h(x)/\pi$.  
An interesting analogous fact is that the Boolean L\'evy measure of $N(0,1)$ is also of the form $x^{-2} \tilde{h}(|x|)\,\dd x$, where $\tilde{h}$ is real analytic with negative derivative on $(0,\infty)$, see \cite[Proposition 4.2]{HNSU}. 

We then clarify the asymptotic behavior of the function $h$ of the normal distribution at zero and at infinity. 

\begin{theorem}[Asymptotic behaviors of $h$] \label{thm:fine_asymptotics_main} The function $h$ in Theorem \ref{thm:freeLevy} fullfills: 
\begin{enumerate}[label=\rm(h${}_\infty$),leftmargin=1.2cm]

\item\label{item:h(x)_infty_main} $\displaystyle h (x) = \frac{1}{e} \sqrt{\frac{\pi}{2}}x^2 e^{-\frac{x^2}{2} } (1 + O(x^{-2}))$\quad  as \quad $x\to\infty$; 
\end{enumerate}
\begin{enumerate}[label=\rm(h${}_0$),leftmargin=1.2cm]
\item \label{item:h(x)_zero_main} $\displaystyle 
h(x) = \sqrt{\log \frac1{\sqrt{2\pi}\, x} + \sqrt{ \left(\log \frac1{\sqrt{2\pi}\, x}\right)^2+ \frac{\pi^2}{4} }} + O(x^\eta)  \quad \text{as} \quad x\to0^+ \quad \text{for any } 0<\eta <1.
$
\end{enumerate}

\end{theorem}
\begin{remark}
Moreover, \ref{item:h(x)_infty_main} can be enhanced to the asymptotic expansion
\[
\displaystyle h (x) = \frac{1}{e} \sqrt{\frac{\pi}{2}}x^2 e^{-\frac{x^2}{2} } \left(1 - \frac{5}{2x^2}  -\frac{43}{8x^4} - \frac{579}{16 x^6}  - \cdots\right), \qquad x\to\infty,    
\]
 see Remark \ref{rem:asymptotic} for further details. 
\end{remark}

Proofs of the above two theorems are sketched here. Theorem \ref{thm:freeLevy} is proved in Section \ref{sec:freeLevy} but a large part is done in Section \ref{sec:cauchy}. We introduce the crucial supplementary domain 
\[
\Xi =  \left\{x+ iy: x \ne0, y >-\frac{\pi}{2|x|} \right\} \cup i\R \subset \C,    
\]
which is the union of the sets $\C^+ $, $\R$ and a connected component of 
\[
\left\{ z \in \C^-: \im\left[ -i e^{-\frac{z^2}{2}} \right]<0 \right\}.
\] 
The last set naturally arises from formula \eqref{eq:CauchyND} for $\widetilde G$, the entire analytic continuation of the Cauchy transform $G_{N(0,1)}$. 
We are able to show that $\widetilde F:= 1/ \widetilde G$ is analytic in $\Xi$ (Lemma \ref{cor:F}) and there exists a simply connected domain $\Omega \subseteq \Xi$ such that $\widetilde F$ restricted to its closure $\text{cl}(\Omega)$ is a homeomorphism onto $(\C^+ \cup \R) \setminus\{0\}$ (Theorem \ref{thm:omega}). In the construction of $\Omega $ we first identify its boundary set $\partial\Omega$ with the preimage of $\R \setminus \{0\}$ by the map $\Restr{\widetilde F}{\Xi}$ (Proposition \ref{prop:curve}). (If we go outside of $\Xi$, then the preimage of $\R \setminus \{0\}$ seems to have irrelevant connected components, see Figure \ref{fig1} below.)   This argument also implies the analyticity of $\partial\Omega$.  

The function $h$ is then defined as $h(x) = -\im[(\Restr{\widetilde F}{\text{cl}(\Omega)})^{-1}(x)], x>0$, which describes the height (from the real line) of the boundary curve  $\partial\Omega$. 
According to \cite[(8.1.6)]{Kerov} or \cite[(3.5)]{BBLS11}, the following ODE is satisfied by $F_{N(0,1)}$:
\begin{align}\label{eq:ODE}
F_{N(0,1)}'(z)= F_{N(0,1)}(z)(z- F_{N(0,1)}(z)), \qquad z\in \C^+. 
\end{align} 
By analytic continuation, this formula holds for $\widetilde F$ on $\Xi$ too. Going to the inverse map, a system of ODEs for $h(x)$ (and for $g(x):=\re[(\Restr{\widetilde F}{\text{cl}(\Omega)})^{-1}(x)]$) can be deduced.
The monotonicity of $h$ in Theorem \ref{thm:freeLevy} is an easy consequence of these ODEs (Proposition \ref{prop:gh}).  Formula \eqref{eq:FLM_normal} easily follows from the Stieltjes inversion (Section \ref{sec:freeLevy}). 

Considering the above, investigating the curve $\partial \Omega$ in further details will reveal fine properties of the free L\'evy measure, which results in Theorem \ref{thm:fine_asymptotics_main}. The two asymptotics \ref{item:h(x)_infty_main} and \ref{item:h(x)_zero_main}  will be separately proved in expanded forms (providing finer descriptions  of $\partial \Omega$) as Theorems \ref{thm:fine_asymptotics} and \ref{thm:main_infty}, respectively. 
The proof  does not use the ODE by contrast to Theorem \ref{thm:freeLevy}. 

The method for proving \ref{item:h(x)_infty_main} is based on asymptotic analyses of the reciprocal Cauchy transform and of its inverse function at infinity on a region $|\arg z| <\epsilon$. As a basis, the Laurent series asymptotic expansions of $\widetilde F$ and its inverse are  obtained in Lemmas \ref{lem:asymptotic_Cauchy} and \ref{lem:F^{-1}(z)_infty}, respectively.  In addition, we need to estimate an exponential decay of $\im[\widetilde F]$, which is invisible in the Laurent series expansion (Proof of Theorem \ref{thm:fine_asymptotics}). This decay is inherited from the tail behavior of the probability density function.  The whole method seems to be applicable to a wider class of freely infinitely distributions with unbounded support. 

The proof of \ref{item:h(x)_zero_main} is based on formula \eqref{eq:CauchyND} for $\widetilde G$. As the curve $\partial \Omega$ goes to infinity as it approaches the negative imaginary axis (cf.~Figure~\ref{fig1}), the contribution of $G(z)$ in formula  \eqref{eq:CauchyND} is negligible because of its order $O(1/z)$ on $\partial \Omega$ (and because the remaining term $-  i \sqrt{2\pi} e^{-\frac{z^2}{2}}$ goes to infinity).  The point $(\Restr{\widetilde F}{\text{cl}(\Omega)})^{-1}(x)$ is given as a unique solution $z\in \Xi$ to the equation $\widetilde F(z)=x$ that reads, for small $x>0$, $-  i \sqrt{2\pi} e^{-\frac{z^2}{2}} +O(1/z) =1/x$. A detailed analysis of this equation yields the asymptotic behavior of $h(x)= - \im(z) =-\im[(\Restr{\widetilde F}{\text{cl}(\Omega)})^{-1}(x)]$ claimed in \ref{item:h(x)_zero_main}. A key observation in this analysis is that $\partial \Omega$ can be well approximated by the curve $\partial\Xi$ near the imaginary axis, cf.~Figure~\ref{fig1}.







\section{The Cauchy transform of the normal distribution}\label{sec:cauchy}


In this section, we analyze the analytic continuation of the Cauchy transform and its reciprocal, and then describe the boundary of $\Omega_{N(0,1)}$ as a solution to a system of differential equations. 

\subsection{Entire analytic continuation of the Cauchy transform}

We simplify the notation of the Cauchy transform of the normal distribution into
\begin{equation}\label{eq:cauchy}
G (z):= G_{N(0,1)}(z) =  \int_{-\infty}^\infty \frac{1}{z-x} \cdot \frac{1}{\sqrt{2\pi}} e^{-\frac{x^2}{2}}\dd x, \qquad z \in \C\setminus\R. 
\end{equation}
A well known fact is that the Cauchy transform $\Restr{G}{\C^+}$ has an analytic continuation to $\C$ (denoted by $\widetilde{G}$) and, on the lower half-plane, the formula 
\begin{align}\label{eq:CauchyND}
\widetilde{G}(z)=G(z)- 2\pi i \cdot \frac{1}{\sqrt{2\pi}} e^{-\frac{z^2}{2}}, \qquad z\in \C^-
\end{align}
holds, see e.g.\ \cite[Theorem 1.2]{Gre60}. On the other hands, due to \cite[p.~362]{Gre60} and the identity theorem, we have
\begin{align}\label{eq:CauchyND2}
\widetilde{G}(z)=e^{-\frac{1}{2}z^2}\left[-i\sqrt{\frac{\pi}{2}}+\sqrt{2} \int_0^{z/\sqrt{2}} e^{t^2}\dd t \right], \qquad z\in \C.
\end{align}
In particular, we have
\begin{equation} \label{eq:Stieltjes}
\im[\widetilde{G}(x)] = - \sqrt{\frac{\pi}{2}}e^{-\frac{x^2}{2}} <0,  \qquad x\in \R,  
\end{equation}
which can also be deduced from the Stieltjes inversion formula. 


Obviously, the reciprocal Cauchy transform $F_{N(0,1)} = 1/G_{N(0,1)}$ analytically extends to the meromorphic function on $\C$
\[
\widetilde{F}(z) := \frac{1}{\widetilde{G}(z)}. 
\]
In view of \eqref{eq:F} and \eqref{eq:Stieltjes}, poles of $\widetilde{F}$ do not exist in $\C^+ \cup \R$. It seems that $\widetilde{F}$ has poles on $\C^-$, see Figure \ref{fig3} below; however, we mostly work on $\widetilde{F}$ in subdomains where $\widetilde{F}$ turns out to have no poles, so that analysis of poles will be rather out of scope of this paper.    




\subsection{Behavior of the Cauchy transform on extended domains}

As preparatory steps, we investigate the behavior of $\widetilde G$ and $\widetilde F$ on the imaginary axis (Lemma \ref{lem:F(ix)}), asymptotic behavior as  $z\to \infty,  - (1/4)\pi + \epsilon < \arg z < (5/4)\pi - \epsilon$ (Lemma \ref{lem:asymptotic_Cauchy} and Lemma \ref{lem:F-transform}) and then behavior on the crucial domain $\Xi$ defined in Subsection \ref{sec:main} (Lemma \ref{cor:F}). 

The next fact corresponds to the exceptional case $c=0$ excluded in \cite[Lemma 3.6]{BBLS11}. 
\begin{lemma}\label{lem:F(ix)}
\begin{enumerate}[label=(\arabic*),leftmargin=1cm]
\item $\displaystyle \lim_{\substack{x\in \R\\ x\rightarrow\infty}} \widetilde{F}(ix)=i\infty$\quad and \quad$\displaystyle \lim_{\substack{x\in \R\\ x\rightarrow-\infty}} \widetilde{F}(ix)=i0$.
\item $\widetilde{F}$ is a bijection from $i\R$ onto $i(0,\infty)$.
\end{enumerate}
\end{lemma}


\begin{proof}
Obviously it suffices to verify the equivalent assertions
\begin{enumerate}[label=(\roman*),leftmargin=1.2cm]
\item\label{item:G1} $\displaystyle \lim_{\substack{x\in \R\\ x\rightarrow\infty}} \widetilde{G}(ix)=i0$\quad and \quad$\displaystyle \lim_{\substack{x\in \R\\ x\rightarrow-\infty}} \widetilde{G}(ix)=-i\infty$; 
\item\label{item:G2} $\widetilde{G}$ is a bijection from $i\R$ onto $i(-\infty,0)$.
\end{enumerate}

\noindent
{\bf Proof of \ref{item:G1} and \ref{item:G2}.}
\begin{enumerate}[label=(\roman*),leftmargin=0.8cm]

\item The first limit is clear from \eqref{eq:cauchy}. By \eqref{eq:CauchyND}, we have
\begin{align}\label{eq:G(ix)}
\widetilde{G}(ix)=G(ix)- 2\pi i \cdot \frac{1}{\sqrt{2\pi}} e^{\frac{x^2}{2}} \rightarrow -i\infty,  \qquad x\rightarrow -\infty.
\end{align}

\item By \eqref{eq:CauchyND2}, for $x\in \R$, we get
\begin{align*} 
\rho(x):= i\widetilde{G}(ix)=\sqrt{\frac{\pi}{2}} e^{\frac{x^2}{2}} \left[1- \frac{2}{\sqrt{\pi}} \int_0^{x/\sqrt{2}} e^{-t^2} \dd t \right] \in (0,\infty),
\end{align*}
 It follows from (1) that $\lim_{x\rightarrow\infty} \rho(x)=0$ and $\lim_{x\rightarrow-\infty} \rho(x)=\infty$, and  hence $\widetilde{G}$ maps $i\R$ onto $i(-\infty,0)$.  It remains to establish the monotonicity of $\rho$. Since
\begin{align*}
\rho'(x)=\sqrt{2}x e^{\frac{x^2}{2}} \int_{x/\sqrt{2}}^\infty e^{-t^2}\,\dd t - 1,
\end{align*}
it obviously follows that $\rho'(x)<0$ for all $x\le 0$. Some calculus also shows that $\rho'(x)<0$ for all $x>0$. Consequently, $\rho$ is strictly decreasing on $\R$, and therefore $\widetilde{G}$ is a bijection from $i\R$ onto $i(-\infty,0)$. 
\end{enumerate}
\vspace{-7mm}
\end{proof}


The Cauchy transform is well known to have an asymptotic expansion as $z\to \infty,$ $ \epsilon < \arg z < \pi -\epsilon$ for any fixed $\epsilon \in (0,\pi)$, see e.g.\ \cite[Theorem 3.2.1]{Akh65}. For the normal distribution, we can see that the asymptotic expansion holds in the larger domain 
\begin{align*}
D_{\epsilon}:=\left\{z\in \C\setminus\{0\}: \arg(z) \in \left(-\frac{\pi}{4}+\epsilon, \ \frac{5}{4}\pi - \epsilon \right) \right\},    
\end{align*}
with $\epsilon\in(0,\pi/4)$ arbitrary but fixed. To state the formula, we denote by $\{m_k\}_{k\ge0}$ the moment sequence of $N(0,1)$, i.e.\ 
\[
m_0=1; \qquad m_n= \begin{cases} (n-1)!! & \text{if~$n$ is even}, \\ 0& \text{if $n$ is odd}
\end{cases}
 \quad(n \in \N).
\] 



\begin{lemma}\label{lem:asymptotic_Cauchy}
For any fixed $N \in \N$ and fixed $\epsilon \in (0, \frac{\pi}{4})$, the asymptotic expansions
\begin{align}
\widetilde{G}(z) &=  \sum_{n=0}^{N-1} \frac{m_{2n}}{z^{2n+1}} + O(z^{-2N-1})   \quad \text{and}   \label{eq:asymptotic_Cauchy1} \\
\widetilde{G}' (z) &=  - \sum_{n=0}^{N-1} \frac{(2n+1)m_{2n}}{z^{2n+2}} + O(z^{-2N-2})  \label{eq:asymptotic_Cauchy2}
\end{align}
hold as $z\rightarrow\infty$ with $z\in D_{\epsilon}$. 
\end{lemma}
\begin{proof}
For $0<\epsilon <\pi/4$ we first observe that
\begin{align}\label{eq:another_rep}
G (z)=\int_{\partial D_\epsilon} \frac{1}{z-w}\frac{1}{\sqrt{2\pi}} e^{-\frac{w^2}{2}}\,\dd w, \qquad z\in \C^+. 
\end{align}
This is an easy consequence of Cauchy's integral formula and the fact that the contour integral over the arc $\{Re^{i\theta}: -\frac{\pi}{4}+\epsilon \le \theta \le 0\}$ converges to zero as $R \to \infty$. The remaining arguments are analogous to the standard one for $ \epsilon < \arg z < \pi -\epsilon$, see e.g.\ the proof of \cite[Theorem 3.2.1]{Akh65}. For the reader's convenience the rest of the proof is included in Appendix \ref{appendix}. 
\end{proof}


By Lemma \ref{lem:asymptotic_Cauchy}, the analytic extension $\widetilde{F}$ has no poles on $D_{\epsilon,R}:= D_\epsilon \cap\{z: |z|>R\}$ for sufficiently large $R>0$ and 
\begin{align}\label{eq:F_asympt}
\widetilde{F}(z)=z(1+o(1)), \qquad z\rightarrow\infty, \ z\in D_{\epsilon}.
\end{align}







The next two lemmas are basic ingredients to construct and analyze a compositional inverse function of $\widetilde{F}$. 
\begin{lemma}\label{lem:F-transform}
For any $0< \epsilon< \epsilon' < \pi/4$   there exist $R>0$ such that $\widetilde{F}$ is univalent on $D_{\epsilon,R}$ and $D_{\epsilon',R'}\subset\widetilde{F}(D_{\epsilon,R})$, where $R' := [1+\sin(\epsilon' - \epsilon)]R$. 
\end{lemma}

\begin{proof}
\textbf{Part 1: $D_{\epsilon',R'}\subset\widetilde{F}(D_{\epsilon,R})$.}
Due to \eqref{eq:F_asympt}, there exists an $R>0$ such that 
\[
|\widetilde{F}(z) -z | < \sin (\epsilon'-\epsilon) |z|, \qquad z\in \partial D_{\epsilon, R}. 
\]



Let $d(z,D_{\epsilon',R'})$ stand for the distance between $z\in \partial D_{\epsilon,R}$ and the domain $D_{\epsilon',R'}$. It is elementary to verify that 
\begin{align}\label{eq:distance}
d(z, D_{\epsilon',R'})\ge \sin(\epsilon'-\epsilon) |z|, \qquad z\in \partial D_{\epsilon,R}, 
\end{align}
which implies that the curve $\widetilde{F}(\partial D_{\epsilon,R})$ does not intersect with $D_{\epsilon',R'}$ and every point of $D_{\epsilon',R'}$ has rotation number 1 with respect to this curve (viewed as a closed curve in the Riemann sphere), and hence $D_{\epsilon',R'}\subset\widetilde{F}(D_{\epsilon,R})$.  


\vspace{2mm}
\noindent
\textbf{Part 2: Univalence of $\widetilde{F}$.}   In order to resort to the Noshiro--Warschawski criterion (see e.g. \cite[Proposition 1.10]{Pom92} or the original articles \cite{No34,Wa35}), we estimate the derivative $\widetilde{F}'$ on $D_{\epsilon,R}$. 
Take $0< \eta <\epsilon< \pi/4$. By Lemma \ref{lem:asymptotic_Cauchy}, we have
\[
\widetilde{F}'(z)=-\frac{\widetilde{G}'(z)}{\widetilde{G}(z)^2} \sim 1 \quad \text{as} \quad z\rightarrow\infty \quad\text{with}\quad z\in D_\epsilon, 
\]
and therefore, we can take $R_0>0$ large enough so that $\re[\widetilde{F}']\ge 1/2$ on $D_{\epsilon,R_0}$. 

Because $D_{\epsilon,R_0}$ is not convex we introduce supplementary convex domains. 
Let $\ell$ be the half-line starting from the point $4R_0 e^{i(-\frac{\pi}{4} +\epsilon)}$, passing $4R_0 i$ and going to infinity. 
Let $U$ be the domain that has the boundary $\ell \cup \{ re^{i(-\frac{\pi}{4} +\epsilon)}: r\ge 4R_0\}$ and contains the point $4R_0(1+i)$.  Let $V$ be the reflection of $U$ with respect to the imaginary axis. Since $U$ and $V$ are convex domains contained in $D_{\epsilon,R_0}$, the Noshiro--Warschawski criterion implies that $\widetilde{F}$ is univalent in $U$ and $V$. Choosing $R=8R_0$ and using the fact that $\widetilde{F}$ is close to the identity map, i.e.\ $\widetilde{F}(z) = z(1 +o(1))$, we can conclude that $\widetilde{F}$ is univalent in $D_{\epsilon,R}$ for large $R_0>0$. 
\end{proof}




\begin{lemma}\label{cor:F}
Let $\Xi$ be the domain of $\C$ defined in Section \ref{sec:main}, i.e.\ the domain of $\C$ with boundary 
\[
\partial \Xi=C_{\pi} \cup C_{-\pi},
\]
where $C_{\pm \pi}:=\{re^{i\theta} : r>0, \ -\pi <\theta<0, \ r^2\sin 2\theta=\pm\pi\}$.  Then the function $\widetilde{G}$ has no zeros in $\Xi \cup \partial \Xi$, and therefore $\widetilde{F}$ is holomorphic in $\Xi \cup \partial \Xi$. Moreover, the function $\Restr{\widetilde{F}}{\Xi \cup \partial \Xi}$ satisfies 
\begin{enumerate}[label=(\arabic*),leftmargin=1.2cm]
\item\label{item:F1} $\im [\widetilde{F}(z)]<0$ for all $z\in \partial \Xi$,
\item $\re [\widetilde{F}(z)]>0$ for all $z\in \Xi \cap \{z: \re(z)>0\}$, 
\item\label{item:F3} $\displaystyle\lim_{\substack{\im(z) \to- \infty \\ z\in \Xi}}\widetilde{F}(z) =0$.
\end{enumerate}
\end{lemma}

\begin{proof} 
It suffices to establish
\begin{enumerate}[label=(\roman*),leftmargin=1.2cm]
\item\label{item:GG1} $\im[\widetilde{G}(z)]>0$ for all $z\in \partial \Xi$, 
\item\label{item:GG2} $\re [\widetilde{G}(z)]>0$ for all $z\in \Xi\cap \{z: \re(z)>0\}$, 
\item\label{item:GG3} $\displaystyle \lim_{\substack{\im(z) \to- \infty \\ z\in \Xi}}\widetilde{G}(z) =\infty$.   
\end{enumerate}
Note that \ref{item:GG1} and \ref{item:GG2} together with Lemma \ref{lem:F(ix)} (and the fact that $\im[\widetilde{G}]<0$ on $\R$) imply that $\widetilde G$ has no zeros in $\Xi \cup \partial \Xi$. 


\vspace{3mm}
\noindent
{\bf Proof of \ref{item:GG1}--\ref{item:GG3}.}  The proofs are based on separate analyses of the two terms of formula \eqref{eq:CauchyND}. 
\begin{enumerate}[label=(\roman*),leftmargin=0.8cm]
\item If $z=re^{i\theta}\in \partial \Xi$, that is, $r^2\sin 2\theta=\pm \pi$ and $-\pi <\theta<0$, then 
\[
e^{-\frac{1}{2}z^2}= \mp ie^{-\frac{r^2}{2}\cos 2\theta},
\]
respectively. Hence for such $z$, we get
\[
\widetilde{G}(z)=G(z)- 2\pi i \cdot \frac{1}{\sqrt{2\pi}} \cdot \left(\mp ie^{-\frac{r^2}{2}\cos 2\theta}\right) \in \C^+.
\]

\item For $z \in \C^+\cap \{z: \re(z)>0\}$, according to \cite[Lemma 3.1]{BH13}, we have $\re[\widetilde{G}(z)] >0$ .  For $z=re^{i\theta} \in \Xi \cap \C^- \cap \{z: \re(z)>0\}$ we have $-\pi<r^2\sin 2\theta<0$ and hence the number
\[
e^{-\frac{z^2}{2}}= e^{-\frac{r^2}{2}\cos 2\theta} e^{-i \cdot \frac{r^2}{2}\sin 2\theta}
\]
has positive imaginary part.  Again by  \cite[Lemma 3.1]{BH13} $G(z)$ also has positive real part, so that  we conclude $\re[\widetilde{G}(z)]>0$. For $z= x \in (0,\infty)$,  the continuity of $G$ implies $\re[G(x -i0)]\ge0$ $(x>0)$, so the conclusion still holds. 

\item As $\im(z) \to- \infty, z\in \Xi$, the argument of $z$ tends to $-\pi/2$, so that $e^{-\frac{z^2}{2}}$ tends to $\infty$. On the other hand, $G(z)$ tends to zero. We are done.  
\end{enumerate}
\vspace{-7mm}
\end{proof}



\subsection{Describing the boundary of $\Omega_{N(0,1)}$} \label{subsec:boundary}



Using results in the previous subsection, we describe the boundary of $\Omega_{N(0,1)}$. 
\begin{proposition} \label{prop:curve}
For every $x>0$ there exists a unique point $H(x)$ in $\Xi \cap \C^- \cap \{z: \re(z)>0\}$ such that $\widetilde F(H(x))=x$. The curve $p_0^+ =\{H(x): x>0\}$ is real analytic and is mapped by $\widetilde{F}$ bijectively onto $(0,\infty)$. 
By symmetry, the curve $p_0^-$ which is the reflection of $p_0^+$ with respect to the imaginary axis is mapped by $\widetilde{F}$ bijectively onto $(-\infty, 0)$. 
\end{proposition}
\begin{proof}
Let $c_R$ be the simple closed curve in the Riemann sphere consisting of $i[-\infty, 0], \{z \in \partial\Xi: 0< \re(z) < R \}, \{R +i y: -\frac{\pi}{2R} \le y \le 0\}$ and $[0, R]$. 
We take $R$ sufficiently large so that $|\widetilde{F}(z) -z| < \frac1{2}|z|$ for $|z| \ge R, z \in c_R$. By Lemma \ref{lem:F(ix)},  Lemma \ref{cor:F} \ref{item:F1}, \ref{item:F3} and the fact that $\im[F]>0$ on $\R$, we can observe that every point of $(0, \frac1{2}R)$ is surrounded by the curve $\widetilde{F}(c_R)$ exactly once. This implies that for every $x \in (0, \frac1{2}R)$ there exists a unique point $H(x)$ in the Jordan domain surrounded by $c_R$ such that $\widetilde{F}(H(x)) =x$. Because $R$ is arbitrary as long as sufficiently large, for every $x \in (0, \infty)$ there exists a unique point $H(x)$ in the domain surrounded by the curve $i[-\infty, 0]\cup \{z \in \partial\Xi: 0< \re(z) < \infty \} \cup [0,\infty)$ such that $\widetilde{F}(H(x)) =x$. It remains to prove the analyticity of the function $H$. First note that $\widetilde F'(z) \ne 0$ holds on $p_0^+$; otherwise the point $H(x)$ would not be unique. Therefore, $\widetilde F$ is locally bijective at each point $H(x)$ and hence its inverse function $H$ is also analytic in a complex neighborhood of each point $x>0$. 
\end{proof}


Let 
$$
g(x) := \re[H(x)] \qquad \text{and} \qquad h(x):= -\im[H(x)].
$$
Obviously, we have $g, h >0$. 
By analytic continuation equation \eqref{eq:ODE} easily extends to 
\begin{align}\label{eq:ODE2}
\widetilde{F}'(z)= \widetilde{F}(z)(z- \widetilde{F}(z)), \qquad z\in \Xi. 
\end{align}
Because $H$ is the compositional inverse map of $\Restr{\widetilde{F}}{p_0^+}$ and $\widetilde{F}' (z)$ does not vanish on $p_0^+$,  
 the ODE \eqref{eq:ODE2} restricted to $p_0^+$ entails the ODE $H'(x) = \frac{1}{x(H(x)-x)}$, which is equivalent to  
\begin{align}
g'(x)&=\frac{g(x)-x}{x((g(x)-x)^2+h(x)^2)} \quad \text{and} \label{eq:g} \\
h'(x)&=-\frac{h(x)}{x((g(x)-x)^2+h(x)^2)}.  \label{eq:h}
\end{align}
Using this ODE we provide some properties of $g$ and $h$ below. Some of the results will be made much finer in Section \ref{sec:freeLevy}. 

\begin{proposition}\label{prop:gh}
The following hold: 
\begin{align*}
g'>0 \quad \text{and} &\quad h'<0 \quad \text{on} \quad (0,\infty);  \\ 
 \lim_{x\to \infty} g(x) =\infty, \quad \lim_{x\to \infty} h(x) =0, &\quad  \lim_{x\to 0^+} g(x) =0 \quad \text{and}\quad  \lim_{x\to 0^+} h(x) =\infty. 
\end{align*}
 \end{proposition}
\begin{proof}
Equation \eqref{eq:h} readily implies $h'(x)<0$. Because of \eqref{eq:F_asympt} and the construction of $p_0^+$, we can conclude that $\lim_{x\to\infty} g(x)=\infty$, which associates the limit $\lim_{x\to\infty} h(x)=0$
because $g(x) - i h(x) \in \Xi$. To show $g'(x)>0$ it is convenient to introduce the function $ \omega(x) := g(x)-x$. Suppose to the contrary that $g'$ takes a nonpositive value. 
Because $g(x) \to \infty$ as $x \to\infty$ there is at least a strictly increasing sequence converging to $\infty$ on which $g'$ takes positive values. 
Therefore, we can find $x_0, x_1 \in (0,\infty)$ with $x_0 <x_1$ such that $g'(x_0)=0$ and $g'(x)>0$ for all $x \in (x_0,x_1)$. In terms of $\omega$ the former reads $\omega'(x_0)=-1$. 
In view of \eqref{eq:g} it also follows that $\omega(x_0)=0$ and $\omega(x)>0$ for $x \in (x_0,x_1)$.  These (in)equalities obviously contradict. The proof of $g'(x)>0$ is thus complete. 

It remains to prove the last two limits.  It suffices to establish $\lim_{x\to 0^+} h(x) =\infty$ because then $\lim_{x\to 0^+} g(x) =0$ follows from the fact that $g(x) - ih(x) \in \Xi$.  

Suppose to the contrary that $\beta := \lim_{x\to 0^+} h(x) <\infty$. We set $\alpha :=  \lim_{x\to 0^+} g(x) \in [0,\infty)$. Then $\alpha-i\beta \in \Xi \cup \partial \Xi$. Taking the limit in the formula $\widetilde F(g(x) - i h(x)) = x$, we get $\widetilde F(\alpha -i \beta) =0$, a contradiction to the fact that $\widetilde F$ does not have zeros on $\Xi \cup \partial \Xi$.
\end{proof}

Finally, we describe the domain $\Omega_{N(0,1)}$ rather explicitly. This result reproduces the known fact that $N(0,1) \in {\bf UI}$, cf.\ Remark \ref{rem:UI}. 

\begin{theorem}\label{thm:omega}
We set  
\begin{align*}
\Omega=\{x+iy \in \C: x,y\in \R,\ x\neq0,\ y>f(x)\} \cup i\R,
\end{align*}
where $f\colon \R\setminus\{0\}\rightarrow (-\infty,0)$ is the real analytic function determined by $f(x) = -h \circ g^{-1}(|x|)$. Then $\Omega$ coincides with $\Omega_{N(0,1)}$, i.e.\ $\widetilde{F}$ is a bijection from $\Omega$ onto $\C^+$. 
\end{theorem}


\begin{proof}
First note that $\Omega$ is exactly the domain that has boundary $p_0^+ \cup p_0^- $ and contains $\C^+$ as a subset. 

Because $\partial \Xi \cup \{\infty\}$ is not a simple closed curve in the Riemann sphere, we introduce an approximating simple closed curve $\gamma_R$ in the Riemann sphere consisting of $\{z \in p_0^+ \cup\{\infty\} \cup p_0^-: |\re(z)| \le R \}$,  the semicircle $\{R e^{i\theta}: 0 \le \theta \le \pi\}$ and two vertical line segments $R+i[-f(R), 0]$ and $-R+i[-f(R), 0]$, where $\infty$ stands for the point corresponding to $\lim_{x\to0^+} H(x)$ and $R>0$. Note that, as a consequence of Lemma \ref{cor:F}\ref{item:F3}, $\widetilde{F}$ can be regarded as a continuous function on closure (in the Riemann sphere) of the Jordan domain surrounded by $\gamma_R$. This fact is important in the next paragraph when we apply the Darboux theorem. 

We prove that $\widetilde{F}$ is univalent on  $\gamma_R$ for sufficiently large $R>0$. Because of the symmetry with respect to the imaginary axis, it suffices to prove the univalence on $\gamma_R \cup \{z: \re(z)>0\}$. First, $\widetilde{F}$ is univalent on $p_0^+$ by the construction of $p_0^+$. Also, we can take $r,R>0$ with $ r, g(r) < R/4$ large enough so that, by Lemma \ref{lem:F-transform},  $\widetilde{F}$ is univalent in $\gamma_R \cap \{x+iy: |x| \ge g(r) \text{~or~} y \ge r\}$ and that, by \eqref{eq:F_asympt},  $|\widetilde{F}(z)- z| < \frac{|z|}{2}$ holds for all $|z| \ge R, z \in \gamma_R$. In this situation one can see the univalence on the whole $\gamma_R$. This furthermore implies that $\widetilde{F}$ is a bijection from the Jordan domain surrounded by $\gamma_R$ onto the Jordan domain surrounded by $\widetilde{F}(\gamma_R)$ according to the Darboux theorem,  see e.g.\ \cite[Exercise 2.3-4]{Pom92} or \cite[Corollary 9.5]{Pom75}.\footnote{These references assume the Jordan curve to be contained in $\C$. Although $\gamma_R$ passes $\infty$, it can be suitably mapped to the complex plane, e.g.\ by the map $T\colon (\C \cup\{\infty\})\setminus\{c\} \to \C, T(z)= 1/(z-c)$ with a fixed $c \in \C\setminus \Xi$. Then \cite[Exercise 2.3-4]{Pom92} or \cite[Corollary 9.5]{Pom75} can be applied to the function $\widetilde{F} \circ T^{-1}$ which maps the Jordan curve $T(\gamma_R) \subset \C$ bijectively to the Jordan curve $\widetilde{F}(\gamma_R) \subset \C$.} 
By letting $R\to\infty$ we obtain the desired conclusion. 
\end{proof}

\begin{remark}\label{rem:homeo}
Together with Carath\'eodory's theorem \cite[Theorem 2.6]{Pom92}, the fact that $\gamma_R$ is a Jordan curve in the Riemann sphere implies that  $\widetilde{F}$ is also a homeomorphism from $\text{cl}(\Omega)$ onto $(\C^+\cup\R) \setminus \{0\}$. 
\end{remark}


\begin{remark} \label{rem:UI}  
The existence of $\Omega_{N(0,1)}$ follows easily from the work \cite{BBLS11}, where probability measures $\mu_c, c\in (-1,0)$, called the Askey-Wimp-Kerov distributions, are shown to be in \textbf{UI}. Because class \textbf{UI} is weakly closed (see \cite[p. 2763]{AH13}) and $N(0,1)$ is the weak limit of $\mu_c$ as $c\to0^-$, we conclude that $N(0,1)$ also belongs to \textbf{UI}. The strategy there for proving $\mu_c \in \mathbf{UI}$ was to construct, for each $t>0$, a simple curve $p_t^c$ which is symmetric with respect to the imaginary axis, passing through a unique point in $i\R$ and so that the reciprocal Cauchy transform maps $p_t^c$ bijectively onto $\R+it$, see \cite[Lemma 3.8]{BBLS11} for the construction of $p_t^c$. 

However, the domain $\Omega_{N(0,1)}$ was not investigated in \cite{BBLS11}. By contrast, our method directly constructed the boundary $p_0^+ \cup p_0^-$ of $\Omega_{N(0,1)}$ as the preimage of $\R \setminus\{0\}$ by the map $\Restr{\widetilde F}{\Xi}$ (for the interested reader, the curves $\im[\widetilde{F}]=t$ for different $t$'s are shown in Figure \ref{fig2}). Further details on the boundary $p_0^+ \cup p_0^-$ will be clarified in Theorems \ref{thm:fine_asymptotics} and \ref{thm:main_infty} below. 
\end{remark}

\begin{figure}[h!]
\begin{center}
\begin{minipage}{0.45\hsize}
\begin{center}
\begin{overpic}[width=6cm]{Omega.eps}
\put(20,55){$p_0^-$}
\put(75,55){$p_0^+$}
\end{overpic}
\end{center}
\caption{the curves $\im[\widetilde{F}]=0$ (undashed) and $\partial \Xi$ (dashed). Outside $\Xi$ there are more preimages (different from $p_0^\pm$) of $\R \setminus\{0\}$ by the mapping $\widetilde{F}$. When the real part is sufficiently small, the curves $p_0^+ \cup p_0^-$ and $\partial \Xi$ are close. } \label{fig1}
\end{minipage}
\hspace{5mm}
\begin{minipage}{0.45\hsize}
\begin{center}
\begin{overpic}[width=6cm]{Omega2.eps}
\put(20,55){$p_0^-$}
\put(75,55){$p_0^+$}
\end{overpic}
\end{center}
\caption{the curves $\im[\widetilde{F}]=t$ for $t=0$ (blue), $t=0.1$ (yellow), $t=0.4$ (green), $t=0.7$ (red), $t=1$ (purple), $t=1.3$ (brown).  \vspace{11mm} }\label{fig2}
\end{minipage}
\end{center}

\begin{center}
\begin{minipage}{0.45\hsize}
\begin{center}
\begin{overpic}[width=6cm]{Poles.eps}
\put(20,55){$p_0^-$}
\put(75,55){$p_0^+$}
\end{overpic}
\end{center}
\caption{the curves $\im[\widetilde{G}]=0$ (undashed) and $\re[\widetilde{G}]=0$ (dashed). The intersection of them is the poles of $\widetilde{F}$. } \label{fig3}
\end{minipage}
\hspace{5mm}
\begin{minipage}{0.45\hsize}
\begin{center}
\vspace{8mm}
\begin{overpic}[width=6cm]{h.eps}
\end{overpic}
\end{center}
\caption{the graph of $h(x)$.  Figures \ref{fig1} -- \ref{fig4} are drawn on Mathematica 12.  \vspace{16mm} }\label{fig4} 
\end{minipage}
\end{center}
\end{figure}



\section{Proofs of the main results}\label{sec:freeLevy}

According to Theorem \ref{thm:omega} and Remark \ref{rem:homeo}, we can define the analytic compositional inverse function $\widetilde{F}^{-1}\colon\C^+ \to \Omega$ which extends to a homeomorphism from $(\C^+\cup\R) \setminus \{0\}$ onto $\text{cl}(\Omega)$. It is even analytic on  $\R\setminus \{0\}$ with values $\widetilde{F}^{-1} (x)  = \text{sign}(x)g(|x|) - i h(|x|)$ according to the previous section and by symmetry. On the other hand, by Lemma \ref{lem:F-transform}, the function $\widetilde{F}$ is a bijection from $D_{\epsilon,R}$ onto its range that contains $D_{\epsilon',R'}$ for sufficiently large $R$. This allows us to get an analytic continuation of $\widetilde{F}^{-1}$ to the domain $(\C^+\cup\R) \cup D_{\epsilon',R'} \setminus \{0\}$. 



\begin{proof}[Proof of Theorem \ref{thm:freeLevy}]
 Recall that the Voiculescu transform $\widetilde{\varphi}(w) := \widetilde{F}^{-1}(w) -w$ is an analytic function from $(\C^+\cup \R)\setminus\{0\}$ to $\C^-\cup \R$. In the Pick--Nevanlinna representation 
\begin{align}\label{eq:VT}
\widetilde{\varphi}(z)=b+\int_\R \frac{1+zx}{z-x} \, \tau(\dd x),  \qquad z \in \C^+, 
\end{align}
where $b\in \R$ and $\tau$ is a finite measure on $\R$, the Stieltjes inversion formula yields 
\begin{align*}
(1+x^2)\mathbf1_{\R\setminus\{0\}}(x)\tau(\dd x)=-\frac{1}{\pi} \im [\widetilde{\varphi}(x)]\,\dd x = \frac{1}{\pi} h(|x|)\,\dd x. 
\end{align*}
According to \eqref{eq:freeLM} the free L\'{e}vy measure of $N(0,1)$ is given as desired. 
\end{proof}

\begin{remark}
Because $N(0,1)$ is symmetric, $\widetilde{\varphi}(i)  \in i \R$ and hence the number $b = \re[\widetilde{\varphi}(i)]$ vanishes. Moreover, we can show that the semicircular component $\tau(\{0\})$ vanishes too. Indeed, according to \eqref{eq:G(ix)}, we have $\widetilde{F}(ix) \sim i (1/\sqrt{2\pi}) e^{-x^2/2}$ as $x\to -\infty$. Using the formula $\tau(\{0\})= \lim_{y \to 0^+} iy \widetilde{\varphi}(iy)$ we deduce that 
\[
\tau(\{0\}) = \lim_{y\to0^+} iy \widetilde{F}^{-1}(iy) = \lim_{x\to -\infty} \widetilde{F}(ix) ix =0 
\]
as desired. 
\end{remark}




Let $\kappa_n$ be the $n$-th free cumulant of $N(0,1)$ below. Because $N(0,1)$ is symmetric, the odd free cumulants vanish. According to \cite[p. 3683]{BBLS11}, some even free cumulants are given as follows: 
\[
\kappa_2=1, \quad \kappa_4=1,  \quad \kappa_6=4 \quad \text{and}\quad \kappa_8=27.
\]
The following fact is a key for understanding the asymptotics of $h(x)$ as $x \to \infty$. We can prove it analogously to \cite[Theorem 1.3 and Proposition A.3]{BG06}. For the reader's convenience a self-contained proof is provided in Appendix \ref{appendix}. 

\begin{lemma} \label{lem:F^{-1}(z)_infty}
For every fixed $N \in \N$ and $\epsilon'\in(0,\pi/4)$ we have 
\begin{equation}  \label{eq:asymptotic_F^{-1}}
\widetilde{F}^{-1}(w) = w + \sum_{n=1}^N \frac{\kappa_{2n}}{w^{2n-1}} + o\left(\frac{1}{w^{2N-1}}\right)  \quad \text{as} \quad w\to\infty, \ w \in  D_{\epsilon'}.  
\end{equation}
\end{lemma}


We provide a proof of Theorem \ref{thm:fine_asymptotics_main} \ref{item:h(x)_infty_main} below. Since the proof requires estimates on $g$ too, we expand the statement of Theorem \ref{thm:fine_asymptotics_main} \ref{item:h(x)_infty_main}  as follows. 



\begin{theorem} \label{thm:fine_asymptotics} The following asymptotic behaviors hold. 

\begin{enumerate}[label=\rm(g${}_\infty$),leftmargin=1.2cm]
\item\label{item:g(x)_infty} 
$\displaystyle 
g(x) = x + \sum_{n=1}^N \frac{\kappa_{2n}}{x^{2n-1}} + o\left(\frac{1}{x^{2N-1}}\right)  \quad \text{as} \quad x\to\infty  
$\quad for every fixed\quad $N \in \N$.  
\end{enumerate}

\begin{enumerate}[label=\rm(h${}_\infty$),leftmargin=1.2cm]
\item \label{item:h(x)_infty_main} 
$\displaystyle h (x) = \frac{1}{e} \sqrt{\frac{\pi}{2}}x^2 e^{-\frac{x^2}{2} } (1 + O(x^{-2}))$\quad  as \quad $x\to\infty$. 
\end{enumerate}
\end{theorem}

\begin{proof} 
\ref{item:g(x)_infty} is just the real part of the formula in Lemma \ref{lem:F^{-1}(z)_infty}. 

For the proof of \ref{item:h(x)_infty_main}, we set $a(x,y):=\im[\widetilde F(x + iy)]$. 
It follows from \eqref{eq:Stieltjes} and Lemma \ref{lem:asymptotic_Cauchy} that, as $x\to \infty$, 
\begin{align}
a(x,0) &= \frac{-\im[\widetilde G(x)]}{|\widetilde G(x)|^2} = \sqrt{\frac{\pi}{2}}x^2e^{-\frac{x^2}{2}} (1 + O(x^{-2})) \quad \text{and}   \label{eq:asymp1} \\
a_y(x,y) &= \im[ i \widetilde F ' (x + iy)] = \re[\widetilde F ' (x + iy)] = 1 + O(x^{-2}),   \label{eq:asymp2}
\end{align}
as long as $x+i y \in D_{\epsilon}$. 
By Taylor's theorem for every $x >0$ and $ y \in [-1,0]$ there exists $\theta \in (0,1)$ such that $a(x,y) = a(x,0) + a_y (x,\theta y) y $, so that 
\[
a(x,y) =  \sqrt{\frac{\pi}{2}}x^2e^{-\frac{x^2}{2}} (1 + O(x^{-2})) +(1 + O(x^{-2})) y 
\]
If we set the function
\[
f_c(x) := - \sqrt{\frac{\pi}{2}}x^2e^{-\frac{x^2}{2}} (1 + c x^{-2}), \qquad c \in \R, 
\]
then we get 
\[
a(x, f_c(x))  =  \sqrt{\frac{\pi}{2}}x^2e^{-\frac{x^2}{2}} (O(x^{-2}) - c x^{-2}(1+O(x^{-2}))),   
\]
where the $O(x^{-2})$'s are all independent of $c$. 
This implies, for a sufficiently large (fixed) $c>0$, that 
\[
a(x, f_c(x))<0 < a(x,f_{-c}(x)) 
\] 
for all sufficiently large $x$ (such that $f_c(x) >-1 $). 
From the results in Subsection \ref{subsec:boundary}, for $x>0$, the function  $f(x)$ introduced in Theorem \ref{thm:omega} is a unique solution $y \in(-\pi/(2x),0)$ to the equation $a(x,y)=0$. We therefore have $f_c(x) < f(x) < f_{-c}(x)$ for sufficiently large $x >0$, i.e.\  
\begin{equation}\label{eq:asymp_f}
f(x) = - \sqrt{\frac{\pi}{2}}x^2e^{-\frac{x^2}{2}} (1 + O(x^{-2})), \qquad x\to\infty.  
\end{equation}
Finally, using $g(x) = x + \frac1{x} + O(x^{-3})$ we have 
\begin{align}
h(x) = - f(g(x)) 
&= \sqrt{\frac{\pi}{2}}g(x)^2e^{-\frac{g(x)^2}{2}} (1 + O(x^{-2})) \notag  \\
&= \sqrt{\frac{\pi}{2}}x^2 (1+O(x^{-2}))^2 e^{-\frac{x^2}{2}(1+\frac{1}{x^2} + O(x^{-4}))^2} (1 + O(x^{-2})) \notag \\
&=\sqrt{\frac{\pi}{2}}x^2 e^{-\frac{x^2}{2}} e^{-1+ O(x^{-2})} (1 + O(x^{-2}))   \notag \\
&= \frac1{e}\sqrt{\frac{\pi}{2}}x^2 e^{-\frac{x^2}{2}}(1 + O(x^{-2})).   \label{eq:asymp_h}
\end{align}
\end{proof}
  
 
 
 
 

\begin{remark}\label{rem:asymptotic}
With some more elaboration,  \ref{item:h(x)_infty_main} can be generalized to higher order expansions of any order
 \begin{equation} \label{eq:precise_h_infty0}
 h (x) = \frac{1}{e} \sqrt{\frac{\pi}{2}}x^2 e^{-\frac{x^2}{2} } \left(1 + \frac{a_2}{x^2} + \frac{a_4}{x^4} + \cdots + \frac{a_{2N}}{x^{2N}} + o(x^{-2N}) \right), \qquad x\to\infty,   
 \end{equation}
where the coefficients $a_{2n}$ are determined by the formula (in the sense of formal power series or asymptotic expansion)
\begin{align}
1 + \sum_{n=1}^\infty \frac{a_{2n}}{x^{2n}}
&= (\widetilde F^{-1})'(x) \exp\left[-\frac{1}{2} \left(\widetilde F^{-1}(x)^2 -x^2 -2\right)\right]   \label{eq:precise_h_infty} \\
&= \left( 1-  \sum_{n=1}^\infty \frac{(2n-1) \kappa_{2n}}{x^{2n}} \right) \exp\left[-\frac{1}{2x^2}  \left(1+ \sum_{n\ge2}\frac{\kappa_{2n}}{x^{2n-2}} \right)^2 -  \sum_{n\ge2}\frac{\kappa_{2n}}{x^{2n-2}}  \right]  \notag \\
&=  1 - \frac{5}{2x^2}  -\frac{43}{8x^4} - \frac{579}{16 x^6}  - \cdots.   \notag
\end{align} 
The proof is sketched below. We first refine \eqref{eq:asymp1} and \eqref{eq:asymp2}. 
From Lemma \ref{lem:asymptotic_Cauchy} we have $\widetilde F(z) = \FN(z) + o(z^{-2N+1})$ and $\widetilde F ' (z) = (\FN)'(z) + o(z^{-2N})$ as $z\to\infty$ with $z \in D_\epsilon$, where 
\[
\FN(x) = x - \sum_{n=1}^N \frac{b_{2n}}{x^{2n-1}} 
\]
for some constants $b_{2n} \in \R ~(n\in \N)$ (the Boolean cumulants of $N(0,1)$). Then we get the following refinement of \eqref{eq:asymp1}: 
\begin{align}
a(x,0) &= \frac{-\im[\widetilde G(x)]}{|\widetilde G(x)|^2} = \sqrt{\frac{\pi}{2}}x^2e^{-\frac{x^2}{2}} \left(\frac{\FN(x)}{x}  + o(x^{-2N})\right)^2, \qquad x\to\infty.   \label{eq:asymp3}
\end{align}
On the other hand, for \eqref{eq:asymp2}, we limit the variables $(x,y)$ to the thin domain 
\[
J := \{x+ iy: x>0, 2 f_0(x) < y < -2f_0(x) \}. 
\]
The point is that $x+i f(x)$ is contained in $J$ for large $x$ thanks to the established \eqref{eq:asymp_f}. Then we can get the following refinement of  \eqref{eq:asymp2}: 
\begin{equation}
a_y(x,y)  = \re[\widetilde F ' (x + iy)]   = (\FN)'(x) + o(x^{-2N}), \qquad x\to \infty \quad \text{with} \quad x+iy \in J.   \label{eq:asymp4}
\end{equation}
The point is that no $y$'s appear in the main term above thanks to the exponential bounds for $y$.  

Then substituting $y=f(x)$ into $a(x,y) = a(x,0) + a_y (x,\theta y) y $ and combining it with \eqref{eq:asymp3} and \eqref{eq:asymp4} yield
\[
f(x) = - \sqrt{\frac{\pi}{2}}x^2e^{-\frac{x^2}{2}} \left( 1 + \frac{c_2}{x^2} + \frac{c_4}{x^4} + \cdots + \frac{c_{2N}}{x^{2N}} + o(x^{-2N}) \right), 
\]
where $c_2,c_4, \dots$ are determined by the equation (in the sense of asymptotic expansion) 
\[
 1+ \sum_{n=1}^\infty \frac{c_{2n}}{x^{2n}}  =  \frac{ \left( 1 - \sum_{n=1}^\infty \dfrac{b_{2n}}{x^{2n}}\right)^2}{1 + \sum_{n=1}^\infty \dfrac{(2n-1)b_{2n}}{x^{2n}}   },   
\]
the right hand side of which can be written as $\frac{\widetilde F(x)^2}{x^2 \widetilde F'(x)}$.  Finally computing $h(x) = -f(g(x))$ as in \eqref{eq:asymp_h} yields the desired formula  \eqref{eq:precise_h_infty0}. Note that we can write $g(x) = \widetilde F^{-1}(x)$ in the sense of asymptotic expansion, which is a key ingredient for deriving \eqref{eq:precise_h_infty}. 
\end{remark}

Next, we provide a proof of Theorem \ref{thm:fine_asymptotics_main} \ref{item:h(x)_zero_improved}. The proof needs estimates on the functions $f$, $g$ and $gh$, so we expand the statement of Theorem \ref{thm:fine_asymptotics_main} \ref{item:h(x)_zero_improved}.
The results also offer a better understanding of the boundary of $\Omega$; especially they justify that the curve $\{H(x): x>0\}$ approaches $\partial \Xi$ as $x\to0^+$ as observed  in  Figure \ref{fig1}. 




\begin{theorem}  \label{thm:main_infty} Let $0<\eta<1$ be fixed. The following asymptotic behavior holds. 

\begin{enumerate}[label=\rm(f${}_0$),leftmargin=1.2cm] 
\item\label{item:f(x)_zero_improved} 
$\displaystyle 
f(x) = -\frac{\pi}{2x }[1+ O(e^{-\frac{\pi^2 \eta}{8x^2}})] \quad \text{as} \quad x\to0^+. 
$ 
\end{enumerate}
\begin{enumerate}[label=\rm(g${}_0$),leftmargin=1.2cm] 
\item\label{item:g(x)_zero_improved} 
$\displaystyle 
g(x) =\sqrt{-\log \frac1{\sqrt{2\pi}\, x} + \sqrt{ \left(\log \frac1{\sqrt{2\pi}\, x}\right)^2+ \frac{\pi^2}{4} }} + O(x^\eta) \quad \text{as} \quad x\to0^+. 
$ 

In particular, $g(x) \sim \frac{\pi}{\sqrt{8\log \frac1{x}}}$. 
\end{enumerate}
\begin{enumerate}[label=\rm(gh${}_0$),leftmargin=1.2cm] 
\item\label{item:g(x)h(x)_zero} 
$\displaystyle 
g(x)h(x) = \frac{\pi}{2}\left[ 1 + O(x^\eta)\right]  \quad \text{as} \quad x\to0^+. 
$
\end{enumerate}
\begin{enumerate}[label=\rm(h${}_0$),leftmargin=1.2cm] 
\item\label{item:h(x)_zero_improved} 
$\displaystyle 
h(x) = \sqrt{\log \frac1{\sqrt{2\pi}\, x} + \sqrt{ \left(\log \frac1{\sqrt{2\pi}\, x}\right)^2+ \frac{\pi^2}{4} }} + O(x^\eta)  \quad \text{as} \quad x\to0^+. 
$

In particular, $h(x) \sim \sqrt{2\log \frac1{x}}$. 
\end{enumerate}
\end{theorem}


\begin{proof}
\begin{enumerate}[leftmargin=1cm]

\item[\ref{item:f(x)_zero_improved}] Recall that, for $x>0$, $f(x) = -h \circ g^{-1} (x)$ is a unique solution $y \in (-\frac{\pi}{2x},0)$ to $\im [\widetilde{G}(x+iy)]=0$.  We set $\epsilon = \epsilon(x) =e^{-\frac{\pi^2 \eta}{8x^2}}$. Formula \eqref{eq:CauchyND} implies that, for $x>0$, 
\begin{align*}
\im\left[ \widetilde{G}\left(x-\frac{\pi}{2x} (1-\epsilon) i \right) \right]
&= \im\left[G\left(x-\frac{\pi}{2x}(1-\epsilon) i\right) - 2\pi i  \frac{1}{\sqrt{2\pi}} e^{-\frac{1}{2}(x^2-\frac{\pi^2}{4x^2}(1-\epsilon)^2)} e^{\frac{\pi}{2}(1-\epsilon) i} \right] \\
&= \im \left[G\left(x-\frac{\pi}{2x}(1-\epsilon) i\right)\right]-\sqrt{2\pi} e^{-\frac{1}{2}(x^2-\frac{\pi^2}{4x^2}(1-\epsilon)^2)} \sin \frac{\pi\epsilon}{2}
\end{align*}
and therefore, as $x\to 0^+$, 
\begin{align*}
\im \left[\widetilde{G}\left(x-\frac{\pi}{2x}(1-\epsilon) i \right) \right] 
 &= O(x) - (1+o(1))  \frac{\sqrt{2\pi^3}}{2} \epsilon e^{\frac{\pi^2}{8x^2}(1-\epsilon)^2} \\
 &=  o(1) - (1+o(1))  \frac{\sqrt{2\pi^3}}{2} e^{\frac{\pi^2}{8x^2}(1-\eta+o(1))}. 
\end{align*}
In particular, $\im \left[\widetilde{G}\left(x-\frac{\pi}{2x}(1-\epsilon) i \right) \right] <0$ for sufficiently small $x>0$. On the other hand, recall from Lemma \ref{cor:F} that $\im \left[\widetilde{G}\left(x-\frac{\pi}{2x} i \right) \right] >0$. Therefore, $-\frac{\pi}{2x} < f(x) < -\frac{\pi}{2x}(1-\epsilon)$ for sufficiently small $x>0$; in particular
\ref{item:f(x)_zero_improved} holds. 

\item[\ref{item:g(x)_zero_improved}]  We begin with estimating 
\begin{align*}
 \widetilde{G}\left(x+ if(x) \right) 
&= G\left(x+ if(x)\right) - 2\pi i  \frac{1}{\sqrt{2\pi}} e^{-\frac{1}{2}[x^2-f(x)^2]} e^{-i x f(x)}  \\
&= o(1) -i  \sqrt{2\pi}e^{-\frac1{2}x^2}  e^{\frac{\pi^2}{8x^2}[1+ O(e^{-\frac{\pi^2 \eta}{8x^2}})]} e^{\frac{i\pi }{2}[1+O(e^{-\frac{\pi^2 \eta}{8x^2}}) ]}  \\
&= \sqrt{2\pi} (1+O(e^{-\frac{\pi^2 \eta'}{8x^2}}))e^{-\frac1{2}x^2} e^{\frac{\pi^2}{8x^2}},   \qquad x \to0^+, 
\end{align*}
for any  $\eta'  \in (0,\eta)$. This yields 
\begin{equation}\label{eq:H(y)}
y:= \widetilde{F}\left(x+ if(x) \right)  = \frac1{\sqrt{2\pi}} (1+O(e^{-\frac{\pi^2 \eta'}{8x^2}}))e^{\frac1{2}x^2} e^{- \frac{\pi^2}{8x^2}}. 
\end{equation}
Note that $y\to 0^+$ as $x\to0^+$ and $H(y) = x + if(x)$, so that $g(y)=x$. Taking the logarithm of \eqref{eq:H(y)} we obtain $x^4 -2x^2 \log (\sqrt{2\pi}\, y) -\frac{\pi^2}{4}+ O(e^{-\frac{\pi^2 \eta'}{8x^2}})=0$ and hence (by the quadratic formula)
\begin{equation} \label{eq:g(y)^2}
x^2 = \log (\sqrt{2\pi}\, y) \pm \sqrt{ (\log (\sqrt{2\pi}\, y) )^2 
+ \frac{\pi^2}{4} +   O(e^{-\frac{\pi^2 \eta'}{8x^2}})}. 
\end{equation}
The indefinite sign above is actually $+$ because $x\to 0^+$ as $y\to0^+$; then,   
in particular, $x^2 \sim \frac1{8}\cdot \frac{\pi^2}{\log \frac{1}{y}}$. This implies $O(e^{-\frac{\pi^2 \eta'}{8x^2}}) = O(y^{\eta'+o(1)})$ and hence 
\begin{align}
\sqrt{ (\log (\sqrt{2\pi}\, y) )^2 + \frac{\pi^2}{4} +   O(e^{-\frac{\pi^2 \eta'}{8x^2}})} 
&= \sqrt{ (\log (\sqrt{2\pi}\, y) )^2   \notag
+ \frac{\pi^2}{4}} \sqrt{1 +   \frac{O(y^{\eta'+o(1)}) }{(\log(\sqrt{2\pi}\, y) )^2 
+ \frac{\pi^2}{4}  } }   \notag  \\
&=   (1+O(y^{\eta'+o(1)}(\log y)^{-2}))\sqrt{ (\log (\sqrt{2\pi}\, y) )^2 
+ \frac{\pi^2}{4}}   \notag \\
&=  \sqrt{ (\log(\sqrt{2\pi}\, y) )^2 
+ \frac{\pi^2}{4}}  +O(y^{\eta' +o(1)}).   \label{eq:landau}
\end{align}

Combining \eqref{eq:g(y)^2} and \eqref{eq:landau}, taking the square root and similarly handling the Landau symbols yields the desired \ref{item:g(x)_zero_improved}.  (Recall that $0< \eta' < \eta<1$ were arbitrary.)

\item[\ref{item:g(x)h(x)_zero}] Substituting $g(x) \sim \frac{\pi}{\sqrt{8 \log 1/x}} ~(x\to0^+)$ into \ref{item:f(x)_zero_improved},  we get 
\begin{align}
g(x) h(x) &= -g(x) f(g(x)) = \frac{\pi}{2} \left[1+ O\left(e^{-\frac{\pi^2 \eta}{8g(x)^2}   }\right) \right]  = \frac{\pi}{2}\left[ 1 + O(x^{\eta+o(1)})\right]. 
\end{align}
This finishes the proof of \ref{item:g(x)h(x)_zero} since $0<\eta <1$ was arbitrary. 

\item[\ref{item:h(x)_zero_improved}] One only needs to combine \ref{item:g(x)_zero_improved}  and \ref{item:g(x)h(x)_zero}. 

\end{enumerate}
\vspace{-7mm}
\end{proof}






% Acknowledgements 
%%%%%%%%%%%%%%%%%%%
 \section*{Acknowledgements}
This research is supported by JSPS Open Partnership Joint Research Projects Grant Number JPJSBP120209921. Moreover, T.H. is supported by JSPS Grant-in-Aid for Young Scientists 19K14546. Y.U. is supported by JSPS Grant-in-Aid for Scientific Research (B) 19H01791 and JSPS Grant-in-Aid for Young Scientists 22K13925. 









\appendix 

\section{Proofs of basic asymptotic expansions} \label{appendix}

\begin{proof}[Proof of Lemma \ref{lem:asymptotic_Cauchy} (continued)]
By taking the derivatives of \eqref{eq:another_rep} with respect to $z$, similar formulas hold for moments: 
\begin{equation}\label{eq:moments}
\int_{D_\epsilon}  w^n   \frac{1}{\sqrt{2\pi}} e^{-\frac{w^2}{2}}\,\dd w 
= \int_{\R}  x^n   \frac{1}{\sqrt{2\pi}} e^{-\frac{x^2}{2}}\,\dd x =m_n, \qquad n \in \N \cup \{0\}. 
\end{equation}
Since the RHS of \eqref{eq:another_rep} is analytic in $D_\epsilon$, the identity theorem yields that
\begin{equation*} 
\widetilde{G}(z)=\int_{\partial D_\epsilon} \frac{1}{z-w}\frac{1}{\sqrt{2\pi}} e^{-\frac{w^2}{2}}\,\dd w, \qquad z\in D_\epsilon. 
\end{equation*}

Take $0< \eta <\epsilon< \pi/4$ here. The transform $\widetilde{G}$ obviously has the following representation:
\begin{equation}\label{eq:cauchy2}
\widetilde{G}(z)=\int_{\partial D_{\eta}} \frac{1}{z-w} \cdot \frac{1}{\sqrt{2\pi}} e^{-\frac{w^2}{2}}\,\dd x, \qquad z\in D_\epsilon.
\end{equation}
Combining \eqref{eq:cauchy2} and \eqref{eq:moments} (the latter for $\eta$ instead of $\epsilon$) and the elementary identity
\begin{equation} \label{eq:geometric_series}
\frac{1}{z-w} - \sum_{k=0}^{2N-1} \frac{w^k}{z^{k+1}} = \frac{w^{2N}}{z^{2N}(z-w)}, 
\end{equation}
 we obtain 
\begin{align}
z^{2N+1}\left( \widetilde{G}(z)-\sum_{k=0}^{2N-1} \frac{m_k}{z^{k+1}}\right) 
&=z^{2N+1} \int_{\partial D_{\eta}}\left(   \frac{1}{z-w} -\sum_{k=0}^{2N-1} \frac{w^{k}}{z^{k+1}}\right) \frac{1}{\sqrt{2\pi}} e^{-\frac{w^2}{2}}\,\dd w \notag \\
&= \int_{\partial D_{\eta}} \frac{w^{2N}z}{z-w} \cdot \frac{1}{\sqrt{2\pi}} e^{-\frac{w^2}{2}}\,\dd w, \qquad z \in D_{\epsilon}.\label{eq:Cauchy2}
\end{align}
We here observe that for $w= re^{i(-\frac{\pi}{4}+\eta)} \in \partial D_{\eta}$
\begin{equation}\label{eq:geometric}
\sup_{z\in D_\epsilon}\left| \frac{z}{z-w} \right| \le 1+  \sup_{z\in D_\epsilon}\left|\frac{w}{z-w}\right| = 1+ \frac{1}{\sin (\epsilon - \eta)},
\end{equation}
so that, by the Lebesgue convergence theorem,  \eqref{eq:Cauchy2} converges to the finite number $m_{2N}$ as $z \to \infty$.  This completes the proof of \eqref{eq:asymptotic_Cauchy1}. 

The asymptotic expansion for $\widetilde{G}' (z)$ can be proved very similarly; one needs to use
\[
\widetilde{G}'(z)=-\int_{\partial D_\eta} \frac{1}{(z-w)^2} \cdot \frac{1}{\sqrt{2\pi}}e^{-\frac{w^2}{2}} \,\dd w, \qquad z\in D_\epsilon 
\]
and the $z$-differentiated  version  of formula \eqref{eq:geometric_series}. 
\end{proof}



\begin{proof}[Proof of Lemma \ref{lem:F^{-1}(z)_infty}]
We verify the formula for $N=2$ which should well explain how to handle the general $N$.  Let $0 < \epsilon < \epsilon' < \pi/4$ be fixed. 

A straightforward calculation translates Lemma \ref{lem:asymptotic_Cauchy} into the asymptotic expansion of $\widetilde{F}$ 
\begin{equation}\label{eq:asymp1b}
\widetilde{F}(z) = z - \frac1{z} - \frac{2}{z^3} + O\left( \frac1{z^5} \right), \qquad z \to\infty, \ z \in D_\epsilon. 
\end{equation} 
As the inverse function of $\widetilde{F}(z) = z(1+o(1))$, we obtain $\widetilde{F}^{-1}(w) = w(1+o(1))$. Plugging $\widetilde{F}^{-1}(w)$ into \eqref{eq:asymp1b} yields 
\[
w = \widetilde{F}( \widetilde{F}^{-1}(w)) = \widetilde{F}^{-1}(w)  - \frac1{w(1+o(1))} - \frac{2}{w^3(1+o(1))} + O\left( \frac1{w^5} \right), 
\]
which can be simplified to 
\begin{equation}\label{eq:asymp2b}
\widetilde{F}^{-1}(w) = w + \frac{1}{w} + o\left( \frac1{w} \right). 
\end{equation}
We then plug the refined asymptotics \eqref{eq:asymp2b} into \eqref{eq:asymp1b} to get 
\begin{align}
w &=  \widetilde{F}^{-1}(w)  - \frac1{w + \frac1{w} +o\left(\frac1{w}\right)} - \frac{2}{w^3(1+o(1))} + O\left( \frac1{w^5} \right) \\
&= \widetilde{F}^{-1}(w)  - \frac1{w} - \frac1{w^3} + o\left( \frac1{w^3} \right),  
\end{align}
and hence 
\[
\widetilde{F}^{-1}(w) = w + \frac{1}{w} +  \frac1{w^3}+o\left( \frac1{w^3} \right). 
\]
Higher order expansions can be computed similarly by induction on $N$:  substituting \eqref{eq:asymptotic_F^{-1}} for $N$ into $\widetilde F(z) = z - \sum_{n=1}^{N+1} \frac{b_{2n}}{z^{2n-1}} +o(\frac{1}{z^{2N+1}})$ yields the asymptotic formula \eqref{eq:asymptotic_F^{-1}} for $N+1$. 
From the procedure, the coefficients of $\widetilde F^{-1}$ are obtained as the coefficients of the formal inverse Laurent series of $\widetilde F(z)$. These coefficients are known to be the free cumulants, see e.g.\ \cite[Theorem 12.5]{NS06}. 
\end{proof}




% References
%%%%%%%%%%%%

\begin{thebibliography}{99}

\bibitem{Akh65} N.\ I.\ Akhiezer, \emph{The Classical Moment Problem}, Oliver \& Boyd, Edinburgh / London, 1965. 

\bibitem{AH13} O.\ Arizmendi and T.\ Hasebe, On a class of explicit Cauchy-Stieltjes transforms related to monotone stable and free Poisson laws, Bernoulli {\bf19(5B)} (2013), 2750--2767. 

\bibitem{AH13b} O.\ Arizmendi and T.\ Hasebe, Classical and free infinite divisibility for Boolean stable laws, Proc.\ Amer.\ Math.\ Soc.\ {\bf142} (2014), 1621--1632. 

\bibitem{AH16} O.\ Arizmendi and T.\ Hasebe, Classical scale mixtures of Boolean stable laws, Trans.\ Amer.\ Math.\ Soc.\ {\bf368} (2016), 4873--4905.

\bibitem{BBLS11} S.\ Belinschi, M.\ Bo\.{z}ejko, F.\ Lehner and R.\ Speicher, The normal distribution is $\boxplus$-infinitely divisible, Adv.\ Math.\ {\bf 226}, No.\ 4 (2011), 3677--3698.

\bibitem{BG06} F.\ Benaych-Georges, Taylor expansions of $R$-transforms: application to supports and moments. Indiana Univ. Math. J. {\bf 55} (2) (2006) 465--481.

\bibitem{BV93} H.\ Bercovici and D.\ Voiculescu, Free convolution of measures with unbounded support. Indiana Univ. Math. J. {\bf 42} (3) (1993) 733--773.

\bibitem{BH13} M.\ Bo\.zejko and T.\ Hasebe, On free infinite divisibility for classical Meixner distributions, Probab.\ Math.\ Stat.\ 33, Fasc. 2 (2013), 363--375. 

\bibitem{Gre60} D.\ Greenstein, On the analytic continuation of functions which map the upper half plane into itself, J.\ Math.\ Anal.\ Appl.\ {\bf1} (1960), 355--362.

\bibitem{Has14} T.\ Hasebe, Free infinite divisibility for beta distributions and related ones, Electron.\ J.\ Probab.\ {\bf19} (2014), No. 81, 1--33. 

\bibitem{Has16} T.\ Hasebe, Free infinite divisibility for powers of random variables, ALEA Lat.\ Am.\ J.\ Probab.\ Math.\ Stat.\ {\bf13} (2016), no.\ 1, 309--336. (Errors corrected in arXiv:1509.08614)

\bibitem{HNSU} T.\ Hasebe, K. Noba, N. Sakuma and Y. Ueda, On Boolean selfdecomposable distributions, arXiv:2206.04932.

\bibitem{HST19} T.\ Hasebe, N.\ Sakuma and S.\ Thorbj{\o}rnsen, The normal distribution is freely selfdecomposable, Int.\ Math.\ Res.\ Not.\ IMRN, vol.\ 2019, Issue 6, 1758--1787.

\bibitem{Kerov} S.\ Kerov, Interlacing measures. Kirillov's seminar on representation theory, 35--83, Amer.\ Math.\ Soc.\ Transl.\ Ser.\ 2, {\bf 181}, Adv.\ Math.\ Sci., 35, Amer.\ Math.\ Soc., Providence, RI, 1998.

\bibitem{MU20} J.\ Morishita and Y. Ueda, Free infinite divisibility for generalized  power distributions with free Poisson term, Probab. Math. Stat. {\bf 40}, Fasc.\ 2 (2020), 245--267. 

\bibitem{NS06} A.\ Nica and R.\ Speicher, Lectures on the combinatorics of free probability. London Mathematical Society Lecture Note Series, 335. Cambridge University Press, Cambridge, 2006. xvi+417 pp.

\bibitem{No34} K.\ Noshiro, On the theory of schlicht functions, J.\ Fac.\ Sci.\ Hokkaido Imper.\ Univ.\ {\bf 2} (1934), 129--155.

\bibitem{Pom75} Ch.\ Pommerenke, {\sl Univalent Functions}, VandenHoeck \& Ruprecht, G\"ottingen, 1975. 

\bibitem{Pom92} Ch.\ Pommerenke, {\sl Boundary behaviour of conformal maps}, Grundlehren der Mathematischen Wissenschaften, 299, Springer-Verlag, Berlin, 1992. 

\bibitem{Wa35} S.\ E.\ Warschawski, On the higher derivatives at the boundary in conformal mapping, Trans.\ Amer.\ Math.\ Soc.\ {\bf 38}, No.\ 2 (1935), 310--340.

\end{thebibliography}


\vspace{4mm}

\begin{enumerate}
\item[]
\hspace{-9mm}Takahiro Hasebe\\
Department of Mathematics, Hokkaido University.\\
North 10 West 8, Kita-ku, Sapporo 060-0810, Japan.\\
Email address: thasebe@math.sci.hokudai.ac.jp

\item[] 
\hspace{-9mm}Yuki Ueda\\
Department of Mathematics, Hokkaido University of Education. \\
Hokumon-cho 9, Asahikawa, Hokkaido, 070-8621, Japan.\\
Email address: ueda.yuki@a.hokkyodai.ac.jp
\end{enumerate}


} %%% end of large

\end{document}