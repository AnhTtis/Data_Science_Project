

\begin{abstract}
In this paper, we propose a new challenge that synthesizes a novel view in a more practical environment, where the number of input multi-view images is limited and illumination variations are significant.
Despite recent success, neural radiance fields (NeRF) require a massive amount of input multi-view images taken under constrained illuminations.
To address the problem, we suggest ExtremeNeRF, which utilizes occlusion-aware multi-view albedo consistency, supported by geometric alignment and depth consistency. 
We extract intrinsic image components that should be illumination-invariant across different views, enabling direct appearance comparison between the input and novel view under unconstrained illumination. 
We provide extensive experimental results for an evaluation of the task, using the newly built NeRF Extreme benchmark, which is the first in-the-wild novel view synthesis benchmark taken under multiple viewing directions and varying illuminations.
The project page is at https://seokyeong94.github.io/ExtremeNeRF/


\end{abstract} 


%\vspace{-20pt}


\begin{figure}[t]
    \centering
     \begin{subfigure}[b]{1.0\linewidth}
         \centering
        \includegraphics[width=\linewidth]{figure/ICCV/teaser_inputs.pdf}
     \end{subfigure}
    \begin{tabular}{>{\centering\arraybackslash}p{.55\linewidth}}
    \scriptsize Sparse inputs with varying illuminations \\ [0.1cm]
    \end{tabular}
     \begin{subfigure}[b]{1.0\linewidth}
         \centering
        \includegraphics[width=\linewidth]{figure/ICCV/teaser_results.pdf}        %
     \end{subfigure}
      \begin{tabular}{>{\centering\arraybackslash}p{.45\linewidth}>{\centering\arraybackslash}p{.45\linewidth}}
        \scriptsize RegNeRF~\cite{niemeyer2022regnerf}  & \scriptsize  \textbf{ExtremeNeRF (Ours)}\\
      \end{tabular}
    \vspace{-15pt}
    \caption{
        \textbf{Few-shot view synthesis results on inputs with varying illuminations.}
    Our ExtremeNeRF shows reliable novel view synthesis results compared to the state-of-the-art method, RegNeRF~\cite{niemeyer2022regnerf}, in an extremely challenging environment where only $3$ view images taken under varying illuminations are available.}
    \label{fig:teaser} \vspace{-10pt}
\end{figure}
% 











