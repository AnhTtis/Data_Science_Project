    \documentclass[10pt,twocolumn,letterpaper]{article}
    
\usepackage{iccv}
\usepackage{times}
\usepackage{epsfig}
\usepackage{graphicx}
\usepackage{amsmath}
\usepackage{amssymb}
\usepackage{booktabs}
\usepackage{verbatim}
\usepackage{wrapfig,lipsum,booktabs}
\usepackage[dvipsnames]{xcolor}
\usepackage{mathtools} 
\usepackage{multirow}
\usepackage{color}
\usepackage{subcaption}
\usepackage{colortbl}
\usepackage{listings}
\usepackage{comment}
\usepackage{algorithm}

% Include other packages here, before hyperref.
\newcommand{\figref}[1]{Fig. \ref{#1}}
\newcommand{\tabref}[1]{Table \ref{#1}}
\newcommand{\equref}[1]{(\ref{#1})}
\newcommand{\secref}[1]{Sec. \ref{#1}}
\newcommand{\algref}[1]{Alg. \ref{#1}}

\usepackage{pifont}
\newcommand{\cmark}{\ding{51}}%
\newcommand{\xmark}{\ding{55}}%

% If you comment hyperref and then uncomment it, you should delete
% egpaper.aux before re-running latex.  (Or just hit 'q' on the first latex
% run, let it finish, and you should be clear).
\usepackage[pagebackref=true,breaklinks=true,letterpaper=true,colorlinks,bookmarks=false]{hyperref}


\iccvfinalcopy % *** Uncomment this line for the final submission


\def\httilde{\mbox{\tt\raisebox{-.5ex}{\symbol{126}}}}

% Pages are numbered in submission mode, and unnumbered in camera-ready
\ificcvfinal\pagestyle{empty}\fi

\begin{document}

%%%%%%%%% TITLE
\title{ExtremeNeRF: Few-shot Neural Radiance Fields Under Unconstrained Illumination}

\author{\vspace{.05cm} SeokyYeong Lee$^{1,2}$ \quad JunYong Choi$^{1,2}$ \quad Seungryong Kim$^{2}$\\ \vspace{.2cm} Ig-Jae Kim$^{1,3,4}$ \quad Junghyun Cho$^{1,3,4}$ \\
\vspace{.05cm}$^1$Korea Institute of Science and Technology, Seoul \quad $^2$Korea University, Seoul \\
$^{3}$AI-Robotics, KIST School, University of Science and Technology \\
$^{4}$Yonsei-KIST Convergence Research Institute, Yonsei University \\
{\tt\small\{shapin94, happily, drjay, jhcho\}@kist.re.kr} \quad \tt\small seungryong\_kim@korea.ac.kr \\
}
\renewcommand\footnotemark{}
%\renewcommand\footnoterule{}
\thanks{This work was partly supported by Institute of Information \& communications Technology Planning \& Evaluation (IITP) grant funded by the Korea government(MSIT)(No.2020-0-00457, 50\%) and KIST Institutional Program(Project No.2E32301, 50\%).}


\maketitle
% Remove page # from the first page of camera-ready.
\ificcvfinal\thispagestyle{empty}\fi

\begin{abstract}
The current study investigated possible human-robot kinaesthetic interaction using a variational recurrent neural network model, called PV-RNN, which is based on the free energy principle.
Our prior robotic studies using PV-RNN showed that the nature of interactions between top-down expectation and bottom-up inference is strongly affected by a parameter, called the meta-prior, which regulates the complexity term in free energy.
% The current study examines how the behaviours of robots alter by changing the meta-prior $w$ in human-robot kinaesthetic interaction.
The current study examines how changing the meta-prior $w$ in the interaction phase affects the counter force generated when an experimenter attempts to induce movement pattern transitions familiar to the robot through its prior training.
The study also compares the counter force generated when trained transitions are induced by a human experimenter and when untrained transitions are induced.
Our experimental results indicated that (1) the human experimenter needs more/less force to induce trained transitions when $w$ is set with larger/smaller values, (2) the human experimenter needs more force to act on the robot when he attempts to induce untrained as opposed to trained movement pattern transitions.
Our analysis of time development of essential variables and values in PV-RNN during bodily interaction clarified the mechanism by which gaps in actional intentions between the human experimenter and the robot can be manifested as reaction forces between them.


%% Hiroki writing 2022-11-4
%Current study investigates the dynamics of the latent states during human-robot kinaesthetic interaction using PV-RNN.
%We have achieved to observe and analyse the internal state of an RNN model based on the free energy principle, during real-time human-robot interaction.
%Essential characteristics observed in the previous study of this variational recurrent neural network model, PV-RNN, is that by changing a meta prior $w$, the balance between the top-down intention and the bottom-up perceptual reality changes.
%In the current study, we examined how changing the weighting parameter $w$ between accuracy and complexity in free energy principle affects the humanoid robot's behaviour through human-robot interaction. We have conducted some human-robot kinaesthetic interaction experiments with various $w$ and quantitatively analysed the latent variable and the force applied to the humanoid robot. We have observed that the force required to change the robot's intention has increased, both when the top-down intention was strengthened by changing the $w$ and when corresponding switch of its primitive was against the experience of the RNN during its training. The study confirms through quantitative analysis that by increasing or decreasing the $w$ in PV-RNN, humanoid robot leads or follows the human counterpart during the human-robot kinaesthetic interaction.

\begin{comment}
Comment from Jun #2
・最後にQualitativeな結果(インパクト)が欲しい
・Current study investigates the problem on~と書き出すのが一般的
・最初の一文と最後の一文を対応させる
・最後の一文はもう少しAbstractかつ包括的に
\end{comment}

\begin{comment}
Comment from Jun #1
We investigated how the kinaesthetic human-robot interaction can affect the internal state of a model based on the free energy principle. 
=> how the internal state is affected is not the most important point in this study. This part should be rewritten.

The key function of this variational recurrent neural network model, PV-RNN, is that by changing a meta prior $w$, it takes a balance between the "complexity” term and the ”accuracy” term which corresponds to a top-down intention and a bottom-up perceptual reality in the free energy principle, respectively. 
=> This is not key function of PV-RNN. It is an essential characteristics observed in the previous study. The grammar after $w$ is something strange. Rewrite these.

This research has conducted a human-robot interaction experiment with a robotic agent in a kinaesthetic sense.
=> The sentence is not good. "in a kinaesthetic sense" is grammatically wrong.
MODIFIED => "In the current study human-robot interaction experiments using the kinaesthetic sense were conducted."

We investigated that when human forces the agent to switch primitives from one to another, larger force was required both when the human intention is conflictive against the top-down the intention of the agent and when the agent has a stronger top-down intention by modifying the $w$.
=> You should write the essential results of the experiments rather than what we investigated and also how these results could contribute to the studies on human-robot interaction.
\end{comment}

\end{abstract}
% \begin{figure}[t]
%     % \begin{subfigure}{1\linewidth}
%     %   \centering
%     % %   \includegraphics[width=1\linewidth]{figs/fig_1_moti_textattn.pdf}  
%     % %   \includegraphics[width=1\linewidth]{figs/fig_1_moti_textattn_v2.pdf}  
%     %   \includegraphics[width=1\linewidth]{figs/fig_1_moti_textattn_v5.pdf}  
%     %   \vspace{-0.5cm}
%     %     \caption{Amount of attention added to each video clip from the source video and query text in the self-attention layers of Moment-DETR encoder.}
%     %     % \caption{Distribution of attention for source and query in Moment-DETR encoder}
%     %     % Visualization of video clip's self-attention score in Moment-DETR encoder.
%     %   \label{fig:fig1_text_attn_ex}
%     % \end{subfigure}%\hfill% or  or \hspace{0.3\textwidth}
%     \vspace{0.2cm}
%     % \begin{subfigure}{1\linewidth}
%       \centering
%     %   \includegraphics[width=1\linewidth]{figs/fig1_moti_negattn.pdf}  
%       \includegraphics[width=1\linewidth]{figs/fig1_moti_negattn_v3.pdf}  
%       \vspace{-0.4cm}
%     %   \caption{Correspondence of saliency scores on the relevance between video clips and the text query.}
%     % \caption{Predicted saliency scores against the video relevant positive query and video irrelevant negative query}
%       \label{fig:fig1_neg_attn_ex}
%     % \end{subfigure}%\hfill% or  or \hspace{0.3\textwidth}
%     \caption{
%     % 원준 원본
%     % (a) Comparison between attention scores of source and query for each video clip~(We sum the attention scores from video and text). 
%     % We observe that the attention scores are dominated by other clips in the source video. 
%     % Text queries do not account for much attention regardless of the relevance to the video clips.
%     % \textbf{(a)} Inspection of the query dependency in Moment-DETR encoder.
%     % % We visualize the attention score of video tokens in the transformer encoder and observe that text query accounts for only a low portion of attention.
%     % % This tendency occurs regardless of the relevance between the text query and video clips. 
%     % We visualize the attention score of video tokens in the transformer encoder and observe 1) text query only accounts for a low portion of attention, and 2) relevance between video-query pair does not affect the attention scores ratio of text.
%     \textbf{(b)} Comparison of highlight-ness when relevant and non-relevant queries are input.
%     As observed in , existing work only uses queries to play an insignificant role, thereby may not be capable of detecting false queries and considering the video-query relevance even when the problem in (a) is resolved. 
%     % \SE{} % 이 부분이 "not capable of" 란 용어가 세다는 피드백이 있는 듯 합니다. 이러한 능력이 없다는 것은 굉장히 강한 어조인거 같기는 하고, 이러한 경우들이 종종 있다거나 좀 약화시킬 필요가 있어보이긴 하네요.
%     On the other hand, our QD-DETR yields a query-dependent representation that the relevance between the source video and query text is updated in the saliency scores.
%     There is a large gap between positive and negative saliency scores, and scores are consistent since the clips are all highly correlated to others.
%     }
%     \label{fig:motivation_ex}
%     % \captionsetup{belowskip=13pt}
%     % \setlength{\belowcaptionskip}{-10pt}
% \end{figure}
\begin{figure}
    \centering
    \includegraphics[width=1\linewidth]{figs/fig1_moti_negattn_1111.pdf}
    % \includegraphics[width=1\linewidth]{figs/fig1_moti_negattn_1109.pdf}
    % \includegraphics[width=1\linewidth]{figs/fig1_moti_negattn_stat.pdf}
    \vspace{-0.6cm}
    \caption{
        % \SE{} % 수정 필요
        Comparison of highlight-ness~(saliency score) when relevant and non-relevant queries are given.
        We found that the existing work only uses queries to play an insignificant role, thereby may not be capable of detecting negative queries and video-query relevance; saliency scores for clips in ground-truth~(GT) moments are low and equivalent for positive and negative queries.
        % This also results in mispredicted moments when ground-truth~(GT) moment is dominated by clips unrelated to GT since their prediction is highly focused on the video.
        % \SE{} % 여기 한번 더 보면 좋을 듯 합니다. GT moment에 unrelated한 clip이 많으면? label이 틀렷을 경우를 말씀하시는건지?
        % As observed in saliency graph, existing work only uses queries to play an insignificant role, thereby may not be capable of detecting false queries and considering the video-query relevance.
        On the other hand, query-dependent representations of QD-DETR result in corresponding saliency scores to the video-query relevance and precisely localized moments.
        % On the other hand, our QD-DETR yields a query-dependent representation that the
        % saliency scores are in accordance with the relevance between the video and query.
        % text is in accordance with the saliency scores.
        % There is a large gap between positive and negative saliency scores, and scores are consistent since the clips are all highly correlated to others.
}
    \label{fig:motivation_ex}
\end{figure}


\section{Introduction}
% 원준 원본
% Along with the advance of digital devices and platforms, video is now one of the most desired data type for consumers. However, although the large information capacity of videos may be beneficial in many aspects, e.g., informative and entertaining, on the contrary perspective, videos are time-consuming, and hard to search for desirable moments. 
% This has led many creators to use extra manpower to crop and edit the video to generate highlight clips to gain the consumer’s attention.
Along with the advance of digital devices and platforms, video is now one of the most desired data types for consumers~\cite{apostolidis2021video,wu2017deep}.
% SE: Video aware deep learning application & survey papers?
Although the large information capacity of videos might be beneficial in many aspects, e.g., informative and entertaining, inspecting the videos is time-consuming, so that it is hard to capture the desired moments~\cite{anne2017localizing,apostolidis2021video}. 
% This has led many creators to use extra manpower to crop and edit the video to generate highlight clips to gain the consumer’s attention.


% On the other side, 
Indeed, the need to retrieve user-requested or highlight moments within videos is greatly raised.
Numerous research efforts were put into the search for the requested moments in the video~\cite{anne2017localizing, gao2017tall, liu2015multi, escorcia2019temporal} and summarizing the video highlights~\cite{zhang2016video, mahasseni2017unsupervised, badamdorj2022contrastive, wei2022learning}.
% Numerous research efforts were put into the search for the requested moments in the video~\cite{anne2017localizing, gao2017tall, liu2015multi, escorcia2019temporal}, summarizing the video to generate highlights was another popular topic~\cite{zhang2016video, mahasseni2017unsupervised, badamdorj2022contrastive, wei2022learning}.
Recently, Moment-DETR~\cite{momentdetr} further spotlighted the topic by proposing a QVHighlights dataset that enables the model to perform both tasks, retrieving the moments with their highlight-ness, simultaneously.

% 원준 원본
% To detect the desired moments, previous works employed transformer encoder-decoder architectural designs to fuse the text query into the video representations. Moment-DETR~\cite{mDETR} modified detection transformer to process capture the moment as a set, and UMT~\cite{umt} implemented transformer decoder as to output clip-wise saliency. 
% Yet to their outstanding breakthroughs in the literature of moment retrieval with the seminal architectures, their limitation is that the role of the given text query is insignificant in representing the query-conditioned video representation; the attention mechanism of moment DETR is not explicitly conditioned on the text query, and the text query is conditioned on multi-modal clips where the differences between the clips are smoothed after encoding process in UMT.



% \begin{figure}[t]
% \centering
%     \begin{subfigure}[l]{0.37\linewidth}
%       \centering
%       \vspace{0.20cm}
%     %   \includegraphics[width=1\linewidth]{figs/fig_1_moti_textattn.pdf}  
%     %   \includegraphics[width=1\linewidth]{figs/fig_1_moti_textattn_v2.pdf}  
%       \includegraphics[width=1\linewidth]{figs/fig1_moti_violin_a.pdf}  
%       \vspace{-0.60cm}
%     %   \caption{text attention}
%         \caption{Importance of queries in video representation}
%       \label{fig:fig1_text_attn}
%     \end{subfigure}%\hfill% or  or \hspace{0.3\textwidth}
%     \vspace{0.2cm}
%     \begin{subfigure}[r]{0.61\linewidth}
%       \centering
%     %   \includegraphics[width=1\linewidth]{figs/fig1_moti_negattn.pdf}  
%       \includegraphics[width=1\linewidth]{figs/fig1_moti_violin_b.pdf}  
%     %   \caption{neg attention}
%         % \caption{Relation between the highlight-ness and the relevance between videos and query texts.}
%         \caption{Highlight-ness~(saliency) histogram of positive and negative video-query pairs\SE{}}
%       \label{fig:fig1_neg_attn}
%     \end{subfigure}%\hfill% or  or \hspace{0.3\textwidth}
%     % \vspace{-0.2cm}
%     \caption{Overall statistics for attention scores in Fig.~\ref{fig:motivation_ex} in QVHighlights dataset. 
%     (a) For the attention scores that measure how much the text query is generally involved in video representation, we use violin plots to show the probability density. We plot the score for each layer in the encoder.
%     % (b) Using the histogram, we compare how the baseline and QD-DETR yield different salient scores given the positive and negative video-text pairs.
%     (b) Saliency histogram shows the distributional gap between positive and negative video-text query pairs of baseline~(Moment-DETR) and proposed QD-DETR.\SE{}
%     }
%     \label{fig:motivation}
%     % \captionsetup{belowskip=13pt}
%     % \setlength{\belowcaptionskip}{-10pt}
% \end{figure}

% \begin{figure}[t]
% \centering

%     \begin{subfigure}[r]{1\linewidth}
%       \centering
%       \hspace{-0.2cm}
%     %   \includegraphics[width=1\linewidth]{figs/fig1_moti_negattn.pdf}  
%       \includegraphics[width=1.1\linewidth]{figs/fig1_moti_violin_a_v2.pdf}  
%     %   \caption{neg attention}
%         % \caption{Relation between the highlight-ness and the relevance between videos and query texts.}
%         \vspace{-0.5cm}
%         % \caption{Saliency histogram of positive and negative video-query pairs}
%         \caption{We plot the histograms and its average value~(dotted line) to compare saliency scores when true and false text queries are given for each method. (left) Since the video representations do not include much textual information, both the true and false queries yield similar saliency scores. (Middle) Even when the video representation is enforced to be updated with the textual information, the issue is not much resolved. (Right) By extracting discriminative features in the text query, distributions are differentiated.
%         % \SE{} % R1@0.5 설명
%         Also, R1@0.5 indicates evaluation metric, Recall at 1 with IoU 0.5 threshold on QVhighlight \textit{val} set.
%         }
%       \label{fig:fig1_neg_attn}
%     \end{subfigure}%\hfill% or  or \hspace{0.3\textwidth}
%     \\
%     \begin{tabular}{cc}
%     \hspace{-0.2cm}
%         \begin{minipage}{.4\linewidth}
%             \begin{subfigure}[l]{1\linewidth}
%               \centering
%             %   \vspace{0.20cm}
%             %   \includegraphics[width=1\linewidth]{figs/fig_1_moti_textattn.pdf}  
%             %   \includegraphics[width=1\linewidth]{figs/fig_1_moti_textattn_v2.pdf}  
%               \includegraphics[width=1\linewidth]{figs/fig1_moti_violin_a.pdf}  
%               \vspace{-0.60cm}
%             %   \caption{text attention}
%                 \caption{Importance of queries in video representation}
%               \label{fig:fig1_text_attn}
%             \end{subfigure}%\hfill% or  or \hspace{0.3\textwidth}
%         \end{minipage}
        
%         \begin{minipage}{.6\linewidth}
%             \vspace{-0.2cm}
%             \caption{Overall statistics of Fig.~\ref{fig:motivation_ex} in QVHighlights dataset. 
%             (a) Saliency histogram shows the distributional gap between positive and negative video-text query pairs.
%             % (a) For the attention scores that measure how much the text query is generally involved in video representation, we use violin plots to show the probability density. We plot the score for each layer in the encoder.
%             % (b) Using the histogram, we compare how the baseline and QD-DETR yield different salient scores given the positive and negative video-text pairs.
%             % (b) Text ratio in self-attention layer to  of Moment-DETR
%             % (b) Ratio of text when representing video tokens in self-attention of Moment-DETR.
%             % (b) Magnitude of attention text query involved.
%             % (b) Attention score of video tokens
%             % (b) Magnitude of text query to refine the video tokens in self-attention layer of Moment-DETR.
%             (b) Probability density depicting the weight of the text query in attention score for video clips. Scores are from the self-attention layers in Moment-DETR encoder.
%             % (b) The text query ratio in attention score of video clips (Self-attention layer in Moment-DETR encoder). We use violin plots to show probability density.
%             % 텍스트 쿼리가, 비디오 피쳐에 얼만큼 attend 하는지
%             }
%         \end{minipage}
    
%     \end{tabular}
%     \vspace{-0.5cm}
%     \label{fig:moti}
%     % \captionsetup{belowskip=13pt}
%     % \setlength{\belowcaptionskip}{-10pt}
% \end{figure}


% \begin{figure}
%     \centering
%     % \includegraphics[width=1\linewidth]{figs/fig1_moti_negattn_1109.pdf}
%     \includegraphics[width=1\linewidth]{figs/fig1_moti_negattn_stat_v2.pdf}
%     \vspace{-0.8cm}
%     \caption{
%         Histogram of saliency when the positive and negative queries are given. We plot the histograms and its average value~(dotted line) to compare saliency scores when relevant~(positive) and irrelevant~(negative) text queries are given for each method. (Left) Since the video representations do not properly reflect textual information, both the positive and negative queries yield similar saliency scores. 
%         % (Middle) Even when the video representation is enforced to be updated with the textual information, the issue is not much resolved. 
%         (Right) By representing video clips in query-dependent manner, distributions are differentiated.
%     }
%     \vspace{-0.6cm}
%     \label{fig:motivation}
% \end{figure}


% One of the demanding task is moment retrieval task, which is detecting the desired moments from the given query, typically the text query.
When describing the moment, one of the most favored types of query is the natural language sentence~(text)\cite{anne2017localizing}. 
While early methods utilized convolution networks~\cite{zhang2020learning, gao2021fast, wang2020temporally}, recent approaches have shown that deploying the attention mechanism of transformer architecture is more effective to fuse the text query into the video representation.
% To handle these modalities, previous works simply employed the attention mechanism of transformer architecture to fuse the text query into the video representation.
For example, Moment-DETR~\cite{momentdetr} introduced the transformer architecture which processes both text and video tokens as input by modifying the detection transformer~(DETR), and UMT~\cite{umt} proposed transformer architectures to take multi-modal sources, e.g., video and audio. 
Also, they utilized the text queries in the transformer decoder.
Although they brought breakthroughs in the field of MR/HD with seminal architectures, they overlooked the role of the text query.
To validate our claim, we investigate the Moment-DETR~\cite{momentdetr} in terms of the impact of text query in MR/HD~(Fig.\ref{fig:motivation_ex}).
Given the video clips with a relevant positive query and an irrelevant negative query, we observe that the baseline often neglects the given text query when estimating the query-relevance scores, i.e., saliency scores, for each video clip.
% the output saliency score, i.e. query-relevance scores.
% Based on the observation, we traced the actual saliency prediction of the model against both the video-relevant query and the irrelevant dummy one where we find that the baseline often neglects the given text query when estimating the query-relevance scores of video clips.
% For example, in Fig.~\ref{fig:motivation_ex}, saliency scores are not affected even when the query is substituted with the dummy.
% % General statistics for Fig.~\ref{fig:motivation_ex} is shown in Fig.~\ref{fig:motivation}. 
% General statistics corresponding to Fig.~\ref{fig:motivation_ex} are also shown in Fig.~\ref{fig:motivation}.



% The limitation of the concrete baseline~\cite{momentdetr} is inspected in two different aspects; 1) Utilization of text-query in the encoding process and 2) the output saliency score, i.e. query-relevance scores.
% Firstly, we visualize the attention score when video clips are given as a query in self-attention. 
% We observe that the text queries have relatively small impacts compared to other video features, as shown in Fig.~\ref{fig:fig1_text_attn_ex}.
% That is, the text does not account for much in representing every video clip, although the goal of MR/HD is to detect query-relevant moments.
% Based on the observation, we traced the actual saliency prediction of the model against both the video-relevant query and the irrelevant dummy one where we find that the baseline often neglects the given text query when estimating the query-relevance scores of video clips.
% For example, in Fig.~\ref{fig:motivation_ex}, saliency scores are not affected even when the query is substituted with the dummy.
% % General statistics for Fig.~\ref{fig:motivation_ex} is shown in Fig.~\ref{fig:motivation}. 
% General statistics are also shown in Fig.~\ref{fig:motivation}.

% Consequently, in Fig.~\ref{fig:fig1_neg_attn_ex}~(b), we found that the baseline often neglects the given text query when estimating the query-relevance scores of video clips; 
% For example, 


% We validate the previous work sometimes neglects the given query when estimating the saliency of video clips.
% For example, there is an example that the saliency scores from positive and negative queries cannot be distinguishable, as shown in Fig.~\ref{fig:fig1_neg_attn_ex}.
% % 우리는 추가로 text attention을 추가도 해봤지만, 효과가 있긴 했으나, still 이슈가 있는 것을 확인하였다?
% % Still, we observe that assuring the high attendance of text queries does not resolve the overlap which motivates us to question the quality of the naive use of task-agnostic text representation~\cite{momentdetr, umt}.
% We found that introducing the text-attention for ensuring the high attendance of text queries relieve the overlap, but there still be a severe overlap.


% To validate their limitations, we inspect the impacts of text queries in the concrete baseline~\cite{momentdetr} with the two different aspects, 1) tendency of attention in self-attention layer and 2) saliency score, i.e. query-relevance scores. \SE{} % attention 이 갑자기 등장하는가?
% Firstly, we visualize the attention score when video clips are given as a query in self-attention. We observe the text queries have relatively low attention scores compared to the video features, as shown in Fig.~\ref{fig:fig1_text_attn_ex}.
% That is, the text does not account for much in representing every video clip, although the goal of MR/HD is to detect query-relevant moments.
% Based on this observation, we trace the actual saliency prediction of the model against both positive and negative text queries.
% We validate the previous work sometimes neglects the given query when estimating the saliency of video clips.
% For example, there is an example that the saliency scores from positive and negative queries cannot be distinguishable, as shown in Fig.~\ref{fig:fig1_neg_attn_ex}.
% % 우리는 추가로 text attention을 추가도 해봤지만, 효과가 있긴 했으나, still 이슈가 있는 것을 확인하였다?
% % Still, we observe that assuring the high attendance of text queries does not resolve the overlap which motivates us to question the quality of the naive use of task-agnostic text representation~\cite{momentdetr, umt}.
% We found that introducing the text-attention for ensuring the high attendance of text queries relieve the overlap, but there still be a severe overlap.



% Thus, we 
% query dependency를 높이기 위해 
% Cross-attention? text-attention? detailed explanation on text-attention should be needed?
% By handling these two issues, we find that more precise retrieval can be achieved.
% 
% 
%
% By projecting video-discriminative text features with high text attendance to source video, we f 
% We also find the need to improve the quality of query features since assuring high text attendance also results in...
% pairs are not finetuned to be discriminative that even the similarity within the pairs does not reflect the relevance between the query and the video clips.
% General statistics for Fig.~\ref{fig:motivation_ex} is shown in Fig.~\ref{fig:motivation}. 
% \SE{} % 이거 ??로 뜨는데, 위처럼 figure 그리면 label이 안되는걸까요
% \SE{}
% 형님 아래 사항 생각 좀 해보는게 좋을 거 같아요.
% fig 1. (a) 그림만 봤을 때 모든 clip에 대해 text attention이 일정이상 존재하긴 하니까, 뭔가 not assured to be conditioned가 와닿지 않는거 같아요.
% + 왜 text가 항상 attend 해야하나?
% not assured to be conditioned --> text shows relatively low affects compared to video 같이 실제 나타난 현상까지 같이 적으면 어떨까 싶어요.
% fig 1. (b) 덜 반영한다?

% \SU{}
% 일단 text가 attend 잘 되어야 한다는 것에 좀 궁금점이 생깁니다. 결국에는 text와 관련있는 frame들을 attend해서 higlight를 찾아야 하는게 아닐까요? 그리고, 현제 저희의 모델 구조상 text query가 Key와 Value로 거의 활용되고 있는데 그렇다면 결국에는 해당 모델은 text에 대한 attention이 전혀 없다고 봐도 무방하지 않을까요? 그런 면에서 text attention을 강조하는게 좀 걸리긴 합니다.

% Specifically, the text query is not assured to be explicitly conditioned on every clip of the video, and as the query texts are evenly treated, discriminative keywords may not be spotlighted.
% attention mechanism of Moment-DETR is not explicitly conditioned on the text query as shown in Fig~\ref{}(d), and in UMT, the text are only used for conditioning the queries while the video representation are refined itself by self-attention.

% \begin{figure}[t]
%     \begin{subfigure}{1\linewidth}
%       \centering
%     %   \includegraphics[width=1\linewidth]{figs/fig_1_moti_textattn.pdf}  
%     %   \includegraphics[width=1\linewidth]{figs/fig_1_moti_textattn_v2.pdf}  
%       \includegraphics[width=1\linewidth]{figs/fig_1_moti_textattn_v4.pdf}  
%       \vspace{-0.5cm}
%     %   \caption{text attention}
%         \caption{Distribution of attention scores in Moment-DETR encoder}
%       \label{fig:fig1_text_attn}
%     \end{subfigure}%\hfill% or  or \hspace{0.3\textwidth}
%     \vspace{0.2cm}
%     \begin{subfigure}{1\linewidth}
%       \centering
%     %   \includegraphics[width=1\linewidth]{figs/fig1_moti_negattn.pdf}  
%       \includegraphics[width=1\linewidth]{figs/fig1_moti_negattn_v2.pdf}  
%       \vspace{-0.5cm}
%     %   \caption{neg attention}
%         \caption{Saliency score against positive and negative text queries}
%       \label{fig:fig1_neg_attn}
%     \end{subfigure}%\hfill% or  or \hspace{0.3\textwidth}
%     \vspace{0.2cm}
%     \begin{subfigure}{1\linewidth}
%       \centering
%     %   \includegraphics[width=1\linewidth]{figs/fig1_moti_violin.pdf}  
%       \includegraphics[width=1\linewidth]{figs/fig1_moti_violin_v2.pdf}  
%       \vspace{-0.5cm}
%       \caption{violin}
%       \label{fig:fig1_violin}
%     \end{subfigure}%\hfill% or  or \hspace{0.3\textwidth}
%     \vspace{-0.2cm}
%     \caption{(a) 1. portion of text attention vs. video attention 2. relation with text query and content (e.g. fg, bg) of clip seems not to affect the attention score
%     (b) 1. high variability even though entire clips are highly correlated with the given text query 2. positive and negative query makes overlaps on saliency score distribution
%     (3) actual distribution on validation dataset.}
%     \label{fig:motivation}
%     % \captionsetup{belowskip=13pt}
%     % \setlength{\belowcaptionskip}{-10pt}
% \end{figure}

To this end, we propose Query-Dependent DETR~(QD-DETR) that produces query-dependent video representation.
% Our key focus is to ensure each clip in predicted moments is explicitly conditioned by the query, particularly on the video-descriptive portion of the text query.
% Our key focus is to ensure that query-relevant clips are predicted by enforcing each clip to be explicitly conditioned by the query.
%Our key focus is to ensure that the model prediction for each clip is highly relevant to the query.
Our key focus is to ensure that the model's prediction for each clip is highly dependent on the query.
% by enforcing each clip to be explicitly conditioned by the query. :)
% hmm...
% \SE {} % "query-relevant clips are predicted" 이 문장이 좀 애매한거 같습니다. relevant 클립을 놓지지 않고 찾는 것을 보장한다? 이런 느낌인지 아니면 높은 saliency 를 주는게 목적이다? model prediction이 query-relevance를 반영하는 것을 보장한다?
% Our key focus is to ensure that the model prediction reflects query-relevance of clips by enforcing each clip to be explicitly conditioned by the query.
First, to fully utilize the contextual information in the query, we revise the transformer encoder to be equipped with cross-attention layers at the very first layers.
% 상익's thought :  single video - query간의 관계만 고려 - 같은 word가 더 많이 쓰이는 것을 보고 
% 교수님's thought : neg pair 를 쓰면 쿼리를 보지 않고서는 video clip간만 고려하는 것이 사라짐. 왜냐면 0으로 내보내야 하기 때문. --> SE: relative difference 만 고려하다가, 
By inserting a video as the query and a text as the key and value of the cross-attention layers, our encoder enforces the engagement of the text query in extracting video representation.
% 원준 교수님 코멘트 반영해서 다시
Then, in order to not only inject a lot of textual information into the video feature but also make it fully exploited, we leverage the negative video-query pairs generated by mixing the original pairs.
Specifically, the model is learned to suppress the saliency scores of such  negative~(irrelevant) pairs.
Our expectation is the increased contribution of the text query in prediction since the videos will be sometimes required to yield high saliency scores and sometimes low ones depending on whether the text query is relevant or not.
% \SE{}
% learns to?
% By suppressing the saliency scores of the irrelevant video-query pairs, the model learns to spotlight only the video-specific discriminative words in the query.
% % \SE{} % ====================== 상익 수정 ========================
% However, this architectural design still lacks the capability of identifying the video-descriptive keywords in the query.
% % However, this architectural design still lacks in identifying proper query relevance.
% This is because the current training scheme only focuses on the interactions of video and clips within a single video while neglecting information shared throughout the entire video.
% % We argue the problem of the current training scheme that only focuses on distinguishing the clips in a single video while neglecting information shared throughout the entire video.
% Therefore, we leverage the negative video-query relationships to enhance the capability of identifying the contextual similarity of query and video clips.
% 
% 원준 원본 
% However, this architectural design heavily relies on the quality of the text query.
% Therefore, we leverage the negative video-query relationships to enable the model to emphasize key corresponding query features.
% By suppressing the saliency scores of the irrelevant video-query pairs, the model learns to spotlight only the video-specific discriminative words in the query.
% =========================================================
Lastly, to apply the dynamic criterion to mark highlights for each instance, we deploy a saliency token to represent the entire video and utilize it as an input-adaptive saliency criterion. 
With all components combined, our QD-DETR produces query-dependent video representation by integrating source and query modalities.
This further allows the use of positional queries~\cite{dabdetr} in the transformer decoder.
% Furthermore, we can exploit the advanced DETR decoder architectures using the positional information, e.g., DAB-DETR, since our encoded tokens consist of identical position representations from a single modality.
% \SE{} % ====================== 상익 수정 ========================
% Furthermore, we can exploit the advanced DETR decoder architectures using the positional information, e.g., DAB-DETR, since our video clip tokens consist of identical position representations from a single modality.
% 원준 원본
% It also enables the use of advanced DETR decoder architectures, e.g., DAB-DETR, for the first time, as these works exploit the position information within a single modality.
% =========================================================
Overall, our superior performances over the existing approaches validate the significance of the role of text query for MR/HD.
% Our extensive experiments on QVHighlights, TVSum, and Charades-STA datasets validate the significance of considering the role and the quality of text query.

% All components combined with dynamic anchor moments for the query of decoder, our FOQUE fosters the query-dependent video representation, thereby making the 
% All components combined, our modified transformer encoding process fosters the query-dependent video representation thereby achieving the state-of-the-art results on various benchmarks of moment-retrieval and highlight detection.
	
% -	Video Platform & Streamer & Consumer의 증가. 
% Video는 다른 데이터 타입보다 정보가 많아 유용하지만, 이는 다른 말로 해석하면 video를 보는 것은 time-consuming 하고, 원하는 것을 찾아보기에는 힘들 수 있음.
% 따라서, 많은 매체에서는 사람들의 더 많은 이목을 끌기 위해 highlight 비디오라는 것을 편집하여 공유도 함.
% 하지만, highlight video를 만들기 위해 사람의 노력이 필요한 현 시점에서, This spotlights the need to retrieve the user-requested / Highlight moments in the video.

% -	이전에도 이러한 문제를 해결하기 위해 (asdfasdf) for moment retrieval, (asdfasdf) for highlight detection 등이 제안 되었지만, 이들은 비디오의 특정 영역을 찾는다는 공통된 목적을 가지고 있으면서도, 데이터 셋의 한계로 인해 따로 연구되었음. 이를 문제 삼으며, 최근에는 두 task를 동시에 학습할 수 있는 dataset이 소개 되었는데, 컴퓨터비전에서 최근 각광을 받고 있는 Transformer 모델 도입과 함께 큰 발전을 거듭하고 있음.

% -	구체적으로, 이 두가지 task를 수행하기 위해서는 transformer를 두가지 방법으로 이용할 수 있는데, moment-DETR 처럼 moment 를 clip의 set 단위로 예측할 수 있고, UMT 처럼 clip-wise prediction을 할 수 있음. 하지만, 이들은 query를 condition이 아닌 video와 동등한 레벨로 취급하거나 [mDETR], 매 클립이 self-attention으로 mixing 된 후에 condition을 걸어주어 clip간의 차이를 확실하지 이용하지 못하였고, 또한, 확실하게 condition으로 주지 못하였고, video와 query 사이의 관계를 한정적으로만 이용하였다.

% -	따라서, we explore three different ways to fully exploit query information. First, we design one-way cross-attention layer to condition every clip with the query features. Then, we utilized the negative video-text pairs to better model the relationships between the video and the text embeddings. Lastly, we define the saliency token to be the video-query dependent saliency estimator.


















% ===================== neg pair 부분 ===========================
% Nevertheless, the current training scheme, only considering the given video-query pair, still disturbs the model from identifying proper query-relevance prediction.
% In detail, the model focus on learning the fine-grained discrepancy between video clips, while neglecting the information they share, which contains significant clues to understand the context of video.
% Therefore, we leverage the negative video-query relationships to enhance the capability of identifying the contextual similarity of query and video clips.
% Therefore, we leverage the negative video-query relationships by suppressing those pairs, so that enhance the capability of identifying the contextual similarity of query and video clips.
% We hypothsize the diversity in query-video pairs are insufficient to learn the general relationship between text query and video.
% Therefore, we leverage the negative video-query relationships by suppressing the saliency scores of the irrelevant video-query pairs.
% However, this architectural design still lacks in identifying proper query relevance.
% We argue that the current training scheme only focuses on learning the fine-grained discrepancy between clips in a single video, while neglecting the information they share, which contains significant clues to understand the context of the video.
% Therefore, we leverage the negative video-query relationships to enhance the capability of identifying the contextual similarity of query and video clips.
% However, this architectural design still lacks in identifying proper query relevance.
% We argue the problem of the current training scheme that only focuses on learning the fine-grained discrepancy between clips in a single video.
% That is, the current design neglects the information shared throughout the video, although it contains significant clues to understand the context of the video.
\section{Related Works}

\begin{figure*}[!ht]
\centering
\includegraphics[width=\linewidth]{body/figures/data_collection2.png}
\caption{\textbf{Hardware Setup.} We use a GelSight Wedge sensor for tactile sensing, an Intel ReslSense D405 camera mounted on the side for RGB vision sensing, and an OptiTrack setup for motion capture. \textbf{Data Collection.} The tactile finger and the camera are fixed to the table at all times. A human operator moves a test object and presses it against the finger. We show sampled tactile and RGB images as well as a reconstructed local tactile depth map on the right.}
\label{datacollection}
\end{figure*}

% \textbf{Tactile sensors}:
% Over the years, researchers have developed tactile sensors working on different sensing principles, such as resistance, capacitance, magnetic, barometric, and optic.
% We refer readers to \cite{kappassov2015tactile} for an in-depth review of different types of tactile sensors and their applications.
% Compared to other sensing principles, GelSight tactile sensors have the advantage of providing high-resolution geometrical information of the contact surface.
% They are usually constructed with an elastic silicone gel, directional colored LEDs, and a camera pointing at the gel.
% The gels are usually coated with reflective paint with printed dots.
% When in contact, the gel deforms and takes the shape of the contact surface.
% Shear force can be retrieved by tracking dot movements.
% Furthermore, the color value of a pixel is correlated with the gradient of the height of the contact surface at the specific location. 
% With a pre-calibrated color table, a depth map can be reconstructed from the color image.
% This type of tactile sensor is selected for our work for its rich output and ease to use.

Researchers of the robotics community have put forward a wide range of tactile sensing solutions.
Sensors working on different sensing principles have been adopted to solve a large set of manipulation tasks.
Among different types of tactile sensors, vision-based ones such as GelSight \cite{yuan2017gelsight} and GelSlim \cite{donlon2018gelslim} stand out for their rich output, ease to use, and affordability.
While we focus on the pose estimation and shape reconstruction task using vision-based tactile sensors, we refer readers to \cite{kappassov2015tactile} for an in-depth review of different types of tactile sensing and their applications.
In this section, we review works on three typical tasks that are most relevant to our solution: slip detection, object property inference, and SLAM.

% Researchers have found that using this class of vision-based tactile sensors can greatly increase the accuracy when reasoning about the contact surface, compared to traditional tactile sensors that are constructed with normal direction force sensors [\todo{add citation}].

\textbf{Slip detection and estimation}:
Using a similar sensor to ours, Yuan \etal compared and analyzed a GelSight tactile sensor's images collected at different stages of slip in \cite{yuan2015measurement} and showed this type of sensors' capability in detecting micro scale movements.
Li \etal and Zhang \etal trained recurrent neural networks on tactile images to detect slip between multiple time steps in a manipulation sequence \cite{li2018slip, zhang2018fingervision}.
Built on their binary slip detection model in \cite{li2018slip}, Li further added rotational slip direction prediction in \cite{li2019rotational}.
Calandra \etal improved a grasp planner for the classic robot bin-picking problem by incorporating slip detection and achieved a higher grasp success rate \cite{calandra2017feeling}. 
However, those methods only detect slip without localizing the object after the slip.
In many precision manipulation tasks we are also interested in the amount of the displacement.

\textbf{Object property inference and localization}:
% With detailed information on the contact surface provided by high-resolution tactile sensors, 
Many works have focused on inferring properties of the in-contact object, such as shape \cite{strub2014using, luo2015tactile, luo2019iclap}, texture \cite{luo2018vitac, yuan2017connecting}, and material \cite{yuan2017connecting, kroemer2011learning, kerr2018material}.
Those learned object properties can be further used for localization.
In order to localize current grasps, Bauza \etal proposed to match new tactile imprints with previously collected tactile imprints \cite{bauza2019tactile}, while Luo \etal learned to match tactile imprints directly to visual images of the whole object \cite{luo2015localizing}.
Assuming known CAD models, Bauza \etal proposed to localize by comparing contact masks generated from tactile images with a large bank of random projections of the CAD model \cite{bauza2022tac2pose}.
To solve the reverse problem, i.e. what a tactile image looks like given an object and a pose, several tactile simulators have been built to automatically generate tactile images given an object's CAD model and a finger pose \cite{si2022taxim, wang2022tacto}.
One major limitation for this category of works is that they all require a known calibrated geometry of the object: a pre-collected tactile map \cite{bauza2019tactile}, a model of the object \cite{bauza2022tac2pose}, or a global image with known geometry \cite{luo2015localizing}.
This requirement can be hard to meet in less constraint environments.

\textbf{Tactile SLAM}:
Recent studies have shown interests in working with unknown objects by leveraging methods from the SLAM problem.
With a focus on 2D shapes, Suresh \etal parameterized shapes as Gaussian Process Implicit Surfaces (GPIS), and learned its parameters from tactile signals collected during pushing \cite{suresh2021tactile}.
Assuming known contact poses, authors of \cite{suresh2022shapemap} first learned a noisy mapping from known surface geometries to corresponding tactile images, then reconstructed an object by combining many noisy local tactile measurements into an optimized global shape using factor graph optimization.
The closest prior work to ours is \cite{sodhi2022patchgraph}, where the authors learned to estimate 6D poses and 3D shapes simultaneously for unknown objects. 
They constructed a pose estimator based on tactile sensing, and a shape reconstruction pipeline that added in new tactile point clouds incrementally on the run.
However, this approach heavily relies on the performance of the tactile pose estimator, which lacks a global understanding of the object and can suffer from repeated patterns or smooth surfaces.
In contrast, our work combines vision and tactile sensing which provides us with both global and local understandings of the scene without requiring any other domain knowledge.
Furthermore, we designed a loop closure mechanism that periodically matches current tactile and vision images to stored key-frames, which significantly reduced accumulated errors.
With this, FingerSLAM is able to produce realistic reconstructions even in long sequences. 
\section{Preliminary}
\label{sec_an_overview_of_cil}

The following is the training pipeline of standard CIL with few-shot exemplars.
%
Assume there are $N$ learning phases. %: one initial phase and $N\!-\!1$ subsequent phases.
%
In the $1$-st phase, we load data $\mathcal{D}_{1}$ containing all training samples of $c_1$ classes, and use $\mathcal{D}_{1}$ to train the initial classification model $(\theta_1, \omega_1)$, where $\theta_1$ and $\omega_1$ 
 denote the parameters of the feature extractor and classifier, respectively. 
%
When the training is done, we evaluate the model performance on the test samples of $c_1$ classes. Before the $2$-nd phase, we discard most of the training samples due to the strict memory budget of CIL.
%
In other words, we preserve only a handful of training samples $\mathcal{E}_1$ (i.e., exemplars) in the memory, selected from $\mathcal{D}_1$. A common method for selecting exemplars is called feature herding~\cite{rebuffi2017icarl} and has been used in many related works~\cite{liu2020mnemonics,yan2021dynamically,wang2022memory,wang2022foster}. We adopt it, too, in this work.
%
In the $i$-th phase ($i\geq2$), we load all exemplars ${\mathcal E}_{1:i-1}=\mathcal{E}_1\cup \dots\cup {\mathcal E}_{i-1}$ from the memory and initialize the current model $(\theta_{i},\omega_{i})$ by the previous model $(\theta_{i-1},\omega_{i-1})$. 
%
We use $\mathcal{E}_{1:i-1}$ and the new coming data $\mathcal D_i$ (containing $c_i$ new classes) to train $(\theta_{i},\omega_{i})$.% as follows,
%
Then, we evaluate the current model using a test set of all $\sum_{j=1}^{i}c_j$ classes seen so far. 
%
After that, we discard most of the training samples in $\mathcal D_i$, and leave few-shot exemplars ${\mathcal E}_{i}$ in the memory. 
%
It is clear that this discarding causes a strong data imbalance between old and new coming classes in the subsequent phase. In the following, we introduce our solution to this problem.



\section{Methodology}
\label{sec:methods}
%\subsection{Problem Formulation}
\subsection{Overview}
The objective of this work is to build an illumination-robust few-shot view synthesis framework by regularizing intrinsic components that should be identical across multi-view images regardless of illumination.
However, there exist three major challenges in comparing the albedo extracted from the novel synthesized view and those of input images: 1) Since our proposed method requires pixel-to-pixel correspondence between different views, a geometric alignment is needed to select the pixel to compare. 2) There always exists a non-intersecting or occluded region between the novel view and the input view images, which should be considered during cross-view regularization. 3) The key feature for successful intrinsic decomposition is the global context of a scene. However, the NeRF-based structure has difficulties in rendering full-resolution images since continuous ray sampling upon the entire image requires massive computational and memory costs.

To address these problems, we suggest a few-shot view synthesis framework that utilizes an offline intrinsic decomposition network, providing global context-aware pseudo-albedo ground truth without the computational overhead. As illustrated in Fig.~\ref{fig:overallframework}, FIDNet~\cite{li2018iidww} provides pseudo-albedo ground truths for the input images before the start of the training. Then PIDNet learns to extract intrinsic components of the synthesized patch, given the pseudo-albedo ground truth of the corresponding patch depicting the same 3D surface. As a result, our NeRF learns illumination-robust few-shot view synthesis. 

In the following subsections, we introduce each component of our framework in detail.  





\subsection{Intrinsic Consistency Regularization}
\paragraph{Geometric alignment.}
As illustrated in Fig.~\ref{fig:overallframework}, a pair of images is selected from a set of inputs and randomly generated novel views, for every iteration.
In order to get a pixel correspondence between the two, we use a projective transformation, similar to several concurrent works~\cite{MVCGAN, xu2022sinnerf}. Given a pixel $x$ in the novel view, we need the corresponding image pixel $x'$ in the input view depicting the same 3D point. If the depth of a given image pixel is known, $x'$ can be obtained using projective transformation as follows:
\begin{equation}
     x' = (KT'^{-1}T)d(x)K^{-1}x ,
\label{eq:homography}
\end{equation}
where $K$ indicates camera intrinsics and $T$ and $T'$ indicate camera-to-world matrices of the novel view and input view. Both $K$ and $T$ are given for the calibrated images. \vspace{-10pt}


\paragraph{Albedo consistency.}
Based on the pixel correspondence obtained above, we can impose image consistency between inputs and novel views.
However, under varying illumination, Eq.~\ref{eq:colorreg} cannot regularize view-dependent color as it does under constrained illumination, for its different interactions within illumination (see Fig.~\ref{fig:intrinsics}).
To overcome this, we present $L2$ normalized albedo consistency loss $\mathcal{L}_{\mathrm{ac}}$ formulated as follows:
\begin{equation}
  \mathcal{L}_{\mathrm{ac}} = \sum_{x \in \mathcal{P}} \omega_{\mathrm{occ}}(x)\|\hat{a}(x) - \hat{a}(x')\|^2_2,
  \label{eq:ac}
\end{equation}
where $\hat{a}(x)$ and $\hat{a}(x')$ indicate the extracted albedo at $x$ and $x'$ from the novel view and input view, respectively.
$\mathit{w}_{\mathrm{occ}}(x)$ indicates the weight term to consider inaccurate correspondences coming from occlusions or out-of-region pixels, while $\mathcal{P}$ denotes all the pixels in the novel view. Details of $\mathit{w}_{\mathrm{occ}}(x)$ are in the following.
\vspace{-10pt}

\paragraph{Occlusion handling.}
Eq.~\ref{eq:homography} described above byproducts $\Tilde{d}(x')$, a depth value at pixel $x'$ in the input view. $\hat{d}(x')$, a synthesized depth value at $x'$ should be identical to $\Tilde{d}(x')$ if there exists neither self-occlusion nor ill-synthesized floating artifacts. 
For all cases, a projection error on $x'$, denote by $\mathcal{E}_\mathrm{proj}$ can by defined as
\begin{equation}    
\begin{array}{cc}
     \mathcal{E}_\mathrm{proj} = (\hat{d}(x') - \Tilde{d}(x'))^2.
\end{array}
  \label{eq:error}
\end{equation}
However, a problem exists in that both inaccurate correspondence and occlusion cause large projection errors.
In order to minimize $\mathcal{E}_\mathrm{proj}$ that came from inaccurate correspondences while protecting the pixel pairs with occlusion, we define an occlusion-aware weight term $\mathit{w}_{\mathrm{occ}}$, on a pixel $x'$ of input view, as
\begin{equation}    
\begin{array}{cc}
     \omega_{\mathrm{occ}} = \mathit{r}_{\mathrm{e}}(1 - (\mathcal{E}_\mathrm{proj}/\mathcal{M}_{\mathrm{proj}})), % 수식 정리 필요
\end{array}
  \label{eq:weight}
\end{equation}
where $\mathit{r}_{\mathrm{e}}$ and $\mathcal{M}_{\mathrm{proj}}$ indicates the error rate coefficient and the maximum value of $\mathcal{E}_\mathrm{proj}(x)$, respectively. By using $\mathit{w}_{\mathrm{occ}}$, the input-novel view pairs with occlusion that are likely to have large projection errors will not be enforced to have albedo consistency.  $\mathit{r}_{\mathrm{e}}$ decays from $1$ to the end criteria ($0.5$), reducing the number of pairs that are enforced to have the same albedo, since inaccurate correspondence pairs will decrease while training. 
\vspace{-10pt}

\begin{figure}
  \centering
  % \begin{tabular}{>{\centering\arraybackslash}p{.29\linewidth}>{\centering\arraybackslash}p{.27\linewidth}>{\centering\arraybackslash}p{.27 \linewidth}}
  %   \scriptsize Input & \scriptsize Albedo & \scriptsize Shading \\
  %  \end{tabular}
  \includegraphics[width=\linewidth]{figure/ICCV/intrinsics2.pdf}
  \vspace{-15pt}
\caption{ \textbf{Examples of intrinsic decomposition on NeRF Extreme.}
   %Despite the imperfect decomposition results, 
   Estimated albedo maps can provide appearances that are more illumination-invariant than input color images, as shown in a lower difference across multi-views.}
\label{fig:intrinsics} \vspace{-10pt}
\end{figure}

\paragraph{Depth consistency.}
Geometric alignment in our setup may utilize incorrect synthesized depth values that are commonly observed in novel view synthesis. 
In order to prevent the model to enforce consistency between pixels with inaccurate correspondence, and to efficiently correct ill-synthesized scene geometry, we present a depth consistency loss $\mathcal{L}_{\mathrm{dc}}$.
A direct minimization of $\mathcal{E}_\mathrm{proj}$, however, can be counterproductive due to occlusion, by smoothing two unrelated surface depths.
Through experimentation, we have discovered that total variation normalization on projection error can better regularize the scene geometry, successfully reducing floating artifacts without suffering adverse effects from occlusion.
Depth consistency loss $\mathcal{L}_{\mathrm{dc}}$ in the input view can be defined such that 
\begin{equation}
     \mathcal{L}_{\mathrm{dc}} = \sum_{x' \in \mathcal{P}'} \sum_{y \in \mathcal{N}(x')}(\mathcal{E}_\mathrm{proj}(y)-\mathcal{E}_\mathrm{proj}(x'))^2 ,
\label{eq:dc}
\end{equation}
where $y$ indicates one of the 4-neighbor adjacent pixels $\mathcal{N}(x')$ for $x'$. $\mathcal{P}'$ denotes all the pixels in the input view. Details of the projection error cases and depth regularization can be found in the supplementary materials.



\subsection{Albedo Estimation}
As discussed previously, a successful intrinsic decomposition inevitably requires the global context of a scene, while NeRF struggles with handling large-resolution inputs.
Moreover, intrinsic rendering methods~\cite{nerfinthewild, boss2021nerd, NeuralPIL, boss2022-samurai, ye2022intrinsicnerf} cannot be used in our case due to the lack of input data (uses 0.06 times fewer data).
To address these challenges without imposing a computational burden or requiring supervision, we propose a two-stage intrinsic decomposition pipeline: a full-image and patch-wise intrinsic decomposition network called FIDNet and PIDNet. Before training begins, FIDNet extracts the intrinsic components of the input images - pseudo-albedo ground truths - offline, in order to provide guidance to PIDNet with global contexts. During training, PIDNet extracts patch-wise intrinsic components ($\hat{a}(x)$) of the synthesized color patch at the novel view ($\hat{c}(x)$). Given the pseudo-albedo ground truth provided by FIDNet, PIDNet is trained to minimize $\mathcal{L}_{\mathrm{ac}}$ (Eq.~\ref{eq:ac}).
Specifically, FIDNet~\cite{li2018iidww} uses a shared encoder and $2$ decoders, one for log-scale albedo and the other for shading images. The network also predicts a 3-dimensional light color $c$ as a side output. PIDNet follows the architecture with a shallower structure. 
%Note that the official implementation and the pre-trained model of IIDWW~\cite{li2018iidww} have been used without fine-tuning for FIDNet. 


\subsection{Total Loss Functions} 
In addition to albedo consistency loss and depth consistency loss, an edge-preserving loss $\mathcal{L}_{\mathrm{edge}}$~\cite{godard2017unsupervised}, an intrinsic smoothness loss $\mathcal{L}_{\mathrm{pid}}$~\cite{li2018iidww}, and chromaticity consistency loss $\mathcal{L}_{\mathrm{chrom}}$~\cite{ye2022intrinsicnerf} are also used with color consistency loss $\mathcal{L}_{\mathrm{color}}$ and depth smoothness loss $\mathcal{L}_{\mathrm{ds}}$. An edge-preserving loss $\mathcal{L}_{\mathrm{edge}}$ gives the constraint that gradients of the novel synthesized view (i.e. edge) should be identical to the one of the input view. A patch-wise intrinsic smoothness loss $\mathcal{L}_{\mathrm{pid}}$ is formulated in the same way as depth consistency loss. A patch-wise chromaticity consistency loss $\mathcal{L}_{\mathrm{chrom}}$ gives the constraint that the chromaticity of the input patch and the extracted albedo is the same.

% chromaticity loss 내용 추가 / 
\section{Experiments}
In this section, we evaluate the effectiveness of pFedPT and compare it with several advanced methods in various datasets and settings. We also conduct a number of exploratory experiments to find out how pFedPT works and verify the effectiveness of pFedPT in terms of  client data distribution. The detailed experimental setup and the more experimental results can be found in the Appendix.





\begin{table*}[!ht]
\centering
\captionsetup{font=small}
\caption{The results of pFedPT and baseline methods on the image datasets with different non-IID settings}
\vspace{-0.8em}
% \ls{split this table as three tables with three types of datasets}}
\label{results}
\renewcommand\arraystretch{1.25}
\resizebox{0.9\textwidth}{!}{
\begin{tabular}{lccccccccccccc}
\hline
\textbf{} & \multicolumn{6}{c}{CIFAR10} & \multicolumn{6}{c}{CIFAR100} &\\ \cline{2-13} 
\#setting & \multicolumn{2}{c}{IID} & \multicolumn{2}{c}{Dirichlet } & \multicolumn{2}{c}{Pathological} & \multicolumn{2}{c}{IID} & \multicolumn{2}{c}{Dirichlet } & \multicolumn{2}{c}{Pathological}  \\ \cline{2-13} 
\#client & ViT & CNN & ViT & CNN & ViT & CNN & ViT & CNN & ViT & CNN & ViT & CNN \\ \hline

FedAvg  & 60.50 & \textbf{67.13 } & 53.01  & 61.92  & 54.98  & 63.62 & 29.60 & 26.42 & 25.93 & 26.50 & 27.71 & 30.28  \\
FedProx  & 57.04 & 66.94  & 53.14  & 61.95  & 55.02  & 63.29  & 27.71 & 26.29 & 26.00  & 26.48 & 27.84 & 30.52  \\


% FedBABU+PT  & 60.75  & 87.04  & 80.45  & 90.05  & 78.50  & 22.49 & 48.80 & 35.07 & 48.61 & 34.52  \\
MOON  & {60.99} & 66.88  & 61.12  & 62.53  & 65.98  & 63.52 & 29.32 & \textbf{26.43} & 24.95 & 26.93 & 27.61 & 29.00  \\

% FedAMP  & 37.54  & 66.31  & 59.75  & 62.01  & 39.46  & 1.47 & 41.04 & 16.38 & 23.35 & 14.53  \\
% FedAMP+PT  & 21.09  & 78.46  & 64.23  & 74.33  & 46.97  & 7.03 & 35.43 & 21.63 & 20.17 & 8.48  \\

% FedMTL+PT  & 45.65  & 85.33  & 73.94  & 87.46  & 70.17  & 7.58 & 44.88 & 25.82 & 41.06 & 25.40  \\
FedPer  & \textbf{61.57} & {51.46 } & {73.16}  & 77.98  & {75.20}  & 79.97  & 29.74 & 10.82 & {36.78} & {27.79} & 35.36 & 31.13  \\

FedRep  & 48.38 & {49.70 } & 74.11  & 77.65 & 74.48  & 78.39  & 17.84 & 9.13 & 35.06 & {27.39} & 36.13 & 32.41  \\
FedMTL  & 45.65 & 45.65  & 68.48 & 73.95  & 65.39   & 70.94 & 17.91  & 7.34 & 26.08 & 25.85 & 25.46 & 26.32  \\
% FedRoD  & {65.43 } & 88.28  & 81.36  & 91.10  & {82.35 } & 20.19 & 49.31 & 34.55 & 49.71 & 37.25 \\
% FedRoD+PT  & {65.90} & 78.50  & 76.01  & 52.64  & {65.97} & 26.34 & 32.35 & 27.31 & 28.60 & 29.58 \\
FedBABU  & 50.41 & 61.17  & 74.21  & 80.11 & 74.30  & 80.69 & 20.61 & 22.55 & 36.17 & 31.66 & 35.74 & 35.45  \\
Local  & 45.37 & 39.04  & 68.40  & 73.98  & 64.83  & 70.76  & 18.01 & 7.33 & 26.23 & 25.15 & 24.65 & 25.34  \\
% Local+PT  & 39.56  & 85.29  & 73.88  & 87.26  & 70.18  & 7.74 & 44.86 & 25.71 & 41.04 & 25.62  \\
\hline
pFedPT (ours) & 60.01  & 66.09 & \textbf{74.92} & \textbf{80.83} & \textbf{75.42} & \textbf{81.16} & \textbf{31.66} & {26.41} & \textbf{36.80} & \textbf{32.47} & \textbf{36.88} & \textbf{37.98}\\ 

\hline
\end{tabular}
}
\end{table*}



\begin{table}[!ht]
\centering
\captionsetup{font=small}
\caption{The results of baseline methods with prompts on the image datasets with CNN model in Non-IID settings.}
\vspace{-0.8em}
% \ls{split this table as three tables with three types of datasets}}
\label{results+pt}
\renewcommand\arraystretch{1.25}
\resizebox{0.45\textwidth}{!}{
\begin{tabular}{lccccc}
\hline
\textbf{} & \multicolumn{2}{c}{CIFAR10} & \multicolumn{2}{c}{CIFAR100} &\\ \cline{2-5} 
\#setting  & \multicolumn{1}{c}{Dirichlet } & \multicolumn{1}{c}{Pathological} &  \multicolumn{1}{c}{Dirichlet } & \multicolumn{1}{c}{Pathological}  \\ \cline{2-5} 
 \hline

FedProx     & 61.95    & 63.29   & 26.48  & 30.52  \\
FedProx+PT    & \textbf{80.47}    & \textbf{81.48}    & \textbf{31.95}  & \textbf{37.88}  \\
\hline
MOON     & 62.53    & 63.52   & 26.93  & 29.00  \\
MOON+PT     & \textbf{77.84}   & \textbf{76.00}    & \textbf{28.67}  & \textbf{34.60}  \\
\hline
FedPer     & 77.98    & 79.97    & {27.79} & 31.13  \\
FedPer+PT     & \textbf{78.40}    & \textbf{80.59}    & \textbf{28.83}  & \textbf{31.14}  \\
\hline
FedRep     & 77.65  & 78.39     & {27.39}  & 32.41  \\
FedRep+PT      & 77.65    & \textbf{79.11 }   & \textbf{29.19}  & \textbf{32.75}  \\
\hline
\end{tabular}
}
\end{table}








\subsection{Experimental Setup}

\paragraph{Baselines.}\ 
We compare the pFedPT with several advanced FL  methods, including FedAvg~\cite{mcmahan2017communication}, 
FedProx~\cite{li2020federated}.
MOON~\cite{li2021model},  FedPer~\cite{arivazhagan2019federated}, 
FedRep~\cite{collins2021exploiting},
FedMTL~\cite{smith2017federated} and FedBABU~\cite{oh2021fedbabu}. We also compare a baseline named \textbf{Local}, where each client trains a model with its local data without federated learning. We conduct experiments on two benchmark datasets:
CIFAR10~\cite{krizhevsky2009learning} and CIFAR100~\cite{krizhevsky2009learning}. CIFAR100 is a more difficult dataset for classification tasks than CIFAR10. We use PyTorch~\cite{oord2018representation} to implement pFedPT and the other baselines.


\paragraph{Datasets.} We consider two different scenarios for simulating non-identical data distributions (Non-IID) across federated clients. Dirichlet Partition follows works~\cite{hsu2019measuring}, where we partition the training data according to a Dirichlet distribution Dir($\alpha$) for each client and generate the corresponding test data for each client following the same distribution. We specify $\alpha$ equal 0.3 for each dataset. In addition, we evaluate with the pathological partition setup similar to \cite{zhang2020personalized}, in which each client is only assigned a limited number of classes at random from the total number of classes. We specify that each client possesses 5 classes for CIFAR10 and 50 classes for CIFAR100. 


\paragraph{Implementation Details.}\ 
We verify the experimental results based on CNN and ViT architectures. The CNN model consists of 2 convolutional layers with 64 $\times$ filters followed by 2 fully connected layers with 394 and 192 neurons and a softmax	layer. We use tiny ViT architecture consisting of 8 blocks with 8 self-attention layers in each block. The corresponding attention head number is 8, the patch size is 4, and the embedding dimension is 128. We set the number of clients to 50, and then each client has a 20\% chance of participating in each communication round. We utilize the SGD algorithm~\cite{cherry1998sgd} as the local optimizer for all methods. We use padding as our prompt method. We set batch size as 16 in the local training phase, the local training epochs for the generator and backbone as 5 in each round, the learning rate for the backbone as 0.005, the learning rate for the prompt generator as 1, and the  padding prompt size as 4. The number of communication rounds is set to 150 for CIFAR10, 300 for CIFAR100, where all FL approaches have very limited or no accuracy gain with more communications.


\subsection{Main Results} \label{main result}
We run vast experiments to determine the superiority of pFedPT on the model performance in different datasets. Our results highlight the benefit of pFedPT compared to the existing PFL optimization approaches.

\paragraph{Better performance of pFedPT.}
Tab.~\ref{results} compares the best accuracy of the pFedPT with baselines on evaluation datasets with various settings. On CIFAR10 and CIFAR100, the pFedPT consistently achieves the best test accuracy with Non-IID setting. For instance, when training on the data of Dirichlet distribution CIFAR10 with CNN, the test accuracy of the pFedPT is 80.83\%, the accuracy of FedAvg is 61.92\%, and the accuracy of the FedPer is 77.98\%.  The improvements of pFedPT indicate that prompts in each client effectively improve the backbone performance in each client. Similarly, in CIFAR100, pFedPT outperforms most baselines in various settings and achieves comparable results in the Dirichlet setting.

\paragraph{Robustness of pFedPT.}  pFedPT achieves clear success on both ViT and CNN models and seems to get better performance as the FL tasks become more difficult (since better performance is observed at a greater Non-IID extent and in datasets that are intrinsically more difficult). Interestingly, in the IID setting, we show that all the personalized solutions exhibit some extent of performance degradation, which becomes more significant as the dataset becomes more challenging. Our interpretation of this phenomenon is that when the data are distributed under the IID setting, the PFL approach does not effectively take advantage of the personalization characteristics among clients, resulting in performance degradation. pFedPT will utilize the data distribution information in the client by visual prompts. When the data is IID, the output will be similar on the various clients and degenerate into the FedAvg.


\paragraph{Improvements  of prompt for other algorithms.}  Visual prompts can improve the performance of backbones on clients by fine-tuning the backbone with hints about the distribution of the client's data. We explore the usefulness of visual prompts as prior knowledge for other FL algorithms, and Tab.~\ref{results+pt} presents these results. In the Dirichlet setting of CIFAR10, the final test accuracy of FedProx increases from 61.95\% to 80.47\% after adding prompts, and the test accuracy of MOON increases from 62.53\% to 77.84\%. We find that a visual prompt enables fine-tuning of the backbone of the client, which helps FL algorithms that pursue high precision fuse client information for personalization. Similarly, PFL algorithms with model decoupling, like FedRep and FedPer, can also yield a performance boost by integrating pFedPT. Therefore, prompt can be used as an additional component to improve the personalization performance of some existing FL algorithms.

Compared with other baselines, the pFedPT takes full advantage of the data improvement space. Additional prompts are added to the data entered into the model to improve the performance of the model on each client.

\subsection{Exploratory Study}
To provide more explanation for pFedPT, we additionally conduct several exploratory studies on pFedPT.  



% \begin{figure}
%     \centering
% w    \includegraphics[width=0.40\textwidth]{figures/exp_cnn_vit.pdf}
%     \caption{Results of using Vit and CNN as backbones for pFedPT and FedAvg, respectively}
%     \label{fig:comparison}
% \end{figure}

 
 
 
% \paragraph{The validity of the ViT.}\  
% We examine the CNN and ViT as the backbone in our pFedPT framework compared with FedAvg. The experiments are under three different CIFAR10 dataset settings, i.e., Dirichlet (0.1), Pathological (2) and IID, where Dirichlet (0.1) and Pathological (2) are non-IID settings. 
% Fig.~\ref{fig:comparison} demonstrates the test accuracy between four different models. Under the non-IID setting, the pFedPT+ViT model outperforms other methods by a large margin, while the other three methods struggle to address the distribution with data heterogeneity. It means that visual prompts should be used together with a ViT instead of only replacing the CNN backbone with a ViT or adding prompts separately. The ViT can notice the prior knowledge contained in the prompts through the multi-head self-attention and map the prompts to the data distribution clients. Hence, the test accuracy of the model has been greatly improved, which indicates that our pFedPT could better mine and exploit heterogeneity among clients.

 
 \begin{figure}[htbp]
    \centering
    \includegraphics[width=0.47\textwidth]{figures/attention_visulization.pdf}
    \vspace{-0.8em}
    \captionsetup{font=small}
    \caption{Visualization results generated by FedAvg and pFedPT with different backbones.}
    \label{fig:attenmap}
\end{figure}

\paragraph{Visualization of attention maps.} 
To illustrate the effectiveness of visual prompts, we conducted some validation experiments. We train ten clients using FedAvg and pFedPT with ViT and CNN backbones under the Dirichlet setting of the CIFAR10 dataset, respectively. As shown in Fig.~\ref{fig:attenmap}, we make a visualization of the attention map of the last layer in the ViT and CNN by Grad-CAM~\cite{DBLP:journals/ijcv/SelvarajuCDVPB20}.
The first three rows in the figure show that  FedAvg focuses on some salient classification features of the raw image.
The fourth row contains the input images with the padding visual prompts, which are added by the prompt generator of pFedPT according to Eq.~(\ref{f4}). Both pFedPT+ViT and pFedPT+CNN shift some attention to the added prompts, which can help obtain the prior knowledge for the model inference process, thus improving the performance of the model.

 \begin{figure}[htbp]
    \centering
    \includegraphics[width=0.45\textwidth]
    {figures/tSNE.pdf}
    \vspace{-0.8em}
    \captionsetup{font=small}
    \caption{t-SNE visualization of embedding for pure color images with learned prompts in different clients.}
    \label{fig:tSNE}
    \vspace{-0.4cm}
\end{figure}


\paragraph{The guidance information contained in the prompts.}\  In order to further explore the influence of visual prompts, we generated 100 different pure color images with the shape of $[3\times32\times32]$. Using the pure color picture, pFedPT can exclude the disturbance of image contents and pay more attention to visual prompts. We feed those color pictures into pFedPT models in different clients with different prompts and visualize the output embeddings of their last MLP layer. We project them into a two-dimensional plane using the t-SNE algorithm~\cite{van2008visualizing}. Fig.~\ref{fig:tSNE} shows that after the visual prompts are added, the model outputs of different clients can be easily distinguished, indicating that the prompts contain prior knowledge of the client model and aid in the classification task. 


\begin{figure}
    \centering
    \includegraphics[width=0.45\textwidth]{figures/prompt.pdf}
    \vspace{-0.8em}
    \captionsetup{font=small}
    \caption{Effect of different types of prompts}
    \label{fig:p_type}
\end{figure}

 
 
\paragraph{Impact of different types of visual prompts.} 
We analyze different choices on how and where to insert prompts in the input images and how they would affect the final performance. We perform an ablation study on different prompt sizes in $p=\{2,4,6,\dots,16 \}$ in CIFAR10 with a Dirichlet distribution. As shown in Fig.~\ref{fig:p_type}, padding prompts reach the highest performance with a size of 4. The test accuracy of fixed location and random location prompts grows gradually with the increase in prompt size, but it is still slightly lower than the padding prompt. In contrast, the accuracy of padding prompts decreases as the prompt size increases. A possible explanation is that the padding method covers more pixels of the original images than the other two methods when using the same length of prompts. As a result, the key information for classification could be blocked by the prompts and harm the performance of the model. Overall, the padding prompts with size 4 achieve the best performance. Note that other visual tasks may require significantly different kinds of prompts. 





% \begin{figure}
%     \centering
%     \includegraphics[width=0.5\textwidth]{figures/out.pdf}
%     \caption{Similarity comparison between the distribution of the predicted classes and the distribution of the local data.}
%     \label{fig:output_dis}
% \end{figure}

% \begin{figure}
%     \centering
%     \includegraphics[width=0.35\textwidth]{figures/the difference of average prompt4.pdf}
%     \caption{The difference of average prompt between two consecutive rounds.}
%     \label{fig:difference}
%         \vspace{-0.4cm}
% \end{figure}


% \begin{figure}
%     \centering
%     \includegraphics[width=0.35\textwidth]{figures/fintune.pdf}
%     \caption{Test accuracy after fine-tuning the head of models trained on a client of CIFAR10 datasets}.
%     \label{fig:fintune}
%     \vspace{-0.4cm}
% \end{figure}


% \paragraph{Empirical analysis of the learned prompts.}\ 
% Fig.~\ref{fig:difference} records the average difference of the prompts generated between the two rounds before and after ten clients during the pFedPT training process. The overall experimental results are divided into two stages: first ascending and then descending. In our settings, the initial prompt generator parameters of each client are the same, and the rising stage is the mapping process between each client and the prompts based on its own data distribution. The descending stage is that the aggregated model tends to converge, and the mapping between the prompt and the client data distribution on each client is completed. Eventually the change of prompts embedding approaches 0, that is, each client establishes stable prompts that conforms to its own data distribution.

 
% \paragraph{Generalization ability of the pFedPT.}
% We evaluate the strength of the backbone learned by pFedPT in terms of adaptation to new clients. To do so, we first train the pFedPT, the FedAvg in the usual setting on the partition of the CIFAR10 dataset with 10 clients and Dir (0.1) partition. Then, we encounter clients with data from Dir (0.3) partition of the CIFAR10 dataset. We assume we have access to a dataset of 400 samples for this new client to fine-tune. For the pFedPT, we fine-tune the prompt generator parameter over multiple epochs while keeping the backbone fixed. For the FedAvg, fine tune the last layer of backbone while keeping other layers. Fig.~\ref{fig:fintune} shows that the pFedPT has significantly better performance than the FedAvg.
\section{Conclusion}
Throughout the paper, we first analyze the current evaluation methods for diffusion-based adversarial purification and then propose a recommendation for the reliable evaluation of the robustness of adversarial purification. We further investigate the influence of hyperparameters of the diffusion model on the robustness of the purification. Based on our analysis, we propose a new strategy to maximize the benefit of the purification methods.

%%%%%%%%% TITLE
\vspace{15pt}
\hspace{-10pt}
\textbf{{\huge Appendix}}
\vspace{5pt}
\appendix

\section{Additional Experimental Results}\label{A}
\subsection{Comparisons with NeROIC}
As mentioned in the paper, we would like to emphasize that none of the previous works deal with the same problem as us, which is a few-shot views synthesis of a non-object-centric scene under unconstrained illumination.
In the paper, we provide comparisons with the baselines of Barron et al.~\cite{barron2021mip} and Niemeyer et al.~\cite{niemeyer2022regnerf}.
While NeROIC~\cite{kuang2022neroic} and other related works such as NeuralPIL~\cite{NeuralPIL}, NeRD~\cite{boss2021nerd}, and SAMURAI~\cite{boss2022-samurai} deal with view synthesis under varying illumination, they have several differences from us, which we describe as follows.

\begin{itemize}
    \item All of these works can only handle object-centric, 360 scenes paired with foreground masks, but our proposed method is targeting non-object-centric, forward-facing scenes without additional masks. 
    \item Previous works~\cite{kuang2022neroic, NeuralPIL, boss2021nerd, boss2022-samurai} are an inverse rendering framework, which aims to decompose images into their geometry, material, and illumination. Therefore, they require massive computational costs with multi-stage training. Our proposed work is the first end-to-end few-shot synthesis framework that handles inputs with unconstrained illuminations with minimum computational costs~($3.5$ hours in a $3$-view setting).  
    \item Boss et al.~\cite{boss2021nerd} and following works~\cite{NeuralPIL, boss2022-samurai} does not assume few-shot settings. Especially, SAMURAI~\cite{boss2022-samurai} assumes unknown camera poses of inputs, unlike other previous works and our proposed method.
\end{itemize}

In the following, we additionally provide comparisons with NeROIC, on our NeRF Extreme dataset. Note that NeROIC is the only baseline that performs few-shot view synthesis with varying illumination inputs, which shows better performance compared to other baselines (detailed experimental results are on their paper, ~\cite{kuang2022neroic}) under unconstrained illumination conditions.
Comparisons with NeRF-W~\cite{nerfinthewild} are not available since 1) it is targeting outdoor scenes and 2) neither the implementation nor the pre-processed data are publicly available. 
\vspace{-10pt}

\paragraph{Comparison on NeRF Extreme.}
We compare our model with NeROIC on our NeRF Extreme dataset in the 3-view setting,  Tab.~\ref{tab:neroic_extreme} and Fig.~\ref{fig:neroic_extreme} show the quantitative and qualitative comparison results, respectively. NeROIC-Geom refers to the geometry network of NeROIC while NeROIC-Full refers to the full rendering network of NeROIC. Both NeROIC-Geom and NeROIC-full show over-smoothed or diverged results on our dataset, while our proposed ExtremeNeRF shows plausible view synthesis results. The results demonstrate the fact that none of the previous works can successfully perform few-shot view synthesis under unconstrained illumination if the target scene is the non-object-centric one. 
See Sec.~\ref{B} for more details about NeROIC. 


\begin{table}
    \centering
    \resizebox{\linewidth}{!}
    {\begin{tabular}{l|c|c|c}
\toprule
 &  PSNR $\uparrow$ & SSIM $\uparrow$ & LPIPS $\downarrow$ \\
    \midrule
    NeROIC~\cite{kuang2022neroic}-Geom \ & 13.06 & 0.25 & 0.64  \\
    \midrule
    \textbf{Extreme NeRF (Ours)} \ & \cellcolor{RoyalBlue!25}14.86 & \cellcolor{RoyalBlue!25}0.36 & \cellcolor{RoyalBlue!25}0.44 \\
\bottomrule
\end{tabular}

%\cellcolor{RoyalBlue!25}
%\cellcolor{SkyBlue!25}}
    \vspace{-.2cm}
    \caption{
    \textbf{Quantitative comparison on NeRF Extreme.} Note that results from NeROIC-Full are not included for their diverged outputs.}
    \label{tab:neroic_extreme} \vspace{-10pt}
\end{table}

\begin{figure}
  \centering
  \begin{tabular}{>{\centering\arraybackslash}p{0.20\textwidth}>{\centering\arraybackslash}p{0.22\textwidth}}
      \scriptsize NeROIC~\cite{kuang2022neroic}-Geom & \scriptsize  \textbf{ExtremeNeRF~(Ours)} \\
    \end{tabular}
  \includegraphics[width=\linewidth]{figure/ICCV/neroic_extreme3.pdf}
  %\fbox{\rule{0pt}{2in} \rule{.9\linewidth}{0pt}}
  \vspace{-15pt}
   \caption{\textbf{Qualitative comparison with NeROIC~\cite{kuang2022neroic} on NeRF Extreme.} A synthesized novel view by NeROIC~\cite{kuang2022neroic}-Geom and our proposed method in 3-view setting. NeROIC-Geom shows unclear, distorted results compared to ours. Note that synthesized results of NeROIC-Full are not included since the network diverged.}
   \label{fig:neroic_extreme} \vspace{-10pt}
\end{figure}



\subsection{Ablation Studies}
As shown in Tab.~4 of the paper, our model with ablation on each component show a larger difference in the quality of the underlying depth. For better analysis, we provide additional qualitative comparison which visualizes the difference in the quality of the synthesized depth. Fig.~\ref{fig:supp_ablation} shows the synthesized color and depth map with and without cross-view albedo consistency regularization. Our model with albedo consistency regularization shows a better performance in synthesizing depth compared to the one without albedo consistency. It also supports the result described in Fig.~8 of the paper, that neglecting illumination can result in distorted depth in few-shot view synthesis under unconstrained illumination. 

\begin{figure}
  \centering
  \begin{tabular}{>{\centering\arraybackslash}p{0.20\textwidth}>{\centering\arraybackslash}p{0.22\textwidth}}
      \scriptsize Ours w/o $\mathcal{L}_{\mathrm{ac}}$ & \scriptsize  Ours \\
    \end{tabular}
  \includegraphics[width=\linewidth]{figure/ICCV/ablation_qual2.pdf}
  \vspace{-15pt}
   \caption{\textbf{Qualitative comparison on ablation studies.} Ablation on our cross-view albedo consistency loss results in degraded synthesizing performance, especially in the depth map.}
   \label{fig:supp_ablation} \vspace{-10pt}
\end{figure}







\subsection{Relighting}
Unlike the other previous works that tried to achieve inverse rendering of a scene given inputs with unconstrained illumination, our proposed method focuses on a few-shot view synthesis task for practical usage. The architectural choice results in better computational efficiency, however, has its limitation in that it cannot explicitly decompose the illumination of the target scene. In this supplementary material, we additionally provide relighting results of our proposed method, inspired by the image formation model we use during intrinsic decomposition. As described in Eq.~\ref{eq:imageformation}, our inputs and synthesized images can be decomposed by their illumination-invariant color~(albedo), illumination-variant shading, light color, and non-Lambertian residuals. Because our proposed ExtremeNeRF provides per-scene optimization, we can relight the synthesized image by replacing the shading image, as long as the synthesized image remains plausible. Fig.~\ref{fig:relit} shows the relighting results, achieved by replacing the shading image of the synthesized scene. Note that shiny, blurry effects in the scenes are caused by \cite{li2018iidww}, which is used as our full-image intrinsic decomposition method.

\begin{figure}
    \begin{tabular}{>{\centering\arraybackslash}p{0.20\textwidth}>{\centering\arraybackslash}p{0.22\textwidth}}
      \scriptsize Before & \scriptsize  After \\
    \end{tabular}
  \centering
  \includegraphics[width=\linewidth]{figure/ICCV/relit.pdf}
  \vspace{-15pt}
   \caption{\textbf{Relighting results.}}
   \label{fig:relit} \vspace{-10pt}
\end{figure}


\subsection{Failure Cases} % outdoor scene cases. 기존 work 보다는 훨씬 distortion 이 적지만, plausible 하지 못하다는 것. 
If the outdoor scenes have challenging illumination, such as the \textit{tent} scene in our dataset, our proposed method may struggle to synthesize plausible results. Figure \ref{fig:failure} shows the synthesized results for the \textit{tent} scene produced by our method and the baseline method from \cite{niemeyer2022regnerf}. The less-qualified results may be due to inadequate decomposition of the intrinsic components of the scene, which causes the cross-view regularization to fail. However, despite the difficulties in regularizing the intrinsic components, our proposed method still shows better synthesizing performance compared to the baseline, which produces a highly distorted color image.

\begin{figure}
  \centering
  \begin{tabular}{>{\centering\arraybackslash}p{0.20\textwidth}>{\centering\arraybackslash}p{0.22\textwidth}}
      \scriptsize RegNeRF~\cite{niemeyer2022regnerf} & \scriptsize  \textbf{ExtremeNeRF(Ours)} \\
    \end{tabular}
  \includegraphics[width=\linewidth]{figure/ICCV/failures.pdf}
  \vspace{-15pt}
   \caption{\textbf{Failure cases.}}
   \label{fig:failure} \vspace{-10pt}
\end{figure}

\begin{figure*}[t]
    \centering
     \begin{tabular}{>{\centering\arraybackslash}p{0.3\textwidth}>{\centering\arraybackslash}p{0.3\textwidth}>{\centering\arraybackslash}p{0.3\textwidth}}
      \scriptsize mip-NeRF~\cite{barron2021mip} & \scriptsize  RegNeRF~\cite{niemeyer2022regnerf} & \scriptsize  \textbf{ExtremeNeRF~(Ours)}\\
    \end{tabular}
     \begin{subfigure}[b]{1.0\textwidth}
         \centering
        \includegraphics[width=\linewidth]{figure/ICCV/extreme_supp_3.pdf}z
         \caption{3 Input Views}
     \end{subfigure}
     \begin{subfigure}[b]{1.0\textwidth}
         \centering
        \includegraphics[width=\linewidth]{figure/ICCV/extreme_supp_6.pdf}        %
         \caption{6 Input Views}
     \end{subfigure}
     \begin{subfigure}[b]{1.0\textwidth}
         \centering
        \includegraphics[width=\linewidth]{figure/ICCV/extreme_supp_9.pdf}        
         \caption{9 Input Views}
     \end{subfigure}
    \vspace{-15pt}
    \caption{
    \textbf{Additional qualitative results on NeRF Extreme.}
    }
    \label{fig:suppEx} \vspace{-10pt}
\end{figure*}

\begin{figure*}[t]
    \centering
     \begin{tabular}{>{\centering\arraybackslash}p{0.3\textwidth}>{\centering\arraybackslash}p{0.3\textwidth}>{\centering\arraybackslash}p{0.3\textwidth}}
      \scriptsize mip-NeRF~\cite{barron2021mip} & \scriptsize  RegNeRF~\cite{niemeyer2022regnerf} & \scriptsize  \textbf{ExtremeNeRF~(Ours)}\\
    \end{tabular}
     \begin{subfigure}[b]{1.0\textwidth}
         \centering
        \includegraphics[width=\linewidth]{figure/ICCV/dtu_add.pdf}
     \end{subfigure}
    \vspace{-20pt}
    \caption{
    \textbf{Additional qualitative results on light-varying DTU~\cite{DTU} for 3 input views.}
    }
    \label{fig:suppDTU} \vspace{-10pt}
\end{figure*}


\subsection{Additional Qualitative Results}
\paragraph{NeRF Extreme.}
Figure \ref{fig:suppEx} shows additional qualitative comparisons for our NeRF Extreme benchmark, using 3, 6, and 9 view scenarios. Our method performs better than the others, especially in synthesizing depth. This demonstrates that our albedo regularization pipeline successfully removes the distortions that can arise in an unconstrained illumination environment.
\vspace{-10pt}

\paragraph{Light-varying DTU.}
Figure \ref{fig:suppDTU} shows additional qualitative results for DTU~\cite{DTU} using 3 view scenarios. Although our proposed method sometimes produces competitive or inferior quantitative results compared to RegNeRF~\cite{niemeyer2022regnerf} (as shown in Table 3 in the paper), it often produces less distorted results for both color and depth maps. We would like to emphasize that the DTU dataset is not an ideal input for our method, due to its insufficient global context, which is important for successful intrinsic decomposition.

\begin{figure*}[t]
    \centering
    \begin{subfigure}[b]{1.0\textwidth}
    \centering
        \includegraphics[width=.15\linewidth]{figure/scene_list/bench.jpg}
        \includegraphics[width=.15\linewidth]{figure/scene_list/cafe.jpg}
        \includegraphics[width=.15\linewidth]{figure/scene_list/flower.jpg}
        \includegraphics[width=.15\linewidth]{figure/scene_list/houseplant.jpg}
        \includegraphics[width=.15\linewidth]{figure/scene_list/kitchen.jpg}
        % \caption{3 Input Views}
        \vspace{-3pt}
     \end{subfigure}
     \begin{tabular}{>{\centering\arraybackslash}p{0.15\textwidth}>{\centering\arraybackslash}p{0.12\textwidth}>{\centering\arraybackslash}p{0.15\textwidth}>{\centering\arraybackslash}p{0.12\textwidth}>{\centering\arraybackslash}p{0.15\textwidth}}
    \scriptsize Bench  & \scriptsize Cafe & \scriptsize Flower  & \scriptsize Houseplant & \scriptsize Kitchen  \\
    \end{tabular}
    \begin{subfigure}[b]{1.0\textwidth}
    \centering
       \includegraphics[width=.15\linewidth]{figure/scene_list/leaves.jpg}
       \includegraphics[width=.15\linewidth]{figure/scene_list/room.jpg}
       \includegraphics[width=.15\linewidth]{figure/scene_list/table.jpg}
       \includegraphics[width=.15\linewidth]{figure/scene_list/tent.jpg}
       \includegraphics[width=.15\linewidth]{figure/scene_list/tree.jpg}
        % \caption{3 Input Views}
        \vspace{-3pt}
    \end{subfigure}
    \begin{tabular}{>{\centering\arraybackslash}p{0.15\textwidth}>{\centering\arraybackslash}p{0.12\textwidth}>{\centering\arraybackslash}p{0.15\textwidth}>{\centering\arraybackslash}p{0.12\textwidth}>{\centering\arraybackslash}p{0.15\textwidth}}
    \scriptsize Leaves  & \scriptsize Room & \scriptsize Table & \scriptsize Tent & \scriptsize Tree \\
    \end{tabular}
    \vspace{-5pt}
    \caption{
        \textbf{NeRF Extreme dataset.} This dataset is newly built to provide multi-view images under varying illumination, which can be used to train and evaluate the robust NeRF. Exemplified images are selected from their test sets with mild-lighting conditions.}
    \label{fig:scenelist} \vspace{-10pt}
\end{figure*}



\section{Experimental Details}\label{B}
In this section, we provide details about the proposed methods and experiments.
\subsection{Datasets}
\paragraph{NeRF Extreme.}
As already mentioned in the paper, we would like to emphasize that our proposed NeRF Extreme dataset, is the first frontal-facing, in-the-wild multi-view dataset with varying illumination. A detailed explanation of the dataset statistics can be found in Sec.~\ref{C}.
Fig.~\ref{fig:scenelist} shows the indoor and outdoor scenes included in our dataset.
In the experiments, we use image IDs of (0, 14, 29), (0, 5, 12, 17, 28, 29), (0, 4, 7, 11, 14, 18, 22, 25, 29) for 3, 6, and 9 views scenarios, respectively.  All images are used in a resolution of $300 \times 400$. %More detailed information about the configuration will be provided with the code. 
\vspace{-10pt}

\paragraph{Light-varying DTU.}
DTU~\cite{DTU} consists of images taken under structured cameras and light sources. There exist 7 number of lighting variations per scene. Previous works which deal with view synthesis under consistent illumination~\cite{mildenhall2020nerf, dietnerf, yu2021pixelnerf, srf, barron2021mip, niemeyer2022regnerf, CLIPnerf, infonerf, chen2021mvsnerf, xu2022sinnerf} have used the dataset with fixed mild lighting condition.% - noted as 3, which refers to the mild lighting condition. 
In this paper, we randomly choose lighting conditions for each scene. Note that view synthesis performance may vary a lot depending on which viewing directions and lighting conditions are selected. Following the evaluation protocol used by the previous works~\cite{yu2021pixelnerf, niemeyer2022regnerf}, we use scan IDs (8, 21, 30, 31, 34, 38, 40, 41, 45, 55, 63, 82, 103, 110, 114) as the dataset, while using image IDs (25, 22, 28) as inputs. In the cases of the lighting condition, (4, 1, 5) are used for (25, 22, 28) images, respectively. All images are used in a resolution of $300 \times 400$, following the evaluation protocols of the previous work~\cite{niemeyer2022regnerf}. Note that all metrics are calculated without masks.




\subsection{Baselines}
In the cases of RegNeRF~\cite{niemeyer2022regnerf}, the official configurations for DTU~\cite{DTU} and LLFF~\cite{LLFF} are used for evaluations of light-varying DTU and NeRF Extreme, respectively, owing to their identical or similar data characteristics. 
For light-varying DTU, 44K train epochs are used for 3 view cases, for both RegNeRF and ours. For NeRF Extreme, 64K, 140K, and 21K train epochs are used for 3, 6, and 9 view cases, respectively. Our ExtremeNeRF shares the configurations with RegNeRF. Note that the official implementation of RegNeRF lacks color regularization done by the normalizing-flow model.
In the case of NeROIC~\cite{kuang2022neroic}, we use the publicly available instructions and code for the experiments. Since the authors have notified that the proposed method is not able to handle the frontal-facing scenes, we adjust the implementation code to be able to run on the frontal-facing scene. Instead of the foreground masks which should be provided to run the code, we enable all the image pixels to be regarded as foreground ones. It may result in degraded performance of the baseline, however, is the only way to make a comparison. We follow the provided configurations for the training options. 

\begin{table*}
    \centering
    \resizebox{\linewidth}{!}
    {\begin{tabular}{l|c|c|c|c|c}
\toprule
  &  Name &  Filter &  Input &  Output &  Connectivity \\
  \midrule
   \multirow{4}{*}{ Encoder} &  Conv1 & kernel=$(4 \times 4)$, strides=$(2 \times 2)$& $32 \times 32 \times 3$ & $16 \times 16 \times 32$ & -\\
                                 &  (LeakyRelu-Conv2-BN)  &  kernel=$(4 \times 4)$, strides=$(2 \times 2)$ & $16 \times 16 \times 32$ & $8 \times 8 \times 64$ & -\\
                                 &  (LeakyRelu-Conv3-BN)  &  kernel=$(4 \times 4)$, strides=$(2 \times 2)$ & $8 \times 8 \times 64$ & $4 \times 4 \times 128$ & -\\
                                 &  (LeakyRelu-Conv4) &  kernel=$(4 \times 4)$, strides=$(2 \times 2)$ & $4 \times 4 \times 128$ & $2 \times 2 \times 256$ & -\\
  \midrule
  \multirow{4}{*}{ Decoder}  &  (LeakyRelu-Deconv1-BN)  &  kernel=$(4 \times 4)$, strides=$(2 \times 2)$ & $2 \times 2 \times 256$ & $4 \times 4 \times 128$ & $4 \times 4 \times 128$\\
                                 &  (LeakyRelu-Deconv2-BN) & kernel=$(4 \times 4)$, strides=$(2 \times 2)$ & $4 \times 4 \times 256$ & $8 \times 8 \times 64$  & $8 \times 8 \times 64$\\
                                 &  (LeakyRelu-Deconv3-BN) & kernel=$(4 \times 4)$, strides=$(2 \times 2)$ & $8 \times 8 \times 128$ & $16 \times 16 \times 32$ & $16 \times 16 \times 32$\\
                                 &  \multirow{2}{*}{(LeakyRelu-Deconv4)} & \multirow{2}{*}{kernel=$(4 \times 4)$, strides=$(2 \times 2)$} & \multirow{2}{*}{$16 \times 16 \times 64$} &  $32 \times 32 \times 3$~(albedo) & \multirow{2}{*}{-} \\
                                 &  & &  & $32 \times 32 \times 1$~(shading) & \\
  \midrule
  \multirow{3}{*}{ Color Prediction}  &  AvgPool  &  kernel=$(1 \times 1)$ & $2 \times 2 \times 256$ & $2 \times 2 \times 256$ & -\\
                                 &  (LeakyRelu-Conv5) & kernel=$(3 \times 3)$, strides=$(1 \times 1)$ & $2 \times 2 \times 256$ & $2 \times 2 \times 128$ & -\\
                                 &  (LeakyRelu-Flatten-FC) & - & $2 \times 2 \times 128$ & $3$ & -\\
  
\bottomrule
\end{tabular}}
    \vspace{-.2cm}
    \caption{
    \textbf{Patch-wise intrinsic decomposition network(PIDNet) architecture. }}
    \label{tab:PIDNetarch} \vspace{-10pt}
\end{table*}

\subsection{Metrics}
\paragraph{Cross-Color-Ratio~(CCR). }
Cross-color-ratio (CCR)~\cite{gevers1999color} refers to the illumination-invariant image gradients that only depend on the albedo transitions of an image. For two adjacent $RGB$ pixels $x_1, x_2$, CCRs are defined as:
\begin{equation}
     M_{RG} = \cfrac{R_{x_1}G_{x_2}}{R_{x_2}G_{x_1}}, 
     M_{RB} = \cfrac{R_{x_1}B_{x_2}}{R_{x_2}B_{x_1}},
     M_{GB} = \cfrac{G_{x_1}B_{x_2}}{G_{x_2}B_{x_1}}, 
\label{eq:CCR}
\end{equation}
\vspace{-10pt}

\paragraph{Cross-Color-Ratio-Difference~(CCRD).}
According to Das et al.~\cite{das2022pie}, CCR reflects albedo properties sufficiently to improve the performance of intrinsic decomposition. By taking the logarithm of both side of Eq.~\ref{eq:CCR} as suggested by \cite{das2022pie}, our cross-color-ratio-difference (CCRD) metric can be formulated as:
\begin{equation}
     \text{CCRD} = \sum_{x \in \mathcal{P}} |\mathrm{ccr}(x) - \mathrm{ccr}_{\text{GT}}(x)|
\label{eq:CCRD}
\end{equation}
where $\mathrm{ccr}(x)$ and $\mathrm{ccr}_{\text{GT}}(x)$ indicate log-scale CCR values at a pixel $x$ of the synthesized image and the ground-truth image, respectively, and $\mathcal{P}$ denotes all pixels of the image.
\vspace{-10pt}

\paragraph{Absolute Relative error~(Abs Rel).} 
Absolute Relative error~(Abs Rel) is one of the most commonly used metrics for depth estimation tasks. In order to measure the scale-invariant plausibility of the synthesized depth, we adopted Abs Rel as:
\begin{equation}
     \text{Abs Rel} = \sum_{x \in \mathcal{P}} |\bar{d}(x) - \bar{d}_{\text{GT}}(x)|,
\label{eq:CCRD}
\end{equation}
where $\bar{d}(x)$ and $\bar{d}_{\text{GT}}(x)$ indicate the synthesized depth and the ground truth depth, which are normalized to scale 0 to 1, respectively. Since most of the view-synthesis datasets except DTU~\cite{DTU} are not providing depth information, we estimated the depth maps by ~\cite{giang2022curvatureguided} and used them as a pseudo-depth ground truth.


\subsection{Network Architectures} 
\paragraph{FIDNet architectures.}
IIDWW\cite{li2018iidww} uses a variant of UNet\cite{pix2pix,unet} architectures with a shared encoder and 2 decoders. A 3-dimensional light color $c$ is also predicted as a by-product of the network. The publicly available model takes $256 \times 384$ sized full-resolution images as inputs. Images are resized before and after intrinsic decomposition.
\vspace{-10pt}

\paragraph{PIDNet architectures.}
As a downsized model of FIDNet, PIDNet consists of 4 numbers of $4 \times 4$ sized convolution/deconvolution layers that are connected to each other using skip-connections, with additional FC layers for light color prediction. Tab.~\ref{tab:PIDNetarch} shows the details of PIDNet. Note that PIDNet takes $32 \times 32$ sized patch as an input. Decoders for albedo and shading are identical except for the last channel dimension ($3$ for the albedo and $1$ for the shading image).

\begin{algorithm}[t]
\caption{Depth Consistency, Pytorch-like}
\label{code:dcloss}
\definecolor{codeblue}{rgb}{0.25,0.5,0.5}
\definecolor{codekw}{rgb}{0.85, 0.18, 0.50}

\lstset{
  backgroundcolor=\color{white},
  basicstyle=\fontsize{7.5pt}{7.5pt}\ttfamily\selectfont,
  columns=fullflexible,
  breaklines=true,
  captionpos=b,
  commentstyle=\fontsize{7.5pt}{7.5pt}\color{codeblue},
  keywordstyle=\fontsize{7.5pt}{7.5pt}\color{codekw},
}
\begin{lstlisting}[language=python]
def get_corresponding_coords(K, coords, c2w_1, c2w_2, depth):
    im_coord = coords*depth
    cam_coord = np.linalg.inv(K) @ im_coord
    w2c = np.linalg.inv(c2w_2)
    cam_mat = w2c @ c2w_1
    return K @ cam_mat @ cam_coord

def init_mask_tg_rays(tg_coords, full_rays):
    lrc_mask = np.ones_like(tg_coords)
    lrc_mask = (0 if outside of image region else 1)
    tg_rays = full_rays[tg_coords]
    return lrc_mask, tg_rays

# src renderings from MLP
src_renderings = model.apply(src_rays)
depth = src_renderings['depth_pred']

# calc target coord & rays
uvd_target = get_corresponding_coords
          (K, src_coords, src_pose, tg_pose, depth)
tg_coords = uvd_target[:2,:]//uvd_target[2,:]

depth_tilde = uvd_target[2,:]
lrc_mask, tg_rays = init_mask_tg_rays
          (tg_coords, tg_full_rays)

# tg renderings from MLP
tg_renderings = model.apply(tg_rays)
depth_hat = tg_renderings['depth_pred']

error_proj = (depth_tilde - depth_hat)**2

loss_dc = np.mean(tv_norm(error_proj))
\end{lstlisting}
\end{algorithm}

\begin{algorithm}[t]
\caption{Albedo Consistency, Pytorch-like}
\label{code:acloss}
\definecolor{codeblue}{rgb}{0.25,0.5,0.5}
\definecolor{codekw}{rgb}{0.85, 0.18, 0.50}

\lstset{
  backgroundcolor=\color{white},
  basicstyle=\fontsize{7.5pt}{7.5pt}\ttfamily\selectfont,
  columns=fullflexible,
  breaklines=true,
  captionpos=b,
  commentstyle=\fontsize{7.5pt}{7.5pt}\color{codeblue},
  keywordstyle=\fontsize{7.5pt}{7.5pt}\color{codekw},
}
\begin{lstlisting}[language=python]

src_patch = src_renderings['rgb']
src_shading, src_albedo, src_rgb = pidnet.apply(src_patch)

# intrinsic smoothness loss
pid_loss = np.mean(tv_norm(src_albedo))

# get target albedo from FIDNet result
tg_albedo = tg_full_albedo[tg_coords]

tg_chrome = rgb_to_chromaticity(tg_albedo)
src_chrome = rgb_to_chromaticity(src_albedo)
src_patch_chrome = rgb_to_chromaticity(src_patch)

# chromaticity consistency loss
loss_chrom = np.mean((src_chrome - tg_chrome)**2 
            + (src_chrome - src_patch_chrome)**2)

# scale value using least square
tg_albedo = ls_scale_val(src_albedo, tg_albedo)

occ_weight = r_e * (1- error_proj/max(error_proj))

# albedo consistency loss
loss_ac = np.mean(occ_weight*(src_albedo - tg_albedo)**2)

src_grad = np.gradient(np.exp(lrc_mask*src_albedo))
tg_grad = np.gradient(np.exp(lrc_mask*tg_albedo))

# edge preserving loss
loss_edge = np.mean(occ_weight*(src_grad - tg_grad)**2)

\end{lstlisting}
\end{algorithm}

\subsection{Loss Functions}
\paragraph{Edge-preserving loss. }
Motivated by \cite{godard2017unsupervised}, we use the gradient-based edge-preserving loss, to enforce the input and the novel view patches to preserve geometric properties. Using an occlusion-aware weight term, $\omega_{occ}(x)$, which already has been discussed in the paper, our edge-preserving loss on the predicted albedo can be formulated as:
\begin{equation} 
\begin{split}
     \mathcal{L}_{\mathrm{edge}} =\sum_{x' \in \mathcal{P}'} \omega_{occ}(x)
     \|\partial(\hat{a}(x) - \hat{a}(x'))\|^2 ,
\end{split}
\label{eq:edge}
\end{equation}
where $\partial$ denotes the partial derivatives of the vertical and the horizontal directions, and $\mathcal{P'}$ are all the pixels in the target image. 
\vspace{-10pt}


\paragraph{Patch-wise intrinsic smoothness loss.}
Similar to the previous work~\cite{li2018iidww} which uses various kinds of smoothness terms to give constraints to the network, we give smoothness constraints to our patch-wise intrinsic decomposition network~(PIDNet). Our patch-wise intrinsic smoothness loss is formulated as:
\begin{equation}
\begin{split}
    \mathcal{L}_{\mathrm{pid}} = \sum_{x' \in \mathcal{P}'} \sum_{y \in \mathcal{N}(x')} \|\hat{a}(y)\!-\hat{a}(x')\|^2 ,
\end{split}
\label{eq:pid}
\end{equation}
where $y$ is one of the 4-neighbor adjacent pixels $\mathcal{N}(x')$ for $x'$, and $\mathcal{P}'$ denotes all the pixels of the target image.
\vspace{-10pt}

\paragraph{Chromaticity consistency loss.}
Similar to \cite{ye2022intrinsicnerf}, we adopted the chromaticity consistency loss to enforce the consistency between the input and novel view patches. Our chromaticity consistency loss is formulated as:
\begin{equation}
  \mathcal{L}_{\mathrm{chrom}} = \sum_{x' \in \mathcal{P'}} \|\hat{ch}(x) - \hat{ch}(x')\|^2_2,
  \label{eq:ac}
\end{equation}
where $\hat{ch}(x)$ and $\hat{ch}(x')$ indicate the extracted albedo at $x$ and $x'$ from the novel view and the input view, respectively.
$\mathcal{P}$ denotes all the pixels in the novel view.
\vspace{-10pt}

\paragraph{Total loss functions. }
For each $x$ and $x'$ in the batch-wisely sampled input and the novel view patch, the total loss of our proposed framework is
\begin{equation}
\begin{split}
    \mathcal{L}_{\mathrm{total}}  =  & \lambda_{\mathrm{c}}\mathcal{L}_{\mathrm{color}} + \lambda_{\mathrm{a}}\mathcal{L}_{\mathrm{ac}} + \lambda_{\mathrm{dc}}\mathcal{L}_{\mathrm{dc}} + \lambda_{\mathrm{ds}}\mathcal{L}_{\mathrm{ds}} \\
    &+ \lambda_{\mathrm{e}}\mathcal{L}_{\mathrm{edge}} + \lambda_{\mathrm{pid}}\mathcal{L}_{\mathrm{pid}} + \lambda_{\mathrm{chrom}}\mathcal{L}_{\mathrm{chrom}}, 
\end{split}
\label{eq:total}
\end{equation}
where $\lambda_{\mathrm{c}}, \lambda_{\mathrm{a}}, \lambda_{\mathrm{dc}}, \lambda_{\mathrm{ds}}, \lambda_{\mathrm{edge}}, \lambda_{\mathrm{pid}}, \lambda_{\mathrm{chrom}}$ are weights parameters for each loss, respectively. In our experiment, losses are weighted as: $\lambda_{\mathrm{c}} = 1.0$, $\lambda_{\mathrm{a}} = 1.0$, $\lambda_{\mathrm{dc}} = 1.0$, $\lambda_{\mathrm{ds}} = 0.1$, $\lambda_{\mathrm{edge}} = 0.01$, $\lambda_{\mathrm{pid}} = 1.0$,  $\lambda_{\mathrm{chrom}} = 0.01$. We provide detailed algorithms in Alg.~\ref{code:dcloss} and \ref{code:acloss}.

\begin{figure}
  \centering
  \includegraphics[width=\linewidth]{figure/projection.pdf}
  \vspace{-15pt}
   \caption{\textbf{Projection situation examples.} Blue and yellow arrows are the synthesized depths $\hat{d}(x)$ and $\hat{d}(x')$, respectively, while green arrow is $\Tilde{d}(x')$ obtained by projective transformation.}
   \label{fig:projection} \vspace{-10pt}
\end{figure}

\subsection{Occlusion Handling}
\paragraph{Problem situations.}
In Fig.~\ref{fig:projection}, we illustrate possible situations that may occur in the geometry alignment stage.
\begin{itemize}
    \item (a): Projection to $x_1$ shows a situation where projection error $\mathcal{E}_\mathrm{proj}(x_1)$ is caused by the self-occlusion. It refers to the expected occlusion error case, which does not relate to the ill-synthesized geometry. Using occlusion mask $m(x)$ can exclude the projection case since it should be maintained even after the end of the optimization.
    \item (b): Projection to $x_2$ shows an ideal projection case. The synthesized depth $\hat{d}(x_2)$ is identical to the depth $\Tilde{d}(x_2)$ obtained by projective-transformation - i.e. $\mathcal{E}_\mathrm{proj}(x_2) = 0$.
    \item (c): Projection to $x_3$ shows a projection error caused by the false-positive density measure of NeRF. By enforcing $\mathcal{E}_\mathrm{proj}(x_3)$ to become zero, the floating artifacts can be expected to be removed.
    \item (d): Projection to $x_4$ shows a zero projection error which cannot be optimized by enforcing $\mathcal{E}_\mathrm{proj} = 0$. It is because the projection error $\mathcal{E}_\mathrm{proj}(x_4)$ is already 0. By assuming that the projection errors caused by self-occlusions would have smooth projection errors while ill-synthesized artifacts are not, we can optimize the depth consistency loss $\mathcal{L}_{\mathrm{dc}}$ described in Eq. 8 in the paper.
\end{itemize}
\vspace{-10pt}
% occlusion handling 관련 내용 추가

\paragraph{Error rate coefficient $r_e$.}
As described earlier, our projection error $\mathcal{E}_\mathrm{proj}$ includes errors that result from both occlusions and ill-synthesized depth. To prevent over-smoothing of synthesized results on occluded areas, errors resulting from occlusion should be disregarded during training. However, our occlusion-aware weight term is formulated as Eq.~8 in the paper. If the network can effectively regularize the ill-synthesized depth, a portion of the occlusion pairs included as a regularization target will increase during training. To prevent the regularization of occlusion pairs, the error rate coefficient $r_e$ will decrease the contribution of the consistency regularization as training progresses.


\subsection{Image Formation Model}
Basic intrinsic decomposition methods are based on the Lambertian assumption. 
However, most real-world objects have surface characteristics whose reflectances vary upon viewing directions. Thus, we use the image formation model suggested by \cite{li2018iidww} as follows:
\begin{equation}
\begin{split}
    \log I = \log A + \log S + c + N ,
\end{split}
\label{eq:imageformation}
\end{equation}
which takes light color vector $c$ and non-Lambertian residuals $N$ into account.



\section{NeRF Extreme}\label{C}


\subsection{Pose Estimation}
Similar to LLFF~\cite{LLFF}, all the camera poses are obtained by COLMAP~\cite{colmap} structure from motion framework. All the images and corresponding camera poses will be publicly available. 


\subsection{Dataset Statistics}

\begin{figure}
  \centering
  \includegraphics[width=\linewidth]{figure/graph_2.png}
  \vspace{-20pt}
   \caption{\textbf{Intensity distribution of our NeRF Extreme and LLFF~\cite{LLFF}.} Our dataset shows a larger variance in per-image lighting intensity distribution than LLFF~\cite{LLFF}. Lighting intensities are obtained from the shading images extracted by \cite{das2022pie}.}
   \label{fig:graph} \vspace{-10pt}
\end{figure}

In order to verify the illumination diversity of our dataset, we extract per-image intensity distribution from shading images, which are obtained by the state-of-the-art intrinsic decomposition framework~\cite{das2022pie}. A shading image is suitable for evaluating the illumination diversity of a dataset since it provides environment-dependent information about the image. Fig.~\ref{fig:graph} shows an intensity distribution of ours and existing LLFF~\cite{LLFF}, which has similar dataset characteristics. Each dot in the distribution indicates per-image intensity characteristics. Our dataset shows more scattered distribution compared to LLFF, indicating a larger illumination diversity.

\subsection{Lighting Variations}
All of the images in our dataset were captured using off-the-shelf cameras. Galaxy z-flip 4 was used to reproduce the casual image-capturing setup. Fig.~\ref{fig:tentall} and \ref{fig:benchall} show all the training and the test images(2 rows from the last) belonging to our \textit{tent} and \textit{bench} scene, respectively. Outdoor scenes are captured at different times and in different sunlight, to be taken under varying illumination conditions. Fig.~\ref{fig:tableall} and \ref{fig:roomall} show all the training and test images belonging to our \textit{table} and \textit{room} scenes, respectively. Indoor scenes are captured with turned-on/off lights, and closed/open curtains, to get illumination variance. 




\clearpage


\begin{figure*}
    \centering
    \begin{subfigure}[b]{1.0\textwidth}
         \centering
        \includegraphics[width=.18\linewidth]{figure/table/000.jpg}
        \includegraphics[width=.18\linewidth]{figure/table/001.jpg}
        \includegraphics[width=.18\linewidth]{figure/table/002.jpg}
        \includegraphics[width=.18\linewidth]{figure/table/003.jpg}
        \includegraphics[width=.18\linewidth]{figure/table/004.jpg}
        \vspace{.125cm}
    \end{subfigure}
    \begin{subfigure}[b]{1.0\textwidth}
    \centering
        \includegraphics[width=.18\linewidth]{figure/table/005.jpg}
        \includegraphics[width=.18\linewidth]{figure/table/006.jpg}
        \includegraphics[width=.18\linewidth]{figure/table/007.jpg}
        \includegraphics[width=.18\linewidth]{figure/table/008.jpg}
        \includegraphics[width=.18\linewidth]{figure/table/009.jpg}
        \vspace{.125cm}
    \end{subfigure}
    \begin{subfigure}[b]{1.0\textwidth}
    \centering
        \includegraphics[width=.18\linewidth]{figure/table/010.jpg}
        \includegraphics[width=.18\linewidth]{figure/table/011.jpg}
        \includegraphics[width=.18\linewidth]{figure/table/012.jpg}
        \includegraphics[width=.18\linewidth]{figure/table/013.jpg}
        \includegraphics[width=.18\linewidth]{figure/table/014.jpg}
        \vspace{.125cm}
    \end{subfigure}
    \begin{subfigure}[b]{1.0\textwidth}
    \centering
        \includegraphics[width=.18\linewidth]{figure/table/015.jpg}
        \includegraphics[width=.18\linewidth]{figure/table/016.jpg}
        \includegraphics[width=.18\linewidth]{figure/table/017.jpg}
        \includegraphics[width=.18\linewidth]{figure/table/018.jpg}
        \includegraphics[width=.18\linewidth]{figure/table/019.jpg}
        \vspace{.125cm}
    \end{subfigure}
    \begin{subfigure}[b]{1.0\textwidth}
    \centering
        \includegraphics[width=.18\linewidth]{figure/table/020.jpg}
        \includegraphics[width=.18\linewidth]{figure/table/021.jpg}
        \includegraphics[width=.18\linewidth]{figure/table/022.jpg}
        \includegraphics[width=.18\linewidth]{figure/table/023.jpg}
        \includegraphics[width=.18\linewidth]{figure/table/024.jpg}
        \vspace{.125cm}
    \end{subfigure}
    \begin{subfigure}[b]{1.0\textwidth}
    \centering
        \includegraphics[width=.18\linewidth]{figure/table/025.jpg}
        \includegraphics[width=.18\linewidth]{figure/table/026.jpg}
        \includegraphics[width=.18\linewidth]{figure/table/027.jpg}
        \includegraphics[width=.18\linewidth]{figure/table/028.jpg}
        \includegraphics[width=.18\linewidth]{figure/table/029.jpg}
        \vspace{.125cm}
    \end{subfigure}
    \begin{subfigure}[b]{1.0\textwidth}
    \centering
        \includegraphics[width=.18\linewidth]{figure/table/101.jpg}
        \includegraphics[width=.18\linewidth]{figure/table/102.jpg}
        \includegraphics[width=.18\linewidth]{figure/table/103.jpg}
        \includegraphics[width=.18\linewidth]{figure/table/104.jpg}
        \includegraphics[width=.18\linewidth]{figure/table/105.jpg}
        \vspace{.125cm}
    \end{subfigure}
    \begin{subfigure}[b]{1.0\textwidth}
    \centering
        \includegraphics[width=.18\linewidth]{figure/table/106.jpg}
        \includegraphics[width=.18\linewidth]{figure/table/107.jpg}
        \includegraphics[width=.18\linewidth]{figure/table/108.jpg}
        \includegraphics[width=.18\linewidth]{figure/table/109.jpg}
        \includegraphics[width=.18\linewidth]{figure/table/110.jpg}
    \end{subfigure}
     \vspace{-20pt}
    \caption{
        \textbf{Illumination variation samples of the \textit{table} scene in NeRF Extreme. }}
    \label{fig:tableall} \vspace{-10pt}
\end{figure*}


\begin{figure*}
    \centering
    \begin{subfigure}[b]{1.0\textwidth}
         \centering
        \includegraphics[width=.18\linewidth]{figure/room/000.jpg}
        \includegraphics[width=.18\linewidth]{figure/room/001.jpg}
        \includegraphics[width=.18\linewidth]{figure/room/002.jpg}
        \includegraphics[width=.18\linewidth]{figure/room/003.jpg}
        \includegraphics[width=.18\linewidth]{figure/room/004.jpg}
        \vspace{.125cm}
    \end{subfigure}
    \begin{subfigure}[b]{1.0\textwidth}
    \centering
        \includegraphics[width=.18\linewidth]{figure/room/005.jpg}
        \includegraphics[width=.18\linewidth]{figure/room/006.jpg}
        \includegraphics[width=.18\linewidth]{figure/room/007.jpg}
        \includegraphics[width=.18\linewidth]{figure/room/008.jpg}
        \includegraphics[width=.18\linewidth]{figure/room/009.jpg}
        \vspace{.125cm}
    \end{subfigure}
    \begin{subfigure}[b]{1.0\textwidth}
    \centering
        \includegraphics[width=.18\linewidth]{figure/room/010.jpg}
        \includegraphics[width=.18\linewidth]{figure/room/011.jpg}
        \includegraphics[width=.18\linewidth]{figure/room/012.jpg}
        \includegraphics[width=.18\linewidth]{figure/room/013.jpg}
        \includegraphics[width=.18\linewidth]{figure/room/014.jpg}
        \vspace{.125cm}
    \end{subfigure}
    \begin{subfigure}[b]{1.0\textwidth}
    \centering
        \includegraphics[width=.18\linewidth]{figure/room/015.jpg}
        \includegraphics[width=.18\linewidth]{figure/room/016.jpg}
        \includegraphics[width=.18\linewidth]{figure/room/017.jpg}
        \includegraphics[width=.18\linewidth]{figure/room/018.jpg}
        \includegraphics[width=.18\linewidth]{figure/room/019.jpg}
        \vspace{.125cm}
    \end{subfigure}
    \begin{subfigure}[b]{1.0\textwidth}
    \centering
        \includegraphics[width=.18\linewidth]{figure/room/020.jpg}
        \includegraphics[width=.18\linewidth]{figure/room/021.jpg}
        \includegraphics[width=.18\linewidth]{figure/room/022.jpg}
        \includegraphics[width=.18\linewidth]{figure/room/023.jpg}
        \includegraphics[width=.18\linewidth]{figure/room/024.jpg}
        \vspace{.125cm}
    \end{subfigure}
    \begin{subfigure}[b]{1.0\textwidth}
    \centering
        \includegraphics[width=.18\linewidth]{figure/room/025.jpg}
        \includegraphics[width=.18\linewidth]{figure/room/026.jpg}
        \includegraphics[width=.18\linewidth]{figure/room/027.jpg}
        \includegraphics[width=.18\linewidth]{figure/room/028.jpg}
        \includegraphics[width=.18\linewidth]{figure/room/029.jpg}
        \vspace{.125cm}
    \end{subfigure}
    \begin{subfigure}[b]{1.0\textwidth}
    \centering
        \includegraphics[width=.18\linewidth]{figure/room/101.jpg}
        \includegraphics[width=.18\linewidth]{figure/room/102.jpg}
        \includegraphics[width=.18\linewidth]{figure/room/103.jpg}
        \includegraphics[width=.18\linewidth]{figure/room/104.jpg}
        \includegraphics[width=.18\linewidth]{figure/room/105.jpg}
        \vspace{.125cm}
    \end{subfigure}
    \begin{subfigure}[b]{1.0\textwidth}
    \centering
        \includegraphics[width=.18\linewidth]{figure/room/106.jpg}
        \includegraphics[width=.18\linewidth]{figure/room/107.jpg}
        \includegraphics[width=.18\linewidth]{figure/room/108.jpg}
        \includegraphics[width=.18\linewidth]{figure/room/109.jpg}
        \includegraphics[width=.18\linewidth]{figure/room/110.jpg}
    \end{subfigure}
     \vspace{-20pt}
    \caption{
        \textbf{Illumination variation samples of the \textit{room} scene in NeRF Extreme. }}
    \label{fig:roomall} \vspace{-10pt}
\end{figure*}


\begin{figure*}
    \centering
    \begin{subfigure}[b]{1.0\textwidth}
         \centering
        \includegraphics[width=.18\linewidth]{figure/tent/000.jpg}
        \includegraphics[width=.18\linewidth]{figure/tent/001.jpg}
        \includegraphics[width=.18\linewidth]{figure/tent/002.jpg}
        \includegraphics[width=.18\linewidth]{figure/tent/003.jpg}
        \includegraphics[width=.18\linewidth]{figure/tent/004.jpg}
        \vspace{.125cm}
    \end{subfigure}
    \begin{subfigure}[b]{1.0\textwidth}
    \centering
        \includegraphics[width=.18\linewidth]{figure/tent/005.jpg}
        \includegraphics[width=.18\linewidth]{figure/tent/006.jpg}
        \includegraphics[width=.18\linewidth]{figure/tent/007.jpg}
        \includegraphics[width=.18\linewidth]{figure/tent/008.jpg}
        \includegraphics[width=.18\linewidth]{figure/tent/009.jpg}
        \vspace{.125cm}
    \end{subfigure}
    \begin{subfigure}[b]{1.0\textwidth}
    \centering
        \includegraphics[width=.18\linewidth]{figure/tent/010.jpg}
        \includegraphics[width=.18\linewidth]{figure/tent/011.jpg}
        \includegraphics[width=.18\linewidth]{figure/tent/012.jpg}
        \includegraphics[width=.18\linewidth]{figure/tent/013.jpg}
        \includegraphics[width=.18\linewidth]{figure/tent/014.jpg}
        \vspace{.125cm}
    \end{subfigure}
    \begin{subfigure}[b]{1.0\textwidth}
    \centering
        \includegraphics[width=.18\linewidth]{figure/tent/015.jpg}
        \includegraphics[width=.18\linewidth]{figure/tent/016.jpg}
        \includegraphics[width=.18\linewidth]{figure/tent/017.jpg}
        \includegraphics[width=.18\linewidth]{figure/tent/018.jpg}
        \includegraphics[width=.18\linewidth]{figure/tent/019.jpg}
        \vspace{.125cm}
    \end{subfigure}
    \begin{subfigure}[b]{1.0\textwidth}
    \centering
        \includegraphics[width=.18\linewidth]{figure/tent/020.jpg}
        \includegraphics[width=.18\linewidth]{figure/tent/021.jpg}
        \includegraphics[width=.18\linewidth]{figure/tent/022.jpg}
        \includegraphics[width=.18\linewidth]{figure/tent/023.jpg}
        \includegraphics[width=.18\linewidth]{figure/tent/024.jpg}
        \vspace{.125cm}
    \end{subfigure}
    \begin{subfigure}[b]{1.0\textwidth}
    \centering
        \includegraphics[width=.18\linewidth]{figure/tent/025.jpg}
        \includegraphics[width=.18\linewidth]{figure/tent/026.jpg}
        \includegraphics[width=.18\linewidth]{figure/tent/027.jpg}
        \includegraphics[width=.18\linewidth]{figure/tent/028.jpg}
        \includegraphics[width=.18\linewidth]{figure/tent/029.jpg}
        \vspace{.125cm}
    \end{subfigure}
    \begin{subfigure}[b]{1.0\textwidth}
    \centering
        \includegraphics[width=.18\linewidth]{figure/tent/101.jpg}
        \includegraphics[width=.18\linewidth]{figure/tent/102.jpg}
        \includegraphics[width=.18\linewidth]{figure/tent/103.jpg}
        \includegraphics[width=.18\linewidth]{figure/tent/104.jpg}
        \includegraphics[width=.18\linewidth]{figure/tent/105.jpg}
        \vspace{.125cm}
    \end{subfigure}
    \begin{subfigure}[b]{1.0\textwidth}
    \centering
        \includegraphics[width=.18\linewidth]{figure/tent/106.jpg}
        \includegraphics[width=.18\linewidth]{figure/tent/107.jpg}
        \includegraphics[width=.18\linewidth]{figure/tent/108.jpg}
        \includegraphics[width=.18\linewidth]{figure/tent/109.jpg}
        \includegraphics[width=.18\linewidth]{figure/tent/110.jpg}
    \end{subfigure}
     \vspace{-20pt}
    \caption{
        \textbf{Illumination variation samples of the \textit{tent} scene in NeRF Extreme. }}
    \label{fig:tentall} \vspace{-10pt}
\end{figure*}

\begin{figure*}
    \centering
    \begin{subfigure}[b]{1.0\textwidth}
         \centering
        \includegraphics[width=.18\linewidth]{figure/bench/000.jpg}
        \includegraphics[width=.18\linewidth]{figure/bench/001.jpg}
        \includegraphics[width=.18\linewidth]{figure/bench/002.jpg}
        \includegraphics[width=.18\linewidth]{figure/bench/003.jpg}
        \includegraphics[width=.18\linewidth]{figure/bench/004.jpg}
        \vspace{.125cm}
    \end{subfigure}
    \begin{subfigure}[b]{1.0\textwidth}
    \centering
        \includegraphics[width=.18\linewidth]{figure/bench/005.jpg}
        \includegraphics[width=.18\linewidth]{figure/bench/006.jpg}
        \includegraphics[width=.18\linewidth]{figure/bench/007.jpg}
        \includegraphics[width=.18\linewidth]{figure/bench/008.jpg}
        \includegraphics[width=.18\linewidth]{figure/bench/009.jpg}
        \vspace{.125cm}
    \end{subfigure}
    \begin{subfigure}[b]{1.0\textwidth}
    \centering
        \includegraphics[width=.18\linewidth]{figure/bench/010.jpg}
        \includegraphics[width=.18\linewidth]{figure/bench/011.jpg}
        \includegraphics[width=.18\linewidth]{figure/bench/012.jpg}
        \includegraphics[width=.18\linewidth]{figure/bench/013.jpg}
        \includegraphics[width=.18\linewidth]{figure/bench/014.jpg}
        \vspace{.125cm}
    \end{subfigure}
    \begin{subfigure}[b]{1.0\textwidth}
    \centering
        \includegraphics[width=.18\linewidth]{figure/bench/015.jpg}
        \includegraphics[width=.18\linewidth]{figure/bench/016.jpg}
        \includegraphics[width=.18\linewidth]{figure/bench/017.jpg}
        \includegraphics[width=.18\linewidth]{figure/bench/018.jpg}
        \includegraphics[width=.18\linewidth]{figure/bench/019.jpg}
        \vspace{.125cm}
    \end{subfigure}
    \begin{subfigure}[b]{1.0\textwidth}
    \centering
        \includegraphics[width=.18\linewidth]{figure/bench/020.jpg}
        \includegraphics[width=.18\linewidth]{figure/bench/021.jpg}
        \includegraphics[width=.18\linewidth]{figure/bench/022.jpg}
        \includegraphics[width=.18\linewidth]{figure/bench/023.jpg}
        \includegraphics[width=.18\linewidth]{figure/bench/024.jpg}
        \vspace{.125cm}
    \end{subfigure}
    \begin{subfigure}[b]{1.0\textwidth}
    \centering
        \includegraphics[width=.18\linewidth]{figure/bench/025.jpg}
        \includegraphics[width=.18\linewidth]{figure/bench/026.jpg}
        \includegraphics[width=.18\linewidth]{figure/bench/027.jpg}
        \includegraphics[width=.18\linewidth]{figure/bench/028.jpg}
        \includegraphics[width=.18\linewidth]{figure/bench/029.jpg}
        \vspace{.125cm}
    \end{subfigure}
    \begin{subfigure}[b]{1.0\textwidth}
    \centering
        \includegraphics[width=.18\linewidth]{figure/bench/101.jpg}
        \includegraphics[width=.18\linewidth]{figure/bench/102.jpg}
        \includegraphics[width=.18\linewidth]{figure/bench/103.jpg}
        \includegraphics[width=.18\linewidth]{figure/bench/104.jpg}
        \includegraphics[width=.18\linewidth]{figure/bench/105.jpg}
        \vspace{.125cm}
    \end{subfigure}
    \begin{subfigure}[b]{1.0\textwidth}
    \centering
        \includegraphics[width=.18\linewidth]{figure/bench/106.jpg}
        \includegraphics[width=.18\linewidth]{figure/bench/107.jpg}
        \includegraphics[width=.18\linewidth]{figure/bench/108.jpg}
        \includegraphics[width=.18\linewidth]{figure/bench/109.jpg}
        \includegraphics[width=.18\linewidth]{figure/bench/110.jpg}
    \end{subfigure}
     \vspace{-20pt}
    \caption{
        \textbf{Illumination variation samples of the \textit{bench} scene in NeRF Extreme. }}
    \label{fig:benchall} \vspace{-10pt}
\end{figure*}


%#################################################################################

%#################################################################################



\clearpage

\clearpage
%%%%%%%%% REFERENCES
%%%%%%%%% REFERENCES
{\small
\bibliographystyle{ieee_fullname}
\bibliography{egbib}
}


\end{document}