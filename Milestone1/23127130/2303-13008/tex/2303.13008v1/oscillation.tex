
\RequirePackage[2020-02-02]{latexrelease}
\documentclass[aps,prd,groupedaddress,graphicx,nofootinbib]{revtex4}
%\usepackage[margin=2cm]{geometry}
\usepackage{amsmath,amssymb,graphics,graphicx,color,epsf}
\usepackage{subfigure}
\usepackage[scanall]{psfrag}  % the scanall option enables \tex inside EPS files
\usepackage{tikz}
\DeclareMathOperator{\sech}{sech}


\begin{document}

\title{On the oscillations of the inflaton field of the simplest $\alpha$-attractor T-model}



\author{Chia-Min Lin}

\affiliation{Fundamental General Education Center, National Chin-Yi University of Technology, Taichung 41170, Taiwan}



%\baselineskip 14pt


%\date{Draft \today}



\begin{abstract}
In this work, we consider homogeneous oscillations of the inflaton field after inflation. In particular, we obtain an analytical result for the (average) equation of state for the oscillating inflaton field for the simplest $\alpha$-attractor T-model. The result is useful for the study of its post-inflationary evolution. The most dramatic possibility is that during inflaton field oscillation, the (average) equation of state is that of a cosmological constant. This implies the end of slow-roll inflation in this model could be the beginning of oscillating inflation. 


\end{abstract}
\maketitle
\large
\baselineskip 18pt
\section{Introduction}

Cosmic inflation \cite{Starobinsky:1980te, Guth:1980zm, Linde:1981mu} is probably the most popular scenario for the very early universe cosmology. 
It can explain many problems of our universe, such as why it is so big and why it is so homogeneous but not that homogeneous. 
%Alan Guth describes inflation as the ``bang" of the big bang \cite{Guth:1997qr}, indeed it is also the ``big" of the big bang.
In many models of cosmic inflation, inflation is driven by the potential energy density of a scalar field called the inflaton field. 
Apparently, inflation has to end (or have a graceful exit, as is often put). We have to recover a decelerating universe at least in our observable universe in order to have successful structure formation.
After slow-roll inflation, the inflaton field starts to enter an oscillation phase. If we can understand better about this oscillation period, we would have a better understanding of the post-inflationary evolution history such as the early dark age, the mechanism of (p)reheating, and/or baryogenesis to name a few.

The (average) equation of state of the oscillation of a scalar field in an expanding universe \cite{Turner:1983he} is well known for a potential of a power function $V=\lambda \phi^n$. It seems this is the only example where we know a closed form solution. In this work, we investigate the simplest\footnote{In general, $\alpha$-attractor T-model takes the potential form $V=V_0 \tanh^{2n}\left( \frac{\phi}{F} \right)$, the simplest one corresponds to $n=1$.} $\alpha$-attractor T-model of inflation \cite{Kallosh:2013hoa, Kallosh:2013yoa, Carrasco:2015pla} with the potential
\begin{equation}
V(\phi)=V_0\tanh^2\left( \frac{\phi}{F} \right).
\label{alpha}
\end{equation}
We show that the (average) equation of state can be solved in this model. For the impatient reader, our result is given in Eq.~(\ref{main}). This result is useful in the study of the simplest $\alpha$-attractor T-model of inflation. In the following, I start a journey of this calculation.


\section{oscillating scalar field in an expanding universe}

The oscillation of a homogeneous scalar field in an expanding universe was considered in \cite{Turner:1983he}. We provide a brief review of the relevant parameters in this section and fill in some gaps in the calculations.

For a homogeneous scalar field $\phi$ with potential $V$, the energy density $\rho$ is given by
\begin{equation}
\rho=\frac{\dot{\phi}^2}{2}+V,
\label{rho}
\end{equation}
and the pressure is
\begin{equation}
p=\frac{\dot{\phi}^2}{2}-V.
\label{p}
\end{equation}
Therefore, for an oscillating scalar field, we have
\begin{equation}
\dot{\phi}^2=\rho+p = \left( 1+\frac{p}{\rho} \right)\rho \equiv \left( 1+w \right) \rho \equiv \left( \gamma+\gamma_p \right) \rho,
\label{gp}
\end{equation}
where $w$ is the parameter for the equation of state. We have defined $\gamma$ as the average of $(\rho+p)$ over an oscillation, and $\gamma_p$ is the periodic part of $\dot{\phi}^2$ following \cite{Turner:1983he}. It is assumed that the frequency of oscillations $\omega \simeq \dot{\phi}/\phi$ satisfies
\begin{equation}
\frac{1}{\omega} \ll \frac{1}{H}.
\label{omega}
\end{equation}
This means the period of oscillation is much smaller than the characteristic time scale (or ``age") of the universe. 

We would like to calculate the (average) equation of state.
In a textbook of cosmology \cite{Mukhanov:2005sc}, the argument goes like this. If we neglect the Hubble expansion (due to the fast oscillation given by Eq.~(\ref{omega})), the equation of motion becomes
\begin{equation}
(\phi\dot{\phi}\dot{)}-\dot{\phi}^2+\phi V^\prime=0.
\end{equation} 
This is nothing but the Klein-Gorden equation for a homogeneous scalar field. The first term drops out upon averaging over one period and gives $\langle \dot{\phi}^2 \rangle = \langle \phi V^\prime \rangle$.
Therefore from Eqs.~(\ref{rho}) and (\ref{p}), we obtain
\begin{equation}
\langle w \rangle=\frac{p}{\rho}=\frac{\langle \phi V^\prime \rangle -\langle 2V \rangle}{\langle \phi V^\prime \rangle +\langle 2V \rangle}.
\end{equation}
For example, if $V=\lambda \phi^n$, we have
\begin{equation}
\langle w \rangle=\frac{n\langle \phi^n \rangle -2\langle \phi^n \rangle}{n\langle \phi^n \rangle +2\langle \phi^n \rangle}=\frac{n-2}{n+2}.
\end{equation}
Here the $\langle \phi^n \rangle$ cancels out and we do not have to worry about it. This cancellation only happens when $\phi V^\prime \propto V$, namely the power function. Usually, it is argued that for a general potential, one can power expand it and when the field value is small, the lowest power term dominates and we can use the above formula. 
%For example, this is the case to consider the $\alpha$-attractor T-model in \cite{Garcia:2020wiy, Ueno:2016dim, Eshaghi:2016kne, German:2020cbw}. 
In this work, we would like to be a little bit more ambitious to calculate the (average) equation of state.

Let us return to the treatment of \cite{Turner:1983he}.
From the continuity equation
\begin{equation}
\frac{d\rho}{dt}=-3H(\gamma+\gamma_p)\rho,
\end{equation}
we have
\begin{equation}
\frac{d\rho}{\rho}=-3\frac{da}{a}\gamma  -3H \gamma_p dt.
\label{c}
\end{equation}
When we integrate Eq.~(\ref{c}), the second term on the right-handed side is negligible due to the condition given by Eq.~(\ref{omega}). In particular, if $\gamma$ is constant, we obtain
\begin{equation}
\rho=a^{-3\gamma}.
\end{equation}
This expression is simpler than that by using $\langle w \rangle$. This is one of the reasons to use $\gamma$.
By definition, $\gamma$ is the average value of $1+w$. Let us calculate the average over one period from $t=0$ to $t=T$. During one oscillation, due to Eq.~(\ref{omega}), $\rho$ can be treated as a constant and we can write $\rho=V_m$, with $V_m$ the potential at the maximum value (turning point of the oscillation) of $\phi$ at $\phi_m$ where $\dot{\phi}_m=0$. The average of $1+w$ is
\begin{equation}
\gamma=1+\langle w \rangle= \frac{1}{T}\int^T_0 \frac{\rho+p}{\rho}dt=\frac{1}{T}\int^T_0 \frac{\dot{\phi}^2}{V_m}dt,
\label{g1}
\end{equation}
where we have used Eqs.~(\ref{rho}) and (\ref{p}) to calculate $\rho+p$.
From Eq.~(\ref{rho}), we have
\begin{equation}
\rho=V_m=\frac{\dot{\phi}^2}{2}+V.
\label{m}
\end{equation}
Therefore 
\begin{equation}
\dot{\phi}=\sqrt{2}\sqrt{V_m-V}.
\label{m2}
\end{equation}
The period is given by
\begin{equation}
T=\int^T_0 dt = 2\int^{\phi_m}_0 \frac{d\phi}{\dot{\phi}}= 2 \int^{\phi_m}_0 \frac{d\phi}{\sqrt{2}\sqrt{V_m-V}}, 
\label{t}
\end{equation}
where we have assumed the potential is an even function $V(\phi)=V(-\phi)$ and set the lower limit of the integral to $\phi=0$. The last equality is from Eq.~(\ref{m2}). Similarly, we can write
\begin{equation}
\int^T_0 \frac{\dot{\phi}^2}{V_m}dt=2 \int^{\phi_m}_0 \frac{\dot{\phi}^2}{V_m}\frac{d\phi}{\dot{\phi}}=\int^{\phi_m}_0 \frac{\sqrt{2}\sqrt{V_m-V}}{V_m}.
\label{f}
\end{equation}
By using Rqs.~(\ref{g1}), (\ref{t}), and (\ref{f}), we obtain \cite{Turner:1983he}
\begin{equation}
\gamma=2 \frac{\int^{\phi_m}_0 \left( 1-\frac{V}{V_m} \right)^{1/2}d\phi}{\int^{\phi_m}_0 \left( 1-\frac{V}{V_m} \right)^{-1/2}d\phi}.
\label{gamma}
\end{equation}
For example, if $V=\lambda \phi^n$, we have
\begin{equation}
\gamma=2 \frac{\int^{\phi_m}_0 \left( 1-\frac{\phi^n}{\phi^n_m} \right)^{1/2}d\phi}{\int^{\phi_m}_0 \left( 1-\frac{\phi^n}{\phi^n_m} \right)^{-1/2}d\phi}.
\label{g}
\end{equation}
In \cite{Turner:1983he}, the author only mentioned that it is ``straightforward" to integrate the above integral to give
\begin{equation}
\gamma=\frac{2n}{n+2}.
\label{pow}
\end{equation} 
I would like to be more explicit here for the readers who do not see the straightforwardness. To my knowledge, I have never seen this calculation presented in a textbook or elsewhere. Let us define
\begin{equation}
x \equiv \left( \frac{\phi}{\phi_m} \right)^n.
\end{equation}
The integral in the numerator becomes
\begin{eqnarray}
\int^{\phi_m}_0 \left( 1-\frac{\phi^n}{\phi^n_m} \right)^{1/2}d\phi &=&\frac{\phi_m}{n} \int^1_0 \left(1-x\right)^{1/2}x^{\frac{1-n}{n}}dx  \\ 
                                                                                                     &=&\frac{\phi_m}{n} \frac{\Gamma\left( \frac{1}{n} \right)\Gamma\left( \frac{3}{2} \right)}{\Gamma\left( \frac{1}{n} +\frac{3}{2} \right)},
\end{eqnarray}
where the result is from a beta function. Similarly, the denominator is
\begin{eqnarray}
\int^{\phi_m}_0 \left( 1-\frac{\phi^n}{\phi^n_m} \right)^{-1/2}d\phi &=&\frac{\phi_m}{n} \int^1_0 \left(1-x\right)^{-1/2}x^{\frac{1-n}{n}}dx  \\ 
                                                                                                     &=&\frac{\phi_m}{n} \frac{\Gamma\left( \frac{1}{n} \right)\Gamma\left( \frac{1}{2} \right)}{\Gamma\left( \frac{1}{n} +\frac{1}{2} \right)},
\end{eqnarray}
Therefore we can obtain
\begin{equation}
\gamma=2 \frac{\Gamma\left( \frac{1}{2}+1 \right)\Gamma\left( \frac{1}{n} +\frac{1}{2} \right)}{\Gamma\left( \frac{1}{2} \right)\Gamma\left( \frac{1}{n} +\frac{1}{2}+1 \right)}=\frac{2n}{n+2},
\end{equation}
where I have used $\Gamma(x+1)=x\Gamma(x)$.
%This is a beautiful result, but apparently, this calculation is also limited to a power function.
This is one method to do the integral, here I have another method to do the integral without using beta functions. Let us write 
\begin{eqnarray}
\int^{\phi_m}_0 \left( 1-\frac{\phi^n}{\phi^n_m} \right)^{1/2}d\phi&=&\int^{\phi_m}_0 \left( 1-\frac{\phi^n}{\phi^n_m} \right)^{-1/2}  \left( 1-\frac{\phi^n}{\phi^n_m} \right)     d\phi  \nonumber  \\
&=&\int^{\phi_m}_0 \left( 1-\frac{\phi^n}{\phi^n_m} \right)^{-1/2} d\phi  -\int^{\phi_m}_0 \frac{\phi^n}{\phi^n_m}\left( 1-\frac{\phi^n}{\phi^n_m} \right)^{-1/2} d\phi \nonumber \\
&=&\int^{\phi_m}_0 \left( 1-\frac{\phi^n}{\phi^n_m} \right)^{-1/2} d\phi  -\int^{\phi_m}_0 n \frac{\phi^{n-1}}{\phi^n_m}\frac{\phi}{n} \left( 1-\frac{\phi^n}{\phi^n_m} \right)^{-1/2} d\phi  \nonumber  \\
&=&\int^{\phi_m}_0 \left( 1-\frac{\phi^n}{\phi^n_m} \right)^{-1/2} d\phi -\frac{2}{n} \int^{\phi_m}_0 \left( 1-\frac{\phi^n}{\phi^n_m} \right)^{1/2}d\phi,
\end{eqnarray}
where we have used integration by parts. Move the last term to the leftmost of the equalities, and the result follows from Eq.~(\ref{g}). I have done the review part, now let us challenge ourselves to calculate something new.


\section{The simplest $\alpha$-attractor T-model}
In this section, we do a similar calculation for the potential given by Eq.~(\ref{alpha}).
The first question we need to deal with is perhaps when would (slow-roll) inflation end and the inflaton field starts to oscillate. Let us set the reduced Planck mass $M_P \simeq 2.4 \times 10^{18}\mbox{ GeV}$ to $M_P=1$. We have the slow-rolling parameters (obtained from taking derivatives of Eq.~(\ref{alpha}))
\begin{equation}
\epsilon \equiv \frac{1}{2}\left( \frac{V^\prime}{V} \right)^2=\frac{2}{F^2 \sinh^2\left( \frac{\phi}{F} \right)\cosh^2\left( \frac{\phi}{F} \right)},
\end{equation}
and 
\begin{equation}
\eta \equiv \frac{V^{\prime\prime}}{V}=\frac{2}{F^2\cosh^2\left( \frac{\phi}{F} \right)} \left(  \frac{1}{\sinh^2\left( \frac{\phi}{F} \right)}-2  \right).
\end{equation}
%Inflation ends at $\phi=\phi_e$ when $\epsilon=1$. It can be solved to obtain
%\begin{equation}
%\frac{\phi_e}{F}=\ln\left( \sqrt{\frac{2\sqrt{2}}{F}+\sqrt{\frac{8}{F}+1}}\right).
%\end{equation}
Slow-roll inflation ends at $\phi=\phi_e$ when either $\epsilon=1$ or $|\eta|=1$ is achieved. The slow-roll parameters depend on $F$. We plot the case $F=1$ in Fig.~\ref{fig2} and the case $F=0.01$ in Fig.~\ref{fig3}. As we can see in the figures\footnote{Similar plots for the case which corresponds to $F \sim 1$ can be found in \cite{German:2021tqs}. Our $F$ corresponds to $1/\lambda$ in \cite{German:2021tqs}. Note that we have $M_P=1$.}, slow-roll inflation ends at $\phi_e/F \sim 1$ when $F=1$ and at $\phi_e/F \sim 6$ when $F=0.01$. We assume the inflaton field starts its rapid oscillation and satisfies Eq.~(\ref{omega}) when $\phi_e$ is achieved.
\begin{figure}[t]
  \centering
\includegraphics[width=0.6\textwidth]{f1.eps}
  \caption{Slow-roll parameters as a function of $\phi/F$ when $F=1$. Slow-roll inflation ends at $\phi_e/F \sim 1$.}
  \label{fig2}
\end{figure}

\begin{figure}[t]
  \centering
\includegraphics[width=0.6\textwidth]{f2.eps}
  \caption{Slow-roll parameters as a function of $\phi/F$ when $F=0.01$. Slow-roll inflation ends at $\phi_e/F \sim 6$.}
  \label{fig3}
\end{figure}

Let us calculate $\gamma$ in this model.
For notational simplicity during calculation, we define 
\begin{equation}
c \equiv \frac{V_0}{V_m}=\frac{1}{\tanh^2\left(\frac{\phi_m}{F}\right)}.
\end{equation}
Note that $c>1$. We have identities such as $\sqrt{c-1}=1/\sinh \left(\phi_m/F\right) $.
The numerator of Eq.~(\ref{gamma}) is 
\begin{eqnarray}
\int^{\phi_m}_0 \left( 1-c \tanh^2\left(\frac{\phi}{F}\right) \right)^{1/2}d\phi   
&=&\int^{\phi_m}_0 \left( 1-c \tanh^2\left(\frac{\phi}{F}\right) \right)^{-1/2}d\phi\left( 1-c \tanh^2\left(\frac{\phi}{F}\right) \right)d\phi   \nonumber  \\ &=&\int^{\phi_m}_0 \left( 1-c \tanh^2\left(\frac{\phi}{F}\right) \right)^{-1/2}d\phi \nonumber \\
&-&c \int^{\phi_m}_0 \left(1- \sech^2\left( \frac{\phi}{F} \right) \right)  \left( 1-c \tanh^2\left(\frac{\phi}{F}\right) \right)^{-1/2}d\phi  \nonumber \\
&=&-(c-1)\int^{\phi_m}_0 \left( 1-c \tanh^2\left(\frac{\phi}{F}\right) \right)^{-1/2}d\phi  \nonumber \\
&+&c \int^{\phi_m}_0 \sech^2\left( \frac{\phi}{F}  \right)  \left( 1-c \tanh^2\left(\frac{\phi}{F}\right) \right)^{-1/2}d\phi.  
\label{nu}
\end{eqnarray}
The last integral can be done as
\begin{eqnarray}
c \int^{\phi_m}_0 \sech^2\left( \frac{\phi}{F}  \right)  \left( 1-c \tanh^2\left(\frac{\phi}{F}\right) \right)^{-1/2}d\phi&=&cF \int^{\phi_m}_0 \frac{d\tanh \left( \frac{\phi}{F} \right)}{\sqrt{1-c \tanh^2 \left( \frac{\phi}{F} \right)}} \nonumber \\
&=&cF\left[ \frac{1}{\sqrt{c}} \sin^{-1}\left( \sqrt{c}\tanh\left( \frac{\phi}{F} \right) \right) \right]^{\phi_m}_0 \nonumber \\  &=&\frac{F\pi \sqrt{c}}{2}  \nonumber \\&=&\frac{F\pi}{2 \tanh\left(\frac{\phi_m}{F} \right)}.
\end{eqnarray}
The denominator of Eq.~(\ref{gamma}) is 
\begin{eqnarray}
\int^{\phi_m}_0 \left( 1-c \tanh^2\left(\frac{\phi}{F}\right) \right)^{-1/2}d\phi= F \int^{\phi_m}_0 \cosh \left( \frac{\phi}{F} \right)\frac{1}{\sqrt{1-(c-1)\sinh^2 \left( \frac{\phi}{F} \right)}} d\left( \frac{\phi}{F}\right).
\end{eqnarray}
Let $u=\sqrt{c-1}\sinh \left(\frac{\phi}{F} \right)$, the integral becomes
\begin{eqnarray}
\frac{F}{\sqrt{c-1}} \int^{\sqrt{c-1}\sinh \left(\frac{\phi_m}{F} \right)}_0 \frac{du}{\sqrt{1-u^2}}&=&\left[\frac{F}{\sqrt{c-1}} \sin^{-1}\left( \sqrt{c-1}\sinh \left(\frac{\phi}{F} \right) \right)\right]^{\phi_m}_0 \nonumber \\  &=& \frac{F\pi}{2} \sinh \left( \frac{\phi_m}{F} \right).
\end{eqnarray}
Therefore 
\begin{equation}
\int^{\phi_m}_0 \left( 1-c \tanh^2\left(\frac{\phi}{F}\right) \right)^{1/2}d\phi  =  \frac{F\pi}{2 \tanh\left(\frac{\phi_m}{F} \right)}-\frac{F\pi}{2 \sinh \left(\frac{\phi_m}{F} \right)}.
\end{equation}
This is also the other integral that needs to be done in Eq.~(\ref{nu}).
Substituting these results into Eq.~(\ref{gamma}), we have
\begin{equation}
\gamma=1+\langle w \rangle=\frac{2}{\cosh \left(\frac{\phi_m}{F} \right)+1}.
\label{main}
\end{equation}
This is the main result of our work. We plot $\gamma$ as a function of $\phi_m/F$ in Fig.~\ref{fig1}. Note that for a fixing $F$, $\gamma$ is a function of $\phi_m$, not $\phi$ which is oscillating rapidly. If one studies the equation of state parameter $w$ as a function of $\phi$, it would be oscillating rapidly. Here $\phi_m$ is the envelope of the oscillating $\phi$. When $\phi_m/F \rightarrow 0$, we have $\gamma \rightarrow 1$. This corresponds to $\langle w \rangle =\gamma-1 \rightarrow 0$. The equation of state is that of nonrelativistic (cold) matter. This is reasonable because when $\phi_m/F \rightarrow 0$, the potential in Eq.~(\ref{alpha}) approaches $V \propto \phi^2$. From Eq.~(\ref{pow}), for a quadratic potential $n=2$, we have $\gamma=1$.

On the other hand, when $\phi_m$ becomes large, $\gamma$ decreases toward zero. This can only happen for a small value of $F$. For example, in order to have $\phi_m/F \gtrsim 6$, we need $F \lesssim 0.01$ as can be seen from Fig.~\ref{fig3}. The result can be dramatic though. When $\gamma \rightarrow 0$, we have $\langle w \rangle \rightarrow -1$ and the (average) equation of state approaches that of a cosmological constant even for an oscillating scalar field! This phenomenon is known for a different potential and has been called oscillating inflation (or oscillatory inflation) \cite{Damour:1997cb} (see also \cite{Liddle:1998pz, Taruya:1998cz, Cardenas:1999cw, Lee:1999pta, Sami:2001zd, Koutvitsky:2016rkw}). Here we have discovered another model of oscillating inflation. The end of slow-roll inflation in this case is the beginning of another period of inflation due to a different mechanism. During oscillating inflation, the inflaton field is not slow-rolling but rapidly oscillating instead. Physically the reason why oscillating inflation can happen is because of the inflaton field spends most of its time on the plateau of the potential during oscillation. In \cite{Kallosh:2013yoa}, the limit of $F \rightarrow 0$ is taken for a universal attractor\footnote{Our $F$ corresponds to $\sqrt{6\alpha}$ of \cite{Kallosh:2013yoa} and $\alpha \rightarrow 0$ is considered.}. Under this condition, oscillating inflation as described above is expected to happen. In the case of a very small $F$, it is possible that the whole period of observable inflation is due to oscillating inflation.

\begin{figure}[t]
  \centering
\includegraphics[width=0.6\textwidth]{g.eps}
  \caption{$\gamma$ as a function of $\phi_m/F$ after slow-roll inflation. $\phi_m$ is the envelope of the oscillating $\phi$.}
  \label{fig1}
\end{figure}


\section{conclusion}
\label{con}
In this work, we provide a detailed treatment for an oscillating homogeneous scalar field in an expanding universe. In particular, we find the (average) equation of state for the simplest $\alpha$-attractor T-model. Our main result is given by Eq.~(\ref{main}) which is plotted in Fig.~\ref{fig1}. It shows that when $\phi_m/F \rightarrow 0$, the equation of state is that of nonrelativistic (cold) matter. On the other hand, when $\phi_m/F \gtrsim 6$, which can only happen for $F \lesssim 0.01$, the equation of state approaches that of a cosmological constant. We discovered that oscillating inflation can occur in this model.

Our result should also have ramifications for the study of (p)reheating \cite{Garcia:2020wiy, Eshaghi:2016kne, Krajewski:2022ezo, Krajewski:2018moi, Ueno:2016dim, German:2020cbw, Shojaee:2020xyr, Podolsky:2005bw} and oscillons \cite{Copeland:1995fq, Zhang:2020ntm, Lozanov:2017hjm, Lozanov:2016hid, Lozanov:2019ylm, Zhang:2020bec} after $\alpha$-attractor T-model of inflation. %Perhaps this can be used as a novel idea of dark energy.

%Although we have so far considered only the simplest T-model. The result does show that a nontrivial average equation of motion can appear during the inflaton field oscillation. This sheds some light to more general cases. It is plausible that a similar phenomenon could also appear for other T-models. 


\acknowledgments
This work is supported by the National Science and Technology Council (NSTC) of Taiwan under Grant No. NSTC 111-2112-M-167-002.

\begin{thebibliography}{99}


\bibitem{Starobinsky:1980te}
A.~A.~Starobinsky,
``A New Type of Isotropic Cosmological Models Without Singularity,''
Phys. Lett. B \textbf{91}, 99-102 (1980)
doi:10.1016/0370-2693(80)90670-X
%6050 citations counted in INSPIRE as of 27 Jan 2023

\bibitem{Guth:1980zm}
A.~H.~Guth,
``The Inflationary Universe: A Possible Solution to the Horizon and Flatness Problems,''
Phys. Rev. D \textbf{23}, 347-356 (1981)
doi:10.1103/PhysRevD.23.347
%9337 citations counted in INSPIRE as of 27 Jan 2023

\bibitem{Linde:1981mu}
A.~D.~Linde,
``A New Inflationary Universe Scenario: A Possible Solution of the Horizon, Flatness, Homogeneity, Isotropy and Primordial Monopole Problems,''
Phys. Lett. B \textbf{108}, 389-393 (1982)
doi:10.1016/0370-2693(82)91219-9
%5894 citations counted in INSPIRE as of 27 Jan 2023

%\bibitem{Guth:1997qr}
%A.~H.~Guth,
%``Was cosmic inflation the 'bang' of the big bang?,''
%SLAC Beam Line \textbf{27N3}, 14 (1997)
%0 citations counted in INSPIRE as of 19 Mar 2023

\bibitem{Turner:1983he}
M.~S.~Turner,
``Coherent Scalar Field Oscillations in an Expanding Universe,''
Phys. Rev. D \textbf{28}, 1243 (1983)
doi:10.1103/PhysRevD.28.1243
%776 citations counted in INSPIRE as of 17 Mar 2023


\bibitem{Kallosh:2013hoa}
R.~Kallosh and A.~Linde,
``Universality Class in Conformal Inflation,''
JCAP \textbf{07}, 002 (2013)
doi:10.1088/1475-7516/2013/07/002
[arXiv:1306.5220 [hep-th]].
%530 citations counted in INSPIRE as of 17 Mar 2023

\bibitem{Kallosh:2013yoa}
R.~Kallosh, A.~Linde and D.~Roest,
``Superconformal Inflationary $\alpha$-Attractors,''
JHEP \textbf{11}, 198 (2013)
doi:10.1007/JHEP11(2013)198
[arXiv:1311.0472 [hep-th]].
%582 citations counted in INSPIRE as of 19 Mar 2023

\bibitem{Carrasco:2015pla}
J.~J.~M.~Carrasco, R.~Kallosh and A.~Linde,
``$\alpha $-Attractors: Planck, LHC and Dark Energy,''
JHEP \textbf{10}, 147 (2015)
doi:10.1007/JHEP10(2015)147
[arXiv:1506.01708 [hep-th]].
%133 citations counted in INSPIRE as of 20 Mar 2023



\bibitem{Mukhanov:2005sc}
V.~Mukhanov,
``Physical Foundations of Cosmology,''
Cambridge University Press, 2005,
ISBN 978-0-521-56398-7
doi:10.1017/CBO9780511790553
%510 citations counted in INSPIRE as of 19 Mar 2023




\bibitem{German:2021tqs}
G.~German,
``On the \ensuremath{\alpha}-attractor T-models,''
JCAP \textbf{09}, 017 (2021)
doi:10.1088/1475-7516/2021/09/017
[arXiv:2105.05426 [astro-ph.CO]].
%2 citations counted in INSPIRE as of 19 Mar 2023

\bibitem{Damour:1997cb}
T.~Damour and V.~F.~Mukhanov,
``Inflation without slow roll,''
Phys. Rev. Lett. \textbf{80}, 3440-3443 (1998)
doi:10.1103/PhysRevLett.80.3440
[arXiv:gr-qc/9712061 [gr-qc]].
%85 citations counted in INSPIRE as of 21 Mar 2023

\bibitem{Liddle:1998pz}
A.~R.~Liddle and A.~Mazumdar,
``Inflation during oscillations of the inflaton,''
Phys. Rev. D \textbf{58}, 083508 (1998)
doi:10.1103/PhysRevD.58.083508
[arXiv:astro-ph/9806127 [astro-ph]].
%46 citations counted in INSPIRE as of 21 Mar 2023

\bibitem{Taruya:1998cz}
A.~Taruya,
``Parametric amplification of density perturbation in the oscillating inflation,''
Phys. Rev. D \textbf{59}, 103505 (1999)
doi:10.1103/PhysRevD.59.103505
[arXiv:hep-ph/9812342 [hep-ph]].
%41 citations counted in INSPIRE as of 21 Mar 2023

\bibitem{Cardenas:1999cw}
V.~H.~Cardenas and G.~Palma,
``Some remarks on oscillating inflation,''
Phys. Rev. D \textbf{61}, 027302 (2000)
doi:10.1103/PhysRevD.61.027302
[arXiv:astro-ph/9904313 [astro-ph]].
%12 citations counted in INSPIRE as of 21 Mar 2023

\bibitem{Lee:1999pta}
J.~w.~Lee, S.~Koh, C.~Park, S.~J.~Sin and C.~H.~Lee,
``Oscillating inflation with a nonminimally coupled scalar field,''
Phys. Rev. D \textbf{61}, 027301 (2000)
doi:10.1103/PhysRevD.61.027301
[arXiv:hep-th/9909106 [hep-th]].
%29 citations counted in INSPIRE as of 21 Mar 2023

\bibitem{Sami:2001zd}
M.~Sami,
``Inflation with oscillations,''
Grav. Cosmol. \textbf{8}, 309-312 (2003)
[arXiv:gr-qc/0106074 [gr-qc]].
%14 citations counted in INSPIRE as of 21 Mar 2023

\bibitem{Koutvitsky:2016rkw}
V.~A.~Koutvitsky and E.~M.~Maslov,
``On the oscillation-driven cosmological expansion at the post-inflationary stage,''
Grav. Cosmol. \textbf{23}, no.1, 35-40 (2017)
doi:10.1134/S0202289317010078
[arXiv:1611.03734 [gr-qc]].
%1 citations counted in INSPIRE as of 21 Mar 2023


\bibitem{Garcia:2020wiy}
M.~A.~G.~Garcia, K.~Kaneta, Y.~Mambrini and K.~A.~Olive,
``Inflaton Oscillations and Post-Inflationary Reheating,''
JCAP \textbf{04}, 012 (2021)
doi:10.1088/1475-7516/2021/04/012
[arXiv:2012.10756 [hep-ph]].
%42 citations counted in INSPIRE as of 19 Mar 2023

\bibitem{Ueno:2016dim}
Y.~Ueno and K.~Yamamoto,
``Constraints on $\alpha$-attractor inflation and reheating,''
Phys. Rev. D \textbf{93}, no.8, 083524 (2016)
doi:10.1103/PhysRevD.93.083524
[arXiv:1602.07427 [astro-ph.CO]].
%82 citations counted in INSPIRE as of 19 Mar 2023


\bibitem{Eshaghi:2016kne}
M.~Eshaghi, M.~Zarei, N.~Riazi and A.~Kiasatpour,
``CMB and reheating constraints to $\alpha$-attractor inflationary models,''
Phys. Rev. D \textbf{93}, no.12, 123517 (2016)
doi:10.1103/PhysRevD.93.123517
[arXiv:1602.07914 [astro-ph.CO]].
%50 citations counted in INSPIRE as of 19 Mar 2023

\bibitem{German:2020cbw}
G.~German,
``Constraining \ensuremath{\alpha}-attractor models from reheating,''
Int. J. Mod. Phys. D \textbf{31}, no.10, 2250081 (2022)
doi:10.1142/S021827182250081X
[arXiv:2010.09795 [astro-ph.CO]].
%12 citations counted in INSPIRE as of 19 Mar 2023



\bibitem{Krajewski:2022ezo}
T.~Krajewski and K.~Turzy\'nski,
``(P)reheating and gravitational waves in \ensuremath{\alpha}-attractor models,''
JCAP \textbf{10}, 005 (2022)
doi:10.1088/1475-7516/2022/10/005
[arXiv:2204.12909 [astro-ph.CO]].
%2 citations counted in INSPIRE as of 19 Mar 2023

\bibitem{Krajewski:2018moi}
T.~Krajewski, K.~Turzy\'nski and M.~Wieczorek,
``On preheating in $\alpha$-attractor models of inflation,''
Eur. Phys. J. C \textbf{79}, no.8, 654 (2019)
doi:10.1140/epjc/s10052-019-7155-z
[arXiv:1801.01786 [astro-ph.CO]].
%46 citations counted in INSPIRE as of 19 Mar 2023

\bibitem{Shojaee:2020xyr}
R.~Shojaee, K.~Nozari and F.~Darabi,
``\ensuremath{\alpha}-Attractors and reheating in a nonminimal inflationary model,''
Int. J. Mod. Phys. D \textbf{29}, no.10, 2050077 (2020)
doi:10.1142/S0218271820500777
[arXiv:2101.03981 [astro-ph.CO]].
%7 citations counted in INSPIRE as of 19 Mar 2023

\bibitem{Podolsky:2005bw}
D.~I.~Podolsky, G.~N.~Felder, L.~Kofman and M.~Peloso,
``Equation of state and beginning of thermalization after preheating,''
Phys. Rev. D \textbf{73}, 023501 (2006)
doi:10.1103/PhysRevD.73.023501
[arXiv:hep-ph/0507096 [hep-ph]].
%219 citations counted in INSPIRE as of 20 Mar 2023

\bibitem{Zhang:2020ntm}
H.~Y.~Zhang,
``Gravitational effects on oscillon lifetimes,''
JCAP \textbf{03}, 102 (2021)
doi:10.1088/1475-7516/2021/03/102
[arXiv:2011.11720 [hep-th]].
%7 citations counted in INSPIRE as of 19 Mar 2023

\bibitem{Copeland:1995fq}
E.~J.~Copeland, M.~Gleiser and H.~R.~Muller,
``Oscillons: Resonant configurations during bubble collapse,''
Phys. Rev. D \textbf{52}, 1920-1933 (1995)
doi:10.1103/PhysRevD.52.1920
[arXiv:hep-ph/9503217 [hep-ph]].
%254 citations counted in INSPIRE as of 19 Mar 2023

\bibitem{Lozanov:2017hjm}
K.~D.~Lozanov and M.~A.~Amin,
``Self-resonance after inflation: oscillons, transients and radiation domination,''
Phys. Rev. D \textbf{97}, no.2, 023533 (2018)
doi:10.1103/PhysRevD.97.023533
[arXiv:1710.06851 [astro-ph.CO]].
%138 citations counted in INSPIRE as of 19 Mar 2023

\bibitem{Lozanov:2016hid}
K.~D.~Lozanov and M.~A.~Amin,
``Equation of State and Duration to Radiation Domination after Inflation,''
Phys. Rev. Lett. \textbf{119}, no.6, 061301 (2017)
doi:10.1103/PhysRevLett.119.061301
[arXiv:1608.01213 [astro-ph.CO]].
%124 citations counted in INSPIRE as of 20 Mar 2023

\bibitem{Lozanov:2019ylm}
K.~D.~Lozanov and M.~A.~Amin,
``Gravitational perturbations from oscillons and transients after inflation,''
Phys. Rev. D \textbf{99}, no.12, 123504 (2019)
doi:10.1103/PhysRevD.99.123504
[arXiv:1902.06736 [astro-ph.CO]].
%72 citations counted in INSPIRE as of 21 Mar 2023

\bibitem{Zhang:2020bec}
H.~Y.~Zhang, M.~A.~Amin, E.~J.~Copeland, P.~M.~Saffin and K.~D.~Lozanov,
``Classical Decay Rates of Oscillons,''
JCAP \textbf{07}, 055 (2020)
doi:10.1088/1475-7516/2020/07/055
[arXiv:2004.01202 [hep-th]].
%43 citations counted in INSPIRE as of 21 Mar 2023

\end{thebibliography}
\end{document} 
















