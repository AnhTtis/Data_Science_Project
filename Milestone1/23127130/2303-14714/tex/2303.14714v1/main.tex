\documentclass[11pt]{article}
\usepackage[utf8]{inputenc}
\usepackage[
papersize={5.5in, 8.5in},
left=0.35in,
right=0.35in,
top=0.35in,
bottom=0.5in]{geometry}
\usepackage{graphicx}
\usepackage{amsthm}
\usepackage{rotating}
\usepackage{amsfonts,amssymb,amsmath}
\usepackage[usenames]{color}
\usepackage[T1]{fontenc}

\usepackage{array}
\usepackage{float}
\usepackage{booktabs}
\usepackage[dvipsnames,usenames,table,xcdraw]{xcolor}

\renewcommand{\Re}{\mathop{\mathrm{Re}}\nolimits}
\renewcommand{\Im}{\mathop{\mathrm{Im}}\nolimits}

\newcommand{\gr}{\cellcolor[HTML]{E0E0E0}}
\newcommand{\tbd}{\vspace{7cm}}

\DeclareMathOperator{\Aut}{Aut}
\graphicspath{ {images/} }
\newtheorem*{theorem}{Theorem}
\frenchspacing
\setcounter{secnumdepth}{3}
\parskip=1mm
 
\title{\bf %\Emerald
On lower bounds of the order \\of $k$-chromatic unit distance graphs %in the plane
}

\author{\normalsize {Aubrey D.N.J. de Grey; \textcolor[rgb]{0.2,0.6,0.4}{Jaan Parts (jaan\_parts@mail.ru)} }}
\date{\normalsize {Mountain View, California, USA; \textcolor[rgb]{0.2,0.6,0.4}{Kazan, Tatarstan, Russia}}}


\begin{document}

\maketitle

\pagestyle{empty}
\thispagestyle{empty}

\begin{abstract}
Here we give refined numerical values for the minimum number of vertices of $k$-chromatic unit distance graphs in the Euclidean plane.
\end{abstract}


\section{Background}

In the previous issue of \textit{Geombinatorics}, an article \cite{gwyn} was published, in which Haydn Gwyn and Jacob Stavrianos obtained new estimates for the \textit{minimum} number of vertices $v_k$ and edges $e_k$ for arbitrary $k$-chromatic\footnote{
For clarity, we emphasize that here $k$ means the number of colors in the proper vertex coloring of the \textit{graph} and differs by one from similar designations in \cite{gwyn} and other publications mentioned below, where it is associated with the \textit{tiling} of the plane.}
\textit{unit distance graphs} in the plane for $k=5$ and $k=6$. 

It is known that $v_3=3$ and $v_4=7$ (provided by the unit triangle and the Moser graph). For $k\ge 5$, exact values of $v_k$ are not known. Initial \textit{lower bounds} $v_5>12$, $v_7>6197$ were obtained by Dan Pritikin \cite{pri}. For $k=5$, the finite \textit{upper bounds} are\footnote{In \cite{par1} the corresponding 509-vertex graph has 2442 edges, but is not edge-critical, which allows us to reduce $e_5$. We were able to discard 36 edges, but we didn't perform an exhaustive search, so further improvements are possible.}
$v_5\le 509$, $e_5\le 2406$, see \cite{par1}. 

Pritikin's approach is based on finding a proper tiling of the plane using $k-1$ colors, which covers most of the plane, so that the average \textit{density} $\delta$ of the \textit{uncolored} area is minimal. Thus, $k-1$ colors are enough for any unit distance graph with at most $\lfloor 1/\delta \rfloor$ vertices, and a $k$-chromatic graph must have at least one more vertex.

In \cite{par2} we slightly improved the lower bounds to $v_6>24$ and $v_7>6992$ by modifying the tiling construction.

The main idea of Gwyn-Stavrianos' approach is to first estimate the minimum number of edges $e_k$, and then proceed to the number of vertices $v_k$ using known relations. To do this, a tiling of the plane in $k-1$ colors is found, minimizing the \textit{probability} $p_{k-1}$ of the formation of a \textit{mono-chromatic edge} (such that both vertices have the same color). The value $1/p_{k-1}$ gives a lower bound on the number of edges $e_k$.

For $k=5$, \cite{gwyn} uses a tiling by regular hexagons of four colors with a diameter of about 1.1335. For $k=6$, it is overlaid with a set of disks of unit diameter, located at a unit distance from each other with an average density $\delta=\pi/(8\sqrt3)$ and colored in the fifth color, and a lemma is used that allows passing from $p_4$ to $p_5$:\; $p_k\le(1-2\delta) \;p_{k-1}$.


According to the calculations of Gwyn and Stavrianos, $e_5\ge 98$, $v_5\ge 22$, $e_6\ge 180$, $v_6\ge 32$.
In this note, we offer several refinements that allow us to slightly increase these numerical values.


\section{Improvements}


First, note that in \cite{gwyn}, to go from the number of edges $e_k$ to the number of vertices $v_k$, the relation $e_k<v_k^{3/2}$ was used, which was obtained by Paul Erdős in \cite{erd}. Péter Ágoston and Dömötör Pálvölgyi recently obtained tighter bounds \cite{ago}:
$e_k\le \sqrt[3]{29/4}\;v_k^{4/3}$. In the range $20\le v_k< 521$, better estimates can be derived from the formula\footnote{
For completeness, note that according to \cite{ago} for $1\le v_k\le15$ the minimum number of edges $e_k$ is known exactly: 0, 1, 3, 5, 7, 9, 12, 14, 18, 20, 23, 27, 30, 33, and 37; while for $16\le v_k\le19$ the upper bounds on $e_k$ are 42, 47, 52, and 57. 
For the next $20\le v_k< 60$, using the formula, we get the following upper bounds on $e_k$: 63, 68, 72, 77, 82, 87, 92, 97, 102, 108, 113, 119, 124, 130, 136, 142, 148, 154, 160, 166, 172, 179, 185, 192, 198, 205, 212, 218, 225, 232, 239,
246, 254, 261, 268, 276, 283, 291, 298, 306.}
\cite{ago}: $v_k \ge 2\,r-11+24/v_k\,+$ $(1-s)\binom{\lfloor r \rfloor}{2} + s \binom{\lceil r \rceil}{2}$, where $r=2e_k/v_k$, $s=r-\lfloor r \rfloor$.
Applying these relations to $e_k$, we get a noticeable improvement in the $v_k$ estimates.

In \cite{gwyn} in the case of $k=6$, the plane was partially covered by disks with density $\delta=\pi/(8\sqrt3)\approx 0.226725$. Instead, one can use %the best known 
Croft's tiling \cite{cro} with rounded 12-gons and a density of about $\delta\approx 0.229365$, which slightly improves the $e_6$ estimate.
Also note that in \cite{gwyn} rounding down was used to get the $e_k$ estimates. It is more correct to use rounding up $e_k\ge\lceil 1/p_{k-1} \rceil$, which gives an increase by one more.

In \cite{gwyn}, a statistical method was used to find the value of $p_k$: a repeating section $S$ of the tiling was selected, on which a random unit edge was repeatedly superimposed, defined by the coordinates $\sigma=(x, y)$ of its first vertex and by the orientation angle $\phi$ of its second one, and the average value of the binary function $M_k(\sigma,\phi)$ was determined, which takes the value 1 or 0 depending on whether an edge is mono-chromatic or not. 
Instead, we took an algebraic method, calculating the integral 
$p_k=\frac{1}{2\pi S}\int_S d\sigma\int_{0}^{2\pi}d\phi \,M_k(\sigma,\phi)$
in the \texttt{Mathematica} package, which allowed us to refine the optimal tiling parameters. However, we encountered some problems with the numerical integration function \texttt{NIntegrate} in the general case, and were forced to limit ourselves to relatively simple tilings.

As a result, we get the bounds: $e_5\ge 99$, $v_5\ge 28$, $e_6\ge 182$, $v_6\ge 42$.

We also tried Gwyn-Stavrianos' approach for the case $k=7$, and obtained the following estimates for a tiling close to Pritikin's construction using pentagonal tiles: $e_7\ge 232646$, $v_7\ge 6456$. (For the construction using hexagonal tiles, we were able to get only $e_7\ge 69451$ and $v_7\ge 2608$.) This is better than \cite{pri}, but falls short of \cite{par2}. However, there is still some hope of beating the latter one by using tiles with curved borders (the so-called "wavy edges" used in \cite{par2}).


%%% -*-BibTeX-*-
%%% Do NOT edit. File created by BibTeX with style
%%% ACM-Reference-Format-Journals [18-Jan-2012].

\begin{thebibliography}{52}

%%% ====================================================================
%%% NOTE TO THE USER: you can override these defaults by providing
%%% customized versions of any of these macros before the \bibliography
%%% command.  Each of them MUST provide its own final punctuation,
%%% except for \shownote{}, \showDOI{}, and \showURL{}.  The latter two
%%% do not use final punctuation, in order to avoid confusing it with
%%% the Web address.
%%%
%%% To suppress output of a particular field, define its macro to expand
%%% to an empty string, or better, \unskip, like this:
%%%
%%% \newcommand{\showDOI}[1]{\unskip}   % LaTeX syntax
%%%
%%% \def \showDOI #1{\unskip}           % plain TeX syntax
%%%
%%% ====================================================================

\ifx \showCODEN    \undefined \def \showCODEN     #1{\unskip}     \fi
\ifx \showDOI      \undefined \def \showDOI       #1{#1}\fi
\ifx \showISBNx    \undefined \def \showISBNx     #1{\unskip}     \fi
\ifx \showISBNxiii \undefined \def \showISBNxiii  #1{\unskip}     \fi
\ifx \showISSN     \undefined \def \showISSN      #1{\unskip}     \fi
\ifx \showLCCN     \undefined \def \showLCCN      #1{\unskip}     \fi
\ifx \shownote     \undefined \def \shownote      #1{#1}          \fi
\ifx \showarticletitle \undefined \def \showarticletitle #1{#1}   \fi
\ifx \showURL      \undefined \def \showURL       {\relax}        \fi
% The following commands are used for tagged output and should be
% invisible to TeX
\providecommand\bibfield[2]{#2}
\providecommand\bibinfo[2]{#2}
\providecommand\natexlab[1]{#1}
\providecommand\showeprint[2][]{arXiv:#2}

\bibitem[\protect\citeauthoryear{Albrecht and Stone}{Albrecht and
  Stone}{2017}]%
        {Albrecht2017ReasoningAH}
\bibfield{author}{\bibinfo{person}{Stefano~V. Albrecht} {and}
  \bibinfo{person}{P. Stone}.} \bibinfo{year}{2017}\natexlab{}.
\newblock \showarticletitle{Reasoning about Hypothetical Agent Behaviours and
  their Parameters}. In \bibinfo{booktitle}{\emph{AAMAS}}.
\newblock


\bibitem[\protect\citeauthoryear{Andrejczuk, Berger, Rodriguez-Aguilar, Sierra,
  and Mar{\'\i}n-Puchades}{Andrejczuk et~al\mbox{.}}{2018}]%
        {andrejczuk2018composition}
\bibfield{author}{\bibinfo{person}{Ewa Andrejczuk}, \bibinfo{person}{Rita
  Berger}, \bibinfo{person}{Juan~A Rodriguez-Aguilar}, \bibinfo{person}{Carles
  Sierra}, {and} \bibinfo{person}{V{\'\i}ctor Mar{\'\i}n-Puchades}.}
  \bibinfo{year}{2018}\natexlab{}.
\newblock \showarticletitle{The composition and formation of effective teams:
  computer science meets organizational psychology}.
\newblock \bibinfo{journal}{\emph{The Knowledge Engineering Review}}
  \bibinfo{volume}{33} (\bibinfo{year}{2018}), \bibinfo{pages}{e17}.
\newblock


\bibitem[\protect\citeauthoryear{Arjona-Medina, Gillhofer, Widrich,
  Unterthiner, Brandstetter, and Hochreiter}{Arjona-Medina
  et~al\mbox{.}}{2019}]%
        {arjona2019rudder}
\bibfield{author}{\bibinfo{person}{Jose~A Arjona-Medina},
  \bibinfo{person}{Michael Gillhofer}, \bibinfo{person}{Michael Widrich},
  \bibinfo{person}{Thomas Unterthiner}, \bibinfo{person}{Johannes
  Brandstetter}, {and} \bibinfo{person}{Sepp Hochreiter}.}
  \bibinfo{year}{2019}\natexlab{}.
\newblock \showarticletitle{Rudder: Return decomposition for delayed rewards}.
\newblock \bibinfo{journal}{\emph{NeurIPS}}  \bibinfo{volume}{32}
  (\bibinfo{year}{2019}).
\newblock


\bibitem[\protect\citeauthoryear{Beal, Changder, Norman, and Ramchurn}{Beal
  et~al\mbox{.}}{2020}]%
        {beal2020learning}
\bibfield{author}{\bibinfo{person}{Ryan Beal}, \bibinfo{person}{Narayan
  Changder}, \bibinfo{person}{Timothy Norman}, {and} \bibinfo{person}{Sarvapali
  Ramchurn}.} \bibinfo{year}{2020}\natexlab{}.
\newblock \showarticletitle{Learning the value of teamwork to form efficient
  teams}. In \bibinfo{booktitle}{\emph{Proceedings of the AAAI Conference on
  Artificial Intelligence}}, Vol.~\bibinfo{volume}{34}.
  \bibinfo{pages}{7063--7070}.
\newblock


\bibitem[\protect\citeauthoryear{Beetz, Hoyningen-Huene, Bandouch,
  Kirchlechner, Gedikli, and Maldonado}{Beetz et~al\mbox{.}}{2006}]%
        {beetz2006camera}
\bibfield{author}{\bibinfo{person}{Michael Beetz}, \bibinfo{person}{Nico~v
  Hoyningen-Huene}, \bibinfo{person}{Jan Bandouch}, \bibinfo{person}{Bernhard
  Kirchlechner}, \bibinfo{person}{Suat Gedikli}, {and} \bibinfo{person}{Alexis
  Maldonado}.} \bibinfo{year}{2006}\natexlab{}.
\newblock \showarticletitle{Camera-based observation of football games for
  analyzing multi-agent activities}. In \bibinfo{booktitle}{\emph{Proceedings
  of the fifth international joint conference on Autonomous agents and
  multiagent systems}}. \bibinfo{pages}{42--49}.
\newblock


\bibitem[\protect\citeauthoryear{Bialkowski, Lucey, Carr, Yue, Sridharan, and
  Matthews}{Bialkowski et~al\mbox{.}}{2014}]%
        {bialkowski2014large}
\bibfield{author}{\bibinfo{person}{Alina Bialkowski}, \bibinfo{person}{Patrick
  Lucey}, \bibinfo{person}{Peter Carr}, \bibinfo{person}{Yisong Yue},
  \bibinfo{person}{Sridha Sridharan}, {and} \bibinfo{person}{Iain Matthews}.}
  \bibinfo{year}{2014}\natexlab{}.
\newblock \showarticletitle{Large-scale analysis of soccer matches using
  spatiotemporal tracking data}. In \bibinfo{booktitle}{\emph{2014 IEEE
  international conference on data mining}}. IEEE, \bibinfo{pages}{725--730}.
\newblock


\bibitem[\protect\citeauthoryear{Bouveret and Lang}{Bouveret and Lang}{2014}]%
        {bouveret2014manipulating}
\bibfield{author}{\bibinfo{person}{Sylvain Bouveret} {and}
  \bibinfo{person}{J{\'e}r{\^o}me Lang}.} \bibinfo{year}{2014}\natexlab{}.
\newblock \showarticletitle{Manipulating picking sequences.}. In
  \bibinfo{booktitle}{\emph{ECAI}}, Vol.~\bibinfo{volume}{14}.
  \bibinfo{pages}{141--146}.
\newblock


\bibitem[\protect\citeauthoryear{Brams and Straffin~Jr}{Brams and
  Straffin~Jr}{1979}]%
        {brams1979prisoners}
\bibfield{author}{\bibinfo{person}{Steven~J Brams} {and}
  \bibinfo{person}{Philip~D Straffin~Jr}.} \bibinfo{year}{1979}\natexlab{}.
\newblock \showarticletitle{Prisoners' dilemma and professional sports drafts}.
\newblock \bibinfo{journal}{\emph{The American Mathematical Monthly}}
  \bibinfo{volume}{86}, \bibinfo{number}{2} (\bibinfo{year}{1979}),
  \bibinfo{pages}{80--88}.
\newblock


\bibitem[\protect\citeauthoryear{Bransen and Van~Haaren}{Bransen and
  Van~Haaren}{2020}]%
        {bransen2020player}
\bibfield{author}{\bibinfo{person}{Lotte Bransen} {and} \bibinfo{person}{Jan
  Van~Haaren}.} \bibinfo{year}{2020}\natexlab{}.
\newblock \showarticletitle{Player chemistry: Striving for a perfectly balanced
  soccer team}.
\newblock \bibinfo{journal}{\emph{Sports Analytics Conference}}
  (\bibinfo{year}{2020}).
\newblock


\bibitem[\protect\citeauthoryear{Dafoe, Bachrach, Hadfield, Horvitz, Larson,
  and Graepel}{Dafoe et~al\mbox{.}}{2021}]%
        {DafoeNature2021}
\bibfield{author}{\bibinfo{person}{Allan Dafoe}, \bibinfo{person}{Yoram
  Bachrach}, \bibinfo{person}{Gillian Hadfield}, \bibinfo{person}{Eric
  Horvitz}, \bibinfo{person}{Kate Larson}, {and} \bibinfo{person}{Thore
  Graepel}.} \bibinfo{year}{2021}\natexlab{}.
\newblock \showarticletitle{Cooperative {AI}: machines must learn to find
  common ground}.
\newblock \bibinfo{journal}{\emph{Nature}}  \bibinfo{volume}{593}
  (\bibinfo{year}{2021}), \bibinfo{pages}{33--36}.
\newblock


\bibitem[\protect\citeauthoryear{Derks and Peters}{Derks and Peters}{1993}]%
        {derks1993shapley}
\bibfield{author}{\bibinfo{person}{Jean Derks} {and} \bibinfo{person}{Hans
  Peters}.} \bibinfo{year}{1993}\natexlab{}.
\newblock \showarticletitle{A Shapley value for games with restricted
  coalitions}.
\newblock \bibinfo{journal}{\emph{International Journal of Game Theory}}
  \bibinfo{volume}{21}, \bibinfo{number}{4} (\bibinfo{year}{1993}),
  \bibinfo{pages}{351--360}.
\newblock


\bibitem[\protect\citeauthoryear{Durugkar, Liebman, and Stone}{Durugkar
  et~al\mbox{.}}{2020}]%
        {Durugkar2020BalancingIP}
\bibfield{author}{\bibinfo{person}{Ishan Durugkar}, \bibinfo{person}{E.
  Liebman}, {and} \bibinfo{person}{P. Stone}.} \bibinfo{year}{2020}\natexlab{}.
\newblock \showarticletitle{Balancing Individual Preferences and Shared
  Objectives in Multiagent Reinforcement Learning}. In
  \bibinfo{booktitle}{\emph{IJCAI}}.
\newblock


\bibitem[\protect\citeauthoryear{Elitzur}{Elitzur}{2020}]%
        {elitzur2020data}
\bibfield{author}{\bibinfo{person}{Ramy Elitzur}.}
  \bibinfo{year}{2020}\natexlab{}.
\newblock \showarticletitle{Data analytics effects in major league baseball}.
\newblock \bibinfo{journal}{\emph{Omega}}  \bibinfo{volume}{90}
  (\bibinfo{year}{2020}), \bibinfo{pages}{102001}.
\newblock


\bibitem[\protect\citeauthoryear{Ellis}{Ellis}{1983}]%
        {ellis1983similarities}
\bibfield{author}{\bibinfo{person}{M Ellis}.} \bibinfo{year}{1983}\natexlab{}.
\newblock \showarticletitle{Similarities and differences in games: A system for
  classification}. In \bibinfo{booktitle}{\emph{International association for
  physical education in higher education Conference}}.
\newblock


\bibitem[\protect\citeauthoryear{Fern{\'a}ndez, Bornn, and
  Cervone}{Fern{\'a}ndez et~al\mbox{.}}{2021}]%
        {fernandez2021framework}
\bibfield{author}{\bibinfo{person}{Javier Fern{\'a}ndez}, \bibinfo{person}{Luke
  Bornn}, {and} \bibinfo{person}{Daniel Cervone}.}
  \bibinfo{year}{2021}\natexlab{}.
\newblock \showarticletitle{A framework for the fine-grained evaluation of the
  instantaneous expected value of soccer possessions}.
\newblock \bibinfo{journal}{\emph{Machine Learning}} \bibinfo{volume}{110},
  \bibinfo{number}{6} (\bibinfo{year}{2021}), \bibinfo{pages}{1389--1427}.
\newblock


\bibitem[\protect\citeauthoryear{Fisac, Bronstein, Stefansson, Sadigh, Sastry,
  and Dragan}{Fisac et~al\mbox{.}}{2019}]%
        {fisac2019hierarchical}
\bibfield{author}{\bibinfo{person}{Jaime~F Fisac}, \bibinfo{person}{Eli
  Bronstein}, \bibinfo{person}{Elis Stefansson}, \bibinfo{person}{Dorsa
  Sadigh}, \bibinfo{person}{S~Shankar Sastry}, {and} \bibinfo{person}{Anca~D
  Dragan}.} \bibinfo{year}{2019}\natexlab{}.
\newblock \showarticletitle{Hierarchical game-theoretic planning for autonomous
  vehicles}. In \bibinfo{booktitle}{\emph{ICRA}}. IEEE,
  \bibinfo{pages}{9590--9596}.
\newblock


\bibitem[\protect\citeauthoryear{Garner, Humphrey, and Simkins}{Garner
  et~al\mbox{.}}{2016}]%
        {garner2016business}
\bibfield{author}{\bibinfo{person}{Jacqueline Garner},
  \bibinfo{person}{Phillip~R Humphrey}, {and} \bibinfo{person}{Betty Simkins}.}
  \bibinfo{year}{2016}\natexlab{}.
\newblock \showarticletitle{The business of sport and the sport of business: A
  review of the compensation literature in finance and sports}.
\newblock \bibinfo{journal}{\emph{International Review of Financial Analysis}}
  \bibinfo{volume}{47} (\bibinfo{year}{2016}), \bibinfo{pages}{197--204}.
\newblock


\bibitem[\protect\citeauthoryear{Goes, Kempe, Meerhoff, and Lemmink}{Goes
  et~al\mbox{.}}{2019}]%
        {goes2019not}
\bibfield{author}{\bibinfo{person}{Floris~R Goes}, \bibinfo{person}{Matthias
  Kempe}, \bibinfo{person}{Laurentius~A Meerhoff}, {and}
  \bibinfo{person}{Koen~APM Lemmink}.} \bibinfo{year}{2019}\natexlab{}.
\newblock \showarticletitle{Not every pass can be an assist: a data-driven
  model to measure pass effectiveness in professional soccer matches}.
\newblock \bibinfo{journal}{\emph{Big data}} \bibinfo{volume}{7},
  \bibinfo{number}{1} (\bibinfo{year}{2019}), \bibinfo{pages}{57--70}.
\newblock


\bibitem[\protect\citeauthoryear{Hu, Xie, Liang, and Chang}{Hu
  et~al\mbox{.}}{2022}]%
        {hu2022policy}
\bibfield{author}{\bibinfo{person}{Siyi Hu}, \bibinfo{person}{Chuanlong Xie},
  \bibinfo{person}{Xiaodan Liang}, {and} \bibinfo{person}{Xiaojun Chang}.}
  \bibinfo{year}{2022}\natexlab{}.
\newblock \showarticletitle{Policy diagnosis via measuring role diversity in
  cooperative multi-agent {RL}}. In \bibinfo{booktitle}{\emph{ICML}}.
  \bibinfo{pages}{9041--9071}.
\newblock


\bibitem[\protect\citeauthoryear{Le, Yue, Carr, and Lucey}{Le
  et~al\mbox{.}}{2017}]%
        {le2017coordinated}
\bibfield{author}{\bibinfo{person}{Hoang~M Le}, \bibinfo{person}{Yisong Yue},
  \bibinfo{person}{Peter Carr}, {and} \bibinfo{person}{Patrick Lucey}.}
  \bibinfo{year}{2017}\natexlab{}.
\newblock \showarticletitle{Coordinated multi-agent imitation learning}. In
  \bibinfo{booktitle}{\emph{International Conference on Machine Learning}}.
  PMLR, \bibinfo{pages}{1995--2003}.
\newblock


\bibitem[\protect\citeauthoryear{Ledezma, Aler, Sanchis, and Borrajo}{Ledezma
  et~al\mbox{.}}{2009}]%
        {ledezma2009ombo}
\bibfield{author}{\bibinfo{person}{Agapito Ledezma}, \bibinfo{person}{Ricardo
  Aler}, \bibinfo{person}{Araceli Sanchis}, {and} \bibinfo{person}{Daniel
  Borrajo}.} \bibinfo{year}{2009}\natexlab{}.
\newblock \showarticletitle{OMBO: An opponent modeling approach}.
\newblock \bibinfo{journal}{\emph{{AI} Communications}} \bibinfo{volume}{22},
  \bibinfo{number}{1} (\bibinfo{year}{2009}), \bibinfo{pages}{21--35}.
\newblock


\bibitem[\protect\citeauthoryear{Lewis}{Lewis}{2004}]%
        {lewis2004moneyball}
\bibfield{author}{\bibinfo{person}{Michael Lewis}.}
  \bibinfo{year}{2004}\natexlab{}.
\newblock \bibinfo{booktitle}{\emph{Moneyball: The art of winning an unfair
  game}}.
\newblock \bibinfo{publisher}{WW Norton \& Company}.
\newblock


\bibitem[\protect\citeauthoryear{Liemhetcharat and Luo}{Liemhetcharat and
  Luo}{2015}]%
        {liemhetcharat2015applying}
\bibfield{author}{\bibinfo{person}{Somchaya Liemhetcharat} {and}
  \bibinfo{person}{Yicheng Luo}.} \bibinfo{year}{2015}\natexlab{}.
\newblock \showarticletitle{Applying the Synergy Graph Model to Human
  Basketball.}. In \bibinfo{booktitle}{\emph{AAMAS}}.
  \bibinfo{pages}{1695--1696}.
\newblock


\bibitem[\protect\citeauthoryear{Liu, Schulte, Poupart, Rudd, and Javan}{Liu
  et~al\mbox{.}}{2020}]%
        {liu2020learning}
\bibfield{author}{\bibinfo{person}{Guiliang Liu}, \bibinfo{person}{Oliver
  Schulte}, \bibinfo{person}{Pascal Poupart}, \bibinfo{person}{Mike Rudd},
  {and} \bibinfo{person}{Mehrsan Javan}.} \bibinfo{year}{2020}\natexlab{}.
\newblock \showarticletitle{Learning agent representations for ice hockey}.
\newblock \bibinfo{journal}{\emph{Advances in Neural Information Processing
  Systems}}  \bibinfo{volume}{33} (\bibinfo{year}{2020}),
  \bibinfo{pages}{18704--18715}.
\newblock


\bibitem[\protect\citeauthoryear{Ljung, Carlsson, and Lambrix}{Ljung
  et~al\mbox{.}}{2018}]%
        {Ljung2018PlayerPV}
\bibfield{author}{\bibinfo{person}{Dennis Ljung}, \bibinfo{person}{Niklas
  Carlsson}, {and} \bibinfo{person}{P. Lambrix}.}
  \bibinfo{year}{2018}\natexlab{}.
\newblock \showarticletitle{Player Pairs Valuation in Ice Hockey}. In
  \bibinfo{booktitle}{\emph{MLSA@PKDD/ECML}}.
\newblock


\bibitem[\protect\citeauthoryear{Lucey, Bialkowski, Carr, Foote, and
  Matthews}{Lucey et~al\mbox{.}}{2012}]%
        {lucey2012characterizing}
\bibfield{author}{\bibinfo{person}{Patrick Lucey}, \bibinfo{person}{Alina
  Bialkowski}, \bibinfo{person}{Peter Carr}, \bibinfo{person}{Eric Foote},
  {and} \bibinfo{person}{Iain Matthews}.} \bibinfo{year}{2012}\natexlab{}.
\newblock \showarticletitle{Characterizing multi-agent team behavior from
  partial team tracings: Evidence from the english premier league}. In
  \bibinfo{booktitle}{\emph{Proceedings of the AAAI Conference on Artificial
  Intelligence}}, Vol.~\bibinfo{volume}{26}. \bibinfo{pages}{1387--1393}.
\newblock


\bibitem[\protect\citeauthoryear{Pourmehr and Dadkhah}{Pourmehr and
  Dadkhah}{2011}]%
        {pourmehr2011overview}
\bibfield{author}{\bibinfo{person}{Shokoofeh Pourmehr} {and}
  \bibinfo{person}{Chitra Dadkhah}.} \bibinfo{year}{2011}\natexlab{}.
\newblock \showarticletitle{An overview on opponent modeling in RoboCup soccer
  simulation 2D}.
\newblock \bibinfo{journal}{\emph{Robot Soccer World Cup}}
  (\bibinfo{year}{2011}), \bibinfo{pages}{402--414}.
\newblock


\bibitem[\protect\citeauthoryear{Raabe, Nabben, and Memmert}{Raabe
  et~al\mbox{.}}{2022}]%
        {raabe2022graph}
\bibfield{author}{\bibinfo{person}{Dominik Raabe}, \bibinfo{person}{Reinhard
  Nabben}, {and} \bibinfo{person}{Daniel Memmert}.}
  \bibinfo{year}{2022}\natexlab{}.
\newblock \showarticletitle{Graph representations for the analysis of
  multi-agent spatiotemporal sports data}.
\newblock \bibinfo{journal}{\emph{Applied Intelligence}}
  (\bibinfo{year}{2022}), \bibinfo{pages}{1--21}.
\newblock


\bibitem[\protect\citeauthoryear{Radke, Brecht, and Radke}{Radke
  et~al\mbox{.}}{2022a}]%
        {radke2022identifying}
\bibfield{author}{\bibinfo{person}{David Radke}, \bibinfo{person}{Tim Brecht},
  {and} \bibinfo{person}{Daniel Radke}.} \bibinfo{year}{2022}\natexlab{a}.
\newblock \showarticletitle{Identifying Completed Pass Types and Improving
  Passing Lane Models}. In \bibinfo{booktitle}{\emph{Link{\"o}ping Hockey
  Analytics Conference}}. \bibinfo{pages}{71--86}.
\newblock


\bibitem[\protect\citeauthoryear{Radke, Larson, and Brecht}{Radke
  et~al\mbox{.}}{2022b}]%
        {Radke2022Exploring}
\bibfield{author}{\bibinfo{person}{David Radke}, \bibinfo{person}{Kate Larson},
  {and} \bibinfo{person}{Tim Brecht}.} \bibinfo{year}{2022}\natexlab{b}.
\newblock \showarticletitle{Exploring the Benefits of Teams in Multiagent
  Learning}. In \bibinfo{booktitle}{\emph{IJCAI}}.
\newblock


\bibitem[\protect\citeauthoryear{Radke, Larson, and Brecht}{Radke
  et~al\mbox{.}}{2022c}]%
        {radke2022importance}
\bibfield{author}{\bibinfo{person}{David Radke}, \bibinfo{person}{Kate Larson},
  {and} \bibinfo{person}{Tim Brecht}.} \bibinfo{year}{2022}\natexlab{c}.
\newblock \showarticletitle{The Importance of Credo in Multiagent Learning}.
\newblock \bibinfo{journal}{\emph{ALA Workshop at AAMAS}}
  (\bibinfo{year}{2022}).
\newblock


\bibitem[\protect\citeauthoryear{Radke, Radke, Brecht, and Pawelczyk}{Radke
  et~al\mbox{.}}{2021}]%
        {Radke2021Passing}
\bibfield{author}{\bibinfo{person}{D.~T. Radke}, \bibinfo{person}{D.~L. Radke},
  \bibinfo{person}{T. Brecht}, {and} \bibinfo{person}{A. Pawelczyk}.}
  \bibinfo{year}{2021}\natexlab{}.
\newblock \showarticletitle{Passing and Pressure Metrics in Ice Hockey}.
\newblock \bibinfo{journal}{\emph{Workshop of AI for Sports Analytics}}
  (\bibinfo{year}{2021}).
\newblock


\bibitem[\protect\citeauthoryear{Rahimian and Toka}{Rahimian and Toka}{2022}]%
        {rahimian2022optical}
\bibfield{author}{\bibinfo{person}{Pegah Rahimian} {and}
  \bibinfo{person}{Laszlo Toka}.} \bibinfo{year}{2022}\natexlab{}.
\newblock \showarticletitle{Optical tracking in team sports}.
\newblock \bibinfo{journal}{\emph{Journal of Quantitative Analysis in Sports}}
  \bibinfo{volume}{18}, \bibinfo{number}{1} (\bibinfo{year}{2022}),
  \bibinfo{pages}{35--57}.
\newblock


\bibitem[\protect\citeauthoryear{Rahwan, Michalak, Wooldridge, and
  Jennings}{Rahwan et~al\mbox{.}}{2015}]%
        {rahwan2015coalition}
\bibfield{author}{\bibinfo{person}{Talal Rahwan}, \bibinfo{person}{Tomasz~P
  Michalak}, \bibinfo{person}{Michael Wooldridge}, {and}
  \bibinfo{person}{Nicholas~R Jennings}.} \bibinfo{year}{2015}\natexlab{}.
\newblock \showarticletitle{Coalition structure generation: A survey}.
\newblock \bibinfo{journal}{\emph{Artificial Intelligence}}
  \bibinfo{volume}{229} (\bibinfo{year}{2015}), \bibinfo{pages}{139--174}.
\newblock


\bibitem[\protect\citeauthoryear{Rashid, Samvelyan, Schroeder, Farquhar,
  Foerster, and Whiteson}{Rashid et~al\mbox{.}}{2018}]%
        {rashid2018qmix}
\bibfield{author}{\bibinfo{person}{Tabish Rashid}, \bibinfo{person}{Mikayel
  Samvelyan}, \bibinfo{person}{Christian Schroeder}, \bibinfo{person}{Gregory
  Farquhar}, \bibinfo{person}{Jakob Foerster}, {and} \bibinfo{person}{Shimon
  Whiteson}.} \bibinfo{year}{2018}\natexlab{}.
\newblock \showarticletitle{Qmix: Monotonic value function factorisation for
  deep multi-agent reinforcement learning}. In
  \bibinfo{booktitle}{\emph{ICML}}. \bibinfo{pages}{4295--4304}.
\newblock


\bibitem[\protect\citeauthoryear{Rein and Memmert}{Rein and Memmert}{2016}]%
        {rein2016big}
\bibfield{author}{\bibinfo{person}{Robert Rein} {and} \bibinfo{person}{Daniel
  Memmert}.} \bibinfo{year}{2016}\natexlab{}.
\newblock \showarticletitle{Big data and tactical analysis in elite soccer:
  future challenges and opportunities for sports science}.
\newblock \bibinfo{journal}{\emph{SpringerPlus}} \bibinfo{volume}{5},
  \bibinfo{number}{1} (\bibinfo{year}{2016}), \bibinfo{pages}{1--13}.
\newblock


\bibitem[\protect\citeauthoryear{Ritchie, Harell, and Shreeves}{Ritchie
  et~al\mbox{.}}{2022}]%
        {ritchie2022pass}
\bibfield{author}{\bibinfo{person}{Robyn Ritchie}, \bibinfo{person}{Alon
  Harell}, {and} \bibinfo{person}{Phillip Shreeves}.}
  \bibinfo{year}{2022}\natexlab{}.
\newblock \showarticletitle{Pass Evaluation in Women's Olympic Ice Hockey}. In
  \bibinfo{booktitle}{\emph{Proceedings of the 5th International ACM Workshop
  on Multimedia Content Analysis in Sports}}. \bibinfo{pages}{65--73}.
\newblock


\bibitem[\protect\citeauthoryear{Sampaio, McGarry, Calleja-Gonz{\'a}lez,
  Jim{\'e}nez~S{\'a}iz, Schelling i~del Alc{\'a}zar, and Balciunas}{Sampaio
  et~al\mbox{.}}{2015}]%
        {sampaio2015exploring}
\bibfield{author}{\bibinfo{person}{Jaime Sampaio}, \bibinfo{person}{Tim
  McGarry}, \bibinfo{person}{Julio Calleja-Gonz{\'a}lez},
  \bibinfo{person}{Sergio Jim{\'e}nez~S{\'a}iz}, \bibinfo{person}{Xavi
  Schelling i~del Alc{\'a}zar}, {and} \bibinfo{person}{Mindaugas Balciunas}.}
  \bibinfo{year}{2015}\natexlab{}.
\newblock \showarticletitle{Exploring game performance in the National
  Basketball Association using player tracking data}.
\newblock \bibinfo{journal}{\emph{PloS one}} \bibinfo{volume}{10},
  \bibinfo{number}{7} (\bibinfo{year}{2015}), \bibinfo{pages}{e0132894}.
\newblock


\bibitem[\protect\citeauthoryear{Santos, Santos, Pacheco, and Levin}{Santos
  et~al\mbox{.}}{2021}]%
        {Santos2021SocialNI}
\bibfield{author}{\bibinfo{person}{F. Santos}, \bibinfo{person}{F.~C. Santos},
  \bibinfo{person}{J. Pacheco}, {and} \bibinfo{person}{S. Levin}.}
  \bibinfo{year}{2021}\natexlab{}.
\newblock \showarticletitle{Social Network Interventions to Prevent
  Reciprocity-driven Polarization}. In \bibinfo{booktitle}{\emph{AAMAS}}.
\newblock


\bibitem[\protect\citeauthoryear{Schr{\"o}der, Hoey, and Rogers}{Schr{\"o}der
  et~al\mbox{.}}{2016}]%
        {schroder2016modeling}
\bibfield{author}{\bibinfo{person}{Tobias Schr{\"o}der}, \bibinfo{person}{Jesse
  Hoey}, {and} \bibinfo{person}{Kimberly~B Rogers}.}
  \bibinfo{year}{2016}\natexlab{}.
\newblock \showarticletitle{Modeling dynamic identities and uncertainty in
  social interactions: Bayesian affect control theory}.
\newblock \bibinfo{journal}{\emph{American Sociological Review}}
  \bibinfo{volume}{81}, \bibinfo{number}{4} (\bibinfo{year}{2016}),
  \bibinfo{pages}{828--855}.
\newblock


\bibitem[\protect\citeauthoryear{Schuckers}{Schuckers}{2011}]%
        {schuckers2011s}
\bibfield{author}{\bibinfo{person}{Michael~E Schuckers}.}
  \bibinfo{year}{2011}\natexlab{}.
\newblock \showarticletitle{What's An NHL Draft Pick Worth? A Value Pick Chart
  for the National Hockey League}.
\newblock \bibinfo{journal}{\emph{Statistical Sports Consulting}}
  (\bibinfo{year}{2011}).
\newblock


\bibitem[\protect\citeauthoryear{Schulte, Khademi, Gholami, Zhao, Javan, and
  Desaulniers}{Schulte et~al\mbox{.}}{2017}]%
        {schulte2017markov}
\bibfield{author}{\bibinfo{person}{Oliver Schulte}, \bibinfo{person}{Mahmoud
  Khademi}, \bibinfo{person}{Sajjad Gholami}, \bibinfo{person}{Zeyu Zhao},
  \bibinfo{person}{Mehrsan Javan}, {and} \bibinfo{person}{Philippe
  Desaulniers}.} \bibinfo{year}{2017}\natexlab{}.
\newblock \showarticletitle{A Markov Game model for valuing actions, locations,
  and team performance in ice hockey}.
\newblock \bibinfo{journal}{\emph{Data Mining and Knowledge Discovery}}
  \bibinfo{volume}{31}, \bibinfo{number}{6} (\bibinfo{year}{2017}),
  \bibinfo{pages}{1735--1757}.
\newblock


\bibitem[\protect\citeauthoryear{Schwind, Demirovic, Inoue, and
  Lagniez}{Schwind et~al\mbox{.}}{2021}]%
        {schwind2021partial}
\bibfield{author}{\bibinfo{person}{Nicolas Schwind}, \bibinfo{person}{Emir
  Demirovic}, \bibinfo{person}{Katsumi Inoue}, {and}
  \bibinfo{person}{Jean-Marie Lagniez}.} \bibinfo{year}{2021}\natexlab{}.
\newblock \showarticletitle{Partial Robustness in Team Formation: Bridging the
  Gap between Robustness and Resilience.}. In
  \bibinfo{booktitle}{\emph{AAMAS}}, Vol.~\bibinfo{volume}{21}.
  \bibinfo{pages}{20th}.
\newblock


\bibitem[\protect\citeauthoryear{Simon}{Simon}{1990}]%
        {simon1990bounded}
\bibfield{author}{\bibinfo{person}{Herbert~A Simon}.}
  \bibinfo{year}{1990}\natexlab{}.
\newblock \showarticletitle{Bounded rationality}.
\newblock In \bibinfo{booktitle}{\emph{Utility and probability}}.
  \bibinfo{publisher}{Springer}, \bibinfo{pages}{15--18}.
\newblock


\bibitem[\protect\citeauthoryear{Spearman}{Spearman}{2018}]%
        {spearman2018beyond}
\bibfield{author}{\bibinfo{person}{William Spearman}.}
  \bibinfo{year}{2018}\natexlab{}.
\newblock \showarticletitle{Beyond expected goals}. In
  \bibinfo{booktitle}{\emph{Proceedings of the 12th MIT sloan sports analytics
  conference}}. \bibinfo{pages}{1--17}.
\newblock


\bibitem[\protect\citeauthoryear{Stone, Riley, and Veloso}{Stone
  et~al\mbox{.}}{2000}]%
        {stone2000defining}
\bibfield{author}{\bibinfo{person}{Peter Stone}, \bibinfo{person}{Patrick
  Riley}, {and} \bibinfo{person}{Manuela Veloso}.}
  \bibinfo{year}{2000}\natexlab{}.
\newblock \showarticletitle{Defining and using ideal teammate and opponent
  agent models}. In \bibinfo{booktitle}{\emph{AAAI/IAAI}}.
  \bibinfo{pages}{1040--1045}.
\newblock


\bibitem[\protect\citeauthoryear{Tuyls, Omidshafiei, Muller, Wang, Connor,
  Hennes, Graham, Spearman, Waskett, Steel, et~al\mbox{.}}{Tuyls
  et~al\mbox{.}}{2021}]%
        {tuyls2021game}
\bibfield{author}{\bibinfo{person}{Karl Tuyls}, \bibinfo{person}{Shayegan
  Omidshafiei}, \bibinfo{person}{Paul Muller}, \bibinfo{person}{Zhe Wang},
  \bibinfo{person}{Jerome Connor}, \bibinfo{person}{Daniel Hennes},
  \bibinfo{person}{Ian Graham}, \bibinfo{person}{William Spearman},
  \bibinfo{person}{Tim Waskett}, \bibinfo{person}{Dafydd Steel},
  {et~al\mbox{.}}} \bibinfo{year}{2021}\natexlab{}.
\newblock \showarticletitle{Game Plan: What {AI} can do for Football, and What
  Football can do for AI}.
\newblock \bibinfo{journal}{\emph{Journal of Artificial Intelligence Research}}
   \bibinfo{volume}{71} (\bibinfo{year}{2021}), \bibinfo{pages}{41--88}.
\newblock


\bibitem[\protect\citeauthoryear{Van Der~Hoek, Jamroga, and Wooldridge}{Van
  Der~Hoek et~al\mbox{.}}{2005}]%
        {van2005logic}
\bibfield{author}{\bibinfo{person}{Wiebe Van Der~Hoek},
  \bibinfo{person}{Wojciech Jamroga}, {and} \bibinfo{person}{Michael
  Wooldridge}.} \bibinfo{year}{2005}\natexlab{}.
\newblock \showarticletitle{A logic for strategic reasoning}. In
  \bibinfo{booktitle}{\emph{Proceedings of the fourth international joint
  conference on Autonomous agents and multiagent systems}}.
  \bibinfo{pages}{157--164}.
\newblock


\bibitem[\protect\citeauthoryear{Vats, Fani, Clausi, and Zelek}{Vats
  et~al\mbox{.}}{2022}]%
        {vats2022evaluating}
\bibfield{author}{\bibinfo{person}{Kanav Vats}, \bibinfo{person}{Mehrnaz Fani},
  \bibinfo{person}{David~A Clausi}, {and} \bibinfo{person}{John~S Zelek}.}
  \bibinfo{year}{2022}\natexlab{}.
\newblock \showarticletitle{Evaluating deep tracking models for player tracking
  in broadcast ice hockey video}.
\newblock \bibinfo{journal}{\emph{arXiv preprint arXiv:2205.10949}}
  (\bibinfo{year}{2022}).
\newblock


\bibitem[\protect\citeauthoryear{Visser, Dr{\"u}cker, H{\"u}bner, Schmidt, and
  Weland}{Visser et~al\mbox{.}}{2000}]%
        {visser2000recognizing}
\bibfield{author}{\bibinfo{person}{Ubbo Visser}, \bibinfo{person}{Christian
  Dr{\"u}cker}, \bibinfo{person}{Sebastian H{\"u}bner}, \bibinfo{person}{Esko
  Schmidt}, {and} \bibinfo{person}{Hans-Georg Weland}.}
  \bibinfo{year}{2000}\natexlab{}.
\newblock \showarticletitle{Recognizing formations in opponent teams}. In
  \bibinfo{booktitle}{\emph{Robot Soccer World Cup}}. Springer,
  \bibinfo{pages}{391--396}.
\newblock


\bibitem[\protect\citeauthoryear{Williamson and Cox}{Williamson and
  Cox}{2014}]%
        {williamson2014distributed}
\bibfield{author}{\bibinfo{person}{Kellie Williamson} {and}
  \bibinfo{person}{Rochelle Cox}.} \bibinfo{year}{2014}\natexlab{}.
\newblock \showarticletitle{Distributed cognition in sports teams: Explaining
  successful and expert performance}.
\newblock \bibinfo{journal}{\emph{Educational Philosophy and Theory}}
  \bibinfo{volume}{46}, \bibinfo{number}{6} (\bibinfo{year}{2014}),
  \bibinfo{pages}{640--654}.
\newblock


\bibitem[\protect\citeauthoryear{Yan, Kroer, and Peysakhovich}{Yan
  et~al\mbox{.}}{2020}]%
        {Yan2020EvaluatingAR}
\bibfield{author}{\bibinfo{person}{Tom Yan}, \bibinfo{person}{Christian Kroer},
  {and} \bibinfo{person}{A. Peysakhovich}.} \bibinfo{year}{2020}\natexlab{}.
\newblock \showarticletitle{Evaluating and Rewarding Teamwork Using Cooperative
  Game Abstractions}.
\newblock \bibinfo{journal}{\emph{NeurIPS}} (\bibinfo{year}{2020}).
\newblock


\end{thebibliography}


\end{document}
