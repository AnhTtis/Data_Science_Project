\begin{abstract}
With the global population aging rapidly, Alzheimer’s disease (AD) is particularly prominent in older adults, which has an insidious onset and leads to a gradual, irreversible deterioration in cognitive domains (memory, communication, etc.). Speech-based AD detection opens up the possibility of widespread screening and timely disease intervention.
% Recent advances in pre-trained models motivate AD detection modeling to shift from low-level features to high-level representations.
% This paper presents several efficient methods to extract better AD-related cues from high-level representations of acoustic, linguistic models and the pertinence between description utterances and task-related keywords.
% Experiments are carried out on the ADReSS dataset in a binary classification setup, and models are evaluated on the unseen test set. Results and comparison with recent literature show the effectiveness of the proposed models, which give superior accuracy scores of 88.79\%, 90.09\% and 84.38\% for acoustic, linguistic and task-oriented methods, respectively, as well as 91.41\% for the combination of these features. The comparable performances from audio-only or task-oriented also promote the feasibility of automation and generalization for AD detection.
Recent advances in pre-trained models motivate AD detection modeling to shift from low-level features to high-level representations.
This paper presents several efficient methods to extract better AD-related cues from high-level acoustic and linguistic features.
Based on these features, the paper also proposes a novel task-oriented approach by modeling the relationship between the participants' description and the cognitive task.
Experiments are carried out on the ADReSS dataset in a binary classification setup, and models are evaluated on the unseen test set. Results and comparison with recent literature demonstrate the efficiency and superior performance of proposed acoustic, linguistic and task-oriented methods.
The findings also show the importance of semantic and syntactic information, and feasibility of automation and generalization with the promising audio-only and task-oriented methods for the AD detection task.

\end{abstract}
\begin{keywords}
Alzheimer’s disease, task-oriented, pretrained embeddings, transfer learning, multimodality
\end{keywords}
