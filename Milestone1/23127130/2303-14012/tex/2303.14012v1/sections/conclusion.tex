\section{Conclusions}
\label{sec:conclusions}

Studying the interaction between deformable and rigid objects is an interesting yet complicated problem due to the complexity in the modeling and simulating such contact. However, with the rapid development of physics-based simulators that support soft bodies, the research gap between rigid and deformable objects is becoming smaller. To leverage the capability of such simulators in this domain, we presented a planning pipeline for manipulation tasks involving the interaction between deformable and rigid objects, which we explicitly demonstrate in the dish cleaning task with a deformable sponge. The key component of the pipeline is the contact map prediction model trained entirely on synthetic data, which learns the mapping from the contact location to the contact map between the two objects. The output of the model is then used to generate the optimal trajectory that achieves full area coverage of the target objects. We demonstrate the performance of the planned trajectories through experiments in both simulation and real-world scenarios on objects with varying sizes and geometrical features. The results show that the proposed planning pipeline is capable of generating high-quality trajectories that efficiently achieve nearly full area coverage of the target objects with a deformable tool. Despite the good results, there is still room for improvements. Some limitations of the proposed pipeline that can be tackled are the lack of real-time knowledge of the en route contact map while traversing between two contact points, or counter-intuitive trajectories compared to those of humans. Another interesting future work would be to close the manipulation loop with real-time contact feedback to counteract uncertainties in the real world such as incorrect contact map prediction or displacement of objects while being manipulated. 



