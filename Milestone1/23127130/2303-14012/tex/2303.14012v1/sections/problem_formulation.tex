\section{Problem Formulation}
\label{sec:prob_form}
% This work addresses the problem of generating antipodal grasps on unknown objects with different stiffnesses lying on a supporting surface. The goal is to calculate a grasp for each pixel in the depth image while taking into account object stiffness. More formally, we train a model $\mathcal{M}$ that takes as input a depth image $\mathbf{I_d}$ and a stiffness image $\mathbf{I_s}$, and produces a grasp map \textbf{G} that incorporates grasp quality and grasp parameters (orientation, gripper width) for grasps centered at each pixel in the input.
% \begin{equation*}
%     \mathcal{M}: (\mathbf{I_d},\mathbf{I_s}) \mapsto \textbf{G}.
% \end{equation*}

% To achieve this goal, we propose to use the \ac{dnn} in \figref{fig:pipeline} to map from depth and stiffness images to grasps \textbf{G} in the image, which we can easily transform to the real world using known coordinate transforms. 
In this work, we focus on manipulation tasks that involve the interaction between a volumetric deformable tool and rigid objects, such as the one shown in \figref{fig:idea}. Specifically, this work addresses the problem of area coverage planning under deformations. The goal is to generate an optimal trajectory to cover the entire surface of a rigid object with a deformable tool while taking into account the interaction of the two bodies. To achieve this, we first need to learn the correlation between the geometric features of rigid objects and the contact areas between a deformable tool and those rigid objects. 
% Let us consider a dish cleaning task using a deformable sponge; we aim to first learn a model that predicts the contact area between the sponge and the target objects at specific contact locations from the point-cloud observation of the target objects. 
More formally, we train a model $\mathcal{M}$ that takes as input a point cloud of the target object $\mathcal{P}_O$ and the contact point $p_c$ ($p_c \in \mathcal{P}_O$) , and produces a contact map $\mathcal{P}_C$ for $N$ object points $(p_i \in \mathcal{P}_O)_{i=1}^N$.
\begin{equation*}
    \mathcal{M}: (\mathcal{P}_O,p_c) \mapsto \mathcal{P}_C.
\end{equation*} 
The learned model $\mathcal{M}$ is then used as input for an area coverage trajectory planning algorithm to generate an optimal trajectory consisting of $K$ contact points in an optimized ordered $\mathcal{T}_{opt} = \{p_{c_i}\}_{i=1}^K$ that achieves full area coverage of the target objects with the deformable tool in the most efficient way.
\begin{figure}[!t]
    \centering
    \def\svgwidth{.6\linewidth}
     {\fontsize{10}{10}%\selectfont\sf
    \input{figures_tex/idea.pdf_tex}} 
    \caption{Different tactics of human to create contact with various complex surface profiles using a deformable sponge.}
    \label{fig:idea}
    \vspace{-0.5cm}
\end{figure}