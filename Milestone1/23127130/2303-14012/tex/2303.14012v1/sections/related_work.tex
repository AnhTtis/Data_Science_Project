\section{Related work}
\label{sec:related_work}
To put our work in context, we next review three complementary viewpoints, the interaction between deformable and rigid objects, planning for deformable objects, and the simulation of deformable objects.
\subsection{Interaction between deformable and rigid objects}
Recently, the robotic research community has made great strides in learning complex dynamics of deformable objects, which in turn enable robots to achieve impressive results in sophisticated manipulation tasks such as folding \cite{hietala2022closing,speedfolding}/ unfolding clothes \cite{lin2021learning, ha2021flingbot}, untangling rope \cite{Grannen2020UntanglingDK} and manipulating deformable objects in target configurations \cite{shen2022acid}. Nevertheless, instead of only looking at the behavior of deformable objects, various methods have been proposed to study the interaction between rigid and deformable objects and take advantage of this knowledge to facilitate different manipulation tasks in different sectors such as grasp generation \cite{Brahmbhatt_2020_ECCV,jiang2021graspTTA,defggcnn}, assistive dressing \cite{erickson17,erickson18,wang2022visual}, assistive surgery \cite{Haouchine18}, or food processing \cite{heiden2021disect,sundaresan2022learning,lin2022planning}. 

% some attempts to push the problem even further by investigating the interaction between deformable and rigid bodies. Various methods have been proposed to study the interaction between rigid and deformable objects and to take advantage of this knowledge to facilitate different manipulation tasks in different sectors such as grasp generation \cite{Brahmbhatt_2020_ECCV,jiang2021graspTTA,defggcnn}, assistive dressing \cite{erickson17,erickson18,wang2022visual}, assistive surgery \cite{}, or food processing \cite{heiden2021disect,sundaresan2022learning,lin2022planning}. 

% Among the aforementioned sectors, capturing human hands - rigid objects interactions has been an active field of study \cite{Brahmbhatt_2020_ECCV, jiang2021graspTTA}. These works take into account the deformation of human hands to improve the different grasp synthesis pipeline. Others studied the interaction between a rigid robot end effector and deformable objects to estimate the contact force for assistive surgical \cite{Haouchine18} and food processing tasks \cite{heiden2021disect,sundaresan2022learning,lin2022planning}. For example, Sundaresan \etal \cite{sundaresan2022learning} proposed a method to learn zero-shot skewering of food items for assisted feeding based on visual and haptic observations during interaction between a rigid fork and deformable food items.  
Previous works attempted to estimate the interaction between deformable and rigid objects using real-world data \cite{Haouchine18,sundaresan2022learning,jiang2021graspTTA,Brahmbhatt_2020_ECCV}, while recent works sought to study this interaction using synthetic data from simulation. For example, Erickson \etal \cite{erickson17} presented a data-driven method trained in physics-based simulation that can infer the force applied onto a person's body from end-effector measurements. The inferred force is used along with model predictive control (MPC) to improve assistive dressing \cite{erickson18}. This idea has recently been leveraged for better generalization across objects and assistive tasks in \cite{wang2022visual}. Specifically, instead of using end effector measurements, Wang \etal \cite{wang2022visual} proposed a visual haptic reasoning method that uses point-cloud observations to estimate the contact force distribution when manipulating cloth on rigid objects. The model was experimentally proven to be reasonably transferred from simulation to different assistive tasks in the real world. It is worth noting that our work is highly inspired by \cite{wang2022visual}, but instead of doing \textit{contact reasoning} for cloth, our work in this paper aims at tackling \textit{contact prediction} for volumetric deformable objects, where our goal is to attempt to predict the contact areas in advance. 
\subsection{Planning for Deformable Object Manipulation}
Planning manipulation tasks involving the interaction of deformable and rigid objects is, compared to that involves only deformable objects, a less explored research area. Only until recently have a few works proposed methods to plan tasks either for rigid tools and deformable objects such as food processing \cite{lin2022planning,Seita2022toolflownet}, or for cloth and rigid objects such as assistive dressing \cite{wang2022visual,erickson18}. Continuing in this line of research, instead of focusing on cloth, in this paper we aim to learn the interaction between volumetric deformable tools and rigid objects and exploit this knowledge in planning manipulation tasks such as cleaning tasks.

\subsection{Deformable Object Simulation}
Research in computer graphics has made significant progress in the development of simulations of deformable objects \cite{Andrew06}. This progress has, in turn, driven recent simulators to manipulate deformable objects. Over the past few years, different physics-based simulators have been developed that allow us to simulate the interaction between deformable and rigid objects for various types of deformable objects, ranging from cloth \cite{clothsim_19, softgym,Qiao2020Scalable}, liquid \cite{softgym,Schenck18}, 3D deformable objects \cite{heiden2021disect,isaacgym}. For example, Wang \etal \cite{wang2022visual} trained the aforementioned visual haptic reasoning model using the data generated by the SoftGym simulator, and the result showed that the model is also well transferred to the real world.

% Heiden \etal \cite{heiden2021disect} introduced a differentiable simulator named DiSECt just to serve the objective of simulating the cutting of soft materials. The author also conducted experiments on a real robot arm equipped with a slicing knife to demonstrate the similarity in simulation and the real world. On a more related note to our work, 

In this work, since our goal is to study the interaction between volumetric deformable objects and rigid objects, we use the existing Isaac Gym simulator to develop our desired simulation environment. The main reason behind this choice is that Isaac Gym has been shown to provide accurate simulation of volumetric deformable objects, which, in turn, powered different downstream robotic tasks such as grasp analysis \cite{defgraspsim,Kim2021IPCGraspSimRT,metrics2022}, grasp synthesis \cite{defggcnn} or learning policies for robotic assembly \cite{Narang2022FactoryFC}.     

