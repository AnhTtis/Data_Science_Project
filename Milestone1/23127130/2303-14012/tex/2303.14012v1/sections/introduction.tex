\section{Introduction}
\label{sec:introduction}
Volumetric or 3D deformable objects, present in various forms, are prevalent in many aspects of daily life, including food, toys, or internal organs of humans. Successful manipulation of such objects can lead to numerous practical applications in areas such as surgical manipulation, or food processing where robots can be used to make pizza dough \cite{lin2022planning}, cut fruits \cite{heiden2021disect}, or in healthcare where robots can be used to rearrange objects in target configurations \cite{shen2022acid}, or clean dishes \cite{wang2022visual}. These tasks are trivial for humans because not only we possess remarkable dexterity but we also excel at task planning, as demonstrated by our ability to perceive objects at hand, and develop a plan to complete the task with precision and accuracy in less than a second.

% Let us consider a traditional dish cleaning task of wiping dishes with a sponge, which is explicitly studied in this work. People often hold dishes in one hand, plan where to place the sponge and which trajectory to wipe in order to clean the whole surface of the dish, and then finally execute the planned trajectory. The whole process from perceiving the dishes to planning the wiping trajectory takes less than one second for humans, as subconsciously one can roughly predict the contact and the force distribution between the sponge and the dishes at a certain location. 

\begin{figure}[!t]
    \centering
    \def\svgwidth{\linewidth}
     {\fontsize{6}{6}%\selectfont\sf
    





%%% Use this command to specify your EasyChair submission number.
%%% In anonymous mode, it will be printed on the first page.

\acmSubmissionID{???}

%%% Use this command to specify the title of your paper.

% \title[Team Sport Analytics is a Multiagent Systems Problem]{Team Sport Analytics is a Multiagent Systems Problem}
\title[Presenting Multiagent Challenges in Team Sports Analytics]{Presenting Multiagent Challenges in Team Sports Analytics}
\subtitle{Blue Sky Ideas Track}
% \title[How Multiagent Systems and Sports Analytics can Help Each Other]{How Multiagent Systems and Sports Analytics can Help Each Other}

%%% Provide names, affiliations, and email addresses for all authors.

% \author{Paper: \#1506}

\author{David Radke}
\email{dradke@blackhawks.com}
\affiliation{%
  \institution{Chicago Blackhawks}
	\city{Chicago, IL}
	\country{USA \\ University of Waterloo}
}
% \email{dtradke@uwaterloo.ca}
% \affiliation{%
%   \institution{University of Waterloo}
% 	% \city{Waterloo, ON}
% 	% \country{Canada}
% }


\author{Alexi Orchard}
\email{alexi.orchard@uwaterloo.ca}
\affiliation{%
  \institution{University of Waterloo}
	\city{Waterloo, ON}
	\country{Canada}
}

% \author {
%     % Authors
%     David Radke\textsuperscript{\rm 1,2},
%     Alexi Orchard\textsuperscript{\rm 2}
% }
% \affiliation {
%     % Affiliations
%     \textsuperscript{\rm 1}Chicago Blackhawks,
%     \textsuperscript{\rm 2}University of Waterloo\\
%     dradke@blackhawks.com, 
%     alexi.orchard@uwaterloo.ca
% }} 
    \caption{SPONGE deployed in the real world to accomplish a dish cleaning task with a deformable sponge. Given a point cloud of the target objects, SPONGE powered by a contact map prediction model trained in simulation, plans an optimal trajectory aiming at achieving full area coverage of target objects with the least number of waypoints.} 
    \label{fig:title}
    \vspace{-0.5cm}
\end{figure}

Planning manipulation tasks involving interactions between deformable and rigid objects, such as wiping a curved surface with a deformable tool, is difficult due to the challenge in predicting such interactions. To date, most existing works disregard the interaction between the deformable tool and target objects, and focus only on the control aspect of the tasks \cite{control_survey}. Only recently have some researchers started to investigate how to estimate and harness such interactions in different tasks such as assistive dressing, or food processing. In the literature, researchers studied the interaction between complex deformable objects such as human hands \cite{Brahmbhatt_2020_ECCV, jiang2021graspTTA} or cloth \cite{wang2022visual,erickson17} and rigid bodies by looking at the concept of \textit{contact reasoning} where the location of contact and the magnitude of applied forces are estimated once the two bodies interact with each other. However, this \textit{contact reasoning} concept is more suitable for the control aspect than for the planning aspect due to its ability to track the interaction in real time. Thus, the question of how to predict the interaction between deformable and rigid objects and exploit such interactions for planning remains open.

To address the aforementioned open issues, we propose \textbf{S}equence \textbf{P}lanning with deformable-\textbf{ON}-rigid contact prediction from \textbf{GE}ometric features (SPONGE), a sequence planning pipeline powered by a contact prediction model that predicts contact between deformable and rigid bodies, with the aim of providing robots with the aforementioned human-like planning skill in order to efficiently automate downstream deformable object manipulation tasks such as cleaning dishes (Fig. \ref{fig:title}). Instead of \textit{contact reasoning}, in this paper we tackle the concept of \textit{contact prediction} of a 3D deformable tool acting on rigid objects, which is better suited for planning purposes. We take a data-driven approach with physics-based simulation to model the interactions between 3D deformable objects and rigid objects. We then use PointNet \cite{qi2016pointnet} architecture to form a mapping from point-cloud observation of the target object, and pose of the deformable tool to 3D representation of the contact points between the two bodies. The trained contact prediction model is then used as the driving force behind the planning of a subsequent task. Finally, we deploy SPONGE in the real world to demonstrate that the contact prediction model trained only with synthetic data from physics-based simulation can help to produce an efficient plan for a manipulation task to be successfully executed in the real world.

In summary, the main contributions of this paper are as follows:
\begin{itemize}
    \item A deep learning-based contact prediction model that predicts the contacts between 3D deformable and rigid objects under interactions.
    \item A planning pipeline powered by the proposed contact prediction model to efficiently automate deformable object manipulation tasks.    
    \item A novel experimental dataset containing 3D contact locations, 3D net force vectors, ands labeled contact areas when pressing deformable objects against rigid objects. The dataset was collected from 10 rigid objects interacting with a deformable sponge, with more than ten thousand data points.    
    \item A thorough empirical evaluation of the proposed method, both in simulation and on real hardware, demonstrating the ability of the planning pipeline to generate high-quality trajectories that efficiently cover the entire desired area to accomplish the manipulation task.
\end{itemize}

