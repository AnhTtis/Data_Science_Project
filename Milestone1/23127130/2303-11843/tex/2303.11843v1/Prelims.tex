\section{Preliminaries}\label{sec:prelims}

We begin with basic notation and definitions. For any positive integer $n$, we write $[n]$ to denote the set $ \{1,2,\dots,n\}$. In what follows, we will fix any metric space $(\cX,d)$. 
A fully dynamic stream is a sequence $(p_1,\sigma_1),\dots,(p_M,\sigma_M)$ of $M$ updates such that $p_i \in \cX$ is a point, and $\sigma_i \in \{+,-\}$ signifies either an insertion or deletion of a point. Naturally, we assume that a point can only be deleted if it was previously inserted. Moreover, we assume that each point is inserted at most once before being deleted.
\awr{rephrased this sentence, note that now we have several objectives. The old version is commented}

We call a point $p \in \cX$ active at time $t$ if $p$ was inserted at some time $t' < t$, and not deleted anytime between $t'$ and $t$. We write $P^{t} \subset \cX$ to denote the set of active points at time $t$. We let $r_{\min},r_{\max}$ be reals such that for all $t \in [M]$ and $x,y \in P^t$, we have $r_{\min} \leq d(x,y) \leq r_{\max}$, and set $\Delta = r_{\max} / r_{\min}$ to be the aspect ratio of the point set. As in prior works \cite{ChaFul18,schmidt2019fully}, we assume that an upper bound on $\Delta$ is known.%

\paragraph{Clustering Objectives.}
In the $k$-center problem, given a point set $P$ living in a metric space $(\cX,d)$, the goal is to output a set of $k$ \textit{centers} $\cC = \{c_1,\dots,c_k\} \subset \cX$, along with a mapping $\ell: P \to \{c_1,\dots,c_k\}$, 
such that the following objective is minimized:
\[ \cost[P]{\cC}[k][\infty] =   \max_{p \in P} d(p,\ell(p))    \]
In other words, one would like for the maximum distance from $p$ to the $\ell(p)$, over all $p \in P$, to be minimized. A related objective is $k$-sum-of-radii, where the maximum distance to a point is considered for each cluster and added instead of taking the maximum:
\[ \cost[P]{\cC}[k][R\Sigma] =   \sum_{c \in \mathcal{C}} \max_{p \in \ell^{-1}(c)} d(p,c)    \]
For the $k$-sum-of-diameters objective, one considers the pairwise distances of points assigned to the same clusters instead of the distances from each point to its center.
\[ \cost[P]{\cC}[k][D\Sigma] =   \sum_{c \in \mathcal{C}} \max_{p,q \in \ell^{-1}(c)} d(p,q)    \]
Additionally, for any real $z >0$, we introduce the $(k,z)$-clustering problem, which is to minimize 
\[ \cost[P]{\cC}[k][z] =   \sum_{p \in P} d^z(p,\ell(p))    \]

We will be primarily concerned the the $k$-center objective, but we introduce the more general $(k,z)$-clustering objective, which includes both $k$-median (for $z=1$) and $k$-means (for $z=2$), as our lower bounds from Section \ref{sec:LB} will hold for these objectives as well.

We remark that while $\ell(p)$ is usually fixed by definition to be the closest point to $p$ in $C$, the closest point may not necessarily be easy to maintain in a fully dynamic setting. Therefore, we will evaluate the cost of our algorithms with respects to both the centers and the mapping $\ell$ from points to centers. For any $p \in P$, we will refer to $\ell(p)$ as the \textit{leader} of $p$ under the mapping $\ell$, and the set of all points lead by a given $c_i$ is the cluster led by $c_i$.

In addition to maintaining a clustering with approximately optimal cost, we would like for our algorithms to be able to quickly answer queries related to cluster membership, and enumeration over all points in a cluster. Specifically, we ask that our algorithm be able to answer the following queries at any time step $t$: 

\begin{figure}[H]
\begin{Frame}[Queries to a Fully Dynamic Clustering Algorithm]
\begin{enumerate}
    \item \textbf{Membership Query:} Given a point $p \in P^{t}$, return the center $c = \ell(p)$ of the cluster $C$ containing~$p$.
    \item  \textbf{Cluster Enumeration:} Given a point $p \in P^{t}$, list all points in the cluster $C$ containing~$p$.
\end{enumerate}
\end{Frame}
\end{figure}

In particular, after processing any given update, our algorithms will be capable of responding to membership queries in $O(1)$-time, and to clustering enumeration queries in time $O(|C|)$, where $C$ is the clustering containing the query point.
