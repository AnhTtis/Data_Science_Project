\section{Lower Bound Against Adaptive Adversaries}
\label{sec:LBadap}

In this section we present our lower bounds on the approximation
guarantees of dynamic algorithms for $k$-center, $k$-median, $k$-means,
$k$-sum-of-radii, $k$-sum-of-diameters, and $(k,z)$-clustering against
a \emph{metric-adaptive} adversary. Recall that a point-adaptive adversary needs to fix the metric space in advance, while a metric-adaptive adversary only needs to answer a distance query consistently with all answers it gave previously (so the metric itself is chosen adaptively). First,
we define a generic strategy for an adversary that in each operation
creates or deletes a point and answers the distance queries of the
algorithm. Then we show how to derive lower bounds for all of our
problems from this single strategy. Some formal details of the
analysis are deferred to \cref{sec:appendix-lower-adaptive}.


\subsection{Strategy of the Adaptive Adversary}

Let $f(k,n)$ be a positive function that is, for every fixed $k$, non-decreasing in $n$. Suppose that there is an algorithm (for any of the problems under
consideration) that in an amortized sense queries the distances
between at most $f(k,n)$ pairs of points per operation of the adversary,
where $n$ denotes the number of points at the beginning of the respective
operation. Note that any algorithm with an amortized update time
of at most $f(k,n)$ fulfills this condition. To determine the distance
between two points the algorithm asks a distance query to the adversary. 
We present now an adversary ${\cal A}$ whose goal
is to maximize the approximation ratio of the algorithm. To
record past answers and to give consistent answers, ${\cal A}$ maintains
a graph $G=(V,E)$ which contains a vertex $v_{p}\in V$ for each
point $p$ that has been inserted previously (including points that
have been deleted already). Intuitively, with the edges in $E$
the adversary keeps track of previous answers to distance queries.
Each vertex $v_{p}$ is labeled as \emph{open, closed, }or \emph{off}.
If a point $p$ has not been deleted yet, then its vertex $v_{p}$
is labeled as open or closed. Once point $p$ is deleted, then
$v_{p}$ is labeled as off. Intuitively, if $p$ has not been deleted
yet, then $v_{p}$ is open if it has small degree and closed if
it has large degree. In the latter case, ${\cal A}$ will delete $p$
soon. All edges in $G$ have length 1, and for two vertices $v,v'\in V$
we denote by $d_{G}(v,v')$ their distance in $G$.

When choosing the next update operation, ${\cal A}$ checks whether there
is a closed vertex $v_{p}$. If yes, ${\cal A}$ picks an arbitrary
closed vertex $v_{p}$, deletes the corresponding point $p$, and
labels the vertex $v_{p}$ as off. Otherwise, it adds a new point
$p$, adds a corresponding vertex $v_{p}$ to $G$, and labels $v_{p}$
as open.

Suppose now that the algorithm queries the distance $d(p,p')$ for
two points $p,p'$ while processing an operation. Note that $p$ and/or
$p'$ might have been deleted already. If both $v_{p}$ and $v_{p'}$
are open then ${\cal A}$ reports to the algorithm that $d(p,p')=1$
and adds an edge $\{v_{p},v_{p'}\}$ to $E$. Intuitively, due to
the edge $\{v_{p},v_{p'}\}$ the adversary remembers that it reported
the distance $d(p,p')=1$ before and ensures that in the future it
will report distances consistently. Otherwise, ${\cal A}$ considers
an augmented graph $G'$ which consists of $G$ and has
in addition an edge $\{v_{\bar{p}},v_{\bar{p}'}\}$ of length 1
between any pair of open vertices $\bar{p},\bar{p}'$. The adversary
computes the shortest path $P$ between $p$ and $p'$ in $G'$ and
reports that $d(p,p')$ equals the length of $P$. If $P$ uses an
edge between two open vertices $\bar{p},\bar{p}'$, then ${\cal A}$
adds the edge $\{v_{\bar{p}},v_{\bar{p}'}\}$ to $G$. Note that $P$
can contain at most one edge between two open vertices since it
is a shortest path in a graph in which all pairs of open vertices
have distance~1. %
Observe that if both $v_{p}$ and $v_{p'}$ are open then this procedure
reports that $d(p,p')=1$ and adds an edge $\{v_{p},v_{p'}\}$ to
$E$ which is consistent with our definition above for this
case. If a vertex $v_{p}$ has degree at least $100f(k,t)$ for the current operation $t$, then $v_{p}$ is labeled as closed. A closed vertex never becomes open again.

In the next lemma we prove some properties about this strategy of
${\cal A}$. For each operation $t$, denote by $G_{t}=(V_{t},E_{t})$
the graph $G$ at the beginning of the operation $t$. 
Recall that the value of $n$ right before the operation $t$ (which is  the number of
current points) equals the number of open and closed
vertices in $G_{t}$.
We show that
the the number of open vertices is $\Theta(t)$, each vertex has
bounded degree, and there exist arbitrarily large values $t$
such that in $G_{t}$ there are no closed vertices (i.e., only open
and off vertices). \begin{lemma} \label{lem:mpt-det-low-adv-prop}
	For every operation $t>0$ the strategy of the adversary ensures the
	following properties for $G_{t}$
	\begin{enumerate}
		\item the number of open vertices in $G_{t}$ is at least $92t/100$,
		\label{it::mpt-det-low-adv-prop-numac} 
		\item each vertex in $G_{t}$ has a degree of at most $101f(k,n)$, \label{it::mpt-det-low-adv-prop-deg} 
		\item there exists an operation $t'$ with $t < t' \leq 2t$ such that $G_{t'}$
		contains only open and off vertices, but no closed vertices. \label{it::mpt-det-low-adv-prop-clean} 
	\end{enumerate}
\end{lemma} We say that an operation $t\in\N$ is a \emph{clean operation}
if in $G_{t}$ there are no closed vertices. For any clean operation
$t$, denote by $\bar{G}_{t}=(\bar{V}_{t},\bar{E}_{t})$ the subgraph
of $G_{t}$ induced by the open vertices in $V_{t}$.

\vspace{-0.2cm}


\paragraph*{Consistent metrics. }

The algorithm does not necessarily know the complete metric of the
given points, it knows only the distances reported by the adversary.
In particular, there might be many possible metrics that are consistent
with the reported distances. For each $t\in\N$ denote by $Q_{t}$
the points that were inserted before operation $t$, including
all points that were deleted before operation $t$, and let
$P_{t}\subseteq Q_{t}$ denote the points in $Q_{t}$ that are not
deleted. Given a metric $M$ on any point set $P'$, for all pairs
of points $p,p'\in P'$ we denote by $d_{M}(p,p')\ge0$ the distance
between $p$ and $p'$ according to $M$. For any $t\in\N$ we say
that a metric $M$ for the point set $Q_{t}$ is \emph{consistent}
if for any pair of points $p,p'\in Q_{t}$ for which the adversary
reported the distance $d(p,p')$ before operation~$t$, it
holds that $d(p,p')=d_{M}(p,p')$. In particular, any consistent metric
might be the true underlying metric for the point set $Q_{t}$.

The key insight is that for each clean operation $t$, we can build
a consistent metric with the following procedure. Take the graph $G_{t}$
and insert an arbitrary set of edges of length $1$ between pairs
of open vertices (but no edges that are incident to off vertices),
and let $G'_{t}$ denote the resulting graph. Let $M$ be the shortest
path metric according to $G'_{t}$. If a metric $M$ for $Q_{t}$
is constructed in this way, we say that $M$ is an \emph{augmented
	graph metric for }$t$.



\begin{lemma} \label{lem:metric-consistent}Let $t\in\N$ be a clean
	operation and let $M$ be an augmented graph metric for $t$. Then
	$M$ is consistent. 
\end{lemma} 
In particular, there are no shortcuts
via off vertices in $G_{t}$ that could make the metric $M$
inconsistent.

We fix a clean operation $t\in\N$. We define some metrics that are consistent
with $Q_{t}$ that we will use later for the lower bounds for our
specific problems. The first one is the ``uniform'' metric $M_{\mathrm{uni}}$
that we obtain by adding to $G_{t}$ an edge between \emph{each} pair
of open vertices in $G_{t}$. As a result, $d_{M_{\mathrm{uni}}}(p,p')=1$
for any $p,p'\in P_{t}$. 
\begin{lemma} \label{lem:muni-consistent}
	For each clean operation $t$ the corresponding metric $M_{\mathrm{uni}}$
	is consistent. 
\end{lemma} 
In contrast to $M_{\mathrm{uni}}$, our
next metric ensures that there are distances of up to $\Omega(\log n)$
between some pairs of points. Let $p^{*}\in P_{t}$ be a point such
that $v_{p^{*}}$ is open. For each $i\in\N$ let $V^{(i)}\subseteq V_{t}$
denote the open vertices $v\in V_{t}$ with $d_{G_{t}}(v_{p^{*}},v)=i$,
and let $V^{(n)}\subseteq V_{t}$ denote the vertices in $G_{t}$
that are in a different connected component than $p^{*}$. Since
the vertices in $G_{t}$ have degree at most $100 f(k,n)$, more than half of all
vertices are in sets $V^{(i)}$ with $i\ge\Omega(\log n/ \log f(k,n))$.

We define now a metric $M(p^{*})$ as the shortest path metric in
the graph defined as follows. We start with~$G_{t}$; for each $i,i'\in\N$
we add to $G_{t}$ an edge $\{v_{p},v_{p'}\}$ between any pair of
vertices $v_{p}\in V^{(i)}$, $v_{p'}\in V^{(i')}$ such that $|i-i'|\le1$.
As a result, for any $i,i'\in\N$ and any $v_{p}\in V^{(i)}$, $v_{p'}\in V^{(i')}$
we have that $d_{M(p^{*})}(p,p')=\max\{|i-i'|,1\}$, i.e., $d_{M(p^{*})}(p,p')=1$
if $i=i'$ and $d_{M(p^{*})}(p,p')=|i-i'|$ otherwise. 
\begin{lemma}
	\label{lem:mstar-consistent} For each clean operation $t$ and each
	point $p^{*}\in V_{t}$ the metric $M(p^{*})$ is consistent. 
\end{lemma}
For any two thresholds $\ell_{1},\ell_{2}\in\N_{0}$ with $\ell_1 < \ell_2$  we define a metric
$M_{\ell_{1},\ell_{2}}(p^{*})$ (which is a variation of $M(p^{*})$)
as the shortest path metric in the following graph. Intuitively,
we group the vertices in $\bigcup_{i=0}^{\ell_{1}}V^{(i)}$ to one
large group and similarly the vertices in $\bigcup_{i=\ell_{2}}^{\infty}V^{(i)}$.
Formally, in addition to the edges defined for $M(p^{*})$, for each
pair of vertices $v_{p}\in V^{(i)}$, $v_{p'}\in V^{(i')}$ we add
an edge $\{v_{p},v_{p'}\}$ if $i\le i'\le\ell_{1}$ or $\ell_{2}\le i\le i'$.%


\begin{lemma} \label{lem:mstarrange-consistent} For each clean operation
	$t$, each point $p^{*}\in V_{t}$, and each $\ell_{1},\ell_{2}\in\N_{0}$
	the metric $M_{\ell_{1},\ell_{2}}(p^{*})$ is consistent. \end{lemma}

\paragraph*{Lower bounds. }

Consider a clean operation $t$. The algorithm cannot distinguish
between $M_{\mathrm{uni}}$ and $M(p^{*})$ for any $p^{*}\in P_{t}$.
In particular, for the case that $k=1$ (for any of our clustering
problems) the algorithm selects a point $p^{*}$ as the center, and
then for each $i$ it cannot determine whether the distance
of the points in $V^{(i)}$ to $p^{*}$ equals 1 or $i$.
However, there are at least $n/2$ points in sets $V^{(i)}$
with $i\ge\Omega(\log n/\log f(k,n)))$ and hence they contribute a large amount
to the objective function value. This yields the following lower bounds,
already for the case that $k=1$. With more effort, we can show them even for bi-criteria approximations,
i.e., for algorithms that may output $O(k)$ centers, but where the approximation
ratio is still calculated with respect to the optimal cost on $k$
centers.

For $k$-center the situation changes
if the algorithm does
not need to be able to report an upper bound of the value of its computed
solution (but only the solution itself), since if $k=1$, then any
point is a 2-approximation. However, for arbitrary $k$ we can argue
that there must be $3k$ consecutive sets $V^{(i)},V^{(i+1)},...,V^{(i+3k-1)}$
such that the algorithm does not place any center on any point corresponding
to the vertices in these sets and hence incurs a cost of at least
$3k/2$. On the other hand, for the metric $M_{i,i+3k-1}(p^{*})$ the
optimal solution selects one center from each set $V^{(i+1)},V^{(i+4)},V^{(i+7)},...$
which yields a cost of only 1. %

\thmDetLB*



\subsection{Deferred Proofs}
\label{sec:appendix-lower-adaptive}

We introduce some formal notation and definitions we use to revisit the adversarial strategy that generates an input stream and answers distance queries on the set of currently known points. Then, we derive lower bound constructions for the aforementioned problems that are based on the metric space defined by the stream and the answers to the algorithm.



\SetKwFunction{FnGenerate}{GenerateStream}
\SetKwFunction{FnAnswer}{AnswerQuery}

We describe a strategy for an adversary $\mathcal{A}$ that generates a stream of update operations~$\s$ and answers distance queries~$\q$ on pairs of points by any dynamic algorithm with a guarantee on its amortized  complexity. In the following presentation, the adversary constructs the underlying metric space ad hoc. More precisely, the adversary constructs two metric spaces simultaneously that cannot be distinguished by the algorithm and its queries. All subsequent lower bounds stem from the fact that the problem at hand has different optimal costs on the input for the two metric spaces. When the algorithm outputs a solution, the adversary can fix a metric space that induces high cost for the centers chosen by the algorithm.

During the execution of the algorithm, the adversary maintains a graph $\kg$.
Each point that was inserted by the adversary is represented by a node in $\kg$. All query answers given to the algorithm by the adversary can be derived from 
$G$
using the shortest path metric $\dsp[\kg]{\cdot}{\cdot}$ on $\kg$. We denote the algorithm's $i$\xth/ query after update operation $t$ by $\q[t][i]$, and the adversary's answer by $\ans{\q[t][i]}$. For $t > 0$, we denote the number of queries asked by the algorithm between the $t$\xth/ and the $(t+1)$\xth/ update operation by $c(t)$. If $t$ is clear from context, we simplify notation and write $c := c(t)$. Note that, in this section, we use a slightly extended notation when indexing graphs when compared to other sections. Details follow.

We number the update operations consecutively starting with 1 using index $t$ and after each update operation, we index the distance queries that the algorithm issues while processing the update operation and the immediately following value- or solution-queries using index $i$. Let $c > 0$ and let $\kg[0][c(0)]$ be the empty graph.  For every $t > 0, i \in [c(t)]$, consider $i$-th query issued by the algorithm processing the $t$-th update operation.  The graph $\kg[t][i]$ has the following structure. For every point $x$ that is inserted in the first $t$ operations, $\V{\kg[t][0]}$ contains a node $x$. All edges have length $1$, and it holds that $\V{\kg[t][i]} \supseteq \V{\kg[t][i-1]}$ and $\E{\kg[t][i]} \supseteq \E{\kg[t][i-1]}$.

Edges are inserted by the adversary as detailed below. Let $\preceq$ denote the predicate that corresponds to the lexicographic order. In particular, the adversary maintains the following invariant, which is parameterized by the update operation $t$ and the corresponding query $i$: for all $(t',j) \preceq (t,i)$ and $(u,v) \defeq \q[t'][j]$, $\ans{\q[t'][j]} = \dsp[\kg[t][i]]{u}{v}$. In other words, any query given by the adversary remains consistent with the shortest path metric on all versions of 
$G$
after the query was answered. The adversary distinguishes the following types of nodes in $\kg[t][i]$ to answer a query. Recall that $f \defeq f(k,n)$ is an upper bound on the amortized complexity per update operation of the algorithm, which is non-decreasing in $n$ for fixed $k$.

\begin{definition}[type of nodes]
	\label{def:node_types}
	Let $t \geq 0, i \in [c]$ and let $u \in \V{\kg[t][i]}$. If $u$ has degree less than $100f(k,i)$ for all $i \in [t]$, it is \emph{open} after update $t$, otherwise it is \emph{closed}. In addition, the adversary can mark closed nodes as \emph{off}. We denote the set of open, closed and off nodes in $\kg[t][i]$ by $\actno[t][i]$, $\pasno[t][i]$ and $\disno[t][i]$, respectively.%
\end{definition}

For operation $t$, the adversary answers the $i$\xth/ query $\q[t][i]$ according to the shortest path metric on $\kg[t][i-1]$ with the additional edge set $\actno[t][i-1] \times \actno[t][i-1]$. In other words, the adversary (virtually) adds edges between all open nodes in $\kg[t][i-1]$ and reports the length of a shortest path between the query points in the resulting graph. After the adversary answered query $\q[t][i]$, the resulting %
graph $\kg[t][i]$ is $\kg[t][i-1]$ plus the (unique) edge $e$ of the shortest path between two open nodes that is not in $\kg[t][i-1]$ if such edge exists. If the connected component of $e$ in the resulting graph does not contain any open node, we also add an edge between the connected component and a node with degree at most $50f(k,i)$. Thus, the algorithm maintains the invariant that each connected component has at least one open vertex (see \cref{lem:kg-actnum} for details). A key element of our analysis is that \empty{all} answers up to operation $t$ and query $i$ are equal to the length of the shortest paths between the corresponding query points in $\kg[t][i]$. The generation of the input stream and the answers to all queries are formally given by \cref{alg:stream} and~\ref{alg:answer}, respectively.


\begin{algorithm}
	\Fn{\FnGenerate{$t$}}{
		\If{there exists a closed node $x \in \V{\kg[t-1][c]}$}{
			mark $x$ as off in $\kg[t][0]$\;
			\Return{$\langle$ delete $x$ $\rangle$}
		}
		\Else{
			let $x$ be a new point, i.e., that was not returned by the adversary before\;
			\Return{$\langle$ insert $x$ $\rangle$}
		}
	}
	\caption{\label{alg:stream} Construction of element $\s[t]$ of $\s$}
\end{algorithm}
\begin{algorithm}
	\Fn{\FnAnswer{$\q[t][i] = (x,y)$}}{
		let $\akg[t][i] = (\V{\kg[t][i-1]}, \E{\kg[t][i-1]} \cup (\actno[t][i-1] \times \actno[t][i-1]))$\;
		let $p = (e_1, \ldots, e_k)$ be a shortest path between $x$ and $y$ in $\akg[t][i]$\;
		set $\kg[t][i] \defeq \kg[t][i-1]$\;
		\If{$\exists e_i \notin \E{\kg[t][i-1]}$}{
			insert $e_i$ into $\kg[t][i]$ \label{lin:alg_answer_instert_a} \;
			let $C$ be the connected component of $e_i$ in $\kg[t][i]$ \;
			\If(\tcp*[f]{make sure $C$ contains an open node}){$C \cap \actno[t][i] = \emptyset$}{
				let $u = \arg\min_{u' \in C} \deg(u')$ \tcp*{pick node with degree $100f(k,t)$}
				let $v = \arg\min_{v' \in \actno[t][i]} \deg(v')$ \tcp*{pick node with degree at most $50f(k,t)$}
				insert $(u,v)$ into $\kg[t][i]$ \label{lin:alg_answer_instert_b} \;
			}
		}
		\Return{length of $p$}
	}
	\caption{\label{alg:answer} Answer of the adversary to query $\q[t][i]$}
\end{algorithm}

\subsubsection{Adversarial Strategy}

Let $n_t$ be the number of open and closed nodes, i.e., the number of current points for the algorithm, after operation $t$. In the whole section we use the notations  of $\kg[t'][i], \actno[t][i]$ etc.  from Definition~\ref{def:node_types}.

\paragraph{Proof of \cref{lem:mpt-det-low-adv-prop}}

The next three lemmas prove the three claims in \cref{lem:mpt-det-low-adv-prop}.

\begin{lemma}[\cref{lem:mpt-det-low-adv-prop} (\ref{it::mpt-det-low-adv-prop-numac})]
	\label{lem:kg-actnum}
	For every $t > 0$, the number of open nodes in $\kg[t][0]$ is at least $92t / 100$, and for every $t > 0, j \geq 0$, there exists at least one node with degree at most $50f(k,t)$ in $\kg[t][j]$.
\end{lemma}
\begin{proof}
	Recall that $f(k,n)$ is a positive function that is non-decreasing in $n$. We prove the first part of the claim by induction. By the properties of $f$, the case $t = 1$ follows trivially. Let $t \geq 2$. For any $i \in [t]$, the algorithm's query budget increases by $f(k, n_i)$ queries after the $i$\xth/ update. Since $f(k,i)$ is non-decreasing, nodes inserted after update operation $i-1$ can only become closed if their degrees increase to at least $100f(k,i)$. Answering a query $i$ inserts at most two edges into $\kg[t][i-1]$, and the sum of degrees increases by at most four. Therefore, it holds that $\pasnum[t][0] \leq \sum_{i \in [t]} 4f(k, n_i) / (100f(k, i)) \leq \sum_{i \in [t]} 4f(k, i) / (100f(k, i)) \leq 4t / 100$. It follows that the adversary will delete at most $4t / 100$ points in the first $t$ operations and insert points in the other at least $(1-4/100)t$ operations. The number of open nodes after update $t$ is $\actnum[t][0] \geq t - \pasnum[t][0] - 4t / 100 \geq 92t / 100$.
	
	To prove the second part of the claim, let $s_{t,j}$ be the number of nodes in $\kg[t][j]$ with degree at most $50f(k,t)$. Similarly as before, we have $n_t - s_{t,j} \leq \sum_{i \in [t]} 4f(k, n_i) / (50f(k, i)) \leq \sum_{i \in [t]} 4f(k, i) / (50f(k, i)) \leq 4t / 50$. Therefore, $s_{t,j} \geq n_t - 4t / 50 \geq (t - 4t / 100) - 4t / 50 \geq 1$.
\end{proof}

\begin{lemma}[\cref{lem:mpt-det-low-adv-prop} (\ref{it::mpt-det-low-adv-prop-deg})]
	\label{lem:max-degree}
	For every $t > 0, i \in [c]$, all nodes have degree at most $100f(k,t) + 1$ in $\kg[t][i]$.
\end{lemma}
\begin{proof}
	By definition, the claim is true for open nodes. Edges are only inserted into $G$
	if the algorithm queries for the distance between two nodes $x,y$ and the adversary determines a shortest path between $x$ and $y$ that contains edges that are not present in $\kg[t][i-1]$ (see \cref{alg:answer}). Since all such edges are edges between open nodes, only degrees of open nodes in $\kg[t][i]$ increase. The adversary finds a shortest path on a supergraph of $\actno[t][i] \times \actno[t][i]$. Therefore, any shortest path it finds contains at most one edge with two open endpoints. It follows that a query increases the degree of any open node in $\kg[t][i]$ by at most one, which may turn it into a closed node with degree $100f(k,t)$. If the connected component of this edge contains no open node, it also inserts an edge from an open node (with degree at most $50 f(k,t)$) to the vertex with smallest degree in the component. Since the vertex of the component that became closed most recently always has degree $100f(k,t)$, this increases the degree of every closed node at most once by $1$.
\end{proof}

\begin{lemma}[\cref{lem:mpt-det-low-adv-prop} (\ref{it::mpt-det-low-adv-prop-clean})]
	\label{lem:no-closed-nodes}
	For every $t > 0$, there exists a clean update operation $t'$, $t < t' \le 2t$, i.e., $\kg[t'][0]$ only contains open and off nodes, but no closed nodes.
\end{lemma}
\begin{proof}
	We prove the claim by induction over the operations $t$ with the properties that $\kg[t][0]$ contains no closed node, but $\kg[t+1][0]$ contains at least one closed node. The claim is true for the initial (empty) graph $\kg[0][0]$. Let $t > 0$. We prove that in at least one operation $t' \in \{ t+1, \ldots, 2t \}$, the number of closed nodes is $0$. For the sake of contradiction, assume that for all $t'$, $t < t' \leq 2t$, the number of closed nodes is non-zero, i.e., $\pasnum[t'][0] > 0$. We call an open node \emph{semi-open} if it has degree greater than $50f(k,i)$ after some update $i \in [2t]$. Otherwise, we call it \emph{fully-open}. Recall that, similarly, a vertex becomes closed if it has degree at least $100f(k,i)$ after some update $i \in [2t]$ (and never becomes open again).
	
	For any $i \in [2t]$, the algorithm's query budget increases by $f(k, n_i)$ queries after the $i$\xth/ update. Since $f(k,i)$ is non-decreasing, nodes inserted after update operation $i-1$ can only become semi-open if their degrees increase to at least $50f(k,i)$ (resp. $100f(k,i)$) by the definition of semi-open (resp. closed). Also due to the monotonicity of $f$, the number of semi-open or closed nodes is maximized if the algorithm invests its query budget as soon as possible. It follows that the number of semi-open or closed nodes up to operation $2t$ is at most $\sum_{j \in [2t]} 4f(k,j) / (50f(k,j)) \leq 4t/50$. Without loss of generality, we may assume that all semi-open nodes are closed (so the algorithm does not need to invest budget to make them closed).
	
	After update operation $2t$, the algorithm's total query budget from all update operations is at most $\sum_{i \in [2t]} f(k,n_i) \leq \sum_{i \in [2t]} f(k,i) \leq 2t f(k,2t)$. Answering a query $i$ inserts at most two edges into $\kg[t][i-1]$, and the sum of degrees increases by at most four. The algorithm may use its budget to increase the degree of at most $2t \cdot 4f(k,2t) / (50f(k,2t)) \leq 8t/50$ fully-open nodes to at least $100f(k,2t)$, i.e., to make them closed nodes. 
	Recall our assumption that $\pasnum[i][0] > 0$ for all $i \in \{t+1, \ldots, 2t\}$. The adversary deletes one point corresponding to a closed node in each update operation from $\{t+1, \ldots, 2t \}$. Therefore, the number of closed nodes after update operation $2t$ is
	\begin{equation*}
		\pasnum[2t][0] \leq \frac{4t}{50} + \frac{8t}{50} - t \leq \frac{12t}{50} - t < 0.
	\end{equation*}
	This is a contradiction to the assumption.
\end{proof}

\paragraph{Proofs of \cref{lem:metric-consistent,lem:muni-consistent,lem:muni-consistent,lem:mstar-consistent,lem:mstarrange-consistent}}

The following observation follows immediately from the properties of shortest path metrics.

\begin{observation}
	\label{lem:convex-compliance}
	Let $G=(V,E)$ and $G'=(V,E')$ be two graphs so that $E \subseteq E'$. If a sequence of queries is consistent with the shortest path metric on $G$ as well as on $G'$, then, for any $E''$, $E \subseteq E'' \subseteq E'$, it is also consistent with the shortest path metric on $(V, E'')$.
\end{observation}

The following lemma together with \cref{lem:convex-compliance} implies \cref{lem:metric-consistent,lem:muni-consistent,lem:muni-consistent,lem:mstar-consistent,lem:mstarrange-consistent,lem:mmulti-consistent} by setting $G = \kg[t][i-1]$ and $G' = (\V{\kg[t][i-1]}, \E{\kg[t][i-1]} \cup (\actno[t][i-1] \times \actno[t][i-1]))$ in \cref{lem:convex-compliance}.

\begin{lemma}
	\label{lem:sp-consistent}
	For any $t, t' > 0$, $i, i' \in [c]$ so that $(t',i') \prec (t,i)$, the answer given to query $\q[t'][i']$ is consistent with the shortest path metric on $G'$.
\end{lemma}
\begin{proof}
	Let $t' > 0$, $i' \in [c]$ so that $(t',i') \prec (t,i)$ and denote $(x,y) \defeq q \defeq \q[t'][i']$. We prove that $\ans{q} = \dsp[\kg[t][i]]{x}{y}$. As neither vertices nor edges are deleted after they have been inserted, for every $(t_1, i_1) \prec (t_2, i_2)$, $\kg[t_2][i_2]$ is a supergraph of $\kg[t_1][i_1]$. Thus, if there exists a path $P$ between $x$ and $y$ in $\kg[t'][i']$, a shortest path in $\kg[t][i]$ between $x$ and $y$ cannot be longer than $P$.
	
	It remains to prove $\ans{q} \leq \dsp[\kg[t][i]]{x}{y}$. For the sake of contradiction, assume that there exist $t'',i''$ 
	so that $\ans{q} = \dsp[\kg[t''][i''-1]]{x}{y}$, but $\ans{q} > \dsp[\kg[t''][i'']]{x}{y}$. By \cref{def:node_types}, closed nodes never become open. For any closed node $v \in \pasno[t''][i''-1]$, it follows that $\dsp[\kg[t''][i'']]{v}{\actno[t][i]} \geq \dsp[\kg[t''][i''-1]]{v}{\actno[t][i-1]}$ as \cref{alg:answer} only inserts edges between vertices in $\actno[t][i]$ into $\kg[t][i]$. Therefore, any shortest path between $x$ and $y$ in the graph $(\V{\kg[t''][i''-1]}, \E{\kg[t''][i''-1] \cup (\actno[t''][i''-1] \times \actno[t''][i''-1])}$ has length at least $\dsp[\kg[t''][i''-1]]{x}{y} = \ans{q}$.
\end{proof}

\subsubsection{Lower Bounds for Clustering}

Our lower bounds apply for the case that the algorithm is allowed to choose as centers any points that have ever been inserted as well as to the case where  centers must belong to the set of current points. In the whole section we use the notations  of $\kg[t'][i], \actno[t][i]$ etc. from the introduction of \cref{sec:appendix-lower-adaptive}.

\paragraph{Proof of \cref{thm:det-1lb}}

For any $\ell\in\N_{0}$, we define a metric $M_{\ell}(P^{*})$
on a subset of points $P^{*}$ as the shortest path metric on the
following graph. For each pair of open vertices $u,v$ we add an edge if $d(u,P^{*})\geq\ell$
and $d(v,P^{*})\geq\ell$.

\begin{lemma} \label{lem:mmulti-consistent} For each clean update operation
	$t$, each subset of points $P^{*}\in V_{t}$, and each $\ell\in\N_{0}$
	the metric $M_{\ell}(P^{*})$ is consistent.
\end{lemma}
\begin{proof}
	The metric $M_{\ell}(P^{*})$ is an augmented graph metric for $t$ and, thus, for a clean update operation $t$, it
	is consistent by Lemma~\ref{lem:metric-consistent}.
\end{proof}

\begin{lemma}[\cref{thm:det-1lb}, part 1]
	Let $k \ge 2$.
	Consider any dynamic algorithm for maintaining an approximate $k$-center solution of a dynamic point set  that (1) queries amortized $f(k,n)$ distances per operation, where $n$ is the number of current points,
	and (2) outputs at most $g(k) \in O(k)$ centers.
	For any $t\geq 2$ such that $t$ is a clean operation, the approximation factor of the algorithm's solution
	(with respect to an optimal $k$-center solution) against an
	adaptive adversary right after operation $t$ is at least
	$\Omega\left(\min\left\{ k,\frac{\log n}{k\log{f(k,2n)}}\right\} \right)$.
\end{lemma}
\begin{proof}
	Denote $G\defeq(V,E)\defeq\kg[t][0]$ and $A\defeq\actno[t][0]$.
	By \cref{lem:kg-actnum}, the number of open nodes in $G$ is
	at least $92t/100\geq t$, which implies that $|A| \ge 92t/100$. 
	Thus,  after operation $t$,  the number $n$ of current points is at least  $92t/100$.
	
	
	
	Without loss of generality, we assume that $G[A]$ has exactly one connected component: If this is not the case, let $C_1, \ldots, C_s$ be the connected components of $G[A]$ and observe that we may insert a path connecting the connected components by inserting
	into $G[A]$ edges  $(v_1,v_2), \ldots, (v_{s-1}, v_s)$ of length $1$, where $v_i \in C_i$ are arbitrary vertices. This increases the maximum degree of nodes in $G$ by at most $2$ and
	the shortest-path metric $M$ on the resulting graph that is constructed in this way is an augmented graph metric for $t$. As $t$ is clean, Lemma~\ref{lem:metric-consistent} shows that $M$ is consistent.
	
	Let $x \in V$ be any node. By \cref{lem:max-degree}, the number $n^{(i)}$ of nodes that have distance at most $i$ to $x$ is at most $\sum_{j \in [i]} (100f(1,t)+3)^j \leq (100f(k,t)+3)^{i+1}$. 
	Consider the largest $\ell$ such that $(100f(k,t)+3)^{\ell+1} < n$. It follows that there exists a node $z$ at distance $\ell+1$ to $x$. Furthermore $(100f(k,t)+3)^{\ell+3} \ge n^{(\ell+1)} \ge n$, which implies that
	$\ell+2 \ge  \log_{100f(k,t)+3} n.$
	
	Let $S = \{ s_1, \ldots, s_{g(k)} \}$ be the solution of the algorithm. For $i \in [\ell+1]$, let us define $V^{(i)}$ to be the set of vertices $v$ in $G[A]$ with $d_{G[A]}(x,v) = i$. By pigeon hole principle, there must exist a consecutive sequence $(i_1, \ldots, i_m)$ so that $m \geq \ell / (g(k)+1)$ and, for all $i \in \{ i_1, \ldots, i_m \}$, $S \cap V^{(i)} = \emptyset$. Let $k' =  \min\{3k-1, m/2\}$. Consider the metric $M_{i_1,i_{k'}}(x)$ and let $y_{i_1}, \ldots, y_{k'}$ be elements from the respective sets $V^{(i_1)}, \ldots,V^{(i_{k'})}$.
	
	The algorithm's solution $S$ has cost at least $k'$ because $\dsp[G]{S}{V_{i_{k'}}} \geq k'$. The solution $\{ y_{3j-2} \mid j \in \N \wedge 3j-2 \in [k'] \}$ is optimal and has cost $1$. It follows that the approximation factor of $S$ is greater than or equal to $k' = \min\{3k-1, m/2\}$.
	
	Let $n$ be the number of points at iteration $t$. We calculate that
	\begin{alignat*}{1}
		\min\left\{ 3k/2,m/2\right\}  & \ge\Omega\left(\min\left\{ k,m\right\} \right)\\
		& \ge\Omega\left(\min\left\{ k,\frac{\ell}{g(k)+1}\right\} \right)\\
		& \ge\Omega\left(\min\left\{ k,\frac{1}{g(k)}\left(\frac{\log n}{\log{(103f(k,t))}}-1\right)\right\} \right)\\
		& \ge\Omega\left(\min\left\{ k,\frac{\log n}{k\log{(103f(k,2n))}}\right\} \right)\\
		& \ge\Omega\left(\min\left\{ k,\frac{\log n}{k\log103+k\log{f(k,2n)}}\right\} \right)\\
		& \ge\Omega\left(\min\left\{ k,\frac{\log n}{k\log{f(k,2n)}}\right\} \right)
	\end{alignat*}
	using that $g(k)=O(k)$.
\end{proof}

\begin{lemma}[\cref{thm:det-1lb}, part 2]
	\label{lem:diam}
	Consider any dynamic algorithm for computing the diameter of a dynamic point set  that queries amortized $f(1,n)$ distances per operation, where $n$ is the number of current points,
	and outputs at most $g\geq1$ centers.
	For any $t\geq 2$ such that $t$ is a clean operation, the approximation factor of the algorithm's solution
	(with respect to the correct diameter) against an
	adaptive adversary right after operation $t$ is at least
	$\log (92t/100) / (\log{103f(1,t)}) -1$ and $92t/100 \le n \le t$.
\end{lemma}
\begin{proof}
	Denote $G\defeq(V,E)\defeq\kg[t][0]$ and $A\defeq\actno[t][0]$.
	By \cref{lem:kg-actnum}, the number of open nodes in $G$ is
	at least $92t/100\geq t$, which implies that $|A| \ge 92t/100$. 
	Thus,  after operation $t$,  $n \ge 92t/100$.
	
	
	
	Without loss of generality, we assume that $G[A]$ has exactly one connected component: If this is not the case, let $C_1, \ldots, C_s$ be the connected components of $G[A]$ and observe that we may insert a path connecting the connected components by inserting
	into $G[A]$ edges  $(v_1,v_2), \ldots, (v_{s-1}, v_s)$ of length $1$, where $v_i \in C_i$ are arbitrary vertices. This increases the maximum degree of nodes in $G$ by at most $2$ and
	the shortest-path metric $M$ on the resulting graph that is constructed in this way is an augmented graph metric for $t$. As $t$ is clean, Lemma~\ref{lem:metric-consistent} shows that $M$ is consistent.
	
	Let $x \in V$ be any node. By \cref{lem:max-degree}, the number $n^{(i)}$ of nodes that have distance $i$ to $x$ is at most $\sum_{j \in [i]} (100f(1,t)+3)^j \leq (100f(1,t)+3)^{i+1}$. 
	Consider the largest $i$ such that $(100f(1,t)+3)^{i+1} < n$. It follows that there exists a node at distance $i+1$ to $x$. Furthermore $(100f(1,t)+3)^{i+2} \ge n^{(i+1)} \ge n$, which implies that
	$i+2 \ge  \log_{100f(1,t)+3} n.$
	As $f(1,t) \ge 1$ for all values of $t$, there exists a shortest path $P$ starting at $x$ of length at least $i+1 \geq \log_{100f(1,t)+3}n - 1 \geq (\log n/ \log{(103f(1,t))}) -1 =: \ell$. It follows that the diameter is at least $\ell$.
	On the other hand, $M$ can be extended by adding an edge between any pair of open nodes, resulting in the
	consistent metric  $M_{\mathrm{uni}}$. For this metric  the diameter of $G$ is 1.
	As the algorithm cannot tell whether $\ell$ or 1 is the correct answer, and it always has to output a value that is as least as large as the correct answer, it will output at least $\ell$. Thus, the approximation ratio is at least $\ell \ge \log (92t/100)/ \log{103f(1,t)} -1$.
	Note that this implies a lower bound for the approximation ratio for $1$-sum-of-radii, and $1$-sum-of-diameter.
\end{proof}

\begin{lemma} \label{lem:det-low-1pclus} 
	Consider any dynamic algorithm for $(1,p)$-clustering that queries amortized $f(1,n)$ distances per operation, where $n$ is the number of current points,
	and outputs at most $1 \le g\leq n$ centers.
	For any $t\geq1$ such that $t$ is a clean operation, the approximation factor of the algorithm's solution
	(with respect to the optimal $(1,p)$-clustering cost) against an
	adaptive adversary right after operation $t$ is at least $\left[\frac{\log(t/4g)}{p + \log(101f(1,t))}\right]^{p}/4$ and $92t/100 \le n \le t$. \end{lemma} 
\begin{proof}
	Denote $G\defeq(V,E)\defeq\kg[t][0]$ and $A\defeq\actno[t][0]$.
	By \cref{lem:kg-actnum}, the number of open nodes in $G$ is
	at least $92t/100\geq t$, which implies that $|A| \ge 92t/100$. 
	Thus,  after operation $t$,  $n \ge 92t/100$.
	
	Let $C$ be the centers that are picked by the algorithm after operation $t$. By the assumption of the lemma, $|C| \le g$.
	Consider $M_{\ell}(C)$, where $\ell\defeq\log(t/4g)/(p+\log(101 f(1,t)))$.
	By \cref{lem:max-degree}, for any $s\in C$, the size of $\lvert\{x\mid x\in A\wedge d(x,s)<\ell\}\rvert$
	is at most $\sum_{i\in[\ell-1]}(101 f(1,t))^{i}<(101 f(1,t))^{\ell}$. 
	Let $V_{+}\defeq\{x\mid x\in A\wedge d(x,C)\geq\ell\}$.
	Since $\lvert C\rvert\leq g$, it holds that $\lvert V_{+}\rvert>|A| -g(101 f(1,t))^{\ell}  \ge 92t / 100 -g(t/4g)^{\log ((101 f(1,t)) / (p + \log (101 f(t,1)))}\ge 92t/100 - t/4 \ge t/2$.
	Since $d(s,V_{+})\geq\ell$ for any $s\in C$, the $(1,p)$-clustering
	cost of $S$, and thus the cost of the algorithm, is at least $\lvert V_{+}\rvert\cdot\ell^{p}\geq t/2\cdot\ell^{p}$.
	
	We will show that if instead a single point corresponding to a vertex of $V_{+}$ is picked as center, then the cost is at most $2t$, which provides an upper bound on the cost of the optimum solution.
	It follows that the approximation factor achieved by the algorithm is at least $\ell^p / 4$.
	
	
	To complete the proof consider a point $x$ whose corresponding point
	$v_{x}$ belongs to $V_{+}$. One can easily show that for
	each $\alpha\in\N$ it holds that $\alpha^{p}\le(101f(1,t)\cdot2^{p})^{2\alpha/3}$
	since for each $\alpha\in\N$ it holds that $\alpha^{p}=2^{p\log\alpha}\le2^{p\cdot\frac{2\alpha}{3}}$.
	Thus, the $(1,p)$-clustering cost if a single point $x$ is chosen
	as center is at most
	
	\begin{alignat*}{1}
		\sum_{i=0}^{\ell-1}g(101f(1,t))^{i}\cdot(\ell-i)^{p}+t\cdot1^{p} & \le g\left((101f(1,t))^{\ell}\cdot\sum_{i=1}^{\ell}\frac{1}{(101f(1,t))^{i}}\cdot i^{p}\right)+t\\
		& \le g\left((101f(1,t))^{\ell}\cdot\sum_{i=1}^{\ell}\frac{(101f(1,t)\cdot2^{p})^{2i/3}}{(101f(1,t))^{i}}\right)+t\\
		& \le g\left((101f(1,t))^{\ell}\cdot\sum_{i=1}^{\ell}\frac{2^{p\cdot2i/3}}{(101f(1,t))^{i/3}}\right)+t\\
		& \le g\left((101f(1,t))^{\ell}\cdot\sum_{i=1}^{\ell}\left(\frac{2^{\frac{2p}{3}}}{(101f(1,t))^{1/3}}\right)^{i}\right)+t\\
		& \le g(101f(1,t))^{\ell}2^{p\ell}+t\\
		& \le g(101f(1,t)\cdot2^{p})^{\ell}+t\\
		& \le5t/4\le2t
	\end{alignat*}
	using that $\ell=\frac{\log(t/4g)}{p+\log(101f(1,t))}=\frac{\log(t/4g)}{\log(2^{p}101f(1,t))}=\log_{2^{p}101f(1,t)}t/4g$.
	This yields an approximation ratio of at least $\frac{t/2\cdot\ell^{p}}{2t}=\Omega(\ell^{p})$. 
\end{proof}

\begin{lemma}[\cref{thm:det-1lb}, part 3]
	For any $k  \geq 1$, any dynamic algorithm which returns a set of $k$-centers against an adaptive adversary
	with an amortized update time of $f(k,n)$, for
	an arbitrary function $f$,  must have an approximation ratio of $\Omega\left(\left(\frac{\log(n)}{z+\log f(1,2n)}\right)^{z}\right)$ for the $(1,z)$-clustering.
\end{lemma}
\begin{proof}
	Let $t\in\N$. By Lemma~\ref{lem:mpt-det-low-adv-prop}, there is
	a value $t'$ with $t<t'\le2t'$ such that $t'$ is a clean operation.
	Let $n$ be the number of open points at iteration $t'$. By \cref{lem:kg-actnum}
	we know that $t'\ge n\ge92t'/100$. Note that hence $t'\le2n$.
	
	Recall that we assumed that the function $f(k,n)$ is non-decreasing
	in $n$ (for any fixed $k$). Suppose that after operation $t'$ we
	query the solution value of an algorithm for $1$-center, $1$-sum-of-radii,
	or $1$-sum-of-diameter. By Lemma~\ref{lem:diam} its approximation
	ratio is at least $\frac{\log(92t'/100)}{\log(103f(k,t'))} -1\ge\frac{\log(92n/100)}{\log(103f(k,2n))} -1=\Omega\left(\frac{\log(n)}{\log(f(k,2n))}\right)$.
	Suppose that instead we query the solution from the algorithm for
	$(1,p)$-clustering. By Lemma~\ref{lem:det-low-1pclus}, its approximation
	ratio is at least
	
	\begin{alignat*}{1}
		\left[\frac{\log(t'/4g)}{p+\log(101f(1,t'))}\right]^{p}/4 & \ge\left[\frac{\log(n/4g)}{p+\log(101f(1,2n))}\right]^{p}/4\\
		& \ge\left[\frac{\log(n)-\log(4g)}{p+\log(101f(1,2n))}\right]^{p}/4\\
		& \ge\left[\frac{\log(n)}{1.1p+1.1\log(101f(1,2n))}\right]^{p}/4\\ 
		& \ge\left[\frac{\log(n)}{1.1p+1.1(7+\log(f(1,2n))}\right]^{p}/4\\ 
		& \ge\left[\frac{\log(n)}{1.1p+8+1.1\log(f(1,2n))}\right]^{p}/4\\ 
		& \ge\left[\frac{\log(n)}{2p+8+2\log(f(1,2n))}\right]^{p}/4\\ 
		& =\Omega\left(\left(\frac{\log n}{2p+8+2\log f(1,2n)}\right)^{p}\right)
	\end{alignat*}
	using that $g=O(1)$. Hence, for $k$-median and $k$-means if we
	take $p=1$ and $p=2$, respectively, this yields bounds of $\Omega\left(\frac{\log n}{10+2\log f(1,2n)}\right)$
	and $\Omega\left(\left(\frac{\log n}{12+2\log f(1,2n)}\right)^{2}\right)$,
	respectively. 	
\end{proof}

