\section{Conclusion}
\label{sec:conclusion}

In this paper, we investigate affective vocal burst recognition (AVBR) by proposing a hierarchical framework with bi-directional regression chains to explicitly consider multiple relationships, (i) between emotional states and diverse cultures, (ii) between low-dimensional and high-dimensional emotion spaces, and (ii) between various emotion classes within the high-dimensional space. To address the data sparsity problem in AVBR, we also integrate SSL representations via a trainable aggregation method. The proposed framework achieves significantly better performance than baseline systems on the HUME-VB dataset. Data analysis on the dataset and the experimental results also supports the necessity of modeling the inherent relationships.
In the future, we will investigate imbalanced learning w.r.t. cultures and labels in the AVBR task. We will also try to interpret the affective cues from the high-level embeddings for VBs.
% In the future, we will conduct research in low-resource VB recognition in cultures with limited training data.