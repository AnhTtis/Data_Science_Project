\section{Experiments}
\label{sec:exp}

\subsection{The A-VB Data}
We use the HUME-VB dataset of emotional non-linguistic vocalizations (vocal bursts)~\cite{Cowen2022HumeVB} that is used in the ACII A-VB Competition 2022 \cite{BairdA-VB2022}. The competition aims to promote research on modeling emotion in vocalizations, and proposes four tasks utilizing the HUME-VB data: the Two-Dimensional (TWO), High-Dimensional (HIGH), Cross-Cultural High-Dimensional (CULTURE) regression tasks, and the Expressive Burst-Type (TYPE) classification task.
The HUME-VB data contains about 37 hours of vocal burst data from 1702 speakers from China, South Africa, the U.S., and Venezuela. Each vocal burst is labeled with intensities in [1:100] of ten different expressed emotions or category in 8 classes from an average of 85.2 raters. The data is subsequently partitioned into training (19,990 VBs from 571 speakers), validation (19,396 VBs from 568 speakers), and test (19,815 VBs from 563 speakers) splits, with consideration of speaker independence and balances across countries and vocalization types.
% , as shown in Table ~\ref{tab:data}.
% % PTMTorrrent
\newcommand{\numberOfModelHub}{5\xspace}

\newcommand{\TotalNumberOfPackages}{{15,913}\xspace}
% 12401 from Hugging Face
% 185 from ONNX
% 33 from Model Hub
% 3245 from Model Zoo
% 49 from PyTorch Hub
% SUM (by Nick): 15,913

\newcommand{\HFNumberOfPackages}{{12,401}\xspace}
\newcommand{\HFNumberOfPackagesMetadata}{{124,427}\xspace}
\newcommand{\MZNumberOfPackages}{3,245\xspace}
\newcommand{\PHNumberOfPackages}{{49}\xspace}
\newcommand{\MHNumberOfPackages}{{33}\xspace}
\newcommand{\ONNXNumberOfPackages}{{185}\xspace}

\newcommand{\TotalDataSize}{\textasciitilde{61TB}\xspace}
\newcommand{\HFDataSize}{{61TB}\xspace}
\newcommand{\MZDataSize}{{115GB}\xspace}
\newcommand{\PHDataSize}{{1.5GB}\xspace}
\newcommand{\MHDataSize}{{721MB}\xspace}
\newcommand{\ONNXDataSize}{{441MB}\xspace}
%%%



% ICSE submission - HFTorrent v1

\newcommand{\PTMDatasetNPackages}{63,182\xspace}
\newcommand{\PTMDatasetPercentage}{{99.7\%}\xspace}
\newcommand{\PTMDatasetFailedPackages}{{186}\xspace}
\newcommand{\PTMDatasetFailedPercentage}{{0.3\%}\xspace}

\newcommand{\PTMDatasetNReposWithSignedCommits}{{132}\xspace}
\newcommand{\PTMDatasetPercentOfSignedCommits}{{0.208\%}\xspace}


\newcommand{\PercentOfVerifiedOrgs}{{3.188\%}\xspace}
\newcommand{\NOrganizations}{{6,243}\xspace}
\newcommand{\NVerifedOrgs}{{199}\xspace}

\newcommand{\NOfRepositoriesWithMalware}{{1}\xspace}
\newcommand{\PercentageOfRepositoriesWithMalware}{{0.002\%}\xspace}
\newcommand{\TotalRepositoriesForMalwareScanning}{{63,366}\xspace}

In this work, our target tasks are the TWO, HIGH and CULTURE tasks, while the classifications of COUNTRY and TYPE are used as auxiliary tasks. The TWO task aims to predict values of arousal and valence (based on 1=unpleasant/subdued, 5=neutral, 9=pleasant/stimulated), while The HIGH task aims to predict a higher dimension, i.e., the intensity of the aforementioned 10 emotions. The CULTURE task is a 10-dimensional, 4-country culture-specific emotion intensity regression task, i.e., it aims to predict the 40 intensity values of emotion (10 from each culture).

% % PTMTorrrent
\newcommand{\numberOfModelHub}{5\xspace}

\newcommand{\TotalNumberOfPackages}{{15,913}\xspace}
% 12401 from Hugging Face
% 185 from ONNX
% 33 from Model Hub
% 3245 from Model Zoo
% 49 from PyTorch Hub
% SUM (by Nick): 15,913

\newcommand{\HFNumberOfPackages}{{12,401}\xspace}
\newcommand{\HFNumberOfPackagesMetadata}{{124,427}\xspace}
\newcommand{\MZNumberOfPackages}{3,245\xspace}
\newcommand{\PHNumberOfPackages}{{49}\xspace}
\newcommand{\MHNumberOfPackages}{{33}\xspace}
\newcommand{\ONNXNumberOfPackages}{{185}\xspace}

\newcommand{\TotalDataSize}{\textasciitilde{61TB}\xspace}
\newcommand{\HFDataSize}{{61TB}\xspace}
\newcommand{\MZDataSize}{{115GB}\xspace}
\newcommand{\PHDataSize}{{1.5GB}\xspace}
\newcommand{\MHDataSize}{{721MB}\xspace}
\newcommand{\ONNXDataSize}{{441MB}\xspace}
%%%



% ICSE submission - HFTorrent v1

\newcommand{\PTMDatasetNPackages}{63,182\xspace}
\newcommand{\PTMDatasetPercentage}{{99.7\%}\xspace}
\newcommand{\PTMDatasetFailedPackages}{{186}\xspace}
\newcommand{\PTMDatasetFailedPercentage}{{0.3\%}\xspace}

\newcommand{\PTMDatasetNReposWithSignedCommits}{{132}\xspace}
\newcommand{\PTMDatasetPercentOfSignedCommits}{{0.208\%}\xspace}


\newcommand{\PercentOfVerifiedOrgs}{{3.188\%}\xspace}
\newcommand{\NOrganizations}{{6,243}\xspace}
\newcommand{\NVerifedOrgs}{{199}\xspace}

\newcommand{\NOfRepositoriesWithMalware}{{1}\xspace}
\newcommand{\PercentageOfRepositoriesWithMalware}{{0.002\%}\xspace}
\newcommand{\TotalRepositoriesForMalwareScanning}{{63,366}\xspace}
% \vspace{-1em}

\subsection{Experimental Setup}
% \textcolor{red}{Introduce the parameters for preprocessing (feature extraction window length, shift, augmentation parameters), parameters of attentive pooling (width), projection layer structure, shared layer, task-specific layers.
% Baseline systems introduction}
In this work, we set the dimensions of projection and shared layers to 128 and 64, respectively. The task-specific Bi-directional chains consist of two linear layers with sigmoid activation that are concatenated and averaged. The $\lambda$ in Eq.~\ref{eq:loss} is set to 0.9. We use  AdamW~\cite{loshchilov2017decoupled} as our optimizer with a learning rate of $1e-5$ for the Wav2vec 2.0 model finetuning and $1e-3$ for the downstream module training. To obtain a stabler CCC loss and alleviate the variance from the large pretrained model, we train the system with a large batch size of 1024 and a weight decay of $1e-3$. A 0.25 dropout is added between every two modules. We also apply early stopping (patience of 10, maximum of 25 epochs) to avoid overfitting the model.
The systems are evaluated on the validation and test datasets with the averaged biased CCC metric for the target tasks.

\subsection{Baselines}
The baseline systems in this challenge include feature-based and end-to-end methods~\cite{BairdA-VB2022}. The feature-based approach extracts 6,373-dimensional ComParE~\cite{schuller2013interspeech}, 88-dimensional eGeMAPS~\cite{eyben2015geneva} acoustic feature sets, and models the features with three fully-connected layers with layer normalization. While the end-to-end approach uses Emo-18~\cite{tzirakis2018end} convolutional neural networks followed by a 2-layer Long-short term memory (LSTM) network.

\subsection{Experimental Results}
% performance of each task
% Cross Comparison
\section{Results}
\label{results}

\begin{figure*}[ht]
    \centering
    \includegraphics[scale=0.15,trim={0 2.5cm 0 5cm},clip]{images/aoi-single_burst}
    \caption{The time average peak Age of Information with burst and \gls{soa} loss values against the dynamic reliability logic for different network topologies.}
    \label{fig:aoi_burst}\vspace{-0.4cm}
\end{figure*}


This paper focuses on both transport layer and application layer metrics to determine the feasibility of dynamic reliability. For this, we have selected the session packet volume, as transmitted, retransmitted, lost and backlogged packets as \glspl{kpi} for the transport layer; while focusing on the \gls{aoi} for the application layer. The \gls{aoi} was chosen as a crucial indicator for the freshness of packets in real-time applications. More specifically, this work adopts the time average peak \gls{aoi} equation \cite{aoi_equation} depicted in Eq. \ref{aoi}, where $\Delta(r_{i+1})$ is the $i$th update at the time it was received at the server, for a session time period of $\tau$.

\begin{equation}
    \label{aoi}
    \gls{aoi}_\tau = \frac{1}{n-1}\sum_{i=1}^{n-1} \Delta(r_{i+1})
\end{equation}

We include a comparison between the vanilla QUIC implementation which does not enjoy the dynamic reliability extension, with a number of dynamic reliability policies. The tests were run a number of times for statistical significance, with the mean value of vanilla implementation used as a baseline for comparison. The topology utilised both random loss and bursty loss to explore the bounds of dynamic reliability. The \gls{soa} loss in the figures correspond to the loss values presented in Table. \ref{tab:path_char}, for ease of comparison between bursty and random loss scenarios.

\subsection{Transport-Layer KPIs}

To analyse the performance gain at the transport layer due to dynamic reliability, the volume of transmitted and backlogged packets is examined. The figures are in the form of boxplots, which take the vanilla implementation as a benchmark, depicted as the red dashed line.

As seen in Fig. \ref{fig:sent_burst}, the loss plays a crucial role in the performance of the reliability policies. The policies under random loss did incredibly well for the networks with a larger capacity, namely \gls{mmwave} and Sub-6~GHz, whereas for burst loss, the lower network capacities had a larger packet reduction. With the increase in burst loss, the behaviour of the set split reliable policies became unpredictable, if a reliable assignment happened to coincide with a burst loss, the number of transmitted packets increases, and vice versa. On the other hand, in smarter policies, such as Loss-Aware, the performance lightly matched the vanilla baseline, as the reliable assignment dominated the session to compensate for a higher burst loss. Not only that but, the burst loss also impacted the variance of the transmitted packets for the policies.

Unsurprisingly, the unreliable focused policy, 80-20 split, outperformed other policies for all topologies in random and bursty loss scenarios, with an approximate reduction of 80\%. That being said, the majority of the policies reduced the transmitted packets on the link by approximately 70\% for random loss, while the reduction started at $\approx 15\%$ and decreased as the loss increased for the burst loss scenario.

The retransmitted and lost packets, not shown due to space limitations, followed the same trend as the transmitted packets for the random loss scenarios. However, for the burst loss scenarios, the larger capacity networks had a lower reduction in the retransmitted and lost packets. This can be seen as a favorable outcome since the lower capacity networks are scarce on resources. It is important to note that the Loss-Aware policy mimicked the vanilla approach as the burst loss increased, signifying the overwhelming appointment of reliable packets in adapting to the harsh burst loss conditions.
 
Alternatively, Fig. \ref{fig:backlog_burst} clearly shows a stark comparison between the policies and loss scenario in the reduction of the backlogged packets. The Loss-Aware policy for random loss scenario reduced the backlogged packets by up to 50\%, beating all other policies by approximately 30\%. Furthermore, it is clear that the unreliability focused policies resulted in the lowest backlog for the session. In comparison, we notice that the burst loss and the backlogged frequency have a positive correlation, where the maximum reduction of the backlogged packets for the policies is at most 20\%. Much like the transmitted packets, the probability of a burst loss occurrence plays a vital role in the number of retransmissions sent and by extension the number of backlogged packets. Thus, we can conclude that the stress placed on the buffer is a result of the reliable packets which is tightly coupled with the congestion on the session. Whereas, unreliable focused policies did not encounter such a phenomenon regardless if it was experiencing a burst loss.


\subsection{Application-Layer KPIs}

The feasibility of dynamic reliability for real-time applications can be determined by the \gls{aoi}, with comparison across different topologies and policies. If we take a strict approach and consider anything below $10$~ms is real-time \cite{real-time}, then all the reliability policies passed that requirement, which is attractive for real-time applications, as shown in Fig. \ref{fig:aoi_burst}. Utilising the median as an estimate of the runs, the policies in the WLAN and Sub-6~GHz topology with random loss floated around $4-5$~ms with negligible difference, while the \gls{aoi} for \gls{mmwave} was $\approx 2-3$~ms. It is clear that the \gls{aoi} and the network capacity have a negative correlation, as the network capacity decreases, the \gls{aoi} increases. The same correlation is extended to the bursty loss scenarios, where \gls{mmwave} dominated the other topologies. That being said, it is crucial to note that the \gls{aoi} for the reliability policies is often slightly better than or equal to the \gls{aoi} of the vanilla implementation, proving that dynamic reliability reduces the congestion of the session at no cost to the \gls{aoi}.

We compare our system with the baselines on the TWO, HIGH and CULTURE tasks in Table~\ref{tab:results}. It can be found that the proposed system outperforms the baselines on all three tasks by a significant margin. This demonstrates the effectiveness of the proposed hierarchical framework. 

\begin{figure}
       \centering
        \setlength{\tabcolsep}{1pt}
        {\scriptsize
        \begin{tabular}{c c c c c c c }
            { Original } &
            \multicolumn{2}{c}{  } &
            \multicolumn{4}{c}{$\longleftarrow$ Object level variations $\longrightarrow$} \\
            \includegraphics[width=0.185\linewidth]{images/ablation/chair.jpg} &
            \multicolumn{2}{c}{  } &
            \includegraphics[width=0.185\linewidth]{images/ablation/1_only_prompt_mixing/bench.jpg} &
            \includegraphics[width=0.185\linewidth]{images/ablation/1_only_prompt_mixing/stool.jpg} &
            \includegraphics[width=0.185\linewidth]{images/ablation/1_only_prompt_mixing/armchair.jpg} &
            \includegraphics[width=0.185\linewidth]{images/ablation/1_only_prompt_mixing/saddle.jpg} \\
            \multicolumn{3}{c}{  } &
            \multicolumn{4}{c}{ Only Prompt Mixing } \\
            \multicolumn{3}{c}{ } &
            \includegraphics[width=0.185\linewidth]{images/ablation/2_with_self_attn_injection/bench.jpg} &
            \includegraphics[width=0.185\linewidth]{images/ablation/2_with_self_attn_injection/stool.jpg} &
            \includegraphics[width=0.185\linewidth]{images/ablation/2_with_self_attn_injection/armchair.jpg} &
            \includegraphics[width=0.185\linewidth]{images/ablation/2_with_self_attn_injection/saddle.jpg} \\
            \multicolumn{3}{c}{  } &
            \multicolumn{4}{c}{ + Attention-Based Shape Localization } \\
            \multicolumn{3}{c}{ } &
            \includegraphics[width=0.185\linewidth]{images/ablation/3_background_blending/bench.jpg} &
            \includegraphics[width=0.185\linewidth]{images/ablation/3_background_blending/stool.jpg} &
            \includegraphics[width=0.185\linewidth]{images/ablation/3_background_blending/armchair.jpg} &
            \includegraphics[width=0.185\linewidth]{images/ablation/3_background_blending/saddle.jpg} \\
            \multicolumn{3}{c}{  } &
            \multicolumn{4}{c}{ + Controllable Background Preservation } \\
        \end{tabular}
        }
    \vspace{1mm}
    \captionof{figure}{
    Ablating our full object variations pipeline. Original image was crated using the prompt ``A \emph{chair} with a dog on it''. 
    }
    \vspace{-10pt}
    \label{fig:ablation}
\end{figure}

We also conducted experiments to verify the effectiveness of the integrated pre-trained representations and the regression chains on the HIGH task. As shown in Table~\ref{tab:ablation}, when the SSL representations are directly used without further fine-tuning on the HUME-VB dataset, the performance drops from 0.7351 to 0.6103, but still outperforms the baseline systems. If the regression chains are removed, the performance also decreases significantly, which demonstrates the effectiveness of the regression chains for the HIGH task. These results also suggest that the combination of fine-tuned SSL representations that implicitly borrow from external data, and the regression chains that model interactions between emotion classes, are both beneficial for performance.  

\begin{table}[htb]
\centering
% \resizebox{\linewidth}{!}{
\begin{tabular}{l | c c | c }
\toprule
Approach & Valence & Arousal & Average \\
\midrule
Ours     & .7622  & .6309 & .6966 \\
\bottomrule
\end{tabular}
% }
\caption{Performance (CCC) of proposed system for the TWO task on all validation data.}
\label{tab:two}
\end{table}

We further analyze the performance of the arousal and valence prediction in the TWO task. The breakdown of performance is shown in Table~\ref{tab:two}. It can be observed that the CCC of predicted valence values is much higher than that of predicted arousal values. This matches well with the characteristics of the HUME-VB dataset -- that the distribution of human valence annotation is more diffuse than the arousal distribution \cite{baird2022acii}. 

\begin{table}[htb]
\centering
\resizebox{\linewidth}{!}{
\begin{tabular}{l | c c c c c }
\toprule
Approach & Awe & Excite. & Amuse. & Awkward. & Fear \\
Ours & .8169 & .6962 & .7928 & .6085 & .7742 \\
\midrule
Approach & Horror & Distress & Triumph & Sadness & Surprise \\
Ours & .7528 & .7010 & .6914 & .7110 & .8063 \\
\bottomrule
\end{tabular}
}
\caption{Performance (CCC) of the proposed method for the HIGH task on the validation data.}
\label{tab:high}
\end{table}


For the HIGH task, the performances of different emotion classes are shown in Table~\ref{tab:high}. It can be seen that all 10 classes have satisfactory performance.  In particular, the \textit{awkward} class is relatively more difficult with a slightly lower performance, which is also observed in \cite{xin2022exploring}.



% \begin{table*}[htb]
\centering
\resizebox{\linewidth}{!}{
\begin{tabular}{l | c c c c c c c c c c | c | c}
\toprule
Country & Awe & Excite. & Amuse. & Awkward. & Fear & Horror & Distress & Triumph & Sadness & Surprise & Average & Total Average \\
\midrule
China  & .3774 & .6387 & .5171 & .5172 & .6430 & .7590 & .6761 & .6146 & .7021 & .7042 & .6149 & \multirow{4}{*}{.6464} \\
U.S. & .7840 & .7051 & .8387 & .6326 & .7269 & .7092 & .6927 & .7134 & .7396 & .7600 & .7302 \\
South Africa & .6260 & .6917 & .7429 & .5253 & .7041 & .6472 & .5830 & .6342 & .6499 & .7157 & .6520 \\
Venezuela & .7036 & .4383 & .7346 & .5048 & .5992 & .6287 & .4500 & .5477 & .6815 & .5964 & .5885 \\
\bottomrule
\end{tabular}
}
\caption{CCC of the proposed method for the CULTURE task on all validation data.}
\label{tab:culture}
\end{table*}

\begin{table}[htb]
\centering
% \resizebox{\linewidth}{!}{
\begin{tabular}{l | c | c c  }
\toprule
Countries  & Average & Train  & Val.  \\
\midrule
China    & .6149  & 79 & 76  \\
U.S.     & .7302 & 206 & 206 \\
South Africa & .6520 & 244 & 244 \\
Venezuela & .5885 & 42 & 42 \\
\bottomrule
\end{tabular}
% }
\caption{Performance (CCC) of the proposed method for the CULTURE task on the validation dataset. Distribution of recording numbers for the four countries on the training and validation sets is also shown.}
\label{tab:country}
\end{table}
In the CULTURE task, it can be found that the performance for the data from Venezuela is significantly worse than the other locations. This is probably caused by the unbalanced distribution in the dataset.  This is shown in Table~\ref{tab:country}, where the training and validation data for Venezuela is much less compared to the data for U.S. and South Africa. Similarly, the performance for China is also inferior to the those for U.S. and South Africa.
%It can be observed that the classes of \textit{excitement} and \textit{distress} have relatively lower performance.
% points (obs from table)


% \textcolor{red}{labels dependency (figs: type_two, high_corr)}
% insights (label dependency)
% figs/type_two: the relationship between low-dimensional labels (valence, arousal) and part of high-dimensional voc_type labels
% figs/high_corr: Pearson's correlation coefficients among 10 high-dimensional labels

% \textcolor{red}{Ablation Study (-Chain, -Hierarchical)}


