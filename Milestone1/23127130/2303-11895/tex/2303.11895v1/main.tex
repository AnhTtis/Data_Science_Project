\documentclass[12pt, a4paper]{amsart}
\usepackage[utf8]{inputenc}
\usepackage{amsthm}
\usepackage{mathrsfs}  
\usepackage[none]{hyphenat}
\usepackage[english]{babel}
\usepackage{setspace}
\usepackage{longtable}
\usepackage{multirow}
\usepackage[a4paper, top=3cm, bottom=4cm,left=3cm,right=3cm]{geometry}
\usepackage{setspace}
%\usepackage{mathabx}
\usepackage{comment}
\usepackage{caption}
\usepackage{adjustbox}


\usepackage{pgf,tikz}
\usepackage{microtype}


\usepackage{tikz-cd}
\usepackage{tikz,pgfplots}
\usetikzlibrary{positioning, angles}
\usetikzlibrary{arrows}
\usetikzlibrary{decorations.pathmorphing, decorations.pathreplacing}
\usetikzlibrary{decorations.markings}
\usetikzlibrary{calc,shapes}
\usetikzlibrary{decorations.markings}
\usepackage{mathtools}
\usepackage{lipsum}
\usepackage{comment}
\usepackage{float}    
\usetikzlibrary{shapes.geometric}

\let\phi\varphi


\usepackage{amsmath, amsthm, amssymb, amsfonts, xfrac}
\newtheorem{prop}{\textsc{Proposition}}[section]
\newtheorem{cor}[prop]{\textsc{Corollary}}
\newtheorem{thm}[prop]{\textsc{Theorem}}
\newtheorem{conj}[prop]{\textsc{Conjecture}}


\theoremstyle{definition}
\newtheorem{defn}[prop]{\textsc{Definition}}
\newtheorem{ex}[prop]{\textsc{Example}}
\newtheorem{question}[prop]{\textsc{Question}}

\newtheorem{remark}[prop]{\textsc{Remark}}

\newtheorem{lemma}[prop]{\textsc{Lemma}}

\usepackage{graphicx}
\newcommand{\Z}{\mathbb{Z}}
\newcommand{\AC}{\widetilde{\mathcal{AC}}}
\newcommand{\A}{\mathcal{A}^{\Z}}
\newcommand{\Bl}{\mathcal{Bl}^{\Z}}
\newcommand{\Q}{\mathbb{Q}}
\newcommand{\cc}{\mathcal{C}}
\newcommand{\Sy}{\mathcal{S}}
\newcommand{\W}{W}
\newcommand{\qb}{\mathfrak{qb}}
\newcommand{\infn}[1]{||#1||_{\infty}}
\newcommand{\map}{\longrightarrow}
\newcommand{\var}{\left(\M,\A\right)}
\newcommand{\R}{\mathbb{R}}
\newcommand{\F}{\mathbb{F}}
\newcommand{\G}{\mathcal{G}}
\newcommand{\C}{\widetilde{\mathcal{C}}}

\newcommand{\nequiv}{\not\equiv}

\newcommand{\T}{\mathcal{T}}
\DeclareMathOperator{\bd}{b}

\DeclareMathOperator{\Hom}{Hom}
\DeclareMathOperator{\Tors}{Tors}
\DeclareMathOperator{\rank}{rank}

\newcommand{\BL}{L_b}
\newcommand{\QBL}{L_{qb}}
\newcommand{\bg}{\widetilde{bg}}


\DeclareMathOperator{\lk}{lk}
\DeclareMathOperator{\trace}{trace}
\DeclareMathOperator{\disc}{disc}
\DeclareMathOperator{\ch}{char}
\DeclareMathOperator{\genus}{genus}
\DeclareMathOperator{\im}{im}
\DeclareMathOperator{\coker}{coker}
\DeclareMathOperator{\sign}{sign}
\DeclareMathOperator{\Diffeo}{Diffeo}
\DeclareMathOperator{\SO}{SO}
\DeclareMathOperator{\Or}{O}
\DeclareMathOperator{\GL}{GL}
\DeclareMathOperator{\Fix}{Fix}
\DeclareMathOperator{\tr}{tr}
\DeclareMathOperator{\id}{id}
\newcommand{\gnorm}[1]{||#1||_B}
\newcommand{\on}[1]{o^+(#1)}
\newcommand{\om}[1]{o^-(#1)}
\newcommand{\Pm}{\mathcal{P}}

\let\intgrl\int % keep the original definition of `\int`
\let\int\relax % to avoid a "spurious" error message
\DeclareMathOperator{\int}{int}
\DeclareMathOperator{\Int}{Int}

\newcommand{\N}{\mathscr{N}}

%\counterwithin{section}{chapter}
%\counterwithin{figure}{section}


 \newcommand{\bigstark}{\mathop{\Large \mathlarger{\mathlarger{*}}}}
\newcommand{\norm}[1]{||#1||}
\newcommand{\+}{\`}
\DeclareMathOperator{\scl}{scl}
\newcommand{\annotation}[1]{%
  \marginpar{\small\bfseries\color{blue}#1}}
\usepackage{hyperref}

\title[Equivariant algebraic concordance of strongly invertible knots]{Equivariant algebraic concordance\\of strongly invertible knots\\
\scriptsize \textnormal{Preliminary version}}

\author{Alessio Di Prisa}
\address{Scuola Normale Superiore, Pisa, Italy \vskip.05in}
\email{alessio.diprisa@sns.it}

\begin{document}
\maketitle

\begin{abstract}
By considering a particular type of invariant Seifert surfaces we define homomorphism from the equivariant concordance group of directed strongly invertible knots $\C$ to a new equivariant algebraic concordance group $\widetilde{\G}^\Z$.
We prove that this homomorphism lifts both Miller and Powell's equivariant algebraic concordance homomorphism \cite{miller_powell} and Alfieri and Boyle's equivariant signature \cite{alfieri2021strongly}.
We show that this new invariant can obstruct equivariant sliceness for knots with Alexander polynomial one.
Finally, we obtain a new lower bound on the equivariant slice genus of a strongly invertible knot from its equivariant algebraic concordance class.
\end{abstract}

\section*{Introduction}

A knot $K\subset S^3$ is said to be \emph{invertible} if there is an orientation-preserving diffeomorphism $\rho$ of $S^3$ such that $\rho(K)=K$ and $\rho$ reverses the orientation on $K$. If such diffeomorphism can be taken to be an involution, that is $\rho^2=\id$, we say that $K$ is \emph{strongly invertible}.

By the solution of the Smith conjecture \cite{smith} it is known that every finite order element in $\Diffeo(S^3)$ is conjugate to an element of $\Or(4)$.
As a consequence, we can think of a strongly invertible knot as a knot which is invariant under a $\pi$-rotation around some unknotted axis in $S^3$. The knot intersects the axis in two points, by which the axis is separated in two half-axes.

In \cite{sakuma} Sakuma defined a notion of \emph{direction} on a strongly invertible knot, which consists in an orientation of the axis of the involution together with the choice of one of the half-axes. Using this additional structure he was able to define unambiguously an operation of \emph{equivariant connected sum} between \emph{directed strongly invertible knots}.

Since strongly invertible knots are naturally equipped with an involution, it is natural to ask whether a strongly invertible slice knot is also \emph{equivariantly slice}, i.e.~it bounds a slice disk in $B^4$ which is invariant under some extension of the involution.
Similarly to the classical case, this leads to the definition of the \emph{equivariant concordance group} $\C$ as the set of classes of directed strongly invertible knots up to an appropriate definition of \emph{equivariant concordance}.
Recently we proved that $\C$ is not abelian \cite{diprisa}, which is in stark contrast with the classical concordance group.


Several authors \cite{sakuma,boyle2021equivariant,alfieri2021strongly,dai_mallick_stoffregen,miller_powell,diprisa_framba} have found invariants and obstructions for the equivariant concordance of strongly invertible knots.

In particular, in \cite{miller_powell} Miller and Powell introduce a notion of \emph{equivariant algebraic concordance}, by studying the action of the strong inversion on the Blanchfield pairing on the Alexander module of a strongly invertible knot.
In this way they define a homomorphism
$$\Psi:\C\longrightarrow\AC$$
from the equivariant concordance group to an \emph{equivariant algebraic concordance group} $\AC$ of \emph{equivariant Blanchfield pairings}.

On the other hand, in \cite{boyle2021equivariant,alfieri2021strongly} the authors define an equivariant version of the classical knot signature for directed strongly invertible knots, obtaining a group homomorphism $$\widetilde{\sigma}:\C\longrightarrow\Z.$$


In this paper we define a notion of \emph{equivariant algebraic concordance} for directed strongly invertible knots, analogous to Levine's algebraic concordance \cite{Levine1969a,Levine1969b}, by considering a particular \emph{type} of \emph{invariant Seifert surfaces}. In Theorem \ref{en_alg_conc} we construct a homomorphism $\Phi$ from the equivariant concordance group $\C$ to an \emph{equivariant algebraic concordance group} $\widetilde{\G}^\Z$ of \emph{equivariant Seifert systems}.

We prove that the map defined by Miller and Powell \cite{miller_powell} and the equivariant signature defined by Alfieri and Boyle \cite{alfieri2021strongly} factor through $\widetilde{\G}^\Z$ as in the diagram below.
\begin{center}
    \begin{tikzcd}
    &\C\ar[d,"\Phi"]\ar[rd,"\Psi"]\ar[ld,swap,"\widetilde{\sigma}"]&\\
    \Z&\widetilde{\G}^\Z\ar[l]\ar[r]&\AC
    \end{tikzcd}
\end{center}

Moreover, we show that $\Phi$ is sensible to the choice of direction (Remark \ref{remark:direction}) and that it can be used to obstruct equivariant sliceness for knots with Alexander polynomial one (Example \ref{alexander_one}), contrary to Miller and Powell's invariant.


Finally, in Theorem \ref{lower_bound_genus} we obtain a new lower bound on the \emph{equivariant slice genus} of a strongly invertible knot using $\widetilde{\G}^\Z$.





\medskip

\subsection*{Organization of the paper}
In Section \ref{sect:preliminaries} we briefly recall some notions and results on equivariant concordance and on algebraic concordance, and we introduce the definition of \emph{$n$-butterfly link}, which is a generalization of the \emph{butterfly link} \cite{boyle2021equivariant}.
Section \ref{ext_transv} contains some results on the extension and transversality of equivariant maps, that are used in the next section.
In Section \ref{sect:equiv_alg_conc} we use Proposition \ref{alg_slice_link} to motivate the definition of the equivariant algebraic concordance group $\widetilde{\G}^\Z$ and to prove Theorem \ref{lower_bound_genus}.
In Section \ref{sect:equiv_GL} we define a reduction of $\widetilde{\G}^\Z$ to an equivariant version $\widetilde{W}(\Q)$ of the Witt group of $\Q$, and we show that the equivariant signature \cite{alfieri2021strongly} factors through $\widetilde{W}(\Q)$.
Finally, we characterize the image of a directed strongly invertible knot in $\widetilde{W}(\Q)$ in terms of (non-equivariant) rational Witt invariants.


\medskip
\subsection*{Remark} In this preliminary version of the paper we work in the smooth category, unless otherwise specified. However, we believe that it should be possible to adapt most of the results in the topological category. We intend to make this statement clear in a future version of the paper.



\section{Preliminaries}\label{sect:preliminaries}


\subsection{Directed strongly invertible knots}\label{SIK}
We recall the definition of \emph{directed strongly invertible knots} and \emph{equivariant concordance group} following \cite{sakuma,boyle2021equivariant}.

Let $(K,\rho)$ be a strongly invertible knot.
By the resolution of the Smith conjecture \cite{smith} we know that $\rho$ is conjugate to an element of $\SO(4)$ and that it acts on $S^3$ as a rotation around the \emph{axis} $\Fix(\rho)$, which is an unknotted $S^1$. Since the restriction of $\rho$ on $K$ is orientation-reversing, the fixed axis intersects $K$ in two points, which separate $\Fix(\rho)$ in two so called \emph{half-axes}.


\begin{defn}
A \emph{direction} $h$ on a strongly invertible knot $(K,\rho)$ is the choice of one of the half-axes $h$ and an orientation on $\Fix(\rho)$. We say that $(K,\rho,h)$ is a \emph{directed strongly invertible knot}.
\end{defn}

\begin{defn}
We say that two directed strongly invertible knots $(K_i,\rho_i,h_i)$, $i=0,1$ are \emph{equivariantly isotopic} if there exists an orientation-preserving diffeomorphism $\varphi:S^3\longrightarrow S^3$ such that:
\begin{itemize}
    \item $\varphi(K_0)=K_1$,
    \item $\varphi\circ\rho_0=\rho_1\circ\varphi$,
    \item $\varphi(h_0)=h_1$.
\end{itemize}
\end{defn}




We will often omit to specify the choice of strong inversion and direction when it is not strictly necessary to specify them.

\begin{remark}
A direction on $(K,\rho)$ induces an ordering on $K\cap\Fix(\rho)$: we say that the \emph{first fixed point} of $K$ is the initial point of the chosen half-axis, while the final point is the \emph{second fixed point}.
\end{remark}


\begin{defn}
Let $K$ and $J$, be two directed strongly invertible knots.
Their \emph{equivariant connected sum} $K\widetilde{\#}J$ is the directed strongly invertible knot obtained by cutting $K$ at its second fixed point and $J$ at its first fixed point, gluing the two knots and axes equivariantly in a way that is compatible with the orientations on the axes, and choosing the half-axis of the sum to be the union of the half-axes of the two components, as depicted in Figure \ref{equivariant_sum}.
\end{defn}

\begin{figure}[ht]
\centering
\begin{tikzpicture}

\node[anchor=south west,inner sep=0] at (0,0){\includegraphics[scale=0.3]{figures/equivariant_sum.png}};
\node[label={$K$}] at (1.2,2.7){};
\node[label={$\widetilde{\#}$}] at (2.45,2.7){};

\node[label={$J$}] at (3.8,2.7){};
\node[label={$=$}] at (5.5,2.7){};
\node[label={$J$}] at (7.6,3.9){};
\node[label={$K$}] at (7.6,1.4){};

\end{tikzpicture}
    \caption{The equivariant connected sum of $K$ and $J$. The vertical axis (colored red) is the axis of the strong inversion. The chosen half-axis is the solid one.}
    \label{equivariant_sum}
\end{figure}


\begin{defn}
Let $(K,\rho,h)$ be a directed strongly invertible knot. We define
\begin{itemize}
    \item the \emph{mirror} of $(K,\rho,h)$ by $mK=(mK,\rho,h)$,
    \item the \emph{axis-inverse} of $(K,\rho,h)$ by $iK=(K,\rho,-h)$, where $-h$ is the direction given by the half-axis $h$ with the opposite orientation,
    \item the \emph{antipode} of $(K,\rho,h)$ by $aK=(K,\rho,h')$, where $h'$ is the direction given by the oriented half-axis complementary to $h$.
\end{itemize}
\end{defn}


\begin{defn}
Let $(K,\rho)$ be a strongly invertible knot. We say that $K$ is \emph{equivariantly slice} if there exists a slice disk $D\subset B^4$ for $K$, invariant with respect to an involution of $B^4$ extending $\rho$.
We define the \emph{equivariant slice genus} of $(K,\rho)$ as
$$\widetilde{g}_4(K)=\min_{\Sigma}\genus(\Sigma)$$
where $\Sigma$ ranges among the orientable surfaces in $B^4$ with boundary $K$, invariant under an involution extending $\rho$.
\end{defn}

\begin{defn}
We say that two directed strongly invertible knots $(K_i,\rho_i,h_i)$, $i=0,1$ are \emph{equivariantly concordant} if there exists a smooth properly embedded annulus $C\cong S^1\times I\subset S^3\times I$, invariant with respect to some involution $\rho$ of $S^3\times I$ such that:
\begin{itemize}
    \item $\partial (S^3\times I,C)=(S^3,K_0)\sqcup -(S^3,K_1)$,
    \item $\rho$ is in an extension of the strong inversion $\rho_0\sqcup\rho_1$ on $S^3\times 0\sqcup S^3\times 1$,
    \item the orientations of $h_0$ and $-h_1$ induce the same orientation on the annulus $\Fix(\rho)$, and $h_0$ and $h_1$ are contained in the same component of $\Fix(\rho)\setminus C$.
\end{itemize}
\end{defn}



The operation of equivariant connected sum induces a group structure on the set $\C$ of classes of directed strongly invertible knots up to equivariant concordance. The group identity is given by the class of the unknot, while the inverse of $K$ can be represented by $K^{-1}:=m(i(K))$.

Notice that, while the direction is essential to define an equivariant connected sum, $K$ is equivariantly slice if and only if $iK$ or $aK$ is so. Therefore we can consider the mirror, axis-inverse and antipode as involutive maps from the $\C$ to itself.





\subsection{Butterfly links}
In \cite{boyle2021equivariant} Boyle and Issa associate with a directed strongly invertible knot the so called \emph{butterfly link}. Using this link they construct several invariants. In this section we recall some of the invariants defined in \cite{boyle2021equivariant}.
Additionally, we introduce the definition of \emph{$n$-butterfly link} of a directly strongly invertible knot, which is important in the following sections. The $n$-butterfly link is a generalization of the butterfly link and it coincides with the definition in \cite{boyle2021equivariant} for $n=0$.



\begin{defn}\label{butterfly_link}
Let $(K,\rho,h)$ be a directed strongly invertible knot. Take an equivariant band $B$, parallel to the preferred half-axis $h$, which attaches to $K$ at the two fixed points.
Performing a band move on $K$ along $B$ produces a $2$-component link with linking number between components depending on the number of twists of $B$ (see for example Figure \ref{bandmove}).
Observe that $\partial B\setminus K$ consists of two arcs parallel to $h$, which we orient as $h$. The arcs lie in different components of the link and we consider on each component the orientation induced from the respective arc.
The \emph{$n$-butterfly link} $L_b^n(K)$, is the $2$-components $2$-periodic link (i.e. the involution $\rho$ exchanges its components) obtained from such a band move on $K$, so that the linking number between its components is $n$.
\end{defn}

\begin{figure}
    \centering
    \includegraphics[scale=0.5]{figures/figura_otto_band_move.png}
    \caption{The band move (in grey) that produces the $0$-butterfly link of $4_1^+$.}
    \label{bandmove}
\end{figure}


Recall that a \emph{semi-orientation} on a link $L$ is the choice of an orientation on each component of $L$, up to reversing the orientation on all components simultaneously.

\begin{defn}
Define $\widehat{L}_b^n(K)$ to be the $n$-butterfly link of $K$ endowed with the opposite semi-orientation.
Observe that the semi-orientation on $\widehat{L}_b^n(K)$ makes the band move along $B$ coherent with the unique semi-orientation on $K$.
\end{defn}

With a slight abuse of notation we will also call $\widehat{L}_b^n(K)$ the $n$-butterfly link of $K$.
\begin{remark}
Notice that the linking number between the components of $\widehat{L}_b^n(K)$, taken with respect to the chosen semi-orientation, is $-n$.
\end{remark}


\begin{defn}
Let $(L_i,\rho_i)$, $i=0,1$, be two $2$-component $2$-periodic links. We say that $(L_0,\rho_0)$ and $(L_1,\rho_1)$ are \emph{equivariantly concordant} if they bound the disjoint union of two properly embedded smooth annuli in $S^3\times I$, which is invariant with respect to some involution of $S^3\times I$ extending $\rho_0\sqcup\rho_1$.
\end{defn}

\begin{prop}\label{link_concordance}
Let $(K_i,\rho_i,h_i)$, $i=0,1$, be two equivariantly concordant directed strongly invertible knots. Then, $L_b^n(K_0)$ and $L_b^n(K_1)$ are also equivariantly concordant.
\end{prop}
\begin{proof}
The proof is identical to the proof of Proposition 2.6 in \cite{diprisa_framba}.
\end{proof}


\begin{remark}\label{slice_link}
It follows from the proposition above that if $K$ is equivariantly slice then also $\widehat{L}_b^0(K)$ is \emph{equivariantly slice}, i.e. it bounds two disjoint equivariant disks in $B^4$. On the other hand, since $\widehat{L}_b^0(K)$ is obtained by an equivariant band move from $K$ (which can be seen as a genus $0$ equivariant cobordism) we have that if $\widehat{L}_b(K)$ is equivariantly slice then so is $K$.
\end{remark}


In spite of Remark \ref{slice_link}, it is not true in general that if $L_b^0(K)$ is equivariantly concordant to $L_b^0(J)$ then $K$ is equivariantly concordant to $J$.



\begin{prop}[\cite{boyle2021equivariant}]\label{butterfly_homomorphisms}
The following are well defined group homomorphisms.
\begin{itemize}
    \item $\mathfrak{b}:\C\longrightarrow\mathcal{C}$, where $\mathfrak{b}(K)$ is one component of $L_b^0(K)$,
    \item $\qb:\C\longrightarrow\mathcal{C}_{top}$, where $\qb(K)$ the knot $L_b^0(K)/\rho$ in $S^3/\rho\cong S^3$, and $\mathcal{C}_{top}$ is the \emph{topological concordance group}.
\end{itemize}
\end{prop}

\begin{defn}\label{double}
Given an oriented knot $K$, its \emph{double} $\mathfrak{r}(K)$ is the directed strongly invertible knot given by $K\#r(K)$, with the involution $\rho$ that exchanges $K$ and $r(K)$ (the $\pi$-rotation around the vertical axis in Figure \ref{rK}). The direction on $\mathfrak{r}(K)$ is given as follows: the connected sum can be performed by a suitable band move along some band $B$, in grey in the figure, in such a way that $\Fix(\rho)\cap B$ is the half-axis $h$. We orient $h$ as the portion of $B$ lying on $K$ (note that $h$ is parallel to $B\cap K$).
\end{defn}


\begin{figure}[ht]
\centering
\begin{tikzpicture}

\node[anchor=south west,inner sep=0] at (0,0){\includegraphics[scale=0.3]{figures/r_K.png}};
\node[label={$K$}] at (1.2,1.7){};
\node[label={$h$}] at (4.1,1.7){};

\node[label={$r(K)$}] at (7.8,1.7){};

\end{tikzpicture}
    \caption{The directed strongly invertible knot $\mathfrak{r}(K)$. The chosen half-axis is the solid one.}
    \label{rK}
\end{figure}


It is not difficult to check (see \cite{boyle2021equivariant}) that $\mathfrak{r}$ defines a homomorphism
$$\mathfrak{r}:\mathcal{C}\longrightarrow\C.$$

It is immediate from the definitions that given an oriented knot $K$, the $0$-butterfly link of $\mathfrak{r}(K)$ is given by two split copies of $K$ (see again \cite{boyle2021equivariant} for details).
Therefore the composition $\mathfrak{b}\circ\mathfrak{r}:\mathcal{C}\longrightarrow\mathcal{C}$ is the identity homomorphism.

Hence, we get that $\mathfrak{b}$ is surjective, $\mathfrak{r}$ is injective and that $\mathfrak{r}(\cc)$ is a copy of the classical concordance group, contained in the center of $\C$ (as noted in \cite{sakuma}).
As a consequence we observe the following corollary.

\begin{cor}\label{split}
The equivariant concordance group splits as
$$\C=\ker(\mathfrak{b})\oplus\mathfrak{r}(\mathcal{C}).$$
\end{cor}



\subsection{Algebraic concordance}
In \cite{Levine1969a,Levine1969b} Levine defined a surjective homomorphism from the classical concordance group to a Witt group of Seifert forms, called the \emph{algebraic concordance group}, which is given by
$$\phi:\mathcal{C}\longrightarrow\G^{\Z}$$
$$[K]\longmapsto [\theta_F]$$
where $F$ is a Seifert surface for $K$ and $\theta_F$ is the Seifert form on $H_1(F,\Z)$.

Taking the symmetrization of the Seifert form gives rise to a group homomorphism
$$\G^\Z\longrightarrow\W(\Q)$$
$$[A]\longmapsto[A+A^t]$$
where $\W(\Q)$ is the Witt group of non-degenerate symmetric forms on finitely dimensional $\Q$-vector spaces.
Denote by $\phi_W:\mathcal{C}\longrightarrow\W(\Q)$ the composition and by a slight abuse of notation we will also denote by $\phi_W$ the similarly defined homomorphism from the topological concordance group $\cc_{top}$ to $\W(\Q)$.
Clearly by composing $\phi_W$ with the signature homorphism $\sigma:\W(\Q)\longrightarrow\Z$ one get the knot signature $\sigma(K)$.

Given a (possibly nonorientable) spanning surface $F$ for a link $L$, Gordon and Litherland \cite{Gordon1978} defined a bilinear form
$$\G_F:H_1(F,\Z)\times H_1(F,\Z)\longrightarrow \Z$$
$$(a,b)\longmapsto \lk(\widetilde{a},b)$$
given by the linking number of $b$ with $a$ pushed off $F$ ``in both directions simultaneously''. This form is bilinear and symmetric and if $F$ is oriented it coincides with the symmetrization of the Seifert form.






In \cite{Gordon1978} Gordon and Litherland proved that it is possible to compute the signature of a knot from the Gordon-Litherland form of any spanning surface, by introducing a corrective term. We briefly recall some of the notation used in \cite{Gordon1978} and we observe in Proposition \ref{GL_refined} how the results of Gordon and Litherland allow us to compute not only the signature of a knot $K$, but the whole Witt class $\phi_W(K)$, using any spanning surface.
This fact is presumably known to the experts but we could not find it in the literature.

\begin{defn}\label{euler_number}
Let $F$ be a spanning surface for a knot $K$ and let $K^F$ be a longitude of $K$ which misses $F$.
The \emph{relative Euler number of $F$} is defined as $$e(F)=-\lk(K,K^F),$$
where $K$ and $K^F$ are coherently oriented. 
\end{defn}
Observe that since $K^F$ and $F$ are disjoint, $[K]=0\in H_1(S^3\setminus K^F,\Z/2\Z)$. Hence $e(F)$ is always an even integer.



\begin{defn}
Let $F_1$, $F_2$ be two surfaces in $S^3$ with $\partial F_1=\partial F_2$ and suppose that there exists a $3$-ball $B^3=B^1\times B^2\subset S^3\setminus \partial F_i$ such that
\begin{itemize}
    \item $F_1\cap B^3=\partial B^1\times B^2$,
    \item $F_2\cap B^3=B^2\times\partial B^1$,
    \item $F_1\setminus B^3=F_2\setminus B^3$.
\end{itemize}
In this situation we say that $F_2$ is obtained from $F_1$ by a \emph{$1$-handle move}.
\end{defn}

\begin{figure}[ht]
\centering
\begin{tikzpicture}

\node[anchor=south west,inner sep=0] at (0,0){\includegraphics[scale=0.3]{figures/add_half_band.png}};
\node[label={$\varepsilon=+1$}] at (8,4.5){};
\node[label={$\varepsilon=-1$}] at (8,0.9){};
\node[label={$F$}] at (1,2.7){};
\end{tikzpicture}
    \caption{The addition of a half-twisted band.}
    \label{half_band}
\end{figure}

\begin{prop}\label{GL_refined}
Let $F$ be spanning surface for $K$, and $A$ a matrix representing Gordon-Litherland form $\G_F$ on $H_1(F)$.
Then, the Witt class of $K$ is represented by
$$
\begin{pmatrix}
A&0\\
0&\varepsilon Id
\end{pmatrix},
$$
where $\varepsilon=\sign(e(F))$ and the $Id$ block has size $n\times n$ with $n=|e(F)|/2$.
\end{prop}

\begin{proof}
Let $G$ be a Seifert surface for $K$ (in particular $e(G)=0$). By \cite[Theorem 11]{Gordon1978} we can obtain $G$ from $F$ by a finite sequence of the following moves (and their inverses):
\begin{itemize}
    \item ambient isotopy,
    \item $1$-handle moves,
    \item addition of a small half-twisted band at the boundary.
\end{itemize}

It is not difficult to check that the first two moves do not change the Witt class of the Gordon-Litherland form. By attaching a half-twisted band the Gordon-Litherland form and the relative Euler number changes as
$$A\longrightarrow\begin{pmatrix}
A&0\\0&\varepsilon
\end{pmatrix},$$
$$e(F)\longmapsto e(F)-2\varepsilon,$$
where $\varepsilon=\pm1$ depends on the twist of the band, as in Figure \ref{half_band}.
Since the matrix $\begin{pmatrix}
1&0\\
0&-1
\end{pmatrix}$ is metabolic, if we attach two bands with opposite half-twists, the overall move leaves the Witt class unchanged.
The conclusion follows by observing that, up to algebraic cancellation, one has to attach $n=|e(F)|/2$ bands with the same half-twist.


\end{proof}







\section{Extension and transversality of equivariant maps}\label{ext_transv}
In this section we show some results on the extension and trasversality of equivariant maps. Then, we use these results to prove Lemma \ref{3manifold}, which is fundamental for the constructions in Section \ref{sect:equiv_alg_conc}.

Let $X$ be a smooth connected manifold with boundary, such that the inclusion of $\partial X$ in $X$ induces an isomorphism $H^1(X,\Z)\longrightarrow H^1(\partial X,\Z)$. Since $S^1$ is a $K(\Z,1)$, every map $\partial X\longrightarrow S^1$ can be extended to a map $X\longrightarrow S^1$, unique up to homotopy.

Consider now the $\Z/2\Z$-action on $S^1$ given by
$$\iota:S^1\longrightarrow S^1$$
$$z\longmapsto \overline{z},$$
and suppose that $X$ is endowed with a smooth $\Z/2\Z$-action, generated by $\rho\in\Diffeo(X).$

\begin{lemma}\label{equiv_ext}
Let $(X,\partial X,\rho)$ be as above. Let $f:(\partial X,\rho)\longrightarrow (S^1,\iota)$ be an equivariant map, i.e. $f=\iota\circ f\circ\rho$, and suppose there exists $x_0\in \partial X$ such that $f(x_0)=1$. Then $f$ admits an equivariant extension $F:(X,\rho)\longrightarrow (S^1,\iota)$.
\end{lemma}

\begin{proof}
Let $G:X\longrightarrow S^1$ be a (possibly non equivariant) extension of $f$, and define $H:X\longrightarrow S^1$ as $H=G\cdot (\iota\circ G\circ \rho)$, where $\cdot$ is the group operation on $S^1$. By construction $H$ is equivariant.
Since $G$ extends $f$ and $f$ is equivariant we have that $G$ and $\iota\circ G\circ \rho$ induce the same map
$$G_*=(\iota\circ G\circ\rho)_*:H_1(X,\Z)\longrightarrow H_1(S^1,\Z)=\Z.$$
Therefore, $H_*=G_*+(\iota\circ G\circ\rho)_*=2G_*$ and then $H$ can be lifted to the two fold covering.

\begin{center}
    \begin{tikzcd}
    &S^1\ar[d]&z\ar[d]\\
    X\ar[r,"H"]\ar[ru,"F", dashed]&S^1&z^2
    \end{tikzcd}
\end{center}

Choose the lift $F$ such that $F(x_0)=1$. Observe that since $f$ is equivariant, $H_{|\partial X}=f\cdot f$, then $F_{|\partial X}=f$. Therefore we only need to prove that $F$ is equivariant. 
Notice that $F\circ\rho$ is a lift of $H\circ\rho=\iota\circ H$. Therefore we have that $(F\cdot F\cdot (F\circ\rho)\cdot (F\circ\rho))=H\cdot \iota H\equiv 1$ and hence $F\cdot (F\circ\rho)\equiv \pm1$.
Since $F(x_0)=f(x_0)=1$ and $f$ is equivariant, we have that $F(x_0)\cdot F(\rho(x_0))=1$ and since $X$ is connected $F\cdot (F\circ\rho)\equiv 1$, i.e.
$F=\iota\circ F\circ \rho$.
\end{proof}

Suppose now that the action of $\rho$ on $X$ is free. Then we have the following result.
\begin{lemma}\label{equiv_transv}
Let $f:(X,\rho)\longrightarrow (S^1,\iota)$ be an equivariant map and suppose that $1\in S^1$ is a regular value for $f_{|\partial X}$. Then there exists an equivariant homotopy $h:(X\times I,\rho\times\id)\longrightarrow (S^1,\iota)$ such that:
\begin{itemize}
    \item $h(\cdot,0)=f$,
    \item $h(x,t)=f(x)$ for every $x\in\partial X$ and $t\in I$,
    \item $1\in S^1$ is a regular value for $g=h(\cdot,1)$.
\end{itemize}
\end{lemma}

\begin{proof}
Consider the $\Z/2\Z$-action induced by $\rho$ and $\iota$ on the product space
$$\widehat{\rho}:X\times S^1\longrightarrow X\times S^1$$
$$(x,z)\longmapsto(\rho(x),\iota(z)).$$
Observe that there is a bijection between equivariant maps $f:X\longrightarrow S^1$ and equivariant sections of the fiber bundle $X\times S^1\longrightarrow X$, given by
$$f\longmapsto F=(\id_X,f).$$
Moreover, notice $1\in S^1$ is a regular value for $f$ (resp. $f_{|\partial X})$ if and only if the associated section $F$ is transverse to $X\times 1$ (resp. $F_{|\partial X}\pitchfork \partial X\times 1$).
Consider now the quotient spaces of $X$ and $X\times S^1$ by the respective involutions. We have the following commutative diagram
\begin{center}
\begin{tikzcd}
    X\times S^1\ar[d,"q"]\ar[r]&X\ar[d,"p"]\\
    (X\times S^1)/\widehat{\rho}\ar[r]& X/\rho
\end{tikzcd}
\end{center}
where the vertical maps are $2$-fold regular covering maps, while the horizontal ones are fiber bundle maps.
Now, let $F:X\longrightarrow X\times S^1$ be the section associated with $f$. Since $F$ is equivariant, it induces a section on the quotient spaces
$$\overline{F}:X/\rho\longrightarrow (X\times S^1)/\widehat{\rho}.$$
By hypothesis $F_{|\partial X}$ is transverse to $\partial X\times 1$, hence $\overline{F}_{|\partial X/\rho}\pitchfork (\partial X\times 1)/\widehat{\rho}$.
By (non-equivariant) general position arguments we can now find a section $G:X/\rho\longrightarrow (X\times S^1)/\widehat{\rho}$ transverse to $(X\times 1)/\widehat{\rho}$, which is a small perturbation of $\overline{F}$ and homotopic to $\overline{F}$ relative to the boundary.
Since $G\circ p$ is homotopic to $\overline{F}\circ p$ and the latter admit a lift to $X\times S^1$, namely $F$, we can find a lift $\widetilde{G}:X\longrightarrow X\times S^1$, such that $\widetilde{G}_{|\partial X}=F_{\partial X}$.
Given that $G$ is a section of the quotient bundle and the perturbation is small, $\widetilde{G}$ is still a section of $X\times S^1\longrightarrow X$. Moreover since $G$ is transverse to $(X\times 1)/\widehat{\rho}$, we have that $\widetilde{G}\pitchfork X\times 1$.
Finally, observe that $\widetilde{G}\circ\rho$ is the other lift of $G\circ p$ and since the former is not equal to $\widetilde{G}$ (by the hypothesis on the boundary), we have that $\widetilde{G}\circ\rho=\widehat{\rho}\circ\widetilde{G}$, i.e. $\widetilde{G}$ is an equivariant section.
Therefore the corresponding map $g:X\longrightarrow S^1$ given by composing $\widetilde{G}$ with the projection on $S^1$ is a $\Z/2\Z$-equivariant map, homotopic to $f$ relative to the boundary, and $1\in S^1$ is a regular value for $g$.
\end{proof}

\begin{lemma}\label{3manifold}
Let $\rho:B^4\longrightarrow B^4$ be an orientation preserving involution, with fixed point set diffeomorphic to a $2$-disk $D$.
Let $F\subset S^3$ and $\Sigma\subset B^4$ be two oriented surface with $L=\partial F=\partial\overline{\Sigma}$.
Suppose that both $F$ and $\Sigma$ are $\rho$-invariant and that $\rho$ reverses their orientation.
If both $F$ and $\Sigma$ are disjoint from $\Fix(\rho)$, then there exists a $\rho$-invariant, oriented $3$-manifold $M\subset B^4$, disjoint from $\Fix(\rho)$ and such that $\partial M=F\cup\Sigma$.
\end{lemma}

\begin{proof}
Let $Y$ be the complement in $S^3$ of an equivariant tubular neighbourhood of $\partial F\cup\partial D$.
Let $N\cong F\times[-1,1]$ be and equivariant tubular neighbourhood of $F$ in $Y$. Observe that the restriction of $\rho$ acts on $N$ as
$$\rho:N\longrightarrow N$$
$$(x,t)\longmapsto (\rho(x),-t).$$

Let $\phi:\R\longrightarrow\R$ be a smooth, odd map such that $\phi'\geq0$ and
$$\phi(x)=\begin{cases}x\text{ for }|x|\leq1/2\\
\sign(x)\text{ for }|x|\geq2/3\end{cases}.$$
Then we can define an equivariant map
$$f:(N,\rho)\longrightarrow (S^1,\iota)$$
$$(x,t)\longmapsto e^{\pi i\phi(t)}$$
and we can extend it smoothly to $Y$ by setting $f$ to be $-1$ outside $N$.
By construction $1\in S^1$ is a regular value for $f$ and $F=f^{-1}(1)$.

Let $X$ be the complement in $B^4$ of an equivariant tubular neighbourhood of $\Sigma\sqcup D$.
Notice that both these tubular neighbourhood are trivial $D^2$-bundles.
Now $f$ can be equivariantly extended to $\partial X$, as follows.
On the boundary of the tubular neighbourhood of $D$ we set $f$ to be identically $-1$.
On the boundary of the tubular neighbourhood of $\Sigma$, which we equivariantly identify with $(\Sigma\times S^1,\rho\times\iota)$, we set $f$ to be
$$\Sigma\times S^1\longrightarrow S^1$$
$$(x,e^{\pi i t})\longmapsto e^{\pi i \phi(t)},$$
coherently with the definition of $f$ on $Y$.

By Lemma \ref{equiv_ext} and Lemma \ref{equiv_transv} we get that $f$ admits an equivariant extension $\widehat{f}:X\longrightarrow S^1$, with $1\in S^1$ as regular value.
Finally $M=\widehat{f}^{-1}(1)$ is a $\rho$-invariant $3$-manifold with boundary $F\cup\Sigma$.
Observe that by construction $M$ is disjoint from the fixed point set $D$.
\end{proof}



\section{Equivariant algebraic concordance}\label{sect:equiv_alg_conc}
In this section we define an equivariant algebraic concordance group $\widetilde{\G}^\Z$ and a homomorphism $\Phi:\C\longrightarrow\widetilde{\G}^\Z$.
We compare $\widetilde{\G}^\Z$ with the equivariant algebraic concordance group $\AC$ defined in \cite{miller_powell}.
Finally, we use $\widetilde{\G}^\Z$ to obtain a lower bound on the equivariant slice genus of a strongly invertible knot.

\subsection{Equivariant Seifert systems}

\begin{defn}\label{inv_seifert_surface}
Let $(K,\rho,h)$ be a directed strongly invertible knot. An \emph{invariant Seifert surface of type $n$} for $K$ is a connected, orientable surface $F\subset S^3$ such that:
\begin{itemize}
    \item $F$ is $\rho$-invariant i.e. $\rho(F)=F$,
    \item $h=\Fix(\rho)\cap F$,
    \item the surface $\widehat{F}$ obtained from $F$ by equivariantly cutting along $h$ is a $\rho$-invariant Seifert surface for $\widehat{L}_b^n(K)$.
\end{itemize}
\end{defn}

\begin{prop}\label{stab_surface}
For any directed strongly invertible knot $K$ and every $n\in\Z$ there exists a invariant Seifert surface of type $n$.
\end{prop}

\begin{proof}
From \cite{hirasawa_hiura_sakuma} we know that for any $(K,\rho,h)$ there exists a $\rho$-invariant Seifert surface $F$ such that $\Fix(\rho)\cap F=h$.
Cutting $F$ along $h$ we obtain a (possibly disconnected) orientable surface $\widetilde{F}$ and the linking number between the components of $\partial\widetilde{F}$ would not be generally $-n$.
Now let $G$ be the equivariant Seifert surface for the unknot described in Figure \ref{unknot_surface}.

\begin{figure}[ht]
\centering
\begin{tikzpicture}

\node[anchor=south west,inner sep=0] at (0,0){\includegraphics[scale=0.6]{figures/unknot_surface.png}};

\end{tikzpicture}
    \caption{The invariant Seifert surface $G$ for the unknot.}
    \label{unknot_surface}
\end{figure}
Observe that cutting $G$ along the fixed point set we obtain a Seifert surface for a link with linking number $+1$ between its components.
In other words, $G$ is an invariant Seifert surface of type $-1$ for the unknot.
Therefore, by taking the equivariant connected sum of $F$ with an appropriate number of copies of $G$ and/or its mirror image, we easily get an invariant Seifert surface of type $n$ for $K$.

\end{proof}


As a consequence of Lemma \ref{3manifold}, we have the following proposition.

\begin{prop}\label{alg_slice_link}
Let $(K,\rho,h)$ be a directed strongly invertible knot and $F$ be an equivariant Seifert surface for $\widehat{L}_b^n(K)$. Suppose that $\widehat{L}_b^n(K)$ bounds an orientable surface $\Sigma\subset B^4$ invariant under an involution of $B^4$ extending $\rho$ (which we still denote by $\rho$). Assume that $\rho$ has no fixed point on $\Sigma\cup F$. Denote by $g_F$ and $g_{\Sigma}$ the genus of $F$ and $\Sigma$ respectively. Then there exists a $\rho_*$-invariant submodule $H\subset H_1(F,\Z)$ such that:
\begin{itemize}
    \item $\rank H\geq g_F-g_{\Sigma}$ if $\Sigma$ is connected and $\rank H\geq g_F-g_{\Sigma}+1$ if $\Sigma$ is not connected,
    \item the Seifert form of $F$ vanishes on $H$,
    \item for every $\alpha\in H$, the linking number between $\alpha$ and the fixed axis is zero.
\end{itemize}
\end{prop}

\begin{proof}
By Lemma \ref{3manifold} there exists a $\rho$-invariant oriented $3$-manifold $M\subset B^4$, such that $\partial M=F\cup\Sigma$ and $M\cap\Fix(\rho)=\emptyset$.

Denote by $V$ the kernel of
$H_1(\partial M,\Q)\longrightarrow H_1(M,\Q).$ It is easy to see that $2\cdot\dim V=\dim H_1(\partial M,\Q)=\genus(\partial M)$, by standard duality argument (half-lives, half-dies principle) or by computing the Euler characteristic of the exact sequence of the couple $(M,\partial M)$.

Suppose now that $\Sigma$ is connected. Then it is easy to see that $\genus(\partial M)=g_F+g_{\Sigma}+1$ and that the map induced by the inclusion $i_*:H_1(F,\Q)\longrightarrow H_1(\partial M,\Q)$ is injective.
Since $\dim V=g_F+g_{\Sigma}+1$ and $\dim H_1(F,\Q)=g_F+1$, we have that the preimage $W$ of $V$ in $H_1(F,\Q)$ has dimension at least $g_F-g_{\Sigma}$.

Suppose now that $\Sigma$ is not connected. Then $\genus(\partial M)=g_F+g_{\Sigma}$ and the map induced by the inclusion $i_*:H_1(F,\Q)\longrightarrow H_1(\partial M,\Q)$ has kernel of dimension $1$.
Since $\dim V=g_F+g_{\Sigma}$ and $\dim H_1(F,\Q)=g_F+1$, we have that the preimage $W$ of $V$ in $H_1(F,\Q)$ has dimension at least $1+g_F-g_{\Sigma}$.

Let $H\subset H_1(F,\Z)$ be
$$H=\{x\in H_1(F,\Z)\;|\;\exists n\in\Z, n\neq 0, nx\in W\}.$$


It is a well known fact that the Seifert form of $F$ is identically zero on $H$. Since all of the maps considered are equivariant, we get that also $H$ is invariant under the action of $\rho_*$ on $H_1(F,\Z)$.

By the considerations above, the rank $H$ satisfy the inequalities stated in the proposition.

Finally, let $\alpha\in H$ and let $\Delta\subset M$ be a $2$-chain such that $\partial \Delta=n\alpha$ for some integer $n\neq0$. Since $M$ is disjoint from the disk $D$ of fixed points, it follows that $\lk(n\alpha, \partial D)=\#(\Delta\cap D)=0$, hence $\lk(\alpha,\partial D)=0$.
\end{proof}

We use now the result given Proposition \ref{alg_slice_link} to define a notion of equivariant algebraic concordance for directed strongly invertible knots.

\begin{defn}
An \emph{equivariant Seifert system} is a tuple $(\theta,\rho,h,\widetilde{\lk})$, where

\begin{itemize}
\item $\theta:M\times M\longrightarrow\Z$ is a bilinear form on a free $\Z$-module $M$ of even rank,
\item $\rho:M\longrightarrow M$ is a linear involution,
\item $\theta(\rho(x),\rho(y))=\theta^t(x,y):=\theta(y,x)$ for every $x,y\in M$,
\item $\theta-\theta^t$ is unimodular,
\item $h,\widetilde{\lk}\in\Hom(M,\Z)$,
\item $h\circ\rho=-h$,
\item $\widetilde{\lk}\circ\rho=\widetilde{\lk}$.
\end{itemize}

An equivariant Seifert system $(\theta,\rho,h,\widetilde{\lk})$ on $M$ is said to be \emph{equivariantly metabolic} if there exists a submodule $H\subset M$ such that
\begin{itemize}
\item $\rank M=2\cdot\rank H$
    \item $\rho(H)=H$, i.e. $H$ is \emph{$\rho$-invariant},
    \item $\theta$ is identically zero on $H\times H$,
    \item $H\subset \ker(h)\cap\ker(\widetilde{\lk})$.
\end{itemize}
\end{defn}


Let $(K,\rho,h)$ be a directed strongly invertible knot and let $F$ be an invariant Seifert surface for $K$ of type $n$ for some $n$. Fix an auxiliary orientation on $F$.

We see now how $F$ determines an equivariant Seifert system.
Since $\rho$ reverses the orientation on $F$, it is immediate to check that $\theta_F(\rho_*(x),\rho_*(y))=\theta_F(y,x)$ for every $x,y\in H_1(F,\Z)$, where $\theta_F$ is the Seifert form of $F$. Since $h\subset F$, we have that $h$ represents a class in $H_1(F,\partial F,\Z)$. By duality and universal coefficients, we can consider $h$ as a homomorphism $h:H_1(F,\Z)\longrightarrow \Z$, which maps an oriented curve $c$ in $F$ to the algebraic intersection $\#(c\cap h)$.
Finally, let $A$ be the oriented fixed axis of $\rho$. Then we have an homomorphism
$$\widetilde{\lk}:H_1(F,\Z)\longrightarrow \Z$$
$$c\longmapsto \lk(c^+,A)+\lk(c^-,A),$$
where the $c^{\pm}$ is a nearby copy of $c$ outside $F$ in the positive/negative direction.
It is immediate to check that the tuple $(\theta_F,\rho_*,h,\widetilde{\lk})$ is an equivariant Seifert system.
We will denote by $\Sy(F)$ the equivariant Seifert system determined by $F$.


\begin{defn}\label{orth_sum}
Let $(\theta_i,\rho_i,h_i,\widetilde{lk}_i)$ for $i=1,2$ be two equivariant Seifert systems defined over $M$ and $N$ respectively. Their \emph{orthogonal sum} $(\theta_1,\rho_1,h_1,\widetilde{lk}_1)\oplus(\theta_2,\rho_2,h_2,\widetilde{lk}_2)$ is the tuple $(\theta,\rho,h,\widetilde{lk})$ defined by
$$\theta:(M\oplus N)\times(M\oplus N)\longrightarrow \Z$$
$$((x_1,x_2),(y_1,y_2))\longmapsto \theta_1(x_1,y_1)+\theta_2(x_2,y_2)$$
$$\rho:M\oplus N\longrightarrow M\oplus N$$
$$\rho(x,y)=(\rho_1(x),\rho_2(y))$$
$$h,\widetilde{lk}:M\oplus N\longrightarrow\Z$$
$$h(x,y)=h_1(x)+h_2(y)$$
$$\widetilde{\lk}(x,y)=\widetilde{\lk}_1(x)+\widetilde{\lk}_2(y).$$
We say that $(\theta_i,\rho_i,h_i,\widetilde{lk}_i)$, $i=1,2$ are \emph{equivariantly concordant} if
the orthogonal sum between $(\theta_1,\rho_1,h_1,\widetilde{lk}_1)$ and $(-\theta_2^t,\rho_2,h_2,\widetilde{lk}_2)$ is equivariantly metabolic.
\end{defn}

\begin{defn}\label{equiv_alg_conc}
We define the \emph{equivariant algebraic concordance group} $\widetilde{\G}^{\Z}$ to be the set of equivalence classes of equivariant Seifert systems up to equivariant concordance. It is not difficult prove that the operation of orthogonal sum defines a group structure on $\widetilde{\G}^{\Z}$, by adapting the proof of Levine \cite{Levine1969a,Levine1969b} in the case of the classical algebraic concordance.
\end{defn}


\begin{thm}\label{en_alg_conc}
Let $F\subset S^3$ be an invariant Seifert surface of type $0$ for a directed strongly invertible knot $(K,\rho,h)$ and choose an orientation on $F$. The class of the equivariant Seifert system $\Sy(F)$ in $\widetilde{\G}^{\Z}$ depends only on the equivariant concordance class of $K$.
In particular we have a well defined group homomorphism
$$\Phi:\C\longrightarrow\widetilde{\G}^{\Z}$$
$$[K,\rho,h]\longmapsto[\Sy(F)].$$
\end{thm}

\begin{proof}
Let $G$ be an invariant surface $G$ of type $0$ for another directed strongly invertible knot $J$. We can equivariantly perform the connected sum of $F$ and $G$ along their boundary so that $F\natural G$ is a invariant Seifert surface of type $0$ for the $K\widetilde{\#}J$. It is immediate to see that $\Sy(F\natural G)=\Sy(F)\oplus\Sy(G)$.
Therefore, in order to prove that the homomorphism $\Phi$ is well defined it is sufficient to show that $\Sy(F)$ is equivariantly metabolic whenever the knot $K=\partial F$ is equivariantly slice.


Let $\widehat{F}$ be the equivariant Seifert surface for $\widehat{L}_b^0(K)$ obtained by cutting $F$.
By Proposition \ref{alg_slice_link} there exists a $\rho_*$-invariant submodule $H$ of $H_1(\widehat{F},\Z)$, such that $2\rank H=\rank H_1(\widehat{F},\Z)+1$ and the Seifert form of $\widehat{F}$ vanishes on it.

Observe that we can regard $H_1(\widehat{F},\Z)$ as a $\rho_*$-invariant codimension $1$ submodule in $H_1(F,\Z)$ through the map induced by the inclusion. Moreover, $H_1(\widehat{F},\Z)$ is easily identified with the kernel of $h:H_1(F,\Z)\longrightarrow\Z$.
The restriction of the Seifert form of $F$ on $H_1(\widehat{F},\Z)$ clearly coincides with the Seifert form of $\widehat{F}$.
Again by Proposition \ref{alg_slice_link}, the linking number homomorphism $\widetilde{\lk}:H_1(F,\Z)\longrightarrow\Z$ vanishes on $H$.
Therefore $H$ is an equivariant metabolizer for the equivariant Seifert system $\Sy(F)$.
\end{proof}


\begin{remark}
Let $G$ the equivariant Seifert surface of type $-1$ for the unknot described in Figure \ref{unknot_surface}.
Let $F$ be an invariant Seifert surface of type $n$ for a directed strongly invertible knot $(K,\rho,h)$.
Then by the proof of Theorem \ref{en_alg_conc} and Proposition \ref{stab_surface} follows easily that we can compute the equivariant algebraic concordance class of $K$ by
$$\Phi(K)=[\Sy(F)+n\Sy(G)]\in\widetilde{\G}^\Z.$$

Similarly, observe that in order to compute $\Phi(K)$ it is not relevant in Definition \ref{inv_seifert_surface} that $F\setminus h$ is connected, since we can replace $F$ with $F\natural G\natural\overline{G}$ without changing the concordance class of its equivariant Seifert system.
\end{remark}

\begin{prop}
Let $\mathcal{A}$ be the concordance group of algebraically slice knots, i.e. the kernel of $\phi:\cc\longrightarrow\mathcal{G}^\Z$.
Then the kernel of $\Phi:\C\longrightarrow\widetilde{\G}^\Z$ contains a copy of $\mathcal{A}$, namely $\mathfrak{r}(\mathcal{A})\subset\ker(\Phi)$.
\end{prop}

\begin{proof}
Let $K$ be an oriented knot representing a class in $\mathcal{A}$ and let $F$ be a Seifert surface for $K$. Then we can compute $\Phi(\mathfrak{r}(K))$ using as invariant surface $\mathfrak{r}(F)=F\natural r(F)$, where analogously to Definition \ref{double}, the involution of $\mathfrak{r}(K)$ exchange $F$ and $r(F)$.


Identifying $H_1(\mathfrak{r}(F),\Z)\cong H_1(F,\Z)\oplus H_1(F,\Z)$, it is not difficult to see that the equivariant Seifert system of $\mathfrak{r}(F)$ is of type $\Sy(\mathfrak{r}(F))=(\theta,\rho,0,0)$, where
$$\theta=\begin{pmatrix}
\theta_F&0\\
0&\theta_F^t
\end{pmatrix}$$
$$\rho=\begin{pmatrix}
0&\id\\
\id&0
\end{pmatrix}.$$
Therefore, if $H\subset H_1(F,\Z)$ is a metabolizer of $\theta_F$ then $H\oplus H\subset H_1(\mathfrak{r}(F),\Z)$ is an equivariant metabolizer for $\Sy(\mathfrak{r}(F))$.

Since $\mathfrak{r}$ is injective (see Corollary \ref{split}), we have that $\ker(\Phi)$ contains a copy of $\mathcal{A}$.

\end{proof}

\begin{remark}
As a consequence, it follows from \cite{jiang1981simple} that $\ker(\Phi)$ contains a subgroup isomorphic to $\Z^\infty$, and from \cite{livingston1999order} that it contains a subgroup isomorphic to $\Z_2^\infty$.
\end{remark}



\subsection{Equivariant Blanchfield pairing}
In this section we show that $\Phi:\C\longrightarrow \widetilde{\G}^\Z$ which lifts the homomorphism $\Psi:\C\longrightarrow\widetilde{\mathcal{AC}}$ defined in \cite{miller_powell}, passing through a \emph{reduced} version $\widetilde{\G}_r^\Z$ of the equivariant algebraic concordance group.
Then we show that $\widetilde{\G}^\Z$ is able to obstruct equivariant sliceness for strongly invertible knot with trivial Alexander polynomial, contrary to $\AC$.

\begin{defn}\label{reduced}
An \emph{equivariant Seifert form} is a couple $(\theta,\rho)$, where

\begin{itemize}
\item $\theta:M\times M\longrightarrow\Z$ is a bilinear form on a free $\Z$-module $M$ of even rank,
\item $\rho:M\longrightarrow M$ is a linear involution,
\item $\theta(\rho(x),\rho(y))=\theta^t(x,y):=\theta(y,x)$ for every $x,y\in M$,
\item $\theta-\theta^t$ is unimodular.
\end{itemize}

An equivariant Seifert form $(\theta,\rho)$ on $M$ is said to be \emph{equivariantly metabolic} if there exists a submodule $H\subset M$ such that
\begin{itemize}
\item $\rank M=2\rank H$
    \item $\rho(H)=H$, i.e. $H$ is \emph{$\rho$-invariant},
    \item $\theta$ is identically zero on $H\times H$.
\end{itemize}
\end{defn}



Similarly to Definition \ref{orth_sum} and \ref{equiv_alg_conc} we can define a notion of orthogonal sum between equivariant Seifert forms and construct the \emph{reduced equivariant algebraic concordance group} $\widetilde{\G}_r^{\Z}$ as the set of equivalence classes of equivariant Seifert forms up to equivariant concordance.

Clearly exists a forgetful map
$$r:\widetilde{\G}^{\Z}\longrightarrow \widetilde{\G}_r^\Z,$$
which is obviously surjective, since it admits a natural section
$$s:\widetilde{\G}_r^\Z\longrightarrow\widetilde{\G}^\Z$$
given by mapping an equivariant Seifert form $(\theta,\rho)$ to the equivariant Seifert systems $(\theta,\rho,0,0)$.
In particular $\widetilde{\G}^\Z$ splits as
$$\widetilde{\G}^\Z\cong \widetilde{\G}_r^\Z\oplus\ker(r).$$

We will denote by $\Phi_r$ the map given by the composition $$\Phi_r:\C\xrightarrow{\Phi}\widetilde{\G}^\Z\xrightarrow{r}\widetilde{\G}_r^\Z.$$

Levine \cite{Levine1969a,Levine1969b} showed that the algebraic concordance group is isomorphic to $\Z^{\infty}\oplus\Z_2^{\infty}\oplus\Z_4^{\infty}$ and that the knot concordance group surjects onto it.
We do not know if $\Phi$ is surjective or the isomorphism type of $\widetilde{\G}^\Z$.
In the future we intend to investigate this questions, and the analogous ones for $\widetilde{\G}_r^\Z$.


In \cite{miller_powell} Miller and Powell study the action of the strong inversion $\rho$ of a strongly invertible knot $K$ on its Alexander module $\A(K)$ and on the Blanchfield pairing on $\A(K)$.
In particular they show that the action induced by $\rho$ on $\A(K)$ is an anti-isometry of the Blanchfield pairing (Proposition 2.8). Moreover, they define an \emph{equivariant algebraic concordance group} $\AC$ as the Witt group of \emph{abstract equivariant Blanchfield pairings} (Definition 4.3) and they prove that taking the Blanchfield form of a strongly invertible knot $(K,\rho)$ together with the involution on $\A(K)$ induced by $\rho$ defines a homomorphism (Proposition 4.6)
$$\Psi:\C\longrightarrow\AC.$$


In \cite{friedl2017calculation} the authors proves that the Alexander module and the Blanchfield pairing of $K$ can be expressed in terms of the Seifert form of a Seifert surface $F$ as follows: if $A$ is a matrix representing the Seifert form of $F$, with respect to a basis $\mathcal{B}$, and $\genus(F)=g$ then $\A(K)\cong\Z[t^{\pm1}]^{2g}/(tA-A^t)\Z[t^{\pm1}]^{2g}$ and under this identification the Seifert pairing is equivalent to
$$\mathcal{BL}:\A(K)\times\A(K)\longrightarrow\Q(t)/\Z[t^{\pm1}]$$
$$(x,y)\longmapsto x(t-1)(A-tA^t)^{-1}\overline{y}$$
where $\overline{\cdot}$ is the $\Z$-linear involution given by $t\longmapsto t^{-1}$.
It is not difficult to see, as pointed out in the examples in \cite{miller_powell}, that if $F$ is $\rho$-invariant and $P$ is the matrix representing the action of $\rho$ on $H_1(F,\Z)$ with respect to $\mathcal{B}$, we have that the action of $\rho$ on $\A(K)$ can be read as
$$\rho_*:\A(K)\longrightarrow\A(K)$$
$$x\longmapsto P\overline{x}.$$

The same construction carried out for abstract equivariant Seifert forms and abstract equivariant Blanchfield pairings shows that there exists a natural group homomorphism
$$\widetilde{\G}_r^\Z\longrightarrow\AC$$
that makes the following diagram commutative
\begin{center}
    \begin{tikzcd}
    \C\ar[r,"\Phi"]\ar[rrd,swap,"\Psi"]&\widetilde{\G}^\Z\ar[r,"r"]&\widetilde{\G}_r^\Z\ar[d]\\
    &&\AC.
    \end{tikzcd}
\end{center}


In the following example we show that $\widetilde{\G}^\Z$ is able to obstruct equivariant sliceness also for strongly invertible knot with trivial Alexander polynomial, contrary to $\AC$.


\begin{ex}\label{ex:alexander_one}
Consider the knot $K13n1496$ as a the directed strongly invertible knot $(K,\rho,h)$ that bounds the surface $F$ in Figure \ref{alexander_one}, where the strong inversion is given by the $\pi$-rotation around the vertical axis and the chosen oriented half-axis is the red one in the figure.


\begin{figure}[ht]
\centering
\begin{tikzpicture}

\node[anchor=south west,inner sep=0] at (0,0){\includegraphics[scale=0.5]{figures/alexander_one.pdf}};
\node[label={$\alpha$}] at (6.1,6){};
\node[label={$\beta$}] at (7.2,7.3){};
\node[label={$\gamma$}] at (7.5,3){};
\node[label={$\delta$}] at (2.6,1){};
\node[label={$h$}] at (4.7,3.75){};
\end{tikzpicture}
    \caption{The invariant surface $F$ with boundary $K13n1496$. The chosen half-axis $h$ is the solid one. The curves $\alpha,\beta,\gamma,\delta$ form a basis of $H_1(F,\Z)$.}
    \label{alexander_one}
\end{figure}

One can easily check that $K$ has trivial Alexander polynomial and hence that its image is trivial image in the equivariant algebraic concordance group $\AC$ defined in \cite{miller_powell}. However, Boyle and Issa \cite{boyle2021equivariant} prove that $K$ is not equivariantly slice. We show that the same result can be obtained by using $\widetilde{\G}^\Z$.

The surface $F$ is an invariant Seifert surface for $K$, so we can use it to compute the class of $K$ in $\widetilde{\G}^{\Z}$.
With respect to the basis $\{\alpha,\beta,\gamma,\delta\}$ of $H_1(F,\Z)$ the Seifert form and the involution $\rho_*$ are represented by the matrices $A$ and $P$ respectively:
$$A=\begin{pmatrix}
-1 & 1 & 0 & 0\\
0 & -1 & -1 & 0\\
0 & -1 & -1 & 0\\
0 & 0 & -1 & 1
\end{pmatrix}$$
$$P=\begin{pmatrix}
1 & 1 & 0 & 0\\
0 & -1 & 0 & 0\\
0 & 0 & -1 & 0\\
0 & 0 & 1 & 1
\end{pmatrix}.$$
The homomorphisms $h$ and $\widetilde{\lk}$ are represented by the covectors:
$$h=\begin{pmatrix}
0 & -1 & 1 & 0
\end{pmatrix}$$
$$\widetilde{\lk}=\begin{pmatrix}
-2 & -1 & 1 & 2
\end{pmatrix}.$$
One can easily check that $H=\langle \alpha-\delta, \beta-\gamma\rangle$ is a $\rho_*$-invariant submodule of rank $2$ on which the Seifert form of $F$ vanishes.

Therefore the class of $K$ represents the identity also in the reduced equivariant algebraic concordance group $\widetilde{\G}_r^\Z$.

However, $H=\ker(h)\cap\ker(\widetilde{\lk})=\langle \beta+\gamma, \alpha+\delta\rangle$ has rank $2$ but the Seifert form do not vanishes on $H$, therefore the class of $K$ is nontrivial in $\widetilde{\G}^{\Z}$.
\end{ex}

It follows from its definition that $\AC$ do not distinguish a directed strongly invertible knot from its antipode.
On the other hand, in Section \ref{sect:equiv_GL} we prove that the equivariant signature \cite{alfieri2021strongly} can be retrieved from $\widetilde{\G}_r^\Z$. Since the equivariant signature is sensible to the choice of half-axis for a strongly invertible knot, so is $\widetilde{\G}_r^\Z$ (see Remark \ref{equiv_signature_direction}).


\subsection{Lower bound on the equivariant slice genus}
In \cite{miller_powell} the authors obtain a lower bound on the equivariant slice genus of a strongly invertible knot using the Blanchfield form.
Since Miller and Powell's invariant factors through $\widetilde{\G}^\Z$, we can get the same lower bound indirectly.
However, we prove in this section that it is possible to obtain a different lower bound on the equivariant slice genus using $\widetilde{\G}^\Z$.

\begin{defn}\label{equiv_complx}
Let $\Sigma\subset B^4$ be a properly embedded orientable surface, with boundary a strongly invertible knot $(K,\rho)$. Suppose that $\Sigma$ is invariant under an involution of $B^4$ which extends $\rho$, and denote by $D\cong D^2$ the fixed point set of $\rho$ in $B^4$.
Then intersection $\Sigma\cap D$ consists on an arc joining the two fixed points on $K$ and finite set $\Gamma$ of fixed $S^1$.
We define the \emph{complexity} of $\Sigma$ as
$$c(\Sigma)=\genus(\Sigma)+\#\Gamma.$$
Then, we define the \emph{slice complexity} of a strongly invertible knot $(K,\rho)$ as
$$c_4(K,\rho)=\min_{\Sigma} c(\Sigma),$$
where $\Sigma$ ranges among the orientable surfaces in $B^4$ with boundary $K$, invariant under an involution of $B^4$ extending $\rho$.
\end{defn}

\begin{remark}
By Smith theory (see \cite{Bredon1972IntroductionTC}), we have that $\#\Gamma\leq\genus(\Sigma)$. Therefore, for every strongly invertible knot $(K,\rho)$ the follwing inequalities hold
$$\widetilde{g}_4(K)\leq c_4(K)\leq 2\cdot\widetilde{g}_4(K).$$
\end{remark}


Let $(K,\rho,h)$ be a directed strongly invertible knot, and let $\Sigma\subset B^4$ an invariant surface for $K$ as in Definition \ref{equiv_complx}.
Denote by $D$ the fixed point set in $B^4$, oriented compatibily with the half-axis $h$.
Observe that $D\setminus \Sigma$ can be subdivided in two subsurfaces in a checkerboard fashion, as described in Figure \ref{orientation_fixed_circles}.


\begin{figure}[ht]
\centering
\begin{tikzpicture}

\node[anchor=south west,inner sep=0] at (0,0){\includegraphics[scale=0.5]{figures/orientation_fixed_circles.pdf}};
\node[label={$h$}] at (8.5,7.5){};
\node[label={$\alpha$}] at (4.8,3.8){};
\end{tikzpicture}
    \caption{An example of how to get the orientation of the fixed point set of $\Sigma$.}
    \label{orientation_fixed_circles}
\end{figure}

Let $S$ be the subsurface containing the chosen half-axis $h$, and orient every $\gamma\in\Gamma$ and the fixed arc $\alpha$ as the boundary of $S$.

Let $D\times D^2$ be an equivariant tubular neighbourhood of $D$ in $B^4$.
Pick an auxiliary orientation on $\Sigma$ and observe that $\Sigma$ induces on every $\gamma$ a nowhere vanishing section $s_\gamma$ of $D\times D^2$, which we can regard as a map $s_\gamma:\gamma\cong S^1\longrightarrow S^1$.
We call the degree of $s_\gamma$ the \emph{framing} $f(\gamma)\in\Z$ of $\gamma$.
It is easy to see that it does not depend on the auxiliary orientation on $\Sigma$.

Similarly, $\Sigma$ induces on $\alpha$ a nowhere zero section of $D\times D^2$, which we complete to a section on $\alpha\cup h$ using the section induced on $h$ by a band $B\subset S^3$ which gives the $0$-butterfly link of $K$ (see Definition \ref{butterfly_link}).
Call the degree of the associated map $S^1\longrightarrow S^1$ the \emph{framing} $f(\alpha)$ of $\alpha$.

Finally, we say that $\Sigma$ is an \emph{invariant surface type $n$} for $(K,\rho,h)$, where
$$n=f(\alpha)+\sum_{\gamma\in\Gamma}f(\gamma).$$

\begin{prop}\label{remove_fix_pt}
Let $\Sigma\subset B^4$ be an invariant surface of type $n$ for a directed strongly invertible knot $(K,\rho,h)$. Then, there exists an invariant oriented surface $\widehat{\Sigma}\subset B^4$ with boundary $\widehat{L}_b^n(K)$, with no fixed points and such that
$$\genus(\widehat{\Sigma})\leq\begin{cases}c(\Sigma)-1\text{ if } \widehat{\Sigma} \text{ is connected,}\\
c(\Sigma)\text{ if } \widehat{\Sigma} \text{ is not connected}.
\end{cases}$$
\end{prop}

\begin{proof}
On the set of fixed circles $X=\Gamma\cup\{\alpha\cup h\}$ consider the partial order given by the nesting of circles, seen as circles in the fixed disk $D$.
First of all, we want to remove all of the fixed circles. We do so by applying two moves.

\textbf{Move 1}: Suppose there exists a minimal element $\gamma\in\Gamma\subset X$ with framing zero. Let $D_{\gamma}\subset D$ be the disk bounded by $\gamma$. Since $f(\gamma)=0$ the section induced by $\Sigma$ on $\gamma$ of the equivariant tubular neighbourhood of $D$ extends over $D_{\gamma}$ to a nowhere vanishing section, which we can take to be equivariant. Therefore, we can perform an equivariant surgery of $\Sigma$ along $D_{\gamma}$, obtaining a surface $\Sigma'$ of the same type, with less genus and fixed circles.
Replace $\Sigma$ by $\Sigma'$.

\textbf{Move 2}: Let $\gamma\in\Gamma$ be a minimal element with $f(\gamma)\neq 0$ and let $\xi\in X$ be a circle such that there exists an arc $\beta\subset D\setminus\Sigma$ joining $\gamma$ and $\xi$.
Then, we can find an equivariant $D^1\times D^2$ inside an equivariant tubular neighbourhood $D^1\times D^3$ of $\beta$ such that $(\partial D^1)\times D^2\subset \Sigma$. We perform an equivariant surgery along $D^1\times D^2$, obtaining a surface $\Sigma'$ with $\genus(\Sigma')=\genus(\Sigma)+1$. One can check that the circles $\gamma$ and $\xi$ were joined during the surgery into a new fixed circle with framing $f(\gamma)+f(\xi)$. Therefore $\Sigma'$ has the same type of $\Sigma$. Replace $\Sigma$ by $\Sigma'$.

\begin{figure}[ht]
\centering
\begin{tikzpicture}

\node[anchor=south west,inner sep=0] at (0,0){\includegraphics[scale=0.5]{figures/order_fixed_circles.pdf}};
\node[label={$h$}] at (8.5,7.5){};
\node[label={$\alpha$}] at (4.8,3.8){};
\end{tikzpicture}
    \caption{An example of choice of the arcs $\beta$, in blue.}
    \label{1surgery}
\end{figure}

Applying Move 1 whenever possible and Move 2 in the other cases, we get an invariant surface $\Sigma'$ of type $n$ with fixed point set consisting of only one arc.
Observe that we have to apply Move 2 at most $\#\Gamma$ times.
As in Remark \ref{slice_link} we can consider the equivariant band move on $K$ along $h$ that gives $\widehat{L}_b^n(K)$ as an equivariant cobordism $C$ between $K$ and $\widehat{L}_b^n(K)$.
Now glue together $C$ and $\Sigma'$ along $K$, obtaining a surface $\Sigma''$, with $\genus(\Sigma'')\leq c(\Sigma)$.
By construction, the fixed point set of the involution on $\Sigma''$ consists of a single circle, with framing induced by $\Sigma''$ equal to zero. Finally, apply Move 1, obtaining an invariant surface $\widehat{\Sigma}$ with boundary $\widehat{L}_b^n(K)$ and without fixed points.
Observe that if the final Move 1 does not disconnect the surface, then $\genus(\widehat{\Sigma})=\genus(\Sigma'')-1\leq c(\Sigma)-1$. Otherwise $\genus(\widehat{\Sigma})=\genus(\Sigma'')\leq c(\Sigma)$.
\end{proof}


\begin{defn}
Let $\Sy=(\theta,\rho,h,\widetilde{lk})$ be an equivariant Seifert system on $M\cong\Z^{2m}$.
A \emph{partial metabolizer} for $\Sy$ is a $\rho$-invariant submodule $H\subset \ker(h)\cap\ker(\widetilde{lk})$ such that $\theta_{H\times H}\equiv 0$.
We define the \emph{algebraic complexity} of $\Sy$ as $ac(\Sy)=m-k$, where
$$k=\max\{\rank(H)\,|\, H \text{ is a partial metabolizer of } \Sy\}.$$
\end{defn}
We do not prove it in this version of the paper, but the algebraic complexity is constant on the equivariant algebraic concordance classes. Therefore it induce a well defined map $ac:\widetilde{\G}^{\Z}\longrightarrow\mathbb{N}$.




\begin{thm}\label{lower_bound_genus}
Let $(K,\rho,h)$ be a directed strongly invertible knot and let $\Phi(K)\in\widetilde{\G}^\Z$ be its equivariant algebraic concordance class.
Then the following inequality holds
$$2\widetilde{g}_4(K)\geq c_4(K)\geq\min_{n\in\Z}ac(\Phi(K)+n\Sy(G)),$$
where $G$ is the invariant Seifert surface of type $-1$ for the unknot in Figure \ref{unknot_surface}.
\end{thm}

\begin{proof}
Let $\Sigma\subset B^4$ be any invariant orientable surface with boundary $K$ and let $n$ be the type of $\Sigma$.
Let $\widehat{\Sigma}\subset B^4$ be the invariant orientable surface with boundary $\widehat{L}_b^n(K)$ and no fixed points obtained from $\Sigma$ by Proposition \ref{remove_fix_pt}.
Take now a an invariant Seifert surface $F\subset S^3$ of type $n$ for $K$.
By Proposition \ref{alg_slice_link} there exists a partial metabolizer $H\subset H_1(F,\Z)$, with $\rank H\geq g_F-c(\Sigma)$.
Therefore $c(\Sigma)\geq g_F-\rank H\geq ac(\Sy(F))\geq\min_{n\in\Z}ac(\Phi(K)+n\Sy(G))$.
Taking the minimum over $\Sigma$, we get
$$c_4(K)\geq\min_{n\in\Z}ac(\Phi(K)+n\Sy(G)).$$
\end{proof}


\begin{remark}
Since the equivariant slice genus and the slice complexity do not depend on the choice of the direction, one can replace $(K,\rho,h)$ by its antipode $(K,\rho,h')$ in Theorem \ref{lower_bound_genus} to obtain a (potentially) better lower bound.
\end{remark}





\section{Equivariant Gordon-Litherland form}\label{sect:equiv_GL}
In this section we define a homomorphism from $\mathcal{\G}_r^\Z$ to a simpler group $\widetilde{\W}(\Q)$ of algebraic concordance, namely an equivariant version of the Witt group of $\Q$.
Then we give a characterization of the image of a directed strongly invertible knot in $\widetilde{\W}(\Q)$ in terms of classical Witt invariants. Finally, we prove that the equivariant signature defined in \cite{alfieri2021strongly} factors through $\widetilde{\W}(\Q)$.

\begin{defn}
Let $\F$ be a field. An \emph{equivariant symmetric form} is a pair $(Q,\rho)$ where $Q$ is a symmetric, bilinear and non-degenerate form on a finite dimensional $\F$-vector space $V$ and $\rho$ is a $Q$-isometric involution of $V$.
We say that $(Q,\rho)$ is \emph{equivariantly metabolic} if $\dim V$ is even and there exists a half-dimensional $\rho$-invariant subspace $W\subset V$ such that $Q_{|W\times W}\equiv 0$.
\end{defn}

\iffalse
\begin{defn}
Let $(Q_1,\rho_1),(Q_2,\rho_2)$ be equivariant symmetric forms defined respectively on $V$, $W$. Their \emph{orthogonal sum} $(Q,\rho)=(Q_1,\rho_1)\oplus(Q_2,\rho_2)$ is defined as
$$Q:(V\oplus W)\times(V\oplus W)\longrightarrow \F$$
$$((x_1,x_2),(y_1,y_2))\longmapsto Q_1(x_1,y_1)+Q_2(x_2,y_2)$$
$$\rho:V\oplus W\longrightarrow V\oplus W$$
$$\rho(x,y)=(\rho_1(x),\rho_2(y)).$$
We say that $(Q_1,\rho_1),(Q_2,\rho_2)$ are \emph{equivariantly concordant} if $(Q_1,\rho_1)\oplus(-Q_2,\rho_2)$ is equivariantly metabolic.
\end{defn}
\fi


Again, analogously to Definition \ref{orth_sum} and \ref{equiv_alg_conc} we can define a notion of orthogonal sum and concordance between equivariant symmetric forms and define the \emph{equivariant Witt group} $\widetilde{\W}(\F)$ of $\F$ to be the set of equivalence classes of equivariant symmetric forms up to equivariant concordance.


Given an equivariant Seifert form $(\theta,\rho)$ defined over a $\Z$-module $M$, we can define an equivariant symmetric form on $M\otimes_{\Z}\Q$ by $(\theta+\theta^t,\rho)$.

It is immediate to see that this association induces a group homomorphism
$$\widetilde{\mathcal{G}}_r^\Z\longrightarrow\widetilde{\W}(\Q).$$
Denote by $\Phi_W:\C\longrightarrow \widetilde{\W}(\Q)$ the map given by the composition

$$\C\longrightarrow\widetilde{\mathcal{G}}_r^\Z\longrightarrow\widetilde{\W}(\Q).$$

Notice that the map $\Phi_W$ maps the equivariant concordance class of a directed strongly invertible knot $(K,\rho,h)$, to the Witt class of the couple $(\G_F,\rho_*)$, where $F$ is an invariant Seifert surface of type $0$ for $K$ and $(\G_F,\rho_*)$ is the couple given by the Gordon-Litherland form on $H_1(F,\Q)$ and the action induced by $\rho$.


\subsection{A characterization of the equivariant Witt class}
Now we show that given a directed strongly invertible knot $K$, the equivariant Witt class $\Phi_W(K)$ depends only on the (classical) Witt class of $K$ and $\mathfrak{qb}(K)$ (see Definition \ref{butterfly_homomorphisms}).

\begin{remark}
Let $(Q,\rho)$ be an equivariant form over $\Q$ and let $E_{\lambda}$ be the $\lambda$-eigenspace of $\rho$, for $\lambda=\pm1$. Given $v\in E_1$, $w\in E_{-1}$ clearly
$$Q(v,w)=Q(v,-w)=-Q(v,w)\Longrightarrow Q(v,w)=0.$$
Hence, $E_{1}$ and $E_{-1}$ are orthogonal and we can decompose the form as
$$(Q,\rho)=(Q_{|E_1},\id)\oplus(Q_{|E_{-1}},-\id).$$
This gives us an isomophism $$(\pi_+,\pi_-):\widetilde{\W}(\Q)\longrightarrow\W(\Q)\oplus\W(\Q)$$
$$[Q,\rho]\longrightarrow ([Q_{|E_1}],[Q_{|E_{-1}}]).$$
We will denote by $\Phi^{\pm}_W=\pi_{\pm}\circ\Phi_W:\C\longrightarrow\W(\Q)$ the induced homomorphisms. 
\end{remark}

Using the description above of $\widetilde{\W}(\Q)$ we are able to give a new definition of the equivariant signature.
\begin{defn}\label{equiv_sign_witt}
Denote by $\sigma:\W(\Q)\longrightarrow\Z$ the signature homomorphism.
Define the \emph{equivariant signature} as $\widetilde{\sigma}=(\sigma\circ\Phi^-_W -\sigma\circ\Phi^+_W):\C\longrightarrow\Z$.
\end{defn}


We show now how the invariants we just defined are related to some of the invariants defined in \cite{boyle2021equivariant} and \cite{alfieri2021strongly}.


\begin{defn}
Let $A$ be a non-degenerate symmetric $n\times n$ matrix and let $k$ be non zero integer. Define $M_k(A)=k\cdot A$.
Clearly if $A$ is metabolic, $M_k(A)$ is so. Moreover $M_k(A)\oplus M_k(B)=M_k(A\oplus B)$. Therefore, this induces a well-defined homomorphism
$$M_k:\W(\Q)\longrightarrow\W(\Q)$$
$$[A]\longmapsto [M_k(A)].$$
It is immediate to see that $M_k\circ M_k$ is the identity, hence $M_k$ is an isomorphism.
\end{defn}

\begin{lemma}\label{euler_quotient}
Let $F$ be an invariant Seifert surface of type $0$ for a directed strongly invertible knot $(K,\rho,h)$ and let $\widetilde{F}$ be the corresponding Seifert surface for $\widehat{L}_b^0(K)$.
Then the quotient surface $\overline{F}=\widetilde{F}/\rho\subset \overline{S^3}=S^3/\rho$ is a spanning surface for $\qb(K)$ with zero relative Euler number $e(\overline{F})$ (see Definition \ref{euler_number}).
\end{lemma}
\begin{proof}
Let pick a representative of the semi-orientation on $\widehat{L}_b^0(K)$ and denote the components of the link by $H$ and $J$.
Let $H^{\widetilde{F}}$ be a nearby longitude of $H$ missing $\widetilde{F}$.
Then the projection $\pi(H^{\widetilde{F}})$ is a longitude of $\qb(K)$ missing $\overline{F}$.
In order to show that $e(\overline{F})=\lk(\qb(K),\pi(H^{\widetilde{F}}))=0$ it is sufficient to prove that $[H^{\widetilde{F}}]=0\in H_1(S^3\setminus H,\Z)$.
Since $H^{\widetilde{F}}$ is disjoint from $\widetilde{F}$, we have that $\lk(H^{\widetilde{F}},H)+\lk(H^{\widetilde{F}},J)=0$.
By definition of $0$-butterfly link we have that $\lk(H^{\widetilde{F}},J)=\lk(H,J)=0$, therefore $\lk(H^{\widetilde{F}},H)=0$.
In other words $[H^{\widetilde{F}}]=0\in H_1(S^3\setminus H,\Z)$.
\end{proof}

\begin{prop}
Let $(K,\rho,h)$ be a directed strongly invertible knot. Then
$$\Phi_W^+(K)=M_2(\phi_W(\mathfrak{qb}(K)),$$
where $\phi_W(\mathfrak{qb}(K))$ is the Witt class of $\mathfrak{qb}(K)$.
\end{prop}
\begin{proof}
Let $F$ be an invariant Seifert surface of type $0$ for $K$ and let $\widetilde{F}$ be the corresponding Seifert surface for $\widehat{L}_b^0(K)$. First of all, observe that $F$ can be obtained from $\widetilde{F}$ by attaching an equivariant band $B$. Since $\rho$ reverses the orientation of the core of $B$, it is not difficult to see that the dimension of the $(-1)$-eigenspace of $\rho_*$ increases by one going from $H_1(\widetilde{F},\Q)$ to $H_1(F,\Q)$. Hence, the $1$-eigenspace of $\rho_*$ is fully contained in $H_1(\widetilde{F})$. Let $\pi:(S^3,\widetilde{F})\longrightarrow(\overline{S^3},\overline{F})$ be the quotient projection, given by the action of $\rho$. The quotient surface $\overline{F}$ is a spanning surface for $\qb(K)$.
Observe that the quotient projection $\pi$ is a $2$-fold covering $\widetilde{F}\longrightarrow \overline{F}$. Take now an oriented curve $c$ in $\overline{F}$, representing a class in $H_1(\overline{F})$, and lift it to a class $\tr(c)=\pi^{-1}(c)\in H_1(\widetilde{F})$.
This defines a transfer homomorphism (see \cite{Bredon1972IntroductionTC} for details) $\tr:H_1(\overline{F})\longrightarrow H_1(\widetilde{F})$. By construction $\rho_*(\tr(c))=\tr(c)$, i.e. the image of the transfer map is contained in the $1$-eigenspace $E_1$ of $\rho_*$. The composition $\pi_*\circ\tr$ is given by $$\pi_*\circ \tr:H_1(\overline{F})\longrightarrow H_1(\overline{F})$$
$$c\longmapsto 2c,$$
hence $\tr$ is injective. Moreover $\tr$ is clearly surjective on $E_1$: given a $\rho$-invariant class $d\in E_1$, we can project it by $\pi_*$ and lift it again, showing that $2d\in \im(\tr)$.
Finally, we show that the transfer map behaves well with respect to the Gordon-Litherland form. Given $c,d\in H_1(\overline{F})$ let $S$ be an oriented surface in $\overline{S^3}$ with $\partial S=\widetilde{d}$, where $\widetilde{d}$ is $d$ pushed out of $\overline{F}$ ``in both directions simultaneously'' (as in the definition of the Gordon-Litherlan form). In this way we have that $\G_{\overline{F}}(c,d)=\lk(c,\widetilde{d})=\#S\cap c$. Up to a small isotopy we can suppose $S$ transverse to the branching locus. The lift $\pi^{-1}(S)$ is an oriented surface in $S^3$ with boundary $\widetilde{\tr(d)}$ and we can use it to calculate $$\lk(\tr(c),\widetilde{\tr(d)})=\#(\pi^{-1}(S)\cap\tr(c))=2(\#S\cap c).$$
It follows that the Gordon-Litherland forms are related by $$\G_{\widetilde{F}}(\tr(c),\tr(d))=2\cdot\G_{\overline{F}}(c,d).$$
Therefore the transfer map gives an isometry
$$\tr:(H_1(\overline{F}),2\cdot\G_{\overline{F}})\longrightarrow (E_1,\G_{\widetilde{F}}).$$
Finally, by Lemma \ref{euler_quotient} we have that $e(\overline{F})=0$ and hence by Proposition \ref{GL_refined} that the Gordon-Litherland form on $\overline{F}$ represents the Witt class of $\qb(K)$.
\end{proof}


As an immediate consequence we get the following corollary.
\begin{cor}\label{qb_phi}
Let $K$ be a directed strongly invertible knot and let $A$ and $B$ be symmetric matrices representing the (non-equivariant) Witt classes of $\mathfrak{qb}(K)$ and $K$ respectively.
Then the equivariant Witt class of $K$ is represented by the couple
$$\Phi_W(K)=\left[\begin{pmatrix}
2A&0&0\\
0&-2A&0\\
0&0&B
\end{pmatrix}, \begin{pmatrix}
\id&0&0\\
0&-\id&0\\
0&0&-\id
\end{pmatrix}\right].$$
\end{cor}



\subsection{The equivariant signature}\label{equiv_sign_sect}
We recall now the definition of equivariant signature introduced by Alfieri and Boyle \cite{alfieri2021strongly} and we prove that it is equivalent to the one in Definition \ref{equiv_sign_witt}.

\medskip
Given a knot $K\subset S^3$ we denote by $\Sigma(K)$ the $2$-fold cover of $S^3$ branched over $K$.
Given a properly embedded and connected surface $F\subset B^4$, we denote by $\Sigma(F)$ the $2$-fold cover of $B^4$ branched over $F$, and by $\tau$ the covering transformation of $\Sigma(F)$.


\begin{lemma}[Proposition 12 \cite{boyle2021equivariant}]\label{lift}
Let $\rho$ be an orientation preserving involution of $B^4$ such that $\Fix(\rho)$ is a $2$-disk $D$. Let $F\subset B^4$ be a properly embedded and connected $\rho$-invariant surface on which $\rho$ acts non-trivially. Then, there exists a lift $\widetilde{\rho}$ of $\rho$, i.e. the following diagram commutes
\begin{center}
\begin{tikzcd}

\Sigma(F)\ar[r,"\widetilde{\rho}"]\ar[d,"\pi"]&\Sigma(F)\ar[d,"\pi"]\\
B^4\ar[r,"\rho"]&B^4.
\end{tikzcd}
\end{center}
In fact, there exists exactly two such lifts, namely $\widetilde{\rho}$ and $\tau\widetilde{\rho}$.
\end{lemma}


Now let $(K,\rho,h)$ be a directed, strongly invertible knot and let $F\subset B^4$ be a properly embedded connected surface with $\partial F=K$ (not necessarily orientable), invariant with respect to some extension of $\rho$ to $B^4$ (which we still denote by $\rho$).


Before introducing the equivariant signature it is useful to better describe the fixed point set of the lifts of $\rho$ given by Lemma \ref{lift}. We do so in the following remark
\begin{remark}\label{fixed_pt_lift}
Let $D$ be the fixed point disk of $B^4$. The intersection $D\cap F=\Fix(\rho_{|F})$ is the disjoint union of an arc joining the fixed point of $K$ and a finite number of $S^1$ and isolated points.
Take $x\in\Fix(\widetilde{\rho})$ and observe that $\rho\circ\pi(x)=\pi\circ\widetilde{\rho}(x)=\pi(x)$, therefore $\Fix(\widetilde{\rho})\subseteq\pi^{-1}(D)$.
Moreover, note that $\widetilde{\rho}\circ\tau(x)=\tau\circ\widetilde{\rho}(x)=\tau(x)$, i.e. $\Fix(\widetilde{\rho})$ is $\tau$-invariant.
Take now $x\in \pi^{-1}(F\cap D)$. Then $\widetilde{\rho}(x)\in\pi^{-1}(\rho\circ\pi(x))=\{x,\tau(x)\}$ and since $\tau(x)=x$ we have that $x\in\Fix(\widetilde{\rho})$, i.e. $\pi^{-1}(F\cap D)$ is fixed pointwise by $\widetilde{\rho}$.
Let $C_1,\dots,C_n$ the connected components of $D\setminus F$. Since $\rho_{|C_i}$ is the identity, $\widetilde{\rho}$ and $\tau\circ\widetilde{\rho}$ act either as the identity or as $\tau$ on $\pi^{-1}(C_i)$. Therefore, exactly one of the lifts fixes pointwise the preimage of $C_i$, while the other one has no fixed point in $\pi^{-1}(C_i)$.
Let $C_i$ and $C_j$ be \emph{adjacent} components, i.e. separated by a circle or an arc in $D\cap F$. Then, $\pi^{-1}(C_i)$ and $\pi^{-1}(C_j)$ cannot be both fixed pointwise by $\widetilde{\rho}$. Otherwise, $\overline{\pi^{-1}(C_i)}$ and $\overline{\pi^{-1}(C_j)}$ would be two fixed surfaces, both contained in $\Fix(\widetilde{\rho})$ and intersecting in a non-trivial way in their interior and this would imply that $\Fix(\widetilde{\rho})$ has a component which is not a manifold.
Therefore, if we decompose $D\setminus F=A\sqcup B$, where $A$ and $B$ are union of non-adjacent components, we have that the fixed point sets of the two lifts of $\rho$ are respectively $\pi^{-1}(\overline{A})$ and $\pi^{-1}(\overline{B})$.
\end{remark}


Observe that by Remark \ref{fixed_pt_lift} exactly one lift $\widetilde{\rho}$ of $\rho$ to $\Sigma(F)$ fixes pointwise $\widetilde{h}=\pi^{-1}(h)$.
The fixed point set of $\widetilde{\rho}$ is the disjoint union of a (eventually disconnected) surface $\Delta$, with $\partial\Delta=\widetilde{h}$, and a finite set of points.
Recall now that the $0$-butterfly link $L_b^0(K)$ is obtained by performing a band move on $K$ along a band parallel to $h$, in such a way that the linking number between the components of $L_b^0(K)$ is zero. Let $\gamma$ be one of the arcs of this band parallel to $h$. Since the endpoints of $\gamma$ meet the branching set, its preimage $\widetilde{\gamma}$ in $\Sigma(F)$ is a closed curve.
Given a perturbation $\Delta'$ of $\Delta$ with $\partial\Delta'=\widetilde{\gamma}$ we define the \emph{relative Euler number} $e(\Delta,\widetilde{\gamma})$ as the algebraic intersection $\#(\Delta\cap\Delta')$.

\begin{defn}[\cite{alfieri2021strongly}]\label{equiv_signature}
The \emph{equivariant signature} of $(K,\rho,h)$ is defined as
$$\widetilde{\sigma}(K)=\sigma(\Sigma(F),\widetilde{\rho})-e(\Delta,\widetilde{\gamma}),$$
where $\sigma(\Sigma(F),\widetilde{\rho})$ is the $g$-signature (see \cite{alfieri2021strongly} or \cite{Gordon1986}) of the pair $(\Sigma(F),\widetilde{\rho})$.
\end{defn}

Using the $G$-signature Theorem \cite{Gordon1986}, Alfieri and Boyle prove that the equivariant signature is a well-defined invariant for equivariant concordance and in particular defines a homomorphism
$$\widetilde{\sigma}:\C\longrightarrow\Z.$$


\begin{remark}
Actually, Alfieri and Boyle \cite{alfieri2021strongly} define the equivariant signature slightly differently, exchanging the role of the two half-axes $h$ and $h'$. It is immediate to check that our definition of equivariant signature for the directed strongly invertible knot $(K,\rho,h)$ coincides with their definition for the antipode $(K,\rho,h')=a(K,\rho,h)$.
Hence the two invariants are essentially the same. However, it is easier to relate Definition \ref{equiv_signature} to the equivariant algebraic concordance group (see Proposition \ref{equivalent_definition}).
\end{remark}


In \cite[Section 6]{alfieri2021strongly} the authors explain how to easily compute the relative Euler number for the equivariant pushoff of a spanning surface in $B^4$.
Using the following proposition it is possible to easily compute the equivariant signature from an equivariant spanning surface.
\begin{prop}[\cite{boyle2021equivariant}, Proposition 13]\label{lattice_isom_equiv}
Let $(K,\rho,h)$ be a directed strongly invertible knot in $S^3$. Let $F$ be a connected spanning surface for $K$, with $\rho(F)=F$.
We still denote by $\rho$ the radial extension of the involution to $B^4$. Let $\widehat{F}$ be the surface obtained by equivariantly pushing the interior of $F$ in $B^4$ and denote by $\widetilde{\rho}$ the preferred lift of $\rho$ to $\Sigma(\widehat{F})$.
Then under the identification $(H_1(F),\G_F)\cong (H_2(\Sigma(\widehat{F})),Q)$ the map of lattices $\widetilde{\rho}_*:(H_2(\Sigma(\widehat{F})),Q)\longrightarrow(H_2(\Sigma(\widehat{F})),Q)$ is equivalent to:
\begin{itemize}
    \item $\rho_*:(H_1(F),\G_F)\longrightarrow(H_1(F),\G_F)$ if $h\not\subset F$,
    \item $-\rho_*:(H_1(F),\G_F)\longrightarrow(H_1(F),\G_F)$ if $h\subset F$,
\end{itemize}
\end{prop}



\begin{prop}\label{equivalent_definition}
The equivariant signature introduced in Definition \ref{equiv_sign_witt} coincides with the one given in Definition \ref{equiv_signature}.
\end{prop}
\begin{proof}
Let $F$ be an invariant Seifert surface of type $0$ for a directed strongly invertible knot $(K,\rho,h)$. According to Definition \ref{equivalent_definition}, $\widetilde{\sigma}(K)$ is the equivariant signature of $(H_1(F),\G_F,-\rho_*)$. By Lemma \ref{lift} and Proposition \ref{lattice_isom_equiv} this quantity coincides with the $g$-signature of the pair $(\Sigma(\widehat{F}),\widetilde{\rho})$, where $\Sigma(\widehat{F})$ is the $2$-fold cyclic cover branched over a copy $\widehat{F}$ of $F$ radially pushed into $B^4$ and $\widetilde{\rho}$ is the preferred lift of the radial extension of $\rho$ to $B^4$. Hence, it is sufficient to prove that the relative Euler number vanishes.
Let $\gamma$ be a parallel copy of $h$ on $F$. Since cutting $F$ along $h$ produces an equivariant Seifert surface for $\widehat{L}_b^0(K)$, the lift $\widetilde{\gamma}$ of $\gamma$ in $\Sigma(\widehat{F})$ is the canonical longitude of $\widetilde{h}$. 
Let $D,D'$ be the traces of $h$ and $\gamma$ respectively along the radial isotopy that pushes the interior of $F$ in $B^4$.
Since $\widehat{F}$ is obtained from $F\subset S^3$, one can see that $\Fix(\rho_{\widehat{F}})$ consists solely on an arc joining the fixed points of $K$.


Then by Remark \ref{fixed_pt_lift} the fixed point set of $\widetilde{\rho}$ consists of the lift $\Delta$ of $D$. The lift $\Delta'$ of $D'$ is a perturbation of $\Delta$ such that $\partial \Delta'=\widetilde{\gamma}$, and since they are disjoint we have $$e(\Delta,\widetilde{\gamma})=\#(\Delta\cap\Delta')=0$$
i.e. the relative Euler number vanishes.
\end{proof}


\begin{remark}\label{remark:direction}
As a consequence of Proposition \ref{equivalent_definition} and Corollary \ref{qb_phi} we obtain the following formula
$$\widetilde{\sigma}(K)=\sigma(K)-2\sigma(\mathfrak{qb}(K))$$
for the equivariant signature of a directed strongly invertible knot $K$ in terms of classical signatures.
\end{remark}

\begin{remark}\label{equiv_signature_direction}

As shown by Alfieri and Boyle \cite[Proposition 7.3]{alfieri2021strongly}, the equivariant signature is sensible to the choice of the half-axis for a strongly invertible knot. For example, they show in the proof of Proposition 7.2 that $\widetilde{\sigma}(7_4b^+\widetilde{\#}m7_4b^-)\neq 0$.

Since the equivariant algebraic concordance homomorphism $\Psi:\C\longrightarrow\AC$ defined in \cite{miller_powell} does not distinguish the choice of half-axis, we get that $7_4b^+\widetilde{\#}m7_4b^-$ has trivial image in $\AC$.

On the other hand, since $\widetilde{\sigma}$ factors through $\widetilde{\G}_r^\Z$, we get that the image of $7_4b^+\widetilde{\#}m7_4b^-$ is non trivial in $\widetilde{\G}_r^\Z$.

\end{remark}



\bibliographystyle{alpha} % We choose the "plain" reference style
\bibliography{refs} % Entries are in the refs.bib file

\end{document}
