In this section, we present additional results of unseen identities and results of sparse views.

\subsubsection{Unseen Identities}
Our method has the ability to adapt to new individuals as the pre-trained template serves as a good initialization. 
To verify this, we consider 3 \textbf{new} identities (Models 552, 555 and 598) and each identity is associated with only 5 views. 

We adopted the pre-trained template, i.e., the one trained on 30 identities the PR-Senior and PR-Young datasets with 10 views for each identity, to learn the fine details for each identity in Stage 2.
We observed that our method also produced plausible results for the unseen identities as shown in Table~\ref{tab:unseen_sparse}. This demonstrates that the pre-trained template can adapt to new identities.

\subsubsection{Small Dataset}

We also conducted another experiment using the 3 unseen identities as a small dataset. In the second experiment, we trained a \textbf{new} template using only the 15 images of the 3 new identities in Stage 1 and then used the template to learn the fine details for each identity in Stage 2.
Without a surprise, the geometry of the newly trained template is worse than that of the pre-trained template due to significantly fewer views involved in Stage 1 training. Still, our method produced a fairly good result and none of the other methods, VolSDF, NeuS and HF-NeuS, were able to reconstruct satisfactory geometry with only 5 views as input as illustrated in Figure~\ref{fig:unseen_result_train} and Figure~\ref{fig:unseen_result_novel}. 

\begin{table*}[htbp]
\setlength\tabcolsep{1pt}
\scriptsize
    \centering
    \begin{tabular}{c|ccc|ccc|ccc|ccc|ccc}
    \hline
        \multirow{2}{*}{Model} & \multicolumn{3}{c|}{NeuS} & \multicolumn{3}{c|}{HF-NeuS} & \multicolumn{3}{c|}{VolSDF} & \multicolumn{3}{c|}{Ours(Template on the 3 ids)} & \multicolumn{3}{c}{Ours(Template on 30 ids)}\\
        \cmidrule[0.5pt](rl){2-16}
        &CD ($10^{-4}$) & PSNR$_\text{t}$ & PSNR$_\text{n}$ 
        &CD ($10^{-4}$) & PSNR$_\text{t}$ & PSNR$_\text{n}$ 
        &CD ($10^{-4}$) & PSNR$_\text{t}$ & PSNR$_\text{n}$
        &CD ($10^{-4}$) & PSNR$_\text{t}$ & PSNR$_\text{n}$
        &CD ($10^{-4}$) & PSNR$_\text{t}$ & PSNR$_\text{n}$\\
        \hline
        552 & 3.769&35.22&21.64&N.A.&35.40&13.36& 8.192 &33.62&23.13&1.815 &33.58&26.51& 1.197 &34.92&25.88 \\
        555 & 3.614&35.37&18.97&N.A.&35.65&12.19&243.4&33.45&11.80&1.254 &33.30&24.14& 1.071&35.04&23.39\\
        598 & 16.25&36.39&21.89&N.A.&36.30&12.81&23.76&35.63&20.27&1.056 &35.70&27.02& 1.020&35.50&27.39\\
        \hline
    \end{tabular}
    \caption{Performance on three unseen identities under 5 views. N.A. indicates no results successfully reconstructed.}
    \label{tab:unseen_sparse}
\end{table*}
\begin{figure*}[htbp]
    \centering
    \rotatebox{90}{\textbf{training view}}
    \includegraphics[width=0.18\textwidth]{./results/sp_view_5view/gt_383_15.jpg}
    \includegraphics[width=0.18\textwidth]{./results/sp_view_5view/neus_5_383_15_render.jpg}
    \includegraphics[width=0.18\textwidth]{./results/sp_view_5view/hfs_5_383_15_render.jpg}
    \includegraphics[width=0.18\textwidth]{./results/sp_view_5view/volsdf_5_383_15_render.jpg}
    \includegraphics[width=0.18\textwidth]{./results/sp_view_5view/ours_5_383_15_render.jpg}\\
    \rotatebox{90}{}
    \includegraphics[width=0.18\textwidth]{./results/template_effects/gt_571_blank.jpg}
    \includegraphics[width=0.18\textwidth]{./results/sp_view_5view/neus_5_383_15_normal.jpg}
    \includegraphics[width=0.18\textwidth]{./results/sp_view_5view/hfs_5_383_15_normal.jpg}
    \includegraphics[width=0.18\textwidth]{./results/sp_view_5view/volsdf_5_383_15_normal.jpg}
    \includegraphics[width=0.18\textwidth]{./results/sp_view_5view/ours_5_383_15_normal.jpg}\\
    \rotatebox{90}{\textbf{novel view}}
    \includegraphics[width=0.18\textwidth]{./results/sp_view_5view/gt_383_12.jpg}
    \includegraphics[width=0.18\textwidth]{./results/sp_view_5view/neus_5_383_12_render.jpg}
    \includegraphics[width=0.18\textwidth]{./results/sp_view_5view/hfs_5_383_12_render.jpg}
    \includegraphics[width=0.18\textwidth]{./results/sp_view_5view/volsdf_5_383_12_render.jpg}
    \includegraphics[width=0.18\textwidth]{./results/sp_view_5view/ours_5_383_12_render.jpg}\\
    \rotatebox{90}{}
    \includegraphics[width=0.18\textwidth]{./results/template_effects/gt_571_blank.jpg}
    \includegraphics[width=0.18\textwidth]{./results/sp_view_5view/neus_5_383_12_normal.jpg}
    \includegraphics[width=0.18\textwidth]{./results/sp_view_5view/hfs_5_383_12_normal.jpg}
    \includegraphics[width=0.18\textwidth]{./results/sp_view_5view/volsdf_5_383_12_normal.jpg}
    \includegraphics[width=0.18\textwidth]{./results/sp_view_5view/ours_5_383_12_normal.jpg}\\
    \makebox[0.18\textwidth]{\scriptsize GT}
    \makebox[0.18\textwidth]{\scriptsize NeuS}
    \makebox[0.18\textwidth]{\scriptsize HF-NeuS}
    \makebox[0.18\textwidth]{\scriptsize VolSDF}
    \makebox[0.18\textwidth]{\scriptsize Ours}
    \caption{Results for Model 383 with only 5 views as input. The template human head was trained using 5 randomly selected views for all 30 identities of the PR-Senior and PR-Young datasets. The images of Model 383 for Stage 1 training and Stage 2 training are the same, therefore no additional views were provided. }
    \label{fig:sparse_view_old}
\end{figure*}

\begin{figure*}
    \centering
    \rotatebox{90}{\textbf{552}}
    \includegraphics[width=0.18\textwidth]{./3_5_newid/552_15.jpg}
    \includegraphics[width=0.18\textwidth]{./3_5_newid/neus_5_552_15_render.jpg}
    \includegraphics[width=0.18\textwidth]{./3_5_newid/hfs_5_552_15_render.jpg}
    \includegraphics[width=0.18\textwidth]{./3_5_newid/volsdf_5_552_15_render.jpg}
    \includegraphics[width=0.18\textwidth]{./3_5_newid/ours_5_552_15_render.jpg}\\
    \rotatebox{90}{}
    \includegraphics[width=0.18\textwidth]{./results/template_effects/gt_571_blank.jpg}
    \includegraphics[width=0.18\textwidth]{./3_5_newid/neus_5_552_15_normal.jpg}
    \includegraphics[width=0.18\textwidth]{./3_5_newid/hfs_5_552_15_normal.jpg}
    \includegraphics[width=0.18\textwidth]{./3_5_newid/volsdf_5_552_15_normal.jpg}
    \includegraphics[width=0.18\textwidth]{./3_5_newid/ours_5_552_15_normal.jpg}\\
    
    \rotatebox{90}{\textbf{555}}
    \includegraphics[width=0.18\textwidth]{./3_5_newid/555_15.jpg}
    \includegraphics[width=0.18\textwidth]{./3_5_newid/neus_5_555_15_render.jpg}
    \includegraphics[width=0.18\textwidth]{./3_5_newid/hfs_5_555_15_render.jpg}
    \includegraphics[width=0.18\textwidth]{./3_5_newid/volsdf_5_555_15_render.jpg}
    \includegraphics[width=0.18\textwidth]{./3_5_newid/ours_5_555_15_render.jpg}\\
    \rotatebox{90}{}
    \includegraphics[width=0.18\textwidth]{./results/template_effects/gt_571_blank.jpg}
    \includegraphics[width=0.18\textwidth]{./3_5_newid/neus_5_555_15_normal.jpg}
    \includegraphics[width=0.18\textwidth]{./3_5_newid/hfs_5_555_15_normal.jpg}
    \includegraphics[width=0.18\textwidth]{./3_5_newid/volsdf_5_555_15_normal.jpg}
    \includegraphics[width=0.18\textwidth]{./3_5_newid/ours_5_555_15_normal.jpg}\\

    \rotatebox{90}{\textbf{598}}
    \includegraphics[width=0.18\textwidth]{./3_5_newid/598_15.jpg}
    \includegraphics[width=0.18\textwidth]{./3_5_newid/neus_5_598_15_render.jpg}
    \includegraphics[width=0.18\textwidth]{./3_5_newid/hfs_5_598_15_render.jpg}
    \includegraphics[width=0.18\textwidth]{./3_5_newid/volsdf_5_598_15_render.jpg}
    \includegraphics[width=0.18\textwidth]{./3_5_newid/ours_5_598_15_render.jpg}\\
    \rotatebox{90}{}
    \includegraphics[width=0.18\textwidth]{./results/template_effects/gt_571_blank.jpg}
    \includegraphics[width=0.18\textwidth]{./3_5_newid/neus_5_598_15_normal.jpg}
    \includegraphics[width=0.18\textwidth]{./3_5_newid/hfs_5_598_15_normal.jpg}
    \includegraphics[width=0.18\textwidth]{./3_5_newid/volsdf_5_598_15_normal.jpg}
    \includegraphics[width=0.18\textwidth]{./3_5_newid/ours_5_598_15_normal.jpg}\\
    \makebox[0.18\textwidth]{\scriptsize GT}
    \makebox[0.18\textwidth]{\scriptsize NeuS}
    \makebox[0.18\textwidth]{\scriptsize HF-NeuS}
    \makebox[0.18\textwidth]{\scriptsize VolSDF}
    \makebox[0.18\textwidth]{\scriptsize Ours}
    \caption{Training view results for 3 unseen identities (552, 555, 598) with only 5 views.}
    \label{fig:unseen_result_train}
\end{figure*}

\begin{figure*}
    \centering
    \rotatebox{90}{\textbf{552}}
    \includegraphics[width=0.18\textwidth]{./3_5_newid/552_19.jpg}
    \includegraphics[width=0.18\textwidth]{./3_5_newid/neus_5_552_19_render.jpg}
    \includegraphics[width=0.18\textwidth]{./3_5_newid/hfs_5_552_19_renderl.jpg}
    \includegraphics[width=0.18\textwidth]{./3_5_newid/volsdf_5_552_19_render.jpg}
    \includegraphics[width=0.18\textwidth]{./3_5_newid/ours_5_552_19_render.jpg}\\
    \rotatebox{90}{}
    \includegraphics[width=0.18\textwidth]{./results/template_effects/gt_571_blank.jpg}
    \includegraphics[width=0.18\textwidth]{./3_5_newid/neus_5_552_19_normal.jpg}
    \includegraphics[width=0.18\textwidth]{./3_5_newid/hfs_5_552_19_normal.jpg}
    \includegraphics[width=0.18\textwidth]{./3_5_newid/volsdf_5_552_19_normal.jpg}
    \includegraphics[width=0.18\textwidth]{./3_5_newid/ours_5_552_19_normal.jpg}\\
    
    \rotatebox{90}{\textbf{555}}
    \includegraphics[width=0.18\textwidth]{./3_5_newid/555_29.jpg}
    \includegraphics[width=0.18\textwidth]{./3_5_newid/neus_5_555_29_render.jpg}
    \includegraphics[width=0.18\textwidth]{./3_5_newid/hfs_5_555_29_render.jpg}
    \includegraphics[width=0.18\textwidth]{./3_5_newid/volsdf_5_555_29_render.jpg}
    \includegraphics[width=0.18\textwidth]{./3_5_newid/ours_5_555_29_render.jpg}\\
    \rotatebox{90}{}
    \includegraphics[width=0.18\textwidth]{./results/template_effects/gt_571_blank.jpg}
    \includegraphics[width=0.18\textwidth]{./3_5_newid/neus_5_555_29_normal.jpg}
    \includegraphics[width=0.18\textwidth]{./3_5_newid/hfs_5_555_29_normal.jpg}
    \includegraphics[width=0.18\textwidth]{./3_5_newid/volsdf_5_555_29_normal.jpg}
    \includegraphics[width=0.18\textwidth]{./3_5_newid/ours_5_555_29_normal.jpg}\\

    \rotatebox{90}{\textbf{598}}
    \includegraphics[width=0.18\textwidth]{./3_5_newid/598_24.jpg}
    \includegraphics[width=0.18\textwidth]{./3_5_newid/neus_5_598_24_render.jpg}
    \includegraphics[width=0.18\textwidth]{./3_5_newid/hfs_5_598_24_render.jpg}
    \includegraphics[width=0.18\textwidth]{./3_5_newid/volsdf_5_598_24_render.jpg}
    \includegraphics[width=0.18\textwidth]{./3_5_newid/ours_5_598_24_render.jpg}\\
    \rotatebox{90}{}
    \includegraphics[width=0.18\textwidth]{./results/template_effects/gt_571_blank.jpg}
    \includegraphics[width=0.18\textwidth]{./3_5_newid/neus_5_598_24_normal.jpg}
    \includegraphics[width=0.18\textwidth]{./3_5_newid/hfs_5_598_24_normal.jpg}
    \includegraphics[width=0.18\textwidth]{./3_5_newid/volsdf_5_598_24_normal.jpg}
    \includegraphics[width=0.18\textwidth]{./3_5_newid/ours_5_598_24_normal.jpg}\\
    \makebox[0.18\textwidth]{\scriptsize GT}
    \makebox[0.18\textwidth]{\scriptsize NeuS}
    \makebox[0.18\textwidth]{\scriptsize HF-NeuS}
    \makebox[0.18\textwidth]{\scriptsize VolSDF}
    \makebox[0.18\textwidth]{\scriptsize Ours}
    \caption{Novel view results for 3 unseen identities (552, 555, 598) with only 5 views.}
    \label{fig:unseen_result_novel}
\end{figure*}

\subsubsection{Sparse Views}
This section provides further results on sparse views as shown in Figure~\ref{fig:sparse_view_old}. Moreover, Table~\ref{tab:unseen_sparse} also demonstrates that our approach surpasses other methods in terms of reconstructing geometry under sparse view conditions.

