\documentclass[10pt,article,superscriptaddress,floatfix,onecolumn,longbibliography]{revtex4-2}

%\pdfoutput=1

\usepackage{amsmath,dcolumn,bm,amsthm,tabularx,subfigure}
\usepackage[colorlinks=true,urlcolor=blue,citecolor=blue,linkcolor=blue]{hyperref}
\usepackage{mathrsfs,times}
\usepackage{bbold,dsfont}
\usepackage{graphicx,graphics,color}
\usepackage{mathtools,nicefrac}
\usepackage{slashed}
\usepackage{amsfonts,amssymb}
\usepackage{tikz}
\usetikzlibrary{positioning}
\allowdisplaybreaks

\newcommand{\beq}{\begin{equation}}
\newcommand{\eeq}{\end{equation}}
\newcommand{\bea}{\begin{eqnarray}}
\newcommand{\eea}{\end{eqnarray}}
\newcommand{\widebar}{\overline}
\newcommand{\wtilde}{\widetilde}
\newcommand{\X}{\widetilde{X}}
\newcommand{\Z}{\widetilde{Z}}
\newcommand{\qp}{\mathrm{qp}}
\newcommand{\SvN}{[S_\textrm{vN}]}
\newcommand{\SvNdef}{S_\textrm{vN}}
\newcommand{\trho}{\widetilde{\rho}}
\newcommand{\I}{{I}}
\newcommand{\R}{{R}}
\newcommand{\ZlrI}{{Z^{lr}\!\!,I}}
\newcommand{\ZlroI}{{Z^{lr}_{1}\!\!,I}}
\newcommand{\ZlrzI}{{Z^{lr}_{0}\!\!,I}}
\newcommand{\ZlrtI}{{Z^{lr}_{2}\!\!,I}}
\newcommand{\sua}{{\sf a}}
\newcommand{\sub}{{\sf b}}

\renewcommand{\thetable}{S\Roman{table}}
\renewcommand{\thefigure}{S\arabic{figure}}
\renewcommand{\thesubsection}{Supplemental Note \arabic{subsection}}
\renewcommand{\theequation}{S\arabic{equation}}

\begin{document}

\title{Supplemental Materials for \lq\lq Symmetry-Resolved Entanglement Dynamics in Disordered Bose-Hubbard Chain''}

\author{Jie~Chen}
\email[Corresponding author.\\]{chenjie666@sjtu.edu.cn}
\affiliation{Key Laboratory of Artificial Structures and Quantum Control (Ministry of Education), School of Physics and Astronomy, Shenyang National Laboratory for Materials Science, Shanghai Jiao Tong University, Shanghai 200240, China}

\author{Chun~Chen}
\email[Corresponding author.\\]{chunchen@sjtu.edu.cn}
\affiliation{Key Laboratory of Artificial Structures and Quantum Control (Ministry of Education), School of Physics and Astronomy, Shenyang National Laboratory for Materials Science, Shanghai Jiao Tong University, Shanghai 200240, China}

\author{Xiaoqun~Wang}
\email[Corresponding author.\\]{xiaoqunwang@zju.edu.cn}
\affiliation{Key Laboratory of Artificial Structures and Quantum Control (Ministry of Education), School of Physics and Astronomy, Shenyang National Laboratory for Materials Science, Shanghai Jiao Tong University, Shanghai 200240, China}
\affiliation{School of Physics, Zhejiang University, Hangzhou 310058, Zhejiang, China}
\affiliation{Tsung-Dao Lee Institute, Shanghai Jiao Tong University, Shanghai 200240, China}
\affiliation{Collaborative Innovation Center of Advanced Microstructures, Nanjing University, Nanjing 210093, China}

\date{\today}

\maketitle

\tableofcontents

\newpage

\subsection{Dynamical generation of ECW in a $L=4,N=2$ free-boson chain}

The formation of the entanglement channel wave (ECW) can be mathematically established on a finite free-boson chain. Here, for the purpose of illustration, it might be illuminating to work out some details for the simplest case where two bosons hop freely on a four-site periodic chain. The tight-binding model is described by the following Hamiltonian,
\beq
H_0=-J\sum^{L}_{i=1}(a^{\dagger}_ia_{i+1}+\textrm{H.c.})=-2J\sum_k\cos k\ a^\dagger_k a_k,
\eeq
where the lattice constant is set to $1$ and the Fourier transform reads $a_i=L^{-\frac{1}{2}}\sum_k e^{ik R_i}a_k$.

Starting from a normalized pure state $|\psi\rangle=\sum_{\sua,\sub}\psi_{\sua,\sub}|\sua\rangle|\sub\rangle$ where $\{|\sua\rangle\},\{|\sub\rangle\}$ are orthonormal bases for the left and right half-chains, the reduced density matrix is simply given by
\beq
\rho_L={\sf Tr}_R\left(|\psi\rangle\langle\psi|\right)=\sum_{\sua,\sua'}\left[\sum_{\sub'}\psi_{\sua,\sub'}\psi^*_{\sua',\sub'}\right]|\sua\rangle\langle\sua'|.
\label{densitymatrix}
\eeq
Under the unitary time evolution, it is easy to see that
\beq
\psi_{\sua,\sub}=\langle\sua,\sub|\psi\rangle\ \ \ \Longrightarrow\ \ \ \psi_{\sua,\sub}(t)=\langle\sua,\sub|e^{-iH_0t}|\psi\rangle.
\label{rhoab}
\eeq

Now the symmetry-resolved density matrix $\rho_{L,n}$ can be constructed by specifying and going through all basis states $|\sua,\sub\rangle$ within the number sectors $n$ and $N-n$. For the case of $L=4,N=2$, it is easy to see that $n=0,1,2$ and due to the observation that $\rho_{L,n=0,2}$ are essentially one dimensional, the corresponding $S^{n=0,2}_{\textrm{vN},L}=0$. We thus only need to examine the nontrivial channel $n=1$. The basis state can then be labelled by the site index where the single boson is located. Concretely,
\beq
(\sua,\sub)=(1,3),(1,4),(2,3),(2,4),
\eeq
where for instance $(\sua,\sub)=(1,3)$ means that $|\sua,\sub\rangle=a^\dagger_1a^\dagger_3|0\rangle$.

The initial $l$-state can be written as follows,
\beq
|\psi\rangle=a^\dagger_1 a^\dagger_2|0\rangle.
\eeq
Upon Fourier transforms, one derives
\beq
e^{-iH_0t}|\psi\rangle=L^{-2}\sum_{k,k'}\sum_{R_i,R_j}e^{-ikR_1}e^{-ik'R_2}e^{i(2J\cos k+2J\cos k')t}e^{ikR_i}e^{ik'R_j}a^\dagger_i a^{\dagger}_j|0\rangle.
\eeq
Throughout this note we always assume that $t\geqslant0$. Inserting the component basis state from channel $n=1$, Eq.~(\ref{rhoab}) then yields
\begin{align}
(\sua,\sub)=(1,3)\ \ &\Longrightarrow\ \ \psi_{1,3}(t)=L^{-2}\sum_{k,k'}\left(e^{-ik'(R_2-R_3)}+e^{-ik(R_1-R_3)}e^{-ik'(R_2-R_1)}\right)\cdot e^{i(2J\cos k+2J\cos k')t}, \\
(\sua,\sub)=(1,4)\ \ &\Longrightarrow\ \ \psi_{1,4}(t)=L^{-2}\sum_{k,k'}\left(e^{-ik'(R_2-R_4)}+e^{-ik(R_1-R_4)}e^{-ik'(R_2-R_1)}\right)\cdot e^{i(2J\cos k+2J\cos k')t}, \\
(\sua,\sub)=(2,3)\ \ &\Longrightarrow\ \ \psi_{2,3}(t)=L^{-2}\sum_{k,k'}\left(e^{-ik(R_1-R_3)}+e^{-ik(R_1-R_2)}e^{-ik'(R_2-R_3)}\right)\cdot e^{i(2J\cos k+2J\cos k')t}, \\
(\sua,\sub)=(2,4)\ \ &\Longrightarrow\ \ \psi_{2,4}(t)=L^{-2}\sum_{k,k'}\left(e^{-ik(R_1-R_4)}+e^{-ik(R_1-R_2)}e^{-ik'(R_2-R_4)}\right)\cdot e^{i(2J\cos k+2J\cos k')t}.
\end{align}
Clearly, all the above four elements vanish identically at $t=0$. 

Taking advantage of the periodicity of momenta $k,k'$, it is ready to show that
\begin{align}
\psi^*_{1,3}(t)&=\left\{L^{-2}\sum_{k,k'}\left[e^{ik'}+e^{2ik}e^{-ik'}\right]\cdot e^{i(2J\cos k+2J\cos k')t}\right\}^* \nonumber \\
&=L^{-2}\sum_{k,k'}\left[e^{-ik'}+e^{-2ik}e^{ik'}\right]\cdot e^{-i(2J\cos k+2J\cos k')t} \nonumber \\
&=L^{-2}\sum_{k,k'}\left[e^{-i(k'+\pi)}+e^{-2i(k+\pi)}e^{i(k'+\pi)}\right]\cdot e^{i(2J\cos k+2J\cos k')t} \nonumber \\
&=-L^{-2}\left\{\sum_{k,k'}\left[e^{-ik'}+e^{-2ik}e^{ik'}\right]\cdot e^{i(2J\cos k+2J\cos k')t}\right\} \nonumber \\
&=-L^{-2}\left\{\sum_{k,k'}\left[e^{ik'}+e^{2ik}e^{-ik'}\right]\cdot e^{i(2J\cos k+2J\cos k')t}\right\}=-\psi_{1,3}(t).
\end{align} 
Therefore, after switching the signs of $k,k'$, it is shown that $\psi_{1,3}(t)$ is purely imaginary, i.e., $\psi^*_{1,3}(t)=-\psi_{1,3}(t)$. By contrast, one finds that $\psi_{1,4}(t)$ is purely real,
\begin{align}
\psi^*_{1,4}(t)&=\left\{L^{-2}\sum_{k,k'}\left[e^{2ik'}+e^{3ik}e^{-ik'}\right]\cdot e^{i(2J\cos k+2J\cos k')t}\right\}^* \nonumber \\
&=L^{-2}\sum_{k,k'}\left[e^{-2ik'}+e^{-3ik}e^{ik'}\right]\cdot e^{-i(2J\cos k+2J\cos k')t} \nonumber \\
&=L^{-2}\sum_{k,k'}\left[e^{-2i(k'+\pi)}+e^{-3i(k+\pi)}e^{i(k'+\pi)}\right]\cdot e^{i(2J\cos k+2J\cos k')t} \nonumber \\
&=L^{-2}\sum_{k,k'}\left[e^{-2ik'}+e^{-3ik}e^{ik'}\right]\cdot e^{i(2J\cos k+2J\cos k')t} \nonumber \\
&=L^{-2}\sum_{k,k'}\left[e^{2ik'}+e^{3ik}e^{-ik'}\right]\cdot e^{i(2J\cos k+2J\cos k')t}=\psi_{1,4}(t).
\end{align}
Therefore, $\psi^*_{1,4}(t)=\psi_{1,4}(t)$. Furthermore, one can show that $\psi_{1,3}(t)=\psi_{2,4}(t)$,
\begin{align}
\psi_{2,4}(t)&=L^{-2}\sum_{k,k'}\left[e^{3ik}+e^{ik}e^{2ik'}\right]\cdot e^{i(2J\cos k+2J\cos k')t} \nonumber \\
&=L^{-2}\sum_{k',k}\left[e^{3ik'}+e^{ik'}e^{2ik}\right]\cdot e^{i(2J\cos k'+2J\cos k)t} \nonumber \\
&=L^{-2}\sum_{k',k}\left[e^{-3ik'}+e^{-ik'}e^{2ik}\right]\cdot e^{i(2J\cos k'+2J\cos k)t} \nonumber \\
&=L^{-2}\sum_{k',k}\left[e^{4ik'}e^{-3ik'}+e^{-ik'}e^{2ik}\right]\cdot e^{i(2J\cos k'+2J\cos k)t}=\psi_{1,3}(t),
\end{align}   
where in the second line we exchange $k\leftrightharpoons k'$, switch the sign of $k'$ only in the third line, and finally use the identity that $e^{ik'L}=e^{4ik'}=1$. Likewise, we can show that $\psi_{1,4}(t)=\psi_{2,3}(t)$,
\begin{align}
\psi_{2,3}(t)&=L^{-2}\sum_{k,k'}\left[e^{2ik}+e^{ik}e^{ik'}\right]\cdot e^{i(2J\cos k+2J\cos k')t} \nonumber \\
&=L^{-2}\sum_{k',k}\left[e^{2ik'}+e^{ik'}e^{ik}\right]\cdot e^{i(2J\cos k'+2J\cos k)t} \nonumber \\
&=L^{-2}\sum_{k',k}\left[e^{-2ik'}+e^{-ik'}e^{-ik}\right]\cdot e^{i(2J\cos k'+2J\cos k)t} \nonumber \\
&=L^{-2}\sum_{k',k}\left[e^{-2ik'}e^{4ik'}+e^{-ik'}e^{-ik}e^{4ik}\right]\cdot e^{i(2J\cos k'+2J\cos k)t}=\psi_{1,4}(t),
\end{align} 
where in the second line we exchange $k\leftrightharpoons k'$, switch both signs of $k,k'$ in the third line, and finally use the identity that $e^{ik'L}=e^{4ik'}=e^{ikL}=e^{4ik}=1$.

To summarize, by resorting to the system's translation and reflection symmetries, we find that
\begin{align}
\psi^*_{1,3}(t)&=-\psi_{1,3}(t), \\
\psi^*_{1,4}(t)&=\psi_{1,4}(t), \\
\psi_{1,3}(t)&=\psi_{2,4}(t), \\
\psi_{1,4}(t)&=\psi_{2,3}(t).
\end{align}

Armed with the above relations, the symmetry-resolved density matrix $\rho_{L,n=1}(t)$ can be readily constructed as per Eq.~(\ref{densitymatrix}) as a two-by-two matrix,
\begin{align}
\rho_{L,n=1}(t)\coloneqq&\begin{bmatrix}
\rho_{1,1}(t) & \rho_{1,2}(t) \\
\rho_{2,1}(t) & \rho_{2,2}(t) 
\end{bmatrix}=\begin{bmatrix}
\psi_{1,3}(t) & \psi_{1,4}(t) \\
\psi_{2,3}(t) & \psi_{2,4}(t) 
\end{bmatrix}\cdot\begin{bmatrix}
\psi^*_{1,3}(t) & \psi^*_{2,3}(t) \\
\psi^*_{1,4}(t) & \psi^*_{2,4}(t) 
\end{bmatrix} \nonumber \\
=&\begin{bmatrix}
\psi_{1,3}(t)\psi^*_{1,3}(t)+\psi_{1,4}(t)\psi^*_{1,4}(t) & \psi_{1,3}(t)\psi^*_{2,3}(t)+\psi_{1,4}(t)\psi^*_{2,4}(t) \\
\psi_{2,3}(t)\psi^*_{1,3}(t)+\psi_{2,4}(t)\psi^*_{1,4}(t) & \psi_{2,3}(t)\psi^*_{2,3}(t)+\psi_{2,4}(t)\psi^*_{2,4}(t) 
\end{bmatrix} \nonumber \\
=&\left[\psi_{1,3}(t)\psi^*_{1,3}(t)+\psi_{1,4}(t)\psi^*_{1,4}(t)\right]\cdot\begin{bmatrix}
1 & 0 \\
0 & 1 
\end{bmatrix},
\label{unnorm_denmat}
\end{align}
where the following relations are employed, $\psi_{1,3}(t)\psi^*_{2,3}(t)+\psi_{1,4}(t)\psi^*_{2,4}(t)=\psi_{1,3}(t)\psi^*_{1,4}(t)+\psi_{1,4}(t)\psi^*_{1,3}(t)=0$ and $\psi_{2,3}(t)\psi^*_{2,3}(t)+\psi_{2,4}(t)\psi^*_{2,4}(t)=\psi_{1,4}(t)\psi^*_{1,4}(t)+\psi_{1,3}(t)\psi^*_{1,3}(t)$. Therefore, assume that $|\psi_{1,3}|^2+|\psi_{1,4}|^2\neq0$, the normalized density matrix for the channel $n=1$ is simply
\beq
\widetilde{\rho}_{L,n=1}(t>0)=\begin{bmatrix}
\frac{1}{2} & 0 \\
0 & \frac{1}{2}
\end{bmatrix}\ \ \ \Longrightarrow\ \ \ S^{n=1}_{\textrm{vN},L}(t>0)=-\frac{1}{2}\log_2\left(\frac{1}{2}\right)-\frac{1}{2}\log_2\left(\frac{1}{2}\right)=1.
\label{norm_denmat}
\eeq
In other words, once $t$ starts to deviate from $0$, $S^{n=1}_{\textrm{vN},L}(t)$ will suddenly jump to $1$ and largely become a constant (up to abrupt drops at particular periodicities due to the vanishing of the diagonal element) in this simplest nontrivial setting.

One peculiarity associated to the dynamics of symmetry-resolved entanglement entropy as encoded in Eqs.~(\ref{unnorm_denmat}) and (\ref{norm_denmat}) is that at any moment $t$, regardless how small the diagonal matrix element $|\psi_{1,3}|^2+|\psi_{1,4}|^2\geqslant0$ is, once it is nonzero, the equality of the dominant diagonal elements $\rho_{1,1}$ and $\rho_{2,2}$ guarantees that $S^{n=1}_{\textrm{vN},L}=1$. Via this demonstration, the existence of ECW can thus be rigorously established.

\newpage

\subsection{Two types of melting processes of ECWs in the ETH phase}

In this work, we exclusively focus on the even chains at half filling with periodic boundary conditions. Under these specifications, we find there exist two types of ECWs depending on the parity of the total number of bosons. As shown by Figs.~\ref{pics1}(a),(b), when the total number of bosons is even, the discontinuous jumps of $S^n_{\textrm{vN},L}$ occurs only for those odd-$n$ channels. In comparison, as shown by Figs.~\ref{pics1}(c),(d), when the total number of bosons is odd, the discontinuous jumps of $S^n_{\textrm{vN},L}$ occurs only for those even-$n$ channels. Here we always start from the $l$-state and choose the disorder strength $\mu=2J$ to be small.

To be pedantic, we should name the ECW in Figs.~\ref{pics1}(a),(b) to be the reflection-symmetric ECW (with respect to the channel axis $n$). While, the ECW in Figs.~\ref{pics1}(c),(d) should be called the reflection-asymmetric ECW for the obvious geometric reasons.

Due to the weak breaking of the spatial reflection symmetry, we know these two types of ECWs will melt in the long-time limit. Figure~\ref{pics1} reveals that the melting processes for these two ECWs are different.

Concretely, from Figs.~\ref{pics1}(a),(b), we observe that for the reflection-symmetric ECW, the $\mathbb{Z}_2$ channel wave pattern disappears but the channel reflection symmetry of the ECW persists. By contrast, from Figs.~\ref{pics1}(c),(d), we observe that for the reflection-asymmetric ECW, the disappearance of the $\mathbb{Z}_2$ channel wave is accompanied by a symmetrization process with the emergence of the channel reflection symmetry. Note that this channel reflection symmetry does not exist in the initially reflection-asymmetric ECW. In this sense, the melting of the reflection-asymmetric ECW is more drastic than that of the reflection-symmetric ECW.

%---------------------------------------------------------------------------------------
\begin{figure*}[hb]
	
	\centering
	
	\subfigbottomskip=0.1pt
	
	\subfigure{
		\begin{minipage}{0.35\linewidth}
			\centering
			\includegraphics[scale=0.4]{svnn_evo_L16_MU02.eps}
			%\caption{fig1}
		\end{minipage}
	}%
	\subfigure{
		\begin{minipage}{0.35\linewidth}
			\centering
			\includegraphics[scale=0.42]{svnn_evo_L16_MU02_col.eps}
			%\caption{fig2}
		\end{minipage}
	}%
	\quad
	\subfigure{
		\begin{minipage}{0.35\linewidth}
			\centering
			\includegraphics[scale=0.4]{svnn_evo_L18_MU02.eps}
			%\caption{fig3}
		\end{minipage}
	}%
	\subfigure{
		\begin{minipage}{0.35\linewidth}
			\centering
			\includegraphics[scale=0.42]{svnn_evo_L18_MU02_col.eps}
			%\caption{fig4}
		\end{minipage}
	}%
	
	\caption{Two types of ECWs and their different melting processes in the thermal regime. (a),(b) show the formation and the time evolution of the channel-reflection-symmetric ECW in a periodic even chain accommodating even number of total bosons. The melting of this first type of ECW removes the $\mathbb{Z}_2$ channel wave pattern but preserves the channel reflection symmetry. (c),(d) show the formation and the time evolution of the channel-reflection-asymmetric ECW in a periodic even chain accommodating odd number of total bosons. The melting of this second type of ECW not only removes the $\mathbb{Z}_2$ channel wave pattern but also allows for the emergence of the channel reflection symmetry.}
	\label{pics1}

\end{figure*}
%---------------------------------------------------------------------------------------

\newpage

\subsection{Comparison to the cut-varying entanglement patterns}

When devising the pattern of entanglement, one usually wants to maximize the inherent nonlocality structure as much as possible. By fixing one cut position at the middle of the chain and the other at the boundary, the ECW based on the half-chain von Neumann entanglement entropy maintains the maximal degree with regard to nonlocality.

Conceptually, the simplest entanglement pattern one may think of is by successively varying the cut position from the first bond at the boundary to the middle bond at the chain's centre. A representative for this type of real-space entanglement patterns is provided by the dimer product state of spin chains, i.e., two nearest spins are first entangled into a singlet and next this two-spin dimer forms the product state with other similar two-spin dimers. This dimer-like entanglement pattern resembles the CDW; one only needs to change the position variable from the site to the bond.

However, as compared to the ECW, the nonlocality property of such an initial-state arrangement is reduced in two aspects. First, for the ideal case of the dimer product state, one can easily understand that the nonlocality property is somehow minimized because the actual entanglement of the state only involves two nearest-neighboring sites. Second, the varying of the cut position necessarily creates some inhomogeneity of the entanglement pattern because the nonlocality feature and the corresponding entanglement entropy have to diminish monotonically when one of the cuts is moved away from the centre bond of the chain to its end bond.

In the main text, we stress that the normalization is not only a crucial conceptual step for the proper definition of the symmetry-resolved entanglement entropy, but it also comprises an indispensable technical procedure to hide the absolute smallness of the pertinent density-matrix elements [for example see Eqs.~(\ref{unnorm_denmat}) and (\ref{norm_denmat})] such that the ECW pattern determined from the relative weights within the number-symmetry block can surface. By contrast, this normalization procedure does not arise in the calculation of the cut-varying entanglement patterns such as the aforementioned dimer product state and the like. Therefore, in view of the exponential suppression of the particle transport under the strong disorders, we anticipate that the many-body dephasing would not be efficient enough to symmetrize the real-space cut-varying entanglement patterns.

Roughly speaking, the preservation of CDW in MBL is due to its finite overlaps with the local integrals of motion (l-bits) configurations. This physical reasoning works well to interpret the imbalanced distribution of $p_{n,\infty}$. However, the same rationale might not be directly applicable to account for the situation of ECW. This is partly owing to the intrinsic nonlocality of ECW. But, more importantly, unlike the significance of the absolute projection amplitudes onto these l-bit states for the persistence of the initial CDW or $p_{n,\infty}$, it is the relative weights among those nonlocal l-bit projections that control the evolution of ECW. 

Likewise, the physical picture based on the Hilbert-space (de)localization may also be largely set by the absolute values of the prefactors associated to the generated component states under the unitary time evolution of the initial state. In this sense, the calculation of the accompanying IPR could be useful in understanding the $p_{n,\infty}$ distribution. Nonetheless, it might not directly concern the symmetrization between $S^{n,N-n}_{\textrm{vN},L}$. Symmetry-resolved IPRs may be a useful alternative for this purpose.

\newpage

\subsection{Individual eigenstate properties at low and high eigenenergy densities under weak disorders}

The purpose of this section is to exhibit the observation that the eigenstates formed by scatter bosons could be drastically different from the eigenstates formed by cluster bosons. To this end, we concentrate on the region of weak disorder strength $(\mu=2J)$, pick up an arbitrary random sample, and then perform an analysis upon the individual eigenstates in an energy-density resolved manner. As $\mu$ is small, we do not expect any significant differences between different random-sample realizations.

Figure~\ref{pics2}(a) displays the channel-resolved $S^n_{\textrm{vN},L}$ as a function of the channel index $n$ for the eleven eigenstates of the Hamiltonian $H_{\textrm{dBH}}$ whose eigenenergies are the closest to that of the initial $l$-state $|l\rangle$, i.e., the eleven closest to $\langle l|H_{\textrm{dBH}}|l\rangle\sim0.25$. Note that here we just pick up one arbitrary random-sample realization. Clearly, these eleven curves nicely overlap with each other, demonstrating the agreement with the eigenstate thermalization hypothesis (ETH), viz., the wavefunction properties in this low-energy window are chiefly controlled by the specified eigenenergy density. Therefore, in this ETH regime, any arbitrary eigenstate could be the representative to extract the relevant physical quantities about the system. It is worth emphasizing that these eleven low-energy eigenstates like the $l$-state are comprised mainly by scatter bosons.

However, once we switch to the regime of the high-energy $p$-state $|p\rangle$, Fig.~\ref{pics2}(b) presents the corresponding expectation value of the local boson occupation number as a function of the site index $i$. Here, as is in panel (a), we pick up eleven eigenstates from an arbitrary random sample whose eigenenergies are the closest to that of the initial $p$-state, i.e., the eleven closest in energy to $\langle p|H_{\textrm{dBH}}|p\rangle\sim0.75$. Strikingly, we find that within this high-energy window, each eigenstate exhibits distinguishable properties. First, almost all bosons in each eigenstates cluster round a particular single lattice site and second, for distinct eigenstates, the sites where the clustered bosons sit are also different. In other words, in this high-energy window, the ETH breaks down, in the sense that although two eigenstates are adjacent in eigenenergy, their physical properties could be dramatically distinct from each other. To some extent, Fig.~\ref{pics2}(b) illustrates the Hilbert-space structure that allows for the formation of the cluster MBL in the presence of weak disorders.      

%--------------------------------------------------------------------------------------
\begin{figure*}[hb]
	\begin{center}
		
		\centering
		
		\subfigbottomskip=0.1pt
		
		\subfigure{
			\begin{minipage}{0.33\linewidth}
				\centering
				\includegraphics[scale=0.43]{svnn_es_L_14_mu_2_st_2.eps}
				%\caption{fig1}
			\end{minipage}
		}%
		\subfigure{
			\begin{minipage}{0.33\linewidth}
				\centering
				\includegraphics[scale=0.43]{n_L_14_mu_2_st_1.eps}
				%\caption{fig2}
			\end{minipage}
		}%
		
		\caption{Individual eigenstate properties at the low- and high-energy densities. (a) depicts the channel-resolved entropy $S^n_{\textrm{vN},L}$ as a function of the channel index $n$ for the eleven eigenstates which are the closest in energy to that of the initial $l$-state. As the $l$-state is formed by scatter bosons, (a) gives the results for the low-energy window, which is consistent with the predictions based on ETH. (b) depicts the local boson occupation number as a function of the lattice site for the eleven eigenstates which are the closest in energy to that of the initial $p$-state. As the $p$-state is formed by cluster bosons, (b) supplies the results from the high-energy window, which clearly violate the predictions based on ETH. Notice that here no disorder averages are needed, because only one arbitrary random-sample realization is used. The length of the periodic chain is $14$, the total number of bosons is $7$, and the disorder strength $\mu=2J$.}
		\label{pics2}
		
	\end{center}
\end{figure*}
%--------------------------------------------------------------------------------------

\newpage

\subsection{Time evolution of the channel-resolved number entropy}

The reference [M.~Kiefer-Emmanouilidis {\it et al}., \href{https://doi.org/10.1103/PhysRevLett.124.243601}{Phys. Rev. Lett. {\bf 124}, 243601 (2020)}] reported that the total number entropy seems to grow double logarithmically over time even deep inside the MBL phase, hinting that the full localization might be unstable in the thermodynamic limit due to the unceasing particle or energy transports. 

In this section, we perform a channel-resolved analysis of the total number entropy $S_N$ for the disordered Bose-Hubbard (dBH) chain model to help clarify the possibility that the observed double-log growth of $S_N$ might not necessarily indicate the breakdown of the full localization.

The total number entropy $S_N$ is defined by
\beq
S_N(t)\coloneqq-\sum_n p_n(t)\log_2 p_n(t).
\eeq
This naturally suggests the channel-resolved number entropy $S^n_N$ given by
\beq
S^n_N(t)\coloneqq-p_n(t)\log_2 p_n(t).
\eeq

Figure~\ref{pics3}(b) shows the quantum quench evolutions of $S^n_N(t)$ in a log-log format starting from the initial $l$-state for a strongly disordered $(\mu=20J)$, periodic BH chain of length $L=18$. There are two salient features in Fig.~\ref{pics3}(b). 
\begin{enumerate}
\item[$\bullet$] First, the temporal $S^n_N$ growth in those small $n<5$ channels fulfils a power law, which is much faster than $\log\log(t)$. While, for those large $n>5$ channels, the corresponding $S^n_N$ growth does however become noticeably slower.
\item[$\bullet$] Second, the absolute values of $S^n_N$ in those small $n<5$ channels are negligibly smaller than the absolute values of $S^n_N$ in those large $n>5$ channels. Particularly, it appears that the contributions from the channels $n=9,8$ dominate the whole time evolution of the total $S_N(t)$.
\end{enumerate}
From Fig.~4(c) in the main text, we have known that the total $S_N$ in this quantum quench setup indeed grows double logarithmically over time. Therefore, combine the above-listed two points, we argue that the double-log growth of $S_N$ should be predominantly controlled by the channels of large $n=9,8$ and the participation of those small-$n$ channels might be negligible. In other words, in this quantum quench evolution, particles are well confined to the left half-chain; the perceived particle-number fluctuations occur mainly as reorganizations within the initial number channels where particles are first released rather than the substantial particle transports involving all available channels, in particular, those remote channels of small $n$.

As a comparison, Fig.~\ref{pics3}(a) plots the channel-resolved buildup of $S^n_N$ for the weak-disorder thermal regime, where we still begin with the initial $l$-state and now set $\mu=2J$. The obtained results are consistent with the predictions of ETH, namely, each channel participates significantly and almost equally. As per Fig.~4(a) in the main text, the overall growth of the total $S_N$ in this case follows a logarithmic function of time. This might be due to the fact that after the transient period, the paired two channels $n'$ and $N-n'$ could exhibit the opposite evolution tendencies before eventually converging to the same saturation value. Particularly, for those large-$n$ channels, the corresponding $S^n_N$ experiences first an increase and then a decrease over time before reaching the saturation, which is unlike the monotonic increase of $S^n_N$ occurring in those small-$n$ channels.      

%--------------------------------------------------------------------------------------
\begin{figure*}[hb]
	\begin{center}
		
		\centering
		
		\subfigbottomskip=0.1pt
		
		\subfigure{
			\begin{minipage}{0.33\linewidth}
				\centering
				\includegraphics[scale=0.43]{snn_evo_1.eps}
				%\caption{fig1}
			\end{minipage}
		}%
		\subfigure{
			\begin{minipage}{0.33\linewidth}
				\centering
				\includegraphics[scale=0.43]{snn_evo_2.eps}
				%\caption{fig2}
			\end{minipage}
		}%
	
		\caption{Time evolution of the channel-resolved number entropy $S^n_N$. Here we start from the initial $l$-state in a disordered, periodic BH chain of length $L=18$. (a) gives the results for the thermal phase realized at the weak-disorder regime of $\mu=2J$. (b) corresponds to the regime deep inside the scatter MBL phase stabilized by $\mu=20J$.}
		\label{pics3}
		
	\end{center}
\end{figure*}
%--------------------------------------------------------------------------------------

%--------------------------------------------------------------------------------------------------------------------------------------------------------------------
\newpage

\subsection{Analysis of the scaling behavior of $S_N$ initialized from the $l$-state}

%---------------------------------------------------------------------------------------
For the later convenience, let $n_r=N-n$ denote the number of particles in the right half-chain, $n$ denote the number of particles in the left half-chain, and $N=L/2$ denote the total number of particles in the whole chain.
%---------------------------------------------------------------------------------------

%---------------------------------------------------------------------------------------
The $l$-state, whose energy density is in the ETH region, thermalizes at small disorder. As shown by Fig.~\ref{pics4}(a), the corresponding long-time $p_{n_r}$ does approximately satisfy the Gaussian distribution. One can thus assume that $p_{n_r}=\alpha \cdot e^{-\left( n_r-N/2 \right) ^2/\beta ^2}$. Further, since $p_{n_r}$ needs to be normalized, as compared to the normalized Gaussian distribution, it is derivable that $\beta \sim N$ and $\alpha \sim \frac{1}{N}$. According to Fig. 2(c) in the main text, it is then reasonable to let $\beta \approx cL$ and $\alpha \approx d\frac{1}{L}$, where $c$, $d$ are two parameters. Therefore, the particle number entropy $S_N$ can be calculated as follows,
\begin{align}
S_N&=\sum_{n_r=0}^N{-p_{n_r}\cdot \log \left( p_{n_r} \right)} \nonumber \\
&=-\log \left( \alpha \right) +\frac{\alpha}{\beta ^2}2\sum_{n_r=0}^{N/2}{\left[ n_{r}^{2}e^{-n_{r}^{2}/\beta ^2} \right]} \nonumber \\
&\approx -\log \left( \alpha \right) +\frac{\alpha}{\beta ^2}2\int_0^{N/2}{n_{r}^{2}e^{-n_{r}^{2}/\beta ^2}dn_r} \nonumber \\
&=\log \left( L \right) -\log \left( d \right) +\frac{1}{4}d\left[ 2c\sqrt{\pi}Erf\left( \frac{1}{4c} \right) -e^{-1/16c^2} \right]. 
\end{align}
We see that $S_N \sim \ln(L)$. This is also verified by the numerical results given in Fig.~\ref{pics4}(c).
%---------------------------------------------------------------------------------------

%---------------------------------------------------------------------------------------
For a free bosonic system without interactions, an ideal case can be assumed where the various possible configurations are assumed to have the same probability in the thermalized state, which means that $p_n$ is determined only by the Hilbert-space dimension of the chunk. The total Hilbert space of a bosonic system with chain length $L$ and the number of particles $N$ is $D_{N}^{L}=\frac{\left( L+N-1 \right) !}{N!\left( L-1 \right) !}$, the dimension of the Hilbert subspace with particle number $n$ on the left half-chain is $D_n=D_{n}^{L/2}D_{N-n}^{L/2}$, so $p_n=D_n/D_{N}^{L}$. For the current half-filling system $N=L/2$, the result of the numerical calculation is shown in Fig.~\ref{pics4}(c), which also shows that $S_N \sim \ln(L)$.
%---------------------------------------------------------------------------------------

%---------------------------------------------------------------------------------------
\begin{figure*}[htb]
	
	\centering
	
	\subfigbottomskip=0.1pt
	
	\subfigure{
		\begin{minipage}{0.24\linewidth}
			\centering
			\includegraphics[scale=0.32]{pn_end_14_state2.eps}
			%\caption{fig1}
		\end{minipage}
	}%
	\subfigure{
		\begin{minipage}{0.24\linewidth}
			\centering
			\includegraphics[scale=0.32]{pn_end_14_state1.eps}
			%\caption{fig2}
		\end{minipage}
	}%
	\subfigure{
		\begin{minipage}{0.24\linewidth}
			\centering
			\includegraphics[scale=0.32]{sn_eth_ln.eps}
			%\caption{fig3}
		\end{minipage}
	}%
	\subfigure{
		\begin{minipage}{0.24\linewidth}
			\centering
			\includegraphics[scale=0.32]{sn_mbl_area.eps}
			%\caption{fig4}
		\end{minipage}
	}%
	
	\caption{The $p_n$ distribution of the steady state for a $L=14$ dBH chain with different disorder strengths. (a) is for the initial $l$-state, under the case of $\mu=2$, fitted by a Gaussian distribution; for $\mu=20$, the distribution is fitted by $p_{n_r}=\left( n_r+1 \right) ^ae^{bn_r+f}$ with $a=1.9$ and $b=-3.4$, where $n_r=N-n$ denotes the number of particles in the right half-chain. (b) is for the initial $p$-state. (c) The logarithmic fit of the particle number entropy $S_N$ with chain length for the steady state with the initial $l$-state and the small disorder. A comparison of the logarithmic fit to the ideal case of the free Bose system is also given. (d) Area law of the particle number entropy $S_N$ for the steady state when the initial state is $l$-state and large disorder, and the inset gives the results of $S_N$ with parameters from the phenomenological analysis of an infinitely long chain. All calculations related to entropy in the figure have taken $2$ as the logarithmic base.}
	\label{pics4}
	
\end{figure*}
%---------------------------------------------------------------------------------------

%---------------------------------------------------------------------------------------
For the case of large disorder, the system is in MBL, as shown in Fig.~\ref{pics4}(a). In this case, $p_{n_r}$ can be fitted with an exponential decay-like function with $p_{n_r}=\left( n_r+1 \right) ^ae^{bn_r+f}$, where $b$ is always less than $0$ and the larger the disorder, the smaller $b$ will be. Using the normalization condition of $p_{n_r}$, we can obtain $p_{n_r}=\frac{\left( n_r+1 \right) ^ae^{b\left( n_r+1 \right)}}{\left[ Li_{-a}\left( e^b \right) -\left( e^b \right) ^{2+N}L\left( e^b,-a,2+N \right) \right]}$, where $Li_{-a}\left( e^b \right) $ is Lerch zeta function, and $L\left( e^b,-a,2+N \right) $ is polylogarithm function. In Fig.~\ref{pics4}(d), the calculation of $S_N$ using the normalized $p_{n_r}$ values is given, which can be seen to behave as an area law. When the chain length is infinite, we can get $p_{n_r}=\left( n_r+1 \right) ^ae^{bn_r}\frac{e^b}{Li_{-a}\left( e^b \right)}$. Taking this $p_{n_r}$ to compute $S_{N}^{\infty}$, turning the summation of which into an integral, a complex analytic expression can be obtained. In the inset of Fig.~\ref{pics4}(d), $S_{N}^{\infty}$ is calculated using numerical integration, and it can be seen that as the disorder increases (i.e., $b$ decreases), $S_{N}^{\infty}$ becomes smaller, and the tendency of $S_{N}^{\infty}$ to become smaller slows down when the disorder is large, which indicates that the eventual entropy does not decrease continuously to zero as the disorder increases.
%---------------------------------------------------------------------------------------

%--------------------------------------------------------------------------------------------------------------------------------------------------------------------

\newpage

\subsection{Miscellaneous others}

Figure~\ref{pics5} below provides additional evidence to support the single-logarithmic temporal growth for the total configuration entropy $S_C(t)$ inside the scatter MBL regime. Similar single-log $S_C$ growth has also been given by Fig.~4(d) in the main text. Here, we choose the disorder strength to be slightly smaller, i.e., $\mu=12J$, and the initial state to be the $l$-state. Via a systematic finite-size scaling analysis which covers the length range from $L=6$ to $L=14$, it has been cleanly demonstrated that the saturated values of $S_C(t=\infty)$ satisfy the expected volume scaling law.

%--------------------------------------------------------------------------------------
\begin{figure*}[htb]
	\begin{center}
		
		\centering
		
		\subfigbottomskip=0.1pt
		
		\subfigure{
			\begin{minipage}{0.33\linewidth}
				\centering
				\includegraphics[scale=0.43]{sc_evo_st2_mu_12.eps}
				%\caption{fig1}
			\end{minipage}
		}%
		
		\caption{The single-log temporal growth of the total configuration entropy $S_C$ in the scatter MBL phase realized upon a significant disorder strength $\mu=12J$ and by initializing the chain in the $l$-state. The finite-size analysis based on a series of $L$ shows that the saturation values of $S_C$ at infinite time fulfil a volume scaling law.}
		\label{pics5}
		
	\end{center}
\end{figure*}
%--------------------------------------------------------------------------------------

To supply further insights for the energy-resolved dynamical phase diagram of the dBH chain model from the perspective of Hilbert-space (de)localization, in Fig.~\ref{pics6} we study the preliminary time evolutions of the inverse participation ratios (IPRs) for the three types of MBL regimes in the phase diagram that are realizable by starting from the initial $p$- and $l$-states respectively under different values of the disorder strength $\mu$. MBL is typically signalled by the vanishing of IPR, and we find this is indeed the case for both scatter and cluster MBLs, where the absolute values of IPRs have already been smaller than $10^{-3}$ for a finite, periodic dBH chain with length $L=12$. 

The two insets in Fig.~\ref{pics6} plot the corresponding saturated values of IPRs at the infinite-time limit for the respective initial $l$- and $p$-states as a function of the disorder strength $\mu$. Interestingly, for the lower inset which describes the results for the initial $p$-state, we notice that the profile of the infinite-time IPR shows an emerging peak at about $\mu=10J$, signalling that there might exist a transition between the hard cluster MBL at small $\mu<10J$ and the soft cluster MBL at large $\mu>10J$. While, for the initial $l$-state, the upper inset demonstrates that the scatter thermal and the scatter MBL phases form two independent plateaus. The thermal one is centering around a finite value for the infinite-time IPR. By contrast, the infinite-time IPR approaches zero when getting deep inside the scatter MBL regime. In between, there exists a sharply decreasing line connecting these two plateaus, which roughly indicates the transition zone in the infinite-time IPR diagram.     

%--------------------------------------------------------------------------------------
\begin{figure*}[htb]
	\begin{center}
		
		\centering
		
		\subfigbottomskip=0.1pt
		
		\subfigure{
			\begin{minipage}{0.33\linewidth}
				\centering
				\includegraphics[scale=0.43]{ipr_evo.eps}
				%\caption{fig1}
			\end{minipage}
		}%
		
		\caption{Temporal evolutions of IPRs for the three types of MBL regimes in the dBH chain. The length of the periodic chain is $L=12$. The two insets give the corresponding saturation values of IPRs at infinite time for the initial $l$- and $p$-states as a function of the disorder strength $\mu$. Particularly, a peak is observed in the lower inset for the initial $p$-state, which might serve as a transition signature between the hard and the soft cluster MBL phases. Notice, however, once the system is fully localized, the absolute values of IPRs are normally small.}
		\label{pics6}
		
	\end{center}
\end{figure*}
%--------------------------------------------------------------------------------------

\end{document}
