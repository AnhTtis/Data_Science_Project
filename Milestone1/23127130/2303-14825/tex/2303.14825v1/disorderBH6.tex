\documentclass[aps,prl,superscriptaddress,floatfix,twocolumn,longbibliography]{revtex4-2}

\usepackage{amsmath,dcolumn,bm,amsthm,tabularx,subfigure}
\usepackage[colorlinks=true,urlcolor=blue,citecolor=blue,linkcolor=blue]{hyperref}
\usepackage{mathrsfs,times}
\usepackage{bbold,dsfont}
\usepackage{graphicx,graphics,color}
\usepackage{mathtools,nicefrac}
\usepackage{slashed}
\usepackage{amsfonts,amssymb}
\usepackage{multirow}
\allowdisplaybreaks

\newcommand{\beq}{\begin{equation}}
\newcommand{\eeq}{\end{equation}}
\newcommand{\bea}{\begin{eqnarray}}
\newcommand{\eea}{\end{eqnarray}}
\newcommand{\widebar}{\overline}
\newcommand{\wtilde}{\widetilde}
\newcommand{\X}{\widetilde{X}}
\newcommand{\Z}{\widetilde{Z}}
\newcommand{\qp}{\mathrm{qp}}
\newcommand{\SvN}{[S_\textrm{vN}]}
\newcommand{\SvNdef}{S_\textrm{vN}}
\newcommand{\trho}{\widetilde{\rho}}
\newcommand{\I}{{I}}
\newcommand{\R}{{R}}

\begin{document}

\title{Symmetry-Resolved Entanglement Dynamics in Disordered Bose-Hubbard Chain}

\author{Jie~Chen}
\email[Corresponding author.\\]{chenjie666@sjtu.edu.cn}
\affiliation{Key Laboratory of Artificial Structures and Quantum Control (Ministry of Education), School of Physics and Astronomy, Shenyang National Laboratory for Materials Science, Shanghai Jiao Tong University, Shanghai 200240, China}

\author{Chun~Chen}
\email[Corresponding author.\\]{chunchen@sjtu.edu.cn}
\affiliation{Key Laboratory of Artificial Structures and Quantum Control (Ministry of Education), School of Physics and Astronomy, Shenyang National Laboratory for Materials Science, Shanghai Jiao Tong University, Shanghai 200240, China}

\author{Xiaoqun~Wang}
\email[Corresponding author.\\]{xiaoqunwang@zju.edu.cn}
\affiliation{Key Laboratory of Artificial Structures and Quantum Control (Ministry of Education), School of Physics and Astronomy, Shenyang National Laboratory for Materials Science, Shanghai Jiao Tong University, Shanghai 200240, China}
\affiliation{School of Physics, Zhejiang University, Hangzhou 310058, Zhejiang, China}
\affiliation{Tsung-Dao Lee Institute, Shanghai Jiao Tong University, Shanghai 200240, China}
\affiliation{Collaborative Innovation Center of Advanced Microstructures, Nanjing University, Nanjing 210093, China}

\date{\today}

\begin{abstract}

Many-body localization (MBL) features long-time persistence of charge-density-like waves (CDWs) of local observables. Is it practical to commence from a modulated state pattern also for nonlocal quantum entanglement? Will such entanglement analogs of CDWs survive still in MBL? From a constituent viewpoint, a great deal of MBL is learnt from $1$D spin or fermion systems where carriers are scatter particles. What about the situation when multiple interacting particles cluster in a random-potential background? To tackle these questions, we study symmetry-resolved entanglement entropy in disordered Bose-Hubbard (dBH) chain using numerical quantum quench dynamics. We concentrate on $2$ types of inhomogeneous initial states after mapping out the energy-resolved dynamical phase diagram of the model. From time-evolving a line-shape initial product state, we find the sudden formation of robust entropy imbalance across different symmetry sectors, termed entanglement-channel wave (ECW). Intriguingly, ECW melts in MBL under strong-disorder limit. It is tempting to conjecture that melting of ECW and freezing of CDW are duo traits inherent to disorder-induced MBL. Further, by exploiting dynamical consequences of loading bosons onto one site, we find the possibility to realize an interaction-facilitated cluster localization unique to BH-type models at weak disorders. Together, the unraveled rich entanglement dynamics manifests the intrinsic complexity of dBH model from a jointly energy- and symmetry-resolved perspective. 

\end{abstract}

\maketitle

{\it {\color{blue}Introduction.}}---MBL comprises a paradigm of nonergodic eigenstate matter beyond Anderson insulator \cite{Anderson,Basko2,Gornyi,Oganesyan,PalPRB,Huse,Abanin}. The interplay between randomness and interaction renders it essential to approach this nonequilibrium problem from the many-body wavefunctions. Since the early days, entanglement entropy and its quantum quench dynamics are deployed to probe the slow albeit unrestricted information propagation in MBL systems driven by dephasing \cite{Znidaric,BardarsonPollmannMoore,Serbyn2013}. A recent advance akin to the entanglement growth is the imposition of symmetry resolution. The ensuing symmetry-resolved entanglement dynamics was first implemented using $^{87}$Rb atoms to witness the logarithmic signature of MBL in dBH chains \cite{LukinGreiner}.

%---------------------------------------------------------------------------------------
\begin{figure}[tb]	
	\begin{center}
		
		\includegraphics[scale=0.4]{phase.eps}
		
		\caption{Dynamical phase diagram of periodic dBH model by contour plotting the averaged level-spacing ratio $r$ \cite{Oganesyan,Atas} with $L=12,N=6$. $\epsilon$ and $\mu$ denote eigenenergy density and disorder strength. To minimize interaction, the initial $l$-state consists of $1$ ($0$) boson on each site of the left (right) half-chain whose energy density is traced by red line. To maximize interaction, the initial $p$-state accommodates all bosons on the leftmost site, leaving the remaining chain unoccupied, whose energy density is delineated by green line. Note, bosons are released within left half-chain, therefore dominant tendencies of particle flows in $l$- and $p$-states are unidirectional toward right part.}
		\label{pic1}
			
	\end{center}
\end{figure}
%---------------------------------------------------------------------------------------

In this Letter, we investigate MBL by highlighting the necessity to integrate energy resolution in initial-state preparation with symmetry resolution in entanglement decomposition for the quench dynamics of dBH chain. Physically, it proves crucial to first scrutinize the dynamical phase diagram of the model (Fig.~\ref{pic1}). $4$ regimes with distinct behaviors are identified. As scatter bosons are low-energy carriers, the lower-lying portion of Fig.~\ref{pic1}, accessible via a line-shape initial product state ($l$-state), resembles the whole phase diagrams of typical disordered spin/fermion chains \cite{Luitz}. Interestingly, an ECW emerges from $l$-state in the coordinates of symmetry index and time. Unlike the freezing of CDW, ECW symmetrizes in strong-disorder limits, implying these $2$ phenomena might be \lq\lq head and tail'' of disorder-induced MBL. The higher-energy portion of Fig.~\ref{pic1} is unique to bosonic systems (bosons cluster). A point-shape initial product state ($p$-state) is devised to assess this regime, where a cluster MBL is stabilized at weak disorders, exhibiting prolonged inhomogeneities in both particle and entropy distributions.

{\it {\color{blue}Model \& $U(1)$-symmetry resolution.}}---The dBH chain is described by the following Hamiltonian,
\beq
H_{\textrm{dBH}}=-J\sum_{i}(a^{\dagger}_ia_{i+1}+\textrm{H.c.})+\sum_i\frac{U}{2}n_i(n_i-1)+\sum_i\mu_in_i, \nonumber
\eeq
where $a^\dagger_i,a_i$ are bosonic creation/annihilation operators at site $i$, $n_i=a^\dagger_ia_i\ (N=\sum_in_i)$ counts the local (total) boson occupation number, $U$ parametrizes the onsite Hubbard interaction, and $\mu_i\in[-\mu,\mu]$ is a diagonal random potential drawn from box distributions with disorder strength controlled by $\mu$. Note, $[N,H_{\textrm{dBH}}]=0$. Throughout this work, all relevant quantities are random averages over at least $1000$ independent realizations solved by exact diagonalization \cite{Zhang2010} and Krylov-iterative method \cite{Paeckel}. We set $J=1$ as the energy unit and fix $U=3J,N=\frac{L}{2}$ in succeeding numeric calculations.

Particle/magnetization conservation splits entanglement into different symmetry sectors, which, after proper normalization, provides a finer characterization of the underlying many-body wavefunction for both equilibrium ground state \cite{Goldstein,Xavier} and out-of-equilibrium global/local quantum quench dynamics \cite{LukinGreiner,Bonsignori,ParezPRB,KieferEmmanouilidisPRL,Luitz2020,KieferEmmanouilidisPRB,Feldman}.

Denote the total conserved operator as $Q=Q_L+Q_R$ which is separable into two parts $L/R$, then for a pure eigenstate $|\psi\rangle$ of $Q$, the reduced density matrix of $L$, $\rho_L={\sf Tr}_R(|\psi\rangle\langle\psi|)$, commutes with $Q_L$, $[\rho_L,Q_L]=0$, implying $\rho_L=\oplus_n\rho_{L,n}$ where $\rho_{L,n}$ is the assembly of blocks in $\rho_L$ that possess the same eigenvalue $n$ of $Q_L$. Because ${\sf Tr}_L\rho_L=\sum_n{\sf Tr}_{L,n}\rho_{L,n}=1$, it appears that for $|\psi\rangle$, $p_n={\sf Tr}_{L,n}\rho_{L,n}$ represents the probability of yielding eigenvalue $n$ in the projective measurement of $Q_L$. Within that subspace, the normalized reduced density matrix assumes $\widetilde{\rho}_{L,n}=p^{-1}_n\rho_{L,n}$ which satisfies ${\sf Tr}_{L,n}\widetilde{\rho}_{L,n}=1$. Consequently, the entanglement entropy of $L$ decomposes into $S_{\textrm{vN},L}=-{\sf Tr}_L\rho_L\log_2\rho_L=-\sum_n{\sf Tr}_{L,n}\rho_{L,n}\log_2\rho_{L,n}=-\sum_np_n\log_2p_n-\sum_np_n{\sf Tr}_{L,n}\widetilde{\rho}_{L,n}\log_2\widetilde{\rho}_{L,n}$. Introducing the symmetry-resolved entanglement entropy $S^n_{\textrm{vN},L}$ then yields, $S_{\textrm{vN},L}=-\sum_np_n\log_2p_n+\sum_np_nS^n_{\textrm{vN},L}$, where $S^n_{\textrm{vN},L}=-{\sf Tr}_{L,n}\widetilde{\rho}_{L,n}\log_2\widetilde{\rho}_{L,n}$ acquires the form of von Neumann entanglement entropy for assembly $n$.

Instead of number/configuration entropies, $S_N=-\sum_np_n\log_2p_n,\ S_C=\sum_np_nS^n_{\textrm{vN},L}$, which quantify the total entropies stemming from charge fluctuations across different sectors and configurational superpositions within each sector weighted by probability \cite{LukinGreiner}, we inspect the parsing of $S_{\textrm{vN},L}$ into $\{n,p_n,S^n_{\textrm{vN},L}\}$ where $p_n$ is the absolute weight and $S^n_{\textrm{vN},L}$ is determined by the relative weights. In terms of the symmetry index $n$, one can dynamically contrast $S^{n'}_{\textrm{vN},L}$ where $p_{n'}$ a maximum with $S^{N-n'}_{\textrm{vN},L}$ where $p_{N-n'}$ a minimum to uncover refined structures beyond the scheme of space-time.

{\it {\color{blue}Entanglement pattern generation via quantum quench.}}---Most quench studies of MBL start from nonentangled product states with predesigned local density imbalance imprinted \cite{SchreiberBloch,BardarsonPollmannMoore,Luitz2016}. As entanglement is absent from the start and usually builds up transiently in a continuous fashion, this construction appears structureless in the initial preparation of entanglement, despite the recognition of subsequent logarithmic entanglement growth as a fingerprint of MBL \cite{LukinGreiner}.

%---------------------------------------------------------------------------------------
\begin{figure}[tb]
	
	\centering
	
	\subfigbottomskip=0.1pt
	
	\subfigure{
		\begin{minipage}{0.5\linewidth}
			\centering
			\includegraphics[scale=0.29]{svnn_evo_1.eps}
			%\caption{fig1}
		\end{minipage}
	}%
	\subfigure{
		\begin{minipage}{0.5\linewidth}
			\centering
			\includegraphics[scale=0.29]{svnn_evo_2.eps}
			%\caption{fig2}
		\end{minipage}
	}%
	\quad
	\subfigure{
		\begin{minipage}{0.5\linewidth}
			\centering
			\includegraphics[scale=0.29]{pn_st2_mu_2.eps}
			%\caption{fig3}
		\end{minipage}
	}%
	\subfigure{
		\begin{minipage}{0.5\linewidth}
			\centering
			\includegraphics[scale=0.29]{pn_st2_mu_20.eps}
			%\caption{fig4}
		\end{minipage}
	}%
	
	\caption{Time evolution of half-chain entanglement entropy from initial $l$-state resolved into each symmetry channel labeled by number index $n$. Top and bottom rows target $S^n_{\textrm{vN},L}(t)$ and the scaling of $p_{n,\infty}$ with the chain length, while left and right columns address weak and strong disorders. A periodic chain of length $L$ and filling $N=\frac{L}{2}$ is used. Upper insets of (a),(b) illustrate the contour plots of $S^n_{\textrm{vN},L}(t)$ in the $(n,t)$ plane where ECW and its melting are demonstrated. Lower insets of (a),(b) present the scaling of the saturation values of $S^n_{\textrm{vN},L}$ at infinite time, where the $n=0,\frac{L}{2}$ components vanish identically. For illustration, error bars are preserved in saturation results but otherwise omitted in time evolutions.}
	\label{pic4}
	
\end{figure}
%---------------------------------------------------------------------------------------

Curiously, can structural features of entanglement develop discontinuously from the product state at an infinitesimal lapse of time? Intriguingly, symmetry-resolved entanglement entropies constitute an apparatus to address this question. Figures~\ref{pic4} and \ref{pic5} show the entanglement growth resolved into each number sector in the numerical quantum quench experiment. Depending on how bosons are initially populated, $2$ distinct dynamical patterns are observed. 

Start from $l$-state where scatter bosons are leading mobile identity, one novelty of the quench dynamics of entanglement is the discontinuous jump of $S^n_{\textrm{vN},L}$ from $0$ to $1$ at $t=0^+$ \cite{SuppMat}. As demonstrated by Figs.~\ref{pic4}(a),(b), for half-filled even chain with odd number of bosons and PBCs, all nontrivial $S^{n}_{\textrm{vN},L}$ jump to $1$ if $n$ is even; for those odd $n$, $S^{n}_{\textrm{vN},L}$ instead develop smoothly from $0$ up to the saturation. In analog to CDW with local particle imbalance among even/odd lattice sites, based on product line state, there arises a nonlocal entanglement imbalance among symmetry channels $n$ of opposite parities.

{\it {\color{blue}Entropy symmetrization in scatter MBL.}}---One significance of the above finding pertains to scrutinize how such an initially-engendered entanglement modulation evolves under the influence of disorders. Will this ECW freeze in localized phases? To this end, we first examine the thermal phases. As exemplified by Figs.~\ref{pic4}(a),(c), it turns out that in the weak-disorder range of the initial $l$-state, both ECWs and CDWs melt to conform with ETH \cite{Deutsch,Srednicki,DAlessio,KimPRE}. This is because under weak disorders, the reflection symmetry of the BH model in thermal regime might be assumed to be broken in a smooth manner, then for each eigenstate within the thermalization energy window, it follows $S^{N-n}_{\textrm{vN},L}\approx S^{N-n}_{\textrm{vN},R}$. Here $S^{N-n}_{\textrm{vN},L(R)}$ designates symmetry-resolved entanglement entropy of left (right) half-chain that accommodates $N-n$ bosons. Next, by virtue of $S^{N-n}_{\textrm{vN},R}=S^{n}_{\textrm{vN},L}$, valid for arbitrary pure states, one derives $S^{N-n}_{\textrm{vN},L}\approx S^{n}_{\textrm{vN},L}$, indicating early-time even/odd-$n$ entanglement imbalance disappears at long-time limit. Parallel rationale carries over to the infinite-time profile of $p_n$ (denoted as $p_{n,\infty}$): thermalization dictates that the initially inhomogeneous boson population melts into the final more uniform density landscape captured by a Gaussian distribution.

Surprisingly, we find that for the initial $l$-state, this ECW symmetrizes with respect to channels $n$ vs. $N-n$ even when subject to strong disorders [Fig.~\ref{pic4}(b)], suggesting that in scatter MBL, nonlocal ECW melts as well. By contrast, the accompanying local boson occupations $p_{n,\infty}$ remain frozen concurrently onto the initially asymmetric form [Fig.~\ref{pic4}(d)], in accord with the phenomenon of localization.

Assume the validity of LIOM phenomenology for MBL \cite{Serbyn,HuseLIOM,Ros}, then eigenstates of the system at large $\mu$ might be prescribable by filling the set of localized bits (l-bits). The diagonal approximation (arising from taking the infinite-time limit) then informs that the quasi-exponential decay of $p_{n,\infty}$ as a function of $n$ in Fig.~\ref{pic4}(d) is determined by the projection coefficients of the initial $l$-state into these l-bit eigenstates. Via interpreting these expansion coefficients as tunneling amplitudes, it is equivalently comprehensible that the probability of the corresponding multi-boson tunneling processes is exponentially suppressed in the localized regime as per a measure set by localization length. 

%---------------------------------------------------------------------------------------
\begin{figure}[b]
	
	\centering
	
	\subfigbottomskip=0.1pt
	
	\subfigure{
		\begin{minipage}{0.5\linewidth}
			\centering
			\includegraphics[scale=0.29]{svnn_evo_3.eps}
			%\caption{fig1}
		\end{minipage}
	}%
	\subfigure{
		\begin{minipage}{0.5\linewidth}
			\centering
			\includegraphics[scale=0.29]{svnn_evo_4.eps}
			%\caption{fig2}
		\end{minipage}
	}%
	\quad
	\subfigure{
		\begin{minipage}{0.5\linewidth}
			\centering
			\includegraphics[scale=0.29]{pn_st1_mu_2.eps}
			%\caption{fig3}
		\end{minipage}
	}%
	\subfigure{
		\begin{minipage}{0.5\linewidth}
			\centering
			\includegraphics[scale=0.29]{pn_st1_mu_20.eps}
			%\caption{fig4}
		\end{minipage}
	}%
	
	\caption{Time evolution of half-chain entanglement entropy from initial $p$-state resolved into each symmetry channel labeled by number index $n$. Other arrangements are parallel to that of Fig.~\ref{pic4}.}
	\label{pic5}
	
\end{figure}
%---------------------------------------------------------------------------------------

One feature of Fig.~\ref{pic4} is the coexistence of the resemblance of $S^n_{\textrm{vN},L}$ between (a),(b) with the contrast of $p_{n,\infty}$ between (c),(d). A phenomenological argument for this disparate trend might run as follows. Take channels $n=1,6$ in an $L=14,N=7$ chain as an example, then for a single random sample, the use of normalized $\widetilde{\rho}_{L,n}$ renders it possible to examine the relative arrangements of the component states within and between each individual channel of the pair. The overall discrepancy in the prefactors between the $2$ channels is thereby hidden. As (d) suggests the system is localized at $\mu=20$, the dimensions of the pertinent blocks in $\widetilde{\rho}_{L,n=1,6}$ whose entries deviate substantially from $0$ are supposed to be controlled by localization length (recall $S^{n=6}_{\textrm{vN},L}=S^{n=1}_{\textrm{vN},R}$). This constraint on the availability of multi-boson configurations in each channel, combined with consideration on the minimization of the resultant energy mismatch, implies that these dominant density-matrix blocks might largely be diagonal. However, in view of the fact that (i) the $n=1$ channel is dominated by configurations with $3$ bosons concentrated near the right and another $3$ near the left entanglement cut and the remaining $1$ boson localized at the middle of the left half-chain; (ii) the $n=6$ channel is dominated by configurations with $6$ bosons evenly distributed along the left half-chain and $1$ boson localized at either left or right cut; (iii) the onsite potentials $\mu_i$ now represent large and uncorrelated random numbers, it is therefore not guaranteed that within a single sample, the equality of the saturation values between $S^{n=1,6}_{\textrm{vN},L}$ ensues. The observed entropy symmetrization in (b) thus hints that only after averaging over a sufficient amount of random realizations, the statistical distributions of the eigenspectra of $\widetilde{\rho}_{L,n=1,6}$ tend to share some notable similarities. Analogous reasonings apply to other pairs of channels. Because the effective dimension of the leading nontrivial block in $\widetilde{\rho}_{L,n}$ increases as $n$ approaches $\frac{N}{2}$, the saturation values of $S^{n,N-n}_{\textrm{vN},L}$ accordingly get raised in a correspondingly successive way. This is consistent with the overall tendency given by (b).

{\it {\color{blue}Entropy inhomogeneity in cluster MBL.}}---One impetus behind invoking initial $l$-state is it minimizes Hubbard interaction. Now we consider the opposite extreme, the initial $p$-state, which maximizes the interaction. As displayed by Figs.~\ref{pic5}(a),(c), in this case both the $S^{n}_{\textrm{vN},L}$ evolution and the $p_{n,\infty}$ distribution alter drastically at small $\mu$.

First, once all bosons are loaded onto a site, the Hubbard term dominates the Hamiltonian, which renders single-boson tunnelings quenched as perturbations. To reduce energy mismatch, the time evolution of the initial $p$-state tends to preserve its cluster structure. Furthermore, the neighboring eigenstates available to the $p$-state also share similar cluster features to sustain their comparable energy densities \cite{SuppMat}. Consequently, within this high-energy window, the translation and reflection symmetries of the model are bound to be broken in an abrupt way by small $\mu$. Numerically, Fig.~\ref{pic5}(c) confirms the scaling trend of $p_{n,\infty}$ toward this interaction-facilitated cluster localization at weak disorders upon the increase of $L$.

Second, unlike the ECW formation in Figs.~\ref{pic4}(a),(b), starting from the initial $p$-state, $S^{n}_{\textrm{vN},L}$ grows continuously from $0$ and no ECW arises. Differing also from the long-time entropy symmetrizations of the initial $l$-state [Figs.~\ref{pic4}(a),(b)], a strong entropy inhomogeneity develops and stabilizes in the cluster MBL [Fig.~\ref{pic5}(a)]. A qualitative justification on this finding may run as follows. Take channels $n=1,6$ in an $L=14,N=7$ chain with small $\mu$ as an example. According to Fig.~\ref{pic5}(c), the $n=1$ channel is dominated by configurations with $6$ bosons moved to near the left entanglement cut and $1$ boson left within the left half-chain. While the $n=6$ channel is dominated by configurations with $6$ bosons localized around the original site and $1$ boson hopping across the left cut into the right half-chain. Because the leading energy mismatch between the initial $p$-state and the $n=6$ channel is small, any additional fluctuations induced by hoppings of single boson within right half-chain are relatively important, thus these processes are restricted and the corresponding $S^{n=6}_{\textrm{vN},L}$ gets suppressed. In comparison, the leading energy mismatch between the initial $p$-state and the $n=1$ channel is large, so comparatively fluctuations within the $n=1$ channel owing to tunnelings of single boson along left half-chain are less influential, suggesting the corresponding hopping processes are more extended. In other words, the size of the pertinent block in $\widetilde{\rho}_{L,n=1}$ increases. Accordingly, after normalization, $S^{n=1}_{\textrm{vN},L}$ becomes enhanced. In this sense, it is the significant energy gap between channels $n=1,6$, together with its interplay with weak disorders, that underpins the dynamics of cluster MBL [Figs.~\ref{pic5}(a),(c)]. Other pairs of channels can be addressed in a similar way. Thereby, manipulations upon the initial $p$-state give rise to the equilibrated coexistence of particle and entropy inhomogeneities in one unified dynamical setting.

%---------------------------------------------------------------------------------------
\begin{figure}[b]
	
	\centering
	
	\subfigbottomskip=0.1pt
	
	\subfigure{
		\begin{minipage}{0.5\linewidth}
			\centering
			\includegraphics[scale=0.29]{sn_time_evo_1.eps}
			%\caption{fig1}
		\end{minipage}
	}%
	\subfigure{
		\begin{minipage}{0.5\linewidth}
			\centering
			\includegraphics[scale=0.29]{sc_time_evo_1.eps}
			%\caption{fig2}
		\end{minipage}
	}%
	\quad
	\subfigure{
		\begin{minipage}{0.5\linewidth}
			\centering
			\includegraphics[scale=0.29]{sn_time_evo_2.eps}
			%\caption{fig3}
		\end{minipage}
	}%
	\subfigure{
		\begin{minipage}{0.5\linewidth}
			\centering
			\includegraphics[scale=0.29]{sc_time_evo_2.eps}
			%\caption{fig4}
		\end{minipage}
	}%
	\quad
	\subfigure{
		\begin{minipage}{0.5\linewidth}
			\centering
			\includegraphics[scale=0.29]{sn_time_evo_3.eps}
			%\caption{fig5}
		\end{minipage}
	}%
	\subfigure{
		\begin{minipage}{0.5\linewidth}
			\centering
			\includegraphics[scale=0.29]{sc_time_evo_3.eps}
			%\caption{fig6}
		\end{minipage}
	}%
	
	\caption{Quench dynamics of half-chain $S_N,S_C$. The $1$st row focuses on weak-disorder regime where scatter ETH and cluster MBL are realizable by commencing from $l$- and $p$-state. The $2$nd ($3$rd) row is devoted to scatter (cluster) MBL at large $\mu$. To evolve longer chains $(L=16,18)$, Krylov-iterative method \cite{Paeckel} is employed.}
	\label{pic2}
	
\end{figure}
%---------------------------------------------------------------------------------------

Finally, the above physical picture carries over to the strongly-disordered circumstance with the addition that now each boson is localized by disorder as evident from Fig.~\ref{pic5}(d), thus being confined to regions set by localization length. That explains why entropy inhomogeneity is reduced at $\mu=20$ [Fig.~\ref{pic5}(b)], in consistency with the entropy-symmetrization hypothesis for the disorder-driven MBL.         

%---------------------------------------------------------------------------------------
\def\arraystretch{1.2}
%\setlength{\tabcolsep}{5pt}
\begin{table}[b]
	\caption{Dynamic $S_N,S_C$ characteristics for dBH model.} 
	\label{table1}
	\centering
	\begin{tabular}{|c|c|cccc|} 
		\hline
		\multicolumn{2}{|c|}{\multirow{3}{*}{phase}} & \multicolumn{1}{c|}{\multirow{3}{*}{ETH}} & \multicolumn{3}{c|}{MBL}    \\
		\cline{4-6}  
		\multicolumn{2}{|c|}{}                       & \multicolumn{1}{c|}{}                     & \multicolumn{1}{c|}{\multirow{2}{*}{scatter}} & \multicolumn{2}{c|}{cluster}                 \\
		\cline{5-6}
		\multicolumn{2}{|c|}{}                       & \multicolumn{1}{c|}{}                     & \multicolumn{1}{c|}{}                         & \multicolumn{1}{c|}{hard} & soft             \\ 
		\hline
		\multirow{2}{*}{$S_N$} & growth              & $\ln t$                                     & $\ln\ln t$                                       & \textrm{no}               & \textrm{no}      \\ 
		\cline{2-2}
		& scaling                 & $\ln L$                                     & \textrm{area}                                 & \textrm{area}             & \textrm{area}   \\ 
		\cline{1-2}
		\multirow{2}{*}{$S_C$} & growth              & $t$                           & $\ln t$                                         & \textrm{no}               & $\ln\ln t$          \\ 
		\cline{2-2}
		& scaling                 & \textrm{volume}                           & \textrm{volume}                               & \textrm{area}             & \textrm{area}  \\
		\hline
	\end{tabular}
\end{table}
%---------------------------------------------------------------------------------------

{\it {\color{blue}Scatter vs. cluster MBLs from $S_N,S_C$.}}---For completeness, we summarize in Fig.~\ref{pic2} the major characteristics of the $4$ types of quench dynamics thus far using complementary evolutions of number and configuration entropies \cite{SuppMat}.

Figures~\ref{pic2}(a),(b) target regime of small $\mu$. For the thermalization dome accessible via initialization in $l$-state, besides reproducing linear-$t$ and linear-$L$ scalings for the growth and the saturation of $S_C$ [inset of Fig.~\ref{pic2}(b)], we find the temporal build-up of $S_N$ obeys a log function of $t$ and its saturation scales as a log function of $L$ [Fig.~\ref{pic2}(a)]. By contrast, once switching to $p$-state, both $S_N,S_C$ evolutions become halted and the scalings of their saturations fulfill area law. The cluster MBL at small $\mu$ thus features a bounded $S_{\textrm{vN}}$ growth and an interaction-stabilized entropy inhomogeneity. The latter never occurs in Anderson insulator. Figures~\ref{pic2}(c)-(f) present the $S_N,S_C$ results for the strongly-disordered regime. Specifically, (c) shows while $S_N$ growth in scatter MBL follows a double-log function of $t$, the scaling of its saturation is obedient to area law. This is in accordance with the result of Fig.~\ref{pic4}(d) but differs from the claim in \cite{KieferEmmanouilidisPRL,KieferEmmanouilidisPRB}. The companion $\ln t$ rise of $S_C$ in scatter MBL, along with its volume-law scaling, is revealed by (d). Interestingly, the cluster MBL at large $\mu$ exhibits no appreciable temporal growth of $S_N$, which thus obeys area scaling law as evidenced by Fig.~\ref{pic2}(e). The accompanying $S_C$, however, grows as a tentatively double-log function of $t$ but likely saturates to area law as well at the long-time limit [Fig.~\ref{pic2}(f)]. Table~\ref{table1} recaps the features of $S_N,S_C$ to differentiate the $4$ dynamical regions in dBH chain.

{\it {\color{blue}Summary \& outlook.}}---We numerically explore the energy-resolved dynamical phase diagram of $1$D dBH model using quantum quench evolutions of symmetry-resolved entanglement entropies. By inspection on an ECW generated from an initial product state, we hypothesize an entropy symmetrization process for the strongly-disordered MBL phases. One peculiarity of the dBH-type models revealed by present work is the appearance of an MBL region formed by clustered bosons. 

In spite of the progress so far, several questions remain open regarding the analytic understanding on the mechanism of ECW, its melting, and the description framework for cluster MBL. Protocols on how to experimentally measure the symmetry-resolved entanglement entropies comprise another promising direction for the future study. These continued efforts will reveal more surprises founded upon the interplay among symmetry, entanglement, randomness, and interaction.

\bibliography{disBH}

\end{document}
