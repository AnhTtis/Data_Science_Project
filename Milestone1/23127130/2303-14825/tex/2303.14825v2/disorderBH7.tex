\documentclass[aps,prl,superscriptaddress,floatfix,twocolumn,longbibliography]{revtex4-2}

\usepackage{mathptmx}
\usepackage{charter}
\usepackage{amsmath,dcolumn,bm,amsthm,tabularx,subfigure}
\usepackage[colorlinks=true,urlcolor=blue,citecolor=blue,linkcolor=blue]{hyperref}
\usepackage{mathrsfs}
\usepackage{bbold}
\usepackage{dsfont}
\usepackage{graphicx,graphics,color}
\usepackage{mathtools,nicefrac}
\usepackage{slashed}
\usepackage{amsfonts,amssymb}
\usepackage{multirow}
\allowdisplaybreaks

\newcommand{\beq}{\begin{equation}}
\newcommand{\eeq}{\end{equation}}
\newcommand{\bea}{\begin{eqnarray}}
\newcommand{\eea}{\end{eqnarray}}
\newcommand{\widebar}{\overline}
\newcommand{\wtilde}{\widetilde}
\newcommand{\X}{\widetilde{X}}
\newcommand{\Z}{\widetilde{Z}}
\newcommand{\qp}{\mathrm{qp}}
\newcommand{\SvN}{[S_\textrm{vN}]}
\newcommand{\SvNdef}{S_\textrm{vN}}
\newcommand{\trho}{\widetilde{\rho}}
\newcommand{\I}{{I}}
\newcommand{\R}{{R}}

\begin{document}

\title{Energy- \& Symmetry-Resolved Entanglement Dynamics in Disordered Bose-Hubbard Chain}

\author{Jie~Chen}
\email[Corresponding author.\\]{chenjie666@sjtu.edu.cn}
\affiliation{Key Laboratory of Artificial Structures and Quantum Control (Ministry of Education), School of Physics and Astronomy, Shenyang National Laboratory for Materials Science, Shanghai Jiao Tong University, Shanghai 200240, China}

\author{Chun~Chen}
\email[Corresponding author.\\]{chunchen@sjtu.edu.cn}
\affiliation{Key Laboratory of Artificial Structures and Quantum Control (Ministry of Education), School of Physics and Astronomy, Shenyang National Laboratory for Materials Science, Shanghai Jiao Tong University, Shanghai 200240, China}

\author{Xiaoqun~Wang}
\email[Corresponding author.\\]{xiaoqunwang@zju.edu.cn}
\affiliation{Key Laboratory of Artificial Structures and Quantum Control (Ministry of Education), School of Physics and Astronomy, Shenyang National Laboratory for Materials Science, Shanghai Jiao Tong University, Shanghai 200240, China}
\affiliation{School of Physics, Zhejiang University, Hangzhou 310058, Zhejiang, China}
\affiliation{Tsung-Dao Lee Institute, Shanghai Jiao Tong University, Shanghai 200240, China}
\affiliation{Collaborative Innovation Center of Advanced Microstructures, Nanjing University, Nanjing 210093, China}

\date{\today}

\begin{abstract}

A great deal of many-body localization (MBL) has been learnt from the one-dimensional ($1$D) spin or fermion systems where the carriers are sort of scatter particles. By contrast, the situation when multiple interacting bosons are clustered in a random potential remains much unexplored, albeit highly demanded. By using the numerical quantum quench dynamics, we study the symmetry-resolved entanglement entropy in a disordered Bose-Hubbard (dBH) chain, concentrating on the two types of inhomogeneous initial states to target the lower- and the higher-energy section of its dynamical phase diagram. From time-evolving a line-shape initial product state, we discover a hidden order: the sudden formation of the entropy imbalance across the different symmetry sectors, resulting in an entanglement-channel wave (ECW). Intriguingly, ECW melts in the strong-disorder limit. We conjecture the melting of ECW and the freezing of CDW are duo traits inherent to the disorder-induced MBL. Relatedly, a channel-resolved reanalysis hints the previously observed double-log growth of the number entropy may not indicate the breakdown of MBL for scatter particles. Through further exploiting the dynamical consequences of loading bosons onto one site, a cluster localization unique to the BH-type models emerges at the weak disorders. Collectively, the unraveled entanglement structures and dynamics manifest the dBH model's richness from a jointly energy- and symmetry-resolved perspective. 

\end{abstract}

\maketitle

{\it {\color{blue}Introduction.}}---MBL comprises a paradigm of nonergodic eigenstate matter beyond Anderson insulator \cite{Anderson,Basko2,Gornyi,Oganesyan,PalPRB,Huse,Abanin}. The interplay between randomness and interaction renders it essential to attack this nonequilibrium problem right from the level of many-body wavefunctions. Since the early days, entanglement entropy and its quench dynamics have been widely deployed to probe the slow albeit unrestricted information propagation in MBL \cite{Znidaric,BardarsonPollmannMoore,Serbyn2013}. A recent advance akin to this reasoning is the imposition of the symmetry resolution. Particularly, the ensuing symmetry-resolved entanglement dynamics has recently been measured using $^{87}$Rb atoms to witness the logarithmic signature of MBL in dBH chains \cite{LukinGreiner}.

%---------------------------------------------------------------------------------------
\begin{figure}[t]	
	\begin{center}
		
		\includegraphics[scale=0.454]{phase.pdf}
		
		\caption{Dynamical phase diagram of the periodic dBH model by contour plotting the averaged level-spacing ratio $r$ \cite{Oganesyan,Atas} of a chain with $L=12,\ N=6$. $\varepsilon$ and $\mu$ denote the eigenenergy density and the disorder strength. To minimize the interaction, the initial $l$-state consists of one (zero) boson on each site of the left (right) half-chain whose energy density is traced by the red line. To maximize the interaction, the initial $p$-state accommodates all bosons on the leftmost site, leaving the remainder unoccupied, whose energy density is delineated by the green line. Note that all bosons are released within the left half-chain, therefore the dominant tendencies of the particle flows in $l$- and $p$-states are unidirectional toward the right part.}
		\label{pic1}
			
	\end{center}
\end{figure}
%---------------------------------------------------------------------------------------

Despite being the paradigmatic model for interacting bosons, in the context of MBL, the dBH chain is much less explored than its fermionic counterpart, the spinless $t$-$V$ model, or equivalently, the Heisenberg $XXZ$ chain via the Jordan-Wigner transformation. To our knowledge, even for the existing literature, the dynamical phase diagram of the dBH model has seldom been worked out. To this end, the central objective of this Letter is to construct and elucidate the dynamical phase diagram for the dBH model (see Fig.~\ref{pic1}). On the one hand, as the scatter bosons are the low-energy carriers, the lower-lying portion of Fig.~\ref{pic1} resembles the whole phase diagrams of the typical disordered spin or fermion chains \cite{Luitz}. On the other hand, the higher-energy portion of Fig.~\ref{pic1} is unique to the bosonic systems (bosons cluster). This binary aspect of the phase diagram, originating from the quantum statistics properties of the model, demands that the dBH chain must be studied in an energy-resolved manner.

We fuse together these two key ingredients, the symmetry resolution in the entanglement decomposition and the energy resolution in the initial-state preparation, as the correct method to scrutinize the quench dynamics for the important but understudied dBH model. Specifically, a line-shape initial product state ($l$-state) is designed to access the lower-energy section of the phase diagram, where an ECW pops out in the coordinate plane of the symmetry index and time. Unlike the freezing of CDW, ECW symmetrizes in the strong-disorder limit. Although the fate of the disorder-induced MBL is still under intense debate \cite{Suntajs,Abanin2021}, the entropy symmetrization we propose relies on the symmetry and the strong disorders but not necessarily on the MBL phase itself, thus it could be more generic. Likewise, a point-shape initial product state ($p$-state) is devised to assess the higher-energy section of Fig.~\ref{pic1}, where a cluster MBL is stabilized at the weak disorders, exhibiting the prolonged inhomogeneities in both particle and entropy distributions. As the mechanism underpinning the cluster MBL need not be the same as that for the MBL in spin or fermion systems, the challenges and critiques from the ongoing debate might not be immediately pertinent to the case at hand, thus leaving the door potentially open toward MBL in boson systems.    

{\it {\color{blue}Model \& symmetry.}}---The dBH chain is describable by the standard Hamiltonian,
\beq
H_{\textrm{dBH}}=-J\sum_{i}(a^{\dagger}_ia_{i+1}+\textrm{H.c.})+\sum_i\frac{U}{2}n_i(n_i-1)+\sum_i\mu_in_i, \nonumber
\eeq
where $a^\dagger_i\ (a_i)$ is the boson creation (annihilation) operator at site $i$, $n_i=a^\dagger_ia_i\ (N=\sum_in_i)$ counts the local (total) boson occupation number, $U$ parametrizes the onsite Hubbard interaction, and $\mu_i\in[-\mu,\mu]$ is a diagonal random potential drawn from the box distribution. Crucially, $[N,H_{\textrm{dBH}}]=0$, so the number-conserved dBH model respects the $U(1)$ symmetry. In this work, all relevant quantities are averages over at least $10^3$ random samples, solved by exact diagonalization \cite{Zhang2010} or Krylov-iterative method \cite{Paeckel}. We set $J=1$ as the energy unit and fix $U=3J,\ N=\frac{L}{2}$ in the succeeding numeric calculations.

Denote the total conserved operator as $Q=Q_L+Q_R$ which is separable into two parts $L$ and $R$, then for an eigenstate $|\psi\rangle$ of $Q$, the reduced density matrix of $L$, $\rho_L={\sf Tr}_R(|\psi\rangle\langle\psi|)$, commutes with $Q_L$, implying $\rho_L=\oplus_n\rho_{L,n}$ where $\rho_{L,n}$ the assembly of the blocks possessing the same eigenvalue $n$ of $Q_L$. Because $\sum_n{\sf Tr}_{L,n}\rho_{L,n}=1$, for $|\psi\rangle$, $p_n={\sf Tr}_{L,n}\rho_{L,n}$ represents the probability of yielding eigenvalue $n$ in the projective measurement of $Q_L$. Within that subspace, the normalized reduced density matrix assumes $\widetilde{\rho}_{L,n}=p^{-1}_n\rho_{L,n}$. Consequently, the entanglement entropy of $L$ decomposes into $S_{\textrm{vN},L}=-\sum_np_n\log_2p_n+\sum_np_nS^n_{\textrm{vN},L}$, where $S^n_{\textrm{vN},L}=-{\sf Tr}_{L,n}\widetilde{\rho}_{L,n}\log_2\widetilde{\rho}_{L,n}$ becomes the symmetry-resolved entanglement entropy for $L$ accommodating $n$ bosons.

Most previous works focus on the so-called number and configuration entropies, $S_N=-\sum_np_n\log_2p_n,\ S_C=\sum_np_nS^n_{\textrm{vN},L}$, which quantify the total entropies from the charge fluctuations across different sectors and the configurational superpositions within each sector weighted by probability \cite{Goldstein,Xavier,LukinGreiner,Bonsignori,ParezPRB,KieferEmmanouilidisPRL,Luitz2020,KieferEmmanouilidisPRB,Feldman}. The drawback of using $S_N$ and $S_C$ is that they are not independent. Instead, we inspect the parsing of $S_{\textrm{vN},L}$ into the set $\{n,p_n,S^n_{\textrm{vN},L}\}$ where $p_n$ is the absolute weight and $S^n_{\textrm{vN},L}$ is determined by the relative weight. In terms of $n$, one can dynamically contrast $S^{n'}_{\textrm{vN},L}$ where $p_{n'}$ a maximum with $S^{N-n'}_{\textrm{vN},L}$ where $p_{N-n'}$ a minimum to uncover the pure entanglement structures beyond the usual scheme of space and time.

Experimentally, time evolutions for both $p_n$ itself and a correlator $C_n\coloneqq\sum_{\{L_n\}}\sum_{\{R_{N-n}\}}|p(L_n\otimes R_{N-n})-p(L_n)p(R_{N-n})|$ complementary to $S^n_{\textrm{vN},L}$ were measured in \cite{LukinGreiner}. Here, $\{L_n\}~(\{R_{N-n}\})$ denotes all possible configurations with $n~(N-n)$ particles in the left (right) half-chain. $C_n$ is not exactly $S^n_{\textrm{vN},L}$, but \cite{LukinGreiner} suggests it might capture the qualitative characteristics of $S^n_{\textrm{vN},L}$ through the quantification of the separability between $L$ and $R$. In this regard, our predictions below may largely be observable.

{\it {\color{blue}Create entanglement pattern in lower-energy section.}}---Most quench studies of MBL start from nonentangled product states with predesigned local density imbalance imprinted \cite{SchreiberBloch,BardarsonPollmannMoore,Luitz2016}. As entanglement is absent from the start and usually builds up transiently in a continuous fashion, this construction appears structureless in the initial preparation of entanglement.

%---------------------------------------------------------------------------------------
\begin{figure}[tb]
	
	\begin{center}
		
	\includegraphics[scale=0.93]{main_pic2.pdf}
	
	\caption{Time evolution of the half-chain entanglement entropy from the initial $l$-state resolved into each symmetry channel labeled by the number index $n$. A periodic chain of length $L$ and filling $N=\frac{L}{2}$ is used. The top and bottom rows target $S^n_{\textrm{vN},L}(t)$ and the scaling of $p_{n,\infty}$ with $L$, while the left and right columns address the weak and strong disorders. Upper insets of (a),(b) illustrate the contour plots of $S^n_{\textrm{vN},L}(t)$ in the $(n,t)$ plane where the ECW and its melting are demonstrated. Lower insets of (a),(b) present the scaling of the saturation values of $S^n_{\textrm{vN},L}$ at the infinite time (the $n=0,\frac{L}{2}$ components vanish identically). For illustration, error bars are preserved in the saturation results but omitted otherwise in the time evolutions.}
	\label{pic4}
	
	\end{center}
	
\end{figure}
%---------------------------------------------------------------------------------------

Curiously, can structural features of entanglement evolve discontinuously from the product state at an infinitesimal lapse of time? Intriguingly, the symmetry-resolved entanglement entropy constitutes an ideal apparatus to address this question. Figures~\ref{pic4} and \ref{pic5} show the entanglement growth resolved into each number sector in the numerical quantum quench experiment. Depending on how bosons are initially populated, two distinct dynamical patterns are observed. 

Start from the $l$-state where the scatter bosons are the leading mobile identity, one novelty of the quench dynamics of entanglement is the discontinuous jump of $S^n_{\textrm{vN},L}$ from $0$ to $1$ at $t=0^+$ (see \cite{SuppMat} for a derivation). As demonstrated by Figs.~\ref{pic4}(a),(b), for half-filled even chain with odd number of bosons and PBCs, all nontrivial $S^{n}_{\textrm{vN},L}$ jump to $1$ if $n$ is even; for those odd $n$, $S^{n}_{\textrm{vN},L}$ instead develop smoothly from $0$ up to the saturation. In analog to the CDW with local particle imbalance among even/odd lattice sites, based on the product line state, there arises a nonlocal entanglement imbalance among the symmetry channels $n$ of the opposite parities.

{\it {\color{blue}Entropy symmetrization in lower-energy section.}}---One significance of the above finding pertains to scrutinize the influence of disorders. Will this ECW freeze in the localized phases? To set the stage, we first examine the thermal phases. As exemplified by Figs.~\ref{pic4}(a),(c), it proves that in the weak-disorder regime for the initial $l$-state, both ECW and CDW melt to conform with ETH \cite{Deutsch,Srednicki,DAlessio,KimPRE}. This is because under weak disorders, the reflection symmetry of the clean BH model is broken through a smooth manner, then for each eigenstate within the thermalization energy window, it follows $S^{N-n}_{\textrm{vN},L}\approx S^{N-n}_{\textrm{vN},R}$. Next, by virtue of $S^{N-n}_{\textrm{vN},R}=S^{n}_{\textrm{vN},L}$, valid for arbitrary pure states, one derives $S^{N-n}_{\textrm{vN},L}\approx S^{n}_{\textrm{vN},L}$, indicating the early-time even/odd-$n$ entanglement imbalance disappears at the long-time limit. Parallel rationale carries over to the infinite-time profile of $p_n$ (denoted as $p_{n,\infty}$): thermalization dictates the initially inhomogeneous boson population melts into the final uniform density landscape captured by a Gaussian.

Interestingly, we find for the initial $l$-state, this ECW symmetrizes with respect to channels $n$ versus $N-n$ even when subject to strong disorders [Fig.~\ref{pic4}(b)], suggesting in scatter MBL, nonlocal ECW melts. Concurrently, the companion local boson occupations $p_{n,\infty}$ remain frozen onto the initially asymmetric form [Fig.~\ref{pic4}(d)], in accord with the localization phenomenon.

Assume the LIOM phenomenology for MBL \cite{Serbyn,HuseLIOM,Ros}, then the system's eigenstates at large $\mu$ might be prescribable by filling the localized bits (l-bits). The diagonal approximation from taking the infinite-time limit then informs that the quasi-exponential decay of $p_{n,\infty}$ as a function of $n$ in Fig.~\ref{pic4}(d) is determined by the projection coefficients of the initial $l$-state into these l-bit eigenstates. Via interpreting these expansion coefficients as the tunneling amplitudes, it is comprehensible that the probability of the corresponding multi-boson tunneling processes is exponentially suppressed in the localized regime as per a measure set by the localization length.

%---------------------------------------------------------------------------------------
\begin{figure}[b]
	
	\begin{center}
		
	\includegraphics[scale=0.95]{main_pic3.pdf}
	
	\caption{Time evolution of the half-chain entanglement entropy from the initial $p$-state resolved into each symmetry channel labeled by $n$. Other arrangements are parallel to that of Fig.~\ref{pic4}.}
	\label{pic5}
	
	\end{center}
	
\end{figure}
%---------------------------------------------------------------------------------------

One salient feature of Fig.~\ref{pic4} is the coexistence of the resemblance of $\{S^n_{\textrm{vN},L}\}$ between (a),(b) with the contrast of $\{p_{n,\infty}\}$ between (c),(d). A phenomenological argument for this disparate trend might run as follows. Take channels $n=1,6$ in an $L=14,\ N=7$ chain as an example, then for a single sample, the use of the normalized $\widetilde{\rho}_{L,n}$ makes it possible to examine the relative arrangements of the component states within and between each individual channel of the pair. The overall discrepancy in the prefactors between the two is hidden. As (d) suggests the system is localized at $\mu=20$, the dimensions of the pertinent nonzero blocks in $\widetilde{\rho}_{L,n=1,6}$ are controlled by the localization length. This constraint on the multi-boson configurations in each channel, combined with the minimization of the resultant energy mismatch, implies that these dominant density-matrix blocks might largely be diagonal. However, in view of the fact that (i) the $n=1$ channel is dominated by configurations with $3$ bosons concentrated near the right and another $3$ near the left entanglement cut and the remaining $1$ localized at the middle of the left half-chain; (ii) the $n=6$ channel is dominated by configurations with $6$ bosons evenly distributed along the left half-chain and $1$ boson localized at either the left or the right cut; (iii) the onsite potentials $\mu_i$ now represent large and uncorrelated random numbers, it is not guaranteed that within a single sample, the equality of the saturation values between $S^{n=1,6}_{\textrm{vN},L}$ ensues. The observed entropy symmetrization in (b) thus hints that only after averaging over a sufficient amount of random realizations, the statistical distributions of the eigenspectra of $\widetilde{\rho}_{L,n=1,6}$ tend to share some notable similarities. Analogous reasonings apply to other pairs of channels. Because the effective dimension of the leading nontrivial block in $\widetilde{\rho}_{L,n}$ increases as $n$ approaches $\frac{N}{2}$, the saturation values of $S^{n,N-n}_{\textrm{vN},L}$ get raised in a successive way. This is consistent with the overall tendency seen in (b).

{\it {\color{blue}Entropy inhomogeneity in higher-energy section.}}---Now we switch to the opposite extreme, the initial $p$-state, which maximizes the interaction. As displayed by Figs.~\ref{pic5}(a),(c), in this case both the $\{S^{n}_{\textrm{vN},L}\}$ evolution and the $\{p_{n,\infty}\}$ distribution alter drastically at small $\mu$.

First, once all bosons are loaded onto a site, the Hubbard term dominates the Hamiltonian, which renders single-boson tunnelings quenched as perturbations. To reduce the energy mismatch, the time evolution of the $p$-state tends to preserve its cluster structure. Further, the neighboring eigenstates available to the $p$-state also share similar cluster features to sustain their comparable energy densities \cite{SuppMat}. Consequently, within this high-energy window, the translation and reflection symmetries of the model are bound to be broken in an abrupt way by the small $\mu$. Numerically, Fig.~\ref{pic5}(c) confirms the scaling trend of $p_{n,\infty}$ toward this interaction-facilitated cluster localization at weak disorders upon increasing $L$.

Second, unlike the ECW formation in Figs.~\ref{pic4}(a),(b), starting from the $p$-state, $S^{n}_{\textrm{vN},L}$ grows continuously from zero and no ECW arises. Differing also from the long-time entropy symmetrizations in Figs.~\ref{pic4}(a),(b), a strong entropy inhomogeneity develops in the cluster MBL region [Fig.~\ref{pic5}(a)]. A qualitative justification for this may run as follows. Take channels $n=1,6$ in an $L=14,~N=7$ chain with small $\mu$ as an example. According to Fig.~\ref{pic5}(c), the $n=1$ channel is dominated by configurations with $6$ bosons moved to near the left entanglement cut and $1$ boson left within the left half-chain. While the $n=6$ channel is dominated by configurations with $6$ bosons localized around the original site and $1$ boson hopping across the left cut into the right half-chain. Because the leading energy mismatch between the initial $p$-state and the $n=6$ channel is small, any additional fluctuations induced by hoppings of single boson within the right half-chain are relatively important, thus these processes are restricted and the corresponding $S^{n=6}_{\textrm{vN},L}$ gets suppressed. In comparison, the leading energy mismatch between the initial $p$-state and the $n=1$ channel is large, so comparatively, the fluctuations within the $n=1$ channel owing to the single-boson tunnelings along the left half-chain are less influential, suggesting the corresponding hopping processes are more extended, i.e., the size of the pertinent block in $\widetilde{\rho}_{L,n=1}$ increases. Accordingly, after normalization, $S^{n=1}_{\textrm{vN},L}$ becomes enhanced. In this sense, it is the significant energy gap between channels $n=1,6$, together with its interplay with the weak disorders, that underpins the dynamics of cluster MBL [Figs.~\ref{pic5}(a),(c)]. Other pairs of channels can be addressed in a similar way. Through manipulating the initial $p$-state, we therefore find the equilibrated coexistence of the particle and entropy inhomogeneities in one unified dynamical setting.

%---------------------------------------------------------------------------------------
\begin{figure}[b]
	
	\begin{center}
		
	\includegraphics[scale=0.935]{main_pic4.pdf}
	
	\caption{Quench dynamics of the half-chain $S_N,~S_C$. The first row focuses on the weak-disorder regime where the scatter ETH and the cluster MBL are realizable by commencing from the $l$- and $p$-state. The second (third) row is devoted to the scatter (cluster) MBL at large $\mu$. To evolve the longer chains $(L=16,18)$, the Krylov-iterative method \cite{Paeckel} is employed.}
	\label{pic2}
	
	\end{center}
	
\end{figure}
%---------------------------------------------------------------------------------------

Finally, the above picture carries over to the strongly-disordered circumstance with the addition that now each boson is localized by disorder as evident from Fig.~\ref{pic5}(d), thus being confined to regions set by localization length. This explains why the entropy inhomogeneity is reduced at $\mu=20$ [Fig.~\ref{pic5}(b)], in consistency with the entropy-symmetrization hypothesis for the disorder-driven MBL.         

{\it {\color{blue}Dynamic distinctions from $S_N$ \& $S_C$.}}---For completeness, Fig.~\ref{pic2} summarizes the characteristics of the four types of quench dynamics thus far using the complementary evolutions of the number and configuration entropies \cite{SuppMat}.

Figures~\ref{pic2}(a),(b) target the regime of small $\mu$. For the thermalization dome accessible via the $l$-state, besides reproducing the linear-$t$ and linear-$L$ scalings for the growth and saturation of $S_C$ [inset of Fig.~\ref{pic2}(b)], we find the temporal build-up of $S_N$ obeys a log function of $t$ and its saturation scales as a log function of $L$ [Fig.~\ref{pic2}(a)]. By contrast, once switch to the $p$-state, both the $S_N,~S_C$ evolutions become halted and the scalings of their saturations fulfill the area law. The cluster MBL at small $\mu$ hence features a bounded $S_{\textrm{vN}}$ growth and an interaction-stabilized entropy inhomogeneity. The latter never occurs in Anderson insulator. Figures~\ref{pic2}(c)-(f) present the $S_N,~S_C$ results for the strong-disorder regime. (c) shows that the $S_N$ growth in scatter MBL follows a double-log function of $t$. Via a channel-resolved reanalysis, we show in \cite{SuppMat} that the perceived particle-number fluctuations occur mainly as the reorganizations within the initial number channels where the particles are released rather than the substantial particle transports involving all available channels, in particular, those remote ones. In other words, the scaling of the $S_N$ saturation is obedient to the area law. This is in accord with Fig.~\ref{pic4}(d) but differs from the claim in \cite{KieferEmmanouilidisPRL,KieferEmmanouilidisPRB}. The companion $\ln t$ rise of $S_C$ in scatter MBL, along with its volume-law scaling, is revealed by (d). Interestingly, the cluster MBL at large $\mu$ exhibits no appreciable temporal growth in $S_N$, which thereby obeys an area scaling law as evidenced by Fig.~\ref{pic2}(e). The accompanying $S_C$, however, grows as a tentatively double-log function of $t$, but likely saturates to the area law as well at the long-time limit [Fig.~\ref{pic2}(f)]. Table~\ref{table1} recaps the features of $S_N,~S_C$ to help differetiate between the four dynamical regions in the dBH chain.

%---------------------------------------------------------------------------------------
\def\arraystretch{1.2}
\begin{table}[t]
	\caption{Dynamic $S_N,~S_C$ characteristics for the dBH model.} 
	\label{table1}
	\centering
	\begin{tabular}{|c|c|cccc|} 
		\hline
		\multicolumn{2}{|c|}{\multirow{3}{*}{phase}} & \multicolumn{1}{c|}{\multirow{3}{*}{ETH}} & \multicolumn{3}{c|}{MBL}    \\
		\cline{4-6}  
		\multicolumn{2}{|c|}{}                       & \multicolumn{1}{c|}{}                     & \multicolumn{1}{c|}{\multirow{2}{*}{scatter}} & \multicolumn{2}{c|}{cluster}                 \\
		\cline{5-6}
		\multicolumn{2}{|c|}{}                       & \multicolumn{1}{c|}{}                     & \multicolumn{1}{c|}{}                         & \multicolumn{1}{c|}{hard} & soft             \\ 
		\hline
		\multirow{2}{*}{$S_N$} & growth              & $\ln t$                                     & $\ln\ln t$                                       & \textrm{no}               & \textrm{no}      \\ 
		\cline{2-2}
		& scaling                 & $\ln L$                                     & \textrm{area}                                 & \textrm{area}             & \textrm{area}   \\ 
		\cline{1-2}
		\multirow{2}{*}{$S_C$} & growth              & $t$                           & $\ln t$                                         & \textrm{no}               & $\ln\ln t$          \\ 
		\cline{2-2}
		& scaling                 & \textrm{volume}                           & \textrm{volume}                               & \textrm{area}             & \textrm{area}  \\
		\hline
	\end{tabular}
\end{table}
%---------------------------------------------------------------------------------------

{\it {\color{blue}Summary \& outlook.}}---We numerically explore the energy-resolved dynamical phase diagram of the $1$D dBH model using the quench evolutions of the symmetry-resolved entanglement entropies. By inspection on an ECW generated from an initial product state, we hypothesize an entropy symmetrization process for the strongly-disordered phases. One peculiarity of the dBH-type models revealed by the present work is the appearance of an MBL region formed by the clustered bosons. 

In spite of the progress so far, several questions remain open regarding the analytic understanding on the mechanism of the ECW, its melting, and the description framework for the cluster MBL. Protocols on how to experimentally measure the symmetry-resolved entanglement entropies comprise another promising direction for the future study. These continued efforts will reveal more surprises founded upon the interplay among statistics, symmetry, entanglement, randomness, and interaction.

We thank Z. Cai for the insightful discussions. J.C. and X.W. were supported by MOST2022YFA1402701 and the NSFC Grant No.~11974244. C.C. was supported by a start-up fund from SJTU.

\bibliography{disBH}

\end{document}
