% Version 1.2 of SN LaTeX, November 2022
%
% See section 11 of the User Manual for version history 
%
%%%%%%%%%%%%%%%%%%%%%%%%%%%%%%%%%%%%%%%%%%%%%%%%%%%%%%%%%%%%%%%%%%%%%%
%%                                                                 %%
%% Please do not use \input{...} to include other tex files.       %%
%% Submit your LaTeX manuscript as one .tex document.              %%
%%                                                                 %%
%% All additional figures and files should be attached             %%
%% separately and not embedded in the \TeX\ document itself.       %%
%%                                                                 %%
%%%%%%%%%%%%%%%%%%%%%%%%%%%%%%%%%%%%%%%%%%%%%%%%%%%%%%%%%%%%%%%%%%%%%

%%\documentclass[referee,sn-basic]{sn-jnl}% referee option is meant for double line spacing

%%=======================================================%%
%% to print line numbers in the margin use lineno option %%
%%=======================================================%%

%%\documentclass[lineno,sn-basic]{sn-jnl}% Basic Springer Nature Reference Style/Chemistry Reference Style

%%======================================================%%
%% to compile with pdflatex/xelatex use pdflatex option %%
%%======================================================%%

%%\documentclass[pdflatex,sn-basic]{sn-jnl}% Basic Springer Nature Reference Style/Chemistry Reference Style


%%Note: the following reference styles support Namedate and Numbered referencing. By default the style follows the most common style. To switch between the options you can add or remove �Numbered� in the optional parenthesis. 
%%The option is available for: sn-basic.bst, sn-vancouver.bst, sn-chicago.bst, sn-mathphys.bst. %  
 
%%\documentclass[sn-nature]{sn-jnl}% Style for submissions to Nature Portfolio journals
%%\documentclass[sn-basic]{sn-jnl}% Basic Springer Nature Reference Style/Chemistry Reference Style
\documentclass[sn-mathphys,Numbered]{sn-jnl}% Math and Physical Sciences Reference Style
%%\documentclass[sn-aps]{sn-jnl}% American Physical Society (APS) Reference Style
%%\documentclass[sn-vancouver,Numbered]{sn-jnl}% Vancouver Reference Style
%%\documentclass[sn-apa]{sn-jnl}% APA Reference Style 
%%\documentclass[sn-chicago]{sn-jnl}% Chicago-based Humanities Reference Style
%%\documentclass[default]{sn-jnl}% Default
%%\documentclass[default,iicol]{sn-jnl}% Default with double column layout

%%%% Standard Packages
%%<additional latex packages if required can be included here>

\usepackage{graphicx}%
\usepackage{multirow}%
\usepackage{amsmath,amssymb,amsfonts}%
\usepackage{amsthm}%
\usepackage{mathrsfs}%
\usepackage[title]{appendix}%
\usepackage{xcolor}%
\usepackage{textcomp}%
\usepackage{manyfoot}%
\usepackage{booktabs}%
\usepackage{algorithm}%
\usepackage{algorithmicx}%
\usepackage{algpseudocode}%
\usepackage{listings}%
%%%%

%%%%%=============================================================================%%%%
%%%%  Remarks: This template is provided to aid authors with the preparation
%%%%  of original research articles intended for submission to journals published 
%%%%  by Springer Nature. The guidance has been prepared in partnership with 
%%%%  production teams to conform to Springer Nature technical requirements. 
%%%%  Editorial and presentation requirements differ among journal portfolios and 
%%%%  research disciplines. You may find sections in this template are irrelevant 
%%%%  to your work and are empowered to omit any such section if allowed by the 
%%%%  journal you intend to submit to. The submission guidelines and policies 
%%%%  of the journal take precedence. A detailed User Manual is available in the 
%%%%  template package for technical guidance.
%%%%%=============================================================================%%%%

%\jyear{2021}%

%% as per the requirement new theorem styles can be included as shown below
\theoremstyle{thmstyleone}%
\newtheorem{theorem}{Theorem}%  meant for continuous numbers
%%\newtheorem{theorem}{Theorem}[section]% meant for sectionwise numbers
%% optional argument [theorem] produces theorem numbering sequence instead of independent numbers for Proposition
\newtheorem{proposition}[theorem]{Proposition}% 
%%\newtheorem{proposition}{Proposition}% to get separate numbers for theorem and proposition etc.
\newtheorem{lemma}{Lemma}
\newtheorem{corollary}{Corollary}

\theoremstyle{thmstyletwo}%
\newtheorem{example}{Example}%
\newtheorem{remark}{Remark}%

\theoremstyle{thmstylethree}%
\newtheorem{definition}{Definition}%

\raggedbottom
%%\unnumbered% uncomment this for unnumbered level heads

\begin{document}

\title[CSIRMs to K\"ahler manifolds]{Clairaut semi-invariant Riemannian maps to K\"ahler manifolds}

%%=============================================================%%
%% Prefix	-> \pfx{Dr}
%% GivenName	-> \fnm{Joergen W.}
%% Particle	-> \spfx{van der} -> surname prefix
%% FamilyName	-> \sur{Ploeg}
%% Suffix	-> \sfx{IV}
%% NatureName	-> \tanm{Poet Laureate} -> Title after name
%% Degrees	-> \dgr{MSc, PhD}
%% \author*[1,2]{\pfx{Dr} \fnm{Joergen W.} \spfx{van der} \sur{Ploeg} \sfx{IV} \tanm{Poet Laureate} 
%%                 \dgr{MSc, PhD}}\email{iauthor@gmail.com}
%%=============================================================%%
\author[1]{\fnm{Murat} \sur{Polat}}\email{murat.polat@dicle.edu.tr}

\author*[2]{\fnm{Kiran} \sur{Meena}}\email{kirankapishmeena@gmail.com}


%\equalcont{These authors contributed equally to this work.}

%\author[1,2]{\fnm{Third} \sur{Author}}\email{iiiauthor@gmail.com}
%\equalcont{These authors contributed equally to this work.}
\affil[1]{\orgdiv{Department of Mathematics,
		Faculty of Science}, \orgname{Dicle University}, \orgaddress{\street{Sur}, \postcode{Diyarbak-21280}, %\state{State},
		\country{Turkey}}}
	
\affil*[2]{\orgdiv{Department of Mathematics, Lady Shri Ram College for Women}, \orgname{University of Delhi}, \orgaddress{\street{Lajpat Nagar}, %\city{New Delhi},
		 \postcode{New Delhi-110024}, %\state{New Delhi}, 
\country{India}}}



%\affil[3]{\orgdiv{Department}, \orgname{Organization}, \orgaddress{\street{Street}, \city{City}, \postcode{610101}, \state{State}, \country{Country}}}

%%==================================%%
%% sample for unstructured abstract %%
%%==================================%%

\abstract{In this paper, first we recall the notion of Clairaut Riemannian map (CRM) ${F}$ using a geodesic curve on the base manifold and give the Ricci equation. We also show that if base manifold of CRM is space form then leaves of $(ker{F}_\ast)^\perp$ become space forms and symmetric as well. Secondly, we define Clairaut semi-invariant Riemannian map (CSIRM) from a Riemannian manifold $(M, g_{M})$ to a K\"ahler manifold $(N, g_{N}, P)$ with a non-trivial example. We find necessary and sufficient conditions for a curve on the base manifold of semi-invariant Riemannian map (SIRM) to be geodesic. Further, we obtain necessary and sufficient conditions for a SIRM to be CSIRM. Moreover, we find necessary and sufficient condition for CSIRM to be harmonic and totally geodesic. In addition, we find necessary and sufficient condition for the distributions $\bar{D_1}$ and $\bar{D_2}$ of $(ker{F}_\ast)^\bot$ (which are arisen from the definition of CSIRM) to define totally geodesic foliations. Finally, we obtain necessary and sufficient conditions for $(ker{F}_\ast)^\bot$ and base manifold to be locally product manifold $\bar{D_1} \times \bar{D_2}$ and $N_{(range{F}_\ast)} \times N_{(range{F}_\ast)^\bot}$, respectively.}

\keywords{K\"ahler manifolds, harmonic maps, Riemannian maps, Semi-invariant Riemannian maps, Clairaut Riemannian maps}

%%\pacs[JEL Classification]{D8, H51}

\pacs[MSC Classification]{53B20, 53B35, 53C43}

\maketitle

%\section{Introduction}\label{sec1}

\section{Introduction}
\noindent First of all we fix abbreviations C, RM, RMs, I, AI, and SI for denoting Clairaut, Riemannian map, Riemannian maps, invariant, anti-invariant and semi-invariant respectively. The detailed geometry of Riemannian submersion was investigated in \cite{Falcitelli}. As a generalization of an isometric immersion and Riemannian submersion, Fischer introduced RM between Riemannian manifolds that satisfies the well known generalized eikonal equation $\left\Vert {F}_{\ast}\right\Vert ^{2}=rank{F} ,$ which is a bridge between geometric optics and physical optics \cite{Fischer}. Further, the geometry of RMs was investigated in \cite{Sahin1}, \cite{Sahin6}, \cite{Sahin_2017b} and \cite{Sahin4}. 

An important Clairaut's relation states that $d \sin \theta$ is constant, where $\theta$ is the angle between the velocity vector of a geodesic and a meridian, and $d$ is the distance to the axis of a surface of revolution. In 1972, Bishop defined Clairaut Riemannian submersion with connected fibers and gave a necessary and sufficient condition for a Riemannian submersion to be Clairaut Riemannian submersion \cite{Bishop}. Further, Clairaut submersions were studied in \cite{Allison}, \cite{Lee} and \cite{Meena1}. In \cite{Sahin5} and \cite{Meena}, CRMs were introduced by using geodesic curve on the total and base spaces respectively, and obtained necessary and sufficient conditions for RMs to be CRMs. Further, \c{S}ahin gave an open problem to find characterizations for CRMs (see: \cite{Sahin6}, p. 165, Open Problem 2).

In \cite{Watson}, Watson studied almost Hermitian submersions. In \cite{Sahin3}, \c{S}ahin introduced holomorphic RMs as generalizations of holomorphic submersions and holomorphic submanifolds. Moreover, IRM, AIRM and SIRM were studied from a Riemannian manifold to a K\"ahler manifold in \cite{Akyol, Akyol1, Sahin, Sahin2}. Recently, in \cite{Yadav}, \cite{Meena_Thesis} and \cite{Meena}, Authors introduced CIRM and CAIRM from Riemannian manifolds to K\"ahler manifolds.

In this paper, we study CSIRMs as generalizations of CR-submanifolds, holomorphic submersions, IRMs, CIRMs, AIRMs, CAIRMs and SIRMs from Riemannian manifolds to K\"ahler manifolds. Thus CSIRMs may have possible applications in mathematical physics. The paper is organized as follows: In Section \ref{sec2}, we recall some basic definitions and facts which are needed for this paper. In Section \ref{sec3}, first we recall the notion of CRM using a geodesic curve on the base manifold and give the Ricci equation. We also show that if base manifold of CRM is space form then leaves of $(ker{F}_\ast)^\perp$ become space forms and symmetric as well. Secondly, we define CSIRM from a Riemannian manifold to a K\"ahler manifold. Then we give a non-trivial example of such CSIRM. Further, we obtain a necessary and sufficient condition for a SIRM to be CSIRM. Moreover, we find necessary and sufficient condition for CSIRM to be harmonic and totally geodesic. In addition, we find necessary and sufficient condition for the distributions $\bar{D_1}$ and $\bar{D_2}$ of $(ker{F}_\ast)^\bot$ to define totally geodesic foliations. Finally, we obtain necessary and sufficient conditions for $(ker{F}_\ast)^\bot$ to be locally product manifold of $\bar{D_1}$ and $\bar{D_2}$, and for $N$ to be locally product manifold of $N_{(range{F}_\ast)}$ and $N_{(range{F}_\ast)^\bot}$.
\newpage
\section{Preliminaries}\label{sec2}
\noindent In this section, we recall the notion of RM between Riemannian manifolds and give a brief review of basic facts.

Let ${F}$ be a smooth map between Riemannian manifolds $(M, g_{M})$ and $(N,g_{N})$ such that $0 < rank{F} < \min \left\{\dim(M), \dim (N)\right\}$. We denote the kernel space of ${F}_{\ast}$ by $V_{q}=ker {F}_{\ast {q}}$ at $q \in M$ and consider the orthogonal complementary space $H_{q}= (ker {F}_{\ast {q}})^{\bot}$ to $ker {F}_{\ast {q}}$ in $T_{q} M$. Then the tangent space $T_{q} M$ of $M$ at $q$ has the decomposition
\begin{equation*}
	T_{q} M = (ker {F}_{\ast {q}}) \oplus (ker {F}_{\ast {q}})^{\bot} = V_{q} \oplus H_{q}.
\end{equation*}
We denote the range of ${F}_{\ast}$ by $range{F}_{\ast}$ at $q \in M$ and consider the orthogonal complementary space $(range {F}_{\ast {q}})^{\bot}$ to $range {F}_{\ast{q}}$ in the tangent space $T_{{F} (q)}N$ of $N$ at ${F} (q) \in N$. Since $rank {F} < \min \left\{\dim (M), \dim (N)\right\}$, we have $(range {F}_{\ast {q}})^{\bot} \neq \left\{ 0 \right\}$. Thus the tangent space $T_{{F} (q)} N$ of $N$ at ${F} (q) \in N$ has the decomposition
\begin{equation*}
	T_{{F} (q)}N=(range{F}_{\ast _{q}})\oplus (range{F}_{\ast_{q}})^{\bot}.
\end{equation*}
Now ${F}$ is called RM at $q \in M$ if the horizontal restriction ${F}_{\ast {q}}^{h}: (ker {F}_{\ast {q}})^{\bot} \to (range {F}_{\ast {q}})$ is a linear isometry between the spaces $((ker {F}_{\ast {q}})^{\bot}, g_{M_{{q}}}|_{(ker {F}_{\ast {q}})^{\bot}})$ and $(range {F}_{\ast {q}}, g_{N {F}(q)}|_{range {F}_{\ast {q}}})$. Equivalently
\begin{equation}\label{2.1}
	g_{N}({F}_{\ast}X, {F}_{\ast}Y) = g_{M}(X,Y) 
\end{equation}
for all $X, Y$ vector field tangent to $\Gamma (ker {F}_{\ast {q}})^{\bot}$. It follows that isometric immersions and Riemannian submersions are particular RMs with $ker {F}_{\ast} = \left\{ 0 \right\}$ and $(range {F}_{\ast})^{\bot} = \left\{ 0 \right\}$, respectively. The differential map ${F}_{\ast}$ of ${F}$ can be viewed as a section of bundle $\hom (TM, {F}^{-1} TN) \to M$, where ${F}^{-1} TN$ is the pullback bundle whose fibers at $q \in M$ is $\left({F}^{-1} TN\right)_{q} = T_{{F} (q)}N$, $q \in M$. The bundle $\hom (TM, {F}^{-1} TN)$ has a connection $\nabla$ induced from the Levi-Civita connection $\nabla^{M}$ and the pullback connection $\overset{N}{\nabla^{{F}}}$. Then the second fundamental form of ${F}$ is defined as \cite{Nore} 
\begin{equation}\label{2.2}
	(\nabla {F}_{\ast})(X, Y)= \overset{N}{\nabla_{X}^{{F} }} {F}_{\ast} (Y) - {F}_{\ast}(\nabla_{X}^{M}Y)
\end{equation}
for all $X, Y \in \Gamma (TM)$, where $\overset{N}{\nabla_{X}^{{F} }} {F}_{\ast} Y \circ {F} = \overset{N}{\nabla_{{F}_{\ast}X}^{{F}}} {F}_{\ast}Y$. In \cite{Sahin4} \c{S}ahin showed that $(\nabla {F}_{\ast})(X, Y) = (\nabla {F}_{\ast})(Y, X)$. In \cite{Sahin}, \c{S}ahin proved for all $X, Y \in \Gamma (ker {F}_{\ast})^{\bot}$
\begin{equation}\label{2.3}
	(\nabla {F}_{\ast}) (X, Y) \in \Gamma (range{F}_{\ast})^{\bot}. 
\end{equation}

\begin{lemma}\label{lemmaforumbilicity}
	\cite{Sahin2} Let ${F}$ be a RM between Riemannian manifolds $(M, g_{M})$ and $(N,g_{N})$. Then ${F} $ is umbilical RM if and only if
	\begin{equation}\label{2.4}
		(\nabla {F}_{\ast})(X,Y) = g_{M}(X, Y) H
	\end{equation}
	for all $X, Y \in \Gamma (ker {F}_{\ast})^{\bot}$ and $H$ is the mean curvature vector field of $range{F}_{\ast}$.
\end{lemma}
For any vector field $X$ on $M$ and any section $V$ of $(range{F}_{\ast})^{\bot}$, we have $\nabla_{X}^{{F} \bot}V$, which is the orthogonal projection of $\nabla_{X}^{N}V$ on $(range{F}_{\ast})^{\bot}$, where $\nabla^{{F} \bot}$ is linear connection on $(range{F}_{\ast})^{\bot}$ such that $\nabla^{{F} \bot} g_{N} = 0$. In addition $(range{F}_{\ast})^{\bot}$ is totally geodesic if and only if $\nabla_{U}^{N} V = \nabla_{U}^{{F} \bot}V$ for all $U, V \in \Gamma (range{F}_{\ast})^{\bot}$ \cite{Yadav_PMD}. Now, for a RM ${F}$, we have $S_{V}$ as (\cite{Sahin4}, p. 188)
\begin{equation}\label{2.5}
	\nabla_{{F}_{\ast}X}^{N}V = - S_{V} {F}_{\ast} X + \nabla_{X}^{{F} \bot}V, 
\end{equation}
where $\nabla^{N}$ is Levi-Civita connection on $N$, $S_{V} {F}_{\ast} X$ is the tangential component (a vector field along ${F}$) of $\nabla_{{F}_{\ast}X}^{N} V$. Thus at $q \in M$, we have $\nabla_{{F}_{\ast} X}^{N}V(q) \in T_{{F} (q)} N, S_{V} {F}_{\ast} X \in {F}_{\ast q}(T_{q} M)$ and $\nabla_{X}^{{F} \bot} V \in ({F}_{\ast q}(T_{q}M))^{\bot}$. It is easy to see that $S_{V} {F}_{\ast} X$ is bilinear in $V$, and ${F}_\ast X$ at $q$ depends only on $V_{q}$ and ${F}_{\ast q} X_{q}$. 

Let $(N, g_{N})$ be an almost Hermitian manifold \cite{Yano}, then $N$ admits a tensor $P$ of type $(1, 1)$ on $N$ such that $P^{2} = - I$ and
\begin{equation}\label{2.8}
	g_{N}(P\tilde{X}, P\tilde{Y}) = g_{N}(\tilde{X}, \tilde{Y})
\end{equation}
for all $\tilde{X}, \tilde{Y} \in \Gamma (TN)$. An almost Hermitian manifold $N$ is called K\"ahler manifold if
\begin{equation*}
	(\nabla_{\tilde{X}}^{N}P)\tilde{Y} = 0,
\end{equation*}
where $\nabla^{N}$ is the Levi-Civita connection on $N$.
\begin{definition}
	\cite{Sahin2} Let ${F}$ be a RM from a Riemannian manifold $(M, g_{M})$ to an almost Hermitian manifold $(N, g_{N}, P)$. Then we say that ${F}$ is a SIRM at $q \in M$ if there are subbundles $D_{1}$ and $D_{2}$ in $range{F}_{\ast}$ such that $P(D_{1}) = D_{1}~ \& ~P(D_{2}) \subseteq (range{F}_{\ast})^{\bot}$. If ${F}$ is a SIRM at every point $q \in M$, then we say that ${F}$ is a SIRM.
\end{definition}
\noindent Then for ${F}_\ast X \in \Gamma (range{F}_{\ast})$, we write
\begin{equation}\label{3.3}
	P {F}_{\ast} X = \phi {F}_{\ast}X + \omega {F}_{\ast}X,
\end{equation}
where $\phi {F}_{\ast} X \in \Gamma (D_{1})$ and $\omega {F}_{\ast} X \in \Gamma (PD_{2})$. Also for ${F}_{\ast} X \in \Gamma(D_1)$ and ${F}_\ast Y \in \Gamma(D_2)$, we have $g_{N} ({F}_\ast X, {F}_{\ast} Y) = 0$. Thus we have two orthogonal distributions $\bar{D_1}$ and $\bar{D_2}$ such that
\[(ker {F}_\ast)^\bot = \bar{D_1} \oplus \bar{D_2}.\] In addition, for $V \in \Gamma((range {F}_{\ast})^{\bot})$, we write
\begin{equation}\label{3.4}
	PV = BV + CV, 
\end{equation}
where $BV \in \Gamma (D_{1})$ and $CV \in \Gamma (\eta)$. Here $\eta$ is the complementary orthogonal distribution to $\omega (D_{2})$ in $(range {F}_{\ast})^{\bot}$. It is easy to see that $\eta$ is invariant with respect to $P$.

\section{CSIRMs from Riemannian manifolds to K\"ahler manifolds}\label{sec3}

\noindent In this section, we introduce CSIRM from a Riemannian manifold to a K\"ahler manifold and investigate the geometry.

The notion of CRM comes from a geodesic curve on a surface of revolution. \c{S}ahin defined the CRM using a geodesic curve on the total manifold in \cite{Sahin5}. %According to the definition, a RM ${F}:(M, g_{M}) \to (N, g_{N})$ between Riemannian manifolds is called CRM if there is a function $r':M \to \mathbb{R}^{+}$ such that for every geodesic $\alpha$ on $M$, the function $(r' \circ \alpha )\sin \vartheta$ is constant, where, for all $t$, $\vartheta (t)$ is the angle between $\dot{\alpha}(t)$ and the horizontal space at $\alpha(t)$. 
Recently, CRM was defined by Meena and Yadav using a geodesic curve on the base manifold in \cite{Meena} as:

\begin{definition}\label{DEF1}
	\cite{Meena} A RM ${F}$ between Riemannian manifolds $(M, g_{M})$ and $(N,g_{N})$ is called CRM if there is a function $s:N\to \mathbb{R}^{+}$ such that for every geodesic $\beta$ on $N$, the function $(s\circ \beta )\sin \theta (t)$ is constant, where, ${F}_{\ast} X \in \Gamma (range{F}_{\ast})$ and $V\in \Gamma (range{F}_{\ast})^{\bot}$ are components of $\dot{\beta}(t)$, and $\theta (t)$ is the angle between $\dot{\beta}(t)$ and $V$ for all $t$.
\end{definition}
\noindent The authors Meena and Yadav obtained the following necessary and sufficient conditions for a RM to be CRM:
\begin{theorem}\label{TH1}
	\cite{Meena} Let ${F}$ be a RM  between Riemannian manifolds $(M, g_{M})$ and $(N,g_{N})$ such that $(range{F}_{\ast})^{\bot}$ is totally geodesic and $range{F}_\ast$ is connected, and let $\alpha, \beta = {F} \circ \alpha$ be geodesic curves on $M$ and $N$, respectively. Then ${F}$ is CRM with $s = e^{f}$ if and only if any one of the following conditions holds:
	\begin{enumerate}[(i)]
		\item $S_{V} {F}_{\ast} X = -V(f) {F}_{\ast}X$, where ${F}_{\ast} X \in \Gamma(range{F}_{\ast})$ and $V \in \Gamma (range{F}_{\ast})^{\bot}$ are components of $\dot{\beta}(t)$, i.e. $\dot{\beta}(t) = {F}_{\ast} X + V$.
		
		\item ${F}$ is umbilical map, and has $H = - \nabla^{N}f$, where $f$ is a smooth function on $N$ and $H$ is the mean curvature vector field of $range{F}_{\ast}$.
	\end{enumerate}
\end{theorem}

\begin{theorem}
	Let ${F}$ be a CRM with $s=e^{f}$ between Riemannian manifolds $(M, g_{M})$ and $(N, g_{N})$. Then the Ricci equation is
	\begin{equation*}
		[S_W, S_V] F_\ast X = 0,
	\end{equation*}
	where $X, Y \in \Gamma(kerF_\ast)^\perp$ and $W, V \in \Gamma (range{F}_{\ast})^{\bot}$.
\end{theorem}

\begin{proof}
	For a RM we have \cite{Sahin4}
	\begin{align*}
		g_N(R^N(F_\ast X, F_\ast Y)V, W)& = g_N(R^{F \perp}(F_\ast X, F_\ast Y)V, W) \\&-g_N((\nabla F_\ast)(X, {}^\ast F_\ast(S_V F_\ast Y)), W) \\& + g_N((\nabla F_\ast)(Y, {}^\ast F_\ast(S_V F_\ast X)), W),
	\end{align*}
	where $X, Y \in \Gamma(kerF_\ast)^\perp$ and $W, V \in \Gamma (range{F}_{\ast})^{\bot}$. Since $F$ is a CRM, above equation can be written as
	\begin{align*}
		g_N(R^N(F_\ast X, F_\ast Y)V, W)& = g_N(R^{F \perp}(F_\ast X, F_\ast Y)V, W) \\&+ g_M(X, {}^\ast F_\ast(S_V F_\ast Y))g_N(\nabla^N f, W)\\& - g_M(Y, {}^\ast F_\ast(S_V F_\ast X))g_N(\nabla^N f, W),
	\end{align*}
	which is equal to
	\begin{align}\label{RFperp}
		g_N(R^N(F_\ast X, F_\ast Y)V, W)& = g_N(R^{F \perp}(F_\ast X, F_\ast Y)V, W) \nonumber \\&+ g_N(F_\ast X, S_V F_\ast Y)g_N(\nabla^N f, W) \nonumber\\&- g_N(F_\ast Y, S_V F_\ast X)g_N(\nabla^N f, W).
	\end{align}
	Using Theorem \ref{TH1} in (\ref{RFperp}), we get
	\begin{align}\label{RFperpreduced}
		g_N(R^N(F_\ast X, F_\ast Y)V, W) = g_N(R^{F \perp}(F_\ast X, F_\ast Y)V, W).
	\end{align}
	We know that the Ricci equation for a RM is given by \cite{Sahin4}
	\begin{align}\label{riccieqn}
		g_N(R^N(F_\ast X, F_\ast Y)V, W)& = g_N(R^{F \perp}(F_\ast X, F_\ast Y)V, W) \nonumber\\& + g_N([S_W, S_V]F_\ast X, F_\ast Y).
	\end{align}
	Thus by (\ref{RFperpreduced}) and (\ref{riccieqn}) we obtain
	\begin{equation*}
		[S_W, S_V] F_\ast X = 0.
	\end{equation*}
\end{proof}
\begin{theorem}
	Let ${F}$ be a CRM with $s=e^{f}$ from a Riemannian manifolds $(M, g_{M})$ to a space form $(N(k), g_{N})$ such that $\|\nabla^N f\| = 1$. If $(ker F_\ast)^\perp$ is integrable distribution then any leaf of $(ker F_\ast)^\perp$ is also space form and symmetric as well. 
\end{theorem}

\begin{proof} For a RM we have the Gauss equation \cite{Sahin4} 
	\begin{align*}
		g_M(R^M(X, Y)Z, T)& = g_N(R^N(F_\ast X, F_\ast Y)F_\ast Z, F_\ast T) - g_N((\nabla F_\ast)(X, Z), (\nabla F_\ast)(Y, T)) \\&+ g_N((\nabla F_\ast)(Y, Z), (\nabla F_\ast)(X, T)),
	\end{align*}
	where $X, Y, Z ,T \in \Gamma(ker F_\ast)^\perp$. Since $N(k)$ is space form and $F$ is CRM, above equation can be written as
	\begin{align*}
		g_M(R^M(X, Y)Z, T)& = kg_N(F_\ast Y, F_\ast Z) g_N(F_\ast X, F_\ast T) - kg_N(F_\ast X, F_\ast Z) g_N(F_\ast Y, F_\ast T)\\& - g_M(X, Z) g_M(Y,T) g_N(\nabla^N f, \nabla^N f) \\&+ g_M(Y, Z) g_M(X, T) g_N(\nabla^N f, \nabla^N f).
	\end{align*}
	Since we have $\|\nabla^N f\| = 1$, we write
	\begin{align*}
		g_M(R^M(X, Y)Z, T)& = (k + 1)\{g_M(Y, Z) g_M(X, T) - g_M(X, Z) g_M(Y, T)\}.
	\end{align*}
	This implies the proof.
\end{proof}

\begin{definition}
	A SIRM from a Riemannian manifold to a K\"ahler manifold is called CSIRM if it satisfies the Definition \ref{DEF1} of CRM.
\end{definition}

\begin{example}
	Let $M$ be an Euclidean space given by
	\begin{equation*}
		M = \left\{ (u_{1}, u_{2}, u_{3}, u_{4}, u_{5}) \in \mathbb{R}^{5} : u_{1} \neq 0, u_{2} \neq 0, u_{5} \neq 0 \right\}.
	\end{equation*}
	We describe the Riemannian metric $g_{M}$ on $M$ given by
	\begin{equation*}
		g_{M} = du_{1}^{2} + du_{2}^{2} + du_{3}^{2} + du_{4}^{2} + du_{5}^{2}.
	\end{equation*}
	Let $N = \left\{(v_{1}, v_{2}, v_{3}, v_{4}) \in \mathbb{R}^{4} \right\}$ be a Riemannian manifold with Riemannian metric $g_{N}$ on $N$ given by 
	\begin{equation*}
		g_{N} = dv_{1}^{2} + dv_{2}^{2} + dv_{3}^{2} + e^{2u_{1}} dv_{4}^{2}.
	\end{equation*}
	We take the complex structure $P$ on $N$ as $P(a, b, c, d) = (-b, a, -d, c)$. 
	Then a basis of $T_q M$ is
	\begin{equation*}
		\left\{ e_{i} = \frac{\partial}{\partial u_{i}} ~\text{for $1 \leq i \leq 5$}\right\},
	\end{equation*}
	and a $P$-basis on $T_{{F}(q)} N$ is
	\begin{equation*}
		\left\{e_{j}^{\ast} = \frac{\partial}{\partial v_{j}} ~\text{for $1 \leq j \leq 3$}, e_{4}^{\ast} = e^{-u_{1}} \frac{\partial}{\partial v_{4}}\right\},
	\end{equation*}
	for all $q \in M$. Now, we define a map ${F}$ between $(M, g_{M})$ and $(N, g_{N}, P)$ by
	\begin{equation*}
		{F}(u_{1}, u_{2}, u_{3}, u_{4}, u_{5}) = (u_{1}, u_{2}, u_{5}, 0).
	\end{equation*}
	Then, we have
	\begin{equation*}
		ker {F}_{\ast} = Span \left\{U_{1} = e_{3}, U_{2} = e_{4} \right\},
	\end{equation*}
	\begin{equation*}
		(ker {F}_{\ast})^{\bot} = Span\left\{X_{1} = e_{1}, X_{2} = e_{2}, X_{3} = e_{5}\right\}.
	\end{equation*}
	It is easy to see that ${F}_{\ast} X_{1} = e_{1}^{\ast}, {F}_{\ast} X_{2} = e_{2}^{\ast}, {F}_{\ast} X_{3} = e_{3}^{\ast}$ and $g_{M} (X_{i}, X_{j}) = g_{N} ({F}_{\ast} \left( X_{i}\right), {F}_{\ast} \left( X_{j} \right))$ for $i, j = 1, 2, 3$. Thus ${F}$ is a RM with 
	\begin{equation*}
		range {F}_{\ast} = Span \left\{e_{1}^{\ast}, e_{2}^{\ast}, e_{3}^{\ast} \right\}, (range{F}_{\ast})^{\bot} = Span \left\{e_{4}^{\ast}\right\},
	\end{equation*}
	\begin{equation*}
		D_1 = Span \left\{e_{1}^{\ast}, e_{2}^{\ast}\right\}, D_2 = Span \left\{e_{3}^{\ast}\right\}.
	\end{equation*}
	Moreover it is easy to see that $P {F}_{\ast} X_{1} = e_{2}^{\ast}, P {F}_{\ast} X_{2} = - e_{1}^{\ast}, P {F}_{\ast} X_{3} = - e_{4}^{\ast}$. Thus ${F}$ is a SIRM. Now to show ${F}$ is CRM we will find smooth function $f$ on $N$ satisfying $(\nabla {F}_{\ast})(X, X) = - g_{M} (X, X) \nabla^{N} f$, for all $X \in \Gamma \left(ker {F}_{\ast}\right)^{\bot}$. 
	Since we have the only non-zero Christoffel symbols $\Gamma^{1}_{44} = -e^{2u_1} ~\text{\normalfont and}~ \Gamma^{4}_{14} = 1$ for $g_{N}$,
	by using (\ref{2.2}) we obtain $(\nabla {F}_{\ast})(X, X) = 0$ for $X = a_1 X_1 + a_2 X_2 + a_3 X_3$ where $a_is \in \mathbb{R}$. Also $\nabla^{N} f = \sum \limits_{i, j = 1}^{4} g_{N}^{ij} \frac{\partial f}{\partial v_{i}}\frac{\partial}{\partial v_{j}} \implies \nabla^{N} f = 0$ for a constant function $f$. Then it is easy to verify that $(\nabla {F}_{\ast}) (X, X) = - g_{M}(X, X) \nabla^{N} f$ for any vector field $X \in \Gamma \left(ker {F}_{\ast}\right)^{\bot}$ with a constant function $f$. Thus ${F}$ is CSIRM from a Riemannian manifold $(M, g_{M})$ to a K\"ahler manifold $(N, g_{N}, P)$.
\end{example}
Note that from now onward, we are taking the map ${F}$ such that $(range {F}_{\ast})^{\bot}$ is totally geodesic.
\begin{theorem}
	Let ${F}$ be a SIRM from a Riemannian manifold $(M, g_{M})$ to a K\"ahler manifold $(N, g_{N}, P)$ and $\alpha : I \to M$ be a geodesic curve on $M$. Then the curve $\beta = {F} \circ \alpha$ is geodesic on $N$ if and only if
	\begin{align}\label{3.5}
		&-S_{\omega {F}_{\ast} X} {F}_{\ast} X + {F}_{\ast} (\nabla_{X}^{M} {}^{\ast} {F}_{\ast} BV) - S_{CV} {F}_{\ast} X \nonumber\\& + \nabla_{V}^{N} \phi {F}_{\ast} X + \nabla_{V}^{N} BV + {F}_\ast(\nabla_X^{M} {}^{\ast}{F}_\ast \phi {F}_\ast X)= 0
	\end{align}
	and
	\begin{align}\label{3.6}
		&(\nabla {F}_\ast) (X, {}^{\ast}{F}_\ast \phi {F}_\ast X) + \nabla_{X}^{{F} \bot} \omega {F}_{\ast} X + (\nabla {F}_{\ast}) (X,{}^{\ast}{F}_{\ast} BV) \nonumber\\&+ \nabla_{X}^{{F} \bot} CV + \nabla_{V}^{{F} \bot} \omega {F}_{\ast} X + \nabla_{V}^{{F} \bot} CV = 0, 
	\end{align}
	where ${F}_{\ast} X \in \Gamma (range{F}_{\ast}), V \in \Gamma (range {F}_{\ast})^{\bot}$ are components of $\dot{\beta}(t)$ and ${}^{\ast} {F}_{\ast}$ is the adjoint map of ${F}_{\ast}$. In addition, $\nabla^{N}$ is the Levi-Civita connection on $N$ and $\nabla^{{F} \bot}$ is a linear connection on $(range{F}_{\ast})^{\bot}$.
\end{theorem}

\begin{proof}
	Let $\alpha : I \to M$ be a geodesic on $M$ and let $\beta = {F} \circ \alpha$ be a regular curve on $N$ with ${F}_{\ast} X \in \Gamma (range {F}_{\ast}), V \in \Gamma(range{F}_{\ast})^{\bot}$ are components of $\dot{\beta}(t)$. Since $N$ is K\"ahler manifold, $\nabla_{\dot{\beta}}^{N} \dot{\beta} = - P \nabla_{\dot{\beta}}^{N} P\dot{\beta}$. Thus
	\begin{equation*}
		\nabla_{\dot{\beta}}^{N} \dot{\beta} = - P \nabla_{\dot{\beta}}^{N} P \dot{\beta} = - P \nabla_{{F}_{\ast} X + V}^{N} P ({F}_{\ast} X + V),
	\end{equation*}
	which implies
	\begin{equation}\label{3.7}
		\nabla_{\dot{\beta}}^{N} \dot{\beta} = - P \left(\nabla_{{F}_{\ast} X}^{N} P {F}_{\ast} X + \nabla_{{F}_{\ast} X}^{N} P V + \nabla_{V}^{N} P {F}_{\ast} X + \nabla_{V}^{N} PV \right). 
	\end{equation}
	Using (\ref{2.5}), (\ref{3.3}) and (\ref{3.4}) in (\ref{3.7}), we have
	\begin{eqnarray}\label{3.8}
		\nabla_{\dot{\beta}}^{N} \dot{\beta} =& - P \left( \begin{array}{c} \nonumber \nabla_{{F}_{\ast} X }^{N} \phi {F}_{\ast} X + \nabla_{{F}_{\ast} X }^{N} \omega {F}_{\ast} X + \nabla_{{F}_{\ast} X }^{N} B V + \nabla_{{F}_{\ast} X }^{N} CV \\ + \nabla_{V}^{N} \phi {F}_{\ast} X + \nabla_{V}^{N} \omega {F}_{\ast} X + \nabla_{V}^{N} B V + \nabla_{V}^{N} C V
		\end{array}
		\right)\\
		= & - P \left( 
		\begin{array}{c}
			{F}_\ast(\nabla_X^{M} {}^{\ast}{F}_\ast \phi {F}_\ast X) + (\nabla {F}_\ast) (X, {}^{\ast}{F}_\ast \phi {F}_\ast X) + \nabla_{X}^{{F} \bot} C V \\- S_{\omega{F}_{\ast}X} {F}_{\ast} X + \nabla_{X}^{{F}\bot} \omega {F}_{\ast} X + \nabla_{{F}_{\ast} X}^{N} B V - S_{CV} {F}_{\ast} X \\+ \nabla_{V}^{N} \phi {F}_{\ast} X + \nabla_{V}^{N} \omega {F}_{\ast} X + \nabla_{V}^{N} B V + \nabla_{V}^{N} C V
		\end{array}
		\right).
	\end{eqnarray}
	Since $\nabla^{N}$ is Levi-Civita connection on $N$, $g_{N}(\nabla_{V}^{N} \phi {F}_{\ast} X, U) = 0$ and $g_{N} (\nabla_{V}^{N} B V, U) = 0$, for any $U \in \Gamma(range {F}_{\ast})^{\bot}, \nabla_{V}^{N} B V, \nabla_{V}^{N} \phi {F}_{\ast} X \in \Gamma(range {F}_{\ast})$. Also by using (\ref{2.2}), we have $\nabla_{{F}_{\ast}X}^{N} B V = \overset{N}{\nabla_{{F}_{\ast}X}^{{F}}} BV \circ {F} = (\nabla {F}_{\ast}) (X, {}^{\ast} {F}_{\ast} B V) + {F}_{\ast}(\nabla_{X}^{M \ast} {F}_{\ast} B V)$. In addition, since $(range {F}_{\ast})^{\bot}$ is totally geodesic, we have $\nabla_{V}^{N} \omega {F}_{\ast} X = \nabla_{V}^{{F} \bot} \omega {F}_{\ast}X$ and $\nabla_{V}^{N} C V = \nabla_{V}^{{F} \bot} C V$. Then by (\ref{3.8}), we obtain
	\begin{equation*}
		\nabla_{\dot{\beta}}^{N} \dot{\beta} = - P\left( \begin{array}{c} - S_{\omega {F}_{\ast}X} {F}_{\ast} X +{F}_\ast(\nabla_X^{M} {}^{\ast}{F}_\ast \phi {F}_\ast X) + (\nabla {F}_\ast) (X, {}^{\ast}{F}_\ast \phi {F}_\ast X) \\+ \nabla_{X}^{{F}\bot} \omega {F}_{\ast}X + (\nabla {F}_{\ast}) (X,{}^{\ast} {F}_{\ast} B V ) + {F}_{\ast}(\nabla_{X}^{M\ast} {F}_{\ast}BV) - S_{CV} {F}_{\ast} X \\ + \nabla_{X}^{{F} \bot} C V + \nabla_{V}^{N} \phi {F}_{\ast} X + \nabla_{V}^{{F} \bot}\omega{F}_{\ast} X + \nabla_{V}^{N} B V + \nabla_{V}^{{F} \bot}C V
		\end{array}
		\right).
	\end{equation*}
	Now, $\beta$ is geodesic on $N \iff \nabla_{\dot{\beta}}^{N} \dot{\beta} = 0 \iff$
	\begin{eqnarray*}
		0 & = & - S_{\omega {F}_{\ast}X} {F}_{\ast} X +{F}_\ast(\nabla_X^{M} {}^{\ast}{F}_\ast \phi {F}_\ast X) + (\nabla {F}_\ast) (X, {}^{\ast}{F}_\ast \phi {F}_\ast X) + \nabla_{X}^{{F}\bot}\omega {F}_{\ast} X \\&&+ (\nabla {F}_{\ast}) (X, {}^{\ast} {F}_{\ast} B V) + {F}_{\ast}(\nabla_{X}^{M \ast} {F}_{\ast} B V) - S_{CV} {F}_{\ast} X + \nabla_{X}^{{F} \bot} C V \\&& + \nabla_{V}^{N} \phi {F}_{\ast} X + \nabla_{V}^{{F} \bot} \omega {F}_{\ast} X + \nabla_{V}^{N} B V + \nabla_{V}^{{F} \bot} C V.
	\end{eqnarray*}
	By separating the vertical and horizontal parts of above equation, we obtain (\ref{3.5}) and (\ref{3.6}). This completes the proof.
\end{proof}

\begin{theorem}
	Let ${F}$ be a SIRM from a Riemannian manifold $(M, g_{M})$ to a K\"ahler manifold $(N, g_{N}, P)$ such that $range{F}_\ast$ is connected, and $\alpha, \beta ={F} \circ \alpha$ be geodesic curves on $M$ and $N$, respectively. Then ${F}$ is CSIRM with $s=e^{f}$ if and only if 
	\begin{align*}
		\left\Vert {F}_{\ast} X \right\Vert^{2} g_{N} (\nabla^{N}f, V)& = g_{N} (S_{\omega {F}_{\ast}X} {F}_{\ast} X + S_{CV}{F}_{\ast} X -\nabla_{V}^{N} \phi {F}_{\ast}X \\&- {F}_\ast(\nabla_X^{M} {}^{\ast}{F}_\ast \phi {F}_\ast X), BV) - g_{N}((\nabla {F}_\ast) (X, {}^{\ast}{F}_\ast \phi {F}_\ast X) \\&+ \nabla_{X}^{{F} \bot}\omega {F}_{\ast} X + (\nabla {F}_{\ast}) (X, {}^{\ast} {F}_{\ast} B V) + \nabla_{V}^{{F} \bot} \omega {F}_{\ast}X, C V),
	\end{align*}
	where $f$ is a smooth function on $N$ and ${F}_{\ast} X \in \Gamma(range {F}_{\ast}), V \in \Gamma(range {F}_{\ast})^{\bot}$ are components of $\dot{\beta}(t)$.
\end{theorem}

\begin{proof}
	Let $\alpha$ be a geodesic curve on $M$ and $\beta = {F} \circ \alpha$ be geodesic on $N$ with ${F}_{\ast} X \in \Gamma (range {F}_{\ast})$ and $V \in \Gamma(range {F}_{\ast})^{\bot}$ are components of $\dot{\beta}(t)$, and $\theta (t)$ denote the angle in $\left[0, \pi \right]$ between $\dot{\beta}$ and $V$. Let $\sqrt{k}$ be constant speed of $\beta$ on $N$ that is, $k = g_{N} (\dot{\beta}(t), \dot{\beta}(t)) = \left\Vert \dot{\beta}(t) \right\Vert^{2}$. Hence we conclude that,
	\begin{equation}\label{3.9}
		g_{N}({F}_{\ast} X, {F}_{\ast} X) = k \sin^{2} \theta (t)
	\end{equation}
	and
	\begin{equation}\label{3.10}
		g_{N}(V, V) = k \cos^{2} \theta (t).
	\end{equation}
	Differentiating (\ref{3.10}), we have
	\begin{equation}\label{3.11}
		\frac{d}{dt} g_{N} (V, V) = - 2 k \sin \theta (t) \cos \theta (t) \frac{d \theta}{dt}. 
	\end{equation}
	On the other hand by (\ref{2.8}), we have 
	\begin{equation}\label{3.12}
		\frac{d}{dt} g_{N} (V, V)=\frac{d}{dt} g_{N} (PV, PV). 
	\end{equation}
	Using (\ref{3.4}) in (\ref{3.12}), we get
	\begin{equation*}
		\frac{d}{dt} g_{N} (V, V) = \frac{d}{dt} \left(g_{N} (BV, BV) + g_{N} (CV, CV)\right),
	\end{equation*}
	which implies
	\begin{equation}\label{3.13}
		\frac{d}{dt} g_{N}(V, V) = 2 g_{N} (\nabla_{\dot{\beta}}^{N}BV, BV) + 2 g_{N}(\nabla_{\dot{\beta}}^{N} C V, C V).
	\end{equation}
	Putting $\dot{\beta} = {F}_{\ast} X + V$ in (\ref{3.13}), we get
	\begin{equation}\label{3.14}
		\begin{array}{ll}
			\frac{d}{dt} g_{N} (V, V) = & 2 g_{N} (\nabla_{{F}_{\ast} X}^{N} B V, B V) + 2 g_{N} (\nabla_{{F}_{\ast} X}^{N} C V,C V) \\& + 2 g_{N} (\nabla_{V}^{N} B V, B V) + 2 g_{N} (\nabla_{V}^{N} C V, C V).
		\end{array}
	\end{equation}
	Since $(range {F}_{\ast})^{\bot}$ is totally geodesic, (\ref{3.14}) can be written as 
	\begin{equation}\label{3.15}
		\begin{array}{ll}
			\frac{d}{dt} g_{N} (V, V) = & 2 g_{N} (\overset{N}{\nabla_{X}^{{F}}} BV \circ {F}, B V) + 2 g_{N}(\nabla_{{F}_{\ast} X}^{N}C V, C V) \\& + 2 g_{N} (\nabla_{V}^{N} B V, B V) + 2 g_{N}(\nabla_{V}^{{F} \bot} C V, C V).
		\end{array}
	\end{equation}
	Using (\ref{2.2}), (\ref{2.3}) and (\ref{2.5}) in (\ref{3.15}), we get
	\begin{align}\label{3.17}
		\frac{d}{dt} g_{N} (V, V) =& 2 g_{N} ({F}_{\ast}(\nabla_{X}^{M\ast} {F}_{\ast} B V) + \nabla_{V}^{N} B V, B V) \nonumber\\&+ 2 g_{N} (\nabla_{X}^{{F} \bot} C V + \nabla_{V}^{{F} \bot} C V, C V). 
	\end{align}
	Using (\ref{3.5}) and (\ref{3.6}) in (\ref{3.17}), we get
	\begin{equation}\label{3.18}
		\begin{array}{ll}
			\frac{d}{dt} g_{N}(V, V) = & 2 g_{N} (S_{\omega {F}_{\ast} X } {F}_{\ast} X + S_{C V} {F}_{\ast} X - \nabla_{V}^{N} \phi {F}_{\ast} X \\&- {F}_\ast(\nabla_X^{M} {}^{\ast}{F}_\ast \phi {F}_\ast X), B V) - 2 g_{N} ((\nabla {F}_\ast) (X, {}^{\ast}{F}_\ast \phi {F}_\ast X) \\&+ \nabla_{X}^{{F} \bot} \omega {F}_{\ast} X + (\nabla {F}_{\ast})(X, {}^{\ast} {F}_{\ast} B V ) + \nabla_{V}^{{F} \bot} \omega {F}_{\ast} X, C V). 
		\end{array}
	\end{equation}
	From (\ref{3.11}) and (\ref{3.18}), we have
	\begin{align}\label{3.19}
		- k \sin \theta (t) \cos \theta (t) \frac{d \theta}{dt} = & g_{N} (S_{\omega {F}_{\ast}X} {F}_{\ast} X + S_{CV} {F}_{\ast} X - \nabla_{V}^{N} \phi {F}_{\ast}X \nonumber \\&- {F}_\ast(\nabla_X^{M} {}^{\ast}{F}_\ast \phi {F}_\ast X), BV) - g_{N} ((\nabla {F}_\ast) (X, {}^{\ast}{F}_\ast \phi {F}_\ast X) \nonumber \\&+ \nabla_{X}^{{F} \bot} \omega {F}_{\ast} X + (\nabla{F}_{\ast}) (X, {}^{\ast} {F}_{\ast} B V) + \nabla_{V}^{{F} \bot} \omega {F}_{\ast} X, C V).
	\end{align}
	Now, ${F}$ is a CSIRM with $s = e^{f}$ if and only if $\frac{d}{dt} (e^{f\circ \beta} \sin \theta (t)) = 0$. Therefore
	\begin{equation*}
		\frac{d}{dt} (e^{f\circ \beta} \sin \theta (t)) = 0 \iff e^{f \circ \beta} \frac{d(f \circ \beta)}{dt} \sin \theta (t) + e^{f \circ \beta} \cos \theta (t) \frac{d \theta}{dt} = 0.
	\end{equation*}
	Since $s$ is a positive function,
	\begin{equation*}
		\frac{d(f\circ \beta)}{dt} \sin \theta (t) + \cos \theta (t) \frac{d \theta}{dt} = 0.
	\end{equation*}
	By multiplying this with non-zero factor $k \sin \theta$ and using (\ref{3.9}), we get
	\begin{equation}\label{3.20}
		-k \sin \theta (t) \cos \theta (t) \frac{d \theta }{dt} = g_{N}({F}_{\ast} X, {F}_{\ast} X) \frac{d(f \circ \beta)}{dt}. 
	\end{equation}
	Since the left-hand sides of (\ref{3.19}) and (\ref{3.20}) are equal, we can write
	\begin{align*}
		g_{N}({F}_{\ast}X, {F}_{\ast} X) g_{N}(\nabla^N f, V) & = g_{N} (S_{\omega{F}_{\ast} X} {F}_{\ast} X + S_{CV}{F}_{\ast} X - \nabla_{V}^{N} \phi {F}_{\ast} X \\&- {F}_\ast(\nabla_X^{M} {}^{\ast}{F}_\ast \phi {F}_\ast X), B V) - g_{N}((\nabla {F}_\ast) (X, {}^{\ast}{F}_\ast \phi {F}_\ast X) \\&+ \nabla_{X}^{{F} \bot} \omega {F}_{\ast} X + (\nabla {F}_{\ast}) (X, {}^{\ast} {F}_{\ast} B V) + \nabla_{V}^{{F} \bot} \omega {F}_{\ast} X, C V).
	\end{align*}
	Since $\frac{d(f\circ \beta)}{dt} = g_{N} (\nabla^N f, \dot{\beta}(t)) = g_{N} (\nabla^N f, {F}_{\ast} X + V) = g_{N}(\nabla^N f, V)$. Hence the theorem is proved.
\end{proof}

\begin{proposition}
	Let ${F}$ be a CSIRM with $s=e^{f}$ from a Riemannian manifold $(M, g_{M})$ to a K\"ahler manifold $(N, g_{N}, P)$. Then following statements are true:
	\begin{enumerate}
		\item[(i)] $\nabla^N_{P F_\ast X} F_\ast Y = -g_N(F_\ast Y, \phi F_\ast X) \nabla^N f + F_\ast(\nabla^M_{{}^{\ast}F_\ast \phi F_\ast X} Y)~\text{if $F_\ast X, F_\ast Y \in \Gamma(D_1)$}$.
		
		\item[(ii)] $\nabla^N_{P F_\ast X} F_\ast Y = F_\ast(\nabla^M_{{}^{\ast}F_\ast \phi F_\ast X} Y)~\text{if $F_\ast X \in \Gamma(D_1)$ and $F_\ast Y \in \Gamma(D_2)$}$.
		
		\item[(iii)] $\nabla^N_{PU} V = V(f) BU + \nabla^{F \perp}_{{}^{\ast}F_\ast BU} V + \nabla^{F \perp}_{CU} V~\text{if $U, V \in \Gamma(rangeF_\ast)^\perp$}$.
		
		\item[(iv)] $\nabla^N_{U} PV = U(f) BU + \nabla^{F \perp}_{{}^{\ast}F_\ast BV} U + \nabla^{F \perp}_{U} CV~\text{if $U, V \in \Gamma(rangeF_\ast)^\perp$}$.
	\end{enumerate}
\end{proposition}

\begin{proof}
	For $F_\ast X, F_\ast Y \in \Gamma(D_1)$:
	\begin{align*}
		\nabla^N_{P F_\ast X} F_\ast Y &=\nabla^N_{\phi F_\ast X} F_\ast Y = \nabla^N_{{}^{\ast}F_\ast \phi F_\ast X} F_\ast Y\\& = (\nabla F_\ast)(Y, {}^{\ast}F_\ast \phi F_\ast X) + F_\ast(\nabla^M_{{}^{\ast}F_\ast \phi F_\ast X} Y) \\& = -g_M(Y, {}^{\ast}F_\ast \phi F_\ast X)\nabla^N f + F_\ast (\nabla^M_{{}^{\ast}F_\ast \phi F_\ast X} Y)\\& = -g_N(F_\ast Y, \phi F_\ast X)\nabla^N f + F_\ast (\nabla^M_{{}^{\ast}F_\ast \phi F_\ast X} Y).
	\end{align*}
	Similarly for $F_\ast X \in \Gamma(D_1)$ and $F_\ast Y \in \Gamma(D_2)$:
	\begin{align*}
		\nabla^N_{P F_\ast X} F_\ast Y & = -g_N(F_\ast Y, \phi F_\ast X)\nabla^N f + F_\ast (\nabla^M_{{}^{\ast}F_\ast \phi F_\ast X} Y) \\&= F_\ast (\nabla^M_{{}^{\ast}F_\ast \phi F_\ast X} Y).
	\end{align*}
	Now for $U, V \in \Gamma(rangeF_\ast)^\perp$:
	\begin{align*}
		\nabla^N_{PU} V& = \nabla^N_{BU + CU} V = \nabla^N_{BU} V + \nabla^{F \perp}_{CU} V\\& = -S_V BU + \nabla^{F \perp}_{{}^{\ast}F_\ast BU} V + \nabla^{F \perp}_{CU} V\\& = V(f) BU + \nabla^{F \perp}_{{}^{\ast}F_\ast BU} V + \nabla^{F \perp}_{CU} V.
	\end{align*}
	Similarly we can obtain
	\begin{align*}
		\nabla^N_{U} PV = U(f) BV + \nabla^{F \perp}_{{}^{\ast}F_\ast BV} U + \nabla^{F \perp}_{U} CV.
	\end{align*}
	This completes the required proof.
\end{proof}

\begin{theorem}\label{thmharmonic} \cite{Meena}
	Let $F$ be a CRM with $s = e^f$ between Riemannian manifolds $(M, g_M)$ and $(N, g_N)$ such that $kerF_\ast$ is minimal. Then $F$ is harmonic if and only if $f$ is constant.
\end{theorem}

\begin{theorem}
	Let ${F}$ be a CSIRM with $s=e^{f}$ from a Riemannian manifold $(M, g_{M})$ to a K\"ahler manifold $(N, g_{N}, P)$ such that $kerF_\ast$ is minimal. Then $F$ is harmonic if and only if
	\begin{equation*}
		\textrm{\normalfont trace} \left\{\omega S_{\omega F_{\ast }X}F_{\ast }X-C\nabla
		_{X}^{F\perp}\omega F_{\ast }X-(\nabla _{X}^{N}P\phi F_{\ast
		}X)^{(rangeF_{\ast })^{\perp }}\right\} = 0,
	\end{equation*}
	where $X\in \Gamma(kerF_{\ast})^{\perp}$.
\end{theorem}

\begin{proof}
	For $X\in \Gamma(kerF_{\ast})^{\perp}$ by using (\ref{2.2}),
	(\ref{3.3}), (\ref{2.5}) and (\ref{3.4}) we have
	\begin{eqnarray*}
		(\nabla F_{\ast })(X,X) &=&\overset{N}{\nabla_{X}^{{F} }} {F}_{\ast} X - {F}_{\ast}(\nabla_{X}^{M}X) \\
		&=&-P\overset{N}{\nabla_{X}^{{F}}} P{F}_{\ast} X - {F}_{\ast}(\nabla_{X}^{M}X) \\
		&=&-\nabla_{F_\ast X}^{N} P\phi F_{\ast }X+\phi S_{\omega F_{\ast }X}F_{\ast
		}X+\omega S_{\omega F_{\ast }X}F_{\ast }X \\
		&&-B\nabla_{X}^{F\perp }\omega F_{\ast }X-C\nabla _{X}^{F\perp }\omega
		F_{\ast }X-F_{\ast }(\nabla _{X}^{M}X).
	\end{eqnarray*}
	We know that the $rangeF_{\ast}$ part of $(\nabla F_{\ast })(X,X)$ is equal to $0$, i.e.
	\begin{equation*}
		\phi S_{\omega F_{\ast} X} F_{\ast} X - B \nabla_{X}^{F \perp} \omega F_{\ast}X - F_{\ast} (\nabla_{X}^{M}X) - (\nabla_{X}^{N} P \phi F_{\ast}X)^{rangeF_{\ast}} = 0.
	\end{equation*}
	Therefore
	\begin{align}\label{3.39}
		(\nabla F_{\ast })(X, X)& = -g_M(X, X) \nabla^{N}f \nonumber \\&= \omega S_{\omega F_{\ast}X} F_{\ast}X - C \nabla_{X}^{F \perp} \omega F_{\ast} X - (\nabla_{X}^{N} P \phi
		F_{\ast}X)^{(rangeF_{\ast})^{\perp}}. 
	\end{align}
	Taking trace of (\ref{3.39}), we get
	\[
	\nabla^{N}f=-\frac{1}{(m-r)} \textrm{trace} \left\{\omega S_{\omega F_{\ast }X}F_{\ast }X-C\nabla
	_{X}^{F\perp}\omega F_{\ast }X-(\nabla _{X}^{N}P\phi F_{\ast
	}X)^{(rangeF_{\ast })^{\perp }}\right\}. 
	\]
	Since $F$ is a CRM with minimal fibers then proof follows by Theorem \ref{thmharmonic}.
\end{proof}

\begin{lemma}\label{Lemma3.1}
	Let ${F}$ be a CSIRM with $s=e^{f}$ from a Riemannian manifold $(M, g_{M})$ to a K\"ahler manifold $(N, g_{N}, P)$. Then either $f$ is constant on $P(D_{2})$ or $D_2$ is one dimensional.
\end{lemma}

\begin{proof}
	Since ${F}$ is CRM with $s=e^{f}$ then using Theorem \ref{TH1} and (\ref{2.2}) in (\ref{2.4}), we have
	\begin{equation}\label{3.21}
		\overset{N} {\nabla_{X}^{{F}}} {F}_{\ast} Y - {F}_{\ast} (\nabla_{X}^{M} Y) = - g_{M}(X, Y) \nabla^{N} f
	\end{equation}
	for $X, Y\in \Gamma (\bar{D_2})$. Taking the inner product of (\ref{3.21}) with $P {F}_{\ast} Z \in \Gamma (PD_{2})$, we have
	\begin{equation}\label{3.22}
		g_{N} (\overset{N}{\nabla_{X}^{{F}}} {F}_{\ast} Y - {F}_{\ast}(\nabla_{X}^{M}Y), P{F}_{\ast} Z) = - g_{M}(X, Y)g_{N}(\nabla^{N}f, P{F}_{\ast} Z).
	\end{equation}
	Since $\overset{N}{\nabla^{{F}}}$ is pull-back connection of the Levi-Civita connection ${\nabla}^{N}$, $\overset{N}{\nabla^{{F}}}$ is also Levi-Civita connection. Then using compatibility condition in (\ref{3.22}), we get
	\begin{equation}\label{3.23}
		g_{N} (P\overset{N}{\nabla_{X}^{{F}}} {F}_{\ast} Z, {F}_{\ast} Y) = g_{M}(X, Y) g_{N}(\nabla^{N}f, P {F}_{\ast} Z).
	\end{equation}
	Using (\ref{2.8}) in (\ref{3.23}), we get
	\begin{equation}\label{3.24}
		-g_{N} (\overset{N}{\nabla_{X}^{{F}}} {F}_{\ast} Z, P {F}_{\ast} Y) = g_{M} (X, Y) g_{N}(\nabla^{N}f, P {F}_{\ast} Z).
	\end{equation}
	Using (\ref{3.21}) in (\ref{3.24}), we get
	\begin{equation}\label{3.25}
		g_{M} (X, Z) g_{N} (\nabla^{N}f, P {F}_{\ast} Y) = g_{M} (X, Y) g_{N} (\nabla^{N}f, P {F}_{\ast} Z).
	\end{equation}
	Now, putting $X = Y$ in (\ref{3.25}), we get
	\begin{equation}\label{3.26}
		g_{M} (X, Z) g_{N} (\nabla^{N}f, P {F}_{\ast} X ) = g_{M} (X, X) g_{N} (\nabla^{N}f, P{F}_{\ast} Z).
	\end{equation}
	Now interchanging $X$ and $Z$ in (\ref{3.26}), we get
	\begin{equation}\label{3.27}
		g_{M} (X, Z) g_{N} (\nabla^{N}f, P {F}_{\ast} Z) = g_{M} (Z, Z) g_{N} (\nabla^{N}f, P {F}_{\ast} X).
	\end{equation}
	From (\ref{3.26}) and (\ref{3.27}), we have
	\begin{equation}\label{3.28}
		g_{N} (\nabla^{N}f, P {F}_{\ast}X) \left(1 - \frac{g_{M} (X, X) g_{M}(Z, Z)} {g_{M} (X, Z) g_{M} (X, Z)}\right) = 0.
	\end{equation}
	Since $\nabla^{N} f \in \Gamma (range {F}_{\ast})^\bot$, (\ref{3.28}) implies the proof.
\end{proof}

\begin{theorem}
	Let ${F}$ be a CSIRM with $s=e^{f}$ from a Riemannian manifold $(M, g_{M})$ to a K\"ahler manifold $(N, g_{N}, P)$ such that $\dim(D_2) > 1$. Then ${F}$ is totally geodesic if and only if
	\begin{enumerate}[(i)]
		\item[(i)] $ker {F}_{\ast}$ is totally geodesic, and
		
		\item[(ii)] $(ker {F}_{\ast})^\bot$ is totally geodesic, and
		
		\item[(iii)] for $X, Y, Z \in \Gamma(ker {F}_{\ast})^\bot$ such that ${F}_\ast Z = \phi {F}_\ast Y$, we have	\begin{align*}
			g_{M} (X, Z)\left\{B(\nabla^{N} f) + C(\nabla^{N} f)\right\} =& \phi {F}_\ast(\nabla_X^{M} Z) + \omega {F}_\ast(\nabla_X^{M} Z) \\&+ B \nabla_{X}^{{F} \bot} \omega {F}_\ast Y + C \nabla_{X}^{{F} \bot} \omega {F}_\ast Y + {F}_\ast(\nabla_X^{M} Y).
		\end{align*}
	\end{enumerate}
\end{theorem}

\begin{proof}
	We know that ${F}$ is totally geodesic if and only if 
	\begin{equation}\label{3.40}
		( \nabla {F}_{\ast})(U, V) = 0,
	\end{equation}
	\begin{equation}\label{3.41}
		( \nabla {F}_{\ast})(X, U) = 0
	\end{equation}
	and
	\begin{equation}\label{3.42}
		( \nabla {F}_{\ast})(X, Y) = 0
	\end{equation}	
	for $U, V \in \Gamma(ker{F}_{\ast})$ and $X, Y \in \Gamma(ker{F}_{\ast})^\bot$. More precisely, (\ref{3.40}) and (\ref{3.41}) have mean $ker{F}_{\ast}$ and $(ker{F}_{\ast})^\bot$ are totally geodesic. In addition, (\ref{3.42}) has means
	\[(\nabla {F}_\ast)(X, Y) = {\nabla_{X}^{{F} }} {F}_{\ast} (Y) - {F}_{\ast}(\nabla_{X}^{M}Y) = -P {\nabla_{X}^{{F}}} P {F}_{\ast} (Y) - {F}_{\ast}(\nabla_{X}^{M}Y) =0.\]
	Then by (\ref{3.3}), we have
	\[-P\nabla_X^{{F}} (\phi {F}_{\ast} Y) - P \nabla_X^{{F}} \omega {F}_{\ast} Y - {F}_{\ast}(\nabla_X^{M} Y) =0.\]
	First using $\phi {F}_\ast Y = {F}_\ast Z$ and (\ref{2.5}) in above equation, and then using (\ref{2.2}), we obtain
	\begin{align*}
		-P((\nabla {F}_\ast)(X, Z) + {F}_{\ast} (\nabla_X^{M} Z)) - P(-S_{\omega {F}_{\ast} Y} {F}_{\ast} X + \nabla_X^{{F} \bot} \omega {F}_{\ast}Y) - {F}_\ast(\nabla_X^{M} Y) =0.
	\end{align*}
	Since ${F}$ is CRM, by using Theorem \ref{TH1} and Lemma \ref{lemmaforumbilicity}, we obtain
	\begin{align*}
		P(g_{M} (X, Z) \nabla^{N} f) = P{F}_{\ast} (\nabla_X^{M} Z) + P( \omega {F}_{\ast} Y)(f) {F}_\ast X + P \nabla_X^{{F} \bot} \omega {F}_{\ast}Y + {F}_\ast(\nabla_X^{M} Y).
	\end{align*}
	Thus proof follows by (\ref{3.3}) and Lemma \ref{Lemma3.1}.
\end{proof}

\begin{theorem}
	Let ${F}$ be a CSIRM with $s=e^{f}$ from a Riemannian manifold $(M, g_{M})$ to a K\"ahler manifold $(N, g_{N}, P)$ such that $\dim(D_2) > 1$. Then following statements are true:
	\begin{enumerate}
		\item[(i)] $\nabla^{N} f \in \Gamma(P D_2)$.
		
		\item[(ii)] $S_\xi = 0$, for all $\xi \in \Gamma(\eta)$.
		
		\item[(iii)] $\nabla^N_{P F_\ast X} F_\ast Y = \nabla_X^{F \perp} P F_\ast Y$ \text{if $F_\ast X, F_\ast Y \in \Gamma(D_2)$}.
		
		
		\item[(iv)] $\nabla^N_{P F_\ast X} F_\ast Y = \nabla^N_{\phi F_\ast X} F_\ast Y + \nabla^{F \perp}_Y \omega F_\ast X$ \text{if $F_\ast X, F_\ast Y \in \Gamma(rangeF_\ast)$}.
	\end{enumerate}
\end{theorem}

\begin{proof}
	Since ${F}$ is CRM, for $X \in \Gamma(\bar{D_1})$, we have
	\begin{equation}\label{3.29}
		\nabla_X^{{F}} {F}_\ast X - {F}_{\ast}(\nabla_X^{M} X) = - g_{M}(X, X) \nabla^{N} f.
	\end{equation}
	Taking inner product with $V \in \Gamma(\eta)$, we obtain
	\begin{equation*}
		g_{N} (\nabla_X^{{F}} {F}_\ast X, V) = - g_{M}(X, X) g_{N}(\nabla^{N} f, V).
	\end{equation*}
	Since $N$ is K\"ahler manifold, we can write
	\begin{equation*}
		g_{N} (P \nabla_X^{{F}} {F}_\ast X, PV) = g_{N} (\nabla_X^{{F}} P {F}_\ast X, PV) = - g_{M}(X, X) g_{N}(\nabla^{N} f, V).
	\end{equation*}
	By using (\ref{3.3}) in above equation, we get
	\begin{equation}\label{3.30}
		g_{N} (\nabla_X^{{F}} \phi {F}_\ast X, PV) = - g_{M}(X, X) g_{N}(\nabla^{N} f, V).
	\end{equation}
	Using (\ref{3.29}) in (\ref{3.30}), we obtain
	\begin{equation*}
		g_{M}(X, {}^\ast {F}_{\ast}(\phi {F}_{\ast}X)) g_{N} (\nabla^{N}f, PV) = g_{M}(X, X) g_{N}(\nabla^{N} f, V).
	\end{equation*}
	Since ${F}$ is RM, we can write 
	\begin{equation}\label{3.31}
		g_{N}({F}_{\ast} X, \phi {F}_{\ast}X) g_{N} (\nabla^{N}f, PV) = g_{M}(X, X) g_{N}(\nabla^{N} f, V).
	\end{equation}
	Interchanging $V$ and $\nabla^{N} f$, we get
	\begin{equation}\label{3.32}
		g_{N}({F}_{\ast} X, \phi {F}_{\ast}X) g_{N} (V, P\nabla^{N}f) = g_{M}(X, X) g_{N}(\nabla^{N} f, V).
	\end{equation}
	Using (\ref{3.32}) in (\ref{3.31}), we obtain
	\begin{equation*}
		g_{N}({F}_{\ast} X, \phi {F}_{\ast}X) g_{N} (PV, \nabla^{N}f) = 0.
	\end{equation*}
	This implies the proof of $(i)$.
	For $\xi \in \Gamma(\eta)$ and $F_\ast X \in \Gamma(D_2)$:
	\begin{align*}
		g_N (S_\xi F_\ast X, F_\ast X) & = - \xi (f) g_N(F_\ast X, F_\ast X) \\& = -g_N(\xi, \nabla^N f) g_N(F_\ast X, F_\ast X).
	\end{align*}
	Since $\nabla^N f \in \Gamma(PD_2)$ and $(\eta) \perp (PD_2)$, by above equation we obtain
	\begin{align*}
		g_N (S_\xi F_\ast X, F_\ast X) = 0.
	\end{align*}
	Similarly for $\xi \in \Gamma(\eta)$ and $F_\ast Y \in \Gamma(D_1)$, we obtain
	\begin{align*}
		g_N (S_\xi F_\ast Y, F_\ast Y) = 0.
	\end{align*}
	Thus $S_\xi F_\ast X = 0 = S_\xi F_\ast Y$ for $\xi \in \Gamma(\eta), F_\ast Y \in \Gamma(D_1)$ and $F_\ast X \in \Gamma(D_2)$. Hence $S_\xi = 0$ on $(range F_\ast)$. Thus the second statement hold.
	The next statements can easily prove by using (\ref{2.5}), $\nabla^{N} f \in \Gamma(P D_2)$ and Theorem \ref{TH1}.
\end{proof}

\begin{proposition}\label{prop1}
	Let ${F}$ be a CSIRM with $s=e^{f}$ from a Riemannian manifold $(M, g_{M})$ to a K\"ahler manifold $(N, g_{N}, P)$ such that $\dim(D_2) > 1$. Then the following statements are true:
	\begin{enumerate}[(i)]
		\item[(i)] $\bar{D_1}$ defines a totally geodesic foliation if and only if $(\nabla {F}_{\ast})(X, U)$ has no component in $D_1$ for $X \in \Gamma(\bar{D_1})$ and $U \in \Gamma(ker {F}_\ast)$.
		
		\item[(ii)] $\bar{D_2}$ defines a totally geodesic foliation if and only if $(\nabla {F}_{\ast})(X_2, U)$ has no component in $D_2$ for $X_2 \in \Gamma(\bar{D_2})$ and $U \in \Gamma(ker {F}_\ast)$.
	\end{enumerate}
\end{proposition}
\begin{proof}
	We know that $\bar{D_1}$ defines totally geodesic foliation if and only if $g_{M}(\nabla_X^{M} Y, U) = 0$ and $g_{M} (\nabla_X^{M} Y, X') = 0$ for $X, Y \in \Gamma(\bar{D_1})$, $U \in \Gamma(ker {F}_\ast)$ and $X' \in \Gamma(\bar{D_2})$. Now since ${F}$ is RM, using (\ref{2.1}) and (\ref{2.2}) we can write \[g_{M}(\nabla_X^{M} Y, U) = - g_{M} (\nabla_X^{M} U, Y) = -g_{N}({F}_\ast (\nabla_X^{M} U), {F}_\ast Y) = g_{N}((\nabla {F}_\ast)(X, U), {F}_\ast Y),\] and similarly 
	\begin{align*}
		g_{M}(\nabla_X^{M} Y, X')& = - g_{M} (\nabla_X^{M} X', Y) = -g_{N}({F}_\ast (\nabla_X^{M} X'), {F}_\ast Y) \\&= -g_{N}(\nabla_X^{{F}}{F}_\ast X', {F}_\ast Y)=-g_{N}(\nabla_{{F}_{\ast} X}^{N}{F}_\ast X', {F}_\ast Y).
	\end{align*}
	Since $N$ is K\"ahler manifold, by using (\ref{2.8}) and then (\ref{2.5}), we obtain
	\begin{align}\label{3.33}
		g_{M}(\nabla_X^{M} Y, X') =-g_{N}(\nabla_{{F}_{\ast} X}^{N} P{F}_\ast X', P{F}_\ast Y) =g_{N}(S_{P {F}_\ast X'} {F}_\ast X, P {F}_\ast Y).
	\end{align}
	Using Theorem \ref{TH1} and then Lemma \ref{Lemma3.1} in (\ref{3.33}), we get
	\begin{align*}
		g_{M}(\nabla_X^{M} Y, X') = -(P {F}_{\ast} X'(f))g_{N} ({F}_\ast X, P {F}_\ast Y) = 0.
	\end{align*}
	This implies the proof of $(i)$.
	
	On the other hand, we know that $\bar{D_2}$ defines totally geodesic if and only if $g_{M}(\nabla_{X_2}^{M} {Y_2}, U) = 0$ and $g_{M} (\nabla_{X_2}^{M} {Y_2}, {X_2}') = 0$ for ${X_2}, {Y_2} \in \Gamma(\bar{D_2})$, $U \in \Gamma(ker {F}_\ast)$ and ${X_2}' \in \Gamma(\bar{D_1})$. Now since ${F}$ is RM, using (\ref{2.1}) and (\ref{2.2}) we can write
	\[g_{M}(\nabla_{X_2}^{M} {Y_2}, U) = - g_{M} (\nabla_{X_2}^{M} U, {Y_2}) = -g_{N}({F}_\ast (\nabla_{X_2}^{M} U), {F}_\ast {Y_2}) = g_{N}((\nabla {F}_\ast)({X_2}, U), {F}_\ast {Y_2}),\] and similarly 
	\begin{align*}
		g_{M}(\nabla_{X_2}^{M} {Y_2}, {X_2}')& = g_{N}({F}_\ast (\nabla_{X_2}^{M} {Y_2}), {F}_\ast {X_2}') \\&= g_{N}(\nabla_{X_2}^{{F}}{F}_\ast {Y_2}, {F}_\ast {X_2}')=g_{N}(\nabla_{{F}_{\ast} {X_2}}^{N}{F}_\ast {Y_2}, {F}_\ast {X_2}').
	\end{align*}
	Since $N$ is K\"ahler manifold, by using (\ref{2.8}) and then (\ref{2.5}), we obtain
	\begin{align}\label{3.34}
		g_{M}(\nabla_{X_2}^{M} {Y_2}, {X_2}') = g_{N}(\nabla_{{F}_{\ast} {X_2}}^{N} P{F}_\ast {Y_2}, P{F}_\ast {X_2}') =-g_{N}(S_{P {F}_\ast {Y_2}} {F}_\ast {X_2}, P {F}_\ast {X_2}').
	\end{align}
	Using Theorem \ref{TH1} and then Lemma \ref{Lemma3.1} in (\ref{3.34}), we get
	\begin{align*}
		g_{M}(\nabla_{X_2}^{M} {Y_2}, {X_2}') = -(P {F}_{\ast} {Y_2}(f))g_{N} ({F}_\ast {X_2}, P {F}_\ast {X_2}') = 0.
	\end{align*}
	This implies the proof of $(ii)$.
\end{proof}

\begin{definition}\label{locallyproduct}
	\cite{Ponge_1993} Let $(M, g_M)$ be a Riemannian manifold and assume that the canonical foliations $L_1$ and $L_2$ such that $L_1 \cap L_2 = \{0\}$ everywhere. Then $(M, g_M)$ is a locally product manifold if and only if $L_1$ and $L_2$ are totally geodesic foliations. 
\end{definition}
\begin{theorem}\label{thm3.5}
	Let ${F}$ be a CSIRM with $s=e^{f}$ from a Riemannian manifold $(M, g_{M})$ to a K\"ahler manifold $(N, g_{N}, P)$ such that $\dim(D_2) > 1$. Then $(ker{F}_{\ast})^\bot$ is a locally product manifold of $\bar{D_1}$ and $\bar{D_2}$ if and only if
	\begin{enumerate}[(i)]
		\item[(i)] $(\nabla {F}_{\ast})(X, U)$ has no component in $D_1$, and
		
		\item[(ii)] $(\nabla {F}_{\ast})(X_2, U)$ has no component in $D_2$,
	\end{enumerate}
	for $X \in \Gamma(\bar{D_1}), X_2 \in \Gamma(\bar{D_2})$ and $U \in \Gamma(ker {F}_\ast)$.
\end{theorem}

\begin{proof}
	The proof is straight forward by Proposition \ref{prop1} and Definition \ref{locallyproduct}.
\end{proof}

\begin{theorem}\label{thm3.6}
	Let ${F}$ be a CSIRM with $s=e^{f}$ from a Riemannian manifold $(M, g_{M})$ to a K\"ahler manifold $(N, g_{N}, P)$ such that $\dim(D_2) > 1$. Then the base manifold is locally a product manifold $N_{{(range{F}_\ast)}} \times N_{{(range{F}_\ast)^\bot}}$ if and only if $g_{N} ({F}_{\ast}(\nabla_{X}^{M} \phi {F}_{\ast} X), BV) + g_{N} (\nabla_{X}^{{F} \bot} \omega {F}_{\ast} X, CV) + g_{N} (S_{CV} {F}_{\ast} X, \phi {F}_{\ast} X)= 0$ for $X \in \Gamma(\bar{D_1})$ and $V \in \Gamma(range {F}_\ast)^\bot$.
\end{theorem}

\begin{proof}
	Since $N$ is K\"ahler manifold, by using (\ref{2.8}), we have
	\begin{equation}\label{3.43}
		g_{N} (\nabla_{{F}_\ast X}^{N} {F}_\ast X, V) = g_{N}( \nabla_{{F}_\ast X}^{N} P{F}_\ast X, PV)
	\end{equation}
	for ${F}_{\ast} X \in \Gamma(range {F}_{\ast})$ and $V \in \Gamma(range {F}_{\ast})^\bot$.
	By using (\ref{3.3}) and (\ref{3.4}) in (\ref{3.43}), we write
	\begin{align*}
		g_{N} (\nabla_{{F}_\ast X}^{N} {F}_\ast X, V)&= g_{N}(\nabla_{{F}_\ast X}^{N} \phi {F}_\ast X, BV) + g_{N}(\nabla_{{F}_\ast X}^{N} \omega {F}_\ast X, BV) \\&+ g_{N}(\nabla_{{F}_\ast X}^{N} \phi {F}_\ast X, CV) + g_{N}(\nabla_{{F}_\ast X}^{N} \omega {F}_\ast X, CV).
	\end{align*}
	Using (\ref{2.5}) in above equation, we write
	\begin{align}\label{3.44}
		g_{N} (\nabla_{{F}_\ast X}^{N} {F}_\ast X, V)&= g_{N}(\nabla_{{F}_{\ast} X}^{N} \phi {F}_{\ast} X, BV) + g_{N} (S_{CV} {F}_{\ast} X, \phi {F}_{\ast}X) \nonumber \\&- g_{N}(S_{\omega {F}_{\ast} X} {F}_{\ast} X, BV) + g_{N}(\nabla_X^{{F} \bot} \omega {F}_{\ast} X, CV).
	\end{align}
	Using Theorem \ref{TH1} and then Lemma \ref{Lemma3.1} in (\ref{3.44}), we get
	\begin{align*}
		g_{N} (\nabla_{{F}_\ast X}^{N} {F}_\ast X, V)&= g_{N}(\nabla_{{F}_{\ast} X}^{N} \phi {F}_{\ast} X, BV) + g_{N}(\nabla_X^{{F} \bot} \omega {F}_{\ast} X, CV) + g_{N} (S_{CV} {F}_{\ast} X, \phi {F}_{\ast}X).
	\end{align*}
	Then by using (\ref{2.2}), we get
	\begin{align*}
		g_{N} (\nabla_{{F}_\ast X}^{N} {F}_\ast X, V)&= g_{N}((\nabla {F}_{\ast})(X, {}^{\ast}{F}_{\ast}(\phi {F}_{\ast}X)), BV) \\&+ g_{N}(\nabla_X^{{F} \bot} \omega {F}_{\ast} X, CV) + g_{N} (S_{CV} {F}_{\ast} X, \phi {F}_{\ast}X).
	\end{align*}
	Then by Definition \ref{locallyproduct} proof is completed.
\end{proof}

%%%%%%%%%%%%%%%%%%%%%%%%%%%%%%%%%%




\backmatter

%\bmhead{Supplementary information}
%
%If your article has accompanying supplementary file/s please state so here. 
%
%Authors reporting data from electrophoretic gels and blots should supply the full unprocessed scans for key as part of their Supplementary information. This may be requested by the editorial team/s if it is missing.
%
%Please refer to Journal-level guidance for any specific requirements.

\bmhead{Acknowledgments}

Kiran Meena gratefully acknowledges the research facilities provided by Harish-Chandra Research Institute, Prayagraj, India.

\section*{Declarations}

\begin{itemize}
	\item Funding: Kiran Meena is partial financial supported by the Department of Atomic Energy, Government of India [Offer Letter No.: HRI/133/1436 Dated 29 November 2022].
	
	\item Conflict of interest/Competing interests: The authors have no conflict of interest and no financial interests for this article.
	
	\item Ethics approval: The submitted work is original and not submitted to more than one journal for simultaneous consideration.
	
	\item Consent to participate: Not applicable.
	
	\item Consent for publication: Not applicable.
	
	\item Availability of data and materials: This manuscript has no associated data.
	
	\item Code availability: Not applicable.
	
	\item Authors' contributions: Conceptualization, methodology, investigation, writing - original draft, validation, review, editing and reading have been performed by both the authors and agreed to the paper.	
\end{itemize}

\begin{thebibliography}{99}
	\bibitem{Akyol} M. A. Akyol and B. \c{S}ahin, \textit{Conformal anti-invariant Riemannian maps to K\"ahler manifolds}, U.P.B. Sci. Bull. Ser. A \textbf{80} (2018), no. 4, 187--198.
	
	\bibitem{Akyol1} M. A. Akyol and B. \c{S}ahin, \textit{Conformal semi-invariant Riemannian maps to K\"ahler manifolds}, Rev. Un. Mat. Argentina \textbf{60} (2019), no. 2, 459--468.
	
	\bibitem{Allison} D. Allison, \textit{Lorentzian Clairaut submersions}, Geom. Dedicata \textbf{63} (1996), 309--319.
	
	\bibitem{Bishop} R. L. Bishop, \textit{Clairaut submersions}, Differential geometry (in Honor of K-Yano), Kinokuniya, Tokyo, (1972), 21--31.
	
	\bibitem{Falcitelli} M. Falcitelli, S. Ianus and A. M. Pastore, \textit{Riemannian Submersions and Related Topics}, World Scientific, River Edge, NJ, (2004).
	
	\bibitem{Fischer} A. E. Fischer, \textit{Riemannian maps between Riemannian manifolds}, Contemp. Math. \textbf{132} (1992), 331--366.
	
	\bibitem{Lee} J. Lee, J. H. Park, B. \c{S}ahin and D. Y. Song, \textit{Einstein conditions for the base space of anti-invariant Riemannian submersions and Clairaut submersions}, Taiwanese J. Math. \textbf{19} (2015), no. 4, 1145--1160.
	
	\bibitem{Meena_Thesis} K. Meena, \textit{Geometry of Riemannian Maps}, PhD Thesis, Banaras Hindu University, (2022).
	
	\bibitem{Meena1} K. Meena and T. Zawadzki, \textit{Clairaut conformal submersions}, arXiv preprint, arXiv:2202.00393 [math.DG].
	
	\bibitem{Meena} K. Meena and A. Yadav, \textit{Clairaut Riemannian maps}, Turkish J. Math. \textbf{47} (2023), no. 2, 794--815.
	
	\bibitem{Nore} T. Nore, \textit{Second fundamental form of a map}, Ann. Mat. Pura Appl. \textbf{146} (1986), 281--310.
	
	\bibitem{Ponge_1993} R. Ponge and H. Reckziegel, \textit{Twisted products in pseudo-Riemannian geometry}, Geom. Dedicata \textbf{48} (1993), no. 1, 15--25.
	
	\bibitem{Sahin3} B. \c{S}ahin, \textit{Holomorphic Riemannian maps}, J. Math. Phys. Anal. Geom. \textbf{10} (2014), no. 4, 422--430.
	
	\bibitem{Sahin} B. \c{S}ahin, \textit{Invariant and anti-invariant Riemannian maps to K\"ahler manifolds}, Int. J. Geom. Methods Mod. Phys. \textbf{7} (2010), no. 3, 337--355.
	
	\bibitem{Sahin2} B. \c{S}ahin, \textit{Semi-invariant Riemannian maps to K\"ahler manifolds}, Int. J. Geom. Methods Mod. Phys. \textbf{8} (2011), no. 7, 1439--1454.
	
	\bibitem{Sahin1} B. \c{S}ahin, \textit{Conformal Riemannian maps between Riemannian manifolds, their harmonicity and decomposition theorems}, Acta Appl. Math. \textbf{109} (2010), 829--847.
	
	\bibitem{Sahin6} B. \c{S}ahin, \textit{A survey on differential geometry of Riemannian maps between Riemannian manifolds}, An. \c{S}tiin\c{t}. Univ. Al. I. Cuza Ia\c{s}i. Mat. \textbf{63} (2017), 151--167.
	
	\bibitem{Sahin_2017b} B. \c{S}ahin, \textit{Notes on Riemannian maps}, U.P.B. Sci. Bull. Ser. A \textbf{79} (2017), no. 3, 131--138.
	
	\bibitem{Sahin4} B. \c{S}ahin, \textit{Riemannian submersions, Riemannian maps in Hermitian geometry, and their applications}, Elsevier, Academic Press, (2017).
	
	\bibitem{Sahin5} B. \c{S}ahin, \textit{Circles along a Riemannian map and Clairaut Riemannian maps}, Bull. Korean Math. Soc. \textbf{54} (2017), no. 1, 253--264.
	
	\bibitem{Watson} B. Watson, \textit{Almost Hermitian submersions}, J. Differential Geom. \textbf{11} (1976), 147--165.
	
	\bibitem{Yadav} A. Yadav and K. Meena, \textit{Clairaut invariant Riemannian maps with K\"ahler structure}, Turkish J. Math. \textbf{46} (2022), no. 3, 1020--1035.
	
	\bibitem{Yadav_PMD} A. Yadav and K. Meena, \textit{Riemannian maps whose base manifolds admit a Ricci soliton}, Publ. Math. Debrecen \textbf{103}, (2023), no. 1-2, 115--139.
	
	\bibitem{Yano} K. Yano and M. Kon, \textit{Structure on Manifolds}, World Scientific, Singapore, (1984).
\end{thebibliography}

\end{document}
