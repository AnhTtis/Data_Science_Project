\section{Related Work}
\begin{figure*}[t!]
	\centering
	\includegraphics[clip, trim=0.1cm 6.3cm 0.1cm 0cm, width=0.97\textwidth]{figures/workflow2.pdf}
\vskip-10pt	\caption{\textbf{An overview of LightPainter.} As the user starts scribbling with our interactive application, the neural modules interactively render a realistic relit image that is faithful to the user input. }\label{fig2}
\vspace{-4mm}
\end{figure*}

\paragraph{Portrait Relighting:}
% There is a significant body of work on portrait/face relighting. 
The pioneering work of Debevec \etal~\cite{debevec2000acquiring} presents an advanced illumination rig (i.e. the light stage) to capture per-person reflectance field, which is used to render the subject under novel illuminations. Such technique has been used to create training data for a number of single-image relighting methods~\cite{nestmeyer2020learning,pandey2021total,sun2019single,zhou2019deep,zhang2021neural,zhang2021neural2}. 
Portrait relighting has also been formulated as style transfer. Shih \etal~\cite{shih2014style} employ a multi-scale technique to transfer the local statistics of an exemplar to the target image. Shu \etal~\cite{shu2017portrait} formulate the lighting transfer as a geometry-aware mass transport problem with 3D morphable face model. Using quotient image to achieve relighting is introduced in~\cite{shashua2001quotient,peers2007post}, where they multiply the source image with a ratio map to render novel illuminations. Intrinsic decomposition based approaches~\cite{pandey2021total,wang2020single,Hou_2022_CVPR,nestmeyer2020learning,barron2014shape,le2019illumination,ji2022relight,li2014intrinsic,shahlaei2015realistic} factorize a source image into geometry, reflectance, and illumination, and apply novel lighting by conditioning on a new illumination. Such technique has received increasing popularity over recent years thanks to recently advanced light stage capturing systems~\cite{guo2019relightables} that make direct supervision of intrinsic components possible. Our method also falls into this category.
\vspace{-4mm}
\paragraph{Lighting Representation:} Lighting representation determines how users interact with the system. Several works~\cite{zhou2019deep,sengupta2018sfsnet} use spherical harmonics, which is limited to only low-frequency illuminations. Reference-based methods~\cite{shu2017portrait, shih2014style} use an image as a proxy to represent lighting, yet the requirement of a matching exemplar image reduces their practicability. A similar argument applies to environment map~\cite{sun2019single,zhou2019deep,wang2020single,yeh2022learning,pandey2021total}, which is inherently challenging to edit, thus hard to interactive with. Other works~\cite{nestmeyer2020learning,Hou_2021_CVPR,Hou_2022_CVPR} model only directional lights, which constrains the type of lighting these methods support.  
\vspace{-4mm}
\paragraph{Image Manipulation with User Scribbles.} Scribbling (or sketching) is one of the most intuitive interactions for human to express creative ideas. Drawing-based interface has been widely exploited in various image manipulation tasks~\cite{chen2018sketchygan,nazeri2019edgeconnect,elder1998image,dekel2018sparse,olszewski2020intuitive,yang2020deep,ghosh2019interactive,fivser2016stylit}. For example, the pioneering work of Eitz \etal~\cite {chen2009sketch2photo} introduces Sketch2Photo to interactively perform image retrieval and synthesize from user sketch. Yu \etal~\cite{yu2019free} develops deepfill-v2, which allows users to conduct free-form inpainting with scribbled ``holes''. SketchHairSalon~\cite{xiao2021sketchhairsalon} makes hair design easy by drawing desired hair structures. Scribbles and sketches have also been used for face manipulation~\cite{zeng2022sketchedit,PortenierHSBFZ18} and image colorization~\cite{LevinLW04,HuangTCWW05}. However, such intuitive interface has not been studied in the context of portrait relighting. 
\vspace{-4mm}
\paragraph{Commercial Lighting Editing Tools.}
Only a few commercial applications support a complete set of lighting editing capability. Applications such as Facetune~\cite{Facetune},``Studio Lighting Mode" in iPhone~\cite{apple} and ``Portrait Relighting" feature in Google Pixel~\cite{google} only support a limited set of editing constrained to changing brightness or adding a fixed-color directional light in 2D. The recently released ClipDrop~\cite{ClipDrop} provides more flexible editing by allowing users to place virtual lights with a chosen color, intensity, distance and radius. However, it does not support removing existing illuminations from the scene. Further, to pursue creative lighting effects, it is possible that users have to manually tune multiple lights at the same time. In the user study, we will show this process is difficult for many novices.