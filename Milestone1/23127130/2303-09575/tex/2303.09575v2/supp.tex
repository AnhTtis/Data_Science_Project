\makeatletter
\renewcommand \thesection{S\@arabic\c@section}
\renewcommand\thetable{S\@arabic\c@table}
\renewcommand \thefigure{S\@arabic\c@figure}
\setcounter{figure}{0}
\setcounter{table}{0}
\makeatother

\section*{Appendix}

\section{Kaplan Meier curves of METABRIC and MSK data\label{app1}}
Figure ~\ref{figfigs1} presents the Kaplan Meier curves of METABRIC and MSK data.

\begin{figure}[!h]
  \centerline{\includegraphics[width=1\linewidth]{img/km-plot.pdf}}
  \caption{Kaplan Meier curve of METABRIC and MSK data}
  \label{figfigs1}
\end{figure}



\section{Impact of number of data points $m$ used for estimating the learning curves\label{app2}}

\vspace*{12pt}

In our main analysis, we use $m=50$ data points, which denote samples with different sizes, for estimating the learning curves. To further investigate the impact of number of data points on the stability of the fitted learning curves, we perform analyses of different sets $m\in\{5, 10, 20, 30, 40, 50\}$, at a varying total sample size $N\in\{100, 300, 600\}$. We use METABRIC data and evaluate a modelling strategy with a logistic regression model trained and evaluated only with clinical data on a task of predicting a binary outcome.

Figure ~\ref{figfigs2} presents the two (NLS and GP) C-statistic learning curves as a function of sample size. Both GP and NLS work well with most predicted C-statistics falling within the 95\% CIs for most sample sizes. Table ~\ref{tabtabs1} presents the point estimation of learning learning curve model. The GP model appears more robust even with only 5 data points and 100 total sample size. GP and NLS had similar results after the total sample size were 300 or more.  

\begin{center}
  \includegraphics[width=1\linewidth]{img/sfig1.pdf}
  \captionof{figure}{Predicted C-statistics of learning curve of METABRIC with GP and NLS of logistic regression with different data points and at different total sample size (lines) and 95\% confidence interval of raw bootstrap results using all samples (shaded area)}
  \label{fig:figs1}
\end{center}

\begin{center}
  \captionof{table}{Point estimation and standard error (SE) of GP and NLS with different data points and at different total sample size}
  \label{tab:tabs1}
  \footnotesize
  \begin{tabular}{lcccccc}
    \tblhead{
    & \multicolumn{3}{c}{\textbf{Gaussian Process}} & \multicolumn{3}{c}{\textbf{Nonlinear Least Squares}}\\
    \textbf{m} & \textbf{a (SE)} & \textbf{b (SE)} & \textbf{c (SE)} & \textbf{a (SE)} & \textbf{b (SE)} & \textbf{c (SE)}
    }
    \multicolumn{6}{l}{\textbf{N = 100}}  \\
    5 & 0.26 (0.09) & 1.77 (1.73) & 0.67 (0.24) & 0.30 (0.76) & 3.88 (36.44) & 1.00 (594401.73)\\
    10 & 0.29 (0.02) & 2.68 (1.05) & 0.80 (0.11) & 0.31 (0.59) & 3.71 (28.94) & 1.00 (482237.84)\\
    20 & 0.25 (0.02) & 1.22 (0.24) & 0.54 (0.07) & 0.00 (145040511.44) & 0.59 (5.66) & 0.12 (22.14)\\
    30 & 0.28 (0.02) & 3.24 (0.92) & 0.85 (0.10) & 0.00 (2337166.71) & 0.65 (3.13) & 0.14 (14.03)\\
    40 & 0.27 (0.02) & 2.80 (1.23) & 0.78 (0.17) & 0.29 (0.52) & 1.94 (8.54) & 0.81 (9.56)\\
    50 & 0.26 (0.02) & 1.38 (0.24) & 0.61 (0.06) & 0.32 (0.30) & 2.89 (15.25) & 1.00 (170639.92)\\
    \multicolumn{6}{l}{\textbf{N = 300}}  \\
    5 & 0.26 (0.03) & 3.19 (1.53) & 0.77 (0.15) & 0.23 (0.25) & 1.34 (0.95) & 0.54 (0.97)\\
    10 & 0.18 (0.04) & 1.03 (0.31) & 0.40 (0.12) & 0.23 (0.19) & 1.37 (0.82) & 0.55 (0.80)\\
    20 & 0.23 (0.01) & 1.60 (0.30) & 0.57 (0.06) & 0.21 (0.20) & 1.22 (0.53) & 0.49 (0.61)\\
    30 & 0.21 (0.01) & 1.14 (0.13) & 0.47 (0.04) & 0.18 (0.38) & 0.85 (0.26) & 0.36 (0.61)\\
    40 & 0.24 (0.01) & 1.75 (0.41) & 0.61 (0.07) & 0.19 (0.29) & 0.87 (0.23) & 0.37 (0.51)\\
    50 & 0.14 (0.02) & 0.80 (0.04) & 0.30 (0.03) & 0.05 (3.40) & 0.71 (0.03) & 0.19 (0.69)\\
    \multicolumn{6}{l}{\textbf{N = 600}}  \\
    5 & 0.24 (0.03) & 2.29 (1.36) & 0.66 (0.16) & 0.21 (0.14) & 1.01 (0.33) & 0.45 (0.45)\\
    10 & 0.22 (0.02) & 1.50 (0.45) & 0.55 (0.09) & 0.22 (0.08) & 1.29 (0.37) & 0.51 (0.35)\\
    20 & 0.23 (0.01) & 1.81 (0.41) & 0.61 (0.07) & 0.23 (0.05) & 1.44 (0.35) & 0.55 (0.28)\\
    30 & 0.21 (0.01) & 1.13 (0.16) & 0.46 (0.05) & 0.22 (0.06) & 1.27 (0.26) & 0.51 (0.24)\\
    40 & 0.22 (0.00) & 1.11 (0.08) & 0.48 (0.02) & 0.22 (0.05) & 1.21 (0.23) & 0.51 (0.22)\\
    50 & 0.21 (0.01) & 1.08 (0.15) & 0.46 (0.05) & 0.22 (0.04) & 1.29 (0.22) & 0.52 (0.19)\\
    \lastline
    \end{tabular}
\end{center}
