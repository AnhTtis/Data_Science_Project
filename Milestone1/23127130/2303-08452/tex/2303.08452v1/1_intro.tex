\section{Introduction}
The early detection of lesions in medical images is critical for the diagnosis and treatment of various conditions, including neurological disorders. Stroke is a leading cause of death and disability, where early detection and treatment can significantly improve patient outcomes. However, the quantification of lesion burden is challenging and can be time-consuming and subjective when performed manually by medical professionals~\cite{atlas2022}. 
While supervised learning methods~\cite{kamnitsas2016deepmedic,kamnitsas2017efficient} have proven to be effective in lesion segmentation, they rely heavily on large annotated datasets for training and tend to generalize poorly beyond the learned labels~\cite{ruff2021unifying}.
On the other hand, unsupervised methods focus on detecting patterns that significantly deviate from the norm by training only on normal data. 

One widely used category of unsupervised methods is latent restoration methods. They involve autoencoders (AEs) that learn low-dimensional representations of data and detect anomalies through inaccurate reconstructions of abnormal samples~\cite{pawlowski2018unsupervised}. However, developing compact and comprehensive representations of the healthy distribution is challenging~\cite{bercea2022ra}, as recent studies suggest AEs perform better reconstructions on out-of-distribution (OOD) samples than on training samples~\cite{schirrmeister2020understanding}. 
Various techniques have been introduced to enhance representation learning, including discretizing the latent space~\cite{mao2020abnormality}, disentangling compounding factors~\cite{bercea2022fedano}, and variational autoencoders (VAEs) that introduce a prior into the latent distribution~\cite{you2019unsupervised,zimmerer2019unsupervised}. However, methods that can enforce the reconstruction of healthy generally tend to produce blurry reconstructions.

In contrast, generative adversarial networks (GANs)\cite{goodfellow2014generative,perera2019ocgan,schlegl2019fanogan} are capable of producing high-resolution images. New adversarial AEs combine VAEs' latent representations with GANs' generative abilities, achieving SOTA results in image generation and outlier detection\cite{chen2018unsupervised,daniel2021soft,bercea2022ra}. %Reversed AEs further improve the reconstruction accuracy of adversarial AEs and demonstrate effective localization of various brain pathologies~\cite{bercea2022ra}.
Nevertheless, latent methods still face difficulties in accurately reconstructing data from their low-dimensional representations, causing false positive detections on healthy tissues.

Several techniques have been proposed that make use of the inherent spatial information in the data rather than relying on constrained latent representations~\cite{kascenas2022denoising,Wyatt_2022_CVPR,zimmerer2018context}. %However, in a reconstruction-based setup, these methods may not perform well as they will simply copy anomalies at inference. To address this limitation, 
These methods are often trained on a pretext task, such as recovering masked input content~\cite{zimmerer2018context}. De-noising AEs~\cite{kascenas2022denoising} are trained to eliminate synthetic noise patterns, utilizing skip connections to preserve the spatial information and achieve SOTA brain tumor segmentation. However, they heavily rely on a learned noise model and may miss anomalies that deviate from the noise distribution~\cite{bercea2022ra}. More recently, diffusion models~\cite{ho2020denoising} apply a more complex de-noising process to detect anomalies~\cite{Wyatt_2022_CVPR}. However, the choice and granularity of the applied noise is crucial for breaking the structure of anomalies~\cite{Wyatt_2022_CVPR}. Adapting the noise distribution to the diversity and heterogeneity of pathology is inherently difficult, and even if achieved, the noising process disrupts the structure of both healthy and anomalous regions throughout the entire image.

In related computer vision areas, such as industrial inspection~\cite{mvtec1}, the top-performing methods do not focus on reversing anomalies, but rather on detecting them by using large nominal banks~\cite{defard2021padim,roth2022towards}, or pre-trained features from large natural imaging datasets like ImageNet~\cite{bergmann2020uninformed,salehi2021multiresolution}. Salehi et al.~\cite{salehi2021multiresolution} have employed multi-scale knowledge distillation to detect anomalies in industrial and medical imaging. However, the application of these networks in medical anomaly segmentation, particularly in brain MRI, is limited by various challenges specific to the medical imaging domain. They include the variability and complexity of normal data, subtlety of anomalies, limited size of datasets, and domain shifts.

This work aims to combine the advantages of constrained latent restoration for understanding healthy data distribution with generative in-painting networks. Unlike previous methods, our approach does not rely on a learned noise model, but instead creates masks of probable anomalies using latent restoration. These guide generative in-painting networks to reverse anomalies, i.e., preserve healthy tissues and produce pseudo-healthy in-painting in anomalous regions. 
We believe that our proposed method will open new avenues for interpretable, fast, and accurate anomaly segmentation and support various clinical-oriented downstream tasks, such as investigating progression of disease, patient stratification and treatment planning. In summary our main contributions are: 
\begin{itemize}
    \item[$\bullet$] We investigate and measure the ability of SOTA methods to reverse synthetic anomalies on real brain T1w MRI data.
    \item[$\bullet$] We propose a novel unsupervised segmentation framework, that we call \textit{PHANES}, that is able to preserve healthy regions and utilize them to generate pseudo-healthy reconstructions on anomalous regions.
    \item[$\bullet$] We demonstrate a significant advancement in the challenging task of unsupervised ischemic stroke lesion segmentation.
   % \item[$\bullet$] We show that our proposed framework can curate anomalous datasets and generate healthy-diseased image pairs, aiding clinical studies. For example, we show that augmenting training with generated pseudo-healthy samples reduces the domain shift and improves anomaly segmentation.
\end{itemize}