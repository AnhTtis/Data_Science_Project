
% We implemented CC-K-CBS with belief-SST as the low-lever planner in Open Motion Planning Library (OMPL), 
% The benchmarks were performed on AMD Ryzen 9 4.5GHz CPU and 64 GB of RAM.
We evaluate our algorithms in several benchmarks. We first focus on characterizing the robot-robot collision checking algorithms from Sec.~\ref{sec:collisionChecking} independent of the planning algorithm. This allows us to isolate the conservatism and computational complexity of the robot-robot collision checkers. We then analyze the full algorithms from Sec.~\ref{sec:method} with different collision checking methods
Our implementation of CC-K-CBS uses belief-SST as the low-lever planner in Open Motion Planning Library (OMPL), 
the benchmarks were performed on AMD 4.5 GHz CPU and 64 GB of RAM. 
% this allows us to examine how the larger planning framework behaves as whole, and better evaluate the situations where one collision checker is advantageous over the other.

\begin{figure}
    \centering
    \begin{subfigure}{0.23\textwidth}
    \includegraphics[width=\textwidth]{figures/conservatism.pdf}
    \vspace{-7mm}
    \caption{}
    \label{fig:conservatism}
    \end{subfigure}
    \hfill
    \begin{subfigure}{0.23\textwidth}
    \includegraphics[width=\textwidth]{figures/CompTime_combined.pdf}
    \vspace{-7mm}
    \caption{}
    \label{fig:compTime_combined}
    \end{subfigure}
    \hfill

\caption{(a) Plot of the conservatism $C_{z}$ for each method in each environment. (b) Box and whiskers plots of the log(computation time) for each method.}
\vspace{-3.5mm}
\end{figure}

We consider two types of robot dynamics and varying number of robots $N_A$. The first is a 2D system taken from~\cite{Bry2011_BeliefProp}:
% \begin{align}
    $x_{k+1}^i=x_k^i+u_k^i+w_k^i$ with $w_k^i \sim \mathcal{N}(0,0.1^2I)$.
% \end{align}
The second is a second-order unicycle:
% \begin{equation}
    % \dot{\text{x}}^i= \text{v} \cos\theta, \quad \dot{\text{y}}^i= \text{v} \sin \theta, \quad \dot{\theta}=\omega, \quad \dot{\text{v}}=a,
% \end{equation}
$\dot{\text{x}}^i= \text{v} \cos\theta,$ $\dot{\text{y}}^i= \text{v} \sin \theta$, $\dot{\theta}=\omega$, $\dot{\text{v}}=a$,
where $\omega$ and $a$ are control inputs. We use the feedback linearization scheme described in \cite{DeLuca2000_feedbackLin} to obtain a linear model, 
%with $x_k^i=[\text{x}_k^i,\dot{\text{x}}_k^i, \text{y}_k^i,\dot{\text{y}}_k^i]$ and noise $Q^i=0.1^2I$. 
which we convert to discrete time. We assume both systems are fully observable with measurement model~\eqref{eq: measurement model} with $C^i=I$ and $R^i=0.1^2I$.


% \subsection{Robot-Robot Collision Checker Benchmarking}
\textbf{Robot-Robot Collision Checker Benchmark: }
We characterize the robot-robot collision checking algorithms based on computation time and conservatism. Conservatism impacts the planners ability to find paths in more difficult, i.e. cluttered, situations. For a given set of beliefs for two robots, we define the conservatism of method $z$ as $C_{z} = R_{z} - R_{MC}$, where $R_{z}$ is the rejection rate obtained by method $z$, and $R_{MC}$ is the rejection rate obtained by a Monte Carlo evaluation of the probability of collision.

% from the \emph{true} probability of collision, $p_{true}^{ij}$. 

% If the true probability is know, the chance constraint can be exactly evaluated: $p_{true}^{ij}<1-p_\safe$. To evaluate conservatism, we randomly sampled $N_{MC}$ sets of two beliefs for each agent, $B_{s_{ij}}=\{\expBelief^i, \expBelief^j\}, s_{ij}=1,...,N_{MC}$, with $N_{MC}=10,000$. For each pair, we then computed the true probability of collision, and evaluated the chance constraint to obtain either valid or invalid, $\bar{v}^{s_{ij}}=1$ or $0$ respectively. The true validity rate can then be calculated as $\bar{V}_{true}=\sum_{s_{ij}=1}^{N_{MC}} \bar{v}^{s_{ij}}/N_{MC}$. For each pair, we then evaluate the chance constraint according to each of the proposed collision checking algorithm, $\bar{v}_{alg}^{s_{ij}}\in\{0,1\}$, yielding an associated algorithmic validity rate $\bar{V}_{alg}=\sum_{s_{ij}=1}^{N_{MC}} \bar{v}_{alg}^{s_{ij}}/N_{MC}$. Conservatism is defined as the difference between the true validity rate and the algorithmic validity rate: $C_{alg}=\bar{V}_{alg}-\bar{V}_{true}$. This indicates how often the algorithm rejects as invalid a sample that actually satisfies the chance constraint.


% \ml{all you're saying is that conservatism of method $z$ is defined as $C_{z} = R_{z} - R_{MC}$, where $R_{z}$ is the rejection rate obtained by method $z$, and $R_{MC}$ the rejection rate obtained by Monte Carlo evaluation of the probability of collision????}

% \ml{is this paragraph needed?}
% As stated earlier, the entire motivation for the proposed agent-agent collision checking algorithms is the difficulty of evaluating the true probability of collision. For the benchmarks performed here, we approximate the true probability of collision via MC sampling. For each pair of sampled beliefs, we further sampled 50,000 states. Each sampled state corresponds to a projection of the robots bodies in the workspace, where collision is indicated by the intersection of the two projections. To calculate the 'true' probability of collision, we found the proportion of the sampled states that resulted in collision.

We sampled 2D beliefs in two difference spaces, 5x5 and 10x10, with $p_\safe=0.95$. For Method 2.2 (the gridded cdf method), we chose 6 different grid discretizations, denoted by Method 2.2($d$). For each discretization, the maximum range of $\mathcal{P}_{whit}^{ij}$ over each axis, $D_w$, was divided equally, such that the discretization step size was $D_w/d$. Note that each of the proposed methods decreases in conservatism, shown in Fig.~\ref{fig:conservatism}, and increases in computation time, shown in Fig.~\ref{fig:compTime_combined}. The 5x5 space is the more `cluttered', with more beliefs in close proximity, which in turn results in higher collision probabilities. Note that the 5x5 space generally results in lower conservatism. This is because for the relatively high choice of $p_\safe$, a high proportion of the sampled beliefs truly violated the chance constraint, making it less conservative compared to a scenario where more beliefs are valid.

% \begin{figure}
%     \centering
%     \includegraphics[width=0.45\textwidth]{figures/PcollVSdist_Comp.pdf}
%     \vspace{-2mm}
%     \caption{Plot of the true probability of collision vs the euclidean distance between the means for each sampled belief pair}
%     \label{fig:PcollvsDist}
%     \vspace{0mm}
% \end{figure}







% \begin{table}[]
% \caption{CC-K-CBS and Centralized planner benchmarking results for simple 2D linear system}
%     \centering
%     \scalebox{0.65}{
%     \begin{tabular}{|c|c|c|c|c|c|}
%          \hline
%             & \multicolumn{4}{c|}{Decentralized (CC-K-CBS)} & Centralized \\
%             & Method 2.2 & Method 2.1* & Method 2.1 & Method 1 & Method 1\\ \hline
%             \multirow{3}{*}{2 Agents}  &     1.00 &   1.00 &   1.00 &   1.00 & 1.00\\ 
%             & 1.26 $\pm$   1.41 &   1.56 $\pm$   1.91&   1.41 $\pm$   1.21&   1.36 $\pm$   1.03& N/A\\ 
%             & 93.72 $\pm$  13.18 &  94.73 $\pm$  14.34&  90.07 $\pm$  13.05&  91.76 $\pm$  12.87& 81.07 $\pm$  17.67\\    \hline
%            \multirow{3}{*}{3 Agents}  &  1.00 &   1.00 &   1.00 &   1.00 &   1.00\\ 
%               & 1.87 $\pm$   1.40 &   2.10 $\pm$   1.84&   2.25 $\pm$   1.32&   3.52 $\pm$   3.77& N/A \\ 
%             & 136.22 $\pm$  12.96 & 130.30 $\pm$  13.41& 133.11 $\pm$  15.35& 133.93 $\pm$  13.63 & 108.76 $\pm$  22.97 \\  \hline
%            \multirow{3}{*}{5 Agents}  &   1.00 &   1.00 &   1.00 &   1.00 & \\ 
%                 &  8.94 $\pm$   7.47 &  15.12 $\pm$  13.22&  13.70 $\pm$  12.06&  16.47 $\pm$  16.99& N/A\\ 
%                 & 167.17 $\pm$  15.65 & 167.34 $\pm$  14.74& 170.24 $\pm$  19.59& 177.61 $\pm$  19.62& N/A\\   \hline
%              \multirow{3}{*}{7 Agents} &   0.76 &   0.44 &   0.66 &   0.52 & \\ 
%                  & 62.75 $\pm$  52.59 &  76.51 $\pm$  38.41&  81.55 $\pm$  52.70&  67.57 $\pm$  45.73& N/A\\ 
%                 & 293.11 $\pm$  17.95 & 286.35 $\pm$  19.10& 281.48 $\pm$  17.85& 299.11 $\pm$  19.40& N/A\\  
%         \hline
        
%     \end{tabular}
%     }
%     \label{tab:benchCCKCBS_linear}
% \end{table}



\begin{table*}[t!]
\caption{CC-K-CBS benchmarking for Simple 2D and unicycle systems}
    \centering
    \scalebox{0.7}{
    \begin{tabular}{|c|c||c|c|c|c||c|c|c|c|}
         \hline
          & & \multicolumn{4}{c||}{Simple Linear} & \multicolumn{4}{c|}{Unicycle}  \\
         $\#$ Robots & Metric & Method 2.2 & Method 2.1* & Method 2.1 & Method 1& Method 2.2 & Method 2.1* & Method 2.1 & Method 1\\ \hline 
        \multirow{3}{*}{2} &  Succ. Rate &  1.00&   1.00 &   1.00 &   1.00 &  1.00&   1.00 &   1.00 &   1.00\\ 
         &  Time (s) &    1.36 $\pm$   0.15 &   1.36 $\pm$   0.19&   1.56 $\pm$   0.27&  \textbf{ 1.26 $\pm$   0.20}&    4.07 $\pm$   0.48 &  \textbf{ 2.77 } $\pm$   0.31&   4.79 $\pm$   0.74&   5.03 $\pm$   0.69\\ \hline 
        \multirow{3}{*}{3} &  Succ. Rate &  1.00&   1.00 &   1.00 &   1.00 &  1.00&   1.00 &   1.00 &   1.00\\ 
         &  Time (s) &    3.52 $\pm$   0.53 &   2.33 $\pm$   0.29&   2.10 $\pm$   0.26&  \textbf{ 1.87 $\pm$   0.20} &    6.18 $\pm$   0.63 &  \textbf{ 4.58 $\pm$   0.54}&  11.30 $\pm$   1.82&   7.36 $\pm$   1.32\\ \hline 
        \multirow{3}{*}{5} &  Succ. Rate &  1.00&   1.00 &   1.00 &   1.00 &  1.00&   1.00 &   1.00 &   1.00\\ 
         &  Time (s) &   16.47 $\pm$   2.40 &  11.89 $\pm$   2.00&  15.12 $\pm$   1.87&  \textbf{ 8.94 $\pm$   1.06} &   13.59 $\pm$   1.72 &  \textbf{ 7.34 $\pm$   1.05}&  21.73 $\pm$   2.70&  13.94 $\pm$   1.71\\ \hline 
        \multirow{3}{*}{7} &  Succ. Rate &  0.52&   0.66 &   0.44 &  \textbf{ 0.76} &  0.40&   0.50 &   0.24 &  \textbf{ 0.56}\\ 
         &  Time (s) &   67.57 $\pm$   8.97 &  62.99 $\pm$   6.81&  76.51 $\pm$   8.19& \textbf{ 62.75 $\pm$   8.53} &  \textbf{ 66.44 $\pm$   8.22 }& 100.77 $\pm$   8.37&  77.15 $\pm$  10.52&  82.92 $\pm$   7.51\\ \hline       
    \end{tabular}
    }
    `\vspace{-4mm}
    \label{tab:benchCCKCBS}
\end{table*}



% \begin{table}[]
% \caption{CC-K-CBS benchmarking for 2D system}
%     \centering
%     \scalebox{0.7}{
%     \begin{tabular}{|c|c|c|c|c|c|}
%          \hline
%          $\#$ Robots & Metric & Method 2.2 & Method 2.1* & Method 2.1 & Method 1\\ \hline 
%         \multirow{3}{*}{2 Robots} &  Succ. Rate &  1.00&   1.00 &   1.00 &   1.00\\ 
%          &  Time (s) &    1.36 $\pm$   0.15 &   1.36 $\pm$   0.19&   1.56 $\pm$   0.27&  \textbf{ 1.26 $\pm$   0.20}\\ \hline 
%         \multirow{3}{*}{3 Robots} &  Succ. Rate &  1.00&   1.00 &   1.00 &   1.00\\ 
%          &  Time (s) &    3.52 $\pm$   0.53 &   2.33 $\pm$   0.29&   2.10 $\pm$   0.26&  \textbf{ 1.87 $\pm$   0.20}\\ \hline 
%         \multirow{3}{*}{5 Robots} &  Succ. Rate &  1.00&   1.00 &   1.00 &   1.00\\ 
%          &  Time (s) &   16.47 $\pm$   2.40 &  11.89 $\pm$   2.00&  15.12 $\pm$   1.87&  \textbf{ 8.94 $\pm$   1.06}\\ \hline 
%         \multirow{3}{*}{7 Robots} &  Succ. Rate &  0.52&   0.66 &   0.44 &  \textbf{ 0.76}\\ 
%          &  Time (s) &   67.57 $\pm$   8.97 &  62.99 $\pm$   6.81&  76.51 $\pm$   8.19& \textbf{ 62.75 $\pm$   8.53}\\ \hline        
%     \end{tabular}
%     }
%     \label{tab:benchCCKCBS_linear}
% \end{table}
% \begin{table}[]
%     \centering
%     \caption{CC-K-CBS benchmarking for unicycle system}
%     \scalebox{0.7}{
%     \begin{tabular}{|c|c|c|c|c|c|}
%          \hline
%         $\#$ Robots & Metric & Method 2.2 & Method 2.1* & Method 2.1 & Method 1\\ \hline 
%         \multirow{3}{*}{2 Robots} &  Succ. Rate &  1.00&   1.00 &   1.00 &   1.00\\ 
%          &  Time (s) &    4.07 $\pm$   0.48 &  \textbf{ 2.77 } $\pm$   0.31&   4.79 $\pm$   0.74&   5.03 $\pm$   0.69\\ \hline 
%         \multirow{3}{*}{3 Robots} &  Succ. Rate &  1.00&   1.00 &   1.00 &   1.00\\ 
%          &  Time (s) &    6.18 $\pm$   0.63 &  \textbf{ 4.58 $\pm$   0.54}&  11.30 $\pm$   1.82&   7.36 $\pm$   1.32\\ \hline 
%         \multirow{3}{*}{5 Robots} &  Succ. Rate &  1.00&   1.00 &   1.00 &   1.00\\ 
%          &  Time (s) &   13.59 $\pm$   1.72 &  \textbf{ 7.34 $\pm$   1.05}&  21.73 $\pm$   2.70&  13.94 $\pm$   1.71\\ \hline 
%         \multirow{3}{*}{7 Robots} &  Succ. Rate &  0.40&   0.50 &   0.24 &  \textbf{ 0.56}\\ 
%          &  Time (s) &  \textbf{ 66.44 $\pm$   8.22 }& 100.77 $\pm$   8.37&  77.15 $\pm$  10.52&  82.92 $\pm$   7.51\\ \hline 
%     \end{tabular}
%     }
%     \label{tab:benchCCKCBS_unicycle}
% \end{table}


% \subsection{Characterizing the MRMP Planners}
\textbf{Characterizing the MRMP Planners: }
% We first compare benchmarks on the performance of the centralized and CC-K-CBS planners proposed in Sec.~\ref{sec:method}. We then present specific plans generated by the decentralized method, along with results from Monte Carlo (MC) simulations demonstrating the plans' validity. 
% \subsubsection{Planner Benchmarking}
We use benchmarking to compare the centralized planner to CC-K-CBS under each of the methods proposed in Sec.~\ref{sec:collisionChecking}. We additionally implement the adaptive risk allocation on Method 2.1, denoted by Method 2.1*. All other methods use equal allocation. We are interested in comparing two quantities: \emph{success rate} and \emph{computation time}. 
% The first is the success rate, defined as the rate at which the planner returns a solution within the given maximum planning duration. The second is the computation time, which indicates the computational complexity of the algorithms. 
Each scenario consisted of a 32x32 environment with 50 obstacles and 2, 3, 5, or 7 robots. Benchmarking was performed using 50 runs for each algorithm with a maximum planning time of 3 minutes. The results of the benchmarks for both systems are in Table \ref{tab:benchCCKCBS}. 

We see that Method 1 is consistently faster than the other methods for the linear system, although this advantage is less pronounced for the smallest system (2 robots), and largest system (7 robots) we studied. It appears to offer only limited benefit for the more complex unicycle system. We can additionally see that the adaptive risk allocation technique consistently outperforms the equal allocation, both in computation time and success rate.

\ml{remove this paragraph... add centralized to the table}
We additionally performed benchmarking on the centralized approach with the simple linear system, using collision checking Method 1. We allocated the full planning time for optimization, so computation time is not reported. The centralized approach allows optimization via the SST algorithm, which does result in improved sum of controls. However, it quickly runs into scalability problems, we were unable to scale beyond 3 robots, where the success rate was $8\%$.

\ml{fix this paragraph}
Finally, we note that CC-K-CBS was able to scale to 20 in a 32x32 environment containing 50 obstacles (with success rate $66\%$), and 25 robots in an empty environment (with success rate $xx\%$), with only collision checking Method 1.  All other methods timed out with more than 7 robots.

% \subsubsection{Planner Behavior}
\ml{show plan for 7 robots and simulations first...}
We present the nominal motion plans for 25 unicycle robots in Figure \ref{fig:unicycle_SampleTraj_big}, with the $p_\safe$ contours plotted as circles. We validated these nominal plans by determining that the contours do not intersect with each other or obstacles at any point. We additionally generated plans for 7 simple linear robots, and Monte Carlo sampled trajectories (shown in Figure \ref{fig:Simple2D_sampleTraj}) to verify that the total probability of collision was less than $1-p_\safe$.


\ml{integrate it with the environments description..}
To highlight how the scenario influences the planner behavior, consider the narrow corridor in Figure \ref{fig:Simple2D_sampleTraj_narrow}, with the simple 2D system, where the noise matrices were modified to be elliptical ($R^i=Q^i=diag(0.05^2,0.1^2)$). This accentuates the conservatism of Method 1: because the ellipse is aligned with the narrow corridor, the true probability of collision is not high, however the circular bound from Method 1 is unable to account for this. In this scenario, implementation of Method 1 in the planner does not find a plan, where implementation of all the other methods do. 

%We present sample plans generated by the K-CBS algorithm under both dynamics models with $p_\safe=0.90$. These plans were generated in a 32x32 environment containing 50 random obstacles. Figure \ref{fig:unicycle_SampleTraj_big} presents the nominal trajectories for 12 robots, with the $90\%$ confidence bound plotted as ellipses. We verified that these bounds do not simultaneously intersect at any point in the nominal trajectories, nor do they intersect with obstacles. We additionally verify the probability of collision over entire sampled trajectories for the simple 2D linear system using 500 MC samples of each robot's trajectory. The empirically obtained probability of collision was $3.7\%$ for the simple 2D system. Examples of the nominal plans and sampled trajectories for the simple 2D system are shown in Figure \ref{fig:Simple2D_sampleTraj}.



% \begin{figure}
%     \centering
%     \includegraphics[width=0.4\textwidth]{figures/unicycle_SampleTraj.pdf}
%     \vspace{-2mm}
%     \caption{For the unicycle system, $95\%$ confidence bound plotted as ellipses, with the nominal trajectory over-plotted as a solid line for each of 7 agents.}
%     \label{fig:unicycle_sampleTraj}
%     \vspace{0mm}
% \end{figure}

% \begin{figure}
%     \centering
%     \begin{subfigure}{0.33\textwidth}
%     \includegraphics[angle=90,width=\textwidth]{figures/simpleLin_SampleTraj.pdf}
%     \vspace{-7mm}
%     \caption{}
%     \label{fig:Simple2D_sampleTraj}
%     \end{subfigure}
%     \hfill
%     \begin{subfigure}{0.14\textwidth}
%     \includegraphics[width=\textwidth]{figures/simpleLin_SampleTraj_narrow.pdf}
%     \vspace{-7mm}
%     \caption{}
%     \label{fig:Simple2D_sampleTraj_narrow}
%     \end{subfigure}
%     \hfill

% \caption{(a) For the simple 2D system in a 32x32 environment: 100 MC sampled trajectories plotted as transparent lines, with the nominal trajectory over-plotted as a solid line. (b) Sample 6x2 narrow environment}
% \vspace{-3.5mm}
% \end{figure}



\begin{figure}
    \centering
    \includegraphics[width=0.3\textwidth]{figures/simpleLin_SampleTraj.pdf}
    \vspace{-2mm}
    \caption{For the simple 2D system in a 32x32 environment: 100 MC sampled trajectories plotted as transparent lines, with the nominal trajectory over-plotted as a solid line.}
    \label{fig:Simple2D_sampleTraj}
    \vspace{-5mm}
\end{figure}


\begin{figure}
    \centering
    \includegraphics[angle=90,width=0.25\textwidth]{figures/simpleLin_SampleTraj_narrow.pdf}
    \vspace{-2mm}
    \caption{Sample 6x2 narrow environment}
    \label{fig:Simple2D_sampleTraj_narrow}
    \vspace{-5mm}
\end{figure}
