
% We implemented CC-K-CBS with belief-SST as the low-lever planner in Open Motion Planning Library (OMPL), 
% The benchmarks were performed on AMD Ryzen 9 4.5GHz CPU and 64 GB of RAM.
We evaluate our algorithms in several benchmarks. We first independently test the robot-robot collision checking algorithms in Sec.~\ref{sec:collisionChecking}, allowing us to isolate the conservatism and computational complexity of the robot-robot collision checkers. We then analyze the full algorithms from Sec.~\ref{sec:method} with different collision checking methods.
Our implementation of CC-K-CBS is publicly available~\cite{kcbs-code}. It uses belief-SST as the low-level planner in Open Motion Planning Library (OMPL) \cite{sucan2012the-open-motion-planning-library}. 
The benchmarks were performed on AMD 4.5 GHz CPU and 64 GB of RAM. 
% \textcolor{blue}{Success rate refers to the percentage of runs of the planner that return a solution within the allotted computation time.}
% this allows us to examine how the larger planning framework behaves as whole, and better evaluate the situations where one collision checker is advantageous over the other.

\begin{figure}
    \centering
    \begin{subfigure}{0.23\textwidth}
    \includegraphics[width=\textwidth]{figures/conservatism.pdf}
    \vspace{-5mm}
    \caption{}
    \label{fig:conservatism}
    \end{subfigure}
    \hfill
    \begin{subfigure}{0.23\textwidth}
    \includegraphics[width=\textwidth]{figures/CompTime_combined.pdf}
    \vspace{-5mm}
    \caption{}
    \label{fig:compTime_combined}
    \end{subfigure}
    \hfill

\caption{(a) Plot of the conservatism $C_{z}$ for each method in each environment. (b) Box and whiskers plots of the log(computation time) for each method.}
% \vspace{-3.5mm}
\end{figure}

We consider square-shaped robots of width 0.25 with two types of dynamics: 2D robotic system taken from~\cite{Bry2011_BeliefProp}; $x_{k+1}^i=x_k^i+u_k^i+w_k^i, w_k^i \sim \mathcal{N}(0,0.1^2I)$, and 2nd-order unicycle; $\dot{\text{x}}^i= \text{v} \cos\theta, \dot{\text{y}}^i= \text{v} \sin \theta, \dot{\theta}=\omega, \dot{\text{v}}=a$, where $\omega$ and $a$ are control inputs. We use the feedback linearization 
scheme described in \cite{DeLuca2000_feedbackLin} 
to obtain a linear model, 
%with $x_k^i=[\text{x}_k^i,\dot{\text{x}}_k^i, \text{y}_k^i,\dot{\text{y}}_k^i]$ and noise $Q^i=0.1^2I$. 
which we convert to discrete time with additive noise. We assume both systems are fully observable with measurement model in \eqref{eq: measurement model} with $C^i=I$ and $R^i=0.1^2I$. 


% \subsection{Robot-Robot Collision Checker Benchmarking}
\textbf{Robot-Robot Collision Checker Benchmark: }
We characterize the robot-robot collision checking algorithms based on computation time and conservatism. Conservatism impacts the planners ability to find paths in more difficult, i.e. cluttered, situations. For a given set of beliefs for two robots, we define the conservatism of method $z$ as $C_{z} = R_{z} - R_{MC}$, where $R_{z}$ is the rejection rate obtained by method $z$, and $R_{MC}$ is the rejection rate obtained by a Monte Carlo evaluation of the probability of collision.

% from the \emph{true} probability of collision, $p_{true}^{ij}$. 

% If the true probability is know, the chance constraint can be exactly evaluated: $p_{true}^{ij}<1-p_\safe$. To evaluate conservatism, we randomly sampled $N_{MC}$ sets of two beliefs for each agent, $B_{s_{ij}}=\{\expBelief^i, \expBelief^j\}, s_{ij}=1,...,N_{MC}$, with $N_{MC}=10,000$. For each pair, we then computed the true probability of collision, and evaluated the chance constraint to obtain either valid or invalid, $\bar{v}^{s_{ij}}=1$ or $0$ respectively. The true validity rate can then be calculated as $\bar{V}_{true}=\sum_{s_{ij}=1}^{N_{MC}} \bar{v}^{s_{ij}}/N_{MC}$. For each pair, we then evaluate the chance constraint according to each of the proposed collision checking algorithm, $\bar{v}_{alg}^{s_{ij}}\in\{0,1\}$, yielding an associated algorithmic validity rate $\bar{V}_{alg}=\sum_{s_{ij}=1}^{N_{MC}} \bar{v}_{alg}^{s_{ij}}/N_{MC}$. Conservatism is defined as the difference between the true validity rate and the algorithmic validity rate: $C_{alg}=\bar{V}_{alg}-\bar{V}_{true}$. This indicates how often the algorithm rejects as invalid a sample that actually satisfies the chance constraint.


% \ml{all you're saying is that conservatism of method $z$ is defined as $C_{z} = R_{z} - R_{MC}$, where $R_{z}$ is the rejection rate obtained by method $z$, and $R_{MC}$ the rejection rate obtained by Monte Carlo evaluation of the probability of collision????}

% \ml{is this paragraph needed?}
% As stated earlier, the entire motivation for the proposed agent-agent collision checking algorithms is the difficulty of evaluating the true probability of collision. For the benchmarks performed here, we approximate the true probability of collision via MC sampling. For each pair of sampled beliefs, we further sampled 50,000 states. Each sampled state corresponds to a projection of the robots bodies in the workspace, where collision is indicated by the intersection of the two projections. To calculate the 'true' probability of collision, we found the proportion of the sampled states that resulted in collision.

We sampled 2D beliefs in two difference spaces, 5$\times$5 and 10$\times$10, with $p_\safe=0.95$. For M.2.2 (the gridded cdf method), we chose 6 different grid discretizations, denoted by M.2.2($d$). For each discretization, the maximum range of $\mathcal{P}_{whit}^{ij}$ over each axis, $D_w$, was divided equally, such that the discretization step size was $D_w/d$. Note that each of the proposed methods decreases in conservatism, shown in Fig.~\ref{fig:conservatism}, and increases in computation time, shown in Fig.~\ref{fig:compTime_combined}. The 5$\times$5 space is the more `cluttered', with more beliefs in close proximity, which in turn results in higher collision probabilities. Note that the 5$\times$5 space generally results in lower conservatism. This is because for the relatively high choice of $p_\safe$, a high proportion of the sampled beliefs truly violated the chance constraint, making it less conservative compared to a scenario where more beliefs are valid.

% \begin{figure}
%     \centering
%     \includegraphics[width=0.45\textwidth]{figures/PcollVSdist_Comp.pdf}
%     \vspace{-2mm}
%     \caption{Plot of the true probability of collision vs the euclidean distance between the means for each sampled belief pair}
%     \label{fig:PcollvsDist}
%     \vspace{0mm}
% \end{figure}



%%%%%%%%%%%%%%%%% SIDE BY SIDE TABLE
% \begin{table*}[]
% \caption{CC-K-CBS benchmarks for 2D system in Env8}
%     \centering
%     \scalebox{0.84}{
%     \begin{tabular}{|c|c||c|c|c|c|c|c||c|c|c|c|c|c|}
%          \hline
%          & & \multicolumn{6}{c||}{Simple Linear} & \multicolumn{6}{c|}{Unicycle}  \\ 
%         $N_A$ & Metric & M.1 & M.2.1 & M.2.1* & M.2.2(2) & M.2.2(5) & M.2.2(10)    \\ \hline 
%         \multirow{3}{*}{2} &  Succ. Rate &  1.00&   1.00 &   1.00 &   1.00 &   1.00 &   1.00 &  1.00&   1.00 &   1.00 &   1.00 &   1.00 &   1.00\\ 
%          &  Time (s) &    1.73 &   0.71 &   \textbf{0.56} &   0.87 &   1.19 &   0.75  & 0.03 &   0.03 &   0.05 &   0.04 &   0.04 &   0.06 \\ \hline 
%         \multirow{3}{*}{4} &  Succ. Rate &  0.00&   1.00 &   1.00 &   1.00 &   1.00 &   1.00  & 1.00&   1.00 &   1.00 &   1.00 &   1.00 &   1.00\\ 
%          &  Time (s) &     - &  11.86 &  41.92 &   \textbf{5.33} &   6.09 &  10.89  &    0.21 &   0.36 &   0.27 &   0.27 &   0.36 &   0.57   \\ \hline 
%         \multirow{3}{*}{6} &  Succ. Rate &  -&   0.00 &   0.20 &   1.00 &   1.00 &   1.00  &  0.00&   0.00 &   0.00 &   0.00 &   0.00 &   0.00 \\ 
%          &  Time (s) &     - &    - & 127.82 &   \textbf{5.53} &   8.61 &  16.43   &     NaN &    NaN &    NaN &    NaN &    NaN &    NaN  \\ \hline 
%         \multirow{3}{*}{8} &  Succ. Rate &  -&   - &   0.00 &   \textbf{0.98} &   0.94 &   0.90   &   1.00  &  0.00&   0.00 &   0.00 &   0.00 &   0.00  \\ 
%          &  Time (s) &    - &   - &   0.00 &  \textbf{18.56} &  29.77 &  61.79  &     NaN &    NaN &    NaN &    NaN &    NaN &    NaN   \\ \hline     
%     \end{tabular}
%     }
%     \label{tab:benchCCKCBS}
% \end{table*}



\begin{figure}[t]
    \centering
    \begin{subfigure}[b]{0.49\columnwidth}
         \centering
         \includegraphics[width=\textwidth]{figures/unicycle_SampleTraj_8x8.pdf}
         \caption{Env8}
         \label{fig:unicycle_SampleTraj_8x8}
     \end{subfigure}
    \begin{subfigure}[b]{0.485\columnwidth}
         \centering
         \includegraphics[width=\textwidth]{figures/unicycle_SampleTraj_32x32obs.png}
         \caption{Env32Obs}
         \label{fig:unicycle_SampleTraj_32x32obs}
     \end{subfigure}
     \hfill
     \hspace{-2mm}
     \caption{(a) Sample plan for 4 robots with 2nd-order unicycle dynamics in Env8. Ellipses are $95\%$ probability contours. (b) Sample plan for 20 robots with 2nd-order unicycle dynamics in Env32Obs. Ellipses are $95\%$ probability contours.}
     \label{fig:envs}
     % \vspace{-2mm}
\end{figure}

%%%%%%%%%%%%%%%%% OLD RESULTS TABLE (JUST LINEAR)

% \begin{table}[]
% \caption{CC-K-CBS benchmarks for Env8 (OLD RESULTS)}
%     \centering
%     \scalebox{0.84}{
%     \begin{tabular}{|c|c|c|c|c|c|c|c|}
%          \hline
%         $N_A$ & Metric & M.1 & M.2.1 & M.2.1* & M.2.2(2) & M.2.2(5) & M.2.2(10) \\ \hline 
%         \multirow{3}{*}{2} &  Succ. Rate &  1.00&   1.00 &   1.00 &   1.00 &   1.00 &   1.00\\ 
%          &  Time (s) &    1.73 &   0.71 &   \textbf{0.56} &   0.87 &   1.19 &   0.75 \\ \hline 
%         \multirow{3}{*}{4} &  Succ. Rate &  0.00&   1.00 &   1.00 &   1.00 &   1.00 &   1.00\\ 
%          &  Time (s) &     - &  11.86 &  41.92 &   \textbf{5.33} &   6.09 &  10.89 \\ \hline 
%         \multirow{3}{*}{6} &  Succ. Rate &  -&   0.00 &   0.20 &   1.00 &   1.00 &   1.00\\ 
%          &  Time (s) &     - &    - & 127.82 &   \textbf{5.53} &   8.61 &  16.43 \\ \hline 
%         \multirow{3}{*}{8} &  Succ. Rate &  -&   - &   0.00 &   \textbf{0.98} &   0.94 &   0.90\\ 
%          &  Time (s) &    - &   - &   0.00 &  \textbf{18.56} &  29.77 &  61.79 \\ \hline   
%     \end{tabular}
%     }
%     \label{tab:benchCCKCBS}
% \end{table}




%%%%%%%%%%%%%%%%% NEW RESULTS STACKED TABLE
\begin{table}[t]
\caption{CC-K-CBS benchmarks for Env8.}
    \centering
    \scalebox{0.82}{
    \begin{tabular}{|c|c|c|c|c|c|c|c|}
         \hline
         \multicolumn{8}{|c|}{2D Linear} \\ \hline
        $N_A$ & Metric & M.1 & M.2.1 & M.2.1* & M.2.2(2) & M.2.2(5) & M.2.2(10) \\ \hline 
        \multirow{3}{*}{2} &  Succ. Rate &  1.00&   1.00 &   1.00 &   1.00 &   1.00 &   1.00\\ 
         &  Time (s) &    1.08 &   \textbf{0.68} &   1.04 &   1.11 &   0.69 &   0.74 \\ \hline 
        \multirow{3}{*}{4} &  Succ. Rate &  0.00&   1.00 &   1.00 &   1.00 &   1.00 &   1.00\\ 
         &  Time (s) &     - &  18.68 &  35.03 &   4.19 &   \textbf{3.97} &   7.61 \\ \hline 
        \multirow{3}{*}{6} &  Succ. Rate &  -&   0.00 &   0.12 &   1.00 &   1.00 &   1.00\\ 
         &  Time (s) &     - &    - &  66.68 &   \textbf{4.44} &   6.15 &  12.72 \\ \hline 
        \multirow{3}{*}{8} &  Succ. Rate &  -&   - &   0.00 &   1.00 &   1.00 &   0.86\\ 
         &  Time (s) &     - &    - &    - &  \textbf{15.38} &  26.54 &  59.08 \\ \hline  
         \multicolumn{8}{|c|}{Unicycle} \\ \hline
        \multirow{3}{*}{2} &  Succ. Rate &  1.00&   1.00 &   1.00 &   1.00 &   1.00 &   1.00\\ 
         &  Time (s) &    0.95 &   0.63 &   0.61 &   \textbf{0.37} &   0.57 &   0.83 \\ \hline 
        \multirow{3}{*}{4} &  Succ. Rate &  0.00&   0.96 &   1.00 &   1.00 &   1.00 &   1.00\\ 
         &  Time (s) &     - &  17.67 &   9.81 &   \textbf{2.59} &   2.86 &   5.24 \\ \hline 
        \multirow{3}{*}{6} &  Succ. Rate & -&   0.00 &   0.00 &   1.00 &   1.00 &   1.00\\ 
         &  Time (s) &     - &    - &    - &   \textbf{4.79} &   5.95 &  10.40 \\ \hline 
        \multirow{3}{*}{8} &  Succ. Rate &  -&   - &   - &   1.00 &   0.98 &   0.96\\ 
         &  Time (s) &     - &    - &    - &  \textbf{18.27} &  25.52 &  45.73 \\ \hline 
    \end{tabular}
    }
    \label{tab:benchCCKCBS}
\end{table}





% \subsection{Characterizing the MRMP Planners}
\textbf{Planners Benchmark: }
% We first compare benchmarks on the performance of the centralized and CC-K-CBS planners proposed in Sec.~\ref{sec:method}. We then present specific plans generated by the decentralized method, along with results from Monte Carlo (MC) simulations demonstrating the plans' validity. 
% \subsubsection{Planner Benchmarking}
We use benchmarking to evaluate the proposed planners under each of the collision-checking methods in Sec.~\ref{sec:collisionChecking}. We additionally implement the adaptive risk allocation on M.2.1, denoted by M.2.1*. All other methods use equal allocation. 
% We are interested in comparing two quantities: \emph{success rate} and \emph{computation time}. 
% The first is the success rate, defined as the rate at which the planner returns a solution within the given maximum planning duration. The second is the computation time, which indicates the computational complexity of the algorithms. 
Benchmarking was performed using 50 runs for each algorithm with a maximum planning time of 3 minutes and $P_\safe = 0.9$. We examine 3 environments: a small 8$\times$8 (Env8), large 32$\times$32 with 50 random obstacles (Env32Obs), and large 32$\times$32 without obstacles (Env32) shown in Figs.~\ref{fig:unicycle_SampleTraj_8x8}, \ref{fig:unicycle_SampleTraj_32x32obs}, and \ref{fig:unicycle_SampleTraj_big}, respectively.
The considered metrics are \emph{success rate} (the ratio of the successful planning instances within 3 minutes to all instances) and \emph{time} (computation time of the successful runs).

%%%%%% BACKUP PARAGRAPH
\textit{Env8: } This small environment allows us to study how the different collision checking methods perform in a cluttered environment with the 2D system. The results are shown in Table~\ref{tab:benchCCKCBS}. Conservatism strongly impacts the planner's ability to find paths with additional robots. All methods reliably find plans for 2 robots, however as the conservatism of the method increases, the success rate decreases when more agents are added. In this cluttered environment robot paths are necessarily close together, so more conservative methods are less likely to find valid paths.  Specifically, M.1 performs poorly and M.2.2 has the best performance.

% \textit{Env8: } This small environment allows us to study how the difference collision checking methods perform in a cluttered environment. The results are shown in Table~\ref{tab:benchCCKCBS}. Here the conservatism of the method strongly impacts the planners ability to find paths when more agents are introduced. All methods reliably find plans for 2 robots, however as the conservatism of the method increases, the success rate decreases when more agents are added. In this cluttered environment robot paths are necessarily close together, so more conservative methods are less likely to find valid paths.  Specifically, Method 1 performs poorly and Method 2.2 has the best performance.

\textit{Env32Obs: } The large environment is far less cluttered, allowing for far more agents and better insight into the scalability of the CC-K-CBS algorithm. The results for Env8 are based on the robot-obstacle collision checker from \cite{Luders2010_CC-RRT}. However, this is very conservative, especially for many obstacles.  CC-K-CBS with this method was unable to find paths in Env32Obs for even 2 robots. To scale to a larger number of robots, we made the very straightforward adaptation to M.1 to check intersection of the bound $\mathcal{C}_{bound}^i$ with obstacles. 
As shown in Fig~\ref{fig:benchmark 30robots}, CC-K-CBS is able to scale to $20$ robots with $68\%$ success rate and runtime of $120.7\pm5.45$ seconds for the 2nd-order unicycle dynamics. A sample plan is shown in Fig.~\ref{fig:unicycle_SampleTraj_32x32obs}.  This illustrates that for large number of robots, M.1 has the best performance because it is not affected by risk allocation, whereas the allocated risk gets prohibitively smaller as the number of robots increases for the other methods, making it difficult to find valid plans.

We also performed benchmarking on the centralized approach with the simpler 2D system in Env32Obs, using collision checking M.1. 
% We allocated the full planning time for optimization, so computation time is not reported. The centralized approach allows optimization via the SST algorithm, which does result in improved sum of controls. However, it quickly runs into scalability problems, 
This algorithm was able to plan for 2 robots with $100\%$ success rate, but the success rate dropped to $8\%$ for 3 robots, illustrating the classical scalability issue with centralized approaches. 

\textit{Env32: } To further test the scalability of CC-K-CBS, we increased the number of agents in the large open environment Env32. 
As shown in Fig~\ref{fig:benchmark 30robots}, CC-K-CBS is able to to scale to $28$ 2nd-order unicycle robots with $56\%$ success rate and $129\pm4.97$ seconds computation time. 
A sample plan is shown in Fig.~\ref{fig:unicycle_SampleTraj_big}.  CC-K-CBS is also able to plan for 30 robots but at much lower success rate of $18\%$.

\begin{figure}
    \centering
    \includegraphics[width=0.86\columnwidth]{figures/compareBench_32x32.pdf}
    % \vspace{-2mm}
    \caption{Benchmark results for CC-K-CBS on varying number of 2nd-oder unicycle robots. For \# robots $\leq 20$, Env32Obs is used (dashed line), and for \# robots $\geq 22$, Env32 is used (solid line). The discontinuity is due to removing obstacles.
    }
    \label{fig:benchmark 30robots}
\end{figure}


% \subsubsection{Planner Behavior}
We validated the robustness of the motion plans using Monte Carlo simulation. We randomly selected motion plans, and with 500 simulations for each robot and observed no more than $2\%$ collision probability over the entire trajectory, illustrating the plans are indeed robust to uncertainty. 

%We present sample plans generated by the K-CBS algorithm under both dynamics models with $p_\safe=0.90$. These plans were generated in a 32x32 environment containing 50 random obstacles. Figure \ref{fig:unicycle_SampleTraj_big} presents the nominal trajectories for 12 robots, with the $90\%$ confidence bound plotted as ellipses. We verified that these bounds do not simultaneously intersect at any point in the nominal trajectories, nor do they intersect with obstacles. We additionally verify the probability of collision over entire sampled trajectories for the simple 2D linear system using 500 MC samples of each robot's trajectory. The empirically obtained probability of collision was $3.7\%$ for the simple 2D system. Examples of the nominal plans and sampled trajectories for the simple 2D system are shown in Figure \ref{fig:Simple2D_sampleTraj}.



% \begin{figure}
%     \centering
%     \includegraphics[width=0.4\textwidth]{figures/unicycle_SampleTraj.pdf}
%     \vspace{-2mm}
%     \caption{For the unicycle system, $95\%$ confidence bound plotted as ellipses, with the nominal trajectory over-plotted as a solid line for each of 7 agents.}
%     \label{fig:unicycle_sampleTraj}
%     \vspace{0mm}
% \end{figure}

% \begin{figure}
%     \centering
%     \begin{subfigure}{0.33\textwidth}
%     \includegraphics[angle=90,width=\textwidth]{figures/simpleLin_SampleTraj.pdf}
%     \vspace{-7mm}
%     \caption{}
%     \label{fig:Simple2D_sampleTraj}
%     \end{subfigure}
%     \hfill
%     \begin{subfigure}{0.14\textwidth}
%     \includegraphics[width=\textwidth]{figures/simpleLin_SampleTraj_narrow.pdf}
%     \vspace{-7mm}
%     \caption{}
%     \label{fig:Simple2D_sampleTraj_narrow}
%     \end{subfigure}
%     \hfill

% \caption{(a) For the simple 2D system in a 32x32 environment: 100 MC sampled trajectories plotted as transparent lines, with the nominal trajectory over-plotted as a solid line. (b) Sample 6x2 narrow environment}
% \vspace{-3.5mm}
% \end{figure}



