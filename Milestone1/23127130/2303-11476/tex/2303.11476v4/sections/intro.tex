
Robotic teams with diverse capabilities provide a powerful tool for efficiently accomplishing complex tasks. 
Applications extend from planetary exploration to automated warehouses and search and rescue missions.  However, planning for many robots is a challenging problem.
A major difficulty of \textit{multi-robot motion planning} (MRMP) lies in the size of the planning space, which grows exponentially with the number of robots. There
are also constraints posed by the robots' kinodynamics as well as robot-robot and robot-obstacle collisions that need to
be respected, which by itself is a difficult problem even for a
single robot. 
% A core challenge in \textit{multi-robot motion planning} (MRMP) is ensuring all robots avoid collision with each other and obstacles while accomplishing the assigned task. 
Additional complexities arise for realistic robots due to uncertainty in both motion and sensing, which the planner needs to account for in collision checking. 
In this work, we focus on the uncertain MRMP problem, and develop a scalable algorithm that guarantees collision avoidance and reaching goal with completeness properties.

Extensive work has considered MRMP problem in deterministic settings.
Proposed methods %is a well established research topic in robotics, and 
are either centralized (coupled)~\cite{Lavalle98rapidly-exploringrandom,wagner2015subdimensional,shome2020drrt,Kottinger:ICRA:2021,sucan2011sampling} or decentralized (decoupled)~\cite{gammell2014bit,tang2018complete,Le:ICAPS:2017,Le:RAL:2019}. %fall into two categories. %: centralized %(coupled) 
%and decentralized. %(decoupled). 
Centralized planners compose all robots' states together into a single meta-robot state, to which single-robot planning methods with completeness guarantees can then be applied. %Single robot planning techniques can then be readily applied to this meta-robot, which inherits their completeness properties. 
However, the exponential state increase makes centralized approaches scale poorly. Decentralized methods address this by planning for each robot individually and considering system collisions separately. These algorithms drastically cut computation time but often sacrifice completeness guarantees. For example, \cite{vandenBerg2005_prioritized} 
plans for one robot at a time, where each robot is constrained by the paths of all previously planned robots. While fast, this method is provably incomplete~\cite{lavalle2006planning}. % \at{citation?}. 
%  Other MRMP approaches use online distributed planning, e.g., \cite{Patwardhan2023_DistMRMP,AlonsoMora2018_CoopCollAvoid,Senbaslar2019_MultiRobotDistributed,VanParys2016_OnlineDistMultiVehicle}
%  \ml{are these for uncertain robots? This paragraph is for deterministic robots..}
%  ,
% %. These involve distributed planners, 
% where each robot plans over a short horizon to account for nearby robots and obstacles. %in the immediate vicinity. 
% These methods perform well in unknown environments where cooperation is not prioritized, but do not provide formal performance guarantees\jk{can we also say they do not provide correctness guarantees?}. %\at{I'm not really sure this paragraph is necessary...}
One recent contribution to scalable MRMP is Kinodynamic Conflict-Based Search (K-CBS), \cite{Kotting2022_KCBS}. K-CBS pairs a low-level kinodynamic motion planner for individual robots with a high-level tree search over the space of possible plans, which checks for robot-robot collisions and constructs the corresponding time dependent constraints to pass to the low-level motion planner. K-CBS offers users scalability and includes a merging technique to uphold probabilistic completeness properties of the low-level motion planner. Although a great step forward, none of these approaches are robust to state and measurement uncertainty, which are inherent to realistic robotic systems.  



\begin{figure}
    \centering
    \includegraphics[width=0.3\textwidth]{figures/unicycle_SampleTraj.png}
    \caption{Sample plans for 30 robots with 2nd-order unicycle dynamics under both motion and sensing uncertainties. Ellipses are $95\%$ safety contours. Star-centered circles are goal regions.}
    \label{fig:unicycle_SampleTraj_big}
    \vspace{-5mm}
\end{figure}

Motion planning under Gaussian uncertainty has largely been studied for single-robot systems. The most efficient of these are sampling-based planners, which are analogous to their deterministic counterparts, e.g., the feedback-based information roadmap \cite{Agha-Mohammadi2014_FIRM} %is an extension of probabilistic roadmaps (PRM), 
and rapidly-exploring random belief trees \cite{Bry2011_BeliefProp}. %are an extension of rapidly-exploring random trees (RRTs). 
%We adopt 
The belief-$\mathcal{A}$ framework introduced in \cite{Ho2022_GBT} provides a generic structure for adapting any sampling-based kinodynamic planner $\mathcal{A}$ to the belief space. It shows that fast planning is possible by using chance constraint formulation of collision avoidance.
Although these approaches can solve uncertain motion planning, they cannot solve uncertain MRMP out-of-the-box. 
% Wrapping this single agent planner in the K-CBS framework extends it to MRMP while maintaining %the planner's efficiency and performance guarantees. 

One approach to uncertain MRMP is to abstract the problem to a multi-agent Markov Decision Process (MMDP), which can be reduced to single-agent MDPs with a joint action space \cite{Boutilier1996_MMDP}. This relies on abstraction of the robot dynamics to a discretized transition probability matrix, which is difficult and scales poorly to many robots. Additionally, MMDPs solely account for uncertain dynamics, whereas measurement noise requires using a Partially-Observation MDP (POMDP), which can be very difficult to solve and also scales poorly.  Other uncertain MRMP approaches use online distributed planning, e.g., \cite{Patwardhan2023_DistMRMP,AlonsoMora2018_CoopCollAvoid,Senbaslar2019_MultiRobotDistributed,VanParys2016_OnlineDistMultiVehicle}, where each robot plans over a short horizon to account for nearby robots and obstacles. %in the immediate vicinity. 
These methods perform well in unknown environments. However, neither POMDP approaches nor online distributed methods provide any formal performance or completeness guarantees. %\at{I'm not really sure this paragraph is necessary...}

% On the other hand, single agent planners exist which do account for measurement noise and provide performance guarantees. 
 
% \nra{Zach Sunberg does, since he knows how to solve certain varieties quite efficiently, so not the best argument-rephrase/justify: *why* wouldn't you want this? doesn't provide any formal guarantees, or what?}. \nra{There is a disconnect in the argument here -- you go from talking about  MDPs/POMDPs for multiple robots to suddenly talking about single-robot planning techniques -- why?} 
%A broader variety of techniques exist for planning under uncertainty with a single robot \nra{...and so what? we can seek to adapt these to accommodate decentralized multi-robot planning under uncertainty? why are these worth looking at -- b/c in single-robot case they can provide guarantees under kinds of uncertainties we want to examine for multi-robot case? or something else?}. 

This work addresses this gap by presenting a scalable MRMP algorithm for robots operating with Gaussian state and measurement noise.
We combine %best aspects of K-CBS and belief-$\mathcal{A}$ to address MRMP under uncertainty: K-CBS enables 
the scalability of K-CBS with the fast belief-space planning of belief-$\mathcal{A}$ 
%enables fast belief space planning, and 
to solve MRMP for robots under Gaussian uncertainty. Our algorithm, Chance-Constrained K-CBS (CC-K-CBS), is probabilistically complete via the completeness inheritance properties of K-CBS and belief-$\mathcal{A}$~\cite{Kotting2022_KCBS,Ho2022_GBT}. 
% We present several new algorithms for probabilistic robot-robot collision checking. 

We present three main contributions: (i) three probabilistic robot-robot collision checking algorithms of varying conservatism and computational complexity; (ii) a scalable decentralized planner that uses novelties from (i) to solve MRMP for robots operating with Gaussian noise; 
%\nra{would be worth saying something a bit more technically insightful/appealing here -- why should reader keep on reading? are these new techniques that have not been developed previously?}\at{The techniques are kind simple, the novelty is in applying them in the multi-robot planning framework. Also, have not found many sources that directly benchmark computation and conservatism of collision checkers}; 
and (iii) case studies and benchmarks demonstrating the applicability of the larger MRMP framework, validity of our collision-checking methods, and comparisons between the algorithms. 
% Though the methods presented here are relatively simple, they have not been previously applied to multi-robot planning under uncertainty, and benchmarks evaluations for computation and conservatism have also not yet been established in this context. 

