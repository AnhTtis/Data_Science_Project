





Recall that the probability of collision between robots $i$ and $j$, as defined in  \eqref{eq:prob robot-robot collision}, requires integration of the joint probability distribution of the two robots over set $\X^{ij}_\coll$. 
This computation is difficult because (i) $\X^{ij}_\coll$ is hard to construct, and (ii) integration of the joint probability distribution function is expensive.  However, the constraint-checking procedure must be extremely fast in sampling-based algorithms because it is called in every iteration of tree extension. In this section, we introduce three approximation methods for this integration that trade off accuracy with computation effort.  
%The methods rely on bounding the projection of $\X^{ij}_{\coll}$ into the lower dimensional workspace, where checking the chance constraint is simpler. 

\begin{figure}[b]
    \centering
    \begin{subfigure}{0.26\textwidth}
    \includegraphics[width=\textwidth]{figures/minkVol.pdf}
    \vspace{-7mm}
    \caption{}
    \label{fig:CspaceIntegration}
    \end{subfigure}
    \centering
    \begin{subfigure}{0.2\textwidth}
    \includegraphics[width=\textwidth]{figures/minkVolSlice.pdf}
    \vspace{-7mm}
    \caption{}
    \label{fig:CspaceIntegration_slice}
    \end{subfigure}
\caption{(a) 3D view of the c-space obstacle $\X_{coll}^{ij}$ 
% approximately constructed as the conjunction of a series of Minkowski sums of uncertain 
for
robot $i$ with fixed robot $j$ calculated at discrete choices of $\theta_k^i$.  
% (b) Example Minkowski sum for a single choice of $\theta_k^i$, depicted by the intersection of a gray plane in the top figure. 
(b)~Top-down view of $\X_{coll}^{ij}$.
The projection of $\X_{coll}^{ij}$ onto x-y plane is bounded by a green circle.
}
% \vspace{-3.5mm}
\end{figure}

For clarity of explanation, consider two deterministic (rotating and translating) robots $i$ and $j$ in a 2D workspace, with configurations $(\text{x}_k^l,\text{y}_k^l,\theta_k^l)$, $l\in \{i,j\}$. We can construct the 4D c-space obstacles for this pair by taking the Minkowski difference of the two agents for every pair of orientation angles (see the c-space construction in Fig.~\ref{fig:CspaceIntegration}, which shows the c-space in 3D for a given orientation of agent $j$). This 4D obstacle is generally difficult to construct, so deterministic collision checkers typically resort to collision checking in the workspace, e.g., by bounding the projection of this set, shown by a green disk in Fig. \ref{fig:CspaceIntegration_slice}. 



Now, consider that each robot is described by a random variable (RV), so the position and orientation of their bodies are no longer deterministic but given by beliefs, such that each point in the 4D c-space maps to a probability. The exact probability of collision is the integral of the probability mass over the c-space obstacle. 
% Hence, the uncertain collision checking problem has two main challenges: construction of a complex c-space obstacle, and integration over this complex shape. 
This results in the two challenges listed above: c-space obstacle construction and integral computation.
Each method 
% listed in the following sections address these two core problems.
proposed below attempts to address these core problems. 
% Approximations and simplifications made to increase efficiency also introduce conservatism, which leads the collision checker to reject valid beliefs. Each of the methods described strikes a difference balance between these two competing factors.

%Then, the c-spaces are 3D and consist of a 2D Euclidean coordinate (position) and an orientation angle, i.e., $x_k^l=(\text{x}_k^l,\text{y}_k^l,\theta_k^l)$, $l\in \{i,j\}$. 
% These robots operate in a 2D workspace $\W$, so their projections $\proj{W}(x_k^l), l={i,j}$ are also 2D.
%Furthermore, assume only robot $i$ has uncertainty, and the state of robot $j$ is fixed and known. This reduces robot-robot collision checking to checking an uncertain robot with a known obstacle. In this case, the c-space is 3-dimensional, and the integration area for collision can be constructed by determining the Minkowski sum of the uncertain robot with the fixed robot for each possible orientation $\theta_k^i$. This is depicted by a pink volume in Figure \ref{fig:CspaceIntegration}. A simple way to bound the projection on $[\text{x}_k^i, \text{y}_k^i]$ space is to bound both robots' bodies with a disk that accounts for all possible orientations. The projection is then bounded by a circle with radius equal to the sum of the radii of the two bounding disks, shown as a green circle in Figure \ref{fig:CspaceIntegration}. The concept of bounding the projection in 2D euclidean space intuitively extends to higher dimensional state spaces. 

%We present three methods for checking the robot-robot collision constraint. Each method is computationally cheap, but has varying levels of conservatism. The sources of conservatism arise from two main sources: bounds on the orientation of the robot body, and approximations introduced to ease evaluation of complex integrals. The results section presents an empirical study of conservatism and computation time for various systems and environments. We present the methods in order of increasing computation time, and decreasing conservatism.
\begin{figure}
     \centering
     \begin{subfigure}[b]{0.45\columnwidth}
         \centering
         \includegraphics[width=\textwidth]{figures/discBody.pdf}
         \vspace{-5mm}
         \caption{}
         \label{fig:diskBoundBody}
     \end{subfigure}
     \hfill
     \begin{subfigure}[b]{0.45\columnwidth}
         \centering
         \includegraphics[width=\textwidth]{figures/individualAgentsProb.pdf}
         \vspace{-5mm}
         \caption{}
         \label{fig:AgentCircPsafe}
     \end{subfigure}
     \newline
     \begin{subfigure}[b]{0.45\columnwidth}
         \centering
         \includegraphics[width=\textwidth]{figures/DiskIntersect.pdf}
         \vspace{-5mm}
         \caption{}
         \label{fig:diskIntersect}
     \end{subfigure}
     \hfill
     \begin{subfigure}[b]{0.45\columnwidth}
         \centering
         \includegraphics[width=\textwidth]{figures/diffProbCirc.pdf}
         \vspace{-5mm}
         \caption{}
         \label{fig:CircularBodyCicularIntegration}
     \end{subfigure}
     \newline
     \begin{subfigure}[b]{0.45\columnwidth}
         \centering
         \includegraphics[width=\textwidth]{figures/diffProbCircPoly.pdf}
         \vspace{-5mm}
         \caption{}
         \label{fig:CircularBodyPolygonIntegration}
     \end{subfigure}
     \hfill
     \begin{subfigure}[b]{0.45\columnwidth}
         \centering
         \includegraphics[width=\textwidth]{figures/diffProbCircPoly_grid.pdf}
         \vspace{-5mm}
         \caption{}
         \label{fig:CircularBodyGridIntegration}
     \end{subfigure}
     \caption{
(a) Circle bound on robots' bodies. (b) $p_\safe$ contour and corresponding circular over bound on $p_\safe$. (c) Inflated circular bounds intersect, indicating possible collision between robots. (d) Integration area for probability of collision with circular disk bound on robots' bodies. (e) Polygonal bound on circular integration area. (f) Transformed distribution and integration area, with over-approximated grid.}
% \vspace{-1mm}
\end{figure}

% \begin{figure}
%     \centering
%     \begin{subfigure}{0.3\textwidth}
%     \includegraphics[width=\textwidth]{figures/diffProbCirc.pdf}
%     \vspace{-7mm}
%     \caption{}
%     \label{fig:CircularBodyCicularIntegration}
%     \end{subfigure}
%     \\
%     \centering
%     \begin{subfigure}{0.3\textwidth}
%     \includegraphics[width=\textwidth]{figures/diffProbCircPoly.pdf}
%     \vspace{-7mm}
%     \caption{}
%     \label{fig:CircularBodyPolygonIntegration}
%     \end{subfigure}
% \caption{(a) Integration area for probability of collision with circular disk bound on robots' bodies. (b) Polygonal bound on circular integration area}
% \vspace{-3.5mm}
% \end{figure}



% \begin{figure}
%     \centering
%     \includegraphics[width=0.3\textwidth]{figures/diffProbCircPoly_grid.pdf}
%     \vspace{-7mm}
%     \caption{}
%     \label{fig:CircularBodyGridIntegration}
% \caption{Transformed distribution and integration area, with over-approximated grid.}
% \vspace{-3.5mm}
% \end{figure}







\begin{comment}
%%% Old table, kept for reference for previous benchmarks    
\begin{table*}[]
    \centering
    \begin{tabular}{c|c|c|c|c}
         Method &  Body Bound & Integration Area & Bound on Integration Area & CC Evaluation\\
         \hline
         1 & Sphere & N/A & N/A & Intersection \\
         2 & Polyhedron & Polyhedron & N/A & Linear Constraint \\
         3 & Sphere & Sphere & Polyhedron & Linear Constraint \\
         4 & Sphere & Sphere & N/A & Grid CDF       
    \end{tabular}
    \caption{Summary table of collision checking methods.}
    \label{tab:ccMethodsSummary}
\end{table*}
\end{comment}

\subsection{Method 1 (M.1): Safety Contour}
\label{sec:Method1}
M.1 is inspired by deterministic collision checking methods, where collisions are detected by intersections of robot bodies in the workspace. We adapt this framework to uncertain collision checking by inflating bounds on the robot bodies to encapsulate $p_\safe$ probability mass.

%We rely on calculating a $w$-sphere for each robot in the workspace that is guaranteed to contain at least $p_{\safe}$ probability mass, we then check for collisions by ensuring these spheres do not intersect. A simplified 2D example is shown in Figures \ref{fig:diskBoundBody}, \ref{fig:AgentCircPsafe}, and \ref{fig:diskIntersect}.

Recall that the expected belief is $\expBelief(x_k^i)= \N(\check{x}_k^i,\Gamma_k^i)$. We define the Gaussian marginal of this distribution in the workspace as $\expBelief_\W(x_k^i)$.
% such that 
% \begin{equation}
%     \expBelief_\W(x_k^i) = \int_{\phi_k^i} \expBelief_{x_k^i\mid _k^i}(x_k^i\mid \phi_k^i)\expBelief_{\phi_k^i}(\phi_k^i)d\phi_k^i
% \end{equation}
Let $\bar{x}^i_k \in \W$ be the position components of $x_k^i$, and $\bar{\Gamma} \in \reals^{w\times w}$ be the corresponding covariance matrix (sub-matrix of $\Gamma^i_k$). Then $\expBelief_\W(x_k^i) = \N(\bar{x}^i_k, \bar{\Gamma}^i_k)$.
% \ml{hm.. why? Why don't we look at the belief of the projection of $x^i_k$ onto the workspace?  Let $\bar{x}^i_k \in \W$ be the position components of $x_k^i$, and $\bar{\Gamma} \in \reals^{w\times w}$ be the corresponding covariance matrix (sub-matrix of $\Gamma^i_k$). Then $\expBelief_\W(x_k^i) = \N(\bar{x}^i_k, \bar{\Gamma}^i_k)$, no? }
% where $\phi_k^i$ are all the state space variables not contained in the workspace. \at{Not so sure about this, is there a better way to write this?} 
The elliptical probability contour containing $p_{\safe}$ probability mass in the workspace marginal distribution is the region $\mathcal{E}_{\safe}^i$ such that
% \begin{equation*}
    $\int_{\mathcal{E}_{\safe}^i} \expBelief_\W(x_k^i) dx=p_{\safe}.$
% \end{equation*}
The axes of this ellipsoid are given by $a_l^i=t_{\chi}\lambda_l, \quad l=1,\ldots,w,$ 
% $a^i_1,\ldots,a^i_w$ are defined by the eigenvalues of covariance $\bar{\Gamma}_k^i$
% , $eigenval(\Gamma_k^i)=(\lambda_1,...,\lambda_n)$, 
% such that 
% \begin{equation*}
%     a_l^i=t_{\chi}\lambda_l, \quad l=1,...,w,
% \end{equation*}
where $\lambda_l$ is an eigenvalue of $\bar{\Gamma}_k^i$, and
$t_{\chi}$ is the inverse $\chi^2$ cumulative density function
evaluated at $p_{\safe}$ with $w$ degrees of freedom.  

Determining intersection of ellipses is difficult, so we bound the ellipsoid with a sphere, where checking for intersection only requires comparing the sum of the radii to the distance between the centers. The sphere containing this ellipsoid, $\mathcal{C}_{\safe}^i$, is defined by the radius $r^i=t_{\chi}\lambda_{max}$, where 
% $\lambda_{max}=max(eigenval(\bar{\Gamma}_k^i))$. 
$\lambda_{\max}=\max\{ \lambda_l \}_{l=1}^w.$
Because $\mathcal{E}_{\safe}^i\subseteq \mathcal{C}_{\safe}^i$, it is assured that the probability mass contained by this sphere is greater than or equal to $p_{\safe}$, i.e., 
% \begin{equation}
    $\int_{\mathcal{C}_{\safe}^i} \expBelief_\W(x_k^i) dx\geq p_{\safe}.$
% \end{equation}
An illustration of both the elliptical contour and its associated spherical bound are shown in Fig.~\ref{fig:AgentCircPsafe} for the robots in Fig~\ref{fig:diskBoundBody}.

Note that this sphere contains only the probability mass associated with $x_k^i$, the origin point on the body frame. 
% \textcolor{blue}{Even if the origin point may satisfy the chance constraint, this does not imply the same for the entire body of the robot.}
It does not account for the collision probability of the other points on the robot body.
%\ml{unclear... reword this sentence}
To address this, we inflate the sphere by a bounding sphere on the robot's body (shown in Fig.~\ref{fig:diskBoundBody}), such that the probability of \emph{any} point on the robot's body being within this inflated sphere is greater than or equal to $p_{\safe}$. The radius, $R_i$, of the bounding sphere on the robot's body is obtained by finding the largest possible Euclidean distance between any pair of points within a robot's body: $R_i = \frac{1}{2}\max_{x_1,x_2\in \B^i} \| x_1-x_2 \|_2$. The inflated bound, $\mathcal{C}_{bound}^i$ is a sphere of radius $r^i+R^i$. The probability that any point of the robot is outside this sphere, the complement of $\mathcal{C}_{bound}^i$, is guaranteed to be less than $1-p_{\safe}$. Thus, if any two robot's bounding spheres
do not intersect, i.e. $\mathcal{C}_{bound}^i \cap \mathcal{C}_{bound}^j = \emptyset$, the probability of collision between them is guaranteed to be less than $1-p_{\safe}$.
% \begin{multline}
%     \mathcal{C}_{bound}^i \cap \mathcal{C}_{bound}^j = \emptyset  \quad \Rightarrow\\
%      P(\proj{\W}(x_k^i)) \cap \proj{\W}(x_k^j))\leq 1-p_{\safe}.
% \end{multline}
The inflated disks and corresponding intersection checking are shown in Fig.~\ref{fig:diskIntersect}.

This method has two main sources of conservatism. The first stems from the bounding sphere on the robot body. This is especially egregious for robots with high aspect ratios. The second source is associated with checking for intersections of the probability mass bound. This conservatively assumes all probability mass not within the sphere is in collision, so any intersection (no matter how small) is considered a possible constraint violation. Violation for this method does not imply that the actual probability of collision exceeds the constraint. Despite these limitations, this method is extremely fast. %collision checking method.
% of all those proposed here.

\subsection{Method 2: Linear Gaussian Transformation}
\label{sec:Method2}
This section introduces a class of methods that reduce conservatism by simplifying the integration itself, namely by reducing its dimensionality. This is done by introducing a new RV representing the difference between two robots' states: $x_k^{ij}=x_k^i-x_k^j$. This RV is Gaussian distributed such that $x_k^{ij}\sim\N(\check{x}_k^i-\check{x}_k^j, \Gamma_k^i+\Gamma_k^j)$, with the workspace marginal $\expBelief_\W(x_k^{ij})$. This new RV represents a vector connecting the origin points of two robots. Collisions correspond to specific realizations of $x_k^{ij}$, %this difference vector, 
defining the required integration  area. 

\begin{comment}
\begin{figure}
    \centering
    \begin{subfigure}{0.3\textwidth}
    \includegraphics[width=\textwidth]{figures/polyBodyBound.pdf}
    \vspace{-7mm}
    \caption{}
    \label{fig:polyBodyBound}
    \end{subfigure}
    \centering
    \begin{subfigure}{0.3\textwidth}
    \includegraphics[width=\textwidth]{figures/diffProbMinkPoly.pdf}
    \vspace{-7mm}
    \caption{}
    \label{fig:PolyBodyMinkowskiInt}
    \end{subfigure}
\caption{(a) Integration area for probability of collision with circular disk bound on robots' bodies. (b) Polygonal bound on circular integration area}
\vspace{-3.5mm}
\end{figure}


\subsubsection{ditching this method}
This method relies on a polyhedral bounds on the bodies of the robots in the workspace, these bounds are then used to construct Minkowski sums which define the integration area. Because this is still a complex integration area, we simplify the constraint checking by using the approximate linear constraints first derived for CC-RRT in \cite{Luders2010_CC-RRT}.

For each robot we define a polyhedra in the workspace, $\mathcal{P}_{poly}^i$,  that bounds the robot's body for all possible orientations. This can be constructed by defining a polyhedral bound over the sphere $S_{body}^i$ described in Section \ref{sec:Method1}, shown in Figure \ref{fig:polyBodyBound}. \at{Need more explanation here?}. We construct the Minkowski Sum for all possible pairs of robots, $\mathcal{P}_{Mink}^{ij}$. 

We then introduce a new random variable, $x_k^{ij}=x_k^i-x_k^j$ representing the difference between two robots' states, this random variable is Gaussian distributed such that $x_k^{ij}\sim\N(\check{x}_k^i-\check{x}_k^j, \Gamma_k^i+\Gamma_k^j)$, with the workspace marginal $\expBelief_\W(x_k^{ij})$. By integrating $\expBelief_\W(x_k^{ij})$ over the Minkowski sum, we can exactly find the probability of the two bounding polyhedra intersecting (indicating possible collision), shown in Figure \ref{fig:PolyBodyMinkowskiInt}. To check the chance constraint, we simply need to ensure this probability does not exceed $1-p_{\safe}$. However, this integral is difficult to calculate, so further approximation is required. We instead make the following approximations to simplify checking the chance constraint. 

The Minkowski sum can be given by the conjunction of a set of $N_p$ half planes: $\mathcal{P}_{Mink}^{ij} = \{ x \mid \mathrm{a}_{h,i}^Tx<b_{h,i}, \forall h\in[1,N_p]\}$, where $a_{h,i}\in\mathbb{R}^{1\times w}$ and $b_{h,i}\in\mathbb{R}$. This reduces the robot-robot chance constraint checking problem to the same problem addressed in \cite{Luders2010_CC-RRT}. In that work, various approximations were used to reduce the probabilistic chance constraint checking problem to checking a set of deterministic linear constraints. One of the key approximations was the allocation of the acceptable probability of collision, or risk allocation: the total acceptable probability of collision ($\Delta = 1-p_{\safe}$) must be distributed across all the possible sources of collision. In our case, we must distribute the probability of collision for an robot among the $N_A-1$ other robots. To keep things general, we we will denote the allowable probability for robot $i$ colliding with any other robot $j$ as $p_c^{ij}$. Two proposed methods of determining $p_c^{ij}$ are described in Section \ref{sec:Risk Allocation}. Under this generalization, we have adapted the linear constraints from \cite{Luders2010_CC-RRT} for the random variable $x_k^{ij}$ as:
\begin{multline}
    \bigvee_{h=1}^{N_p} a_{h,i}^T (\check{x}_k^i-\check{x}_k^j) \geq b_{h,i} + \bar{b}_{h,i}  \quad \Rightarrow  \\
    P(\proj{\W}(X_k^i)) \cap \proj{\W}(X_k^j))\leq 1-p_{\safe}
\end{multline}
where
\begin{align}    
    &\bar{b}_{h,i} = \sqrt{2}P_v erf^{-1}\left( 1-2 p_c^{ij} \right), \\
    &P_v = \sqrt{a_{h,i}^T(\Gamma_k^i+\Gamma_k^j)a_{h,i}}
\end{align}

There are two main sources of conservatism here. The first stems from the same source as the prior method: the polyhedral bound on the body is even more conservative than the spherical bound. This results in violation of the constraint in cases where the true orientations of the body do not actually intersect. The second source of conservatism arises from the use of deterministic linear constraints for the integration itself. This source of conservatism scales with the number of robots and the number of half planes in the Minkowski sum. This method is only slightly more computationally intensive than the first method. While checking linear constraints is very computationally simple, this method must at worst check every constraint for associated with each pair of robots. This yields slightly longer computation times than the first method, which required checking a single condition for every robot pair.
\end{comment}



\subsubsection{Method 2.1 (M.2.1): Convex Polytopic Bounding}
\label{sec:Method2.1}
%Note that 
The difference RV somewhat simplifies the high dimensional c-space obstacle representation, shown in Fig.~\ref{fig:CspaceIntegration}, by merging the workspace dimensions for the two robots. But the orientation of the robots must still be considered. To obtain a tractable integration area, we use the same bounding spheres described in Sec.~\ref{sec:Method1}. Intersection of the bounding spheres is considered a collision state, corresponding to a spherical integration area $S_{body}^{ij}$, with radius $R^i + R^j$, over difference distribution $\expBelief_\W(x_k^{ij})$ (shown in Fig.~\ref{fig:CircularBodyCicularIntegration}). 

Because exact integration of a Gaussian distribution over a sphere is difficult, we bound $S_{body}^{ij}$ with a polytope, $\mathcal{P}_{body}^{ij}$, which can be defined by a set of $N_s$ half planes, $\mathcal{P}_{body}^{ij} = \{ x \mid c_{h,i}^Tx<d_{h,i}, \forall h\in[1,N_s]\}$, where $c_{h,i}\in\mathbb{R}^{n\times 1}$ and $d_{h,i}\in\mathbb{R}$. This reduces the robot-robot chance constraint checking problem to the same problem addressed in \cite{Luders2010_CC-RRT},  where various (over-)approximations are used to reduce the probabilistic chance constraint checking to deterministic linear constraints checking. We employ the same methods to check the chance constraints, with the distribution given by $\N(\check{x}_k^i-\check{x}_k^j, \Gamma_k^i+\Gamma_k^j)$, and polytope given by $\mathcal{P}_{body}^{ij}$.

A key approximation in this method is the allocation of the acceptable probability of collision, or risk allocation: the total acceptable probability of collision $p_{coll} = 1-p_{\safe}$ must be distributed across all the possible sources of collision. In our case, we must distribute the probability of collision for a robot among the $N_A-1$ other robots. We denote the allowable probability for robot $i$ colliding with any other robot $j$ as $p_c^{ij}$. Two proposed methods of determining $p_c^{ij}$ are described in Sec.~\ref{sec:Risk Allocation}. 
% Under this generalization, we have adapted the linear constraints from \cite{Luders2010_CC-RRT} for the random variable $x_k^{ij}$ as:
% \begin{multline}
%     \text{if} \quad \bigvee_{h=1}^{N_c} c_{h,i}^T (\check{x}_k^i-\check{x}_k^j) \geq d_{h,i} + \bar{d}_{h,i} \\
%     \rightarrow P(\proj{\W}(X_k^i)) \cap \proj{\W}(X_k^j))\leq 1-p_{\safe}
% \end{multline}
% where 
% \begin{multline}    
%     \bar{d}_{h,i} = \sqrt{2}P_v erf^{-1}\left( 1-2 p_c^{ij} \right), \\
%     P_v = \sqrt{c_{h,i}^T(\Gamma_k^i+\Gamma_k^j)c_{h,i}}
% \end{multline}

This method has two main sources of conservatism. The first occurs with the spherical bound on the body, the same bound used in M.1 (Sec.~\ref{sec:Method1}) while the second arises from the use of deterministic linear constraints for the integration. This source of conservatism scales with the number of robots and the number of half planes in the polytope bound. This method is only slightly more computationally expensive than the first method: while checking linear constraints is very computationally simple, this method must at worst check every constraint associated with each pair of robots. This yields slightly longer computation times than the M.1, which requires checking a single condition for every robot pair. 


\subsubsection{Method 2.2 (M.2.2): Grid Integration}
This method follows directly from M.2.1, but with a direct integration technique, rather than a check on deterministic linear constraints. This makes this method more accurate and less conservative, but also more computationally intensive. 

We begin with the same difference distribution over $\expBelief_\W(x_k^{ij})$ 
and integration over $\mathcal{P}_{body}^{ij}$. We then perform a whitening (Mahalanobis) transformation on the distribution and $\mathcal{P}_{body}^{ij}$. This transformation decorrelates a Gaussian distribution, yielding the standard normal distribution, as well as rotating and translating $\mathcal{P}_{body}^{ij}$. We define this new integration area as $\mathcal{P}_{white}^{ij}$ (see Fig.~\ref{fig:CircularBodyGridIntegration}). 
Because we have transformed the probability distribution to a standard normal, the cumulative distribution function can be easily computed via the error function (erf); we do this over a grid. This technique is agnostic to the choice of grid cells and only requires rectangular cells that completely cover the polytope $\mathcal{P}_{white}^{ij}$, as shown in green in Fig. \ref{fig:CircularBodyGridIntegration}. The probability of collision is the sum of the probability mass contained in each grid cell, defined as $p_{poly}$. The chance constraint can then be simply checked as $p_{poly}\leq p_c^{ij}$,
where $p_c^{ij}$ again is the risk allocated to a pair of robots.

%we will define a grid discretization over the entire workspace, with discretization step size for each workspace axis as $d\W_w$.  Each grid cell, $\W_g$, is a defined as set of vertices, $v_\W\in\W$, such that the $w$-th element of $v_\W$, $v_{\W}(w)$ is an element of the adjacent set of grid points in axis $w$:
%\begin{equation}
%    v_{\W}(w) \in \{w_Nd\W_w, (w_N+1)d\W_w\}, \quad w_N\in\mathbb{Z}
%\end{equation}
%The set of cells that over-approximate the integration area is therefore: $\W_{g,poly}=(\W_g \mid \exists v_\W(w) \in \mathcal{P}_{whit}^{ij})$. This grid over approximation of the integration area is shown in Figure \ref{fig:CircularBodyGridIntegration}.
%\at{Bonkers way to define a grid, there must be a better way...}

%The probability contained in a grid cell, $p(\W_g)$ can then be defined in terms of the cdf evaluated at each vertex:
%\begin{equation}
%    cdf(v_\W) = \prod_{w} \frac{1}{2}\left( 1 + erf\left( \frac{v_\W(w)}{\sqrt{2}} \right) \right)
%\end{equation}

%The over approximation of the probability mass contained in $\mathcal{P}_{whit}^{ij}$, defined as $p_{poly}$ is 
%\begin{equation}
%    p_{poly} = \sum_{\W_g\in\W_{g,poly}} p(\W_g)
%\end{equation}

% The chance constraint can then be simply checked as:
% \begin{multline}
%     \text{if} \quad p_{poly}\leq p_\safe \\
%     \rightarrow P(\proj{\W}(X_k^i)) \cap \proj{\W}(X_k^j))\leq 1-p_{\safe}
% \end{multline}
This method inherits the same conservatism associated with the spherical body bound and polytope integration area bound in M.2.1. However, the approximate grid calculation is far less conservative than the deterministic linear constraints. The grid introduces more computational cost, particularly in transforming the distribution and evaluating the probability of each grid cell. Finer grid discretization reduces conservatism, but also increases computation time.

\subsection{Risk Allocation}
\label{sec:Risk Allocation}
%Note that 
Risk allocation is only necessary for M.2.1 and M.2.2 and not required in M.1. Equally distributing the allowable probability of collision $p_{coll}$ across the robots and obstacles, as introduced in \cite{Luders2010_CC-RRT}, is the simplest allocation method. For robot-robot collision this corresponds to $p_c^{ij}=\frac{p_{coll}}{N_O + N_A - 1}$. However, this can be overly conservative and make it difficult to find viable motion plans. Instead, we propose allocating $p_{coll}$ more effectively, such that robots with a higher likelihood of collision are assigned a higher proportion of $p_{coll}$.

We begin by setting $p_{coll}=P_c^A+P_c^O$, where $P_c^A$ 
% is the allowable probability of collision with all the robots, and 
and
$P_c^O$ are the allowable probabilities of collision with all the robots and all the static obstacles, respectively. This naturally fits the high-level low-level division in CC-K-CBS because obstacle collision checking occurs only in the low-level planner.
Define the total volume of all the robots as $V_A$, and the volume of all the obstacles as $V_O$. Then, we set  $P_c^A=\frac{V_A p_{coll}}{V_A+V_O}$ and $P_c^O=\frac{V_Op_{coll}}{V_A+V_O}$. This allocates more of the probability of collision to the category that is more likely to cause collisions. 
We then divide $P_c^A$ based on the distance between robots, with robots that are closer together receiving a larger portion of $P_c^A$. Let $d_k^{j}=\|\bar{x}_k^i-\bar{x}_k^j\|$ be the workspace distance between robot $i$'s and $j$'s means. Then, the allowable probability of collision for robot $i$ with $j$ is $p_c^{ij} = \frac{\alpha}{d_k^j}P_c^A$, where 
% $\alpha =\frac{1}{\sum_j^{N_A-1} d_k^{j}}$ 
$\alpha =1/ (\sum_j^{N_A-1} d_k^{j})$ 
is the normalizing factor.
