%%%%%%%%%%%%%%%%%%%%%%%%%%%%%%%%%%%%%%%%%%%%%%%%%%%%%%%%%%%%%%%%%%%%%%%%%%%%%%%%
%2345678901234567890123456789012345678901234567890123456789012345678901234567890
%        1         2         3         4         5         6         7         8

\documentclass[letterpaper, 10 pt, conference]{ieeeconf}

%\documentclass[a4paper, 10pt, conference]{ieeeconf}      % Use this line for a4 paper

\IEEEoverridecommandlockouts                              % This command is only needed if 
                                                          % you want to use the \thanks command

%\overrideIEEEmargins                                      % Needed to meet printer requirements.

%In case you encounter the following error:
%Error 1010 The PDF file may be corrupt (unable to open PDF file) OR
%Error 1000 An error occurred while parsing a contents stream. Unable to analyze the PDF file.
%This is a known problem with pdfLaTeX conversion filter. The file cannot be opened with acrobat reader
%Please use one of the alternatives below to circumvent this error by uncommenting one or the other
%\pdfobjcompresslevel=0
%\pdfminorversion=4

% See the \addtolength command later in the file to balance the column lengths
% on the last page of the document

% The following packages can be found on http:\\www.ctan.org
%\usepackage{graphics} % for pdf, bitmapped graphics files
%\usepackage{epsfig} % for postscript graphics files
%\usepackage{mathptmx} % assumes new font selection scheme installed
%\usepackage{times} % assumes new font selection scheme installed
%\usepackage{amsmath} % assumes amsmath package installed
%\usepackage{amssymb}  % assumes amsmath package installed


\let\proof\relax
\let\endproof\relax
\usepackage{amsthm}
\usepackage[utf8]{inputenc}
\usepackage{amsmath,amssymb}
\usepackage{booktabs}
\usepackage[font=small]{subcaption}
\usepackage{numprint}
\usepackage{csvsimple}
\usepackage{comment}
\usepackage{amsfonts}
\usepackage{subcaption}
\usepackage{cite}
\usepackage{multirow}
\usepackage{eso-pic}

\newtheorem{theorem}{Theorem} %[section]
\newtheorem{corollary}{Corollary}
\newtheorem{lemma}[theorem]{Lemma}
\newtheorem{definition}{Definition} %[section]
\newtheorem{assumption}{Assumption}
\usepackage[linesnumbered,ruled,noend]{algorithm2e}
\usepackage{algpseudocode}
\usepackage{graphicx}
\usepackage{xcolor}


\newcommand{\ml}[1]{\textcolor{red}{[ML: #1]}}
\newcommand{\at}[1]{\textcolor{purple}{[AT: #1]}}
\newcommand{\nra}[1]{\textcolor{red}{[NRA: #1]}}
\newcommand{\jk}[1]{\textcolor{cyan}{JK: #1}}

\newcommand{\reals}{\mathbb{R}}
\newcommand{\naturals}{\mathbb{R}}
\newcommand{\expect}{\mathbb{E}}
\newcommand{\positivereals}{\mathbb{R}_{>0}}
\newcommand{\W}{\mathcal{W}}
\newcommand{\X}{\mathcal{X}}
\newcommand{\C}{\mathcal{C}}
\newcommand{\B}{\mathcal{B}}
\newcommand{\N}{\mathcal{N}}
\newcommand{\Y}{\mathcal{Y}}
\newcommand{\proj}[1]{\textsc{proj}_{#1}}
\newcommand{\safe}{\text{safe}}
\newcommand{\coll}{\text{coll}}
\newcommand{\expBelief}{\textbf{b}}



% \title{\LARGE \bf
% IMPETUS: Integrated Motion Planning and Event Triggered Estimation which USually works
% }

\title{\LARGE \bf
Chance-Constrained Multi-Robot Motion Planning under \\ Gaussian Uncertainties
}

%\title{\LARGE \bf
%Chance Constrained Motion Planning with Event Triggered Estimation}

\author{Anne Theurkauf$^*$, Justin Kottinger$^*$, Nisar Ahmed, and Morteza Lahijanian% <-this % stops a space
\thanks{This work was supported by NASA STTR award 80NSSC20C0314.}
\thanks{$^*$A. Theurkauf and J. Kottinger have equal contributions in this work.}
\thanks{Authors are with the department of Aerospace Engineering Sciences at the University of Colorado Boulder, CO, USA
        {\tt\small \{\textit{firstname}.\textit{lastname}\}@colorado.edu}}
}


\begin{document}


\AddToShipoutPictureBG*{%
  \AtPageUpperLeft{%
    \hspace{14.5cm}%
    \raisebox{-1.5cm}{%
      \makebox[0pt][r]{Submitted to IEEE Robotics and Automation Letters}}}}


\maketitle
\thispagestyle{plain}
\pagestyle{plain}


%%%%%%%%%%%%%%%%%%%%%%%%%%%%%%%%%%%%%%%%%%%%%%%%%%%%%%%%%%%%%%%%%%%%%%%%%%%%%%%%
% Abstract


Over the past few years, there has been a significant amount of research focused on studying the ReLU activation function, with the aim of achieving neural network convergence through over-parametrization. However, recent developments in the field of Large Language Models (LLMs) have sparked interest in the use of exponential activation functions, specifically in the attention mechanism.

Mathematically, we define the neural function $F: \R^{d \times m} \times  \mathbb{R}^d \rightarrow \mathbb{R}$ using an exponential activation function. Given a set of data points with labels $\{(x_1, y_1), (x_2, y_2), \dots, (x_n, y_n)\} \subset \mathbb{R}^d \times \mathbb{R}$ where $n$ denotes the number of the data. Here $F(W(t),x)$ can be expressed as $F(W(t),x) := \sum_{r=1}^m a_r \exp(\langle w_r, x \rangle)$, where $m$ represents the number of neurons, and $w_r(t)$ are weights at time $t$. It's standard in literature that $a_r$ are the fixed weights and it's never changed during the training. We initialize the weights $W(0) \in \mathbb{R}^{d \times m}$ with random Gaussian distributions, such that $w_r(0) \sim \mathcal{N}(0, I_d)$ and initialize $a_r$ from random sign distribution for each $r \in [m]$.

Using the gradient descent algorithm, we can find a weight $W(T)$ such that $\| F(W(T), X) - y \|_2 \leq \epsilon$ holds with probability $1-\delta$, where $\epsilon \in (0,0.1)$ and $m = \Omega(n^{2+o(1)}\log(n/\delta))$. To optimize the over-parametrization bound $m$, we employ several tight analysis techniques from previous studies [Song and Yang arXiv 2019, Munteanu, Omlor, Song and Woodruff ICML 2022]. 

 



%%%%%%%%%%%%%%%%%%%%%%%%%%%%%%%%%%%%%%%%%%%%%%%%%%%%%%%%%%%%%%%%%%%%%%%%%%%%%%%%
\section{INTRODUCTION}
    \label{sec:intro}
    \input{sections/intro_RAL}

\section{PROBLEM FORMULATION}
    \label{sec:problem}
    \input{sections/problem_RAL}

\section{Chance-Constrained MRMP Algorithms}
    \label{sec:method}
    

We now describe two algorithms for Chance-Constrained MRMP (CC-MRMP). We begin by formalizing the composed belief space for $N_A$ robots with models in \eqref{eq: dynamics model} and~\eqref{eq: measurement model}. We then propose a centralized approach to solving CC-MRMP by employing an existing single-robot belief-space planning algorithm called \emph{Belief-}$\mathcal{A}$ \cite{Ho2022_GBT}.
% which provides asymptotically optimal trajectories for systems described by Eqns.~\ref{eq: dynamics model} and~\ref{eq: measurement model}. 
To accommodate larger multi-robot systems, we introduce a decentralized algorithm called \emph{CC-K-CBS} which adapts the novel K-CBS algorithm to plan for systems with state and measurement uncertainty. We describe the empirical advantages and disadvantages of these methods in Sec.~\ref{sec:eval}.

\subsection{Centralized Chance-Constrained MRMP}
%Utilizing 
Centralized methods for uncertain MRMP require composing the system beliefs into a single \emph{meta-belief}, $\textbf{B}_k$, with a single dynamical constraint. We can then leverage Belief-$\mathcal{A}$ on this meta-belief. Belief-$\mathcal{A}$ provides a framework for adapting any kinodynamic sampling-based planner $\mathcal{A}$ to efficiently plan in the Gaussian belief space under chance constraints as described by \eqref{eq:coll cc} and \eqref{eq: goal cc}. See \cite{Ho2022_GBT} for details. 
%, while providing guarantees on satisfying chance constraints on the probability of terminating in a goal region and the point-wise probability of collision
%\ml{delete the explanation for CC to save space - could refer to eqns 7a and 7b instead if we want to be elaborate.}
%. For brevity we only present an overview here, see \cite{Ho2022_GBT} for details. The algorithm builds a motion tree in Gaussian belief space the same way $\mathcal{A}$ does in the state space, e.g., if $\mathcal{A}$ is RRT, Belief-RRT extends the tree by sampling a random Gaussian belief, determining the closest node to that belief via the 2-Wasserstein distance, then extending that node using a computed nominal control input
%\ml{another place to save space}
%. The Gaussian belief is propagated 
% using the techniques derived in \cite{Bry2011_BeliefProp}. The propagated belief is 
%and used for validity checking on the new node.  For collision checking, it adapts the method in \cite{Luders2010_CC-RRT}, which is based on linear constraint checking for static convex polytopic obstacles. 
% In our implementation we used this framework to adapt the Stable Sparse Rapidly Exploring Trees (SSTs)  algorithm, \cite{Li2016_SST}, to Belief-SST.

For centralized CC-MRMP, because all the robots are assumed to be independent, sampling and propagation of the meta-belief can be performed independently for each robot, and the meta-agent belief reconstructed from the individual robots' \emph{expected beliefs}, $\expBelief(x_k^i)$. These individual beliefs are obtained through \cite{Bry2011_BeliefProp}, which provides a method for predicting the belief over states for a linearizable and controllable system that uses a KF for estimation. This method forecasts the belief with a priori unknown measurements, yielding an expected belief defined as:
\begin{align*}
    \label{eq: expected belief}
    \expBelief(x_k^i) &= \mathbb{E}_Y[b(x_k^i | x_0^i, y_{0:k}^i)] 
    = \int_{y_{0:k}^i} \hspace{-3mm} b(x_k^i | x_0^i, y_{0:k}^i) pr(y_{0:k}^i)dy.
    % &= \mathcal{N}(\check{x}_k^i,\Sigma_k^i+\Lambda_k^i) = \mathcal{N}(\mu_k^i,\Gamma_k^i)
     % = \mathcal{N}(\mu_k^i,\Gamma_k^i),
\end{align*}
We use this expected belief for offline motion planning and evaluation of chance constraints in \eqref{eq:coll cc} and \eqref{eq: goal cc}. For a given initial belief $b(x_0^i)$ and nominal trajectory, $\check{X}^i$, the expected belief is
$\expBelief(x_k^i) = \mathcal{N}(\check{x}_k^i,\Gamma_k^i)$, where $\Gamma_k^i = \Sigma_k^i+\Lambda_k^i$ can be recursively calculated as derived in \cite{Bry2011_BeliefProp}:
\begin{align}
    \Sigma_k^{i-} &= A^i \Sigma^i_{k-1}A^{iT} + Q^i, \quad \Sigma_k^i = \Sigma_k^{i-} - L_k^i C^i\Sigma_k^{i-}, \\
    \Lambda_k^i &= (A^i-B^iK^i)\Lambda_{k-1}^i(A^i-B^iK^i)^T + L_k^i C^i \Sigma_k^{i-},
\end{align}
where $\Sigma_k^i$ is the online KF uncertainty, and $\Lambda_k^i$ is covariance of the expected state estimates $\hat{x}_k^i$. Intuitively, this distribution can be thought of as the sum of the online estimation error and the uncertainty from the a priori unknown measurements made during execution. Given the expected beliefs for the agents, the meta-belief can be constructed such that $\textbf{B}_k=\mathcal{N}(\check{X}_k^{M}, \Gamma_k^M)$, where $\check{X}_k^{M}=(\check{x}_k^{iM}, ...,\check{x}_k^{N_A M})$, and $\Gamma_k^M$ is constructed as a block diagonal with elements $\Gamma_k^i, i=1,...,N_A$. 
Similarly, the validity checkers can be iterated for each robot, such that an entire meta-belief node is considered invalid if a single robot's belief is deemed invalid.

This centralized algorithm is the simplest of our two approaches, however it suffers from poor scalability with the number of agents, just as centralized deterministic planners do. To this end, we introduce a decentralized planner. 
% This formulation allows us to use any single-agent belief-space planner to solve uncertain MRMP. We denote this general algorithm as belief-$\mathcal{A}$, where $\mathcal{A}$ is any deterministic sampling-based tree planner. Originally introduced in \jk{cite}, belief-$\mathcal{A}$ works by sampling beliefs according to the Wasserstein distance metric, propagating the beliefs using Eqn.~\ref{eq: dynamics model} and~\ref{eq: measurement model} according to methods introduced in Bry/Roy \jk{cite}, and using off-the-shelf chance constrained agent-to-obstacle collision checkers to ensure safe solutions. Using belief-$\mathcal{A}$ for Uncertain MRMP adds an additional requirement of implementing an agent-to-agent collision checking procedure.  

\subsection{Decentralized Chance-Constrained K-CBS}
\begin{algorithm}[t]
\KwIn{$\W$, $\{x_0^i\}_{i=1}^{N_A}$, N, B, $p_{\safe}$}
\KwOut{Valid motion plans $\{(\check{U}^i, \check{X}^i)\}_{i=1}^{N_A}$}
$Q, n_0, \leftarrow \emptyset$;
$n_0$.plan $\leftarrow$ \underline{\textsc{getInitPlan}}$()$;$Q$.add($n_0$)\label{ln:rootNode}\;
%$p_{\safe}^A$,$p_{\safe}^O \leftarrow$\underline{\textsc{allocSafety}}$(p_{\safe})$\label{ln:allocate safety}\;
\While{solution not found \label{ln:mainBegin}}
{
    $c \leftarrow$ $Q$.top()\label{ln:selectKCBS}\;
    \eIf{$c$.plan is empty \label{ln:retry}}
    {
        $Q$.pop(); \underline{\textsc{replan}}(N, $c$);
        % $\mathcal{C}_{\max} \leftarrow$
        % \If{shouldMerge($\mathcal{C}_{\max}$, B)\label{ln:mergeIf1}}
        % {
        %     \KwRet{mrg\&rstrt($\W$, $\{x_0^i\}_{i=1}^{N_A}$, N, B,$\mathcal{C}_{max}$)}
        % }
        $Q$.add($c$)\label{ln:retryAdd}\;
    }
    {
        K $\leftarrow$ \underline{\textsc{validatePlan}}($c$.plan, $p_{\safe}$)\label{ln:validate}\;
        \uIf{K is empty\label{ln:isKempty}}
        {
            \KwRet{$c$.plan\label{ln:retPlan}}
        }
        \uElseIf{shouldMerge(K, B)\label{ln:shouldMerge}}
        {
            \KwRet{mrg\&rstrt($\W$, $\{x_0^i\}_{i=1}^{N_A}$, N, B, K)}
        }
        \Else
        {
            $Q$.pop()\label{ln:branchPop}\;
            \For{every robot $i$ in K\label{ln:branchForLoop}}
            {
                $c_{new}.\mathcal{C} \leftarrow$ $c.\mathcal{C} \: \cup$ \underline{\textsc{createCSTR}}(K, $i$)\label{ln:create constraint}\;
                \underline{\textsc{replan}}(N, $c_{new}$); $Q$.add($c_{new}$)\label{ln:branchAdd}\; 
                %$\mathcal{C}_{\max} \leftarrow$ 
            }
        }
    }
    \label{ln:mainEnd}
}
\caption{Chance Constrained K-CBS}
\label{alg:G-KCBS}
\end{algorithm}
\setlength{\textfloatsep}{10pt}

Our decentralized CC-MRMP algorithm is based on Kinodynamic Conflict-Based Search (K-CBS)~\cite{Kotting2022_KCBS}. K-CBS consists of a high-level conflict-tree search and a low-level motion planner. At the high-level, K-CBS tracks a constraint-tree (CT), in which each node represents a suggested plan, which might contain conflicts (i.e., collisions). At each iteration, K-CBS picks an unexplored node from the CT based on some heuristic and identifies the conflicts in the node’s corresponding plan. Then, K-CBS tries to resolve the conflicts by creating child nodes as follows: if robots $i$ and $j$ are in collision for some time duration $[t_s,t_e]$ (denoted $\langle i,j,[t_s,t_e]\rangle$), then two child nodes are created, one with the constraint that $\mathcal{B}^i$ cannot intersect with $\mathcal{B}^j$ for all $t\in[t_s,t_e]$ and the other dually for robot $j$.
Then, in each child node, a low-level sampling-based planner attempts to replan a trajectory that satisfies the set of all constraints in that branch of the CT within $N>0$ iterations. This process repeats until a non-colliding plan is found. In the case where greater than $B>0$ conflicts arise, K-CBS includes a merge and restart procedure that recursively combines a subset of the robots. We refer the reader to~\cite{Kotting2022_KCBS} for details on the original K-CBS.
We now describe CC-K-CBS (Alg.~\ref{alg:G-KCBS}), which requires several adaptations of K-CBS (underlined in Alg.~\ref{alg:G-KCBS}).
% The pseudocode for (CC)-K-CBS is shown in Alg.~\ref{alg:G-KCBS}. We refer the reader to~\cite{Kotting2022_KCBS} for details on the original K-CBS. 

Firstly, note that in K-CBS, collision checking between robots and static obstacles are performed separately. The low-level planner keeps track of workspace static obstacles, and the CT node accounts for robot-robot collisions at the high-level.  This separation is key for scalability, but it introduces a major challenge in chance-constrained planning for uncertain robots.  We expand on this further in Sec.~\ref{sec:collisionChecking}.
% We extended K-CBS Chance-Constrained K-CBS (CC-K-CBS) for CC-MRMP with three adaptations. 
% The procedures that require changes shown in red, blue, and violate fonts in Alg.~\ref{alg:G-KCBS}.
Secondly, since states are now beliefs, CC-K-CBS requires a Gaussian belief low-level motion planner.  We propose to use Belief-$\mathcal{A}$ for the \textsc{replan} and \textsc{getInitPlan} procedures. Thirdly, CC-K-CBS requires a definition of constraints; we introduce \emph{belief-constraints}. This simply replaces the deterministic shape with a non-deterministic belief. Thus, given a conflict $K=\langle i,j,[t_s,t_e]\rangle$, one constraint forces robot $i$ to satisfy the chance-constraint with $b(x_k^j)$ for all $t\in[t_s,t_e]$ and dually for robot $j$. This enables \textsc{createCSTR} 
(Line~\ref{ln:create constraint}) 
to define meaningful constraints during replanning with Belief-$\mathcal{A}$. 

Lastly, \textsc{validatePlan} (Line~\ref{ln:validate}) requires an efficient way of validating the chance constraints (Equations \ref{eq:coll cc} and \ref{eq: goal cc}) while simulating prospective MRMP plans. We propose several techniques for this computation, each of which is described in detail in Sec.~\ref{sec:collisionChecking}. Every method requires comparison with the maximum allowable probability of collision $p_{coll} = 1-p_{\safe}$. 
We say \textsc{validatePlan} procedure is \emph{complete} if it can correctly evaluate the constraints \eqref{eq:coll cc} and \eqref{eq: goal cc} for all Gaussian beliefs.

The following theorem provides the sufficient conditions for CC-K-CBS to be probabilistically complete. 
% Because deterministic K-CBS has already been proven to be probabilistically complete in \cite{Kotting2022_KCBS}, we focus on the aspects of the algorithm which have been modified for uncertainty.


\begin{theorem}
    \label{thm:completeness}
    CC-K-CBS is probabilistically complete if the underlying planner Belief-$\mathcal{A}$ is  probabilistically complete and the \textsc{validatePlan} procedure is complete.
\end{theorem}
\begin{proof}
The proof builds on Theorem 1 in \cite{Kotting2022_KCBS}, which shows that 
% shows that
% According to \cite{Kotting2022_KCBS},
K-CBS inherits the probabilistic completeness property of the low-level planner. Thus, probabilistic completeness of Belief-$\mathcal{A}$ is required for CC-K-CBS. The completeness of \textsc{validatePlan} ensures that CC-K-CBS does not reject valid plans. It follows that \textsc{createCSTR} does not pass spurious constraints to the low-level planner. Therefore, under these conditions, the probabilistic completeness proof for K-CBS extends to CC-K-CBS.
\end{proof}

The completeness condition on \textsc{validatePlan} 
% requires exact evaluation of the chance constraints in \eqref{eq:coll cc} and \eqref{eq: goal cc}, which is possible 
can be satisfied
by exact evaluation of the probabilities in \eqref{eq:prob goal}-\eqref{eq:prob robot-robot collision}. However, this operation can be computationally expensive. To achieve tractability and scalability, we introduce conservatism into \textsc{validatePlan} by over-approximating the probability of collision.
% in \eqref{eq:prob collision} and \eqref{eq:prob robot-robot collision} and under-approximating the goal probability in \eqref{eq:prob goal}. 
% The use of an over-approximation guarantees satisfaction of the chance constraint, however can result in valid plans being incorrectly pruned.
With this \textsc{validatePlan}, we can still guarantee soundness of the CC-K-CBS solutions, but completeness is affected since valid plans may be rejected.


\begin{corollary}
If \textsc{validatePlan} uses an over-approximation of the probability of collision, then CC-K-CBS is probabilistically complete up to the accuracy of the over-approximation.
\end{corollary}

Below, we introduce several methods for over-approximating the probability of collision. 
% and evaluating the chance constraints.

\begin{comment}
    
\textcolor{blue}{Given an underlying sampling-based tree-search planner, $\mathcal{A}$, we prove the proposed CC-K-CBS algorithm inherits the same completeness properties as $\mathcal{A}$.
\ml{hmm... need to elaborate more here... I think the assumption is that \textsc{validatePlan} is complete w.r.t. $p_\safe$, no?}
}
\ml{another point regarding the theorem/proof, risk allocation \textsc{allocSafety} is only required for Method 1, not Method 2, right?}
\begin{theorem}[]
    \label{thm:completeness}
    \textcolor{blue}{If planner $\mathcal{A}$ is probabilistically complete, then so is CC-K-CBS.}
\end{theorem}
\begin{proof}
\textcolor{blue}{The proof of follows directly from the completeness proofs of the constituent algorithms, K-CBS and Belief-$\mathcal{A}$, which are detailed in \cite{Kotting2022_KCBS} and \cite{Ho2022_GBT} respectively.}

\textcolor{blue}{Note that the introduction of probabilistic chance constraints does not violate the completeness proof of the K-CBS algorithm from \cite{Kotting2022_KCBS}. We can simply substitute the robot-robot chance constraint condition into the deterministic collision avoidance criterion of K-CBS. This does not change the completeness proof, thus preserving the probabilistic completeness property of K-CBS for returning nominal trajectories that respect the chance constraint.}
\ml{This is not a proof... it's an explanation/elaboration.}

\textcolor{blue}{To complete the proof: if $\mathcal{A}$ is probabilistically complete, then so is Belief-$\mathcal{A}$~\cite{Ho2022_GBT}, which in turn implies K-CBS~\cite{Kotting2022_KCBS} is probabilistically complete. Therefore, if $\mathcal{A}$ is probabilistically complete, then so is CC-K-CBS.}
\ml{the last sentence is a repetitive}
\end{proof}

\ml{reword the following paragraph and make it concise given the comments above the theorem.}
\textcolor{blue}{The completeness of CC-K-CBS is complicated by the difficulty of probabilistic collision checking, where the exact probability of collision is almost never available for the collision checking step. To this end we introduce approximations that over-bound the probability of collision. We can thus guarantee that any returned solution satisfies the chance constraint. However, the over approximation means we cannot guarantee that if a valid solution exists it will be found, even as the computation time approaches infinity. Thus the proposed approximate collision checking algorithm is not probabilistically complete with respect to the original chance constraint. However, the algorithm is probabilisitically complete \emph{with respect to the approximations}, meaning if a solution exists under the approximations, the algorithm will return that solution with probability 1 as the computation time approaches infinity.}
\end{comment}


%These three new requirements present themselves inside the red, violet, and blue procedures of Alg.~\ref{alg:G-KCBS}, respectively. 


% \subsection{Belief Propagation}
% \label{sec:beliefProp}
% Work \cite{Bry2011_BeliefProp} provides a method for predicting the belief over states for a linearizable and controllable system that uses a KF for estimation. This method forecasts the belief with a priori unknown measurements, yielding an \emph{expected belief}, $\expBelief(x_k^i)$, defined as:
% \begin{align*}
%     \label{eq: expected belief}
%     \expBelief(x_k^i) &= \mathbb{E}_Y[b(x_k^i \mid x_0^i, y_{0:k}^i)] 
%     = \int_{y_{0:k}^i} \hspace{-3mm} b(x_k^i \mid x_0^i, y_{0:k}^i) pr(y_{0:k}^i)dy.
%     % &= \mathcal{N}(\check{x}_k^i,\Sigma_k^i+\Lambda_k^i) = \mathcal{N}(\mu_k^i,\Gamma_k^i)
%      % = \mathcal{N}(\mu_k^i,\Gamma_k^i),
% \end{align*}
% We use this expected belief for offline motion planning and evaluation of chance constraints \eqref{eq:coll cc} and \eqref{eq: goal cc}. For a given initial belief $b(x_0^i)$ and nominal trajectory, $\check{X}^i$, the expected belief
% $\expBelief(x_k^i) = \mathcal{N}(\check{x}_k^i,\Gamma_k^i)$, where $\Gamma_k^i = \Sigma_k^i+\Lambda_k^i$ can be recursively calculated as derived in \cite{Bry2011_BeliefProp}:
% \begin{align}
%     \Sigma_k^{i-} &= A\Sigma_{k-1}A^{iT} + Q^i, \quad \Sigma_k^i = \Sigma_k^{i-} - L_k^i C^i\Sigma_k^{i-}, \\
%     \Lambda_k^i &= (A^i-B^iK^i)\Lambda_{k-1}^i(A^i-B^iK^i)^T + L_k^i C^i \Sigma_k^{i-},
% \end{align}
% where $\Sigma_k^i$ is the online KF uncertainty, and $\Lambda_k^i$ is covariance of the expected state estimates $\hat{x}_k^i$. Intuitively, this distribution can be thought of as the sum of the online estimation error and the uncertainty from the a priori unknown measurements made during execution.

    
\section{Robot-Robot Collision Checking Algorithms}
    \label{sec:collisionChecking}
    





Recall that the probability of collision between robots $i$ and $j$, as defined in  \eqref{eq:prob robot-robot collision}, requires integration of the joint probability distribution of the two robots over set $\X^{ij}_\coll$. 
This computation is difficult because (i) $\X^{ij}_\coll$ is hard to construct, and (ii) integration of the joint probability distribution function is expensive.  However, the constraint-checking procedure must be extremely fast in sampling-based algorithms because it is called in every iteration of tree extension. In this section, we introduce three approximation methods for this integration that trade off accuracy with computation effort.  
%The methods rely on bounding the projection of $\X^{ij}_{\coll}$ into the lower dimensional workspace, where checking the chance constraint is simpler. 

\begin{figure}[b]
    \centering
    \begin{subfigure}{0.26\textwidth}
    \includegraphics[width=\textwidth]{figures/minkVol.pdf}
    \vspace{-7mm}
    \caption{}
    \label{fig:CspaceIntegration}
    \end{subfigure}
    \centering
    \begin{subfigure}{0.2\textwidth}
    \includegraphics[width=\textwidth]{figures/minkVolSlice.pdf}
    \vspace{-7mm}
    \caption{}
    \label{fig:CspaceIntegration_slice}
    \end{subfigure}
\caption{(a) 3D view of the c-space obstacle $\X_{coll}^{ij}$ 
% approximately constructed as the conjunction of a series of Minkowski sums of uncertain 
for
robot $i$ with fixed robot $j$ calculated at discrete choices of $\theta_k^i$.  
% (b) Example Minkowski sum for a single choice of $\theta_k^i$, depicted by the intersection of a gray plane in the top figure. 
(b)~Top-down view of $\X_{coll}^{ij}$.
The projection of $\X_{coll}^{ij}$ onto x-y plane is bounded by a green circle.
}
% \vspace{-3.5mm}
\end{figure}

For clarity of explanation, consider two deterministic (rotating and translating) robots $i$ and $j$ in a 2D workspace, with configurations $(\text{x}_k^l,\text{y}_k^l,\theta_k^l)$, $l\in \{i,j\}$. We can construct the 4D c-space obstacles for this pair by taking the Minkowski difference of the two agents for every pair of orientation angles (see the c-space construction in Fig.~\ref{fig:CspaceIntegration}, which shows the c-space in 3D for a given orientation of agent $j$). This 4D obstacle is generally difficult to construct, so deterministic collision checkers typically resort to collision checking in the workspace, e.g., by bounding the projection of this set, shown by a green disk in Fig. \ref{fig:CspaceIntegration_slice}. 



Now, consider that each robot is described by a random variable (RV), so the position and orientation of their bodies are no longer deterministic but given by beliefs, such that each point in the 4D c-space maps to a probability. The exact probability of collision is the integral of the probability mass over the c-space obstacle. 
% Hence, the uncertain collision checking problem has two main challenges: construction of a complex c-space obstacle, and integration over this complex shape. 
This results in the two challenges listed above: c-space obstacle construction and integral computation.
Each method 
% listed in the following sections address these two core problems.
proposed below attempts to address these core problems. 
% Approximations and simplifications made to increase efficiency also introduce conservatism, which leads the collision checker to reject valid beliefs. Each of the methods described strikes a difference balance between these two competing factors.

%Then, the c-spaces are 3D and consist of a 2D Euclidean coordinate (position) and an orientation angle, i.e., $x_k^l=(\text{x}_k^l,\text{y}_k^l,\theta_k^l)$, $l\in \{i,j\}$. 
% These robots operate in a 2D workspace $\W$, so their projections $\proj{W}(x_k^l), l={i,j}$ are also 2D.
%Furthermore, assume only robot $i$ has uncertainty, and the state of robot $j$ is fixed and known. This reduces robot-robot collision checking to checking an uncertain robot with a known obstacle. In this case, the c-space is 3-dimensional, and the integration area for collision can be constructed by determining the Minkowski sum of the uncertain robot with the fixed robot for each possible orientation $\theta_k^i$. This is depicted by a pink volume in Figure \ref{fig:CspaceIntegration}. A simple way to bound the projection on $[\text{x}_k^i, \text{y}_k^i]$ space is to bound both robots' bodies with a disk that accounts for all possible orientations. The projection is then bounded by a circle with radius equal to the sum of the radii of the two bounding disks, shown as a green circle in Figure \ref{fig:CspaceIntegration}. The concept of bounding the projection in 2D euclidean space intuitively extends to higher dimensional state spaces. 

%We present three methods for checking the robot-robot collision constraint. Each method is computationally cheap, but has varying levels of conservatism. The sources of conservatism arise from two main sources: bounds on the orientation of the robot body, and approximations introduced to ease evaluation of complex integrals. The results section presents an empirical study of conservatism and computation time for various systems and environments. We present the methods in order of increasing computation time, and decreasing conservatism.
\begin{figure}
     \centering
     \begin{subfigure}[b]{0.45\columnwidth}
         \centering
         \includegraphics[width=\textwidth]{figures/discBody.pdf}
         \vspace{-5mm}
         \caption{}
         \label{fig:diskBoundBody}
     \end{subfigure}
     \hfill
     \begin{subfigure}[b]{0.45\columnwidth}
         \centering
         \includegraphics[width=\textwidth]{figures/individualAgentsProb.pdf}
         \vspace{-5mm}
         \caption{}
         \label{fig:AgentCircPsafe}
     \end{subfigure}
     \newline
     \begin{subfigure}[b]{0.45\columnwidth}
         \centering
         \includegraphics[width=\textwidth]{figures/DiskIntersect.pdf}
         \vspace{-5mm}
         \caption{}
         \label{fig:diskIntersect}
     \end{subfigure}
     \hfill
     \begin{subfigure}[b]{0.45\columnwidth}
         \centering
         \includegraphics[width=\textwidth]{figures/diffProbCirc.pdf}
         \vspace{-5mm}
         \caption{}
         \label{fig:CircularBodyCicularIntegration}
     \end{subfigure}
     \newline
     \begin{subfigure}[b]{0.45\columnwidth}
         \centering
         \includegraphics[width=\textwidth]{figures/diffProbCircPoly.pdf}
         \vspace{-5mm}
         \caption{}
         \label{fig:CircularBodyPolygonIntegration}
     \end{subfigure}
     \hfill
     \begin{subfigure}[b]{0.45\columnwidth}
         \centering
         \includegraphics[width=\textwidth]{figures/diffProbCircPoly_grid.pdf}
         \vspace{-5mm}
         \caption{}
         \label{fig:CircularBodyGridIntegration}
     \end{subfigure}
     \caption{
(a) Circle bound on robots' bodies. (b) $p_\safe$ contour and corresponding circular over bound on $p_\safe$. (c) Inflated circular bounds intersect, indicating possible collision between robots. (d) Integration area for probability of collision with circular disk bound on robots' bodies. (e) Polygonal bound on circular integration area. (f) Transformed distribution and integration area, with over-approximated grid.}
% \vspace{-1mm}
\end{figure}

% \begin{figure}
%     \centering
%     \begin{subfigure}{0.3\textwidth}
%     \includegraphics[width=\textwidth]{figures/diffProbCirc.pdf}
%     \vspace{-7mm}
%     \caption{}
%     \label{fig:CircularBodyCicularIntegration}
%     \end{subfigure}
%     \\
%     \centering
%     \begin{subfigure}{0.3\textwidth}
%     \includegraphics[width=\textwidth]{figures/diffProbCircPoly.pdf}
%     \vspace{-7mm}
%     \caption{}
%     \label{fig:CircularBodyPolygonIntegration}
%     \end{subfigure}
% \caption{(a) Integration area for probability of collision with circular disk bound on robots' bodies. (b) Polygonal bound on circular integration area}
% \vspace{-3.5mm}
% \end{figure}



% \begin{figure}
%     \centering
%     \includegraphics[width=0.3\textwidth]{figures/diffProbCircPoly_grid.pdf}
%     \vspace{-7mm}
%     \caption{}
%     \label{fig:CircularBodyGridIntegration}
% \caption{Transformed distribution and integration area, with over-approximated grid.}
% \vspace{-3.5mm}
% \end{figure}







\begin{comment}
%%% Old table, kept for reference for previous benchmarks    
\begin{table*}[]
    \centering
    \begin{tabular}{c|c|c|c|c}
         Method &  Body Bound & Integration Area & Bound on Integration Area & CC Evaluation\\
         \hline
         1 & Sphere & N/A & N/A & Intersection \\
         2 & Polyhedron & Polyhedron & N/A & Linear Constraint \\
         3 & Sphere & Sphere & Polyhedron & Linear Constraint \\
         4 & Sphere & Sphere & N/A & Grid CDF       
    \end{tabular}
    \caption{Summary table of collision checking methods.}
    \label{tab:ccMethodsSummary}
\end{table*}
\end{comment}

\subsection{Method 1 (M.1): Safety Contour}
\label{sec:Method1}
M.1 is inspired by deterministic collision checking methods, where collisions are detected by intersections of robot bodies in the workspace. We adapt this framework to uncertain collision checking by inflating bounds on the robot bodies to encapsulate $p_\safe$ probability mass.

%We rely on calculating a $w$-sphere for each robot in the workspace that is guaranteed to contain at least $p_{\safe}$ probability mass, we then check for collisions by ensuring these spheres do not intersect. A simplified 2D example is shown in Figures \ref{fig:diskBoundBody}, \ref{fig:AgentCircPsafe}, and \ref{fig:diskIntersect}.

Recall that the expected belief is $\expBelief(x_k^i)= \N(\check{x}_k^i,\Gamma_k^i)$. We define the Gaussian marginal of this distribution in the workspace as $\expBelief_\W(x_k^i)$.
% such that 
% \begin{equation}
%     \expBelief_\W(x_k^i) = \int_{\phi_k^i} \expBelief_{x_k^i\mid _k^i}(x_k^i\mid \phi_k^i)\expBelief_{\phi_k^i}(\phi_k^i)d\phi_k^i
% \end{equation}
Let $\bar{x}^i_k \in \W$ be the position components of $x_k^i$, and $\bar{\Gamma} \in \reals^{w\times w}$ be the corresponding covariance matrix (sub-matrix of $\Gamma^i_k$). Then $\expBelief_\W(x_k^i) = \N(\bar{x}^i_k, \bar{\Gamma}^i_k)$.
% \ml{hm.. why? Why don't we look at the belief of the projection of $x^i_k$ onto the workspace?  Let $\bar{x}^i_k \in \W$ be the position components of $x_k^i$, and $\bar{\Gamma} \in \reals^{w\times w}$ be the corresponding covariance matrix (sub-matrix of $\Gamma^i_k$). Then $\expBelief_\W(x_k^i) = \N(\bar{x}^i_k, \bar{\Gamma}^i_k)$, no? }
% where $\phi_k^i$ are all the state space variables not contained in the workspace. \at{Not so sure about this, is there a better way to write this?} 
The elliptical probability contour containing $p_{\safe}$ probability mass in the workspace marginal distribution is the region $\mathcal{E}_{\safe}^i$ such that
% \begin{equation*}
    $\int_{\mathcal{E}_{\safe}^i} \expBelief_\W(x_k^i) dx=p_{\safe}.$
% \end{equation*}
The axes of this ellipsoid are given by $a_l^i=t_{\chi}\lambda_l, \quad l=1,\ldots,w,$ 
% $a^i_1,\ldots,a^i_w$ are defined by the eigenvalues of covariance $\bar{\Gamma}_k^i$
% , $eigenval(\Gamma_k^i)=(\lambda_1,...,\lambda_n)$, 
% such that 
% \begin{equation*}
%     a_l^i=t_{\chi}\lambda_l, \quad l=1,...,w,
% \end{equation*}
where $\lambda_l$ is an eigenvalue of $\bar{\Gamma}_k^i$, and
$t_{\chi}$ is the inverse $\chi^2$ cumulative density function
evaluated at $p_{\safe}$ with $w$ degrees of freedom.  

Determining intersection of ellipses is difficult, so we bound the ellipsoid with a sphere, where checking for intersection only requires comparing the sum of the radii to the distance between the centers. The sphere containing this ellipsoid, $\mathcal{C}_{\safe}^i$, is defined by the radius $r^i=t_{\chi}\lambda_{max}$, where 
% $\lambda_{max}=max(eigenval(\bar{\Gamma}_k^i))$. 
$\lambda_{\max}=\max\{ \lambda_l \}_{l=1}^w.$
Because $\mathcal{E}_{\safe}^i\subseteq \mathcal{C}_{\safe}^i$, it is assured that the probability mass contained by this sphere is greater than or equal to $p_{\safe}$, i.e., 
% \begin{equation}
    $\int_{\mathcal{C}_{\safe}^i} \expBelief_\W(x_k^i) dx\geq p_{\safe}.$
% \end{equation}
An illustration of both the elliptical contour and its associated spherical bound are shown in Fig.~\ref{fig:AgentCircPsafe} for the robots in Fig~\ref{fig:diskBoundBody}.

Note that this sphere contains only the probability mass associated with $x_k^i$, the origin point on the body frame. 
% \textcolor{blue}{Even if the origin point may satisfy the chance constraint, this does not imply the same for the entire body of the robot.}
It does not account for the collision probability of the other points on the robot body.
%\ml{unclear... reword this sentence}
To address this, we inflate the sphere by a bounding sphere on the robot's body (shown in Fig.~\ref{fig:diskBoundBody}), such that the probability of \emph{any} point on the robot's body being within this inflated sphere is greater than or equal to $p_{\safe}$. The radius, $R_i$, of the bounding sphere on the robot's body is obtained by finding the largest possible Euclidean distance between any pair of points within a robot's body: $R_i = \frac{1}{2}\max_{x_1,x_2\in \B^i} \| x_1-x_2 \|_2$. The inflated bound, $\mathcal{C}_{bound}^i$ is a sphere of radius $r^i+R^i$. The probability that any point of the robot is outside this sphere, the complement of $\mathcal{C}_{bound}^i$, is guaranteed to be less than $1-p_{\safe}$. Thus, if any two robot's bounding spheres
do not intersect, i.e. $\mathcal{C}_{bound}^i \cap \mathcal{C}_{bound}^j = \emptyset$, the probability of collision between them is guaranteed to be less than $1-p_{\safe}$.
% \begin{multline}
%     \mathcal{C}_{bound}^i \cap \mathcal{C}_{bound}^j = \emptyset  \quad \Rightarrow\\
%      P(\proj{\W}(x_k^i)) \cap \proj{\W}(x_k^j))\leq 1-p_{\safe}.
% \end{multline}
The inflated disks and corresponding intersection checking are shown in Fig.~\ref{fig:diskIntersect}.

This method has two main sources of conservatism. The first stems from the bounding sphere on the robot body. This is especially egregious for robots with high aspect ratios. The second source is associated with checking for intersections of the probability mass bound. This conservatively assumes all probability mass not within the sphere is in collision, so any intersection (no matter how small) is considered a possible constraint violation. Violation for this method does not imply that the actual probability of collision exceeds the constraint. Despite these limitations, this method is extremely fast. %collision checking method.
% of all those proposed here.

\subsection{Method 2: Linear Gaussian Transformation}
\label{sec:Method2}
This section introduces a class of methods that reduce conservatism by simplifying the integration itself, namely by reducing its dimensionality. This is done by introducing a new RV representing the difference between two robots' states: $x_k^{ij}=x_k^i-x_k^j$. This RV is Gaussian distributed such that $x_k^{ij}\sim\N(\check{x}_k^i-\check{x}_k^j, \Gamma_k^i+\Gamma_k^j)$, with the workspace marginal $\expBelief_\W(x_k^{ij})$. This new RV represents a vector connecting the origin points of two robots. Collisions correspond to specific realizations of $x_k^{ij}$, %this difference vector, 
defining the required integration  area. 

\begin{comment}
\begin{figure}
    \centering
    \begin{subfigure}{0.3\textwidth}
    \includegraphics[width=\textwidth]{figures/polyBodyBound.pdf}
    \vspace{-7mm}
    \caption{}
    \label{fig:polyBodyBound}
    \end{subfigure}
    \centering
    \begin{subfigure}{0.3\textwidth}
    \includegraphics[width=\textwidth]{figures/diffProbMinkPoly.pdf}
    \vspace{-7mm}
    \caption{}
    \label{fig:PolyBodyMinkowskiInt}
    \end{subfigure}
\caption{(a) Integration area for probability of collision with circular disk bound on robots' bodies. (b) Polygonal bound on circular integration area}
\vspace{-3.5mm}
\end{figure}


\subsubsection{ditching this method}
This method relies on a polyhedral bounds on the bodies of the robots in the workspace, these bounds are then used to construct Minkowski sums which define the integration area. Because this is still a complex integration area, we simplify the constraint checking by using the approximate linear constraints first derived for CC-RRT in \cite{Luders2010_CC-RRT}.

For each robot we define a polyhedra in the workspace, $\mathcal{P}_{poly}^i$,  that bounds the robot's body for all possible orientations. This can be constructed by defining a polyhedral bound over the sphere $S_{body}^i$ described in Section \ref{sec:Method1}, shown in Figure \ref{fig:polyBodyBound}. \at{Need more explanation here?}. We construct the Minkowski Sum for all possible pairs of robots, $\mathcal{P}_{Mink}^{ij}$. 

We then introduce a new random variable, $x_k^{ij}=x_k^i-x_k^j$ representing the difference between two robots' states, this random variable is Gaussian distributed such that $x_k^{ij}\sim\N(\check{x}_k^i-\check{x}_k^j, \Gamma_k^i+\Gamma_k^j)$, with the workspace marginal $\expBelief_\W(x_k^{ij})$. By integrating $\expBelief_\W(x_k^{ij})$ over the Minkowski sum, we can exactly find the probability of the two bounding polyhedra intersecting (indicating possible collision), shown in Figure \ref{fig:PolyBodyMinkowskiInt}. To check the chance constraint, we simply need to ensure this probability does not exceed $1-p_{\safe}$. However, this integral is difficult to calculate, so further approximation is required. We instead make the following approximations to simplify checking the chance constraint. 

The Minkowski sum can be given by the conjunction of a set of $N_p$ half planes: $\mathcal{P}_{Mink}^{ij} = \{ x \mid \mathrm{a}_{h,i}^Tx<b_{h,i}, \forall h\in[1,N_p]\}$, where $a_{h,i}\in\mathbb{R}^{1\times w}$ and $b_{h,i}\in\mathbb{R}$. This reduces the robot-robot chance constraint checking problem to the same problem addressed in \cite{Luders2010_CC-RRT}. In that work, various approximations were used to reduce the probabilistic chance constraint checking problem to checking a set of deterministic linear constraints. One of the key approximations was the allocation of the acceptable probability of collision, or risk allocation: the total acceptable probability of collision ($\Delta = 1-p_{\safe}$) must be distributed across all the possible sources of collision. In our case, we must distribute the probability of collision for an robot among the $N_A-1$ other robots. To keep things general, we we will denote the allowable probability for robot $i$ colliding with any other robot $j$ as $p_c^{ij}$. Two proposed methods of determining $p_c^{ij}$ are described in Section \ref{sec:Risk Allocation}. Under this generalization, we have adapted the linear constraints from \cite{Luders2010_CC-RRT} for the random variable $x_k^{ij}$ as:
\begin{multline}
    \bigvee_{h=1}^{N_p} a_{h,i}^T (\check{x}_k^i-\check{x}_k^j) \geq b_{h,i} + \bar{b}_{h,i}  \quad \Rightarrow  \\
    P(\proj{\W}(X_k^i)) \cap \proj{\W}(X_k^j))\leq 1-p_{\safe}
\end{multline}
where
\begin{align}    
    &\bar{b}_{h,i} = \sqrt{2}P_v erf^{-1}\left( 1-2 p_c^{ij} \right), \\
    &P_v = \sqrt{a_{h,i}^T(\Gamma_k^i+\Gamma_k^j)a_{h,i}}
\end{align}

There are two main sources of conservatism here. The first stems from the same source as the prior method: the polyhedral bound on the body is even more conservative than the spherical bound. This results in violation of the constraint in cases where the true orientations of the body do not actually intersect. The second source of conservatism arises from the use of deterministic linear constraints for the integration itself. This source of conservatism scales with the number of robots and the number of half planes in the Minkowski sum. This method is only slightly more computationally intensive than the first method. While checking linear constraints is very computationally simple, this method must at worst check every constraint for associated with each pair of robots. This yields slightly longer computation times than the first method, which required checking a single condition for every robot pair.
\end{comment}



\subsubsection{Method 2.1 (M.2.1): Convex Polytopic Bounding}
\label{sec:Method2.1}
%Note that 
The difference RV somewhat simplifies the high dimensional c-space obstacle representation, shown in Fig.~\ref{fig:CspaceIntegration}, by merging the workspace dimensions for the two robots. But the orientation of the robots must still be considered. To obtain a tractable integration area, we use the same bounding spheres described in Sec.~\ref{sec:Method1}. Intersection of the bounding spheres is considered a collision state, corresponding to a spherical integration area $S_{body}^{ij}$, with radius $R^i + R^j$, over difference distribution $\expBelief_\W(x_k^{ij})$ (shown in Fig.~\ref{fig:CircularBodyCicularIntegration}). 

Because exact integration of a Gaussian distribution over a sphere is difficult, we bound $S_{body}^{ij}$ with a polytope, $\mathcal{P}_{body}^{ij}$, which can be defined by a set of $N_s$ half planes, $\mathcal{P}_{body}^{ij} = \{ x \mid c_{h,i}^Tx<d_{h,i}, \forall h\in[1,N_s]\}$, where $c_{h,i}\in\mathbb{R}^{n\times 1}$ and $d_{h,i}\in\mathbb{R}$. This reduces the robot-robot chance constraint checking problem to the same problem addressed in \cite{Luders2010_CC-RRT},  where various (over-)approximations are used to reduce the probabilistic chance constraint checking to deterministic linear constraints checking. We employ the same methods to check the chance constraints, with the distribution given by $\N(\check{x}_k^i-\check{x}_k^j, \Gamma_k^i+\Gamma_k^j)$, and polytope given by $\mathcal{P}_{body}^{ij}$.

A key approximation in this method is the allocation of the acceptable probability of collision, or risk allocation: the total acceptable probability of collision $p_{coll} = 1-p_{\safe}$ must be distributed across all the possible sources of collision. In our case, we must distribute the probability of collision for a robot among the $N_A-1$ other robots. We denote the allowable probability for robot $i$ colliding with any other robot $j$ as $p_c^{ij}$. Two proposed methods of determining $p_c^{ij}$ are described in Sec.~\ref{sec:Risk Allocation}. 
% Under this generalization, we have adapted the linear constraints from \cite{Luders2010_CC-RRT} for the random variable $x_k^{ij}$ as:
% \begin{multline}
%     \text{if} \quad \bigvee_{h=1}^{N_c} c_{h,i}^T (\check{x}_k^i-\check{x}_k^j) \geq d_{h,i} + \bar{d}_{h,i} \\
%     \rightarrow P(\proj{\W}(X_k^i)) \cap \proj{\W}(X_k^j))\leq 1-p_{\safe}
% \end{multline}
% where 
% \begin{multline}    
%     \bar{d}_{h,i} = \sqrt{2}P_v erf^{-1}\left( 1-2 p_c^{ij} \right), \\
%     P_v = \sqrt{c_{h,i}^T(\Gamma_k^i+\Gamma_k^j)c_{h,i}}
% \end{multline}

This method has two main sources of conservatism. The first occurs with the spherical bound on the body, the same bound used in M.1 (Sec.~\ref{sec:Method1}) while the second arises from the use of deterministic linear constraints for the integration. This source of conservatism scales with the number of robots and the number of half planes in the polytope bound. This method is only slightly more computationally expensive than the first method: while checking linear constraints is very computationally simple, this method must at worst check every constraint associated with each pair of robots. This yields slightly longer computation times than the M.1, which requires checking a single condition for every robot pair. 


\subsubsection{Method 2.2 (M.2.2): Grid Integration}
This method follows directly from M.2.1, but with a direct integration technique, rather than a check on deterministic linear constraints. This makes this method more accurate and less conservative, but also more computationally intensive. 

We begin with the same difference distribution over $\expBelief_\W(x_k^{ij})$ 
and integration over $\mathcal{P}_{body}^{ij}$. We then perform a whitening (Mahalanobis) transformation on the distribution and $\mathcal{P}_{body}^{ij}$. This transformation decorrelates a Gaussian distribution, yielding the standard normal distribution, as well as rotating and translating $\mathcal{P}_{body}^{ij}$. We define this new integration area as $\mathcal{P}_{white}^{ij}$ (see Fig.~\ref{fig:CircularBodyGridIntegration}). 
Because we have transformed the probability distribution to a standard normal, the cumulative distribution function can be easily computed via the error function (erf); we do this over a grid. This technique is agnostic to the choice of grid cells and only requires rectangular cells that completely cover the polytope $\mathcal{P}_{white}^{ij}$, as shown in green in Fig. \ref{fig:CircularBodyGridIntegration}. The probability of collision is the sum of the probability mass contained in each grid cell, defined as $p_{poly}$. The chance constraint can then be simply checked as $p_{poly}\leq p_c^{ij}$,
where $p_c^{ij}$ again is the risk allocated to a pair of robots.

%we will define a grid discretization over the entire workspace, with discretization step size for each workspace axis as $d\W_w$.  Each grid cell, $\W_g$, is a defined as set of vertices, $v_\W\in\W$, such that the $w$-th element of $v_\W$, $v_{\W}(w)$ is an element of the adjacent set of grid points in axis $w$:
%\begin{equation}
%    v_{\W}(w) \in \{w_Nd\W_w, (w_N+1)d\W_w\}, \quad w_N\in\mathbb{Z}
%\end{equation}
%The set of cells that over-approximate the integration area is therefore: $\W_{g,poly}=(\W_g \mid \exists v_\W(w) \in \mathcal{P}_{whit}^{ij})$. This grid over approximation of the integration area is shown in Figure \ref{fig:CircularBodyGridIntegration}.
%\at{Bonkers way to define a grid, there must be a better way...}

%The probability contained in a grid cell, $p(\W_g)$ can then be defined in terms of the cdf evaluated at each vertex:
%\begin{equation}
%    cdf(v_\W) = \prod_{w} \frac{1}{2}\left( 1 + erf\left( \frac{v_\W(w)}{\sqrt{2}} \right) \right)
%\end{equation}

%The over approximation of the probability mass contained in $\mathcal{P}_{whit}^{ij}$, defined as $p_{poly}$ is 
%\begin{equation}
%    p_{poly} = \sum_{\W_g\in\W_{g,poly}} p(\W_g)
%\end{equation}

% The chance constraint can then be simply checked as:
% \begin{multline}
%     \text{if} \quad p_{poly}\leq p_\safe \\
%     \rightarrow P(\proj{\W}(X_k^i)) \cap \proj{\W}(X_k^j))\leq 1-p_{\safe}
% \end{multline}
This method inherits the same conservatism associated with the spherical body bound and polytope integration area bound in M.2.1. However, the approximate grid calculation is far less conservative than the deterministic linear constraints. The grid introduces more computational cost, particularly in transforming the distribution and evaluating the probability of each grid cell. Finer grid discretization reduces conservatism, but also increases computation time.

\subsection{Risk Allocation}
\label{sec:Risk Allocation}
%Note that 
Risk allocation is only necessary for M.2.1 and M.2.2 and not required in M.1. Equally distributing the allowable probability of collision $p_{coll}$ across the robots and obstacles, as introduced in \cite{Luders2010_CC-RRT}, is the simplest allocation method. For robot-robot collision this corresponds to $p_c^{ij}=\frac{p_{coll}}{N_O + N_A - 1}$. However, this can be overly conservative and make it difficult to find viable motion plans. Instead, we propose allocating $p_{coll}$ more effectively, such that robots with a higher likelihood of collision are assigned a higher proportion of $p_{coll}$.

We begin by setting $p_{coll}=P_c^A+P_c^O$, where $P_c^A$ 
% is the allowable probability of collision with all the robots, and 
and
$P_c^O$ are the allowable probabilities of collision with all the robots and all the static obstacles, respectively. This naturally fits the high-level low-level division in CC-K-CBS because obstacle collision checking occurs only in the low-level planner.
Define the total volume of all the robots as $V_A$, and the volume of all the obstacles as $V_O$. Then, we set  $P_c^A=\frac{V_A p_{coll}}{V_A+V_O}$ and $P_c^O=\frac{V_Op_{coll}}{V_A+V_O}$. This allocates more of the probability of collision to the category that is more likely to cause collisions. 
We then divide $P_c^A$ based on the distance between robots, with robots that are closer together receiving a larger portion of $P_c^A$. Let $d_k^{j}=\|\bar{x}_k^i-\bar{x}_k^j\|$ be the workspace distance between robot $i$'s and $j$'s means. Then, the allowable probability of collision for robot $i$ with $j$ is $p_c^{ij} = \frac{\alpha}{d_k^j}P_c^A$, where 
% $\alpha =\frac{1}{\sum_j^{N_A-1} d_k^{j}}$ 
$\alpha =1/ (\sum_j^{N_A-1} d_k^{j})$ 
is the normalizing factor.


\section{EVALUATIONS}
    \label{sec:eval}
    
% We implemented CC-K-CBS with belief-SST as the low-lever planner in Open Motion Planning Library (OMPL), 
% The benchmarks were performed on AMD Ryzen 9 4.5GHz CPU and 64 GB of RAM.
We evaluate our algorithms in several benchmarks. We first independently test the robot-robot collision checking algorithms in Sec.~\ref{sec:collisionChecking}, allowing us to isolate the conservatism and computational complexity of the robot-robot collision checkers. We then analyze the full algorithms from Sec.~\ref{sec:method} with different collision checking methods.
Our implementation of CC-K-CBS is publicly available~\cite{kcbs-code}. It uses belief-SST as the low-level planner in Open Motion Planning Library (OMPL) \cite{sucan2012the-open-motion-planning-library}. 
The benchmarks were performed on AMD 4.5 GHz CPU and 64 GB of RAM. 
% \textcolor{blue}{Success rate refers to the percentage of runs of the planner that return a solution within the allotted computation time.}
% this allows us to examine how the larger planning framework behaves as whole, and better evaluate the situations where one collision checker is advantageous over the other.

\begin{figure}
    \centering
    \begin{subfigure}{0.23\textwidth}
    \includegraphics[width=\textwidth]{figures/conservatism.pdf}
    \vspace{-5mm}
    \caption{}
    \label{fig:conservatism}
    \end{subfigure}
    \hfill
    \begin{subfigure}{0.23\textwidth}
    \includegraphics[width=\textwidth]{figures/CompTime_combined.pdf}
    \vspace{-5mm}
    \caption{}
    \label{fig:compTime_combined}
    \end{subfigure}
    \hfill

\caption{(a) Plot of the conservatism $C_{z}$ for each method in each environment. (b) Box and whiskers plots of the log(computation time) for each method.}
% \vspace{-3.5mm}
\end{figure}

We consider square-shaped robots of width 0.25 with two types of dynamics: 2D robotic system taken from~\cite{Bry2011_BeliefProp}; $x_{k+1}^i=x_k^i+u_k^i+w_k^i, w_k^i \sim \mathcal{N}(0,0.1^2I)$, and 2nd-order unicycle; $\dot{\text{x}}^i= \text{v} \cos\theta, \dot{\text{y}}^i= \text{v} \sin \theta, \dot{\theta}=\omega, \dot{\text{v}}=a$, where $\omega$ and $a$ are control inputs. We use the feedback linearization 
scheme described in \cite{DeLuca2000_feedbackLin} 
to obtain a linear model, 
%with $x_k^i=[\text{x}_k^i,\dot{\text{x}}_k^i, \text{y}_k^i,\dot{\text{y}}_k^i]$ and noise $Q^i=0.1^2I$. 
which we convert to discrete time with additive noise. We assume both systems are fully observable with measurement model in \eqref{eq: measurement model} with $C^i=I$ and $R^i=0.1^2I$. 


% \subsection{Robot-Robot Collision Checker Benchmarking}
\textbf{Robot-Robot Collision Checker Benchmark: }
We characterize the robot-robot collision checking algorithms based on computation time and conservatism. Conservatism impacts the planners ability to find paths in more difficult, i.e. cluttered, situations. For a given set of beliefs for two robots, we define the conservatism of method $z$ as $C_{z} = R_{z} - R_{MC}$, where $R_{z}$ is the rejection rate obtained by method $z$, and $R_{MC}$ is the rejection rate obtained by a Monte Carlo evaluation of the probability of collision.

% from the \emph{true} probability of collision, $p_{true}^{ij}$. 

% If the true probability is know, the chance constraint can be exactly evaluated: $p_{true}^{ij}<1-p_\safe$. To evaluate conservatism, we randomly sampled $N_{MC}$ sets of two beliefs for each agent, $B_{s_{ij}}=\{\expBelief^i, \expBelief^j\}, s_{ij}=1,...,N_{MC}$, with $N_{MC}=10,000$. For each pair, we then computed the true probability of collision, and evaluated the chance constraint to obtain either valid or invalid, $\bar{v}^{s_{ij}}=1$ or $0$ respectively. The true validity rate can then be calculated as $\bar{V}_{true}=\sum_{s_{ij}=1}^{N_{MC}} \bar{v}^{s_{ij}}/N_{MC}$. For each pair, we then evaluate the chance constraint according to each of the proposed collision checking algorithm, $\bar{v}_{alg}^{s_{ij}}\in\{0,1\}$, yielding an associated algorithmic validity rate $\bar{V}_{alg}=\sum_{s_{ij}=1}^{N_{MC}} \bar{v}_{alg}^{s_{ij}}/N_{MC}$. Conservatism is defined as the difference between the true validity rate and the algorithmic validity rate: $C_{alg}=\bar{V}_{alg}-\bar{V}_{true}$. This indicates how often the algorithm rejects as invalid a sample that actually satisfies the chance constraint.


% \ml{all you're saying is that conservatism of method $z$ is defined as $C_{z} = R_{z} - R_{MC}$, where $R_{z}$ is the rejection rate obtained by method $z$, and $R_{MC}$ the rejection rate obtained by Monte Carlo evaluation of the probability of collision????}

% \ml{is this paragraph needed?}
% As stated earlier, the entire motivation for the proposed agent-agent collision checking algorithms is the difficulty of evaluating the true probability of collision. For the benchmarks performed here, we approximate the true probability of collision via MC sampling. For each pair of sampled beliefs, we further sampled 50,000 states. Each sampled state corresponds to a projection of the robots bodies in the workspace, where collision is indicated by the intersection of the two projections. To calculate the 'true' probability of collision, we found the proportion of the sampled states that resulted in collision.

We sampled 2D beliefs in two difference spaces, 5$\times$5 and 10$\times$10, with $p_\safe=0.95$. For M.2.2 (the gridded cdf method), we chose 6 different grid discretizations, denoted by M.2.2($d$). For each discretization, the maximum range of $\mathcal{P}_{whit}^{ij}$ over each axis, $D_w$, was divided equally, such that the discretization step size was $D_w/d$. Note that each of the proposed methods decreases in conservatism, shown in Fig.~\ref{fig:conservatism}, and increases in computation time, shown in Fig.~\ref{fig:compTime_combined}. The 5$\times$5 space is the more `cluttered', with more beliefs in close proximity, which in turn results in higher collision probabilities. Note that the 5$\times$5 space generally results in lower conservatism. This is because for the relatively high choice of $p_\safe$, a high proportion of the sampled beliefs truly violated the chance constraint, making it less conservative compared to a scenario where more beliefs are valid.

% \begin{figure}
%     \centering
%     \includegraphics[width=0.45\textwidth]{figures/PcollVSdist_Comp.pdf}
%     \vspace{-2mm}
%     \caption{Plot of the true probability of collision vs the euclidean distance between the means for each sampled belief pair}
%     \label{fig:PcollvsDist}
%     \vspace{0mm}
% \end{figure}



%%%%%%%%%%%%%%%%% SIDE BY SIDE TABLE
% \begin{table*}[]
% \caption{CC-K-CBS benchmarks for 2D system in Env8}
%     \centering
%     \scalebox{0.84}{
%     \begin{tabular}{|c|c||c|c|c|c|c|c||c|c|c|c|c|c|}
%          \hline
%          & & \multicolumn{6}{c||}{Simple Linear} & \multicolumn{6}{c|}{Unicycle}  \\ 
%         $N_A$ & Metric & M.1 & M.2.1 & M.2.1* & M.2.2(2) & M.2.2(5) & M.2.2(10)    \\ \hline 
%         \multirow{3}{*}{2} &  Succ. Rate &  1.00&   1.00 &   1.00 &   1.00 &   1.00 &   1.00 &  1.00&   1.00 &   1.00 &   1.00 &   1.00 &   1.00\\ 
%          &  Time (s) &    1.73 &   0.71 &   \textbf{0.56} &   0.87 &   1.19 &   0.75  & 0.03 &   0.03 &   0.05 &   0.04 &   0.04 &   0.06 \\ \hline 
%         \multirow{3}{*}{4} &  Succ. Rate &  0.00&   1.00 &   1.00 &   1.00 &   1.00 &   1.00  & 1.00&   1.00 &   1.00 &   1.00 &   1.00 &   1.00\\ 
%          &  Time (s) &     - &  11.86 &  41.92 &   \textbf{5.33} &   6.09 &  10.89  &    0.21 &   0.36 &   0.27 &   0.27 &   0.36 &   0.57   \\ \hline 
%         \multirow{3}{*}{6} &  Succ. Rate &  -&   0.00 &   0.20 &   1.00 &   1.00 &   1.00  &  0.00&   0.00 &   0.00 &   0.00 &   0.00 &   0.00 \\ 
%          &  Time (s) &     - &    - & 127.82 &   \textbf{5.53} &   8.61 &  16.43   &     NaN &    NaN &    NaN &    NaN &    NaN &    NaN  \\ \hline 
%         \multirow{3}{*}{8} &  Succ. Rate &  -&   - &   0.00 &   \textbf{0.98} &   0.94 &   0.90   &   1.00  &  0.00&   0.00 &   0.00 &   0.00 &   0.00  \\ 
%          &  Time (s) &    - &   - &   0.00 &  \textbf{18.56} &  29.77 &  61.79  &     NaN &    NaN &    NaN &    NaN &    NaN &    NaN   \\ \hline     
%     \end{tabular}
%     }
%     \label{tab:benchCCKCBS}
% \end{table*}



\begin{figure}[t]
    \centering
    \begin{subfigure}[b]{0.49\columnwidth}
         \centering
         \includegraphics[width=\textwidth]{figures/unicycle_SampleTraj_8x8.pdf}
         \caption{Env8}
         \label{fig:unicycle_SampleTraj_8x8}
     \end{subfigure}
    \begin{subfigure}[b]{0.485\columnwidth}
         \centering
         \includegraphics[width=\textwidth]{figures/unicycle_SampleTraj_32x32obs.png}
         \caption{Env32Obs}
         \label{fig:unicycle_SampleTraj_32x32obs}
     \end{subfigure}
     \hfill
     \hspace{-2mm}
     \caption{(a) Sample plan for 4 robots with 2nd-order unicycle dynamics in Env8. Ellipses are $95\%$ probability contours. (b) Sample plan for 20 robots with 2nd-order unicycle dynamics in Env32Obs. Ellipses are $95\%$ probability contours.}
     \label{fig:envs}
     % \vspace{-2mm}
\end{figure}

%%%%%%%%%%%%%%%%% OLD RESULTS TABLE (JUST LINEAR)

% \begin{table}[]
% \caption{CC-K-CBS benchmarks for Env8 (OLD RESULTS)}
%     \centering
%     \scalebox{0.84}{
%     \begin{tabular}{|c|c|c|c|c|c|c|c|}
%          \hline
%         $N_A$ & Metric & M.1 & M.2.1 & M.2.1* & M.2.2(2) & M.2.2(5) & M.2.2(10) \\ \hline 
%         \multirow{3}{*}{2} &  Succ. Rate &  1.00&   1.00 &   1.00 &   1.00 &   1.00 &   1.00\\ 
%          &  Time (s) &    1.73 &   0.71 &   \textbf{0.56} &   0.87 &   1.19 &   0.75 \\ \hline 
%         \multirow{3}{*}{4} &  Succ. Rate &  0.00&   1.00 &   1.00 &   1.00 &   1.00 &   1.00\\ 
%          &  Time (s) &     - &  11.86 &  41.92 &   \textbf{5.33} &   6.09 &  10.89 \\ \hline 
%         \multirow{3}{*}{6} &  Succ. Rate &  -&   0.00 &   0.20 &   1.00 &   1.00 &   1.00\\ 
%          &  Time (s) &     - &    - & 127.82 &   \textbf{5.53} &   8.61 &  16.43 \\ \hline 
%         \multirow{3}{*}{8} &  Succ. Rate &  -&   - &   0.00 &   \textbf{0.98} &   0.94 &   0.90\\ 
%          &  Time (s) &    - &   - &   0.00 &  \textbf{18.56} &  29.77 &  61.79 \\ \hline   
%     \end{tabular}
%     }
%     \label{tab:benchCCKCBS}
% \end{table}




%%%%%%%%%%%%%%%%% NEW RESULTS STACKED TABLE
\begin{table}[t]
\caption{CC-K-CBS benchmarks for Env8.}
    \centering
    \scalebox{0.82}{
    \begin{tabular}{|c|c|c|c|c|c|c|c|}
         \hline
         \multicolumn{8}{|c|}{2D Linear} \\ \hline
        $N_A$ & Metric & M.1 & M.2.1 & M.2.1* & M.2.2(2) & M.2.2(5) & M.2.2(10) \\ \hline 
        \multirow{3}{*}{2} &  Succ. Rate &  1.00&   1.00 &   1.00 &   1.00 &   1.00 &   1.00\\ 
         &  Time (s) &    1.08 &   \textbf{0.68} &   1.04 &   1.11 &   0.69 &   0.74 \\ \hline 
        \multirow{3}{*}{4} &  Succ. Rate &  0.00&   1.00 &   1.00 &   1.00 &   1.00 &   1.00\\ 
         &  Time (s) &     - &  18.68 &  35.03 &   4.19 &   \textbf{3.97} &   7.61 \\ \hline 
        \multirow{3}{*}{6} &  Succ. Rate &  -&   0.00 &   0.12 &   1.00 &   1.00 &   1.00\\ 
         &  Time (s) &     - &    - &  66.68 &   \textbf{4.44} &   6.15 &  12.72 \\ \hline 
        \multirow{3}{*}{8} &  Succ. Rate &  -&   - &   0.00 &   1.00 &   1.00 &   0.86\\ 
         &  Time (s) &     - &    - &    - &  \textbf{15.38} &  26.54 &  59.08 \\ \hline  
         \multicolumn{8}{|c|}{Unicycle} \\ \hline
        \multirow{3}{*}{2} &  Succ. Rate &  1.00&   1.00 &   1.00 &   1.00 &   1.00 &   1.00\\ 
         &  Time (s) &    0.95 &   0.63 &   0.61 &   \textbf{0.37} &   0.57 &   0.83 \\ \hline 
        \multirow{3}{*}{4} &  Succ. Rate &  0.00&   0.96 &   1.00 &   1.00 &   1.00 &   1.00\\ 
         &  Time (s) &     - &  17.67 &   9.81 &   \textbf{2.59} &   2.86 &   5.24 \\ \hline 
        \multirow{3}{*}{6} &  Succ. Rate & -&   0.00 &   0.00 &   1.00 &   1.00 &   1.00\\ 
         &  Time (s) &     - &    - &    - &   \textbf{4.79} &   5.95 &  10.40 \\ \hline 
        \multirow{3}{*}{8} &  Succ. Rate &  -&   - &   - &   1.00 &   0.98 &   0.96\\ 
         &  Time (s) &     - &    - &    - &  \textbf{18.27} &  25.52 &  45.73 \\ \hline 
    \end{tabular}
    }
    \label{tab:benchCCKCBS}
\end{table}





% \subsection{Characterizing the MRMP Planners}
\textbf{Planners Benchmark: }
% We first compare benchmarks on the performance of the centralized and CC-K-CBS planners proposed in Sec.~\ref{sec:method}. We then present specific plans generated by the decentralized method, along with results from Monte Carlo (MC) simulations demonstrating the plans' validity. 
% \subsubsection{Planner Benchmarking}
We use benchmarking to evaluate the proposed planners under each of the collision-checking methods in Sec.~\ref{sec:collisionChecking}. We additionally implement the adaptive risk allocation on M.2.1, denoted by M.2.1*. All other methods use equal allocation. 
% We are interested in comparing two quantities: \emph{success rate} and \emph{computation time}. 
% The first is the success rate, defined as the rate at which the planner returns a solution within the given maximum planning duration. The second is the computation time, which indicates the computational complexity of the algorithms. 
Benchmarking was performed using 50 runs for each algorithm with a maximum planning time of 3 minutes and $P_\safe = 0.9$. We examine 3 environments: a small 8$\times$8 (Env8), large 32$\times$32 with 50 random obstacles (Env32Obs), and large 32$\times$32 without obstacles (Env32) shown in Figs.~\ref{fig:unicycle_SampleTraj_8x8}, \ref{fig:unicycle_SampleTraj_32x32obs}, and \ref{fig:unicycle_SampleTraj_big}, respectively.
The considered metrics are \emph{success rate} (the ratio of the successful planning instances within 3 minutes to all instances) and \emph{time} (computation time of the successful runs).

%%%%%% BACKUP PARAGRAPH
\textit{Env8: } This small environment allows us to study how the different collision checking methods perform in a cluttered environment with the 2D system. The results are shown in Table~\ref{tab:benchCCKCBS}. Conservatism strongly impacts the planner's ability to find paths with additional robots. All methods reliably find plans for 2 robots, however as the conservatism of the method increases, the success rate decreases when more agents are added. In this cluttered environment robot paths are necessarily close together, so more conservative methods are less likely to find valid paths.  Specifically, M.1 performs poorly and M.2.2 has the best performance.

% \textit{Env8: } This small environment allows us to study how the difference collision checking methods perform in a cluttered environment. The results are shown in Table~\ref{tab:benchCCKCBS}. Here the conservatism of the method strongly impacts the planners ability to find paths when more agents are introduced. All methods reliably find plans for 2 robots, however as the conservatism of the method increases, the success rate decreases when more agents are added. In this cluttered environment robot paths are necessarily close together, so more conservative methods are less likely to find valid paths.  Specifically, Method 1 performs poorly and Method 2.2 has the best performance.

\textit{Env32Obs: } The large environment is far less cluttered, allowing for far more agents and better insight into the scalability of the CC-K-CBS algorithm. The results for Env8 are based on the robot-obstacle collision checker from \cite{Luders2010_CC-RRT}. However, this is very conservative, especially for many obstacles.  CC-K-CBS with this method was unable to find paths in Env32Obs for even 2 robots. To scale to a larger number of robots, we made the very straightforward adaptation to M.1 to check intersection of the bound $\mathcal{C}_{bound}^i$ with obstacles. 
As shown in Fig~\ref{fig:benchmark 30robots}, CC-K-CBS is able to scale to $20$ robots with $68\%$ success rate and runtime of $120.7\pm5.45$ seconds for the 2nd-order unicycle dynamics. A sample plan is shown in Fig.~\ref{fig:unicycle_SampleTraj_32x32obs}.  This illustrates that for large number of robots, M.1 has the best performance because it is not affected by risk allocation, whereas the allocated risk gets prohibitively smaller as the number of robots increases for the other methods, making it difficult to find valid plans.

We also performed benchmarking on the centralized approach with the simpler 2D system in Env32Obs, using collision checking M.1. 
% We allocated the full planning time for optimization, so computation time is not reported. The centralized approach allows optimization via the SST algorithm, which does result in improved sum of controls. However, it quickly runs into scalability problems, 
This algorithm was able to plan for 2 robots with $100\%$ success rate, but the success rate dropped to $8\%$ for 3 robots, illustrating the classical scalability issue with centralized approaches. 

\textit{Env32: } To further test the scalability of CC-K-CBS, we increased the number of agents in the large open environment Env32. 
As shown in Fig~\ref{fig:benchmark 30robots}, CC-K-CBS is able to to scale to $28$ 2nd-order unicycle robots with $56\%$ success rate and $129\pm4.97$ seconds computation time. 
A sample plan is shown in Fig.~\ref{fig:unicycle_SampleTraj_big}.  CC-K-CBS is also able to plan for 30 robots but at much lower success rate of $18\%$.

\begin{figure}
    \centering
    \includegraphics[width=0.86\columnwidth]{figures/compareBench_32x32.pdf}
    % \vspace{-2mm}
    \caption{Benchmark results for CC-K-CBS on varying number of 2nd-oder unicycle robots. For \# robots $\leq 20$, Env32Obs is used (dashed line), and for \# robots $\geq 22$, Env32 is used (solid line). The discontinuity is due to removing obstacles.
    }
    \label{fig:benchmark 30robots}
\end{figure}


% \subsubsection{Planner Behavior}
We validated the robustness of the motion plans using Monte Carlo simulation. We randomly selected motion plans, and with 500 simulations for each robot and observed no more than $2\%$ collision probability over the entire trajectory, illustrating the plans are indeed robust to uncertainty. 

%We present sample plans generated by the K-CBS algorithm under both dynamics models with $p_\safe=0.90$. These plans were generated in a 32x32 environment containing 50 random obstacles. Figure \ref{fig:unicycle_SampleTraj_big} presents the nominal trajectories for 12 robots, with the $90\%$ confidence bound plotted as ellipses. We verified that these bounds do not simultaneously intersect at any point in the nominal trajectories, nor do they intersect with obstacles. We additionally verify the probability of collision over entire sampled trajectories for the simple 2D linear system using 500 MC samples of each robot's trajectory. The empirically obtained probability of collision was $3.7\%$ for the simple 2D system. Examples of the nominal plans and sampled trajectories for the simple 2D system are shown in Figure \ref{fig:Simple2D_sampleTraj}.



% \begin{figure}
%     \centering
%     \includegraphics[width=0.4\textwidth]{figures/unicycle_SampleTraj.pdf}
%     \vspace{-2mm}
%     \caption{For the unicycle system, $95\%$ confidence bound plotted as ellipses, with the nominal trajectory over-plotted as a solid line for each of 7 agents.}
%     \label{fig:unicycle_sampleTraj}
%     \vspace{0mm}
% \end{figure}

% \begin{figure}
%     \centering
%     \begin{subfigure}{0.33\textwidth}
%     \includegraphics[angle=90,width=\textwidth]{figures/simpleLin_SampleTraj.pdf}
%     \vspace{-7mm}
%     \caption{}
%     \label{fig:Simple2D_sampleTraj}
%     \end{subfigure}
%     \hfill
%     \begin{subfigure}{0.14\textwidth}
%     \includegraphics[width=\textwidth]{figures/simpleLin_SampleTraj_narrow.pdf}
%     \vspace{-7mm}
%     \caption{}
%     \label{fig:Simple2D_sampleTraj_narrow}
%     \end{subfigure}
%     \hfill

% \caption{(a) For the simple 2D system in a 32x32 environment: 100 MC sampled trajectories plotted as transparent lines, with the nominal trajectory over-plotted as a solid line. (b) Sample 6x2 narrow environment}
% \vspace{-3.5mm}
% \end{figure}




%      \section{Benchmarks and Evaluation}
\label{sec:eval}

We evaluate \krakenSpace to answer the following set of questions:
\begin{itemize}
\item How much improvement does partial evaluation and our implemented compiler optimizations give \kraken? %(\S \ref{sec:eval2})
\item How much faster is our purely functional f-expr language, \krakenSpace, compared to other implementations of fexprs? %(\S \ref{sec:eval1} - \ref{sec:eval2})
\item How does \kraken's performance, with its fexprs, compare to macros? %(\S \ref{sec:eval1}, \S \ref{sec:eval3})
\item How do the different partial evaluation mechanisms/optimizations in \krakenSpace contribute towards reduction in overall runtime?
%\item What does \krakenSpace do internally when we create a data structure and evaluate it for some function? (\S \ref{sec:casestudy})
\end{itemize}

\textbf{Experimental Setup}: 
We ran these experiments in a reproducible Nix environment on a NixOS install \cite{10.1145/1411203.1411255} (Kernel 6.0.0) on a laptop with 8 cores / 16 threads and 64 GB of RAM.
Our code contains the scripts and Nix Flakes needed to reproduce the exact set of dependencies to run our tests.
%The code can be found at \url{https://github.com/limvot/kraken}.

The Kraken benchmarks were run using both the Wasmtime and WAVM WebAssembly engines for most benchmarks.
The Wasmtime WebAssembly engine is one of the most popular, developed by the Bytecode Alliance itself, and uses the CraneLift code generation backend.
The WAVM WebAssembly engine is interesting for its use of LLVM, and it often produces the fastest code on benchmarks but has a higher startup time.
We eliminated the Cfold Wasmtime benchmark due to problems running out of stack space (a known property of the Cfold benchmark).

\textbf{Benchmarks}: 
To showcase the capability of Kraken, we created benchmarks that are commonly implemented in functional languages and have been used as benchmarks in other papers \cite{reinking2021perceus, 10.1145/3547646}.
The benchmarks are
\begin{itemize}
\item Fib - Calculating the nth Fibonacci number
\item RB-Tree - Inserting n items into a red-black tree, then traversing the tree to sum its values
\item Deriv - Computing a symbolic derivative of a large expression
\item Cfold - Constant-folding a large expression
\item NQueens - Placing n number of queens on the board such that no two queens are diagonal, vertical, or horizontal from each other
\end{itemize}
All benchmarks besides Fibonacci use the fexpr version of match for pattern matching in \kraken, which is equivalent to the macro version in NewLisp. We also RB-Tree using NewLisp's~\cite{mueller2018newlisp} version of fexpr match. We modified the sizes of the problems presented to the benchmark to account for the longer running times of some of the less-optimized implementations.
The code for Kraken and NewLisp is very similar, and we should note that it is very unidiomatic NewLisp.
Our goal was not to compare Kraken and NewLisp as implementation languages for Red-Black Trees, but to stress test a single reasonably complex fexpr/macro, namely pattern matching.
% \textbf{Comparison with other languages}: We evaluated \krakenSpace against a language that contains f-exprs, as well as against itself with various optimizations disabled. The only other language we could find which contains a real f-expr mechanism is NewLisp~\cite{mueller2018newlisp} and so we ported \kraken's benchmark implementation to NewLisp.

%The six state-of-the-art languages are Java 17.0.1, Swift 5.4.2, Koka 2.3.2, C++, Haskell 8.10.7, and OCaml 4.12.
%The language choices were taken directly from Perceus reference-counting paper \cite{reinking2021perceus}.
%The Fibonacci benchmark additionally tests Python 3.9.11 and Chez Scheme 9.5.4.
%Koka, Ocaml and Haskell are good comparison points as statically-typed, compiled, functional programming languages, while Chez Scheme is a good comparison point as a mature and industrial strength dynamically-typed Scheme implementation known for its performance. 
%\subsection{Basic Level Comparison}
\subsection{The Effect of Partial Evaluation on Eval Calls}

\begin{table}[h]
\caption{Number of eval calls with no partial evaluation for Fexprs}
	\begin{tabular}{||c | c c c c c ||} 
		\hline
		&Evals & Eval w1 Calls & Eval w0 Calls & Comp Dyn & Comp Dyn\\ 
        & & & & w1 Calls & w0 Calls\\ [0.5ex] 
		\hline\hline
		Cfold 5 & 10897376 & 2784275 & 879066  & 1 & 0 \\ 
		\hline
		  Deriv 2  & 11708558 & 2990090 & 946500 & 1 & 0 \\ 
        \hline
		  NQueens 7 & 13530241 & 3429161 & 1108393 & 1 & 0 \\ 
    \hline
		  Fib 30 & 119107888 & 30450112 & 10770217 & 1 & 0 \\ 
    \hline
		  RB-Tree 10 & 5032297 & 1291489 & 398104 & 1 & 0 \\ 
		\hline
	\end{tabular}
    \label{npe:calls}
 \end{table}

As mentioned before, using fexprs without partial evaluation will prelude optimization and cause a massive amount of repeated work. Table \ref{npe:calls} and Table \ref{pe:calls} show the number of calls to the \krakenSpace runtime's eval function, the number of times the runtime's eval function executed a call to an applicative with wrap\_level=1, the number of times the runtime's eval function executed a call to an operative with wrap\_level=0, the number of compiled dynamic calls to applicatives with wrap\_level=1, and the number of compiled dynamic calls to operatives with wrap\_level=0.
These are shown for \krakenSpace test cases with partial evaluation turned off and turned on. 
\begin{table}[h]
\caption{Number of eval calls in Partially Evaluated Fexprs}
	\begin{tabular}{||c | c c c c c ||} 
		\hline
		&Evals & Eval w1 Calls & Eval w0 Calls & Comp Dyn & Comp Dyn\\ 
        & & & & w1 Calls & w0 Calls\\ [0.5ex] 
		\hline\hline
		Cfold 5 & 0 & 0 & 0  & 0 & 0 \\ 
		\hline
		  Deriv 2  & 0 & 0 & 0 & 2 & 0 \\ 
        \hline
		  NQueens 7 & 0 & 0 & 0 & 0 & 0 \\ 
    \hline
		  Fib 30 & 0 & 0 & 0 & 0 & 0 \\ 
    \hline
		  RB-Tree 10 & 0 & 0 & 0 & 10 & 0 \\ 
		\hline
	\end{tabular}
    \label{pe:calls}
 \end{table}

\begin{table}[h]
\caption{Number of calls to the runtime's eval function for RB-Tree. The table shows the non-partial evaluation numbers -> partial evaluation numbers.}
	\begin{tabular}{||c | c c c c c ||} 
		\hline
		&Evals & Eval w1 Calls & Eval w0 Calls & Comp Dyn & Comp Dyn\\ 
        & & & & w1 Calls & w0 Calls\\ [0.5ex] 
		\hline\hline
		  RB-Tree 7 & 2952848 -> 0 & 757932 -> 0 & 233513 -> 0 & 1 -> 7 & 0 -> 0\\ 
        \hline
		  RB-Tree 8 & 3532131 -> 0 & 906548 -> 0 & 279379 -> 0 & 1 -> 8 & 0 -> 0\\ 
        \hline
		  RB-Tree 9 & 4278001 -> 0 & 1097965 -> 0 & 3383831 -> 0 & 1 -> 9 & 0 -> 0\\ 
		\hline
	\end{tabular}
    \label{pe:rb}
    \vspace{-4mm}
 \end{table}

Without partial evaluation, no compilation can be done because it is impossible to tell if arguments to calls will be evaluated. In all benchmarks, partial evaluation removed all calls to the runtime's eval function, resulting in a completely compiled program. Looking at RB-Tree, there are over a million calls to combiners with wrap level 1 (normal functions), and 398,000 calls to combiners with wrap level 0 (operatives replacing macros). This massive blowup in the number of calls is due to the repeated and exponential re-execution of macro-like-combiners in the definition of other macro-like-combiners, as discussed in the Introduction.

The non-partially-evaluated benchmarks show 1 compiled dynamic call to an applicative (its the first call into eval) and 0 compiled dynamic calls to operatives, because there is no compilation at all. For the partially evaluated benchmarks, there are a few compiled dynamic calls to applicatives due to higher-order function use in the benchmarks, and there are no compiled dynamic calls to operatives, as all operative use has been eliminated.
We also varied the inputs for RB-Tree shown in Table \ref{pe:rb} to give a sense for how the number scale with respect to input size.

The incredible slowdown implied by these tables comes to full fruition in our RB-Tree test in Fig.~\ref{fig:kraken_nqueens_rbtree}.
We kept this run shorter because Kraken's non-partial-evaluating interpreter takes an incredibly long time even for 100 insertions (40 minutes).
The compounding layers of repeated macro-like operative calls in the non-partially-evaluated Kraken version cause a ~70,000x slowdown relative to the partial evaluated, optimized, and compiled version.
For the remaining benchmarks, we remove the naive interpreted \krakenSpace version, as in each case its performance is so bad as to blow out the graph and make it impossible to do any comparison.
In our optimized Kraken, our partial evaluation algorithm is able to fully collapse these levels of inefficiency, evaluate and inline the results, and give the backend more specialized code to optimize, emitting a compiled version that handily beats not only the NewLisp-fexpr implementation but even the NewLisp-macro implementation, as can be seen in Fig.~\ref{fig:kraken_vs_world_fib}.
We kept the benchmark sizes small in this test because the stack limits of NewLisp prevent sizes larger then ~880, while the Tail Call Elimination performed by the \krakenSpace compiler allows us to run much larger benchmarks, including the run of 4,800,000 inserts to the RB-Tree.
This result shows the dramatic effect of partial evaluation and compiler optimizations on runtime for \kraken. Our technique takes the performance of a fully fexpr based language from being completely infeasible to being faster than a macro-based dynamic scripting language currently in use.
% \begin{center}
% \begin{table}[ht]
% \caption{Number of call to the runtime's eval function for Fib. The table shows the non-partial evaluation numbers -> partial evaluation numbers}
% 	\begin{tabular}{||c | c c c c c ||} 
% 		\hline
% 		&Evals & Eval w1 Calls & Eval w0 Calls & Comp Dyn w1 Calls & Comp Dyn w0 Calls\\ [0.5ex] 
% 		\hline\hline
% 		Fib 10 & 8468 -> 0 & 2167 -> 0  & 777 -> 0 & 1 -> 0 & 0 -> 0 \\ 
% 		\hline
% 		  Fib 15  & 87916 -> 0 & 22478 -> 0 & 7961 -> 0 & 1 -> 0 & 0 -> 0 \\ 
%         \hline
% 		  Fib 20 & 969010 -> 0 & 247731 -> 0 & 87633 -> 0 & 1 -> 0 & 0 -> 0 \\ 
%     \hline
% 		  Fib 25 & 10740492 -> 0 & 2745825 -> 0  & 971209 -> 0 & 1 -> 0 & 0 -> 0 \\ 
% 		\hline
% 	\end{tabular}
%     \label{pe:fib}
%  \end{table}
% \end{center}

\begin{figure}[h]
\caption{Constant Fold and Deriv}
\includegraphics[width=0.45\textwidth]{cfold_table.csv_}
\includegraphics[width=0.45\textwidth]{deriv_table.csv_}
\label{fig:kraken_const_deriv}
\vspace{-6mm}
\end{figure}
\subsection{Comparison between Kraken Versions}
Beyond the massive speedup from partial-evaluation, Fig. \ref{fig:kraken_const_deriv} and \ref{fig:kraken_nqueens_rbtree} show the effect of the various compiler optimizations we described by disabling them one by one.
 Our main four optimizations have a strong positive effect on runtime, with the exception of lazy environment instantiation. Lazy environment instantiation helps massively on fib, and some on Deriv, but generally hurts the rest slightly.


\begin{figure}[h]
\caption{N-Queens}
\includegraphics[width=0.45\textwidth]{nqueens_table.csv_}
\includegraphics[width=0.45\textwidth]{slow_rbtree_table.csv_}
\label{fig:kraken_nqueens_rbtree}
\vspace{-4mm}
\end{figure}


\subsection{Comparison against Others}


To give a general idea of our current performance, we also show a Fibonacci benchmark that mostly exercises pure function-call speed and inlining as seen in Fig. ~\ref{fig:kraken_vs_world_fib}.
We include Python and Chez Scheme to give a general idea for where an exemplar slow and an exemplar fast dynamic language would fall.
With the benefit of our partial evaluation, compilation, and leaning upon mature WebAssembly implementations, we beat both, but this should be taken with a grain of salt, as this is a very limited micro-benchmark only meant to give a general sense of the order of magnitude of our performance.



\label{sec:eval1}
\begin{figure}[h]
\caption{Kraken vs. Others. Ordered by fastest to slowest}
\includegraphics[width=0.45\textwidth]{fib_table.csv_}
\includegraphics[width=0.45\textwidth]{rbtree_table.csv_}
\label{fig:kraken_vs_world_fib}
\end{figure}

%\label{sec:eval_nqueens}
%\begin{figure}[h]
%\caption{N-Queens}
%\includegraphics[width=0.45\textwidth]{nqueens_table.csv_}
%\includegraphics[width=0.45\textwidth]{slow_nqueens_table.csv_}
%\label{fig:kraken_nqueens}
%\end{figure}

%\label{sec:eval_nqueens}
%\begin{figure}[h]
%\caption{Kraken, N-Queens, absolute value and log-scale}
%\includegraphics[width=0.45\textwidth]{nqueens_table.csv_}
%\includegraphics[width=0.45\textwidth]{nqueens_table.csv_log}
%\label{fig:kraken_nqueens}
%\end{figure}
%\label{sec:eval_nqueensp}
%\begin{figure}[h]
%\caption{Kraken, N-Queens, absolute value and log-scale}
%\includegraphics[width=0.45\textwidth]{slow_nqueens_table.csv_}
%\includegraphics[width=0.45\textwidth]{slow_nqueens_table.csv_log}
%\label{fig:kraken_nqueensp}
%\end{figure}

%\label{sec:eval_cfold}
%\begin{figure}[h]
%\caption{C-Fold}
%\includegraphics[width=0.45\textwidth]{cfold_table.csv_}
%\includegraphics[width=0.45\textwidth]{slow_cfold_table.csv_}
%\label{fig:kraken_cfold}
%\end{figure}
%\label{sec:eval_cfold}
%\begin{figure}[h]
%\caption{Kraken, C-Fold, absolute value and log-scale}
%\includegraphics[width=0.45\textwidth]{cfold_table.csv_}
%\includegraphics[width=0.45\textwidth]{cfold_table.csv_log}
%\label{fig:kraken_cfold}
%\end{figure}
%\label{sec:eval_cfoldp}
%\begin{figure}[h]
%\caption{Kraken, C-Fold, absolute value and log-scale}
%\includegraphics[width=0.45\textwidth]{slow_cfold_table.csv_}
%\includegraphics[width=0.45\textwidth]{slow_cfold_table.csv_log}
%\label{fig:kraken_cfoldp}
%\end{figure}

%\label{sec:eval_deriv}
%\begin{figure}[h]
%\caption{Deriv}
%\includegraphics[width=0.45\textwidth]{deriv_table.csv_}
%\includegraphics[width=0.45\textwidth]{slow_deriv_table.csv_}
%\label{fig:kraken_deriv}
%\end{figure}
%\label{sec:eval_deriv}
%\begin{figure}[h]
%\caption{Kraken, Deriv, absolute value and log-scale}
%\includegraphics[width=0.45\textwidth]{deriv_table.csv_}
%\includegraphics[width=0.45\textwidth]{deriv_table.csv_log}
%\label{fig:kraken_deriv}
%\end{figure}
%\label{sec:eval_derivp}
%\begin{figure}[h]
%\caption{Kraken, Deriv, absolute value and log-scale}
%\includegraphics[width=0.45\textwidth]{slow_deriv_table.csv_}
%\includegraphics[width=0.45\textwidth]{slow_deriv_table.csv_log}
%\label{fig:kraken_derivp}
%\end{figure}

%\subsection{Comparison against state-of-the-art languages}
%\label{sec:eval3}

%\begin{figure}[h]
%\caption{Kraken vs. S.o.t.A.}
%\includegraphics[width=0.45\textwidth]{cfold_table.csv_}
%\includegraphics[width=0.45\textwidth]{rbtree_table.csv_}
%\label{fig:kraken_vs_world1}
%\end{figure}

%\begin{figure}[h]
%\caption{Kraken vs. S.o.t.A.}
%\includegraphics[width=0.45\textwidth]{deriv_table.csv_}
%\includegraphics[width=0.45\textwidth]{nqueens_table.csv_}
%\label{fig:kraken_vs_world2}
%\end{figure}

% \begin{figure}[h]
% \caption{Kraken vs. S.o.t.A. (Log)}
% \includegraphics[width=0.45\textwidth]{cfold_table.csv_log}
% \includegraphics[width=0.45\textwidth]{rbtree_table.csv_log}
% \label{fig:kraken_vs_world_log_1}
% \end{figure}
% \begin{figure}[h]
% \caption{Kraken vs. S.o.t.A. (Log)}
% \includegraphics[width=0.45\textwidth]{deriv_table.csv_log}
% \includegraphics[width=0.45\textwidth]{nqueens_table.csv_log}
% \label{fig:kraken_vs_world_log_2}
% \end{figure}

%As we noted before with the Fib(30) microbenchmark in Section \ref{sec:eval1}, we remain significantly slower than state-of-the-art compiled languages.
%This is particularly true for memory-intensive benchmarks due to our naive reference-counting and malloc/free implementations.
%However, our results are of a similar order of magnitude to the difference between the state-of-the-art compiled languages and dynamic scripting languages, like Python's results in the Fib(30) microbenchmark.
%We assert that is not a fundamental limitation because the classic f-expr slowness is being eliminated, as shown by Fig. \ref{fig:kraken_vs_newlisp1} and Fig. \ref{fig:kraken_vs_newlisp2}.
%In future work, we plan to expand our compile-time analysis and optimization to implement a modified, dynamic-language version of Perceus reference counting.
%With this change, we belive \krakenSpace can be competitive with these state-of-the-art languages.

%\subsection{Case Study: Red-Black Tree}
%\label{sec:casestudy}

%\begin{figure}[h]
%\caption{Kraken vs. S.o.t.A. - RB-Tree Focus}
%\includegraphics[width=0.4\textwidth]{rbtree_table.csv_}
%\includegraphics[width=0.4\textwidth]{rbtree_table.csv_log}
%\label{fig:kraken_vs_world_rbtree}
%\end{figure}


%To evaluate our partial evaluation algorithm and compiler, we extracted the benchmarks used by the Koka language project from their code repository and added Kraken versions, as well as implementing a naive Fibonacci microbenchmark ourselves to evaluate pure function call speed.\\
%With partial evaluation and the compiler optimizations listed above, we get fairly strong performance on purely numerical computations, such as the naive Fibonacci microbenchmark.
%Unfortunately, the overhead of our unsophisticated reference counting, dynamic type checking, and bounds checking causes poor performance on benchmarks involving data structures relative to mainstream programming language implementations.
%This is not a fundamental limitation, and will be addressed in future work, as recounted in the next section.
%It should be noted, however, that while the performance relative to established language implementations is very poor for the memory-intensive benchmarks (600-900x slower), we still realize a massive speedup compared to an unoptimized and non-partial-evaluated f-expr implementation (100,000x faster)!

    
\section{CONCLUSION}
    \label{sec:conclusion}
    In this work, we consider the MRMP problem under Gaussian uncertainty and introduce CC-K-CBS, which generates motion plans for each robot with chance-constraint guarantees.
% on the probability of collision with obstacles and other robots. 
This algorithm is based on extending K-CBS to accommodate belief space planning. 
% using belief-$\mathcal{A}$. 
Our main contribution includes three methods for fast evaluation of collision probabilities between robots. 
% We provide benchmarks comparing the computation time and conservatism of each proposed method, as well as benchmarks characterizing the performance of CC-K-CBS as a whole. 
This offline algorithm successfully generated motion plans across a variety of scenarios and scaled to 30 robots.
% in an empty environment, however 
Future directions could encompass other types of noise and nonlinear dynamics as well as new collision checking methods that could further 
improve computation time and conservatism.


%%%%%%%%%%%%%%%%%%%%%%%%%%%%%%%%%%%%%%%%%%%%%%%%%%%%%%%%%%%%%%%%%%%%%%%%%%%%%%%
% \addtolength{\textheight}{-10cm}   % This command serves to balance the column lengths
                                  % on the last page of the document manually. It shortens
                                  % the textheight of the last page by a suitable amount.
                                  % This command does not take effect until the next page
                                  % so it should come on the page before the last. Make
                                  % sure that you do not shorten the textheight too much.



% %%%%%%%%%%%%%%%%%%%%%%%%%%%%%%%%%%%%%%%%%%%%%%%%%%%%%%%%%%%%%%%%%%%%%%%%%%%%%%%% REFERENCES
% \newpage
\bibliographystyle{IEEEtran}
\bibliography{refs}

\end{document}
