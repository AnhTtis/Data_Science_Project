% problem formulation

\subsection{Uncertain Multi-Robot System}
We consider $N_A$ uncertain robots in a shared workspace $\W \subset \reals^w$, $w\in\{2,3\}$ with a set of static obstacles $\W_O \subset \W$. Each robot $i \in \{1,\ldots, N_A\}$ has a rigid body $\B^i \subset \W$ and uncertain dynamics
\begin{equation}
\label{eq: dynamics model}
    x_{k+1}^i=A^ix_k^i + B^i u_k^i + w^i_k, \quad w_k^i \sim \N(0,Q^i), 
\end{equation}
where $x_k^i\in\X^i \subseteq\reals^{n_i}$ and $u_k^i\in\mathcal{U}^i \subseteq \reals^{p_i}$ are the state and controls, respectively, with the associated $A^i \in\reals^{n_i\times n_i}$ and $B^i \in\reals^{n_i\times p_i}$ matrices, and $w^i_k$ is the motion (process) noise sampled from a Gaussian distribution with zero mean and covariance $Q^i \in \reals^{n_i \times n_i}$. 

Note that state $x_k^i \in \X^i$ corresponds to the origin of the body frame of robot $\B^i$ w.r.t. to the global frame. The robot body $\B^i$ is defined as a set of points in its local frame. 
Since state $x_k^i$ includes configuration (position and orientation) as well as velocity terms, the body of the robot is well-defined w.r.t. global frame by projection of the state to either the configuration space (C-space) or workspace ($\W$).
 We define operator $\proj{\mathcal{S}}: \cup_{i=1}^{N_A} \X^i \to \mathcal{S}$ to be this projection of $x_k^i \in \X^i$ to its the lower dimensional space $\mathcal{S}\subset \X^i$, e.g., $\proj{\W}(x_k^i)\subset\W$ is the set of points ($\B^i$) that robot $i$ at state $x_k^i$ occupies in the workspace.

% \ml{do we have to assume measurement noise? it just makes it more complicated, no?}\at{Yes, but that's how the Gaussian belief trees are formulated, should keep it consistent with that?}
% \ml{GBT can be used with perfect measurement.... with process noise, we're still in the belief space}
Each robot $i \in \{1, \dots, N_A\}$ also has sensor (measurement) uncertainty described by
\begin{equation}
\label{eq: measurement model}
    y_k^i=C^ix_k^i + v^i_k, \quad v_k^i \sim\N(0,R^i), 
\end{equation}
where $y_k^i\in\Y^i \subseteq\reals^{m_i}$ is the sensor measurement, $C^i \in\reals^{m_i \times n_i}$ is the corresponding matrix, and $v_k^i$ is the measurement noise, sampled from Gaussian distribution with zero mean and covariance $R^i$. We additionally assume the system is both controllable and observable.


\subsection{Estimation, Control, and Motion Plan}
Each robot is equipped with a standard Kalman Filer (KF) and feedback controller.  The uncertainty of robot $i$ state, also known as the \textit{belief} of $x^i_k$ and denoted by $b(x^i_k)$,  is given by the KF as a Gaussian distribution at each time step $k$, i.e.,  
$$x^i_k \sim b(x_k^i)=\N(\hat{x}_k^i,\Sigma_k^i,)$$ 
where $\hat{x}_k^i \in \X^i$ is the state estimate (mean), and $\Sigma_k^i \in \reals^{n_i \times n_i}$ is the covariance matrix.  

We define \emph{motion plan} for robot $i$ to be a pair $(\check{U}^i,\check{X}^i)$ of nominal controls $\check{U}^i=(\check{u}_0^i, \check{u}_1^i, \ldots, \check{u}_{T-1}^i)$ and the corresponding trajectory of nominal states $\check{X}^i=(\check{x}_0^i, \check{x}_1^i, \ldots, \check{x}_T^i)$, where $\check{X}^i$ is obtained via propagation of the dynamics \eqref{eq: dynamics model} under $\check{U}^i$ in the absence of process noise.  Then, robot $i$ executes motion plan $(\check{U}^i, \check{X^i})$ by using a proportional stabilizing feedback controller
\begin{equation}
\label{eq:feedbackController}
    u_k^i = \check{u}_{k-1}^i - K(\hat{x}_k^i - \check{x}_k^i),
\end{equation}
where $K$ is the control gain.

% Each robot's initial state is Gaussian distributed with mean $\hat{x}_0^i$ and covariance $\Sigma_0^i$, i.e. $x_0^i\sim\N(\hat{x}_0^i,\Sigma_0^i)$.

\subsection{Probabilistic Collisions and Task Completion}
% \ml{Probabilistic Collision and Goal something?}

% \at{start with deterministic, then talk about probabilities, then constratin}
% \ml{let's not call it mission objectives... perhaps don't need to have a subsection here}\at{But we have to formally define the probability of collision before we can state the constraint in the problem formulation?}
% \ml{what I meant is that we don't need to have a subsection... it can be part of the main section.}



% We consider the task of navigating to a goal location while avoiding collision with obstacles and other agents. In a deterministic setting, where the state of the robot is known, this has been addressed by numerous sampling-based algorithms, \at{cite a few}. In each of these the projection of the body of the robot onto the workspace is always known, so checking for collision only involves checking known polyhedra for intersections. When the system dynamics are uncertain, as in \eqref{eq: dynamics model}, accomplishing the task becomes probabilistic. We can define the probability of reaching the goal, colliding with obstacles, and colliding with other agents in terms of the distributions across the agents' states.  

Each robot $i$ is assigned a goal region $\X_{G}^i\subset \X^i$  in its state space. 
The workspace contains $ |\W_O| = N_O$ static obstacles and $N_A$ robots (moving obstacles).  The motion planning task is to compute a plan for every robot to reach its goal and avoid collisions with all (static and moving) obstacles.  In a deterministic setting, these objectives (reaching goal and obstacle avoidance) are binary.  In a stochastic setting, they are probabilistic, which we define below.  

The probability that robot $i$ is in the goal region at time $k$ is given by 
\begin{equation}
    \label{eq:prob goal}
    P^{i_k}_G = P(x^i_k \in \X^i_G) = \int_{\X^i_G} b(x_k^i)(s) ds,
\end{equation}
where $b(x^i_k)(s)$ is the distribution $b(x^i_k)$ evaluated at state $s \in \X^i$.
For collision probability, we consider static and moving obstacles separately.  Let $\X^i_O \subseteq \X^i$ be the set of static obstacles in the robot $i$'s state space.  It is constructed from $\W_O$ (and the limits on the velocity terms).  Then, similar to probability of goal, the probability of collision of robot $i$ at time step $k$ with the static obstacles is
\begin{equation}
    \label{eq:prob collision}
    P^{i_k}_O = P(x^i_k \in \X^i_O) = \int_{\X^i_O} b(x_k^i)(s) ds.
\end{equation}
% \noindent

The probability of one robot colliding with another is more difficult to define because we must account for uncertainty in both robot bodies. To describe the collision states, it is best to consider the composed space of the two robots.  That is, for robot $i$ and $j$, $i \neq j$, define the composed state as $x^{ij} = (x^i, x^j) \in \X^i \times \X^j = \X^{ij}$.  The set of points that correspond to the two robots colliding is described by
\begin{multline*}
    \X^{ij}_\coll = \{x^{ij}=(x^i, x^j) \mid \proj{\W}(x^i) \cap \proj{\W}(x^i) \\
    \forall x^i \in \X^i, \; \forall x^j \in \X^j \}.
\end{multline*}
The probability of collision is then given by
\begin{align}
    P^{ij_k}_{\coll} &= P((x^i_k,x^j_k) \in \X^{ij}_\coll) = \int_{\X^{ij}_\coll} b(x_k^{ij})(s) ds \nonumber \\
    &= \int_{(s^i,s^j) \in \X^{ij}_\coll} \hspace{0 mm} b(x_k^{i})(s^i) b(x_k^{j})(s^j) ds^i ds^j,
    \label{eq:prob robot-robot collision}
\end{align}
where the equality in the second line is due to independence of the robots' uncertainties from each other.

In this work, we use a chance constraint to bound the safety (violating) probabilities.  That is, given a minimum probability of safety $p_\safe$, we want to ensure that probability of collision (with static and moving obstacles) at every time point is less than $1-p_\safe$, and the probability of ending in the goal is at least $p_\safe$.  

% While obstacles can be given in the state space $\X$, we assume these are linear constraints (such as velocity limits), that are easy to evaluate. Therefore, we focus our effort in this work on obstacles solely defined in workspace. A discrete trajectory of length $T^i$ agent $i$'s states is given by $X_{0:T}^i=(x_0^i, x_1^i, ..., x_T^i)$. We define the set of agent $i$ states for which the projection of its body intersects with an obstacle as $\X_{coll}^i=\{x^i \mid \proj{\W}(x_k^i))\cap\W_O\}$. 

% The probability of agent $i$ colliding with an obstacle at any time step $k$ in its trajectory is given by:
% \begin{equation}
%   P(\proj{\W}(x_k^i)) \cap \X_O) = \int_{\X_{coll}^i} p(x_k^i) dx,
% \end{equation}
% where $p(x_k^i)$ defines the probability distribution across agent $i$'s states at time step $k$. Note this probability describes the a priori distribution across states, and, while related, differs from the online estimate $b_k^i$ maintained by the KF. The method of obtaining this a priori distribution is discussed in Section \ref{sec:GBT}. The probability of agent $i$ terminating within its goal region is given by:
% \begin{equation}
%     P(x_T^i\in \X_{G}^i) = \int_{\X_{G}^i} p(x_T^i) dx
% \end{equation}


% The probability of one robot colliding with another is more difficult to define, because we must account for uncertainty in both robot bodies. 
% Consider a combined c-space representation for a pair of agents, $x_k^ij=(x_k^i, x_k^j)$. The probability at each point in this space (associated with a realization of paired states) is $p(x_k^i,x_k^j)$. Because we assumed the agents are independent we have $p(x_k^i,x_k^j)=p(x_k^i)p(x_k^j)$. We define the set of c-space points that result in collision as $\C_{coll}^{ij}=\{(x_k^i,x_k^j) \mid \proj{\W}(x_k^i) \cap \proj{\W}(x_k^j)\}$. Therefore, the probability of collision between agents $i$ and $j$ is given by:
% \begin{equation}
%     \label{eq:prob-coll}
%     P(\proj{\W}(X_k^i)) \cap \proj{\W}(X_k^j)) = \int_{\mathcal{C}_{coll}^{ij}} p(x_k^i)p(x_k^j) dx
% \end{equation}

% Note that the integration area $\C_{coll}^{ij}$ can be defined a priori, it is only the distribution $p(x_k^i,x_k^j)$ that changes with time. However, this area is still difficult to define, and very cumbersome to integrate. We address this in two ways. The fiest is the main contribution of this paper, and involves various methods to simplify this integral for efficient collision checking. The second is to consider only checking some constraint on the probability of collision, rather than calculating the actual probability of collision. Using chance constraints is a well established technique for planning with uncertain sysyems, \at{cite some things}. For each agent, we seek to constrain the point-wise probability of collision with obstacles an other agents, as well as constrain the probability of terminating in the desired goal region.


\subsection{Chance-Constrained MRMP Problem}
% \jk{Maybe put this in its own environment so we can reference ``Problem 1" throughout the paper.}
Given $N_A$ robots with uncertain dynamics in \eqref{eq: dynamics model}, noisy measurements in \eqref{eq: measurement model}, and controller \eqref{eq:feedbackController} in workspace $\W$ with static obstacles $\W_O$, 
a set of initial distributions $\{x_0^i = \N(\hat{x}_0^i,\Sigma^i_0)\}_{i=1}^{N_A}$, 
a set of goal regions $\{\X_G^i\}_{i=1}^{N_A}$, and a safety probability threshold $p_\safe$, for each robot $i \in \{1,\ldots,N_A\}$, compute a motion plan $(\check{U}^i, \check{X}^i) = ((\check{u}_0^i, \check{u}_1^i, \ldots, \check{u}_{T-1}^i), (\check{x}_0^i, \check{x}_1^i, \ldots, \check{x}_T^i))$ 
such that 
$\check{x}_0^i = \hat{x}_0^i$,
\begin{subequations}
    \begin{align}
        % P(\B^i \cap \X_O) \leq 1-p_\safe, \forall k=1,..,T, i=1,...,N_A \\
        % P(x_T^i\in \X_{goal}^i)\geq p_{\safe}, \forall i=1,...,N_A \\
        % P(\proj{\W}(X_{0:T}^i)) \cap \proj{\W}(X_{0:T}^j)) \leq 1-p_{\safe}, \\
        % \forall i\neq j, \forall t\in [0,max(T^i,T^j)]    
        \label{eq:coll cc}
        &P^{i_k}_O + \sum_{j=1, j \neq i}^{N_A} P^{ij_k}_{\coll} \leq 1-p_\safe && \forall k\in \{1,..,T\} \\
        \label{eq: goal cc}
        &P^{i_T}_G \geq p_{\safe} &&
    \end{align}    
\end{subequations}
where probability terms $P^{i_k}_O$, $P^{ij_k}_{\coll}$, and $P^{i_T}_G$ are given in \eqref{eq:prob collision}, \eqref{eq:prob robot-robot collision}, and \eqref{eq:prob goal}, respectively.
