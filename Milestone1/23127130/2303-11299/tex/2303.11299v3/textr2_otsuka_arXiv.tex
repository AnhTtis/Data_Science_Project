%%  
%%   This file is part of the APS files in the REVTeX 4 distribution. 
%%   Version 4.0 of REVTeX, August 2001 
%% 
%%   Copyright (c) 2001 The American Physical Society. 
%% 
%%   See the REVTeX 4 README file for restrictions and more information.
% 
% This is a template for producing manuscripts for use with REVTEX 4.0
% Copy this file to another name and then work on that file.
% That way, you always have this original template file to use.
% 
% Group addresses by affiliation; use superscriptaddress for long
% author lists, or if there are many overlapping affiliations.
% For Phys. Rev. appearance, change preprint to twocolumn
% Choose pra, prb, prc, prd, pre, prl, prstab, or rmp for journal
%  Add 'draft' option to mark overfull boxes with black boxes
%  Add 'showpacs' option to make PACS codes appear
%  Add 'showkeys' option to make keywords appear
%\documentclass[aps,prl,preprint,groupedaddress]{revtex4}
%\documentclass[aps,prl,preprint,superscriptaddress]{revtex4}
%\documentclass[aps,prl,twocolumn,superscriptaddress,nobibnotes]{revtex4-1}
%% arXiv  
\documentclass[aps,prx,twocolumn,superscriptaddress,showpacs,longbibliography]{revtex4-2}
%%\documentclass[aps,prx,twocolumn,superscriptaddress,showpacs,longbibliography,linenumbers]{revtex4-2}
% You should use BibTeX and apsrev.bst for references
% Choosing a journal automatically selects the correct APS
% BibTeX style file (bst file), so only uncomment the line
% below if necessary.
%\bibliographystyle{apsrev}

% packages
\usepackage{amsmath,amssymb}
%\usepackage[dvips]{graphicx}
%\usepackage[dvipdfmx, draft]{graphicx}
\usepackage{graphicx}
\usepackage{color} 
\usepackage{txfonts} 

%%%
\usepackage{ulem}
%%%

%\usepackage{CJK}
\usepackage{xspace}  
\usepackage{multirow}
\usepackage{dcolumn} %centers to the decimal point in tables using ``d''
%Show the citations within the text

%\usepackage{pstricks}
%\usepackage{psfrag} 
%\let\origcite=\cite
%\renewcommand{\cite}[2][]{\rput[b](0,1.7ex){\tiny\red #2}\origcite[#1]{#2}}
%\usepackage{cite}
\usepackage{booktabs}

%\renewcommand{\familydefault}{\sfdefault}
%\usepackage{helvet}

% simple new shortcuts 
\usepackage{xspace}  
% for simplification of typing
\newcommand{\ts}{\textsuperscript} 
\newcommand{\etal}{\textit{et al.}\xspace}

\newcommand{\eb}{\ensuremath{e^{2}}b\ensuremath{^{2}}\xspace}
 
\newcommand{\kn}[1]{{\footnotesize (#1)}}   % Kanji-name (smaller font)
% for exp. observables.
\newcommand{\etwop}{\ensuremath{E(2^+_1)}\xspace}
\newcommand{\efourp}{\ensuremath{E(4^+_1)}\xspace}
\newcommand{\stwop}{\ensuremath{2^+_1}\xspace}
\newcommand{\sfourp}{\ensuremath{4^+_1}\xspace}
\newcommand{\eratio}{\ensuremath{E(4^+_1)/E(2^+_1)}\xspace}
\newcommand{\eratior}{\ensuremath{R_{4/2}}\xspace}
\newcommand{\etwopt}{\ensuremath{2^+_1 \rightarrow 0^{+}_{\mathrm{gs}}}\xspace}
\newcommand{\efourpt}{\ensuremath{4^+_1 \rightarrow 2^+_1}\xspace}
\newcommand{\beupl}{\ensuremath{B(E2;~0^+_{\mathrm{gs}}\rightarrow2^+_1)}\xspace} % long B(E2)
\newcommand{\beup}{\ensuremath{{B(E2)\!\!\uparrow}}\xspace}  % short B(E2)

%%prevent hyphenation
%\hyphenation{}

\begin{document}

% Use the \preprint command to place your local institutional report
% number in the upper righthand corner of the title page in preprint mode.
% Multiple \preprint commands are allowed.
% Use the 'preprintnumbers' class option to override journal defaults
% to display numbers if necessary 
%\preprint{}

%%%%   T I T L E   of  paper   %%%%
\title{Prevailing Triaxial Shapes in Heavy Nuclei Driven by Nuclear Tensor Force}

% \affiliation can be followed by \email, \homepage, \thanks as well.

\newcommand{\ariken}{      \affiliation{RIKEN Nishina Center, 2-1 Hirosawa, Wako, Saitama 351-0198, Japan}}
\newcommand{\aut}{         \affiliation{Department of Physics, The University of Tokyo, 7-3-1 Hongo, Bunkyo, Tokyo 113-0033, Japan}}
\newcommand{\acns}{        \affiliation{Center for Nuclear Study, The University of Tokyo, 7-3-1 Hongo, Bunkyo, Tokyo 113-0033, Japan}}
\newcommand{\akul}{         \affiliation{KU Leuven, Instituut voor Kern- en Stralingsfysica, 3000 Leuven, Belgium}}
\newcommand{\atsuk}{         \affiliation{Center for Computational Sciences, University of Tsukuba, 1-1-1 Tennodai, Tsukuba, Ibaraki, 305-8577, Japan}}
\newcommand{\ajar}{         \affiliation{Advanced Science Research Center, Japan Atomic Energy Agency, Tokai, Ibaraki 319-1195, Japan} }

\newcommand{\aemp}{\email{otsuka@phys.s.u-tokyo.ac.jp}}  % author email
 
\author{T.~Otsuka}   \aemp  \aut \ariken \ajar \akul%\aemt  $^*$
\author{Y.~Tsunoda}    \acns  \atsuk
\author{Y.~Utsuno}    \ajar \acns
\author{N.~Shimizu}    \acns \atsuk
\author{T.~Abe}   \ariken
\author{H.~Ueno} \ariken

\date{\today}

\begin{abstract}     
Virtually any object can rotate: the rotation of a rod or a linear molecule appears evident, but a number of objects, including a simple example of H$_2$O molecule, are of complex shapes and rotate in complicated manners. While the rotation provides various intriguing physics cases, the primary picture for atomic nuclei was simple.  Rotational bands have been observed for many nuclei, and their basic picture is considered to have been established in 1950’s. We, however, show that this traditional picture is superseded with a novel picture arising from basic characteristics of the nuclear forces.  In the traditional view, as stressed by Aage Bohr in his Nobel lecture, most of heavy nuclei are like axially-symmetric prolate ellipsoids (i.e., with two shorter axes of equal length), rotating about one of the short axes, like a rod.   In the present picture, however, in many cases, the lengths of these three axes are all different, called triaxial.  The triaxial shape yields more complex rotations, which actually reproduce experimental data as shown by state-of-the-art Configuration Interaction calculations, 
on supercomputers.  The key to differentiate the two pictures is the nuclear tensor force, which is known to produce the shell evolution in exotic nuclei, a major agenda of Rare-Isotope physics. We now demonstrate that the same tensor force generates, in many nuclei, the triaxiality, fading the prolate-ellipsoid dominance away.  \textcolor{black}{Experimental findings obtained decades ago are revisited with unbiased eyes, and are shown to be supportive of the present idea.}  The tensor force is a direct consequence of one $\pi$ meson exchange between nucleons, and the present finding is regarded as the first explicit or visible case that elementary particles directly affect nuclear shapes.  The importance of the explicit tensor force is in the same line as Weinberg's modeling of nuclear forces.  This feature makes the new picture robust, and may cast challenges for other many-body systems \textcolor{black}{having forces similar to the tensor force.}  Substantial impacts on superheavy nuclei and fission are anticipated.  This study sheds lights on the earlier suggestion of dominant triaxiality by Davydov, a Ukrainian physicist.
\end{abstract}  

\maketitle

\section{Introduction Referring to Molecular Rotation} 

The atomic nucleus comprises $Z$ protons and $N$ neutrons, which are collectively called nucleons.   With the mass number $A$=$Z$+$N$, a given nucleus is labelled as $^A$X where X denotes the element, e.g., $^{166}$Er for erbium-166 ($Z$=68).    The nucleus, as an assembly of many nucleons, exhibits a clear surface with a certain shape.  In many nuclei, the shape is an ellipsoid, which rotates \cite{bohr_mottelson_book2,bohr_1952,rainwater_1950}.  There has been the conventional text-book picture for nuclear shape and rotation, as stressed by Aage Bohr in his Nobel-prize lecture in 1975 \cite{bohr_nobel}.  The present work, however, depicts a different picture based on recent studies that were infeasible in earlier days.  Before an in-depth description of the major outcome, we overview the rotation of molecular systems, because 
it is easier to view.
Figures~\ref{fig:image}{\bf a-b} schematically display the rotation of diatomic (O$_2$) and triatomic (H$_2$O) molecules \cite{atkins}.   The O$_2$ molecule is like a rod, and rotates about an axis perpendicular to the axis connecting the two O atoms (see Fig.~\ref{fig:image}{\bf a}).  It cannot rotate about the axis connecting two atoms, because the quantum state does not change by such rotation.  

%%%%%%%%%%%%  FIGURE 1  %%%%%%%%%%%%%
% Fig1: Classical cluster models

\begin{figure*}[tb]
  \centering
  \includegraphics[width=16cm]{Fig/Fig1.eps}
    \caption{ {Schematic illustrations of the rotations of molecules and atomic nuclei.}
    {\bf a} O$_2$ molecule.    {\bf b} H$_2$O molecule.
    {\bf c} prolate and {\bf d} triaxial nuclear shapes, with associated rotations $\vec{R}$ and $\vec{K}$.
    {\bf e} Legend. {\bf f, g}   Principal axes of prolate and triaxial ellipsoids.
} 
  \label{fig:image}  
\end{figure*}  

%%%%%%%%%%%%%%%%%%%%%%%%%%%%%%%%
 
The situation with the H$_2$O molecule is more complex.  As shown in Fig.~\ref{fig:image}{\bf b}, %in the view of classical mechanics, 
this molecule can rotate about more than one axis.  
%Although the rotational motions are different in the quantum mechanics, the appearance of multiple rotational axes remains.  
Thus, if the molecule has a complex configuration of atoms, its rotational motion occurs about multiple axes.   

A nuclear analogy to the O$_2$ molecule can be found in a nucleus deformed to a prolate ellipsoid (see Fig.~\ref{fig:image}{\bf c}), where the ellipsoid is stretched in the vertical axis in the paper plane (side view), with a nuclear quantum state invariant under the rotation about the vertical axis (top view).  
This invariance is called {\it axial-symmetry}, and the vertical axis here is called the {\it symmetry axis}.  In quantum mechanics, no rotational motion arises about the symmetry axis because the state does not change, like di-atomic molecule (see  Fig.~\ref{fig:image}{\bf a}).  On the other hand, the ellipsoid can rotate about an axis perpendicular to the symmetry axis (side view).  

The axial-symmetry can be broken in reality.  The cross section of the ellipsoid then becomes an ellipse as shown in the top view of Fig.~\ref{fig:image}{\bf d}.  The rotation can now occur about the axis perpendicular to this cross section, as seen in the right drawing of this panel.  Although details are different, this is basically similar to the H$_2$O molecule case: a complex object can rotate in multiple ways.  

%%%  Nuclear ellipsoidal deformation  %%%%%%
\section{Nuclear Ellipsoidal Deformation}

Assuming an ellipsoidal deformation of nuclear shape, it is defined concretely with axes $R_x, R_y$ and $R_z$ shown in Fig.~\ref{fig:image}{\bf e}, where the standard convention for the axis lengths $R_z \ge R_x \ge R_y$ is taken.  The ellipsoid is viewed from two different angles, A and B. The prolate ellipsoid is shown in Fig.~\ref{fig:image}{\bf f} where $R_x$=$R_y$ holds.  The ellipsoid without the axial-symmetry is shown in Fig.~\ref{fig:image}{\bf g}, where the three axes take different lengths.  This feature is called {\it triaxial}.   The ellipsoid shows quadrupole moments: $Q_0$ $\propto \langle 2 z^2-x^2-y^2\rangle$ and $Q_2$ $\propto \langle x^2-y^2\rangle$, where $\langle  \rangle$ implies integral inside the surface with a uniform density.  The prolate shape is characterized by positive $Q_0$ and vanishing $Q_2$, whereas the triaxiality by finite $Q_2$.  We show, in this paper, the unexpected preponderance of the triaxiality, its consequences in nuclear rotation and its robustness rooted in nuclear forces.  
\textcolor{black}{Although the oblate ellipsoid $R_z = R_x > R_y$ is another possible case, it emerges at rather high excitation energies in the nuclei to be discussed.  Consequently, the oblate shape will appear in limited cases in this article.
}
%%%%%%%%%%%%  FIGURE 2  %%%%%%%%%%%%%

\begin{figure*}[tb]
  \centering
  \includegraphics[width=16cm]{Fig/Fig2.eps}
    \caption{ {Overview of the 2$^+_1$ states.}
    {\bf  a} Mechanism to drive ellipsoidal deformation. 
    {\bf  b} Examples of spherical and deformed cases with the rotational level energies.
    {\bf  c} All measured lowest 2$^+$ level energies of even-even nuclei as of the year 2022, based on the data taken from NuDat 3.0 \cite{nudat3}.
    } 
  \label{fig:2+}  
\end{figure*}  

%%%%%%%%%%%%%%%%%%%%%%%%%%%%%%%

The mechanism towards deformed nuclear shapes is schematically depicted in Fig.~\ref{fig:2+}{\bf a}, where short-range attractive forces between nucleons produce more binding energies for  nucleons closely configured (lower object) than for nucleons sparsely spread with longer mutual distances (upper object).  The Jahn-Teller effect \cite{jahn_1937} also contributes to the present formation of the ellipsoidal shape.  From now on, the nucleus is treated as a multi-nucleon quantum system, and the quadrupole moments are calculated from their wave functions.  The nucleus is then connected to the afore-mentioned ellipsoid with the same quadrupole moments.  

The present attraction is much stronger between a proton and a neutron than between two neutrons or between two protons.  This implies that the deformation becomes strong if there are sufficiently large numbers of active (or valence) protons and active neutrons, outside the inert core (see Fig.~\ref{fig:2+}{\bf a}).    On the other hand, if the protons (neutrons) form a closed shell, as the closed shell is spherical, the neutrons (protons) also form a spherical shape.

Different nuclear shapes result in visible differences in observables.  
Figure~\ref{fig:2+}{\bf c} exhibits the global systematics of the level energies of the lowest state of spin/parity $J^{P}$=2$^+$ for even-$Z$ and even-$N$ ({\it even-even}) nuclei, on top of the $J^{P}$=0$^+$ ground state.  
For doubly-magic nuclei, the excitation from the ground state can be made by moving a nucleon over a magic gap.  This requires a large amount of energy, meaning a high excitation energy.
There are high spikes in Fig.~\ref{fig:2+}{\bf c} with the excitation energies more than 2 MeV (see green open circles).  These spikes correspond to conventional ($N$=8, 20, 28, 50, 82, 126) or new ($N$=16, 32, 34, 40) magic numbers.
Beyond $N \sim$ 40, most of the points (blue closed circles) show low excitation energies down to several tens of keV.
An example of such low excitation energies is shown for the nucleus, $^{166}$Er, in the yellow-shaded part of Fig.~\ref{fig:2+}{\bf b}, which displays the lowest states of $J^{P}$=2$^+$, 4$^+$ and 6$^+$.  Next to these experimental levels, the rotational excitation energy, $E_x(J) = \alpha J(J+1)$, is shown with the value of $\alpha$ adjusted to the measured 2$^+$ level energy, in a remarkable agreement for $J^{P}$=4$^+$ and 6$^+$.  
We thus consider that a strongly deformed ellipsoid like the lower part of panel {\bf a}, rotates with $J^{P}$=2$^+$, 4$^+$ and 6$^+$.  Similar rotational bands have been observed in many heavy nuclei.
The systematics shown in Fig.~\ref{fig:2+}{\bf c} exhibits many nuclei with the first 2$^+$ state, denoted as 2$^+_1$, at a very low excitation energy.  This implies large moments of inertia indicative of strong ellipsoidal deformation.  

Figure~\ref{fig:2+}{\bf b} displays level energies of $^{124}$Sn, where the $Z$=50 proton closed shell results in a spherical shape for its ground state.  Its $J^{P}$=2$^+_1$ level is lying high, partly because of the absence of a mechanism like Fig.~\ref{fig:2+}{\bf a}.  This feature is depicted by closed green circles in Fig.~\ref{fig:2+}{\bf c}, forming a minor fraction of the points.  

The strong ellipsoidal deformation (blue circles in Fig.~\ref{fig:2+}{\bf c}) is thus an important and dominating trend of heavy nuclei.  
While Fig.~\ref{fig:image}{\bf c, d} display two cases of the shape deformation, the axially symmetric shape (panel {\bf c}) has been considered to occur in most cases, since Bohr and Mottelson argued \cite{bohr_mottelson_book2,bohr_nobel}.  
This picture has been a textbook item over 70 years \cite{rowe_book,deshalit_book,ring_schuck_book}, and many works were made
on top of it, for instance, the fission mechanism \cite{nix_fissionshape,ichikawa_fission}.  

%%%%  R and K
\textcolor{black}{
The axially symmetric ellipsoid can rotate about one axis as shown in Fig.~\ref{fig:image}{\bf c}, with the corresponding angular momentum, $\vec{R}$.
This is nothing but the total angular momentum $\vec{J}$ (=$\vec{R}$).  Its magnitude, $J$,  and one of its components are conserved.  If the triaxial shape sets in, another rotation of the ellipsoid emerges, because of ellipse, as shown in the right part of Fig.~\ref{fig:image}{\bf d}, with its angular momentum, $\vec{K}$.  The total angular momentum becomes $\vec{J}=\vec{R}+\vec{K}$.  The conservation law for $\vec{J}$ is not applied to $\vec{R}$ or $\vec{K}$, implying that different components are superposed in the eigenstate.  However, in actual triaxial cases to be discussed below, the magnitude of $\vec{K}$, called $K$, seems to be approximately conserved or not mixed significantly, primarily due to larger excitation energies due to this rotation.  The phrases, $K$ rotation and $K$ quantum number, will appear in this sense. 
}

%We sketch this traditional picture first.


%%%%%%%%%%%%  FIGURE 3  %%%%%%%%%%%%%

\begin{figure*}[tb]
  \centering
  \includegraphics[width=12.5cm]{Fig/Fig3.eps}
    \caption{ {Schematic pictures of rotational motion and actual features.}
    {\bf  a} Conventional picture (prolate shape) and $\gamma$ vibration from it.
    {\bf  b} Present picture (triaxial shape).
    {\bf  c} Levels and E2 properties of $^{166}$Er compared to experimental data \cite{ensdf}.
    B(E2) values are in W.u., and spectroscopic electric quadrupole moments, Q, are in $e$ barn. 
      } 
  \label{fig:level}  
\end{figure*}  

%%%%%%%%%%%%%%%%%%%%%%%%%%%%%%%%

The surface tension of a liquid drop generally makes its shape spherical.  The nuclear shape was considered to be an equilibrium between this tension and the force causing the deformation {\it \`a la} Fig.~\ref{fig:2+}{\bf a}.   An axially-symmetric ellipsoid is then favored over a triaxial one, because of the constant curvature in the $xy$ plane (see Fig.~\ref{fig:image}{\bf f}).  
The rotation of the rigid body is described, for instance, in textbooks \cite{rowe_book,deshalit_book}.  
As a many-nucleon problem, Kumar and Baranger \cite{kumar_baranger1968} performed the ``pairing-plus-quadruple model'' calculation, as a then most advanced many-body calculation, and advocated the ``preponderance of axially symmetric shapes'' over many species of nuclei. 
A similar work was reported by Bes and Sorensen \cite{bes_sorensen1969}.   These works indicated that once many nucleons in many single-particle orbits of a mean potential coherently contribute, the axially-symmetric prolate deformation likely occur.  The SU(3) model is a symmetry based many-body approach, and favors the same feature \cite{elliott_1958a,elliott_1958b}, 
as well as its approximate extensions \cite{caurier_2005}.  
 
Figure~\ref{fig:level}{\bf a} shows a schematic picture of the prolate deformed ground state and its rotational excitations.   Figure~\ref{fig:level}{\bf a} includes another type of excitation, called $\gamma$ vibration \cite{bohr_mottelson_book2, bohr_nobel}.  This excitation stands for a vibrational excitation of the circle in Fig.~\ref{fig:image}{\bf c, f}, represented by $\gamma$ phonon.  The 2$^+_2$ state is then identified as the one-$\gamma$-phonon state on top of the deformed equilibrium \cite{bohr_mottelson_book2, bohr_nobel} (see Fig.~\ref{fig:level}{\bf a}).

Figure~\ref{fig:level}{\bf c} depicts measured excitation energies as well as electromagnetic decays and moments \cite{ensdf}, for $^{166}$Er as an example of observed deformed nuclei.  The decays are actually electric quadrupole (E2) transitions, and their strengths are expressed by squared E2 strengths called $B(E2;J^+\rightarrow J'^+)$.  The 2$^+_2$ state decays to the ground state through relatively strong transition with $B(E2;2^+_2\rightarrow 0^+_1)$ = 5.17$\pm$0.21 (W.u.) (see panel {\bf c}), where W.u. implies the single-particle (Weisskopf) unit \cite{bohr_mottelson_book1}.  Such somewhat large $B(E2)$ value was interpreted as an indication that the 2$^+_2$ state is a phonon excitation which is collective to a certain extent \cite{bohr_nobel}.  

%%%  Multi-nucleon Structure by Configuration Interaction Simulation  %%%%%%
\section{Multi-nucleon Structure by Configuration Interaction Simulation} 

We present a recent work on this subject from a modern viewpoint combined with state-of-the-art configuration interaction (CI) calculations with a realistic effective nucleon-nucleon ($NN$) interaction.  The CI calculation in nuclear physics is called {\it shell model}, and some introductory details are found in Appendix~\ref{Ap_shell}. 
The CI calculation here means Monte Carlo Shell Model (MCSM) \cite{mcsm_1995,mcsm_1998,mcsm_2001,mcsm_2012}, which enabled us to carry out CI calculations far beyond the limit of the conventional CI approaches, on contemporary challenges \cite{togashi_2016, leoni_2017, togashi_2018, marsh_2018, ichikawa_2019, taniuchi_2019, otsuka_2019, marginean_2020, tsunoda_2020, abe_2021, otsuka_2022_nature}.  
Some details of the MCSM are presented in Appendix~\ref{Ap_MCSM}. 
A large number of single-particle orbits are taken so that the ellipsoidal deformation can be described, 8 for protons and 10 for neutrons.  The nucleus $^{166}$Er is mainly discussed as an example, with 28 active protons and 28 active neutrons.  The present work goes far beyond the  earlier paper \cite{otsuka_2019}, \textcolor{black}{where the prevailing of the triaxiality and the fundamental robust mechanism were both untouched.  We will return to these points after all novel features are presented.}
%, by covering a wider range of nuclei, and by looking for future impacts.   
The present CI calculation is carried out by the most advanced methodology using the number-projected quasiparticle vacua for the MCSM basis vectors instead of Slater determinants, which can be specified as the Quasiparticle Vacua Shell Model (QVSM) \cite{shimizu_2021}.  

A prototype of the present Hamiltonian was obtained and used in the earlier work \cite{otsuka_2019}.  
The proton-neutron (interaction) part was obtained from the monopole-based-universal  interaction, $V_{\rm MU}$ \cite{otsuka_2010}, which was derived based on the shell-model and microscopic interactions \cite{otsuka_2010} and has been used in many studies, up to Hg isotopes \cite{marsh_2018}.  
The proton-proton and neutron-neutron interactions were fixed in Ref. \cite{otsuka_2019} mainly based on the microscopic interactions used in Ref. \cite{brown_2000_Pb}.  In the earlier work, the pairing correlations in wave functions can be underestimated because of the Slater-determinant expansion in the original MCSM, and its strengths were likely taken to be stronger, in order to compensate this possible underestimation in reproducing experimental data.  This problem vanishes in the present calculation to a large extent, and the pairing strengths are rescaled to be somewhat weaker.  The SPE values are also changed slightly.  \textcolor{black}{The basic outcomes of this earlier work \cite{otsuka_2019} remain in the present work, besides other new points.}

%%%%%%%%%%%%  FIGURE 4  %%%%%%%%%%%%%

\begin{figure*}[tb]
  \centering
  \includegraphics[width=16.5cm]{Fig/Fig4.eps}
    \caption{ {PES and T-plots for $^{166}$Er} 
    {\bf a.} Legend.   
    {\bf b} PES. 
    {\bf c} T-plots of ground and low-lying states for the red-square region of {\bf b}.
    All T-plots are independently obtained, despite resulting resemblances.
    {\bf d} T-plots for reduced crucial monopole interactions.
    {\bf e} T-plots for monopole-frozen Hamiltonian.
    } 
  \label{fig:tplot}  
\end{figure*}  

%%%%%%%%%%%%%%%%%%%%%%%%%%%%%%%
 
Figure~\ref{fig:level}{\bf c} depicts calculated level energies and E2 properties of $^{166}$Er, in a good agreement with experiment, including both large and small values of $B(E2)$.  The present value of $B(E2;2^+_2\rightarrow 0^+_1)$ is 5.3 W.u. in a salient agreement with experiment, and urges us to look into underlying multi-nucleon structure of the eigenstates.  

For this purpose, deformation parameters $\beta_2$ and $\gamma$ are introduced \cite{bohr_mottelson_book2} to indicate how the sphere with the radius $R_0$ is deformed.  The $\beta_2$ parameter (typically $\sim$0.3) denotes the degree of deformation.  The $\gamma$ parameter is an angle related to the ratio between $R_x$ and $R_y$.  They are mutually linked as (see Fig.~\ref{fig:image}{\bf e}), 
\begin{equation}
R_z =\{1+0.63 \beta_2 \cos \gamma\} \, R_0,
\label{eq:R1}
\end{equation}
\begin{equation}
R_x =  \{1+0.63 \beta_2 \sin(\gamma-30^\circ)\} \, R_0, 
\label{eq:R2}
\end{equation}
\begin{equation}
R_y = \{1-0.63 \beta_2 \cos(60^\circ-\gamma)\} \, R_0.
\label{eq:R3}
\end{equation}
Prolate shapes ($R_x$=$R_y$) correspond to $\gamma$=0, while triaxial shapes emerge for 0$^{\circ} < \gamma < 60^{\circ}$.  
Usually, $\beta_2 \ge 0$ and $0^{\circ} \le \gamma \le 60^{\circ}$ are considered, implying
$R_y \le R_x \le R_z$.   
The $\beta_2$ and $\gamma$ parameters for a quantum state can be obtained by evaluating quadrupole moments $Q_0$ and $Q_2$ \cite{utsuno_2015,otsuka_2022}, and by calculating $R_{x,y,z}$ values of the afore-mentioned uniform ellipsoid having the same $Q_0$ and $Q_2$ values.  

A constrained mean-field (more concretely Hartree-Fock-Bogoliubov) calculation is carried out by imposing various values of $(\beta_2, \gamma)$ as constraints.  The obtained energy expectation value is displayed by contour plot in Fig.~\ref{fig:tplot}{\bf b}, with a legend in Fig.~\ref{fig:tplot}{\bf a}.  This plot is usually called the potential energy surface (PES).  The same Hamiltonian as the present CI, or MCSM, calculations is consistently used.  The PES in Fig.~\ref{fig:tplot}{\bf b} suggests that the bottom part of the PES spreads around a finite $\gamma$ angle, and the precise inspection actually indicates that the minimum is not at $\gamma$=0 but around $\gamma \sim$ 9$^{\circ}$.  This contradicts the prolate preponderance hypothesis believed over seven decades.  For $\gamma$=9$^{\circ}$, the values of $R_{x,y,z}/R_0$ are 0.93, 0.88 and 1.19, respectively. 

The wave functions of the MCSM are expanded by so-called MCSM basis vectors, which are presently number-projected quasiparticle vacua \cite{shimizu_2021}. 
These basis vectors are selected from a large group of candidates generated stochastically, so that the maximum possible lowering of the energy eigenvalues of interest can be obtained.  The selected basis vectors are further optimized by variational procedures \cite{mcsm_2001}.  Some details of the computational methodology are presented in Appendix~\ref{Ap_MCSM}.
Each of basis vectors has intrinsic quadrupole moments, from which the corresponding $(\beta_2, \gamma)$ value is obtained.  The individual basis vector can be represented by a circle pinned down on the PES according to this $(\beta_2, \gamma)$ value, which now serves as a ``partial labelling'' of the basis vector.   
The importance of each basis is depicted by circle's area proportional to the overlap probability with the eigenstate.  This visualization is called T-plot \cite{tsunoda_2014}, and turned out to be very useful (see Appendix~\ref{Ap_MCSM}).

\textcolor{black}{If $K$ quantum number is conserved, E2 matrix elements follow certain regularities \cite{bohr_mottelson_book2}.  The actual values indeed follow them to a good extent.  For instance, the spectroscopic quadrupole moment vanishes for $K$=2 and $J$=3, and this is the case for the 3$^+_1$ state.  Combining such properties with strong E2 transitions expected within individual rotational bands, the band structure is assigned as $K$$\sim$0 for the 0$^+_1$, 2$^+_1$, 4$^+_1$ states forming the so-called ground band, $K$$\sim$2 for the 2$^+_2$, 3$^+_1$, 4$^+_2$ states forming the so-called $\gamma$ band, and $K$$\sim$4 for the 4$^+_3$ state.
The approximate conservation of $K$ quantum number is well known feature, for instance \cite{polishK_2021}, and is discussed for $^{166}$Er in Ref. \cite{tsunoda_2021}.
}

Figure~\ref{fig:tplot}{\bf c} displays the T-plots for the 0$^+_1$, 2$^+_1$, 4$^+_1$ states (ground-band members), and the 2$^+_2$, 3$^+_1$, 4$^+_2$ states ($\gamma$-band members), as well as the 4$^+_3$ state.  All of them exhibit remarkably similar T-plot patterns, {\it i.e.}, the mean position and fluctuation.  This similarity suggests a rather common triaxiality among these states, which is consistent with the picture shown in Figure~\ref{fig:level}{\bf b}: all these states can be generated, to a good extent, by rotating a common triaxial ellipsoidal state in multiple ways.   

The mean value of $\gamma$ is calculated, \textcolor{black}{by the method described in \cite{otsuka_2022},} to be 8.2$^{\circ}$ 
\textcolor{black}{for each member of the ground band, 9.1$^{\circ}$ for each member of the $\gamma$ band, and 9.5$^{\circ}$ for the 4$^+_3$ state, where small variations within a band (beyond two digits) are ignored. } 
These values are almost the same, and in particular, are almost exactly so within a band.  This fact suggests that a fixed triaxial shape governs $^{166}$Er, \textcolor{black}{and the fine details are consistent with the band assignment of individual states.}   
This property holds up to the 4$^+_3$ state, where the CI calculation stopped due to the computer resource.   \textcolor{black}{The value of $\gamma$ can be evaluated within the Kumar invariant approach \cite{kumar_1972}, which shows a similar value.}

%%%  Novel picture of the $\gamma$ band  %%%%%%
\section{Novel picture of the $\gamma$ band \label{gamma band}} 

The mean value of the angle $\gamma$ is about 1$^{\circ}$ larger for the $\gamma$ band than for the ground band.  This difference reflects structure changes from the ground band to the $\gamma$ band, and its impact on the relative energy between these two bands is of great interest.  While the energy of the 0$^+_1$ state is calculated with many basis vectors (see Appendix~\ref{Ap_MCSM} for details), the first basis vector, denoted $\xi_0$ here ($\phi_1$  in eq.~(\ref{eq:mcsm_psi}) for 0$^+_1$), is most important among them: it indeed shows a large overlap probability with the final solution, 89\% after the projections and normalization.  
We next look at the 3$^+_1$ member of the $\gamma$ band.  This state has no counter part in the ground band and should well represent features of the $\gamma$ band.  We first calculate its energy by using $\xi_0$ alone, similarly to the 0$^+_1$ state.   Although the obtained energy of 3$^+_1$ state is not too bad, its difference from the 0$^+_1$-state value is 1.27 MeV, which is notably larger than the 3$^+_1$ excitation energy by the present MCSM calculation shown in Fig.~\ref{fig:level}{\bf c}, that is 0.83 MeV.  However, instead of $\xi_0$, by using another state, $\xi_3$, determined by solely lowering the energy of the 3$^+_1$ state, the obtained 3$^+_1$ state comes down from 1.27 MeV to 0.76 MeV above the 0$^+_1$ state obtained from $\xi_0$.  \textcolor{black}{This difference is apparently closer to the \textcolor{black}{calculated 3$^+_1$ level energy (at 0.83 MeV in Fig.~\ref{fig:level}{\bf c}) mentioned just above.}}  The overlap probability with the final solution becomes as high as 88\%.  Thus, the most optimum basis vector is slightly different between the 0$^+_1$ and 3$^+_1$ states.  
By including additional basis vectors into the actual MCSM calculation, the energies of the 0$^+_1$ and 3$^+_1$ states come down \textcolor{black}{approximately in parallel}, and their difference swiftly reaches the value shown in Fig.~\ref{fig:level}{\bf c}.  

The value of $\gamma$ is 8.5$^{\circ}$ and 9.8$^{\circ}$ for $\xi_0$ and $\xi_3$, respectively, being similar to the mean $\gamma$ values of the ground and $\gamma$ bands.  The analysis above indicates that this difference lowers the $\gamma$-band excitation energy by $\sim$0.5 MeV.  Once this lowering occurs, additional basis vectors do not substantially change the $\gamma$-band excitation energy, \textcolor{black}{ending up} with a 0.07 MeV shift in the final solution.      
These features suggest that the $\gamma$-band excitation energy is substantially lowered due to the enlargement of the triaxial deformation.
We stress that this effect is due to $NN$ interactions incorporated into the MCSM calculation.  

%The rotation shown in the lower-right corner of Fig.~\ref{fig:image}{\bf d} is interpreted to approximately carry spin 0, 2 and 4, respectively, for the (0$^+_1$, 2$^+_1$, 4$^+_1$),  (2$^+_2$, 3$^+_1$, 4$^+_2$), and 4$^+_3$ states.  It corresponds to so-called $K$ quantum number.  The vector coupling with the other rotation in Fig.~\ref{fig:image}{\bf d} forms the total spin of the nucleus.   
%The $K$ quantum number is approximately conserved in the present case 
%as the actual $\gamma$ value is well below 15$^{\circ}$.  

\textcolor{black}{Considering $K$ rotation,} 
the enlargement of the $\gamma$ angle from the ground to the $\gamma$ band can be interpreted as a kind of the centrifugal effect that more distorts the ellipse in the right-lower part of Fig.~\ref{fig:image}{\bf d}.  The deformation along the longest (or vertical) axis in the left part of Fig.~\ref{fig:image}{\bf d} is considered to be %more stable 
\textcolor{black}{quite rigid against $R$ rotation due} to a strong binding mechanism, and creates practically perfect rotors not only within the ground band but also within the $\gamma$ band.  The deformation perpendicular to this axis \textcolor{black}{(or the value of $\gamma$)}, however, appears to increases for \textcolor{black}{``faster $K$ rotation''}.  Because of this ``stretching'', the excitation energies associated with the \textcolor{black}{(approximate)} $K$ quantum number are lowered due to enlarged moment of inertia, as seen in the experiment and also in the present MCSM calculation.  
\textcolor{black}{This stretching effect and its impact on the excitation energy of the $\gamma$ band are one of novel and crucial features, not known in the earlier work \cite{otsuka_2019}.
}
Consistently with the stretching picture, the $\gamma$ band depicts, as compared to the ground band, the weakening (enlargement) of the 0$^+$-pairing (2$^+$-pairing) correlation, as well as enhanced effect of the proton-neutron interaction, which is the origin of deformed shapes (see Fig.~\ref{fig:2+}{\bf a}) \cite{remark1}.

\textcolor{black}{The picture described above is confronted with the interpretation in terms of the $\gamma$ vibration raised \cite{bohr_1952} and stressed \cite{bohr_mottelson_book2,bohr_nobel} by A. Bohr.  The $\gamma$ vibration is a distortion of the circular cross section (view B) of Fig.~\ref{fig:image} {\bf f}, leading to the intuitive image of Fig.~\ref{fig:level} {\bf a}.  This interpretation contradicts the present result, as this cross section is not a circle but an ellipse as shown in Fig.~\ref{fig:image} {\bf f}.   The excitation from the ground state to the 2$^+_2$ state is considered, presently, to be rotational (see Fig.~\ref{fig:level} {\bf b}), with the intuitive understanding as the change of the $K$ quantum number from $\sim$0 to $\sim$2.  }

\textcolor{black}{We here stress that the terminology, $\gamma$ band, is used, in this work, only for referring to the rotational band built on a low-lying 2$^+$ state, excluding the 2$^+_1$ state, irrespectively of $\gamma$ vibration.   This band head is sometimes called the 2$^+_{\gamma}$ state in literatures.  As this nomenclature is confusing, it is avoided in this article.  
We just report that the present MCSM calculation shows no evidence or hint of a $\gamma$ vibration which is supposed to be built on a prolate (or oblate) ground state.  
}

\textcolor{black}{Microscopic investigation on the presence of the $\gamma$ vibration, without explicitly assuming an axially-symmetric ground state, has been very difficult.   As a related work, in \cite{delaroche_2010}, this question was indirectly argued, among other issues, from a broader microscopic manner with the Gogny interaction, a well-accepted mean-field model.  In this approach \cite{delaroche_2010}, the Bohr Hamiltonian was derived microscopically and was diagonalized.  It successfully described the excitation energies of the first 2$^+$ states of many nuclei.  The vast majority of second 2$^+$ states were considered as $\gamma$ vibrations, but their excitation energies are systematically overestimated \cite{delaroche_2010}, casting a mystery to date.  Any further survey for the $\gamma$ vibration will be of high interest, while 
low-energy vibrational modes from the deformed ellipsoid were extensively assessed from more general viewpoints in \cite{Sharpey-Schafer_2019,sun_2002}.
}

%%%%   DOUBLE gamma
\section{Double gamma phonon state}
\textcolor{black}{
The $\gamma$ vibration can be described by the so-called $\gamma$ phonon, having the $\gamma$-vibrational 2$^+$ state as $\gamma$ phonon state created on the prolate (or oblate) ground state as the zero-phonon state.  The double $\gamma$ phonon ($\gamma\gamma$) state can then be considered, and we now discuss such $\gamma\gamma$ 4$^+$ state.  
%There might be other 4$^+$ state(s) below this state, but we just focus on this.  
A simple estimation of its level energy is twice that of the $\gamma$ phonon 2$^+$ state, which turns out to be 2 $\times$ 0.79 MeV for $^{166}$Er.  Its experimental candidate state has been observed based on E2 decay to the 2$^+_2$ state, as it should be relatively strong due to one phonon annihilation.  The obtained excitation energy, however, appears to be substantially higher, 2.03 MeV (see \cite{ensdf,fahlander_1996,garrett_1997,tsunoda_2021}).}  
This discrepancy has attracted attentions. 

\textcolor{black}{The present theoretical 4$^+_3$ state is \textcolor{black}{$K$$\sim$4 member of almost the same triaxial shape.}  The triaxiality is somewhat enlarged (see Fig.~\ref{fig:tplot} {\bf c}) as a consequence of stretching effect  \textcolor{black}{with $K$ larger}.
%, as shown in the right part of Fig.~\ref{fig:image}{\bf d}  (see Ref. \cite{tsunoda_2021}). 
Its calculated excitation energy is 2.22 MeV, being closer to the observed value, 2.03 MeV \cite{ensdf,fahlander_1996,garrett_1997}.  The calculated value $B(E2; 4^+\rightarrow 2^+_2)$=10.4 W.u. is also in agreement with the experimental value 8 $\pm$ 3 W.u. \cite{ensdf}.  Thus, the calculated 4$^+_3$ state appears to correspond to the 4$^+$ state assigned experimentally as the $\gamma\gamma$ state, and the problem of too high excitation energy  disappears, with the $\gamma\gamma$ assignment replaced by the $K$$\sim$4 rotation.  
}

%%%%%  Davydov model
\section{Davydov model}
The model with rigid and triaxial shape was intensively discussed around 1958 by Davydov and his collaborators \cite{davydov1,davydov2}.   For $\gamma=$9$^{\circ}$, the ratio $B(E2;2^+_2\rightarrow 0^+_1)/B(E2;2^+_1\rightarrow 0^+_1)$ is 0.0238 in this Davydov model \cite{tsunoda_2021}, in agreement with the experiment and the present calculation (see Fig.~\ref{fig:level}{\bf c}).  As electromagnetic properties directly scan the shapes, the Davydov model appears to be consistent with the present scheme for the shapes.   Excitation energies are another story, however.  
The experimental ratio $E_x(2^+_2)$/$E_x(2^+_1)$ is 9.8.  The Davydov model yields $\sim$20 for $\gamma=$9$^{\circ}$, and requires $\gamma\sim$14$^{\circ}$ to reproduce this experimental value.   Thus, the Davydov model shows inconsistency within the model.  The present CI calculation, in contrast, incorporates variable triaxiality, which indeed changes more or less from band to another, and possibly other effects.  \textcolor{black}{It consequently reproduces} the observed $\gamma$-band excitation energy, as presented above.  \textcolor{black}{The Davydov model is, on the other hand, too rigid to handle such effects, and fails in predicting excitation energies.}  The concept of triaxial shapes \textcolor{black}{is definitely} appropriate. 

We note that the deformed shell model by Sun {\it et al.} \cite{sun_2000,sun_2002}, was applied, with assumed triaxiality, to a number of heavy nuclei. 

%%%   
\section{Monopole Interaction and Tensor Force} 
\label{sec:mono}

It is of interest and importance to find the origin of the present triaxiality in nuclear forces.
A crucial key here is the monopole interaction (see a review in \cite{otsuka_2020}), which is a part of any two-body interaction (see also Ref. \cite{otsuka_2022_emerging} for concise description).  When two nucleons are in single-particle orbitals $j$ and $j'$, the orbital motions occur about the corresponding axes.  The monopole interaction stands for an averaged effect taken over all possible axis orientations, for a given interaction.  
The monopole interaction between a proton and a neutron is expressed as \cite{otsuka_2020,otsuka_2022_emerging}
\begin{equation}
v_{mono} = \Sigma_{j,j'} \, v^{mono}_{j,j'} \, n^{(p)}_j \, n^{(n)}_{j'} 
\label{eq:mono}
\end{equation}
where $n^{(p)}_j$ ($n^{(n)}_{j'}$) denotes the number of protons in orbital $j$ (neutrons in $j'$), and $v^{mono}_{j,j'}$ is the coefficient called monopole matrix element.  
The neutron single-particle energy (SPE) of the orbital $j'$ is effectively shifted by $\Sigma_j \, v^{mono}_{j,j'} \, n^{(p)}_j$ or vice versa.  
Evidently, these effects of monopole interaction are configuration dependent, or dynamically vary.  
If the monopole interaction is particularly strong (or $v^{mono}_{j,j'}$ is large) between specific $j$ and $j'$, it can have crucial impacts on the nuclear structure.  

%%%%%%%%%%%%  FIGURE 5  %%%%%%%%%%%%%

\begin{figure}[tb]
  \centering
%%%
  \includegraphics[width=8cm]{Fig/Fig5.eps}
    \caption{ {Mechanism making triaxial ground and low-lying states.}
   The blue wavy line indicates monopole interaction which is particularly strongly attractive due to
   coherent contribution from tensor and central nuclear forces.  The enhancement of the $Q_2$ quadrupole moment is indicated by round arrows.
    } 
  \label{fig:mono}  
\end{figure}    

%%%%%%%%%%%%%%%%%%%%%%%%%%%%%%%

The monopole interaction of the tensor force brings about such impacts \cite{otsuka_2005,otsuka_2020}.  It generates a strong attraction, for instance, between a proton in the $j=l+1/2$ orbital and a neutron in the $j'=l'-1/2$ orbital, where $j$ and $j'$ ($l$ and $l'$) denote the total (orbital) angular momenta, and $1/2$ implies the intrinsic spin of a nucleon.  
Note that $j$ also means the index of the orbital.  As an example, 
the monopole interaction of the tensor force is indeed strongly attractive between the proton $1h_{11/2}$ orbital and the neutron $1h_{9/2}$ orbital with $v^{mono}_{1h_{11/2},1h_{9/2}}=-0.08$ MeV.  Combined with the contribution from the central force \cite{otsuka_2010}, the total monopole interaction becomes strongly attractive with $v^{mono}_{1h_{11/2},1h_{9/2}}=-0.39$ MeV, well beyond the average value of $v^{mono}_{j,j'}, -0.25$ MeV over all orbitals.  The tensor force contribution is more than half the extra gain (see Fig.~\ref{fig:mono}).  As the monopole effect is linearly dependent on $n^{(p)}_j$ and $n^{(n)}_{j'}$ (see eq.~(\ref{eq:mono})), its magnitude can be magnified by a factor 10 or more.  This is a unique property absent in any part of the interaction but the monopole.  The monopole interaction between two protons or between two neutrons is not so relevant here.
 
The tensor force is a part of the nuclear forces.  After the prediction of meson mediated nuclear forces by Yukawa \cite{yukawa_1935}, Bethe formulated the tensor force \cite{bethe_1940}.  
If one $\pi$ meson is emitted from a nucleon and is absorbed by another nucleon, a tensor force arises between these two nucleons.  In other words, the one-$\pi$-exchange process and a tensor force are the same thing to a great extent.    This is the major origin of the tensor force working between nucleons, with the cancellation by about a quarter due to $\rho$ meson exchange, a resonance of two $\pi$ mesons.  The proton-neutron interaction is given by the $V_{\rm MU}$ interaction as mentioned earlier, and this interaction includes these tensor forces with strengths evaluated in Ref. \cite{osterfeld_1992}.
The $NN$ interaction acting on nucleons bound in nuclei undergoes substantial renormalization in general, and is changed.   It was shown, however, that the tensor force, particularly its monopole interaction, remains rather unchanged.  This unique property is referred to as the renormalization persistency \cite{ntsunoda_2011}.    Thus, the $\pi$ meson exchange is the major origin of the tensor force in nuclei.


%%%%%%%%%%   Figure 6     %%%%%%%%%%

\begin{figure*}[tb]
  \centering
  \includegraphics[width=17.5cm]{Fig/Fig6.eps}
    \caption{ {Triaxiality development in single-particle orbital h$_{11/2}$.}
    {\bf a} Prolate case ($\gamma$=0) with degenerate single-particle orbitals.
    {\bf b} Splitting and mixing of single-particle states of the h$_{11/2}$ orbital shown by 
    blue ($\gamma$=0) and red ($\gamma$=9$^{\circ}$) horizontal bars.  The red asterisks 
    imply that the occupation of these states can generate the triaxiality (or finite $Q_2$). } 
  \label{fig:Q2}  
\end{figure*}  

%%%%%%%%%%%%%%%%%%%%%%%%%%%%%%%%%%%%%%

\section{Mechanism of the enlargement of $Q_2$ moment}

We here discuss basic mechanisms of the appearance of triaxiality in the present case, in a pedagogical way using schematic examples.    
The single-particle orbitals in the harmonic oscillator potential are degenerate if they belong to the same oscillator quanta.   
Figure~\ref{fig:Q2}{\bf a} shows one of such cases (the oscillator quanta = 5, comprising 1$h_{11/2,9/2}$, 2$f_{7/2,5/2}$ and 3$p_{3/2,1/2}$ orbitals).  If a quadrupole deformation sets in as Fig.~\ref{fig:2+}{\bf a} exhibits, it is known that a prolate shape, which is axially-symmetric, arises in the ground and low-lying states.  This feature remains even if the degeneracy is lifted (or loosened) to a certain extent, as demonstrated in early studies \cite{kumar_baranger1968,bes_sorensen1969}, leading to the preponderance of axially-symmetric prolate shapes.   A similar situation occurs with the SU(3) symmetry \cite{elliott_1958a,elliott_1958b}.   Another work showing the dominance of the prolate nuclear shapes was
a systematic survey in terms of the Skyrme Hartree-Fock calculation \cite{tajima_1996}.  The result appeared to be consistent with the above-mentioned other works.
The universal finite-range liquid-drop model (FRLDM) similarly predicts no triaxial ground states for the deformed nuclei discussed in this article (see Fig. 2 of Ref. \cite{moller_2006}). 

This has been the conventional picture for the deformed nuclear shapes of heavy nuclei, but a different picture now emerges.   
Figure~\ref{fig:Q2}{\bf b} exhibits an illustrative example focusing on the lowering of the 1$h_{11/2}$ orbital.  
This lowering is partly due to a typical phenomenon in nuclei.  The orbitals of large $j=l+1/2$ are shifted down in energy due to the spin-orbit splitting and the so-called $l^2$ term which are stronger for larger $l$ values \cite{bohr_mottelson_book1}.   This effect was included in earlier studies such as Refs. \cite{kumar_baranger1968,bes_sorensen1969}, and does not change the preponderance of prolate shapes.   There is another effect, however.
As graphically shown in Fig.~\ref{fig:mono}, an extra attraction emerges, for instance, between the proton 1$h_{11/2}$ orbital and the neutron 1$h_{9/2}$ orbital.  

Figure~\ref{fig:Q2}{\bf b} exhibits that the energies of single-particle states of the 1$h_{11/2}$ orbital in the body-fixed frame, for z-component of $j$: $j_z$=$\pm 1/2$, $j_z$=$\pm 3/2$, ..., $j_z$=$\pm 11/2$.
For $\beta_2$=0, the nucleus is spherical, and the energies of these single-particle states are degenerate (see below $\beta_2$=0).
Once the nucleus is deformed, this degeneracy is lifted.   For a prolate ellipsoid 
(in gray) with $\beta_2>0$ and $\gamma$=0, the $j_z$ quantum number is conserved in each single-particle state.  The energies of these states are split in the ascending order, $j_z$=$\pm 1/2$, $j_z$=$\pm 3/2$, ..., $j_z$=$\pm 11/2$.  The degeneracy between positive and negative signs is due to the time reversal symmetry.  

There are sizable matrix elements of the $Q_2$ moment, for instance, between the states of $j_z$=1/2 and 5/2, as the $Q_2$ operator can change the $j_z$ by two units.  The single-particle states of different $j_z$'s are mixed in the eigenstates of a  quadrupole field involving certain triaxiality ($\gamma\ne$0).   Such eigenstates then yield non-vanishing $Q_2$ moments.  These $Q_2$ moments are coupled with the $Q_2$ deformed field (produced by other nucleons), shifting the energies.  Such energy shifts are depicted, for $\gamma$=9$^\circ$ in the far right column of  
Fig.~\ref{fig:Q2}{\bf b}.  The lowest state shows extra binding energy.
If the lowest state or the two lowest states, marked with asterisks, are occupied, certain $Q_2$ moments are produced, driving the nucleus into triaxial shape.  This is one of the basic mechanisms to yield finite $Q_2$ moment, or triaxial shape.

Figure~\ref{fig:Q2}{\bf b} indicates that the lowering of 1$h_{11/2}$ orbitals leaves an incomplete harmonic oscillator shell.  As stated above, the $Q_2$ moment of the nucleus is zero or small in magnitude for strongly deformed states, if all orbitals of a harmonic oscillator shell are degenerate or almost so.  As individual matrix elements of the $Q_2$ operator are not small, the cancellation should occur among individual effects.  With the $h_{11/2}$ orbital separated from the rest of the orbitals, this cancellation becomes less perfect, which results in another source of finite $Q_2$ moment. 

Figure~\ref{fig:Q2}{\bf b} suggests that two to four nucleons in high-$j$ orbitals can substantially contribute to $Q_2$ moment.  \textcolor{black}{Such occupation of high-$j$ orbitals does not occur in general, unless substantial additional binding energy is brought in.
Regarding $^{166}$Er, this indeed} occurs for the combination (proton 1$h_{11/2}$, neutron 1$h_{9/2}$) and for the combination (proton 1$g_{7/2}$, neutron 1$i_{13/2}$), because of strongly attractive proton-neutron monopole interactions, $v^{mono}_{1h_{11/2},1h_{9/2}}$ and $v^{mono}_{1g_{7/2},1i_{13/2}}$.  Note that the proton 1$g_{7/2}$ orbital contributes to the monopole effect but not to $Q_2$ moment, as it is nearly fully occupied.

%%%%%%%%%%%%%  Figure  7   %%%%%%%%%%%%%%%%%%%%%%

\begin{figure}[!tb]
  \centering
  \includegraphics[width=8.7cm]{Fig/Fig7.eps}
    \caption{ {PES near the minimum by the constrained HFB after the projection
    on to $J^P$=$0^+$.}
    The calculations are made with  {\bf a} the original Hamiltonian, {\bf b} the Hamiltonian
    where two crucial monopole interactions are reduced , {\bf c} the Hamiltonian 
    monopole-frozen with spherical HFB reference state. 
    Dashed lines are drawn to show representative $\gamma$ angles.
%    {\bf d (e)} T-plot corresponding to {\bf b (c)}.  The background is the unprojected PES.
    } 
  \label{fig:3pec}  
\end{figure}  

%%%%%%%%%%%%%%%%%%%%%%%%%%%%%%%%%%%%%%%%%

%%%  Analyses
\section{Analyses with reduced or eliminated monopole interactions}
This monopole-quadrupole combined effect naturally occurs in the actual CI wave functions, and provide more binding energies to triaxial states over prolate states, making triaxial ground and low-lying states.  We look into this effect with two analyses. 

%%%  Analyses with reduced monopole interactions
\subsection{Analysis with reduced monopole interactions and the self-organization}
   
Figure~\ref{fig:3pec} displays details of the PES near the minimum.  \textcolor{black}{In this subsection, for a more refined inspection, the PES is calculated by the constrained HFB after the projection on to $J^P$=$0^+$.}  The PES minimum of the original Hamiltonian (panel {\bf a}) is located at $\gamma$=9.5$^{\circ}$ and its energy is lower by $\sim$0.4 MeV than the lowest prolate point.  

%Thus, instead of conventionally conceived mechanism (see Fig.~\ref{fig:level}{\bf a}), another one in Fig.~\ref{fig:level}{\bf b} emerges for the original Hamiltonian with realistic monopole interactions.  
The CI results indicate that the ground state of  $^{166}$Er contains about four protons in $1h_{11/2}$ and about four neutrons in $1h_{9/2}$, with which a strong quadrupole deformation with triaxiality is generated, and the extra monopole effect beyond the average one reaches about -0.14$\times$4$\times4=-2.24$ MeV, a substantial additional binding.  A similar but weaker effect arises from the (proton 1$g_{7/2}$, neutron 1$i_{13/2}$) combination. \textcolor{black}{These effects combined produce the triaxial minimum in the PES just mentioned, as well as T-plot circles around $\gamma$=9$^{\circ}$ shown in Fig.~\ref{fig:tplot}{\bf c}.} 

\textcolor{black}{In order to further examine the monopole-quadrupole interplay,} $v^{mono}_{1h_{11/2},1h_{9/2}}$ and $v^{mono}_{1g_{7/2},1i_{13/2}}$ are reduced to the average value, -0.25 MeV (see Sec.~\ref{sec:mono}). 
Figure~\ref{fig:3pec} {\bf b} shows the result of this monopole-reduced Hamiltonian: the minimum at $\gamma$=6.3$^{\circ}$ is only $\sim$0.1 MeV below the prolate minimum. 
\textcolor{black}{This trend is confirmed by the T-plot in Fig.~\ref{fig:tplot}{\bf d}, which displays the  movement} of T-plot circles towards a prolate shape.  

The monopole interaction favors no specific shape, but can yield crucial effects for nuclear shapes by effectively changing single-particle energies in configuration-dependent manners.  
\textcolor{black}{In other words, the single-particle environment can be tailored for certain shapes.}  This mechanism is referred to, in general, as the self-organization \cite{otsuka_2019,otsuka_2022_emerging}.   \textcolor{black}{In actual cases, the final shape may emerge from a pool of shapes with more or less similar deformation (or quadrupole) energies.  As  the maximum binding energy includes both monopole and quadrupole contributions, the monopole-interaction effect can be crucial.}  In the case of $^{166}$Er, certain configurations favoring triaxiality indeed gain more binding energy due to this mechanism. 

For the mechanism of the triaxiality, it is crucial to put proper numbers of nucleons in the relevant large-$j$ orbitals.  \textcolor{black}{If they are too few or too many, the $Q_2$ moment cannot be large enough, and the shape moves to a spherical or a prolate one. 
The prolate shapes in Hg isotopes \cite{marsh_2018} turn out to be an example of too many neutrons in the   
$1i_{13/2}$ orbit. }
 
%%%  Monopole frozen
\subsection{Frozen Analysis and appearance of prolate shape}
As another way to see the crucial contribution of the monopole interaction, we calculate mean shifts of the SPEs by taking expectation values of $n^{(p)}_j$ {\it etc.} in eq.~(\ref{eq:mono}) with a reference state.  We then remove the monopole interaction and instead include such mean shifts of the SPEs.  Namely, dynamical effects of the monopole interaction are {\it frozen} with this reference state \cite{otsuka_2019}. The CI calculation is then performed.  Presently, a spherical state, given by constrained HFB calculations, is used as the reference state.      
Figure~\ref{fig:3pec}{\bf c} exhibits the PES of this monopole-frozen Hamiltonian.  The minimum is even closer to the prolate line than in panels {\bf a,b}.
Figure~\ref{fig:tplot}{\bf e} displays the T-plot of the 0$^+_1$ state obtained by the same monopole-frozen Hamiltonian.  The strong deformation remains, but the triaxiality disappears. This further confirms that the activation of the monopole interaction is crucial for the triaxiality. 


\section{Nuclei around $^{166}$Er in the Segr\`e chart}

\textcolor{black}{
We now expand our scope to some nuclei around $^{166}$Er in the Segr\`e (nuclear) chart.}

One of the signatures of the triaxiality can be the 2$^+_2$ level below the 0$^+_2$ level, \textcolor{black}{and we adopt it.}  Figure~\ref{fig:chart} displays a part of the Segr\`e (nuclear) chart (even-even nuclei around $^{166}$Er), where star symbols indicate 17 deformed nuclei ($E_x(2^+_1)<$ 0.15 MeV) with this signature \textcolor{black}{experimentally observed} \cite{ensdf,nudat3}.   These 17 nuclei are traditionally supposed to have prolate ground states.  The purple star symbols form a separate group, some of which \textcolor{black}{may have been considered, in the past,} to be triaxial.  In fact, its neighbor, $^{188}$Os, was considered to have $\gamma$=30$^{\circ}$ \cite{hayashi_1984}.  
Missing star symbols do not necessarily imply prolate shapes but rather mean unavailable experimental data. 

%%%%%%%%%%%%  FIGURE 8  %%%%%%%%%%%%%

%\begin{figure*}[tb]
\begin{figure}[tb]
  \centering
  \includegraphics[width=8.8cm]{Fig/Fig8.eps}
    \caption{ {Nuclear chart (part) and triaxially deformed nuclei (red and purple star symbols).}
   Red squares indicate the lowest 2$^+$ states with higher level energies ($>$ 1 MeV), 
   while green ones stand for those with lower level energies ($<$ 0.1 MeV).  The colors in between indicate intermediate cases.   Background given by NuDat 3.0 \cite{nudat3}.
    } 
  \label{fig:chart}  
\end{figure}    
%\end{figure*}  

%%%%%%%%%%%  Figure 9  (old  Extended FIGURE 1)  %%%%%%%%%%%%%

%%%  levels of 162,166Er and 164Dy as well as 170Er
\begin{figure*}[!tb]
  \centering
  \includegraphics[width=14cm]{Fig/Fig9.eps}
    \caption{ {Lowest level energies of triaxial nuclei.} Selected triaxial nuclei are {\bf a} $^{166}$Er,  {\bf b} $^{162}$Er, {\bf c}  $^{164}$Dy and {\bf d} $^{158}$Gd.  {\bf e} Lowest level energies of a prolate nucleus, $^{170}$Er, are depicted for comparison.
For these triaxial nuclei, theoretical and experimental \cite{ensdf} values of $B(E2;2^+_1\rightarrow 0^+_1)$ and $B(E2;2^+_2\rightarrow 0^+_1)$ (see arrows) are shown in W.u.  The spectroscopic electric quadrupole moment is shown (see "Q") in the unit of $e$ barn for the $2^+_1$ and $2^+_2$ states for some nuclei \cite{ensdf}.  }
  \label{fig:levels}
\end{figure*}  

%%%%%%  162Er   164Dy   158Gd   %%%%%%%%%%%%%%%

The CI calculations discussed in this article require huge computer resources, and thereby we have performed them for $^{162}$Er, $^{164}$Dy, $^{158}$Gd and $^{170}$Er besides $^{166}$Er.  
Figure~\ref{fig:levels} {\bf a-e} show primary results of them.  
These panels show not only the level energies but also the values of $B(E2;2^+_1\rightarrow 0^+_1)$, $B(E2;2^+_2\rightarrow 0^+_1)$ (in W.u.) and spectroscopic electric quadrupole moments (in $e$ barn).    
A good agreement to experiment \cite{ensdf} is visible, which suggests the validity of the Hamiltonian.  As an example, the 2$^+_2$, 3$^+_1$ and 4$^+_2$ states of $^{162}$Er (panel {\bf b}) are shifted upwards by the almost same amounts between theory and experiment, compared to those of $^{166}$Er (panel {\bf a}).   A further shift is seen in $^{158}$Gd similarly between experiment and theory (panels {\bf a} and {\bf d}).  All shown levels remain quite unchanged between $^{164}$Dy and $^{166}$Er (panels {\bf a} and {\bf c}), but the $B(E2;2^+_2\rightarrow 0^+_1)$ value differs between these two nuclei also similarly in both theory and experiment.  
Thus, a good overall agreement is obtained between theory and experiment.   In particular, the $B(E2;2^+_2\rightarrow 0^+_1)$ value changes by up to a factor of two from nucleus to nucleus, and the present calculation reproduces this variation rather well.  
\textcolor{black}{The nuclei, $^{158}$Gd,  $^{164}$Dy, and $^{162,166}$Er fulfill the criteria of the triaxiality mentioned above, also in theory.}  

\textcolor{black}{In the level scheme of $^{170}$Er, on the other hand, the calculated 0$^+_2$ level is lower than the  2$^+$ levels except for the 2$^+_1$ level, consistent with experiment.
This level scheme places this nucleus outside the group of the triaxial nuclei, in the same way as the experimental classification shown in Fig.~\ref{fig:chart}.  It is remarkable that the same Hamiltonian reproduces this transition from triaxial to prolate shapes as a function of the neutron number.} 

%%%%%%%%%%  Figure 10  (old Extended FIGURE 2)  %%%%%%%%%%%%%
\begin{figure*}[!tb]
  \centering
  \includegraphics[width=17.5cm]{Fig/Fig10.eps}
    \caption{ {PES of 17 triaxial deformed nuclei and their neighbors.}
    The present deformed triaxial nuclei are in dark-pink background and are named in red.
    The neighboring nuclei are in blanc background and are named in black.  The PES is 
    calculated by the HFB calculation without angular-momentum projection.  The particle
    numbers are projected.
    } 
  \label{fig:17pes}  
\end{figure*}  
%%%%%%%%%%%%%%%%%%%%%%%%%%%%%%%
 
\textcolor{black}{Figure~\ref{fig:17pes} displays the PESs of the ``17 experimental triaxial nuclei'' (red stars) defined in Fig.~\ref{fig:chart}, being embedded in the Segr\`e chart.  Other nuclei surrounding them are also shown, where the shape is less deformed ({\it e.g.} $N$ smaller)
or becomes closer to prolate ({\it e.g.} $N$ larger).} 

\textcolor{black}{
The MCSM results for $^{162}$Er, $^{166}$Er, $^{164}$Dy, and $^{158}$Gd confirm the triaxiality inferred from the PES's shown in Fig.~\ref{fig:17pes}.  As the patterns in Fig.~\ref{fig:17pes} vary only gradually within the ``17 triaxial nuclei'',  most of them, at least, are expected to show profound triaxiality once the MCSM calculation is performed.  
As these 17 nuclei are located in the middle of the deformed rare-earth region in the Segr\`e chart, and are traditionally supposed to have prolate ground states, the overall picture of nuclear shape is drastically changed.    We are thus led to the prevailing triaxiality in heavy deformed nuclei.}
 
%%%%%%%%%%%%%%%%%%%%%%%%%%%%%%%

%%%   Re-visit to early experiments 
\textcolor{black}{\section{Re-visit to early experiments} }

\textcolor{black}{
Quite a few experiments were conducted decades ago for the nuclei of current interest, yielding extensive data by various probes. Examples are found in \cite{cline_1986} for $^{168}$Er, in 
\cite{fahlander_1992} for $^{166}$Er, in \cite{werner_2005} for $^{158}$Gd and $^{166}$Dy, where the value of the angle $\gamma$ was deduced {\it \`a la} Cline \cite{cline_1986}, with some variations, based on the Kumar invariant \cite{kumar_1972}.  
The typical indication is represented by excerpts from \cite{cline_1986}, ``The asymmetric rigid rotor using $\gamma$=9 $^\circ$ reproduces the data well ...'' and ``The individual E2 matrix elements and the rotational invariants for the ground and $\gamma$ bands in $^{168}$Er all are consistent with rotation of a quadrupole deformed rotor with asymmetry centroid of $\gamma$ $\approx$ 9$^\circ$, ...''.   Despite such clear experimental message, there was no statement to deny the prolate ground state, which might be due to the paradigm of the ``preponderance of axially symmetric shapes'' \cite{bohr_nobel,kumar_baranger1968, bes_sorensen1969}.  
The structure of $^{168}$Er resembles that of $^{166}$Er, where the MCSM calculation points to $\gamma$ $\sim$ 9$^\circ$ as discussed so far.   
The value of $\gamma$ is reported as $\gamma$$\approx$10$^\circ$ for $^{166}$Er \cite{fahlander_1992}, $\gamma$=6 (2)$^\circ$ for $^{158}$Gd and $\gamma$=7 (4)$^\circ$ for $^{164}$Dy \cite{werner_2005}.  The present MCSM calculation shows $\gamma$=5.9$^\circ$ 
for $^{158}$Gd and $\gamma$=7.4$^\circ$ for $^{164}$Dy, in a salient agreement with the values deduced from experiment.      
Thus, with unbiased eyes, there were experimental evidences for the present view with the triaxiality.  Once the triaxiality is set in, the $\gamma$ band naturally arises as discussed above, and a $\gamma$ phonon or something like that is redundant.  Moreover, the present  MCSM calculations show no hint of such states in relevant energy region. 
}

\textcolor{black}{
The existing spectroscopic data show reasonable agreement to the present MCSM calculations, and the $\gamma$ value deduced from these data based on a standard method depicts also good agreement with the MCSM result.  Thus, individual data and systematic analysis are quite consistent with the present calculation and the view of prevailing triaxiality.  The idea of ``preponderance of axially symmetric shapes'' was supported by microscopic calculations in the 1950s, but such calculations did not include the tensor force.  The tensor force effect had been practically overlooked for decades also in the systematic study of the shell structure \cite{otsuka_2020}.   A similar history may have been  repeated for the shapes, even casting substantial influence, leading to an improper treatment of correct experimental data. 
}
 
\textcolor{black}{
It is worth noting that both E2 transitions and excitation energies are discussed together in some of the attempts to extract $\gamma$ values, but the latter cannot be assessed so precisely within the Davydov model, as indicated above.  This drawback, which has become clear by the present work, caused some confusions in the past.  
}
\\ 
%%%%%%%%%%%%%%%%%%%%%%%%%%%%%%%
 
%%%   Summary
\section{Summary, Prospect and a note on Davydov} 

\textcolor{black}{This article shows, for the first time, the unexpected role of the tensor force} for the ellipsoidal nuclear shapes.  The tensor force represents a major part of the one-$\pi$-meson exchange effects, and was shown to be responsible for the shell evolution in exotic nuclei \cite{otsuka_2005,otsuka_2020,otsuka_2022_emerging}, 
giving rise to new magic numbers (see Fig.~\ref{fig:2+}).  The shell structure becomes invisible in most of heavier nuclei as single-particle orbitals are mixed due to strong deformation.  
In these nuclei, the tensor force can produce completely different but still important effects towards triaxiality, through a quantum mechanical way of the self-organization \cite{otsuka_2019}.   
Thus, the tensor force produces the shell evolution in exotic nuclei and the triaxial shapes in heavy deformed nuclei.  
Due to the renormalization persistency \cite{ntsunoda_2011}, the tensor force \textcolor{black}{in nuclei} is the major consequence of the exchange of one $\pi$ meson, being rather free from in-medium corrections, and we now see the direct connections between such an ``elementary-particle'' process and basic nuclear shapes. 
These findings make the trend towards triaxiality quite robust.   
The $\pi$-meson exchange is treated separately also in the modern theory of nuclear forces \cite{weinberg_1990} as materialized as the chiral Effective Field Theory \cite{machleidt_2011} of the Quantum Chromo Dynamics. 

\textcolor{black}{Based on the robustness of the triaxiality, the prevailing triaxiality is addressed for heavy deformed nuclei.
The preponderance of prolate shapes, stressed by Aage Bohr, has been a textbook item, but it will be superseded, in future textbooks, by the present novel concept.    
Consequently, traditional $\gamma$ vibration is replaced by triaxial rotation \textcolor{black}{quantified} by the $K$ quantum number approximately conserved.
These new picture and concept are supported by state-of-the-art extra-large-scale CI calculations (i.e., Monte Carlo Shell Model) with realistic effective interaction.   The agreement with measured spectroscopic data is quite good.}
\textcolor{black}{The earlier experiments, which did not explicitly claim the prevailing triaxility, turn out to support this novel picture.   
}

\textcolor{black}{
The excitation energy of the $\gamma$ band is lowered as the triaxiality slightly grows from the ground band to the $\gamma$ band.  This is nothing but a stretching by the \textcolor{black}{$K$ rotation due to the triaxiality}, whereas the ellipsoid is rigid against the \textcolor{black}{$R$} rotation within individual bands.  This is another novel feature, and is \textcolor{black}{linked to} realistic nuclear forces, suggesting a possible crucial role of the CI calculation for the nuclear collectivity.   
}

It is still a major experimental challenge to \textcolor{black}{explicitly or critically scan} the shape of just ground state.    
Magnetic excitation is a possibility: another mode may appear besides the scissors mode \cite{pietralla_1998}, as a contra oscillation between the proton and neutron ellipses about the longest axis (see the right-lower part of Fig.~\ref{fig:image}{\bf d}), which may be called a rolling mode.
Another intriguing possibility lies in relativistic heavy-ion collision as a new tool for scanning the nuclear shape \cite{giacalone_phdthesis_2020}.  The former experiment may be done in facilities like HI$\gamma$S and RCNP, while the latter in LHC/CERN or RHIC.  
\textcolor{black}{Hyper nuclei containing a $\Lambda$ partcile may provide with another opportunity to see the triaxiality, in the form of splitting of the level energies, which can be seen in J-Lab or J-PARC.}
Another relevant feature is the chirality.  As a result of substantial triaxiality in the strongly deformed region, nuclear chiral doublet bands \cite{frauendorf_2001} may appear 
more widely than expected, providing new experimental opportunities.

The binding energy gain by the deformation is likely more crucial in heavy and superheavy nuclei, and possible appearance of triaxiality was shown even without including the present mechanism, for instance, Ref. \cite{algerian_2017}.  The triaxiality has also been addressed over decades as it may lower the first fission barrier \cite{leander}.  As the triaxiality is a robust reality rather than an accidental incident, it may produce more visible effects on these nuclei than previously expected.

\textcolor{black}{It is of great interest to explore, in other physical systems, possible effects similar to the ones by the nuclear tensor force.  The interaction between electric dipole moments, carried possibly by molecules, depicts such similarity, and may show up in microclusters of water molecules, metal molecules, {\it etc}.  
}

We finally note that the preponderance of triaxiality in heavy nuclei was suggested by a Ukrainian physicist, Dr. A. S. Davydov (Crimea 1912 - Kyiv 1993), but this suggestion has not been well appreciated due to some reasons.  Since the basic idea is shown to be relevant, he and his work can be better appreciated.    

%%%%%%%%%%%%%%%%%%%%%%%%%%%%%%%%%%%%%%%%%%%%%

\section*{A\lowercase{cknowledgements}}

The authors are grateful to Drs. P. Van Duppen, N. Pietralla, P. von Neumann-Cosel, A. Tamii, G. Giacalone, K. Nishio, Y. Aritomo, T. Azuma, K. Yabana and S. Yamamoto for valuable suggestions and/or discussions.  
The MCSM calculations were performed on the supercomputer Fugaku at RIKEN AICS  (hp190160, hp200130, hp210165, hp220174).  
This work was supported in part by MEXT as "Priority Issue on Post-K computer" (Elucidation of the Fundamental Laws and Evolution of the Universe) (hp160211, hp170230, hp180179, hp190160) and "Program for Promoting Researches on the Supercomputer Fugaku" (Simulation for basic science: from fundamental laws of particles to creation of nuclei) (JPMXP1020200105), and by JICFuS.  
This work was supported by JSPS KAKENHI Grant Numbers JP19H05145, JP21H00117, JP21K03564, JP20K03981, JP17K05433  and JP18H05462.  
  
%%%%%%%%%%%%%%%%%%%%%%%%%%%%%%%%%%%%%%%%%%%%%

\appendix
\def\thesection{\Alph{section}}

\section{Configuration Interaction (CI) calculation or Shell-model calculation}
\label{Ap_shell}

\noindent
We sketch the shell model for atomic nuclei in this Appendix.
The shell-model calculation is one of the standard methods for the nuclear many-body problem \cite{deshalit_book,ring_schuck_book,talmi_book,caurier_2005}.  It belongs to the category of the Configuration Interaction (CI) calculation, which is more familiar to broad audience and is also used in this article for the meaning of the shell model.   The ingredients of the shell model are (i) single-particle orbitals and their energies, (ii) the numbers of protons and neutrons in these orbitals, (iii) nucleon-nucleon ($NN$) interaction.  Thus, the properties of the nuclear states are determined by them, without other a priori assumption.  The protons and neutrons can move in these orbitals, scattering each other through the $NN$ interaction adopted.   These protons and neutrons (collectively called nucleons) do not include the nucleons in the inert core, as it is a closed shell and is treated as the vacuum.    
The single-particle energies (SPE) and the $NN$ interaction are taken from some models and/or theories.
The matrix elements of the SPE and the $NN$ interaction are expressed, as one- and two-body operators, respectively,  with respect to all possible combinations of single-particle  states, which are magnetic substates of each single-particle orbital.  

The Hamiltonian consists of one-body term and two-body term, as usual.
The one-body term is expressed by the SPEs, and the two-body term is expressed in terms of the matrix elements of the $NN$ interaction with respect to antisymmetrized two-nucleon states.  The Hamiltonian is thus constructed, and the actually used Hamiltonian is mentioned in the main text.

The many-body states are described by superpositions of Slater determinants in many {\it conventional} shell-model calculations.
The many-body Schr\"odinger equation is solved for the given Hamiltonian as,  
\begin{equation}
H \, \Psi \,=\, E \, \Psi,             
\label{eq:Schr}
\end{equation}
where $\Psi$ is an eigenstate for an eigenvalue $E$, and the Hamiltonian {\it matrix} for this many-body system is represented by matrix elements of the Hamiltonian {\it operator} for all combinations (bra and ket vectors) of all possible Slater determinants.
By diagonalizing this Hamiltonian matrix, energy eigenvalues and wave functions of eigenstates are obtained.  We can calculate various physical quantities from these wave functions.  
The number of Slater determinants is called shell model dimension.  
This is the general framework of the shell model, while many-body states can be equivalently expressed otherwise.

The shell model dimension becomes huge in many interesting cases, restricting the actual feasibility of the calculation.  This is the major obstacle of the conventional shell-model calculation, and the current limit of the shell-model dimension is around 10$^{11}$ as of 2019 \cite{shimizu_2019}.  In order to overcome this difficulty, the Monte Carlo Shell Model was introduced, as described in the next Appendix.

The practical setup of the present work is mentioned \cite{otsuka_2019}.  Proton single-particle orbitals are, 1$g_{9/2,7/2}$, 2$d_{5/2,3/2}$, 3$s_{1/2}$, 1$h_{11/2}$, 2$f_{7/2}$, and 3$p_{3/2}$.   The neutron single-particle orbitals are, 1$h_{11/2,9/2}$, 2$f_{7/2,5/2}$, 3$p_{3/2,1/2}$, 1$i_{13/2}$, 2$g_{9/2}$, 3$d_{5/2}$ and 4$s_{1/2}$.  
These orbitals define the model space, which is built on top of the $Z$=40 and $N$=70 magic numbers of the Harmonic Oscillator potential.
In the case of $^{166}$Er, 28 protons and 28 neutrons are put into the model space formed by these orbitals.  
The model space and the number of nucleons are much larger than those in the usual shell-model calculation, so that the shape deformation can be described.  The shell-model dimension becomes as large as 4.8$\times$10$^{33}$ for the case of $^{166}$Er.  This is far beyond the limit of the conventional shell-model calculation, but can be overcome by using the MCSM described in the next Appendix.


\def\thesection{\Alph{section}}
\section{Monte Carlo Shell Model and T-plot}
\label{Ap_MCSM}

\noindent
In this Appendix, we briefly describe the Monte Carlo Shell Model (MCSM).
The MCSM was initially proposed in Ref. \cite{mcsm_1995}.  A prototype of its present version was shown in Ref. \cite{mcsm_1998}.  The method and applications of the MCSM were reviewed, for instance, in Refs. \cite{mcsm_2001,mcsm_2012}.  The MCSM is the methodology fully exploited in this work.  It uses Slater determinants as the basis vectors, similarly to the conventional shell-model calculation.  However, the Slater determinants are not the same as those used in the conventional shell-model calculation.   In the conventional one, each Slater determinant is a direct product of some single-particle states, each of which is a magnetic substate of the single-particle orbital.   

An MCSM eigenstate is schematically written as (see Ref. \cite{otsuka_2022_emerging} for more detailed but pedagogical concise explanation),
\begin{equation}
\Psi \, = \, \sum_k \, f_k \, \hat{{\mathcal P}}_{J^{P}} \, \phi_k \,\,, 
\label{eq:mcsm_psi}
\end{equation}
where $f_k$ denotes the amplitude, $\hat{{\mathcal P}}_{J^{P}}$ means the projection operator on to the spin/parity $J^{P}$ (this part is more complicated in practice), and $\phi_k$ stands for Slater determinant called ($k$-th) MCSM basis vector: $\phi_k$ = $\Pi_i \, c^{(k)\dagger}_i \,|0 \rangle$.  Here, $|0 \rangle$ is the inert core (closed shell), $c^{(k)\dagger}_i$ refers to a superposition, 
\begin{equation}
\label{eq:mcsm_bv} 
c^{(k)\dagger}_i \,=\, \sum_n \, D^{(k)}_{i,n} \, a^{\dagger}_n \,\, ,
\end{equation}
with $a^{\dagger}_n$ being the creation operator of a usual single-particle state, for instance, that of the HO potential, and $D^{(k)}_{i,n}$ denoting a matrix element.  By choosing an optimal matrix $D^{(k)}$, we can select $\phi_k$ so that such $\phi_k$ better contributes to the lowering of the corresponding energy eigenvalue.  Thus, the determination of $D^{(k)}$ is the core of the MCSM calculation. The index $k$ runs up to 50-100, but can be more.  These are much smaller than the dimension of the many-body Hilbert space, which is 4.8$\times$10$^{33}$ for the case of $^{166}$Er, as already mentioned.
  
Thus, the basis vectors of the MCSM calculation are composed of ``stochastically - variationally deformed'' single-particle states.  The adopted basis vectors are mutually independent, otherwise no energy gain.    By having a set of these MCSM basis vectors thus fixed, we diagonalize the Hamiltonian, and obtain energy eigenvalues and eigenstates.  
A large number of MCSM calculations have been performed as exemplified in Refs. \cite{tsunoda_2014,togashi_2016,otsuka_2016,leoni_2017,marsh_2018,togashi_2018,ichikawa_2019,otsuka_2019,taniuchi_2019,tsunoda_2020,marginean_2020,abe_2021,otsuka_2022_nature}.

Besides the breakthrough in the computational limit, the MCSM also has the advantage of providing a very useful way to visualize the intrinsic shape of each MCSM eigenstate through what is called the T-plot \cite{tsunoda_2014,otsuka_2016}.  Because the MCSM basis vector is a deformed Slater determinant, one can calculate its intrinsic quadrupole moments, {\it i.e.} the quadrupole moments in the body-fixed frame, denoted as $Q_0$ and $Q_2$.  They can be expressed by two parameters $\beta_2$ and $\gamma$, as described in the main text.    The importance of each MCSM basis vector to a given eigenstate (its overlap probability in the MCSM eigenstate) is represented by the size (area) of its circular symbol in the T-plot.  The T-plot can be made on the $\beta_2$-$\gamma$ plane, but is usually on the PES, and intuitively exhibits the underlying physical pictures for the states of interest, as demonstrated in a variety of studies, {\it e.g.} in Refs. \cite{leoni_2017,marsh_2018,ichikawa_2019,togashi_2016,togashi_2018,otsuka_2019,marginean_2020,otsuka_2022_nature}.

We stress that the T-plot played a major role in the present study, displaying not only triaxial shapes of the eigenstates but also their rigidity over different eigenstates.   

The present calculation was performed by the most advanced methodology of the MCSM.
This is called Quasiparticle Vacua Shell Model (QVSM) \cite{shimizu_2021}.  In the original version of the MCSM, the pairing correlations are mainly incorporated by superposing different MCSM basis vectors, which are deformed Slater determinants as stated above.  The QVSM basis vectors are somewhat like a Hartree-Fock-Bogoliubov ground state (which is a generalization of the BCS ground state), and this feature enables each QVSM basis vector to contain both effects of the deformed mean field and effects of the pairing correlations.  This advantage is particularly exploitable for heavy nuclei where the pairing correlations over different single-particle orbitals become more important than for lighter nuclei.  In the original MCSM, the pairing correlations are largely carried by superpositions of different MCSM basis vectors.  As the QVSM calculation is computationally heavier, this merit makes sense for some nuclei heavier than $A\sim$100. 

The spurious center-of-mass motion is removed to a sufficient extent by the Lawson method \cite{lawson_method}.



%%%%%%%%%%%%%%%%%%%%%%%%%%%%%%%%%%%%%%%%%%%%%

\makeatletter
\renewcommand\@biblabel[1]{#1.}
%\renewcommand\@cite[2]{\ts{#1\if@tempswa , #2\fi}}
\makeatother

\def\bibsection{\section*{\bf references}}
%%%%%%%%%%
\begin{thebibliography}{99}

\aemp

%%%%  Bohr-Mottelson II
\bibitem{bohr_mottelson_book2}
Bohr, A. \& Mottelson, B.R., Nuclear Structure (Benjamin, New York, 1975), Vol. II.

% beta, gamma vibration   (vibration quanta, not on band formation) 
\bibitem{bohr_1952}
Bohr, A., The Coupling of Nuclear Surface Oscillations to the Motion of Individual Nucleons, 
Mat. Fys. Medd. Dan. Vid. Selsk. {\bf 26}, 14 (1952).

%\bibitem{bohr_mottelson_1953}
%Bohr, A. Mottelson, B. R., 
%Collective and Individual-Particle Aspects of Nuclear Structure, 
%Mat. Fys. Medd. Dan. Vid. Selsk. {\bf 27}, 16 (1953).

%%%%   Rainwater   quadrupole deformation
\bibitem{rainwater_1950}
Rainwater, J., 
Nuclear energy level argument for a spheroidal nuclear model,
Phys. Rev. {\bf 79}, 432 (1950).

%%%%  Bohr  Nobel  lecture
\bibitem{bohr_nobel}
Bohr, A.
Rotational Motion in Nuclei.
In {\it Nobel Lectures, Physics 1971—1980}, 
Lundqvist S., Ed.; World Scientific: Singapore, 1992; pp. 213--232;
https://www.nobelprize.org/prizes/physics/1975/bohr/facts/.

%%%%  Atkins
\bibitem{atkins}
Atkins, P., de Paula, J., Keeler J.,
Physical Chemistry (Oxford University Press, Oxford, 2018).

%nudat3
\bibitem{nudat3} 
National Nuclear Data Center. NuDat 3.0.\\
https://www.nndc.bnl.gov/nudat3/

%%%  Jahn-Teller
\bibitem{jahn_1937}
Jahn, H. A.,  Teller, E., 
Stability of Polyatomic Molecules in Degenerate Electronic States I - Orbital Degeneracy,
Proc. R. Soc. A, \textbf{161}, 220, (1937).

%%%%  Rowe book
\bibitem{rowe_book}
Rowe, D. J., 
Nuclear collective motion:: models and theory, 
(World Scientific, Singapore, 2010).

%%%%  de Shalit Feshbach book
\bibitem{deshalit_book}
De Shalit, A., Feshbach, H., 
Nuclear Structure (theoretical Nuclear Physics), 
(John Wiley and Sons, New York, 1974).

%%%  Ring - Schuck  Book
\bibitem{ring_schuck_book}
Ring, P., Schuck, P., 
The Nuclear Many-Body Problem,
(Springer-Verlag: Berlin, 1980).
 
%%%%  FISSION 2 papres
\bibitem{nix_fissionshape}
Nix, J. R., 
Further studies in the liquid-drop theory of nuclear fission,
Nucl. Phys. A {\bf 130}, 241 (1969).  

\bibitem{ichikawa_fission}
Ichikawa, T., Iwamoto, A., M\"oller, P., Sierk, A. J.. 
Contrasting fission potential-energy structure of actinides and mercury isotopes,
Phys. Rev. C {\bf 86}, 024610 (2012).

%%%  axial symmetry
\bibitem{kumar_baranger1968}   %preponderance of axially symmetric shapes  bes_sorensen1969
Kumar, K.,  \&  Baranger, M., 
Nuclear deformations in the pairing-plus-quadrupole model (III). 
Static nuclear shapes in the rare-earth region, 
Nucl. Phys. A  \textbf{110}, 529 (1968).   %–554
Proc. R. Soc. A, \textbf{161} 220, (1937).

\bibitem{bes_sorensen1969}   %axially symmetric shapes
Bes, D.R., \& Sorensen, R.A., 
The Pairing-Plus-Quadrupole Model. 
{\it Advances in Nuclear Physics}, Eds: Baranger, M, \& Vogt, E., (Plenum Press, New York, 1969), vol. ??,
(Benjamin, New York, 1975), Vol. II.

%  Elliott   QQ, rotation 
\bibitem{elliott_1958a}
Elliott, J.P.,
Collective motion in the nuclear shell model I.  Classification schemes for states of mixed configurations,  
Proc. Roy. Soc. (London) A {\bf 245}, 128 (1958).

%  Elliott   QQ, rotation 
\bibitem{elliott_1958b}
Elliott, J.P.,  
Collective motion in the nuclear shell model II.  The introduction of intrinsic wave-functions, 
Proc. Roy. Soc. (London) A {\bf 245}, 562 (1958).

\bibitem{caurier_2005}
Caurier, E., Martínez-Pinedo, G., Nowacki, F., Poves, A., Zuker, A. P.,
The shell model as a unified view of nuclear structure,
Rev. Mod. Phys. {\bf 77}, 427 (2005)

%ensdf
\bibitem{ensdf} 
National Nuclear Data Center. Evaluated Nuclear Structure Data File.\\
http://www.nndc.bnl.gov/ensdf/.

% Bohr-Mottelson vol. I   ˆ   For W.u.
\bibitem{bohr_mottelson_book1}
Bohr, A. \& Mottelson, B.R. Nuclear Structure (Benjamin, New York, 1969), Vol. I.

%%%%  MCSM   review
\bibitem{mcsm_2001}
Otsuka, T., Honma, M., Mizusaki, T., Shimizu, N. \& Utsuno, Y.,
Monte Carlo Shell Model for Atomic Nuclei,
Prog. Part. Nucl. Phys. {\bf 47}, 319-400 (2001).

%%%%  MCSM   original papers
\bibitem{mcsm_1995}
Honma, M., Mizusaki, T., Otsuka, T.,  
Diagonalization of Hamiltonians for Many-Body Systems by Auxiliary Field Quantum Monte Carlo Technique,
Phys. Rev. Lett. \textbf{75}, 1284 (1995).

\bibitem{mcsm_1998}
Otsuka, T., Mizusaki, T., Honma, M.,   
Structure of the N=Z=28 Closed Shell Studied by Monte Carlo Shell Model Calculation,
Phys. Rev. Lett. \textbf{81}, 1588 (1998).

\bibitem{mcsm_2012}
Shimizu, N. \etal, 
New-generation Monte Carlo shell model for the K computer era, 
Prog. Theor. Exp. Phys. \textbf{2012}, 01A205 (2012).

%%%%%%%%%%%%%   MCSM   examples 
\bibitem{togashi_2016}
Togashi, T., Tsunoda, Y., Otsuka, T.,  \& Shimizu, N.,
Quantum Phase Transition in the Shape of Zr isotopes, 
Phys. Rev. Lett. {\bf 117}, 172502 (2016).

\bibitem{leoni_2017}
Leoni, S., \etal., 
Multifaceted Quadruplet of Low-Lying Spin-Zero States in $^{66}$Ni: Emergence of Shape Isomerism in Light Nuclei, 
Phys. Rev. Lett. {\bf 118}, 162502 (2017).

\bibitem{togashi_2018}
Togashi, T., Tsunoda, Y., Otsuka, T., Shimizu, N.,  \& Honma, M.,
Novel Shape Evolution in Sn Isotopes from Magic Numbers 50 to 82, 
Phys. Rev. Lett. {\bf 121}, 062501 (2018).

\bibitem{marsh_2018} 
Marsh, B. A. \etal 
Characterization of the shape-staggering effect in mercury nuclei, 
Nature Physics \textbf{14}, 1163-1167 (2018).

\bibitem{ichikawa_2019}
Ichikawa, Y. \etal,
Interplay between nuclear shell evolution and shape deformation revealed by the magnetic
moment of $^{75}$Cu,
Nature Physics {\bf 15}, 321 (2019).

\bibitem{taniuchi_2019}
Taniuchi, R.  \etal, 
$^{78}$Ni revealed as a doubly magic stronghold against nuclear deformation,
Nature {\bf 569}, 53 (2019).

%%%%   Otsuka  self-organization
\bibitem{otsuka_2019}
Otsuka, T., Tsunoda, Y., Abe, T., Shimizu, N., Van Duppen, P.,   
Underlying Structure of Collective Bands and Self-Organization in Quantum Systems,
Phys. Rev. Lett. \textbf{123}, 222502 (2019).

\bibitem{marginean_2020}
M\u{a}rginean, S., \etal.,
Shape Coexistence at Zero Spin in $^{64}$Ni Driven by the Monopole Tensor Interaction,
Phys. Rev. Lett. {\bf 125}, 102502 (2020)

\bibitem{tsunoda_2020}
Tsunoda, N., Otsuka, T., Takayanagi, K., Shimizu, N., 
Suzuki, T., Utsuno, Y., Yoshida, S., \& Ueno, H.,
The impact of nuclear shape on the emergence of the neutron dripline,
Nature {\bf 587}, 66 (2020).

\bibitem{abe_2021}
Abe, T., Maris, P., Otsuka, T., Shimizu, N., Utsuno, Y., Vary, J. P.,
Ground-state properties of light $4n$ self-conjugate nuclei in ab initio no-core Monte Carlo shell model calculations with nonlocal $NN$ interactions, 
Phys. Rev. C {\bf 104}, 054315 (2021).

\bibitem{otsuka_2022_nature}
Otsuka, T., Abe, T., Yoshida, T., Tsunoda, Y., Shimizu, N., Itagaki, N., Utsuno, Y., Vary, J., Maris, P., Ueno, H.,
$\alpha$-Clustering in atomic nuclei from first principles with statistical learning and the Hoyle state character, 
Nature Communications {\bf 13}, 2234 (2022).

%%%%   QVSM
\bibitem{shimizu_2021}
Shimizu, N., Tsunoda, Y., Utsuno, Y., Otsuka, T.,
Variational approach with the superposition of the symmetry-restored quasiparticle vacua for nuclear shell-model calculations,
Phys. Rev. C {\bf 103}, 014312 (2021).

%%%%  VMU
\bibitem{otsuka_2010}
Otsuka, T., Suzuki, T., Honma, M., Utsuno, Y., Tsunoda, N., Tsukiyama, K., and Hjorth-Jensen, M., 
Novel features of nuclear forces and shell evolution in exotic nuclei, 
Phys. Rev. Lett. \textbf{104}, 012501 (2010).

%%%%  Brown  Pb interaction
\bibitem{brown_2000_Pb}
Brown, B. A., 
Double-Octupole States in $^{208}$Pb,
Phys. Rev. Lett. \textbf{85}, 5300 (2000).

%%%%  Formulas for beta2 and gamma
\bibitem{utsuno_2015}
Utsuno, Y., Shimizu, N., Otsuka, T., Yoshida, T., Tsunoda, Y.
Nature of Isomerism in Exotic Sulfur Isotopes.
Phys. Rev. Lett., {\bf 114}, 032501 (2015).

%%%%   beta2, gamma from quadrupole moments, T-plot 重心
\bibitem{otsuka_2022}
Otsuka, T., Shimizu, N., Tsunoda, Y., 
Moments and radii of exotic Na and Mg isotopes.
Phys. Rev. C {\bf 105}, 014319 (2022).

%%%%%%  T - p l o t    %%%%%%%%%%%
\bibitem{tsunoda_2014}
Tsunoda, Y., Otsuka, T., Shimizu, N., Honma, M., Utsuno,
 Y., Novel shape evolution in exotic Ni isotopes and configuration-dependent shell structure, 
Phys. Rev. C \textbf{89}, 031301(R) (2014).

%%%%  Samorajczyk-Pysk   K quantum number
\bibitem{polishK_2021}
Samorajczyk-Pysk, J., \etal, 
Low-spin levels in $^{140}$Sm: Five 0$^+$ states and the question of softness against nonaxial deformation,
Phys. Rev. C {\bf 104}, 024322 (2021).

%%%%  Tsunoda  Bohr Hamiltonian
\bibitem{tsunoda_2021}
Tsunoda, Y., Otsuka, T., 
Triaxial rigidity of $^{166}$Er and its Bohr-model realization,
Phys. Rev. C, {\bf 103}, L021303 (2021).

\bibitem{kumar_1972}
Kumar, K.,
Intrinsic Quadrupole Moments and Shapes of Nuclear Ground States and Excited States,
Phys. Rev. Lett. \textbf{28}, 249 (1972).

%%%%  REMARK1
\bibitem{remark1}
Here, the proton-neutron interaction actually means the multipole part of it.
Any given interaction can be decomposed  into the monopole part and the multipole
part, of which the former will be discussed in Sec.~\ref{sec:mono}.  The multipole part is the rest of the interaction, and can be the direct source to produce the surface deformation \textcolor{black}{of the ground and low-lying states of atomic nuclei.}

%%%%  5DCH
\bibitem{delaroche_2010}
Delaroche, J.-P., Girod, M., Libert, J., Goutte, H., Hilaire, S., P´eru, S., Pillet, N., Bertsch, G. F., 
Structure of even-even nuclei using a mapped collective Hamiltonian and the D1S Gogny interaction, 
Phys. Rev. C {\bf 81}, 014303 (2010).

%%%%%  Sharpey-Schafer
\bibitem{Sharpey-Schafer_2019}
Sharpey-Schafer, J. F., Bark, R. A., Bvumbi, S. P., Dinoko, T. R. S., and Majola, S. N. T.,
``Stiff'' deformed nuclei, configuration dependent pairing and the $\beta$ and $\gamma$ degrees of freedom,
Eur. Phys. J. A. {\bf 55}, 15 (2019).

%%%%%  YangSun
\bibitem{sun_2000}
Sun, Y., Hara, K., Sheikh, J. A., Hirsch, J. G., Vel\'azquez, V., Guidry, M.,
Multiphonon $\gamma$-vibrational bands and the triaxial projected shell model.
Phys. Rev. C {\bf 61}, 064323 (2000).

%%%%%  Fahlander
\bibitem{fahlander_1996} %garrett_1997
Fahlander, C., Axelsson, A., Heinebrodt, M., Hartlein, T., Schwalm, D.,
Two-phonon $\gamma$-vibrational states in $^{166}$Er,
Phys. Lett. B 388, 475 (1996).

%%%%%  P. Garrett
\bibitem{garrett_1997}
Garrett, P. E.; Kadi, M.; Li, Min; McGrath, C. A.; Sorokin, V.; Yeh, Minfang; Yates, S. W.,  
$K^{\pi}$=0$^+$ and 4$^+$ Two-Phonon $\gamma$-Vibrational States in $^{166}$Er,
Phys. Rev. Lett. {\bf 78}, 4545 (1997).

%%%%  Davydov model
\bibitem{davydov1}
Davydov, A.S., Filippov, G.F.,  
Rotational states in even atomic nuclei,
Nucl. Phys. {\bf 8}, 237 (1958).

\bibitem{davydov2}
Davydov, A.S., Rostovsky, V.S.,
Relative transition probabilities between rotational levels of non-axial nuclei.
Nucl. Phys. {\bf 12}, 58 (1959).  

%%%%   Deformed shell model   application
\bibitem{sun_2002}
Boutachkov, P., Aprahamian, A., Sun, Y., Sheikh, J.A., Frauendorf, S., 
In-band and inter-band B(E2) values within the Triaxial Projected Shell Model.
Eur. Phys. J. A {\bf 15}, 455 (2002).

%%%%  shell evolution  RMP
\bibitem{otsuka_2020} 
Otsuka, T.,  Gade, A., Sorlin, O., Suzuki, T., Utsuno, Y.
Evolution of shell structure in exotic nuclei, 
Rev. Mod. Phys. {\bf 92}, 015002  (2020).

%%%%   Emerging 
\bibitem{otsuka_2022_emerging} 
Otsuka, T.,
Emerging concepts in nuclear structure based on the shell model, 
Physics {\bf 4}, 258 (2022), https://www.mdpi.com/2624-8174/4/1/18.

%%%%   Shell evolution by tensor
\bibitem{otsuka_2005}
Otsuka, T., Suzuki, T., Fujimoto, R., Grawe, H., Akaishi, Y., 
Evolution of the nuclear shells due to the tensor force, 
Phys. Rev. Lett. \textbf{95}, 232502 (2005).

\bibitem{yukawa_1935}
Yukawa, H., 
On the Interaction of Elementary Particles. I,
Proc. Phys. Math. Soc. Japan {\bf 17}, 48 (1935)

\bibitem{bethe_1940}
Bethe, H. A., 
The Meson Theory of Nuclear Forces I. General Theory,
Phys. Rev. {\bf 57}, 260 (1940)

%\bibitem{bethe_1940b}
%Bethe, H. A., 
%The Meson Theory of Nuclear Forces. Part II. Theory of the Deuteron,
%Phys. Rev. {\bf 57}, 390 (1940)

\bibitem{osterfeld_1992}
Osterfeld, F.,
Nuclear spin and isospin excitations,
Rev. Mod. Phys. {\bf 64}, 491 (1992)

%%%%  Renormalization Persistency   N Tsunoda
\bibitem{ntsunoda_2011}
Tsunoda, N., Otsuka, T., Tsukiyama, K., Hjorth-Jensen, M., 
Renormalization persistency of the tensor force in nuclei,
Phys. Rev. C {\bf 84}, 044322 (2011).

%%%%  Tajima et al.  Skyrme systematics
\bibitem{tajima_1996}
Tajima, N., Takahara, S., Onishi, N.,
Extensive Hartree-Fock+BCS calculation with Skyrme SIII force,
Nucl. Phys. A {\bf 603} 23 (1996).

%%%%  P. Moller
\bibitem{moller_2006}
M\"oller, P., Bengtsson, R., Carlsson, G., Olivius, P., Ichikawa, T., 
Global Calculations of Ground-State Axial Shape Asymmetry of Nuclei,
Phys. Rev. Lett. \textbf{97}, 162502 (2006). 

%%%%%  Hayashi-Hara-Ring
\bibitem{hayashi_1984}
Hayashi, A., Hara, K., Ring, P., 
Existence of Triaxial Shapes in Transitional Nuclei.
Phys. Rev. Lett. {\bf 53}, 337 (1984).

%%%%%  Cline   168Er   9 deg
%  "The asymmetric rigid rotor using gamma=9deg reproduces the data well ..."  p.705
\bibitem{cline_1986}
Cline, D., 
NUCLEAR SHAPES STUDIED BY COULOMB EXCITATION,
Ann. Rev. Nucl. Part. Sci. {\bf 36}, 683 (1986).

%%%%%  Cline   168Er     8 deg.   energy-> 12deg
%\bibitem{cline_1990}
%Kotli\'nski, B., Cline, D., B\"acklin', A., {\it et al.},
%COULOMB EXCITATION OF $^{166}$Er,
%Nucl. Phys. A {\bf 517}, 365 (1990).

%%%   Fahlander  166Er   10deg   confusion for energy
\bibitem{fahlander_1992}
Fahlander, C., Thorslund, I., Varnestig, B., {\it et al.},
TRIAXIALITY IN $^{166}$Er,
Nucl. Phys. A {\bf 537}, 183 (1992).

%%%   Werner  164Dy 7 deg   158Gd   8 deg
\bibitem{werner_2005}  
Werner, V., Scholl, C., von Brentano, P., 
Triaxiality and the determination of the cubic shape parameter K$_3$ from five observables,
Phys. Rev. C {\bf 71}, 054314 (2005).

%%%%%%%%%%%%%%%%%%%%%%%%%%%%%%%%%%%%%%%%%%%%%%%
%%%%%  Weinberg
\bibitem{weinberg_1990} 
Weinberg, S., 
Nuclear forces from chiral lagrangians,
Phys. Lett. B {\bf 251}, 288 (1990).

%%%% chiral EFT
\bibitem{machleidt_2011}
Machleidt, R., Entem, D. R., 
Chiral effective field theory and nuclear forces,
Phys. Rep. {\bf 503}, 1 (2011).

%%%   scissors   Pietralla
\bibitem{pietralla_1998}  
Pietralla, N., von Brentano, P., Herzberg, R.-D., Kneissl, U., Lo Iudice, N., Maser, H., Pitz, H. H., Zilges, A., 
Systematics of the excitation energy of the 1$^+$ scissors mode and its empirical dependence on the nuclear deformation parameter,
Phys. Rev. C {\bf 58}, 184 (1998).

%%%%   Giacalone, Giuliano   Thesis   
\bibitem{giacalone_phdthesis_2020} 
Giacalone, G.,
A matter of shape: seeing the deformation of atomic nuclei at high-energy colliders, 
Ph.D thesis, Universit\'{e} Paris-Saclay, CNRS, CEA,
https://arxiv.org/abs/2101.00168.

\bibitem{frauendorf_2001}
Frauendorf, S.,
Spontaneous symmetry breaking in rotating nuclei,
Rev. Mod. Phys. {\bf 73}, 463 (2001)

%%%   algerian 
\bibitem{algerian_2017}  
Benrabia, K., Medjadi, D. E., Imadalou, M., Quentin, P.,
Triaxial quadrupole dynamics and the inner fission barrier of some heavy even-even nuclei,
Phys. Rev. C {\bf 96}, 034320 (2017).

%%%%  Leander
\bibitem{leander}   %axially symmetric shapes
Larsson, S. E., Leander, G., 
Fission Barriers for Heavy Elements with Quadrupole, Hexadecapole and Axially Asymmetric Distortions Taken into Account Simultaneously, 
{\it Physics and Chemistry of Fission 1973, Proc. of the 3rd IAEA Symposium on the Physics and Chemistry of Fission} (International Atomic Energy Agency, Vienna, 1974), vol. I, p. 177.
%%%%%%%%%%%% no editor in the usual sense


%%%%  Talmi book
\bibitem{talmi_book}
Talmi, I., 
Simple Models of Complex Nuclei: The Shell Model and Interacting Boson Model, 
(CRC Press, Boca Raton, 2017).


%shell   dimension  limit  
\bibitem{shimizu_2019}
Shimizu, N., Mizusaki, T., Utsuno, Y. \& Tsunoda, Y.,
Thick-Restart Block Lanczos Method for Large-Scale Shell-Model Calculations, 
Comp. Phys. Comm. \textbf{244}, 372 (2019).

%%%%%%%%%%%%%%%

%%%  J. Phys. Review
\bibitem{otsuka_2016}
Otsuka, T.  and Tsunoda, Y.,
The role of shell evolution in shape coexistence,
J. Phys. G, \textbf{43}, 024009 (2016).

%%%%%%%%%%%%%%%
\bibitem{lawson_method}
Gloeckner, D. H. and Lawson, R. D.,
Spurious Center-of-Mass Motion,
Phys. Lett. B \textbf{53}, 313 (1974).


%%%   polish 
%\bibitem{polish_2017}  
%Jachimowicz, P., Kowal, M., Skalski, M.,
%Effect of non-axial octupole shapes in heavy and superheavy nuclei,
%Phys. Rev. C {\bf 95}, 034329 (2017).

%%%%   K computer  使ってない?
%%\bibitem{Kcomputer}
%% K computer, https://www.r-ccs.riken.jp/en/k-computer/about/.
 
%%%%   Fugaku
% \bibitem{Fugaku}
% Fugaku computer, https://www.r-ccs.riken.jp/en/fugaku/.

%%%%%  END the BIBLIOGRAPHY %%%%%%%%%%%%

\end{thebibliography}

%%%%%%%%%%%%%%%%%%%%%%%%%%%%%%%
%%%   Acknowledgements   %%%

\end{document}


\section*{Author Contributions}
T.O. promoted the whole study and wrote the manuscript; Y.T. performed the CI calculations, drew T-plots and contributed to the code development; Y.U. contributed to various in-depth discussions; N.S. made the main part of the computer codes; T.A., P.V.D. and H.U. made valuable discussions.   
All authors discussed the results and commented on the manuscript.








