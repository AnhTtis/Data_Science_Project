% ****** Start of file apssamp.tex ******
%
%   This file is part of the APS files in the REVTeX 4.2 distribution.
%   Version 4.2a of REVTeX, December 2014
%
%   Copyright (c) 2014 The American Physical Society.
%
%   See the REVTeX 4 README file for restrictions and more information.
%
% TeX'ing this file requires that you have AMS-LaTeX 2.0 installed
% as well as the rest of the prerequisites for REVTeX 4.2
%
% See the REVTeX 4 README file
% It also requires running BibTeX. The commands are as follows:
%
%  1)  latex apssamp.tex
%  2)  bibtex apssamp
%  3)  latex apssamp.tex
%  4)  latex apssamp.tex
%
\documentclass[aps,reprint,nofootinbib,nobibnotes,notitlepage,superscriptaddress,onecolumn,prd,
%superscriptaddress,
%groupedaddress,
%unsortedaddress,
%runinaddress,
%frontmatterverbose, 
%preprint,
%preprintnumbers,
%nobibnotes,
%bibnotes,
 amsmath,amssymb
%floatfix,
]{revtex4-2}

\usepackage[caption=false]{subfig}
\usepackage{braket}
\usepackage{float}
\usepackage{lipsum}
% \usepackage[margin=0.75in]{geometry}
\usepackage{graphicx}% include figure files
\usepackage{dcolumn}% align table columns on decimal point
\usepackage{bm}% bold math
\usepackage{natbib}
\usepackage{hyperref}
\hypersetup{
	colorlinks = true,
    linkcolor = Red,
    urlcolor  = Red,
    citecolor = Red
}

\usepackage{microtype}

\usepackage{verbatim}
\usepackage[amssymb]{SIunits}
\usepackage{tabularx}
\usepackage[dvipsnames]{xcolor}
\usepackage{wasysym}

\usepackage{tikz}

\renewcommand{\arraystretch}{1.3}
\def\be{\begin{equation}}
\def\ee{\end{equation}}


%\usepackage[mathlines]{lineno}% Enable numbering of text and display math
%\linenumbers\relax % Commence numbering lines

%\usepackage[showframe,%Uncomment any one of the following lines to test 
%%scale=0.7, marginratio={1:1, 2:3}, ignoreall,% default settings
%%text={7in,10in},centering,
%%margin=1.5in,
%%total={6.5in,8.75in}, top=1.2in, left=0.9in, includefoot,
%%height=10in,a5paper,hmargin={3cm,0.8in},
%]{geometry}

% Comment commands
\usepackage{color}
\definecolor{darkgreen}{RGB}{0,120,0}
\newcommand{\oliver}[1]{{\color{red}[Oliver: #1]}}
\definecolor{darkgreen}{RGB}{0,120,0}
\newcommand{\gc}[1]{{\color{darkgreen}[Gio: #1]}}

% Useful commands

\newcommand{\Mpch}{h^{-1}\mathrm{Mpc}}
\newcommand{\hMpc}{h\,\mathrm{Mpc}^{-1}}
\newcommand{\inn}[2]{\left(\left.{#1}\right|{#2}\right)}
\newcommand{\delD}[1]{(2\pi)^3\delta_\mathrm{D}\left({#1}\right)}
\newcommand{\nhat}{\hat{\vec{n}}}
\newcommand{\av}[1]{\left\langle{#1}\right\rangle} 
\newcommand{\obs}{\mathrm{obs}}
\newcommand{\vk}{\vec k}
\newcommand{\hk}{\hat{\vec k}}
\newcommand{\vK}{\vec K}
\newcommand{\vp}{\vec p}
\newcommand{\vm}{\vec m}
\newcommand{\vl}{\vec l}
\newcommand{\vL}{\vec L}
\newcommand{\vd}{\vec d}
\newcommand{\vs}{\vec s}
\newcommand{\vq}{\vec q}
\newcommand{\vx}{\vec x}
\newcommand{\hx}{\hat{\vec x}}
\newcommand{\vy}{\vec y}
\newcommand{\vn}{\vec n}
\newcommand{\Hi}{\mathsf{H}^{-1}}
\newcommand{\C}{\mathsf{C}}
\newcommand{\F}{\mathcal{F}}
\newcommand{\G}{\mathcal{G}}
\newcommand{\M}{\mathcal{M}}
\newcommand{\B}{\mathsf{B}}
\newcommand{\T}{\mathsf{T}}
\newcommand{\Q}{\mathsf{Q}}
\newcommand{\Sig}{\mathsf{S}}
\newcommand{\N}{\mathsf{N}}
\newcommand{\Ci}{\mathsf{C}^{-1}}
\newcommand{\tCi}{\tilde{\mathsf{C}}^{-1}}
\newcommand{\g}[1]{g_\alpha^{({#1})}}
\newcommand{\tr}[1]{\operatorname{tr}\left({#1}\right)}
\newcommand{\ft}[1]{\mathcal{F}\left[{#1}\right]}
\newcommand{\ift}[1]{\mathcal{F}^{-1}\left[{#1}\right]}
\newcommand{\fid}{\mathrm{fid}}
\newcommand{\Tr}[1]{\operatorname{Tr}\left[{#1}\right]}
\newcommand{\hR}{\hat{\vec R}}
\newcommand{\hr}{\hat{\vec r}}
\newcommand{\hn}{\hat{\vec n}}
\newcommand{\hz}{\hat{\vec z}}
\newcommand{\hs}{\hat{\vec s}}
\newcommand{\tjo}[3]{\begin{pmatrix} {#1} & {#2} & {#3}\\ 0 & 0 & 0\end{pmatrix}}
\newcommand{\tj}[6]{\begin{pmatrix} {#1} & {#2} & {#3}\\ {#4} & {#5} & {#6}\end{pmatrix}}
\newcommand{\cg}[3]{\left\langle {#1} ; {#2} | {#3} \right\rangle}
\newcommand{\expect}[1]{\mathbb{E}\left[{#1}\right]}
\renewcommand{\vr}{\vec r}
\renewcommand{\L}{\Lambda}
\renewcommand{\P}{\mathcal{P}}

\def\beq{\begin{eqnarray}}
\def\eeq{\end{eqnarray}}
\def\k{\textbf{k}}
\def\bfk{\textbf{k}}
\def\bfs{\textbf{s}}
\def\bft{\textbf{t}}
\def\bfu{\textbf{u}}
\let\vec\mathbf


\usepackage{empheq}

\newcommand*\widefbox[1]{\fbox{\hspace{2em}#1\hspace{2em}}}

% Journal names
\def\mnras{\rm{MNRAS}}
\def\aj{\rm{AJ}}
\def\aap{\rm{A\&A}}
\def\baas{\rm{BAAS}}
\def\physrep{\rm{Phys.~Rep.}}
\def\apj{\rm{ApJ}}                 % Astrophysical Journal
\def\apjl{\rm{ApJ}}                % Astrophysical Journal, Letters
\def\apjs{\rm{ApJS}}               % Astrophysical Journal, Supplement
\def\jcap{\rm{JCAP}}
\def\pasj{\rm{PASJ}}


\begin{document}

\title{Cross-Field Trispectra}
%\date{\today}

\author{Oliver~H.\,E.~Philcox}
\email{ohep2@cantab.ac.uk}
%\affiliation{Center for Theoretical Physics, Department of Physics,
%Columbia University, New York, NY 10027, USA}
%\affiliation{Simons Society of Fellows, Simons Foundation, New York, NY 10010, USA}

\begin{abstract} 
    %\noindent We derive optimal estimators for the two-, three-, and four-point correlators of statistically isotropic scalar fields defined on the sphere, such as the Cosmic Microwave Background temperature fluctuations, allowing for arbitrary (linear) masking and inpainting schemes. In each case, we give the optimal unwindowed estimator (obtained via a maximum-likelihood prescription, with an associated Fisher deconvolution matrix), and an idealized form, and pay close attention to their efficient computation. For the trispectrum, we include both parity-even and parity-odd contributions, as allowed by symmetry. The estimators can include arbitrary weighting of the data (and remain unbiased), but are shown to be optimal in the limit of inverse-covariance weighting and Gaussian statistics. The normalization of the estimators is computed via Monte Carlo methods, with the rate-limiting steps (involving spherical harmonic transforms) scaling linearly with the number of bins. An accompanying code package, \href{https://github.com/oliverphilcox/PolyBin}{\textsc{PolyBin}}, implements these estimators in \textsc{python}, and we demonstrate the estimators' efficacy via a suite of validation tests.
\end{abstract}

\maketitle


We define the cross-field full-sky parity-even unnormalized scalar trispectrum estimator as
\beq
    \widehat{\rho}[abcd]^{\ell_1\ell_2}_{\ell_3\ell_4}(L) = \sum_{m_1\cdots m_4M}(-1)^{M}\G^{\ell_1\ell_2L}_{m_1m_2(-M)}\G^{\ell_3\ell_4L}_{m_3m_4M}a_{\ell_1m_1}b_{\ell_2m_2}c_{\ell_3m_3}d_{\ell_4m_4}.
\eeq
Let's define a bin by $\Theta_\ell(b)$ equal to unity if $\ell$ is in $b$ and zero else. The binned estimator is thus
\beq\label{eq: binned-rho}
    \widehat{\rho}[abcd](\vec b, B) = \sum_{m_1\cdots m_4M}\sum_{\ell_1\cdots \ell_4L}\Theta_{\ell_1}(b_1)\cdots \Theta_{\ell_4}(b_4)\Theta_L(B)\times (-1)^{M}\G^{\ell_1\ell_2L}_{m_1m_2(-M)}\G^{\ell_3\ell_4L}_{m_3m_4M}a_{\ell_1m_1}b_{\ell_2m_2}c_{\ell_3m_3}d_{\ell_4m_4},
\eeq
where $\vec b\equiv\{b_1,\cdots,b_4\}$. Consider the following contribution to the above:
\beq
    I[ab](b_1,b_2;\ell',m') &\equiv& \sum_{\ell_1\ell_2m_1m_2}\Theta_{\ell_1}(b_1)\Theta_{\ell_2}(b_2)\times\G^{\ell_1\ell_2\ell'}_{m_1m_2m'}a_{\ell_1m_1}b_{\ell_2m_2}\\\nonumber
    &=&\left(\int d\hn\,Y^*_{\ell'm'}(\hn)[a]_{b_1}(\hn)[b]_{b_2}(\hn)\right)^*\\\nonumber
\eeq
with $[x]_b(\hn) = \sum_{\ell m}Y_{\ell m}x_{\ell m}\Theta_\ell(b)$, which is an inverse harmonic transform (and assumed to be real), and we have written the Gaunt integral as an integral over spherical harmonics. We can thus obtain $I[ab]$ as an inverse harmonic transform. 

In full;
\beq
    \widehat{\rho}[abcd](\vec b, B) = \sum_{LM}\Theta_L(B)\times \bigg(I[ab](b_1,b_2;L,M)\bigg)^*\bigg(I[cd](b_3,b_4;L,M)\bigg);
\eeq
this is just a sum in harmonic space over $(L,M)$. As such, we can evaluate $\widehat{\rho}$ for all fields, without resorting to computing $\ell$-dependent quantities. Note that this doesn't subtract off the disconnected terms; this is possible as for the previous estimators, except that we now need to include $\ell$-space binning. [NB: the ``ideal'' version of the PolyBin estimator does all this mechanics, except that it uses spins $(-1,-1,2)$ instead of $(0,0,0)$, adds weights $S_\ell$, and normalizes to get the full trispectrum quantity.]

How does the binned $\widehat{\rho}$ relate to the unbinned one? We could, in principle, add any $\ell$-dependent weighting we like when defining \eqref{eq: binned-rho}. Adding an factor that is not multiplicatively separable in $\ell_i$ will break the factorizability however, so our only choice is something of the form $w_{\ell_1}w_{\ell_2}w_{\ell_3}w_{\ell_4}W_L$ (for some $w,W$). Assuming that the quantity entering our formulae is $\widehat{\rho}[abcd]^{\ell_1\ell_2}_{\ell_3\ell_4}(L)$ directly and ignoring any weights, we can just form an estimate for the unbinned quantity as
\beq
    \widehat{\rho}[abcd]^{\ell_1\ell_2}_{\ell_3\ell_4}(L) = \frac{1}{N(\ell_1,\ell_2,\ell_3,\ell_4,L)}\widehat{\rho}[abcd]^{b(\ell_1)b(\ell_2)}_{b(\ell_3)b(\ell_4)}(b(L)),
\eeq
where $b(\ell)$ is the bin containing $\ell$. Here, the normalization is the number of $\ell$'s entering the $\ell$ summation in \eqref{eq: binned-rho}; this is given by
\beq
    \sum_{\ell_1\ell_2\ell_3\ell_4L}\Theta_{\ell_1}(b_1)\cdots \Theta_{\ell_4}(b_4)\Theta_L(B) \times \Delta(\ell_1,\ell_2,L)\Delta(\ell_3,\ell_4,L) &=& \sum_{L}\Theta_L(B)\left[\sum_{\ell_1\ell_2}\Theta_{\ell_1}(b_1)\Theta_{\ell_2}(b_2)\Delta(\ell_1,\ell_2,L)\right]\\\nonumber
    &&\qquad\times\,\left[\sum_{\ell_3\ell_4}\Theta_{\ell_3}(b_3)\Theta_{\ell_4}(b_4)\Delta(\ell_3,\ell_4,L)\right],
\eeq
where $\Delta = 1$ if the triangle conditions are obeyed and zero else.

An alternative, and likely profitable, approach would be to directly compute the summed-over-$\ell$ quantities entering the calculation of interest. This probably just returns the power spectrum however, making this approach likely moot...

 
%\acknowledgments
% {\small
% \noindent We thank Will Coulton, Cyril Creque-Sarbinowski, Colin Hill and, in particular, Adri Duivenvoorden for insightful discussions. OHEP is a Junior Fellow of the Simons Society of Fellows and thanks the inhabitants of Bukit Lawang for inspiration. The author is pleased to acknowledge that the work reported in this paper was substantially performed using the Princeton Research Computing resources at Princeton University, which is a consortium of groups led by the Princeton Institute for Computational Science and Engineering (PICSciE) and the Office of Information Technology's Research Computing Division.
% }
%\clearpage


\appendix

\bibliographystyle{apsrev4-1}
\bibliography{refs}% Produces the bibliography via BibTeX.

\end{document}
%
% ****** End of file apssamp.tex ******
