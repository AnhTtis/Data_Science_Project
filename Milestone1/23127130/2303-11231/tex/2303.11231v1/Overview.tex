\section{First operation: Delayed extension}

Given a positive integer $s$, we denote by $[s]$ the set $\{1, 2, \ldots, s\}$. If $G$ is a graph, $\chi(G)$ denotes its chromatic number and $\omega(G)$ its clique number. When $X$ is a subset of vertices, we denote by $G[X]$ the graph induced by $G$ on $X$. We recall that $X\subseteq V(G)$ is a \emph{module} of $G$ if for every $y\in V(G)\setminus X$ we have all edges between $y$ and $X$, or no edge between $y$ and $X$.

We first show that every graph $G$ on vertex set $v_1, \ldots, v_n$ has a canonical decomposition tree $T_d$, called \emph{delayed decomposition tree}. Its leaves are the vertices of $G$, and every internal node $x$ of $T_d$ is labelled by a graph $G_x$ defined on the grandchildren of $x$. It is the analogue of the usual decomposition tree for modules, where every $G_x$ is defined on children instead of grandchildren (hence the "delayed"), with the noticeable difference that \emph{every graph} is entirely decomposable. 

The tree $T_d$ is defined via a sequence of refining partitions $P_0,P_1,\dots ,P_k$ of the vertex set $V$ of $G$ starting with the root vertex $P_0=\{V\}$ and ending on $P_k=\{\{v_1\}, \ldots, \{v_n\}\}$, the partition into singletons corresponding to the vertices. We now describe how to construct the refinement $P_{i+1}$ of $P_i=\{B_1, \ldots, B_m\}$, that is how a part $B_j$ is further refined. In this process, all parts will consist of consecutive vertices in the ordering $v_1, \ldots, v_n$.

For this, we define a partition $P(I)$ of an arbitrary interval of consecutive vertices $I=\{v_i,\dots, v_j\}$.

\begin{itemize}
    \item If $i=j$, it is not refined.
    \item If $I$ is a module, we divide it into two parts $\{v_i,\dots ,v_{\lfloor (i+j)/2\rfloor}\}$ and $\{v_{\lfloor (i+j)/2\rfloor+1},\dots, v_j\}$.
    \item If $I$ is not a module, we partition it into maximal subsets $S$ of consecutive vertices such that $S$ is a module in $G[(V\setminus I)\cup S]$. We call these parts \emph{local modules}.
   In other words $v_s,v_{s+1}\in I$ are in the same local module of $I$ if and only if they have the same neighbours in $V\setminus I$.
\end{itemize}

 To form the refinement $P_{i+1}$ of $P_i=\{B_1, \ldots, B_m\}$, we just refine every $B_j$ into $P(B_j)$. Observe that since $P_0=\{V\}$ is a module, we have $P_1 = \{\{v_1, \ldots, v_{\lceil n/2 \rceil}\}, \{v_{\lceil n/2 \rceil + 1}, \ldots, v_n\}\}$. We stop the process when $P_k=P_{k-1}$, which happens when all parts are singletons. For technical reasons (made clearer in the next definition) we keep the two identical partitions into singletons $P_{k-1},P_k$ instead of simply stopping at step $k-1$.

We next consider the tree $T_d$ corresponding to this decomposition process, where the nodes at depth $i$ correspond to the parts of $P_i$, and the children of a node $x\in P_i$ are the parts $y\in P_{i+1}$ such that $y\subseteq x$ (we usually identify the nodes of $T_d$ to subsets of $V$). The leaves of $T_d$ are the vertices $v_i$ of $G$. Moreover, the parent of a leaf $v_i$ is $v_i$ since $P_{k-1}=P_k$. We now describe how to structure $T_d$ as a \emph{delayed tree decomposition} in order to encode the graph $G$. The crucial remark here is that if a node $x$ of $T_d$ has two grandchildren $y,z$ which are not siblings (we say that $y,z$ are \emph{cousins}, i.e. their parents are distinct), then we have all edges between $y$ and $z$, or no edge between them. In other words, cousins are modules with respect to each other.

From this observation, we define a function $g$ associating to every node $x\in T_d$ a graph $G_x:=g(x)$ whose vertex set is the set of  grandchildren of $x$ and such that $yz$ is an edge of $G_x$ if $y,z$ are cousins and there exists an edge between $y$ and $z$ in $G$ (and thus $y$ is fully joined to $z$). 

Given a pair $(T,g)$, now simply seen as a rooted tree $T$ in which the parent of every leaf only has one child, and each $g(x)$ is a graph on the grandchildren of $x$ (if any, otherwise $g(x)$ is empty), we define the \emph{realization} $R(T,g)$ as the graph such that: 
\begin{itemize}
    \item its vertex set is the set of leaves $L$ of $T$,
    \item two vertices $x,y\in L$ are joined by an edge if, given that $z$ is their closest ancestor in $T$ and $x',y'$ are the respective grandchildren of $z$ which are the ancestors of $x,y$, the edge $x'y'$ belongs to $g(z)$.
\end{itemize}

The crucial observation is that $G$ is equal to $R(T_d,g)$, hence every graph can be expressed as a delayed decomposition tree. Given a class of graphs $\cal C$, we denote by ${\cal C}_d$ the class of graphs $G$ admitting a delayed tree decomposition $(T_d,g)$ (for some enumeration of their vertex set) in which all graphs $g(x)$ belong to $\cal C$. We call ${\cal C}_d$ the \emph{delayed extension} of $\cal C$. It is not strictly speaking a closure since applying it twice can produce more graphs than applying it once.

The delayed extension is very similar to the \emph{substitution closure} ${\cal C}_s$ of $\cal C$, in which $G\in {\cal C}_s$ if all its induced \emph{prime} subgraphs $H$ (i.e. with no modules of size $k$ where $1<k<|V(H)|$) belong to $\cal C$. We have ${\cal C}_s\subseteq {\cal C}_d$, but we shall not need it. Conversely, we can express ${\cal C}_d$ in terms of ${\cal C}_s$.
\begin{lemma}\label{lem:delayedpartition}
Every graph in ${\cal C}_d$ is the edge union of two graphs in ${\cal C}_s$.
\end{lemma}
\begin{proof}
Let $G:=R(T,g)$ be a graph such that $g(x)\in \cal C$ for all nodes of $T$. We consider a function $g_o$ (o for odd)  defined on $T$ such that  $g_o(x)=g(x)$ if $x$ has odd depth in $T$ and $g_o(x)$ is the edgeless graph (on the grandchildren of $x$) if $x$ has even depth. We define analogously $g_e$ in which $g_e(x)=g(x)$ if $x$ has even depth, and is edgeless otherwise. We define $G_o:=R(T,g_o)$ and $G_e:=R(T,g_e)$.

By construction, $G$ is the edge union $G_o\cup G_e$. We now show that $G_o$ is in ${\cal C}_s$. Observe that every node $x$ of $T$ with odd depth  (seen as a subset of vertices of $G$) is a module of $G_o$. Indeed, for every vertex $v$, $x$ is a module of $G_o[x\cup v]$: this is by definition of $G_o$ if $v$ is not a descendant of a sibling of $x$, and if $v$ is a descendant of a sibling of $x$ there is no edge between $v$ and $x$ since their closest ancestor, the parent of $x$, has even depth.
In particular, if $H$ is a prime induced subgraph of $G_o$, we consider the deepest node $x$ with odd depth such that $V(H)\subseteq x$, and since $|V(H)\cap y|\leq 1$ for every grandchild $y$ of $x$ ($H$ is prime), we have that $H$ is an induced subgraph of $g(x)$, hence $H\in \cal C$. The same argument holds for $G_e$.
\end{proof}

We now recall the result of Chudnovsky, Penev, Scott and Trotignon~\cite{CPST13}.

\begin{theorem}\label{th:cpst}
If $\cal C$ is polynomially $\chi$-bounded with function $\omega^k$, then ${\cal C}_s$ is $\chi$-bounded by $\omega^{3k+11}$.
\end{theorem}

\begin{corollary}\label{cor:delay}
If $\cal C$ is polynomially $\chi$-bounded with function $\omega^k$, then ${\cal C}_d$ is $\chi$-bounded by $\omega^{6k+22}$.
\end{corollary}

\begin{proof} This directly follows from Lemma~\ref{lem:delayedpartition} and the fact that $\chi (G)\leq \chi (G_o)\chi (G_e)$.
\end{proof}

Thus, it suffices to focus on $\cal C$ if we want to show a polynomial $\chi$-bounding function for ${\cal C}_d$. The key to simplify a graph $G$ is an enumeration of its vertices for which the canonical delayed decomposition $(T_d,g)$ satisfies that all $g(x)$ are \emph{simpler} than $G$. This is better said for classes: A class ${\cal C}_0$ can be \emph{delayed} to ${\cal C}_1$ if ${\cal C}_0=({{\cal C}_1})_d$. For instance, cographs are the class $({{\cal G}_2})_d$ where ${\cal G}_2$ are the graphs of size at most 2 (indeed, it is $({{\cal G}_2})_s$). Let us say that a class ${\cal C}_0$ can be \emph{finitely delayed} to a class ${\cal C}_k$ if there exist ${\cal C}_0, {\cal C}_1,\dots ,{\cal C}_k$ such that ${\cal C}_i=({{\cal C}_{i+1}})_d$ for all $i=0,\dots ,k-1$. Also, we say that ${\cal C}_0$ has \emph{finite delay} if it can be finitely delayed to ${\cal G}_2$. By Corollary~\ref{cor:delay}, every class with finite delay is \pcb, as well as every class which can be finitely delayed to a \pcb~class. 

An exciting line of future research would be to explore which classes of graphs have finite delay to simpler classes. This could prove useful for understanding $\chi$-boundedness. Of course, it is a particular case of a more complex question asking if a graph can be edge-partitioned into two simpler graphs, but it has a great advantage: looking for arbitrary edge-partitions can be extremely complex due to the huge number of possibilities. Focusing instead on delayed decompositions only requires guessing the right vertex-ordering, which is a deeply explored field in graphs. This is why delayed decompositions behave so well with twin-width, which has alternative definitions involving vertex orderings.






