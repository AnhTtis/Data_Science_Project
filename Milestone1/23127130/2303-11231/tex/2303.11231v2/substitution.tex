\section{More on the $\chi$-boundedness of substitution-closure}\label{sec:subs}


The goal of this section is to provide a proof of the following theorem by slightly modifying the original argument of Chudnovsky, Penev, Scott and Trotignon~\cite{CPST13}.


\begin{theorem}\label{th:chibordec}
If $\cal C$ is hereditary and \pcb~with function $\chi \leq \omega^k$, then ${\cal C}_s$ is \pcb~with function $\chi \leq \omega^{2k+3}$
\end{theorem}

Given a class of graphs $\cal C$, a \emph{$\cal C$-tree-decomposition} is a pair $(T,g)$ in which $T$ is a rooted tree and $g$ is a function associating to every internal node $x$ of $T$ a graph $g(x)\in \cal C$ whose vertices are the children of $x$. The \emph{realization} $R(T,g)$ is the graph such that: 
\begin{itemize}
    \item its vertex set is the set of leaves $L$ of $T$
    \item two vertices $x,y\in L$ are joined by an edge if, given that $z$ is their closest ancestor in $T$ and $x',y'$ are the respective children of $z$ which are the ancestors of $x,y$, the edge $x'y'$ belongs to $g(z)$.
\end{itemize}

Given a class $\cal C$, we denote by ${\cal C}_s$ the class of all $R(T,g)$ where $(T,g)$ is a $\cal C$-tree-decomposition. For instance, if $\cal C$ is the class of cliques and independent sets, then  ${\cal C}_s$ is the class of cographs.
We say that $(T,g)$ is \emph{independent} if whenever $x, y$ are internal nodes of $T$ with the same parent $z$, then $xy$ is not an edge of $g(z)$. We denote by ${\cal C}_i$ the class of all $R(T,g)$ where $(T,g)$ is an independent $\cal C$-tree-decomposition. \\
Let $\cal C$ be a class of graphs, and $(T, g)$ be a $\cal C$-tree-decomposition. 
Let $u$ be any node of $T$ which is not the root, and let $v$ be the parent of $u$. We say that $u$ is \emph{isolated} in $T$ if it is isolated in $g(v)$. \\
The \emph{depth} $d(x)$ of a node $x$ of $T$ is the number of strict ancestors of $x$ in $T$ that are not isolated.
The \emph{depth} $d(T, g)$ of $(T, g)$ is the maximum depth of a leaf of $T$. 
Finally, if $G \in \mathcal{C}_s$, the \emph{depth} of $G$ is the minimum depth of a $\cal C$-tree-decomposition $(T, g)$ such that $G = R(T, g)$.

\begin{lemma}\label{lem:depth}
For every class $\cal C$, if $(T, g)$ is a $\mathcal{C}$-tree-decomposition of depth $d$, then $\omega(R(T, g)) \geq d + 1$.
\end{lemma}

\begin{proof} We consider a leaf $x$ with depth $d$. We denote by $y_1,\dots, y_d$ the non-isolated ancestors of $x$, and by $z_1,\dots, z_d$ their respective parents. We also denote by $w_1,\dots, w_d$ some respective children of $z_1,\dots, z_d$ such that $y_iw_i$ is an edge in $g(z_i)$. We now pick some leaves $x'_1,\dots ,x'_d$ which are respective descendants of $w_1,\dots, w_d$. Observe that $x,x'_1,\dots ,x'_d$ is a clique of $R(T, g)$.
\end{proof}

\begin{theorem}\label{th:inddec}
If $\cal C$ is \pcb~with function $\chi \leq \omega^k$, then ${\cal C}_i$ is \pcb~with function $\chi \leq \omega^{k+1}$
\end{theorem}

\begin{proof}
Let $G \in \mathcal{C}_i$. There exists an independent $\cal C$-tree-decomposition $(T, g)$ such that $G = R(T, g)$. By Lemma \ref{lem:depth}, we have $\omega(G) \geq d(T, g) + 1$. For every internal node $x$, we have $g(x) \in \cal C$ and $\omega(g(x)) \leq \omega(G)$, thus $g(x)$ is $\omega(G)^k$-colorable. For every internal node $x$, let $\alpha_x$ be an $\omega(G)^k$-coloring of $g(x)$. Let $v$ be any vertex of $G$ (equivalently $v$ is a leaf of $T$). If $T$ has only one node, then $G$ is the graph on a single vertex, so $G$ is $\omega(G)^{k+1}$-colorable. Otherwise, let $x$ be the parent of $v$ in $T$. We set $c(v) = (d(v), \alpha_x(v))$, where $d(v)$ is the depth of $v$ in $(T, g)$. \\
We claim that $c$ is a proper $\omega(G)^{k+1}$-coloring of $G$. First, for every vertex $v$ of $G$, we have $0 \leq d(v) \leq d(T, g) \leq \omega(G) - 1$, so we indeed use at most $\omega(G)^{k+1}$ colors. Then, let $uv$ be an edge of $G$. Let $z$ be the closest ancestor of $u$ and $v$ in $T$, and $u', v'$ be the respective children of $z$ which are the ancestors of $u, v$. The edge $u'v'$ belongs to  $g(z)$. Since $(T, g)$ is an independent $\cal C$-tree-decomposition, this implies that either $u'$ or $v'$ is a leaf (or both). If they are both leaves, we have $u = u'$ and $v = v'$, therefore $u$ and $v$ are children of $z$ which are adjacent in $g(z)$. Thus, $\alpha_z(u) \neq \alpha_z(v)$ so $c(u) \neq c(v)$. If only one of $u', v'$ is a leaf, say $u'$, then we have that $v'$ is a strict ancestor of $v$ which is not isolated, so $d(v) > d(v') = d(u') = d(u)$ hence $c(u) \neq c(v)$. 
\end{proof}

\begin{proof} We now prove \ref{th:chibordec} by induction on the depth of $G$. \\
If $G$ has depth 0, then $G$ is a disjoint union of graphs of $\cal C$, so $\chi(G) \leq \omega(G)^k \leq \omega(G)^{2k+3}$.
Now, suppose $G$ has depth 1. By a previous lemma, we have $\omega(G) \geq 2$. We can assume that $G$ is connected since we can deal with each connected component separately. Then $G$ can be obtained by starting from a graph $G_0 \in \mathcal{C}$ on vertex set $v_1, \ldots, v_n$ and substituting each $v_i$ by a graph $G_i \in \mathcal{C}$. Let $X = \{v_i, \omega(G_i) \leq \sqrt{\omega(G)}\}$ and $Y = V(G_0) \setminus X$. Let $G_X$ be the subgraph of $G$ induced by the leaves which are descendant of nodes of $X$ and $G_Y$ be the graph induced on the other vertices. The graph $G_X$ is obtained by substituting graphs of $\mathcal{C}$ of clique number at most $\sqrt{\omega(G)}$ inside a graph of $\mathcal{C}$ of clique number at most $\omega(G)$. Hence, $G_X$ can be colored with at most  $\omega(G)^{1.5k}$ colors (consider the product of the colorings). Similarly, the graph $G_Y$ is obtained by substituting graphs of $\mathcal{C}$ of clique number at most $\omega(G)$ inside a graph of $\mathcal{C}$ of clique number at most $\sqrt{\omega(G)}$ so it can be colored with at most  $\omega(G)^{1.5k}$ colors. Hence $G$ can be colored with $2\omega(G)^{1.5k} \leq \omega(G)^{1.5k+1} \leq \omega(G)^{2k+3}$ colors. \\

Now, suppose that $G$ has depth at least 2. Thus, $\omega(G) \geq 3$. Let $(T, g)$ be a $\mathcal{C}$-tree-decomposition of $G$. For every node $x$ of $T$, let $T_x$ be the subtree of $T$ that is rooted in $x$, $g_x$ be the restriction of $g$ to the internal nodes of $T_x$, and $G_x = R(T_x, g_x)$. Finally, let $X = \{x, \omega(G_x) > \omega(G)/2\}$. Note that if $z$ is the parent of $x$ in $T$, then $G_x$ is an induced subgraph of $G_z$, so if $x \in X$, then we also have $z \in X$. In particular, $T[X]$ is a subtree of $T$. Let $C(X)$ be the set of children of an element of $X$ in $T$ and $Y = X \cup C(X)$. Let $T' = T[Y]$. $T'$ is also a subtree of $T$. Let $g'$ be the restriction of $g$ to the internal nodes of $T'$, and let $H = R(T', g')$. Note that $H$ is an induced subgraph of $G$ so $\omega(H) \leq \omega(G)$. Now, suppose $x, y$ are internal nodes of $T'$ with the same parent $z$. Then, $x, y \in X$ so $\omega(G_x), \omega(G_y) > \omega(G)/2$. Thus, $x$ and $y$ cannot be adjacent in $g'(z)$ otherwise we would have $\omega(H) > \omega(G)$. This means that $(T', g')$ is an independent $\mathcal{C}$-tree-decomposition of $H$, so $H \in \mathcal{C}_i$. Thus, $H$ is $\omega(H)^{k+1} \leq \omega(G)^{k+1}$-colorable. \\
Let $v_1, \ldots, v_n$ be the vertices of $H$. We have that $G$ can be obtained by substituting each $v_i$ by some $G_{v_i} \in \mathcal{C}_s$. We can assume that $G$ is connected since we can deal with each connected component separately. Thus, we get that for every $i \in [n]$, $d(G_{v_i}) < d(G)$ (where $d$ stands for the depth). Futhermore, if $v_i$ is a vertex of $H$, it means that it is a leaf of $T'$. Indeed, since $\omega(G) \geq 3$, if $v_i \in X$ then $\omega(G_{v_i}) \geq 2$ so $v_i$ has children in $T$. Thus, $v_i \notin X$, which means $\omega(G_{v_i}) \leq \omega(G)/2$. Let $\omega_i = \omega(G_{v_i})$. By induction hypothesis, we have $\chi(G_{v_i}) \leq \omega_i^{2k+3}$. \\
Let $m = \lfloor\log(\omega(G))\rfloor$. For $j \in [m]$, let $V_j = \{v_i: \frac{\omega(G)}{2^{j+1}} < \omega_i \leq \frac{\omega(G)}{2^j}\}$. Note that the $V_j$'s partition  $\{v_1, \ldots, v_n\}$. For $j \in [m]$, let $H_j = H[V_j]$ and $G_j$ be the corresponding induced subgraph of $G$. 

If $j \in [m]$ and $v_i \in V_j$ then $\omega_i > \frac{\omega(G)}{2^{j+1}}$ so $\omega(H_j) \leq 2^{j+1}$. Thus, $\chi(H_j) \leq 2^{(j+1)(k+1)}$ because $H_j$ is an induced sugraph of $H\in \mathcal{C}_i$. Thus, we have $\chi(G_j) \leq 2^{(j+1)(k+1)} \cdot \left(\frac{\omega(G)}{2^{j}}\right)^{2k+3}$ (by doing the product of the colorings). Now, let's color separately each $V_j$ with a different palette. The number of colors used is: \begin{align*}
    \sum_{j=1}^m \chi(G_j) &\leq \sum_{j = 1}^m 2^{(j+1)(k+1)} \cdot \left(\frac{\omega(G)}{2^{j}}\right)^{2k+3} \\
                            &= 2^{k+1} \omega(G)^{2k+3} \sum_{j = 1}^m 2^{-j(k+2)} \\
                            &\leq 2^{k+1} \omega(G)^{2k+3} \frac{2^{-(k+2)}}{1-2^{-(k+2)}} \\
                            &\leq \omega(G)^{2k+3} \times \frac{2^{k+2}}{2(2^{k+2} - 1)} \\
                            &\leq \omega(G)^{2k+3}
\end{align*}
\end{proof}