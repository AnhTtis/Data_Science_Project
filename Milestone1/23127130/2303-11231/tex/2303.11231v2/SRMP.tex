\section{Second operation: Right extension}

Our goal in this section is to define an extension of a class of graphs $\cal C$ which preserves $\chi$-boundedness, and even polynomial $\chi$-boundedness when the twin-width of $\mathcal{C}$ is bounded. Given a graph $G$, a \emph{right module partition} (RMP) is a partition $V_1, \ldots, V_k$ of the vertices of $G$ such that \begin{enumerate}
    \item Each $V_i$ is a stable set.
    \item For every $i < j$, $V_i$ is a module with respect to $V_j$ (i.e. $V_i$ is a module in $G[V_i\cup V_j]$).
\end{enumerate}

Note that every graph $G$ has a trivial RMP where each $V_i$ consists of a single vertex. Therefore, there should be some limitations to the definition of RMP. A first attempt is to consider a class of graphs $\mathcal{C}$, and insist that every induced subgraph intersecting every $V_j$ on at most one vertex (called a \emph{transversal}) belongs to $\mathcal{C}$. 
Unfortunately, even RMP with forests transversals are not $\chi$-bounded. To see this, consider $S_{n,2}$, the $n$-th shift graph, whose vertex set is $\{(i, j), 1 \leq i < j \leq n\}$ and such that there is an edge between $(i, j)$ and $(i', j')$ if and only if $j = i'$. Observe that the graphs $S_{n,2}$ have unbounded chromatic number and are triangle-free. However, the partition $(V_2, \ldots, V_n)$ where $V_j = \{(i, j), 1 \leq i < j\}$ for $2 \leq j \leq n$ is an RMP (with $V_1$ empty) such that the only neighbours of $(i,j)\in V_j$ in parts $V_k$ where $k<j$ are in $V_i$. Thus if $(i,j)$ belongs to a transversal, its degree is at most one with respect to the vertices in $V_k$ with $k < j$. Hence, all transversals are forests, while the graphs $S_{n,2}$ are not $\chi$-bounded.


For this reason, we introduce a stronger notion of RMP, meant to preserve $\chi$-boundedness.
    If $\mathcal{P} = (V_1, \ldots, V_k)$ is an RMP of a graph $G$, for every $1 \leq j_1 < j_2 < \ldots < j_\ell \leq k$ and every $W_{j_1} \subseteq V_{j_1}, \ldots, W_{j_\ell} \subseteq V_{j_\ell}$, all non-empty, we denote by $G/\{W_{j_1}, \ldots ,W_{j_\ell}\}$ the graph on vertex set $[\ell]$ such that there is an edge $ii'$ if and only if there is an edge between $W_{j_i}$ and $W_{j_{i'}}$. We call such a graph a \emph{transversal minor} of $(G, \mathcal{P})$.
    Given a class $\mathcal{C}$, an RMP such that all transversal minors are in $\mathcal{C}$ is called a \emph{$\mathcal{C}$-RMP}. The class of graphs $G$ admitting a $\mathcal{C}$-RMP is denoted by $RM(\mathcal{C})$ and is called the \emph{right extension} of $\cal C$.

    \begin{figure}[h]
\centering
\begin{tikzpicture}
\foreach \x/\y/\z in {1/2/A, 1/3/B, 2/3/C, 1/4/D, 2/4/E, 3/4/F, 1/5/G, 2/5/H, 3/5/I, 4/5/J} {
  \node at (2.1*\y-2.1, 1.5*\x+1.5-1.5*\y) (\z) {\x, \y};
}
\node[draw,dotted,fit=(A)] (2) {};
\node[draw,dotted,fit=(B) (C)] (3) {};
\node[draw,dotted,fit=(D) (E) (F)] (4) {};
\node[draw,dotted,fit=(G) (H) (I) (J)] (5) {};
\draw (C) to [bend left = 70] (2.south);
\draw (E.west) to [bend left = 70] (2.south);
\draw (F) -- (3.east);
\draw (H.west) to [bend left = 70] (2.south);
\draw (I.west) to [bend left = 70] (3.east);
\draw (J) -- (4);
\end{tikzpicture}
\caption{The RMP for $S_{5, 2}$. Here, an edge means that we have all edges from the stable set on the left to the vertex on the right. Observe that every transversal is a forest, however we can form every graph on 4 vertices as a transversal minor.}
\end{figure}


Our goal is now to show that right extension preserves $\chi$-boundedness. We will express our result in the language of ordered graphs (graphs with a total order on vertices). An RMP for an ordered graph must respect the order, that is the parts of the partition must consist of consecutive vertices. The following result is not used in the proof that bounded twin-width graphs are \pcb, but it could be useful in other context and its proof is similar to a later argument.  
\begin{proposition}\label{prop:chibound} If $\mathcal{C}$ is a  $\chi$-bounded class of ordered graphs, then $RM(\mathcal{C})$ is $\chi$-bounded.
\end{proposition}

The proof of Proposition~\ref{prop:chibound} is heavily based on the following lemma about the clique number. Given an ordered graph $G$ and an RMP ${\mathcal P}=(V_1, \ldots, V_k)$, we denote by $G/\cal P$ the ordered graph obtained by contracting all parts of $\cal P$, i.e. $G / \{V_1, \ldots, V_k\}$. This is a particular transversal minor of $G$, with the property that $\chi (G) \leq \chi (G/{\cal P})$. 
A class $\cal C$ of ordered graphs is \emph{$h$-free} if it does not contain some ordered graph on $h$ vertices.

\begin{lemma} There exists a function $\phi$ such that for every $h$-free class $\cal C$ and every ordered graph $G$ with a $\cal C$-right module partition $\cal P$, we have $\omega (G/{\cal P})\leq \phi (\omega (G),h)$
\end{lemma}

\begin{proof}
    The proof is by induction on $\omega :=\omega(G)$ and $h$. 
    If $h=1$ or $\omega =1$ then $G$ is edgeless, so we can set $\phi(x,1)=\phi(1,y)=1$. 
    Now, $\omega \geq 2$ and $h \geq 2$, and we assume that we proved the existence of $\phi(\omega - 1, h)$ and $\phi(\omega, h-1)$.
    We denote $\mathcal{P} = (V_1, \ldots, V_k)$ and consider an ordered graph $H$ on vertices $v_1,\dots ,v_h$ which is not in $\cal C$. Observe that we can restrict ourselves to the case where $G/{\cal P}$ is a clique, as we can only consider a maximal clique of $G/{\cal P}$.
    
    Thus, for every $i < j \in [k]$, there is an edge between $V_i$ and $V_j$.
    Let $D_k$ be a subset of $V_k$ of minimal size such that there is an edge between $V_i$ and $D_k$ for every $i < k$. By minimality of $D_k$, for every $d \in D_k$ there exists $i_d \in [k-1]$ such that there is an edge between $V_{i_d}$ and $d$ (and thus $d$ dominates $V_{i_d}$) but no edge between $V_{i_d}$ and $D_k \setminus \{d\}$. We consider two cases depending on the size of $D_k$. 
    
    \begin{enumerate}
        \item If $|D_k| > \phi(\omega, h-1)+1$, we show that we reach a contradiction. Select some $x\in D_k$ and consider the set $\cal P '$ of $|D_k|-1$ parts $V_{i_d}$ as defined above, except for the part $V_{i_x}$ which is not selected in $\cal P '$. Since $G/\cal P'$ is a clique of size at least $\phi(\omega, h-1)+1$, it follows by induction that $(G[\cup {\cal P'}],\cal P')$ contains all ordered graphs of size $h-1$ as transversal minors. In particular, the ordered graph $H'=H\setminus v_h$ is a transversal minor of $\cal P '$.
        If $v_h$ is isolated in $H$, we reach a contradiction since $H'\cup x$ is isomorphic to $H$ and is a transversal minor of $(G,\cal P)$. Otherwise, observe that one can extend $H'$ in all possible ways as a transversal minor of $\cal P$ by selecting some vertices in $D_k$. Indeed, every $V_{i_d}$ corresponds to a vertex $d$ in $D_k$ which is only joined to $V_{i_d}$. We can then select vertices in $D_k$ to extend $H'$ to $H$, a contradiction.
        
        \item If $|D_k| \leq \phi(\omega, h-1)+1$. For $d \in D_k$, let $S_d$ be the set of neighbours of $d$ in $G$. In particular, $\omega(G[S_d]) \leq \omega - 1$. Furthermore, $\mathcal{P}$ induces by restriction a $\mathcal{C}$-RMP $\mathcal{P}_d$ of $G[S_d]$. We then have $\omega(G[S_d]/\mathcal{P}_d) \leq \phi(\omega - 1, h)$. By taking the union over all $d \in D_k$, we deduce $\omega(G/\mathcal{P}) \leq \phi(\omega - 1, h) \cdot (\phi(\omega, h-1) +1)+ 1$ (the additional +1 stands for the last class $V_k$ which is not dominated).
        
    
        
    \end{enumerate}

Therefore, we can choose $\phi(\omega, h) = \phi(\omega - 1, h) \cdot (\phi(\omega, h-1) +1)+ 1$.
\end{proof}

We are now ready to prove Proposition~\ref{prop:chibound}. If $\cal C$ has a $\chi$-bounding function $f$, by the fact that the class of all graphs is not $\chi$-bounded, there is a graph $H$ of size $h$ which is not in $\cal C$. Consider now any graph $G$ in $RM(\cal C)$ with clique number $\omega$ and $\mathcal{C}$-RMP $\cal P$. We have $$\chi (G)\leq \chi (G/{\cal P})\leq f(\omega (G/{\cal P}))\leq f(\phi(\omega(G),h)).$$

Therefore the function $f(\phi(\omega(G),h))$ is $\chi$-bounding for $RM(\cal C)$. This approach does not provide a polynomial bound if the class $\cal C$ is polynomially $\chi$-bounded. We could not prove (or disprove) that polynomial $\chi$-boundedness is preserved by RMP. However, this is the case when the twin-width of $\mathcal{C}$ is bounded, as shown in the following section. 