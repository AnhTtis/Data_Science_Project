\section{Twin-width and almost-mixed minors}
  
  We recall here some definitions and results related to twin-width. For a more intuitive and pedestrian introduction, see~\cite{BKTW21}. We adopt here the matrix point of view of twin-width, where every graph $G$ is represented via its symmetric adjacency matrix $(a_{u,v})$ where $u,v$ are over couples of vertices. The entry $a_{u,v}$ is 1 if $uv$ is an edge, 0 if $uv$ is not an edge, and $*$ if $u=v$. The addition of the $*$ symbol slightly simplifies some technicalities, but is not necessary for the argument.
  
    A $01*$-matrix is \emph{horizontal} if all its rows are constant. It is \emph{vertical} if all its columns are constant. It is \emph{constant} if it is both horizontal and vertical. It is \emph{mixed} if it is neither horizontal nor vertical, or if it has at least 2 rows and 2 columns and contains a $*$ entry.
A \emph{corner} in a matrix $M$ is a mixed $2 \times 2$ submatrix of $M$.


\begin{lemma}[\cite{BKTW21}]\label{lem:corner}
 A matrix is mixed if and only if it contains a corner.
\end{lemma}

   Let $M$ be a matrix. 
    A \emph{row partition} of $M$ is a partition of the rows of $M$ in which each part of the partition consists of consecutive rows. We define \emph{column partitions} in a similar way. 
    A \emph{division} $\mathcal{D}$ of $M$ is a pair $(\mathcal{R}, \mathcal{C})$ where $\mathcal{R}$ is a row partition of $M$ and $\mathcal{C}$ is a column partition of $M$. If $\mathcal{R}$ and $\mathcal{C}$ both have the same number of parts, say $k$, we say that $(\mathcal{R}, \mathcal{C})$ is a \emph{$k$-division} of $M$. In this case, we index the row blocks and the column blocks of $\mathcal{D}$ with integers from $[k]$ in the natural order of the blocks. 
    If $\mathcal{D}$ is a $k$-division of $M$, for $i, j \in [k]$, we denote by $\mathcal{D}[i, j]$ the intersection of the $i$-th row block with the $j$-th column block, which we call a \emph{zone} of $\mathcal{D}$.  If $R_i\in \mathcal{R}$ and $C_j\in\mathcal{C}$, we also adopt the notation $[R_i,C_j]$ for the zone $\mathcal{D}[i, j]$. It is a contiguous submatrix of $M$.
    
    We say that a zone of a matrix is \emph{mixed} if it is mixed as a submatrix.
    If $M$ is symmetric, we say that a division $(\mathcal{R}, \mathcal{C})$ is \emph{symmetric} if $\mathcal{R}$ and $\mathcal{C}$ partition rows and columns in the same way (i.e. $\mathcal{R}$ is the transpose of $\mathcal{C}$).
    
    Let $M$ be a matrix and $\mathcal{D}$ be a $d$-division of $M$. We say that $\mathcal{D}$ is a \emph{$d$-mixed minor} if each zone of $\mathcal{D}$ is mixed. If $M$ does not have any $d$-mixed minor, we say that $M$ is \emph{$d$-mixed free}. The twin-width parameter and mixed-minor freeness are functionally equivalent. In particular, it was shown in \cite{BKTW21} that if a graph has twin-width at most $d$, it has a vertex ordering for which the adjacency matrix of $G$ is $f_d$-mixed free for some constant $f_d$.


    We also recall the following result, which is a direct consequence of the Marcus-Tardos theorem, see \cite{MT04}:

\begin{theorem}\label{th:marcustardos}
    For every positive integer $d$, there is a constant $mt_d$ such that for every $d$-mixed free matrix $M$ and every $k$-division of $M$, the number of mixed zones is at most $mt_d \cdot k$
\end{theorem}
   
As we will perform some operations on our graphs (such as deleting edges and contracting subsets of vertices), we show in the next results how mixed zones are affected. Let $M$ be a $01*$-matrix (not necessarily an adjacency matrix) with exactly two row blocks $R,R'$ and two columns blocks $C,C'$, each of size at least two. 

\begin{lemma}\label{lem:4corner}
If all four zones of $M$ are mixed, there is a corner intersecting all zones.
\end{lemma}

\begin{proof}
This is clear if $M$ contains a $*$. Consider a non constant row $r$ in $[R,C]$ and a non constant column $c'$ in $[R',C']$. Let $e$ be the entry $(r,c')$. Let $c\in C$ such that $e\neq (r,c)$ and $r'\in R'$ such that $e\neq (r',c')$. Now $\{r,r'\},\{c,c'\}$ is a corner.
\end{proof}

The \emph{contraction} $M'=M/\{R,R';C,C'\}$ is the $2\times 2$-matrix obtained by keeping a single value $x$ for each of the 4 zones, with the following rule: $x$ is the maximum value of the zone according to the order $0<1<*$. Thus, we get $*$ as soon as there exists a $*$, and we get $0$ only if the zone is full $0$.

\begin{lemma}\label{lem:contract}
If $M'$ is mixed, then $M$ is mixed.
\end{lemma}

\begin{proof}
This is clear if $M'$ (thus equivalently $M$) contains a $*$. Let us assume by contrapositive that $M$ is non-mixed. If $M$ is vertical (resp. horizontal), then observe that $M'$ is also vertical (resp. horizontal).
\end{proof}
 
 We keep the same notations as before, and assume moreover that none of the four zones of $M$ is mixed (in particular $M$ has no value $*$). The \emph{horizontal-deletion} $M_H$ of $M$ is the matrix obtained from $M$ by setting all values to 0 in each zone which is not vertical (or equivalently each zone which is horizontal and non-constant). We similarly define the \emph{vertical-deletion} $M_V$.
 
\begin{lemma} \label{lem:deletion}
If $M_H$ is mixed, then $M$ is mixed.
\end{lemma}

\begin{proof}
 Let us assume by contrapositive that $M$ is non-mixed. By assumption, there is no $*$ in  $M_H$. If $M$ is vertical, then observe that $M_H$ is also vertical. Now if $M$ is horizontal, note that the zones $R,C$ and $R,C'$ are either both set to 0 (if they are not vertical), or both left as in $M$. In both cases the rows of $R$ in $M_H$ are constant. The same applies to $R'$, so $M_H$ is horizontal. 
\end{proof}


Here is the key-definition of \cite{PS22}.   A $d$-division $\mathcal{D}$ of a matrix $M$ is a \emph{$d$-almost mixed minor} if for every $i \neq j \in [d]$, the zone $\mathcal{D}[i, j]$ is mixed. If $M$ does not have any $d$-almost mixed minor, we say that $M$ is \emph{$d$-almost mixed free}. By extension, a graph is \emph{$d$-almost mixed free} if we can order its vertices in such a way that its adjacency matrix is $d$-almost mixed free.

Observe that every $d$-almost mixed free matrix is also $d$-mixed free. Conversely, every $d$-mixed free matrix is also $2d$-almost mixed free. Indeed, if $M$ has a $2d$-almost mixed minor, then merging the first $d+1$ row blocks, and the last $d+1$ column blocks gives a $d$-mixed minor of $M$. Note that every submatrix of a $d$-(almost) mixed free matrix is also $d$-(almost) mixed free, hence every subgraph of a $d$-almost mixed free graph is also $d$-almost mixed free. 

Let $G$ be a graph with an RMP $\mathcal{P} = (V_1, \ldots, V_k)$. We say that $(G,\mathcal{P})$ is \emph{$d$-almost mixed free}, if for every coarsening $\cal P'$ of $\cal P$ into $d$ parts $(V'_1, \ldots, V'_d)$, where each $V'_i$ consists of consecutive parts of $\cal P$, some zone $[V'_i,V'_j]$, where $i\neq j$, is not mixed in the adjacency matrix of $G$. Note that we only speak here of restrictions on symmetric divisions of $G$, which encompass much larger classes than bounded twin-width. 

\begin{lemma} If $(G,\cal P)$ is $d$-almost mixed free, then  $\omega (G/{\cal P})\leq \omega(G)^d$.
\end{lemma}

\begin{proof}
We write $\omega:=\omega(G)$ and denote $\mathcal{P} = (V_1, \ldots, V_k)$. Let $\phi$ such that $\omega(G/{\cal P}) \leq \phi(\omega, d)$. We have $\phi(\cdot, 1)=0$ (empty graph) and $\phi(1, \cdot) = 1$ (edgeless graph). We assume $\omega\geq 2$ and $d\geq 2$ and show that $\phi(\omega, d)= \phi(\omega-1, d)+\phi(\omega, d-1)+1$ upper bounds $\omega(G/{\cal P})$. We can restrict ourselves to a maximal clique of $G/{\cal P}$, so we can assume that there is an edge between $V_i$ and $V_j$ whenever $i < j \in [k]$.    

Let us consider the smallest $\ell$ such that $\omega(G[V_1\cup \dots \cup V_{\ell}])=\omega$. Denote by $Y$ the set $V_1\cup \dots \cup V_{\ell}$, and consider any $V_i$ where $i\geq \ell +1$. Note that $Y$ is not a module with respect to $V_i$, as some vertex in $V_i$ would dominate $Y$, hence forming a clique of size $\omega +1$ in $G$. Conversely, if $V_i$ is a module with respect to $Y$, since $\cal P$ is an RMP, and there exists an edge between all pairs of parts, $V_i$ would dominate $Y$, with the same contradiction.

Consider the graph $G'=G[V_{\ell+1}\cup \dots \cup V_{k}]$ and its RMP $\mathcal{P'} = (V_{\ell +1}, \ldots, V_k)$. We claim that $(G',\cal P')$ is $d-1$-almost mixed free, otherwise any $d-1$-almost mixed minor coarsening $(V'_1, \ldots, V'_{d-1})$ of $\mathcal{P'}$ could be extended to the $d$-almost mixed minor $(Y,V'_1, \ldots, V'_{d-1})$ of $(G,\cal P)$. 

Thus $k=\omega (G/\cal P)$ satisfies by induction that $k\leq \ell + \phi(\omega, d-1)$. And since the first $\ell -1$ parts do not contain a clique of size $\omega$, we have $k \leq \phi(\omega - 1, d) + 1 + \phi(\omega, d-1)= \phi(\omega, d)$. Setting  $\psi(\cdot,\cdot)=\phi(\cdot,\cdot)-1$, we have that $\psi (\omega,d)=\psi(\omega - 1, d) + \psi(\omega, d-1)$. Moreover, we both have $\psi (\omega,1)=-1$ and $\psi (1,d)=0$, so $\psi (\omega,d)\leq \binom{\omega + d - 2}{d-1}\leq \omega ^{d-1}$.
\end{proof}

\begin{proposition} \label{prop:RMPpcb} Let $\cal C$ be a class of graphs with polynomial $\chi$-bounding function $f(x)=x^c$. If $\cal P$ is a $\cal C$-RMP of $G$ such that $(G,\cal P)$ is $d$-almost mixed free, then $\chi (G)\leq \omega ^{cd}$.
\end{proposition}

\begin{proof}
   Since $\omega(G/{\cal P}) \leq \omega(G)^{d}$, we have $\chi(G)\leq \chi(G/{\cal P})\leq \omega(G/{\cal P})^c\leq \omega(G)^{cd}$.
\end{proof} 