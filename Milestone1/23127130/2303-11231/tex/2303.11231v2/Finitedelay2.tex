\section{Mixed extensions}

We conclude with another type of class extension which preserves  $\chi$-boundedness. Given an ordered graph $G$ with vertex ordering  $v_1,\dots ,v_n$ (denoted by $<$), we only ask here that the restriction to "mixed parts" of $G$ are $\chi$-bounded. Precisely, given a class $\cal C$ of graphs, we say that $(G,<)$ is a \emph{$\cal C$-mixed extension} if for every partition $\cal P$ into intervals $I_1,\dots ,I_k$, the subgraph $\text{Mix}(G,<,\cal P)$ of $G$ in which we only keep edges between intervals $I_s,I_t$ for which $I_s,I_t$ is mixed, belongs to $\cal C$. The class of graphs $G$ admitting a vertex-ordering $<$ such that $(G,<)$ is a $\cal C$-mixed extension is denoted by ME$(\cal C)$.

Observe that graphs with twin-width at most $d$ are mixed extensions of the class of graphs with chromatic number at most $f(d)$. To see this, observe that they admit vertex-orderings with no large mixed-minors, hence the Marcus-Tardos theorem implies that every partition $\cal P$ has a linear number of mixed zones. So $\text{Mix}(G,<,\cal P)$ has bounded chromatic number by degeneracy. Hence the fact that bounded twin-width graphs are $\chi$-bounded is a particular case of the following result (whose proof mimics the one of Proposition~\ref{prop:RMPpcb}):



\begin{theorem}\label{th:chimix}
If $\cal C$ is $\chi$-bounded, then ME$(\cal C)$ is $\chi$-bounded.
\end{theorem}

\begin{proof} Assume that all graphs $H$ in $\mathcal{C}$ satisfy $\chi(H)\leq f(\omega (H))$. Consider $G\in \cal C$ with clique number $\omega$ and assume that we have already proved that the chromatic number of every graph in $ME(\mathcal{C})$ with smaller value of $\omega$ is at most $c$. Consider $<$, a vertex ordering $v_1,\dots ,v_n$ of $G$ such that $(G,<)$ is a $\cal C$-mixed extension. Pick a smallest initial interval $I_1=v_1,\dots ,v_{i_1}$ such that $G[I_1]$ has a clique of size $\omega$. Then pick a smallest interval $I_2=v_{i_1+1},\dots ,v_{i_2}$ such that $G[I_2]$ has a clique of size $\omega$, otherwise pick all remaining vertices. Continue the process to get a partition $\cal P$ into intervals $I_1,\dots ,I_k$. Observe that if there is an edge between $I_s$ and $I_t$, where $s<t<k$, then $I_s,I_t$ is mixed, otherwise one of the intervals would be a module with respect to the other, and we would find a clique of size $\omega +1$. Therefore the chromatic number of $G$ is at most $(c+1)f(\omega )+c+1$ since $\chi(G[I_k])\leq c+1$ and  $G[I_1\cup \dots \cup I_{k-1}]$ is the edge union of two graphs, one of chromatic number at most $c+1$ inside the $I_i$'s, and one of chromatic number at most $f(\omega)$ across the $I_i$'s.
\end{proof}

