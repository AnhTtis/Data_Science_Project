\section{Polynomial $\chi$-boundedness of bounded twin-width graphs}

By Lemma~\ref{lem:mixedtww}, in order to prove our main result, Theorem~\ref{th:BT}, we just have to show that the class of $d$-mixed free graphs is polynomially $\chi$-bounded. Since mixed freeness is functionally
equivalent to almost mixed freeness, we only consider this last notion.


To prove that a class $\cal C$ is \pcb, a strategy is to show that every graph $G$ of $\cal C$ has a vertex-partition or an edge-partition into a bounded number of graphs, each of them belonging to some known \pcb~class $\cal C '$. We will use this argument several times here.

\begin{theorem}\label{thm:damf-pcb} The class of $d$-almost mixed free graphs is \pcb.
\end{theorem}

\begin{proof}
The proof is by induction on $d$. For $d=2$, if $G$ has a 2-almost mixed free adjacency matrix, $G$ is a cograph (see \cite{PS22}), hence $G$ is perfect so the property holds. Now, let $d \geq 3$ and consider a graph $G$, which is $d$-almost mixed free with respect to the vertex ordering $v_1,\ldots ,v_n$. We first partition the set of vertices $V' =\{v_2,\dots, v_{n-1}\}$ into four subsets $V'_{00},V'_{01},V'_{10},V'_{11}$ according to their neighbourhood in $\{v_1,v_n\}$. For instance, $V'_{01}=(V'\setminus N(v_1))\cap N(v_n)$. It suffices to show that all four graphs $G[V'_{ij}]$ belong to some \pcb~class. Hence, without loss of generality, we can assume that $V'$ is a module in $G$. 
    We now consider the delayed decomposition tree $(T_d,g)$ of $G':=G[V']$ and consider the class $\cal C$ containing all $g(x)$ and their induced subgraphs. By Corollary~\ref{cor:delay}, we just have to show that $\cal C$ is \pcb. The graphs $g(x)$ are obtained by starting from an interval $I=\{v_s,\dots, v_t\}$ of vertices of $G'$, partitioning it into local modules $L_1,\dots, L_k$, and then partitioning each local module into local submodules. Let $H$ be the graph we obtain from $G[I]$ by removing all edges inside all local modules $L_i$. Note that $H$ can also be obtained by substituting the vertices of $g(x)$ by stable sets. This does not change $\chi$ and $\omega$. Therefore, to show that $\mathcal{C}$ is \pcb, it suffices to prove the following.

\begin{claim} Let $I=\{v_s,\dots, v_t\}$ be any interval of vertices of $G'$. Consider its partition into local modules $L_1,\dots, L_k$ and denote by $H$ the graph on vertex set $I$ obtained from $G[I]$ by deleting the edges inside the local modules. Then, $H$ is a bounded (vertex and edge) union of graphs from hereditary \pcb~classes.
\end{claim}

Note that the claim holds when $I$ is a non trivial module since it is cut into two parts. Indeed, in this case $H$ is bipartite and thus $\chi$-bounded by 2. Assume now that we have local modules $L_1,\dots ,L_k$, and consider all pairs $i<j$ such that $G[L_i,L_j]$ is mixed (call \emph{mixed pairs}). The $L_i$'s form a $k$-division of the adjacency matrix of $G[I]$, which is $d$-mixed free. Thus, by Theorem \ref{th:marcustardos}, the graph $R$ on vertex set $[k]$ whose edges correspond to mixed pairs has at most $\frac{mt_d}{2} \cdot k$ edges. In particular, $R$ can be vertex-colored into $(mt_d+1)$ classes. In other words, one can partition the set of local modules into $(mt_d+1)$ subsets, in which local modules are not pairwise mixed. We denote by $L'$ such a subset of local modules. To prove our claim, we just have to show that $H':=H[L']$ belongs to a \pcb~class of graphs.

Observe that for every $i<j$ and $L_i,L_j\in L'$, we have that $L_i$ is a module in $H[L_i\cup L_j]$ (denoted $L_i\rightarrow L_j$) or $L_j$ is a module in $H[L_i\cup L_j]$, that is $L_j\rightarrow L_i$. Note that if we both have $L_j\rightarrow L_i$ and $L_i\rightarrow L_j$, we have all edges or no edge between $L_i$ and $L_j$. We now define two subgraphs $H'_{\rightarrow}$ and $H'_{\leftarrow}$ of $H'$: in $H'_{\rightarrow}$ we only keep the edges of $H'$ between pairs $L_i\rightarrow L_j$ where $i<j$, and in $H'_{\leftarrow}$ we only keep the edges of $H'$ between pairs $L_i\leftarrow L_j$ where $i<j$.
Note that $H'=H'_{\rightarrow}\cup H'_{\leftarrow}$, and thus we just have to show that (for instance) $H'_{\rightarrow}$ belongs to a \pcb~class of graphs.

The graph $H'_{\rightarrow}$ with the partition $L'$ is a right module partition. Note that the same holds for $H'_{\leftarrow}$ if we reverse the order of the local modules. 
We further partition $H'_{\rightarrow}$: let us say that a local module $L_i$ is \emph{left} if $i>1$ and there is a vertex $v_j$ among $v_1,v_2,\dots,v_{s-1}$ (i.e. to the left of $I$) which distinguishes $L_{i-1}$ from $L_i$. Precisely, $v_j$ is not joined in the same way to the last vertex of $L_{i-1}$ and to the first of $L_i$. If $L_i$ (with $i>1$) is not left, then it is \emph{right} (and indeed some vertex $v_j$ to the right of $I$ distinguishes $L_{i-1}$ from $L_i$). We neglect $L_1$ in this definition (it only adds 1 to the chromatic number of $H'$). We now partition $L'$ into $L'_{ri}$ containing all right local modules $L_i$ of $L'$, and $L'_{le}$ containing the left local modules. Again, by vertex partition, we just have to show that the RMP $H'_{\rightarrow,ri}$ which is the induced restriction of $H'_{\rightarrow}$ to $L'_{ri}$ is \pcb. To apply Proposition~\ref{prop:RMPpcb}, we first show that the transversal minors of  $(H'_{\rightarrow,ri},L'_{ri})$ are $d-1$-almost mixed free, which by induction implies that they are \pcb. Then we argue that $(H'_{\rightarrow, ri},L'_{ri})$ has no large almost mixed minor. It suffices here to show that it is $2d$-almost mixed free. These are our last results.

\begin{claim}[\cite{PS22}]\label{claim:twwreduction}
    Every transversal minor of  $(H'_{\rightarrow,ri},L'_{ri})$ is $d-1$-almost mixed free.
\end{claim}

\begin{proof}
Assume for contradiction that we can find a sequence of local modules $L'_1,\dots ,L'_t$ in $L'_{ri}$, each of them containing a non empty subset of vertices $W_1,\dots ,W_t$, such that the graph $Q=H'_{\rightarrow,ri}/\{W_1,\dots ,W_t\}$ has a $d-1$-almost mixed minor. The vertices of $Q$ are denoted $W=\{w_1,\dots ,w_t\}$, where $W_i$ is contracted to $w_i$. Moreover, there exist two partitions of $W$ into consecutive blocks of vertices $(R_1,\dots ,R_{d-1})$ and $(C_1,\dots ,C_{d-1})$ such that $Q$ is mixed on the zone $[R_i,C_j]$ whenever $i\neq j$ (thus all $R_i$ and $C_j$ have size at least 2). We now show how to "lift" these partitions to $G$ in order to get a contradiction.

Consider any partition ${\cal R}'=(R'_1,\dots ,R'_{d-1})$ of $I$ (where parts consist of consecutive local modules) satisfying that $w_i\in R_j$ implies $L'_i \subseteq R'_j$. Similarly ${\cal C}'=(C'_1,\dots ,C'_{d-1})$ partitions $I$ and $w_i\in C_j$ implies $L'_i \subseteq C'_j$. We now extend the partitions ${\cal R}',{\cal C}'$ of $I$ to the whole vertex set $V$ of $G$ by first setting $R'_1:=R'_1\cup \{v_1,\dots ,v_{s-1}\}$ and $C'_1:=C'_1\cup \{v_1,\dots ,v_{s-1}\}$, and then adding a new part $\{v_{t+1},\dots ,v_{n}\}=R'_d=C'_d$ to both ${\cal R}'$ and ${\cal C}'$. These new partitions are called ${\cal R}$ and ${\cal C}$. Observe that if we were working with $H'_{\rightarrow,le}$, we would have added the part $R'_0=C'_0=\{v_1,\dots ,v_{s-1}\}$ to both ${\cal R}'$ and ${\cal C}'$ and extended the parts $R'_{d-1}$ and $C'_{d-1}$ by adding $\{v_{t+1},\dots ,v_{n}\}$. 


We now show that ${\cal R}$ and ${\cal C}$ form a $d$-almost mixed minor for $G$, which will be our contradiction. We need to focus on two points: the added parts $R'_d,C'_d$ should be mixed with respect to the others, and the original mixed zones $[R_i,C_j]$ of $Q$ should yield mixed zones $[R'_i,C'_j]$ of $G$. We separate the two arguments:
\begin{itemize}
    \item Consider first some zone $[R'_i,C'_d]$ where $i<d$ (similar argument for $[R'_d,C'_i]$). By the fact that $R_i$ contains two vertices $w_a,w_b$, the part $R'_i$ contains the right local modules $L'_a,L'_b$ (where $a<b$). Focus now on the vertex $v_j$ which is the first vertex of $L'_b$, and note that $v_{j-1}\in R'_i$ since $L'_a\subseteq R'_i$. Since $L'_b$ is a right local module, there exists $v_k$, where $k>t$ such that $v_k$ is differently joined to $v_{j-1}$ and $v_j$. Recall that $v_n$ is joined in the same way to $v_{j-1}$ and $v_j$. Since $v_{j-1},v_j\in R'_i$ and $v_k,v_n\in C'_d$, they witness the fact that $[R'_i,C'_d]$ is mixed. 
    \item Now, consider any zone $[R'_i,C'_j]$ where $i,j<d$ and $i\neq j$. If the zone $[R_i,C_j]$ contains a $*$ (i.e. some $w_a$ both belongs to $R_i$ and $C_j$), then $[R'_i,C'_j]$ also contains a $*$. Otherwise, by Lemma \ref{lem:corner}, $R_i$ contains two vertices $w_a,w_b$ and $C_j$ contains two vertices $w_c,w_d$ such that $\{w_a,w_b\},\{w_c,w_d\}$ is a corner. Moreover, since there is no $*$ value, we have $a<b<c<d$ or $c<d<a<b$. Without loss of generality, we assume $a<b<c<d$. By Lemma~\ref{lem:contract}, the restriction of the adjacency matrix of $H'_{\rightarrow,ri}$ on $[W_a\cup W_b,W_c\cup W_d]$ is mixed since its contraction is the corner $\{w_a,w_b\},\{w_c,w_d\}$. So the submatrix $[L'_a\cup L'_b,L'_c\cup L'_d]$ is also mixed.
    By definition of $H'_{\rightarrow,ri}$, if $L'_a$ (or $L'_b$) is not a module with respect to $L'_c$ (or to $L'_d$), then the zone $[L'_a,L'_c]$ is set to 0. In other words, the adjacency matrix of $H'_{\rightarrow,ri}$ restricted to $[L'_a\cup L'_b,L'_c\cup L'_d]$, is the horizontal-deletion of the adjacency matrix of $G$. Thus the zone $[R'_i,C'_j]$ is mixed by Lemma~\ref{lem:deletion}.
\end{itemize}
\end{proof}
\begin{claim}
    The pair $(H'_{\rightarrow,ri},L'_{ri})$ is $2d$-almost mixed free.
\end{claim}

\begin{proof}
    Assume for contradiction that we can find a coarsening $W'_1,\dots ,W'_{2d}$ of $L'_{ri}$ which forms a $2d$-almost mixed minor of $H'_{\rightarrow,ri}$. We now set $W_i=W'_{2i-1}\cup W'_{2i}$ for all $i=1,\dots ,d$. 
    By Lemma~\ref{lem:corner} every (mixed) zone $[W_i,W_j]$ with $i\neq j$ of $H'_{\rightarrow,ri}$ contains a corner $\{w_a,w_b\},\{w_c,w_d\}$.  By Lemma~\ref{lem:4corner}, we can assume that $w_a,w_b,w_c,w_d$ belong respectively to $W'_{2i-1},W'_{2i},W'_{2j-1},W'_{2j}$, hence to respective distinct local modules $L'_a,L'_b,L'_c,L'_d$.
    We assume without loss of generality that $i<j$, and thus $a<b<c<d$. 
    By definition of $H'_{\rightarrow,ri}$, if $L'_a$ (or $L'_b$) is not a module with respect to $L'_c$ (or to $L'_d$), then the zone $[L'_a,L'_c]$ is set to 0. In other words, the adjacency matrix of $H'_{\rightarrow,ri}$ restricted to $[L'_a\cup L'_b,L'_c\cup L'_d]$, is the horizontal-deletion of the adjacency matrix of $G$. Hence the zone $[W_i,W_j]$ is mixed in $G$ by Lemma~\ref{lem:deletion}. Thus $G$ restricted to $W_1,\dots ,W_d$ has a $d$-almost mixed minor, a contradiction.
\end{proof}
We can conclude the proof of Theorem~\ref{thm:damf-pcb} by using the two previous Claims and Proposition~\ref{prop:RMPpcb}.
\end{proof}

Theorem~\ref{th:BT} now follows from Lemma~\ref{lem:mixedtww} and Theorem~\ref{thm:damf-pcb}. From the previous proof, the order of magnitude for the $\chi$-bounding function of graphs with no $d$-almost mixed minor is $\omega^{d^{O(d)}}$.
