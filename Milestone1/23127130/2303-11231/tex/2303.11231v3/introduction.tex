\section{Introduction}

One of the first questions regarding the chromatic number $\chi (G)$ of a graph $G$ is how it compares to the clique number $\omega (G)$. Indeed, while $\omega (G)\leq \chi (G)$, early constructions by Blanche Descartes \cite{BD54}, Zykov \cite{ZY52} and Mycielski \cite{MY55} show that there are triangle-free graphs with arbitrarily large $\chi$. This question was also one of the key motivations for the introduction of the probabilistic method by Erd\H{o}s \cite{ER59}, to get randomized constructions of graphs with large girth and large chromatic number.

\subsection{Polynomial $\chi$-bounded classes}
A hereditary class $\cal C$ of graphs (or just \emph{class} in this paper, i.e. closed under induced subgraphs)  is \emph{$\chi$-bounded} if there exists a function $f$ such that $\chi(G)\leq f(\omega (G))$ for every $G\in \cal C$. For instance, the well-known class of \emph{perfect graphs} is the (hereditary) class of graphs such that $f(x)=x$. When $f$ can be chosen as a polynomial, the class $\cal C$ is 
\emph{polynomially $\chi$-bounded}. Let us also say that $\cal C$ is \emph{$k$-initially $\chi$-bounded} if $f(i)$ exists for all $i\leq k$. 
Recently, Carbonero, Hompe, Moore and Spirkl~\cite{CHMS23} showed that there are classes which are $2$-initially $\chi$-bounded but not $\chi$-bounded. 
This result was then extended to arbitrary values of $k$, independently by Bria\'nski, Davies and Walczak~\cite{BDW22} and by Girão, Illingworth, Powierski, Savery, Scott, Tamitegama and Tan~\cite{GIPSSTT24}. Notably, there are classes which are $\chi$-bounded but not polynomially $\chi$-bounded, see~\cite{BDW22}. The field is developing at a very fast pace: for a recent survey, see Scott and Seymour~\cite{SS20}.

Polynomial $\chi$-boundedness is one of the tamest behaviours a class can have. The most natural example of \pcb~class is the class of perfect graphs, which can be defined by forbidden induced subgraphs (odd holes and antiholes as the Strong Perfect Graph Theorem asserts~\cite{CRST06}) and also admit some structural decompositions. This leads to three main directions of research: Which forbidden induced subgraphs give polynomial $\chi$-boundedness? 
Which structural parameters yield polynomial $\chi$-boundedness? Which operations preserve $\chi$-boundedness? These three questions are intimately connected. The simplest graphs which are \pcb, cographs, are altogether $P_4$-free graphs, have clique-width at most 2 and are the closure under substitutions of graphs of size at most 2.

From the forbidden induced subgraph perspective, the Strong Perfect Graph Theorem, proved by Chudnovsky, Robertson, Seymour
and Thomas~\cite{CRST06} is definitely the masterpiece of the field. One of the most influential questions in the domain was proposed by Gy\'arf\'as and Sumner: is the class of graphs excluding some fixed induced tree $\chi$-bounded? This is true for paths, but the $\chi$-bounding function is not known to be polynomial. Scott, Seymour and Spirkl~\cite{SSS23} showed that sparse graphs (i.e. with no $K_{t,t}$ subgraph) excluding a tree are \pcb. Liu, Schroeder, Wang and Yu showed it for $t$-broom free graphs~\cite{LSWY21}. 
Davies and McCarty~\cite{DM21} proved that circle graphs are polynomially $\chi$-bounded.

From the point of view of operations preserving (polynomial) $\chi$-boundedness, the landscape is less developed. Even deciding if modules can be safely contracted requires a careful argument. To this end,
Chudnovsky, Penev, Scott, and Trotignon~\cite{CPST13} showed that if a class is \pcb, its closure under substitutions is also \pcb. It is also natural to wonder whether combining two closure operations each preserving $\chi$-boundedness still preserves $\chi$-boundedness. This is unfortunately not always the case as substitutions and 2-cuts lead to triangle-free graphs with arbitrarily large $\chi$~\cite{BBDGT23}. On the positive side, Dvo\v{r}\'ak and  Kr\'al’ also showed in~\cite{DK12} that closure by bounded rank cuts also  preserves $\chi$-boundedness. For this, they introduced a new tree-decomposition which is reminiscent of our delayed decomposition tree.

\subsection{Twin-width}
With the exception of VC-dimension, classes of graphs with bounded complexity parameter are usually $\chi$-bounded. This is the case for bounded tree-width, as it implies bounded degeneracy, but also the case for rank-width, as shown in~\cite{DK12}. Building on this result,
Bonamy and Pilipczuk~\cite{BP19} showed that graphs with bounded clique-width are \pcb.

Twin-width was introduced in~\cite{BKTW21} as a new structural complexity measure, capturing at the same time minor-closed classes, strict classes of permutations and bounded clique-width classes.
It was shown in~\cite{BGKTW21} that graphs with bounded twin-width are $\chi$-bounded. In the same paper, a polynomial bound was posed as an open problem, which would extend the polynomial $\chi$-boundedness of graphs with bounded clique-width~\cite{BP19}.


A natural step when trying to achieve polynomial bounds is to first look for quasi-polynomial ones, that is of order $n^{\log^c n}$. For instance, $P_5$-free graphs are quasi-polynomially $\chi$ bounded, see~\cite{SSS22}. In their breakthrough result, Pilipczuk and Soko\l{}owski~\cite{PS22} showed the following result:

\begin{theorem}\label{th:PS}
For every $t \in \mathbb{N}$ there is a constant $\gamma_t$ such that every graph with twin-width at most $t$ and clique number $\omega$ has chromatic number bounded by $2^{\gamma_t \log^{4t+3}\omega}$.
\end{theorem}

Reaching quasi-polynomiality often requires the right tools and definitions, and adding a little bit more of structure can sometimes save the logarithmic term.
This paper is no exception, as the fundamental idea (a reduction to lower twin-width), was already proposed by Pilipczuk and Soko\l{}owski~\cite{PS22}. Their twin-width reduction is based on a clever definition, $d$-almost mixed minors, which are particularly well-behaved for induced subgraphs. 
They use the fact that twin-width is functionally equivalent to finding a particular vertex ordering $v_1,\dots ,v_n$ for which the adjacency matrix cannot be partitioned into $k\times k$ mixed blocks, see~\cite{BKTW21}. Here mixed means that there are at least two distinct rows and 
two distinct columns. However, this definition is too constrained and they relaxed it by lifting the condition for the diagonal blocks ($k$-almost mixed minors). The difficult technical part of their approach is to partition $G$ into some restrictions $G[v_i,\dots,v_j]$ which do not ``contain" a $k-1$-almost mixed minor, in order to apply induction. The ``containment" notion is a very clever mix of edge-partition and vertex-quotient. This is the central reduction of their argument.
Our proof also uses this reduction and inserts it inside a different analysis of the decomposition tree. As a result we obtain:

\begin{theorem}\label{th:BT}
For every $t \in \mathbb{N}$ the class of graphs with twin-width at most $t$ is polynomially $\chi$-bounded. 
\end{theorem}

\subsection{Overview of the proof}
The first step is to consider a decomposition tree analogous to the substitution tree used for modules. The idea is simple: we first start with the partition $$\{\{v_1, \ldots, v_{\lceil n/2 \rceil}\}, \{v_{\lceil n/2 \rceil + 1}, \ldots, v_n\}\}$$ and we greedily continue to partition every part $B$ into modules with respect to the outside of $B$. If we reach a real module, we c
ut it in half and iterate. We now have a decomposition tree $T_d$ whose leaves are the vertices of $G$.

The next step is to observe that we can structure $T_d$ (i.e. associate a graph $g(x)$ to every internal node $x$) so that the information $(T,g)$ is enough to retrieve $G$. This is our \emph{delayed decomposition}. The key-fact is that if the class of graphs $g(x)$ is \pcb, one can derive polynomial $\chi$-boundedness for $G$. This does not involve twin-width and is a general decomposition method: delayed extensions preserve (polynomial) $\chi$-boundedness.

When $G$ is $k$-almost-mixed free, we can argue as in~\cite{PS22} that all $g(x)$ are ``simpler". Here again, we encapsulate the argument into a general framework, \emph{right extensions}, for which we prove that they preserve $\chi$-boundedness for all graphs, and polynomial $\chi$-boundedness for bounded twin-width graphs.

To sum up our approach: we show that $d$-almost mixed free graphs are delayed extensions of (vertex and edge unions of) right extensions of $d-1$-almost mixed free graphs and bounded $\chi$ graphs.

\subsection{Future directions}

In this paper, we stress the roles of two main operations: delayed extensions and right extensions, which apply to general graphs. Whether they can be used for other classical problems on $\chi$-boundedness is left for future research.

The fact that delayed extensions preserve polynomial $\chi$-boundedness directly results from the stability of polynomial $\chi$-boundedness by substitution. Since we crucially need it, we take a closer look at the proof from \cite{CPST13}. We slightly simplify the argument and get a better bound, again by tweaking the decomposition tree, but the whole argument is still surprisingly non trivial.

We also describe a new operation, the mixed extension, which appears naturally in a proof of the $\chi$-boundedness of bounded twin-width classes. A mixed subgraph of an ordered graph $G$ is obtained by only keeping the edges between mixed pairs of intervals, for some vertex-partition of $G$ into intervals. When $G$ has bounded twin-width (and the order is $d$-mixed free), every mixed subgraph has bounded chromatic number (using degeneracy via Marcus-Tardos). We show more generally that if the class $\cal C'$ of mixed subgraphs of a class $\cal C$ is $\chi$-bounded, then $\cal C$ is also $\chi$-bounded. Thus, mixed extensions preserve $\chi$-boundedness.

