\section{Problem Setup}

\noindent {\bf Constrained multi-objective optimization w/ preferences.} Constrained MOO is the problem of optimizing $\mathbf{K} \geq 2$ real-valued objective functions $\{f_1(x), \cdots, f_{\mathbf{K}}(x)\}$, while satisfying $\mathbf{L}$ black-box constraints of the form $c_1 \geq 0, \cdots, c_\mathbf{L}(x) \geq 0$ over the given design space $\mathfrak{X}$. A function evaluation with the candidate parameters $\vec{x} \in \mathfrak{X}$ generates two vectors, one consisting of objective values and one consisting of constraint values $\vec{y} = (y_{f_1}, \cdots, y_{f_{\mathbf{K}}}, y_{c_1}, \cdots, y_{c_{\mathbf{L}}})$ where $y_{f_j} = f_j(x)$ for all $j \in \{1, \cdots, K\}$ and $y_{c_i} = C_i(x)$ for all $i \in \{1, \cdots, L\}$. We define an input vector $\vec{x}$ as feasible if and only if it satisfies all constraints. The input vector $\vec{x}$ {\em Pareto-dominates} another input vector $\vec{x'}$ if $f_j(\vec{x}) \leq f_j(\vec{x'}) \hspace{1mm} \forall{j}$ and there exists some $j \in \{1, \cdots, K\}$ such that $f_j(\vec{x}) < f_j(\vec{x'})$. 

The optimal solution of the MOO problem with constraints is a set of input vectors $\mathcal{X}^* \subset \mathfrak{X}$ such that no configuration $\vec{x'} \in \mathfrak{X} \setminus \mathcal{X}^*$ Pareto-dominates another input $\vec{x} \in \mathcal{X}^*$ and all configurations in $\mathcal{X}^*$ are feasible. The solution set $\mathcal{X}^*$ is called the optimal constrained {\em Pareto set} and the corresponding set of function values $\mathcal{Y}^*$  is called the optimal constrained {\em Pareto front}. The most commonly used measure to evaluate the quality of a given Pareto set is by calculating the Pareto hypervolume (PHV) indicator \cite{10.1145/1527125.1527138} of the corresponding Pareto front of $\vec{(y_{f_1}, y_{f_2}, \cdots, y_{f_{\mathbf{K}}}})$ with respect to a reference point $\vec{r}$. Our overall goal is to approximate the constrained Pareto set $\mathcal{X}^*$ by minimizing the total number of expensive function evaluations. When a preference specification $p$ over the objectives is provided, the MOO algorithm should prioritize producing a Pareto set of inputs that optimize the preferred objective functions.

\noindent {\bf Preferences over black-box functions.} The designer/practitioner can define input preferences over multiple black-box functions through the notion of preference specification, which is defined as a vector of scalars $\vec{p} = \{p_{f_1}, \cdots, p_{f_K}, p_{c_1}, \cdots, p_{c_L}\}$ with $0\leq p_i \leq1 $ and $\sum_{i \in \mathcal{I}} p_i = 1$ such that $\mathcal{I} = \{f_1, \cdots, f_K,c_1, \cdots, c_L\}$. Higher values of $p_i$ mean that the corresponding objective function $f_i$ is highly preferred. In such cases, the solution to the MOO problem should prioritize producing design parameters that optimize the preferred objective functions.