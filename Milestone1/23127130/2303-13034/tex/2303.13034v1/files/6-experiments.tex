\section{Experimental Setup and Results}

In this section, we present experimental evaluation of PAC-MOO and baseline methods on two challenging analog circuit design problems.

\noindent\textbf{Baselines.} We compare PAC-MOO with state-of-the-art constrained MOO evolutionary algorithms, namely, NSGA-II \cite{deb2002fast} and MOEAD \cite{4358754}. We also compare to the constrained MOO method, the Uncertainty aware search framework for multi-objective Bayesian optimization with constraints (USEMOC) \cite{belakaria2020uncertainty}. We evaluated two variants of USEMOC: USEMOC-EI and USEMOC-TS, using expected improvement (EI) and Thompson sampling (TS) acquisition functions.

\noindent\textbf{PAC-MOO}: We employ a Gaussian process (GP) with squared exponential kernel for all our surrogate models. We evaluated several preference values for the efficiency objective function.
PAC-MOO-0 refers to the preference being equal over all objectives and constraints. PAC-MOO-1 refers to assigning 80\% preference to the efficiency objective and equal importance to other functions and constraints, resulting in a preference value $p_i=0.5 \times 0.8=0.4$ for the efficiency. With PAC-MOO-2, we assign a total preference of 85\% to the objective functions with 92\% importance to the efficiency resulting in a preference value of $p_i=0.85 \times 0.92=0.782$. We assign equal preference to all other functions.  With PAC-MOO-3, we assign more importance to the objective functions by assigning a total of 0.65 preference to them and 0.35 to the constraints. Additionally, we provide 88\% importance to the efficiency resulting in a preference value of $p_i=0.65 \times 0.88=0.572$.

\noindent\textbf{Evaluation Metrics}: The \textit{Pareto Hypervolume (PHV)} indicator is a commonly used metric to measure the quality of the Pareto front \cite{phd/dnb/Zitzler99}. PHV is defined as the volume between a reference point and the Pareto front.  After each circuit simulation, we measure the PHV for all algorithms and compare them. To demonstrate the efficacy of the preference-based PAC-MOO, we compare different algorithms using the maximum efficiency of the optimized circuit configurations as a function of the number of circuit simulations.


\noindent {\bf Benchmarks: } { \em 1.Switched-Capacitor Voltage Regulator (SCVR) design optimization setup.}  
The constrained MOO problem for SCVR circuit design consists of 33 input design variables, nine objective functions, and 14 constraints. 
Every method is initialized with 24 randomly sampled circuit configurations. 
{\em 2. High Conversion Ratio (HCR) design setup.} 
The constrained MOO problem for HCR circuit design consists of 32 design variables, 5 objective functions, and 6 constraints.
Considering that the fraction of feasible circuit configurations in the design space is extremely low (around 4\%), every method is initialized with 32 initial feasible designs provided by a domain expert.

In all our preference-based experiments, we assign a preference value to the efficiency objective and assign all other black-box functions (the rest of the objectives and the constraints) equal preference.

It is noteworthy that neither evolutionary algorithms nor the baseline BO method USEMOC are capable of handling preferences over objectives. This is an important advantage of our PAC-MOO algorithm, which we demonstrate through our experiments.

\begin{figure}[ht!]
\begin{subfigure}{.5\textwidth}
\includegraphics[width=\textwidth]{figures/scvr_hv_plot.png}
\caption{Hypervolume - SCVR}\label{scvr_hv_plot}
\end{subfigure}
\begin{subfigure}{.5\textwidth}
\includegraphics[width=\textwidth]{figures/hcr_hv_plot.png}
\caption{Hypervolume - HCR}\label{hcr_hv_plot}
\end{subfigure}
\begin{subfigure}{.5\textwidth}
\includegraphics[width=\textwidth]{figures/scvr_max_efficiency_plot.png}
\caption{Efficiency - SCVR}\label{scvr_max_efficiency_plot}
\end{subfigure}
\begin{subfigure}{.5\textwidth}
\includegraphics[width=\textwidth]{figures/hcr_max_efficiency_plot.png}
\caption{Efficiency - HCR}\label{hcr_max_efficiency_plot}
\end{subfigure}
\caption{Hypervolume and Efficiency of optimized circuits with preferences vs. No of simulations}
\end{figure}

\noindent {\bf Hypervolume of Pareto set vs. No of circuit simulations.} Figures \ref{scvr_hv_plot} and \ref{hcr_hv_plot} show the results for PHV of Pareto set as a function of the number of circuit simulations for SCVR and HCR design, respectively. An algorithm is considered relatively better if it achieves higher hypervolume with a lower number of circuit simulations. 
We make the following observations. {\bf 1)} PAC-MOO with no preferences (i.e., PAC-MOO-0) outperforms all the baseline methods. This is attributed to the efficient information-theoretic acquisition function and the exploitation approach to finding feasible regions in the circuit design space. {\bf 2)} At least one version of USEMOC performs better than all evolutionary baselines: USEMOC-EI for both SCVR and HCR designs. These results demonstrate that BO methods have the potential for accelerating analog circuit design optimization over evolutionary algorithms .
{\bf 3)} The performance of PAC-MOO with preference (i.e., PAC-MOO-1,2,3) is lower in terms of the hypervolume since the metric evaluates the quality of general Pareto front, while our algorithm puts emphasis on specific regions of the Pareto front via preference specification. This behavior is expected, nevertheless, we notice that the PHV with  PAC-MOO-1 and PAC-MOO-2 is still competitive and degrades only when a significantly high preference is given to efficiency (PAC-MOO-3). 

\noindent {\bf Efficiency of optimized circuits with preferences.} Since efficiency is the most important objective for both SCVR and HCR circuits, we evaluate PAC-MOO by giving higher preference to efficiency over other objectives. Figures \ref{scvr_max_efficiency_plot} and \ref{hcr_max_efficiency_plot} show the results for maximum efficiency of the optimized circuit configurations as a function of the number of circuit simulations for SCVR and HCR design optimization. {\bf 1)} As intended by design, PAC-MOO with preferences outperforms all baseline methods, including PAC-MOO without preferences. {\bf 2)} The improvement in maximum efficiency of uncovered circuit configurations for PAC-MOO with preferences comes at the expense of loss in hypervolume metric as shown in Figure \ref{scvr_hv_plot} and Figure \ref{hcr_hv_plot}.  
