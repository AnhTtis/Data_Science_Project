\section{Summary}

Motivated by challenges in hard engineering design optimization problems (e.g., large design spaces, expensive simulations, a small fraction of configurations are feasible, and the existence of preferences over objectives), this paper proposed a principled and efficient Bayesian optimization algorithm referred to as PAC-MOO. The algorithm builds Gaussian process based surrogate models for both objective functions and constraints and employs them to intelligently select the sequence of input designs for performing experiments. The key innovations behind PAC-MOO include a scalable and efficient acquisition function based on the principle of information gain about the optimal constrained Pareto front; an effective exploitation approach to find feasible regions of the design space; and incorporating preferences over multiple objectives using a convex combination of the corresponding acquisition functions. Experimental results on two challenging analog circuit design optimization problems demonstrated that PAC-MOO outperforms baseline methods in finding a Pareto set of feasible circuit configurations with high hyper-volume using a small number of circuit simulations. With preference specification, PAC-MOO was able to find circuit configurations that optimize the preferred objective functions better. 

\vspace{1.0ex}

\noindent {\bf Acknowledgements.} The authors gratefully acknowledge the support from National Science Foundation (NSF) grants IIS-1845922, OAC-1910213, and SII-2030159. The views expressed are those of the authors and do not reflect the official policy or position of the NSF.
