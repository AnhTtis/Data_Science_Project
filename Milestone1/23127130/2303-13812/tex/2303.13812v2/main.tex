\documentclass[11pt]{amsart}

\usepackage[margin=1.1in]{geometry}
\usepackage{setspace}
\linespread{1.05}
\usepackage[parfill]{parskip}
\setlength{\parindent}{15pt}

\usepackage{amssymb}
\usepackage{graphicx}
\usepackage{hyperref}
\usepackage{comment}
\usepackage{mathtools}
\usepackage{multirow}

% Script font for denoting point processes
\usepackage{calrsfs}
\DeclareMathAlphabet{\pazocal}{OMS}{zplm}{m}{n}

%Drawing contour integrals
\usepackage{tikz}
\usetikzlibrary{decorations.markings}

%For drawing Young tableaux
\usepackage{ytableau}

% Theorem environments


\usepackage{amssymb}
\usepackage{graphicx}
\usepackage{hyperref}
\usepackage[shortlabels]{enumitem}

% Script font for denoting point processes
\usepackage{calrsfs}
\DeclareMathAlphabet{\pazocal}{OMS}{zplm}{m}{n}

%Drawing contour integrals
\usepackage{tikz}
\usetikzlibrary{decorations.markings}

%For drawing Young tableaux
\usepackage{ytableau}

% Theorem environments
\theoremstyle{plain}
\newtheorem{thm}{Theorem}[section]
\newtheorem{cor}[thm]{Corollary}
\newtheorem{lemma}[thm]{Lemma}
\newtheorem{prop}[thm]{Proposition}
\newtheorem{definition}[thm]{Definition}
\newtheorem{ex}[thm]{Example}

\theoremstyle{remark}
\newtheorem{remark}[thm]{Remark}

\numberwithin{equation}{section}


%Symbols and variable shortcuts
\newcommand{\eps}{\epsilon}
\newcommand{\C}{\mathbb{C}}
\newcommand{\T}{\mathbb{T}}

\newcommand{\B}{\mathbb{B}}
\newcommand{\F}{\mathcal{F}}
\newcommand{\Lb}{\pazocal{L}}
\newcommand{\R}{\mathbb{R}}

\newcommand{\Tb}{\mathbf{T}}
\newcommand{\X}{\pazocal{X}}
\newcommand{\Y}{\mathcal{Y}}
\newcommand{\Z}{\mathbb{Z}}
\newcommand{\lm}{\lambda}
\newcommand{\sst}{\Delta_n}
\newcommand{\st}{\Delta_{\infty}}
\newcommand{\ii}{\mathbf i}
\newcommand{\YD}{\mathbb{YD}}
\newcommand{\J}{\pazocal{J}}
\newcommand{\EE}{\mathbb{E}}

\renewcommand{\d}{\mathrm{d}}

%Operators
\newcommand{\pr}[1]{\mathbb{P}\left [ #1 \right]}
\newcommand{\E}[1]{\mathbb{E}\left [ #1 \right ]}
\newcommand{\ind}[1]{\mathbf{1}_{\{#1\}}}
\newcommand{\G}[1]{\Gamma \left(#1 \right)}
\newcommand{\dint}[2]{\frac{1}{(2\pi \mathbf{i})^2} \oint \limits_{C_z #1} dz \oint \limits_{C_w #2} dw\,}


\title{RECTANGULAR MATRIX ADDITIONS IN LOW AND HIGH TEMPERATURES}

\author{Jiaming Xu}



\address[J.X.]{Department of Mathematics, University of Wisconsin - Madison. jxu385@wisc.edu}


\begin{document}
\begin{abstract}
We study the addition of two random independent $M\times N$ rectangular random matrices with invariant distributions in two limit regimes, where the parameter beta (inverse temperature) goes to infinity and zero. In low temperature regime the random singular values of the sum concentrate at deterministic points, while in high temperature regime we obtain a Law of Large Numbers for the empirical measures. As a consequence, we deliver a duality between low and high temperatures. Our proof uses the type BC Bessel function as characteristic function of rectangular matrices, and through the analysis of this function we introduce a new family of cumulants, that linearize the addition in high temperature limit, and degenerate to the classical or free cumulants in special cases.
\end{abstract}
\maketitle
\section{Introduction}
\subsection{Overview}
Addition is one of the most natural operations on matrices. For deterministic matrices, there was a classical question posed by Weyl \cite{W} in 1912, which considers eigenvalues $c_{1}\le ...\le c_{N}$ of $C=A+B$, where $A,B$ are two arbitrary self-adjoint $N\times N$ matrices with fixed real eigenvalues $a_{1}\le ...\le  a_{N}$, $b_{1}\le ...\le b_{N}$, and try to describe all the possible values of  $c_{1}\le...\le c_{N}$. Solved by the end of $XX^{th}$ centery due to combined efforts by Horn, Klyochko, Kuntson-Tao, and others, see e.g \cite{Ho}, \cite{Kl}, \cite{KT}. In random matrices, people usually assume the summands $A$ and $B$ are random, independent and share some certain symmetries, and the study of this type of questions have significant connections with free probability theory. 

A well-known classical result connecting random matrix addition and free probability is stated by Volculescu in \cite{Vo}, which considers the addition of two independent real/complex/real quaternionic self-adjoint matrices, and relates its asymptotic behavior as the size of the matrix grows with the notion of \emph{free convolution}. There's also another classical result of similar flavor in rectangular setting. Take $\{A_{M}\}_{M=1}^{\infty}$, $\{B_{M}\}_{M=1}^{\infty}$ to be two independent sequences of $M\times N$ ($M\le N$) matrices with real/complex/real quaternionic entries, that are uniformly chosen from the set of rectangular matrices with given singular values $a_{M,1}\ge...\ge a_{M,M}\ge 0$ and $b_{M,1}\ge...\ge b_{M,M}\ge 0$, and let $C_{M}=A_{M}+B_{M}$ with random singular values $c_{M,1}\ge...\ge c_{M,M}\ge 0$. 
\begin{definition}\label{def:empiricalmeas}
    For  a $M\times N$ ($M\le N$) matrix $A$ with singular values $a_{1},...,a_{M}\ge 0$, define its (symmetric) empirical measure to be 
    $$\mu_{A}:=\frac{1}{2M}\sum_{i=1}^{M}(\delta_{a_{i}}+\delta_{-a_{i}}).$$
\end{definition}
\begin{thm}{\cite[Proposition 2.1]{B1}}\label{thm:recfreeconvolution}
Define $\{A_{M}\}_{M=1}^{\infty}$, $\{B_{M}\}_{M=1}^{\infty}$ as above. Assume that $M,N\rightarrow \infty$ in a way that $N(M)/M\rightarrow q$ for some constant $q\ge 1$, and there exists deterministic probability measures $\mu_{A}$, $\mu_{B}$ on $\R$, such that 
$$\lim_{M\rightarrow \infty}\mu_{A_{M}}=\mu_{A},\quad \lim_{M\rightarrow \infty}\mu_{B_{M}}=\mu_{B}.$$
Then the random empirical measure of $C$, $\mu_{C_{M}}=\frac{1}{2M}\sum_{i=1}^{M}(\delta_{c_{M,i}}+\delta_{-c_{M,i}})$, converges weakly in probability to some deterministic probability measure $\mu_{C}$ on $\R$.

$\mu_{C}=\mu_{A}\boxplus_{q} \mu_{B}$ is called the \emph{rectangular free convolution} of $\mu_{A}$ and $\mu_{B}$.
\end{thm}

The rectangular free convolution is a deterministic binary operation of measures on $\R$, that itself does not rely on random matrix structure, and it was well-studied in free probability theory from different aspects. In particular, for each measure $\mu$ with finite moments, there exists a collection of \emph{rectangular free cumulants} $\{c^{q}_{l}\}_{l=1}^{\infty}$ (see \cite[Section 3.1]{B1}) that are polynomials of moments with explicit expressions, and these quantities linearize free convolution, i.e, $c^{q}_{l}(\mu_{A}\boxplus \mu_{B})=c^{q}_{l}(\mu_{A})+c^{q}_{l}(\mu_{B})$ for all $l$'s. It turns out that the existence of such cumulants is a common feature of the various version of convolutions in free probability theory, and each convolution is characterized by its corresponding cumulants.

On the other side, there has been lots of papers studying additions of $\beta$-ensembles of random matrix theory that generalize the above theory in different parameter regimes. The parameter $\beta>0$ is interpreted in physics as inverse temperature, and the cases $\beta=1,2,4$ correspond to matrices with real/complex/real quaternionic entries. There are two classes of matrix ensembles, the $N\times N$ self-adjoint matrix and the $M\times N$ rectangular matrices, and the most classical examples are Gaussian ensembles and Laguerre ensembles, respectively. For the first class, \cite{GM} studies the limit behavior of eigenvalues of $C=A+B$ when $N$ is fixed and $\beta\rightarrow\infty$, \cite{BCG} proves a Law of Large Numbers similar to Theorem \ref{thm:recfreeconvolution} when $N\rightarrow\infty, \theta\rightarrow0, N\theta\rightarrow \gamma>0$, and \cite{MSS},\cite{AP},\cite{GM} extend the theory of convolution and cumulants to finite matrix additions for $\beta>0$. However, extending Theorem \ref{thm:recfreeconvolution} to general $\beta>0$ remained open. The second class is relatively less understood. \cite{B1},\cite{B2} study the so-called rectangular free convolution for $\beta=1,2$, $ N,M\rightarrow\infty$ and $N/M\rightarrow q\ge 1$, and \cite{GrM},\cite{Gri} study the finite free convolution and cumulants for rectangular matrix additions for $\beta=1,2$. 

The matrix ensemble considered in this text belongs to the second class, and we study the limiting behavior of singular values of $C=A+B$ in both low and high temperature regimes, more precisely, when $M,N$ are fixed, $\theta\rightarrow\infty$, and when $M, N\rightarrow\infty, \theta\rightarrow0, M\theta\rightarrow \gamma>0, N\theta\rightarrow q\gamma$ for some $q\ge 1$. Note that even defining the operation $C=A+B$ for $\beta\ne 1,2,4$ is non-trivial, and this is one of our tasks.

Our approach is based on distributions of rectangular matrices in a version of characteristic function. The symmetry of self-adjoint/rectangular matrices with fixed eigenvalues/singular values are given by invariance under actions of classical Lie group $O(N)/U(N)/Sp(N)$, and when $\beta=1,2,4$, the matrix characteristic functions have matrix integral representations with representation-theoretic background. Such functions have natural analytic continuation to all $\beta>0$, and can be identified as eigenfunctions of certain differential operators. See the following papers \cite{BCG}, \cite{GM}, \cite{GS} for application of such idea in random matrices, and also \cite{BG1}, \cite{BG2}, \cite{H} for the study of more general $N$-particle system using symmetric characteristic functions of similar flavor. While the above works deal with self-adjoint matrices, or more generally $N$-particle systems that are corresponding to root system of type A with a single root multiplicity $\theta=\beta/2$, rectangular matrices are corresponding to root system of type BC, that has two distinct root multiplicities parameterized by $\beta$ in a more involved way. For more connections of type BC Lie theoretic object with probability, see e.g \cite{KVW}, \cite{V}, \cite{VW}.

In this text, the randomness of a M-tuple of nonnegative real numbers (which should be thought as singular values of some $M\times N$ random matrices) is characterized by a multivariate symmetric function, known as type BC Bessel function in special functions literature. It is also a special case of the symmetric Dunkl kernel, that generalizes the usual Fourier kernel to nontrivial root multiplicities, see \cite{A} for a review. Motivated by the asymptotic behavior in high temperature regime, we apply and further develop a philosophy, that the limit of partial derivatives of logarithm of our characteristic function at 0 give a collection of cumulants, and the existence of such cumulants is equivalent to the existence of limiting moments, which implies that the empirical measure of the random M-tuples satisfy a Law of Large Numbers. These new q-$\gamma$ cumulants are designed to linearize the rectangular addition in the regime $M\theta\rightarrow \gamma, N\theta\rightarrow q\gamma$, while the operation itself in this limit regime is called q-$\gamma$ convolution. Similar to classical and free cumulants, they have nice combinatorial relation with moments. Finally, we point out that there's a surprising identification of q-$\gamma$ theory with the rectangular finite free probability theory, which was developed in \cite{MSS},\cite{GrM},\cite{Gri} while studying finite rectangular matrix additions.  

\subsection{Rectangular matrix addition}
Throughout the text we always take $\beta=2\theta>0$, and $\beta=1, 2, 4 \ (\theta=\frac{1}{2},1,2$) correspond to the (skew) field real, complex and real quaterion (whose real dimension are given by $\beta$). For $M\le N$, given two $M\times N$ independent random matrices $A$ and $B$, we study the randomness of the sum 
$$C=A+B.$$

Inspired by the classical theory of summing independent random variables $X+Y$, namely, that the charateristic function satiesfies
$$\phi_{X+Y}(t)=\phi_{X}(t)\cdot\phi_{Y}(t),$$
where $t\in \R$ is the parameter variable, we have
\begin{prop}\label{prop:characteristic1}
For $\theta=\frac{1}{2},1,2$, let $A$ and $B$ be $M\times N$ rectangular independent random matrices, let $Z$ be $N\times M$ an arbitrary deterministic matrix with real/complex/real quaternionic entries, and let $C=A+B$. We have  
\begin{equation}\label{eq_matrixfouriertransform}
\EE \Big[\exp\Big(Re(Tr(CZ))\Big)\Big]=\EE\Big[\exp\Big(Re(Tr(AZ))\Big)\Big]\cdot\EE\Big[\exp\Big(Re(Tr(BZ))\Big)\Big].
\end{equation}
\end{prop}
\begin{proof}
    $Re(Tr(CZ))=Re(Tr(AZ))+Re(Tr(BZ)),$ and since $A, B$ are independent, the expectation of the exponential function factors.
\end{proof}


Let us now rewrite (\ref{eq_matrixfouriertransform}) in terms of singular values of $A, B$ and $C$, and for simplicity first take $\theta =1$, i.e, we deal with complex matrices. In this text, we are considering the summands $A$ and $B$ that have distribution invariant under left and right unitary actions, i.e, 
\begin{equation}\label{eq_invariance}
A\stackrel{d}{=} UAV, \ B\stackrel{d}{=} UBV,  
\end{equation}
where $U\in U(M), V\in U(N)$ are arbitrary unitary matrices. One example is the real/complex/real quaternionic $M\times N$ Wishart matrix, with i.i.d mean 0 Gaussian entries.
 
 Note that if $A,B$ satisfy (\ref{eq_invariance}), so is C. For simplicity, in the following discussions we usually stick to A. By singular value decomposition, it's useful to write A as $U\Lambda V$, where 
 \begin{equation}\label{eq_matrixLambda}
 \Lambda=\begin{bmatrix}
a_{1} &  & & &&0&...&0\\
 & a_{2} & & &&0&...& 0\\
  &             &...&   &&\\
  &&&...&\\
  &&&&a_{M} &0&...&0
\end{bmatrix}_{M\times N},
\end{equation}
$\Vec{a}=(a_{1},...,a_{M})\in \R_{\ge 0}$, and $U\in U(M), V\in U(N)$ are random elements under Haar measures on the corresponding unitary groups. For now, assume that the singular values of A are deterministic. One can consider rectangular matrices A with real/real quaternionic entries and define invariant distribution in exactly the same way, while replacing the Haar distributed $U,V$ by elements in orthogonal/unitary sympletic groups $O(M)/Sp(M)$, $O(N)/Sp(N)$. Similarly for $B$.

The eigenvectors of $AA^{*}, A^{*}A$, $BB^{*}, B^{*}B$ are distributed uniformly, and so are eigenvectors of $CC^{*}, C^{*}C$. Hence the nontrivial randomness of $C$ is about its singular values. Also, because of the singular value decomposition in (\ref{eq_invariance}), we can replace the parameter matrix $Z$ by the form in (\ref{eq_diagzmatrix}), where $z_{1},...,z_{M}$ are its singular values. Therefore, we can rewrite the matrix Fourier transform of $A$ in (\ref{eq_matrixfouriertransform}) as a function $\B(\Vec{a};z_{1},...,z_{M};\theta,N)$, such that 
\begin{equation}\label{eq_matrixint}
    \B(\Vec{a};z_1,z_2,...,z_M;\theta, N)=\int  dU\int  dV \exp\left(\frac{1}{2}Tr(U\Lambda VZ+Z^{*}V^{*}\Lambda^{*}U^{*})\right),
\end{equation}
where $\Lambda$ is defined in the same way as in (\ref{eq_matrixLambda}), and
\begin{equation}\label{eq_diagzmatrix}
    Z=\begin{bmatrix}
z_{1} &  & & &\\
 & z_{2} & & &\\
  &             &...&\\
  &&&...&\\
  &&&&z_{M} &\\
  0&&...&&0\\
  && ...&&\\
  0&&...&&0
\end{bmatrix}_{N\times M},\end{equation}
$\theta$ takes the value $\frac{1}{2},1,2$, $(z_{1},...,z_{M})\in \R_{M}$, and $U\in O(M)/U(M)/Sp(M)$, $V\in O(N)/U(N)/Sp(N)$ are integrated under Haar measures respectively.

The function $\B(\Vec{a};z_{1},...,z_{M};\theta,N)$ is known as multivariate type BC Bessel function in a general theoretical framework of special functions. We briefly review this framework in Appendix A, and give more information about the type BC Bessel function in Section 2.2, 2.3 based on the theory.

We note that because of the symmetry of Haar measure, $\B(\Vec{a};z_{1},...,z_{M};\theta,N)$ is symmetric in both $(a_{1},...,a_{M})$ and $(z_{1},...,z_{M})$, and without loss of generality we can always take $a_{1}\ge a_{2}\ge ...\ge a_{M}$. Same for matrix $B$ and $C$.
Then Proposition \ref{prop:characteristic1} is rewritten as the following result.
\begin{prop}\label{prop:characteristic2}
For $\theta=\frac{1}{2},1,2$, fix $a_{1}\ge...\ge a_{M}\ge 0$,  $b_{1}\ge...\ge b_{M}\ge 0$, let 
$A_{M\times N}$ and $B_{M\times N}$ be real/complex/real quaternionic rectangular matrices with deterministic singular values $\{a_{i}\}_{i=1}^{M},\{b_{i}\}_{i=1}^{M}$ and invariant distribution, as in (\ref{eq_invariance}), $C=A+B$ with singular values $\Vec{c}=(c_{1}\ge ...\ge c_{M}\ge 0)$,
then 
\begin{equation}
\E{\B(\Vec{c};z_{1},...,z_{M};\theta,N)}=
\B(\Vec{a};z_{1},...,z_{M};\theta,N)\cdot \B(\Vec{b};z_{1},...,z_{M};\theta,N), \quad (z_{1},...,z_{M})\in \R^{M}.
\end{equation}
\end{prop}

For general $\beta>0$, there is no skew field of real dimension $\beta$ and therefore no concrete $\beta$-rectangular matrices. But motivated by Proposition \ref{prop:characteristic2}, we first identify an invariant $M\times N$ matrix with uniform "singular vectors" and deterministic singular values $a_{1},...,a_{M}$, with the M-tuples $\Vec{a}$. Moreover, it's known (see e.g \cite[Section 13.4.3]{Forrester}, \cite{GT}) that multivariate Bessel functions admit a natural exterpolation from $\theta=\frac{1}{2},1,2$ to arbitrary real $\theta>0$. We still denote this function as $\B(\Vec{a};z_{1},...,z_{M};\theta,N)$ for all $\theta>0$. Hence, the  $M\times N$ matrix addition (of independent summands) is extended to $\theta>0$, by generalizing Proposition \ref{prop:characteristic2}. 

\begin{definition}\label{def:deterministicaddition}
Fix $\theta>0$, $M\le N$, $\Vec{a}=(a_{1}\ge...\ge a_{M}\ge 0)$, $\Vec{b}=(b_{1}\ge...\ge b_{M}\ge 0)$, let $\Vec{c}$ be a symmetric random vector in $\R_{\ge 0}^{M}$, such that 
\begin{equation}\label{eq_additioindef}
\EE\Big[\B(\Vec{c};z_{1},...,z_{M};\theta,N)\Big]=
\B(\Vec{a};z_{1},...,z_{M};\theta,N)\cdot \B(\Vec{b};z_{1},...,z_{M};\theta,N), \quad (z_{1},...,z_{M})\in \R^{M}.
\end{equation}
We write 
$$\Vec{c}=\Vec{a}\boxplus_{M,N}^{\theta}\Vec{b}.$$
\end{definition}

From the probablistic point of view, $\Vec{c}$ is identified as the singular values of the "virtual" random $M\times N$ matrix $C=A\boxplus^{\theta}_{M,N} B$, and $\B(\Vec{c};z_{1},...,z_{M};\theta,N)$ serves as the characterstic function of $C$. From the analytic point of view, the operation in (\ref{eq_additioindef}) has been studied previously, in the context of Dunkl kernel and Dunkl translation, see \cite[Section 3.6]{A}.  The expectation symbol on the left of (\ref{eq_additioindef}) holds in the sense that, there exists a unique generalized function\footnote{Throughout this text, we use the term "generalized function" instead of "distribution" to denote a linear functional on smooth functions, in order to avoid confusion with the probability distribution.} $\mathfrak m$ on $\R_{\ge 0}^{M}$ depending on $\Vec{a}$ and $\Vec{b}$, such that for any $(z_{1},...,z_{M})\in \R^{M}$, when testing on $\B(\cdot;z_{1},...,z_{M};\theta, N)$ we get the right side of (\ref{eq_additioindef}), and in particular when taking $z_{1}=...=z_{M}=0$ we have
$\mathfrak m(1)=1$. 
Note that $\mathfrak m$ is symmetric in the sense that, for any proper test function $f$ and any permutation $\sigma$,
$$\langle \mathfrak m, f(c_{1},...,c_{M})\rangle=\langle \mathfrak m, f(c_{\sigma(1)},...,c_{\sigma(M)})\rangle,$$
where $\langle \mathfrak m, f\rangle$ denotes the value of the functional $\mathfrak m$ testing on the f. Moreover, by \cite[Lemma 3.23]{A} $\mathfrak m$ is compactly supported.



The rectangular addition $\Vec{a}\boxplus_{M,N}^{\theta}\Vec{b}$ can also be naturally generalized to independent random M-tuples $\Vec{a},\Vec{b}$, by first taking the conditional event that $\Vec{a},\Vec{b}$ taking some fixed value, then applying Definition \ref{def:deterministicaddition}. Formally, for random M-tuples $\Vec{a}$ we replace the type BC Bessel function by 
\begin{equation}\label{eq_bgf0}
G_{N;\theta}(z_{1},...,z_{M}):=\EE \Big[\B(\Vec{a},z_{1},...,z_{M};N,\theta)\Big],
\end{equation}
the type BC Bessel generating function of $\Vec{a}$, and we assume the randomness of $\Vec{a}$ to be reasonable, in the sense that the right side of (\ref{eq_bgf0}) is finite and well-behaved as an analytic function of $(z_{1},...,z_{M})\in \R_{M}$. See Section \ref{sec:bgf} for more details.

\subsection{Low and high temperature behavior}
By viewing $\Vec{c}=\Vec{a}\boxplus_{M,N}^{\theta}\Vec{b}$ as the random M-tuples of singular values of some $M\times N$ virtual rectangular matrix with invariant distribution, it's then natural to study the behavior of $\Vec{c}$ from a random matrix point of view. The distribution of $\Vec{c}$ depends on summands $\Vec{a},\Vec{b}$ and parameters $M\le N, \theta>0$. In various regimes of parameters, one can propose the following questions:

1. For fixed $M,N,\theta$, is the distribution of $\Vec{c}$ given by a probability measure on $\R_{\ge 0}^{M}$, and how do $\Vec{a},\Vec{b}$ explicitly determine this measure?

2. What's the "low temperature" behavior of $\Vec{c}$, i.e, when taking $M,N$ to be fixed, and  $\theta\rightarrow\infty$?

3. What's the "fixed temperature" behavior of $\Vec{c}$, i.e, when taking $\theta$ to be fixed, and $M,N\rightarrow \infty$?

4. What's the "high temperature" behavior of $\Vec{c}$, i.e, when taking $\theta\rightarrow 0$, and $M,N\rightarrow \infty$, growing in potentially different speed?

This text answers question 2 and 4. For question 1, it is well-believed (but still open) that the generalized function under $\Vec{c}$ is indeed a probability measure, see e.g \cite[Section 3.5]{A}, and this is related to the positivity conjecture of Littlewood-Richardson coefficients in the theory of symmetric functions, see \cite[Conjecture 8.3]{St}, \cite{Ro2} or \cite[Conjecture 2.1]{GM}. We do not rely on this conjecture and instead analyze moments of the distribution of $\Vec{c}$, which can be defined no matter the positivity conjecture holds or not. See Proposition \ref{prop:polyexpectation} for the precise statement.

In low temperature regime, we observe that the random M-tuples are becoming "frozen" at some deterministic positions. More precisely we have the following statement.
Let $1\le M\le N$, $z$ be a formal variable, $(a_{1},...,a_{M}), (b_{1},...,b_{M})\in \R^{M}_{\ge 0}$, we define a polynomial $P_{M,N}(z)$ by
\begin{equation}\label{eq_charpoly}
\begin{split}
P_{M,N}(z)=\sum_{l=0}^{M}(-1)^{l}\Bigg(&\sum_{i\ge 0, j\ge 0, i+j=l}\frac{(M-i)!(M-j)!}{M!(M-l)!}\\&\frac{(N-i)!(N-j)!}{N!(N-l)!}
e_{i}(a_{1}^{2},...,a_{M}^{2})e_{j}(b_{1}^{2},...,b_{M}^{2})\Bigg)z^{M-l}.
\end{split}
\end{equation}

\begin{thm}\label{thm:lln}
Fix $M\le N$, given $\Vec{a}$ and $\Vec{b}$, let $\Vec{c}=\Vec{a}\boxplus_{M,N}^{\theta}\Vec{b}$. Then as 
 $\theta\rightarrow \infty$, the distribution of $\Vec{c^{2}}=(c_{1}^{2},...,c_{M}^{2})$ converges on polynomial test functions to $\delta$-measures on roots of $P_{M,N}(z)$. 
\end{thm}
\begin{remark}
   The polynomial $P_{M,N}(z)$ has previously appeared in \cite{GrM}, and it's shown in \cite[Theorem 2.3]{GrM} that all roots of $P_{M,N}(z)$ are real and nonnegative, given that $a_{1}^{2},...,a_{M}^{2}, b_{1}^{2},...,b_{M}^{2}$ are all real and nonnegative, using the theory of stable polynomials.
\end{remark}

In the fixed temperature regime, it was shown in \cite{B1} (see Theorem \ref{thm:recfreeconvolution}) that for $\theta=\frac{1}{2},1$, as $M,N\rightarrow\infty$ in a way that $\frac{N}{M}\rightarrow q$, we get the rectangular free convolution. We believe the same result holds for any fixed $\theta>0$.  

In high temperature regime, when taking $\theta\rightarrow 0$, $N\rightarrow\infty$ and let $M$, the number of singular values to be fixed, the type BC multivariate Bessel function becomes a simple symmetric combination of exponents: 
\begin{equation}
        \B(\Vec{a},N\theta z_{1},...,N\theta z_{M};\theta, N)\longrightarrow \frac{1}{M!}\sum_{\sigma\in S_{M}}\prod_{i=1}^{M}e^{a_{i}^{2}z_{\sigma(i)}^{2}}.
\end{equation}
See Appendix D for more details. Such limit expression has a clear probabilistic interpretation. Given deterministic M-tuples $\Vec{a}$ and $\Vec{b}$ as before, let $\Vec{c}=(c_{1},...,c_{M}\ge 0)$ be obtained by choosing uniformly random an element $\sigma$ in $S_{M}$, and taking 
    $$(c_{1}^{2},...,c_{M}^{2})=(a_{1}^{2}+b_{\sigma(1)}^{2},...,a_{M}^{2}+b_{\sigma(M)}^{2}).$$
    Moreover, when taking $M\rightarrow \infty$ after and assume the empirical measure 
    $$\frac{1}{M}\sum_{i=1}^{M}\delta_{x_{i}^{2}}\ (x_{i}=a_{i,M}\ \text{or}\ b_{i,M})$$ 
    of $\{\Vec{a}_{M}\}_{M=1}^{\infty}$, $\{\Vec{b}_{M}\}_{M=1}^{\infty}$ converge to some probability measure $\mu_{a}, \mu_{b}$ on $\R_{\ge 0}$ weakly, then so is $\{\Vec{c}_{M}\}_{M=1}^{\infty}$, and the empirical measure
    $$\mu_{c}=\mu_{a}*\mu_{b},$$
    where $*$ denotes the usual convolution of measures on $\R$.

We see two different limiting behavior of $\Vec{a}\boxplus_{M,N}^{\theta}\Vec{b}$ as $M\rightarrow\infty$. For $\theta=0$ we get the usual convolution, and for $\theta>0$ we get the rectangular free convolution. This motivates us to look at the intermediate regime between the above two settings, such that we take $M\rightarrow \infty, N\rightarrow \infty, \theta\rightarrow 0, M\theta\rightarrow\gamma>0$, $N\theta\rightarrow q\gamma$ for some $q\ge 1$. We are interested in the sequence of (random) virtual singular values $\{\Vec{c}_{M}=(c_{M,1}\ge...\ge c_{M,M}\ge 0)\}_{M=1}^{\infty}$, and we study the limiting behavior of the symmetric empirical measures of $\Vec{c}_{M}$. 

\begin{definition}\label{def:llnsatisfaction0}
    Let $\{\Vec{a}_{M}\}_{M=1}^{\infty}$ be a sequence of random M-tuples such that $\Vec{a}_{M}=(a_{M,1}\ge ...\ge a_{M,M}\ge 0)$. Denote 
    $$p^{M}_{k}=\frac{1}{2M}\sum_{i=1}^{M}\left[a_{M,i}^{k}+(-a_{M,i})^{k}\right].$$
    We say $\{\Vec{a}_{M}\}$ converges in moments, if there exist deterministic nonnegative real numbers $\{m_{k}\}_{k=1}^{\infty}$ such that for any s=1,2,... and any $k_{1},...,k_{s}\in \Z_{\ge 1}$, we have
    \begin{equation}
        \lim_{M\rightarrow \infty}\E{\prod_{i=1}^{s}p_{k_{i}}^{M}}=\prod_{i=1}^{s}m_{k_{i}}.
    \end{equation}
    We write 
    \begin{equation}
        \Vec{a}_{M}\xrightarrow[M\rightarrow \infty]{m} \{m_{k}\}_{k=1}^{\infty}. 
    \end{equation}
\end{definition}
\begin{remark}
    Definition \ref{def:llnsatisfaction0} is stating that the empirical measure of $(a_{M,1},...,a_{M,M})$ is converging weakly to some deterministic probability measure with moments $\{m_{k}\}_{k=1}^{\infty}$, as long as the moment problem of $\{m_{k}\}_{k=1}^{\infty}$ has a unique solution. By definition $p^{M}_{k}=0$ for all odd $k$'s, and therefore one can immediately see that $m_{k}=0$ for all odd $k$'s. The reason why we use symmetric empirical measure is that there's no canonical choice of the sign of singular values.
\end{remark}
\begin{remark}
    The convergence is well-posed as long as the randomness of $\Vec{a}_{M}$'s are given by compactly supported generalized function, where the expectation $\EE$ is testing the generalized function by the  
polynomial function $p^{M}_{k}$ of $\Vec{a}_{M}$.

\end{remark}

We prove a law of large numbers of the symmetric empirical measure of $\Vec{c}_{M}$, which is interpreted as the empirical measure of the $M\times N$ matrix $C$ with singular values $c_{M,1},...,c_{M,M}$. We assume the the distribution of each  $\Vec{a}_{M}$, $\Vec{b}_{M}$ is given by some real valued compactly supported generalized function or exponentially decaying measure. For the precise meaning of the latter notion and more details of this technicality, see Section \ref{sec:bgf}. 

\begin{thm}\label{thm:hightemplln}
    Fix $\gamma>0, q\ge 1$. For $M=1,2,...$, let $N(M)\ge M$, $\theta(M)>0$ be two sequences satisfying $N\rightarrow\infty$, $\theta\rightarrow 0$, $M\theta\rightarrow \gamma$, $N\theta\rightarrow q\gamma$ as $M\rightarrow \infty$. Suppose for two sequences of random tuples $\{\Vec{a}_{M}\}_{M=1}^{\infty}, \{\Vec{b}_{M}\}_{M=1}^{\infty}$, 
    $$\Vec{a}_{M}\xrightarrow[M\rightarrow \infty]{m}\{m^{a}_{k}\}_{k=1}^{\infty},\quad \Vec{b}_{M}\xrightarrow[M\rightarrow \infty]{m}\{m^{b}_{k}\}_{k=1}^{\infty}.$$ 
    Then
    $$\Vec{a}_{M}\boxplus^{\theta}_{M,N}\Vec{b}_{M}\xrightarrow[M\rightarrow\infty]{m}\{m^{c}_{k}\}_{k=1}^{\infty},$$
    where $\{m^{c}_{k}\}_{k=1}^{\infty}$ is a sequence of deterministic nonnegative real numbers.

    We say $\{m^{c}_{k}\}_{k=1}^{\infty}$ is the q-$\gamma$ convolution of $\{m^{a}_{k}\}_{k=1}^{\infty}$ and $\{m^{b}_{k}\}_{k=1}^{\infty}$, written as 
    $$\{m^{c}_{k}\}_{k=1}^{\infty}=\{m^{a}_{k}\}_{k=1}^{\infty}\boxplus_{q,\gamma}\{m^{b}_{k}\}_{k=1}^{\infty}.$$
\end{thm}

We provide more properties of the q-$\gamma$ convolution in the following two theorems.
\begin{thm}\label{thm:momentcumulantsummary}
There exists a invertible map $\mathrm{T}^{q,\gamma}_{m\rightarrow k}:\R^{\infty}\rightarrow \R^{\infty}$, that corresponds each $\{m_{2k}\}_{k=1}^{\infty}$ with a collection of \emph{q-$\gamma$ cumulants} $\{k_{l}\}_{l=1}^{\infty}$, i.e, $\{k_{l}\}_{l=1}^{\infty}=\mathrm{T}^{q,\gamma}_{m\rightarrow k}(\{m_{2k}\}_{k=1}^{\infty})$. The $q$-$\gamma$ cumulants linearizes q-$\gamma$ convolution: for $l=1,2,...$,
$$k_{l}(\{m^{c}_{2k}\}_{k=1}^{\infty})=k_{l}(\{m^{a}_{2k}\}_{k=1}^{\infty})+k_{l}(\{m^{b}_{2k}\}_{k=1}^{\infty}).$$
$k_{l}=0$ for all odd $l$'s. Also $m_{k}^{a}$, $m_{k}^{b}$, $m_{k}^{c}$ are 0 for all odd $k$'s.

Treating each $k_{l}$ as a variable of degree $l$, then each $m_{2k}$ is a homogeneous polynomial in $k_{l}$'s of degree $2k$, whose coefficients are polynomials of $q, \gamma$ with explicit combinatorial description. 
Conversely, treating $m_{2k}$ as a variable of degree $2k$, each even q-$\gamma$ cumulant $k_{2l}$ is a homogeneous polynomial in $m_{2k}$'s of degree $2l$. 
\end{thm}
\begin{thm}\label{thm:degenerationsummary}
When $q\rightarrow 0, q\gamma\rightarrow \infty$, the q-$\gamma$ convolution of $\{m_{k}^{a}\}_{k=1}^{\infty}$ and $\{m_{k}^{b}\}_{k=1}^{\infty}$ turns into usual convolution of the two corresponding independent random variables, and the q-$\gamma$ cumulants of $\{m_{2k}^{a}\}_{k=1}^{\infty}$, $\{m_{2k}^{b}\}_{k=1}^{\infty}$ turn into the usual cumulants after proper rescaling.
Similarly, when $q$ is fixed, $\gamma\rightarrow\infty$, the q-$\gamma$ convolution turns into rectangular free convolution, and the q-$\gamma$ cumulants turn into rectangular free cumulants after proper rescaling.

\end{thm}
Theorem \ref{thm:hightemplln} is proved in Section \ref{sec:lln}. Theorem \ref{thm:momentcumulantsummary} summarizes results in Section \ref{sec:cumulanttomoment} and \ref{sec:momenttocumulant}, such that the combinatorial moment-cumulant formula is given in Theorem \ref{thm:cumulanttomomentcomb}, and the relation between moment generating function and cumulant generating function is given in Theorem \ref{thm:momenttocumulant}. Theorem \ref{thm:degenerationsummary} summarizes the connections of q-$\gamma$ convolution to classical and free convolution, which are given in Theorem \ref{thm:usualcumulant} and Theorem \ref{thm:connectiontofreecumulant} respectively. We also provide a limit transition of our $q$-$\gamma$ convolution to the $\gamma$-convolution defined in \cite{BCG} in Theorem \ref{thm:gammacumulant}, which is related to the asymptotic behavior of self-adjoint matrix additions in high temperature regime.

\subsection{Duality between low and high temperatures}
It is observed that in $\beta$-random matrix theory, there's a duality between parameters $\beta$ and $\frac{4}{\beta}$ (or $\theta$ with $\frac{1}{\theta}$). For example in \cite{De}, the author gives an equality of average products of characteristic polynomials of Gaussian/Chiral $\beta$-ensembles at $\beta$ and $\frac{4}{\beta}$. Similarly in \cite{F2}, it's shown that for Gaussian/Laguerre/Jocobi $\beta$-ensembles, the one-point or higher-point functions that describe the linear statistic of eigenvalues at low and high temperature can be identified with each other. The phenomena is not yet fully understood, and one analog exists in the theory of symmetric polynomials, where there is an automorphism that sends Jack polynomial to its dual by taking the transpose of its labelling Young diagram and invert the parameter $\theta$ at the same time, see \cite[Section 3]{S} or \cite[(10.17)]{M} for the precise statement. 
Since in this text we are considering low and high temperature regimes at the same time, the duality is indicating some connection between the two regimes. When $M,N$ are fixed, $\theta\rightarrow\infty$, $\Vec{c}=\Vec{a}\boxplus_{M,N}^{\theta}\Vec{b}$ concentrate at roots of $P_{M,N}(z)$, which are identified as the rectangular finite free convolution of $\Vec{a}$ and $\Vec{b}$ defined in \cite{MSS}, \cite{GrM}. When $M,N\rightarrow\infty$, $\theta\rightarrow 0$, $M\theta\rightarrow\gamma,\ N\theta\rightarrow q\gamma$, $\Vec{c}$ converges in moments to the q-$\gamma$ convolution of $\Vec{a}$, $\Vec{b}$.  We find that the $(M,N)$-rectangular finite free convolution and the $q$-$\gamma$ convolution match with each other under certain identification of parameters. More precisely, \cite{Gri} introduces a degree M polynomial as the so-called rectangular R-transform, that linearizes rectangular finite free convolution. We treat the coefficients of rectangular R-transform as the rectangular finite cumulants, and show that if identifying $M$ in rectangular finite convolution with $-\gamma$ in q-$\gamma$ convolution, the moment-cumulant relation of rectangular finite convolution and q-$\gamma$ convolution match perfectly. In addition, since both $M$ and $\gamma$ are positive, these two operations are analytic continuation of each other, and they together extend the moment-cumulant relation to $\gamma\in \R_{\ge 0}\bigcup\Z_{\le -1}$. See Section \ref{sec:duality} for more details.

We also note that similar identification of low and high temperature regimes holds also for self-adjoint matrix addtions. \cite{BCG} studies addition of two $N\times N$ self-adjoint matrices in the high temperature regime $N\rightarrow\infty, \theta\rightarrow 0, N\theta\rightarrow\gamma>0$, and introduces the so-called $\gamma$-convolution and $\gamma$-cumulants. On the other hand, \cite{AP} introduces a family of $d\times d$ free cumulants from finite self-adjoint matrix additions in low temperature, and the authors of these two papers discovered that their moment-cumulant relations can also match by identifying $d$ with $-\gamma$. We believe that such matching appearing in both self-adjoint and rectangular matrix additions should not be just coincidence. 

\subsection{Techniques and difficulties}
Unlike many other classes of $\beta$-ensembles, we don't have a density function of our object $\Vec{c}=\Vec{a}\boxplus_{M,N}^{\theta}\Vec{b}$, and because of the openness of the positivity conjecture we can't even guarantee that such density exists. Therefore, the proof of main results in low and high temperature regime, Theorem \ref{thm:lln} and Theorem \ref{thm:hightemplln}, both rely heavily on moment calculations.  We characterize its distribution using the type BC Bessel generating function $G_{N;\theta}(z_{1},...,z_{M};\Vec{c})$, which is a new object in random matrix literature, and apply two different approaches in the low and high temperature regime respectively, to extract the moment information of $\Vec{c}$.

In order to apply such approaches, it's necessary to figure out the correct notion of Bessel function $\B(\vec{c};z_{1},...,z_{M};\theta,N)$ for rectangular matrices. On one hand, we start from the case $\theta=\frac{1}{2},1,2$ and define $\B(\Vec{c};z_{1},...,z_{M};\theta,N)$ as the matrix integral in (\ref{eq_matrixint}), based on the probabilistic intuition of rectangular random matrices. On the other hand, for arbitrary $\theta>0$, we define our type BC Bessel function to be a symmetric Dunkl kernel, that is known as the joint eigenfunction of the corresponding type BC Dunkl operators, with eigenvalues given by the symmetric moments of $\Vec{c}$. While there are infinite versions of Dunkl kernels, we choose the root multiplicities $m_{\pm e_{i}}$, $m_{\pm e_{i}\pm e_{j}}$ in a unique way that 

1. For $\theta=\frac{1}{2},1,2$, it coincides with (\ref{eq_matrixint}).

2. For general $\theta>0$, it has nice explicit power series expansion that naturally extrapolates from $\theta=\frac{1}{2},1,2$.

We find such root multiplicities and verify the analytic and combinatorial properties of $\B(\cdot;z_{1},...,z_{M};\theta,N)$ in Section \ref{sec:pre}, by applying the general theory of special functions and symmetric spaces under random matrix motivations.

In low temperature regime, we use the explicit expansion of Bessel generating function to calculate the limiting distribution of $\Vec{c}$. And in high temperature regime, we study the asymptotic behavior of the action of Dunkl operators on $G_{N;\theta}(z_{1},...,z_{M};\Vec{c})$, which extracts moment information. More precisely, in Theorem \ref{thm:hightemperaturemainthm} we build an equivalence of the following two conditions of a sequence of random M-tuples $\Vec{c}_{M}=(c_{M,1},...,c_{M,M})\in \R_{\ge 0}^{M}$, $M=1,2,...$, in the regime $M,N\rightarrow\infty, \theta\rightarrow 0, M\theta\rightarrow \gamma, N\theta\rightarrow q\gamma$:

1. $\{\Vec{c}_{M}\}_{M=1}^{\infty}$ converges in moments as in Definition \ref{def:llnsatisfaction0}.

2. The $l^{th}$ order partial derivative in $z_{1}$ of $\ln\Big(G_{N;\theta}(z_{1},...,z_{M};\Vec{c})\Big)$ at 0 converges to some real number for all $l=1,2,...$, and the partial derivatives in more than one variables among $z_{1},...,z_{M}$ at 0 all converge to 0.

The nontrivial limit of $l^{th}$ order derivative in condition 2 gives the q-$\gamma$ cumulant $k_{l}$ of $\{\Vec{c}_{M}\}_{M=1}^{\infty}$, up to some constant. Note that this equivalence itself is independent of the addition operation, and can be applied to a single sequence of (virtual) rectangular matrices. See Section \ref{sec:laguerre} as an example.

Compared to the previous studies of rectangular additions, which mostly deal with real/complex matrices, our text defines and considers the general $\beta$-additions that do not rely on concrete matrix structure. Compared to the study of self-adjoint additions, there are some extra technicality that arise in this text. Firstly, there are two parameters $M, N$ of the matrix size, and we allow $M$ and $N$ to grow in different speed. More importantly, because of the more involved root multiplicities,  the type BC Bessel generating functions, type BC Dunkl operators have more complicated expressions, and this makes the combinatoric in the asymptotic analysis more complicated as well. Because of the above two issues, and because of the fact that rectangular matrices are relatively less studied in literatures, it takes more efforts for us to properly define the rectangular version of empirical measures, moments, cumulants etc, and figure out the limit regime that nontrivial behavior and connection to known objects occur. The readers will also see a more complicated moment-cumulant relation of our q-$\gamma$ convolution, that can degenerate to the usual, free, rectangular free and $\gamma$-convolutions which are operations that characterize several other random matrix additions.

\subsection{Further Studies}
We point out several possible directions for further studies in rectangular matrix addition. In the regime $\theta\rightarrow\infty$ and $M,N$ fixed, we believe that the fluctuation of $\Vec{c}=\Vec{a}\boxplus_{M,N}^{\theta}\Vec{b}$ around roots of $P_{M,N}(z)$ will converge in distribution to some Gaussian vector, since the similar limiting behavior holds for Laguerre $\beta$ ensemble. We are only able to prove this for the single row matrix, i.e, when $M=1$, and the general case remains as an open problem.

This text does not consider the fixed temperature regime, where $M,N\rightarrow\infty$, $N/M\rightarrow q\ge 1$ and $\theta$ is fixed. As mentioned previously, we believe that for general $\theta>0$, we will get rectangular free convolution of the empirical measure in the limit.

In the regime $M,N\rightarrow \infty, \theta\rightarrow 0, M\theta\rightarrow \gamma, N\theta\rightarrow q\gamma$, we prove a Law of Large Numbers of the empirical measure of rectangular matrix addition, and it might be of interest to go further, and prove a Central Limit Theorem of it under proper assumption of the summands: for a class of well-behaved test function $\phi$, testing the empirical measure with $\phi$ always gives a Gaussian random variable in the limit. We refer to \cite{GuM} for the result in this flavor on a collection of $\beta$-ensembles.

We also note that we only consider the global behavior of the limiting empirical measure in this text. However, our approach of using Dunkl operators to extract moment information, might be applicable in the study of the bulk or edge limit of certain matrix ensemble, including but not limited to rectangular matrix addition.

The paper is organized as follows. In Section \ref{sec:pre} we introduce type BC Bessel function and Bessel generating function, which play the role of characteristic function for rectangular matrices. In Section \ref{sec:lowtemp} we study the low temperature behavior. In Section \ref{sec:lln} we prove the main theorem in high temperature regime, and introduce the q-$\gamma$ cumulants in an analytic way. Then we study the moment-cumulant relation of q-$\gamma$ convolution in more details, provide an explicit combinatorial description, and point out its connection with the classical free probability theory in Section \ref{sec:momentcumulant}. Finally in Section \ref{sec:duality}, we check the quantitative connection between low and high temperature regimes.
\subsection*{Acknowledgements} The author is grateful to Vadim Gorin for a lot of stimulating discussions, and all his useful suggestions on the presentation of this text. We thank Grigori Olshanski for pointing out one useful reference, Simon Marshall for explaining some basics of symmetric spaces, and Margit Roesler for clarification of a technical issue in her lecture notes.

\section{Bessel functions and Dunkl operators} \label{sec:pre}
\subsection{Symmetric polynomials}\label{sec:sympoly}
Symmetric polynomials are common objects appearing in combinatoric, representation theory, and random matrices.  This section recalls basic definitions of several objects in this subject, that we will use in this text. For a detailed introduction of classical results of symmetric polynomials, see e.g \cite{M}.

\begin{definition}\label{def:YD}
    A \emph{partition} $\lambda$ is a M-tuple of nonnegative integers $(\lambda_{1}\ge \lambda_{2}\ge ...\ge \lambda_{M}\ge 0)$. We identity $(\lambda_{1},...,\lambda_{M})$ with $(\lambda_{1},...,\lambda_{M},0,...,0)$, and denote the \emph{length} of $\lambda$ by $l(\lambda)\in \Z_{\ge 1}$, which is the number of strictly positive $\lambda_{i}$'s. We say a partition is even, if $\lambda_{1},...,\lambda_{l(\lambda)}$ are all even. 

    Let 
    $|\lambda|=\sum_{i=1}^{l(\lambda)}\lambda_{i}.$ For two partitions $\lambda, \mu$ such that $|\lambda|=|\mu|$, there's a lexicographical order between them, that is, $\lambda>\mu$ if and only if for some $j\in \Z_{\ge 1}$,
    $$\lambda_{1}=\mu_{1},...,\lambda_{j-1}=\mu_{j-1}\ \text{and}\ \lambda_{j}>\mu_{j}.$$
\end{definition}

The combinatoric expressions of symmetric polynomials are often given by sums over partitions, for which we introduce the following notions. 

\begin{definition}
    A \emph{Young diagram} is graphical representation of a partition. Given a partition $\lambda$, view it as a collection of $|\lambda|$ boxes, that there are $\lambda_{i}$ boxes in the $i^{th}$ row. In this text we do not distinguish a partition and its corresponding Young diagram. Let $s=(i,j)\in \lambda$ be the coordinate of the box on the $j^{th}$ column and the $i^{th}$ row in $\lambda$. Moreover, let $\lambda_{j}^{'}$ be the number of boxes on the $j^{th}$ column of $\lambda$, and
    $$a(s)=a(i,j)=\lambda_{i}-j, \quad l(s)=l(i,j)=\lambda_{j}^{'}-i,\quad \lambda^{'}=(\lambda_{1}^{'},...,\lambda_{\lambda_{1}}^{'}).$$
\end{definition}
\begin{definition}\label{def:sympoly}
    For $M\in \Z_{\ge 1}$, a symmetric polynomial $g(z_{1},...,z_{M})$ is a multivariate polynomial of variables $z_{1},...,z_{M}$ with complex coefficients, such that for any $\sigma\in S_{M}$, the symmetric group of M elements, we have
    $$f(z_{1},...,z_{M})=f(z_{\sigma(1)},...,z_{\sigma(M)}).$$

    We denote the space of all symmetric polynomials in M variables by $\Lambda_{M}$, which has the structure of an (complex) algebra. 
\end{definition}

We introduce several classical symmetric polynomials as elements in $\Lambda_{M}$.
\begin{definition}
    The monomial symmetric polynomial $m_{\lambda}$'s are a collection of elements in $\lambda$ indexed by partition $\lambda$, such that for $l(\lambda)\le M$,
    $$m_{\lambda}(\Vec{z})=\sum_{(k_{1},...,k_{N})}\sum_{1\le i_{1}<i_{2}<...<i_{k}\le  M}z_{i_{1}}^{k_{1}}z_{i_{2}}^{k_{2}}...\ z_{i_{k}}^{k_{N}},$$
    where $(k_{1},...,k_{N})$ go over all rearrangements of $\lambda_{1}\ge...\ge \lambda_{N}$ without repetitions. We also take $m_{\lambda}(\Vec{z})=0$ for $l(\lambda)>M$.

    The elementary symmetric polynomials  $\{e_{k}\}_{i=1}^{M}$ are
    $$e_{k}(\vec{z})=\sum_{1\le i_{1}<i_{2}<...<i_{k}\le M}z_{i_{1}}z_{i_{2}}...z_{i_{N}}.$$ 
    By definition $e_{k}=m_{1^{k}}$, where $1^{k}$ denotes the partition $(1,1,...,1)$ of length k.

    The power sums  $\{p_{k}\}_{k=1}^{\infty}$ are 
    $$p_{k}(\Vec{z})=z_{1}^{k}+z_{2}^{k}+...
    +z_{M}^{k}.$$
    By definition $e_{k}=m_{(k)}$, where $(k)$ denotes the length 1 partition $\lambda$ such that $\lambda_{1}=k$. 
\end{definition}
\begin{remark}
    It's clear from definition that $\{m_{\lambda}\}$ form a linear basis of $\Lambda_{M}$. Another important fact is that, $\{e_{k}\}_{k=1}^{M}$ and $\{p_{k}\}_{k=1}^{\infty}$ are two sets of algebraic generators of $\Lambda_{M}$, see \cite[Chapter 1]{M}.
\end{remark}

The Jack polynomials play a central role in this text. Let $\Vec{z}$ denote $(z_{1},...,z_{M})$ for some $M\ge 1$,
fix $\theta>0$, and let $X$ be a formal auxiliary variable, $\partial_{i}, i=1,2,...,M$, be the partial derivative operator in $z_{i}$; $V(\Vec{z})=\prod_{1\le i<j\le M}(z_{i}-z_{j})$ be the Vandermonde determinant. 

\begin{definition}\cite[Chapter VI]{M}\label{def:jack}
    Let $D_{M}(X;\theta)$ be a differential operator of the form
    \begin{equation}
        D_{M}(X;\theta)=V(\Vec{z})^{-1}\det\Bigg[z_{i}^{M-j}\left(z_{i}\frac{\partial}{\partial z_{i}}+(M-j)\theta+X\right)\Bigg]_{1\le i,j\le M}.
    \end{equation}
    $D_{M}(X;\theta)$ is a generating function (with variable X) of linear differential operators $D_{M}^{1},...,D_{M}^{M}$ acting on $\Lambda_{M}$, such that 
    $$D_{M}(X;\theta)=\sum_{r=0}^{M}D^{r}_{M}X^{M-r}.$$

    The Jack polynomials in M-variables are a collection of elements $P_{\lambda}(\Vec{z})$ in $\Lambda_{M}$, indexed by partitions $\lambda$ such that $l(\lambda)\le M$. Each $P_{\lambda}(\Vec{z})$ is uniquely determined by the following two properties:
    \begin{equation}\label{eq_jack1}
        P_{\lambda}(\Vec{z})=m_{\lambda}(\Vec{z})+\sum_{\mu<\lambda}u_{\mu}^{\lambda}(\theta)m_{\mu}(\Vec{z}),  
    \end{equation}
   where $u_{\mu}^{\lambda}(\theta)\in \R$ are parameterized by $\theta$, and

\begin{equation}\label{eq_jack2}
    D_{M}(X;\theta)P_{\lambda}(\Vec{z})=c_{\lambda}^{\lambda}(\theta)P_{\lambda}(\Vec{z}),
\end{equation}
    
    where $c_{\lambda}^{\lambda}(\theta)=\prod_{i=1}^{M}(X+\theta^{-1}\lambda_{i}+M-i)$.
\end{definition}

\begin{prop}\label{prop:jack}\cite[chapter VI]{M}
    For $M\ge l(\mu)$, $u_{\mu}^{\lambda}(\theta)$ in (\ref{eq_jack1}) is independent of $M$.
\end{prop}
Because of last proposition, we write $P_{\lambda}(\cdot;\theta)=m_{\lambda}(\cdot)+\sum_{\mu<\lambda}u_{\mu}^{\lambda}(\theta)m_{\mu}(\cdot)$, where $\cdot$ denotes $(z_{1},...,z_{M})$ for arbitrary $M\ge l(\lambda)$, which does not affect the combinatorial expansion in $m_{\mu}$'s. We also introduce another version of Jack polynomial.
\begin{definition}\label{def:jack2}
    The dual of Jack polynomial $Q_{\lambda}(\cdot;\theta)$ as 
    $$Q_{\lambda}(\cdot;\theta)=b_{\lambda}(\theta)\cdot P_{\lambda}(\cdot;\theta),$$
    where $b_{\lambda}(\theta)=\prod_{s\in \lambda}\frac{a(s)+\theta l(s)+\theta}{a(s)+\theta l(s)+1}$.
\end{definition}

\begin{remark}
    It's a nontrivial fact that Jack polynomials satisfying the two defining properties exist. The differential operator $D_{M}(X;\theta)$ was discovered by Sekiguchi in \cite{S}.
\end{remark}


Given two Jack polynomials $P_{v}(\cdot;\theta)$ and  $P_{\mu}(\cdot;\theta)$, their project $P_{v}(\cdot;\theta)\cdot P_{\mu}(\cdot;\theta)$ is again a symmetric polynomial, and hence can be written as a unique linear combination of Jack polynomials. Namely we have the following equality, where $C^{v,\mu}_{\lambda}(\theta)$ is the coefficient of $P_{\lambda}(\cdot;\theta)$ in the expansion:
\begin{equation}\label{eq_coefficient1} 
P_{v}(\cdot;\theta)P_{\mu}(\cdot;\theta)=\sum_{\lambda}C^{v,\mu}_{\lambda}(\theta)P_{\lambda}(\cdot;\theta)
.\end{equation}

We note that $C^{v,\mu}_{\lambda}(\theta)$ is also independent of $M$ because of Proposition $\ref{prop:jack}$.


\subsection{Type BC Bessel functions}\label{sec:typebc}
For positive integers $M\le N$, take a M-tuples of nonnegative real numbers $\Vec{a}=(a_{1}\ge a_{2}\ge...\ge a_{M})$ as the given data. The idea of type BC Bessel function $\B(\Vec{a},z_{1},...,z_{M});\theta)$ is a version of multivariate symmetric Fourier kernel, with certain nontrivial root multiplicities given by parameter $\theta>0$. In the special functions literature this is a special case of the so-called symmetric Dunkl kernel, see Section \ref{sec:dunkl}.

\begin{definition}\label{def:matrixintegral}
For $\theta=\frac{1}{2},1,2$, $M\le N$, the type BC multivariate Bessel functions are defined with parameter $\theta$, and M-tuples of ordered real labels $a=(a_{1}\ge a_{2}\ge \cdots a_{M})$, that
$$\B(\Vec{a};z_1,z_2,...,z_M;\theta, N)=\int  dU\int  dV\ \exp \left(\frac{1}{2}Tr(U\Lambda VZ+Z^{*}V^{*}\Lambda^{*}U^{*})\right),$$
where  \begin{equation}
    \Lambda=\begin{bmatrix}
a_{1} &  & & &&0&...&0\\
 & a_{2} & & &&0&...& 0\\
  &             &...&   &&\\
  &&&...&\\
  &&&&a_{M} &0&...&0
\end{bmatrix}_{M\times N},\end{equation}
\begin{equation}
    Z=\begin{bmatrix}
z_{1} &  & & &\\
 & z_{2} & & &\\
  &             &...&\\
  &&&...&\\
  &&&&z_{M} &\\
  0&&...&&0\\
  && ...&&\\
  0&&...&&0
\end{bmatrix}_{N\times M},\end{equation}
$U\in O(M)/U(M)/Sp(M)$, $V\in O(N)/U(N)/Sp(N)$ are integrated under Haar measures.
\end{definition}

Definition \ref{def:matrixintegral} provides an explicit connection with rectangular matrices, where the integral is of the form as a "Fourier transform"/characteristic function of $A=U\Lambda V$. However, since there is no (skew) field with real dimension $\beta$ for general $\beta>0$, one need to define the Bessel functions in an alternate way that does not rely on explicit matrix structure.
For this purpose, we introduce the notion of type BC Jacobi polynomial, which is indeed the multivariate Jacobi polynomial in Appendix A with a specified root multiplicity function parametrized by $\theta>0$, and was studied in \cite{OO2}.

For $M\in \Z_{\ge 1}$, let $W$ denote the $BC_{M}$ Weyl group 
$W=S_{M}\ltimes \Z_{2}^{M}$, which acts on functions of $\Vec{x}=(x_{1},...,x_{M})$. The $S_{M}$ part permutes $x_{1},...,x_{M}$, and the $\Z_{2}^{M}$ part acts by 
$f(\Vec{x})\mapsto f(x_{1}^{\pm},...,x_{M}^{\pm})$.
\begin{definition}{\cite{OO2}}
    Take three parameters $\theta>0, a,b>-1$.
    The type BC Jacobi polynomials are a collection of functions $J_{\lambda}(\Vec{x};\theta,a,b)$ on the M-dimensional torus
    $$\T=\{(x_{1},...,x_{M})\subset \C^{M}, |x_{1}|=...=|x_{M}|=1\},$$
    indexed by partition $\lambda$. And $J_{\lambda}$'s are determined by the following:

    (1). $J_{\lambda}(\Vec{x};\theta,a,b)=x_{1}^{\lambda_{1}}\cdots x_{M}^{\lambda_{M}}+...$, 
    where the dots stand for lower monomials in the lexicographic order as in Definition \ref{def:YD}, and $J_{\lambda}$ is $W$-invariant,

    (2). $J_{\lambda}$'s are mutually orthogonal in $L^{2}(\T,w)$, with scalar product given by 
    $$\langle f,g\rangle:=\int_{\T}f(\Vec{x})\Bar{g}(\Vec{x})w(\Vec{x})\cdot \text{Haar}(d\Vec{x}),$$
    where 
    $$w(\Vec{x})=\prod_{1\le i<j\le M}|x_{i}-x_{j}|^{2\theta}|1-x_{i}x_{j}|^{2\theta}\prod_{1\le i\le M}|1-x_{i}|^{2a+1}|1+x_{i}|^{2b+1}.$$

    Let $\Phi_{\lambda}(x_{1},...,x_{M};\theta,a,b)$ be the normalized type BC multivariate Jacobi polynomials where $\Phi_{\lambda}(0,...,0;\theta,a,b)=1$. 
\end{definition}
\begin{remark}\label{rem:idjacobi}
    By taking $x_{i}=e^{2z_{i}i}$ for i=1,2,...,M, each $J_{\lambda}$ is identified with the Jacobi polynomial $\mathfrak{J}_{\Tilde{\lambda}}$ in Definition \ref{def:generaljacobi}, where $\Tilde{\lambda_{k}}=2\Tilde{\lambda_{k}}$ for $k=1,2,...,l(\lambda)$.
\end{remark}
We define the type BC Bessel function as a limit of type BC Jacobi polynomial, then present a more concrete power series expression of it in terms of Jack polynomials, using the limit transition.

\begin{definition}\label{def:bessel}
Take $\theta>0$, $M\le N$, $\lambda=\lfloor \epsilon^{-1}(a_{1},...,a_{M}) \rfloor$, $a=\theta(N-M+1)-1$, $b=\theta-1$, $\rho$ be a fixed vector defined as in (\ref{eq_rho}), the type BC multivariate Bessel function labeled by $\Vec{a}=(a_{1}\ge a_{2}\ge...\ge a_{M})$ is an multivariate analytic function in both $\Vec{a}$ and $(z_{1},...,z_{M})$, defined by 
$$\B(i\Vec{a},z_{1},...,z_{M};\theta, N):=\lim_{\epsilon\rightarrow 0}\Phi_{\lfloor\frac{\Vec{a}}{2\epsilon}-\frac{\rho}{2}\rfloor}(e^{2\epsilon z_{1}i},...,e^{2\epsilon z_{M}i};a,b,\theta).$$
\end{definition}
\begin{remark}\label{rem:idbessel}
    Because of Remark \ref{rem:idjacobi}, each $\B(\Vec{a},z_{1},...,z_{M};\theta, N)$ is identified with $f_{\Vec{a}}$ in Definition \ref{def:generalbessel}.  Moreover, by specifying $a,b$ in this way we take the root multiplicities $m_{\pm e_{i}}=2\theta (N-M), m_{\pm 2e_{i}}=2\theta-1, m_{\pm e_{i}\pm e_{j}}=2\theta$, which are parameterized by a single variable $\theta>0$. 
   

\end{remark}

\begin{definition}
For a partition $\mu$, $t\in \R$, $\theta>0$, let 
\begin{equation}\label{eq_Hmu}
H(\mu)=\prod_{s\in \mu}[a(s)+1+\theta l(s)],
\end{equation}

\begin{equation}\label{eq_Hmu'}
H^{'}(\mu)=\prod_{s\in \mu}[a(s)+\theta+\theta l(s)],
\end{equation}
and
\begin{equation}\label{eq_tmu}
(t)_{\mu}=\prod_{s\in \mu}[(t+j-1-\theta (i-1)].    
\end{equation}
\end{definition}

\begin{prop}\label{prop:bessel}
The limit in Definition \ref{def:bessel} exists, and 
\begin{equation}\label{eq_expansion}
\begin{split}
&\  \B(\Vec{a},z_{1},...,z_{M};\theta,N)\\
=&\sum_{\mu}\prod_{i=1}^{M}\frac{\Gamma(\theta N -\theta (i-1))}{\Gamma(\theta N-\theta(i-1)+\mu_{i})}\frac{1}{H(\mu)}2^{-2|\mu|}\frac{P_{\mu}(a_{1}^{2},\cdots,a_{M}^{2};\theta)P_{\mu}(z_{1}^{2},\cdots,z_{M}^{2};\theta)}{P_{\mu}(1^{M};\theta)}\\
=&\sum_{\mu}\prod_{i=1}^{M}\frac{\Gamma(\theta N -\theta (i-1))}{\Gamma(\theta N-\theta(i-1)+\mu_{i})}\frac{\Gamma(\theta M -\theta (i-1))}{\Gamma(\theta M-\theta(i-1)+\mu_{i})}\frac{H^{'}(\mu)}{H(\mu)}2^{-2|\mu|}P_{\mu}(a_{1}^{2},\cdots,a_{M}^{2};\theta)P_{\mu}(z_{1}^{2},\cdots,z_{M}^{2};\theta),
\end{split}
\end{equation}
where $\mu$ is summed over all partitions of length at most $M$.
\end{prop}
\begin{proof}
The existence of the limit is guaranteed by Proposition \ref{prop:generallimittransition}, Theorem \ref{thm:idjacobihypergeometric}, Remark \ref{rem:idjacobi} and \ref{rem:idbessel}. More precisely, by \cite[Theorem 2.32]{Ro}, the multivariate Bessel function is meromorphic on $m$, the root multiplicity function, and the pole set $K^{sing}$ of $m$ is explicitly given in \cite{DJO}. One can check that for all $\theta>0$, $m\notin K^{sing}$. Hence we can do an analytically continuation of (\ref{eq_expansion}) from nonnegative root multiplicities to all $\theta>0$.

We do a concrete calculation for the explicit expression on the right of (\ref{eq_expansion}). 
By \cite[Proposition 2.3]{OO2}, 
\begin{align*}
&\ \Phi_{\lfloor \frac{\Vec{a}}{2\epsilon}-\frac{\rho}{2}\rfloor}(e^{2\epsilon z_{1}i},...,e^{2\epsilon z_{M}i};\theta,a,b) \\
=&\sum_{\mu\le \lfloor \frac{\Vec{a}}{2\epsilon}-\frac{\rho}{2}\rfloor}\frac{I_{\mu}(\lfloor \frac{\Vec{a}}{2\epsilon}-\frac{\rho}{2}\rfloor;\theta;\sigma+M)P_{\mu}(2cos(2\epsilon z_{j})-2;\theta)}{C(M,\mu;\theta;a,b)},
\end{align*}
where $I_{\mu}(x_{1},...,x_{M};\theta,h)$ is defined in \cite [Proposition 2.2]{OO2}, $\sigma=(a+b+1)/2$, and 
$$C(M,\mu;\theta,a,b)=I_{\mu}(\mu;\theta,\sigma+\theta M)J_{\mu}(1^{M};\theta,a,b).$$
By comparing \cite[(2.3) and (2.4)]{OO2}, we see that as an inhomogeneous polynomial of $x_{1},..., x_{M}$, $I_{\mu}(x_{1},..., x_{M};\theta;h)$ has highest degree term $P_{\mu}(x_{1}^{2},..., x_{M}^{2};\theta)$. Therefore asymptotically 
$$I_{\mu}(\lfloor \frac{\Vec{a}}{2\epsilon}-\frac{\rho}{2}\rfloor;\theta;\sigma+M)\approx P_{\mu}(a_{1}^{2},..., a_{M}^{2};\theta)2^{-2|\mu|}\epsilon^{-2|\mu|}.$$

On the other hand, when $\epsilon$ is small, 
$$P_{\mu}(2cos(2\epsilon z_{j})-2;\theta)\approx P_{\mu}(-(2\epsilon z_{j})^{2};\theta)= P_{\mu}(z_{j}^{2};\theta)(-4)^{|\mu|}\epsilon^{2|\mu|},$$ 
so it remains to match the coefficients. This follows by (see \cite[(10,20)]{M}) 
\begin{equation}\label{eq_Jack(1)}
P_{\mu}(1^{M};\theta)=\frac{(M\theta)_{\mu}}{H^{'}(\mu)},   \end{equation}
and (see \cite[Remark 2.5]{OO2}) 
\begin{align*}
\begin{split}
&\ C(M,\mu;\theta,\theta(N-M+1)-1,\theta-1)\\
=&4^{\mu}\cdot \frac{H(\mu)}{H^{'}(\mu)}\prod_{i=1}^{M}\frac{\Gamma(\mu_{i}+(M-i+1)\theta)}{\Gamma((M-i+1)\theta)}\frac{\Gamma(\mu_{i}+(N-i+1)\theta)}{\Gamma((N-i+1)\theta)}. \qedhere
\end{split}
\end{align*}
\end{proof}

The following example gives a connection of $\B(\cdot,z_{1},...,z_{M};\theta, N)$ with the usual single variable Bessel function.
\begin{ex}\label{ex:usualbessel}
    When  $M=1$, 
    $$\B(a,iz;\theta, N)=\Gamma(N\theta)\cdot (\frac{az}{2})^{-(N\theta-1)} B_{N\theta-1}(az),$$
    where $B_{\alpha}$ is the Bessel function of the first kind.
\end{ex}

Definition \ref{def:bessel} generalizes the notion of the type BC Bessel function to any $\theta>0$. In particular, when $\theta=\frac{1}{2}, 1, 2$, Definition \ref{def:bessel} provides an explicit power series expansion of the matrix integral in Definition \ref{def:matrixintegral}. 
There are more than one way to show the equivalence of these two expressions, and the one we present below relies on the representation theory lying behind the concrete objects.

\begin{thm}\label{thm:fouriertransform}
For $\theta=\frac{1}{2}, 1, 2$, 
\begin{align*}
\begin{split}
& \int  dU\int  dV\ exp(i\cdot Tr(U\Lambda VZ+Z^{*}V^{*}\Lambda^{*}U^{*}))\\
=&\lim_{\epsilon\rightarrow 0}\Phi_{\lfloor\frac{\Vec{a}}{\epsilon}-\frac{\rho}{2}\rfloor}(e^{2\epsilon z_{1}i},...,e^{2\epsilon z_{M}i};a,b,\theta)
\end{split}
\end{align*}
where the matrix integral on the left is defined in the same way as in Definition \ref{def:matrixintegral}, only differs by a constant $2i$ in the exponent.
\end{thm}
\begin{proof}
 $\Phi_{\lfloor\frac{\Vec{a}}{\epsilon}-\frac{\rho}{2}\rfloor}(e^{2 z_{1}i},...,e^{2 z_{M}i};a,b,\theta)$ is identified with spherical function of $O(M+N)/O(M)\times O(N),U(M+N)/U(M)\times U(N),Sp(M+N)/Sp(M)\times Sp(N)$ respectively according to Theorem \ref{thm:identification}, and the root multiplicity list in Appendix B. After limit transition in Proposition \ref{prop:generallimittransition} or Remark \ref{rem:spherical}, it suffices to identify the matrix integral with the corresponding Euclidean spherical function, which we again refer to \cite{Hel1}, \cite{Hel2}.
\end{proof}

\begin{remark}
The expression of matrix integral in Definition \ref{def:matrixintegral} as power series in Definition \ref{def:bessel} is not new, and could be found in \cite[ Section 13.4.3]{Forrester} with a different proof. See the Appendix C for more information and yet another short proof of this result.
\end{remark}

\subsection{Type BC Dunkl operators}\label{sec:dunkl}
As a special class of differential operators, Dunkl operators were introduced in \cite{D}, and can be thought as a generalization of the usual partial derivatives on multivariate analytic functions, that take Fourier kernels as eigenfunction. We briefly review the basic general theory of Dunkl opertors in Appendix A, and in this section, we specify to a special class of rational Dunkl operators under root system of type BC, which is parametrized by a single variable $\theta>0$ and plays a central role in Section \ref{sec:lowtemp}. For the convenience of readers, we redefine this operator in a more concrete and straightforward way.


\begin{definition}\label{def:dunkl}
    For $N\ge M\ge 2, \theta>0$,
    let $D_{i}$ be a differential operator acting on analytic functions on $\C^{M}$ with variables $z_{1},..,z_{M}$, that
    \begin{equation}
        D_{i}=\partial_{i}+\Big[\theta(N-M+1)-\frac{1}{2}\Big]\frac{1-\sigma_{i}}{z_{i}}+\theta\sum_{j\ne i}\Big[\frac{1-\sigma_{ij}}{z_{i}-z_{j}}+\frac{1-\tau_{ij}}{z_{i}+z_{j}}\Big],
    \end{equation}
    where $\sigma_{i}$ interchanges $z_{i}$ and $-z_{i}$, $\sigma_{ij}$ interchanges $z_{i}$ and $z_{j}$, and $\tau_{ij}$ interchanges $z_{i}$ and $-z_{j}$.
\end{definition}

\begin{remark}
    $D_{i}$'s are special cases of the rational Dunkl operator in Definition \ref{def:rationaldunkl}, such that the reflections $s_{\alpha}$ for $\alpha\in R$ are specified as following: 
    $$\sigma_{i}=s_{e_{i}},\ \ \sigma_{ij}=s_{e_{i}-e_{j}},\ \ \tau_{ij}=s_{e_{i}+e_{j}}.$$
    Moreover, the root multiplicity function is given by $m_{\pm e_{i}}=2\theta (N-M), m_{\pm 2e_{i}}=2\theta-1, m_{\pm e_{i}\pm e_{j}}=2\theta$, the same as type BC Bessel function in Section \ref{sec:typebc}. 
\end{remark}

\begin{prop}\cite{D}\label{prop:commutativity}
    The Dunkl operators of same root multiplicities commute, i.e, 
    $$D_{i}D_{j}=D_{j}D_{i}$$
    for any $1\le i,j\le M$.
\end{prop}
The following result provides connection of type BC multivariate Bessel functions and Dunkl operators, namely, the former are eigenfunctions of the latter.

\begin{definition}
    Fix $M\ge 1$. For $k=1,2,...$, denote
    $$\mathrm{P}_{k}=D_{1}^{k}+...+D_{M}^{k}.$$
\end{definition}
\begin{thm}\label{thm:dunklonbessel}
Given $\Vec{a}=(a_{1}\ge ...\ge a_{M})$ for each $k=1,2,...$,
    \begin{equation}\label{eq_eigenfunction}
        \mathrm{P}_{2k}\B(\Vec{a}, z_{1},..,z_{M};\theta)=\left(\sum_{i=1}^{M}(a_{i})^{2k}\right)\cdot \B(\Vec{a},z_{1},...,z_{M};\theta).
    \end{equation}
\end{thm}
\begin{proof}
    This is a special case of Definition \ref{def:generalbessel}. 
\end{proof}
\begin{remark}
     From Proposition \ref{prop:bessel}, one can see that $\B(\Vec{a},z_{1},...,z_{M};\theta)$ is symmetric under actions of Weyl group of root system $BC_{M}$, namely, invariant by interchanging $z_{i}$ with $z_{j}$ and replacing $z_{i}$ by $-z_{i}$. Similarly, it's necessary to take symmetric power sum of $D_{i}'s$ with even power, which satisfies the same symmetry. 

\end{remark}

\subsection{Matrix addition and moments}\label{sec:addition}

For $\Vec{c}=\Vec{a}\boxplus_{M,N}^{\theta}\Vec{b}$, we assume in this section that $\Vec{a},\Vec{b}$ are deterministic, and recall from Definition \ref{def:deterministicaddition} that the distribution $\mathfrak{m}$ of $\Vec{c}$ is given by testing on type BC Bessel function.
Note that polynomials are bounded and smooth on compact sets, and therefore are legitimate test functions of $\mathfrak{m}$. Moreover, by Proposition \ref{prop:characteristic2} Bessel function is analytic and symmetric on $\C^{M}$, so we can view it as a generating function of symmetric polynomials of M variables $c_{1},...,c_{M}$. More precisely, by expanding Bessel functions on both sides of (\ref{eq_additioindef}) using (\ref{eq_expansion}), we have the following:

\begin{prop}\label{prop:polyexpectation}
For each partition $\lambda$ with $l(\lambda)\le M$, let $\Vec{c}=\Vec{a}\boxplus_{M,N}^{\theta}\Vec{b}$, then
\begin{equation}\label{eq_polyexpectation}
\begin{split}
\E{P_{\lambda}(c_{1}^{2},...,c_{M}^{2};\theta)}
=&\sum_{|v|+|\mu|=|\lambda|}\frac{H(\lambda)}{H(v)H(\mu)}\frac{\prod_{i=1}^{M}\Gamma(\theta (N-i+1))\Gamma(\theta (N-i+1)+\lambda_{i})}{\prod_{i=1}^{M}\Gamma(\theta (N-i+1)+v_{i})\Gamma(\theta (N-i+1)+\mu_{i})}\\
&\frac{P_{\lambda}(1^{M};\theta)}{P_{v}(1^{M};\theta)P_{\mu}(1^{M};\theta)}C^{v,\mu}_{\lambda}(\theta)P_{v}(a_{1}^{2},...,a_{M}^{2};\theta)P_{\mu}(b_{1}^{2},...,b_{M}^{2};\theta)
\end{split}
\end{equation}
where $v,\mu$ are two partitions of length at most $M$.
\end{prop}

Proposition \ref{prop:polyexpectation} provides explicit data of the distribution of random singular values $\Vec{a}\boxplus_{M,N}^{\theta}\Vec{b}$ in terms of moments. It is believed, but not yet proved that $\mathfrak m$ (with such moments) is indeed a (symmetric) positive probability measure on $\R^{M}$. For $\beta=1,2,4$ this holds automatically because the probability measure is constructed explicitly by the matrix structure, while for general $\beta>0$, the randomness of $\Vec{a}\boxplus_{M,N}^{\theta}\Vec{b}$ holds and is studied in this text in the weaker sense given by (\ref{eq_polyexpectation}).

\subsection{Type BC Bessel generating functions}\label{sec:bgf}
For $\Vec{a}=(a_{1}\ge ...\ge a_{M}\ge 0)$, we assume that $\Vec{a}$ is random, and its distribution is given by a symmetric generalized function $\mathfrak{m}$, testing on smooth functions and in particular polynomials on $\R_{\ge 0}^{M}$. 

\begin{definition}\label{def:bgf}
    Fix $M\le N, \theta>0$. Given a compactly supported symmetric generalized function $\mathfrak{m}$ on $\R_{\ge 0}^{M}$ defined as above, let the Bessel generating function of $\mathfrak{m}$ be a function of $z_{1},...,z_{M}$ given by 
    \begin{equation}\label{eq_bgf}
        G_{N,\theta}(z_{1},...,z_{M};\mathfrak{m}):=\langle \mathfrak{m}, \B(\Vec{a},z_{1},...,z_{M};N,\theta)\rangle,
    \end{equation}
    where the bracket denotes testing $\mathfrak{m}$ by $\B(\Vec{a},z_{1},...,z_{M};\theta, N)$, in which $\Vec{a}$ are the variables and $z_{1},...,z_{M}$ are parameters.
\end{definition}

We also define the Bessel generating function for a class of fast decaying probability measures, for potential applications of our theory (see e.g Section \ref{sec:laguerre}). As preparation, we state a uniform upper bound of multivariate Bessel functions.
\begin{prop}\label{prop:besselbound}
    For  any  $\theta>0$,  $M\le N$, $\Vec{a}=(a_{1}\ge...\ge a_{M})\in \R_{\ge 0}^{M}$, $z=(z_{1},...,z_{M})\in \R^{M}$,  we have
    \begin{equation}\label{eq_besselbound1}
        0\le \B(\Vec{a},z_{1},...,z_{M};\theta,N)\le \left[1+\frac{1}{\theta}\left(\frac{a_{1}|z|}{2}\right)^{2}e^{\frac{a_{1}|z|}{2}}\right]^{M},
    \end{equation}
    and for any $k_{1},...,k_{s}\in \Z_{\ge 1}$,
    \begin{equation}\label{eq_besselbound2}
        \left|\left(\prod_{i=1}^{s}P_{2k_{i}}\right)\B(\Vec{a},z_{1},...,z_{M};\theta,N)\right|\le \prod_{i=1}^{s}\left(\sum_{j=1}^{M}a_{i}^{2k_{i}}\right)\left[1+\frac{1}{\theta}\left(\frac{a_{1}|z|}{2}\right)^{2}e^{\frac{a_{1}|z|}{2}}\right]^{M}.
    \end{equation}
\begin{proof}
    From Proposition \ref{prop:characteristic2}, it's clear that $\B(\Vec{a},z_{1},...,z_{M};\theta,N)\ge 0$, and since $P_{\mu}(1^{M};\theta)=\frac{(M\theta)_{\mu}}{H^{'}(\mu)}$,
    \begin{equation}
    \begin{split}
        &\B(\Vec{a},z_{1},...,z_{M};\theta,N)\\
        =&\sum_{\mu}\prod_{i=1}^{M}\frac{\Gamma(\theta N -\theta (i-1))}{\Gamma(\theta N-\theta(i-1)+\mu_{i})}\frac{(M\theta)_{\mu}}{H(\mu)H^{'}(\mu)}2^{-2|\mu|}\frac{P_{\mu}(a_{1}^{2},\cdots,a_{M}^{2};\theta)P_{\mu}(z_{1}^{2},\cdots,z_{M}^{2};\theta)}{P_{\mu}(1^{M};\theta)^{2}}\\
        \le&\sum_{\mu}\prod_{i=1}^{M}\frac{\Gamma(\theta N -\theta (i-1))}{\Gamma(\theta N-\theta(i-1)+\mu_{i})}\frac{(M\theta)_{\mu}}{H(\mu)H^{'}(\mu)}2^{-2|\mu|}a_{1}^{2|\mu|}z_{1}^{2|\mu|}\\
        \le&\sum_{\mu}\prod_{i=1}^{M}\left[\frac{\Gamma(\theta N -\theta (i-1))}{\Gamma(\theta N-\theta(i-1)+\mu_{i})}\frac{\Gamma(\theta M-\theta(i-1)+\mu_{i})}{\Gamma(\theta M-\theta(i-1))}\right]\frac{1}{\prod_{i=1}^{M}\mu_{i}!}\frac{1}{\prod_{i=1}^{M}\prod_{j=0}^{\mu_{i}-1}(\theta+j)}2^{-2|\mu|}a_{1}^{2|\mu|}z_{1}^{2|\mu|}\\
        \le&\sum_{\mu_{1}\ge...\ge\mu_{M}\ge 0}\frac{1}{\prod_{i=1}^{M}\mu_{i}!}\frac{1}{\prod_{i=1}^{M}\prod_{j=0}^{\mu_{i}-1}(\theta+j)}\left(\frac{a_{1}z_{1}}{2}\right)^{2|\mu|}\\
        \le&\prod_{i=1}^{M}\left(\sum_{\mu_{i}=0}^{\infty}\frac{1}{\mu_{i}!\prod_{j=0}^{\mu_{i}-1}(\theta+j)}\left(\frac{a_{1}z_{1}}{2}\right)^{2\mu_{i}}\right)
        \le\prod_{i=1}^{M}\left(1+\sum_{\mu_{i}=1}^{\infty}\frac{1}{\theta}\frac{1}{[(\mu_{i}-1)!]^{2}}\left(\frac{a_{1}|z|}{2}\right)^{2\mu_{i}}\right)\\
        \le &\prod_{i=1}^{M}\left(1+\frac{1}{\theta}\left(\frac{a_{1}|z|}{2}\right)^{2}e^{\frac{a_{1}|z|}{2}}\right)=\left[1+\frac{1}{\theta}\left(\frac{a_{1}|z|}{2}\right)^{2}e^{\frac{a_{1}|z|}{2}}\right]^{M}.
    \end{split}    
    \end{equation}
    This verifies (\ref{eq_besselbound1}). (\ref{eq_besselbound2}) follows from (\ref{eq_besselbound1}) and Theorem \ref{thm:dunklonbessel}.
\end{proof}
\end{prop}
\begin{definition}\label{def:expdecaying}
    We  say a measure $\mathfrak{m}$ on M-tuples $a_{1}\ge...\ge a_{M}\ge 0$ is \emph{exponentially decaying} with exponent $R>0$, if 
    $$\int e^{MRa_{1}} \mu(da_{1},...,da_{M})<\infty.$$
\end{definition}

    By Proposition \ref{prop:besselbound} and Definition \ref{def:expdecaying}, the Bessel generating function of $\mathfrak{m}$, where $\mathfrak{m}$ is a compactly supported generalized function or exponentially decaying measure, is well-defined on a domain near 0. Moreover,
    we will take $\mathfrak{m}$ to be of total mass $1$, which means $\langle \mathfrak{m}, 1\rangle=1$, where $1$ is the constant function 1. So we have 
    $$G_{N,\theta}(0,...,0;\mathfrak{m})=1.$$



Now we generalize the addition to random vectors $\Vec{a}$ and $\Vec{b}$ following Definition \ref{def:deterministicaddition}.

\begin{definition}\label{def:additioningeneral}
Given $\theta>0$, $M\le N$, let $\Vec{a}=(a_{1}\ge...\ge a_{M}\ge 0)$, $\Vec{b}=(b_{1}\ge...\ge b_{M}\ge 0)$ be two random M-tuples whose distribution are given by generalized functions $\mathfrak{m}_{a}$ and $\mathfrak{m}_{b}$ on $\R_{\ge 0}^{M}$. Let $\Vec{c}$ be a symmetric random vector in $\R_{\ge 0}^{M}$ whose distribution is given by generalized function $\mathfrak{m}_{c}$, such that 
\begin{equation}\label{eq_additioingeneral}
G_{N,\theta}(z_{1},...,z_{M};\mathfrak{m}_{c})=
G_{N,\theta}(z_{1},...,z_{M};\mathfrak{m}_{a})\cdot G_{N,\theta}(z_{1},...,z_{M};\mathfrak{m}_{b}).    
\end{equation}
We write 
$$\Vec{c}=\Vec{a}\boxplus_{M,N}^{\theta}\Vec{b}.$$    
\end{definition}


Since $\B(z_{1},...,z_{M};\theta,N)$ is behaving nice enough in the analytic sense, one can interchange the differentiation over $z_{1},...,z_{M}$ and the pairing with $\mathfrak{m}$, and therefore Theorem 
\ref{thm:dunklonbessel} generalizes to the following.
\begin{thm}\label{thm:dunklonbgf}
    Let $\mathfrak{m}$ be a symmetric compactly supported generalized function on $\R^{M}$, or a exponential decaying measure as in Definition \ref{def:expdecaying} with exponent $R$. Let $k_{1},...,k_{s}\in \Z_{\ge 1}$. Then $G_{N,\theta}( z_{1},..,z_{M};\mathfrak{m})$ is analytic as a function of $(z_{1},...,z_{M})$ (in the domain $\{z\in \R^{M}: |z|<R\}$ in the second case). Moreover,
    \begin{equation}\label{eq_eigenfunction2}
        \left(\prod_{i=1}^{s}\mathrm{P}_{2k_{i}}\right)G_{N,\theta}( z_{1},..,z_{M};\mathfrak{m})\Bigr|_{z_{1}=...z_{M}=0}=\biggl\langle\mathfrak{m}, \prod_{i=1}^{s}\Big(\sum_{j=1}^{M}(a_{j})^{2k_{i}}\Big)\biggr\rangle.
    \end{equation}
    The above properties also hold for $$G_{N,\theta}(z_{1},...,z_{M};\mathfrak{m}_{c})=
G_{N,\theta}(z_{1},...,z_{M};\mathfrak{m}_{a})\cdot G_{N,\theta}(z_{1},...,z_{M};\mathfrak{m}_{b}),$$ where $\mathfrak{m}_{a}$, $\mathfrak{m}_{b}$ are of the above two types.
\end{thm}
\begin{proof}
    This follows from dominated convergence theorem, where the uniform upper bounds of $\B(\cdot,z_{1},...,z_{M};\theta, N)$ and its derivatives are given by Proposition \ref{prop:besselbound}.
\end{proof}


\section{Concentration in low temperature}\label{sec:lowtemp}
In this section, we fix the size of matrices $M$,$N$ and the input as deterministic input $\Vec{a}$, $\Vec{b}$, and study the behavior of $\Vec{c}=\Vec{a}\boxplus_{M,N}^{\theta}\Vec{b}$ as $\theta\rightarrow \infty$. According to the statistical physic interpretation, when $\theta\rightarrow \infty$ the temperature is going down to 0, and hence the random vector $\Vec{c}$ will freeze at some deterministic M-tuples. 
\subsection{Finite Law of Large Numbers}
Before taking the limit, we consider the expected characteristic polynomial of $CC^{*}$ for each $\theta<\infty$. It turns out that the expression does not really depend on $\theta$. The following lemma will be used later in the proof. Let $C^{v,\mu}_{\lambda}(\theta)$ be the coefficient defined in $\ref{eq_coefficient1}$.

\begin{lemma}\label{lem:automorphism}
When $\lambda=1^{l}$, $C^{v,\mu}_{\lambda}(\theta)\ne 0$ only when $v=1^{i}$, $\mu=1^{j}$, and $i+j=l$. Moreover, 
\begin{equation}\label{eq_coefficient2}
C^{1^{i},1^{j}}_{1^{l}}(\theta)=\frac{\prod_{m=1}^{l}(\frac{l\theta-m\theta+1}{l\theta-m\theta+1})}{\prod_{m=1}^{i}(\frac{i\theta-m\theta+\theta}{i\theta-m\theta+1})\prod_{m=1}^{j}(\frac{j\theta-m\theta+\theta}{j\theta-m\theta+1})}.    
\end{equation}
\end{lemma}
\begin{proof}
This is studied in \cite{GM}, and for the convenience of the readers we reproduce the proof. Applying the automorphism $\omega_{\theta}$ of the algebra of symmetric functions (see \cite[Chapter VI, Section 10]{M}, which acts on Jack polynomials in the following way:
\begin{equation}\label{eq_Jackduality}\omega_{\theta}(P_{\lambda}(\cdot;\theta))=Q_{\lambda^{'}}(\cdot;\theta^{-1}),
\end{equation}
(\ref{eq_coefficient1}) becomes

\begin{equation}\label{eq_coefficient3}
Q_{(i,0,...)}(\cdot;\theta^{-1})\cdot Q_{(j,0,...)}(\cdot;\theta^{-1})=\sum_{\mu}C^{1^{i},1^{j}}_{\mu}(\theta)\cdot Q_{\mu^{'}}(\cdot;\theta^{-1}). 
\end{equation}
Recall that $Q_{\lambda}(\cdot;\theta)=b_{\lambda}(\theta)P_{\lambda}(\cdot;\theta)=\frac{H(\lambda)}{H^{'}(\lambda)}P_{\lambda}(\cdot;\theta)$. By comparing the coefficient of the leading monomial $z_{1}^{l}$, we have 
\begin{equation}\label{eq_coefficient4}
C^{1^{i},1^{j}}_{1^{l}}(\theta)=\frac{b_{1^{i}}(\theta^{-1})b_{1^{j}}(\theta^{-1})}{b_{1^{l}}(\theta^{-1})},
\end{equation}
and $C^{v,\mu}_{1^{l}}=0$ if $v$ or $\mu$ has more than one column.
\end{proof}

\begin{thm}\label{thm:characteristicpoly}
Fix $M\le N$, given $\Vec{a}$ and $\Vec{b}$, let $\Vec{c}=\Vec{a}\boxplus_{M,N}^{\theta}\Vec{b}$. Take $z$ as a formal variable, and let 
\begin{equation}\label{eq_characteristicpoly}
P_{M,N}^{\theta}(z)=\E{\prod_{i=1}^{M}(z-c_{i}^{2})}.
\end{equation}
Then the explicit expression of $P_{M,N}^{\theta}(z)$ is $\theta$-independent, and 
\begin{align*}
P_{M,N}^{\theta}(z)
=P_{M,N}(z)         
\end{align*}
for all $\theta>0$, where $P_{M,N}(z)$ is defined in (\ref{eq_charpoly}).
\end{thm}

\begin{proof}
Rewrite the product on the right side of (\ref{eq_characteristicpoly}) as 
$$\prod_{i=1}^{M}(z-c_{i}^{2})=\sum_{l=0}^{M}(-1)^{l}e_{l}(c_{1}^{2},...,c_{M}^{2})z^{M-l},$$
it turns out that $P_{M,N}^{\theta}(z)$ is given by the moments of $\{c_{i}^{2}\}_{i=1}^{M}$ only in terms of elementary symmetric polynomials.

Taking the partition $\lambda=(1^{j},0^{M-j})$, $P_{\lambda}(x;\theta)=e_{j}(x)$ for any $\theta>0$. We use Proposition \ref{prop:polyexpectation} and it remains to specify the coefficients. 
From Lemma \ref{lem:automorphism} we get 
$$\frac{H(1^{l})}{H(1^{i})H(1^{j})}C^{1^{i},1^{j}}_{1^{l}}(\theta)=\frac{H^{'}(1^{l})}{H^{'}(1^{i})H^{'}(1^{j})}=\frac{l!}{i!j!}.$$
Moreover, direct calculation yields
$$\frac{e_{l}(1^{M})}{e_{i}(1^{M})e_{j}(1^{M})}=\frac{i!(M-i)!j!(M-j)!}{M!l!(M-l)!},$$
and when $\lambda=1^{l}, v=1^{i}, \mu=1^{j}$,
$$\frac{\prod_{i=1}^{M}\Gamma(\theta (N-i+1))\Gamma(\theta (N-i+1)+\lambda_{i})}{\prod_{i=1}^{M}\Gamma(\theta (N-i+1)+v_{i})\Gamma(\theta (N-i+1)+\mu_{i})}=\frac{(N-i)!}{N!}\frac{(N-j)!}{(N-l)!}.$$ Combine all these together finishes the proof.
\end{proof}

We highlight the connection of our result with the so-called finite free probability, which was initiated in recent years by Marcus, Spielman and Srivastava and studies convolution of polynomials. Given two polynomials $p(z)=\sum_{i=0}^{M}z^{M-i}a_{i}, q(z)=\sum_{i=0}^{M}z^{M-i}b_{i}$ with degree at most $M$, \cite{MSS} defines the rectangular additive convolution for two $M\times M$ matrices, and \cite{GrM} generalizes it to arbitrary rectangular matrices, such that the $(M,N)^{th}$ rectangular additive convolution of $p(z)$ and $q(z)$ is defined as 
$$p(z)\boxplus\boxplus_{M}^{N}q(z)=\sum_{l=0}^{M}z^{M-l}(-1)^{l}\Bigg(\frac{(M-i)!(M-j)!}{M!(M-l)!}\frac{(N-i)!(N-j)!}{N!(N-l)!}\Bigg)a_{i}b_{j}.$$

In \cite{GrM}, it is shown that taking $p(z)=\chi_{z}(AA^{*})$, $q(z)=\chi_{z}(BB^{*})$, where $A$ and $B$ are two $M\times N$ real/complex matrices, taking $\chi_{z}(\cdot)$ to be the characteristic polynomial  $\det(zI-\cdot)$, and let $U_{M\times M},V_{N\times N}$ are independent Haar orthogonal/unitary, then $p(z)\boxplus\boxplus_{M}^{N}q(z)=\E{\chi_{z}((A+UBV)(A+UBV)^{*})}$. Theorem \ref{thm:characteristicpoly} generalizes this operation from $\beta=1,2$ to arbitrary $\beta>0$, with a different approach not relying on the concrete matrix structure. In particular it shows that the rectangular additive convolution is $\beta-$independent.

Our next result is the law of large number of $\Vec{c}=\Vec{a}\boxplus_{M,N}^{\theta}\Vec{b}$ in the regime $\theta\rightarrow \infty$. As preparation we state a combinatorial result. Given partitions $v,\mu,\lambda$ such that $v_{1}, \mu_{1}\le \lambda_{1}$, $l(v),\ l(\mu),\ l(\lambda)\le M$, 
let $\{k_{l}\}$ be an index set that $l=1,2,...,\lambda_{k}-\lambda_{k+1}, k=1,2,...,M$, and $\{i_{k_{l}}\}, \{j_{k_{l}}\}$ be two collections of nonnegative integers. We do not distinguish $\{i_{k_{l}}\}_{l=1}^{\lambda_{k}-\lambda_{k+1}}$ with $\{i_{k_{\sigma(l)}}\}_{l=1}^{\lambda_{k}-\lambda_{k+1}}$, where $\sigma\in S_{\lambda_{k}-\lambda_{k+1}}$ is an arbitrary permutation, and same for $\{j_{k_{l}}\}_{l=1}^{\lambda_{k}-\lambda_{k+1}}$.

\begin{prop}\label{prop:combcounting}
Let $C^{v,\mu}_{\lambda}$ be the coefficient of $m_{\lambda}(\cdot)$ in the expansion 
$$m_{v}(\cdot)\cdot m_{\mu}(\cdot)=\sum_{\lambda}C^{v,\mu}_{\lambda}m_{\lambda}(\cdot).$$
Then  
\begin{align*}
C^{v^{'},\mu^{'}}_{\lambda^{'}}= \text{\#\ ways\ to\ choose}\ \{i_{k_{l}}\}, \{j_{k_{l}}\}\ \text{such\ that\ for}\ m=1,2,...,M,
\end{align*}
\begin{equation}\label{eq_combcounting}
\begin{split}
\begin{dcases}
    v_{m}=\sum_{k=1}^{M}\sum_{l=1}^{\lambda_{k}-\lambda_{k+1}}I_{i_{k_{l}}\ge m}; \\  
    \mu_{m}=\sum_{k=1}^{M}\sum_{l=1}^{\lambda_{k}-\lambda_{k+1}}I_{j_{k_{l}}\ge m}.
\end{dcases}    
\end{split}
\end{equation}
\end{prop}
\begin{proof}
We choose $\{i_{k_{l}}\},\{j_{k_{l}}\}$ in an explicit way. By definition of $C^{v,\mu}_{\lambda}$, we are combining column $v^{'}_{l_{1}}$ with column $\mu^{'}_{l_{2}}$ to get a column $\lambda^{'}_{l_{3}}$, where $l_{1}, l_{2}, l_{3}$ are chosen among $1,2,...,\lambda_{1}$, and $v^{'}_{l_{1}}, \mu^{'}_{l_{2}}$ might be of length 0. Inspired by this, let $\{i_{k_{l}}\}_{l=1}^{\lambda_{k}-\lambda_{k+1}}$ be the length of (distinct) columns of $v$, $\{j_{k_{l}}\}_{l=1}^{\lambda_{k}-\lambda_{k+1}}$ be the length of (distinct) columns of $\mu$, which are chosen to contribute to $\lambda^{'}_{\lambda_{k+1}+1},...,\lambda^{'}_{\lambda_{k}}$.

We immediately see that the above way to choose $\{i_{k_{l}}\}, \{j_{k_{l}}\}$ satisfy (\ref{eq_combcounting}), whose total number is equal to $C^{v^{'},\mu^{'}}_{\lambda^{'}}$. It remains to check each way of choosing $\{i_{k_{l}}\}, \{j_{k_{l}}\}$ can be interpreted in this way. Given a sequence of nonnegative integers $\{i_{k_{l}}\}, \{j_{k_{l}}\}$ satisfying (\ref{eq_combcounting}), we have for $m=1,2,...,M$,
\begin{equation}\label{eq_combcounting2}
\begin{split}
\begin{dcases}
    v_{m}-v_{m+1}=\sum_{k=1}^{M}\sum_{l=1}^{\lambda_{k}-\lambda_{k+1}}I_{i_{k_{l}}=m}; \\  
    \mu_{m}-\mu_{m+1}=\sum_{k=1}^{M}\sum_{l=1}^{\lambda_{k}-\lambda_{k+1}}I_{j_{k_{l}}= m}.
\end{dcases}    
\end{split}
\end{equation}
Then one can split $\{i_{k_{l}}\}_{l=1}^{\lambda_{k}-\lambda_{k+1}}(k=1,2,...,M)$ into disjoint groups, such that the number of elements in group $m$ is exactly $v_{m}-v_{m+1}$, which is equal to the number of length $m$ columns in $v$. Vice versa for $\{j_{k_{l}}\}_{l=1}^{\lambda_{k}-\lambda_{k+1}}(k=1,2,...,M)$.
\end{proof}


\begin{proof}[Proof of Theorem \ref{thm:lln}:]
The weak convergence to a delta function on polynomial test functions is equivalent to the statement that, given any arbitrary collection of polynomials $f_{1},...,f_{n}$ of M variables, we have 
\begin{equation}\label{eq_lln1}
\lim_{\theta\rightarrow \infty}\E{\prod_{i=1
}^{n}f_{i}(\Vec{c^{2}})}=\lim_{\theta\rightarrow\infty}\prod_{i=1}^{n}\left[\E{f_{i}(\Vec{c^{2}})}\right].
\end{equation}

Since $\Vec{c}$ is symmetric in distribution, if suffices to consider symmetric polynomials in $\Lambda_{M}$, which can be generated (in the sense of algebra) by elementary symmetric functions $e_1,...,e_{M}$. Since (\ref{eq_lln1}) is multilinear in $f_{i}'s$, we reduce to showing for any positive integers $k_1,...,k_{M}$,
\begin{equation}\label{eq_lln2}
\lim_{\theta\rightarrow \infty}\E{\prod_{i=1
}^{M}e_{i}(\Vec{c^{2}})^{k_{i}}}\stackrel{?}{=}\lim_{\theta\rightarrow\theta}\prod_{i=1}^{M}\left[\E{e_{i}(\Vec{c^{2}}}\right]^{k_{i}}.
\end{equation}
Once we show this, the deterministic limit of $\Vec{c}$ will be a M-tuples $\Vec{\lambda}$, 
such that $\EE\Big[e_{i}(\Vec{c})\Big]=e_{i}(\Vec{\lambda})$ for all $i=1,2,...,M$. Then Theorem \ref{thm:characteristicpoly} identifies $\Vec{\lambda}$ with roots of $P^{\theta}_{M,N}(z)$. 

We connect the left side of (\ref{eq_lln2}) with Jack polynomials, using the following result (\cite[Proposition 7.6]{St}):
\begin{equation}\label{eq_lln3}
\lim_{\theta\rightarrow \infty}P_{\lambda}(z_{1},...,z_{M};\theta)=\prod_{i=1}^{M}[e_{i}(z_{1},...,z_{M})]^{\lambda_{i}-\lambda_{i+1}},  \end{equation}
for any partition $\lambda$.
Then, let $\lambda_{i}=k_{i}+...+k_{M}$, the left side of (\ref{eq_lln2}) becomes the limit of $\E{P_{\lambda}(\Vec{c^{2}};\theta)}$, since each product of $e_{i}(\Vec{c^{2}})'s$ has bounded expectation for $\theta>0$ due to the fact that $\Vec{c}$ is bounded supported.
Again by Proposition \ref{prop:polyexpectation}, 
\begin{equation}\label{eq_lln6}
\begin{split}
\E{P_{\lambda}(c_{1}^{2},...,c_{M}^{2};\theta)}
=&\sum_{|v|+|\mu|=|\lambda|}\frac{H(\lambda)}{H(v)H(\mu)}\frac{\prod_{i=1}^{M}\Gamma(\theta (N-i+1))\Gamma(\theta (N-i+1)+\lambda_{i})}{\prod_{i=1}^{M}\Gamma(\theta (N-i+1)+v_{i})\Gamma(\theta (N-i+1)+\mu_{i})}\\
&\frac{P_{\lambda}(1^{M};\theta)}{P_{v}(1^{M};\theta)P_{\mu}(1^{M};\theta)}C^{v,\mu}_{\lambda}(\theta)P_{v}(a_{1}^{2},...,a_{M}^{2};\theta)P_{\mu}(b_{1}^{2},...,b_{M}^{2};\theta).
\end{split}
\end{equation}
Taking $\theta\rightarrow\infty$, 
$$\frac{\prod_{i=1}^{M}\Gamma(\theta (N-i+1))\Gamma(\theta (N-i+1)+\lambda_{i})}{\prod_{i=1}^{M}\Gamma(\theta (N-i+1)+v_{i})\Gamma(\theta (N-i+1)+\mu_{i})}\longrightarrow \prod_{m=1}^{M}(N-m+1)^{\lambda_{m}-v_{m}-\mu_{m}},$$
and since by definition 
$$P_{v}(\cdot;\theta)P_{\mu}(\cdot;\theta)=\sum_{\lambda}C^{v,\mu}_{\lambda}(\theta)P_{\lambda}(\cdot;\theta),$$
applying $\omega_{\theta}$ on both sides (c.f. the proof of Lemma \ref{lem:automorphism}), and use the fact that (see \cite[Proposition 7.6]{St}) 
\begin{equation}\label{eq_lln4}
\lim_{\theta\rightarrow 0}P_{\lambda}(z_{1},...,z_{M};\theta)=m_{\lambda}(z_{1},...,z_{M}),   
\end{equation}
we have
\begin{equation}\label{eq_lln5}
\begin{split}
C^{v,\mu}_{\lambda}(\theta)\frac{H(\lambda)}{H(v)H(\mu)}&\longrightarrow C^{v^{'},\mu^{'}}_{\lambda^{'}}\cdot \frac{\lim_{\theta\rightarrow \infty} H^{'}(\lambda)}{\lim_{\theta\rightarrow \infty} H^{'}(v)\cdot \lim_{\theta\rightarrow \infty} H^{'}(\mu)}\\
&=C^{v^{'},\mu^{'}}_{\lambda^{'}}\cdot \frac{\prod_{s\in \lambda}(l(s)+1)}{\prod_{s\in v}(l(s)+1)\cdot \prod_{s\in \mu}(l(s)+1)}.  
\end{split}
\end{equation}
Moreover, applying (\ref{eq_lln3}) again on $\frac{P_{\lambda}(1^{M};\theta)}{P_{v}(1^{M};\theta)P_{\mu}(1^{M};\theta)}$, $P_{v}(a_{1}^{2},...,a_{M}^{2};\theta)P_{\mu}(b_{1}^{2},...,b_{M}^{2};\theta)$, the right side of (\ref{eq_lln6}) goes to 
\begin{equation}\label{eq_lln7}
\begin{split}
\sum_{|v|+|\mu|=|\lambda|}C^{v^{'},\mu^{'}}_{\lambda^{'}}&\frac{\prod_{s\in \lambda}(l(s)+1)}{\prod_{s\in v}(l(s)+1)\cdot \prod_{s\in \mu}(l(s)+1)}\frac{\prod_{i=1}^{M}\binom{M}{i}^{\lambda_{i}-\lambda_{i+1}}}{\prod_{i=1}^{M}\binom{M}{i}^{v_{i}-v_{i+1}}\prod_{i=1}^{M}\binom{M}{i}^{\mu_{i}-\mu_{i+1}}}\\
&\prod_{m=1}^{M}(N-m+1)^{\lambda_{i}-v_{i}-\mu_{i}}\prod_{i=1}^{M}\left[e_{i}(a_{1}^{2},...,a_{M}^{2})\right]^{v_{i}-v_{i+1}}\prod_{i=1}^{M}\left[e_{i}(b_{1}^{2},...,b_{M}^{2})\right]^{\mu_{i}-\mu_{i+1}}.
\end{split}
\end{equation}

On the other hand, by Theorem \ref{thm:characteristicpoly}, the right side of (\ref{eq_lln2}) is equal to
\begin{equation}\label{eq_lln8}
\begin{split}
&\prod_{k=1}^{M}\left[\EE\big[e_{i}(\Vec{c^{2}})\big]\right]^{\lambda_{k}-\lambda_{k+1}}\\
=&\prod_{k=1}^{M}\left[\sum_{i+j=k}\frac{(M-i)!(M-j)!}{M!(M-k)!}\frac{(N-i)!(N-j)!}{N!(N-k)!}e_{i}(a_{1}^{2},...,a_{M}^{2})e_{j}(b_{1}^{2},...,e_{M}^{2})\right]^{\lambda_{k}-\lambda_{k+1}}.
\end{split}
\end{equation}
It remains to check that (\ref{eq_lln7}) is equal to (\ref{eq_lln8}).

We open the bracket in (\ref{eq_lln8}), and identify each term in the sum with a unique collection of nonnegative integer valued indices $\{k_{l}\}_{l=1}^{\lambda_{k}-\lambda_{k+1}}(k=1,2,...,M)$, such that $i_{k_{l}}+j_{k_{l}}=k$ for each $l$. Moreover, such term is a multiple of $\prod_{i=1}^{M}\left[e_{i}(a_{1}^{2},...,a_{M}^{2})\right]^{v_{i}-v_{i+1}}\prod_{i=1}^{M}\left[e_{i}(b_{1}^{2},...,b_{M}^{2})\right]^{\mu_{i}-\mu_{i+1}}$, where for $i=1,2,...,M$,
\begin{equation}\label{eq_lln9}
\begin{split}
v_{i}-v_{i+1}=\sum_{k=1}^{M}\sum_{l=1}^{\lambda_{k}-\lambda_{k+1}}I_{i_{k_{l}}=m},\\
\mu_{i}-\mu_{i+1}=\sum_{k=1}^{M}\sum_{l=1}^{\lambda_{k}-\lambda_{k+1}}I_{j_{k_{l}}=m},\\
\lambda_{i}-\lambda_{i+1}=\sum_{k=1}^{M}\sum_{l=1}^{\lambda_{k}-\lambda_{k+1}}I_{k=m},
\end{split}
\end{equation}
which matches (\ref{eq_combcounting2}). Hence it remains to match the coefficients, i.e, to show that 
\begin{equation}\label{eq_lln14}
\begin{split}
&\sum_{i_{k_{l}}+j_{k_{l}}=k}\prod_{k=1}^{M}\prod_{l=1}^{\lambda_{k}-\lambda_{k+1}}\left[\frac{(M-i_{k_{l}})!(M-j_{k_{l}})!}{M!(M-k)!}\frac{(N-i_{k_{l}})!(N-j_{k_{l}})!}{N!(N-k)!}\right]\\
\stackrel{?}{=}&C^{v^{'},\mu^{'}}_{\lambda^{'}}\frac{\prod_{s\in \lambda}[l(s)+1]}{\prod_{s\in v}[l(s)+1]\cdot \prod_{s\in \mu}[l(s)+1]}\\
&\cdot\frac{\prod_{i=1}^{M}\binom{M}{i}^{\lambda_{i}-\lambda_{i+1}}}{\prod_{i=1}^{M}\binom{M}{i}^{v_{i}-v_{i+1}}\prod_{i=1}^{M}\binom{M}{i}^{\mu_{i}-\mu_{i+1}}}\prod_{m=1}^{M}(N-m+1)^{\lambda_{m}-v_{m}-\mu_{m}}.   
\end{split}    
\end{equation}

We first rewrite the left side:
\begin{equation}
\begin{split}
\frac{(M-i_{k_{l}})!(M-j_{k_{l}})!}{(M-k)!M!}
=\prod_{m=1}^{M}(M-m+1)^{\sum_{k=1}^{M}(I_{m\le k}-I_{m\le i_{k_{l}}}-I_{m\le j_{k_{l}}})},
\end{split}
\end{equation}
hence 
\begin{equation}\label{eq_lln10}
\begin{split}
&\prod_{k=1}^{M}\prod_{l=1}^{\lambda_{k}-\lambda_{k+1}}\left[\frac{(M-i_{k_{l}})!(M-j_{k_{l}})!}{(M-k)!M!}\frac{(N-i_{k_{l}})!(N-j_{k_{l}})!}{(N-k)!N!}\right]\\
=&\prod_{m=1}^{M}(M-m+1)^{\sum_{k=1}^{M}\sum_{l=1}^{\lambda_{k}-\lambda_{k+1}}\left[I_{m\le k}-I_{m\le i_{k_{l}}}-I_{m\le j_{k_{l}}}\right]}\\
\cdot&\prod_{m=1}^{M}(N-m+1)^{\sum_{k=1}^{M}\sum_{l=1}^{\lambda_{k}-\lambda_{k+1}}\left[I_{m\le k}-I_{m\le i_{k_{l}}}-I_{m\le j_{k_{l}}}\right]}. 
\end{split}
\end{equation}
On the right side,
\begin{equation}\label{eq_lln11}
\begin{split}
    \prod_{m=1}^{M}(N-m+1)^{\lambda_{m}-v_{m}-\mu_{m}}
    =\prod_{m=1}^{M}(N-m+1)^{\sum_{k=1}^{M}\sum_{l=1}^{\lambda_{k}-\lambda_{k+1}}(I_{m\le k}-I_{m\le i_{k_{l}}}-I_{m\le j_{k_{l}}})}.
\end{split}
\end{equation}
And
\begin{equation}
\begin{split}
    \binom{M}{i}=\prod_{m=1}^{M}(M-m+1)^{1-I_{m\ge i+1}-I_{m\ge M-i+1}},
\end{split}
\end{equation}
hence
\begin{equation}\label{eq_lln12}
\begin{split}
&\frac{\prod_{i=1}^{M}\binom{M}{i}^{\lambda_{i}-\lambda_{i+1}}}{\prod_{i=1}^{M}\binom{M}{i}^{v_{i}-v_{i+1}}\prod_{i=1}^{M}\binom{M}{i}^{\mu_{i}-\mu_{i+1}}}\\
=&\prod_{i=1}^{M}\binom{M}{i}^{\sum_{k=1}^{M}\sum_{l=1}^{\lambda_{k}-\lambda_{k+1}}\left[I_{k=i}-I_{i_{k_{l}}=i}-I_{j_{k_{l}}=i}\right]}\\
=&\prod_{m=1}^{M}(M-m+1)^{\Big(\sum_{i=1}^{M}\left[1-I_{m\ge i+1}-I_{m\ge M-i+1}\right]\Big)\Big(\sum_{k=1}^{M}\sum_{l=1}^{\lambda_{k}-\lambda_{k+1}}\left[I_{k=i}-I_{i_{k_{l}}=i}-I_{j_{k_{l}}=i}\right]\Big)}\\
=&\prod_{m=1}^{M}(M-m+1)^{\sum_{k=1}^{M}\sum_{l=1}^{\lambda_{k}-\lambda_{k+1}}\left[I_{m\le k}-I_{m\le i_{k_{l}}}-I_{m\le j_{k_{l}}}\right]+\left[I_{m\ge M-i_{k_{l}}+1}+I_{m\ge M-j_{k_{l}}+1}-I_{m\ge M-k+1}\right]}.
\end{split}
\end{equation}

Finally, 
\begin{equation}\label{eq_lln13}
\begin{split}
    &\prod_{m=1}^{M}(M-m+1)^{\sum_{k=1}^{M}\sum_{l=1}^{\lambda_{k}-\lambda_{k+1}}\Big[-\Big(I_{m\ge M-i_{k_{l}}+1}+I_{m\ge M-j_{k_{l}}+1}-I_{m\ge M-k+1}\Big)\Big]}\\
    =&\prod_{m=1}^{M}(M-m+1)^{\sum_{k=1}^{M}\sum_{l=1}^{\lambda_{k}-\lambda_{k+1}}\Big[-\Big(I_{i_{k_{l}}\ge M-m+1}-I_{j_{k_{l}}\ge M-m+1}+I_{k\ge M-m+1}\Big)\Big]}\\
    =&\prod_{m=1}^{M}m^{\sum_{k=1}^{M}\sum_{l=1}^{\lambda_{k}-\lambda_{k+1}}\Big[I_{m\le k}-I_{m\le i_{k_{l}}}-I_{m\le j_{k_{l}}}\Big]}\\
    =&\frac{\prod_{j=1}^{\lambda_{1}}\lambda_{j}^{'}!}{\prod_{j=1}^{\lambda_{1}}v_{j}^{'}!\prod_{j=1}^{\lambda_{1}}\mu_{j}^{'}!}
    =\frac{\prod_{s\in \lambda}[l(s)+1]}{\prod_{s\in v}[l(s)+1]\prod_{s\in \mu}[l(s)+1]}.
\end{split}   
\end{equation}
(\ref{eq_lln14}) follows from (\ref{eq_lln11}), (\ref{eq_lln12}), (\ref{eq_lln13}) and Proposition \ref{prop:combcounting}.
\end{proof}

\subsection{Gaussian Fluctuation for $1\times N$ matrix}
Take $M=1$, so that $A$ and $B$ are two $1\times N$ matrices with singular values $a_{1}, b_{1}\ge 0$, and let $c_{1}=a_{1}\boxplus_{1,N}^{\theta}b_{1}$. When taking $\theta\rightarrow\infty$, Theorem \ref{thm:lln} shows 
$$c_{1}^{2}\longrightarrow \lambda_{1}^{2}$$
in moments, where 
\begin{equation}\label{eq_fluctuation1}
    \E{c_{1}^{2}}=\E{e_{1}(c^{2})}=a_{1}^{2}+b_{1}^{2}=\lambda_{1}^{2}.
\end{equation}

Based on this result, we consider further the fluctuation of $c_{1}$ around $\lambda_{1}$ in $\theta\rightarrow\infty$ regime, which turns out to be a Gaussian random variable under proper rescaling. 
\begin{thm}\label{thm:fluctuation}
    For $a_{1},b_{1}\ge 0$, let $\lambda_{1}^{2}=a_{1}^{2}+b_{1}^{2}$, and $c_{1}=a_{1}\boxplus_{1,N}^{\theta}b_{1}$. As $\theta\rightarrow\infty$, we have:
\begin{equation}\label{eq_fluctuation2}
\sqrt{\theta}(c_{1}^{2}-\lambda_{1}^{2})\stackrel{d}{\longrightarrow} Z,  
\end{equation}
where $Z\sim\mathcal{N}(0, \frac{2}{N}a_{1}^{2}b_{1}^{2})$.
\end{thm}
\begin{remark}
    We expect the Gaussian fluctuation behavior  of $\Vec{c}=\Vec{a}\boxplus_{M,N}^{\theta}\Vec{b}$ when $\theta\rightarrow\infty$, for general $M>1$, and we leave the generalization of Theorem \ref{thm:fluctuation} as an open problem.
\end{remark}
\begin{proof}
    We first show that the convergence holds in the sense of moments. By Proposition \ref{prop:polyexpectation}, for all $l=1,2,...$
\begin{equation}\label{eq_fluctuation3}
\E{c_{1}^{2l}}=\sum_{k_{1}+k_{2}=l}\frac{l!}{k_{1}!k_{2}!}\frac{\Gamma(\theta N)\Gamma(\theta N+l)}{\Gamma(\theta N+k_{1})\Gamma(\theta N+k_{2})}a_{1}^{2k_{1}}b_{1}^{2k_{2}},    
\end{equation}
since in (\ref{eq_polyexpectation}) $\lambda=(l,0,...)$, $v=(k_{1},0,...)$, $\mu=(k_{2},0,...)$ and $C^{v,\mu}_{\lambda}(\theta)\equiv 1$.

Then for $m\in \Z_{\ge 0}$,
\begin{align*}
    &\E{(c_{1}^{2}-\lambda_{1}^{2})^{m})}\\
    =&\sum_{k_{1}+k_{2}+k=m}\frac{(-1)^{k}m!}{(k_{1}+k_{2})!k!}\frac{(k_{1}+k_{2})!}{k_{1}!k_{2}!}\frac{\Gamma(\theta N)\Gamma(\theta N+k_{1}+k_{2})}{\Gamma(\theta N+k_{1})\Gamma(\theta N+k_{2})}a_{1}^{2k_{1}}b_{1}^{2k_{2}}(a_{1}^{2}+b_{1}^{2})^{k}\\
    =&\sum_{k_{1}+k_{2}+k_{3}+k_{4}=m}(-1)^{k}\frac{m!}{k_{1}!k_{2}!k_{3}!k_{4}!}\frac{\Gamma(\theta N)\Gamma(\theta N+k_{1}+k_{2})}{\Gamma(\theta N+k_{1})\Gamma(\theta N+k_{2})}a_{1}^{2k_{1}+2k_{3}}b_{1}^{2k_{2}+2k_{4}}.
\end{align*}
This implies for fixed $l_{1}+l_{2}=m\ge 0$, coefficient of monomial $a_{1}^{2l_{1}}b_{1}^{2l_{2}}$ in $\E{[\sqrt{\theta}(c_{1}^{2}-\lambda_{1}^{2})]^{m}}$ is 
\begin{align}
 &\sqrt{\theta}^{l_{1}+l_{2}}m!\sum_{k_{3}=0}^{l_{1}}\sum_{k_{4}=0}^{l_{2}}\frac{(-1)^{k_{3}}}{(l_{1}-k_{3})!k_{3}!}\frac{(-1)^{k_{4}}}{(l_{2}-k_{4})!k_{4}!}\frac{\Gamma(\theta N)\Gamma(\theta N+l_{1}+l_{2}-k_{3}-k_{4})}{\Gamma(\theta N+l_{1}-k_{3})\Gamma(\theta N+l_{2}-k_{4})}\\
    =&\sqrt{\theta}^{l_{1}+l_{2}}m!\sum_{k_{3}=0}^{l_{1}}\sum_{k_{4}=0}^{l_{2}}\frac{(-1)^{l_{1}-k_{3}}}{(l_{1}-k_{3})!k_{3}!}\frac{(-1)^{l_{2}-k_{4}}}{(l_{2}-k_{4})!k_{4}!}\frac{\Gamma(\theta N)\Gamma(\theta N++k_{3}+k_{4})}{\Gamma(\theta N+k_{3})\Gamma(\theta N+k_{4})}.\label{eq_fluctuation4}
\end{align}

It remains to match the above expression with moments of $Z$. We use the following lemma, whose prove is postponed.
\begin{lemma}\label{lem:fluctuation}
    For any $l=1,2,...$, with $z$ as a formal variable,
    
    (a).\begin{equation}\label{eq_fluctuationlemma1}
        \sum_{p=0}^{l}\frac{(-1)^{(l-p)}}{(l-p)!p!}(z+p)(z+p+1)...(z+p+q-1)=0
        \end{equation} 
if $q=0,1,2,...,l-1$.

    (b).\begin{equation}\label{eq_fluctuationlemma2}
        \sum_{p=0}^{l}\frac{(-1)^{(l-p)}}{(l-p)!p!}(z+p)(z+p+1)...(z+p+q-1)=1
    \end{equation}
if $q=l$.
\end{lemma}

Without loss of generality, assume that $l_{1}\ge l_{2}$ in (\ref{eq_fluctuation4}), and we rewrite (\ref{eq_fluctuation4}) as
$$\sqrt{\theta}^{l_{1}+l_{2}}m!\sum_{k_{4}=0}^{l_{2}}\frac{(-1)^{l_{2}-k_{4}}}{(l_{2}-k_{4})!k_{4}!}\frac{\Gamma(\theta N)}{\Gamma(\theta N+k_{4})}\Bigg[\sum_{k_{3}=0}^{l_{1}}\frac{(-1)^{l_{1}-k_{3}}}{(l_{1}-k_{3})!k_{3}!}(\theta N+k_{3})(\theta N+k_{3}+1)...(\theta N+k_{3}+k_{4}-1)\Bigg].$$
By Lemma \ref{lem:fluctuation}, the sum in the bracket is nonzero only when $k_{4}=l_{1}$, which implies $l_{1}=l_{2}$, $m=l_{1}+l_{2}=2l_{1}$, and (\ref{eq_fluctuation4}) becomes 
$$\sqrt{\theta}^{2l_{1}}\frac{(2l_{1})!}{l_{1}!}\frac{1}{\theta N(\theta N+1)...(\theta N+l_{1}-1)}.$$
Therefore, the odd moments of $\sqrt{\theta}(c_{1}^{2}-\lambda_{1}^{2})$ are all zero, and the $2k^{th}$ moment of $\sqrt{\theta}(c_{1}^{2}-\lambda_{1}^{2})$ is equal to 
$$\sqrt{\theta}^{2k}\frac{(2k)!}{k!}\frac{1}{\theta N(\theta N+1)...(\theta N+k-1)}a_{1}^{2k}b_{1}^{2k},$$
which converges to 
$$\frac{(2k)!}{k!}\frac{1}{N^{k}}a_{1}^{2k}b_{1}^{2k}=(2k-1)!!(\frac{2}{N})^{k}a_{1}^{2k}b_{1}^{2k}$$
as $\theta\rightarrow\infty$.
This coincides with the moments of $Z\sim \mathcal{N}(0,\frac{2}{N}a_{1}^{2}b_{1}^{2})$. By Example \ref{ex:usualbessel} and \cite[Theorem 2]{MP}, which states that products of two usual Bessel functions can be written as a convex combination of Bessel functions, we have that for each $\theta>0$, $c_{1}^{2}$ is supported by a legitimate probability measure $\mu_{\theta}$. The convergence of second moment when $\theta\rightarrow\infty$ implies that $\{\mu_{\theta}\}_{\theta>0}$ are tight, hence (\ref{eq_fluctuation2}) follows from the moment convergence.
\end{proof}

\begin{proof}[Proof of Lemma \ref{lem:fluctuation}.]
    (a). Expand the polynomial of $z$. For each coefficient of $z^{0}, z^{1}, z^{2},..., z^{q}$, it can be written as a polynomial of $p$ with degree at most $q$ with integer coefficients, and hence an integral linear combination of $p, p(p-1), p(p-1)(p-2),..., p(p-1)\cdots (p-q+1)$, so the left side of (\ref{eq_fluctuationlemma1}) becomes an integral linear combination of binomial sums of $$\sum_{p=q}^{l}\frac{(-1)^{l-p}}{(l-p)!(p-q)!}=(1+(-1))^{l-q},$$ which equals to 0 when $q\le l-1$.

    (b). Following the same idea as (a), the only nonvanishing term is the coefficient of $z^{0}$, which equals to
    $$\sum_{p=0}^{l}\frac{(-1)^{l-p}}{(l-p)!p!}p(p-1)\cdots(p-l+1)=\sum_{p=l}^{l}\frac{(-1)^{l-p}}{(l-p)!(p-l)!}=1.\qedhere$$
\end{proof}

\section{Law of large number in high temperature}\label{sec:lln}
In this section, fix two parameters $\gamma>0, q\ge 1$. we explore the behavior of empirical measures of a $M\times N$ random matrix $C$, in the regime that taking $M,N\rightarrow\infty$, $\theta\rightarrow 0$, $M\theta\rightarrow \gamma$, $N\theta\rightarrow q\gamma$. To simplify the notation, sometimes we only write $M\theta\rightarrow\infty, N\theta\rightarrow q\gamma$ to denote the same regime.

\subsection{Main results}

Consider M-tuples of real numbers $\Vec{c}=(c_{1}\ge ...\ge c_{M}\ge 0)$, which should be thought as singular values of some (virtual) rectangular matrix. Suppose that there is a sequence of random M-tuples $\{\Vec{c}_{M}\}_{M=1}^{\infty}$ where $\Vec{c}_{M}=(c_{M,1}\ge...\ge c_{M,M}\ge 0)$, and the distribution of $\Vec{c}_{M}$ is given in the sense as in Theorem \ref{thm:dunklonbgf}. Denote its empirical measure by $\mu_{M}=\frac{1}{M}\sum_{i=1}^{M}(\delta_{c_{M,i}}+\delta_{-c_{M,i}})$.

We set up a condition in terms of moments, that under some mild technical assumption, is equivalent to the weak convergence, in probability, of the random empirical measures $\{\mu_{M}\}$ to some limiting probability measure $\mu$ when $M\rightarrow \infty$. The moments of $\mu$ are all finite and given by $m_{k}$'s.

\begin{definition}{(LLN)}\label{def:llnsatisfaction}
    Let $\{\Vec{c}_{M}\}_{M=1}^{\infty}$ be a sequence of random M-tuples defined as above. For $k=1,2,...,$ denote 
    $$p^{M}_{k}=\frac{1}{M}\sum_{i=1}^{2M}[c_{M,i}^{k}+(-c_{M,i})^{k}].$$
    We say $\{\Vec{c}_{M}\}$ \emph{satisfies a law of large numbers}, if there exists deterministic real numbers $\{m_{k}\}_{k=1}^{\infty}$ such that for any s=1,2,... and any $k_{1},...,k_{s}\in \Z_{\ge 1}$, we have
    \begin{equation}\label{eq_rectangularlln}
        \lim_{M\rightarrow \infty}\E{\prod_{i=1}^{s}p_{k_{i}}^{M}}=\prod_{i=1}^{s}m_{k_{i}}.
    \end{equation}
\end{definition}

Denote the Bessel generating function of $\Vec{c}_{M}$ by 
$$G_{M,N;\theta}(z_{1},...,z_{M}):=G_{N,\theta}(z_{1},...,z_{M};\mathfrak{m}_{\Vec{c}_{M}}).$$
Recall from the Section \ref{sec:bgf} that $G_{M,N;\theta}(0,...,0)=1$, and $G_{M,N;\theta}(z_{1},...,z_{M})$ is analytic on a domain near $(0,...,0)$. Under these conditions, $\ln(G_{M,N;\theta}(z_{1},...,z_{M}))$ is analytic near $(0,...,0)$, and $\ln(G_{M,N;\theta}(0,...,0))=0$. 

Next, we introduce a condition of the partial derivatives of ln$(G_{M,N;\theta}(z_{1},...,z_{M}))$ at 0, as $M\rightarrow\infty$.

\begin{definition}{($q$-$\gamma$-LLN-appropriateness)}\label{def:llnappropriateness}
Given the sequence $\{\Vec{c}_{M}\}_{M=1}^{\infty}$, if for a sequence of real numbers $\{k_{l}\}_{l=1}^{\infty}$, the following limits hold:
\begin{equation*}
\begin{split}
    &\text{(a).} \  \lim_{M\theta\rightarrow \gamma, N\theta\rightarrow q\gamma}\frac{\partial^{l}}{\partial z_{i}^{l}}ln(G_{M,N;\theta})\Bigr|_{z_{1}=,...,z_{M}=0}=(l-1)!\cdot k_{l},\ \text{for}\  \text{all}\  l,i\in\Z_{\ge 1}.\\
    &\text{(b).}  \ \lim_{M\theta\rightarrow \gamma, N\theta\rightarrow q\gamma}\frac{\partial}{\partial z_{i_{1}}}...\frac{\partial}{\partial z_{i_{r}}}ln(G_{M,N;\theta})\Bigr|_{z_{1},..,z_{M}=0}=0,\ \text{for}\  \text{all} \ r\ge 2,\  \text{and}\ i_{1},...,i_{r}\in \Z_{\ge 1}\ \text{such\ that}\\ 
    & \text{the\ set}\ \{i_{1},...,i_{r}\} \text{is\ of\ cardinality\ at\ least\ two}.  
\end{split}
\end{equation*}
    We say $\{k_{l}\}_{l=1}^{\infty}$ are the limiting $q$-$\gamma$ cumulants of $\{\Vec{c}_{M}\}$.
\end{definition}
\begin{remark}
    By Proposition \ref{prop:polyexpectation}, $k_{l}$ are always 0 for all odd l's.
\end{remark}
\begin{remark}
    Writing 
    $$g^{M,N,\theta}(z)=\frac{\partial}{\partial z}ln(G_{M,N;\theta}(z,0,...,0))=\sum_{l=1}^{\infty}k^{M,N,\theta}_{l}z^{l-1},$$
    we have $$k^{M,N,\theta}_{l}\longrightarrow k_{l}$$
as $M\theta\rightarrow \gamma, N\theta\rightarrow q\gamma$. 
\end{remark}

Our main theorem connects Definition \ref{def:llnsatisfaction}, \ref{def:llnappropriateness} and gives a quantitative relation between moments and q-$\gamma$ cumulants of the limiting empirical measure of $\{\mu_{M}\}_{M=1}^{\infty}$, which is stated using generating function. Consider $\R[[z]]$, the space of all formal power series of variable $z$ with real coefficients.
\begin{definition}
    Let $a(z)$ be an element in $\R[[z]]$. We define four linear operators acting on $\R[[z]]$ to itself, such that for any $n=0,1,2,...$
    \begin{align*}
    &(1). \ \partial(z^{n}):=n\cdot z^{n-1}\\
    &(2).\ d(z^{n}):=\begin{dcases}
    0  &  n=0; \\  
    z^{n-1} & n\ge 1,\\
\end{dcases}\\
    &(3).\ d^{'}(z^{n}):=\begin{dcases}
    0  &  n \;\; \text{is}\;\;\text{even}; \\  
    2z^{n-1} & n \;\;\text{is}\;\;\text{odd},\\
\end{dcases}\\    
    &(4).\  *_{a}(z^{n}):=a(z)\cdot z^{n}.
   \end{align*}
\end{definition}

\begin{definition}\label{def:operators}
   Let $\mathrm{T_{k\rightarrow m}^{q,\gamma}}:\R^{\infty}\rightarrow \R^{\infty}$ 
be an operation sending a countable sequence $\{k_{l}\}_{l=1}^{\infty}$ to another sequence $\{m_{2k}\}_{k=1}^{\infty}$, such that for each $k=1,2,...$
    \begin{equation}\label{eq_cumulanttomoment}
        m_{2k}=[z^{0}]\left(\partial+2\gamma d+((q-1)\gamma-\frac{1}{2})d^{'}+*_{g}\right)^{2k-1}g(z),
    \end{equation}
where $[z^{0}]$ takes the constant term of the formal power series in $\R[[z]]$, and 
$$g(z)=\sum_{l=1}^{\infty}k_{l}z^{l-1}.$$
\end{definition}
\begin{remark}
    Note by a simple induction on $k=1,2,...$ that (\ref{eq_cumulanttomoment}) implies each $m_{2k}$ is given by 
    \begin{equation*}
        \text{a\ positive\ constant\ time}\ k_{2k}+ \text{a\ polynomial\ of}\ k_{2},k_{4},...,k_{2k-2}.
    \end{equation*}
    Hence, $\mathrm{T_{k\rightarrow m}^{q,\gamma}}$ is an invertible map, such that given a sequence of real numbers $\{m_{2k}\}_{k=1}^{\infty}$, there exists a unique real sequence $\{k_{l}\}_{l=1}^{\infty}$ with $k_{l}=0$ for all odd l's, and $\mathrm{T}_{k\rightarrow m}^{q,\gamma}\left(\{k_{l}\}_{l=1}^{\infty}\right)=\{m_{2k}\}_{k=1}^{\infty}$. More precisely, $\{m_{2j}\}_{j=1}^{k}$ are corresponding to $\{k_{l}\}_{l=1}^{2k}$. We denote the inverse map by 
    $$\{k_{l}\}=\mathrm{T_{m\rightarrow k}^{q,\gamma}}(\{m_{2k}\}).$$
    In Section \ref{sec:momentcumulant} we provide various points of views on the maps $\mathrm{T}_{k\rightarrow m}^{q,\gamma}$ and $\mathrm{T}_{m\rightarrow k}^{q,\gamma}$.
\end{remark}
We are ready to present the main result now.
\begin{thm}{(Convergence of empirical measure in high temperature)}\label{thm:hightemperaturemainthm}
    The sequence of random M-tuples $\{\Vec{c}_{M}\}_{M=1}^{\infty}$ satisfies LLN, if and only if it is q-$\gamma$-LLN-appropriate. 
    
    If this occurs, we have
    \begin{equation}\label{eq_cumulantmomentformula}
        \{m_{2k}\}_{k=1}^{\infty}=\mathrm{T}_{k\rightarrow m}^{q,\gamma}(\{k_{l}\}_{l=1}^{\infty}),
    \end{equation}
    where $\{k_{l}\}_{l=1}^{\infty}$ are the q-$\gamma$ cumulants corresponding to $\{m_{2k}\}_{k=1}^{\infty}$.
\end{thm}
\subsection{Asymptotic expression under Dunkl actions}
The proof of Theorem \ref{thm:hightemperaturemainthm} is relying on the actions of Dunkl operator introduced in Section \ref{sec:dunkl} on Bessel generating functions. Before proceeding to the proof, we first study the explicit expression of this action in detail.

Consider a symmetric function $F(z_{1},..,z_{M})$ which is analytic on a complex domain near 0. Then the Talor expansion of $F$ of $k^{th}$ order is 
\begin{equation}\label{eq_talorpolynomial}
   F(z_{1},...,z_{M})=\sum_{\lambda:|\lambda|\le k,\ l(\lambda)\le M}c^{\lambda}_{F}\cdot m_{\lambda}(\Vec{z})+O(||z||^{k+1}), 
\end{equation}

where $m_{\lambda}(\Vec{z})$ is the monomial symmetric polynomial indexed by $\lambda$. If we further assume $F$ to be a symmetric function in $z_{1}^{2},...,z_{M}^{2}$, then
\begin{equation}\label{eq_doublecolumn}
  c^{\lambda}_{F}\;\; \text{is\ nonzero\ only\ if}\ \lambda\ \text{is\ even.}  
\end{equation}

Fix $M\ge 1$. Recall we denote 
$$P_{k}=D_{1}^{k}+...+D_{M}^{k},$$
where $D_{i}$ is defined in Section \ref{sec:dunkl}.

The following theorem is a technical result on the explicit expansion of $\exp(F(z_{1},...,z_{M}))$ under the action of $P_{k}$'s, and it serves as a stepping stone to the proof of Theorem \ref{thm:hightemperaturemainthm}.
\begin{thm}\label{thm:technicalequivalence}
    Fix $k=2,4,...$ and a even partition $\lambda$ and $|\lambda|=2k$. Let $F(z_{1},...,z_{M})$ be a symmetric function on $\R^{M}$ satisfying (\ref{eq_doublecolumn}), analytic on a domain near $(0,...,0)$ and $F(0,...,0)=0$. Then
    \begin{equation}\label{eq_dunklaction}
    \begin{split}
        M^{-l(\lambda)}\left[\prod_{i=1}^{l(\lambda)}P_{\lambda_{i}}\right]&\exp(F(z_{1},...,z_{M})\Bigr|_{z_{1}=...z_{M}=0}=b^{\lambda}_{\lambda}\cdot c^{\lambda}_{F}+\sum_{\mu:|\mu|=k,l(\mu)>l(\lambda)}b^{\lambda}_{\mu}\cdot c^{\mu}_{F}\\
        &+L(c_{F}^{(i)},1\le i\le 2k-1)+R_{1}(c^{v}_{F},|v|<2k)+M^{-1}R_{2}(c^{v}_{F},|v|\le 2k),
    \end{split}
    \end{equation}
    where $b^{\lambda}_{\mu}$ are coefficients that are uniformly bounded in the limit regime $M\rightarrow\infty, N\rightarrow\infty, \theta\rightarrow0, M\theta\rightarrow \gamma, N\theta\rightarrow q\gamma$, and the notation $(i)$ denotes the partition $(i,0,...,0)$.
    In particular,
    \begin{equation}\label{eq_leadingcoefficient}
    \begin{split}
        \lim_{M\theta\rightarrow,N\theta\rightarrow q\gamma}b^{\lambda}_{\lambda}=\prod_{i=1}^{l(\lambda)}&\Big[\lambda_{i}(\lambda_{i}-2+2q\gamma)(\lambda_{i}-2+2q\gamma)(\lambda_{i}-4-2q\gamma)(\lambda_{i}-4+2\gamma)\\&...
        (2+2q\gamma)(2+2\gamma)2q\gamma \Big],
    \end{split}
    \end{equation}
    and 
    \begin{equation}\label{eq_L}
    \begin{split}
        L(c^{(i)}_{F},1\le i\le 2k-1)=&\prod_{i=1}^{l(\lambda)}\left([z^{0}](\partial+2\gamma d+((q-1)\gamma-\frac{1}{2})d^{'}+*_{g})^{\lambda_{i}-1}g(z)\right)\\
        -&2k(2k-2+2q\gamma)(2k-2+2\gamma)(2k-4-2q\gamma)(2k-4+2\gamma)\\
        &...(2+2q\gamma)(2+2\gamma)2q\gamma\cdot c^{(k)}_{F}\cdot \mathrm{I}_{l(\lambda)=1}.
    \end{split}
    \end{equation}
    The operator $\partial, d, d^{'}$ and $*_{g}$ are defined in the same way as in Definition \ref{def:operators}, and 
    $$g(z):=\sum_{i=1}^{\infty}nc^{(n)}_{F}z^{n-1}.$$

    Here, L, $R_{1}$ and $R_{2}$ are all polynomials of $c^{v}_{F}$'s, whose corresponding variables are 
    given in the parenthesis, and the coefficient of each monomial is uniformly bounded in the limit regime. Moreover, each monomial in $R_{1}$ contains at least one $C^{v}_{F}$ where $l(v)\ge 2$.
    If we assign $c^{v}_{F}$ with degree $|v|$, each summand on the right of (\ref{eq_dunklaction}) is homogeneous of degree $k$.
\end{thm}
We postpone the proof of Theorem \ref{thm:technicalequivalence} to next section, and using its result, we are able to prove Theorem \ref{thm:hightemperaturemainthm}.

\noindent{\textbf{Proof of Theorem \ref{thm:hightemperaturemainthm}}.}
We first assume the sequence $\{\Vec{c}_{M}\}_{M=1}^{\infty}$ is q-$\gamma$-appropriate, with limiting q-$\gamma$-cumulants $\{k_{l}\}_{l=1}^{\infty}$. We need to show $\{\Vec{c}_{M}\}_{M=1}^{\infty}$ is satisfying LLN with moments $\{m_{2k}\}_{k=1}^{\infty}=\mathrm{T}_{k\rightarrow m}^{q,\gamma}(\{k_{l}\}_{l=1}^{\infty})$.

Denote the type-BC Bessel generating function of $\Vec{c}_{M}$ by $G_{M,N;\theta}(z_{1},...,z_{M})$. By Theorem \ref{thm:dunklonbgf}, the left side of (\ref{eq_rectangularlln}) before taking the limit is given by 
\begin{equation}\label{eq_symmetricdunklonbessel}
    M^{-s}\Bigg(\prod_{i=1}^{s}P_{2k_{i}}\Bigg)G_{M,N;\theta}(z_{1},...,z_{M})\Bigr|_{z_{1}=...=z_{M}=0}.
\end{equation}
For each $M=1,2,...$, without loss of generality assume $k_{1}\ge k_{2}\ge...\ge k_{s}$ and identify $(2k_{1},...,2k_{s})$ with a partition $\lambda$. Also since $G_{M,N;\theta}$ is analytic on a domain near 0 and $G_{M,N\theta}(0,...,0)=1$, there is a function $F_{M,N;\theta}(z_{1},...,z_{M})$ analytic near 0 and $\exp(F_{M,N;\theta}(z_{1},...,z_{M}))=G_{M,N;\theta}(z_{1},...,z_{M})$, $F_{M,N;\theta}(0,...,0)=0$. We write $F_{M,N;\theta}$ in terms of its $k^{th}$ order Talor polynomial
$$F_{M,N;\theta}(z_{1},...,z_{M})=\sum_{\mu:|\mu|\le k,\ l(\mu)\le M}c^{\lambda}_{F_{M,N;\theta}}\cdot m_{\mu}(\Vec{z})+O(||z^{k+1}||).$$

After the above identifications (\ref{eq_symmetricdunklonbessel}) satisfies the condition of Theorem \ref{thm:technicalequivalence}. Then we turn it into the expression on the right of (\ref{eq_dunklaction}), and take the limit $M\theta\rightarrow\infty$,$N\theta\rightarrow q\gamma$. By q-$\gamma$-appropriateness,
\begin{equation*}
    \lim_{M\theta\rightarrow\infty,\ N\theta\rightarrow q\gamma}c^{(n)}_{F_{M,N;\theta}}=\frac{k_{n}}{n},\quad \lim_{M\theta\rightarrow\infty,\ N\theta\rightarrow q\gamma}c^{(n)}_{F_{M,N;\theta}}=0,\; \text{if}\ l(\mu)>1.
\end{equation*}
Hence $\sum_{\mu:|\mu|=k,l(\mu)>l(\lambda)}b^{\lambda}_{\mu}\cdot c^{v}_{F_{M,N;\theta}}$ turns to 0, since each summand contains some term converging to 0, and 
$$b^{\lambda}_{\lambda}\cdot c^{\lambda}_{F_{M,N;\theta}}\longrightarrow \begin{dcases}
  0& \quad\quad \text{if}\ s>1\\
  (2k_{1}-2+2q\gamma)(2k_{1}-2+2\gamma)(2k_{1}-4-2q\gamma)(2k_{1}-4+2\gamma)\\
        ...(2+2q\gamma)(2+2\gamma)2q\gamma\cdot c^{(2k_{1})}_{F}k_{2k_{1}}&\quad\quad \text{if}\ s=1. 
\end{dcases}
$$

The polynomial L converges to 
\begin{equation*}
    \begin{split}
        &\prod_{i=1}^{s}\left([z^{0}](\partial+2\gamma d+((q-1)\gamma-\frac{1}{2})d^{'}+*_{g})^{2k_{i}-1}g(z)\right)\\
        -&(2k_{1}-2+2q\gamma)(2k_{1}-2+2\gamma)(2k_{1}-4-2q\gamma)(2k_{1}-4+2\gamma)
        ...(2+2q\gamma)(2+2\gamma)2q\gamma\cdot k_{2k_{1}}\cdot \mathrm{I}_{s=1},
    \end{split}
\end{equation*}
where $g(z)=\sum_{n=1}k_{n}z^{n-1}$, since $\sum_{n=1}^{\infty}nc^{(n)}_{F_{M,N;\theta}}z^{n-1}$ converges coefficient-wise to $g(z)$.

The polynomial $R_{1}$ converges to 0, also because each summand contains some factor $c^{v}_{F_{M,N;\theta}}$ with $l(v)>1$, that vanishes in the limit regime. The polynomial $M^{-1}R_{2}$ vanishes as well in the limit since all its coefficients converge to 0. Combining all the results above gives
$$\lim_{M\rightarrow \infty}\E{\prod_{i=1}^{s}p_{2k_{i}}^{M}}=\prod_{i=1}^{s}\left([z^{0}](\partial+2\gamma d+((q-1)\gamma-\frac{1}{2})d^{'}+*_{g})^{2k_{i}-1}g(z)\right),$$
which is equal to $\prod_{i=1}^{s}m_{2k_{i}}$ that $\{m_{2k}\}_{k=1}^{\infty}=\mathrm{T}_{k\rightarrow m}^{q,\gamma}(\{k_{l}\}_{l=1}^{\infty})$. Hence the LLN condition of $\{\Vec{c}_{M}\}_{M=1}^{\infty}$ is proved.

Now we go in the opposite direction, that assuming $\{\Vec{c}_{M}\}_{M=1}^{\infty}$ satisfies LLN for some $\{m_{k}\}_{k=1}^{\infty}$, i.e, for all even partition $\lambda$, 
$$M^{-l(\lambda)}\left[\prod_{i=1}^{l(\lambda)}P_{\lambda_{i}}\right]\exp\Big(F_{M,N;\theta}(z_1,...,z_{M})\Big)\Bigr|_{z_{1}=...=z_{M}=0}\xrightarrow[M\theta\rightarrow \infty]{N\theta \rightarrow q\gamma}\prod_{i=1}^{l(\lambda)}m_{\lambda_{i}},$$
where $F_{M,N;\theta}(z_{1},...,z_{M})=\sum_{\mu:l(\mu)\le M}c^{v}_{F_{M,N;\theta}}\cdot m_{\mu}(\Vec{z})$ is an analytic function near 0 satisfying $\exp(F_{M,N;\theta})=G_{M,N;\theta}$. We need to show:
\begin{equation}
    c^{\lambda}_{F_{M,N;\theta}}\longrightarrow \begin{dcases}
        0& \quad l(\lambda)>1\ \text{or}\ \lambda\ \text{is\ not\ even}\\
        \frac{k_{2k}}{2k}& \quad \lambda=(2k)
    \end{dcases}
\end{equation}
in the limit regime $M\theta\rightarrow \gamma, N\theta\rightarrow q\gamma$, where $\{k_{l}\}_{l=1}^{\infty}=\mathrm{T_{m\rightarrow k}^{q,\gamma}}(\{m_{2k}\}_{k=1}^{\infty}).$

Note that we only need to consider the case $|\lambda|$ is even, and $c^{v}_{F_{M,N;\theta}}=0$ for all $v$ not even, since the type BC Bessel function of each $\Vec{c}_{M}$ is a symmetric function in $z_{1}^{2},...,z_{M}^{2}$, by Definition \ref{def:bgf} and Proposition \ref{prop:polyexpectation}. We proceed by induction on $|\lambda|$. 

For $|\lambda|=0$ there's nothing to show. Suppose the result holds for all $|\lambda|\le 2k-2$, we now consider the partition that $|\lambda|=2k$. By Theorem \ref{thm:technicalequivalence}, for each $M,N,\theta$, we have a (finite) system of linear equations of $$\{c^{\mu}_{F_{M,N;\theta}}\}_{v:\ |v|=2k,\ l(v)>l(\lambda),\ v \text{\ is\ even}}.$$
\begin{equation}\label{eq_linearsystem}
    \begin{split}
       b^{\lambda}_{\lambda}\cdot c^{\lambda}_{F_{M,N;\theta}}+&\sum_{v:\ |v|=2k,\ l(v)>l(\lambda),\ \mu \text{\ is\ even}}b^{\lambda}_{v}\cdot c^{v}_{F_{M,N;\theta}}\\=& M^{-l(\lambda)}\left[\prod_{i=1}^{l(\lambda)}P_{\lambda_{i}}\right]\exp\Big(F_{M,N;\theta}(z_{1},...,z_{M})\Big)\Bigr|_{z_{1}=...z_{M}=0}
        -L(c_{F_{M,N;\theta}}^{(i)},1\le i\le 2k-1)\\
        &\quad\quad\quad-R_{1}(c^{v}_{F_{M,N;\theta}},|v|<2k)-M^{-1}R_{2}(c^{v}_{F_{M,N;\theta}},|v|\le 2k).
    \end{split}
\end{equation}
We observe that if we write it in the matrix form in the lexicographical order of $v$'s introduced in Section \ref{sec:sympoly}, the above system is upper triangular, and again by Theorem \ref{thm:technicalequivalence}, its diagonal entries $b^{\mu}_{\mu}$'s all converge to some nonzero constant in the limit regime, and the off-diagonal entries are uniformly bounded. Hence the matrix is invertible asymptotically, and its inverse has uniformly bounded entries.


\noindent{\textbf{Claim:}} If $\lambda\ne (2k)$, the right side of (\ref{eq_linearsystem}) converges to 0 in the limit regime.

\begin{proof}[\textbf{Proof of the claim:}] $R_{1}\rightarrow 0$ by induction hypothesis (recall that each of its term involves some partition $v$ with $l(v)\ge 2$), and $R_{2}\rightarrow 0$ since the coefficients all vanish in the limit.

By the LLN condition, 
$$M^{-l(\lambda)}\left[\prod_{i=1}^{l(\lambda)}P_{\lambda_{i}}\right]\exp(F(z_{1},...,z_{M})\Bigr|_{z_{1}=...z_{M}=0}\longrightarrow \prod_{i=1}^{l(\lambda)}m_{\lambda_{i}},$$
and by Theorem \ref{thm:technicalequivalence} and Definition \ref{def:operators}, when $\lambda\ne (2k)$, each $\lambda_{i}<2k$ and
$$L(c_{F_{M,N;\theta}}^{(i)},1\le i\le 2k-1)=\prod_{i=1}^{l(\lambda)}m^{M,N;\theta}_{\lambda_{i}},$$ 
where $\{m^{M,N;\theta}_{k}\}_{k=1}^{\infty}=\mathrm{T}_{k\rightarrow m}^{q,\gamma}(\{l\cdot c^{(l)}_{F_{M,N;\theta}}\}_{l=1}^{\infty}).$ 
By induction hypothesis $\{l\cdot c^{(l)}_{F_{M,N;\theta}}\}_{l=1}^{\infty}\longrightarrow\{k_{l}\}_{l=1}^{\infty}$ pointwisely for $l<2k$, and hence $m^{M,N;\theta}_{j}\rightarrow m_{j}$ pointwisely for $j<k$, and 
$$L(c_{F_{M,N;\theta}}^{(i)},1\le i\le 2k-1)\longrightarrow \prod_{i=1}^{l(\lambda)}m_{\lambda_{i}}\ $$
as well. \qedhere\end{proof}

Because of this claim, we conclude that when $M\theta\rightarrow \gamma, N\theta\rightarrow q\gamma$, the solutions of the linear system converge to the zero vector, in particular, 
\begin{equation}\label{eq_llnapproxexceptonerowYD}
 c^{\lambda}_{F_{M,N;\theta}}\longrightarrow 0\
\text{for\ all}\ |\lambda|=2k, \lambda\ne (2k).   
\end{equation}

It remains to consider $\lambda=(2k)$. This time we write down a single identity
\begin{equation}\label{eq_linearsystem2}
    \begin{split}
       b^{(2k)}_{(2k)}\cdot c^{(2k)}_{F_{M,N;\theta}}+&\sum_{v:\ |v|=2k,\ l(v)>1,\ \mu\ \text{is\ even}}b^{(2k)}_{v}\cdot c^{v}_{F_{M,N;\theta}}\\=& M^{-1}P_{2k}\left[\exp(F_{M,N;\theta}(z_{1},...,z_{M})\right]\Bigr|_{z_{1}=...z_{M}=0}
        -L(c_{F_{M,N;\theta}}^{(i)},1\le i\le 2k-1)\\
        &\quad\quad\quad-R_{1}(c^{v}_{F_{M,N;\theta}},|v|<2k)-M^{-1}R_{2}(c^{v}_{F_{M,N;\theta}},|v|\le 2k).
    \end{split}
\end{equation}
We have that
$$\mathrm{LHS}=b^{(2k)}_{(2k)}\cdot c^{(2k)}_{F_{M,N;\theta}}+o(1),$$ where $g_{M,N;\theta}(z)=\sum_{n=1}^{\infty}nc^{(n)}_{F_{M,N;\theta}}z^{n-1}$,
        because of Theorem \ref{thm:technicalequivalence} and (\ref{eq_llnapproxexceptonerowYD}).
        And
$$\mathrm{RHS}=m_{2k}-[z^{0}]\Big(\partial+2\gamma d+((q-1)\gamma-\frac{1}{2}) d^{'}+*_{g_{M,N;\theta}}\Big)^{2k-1}g_{M,N;\theta}(z)+b^{(2k)}_{(2k)}\cdot c^{(2k)}_{F_{M,N;\theta}}+o(1)$$
because of Theorem \ref{thm:technicalequivalence} and the 
 LLN assumption. Hence, when $M\theta\rightarrow \gamma, N\theta\rightarrow q\gamma$,
 $$[z^{0}]\Big(\partial+2\gamma d+((q-1)\gamma-\frac{1}{2}) d^{'}+*_{g_{M,N;\theta}}\Big)^{2k-1}g_{M,N;\theta}(z)\longrightarrow m_{2k}.$$
 By Definition \ref{def:operators}, the invertibility of $\mathrm{T}_{k\rightarrow m}^{q,\gamma}$ and the induction hypothesis, this is equivalent to 
 $$(2k)\cdot c^{(2k)}_{F_{M,N;\theta}}\longrightarrow k_{2k}$$ in the limit regime, that $k_{2k}$ is in the image of $\mathrm{T}_{m\rightarrow k}^{q,\gamma}(\{m_{2j}\}_{j=1}^{\infty})$. This finishes the induction step and therefore the proof. \qed
\subsection{Proof of Theorem  \ref{thm:technicalequivalence}}
We start by reducing $F(z_{1},...,z_{M})$ from a (locally) analytic function to its $2k^{th}$ Talor polynomial.
\begin{lemma}\label{lem:reducetopoly}
    For $F(z_{1},...,z_{M})$ of the form (\ref{eq_talorpolynomial}), denote $F{'}(z_{1},...,z_{M})=\sum_{\lambda:|\lambda|\le 2k,\ l(\lambda)\le M}c^{\lambda}_{F}\cdot m_{\lambda}(\Vec{z})$. Then for a partition $\lambda$ with $|\lambda|=2k$, we have 
    $$\Bigg[\prod_{i=1}^{l(\lambda)}P_{\lambda_{i}}\Bigg]\exp(F^{'}(z_{1},...,z_{M}))\Bigr|_{z_{1}=...=z_{M}=0}=\Bigg[\prod_{i=1}^{l(\lambda)}P_{\lambda_{i}}\Bigg]\exp(F(z_{1},...,z_{M}))\Bigr|_{z_{1},...,z_{M}=0}.$$
\end{lemma}
\begin{proof}
    Since $F$ is analytic near 0, write $\exp(F(z_{1},...,z_{M}))$ and $\exp(F^{'}(z_{1},...,z_{M}))$ as symmetric power series. Their difference $R(\Vec{z})$ is a power series of order $O(||z||^{k+1})$. Since $\prod_{i=1}^{l(\lambda)}P_{\lambda_{i}}$ is a homogeneous polynomial of $D_{i}$'s and each $D_{i}$ reduces the total power of a monomial by 1, 
 $$\Bigg[\prod_{i=1}^{l(\lambda)}P_{\lambda_{i}}\Bigg]R(\Vec{z})=0.\qedhere$$
\end{proof}
By Lemma \ref{lem:reducetopoly}, in the remaining of this section we take 
$$F(z_{1},...,z_{M})=\sum_{\lambda:|\lambda|\le 2k,\ l(\lambda)\le M, \lambda\ \text{even}}c^{\lambda}_{F}\cdot m_{\lambda}(\Vec{z}).$$

$\prod_{i=1}^{l(\lambda)}P_{\lambda_{i}}$ is a sum of products of $D_{i}$'s ($i=1,2,...,M$). For each product of the form $D_{1}^{n_{1}}...D_{M}^{n_{M}}$ acting on $\exp(F(z_{1},...,z_{M}))$, by (\ref{prop:commutativity}) the order does not matter.  
Recall
$$D_{i}=\partial_{i}+\left[\theta(N-M+1)-\frac{1}{2}\right]\frac{1-\sigma_{i}}{z_{i}}+\theta\sum_{j\ne i}\left[\frac{1-\sigma_{ij}}{z_{i}-z_{j}}+\frac{1-\tau_{ij}}{z_{i}+z_{j}}\right].$$
Observe that $D_{i}[\exp(F(z_{1},...,z_{M}))]$ is of the form $H(z_{1},...,z_{M})\exp(F(z_{1},...,z_{M}))$, where $H(z_{1},...,z_{M})$ is a polynomial of $z_{1},...,z_{M}$, and for any i, $D_{i}[H(z_{1},...,z_{M})\exp(F(z_{1},...,z_{M}))]$ is still $\exp(F(z_{1},...,z_{M}))$ multiplied by a polynomial. More precisely, 
\begin{equation}\label{eq_hz1}
\begin{split}
    \partial_{i}[&H(z_{1},...,z_{M})\exp(F(z_{1},...,z_{M}))]\\
    &=\Big(\partial_{i}H(z_{1},...,z_{M})+H(z_{1},...,z_{M})\partial_{i} F(z_{1},...,z_{M})\Big)\cdot \exp(F(z_{1},...,z_{M})),
\end{split}
\end{equation}
\begin{equation}
\begin{split}
    \frac{1-\sigma_{i}}{z_{i}}[&H(z_{1},...,z_{M})\exp(F(z_{1},...,z_{M}))]\\
    &=\Big(\frac{1-\sigma_{i}}{z_{i}}H(z_{1},...,z_{M})\Big)\cdot \exp(F(z_{1},...,z_{M})),
\end{split}
\end{equation}
\begin{equation}
\begin{split}
    \frac{1-\sigma_{ij}}{z_{i}-z_{j}}[&H(z_{1},...,z_{M})\exp(F(z_{1},...,z_{M}))]\\
    &=\Big(\frac{1-\sigma_{ij}}{z_{i}-z_{j}}H(z_{1},...,z_{M})\Big)\cdot \exp(F(z_{1},...,z_{M})),
\end{split}
\end{equation}
\begin{equation}\label{eq_hz4}
\begin{split}
    \frac{1-\tau_{ij}}{z_{i}+z_{j}}[&H(z_{1},...,z_{M})\exp(F(z_{1},...,z_{M}))]\\
    &=\Big(\frac{1-\tau_{ij}}{z_{i}+z_{j}}H(z_{1},...,z_{M})\Big)\cdot \exp(F(z_{1},...,z_{M})).
\end{split}
\end{equation}
 We see that $\prod_{i=1}^{l(\lambda)}[D_{1}^{n_{1}}...D_{M}^{n_{M}}]\exp(F(z_{1},...,z_{M}))\Bigr|_{z_{1},...,z_{M}=0}$ is obtained by acting a polynomial of $\partial_{i}, \frac{1-\sigma_{i}}{z_{i}},\frac{1-\sigma_{ij}}{z_{i}-z_{j}},\frac{1-\tau_{ij}}{z_{i}+z_{j}}$ on $F(z_{1},...,z_{M})$, then take the constant term. Then we have the following basic observation.
\begin{prop}\label{prop:unifombounded}
    For any M-tuples of nonnegative integers $n_{1},...,n_{M}$, 
    \begin{equation}
        \Big[D_{1}^{n_{1}}\cdots D_{M}^{n_{M}}\Big]\exp(F(z_{1},...,z_{M}))\Bigr|_{z_{1},...,z_{M}=0}
    \end{equation}
    is a homogeneous polynomial in $c^{v}_{F}$'s of degree $\sum_{i=1}^{M}n_{i}$, if taking $c^{v}_{F}$ to be of degree $|v|$. Moreover, the coefficients of this polynomial are all uniformly bounded in the limit regime $M\theta\rightarrow\gamma$, $N\theta\rightarrow q\gamma$.
\end{prop}
\begin{proof}
    Each of $\partial_{i}, \frac{1-\sigma_{i}}{z_{i}},\frac{1-\sigma_{ij}}{z_{i}-z_{j}},\frac{1-\tau_{ij}}{z_{i}+z_{j}}$ reduces the degree of a monomial by 1, the constant term of 
$\Big[D_{1}^{n_{1}}...D_{M}^{n_{M}}\Big]\exp(F(z_{1},...,z_{M}))$ is then obtained from some monomials of $z_{1},...,z_{M}$ of degree $\sum_{i=1}^{M}n_{i}$. Since $c^{v}_{F}$ is the coefficient of $m_{v}(\Vec{z})$ which is of degree $|v|$, by assigning $c^{v}_{F}$ with degree $|v|$ one can pass the degree of the original monomials to their resulting constant terms. 

Each $D_i$ is a sum of 2M single operators $\partial_{i}, \frac{1-\sigma_{i}}{z_{i}},\frac{1-\sigma_{ij}}{z_{i}-z_{j}}$ and $\frac{1-\tau_{ij}}{z_{i}+z_{j}}$, in which $2M-2$ terms involves a factor $\theta$, and hence $D_{1}^{n_{1}}...D_{M}^{n_{M}}$ is a sum of $2M^{\sum_{i=1}^{M}n_{i}}$ products of single operators. The constant term of each of these products acting on $\exp(F(z_{1},...,z_{M}))$ is changing with $M,N,\theta$ as a muliple of $\theta^{\sum_{i=1}^{M}n_{i}-\#\partial_{i}\text{'s\ in\ the\ product}}$, and the number of such products is of order $O(M^{\sum_{i=1}^{M}n_{i}-\#\partial_{i}\text{'s\ in\ the\ product}})$. Hence as $M\theta\rightarrow\gamma$ the coefficient is uniformly bounded.
\end{proof}
\begin{prop}\label{prop:dunklptodunkld}
    For a partition $\lambda$, we have 
    \begin{equation}
    \begin{split}
        M^{-l(\lambda)}\left[\prod_{i=1}^{l(\lambda)}P_{\lambda_{i}}\right]&\exp(F(z_{1},...,z_{M}))\Bigr|_{z_{1},...,z_{M}=0}\\
        =&\left[\prod_{i=1}^{l(\lambda)}(D_{i})^{\lambda_{i}}\right]\exp(F(z_{1},...,z_{M}))\Bigr|_{z_{1},...,z_{M}=0}+O(\frac{1}{M}),
    \end{split}
    \end{equation}
    in the limit regime $M\theta\rightarrow\gamma, N\theta\rightarrow q\gamma$, where $O(\frac{1}{M})$ is a homogeneous polynomial of $c^{v}_{F}$'s (taking $c^{v}_{F}$ to be of degree $|v|$) whose coefficients are of order $O(\frac{1}{M})$.
\end{prop}
\begin{proof}
    Each $P_{\lambda_{i}}$ is a sum of M terms $D_{j}^{\lambda_{i}}(j=1,2,...,M)$, hence $\prod_{i=1}^{l(\lambda)}P_{\lambda_{i}}$ is a sum of $M^{l(\lambda)}$ such terms, in which  $O(M^{l(\lambda)-1})$ terms have not all distinct indices $j$'s. By Proposition \ref{prop:unifombounded} each of these terms has uniformly bounded coefficient, hence they together contribute $O(\frac{1}{M})$. As for the remaining terms with all distinct indices, by symmetry of $F(z_{1},...,z_{M})$, their action are all the same as $\prod_{i=1}^{l(\lambda)}(D_{i})^{\lambda_{i}}$.
\end{proof}
After all the reductions above, it remains to study $$\left[\prod_{i=1}^{l(\lambda)}(D_{i})^{\lambda_{i}}\right]\exp(F(z_{1},...,z_{M}))\Bigr|_{z_{1},...,z_{M}=0}$$for an arbitrary partition $\lambda$, whose expression should match the right side of (\ref{eq_dunklaction}).
The expression on the right side of (\ref{eq_dunklaction}) can be splitted into three parts: the linear polynomials of $c^{v}_{F}$'s, the terms involving only $c^{v}_{F}$'s where $v$ are length 1 partitions, and all the other remaining terms. In the next two Propositions, we deal with the first two cases separately. Before that we present several lemmas that will be used in the proof. Consider the action of $\prod_{i=1}^{l(\lambda)}D_{i}^{\lambda_{i}}$ on $m_{\mu}(\Vec{z})$.  Each $D_{i}$ is a combination of $\partial_{i}+[\theta(N-M+1)-\frac{1}{2}]\frac{1-\sigma_{i}}{z_{i}}$ and $\theta\left[\frac{1-\sigma_{ij}}{z_{i}-z_{j}}+\frac{1-\tau_{ij}}{z_{i}+z_{j}}\right]$ with $M-1$ choices of $j\ne i$, hence $\prod_{i=1}^{l(\lambda)}D_{i}^{\lambda_{i}}$ will lead to a big sum, whose summands are products of these two terms. 

 \begin{lemma}\label{lem:generic}
        For arbitrary partitions $\lambda$ and $\mu$, the constant term of $\prod_{i=1}^{l(\lambda)}D_{i}^{\lambda_{i}}m_{\mu}(\Vec{z})$ has a generic part, which is contributed by the summands of $\prod_{i=1}^{l(\lambda)}D_{i}^{\lambda_{i}}$ in which all the indices $j$ are distinct, and all bigger than $l(\lambda)$. The remaining part is of order $O(\frac{1}{M})$ in the limit regime $M\theta\rightarrow \gamma, N\theta\rightarrow q\gamma$.  
    \end{lemma}
    \begin{proof}
        For $k=0,1,...,|\lambda|$, the number of summands in the remaining part (which means their exists a pair of indices that coincides) with $k$ components of $\theta\left[\frac{1-\sigma_{ij}}{z_{i}-z_{j}}+\frac{1-\tau_{ij}}{z_{i}+z_{j}}\right]$ is of order $O(M^{k-1})$, and the power of $\theta$ in these summands is $k$. Since $M\theta\rightarrow \gamma>0$, the remaining part is a finite sum of order $O(M^{k-1}\theta^{k})$, which is $O(\frac{1}{M})$.
    \end{proof} 
    
    Because of this, we only consider the limit of the generic part of the expression. For simplicity we write $l=l(\lambda)$.
    
    \begin{lemma}\label{lem:generic2}
         The generic part of constant term of $\prod_{i=1}^{l}D_{i}^{\lambda_{i}}m_{\mu}(\Vec{z})$ is given by 
        \begin{equation}\label{eq_symmetricdunklaction}
            \Big[D_{l}^{\lambda_{l}-1}\partial_{l}\Big]\cdots\Big[D_{2}^{\lambda_{2}-1}\partial_{2}\Big]\cdot\Big[D_{1}^{\lambda_{1}-1}\partial_{1}\Big]m_{\mu}(\Vec{z}).
        \end{equation}
    \end{lemma} 
    \begin{proof}
    For $m=1$, because of the symmetry, $m_{\mu}(\Vec{z})$ is invariant under the action of $\sigma_{i}$, $\sigma_{ij}$ and $\tau_{ij}$ and hence $D_{i}m_{\mu}(\Vec{z})$ is equal to $\partial_{i}m_{\mu}(\Vec{z})$.  
    
    For $m=2,3,...,l$, after acting $\Big[D_{m-1}^{\lambda_{m-1}-1}\partial_{m}\Big]\cdots\Big[D_{1}^{\lambda_{1}-1}\partial_{1}\Big]$ on $m_{\mu}(\Vec{z})$, since the generic part has distinct indices j's bigger than $l(\lambda)$, we get 
    some polynomial $H(z_1,...,z_{M})$,
    where the operators act on variables $z_{1},...,z_{m-1}$ and $z_{j}$'s ($j>l$). Hence $H(z_1,...,z_{M})$ is still symmetric as function of $z_{m}^{2}$ and $z_{j^{'}}^{2}$ for another different $j^{'}$ in the first $D_{m}$, invariant again under the action of $\sigma_{i}$, $\sigma_{ij}$ and $\tau_{ij}$. We conclude that 
    $$D_{m}\Big[D_{m-1}^{\lambda_{m-1}-1}\partial_{m-1}\Big]\cdots\Big[D_{1}^{\lambda_{1}-1}\partial_{1}\Big]m_{\mu}(\Vec{z})=\partial_{m}\Big[D_{m-1}^{\lambda_{m-1}-1}\partial_{m-1}\Big]\cdots\Big[D_{1}^{\lambda_{1}-1}\partial_{1}\Big]m_{\mu}(\Vec{z}).$$
    \end{proof}
    \begin{remark}
        One can replace $m_{\mu}(\Vec{z})$ by $F(z_{1},...,z_{M})$ or $\exp(F(z_{1},...,z_{M}))$ in last lemma, since these functions satisfy the same symmetry.
    \end{remark}

    The next lemma considers the concrete action of $D_{i}$ on a polynomial of $z_{1},...,z_{l}$.
    \begin{lemma}\label{lem:dunklonmonomial}
        For an arbitrary $l$-tuple $(n_{1},...,n_{l})\in \Z_{\ge 0}^{l}$ and arbitrary $i=1,2,...,l$,
    we have that for the generic part of $D_{i}$,
    \begin{equation}\label{eq_singledunklaction1}
        D_{i}[z_{1}^{n_{1}}...z_{l}^{n_{l}}]=\Big(\partial_{i}+\left[\theta(N-M+1)-\frac{1}{2}\right]d_{i}^{'}+2\theta(M-1)d_{i}\Big)[z_{1}^{n_{1}}\cdots z_{l}^{n_{l}}]+\sum_{j\ne i}(z_{j}p_{1}^{j}+z_{j}p_{2}^{j}),
    \end{equation}
    where $p_{1}^{j},p_{2}^{j}$ are some polynomials of $z_{1},...,z_{l}$ depending on $(n_{1},...,n_{l})$, and
    $d_{i}$, $d^{'}_{i}$ are  linear operators on polynomials of $z_{1},...,z_{M}$ such that
\begin{equation}
    \begin{split}
    d_{i}(z_{i}^{n})=
        \begin{dcases}
    0 & \;\; n=0;\\
    2z_{i}^{n-1}&\;\; n>0,
        \end{dcases}\quad\quad
        d^{'}_{i}(z_{i}^{n})=
        \begin{dcases}
    0 & \;\; \text{n\ is\ even;}\\
    2z_{i}^{n-1}&\;\; \text{n\ is\ odd.}
        \end{dcases}
    \end{split}
\end{equation}  
Note that the the action depends whether the power of $z_{i}$ is odd or even.
    \end{lemma}
    \begin{proof}
        This follows directly from definition. More precisely, for $j>l$, 
        $$\theta\frac{1-\sigma_{ij}}{z_{i}-z_{j}}[z_{1}^{n_{1}}\cdots z_{l}^{n_{l}}]=d_{i}[z_{1}^{n_{1}}\cdots z_{l}^{n_{l}}]+z_{j}p_{1}^{j},$$
        $$\theta\frac{1-\tau_{ij}}{z_{i}+z_{j}}[z_{1}^{n_{1}}\cdots z_{l}^{n_{l}}]=d_{i}[z_{1}^{n_{1}}\cdots z_{l}^{n_{l}}]+z_{j}p_{2}^{j},$$
        and 
        $$\theta\frac{1-\sigma_{i}}{z_{i}}[z_{1}^{n_{1}}\cdots z_{l}^{n_{l}}]=d_{i}^{'}[z_{1}^{n_{1}}\cdots z_{l}^{n_{l}}].$$
    \end{proof}
\begin{prop}\label{prop:dunkllinear}
    For even partition $\lambda$ with $|\lambda|=2k$, we have 
    \begin{equation*}
        \left[\prod_{i=1}^{l(\lambda)}(D_{i})^{\lambda_{i}}\right]\exp(F(z_{1},...,z_{M}))\Bigr|_{z_{1},...,z_{M}=0}=b^{\lambda}_{\lambda}\cdot c^{\lambda}_{F}+\sum_{\mu:|\mu|=2k,\ l(\mu)>l(\lambda)}b^{\lambda}_{\mu}\cdot c^{\mu}_{F}+R+O(\frac{1}{M}).
    \end{equation*}
    In particular, 
    $$\lim_{M\theta\rightarrow \gamma,N\theta\rightarrow q\gamma}b^{\lambda}_{\lambda}=\prod_{i=1}^{l(\lambda)}\Big[\lambda_{i}(\lambda_{i}-2+2q\gamma)(\lambda_{i}-2+2q\gamma)(\lambda_{i}-4-2q\gamma)(\lambda_{i}-4+2\gamma)\cdots
        (2+2q\gamma)(2+2\gamma)2q\gamma\Big].$$
    The summand $R$ is a polynomial of $c^{v}_{F}$ that $|v|<2k$. And $O(\frac{1}{M})$ denotes a linear polynomial of $c^{v}_{F}$ such that $|v|=2k$, and the coefficients are of order $O(\frac{1}{M})$ in the limit regime of this section. 
\end{prop}
\begin{proof}
    Since the expression on the right is homogeneous of degree 2k, all the nonlinear terms are collected as $R$, and it suffices to consider the linear terms, which is 
    $$\sum_{\mu:|\mu|=2k,\ \mu\ \text{is\ even}}b^{\lambda}_{\mu}\cdot c^{\mu}_{F}.$$
    We classify all the even partition $\mu$ with $|\mu|=2k$ in terms of their length. When $l(\mu)>l(\lambda)$, there's nothing to show.
    When $l(\mu)\le l(\lambda)$, we want to show when $\mu\ne \lambda$, $b^{\lambda}_{\mu}$ is of order $O(\frac{1}{M})$. 
    
    Writing $\exp(F(z_{1},...,z_{M}))$ as power series of $F(z_{1},...,z_{m})$. Since each term in $D_{i}$ reduces the total power of a monomial by 1, and $F(z_{1},...,z_{M})=\sum_{\mu:|\mu|\le 2k}c^{\mu}_{F}\cdot m_{\mu}(\Vec{z})$ where each $m_{\mu}(\Vec{z})$ is homogeneous of degree $|\mu|$,  we see that $b^{\lambda}_{\mu}$ is obtained from the action of $\prod_{i=1}^{l(\lambda)}D_{i}$ on the single symmetric monomial $m_{\mu}(\Vec{z})$.
    
    Again let $l=l(\lambda)$. By Lemma \ref{lem:generic} and \ref{lem:generic2}, we first consider the generic part of $\prod_{i=1}^{l(\lambda)}D_{i}m_{\mu}(\Vec{z})$. When $l(\mu)<l$, each monomial of $m_{\mu}(\Vec{z})$ is missing some variable among $z_{1},...,z_{M}$, say $z_{m}$. Then when acting the $\partial_{m}$ in (\ref{eq_symmetricdunklaction}), we get 0 since $\Big[D_{m-1}^{\lambda_{m-1}-1}\partial_{m-1}\Big]...\Big[D_{1}^{\lambda_{1}-1}\partial_{1}\Big]$ does not produce any power of $z_{m}$ to $m_{\mu}(\Vec{z})$. And the remaining part of the action gives $O(\frac{1}{M})$.

     What remains is to consider the case $|\mu|=|\lambda|$, and we calculate the limit of $b^{\lambda}_{\lambda}$. 
    Again $b^{\lambda}_{\lambda}$ is obtained from the action of $\prod_{i=1}^{l(\lambda)}D_{i}$ on  $m_{\lambda}(\Vec{z})$. By Lemma \ref{lem:generic}, \ref{lem:generic2}, we consider only the generic part of the Dunkl product, and act it on the monomials of $m_{\mu}(\Vec{z})$ separately. The monomials with variables other than $z_{1},...,z_{l}$ are missing some variables, say $z_{m}\ (1\le m\le l)$. Then (\ref{eq_symmetricdunklaction}) again tells that these monomials only contribute $O(\frac{1}{M})$.

    Now we consider monomials formed by $z_{1},...,z_{l}$. For an arbitrary $l$-tuple $(n_{1},...,n_{l})$ and arbitrary $i=1,2,...,l$, by Lemma \ref{lem:dunklonmonomial}
    we have 
    \begin{equation}\label{eq_singledunklaction}
        D_{i}[z_{1}^{n_{1}}...z_{l}^{n_{l}}]=\Big(\partial_{i}+[\theta(N-M+1)-\frac{1}{2}][1-(-1)^{n_{i}}]d_{i}+\theta(M-1)d_{i}+\theta(M-1)d_{i}\Big)[z_{1}^{n_{1}}\cdots z_{l}^{n_{l}}]+\sum_{j\ne i}(z_{j}p_{1}^{j}+z_{j}p_{2}^{j}),
    \end{equation}
   
     One can see from the above expression that, after a single action of $D_{i}$, $z_{1}^{n_{1}}...,z_{l}^{n_{l}}$ splits into two parts. The $z_{i}$-power of the first part decreases by 1. The second part has a common factor $z_{j}$, and its $z_{i}$-powers decreases as well while the powers of other variables are unchanged. For the action of $\prod_{i=1}^{l}D_{i}^{\lambda_{i}}$, we repeat the above action by another $|\lambda|-1$ times. Since all indices $j$ are distinct, the second part has no chance to become a constant. Hence we only apply the first part each time we apply one more single $D_{i}$, and $\prod_{i=1}^{l}D_{i}^{\lambda_{i}}$ results in reducing power of $z_{i}$ by $\lambda_{i}$. In the monomials of $m_{\mu}(\Vec{z})'s$ where $|\mu|=|\lambda|$, only $z_{1}^{\lambda_{1}},...,z_{l}^{\lambda_{l}}$ survives as a nonzero constant. More precisely (we use $\approx$ to omit the $O(\frac{1}{M})$ part),
    \begin{align*}
       b^{\lambda}_{\lambda}\approx&
[z^{0}]\prod_{i=1}^{l}D_{i}^{\lambda_{i}}m_{\lambda}(\Vec{z})
\approx[z^{0}]\prod_{i=1}^{l}D_{i}^{\lambda_{i}}z_{1}^{\lambda_{1}}....z_{M}^{\lambda_{M}}\\
       \approx&[z^{0}]\Big[\partial_{l}+\Big(2(M-1)\theta+2(N-M+1)\theta-1\Big)d_{l}\Big]^{\frac{\lambda_{l}}{2}}\Big[\partial_{l}+2(M-1)\theta d_{l}\Big]^{\frac{\lambda_{l}}{2}-1}\partial_{l}\\
       &\cdots\Big[\partial_{1}+\Big(2(M-1)\theta+2(N-M+1)\theta-1\Big)d_{1}\Big]^{\frac{\lambda_{1}}{2}}\Big[\partial_{1}+2(M-1)\theta d_{1}\Big]^{\frac{\lambda_{1}}{2}-1}\partial_{1}\Big[z_{1}^{\lambda_{1}}\cdots z_{l}^{\lambda_{l}}\Big]\\
        =&\Big[\partial_{l}+(2N\theta-1)d_{l}\Big]^{\frac{\lambda_{l}}{2}}\Big[\partial_{l}+2(M-1)\theta d_{l}\Big]^{\frac{\lambda_{l}}{2}-1}\partial_{l}\\
        &\cdots\Big[\partial_{1}+(2N\theta-1)d_{1}\Big]^{\frac{\lambda_{1}}{2}}\Big[\partial_{1}+2(M-1)\theta d_{1}\Big]^{\frac{\lambda_{1}}{2}-1}\partial_{1}\Big[z_{1}^{\lambda_{1}}\cdots z_{l}^{\lambda_{l}}\Big]\\
\xrightarrow[M\theta\rightarrow\gamma]{N\theta\rightarrow q\gamma}&\Big[\partial_{l}+(2q\gamma-1)d_{l}\Big]^{\frac{\lambda_{l}}{2}}\Big[\partial_{l}+2\gamma d_{l}\Big]^{\frac{\lambda_{l}}{2}-1}\partial_{l}\\
&\cdots\Big[\partial_{1}+(2q\gamma-1)d_{1}\Big]^{\frac{\lambda_{1}}{2}}]\Big[\partial_{1}+2\gamma d_{1}\Big]^{\frac{\lambda_{1}}{2}-1}\partial_{1}\Big[z_{1}^{\lambda_{1}}\cdots z_{l}^{\lambda_{l}}\Big]\\
=&\prod_{i=1}^{l}(\lambda_{i}-1+2q\gamma-1)(\lambda_{i}-2+2\gamma)(\lambda_{i}-3+2q\gamma-1)\cdots(2+2\gamma)(1+2q\gamma-1)\\
       =&\prod_{i=1}^{l}\lambda_{i}(\lambda_{i}-2+2q\gamma)(\lambda_{i}-2+2\gamma)(\lambda_{i}-4+2q\gamma)(\lambda_{i}-4+2\gamma)\cdots(2+2q\gamma)(2+2\gamma)2q\gamma.
\end{align*}
The above argument also implies for $|\mu|=|\lambda|$,
$\prod_{i=1}^{l(\lambda)}D_{i}^{\lambda_{i}}m_{\mu}(\Vec{z})$ is $O(\frac{1}{M})$, and so is $b^{\lambda}_{\mu}$.
\end{proof}
The next proposition deals with the terms involving only length 1 partitions, and identify them with $L$ in (\ref{eq_dunklaction}).
\begin{prop}\label{prop:dunklactionnonlinear}
    For even partition $\lambda$ with $|\lambda|=2k$, we have
    \begin{equation}\label{eq_dunklactionnonlinear}
    \begin{split}
        \left[\prod_{i=1}^{l(\lambda)}(D_{i})^{\lambda_{i}}\right]&\exp(F(z_{1},...,z_{M}))\Bigr|_{z_{1}=...z_{M}=0}\\
        &=\prod_{i=1}^{l(\lambda)}\left([z^{0}](\partial+2\gamma d+((q-1)\gamma-\frac{1}{2})d^{'}+*_{g})^{\lambda_{i}-1}g(z)\right)+R+O(\frac{1}{M}),
    \end{split}
    \end{equation}
    where $g(z)=\sum_{n=1}^{\infty}nc^{(n)}_{F}z^{n-1}$, and $\partial$,$\ d$,$\ d^{'}$ and $*_{g}$ are defined in Definition \ref{def:operators}. Moreover, $R$ is a homogeneous polynomial of $c^{v}_{F}$'s that $|v|\le 2k$, and each monomial contains at least one $c^{v}_{F}$ that $l(v)>1$, and $O(\frac{1}{M})$ is a homogeneous polynomial of $c^{v}_{F}$'s whose coefficients are of order $O(\frac{1}{M})$ in the limit regime $M\theta\rightarrow\gamma, N\theta\rightarrow q\gamma$.
    
\end{prop}
\begin{proof}
 Again by Lemma \ref{lem:generic}, \ref{lem:generic2}, we only take the generic part of the action of Dunkl operators, namely, all indices $j$'s involved are distinct and bigger than $l(\lambda)$, and the remaining part becomes $O(\frac{1}{M})$ in (\ref{eq_dunklactionnonlinear}).
 
Moreover, we only consider the polynomials involving only $c^{(n)}_{F}$'s ($n=2, 4,...,2k)$, which are corresponding to $m_{(n)}(\Vec{z})$, and all other terms are collected in $R$ and $O(\frac{1}{M})$.
Hence, we only look at the action 
    \begin{equation}\label{eq_dunklactionnonlinear2}
    \begin{split}
        \left[\prod_{i=1}^{l(\lambda)}(D_{i})^{\lambda_{i}}\right]&\exp(\sum_{n=2}^{2k}c^{(n)}_{F}m_{(n)}(\Vec{z}))\Bigr|_{z_{1},...,z_{M}=0}\\
        &=\left[\prod_{i=1}^{l(\lambda)}(D_{i})^{\lambda_{i}}\right]\prod_{t=1}^{M}\exp(\sum_{n=2}^{2k}c^{(n)}_{F}(z_{t}^{n}))\Bigr|_{z_{1},...,z_{M}=0}.
    \end{split}
    \end{equation}

\noindent{\textbf{Claim:}}
\begin{equation}\label{eq_claim}
\begin{split}
\left[\prod_{i=1}^{l(\lambda)}(D_{i})^{\lambda_{i}}\right]\prod_{t=1}^{M}&\exp(\sum_{n=2}^{2k}c^{(n)}_{F}(z_{j}^{n}))\Bigr|_{z_{1}=...z_{M}=0}\\
&=\prod_{i=1}^{l(\lambda)}\Bigg((D_{i})^{\lambda_{i}-1}\partial_{i}\Big[ \prod_{t=1}^{M}\exp(\sum_{n=2}^{2k}c^{(n)}_{F}(z_{t}^{n})\Big]\Bigr|_{z_{i}=0}\Bigg).
\end{split}
\end{equation}
\begin{proof}[\textbf{Proof of the Claim:}]
Since the indices $j$'s of $D_{i}$'s are distinct and bigger than $l(\lambda)$,  for $i_{1}\ne i_{2}$, $D_{i_{1}}^{\lambda_{i_{1}}}$ and $D_{i_{2}}^{\lambda_{i_{2}}}$ are acting on two groups of disjoint variables. Hence 
the action of each $(D_{i})^{\lambda_{i}}$ factors. Moreover, the first $D_i$ acts as $\partial_{i}$ for the same reason as in the proof of Lemma \ref{lem:generic2}.\qedhere\end{proof}
    

Without loss of generality consider $i=1$. 
$$\partial_{1}\Big[\prod_{t=1}^{M}\exp\Big(\sum_{n=2}^{2k}c^{(n)}_{F}(z_{t}^{n})\Big)\Big]=g(z_{1})\prod_{t=1}^{M}\exp\Big(\sum_{n=2}^{2k}c^{(n)}_{F}(z_{t}^{n})\Big).$$
By (\ref{eq_hz1})-(\ref{eq_hz4}), it suffices to consider the explicit action of $D_{i}$ on $H(z_{1})\prod_{t=1}^{M}\exp\left(\sum_{n=2}^{2k}c^{(n)}_{F}(z_{t}^{n})\right)$, where $H(z_1)$ is a polynomial of $z_{1}$, and 
$$D_{1}\Big[H(z_{1})\prod_{t=1}^{M}\exp\left(\sum_{n=2}^{2k}c^{(n)}_{F}(z_{t}^{n})\right)\Big]=\Big[D_{1}H(z_{1})+g(z_{1})\Big]\prod_{t=1}^{M}\exp\left(\sum_{n=2}^{2k}c^{(n)}_{F}(z_{t}^{n})\right).$$
Hence we have for $i=1,2,...,l(\lambda)$,
$$D_{i}^{\lambda_{i}-1}\partial_{i}\left[\prod_{t=1}^{M}\Big(\exp(\sum_{n=2}^{2k}c^{(n)}_{F}(z_{t}^{n})\Big)\right]\Bigr|_{z_{1}=...=z_{M}=0}=(D_{i}+*_{g})^{\lambda_{i}-1}g(z_{i})\Bigr|_{z_{i}=0}.$$



Again by Lemma \ref{lem:generic}, \ref{lem:generic2}, \ref{lem:dunklonmonomial}, up to $O(\frac{1}{M})$ error,
\begin{equation}
\begin{split}
(D_{i}+*_{g})^{\lambda_{i}-1}g(z_{i})\Bigr|_{z_{i}=0}\approx &\Bigg(\partial_{i}+2(M-1)\theta d_{i}+\Big[\theta(N-M+1)-\frac{1}{2}\Big]d_{i}^{'}\Bigg)^{\lambda_{i}-1}g(z_{i})\Bigr|_{z_{i}=0}\\
\xrightarrow[M\theta\rightarrow\gamma]{N\theta\rightarrow q\gamma}&\Bigg(\partial_{i}+2\gamma d_{i}+\Big[(q-1)\gamma-\frac{1}{2}\Big]d_{i}^{'}+*_{g}\Bigg)^{\lambda_{i}-1}g(z_{i})\Bigr|_{z_{i}=0},
\end{split}
\end{equation}
and plugging this back to (\ref{eq_claim}) gives
\begin{equation}
\begin{split}
    \left[\prod_{i=1}^{l(\lambda)}(D_{i})^{\lambda_{i}}\right]\prod_{t=1}^{M}&\exp\Big(\sum_{n=1}^{2k}c^{(n)}_{F}(z_{j}^{n})\Big)\Bigr|_{z_{1}=...z_{M}=0}\\
=&\prod_{i=1}^{l(\lambda)}\Bigg([z^{0}]\Big(\partial+2\gamma d+\Big[(q-1)\gamma-\frac{1}{2}\Big]d^{'}+*_{g}\Big)^{\lambda_{i}-1}g(z)\Bigg)+O(\frac{1}{M}).
\end{split}
\end{equation}
Proposition \ref{prop:dunklactionnonlinear} then follows.
\end{proof}
Combining all the results above in this section, we arrive at the expansion (\ref{eq_dunklaction}) representing action of Dunkl operators on $\exp(F(z_{1},...,z_{M}))$.

\noindent{\textbf{Proof\ of\ Theorem\ \ref{thm:technicalequivalence}:}}
By Proposition \ref{prop:unifombounded} and \ref{prop:dunklptodunkld} the left side of (\ref{eq_dunklaction}) is a homogeneous polynomial of $c^{v}_{F}$'s of degree $2k$ with uniformly bounded coefficients in the limit regime. The right side of (\ref{eq_dunklaction}) is a combination of Proposition \ref{prop:dunkllinear} (which gives $b^{\lambda}_{\lambda}\cdot c^{\lambda}_{F}+\sum_{\mu:|\mu|=k,\ l(\mu)>l(\lambda)}b^{\lambda}_{\mu}\cdot c^{v}_{F}$) and (\ref{prop:dunklactionnonlinear}) (which gives polynomial L), and note that the only possible overlap of linear terms and terms involving only length 1 partitions is when $\lambda$ itself is $(2k)$, which gives the term subtracted in (\ref{eq_L}).\qed
\subsection{q-$\gamma$ convolution.}
After stating the equivalence in Theorem \ref{thm:hightemperaturemainthm}, Theorem \ref{thm:hightemplln} follows as a direct consequence.

\noindent{\textbf{Proof of Theorem \ref{thm:hightemplln}.}}
For each $M\le N, \theta>0$, let $G_{M,N,\theta}^{a}$, $G_{M,N,\theta}^{b}$, $G_{M,N,\theta}^{c}$ denote the type BC Bessel generating function of $\Vec{a}_{M}, \Vec{b}_{M}$ and $\Vec{c}_{M}=\Vec{a}_{M}\boxplus_{M,N}^{\theta}\Vec{b}_{M}$. Then 
$$G_{M,N,\theta}^{c}(z_{1},...,z_{M})=G_{M,N,\theta}^{a}(z_{1},...,z_{M})\cdot G_{M,N,\theta}^{b}(z_{1},...,z_{M}),$$
and hence partial derivatives of $\ln(G_{M,N,\theta}^{c})$ are equal to the sum of the ones of $\ln(G_{M,N,\theta}^{b})$ and $\ln(G_{M,N,\theta}^{a})$. By assumption of the theorem, $\{\Vec{a}_{M}\}$ and $\{\Vec{b}_{M}\}$ satisfy LLN condition, then by Theorem \ref{thm:hightemperaturemainthm} they are q-$\gamma$-LLN appropriate. Hence by Definition \ref{def:llnappropriateness} $\{\Vec{c}_{M}\}$ is also q-$\gamma$-LLN appropriate. By Theorem \ref{thm:hightemperaturemainthm} again $\{\Vec{c}_{M}\}$ satisfies LLN. \quad \qed

\section{$q$-$\gamma$ cumulants and moments}\label{sec:momentcumulant}
Fix $q\ge 1, \gamma>0$, in this section we continue with the limit regime $M,N\rightarrow\infty$, $\theta\rightarrow 0$, $M\theta\rightarrow \gamma$, $N\theta\rightarrow q\gamma$ . Definition \ref{def:operators} introduces a map $\mathrm{T}_{k\rightarrow m}^{q,\gamma}$ in terms of operators, that sends the real sequence $\{k_{l}\}_{l=1}^{\infty}$ to another real sequence $\{m_{k}\}_{k=1}^{\infty}$. We keep the interpretation from Theorem \ref{thm:hightemperaturemainthm}, that is, we call $\{k_{l}\}_{l=1}^{\infty}$ the q-$\gamma$ cumulants and take $k_{l}=0$ for all odd l's. In this section we give a more combinatorical descriptions of $T_{k\rightarrow m}^{q,\gamma}$. After that, we also provide an explicit relation of $T_{m\rightarrow k}^{q,\gamma}$ in terms of generating functions, and by taking $q,\gamma$ to some extreme values, we set the connections of our $q$-$\gamma$-cumulants to the usual cumulants and (rectangular) free cumulants in free probability theory, and also to the $\gamma$-cumulants defined in \cite{BCG}, that arises in the high temperature regime of self-adjoint matrix additions.
\subsection{From q-$\gamma$ cumulants to moments}\label{sec:cumulanttomoment}
We start by introducing some basic notions of set partitions, which are necessary for the statement of the main theorem.

For $k\in \Z_{\ge 1}$, a \emph{set partition} $\pi$ of $[k]$ is a way to write  $[k]:=\{1,2,...,k\}$ as disjoint union of sets $B_{1},...,B_{n}$ for some $m$. We write $\pi=B_{1}\sqcup B_{2}\sqcup...\sqcup B_{m}$, and denote the space of all set partitions of $[k]$ by $P(k)$. Given a set partition $\pi$, for each $B_{i}$ let $\min(B_{i})$ and $\max(B_{i})$ denote the minimal and maximal number in the subset $B_{i}$ of $[k]$, and for simplicity, we label $B_{1},...,B_{n}$ by $\min(B_{i})$ in increasing order.

In this text, we are in particular interested in the non-crossing partitions. 
\begin{definition}\label{def:noncrossingpartition}
    Fix $k\in \Z_{\ge 1}$, a set partition $\pi=B_{1}\sqcup...\sqcup B_{n}$ of $[k]$ is non-crossing if for any $l=2,...,m$, and any $j=1,2,...,l-1$, the elements in $B_{j}$ are either bigger than $\max(B_{l})$ or smaller than $\min(B_{l})$. See Figure \ref{Figure-partition}. Denote the set of all non-crossing partition of $[k]$ by $NC(k)$.

    
\end{definition}

Each set partition can be realized visually as a collection of blocks $B_{1},...,B_{n}$ with $k$ legs in total, and the block $B_{i}$ has $|B_{i}|$ legs, which is the number of elements in $B_{i}$. See Figure \ref{Figure-partition}. From this point of view, $\pi$ is non-crossing if and only if the legs of one block does not cross any other blocks. 
\begin{figure}[htpb]
    \begin{center}
        
        \includegraphics[width=0.75\linewidth]{noncrossing.pdf}
        \caption{\small{The graph on the left represents a noncrossing partition $\pi$ of $[6]$, where $B_{1}=\{1,4,6\}$, $B_{2}=\{2,3\}$, $B_{3}=\{5\}$, and the graph on the right represents a crossing partition $\pi^{'}$ of $[6]$, where $B_{1}=\{1,3,6\}$, $B_{2}=\{2,4\}$, $B_{3}=\{5\}$.}}\label{Figure-partition}
    \end{center}
\end{figure}

Next we define a quantity associated with the non-crossing set partition $\pi$. 
\begin{definition}\label{def:weight}
    Given $\pi=B_{1}\sqcup...\sqcup B_{m}\in NC(k)$, for $i=1,2,...,n$, let $P_{i}=\#$ of elements in $B_{1},...,B_{i}$ bigger than $\min(B_{i})$, and $Q_{i}=\#$ of elements in $B_1,...,B_{i}$ bigger than $\max(B_{i})$(note that $P_{i}-Q_{i}=|B_{i}|-1$ by definition). Let $C_{1},C_{2},...$ be the countable sequence of constants $$2q\gamma,\ 2\gamma+2,\ 2q\gamma+2,\ 2\gamma+4,\ 2q\gamma+4,\ 2\gamma+6,\ 2q\gamma+6\ ...$$ respectively. Then we define
    \begin{equation}\label{eq_w(pi)}
        W(\pi)=\prod_{i=1}^{m}\Big[C_{Q_{i}+1}C_{Q_{i}+2}\cdots C_{P_{i}}\Big].
    \end{equation}
\end{definition}
\begin{ex}\label{ex:partition}
In Figure \ref{Figure-partition_example}, $\pi$ is a non-crossing partition of $[14]$, such that $B_{1}=\{1,7,8\}$, $B_{2}=\{2,3,6\}$, $B_{3}=\{4,5\}$, $B_{4}=\{9,10,13,14\}$, and $B_{5}=\{11,12\}$. Moreover, $P_{1}=2$, $Q_{1}=0$, $P_{2}=4$, $Q_{2}=2$, $P_{3}=4$, $Q_{3}=3$, $P_{4}=3$, $Q_{4}=0$, $P_{5}=1$, $Q_{5}=0$, so $W(\pi)=C_{1}\cdot C_{3}C_{4}\cdot C_{4}\cdot C_{1}C_{2}C_{3}\cdot C_{1}=C_{1}^{3}C_{2}C_{3}^{2}C_{4}^{2}=(2q\gamma)^{3}(2\gamma+2)(2q\gamma+2)^{2}(2\gamma+4)^{2}$.
\end{ex}
\begin{figure}[htpb]
    \begin{center}
        
        \includegraphics{partition_example.pdf}
        \caption{\small{The graphical representation of the non -crossing partition in Example \ref{ex:partition}.}\label{Figure-partition_example}}
    \end{center}
\end{figure}
We also introduce a notion of \emph{even partition} that will be used later.
\begin{definition}\label{def:evenpartition}
    We say $\pi$ is even if $|B_{1}|,...,|B_{n}|$ are all even, and denote the collection of all non-crossing even set partition of $[2k]$ by $\mathfrak{NC}(2k)$, for some $k\in \Z_{\ge 1}$.
\end{definition}
The following main theorem of this section gives the combinatorial expression of moments as polynomials of q-$\gamma$ cumulants, whose coefficients are given by $W(\pi)$.
\begin{thm}{(q-$\gamma$ cumulants to moments formula)}\label{thm:cumulanttomomentcomb}
    Let $\{k_{l}\}_{l=1}^{\infty}$, $\{m_{k}\}_{k=1}^{\infty}$ be two real sequences such that $k_{l}=0$ for all odd $l$'s, and $\{m_{2k}\}_{k=1}^{\infty}=\mathrm{T}_{k\rightarrow m}^{q,\gamma}(\{k_{l}\}_{l=1}^{\infty})$. Then for any $k=1,2,...$,
    \begin{equation}\label{eq_cumulanttomomentcomb}
        m_{2k-1}=0,\quad m_{2k}=\sum_{\pi\in \mathfrak{NC}(2k)}W(\pi)\prod_{B\in \pi}k_{|B_{i}|}.
    \end{equation}
    
\end{thm}

\begin{ex}
    By manipulating (\ref{eq_cumulanttomomentcomb}), we have the explicit expression of the first few q-$\gamma$ cumulants in terms of moments:
    \begin{equation}\label{eq_cumulanttomomentex}
        \begin{split}
            m_{2}=&2q\gamma k_{2},\\
            m_{4}=&2q\gamma(2\gamma+2)(2q\gamma+2)k_{4}+[(2q\gamma)^{2}+2q\gamma(2\gamma+2)]k_{2}^{2},\\
            m_{6}=&\Big[2q\gamma(2\gamma+2)(2q\gamma+2)(2\gamma+4)(2q\gamma+4)\Big]k_{6}\\
            +&\Big[2q\gamma(2\gamma+2)(2q\gamma+2)(3\times2q\gamma+2\gamma+2+2q\gamma+2+2\gamma+4+2q\gamma+4)\Big]k_{4}k_{2}\\
            +&\Big[(2q\gamma)^{3}+2(2q\gamma)^{2}(2\gamma+2)+2q\gamma(2\gamma+2)(2q\gamma+2)\Big]k_{2}^{3}.
        \end{split}
    \end{equation}
\end{ex}
\begin{proof}[Proof of Theorem \ref{thm:cumulanttomomentcomb}]
    Recall from Definition \ref{def:operators} that 
    $$m_{2k}=[z^{0}]\left(\partial+2\gamma d+\left[(q-1)\gamma-\frac{1}{2}\right]d^{'}+*_{g}\right)^{2k-1}g(z)$$
    where $g(z)=\sum_{l=1}^{\infty}k_{l}z^{l-1}$, i.e, we act the operator $D:=\partial+2\gamma d+((q-1)\gamma-\frac{1}{2})d^{'}+*_{a}$ on $g(z)$ by $2k-1$ times, then take the constant term of the resulting expression. Here $a(z)=\sum_{l=1}^{\infty}a_{l}z^{l-1}$ such that $a_{l}=k_{l}$, and the resulting polynomial $D^{p-1}g(z)$, before the acting $D$ by the $p^{th}$ time, contains only odd powers of $z$ when $p$ is odd, and contains only even power of $z$ when $p$ is even.

    Because of this, we claim that $D^{2k-1}$ acting on $g(z)$ is equivalent to the alternate product of two operators 
    \begin{equation}\label{eq_operatorproduct}
       D^{'}\circ(D^{''}\circ D^{'})^{k-1}. 
    \end{equation}
    More precisely, $D^{''}\circ D^{'}$ has the following explicit effect:
    $$...\longrightarrow \sum_{l=1}^{\infty}b_{l}z^{l-1}\xrightarrow{D^{''}}\sum_{l=1}^{\infty}c_{l}z^{l-1}\xrightarrow{D^{'}}\sum_{l=1}^{\infty}d_{l}z^{l-1}\longrightarrow...,$$
    where $D^{'}$ acts on the polynomial $b(z)=\sum_{l=1}b_{l}z^{l-1}$ which contains only odd powers, with output $c(z)=\sum_{l=1}b_{l}z^{l-1}$ such that
    \begin{equation}\label{eq_d"}
        c_{l}=\begin{dcases}
            0& \;\; \text{l\ is\ even};\\
            (2q\gamma+l-1)b_{l+1}+\sum_{j=1}^{l}a_{j}b_{l+1-j}&\;\; \text{l\ is\ odd},
        \end{dcases}
    \end{equation}
    and then $D^{''}$ acts on $c(z)$ which contains only odd powers, with output $d(z)=\sum_{l=1}^{\infty}d_{l}z^{l-1}$ such that 
    \begin{equation}\label{eq_d""}
        d_{l}=\begin{dcases}
            0& \;\; \text{l\ is\ odd};\\
            (2\gamma+l)c_{l+1}+\sum_{j=1}^{l}a_{j}c_{l+1-j}&\;\; \text{l\ is\ even}.
        \end{dcases}
    \end{equation}
    We verify the expression (\ref{eq_cumulanttomomentcomb}) inductively on $k$, by visualizing the action of $D^{'}$ or $D{'}\circ D^{''}$ under the graphical representation of set partitions.

    When $k=1$, 
    \begin{equation}\label{eq_singleDaction}
    \begin{split}
        m_{2}=&[z^{0}]\Big(D^{'}(g(z)\Big)\\
        =&(2q\gamma)k_{2}+\sum_{j=1}^{1}a_{1}k_{1+1-1}=(2q\gamma)k_{2}+a_{1}k_{1}
    \end{split}
    \end{equation} by (\ref{eq_d"}). 
    
    Correspondingly, we realize this expression by the following concrete operations:
    
    (0). Start with an empty set partition, corresponding to a blank graph with no leg.
    
    (1). Draw a single leg, and label it with monomial $k_{1}$.
    
    (2). Add one more leg on the right of the first leg. Now we have two options: connect this leg with the first leg to form a block of size 2, and label the new block by $C_{1}=2q\gamma k_{2}$ or keep them to be separate as two blocks with single leg, and label the second block by $a_{1}$. 

    After step 2 the first configuration corresponds to monomial $k_{2}$ with weight $2q\gamma$, and the second one corresponds to monomial $k_{1}\cdot a_{1}$ with  weight 1 (which is the product of the labels of the two blocks). Adding them together gives $m_{2}$. To match (\ref{eq_cumulanttomomentcomb}), note that $W(\{1,2\})=C_{1}=2q\gamma$, and $k_{1}=a_{1}=0$ so the second term vanishes.

    To obtain $m_{2k}$, suppose $m_{2},...,m_{2k-2}$ obtained from step 0, 1, 2,..., $2k-3$, $2k-2$ all match (\ref{eq_cumulanttomomentcomb}). We give the operations in step $2k-1$, $2k$:
    
    ($2k-1$). Given a configuration $\pi=B_{1}\sqcup...\sqcup B_{n}$ of $2k-2$ legs in total, with a corresponding weight $\Tilde{W}(\pi)$, insert one more leg right after the first leg to $B_{1}$, so that we get a new configuration $\pi^{'}=B_{1}^{'}\sqcup...\sqcup B_{n}^{'}$, where $B_{1}^{'}=\{1\}\cup \{j+1:j\in B_{1}\}$, and $B_{i}^{'}=\{j+1:j\in B_{i}\}$.

    Then one have $|B_{1}^{'}|$ options: either keep everything unchanged and let $B^{''}_{i}=B^{'}_{i}$ for $i=1,2,...,n$, $$\Tilde{W}(\pi^{''})=\begin{dcases}
        \Tilde{W}(\pi)\cdot (2\gamma+|B^{''}_{1}|-1)&\;\; \text{if\ }|B^{''}_{1}|\ \text{is\ odd};\\
        \Tilde{W}(\pi)\cdot (2q\gamma+|B^{''}_{1}|-2)&\;\; \text{if\ }|B^{''}_{1}|\ \text{is\ even},
    \end{dcases}$$ 
    or split $B_{1}^{'}$ into two non-crossing blocks $B_{1}^{''}\sqcup B_{2}^{''}$, where $\min(B_{1}^{''})=1$, $\min(B_{2}^{''})=2$, and let $B_{i}^{''}=B_{i-1}^{'}$ for $i=3,4,...,n+1$, $\Tilde{W}(\pi^{''})=\Tilde{W}(\pi)$. 

    ($2k$). Repeat the operations in step $2k-1$ one more time, and still denote the output configuration with 2k legs by $\pi^{''}=B_{1}^{''}\sqcup...\sqcup B_{n^{''}}^{''}$($n^{''}$ denotes the number of blocks) with weight $W^{''}(\pi^{''})$.

    Then delete all configurations with at least one odd block. For each remaining configuration $\pi^{''}$, assign it with monomial 
    \begin{equation}\label{eq_inductionmonomial}
        W^{''}(\pi^{''}) k_{|B_{1}^{''}|}\cdot a_{|B_{2}^{''}|} ...\ a_{|B_{n^{''}}^{''}|}.
    \end{equation}
    
    
    We claim that the above steps represent the action of (\ref{eq_operatorproduct}). Indeed, both step 0-2 and $D^{'}$ generate $(2q\gamma)k_{2}+a_{1}k_{1}$, and for $l\ge 3$, step $l$ is corresponding to the $(l-1)^{th}$ term (from left to right) in the product. More precisely, the expression of $d_{l}\ (c_{l}$ resp.) in (\ref{eq_d""}) ((\ref{eq_d"}) resp.) is recording the $l+1$ options one can choose on a configuration whose first block is of size $l$, that choosing to enlarge the first block by 1 gives one extra factor $(2\gamma+l)\ (2q\gamma+l-1$ resp.), and splitting $d_{l}\ ((c_{l}$ resp.) into $a_{j}$ and $c_{l+1-j}$ corresponds to splitting the first block into two new blocks of size $l+1-j$ and $j$ respectively. 
Therefore, acting (\ref{eq_operatorproduct}) on $g(z)$ and take the constant term is equivalent to a chain of compositions of (\ref{eq_d"}) and (\ref{eq_d""}). Compared to (\ref{eq_inductionmonomial}), the output is also a large sum of monomials of $k_{l}$'s (coefficients of $g(z)$) and $a_{l}$'s
(coefficients of $a(z)$), that each non-vanishing monomial is corresponding to a unique non-crossing even partition $\pi$ (recall $k_{l}=a_{l}=0$ for odd l's), and the unique $k_{l}$ it has is giving the size of the first block of $\pi$. To see that each non-crossing even set partition can be realized in this way, we do induction on $k$ and assume this holds for the set partitions of size up to $2k-2$. For $\pi=B_{1}\sqcup...\sqcup B_{n}\in \mathfrak{NC}(2k)$, just combine $B_{1}$ with $B_{2}$ and $ B_{3}$(both might be $\varnothing$) as a single block, then remove two legs from this new block. What we get is an element $\Tilde{\pi}$ in $\mathfrak{NC}(2k-2)$, which can be realized by induction hypothesis, and one can construct $\Bar{\pi}$ from $\Tilde{\pi}$ using the step $2k-1$ and $2k$ above.  

Since $a_{l}=k_{l}$ for all $l=1,2,...$, to match $m_{k}$ with the right side of (\ref{eq_cumulanttomomentcomb}), it remains to match the coefficient of each monomials.   
    Given $\pi\in \mathfrak{NC}(2k-1)\ (\mathfrak{NC}(2k-1)$ resp.), define its degeneration $\Tilde{\pi}=\Tilde{B}_{1}\sqcup...\sqcup \Tilde{B}_{n}\in \mathfrak{NC}(2k-2)\ (\mathfrak{NC}(2k-2)$ resp.) by taking $\Tilde{B}_{1}$ as the combination of the first two blocks removing 1 leg, or simply removing 1 leg from $B_{1}$, similarly as in last paragraph. Then compared to $W(\Tilde{\pi})$, we only replace $C_{|\Tilde{Q}_{1}|+1}...C_{|\Tilde{P}|_{1}}$ by $C_{|Q_{2}|+1}...C_{|P_{2}|}\cdot C_{|Q_{1}|+1}...C_{|P_{1}|}$ in $W(\pi)$ when we choose to split the first block, and one can check that
    $W(\pi)=W(\Tilde{\pi})$ after the change, and when we choose to enlarge the first block by 1, $W(\pi)$ has one more factor $C_{|P_{1}|}=C_{|\Tilde{B}_{1}}|$ than $W(\Tilde{\pi})$. 
    
    On the other hand, for $l=1,2,...,2k$, as pointed out in step $2k-1$ and $2k$, since $a_{l}=k_{l}=0$ for $l$ odd, in order to get a non-vanishing term after $2k$ steps, one can only enlarge $|\Tilde{B}_{1}|$ by 1 in even steps, when $|\Tilde{B}_{1}|$ is odd, then multiply the new factor $2q\gamma+|\Tilde{B}_{1}|-1$ to the monomial, and enlarge $|\Tilde{B}_{1}|$ by 1 in odd steps, when $|\Tilde{B}_{1}|$ is even, then multiply the new factor $2\gamma+|\Tilde{B}_{1}|$. In both cases this factor matches $C_{|\Tilde{B}_{1}}|$. This finishes the proof.
    
    
\end{proof}
\subsection{From moments to q-$\gamma$-cumulants}\label{sec:momenttocumulant}
Recall that $T_{k\rightarrow m}^{q,\gamma}$ is invertible, and for each $l=1,2,...$, $k_{2l}$ is a polynomial of $m_{2}, m_{4},...,m_{2l}$ with leading term as a multiple of $m_{2l}$. For example, by reversing (\ref{eq_cumulanttomomentex}), we have
\begin{equation}
        \begin{split}
            k_{2}=&\frac{1}{2q\gamma} m_{2},\\
            k_{4}=&\frac{1}{2q\gamma(2\gamma+2)(2q\gamma+2)}\Big[m_{4}-(1+\frac{\gamma+1}{q\gamma})m_{2}^{2}\Big],\\
            k_{6}=&\frac{1}{2q\gamma(2\gamma+2)(2q\gamma+2)(2\gamma+4)(2q\gamma+4)}\\
            \cdot&\Bigg(m_{6}-\Big[(3\times 2q\gamma+2\gamma+2+2q\gamma+2+2\gamma+4+2q\gamma+4)\cdot \frac{1}{2q\gamma}\Big]
            \Big[m_{4}-(1+\frac{\gamma+1}{q\gamma})m_{2}^{2}\Big]m_{2}\\
            &-\Big[1+\frac{\gamma+1}{q\gamma}+\frac{(\gamma+1)(q\gamma+1)}{(q\gamma)^{2}}\Big]m_{2}^{3}\Bigg).
        \end{split}
    \end{equation}

For the more general cases, we express the generating function of q-$\gamma$ cumulants by the generating function of moments.

\begin{thm}\label{thm:momenttocumulant}
    Let $\{m_{2k}\}_{k=1}^{\infty}$, $\{k_{l}\}_{l=1}^{\infty}$ be two real sequences such that $\{k_{l}\}_{l=1}^{\infty}=\mathrm{T}_{m\rightarrow k}^{q,\gamma}(\{m_{2k}\}_{k=1}^{\infty})$. Then $k_{l}=0$ for all odd l's, and
    \begin{equation}\label{eq_momenttocumulant}
        \begin{dcases}
            \exp\Big[\gamma\sum_{k=1}^{\infty}\frac{m_{2k}}{k}y^{2k}\Big]=\sum_{n=0}^{\infty}
c_{n}\cdot y^{2n},\\
\exp\Big[\sum_{l=1}^{\infty}\frac{k_{2l}}{2l}y^{2l}\Big]=\sum_{n=0}^{\infty}\frac{c_{n}}{(q\gamma)_{n}(\gamma)_{n}}2^{-2n}y^{2n}
\end{dcases}
    \end{equation}
    for some auxiliary sequence $\{c_{n}\}_{n=0}^{\infty}$. Here we use the Pochhammmer symbol notation 
    $$(x)_{n}:=\begin{dcases}
        x(x+1)\cdot(x+n-1),\;&\text{if}\ n\in \Z_{\ge 1},\\
        1,\;&\text{if}\ n=0.
    \end{dcases}$$

    Alternatively one has the more compact expression
    \begin{equation}\label{eq_momenttocumulantcmpt}
        \exp\Big[\sum_{l=1}^{\infty}\frac{k_{2l}}{2l}y^{2l}\Big]=[z^{0}]\Bigg\{\sum_{n=0}^{\infty}\frac{(yz)^{2n}}{(q\gamma)_{n}(\gamma)_{n}}2^{-2n}\cdot \exp\Big[\gamma\sum_{k=1}^{\infty}\frac{m_{2k}}{k}z^{-2k}\Big]\Bigg\}.
    \end{equation}
\end{thm}

Before giving the proof, we first present two technical results that will be used.

\begin{lemma}\label{lem:forthmmomenttocumulant} 

   (a).\ The following Talor series expansion holds:
    \begin{equation}\label{eq_onerowjackgeneratingfunction}
        \sum_{k=0}^{\infty}Q_{(k)}(a_{1}^{2},...,a_{M}^{2};\theta)y^{2k}=\prod_{i=1}^{M}(1-a_{i}^{2}y^{2})^{-\theta}.
    \end{equation}
    
   (b).\ For $\theta>0$, $y\in \C$ and $\Vec{a}=(a_{1}\ge ...\ge a_{M}\ge 0)$,
    \begin{equation}
        \B(\Vec{a},y,0^{M-1};\theta)=\sum_{k=0}^{\infty}\frac{1}{(N\theta)_{k}(M\theta)_{k}}2^{-2k}Q_{(k)}(a_{1}^{2},...,a_{M}^{2};\theta)y^{2k},
    \end{equation}
    where $Q_{(k)}(a_{i,M}^{2};\theta)$ is defined in Definition \ref{def:jack2}, and $(k)$ denotes the partition $(k,0,...,0)\in \Lambda_{M}$. Moreover, the power series converges uniformly in a domain near 0.

   
\end{lemma}
\begin{proof}
   (a) is a well known result that can be found in \cite[p378 and p380]{M}. (b) follows from Proposition \ref{prop:bessel} and (\ref{eq_jack2}), after specifying all but one variables to 0. 
\end{proof}


\begin{proof}[Proof of Theorem \ref{thm:momenttocumulant}]
    First we note that (\ref{eq_momenttocumulant}) and (\ref{eq_momenttocumulantcmpt}) are equivalent by comparing the coefficients for $y^{2n}$ for each $n=0,1,2,...$, and we will prove (\ref{eq_momenttocumulant}).

    For now, we assume that there exists a probability measure $\mu$ supported on $[a,b]\subset \R_{\ge 0}$, such that for $k=1,2,...$
    $$m_{2k}=\int x^{k} d\mu.$$

    We take a sequence of deterministic M-tuples $\{\Vec{a}_{M}\}_{M=1}^{\infty}$ such that $\Vec{a}_{M}=(a_{1,M},...,a_{M,M})\in [-\sqrt{b},\sqrt{b}]^{M}$, and define $\mu_{M}=\frac{1}{M}\sum_{i=1}^{M}\delta_{a_{i,M}^{2}}$. We choose $\{\Vec{a}_{M}\}$ in a way that $\mu_{M}\rightarrow \mu$ weakly as $M\rightarrow \infty$. This implies that the moments of $\mu_{M}$ also converge pointwisely to the corresponding moments of $\mu$, i.e,
    $$\frac{1}{M}\sum_{i=1}^{M}a_{i,M}^{2}\longrightarrow m_{2k}.$$
    In other words, $\{\Vec{a}_{M}\}_{M=1}^{\infty}$ satisfies the LLN condition, and by Theorem \ref{thm:hightemperaturemainthm} $\{\Vec{a}_{M}\}_{M=1}^{\infty}$ is q-$\gamma$-LLN-appropriate.

  By Lemma \ref{lem:forthmmomenttocumulant} (a),  
    \begin{equation}\label{eq_ck1}
    \begin{split}
        &\sum_{k=0}^{\infty}Q_{(k)}(a_{i,M}^{2};\theta)y^{2k}=\prod_{i=1}^{M}(1-a_{i,M}^{2}y^{2})^{-\theta}\\
        =&\exp\Bigg[-\theta\sum_{i=1}^{M}\ln\Big(1-a_{i,M}^{2}y^{2}\Big)\Bigg]=\exp\Bigg[\theta M\sum_{k=1}^{\infty}\frac{y^{2k}}{k}\frac{1}{M}\sum_{i=1}^{M}(a_{i,M})^{2k}\Bigg]
    \end{split}
    \end{equation}
 as a formal power series. Taking $M\rightarrow\infty, \theta\rightarrow 0, M\theta\rightarrow \gamma$, the above equality becomes 
 \begin{equation}\label{eq_ck2}
     \sum_{k=0}^{\infty}c_{k}\cdot y^{2k}=\exp\Big[\gamma\sum_{k=1}^{\infty}\frac{m_{k}}{k}y^{2k}\Big],
 \end{equation}
 where $c_{k}$ is the pointwise limit of $Q_{(k)}(a_{i,M}^{2};\theta)$ in the above limit regime. This defines $\{c_{k}\}_{k=1}^{\infty}$ in terms of $\{m_{2k}\}_{k=1}^{\infty}$.
 
   On the other hand, since $\mu_{M}$ is deterministic, its type BC Bessel generating function is equal to its Bessel function. And we have for $l=1,2,...$
   \begin{equation}
       (\frac{\partial}{\partial y})^{2l} \ln[ \B(\Vec{a}_{M},y,0^{M-1};\theta)]\Bigr|_{y=0}\xrightarrow[M\theta\rightarrow \gamma]{N\theta\rightarrow q\gamma}(2l-1)!\cdot k_{2l}.
   \end{equation}

By Lemma \ref{lem:forthmmomenttocumulant} (b), the above equation is equivalent to 
\begin{equation}\label{eq_derivativeofqgeneratingfunction}
    (\frac{\partial}{\partial y})^{2l} \ln \Big[\sum_{k=0}^{\infty}\frac{1}{(N\theta)_{k}(M\theta)_{k}}2^{-2k}Q_{(k)}(a_{i,M}^{2};\theta)y^{2k}\Big]\Bigr|_{y=0}\xrightarrow[M\theta\rightarrow \gamma]{N\theta\rightarrow q\gamma}(2l-1)!\cdot k_{2l}.
\end{equation}
   Also, since type BC Bessel function is analytic over $z_{1},...,z_{M}$, so is its logarithm near 0, and by Talor expanding $\ln \Big[\sum_{k=0}^{\infty}\frac{1}{(N\theta)_{k}(M\theta)_{k}}2^{-2k}Q_{k}(a_{i,M}^{2};\theta)y^{2k}\Big]$ we see that each $k_{2l}$ is a polynomial of finitely many terms $\frac{1}{(N\theta)_{k}(M\theta)_{k}}2^{-2k}Q_{k}(a_{i,M}^{2};\theta)$, each of which converges to $\frac{1}{(q\gamma)_{k}(\gamma)_{k}}2^{-2k}c_{k}$. 

   We claim that as $M\theta\rightarrow \gamma, N\theta\rightarrow q\gamma$, $\sum_{k=0}^{\infty}\frac{1}{(N\theta)_{j}(M\theta)_{j}}2^{-2k}Q_{(k)}(a_{i,M}^{2};\theta)y^{2k}$ converges uniformly on a domain near 0. Indeed, the pointwise convergence of coefficient of $y$ is already given above, and to obtain a tail bound of the power series, first note that we have assumed $a_{i,M}$'s are uniformly bounded, then by writing each $Q_{(k)}(a_{1,M}^{2},...,a_{M,M}^{2};\theta)$ as a contour integral of the right side of (\ref{eq_onerowjackgeneratingfunction}) on the circle $\{z:|z|=r\}$ for some $r$ small enough, we see that $Q_{(k)}(a_{1,M}^{2},...,a_{M,M}^{2};\theta)$ are uniformly bounded by $C\cdot r^{-2k}$ for some constant $C$ and $r$.  
   By (\ref{eq_ck1}), (\ref{eq_ck2}), the limit is $$\sum_{k=0}^{\infty}\frac{c_{k}}{(q\gamma)_{k}(\gamma)_{k}}y^{2k},$$ but since the uniform convergence of analytic functions implies convergence of derivatives, it's also equal to $\exp\Big[\sum_{l=1}^{\infty}\frac{k_{2l}}{2l}y^{2l}\Big]$ by (\ref{eq_derivativeofqgeneratingfunction}). Hence these two functions are equal.

   It remains to generalize to the case where $m_{2}, m_{4},...$ are arbitrary real sequence. For each $l=1,2,...$, $k_{2l}$ is a polynomial $h_{l}(m_{2},m_{4},...,m_{2l})$ of degree at most $l$, where the expression of $h_{l}$ is given by (\ref{eq_momenttocumulantcmpt}), while on the other hand, for each $k_{2l}$, (\ref{eq_cumulanttomomentcomb}) gives another polynomial of degree at most $l$, such that $k_{2l}=h^{'}_{l}(m_{2},m_{4},...,m_{2l})$. What we need to show is that $h_{l}=h_{l}^{'}$ for all $l$'s. 

   Fix $l\ge 1$. We have already shown that $h_{l}(m_{2},m_{4},...,m_{2l})=h_{l}^{'}(m_{2},m_{4},...,m_{2l})$, when $m_{2},m_{4},...,m_{2l}$ are the first $l$ moments of some compactly supported probability measure $\mu$ on $\R_{\ge 0}$. Clearly there exists more than $l$ such choices of $m_{2},m_{4},...,m_{2l}$, and therefore by fundamental theorem of algebra these two polynomials coincide. The same argument holds for arbitrary $l\ge 1$.
    \end{proof}

\subsection{Connections to self-adjoint additions}
Let $A$,$B$ be two independent $N\times N$ matrices, uniformly chosen from the sets of self-adjoint matrices with deterministic eigenvalues $a_{1}\ge...\ge a_{N}$ and $b_{1}\ge...\ge b_{N}$ respectively. The study of eigenvalues of $C=A+B$ dates back to \cite{Vo}, which considers the empirical measure of $C$ in fixed temperature regime. In high temperature regime, it was proved in \cite{BCG} that when $N\rightarrow\infty$, $\theta\rightarrow0$ and $N\theta\rightarrow\gamma$, assuming the empirical measure of $A$, $B$ converge to some deterministic probability measure $\mu_{A}$, $\mu_{B}$ on $\R$, then the empirical measure of $C$ converges to some deterministic probability measure $\mu_{C}$, which is named as the $\gamma$-convolution of $\mu_{A}$ and $\mu_{B}$.

There is a collection of quantities $\{k_{l}^{\gamma}\}_{l=1}^{\infty}$ introduced in \cite{BCG}, such that 
$$k^{\gamma}_{l}(\mu_{C})=k^{\gamma}_{l}(\mu_{A})+k^{\gamma}_{l}(\mu_{B})$$
for each $l\ge 1$. We write $\{m_{k}^{'}\}_{k=1}^{\infty}=\mathrm{T}_{k\rightarrow m}^{\gamma}(\{k^{\gamma}_{l}\}_{l=1}^{\infty})$, where $m^{'}_{k}\in \R$ denotes the $k^{th}$ moment of the limiting empirical measure, and $\mathrm{T}_{k\rightarrow m}^{\gamma}$ is a map that gives moment-cumulant relation of $\gamma$-convolution. While in this text we are considering addition of a different type of matrices, we find a limit transition in high temperature regime from rectangular addition to self-adjoint addition, which is stated in terms of cumulants.

\begin{thm}\label{thm:gammacumulant}
    Given a real  sequence  $\{k_{l}\}_{l=1}^{\infty}$ such that $k_{l}=0$ for all odd $l$'s, let $\{m_{2k}\}_{k=1}^{\infty}=\mathrm{T}_{k\rightarrow m}^{q,\gamma}(\{k_{l}\}_{l=1}^{\infty})$, $m_{k}^{'}=\frac{m_{2k}}{(q\gamma)^{k}}$, $k_{l}^{'}=2^{2l-1}k_{2l}$ for $l=1,2,...$. Then
    \begin{equation}
        \lim_{q\rightarrow\infty}\{m_{k}^{'}\}_{k=1}^{\infty}=\mathrm{T}_{k\rightarrow m}^{\gamma}(\{k_{l}^{'}\})_{l=1}^{\infty}.
    \end{equation}
\end{thm}
\begin{proof}
    This follows from a straightforward limit transition of  (\ref{eq_momenttocumulant}) under the assigned rescaling, and the moment-cumulant relation of $\gamma$-convolution in \cite[Theorem 3.11]{BCG}.
\end{proof}

Moreover, we point out that the combinatorial moment-cumulant formula of $\gamma$-convolution, given in \cite[Theorem 3.10]{BCG} can be expressed in an alternate way similar to our Theorem \ref{thm:cumulanttomomentcomb}. 
\begin{prop}\label{prop:gammacumulantcomb}
    Let $\{m_{k}^{'}\}_{k=1}^{\infty}=\mathrm{T}_{k\rightarrow m}^{\gamma}(\{k^{\gamma}_{l}\}_{l=1}^{\infty})$. Then for each $k=1,2,...$,
    \begin{equation}\label{eq_gammacumulantcomb}
        m^{'}_{k}=\sum_{\pi\in NC(k)}W(\pi)\prod_{B\in \pi}k^{\gamma}_{|B_{i}|},
    \end{equation}
    where $W(\pi)$ is defined in the same way as in Definition \ref{def:weight}, after replacing the values of $C_{1}$, $C_{2}$... to be $\gamma+1$, $\gamma+2$...
\end{prop}
\begin{proof}
    By \cite[Definition 3.7 and Theorem 3.8]{BCG}, 
    $$m_{k}=[z^{0}](\partial+\gamma d+*_{g})^{k-1}g(z),\ k=1,2,...$$
    The statement then follows from the similar argument as in the proof of Theorem \ref{thm:cumulanttomomentcomb}.
\end{proof}
\subsection{Connections to the classical convolutions}\label{sec:connection}
Recall from previous sections that, limit of $\boxplus_{M,N}^{\theta}$ gives the q-$\gamma$ convolution $\boxplus_{q,\gamma}$ of two (virtual) probability measures on $\R$, which is linearized by q-$\gamma$ cumulants. We show in this section that, under certain limit transition of the parameters $q,\gamma$, $\boxplus_{q,\gamma}$ converge to the usual convolution, classical free convolution, and rectangular free convolution respectively.

We first provide the connection of q-$\gamma$ cumulants to usual cumulants. For this we recall the combinatorial classical moment-cumulant formula: for $k=1,2,...$ 
    $$m_{k}=\sum_{\pi^{'}\in P(k)}\prod_{i=1}^{n}k^{'}_{|B_{i}|}$$
    where $\{k^{'}_{l}\}_{l=1}^{\infty}$ stands for the usual cumulants, and $P(k)$ is the set of all set partitions of $[k]$. We denote the map that sends $\{m_{k}\}_{k=1}^{\infty}$ to $\{k^{'}_{l}\}_{l=1}^{\infty}$ by $\mathrm{T}_{m\rightarrow k}^{0}$.

    \begin{thm}\label{thm:usualcumulant}
        Given a real sequence $\{m_{2k}\}_{k=1}^{\infty}$, let $\{k_{l}\}_{l=1}^{\infty}=\mathrm{T}_{m\rightarrow k}^{q,\gamma}(\{m_{2k}\}_{k=1}^{\infty})$, 
        $k^{'}_{l}=(q\gamma)^{l}2^{2l-1}(l-1)!k_{2l}$, 
        then 
        \begin{equation}\label{eq_usualcumulantlimit}
            \lim_{\gamma\rightarrow 0,\ q\gamma\rightarrow\infty}\{k^{'}_{l}\}_{l=1}^{\infty}=\mathrm{T}_{m\rightarrow k}^{0}(\{m_{2k}\}_{k=1}^{\infty}).
        \end{equation}
    \end{thm}
    \begin{proof}
        By Theorem \ref{thm:cumulanttomomentcomb}, after rescaled by $(q\gamma)^{k}$, the coefficient $W(\pi)$ does not vanish asymptotically only if for each $i=1,2,...,m$, $2l_{i}-1:=P_{i}-Q_{i}$,
        there are $l_{i}$ terms in $C_{Q_{i}+1}\cdots C_{P_{i}}$ that contain $q\gamma$. Hence each $Q_{i}$ must be even.

        Recall that $NC(k)$ denote the space of all (not neccesary even) non-crossing partitions. We say a non-crossing even partition $\pi$ of $[2k]$ is \emph{equivalent to $\pi^{'}\in NC(k)$}, if there exists some $\pi^{'}\in NC(k)=B_{1}^{'}\sqcup ...\sqcup B_{m}^{'}$, such that by replacing all element $j\in B_{i}$ by $\{2j-1,2j\}$, we get the set $B_{i}$, for any $i=1,2,...,m$. 

        \noindent{\textbf{Claim:}} For $\pi=B_{1}\sqcup ...\sqcup B_{m}\in \mathfrak{NC}(2k)$,  each $Q_{i}$ is even if and only if $\pi$ is equivalent to some $\Tilde{\pi}\in NC(k)$.
        \begin{proof}[\textbf{Proof of the claim}]
            The "if" part is clear. For the "only if" part, just notice that when $\pi$ is even, non-crossing and each $Q_{i}$ is even, $max(B_{i})$ turn out to be all even. The statement then follows by going over all the legs in the graphical representation of $\pi$ from right to left.
        \end{proof}

        Set $\Tilde{C}_{i}=i$, then after taking the limit,
        $$\frac{C_{Q_{i}+1}\cdots C_{P_{i}}}{(q\gamma)^{l_{i}}}\xrightarrow[q\gamma\rightarrow\infty]{\gamma\rightarrow0} 2^{2l_{i}-1}\Tilde{C}_{Q_{i}+1}\cdots \Tilde{C}_{P_{i}}.$$
        
        In other words, 
        \begin{equation}
            \begin{split}
                m_{2k}=&\sum_{\pi=B_{1}\sqcup ...\sqcup B_{m}\in \mathfrak{NC}(2k)}W(\pi)\prod_{B\in\pi}k_{|B_{i}|}\\
                \xrightarrow[q\gamma\rightarrow \infty]{\gamma\rightarrow 0}
                &\sum_{\Tilde{\pi}=\Tilde{B}_{1}\sqcup ...\sqcup \Tilde{B}_{m}\in NC(k)}\prod_{i=1}^{m}\Big[ (Q_{i}+1)\cdots (P_{i})\cdot k^{'}_{|\Tilde{B}_{i}|}\Big]\\
                =&\lim_{\gamma\rightarrow0}\sum_{\Tilde{\pi}=\Tilde{B}_{1}\sqcup ...\sqcup \Tilde{B}_{m}\in NC(k)}\prod_{i=1}^{m}\Big[ (\gamma+Q_{i}+1)\cdots (\gamma+P_{i})\cdot k^{'}_{|\Tilde{B}_{i}|}\Big]\\
                =&\lim_{\gamma\rightarrow0}\mathrm{T}_{k\rightarrow m}^{\gamma}(\{k_{l}^{'}\}_{l=1}^{\infty})_{k}=\mathrm{T}_{k\rightarrow m}^{0}(\{k_{l}^{'}\}_{l=1}^{\infty})_{k}\\
                =&\sum_{\pi^{'}=B_{1}^{'}\sqcup ...\sqcup B_{m}^{'}\in P(k)}\prod_{i=1}^{m}k^{'}_{|B^{'}_{i}|}.
            \end{split}
        \end{equation}
The two equalities in the second to last row hold by Proposition \ref{prop:gammacumulantcomb} and \cite[Theorem 8.2]{BCG} respectively. Then (\ref{eq_usualcumulantlimit}) follows from acting $\mathrm{T}_{m\rightarrow k}^{0}$ on both sides.       
\end{proof}

\begin{cor} For two real sequences $\{m_{2k}^{a}\}_{k=1}^{\infty}$, $\{m_{2k}^{b}\}_{k=1}^{\infty}$, set $m_{2k-1}^{a}=m_{2k-1}^{b}=0$ for $k=1,2,...$, and define 
    \begin{equation}
        \{m^{c}_{k}\}_{k=1}^{\infty}:=\lim_{\gamma\rightarrow 0,\ q\gamma\rightarrow \infty}\Big[\{m^{a}_{k}\}_{k=1}^{\infty}\boxplus_{q,\gamma}\{m^{b}_{k}\}_{k=1}^{\infty}\Big].
    \end{equation}
    Then $m^{c}_{2k-1}=0$ for $k=1,2,...$, and the usuall cumulants of $\{m^{c}_{2k}\}_{k=1}^{\infty}$'s are given by the sum of the corresponding usual cumulants of $\{m^{a}_{2k}\}_{k=1}^{\infty}$ and $\{m_{2k}^{b}\}_{k=1}^{\infty}$, i.e,
    \begin{equation}
        \mathrm{T}_{m\rightarrow k}^{0}\Big(\{m^{c}_{2k}\}_{k=1}^{\infty}\Big)=\mathrm{T}_{m\rightarrow k}^{0}\Big(\{m^{a}_{2k}\}_{k=1}^{\infty}\Big)+\mathrm{T}_{m\rightarrow k}^{0}\Big(\{m^{b}_{2k}\}_{k=1}^{\infty}\Big).
    \end{equation}
    \end{cor}
    \begin{remark}
        Suppose $\mu_{a}, \mu_{b}$ are two probability measures on $\R_{\ge 0}$ that for $k=1,2,...$
        $$m_{2k}^{a}=\int x^{k} d\mu_{a},\; m_{2k}^{b}=\int x^{k} d\mu_{b},$$
        then $\{m^{c}_{2k}\}_{k=1}^{\infty}$ are the moments of the usual convolution of $\mu_{a}$ and $\mu_{b}$.
    \end{remark}


    Next, we consider $\boxplus_{q,\gamma}$ and match its asymptotic behavior with the classical and rectangular free convolution. Before that we recall the definitions of their corresponding cumulants. For $k\in \Z_{\ge 1}$, let $\mathrm{T}_{r\rightarrow m}^{'}$ denote the map sending the real sequence $\{r_{l}\}_{l=1}^{\infty}$ of classical free cumulants to the sequence $\{m_{k}\}_{k=1}^{\infty}$ of moments. Then there is a moment-cumulant formula:
    \begin{equation}\label{eq_freecumulanttomoment}
        m_{k}=\sum_{\pi=B_{1}\sqcup...\sqcup B_{m}\in NC(k)}\prod_{B\in\pi}r_{|B_{i}|}     
\end{equation}
for $\{m_{k}\}_{k=1}^{\infty}=\mathrm{T}_{r\rightarrow m}^{'}
(\{r_{l}\}_{l=1}^{\infty})$. See e.g \cite{No} for a reference.

Similarly, as defined in \cite[Section 3.1]{B1}, rectangular free cumulants are a real sequence $\{c_{l}^{q}\}_{l=1}^{\infty}$ parametrized by $q\ge 1$, such that for $l=1,2,...$, $c^{q}_{2l-1}=0$, and $c^{q}_{2l}$ are related with moments $\{m_{k}\}_{k=1}^{\infty}$ by the following identities:
\begin{equation}\label{eq_recfreecumulant}
    m_{2k}=\sum_{\pi\in \mathfrak{NC}(2k)}q^{-e(\pi)}\prod_{B\in \pi}c_{|B_{i}|},
\end{equation}
where $e(\pi)=\#$ of block $B_{i}$'s with even $\min(B_{i})$,  
and $m_{2k-1}=0$ for $k=1,2,...$ Denote the map sending even moments to rectangular free cumulants by $\mathrm{T}^{\infty}_{m\rightarrow k}$, i.e, $\mathrm{T}^{\infty}_{m\rightarrow k}(\{m_{2k}\}_{k=1}^{\infty})=\{c_{l}\}_{l=1}^{\infty}$.
    \begin{thm}\label{thm:connectiontofreecumulant}
        Given a real sequence $\{m_{2k}\}_{k=1}^{\infty}$, $q\ge 1$, let
        $$\{k_{l}\}_{l=1}^{\infty}=\mathrm{T}_{m\rightarrow k}^{q,\gamma}(\{m_{2k}\}_{k=1}^{\infty}),\quad r_{l}^{\gamma}=(2q\gamma)^{l-1}\cdot k_{l}.$$
        Then we have the following.
        
        (a).
$$\lim_{\gamma\rightarrow\infty}\{r_{l}^{\gamma}\}_{l=1}^{\infty}=\mathrm{T}_{m\rightarrow k}^{\infty}(\{m_{2k}\}_{k=1}^{\infty}).$$

        (b).
        $$\lim_{\gamma\rightarrow\infty}\{r_{l}^{\gamma}\}_{l=1}^{\infty}=\mathrm{T}_{m\rightarrow r}^{'}(\{m_{2k}\}_{k=1}^{\infty})$$
        when $q=1$.
    \end{thm}
    \begin{remark}
        (b) is a special case of (a) when $q=1$. Such connection of rectangular free convolution and classical free convolution was first pointed out in \cite[Remark 2.2]{B1}.
    \end{remark}
    \begin{proof}
    It suffices to prove (a).
        By Theorem \ref{thm:cumulanttomomentcomb}, 
        $$m_{2k}=\sum_{\pi\in \mathfrak{NC}(2k)}W(\pi)\prod_{B\in \pi}k_{|B_{i}|},$$
        where $C_{i}(i=1,2,...)$ are $2\gamma, 2\gamma+2, 2\gamma+2, 2\gamma+4, 2\gamma+4$...
        Hence by taking $\gamma\rightarrow \infty$,  the right side above becomes 
        \begin{equation}\label{eq_cpi}
        \sum_{\pi\in \mathfrak{NC}(2k)}q^{-c(\pi)}\prod_{B\in \pi}c_{|B_{i}|},\end{equation}
        where $c(\pi):=\#$ of block $B_{i}$'s such that $Q_{i}$ is odd.
        Since 
        \begin{align*}
            Q_{i}\ \text{is\ odd}&\iff \text{\ there\ are\ odd\ elements\ of\ } B_{1},...,B_{i-1} \text{\ bigger\ than\ }\max(B_{i})\\
            &\iff \text{\ there\ are\ odd\ elements\ of\ } B_{1},...,B_{i-1} \text{\ smaller\ than\ }\min(B_{i})\iff \min(B_{i}) \text{\ is\ even},
        \end{align*}
        $c(\pi)=e(\pi)$, and (\ref{eq_cpi}) is equal to the right side of (\ref{eq_recfreecumulant}).
    \end{proof}
    
    Recall also that similar to q-$\gamma$ convolution, free convolution and rectangular free convolution are both binary operation of two probability measures linearized by free cumulants. Therefore Theorem \ref{thm:connectiontofreecumulant} implies the following.

    \begin{cor} Given $q\ge 1$, for two real sequences $\{m_{2k}^{a}\}_{k=1}^{\infty}$, $\{m_{2k}^{b}\}_{k=1}^{\infty}$, set $m_{2k-1}^{a}=m_{2k-1}^{b}=0$ for $k=1,2,...$, and define 
    \begin{equation}
        \{m^{c}_{k}\}_{k=1}^{\infty}:=\lim_{\gamma\rightarrow \infty}\Big[\{m^{a}_{k}\}_{k=1}^{\infty}\boxplus_{q,\gamma}\{m^{b}_{k}\}_{k=1}^{\infty}\Big].
    \end{equation}
    Then the free cumulants of $\{m^{c}_{k}\}$'s are given by the sum of the corresponding rectangular free cumulants of $\{m^{a}_{k}\}_{k=1}^{\infty}$ and $\{m_{k}^{b}\}_{k=1}^{\infty}$, i.e,
    \begin{equation}
        \mathrm{T}_{m\rightarrow k}^{\infty}\Big(\{m^{c}_{k}\}_{k=1}^{\infty}\Big)=\mathrm{T}_{m\rightarrow k}^{\infty}\Big(\{m^{a}_{k}\}_{k=1}^{\infty}\Big)+\mathrm{T}_{m\rightarrow k}^{\infty}\Big(\{m^{b}_{k}\}_{k=1}^{\infty}\Big).
    \end{equation}
    \end{cor}
    \begin{remark}
        Suppose $\mu_{a}, \mu_{b}$ are two symmetric probability measures on $\R$ that for $k=1,2,...$
        $$m_{k}^{a}=\int x^{k} d\mu_{a},\; m_{k}^{a}=\int x^{k} d\mu_{a},$$
        then $\{m^{c}_{k}\}_{k=1}^{\infty}$ are the moments of the rectangular free convolution of $\mu_{a}$ and $\mu_{b}$.
    \end{remark}
    \begin{remark}
        Similar results hold for classical free convolution when $q=1$.
    \end{remark}



\subsection{Law of large numbers of Laguerre $\beta$ ensembles}\label{sec:laguerre}
For $M\le N$ and $\theta=\frac{1}{2},1,2$, a $M\times N$ Wishart matrix $X$ is a rectangular random matrix, whose entries are real/complex/real quaternionic i.i.d Gaussian random variables $\mathcal{N}(0,1)$/$\mathcal{N}(0,1)+i \mathcal{N}(0,1)$/$\mathcal{N}(0,1)+i \mathcal{N}(0,1)+j \mathcal{N}(0,1)+k \mathcal{N}(0,1)$. One can check directly that $X$ satisfies the same invariant property given in Section \ref{sec:addition}, with $M$ random singular values $\Vec{x}_{M}=(x_{1,M}\ge ...\ge x_{M,M}\ge 0)$.

 The density of $\Vec{x}_{M}$ is (see e.g \cite[Chapter 3]{Forrester})
\begin{equation}f(\Vec{x}_{M};M,N,\theta)=\frac{1}{Z_{M,N,\theta}}\prod_{i=1}^{M}\Big[x_{M,i}^{2\theta (N-M+1)-1}\exp(-\frac{1}{2} x_{M,i}^{2})\Big]\prod_{1\le j\le k\le M}(x_{M,j}^{2}-x_{M,k}^{2})^{2\theta},\end{equation}
where $Z_{N,M,\theta}$ is the normalizing constant. While for general $\theta>0$
there's again no skew field of real dimension $2\theta$,
$f(\Vec{x}_{M};M,N,\theta)$ continues to make sense, and is defined as the so-called Laguerre $\beta$ ensemble.

\begin{remark}
    It's easy to check that $f(\Vec{x}_{M};M,N,\theta)$ is an exponential decaying measure defined in Definition \ref{def:expdecaying}, and therefore by Theorem \ref{thm:dunklonbgf}, its type BC Bessel generating function is defined and well-behaved under the action of type BC Dunkl operators. 
\end{remark}

\begin{prop}
    Let $G_{M,N,\theta}^{L}(z_{1},...,z_{M})$ denote the type BC Bessel generating function 
    $$\int_{x_{M,1}\ge ...\ge x_{M,M}\ge 0}\B(\Vec{x}_{M},z_{1},...,z_{M};\theta,N)f(\Vec{x}_{M};M,N,\theta)dx_{M,1}\cdots dx_{M,M},$$ then 
     
    $$G_{M,N,\theta}^{L}(z_{1},...,z_{M})=\exp\Big[\frac{1}{2}(z_{1}^{2}+...+z_{M}^{2})\Big].$$
    
\end{prop}
\begin{proof}
    For $\theta=\frac{1}{2},1,2$, one can use Definition \ref{def:matrixintegral} and check this by hand. For general $\theta>0$, this is a special case of \cite[Proposition 2.37.(2)]{Ro}, such that in that identity $y$ is set to be 0, and our $\B(\Vec{a},z_{1},...,z_{M};\theta,N)$ is a symmetric version of $E_{k}(x,z)$, see \cite[Definition 2.35]{Ro}
\end{proof}
For each $M,N,\theta$, Denote the random empirical measure of $f(\Vec{x}_{M};M,N,\theta)$ by $\mu_{M,N,\theta}:=\frac{1}{M}\sum_{i=1}^{M}\delta_{x_{M,i}^{2}}$. 
\begin{thm}
    As $M\rightarrow \infty, N\rightarrow \infty, \theta\rightarrow 0, M\theta\rightarrow \gamma, N\theta\rightarrow q\gamma$, 
    $$\mu_{M,N,\theta}\longrightarrow \mu_{q,\gamma}$$
    weakly in moments, where $\mu_{q,\gamma}$ is a probability measure on $\R_{\ge 0}$, which is uniquely determined by its moments:
    \begin{equation}\label{eq_cumulanttomomentlaguerre}
        m^{'}_{k}=\int_{\R \ge 0} x^{k} d\mu_{q,\gamma}=\sum_{\pi}\prod_{i=1}^{k}C_{P_{i}},\ \text{for}\ k=1,2,...
    \end{equation}
    where $C_{l}$'s, $P_{i}$'s are defined in the same way as in Section \ref{sec:cumulanttomoment}, and $\pi$ goes over all non-crossing perfect matching of $[2k]$.
\end{thm}
\begin{remark}
    With a bit more efforts (e.g, a tightness argument), it's likely that one can show the convergence of the empirical measure holds weakly in probability.
\end{remark}
\begin{proof}
    By taking logarithm and partial derivatives of $G_{M,N,\theta}^{L}$ we have that $\{\Vec{x}_{M}\}$ is $q$-$\gamma$-LLN appropriate with q-$\gamma$ cumulant $k_{2}=1$, $k_{l}=0$ for $l\ne 2$. By Theorem \ref{thm:cumulanttomomentcomb}, only the set partitions that are formed by blocks of size two survive, and in this case $P_{i}=Q_{i}+1$. (\ref{eq_cumulanttomomentlaguerre}) is then specified from (\ref{eq_cumulanttomomentcomb}).

    It remains to show that the moments in (\ref{eq_cumulanttomomentlaguerre}) does correspond to a unique probability measure. This is the so-called Stieltjes moment problem, since the (potential) corresponding measure lies on $[0,\infty)$, see e.g \cite{Ak}.
    We need to check
    $$\sum_{k=1}^{\infty}(m^{'}_{k})^{-\frac{1}{2k}}=\infty.$$
    Again by (\ref{eq_cumulanttomomentlaguerre}), $m_{k}$ is a sum of $\prod_{i=1}^{k}C_{P_{i}}$'s. Among these summands the biggest term corresponds to $P_{i}=i$ for $i=1,2,...,k$, and 
    $$\prod_{i=1}^{k}C_{P_{i}}\le C(C+2)(C+4)\cdots (C+2k-2)=2^{k}\frac{\Gamma(\frac{C}{2}+k)}{\Gamma(\frac{C}{2})},$$
    where $C:= max\{2q\gamma-2,2\gamma\}$.
    The number of non-crossing perfect matching is $Cat(k)=\frac{1}{k+1}\binom{2k}{k}$, the $k^{th}$ Catalan number. Multiplying these two gives an upper bound of $m_{k}$. By Stirling approximation, it turns out that 
    $$(m^{'}_{k})^{\-\frac{1}{2k}}\le C_{1}\cdot \sqrt{k}$$
    for some positive constant $C_{1}$. Hence the series diverges.
\end{proof}

The limiting measure $\mu_{q,\gamma}$ is an q-$\gamma$ analog of the Gaussian and semicircle law, in the sense that their only nonvanishing (q-$\gamma$/classical/free) cumulant is $k_{2}=1$. 

Moreover, the connections to the usual and free convolution in Section \ref{sec:connection} continues to hold in this special case. Indeed, one can show from (\ref{eq_cumulanttomomentlaguerre}) that 
$$\frac{m^{'}_{k}}{(q\gamma)^{k}}\xrightarrow[q\gamma\rightarrow\infty]{\gamma\rightarrow 0}\sum_{\pi}\prod_{i=1}^{k}(2),$$
where $\pi$ goes over all set partitions of $[k]$ into k blocks 
(which is indeed a single one), since any other non-crossing perfect matching has coefficient with $C_{2}=2\gamma+2$, and therefore after rescaled by $(q\gamma)^{k}$ this term vanishes in the limit. The sum on the right is equal to $2^{k}$, which means $m_{k}^{'}=2^{k}$, and 
$$\mu_{q,\gamma}\longrightarrow \delta_{2}$$
weakly when $q\gamma\rightarrow\infty, \gamma\rightarrow 0$.
\begin{remark}
    One can obtain the same result, by doing limit transition on the density of Laguerre ensemble $f(\Vec{x}_{M};M,N,\theta)$ after a change of variables $\lambda_{i}=x_{i}^{2}$, in the regime  $M\rightarrow \infty, N\rightarrow \infty, \theta\rightarrow 0, M\theta\rightarrow 0, N\theta\rightarrow \infty$.
\end{remark}

On the other hand, by taking $q=1$, $\gamma\rightarrow \infty$, (\ref{eq_cumulanttomomentlaguerre}) becomes
\begin{equation}
    \frac{m^{'}_{k}}{(2\gamma)^{k}}\longrightarrow \#\ \text{of non-crossing\ perfect\ matchings\ of\ } [2k]=Cat(k).
\end{equation}
$Cat(k)$ is exactly the $2k^{th}$ moment of the semicircle law.
\section{Duality between convolutions in high and low temperature}\label{sec:duality}
After studying the behavior of rectangular matrix additions in both high and low temperatures, we present a quantitative connection between these two regimes.

Recall that in low temperature regime, given two deterministic M-tuples $\Vec{a}$, $\Vec{b}$, the limit of $\vec{c}=\Vec{a}\boxplus_{M,N}^{\theta}\Vec{b}$ is a deterministic M-tuples $\Vec{\lambda}$, where $\Vec{\lambda}$ is the $(M,N)$-rectangular finite convolution of $\Vec{a}$ and $\Vec{b}$. For a M-tuples $\Vec{a}=(a_{1},...,a_{M})$, let $r_{i}=a_{i}^{2}$ for $i=1,2,...,M$. Let $m_{k}^{'}=\frac{1}{M}(r_{1}^{k}+...+r_{d}^{k})$ be the finite version of moments, for $k=1,2,...,M$. Then the $(M,N)$-rectangular finite convolution of $\Vec{a}$ and $\Vec{b}$ can be thought as a deterministic binary operation of $\{m^{'}_{k}(\Vec{a})\}_{k=1}^{M}$ and $\{m^{'}_{k}(\Vec{b})\}_{k=1}^{M}$. Similarly, we view the q-$\gamma$ convolution of $\{m_{k}^{a}\}\}_{k=1}^{\infty}$ and $\{m_{k}^{b}\}\}_{k=1}^{\infty}$ as a deterministic binary operation of $\{m_{k}^{a}\}\}_{k=1}^{M}$ and $\{m_{k}^{b}\}\}_{k=1}^{M}$.
\begin{thm}\label{thm:duality}
     By identifying $M$ with $-\gamma$, $\frac{M}{N}$ with $q$, $m_{2k}$ with $m^{'}_{k}(-N)^{k}$ for $k=1,2,...,M$, the $(M,N)$-rectangular convolution matches the $q$-$\gamma$ convolution as binary operation of first $M$ nontrivial moments.
\end{thm}

The theorem is claiming that under the above identification, the moment-cumulant formula of these two convolutions are the same, and therefore we need to introduce a version of cumulants for the rectangular finite convolution. For this we refer to 
\cite{Gri}, which considers sum of two invariant $M\times N\ (M=N\lambda,\ \lambda\in [0,1])$ rectangular matrices as in Section \ref{sec:lowtemp}, and defines the rectangular finite R-transform as the analog of the R-transform in (classical) free probability theory, in the sense that it linearizes the finite rectangular addition.
\begin{definition}{\cite[Definition 3.7]{Gri}}
$R^{M,\lambda}_{S_{p_{A}}}(z)$ is the unique polynomial of degree $M$ verifying 
\begin{equation}\label{eq_finiterecrtransform}
    R^{M,\lambda}_{S_{p_{A}}}(z)\equiv \frac{-1}{M}z\frac{d}{dz}\ln\Big(\E{e^{-T^{(N,M)}_{S_{p_{A}}}zNM}}\Big)\quad \text{mod}\ [z^{M+1}],
\end{equation}
where $T^{(N,M)}_{S_{p_{A}}}$ is a random variable. By \cite[p13]{Gri}, for $i=1,2,...,M$,
\begin{equation}\label{eq_finiterecadditionrv}
    \E{(T^{(N,M)}_{S_{p_{A}}})^{i}}=\frac{i!(m-i)!}{m!}\frac{(d-i)!}{d!}a_{i},
\end{equation}
where $a_{i}=e_{i}(\Vec{r})$.
\end{definition}
Inspired by the fact that (classical) R-transform is the generating function of free cumulants, We define the rectangular finite cumulants, such that 
\begin{equation}
    R^{d,\lambda}_{S_{p_{A}}}(z)=\sum_{l=1}^{d}k^{m,d}_{l}z^{l}.
\end{equation}
$k^{m,d}_{1},...,k^{m,d}_{d}$ uniquely determine $r_{1},...,r_{d}$. 

\begin{proof}[Proof of Theorem \ref{thm:duality}]
    We prove that under the following identification of parameters
    \begin{equation}\label{eq_parameterid}
        \begin{split}
            k^{N,M}_{l}&\longleftrightarrow \frac{k_{2l}}{2}\gamma^{l-1}\; \text{for}\ l=1,2,...,M\\
            M&\longleftrightarrow -\gamma\\
            \frac{N}{M}&\longleftrightarrow q\\
            N^{n}a_{n}&\longleftrightarrow c_{n}\\
          m_{2k}&\longleftrightarrow  m_{k}^{'}\cdot (-N)^{k}  ,
        \end{split}
    \end{equation}
    the moment-cumulant relation in rectangular finite convolution and (\ref{eq_momenttocumulant}) match exactly.

    We match the second formula of (\ref{eq_momenttocumulant}) and (\ref{eq_finiterecrtransform}), which play the role of cumulant generating function in their own setting. Let $y^{2}=(-M)z$, then the first formula of (\ref{eq_momenttocumulant}) becomes 
    \begin{equation}\label{eq_cumulanttomomentmatchform}
    \begin{split}
        &\exp\Big(\sum_{l=1}^{\infty}\frac{k_{2l}}{2l}\gamma^{l}z^{l}\Big)=\sum_{k=0}^{\infty}\frac{c_{k}}{(q\gamma)_{k}(\gamma)_{k}}(-M)^{k}z^{k}\\
    \Longrightarrow &\sum_{l=1}^{\infty}k_{2l}\gamma^{l-1}z^{l}=-\frac{1}{d}z\frac{d}{dz}\ln\Big(\sum_{k=0}^{\infty}\frac{c_{k}}{(q\gamma)_{k}(\gamma)_{k}}(-M)^{k}z^{k}\Big).
    \end{split}
    \end{equation} 

    It remains to match the right side of (\ref{eq_cumulanttomomentmatchform}) and (\ref{eq_finiterecrtransform}), i.e, matching 
    $$\E{e^{-T^{(N,M)}_{S_{p_{A}}}zNM}}\quad\quad  \text{with}\quad\quad \sum_{k=0}^{\infty}\frac{c_{k}}{(q\gamma)_{k}(\gamma)_{k}}(-M)^{k}z^{k}$$
    for $k=1,2,...,M$.
    This follows by Talor expanding $e^{-T^{(N,M)}_{S_{p_{A}}}zNM}$, (\ref{eq_finiterecadditionrv}) and (\ref{eq_parameterid}).

    Then we identify the first formula of (\ref{eq_momenttocumulant}) with the moment generating function in rectangular finite convolution. In the latter setting, recall that  $a_{n}=e_{n}(\Vec{r})$ for $n=1,2,...,M$, and $m_{k}^{'}=\frac{1}{M}p_{k}(\Vec{r})$ for $k=1,2,....$. Moreover, take $r_{i}=0$ for all $i>M$, and identify $a_{n}$ with $e_{n}(\Vec{r})$ formally for $n>M$ (both have value 0) as well. Then on the rectangular finite addition side,
    \begin{equation}
        \begin{split}
            &\sum_{n=0}^{\infty}N^{n}a_{n}y^{2n}=\sum_{n=0}^{\infty}e_{n}(\Vec{r})(Ny^{2})^{n}\\
            =&\prod_{n=1}^{\infty}(1+r_{n}Ny^{2})=\exp\Big(-\sum_{k=1}^{\infty}\frac{p_{k}(\Vec{r})(-N)^{k}y^{2k}}{k}\Big)=\exp\Big(-M\sum_{k=1}^{\infty}\frac{m^{'}_{k}(-N)^{k}y^{2k}}{k}\Big).
        \end{split}
    \end{equation}
    This matches the first formula of (\ref{eq_momenttocumulant}) under the identification of parameters.
\end{proof}
\begin{remark}
    After identifying the $k_{1}^{m,d},...,k_{d}^{m,d}$ with the first d even q-$\gamma$ cumulants, one can define $k_{l}^{m,d}$ for $l\ge d+1$ for rectangular finite convolution, by the moment-cumulant relation of q-$\gamma$ convolution under the same parameter identification in (\ref{eq_parameterid}). 
\end{remark}
\begin{remark}
    Note that in (\ref{eq_parameterid}), both $M$ and $\gamma$ are positive, hence there's no choice of parameters that the finite rectangular cumulants coincide with q-$\gamma$ cumulants. Instead, one can combine the domain of these two groups of parameters, and treat the result as an extension of the moment-cumulant relation to, say, $\gamma\in \R_{>0}\bigcup \Z_{\le -1}$. 
\end{remark}

\section{Appendix A:Dunkl operators and hypergeometric functions}
In this appendix, we give a brief review of the basic settings of multivariate hypergeometric functions defined by abstract root systems, the differential operators acting on them, their connection to symmetric spaces and their spherical functions, and the limit transition to multivariate Bessel functions. The purpose is to provide a theoretical background to the particular objects appearing and used later in this text, and explain the connections between them. In Section 2.2 to 2.5, we specify from general theory to the special case and provide more concrete formulas, that we  operate with in Section 3-6. 

A large part of our presentation is a simplification of  \cite{RR} Section 2 and 3, which gives a brief and clear review of the theory with more explanations of the concepts. For more detailed exposition of Dunkl theory, see \cite{Ro} and \cite{A}.


For any $M\ge 1$, consider the Euclidean space $\R^{M}$ with the standard scalar product $\langle x,y\rangle=\sum_{i=1}^{M}x_{i}y_{i}$. For $\alpha\in \R^{M}\setminus \{0\}$, denote the reflection of point $x$ about the hyperplane $\langle\alpha\rangle^{\perp}$ by $\sigma_{\alpha}$, such that 
$$\sigma_{\alpha}(x)=x-2\frac{\langle\alpha,x\rangle}{\langle\alpha,\alpha \rangle^{2}}\alpha.$$
    

Clearly each $\sigma_{\alpha}$ is an element in the orthogonal group $O(M)$.

\begin{definition}
    A root system $R$ is a finite set of vectors in $\R^{M}\setminus \{0\}$, such that $\sigma_{\alpha}(R)=R$ for all $\alpha\in R$. We say $R$ is irreducible if it cannot be decomposed into two disjoint subsets whose elements are mutually orthogonal. $R$ is crystallographic, if for any $\alpha, \beta\in R$ we have 
$$\frac{2\langle \alpha,\beta \rangle}{\langle \beta,\beta \rangle}\in \Z.$$

    $\{\sigma_{\alpha}\}_{\alpha\in R}$ generate a subgroup of $O(M)$, which is called the Weyl group of root system $R$ and record the symmetry that $R$ has.

    Each root system can be written as a disjoint union $R=R_{+}\bigcup (-R_{+})$, such that $R_{+}$ and $-R_{+}$ are separated by some hyperplane through the origin. We call $R_{+}$ the positive part of $R$. 
\end{definition}
\begin{remark}
    The choice of $R_{+}$ is not unique, but all the choices are identical under a linear transformation.
\end{remark}
\begin{remark}
    In this text, we do not require $R$ to be reduced, i.e, $R\bigcap \R\alpha=\pm \alpha$, for all $\alpha\in R$.
\end{remark}
\begin{ex}
In practise, people care mostly about the classical root systems of type A-D. The following two crystallographic root systems appear in random matrices: the root system of $A_{M}$, $M=1,2,...$, that is
$$R=\{e_{i}-e_{j}, 1\le i<j\le M\},$$
and the root system $BC_{M}$, $M=1,2,...$, that is 
$$R=\{\pm e_{i}, 1\le i\le M, \pm 2e_{i}, 1\le i\le M, \pm e_{i}\pm e_{j}, 1\le i<j\le M\},$$
where $e_{i}$, $i=1,2,...,M$ denotes the $i^{th}$ standard basis in $\R^{M}$. 
\end{ex}
After recalling the notion of root system, we are now able give the definition of hypergeometric and Bessel functions, which were introduced in a series of works by Heckman and Opdam, see \cite{HS},\cite{O1},\cite{O2} for details. Fix $M\in \Z_{\ge 1}$, take $\mathfrak{A}$ to be a $M$-dimensional Euclidean space, let $\mathfrak{A}_{\C}$ be the complexification of $\mathfrak{A}$, which is isomorphic to $\C^{M}$, and let $R$ be a crystallographic root system on $\mathfrak{A}$ with Weyl group $W$, $R^{+}$ be the positive part of $R$, and $P^{+}$ be the set of dominant weights associated with $R^{+}$, i.e,
$$P^{+}:=\Big\{\lambda\in \mathfrak{A}: \frac{\langle \lambda, \alpha\rangle}{\langle \alpha,\alpha\rangle}\in \Z^{+}\ \text{for\ all\ }\alpha\in R^{+}\Big\}.$$

\begin{remark}
    View a partition $(\lambda_{1},...,\lambda_{M})$ as a vector with nonnegative integer entries. For root system of type A, $P^{+}$ can be identified with the set of all partitions of length at most $M$, and for root system of type BC, $P^{+}$ can be identified with the set of all even partitions of length at most $M$. 

    For $\mu,\lambda\in P^{+}$, we write $\mu\le \lambda$ if $\mu_{i}\le \lambda_{i}$ for all $i=1,2,...,M$.
\end{remark}

A root multiplicity function $m_{\alpha}: \alpha\in R$ on $R$ is a W-invariant map which assigns to each root in $R$ a real number. For $\alpha\in R$, let 
\begin{equation}\label{eq_rho}
\rho=\rho(m):=\frac{1}{2}\sum_{\alpha\in R^{+}}m_{\alpha} \alpha,
\end{equation}
and $\alpha_{i},\rho_{i}$ be the $i^{th}$ component of $\alpha, \rho$ respectively.

Let $s_{\alpha}$ to be the reflaction operator about $\{\alpha\}^{\perp}$ such that $s_{\alpha}f(x):=f(\sigma_{\alpha}(x))$. The following differential operator acts on any smooth function on $\mathfrak{A}$.
\begin{definition}\label{def:tridunkl}
    For $i=1,2,...,M$, given a root multiplicity function $m_{\alpha}$, the \emph{trigonometric Dunkl operator} associated with $R$ and $m$ is 
    \begin{equation}
        T_{i}=\partial_{i}+\sum_{\alpha\in R^{+}}m_{\alpha}\frac{\alpha_{i}}{1-e^{-2\langle \alpha, \cdot\rangle}}(1-s_{\alpha})-\rho_{i}
    \end{equation}
\end{definition}

Let $S(\mathfrak{A}_{\C})$ denote the space of complex polynomials $p$ in M variables, such that when identifying the $i^{th}$ variable with the $i^{th}$ standard basis $e_{i}$ in $\mathfrak{A}$,  $p\in S(\mathfrak{A}_{\C})$ is invariant under action of W. 
\begin{definition}\label{def:generalhypergeometric}
    For $\lambda\in \mathfrak{A}_{\C}$, the \emph{hypergeometric function associated with R} is an analytic W-invariant function $F_{\lambda}(x;m)$ on $\mathfrak{A}$, such that for each $p\in S(\mathfrak{A}_{\C})$, 
    \begin{equation}
        p(T)F_{\lambda}=p(\lambda)F_{\lambda}.
    \end{equation}
    We set $F_{\lambda}(0;m)=1$.
\end{definition}
\begin{thm}{\cite{HS}}\label{thm:existenceofhypergeometric}
    There exists an open set of root multiplicity functions $M_{reg}$, which contains all nonnegative $m$'s, such that if $m\in M_{reg}$, for each $\lambda\in \mathfrak{A}_{\C}$ there exists a unique function $F_{\lambda}(z;m)$ satisfying Definition \ref{def:generalhypergeometric}. Moreover, $F:\mathfrak{A}_{\C}\times M_{reg}\times \mathfrak{A}\rightarrow \C$ is analytic.
\end{thm}

Similar to hypergeometric functions, the multivariate Bessel functions are also defined as W-invariant eigenfunctions of some differential operators, which are called rational Dunkl operators.

\begin{definition}\label{def:rationaldunkl}
    For i=1,2,...,M, given a root multiplicity function $m_{\alpha}$, the \emph{rational Dunkl operator} associated with $R$ and $m$ is 
    \begin{equation}
        D_{i}=\partial_{i}+\sum_{\alpha\in R^{+}}m_{\alpha}\frac{\alpha_{i}}{2\langle\alpha,\cdot\rangle}(1-s_{\alpha})
    \end{equation}
\end{definition}

\begin{definition}\label{def:generalbessel}
    For $\lambda\in \mathfrak{A}_{\C}$, $m\ge 0$, the \emph{Bessel function associated with R} is an analytic W-invariant function $f_{\lambda}(x;m)$ on $\mathfrak{A}$, such that for each $p\in S(\mathfrak{A}_{\C})$, 
    \begin{equation}
        p(D)f_{\lambda}=p(\lambda)f_{\lambda}.
    \end{equation}
\end{definition}

Note that both $F_{\lambda}, f_{\lambda}$ are invariant in both $\lambda$ and $z$. Indeed, Bessel functions can be obtained from hypergeometric functions by a limit transition. This is simply because under the same limit transition, the trigonometric Dunkl operator $T_{i}$ converges to the corresponding rational Dunkl operator $D_{i}$.
\begin{prop}{\cite{SO1}, \cite[Section 4.4]{A}}\label{prop:generallimittransition}
For $\lambda\in \mathfrak{A}_{\C}$, $m\ge 0$,
    \begin{equation}
        f_{\lambda}(z;m)=\lim_{\epsilon\rightarrow 0}F_{\epsilon^{-1}\lambda}(\epsilon z;m).
    \end{equation}
\end{prop}
From now consider only $\lambda\in P^{+}$. Let 
$M_{\lambda}:=\sum_{\mu\in W.\lambda}e^{i\langle\mu,\cdot\rangle}$ be the trigonometric symmetric monomial on $\T$ indexed by $\lambda$, where $\T$ is a torus obtained as the quotient space of $\mathfrak{A}$, on which $e^{i\langle\mu,\cdot\rangle}$ is periodic. 

For simplicity take $m$ to be a nonnegative root multiplicity function (which is in $M_{reg}$ by Theorem \ref{thm:existenceofhypergeometric}). Let 
\begin{equation}\label{eq_weightgeneralform}
    w_{m}(z):=\prod_{\alpha\in R^{+}}\Bigr|e^{i\langle \alpha,z\rangle}-e^{-i\langle \alpha,z\rangle}\Bigr|^{m_{\alpha}}.
\end{equation}
\begin{definition}\label{def:generaljacobi}
    The \emph{Jacobi polynomials} (Heckman-Opdam polynomials) associated with $R$ and $m\ge 0$ are a collection of functions $\mathfrak{J}_{\lambda}$ on $T$ indexed by $\lambda\in P^{+}$, where 
    $$\mathfrak{J}_{\lambda}(\cdot;m)=\sum_{\mu\in P^{+},\ \mu\le \lambda}c_{\lambda\mu}(m)M_{\mu},$$
    and coefficients $c_{\lambda\mu}(m)$'s are uniquely determined by
    
    (1). $c_{\lambda\lambda}(m)=1$
    
    (2). $\mathfrak{J}_{\lambda}$'s are mutually orthogonal in $L^{2}(\T;w_{m})$.
\end{definition}
Note that $\mathfrak{J}_{\lambda}$'s form an orthogonal basis of $L^{2}(\T;w_{m})^{W}$, the subspace of W-invariant elements in $L^{2}(\T;w_{m})$. 
\begin{ex}
    When taking $R$ to be the type A root system, the Jacobi polynomials of type A are Jack polynomial $P_{\lambda}(\Vec{x};\theta)\in \Lambda_{M}$'s, where $x_{i}$ is identified with $e^{iz_{i}}$.
\end{ex}
The following important result identifies each hypergeometric function with a Jacobi polynomial, when the corrsponding weight $\lambda$ is positive integer-valued.
\begin{thm}{\cite[Section 4.4]{HS}}\label{thm:idjacobihypergeometric}
For all $\lambda\in P^{+}$, $m\ge 0$, the function $F_{\lambda+\rho}(x;m)$ extends holomorphically to $\mathfrak{A}_{\C}$, and 
    \begin{equation}
        F_{\lambda+\rho}(iz;m)=c(\lambda+\rho,m)\mathfrak{J}_{\lambda}(z;m),
    \end{equation}
    where $c(\lambda+\rho,m)$ is a constant depending on $\lambda$ and $m$.
\end{thm}
\begin{remark}
    For some multiplicity function $m$ that is not nonnegative, as long as the $L^{2}$ kernel $w_{m}(x)$ is integrable on $T$, $\mathfrak{J}_{\lambda}$'s are still well-defined and Theorem \ref{thm:idjacobihypergeometric} holds for such m. See Section \ref{sec:typebc}.
\end{remark}

When the root multiplicity function $m$ takes some special values and $\lambda\in P^{+}$, the Jacobi polynomial $\mathfrak{J}_{\lambda}$'s can be identified with the spherical functions on one of the classical symmetric spaces. 

The theory of symmetric spaces are classical, and the standard references are \cite{Hel1}, \cite{Hel2}, which discuss the classification problem, representation theory and analytic properties of symmetric spaces. In a word, a Riemannian symmetric space is a certain quotient of classical Lie groups $G/K$ or $U/K$, where $G$ is noncompact, $U$ is compact and $K$ is a compact subgroup. $G/K$ and $U/K$ are of the so-called noncompact type and compact type, respectively, and there's a duality between one noncompact symmetric space and one compact symmetric space.

Let $G/K$ and $U/K$ be the dual of each other, and let $g,u,l$ denote the Lie algebra of $G, U, K$ respectively, and let $\mathfrak{A}$ denote the maximal abelian subspace in $g/l$, then by duality $i\mathfrak{A}$ is the maximal abelian subspace in $u/l$. The restricted root system of $G/K$ (or $U/K$) is set to be on $\mathfrak{A}$ (or $i\mathfrak{A})$, and they share the same root multiplicities. In Appendix B we list the (restricted) root multiplicity functions of several compact symmetric spaces that are connected to random matrices.

For a noncompact symmetric space $G/K$, its spherical function is defined as a nonzero $K$-biinvariant function $\phi_{\lambda}: G\rightarrow \C$, which is an eigenfunction of any so-called invariant differential operator on $G/K$, indexed by $\lambda\in \mathfrak{A}_{\C}$. For a compact symmetric space $U/K$ with restricted root system $R$, its spherical function $\psi: U\rightarrow \C$ are indexed by the highest weights of unitary irreducible $K$-spherical representations (the representation with a 1-dimensional invariant subspace $V_{\lambda}^{K}$) $\pi_{\lambda}$ of $U$ , where the space of all highest weight is identified as $P^{+}(R)$, and each spherical function is given by 
$$\psi_{\lambda}(u)=\langle\pi_{\lambda}(u)e_{\lambda},e_{\lambda}\rangle,$$
where $e_{\lambda}$ is the unique unit vector in $V_{\lambda}^{K}$.

For a spherical function $\phi$ of $G/K$, by Cartan decomposition and the $K$-biinvariance $\phi$ is determined by the value on $\mathfrak{A}$, and same for $\psi$ of $U/K$. Moreover we have the following result.

\begin{thm}\label{thm:identification}
    For $x\in \mathfrak{A}$, any $\lambda\in P^{+}$, 
    \begin{equation}
        \psi_{\lambda}(\exp(ix))=\phi_{\lambda+\rho}(\exp(ix))=F_{\lambda+\rho}(ix;m)=c(\lambda+\rho,m)\mathfrak{J}_{\lambda}(x;m).
    \end{equation}
\end{thm}
The first equality is given in \cite{Hel2}, and the second equality is \cite[Theorem 5.2.2]{HS}.  
Because of this identification, we no longer distinguish spherical function (indexed by $\lambda\in P^{+}$ of noncompact and compact symmetric spaces, and treat them simply as an analytic function on M dimensional Euclidean space $\mathfrak{A}$.
\begin{remark}\label{rem:spherical}
The Heckman-Opdam Laplacian is an analog of the usual Laplace operator on $\R^{M}$, given by
    $p_{2}(T)=\sum_{i=1}^{M}T_{i}^{2}$. In the case of Theorem \ref{thm:identification}, 
    $$p_{2}(T)=\Delta+\langle\rho,\rho\rangle,$$
    where $\Delta$ is the Laplace-Beltrami operator on $G/K$ (or $U/K)$.
Moreover, the limit transition in Proposition \ref{prop:generallimittransition} specifies to a contraction of a Riemannian symmetric space to a corresponding Euclidean symmetric space, and convergence of the spherical function of the former to the latter. See \cite[Theorem 3.4]{SO2}.

The spherical functions of Riemannian and Euclidean symmetric spaces can both be written as the so-called Harish-Chandra integrals, see \cite[Chapter IV]{Hel2} or \cite[Section 2]{SO2} for their explicit forms. We give calculations of the Harish-Chandra integral corresponding to several Euclidean symmetric spaces in Appendix C.
\end{remark}

\section{Appendix B: Root multiplicities}
Let $U$ be a classical compact Lie group, and $K$ be a Lie subgroup of $U$. We list several examples that $U/K$ is a compact Riemannian symmetric space of rank $M$, and give its root multiplicities. For a complete list of classifications of irreducible Riemannian symmetric spaces, see \cite[Chapter X]{Hel1}.

Let $O(M),U(M),Sp(M)$ denote the $M\times M$ orthogonal/unitary/compact sympletic group. Let $M\le N$, $1\le i\ne j\le M$, $\{e_{i}\}_{i=1}^{M}$ be the standard basis of $\mathfrak{A}=\R^{M}$.

\begin{tabular}{ |p{5cm}||p{2cm}|p{2cm}|p{2cm}|p{2cm}|  }
 \hline
 Compact symmetric space & $m_{e_{i}-e_{j}}$ &$m_{\pm e_{i}\pm e_{j}}$&$m_{\pm e_{i}}$&$m_{\pm 2e_{i}}$  \\
\hline
$U(M+1)/O(M+1)$&  1 &  0   &0 &0\\

 $U(M+1)$&  2   & 0   &0&0\\

 $U(2M+2)/Sp(M+1)$&4 &0 & 0&  0\\
 
 $O(N+M)/O(N)\times O(M)$&0    &1 & $N-M$&  0\\
 
$U(N+M)/U(N)\times U(M)$& 0    & 2&$2(N-M)$&1\\

 $Sp(N+M)/Sp(N)\times Sp(M)$&0   & 4   &$4(N-M)$&3\\

 \hline
\end{tabular}

Now let $m$ be a root multiplicity function such that $m_{\pm e_{i}\pm e_{j}}=2\theta$, $m_{\pm e_{i}}=2\theta(N-M)$, $m_{\pm 2e_{i}}=2\theta-1$, the $\theta=\frac{1}{2},1,2$ cases correspond to the last three rows in the above table. On the other hand, $m_{e_{i}-e_{j}}=2\theta$ corresponds to the first three rows when $\theta=\frac{1}{2},1,2$, and the Bessel functions with these root multiplicities are of type A, and were studied in \cite{BCG}.

\section{Appendix C: Power series expression of Harish-Chandra integrals}
In this appendix we consider matrix integrals which appear as spherical functions of certain Euclidean type symmetric spaces, whose root systems are of type A or type BC. These so-called Harish-Chandra integral originated in the representation theory gain independent interests from physics and special functions, and were also studied intensively. 

One goal of people is to calculate out an explicit expression of these integrals. There were a large amount of literature studying this, but while different people care about Harish-Chandra integral of different forms, the explicit expressions they provide and the level of explicitness also differ a lot. In the remaining pages, we try to give a relatively systematic summary of explicit expressions of Harish-Chandra integrals related to several classical symmetric spaces, highlight their connections with some classical random matrix ensembles, and we give the expressions in terms of the symmetric polynomials. While the results and proof ingredient here are well known by experts, parts of them might not been formally published and might be helpful to readers. 

Fix $N\in \Z_{>0}$. Let $U$ denote the compact orthogonal/unitary/unitary sympletic Lie group $O(N)/U(N)/Sp(N)$, and $dU$ be the corresponding Haar measure while the parameter $\theta=\frac{1}{2}, 1, 2$.

\begin{prop}\cite[Proposition 13.4.1]{Forrester}\label{prop:integral1}

For $N\ge 2$, $\theta=\frac{1}{2}, 1, 2$, let $A=diag(a_{1},...,a_{N}), Z=diag(z_{1},...,z_{N})$,
\begin{equation}\label{eq_typeAintegral}
\begin{split}
 \int \exp(Tr(ZUAU^{-1})dU 
= \sum_{\mu}\frac{1}{H(\mu)}\frac{P_{\mu}(a_{1},...,a_{N};\theta) P_{\mu}(z_{1},...,z_{N};\theta)}{P_{\mu}(1^{N};\theta)}.
\end{split}
\end{equation}
\end{prop}

For $\theta=\frac{1}{2}, 1, 2$, (\ref{eq_typeAintegral}) corresponds by a limit transition to spherical function of compact symmetric spaces $U(N)/O(N), U(N), U(2N)/Sp(N)$ of type $A_{N-1}$, where $\theta$ is the (restricted) root multiplicity $k_{e_{i}-e_{j}}=\frac{1}{2}m_{e_{i}-e_{j}}\ (1\le i\ne j\le N)$. See \cite[Section 4]{OO1}. In probablistic context, (\ref{eq_typeAintegral}) is introduced as "characteristic function" of $N\times N$ real/complex/real quaternionic self-adjoint random matrices whose distribution is invariant under unitary conjugations. The typical examples are GOE/GUE/GSE. See \cite{GM}, \cite{BCG} for more details.

\cite[Section 4]{OV} gives a proof of Proposition \ref{prop:integral1} for the case $\theta=1$. Inspired by their approach, we provide another proof of Theorem \ref{thm:fouriertransform} following the same line.

Fix $M\le N$, $\theta=\frac{1}{2}, 1, 2$, define $$\Lambda=\begin{bmatrix}
a_{1} &  & & &&0&...&0\\
 & a_{2} & & &&0&...& 0\\
  &             &...&   &&\\
  &&&...&\\
  &&&&a_{M} &0&...&0
\end{bmatrix}_{M\times N},$$
$$Z=\begin{bmatrix}
z_{1} &  & & &\\
 & z_{2} & & &\\
  &             &...&\\
  &&&...&\\
  &&&&z_{M} &\\
  0&&...&&0\\
  && ...&&\\
  0&&...&&0
\end{bmatrix}_{N\times M},$$
$U\in O(M)/U(M)/Sp(M)$, $V\in O(N)/U(N)/Sp(N)$ are integrated under Haar measures.

\begin{lemma}\cite[Chapter I, (7.8)]{M}\label{lem:hooklength}
For $m\in \Z_{\ge 1}$, expand $p_{1}^{m}$ in terms of Schur polynomials, i.e,
\begin{equation}
p_{1}^{m}=\sum_{|\lambda|=m}C^{\lambda}_{m}S_{\lambda}.   
\end{equation}
Then \begin{equation}\label{eq_coefficient}
C^{\lambda}_{m}=\frac{m!}{\prod_{s\in \lambda}[a(s)+l(s)+1]}.    
\end{equation}
\end{lemma}

\begin{remark}
$C^{\lambda}_{m}$ can be interpreted in view of both representation theory of symmetric group and combinatoric: $C^{\lambda}_{m}=\chi_{\lambda}(1^{m})=dim_{S_{m}}(\lambda)$, the character of $S_{m}$ at identity, and it's equal to the number of standard Young tableaux of shape $\lambda$.
\end{remark}

\begin{prop}\cite[Proposition 13.4.1]{Forrester}\label{prop:integral2}
For $\theta=\frac{1}{2}, 1, 2$, 
\begin{equation}\label{eq_integralappendix}
\begin{split}
&\ \int  dU\int  dV\ \exp(Tr(U\Lambda VZ+Z^{*}V^{*}\Lambda^{*}U^{*}))\\
=&\sum_{\mu}\prod_{j=1}^{M}\frac{\Gamma(\theta N -\theta (j-1))}{\Gamma(\theta N-\theta(j-1)+\mu_{j})}\frac{1}{H(\mu)}\frac{P_{\mu}(a_{1}^{2},\cdots,a_{M}^{2};\theta)P_{\mu}(z_{1}^{2},\cdots,z_{M}^{2};\theta)}{P_{\mu}(1^{M};\theta)},  
\end{split}
\end{equation}    
where $H(\mu)$ is defined in (\ref{eq_Hmu}).
\end{prop}
\begin{remark}
\cite[Chapter 13]{Forrester} provides a different self-contained proof of Proposition \ref{prop:integral1} and \ref{prop:integral2}.
\end{remark}

\begin{proof}
Throughout this proof, for a symmetric polynomial $f(x_{1},...,x_{M})$ and a $M\times M$ matrix X, let
$f(X)$ be the value of $f$ evaluated at eigenvalues of X.

For $\theta=1$ this integral and its various generalizations were well-studied in physic literature, e.g, in  \cite{GT}, \cite{SW},\cite{GW}. \cite{GT} gives the same power series expansion as the right side, which degenerates to 
$$\sum_{\mu}\frac{\prod_{i=1}^{M}(N-i)!}{\prod_{i=1}^{M}(N-i+\mu_{i})!}\frac{\prod_{i=1}^{M}(M-i)!}{\prod_{i=1}^{M}(M-i+\mu_{i})!}S_{\mu}(a_{1}^{2},...,a_{M}^{2})S_{\mu}(b_{1}^{2},...,b_{M}^{2}),$$
and its proof relies on the well known fact that Schur polynomials $s_{\mu}(x_{1},...,x_{N})$ are the characters of $U(N)$ and $SL_{N}(\R)$.

For $\theta=\frac{1}{2}$,
\begin{align*}
\exp(Tr(U\Lambda VZ+Z^{*}V^{*}\Lambda^{*}U^{*})=\exp(Tr(2U\Lambda VZ)
=\sum_{l(\mu)\le M}\frac{C^{\mu}_{|\mu|}}{|\mu|!}S_{\mu}(2U\Lambda VZ). 
\end{align*}
By \cite[Chapter VII, Section 3, (2.23)]{M} and \cite[Chapter VI, (10.22)]{M}, 
\begin{equation}\label{eq_ortho_1}
\begin{split}
 \int S_{\mu}(U\Lambda VZ)dU
=\begin{dcases}
    0  &  \mu \;\; \text{is}\;\;\text{not}\;\;\text{even}; \\  
    \Omega_{\mu}(\Lambda VZ)=\frac{P_{\lambda}(Z^{T}V^{T}\Lambda^{T}\Lambda VZ;\frac{1}{2})}{P_{\mu}(1^{M};\frac{1}{2})} & \mu \;\; \text{is}\;\;\text{even}, \ \mu=2\lambda.
\end{dcases}    
\end{split}
\end{equation}
\begin{equation}\label{eq_ortho_2}
\begin{split}
&\ \int P_{\lambda}(Z^{T}V^{T}\Lambda^{T}\Lambda VZ;\frac{1}{2})dV=\int P_{\lambda}(ZZ^{T}V^{T}\Lambda^{T}\Lambda V ;\frac{1}{2})dV\\
=&\frac{P_{\lambda}(ZZ^{T};\frac{1}{2})P_{\lambda}(\Lambda^{T}\Lambda;\frac{1}{2})}{P_{\lambda}(1^{N};\frac{1}{2})}= \frac{P_{\lambda}(Z^{T}Z)P_{\lambda}(\Lambda\Lambda^{T})}{P_{\lambda}(1^{N};\frac{1}{2})},
\end{split}
\end{equation}
where the second equality holds by \cite[VII.4.2]{M}. 
Then
\begin{align*}
&\ \int \int \exp(Tr(U\Lambda VZ+Z^{*}V^{*}\Lambda^{*}U^{*})dUdv\\
=& \sum_{\lambda}\frac{C^{2\lambda}_{2|\lambda|}}{|\lambda|!}\frac{4^{|\lambda|}}{P_{\lambda}(1^{M};\frac{1}{2})P_{\lambda}(1^{N};\frac{1}{2})}P_{\lambda}(Z^{T}Z;\frac{1}{2})P_{\lambda}(\Lambda\Lambda^{T};\frac{1}{2})\\
=& \sum_{\lambda}\frac{C^{2\lambda}_{2|\lambda|}}{|\lambda|!}\frac{4^{|\lambda|}}{P_{\lambda}(1^{N};\frac{1}{2})}\frac{1}{P_{\lambda}(1^{M};\frac{1}{2})}P_{\lambda}(a_{1}^{2},...,a_{M}^{2};\frac{1}{2})P_{\lambda}(z_{1}^{2},...,z_{M}^{2};\frac{1}{2})\\
=& \sum_{\lambda}\frac{4^{|\lambda|}}{\prod_{s\in 2\lambda}[a(s)+l(s)+1]}\frac{\prod_{s\in \lambda}[a(s)+\frac{1}{2}l(s)+\frac{1}{2}]}{\prod_{s\in \lambda}[\frac{N}{2}+j-1-\frac{1}{2}(i-1)]}\frac{1}{P_{\lambda}(1^{M};\frac{1}{2})}P_{\lambda}(a_{1}^{2},...,a_{M}^{2};\frac{1}{2})P_{\lambda}(z_{1}^{2},...,z_{M}^{2};\frac{1}{2})\\
=& \sum_{\lambda}\prod_{i=1}^{M}\frac{\Gamma(\frac{N}{2} -\frac{1}{2} (i-1))}{\Gamma(\frac{1}{2} N-\frac{1}{2}(i-1)+\lambda_{i})}\frac{1}{\prod_{s\in \lambda}[a(s)+1+\frac{1}{2}l(s)]}\frac{P_{\lambda}(a_{1}^{2},\cdots,a_{M}^{2};\frac{1}{2})P_{\lambda}(z_{1}^{2},\cdots,z_{M}^{2};\frac{1}{2})}{P_{\lambda}(1^{M};\frac{1}{2})}
\end{align*}
where the third equality follows from Lemma \ref{lem:hooklength} and (\ref{eq_Jack(1)}).

For $\theta=2$, let $\eta$ denote the map embedding real quaternion into $M_{2\times 2}(\C)$, such that for $x=a+bi+cj+dk$, $a,b,c,d\in \R$, 
\begin{equation}
 \eta(x)=\begin{bmatrix}
a+bi &c+di\\
 -c+di& a-bi \\
\end{bmatrix}.
\end{equation}
Similarly for $\eta$ embeds $GL_{M}(Sp)$ into $GL_{2M}(\C)$ by 
\begin{equation}\eta(X)=[\eta(X_{ij})]_{1\le i,j\le M}.\end{equation}
$$\exp(Tr(U\Lambda VZ+Z^{*}V^{*}\Lambda^{*}U^{*})=\exp(Tr(\eta(U\Lambda VZ))
=\sum_{l(\mu)\le M}\frac{C^{\mu}_{|\mu|}}{|\mu|!}S_{\mu}(\eta(U\Lambda VZ)). $$
By \cite[Chapter VII, (6.13), (6.14), (6.20), Exercise 2.7 and Chapter VI, (10.22)]{M}, 
\begin{equation}\label{eq_ortho_3}
\begin{split}
 \int S_{\mu}(\eta(U\Lambda VZ))dU
=\begin{dcases}
    0  &  \mu^{'} \;\; \text{is}\;\;\text{not}\;\;\text{even}; \\  
    \Omega_{\mu}(\Lambda VZ)=\frac{P_{\lambda}(Z^{*}V^{*}\Lambda^{*}\Lambda VZ;2)}{P_{\mu}(1^{M};2)} & \mu^{'} \;\; \text{is}\;\;\text{even}, \ \mu=\lambda\cup \lambda,
\end{dcases}    
\end{split}
\end{equation}
where $*$ denotes the conjugate transpose of quaternionic matrices.
\begin{equation}\label{eq_ortho_4}
\begin{split}
&\ \int P_{\lambda}(Z^{*}V^{*}\Lambda^{*}\Lambda VZ;2)dV=\int P_{\lambda}(ZZ^{*}V^{*}\Lambda^{*}\Lambda V ;2)dV\\
=&\frac{P_{\lambda}(ZZ^{*};2)P_{\lambda}(\Lambda^{*}\Lambda;2)}{P_{\lambda}(1^{N};2)}= \frac{P_{\lambda}(Z^{*}Z;2)P_{\lambda}(\Lambda\Lambda^{*};2)}{P_{\lambda}(1^{N};2)},
\end{split}
\end{equation}
where the second equality holds by \cite[Chapter VII, Exercise 6.4]{M}. 
Then
\begin{align*}
&\ \int \int \exp(Tr(U\Lambda VZ+Z^{*}V^{*}\Lambda^{*}U^{*})dUdV\\
=& \sum_{\lambda}\frac{C^{\lambda\cup\lambda}_{|\lambda\cup\lambda|}}{|\lambda\cup\lambda|!}\frac{1}{P_{\lambda}(1^{M};2)P_{\lambda}(1^{N};2)}P_{\lambda}(Z^{*}Z;2)P_{\lambda}(\Lambda\Lambda^{*};2)\\
=& \sum_{\lambda}\frac{C^{\lambda\cup\lambda}_{|\lambda\cup\lambda|}}{|\lambda\cup\lambda|!}\frac{1}{P_{\lambda}(1^{N};2)}\frac{1}{P_{\lambda}(1^{M};2)}P_{\lambda}(a_{1}^{2},...,a_{M}^{2};2)P_{\lambda}(z_{1}^{2},...,z_{M}^{2};2)\\
=& \sum_{\lambda}\frac{1}{\prod_{s\in \lambda\cup\lambda}[a(s)+l(s)+1]}\frac{\prod_{s\in \lambda}[a(s)+2l(s)+2]}{\prod_{s\in \lambda}[2N+j-1-2(i-1)]}\frac{1}{P_{\lambda}(1^{M};2)}P_{\lambda}(a_{1}^{2},...,a_{M}^{2};2)P_{\lambda}(z_{1}^{2},...,z_{M}^{2};2)\\
=& \sum_{\lambda}\prod_{i=1}^{M}\frac{\Gamma(2N -2 (i-1))}{\Gamma(2 N-2(i-1)+\lambda_{i})}\frac{1}{\prod_{s\in \lambda}[a(s)+1+2l(s)]}\frac{P_{\lambda}(a_{1}^{2},\cdots,a_{M}^{2};2)P_{\lambda}(z_{1}^{2},\cdots,z_{M}^{2};2)}{P_{\lambda}(1^{M};2)},
\end{align*}
where the third equality again follows from Lemma \ref{lem:hooklength} and (\ref{eq_Jack(1)})
\end{proof}

\section{Appendix D: Limit transition of type BC Bessel functions}
In this section we provide a limit transition of $\B(\Vec{a},z_{1},...,z_{M};\theta)$ to a simple symmetric combination of exponents. This transition implies that in the $\theta=0$ regime, the rectangular addition $\Vec{a}\boxplus_{M,N}^{\theta}\Vec{b}$ becomes the usual convolution of the empirical measures $\frac{1}{M}\sum_{i=1}^{M}\delta_{a_{i}^{2}}$ and $\frac{1}{M}\sum_{i=1}^{M}\delta_{b_{i}^{2}}$.

\begin{prop}\label{prop:transitiontoexponent}
    Given $\Vec{a}=(a_{1}\ge a_{2}\ge ...\ge a_{M})$, take $M$ to be fixed, $N\rightarrow \infty, \theta\rightarrow 0, N\theta \rightarrow \infty$, then 
    \begin{equation}
        \B(\Vec{a},N\theta z_{1},...,N\theta z_{M};\theta, N)\longrightarrow \frac{1}{M!}\sum_{\sigma\in S_{M}}\prod_{i=1}^{M}e^{a_{i}^{2}z_{\sigma(i)}^{2}}.
    \end{equation}
\end{prop}

\begin{proof}
    This follows from a straightforward calculation. Indeed, by Proposition \ref{prop:bessel}, 
    \begin{equation}
        \begin{split}
            &\B(\Vec{a},N\theta z_{1},...,N\theta z_{M};\theta, N)=\sum_{\mu\in \mathrm{YD}}\prod_{j=1}^{l(\mu)}\frac{(N\theta)^{\mu_{j}}}{[\theta(N-j+1)]\cdots [\theta(N-j+1)+\mu_{j}-1]}\\
            \cdot &\frac{\prod_{s\in\mu}\Big[a(s)+\theta l(s)+\theta\Big]}{\prod_{s\in \mu}\Big[M\theta+(j-1)-\theta(i-1)\Big]}\cdot\frac{1}{\prod_{s\in \mu}\Big[a(s)+1+\theta l(s)\Big]}P_{\mu}(a_{1}^{2},...,a_{M}^{2};\theta)P_{\mu}(z_{1}^{2},...,z_{M}^{2};\theta).
        \end{split}
    \end{equation}
    When taking the limit in the above way, 
    $$\prod_{j=1}^{l(\mu)}\frac{(N\theta)^{\mu_{j}}}{[\theta(N-j+1)]\cdots [\theta(N-j+1)+\mu_{j}-1]}\longrightarrow 1,$$
and 
    $$\frac{1}{\prod_{s\in \mu}\Big[a(s)+1+\theta l(s)\Big]}\longrightarrow \prod_{j=1}^{l(\mu)}\frac{1}{\mu_{j}!}.$$ Also note that $a(s)+\theta l(s)+\theta$ does not go to 0 only if $a(s)=0$, and $M\theta +(j-1)-\theta(i-1)$ does not go to 0 only if $j=1$. These terms contribute 
    $$\prod_{i\ge 1}k_{i}!\cdot \frac{\Big(M-l(\mu)\Big)!}{M!}=\prod_{i \ge 0}k_{i}!\cdot \frac{1}{M!},$$
    where $k_{i}$ denotes the number of rows in $\mu$ of length $i$. And the remaining part of 
    $$\frac{\prod_{s\in\mu}\Big[a(s)+\theta l(s)+\theta\Big]}{\prod_{s\in \mu}\Big[M\theta+(j-1)-\theta(i-1)\Big]}$$
    converges to 1. Together with (\ref{eq_lln4}), we have the limit is equal to 
    $$\sum_{\mu}\frac{\prod_{i\ge 0}k_{i}!}{M!}\frac{1}{\prod_{j=1}^{l(\mu)}\mu_{j}!}m_{\mu}(a_{1}^{2},...,a_{M}^{2})m_{\mu}(z_{1}^{2},...,z_{M}^{2})$$ which is the Talor expansion of 
    $$\frac{1}{M!}\sum_{\sigma\in S_{M}}\prod_{i=1}^{M}e^{a_{i}^{2}z_{\sigma(i)}^{2}}.$$
\end{proof}

\begin{thebibliography}{ADHV}
\bibitem[A]{A} J.-P. Anker, An introduction to Dunkl theory and its analytic aspects. \emph{Analytic, Algebraic
and Geometric Aspects of Differential Equations}. Birkhauser, Cham (2017), 3-58.

\bibitem[Ak]{Ak} N. I. Akhiezer, The Classical Moment Problem and Some Related Questions in Analysis, \emph{SIAM}, (2020).

\bibitem[AP]{AP}O. Arizmendi, D. Perales, Cumulants for nite free convolution. \emph{Journal of Combinatorial
Theory, Series A} 155, (2018), 244-266. arXiv:1611.06598

\bibitem[B1]{B1} F. Benaych-Georges, Rectangular random matrices, related convolution, \emph{Probab. Theory Relat. Fields},
144, 471-515, (2009). 	arXiv:math/0507336

\bibitem[B2]{B2} F. Benaych-Georges, Rectangular R-Transform as the Limit of Rectangular Spherical Integrals, \emph{Journal of Theoretical Probability},
24, 969-987, (2011). 	arXiv:0909.0178

\bibitem[BCG]{BCG} F. Benaych-Georges, C. Cuenca, V. Gorin, Matrix Addition and the Dunkl Transform at High Temperature, \emph{Communications in Mathematical Physics},
394, 735-795, (2022). 	arXiv:2105.03795

\bibitem[BG1]{BG1}  A. Bufetov, V. Gorin. Fluctuations of particle systems determined by schur generating
functions, \emph{Advances in Math.}, 338 (2018), 702-781.

\bibitem[BG2]{BG2} A. Bufetov, V. Gorin. \emph{Fourier transform on high-dimensional unitary groups with applications to random tilings}, 	\emph{Duke Math. J.}, 168, no. 13 (2019), 2559-2649

\bibitem[D]{D} C. Dunkl, Differential-Difference Operators Associated to Reflection Groups. \emph{Transac-
tions of the American Mathematical Society}, 311, no. 1 (1989), 167-183.

\bibitem[De]{De} P. Desrosiers. "Duality in random matrix ensembles for all $\beta$." \emph{Nuclear Physics B} 817.3 (2009): 224-251.

\bibitem[DJO]{DJO} C. F. Dunkl, M. F. E. De Jeu, Singular Polynomials for Finite Reflection Groups. \emph{Transactions of the American Mathematical Society}, Vol. 346 (1994), 237-256.

\bibitem[F]{Forrester} P. J. Forrester, Log-Gases and Random Matrices, \emph{Princeton University Press}, 2010.

\bibitem[F2]{F2}P. J. Forrester, "High–low temperature dualities for the classical $\beta$-ensembles." \emph{Random Matrices: Theory and Applications} 11.04 (2022): 2250035.

\bibitem[GM]{GM} V. Gorin, A. W. Marcus, Crystallization of random matrix orbits. \emph{International Mathe-
matics Research Notices 2020}, no. 3 (2020), 883-913.

\bibitem[GuM]{GuM} C. D. Guera, R. Memin, CLT for real $\beta$-Ensembles at High Temperature, arXiv:2301.05516

\bibitem[Gri]{Gri} A. Gribinski, A theory of singular values for finite free probability. arXiv:2208.09768 (2022) 

\bibitem[GrM]{GrM} A. Gribinski, A. Marcus, A rectangular additive convolution for polynomials, \emph{Journal of Combinatorial
Theory}. arXiv:1904.11552 

\bibitem[GS]{GS} V. Gorin, Y. Sun, Gaussian 
uctuations for products of random matrices, too appear in \emph{
American Journal of Mathematics}, arXiv:1812.06532 (2018).

\bibitem[GT]{GT} A. Ghaderipoor, C. Tellambura, Generalization of some integrals over unitary matrices by character expansion of
groups, \emph{Journal of Mathematical Physics} 49, 073519, (2008).

\bibitem[GW]{GW} T. Guhr, T. Wettig, An Itzykson–Zuber‐like integral and diffusion for complex ordinary and supermatrices, 37, 6395, (1996).

\bibitem[HS]{HS} G. Heckman, H. Schlichtkrull, Harmonic analysis and special functions on symmetric spaces, \emph{Academic Press}, 1995.

\bibitem[H]{H} J. Huang, Law of Large Numbers and Central Limit Theorems by Jack Generating
Functions. \emph{Advances in Mathematics} 380 (2021), 107545. arXiv:1807.09928

\bibitem[Hel1]{Hel1} S. Helgason, Differential Geometry, Lie Groups, and Symmetric Spaces, \emph{American Mathematical Society}, (2001)

\bibitem[Hel2]{Hel2} S. Helgason, Groups and Geometric Analysis, \emph{American Mathematical Society}, (2012)

\bibitem[Ho]{Ho} A. Horn, Eigenvalues of sums of Hermitian matrices, \emph{Pacific J. Math.}, 12, 225-241, (1962).

\bibitem[Kl]{Kl}  A.A. Klyachko, Stable vector bundles and Hermitian operators, \emph{Selecta Math.} (N.S.) 4, 419-445, (1998).

\bibitem[KT]{KT} A. Knutson, T. Tao, Honeycombs and Sums of Hermitian Matrices. Notices of American
Mathematical Society 48, no. 2 (2001), 175-186.

\bibitem[KVW]{KVW} M. Kornyik, M. Voit, J. Woerner. Some Martingales Associated With Multivariate Bessel Processes. \emph{Acta Math. Hungar.} 163, 194–212 (2021).

\bibitem[M]{M} I. G. Macdonald, Symmetric functions and Hall polynomials, second edition, \emph{Oxford University
Press}, (1995)

\bibitem[MP]{MP} D.J. Marserevic, T.K. Pogany, Integral Representations for Products of Two Bessel
or Modified Bessel Functions, \emph{Mathematics} 7(10), 978 (2019). 

\bibitem[MSS]{MSS} A. Marcus, D. Spielman, N. Srivastava. Finite free convolutions of polynomials, \emph{Probab. Theory Relat. Fields} 182, 807–848 (2022). arXiv:1504.00350

\bibitem[No]{No} J. Novak, "Three lectures on free probability." \emph{Random matrix theory, interacting particle systems, and integrable systems} 65.309-383 (2014): 13. arXiv:1205.2097

\bibitem[O1]{O1} E. Opdam, Harmonic analysis for certain representations of graded Hecke algebras, \emph{Acta Math.} 175 (1995) 75–121.

\bibitem[O2]{O2} Lectures on Dunkl Operators for Real and Complex Reflection Groups, in: MSJ Memoirs, vol. 8, Math.
Soc. of Japan, 2000.

\bibitem[OO1]{OO1} A. Okounkov, G. Olshanski, Shifted Jack polynomials, binomial formula, and applications.
\emph{Mathematical Research Letters}, 4 (1997), 69-78, arXiv:q-alg/9608020.

\bibitem[OO2]{OO2} A. Okounkov, G. Olshanski, Limits of BC-type orthogonal polynomials as the number of variables goes to infinity,
\emph{Jack, Hall-Littlewood and Macdonald Polynomials (E.B.Kuznetsov and S.Sahi,eds). Amer. Math. Soc., Contemporary Math. 
} vol. 417, (2006), 281-319. 	arXiv:math/0606085.

\bibitem[OV]{OV} G. Olshanski, A. Vershik, Ergodic unitarily invariant measures on the space of infinite
Hermitian matrices, In: Contemporary Mathematical Physics. F. A. Berezin's memorial
volume. Amer. Math. Transl. Ser. 2, vol. 175 (R. L. Dobrushin et al., eds), 1996, pp.
137-175. arXiv:math/9601215

\bibitem[Ro]{Ro}  M. Roesler, Dunkl Operators: Theory and Applications. \emph{Lecture Notes in Mathematics} vol. 1817, (2003), 93-136.

\bibitem[Ro2]{Ro2} M. Roesler, A positive radial product formula for the Dunkl kernel. \emph{Transactions of the
American Mathematical Society} 355, no. 6 (2003), 2413-2438. arXiv:math/0210137.

\bibitem[RR]{RR} H. Remling, M. Roesler, Convolution algebras for Heckman–Opdam
polynomials derived from compact Grassmannians. \emph{Journal of Approximation Theory}, 197 (2015) 30–48.

\bibitem[S]{S} J. Sekiguchi, Zonal Spherical Functions on Some Symmetric Spaces. Publ. Res. Inst. Math. Sci. 12, no. 99, (1976),  455–464.

\bibitem[SO1]{SO1} S. B. Saïd, B. Ørsted, Bessel functions for root systems via the trigonometric setting, \emph{International Mathematics Research Notices}, Volume 2005, Issue 9, (2005), 551–585.

\bibitem[SO2]{SO2} S. B. Saïd, B. Ørsted, Analysis on flat symmetric spaces, \emph{Journal de Mathématiques Pures et Appliquées}, Volume 84, Issue 10, (2005), 1393-1426.

\bibitem[St]{St} R. P. Stanley, Some Combinatorial Properties of Jack Symmetric Functions. \emph{Advances
in Mathematics} 77 (1989), 76-115.

\bibitem[SW]{SW} B. Schlittgen, T. Wettig, \emph{Journal of Physics A: Mathematical and General}, 36, (2003), 3196-3201. 	arXiv:hep-th/9605110

\bibitem[V]{V} M. Voit, Freezing Limits for Beta-Cauchy Ensembles, \emph{SIGMA} 18 (2022), 69-94. arXiv:2205.08153

\bibitem[Vo]{Vo} D. Voiculescu, Limit laws for random matrices and free products, \emph{Inventiones Mathe-
maticae}, 104 (1991), 201-220.

\bibitem[VW]{VW} M. Voit, J. H. C. Woerner, Functional central limit theorems for multivariate Bessel processes in the freezing regime, \emph{Stochastic Analysis and Applications}, Vol. 39, (2021), 136-156. arXiv:1901.08390 

\bibitem[W]{W} H. Weyl, Das asymptotische Verteilungsgesetz der Eigenwerte lineare partieller Differentialgleichungen.
Mathematische Annalen, 71 (1912), 441-479.


\end{thebibliography}
\end{document}