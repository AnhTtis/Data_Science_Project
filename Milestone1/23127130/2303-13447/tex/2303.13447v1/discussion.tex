\section{Discussion}
\label{sec:discuss}

\stitle{Study limitations.} We acknowledge that the scope of our evaluation is limited, specifically targeting the graph domain in the broader landscape of data science workflows while focusing on an industrial setting. Moreover, the study participants were employed at a single company. 
While the observed practices
exist widely in industry and academia, the choice of participants inevitably impacted the generalizability of the findings due to organizational norms, policies, and infrastructures. Additional studies could explore \system's usage benefits and limitations in diverse settings.

\stitle{Documentation and usage.} Even so, the observed usage benefits and limitations of case studies highlight several opportunities for improvement. 
While participants appreciated the freedom of customizing operations that helped accommodate ephemeral and ever-evolving objectives within their exploratory analysis, there were suggestions for adding more features by default. This tension indicates the inherent trade-off between user empowerment and user-friendliness. We are conducting participatory design studies within \company to identify features that could be added as built-in functionalities of the widgets. The study also aims to identify additional components for supporting new workflows within the graph construction and serving platform. To improve the comprehension of a widget's features, we plan to implement a \emph{describe} method to explain component definitions, \ie their data types, supported interactions and their corresponding data operations, and the scope of the shared actions. 

\stitle{Authoring strategies for widget developers.} One of the key strengths of the \system framework is the capability to compose task-oriented widgets by combining components. Therefore, a possible research direction can be to investigate various component authoring strategies for widget developers --- for example, via declarative specification (\eg by leveraging the grammar of interactive graphics~\cite{satyanarayan2016vega}) or using direct manipulation-based (\eg Lyra~\cite{satyanarayan2014lyra}).
Graphileon~\cite{graphileon} is a graph-driven dashboard development environment that uses a graph database to store user-interface components (\eg Networks, Tables, Forms) in nodes. Events modeled as relations define the interactions and data flows between the UI components. Future research may explore such view composition strategies for \system widgets.

\stitle{Utilizing interaction provenance.} As highlighted in the case studies, interaction on the interface, even with specific tasks and goals, may lead to many events being recorded. As a result, the history view may become challenging to use due to perceptual scalability limitations~\cite{liu2013immens, bendre2019faster, rahman2021noah}. Future versions of \system can enhance the history view to support typical exploratory data analysis operations such as search and filter. Moreover, these widgets capture the context (where and when an interaction occurred) and scope (specific data domain, \eg graph) of interactions. Managing such interaction provenance may have additional benefits for intelligent agents that utilize interaction history.
For instance, Solas~\cite{epperson2022leveraging} may leverage rich interaction history across sessions to recommend visualizations or subsequent interactions. Future work may explore ways to complement user-driven exploration with more prescriptive guidance besides expanding on existing research on managing notebook provenance~\cite{brachmann2020your}.    

\stitle{Towards scalable data science.} The interaction-aware \system widgets can serve as the presentation layer of provenance-preserving end-to-end systems for data science~\cite{rahman2022ie, rahman2023mhcai}. However, for large-scale systems, the latency of rendering widgets remains a bottleneck. Recent work on enhancing the scalability of Vega visualization generation introduced automatic
server-side scaling via partitioning strategies~\cite{kruchten2022vegafusion}. Similar strategies can be adopted by \system. Other approaches that may be employed to redesign widgets for scale include applying classical database optimization techniques such as caching, pre-fetching, indexing, materialization, and incremental view maintenance on the server side. 

\stitle{The role of collaboration in widget design.} Since notebooks can be collaborative, \system widgets may need to accommodate shared workflows. Collaboration introduces new challenges, such as enforcing access control mechanisms, characterizing the role of users, and instrumenting conflict resolution techniques~\cite{rahman2020mixtape}, all of which indicate the possibility of interesting future research. Within the collaborative setting, another dimension is the plasticity of interfaces~\cite{passi2018trust} --- 
the perceived value of an interface may vary across stakeholders. To this end, another extension could be to equip \system widgets with cross-platform capabilities~\cite{bauerle2022symphony, rahman2023mhcai}. 