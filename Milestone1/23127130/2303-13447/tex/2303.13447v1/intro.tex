\section{Introduction}
Data science workflows are iterative wherein users often switch between multiple tools, including programming environments (\eg Jupyter Notebooks), visualization tools (\eg Tableau or PowerBI), and  spreadsheets (\eg Excel)~\cite{rahman2020leam,wongsuphasawat2019goals}.
Such back-and-forth switching results in a discontinuous workflow --- users are forced to execute repetitive \emph{glue} tasks manually
to bridge the gap between two systems during each switch~\cite{chattopadhyay2020s}.
The overhead of frequent context switching discourages users from using analysis 
tools and restricts them to working
only in code~\cite{wongsuphasawat2019goals}. 
Alspaugh et al.~\cite{futzing2019moseying} advocated for systems
that combine the expressivity of coding
platforms and the ease of use of visual analysis tools.
Computational notebooks (\eg Jupyter~\cite{kluyver2016jupyter} and Observable~\cite{observable}) embedded with interactive interfaces called \emph{widgets}~\cite{IPyWidgets}
support these objectives.
Data practitioners
employ these interfaces for auditing, exploring, and sharing data insights~\cite{wu2020b2,bauerle2022symphony,kery2020mage, rahman2020leam}. 

\begin{figure*}[!htb] 
  \centering
  \includegraphics[width=\linewidth]{figures/teasersecond_lbw_counter.png}
  \caption{Overview of \system features (counter clock-wise). (A) User instantiates a graph exploration widget from the notebook. (B) A multiple-coordinated view consisting of a graph schema and corresponding node (top) and relation (bottom) distribution components is displayed. (C) A customized widget displaying node distribution in alphabetic order --- (D) 
  user defines an initialization function \code{init()} to customize the sort order and passes it as a callback function during widget initialization. (E) User exports the widget state using the \code{export\_data()} accessor function. }
  \label{fig:teaser} 
  \Description{Overview of the Magneton features in a counter clock-wise fashion. (A) User instantiates a graph exploration widget from the notebook. (B) A multiple-coordinated view consisting of a graph schema and corresponding node (top) and relation (bottom) distribution components is displayed. (C) A customized widget displaying node distribution in alphabetic order --- (D) 
  user defines an intialization function which customizes the sort order and passes it as a callback function during widget initialization. (E) User exports the current widget state using the data export accessor function.}
\end{figure*} 


% \hkc{reusability: up until section 3-D2, I thought that reusability meant reusable component which is possible by reconstructing and customizing widget. In here, it seems to mean more easily exportable and passable to next steps?}
% \hkc{regular widgets: predefined interaction -> predefined data operations -> updated state --vs-- magneton: predefined interaction -> customizable data operations -> stored and updated state}
%\hkc{explain what is a state here?}. Here, state refers the value of the component properties.
% A back-end kernel executes the data operations and propagates the updates to the front-end interface.

However, gaps remain with respect to transparency, reusability, and customizability of user actions while using these programmable and interactive interfaces. 
Widgets operate as embeddable and lightweight interfaces with interactive components. User actions on the front-end trigger pre-defined operations called \emph{data operations} that update the widget state and re-render components accordingly. The widget state maintains the values of the front-end component properties, \eg frequency distribution corresponding to a bar chart.
Figure~\ref{fig:teaser}B displays such an interface that summarizes a graph database --- the component on the left displays the schema with node and relation types, and the two components on the right shows the corresponding node (top) and relation (bottom) distributions. Clicking a node type in the schema triggers data operations that recompute the corresponding distributions of the bar charts.
Existing widgets such as \emph{ipywidgets}~\cite{IPyWidgets} 
only maintain the most recent state and do not instrument mechanisms to track the state transitions triggered by user interactions.
Therefore, these widgets lack transparency as the history of interactions and their corresponding states are lost and, 
reusability as recovering previous states requires users to execute the interactions from scratch.  
Lack of such state management 
capabilities contribute to loss of knowledge when data science teams share information~\cite{zhang2020data} and limit the reproducibility of workflows~\cite{Kery2018InteractionsFU, kery2019towards36, rahman2022ie}. Furthermore, these interfaces lack the affordances for end-users to customize the built-in data operations. For example, users cannot override the data operation to update the sort order of the bar charts in Figure~\ref{fig:teaser}B from descending order of frequency to alphabetic order of the labels of the bars. However, existing work advocates for such on-demand customizations to ensure more flexibility in exploring diverse objectives~\cite{rahman2022ie, teddy2020chi}. 

% \danc{is \system restricted to KG exploration?}

% \sajc{yes. for now. In Section~\ref{sec:discuss} we will discuss generalizabiliy.}

% \hkc{why are these existing gaps important in KG exploration? e.g., capturing provenance is essential in KG workflow; KG analysis requires various kinds of zooming-in (by node, by relation, etc?) thus the need for customization}
% \sajc{made updates. please review}

To address these gaps, we implement \system\footnote{Similar to Magneton, a robot-like Pok\'emon, our proposed framework stitches together three objectives (Magnemite): transparency, reusability, and customization, to enable robust \emph{programmable and interactive} interfaces in computational notebooks.}, a framework for authoring programmable and interactive interfaces 
equipped with built-in state management and on-demand customization capabilities. These interfaces are developed by extending existing widgets.
Users can author their workflows in the Jupyter Notebook (Figure~\ref{fig:teaser}A) by instantiating task-specific widgets (Figure~\ref{fig:teaser}B). Users can \textbf{customize} an interaction's underlying data operation --- \eg the top bar chart in Figure~\ref{fig:teaser}C --- by writing custom code in the notebook (Figure~\ref{fig:teaser}D). 
Each widget has a built-in history view, enabling users to explore their interaction history and access corresponding states, thereby ensuring \textbf{transparency} and \textbf{reusability} of widget states. Users can export the widget state programmatically as \code{JSON} objects (Figure~\ref{fig:teaser}E). 
%We discuss the history view in Section~\ref{sec:system} ( Figure~\ref{fig:history}.) 
%We included a video demo of \system in the supplementary material. 
%Finally, users can export any widget state by interactively revisit previous actions using the history view and then leverage data accessors to export the desired information (Figure~\ref{fig:teaser}C). The exported data can then be reused in subsequent steps, other projects, or loaded into another widget for follow-up analysis. 

We implemented a suite of \system widgets
to support various tasks within an in-house knowledge graph construction and serving platform at \company, such as graph curation and knowledge integration. 
We conducted three preliminary case studies involving these tasks where  
data practitioners at \company used \system within their real-world workflows.
The interactive interfaces embedded within computational notebooks
positively impacted their experiences and helped uncover
interesting insights which remained unnoticed in their regular workflows without \system. Examples include identifying low-quality knowledge acquisition candidates and incorrect knowledge integration recommendations. 
Participants found the on-demand customization feature empowering and the ability to access interactions states helpful in managing \emph{glue} tasks between various steps within their workflows. However, participants also identified several limitations that encourage fundamental research questions related to the learning curve, composing widgets, and balancing automation with user agency. We open-source the \system framework at \url{https://github.com/megagonlabs/magneton}.
%to help accelerate the application of these widgets to data science workflows.

% The key contributions of this work are as follows: 
% \begin{itemize}
%     \item We developed \system, a framework for composing interactive interfaces within computational notebooks that enable reproducible, reusable, and customizable workflows involving graphs.
%     \item We launched contextual inquiries to motivate the design goals of \system, organized participatory design sessions for augmenting and focusing the design of \system widgets and conducted case studies on three real-world projects within an industry setting.
%     \item The analysis of the study results highlighted the benefits of \system and reported discussion on possible extensions to other tasks and workflows. The follow-up discussion highlighted potential future research opportunities spanning several disciplines within computer science.
%     \item \todo{We open-source the \system library\footnote{Link withheld for anonymity.} to help accelerate the application of transparent, reusable, and customizable widgets to data science workflows and the building of a community.}
% \end{itemize}
  
% \system combines the following principles to improve upon programmable and interactive interfaces for data science workflows:
% \begin{itemize}
%     \item \textbf{Robust interaction provenance.} Track the interaction state as well as the visualization state and ensure appropriate mapping among interaction and visualization states and data operations. 
%     \item \textbf{Reusable components.} Persist visualization states and provide mechanisms to access any states within user's current analytics session. 
%     \item \textbf{Customization of interaction outcomes.}
% Enable users to customize the data operation underlying a visual interaction to provide more flexibility in meeting their analytics goal.
% \end{itemize}
% \hkc{current principles are more about general EDA in notebooks. (for better flow)ground principles in KG workflows and mention that these principles are also applicable to any non-KG exploration workflow.}




% \begin{figure}[!htb] 
%   \centering
%   \includegraphics[width=0.8\linewidth]{figures/ds_workflow.png}
%   \caption{An iterative graph analysis workflow consisting of data loading, overviewing, navigation, and decision making.}
%   \label{fig:ds_workflow} 
%   \Description{Place holder figure.}
% \end{figure}



