\begin{abstract}
Data science workflows are human-centered processes involving on-demand programming and analysis. While programmable and interactive interfaces such as widgets embedded within computational notebooks are suitable for these workflows, they lack robust state management capabilities and do not support user-defined customization of the interactive components. The absence of such capabilities hinders workflow reusability and transparency while limiting the scope of exploration of the end-users. In response, we developed \system, a framework for authoring interactive widgets within computational notebooks that enables transparent, reusable, and customizable data science workflows. The framework enhances existing widgets to support fine-grained interaction history management, reusable states, and user-defined customizations. We conducted three case studies in a real-world knowledge graph construction and serving platform to evaluate the effectiveness of these widgets. Based on the observations, we discuss future implications of employing \system widgets for general-purpose data science workflows.


\end{abstract}

%submitted
% 


%Alternate:
% Data science workflows are human-centered processes involving on-demand programming and analysis. While programmable and interactive interfaces such as widgets embedded within computational notebooks are suitable for these workflows, they lack robust state management capabilities and do not support user-defined customization of the interactive components. The absence of such capabilities hinders the reusability and reproducibility of workflows while limiting the scope of exploration of the end-users. In response, we developed \system, a framework for authoring interactive widgets within computational notebooks that enables reusable, reproducible, and customizable data science workflows. The framework redesigns widgets to support fine-grained interaction history management, reusable states, and user-defined customizations. We conducted three case studies in a real-world knowledge graph construction and serving platform to evaluate the effectiveness of these widgets. Based on the observations, we discuss future implications of employing \system widgets for general-purpose data science workflows. 

 