\section{Case Study}
\label{sec:study} 
We conducted three case studies, each involving one participant (one female and two male), related to tasks such as knowledge curation and integration within an in-house knowledge graph construction and serving platform at \company. 
Each study lasted for about an hour. In the first phase of the study, we first provided a brief overview of the general capabilities of
the widgets. We then provided each participant with a Jupyter notebook with relevant \system widgets already imported. 
In the next phase, the participants imported their data and 
then employed the \system widgets
to accomplish their tasks. 
Finally, we asked the participants about their experience using \system widgets. 
Note that the case studies involved \company's proprietary knowledge graph in HR domain and datasets, \eg a job description corpus. Therefore, for the screenshots in the paper, we used knowledge graphs constructed from an open-source resource called O*NET~\cite{peterson1999occupational}. However, the information conveyed in the screenshots reflects the original work setting and experience of the participants. 

\stitle{Case Study Details.} Two case studies involved graph exploration for curating knowledge graphs and identifying opportunities for graph expansion. For graph exploration, the participants employed an \emph{Explorer} widget (Figure~\ref{fig:teaser}A). To explore the graph, the participants performed ($P1$ and $P2$) various interactions on the widget components. Examples include panning and zooming to understand the graph schema and selection to view node and relation distribution which provides additional details helpful for uncovering inconsistencies that require further curation. 
% Participants  performed various interactions to explore the graph. Examples include: viewing the schema and distributions, zooming into the schema to explore sub-graphs of specific nodes, and selecting specific entities of an entity type and understanding the incoming and outgoing relations, among others.
In the final case study, the participant ($P3$) performed a knowledge integration task where they focused on verifying alignment candidates extracted from a text corpus and assigning merging decisions, such as insert, ignore, and defer, by exploring candidate entities in the in-house knowledge graph. We provided an \emph{alignment-verification} widget for the verification task (see Figure~\ref{fig:alignment-verification}). The widget contains an interactive table component with alignment candidates from the text corpus and the graph in two columns and a selection menu of alignment decisions in another column. The widget also displays the context of alignment candidates --- a table component showing descriptions of an entity extracted from the text and a graph component displaying the sub-graph of the corresponding graph node. 
%\todo{We include screenshots of widgets used by $P1$ and $P2$ in the supplementary material}. 

\subsection{Study Observations}

\stitle{\system Widget Usage.} 
%All three participants found it convenient to work in the computational notebook setting. 
The interactive widget-based setup enabled the participant to stitch together different tasks in the same ecosystem, such as writing code and interactively exploring data. In their previous setting, all participants had to switch among multiple tools, such as spreadsheets, notebooks or IDEs, and Neo4j graph browser~\cite{miller2013graph}, which was cumbersome. $P3$ commented --- ``$\ldots$ writing code and visualization; this interactive and graphical feature is much better.'' $P3$ appreciated the ability to view the alignment candidates in the context of the corpus and graph (as shown by the bottom two components in Figure~\ref{fig:alignment-verification}) and make decisions more confidently. Participants ($P1$ and $P2$) were able to identify low-quality long-tail nodes and relations in the graph using the exploration widget. The multiple-coordinated views helped explore node and relation distributions --- low-frequency nodes in the graph were good candidates for further expansion. 
$P1$ commented --- ``I like the feature that you can filter the graph by selecting a node and a corresponding incoming or outgoing relation.'' 
%Moreover, The distribution component helped steer the participant's exploration --- low-frequency entities in the graph are good candidates for further expansion based on an external source.

\begin{figure*}[!htb] 
  \centering
  \includegraphics[width=0.8\linewidth]{figures/verification.png}
  \caption{Alignment verification widget.}
  \label{fig:alignment-verification} 
  \Description{Components in the alignment verification widget.}
\end{figure*}

\stitle{Impact of \system Features.} One participant ($P1$) greatly appreciated the general features of \system widgets, such as the history view, which enabled them to explore their interaction history to revisit specific states in the widget. $P1$ mentioned --- ``I wish that it (history view) were integrated within Neo4j browser''. Moreover, the data accessor feature helped participants access the data underlying a given state. $P1$ also commented that the insights obtained through the history view and data obtained through the accessor functions could be shared with other team members during team meetings.
$P2$ utilized the shared action feature to override the data operation underlying the distribution component. They employed a sort wrapper to customize the order of the bars alphabetically to access nodes that the participant was interested in quickly. For example, the participant was interested in a node whose title started with ``c'' and could not scroll and locate the node among approximately 1000 bars in the distribution component. The feature helped improve the discoverability of desired information in the presence of too many bars.
$P3$ also used the shared action feature to declutter the ``sub-graph'' component visualization within the verification widget: instead of showing the entire node neighborhood, which may seem cluttered, the participant used a filtering function --- as a wrapper of the underlying sub-graph computation data operation --- to only display the node, and it's parent. The participant characterized the experience as \emph{transient customizability}: ``When I am exploring, I am not attached to one objective. Right now, I am looking at only the parent (and the node), but later maybe I want to also view siblings $\ldots$ so it can be ephemeral. I like the transient customizability.'' 

\stitle{Limitations.} The participants reported several usability issues with the \system widgets.
$P3$ suggested adding interactions such as filtering and grouping to enable more effective exploration of the interaction states in the history view. $P2$ requested mechanisms to avoid scrolling through the entire interaction history, ``it would be helpful to add a bookmark feature similar to web browsers.''  
While the participants utilized the shared action feature to further customize data operations in widgets, they requested adding more features out-of-the-box to minimize such customization requirements wherever possible. $P2$ commented --- ``the shared action feature is helpful, but a sort could have been avoided had there been a search feature in the bar chart component.''
Moreover, to help design UDFs for overriding shared actions, the participants ($P2$ and $P3$) also requested more clarification in the form of ``standardized documentation'' to explain the components of the widget, their data types, and the supported interactions.  
%Finally, the participant ($P3$) also pointed out that these notebooks can be shared. So in a collaborative setting, \system widgets need to capture user roles, and track user-specific decision-making.






