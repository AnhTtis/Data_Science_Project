\section{Related Work}
\label{sec:related}
%\system bridges three areas of related work: data science workflows, authoring tools in interactive programming environments, and graph visualization and exploration. 

 

% \subsection{Authoring Tools within Interactive Programming Environments}
% \label{sec:related_ipe}
% \danc{This subsection talks more about tools supporting data science workflows in interactive programming environments rather than the env themselves. The second paragraph focuses on tools not specifically designed for notebooks. The title can be a bit misleading. something like "existing interactive data exploration and analysis"?}
Data practitioners often use interactive programming environments
which enable interactive exploration of data~\cite{kery2018story,rahman2022ie}. Computational notebooks are examples
of such environments commonly used by practitioners. Charting libraries such as Altair~\cite{vanderplas2018altair} and Plotly~\cite{plotly} also enable interactive visualization. However, unlike \system these libraries lack affordances for customizing the data operations and do not capture interaction history. 
B2~\cite{wu2020b2}, a dataframe wrapper with chart recommendation capabilities, captures interaction history by transforming UI interactions into dataframe operations and appending those operations in a notebook cell. The intermediate states can be reproduced by uncommenting and executing a python statement which is a cumbersome experience. Moreover, removing the notebook cell erases the interaction history. Therefore, additional version control mechanisms are required to achieve true persistence~\cite{brachmann2020your}. Additionally, the mapping from user interactions to dataframe operations are pre-defined and cannot be customized.
In contrast, \system extends widgets in computational notebooks to enable persistent interaction history and on-demand customization. While B2 is limited to tabular dataframes, \system may support other data domains as users can plug in any functions to overwrite the underlying data operation corresponding to an interaction.
\emph{mage}~\cite{kery2020mage} is another tool similar to B2 that translates interactions on interactive components within widgets to code, \eg dataframe operations, and exhibits similar limitations. Other interactive programming environments are not limited to the notebook paradigm. For example, interactions with visualizations in GUESS~\cite{adar2006guess} and Leam~\cite{rahman2020leam, griggs2021towards} are captured in a Python environment. While visualizations in these tools are programmable via declarative commands, they do not track and persist the user's interaction history, thereby impeding transparency and reusability
of user actions. Variolite~\cite{kery2017variolite} enables users to explore their previous interactions with code only. However, these bespoke tools require users to transition to and learn a new platform. 
%\hkc{reproducibility? those tools seem to allow data export} \saj{only the most recent can be exported. you cannot revisit previous states and reuse.}

% \subsection{Component-driven Frameworks for Data Science}
% \label{sec:related_component}
Frameworks such as Panel~\cite{panel},
Plotly Dash~\cite{plotlydash}, and Symphony~\cite{bauerle2022symphony} 
use independent components to create visualizations that can be used in both Jupyter notebooks and standalone
websites. However, Plotly Dash does not easily extend to custom visualizations, unlike Panel and Symphony.
\system also supports a wider range of visualization libraries. 
However, these frameworks lack support for on-demand customization of underlying data operations and do not track interaction history, both of which are supported as built-in features by \system widgets.
 Streamlit~\cite{streamlit} is another platform for generating interactive web dashboards using Python script. However, the platform prioritizes web applications rather than exploratory data science and does not focus on objectives such as transparency, reusability, and customization. Moreover, a significant learning curve is associated with learning a new platform.
% \hkc{change sent to focus on vis lib not react: '\system also support a wider range of visualization libraries by using React-based independent components.'}

%\todo{ADD hex tech. also add more constrasting discussion with the related work}

% \hide{\subsection{Graph Visualization and Exploration}}
% \label{sec:related_graph}
% \hide{A graph or a network is an abstract data type consisting of a finite set of nodes (entities) and edges (relations)~\cite{liu2018graph}. Depending on the properties of the nodes and edges, graphs can be of various types: homogeneous or heterogeneous and directed or undirected~\cite{liu2018graph, 2019_eurovis_mvnv}. Furthermore, graphs can be static or dynamic. In this paper, we focus on static \emph{Knowledge Graphs} (KGs) --- a directed heterogeneous graph with edges encoding facts among nodes (\eg Person \emph{lives\_in} City)~\cite{kg2019def}. Visualizing KGs requires displaying both graph topology (\ie structure) and attributes associated with the nodes and edges.
% We refer the readers to a survey of heterogeneous graph visualization for further details~\cite{2019_eurovis_mvnv}. Besides the node and edge properties, any graph visualization system must consider other aspects such as scale, sparsity, and visualization goal. In \system, we made such rendering decisions by referring to the usage guidelines communicated through the decades-long research conducted by the data visualization and HCI communities on defining graph layouts, encoding, and underlying data operations~\cite{liu2018graph, 2019_eurovis_mvnv, pienta2015scalable}. 
% Instead of generating visualizations from scratch, \system leverages existing graph visualization libraries. 
% These libraries provide built-in features such as
% choice of multiple layout algorithms, scaling to larger datasets, customization and extensibility for add-ons, compatibility with different setups (\eg browsers, touch screens), rendering capabilities in various formats (\eg WebGL, SVG, and CSS-based), and integration with existing graph databases such as Neo4j~\cite{miller2013graph}. We utilize an existing embeddable graph visualization library called Cytoscape.js~\cite{franz2016cytoscape} as \system embeds visualizations in Jupyter Notebook cells. Other popular libraries include
% D3.js~\cite{bostock2011d3}, G6~\cite{g6}, Neovis.js~\cite{neovis},  Sigma.js~\cite{sigmajs}, and Vivagraph.js~\cite{vivagraphjs}, among others.} 

% \hide{The primary goal of \system is to support various exploration tasks within knowledge graphs as users work on projects related to KG construction and serving. 
% To this end, we refer to existing graph task taxonomies to define various interactions representing a user's exploration goal~\cite{pretorius2014tasks, kerracher2015task, ahn2013task, lee2006task}. 
% Various bespoke graph exploration tools are available in industry~\cite{graphileon, graphistry, graphpolaris, graphxr, hume, linkurious, neobloom, neodash, yworks}. 
% These tools enable graph exploration via direction manipulation or graph querying. These tools do not support blending codes with visualizations, forcing users to use different tools for their analysis workflow.
% Graphistry~\cite{graphistry}, igraph~\cite{igraph}, and NetworkX~\cite{networkx} offer python packages and libraries for network analysis in computational notebooks such as Jupyter Notebook. Graphistry additionally supports GPU-accelerated rendering of large graph visualizations. However, the generated visualizations are static objects displayed in notebook cells lacking interactivity and are unsuitable for interactive graph exploration. 
% There are academic prototypes for general purpose and targeted graph exploration as outlined in various surveys~\cite{pienta2015scalable,liu2018graph}. 
% While these are also bespoke tools lacking integration with interactive programming environments, they inform different layouts for visualizing graphs~\cite{henry2007nodetrix, shneiderman2006network}, interactions~\cite{pienta2017facets, pienta2017vigor, chau2011apolo}, and underlying data operations~\cite{shen2006visual}. 
% \system, on the other hand, is embedded within the existing data science ecosystem, \ie in computational notebooks. In doing so, \system must address additional challenges that these prior systems do not handle --- supporting the highly iterative workflows while capturing the state of exploration, the provenance of interactions on visualizations, and on-demand customization of interactions. }




