\documentclass[10pt,twocolumn,letterpaper]{article}

\usepackage{iccv}
\usepackage{times}
\usepackage{epsfig}
\usepackage{graphicx}
\usepackage{amsmath}
\usepackage{amssymb}
\usepackage{eso-pic}

% Include other packages here, before hyperref.
\usepackage{booktabs}
% fred's own
% \usepackage[small,compact]{titlesec}
\newcommand{\PAR}[1]{\vskip4pt \noindent{\bf #1~}}

% choose either one
\usepackage[inline, shortlabels]{enumitem}
% \usepackage{enumitem}
% \usepackage{paralist}
\usepackage{soul}
\usepackage{verbatim}
\usepackage{array,calc}
\usepackage{placeins}
\usepackage{graphicx}
\usepackage{multirow}
\usepackage{wrapfig}
\usepackage{color}
\usepackage{colortbl}
\usepackage[dvipsnames]{xcolor}
\usepackage{inconsolata}
\usepackage{floatrow}
\usepackage{gensymb}
\usepackage{makecell}

% If you comment hyperref and then uncomment it, you should delete
% egpaper.aux before re-running latex.  (Or just hit 'q' on the first latex
% run, let it finish, and you should be clear).
\usepackage[pagebackref=true,breaklinks=true,letterpaper=true,colorlinks,bookmarks=false]{hyperref}
% Support for easy cross-referencing
\usepackage[capitalize]{cleveref}
\crefname{section}{Sec.}{Secs.}
\Crefname{section}{Section}{Sections}
\Crefname{table}{Table}{Tables}
\crefname{table}{Tab.}{Tabs.}

\iccvfinalcopy % *** Uncomment this line for the final submission

\def\iccvPaperID{1430} % *** Enter the ICCV Paper ID here
\def\httilde{\mbox{\tt\raisebox{-.5ex}{\symbol{126}}}}

% Pages are numbered in submission mode, and unnumbered in camera-ready
% \ificcvfinal\pagestyle{empty}\fi

\begin{document}

%%%%%%%%% TITLE
\title{Cross-Dimensional Refined Learning for Real-Time 3D Visual Perception from Monocular Video}

\author{Ziyang Hong \qquad C. Patrick Yue \\
Hong Kong University of Science and Technology \\
% Hong Kong \\
{\tt\small frederick.hong@connect.ust.hk \qquad eepatrick@ust.hk}
% For a paper whose authors are all at the same institution,
% omit the following lines up until the closing ``}''.
% Additional authors and addresses can be added with ``\and'',
% just like the second author.
% To save space, use either the email address or home page, not both
% \and
% C. Patrick Yue\\
% Institution2\\
% Hong Kong University of Science and Technology \\
% {\tt\small secondauthor@i2.org}
}
\vspace{-2.2cm}
\twocolumn[{%
\renewcommand\twocolumn[1][]{#1}%
\maketitle
\includegraphics[width=1\linewidth]{figs/teaser.pdf}
    \captionof{figure}{
        \textbf{Comparison between the proposed approach and baselines.} 
        Our model is more accurate and coherent in real time, comparing to two baseline methods with input from monocular video, Atlas~\cite{murez2020atlas} and NeuralRecon~\cite{sun2021neuralrecon} + Semantic-Heads. Real-time 3D perception efficiency $\eta_{3D}$ the higher the better. Color denotes different semantic segmentation labeling.
        \vspace{15pt}
    }\label{fig:teaser}
}]

% \maketitle
% Remove page # from the first page of camera-ready.
% \ificcvfinal\thispagestyle{empty}\fi

%%%%%%%%% ABSTRACT
\begin{abstract}
We present a novel real-time capable learning method that jointly perceives a 3D scene’s geometry structure and semantic labels.
Recent approaches to real-time 3D scene reconstruction mostly adopt a  volumetric scheme, where a truncated signed distance function (TSDF) is directly regressed.
However, these volumetric approaches tend to focus on the global coherence of their reconstructions, which leads to a lack of local geometrical detail.
To overcome this issue, we propose to leverage the latent geometrical prior knowledge in 2D image features 
by explicit depth prediction and anchored feature generation, to refine the occupancy learning in TSDF volume.
Besides, we find that this cross-dimensional feature refinement methodology can also be adopted for the semantic segmentation task.
Hence, we proposed an end-to-end cross-dimensional refinement neural network (CDRNet) to extract both 3D mesh and 3D semantic labeling in real time.
The experiment results show that the proposed method achieves state-of-the-art 3D perception efficiency on multiple datasets, which indicates the great potential of our method for industrial applications.
\end{abstract}

%%%%%%%%% BODY TEXT
\section{Introduction}
\label{sec:intro}
Recovering 3D geometry of objects or environment scenes prevails these days with the advent of the ubiquitous digitization. 
The digitization of the world where people live can not only help them better understand their environment scenes, but also enable robots to comprehend what they need to know about the world and therewith conducting assigned tasks.
Generally, with surrounding environment measurements as input, 3D reconstruction and 3D semantic segmentation are two key 3D perception techniques~\cite{dahnert2021panoptic, sun2020scalability, han2020live} in the computer vision society, which enable a wide range of applications, including digital twins~\cite{jiang2022ditto, bozic2021neural}, virtual/augmented reality (VR/AR)~\cite{sun2021neuralrecon, xie2022planarrecon}, building information modeling~\cite{mahmud2020boundary, vanegas2010building}, and autonomous driving~\cite{cao2022monoscene, li2022reconstruct}.

Tremendous research efforts have been made for 3D perception techniques. Based on the sensor types, researches on 3D perception can be divided into two main streams, namely active range sensors that capture surface geometry information and RGB cameras that capture texture with perspective projection. 
Originated from KinectFusion~\cite{newcombe2011kinectfusion}, the commodity RGB-D range sensor is used to measure depth data first and then fuse it into TSDF volume for 3D reconstruction. Although the follow-up depth-based TSDF fusion methods~\cite{weder2020routedfusion, weder2021neuralfusion, azinovic2022neural, xu2022hrbf, sommer2022gradient} achieve detailed dense reconstruction result, they suffer from the global incoherence due to the lack of sequential correlation, the tendency of noise disturbance due to redundant overlapped calculations, and the incapability of semantic deduction due to the lack of texture features.

On the other hand, as the camera-equipped smartphones become readily available with the built-in inertial measurement units, 
% researchers
recent advances have emerged to explore 3D perception with RGB camera on mobile devices. The problem of reconstructing 3D geometry with posed RGB images input only is referred as multi-view stereo (MVS).
Existing methods for MVS methods based on deep learning tend to adopt a volumetric scheme by directly regressing the TSDF volume~\cite{murez2020atlas, stier2021vortx, choe2021volumefusion, sun2021neuralrecon} either as a whole or in fragments. 
However, these volumetric schemes lack local details on the reconstructed mesh and therefore result in inferior semantic deduction based on its 3D reconstruction prediction, and besides, these volumetric learning methods extract 3D geometrical feature representation simply from the back projection of 2D image features, resulting in the mismatch to the 2D information priors for the predicted 3D reconstruction. Moreover, due to the low-latency requirement in SLAM of robotics, it is also necessary to maintain low computation overhead for 3D perception to achieve real-time capability.

These limitations motivate our key idea to utilize 2D explicit predictions to further impose feature refinement on the 3D features input, while keeping the global coherence within the fragments. Unlike these preceding learning-based volumetric works, we argue that the utilization of 2D prior knowledge coming out of explicit predictions as a latent feature refinement plays a significant role in learning the feature representation 
in 3D perception. In addition, the feature refinement brought by 2D explicit prediction can be operated within the fragment input for keeping the computation redundancy and thus overhead low, while having the global coherence by correlating different fragments to extract the target 3D mesh.

In this paper, we propose a novel framework, \textit{CDRNet}, to accomplish both 3D meshing and 3D semantic labeling tasks in real-time, with the help of cross-dimensional feature refinement.
% \newpage
\noindent
Our key contributions are as follows.
\begin{itemize}[itemsep=0pt,topsep=3pt,leftmargin=*]
    \item We propose a novel, end-to-end trainable network architecture, which cross-dimensionally refines the 3D features with the prior knowledge extracted from the explicit estimations of depths and 2D semantics.
    \item The proposed cross-dimensional refinement yields more accurate and robust 3D reconstruction and semantic segmentation results. We highlight that the explicit estimations of both depths and 2D semantics serve as efficient yet effective prior knowledge for 3D perception learning.
    \item To achieve the real-time 3D perception capability, our approach performs both geometric and semantic localized updates to the global map.
    We present a progressive 3D perception system that is capable of the real-time interaction with the monocular camera on cellphones.
\end{itemize}

%------------------------------------------------------------------------
\section{Related Work}
\label{sec:related_work}
\section{Related Work}
\label{sec:related_work}
\subsection{Co-Speech Gesture Synthesis}
The early approaches for generating co-speech gestures often involve creating linguistic rules to translate speech input into a sequence of pre-collected gesture segments, which are typically referred to as rule-based methods \cite{cassell1994rulefullbody,cassell2001beat,kipp2004gesture,kopp2006bml}. \citet{wagner2014rulereview} provide a comprehensive review of these methods. Rule-based methods produce interpretable and controllable results, but creating gesture datasets and rules requires significant effort. To alleviate the manual effort of designing rules in rule-based methods, data-driven approaches have gradually become predominant in this field. \citet{nyatsanga2023data_driven_gesture_survey} offer a thorough survey of these methods. Early data-driven approaches aim to directly learn mapping rules from data through statistical models \cite{neff2008videogesture,levine2009prosodygesture,levine2010gesturecontroller} and combine them with predefined gesture units for gesture generation. Later, the powerful modeling capability of deep neural networks makes it possible to train complex end-to-end models using raw speech-gesture data directly. One option is deterministic models, such as MLP \cite{kucherenko2020gesticulator}, CNN \cite{habibie2021videogesture}, RNN \cite{yoon2019robot,yoon2020trimodalgesture,bhattacharya2021affectivegesture,liu2022hierarchicalgesture}, and Transformer \cite{bhattacharya2021text2gestures}. Another choice is generative models, including flow-based models \cite{alexanderson2020stylegesture,ye2022styleflowgesture}, VAEs \cite{li2021audio2gesture,ghorbani2022zeroeggs}, and VQ-VAE \cite{yi2022talkshow,yazdian2022gesture2vec,liu2022vqgesturevideo}. Due to the inherent many-to-many relationship between speech and gesture, end-to-end models can generate natural-looking gestures but face challenges in ensuring content matching between speech and generated gestures \cite{yoon2022genea}. To address this issue, some neural systems aim to explicitly model both rhythm and semantics from the perspective of model structure \cite{kucherenko2021speech2properties2gestures,ao2022rhythmicgesticulator,liu2022disco} or training supervision strategy \cite{liang2022seeg}. Furthermore, hybrid systems, such as the combination of deep features and motion graphs \cite{zhou2022gesturemaster}, have been proposed to harness the advantages of different approaches. Recently, diffusion models \cite{sohldickstein2015diffusion,song2020improvedscore,ho2020ddpm} have demonstrated impressive results in image synthesis \cite{ramesh2022dalle2} and human motion generation \cite{tevet2022humanmotiondiffusion, zhang2022motiondiffuse}. Inspired by these works, our system adapts the latent diffusion model \cite{rombach2022latentdiffusion} for the co-speech gesture generation task and achieves appealing results.

\subsection{Style Control for Human Motion}
A typical approach to style control for human motion involves specifying a motion clip as a reference and transferring the reference clip's style to the source motion. This task is also known as \emph{style transfer}. Early works in motion style transfer integrate traditional machine learning techniques with manually defined features to infer motion styles \cite{hsu2005motion_style_translation,ma2010motion_style_transfer,xia2015realtime_motion_style_transfer,yumer2016spectral_motion_style_transfer}. Recently, deep learning-based methods have significantly enhanced motion quality. \citet{holden2016deepmotion} first propose a learning framework enabling motion style control through optimization in the motion manifold space. \citet{du2019stylemotioncvae} improve transfer efficiency by training a conditional VAE. \citet{mason2018few-shot_motion_style_transfer} use few-shot learning to generate stylized locomotion. \citet{aberman2020adain} employ a temporally invariant adaptive instance normalization (AdaIN) layer for target style injection, eliminating the need for paired data during training. \citet{wen2021stylemotionflow} achieve unsupervised style transfer using a flow model. \citet{jang2022motionpuzzle} introduce a method capable of controlling styles for individual body parts.

Previous co-speech gesture synthesis systems with style control can be categorized based on whether or not they require style labels. For methods needing labeled data, early works can only learn an individual style for one generator \cite{levine2010gesturecontroller,neff2008videogesture,ginosar2019stylegesture}. \citet{ahuja2022lowresource} propose a strategy that efficiently adapts the source generator to another speaker style using low-resource data. Some works learn a speaker style embedding space with labeled speaker-motion data, enabling gesture style control by sampling from this space \cite{ahuja2020stylegesture,yoon2020trimodalgesture,bhattacharya2021affectivegesture}. \citet{alexanderson2020stylegesture} aimat controlling fine-grained styles, such as gesturing speed and spatial scope, using preprocessed control signal-motion data. Their later work \cite{alexanderson2022diffusiongesture} utilizes a diffusion model for audio-driven motion synthesis, achieving label-based style control by training the model on labeled data. For methods not requiring style labels, \citet{habibie2022motionmatching} propose a motion matching framework to achieve flexible style control. Other studies achieve arbitrary style control by imitating an example given as a video \cite{liu2022hierarchicalgesture} or a motion clip \cite{ghorbani2022zeroeggs,ye2022styleflowgesture,kuriyama2022tokenizedgestures}.  In this work, we utilize a CLIP-based encoder to extract a style embedding from an arbitrary text prompt and incorporate it into the generator via an AdaIN layer, guiding the synthesis of stylized gestures. Our system supports fine-grained multimodal style prompts as opposed to label-based style control. It employs a self-supervised learning scheme and eliminates the need for labeled data. Additionally, we use an autoregressive model rather than a parallel model, making it potentially suitable for real-time applications.

%------------------------------------------------------------------------
\section{Methods}
\label{sec:methods}
\section{Methods}
\label{sec:methods}
\subsection{Preliminary}
In this section, we introduce each component of MAPSeg (\hyperref[fig2]{Fig.2}) and how MAPSeg can serve as a unified solution to centralized, federated, and test-time UDA (\hyperref[fig:overview]{Fig.1b}). We deploy MAPSeg for domain adaptative 3D segmentation of heterogeneous medical images and it consists of three components: (1) 3D masked multi-scale autoencoding for self-supervised pre-training, (2) 3D masked pseudo-labeling for domain adaptive self-training, and (3) global-local feature collaboration to fuse global and local contexts for the final segmentation task. The hybrid cross-entropy and Dice loss (\hyperref[eq:L_seg]{Eq.1}) is often adopted for regular supervised segmentation training, and we employ it as the basic component of the objective functions for MAPSeg:
\begin{equation}
    \label{eq:L_seg}
    \mathcal{L}_{seg}(\hat{y},y) = -\frac{1}{n}\sum_i\sum_jy_{i,j}\log(\hat{y}_{i,j}) -\frac{2\sum y\hat{y}+\epsilon}{\sum y+\sum \hat{y}+\epsilon}
\end{equation}
where $n$ denotes the number of pixels, $y_{i,j}$ and $\hat{y}_{i,j}$ represent the ground truth label and predicted probability for the $i$th pixel to belong to the $j$th class, and $\epsilon$ is used to prevent zero-division. 

In the following sections, notations are defined as: $x$ and $y$ indicate the original image and label of the randomly sampled local patch; $X$ and $Y$ refer to downsampled global scan and label; the subscripts $s$ and $t$ refer to the source and target domains, respectively; the superscript $M$ indicates the image is masked (\eg, $x_t^M$ refers to a masked local patch from the target domain).

\begin{figure*}
\centering
\includegraphics[width=0.85\linewidth]{./figs/fig2-11.pdf}
\caption{Components of the proposed MAPSeg framework. (a) 3D multi-scale masked autoencoding. (b) 3D masked pseudo labeling in source and target domains. (c) 3D Global-local collaboration.} 
\label{fig2}
\end{figure*}
\subsection{3D Multi-Scale Masked Autoencoder (MAE)}
In this study, we propose a 3D variant of MAE using a 3D CNN backbone (\hyperref[fig2]{Fig.2a}). The detailed configuration can be found in Appendix \cref{sec:archite}. Training is jointly performed on two image sources with identical size ($96^3$ voxels): local patches $x$ randomly sampled from the volumetric scan, and the whole scan downsampled to the same size, denoted as $X$. 
Both $x$ and $X$ are masked before feeding into the MAE: $x$ is divided into non-overlapping 3D sub-patches with size $8^3$, of which 70\% are masked out randomly based on a uniform distribution (\hyperref[fig2]{Fig.2a}); The same procedure is applied to $X$ with patch size $4^3$ since it contains a larger field-of-view (FOV). The masked versions of $x$ and $X$ are denoted as $x^M$ and $X^M$, respectively. We train the MAE encoder and decoder to reconstruct $x/X$ based on $x^M/X^M$ using mean squared error on the masked-out regions as the objective function.

\subsection{3D Masked Pseudo-Labeling (MPL)}
MPL uses a teacher-student framework which is a standard strategy in semi-/self-supervised learning~\cite{grill2020bootstrap,NIPS2017_68053af2} to provide stable pseudo labels on an unlabeled target domain during training. 
After MAE pre-training, we keep the MAE encoder $g$ and append a segmentation decoder $h$ to build the segmentation model $f=h\circ g$ (\hyperref[fig2]{Fig.2b-c}). Given an input image $x_s$ and label $y_s$ from the source domain and an input image $x_t$ from the target domain, the teacher model $f_\theta$ takes as input the target image $x_t$ and generates pseudo labels $f_\theta(x_t)$, with gradient detached. The student model $f_\phi$ is then optimized by minimizing the segmentation loss between the predictions of $x_t^M$/$x_s^M$ and $f_{\theta}(x_t)$/$y_s$, which can be formulated as:  
\begin{equation}
\label{eq:L_mpl}
\mathcal{L}_{MPL} = \mathcal{L}_{Seg}(f_{\phi}(x_t^M),f_{\theta}(x_t))+\beta\mathcal{L}_{Seg}(f_{\phi}(x_s^M),y_s)
\end{equation}
where $\beta$ is the weight of source prediction and set as 0.5. 
The teacher model's parameters $\theta$ are then updated during training via exponential moving average (EMA) based on the student model's parameters $\phi$~\cite{NIPS2017_68053af2}.

\begin{equation}
\label{eq:ema_update}
\theta_{t+1} \gets \alpha \theta_{t} + (1-\alpha)\phi_t, 
\end{equation}
where $t$ and $t+1$ indicate training iterations and $\alpha$ is the EMA update weight. For model initialized from the large-scale MAE pretraining, we set $\alpha$ as 0.999 during the first 1,000 steps and 0.9999 afterwards. For model pretrained on small-scale source and target datasets (\eg, only dozens of scans), we set $\alpha$ as 0.99 during the first 1,000 steps, 0.999 during the next 2,000 steps, and 0.9999 for the remaining training. The teacher model $f_{\theta}$ is initialized with student model's parameters $\phi$ after some warm-up training (\eg, 1,000 iterations) on the source-domain data. 

\subsection{3D Global-Local Collaboration (GLC)}
Directly applying MPL for UDA segmentation with large domain shift (\eg, cross-modality/sequence) may lead to unreliable pseudo-label and disrupt the training. Therefore, we design a GLC module (\hyperref[fig2]{Fig.2c}) to improve pseudo-labeling by leveraging the spatial global-local contextual relations induced by the inherent anatomical distribution prior in medical images. With the image encoder pretrained to extract image features at both local and global levels during multi-scale MAE, we take advantage of the global-local contextual relations by concatenating local and global semantic features in the latent space and make prediction based on the fused features. We differ from previous study~\cite{Chen_2019_CVPR} by only applying GLC on the output of the encoder $g$ instead of all layers to save computation cost and employing a different regularization to prevent segmentation decoder from predicting solely based on local features. 

In GLC, a binary mask $M$ is used to indicate the corresponding location of the local patch $x$ inside the downsampled global volume $X$. The encoder $g$ takes as input $x$ and $X$ and generates the local latent feature $\chi_{loc} = g(x)$ as well as cropped and resized global latent feature $\chi_{glo}=\mathit{upsample}(M \odot g(X))$, where $\odot$ indicates cropping $g(X)$ based on $M$ followed by upsampling to match the spatial size of $\chi_{loc}$. Therefore, segmenting a local patch $x$ can be rewritten as $f(x)=h(\chi_{loc}\oplus\chi_{glo})$, where $\oplus$ is the concatenation along channel dimension (\hyperref[fig2]{Fig.2c}). In addition, $f$ is also trained on downsampled global volume $X$ with $\mathcal{L}_{Seg}(f(X),Y)$), in which the global latent feature $g(X)$ is duplicated and $f(X) = h(g(X)\oplus g(X))$, to prevent model from solely relying on local semantic features and encourage the encoder to extract meaningful semantic features from both local and global levels.

We also add a regularization term between the $\chi_{loc}$ and $\chi_{glo}$ to maintain their similarity following~\cite{Chen_2019_CVPR}. Instead of the $\mathcal{L}_2$ regularization used in~\cite{Chen_2019_CVPR}, we maximize the cosine similarity between the $\chi_{loc}$ and $\chi_{glo}$ as:
\begin{equation}
\mathcal{L}_{cos}(x, X) = 1 - \frac{\chi_{loc}\cdot\chi_{glo}}{\max(\| \chi_{loc} \|_2, \| \chi_{glo} \|_2, \epsilon)}
\end{equation}
where $\epsilon$ is used to prevent zero-division. The loss function for GLC calculated on the source data is formulated as: 
\begin{align}
\label{eq.L_gs}
\mathcal{L}_{GLC}^{S} &= \gamma(\mathcal{L}_{Seg}(f_{\phi}(X_s),Y_s)+\mathcal{L}_{Seg}(f_{\phi}(X_s^M),Y_s))
\nonumber\\
&+\delta(\mathcal{L}_{cos}(x_s, X_s) + \mathcal{L}_{cos}(x_s^M, X_s^M))
\end{align}
where $\gamma$ and $\delta$ are the weights of the auxiliary global loss and cosine similarity, and set as $\gamma=0.05$ and $\delta= 0.025$ in our experiments. Similarly, the GLC loss is also calculated on the target data based on pseudo-label $f_{\theta}(X_t)$ and formulated as:
\begin{align}
\label{eq.L_gt}
\mathcal{L}_{GLC}^{T} &= 2\gamma\mathcal{L}_{Seg}(f_{\phi}(X_t^M),f_{\theta}(X_t)) + 2\delta\mathcal{L}_{cos}(x_t^M, X_t^M)
\end{align}
Therefore, the overall loss function of GLC is:
\begin{align}
\label{eq.L_global}
\mathcal{L}_{GLC} &= \mathcal{L}_{GLC}^{S}+\mathcal{L}_{GLC}^{T}
\end{align}
With the regular fully-supervised segmentation loss on source data $\mathcal{L}_{FSS} = \beta\mathcal{L}_{Seg}(f_{\phi}(x_s),y_s)$, where $\beta$ is defined as in \hyperref[eq:L_mpl]{Eq.2}, the overall objective function $\mathcal{L}$ for centralized UDA is formulated as:
\begin{equation}
\label{eq.L_center}
\mathcal{L} = \mathcal{L}_{FSS}+\mathcal{L}_{MPL}+\mathcal{L}_{GLC}
\end{equation}
It is clear that \hyperref[eq.L_center]{Eq.8} requires centralized and synchronous access to source and target data. In the section \hyperref[sec.fuda]{3.5} and \hyperref[sec.ttuda]{3.6}, we demonstrate how MAPSeg can be adapted to federated (decentralized and synchronous access to data) and test-time (decentralized and asynchronous access to data) UDA scenarios. 

\subsection{Extension to Federated UDA}
\label{sec.fuda}
In reality, labeled source-domain data and unlabeled target-domain data are often collected at different sites. We consider a practical scenario where a server (\eg a major hospital) hosts potentially large amount of both labeled and unlabeled scans, and distributed clients (\eg clinics or imaging sites) possess only unlabeled images. This is an under-explored scenario as FL typically assumes either fully or partially labeled data from all clients. We extend MAPSeg to solve this federated multi-target UDA problem according to the details in Algorithm 1 of Appendix \cref{sec:recipe}. Specifically, the server updates the student model $f_\phi$ by minimizing the loss for the labeled source-domain data $D_S$:
\begin{align}
    \mathcal{L}_s 
    &= \beta(\mathcal{L}_{seg}(f_\phi(x_s), y_s)+\mathcal{L}_{seg}(f_\phi(x_s^M), y_s)) \nonumber\\
    &+\gamma(\mathcal{L}_{seg}(f_\phi(X_s), Y_s)+\mathcal{L}_{seg}(f_\phi(X_s^M), Y_s)) \nonumber\\
    &+ \delta(\mathcal{L}_{cos}(x_s, X_s) + \mathcal{L}_{cos}(x_s^M, X_s^M)) \label{eq:loss_server}
\end{align}
The clients update the student model $f_\phi$ by minimizing the loss for its own unlabeled target-domain data $D_T^k$:
\begin{align}
    \mathcal{L}_u
    &= \beta(\mathcal{L}_{seg}(f_\phi(x_t^M), f_\theta(x_t))+\mathcal{L}_{seg}(f_\phi(x_t), f_\theta(x_t))) \nonumber\\
    &+ \gamma(\mathcal{L}_{seg}(f_\phi(X_t^M), f_\theta(X_t))+\mathcal{L}_{seg}(f_\phi(X_t), f_\theta(X_t))) \nonumber\\
    &+ \delta(\mathcal{L}_{cos}(x_t, X_t) + \mathcal{L}_{cos}(x_t^M, X_t^M)) \label{eq:loss_client}
\end{align}
Comparing to the centralized UDA loss (\hyperref[eq.L_center]{Eq.8}), we decompose it into two components: fully supervised loss for server training (\hyperref[eq:loss_server]{Eq.9}) and self-supervised loss for client updates (\hyperref[eq:loss_client]{Eq.10}), which avoids the need for centralized data. After each local update, each client sends the EMA teacher model parameters $\theta$ to the server for aggregation following typical federated averaging\cite{mcmahan2017communication}.

\subsection{Extension to Test-time UDA}
\label{sec.ttuda}
Test-time UDA often involves two separate stages of training, including the source-only training at one center and the target-only finetuning at another site. In the federated UDA setting, \hyperref[eq:loss_server]{Eq.9} and \hyperref[eq:loss_client]{Eq.10} are jointly used to update the server model through synchronous federated averaging after each round. We can further ease the constraint of synchronous communication between source and target sites by training $f_\phi$ on the source data using \hyperref[eq:loss_server]{Eq.9} for some (\eg 1,000) warm-up steps before distributing the model parameters $\phi$ to the target site for initializing the teacher model $f_\theta$. On the target site, $f_\theta$ provides stable pseudo-labels to guide the self-supervised training with \hyperref[eq:loss_client]{Eq.10} and is updated by the EMA of $\phi$ following \hyperref[eq:ema_update]{Eq.3}. We find that in this asynchronous setting MAPSeg still performs well on the target-domain data, albeit with a minor performance tradeoff on the source-domain data (see \hyperref[tab:testtime]{Tab.3}).

%-------------------------------------------------------------------------
\subsection{Implementation Details} \label{section:2.1}

\noindent\textbf{Model architecture and implementation.} We implement the encoder backbone $g$ using 3D-ResNet-like CNN. The segmentation decoder $h$ is adapted from DeepLabV3~\cite{chen2017rethinking}. The framework is implemented using PyTorch. More details of the model and the training procedure are provided in Appendix \cref{sec:archite} and \cref{sec:recipe}. 

\noindent\textbf{Selecting the best model.} For choosing the best model during training, some studies choose to train for fixed iterations and use the last checkpoint. On the other hand, some of the previous UDA studies~\cite{8988158,Chen_Dou_Chen_Qin_Heng_2019} face a dilemma in selecting the best model during training by validating against a hold-out portion of target-domain labels, which is unrealistic as UDA assumes full absence of target labels. We demonstrate that MPL not only provides an efficient pathway to domain adaptative segmentation but also serves as an indicator of how well the model is being adapted to the target domain. We validate the model after each epoch and the best model is selected based on the score: 
$\mathit{Score}=\mathit{Dice}_{Src}-0.5\times\overline{\mathcal{L}_{Seg}}(f_{\phi}(x_t^M),f_{\theta}(x_t))$, where $\mathit{Dice}_{Src}$ is the Dice score on source-domain validation set and $\overline{\mathcal{L}_{Seg}}(f_{\phi}(x_t^M),f_{\theta}(x_t))$ is the mean of $\mathcal{L}_{Seg}(f_{\phi}(x_t^M),f_{\theta}(x_t))$ during the last training epoch. From \hyperref[eq:L_seg]{Eq.1}, it is clear that $ \lim_{\hat{y}\to y} \mathcal{L}_{seg}(\hat{y},y)=-1$, therefore, $Score$ has an upper bound of $1.5$. We demonstrate in \hyperref[tab:cardiac]{Tab.4} that the difference between validation using target labels versus $Score$ is acceptable (81.2 vs. 80.3). Even without accessing target labels for validation, MAPSeg still surpasses the previous SOTA results that use target labels for validation. It is worth noting that we only use target labels for validation in \hyperref[tab:cardiac]{Tab.4} for a fair comparison with previously reported results; other results presented use $Score$ for validation by default. For federated and test-time UDA, $\mathit{Score} = -\overline{\mathcal{L}_{Seg}}(f_{\phi}(x_t^M),f_{\theta}(x_t))$.


%------------------------------------------------------------------------
\section{Experiments}
\label{sec:experiments}
\section{Experimental Setup}
\label{sec:experiments}
\begin{figure}[t]
    \centering 
    \hspace{-.04\columnwidth}
    \includegraphics[width=1.025\columnwidth]{results/VOC/figures/pareto_example.pdf}
    \caption{\textbf{Selecting models for evaluation.} For each configuration, we evaluate every model at every checkpoint and measure its performance across various metrics (\fone, \epg, \iou) on the validation set; \ie every point in the left graph corresponds to one model (for \bcos models optimized via the \epgloss loss at the input layer). Instead of evaluating a single model on the test set, we evaluate \emph{all Pareto-dominant} models, as indicated in the center and right plot.
    % \moritz{Did we not update the results to be consistent with this? I distinctly remember creating the plots for this. (The Pareto front here as a lot more points than those in the result figures...)}
    }
    \label{fig:pareto_example}
\end{figure}

In this section, we describe our experimental setup
and how we select the best models across metrics. {Full training details can be found in the supplement.} We evaluate across the full sweep of combinations of choices for each category, and discuss our results in \cref{sec:results}. 

\myparagraph{Datasets:} We evaluate on \voc \citeMain{everingham2009pascal} and \coco \citeMain{lin2014microsoft} for multi-label image classification. {In \cref{sec:results:waterbirds}, to understand the effectiveness of model guidance in mitigating spurious correlations, we also evaluate on the synthetically constructed Waterbirds-100 dataset \citeMain{sagawa2019distributionally,petryk2022guiding}, where landbirds are perfectly correlated with land backgrounds on the training and validation sets, but are equally likely to occur on land or water in the test set (similar for waterbirds and water). With this dataset, we evaluate model guidance for suppressing undesired features.}

\myparagraph{Attribution Methods and Architectures:} As described in \cref{sec:method:attributions}, we evaluate with \ixg \citeMain{shrikumar2017learning}, \intgrad \citeMain{sundararajan2017axiomatic}, \bcos \citeMain{bohle2022b}, and \gradcam \citeMain{selvaraju2017grad} using models with a \resnet \citeMain{he2016deep} backbone. For \intgrad, we use an \xdnn \resnet \citeMain{hesse2021fast} to reduce the computational cost, and a \bcos \resnet for the \bcos attributions. We optimize the attributions at the input and final layer\footnote{As typically used in \ixg (input) and \gradcam (final) respectively.}; for intermediate layer results, see supplement. Given the similarity of the results between \gradcam and \ixg, and since \bcos attributions performed better than \gradcam for \bcos models, we show \gradcam results in the supplement. 
All models were pretrained on \imagenet \citeMain{imagenet}, and model guidance was performed starting from a baseline model fine-tuned on the target dataset.

\myparagraph{Localization Losses:} As described in \cref{sec:method:losses}, we compare four localization losses in our evaluation: (i) \energyloss, (ii) \loneloss \citeMain{gao2022aligning,gao2022res}, (iii) \ppceloss \citeMain{shen2021human}, and (iv) \rrrloss (cf.~\cref{sec:method:losses}, \citeMain{ross2017right}).

\myparagraph{Evaluation Metrics:} As discussed in \cref{sec:method:metrics}, we evaluate both for classification and localization performance of the models. For classification, we report the F1 scores, similar results with \map scores can be found in the supplement. For localization, we evaluate using the \epg and \iou scores.

\myparagraph{Selecting the best models:} As we evaluate for two distinct objectives (classification and localization), it is non-trivial to decide which models to select during training. \Eg, a model that provides the best classification performance might provide significantly worse localization performance than a model that provides slightly lower classification performance but much better localization. Finding the right balance and deciding which of those models in fact constitutes the `better' model depends on the preference of the end user. 
Hence, instead of selecting models based on a single metric, we select the set of Pareto-dominant models \citeMain{pareto1894massimo,pareto2008maximum,backhaus1980pareto} across three metrics---F1, \epg, and \iou---for each training configuration, as defined by a combination of attribution method, layer, and loss. Specifically, as shown in \cref{fig:pareto_example}, we train for each configuration using three different choices of $\lambda_\text{loc}$, and select the set of Pareto-dominant models among all checkpoints (epochs and $\lambda_\text{loc}$). This provides a more holistic view of the general trends on the effectiveness of model guidance for each configuration.

%------------------------------------------------------------------------
\section{Conclusion}
\label{sec:conclusion}
In this paper, we proposed a light-weight volumetric method, \textit{CDRNet}, that leverages the 2D latent information about depths and semantics as the feature refinement to handle 3D reconstruction and semantic segmentation tasks effectively. We demonstrated that our method has the capability to solve 3D perception tasks in real time, and justified the significance of utilizing 2D prior knowledge when solving 3D perception tasks.
% Extensive experiments on various datasets
Experiments on ScanNet justify the 3D perception performance improvement of our method comparing to prior arts. From the application point of view, the great scalability shown by \textit{CDRNet} proves that 2D priors should not be ignored in 3D perception tasks and further enables a new paradigm to achieve real-time 3D perception on personal commodity devices, such as cellphones and tablets.

{\small
\bibliographystyle{ieee_fullname}
\bibliography{egbib}
}

\end{document}