%%
%% This is file `sample-manuscript.tex',
%% generated with the docstrip utility.
%%
%% The original source files were:
%%
%% samples.dtx  (with options: `all,proceedings,bibtex,manuscript')
%% 
%% IMPORTANT NOTICE:
%% 
%% For the copyright see the source file.
%% 
%% Any modified versions of this file must be renamed
%% with new filenames distinct from sample-manuscript.tex.
%% 
%% For distribution of the original source see the terms
%% for copying and modification in the file samples.dtx.
%% 
%% This generated file may be distributed as long as the
%% original source files, as listed above, are part of the
%% same distribution. (The sources need not necessarily be
%% in the same archive or directory.)
%%
%%
%% Commands for TeXCount
%TC:macro \cite [option:text,text]
%TC:macro \citep [option:text,text]
%TC:macro \citet [option:text,text]
%TC:envir table 0 1
%TC:envir table* 0 1
%TC:envir tabular [ignore] word
%TC:envir displaymath 0 word
%TC:envir math 0 word
%TC:envir comment 0 0
%%
%%
%% The first command in your LaTeX source must be the \documentclass
%% command.
%%
%% For submission and review of your manuscript please change the
%% command to \documentclass[manuscript, screen, review]{acmart}.
%%
%% When submitting camera ready or to TAPS, please change the command
%% to \documentclass[sigconf]{acmart} or whichever template is required
%% for your publication.
%%
%%
\documentclass[manuscript]{acmart}
\settopmatter{printacmref=false}

%%
%% \BibTeX command to typeset BibTeX logo in the docs
\AtBeginDocument{%
  \providecommand\BibTeX{{%
    Bib\TeX}}}

%% Rights management information.  This information is sent to you
%% when you complete the rights form.  These commands have SAMPLE
%% values in them; it is your responsibility as an author to replace
%% the commands and values with those provided to you when you
%% complete the rights form.
\setcopyright{acmlicensed}
\copyrightyear{2025}
\acmYear{2025}
\acmDOI{XXXXXXX.XXXXXXX}

%%
%% For managing citations, it is recommended to use bibliography
%% files in BibTeX format.
%%
%% You can then either use BibTeX with the ACM-Reference-Format style,
%% or BibLaTeX with the acmnumeric or acmauthoryear sytles, that include
%% support for advanced citation of software artefact from the
%% biblatex-software package, also separately available on CTAN.
%%
%% Look at the sample-*-biblatex.tex files for templates showcasing
%% the biblatex styles.
%%

%%
%% The majority of ACM publications use numbered citations and
%% references.  The command \citestyle{authoryear} switches to the
%% "author year" style.
%%
%% If you are preparing content for an event
%% sponsored by ACM SIGGRAPH, you must use the "author year" style of
%% citations and references.
%% Uncommenting
%% the next command will enable that style.
%%\citestyle{acmauthoryear}
\usepackage{multirow}
\usepackage{threeparttable}
\DeclareCaptionLabelFormat{appendixtable}{Appendix Table~#2}
\captionsetup[table]{labelformat=appendixtable, labelsep=period}

%%
%% end of the preamble, start of the body of the document source.
\begin{document}

%%
%% The "title" command has an optional parameter,
%% allowing the author to define a "short title" to be used in page headers.
\title{Label-Efficient Deep Learning in Medical Image Analysis: Challenges and Future Directions}

%%
%% The "author" command and its associated commands are used to define
%% the authors and their affiliations.
%% Of note is the shared affiliation of the first two authors, and the
%% "authornote" and "authornotemark" commands
%% used to denote shared contribution to the research.
\author{Cheng Jin}
\authornote{Both authors contributed equally to this research.}
\email{cheng.jin@connect.ust.hk}
\orcid{0000-0002-3522-3592}
\author{Zhengrui Guo}
\authornotemark[1]
\email{zguobc@connect.ust.hk}
\orcid{0009-0006-9920-0978}
\affiliation{%
  \institution{The Hong Kong University of Science and Technology}
  \country{Hong Kong SAR}
}

\author{Yi Lin}
\affiliation{
  \institution{Weill Cornell Medicine, New York, NY}
  \country{USA}
}
\orcid{0000-0002-7635-2518}
\email{yil4033@med.cornell.edu}

\author{Luyang Luo}
\affiliation{
  \institution{Harvard University, Cambridge, MA}
  \country{USA}
}
\orcid{0000-0002-7485-4151}
\email{luyang_luo@hms.harvard.edu}

\author{Hao Chen}
\authornote{Corresponding author.}
\affiliation{
 \institution{The Hong Kong University of Science and Technology}
 \country{Hong Kong SAR}
}
\orcid{0000-0002-8400-3780}
\email{jhc@cse.ust.hk}

\thanks{This work was supported by the National Natural Science Foundation of China (No. 62202403), Hong Kong Innovation and Technology Commission (Project No. MHP/002/22 and ITCPD/17-9), and Research Grants Council of the Hong Kong Special Administrative Region, China (Project No. R6003-22 and C4024-22GF).}

%%
%% By default, the full list of authors will be used in the page
%% headers. Often, this list is too long, and will overlap
%% other information printed in the page headers. This command allows
%% the author to define a more concise list
%% of authors' names for this purpose.
\renewcommand{\shortauthors}{Jin et al.}

%%
%% The abstract is a short summary of the work to be presented in the
%% article.
\begin{abstract}
Deep learning has significantly advanced medical imaging analysis (MIA), achieving state-of-the-art performance across diverse clinical tasks. However, its success largely depends on large-scale, high-quality labeled datasets, which are costly and time-consuming to obtain due to the need for expert annotation. To mitigate this limitation, label-efficient deep learning methods have emerged to improve model performance under limited supervision by leveraging labeled, unlabeled, and weakly labeled data. In this survey, we systematically review over 350 peer-reviewed studies and present a comprehensive taxonomy of label-efficient learning methods in MIA. These methods are categorized into four labeling paradigms: \textit{no label}, \textit{insufficient label}, \textit{inexact label}, and \textit{label refinement}. For each category, we analyze representative techniques across imaging modalities and clinical applications, highlighting shared methodological principles and task-specific adaptations. We also examine the growing role of health foundation models (HFMs) in enabling label-efficient learning through large-scale pre-training and transfer learning, enhancing the use of limited annotations in downstream tasks. Finally, we identify current challenges and future directions to facilitate the translation of label-efficient learning from research promise to everyday clinical care.
\end{abstract}

%%
%% The code below is generated by the tool at http://dl.acm.org/ccs.cfm.
%% Please copy and paste the code instead of the example below.
%%
\begin{CCSXML}
<ccs2012>
   <concept>
       <concept_id>10002944.10011122.10002945</concept_id>
       <concept_desc>General and reference~Surveys and overviews</concept_desc>
       <concept_significance>500</concept_significance>
       </concept>
   <concept>
       <concept_id>10010147.10010257.10010258</concept_id>
       <concept_desc>Computing methodologies~Learning paradigms</concept_desc>
       <concept_significance>300</concept_significance>
       </concept>
   <concept>
       <concept_id>10010147.10010257.10010321</concept_id>
       <concept_desc>Computing methodologies~Machine learning algorithms</concept_desc>
       <concept_significance>300</concept_significance>
       </concept>
   <concept>
       <concept_id>10010147.10010178.10010224</concept_id>
       <concept_desc>Computing methodologies~Computer vision</concept_desc>
       <concept_significance>300</concept_significance>
       </concept>
 </ccs2012>
\end{CCSXML}

\ccsdesc[500]{General and reference~Surveys and overviews}
\ccsdesc[300]{Computing methodologies~Learning paradigms}
\ccsdesc[300]{Computing methodologies~Machine learning algorithms}
\ccsdesc[300]{Computing methodologies~Computer vision}

%%
%% Keywords. The author(s) should pick words that accurately describe
%% the work being presented. Separate the keywords with commas.
\keywords{Medical Image Analysis, Label-Efficient Learning, Health Foundation Model.}

\received{7 March 2025}

%% This command processes the author and affiliation and title
%% information and builds the first part of the formatted document.
\maketitle

\section{Introduction}
Deep learning (DL) has revolutionized medical image analysis (MIA), significantly improving the efficiency and accuracy of disease detection, diagnosis, and treatment \cite{de2018clinically, cao2023large, vorontsov2024foundation}. By providing a data-driven framework for interpreting large and diverse medical image datasets, DL models have achieved unprecedented performance.
% transition, importance
Despite these advancements, the success of DL models remains heavily dependent on large volumes of precisely annotated data, which are costly and time-consuming to obtain due to the need for expert input \cite{yu2021convolutional}. This growing demand for annotation contrasts sharply with the limited availability of medical experts \cite{rosenkrantz2016us, lu2020national}, creating a widening gap between the increasing volume of medical images and the capacity to label them. Reducing annotation costs, accelerating annotation workflows, and alleviating the burden on annotators have thus become critical challenges in DL-based MIA.

To address the annotation bottleneck in medical imaging, researchers have proposed a variety of learning paradigms, including self-supervised, semi-supervised, weakly supervised, and active learning. These approaches are designed to handle scenarios where annotations are missing, limited, imprecise, or require refinement. By leveraging different levels of supervision, ranging from pixel-level labels to weaker forms such as points, scribbles, bounding boxes, or even unlabeled data, they provide flexibility across diverse labeling conditions. The emergence of health foundation models (HFMs) has further strengthened these strategies by pretraining on large-scale medical datasets to extract generalizable features. These features can be effectively transferred to downstream tasks such as classification, segmentation, or detection, which improves performance and reduces the need for extensive labeled data during fine-tuning \cite{he2024foundation}. In this paper, we refer to the full spectrum of these methods, both traditional and HFM-based, as \textbf{label-efficient learning}.
As illustrated in Fig. \ref{fig_class}, label-efficient learning methods have rapidly expanded in recent years. High-level tasks such as classification, segmentation, and detection remain their primary focus, while applications to low-level tasks, including denoising, image registration, and super-resolution, are also gaining momentum. This growing versatility underscores the increasing impact of label-efficient learning across the MIA pipeline, supporting its broader integration into both research and clinical workflows.

\begin{figure}[htbp]
\centering
\includegraphics[width=\textwidth]{Figures/Figs/categories.png}        
\caption{Overview of this survey. This survey categorizes approaches based on four labeling scenarios: No label (Section \ref{sec:ssl}), insufficient label (Section \ref{sec:semi}), inexact label (Section \ref{sec:mil}), and label refinement (Section \ref{sec:al}). This figure illustrates the disparity between data growth and annotator scarcity, the core techniques employed in each scenario, and trends in label-efficient learning applications. Detailed survey scope can be referred to Appendix \ref{appendix1}.}
	\label{fig_class}
\end{figure}

Several surveys have previously addressed label-efficient learning in MIA, each offering valuable insights yet exhibiting certain limitations. Cheplygina et al. \cite{cheplygina2019not} introduced the term ``not-so-supervised'' learning and categorized methods into supervised, semi-supervised, multi-instance, and transfer learning. However, their work focused on theoretical concepts and lacked practical relevance to clinical MIA applications. Budd et al. \cite{budd2021survey} emphasized human-in-the-loop strategies, while Wang et al. \cite{wang2024comprehensive} provided a more recent survey, but covered only a subset of label-efficient methods, limiting its scope. In contrast, our taxonomy is organized around specific annotation scenarios, offering a more intuitive and application-oriented framework. Furthermore, we analyze how HFMs are reshaping traditional paradigms within each scenario, highlighting open research questions that merit future exploration, as shown in Fig. \ref{fig_class}.

To provide a comprehensive overview of label-efficient learning in MIA, we review over 350 peer-reviewed studies and categorize them into four annotation scenarios: \textit{no label}, where data lacks annotations; \textit{insufficient label}, where labeled data is limited; \textit{inexact label}, where annotations are noisy or coarse; and \textit{label refinement}, where existing labels require improvement. To the best of our knowledge, this is the first extensive review to systematically cover all four scenarios. For each, we define the core challenges, provide essential background, and examine the role of HFMs in enhancing performance. Our analysis not only synthesizes recent progress but also identifies key limitations and outlines future research directions, offering a roadmap for advancing label-efficient learning in MIA. The remainder of this paper is organized as follows. Sections \ref{sec:ssl}--\ref{sec:al} introduce the four annotation scenarios: \textit{no label} in Section \ref{sec:ssl}, \textit{insufficient label} in Section \ref{sec:semi}, \textit{inexact label} in Section \ref{sec:weakly}, and \textit{label refinement} in Section \ref{sec:al}. Section \ref{sec:cnfd} discusses current challenges and explores potential solutions and research opportunities. Finally, we conclude this survey in Section \ref{sec:con}.
\section{Self-supervised Learning in MIA} \label{sec:ssl}
\begin{table*}[ht]
\centering
\caption{Surveyed Self-supervised Learning-based Studies in Medical Image Analysis.}

\resizebox{\textwidth}{.56\textwidth}{
\begin{threeparttable}
\begin{tabular}{@{}llllll@{}}
\toprule
 &Reference & Organ & Proxy Task Design & Dataset & Publication \\ \midrule
\rule{0pt}{2ex} \multirow{20}{*}{\rotatebox{90}{Classification}}&  Li \textit{et al.} \cite{li2020self} & Retina & Multi-modal Contrastive Learning  & ADAM; PALM& TMI 2020\\
\cline{2-6}
% \rule{0pt}{2.5ex}&Zhao \textit{et al.} \cite{zhao2021anomaly}  & Retina; Lung & Inpainting; Local Pixel Shuffling;  & RetinalOCT; ChestX &RetinalOCT: AUC: 0.9642; F1: 0.9342\\
% &&&Non-Linear Intensity Transformation&&ChestX: AUC: 0.8265; F1: 0.8214\rule[-1.2ex]{0pt}{0pt}\\
% \cline{2-6}
\rule{0pt}{2.5ex}&Koohbanani \textit{et al.} \cite{koohbanani2021self} & Breast;  &Magnification Prediction;  & CAMELYON 2016; &TMI 2021  \\
&&Cervix;&Solving Magnification Puzzle;&KATHER;& \\
&&Colon&Hematoxylin Channel Prediction&Private Dataset: 217 Images&\rule[-1.2ex]{0pt}{0pt}\\
\cline{2-6}
% \rule{0pt}{2.5ex}&Li \textit{et al.} \cite{li2021rotation} & Retina &Image Rotation  & ADAM; PALM; DRD & ADAM: AUC: 0.7811; PALM: AUC: 0.9912 \rule[-1.2ex]{0pt}{0pt}\\
% \cline{2-6}
\rule{0pt}{2.5ex}&Azizi \textit{et al.} \cite{azizi2021big} & Skin; Lung &Multi-Instance Contrastive Learning & Priavte Dermatology Dataset; CheXpert  &CVPR 2021\rule[-1.2ex]{0pt}{0pt}\\
\cline{2-6}
% \rule{0pt}{2.5ex}&Yang \textit{et al.} \cite{yang2022cs} & Colon & Cross-stain prediction + Contrastive Learning& KATHER& Acc: 0.918\rule[-1.2ex]{0pt}{0pt}\\
% \cline{2-6}
\rule{0pt}{2.5ex}&Tiu \textit{et al.} \cite{tiu2022expert} & Lung & Contrastive Learning  & CheXpert&Nature BME 2022\rule[-1.2ex]{0pt}{0pt}\\
\cline{2-6}
\rule{0pt}{2.5ex}&Chen \textit{et al.} \cite{chen2022scaling} & Breast; Lung; Kidney & Contrastive Learning  & TCGA-BRCA; TCGA-NSCLS; TCGA-RCC&CVPR 2022\rule[-1.2ex]{0pt}{0pt}\\
\cline{2-6}
\rule{0pt}{2.5ex}&Mahapatra \textit{et al.} \cite{mahapatra2022self} & Lymph; Lung; & Contrastive Learning Variant  & CAMELYON 2017; DRD; GGC&TMI 2022\\
&&Retina; Prostate&&\rule[-1.2ex]{0pt}{0pt}\\
\cline{2-6}
\rule{0pt}{2.5ex}&Wang \textit{et al.} \cite{wang2023ssd} & Skin & Self-supervised Knowledge Distillation  & ISIC 2019 &MIA 2023 \rule[-1.2ex]{0pt}{0pt}\\
\cline{2-6}
\rule{0pt}{2.5ex}& Huang \textit{et al.} \cite{huang2024systematic}  &  Multi-Organ   & SimCLR, MOCOv2,
SwAV, BYOL, SimSiam, DINO, BarlowTw & TissueMNIST; PathMNIST; TMED-2; AIROGS& CVPR 2024 \rule[-1.2ex]{0pt}{0pt}\\
\cline{2-6}
\rule{0pt}{2.5ex}& Tang \textit{et al.} \cite{tang2024self}  &  Lung & Self-Supervised Representation Distribution Learning & TCGA-EGFR; TCGA-Lung; Private Lung Dataset & TMI 2024 \rule[-1.2ex]{0pt}{0pt}\\
\cline{2-6}
\rule{0pt}{2.5ex}& Vorontsov \textit{et al.} \cite{vorontsov2024foundation}  & Multi-Organ & Self-distillation + Masked Image Modeling (DINOv2) & Cancer-related Diagnosis Datasets & Nat. Med. 2024 \rule[-1.2ex]{0pt}{0pt}\\
\cline{2-6}
\rule{0pt}{2.5ex}& Chen \textit{et al.} \cite{chen2024towards}  & Multi-Organ & Self-distillation + Masked Image Modeling (DINOv2) & Cancer-related Diagnosis Datasets & Nat. Med. 2024 \rule[-1.2ex]{0pt}{0pt}\\
\cline{2-6}
\rule{0pt}{2.5ex}& Lu \textit{et al.} \cite{lu2024visual}  & Multi-Organ & Visual-language Contrastive Learning + Captioning (CoCa)  & Cancer-related Diagnosis Datasets & Nat. Med. 2024 \rule[-1.2ex]{0pt}{0pt}\\
\hline
\rule{0pt}{2.5ex} \multirow{14}{*}{\rotatebox{90}{Segmentation}}& Hervella \textit{et al.} \cite{hervella2018retinal}$_{2018}$ & Retina &Multi-modal Reconstruction & Isfahan MISP & MICCAI 2018 \rule[-1.2ex]{0pt}{0pt}\\
\cline{2-6}
\rule{0pt}{2.5ex}&Spitzer \textit{et al.} \cite{spitzer2018improving}$_{2018}$ & Brain & Patch Distance Prediction   & BigBrain & MICCAI2018\rule[-1.2ex]{0pt}{0pt}\\
\cline{2-6}
\rule{0pt}{2.5ex} &  Bai \textit{et al.} \cite{bai2019self} & Heart & Anatomical Position Prediction & Private Dataset: 3825 Subjects & MICCAI 2019\rule[-1.2ex]{0pt}{0pt}\\
\cline{2-6}
\rule{0pt}{2.5ex}& Sahasrabudhe \textit{et al.} \cite{sahasrabudhe2020self}  & Multi-Organ & WSI Patch Magnification Identification &MoNuSeg & MICCAI 2020\rule[-1.2ex]{0pt}{0pt}\\
\cline{2-6}
\rule{0pt}{2.5ex}& Tao \textit{et al.} \cite{tao2020revisiting} & Pancreas &  Rubik's Cube Recovery& NIH PCT; MRBrainS18& MICCAI 2020\rule[-1.2ex]{0pt}{0pt}\\
\cline{2-6}
\rule{0pt}{2.5ex}&Lu \textit{et al.} \cite{lu2021volumetric}& Brain & Fiber Streamlines Density Map Prediction;& dHCP &MIA 2021\\
&&& Registration-based Segmentation Imitation& & \rule[-1.2ex]{0pt}{0pt}\\
\cline{2-6}
\rule{0pt}{2.5ex}&Tang \textit{et al.} \cite{tang2022self} & Abdomen; Liver; &Contrastive Learning; Masked Volume Inpainting;    & DECATHLON; &CVPR 2022\\
&&Prostate&3D Rotation Prediction &BTCV&\rule[-1.2ex]{0pt}{0pt}\\
\cline{2-6}
\rule{0pt}{2.5ex}&Jiang \textit{et al.} \cite{jiang2023anatomical} & Multi-organ &Anatomical-invariant Contrastive Learning    & FLARE 2022; BTCV &CVPR 2023\rule[-1.2ex]{0pt}{0pt}\\
\cline{2-6}
\rule{0pt}{2.5ex}&He \textit{et al.} \cite{he2023geometric} & Heart; Artery; Brain &Geometric Visual Similarity Learning    & MM-WHS-CT; ASOCA; CANDI; STOIC &CVPR 2023\rule[-1.2ex]{0pt}{0pt}\\
\cline{2-6}
\rule{0pt}{2.5ex}&Liu \textit{et al.} \cite{liu2023hierarchical} & Tooth &Hierarchical Global-local Contrastive Learning    & Private Dataset: 13,000 Scans &TMI 2023\rule[-1.2ex]{0pt}{0pt}\\
\cline{2-6}
\rule{0pt}{2.5ex}&Zheng \textit{et al.} \cite{zheng2023msvrl} & Multi-Organ &Multi-scale Visual Representation Self-supervised Learning    & BCV; MSD; KiTS &TMI 2023\rule[-1.2ex]{0pt}{0pt}\\
\cline{2-6}
\rule{0pt}{2.5ex}&Peng \textit{et al.} \cite{peng2024boundary} & Heart; Prostate & Contrastive Learning  & ACDC; PROMISE12 & MIA 2024\rule[-1.2ex]{0pt}{0pt}\\
\cline{2-6}
\rule{0pt}{2.5ex}&Purma \textit{et al.} \cite{purma2024genselfdiff} & Multi-Organ & Diffusion-based Reconstruction & Head and Neck Cancer; GlaS; MoNuSeg & TMI 2024\rule[-1.2ex]{0pt}{0pt}\\
\hline
\rule{0pt}{2.5ex} \multirow{6}{*}{\rotatebox{90}{Regression}} & Abbet \textit{et al.} \cite{abbet2020divide}  & Gland &Image Colorization & Private Dataset: 660 Images&MICCAI 2020 \rule[-1.2ex]{0pt}{0pt}\\
\cline{2-6}
\rule{0pt}{2.5ex}& \multirow{3}{*}{Srinidhi \textit{et al.} \cite{srinidhi2022self}}  & Breast; & WSI Patch Resolution Sequence & BreastPathQ; &MIA 2022\\
&& Colon& Prediction&CAMELYON 2016; &  \\
&&&&KATHER &  \rule[-2ex]{0pt}{0pt}\\
\cline{2-6}
\rule{0pt}{2.5ex}&Fan \textit{et al.} \cite{fan2023cancerself}  &  Brain; Lung  & Image Colorization; Cross-channel   &  GBM; TCGA-LUSC; NLST &TMI 2023\rule[-1.2ex]{0pt}{0pt}\\
\hline
\rule{0pt}{2.5ex} \multirow{32}{*}{\rotatebox{90}{Others}} & Zhuang \textit{et al.} \cite{zhuang2019selfsupervised} & Brain &  Rubik's Cube Recovery & BraTS 2018; Private Dataset: 1,486 Images & MICCAI 2019\rule[-1.2ex]{0pt}{0pt}\\ 
\cline{2-6}
\rule{0pt}{2.5ex}& \multirow{3}{*}{Chen \textit{et al.} \cite{chen2019self}}  & \multirow{3}{*}{Multi-Organ} & \multirow{3}{*}{Disturbed Image Context Restoration} & Private Fetus Dataset: 2,694 Images; &MIA 2019\\
&&&&Private Multi-organ Dataset: 150 Images;& \\
&&&&BraTS 2017& \rule[-1.2ex]{0pt}{0pt}\\
\cline{2-6}
\rule{0pt}{2.5ex}&Zhao \textit{et al.} \cite{zhao2020smore}  &  Brain  & Super-resolution Reconstruction   &  Private Dataset: 47 Images &TMI 2020\rule[-1.2ex]{0pt}{0pt}\\
\cline{2-6}
\rule{0pt}{2.5ex}&Li \textit{et al.} \cite{li2020sacnn} & Abdomen  & CT Reconstruction  & LDCTGC &TMI 2020\rule[-1.2ex]{0pt}{0pt}\\
\cline{2-6}
\rule{0pt}{2.5ex}&Cao \textit{et al.} \cite{cao2020auto}  & Brain & Missing Modality Synthesis & BraTS 2015; ADNI & AAAI 2020\rule[-1.2ex]{0pt}{0pt}\\ 
\cline{2-6}
\rule{0pt}{2.5ex}&Haghighi \textit{et al.} \cite{haghighi2020learning} & Lung &  Self-Discovery + Self-Classification  & LUNA; LiTS; CAD-PE; BraTS 2018; &   MICCAI 2020   \\
&&&+Self-Restoration& ChestX-ray14; LIDC-IDRI; SIIM-ACR& \rule[-1.2ex]{0pt}{0pt}\\
\cline{2-6}
\rule{0pt}{2.5ex}&Taleb \textit{et al.} \cite{taleb20203d}& Brain; Retina;  & 3D Contrastive Predictive Coding; 3D Jigsaw Puzzles; & BraTS 2018; & NeurIPS 2020\\
&&Pancreas &3D Rotation Prediction; 3D Exemplar Networks&DECATHLON;& \\
&&&Relative 3D Patch Location;&DRD& \rule[-1ex]{0pt}{0pt}\\
\cline{2-6}
\rule{0pt}{2.5ex}&Li \textit{et al.} \cite{li2021single} &  Breast; Pancreas; Kidney &Super-resolution Reconstruction;  Color Normalization  & WTS; Private Dataset: 533 Images& MIA 2021 \rule[-1.2ex]{0pt}{0pt}\\
\cline{2-6}
\rule{0pt}{2.5ex}&Wang \textit{et al.} \cite{wang2021transpath} & Multi-Organ & Contrastive Learning & TCGA; KATHER; MHIST & MICCAI 2021\\
&&&&PAIP; PatchCAMELYON& \rule[-1.2ex]{0pt}{0pt} \\
\cline{2-6}
\rule{0pt}{2.5ex}&Zhou \textit{et al.} \cite{zhou2021preservational}& Lung; Brain; Liver & Contrastive Learning + Image Reconstruction & ChestX-ray14; CheXpert; LUNA &  CVPR 2021\\
&&&&BraTS 2018; LiTS;& \rule[-1.2ex]{0pt}{0pt}\\
\cline{2-6}
\rule{0pt}{2.5ex}&Yan \textit{et al.} \cite{yan2022sam}& Multi-Organ  & Global and Local Contrastive Learning  & DeepLesion; NIH LN; Private Dataset: 94 Patients& TMI 2022\rule[-1.2ex]{0pt}{0pt}\\
\cline{2-6}
\rule{0pt}{2.5ex}&Haghighi \textit{et al.} \cite{haghighi2022dira}& Lung & Contrastive Learning + Reconstruction + & ChestX-ray14; CheXpert; & CVPR 2022 \\
&&&Adversarial Learning & Montgomery&  \rule[-1.2ex]{0pt}{0pt}\\
\cline{2-6}
\rule{0pt}{2.5ex}&Cai \textit{et al.} \cite{cai2023dualself}& Lung; Brain; Retina& Dual-Distribution Reconstruction & RSNA-Lung; LAG; VinDr-CXR; Brain Tumor MRI;& MIA 2023 \\
&& &&Private Lung Dataset: 5,000 Images& \rule[-1.2ex]{0pt}{0pt}\\
\cline{2-6}
\rule{0pt}{2.5ex}&Li \textit{et al.} \cite{li2023generic}& Retina  & Frequency-boosted Image Enhancement  &EyePACS;& MIA 2023 \\
&&&&Private Dataset: more than 10,000 Images& \rule[-1.2ex]{0pt}{0pt}\\
% \cline{2-6}
% \rule{0pt}{2.5ex}&Xie \textit{et al.} \cite{xie2022unimiss}$_{2022\text{CS}}$ & Multi-Organ & Contrastive Learning  & BCV; RICORD; &BCV: DSC: 0.8499; RICORD: AUC: 0.8906;\\
% &&&&JSRT Database; ChestXR&JSRT Database: DSC: 0.9408; ChestXR: AUC: 0.9907\rule[-1.2ex]{0pt}{0pt}\\
\bottomrule
\end{tabular}
\begin{tablenotes}    
        \footnotesize               
        \item[1] For the sake of brevity, we denote references that contain more than one task in the following abbreviations: \textbf{C}: Classification, \textbf{S}:Segmentation, \textbf{D}:Detection, \textbf{SR}: Super-resolution, \textbf{DN}: Denoising, \textbf{IT}: Image Translation, \textbf{RE}: Registration. 
      \end{tablenotes}
\end{threeparttable}
}
\label{tab:self}
\end{table*}

\subsection{Reconstruction-Based Methods}
\textbf{Reconstruction-based methods} in Self-SL focus on exploring the inherent structures of data without the help of human annotations. These methods are conducted on several tasks including super-resolution \citep{li2021single,zhao2020smore}, inpainting \citep{zhao2021anomaly}, colorization \citep{abbet2020divide}, and the emerging MIA-specific application, multi-modal reconstruction \citep{cao2020auto,hervella2018retinal}. 

A straightforward way to establish the reconstruction task is proposed by \citet{li2020sacnn}, who adopt an auto-encoder network to encode and reconstruct normal-dose computed tomography (CT) images for learning the latent features by minimizing the mean squared error (MSE) loss. After self-supervised pre-training, the encoder is utilized for feature extraction, and a supervised loss is computed with the encoded latent features. However, the self-supervised pre-training based on the minimization of reconstruction loss might neglect the basic structure of the input image and capture the color space distribution instead \citep{abbet2020divide}. More proxy tasks have been motivated to solve this challenge.

The super-resolution reconstruction task is to generate fine-grained and realistic high-resolution images by utilizing low-resolution input images. In this proxy task, the targeted model can learn the underlying semantic features and structures of data. %In , 
\citet{zhao2020smore} propose an anti-aliasing algorithm based on super-resolution reconstruction to reduce aliasing and restore the quality of magnetic resonance images (MRIs). While \citet{li2023generic} utilize the frequency information in fundus image as guidance to conduct image enhancement. 
In the meantime, super-resolution reconstruction is also an appropriate proxy task for gigapixel histopathology whole-slide images (WSIs) because low-resolution WSIs are rather easy to store and process. From this application, \citet{li2021single} conduct single image super-resolution for WSIs using GAN. 

The image colorization task is to predict the RGB version of the gray-scale images. During this process, the network is trained to capture the contour and shape of different tissues in the sample and fill them with respective colors \citep{abbet2020divide, fan2023cancerself, lin2023nuclei}. \citet{abbet2020divide} introduce the image colorization task into survival analysis of colorectal cancer. They train a convolutional auto-encoder to convert the original input image into a two-channel image, namely, hematoxylin and eosin. Then, MSE loss is applied to measure the difference between the original input image and its converted counterpart. Moreover, in the context of survival analysis,  \citet{fan2023cancerself} extend their methodology beyond image colorization to include a cross-channel pre-text task. This additional task challenges the model to restore the lightness channel in image patches, utilizing the information from their color channels.

The image inpainting task aims to predict and fill in missing parts based on the remaining regions of the input image. This proxy task allows the model to recognize the common features of identical objects, such as color and structure, and thus to predict the missing parts consistently with the rest of the image. \citet{zhao2021anomaly} propose a restoration module based on Self-SL to facilitate the anomaly detection of optical coherence tomography (OCT) and chest X-ray. It demonstrates that the restoration of missing regions facilitates the model's learning of the anatomic information.

In recent years, the multi-modal reconstruction task has emerged  \citep{cao2020auto,hervella2018retinal}. In this task, the model uses the aligned multi-modal images of a patient to reconstruct an image in one modality by taking another modality as the input. \citet{hervella2018retinal} propose this proxy task to enrich the model with joint representations of different modalities, arguing that each modality offers a complementary aspect of the object. Therefore, they take retinography and fluorescein angiography into consideration to facilitate retinal image understanding. Meanwhile,  \citet{cao2020auto} develop a self-supervised collaborative learning algorithm, aiming at learning modality-invariant features for medical image synthesis by generating the missing modality with auto-encoder and GAN.

\subsection{Context-Based Methods}
\textbf{Context-based methods} utilize the inherent context information of the input image. Recent years have witnessed attempts to design novel predictive tasks for specific MIA tasks by training the network for prediction of the output class or localization of objects with the original image as the supervision signal \citep{bai2019self,spitzer2018improving,srinidhi2022self}. \citet{bai2019self} propose a proxy task to predict the anatomical positions from cardiac chamber view planes by applying an encoder-decoder structure. This proxy task properly employs the chamber view plane information, which is available from cardiac MR scans easily. While \citet{zheng2023msvrl} aims to perform finer-grained representation and deal with different target scales by designing a multi-scale consistency objective to boost medical image segmentation. Further advancements in proxy tasks for 3D medical images are presented by \citet{he2023geometric}. They propose a novel paradigm, termed Geometric Visual Similarity Learning, which integrates a topological invariance prior into the assessment of inter-image similarity. This approach aims to ensure consistent representation of semantic regions.
In addition, \citet{srinidhi2022self} propose an MIA-specific proxy task, Resolution Sequence Prediction, which utilizes the multi-resolution information contained in the pyramid structure of WSIs. A neural network is employed to predict the order of multi-resolution image patches out of all possible sequences that can be generated from these patches. In this way, both contextual structure and local details can be captured by the network at lower and higher magnifications, respectively. 

Other efforts have been made to explore the spatial context structure of input data, such as the order of different patches constituting an image, or the relative position of several patches in the same image, which can provide useful semantic features for the network. \citet{chen2019self} focus on the proxy task, dubbed context restoration, of randomly switching the position of two patches in a given image iteratively and restoring the original image. During this process, semantic features can be learned in a straightforward way. Instead of concentrating on the inherent intensity distribution of an image, \citet{li2021rotation} aims to improve the performance of a network with rotation angle prediction as the proxy task. The input retinal images are first augmented, generating several views, then randomly rotated. %in angles \{0°, 90°, 180°, 270°\}
The model is encouraged to predict the rotation angle and cluster the representations with similar features. More advanced proxy tasks such as Jigsaw Puzzles \citep{freeman1964apictorial} and Rubik's Cube \citep{korf1985macro} are also attracting an increasing number of researchers. \citet{taleb2021multimodal} improve the Jigsaw Puzzle task with multi-modal data. Concretely, an input image is constituted of out-of-order patches of different modalities and the model is expected to restore the original image. Rubik's Cube is a task set for 3-dimensional data. \citet{zhuang2019selfsupervised} and \citet{tao2020revisiting} introduce Rubik's Cube into the MIA area, and significantly boost the performance of a deep learning model on 3D data. In this method, the 3D volume will first be cut into a grid of cubes and a random rotation operation will be conducted on these cubes. The aim of this proxy task is to recover the original volume. 

However, for histopathology images, common proxy tasks such as prediction of the rotation or relative position of objects may only provide minor improvements to the model in histopathology due to the lack of a sense of global orientation in WSIs \citep{graham2020dense,koohbanani2021self}. Therefore, \citet{koohbanani2021self} propose proxy tasks targeted at histopathology, namely, magnification prediction, solving magnification puzzle, and hematoxylin channel prediction. In this way, their model can significantly integrate and learn the contextual, multi-resolution, and semantic features inside the WSIs.

\subsection{Contrastive-Based Methods}
\textbf{Contrastive-based methods} are based on the idea that the learned representations of different views of the same image should be similar and those of different images should be clearly distinguishable. Intriguingly, the ideas behind several high-performance algorithms such as SimCLR \citep{chen2020simple} and BYOL \citep{grill2020bootstrap} have been incorporated into the MIA field \citep{azizi2021big,wang2021transpath}. Multi-Instance Contrastive Learning (MICLe), is proposed by  \citet{azizi2021big}, is a refinement and improvement of SimCLR. Instead of using one input to generate augmented views for contrastive learning, they propose to minimize the disagreement of several views from multiple input images of the same patient, creating of more positive pairs. Meanwhile, \citet{wang2021transpath} adopt the BYOL architecture to facilitate histopathology image classification. A contribution of their work was to collect the currently largest WSI dataset for Self-SL pre-training. It includes 2.7 million patches cropped from 32,529 WSIs covering over 25 anatomic sites and 32 classes of cancer subtypes. Similarly, \citet{ghesu2022self} develop a contrastive learning and online clustering algorithm based on over 100 million radiography, CT, MRI, and ultrasound images. By leveraging this large unlabeled dataset for pre-training, the performance and convergence rate of the proposed model show a significant improvement over the state-of-the-art. Another line of work that utilizes large-scale unsupervised dataset is \citep{nguyen2023lvm}, in which over 1.3 million multi-modal data from 55 publicly available datasets are integrated. In addition to considering different perspectives of the same input, \citet{jiang2023anatomical} introduce a contrastive objective for the learning of anatomically invariant features. This approach is designed to fully exploit the inherent similarities in anatomical structures across diverse medical imaging volumes.

Further studies take into account the global and local contrast for better representation learning. Their methods usually minimize the InfoNCE loss \citep{oord2018representation} to capture the global and local level information. In \citep{yan2022sam}, the authors implement the InfoNCE by encoding each pixel of the input image. Their goal is to generate embeddings that can precisely describe the anatomical location of that pixel. To achieve this, they develop a pixel-level contrastive learning framework to generate embeddings at both the global and local level. Further, \citet{liu2023hierarchical} propose a hierarchical contrastive learning objective to capture the unsupervised representation of intra-oral mesh scans from point-level, region-level, and cross-level.

\subsection{Hybrid Methods}
Studies have made efforts to combine some or all of the different types of Sefl-SL methods into a universal framework to learn latent representations from multiple perspectives, such as semantic features and structure information inside unlabeled data \citep{haghighi2020learning,tang2022self,yang2022cs,zhou2021preservational}. For instance, \citet{tang2022self} combine masked volume inpainting, contrastive coding, and image rotation tasks into a Swin Transformer encoder architecture for medical image segmentation.

\subsection{Discussion}
Self-SL methods aim to learn and obtain a model with prior knowledge by manipulating and exploiting unlabeled data. The key to the superior performance of Self-SL models is the design of proxy tasks. Numerous existing Self-SL methods directly adopt proxy tasks prevailing in natural image processing into the MIA field. However, the unique properties of medical images, such as CT, WSI, and MRI, should be exclusively considered and injected into the design process of proxy tasks. The medical field has witnessed pioneering research efforts, exemplified by \citet{zhang2023dive}, that aim to establish guidelines for the design of Self-SL proxy tasks.
Further, proxy task design based on the combination of different medical image modalities is a prospective research direction, during which the model can capture disentangled features of each modality, leading to a robust pre-trained network. For example, large vision-language pre-trained models \citep{park2023self,zhou2022generalized,zhou2023advancing} are emerging in chest X-ray and obtaining ever-increasing research interests.
\section{Insufficient Label} \label{sec:semi}
The \textit{insufficient label} scenario arises when only a small portion of available medical imaging data is annotated, while the majority remains unlabeled. This scenario is prevalent in clinical settings, where expert annotations are costly and time-consuming, yet raw images are readily accessible. In such cases, supervised learning alone proves inadequate due to limited labeled data. 
\begin{figure}[htbp]
	\centering
\includegraphics[width=0.49\textwidth]{./Figures/Figs/semi-workflow.png}
	\caption{Overview of semi-supervised learning paradigm. Semi-SL includes a small set of labeled data and a large amount of unlabeled data to conduct learning jointly, aiming at leveraging the unlabeled data to boost learning performance. Semi-SL typically seeks to optimize the combination of a supervised loss function $\mathcal{L}_{sup}$ and an unsupervised loss function $\mathcal{L}_{unsup}$.  }
	\label{fig_semi_schematic}
\end{figure}
As illustrated in Fig.~\ref{fig_semi_schematic}, \textbf{semi-supervised learning (Semi-SL)} addresses this challenge by leveraging both labeled and unlabeled data during training. A core principle of Semi-SL is supervision propagation, which assumes that unlabeled data should have predictions consistent with labeled data. In this way, Semi-SL enables better generalization while reducing the need for manual annotation.

In the following, we categorize existing Semi-SL methods in MIA how each method enforces or approximates supervision consistency between labeled and unlabeled data into three categories: \textbf{proxy-labeling}, \textbf{generative modeling}, and \textbf{regularization}. Representative works are summarized in Appendix Tab.~\ref{tab:semi}.

\subsection{Proxy-labeling Methods}
\textbf{Proxy-labeling methods} utilize the idea of supervision consistency propagation by assigning pseudo labels to unlabeled samples and incorporate high-confidence examples into the training process through an iterative approach. These methods can be divided into two principal subcategories: \textit{Self-training methods} and \textit{multi-view learning methods}.

\subsubsection{Self-training Methods}

\textit{Self-training methods} operate through the bootstrapping mechanism. Initially, a prediction function $f_{\theta}$ with parameters $\theta$ is trained using available labeled data samples $x\in X_L$. Subsequently, this trained model generates predictions for unlabeled data samples $x\in X_U$. A confidence threshold $\tau$ is established, and sample-label pairs $(x, \mathrm{argmax}{f_{\theta}(x)})$ whose prediction confidence exceeds $\tau$ are added to the labeled dataset $X_L$. This augmented labeled dataset is then used to retrain the prediction function, creating an iterative cycle that continues until the model can no longer make sufficiently confident predictions on remaining unlabeled data.

Entropy minimization \cite{grandvalet2004semi} represents a foundational approach in this category, regularizing models based on the low-density assumption by encouraging low-entropy predictions for unlabeled data. Building on this concept, \textbf{Pseudo-label} \cite{lee2013pseudo} provides a straightforward yet effective self-training mechanism that applies entropy minimization principles in prediction space. While labeled samples undergo supervised training, unlabeled data receive labels corresponding to the model's most confident predictions. In medical image analysis, Pseudo-label has been widely employed as an auxiliary component to enhance model performance across various applications \cite{fan2020inf,zhang2022boostmis,chaitanya2023local}.

A significant challenge with proxy labels is their inherent noise and potential deviation from ground truth. To address this limitation, researchers have developed quality assurance mechanisms including uncertainty-aware confidence evaluation \cite{wang2021semiself}, conditional random field-based proxy label refinement \cite{bai2017semi}, and adversarial training-based methods \cite{zhou2019collaborative}. These approaches help ensure that proxy labels provide reliable supervisory signals during training. Pseudo-label has also been used in MIA to refine a given annotation with the assistance of unlabeled data. Qu \textit{et al.} \cite{qu2020weakly} introduce pseudo-label into nuclei segmentation and design an iterative learning algorithm to refine the background of weakly labeled images where only nuclei are annotated, leaving large areas ignored. Similar ideas can also be seen in \cite{nie2018asdnet}.


\subsubsection{Multi-view learning methods}
\textit{Multi-view learning methods} assume that each sample has two or multiple complementary views and features of the same sample extracted with different views are supposed to be consistent. Therefore, the key idea of multi-view learning methods is to train the model with multiple views of the sample or train multiple learners and minimize the disagreement between them, thus learning the underlying features of the data from multiple aspects. \textbf{Co-training} is a method that falls into this category. It assumes that data sample $x$ can be represented by two views, $\textbf{v}_1(x)$ and $\textbf{v}_{2}(x)$, and each of them are capable of solely training a good learner, respectively. Consequently, the two learners are set to make predictions of each view's unlabeled data, and iteratively choose the candidates with the highest confidence for the other model \cite{yang2021survey}. 
Another variation of multi-view learning methods is Tri-training \cite{zhou2005tri}, which is proposed to tackle the lack of multi-view data and mistaken labels of unlabeled data produced by self-training methods. Tri-training aims to learn three models from three different training sets obtained with bootstrap sampling. A deep learning version of Tri-training, i.e. Tri-Net, has been further proposed in \cite{dong2018tri}.

Co-training, or deep co-training, is dominant in multi-view learning in MIA, with a steady flow of publications \cite{zhao2019multi,zhou2019semi,xia2020uncertainty,wang2021selfco,fang2020dmnet,zeng2023pefat}. To conduct whole brain segmentation, Zhao \textit{et al.} \cite{zhao2019multi} implements co-training by obtaining different views of data with data augmentation. A similar idea can be seen for 3D medical image segmentation in \cite{xia2020uncertainty} and \cite{zhou2019semi}. These two works both utilize co-training by learning individual models from different views of 3D volumes such as the sagittal, coronal, and axial planes. Further works have been proposed to refine co-training. To produce reliable and confident predictions, Wang \textit{et al.} \cite{wang2021selfco} develops a self-paced learning strategy for co-training, forcing the network to start with the easier-to-segment regions and transition to the difficult areas gradually. Rather than discarding samples with low-quality pseudo-labels, Zeng \textit{et al.} \cite{zeng2023pefat} introduces a novel regularization approach, which focuses on extracting discriminative information from such samples by injecting adversarial noise at the feature level, thereby smoothing the decision boundary.
Meanwhile, to avoid the errors of different model components accumulating and causing deviation, Fang and Li \cite{fang2020dmnet} develop an end-to-end model called difference minimization network for medical image segmentation by conducting co-training with an encoder shared by two decoders.

\subsection{Generative Modeling Methods}
While proxy-labeling methods directly assign labels to unlabeled data, \textbf{generative modeling methods} realize supervision consistency propagation by assuming that both labeled and unlabeled data are sampled from a shared latent distribution. By learning this underlying distribution with the help of unlabeled data, these methods enable the model to transfer information across the entire dataset. The learned latent representation is then combined with supervised information from labeled examples to further improve performance.

% Concise version
\textbf{Generative adversarial networks (GANs)} effectively leverage both labeled and unlabeled data through a two-player minimax game between a generator $\mathcal{G}$ and discriminator $\mathcal{D}$ \cite{goodfellow2014generative}. In medical image analysis, several semi-supervised approaches incorporate unlabeled data during adversarial training. Chaitanya \textit{et al.} \cite{chaitanya2021semi} and Hou \textit{et al.} \cite{hou2022semi} utilize unlabeled samples to introduce greater variation in shape and intensity, enhancing model robustness. Zhou \textit{et al.} \cite{zhou2019collaborative} generate pseudo lesion masks for unlabeled data with quality facilitated by the discriminator. Other researchers modify the discriminator's objective beyond binary classification: Odena \textit{et al.} \cite{odena2016semi} extend it to predict $K$ classes plus an additional real/fake class, allowing unlabeled data to contribute to multi-class discrimination. This architecture has been successfully applied to retinal image synthesis \cite{kamran2021vtgan, diaz2019retinal, xie2023fundus}, glaucoma assessment \cite{diaz2019retinal}, chest X-ray classification \cite{madani2018semi}, and other medical imaging tasks \cite{hou2022semi}.

\textbf{Variational autoencoders (VAEs)} offer another effective approach for utilizing unlabeled data. Based on Bayesian inference theory \cite{kingma2013auto}, VAEs encode data into latent variables and reconstruct inputs by maximizing the variational lower bound. In medical image analysis, VAEs primarily learn feature similarities from large unlabeled datasets, creating well-constrained latent spaces that reduce dependence on labeled data \cite{sedai2017semi, wang2022rethinking}. Sedai \textit{et al.} \cite{sedai2017semi} proposed a dual-VAE framework for semi-supervised optic cup segmentation in retinal images, where one VAE learns data distribution from unlabeled data and transfers this knowledge to a second VAE performing segmentation with labeled data. Wang \textit{et al.} \cite{wang2022rethinking} adapted VAEs for 3D medical image segmentation by replacing the conventional mean vector and variance vector with a mean vector and covariance matrix, accounting for correlations between different slices of an input volume.

More recently, \textbf{diffusion models} \cite{ho2020denoising,yang2022diffusion,rombach2022high} have emerged as powerful alternatives in the generative Semi-SL domain. These models offer enhanced stability and sample quality through iterative denoising processes, showing potential in midline shift quantification \cite{gong2023diffusion} and medical image segmentation \cite{liu2024diffrect}. Their ability to model complex anatomical structures while enabling uncertainty quantification makes them particularly valuable for label-scarce scenarios.


\subsection{Regularization-based Methods}
In contrast to the explicit labeling approach of proxy methods and the distribution modeling of generative techniques, regularization-based methods enforce consistency through direct constraints on the model's behavior by assuming that the perturbation of data points does not change the prediction of the model, without requiring any label information. 

$\boldsymbol{\Pi}$\textbf{-model} \cite{sajjadi2016regularization} effectively implements consistency regularization by using a shared encoder to process differently augmented views of the same input and enforcing consistent predictions across these views, while incorporating label information to improve classifier performance. Li \textit{et al.} \cite{li2018semipi} achieved state-of-the-art skin lesion segmentation using this approach with only 300 labeled images, outperforming fully-supervised methods that required 2,000 labeled images. Similar consistency-based approaches appear in Bortsova \textit{et al.} \cite{bortsova2019semi}, who enforce prediction consistency across transformations for chest X-ray segmentation, and Meng \textit{et al.} \cite{meng2023dual}, who employ graph convolution networks to maintain regional and marginal consistency for semi-supervised optic disc and cup segmentation.

\textbf{Temporal ensembling} \cite{laine2016temporal} improves the $\Pi$-model's prediction stability by incorporating exponential moving averages, an approach widely adopted in medical image analysis \cite{cao2020uncertainty,gyawali2019semi,shi2020graph,luo2020deep}. For breast mass segmentation, Cao \textit{et al.} \cite{cao2020uncertainty} integrate uncertainty maps as guidance to ensure prediction reliability. Similarly, Luo \textit{et al.} \cite{luo2020deep} propose uncertainty-aware temporal ensembling for chest X-ray screening with partially labeled data. Gyawali \textit{et al.} \cite{gyawali2019semi} enhance the method by first using a VAE to extract disentangled latent space representations as stochastic embeddings, improving chest X-ray classification performance. A key characteristic of temporal ensembling is that each training sample's activation is updated only once per epoch.
 
\textbf{Mean teacher} \cite{tarvainen2017mean} applies exponentially moving average to model parameters rather than network activations, addressing the limitations of temporal ensembling and finding various applications in medical imaging \cite{li2020transformation,yu2019uncertainty,wang2020double,xu2023ambiguity,adiga2023anatomically}. Li \textit{et al.} \cite{li2020transformation} apply this approach to transformation-consistent medical image segmentation. Since teacher models can generate inaccurate targets for unlabeled data, Yu \textit{et al.} \cite{yu2019uncertainty} and Adiga \textit{et al.} \cite{adiga2023anatomically} incorporate uncertainty maps to ensure target reliability. Wang \textit{et al.} \cite{wang2020double} further propose a double-uncertainty-weighted method for left atrium and kidney segmentation, extending uncertainty from segmentation to feature level. Xu \textit{et al.} \cite{xu2023ambiguity} focus on selecting productive unsupervised consistency targets through an ambiguity-consensus mean-teacher model that better exploits complementary information from unlabeled data.

\subsection{Discussion}
Semi-supervised learning addresses the scarcity of labeled data by exploiting large amounts of unlabeled samples and enforcing supervision consistency across the dataset. 
% Newly Added
Data characteristics and task requirements should guide the choice of the Semi-SL strategy: proxy-labeling methods like self-training tend to perform well when high-confidence predictions can be reliably identified; multi-view learning approaches appear particularly suited for volumetric data where different perspectives provide complementary information; generative modeling shows promise with complex anatomical structures that benefit from learned prior distributions; while regularization-based methods often demonstrate robustness across diverse imaging modalities.
%
A persistent challenge in Semi-SL lies in the utilization of noisy or imperfect unlabeled data. The generation and selection of reliable pseudo labels are critical, as label noise can easily propagate through the training process and undermine model performance. Moreover, the theoretical understanding of how different Semi-SL techniques interact within hybrid systems remains limited, especially when dealing with heterogeneous data sources~\cite{wang2021deephybrid, zhang2022boostmis, wang2020focalmix, gyawali2020semi,miao2023caussl}. 

Meanwhile, the emergence of HFMs has significantly reshaped the landscape of semi-supervised learning in medical image analysis. As demonstrated by approaches like SemiSAM \cite{zhang2023semisam} and SemiSAM+ \cite{zhang2025semisam+}, these models introduce a paradigm shift from traditional model-centric Semi-SL methods focused on regularization strategies toward leveraging pre-trained knowledge to guide the learning process. Foundation models trained on large-scale datasets provide robust prior knowledge that helps specialist models learn more effectively with extremely limited labeled data—a scenario where conventional Semi-SL methods often struggle. This collaborative learning approach, where trainable specialist models interact with frozen foundation models, offers several advantages: it enhances performance in low-annotation regimes, provides more stable training due to knowledge transfer, and exhibits strong generalization capabilities across different medical imaging modalities and targets. As HFMs continue to evolve with improved architectures and more diverse training data, they will likely further transform Semi-SL in MIA, potentially reducing annotation requirements while increasing effectiveness and robustness.
\section{Inexact Label} \label{sec:weakly}
Obtaining precise, pixel-level annotations in MIA is often prohibitively expensive or time-consuming. In practice, clinicians frequently provide only incomplete or coarse annotations—such as sparse points or image-level labels—resulting in the so-called \textbf{inexact label} setting, where the available supervision does not match the granularity required for the target task. \textbf{Weakly supervised learning (WSL)} provides a general framework for addressing the challenges posed by inexact labels. The core issue in this context is label granularity alignment, which refers to bridging the gap between coarse-grained annotations and the fine-grained predictions demanded by clinical applications.
\begin{figure}[htbp]
	\centering
\includegraphics[width=\textwidth]{./Figures/Figs/wsl-workflow.png}
	\caption{Overview of weakly-supervised learning paradigm which addresses the \textit{inexact label} scenario. On the annotation side, annotation-efficient learning strategy aims to maximize the utility of limited or partially annotated data while minimizing annotation burden. On the task side, multi-instance learning strategy aims to bridge the label granularity gap by inferring fine-grained predictions from coarse supervision.
    }
	\label{fig_wsl_schematic}
\end{figure}
Granularity alignment in WSL can be approached from two complementary perspectives. On the annotation side, \textbf{annotation-efficient learning} scheme aims to maximize the utility of limited or partially annotated data while minimizing annotation burden. These methods leverage both labeled and unlabeled data to enhance model performance with reduced annotation cost. On the task side, the challenge is to infer fine-grained information, such as pixel-level segmentation or lesion localization, from supervision that is only available at a coarser level, such as the whole image. \textbf{Multiple instance learning (MIL)} provides a representative learning scheme for this scenario by associating global image-level labels with collections of instances, such as patches or regions, and enabling the model to identify relevant instances based on weak supervision.

In the following, we elaborate on the principles and representative applications of these strategies, with a focus on how label granularity alignment enables robust learning from inexact labels in MIA.

\subsection{Annotation-Efficient Learning}\label{sec:anno}
To address label granularity alignment from the annotation perspective, \textbf{annotation-efficient learning} leverages deep learning techniques with partially labeled data to enhance labeling efficiency for dense predictions. A strategic approach to boost annotation efficiency is to utilize markings other than complete dense annotations. 
While there may be some overlap with other methodologies, annotation-efficient learning methods are specifically tailored to harness the unique attributes of various annotation forms to improve efficiency and bridge the granularity disparity between annotation and prediction.
In this section, we review representative annotation-efficient learning strategies that address label granularity alignment through a coarse-to-fine approach, focusing on techniques related to \textbf{image-level}, \textbf{point}, \textbf{scribble}, and \textbf{box} annotations. 
Appendix Tab.~\ref{tab:anno} provides an overview of notable publications in this domain.

\subsubsection{Image-level Annotation}\label{sec:anno_tag}

Image-level annotation, or tag annotation, is a concise text or binary label assigned to each image, stands out as the most efficient form of annotation. 
Many image-level annotation-based methods draw inspiration from the concept of class activation mapping (CAM)~\cite{zhou2016learning}. 
Several approaches utilize CAM to generate object localization proposals or achieve pixel-wise segmentation of entire objects.
For \textbf{detection} tasks, Wang \textit{et al.}~\cite{hwang2016self} introduced a dual-branch network that concurrently optimizes classification and lesion detection. This approach involves supervising the CAM-based lesion detection network solely with image-level annotations. 
The two branches are guided in tandem through a weight-sharing technique, employing a weighting parameter to regulate the learning focus between classification and detection tasks. 
In the realm of lesion detection, Dubost \textit{et al.}~\cite{dubost2020weakly} proposed a weakly supervised regression network, validated on both 2D and 3D medical images. 
For the \textbf{segmentation} task, Li \textit{et al.}~\cite{li2022deep} proposed a breast tumor segmentation method with only image-level annotations based on CAM and deep-level set (CAM-DLS).
It integrates domain-specific anatomical information from breast ultrasound to reduce the search space for breast tumor segmentation.
Similarly, Li \textit{et al.}~\cite{li_sg-mian_2024} a self-guided multiple information aggregation network for skin lesion segmentation using multiple spatial perceptron solely using classification information as guidance to discriminate the key classification features of lesion areas.
Meanwhile, Chen \textit{et al.}~\cite{chen2022c} proposes a causal CAM method for organ segmentation, which is based on the idea of causal inference with a category-causality chain and an anatomy-causality chain.
In addition, several studies~\cite{lin2019seg4reg,lin2021seg4reg+} demonstrate that bridging the classification task and dense prediction task (e.g., detection and segmentation) via CAM-based methods is beneficial for both tasks.
Compared to natural images, medical images are usually from low contrast, limited texture, and varying acquisition protocols~\cite{zhang2021weakly}, which makes directly applying CAM-based methods less effective.
Fortunately, incorporating the clinical priors (e.g., objects' size~\cite{fruh2021weakly}) into the weakly supervised detection task is promising to improve the performance.
% ---------------------------------------------------------------
\subsubsection{Point Annotation}\label{sec:anno_point}
% ---------------------------------------------------------------
Point annotation refers to the annotation of a single point of an object.
Several studies~\cite{roth2021going,dorent2021inter,khan2019extreme} focus on using extreme points as the annotation to perform pixel-level segmentation.
These methods typically consist of three steps: 1) extreme point selection; 2) initial segmentation with a random walk algorithm; 3) training of the segmentation model with the initial segmentation results. 
The last two steps can be iterated until the segmentation results are stable.
However, these methods require the annotators to locate the boundary of the objects, which is still laborious in practice.
In contrast, other studies~\cite{yoo2019pseudoedgenet,zhao2020weakly,qu2020weakly,qu2019weakly,tian2020weakly,belharbi2021deep,lin2023nuclei,valvano2021learning} use center point annotation to perform pixel-level segmentation for the task of cell/nuclear segmentation. 
These methods typically adopt the Vorinor~\cite{kise1998segmentation} and cluster algorithms to perform coarse segmentation. 
Then different methods are used to refine the segmentation results, such as iterative optimization~\cite{qu2019weakly,qu2020weakly}, self-training~\cite{zhao2020weakly}, and co-training~\cite{lin2023nuclei}. 

Compared with full annotation, point annotation can reduce the annotation time by around 80\%~\cite{qu2020weakly}.
However, some issues have not been addressed. 
First, existing methods typically derived pseudo labels from the point annotation, which are based on strong constraints or assumptions (e.g., Voronoi) from the data, restricting the generalization of these methods to other datasets~\cite{lin2023nuclei}.
Second, due to the lack of explicit boundary supervision, there is a non-negligible performance gap between the weakly supervised methods with points and the fully supervised methods.

\subsubsection{Scribble Annotation}\label{sec:anno_scribble}
% ---------------------------------------------------------------
Scribble annotation, a set of scribbles drawn on an image by the annotators, has been recognized as a user-friendly alternative to bounding box annotation~\cite{tajbakhsh2020embracing}. 
Compared with point annotation, it provides the rough shape and size information of the objects, which is promising to improve the segmentation performance, especially for objects with complex shapes.
Wang \textit{et al.}~\cite{wang2018interactive} propose a self-training framework with differences in model predictions and user-provided scribbles. 
Can \textit{et al.}~\cite{can2018learning} develop a random walk algorithm that incrementally performs region growing method around the scribble ground truth, while
Lee \textit{et al.}~\cite{lee2019scribble2label} introduce Scribble2Label, a method that integrates the supervision signals from both scribble annotations and pseudo labels with the exponential moving average. 
Furthermore, Dorent \textit{et al.}~\cite{dorent2020scribble} extend the Scribble-Pixel method to the domain adaptation scenario, where a new formulation of domain adaptation is proposed based on CRF and co-segmentation with the scribble annotation. 
Zhang \textit{et al.}~\cite{zhang2022cyclemix} adopt mix augmentation and cycle consistency for the Scribble-Pixel method, demonstrating the improvement of both weakly and fully supervised segmentation methods.
Zhou \textit{et al.}~\cite{zhou_weakly_2023} proposed a scribble-supervised approach that combines superpixel-guided scribble walking with class-wise contrastive regularization to augment the structural priors into the weak annotations.

\subsubsection{Box Annotation}\label{sec:anno_box}

Box annotation encloses the segmented region within a rectangle, and various studies have focused on this Box-Pixel scenario. 
Rajchl \textit{et al.}~\cite{rajchl2016deepcut} employ a densely-connected random field (DCRF) with an iterative optimization method for MRI segmentation. 
Wang \textit{et al.}~\cite{wang2021accurate,wang2021bounding} adopt smooth maximum approximation based on the bounding box tightness prior~\cite{hsu2019weakly}, that is, an object instance should touch all four sides of its bounding box. 
Thus, a vertical or horizontal crossing line within a box yields a positive bag because it covers at least one foreground pixel. 
Studies~\cite{wang2021bounding} demonstrate that the Box-Pixel method yields promising performance, being only 1--2\% inferior to the fully supervised methods.
In a recent work, Wong \textit{et al.}~\cite{wong_scribbleprompt_2025} proposed ScribblePrompt, which is flexible to different annotation styles, including bounding boxes, points, and scribbles. Overall, box annotation offers a strong balance between efficiency and accuracy, but may struggle with irregular or overlapping shapes.

\subsection{Multi-instance Learning}

To address label granularity alignment from the task perspective, \textbf{multi-instance learning (MIL)} enables deep learning models to make fine-grained predictions using only coarse, image-level supervision. MIL organizes each image or specimen as a \textit{bag} of multiple \textit{instances} (e.g., patches, regions, or cells), with only the bag-level label observed during training. Under the standard MIL assumption, a bag is positive if at least one of its instances is positive. This framework supports weak supervision and allows for fine-grained analysis—such as localizing disease regions—even when only image-level labels are available.

Distinct from approaches that primarily calibrate the form of supervision, MIL is specifically tailored to bridge the granularity gap between global supervision and dense prediction requirements. By modeling the instance collections within each image, MIL enables inference of instance-level relevance from bag labels, thereby supporting fine-grained pattern recognition using only bag-level supervision.

In this survey, we categorize MIL approaches by their prediction target: \textbf{local-context MIL} focuses on detecting and classifying patterns at the instance level, while \textbf{global-context MIL} aims for bag-level predictions. Representative works are summarized in Appendix Tab.~\ref{tab:mil}.

\subsubsection{Local-Context MIL}
Within the MIL paradigm, local-context MIL approaches focus on the fine-grained identification and localization of specific disease patterns within medical images. These methods are designed to infer instance-level labels for individual patches, enabling precise delineation of pathological regions. By determining the status of each constituent patch, local-context MIL naturally encompasses global-context capabilities—the image-level classification emerges as a function of the detected local annotations. This hierarchical inference process allows clinicians not only to receive diagnostic outcomes but also to visualize the spatial distribution of disease manifestations, enhancing interpretability and clinical utility. The ability to simultaneously perform localization and diagnosis makes local-context MIL particularly valuable in applications requiring both detection sensitivity and anatomical precision.

Schwab \textit{et al.} \cite{schwab2020localization} apply MIL to chest X-ray classification and localization by processing image patches through a CNN to assess their probability of containing critical findings. While traditional MIL approaches integrate patch-level predictions using max-pooling or average-pooling, Couture \textit{et al.} \cite{couture2018multiple} enhance this by implementing a quantile function for pooling, providing better characterization of sample heterogeneity. The field has evolved toward learning-based aggregation methods, notably the attention-based MIL developed by Ilse \textit{et al.} \cite{ilse2018attention}, which better captures and interprets regions of interest. Wang \textit{et al.} \cite{wang2021learning} extend this concept with an inductive attention guidance network for pancreatic ductal adenocarcinoma, where the attention mechanism bridges global classification and local segmentation by identifying relevant regions.

Other intriguing improvements in local-context MIL are springing up as well. Researchers have tried many different ways to facilitate instance prediction \cite{dov2021weakly,manivannan2017subcategory,jia2017constrained,xu2019camel}. Dov \textit{et al.} \cite{dov2021weakly} address cytopathology challenges—where informative instances are sparse and exhibit varied abnormalities—by implementing maximum likelihood estimation to simultaneously predict bag-level labels, diagnostic scores, and instance-level labels. For retinal nerve fiber layer classification, Manivannan \textit{et al.} \cite{manivannan2017subcategory} overcome strong intra-class variation by mapping instances into a discriminative subspace that enhances feature disentanglement. Jia \textit{et al.} \cite{jia2017constrained} incorporate multi-scale image features to extract more latent information from histopathology images. Addressing the limitation of image-level-only labels in MIL, Xu \textit{et al.} \cite{xu2019camel} develop an automatic instance-level label generation method, creating a promising direction for local-context MIL algorithm development.

In parallel, there has been significant progress in related domains such as phenotype categorization \cite{yao2020whole, hashimoto2020multi, yao2019deep} and multi-label classification \cite{mercan2017multi}. These investigations have further exemplified the versatility and potential of the MIL algorithm in addressing complex challenges across various subfields. 

\subsubsection{Global-Context MIL}
In contrast to the localization focus of local-context approaches, global-context MIL aims to detect whether or not target patterns exist. For example, for the COVID-19 screening problem, researchers \cite{li2021novel} have designed MIL algorithms to classify an input sample as severe or not instead of locating every abnormal patch. 

To facilitate the prediction of image-level labels (\textit{e.g.} WSI-level label in the context of computational pathology), researchers normally start from one of two aspects, namely instance- and bag-level. Most existing MIL algorithms \cite{tomita2019attention, hashimoto2020multi,naik2020deep,lu2021data} are based on the basic assumption that instances of the same bag are independent and identically distributed. Consequently, the correlations among instances are neglected, which is not realistic. Several subsequent works have taken the correlation among instances or tissues into consideration \cite{shao2021transmil,wang2022lymph,wang2019rmdl,raju2020graph,han2020accurate}. In \cite{shao2021transmil}, Shao \textit{et al.} introduce Vision Transformer (ViT) into MIL for gigapixel WSIs due to its great advantage in capturing the long-distance information and correlation among instances in a sequence. Meanwhile, to conduct precise lymph node metastasis prediction, Wang \textit{et al.} \cite{wang2022lymph} not only incorporate a pruned Transformer into MIL but also develop a knowledge distillation mechanism based on other similar datasets, such as a papillary thyroid carcinoma dataset, effectively avoiding the overfitting problem caused by the insufficient number of samples in the original dataset. Similarly, Raju \textit{et al.} \cite{raju2020graph} design a graph attention MIL algorithm for colorectal cancer staging, which utilizes different tissues as nodes to construct graphs for instance relation learning. Further, in order to utilize the multi-resolution characteristics of WSIs, Shi \textit{et al.} \cite{shi2023structure} consider WSIs as multi-scale graphs and utilize attention mechanism to integrate their information for primary tumor stage prediction. Similar idea can be found in \cite{yan2023genemutation, shi2023mg, xiang2023multi}. Besides, Liu \textit{et al.} \cite{liu2024advmil} firstly propose an integration of GAN with MIL mechanism for robust and interpretable WSI survival analysis by more accurately estimating target distribution. 

For bag-level improvement, recent years have witnessed two feasible approaches, namely, improved pooling methods and pseudo bags. On the one hand, in order to aggregate the instances with the most information, some researchers have developed novel aggregation methods in MIL algorithms instead of the traditional max pooling \cite{chikontwe2020multiple,das2018multiple,jin2024hmil,guo2024focus}. For example, in \cite{chikontwe2020multiple}, the authors design a pyramid feature aggregation method to directly obtain a bag-level feature vector. On the other hand, however, there is an inherent problem for MIA, especially for histopathology --- the number of WSIs (bags) is usually small, while in contrast, one WSI has numerous patches, leading to an imbalance in the number of bags and instances. To address this problem, Zhang \textit{et al.} \cite{zhang2022dtfd} randomly split the instances of a bag into several smaller bags, called "pseudo bags", with labels that are consistent with the original bag. A similar idea can also be seen in \cite{li2021novel}. Moreover, Jin \textit{et al.} \cite{jin2024hmil} introduce a hierarchical multi-instance learning (HMIL) framework that enhances WSI classification through explicit modeling of the hierarchical relationships between instance-level and bag-level label distributions, creating a more cohesive cross-granularity learning paradigm.

Other improvements in MIL algorithms are also worth mentioning \cite{su2022attention2majority,tennakoon2019classification,wang2020ud}. In \cite{su2022attention2majority}, a novel sampling method is developed to collect instances with high confidence. This method excludes patches shared among different classes and tends to select the patches that match with the bag-level label. In \cite{tennakoon2019classification}, the authors utilize the extreme value theory to measure the maximum feature deviations and consequently leverage them to recognize the positive instances, while in \cite{wang2020ud}, Wang \textit{et al.} introduce an uncertainty evaluation mechanism into MIL for the first time, and train a robust classifier based on this mechanism to cope with OCT image classification problem. 

\subsection{Discussion}
Weakly supervised learning addresses the challenges of limited and inexact annotations in MIA by calibrating both the annotation form and granularity of supervision. Within this paradigm, annotation-efficient learning and MIL represent two complementary facets for coping with weak supervision.

Annotation-efficient learning leverages diverse annotation types—points, scribbles, boxes, and image tags—each suited to different object characteristics and annotation budgets. Points and scribbles are efficient for objects with uniform shape, while boxes handle morphological variation, and image tags offer the lowest annotation cost at the expense of spatial detail. Recent advances in HFMs, especially vision-language models, have enabled high-quality predictions from minimal supervision by integrating multi-modal cues and prompt-based learning~\cite{li2023blip,ren2024grounded}. Future directions include unifying multiple weak supervision signals, leveraging human-in-the-loop strategies, and mining knowledge from multi-modal data to further reduce annotation costs.

Multi-instance learning (MIL) addresses the calibration of annotation granularity, particularly for tasks such as gigapixel-sized WSI analysis, where only bag-level labels are available. MIL enables fine-grained, instance-level inference by modeling bag-instance and instance-instance relationships, effectively aligning global supervision with the need for precise localization or characterization. The rise of HFMs trained on large and diverse datasets has significantly enhanced MIL by providing rich, transferable features, boosting performance without extra annotation~\cite{vorontsov2024foundation,chen2024towards,lu2024visual,ma2024towards,xu2024multimodal,xu2024whole}. Key future directions include developing more interpretable and privacy-preserving MIL frameworks~\cite{javed2022additive,kapse2024si}, designing efficient attention mechanisms for high-dimensional features~\cite{guo2024histgen,li2024rethinking,tang2024feature,guo2024focus}, and integrating foundation models with domain-specific knowledge~\cite{zhang2023textadaptation,yin2024prompting,lu2024pathotune}.
\section{Active Learning in MIA} \label{sec:al}
\begin{table*}[ht]
\centering
\caption{Overview of Active Learning-based Studies in Medical Image Analysis}

\resizebox{\textwidth}{.13\textwidth}{
\begin{threeparttable}
\begin{tabular}{@{}llllll@{}}
\toprule
 &Reference (Year) & Organ & Sampling Method & Dataset & Result \\ \midrule
\rule{0pt}{2ex} \multirow{1}{*}{\rotatebox{90}{Classification}}&  
%Gal \textit{et al.} 
\citet{gal2017deep} &  Skin  & BALD + KL-divergence  &  ISIC 2016 & 22\%  image input: AUC: 0.75\rule[-1.5ex]{0pt}{0pt}\\ 
\cline{2-6}
\rule{0pt}{2.5ex}& 
%Wu \textit{et al.} 
\citet{wu2021covid} &  Lung  & Loss Prediction Network  & CC-CCII Dataset & 42\% Chest X-Ray input: Acc: 86.6\%\rule[-1.5ex]{0pt}{0pt}\\ 
\cline{2-6}
\rule{0pt}{3ex}&
%Li \textit{et al.} 
\citet{li2021pathal}  &  Prostate  & CurriculumNet + O2U-Net  &  ISIC 2017; PANDA Dataset & 60\% input: QWK: 0.895\rule[-3.5ex]{0pt}{0pt}\\
\hline
\rule{0pt}{2.5ex} \multirow{8}{*}{\rotatebox{90}{Segmentation}}&
%Yang \textit{et al.} 
\citet{yang2017suggestive} & Gland; Lymph  & Cosine Similarity + Bootstrapping + FCN& GlaS 2015; Private Dataset: 80 US images & MICCAI 2015: 50\% input: F1: 0.921; Private Dataset: 50\% input: F1: 0.871\rule[-1.5ex]{0pt}{0pt}\\ 
\cline{2-6}
\rule{0pt}{2.5ex}&    
%Konyushkova \textit{et al.}
\citet{konyushkova2019geometry}& Brain (Striatum; Hippocampus)  &    Geometric Priors + Boosted Trees  &  BraTS 2012; EFPL EM Dataset & MRI Data: 60\% input: DSC$\approx$0.76;  EM Data: 40\% input: DSC$\approx$0.60 \rule[-1.5ex]{0pt}{0pt} \\
\cline{2-6}
\rule{0pt}{2.5ex} &    
%Nath \textit{et al.} 
\citet{nath2020diminishing} &  Brain  & Entropy + SVGD Optimization  &  MSD 2018 Dataset & 22.69\% Hippocampus MRI input: DSC: 0.7241 \rule[-1.5ex]{0pt}{0pt}\\
\cline{2-6}
\rule{0pt}{2.5ex}& 
%Ozdemir \textit{et al.} 
\citet{ozdemir2021active}  & Shoulder & BNN +  MMD Divergence & Private Dataset: 36 Volume of MRIs & 48\% MRI input: DSC$\approx$0.85 \rule[-2ex]{0pt}{0pt}\\
\cline{2-6}
\rule{0pt}{2.5ex}& 
%Zhao \textit{et al.} 
\citet{zhao2021dsal} &  Hand; Skin  & U-Net  & RSNA-Bone; ISIC 2017 & 9 AL Iteration: DSC: 0.834 \rule[-1.5ex]{0pt}{0pt}\\
\hline
\rule{0pt}{2.5ex} \multirow{5}{*}{\rotatebox{90}{Others}} & \multirow{2}{*}{\citet{mahapatra2018efficient}$_{\text{CS}}$} %Mahapatra \textit{et al.}
&  Chest  & Bayesian Neural Network + & JSRT Database;& Classification: 35\% input: AUC: 0.953;\\ 
& && cGAN Data Augmentation & ChestX-ray8 & Segmentation: 35\% input: DSC: 0.910\rule[-1.5ex]{0pt}{0pt}\\
\cline{2-6}
\rule{0pt}{2.5ex}& \multirow{2}{*}{
%Zhou \textit{et al.}
\citet{zhou2021active}$_{\text{CD}}$} & 
Colon & Traditional Data Augmentation& Private Dataset: 6 colonoscopy videos & Classification: 4\% input: AUC: 0.9204;\\
& &&Entropy + Diversity & 38 polyp videos + 121 CTPA datasets & Detection: 2.04\% input: AUC: 0.9615\rule[-1.5ex]{0pt}{0pt}\\
\bottomrule
\end{tabular}
\begin{tablenotes}    
        \footnotesize               
        \item[1] For the sake of brevity, we denote references that contain more than one task in the following abbreviations: \textbf{C}: Classification, \textbf{S}: Segmentation, \textbf{D}: Detection. 
      \end{tablenotes}
\end{threeparttable}
}
\label{tab:al}
\end{table*}
\subsection{Data Uncertainty-Based Methods}
Developed from the conventional entropy uncertainty metrics\footnote{To aid the understanding of these metrics, a detailed description of the prior knowledge is provided in Appendix A.2.}, 
%Konyushkova \textit{et al.} 
\citet{konyushkova2019geometry} defined geometric smoothness priors with boosted trees to classify the formed graph representation of electron microscopy images. Here, they flatten 3D images into supervoxels with the SLIC algorithm \citep{achanta2012slic} to conduct graph representations. 
%Yang \textit{et al.} 
\citet{yang2017suggestive} use cosine similarity and a bootstrapping technique to evaluate the uncertainty and representativeness of the output feature with a DCAN \citep{chen2016dcan}-like network.
%Zhou \textit{et al.} 
\citet{zhou2021active} propose the concept of ``active selection" policies, which is the highest confidence based on the entropy and diversity results from sampled data in the mean prediction results. 

Aside from leveraging conventional metrics, utilizing metrics from the deep learning model is another trend. Intuitively, 
%Wu \textit{et al.} 
\citet{wu2021covid} utilize network loss as well as the diversity condition as the uncertainty metric for sampling from a loss prediction network, and conduct the COVID-19 classification task from another classification network. 
%Nath \textit{et al.} 
\citet{nath2020diminishing} leverage marginal probabilities between the query images and the labeled ones, they build a mutual information metric as the diversity metric to serve as a regularizer. Moreover, they adopt Dice log-likelihood instead of its original entropy-based log-likelihood for Stein variational gradient descent optimizer \citep{liu2017stein} to solve the label imbalance problem. 
%Zhao \textit{et al.} 
\citet{zhao2021dsal} utilize Dice's coefficient of the predicted mask calculated between the middle layer and the final layer of the model as the uncertainty metric for the image segmentation task. They use their DS-UNet with a denseCRF \citep{krahenbuhl2011efficient} refiner to annotate low uncertainty samples and oracle annotators for the others. 
%Li \textit{et al.} 
\citet{li2021pathal} use k-means clustering and curriculum classification (CC) based on the CurriculumNet \citep{guo2018curriculumnet} for uncertainty and representativeness estimation. Furthermore, they consider the condition under which noisy medical labels are present and accomplish their automatic exclusion using O2U-Net \citep{huang2019o2u}.  
\subsection{Model Uncertainty-Based Methods}
Bayesian neural networks have attracted increasing attention for their ability to represent and propagate the probability of the DL model. 
%Gal \textit{et al.} 
\citet{gal2017deep} employ Bayesian CNNs for skin cancer classification with Bayesian active learning by disagreement (BALD) \citep{houlsby2011bayesian}.
%Ozdemir \textit{et al.} 
\citet{ozdemir2021active} form a Bayesian network and employ Monte Carlo dropout \citep{gal2016dropout} to obtain the variance information as the model uncertainty. They also construct a representativeness metric produced by infoVAE \citep{zhao2017infovae} for maximum likelihood sampling in the latent space. 
%Mahapatra \textit{et al.} 
\citet{mahapatra2018efficient} also uses a Bayesian neural network to sample the training data. Meanwhile, they use conditional GAN to generate realistic medical images for data augmentation.
\subsection{Discussion}
Whether from the data or from the model, uncertainty measurement is a critical task throughout the whole AL process. The current research directions regarding label-efficient AL methods in MIA focus primarily on the improvement of AL query strategies and the optimization of training methods. For the future, researchers could i) delve into hybrid AL query strategies together with diversity assessment, ii) concentrate on hybrid training schemes (\textit{i.e.}, combined Semi-SL, Self-SL schemes) to yield an intermediate feature representation to further guide the training process, iii) mitigate the degradation of annotation quality when encountering noisy labels.
\section{Challenges and Future Directions} \label{sec:cnfd}
Our comprehensive discussion of label-efficient learning schemes in MIA raises several challenges that should be taken into account to improve the performance of the DL model. In this section, we describe the crucial challenges and shed light on potential future directions for solving these challenges.

\subsection{Omni-Supervised Learning} 
Although the methods we have presented have achieved promising performance, many of them are targeted at addressing \textit{ad hoc} label shortage problems, \textit{i.e.}, these methods do not utilize as much supervision as possible.
Served as a special regime of Semi-SL, \textbf{Omni-supervised learning} is a crucial trend for label-efficient learning in MIA for the simultaneous utilization of different forms of supervision. Studies \cite{luo2021oxnet, chai2022orf} have demonstrated the feasibility of omni-supervised learning under teacher-student\cite{tarvainen2017mean} and the dynamic label assignment \cite{chai2022orf} pipeline, respectively. In the teacher-student training approach, the model trained on fully annotated datasets serves as the teacher model, and features extracted from the weakly-/un-annotated datasets serve as guidance to refine the model. 
Through designated mechanisms, the student model utilizes the teacher model with the provided guidance to further improve performance.  Meanwhile, the dynamic label assignment approach forms the crafted metric from different types of labels in the training process and dynamically gives the final predicted labels. 

During the process of omni-supervised learning, however, centralizing or releasing different supervision health data raises multiple ethical, legal, regulatory, and technological issues \cite{rieke2020future}. On the one hand, collecting and maintaining a high-quality medical dataset consumes a large amount of expense, time, and effort. 
On the other hand, the privacy of patients may be compromised during the centralization or release of health datasets, even with techniques such as anonymization and safe transfer. To address the privacy preservation problem during model development, researchers proposed \textbf{federated learning (FL)} to conduct training in a data-decentralized manner. 
This approach has yielded fruitful results in The field of MIA \cite{dayan2021federated,li2020multi,lu2022federated}. However, current FL algorithms are primarily trained in a supervised manner. When applying the FL to real-world scenarios in MIA, a crucial problem, namely, label deficiency, may appear in local health datasets.
Labels may be missing to varying degrees between medical centers, or the granularity of the labels will vary. A promising research direction is to design label-efficient federated learning methods to address this significant problem. For example, semi-supervised learning\cite{liu2021federated}, active learning, and self-supervised learning \cite{dong2021federated} are suitable to be incorporated into this setting.

\subsection{Human-in-the-loop Interaction}
The application of expert knowledge to refine the output of the model is often carried out in practice, and there have been various efforts to investigate this field, known as human-in-the-loop (HITL). The AL scheme can be considered a part of HITL as it involves the introduction of expert knowledge to refine data supervision. Meanwhile, expert knowledge can also be introduced as action supervision under the \textbf{reinforcement learning (RL)} schemes to improve the performance of the DL model \cite{liao2020iteratively, ma2020boundary}. In RL scheme, a set of “agents” is formulated to learn expert behaviors in an interactive environment via trial and error. In MIA tasks, RL methods mainly treat the interactive refinement process as the Markov decision process (MDP) and give the solution by the RL process. RL-based interventional model training brings the potential for dealing with rare cases in MIA, since the expert-provided interactions can refine the prediction result at the final stage to hinge samples that failed to process by the DL model.
In addition, recent developments in diverse learning methodologies, including but not limited to few-shot learning \cite{feng2021interactive, al2021ifss} and interpretability-guided learning \cite{mahapatra2021interpretability}, have contributed to improved efficacy of human-in-the-loop workflows, thereby reducing labor costs in MIA. This indicates a positive trend towards increased cost-effectiveness in this field.

\subsection{Generation-based Data Augmentation}
Data augmentation with synthesized images produced by generative-based methods is regarded as a way to unlock additional information from the dataset, and leads the way in computation speed and quality of results in the scope of generative methods~\cite{shorten2019survey}.
In the field of MIA, numerous studies~\cite{wang2020semi,lin2022insmix} have investigated data augmentation with the original GAN~\cite{goodfellow2014generative}  and its variations. 
 However, the unique adversarial training procedure of GANs may suffer from training instability~\cite{gulrajani2017improved} and mode collapse~\cite{lin2018pacgan}, yielding ``Copy GAN", which only generates a limited set of samples~\cite{yang2019bi}.
Thus, synthesizing augmented data with a high visual realism and diversity is the key challenge of GAN.
Meanwhile, the \textbf{probabilistic diffusion model} \cite{ho2020denoising}, has recently sparked much interest in MIA applications\cite{kazerouni2022diffusion}. 
This model establishes a forward diffusion stage in which the input data is gradually disrupted by adding Gaussian noise over multiple stages and then learns to reverse the diffusion process to obtain the required noise-free data from noisy data samples. 
Despite their recognized computational overhead \cite{xiao2022tackling}, diffusion models are generally praised for their high mode coverage and sample quality, and various of efforts have been made to ease the computational cost and further improve their generalization capability.

\subsection{Generalization Across Domains and Datasets}
From semi-supervised learning to annotation-efficient learning, we have introduced a considerable number of methods that address the problem of the low-quantity and/or -quality of labels. Nevertheless, recent results reveal that these novel methods may encounter significant performance degradation when shifting to different domains or datasets. The generalization problem in MIA field arises due to multiple causes, such as variance among scanner manufacturers, scanning parameters, and subject cohorts. And various current deep learning algorithms cannot be robustly deployed in various real scenarios. To address this practical problem, the concept of \textbf{domain generalization} has been introduced, of which the key idea is to learn a trained model that encapsulates general knowledge so as to adapt to unseen domains and new datasets with little effort and cost. A plethora of methods have been developed to tackle the domain generalization problem \cite{zhou2022domain}, such as domain alignment \cite{li2018domain}, meta-learning \cite{li2019feature}, data augmentation \cite{qiao2020learning} and so on. MIA has also seen some publications with respect to domain generalization \cite{li2020domain,liu2021feddg}. Further, another challenge for generalization across domains and datasets is that the proposed methods may require numerous labeled multi-source data to extract domain-invariant features. For example, Yuan \textit{et al.} \cite{yuan2022label} have made a successful attempt to achieve model generalization in source domains with limited annotations by leveraging active learning and semi-supervised domain generalization, eliminating the dilemma between domain generalization and expensive annotations.

\subsection{Benchmark Establishment and Comparison}
Label-efficient learning in MIA spans multiple tasks, such as classification, segmentation, and detection, as well as multiple organs, such as the retina, lung, and kidney. Differences and variances in tasks and target organs lead to confounding experiment settings and unfair performance comparisons. Meanwhile, a lack of sufficient public health datasets also contributes to this dilemma. For example, many researchers can only conduct experiments to measure the performance of their proposed algorithms based on their own private datasets due to reasons such as privacy. However, few publications have emerged \cite{gut2022benchmarking} to address the problem, especially for label-efficient learning. Thus, benchmarkng remains a pressing problem for model evaluation. On the one hand, the public should urge for the availability of large datasets. On the other hand, a clearly defined set of benchmarking tasks and the corresponding evaluation procedures should be established. Further, specific experimental details should be stipulated to facilitate the comparability of different label-efficient learning algorithms.
\section{Conclusion}\label{sec:con}
Label-efficient learning has emerged as a pivotal direction in medical image analysis, addressing not only the practical constraints of annotation scarcity but also prompting a re-examination of the fundamental relationship between data, supervision, and clinical value. In this survey, we have introduced a unified taxonomy that spans scenarios of no label, insufficient label, inexact label, and label refinement. This framework has clarified the methodological landscape and illuminated the distinct challenges and opportunities inherent to each paradigm. Through a critical analysis of state-of-the-art methodologies, we have highlighted both recent advances and the persistent barriers to clinical adoption. Our synthesis demonstrates that genuine progress in label efficiency requires not only algorithmic innovation but also coordinated advances in large-scale data curation, adaptive learning strategies, and standardized evaluation. As the field moves toward broader clinical integration, the challenges of generalization, interoperability, and meaningful assessment remain central. By clarifying these issues and outlining future directions, this survey aims to serve as both a reference and a foundation for the continued evolution of label-efficient learning in medical imaging.

%%
%% The next two lines define the bibliography style to be used, and
%% the bibliography file.
\bibliographystyle{ACM-Reference-Format}
\bibliography{reference}


%%
%% If your work has an appendix, this is the place to put it.
\appendix
\newpage
\noindent\Large{\textbf{Appendix}}
\section{Survey Scope}
\label{appendix1}
To ensure comprehensive coverage of relevant literature, we conducted a systematic search using Google Scholar for publications related to label-efficient medical imaging up to March 2025. Additionally, we queried the arXiv preprint server for manuscripts containing key terms pertinent to label-efficient learning in medical imaging. Major conference proceedings—including CVPR, ICCV, ECCV, NeurIPS, AAAI, and MICCAI—as well as leading journals such as Medical Image Analysis (MIA), IEEE Transactions on Medical Imaging (TMI), and Nature Biomedical Engineering, were thoroughly reviewed based on paper titles and abstracts. Furthermore, the reference lists of selected papers were examined to identify additional relevant works. In cases where similar work appeared in multiple venues, only the most impactful or comprehensive publication(s) were included in this survey.

\clearpage

\section{Surveyed Literature and Datasets}

\subsection{Surveyed Literature}
\begin{table*}[ht]
\centering
\caption{Surveyed Self-supervised Learning-based Studies in Medical Image Analysis.}

\resizebox{\textwidth}{.56\textwidth}{
\begin{threeparttable}
\begin{tabular}{@{}llllll@{}}
\toprule
 &Reference & Organ & Proxy Task Design & Dataset & Publication \\ \midrule
\rule{0pt}{2ex} \multirow{20}{*}{\rotatebox{90}{Classification}}&  Li \textit{et al.} \cite{li2020self} & Retina & Multi-modal Contrastive Learning  & ADAM; PALM& TMI 2020\\
\cline{2-6}
% \rule{0pt}{2.5ex}&Zhao \textit{et al.} \cite{zhao2021anomaly}  & Retina; Lung & Inpainting; Local Pixel Shuffling;  & RetinalOCT; ChestX &RetinalOCT: AUC: 0.9642; F1: 0.9342\\
% &&&Non-Linear Intensity Transformation&&ChestX: AUC: 0.8265; F1: 0.8214\rule[-1.2ex]{0pt}{0pt}\\
% \cline{2-6}
\rule{0pt}{2.5ex}&Koohbanani \textit{et al.} \cite{koohbanani2021self} & Breast;  &Magnification Prediction;  & CAMELYON 2016; &TMI 2021  \\
&&Cervix;&Solving Magnification Puzzle;&KATHER;& \\
&&Colon&Hematoxylin Channel Prediction&Private Dataset: 217 Images&\rule[-1.2ex]{0pt}{0pt}\\
\cline{2-6}
% \rule{0pt}{2.5ex}&Li \textit{et al.} \cite{li2021rotation} & Retina &Image Rotation  & ADAM; PALM; DRD & ADAM: AUC: 0.7811; PALM: AUC: 0.9912 \rule[-1.2ex]{0pt}{0pt}\\
% \cline{2-6}
\rule{0pt}{2.5ex}&Azizi \textit{et al.} \cite{azizi2021big} & Skin; Lung &Multi-Instance Contrastive Learning & Priavte Dermatology Dataset; CheXpert  &CVPR 2021\rule[-1.2ex]{0pt}{0pt}\\
\cline{2-6}
% \rule{0pt}{2.5ex}&Yang \textit{et al.} \cite{yang2022cs} & Colon & Cross-stain prediction + Contrastive Learning& KATHER& Acc: 0.918\rule[-1.2ex]{0pt}{0pt}\\
% \cline{2-6}
\rule{0pt}{2.5ex}&Tiu \textit{et al.} \cite{tiu2022expert} & Lung & Contrastive Learning  & CheXpert&Nature BME 2022\rule[-1.2ex]{0pt}{0pt}\\
\cline{2-6}
\rule{0pt}{2.5ex}&Chen \textit{et al.} \cite{chen2022scaling} & Breast; Lung; Kidney & Contrastive Learning  & TCGA-BRCA; TCGA-NSCLS; TCGA-RCC&CVPR 2022\rule[-1.2ex]{0pt}{0pt}\\
\cline{2-6}
\rule{0pt}{2.5ex}&Mahapatra \textit{et al.} \cite{mahapatra2022self} & Lymph; Lung; & Contrastive Learning Variant  & CAMELYON 2017; DRD; GGC&TMI 2022\\
&&Retina; Prostate&&\rule[-1.2ex]{0pt}{0pt}\\
\cline{2-6}
\rule{0pt}{2.5ex}&Wang \textit{et al.} \cite{wang2023ssd} & Skin & Self-supervised Knowledge Distillation  & ISIC 2019 &MIA 2023 \rule[-1.2ex]{0pt}{0pt}\\
\cline{2-6}
\rule{0pt}{2.5ex}& Huang \textit{et al.} \cite{huang2024systematic}  &  Multi-Organ   & SimCLR, MOCOv2,
SwAV, BYOL, SimSiam, DINO, BarlowTw & TissueMNIST; PathMNIST; TMED-2; AIROGS& CVPR 2024 \rule[-1.2ex]{0pt}{0pt}\\
\cline{2-6}
\rule{0pt}{2.5ex}& Tang \textit{et al.} \cite{tang2024self}  &  Lung & Self-Supervised Representation Distribution Learning & TCGA-EGFR; TCGA-Lung; Private Lung Dataset & TMI 2024 \rule[-1.2ex]{0pt}{0pt}\\
\cline{2-6}
\rule{0pt}{2.5ex}& Vorontsov \textit{et al.} \cite{vorontsov2024foundation}  & Multi-Organ & Self-distillation + Masked Image Modeling (DINOv2) & Cancer-related Diagnosis Datasets & Nat. Med. 2024 \rule[-1.2ex]{0pt}{0pt}\\
\cline{2-6}
\rule{0pt}{2.5ex}& Chen \textit{et al.} \cite{chen2024towards}  & Multi-Organ & Self-distillation + Masked Image Modeling (DINOv2) & Cancer-related Diagnosis Datasets & Nat. Med. 2024 \rule[-1.2ex]{0pt}{0pt}\\
\cline{2-6}
\rule{0pt}{2.5ex}& Lu \textit{et al.} \cite{lu2024visual}  & Multi-Organ & Visual-language Contrastive Learning + Captioning (CoCa)  & Cancer-related Diagnosis Datasets & Nat. Med. 2024 \rule[-1.2ex]{0pt}{0pt}\\
\hline
\rule{0pt}{2.5ex} \multirow{14}{*}{\rotatebox{90}{Segmentation}}& Hervella \textit{et al.} \cite{hervella2018retinal}$_{2018}$ & Retina &Multi-modal Reconstruction & Isfahan MISP & MICCAI 2018 \rule[-1.2ex]{0pt}{0pt}\\
\cline{2-6}
\rule{0pt}{2.5ex}&Spitzer \textit{et al.} \cite{spitzer2018improving}$_{2018}$ & Brain & Patch Distance Prediction   & BigBrain & MICCAI2018\rule[-1.2ex]{0pt}{0pt}\\
\cline{2-6}
\rule{0pt}{2.5ex} &  Bai \textit{et al.} \cite{bai2019self} & Heart & Anatomical Position Prediction & Private Dataset: 3825 Subjects & MICCAI 2019\rule[-1.2ex]{0pt}{0pt}\\
\cline{2-6}
\rule{0pt}{2.5ex}& Sahasrabudhe \textit{et al.} \cite{sahasrabudhe2020self}  & Multi-Organ & WSI Patch Magnification Identification &MoNuSeg & MICCAI 2020\rule[-1.2ex]{0pt}{0pt}\\
\cline{2-6}
\rule{0pt}{2.5ex}& Tao \textit{et al.} \cite{tao2020revisiting} & Pancreas &  Rubik's Cube Recovery& NIH PCT; MRBrainS18& MICCAI 2020\rule[-1.2ex]{0pt}{0pt}\\
\cline{2-6}
\rule{0pt}{2.5ex}&Lu \textit{et al.} \cite{lu2021volumetric}& Brain & Fiber Streamlines Density Map Prediction;& dHCP &MIA 2021\\
&&& Registration-based Segmentation Imitation& & \rule[-1.2ex]{0pt}{0pt}\\
\cline{2-6}
\rule{0pt}{2.5ex}&Tang \textit{et al.} \cite{tang2022self} & Abdomen; Liver; &Contrastive Learning; Masked Volume Inpainting;    & DECATHLON; &CVPR 2022\\
&&Prostate&3D Rotation Prediction &BTCV&\rule[-1.2ex]{0pt}{0pt}\\
\cline{2-6}
\rule{0pt}{2.5ex}&Jiang \textit{et al.} \cite{jiang2023anatomical} & Multi-organ &Anatomical-invariant Contrastive Learning    & FLARE 2022; BTCV &CVPR 2023\rule[-1.2ex]{0pt}{0pt}\\
\cline{2-6}
\rule{0pt}{2.5ex}&He \textit{et al.} \cite{he2023geometric} & Heart; Artery; Brain &Geometric Visual Similarity Learning    & MM-WHS-CT; ASOCA; CANDI; STOIC &CVPR 2023\rule[-1.2ex]{0pt}{0pt}\\
\cline{2-6}
\rule{0pt}{2.5ex}&Liu \textit{et al.} \cite{liu2023hierarchical} & Tooth &Hierarchical Global-local Contrastive Learning    & Private Dataset: 13,000 Scans &TMI 2023\rule[-1.2ex]{0pt}{0pt}\\
\cline{2-6}
\rule{0pt}{2.5ex}&Zheng \textit{et al.} \cite{zheng2023msvrl} & Multi-Organ &Multi-scale Visual Representation Self-supervised Learning    & BCV; MSD; KiTS &TMI 2023\rule[-1.2ex]{0pt}{0pt}\\
\cline{2-6}
\rule{0pt}{2.5ex}&Peng \textit{et al.} \cite{peng2024boundary} & Heart; Prostate & Contrastive Learning  & ACDC; PROMISE12 & MIA 2024\rule[-1.2ex]{0pt}{0pt}\\
\cline{2-6}
\rule{0pt}{2.5ex}&Purma \textit{et al.} \cite{purma2024genselfdiff} & Multi-Organ & Diffusion-based Reconstruction & Head and Neck Cancer; GlaS; MoNuSeg & TMI 2024\rule[-1.2ex]{0pt}{0pt}\\
\hline
\rule{0pt}{2.5ex} \multirow{6}{*}{\rotatebox{90}{Regression}} & Abbet \textit{et al.} \cite{abbet2020divide}  & Gland &Image Colorization & Private Dataset: 660 Images&MICCAI 2020 \rule[-1.2ex]{0pt}{0pt}\\
\cline{2-6}
\rule{0pt}{2.5ex}& \multirow{3}{*}{Srinidhi \textit{et al.} \cite{srinidhi2022self}}  & Breast; & WSI Patch Resolution Sequence & BreastPathQ; &MIA 2022\\
&& Colon& Prediction&CAMELYON 2016; &  \\
&&&&KATHER &  \rule[-2ex]{0pt}{0pt}\\
\cline{2-6}
\rule{0pt}{2.5ex}&Fan \textit{et al.} \cite{fan2023cancerself}  &  Brain; Lung  & Image Colorization; Cross-channel   &  GBM; TCGA-LUSC; NLST &TMI 2023\rule[-1.2ex]{0pt}{0pt}\\
\hline
\rule{0pt}{2.5ex} \multirow{32}{*}{\rotatebox{90}{Others}} & Zhuang \textit{et al.} \cite{zhuang2019selfsupervised} & Brain &  Rubik's Cube Recovery & BraTS 2018; Private Dataset: 1,486 Images & MICCAI 2019\rule[-1.2ex]{0pt}{0pt}\\ 
\cline{2-6}
\rule{0pt}{2.5ex}& \multirow{3}{*}{Chen \textit{et al.} \cite{chen2019self}}  & \multirow{3}{*}{Multi-Organ} & \multirow{3}{*}{Disturbed Image Context Restoration} & Private Fetus Dataset: 2,694 Images; &MIA 2019\\
&&&&Private Multi-organ Dataset: 150 Images;& \\
&&&&BraTS 2017& \rule[-1.2ex]{0pt}{0pt}\\
\cline{2-6}
\rule{0pt}{2.5ex}&Zhao \textit{et al.} \cite{zhao2020smore}  &  Brain  & Super-resolution Reconstruction   &  Private Dataset: 47 Images &TMI 2020\rule[-1.2ex]{0pt}{0pt}\\
\cline{2-6}
\rule{0pt}{2.5ex}&Li \textit{et al.} \cite{li2020sacnn} & Abdomen  & CT Reconstruction  & LDCTGC &TMI 2020\rule[-1.2ex]{0pt}{0pt}\\
\cline{2-6}
\rule{0pt}{2.5ex}&Cao \textit{et al.} \cite{cao2020auto}  & Brain & Missing Modality Synthesis & BraTS 2015; ADNI & AAAI 2020\rule[-1.2ex]{0pt}{0pt}\\ 
\cline{2-6}
\rule{0pt}{2.5ex}&Haghighi \textit{et al.} \cite{haghighi2020learning} & Lung &  Self-Discovery + Self-Classification  & LUNA; LiTS; CAD-PE; BraTS 2018; &   MICCAI 2020   \\
&&&+Self-Restoration& ChestX-ray14; LIDC-IDRI; SIIM-ACR& \rule[-1.2ex]{0pt}{0pt}\\
\cline{2-6}
\rule{0pt}{2.5ex}&Taleb \textit{et al.} \cite{taleb20203d}& Brain; Retina;  & 3D Contrastive Predictive Coding; 3D Jigsaw Puzzles; & BraTS 2018; & NeurIPS 2020\\
&&Pancreas &3D Rotation Prediction; 3D Exemplar Networks&DECATHLON;& \\
&&&Relative 3D Patch Location;&DRD& \rule[-1ex]{0pt}{0pt}\\
\cline{2-6}
\rule{0pt}{2.5ex}&Li \textit{et al.} \cite{li2021single} &  Breast; Pancreas; Kidney &Super-resolution Reconstruction;  Color Normalization  & WTS; Private Dataset: 533 Images& MIA 2021 \rule[-1.2ex]{0pt}{0pt}\\
\cline{2-6}
\rule{0pt}{2.5ex}&Wang \textit{et al.} \cite{wang2021transpath} & Multi-Organ & Contrastive Learning & TCGA; KATHER; MHIST & MICCAI 2021\\
&&&&PAIP; PatchCAMELYON& \rule[-1.2ex]{0pt}{0pt} \\
\cline{2-6}
\rule{0pt}{2.5ex}&Zhou \textit{et al.} \cite{zhou2021preservational}& Lung; Brain; Liver & Contrastive Learning + Image Reconstruction & ChestX-ray14; CheXpert; LUNA &  CVPR 2021\\
&&&&BraTS 2018; LiTS;& \rule[-1.2ex]{0pt}{0pt}\\
\cline{2-6}
\rule{0pt}{2.5ex}&Yan \textit{et al.} \cite{yan2022sam}& Multi-Organ  & Global and Local Contrastive Learning  & DeepLesion; NIH LN; Private Dataset: 94 Patients& TMI 2022\rule[-1.2ex]{0pt}{0pt}\\
\cline{2-6}
\rule{0pt}{2.5ex}&Haghighi \textit{et al.} \cite{haghighi2022dira}& Lung & Contrastive Learning + Reconstruction + & ChestX-ray14; CheXpert; & CVPR 2022 \\
&&&Adversarial Learning & Montgomery&  \rule[-1.2ex]{0pt}{0pt}\\
\cline{2-6}
\rule{0pt}{2.5ex}&Cai \textit{et al.} \cite{cai2023dualself}& Lung; Brain; Retina& Dual-Distribution Reconstruction & RSNA-Lung; LAG; VinDr-CXR; Brain Tumor MRI;& MIA 2023 \\
&& &&Private Lung Dataset: 5,000 Images& \rule[-1.2ex]{0pt}{0pt}\\
\cline{2-6}
\rule{0pt}{2.5ex}&Li \textit{et al.} \cite{li2023generic}& Retina  & Frequency-boosted Image Enhancement  &EyePACS;& MIA 2023 \\
&&&&Private Dataset: more than 10,000 Images& \rule[-1.2ex]{0pt}{0pt}\\
% \cline{2-6}
% \rule{0pt}{2.5ex}&Xie \textit{et al.} \cite{xie2022unimiss}$_{2022\text{CS}}$ & Multi-Organ & Contrastive Learning  & BCV; RICORD; &BCV: DSC: 0.8499; RICORD: AUC: 0.8906;\\
% &&&&JSRT Database; ChestXR&JSRT Database: DSC: 0.9408; ChestXR: AUC: 0.9907\rule[-1.2ex]{0pt}{0pt}\\
\bottomrule
\end{tabular}
\begin{tablenotes}    
        \footnotesize               
        \item[1] For the sake of brevity, we denote references that contain more than one task in the following abbreviations: \textbf{C}: Classification, \textbf{S}:Segmentation, \textbf{D}:Detection, \textbf{SR}: Super-resolution, \textbf{DN}: Denoising, \textbf{IT}: Image Translation, \textbf{RE}: Registration. 
      \end{tablenotes}
\end{threeparttable}
}
\label{tab:self}
\end{table*}
\begin{table*}[ht]
\centering
\caption{Overview of Semi-supervised Learning-based Studies in Medical Image Analysis.}

\resizebox{\textwidth}{.49\textwidth}{
\begin{threeparttable}
\begin{tabular}{@{}llllll@{}}
\toprule
 &Reference & Organ &  Semi-SL Algorithm Design & Dataset & Publication \\ \midrule
\rule{0pt}{2ex} \multirow{20}{*}{\rotatebox{90}{Classification}}&  Madani \textit{et al.} \cite{madani2018semi}$_{2018}$ & Lung & Semi-superivsed GAN & NIH PLCO; NIH Chest X-Ray&  ISBI 2018\rule[-1.2ex]{0pt}{0pt} \\ 
\cline{2-6}
\rule{0pt}{2.5ex}&  \rule{0pt}{2.5ex}Diaz-Pinto \textit{et al.} \cite{diaz2019retinal} &  Retina  & Semi-supervised DCGAN  & ORIGA-light; DRISHTI-GS; RIM-ONE; HRF; DRD; &TMI 2019\\
&&&&sjchoi86-HRF; ACRIMA; DRIVE; Messidor &\rule[-1.2ex]{0pt}{0pt}\\
\cline{2-6}
% \rule{0pt}{2.5ex}& Gyawali \textit{et al.} \cite{gyawali2019semi} & Lung & Temporal Ensembling + VAE  & Chexpert& AUC: 0.6697\rule[-1.2ex]{0pt}{0pt}\\
% \cline{2-6}
% \rule{0pt}{2.5ex}& Su \textit{et al.} \cite{su2019local} & Nuclei & Mean Teacher + Label Propagation& MoNuseg; Private Dataset: 17516 nuleus&MoNuseg: F1: 0.783; Private Dataset: F1: 0.7991\rule[-1.2ex]{0pt}{0pt}\\
% \cline{2-6}
\rule{0pt}{2.5ex}& Shi \textit{et al.} \cite{shi2020graph} & Lung; Breast & Graph Temporal Ensembling  & TCGA-Lung; TCGA-Breast&MIA 2020\rule[-1.2ex]{0pt}{0pt}\\
\cline{2-6}
% \rule{0pt}{2.5ex}& Gyawali \textit{et al.} \cite{gyawali2020semi} & Lung; Skin& Mixup on Input and Latent Space & Chexpert; ISIC 2018 &Chexpert: AUC: 0.6847; ISIC: AUC: 0.9073 \rule[-1.2ex]{0pt}{0pt}\\
% \cline{2-6}
% \rule{0pt}{2.5ex}& Kamran \textit{et al.} \cite{kamran2021vtgan}  &  Retina   &     Semi-supervised GAN + ViT &  FFA  &   FID (Fréchet inception distance): 17.3; KID (Kernel inception distance):0.00053 \rule[-1.2ex]{0pt}{0pt}\\
% \cline{2-6}
\rule{0pt}{2.5ex}& Yu \textit{et al.} \cite{yu2021accurate} & Colon & Mean Teacher  & Private Dataset: 13,111 Images &Nature Commun. 2021\rule[-1.2ex]{0pt}{0pt}\\
\cline{2-6}
% \rule{0pt}{2.5ex}& Marini \textit{et al.} \cite{marini2021semi}  &  Prostate  & Teacher-student model & TMA-Zurich; TCGA-PRAD & QWK (Quadratic Weighted Kappa): TMA-Zurich: 0.7645; TCGA-PRAD: 0.4529\rule[-1.2ex]{0pt}{0pt}\\
% \cline{2-6}
\rule{0pt}{2.5ex}& Wang \textit{et al.} \cite{wang2021deephybrid} & Breast; Retina & Virtual Adversarial Training + Self-training & RetinalOCT; Private Dataset: 39,904 Images&  MIA 2021\rule[-1ex]{0pt}{0pt}\\
\cline{2-6}
\rule{0pt}{2.5ex}&Liu \textit{et al.} \cite{liu2022acpl} & Lung; Skin &  Anti-curriculum Self-training &ChestX-Ray14; ISIC 2018&CVPR 2022\rule[-1.2ex]{0pt}{0pt}\\
\cline{2-6}
% \rule{0pt}{2.5ex}& Wen \textit{et al.} \cite{wen2022multi}    &  Brain   & Multi-scale clustering  &  UKBB; ADNI & Acc: 0.89 \rule[-1.2ex]{0pt}{0pt}\\
% \cline{2-6}
\rule{0pt}{2.5ex}& Zhang \textit{et al.} \cite{zhang2022boostmis}    &  Spinal cord   &Consistency Regularization + Pseudo-labeling + Active Learning & Private Dataset: 7,295 Images; & CVPR 2022 \rule[-1.2ex]{0pt}{0pt}\\
\cline{2-6}
\rule{0pt}{2.5ex}&Gao \textit{et al.} \cite{gao2023semi}  &  Multi-Organ   & Dual-task Consistency & TGCA-RCC; TCGA-BR; TGCA-LU& MIA 2023 \rule[-1.2ex]{0pt}{0pt}\\
\cline{2-6}
\rule{0pt}{2.5ex}&Zeng \textit{et al.} \cite{zeng2023pefat}  &  Colon; Skin; Chest  & Self-training + Feature Adversarial Training &  NCT-CRC-HE; ISIC 2018; Chest X-Ray14 & CVPR 2023 \rule[-1.2ex]{0pt}{0pt}\\
\cline{2-6}
\rule{0pt}{2.5ex}&Xie \textit{et al.} \cite{xie2023fundus}  &  Retina   & Semi-supervised GAN & iChallenge; ODIR& TMI 2023 \rule[-1.2ex]{0pt}{0pt}\\
\cline{2-6}
\rule{0pt}{2.5ex}&Yang \textit{et al.} \cite{yang2023hierarchical}  &  Multi-Organ   & Self-training & KC Dataset; ISIC 2018; RSNA Dataset& TMI 2023 \rule[-1.2ex]{0pt}{0pt}\\
\cline{2-6}
\rule{0pt}{2.5ex}& Huang \textit{et al.} \cite{huang2024systematic}  &  Multi-Organ   & Pseudo Label; Mean Teacher; MixMatch; FixMatch; FlexMatch; CoMatch & TissueMNIST; PathMNIST; TMED-2; AIROGS& CVPR 2024 \rule[-1.2ex]{0pt}{0pt}\\
\cline{2-6}
\rule{0pt}{2.5ex}& Berenguer \textit{et al.} \cite{berenguer2024semi}  & Lung; Skin; Knee & Consistency Regularization; Graph Attention & NIH-14 Chest X-ray; ISIC 2018; COV19-CT; Knee MRNet& MIA 2024 \rule[-1.2ex]{0pt}{0pt}\\
\hline
\rule{0pt}{2.5ex} \multirow{38}{*}{\rotatebox{90}{Segmentation}}& Bai \textit{et al.} \cite{bai2017semi}$_{2017}$ & Heart &  CRF-based Self-training   & Private Dataset: 8050 Images&MICCAI 2017 \rule[-1.2ex]{0pt}{0pt}\\
% \cline{2-6}
% \rule{0pt}{2.5ex}& Sedai \textit{et al.} \cite{sedai2017semi}$_{2017}$ & Retina & Teacher-student VAE  &  DRD & DSC: 0.80%$\pm$0.03 
% \rule[-1.2ex]{0pt}{0pt}\\
\cline{2-6}
\rule{0pt}{2.5ex} &    Li \textit{et al.} \cite{li2018semipi}$_{2018}$ &  Skin & $\Pi$-model & ISIC 2017 & ArXiv 2018 \rule[-1.2ex]{0pt}{0pt}\\
\cline{2-6}
\rule{0pt}{2.5ex}& Nie \textit{et al.} \cite{nie2018asdnet}$_{2018}$ & Prostate & Self-training & Private Dataset: 70 Images &MICCAI 2018 \rule[-1.2ex]{0pt}{0pt}\\
\cline{2-6}
\rule{0pt}{2.5ex}& Yu \textit{et al.} \cite{yu2019uncertainty} &  Heart  & Uncertainty-aware Mean Teacher  & ASG  & MICCAI 2019 \rule[-1.2ex]{0pt}{0pt}\\ 
\cline{2-6} 
% \rule{0pt}{2.5ex}&Sedai \textit{et al.} \cite{sedai2019uncertainty} &  Retina  & Uncertainty-aware Teacher-student model  & Private Dataset: 570 Images&DSC: 0.91\rule[-1.2ex]{0pt}{0pt}\\ 
% \cline{2-6}
\rule{0pt}{2.5ex}&Zhou \textit{et al.} \cite{zhou2019semi} & Multi-Organ &  Multi-planar Co-training   & Private Dataset: 310 Volumes& WACV 2019 \rule[-1.2ex]{0pt}{0pt}\\
\cline{2-6}
% \rule{0pt}{2.5ex}&Zhao \textit{et al.} \cite{zhao2019multi} & Brain & Co-training + Knowledge Distillation & MALC; OASIS & MALC: DSC: 0.792\rule[-1.2ex]{0pt}{0pt}\\
% \cline{2-6}
% \rule{0pt}{2.5ex}&Cao \textit{et al.} \cite{cao2020uncertainty} &  Breast  & Uncertainty-aware Temporal Ensembling  & Private Dataset: 170 Volumes; ISIC 2017 &Private Dataset: DSC: 0.7287; ISIC 2017: DSC: 0.8178\rule[-1.2ex]{0pt}{0pt}\\
% \cline{2-6}
\rule{0pt}{2.5ex}&Li \textit{et al.} \cite{li2020transformation} &  Liver; Retina; Skin  & Transformation-consistent Mean Teacher& ISIC 2017; REFUGE; LiTS& TNNLS 2020\rule[-1.2ex]{0pt}{0pt}\\
\cline{2-6}
\rule{0pt}{2.5ex}&Liu \textit{et al.} \cite{liu2020semi} &  Skin; Lung & Mean Teacher + Sample Relation Consistency &  ISIC 2018; ChestX-ray14& TMI 2020 \rule[-1.2ex]{0pt}{0pt}\\
\cline{2-6}
% \rule{0pt}{2.5ex}&Xie \textit{et al.} \cite{xie2020pairwise} & Colon & Pairwise Relation Network  & GlaS; CRAG& GlaS: DSC: 0.906; CRAG: DSC: 0.892\rule[-1.2ex]{0pt}{0pt}\\
% \cline{2-6}
\rule{0pt}{2.5ex}&Li \textit{et al.} \cite{li2020shape} &  Heart & Shape-aware Consistency Regularization  & ASG& MICCAI 2020\rule[-1.2ex]{0pt}{0pt}\\
\cline{2-6}
% \rule{0pt}{2.5ex}&Wang \textit{et al.} \cite{wang2020double} & Heart; Kidney & Double Uncertainty-Weighted Mean Teacher  & ASG; KiTS& ASG: DSC: 0.8965; KiTS: DSC: 0.8879\rule[-1.2ex]{0pt}{0pt}\\
% \cline{2-6}
% \rule{0pt}{2.5ex}&Zhou \textit{et al.}\cite{zhou2020deep} & Cervix  & Masked Mean Teacher   & Private Dataset: 5,480 Images&AJI (Averaged Jaccard Index): 0.6694; mAP: 0.4052 \rule[-1.2ex]{0pt}{0pt}\\
% \cline{2-6}
% \rule{0pt}{2.5ex}&Qu \textit{et al.}\cite{qu2020weakly} & Multi-Organ & Self-training + Background Propagation & MoNuSeg; Private Dataset: 40 Images & MoNuSeg: F1: 0.8282; Private Dataset: F1: 0.8879 \rule[-1.2ex]{0pt}{0pt}\\
% \cline{2-6}
% \rule{0pt}{2.5ex}& Xia \textit{et al.} \cite{xia2020uncertainty} & Pancreas; Abdomen & Uncertainty-aware Co-training   & NIH PCT; MoCTSeg&NIH PCT: DSC: 0.8118\rule[-1.2ex]{0pt}{0pt}\\
% \cline{2-6}
% \rule{0pt}{2.5ex}&Li \textit{et al.} \cite{li2020selftraining} & Multi-Organ &  Self-training + Self-supervised Learning  & MoNuSeg; ISIC 2017&MoNuSeg: F1: 0.7910; ISIC: F1: 0.8617\rule[-1.2ex]{0pt}{0pt}\\
% \cline{2-6}
\rule{0pt}{2.5ex}&Fan \textit{et al.} \cite{fan2020inf} & Lung &  Attention Self-training & IS-COVID & TMI 2020\rule[-1.2ex]{0pt}{0pt}\\
\cline{2-6}
% \rule{0pt}{2.5ex}&Fang \textit{et al.} \cite{fang2020dmnet} & Kidney; Brain &  End-to-end Co-training & KiTS; BraTS 2018 & KiTS: mIoU: 0.902; BraTS: mIoU: 0.854\rule[-1.2ex]{0pt}{0pt}\\
% \cline{2-6}
\rule{0pt}{2.5ex}&Chaitanya \textit{et al.} \cite{chaitanya2021semi}  &  Heart; Prostate; Pancreas   &  Semi-supervised GAN + Deformation and Additive Intensity Field & ACDC; DECATHLON&MIA 2021\rule[-1.2ex]{0pt}{0pt}\\
\cline{2-6}
% \rule{0pt}{2.5ex}&Xu \textit{et al.} \cite{xu2021shadow} &  Prostate & Shadow-consistent Mean Teacher   & Prostate-MRI-US-Biopsy; Private Dataset: 662 volumes&Prostate: DSC: 0.9225; Private Dataset: DSC: 0.9012\rule[-1.2ex]{0pt}{0pt}\\
% \cline{2-6}
% \rule{0pt}{2.5ex}&Huo \textit{et al.} \cite{huo2021atso} & Pancreas & Asynchronous Teacher-student  & NIH PCT; DECATHLON &NIH PCT: DSC: 0.8370; DECATHLON: DSC: 0.7671\rule[-1.2ex]{0pt}{0pt}\\
% \cline{2-6}
% \rule{0pt}{2.5ex}&Wang \textit{et al.} \cite{wang2021selfco} & Heart; Spleen; Prostate & Self-paced Co-training  & ACDC; DECATHLON; ProMRI&DSC: ACDC: 0.8778; DECATHLON: 0.8835; ProMRI: 0.7616\rule[-1.2ex]{0pt}{0pt}\\
% \cline{2-6}
\rule{0pt}{2.5ex}&Luo \textit{et al.} \cite{luo2021efficient}   &  Nasopharynx   &   Uncertainty Rectified Pyramid Consistency  &   Private Dataset: 258 MR Images &   MICCAI 2021\rule[-1.2ex]{0pt}{0pt}\\
\cline{2-6}
% \rule{0pt}{2.5ex}&Shi \textit{et al.} \cite{shi2021inconsistency}  &  Pancreas; Heart   &     Uncertainty Estimation + Self-training & NIH PCT; ACDC; Endocardium MRI & DSC: NIH PCT: 0.6701; ACDC: 0.8610; Endocardium MRI: 0.8667 \rule[-1.2ex]{0pt}{0pt}\\
% \cline{2-6}
% \rule{0pt}{2.5ex}&Hu \textit{et al.} \cite{hu2021semi} &  Hippocampus; Heart  & Semi-supervised Contrastive Learning & DECATHLON; MM-WHS & DECATHLON: DSC: 0.866; MM-WHS: DSC: 0.764;\rule[-1.2ex]{0pt}{0pt}\\
% \cline{2-6}
\rule{0pt}{2.5ex}&Luo \textit{et al.} \cite{luo2021semi}  &  Heart;   & Dual-task Consistency & ASG; NIH PCT & AAAI 2021\rule[-1.2ex]{0pt}{0pt}\\
\cline{2-6}
% \rule{0pt}{2.5ex}&Li \textit{et al.} \cite{li2021dualconsistency}    &  Lung   & Uncertainty-guided Dual Consistency& Private Dataset: 852 Volumes & DSC: 0.774; JI: 0.645\rule[-1.2ex]{0pt}{0pt}\\
% \cline{2-6}
\rule{0pt}{2.5ex}&Li \textit{et al.} \cite{li2021semantic} & Lung; Skin; Liver& StyleGAN2 & ChestX-ray14; JSRT Database; ISIC 2018; LiTS; CHAOS& CVPR 2021\rule[-1.2ex]{0pt}{0pt}\\
\cline{2-6}
\rule{0pt}{2.5ex}&You \textit{et al.} \cite{you2022simcvd} &  Heart; Pancreas& Mean Teacher + Contrastive Learning  & ASG; NIH PCT& TMI 2022\rule[-1.2ex]{0pt}{0pt}\\
\cline{2-6}
\rule{0pt}{2.5ex}&Wang \textit{et al.} \cite{wang2022semi} &  Heart; Prostate & Mean Teacher + Contrastive Learning& ACDC; ProMRI &MIA 2022\rule[-1.2ex]{0pt}{0pt}\\
\cline{2-6}
\rule{0pt}{2.5ex}&Wu \textit{et al.} \cite{wu2022mutual}  &  Heart; Pancreas  & Uncertainty-based Mutual Consistency  &   ASG; NIH PCT; ACDC & MIA 2022\rule[-1.2ex]{0pt}{0pt}\\
\cline{2-6}
% \rule{0pt}{2.5ex}&Wu \textit{et al.} \cite{wu2022minimizing} &  Heart; Retina  & Risk Minimization  & ACDC; REFUGE; MM-WHS &  DSC: ACDC: 0.8392; REFUGE: 0.9126; MM-WHS: 0.8317\rule[-1.2ex]{0pt}{0pt}\\
% \cline{2-6}
% \rule{0pt}{2ex}&Wang \textit{et al.} \cite{wang2022rethinking}  &  Kidney; Heart; Liver & Generative Bayesian Deep Learning & KiTS; ASG; DECATHLON &  DSC: KiTS: 0.898; ASG: 0.884; DECATHLON: 0.935\rule[-1.2ex]{0pt}{0pt}\\
% \cline{2-6}
% \rule{0pt}{2.5ex}&Gao \textit{et al.} \cite{gao2022segmentation} &  Heart   & Mean Teacher & ACDC & DSC: 0.7954 \rule[-1.2ex]{0pt}{0pt}\\
% \cline{2-6}
\rule{0pt}{2.5ex}&Luo \textit{et al.} \cite{luo2022semi}  &  Heart   & Co-training Variant & ACDC & MIDL 2022 \rule[-1.2ex]{0pt}{0pt}\\
\cline{2-6}
\rule{0pt}{2.5ex}&Lei \textit{et al.} \cite{lei2022semi}  & Liver; Skin & Adversarial Consistency + Dynamic Convolution Network & LiTS; ISIC 2018 & TMI 2022\rule[-1.2ex]{0pt}{0pt}\\
\cline{2-6}
\rule{0pt}{2.5ex}&Meng \textit{et al.} \cite{meng2022dual}  &  Multi-Organ   & Consistency Regularization + Adaptive Graph Neural Network & SEG (Combined by Refuge; Drishti-GS; ORIGA; RIGA; & TMI 2022 \rule[-1.2ex]{0pt}{0pt}\\
\cline{2-6}
\rule{0pt}{2.5ex}&Shi \textit{et al.} \cite{shi2023deep}  &  Multi-Organ   & Consistency Regularization + Teacher-student Model & CVC; ETIS-Larib Polyp; Private Dataset: 1,100 images & MIA 2023 \rule[-1.2ex]{0pt}{0pt}\\
\cline{2-6}
\rule{0pt}{2.5ex}&Xu \textit{et al.} \cite{xu2023ambiguity}  &  Multi-Organ   & Mean Teacher & Left Atrium (LA) Dataset; BraTS 2019& MIA 2023 \rule[-1.2ex]{0pt}{0pt}\\
% \cline{2-6}
% \rule{0pt}{2.5ex}&Adiga \textit{et al.} \cite{adiga2023anatomically}  &  Multi-Organ   & Uncertainty-based Teacher-student Model & Left Atrium (LA) Dataset; FLARE 2021& DSC: 0.8660 \rule[-1.2ex]{0pt}{0pt}\\
\cline{2-6}
\rule{0pt}{2.5ex}&Bai \textit{et al.} \cite{bai2023bidirectional}  &  Multi-Organ  & Mean Teacher & Left Atrium (LA) Dataset; Pancreas-NIH; ACDC & CVPR 2023 \rule[-1.2ex]{0pt}{0pt}\\
\cline{2-6}
\rule{0pt}{2.5ex}&Miao \textit{et al.} \cite{miao2023caussl}  &  Multi-Organ   & Causality Co-training & ACDC; Pancreas-CT; BraTS 2019& ICCV 2023\rule[-1.2ex]{0pt}{0pt}\\
\cline{2-6}
\rule{0pt}{2.5ex}&Chaitanya \textit{et al.} \cite{chaitanya2023local}  &  Heart; Prostate &Self-training & ACDC; MICCAI 2019; MMWHS & MIA 2023\rule[-1.2ex]{0pt}{0pt}\\
\cline{2-6}
\rule{0pt}{2.5ex}&Wang \textit{et al.} \cite{wang2023mcf}  & Heart; Pancreas & Co-training & Left Atrial (LA) Dataset; NIH-Pancreas & CVPR 2023\rule[-1.2ex]{0pt}{0pt}\\
\cline{2-6}
\rule{0pt}{2.5ex}&Basak \textit{et al.} \cite{basak2023pseudo}  & Heart; Kidney; Gland & Contrastive Self-training & ACDC; KiTS19; CRAG & CVPR 2023\rule[-1.2ex]{0pt}{0pt}\\
\cline{2-6}
\rule{0pt}{2.5ex}&Zhang \textit{et al.} \cite{zhang2023multi}  & Heart; Pancreas & Co-training & Left Atrial (LA) Dataset; NIH-Pancreas & MIA 2023\rule[-1.2ex]{0pt}{0pt}\\
\cline{2-6}
\rule{0pt}{2.5ex}&Chen \textit{et al.} \cite{chen2022semi}  & Heart; Brain & Task-specific Consistency Regularization & MICCAI 2018; DECATHLON & TMI 2022\rule[-1.2ex]{0pt}{0pt}\\
\cline{2-6}
\rule{0pt}{2.5ex}&Bashir \textit{et al.} \cite{bashir2024consistency}  &  Multi-Organ   & Consistency Regularization & MoNuSeg; BCSS & MIA 2024\rule[-1.2ex]{0pt}{0pt}\\
\cline{2-6}
\rule{0pt}{2.5ex}&Su \textit{et al.} \cite{su2024mutual}  &  Heart; Stomach; Brain & Mutual Learning; Pseudo Label & Left Atrium (LA) Dataset; Pancreas-CT Dataset; Brats-2019 & MIA 2024\rule[-1.2ex]{0pt}{0pt}\\
\cline{2-6}
\rule{0pt}{2.5ex}&Chi \textit{et al.} \cite{chi2024adaptive}  &  Heart; Prostate & Consistency Regularization & ACDC; PROMISE12 & CVPR 2024\rule[-1.2ex]{0pt}{0pt}\\
\cline{2-6}
\rule{0pt}{2.5ex}&Ma \textit{et al.} \cite{ma2024constructing}  & Eye; Prostate; Heart & Mean Teacher with Domain Knowledge Transfer & Fundus dataset; Prostate dataset; M\&Ms dataset  & CVPR 2024\rule[-1.2ex]{0pt}{0pt}\\
\cline{2-6}
\rule{0pt}{2.5ex}&Miao \textit{et al.} \cite{miao2024cross}  &  Breast; Heart & Prompting Segment Anything Model & BUSI; ACDC  & MICCAI 2024\rule[-1.2ex]{0pt}{0pt}\\
\hline
\rule{0pt}{2.5ex} \multirow{4}{*}{\rotatebox{90}{Detection}} & Wang \textit{et al.} \cite{wang2020focalmix} & Lung & MixMatch + Focal Loss & LUNA; NLST & CVPR 2020\rule[-1.2ex]{0pt}{0pt}\\
\cline{2-6}
\rule{0pt}{2.5ex}&Zhou \textit{et al.} \cite{zhou2021ssmd} &  Multi-Organ  & Teacher-student Model + Adaptive Consistency Loss & DSB; DeepLesion & MIA 2021\rule[-1.2ex]{0pt}{0pt}\\ 
\cline{2-6}
\rule{0pt}{2.5ex}&Zhang \textit{et al.} \cite{zhang2024spatial} & Lung; Eye & GAN with Spatial Attention & RSNA Dataset; VinDr-CXR & MICCAI 2024\rule[-1.2ex]{0pt}{0pt}\\ 
\bottomrule
\end{tabular}
\end{threeparttable}}
\label{tab:semi}
\end{table*}
\begin{table*}[ht]
\centering
\caption{Overview of Multi-instance Learning-based Studies in Medical Image Analysis}

\resizebox{\textwidth}{.33\textwidth}{
\begin{threeparttable}
\begin{tabular}{@{}llllll@{}}
\toprule
 &Reference$_{\text{Year}}$ & Organ & MIL Algorithm Design & Dataset & Result \\ \midrule
\rule{0pt}{2ex} \multirow{43}{*}{\rotatebox{90}{Classification}}&  Manivannan \textit{et al.} \cite{manivannan2017subcategory}$_{2017}$ &  Retina;   & Discriminative Subspace Transformation +  & Messidor; TMA-UCSB;  &Messidor: Acc: 0.728; TMA-UCSB: AUC: 0.967;\\
&&Breast&Margin-based Loss&DR Dataset; Private Dataset: 884 Images&DR Dataset: Acc: 0.8793; Private: Kappa: 0.7212\rule[-1.2ex]{0pt}{0pt}\\
\cline{2-6}
% \rule{0pt}{3ex}&Zhu \textit{et al.} \cite{zhu2017deep}$_{2017}$ & Breast & Sparse MIL & INBreast & AUC: 0.89\rule[-1.2ex]{0pt}{0pt}\\
% \cline{2-6}
% \rule{0pt}{3ex}& \rule{0pt}{2.5ex}Mercan \textit{et al.} \cite{mercan2017multi}$_{2017}$& Breast & Multi-Label MIL & BCSC & Average-P (Average-Precision): 0.8068\rule[-1.2ex]{0pt}{0pt}\\
% \cline{2-6}
\rule{0pt}{2.5ex}&Ilse \textit{et al.} \cite{ilse2018attention}$_{2017}$ &  Breast; Colon & Attention-based MIL   &  TMA-UCSB; CRCHistoPhenotypes &TMA-UCSB: Acc: 0.755; CRCHistoPhenotypes: Acc: 0.898\rule[-1.2ex]{0pt}{0pt}\\
\cline{2-6}
\rule{0pt}{2.5ex}&Couture \textit{et al.} \cite{couture2018multiple}$_{2018}$ &  Breast  & Quantile Function-based MIL   &  CBCS3  &Acc: 0.952\rule[-1.2ex]{0pt}{0pt}\\
\cline{2-6}
% \rule{0pt}{2.5ex}&Das \textit{et al.} \cite{das2018multiple}$_{2018}$ &  Breast &  Multiple Instance Pooling   & BreakHis & Acc: 0.8906\rule[-1.2ex]{0pt}{0pt}\\
% \cline{2-6}
\rule{0pt}{2.5ex}&Liu \textit{et al.} \cite{liu2018landmark}$_{2018}$  & Brain & Landmark-based MIL    & ADNI; MIRIAD & ADNI: AUC: 0.9586; MIRIAD: AUC: 0.9716\rule[-1.2ex]{0pt}{0pt}\\
\cline{2-6}
\rule{0pt}{2.5ex}&Campanella \textit{et al.} \cite{campanella2019clinical}$_{2019}$ &  Prostate; Skin; Lymph  & MIL + RNN   &  Private Dataset: 44,732 Images  & AUC: Prostate: 0.986; Skin: 0.986; Lymph: 0.965\rule[-1.2ex]{0pt}{0pt}\\
\cline{2-6}
\rule{0pt}{2.5ex}&Wang \textit{et al.} \cite{wang2019rmdl}$_{2019}$ &  Breast &  Instance Features Recalibration   & Private Dataset: 608 Images & Acc: 0.865\rule[-1.2ex]{0pt}{0pt}\\
\cline{2-6}
% \rule{0pt}{2.5ex}&Tennakoon \textit{et al.} \cite{tennakoon2019classification}$_{2019}$ & Retina; Lung & Extreme Value Theorem-based MIL & ReTOUCH; DLCST & DLCST: AUC: 0.96\\
% \cline{2-6}
\rule{0pt}{2.5ex}&Yao \textit{et al.} \cite{yao2019deep}$_{2019}$ & Lung; Brain & Multiple Instance FCN & NLST; TCGA&NLST: C-Index: 0.678; TCGA: C-Index: 0.657\rule[-1.2ex]{0pt}{0pt}\\
\cline{2-6}
\rule{0pt}{2.5ex}&Wang \textit{et al.} \cite{wang2020ud}$_{2020}$  &  Retina  & Uncertainty-aware MIL + RNN Aggregation  &  Duke-AMD; Private Dataset: 4,644 Volumes & Acc: Duke-AMD: 0.979; Private Dataset: 0.951 \rule[-1.2ex]{0pt}{0pt}\\
\cline{2-6}
\rule{0pt}{2.5ex}&Zhao \textit{et al.} \cite{zhao2020predicting}$_{2020}$  &  Colon & VAE-GAN Feature Extraction +    & TCGA-COAD & Acc: 0.6761; F1: 0.6667;\\
&&&GNN Bag-level Representation Learning&&AUC: 0.7102\rule[-1.2ex]{0pt}{0pt}\\
\cline{2-6}
\rule{0pt}{2.5ex}&Chikontwe \textit{et al.} \cite{chikontwe2020multiple}$_{2020}$ & Colon & Jointly Learning of Instance- and Bag-level Feature   & Private Dataset: 366 Images & F1: 0.9236; P (Precision): 0.9254; R (Recall): 0.9231; Acc: 0.9231\rule[-1.2ex]{0pt}{0pt}\\
\cline{2-6}
\rule{0pt}{2.5ex}&Raju \textit{et al.} \cite{raju2020graph}$_{2020}$ & Colon & Graph Attention MIL  & MCO& Acc: 0.811; F1: 0.798\rule[-1.2ex]{0pt}{0pt}\\
\cline{2-6}
\rule{0pt}{2.5ex}&Han \textit{et al.} \cite{han2020accurate}$_{2020}$ & Lung & Automatic Instance Generation & Private Dataset: 460 Examples & AUC: 0.99\rule[-1.2ex]{0pt}{0pt}\\
\cline{2-6}
\rule{0pt}{2.5ex}&Yao \textit{et al.} \cite{yao2020whole}$_{2020}$  & Lung; Colon & Siamese Multi-instance FCN + Attention MIL  & NLST; MCO &NLST: AUC: 0.7143; MCO: AUC: 0.644\rule[-1.2ex]{0pt}{0pt}\\
\cline{2-6}
\rule{0pt}{2.5ex}&Hashimoto \textit{et al.} \cite{hashimoto2020multi}$_{2020}$ & Lymph & Domain Adversarial + Multi-scale MIL  & Private Dataset: 196 Images & Acc: 0.871 \rule[-1.2ex]{0pt}{0pt}\\
\cline{2-6}
\rule{0pt}{2.5ex}&Shao \textit{et al.} \cite{shao2021transmil}$_{2021}$  &  Breast; Lung; Kidney   & Transformer-based MIL  & CAMELYON 2016; TCGA-NSCLC; TCGA-RCC&Acc: CAMELYON: 0.8837; TCGA-NSCLC: 0.8835; TCGA-RCC: 0.9466\rule[-1.2ex]{0pt}{0pt}\\
\cline{2-6}
\rule{0pt}{2.5ex}&Li \textit{et al.} \cite{li2021dual}$_{2021}$ & Breast; Lung & Dual-stream MIL + Contrastive Learning  & CAMELYON 2016; TCGA Lung Cancer &CAMELYON 2016: AUC: 0.9165; TCGA: AUC: 0.9815\rule[-1.2ex]{0pt}{0pt}\\
\cline{2-6}
\rule{0pt}{2.5ex}&Li \textit{et al.} \cite{li2021novel}$_{2021}$ & Lung & Virtual Bags + Self-SL Location Prediction  & Private Dataset: 460 Examples & AUC: 0.981; Acc: 0.958; F1: 0.895; Sens: 0.936 \rule[-1.2ex]{0pt}{0pt}\\
\cline{2-6}
\rule{0pt}{2.5ex}&Lu \textit{et al.} \cite{lu2021data}$_{2021}$ & Kidney; Lung;& Attention-based MIL + Clustering  & TCGA-RCC + Private Dataset: 135 WSIs; & Kidney: AUC: 0.972;\\
&&Lymph node&&CPTAC-NSCLC + Private Dataset: 131 WSIs;&Lung: AUC: 0.975;\\
&&&&CAMELYON 2016,17 + Private Dataset: 133 WSIs&Lymph node: AUC: 0.940\rule[-1.2ex]{0pt}{0pt}\\
\cline{2-6}
\rule{0pt}{2.5ex}&Wang \textit{et al.} \cite{wang2022lymph}$_{2022}$ & Thyroid & Transformer-based MIL + Knowledge Distillation  & Private Dataset: 595 Images &AUC: 0.9835; P: 0.9482; R: 0.9151; F1: 0.9297\rule[-1.2ex]{0pt}{0pt}\\
\cline{2-6}
\rule{0pt}{2.5ex}&Zhang \textit{et al.} \cite{zhang2022dtfd}$_{2022}$ & Breast; Lung  & Double-Tier Feature Distillation MIL & CAMELYON 2016; TCGA-Lung &CAMELYON 2016: AUC: 0.946; TCGA-Lung: AUC: 0.961 \rule[-1.2ex]{0pt}{0pt}\\
\cline{2-6}
\rule{0pt}{2.5ex}&Schirris \textit{et al.}\cite{schirris2022deepsmile}$_{2022}$ & Breast; Colon & Heterogeneity-aware MIL + Contrastive Learning  & TCGA-CRCk; TCGA-BC  &TCGA-CRCk: AUC: 0.87; TCGA-BC: AUC: 0.81\rule[-1.2ex]{0pt}{0pt}\\
\cline{2-6}
\rule{0pt}{2.5ex}&Su \textit{et al.} \cite{su2022attention2majority}$_{2022}$ & Breast; Kidney & Intelligent Sampling Method + Attention MIL & CAMELYON 2016; Private Dataset: 112 Images &CAMELYON 2016: AUC: 0.891; Private: AUC: 0.974\rule[-1.2ex]{0pt}{0pt}\\
\cline{2-6}
\rule{0pt}{2.5ex}&Zhu \textit{et al.} \cite{zhu2022murcl}$_{2022}$ & Breast; Lung; Kidney& Reinforcement Learning + Contrastive Learning + MIL  & CAMELYON 2016; TCGA-Lung; TCGA-Kidney & AUC: CAMELYON: 0.9452; TCGA-Lung: 0.9637; TCGA-Kidney:  0.9573\rule[-1.2ex]{0pt}{0pt}\\
\cline{2-6}
\rule{0pt}{2.5ex}&Yang \textit{et al.} \cite{yang2022micl}$_{2022}$ & Colon; Muscle & Curriculum Learning + MIL  & CRCHistoPhenotypes; Private Muscle Dataset: 266 Images&CRCHistoPhenotypes: AUC: 0.898; Private: AUC: 0.907\rule[-1.2ex]{0pt}{0pt}\\
\cline{2-6}
\rule{0pt}{2.5ex}&Shi \textit{et al.} \cite{shi2023structure}$_{2023}$ & Stomach; Bladder & Multi-scale Graph MIL  & TCGA-STAD; TCGA-BLCA; Private Stomach Dataset: 574 Images&AUC: TCGA-STAD: 0.829; TCGA-BLCA: 0.886; Private: 0.907\rule[-1.2ex]{0pt}{0pt}\\
\cline{2-6}
\rule{0pt}{2.5ex}&Yan \textit{et al.} \cite{yan2023genemutation}$_{2023}$ & Bladder & Hierarchical Deep MIL  & TCGA-Bladder & TCGA-Bladder: AUC: 0.92\rule[-1.2ex]{0pt}{0pt}\\
\cline{2-6}
\rule{0pt}{2.5ex}&Shi \textit{et al.} \cite{shi2023mg}$_{2023}$ & Breast; Kidney& Multi-scale Transformer + MIL  & BRIGHT; TCGA-BRCA; TCGA-RCC & AUC: BRIGHT: 0.848; TCGA-BRCA: 0.921; TCGA-RCC: 0.990\rule[-1.2ex]{0pt}{0pt}\\
\cline{2-6}
\rule{0pt}{2.5ex}&Liu \textit{et al.} \cite{liu2024advmil}$_{2024}$ & Lung; Breast; Brain & GAN + MIL  & NLST; TCGA-BRCA; TCGA-LGG&C-Index: NLST:  0.672; TCGA-BRCA: 0.566; TCGA-LGG: 0.642\rule[-1.2ex]{0pt}{0pt}\\
\hline
\rule{0pt}{3.5ex} \multirow{2}{*}{\rotatebox{90}{Segmentation}}& Jia \textit{et al.} \cite{jia2017constrained}$_{2017}$ &  Colon  & Multi-scale MIL + Area Constraint Regularization &  Private TMA/Colon Dataset: 60 Images/910 Images &F1: TMA: 0.622; Colon: 0.836\rule[-2ex]{0pt}{0pt}\\
\cline{2-6}
\rule{0pt}{3.5ex}&Xu \textit{et al.} \cite{xu2019camel}$_{2019}$ &  Breast  & Instance-level and Pixel-level Label Generation &  CAMELYON 2016 & Image-level Acc: 0.929; Pixel-level IoU: 0.847\rule[-2ex]{0pt}{0pt}\\
\cline{2-6}
\rule{0pt}{3.5ex} &  Dov \textit{et al.} \cite{dov2021weakly}$_{2021}$ &  Thyroid  & Maximum Likelihood Estimation-based MIL &  Private Dataset: 908 Images& AUC: 0.87\rule[-2ex]{0pt}{0pt}\\
\hline
\rule{0pt}{2.5ex} \multirow{4}{*}{\rotatebox{90}{Others}} &Schwab \textit{et al.} \cite{schwab2020localization}$_{2020\text{CD}}$ &  Lung  & Jointly Classification and Localization &  RSNA-Lung; MIMIC-CXR; Private Dataset: 1,003 Images&  AUC: 0.93\rule[-1.2ex]{0pt}{0pt}\\
\cline{2-6}
\rule{0pt}{2.5ex}&Wang \textit{et al.} \cite{wang2021learning}$_{2021\text{CS}}$  &  Pancreas  & Jointly Global-level Classification and    &  Private Dataset: 800 Images  & DSC: 0.6029; \\
&&&Local-level Segmentation&&Sens: 0.9975\rule[-1.2ex]{0pt}{0pt}\\
\bottomrule
\end{tabular}
\begin{tablenotes}    
        \footnotesize               
        \item[1] For the sake of brevity, we denote references that contain more than one task in the following abbreviations: \textbf{C}: Classification, \textbf{S}:Segmentation, \textbf{D}:Detection. 
        % \item[1] More Semi-supervised learning studies are included in Table \ref{tab:semi2}.
      \end{tablenotes}
\end{threeparttable}
}
\label{tab:mil}
\end{table*}
\begin{table*}[!ht]
\centering
\caption{Overview of Annotation-Efficient Learning Studies in Medical Image Analysis}
\resizebox{\textwidth}{.23\textwidth}{
\begin{tabular}{@{}lllllll@{}}
\toprule
& Reference (Year) & Application & Organ & Method & Dataset & Results \\
\midrule
\rule{0pt}{2ex} \multirow{15}{*}{\rotatebox{90}{Tag}} \rule{0pt}{2ex}&\citet{hwang2016self}                                  & Detection                   & Lung; Breast                                  & CAM + Self-Transfer Learning                                & Private Dataset: 11K X-rays;                             & AP Shenzhen set: 0.872;  \rule[-1ex]{0pt}{0pt}\\
\rule{0pt}{2ex} &&&&&  DDSM; MIAS  & MC set: 0.892; MIAS set: 0.326\rule[-1ex]{0pt}{0pt}\\
\cline{2-7}
% --------------------------------------------------------------
\rule{0pt}{2ex} &\citet{gondal2017weakly}                              & Detection                   & Eye                                           & CAM                                                        & DRD; DiaretDB1   & Hemorrhages SE: 0.91; FP s/I 1.5; Hard Exudates SE: 0.87; FPs/I 1.5; \\
&&&&&& Soft Exudates SE: 0.89; FPs/I: 1.5; RSD SE: 0.52; FPs/I: 1.5 \rule[-1ex]{0pt}{0pt}\\
\cline{2-7}
% --------------------------------------------------------------
\rule{0pt}{2ex} &\citet{wang2018weakly}                                  & Detection                   & Eye                                           & Expectation-Maximization + & DRD; Messidor                                            & mAP Kaggle: 0.8394; Messidor: 0.9091  \rule[-1ex]{0pt}{0pt}\\
\rule{0pt}{2ex} & &  & & Low-Rank Subspace Learning                          & & \rule[-1ex]{0pt}{0pt}\\
\cline{2-7}
% --------------------------------------------------------------
&\rule{0pt}{2ex}\citet{nguyen2019novel}                               & Segmentation                & Eye                                           & CAM + CRF + Active Shape Model                               & Private Dataset: 40 MRI Images                                              & DSC: T1w: 0.845±0.056; T2w: 0.839±0.049\rule[-1ex]{0pt}{0pt}\\
\cline{2-7}
% --------------------------------------------------------------
\rule{0pt}{2ex} &\citet{wang2020weakly}                                  & Detection                   & Lung                                          & CAM + Unsupervised Segmentation                             & Private Dataset: 540 CT Images                                          & Hit Rate: 0.865\rule[-1ex]{0pt}{0pt}\\
\cline{2-7}
% --------------------------------------------------------------
\rule{0pt}{2ex} &\citet{shen2021interpretable}                           & Detection                   & Breast                                        & Globally-aware Multiple Instance                 & NYUBCS; CBIS-DDSM                                           & DSC malignant: 0.325 ± 0.231; DSC Benign: 0.240 ± 0.175; \rule[-1ex]{0pt}{0pt}\\
\rule{0pt}{2ex}&&&&& Classifier  & AP malignant: 0.396 ± 0.275; AP Benign: 0.283 ± 0.24 \rule[-1ex]{0pt}{0pt}\\
\cline{2-7}
% --------------------------------------------------------------
\rule{0pt}{2ex} &\citet{chen2022c}                             & Segmentation                & prostate; Cardiac;              & Causal Inference; CAM & ACDC; ProMRI; CHAOS  & ProMRI DSC: 0.864$\pm$0.004; ASD: 3.86$\pm$1.20; MSD: 3.85$\pm$1.33\rule[-1ex]{0pt}{0pt}\\
&&& Abdominal Organ &&&   ACDC DSC: 0.875$\pm$0.008; ASD: 1.62$\pm$0.41; MSD: 1.17$\pm$0.24\rule[-1ex]{0pt}{0pt}\\
&&  &&&  & CHAOS DSC: 0.781\rule[-1ex]{0pt}{0pt}\\
\cline{2-7}
% --------------------------------------------------------------
\rule{0pt}{2ex} &\citet{liu_tssk-net_2023} & detection & Eye & contrastive learning; knowledge distillation & Private: 7,000 OCT & AUC: 98.05; Dice: 50.95 \\
\midrule
% --------------------------------------------------------------
% Point
% --------------------------------------------------------------
\rule{0pt}{2ex} \multirow{5}{*}{\rotatebox{90}{Point}} &\citet{khan2019extreme}                                 & Segmentation                & Multi-organ                                 & Confidence Map Supervision            & SegTHOR                                                     & DSC Aorta: 0.9441 $\pm$ 0.0187; Esophagus 0.8983 $\pm$ 0.0416; \rule[-1ex]{0pt}{0pt}\\
\cline{2-7}
% --------------------------------------------------------------
\rule{0pt}{2ex} &\citet{zhao2020weakly}                                  & Segmentation                & Cell                                        & Self-/Co-/Hybrid-Training                                  & PHC; Phase100                                               & DSC PHC: 0.871; Phase 100: 0.811  \rule[-1ex]{0pt}{0pt}\\
\cline{2-7}
% --------------------------------------------------------------
\rule{0pt}{2ex}  &\citet{dorent2021inter}                               & Segmentation                & Brain                                       & CNN + CRF                                                   & Vestibular-Schwannoma-SEG                                   & DSC: 0.819$\pm$0.080; HD95: 3.7$\pm$7.4; P: 0.929$\pm$0.059\rule[-1ex]{0pt}{0pt}\\
\cline{2-7}
% --------------------------------------------------------------
\rule{0pt}{2ex} &\citet{guo2023sac} & Segmentation & Multi-organ & Superpixel; Confident learning & MoNuSeg & Dice: 79.42; IoU: 65.15 \\
\cline{2-7}
% --------------------------------------------------------------
\rule{0pt}{2ex} &\citet{xia_weakly_2023} & Segmentation & Multi-organ & Multi-task & MoNuSeg & Dice: 75.39; AJI: 58.19 \\
\midrule
% --------------------------------------------------------------
% Scribble
% --------------------------------------------------------------
\rule{0pt}{2ex} \multirow{3}{*}{\rotatebox{90}{Scribble}} &\citet{wang2018interactive} & Segmentation & Body & Image-Specific Fine-Tuning & Private Dataset: 18 MRI Images; BRATS & Private DSC: 0.8937$\pm$0.0231; BRATS DSC: 0.8811$\pm$0.0609 \rule[-1ex]{0pt}{0pt}\\
\cline{2-7}
% --------------------------------------------------------------
\rule{0pt}{2ex} &\citet{lee2020scribble2label}                            & Segmentation                & Cell                                     & Exponential Moving Average                                 & MoNuSeg                                                     & DSC: 0.6408; mIoU: 0.5811 \rule[-1ex]{0pt}{0pt}\\
\cline{2-7}
% --------------------------------------------------------------
\rule{0pt}{2ex} &\citet{zhang2022cyclemix}                              & Segmentation                & Heart                                    & Mixup + Consistency                 & ACDC; MSCMRseg                                              & ACDC DSC: 0.848; MSCMRseg DSC: 0.800 \rule[-1ex]{0pt}{0pt}\\
\midrule
% --------------------------------------------------------------
% Box
% --------------------------------------------------------------
\rule{0pt}{2ex} \multirow{5}{*}{\rotatebox{90}{Box}} &\citet{rajchl2016deepcut}                             & Segmentation                & Brain; Lung                                   & DenseCRF                                                   & Private Dataset: 55 MRI Images                                              & Brain DSC: 0.941$\pm$0.041; Lung DSC: 0.829$\pm$0.100\rule[-1ex]{0pt}{0pt}\\
\cline{2-7}
% --------------------------------------------------------------
\rule{0pt}{2ex} &\citet{wang2022recistsup}                                  & Segmentation                & lymph; Lung; Skin            &  RECIST
measurement propagation algorithm;                                         & TCIA;   & TCIA ASSD: 0.866; HD95: 3.263; DSC: 0.785  \rule[-1ex]{0pt}{0pt}\\
\rule{0pt}{2ex} &&  & & RECIST Loss;  RECIST3D Loss                                         &  LIDC–IDRI;                                & TCIA ASSD: 0.990; HD95: 3.628; DSC: 0.753\rule[-1ex]{0pt}{0pt}\\
\rule{0pt}{2ex} && & & &  HAM10000;                           & HAM10000 ASSD: 0.314; HD95: 1.299; DSC: 0.832\rule[-1ex]{0pt}{0pt}\\
\cline{2-7}
% --------------------------------------------------------------
\rule{0pt}{2ex} &\citet{zhu_feddm_2023} & Segmentation & Prostate & Annotation calibration; Gradient de-conflicting & PROMISE12 & Dice: 81.01; IoU: 68.77 \\
% --------------------------------------------------------------
\bottomrule\label{tab:anno}
\end{tabular}}
\end{table*}

\begin{table*}[ht]
\centering
\caption{Overview of Active Learning-based Studies in Medical Image Analysis.}

\resizebox{\textwidth}{.32\textwidth}{
\begin{threeparttable}
\begin{tabular}{@{}llllll@{}}
\toprule
 &Reference & Organ & Sampling Method & Dataset & Publication \\ \midrule
\rule{0pt}{2ex} \multirow{10}{*}{\rotatebox{90}{Classification}}&  Gal \textit{et al.} \cite{gal2017deep} &  Skin  & BALD + KL-divergence  &  ISIC 2016 & ICML 2017\rule[-1.5ex]{0pt}{0pt}\\ 
\cline{2-6}
\rule{0pt}{2.5ex}&  Wu \textit{et al.} \cite{wu2021covid} &  Lung  & Loss Prediction Network  & CC-CCII Dataset & MIA 2021\rule[-1.5ex]{0pt}{0pt}\\ 
\cline{2-6}
\rule{0pt}{3ex}& Li \textit{et al.} \cite{li2021pathal}  &  Prostate  & CurriculumNet + O2U-Net  &  ISIC 2017; PANDA Dataset & TMI 2021\rule[-1.5ex]{0pt}{0pt}\\
\cline{2-6}
\rule{0pt}{2.5ex}&  Qu \textit{et al.} \cite{qu2023openal} &  Colon  & Mahalanobis Distance  & NCT-CRC-HE-100K Dataset & MICCAI 2023\rule[-1.5ex]{0pt}{0pt}\\ 
\cline{2-6}
\rule{0pt}{2.5ex}&  Hu \textit{et al.} \cite{hu2023learning} & Colon; Lung & Gradient + Uncertainty & NCT-CRC-HE-100K Dataset; LC25000 & TMI 2023\rule[-1.5ex]{0pt}{0pt}\\ 
\cline{2-6}
\rule{0pt}{2.5ex}& \multirow{2}{*}{Mahapatra \textit{et al.} \cite{mahapatra2024gandalf}} &  \multirow{2}{*}{Chest}  & Graph Informativeness + & \multirow{2}{*}{CheXpert; ChestXray14; MedMNIST} & \multirow{2}{*}{MIA 2024}\rule[-1.5ex]{0pt}{0pt}\\
&&& Informative Augmentation && \\
\cline{2-6}
\rule{0pt}{2.5ex}& Schmidt \textit{et al.} \cite{schmidt2024focused} & Prostate  & Uncertainty + Out-of-Distribution Detection & PANDA Dataset & MIA 2024 \rule[-1.5ex]{0pt}{0pt}\\
\hline
\rule{0pt}{2.5ex} \multirow{18}{*}{\rotatebox{90}{Segmentation}}& Yang \textit{et al.} \cite{yang2017suggestive} & Gland; Lymph  & Cosine Similarity + Bootstrapping + FCN& GlaS 2015; Private Dataset: 80 US images &MICCAI 2017\rule[-1.5ex]{0pt}{0pt}\\ 
\cline{2-6}
\rule{0pt}{2.5ex} & Li \textit{et al.} \cite{li2020attention} & Gland; Brain & Cosine Similarity + Attention FCN & GlaS 2015; iSeg 2017 &MICCAI 2020\rule[-1.5ex]{0pt}{0pt}\\ 
\cline{2-6}
\rule{0pt}{2.5ex} & Shen \textit{et al.} \cite{shen2020deep} & Breast  & Cosine Similarity + Entropy + Dice& Private Dataset: 2,154 IHC images &MICCAI 2020\rule[-1.5ex]{0pt}{0pt}\\ 
\cline{2-6}
\rule{0pt}{2.5ex} & Nath \textit{et al.} \cite{nath2020diminishing} &  Brain  & Entropy + SVGD Optimization  &  MSD 2018 & TMI 2020 \rule[-1.5ex]{0pt}{0pt}\\
\cline{2-6}
\rule{0pt}{2.5ex}& Ozdemir \textit{et al.} \cite{ozdemir2021active}  & Shoulder & BNN +  MMD Divergence & Private Dataset: 36 Volume of MRIs & Knowl.-Based Syst. 2021 \rule[-2ex]{0pt}{0pt}\\
\cline{2-6}
\rule{0pt}{2.5ex}& Zhao \textit{et al.} \cite{zhao2021dsal} &  Hand; Skin  & Dice + U-Net  & RSNA-Bone; ISIC 2017 & JBHI 2021 \rule[-1.5ex]{0pt}{0pt}\\
\cline{2-6}
\rule{0pt}{2.5ex}& \multirow{2}{*}{Atzeni \textit{et al.} \cite{atzeni2022deep}} &  \multirow{2}{*}{Brain}  & \multirow{2}{*}{Dice + FCN}  & MICCAI 2013 SATA Challenge; & \multirow{2}{*}{MIA 2022} \rule[-1.5ex]{0pt}{0pt}\\
&&&&Private Dataset: 266 Histology ROIs&\\
\cline{2-6}
\rule{0pt}{2.5ex}& Tang \textit{et al.} \cite{tang2023pld} &  Carotid  & KL-divergence + Mean Teacher U-Net++  & CUBS; Private Dataset: 350 US Images & MICCAI 2023 \rule[-1.5ex]{0pt}{0pt}\\
\cline{2-6}
\rule{0pt}{2.5ex}& \multirow{2}{*}{Li \textit{et al.} \cite{li2023hal} }&  \multirow{2}{*}{Breast; Chest; Liver}  & \multirow{2}{*}{Entropy + Consistency + Diversity} & BUSI; CXRSet;  & \multirow{2}{*}{MIA 2023} \rule[-1.5ex]{0pt}{0pt}\\
&&&& Private Dataset: 90 3D CT Images; 3,200 US Images  & \\
\cline{2-6}
\rule{0pt}{2.5ex}& Kadir \textit{et al.} \cite{kadir2023edgeal} &  Eye  & KL-divergence + Entropy  & Duke; AROI; UMN & MICCAI 2023 \rule[-1.5ex]{0pt}{0pt}\\
\cline{2-6}
\rule{0pt}{2.5ex}& Bai \textit{et al.} \cite{bai2023slpt} &  Breast  & Diversity + Uncertainty & Private Dataset: 941 CT Images & MICCAI 2023 \rule[-1.5ex]{0pt}{0pt}\\
\cline{2-6}
\rule{0pt}{2.5ex}& Wang \textit{et al.} \cite{wang2024dual} &  Head; Neck  & Euclidean Distance & Private Dataset: 1,057 Volume of MRIs & TMI 2024 \rule[-1.5ex]{0pt}{0pt}\\
\cline{2-6}
\rule{0pt}{2.5ex}& \multirow{2}{*}{Mahapatra \textit{et al.} \cite{mahapatra2024alfredo}} &  \multirow{2}{*}{Chest; Breast} & Uncertainty + Similarity + & \multirow{2}{*}{ChestXray14; CheXpert; Camelyon17} & \multirow{2}{*}{MIA 2024} \rule[-1.5ex]{0pt}{0pt}\\
&&&  Data Augmentation &&\\
\hline
\rule{0pt}{2.5ex} \multirow{5}{*}{\rotatebox{90}{Others}} & \multirow{2}{*}{Mahapatra \textit{et al.} \cite{mahapatra2018efficient}$_{\text{CS}}$}  &  \multirow{2}{*}{Chest}  & Bayesian Neural Network + & JSRT Database;& \multirow{2}{*}{MICCAI 2018}\\ 
& && cGAN Data Augmentation & ChestX-ray8 & \rule[-1.5ex]{0pt}{0pt}\\
\cline{2-6}
\rule{0pt}{2.5ex}& \multirow{2}{*}{Zhou \textit{et al.} \cite{zhou2021active}$_{\text{CD}}$} & 
\multirow{2}{*}{Colon} & Traditional Data Augmentation& Private Dataset: 6 colonoscopy videos & \multirow{2}{*}{MIA 2021} \\
& &&Entropy + Diversity & 38 polyp videos + 121 CTPA datasets & \rule[-1.5ex]{0pt}{0pt}\\
\bottomrule
\end{tabular}
\begin{tablenotes}    
        \footnotesize               
        \item[1] For the sake of brevity, we denote references that contain more than one task in the following abbreviations: \textbf{C}: Classification, \textbf{S}: Segmentation, \textbf{D}: Detection. 
      \end{tablenotes}
\end{threeparttable}
}
\label{tab:al}
\end{table*}

\clearpage

\subsection{Datasets}
\begin{table*}[h]
\centering
\begin{center}
\small
\caption*{Appendix Table 6. Summary of publicly available databases for label-efficient learning in MIA.}
%\resizebox{\textwidth}{.64\textwidth}{
\begin{tabular}{@{}c|p{4cm}lp{0.39\textwidth}@{}}
% \begin{tabular}{@{}p{0.5cm}|p{3cm} p{2cm} p{9cm}@{}}
\toprule
\multicolumn{1}{c}{Organ}               & Dataset (Year)      & Task         & Link  \\ \hline
\multirow{32}{*}{Brain} & \rule{0pt}{2.5ex}BraTS (2012) & Segmentation & \url{http://www.imm.dtu.dk/projects/BRATS2012/data.html}   \\ 
\cline{2-4}
                       & \rule{0pt}{2.5ex}BraTS (2013)~\cite{menze2014multimodal}  & Segmentation & \url{https://www.smir.ch/BRATS/Start2013#!#download}   \rule[-1ex]{0pt}{0pt}\\
\cline{2-4}
                       & \rule{0pt}{2.5ex}BraTS (2015) & Segmentation & \url{https://www.smir.ch/BRATS/Start2015}   \rule[-1ex]{0pt}{0pt}\\
\cline{2-4}
                       & \rule{0pt}{2.5ex}BraTS (2017) & Segmentation & \url{https://sites.google.com/site/braintumorsegmentation/}   \rule[-1ex]{0pt}{0pt}\\
\cline{2-4}
                       & \rule{0pt}{2.5ex}BraTS (2018) & Segmentation & \url{https://wiki.cancerimagingarchive.net/pages/viewpage.action?pageId=37224922}   \rule[-1ex]{0pt}{0pt}\\
\cline{2-4}
                       & \rule{0pt}{2.5ex}MSD (2018)\cite{simpson2019large} &Segmentation &\url{https://drive.google.com/drive/folders/1HqEgzS8BV2c7xYNrZdEAnrHk7osJJ--2}   \rule[-1ex]{0pt}{0pt}\\
\cline{2-4}
                       & \rule{0pt}{2.5ex}dHCP (2018) \cite{makropoulos2018developing}            &Segmentation     &\url{http://www.developingconnectome.org/data-release/}       \rule[-1ex]{0pt}{0pt}\\ 
\cline{2-4}
                       &   \rule{0pt}{2.5ex}JSRT Database (2000)\cite{shiraishi2000development}            & Classification             &\url{http://db.jsrt.or.jp/eng.php}       \rule[-1ex]{0pt}{0pt}\\ 
% \cline{2-4}
%                        &\rule{0pt}{2.5ex}EFPL EM (2013) \cite{lucchi2013learning}              &Segmentation              &\url{https://www.epfl.ch/labs/cvlab/data/data-em/}    \rule[-1ex]{0pt}{0pt}   \\
\cline{2-4}
                       &\rule{0pt}{2.5ex}MRBrainS18 (2018)              &Segmentation              &\url{https://mrbrains18.isi.uu.nl/data/}       \rule[-1ex]{0pt}{0pt}\\
\cline{2-4}
                       &\rule{0pt}{2.5ex}BigBrain (2013)~\cite{amunts2013bigbrain}              &Segmentation              &\url{https://bigbrainproject.org/maps-and-models.html#download}       \rule[-1ex]{0pt}{0pt}\\
\cline{2-4}
                       &\rule{0pt}{2.5ex}MALC (2012)              &Segmentation              &\url{http://www.neuromorphometrics.com/2012_MICCAI_Challenge_Data.html}       \rule[-1ex]{0pt}{0pt}\\
% \cline{2-4}
%                        &\rule{0pt}{2.5ex}Vestibular-Schwannoma-SEG (2021)~\cite{shapey2021segmentation}              &Segmentation              &\url{https://wiki.cancerimagingarchive.net/pages/viewpage.action?pageId=70229053}       \rule[-1ex]{0pt}{0pt}\\
\cline{2-4}
                       &\rule{0pt}{2.5ex}TCIA (2015)~\cite{seff2015leveraging}              &Segmentation              &\url{https://www.cancerimagingarchive.net/}       \rule[-1ex]{0pt}{0pt}\\
\cline{2-4}
                       &\rule{0pt}{2.5ex}OASIS (2007)              &Segmentation              &\url{https://www.oasis-brains.org/#data}       \rule[-1ex]{0pt}{0pt}\\
\cline{2-4}
                       &\rule{0pt}{2.5ex}UKBB (2016)              &Classification              &\url{https://www.ukbiobank.ac.uk/}       \rule[-1ex]{0pt}{0pt}\\
\cline{2-4}
                       &\rule{0pt}{2.5ex}ADNI (2010)              &Classification              &\url{https://www.adni-info.org/}       \rule[-1ex]{0pt}{0pt}\\
\cline{2-4}
                       &\rule{0pt}{2.5ex}ABIDE (2016)              &Classification              &\url{https://fcon_1000.projects.nitrc.org/indi/abide/}       \rule[-1ex]{0pt}{0pt}\\
% \cline{2-4}
%                        &\rule{0pt}{2.5ex}FTD              &Classification              &\url{https://cind.ucsf.edu/research/grants/frontotemporal-lobar-}      \rule[-1ex]{0pt}{0pt}\\
%                        &&& \url{degenerationneuroimaging-initiative-0}\\
% \cline{2-4}
%                        &\rule{0pt}{2.5ex}Brain Tumor MRI (2022)              &Segmentation              &\url{https://www.kaggle.com/datasets/masoudnickparvar/brain-tumor-mri-dataset}       \rule[-1ex]{0pt}{0pt}\\
\cline{2-4}
                       &\rule{0pt}{2.5ex}MIRIAD (2012)              &Classification              & \url{https://www.ucl.ac.uk/drc/research/research-methods/minimal-interval-resonance-imaging-alzheimers-disease-miriad}      \rule[-1ex]{0pt}{0pt}\\
                       
\hline
\multirow{13}{*}{Chest}  &\rule{0pt}{2.5ex}IS-COVID (2020)~\cite{fan2020inf}          &Segmentation              &\url{http://medicalsegmentation.com/covid19/}       \rule[-1ex]{0pt}{0pt}\\ 
\cline{2-4}
                       &\rule{0pt}{2.5ex}CC-COVID (2020)~\cite{zhang2020clinically}              &Segmentation              &\url{https://ncov-ai.big.ac.cn/download?lang=en}       \rule[-1ex]{0pt}{0pt}\\ 
\cline{2-4}
                       &\rule{0pt}{2.5ex}NLST (2009)              &Detection              &\url{https://cdas.cancer.gov/datasets/nlst/}       \rule[-1ex]{0pt}{0pt}\\ 
\cline{2-4}
                       &\rule{0pt}{2.5ex}NIH Chest X-ray (2017)~\cite{wang2017chestx}              &Classification              &\url{https://www.kaggle.com/datasets/nih-chest-xrays/data}       \rule[-1ex]{0pt}{0pt}\\
\cline{2-4}
                       &\rule{0pt}{2.5ex}TCGA-Lung              &Classification              &\url{https://portal.gdc.cancer.gov/repository}       \rule[-1ex]{0pt}{0pt}\\
% \cline{2-4}
%                        &\rule{0pt}{2.5ex}DLCST (2007)              &Classification              &\url{https://clinicaltrials.gov/ct2/show/NCT00496977}       \rule[-1ex]{0pt}{0pt}\\
\cline{2-4}
                       &\rule{0pt}{2.5ex}LDCTGC (2016)              &Detection              &\url{https://www.aapm.org/grandchallenge/lowdosect/}       \rule[-1ex]{0pt}{0pt}\\
\cline{2-4}
                       &\rule{0pt}{2.5ex}ChestX (2018)~\cite{kermany2018identifying}              &Classification              &\url{https://data.mendeley.com/datasets/rscbjbr9sj/3}       \rule[-1ex]{0pt}{0pt}\\
\cline{2-4}
                       &\rule{0pt}{2.5ex}LUNA (2016)              &Detection              &\url{https://luna16.grand-challenge.org/}       \rule[-1ex]{0pt}{0pt}\\
\hline
\end{tabular}%}
\label{tab:Dataset1}
% \end{tabular}
\end{center}
\end{table*}

\begin{table*}[h]
\small
\centering
\begin{center}
\caption*{Appendix Table 6. Summary of publicly available databases for label-efficient learning in MIA. (continued)}
\begin{tabular}{@{}c|p{4cm}lp{0.4\textwidth}@{}}
\toprule
\multicolumn{1}{c}{Organ}               & Dataset (Year)      & Task         & Link  \\ \hline
\multirow{23}{*}{Chest (Continued)} 
                &\rule{0pt}{2.5ex}CAD-PE (2019)              &Segmentation              &\url{https://ieee-dataport.org/open-access/cad-pe}       \rule[-1ex]{0pt}{0pt}\\
\cline{2-4}
                       &\rule{0pt}{2.5ex}SIIM-ACR (2019)              &Segmentation              &\url{https://www.kaggle.com/c/siim-acr-pneumothorax-segmentation}       \rule[-1ex]{0pt}{0pt}\\
\cline{2-4}
                       &\rule{0pt}{2.5ex}RSNA-Lung (2018)              &Detection              &\url{https://www.kaggle.com/c/rsna-pneumonia-detection-challenge}       \rule[-1ex]{0pt}{0pt}\\
\cline{2-4}
                       &\rule{0pt}{2.5ex}VinDr-CXR (2021)              &Detection              &\url{https://vindr.ai/datasets/cxr}       \rule[-1ex]{0pt}{0pt}\\
\cline{2-4}
                       &\rule{0pt}{2.5ex}Montgomery (2022)              &Segmentation              &\url{https://www.kaggle.com/datasets/raddar/tuberculosis-chest-xrays-montgomery}       \rule[-1ex]{0pt}{0pt}\\
% \cline{2-4}
%                        &\rule{0pt}{2.5ex}RICORD (2021)~\cite{tsai2021rsna}              &Segmentation              &\url{https://wiki.cancerimagingarchive.net/pages/viewpage.action?pageId=80969742}       \rule[-1ex]{0pt}{0pt}\\
\cline{2-4}
                       &\rule{0pt}{2.5ex}ChestXR (2021)              &Classification              &\url{https://cxr-covid19.grand-challenge.org/Dataset/}       \rule[-1ex]{0pt}{0pt}\\
\cline{2-4}
                       &\rule{0pt}{2.5ex}MIMIC-CXR (2019)~\cite{johnson2019mimic}             &Detection              &\url{https://physionet.org/content/mimic-cxr/2.0.0/}       \rule[-1ex]{0pt}{0pt}\\
\cline{2-4}
                       &\rule{0pt}{2.5ex}CC-CCII (2020)\cite{zhang2020clinically} &  Classification & \url{http://ncov-ai.big.ac.cn/download/}       \rule[-1ex]{0pt}{0pt}\\
\cline{2-4}
                       &\rule{0pt}{2.5ex}ChestX-ray8 (2017) \cite{wang2017chestx} & Segmentation  & \url{https://nihcc.app.box.com/v/ChestXray-NIHCC/}      \rule[-1ex]{0pt}{0pt}\\
\cline{2-4}
                       &\rule{0pt}{2.5ex}ChestX-ray14 (2019) & Classification & \url{https://stanfordmlgroup.github.io/competitions/chexpert/}       \rule[-1ex]{0pt}{0pt}\\
\cline{2-4}
                       &\rule{0pt}{2.5ex}CheXpert (2019) \cite{wang2017chestx} & Segmentation  & \url{https://stanfordmlgroup.github.io/competitions/chexpert/}       \rule[-1ex]{0pt}{0pt}\\
\hline
\multirow{4}{*}{Gland} & \rule{0pt}{2.5ex}GlaS (2015)\cite{sirinukunwattana2015stochastic} & Segmentation & \url{https://warwick.ac.uk/fac/cross_fac/tia/data/glascontest/download/}   \\ 
\cline{2-4}
                       &\rule{0pt}{2.5ex}CRAG (2017)  &Segmentation   &\url{https://warwick.ac.uk/fac/sci/dcs/research/tia/data/mildnet}   \rule[-1ex]{0pt}{0pt}\\
\hline
\multirow{10}{*}{Prostate} & \rule{0pt}{2.5ex}Prostate-MRI-US-Biopsy (2013)~\cite{sonn2013targeted} & Segmentation & \url{https://wiki.cancerimagingarchive.net/pages/viewpage.action?pageId=68550661}   \rule[-1ex]{0pt}{0pt}\\
\cline{2-4}
                       & \rule{0pt}{2.5ex}PANDA (2020) \cite{bulten2020panda}  &Classification   &\url{https://www.kaggle.com/c/prostate-cancer-grade-assessment/data/}   \rule[-1ex]{0pt}{0pt}\\
\cline{2-4}
                       & \rule{0pt}{2.5ex}ProMRI (2012)~\cite{litjens2014evaluation,tian2015superpixel}  &Segmentation   &\url{https://promise12.grand-challenge.org/}   \rule[-1ex]{0pt}{0pt}\\
\cline{2-4}
                       & \rule{0pt}{2.5ex}TMA-Zurich (2018)~\cite{arvaniti2018automated}  &Classification   &\url{https://www.nature.com/articles/s41598-018-30535-1?source=app#data-availability}   \rule[-1ex]{0pt}{0pt}\\
\cline{2-4}
                       & \rule{0pt}{2.5ex}GGC (2019)  &Classification   &\url{https://gleason2019.grand-challenge.org/Register/}   \rule[-1ex]{0pt}{0pt}\\
\hline
\multirow{7}{*}{Heart} & \rule{0pt}{2.5ex}MSCMRseg~\cite{zhuang2016multivariate} & Segmentation & \url{https://zmiclab.github.io/zxh/0/mscmrseg19/}   \rule[-1ex]{0pt}{0pt}\\
\cline{2-4}
                       & \rule{0pt}{2.5ex}MM-WHS (2017)  &Segmentation   &\url{https://zmiclab.github.io/zxh/0/mmwhs/}   \rule[-1ex]{0pt}{0pt}\\
\cline{2-4}
                       & \rule{0pt}{2.5ex}Endocardium-MRI (2008)~\cite{andreopoulos2008efficient}  &Segmentation   &\url{https://www.sciencedirect.com/science/article/pii/S1361841508000029#aep-e-component-id41}   \rule[-1ex]{0pt}{0pt}\\
\cline{2-4}
                       & \rule{0pt}{2.5ex}M\&Ms (2020)& Segmentation & \url{https://www.ub.edu/mnms/}\rule[-1ex]{0pt}{0pt}\\
\cline{2-4}
                       & \rule{0pt}{2.5ex}ASG (2018)~\cite{xiong2021global}& Segmentation & \url{http://atriaseg2018.cardiacatlas.org/}\rule[-1ex]{0pt}{0pt}\\
\hline
\multirow{2}{*}{Eye} & \rule{0pt}{2.5ex}DRISHTI-GS (2014)~\cite{sivaswamy2014drishti} & Segmentation & \url{https://www.kaggle.com/datasets/lokeshsaipureddi/drishtigs-retina-dataset-for-onh-segmentation}   \rule[-1ex]{0pt}{0pt}\\
% \cline{2-4}
%                        & \rule{0pt}{2.5ex}RIM-ONE (2015)~\cite{fumero2015interactive}& Segmentation & \url{https://www.idiap.ch/software/bob/docs/bob/bob.db.rimoner3/stable/index.html}\rule[-1ex]{0pt}{0pt}\\
\hline
\end{tabular}\label{tab:Dataset2}
\end{center}
\end{table*}

\begin{table*}[htbp]
\centering
\begin{center}
\captcont*{Summary of publicly available databases for label-efficient learning in MIA (continued)}
\begin{tabular}{@{}c|p{3.5cm}lp{0.5\textwidth}@{}}
% \begin{tabular}{@{}p{0.5cm}|p{3cm} p{2cm} p{9cm}@{}}
\toprule
\multicolumn{1}{c}{Domain}               & Dataset (Year)      & Task         & Link  \\ \hline
\multirow{28}{*}{Breast} & \rule{0pt}{2.5ex}BACH (2018) \cite{aresta2019bach} &    Classification & \url{https://iciar2018-challenge.grand-challenge.org/Dataset/}   \\ 
\cline{2-4}
                       &\rule{0pt}{2.5ex}NYUBCS (2019)~\cite{wu2019nyu}& Segmentation & \url{https://datacatalog.med.nyu.edu/dataset/10518}  \rule[-1ex]{0pt}{0pt}\\
\cline{2-4}
                       &\rule{0pt}{2.5ex}CBIS-DDSM (2017)~\cite{lee2017curated}& Segmentation & \url{https://www.kaggle.com/datasets/awsaf49/cbis-ddsm-breast-cancer-image-dataset}  \rule[-1ex]{0pt}{0pt}\\
% \cline{2-4}
%                        &\rule{0pt}{2.5ex}DDSM (1998)~\cite{heath1998current}& Detection & \url{https://www.kaggle.com/datasets/skooch/ddsm-mammography} \rule[-1ex]{0pt}{0pt}\\
\cline{2-4}
                       &\rule{0pt}{2.5ex}MIAS (2015)~\cite{suckling2015mammographic}& Detection & \url{https://www.kaggle.com/datasets/kmader/mias-mammography} \rule[-1ex]{0pt}{0pt}\\
\cline{2-4}
                       &\rule{0pt}{2.5ex}TCGA-Breast & Classification & \url{https://portal.gdc.cancer.gov/repository} \rule[-1ex]{0pt}{0pt}\\
\cline{2-4}
                       &\rule{0pt}{2.5ex}INBreast (2012) & Classification & \url{https://biokeanos.com/source/INBreast} \rule[-1ex]{0pt}{0pt}\\
% \cline{2-4}
%                        &\rule{0pt}{2.5ex}BCSC (2013)~\cite{oster2013development} & Classification & \url{https://www.bcsc-research.org/data}\rule[-1ex]{0pt}{0pt}\\
\cline{2-4}
                       &\rule{0pt}{2.5ex}BreastPathQ (2019)& Classification & \url{https://breastpathq.grand-challenge.org/Overview/} \rule[-1ex]{0pt}{0pt}\\
\cline{2-4}
                       &\rule{0pt}{2.5ex}CAMELYON (2016)& Classification & \url{https://camelyon16.grand-challenge.org/Data/} \rule[-1ex]{0pt}{0pt}\\
\cline{2-4}
                       &\rule{0pt}{2.5ex}CAMELYON (2017)& Classification & \url{https://camelyon17.grand-challenge.org/Data/} \rule[-1ex]{0pt}{0pt}\\
\cline{2-4}
                       &\rule{0pt}{2.5ex}BreakHis (2016)& Classification & \url{https://web.inf.ufpr.br/vri/databases/breast-cancer-} \rule[-1ex]{0pt}{0pt}\\
                       &&& \url{histopathological-database-breakhis/} \\
\cline{2-4}
                       &\rule{0pt}{2.5ex}CBCS3 (2018)~\cite{troester2018racial}& Classification & \url{https://unclineberger.org/cbcs/for-researchers/}  \rule[-1ex]{0pt}{0pt}\\
\cline{2-4}
                       &\rule{0pt}{2.5ex}TNBC (2018)~\cite{naylor2018segmentation}& Segmentation & \url{https://ega-archive.org/datasets/EGAD00001000063} \rule[-1ex]{0pt}{0pt}\\
\cline{2-4}
                       &\rule{0pt}{2.5ex}TUPAC (2016)~\cite{veta2019predicting}& Segmentation & \url{https://github.com/CODAIT/deep-histopath}  \rule[-1ex]{0pt}{0pt}\\
\cline{2-4}
                       &\rule{0pt}{2.5ex}MITOS12~\cite{ludovic2013mitosis}& Segmentation & \url{http://ludo17.free.fr/mitos_2012/dataset.html}  \rule[-1ex]{0pt}{0pt}\\
\cline{2-4}
                       &\rule{0pt}{2.5ex}MITOS14~\cite{icpr2014mitosis}& Segmentation & \url{https://mitos-atypia-14.grand-challenge.org/Dataset/}  \rule[-1ex]{0pt}{0pt}\\
\cline{2-4}
                       &\rule{0pt}{2.5ex}TMA-UCSB (2014)~\cite{kandemir2014empowering}& Classification & \url{https://bioimage.ucsb.edu/research/biosegmentation}  \rule[-1ex]{0pt}{0pt}\\
\hline
\multirow{2}{*}{Cell} & \rule{0pt}{2.5ex}PHC (2013)~\cite{mavska2014benchmark}& Segmentation & \url{http://celltrackingchallenge.net/}   \\ 
% \cline{2-4}
%                        &\rule{0pt}{2.5ex}Phase100~\cite{zhao2018pyramid}& Segmentation & \url{http://celltrackingchallenge.net/2d-datasets/}  \rule[-1ex]{0pt}{0pt}\\
\cline{2-4}
                       &\rule{0pt}{2.5ex}CPM (2017)~\cite{vu2019methods}& Segmentation & \url{http://simbad.u-strasbg.fr/simbad/sim-id?Ident=CPM+17}  \rule[-1ex]{0pt}{0pt}\\
\hline
\multirow{2}{*}{Liver} & \rule{0pt}{2.5ex}LiTS (2017)& Segmentation & \url{https://competitions.codalab.org/competitions/17094}  \\
\cline{2-4}
                       &\rule{0pt}{2.5ex}PAIP (2019) & Segmentation & \url{https://paip2019.grand-challenge.org/Dataset/} \rule[-1ex]{0pt}{0pt}\\
\hline
\multirow{3}{*}{Lymph Node} & \rule{0pt}{2.5ex}PatchCAMELYON (2017) & Classification  & \url{https://patchcamelyon.grand-challenge.org/Download/}  \\
\cline{2-4}
                       &\rule{0pt}{2.5ex}NIH LN (2016) & Classification &  \url{https://wiki.cancerimagingarchive.net/pages/viewpage.action?pageId=19726546} \rule[-1ex]{0pt}{0pt}\\
\hline
\multirow{1}{*}{Pancreas} & \rule{0pt}{2.5ex}NIH PCT& Segmentation & \url{https://wiki.cancerimagingarchive.net/display/Public/Pancreas-CT}  \\
\hline
\multirow{18}{*}{Multi-organ} & \rule{0pt}{2.5ex}DSB (2018)& Segmentation & \url{https://www.kaggle.com/competitions/data-science-bowl-2018/data}  \\
\cline{2-4}
                       &\rule{0pt}{2.5ex}DeepLesion (2018)~\cite{yan2018deeplesion}& Detection & \url{https://nihcc.app.box.com/v/DeepLesion} \rule[-1ex]{0pt}{0pt}\\
\cline{2-4}
                       &\rule{0pt}{2.5ex}WTS (2020)~\cite{keikhosravi2020non}& Super-resolution & \url{https://www.nature.com/articles/s42003-020-01151-5} \rule[-1ex]{0pt}{0pt}\\
                       &&&\url{#data-availability}\rule[-1ex]{0pt}{0pt}\\
\cline{2-4}
                       &\rule{0pt}{2.5ex}DECATHLON (2019)~\cite{simpson2019large}& Segmentation & \url{http://medicaldecathlon.com/} \rule[-1ex]{0pt}{0pt}\\
\cline{2-4}
                       &\rule{0pt}{2.5ex}MoNuSeg (2017)~\cite{kumar2017dataset}& Segmentation & \url{https://monuseg.grand-challenge.org/} \rule[-1ex]{0pt}{0pt}\\
\cline{2-4}
                       &\rule{0pt}{2.5ex}MoCTSeg (2018)~\cite{gibson2018automatic}& Segmentation & \url{https://www.synapse.org/#!Synapse:syn3376386} \rule[-1ex]{0pt}{0pt}\\
\cline{2-4}
                       &\rule{0pt}{2.5ex}BTCV (2017)~\cite{gibson2018multi}& Segmentation & \url{https://zenodo.org/record/1169361#.Y8Ud-OxBwUE} \rule[-1ex]{0pt}{0pt}\\
\cline{2-4}
                       &\rule{0pt}{2.5ex}CT-ORG~\cite{roth2015deeporgan,rister2020ct}& Segmentation & \url{https://wiki.cancerimagingarchive.net/pages/viewpage.action?pageId=61080890}  \rule[-1ex]{0pt}{0pt}\\
\cline{2-4}
                       &\rule{0pt}{2.5ex}NIH PLCO (2011)~\cite{oken2011screening}& Classification & \url{https://cdas.cancer.gov/datasets/plco/} \rule[-1ex]{0pt}{0pt}\\
\cline{2-4}
                       &\rule{0pt}{2.5ex}BCV (2017) & Segmentation & \url{https://www.synapse.org/#!Synapse:syn3193805/files/} \rule[-1ex]{0pt}{0pt}\\
\cline{2-4}
                       &\rule{0pt}{2.5ex}MIDOG~\cite{aubreville2021mitosis}& Segmentation & \url{https://imig.science/midog/the-dataset/} \rule[-1ex]{0pt}{0pt}\\
\hline
\end{tabular}\label{tab:Dataset3}
\end{center}
\end{table*}

\begin{table*}[h]
\small
\centering
\begin{center}
\caption*{Appendix Table 6. Summary of publicly available databases for label-efficient learning in MIA. (continued)}

\begin{tabular}{@{}c|p{4cm}lp{0.4\textwidth}@{}}
% \begin{tabular}{@{}p{0.5cm}|p{3cm} p{2cm} p{9cm}@{}}
\toprule
\multicolumn{1}{c}{Organ}               & Dataset (Year)      & Task         & Link  \\ \hline
\multirow{3}{*}{Abdomen} & \rule{0pt}{2.5ex}ACDC (2018)~\cite{bernard2018deep}& Segmentation  & \url{https://www.creatis.insa-lyon.fr/Challenge/acdc/databases.html}   \rule[-1ex]{0pt}{0pt}\\
\cline{2-4}
                       & \rule{0pt}{2.5ex}CHAOS (2021)~\cite{kavur2021chaos}& Segmentation & \url{https://chaos.grand-challenge.org/} \rule[-1ex]{0pt}{0pt}\\
\hline
\multirow{28}{*}{Breast} & \rule{0pt}{2.5ex}BACH (2018) \cite{aresta2019bach} &    Classification & \url{https://iciar2018-challenge.grand-challenge.org/Dataset/}   \\ 
\cline{2-4}
                       &\rule{0pt}{2.5ex}NYUBCS (2019)~\cite{wu2019nyu}& Segmentation & \url{https://datacatalog.med.nyu.edu/dataset/10518}  \rule[-1ex]{0pt}{0pt}\\
\cline{2-4}
                       &\rule{0pt}{2.5ex}CBIS-DDSM (2017)~\cite{lee2017curated}& Segmentation & \url{https://www.kaggle.com/datasets/awsaf49/cbis-ddsm-breast-cancer-image-dataset}  \rule[-1ex]{0pt}{0pt}\\
% \cline{2-4}
%                        &\rule{0pt}{2.5ex}DDSM (1998)~\cite{heath1998current}& Detection & \url{https://www.kaggle.com/datasets/skooch/ddsm-mammography} \rule[-1ex]{0pt}{0pt}\\
\cline{2-4}
                       &\rule{0pt}{2.5ex}MIAS (2015)~\cite{suckling2015mammographic}& Detection & \url{https://www.kaggle.com/datasets/kmader/mias-mammography} \rule[-1ex]{0pt}{0pt}\\
\cline{2-4}
                       &\rule{0pt}{2.5ex}TCGA-Breast & Classification & \url{https://portal.gdc.cancer.gov/repository} \rule[-1ex]{0pt}{0pt}\\
\cline{2-4}
                       &\rule{0pt}{2.5ex}INBreast (2012) & Classification & \url{https://biokeanos.com/source/INBreast} \rule[-1ex]{0pt}{0pt}\\
% \cline{2-4}
%                        &\rule{0pt}{2.5ex}BCSC (2013)~\cite{oster2013development} & Classification & \url{https://www.bcsc-research.org/data}\rule[-1ex]{0pt}{0pt}\\
\cline{2-4}
                       &\rule{0pt}{2.5ex}BreastPathQ (2019)& Classification & \url{https://breastpathq.grand-challenge.org/Overview/} \rule[-1ex]{0pt}{0pt}\\
\cline{2-4}
                       &\rule{0pt}{2.5ex}CAMELYON (2016)& Classification & \url{https://camelyon16.grand-challenge.org/Data/} \rule[-1ex]{0pt}{0pt}\\
\cline{2-4}
                       &\rule{0pt}{2.5ex}CAMELYON (2017)& Classification & \url{https://camelyon17.grand-challenge.org/Data/} \rule[-1ex]{0pt}{0pt}\\
\cline{2-4}
                       &\rule{0pt}{2.5ex}BreakHis (2016)& Classification & \url{https://web.inf.ufpr.br/vri/databases/breast-cancer-histopathological-database-breakhis/} \rule[-1ex]{0pt}{0pt}\\
\cline{2-4}
                       &\rule{0pt}{2.5ex}CBCS3 (2018)~\cite{troester2018racial}& Classification & \url{https://unclineberger.org/cbcs/for-researchers/}  \rule[-1ex]{0pt}{0pt}\\
\cline{2-4}
                       &\rule{0pt}{2.5ex}TNBC (2018)~\cite{naylor2018segmentation}& Segmentation & \url{https://ega-archive.org/datasets/EGAD00001000063} \rule[-1ex]{0pt}{0pt}\\
\cline{2-4}
                       &\rule{0pt}{2.5ex}TUPAC (2016)~\cite{veta2019predicting}& Segmentation & \url{https://github.com/CODAIT/deep-histopath}  \rule[-1ex]{0pt}{0pt}\\
\cline{2-4}
                       &\rule{0pt}{2.5ex}MITOS12~\cite{ludovic2013mitosis}& Segmentation & \url{http://ludo17.free.fr/mitos_2012/dataset.html}  \rule[-1ex]{0pt}{0pt}\\
\cline{2-4}
                       &\rule{0pt}{2.5ex}MITOS14& Segmentation & \url{https://mitos-atypia-14.grand-challenge.org/Dataset/}  \rule[-1ex]{0pt}{0pt}\\
\cline{2-4}
                       &\rule{0pt}{2.5ex}TMA-UCSB (2014)~\cite{kandemir2014empowering}& Classification & \url{https://bioimage.ucsb.edu/research/biosegmentation}  \rule[-1ex]{0pt}{0pt}\\
\hline
\multirow{5}{*}{Cell} & \rule{0pt}{2.5ex}PHC (2013)~\cite{mavska2014benchmark}& Segmentation & \url{http://celltrackingchallenge.net/}   \\ 
% \cline{2-4}
%                        &\rule{0pt}{2.5ex}Phase100~\cite{zhao2018pyramid}& Segmentation & \url{http://celltrackingchallenge.net/2d-datasets/}  \rule[-1ex]{0pt}{0pt}\\
\cline{2-4}
                       &\rule{0pt}{2.5ex}CPM (2017)~\cite{vu2019methods}& Segmentation & \url{http://simbad.u-strasbg.fr/simbad/sim-id?Ident=CPM+17}  \rule[-1ex]{0pt}{0pt}\\
\hline
\multirow{3}{*}{Liver} & \rule{0pt}{2.5ex}LiTS (2017)& Segmentation & \url{https://competitions.codalab.org/competitions/17094}  \\
\cline{2-4}
                       &\rule{0pt}{2.5ex}PAIP (2019) & Segmentation & \url{https://paip2019.grand-challenge.org/Dataset/} \rule[-1ex]{0pt}{0pt}\\
\hline
\multirow{4}{*}{Lymph Node} & \rule{0pt}{2.5ex}PatchCAMELYON (2017) & Classification  & \url{https://patchcamelyon.grand-challenge.org/Download/}  \\
\cline{2-4}
                       &\rule{0pt}{2.5ex}NIH LN (2016) & Classification &  \url{https://wiki.cancerimagingarchive.net/pages/viewpage.action?pageId=19726546} \rule[-1ex]{0pt}{0pt}\\
\hline
\multirow{2}{*}{Pancreas} & \rule{0pt}{2.5ex}NIH PCT& Segmentation & \url{https://wiki.cancerimagingarchive.net/display/Public/Pancreas-CT}  \\
\hline
\end{tabular}\label{tab:Dataset3-2}
\end{center}
\end{table*}

\begin{table*}[h]
\small
\centering
\begin{center}
\caption*{Appendix Table 6. Summary of publicly available databases for label-efficient learning in MIA. (continued)}

\begin{tabular}{@{}c|p{4cm}lp{0.4\textwidth}@{}}
% \begin{tabular}{@{}p{0.5cm}|p{3cm} p{2cm} p{9cm}@{}}
\toprule
\multicolumn{1}{c}{Organ}               & Dataset (Year)      & Task         & Link  \\ \hline
\multirow{18}{*}{Multi-organ} & \rule{0pt}{2.5ex}DSB (2018)& Segmentation & \url{https://www.kaggle.com/competitions/data-science-bowl-2018/data}  \\
\cline{2-4}
                       &\rule{0pt}{2.5ex}DeepLesion (2018)~\cite{yan2018deeplesion}& Detection & \url{https://nihcc.app.box.com/v/DeepLesion} \rule[-1ex]{0pt}{0pt}\\
\cline{2-4}
                       &\rule{0pt}{2.5ex}WTS (2020)~\cite{keikhosravi2020non}& Super-resolution & \url{https://www.nature.com/articles/s42003-020-01151-5#data-availability} \rule[-1ex]{0pt}{0pt}\\
                       
\cline{2-4}
                       &\rule{0pt}{2.5ex}DECATHLON (2019)~\cite{simpson2019large}& Segmentation & \url{http://medicaldecathlon.com/} \rule[-1ex]{0pt}{0pt}\\
\cline{2-4}
                       &\rule{0pt}{2.5ex}MoNuSeg (2017)~\cite{kumar2017dataset}& Segmentation & \url{https://monuseg.grand-challenge.org/} \rule[-1ex]{0pt}{0pt}\\
\cline{2-4}
                       &\rule{0pt}{2.5ex}MoCTSeg (2018)~\cite{gibson2018automatic}& Segmentation & \url{https://www.synapse.org/#!Synapse:syn3376386} \rule[-1ex]{0pt}{0pt}\\
\cline{2-4}
                       &\rule{0pt}{2.5ex}BTCV (2017)~\cite{gibson2018multi}& Segmentation & \url{https://zenodo.org/record/1169361#.Y8Ud-OxBwUE} \rule[-1ex]{0pt}{0pt}\\
\cline{2-4}
                       &\rule{0pt}{2.5ex}CT-ORG~\cite{roth2015deeporgan,rister2020ct}& Segmentation & \url{https://wiki.cancerimagingarchive.net/pages/viewpage.action?pageId=61080890}  \rule[-1ex]{0pt}{0pt}\\
\cline{2-4}
                       &\rule{0pt}{2.5ex}NIH PLCO (2011)~\cite{oken2011screening}& Classification & \url{https://cdas.cancer.gov/datasets/plco/} \rule[-1ex]{0pt}{0pt}\\
\cline{2-4}
                       &\rule{0pt}{2.5ex}BCV (2017) & Segmentation & \url{https://www.synapse.org/#!Synapse:syn3193805/files/} \rule[-1ex]{0pt}{0pt}\\
\cline{2-4}
                       &\rule{0pt}{2.5ex}MIDOG~\cite{aubreville2021mitosis}& Segmentation & \url{https://imig.science/midog/the-dataset/} \rule[-1ex]{0pt}{0pt}\\
\hline
\end{tabular}\label{tab:Dataset3-3}
\end{center}
\end{table*}
\end{document}
\endinput
%%
%% End of file `sample-manuscript.tex'.
