%% This is file `medima-template.tex',
%% 
%% Copyright 2018 Elsevier Ltd
%% 
%% This file is part of the 'Elsarticle Bundle'.
%% ---------------------------------------------
%% 
%% It may be distributed under the conditions of the LaTeX Project Public
%% License, either version 1.2 of this license or (at your option) any
%% later version.  The latest version of this license is in
%%    http://www.latex-project.org/lppl.txt
%% and version 1.2 or later is part of all distributions of LaTeX
%% version 1999/12/01 or later.
%% 
%% The list of all files belonging to the 'Elsarticle Bundle' is
%% given in the file `manifest.txt'.
%% 
%% Template article for Elsevier's document class `elsarticle'
%% with harvard style bibliographic references
%%
%% $Id: medima-template.tex 153 2018-12-01 11:38:32Z rishi $
%% $URL: http://lenova.river-valley.com/svn/elsarticle/trunk/medima-template.tex $
%%
%% Use the option review to obtain double line spacing
%\documentclass[times,review,preprint,authoryear]{elsarticle}

%% Use the options `twocolumn,final' to obtain the final layout
%% Use longtitle option to break abstract to multiple pages if overfull.
%% For Review pdf (With double line spacing)
%\documentclass[times,twocolumn,review]{elsarticle}
%% For abstracts longer than one page.
%\documentclass[times,twocolumn,review,longtitle]{elsarticle}
%% For Review pdf without preprint line
%\documentclass[times,twocolumn,review,nopreprintline]{elsarticle}
%% Final pdf
\documentclass[times,twocolumn,final]{elsarticle}
%%
%\documentclass[times,twocolumn,final,longtitle]{elsarticle}
%%


%% Stylefile to load MEDIMA template
\usepackage{medima}
\usepackage{framed,multirow}
\usepackage{threeparttable}
\usepackage{booktabs}
\usepackage{amsmath}
\usepackage{captcont}
% \usepackage[authoryear]{natbib}

%% The amssymb package provides various useful mathematical symbols
\usepackage{amssymb}
\usepackage{latexsym}

% Following three lines are needed for this document.
% If you are not loading colors or url, then these are
% not required.
\usepackage{url}
\usepackage{xcolor}

\usepackage{hyperref}

\definecolor{newcolor}{rgb}{.8,.349,.1}

\journal{Medical Image Analysis}

\begin{document}

\verso{Cheng Jin \textit{et~al.}}

\begin{frontmatter}

\title{Label-Efficient Deep Learning in Medical Image Analysis: Challenges and Future Directions}
% \tnotetext[tnote1]{This is an example for title footnote coding.}

\author[1]{Cheng \snm{Jin}\fnref{fn1}}
%\ead{author3@author.com}
\author[1]{Zhengrui \snm{Guo}\fnref{fn1}}
\fntext[fn1]{Equal contribution.}
\author[1]{Yi \snm{Lin}}
\author[1]{Luyang \snm{Luo}}
\author[1,2,3]{Hao \snm{Chen}\corref{cor1}}
\cortext[cor1]{Corresponding author: Hao Chen (jhc@cse.ust.hk)}


\address[1]{Department of Computer Science and Engineering, The Hong Kong University of Science and Technology, Kowloon, Hong Kong}
\address[2]{Department of Chemical and Biological Engineering, The Hong Kong University of Science and Technology, Kowloon, Hong Kong}
\address[3]{HKUST Shenzhen-Hong Kong Collaborative Innovation Research Institute, Futian, Shenzhen, China}


\received{20 December 2023}
%\finalform{10 May 2013}
%\accepted{13 May 2013}
%\availableonline{15 May 2013}
%\communicated{S. Sarkar}


\begin{abstract}
%%%
Deep learning has seen rapid growth in recent years and achieved state-of-the-art performance in a wide range of applications. However, training models typically requires expensive and time-consuming collection of large quantities of labeled data. This is particularly true within the scope of medical imaging analysis (MIA), where data are limited and labels are expensive to acquire. Thus, label-efficient deep learning methods are developed to make comprehensive use of the labeled data as well as the abundance of unlabeled and weak-labeled data. In this survey, we extensively investigated over 300 recent papers to provide a comprehensive overview of recent progress on label-efficient learning strategies in MIA. We first present the background of label-efficient learning and categorize the approaches into different schemes. Next, we examine the current state-of-the-art methods in detail through each scheme. Specifically, we provide an in-depth investigation, covering not only canonical semi-supervised, self-supervised, and multi-instance learning schemes but also recently emerged active and annotation-efficient learning strategies. Moreover, as a comprehensive contribution to the field, this survey not only elucidates the commonalities and unique features of the surveyed methods but also presents a detailed analysis of the current challenges in the field and suggests potential avenues for future research.
%%%%
\end{abstract}

\begin{keyword}
%% MSC codes here, in the form: \MSC code \sep code
%% or \MSC[2008] code \sep code (2000 is the default)
% \MSC 41A05\sep 41A10\sep 65D05\sep 65D17
%% Keywords
\KWD Medical Image Analysis \sep Label-Efficient Learning \sep Annotation-Efficient Learning \sep Weakly-Supervised Learning
\end{keyword}

\end{frontmatter}

%\linenumbers

%% main text
\section{Introduction}
\label{sec:introduction}
Computer-aided medical image analysis (MIA) plays a more and more critical role in achieving efficiency and accuracy in the early detection, diagnosis, and treatment of diseases. In recent years, MIA systems powered by deep learning (DL) have provided a more objective approach to learning from large and heterogeneous medical image datasets and improved disease diagnosis accuracy. However, DL models require abundant precisely annotated data to effectively capture anatomical heterogeneity and disease-specific traits \citep{yu2021convolutional} due to their data-driven nature. Unfortunately, due to a shortage of available annotators \citep{lu2020national}, there is a significant gap between the demand for annotation and the available annotated datasets. Hence, the urgency to curtail annotation expenses, expedite the annotation procedure, and alleviate the load on annotators has emerged as a crucial hurdle in DL-based MIA tasks. Traditional fully-supervised DL methods, on the other hand, depend solely on comprehensively annotated datasets. Recently, strategies based on semi-supervised, self-supervised, and multi-instance learning have been widely utilized to maximize the utility of existing medical data that may be only partially annotated by point, scribble, box, pixel-wise, \textit{etc.} or even completely unannotated data. In this paper, we dub these methods as label-efficient learning.  
As seen in Fig. \ref{fig_trend}, label-efficient learning methods have significantly proliferated in recent years. Meanwhile, label-efficient learning methods excelling in other MIA tasks like denoising, image registration, and super-resolution have also been rising beyond common classification, segmentation, and detection.
\begin{figure}[htbp]
	\centering
\includegraphics[width=0.44\textwidth]{./Figures/Figs/trend.png}
	\caption{The publications label-efficient learning papers in MIA from 2016.}
	\label{fig_trend}
\end{figure} 

\begin{figure}[htbp]
	\centering
	\includegraphics[width=0.44\textwidth]{./Figures/Figs/taxonomy.png}
	\caption{The taxonomy for label-efficient MIA research.}
	\label{fig_taxonomy}
\end{figure}
Several surveys related to label-efficient learning in medical image analysis have been published in recent years. \citet{cheplygina2019not} categorized methods under supervised, semi-supervised, multi-instance, and transfer learning and named them ``not-so-supervised" learning, while \citet{budd2021survey} surveyed human-in-the-loop strategies for MIA tasks. However, methods in these surveys are either limited in scope or lag behind the current trends.  
% While \citep{kumari2023dataefficient} present a contemporary review focusing on data- and label-efficient learning in the medical domain, their taxonomy is relatively ambiguous for readers to understand, while our taxonomy is based on learning schemes, which gives clear guidance for readers. Besides, above related surveys fail to fully address key questions of significant interest to researchers in their discussions, such as the impact of foundation models and omni-supervised learning with diverse label types, \textit{etc.}
% their discussion on future research directions does not fully address key questions of significant interest to researchers. Critical topics, such as the impact of foundation models and omni-supervised learning with diverse label types, receive limited attention in their work.  
While \citet{kumari2023dataefficient} present a contemporary review focused on data- and label-efficient learning in the medical domain, the taxonomy they present lacks sufficient clarity and directness, which may lead to interpretational difficulties for readers. Conversely, our taxonomy is based on learning schemes and provides distinct and straightforward guidance. Furthermore, the above reviews fall short in addressing several crucial questions of significant interest to researchers. 
% These overlooked aspects include the impact of foundation models and the intricacies of omni-supervised learning with diverse label types, among other pertinent topics
In contrast, our paper comprehensively addresses these topics, providing an in-depth exploration of these critical aspects % To address this issue, we conduct a systematic review of current label-efficient methodologies, of which 
and the outline is illustrated in Fig. \ref{fig_taxonomy}.
% \citet{kumari2023dataefficient} present a contemporary review focusing on data- and label-efficient learning in the medical domain, yet their work omits a discussion on future directions pertaining to key questions of interest to researchers. These include the impact of foundation models and omni-supervised learning across diverse label types, among others.

Aiming to provide a comprehensive overview and future challenges of label-efficient learning methods in MIA, we review more than 300 quality-assured and recent label-efficient learning methods based on semi-supervised, multi-instance, self-supervised, active, and annotation-efficient learning strategies. To pinpoint pertinent contributions, Google Scholar was employed to search for papers with related topics. ArXiv was combined through for papers citing one of a set of terms related to label-efficient medical imaging. Additionally, conference proceedings like CVPR, ICCV, ECCV, NIPS, AAAI, and MICCAI were scrutinized based on the titles of the papers, as well as journals such as MIA, IEEE TMI, and Nature Bioengineering. References in all chosen papers were examined. When overlapping work had been reported in multiple publications, only the publication(s) considered most significant were incorporated.

To the best of our knowledge, this is the first comprehensive review in the field of label-efficient MIA. In each learning scheme, we formulate the fundamental problem, offer the necessary background, and display the experimental results case by case. With the challenges proposed at the end of the survey, we explore feasible future directions in several branches to potentially enlighten the follow-up research on label-efficient learning.

The remainder of this paper is organized as follows. In Section \ref{sec:background}, the necessary background and categorization is presented. In Sections \ref{sec:semi}--\ref{sec:anno}, we introduce the primary label-efficient learning schemes in MIA, including semi-supervised learning in Section \ref{sec:semi}, self-supervised learning in Section \ref{sec:ssl}, multi-instance learning in Section \ref{sec: mil}, active learning in Section \ref{sec:al}, few-shot learning in Section \ref{sec:fsl}, and annotation-efficient learning in Section \ref{sec:anno}. We discuss the existing challenges in label-efficient learning and present several heuristic solutions for these open problems in Section \ref{sec:cnfd}, where promising future research directions are proposed as well. Finally, we conclude the paper in Section \ref{sec:con}.


\section{Background and Categorization}\label{sec:background}
In this section, we review the background of the learning schemes covering label-efficient learning. In addition, we present the categorization of each learning scheme in MIA.
\subsection{Semi-Supervised Learning}
\begin{figure}[htbp]
	\centering
\includegraphics[width=0.49\textwidth]{./Figures/Figs/semi-workflow.png}
	\caption{Overview of semi-supervised learning paradigm. Semi-SL includes a small set of labeled data and a large amount of unlabeled data to conduct learning jointly, aiming at leveraging the unlabeled data to boost learning performance. Semi-SL typically seeks to optimize the combination of a supervised loss function $\mathcal{L}_{sup}$ and an unsupervised loss function $\mathcal{L}_{unsup}$.  }
	\label{fig_semi_schematic}
\end{figure}
As illustrated in Fig. \ref{fig_semi_schematic}, \textbf{Semi-supervised learning (Semi-SL)} introduces an additional unlabeled dataset to help the model learn task-related invariant features and aim to achieve better performance than supervised learning. Concretely, one has a set of $L$ labeled data points $X_L = \{(x_i, y_i)\}_{i=1}^{L}$, in which $x_i$ represents the raw data sample from the given input space $\mathcal{X}$ and $y_i$ is the corresponding label. In the meantime, an unlabeled dataset $X_U = \{x_i\}_{i=L+1}^{L+U}$ with a much larger scale is involved, \textit{i.e.}, $U\gg L$. And $X=X_L \cup X_U$ denotes the entire dataset. During the training process, the optimization problem\footnote{Several assumptions and prior knowledge of Semi-SL can be referred to Appendix A.1.} that Semi-SL intends to solve is defined as:
\begin{equation}\label{eq:semi}
    \min_{\theta}\sum_{(x,y)\in X_{L}}\mathcal{L}_{s}(x, y, \theta)+\alpha \sum_{x \in X_U}\mathcal{L}_{u}(x,\theta)+\beta \sum_{x\in X}\mathcal{R}(x,\theta),
\end{equation}
where $\theta$ represents the model parameters, $\mathcal{L}_{s}$ is the supervised loss function, $\mathcal{L}_{u}$ represents the unsupervised loss function, and $\mathcal{R}$ is a regularization term. In addition, $\alpha, \beta \in \mathbb{R}^{+}$ control the trade-off between unsupervised loss $\mathcal{L}_u$ and regularization term $\mathcal{R}$.

Based on how the model incorporates and leverages unlabeled data, we will discuss the categories of Semi-SL methods and their applications in MIA starting from \textbf{proxy-labeling methods}, followed by \textbf{generative methods}, \textbf{consistency regularization methods}, and finally \textbf{hybrid methods}. Meanwhile, we present a brief summary of the representative publications in Tab. \ref{tab:semi}
% \footnote{The summary of all collected publications in Semi-SL and the rest of the learning schemes can be referred to in Appendix B.}. 
\subsection{Self-Supervised Learning}
\begin{figure}[htbp]
	\centering
\includegraphics[width=0.44\textwidth]{./Figures/Figs/self-workflow.png}
	\caption{Overview of self-supervised learning paradigm. Self-SL aims to learn a pre-trained model by developing various proxy tasks based solely on unlabeled data. Then the pre-trained model can be fine-tuned on different downstream tasks with labeled datasets. The process of Self-SL creates a generalizable model based on proxy tasks and avoids the overfitting which might occur if the model is trained only using the labeled datasets of downstream tasks.}
	\label{fig_self_schematic}
\end{figure}
\textbf{Self-supervised learning (Self-SL)} was proposed to extract and learn the underlying features of a large-scale unlabeled dataset without human annotation. Generally, Self-SL methods build proxy tasks for the model to learn the latent features and representations from a massive amount of unlabeled data, thus facilitating the performance on downstream tasks, as shown in Fig. \ref{fig_self_schematic}. Concretely, the training procedure of Self-SL can be divided into two stages: pre-training with proxy tasks and fine-tuning on different downstream tasks. During the pre-training phase, researchers design proxy tasks that satisfy the following two properties \citep{jing2020self}: (1) The label of the input data for the proxy task can be generated automatically by the data itself; (2) the neural network can learn related representations or features of the input data by solving the proxy task.

After the pre-training with proxy tasks, the learned representations will be utilized to solve the main task. The advantages of utilizing proxy tasks are two-fold: on the one hand, by defining particular tasks, the model can be targeted to learn features or representations of the specific studied data; on the other hand, by using a large amount of unlabeled data for pre-training, the model can significantly avoid overfitting during fine-tuning compared to supervised learning, especially for small datasets, in downstream training.

Based on the characteristics of the proxy tasks, we group the mainstream Self-SL methods in MIA into the following four general categories: \textbf{Reconstruction-Based Methods}, \textbf{Context-Based Methods}, \textbf{Contrastive-Based Methods}, and \textbf{Hybrid Methods} with a summary of the representative publications in Tab. \ref{tab:self}.


\subsection{Multi-instance Learning}
\begin{figure}[htbp]
	\centering
\includegraphics[width=0.49\textwidth]{./Figures/Figs/mil-workflow.png}
	\caption{Overview of multi-instance learning paradigm. %Take WSIs for example. 
    The inputs are cut into patches and selected patches are used to form bags, in which each patch is an instance. Given the bag-level labels, the model are trained to predict the category of bags, instances, and/or the original inputs.}
	\label{fig_mil_schematic}
\end{figure}
% As illustrated in Fig. \ref{fig_mil_schematic}, the goal for \textbf{multi-instance learning (MIL)} is to detect and/or classify target patterns of weakly-labeled data. 
In \textbf{multi-instance learning (MIL)}, the concept of a \textit{bag} is introduced. A bag $X_i$ is composed of $k$ instances:
$X_i = \{x_{i,1}, x_{i,2},\cdots,x_{i,k_i}\}$, where $x_{i,j}$ denotes an instance in bag $X_i$,
and the training dataset $\mathcal{X}$ consists of $N$ bags: $\mathcal{X}=\{X_1,X_2,\cdots,X_N\}$. Next, suppose $Y_i \in \{1,0\}$ and $y_{i,j} \in \{1,0\}$ are the labels of bag $X_i$ and the instance $x_{i,j}$ inside it, respectively, in which $1$ denotes positive and $0$ denotes negative for the binary classification scenario. Two common assumptions can be made based on this basic definition of MIL:
\begin{itemize}
    \item If bag $X_i$ is positive, then there exists at least one positive instance $x_{i,m} \in X_i$ and $m\in \{1,2,\cdots,k_i\}$ is unknown. This assumption can be summarized as: if $Y_i=1$, then $\sum^{k_i}_{j=1}{y_{i,j}}\geq 1$.
    \item If bag $X_i$ is negative, then all the instances in $X_i$ are negative, namely, if $Y_i=0$, then $\sum^{k_i}_{j=1}{y_{i,j}}=0$.
\end{itemize}

Based on the assumptions, MIL methods can perform both bag-level and instance-level tasks (illustrated in Fig. \ref{fig_mil_schematic}), with the latter often used in weakly-supervised learning. Concretely, MIL algorithms leverage the instances to identify positive or negative bags, which contributes not only to the image-level diagnosis but also to precise abnormal region detection and localization. This great interpretability of the MIL algorithm fits well in MIA, as both the global structure and local details are crucial components for solving such problems.

In this survey, we categorize MIL methods that aim at detecting all the particular target patterns in the data, such as every patch with a special disease manifestation in a large histopathology image, as \textbf{local detection}; and methods that aim at simply detecting whether or not the particular target patterns exist in the given sample as \textbf{global detection}. Note that taxonomy is in line with the methodology of MIL, \textit{i.e.}, to classify bag-level label (global detection) or to classify instance-level label (local detection). Tab. \ref{tab:mil} presents an overview of the representative publications of each method.

\subsection{Active Learning}
\begin{figure}[htbp]
	\centering
\includegraphics[width=0.49\textwidth]{./Figures/Figs/al_schematic.png}
	\caption{Overview of active learning paradigm. In a cycle, a deep learning model $f$ is trained from a labeled medical dataset $X_L$. Then, active sampling strategies based on different criteria (i.e., data uncertainty $\mathcal{U}(X_U)$, model uncertainty $\mathcal{U}(f)$) are implemented to select the data that is most valuable to the model from unlabeled medical dataset $X_U$. Finally, oracles are employed to annotate the selected data.
 }
	\label{fig_al_schematic}
\end{figure}
\textbf{Active learning (AL)} is a relatively understudied area in the MIA field. It attempts to maintain the performance of a deep learning model while annotating the fewest data with the help of an oracle, which resonates with the philosophy of label-efficient learning, \textit{i.e.}, how to effectively use noisy, limited, and unannotated data throughout the deep learning process. More specifically, its goal is to select the most valuable samples and forward them to the oracle (\textit{e.g.}, human annotator) for labeling to improve the generalization capability of the model. In active learning (AL) practice, the measurement of annotation uncertainty using various strategies is often considered as the metric for sample value. Meanwhile, in order to preserve the network's generalization capability, different mechanisms have been developed to ensure that the sampled images are distributed diversely. % since the annotation uncertainty tends to occur in ambiguous boundaries and the region with partial volume effect\cite{karimi2023learning}, etc. 

As Fig. \ref{fig_al_schematic} illustrates, before the start of the data selection process, a deep learning model is initialized or pre-trained from a labeled dataset $X_L$ with its corresponding parameter $\theta$. After that, AL sampling algorithms construct an uncertainty metric $\mathcal{U}$ for each item of unlabeled dataset $X_U$. This metric determines whether an oracle is required for annotation, and we denote this newly annotated dataset as $X_{L^{\prime}}\subset X_U$. Then the network model will either use the combined labeled data $X_{\hat{L}}=X_{L}\cup X_{L^{\prime}}$ to train from scratch or only use them to fine-tune the model. Denoting the fully labeled version of $X_U$ as $X^L_U$, the goal of AL is to build a model $f(\theta\mid X_L^{*})$ with $|X_L^{*}|
\ll |X^L_U|$ to perform equivalently or better than $f(\theta\mid X_L)$. 

Based on how the uncertainty is obtained, we categorize AL methods into \textbf{data uncertainty-based methods} and \textbf{model uncertainty-based methods}. Data uncertainty-based methods attempt to get a sample with the greatest uncertainty from a batched dataset, while model uncertainty-based methods tend to sample the samples that cause the greatest uncertainty of the deep learning model's performance. A brief summary of surveyed AL papers is presented in Tab. \ref{tab:al}.%This, however, is likely to overlook the link between samples. 
%As a result, numerous sampling approaches have been proposed recently.  and  %For a review of general AL methods, please refer to \cite{ren2021survey}.

\subsection{Few-Shot Learning}
\begin{figure}[htbp]
	\centering
\includegraphics[width=0.49\textwidth]{./Figures/Figs/fsl-workflow.png}
	\caption{Overview of few-shot learning paradigm. The model $f$ learns to generalize from the support set $S$ and infers on the query set $Q$, which contains new, unlabeled examples. The design of $f$ varies: it can either employ deep metric learning to discern patterns within the support set or function as a meta-learner, extracting meta-knowledge pertinent to the task.
 }
	\label{fig_al_schematic}
\end{figure}
Few-shot learning (FSL) is the problem of building a deep learning model to make predictions based on a limited number of samples. This limited sample size restricts the model's generalization ability in conventional learning schemes.  In FSL literature, the terms \textit{support set} $S$ and \textit{query set} $Q$ represent the training and testing sets, respectively. Each support set $S$ comprises $C$ distinct categories, each containing $K$ training samples, thus establishing a \textit{C-way K-shot} configuration.  In this section, we categorize FSL methods based on mainstream MIA literature into two categories: \textbf{metric-based methods}, \textbf{meta-based methods}\footnote{To aid the understanding of each subfield, a detailed description is provided in Appendix A.3.}. For a comprehensive review of general FSL methods, please refer to \citet{song2023comprehensive}.

\subsection{Annotation-Efficient Learning}
% \begin{figure}[htbp]
% 	\centering
% 		\subfigure[]{
% 		\includegraphics[width=0.10\textwidth]{./Figures/Figs/point.png}
% 	}
% 	\subfigure[]{
% 		\includegraphics[width=0.10\textwidth]{./Figures/Figs/scribble.png}
% 	}
% 		\subfigure[]{
% 		\includegraphics[width=0.10\textwidth]{./Figures/Figs/box.png}
% 	}
% 	\subfigure[]{
% 		\includegraphics[width=0.10\textwidth]{./Figures/Figs/full.png}
% 	}
% 	\caption{Annotation types.
% 		(a) Point annotation; (b) Scribble annotation; (c) Bounding box annotation; (d) Pixel-wise annotation.}
% 	\label{fig_anno_effi}
% \end{figure}

\begin{figure}[htbp]
\centering
\includegraphics[width=0.44\textwidth]{Figures/Figs/anno_effi.pdf}        
\caption{Annotation types. (a) Pixel-wise annotation from two independent annotator; (b) Bounding box annotation; (c) Scribble annotation; (d) Point annotation.}
	\label{fig_anno_effi}
\end{figure}

% In practice, accurate and complete annotation is time-consuming and labor-intensive. 
% The main challenges~\cite{dudea2012ultrasonography} lie in
% 1) the annotation process requiring multiple annotators to work independently and the inevitable diagnostic inter-/intra-individual variability; 
% 2) the gold standard for diagnosis needing extra tests to acquire sufficient data; 
% and 3) incomplete or low-quality annotations significantly impacting the model's performance. 
% Other factors, such as domain-specific knowledge, patients' privacy, and large data volumes~\cite{lin2019automated}, may also affect the annotation process.
% Annotation-efficient learning is a technique that utilizes deep learning methods with partially labeled data for dense predictions to improve labeling efficiency.
% The intuitive approach to increase annotation efficiency is to provide markings other than fully dense annotations. 
% Fig.~\ref{fig_anno_effi} shows different forms of annotation, 
% and we will separately review the annotation-efficient learning methods regarding the ``not exact label'' from coarse-to-fine granularity, \textit{i.e.}, \textbf{Tag}, \textbf{Point}, \textbf{Scribble}, and \textbf{Box}. The overview of the representative publications of this category is presented in Tab. \ref{tab:anno}.
\textbf{Annotation-efficient learning} is a technique that utilizes deep learning methods with partially labeled data for dense predictions to improve labeling efficiency.
The intuitive approach to increase annotation efficiency is to provide markings other than fully dense annotations. 
While there may be overlapping techniques with the aforementioned categories, annotation-efficient learning methods specifically focus on leveraging the specific characteristics of the different forms of annotation to enhance the annotation efficiency and hence minimize the granularity difference between the annotation and the prediction.
Fig.~\ref{fig_anno_effi} shows different forms of annotation, and we will separately review the annotation-efficient learning methods that address the ``not exact label'' through a coarse-to-fine way. 
Specifically, we will discuss the techniques related to \textbf{Tag}, \textbf{Point}, \textbf{Scribble}, and \textbf{Box} annotations. 
Tab.~\ref{tab:anno} provides an overview of representative publications in this category.


\section{Insufficient Label} \label{sec:semi}
The \textit{insufficient label} scenario arises when only a small portion of available medical imaging data is annotated, while the majority remains unlabeled. This scenario is prevalent in clinical settings, where expert annotations are costly and time-consuming, yet raw images are readily accessible. In such cases, supervised learning alone proves inadequate due to limited labeled data. 
\begin{figure}[htbp]
	\centering
\includegraphics[width=0.49\textwidth]{./Figures/Figs/semi-workflow.png}
	\caption{Overview of semi-supervised learning paradigm. Semi-SL includes a small set of labeled data and a large amount of unlabeled data to conduct learning jointly, aiming at leveraging the unlabeled data to boost learning performance. Semi-SL typically seeks to optimize the combination of a supervised loss function $\mathcal{L}_{sup}$ and an unsupervised loss function $\mathcal{L}_{unsup}$.  }
	\label{fig_semi_schematic}
\end{figure}
As illustrated in Fig.~\ref{fig_semi_schematic}, \textbf{semi-supervised learning (Semi-SL)} addresses this challenge by leveraging both labeled and unlabeled data during training. A core principle of Semi-SL is supervision propagation, which assumes that unlabeled data should have predictions consistent with labeled data. In this way, Semi-SL enables better generalization while reducing the need for manual annotation.

In the following, we categorize existing Semi-SL methods in MIA how each method enforces or approximates supervision consistency between labeled and unlabeled data into three categories: \textbf{proxy-labeling}, \textbf{generative modeling}, and \textbf{regularization}. Representative works are summarized in Appendix Tab.~\ref{tab:semi}.

\subsection{Proxy-labeling Methods}
\textbf{Proxy-labeling methods} utilize the idea of supervision consistency propagation by assigning pseudo labels to unlabeled samples and incorporate high-confidence examples into the training process through an iterative approach. These methods can be divided into two principal subcategories: \textit{Self-training methods} and \textit{multi-view learning methods}.

\subsubsection{Self-training Methods}

\textit{Self-training methods} operate through the bootstrapping mechanism. Initially, a prediction function $f_{\theta}$ with parameters $\theta$ is trained using available labeled data samples $x\in X_L$. Subsequently, this trained model generates predictions for unlabeled data samples $x\in X_U$. A confidence threshold $\tau$ is established, and sample-label pairs $(x, \mathrm{argmax}{f_{\theta}(x)})$ whose prediction confidence exceeds $\tau$ are added to the labeled dataset $X_L$. This augmented labeled dataset is then used to retrain the prediction function, creating an iterative cycle that continues until the model can no longer make sufficiently confident predictions on remaining unlabeled data.

Entropy minimization \cite{grandvalet2004semi} represents a foundational approach in this category, regularizing models based on the low-density assumption by encouraging low-entropy predictions for unlabeled data. Building on this concept, \textbf{Pseudo-label} \cite{lee2013pseudo} provides a straightforward yet effective self-training mechanism that applies entropy minimization principles in prediction space. While labeled samples undergo supervised training, unlabeled data receive labels corresponding to the model's most confident predictions. In medical image analysis, Pseudo-label has been widely employed as an auxiliary component to enhance model performance across various applications \cite{fan2020inf,zhang2022boostmis,chaitanya2023local}.

A significant challenge with proxy labels is their inherent noise and potential deviation from ground truth. To address this limitation, researchers have developed quality assurance mechanisms including uncertainty-aware confidence evaluation \cite{wang2021semiself}, conditional random field-based proxy label refinement \cite{bai2017semi}, and adversarial training-based methods \cite{zhou2019collaborative}. These approaches help ensure that proxy labels provide reliable supervisory signals during training. Pseudo-label has also been used in MIA to refine a given annotation with the assistance of unlabeled data. Qu \textit{et al.} \cite{qu2020weakly} introduce pseudo-label into nuclei segmentation and design an iterative learning algorithm to refine the background of weakly labeled images where only nuclei are annotated, leaving large areas ignored. Similar ideas can also be seen in \cite{nie2018asdnet}.


\subsubsection{Multi-view learning methods}
\textit{Multi-view learning methods} assume that each sample has two or multiple complementary views and features of the same sample extracted with different views are supposed to be consistent. Therefore, the key idea of multi-view learning methods is to train the model with multiple views of the sample or train multiple learners and minimize the disagreement between them, thus learning the underlying features of the data from multiple aspects. \textbf{Co-training} is a method that falls into this category. It assumes that data sample $x$ can be represented by two views, $\textbf{v}_1(x)$ and $\textbf{v}_{2}(x)$, and each of them are capable of solely training a good learner, respectively. Consequently, the two learners are set to make predictions of each view's unlabeled data, and iteratively choose the candidates with the highest confidence for the other model \cite{yang2021survey}. 
Another variation of multi-view learning methods is Tri-training \cite{zhou2005tri}, which is proposed to tackle the lack of multi-view data and mistaken labels of unlabeled data produced by self-training methods. Tri-training aims to learn three models from three different training sets obtained with bootstrap sampling. A deep learning version of Tri-training, i.e. Tri-Net, has been further proposed in \cite{dong2018tri}.

Co-training, or deep co-training, is dominant in multi-view learning in MIA, with a steady flow of publications \cite{zhao2019multi,zhou2019semi,xia2020uncertainty,wang2021selfco,fang2020dmnet,zeng2023pefat}. To conduct whole brain segmentation, Zhao \textit{et al.} \cite{zhao2019multi} implements co-training by obtaining different views of data with data augmentation. A similar idea can be seen for 3D medical image segmentation in \cite{xia2020uncertainty} and \cite{zhou2019semi}. These two works both utilize co-training by learning individual models from different views of 3D volumes such as the sagittal, coronal, and axial planes. Further works have been proposed to refine co-training. To produce reliable and confident predictions, Wang \textit{et al.} \cite{wang2021selfco} develops a self-paced learning strategy for co-training, forcing the network to start with the easier-to-segment regions and transition to the difficult areas gradually. Rather than discarding samples with low-quality pseudo-labels, Zeng \textit{et al.} \cite{zeng2023pefat} introduces a novel regularization approach, which focuses on extracting discriminative information from such samples by injecting adversarial noise at the feature level, thereby smoothing the decision boundary.
Meanwhile, to avoid the errors of different model components accumulating and causing deviation, Fang and Li \cite{fang2020dmnet} develop an end-to-end model called difference minimization network for medical image segmentation by conducting co-training with an encoder shared by two decoders.

\subsection{Generative Modeling Methods}
While proxy-labeling methods directly assign labels to unlabeled data, \textbf{generative modeling methods} realize supervision consistency propagation by assuming that both labeled and unlabeled data are sampled from a shared latent distribution. By learning this underlying distribution with the help of unlabeled data, these methods enable the model to transfer information across the entire dataset. The learned latent representation is then combined with supervised information from labeled examples to further improve performance.

% Concise version
\textbf{Generative adversarial networks (GANs)} effectively leverage both labeled and unlabeled data through a two-player minimax game between a generator $\mathcal{G}$ and discriminator $\mathcal{D}$ \cite{goodfellow2014generative}. In medical image analysis, several semi-supervised approaches incorporate unlabeled data during adversarial training. Chaitanya \textit{et al.} \cite{chaitanya2021semi} and Hou \textit{et al.} \cite{hou2022semi} utilize unlabeled samples to introduce greater variation in shape and intensity, enhancing model robustness. Zhou \textit{et al.} \cite{zhou2019collaborative} generate pseudo lesion masks for unlabeled data with quality facilitated by the discriminator. Other researchers modify the discriminator's objective beyond binary classification: Odena \textit{et al.} \cite{odena2016semi} extend it to predict $K$ classes plus an additional real/fake class, allowing unlabeled data to contribute to multi-class discrimination. This architecture has been successfully applied to retinal image synthesis \cite{kamran2021vtgan, diaz2019retinal, xie2023fundus}, glaucoma assessment \cite{diaz2019retinal}, chest X-ray classification \cite{madani2018semi}, and other medical imaging tasks \cite{hou2022semi}.

\textbf{Variational autoencoders (VAEs)} offer another effective approach for utilizing unlabeled data. Based on Bayesian inference theory \cite{kingma2013auto}, VAEs encode data into latent variables and reconstruct inputs by maximizing the variational lower bound. In medical image analysis, VAEs primarily learn feature similarities from large unlabeled datasets, creating well-constrained latent spaces that reduce dependence on labeled data \cite{sedai2017semi, wang2022rethinking}. Sedai \textit{et al.} \cite{sedai2017semi} proposed a dual-VAE framework for semi-supervised optic cup segmentation in retinal images, where one VAE learns data distribution from unlabeled data and transfers this knowledge to a second VAE performing segmentation with labeled data. Wang \textit{et al.} \cite{wang2022rethinking} adapted VAEs for 3D medical image segmentation by replacing the conventional mean vector and variance vector with a mean vector and covariance matrix, accounting for correlations between different slices of an input volume.

More recently, \textbf{diffusion models} \cite{ho2020denoising,yang2022diffusion,rombach2022high} have emerged as powerful alternatives in the generative Semi-SL domain. These models offer enhanced stability and sample quality through iterative denoising processes, showing potential in midline shift quantification \cite{gong2023diffusion} and medical image segmentation \cite{liu2024diffrect}. Their ability to model complex anatomical structures while enabling uncertainty quantification makes them particularly valuable for label-scarce scenarios.


\subsection{Regularization-based Methods}
In contrast to the explicit labeling approach of proxy methods and the distribution modeling of generative techniques, regularization-based methods enforce consistency through direct constraints on the model's behavior by assuming that the perturbation of data points does not change the prediction of the model, without requiring any label information. 

$\boldsymbol{\Pi}$\textbf{-model} \cite{sajjadi2016regularization} effectively implements consistency regularization by using a shared encoder to process differently augmented views of the same input and enforcing consistent predictions across these views, while incorporating label information to improve classifier performance. Li \textit{et al.} \cite{li2018semipi} achieved state-of-the-art skin lesion segmentation using this approach with only 300 labeled images, outperforming fully-supervised methods that required 2,000 labeled images. Similar consistency-based approaches appear in Bortsova \textit{et al.} \cite{bortsova2019semi}, who enforce prediction consistency across transformations for chest X-ray segmentation, and Meng \textit{et al.} \cite{meng2023dual}, who employ graph convolution networks to maintain regional and marginal consistency for semi-supervised optic disc and cup segmentation.

\textbf{Temporal ensembling} \cite{laine2016temporal} improves the $\Pi$-model's prediction stability by incorporating exponential moving averages, an approach widely adopted in medical image analysis \cite{cao2020uncertainty,gyawali2019semi,shi2020graph,luo2020deep}. For breast mass segmentation, Cao \textit{et al.} \cite{cao2020uncertainty} integrate uncertainty maps as guidance to ensure prediction reliability. Similarly, Luo \textit{et al.} \cite{luo2020deep} propose uncertainty-aware temporal ensembling for chest X-ray screening with partially labeled data. Gyawali \textit{et al.} \cite{gyawali2019semi} enhance the method by first using a VAE to extract disentangled latent space representations as stochastic embeddings, improving chest X-ray classification performance. A key characteristic of temporal ensembling is that each training sample's activation is updated only once per epoch.
 
\textbf{Mean teacher} \cite{tarvainen2017mean} applies exponentially moving average to model parameters rather than network activations, addressing the limitations of temporal ensembling and finding various applications in medical imaging \cite{li2020transformation,yu2019uncertainty,wang2020double,xu2023ambiguity,adiga2023anatomically}. Li \textit{et al.} \cite{li2020transformation} apply this approach to transformation-consistent medical image segmentation. Since teacher models can generate inaccurate targets for unlabeled data, Yu \textit{et al.} \cite{yu2019uncertainty} and Adiga \textit{et al.} \cite{adiga2023anatomically} incorporate uncertainty maps to ensure target reliability. Wang \textit{et al.} \cite{wang2020double} further propose a double-uncertainty-weighted method for left atrium and kidney segmentation, extending uncertainty from segmentation to feature level. Xu \textit{et al.} \cite{xu2023ambiguity} focus on selecting productive unsupervised consistency targets through an ambiguity-consensus mean-teacher model that better exploits complementary information from unlabeled data.

\subsection{Discussion}
Semi-supervised learning addresses the scarcity of labeled data by exploiting large amounts of unlabeled samples and enforcing supervision consistency across the dataset. 
% Newly Added
Data characteristics and task requirements should guide the choice of the Semi-SL strategy: proxy-labeling methods like self-training tend to perform well when high-confidence predictions can be reliably identified; multi-view learning approaches appear particularly suited for volumetric data where different perspectives provide complementary information; generative modeling shows promise with complex anatomical structures that benefit from learned prior distributions; while regularization-based methods often demonstrate robustness across diverse imaging modalities.
%
A persistent challenge in Semi-SL lies in the utilization of noisy or imperfect unlabeled data. The generation and selection of reliable pseudo labels are critical, as label noise can easily propagate through the training process and undermine model performance. Moreover, the theoretical understanding of how different Semi-SL techniques interact within hybrid systems remains limited, especially when dealing with heterogeneous data sources~\cite{wang2021deephybrid, zhang2022boostmis, wang2020focalmix, gyawali2020semi,miao2023caussl}. 

Meanwhile, the emergence of HFMs has significantly reshaped the landscape of semi-supervised learning in medical image analysis. As demonstrated by approaches like SemiSAM \cite{zhang2023semisam} and SemiSAM+ \cite{zhang2025semisam+}, these models introduce a paradigm shift from traditional model-centric Semi-SL methods focused on regularization strategies toward leveraging pre-trained knowledge to guide the learning process. Foundation models trained on large-scale datasets provide robust prior knowledge that helps specialist models learn more effectively with extremely limited labeled data—a scenario where conventional Semi-SL methods often struggle. This collaborative learning approach, where trainable specialist models interact with frozen foundation models, offers several advantages: it enhances performance in low-annotation regimes, provides more stable training due to knowledge transfer, and exhibits strong generalization capabilities across different medical imaging modalities and targets. As HFMs continue to evolve with improved architectures and more diverse training data, they will likely further transform Semi-SL in MIA, potentially reducing annotation requirements while increasing effectiveness and robustness.
\section{Self-supervised Learning in MIA} \label{sec:ssl}
\begin{table*}[ht]
\centering
\caption{Surveyed Self-supervised Learning-based Studies in Medical Image Analysis.}

\resizebox{\textwidth}{.56\textwidth}{
\begin{threeparttable}
\begin{tabular}{@{}llllll@{}}
\toprule
 &Reference & Organ & Proxy Task Design & Dataset & Publication \\ \midrule
\rule{0pt}{2ex} \multirow{20}{*}{\rotatebox{90}{Classification}}&  Li \textit{et al.} \cite{li2020self} & Retina & Multi-modal Contrastive Learning  & ADAM; PALM& TMI 2020\\
\cline{2-6}
% \rule{0pt}{2.5ex}&Zhao \textit{et al.} \cite{zhao2021anomaly}  & Retina; Lung & Inpainting; Local Pixel Shuffling;  & RetinalOCT; ChestX &RetinalOCT: AUC: 0.9642; F1: 0.9342\\
% &&&Non-Linear Intensity Transformation&&ChestX: AUC: 0.8265; F1: 0.8214\rule[-1.2ex]{0pt}{0pt}\\
% \cline{2-6}
\rule{0pt}{2.5ex}&Koohbanani \textit{et al.} \cite{koohbanani2021self} & Breast;  &Magnification Prediction;  & CAMELYON 2016; &TMI 2021  \\
&&Cervix;&Solving Magnification Puzzle;&KATHER;& \\
&&Colon&Hematoxylin Channel Prediction&Private Dataset: 217 Images&\rule[-1.2ex]{0pt}{0pt}\\
\cline{2-6}
% \rule{0pt}{2.5ex}&Li \textit{et al.} \cite{li2021rotation} & Retina &Image Rotation  & ADAM; PALM; DRD & ADAM: AUC: 0.7811; PALM: AUC: 0.9912 \rule[-1.2ex]{0pt}{0pt}\\
% \cline{2-6}
\rule{0pt}{2.5ex}&Azizi \textit{et al.} \cite{azizi2021big} & Skin; Lung &Multi-Instance Contrastive Learning & Priavte Dermatology Dataset; CheXpert  &CVPR 2021\rule[-1.2ex]{0pt}{0pt}\\
\cline{2-6}
% \rule{0pt}{2.5ex}&Yang \textit{et al.} \cite{yang2022cs} & Colon & Cross-stain prediction + Contrastive Learning& KATHER& Acc: 0.918\rule[-1.2ex]{0pt}{0pt}\\
% \cline{2-6}
\rule{0pt}{2.5ex}&Tiu \textit{et al.} \cite{tiu2022expert} & Lung & Contrastive Learning  & CheXpert&Nature BME 2022\rule[-1.2ex]{0pt}{0pt}\\
\cline{2-6}
\rule{0pt}{2.5ex}&Chen \textit{et al.} \cite{chen2022scaling} & Breast; Lung; Kidney & Contrastive Learning  & TCGA-BRCA; TCGA-NSCLS; TCGA-RCC&CVPR 2022\rule[-1.2ex]{0pt}{0pt}\\
\cline{2-6}
\rule{0pt}{2.5ex}&Mahapatra \textit{et al.} \cite{mahapatra2022self} & Lymph; Lung; & Contrastive Learning Variant  & CAMELYON 2017; DRD; GGC&TMI 2022\\
&&Retina; Prostate&&\rule[-1.2ex]{0pt}{0pt}\\
\cline{2-6}
\rule{0pt}{2.5ex}&Wang \textit{et al.} \cite{wang2023ssd} & Skin & Self-supervised Knowledge Distillation  & ISIC 2019 &MIA 2023 \rule[-1.2ex]{0pt}{0pt}\\
\cline{2-6}
\rule{0pt}{2.5ex}& Huang \textit{et al.} \cite{huang2024systematic}  &  Multi-Organ   & SimCLR, MOCOv2,
SwAV, BYOL, SimSiam, DINO, BarlowTw & TissueMNIST; PathMNIST; TMED-2; AIROGS& CVPR 2024 \rule[-1.2ex]{0pt}{0pt}\\
\cline{2-6}
\rule{0pt}{2.5ex}& Tang \textit{et al.} \cite{tang2024self}  &  Lung & Self-Supervised Representation Distribution Learning & TCGA-EGFR; TCGA-Lung; Private Lung Dataset & TMI 2024 \rule[-1.2ex]{0pt}{0pt}\\
\cline{2-6}
\rule{0pt}{2.5ex}& Vorontsov \textit{et al.} \cite{vorontsov2024foundation}  & Multi-Organ & Self-distillation + Masked Image Modeling (DINOv2) & Cancer-related Diagnosis Datasets & Nat. Med. 2024 \rule[-1.2ex]{0pt}{0pt}\\
\cline{2-6}
\rule{0pt}{2.5ex}& Chen \textit{et al.} \cite{chen2024towards}  & Multi-Organ & Self-distillation + Masked Image Modeling (DINOv2) & Cancer-related Diagnosis Datasets & Nat. Med. 2024 \rule[-1.2ex]{0pt}{0pt}\\
\cline{2-6}
\rule{0pt}{2.5ex}& Lu \textit{et al.} \cite{lu2024visual}  & Multi-Organ & Visual-language Contrastive Learning + Captioning (CoCa)  & Cancer-related Diagnosis Datasets & Nat. Med. 2024 \rule[-1.2ex]{0pt}{0pt}\\
\hline
\rule{0pt}{2.5ex} \multirow{14}{*}{\rotatebox{90}{Segmentation}}& Hervella \textit{et al.} \cite{hervella2018retinal}$_{2018}$ & Retina &Multi-modal Reconstruction & Isfahan MISP & MICCAI 2018 \rule[-1.2ex]{0pt}{0pt}\\
\cline{2-6}
\rule{0pt}{2.5ex}&Spitzer \textit{et al.} \cite{spitzer2018improving}$_{2018}$ & Brain & Patch Distance Prediction   & BigBrain & MICCAI2018\rule[-1.2ex]{0pt}{0pt}\\
\cline{2-6}
\rule{0pt}{2.5ex} &  Bai \textit{et al.} \cite{bai2019self} & Heart & Anatomical Position Prediction & Private Dataset: 3825 Subjects & MICCAI 2019\rule[-1.2ex]{0pt}{0pt}\\
\cline{2-6}
\rule{0pt}{2.5ex}& Sahasrabudhe \textit{et al.} \cite{sahasrabudhe2020self}  & Multi-Organ & WSI Patch Magnification Identification &MoNuSeg & MICCAI 2020\rule[-1.2ex]{0pt}{0pt}\\
\cline{2-6}
\rule{0pt}{2.5ex}& Tao \textit{et al.} \cite{tao2020revisiting} & Pancreas &  Rubik's Cube Recovery& NIH PCT; MRBrainS18& MICCAI 2020\rule[-1.2ex]{0pt}{0pt}\\
\cline{2-6}
\rule{0pt}{2.5ex}&Lu \textit{et al.} \cite{lu2021volumetric}& Brain & Fiber Streamlines Density Map Prediction;& dHCP &MIA 2021\\
&&& Registration-based Segmentation Imitation& & \rule[-1.2ex]{0pt}{0pt}\\
\cline{2-6}
\rule{0pt}{2.5ex}&Tang \textit{et al.} \cite{tang2022self} & Abdomen; Liver; &Contrastive Learning; Masked Volume Inpainting;    & DECATHLON; &CVPR 2022\\
&&Prostate&3D Rotation Prediction &BTCV&\rule[-1.2ex]{0pt}{0pt}\\
\cline{2-6}
\rule{0pt}{2.5ex}&Jiang \textit{et al.} \cite{jiang2023anatomical} & Multi-organ &Anatomical-invariant Contrastive Learning    & FLARE 2022; BTCV &CVPR 2023\rule[-1.2ex]{0pt}{0pt}\\
\cline{2-6}
\rule{0pt}{2.5ex}&He \textit{et al.} \cite{he2023geometric} & Heart; Artery; Brain &Geometric Visual Similarity Learning    & MM-WHS-CT; ASOCA; CANDI; STOIC &CVPR 2023\rule[-1.2ex]{0pt}{0pt}\\
\cline{2-6}
\rule{0pt}{2.5ex}&Liu \textit{et al.} \cite{liu2023hierarchical} & Tooth &Hierarchical Global-local Contrastive Learning    & Private Dataset: 13,000 Scans &TMI 2023\rule[-1.2ex]{0pt}{0pt}\\
\cline{2-6}
\rule{0pt}{2.5ex}&Zheng \textit{et al.} \cite{zheng2023msvrl} & Multi-Organ &Multi-scale Visual Representation Self-supervised Learning    & BCV; MSD; KiTS &TMI 2023\rule[-1.2ex]{0pt}{0pt}\\
\cline{2-6}
\rule{0pt}{2.5ex}&Peng \textit{et al.} \cite{peng2024boundary} & Heart; Prostate & Contrastive Learning  & ACDC; PROMISE12 & MIA 2024\rule[-1.2ex]{0pt}{0pt}\\
\cline{2-6}
\rule{0pt}{2.5ex}&Purma \textit{et al.} \cite{purma2024genselfdiff} & Multi-Organ & Diffusion-based Reconstruction & Head and Neck Cancer; GlaS; MoNuSeg & TMI 2024\rule[-1.2ex]{0pt}{0pt}\\
\hline
\rule{0pt}{2.5ex} \multirow{6}{*}{\rotatebox{90}{Regression}} & Abbet \textit{et al.} \cite{abbet2020divide}  & Gland &Image Colorization & Private Dataset: 660 Images&MICCAI 2020 \rule[-1.2ex]{0pt}{0pt}\\
\cline{2-6}
\rule{0pt}{2.5ex}& \multirow{3}{*}{Srinidhi \textit{et al.} \cite{srinidhi2022self}}  & Breast; & WSI Patch Resolution Sequence & BreastPathQ; &MIA 2022\\
&& Colon& Prediction&CAMELYON 2016; &  \\
&&&&KATHER &  \rule[-2ex]{0pt}{0pt}\\
\cline{2-6}
\rule{0pt}{2.5ex}&Fan \textit{et al.} \cite{fan2023cancerself}  &  Brain; Lung  & Image Colorization; Cross-channel   &  GBM; TCGA-LUSC; NLST &TMI 2023\rule[-1.2ex]{0pt}{0pt}\\
\hline
\rule{0pt}{2.5ex} \multirow{32}{*}{\rotatebox{90}{Others}} & Zhuang \textit{et al.} \cite{zhuang2019selfsupervised} & Brain &  Rubik's Cube Recovery & BraTS 2018; Private Dataset: 1,486 Images & MICCAI 2019\rule[-1.2ex]{0pt}{0pt}\\ 
\cline{2-6}
\rule{0pt}{2.5ex}& \multirow{3}{*}{Chen \textit{et al.} \cite{chen2019self}}  & \multirow{3}{*}{Multi-Organ} & \multirow{3}{*}{Disturbed Image Context Restoration} & Private Fetus Dataset: 2,694 Images; &MIA 2019\\
&&&&Private Multi-organ Dataset: 150 Images;& \\
&&&&BraTS 2017& \rule[-1.2ex]{0pt}{0pt}\\
\cline{2-6}
\rule{0pt}{2.5ex}&Zhao \textit{et al.} \cite{zhao2020smore}  &  Brain  & Super-resolution Reconstruction   &  Private Dataset: 47 Images &TMI 2020\rule[-1.2ex]{0pt}{0pt}\\
\cline{2-6}
\rule{0pt}{2.5ex}&Li \textit{et al.} \cite{li2020sacnn} & Abdomen  & CT Reconstruction  & LDCTGC &TMI 2020\rule[-1.2ex]{0pt}{0pt}\\
\cline{2-6}
\rule{0pt}{2.5ex}&Cao \textit{et al.} \cite{cao2020auto}  & Brain & Missing Modality Synthesis & BraTS 2015; ADNI & AAAI 2020\rule[-1.2ex]{0pt}{0pt}\\ 
\cline{2-6}
\rule{0pt}{2.5ex}&Haghighi \textit{et al.} \cite{haghighi2020learning} & Lung &  Self-Discovery + Self-Classification  & LUNA; LiTS; CAD-PE; BraTS 2018; &   MICCAI 2020   \\
&&&+Self-Restoration& ChestX-ray14; LIDC-IDRI; SIIM-ACR& \rule[-1.2ex]{0pt}{0pt}\\
\cline{2-6}
\rule{0pt}{2.5ex}&Taleb \textit{et al.} \cite{taleb20203d}& Brain; Retina;  & 3D Contrastive Predictive Coding; 3D Jigsaw Puzzles; & BraTS 2018; & NeurIPS 2020\\
&&Pancreas &3D Rotation Prediction; 3D Exemplar Networks&DECATHLON;& \\
&&&Relative 3D Patch Location;&DRD& \rule[-1ex]{0pt}{0pt}\\
\cline{2-6}
\rule{0pt}{2.5ex}&Li \textit{et al.} \cite{li2021single} &  Breast; Pancreas; Kidney &Super-resolution Reconstruction;  Color Normalization  & WTS; Private Dataset: 533 Images& MIA 2021 \rule[-1.2ex]{0pt}{0pt}\\
\cline{2-6}
\rule{0pt}{2.5ex}&Wang \textit{et al.} \cite{wang2021transpath} & Multi-Organ & Contrastive Learning & TCGA; KATHER; MHIST & MICCAI 2021\\
&&&&PAIP; PatchCAMELYON& \rule[-1.2ex]{0pt}{0pt} \\
\cline{2-6}
\rule{0pt}{2.5ex}&Zhou \textit{et al.} \cite{zhou2021preservational}& Lung; Brain; Liver & Contrastive Learning + Image Reconstruction & ChestX-ray14; CheXpert; LUNA &  CVPR 2021\\
&&&&BraTS 2018; LiTS;& \rule[-1.2ex]{0pt}{0pt}\\
\cline{2-6}
\rule{0pt}{2.5ex}&Yan \textit{et al.} \cite{yan2022sam}& Multi-Organ  & Global and Local Contrastive Learning  & DeepLesion; NIH LN; Private Dataset: 94 Patients& TMI 2022\rule[-1.2ex]{0pt}{0pt}\\
\cline{2-6}
\rule{0pt}{2.5ex}&Haghighi \textit{et al.} \cite{haghighi2022dira}& Lung & Contrastive Learning + Reconstruction + & ChestX-ray14; CheXpert; & CVPR 2022 \\
&&&Adversarial Learning & Montgomery&  \rule[-1.2ex]{0pt}{0pt}\\
\cline{2-6}
\rule{0pt}{2.5ex}&Cai \textit{et al.} \cite{cai2023dualself}& Lung; Brain; Retina& Dual-Distribution Reconstruction & RSNA-Lung; LAG; VinDr-CXR; Brain Tumor MRI;& MIA 2023 \\
&& &&Private Lung Dataset: 5,000 Images& \rule[-1.2ex]{0pt}{0pt}\\
\cline{2-6}
\rule{0pt}{2.5ex}&Li \textit{et al.} \cite{li2023generic}& Retina  & Frequency-boosted Image Enhancement  &EyePACS;& MIA 2023 \\
&&&&Private Dataset: more than 10,000 Images& \rule[-1.2ex]{0pt}{0pt}\\
% \cline{2-6}
% \rule{0pt}{2.5ex}&Xie \textit{et al.} \cite{xie2022unimiss}$_{2022\text{CS}}$ & Multi-Organ & Contrastive Learning  & BCV; RICORD; &BCV: DSC: 0.8499; RICORD: AUC: 0.8906;\\
% &&&&JSRT Database; ChestXR&JSRT Database: DSC: 0.9408; ChestXR: AUC: 0.9907\rule[-1.2ex]{0pt}{0pt}\\
\bottomrule
\end{tabular}
\begin{tablenotes}    
        \footnotesize               
        \item[1] For the sake of brevity, we denote references that contain more than one task in the following abbreviations: \textbf{C}: Classification, \textbf{S}:Segmentation, \textbf{D}:Detection, \textbf{SR}: Super-resolution, \textbf{DN}: Denoising, \textbf{IT}: Image Translation, \textbf{RE}: Registration. 
      \end{tablenotes}
\end{threeparttable}
}
\label{tab:self}
\end{table*}

\subsection{Reconstruction-Based Methods}
\textbf{Reconstruction-based methods} in Self-SL focus on exploring the inherent structures of data without the help of human annotations. These methods are conducted on several tasks including super-resolution \citep{li2021single,zhao2020smore}, inpainting \citep{zhao2021anomaly}, colorization \citep{abbet2020divide}, and the emerging MIA-specific application, multi-modal reconstruction \citep{cao2020auto,hervella2018retinal}. 

A straightforward way to establish the reconstruction task is proposed by \citet{li2020sacnn}, who adopt an auto-encoder network to encode and reconstruct normal-dose computed tomography (CT) images for learning the latent features by minimizing the mean squared error (MSE) loss. After self-supervised pre-training, the encoder is utilized for feature extraction, and a supervised loss is computed with the encoded latent features. However, the self-supervised pre-training based on the minimization of reconstruction loss might neglect the basic structure of the input image and capture the color space distribution instead \citep{abbet2020divide}. More proxy tasks have been motivated to solve this challenge.

The super-resolution reconstruction task is to generate fine-grained and realistic high-resolution images by utilizing low-resolution input images. In this proxy task, the targeted model can learn the underlying semantic features and structures of data. %In , 
\citet{zhao2020smore} propose an anti-aliasing algorithm based on super-resolution reconstruction to reduce aliasing and restore the quality of magnetic resonance images (MRIs). While \citet{li2023generic} utilize the frequency information in fundus image as guidance to conduct image enhancement. 
In the meantime, super-resolution reconstruction is also an appropriate proxy task for gigapixel histopathology whole-slide images (WSIs) because low-resolution WSIs are rather easy to store and process. From this application, \citet{li2021single} conduct single image super-resolution for WSIs using GAN. 

The image colorization task is to predict the RGB version of the gray-scale images. During this process, the network is trained to capture the contour and shape of different tissues in the sample and fill them with respective colors \citep{abbet2020divide, fan2023cancerself, lin2023nuclei}. \citet{abbet2020divide} introduce the image colorization task into survival analysis of colorectal cancer. They train a convolutional auto-encoder to convert the original input image into a two-channel image, namely, hematoxylin and eosin. Then, MSE loss is applied to measure the difference between the original input image and its converted counterpart. Moreover, in the context of survival analysis,  \citet{fan2023cancerself} extend their methodology beyond image colorization to include a cross-channel pre-text task. This additional task challenges the model to restore the lightness channel in image patches, utilizing the information from their color channels.

The image inpainting task aims to predict and fill in missing parts based on the remaining regions of the input image. This proxy task allows the model to recognize the common features of identical objects, such as color and structure, and thus to predict the missing parts consistently with the rest of the image. \citet{zhao2021anomaly} propose a restoration module based on Self-SL to facilitate the anomaly detection of optical coherence tomography (OCT) and chest X-ray. It demonstrates that the restoration of missing regions facilitates the model's learning of the anatomic information.

In recent years, the multi-modal reconstruction task has emerged  \citep{cao2020auto,hervella2018retinal}. In this task, the model uses the aligned multi-modal images of a patient to reconstruct an image in one modality by taking another modality as the input. \citet{hervella2018retinal} propose this proxy task to enrich the model with joint representations of different modalities, arguing that each modality offers a complementary aspect of the object. Therefore, they take retinography and fluorescein angiography into consideration to facilitate retinal image understanding. Meanwhile,  \citet{cao2020auto} develop a self-supervised collaborative learning algorithm, aiming at learning modality-invariant features for medical image synthesis by generating the missing modality with auto-encoder and GAN.

\subsection{Context-Based Methods}
\textbf{Context-based methods} utilize the inherent context information of the input image. Recent years have witnessed attempts to design novel predictive tasks for specific MIA tasks by training the network for prediction of the output class or localization of objects with the original image as the supervision signal \citep{bai2019self,spitzer2018improving,srinidhi2022self}. \citet{bai2019self} propose a proxy task to predict the anatomical positions from cardiac chamber view planes by applying an encoder-decoder structure. This proxy task properly employs the chamber view plane information, which is available from cardiac MR scans easily. While \citet{zheng2023msvrl} aims to perform finer-grained representation and deal with different target scales by designing a multi-scale consistency objective to boost medical image segmentation. Further advancements in proxy tasks for 3D medical images are presented by \citet{he2023geometric}. They propose a novel paradigm, termed Geometric Visual Similarity Learning, which integrates a topological invariance prior into the assessment of inter-image similarity. This approach aims to ensure consistent representation of semantic regions.
In addition, \citet{srinidhi2022self} propose an MIA-specific proxy task, Resolution Sequence Prediction, which utilizes the multi-resolution information contained in the pyramid structure of WSIs. A neural network is employed to predict the order of multi-resolution image patches out of all possible sequences that can be generated from these patches. In this way, both contextual structure and local details can be captured by the network at lower and higher magnifications, respectively. 

Other efforts have been made to explore the spatial context structure of input data, such as the order of different patches constituting an image, or the relative position of several patches in the same image, which can provide useful semantic features for the network. \citet{chen2019self} focus on the proxy task, dubbed context restoration, of randomly switching the position of two patches in a given image iteratively and restoring the original image. During this process, semantic features can be learned in a straightforward way. Instead of concentrating on the inherent intensity distribution of an image, \citet{li2021rotation} aims to improve the performance of a network with rotation angle prediction as the proxy task. The input retinal images are first augmented, generating several views, then randomly rotated. %in angles \{0°, 90°, 180°, 270°\}
The model is encouraged to predict the rotation angle and cluster the representations with similar features. More advanced proxy tasks such as Jigsaw Puzzles \citep{freeman1964apictorial} and Rubik's Cube \citep{korf1985macro} are also attracting an increasing number of researchers. \citet{taleb2021multimodal} improve the Jigsaw Puzzle task with multi-modal data. Concretely, an input image is constituted of out-of-order patches of different modalities and the model is expected to restore the original image. Rubik's Cube is a task set for 3-dimensional data. \citet{zhuang2019selfsupervised} and \citet{tao2020revisiting} introduce Rubik's Cube into the MIA area, and significantly boost the performance of a deep learning model on 3D data. In this method, the 3D volume will first be cut into a grid of cubes and a random rotation operation will be conducted on these cubes. The aim of this proxy task is to recover the original volume. 

However, for histopathology images, common proxy tasks such as prediction of the rotation or relative position of objects may only provide minor improvements to the model in histopathology due to the lack of a sense of global orientation in WSIs \citep{graham2020dense,koohbanani2021self}. Therefore, \citet{koohbanani2021self} propose proxy tasks targeted at histopathology, namely, magnification prediction, solving magnification puzzle, and hematoxylin channel prediction. In this way, their model can significantly integrate and learn the contextual, multi-resolution, and semantic features inside the WSIs.

\subsection{Contrastive-Based Methods}
\textbf{Contrastive-based methods} are based on the idea that the learned representations of different views of the same image should be similar and those of different images should be clearly distinguishable. Intriguingly, the ideas behind several high-performance algorithms such as SimCLR \citep{chen2020simple} and BYOL \citep{grill2020bootstrap} have been incorporated into the MIA field \citep{azizi2021big,wang2021transpath}. Multi-Instance Contrastive Learning (MICLe), is proposed by  \citet{azizi2021big}, is a refinement and improvement of SimCLR. Instead of using one input to generate augmented views for contrastive learning, they propose to minimize the disagreement of several views from multiple input images of the same patient, creating of more positive pairs. Meanwhile, \citet{wang2021transpath} adopt the BYOL architecture to facilitate histopathology image classification. A contribution of their work was to collect the currently largest WSI dataset for Self-SL pre-training. It includes 2.7 million patches cropped from 32,529 WSIs covering over 25 anatomic sites and 32 classes of cancer subtypes. Similarly, \citet{ghesu2022self} develop a contrastive learning and online clustering algorithm based on over 100 million radiography, CT, MRI, and ultrasound images. By leveraging this large unlabeled dataset for pre-training, the performance and convergence rate of the proposed model show a significant improvement over the state-of-the-art. Another line of work that utilizes large-scale unsupervised dataset is \citep{nguyen2023lvm}, in which over 1.3 million multi-modal data from 55 publicly available datasets are integrated. In addition to considering different perspectives of the same input, \citet{jiang2023anatomical} introduce a contrastive objective for the learning of anatomically invariant features. This approach is designed to fully exploit the inherent similarities in anatomical structures across diverse medical imaging volumes.

Further studies take into account the global and local contrast for better representation learning. Their methods usually minimize the InfoNCE loss \citep{oord2018representation} to capture the global and local level information. In \citep{yan2022sam}, the authors implement the InfoNCE by encoding each pixel of the input image. Their goal is to generate embeddings that can precisely describe the anatomical location of that pixel. To achieve this, they develop a pixel-level contrastive learning framework to generate embeddings at both the global and local level. Further, \citet{liu2023hierarchical} propose a hierarchical contrastive learning objective to capture the unsupervised representation of intra-oral mesh scans from point-level, region-level, and cross-level.

\subsection{Hybrid Methods}
Studies have made efforts to combine some or all of the different types of Sefl-SL methods into a universal framework to learn latent representations from multiple perspectives, such as semantic features and structure information inside unlabeled data \citep{haghighi2020learning,tang2022self,yang2022cs,zhou2021preservational}. For instance, \citet{tang2022self} combine masked volume inpainting, contrastive coding, and image rotation tasks into a Swin Transformer encoder architecture for medical image segmentation.

\subsection{Discussion}
Self-SL methods aim to learn and obtain a model with prior knowledge by manipulating and exploiting unlabeled data. The key to the superior performance of Self-SL models is the design of proxy tasks. Numerous existing Self-SL methods directly adopt proxy tasks prevailing in natural image processing into the MIA field. However, the unique properties of medical images, such as CT, WSI, and MRI, should be exclusively considered and injected into the design process of proxy tasks. The medical field has witnessed pioneering research efforts, exemplified by \citet{zhang2023dive}, that aim to establish guidelines for the design of Self-SL proxy tasks.
Further, proxy task design based on the combination of different medical image modalities is a prospective research direction, during which the model can capture disentangled features of each modality, leading to a robust pre-trained network. For example, large vision-language pre-trained models \citep{park2023self,zhou2022generalized,zhou2023advancing} are emerging in chest X-ray and obtaining ever-increasing research interests.
\section{Multi-instance Learning in MIA} \label{sec: mil}
\begin{table*}[ht]
\centering
\caption{Overview of Multi-instance Learning-based Studies in Medical Image Analysis}

\resizebox{\textwidth}{.33\textwidth}{
\begin{threeparttable}
\begin{tabular}{@{}llllll@{}}
\toprule
 &Reference$_{\text{Year}}$ & Organ & MIL Algorithm Design & Dataset & Result \\ \midrule
\rule{0pt}{2ex} \multirow{43}{*}{\rotatebox{90}{Classification}}&  Manivannan \textit{et al.} \cite{manivannan2017subcategory}$_{2017}$ &  Retina;   & Discriminative Subspace Transformation +  & Messidor; TMA-UCSB;  &Messidor: Acc: 0.728; TMA-UCSB: AUC: 0.967;\\
&&Breast&Margin-based Loss&DR Dataset; Private Dataset: 884 Images&DR Dataset: Acc: 0.8793; Private: Kappa: 0.7212\rule[-1.2ex]{0pt}{0pt}\\
\cline{2-6}
% \rule{0pt}{3ex}&Zhu \textit{et al.} \cite{zhu2017deep}$_{2017}$ & Breast & Sparse MIL & INBreast & AUC: 0.89\rule[-1.2ex]{0pt}{0pt}\\
% \cline{2-6}
% \rule{0pt}{3ex}& \rule{0pt}{2.5ex}Mercan \textit{et al.} \cite{mercan2017multi}$_{2017}$& Breast & Multi-Label MIL & BCSC & Average-P (Average-Precision): 0.8068\rule[-1.2ex]{0pt}{0pt}\\
% \cline{2-6}
\rule{0pt}{2.5ex}&Ilse \textit{et al.} \cite{ilse2018attention}$_{2017}$ &  Breast; Colon & Attention-based MIL   &  TMA-UCSB; CRCHistoPhenotypes &TMA-UCSB: Acc: 0.755; CRCHistoPhenotypes: Acc: 0.898\rule[-1.2ex]{0pt}{0pt}\\
\cline{2-6}
\rule{0pt}{2.5ex}&Couture \textit{et al.} \cite{couture2018multiple}$_{2018}$ &  Breast  & Quantile Function-based MIL   &  CBCS3  &Acc: 0.952\rule[-1.2ex]{0pt}{0pt}\\
\cline{2-6}
% \rule{0pt}{2.5ex}&Das \textit{et al.} \cite{das2018multiple}$_{2018}$ &  Breast &  Multiple Instance Pooling   & BreakHis & Acc: 0.8906\rule[-1.2ex]{0pt}{0pt}\\
% \cline{2-6}
\rule{0pt}{2.5ex}&Liu \textit{et al.} \cite{liu2018landmark}$_{2018}$  & Brain & Landmark-based MIL    & ADNI; MIRIAD & ADNI: AUC: 0.9586; MIRIAD: AUC: 0.9716\rule[-1.2ex]{0pt}{0pt}\\
\cline{2-6}
\rule{0pt}{2.5ex}&Campanella \textit{et al.} \cite{campanella2019clinical}$_{2019}$ &  Prostate; Skin; Lymph  & MIL + RNN   &  Private Dataset: 44,732 Images  & AUC: Prostate: 0.986; Skin: 0.986; Lymph: 0.965\rule[-1.2ex]{0pt}{0pt}\\
\cline{2-6}
\rule{0pt}{2.5ex}&Wang \textit{et al.} \cite{wang2019rmdl}$_{2019}$ &  Breast &  Instance Features Recalibration   & Private Dataset: 608 Images & Acc: 0.865\rule[-1.2ex]{0pt}{0pt}\\
\cline{2-6}
% \rule{0pt}{2.5ex}&Tennakoon \textit{et al.} \cite{tennakoon2019classification}$_{2019}$ & Retina; Lung & Extreme Value Theorem-based MIL & ReTOUCH; DLCST & DLCST: AUC: 0.96\\
% \cline{2-6}
\rule{0pt}{2.5ex}&Yao \textit{et al.} \cite{yao2019deep}$_{2019}$ & Lung; Brain & Multiple Instance FCN & NLST; TCGA&NLST: C-Index: 0.678; TCGA: C-Index: 0.657\rule[-1.2ex]{0pt}{0pt}\\
\cline{2-6}
\rule{0pt}{2.5ex}&Wang \textit{et al.} \cite{wang2020ud}$_{2020}$  &  Retina  & Uncertainty-aware MIL + RNN Aggregation  &  Duke-AMD; Private Dataset: 4,644 Volumes & Acc: Duke-AMD: 0.979; Private Dataset: 0.951 \rule[-1.2ex]{0pt}{0pt}\\
\cline{2-6}
\rule{0pt}{2.5ex}&Zhao \textit{et al.} \cite{zhao2020predicting}$_{2020}$  &  Colon & VAE-GAN Feature Extraction +    & TCGA-COAD & Acc: 0.6761; F1: 0.6667;\\
&&&GNN Bag-level Representation Learning&&AUC: 0.7102\rule[-1.2ex]{0pt}{0pt}\\
\cline{2-6}
\rule{0pt}{2.5ex}&Chikontwe \textit{et al.} \cite{chikontwe2020multiple}$_{2020}$ & Colon & Jointly Learning of Instance- and Bag-level Feature   & Private Dataset: 366 Images & F1: 0.9236; P (Precision): 0.9254; R (Recall): 0.9231; Acc: 0.9231\rule[-1.2ex]{0pt}{0pt}\\
\cline{2-6}
\rule{0pt}{2.5ex}&Raju \textit{et al.} \cite{raju2020graph}$_{2020}$ & Colon & Graph Attention MIL  & MCO& Acc: 0.811; F1: 0.798\rule[-1.2ex]{0pt}{0pt}\\
\cline{2-6}
\rule{0pt}{2.5ex}&Han \textit{et al.} \cite{han2020accurate}$_{2020}$ & Lung & Automatic Instance Generation & Private Dataset: 460 Examples & AUC: 0.99\rule[-1.2ex]{0pt}{0pt}\\
\cline{2-6}
\rule{0pt}{2.5ex}&Yao \textit{et al.} \cite{yao2020whole}$_{2020}$  & Lung; Colon & Siamese Multi-instance FCN + Attention MIL  & NLST; MCO &NLST: AUC: 0.7143; MCO: AUC: 0.644\rule[-1.2ex]{0pt}{0pt}\\
\cline{2-6}
\rule{0pt}{2.5ex}&Hashimoto \textit{et al.} \cite{hashimoto2020multi}$_{2020}$ & Lymph & Domain Adversarial + Multi-scale MIL  & Private Dataset: 196 Images & Acc: 0.871 \rule[-1.2ex]{0pt}{0pt}\\
\cline{2-6}
\rule{0pt}{2.5ex}&Shao \textit{et al.} \cite{shao2021transmil}$_{2021}$  &  Breast; Lung; Kidney   & Transformer-based MIL  & CAMELYON 2016; TCGA-NSCLC; TCGA-RCC&Acc: CAMELYON: 0.8837; TCGA-NSCLC: 0.8835; TCGA-RCC: 0.9466\rule[-1.2ex]{0pt}{0pt}\\
\cline{2-6}
\rule{0pt}{2.5ex}&Li \textit{et al.} \cite{li2021dual}$_{2021}$ & Breast; Lung & Dual-stream MIL + Contrastive Learning  & CAMELYON 2016; TCGA Lung Cancer &CAMELYON 2016: AUC: 0.9165; TCGA: AUC: 0.9815\rule[-1.2ex]{0pt}{0pt}\\
\cline{2-6}
\rule{0pt}{2.5ex}&Li \textit{et al.} \cite{li2021novel}$_{2021}$ & Lung & Virtual Bags + Self-SL Location Prediction  & Private Dataset: 460 Examples & AUC: 0.981; Acc: 0.958; F1: 0.895; Sens: 0.936 \rule[-1.2ex]{0pt}{0pt}\\
\cline{2-6}
\rule{0pt}{2.5ex}&Lu \textit{et al.} \cite{lu2021data}$_{2021}$ & Kidney; Lung;& Attention-based MIL + Clustering  & TCGA-RCC + Private Dataset: 135 WSIs; & Kidney: AUC: 0.972;\\
&&Lymph node&&CPTAC-NSCLC + Private Dataset: 131 WSIs;&Lung: AUC: 0.975;\\
&&&&CAMELYON 2016,17 + Private Dataset: 133 WSIs&Lymph node: AUC: 0.940\rule[-1.2ex]{0pt}{0pt}\\
\cline{2-6}
\rule{0pt}{2.5ex}&Wang \textit{et al.} \cite{wang2022lymph}$_{2022}$ & Thyroid & Transformer-based MIL + Knowledge Distillation  & Private Dataset: 595 Images &AUC: 0.9835; P: 0.9482; R: 0.9151; F1: 0.9297\rule[-1.2ex]{0pt}{0pt}\\
\cline{2-6}
\rule{0pt}{2.5ex}&Zhang \textit{et al.} \cite{zhang2022dtfd}$_{2022}$ & Breast; Lung  & Double-Tier Feature Distillation MIL & CAMELYON 2016; TCGA-Lung &CAMELYON 2016: AUC: 0.946; TCGA-Lung: AUC: 0.961 \rule[-1.2ex]{0pt}{0pt}\\
\cline{2-6}
\rule{0pt}{2.5ex}&Schirris \textit{et al.}\cite{schirris2022deepsmile}$_{2022}$ & Breast; Colon & Heterogeneity-aware MIL + Contrastive Learning  & TCGA-CRCk; TCGA-BC  &TCGA-CRCk: AUC: 0.87; TCGA-BC: AUC: 0.81\rule[-1.2ex]{0pt}{0pt}\\
\cline{2-6}
\rule{0pt}{2.5ex}&Su \textit{et al.} \cite{su2022attention2majority}$_{2022}$ & Breast; Kidney & Intelligent Sampling Method + Attention MIL & CAMELYON 2016; Private Dataset: 112 Images &CAMELYON 2016: AUC: 0.891; Private: AUC: 0.974\rule[-1.2ex]{0pt}{0pt}\\
\cline{2-6}
\rule{0pt}{2.5ex}&Zhu \textit{et al.} \cite{zhu2022murcl}$_{2022}$ & Breast; Lung; Kidney& Reinforcement Learning + Contrastive Learning + MIL  & CAMELYON 2016; TCGA-Lung; TCGA-Kidney & AUC: CAMELYON: 0.9452; TCGA-Lung: 0.9637; TCGA-Kidney:  0.9573\rule[-1.2ex]{0pt}{0pt}\\
\cline{2-6}
\rule{0pt}{2.5ex}&Yang \textit{et al.} \cite{yang2022micl}$_{2022}$ & Colon; Muscle & Curriculum Learning + MIL  & CRCHistoPhenotypes; Private Muscle Dataset: 266 Images&CRCHistoPhenotypes: AUC: 0.898; Private: AUC: 0.907\rule[-1.2ex]{0pt}{0pt}\\
\cline{2-6}
\rule{0pt}{2.5ex}&Shi \textit{et al.} \cite{shi2023structure}$_{2023}$ & Stomach; Bladder & Multi-scale Graph MIL  & TCGA-STAD; TCGA-BLCA; Private Stomach Dataset: 574 Images&AUC: TCGA-STAD: 0.829; TCGA-BLCA: 0.886; Private: 0.907\rule[-1.2ex]{0pt}{0pt}\\
\cline{2-6}
\rule{0pt}{2.5ex}&Yan \textit{et al.} \cite{yan2023genemutation}$_{2023}$ & Bladder & Hierarchical Deep MIL  & TCGA-Bladder & TCGA-Bladder: AUC: 0.92\rule[-1.2ex]{0pt}{0pt}\\
\cline{2-6}
\rule{0pt}{2.5ex}&Shi \textit{et al.} \cite{shi2023mg}$_{2023}$ & Breast; Kidney& Multi-scale Transformer + MIL  & BRIGHT; TCGA-BRCA; TCGA-RCC & AUC: BRIGHT: 0.848; TCGA-BRCA: 0.921; TCGA-RCC: 0.990\rule[-1.2ex]{0pt}{0pt}\\
\cline{2-6}
\rule{0pt}{2.5ex}&Liu \textit{et al.} \cite{liu2024advmil}$_{2024}$ & Lung; Breast; Brain & GAN + MIL  & NLST; TCGA-BRCA; TCGA-LGG&C-Index: NLST:  0.672; TCGA-BRCA: 0.566; TCGA-LGG: 0.642\rule[-1.2ex]{0pt}{0pt}\\
\hline
\rule{0pt}{3.5ex} \multirow{2}{*}{\rotatebox{90}{Segmentation}}& Jia \textit{et al.} \cite{jia2017constrained}$_{2017}$ &  Colon  & Multi-scale MIL + Area Constraint Regularization &  Private TMA/Colon Dataset: 60 Images/910 Images &F1: TMA: 0.622; Colon: 0.836\rule[-2ex]{0pt}{0pt}\\
\cline{2-6}
\rule{0pt}{3.5ex}&Xu \textit{et al.} \cite{xu2019camel}$_{2019}$ &  Breast  & Instance-level and Pixel-level Label Generation &  CAMELYON 2016 & Image-level Acc: 0.929; Pixel-level IoU: 0.847\rule[-2ex]{0pt}{0pt}\\
\cline{2-6}
\rule{0pt}{3.5ex} &  Dov \textit{et al.} \cite{dov2021weakly}$_{2021}$ &  Thyroid  & Maximum Likelihood Estimation-based MIL &  Private Dataset: 908 Images& AUC: 0.87\rule[-2ex]{0pt}{0pt}\\
\hline
\rule{0pt}{2.5ex} \multirow{4}{*}{\rotatebox{90}{Others}} &Schwab \textit{et al.} \cite{schwab2020localization}$_{2020\text{CD}}$ &  Lung  & Jointly Classification and Localization &  RSNA-Lung; MIMIC-CXR; Private Dataset: 1,003 Images&  AUC: 0.93\rule[-1.2ex]{0pt}{0pt}\\
\cline{2-6}
\rule{0pt}{2.5ex}&Wang \textit{et al.} \cite{wang2021learning}$_{2021\text{CS}}$  &  Pancreas  & Jointly Global-level Classification and    &  Private Dataset: 800 Images  & DSC: 0.6029; \\
&&&Local-level Segmentation&&Sens: 0.9975\rule[-1.2ex]{0pt}{0pt}\\
\bottomrule
\end{tabular}
\begin{tablenotes}    
        \footnotesize               
        \item[1] For the sake of brevity, we denote references that contain more than one task in the following abbreviations: \textbf{C}: Classification, \textbf{S}:Segmentation, \textbf{D}:Detection. 
        % \item[1] More Semi-supervised learning studies are included in Table \ref{tab:semi2}.
      \end{tablenotes}
\end{threeparttable}
}
\label{tab:mil}
\end{table*}

\subsection{Local Detection}
Since we define \textbf{local detection} as detecting or localizing all the particular disease patterns of an input image, papers with the purpose of segmentation or localization can be classified into this category. Most researchers design their local detection model to infer every patch label and thereby obtain both the local annotations and the global labels. Thus, the \textbf{local detection} methods often include \textbf{global detection} methods since inferring image-level labels after obtaining the local annotations. 

Schwab \textit{et al.} \cite{schwab2020localization}, apply the basic MIL algorithm to conduct the localization and classification of chest X-rays. They input every patch of the original sample into a CNN, and the model outputs a score for the patch representing its probability of containing a critical finding. Once the patch-level classifier is trained, the most straightforward way to perform slide-level classification is to integrate the patch-level predictions with max-pooling or average-pooling. The design for the pooling function plays an important role in the performance improvement of the MIL algorithm. For instance, in \cite{couture2018multiple}, the authors design a more general MIL aggregation method by utilizing a quantile function as the pooling function. By doing so, a more thorough description of the heterogeneity of each sample can be provided, enhancing the quality of global classification.
Other studies \cite{ilse2018attention,wang2019rmdl} propose learning-based aggregation operators to provide insight into the contribution of each instance to the bag. Among them, several are based on the attention-based MIL developed by Ilse \textit{et al.} in \cite{ilse2018attention}. By introducing the attention mechanism into MIL, their model can better capture the key features of regions of interest with interpretation. For the pancreatic ductal adenocarcinoma (PDAC) prediction problem, Wang \textit{et al.} \cite{wang2021learning} design an inductive attention guidance network for both classification and segmentation. The attention mechanism works as a connection between the global classifier and local (instance) segmenter by guiding the location of PDAC regions. 

Other intriguing improvements in \textbf{local detection} are springing up as well. Researchers have tried many different ways to facilitate instance prediction\cite{dov2021weakly,manivannan2017subcategory,jia2017constrained,xu2019camel}. Dov \textit{et al.} \cite{dov2021weakly} demonstrate that the general MIL methods perform poorly on cytopathology data for two reasons: instances that contain key information are located sparsely in a gigapixel pathology image, and the informative instances have various characteristics of abnormality. Thus, they propose a MIL structure involving maximum likelihood estimation to predict multiple labels, i.e., bag-level labels and diagnostic scores; instance-level labels and informativeness, simultaneously. Similarly, when studying the classification of the retinal nerve fiber layer (RNFL), Manivannan \textit{et al.} \cite{manivannan2017subcategory} have observed that regions that contain the RNFL generally have strong intra-class variation, making them difficult to distinguish from other regions. Therefore, they map the instances into a discriminative subspace to increase the discrepancy for disentangled instance feature learning. Jia \textit{et al.} \cite{jia2017constrained} incorporate the multi-scale image feature into the learning process to obtain more latent information on histopathology images. Finally, to address the problem that only image-level labels are provided in MIL, Xu \textit{et al.} \cite{xu2019camel} design an automatic instance-level label generation method. Their work has led to an interesting MIL algorithm design direction and may shed light on how to improve the performance of \textbf{local detection} algorithms.

In parallel, there has been significant progress in related domains such as phenotype categorization \cite{yao2020whole, hashimoto2020multi, yao2019deep} and multi-label classification \cite{mercan2017multi}. These investigations have further exemplified the versatility and potential of the MIL algorithm in addressing complex challenges across various subfields. 

\subsection{Global Detection}
\textbf{Global detection} refers to methods that simply aim to find out whether or not target patterns exist. For example, for the COVID-19 screening problem, researchers \cite{li2021novel} have designed MIL algorithms to classify an input sample as severe or not instead of locating every abnormal patch. 

To facilitate the prediction of image-level labels (\textit{e.g.} WSI-level label), researchers normally start from one of two aspects, namely instance- and bag-level. Most existing MIL algorithms \cite{tomita2019attention, hashimoto2020multi,naik2020deep,lu2021data} are based on the basic assumption that instances of the same bag are independent and identically distributed. Consequently, the correlations among instances are neglected, which is not realistic. Recently several works have taken the correlation among instances or tissues into consideration\cite{shao2021transmil,wang2022lymph,wang2019rmdl,raju2020graph,han2020accurate}. In \cite{shao2021transmil}, Shao \textit{et al.} introduce Vision Transformer (ViT) into MIL for gigapixel WSIs due to its great advantage in capturing the long-distance information and correlation among instances in a sequence. Meanwhile, to conduct precise lymph node metastasis prediction, Wang \textit{et al.} \cite{wang2022lymph} not only incorporate a pruned Transformer into MIL but also develop a knowledge distillation mechanism based on other similar datasets, such as a papillary thyroid carcinoma dataset, effectively avoiding the overfitting problem caused by the insufficient number of samples in the original dataset. Similarly, Raju \textit{et al.} \cite{raju2020graph} design a graph attention MIL algorithm for colorectal cancer staging, which utilizes different tissues as nodes to construct graphs for instance relation learning. Further, in order to utilize the multi-resolution characteristics of WSIs, Shi \textit{et al.} \cite{shi2023structure} consider WSIs as multi-scale graphs and utilize attention mechanism to integrate their information for primary tumor stage prediction. Similar idea can be found in \cite{yan2023genemutation, shi2023mg, xiang2023multi}. Besides, Liu \textit{et al.} \cite{liu2024advmil} firstly propose an integration of GAN with MIL mechanism for robust and interpretable WSI survival analysis by more accurately estimating target distribution. 

For bag-level improvement, recent years have witnessed two feasible approaches, namely, improved pooling methods and pseudo bags. On the one hand, in order to aggregate the instances with the most information, some researchers have developed novel aggregation methods in MIL algorithms instead of the traditional max pooling \cite{chikontwe2020multiple,das2018multiple}. For example, in \cite{chikontwe2020multiple}, the authors design a pyramid feature aggregation method to directly obtain a bag-level feature vector. On the other hand, however, there is an inherent problem for MIA, especially for histopathology --- the number of WSIs (bags) is usually small, while in contrast, one WSI has numerous patches, leading to an imbalance in the number of bags and instances. To address this problem, Zhang \textit{et al.} \cite{zhang2022dtfd} randomly split the instances of a bag into several smaller bags, called "pseudo bags", with labels that are consistent with the original bag. A similar idea can also be seen in \cite{li2021novel}.

Other improvements in MIL algorithms are also worth mentioning\cite{su2022attention2majority,tennakoon2019classification,wang2020ud}. In \cite{su2022attention2majority}, an intelligent sampling method is developed to collect instances with high confidence. This method excludes patches shared among different classes and tends to select the patches that match with the bag-level label. In \cite{tennakoon2019classification}, the authors utilize the extreme value theory to measure the maximum feature deviations and consequently leverage them to recognize the positive instances, while in \cite{wang2020ud}, Wang \textit{et al.} introduce an uncertainty evaluation mechanism into MIL for the first time, and train a robust classifier based on this mechanism to cope with OCT image classification problem. 

\subsection{Discussion}
Multi-instance learning in MIA is mainly applied to whole slide image analysis, which can be described as ``a needle in a haystack" problem, making bag-level decisions out of thousands of instances. MIL methods are developed to locate the discriminative patches as a basis for diagnosis. To achieve this goal, MIL research can be divided into several focuses. For the bag-instance correlation, a WSI is represented as a bag containing selected patches during training, which leads to the question of how the patches should be selected to make the bag representative of the WSI. Further, how to handle and leverage the imbalance of positive and negative samples could have a significant impact on model performance. For the instance-instance correlation, the proper modeling and utilization of instance relations can boost the performance of MIL algorithms and advance the interpretability of the model.



\section{Active Learning in MIA} \label{sec:al}
\begin{table*}[ht]
\centering
\caption{Overview of Active Learning-based Studies in Medical Image Analysis}

\resizebox{\textwidth}{.13\textwidth}{
\begin{threeparttable}
\begin{tabular}{@{}llllll@{}}
\toprule
 &Reference (Year) & Organ & Sampling Method & Dataset & Result \\ \midrule
\rule{0pt}{2ex} \multirow{1}{*}{\rotatebox{90}{Classification}}&  
%Gal \textit{et al.} 
\citet{gal2017deep} &  Skin  & BALD + KL-divergence  &  ISIC 2016 & 22\%  image input: AUC: 0.75\rule[-1.5ex]{0pt}{0pt}\\ 
\cline{2-6}
\rule{0pt}{2.5ex}& 
%Wu \textit{et al.} 
\citet{wu2021covid} &  Lung  & Loss Prediction Network  & CC-CCII Dataset & 42\% Chest X-Ray input: Acc: 86.6\%\rule[-1.5ex]{0pt}{0pt}\\ 
\cline{2-6}
\rule{0pt}{3ex}&
%Li \textit{et al.} 
\citet{li2021pathal}  &  Prostate  & CurriculumNet + O2U-Net  &  ISIC 2017; PANDA Dataset & 60\% input: QWK: 0.895\rule[-3.5ex]{0pt}{0pt}\\
\hline
\rule{0pt}{2.5ex} \multirow{8}{*}{\rotatebox{90}{Segmentation}}&
%Yang \textit{et al.} 
\citet{yang2017suggestive} & Gland; Lymph  & Cosine Similarity + Bootstrapping + FCN& GlaS 2015; Private Dataset: 80 US images & MICCAI 2015: 50\% input: F1: 0.921; Private Dataset: 50\% input: F1: 0.871\rule[-1.5ex]{0pt}{0pt}\\ 
\cline{2-6}
\rule{0pt}{2.5ex}&    
%Konyushkova \textit{et al.}
\citet{konyushkova2019geometry}& Brain (Striatum; Hippocampus)  &    Geometric Priors + Boosted Trees  &  BraTS 2012; EFPL EM Dataset & MRI Data: 60\% input: DSC$\approx$0.76;  EM Data: 40\% input: DSC$\approx$0.60 \rule[-1.5ex]{0pt}{0pt} \\
\cline{2-6}
\rule{0pt}{2.5ex} &    
%Nath \textit{et al.} 
\citet{nath2020diminishing} &  Brain  & Entropy + SVGD Optimization  &  MSD 2018 Dataset & 22.69\% Hippocampus MRI input: DSC: 0.7241 \rule[-1.5ex]{0pt}{0pt}\\
\cline{2-6}
\rule{0pt}{2.5ex}& 
%Ozdemir \textit{et al.} 
\citet{ozdemir2021active}  & Shoulder & BNN +  MMD Divergence & Private Dataset: 36 Volume of MRIs & 48\% MRI input: DSC$\approx$0.85 \rule[-2ex]{0pt}{0pt}\\
\cline{2-6}
\rule{0pt}{2.5ex}& 
%Zhao \textit{et al.} 
\citet{zhao2021dsal} &  Hand; Skin  & U-Net  & RSNA-Bone; ISIC 2017 & 9 AL Iteration: DSC: 0.834 \rule[-1.5ex]{0pt}{0pt}\\
\hline
\rule{0pt}{2.5ex} \multirow{5}{*}{\rotatebox{90}{Others}} & \multirow{2}{*}{\citet{mahapatra2018efficient}$_{\text{CS}}$} %Mahapatra \textit{et al.}
&  Chest  & Bayesian Neural Network + & JSRT Database;& Classification: 35\% input: AUC: 0.953;\\ 
& && cGAN Data Augmentation & ChestX-ray8 & Segmentation: 35\% input: DSC: 0.910\rule[-1.5ex]{0pt}{0pt}\\
\cline{2-6}
\rule{0pt}{2.5ex}& \multirow{2}{*}{
%Zhou \textit{et al.}
\citet{zhou2021active}$_{\text{CD}}$} & 
Colon & Traditional Data Augmentation& Private Dataset: 6 colonoscopy videos & Classification: 4\% input: AUC: 0.9204;\\
& &&Entropy + Diversity & 38 polyp videos + 121 CTPA datasets & Detection: 2.04\% input: AUC: 0.9615\rule[-1.5ex]{0pt}{0pt}\\
\bottomrule
\end{tabular}
\begin{tablenotes}    
        \footnotesize               
        \item[1] For the sake of brevity, we denote references that contain more than one task in the following abbreviations: \textbf{C}: Classification, \textbf{S}: Segmentation, \textbf{D}: Detection. 
      \end{tablenotes}
\end{threeparttable}
}
\label{tab:al}
\end{table*}
\subsection{Data Uncertainty-Based Methods}
Developed from the conventional entropy uncertainty metrics\footnote{To aid the understanding of these metrics, a detailed description of the prior knowledge is provided in Appendix A.2.}, 
%Konyushkova \textit{et al.} 
\citet{konyushkova2019geometry} defined geometric smoothness priors with boosted trees to classify the formed graph representation of electron microscopy images. Here, they flatten 3D images into supervoxels with the SLIC algorithm \citep{achanta2012slic} to conduct graph representations. 
%Yang \textit{et al.} 
\citet{yang2017suggestive} use cosine similarity and a bootstrapping technique to evaluate the uncertainty and representativeness of the output feature with a DCAN \citep{chen2016dcan}-like network.
%Zhou \textit{et al.} 
\citet{zhou2021active} propose the concept of ``active selection" policies, which is the highest confidence based on the entropy and diversity results from sampled data in the mean prediction results. 

Aside from leveraging conventional metrics, utilizing metrics from the deep learning model is another trend. Intuitively, 
%Wu \textit{et al.} 
\citet{wu2021covid} utilize network loss as well as the diversity condition as the uncertainty metric for sampling from a loss prediction network, and conduct the COVID-19 classification task from another classification network. 
%Nath \textit{et al.} 
\citet{nath2020diminishing} leverage marginal probabilities between the query images and the labeled ones, they build a mutual information metric as the diversity metric to serve as a regularizer. Moreover, they adopt Dice log-likelihood instead of its original entropy-based log-likelihood for Stein variational gradient descent optimizer \citep{liu2017stein} to solve the label imbalance problem. 
%Zhao \textit{et al.} 
\citet{zhao2021dsal} utilize Dice's coefficient of the predicted mask calculated between the middle layer and the final layer of the model as the uncertainty metric for the image segmentation task. They use their DS-UNet with a denseCRF \citep{krahenbuhl2011efficient} refiner to annotate low uncertainty samples and oracle annotators for the others. 
%Li \textit{et al.} 
\citet{li2021pathal} use k-means clustering and curriculum classification (CC) based on the CurriculumNet \citep{guo2018curriculumnet} for uncertainty and representativeness estimation. Furthermore, they consider the condition under which noisy medical labels are present and accomplish their automatic exclusion using O2U-Net \citep{huang2019o2u}.  
\subsection{Model Uncertainty-Based Methods}
Bayesian neural networks have attracted increasing attention for their ability to represent and propagate the probability of the DL model. 
%Gal \textit{et al.} 
\citet{gal2017deep} employ Bayesian CNNs for skin cancer classification with Bayesian active learning by disagreement (BALD) \citep{houlsby2011bayesian}.
%Ozdemir \textit{et al.} 
\citet{ozdemir2021active} form a Bayesian network and employ Monte Carlo dropout \citep{gal2016dropout} to obtain the variance information as the model uncertainty. They also construct a representativeness metric produced by infoVAE \citep{zhao2017infovae} for maximum likelihood sampling in the latent space. 
%Mahapatra \textit{et al.} 
\citet{mahapatra2018efficient} also uses a Bayesian neural network to sample the training data. Meanwhile, they use conditional GAN to generate realistic medical images for data augmentation.
\subsection{Discussion}
Whether from the data or from the model, uncertainty measurement is a critical task throughout the whole AL process. The current research directions regarding label-efficient AL methods in MIA focus primarily on the improvement of AL query strategies and the optimization of training methods. For the future, researchers could i) delve into hybrid AL query strategies together with diversity assessment, ii) concentrate on hybrid training schemes (\textit{i.e.}, combined Semi-SL, Self-SL schemes) to yield an intermediate feature representation to further guide the training process, iii) mitigate the degradation of annotation quality when encountering noisy labels.
\section{Few-Shot Learning in MIA} \label{sec:fsl}

\begin{table*}[ht]
\centering
\caption{Overview of Few-Shot Learning-based studies in Medical Image Analysis}
\resizebox{\textwidth}{.22\textwidth}{
\begin{threeparttable}
\begin{tabular}{@{}llllll@{}}
\toprule
&Reference (Year) & Organ & Prior Knowledge Source & Dataset & Result \\ \midrule
\multirow{12}{*}{\rotatebox{90}{Classification}}& 
%Medela \textit{et al.} 
\citet{medela2019few} & Colon; Breast; & Siamese Network & \citep{kather2016multi}; Private Dataset & 15-shot Balanced Acc$\approx$93\%\\
&& Lung &&\rule[-1ex]{0pt}{0pt}\\
\cline{2-6}
% \rule{0pt}{2.5ex}&Quellec \textit{et al.} \cite{quellec2020automatic}$_{2020}$  &  Retina   &   CNN + Probabilistic Model &  OPHDIAT Dataset \cite{decenciere2013teleophta}\rule[-1ex]{0pt}{0pt}\\
% \cline{2-6}
% \rule{0pt}{2.5ex}&Mahajan \textit{et al.} \cite{mahajan2020meta}$_{2020}$ & Skin & Reptile; Prototype Network & ISIC 2018, Derm7pt Dataset \cite{Kawahara2018-7pt}; \\
% &&&& SD-198 Dataset \cite{sun2016benchmark}\\
%\cline{2-6}
\rule{0pt}{2.5ex}&
%Chao \textit{et al.} 
\citet{chao2021generalizing} & Liver; Kidney & MAML & TGCA Dataset & 8-shot AUC: 0.6944\\
&& Colon; Breast && \rule[-1ex]{0pt}{0pt}\\
\cline{2-6}
\rule{0pt}{2.5ex}&
%Singh \textit{et al.} 
\citet{singh2021metamed} & Breast; Skin & Reptile & BreakHis dataset \citep{spanhol2015dataset}; ISIC 2018 & BreakHis (with CutMix): 10-shot Acc: 0.8612; \\
&&&&& ISIC 2018 (with MixUp): 10-shot Acc: 0.8425 \rule[-1ex]{0pt}{0pt}\\
\cline{2-6}
\rule{0pt}{2.5ex}&
%Deusche \textit{et al.} 
\citet{deuschel2021multi} & Colon & Prototype Network & Private Dataset: 356 Annotated WSIs & 20-shot (with data augmentation):\\
&&&&&  Acc: 0.489; F1 Score: 0.728\rule[-1ex]{0pt}{0pt}\\
\cline{2-6}
\rule{0pt}{2.5ex}&
%Tang \textit{et al.} 
\citet{tang2021recurrent} & Brain &  Prototype Network & ABD-110 Dataset \citep{tang2021spatial}; ABD-30 Dataset \citep{landman2015miccai} & ADB-110: One-shot Dice Score: 0.8191\\ %Segmentation
&&&& ABD-MR Dataset \citep{kavur2021chaos} & ABD-30 One-shot: Dice Score: 0.7248\\
&&&&& ABD-MR One-shot: Dice Score: 0.7926\rule[-1ex]{0pt}{0pt}\\
\hline
\multirow{15}{*}{\rotatebox{90}{Segmentation}} & 
%Cui \textit{et al.} 
\citet{cui2020unified}     &   Brain; Liver   &  Prototype Network  &  MRBrainS18;  & MRBrainS18: Three-shot Dice Score: 0.8198  \\ %Segmentation
&&&&BTCV Abdomen Dataset \citep{gibson2018multi} & BTCV: One-shot Dice Score: 0.6913 \rule[-1ex]{0pt}{0pt}\\
\cline{2-6}
\rule{0pt}{2.5ex}&
%Roy \textit{et al.} 
\citet{roy2020squeeze}& Liver; Spleen; & U-Net + & VISCERAL Dataset \citep{jimenez2016cloud}& One-shot: Dice Score: 0.485; ASD: 10.48\\
&& Kidney; Psoas & Channel Squeeze-and-Excitation Blocks & &\rule[-1ex]{0pt}{0pt}\\
\cline{2-6}
\rule{0pt}{2.5ex}&
%Khandelwal \textit{et al.} 
\citet{khandelwal2020domain}  & Spine; Vertebrae & Meta-Learning Domain Generalization & MICCAI 2014; xVertSeg Dataset \citep{korez2015framework}; & 5-shot Dice Score: 0.8052\\
&&&& Versatile Dataset \citep{sekuboyina2021verse} \rule[-1ex]{0pt}{0pt}\\
\cline{2-6}
\rule{0pt}{2.5ex}&
%Wang \textit{et al.} 
\citet{wang2021alternative}         &  Spleen; Kidney; Liver;   &  Siamese Network  & CANDI Dataset \citep{kennedy2012candishare}; Multi-organ Dataset & 4-shot: Dice Score: 0.890; Jaccard: 0.804\\ %Segmentation
&& Stomach; Pancreas; && \citep{gibson2018automatic, roth2015deeporgan, clark2013cancer,xu2016evaluation}\\
&&  Duodenum; Esophagus &&  \rule[-1ex]{0pt}{0pt}\\
\cline{2-6}
\rule{0pt}{2.5ex}&
%Yu \textit{et al.} 
\citet{yu2021location}$_{}$         &  Liver; Spleen;   &    Prototype Network  &  VISCERAL Dataset \citep{jimenez2016cloud}& One-Shot Dice Score: 0.703\\
&& Kidney; Psoas && \rule[-1ex]{0pt}{0pt}\\ %   Segmentation
\cline{2-6}
\rule{0pt}{2.5ex}&
%Guo \textit{et al.}
\citet{guo2021multi} & Heart & Multi-level Semantic Adaptation &  MICCAI 2018; Private Dataset: 3,000 & 5-shot: Dice Score: 0.9564; IoU: 0.9185\\
&&&& CT Images and 13,500 Echo Images  \rule[-1ex]{0pt}{0pt}\\
\hline
\multirow{-1}{*}{\rotatebox{90}{Others}}&\multirow{2}{*}{\citet{he2021few}$_{\text{R}}$}   & Heart; Vertebrae & Perception CNN &  MM-WHS 2017 \citep{zhuang2016multi};& 10-shot: Dice Score: 0.867; ASD: 0.41 \rule[-1ex]{0pt}{0pt}\\
&&&&LBPA40 Dataset \citep{gibson158whyntie}\\
% &&&& REFUGE Dataset \cite{orlando2020refuge}; RIM-One Dataset \cite{fumero2011rim}\\
% Lu \textit{et al.} \cite{lu2022transfer}   &  Segmentation  &  Brain   &   Transfer Learning &  HCP Dataset \cite{van2013wu}; Private Dataset    \\
% Mai \textit{et al.} \cite{mai2021few} & Classification & Retina & Transfer Learning & Cell Dataset \cite{kermany2018labeled}; BOE Dataset \cite{srinivasan2014fully}\\
% Paul \textit{et al.} \cite{paul2021discriminative}          &    Classification  &  Chest   &  Ensenble Learning &  NIH Dataset \cite{wang2017chestx}\\ 
\bottomrule
\end{tabular}
\begin{tablenotes}    
        \footnotesize               
        \item[1] For the sake of brevity, we denote references in Others class in the following abbreviations: \textbf{R}: Registration. 
      \end{tablenotes}
\end{threeparttable}
}
\end{table*}

\subsection{Metric-Based Methods}
Due to the high-dimensionality of input images, it is natural to design feature extractor specifically targets at the sparse data to obtain better embedding for the inputs.
\citet{roy2020squeeze} utilized their channel squeeze \& spatial excitation (sSE) Blocks \citep{roy2018concurrent} to import the feature extracted by U-Net-like architecture from the support set. In addition, they offer an effective technique for volumetric segmentation by optimally matching a small number of support volume slices with all query volume slices. Multi-scale information is another source for extracting feature for FSL. \citet{guo2021multi} proposed their multi-level semantic adaption (MSA) mechanism that can self-adaptively handle sequence-level, frame-level and pixel-level features, thus the MSA can process the hierarchical attention metric. Particularly, they utilize LSTMs to consider the temporal correlation among each frame of the sequence data. 

%denoted as a function $g: \mathbb{R}^{m} \rightarrow \mathbb{R}^{n}, m>n$. 
%Consequently, the idea behind metric-based FSL methods is to either learn the embedding function $g$ under the conventional distance function $d$ or to learn both the embedding function $g$ and the distance function $d$ parameterized by neural networks. 

%The key to designing a metric-based method is to distinguish the difference between the query sample and the support sample. 
Exploiting existing metric-learning architectures targeting general FSL tasks is another common approach in this subfield. %As Fig. \ref{fig_prototype} illustrates, 
prototype networks \citep{snell2017prototypical}  computes prototype representations for each base class by averaging the feature vectors. They then measure the distances between these prototype representations and each query image.
\citet{deuschel2021multi} broadened the prototypes into a latent space and designed a COREL loss to discriminate the prototypes and features. \citet{tang2021recurrent} utilized a recurrent mask fashion to progressively learn the correlations between mask and features, and incorporated PANet \citep{wang2019panet}, an extended version of the Prototype Networks for few-shot medical image segmentation. \citet{cui2020unified} discovers a multi-modal mixed prototype for each category and makes dense predictions based on cosine distances between the deep embeddings of the pixels and the category prototypes. %Their multi-modal representations make excellent use of inter-subject similarities and intra-class variances to prevent overfitting caused by very little data. In the segmentation experiments on brain MRI and abdominal CT datasets, the proposed framework outperforms standard 3D U-Net \cite{cciccek20163d} and classical registration-based ANTs \cite{avants2011reproducible} methods for few-shot segmentation. 
\citet{yu2021location} present a location-sensitive local prototype network that exploits spatial priors to perform few-shot medical image segmentation. The method reduces the challenging problem of segmenting the entire image into easily solvable sub-problems of segmenting particular regions using local grid information. %For organ segmentation studies on the VISCERAL CT image dataset, their method utilize 85 CT scans and outperforms the state-of-the-art techniques by a mean Dice coefficient of 10\%.  

%\input{Figures/fig_siamese}
Other than prototype networks, siamese networks \citep{koch2015siamese} 
%are illustrated in Fig. \ref{fig_siamese} 
are another type of prior feature extraction method for FSL. In this architecture, two identical subnetworks, that share the same architecture and weights. Each subnetwork takes one input from a pair and independently processes it. The outputs of the subnetworks are then combined, usually through a similarity metric, to determine the similarity or dissimilarity of the input pair. \citet{medela2019few} incorporates a siamese network with a pre-trained VGG16 \citep{simonyan2014very} backbone with triplet loss to classify tumor types. \citet{wang2021alternative} exploit anatomical similarities to actively learn dense correspondences between the support and query images. The core principles are inspired by the traditional practice of multi-atlas segmentation, in which registration, label propagation, and label fusion, are combined into a single framework in their work. %The provided two baselines, named Siamese-Baseline and the Individual Difference-Aware Baseline are targeted at different morphological structure, where the former is aimed for anatomically stable structures while the latter is for organs with huge morphological variances. %In conclusion, their study establishes a baseline for few-shot 3D medical picture segmentation and its performance are close to the supervised U-Net upper bounds.


\subsection{Meta-Based Methods}

% In the framework of meta-learning, the methods can be distinguished from the formulation of the meta-learner in two sub-categories: Metric-Based Strategies and Gradient-Based Strategies.

% \subsubsection{Gradient-Based Strategics}
%Meta-learning methods typically employ a meta-training approach, where they train a model on a sequence of few-shot tasks derived from the base classes during the training phase. The objective is to equip the pre-trained model with the capability to quickly adapt to entirely new tasks during the testing phase.
%Training on limited data from scratch using vanilla gradient backward propagation to optimize the model would lead an unsatisfactory result due to the limited generalization capability of the model. Aside from obtaining prior knowledge from the samples, targeting at optimization process is another thinking path to improve the performance. Gradient-Based Methods tend to develop the gradient descend approach in order to achieve better generalization performance. %It is worth to point out that the gradient-based strategics do not depend on the architecture of the neural network, as they are targeting at the update process and may apply to any framework.

Most of the advancements in meta-based algorithms are focused on gradient update policy. \citet{khandelwal2020domain} adapted meta-learning domain generalization (MLDG) \citep{li2018learning} method to minimize the loss on the meta-training domains. Based on the 3D-UNet architecture, their hand-crafted gradient update policy aims to integrate the knowledge from the meta-training phase into the meta-testing stage and is stated as: $\theta \longleftarrow \theta-\gamma \frac{\partial\left(F(\hat{S} ; \theta)+\beta G\left(\bar{S} ; \theta-\alpha \nabla_{\theta}\right)\right)}{\partial \theta}$, where $\hat{S}$ denotes the meta-training data, $\overline{S}$ denotes the meta-testing data, $F, G$ denote two losses from meta-training phase as well as meta-testing phase respectively, $\theta$ denotes the network parameters, and $\alpha, \beta, \gamma$ denote the learning rate-like hyper parameters. The modification brings accurate and generalized few-shot segmentation outcomes in three datasets by up to 10\% Dice score improvement compared to the human oracle. 

%\input{Figures/fig_maml}
\citet{chao2021generalizing} integrated the well-known model-agnostic meta-learning (MAML) \citep{finn2017model} into the classification task, which can fast adapt over insufficient samples. The framework iteratively samples a large number of meta-training tasks from the support set to obtain a strong enough generalization ability, so that when faced with a new task, it can be fitted quickly. %As Fig. \ref{fig_maml} denotes, 
The framework generally contains two loops: One is the outer loop that updates the parameters of the whole framework using the gradient information from the inner loop. In the inner loop, it samples the tasks from the support set, tests them in the query set, and updates the parameters individually. The parameter update policy is as: $\theta \leftarrow \theta-\eta \nabla_{\theta} \sum_{\mathcal{Z}_{i} \sim p(\mathcal{Z})}\left(\mathcal{L}_{\mathcal{Z}_{i}}\left(f_{\theta^{\prime}}\right)\right)$,
where $f$ denotes the ResNet18 \citep{he2016deep} backbone used in the work, $\mathcal{Z}_i$ denotes the sampled batch, $p(\mathcal{Z}_i)$ denotes the cancer type distribution of the sampled batch, $\theta^\prime$ denotes the parameters updated in the sample batch, $\mathcal{L}$ denotes the loss function, and $\eta$ denotes the hyper-parameter. The loop of sampling performs meta-learning using the cumulative test error of the backbone model and obtains the parameters as output. %They examine the performance by conducting a few-shot classification problem using a train set size of 8 slides and achieve a state-of-the-art performance.

\citet{singh2021metamed} integrated another popular method in meta-learning field, named Reptile \citep{nichol2018reptile}, which is a similar framework structure as the MAML. 
%\input{Figures/fig_reptile}
%As Fig. \ref{fig_reptile} denotes, 
The only difference is that it iterates several times in the inner loop and takes the final gradient message back to the outer loop. The parameter update rule goes in: $\theta \leftarrow \theta^{\prime} +\epsilon \frac{1}{m} \sum_{k=1}^{m}\left(f_{\theta_{k}}-f_{\theta^{\prime}}\right)$, 
where $\theta^{\prime}$ denotes the model parameters in the outer loop, $\theta_k$ denotes the parameters of the $k$-th sample in the inner sampling loop, $m$ denotes the number of sampled data, and $\epsilon$ denotes the hyper-parameter. The use of the Reptile algorithm reduces the computational cost due to the gradient calculations are fewer than the MAML method. To ensure the performance of the method, the authors combined random augmentation strategies to enhance the generalization capability of the model.

Moreover, methods mixing the above-mentioned types are rising recently, \citet{mahajan2020meta} separately take metric-learning and gradient-learning type methods for their MetaDermDiagnosis Network to solve the skin lesion classification problem. In addition, they implement Group Equivalent convolutions (G-convolutions) \citep{cohen2016group} to improve disease identification performance, as these images typically lack any prevailing global orientation/canonical structure, and G-convolutions make the network equivalent to discrete transformations.

\subsection{Discussions}
While FSL advancements centered around metric-based and meta-based methods and shown promising results. However,  metric-based models often struggle with highly heterogeneous medical datasets, while meta-based approaches can be computationally intensive and complex to implement \citep{song2023comprehensive}. As we look to the future, the integration of other learning schemes could address these shortcomings. This includes, but is not limited to semi-supervised learning \citep{mai2021few}, transfer learning \citep{lu2022transfer}, and ensemble learning \citep{paul2021discriminative}, to name a few \citep{he2021few}. Such a holistic approach promises the development of models that are not only more robust but also exhibit greater versatility.

%Attention Mechanism is favored since its lower parameter cost, parallel interference capability and better performance. Experiment on three datasets demonstrated that their method has outperformed the compared state-of-the-art methods as the Dice, IoU and HD scores are the best with the usage of as low as 0.6\% of the total dataset.

% Quellec \textit{et al.}  \cite{quellec2020automatic} designed a probabilistic model powered by PCA and t-SNE \cite{van2008visualizing} observation for determining classification probability. In detail, it is based on the discovery that CNNs often view images with comparable anomalies as similar, despite the fact that these CNNs were taught to identify unrelated circumstances. For experiments, it compared with other SOTA FSL learning methods and proved its performance superiority.

%Besides in search of mathematical expressions, 
%Besides segmentation tasks, \citet{he2021few} proposed a two-stage few-shot medical image registration network. Firstly, the inputs are encoded by the proposed perception and correspondence networks, and send to the second stage as past knowledge. In the second phase, their Reverse Teaching method for aligning labeled and unlabeled pictures provides adequate supervisory information to the structure and style knowledge for unlabeled images, hence generating more training data. 
%Compared to the state-of-the-art methods, they achieve 12.5\%, 6.3\%, and 1\% improvements on three datasets with less than 60 unregistered images in total, suggesting their method has great clinical application potential.

%Transfer learning has been another approach that going viral in FSL sphere. Mai \textit{et al.} \cite{mai2021few} formulate few-shot retinal diseases recognition problems as a Student-Teacher Transfer Learning task with a discriminative feature space and Knowledge Distillation (KD) from the auxiliary dataset. Lu \textit{et al.} \cite{lu2022transfer} proposed a Transfer Learning-based method for Few-Shot novel White Matter (WM) tract segmentation. It transfers the knowledge acquired for segmenting existing WM tracts to the segmentation of novel WM tracts via a fine-tuning technique, in which a CNN pretrained for segmenting existing WM tracts is fine-tuned without losing the information of the final task-specific layer, as conventional fine-tuning process would neglect it completely. Moreover, they derive a superior initialization for the final task-specific layer of the target model that segments fresh WM tracts from the weights of the pretrained task-specific layer. At last, they offer a data augmentation technique that generates synthetic annotated images with tract-aware image mixing. %Using the HCP dataset and a private dataset, they demonstrate their method on 22 novel WM tracts in total with significantly better performance than the baselines.

%Paul \textit{et al.} \cite{paul2021discriminative} takes the ensemble learning approach for FSL scenario. In the first stage, a CNN-based coarse-learner is used to learn the general properties of chest X-rays. In the second phase, they incorporate a self-designed, saliency-based discriminative autoencoder-based classifier to extract disease-specific salient characteristics from the coarse-learner output and classify based on them. Besides, part of the method is subjected to meta-training and meta-testing, in which they teach the coarse-learner during the meta-training phase. During the meta-testing phase, however, they exclusively train the saliency-based classifier. Experiments reveal a 19\% improvement in F1-score relative to the baseline in the diagnosis of chest X-rays.

%Lastly, Zhao \textit{et al.} \cite{zhao2019data} pushed FSL to an extreme condition, that is, one-shot learning. They proposed a learned data augmentation method to conduct a one-shot segmentation task, i.e., the training data only has one sample. This method has utilized U-Net-based models as general network architecture to learn spatial and appearance transform from the atlas (image with segmentation label) separately. For spatial alignment, they leverage the learning scheme of the VoxelMorph \cite{balakrishnan2019voxelmorph} model, which learns to generate a smooth displacement vector field that registers one image to another by concurrently optimizing their normalized cross-correlation loss and a smoothness term for the displacement field. For appearance alignment, they design a similarity loss with a smooth regularizer. Then synthesize the augmented training image and the corresponding segmentation maps. The experiments on eight public datasets show that their method has outperformed the state-of-the-art one-shot learning methods by increased Dice score.
\section{Annotation-Efficient Learning in MIA}\label{sec:anno}
% ---------------------------------------------------------------
\subsection{Tag Annotation}\label{sec:anno_tag}
% ---------------------------------------------------------------
Tag annotation, which is a text/binary label for each image, is the most efficient form.
Most of such are based on the concept of class activation mapping (CAM)~\cite{zhou2016learning}. 
Several works propose to use of CAM to generate object localization proposals or even to perform whole-object pixel-wise segmentation.
For the \textbf{detection} task, Wang \textit{et al.}~\cite{hwang2016self} propose a two-branch network that jointly optimizes the classification and lesion detection tasks. In this approach, the CAM-based lesion detection network is supervised with only image-level annotations, and
the two branches are mutually guided by the weight-sharing technique, where a weighting parameter is adopted to control the focus of learning from the classification task to the detection task.
For lesion detection, Dubost \textit{et al.}~\cite{dubost2020weakly} propose a weakly-supervised regression network.
The proposed method is validated on both 2D and 3D medical images.
For the \textbf{segmentation} task, Li \textit{et al.}~\cite{li2022deep} propose a breast tumor segmentation method with only image-level annotations based on CAM and deep-level set (CAM-DLS).
It integrates domain-specific anatomical information from breast ultrasound to reduce the search space for breast tumor segmentation.
Meanwhile, Chen \textit{et al.}~\cite{chen2022c} proposes a causal CAM method for organ segmentation, which is based on the idea of causal inference with a category-causality chain and an anatomy-causality chain.
In addition, several studies~\cite{lin2019seg4reg,lin2021seg4reg+} demonstrate that bridging the classification task and dense prediction task (e.g., detection and segmentation) via CAM-based methods is beneficial for both tasks.
Compared to natural images, medical images are usually from low contrast, limited texture, and varying acquisition protocols~\cite{zhang2021weakly}, which makes directly applying CAM-based methods less effective.
Fortunately, incorporating the clinical priors (e.g., objects' size~\cite{fruh2021weakly}) into the weakly supervised detection task is promising to improve the performance.
% ---------------------------------------------------------------
\subsection{Point Annotation}\label{sec:anno_point}
% ---------------------------------------------------------------
Point annotation refers to the annotation of a single point of an object.
Several studies~\cite{roth2021going,dorent2021inter,khan2019extreme} focus on using extreme points as the annotation to perform pixel-level segmentation.
These methods typically consist of three steps: 1) extreme point selection; 2) initial segmentation with a random walk algorithm; 3) training of the segmentation model with the initial segmentation results. 
The last two steps can be iterated until the segmentation results are stable.
However, these methods require the annotators to locate the boundary of the objects, which is still laborious in practice.
In contrast, other studies~\cite{yoo2019pseudoedgenet,zhao2020weakly,qu2020weakly,qu2019weakly,tian2020weakly,belharbi2021deep,lin2022label,valvano2021learning} use center point annotation to perform pixel-level segmentation for the task of cell/nuclear segmentation. 
These methods typically adopt the Vorinor~\cite{kise1998segmentation} and cluster algorithms to perform coarse segmentation. 
Then different methods are used to refine the segmentation results, such as iterative optimization~\cite{qu2019weakly,qu2020weakly}, self-training~\cite{zhao2020weakly}, and co-training~\cite{lin2022label}. 

Compared with full annotation, point annotation can reduce the annotation time by around 80\%~\cite{qu2020weakly}.
However, some issues have not been addressed. 
First, existing methods typically derived pseudo labels from the point annotation, which are based on strong constraints or assumptions (e.g., Voronoi) from the data, restricting the generalization of these methods to other datasets~\cite{lin2022label}.
\begin{table*}[!ht]
\centering
\caption{Overview of Annotation-Efficient Learning Studies in Medical Image Analysis}
\resizebox{\textwidth}{.23\textwidth}{
\begin{tabular}{@{}lllllll@{}}
\toprule
& Reference (Year) & Application & Organ & Method & Dataset & Results \\
\midrule
\rule{0pt}{2ex} \multirow{15}{*}{\rotatebox{90}{Tag}} \rule{0pt}{2ex}&\citet{hwang2016self}                                  & Detection                   & Lung; Breast                                  & CAM + Self-Transfer Learning                                & Private Dataset: 11K X-rays;                             & AP Shenzhen set: 0.872;  \rule[-1ex]{0pt}{0pt}\\
\rule{0pt}{2ex} &&&&&  DDSM; MIAS  & MC set: 0.892; MIAS set: 0.326\rule[-1ex]{0pt}{0pt}\\
\cline{2-7}
% --------------------------------------------------------------
\rule{0pt}{2ex} &\citet{gondal2017weakly}                              & Detection                   & Eye                                           & CAM                                                        & DRD; DiaretDB1   & Hemorrhages SE: 0.91; FP s/I 1.5; Hard Exudates SE: 0.87; FPs/I 1.5; \\
&&&&&& Soft Exudates SE: 0.89; FPs/I: 1.5; RSD SE: 0.52; FPs/I: 1.5 \rule[-1ex]{0pt}{0pt}\\
\cline{2-7}
% --------------------------------------------------------------
\rule{0pt}{2ex} &\citet{wang2018weakly}                                  & Detection                   & Eye                                           & Expectation-Maximization + & DRD; Messidor                                            & mAP Kaggle: 0.8394; Messidor: 0.9091  \rule[-1ex]{0pt}{0pt}\\
\rule{0pt}{2ex} & &  & & Low-Rank Subspace Learning                          & & \rule[-1ex]{0pt}{0pt}\\
\cline{2-7}
% --------------------------------------------------------------
&\rule{0pt}{2ex}\citet{nguyen2019novel}                               & Segmentation                & Eye                                           & CAM + CRF + Active Shape Model                               & Private Dataset: 40 MRI Images                                              & DSC: T1w: 0.845±0.056; T2w: 0.839±0.049\rule[-1ex]{0pt}{0pt}\\
\cline{2-7}
% --------------------------------------------------------------
\rule{0pt}{2ex} &\citet{wang2020weakly}                                  & Detection                   & Lung                                          & CAM + Unsupervised Segmentation                             & Private Dataset: 540 CT Images                                          & Hit Rate: 0.865\rule[-1ex]{0pt}{0pt}\\
\cline{2-7}
% --------------------------------------------------------------
\rule{0pt}{2ex} &\citet{shen2021interpretable}                           & Detection                   & Breast                                        & Globally-aware Multiple Instance                 & NYUBCS; CBIS-DDSM                                           & DSC malignant: 0.325 ± 0.231; DSC Benign: 0.240 ± 0.175; \rule[-1ex]{0pt}{0pt}\\
\rule{0pt}{2ex}&&&&& Classifier  & AP malignant: 0.396 ± 0.275; AP Benign: 0.283 ± 0.24 \rule[-1ex]{0pt}{0pt}\\
\cline{2-7}
% --------------------------------------------------------------
\rule{0pt}{2ex} &\citet{chen2022c}                             & Segmentation                & prostate; Cardiac;              & Causal Inference; CAM & ACDC; ProMRI; CHAOS  & ProMRI DSC: 0.864$\pm$0.004; ASD: 3.86$\pm$1.20; MSD: 3.85$\pm$1.33\rule[-1ex]{0pt}{0pt}\\
&&& Abdominal Organ &&&   ACDC DSC: 0.875$\pm$0.008; ASD: 1.62$\pm$0.41; MSD: 1.17$\pm$0.24\rule[-1ex]{0pt}{0pt}\\
&&  &&&  & CHAOS DSC: 0.781\rule[-1ex]{0pt}{0pt}\\
\cline{2-7}
% --------------------------------------------------------------
\rule{0pt}{2ex} &\citet{liu_tssk-net_2023} & detection & Eye & contrastive learning; knowledge distillation & Private: 7,000 OCT & AUC: 98.05; Dice: 50.95 \\
\midrule
% --------------------------------------------------------------
% Point
% --------------------------------------------------------------
\rule{0pt}{2ex} \multirow{5}{*}{\rotatebox{90}{Point}} &\citet{khan2019extreme}                                 & Segmentation                & Multi-organ                                 & Confidence Map Supervision            & SegTHOR                                                     & DSC Aorta: 0.9441 $\pm$ 0.0187; Esophagus 0.8983 $\pm$ 0.0416; \rule[-1ex]{0pt}{0pt}\\
\cline{2-7}
% --------------------------------------------------------------
\rule{0pt}{2ex} &\citet{zhao2020weakly}                                  & Segmentation                & Cell                                        & Self-/Co-/Hybrid-Training                                  & PHC; Phase100                                               & DSC PHC: 0.871; Phase 100: 0.811  \rule[-1ex]{0pt}{0pt}\\
\cline{2-7}
% --------------------------------------------------------------
\rule{0pt}{2ex}  &\citet{dorent2021inter}                               & Segmentation                & Brain                                       & CNN + CRF                                                   & Vestibular-Schwannoma-SEG                                   & DSC: 0.819$\pm$0.080; HD95: 3.7$\pm$7.4; P: 0.929$\pm$0.059\rule[-1ex]{0pt}{0pt}\\
\cline{2-7}
% --------------------------------------------------------------
\rule{0pt}{2ex} &\citet{guo2023sac} & Segmentation & Multi-organ & Superpixel; Confident learning & MoNuSeg & Dice: 79.42; IoU: 65.15 \\
\cline{2-7}
% --------------------------------------------------------------
\rule{0pt}{2ex} &\citet{xia_weakly_2023} & Segmentation & Multi-organ & Multi-task & MoNuSeg & Dice: 75.39; AJI: 58.19 \\
\midrule
% --------------------------------------------------------------
% Scribble
% --------------------------------------------------------------
\rule{0pt}{2ex} \multirow{3}{*}{\rotatebox{90}{Scribble}} &\citet{wang2018interactive} & Segmentation & Body & Image-Specific Fine-Tuning & Private Dataset: 18 MRI Images; BRATS & Private DSC: 0.8937$\pm$0.0231; BRATS DSC: 0.8811$\pm$0.0609 \rule[-1ex]{0pt}{0pt}\\
\cline{2-7}
% --------------------------------------------------------------
\rule{0pt}{2ex} &\citet{lee2020scribble2label}                            & Segmentation                & Cell                                     & Exponential Moving Average                                 & MoNuSeg                                                     & DSC: 0.6408; mIoU: 0.5811 \rule[-1ex]{0pt}{0pt}\\
\cline{2-7}
% --------------------------------------------------------------
\rule{0pt}{2ex} &\citet{zhang2022cyclemix}                              & Segmentation                & Heart                                    & Mixup + Consistency                 & ACDC; MSCMRseg                                              & ACDC DSC: 0.848; MSCMRseg DSC: 0.800 \rule[-1ex]{0pt}{0pt}\\
\midrule
% --------------------------------------------------------------
% Box
% --------------------------------------------------------------
\rule{0pt}{2ex} \multirow{5}{*}{\rotatebox{90}{Box}} &\citet{rajchl2016deepcut}                             & Segmentation                & Brain; Lung                                   & DenseCRF                                                   & Private Dataset: 55 MRI Images                                              & Brain DSC: 0.941$\pm$0.041; Lung DSC: 0.829$\pm$0.100\rule[-1ex]{0pt}{0pt}\\
\cline{2-7}
% --------------------------------------------------------------
\rule{0pt}{2ex} &\citet{wang2022recistsup}                                  & Segmentation                & lymph; Lung; Skin            &  RECIST
measurement propagation algorithm;                                         & TCIA;   & TCIA ASSD: 0.866; HD95: 3.263; DSC: 0.785  \rule[-1ex]{0pt}{0pt}\\
\rule{0pt}{2ex} &&  & & RECIST Loss;  RECIST3D Loss                                         &  LIDC–IDRI;                                & TCIA ASSD: 0.990; HD95: 3.628; DSC: 0.753\rule[-1ex]{0pt}{0pt}\\
\rule{0pt}{2ex} && & & &  HAM10000;                           & HAM10000 ASSD: 0.314; HD95: 1.299; DSC: 0.832\rule[-1ex]{0pt}{0pt}\\
\cline{2-7}
% --------------------------------------------------------------
\rule{0pt}{2ex} &\citet{zhu_feddm_2023} & Segmentation & Prostate & Annotation calibration; Gradient de-conflicting & PROMISE12 & Dice: 81.01; IoU: 68.77 \\
% --------------------------------------------------------------
\bottomrule\label{tab:anno}
\end{tabular}}
\end{table*}

Second, due to the lack of explicit boundary supervision, there is a non-negligible performance gap between the weakly supervised methods with points and the fully supervised methods.

% ---------------------------------------------------------------
\subsection{Scribble Annotation}\label{sec:anno_scribble}
% ---------------------------------------------------------------
Scribble annotation, a set of scribbles drawn on an image by the annotators, has been recognized as a user-friendly alternative to bounding box annotation~\cite{tajbakhsh2020embracing}. 
Compared with point annotation, it provides the rough shape and size information of the objects, which is promising to improve the segmentation performance, especially for objects with complex shapes.
Wang \textit{et al.}~\cite{wang2018interactive} propose a self-training framework with differences in model predictions and user-provided scribbles. 
Can \textit{et al.}~\cite{can2018learning} develop a random walk algorithm that incrementally performs region growing method around the scribble ground truth, while
Lee \textit{et al.}~\cite{lee2019scribble2label} introduce Scribble2Label, a method that integrates the supervision signals from both scribble annotations and pseudo labels with the exponential moving average. 
Furthermore, Dorent \textit{et al.}~\cite{dorent2020scribble} extend the Scribble-Pixel method to the domain adaptation scenario, where a new formulation of domain adaptation is proposed based on CRF and co-segmentation with the scribble annotation. 
In recent work, Zhang \textit{et al.}~\cite{zhang2022cyclemix} adopted mix augmentation and cycle consistency for the Scribble-Pixel method, demonstrating the improvement of both weakly and fully supervised segmentation methods.

% ---------------------------------------------------------------
\subsection{Box Annotation}\label{sec:anno_box}
% ---------------------------------------------------------------
Box annotation encloses the segmented region within a rectangle, and various recent studies have focused on this Box-Pixel scenario. 
Rajchl \textit{et al.}~\cite{rajchl2016deepcut} employs a densely-connected random field (DCRF) with an iterative optimization method for MRI segmentation. 
Wang \textit{et al.}~\cite{wang2021accurate,wang2021bounding} adopt MIL and smooth maximum approximation based on the bounding box tightness prior~\cite{hsu2019weakly}, that is, an object instance should touch all four sides of its bounding box. 
Thus, a vertical or horizontal crossing line within a box yields a positive bag because it covers at least one foreground pixel. 
Studies~\cite{wang2021bounding} demonstrate that the Box-Pixel method yields promising performance, being only 1--2\% inferior to the fully supervised methods.

\subsection{Discussion}
Points are most suitable for objects with uniform shapes and sizes, particularly when there is a large number of objects present. These points indicate the location of the objects.
Scribbles, on the other hand, are used to label different semantic elements by marking them and are best suited for objects with uniform shapes but varied sizes.
Boxes can provide an approximation of the shape and size information of objects, making them ideal for tasks such as segmentation or detection where objects have high variations in their shape and size.
Out of all these annotation types, image tagging is the most efficient, requiring the least amount of annotation cost.
Several studies have aimed to reduce the performance gap between different annotation-efficient methods based on various annotations.
Future work could explore the following topics: i) integrating multiple supervision signals into a unified learning framework, such as multi-task learning and omni-supervised learning; ii) actively reducing the annotation cost through human-in-the-loop techniques, such as active learning; and iii) mining inherent knowledge from multi-modality data.







\section{Challenges and Future Directions} \label{sec:cnfd}
Our comprehensive discussion of label-efficient learning schemes in MIA raises several challenges that should be taken into account to improve the performance of the DL model. In this section, we describe the crucial challenges and shed light on potential future directions for solving these challenges.

\subsection{Omni-Supervised Learning} 
Although the methods we have presented have achieved promising performance, many of them are targeted at addressing \textit{ad hoc} label shortage problems, \textit{i.e.}, these methods do not utilize as much supervision as possible.
Served as a special regime of Semi-SL, \textbf{Omni-supervised learning} is a crucial trend for label-efficient learning in MIA for the simultaneous utilization of different forms of supervision. Studies \cite{luo2021oxnet, chai2022orf} have demonstrated the feasibility of omni-supervised learning under teacher-student\cite{tarvainen2017mean} and the dynamic label assignment \cite{chai2022orf} pipeline, respectively. In the teacher-student training approach, the model trained on fully annotated datasets serves as the teacher model, and features extracted from the weakly-/un-annotated datasets serve as guidance to refine the model. 
Through designated mechanisms, the student model utilizes the teacher model with the provided guidance to further improve performance.  Meanwhile, the dynamic label assignment approach forms the crafted metric from different types of labels in the training process and dynamically gives the final predicted labels. 

During the process of omni-supervised learning, however, centralizing or releasing different supervision health data raises multiple ethical, legal, regulatory, and technological issues \cite{rieke2020future}. On the one hand, collecting and maintaining a high-quality medical dataset consumes a large amount of expense, time, and effort. 
On the other hand, the privacy of patients may be compromised during the centralization or release of health datasets, even with techniques such as anonymization and safe transfer. To address the privacy preservation problem during model development, researchers proposed \textbf{federated learning (FL)} to conduct training in a data-decentralized manner. 
This approach has yielded fruitful results in The field of MIA \cite{dayan2021federated,li2020multi,lu2022federated}. However, current FL algorithms are primarily trained in a supervised manner. When applying the FL to real-world scenarios in MIA, a crucial problem, namely, label deficiency, may appear in local health datasets.
Labels may be missing to varying degrees between medical centers, or the granularity of the labels will vary. A promising research direction is to design label-efficient federated learning methods to address this significant problem. For example, semi-supervised learning\cite{liu2021federated}, active learning, and self-supervised learning \cite{dong2021federated} are suitable to be incorporated into this setting.

\subsection{Human-in-the-loop Interaction}
The application of expert knowledge to refine the output of the model is often carried out in practice, and there have been various efforts to investigate this field, known as human-in-the-loop (HITL). The AL scheme can be considered a part of HITL as it involves the introduction of expert knowledge to refine data supervision. Meanwhile, expert knowledge can also be introduced as action supervision under the \textbf{reinforcement learning (RL)} schemes to improve the performance of the DL model \cite{liao2020iteratively, ma2020boundary}. In RL scheme, a set of “agents” is formulated to learn expert behaviors in an interactive environment via trial and error. In MIA tasks, RL methods mainly treat the interactive refinement process as the Markov decision process (MDP) and give the solution by the RL process. RL-based interventional model training brings the potential for dealing with rare cases in MIA, since the expert-provided interactions can refine the prediction result at the final stage to hinge samples that failed to process by the DL model.
In addition, recent developments in diverse learning methodologies, including but not limited to few-shot learning \cite{feng2021interactive, al2021ifss} and interpretability-guided learning \cite{mahapatra2021interpretability}, have contributed to improved efficacy of human-in-the-loop workflows, thereby reducing labor costs in MIA. This indicates a positive trend towards increased cost-effectiveness in this field.

\subsection{Generation-based Data Augmentation}
Data augmentation with synthesized images produced by generative-based methods is regarded as a way to unlock additional information from the dataset, and leads the way in computation speed and quality of results in the scope of generative methods~\cite{shorten2019survey}.
In the field of MIA, numerous studies~\cite{wang2020semi,lin2022insmix} have investigated data augmentation with the original GAN~\cite{goodfellow2014generative}  and its variations. 
 However, the unique adversarial training procedure of GANs may suffer from training instability~\cite{gulrajani2017improved} and mode collapse~\cite{lin2018pacgan}, yielding ``Copy GAN", which only generates a limited set of samples~\cite{yang2019bi}.
Thus, synthesizing augmented data with a high visual realism and diversity is the key challenge of GAN.
Meanwhile, the \textbf{probabilistic diffusion model} \cite{ho2020denoising}, has recently sparked much interest in MIA applications\cite{kazerouni2022diffusion}. 
This model establishes a forward diffusion stage in which the input data is gradually disrupted by adding Gaussian noise over multiple stages and then learns to reverse the diffusion process to obtain the required noise-free data from noisy data samples. 
Despite their recognized computational overhead \cite{xiao2022tackling}, diffusion models are generally praised for their high mode coverage and sample quality, and various of efforts have been made to ease the computational cost and further improve their generalization capability.

\subsection{Generalization Across Domains and Datasets}
From semi-supervised learning to annotation-efficient learning, we have introduced a considerable number of methods that address the problem of the low-quantity and/or -quality of labels. Nevertheless, recent results reveal that these novel methods may encounter significant performance degradation when shifting to different domains or datasets. The generalization problem in MIA field arises due to multiple causes, such as variance among scanner manufacturers, scanning parameters, and subject cohorts. And various current deep learning algorithms cannot be robustly deployed in various real scenarios. To address this practical problem, the concept of \textbf{domain generalization} has been introduced, of which the key idea is to learn a trained model that encapsulates general knowledge so as to adapt to unseen domains and new datasets with little effort and cost. A plethora of methods have been developed to tackle the domain generalization problem \cite{zhou2022domain}, such as domain alignment \cite{li2018domain}, meta-learning \cite{li2019feature}, data augmentation \cite{qiao2020learning} and so on. MIA has also seen some publications with respect to domain generalization \cite{li2020domain,liu2021feddg}. Further, another challenge for generalization across domains and datasets is that the proposed methods may require numerous labeled multi-source data to extract domain-invariant features. For example, Yuan \textit{et al.} \cite{yuan2022label} have made a successful attempt to achieve model generalization in source domains with limited annotations by leveraging active learning and semi-supervised domain generalization, eliminating the dilemma between domain generalization and expensive annotations.

\subsection{Benchmark Establishment and Comparison}
Label-efficient learning in MIA spans multiple tasks, such as classification, segmentation, and detection, as well as multiple organs, such as the retina, lung, and kidney. Differences and variances in tasks and target organs lead to confounding experiment settings and unfair performance comparisons. Meanwhile, a lack of sufficient public health datasets also contributes to this dilemma. For example, many researchers can only conduct experiments to measure the performance of their proposed algorithms based on their own private datasets due to reasons such as privacy. However, few publications have emerged \cite{gut2022benchmarking} to address the problem, especially for label-efficient learning. Thus, benchmarkng remains a pressing problem for model evaluation. On the one hand, the public should urge for the availability of large datasets. On the other hand, a clearly defined set of benchmarking tasks and the corresponding evaluation procedures should be established. Further, specific experimental details should be stipulated to facilitate the comparability of different label-efficient learning algorithms.

\section{Conclusion}\label{sec:con}
Despite significant advances in computer-aided MIA, the question of how to endow deep learning models with enormous data remains a daunting challenge. Deep learning models under label-efficient schemes have shown significant flexibility and superiority in dealing with high degree of quality- and quantity-variant data. To that end, we have presented the first comprehensive label-efficient learning survey in MIA. A variety of learning schemes, including semi-supervised, self-supervised, multi-instance, active and annotation-efficient learning in the general field are classified and analyze thoroughly. We hope that by systematically sorting out the methodologies for each learning schemes, this survey will shed light on more progress in the future.

\section*{Acknowledgments}
This work was supported by National Natural Science Foundation of China (No. 62202403), Hong Kong Innovation and Technology Fund (No. PRP/034/22FX), and the Project of Hetao Shenzhen-Hong Kong Science and Technology Innovation Cooperation Zone (HZQB-KCZYB-2020083). 

\appendix
\section{Concepts of Prior Knowledge}
\label{appendix1}
\subsection{Assumptions and Detail in Semi-supervised Learning}
\subsubsection{Assumptions in Semi-supervised Learning}
Semi-SL is not universally effective, as stated in \citep{van2020survey,xiaojin2008semi}, a necessary condition for Semi-SL algorithms to work is that the marginal data distribution $p(x)$ contains underlying information about the posterior distribution $p(y|x)$, where $x$ and $y$ represent the data sample over input space $\mathcal{X}$ and the associated label, respectively. Otherwise the additional unlabeled data will be useless to infer information about $p(y|x)$, which means the Semi-SL algorithms may achieve similar or even worse performance compared with supervised learning algorithms. Therefore, several assumptions over the input data distribution have been proposed to constrain the data structure and ensure the algorithms can be generalized from a limited labeled dataset to a large-scale unlabeled dataset. Following \citep{van2020survey, ouali2020overview}, the assumptions in Semi-SL are introduced as follows:

\textbf{Smoothness assumption}. Suppose $x_1,~x_2 \in X$ are two input data samples over input space $\mathcal{X}$. If the distance between $x_1$ and $x_2$ is very close, \textit{i.e.}, $D(x_1, x_2) < \varepsilon$, where $\varepsilon$ is an artificially set threshold, then the associated labels $y_1$ and $y_2$ should also be the same. Note that sometimes there is an additional constraint in the smoothness assumption. In \citep{ouali2020overview}, $x_1$ and $x_2$ are required to belong to the same high-density region, so as to avoid the situation that these two samples reside on the brink of different high-density regions and are misclassified as one category.

\textbf{Cluster assumption}. In this assumption, we assume that data points with similar underlying information are likely to form high-density regions, i.e., clusters. If the two data points $x_1$ and $x_2$ lie in the same cluster, then they are expected to have the same label. In fact, the cluster assumption can be considered as a special case of the smoothness assumption. According to \citep{chapelle2009semi}, if the two data points $x_1$ and $x_2$ can be connected with a line that does not pass through any low-density area, they belong to the same cluster. 

\textbf{Low-density assumption}. The decision boundary of the classifier is assumed to lie in the low-density areas instead of high-density ones, which can be derived from the cluster assumption and smoothness assumption. On the one hand, if the decision boundary resides in the high-density regions, the two data points $x_1$ and $x_2$ located in the same cluster but opposite sided of the decision boundary will be categorized as different classes, which obviously violates the cluster assumption and smoothness assumption. On the other hand, following the cluster and smoothness assumption, data points in any high-density areas are expected to be assigned the same label, which means the decision boundary of the model can only lie in the low-density areas, thus satisfying the low-density assumption.

\textbf{Manifold assumption}. A manifold is a concept in geometry, that represents a geometric structure in a high-dimensional space, i.e., a collection of data points in the input space $\mathcal{X}$. For example, a curve in 2-dimensional space can be thought of as a 1-dimensional manifold, and a surface in 3-dimensional space can be seen as a 2-dimensional manifold. The manifold assumption states that there is a certain geometry of the data distribution in the high-dimensional space, namely that the data are concentrated around a certain low-dimensional manifold. Due to the fact that high-dimensional data not only poses a challenge to machine learning algorithms, but also leads to a large computational load and the problem of dimensional catastrophe, it will be much more effective to estimate the data distribution if they lie in a low-dimensional manifold. 

\subsubsection{Detail of Key Generative Methods}
Researchers can obtain various generative methods according to different assumptions on the latent distribution. On the one hand, it can be easy to formulate a generative method once an assumption on the distribution is made, whereas on the other, the hypothetical generative model must match the real data distribution to avoid the unlabeled data in turn degrading the generalization performance. One can formulate the modeling process of generative methods as follows:

\begin{equation}\label{generative}
    \begin{aligned}
    y^{*}&=\arg \max _{y} p(y | x)=\arg \max _{y} \frac{p(x | y) p(y)}{p(x)}\\
    &=\arg \max _{y} p(x | y) p(y),\\
    \end{aligned}
\end{equation}
where the generative methods models the joint distribution $p(x,y)$. Eq. (\ref{generative}) indicates that if the correct assumption on prior $p(y)$ and conditional distribution $p(x | y)$ is made, the input data can be expected to come from the latent distribution. 

\textbf{Definition of Generative Adversarial Network (GAN).}
The aim of generator $\mathcal{G}$ is to iteratively learn the latent distribution from real data $x$  starting from generating data with random noise distribution $p(z)$. Meanwhile, the goal of discriminator $\mathcal{D}$ is to correctly distinguish the fake input generated by $\mathcal{G}$ and real data $x$. Formally, we can formulate the optimization problem of a GAN as follows:

\begin{equation*}
    \begin{aligned}
    \min_{\mathcal{G}}\max_{\mathcal{D}} \mathcal{L}(\mathcal{G},\mathcal{D}) &= \mathbb{E}_{x\sim p(x)}[\mathrm{log}\mathcal{D}(x)]\\
    &+\mathbb{E}_{z\sim p(z)}[1-\mathrm{log}(\mathcal{D}(\mathcal{G}(z)))],
    \end{aligned}
\end{equation*}
where $\mathcal{L}$ represents the loss function of generator $\mathcal{G}$ and discriminator $\mathcal{D}$. Concretely, $\mathcal{G}$ aims to minimize the objective function by confusing $\mathcal{D}$ with generated data $\mathcal{G}(z)$, while $\mathcal{D}$ aims at maximizing the objective function by making correct predictions.

\textbf{Definition of Variational Autoencoder (VAE).}
The typical VAE consists of two objectives: one is to minimize the discrepancy between input data $x$ and its reconstruction version $\hat{x}$ produced by the decoder, and the other is to model a latent space $p(z)$ following a simple distribution, such as a standard multivariate Gaussian distribution. Thus, the loss function for training a VAE can be formulated as follows:
\begin{equation*}
    \begin{aligned}
    \min_{\theta}\sum_{x\in X} \mathcal{L}(x, \theta) = \mathcal{L}_{MSE}(x,\hat{x}_{\theta})+\mathcal{L}_{KL}(p_{\theta}(z|x)||p(z)),
    \end{aligned}
\end{equation*}
where $\mathcal{L}_{MSE}$ represents the mean square error; $\hat{x}_{\theta}$ is the reconstruction version of input data $x$ generated by the decoder $p_{\phi}(x|z)$ given parameters $\phi$; $\mathcal{L}_{KL}(\cdot||\cdot)$ represents the Kullback-Leibler divergence which measures the distance between two distributions; and $p_{\theta}(z|x)$ denotes the posterior distribution produced by the encoder given parameters $\theta$.


\subsection{Conventional Uncertainty Metrics in Active Learning}
The uncertainty measure reflects the degree of dispersion of a random input. There are many ways to measure the uncertainty of inputs. Starting with simple metrics like standard deviation and variance, current studies in MIA mainly focus on \textbf{margin sampling} \citep{campbell2000query} and \textbf{entropy sampling} \citep{holub2008entropy}. Denote the probability as $p$, we give the definition of these metrics as follows.

\textbf{Margin sampling} \citep{campbell2000query} estimates the probability difference $\mathcal{M}$ between the first and second most likely labels $\hat{y}_{1}, \hat{y}_{2}$ according to the deep model parameter $\theta$ and expect the least residual value by the following notation:

\begin{equation*}
    \mathcal{M}=\underset{x}{\operatorname{argmin}}   [p_{\theta}\left(\hat{y}_{1} \mid x\right)-p_{\theta}\left(\hat{y}_{2} \mid x\right)]
\end{equation*}

\textbf{Entropy sampling} \citep{holub2008entropy} is another conventional metric for sampling. In a binary or multi-classification scenario, the sampled data with higher entropy can be selected as the expected annotation data. For a $C$-class task, entropy sampling metric $\mathcal{E}$ can be denoted as follows:
\begin{equation*}
    \mathcal{E}=\underset{x}{\operatorname{argmax}}\left(-\sum_{c=1}^{C}p(y_c\mid x)\log p(y_c\mid x)\right)
\end{equation*}

\subsection{Detail of Key Few-shot Methods}
\textbf{Definition of Metric-Learning.} The objective of a metric-based strategy is to measure the distance across limited data samples and attempt to generalize the measurement on more data. Consider two embedded image-label pairs $(f(x_1), y_1)$ and $(f(x_2), y_2)$ and a distance function $d$ parameterized by neural networks to determine the separation between them. For any image $x_3$ to be predicted, this strategy calculates the two distances between the embeddings $d(f(x_1), f(x_3))$ and $d(f(x_2), f(x_3))$ and assigns the class label to the output with fewer values.

\textbf{Definition of Meta-Learning.} Meta-learning, or learning to learn \citep{thrun1998learning} technique divides the learning scheme together with the samples into the meta-training phase and the meta-testing phase. In each phase, it samples the divided data, forming several meta-tasks. Here we denote the corresponding tasks as $\tau_{meta\_train}=\{\tau_0, \tau_1, \ldots, \tau_n\}$ and $\tau_{meta\_test}=\{\tau_{n+1}, \tau_{n+2}, \ldots, \tau_{n+k}\}$. In the meta-training phase, the meta-learner $\mathcal{F}$ would learn the common representation from the support set. Then, the learned parameters are fed into a regular learner $f$ to test the performance of the query set from the same meta-task and optimize the model parameters $\theta_\mathcal{F}, \theta_f$ through the designated loss $\mathcal{L}$. After meta-training, the trained model would have the ability to deal with the assigned problem with very few samples. Denote the support set and the corresponding query set pair as $D_i=(S_i, Q_i)\in\tau_{i}$, the whole process can be depicted in the following notation:
\begin{equation*}
\begin{aligned}
    \theta = \arg\min_\theta \sum_{D_{train}\in\tau_{meta\_train}}\sum_{D_{test}\in\tau_{meta\_test}}\\ \mathcal{L}(f(S_{test}, \mathcal{F}(D_{train};\theta_\mathcal{F})), Q_{test}; \theta_f)
\end{aligned}
\end{equation*}

\renewcommand{\thetable}{B\arabic{table}}
\setcounter{table}{0} % Reset table counter in the appendix
\section{Datasets}
As a supplement to the main text, we summarize representative publicly available datasets across 16 different organs such as the brain, chest, prostate, \textit{etc.} in Tab. \ref{tab:Dataset1}. 
These publicly available MIA datasets can be leveraged to construct label-efficient learning algorithms for numerous purposes, including classification, detection, and segmentation.

\begin{table*}[h]
\centering
\begin{center}
\small
\caption*{Appendix Table 6. Summary of publicly available databases for label-efficient learning in MIA.}
%\resizebox{\textwidth}{.64\textwidth}{
\begin{tabular}{@{}c|p{4cm}lp{0.39\textwidth}@{}}
% \begin{tabular}{@{}p{0.5cm}|p{3cm} p{2cm} p{9cm}@{}}
\toprule
\multicolumn{1}{c}{Organ}               & Dataset (Year)      & Task         & Link  \\ \hline
\multirow{32}{*}{Brain} & \rule{0pt}{2.5ex}BraTS (2012) & Segmentation & \url{http://www.imm.dtu.dk/projects/BRATS2012/data.html}   \\ 
\cline{2-4}
                       & \rule{0pt}{2.5ex}BraTS (2013)~\cite{menze2014multimodal}  & Segmentation & \url{https://www.smir.ch/BRATS/Start2013#!#download}   \rule[-1ex]{0pt}{0pt}\\
\cline{2-4}
                       & \rule{0pt}{2.5ex}BraTS (2015) & Segmentation & \url{https://www.smir.ch/BRATS/Start2015}   \rule[-1ex]{0pt}{0pt}\\
\cline{2-4}
                       & \rule{0pt}{2.5ex}BraTS (2017) & Segmentation & \url{https://sites.google.com/site/braintumorsegmentation/}   \rule[-1ex]{0pt}{0pt}\\
\cline{2-4}
                       & \rule{0pt}{2.5ex}BraTS (2018) & Segmentation & \url{https://wiki.cancerimagingarchive.net/pages/viewpage.action?pageId=37224922}   \rule[-1ex]{0pt}{0pt}\\
\cline{2-4}
                       & \rule{0pt}{2.5ex}MSD (2018)\cite{simpson2019large} &Segmentation &\url{https://drive.google.com/drive/folders/1HqEgzS8BV2c7xYNrZdEAnrHk7osJJ--2}   \rule[-1ex]{0pt}{0pt}\\
\cline{2-4}
                       & \rule{0pt}{2.5ex}dHCP (2018) \cite{makropoulos2018developing}            &Segmentation     &\url{http://www.developingconnectome.org/data-release/}       \rule[-1ex]{0pt}{0pt}\\ 
\cline{2-4}
                       &   \rule{0pt}{2.5ex}JSRT Database (2000)\cite{shiraishi2000development}            & Classification             &\url{http://db.jsrt.or.jp/eng.php}       \rule[-1ex]{0pt}{0pt}\\ 
% \cline{2-4}
%                        &\rule{0pt}{2.5ex}EFPL EM (2013) \cite{lucchi2013learning}              &Segmentation              &\url{https://www.epfl.ch/labs/cvlab/data/data-em/}    \rule[-1ex]{0pt}{0pt}   \\
\cline{2-4}
                       &\rule{0pt}{2.5ex}MRBrainS18 (2018)              &Segmentation              &\url{https://mrbrains18.isi.uu.nl/data/}       \rule[-1ex]{0pt}{0pt}\\
\cline{2-4}
                       &\rule{0pt}{2.5ex}BigBrain (2013)~\cite{amunts2013bigbrain}              &Segmentation              &\url{https://bigbrainproject.org/maps-and-models.html#download}       \rule[-1ex]{0pt}{0pt}\\
\cline{2-4}
                       &\rule{0pt}{2.5ex}MALC (2012)              &Segmentation              &\url{http://www.neuromorphometrics.com/2012_MICCAI_Challenge_Data.html}       \rule[-1ex]{0pt}{0pt}\\
% \cline{2-4}
%                        &\rule{0pt}{2.5ex}Vestibular-Schwannoma-SEG (2021)~\cite{shapey2021segmentation}              &Segmentation              &\url{https://wiki.cancerimagingarchive.net/pages/viewpage.action?pageId=70229053}       \rule[-1ex]{0pt}{0pt}\\
\cline{2-4}
                       &\rule{0pt}{2.5ex}TCIA (2015)~\cite{seff2015leveraging}              &Segmentation              &\url{https://www.cancerimagingarchive.net/}       \rule[-1ex]{0pt}{0pt}\\
\cline{2-4}
                       &\rule{0pt}{2.5ex}OASIS (2007)              &Segmentation              &\url{https://www.oasis-brains.org/#data}       \rule[-1ex]{0pt}{0pt}\\
\cline{2-4}
                       &\rule{0pt}{2.5ex}UKBB (2016)              &Classification              &\url{https://www.ukbiobank.ac.uk/}       \rule[-1ex]{0pt}{0pt}\\
\cline{2-4}
                       &\rule{0pt}{2.5ex}ADNI (2010)              &Classification              &\url{https://www.adni-info.org/}       \rule[-1ex]{0pt}{0pt}\\
\cline{2-4}
                       &\rule{0pt}{2.5ex}ABIDE (2016)              &Classification              &\url{https://fcon_1000.projects.nitrc.org/indi/abide/}       \rule[-1ex]{0pt}{0pt}\\
% \cline{2-4}
%                        &\rule{0pt}{2.5ex}FTD              &Classification              &\url{https://cind.ucsf.edu/research/grants/frontotemporal-lobar-}      \rule[-1ex]{0pt}{0pt}\\
%                        &&& \url{degenerationneuroimaging-initiative-0}\\
% \cline{2-4}
%                        &\rule{0pt}{2.5ex}Brain Tumor MRI (2022)              &Segmentation              &\url{https://www.kaggle.com/datasets/masoudnickparvar/brain-tumor-mri-dataset}       \rule[-1ex]{0pt}{0pt}\\
\cline{2-4}
                       &\rule{0pt}{2.5ex}MIRIAD (2012)              &Classification              & \url{https://www.ucl.ac.uk/drc/research/research-methods/minimal-interval-resonance-imaging-alzheimers-disease-miriad}      \rule[-1ex]{0pt}{0pt}\\
                       
\hline
\multirow{13}{*}{Chest}  &\rule{0pt}{2.5ex}IS-COVID (2020)~\cite{fan2020inf}          &Segmentation              &\url{http://medicalsegmentation.com/covid19/}       \rule[-1ex]{0pt}{0pt}\\ 
\cline{2-4}
                       &\rule{0pt}{2.5ex}CC-COVID (2020)~\cite{zhang2020clinically}              &Segmentation              &\url{https://ncov-ai.big.ac.cn/download?lang=en}       \rule[-1ex]{0pt}{0pt}\\ 
\cline{2-4}
                       &\rule{0pt}{2.5ex}NLST (2009)              &Detection              &\url{https://cdas.cancer.gov/datasets/nlst/}       \rule[-1ex]{0pt}{0pt}\\ 
\cline{2-4}
                       &\rule{0pt}{2.5ex}NIH Chest X-ray (2017)~\cite{wang2017chestx}              &Classification              &\url{https://www.kaggle.com/datasets/nih-chest-xrays/data}       \rule[-1ex]{0pt}{0pt}\\
\cline{2-4}
                       &\rule{0pt}{2.5ex}TCGA-Lung              &Classification              &\url{https://portal.gdc.cancer.gov/repository}       \rule[-1ex]{0pt}{0pt}\\
% \cline{2-4}
%                        &\rule{0pt}{2.5ex}DLCST (2007)              &Classification              &\url{https://clinicaltrials.gov/ct2/show/NCT00496977}       \rule[-1ex]{0pt}{0pt}\\
\cline{2-4}
                       &\rule{0pt}{2.5ex}LDCTGC (2016)              &Detection              &\url{https://www.aapm.org/grandchallenge/lowdosect/}       \rule[-1ex]{0pt}{0pt}\\
\cline{2-4}
                       &\rule{0pt}{2.5ex}ChestX (2018)~\cite{kermany2018identifying}              &Classification              &\url{https://data.mendeley.com/datasets/rscbjbr9sj/3}       \rule[-1ex]{0pt}{0pt}\\
\cline{2-4}
                       &\rule{0pt}{2.5ex}LUNA (2016)              &Detection              &\url{https://luna16.grand-challenge.org/}       \rule[-1ex]{0pt}{0pt}\\
\hline
\end{tabular}%}
\label{tab:Dataset1}
% \end{tabular}
\end{center}
\end{table*}

\begin{table*}[h]
\small
\centering
\begin{center}
\caption*{Appendix Table 6. Summary of publicly available databases for label-efficient learning in MIA. (continued)}
\begin{tabular}{@{}c|p{4cm}lp{0.4\textwidth}@{}}
\toprule
\multicolumn{1}{c}{Organ}               & Dataset (Year)      & Task         & Link  \\ \hline
\multirow{23}{*}{Chest (Continued)} 
                &\rule{0pt}{2.5ex}CAD-PE (2019)              &Segmentation              &\url{https://ieee-dataport.org/open-access/cad-pe}       \rule[-1ex]{0pt}{0pt}\\
\cline{2-4}
                       &\rule{0pt}{2.5ex}SIIM-ACR (2019)              &Segmentation              &\url{https://www.kaggle.com/c/siim-acr-pneumothorax-segmentation}       \rule[-1ex]{0pt}{0pt}\\
\cline{2-4}
                       &\rule{0pt}{2.5ex}RSNA-Lung (2018)              &Detection              &\url{https://www.kaggle.com/c/rsna-pneumonia-detection-challenge}       \rule[-1ex]{0pt}{0pt}\\
\cline{2-4}
                       &\rule{0pt}{2.5ex}VinDr-CXR (2021)              &Detection              &\url{https://vindr.ai/datasets/cxr}       \rule[-1ex]{0pt}{0pt}\\
\cline{2-4}
                       &\rule{0pt}{2.5ex}Montgomery (2022)              &Segmentation              &\url{https://www.kaggle.com/datasets/raddar/tuberculosis-chest-xrays-montgomery}       \rule[-1ex]{0pt}{0pt}\\
% \cline{2-4}
%                        &\rule{0pt}{2.5ex}RICORD (2021)~\cite{tsai2021rsna}              &Segmentation              &\url{https://wiki.cancerimagingarchive.net/pages/viewpage.action?pageId=80969742}       \rule[-1ex]{0pt}{0pt}\\
\cline{2-4}
                       &\rule{0pt}{2.5ex}ChestXR (2021)              &Classification              &\url{https://cxr-covid19.grand-challenge.org/Dataset/}       \rule[-1ex]{0pt}{0pt}\\
\cline{2-4}
                       &\rule{0pt}{2.5ex}MIMIC-CXR (2019)~\cite{johnson2019mimic}             &Detection              &\url{https://physionet.org/content/mimic-cxr/2.0.0/}       \rule[-1ex]{0pt}{0pt}\\
\cline{2-4}
                       &\rule{0pt}{2.5ex}CC-CCII (2020)\cite{zhang2020clinically} &  Classification & \url{http://ncov-ai.big.ac.cn/download/}       \rule[-1ex]{0pt}{0pt}\\
\cline{2-4}
                       &\rule{0pt}{2.5ex}ChestX-ray8 (2017) \cite{wang2017chestx} & Segmentation  & \url{https://nihcc.app.box.com/v/ChestXray-NIHCC/}      \rule[-1ex]{0pt}{0pt}\\
\cline{2-4}
                       &\rule{0pt}{2.5ex}ChestX-ray14 (2019) & Classification & \url{https://stanfordmlgroup.github.io/competitions/chexpert/}       \rule[-1ex]{0pt}{0pt}\\
\cline{2-4}
                       &\rule{0pt}{2.5ex}CheXpert (2019) \cite{wang2017chestx} & Segmentation  & \url{https://stanfordmlgroup.github.io/competitions/chexpert/}       \rule[-1ex]{0pt}{0pt}\\
\hline
\multirow{4}{*}{Gland} & \rule{0pt}{2.5ex}GlaS (2015)\cite{sirinukunwattana2015stochastic} & Segmentation & \url{https://warwick.ac.uk/fac/cross_fac/tia/data/glascontest/download/}   \\ 
\cline{2-4}
                       &\rule{0pt}{2.5ex}CRAG (2017)  &Segmentation   &\url{https://warwick.ac.uk/fac/sci/dcs/research/tia/data/mildnet}   \rule[-1ex]{0pt}{0pt}\\
\hline
\multirow{10}{*}{Prostate} & \rule{0pt}{2.5ex}Prostate-MRI-US-Biopsy (2013)~\cite{sonn2013targeted} & Segmentation & \url{https://wiki.cancerimagingarchive.net/pages/viewpage.action?pageId=68550661}   \rule[-1ex]{0pt}{0pt}\\
\cline{2-4}
                       & \rule{0pt}{2.5ex}PANDA (2020) \cite{bulten2020panda}  &Classification   &\url{https://www.kaggle.com/c/prostate-cancer-grade-assessment/data/}   \rule[-1ex]{0pt}{0pt}\\
\cline{2-4}
                       & \rule{0pt}{2.5ex}ProMRI (2012)~\cite{litjens2014evaluation,tian2015superpixel}  &Segmentation   &\url{https://promise12.grand-challenge.org/}   \rule[-1ex]{0pt}{0pt}\\
\cline{2-4}
                       & \rule{0pt}{2.5ex}TMA-Zurich (2018)~\cite{arvaniti2018automated}  &Classification   &\url{https://www.nature.com/articles/s41598-018-30535-1?source=app#data-availability}   \rule[-1ex]{0pt}{0pt}\\
\cline{2-4}
                       & \rule{0pt}{2.5ex}GGC (2019)  &Classification   &\url{https://gleason2019.grand-challenge.org/Register/}   \rule[-1ex]{0pt}{0pt}\\
\hline
\multirow{7}{*}{Heart} & \rule{0pt}{2.5ex}MSCMRseg~\cite{zhuang2016multivariate} & Segmentation & \url{https://zmiclab.github.io/zxh/0/mscmrseg19/}   \rule[-1ex]{0pt}{0pt}\\
\cline{2-4}
                       & \rule{0pt}{2.5ex}MM-WHS (2017)  &Segmentation   &\url{https://zmiclab.github.io/zxh/0/mmwhs/}   \rule[-1ex]{0pt}{0pt}\\
\cline{2-4}
                       & \rule{0pt}{2.5ex}Endocardium-MRI (2008)~\cite{andreopoulos2008efficient}  &Segmentation   &\url{https://www.sciencedirect.com/science/article/pii/S1361841508000029#aep-e-component-id41}   \rule[-1ex]{0pt}{0pt}\\
\cline{2-4}
                       & \rule{0pt}{2.5ex}M\&Ms (2020)& Segmentation & \url{https://www.ub.edu/mnms/}\rule[-1ex]{0pt}{0pt}\\
\cline{2-4}
                       & \rule{0pt}{2.5ex}ASG (2018)~\cite{xiong2021global}& Segmentation & \url{http://atriaseg2018.cardiacatlas.org/}\rule[-1ex]{0pt}{0pt}\\
\hline
\multirow{2}{*}{Eye} & \rule{0pt}{2.5ex}DRISHTI-GS (2014)~\cite{sivaswamy2014drishti} & Segmentation & \url{https://www.kaggle.com/datasets/lokeshsaipureddi/drishtigs-retina-dataset-for-onh-segmentation}   \rule[-1ex]{0pt}{0pt}\\
% \cline{2-4}
%                        & \rule{0pt}{2.5ex}RIM-ONE (2015)~\cite{fumero2015interactive}& Segmentation & \url{https://www.idiap.ch/software/bob/docs/bob/bob.db.rimoner3/stable/index.html}\rule[-1ex]{0pt}{0pt}\\
\hline
\end{tabular}\label{tab:Dataset2}
\end{center}
\end{table*}

\begin{table*}[htbp]
\centering
\begin{center}
\captcont*{Summary of publicly available databases for label-efficient learning in MIA (continued)}
\begin{tabular}{@{}c|p{3.5cm}lp{0.5\textwidth}@{}}
% \begin{tabular}{@{}p{0.5cm}|p{3cm} p{2cm} p{9cm}@{}}
\toprule
\multicolumn{1}{c}{Domain}               & Dataset (Year)      & Task         & Link  \\ \hline
\multirow{28}{*}{Breast} & \rule{0pt}{2.5ex}BACH (2018) \cite{aresta2019bach} &    Classification & \url{https://iciar2018-challenge.grand-challenge.org/Dataset/}   \\ 
\cline{2-4}
                       &\rule{0pt}{2.5ex}NYUBCS (2019)~\cite{wu2019nyu}& Segmentation & \url{https://datacatalog.med.nyu.edu/dataset/10518}  \rule[-1ex]{0pt}{0pt}\\
\cline{2-4}
                       &\rule{0pt}{2.5ex}CBIS-DDSM (2017)~\cite{lee2017curated}& Segmentation & \url{https://www.kaggle.com/datasets/awsaf49/cbis-ddsm-breast-cancer-image-dataset}  \rule[-1ex]{0pt}{0pt}\\
% \cline{2-4}
%                        &\rule{0pt}{2.5ex}DDSM (1998)~\cite{heath1998current}& Detection & \url{https://www.kaggle.com/datasets/skooch/ddsm-mammography} \rule[-1ex]{0pt}{0pt}\\
\cline{2-4}
                       &\rule{0pt}{2.5ex}MIAS (2015)~\cite{suckling2015mammographic}& Detection & \url{https://www.kaggle.com/datasets/kmader/mias-mammography} \rule[-1ex]{0pt}{0pt}\\
\cline{2-4}
                       &\rule{0pt}{2.5ex}TCGA-Breast & Classification & \url{https://portal.gdc.cancer.gov/repository} \rule[-1ex]{0pt}{0pt}\\
\cline{2-4}
                       &\rule{0pt}{2.5ex}INBreast (2012) & Classification & \url{https://biokeanos.com/source/INBreast} \rule[-1ex]{0pt}{0pt}\\
% \cline{2-4}
%                        &\rule{0pt}{2.5ex}BCSC (2013)~\cite{oster2013development} & Classification & \url{https://www.bcsc-research.org/data}\rule[-1ex]{0pt}{0pt}\\
\cline{2-4}
                       &\rule{0pt}{2.5ex}BreastPathQ (2019)& Classification & \url{https://breastpathq.grand-challenge.org/Overview/} \rule[-1ex]{0pt}{0pt}\\
\cline{2-4}
                       &\rule{0pt}{2.5ex}CAMELYON (2016)& Classification & \url{https://camelyon16.grand-challenge.org/Data/} \rule[-1ex]{0pt}{0pt}\\
\cline{2-4}
                       &\rule{0pt}{2.5ex}CAMELYON (2017)& Classification & \url{https://camelyon17.grand-challenge.org/Data/} \rule[-1ex]{0pt}{0pt}\\
\cline{2-4}
                       &\rule{0pt}{2.5ex}BreakHis (2016)& Classification & \url{https://web.inf.ufpr.br/vri/databases/breast-cancer-} \rule[-1ex]{0pt}{0pt}\\
                       &&& \url{histopathological-database-breakhis/} \\
\cline{2-4}
                       &\rule{0pt}{2.5ex}CBCS3 (2018)~\cite{troester2018racial}& Classification & \url{https://unclineberger.org/cbcs/for-researchers/}  \rule[-1ex]{0pt}{0pt}\\
\cline{2-4}
                       &\rule{0pt}{2.5ex}TNBC (2018)~\cite{naylor2018segmentation}& Segmentation & \url{https://ega-archive.org/datasets/EGAD00001000063} \rule[-1ex]{0pt}{0pt}\\
\cline{2-4}
                       &\rule{0pt}{2.5ex}TUPAC (2016)~\cite{veta2019predicting}& Segmentation & \url{https://github.com/CODAIT/deep-histopath}  \rule[-1ex]{0pt}{0pt}\\
\cline{2-4}
                       &\rule{0pt}{2.5ex}MITOS12~\cite{ludovic2013mitosis}& Segmentation & \url{http://ludo17.free.fr/mitos_2012/dataset.html}  \rule[-1ex]{0pt}{0pt}\\
\cline{2-4}
                       &\rule{0pt}{2.5ex}MITOS14~\cite{icpr2014mitosis}& Segmentation & \url{https://mitos-atypia-14.grand-challenge.org/Dataset/}  \rule[-1ex]{0pt}{0pt}\\
\cline{2-4}
                       &\rule{0pt}{2.5ex}TMA-UCSB (2014)~\cite{kandemir2014empowering}& Classification & \url{https://bioimage.ucsb.edu/research/biosegmentation}  \rule[-1ex]{0pt}{0pt}\\
\hline
\multirow{2}{*}{Cell} & \rule{0pt}{2.5ex}PHC (2013)~\cite{mavska2014benchmark}& Segmentation & \url{http://celltrackingchallenge.net/}   \\ 
% \cline{2-4}
%                        &\rule{0pt}{2.5ex}Phase100~\cite{zhao2018pyramid}& Segmentation & \url{http://celltrackingchallenge.net/2d-datasets/}  \rule[-1ex]{0pt}{0pt}\\
\cline{2-4}
                       &\rule{0pt}{2.5ex}CPM (2017)~\cite{vu2019methods}& Segmentation & \url{http://simbad.u-strasbg.fr/simbad/sim-id?Ident=CPM+17}  \rule[-1ex]{0pt}{0pt}\\
\hline
\multirow{2}{*}{Liver} & \rule{0pt}{2.5ex}LiTS (2017)& Segmentation & \url{https://competitions.codalab.org/competitions/17094}  \\
\cline{2-4}
                       &\rule{0pt}{2.5ex}PAIP (2019) & Segmentation & \url{https://paip2019.grand-challenge.org/Dataset/} \rule[-1ex]{0pt}{0pt}\\
\hline
\multirow{3}{*}{Lymph Node} & \rule{0pt}{2.5ex}PatchCAMELYON (2017) & Classification  & \url{https://patchcamelyon.grand-challenge.org/Download/}  \\
\cline{2-4}
                       &\rule{0pt}{2.5ex}NIH LN (2016) & Classification &  \url{https://wiki.cancerimagingarchive.net/pages/viewpage.action?pageId=19726546} \rule[-1ex]{0pt}{0pt}\\
\hline
\multirow{1}{*}{Pancreas} & \rule{0pt}{2.5ex}NIH PCT& Segmentation & \url{https://wiki.cancerimagingarchive.net/display/Public/Pancreas-CT}  \\
\hline
\multirow{18}{*}{Multi-organ} & \rule{0pt}{2.5ex}DSB (2018)& Segmentation & \url{https://www.kaggle.com/competitions/data-science-bowl-2018/data}  \\
\cline{2-4}
                       &\rule{0pt}{2.5ex}DeepLesion (2018)~\cite{yan2018deeplesion}& Detection & \url{https://nihcc.app.box.com/v/DeepLesion} \rule[-1ex]{0pt}{0pt}\\
\cline{2-4}
                       &\rule{0pt}{2.5ex}WTS (2020)~\cite{keikhosravi2020non}& Super-resolution & \url{https://www.nature.com/articles/s42003-020-01151-5} \rule[-1ex]{0pt}{0pt}\\
                       &&&\url{#data-availability}\rule[-1ex]{0pt}{0pt}\\
\cline{2-4}
                       &\rule{0pt}{2.5ex}DECATHLON (2019)~\cite{simpson2019large}& Segmentation & \url{http://medicaldecathlon.com/} \rule[-1ex]{0pt}{0pt}\\
\cline{2-4}
                       &\rule{0pt}{2.5ex}MoNuSeg (2017)~\cite{kumar2017dataset}& Segmentation & \url{https://monuseg.grand-challenge.org/} \rule[-1ex]{0pt}{0pt}\\
\cline{2-4}
                       &\rule{0pt}{2.5ex}MoCTSeg (2018)~\cite{gibson2018automatic}& Segmentation & \url{https://www.synapse.org/#!Synapse:syn3376386} \rule[-1ex]{0pt}{0pt}\\
\cline{2-4}
                       &\rule{0pt}{2.5ex}BTCV (2017)~\cite{gibson2018multi}& Segmentation & \url{https://zenodo.org/record/1169361#.Y8Ud-OxBwUE} \rule[-1ex]{0pt}{0pt}\\
\cline{2-4}
                       &\rule{0pt}{2.5ex}CT-ORG~\cite{roth2015deeporgan,rister2020ct}& Segmentation & \url{https://wiki.cancerimagingarchive.net/pages/viewpage.action?pageId=61080890}  \rule[-1ex]{0pt}{0pt}\\
\cline{2-4}
                       &\rule{0pt}{2.5ex}NIH PLCO (2011)~\cite{oken2011screening}& Classification & \url{https://cdas.cancer.gov/datasets/plco/} \rule[-1ex]{0pt}{0pt}\\
\cline{2-4}
                       &\rule{0pt}{2.5ex}BCV (2017) & Segmentation & \url{https://www.synapse.org/#!Synapse:syn3193805/files/} \rule[-1ex]{0pt}{0pt}\\
\cline{2-4}
                       &\rule{0pt}{2.5ex}MIDOG~\cite{aubreville2021mitosis}& Segmentation & \url{https://imig.science/midog/the-dataset/} \rule[-1ex]{0pt}{0pt}\\
\hline
\end{tabular}\label{tab:Dataset3}
\end{center}
\end{table*}

\clearpage

% \section*{\itshape Reference style}

% Text: All citations in the text should refer to:
% \begin{enumerate}
% \item Single author: the author's name (without initials, unless there
% is ambiguity) and the year of publication;
% \item Two authors: both authors' names and the year of publication;
% \item Three or more authors: first author's name followed by `et al.'
% and the year of publication.
% \end{enumerate}
% Citations may be made directly (or parenthetically). Groups of
% references should be listed first alphabetically, then chronologically.

%%Harvard
\bibliographystyle{model2-names.bst}\biboptions{authoryear}
\bibliography{refs}

\end{document}

%%
