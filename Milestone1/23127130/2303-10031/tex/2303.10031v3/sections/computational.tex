In this section,  we derive tractable conditions for the verification of dissipativity properties of LPV systems based on data. To this end, we can observe that verifying~\eqref{eq:thm_LPV_diss} can be seen as a robust optimization problem~\cite{ben2009robust}. Hence, we will solve the data-driven analysis problem of Theorem~\ref{thm:LPV_diss} by applying Finsler's Lemma, which translates it into a feasibility check of a scheduling-dependent definiteness condition for all $\bar p_{[1,L]}\in\ms{P}_{[1,L]}$. While this feasibility check involves verifying infinitely many definiteness conditions (one for every trajectory in $\ms{P}_{[1,L]}$), we will show that it can be reduced to a finite-dimensional LMI feasibility problem by taking structural assumptions on $\ms{P}_{[1,L]}$ and extending the analysis results from the model-based setting~\cite{apkarian2000parameterized, HoffmannWerner2014, parrilo2000structured, oliveira2007parameter}.

\subsection{Main concept}

{To derive a computable form of the data-driven analysis problem of Theorem~\ref{thm:LPV_diss},} %
we introduce the \emph{non-strict version} of Finsler's lemma~\cite{meijer2024general}
{that will prove to be an essential building block of our proposed approaches.}
\begin{lemma}[{Non-strict Finsler's Lemma \cite{meijer2024general}}]\label{lem:Finsler}
Let $A\in\mathbb{R}^{n\times n}$, $B\in\mathbb{R}^{q\times n}$.
Then, the following statements: %
\begin{enumerate}[label={(\roman*)}]
\item $x^\top Ax\geq0$ for all $x$ satisfying $Bx=0$,

\item $(B^\perp)^\top AB^\perp\succeq0$,

\item There exists a $\mu\in\mathbb{R}$ such that $A+\mu B^\top B\succeq0$,

\item There is an $X\in\mathbb{R}^{n\times q}$ such that $XB+B^\top X^\top - A \preceq 0$,
\end{enumerate}
satisfy that ``(i) $\Leftrightarrow$ (ii)'', ``(iii) $\Rightarrow$ (ii)'', and ``(iv) $\Rightarrow$ (ii)''.
\end{lemma}
According to Theorem~\ref{thm:LPV_diss}, analyzing dissipativity amounts to verifying Condition~\eqref{eq:thm_LPV_diss} for all {admissible} scheduling signals $\bar{p}_{[1,L]}\in\ms{P}_{[1,L]}$. Based on ``(i) $\Leftrightarrow$ (ii)'' in Lemma~\ref{lem:Finsler}, we infer that~\eqref{eq:thm_LPV_diss} holds if and only if
\begin{align}\label{eq:dd_Ldiss_Finsler}
\big(F(\bar{p}_{[1,L]})^\perp\big)^\top \Pi_H F(\bar{p}_{[1,L]})^\perp\succeq0.
\end{align}
The same idea has been been employed in the LTI case in~\cite{romer2019one}, where this reformulation 
provides a simple and elegant condition to verify dissipativity, only requiring to check positive semi-definiteness of a given data-dependent matrix.
For LPV systems of the form \eqref{eq:sys}, condition~\eqref{eq:dd_Ldiss_Finsler} allows us to check whether the dissipation inequality holds for a \emph{fixed} scheduling trajectory $\bar{p}_{[1,L]}$. However, we need to verify dissipativity as a \emph{system} property for the whole $\ms{P}_{[1,L]}$, which makes, together with the fact that
$F(\bar{p}_{[1,L]})^\perp$ depends nonlinearly on $\bar{p}_{[1,L]}$, the verification of~\eqref{eq:dd_Ldiss_Finsler} intractable.

Therefore, in the remainder of this section, we exploit items~(iii) and~(iv) of Lemma~\ref{lem:Finsler} in order to derive computational procedures for the verification of~\eqref{eq:thm_LPV_diss}, which are tailored to the assumed {definition of $\ms{P}_{[1,L]}$}.
First, in Section~\ref{subsec:computational_polytopic}, we employ item (iv) to address {the case when} $\mathbb{P}$ {is a convex, polytopic set, by describing} $\ms{P}_{[1,L]}$ based on a convex hull argument.
Next, in Section~\ref{subsec:computational_S_procedure}, we use the S-procedure to handle admissible scheduling sets $\ms{P}_{[1,L]}$ defined by quadratic inequalities.

\subsection{Dissipativity verification via a convex hull argument}\label{subsec:computational_polytopic}
\subsubsection{Core concept} We now provide a computational procedure to verify~\eqref{eq:thm_LPV_diss} for any $\bar{p}_{[1,L]}\in\ms{P}_{[1,L]}$, where $\ms{P}_{[1,L]}:=\mb{P}^L$ and $\mathbb{P}$ is a %
convex polytope {with finite many vertices}.


\begin{assumption}[{Polytopic description of $\mb{P}$}]\label{ass:polytopic}
The set $\mathbb{P}\subset\mb{R}^{\dnp}$ is a convex polytope, {generated by a finite {number} of vertices, i.e.,   %
$\mathbb{P}=\mathrm{co}(\{\msf{p}_i\}_{i=1}^{n_\mathrm{v}})$, where $\mathrm{co}$ denotes the convex hull.} \hfill$\square$
\end{assumption}
As the scheduling variable varies in $\mb{P}$, i.e., {$\forall k \in\mathbb{Z}:\bar{p}_k\in\mb{P}$}, every trajectory {$\bar{p}_{[1,L]}$} is confined in the space 
\[\ms{P}_{[1,L]}=\mb{P}\times\dots\times\mb{P}=\mb{P}^L,\]
which is generated with $n_\mathrm{v}^L$ vertices {that} are all permutations of the original vertices of $\mb{P}$, i.e., $\ms{P}_{[1,L]}= \mathrm{co}( \mathrm{perm}_L(\{\msf{p}_i\}_{i=1}^{n_\mathrm{v}}) )$.
We now apply ``(iv) $\Rightarrow$ (i)'' in Lemma~\ref{lem:Finsler} to derive a tailored condition for verifying~\eqref{eq:thm_LPV_diss} via a convex hull argument, i.e., by means of Assumption~\ref{ass:polytopic}. {For this, we will explicitly write $F(\bar{p}_{[1,L]})$ in its affine form, i.e., with $\bar{\vect{p}}_{[1,L]}\in\mb{R}^{L\dnp}$,
\begin{equation}\label{eq:affineF}
    F(\bar{p}_{[1,L]}) = \hat{F}_0 + \sum_{i=1}^{L\dnp} \bar{\vect{p}}_{i}\hat{F}_i. %
\end{equation}
}

\begin{prop}[$L$-dissipativity via convex hull argument]\label{prop:LPV_diss_polytopic}
Suppose Assumption~\ref{ass:polytopic} holds and let $\{\bar{\msf{p}}_\nu\}_{\nu=1}^{n_\mathrm{v}^L}$ denote the vertices of $\mb{P}^L$. 
Then,~\eqref{eq:thm_LPV_diss} holds {if} %
there exist matrices $X_i\in\mathbb{R}^{(N-L+1)\times(N-L+1)}$, {$i \in \{ 0,\dots,L\dnp\}$},
\begin{subequations}\label{eq:prop_LPV_diss_polytopic}
satisfying{
\begin{align}
    \sum_{i=0}^{L\dnp}\sum_{j=0}^{L\dnp} \bar{\vect{p}}_{i}\bar{\vect{p}}_{j}\left(X_i \hat{F}_j + \hat{F}_j^\top X_i^\top\right) -\Pi_H\preceq0, \label{eq:prop_LPV_diss_polytopic:a}\\
    X_i \hat{F}_i + \hat{F}_i^\top X_i^\top \succeq 0, \quad {\forall i \in \{ 0,\dots,L\dnp \}},\label{eq:prop_LPV_diss_polytopic:b}
\end{align}}%
\end{subequations}
{for all vertices {$\{ \bar{\msf{p}}_\nu\}_{\nu=1}^{n_\mathrm{v}^L}$}.}
\end{prop}
\begin{proof}
For any given scheduling trajectory $\bar{p}_{[1,L]}\in\ms{P}_{[1,L]}$, Lemma~\ref{lem:Finsler} implies that~\eqref{eq:thm_LPV_diss} holds if there exists {a} matrix function $X(\bar{p}_{[1,L]}):\ms{P}_{[1,L]} \to \mathbb{R}^{(N-L+1)\times(N-L+1)}$ associated with $\bar{p}_{[1,L]}$ such that
\begin{align}\label{eq:dd_Ldiss_Finsler1}
	X(\bar{p}_{[1,L]})F(\bar{p}_{[1,L]})+F(\bar{p}_{[1,L]})^\top X(\bar{p}_{[1,L]})^\top\!\! -\Pi_H \preceq 0.
\end{align}
{By considering $X$ to have affine dependency on $\bar{p}_{[1,L]}$, i.e., $X(\bar{p}_{[1,L]})=X_0+\sum_{i=1}^{L\dnp} \bar{\vect{p}}_{i}X_i$,  \eqref{eq:dd_Ldiss_Finsler1} under~\eqref{eq:affineF} reads as
\begin{equation}\label{eq:pf:nonconvex}
    \sum_{i=0}^{L\dnp}\sum_{j=0}^{L\dnp} \bar{\vect{p}}_{i}\bar{\vect{p}}_{j}\left(X_i \hat{F}_j + \hat{F}_j^\top X_i^\top\right) -\Pi_H\preceq0, \ \forall \bar{p}_{[1,L]}\in\ms{P}_{[1,L]},
\end{equation}
which {is quadratic, but not necessarily convex in $\bar{p}_{[1,L]}$}}. %
Based on the application of the multi-convexity relaxation from~\cite{apkarian2000parameterized, gahinet1996affine}, condition~\eqref{eq:prop_LPV_diss_polytopic:b} enforces convexity of~\eqref{eq:pf:nonconvex}, and hence if~\eqref{eq:prop_LPV_diss_polytopic:a} and~\eqref{eq:prop_LPV_diss_polytopic:b} hold for all vertices $\{ \bar{\msf{p}}_\nu\}_{\nu=1}^{n_\mathrm{v}^L}$ of $\mathbb{P}^L$, then~\eqref{eq:pf:nonconvex} holds for all $\bar{p}_{[1,L]}\in\ms{P}_{[1,L]}$.
\end{proof}


Proposition~\ref{prop:LPV_diss_polytopic} provides a computational approach to verify condition~\eqref{eq:thm_LPV_diss} in case the scheduling variable is varying in a convex polytope (Assumption~\ref{ass:polytopic}).
Thus, we have reduced the {condition of} data-driven dissipativity %
in Theorem~\ref{thm:LPV_diss}, {which needs to be checked for all $\bar{p}_{[1,L]}\in\ms{P}_{[1,L]}$,}  %
to the existence of matrices $X_i$ such that~\eqref{eq:prop_LPV_diss_polytopic} holds, i.e., to an SDP subject to LMI constraints.  %
 
\subsubsection{{Including rate bounds on $\mathit{p}$}}\label{sss:conservatism}
{We now discuss the inclusion of additional system properties into the analysis problem in terms of incorporating rate bounds on the scheduling signal.}
The analysis in Proposition~\ref{prop:LPV_diss_polytopic} allows for maximal \emph{variation} of the scheduling variable, i.e., $p_k-p_{k-1}$ is only limited to remain inside $\mb{P}$. This can result in conservative conclusions, e.g., on the $\ell_2$-gain of the LPV system, as the analysis also considers (possibly non-existent) fast variations of the scheduling signal. We can reduce this %
by {incorporating} %
rate bounds on $p$, {i.e.,} %
defining the {admissible scheduling set as %
\begin{equation}
    {\ms{P}_{[1,L]}} %
    = \{ p_{[1,L]}\in\mb{P}^L \mid p_k-p_{k-1}\in\mb{D}, \ \forall k = 2, \dots, L \},
\end{equation}
where $\mb{P}$ satisfies Assumption~\ref{ass:polytopic}} and $\mb{D}$ is a convex polytope that defines the rate bounds on $p$. {Note that {$\ms{P}_{[1,L]}$}  is again a convex polytope.} By verifying \eqref{eq:prop_LPV_diss_polytopic} on the vertices of {$\ms{P}_{[1,L]}$}, %
{the analysis yields a %
less conservative
conclusion on the dissipativity property of the system.}
\begin{remark}[Computational complexity vs. conservatism]\label{rem:manylmis}
	Verifying dissipativity via \eqref{eq:prop_LPV_diss_polytopic} \emph{without} incorporation of the rate bounds on $p$ requires solving an SDP with {$(1+L\dnp)n_\mathrm{v}^L$} LMI constraints. This number further increases when including rate bounds, which increases the computational complexity exponentially for larger $L$. Explosion of the computational load for large $L$ can be mitigated by replacing~\eqref{eq:prop_LPV_diss_polytopic} with
\[ XF(\bar{\msf{p}}_\nu) + F(\bar{\msf{p}}_\nu)^\top X^\top  -\Pi_H \preceq0, \quad \nu=1,\dots, n_\mr{v}^L.\]
Introducing a constant $X$ can introduce conservatism, however, it also allows to alleviate the computational load of the dissipativity test.
\end{remark}
{We will see in Section~\ref{sec:examples} that Proposition~\ref{prop:LPV_diss_polytopic} is readily applicable to small-scale LPV systems and it provides tight conditions for verifying $L$-dissipativity from data.}

\subsection{Dissipativity verification via the S-procedure}\label{subsec:computational_S_procedure}
In this section, we assume that $\bar{p}_{[1,L]}$ is varying in a space that is described by a quadratic inequality instead of a convex polytope, i.e., we assume that the scheduling signals in $\ms{P}_{[1,L]}$ are described as follows:
\begin{assumption}[{Quadratic description of $\ms{P}_{[1,L]}$}]\label{ass:ellipsoidal}
The scheduling signal $\bar{p}_{[1,L]}$ varies around a nominal scheduling trajectory $\breve{p}_{[1,L]}$, such that $\bar{p}_{[1,L]}$ satisfies
\begin{align}\label{eq:p_norm_bound}
\lVert \bar{p}_k-\breve{p}_{k}\rVert_W\leq p_{\max}
\end{align}
for all $k=1,\dots,L$ with some $p_{\max}>0$. The set $\ms{P}_{[1,L]}$ can then be trivially described by  
\begin{equation}\label{eq:ass_ellipsoidal}
	{\ms{P}_{[1,L]}}=\left\{\bar{p}_{[1,L]}\in\mb{P}^L \Bigm| \vphantom{ \begin{bmatrix} \hat{\mc{P}}^{\dnu+\dny} \\ I \end{bmatrix}^\top} %
	\begin{bmatrix} \bar{\mc{P}}^{\dnu,\dny} \\ I \end{bmatrix}^{\!\top}\!\! M_P \begin{bmatrix} \bar{\mc{P}}^{\dnu,\dny} \\ I \end{bmatrix}\succeq 0 \right\}, %
\end{equation}
with $\bar{\mc{P}}^{\dnu,\dny}=\begin{bmatrix} \bar{\mathcal{P}}^{\dnu} & 0 \\ 0 & \bar{\mathcal{P}}^{\dny}\end{bmatrix}$, $\bar{\mathcal{P}}^{n_\bullet}=\bar{p}_{[1,L]}\bkron I_{n_\bullet}$.
\hfill$\square$
\end{assumption}
{Note that for $W=2$ in~\eqref{eq:p_norm_bound},} %
we have that $M_P$ in~\eqref{eq:ass_ellipsoidal} is
\begin{align}\label{eq:def_of_MP_ellipsiodal}
	M_P = \begin{bmatrix}
		-I & \breve{\mathcal{P}}^{\dnu,\dny} \\
		(\breve{\mathcal{P}}^{\dnu,\dny})^\top & p_{\max}^2 I -(\breve{\mathcal{P}}^{\dnu,\dny})^\top\breve{\mathcal{P}}^{\dnu,\dny}
	\end{bmatrix},
\end{align}
with $\breve{\mc{P}}^{\dnu,\dny} =\begin{bmatrix}\breve{\mc{P}}^{\dnu}&0\\0&\breve{\mc{P}}^{\dny}\end{bmatrix}$, $\breve{\mc{P}}^{n_\bullet}=\breve{p}_{[1,L]}\bkron I_{n_\bullet}$. Note that with Assumption~\ref{ass:ellipsoidal}, we can describe types of $\ms{P}_{[1,L]}$ that have a spherical or {hyper-rectangular} form. %
Next, we employ the S-procedure (see, e.g.,~\cite{scherer1997full, scherer2001lpv}) to derive a tractable reformulation of~\eqref{eq:thm_LPV_diss} for scheduling sets $\ms{P}_{[1,L]}$ that satisfy Assumption~\ref{ass:ellipsoidal}.
To this end, let us write $F(\bar{p}_{[1,L]})$ in~\eqref{eq:Pi_H_F_p_def} as
\begin{equation*}
\hspace{-0.5mm}	F(\bar{p}_{[1,L]})\!  =\!\!  %
	\underbrace{\begin{bmatrix} F_1 \\ \mc{H}_L(u_{[1,N]}^{\mt{p}}) \\ \mc{H}_L(y_{[1,N]}^{\mt{p}}) \end{bmatrix}}_{F_3}\!  -\! 	\underbrace{\begin{bmatrix} 0 & 0 \\ I & 0 \\ 0 & I \end{bmatrix}}_{F_4}\!\! \begin{bmatrix} \bar{\mathcal{P}}^{\dnu}\!\!\! & 0 \\ 0 & \!\!\!\bar{\mathcal{P}}^{\dny}\end{bmatrix} \!\! \underbrace{\begin{bmatrix} \mc{H}_L(u_{[1,N]}) \\ \mc{H}_L(y_{[1,N]}) \end{bmatrix}}_{F_5}\!.
\end{equation*}
\begin{prop}[$L$-dissipativity via the S-procedure]\label{prop:LPV_diss_ellipsoidal}
Suppose Assumption~\ref{ass:ellipsoidal} holds.
Then,~\eqref{eq:thm_LPV_diss} holds if there exist a $\mu\in\mathbb{R}$ and a $\lambda\geq0$ such that
\begin{align}\label{eq:prop_LPV_diss_ellipsoidal}
\begin{bmatrix}\mu F_4^\top F_4&\mu F_4^\top F_3\\
\mu F_3^\top F_4&\mu F_3^\top F_3+\Pi_H\end{bmatrix}-\lambda \begin{bmatrix}I&0\\0&F_5\end{bmatrix}^{\!\top} \!\!
M_P
\begin{bmatrix}I&0\\0&F_5\end{bmatrix}\succeq0.
\end{align}
\end{prop}


\begin{proof}
{Using}
the implication of (i) by (iii) in Lemma~\ref{lem:Finsler},~\eqref{eq:thm_LPV_diss} holds if for each $\bar{p}_{[1,L]}\in\ms{P}_{[1,L]}$ there exists a $\mu\in\mathbb{R}$ satisfying
\begin{align}\label{eq:prop_LPV_diss_ellipsoidal_proof1}
	\Pi_H+\mu F^\top\!(\bar{p}_{[1,L]}) F(\bar{p}_{[1,L]})\succeq0.
\end{align}
 {Note that {in terms of this condition,} $\mu$ {can be different for each} $\bar{p}_{[1,L]}$. To obtain a convex problem, we {enforce} $\mu$ to be {the same}\footnote{{See Remark~\ref{rem:PVmulti} for a discussion on a scheduling dependent $\mu$.}} for all $\bar{p}_{[1,L]}\in\ms{P}_{[1,L]}$}, {which inherently introduces conservatism.}
{Then, we can rewrite}~\eqref{eq:prop_LPV_diss_ellipsoidal_proof1} {as} %
\begin{align}\label{eq:prop_LPV_diss_ellipsoidal_proof2}
\begin{bmatrix}\bar{\mathcal{P}}^{\dnu,\dny}F_5\\I\end{bmatrix}^\top
\begin{bmatrix}\mu F_4^\top F_4&\mu F_4^\top F_3\\
\mu F_3^\top F_4&\mu F_3^\top F_3+\Pi_H\end{bmatrix}
\begin{bmatrix}\bar{\mathcal{P}}^{\dnu,\dny}F_5\\I\end{bmatrix}\succeq0.
\end{align}
{By} left and right {multiplication of {the}} inequality in~\eqref{eq:ass_ellipsoidal} {with} $F_5^\top$ and $F_5$, respectively, we infer that $\bar{p}_{[1,L]}\in\ms{P}_{[1,L]}$ {if and only if} %
\begin{align}\label{eq:prop_LPV_diss_ellipsoidal_proof3}
	\begin{bmatrix} \bar{\mathcal{P}}^{\dnu,\dny}F_5 \\ I \end{bmatrix}^\top
	\begin{bmatrix} I & 0 \\ 0 & F_5 \end{bmatrix}^\top
	M_P
	\begin{bmatrix} I & 0 \\ 0 & F_5 \end{bmatrix}
	\begin{bmatrix} \bar{\mathcal{P}}^{\dnu,\dny}F_5 \\ I \end{bmatrix}\succeq 0.
\end{align}
Multiplying~\eqref{eq:prop_LPV_diss_ellipsoidal} with $\begin{bmatrix}(\bar{\mathcal{P}}^{\dnu,\dny}F_5)^\top & I \end{bmatrix}$ from the left and with $\begin{bmatrix}(\bar{\mathcal{P}}^{\dnu,\dny}F_5)^\top & I \end{bmatrix}^\top$ from the right and using~\eqref{eq:prop_LPV_diss_ellipsoidal_proof3}, we {get that if \eqref{eq:prop_LPV_diss_ellipsoidal} holds, then}~\eqref{eq:prop_LPV_diss_ellipsoidal_proof2} is implied, which concludes the proof.
\end{proof}

Proposition~\ref{prop:LPV_diss_ellipsoidal} provides a sufficient condition for~\eqref{eq:thm_LPV_diss} and hence, by Theorem~\ref{thm:LPV_diss}, for $(L-\tau)$-dissipativity of the underlying LPV system.
Verifying feasibility of~\eqref{eq:prop_LPV_diss_ellipsoidal} is an SDP with only two decision variables $\mu$ and $\lambda$, and thus it can be {efficiently} implemented from a {computational} perspective for {moderate and even large choices of $L$ and $N$} %
data lengths (see Section~\ref{sec:examples} for a numerical example).

Note that, {similar to} Proposition~\ref{prop:LPV_diss_polytopic}, Proposition~\ref{prop:LPV_diss_ellipsoidal} only provides a \emph{sufficient} condition for~\eqref{eq:thm_LPV_diss}.
This is because the multiplier $\mu\in\mathbb{R}$ %
is chosen {to be the same for all} %
scheduling {trajectories} $\bar{p}_{[1,L]}$.
Therefore, if, e.g., a norm bound similar to~\eqref{eq:p_norm_bound} is available, then translating this bound into a convex polytope as in Assumption~\ref{ass:polytopic} and applying Proposition~\ref{prop:LPV_diss_polytopic} will %
generally lead to less conservative results than considering the quadratic description in Assumption~\ref{ass:ellipsoidal} and applying Proposition~\ref{prop:LPV_diss_ellipsoidal}, even for a constant $X$.
On the other hand, as we will also demonstrate with a numerical example in Section~\ref{sec:examples}, Proposition~\ref{prop:LPV_diss_ellipsoidal} is significantly more efficient  from a computational perspective than Proposition~\ref{prop:LPV_diss_polytopic}.

{Finally, we want to highlight that for all the introduced computational methods, we assume that the `true' admissible scheduling set $\ms{P}_{[1,L]}$ is equivalent with the assumed descriptions (Assumptions~\ref{ass:polytopic}~and~\ref{ass:ellipsoidal}). If the assumed descriptions are in fact \emph{over approximating} $\ms{P}_{[1,L]}$, which can happen when the LPV representation is in fact a \emph{surrogate} model for a nonlinear system, then this again introduces a source of conservatism in the analysis. This issue of conservatism and minimizing it is actively studied in %
constructing LPV surrogate models of nonlinear systems \cite{SADEGHZADEH20204737}.}


\begin{remark}[{Sampling-based $L$-dissipativity analysis}]\label{rem:PVmulti}
The described conservatism of Proposition~\ref{prop:LPV_diss_ellipsoidal} can be alleviated by considering \emph{parameter-dependent} multipliers for the application of Finsler's Lemma (Lemma~\ref{lem:Finsler}), similar to Proposition~\ref{prop:LPV_diss_polytopic}.
To be precise,~\eqref{eq:prop_LPV_diss_ellipsoidal_proof1} can be replaced by
\begin{align}\label{eq:prop_LPV_diss_ellipsoidal_pv}
	\Pi_H+\mu(\bar{p}_{[1,L]}) F(\bar{p}_{[1,L]})^\top F(\bar{p}_{[1,L]})\succeq0,
\end{align}
for some $\mu:\ms{P}_{[1,L]}\to\mathbb{R}$.
In fact, by applying Finsler's Lemma point-wise for any parameter trajectory $\bar{p}_{[1,L]}\in\ms{P}_{[1,L]}$, the existence of a mapping $\mu(\cdot)$ such that~\eqref{eq:prop_LPV_diss_ellipsoidal_pv} holds is a less conservative\footnote{In fact, the provided condition is also necessary, and therefore non-conservative, for \emph{strict} $L$-dissipativity.\label{footnote}} condition for~\eqref{eq:thm_LPV_diss} than~\eqref{eq:prop_LPV_diss_ellipsoidal}.
In practice, inequality~\eqref{eq:prop_LPV_diss_ellipsoidal_pv} can be checked by drawing $N_\mathrm{s}$ (e.g., uniformly distributed) samples $\hat{p}_{[1,L]}^i$ from $\ms{P}_{[1,L]}$ and verifying the existence of $\mu_i\in\mathbb{R}$, $i=1,\dots,N_\mathrm{s}$ such that
\begin{align}
	\Pi_H+\mu_i F(\hat{p}_{[1,L]}^i)^\top F(\hat{p}_{[1,L]}^i)\succeq0
\end{align}
for all $i=1,\dots,N_\mathrm{s}$.
For any fixed number $N_\mathrm{s}$, this yields a less conservative\tss{\ref{footnote}} condition for~\eqref{eq:thm_LPV_diss} and, hence, for $L$-dissipativity compared to Proposition~\ref{prop:LPV_diss_ellipsoidal}.
The resulting dissipativity test is \emph{tight} in the limit $N_\mathrm{s}\to\infty$.
While choosing a dense grid over $\ms{P}_{[1,L]}$, i.e., a large value of $N_\mathrm{s}$, can be prohibitive due to the curse of dimensionality, good results can often be achieved in practice for reasonable values of $N_\mathrm{s}$, see the numerical example in Section~\ref{sec:examples}.
\end{remark}
\begin{remark}[Reducing conservatism with matrix-valued multipliers]
    Another option to reduce the conservatism of Proposition~\ref{prop:LPV_diss_ellipsoidal} is to consider a matrix-valued~$\lambda$. As \eqref{eq:p_norm_bound} constitutes~$L$ inequalities, we can %
    {introduce} at least~$L$ degrees of freedom in~$\lambda$ using %
    full-block multipliers~\cite{scherer1997full}. {Using full-block multipliers} does, on the other hand, significantly increase the complexity of the problem. Exploring such a formulation of the analysis is interesting for further research.
\end{remark}