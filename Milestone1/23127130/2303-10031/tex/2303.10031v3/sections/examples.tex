This section demonstrates the effectiveness of the developed theory by means of determining the $\ell_2 $-gain of the considered LPV systems {in two academic examples} {and one realistic case-study on the LPV embedding of a nonlinear unbalanced disc system}. All the computations discussed in this section have been executed using \matlab on a MacBook Pro~(2020) with Intel Core i5 processor and solved using YALMIP \cite{YALMIP} with the MOSEK solver \cite{MOSEK}.


\subsection{Example I: {Model-based vs. data-driven analysis}} \label{example:1}
In this example, we compare our direct data-driven {analysis }approaches with a model-based and an indirect data-driven\footnote{With indirect data-driven method we mean to first perform system identification that results in a model on which model-based analysis {is performed}.} method. Consider an LPV system {described in the form of} \eqref{eq:sys} with $\dny=\dnu=\dnp=1$, $\dna=\dnb=3$ and $a,b$  as
\begin{align*}
    \begin{bmatrix} a_{1,0} & a_{2,0} & a_{3,0} \\ a_{1,1} & a_{2,1} & a_{3,1} \end{bmatrix} & =%
    \begin{bmatrix} 0.0826    & 0.1491    & -0.1196   \\ 0.0311    & 0.0570    & -0.0650   \end{bmatrix}, \\
    \begin{bmatrix} b_{1,0} & b_{2,0} & b_{3,0} \\ b_{1,1} & b_{2,1} & b_{3,1} \end{bmatrix} & =%
    \begin{bmatrix} 0.5007    & -0.5588   & 0.0784    \\ 0.6953    & 1.8192    & -1.7192   \end{bmatrix}.
\end{align*}
Moreover, we define $\mb{P}$ as $[-0.1,\> 0.1]$ with no rate bounds on~$p$, i.e., $\ms{P}_{[1,L]}=[-0.1,\> 0.1]^L$. For our data-driven dissipativity analysis, we want to determine the $\ell_2 $-gain of this system for a horizon of 7. Hence, we choose $\tau=\max\{\dna,\dnb\}=3$ and set $L$ to 10. {As the system is {SISO}, we have that $\dnx=\max\{\dna,\dnb\}={n_\mathrm{r}}= 3$ \cite{Toth2011_SSrealizationTCST}.} By exciting the system with a {white noise} input $u_k\sim\mc{N}(0,1)$ and scheduling $p_k\sim\mc{U}(\mb{P})$, {where $\mc{N}$ and $\mc{U}$ denote normal and uniform distributions, respectively}, the data-dictionary $\dataset$  with $N=42$ shown in Fig.~\ref{fig:datadictionary} is persistently exciting of order $(L,\dnx)$ {in terms of Proposition~\ref{prop:FLeasy}}.
\begin{figure}
\centering
\includegraphics[scale=1]{figures/datadictionary} %
\caption{Data-dictionary for Example I.}\label{fig:datadictionary} %
\end{figure}
{We compute an upper bound $\gamma$ on the $(L-\tau)$-horizon $\ell_2 $-gain with the model-based approach that is %
{given}
in Appendix~\ref{app:modelbasedLdissip}. We verify~\eqref{eq:LMImodelbased} by using a polytopic description of $\ms{P}_{[1,L]}$ and solving~\eqref{eq:LMImodelbased} on the vertices of $\ms{P}_{[1,L]}$, while minimizing the upper bound $\gamma$. Solving this problem yields  $\gamma=1.362$.} 
Next, we employ the \lpvcore toolbox\footnote{See \cite{BoefCoxToth2021} and \texttt{lpvcore.net} for {this} open-source \matlab toolbox.} to identify {a state-space model of} the LPV system using $\dataset$ and compute the $(L-\tau)$-horizon $\ell_2 $-gain of the resulting identified model. For the identification, we use {the} LPV PEM-SS {method} (see \cite{Toth18cAUT, Cox2018})
with the default settings in \lpvcore~{via} the function \texttt{lpvssest}. The {resulting}  {upper bound $\gamma$ on the $(L-\tau)$-horizon} $\ell_2 $-gain of the identified {LPV} model is 
1.375.%


Finally, we determine the $(L-\tau)$-horizon $\ell_2 $-gain of the system \emph{directly} with $\dataset$ using the results in this paper {to demonstrate} %
 that our direct data-driven dissipativity verification methods are competitive with their (indirect) model-based counter-parts. First, we employ the direct data-driven verification method via the convex hull argument. With the defined range for $p$, we construct the hypercube $\mb{P}^L$  {that defines $\ms{P}_{[1,L]}$}. The resulting hypercube has $n_\mathrm{v}=1024$ vertices. Together with the construction of $F$ using \emph{only} the data-dictionary $\dataset$, we can now solve~\eqref{eq:prop_LPV_diss_polytopic} with a constant $X$ on the vertices of $\mb{P}^L$ and minimize the value $\gamma$ that corresponds to an  {upper bound on the $(L-\tau)$-horizon} $\ell_2 $-gain of the system. The corresponding SDP yields $\gamma=1.362$,  {which demonstrates equivalence with the model-based method}. For the direct data-driven verification method using the S-procedure, we again use $\dataset$ to construct the necessary matrices to apply Proposition~\ref{prop:LPV_diss_ellipsoidal}. Note that by the definition of the scheduling region,  {we can equivalently describe $\ms{P}_{[1,L]}$ using \eqref{eq:def_of_MP_ellipsiodal} by choosing $p_\mr{max}=0.2$ and $\breve{p}_{[1,L]}$ as a zero trajectory.} Solving~\eqref{eq:prop_LPV_diss_ellipsoidal} yields a value for $\gamma$ of 1.831, representing a bound on the $(L-\tau)$-horizon $\ell_2 $-gain of the LPV system. This illustrates the aforementioned conservatism of Proposition~\ref{prop:LPV_diss_ellipsoidal}. On the other hand, the computational load of the latter method is significantly smaller, which we will further illustrate in Example~II, after demonstrating the influence of noise on the data-driven dissipativity analysis.



\subsubsection*{Influence of noisy data}
To study the influence of noise on the $L$-dissipativity analysis, {we repeat} the analysis of this example for different levels of noise injected into the system {in terms of an ARX-type of structure:
\begin{align}\label{eq:noisy}
y_k+\sum_{i=1}^{\dna}a_i(p_{k-i})y_{k-i}=\sum_{i=0}^{\dnb}b_i(p_{k-i})u_{k-i} + e_k,
\end{align} 
where $e_k$ is an i.i.d. white noise process with $e_k\in\mathcal{N}(0,\sigma_\mathrm{e}^2)$.}
 For every analysis experiment, we use the same noise realization (sampled from a normal distribution) and adjust the variance of the noise by means of multiplication {of the sequence} with a constant. {In this way,} we obtain data-dictionaries with a \emph{signal-to-noise}~(SNR) between 26 and 47 dB. %
 {To provide robustness against the noise for the analysis methods based on Proposition~\ref{prop:LPV_diss_polytopic} and~\ref{prop:LPV_diss_ellipsoidal}, we have introduced a regularization parameter in LMI~\eqref{eq:prop_LPV_diss_polytopic:a} to force the left-hand side to be less than $-\delta I$, and in LMI~\eqref{eq:prop_LPV_diss_ellipsoidal} to be larger than $\delta I$. The parameter~$\delta$ can be chosen as the standard deviation $\sigma_\mathrm{e}$ of the noise, which can be estimated in practice. While such a regularization introduces conservatism in the obtained performance bounds, it provides robust feasibility of the analysis against the noise.} The analysis results {under} the noisy data-dictionaries, where we took $\delta=\sigma_\mathrm{e}$,  are shown in Fig.~\ref{fig:influencenoise}.
\begin{figure}
\centering
\includegraphics[scale=1]{figures/influencenoise} 
\caption{Influence of noisy data on the $(L-\tau)$-dissipativity analysis.}\label{fig:influencenoise} 
\end{figure}
The horizontal lines indicate the results for noise-free data, i.e., the earlier obtained results, while the blue circles and red crosses correspond to the upper bound $\gamma$ on the $(L-\tau)$-horizon $\ell_2$-gain calculated with Proposition~\ref{prop:LPV_diss_polytopic} and~\ref{prop:LPV_diss_ellipsoidal}, respectively. 
The plot shows that as {$\sigma_\mathrm{e}$ grows}, i.e., the SNR drops, the results of the analysis become more conservative. However, approaching to high noise levels in terms of a SNR $< 33$ dB, the optimization requires excessively large regularization to remain feasible, even beyond $\delta=\sigma_\mathrm{e}$ for Proposition~\ref{prop:LPV_diss_ellipsoidal}. This  indicates that our methods are applicable under low and moderate levels of noise (SNR $\geq 35$ dB), but a systematic stochastic way of handling the noise is required for further robustifying the analysis results against more severe noise conditions.

\subsection{Example II: {$(\mathit{L}-\tau)$-dissipativity for multiple $\mathit{L}$}}\label{ss:example2}
In this example, we analyze $(L-\tau)$-dissipativity for multiple values of $L$ and, with this, demonstrate the relationship between $(L-\tau)$-horizon dissipativity and infinite-horizon dissipativity in terms of the $\ell_2 $-gain. This example also illustrates the difference in computational load for the proposed methods. {Consider} a similar LPV system {as in Example \ref{example:1}}, now with $\dna=\dnb=2$. {In this case, the coefficients $a$ and $b$} are %
\begin{align*}
    \begin{bmatrix} a_{1,0} & a_{2,0}  \\ a_{1,1} & a_{2,1}  \end{bmatrix} & =%
    \begin{bmatrix} -0.00569  &  0.0706   \\ 0.1137    & -0.0210   \end{bmatrix} \\
    \begin{bmatrix} b_{1,0} & b_{2,0}  \\ b_{1,1} & b_{2,1}  \end{bmatrix} & =%
    \begin{bmatrix}   1.3735  & -0.2941   \\ 0.1254    &  0.6617   \end{bmatrix}.
\end{align*}
The scheduling set {$\mb{P}$} is defined as $[-0.2,\>0.2]$. We also impose a rate bound on the scheduling signal: %
$p_{k}-p_{k-1}\in[-0.1,\> 0.1]={\mb{D}}$. In this example, we determine upper bounds for the $(L-\tau)$-horizon $\ell_2 $-gain of this system using our direct dissipativity analysis approach{es} for $L=3,\dots,8$, and compare these results to the upper bound for the infinite-horizon $\ell_2 $-gain, which is obtained {via model-based analysis} using the \lpvcore toolbox.
\begin{figure}
\centering
\includegraphics[scale=1]{figures/datadictionary2}  %
\caption{Data-dictionary for Example II.}\label{fig:datadictionary2}  %
\end{figure}
We choose $\tau=2$ and generate our data-dictionary $\dataset$ with $N=33$, by exciting the system with an input  {$u_k\sim\mc{N}(0,1)$} and scheduling $p_k$, {whose samples are drawn from an i.i.d. uniform distribution {on $\mathbb{P}$} that is truncated to satisfy {the rate bound}. The resulting $\dataset$} is PE {of order $(L\le8,\dnx=2)$}. The generated data-dictionary is shown in Fig.~\ref{fig:datadictionary2}.

We use the model of the LPV system to compute {an upper bound $\gamma$ on the (infinite-horizon)} $\ell_2 $-gain with \lpvcore~{using a model-based approach}, which yields $\gamma=1.667$. {When~$\mb{D}$ is not considered in the analysis, we obtain $\gamma=1.754$}. With $\dataset$, we identify two LPV-SS representations using the same settings as in Example I, one considering $\mb{P}$ and $\mb{D}$ and one considering only $\mb{P}$. Calculating the upper bound $\gamma$ on the (infinite-horizon) $\ell_2 $-gain of the two identified models yields $\gamma=1.660$ for the former and $\gamma=1.754$ for the latter. 

We will now compare these (indirect) model-based analysis results to the results of direct data-driven $(L-\tau)$-dissipativity analysis for $L=3,\dots,8$. We will apply all four verification methods that we {have discussed} in this paper to this example, i.e., we verify \eqref{eq:thm_LPV_diss} by means of:
\begin{itemize}
	\item A polytopic description of $\mb{P}$, using the convex hull argument in Proposition~\ref{prop:LPV_diss_polytopic} ({denoted as} CHA),
	\item A quadratic description of $\mb{P}$, using the S-procedure in Proposition~\ref{prop:LPV_diss_ellipsoidal} ({denoted as} SP).
\end{itemize}
Furthermore, we reduce the conservatism {of the analysis} introduced in the aforementioned methods by:
\begin{itemize}
	\item Including {the rate bound $\mb{D}$ on $p$ in} the CHA method, as discussed in Section~\ref{sss:conservatism} ({denoted as} CHAr),
	\item Considering {$p$-}dependent multipliers and, as discussed in Remark~\ref{rem:PVmulti}, {solve SP with a sampling-based approach}  ({denoted as} SDM). {For this computation, we generate} $N_\mr{s}=1000$ sample trajectories from $\ms{P}_{[1,L]}$. 
\end{itemize}
{For the CHA(r) methods, we will use a constant $X$.}

With the above {given} four methods, computing {an} upper bound $\gamma$ on the $(L-\tau)$-horizon $\ell_2 $-gain of the system using $\dataset$ {gives the results depicted in  Fig.~\ref{fig:result_multL} for $L=3,\dots,8$}. 
\begin{figure}
\centering
\begin{subfigure}[b]{\linewidth}
\centering
\includegraphics[scale=1]{figures/resultsex2}
\caption{}\label{fig:result_multL:con}
\end{subfigure}
\begin{subfigure}[b]{\linewidth}
\includegraphics[scale=1]{figures/resultsex2_rates}
\caption{}\label{fig:result_multL:ncon}
\end{subfigure}
\caption{Model-based and indirect data-driven dissipativity analysis versus direct data-driven $(L-\tau)$-dissipativity analysis of an LPV system for an increasing horizon  $L$. Plot (a) shows the results {when} the rate bounds on $p$ are not included in the analysis, while (b) shows the results {when} the {rate bounds are included.}}\label{fig:result_multL} %
\end{figure}
{The plot in} Fig.~\ref{fig:result_multL:con} shows {the results of the CHA and SP {data-driven} methods} %
together with the model-based and indirect dissipativity analysis methods, while the %
plot in Fig.~\ref{fig:result_multL:ncon} depicts the results of the {CHAr and SDM direct data-driven dissipativity analysis methods}
together with the model-based and indirect dissipativity analysis methods. %
The plots show for this simple system {that} the finite-horizon direct data-driven dissipativity analysis results converge relatively quick{ly} to the infinite-horizon results {of the} model-based approaches. As expected, Fig.~\ref{fig:result_multL:con} demonstrates that the dissipativity verification via Proposition~\ref{prop:LPV_diss_ellipsoidal} is slightly more conservative, i.e., the computed upper bound on the $\ell_2$-gain is slightly larger. However, this conservatism is negated when the multiplier is considered to be scheduling dependent, as shown in Fig.~\ref{fig:result_multL:ncon}. Hence, our methods give comparable results to the model-based approach when the horizon is chosen sufficiently large {even if} we \emph{only} make use of the information that is encoded in the data-dictionary $\dataset$. 

As highlighted throughout the paper, e.g., Remark~\ref{rem:manylmis}~and~\ref{rem:PVmulti}, the computational complexity of the CHA(r) method(s) can be significant, especially when compared to the method based on the S-procedure or the sampling-based method. This is {illustrated by} Table~\ref{tab:comp}, where we have listed the computation time for every method for an increasing $L$. For comparison, we also included the computation time for model-based $L$-dissipativity verification for the increasing horizon.
\begin{table}
\centering
\caption{Computation time (in seconds) for verifying $(L-\tau)$-dissipativity for $L=3,\dots,8$ using the methods discussed in this paper{:} Convex Hull Argument (CHA), S-Procedure (SP), CHA that incorporates rate bounds (CHAr) and sampling-based approach with Scheduling Dependent Multipliers (SDM). {The last column contains the model-based $(L-\tau)$-dissipativity verification approach (MBA) for comparison.}}\label{tab:comp}
\begin{tabular}{c|rrrrr}
$L$ & \multicolumn{1}{c}{CHA} & \multicolumn{1}{c}{SP} & \multicolumn{1}{c}{CHAr} & \multicolumn{1}{c}{SDM} & \multicolumn{1}{c}{MBA}  \\ \hline
3 & 0.82 & 0.51 & 1.10 & 9.65 & 0.41\\
4 & 1.85 & 0.38 & 3.99 & 15.00 & 0.32\\
5 & 4.31 & 0.39 & 9.10 & 12.31 & 0.33\\
6 & 6.90 & 0.40 & 21.39 & 10.88 & 0.35\\
7 & 16.55 & 0.42 & 56.25 & 12.91 & 0.32\\
8 & 38.31 & 0.45 & 168.80 & 12.08 & 0.36\\
\end{tabular} 
\end{table}
Due to the exponential growth of the number of constraints and decision variables {with horizon}  $L$ of the dissipativity verification problem using the CHA(r) methods, the computation time significantly increases for relatively small horizons, even for simple systems. This makes the approach based on the S-procedure much more suitable for larger systems/horizons at the price of conservatism.

{Finally, we want to comment on the choice of $L$. {We} have seen in this example that a larger $L$ yields {$\ell_2$-gains} `closer' to the infinite-horizon {$\ell_2 $-gain}, which is in line with the discussion of Remark~\ref{rem:fin_vs_inf}. 
In this particular case, we observed that the condition $(L-\tau)\ge2\dnr$ serves as a good heuristic for obtaining an $\ell_2$-gain estimate close to the true infinite-horizon value.
However, a generic guideline for choosing $L$ will require proper understanding {of} the fundamental relationship between $L$-dissipativity and infinite-horizon dissipativity, which is outside  the scope of this paper.}

\subsection{Example III: Unbalanced disc}
{As highlighted in Section~\ref{sec:introduction}, the LPV framework is often used for the analysis and control of nonlinear systems. Therefore, we will %
 {apply the proposed} data-driven dissipativity analysis methods {in a simulation study of} %
an unbalanced disc system, {showing} %
applicability of the methods on physical systems. In this example, we analyze $L$-dissipativity of the {unbalanced disc} 
in closed-loop with an LPV controller.

\subsubsection{Unbalanced disc system} 
The unbalanced disc system consists of a disc with an off-centered mass whose angular position can be controlled by {an} attached DC motor. {The system can be seen as a rotational} %
pendulum, {whose} continuous-time dynamics %
are described by
\begin{equation}\label{eq:unbalanced-disc}
    \ddot{\theta}(t)=-\tfrac{mgl}{J}\sin(\theta(t))-\tfrac{1}{\kappa}\dot{\theta}(t)+\tfrac{K_\mr{m}}{\kappa}u(t),
\end{equation}
where $\theta$ is the angular position of the disc in radians, $u$ is the input voltage to the system, which {can be used as a} control input, and $m,g,l,J,\kappa,K_\mr{m}$ are the physical parameters of the system, {given in} \cite{BoefCoxToth2021}. As we work in discrete time, we discretize the dynamics using a first-order Euler method with sampling-time $T_\mr{s}$, which yields
\begin{multline}\label{eq:UB_IO_NL}
    \theta_k + (\tfrac{T_\mr{s}}{\kappa}-2)\theta_{k-1} + (1 -\tfrac{T_\mr{s}}{\kappa}) \theta_{k-2} \\
    + T_\mr{s}^2\tfrac{mgl}{J}\sin(\theta_{k-2}) = T_\mr{s}^2\tfrac{K_\mr{m}}{\kappa}u_{k-2}. 
\end{multline}
We can \emph{embed} \eqref{eq:UB_IO_NL} {into an} LPV system by defining the scheduling as $p_k = \sinc(\theta_k)=\tfrac{\sin(\theta_k)}{\theta_k}$, which yields that $\mb{P}:=[-0.22,1]$. We choose $T_\mr{s}=0.01$ [s], which gives a negligible discretization error. %

\subsubsection{Controller design}
We design a reference tracking LPV controller for the unbalanced disc system according to the block-diagram depicted in Fig.~\ref{fig:controlscheme},
\begin{figure}
\centering
\includegraphics[width=\linewidth]{figures/controlscheme}
\caption{Control scheme {considered in Example III}.}\label{fig:controlscheme}
\end{figure}
such that we can analyze $(L-\tau)$-dissipativity of the mapping $r_k\to e_k$, where $r_k$ represents the reference for $\theta$, and $e_k=r_k-\theta_k$. To use our developed methods, the closed-loop, i.e., the mapping $r_k\to e_k$, must admit a shifted-affine form as in \eqref{eq:sys}. This is ensured by designing  {a simple FIR-type of} LPV controller with shifted-affine scheduling dependence, i.e., 
\[
    u_k = k_0(p_k)e_k + k_1(p_{k-1})e_{k-1} + \dots + k_{n_\mr{k}}(p_{k-n_\mr{k}})e_{k-n_\mr{k}}.
\]
For $n_\mr{k}=2$, we design the LPV controller by manually tuning the coefficients $k_{0,0},\dots,k_{2,1}$. Choosing the coefficients as
\[
    \begin{bmatrix} k_{1,0} & k_{2,0} & k_{3,0} \\ k_{1,1} & k_{2,1} & k_{3,1} \end{bmatrix} = \begin{bmatrix}
        -80 & 80 & -3 \\ 1.5 & -3 & 0.3
    \end{bmatrix},
\]
yields a response as in Fig.~\ref{fig:closedloopresponse}.  
\begin{figure}
\centering
\includegraphics[scale=1]{figures/closedloopresponse}
\caption{Closed-loop response {of the unbalanced disc system with the considered LPV controller, which is used as the data-dictionary in Example III.}}
\label{fig:closedloopresponse}
\end{figure}
The closed-loop behavior can hence be described as a 4\tss{th}-order LPV-IO difference equation with shifted-affine scheduling dependence, where $r_k$ is the input, $e_k$ is the output and $p_k:=\sinc(r_k-e_k)$ is the scheduling.
\subsubsection{Closed-loop analysis}
As in the previous examples, we first calculate the upper bound $\gamma$ on the (infinite-horizon) $\ell_2$-gain of the closed-loop {with \lpvcore} system using the exact model, which yields $\gamma=1.44$. Note that, as $\mf{B}_\mr{NL}\subseteq\mf{B}_\mr{LPV}$, this implies {an} upper bound on the $\ell_2 $-gain of the nonlinear closed-loop system. For the data-driven dissipativity analysis, we calculate the finite-horizon $\ell_2$-gain for twice the order of the system, i.e., $(L-\tau)=8$ with $L=12$ and $\tau=4$. Furthermore, we construct our data-dictionary $\dataset$ directly from the response depicted in Fig.~\ref{fig:closedloopresponse}, which {corresponds} to taking nominal operational data from the system, instead of {a} carefully designed experiment. Note that for this case, the scheduling data in $\dataset$ is constructed via $p_k=\sinc(r_k-e_k)$. We determine {the} $(L-\tau)$-dissipativity {based $\ell_2 $-gain} with the SP, SDM and the MBA approaches. For the SDM approach, we sample 100 trajectories from $\ms{P}_{[1,L]}$. The analysis results with {all the considered} approaches are given in Table~\ref{tab:unbal}.
\begin{table}
\centering
\caption{$(L-\tau)$-dissipativity analysis on the LPV embedding of the controlled unbalanced disc system for $L=12$, $\tau=4$ using the SP, SDM and MBA approaches.}\label{tab:unbal}
\begin{tabular}{c|cccc}
 & {True $\ell_2 $-gain}& SP & SDM & MBA  \\ \hline
Upper bound $\gamma$ & 1.449 & 1.548 & 1.319 & 1.263 \\
\end{tabular}
\end{table}
We {can} see that the upper bound given by \lpvcore is between the obtained bounds of the more conservative SP and the SDM {approaches}. (Note that the MBA and the SDM approaches under-estimate the infinite-horizon $\ell_2 $-gain, which is due to the difference between finite- and infinite-horizon dissipativity, see also Remark~\ref{rem:fin_vs_inf}). When we increase $L$ to 25, the resulting bound on the finite-horizon $\ell_2 $-gain with the SDM approach is 1.325, while the SP gives 1.807. We can conclude that application of the presented methods on LPV embeddings of nonlinear systems can yield, for sufficiently large $L$, good indications of (an upper bound on) the infinite-horizon $\ell_2 $-gain. %
}

























