 \begin{IEEEbiography}[{\includegraphics[width=1in,height=1.25in,clip,keepaspectratio]{figures/chris.jpg}}]{Chris Verhoek} received his B.Sc. degree in Mechatronics from the Avans University of Applied Sciences and M.Sc. degree (Cum Laude) in Systems and Control from the Eindhoven University of Technology (TU/e), in 2017 and 2020 respectively. His M.Sc. thesis was selected as best thesis of the Electrical Engineering department in the year 2020. 
 
He is currently pursuing a Ph.D. degree under the supervision of Roland T\'oth and Sofie Haesaert at the Control Systems Group, Dept. of Electrical Engineering, TU/e. His main research interests include (data-driven) analysis and control of nonlinear and LPV systems and learning-for-control techniques with stability and performance guarantees.
\end{IEEEbiography}

\begin{IEEEbiography}[{\includegraphics[width=1in,height=1.25in,clip,keepaspectratio]{figures/julian.jpg}}]{Julian Berberich} received a Master's degree in Engineering Cybernetics from the University of Stuttgart, Germany, in 2018. In 2022, he obtained a Ph.D. in Mechanical Engineering, also from the University of Stuttgart, Germany. He is currently working as a Lecturer (Akademischer Rat) at the Institute for Systems Theory and Automatic Control at the University of Stuttgart, Germany. In 2022, he was a visiting researcher at the ETH Z{\"u}rich, Switzerland. He has received the Outstanding Student Paper Award at the 59th IEEE Conference on Decision and Control in 2020 and the 2022 George S. Axelby Outstanding Paper Award. His research interests include data-driven analysis and control as well as quantum computing.
\end{IEEEbiography}

\begin{IEEEbiography}[{\includegraphics[width=1in,height=1.25in,clip,keepaspectratio]{figures/sofie.jpg}}]{Sofie Haesaert} received the B.Sc. degree Cum Laude in mechanical engineering and the M.Sc. degree Cum Laude in systems and control from the Delft University of Technology (TUDelft), Delft, The Netherlands, in 2010 and 2012, respectively, and the Ph.D. degree from TU/e, Eindhoven, The Netherlands, in 2017.

She is currently an Assistant Professor with the Control Systems Group, Department of Electrical Engineering, TU/e. From 2017 to 2018, she was a Postdoctoral Scholar with Caltech. Her research interests are in the identification, verification, and control of cyber-physical systems for temporal logic specifications and performance objectives.
\end{IEEEbiography}

\begin{IEEEbiography}[{\includegraphics[width=1in,height=1.25in,clip,keepaspectratio]{figures/frank.jpg}}]{Frank Allg\"ower} is professor of mechanical engineering at the University of Stuttgart, Germany, and Director of the Institute for Systems Theory and Automatic Control (IST) there.

Frank is active in serving the community in several roles: Among others he has been President of the International Federation of Automatic Control (IFAC) for the years 2017-2020, Vice-president for Technical Activities of the IEEE Control Systems Society for 2013/14, and Editor of the journal Automatica from 2001 until 2015. From 2012 until 2020 Frank served in addition as Vice-president for the German Research Foundation (DFG), which is Germany's most important research funding organization. 

His research interests include predictive control, data-based control, networked control, cooperative control, and nonlinear control with application to a wide range of fields including systems biology.
\end{IEEEbiography}

\begin{IEEEbiography}[{\includegraphics[width=1in,height=1.25in,clip,keepaspectratio]{figures/roland.jpg}}]{Roland T\'oth} received his Ph.D. degree with Cum Laude distinction at the Delft Center for Systems and Control (DCSC), TUDelft, Delft, The Netherlands in 2008.  He was a Post-Doctoral Research Fellow at TUDelft in 2009 and Berkeley in 2010. He held a position at DCSC, TUDelft in 2011-12. Currently, he is an Associate Professor at the Control Systems Group, TU/e and a senior researcher at SZTAKI, Budapest, Hungary. 

His research interests are in identification and control of linear parameter-varying (LPV) and nonlinear systems, developing machine learning methods with performance and stability guarantees for modeling and control, model predictive control and behavioral system theory.
\end{IEEEbiography}