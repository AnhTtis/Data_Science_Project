This section demonstrates the effectiveness of the developed theory by means of determining the $\mc{L}_2$-gain of the considered LPV systems \TR{in two academic examples}. All the computations discussed in this section have been executed using \matlab on a MacBook Pro (2020) with Intel Core i5 processor and solved using YALMIP \cite{YALMIP} with the MOSEK solver \cite{MOSEK}.
\subsection{Example I} \label{example:1}
In this example, we compare our direct data-driven \TR{analysis }approaches with a model-based and an indirect data-driven\footnote{With indirect data-driven method we mean to first perform system identification that results in a model on which model-based analysis \TR{is performed}.} method. Consider an LPV system \TR{described in the form of} \eqref{eq:sys} with $\dny=\dnu=\dnp=1$, $\dna=\dnb=3$ and $a,b$  as
\begin{align*}
    \begin{bmatrix} a_1^{[0]} & a_2^{[0]} & a_3^{[0]} \\ a_1^{[1]} & a_2^{[1]} & a_3^{[1]} \end{bmatrix} & =%
    \begin{bmatrix} 0.0826    & 0.1491    & -0.1196   \\ 0.0311    & 0.0570    & -0.0650   \end{bmatrix}, \\
    \begin{bmatrix} b_1^{[0]} & b_2^{[0]} & b_3^{[0]} \\ b_1^{[1]} & b_2^{[1]} & b_3^{[1]} \end{bmatrix} & =%
    \begin{bmatrix} 0.5007    & -0.5588   & 0.0784    \\ 0.6953    & 1.8192    & -1.7192   \end{bmatrix}.
\end{align*}
Moreover, we define $\mb{P}$ as $[-0.1,\> 0.1]$ with no rate bounds on $p$, i.e., $\ms{P}_{[1,L]}=[-0.1,\> 0.1]^L$. For our data-driven dissipativity analysis, we want to determine the $\mc{L}_2$-gain of this system for a horizon of 7. Hence, we choose $\ell=\max\{\dna,\dnb\}=3$ and set $L$ to 10. {As the system is \TR{\emph{single-input single output}} (SISO), we have that $\dnx=\max\{\dna,\dnb\}=\TR{n_\mathrm{r}}= 3$ \cite{Toth2011_SSrealizationTCST}.} By exciting the system with an input $u_k\sim\mc{N}(0,1)$ and scheduling $p_k\sim\mc{U}(\mb{P})$, we have found via Theorem~\ref{thm:PE} that a data dictionary $\dataset$ with $N=42$ is persistently exciting of order $(L,\dnx)$. The data dictionary is shown in Fig.~\ref{fig:datadictionary}.
\begin{figure}
\centering
\includegraphics[scale=1]{figures/datadictionary} \vspace{-6mm}
\caption{Data dictionary for Example I.}\label{fig:datadictionary} \vspace{-4mm}
\end{figure}
{We compute an upper bound $\gamma$ on the $(L-\ell)$-horizon $\mc{L}_2$-gain with the model-based approach that is worked out in Appendix~\ref{app:modelbasedLdissip}. We verify \eqref{eq:LMImodelbased} by using a polytopic description of $\ms{P}_{[1,L]}$ and solving \eqref{eq:LMImodelbased} on the vertices of $\ms{P}_{[1,L]}$, while minimizing the upper bound $\gamma$. Solving this problem yields  $\gamma=1.362$.} 
Next, we employ the \lpvcore toolbox\footnote{See \cite{BoefCoxToth2021} and \texttt{lpvcore.net} for \TR{this} open-source \matlab toolbox.} to identify the LPV system using $\dataset$ and compute the $(L-\ell)$-horizon $\mc{L}_2$-gain of the resulting identified model. For the identification, we use LPV PEM-SS identification (see \cite{Toth18cAUT, Cox2018})
with the default settings in \lpvcore~\TR{via} the function \texttt{lpvssest}. The {upper bound $\gamma$ on the $(L-\ell)$-horizon} $\mc{L}_2$-gain of the identified \TR{LPV} model is calculated to be 1.375.

Finally, we determine the $(L-\ell)$-horizon $\mc{L}_2$-gain of the system \emph{directly} with $\dataset$ using the results in this paper, {and show that our direct data-driven dissipativity verification methods are competitive with their (indirect) model-based counter-parts.} First, we employ the direct data-driven verification method via the convex hull argument. With the defined range for $p$, we construct the hypercube $\mb{P}^L$  {that defines $\ms{P}_{[1,L]}$}. The resulting hypercube has $n_\mathrm{v}=1024$ vertices. Together with the construction of $F$ using \emph{only} the data dictionary $\dataset$, we can now solve \eqref{eq:prop_LPV_diss_polytopic} on the vertices of $\mb{P}^L$ and minimize the value $\gamma$ that corresponds to an  {upper bound on the $(L-\ell)$-horizon} $\mc{L}_2$-gain of the system. The corresponding SDP yields $\gamma=1.362$,  {which demonstrates the equivalence with the model-based method}. For the direct data-driven verification method using the S-procedure, we again use $\dataset$ to construct the necessary matrices to apply Proposition~\ref{prop:LPV_diss_ellipsoidal}. Note that by the definition of the scheduling region,  {we can equivalently describe $\ms{P}_{[1,L]}$ using \eqref{eq:def_of_MP_ellipsiodal} by choosing $p_\mr{max}=0.2$ and $\breve{p}_{[1,L]}$ as a zero trajectory.} Solving \eqref{eq:prop_LPV_diss_ellipsoidal} yields a value for $\gamma$ of 1.831, representing a bound on the $(L-\ell)$-horizon $\mc{L}_2$-gain of the LPV system. {This results illustrates the aforementioned conservatism of Proposition~\ref{prop:LPV_diss_ellipsoidal}. On the other hand, the computational load of the latter method is significantly smaller, which we will further illustrate in next example.}


\subsection{Example II}\label{ss:example2}
In this example, we analyze $(L-\ell)$-dissipativity for multiple values of $L$ and, with this, demonstrate the relationship between $(L-\ell)$-horizon dissipativity and infinite-horizon dissipativity in terms of the $\mc{L}_2$-gain. This example also illustrates the difference in computational load for the proposed methods. \TR{Consider} a similar LPV system \TR{as in Example \ref{example:1}}, now with $\dna=\dnb=2$. \TR{In this case, the coefficients $a$ and $b$} are %defined as
\begin{align*}
    \begin{bmatrix} a_1^{[0]} & a_2^{[0]} \\ a_1^{[1]} & a_2^{[1]} \end{bmatrix} & =%
    \begin{bmatrix} -0.00569  &  0.0706   \\ 0.1137    & -0.0210   \end{bmatrix} \\
    \begin{bmatrix} b_1^{[0]} & b_2^{[0]} \\ b_1^{[1]} & b_2^{[1]} \end{bmatrix} & =%
    \begin{bmatrix}   1.3735  & -0.2941   \\ 0.1254    &  0.6617   \end{bmatrix}.
\end{align*}
The scheduling set \TR{$\mb{P}$} is defined as $[-0.2,\>0.2]$. We also impose a rate bound on the scheduling signal: %$\mb{D}= [-0.1,\> 0.1]$, i.e., 
$p_{k}-p_{k-1}\in[-0.1,\> 0.1]=\TR{\mb{D}}$. In this example, we determine upper bounds for the $(L-\ell)$-horizon $\mc{L}_2$-gain of this system using our direct dissipativity analysis approach\TR{es} for $L=3,\dots,8$ and compare these results to the upper bound for the infinite-horizon $\mc{L}_2$-gain, which is obtained \TR{via model-based analysis} using the \lpvcore toolbox.
\begin{figure}
\centering
\includegraphics[scale=1]{figures/datadictionary2}  \vspace{-6mm}
\caption{Data dictionary for Example II.}\label{fig:datadictionary2}  \vspace{-4mm}
\end{figure}
We choose $\ell=2$ and generate our data-dictionary $\dataset$, with $N=33$, by exciting the system with an input  {$u_k\sim\mc{N}(0,1)$} and scheduling $p_k$, {whose samples are drawn from an i.i.d. uniform distribution \TR{on $\mathbb{P}$} that is truncated to satisfy \TR{the rate bound}. The resulting $\dataset$} is PE {of order $(L\le8,\dnx=2)$}. The generated data dictionary is shown in Fig.~\ref{fig:datadictionary2}.

We use the model of the LPV system to compute {an upper bound $\gamma$ on the (infinite-horizon)} $\mc{L}_2$-gain with \lpvcore~\TR{using a model-based approach}, which yields $\gamma=1.667$. {When $\mb{D}$ is not considered in the analysis, we obtain $\gamma=1.754$}. With $\dataset$, we identify two LPV-SS representations using the same settings as in Example I, one considering $\mb{P}$ and $\mb{D}$ and one considering only $\mb{P}$. Calculating the upper bound $\gamma$ on the (infinite-horizon) $\mc{L}_2$-gain of the two identified models yields $\gamma=1.660$ for the former and $\gamma=1.754$ for the latter. 

We will now compare these (indirect) model-based analysis results to the results of direct data-driven $(L-\ell)$-dissipativity analysis for $L=3,\dots,8$. We will apply all four verification methods that we discuss in this paper to this example, i.e., we verify \eqref{eq:thm_LPV_diss} by means of
\begin{itemize}
	\item A polytopic description of $\mb{P}$, using the convex hull argument in Proposition~\ref{prop:LPV_diss_polytopic} (\TR{denoted as} CHA),
	\item A quadratic description of $\mb{P}$, using the S-procedure in Proposition~\ref{prop:LPV_diss_ellipsoidal} (\TR{denoted as} SP).
\end{itemize}
Furthermore, we reduce the conservatism introduced in the aforementioned methods by
\begin{itemize}
	\item Including \TR{the rate bound $\mb{D}$ on $p$ in} the CHA method, as discussed in Section~\ref{sss:conservatism} (\TR{denoted as} CHAr),
	\item Considering \TR{$p$-}dependent multipliers and, as discussed in Remark~\ref{rem:PVmulti}, \TR{solve SP with a sampling-based approach}  (\TR{denoted as} SDM). \TR{For this computation, we generate} $N_\mr{s}=1000$ sample trajectories from $\ms{P}_{[1,L]}$. 
\end{itemize}


\TR{With the four methods,} computing \TR{an} upper bound $\gamma$ on the $(L-\ell)$-horizon $\mc{L}_2$-gain of the system using $\dataset$ \TR{gives the results depicted in  Fig.~\ref{fig:result_multL} for $L=3,\dots,8$}. 
\begin{figure}
\centering
\begin{subfigure}[b]{\linewidth}
\centering
\includegraphics[scale=1]{figures/resultsex2}
\caption{}\label{fig:result_multL:con}
\end{subfigure}
\begin{subfigure}[b]{\linewidth}
\includegraphics[scale=1]{figures/resultsex2_rates}
\caption{}\label{fig:result_multL:ncon}
\end{subfigure}
\caption{Model-based and indirect data-driven dissipativity analysis versus direct data-driven $(L-\ell)$-dissipativity analysis of an LPV system for an increasing horizon of $L$ . Plot (a) shows the results \TR{when} the rate bounds on $p$ are not included in the analysis, while (b) shows the results \TR{when} the \TR{rate-bounds are included.}}\label{fig:result_multL}
\end{figure}
\TR{The plot in} Fig.~\ref{fig:result_multL:con} shows \TR{the results of the CHA and SP \TR{data-driven} methods} %the former two methods (CHA and SP) 
together with the model-based and indirect dissipativity analysis methods \TR{when} $\mb{D}$ is \emph{not} considered in the analysis. The plot in Fig.~\ref{fig:result_multL:ncon} depicts the results of the \TR{CHAr and SDM direct data-driven dissipativity analysis methods}
%latter two direct data-driven dissipativity analysis methods (CHAr and SDM), 
together with the model-based and indirect dissipativity analysis methods \TR{when $\mb{D}$ is considered in the analysis}. We want to highlight again that we analyze here $(L-\ell)$-dissipativity.
The plots show that for this simple system, the finite-horizon direct data-driven dissipativity analysis results converge relatively quick\TR{ly} to the infinite-horizon results \TR{of the} model-based approaches. As expected, Fig.~\ref{fig:result_multL:con} demonstrates that the dissipativity verification via Proposition~\ref{prop:LPV_diss_ellipsoidal} is slightly more conservative, i.e., the computed upper bound on the $\mathcal{L}_2$-gain is slightly larger. However, this conservatism is negated when the multiplier is considered to be scheduling dependent, as shown in Fig.~\ref{fig:result_multL:ncon}. Hence, our methods give comparable results to the model-based approach when the horizon is chosen sufficiently large \TR{even if} we \emph{only} make use of the information that is encoded in the data dictionary $\dataset$. 

As highlighted throughout the paper, e.g., Remark~\ref{rem:manylmis}~and~\ref{rem:PVmulti}, the computational complexity of the CHA(r) method(s) can be significant, especially when compared to the method based on the S-procedure or the sampling-based method. This can be observed in Table~\ref{tab:comp}, where we have listed the computation time for every method for an increasing $L$ \chris{For comparison, we also included the computation time for model-based $L$-dissipativity verification for the increasing horizon}.
\begin{table}
\centering
\caption{Computation time (in seconds) for verifying $(L-\ell)$-dissipativity for $L=3,\dots,8$ using the methods discussed in this paper\TR{:} Convex Hull Argument (CHA), S-Procedure (SP), CHA that incorporates the rate bounds (CHAr) and sampling-based approach with Scheduling Dependent Multipliers (SDM). \chris{The last column contains the model-based $(L-\ell)$-dissipativity verification approach (MBA) for comparison.}}\label{tab:comp}
\begin{tabular}{c|rcrcc}
$L$ & \multicolumn{1}{c}{CHA}    & SP     & \multicolumn{1}{c}{CHAr}  & SDM & MBA  \\ \hline
3   & 0.901 %4 
& 0.5147 & 1.3\TR{1} %05 
& 12.15 & 0.490 \\
4   & 1.637  & 0.5045 & 3.28  & 16.99 & 0.417 \\
5   & 3.437  & 0.4525 & 8.1\TR{9} %85 
& 11.02 & 0.550\\
6   & 7.687  & 0.4301 & 23.67 & 12.16 & 0.339\\
7   & 16.85\TR{0}  & 0.4291 & 57.83 & 10.58 & 0.496\\
8   & 35.67\TR{0}  & 0.4598 & 152.0\TR{0} & 13.20  & 0.475\\
\end{tabular}  \vspace{-4mm}
\end{table}
Due to the exponential growth of the number of constraints and decision variables along $L$ of the dissipativity verification problem using the CHA(r) methods, the computation time significantly increases for relatively small horizons, even for simple systems. This makes the approach based on the S-procedure much more suitable for larger systems/horizons at the price of conservatism. 



