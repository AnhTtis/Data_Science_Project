In this section, we provide an equivalent characterization of dissipativity analysis of LPV systems based only on measured data.
First, in Section~\ref{subsec:datadrivendissipativity_finite}, we refine the notion of  $L$-dissipativity for LPV systems.
Section~\ref{subsec:datadrivendissipativity_LPV} then provides a necessary and sufficient condition for $L$-dissipativity of LPV systems based on data.


\subsection{Finite-horizon dissipativity}\label{subsec:datadrivendissipativity_finite}
%
As we consider finite amount of data, we will make use of %the finite-horizon form of dissipativity in terms of 
Definition~\ref{def:L_diss}, requiring to satisfy the dissipation inequality over all time intervals up to a finite time horizon $L$. The following result, based on \cite[Proposition 1]{romer2019one}, shows that it suffices to verify the dissipation inequality for only $L$.
\begin{prop}[$L$-dissipativity of LPV systems]\label{prop:L_diss}
The system defined by \eqref{eq:sys} is $L$-dissipative (see Definition \ref{def:L_diss}) w.r.t.~the supply rate~\eqref{eq:supplyrate} if
%
\begin{align}\label{eq:def_L_diss2}
\sum_{k=1}^{L}\begin{bmatrix}u_k\\y_k\end{bmatrix}^{\!\top} \!\!
\Pi\begin{bmatrix}u_k\\y_k\end{bmatrix}\geq0
\end{align}
%
for all trajectories $\{u_k,y_k\}_{k=1}^{L}\in\Bfint{\mr{IO}}{[1,L]}$ with $w_0=0$ (zero initial condition). 
\end{prop}
\begin{proof} 
As for any given ${p}_{[1,L]}\in\ms{P}_{[1,L]}$, the input-output behavior for LPV systems is \emph{linear}, therefore -- with the assumption of zero initial conditions -- the proof in \cite[Prop~1]{romer2019one} can be used to prove this result. 
\end{proof}
%
Thus, to determine whether the LPV system~\eqref{eq:sys} is dissipative, we need to verify \eqref{eq:def_L_diss2} for all of its trajectories $(u_{[1,L]},y_{[1,L]})\in\Bfint{\mr{IO}}{[1,L]}$.
As we show next, the data-driven representation of \eqref{eq:sys}, which exists based on the Fundamental Lemma for LPV systems (see Theorem~\ref{cor:LPVDDR}), allows us to translate this condition into a test. This test only depends on input-output data and the set $\mathbb{P}$.

\begin{remark}[Finite- vs. infinite-horizon dissipativity]\label{rem:fin_vs_inf}
    In general, $L$-dissipativity is a more `optimistic' property than the infinite-horizon dissipativity property, given in Definition~\ref{def:dissipativity-hillmoylan}. In fact, 
for the case of single-input single-output LTI systems, it is shown in
\cite{boettcher2000toeplitz, tu2018approximation} that $L$-dissipativity with $\Pi$ defined with $(Q,S,R)=(\gamma^2, 0, -1)$ \emph{implies} (infinite-horizon) dissipativity with $\Pi$ defined as $(Q,S,R)=((\gamma+\epsilon)^2, 0, -1)$, where $\epsilon$ decays with $\tfrac{1}{L^2}$.
A similar statement holds true for more general dissipativity properties under suitable assumptions on a transformed LTI system and even extends to \emph{integral quadratic constraints}
(IQC) based analysis as shown in~\cite[Thm.~20]{koch2021determining}.
We conjecture that comparable bounds also apply to the relation between finite- and infinite-horizon dissipativity in the LPV case, which would result in data-driven dissipativity analysis that guarantees bounds on the performance of an unknown LPV system. In Section~\ref{ss:example2}, we illustrate this decay of $\epsilon$ and thus convergence towards the infinite-horizon dissipativity property by means of an example.
\end{remark}
%
\begin{remark}[Model-based $L$-dissipativity]
    It is rather trivial to work out a method to verify $L$-dissipativity in a model-based setting. For completeness, we give the derivation for $L$-dissipativity verification for LPV systems in Appendix~\ref{app:modelbasedLdissip}.
\end{remark}




\subsection{Data-driven dissipativity of LPV systems}\label{subsec:datadrivendissipativity_LPV}

Next, we derive an equivalent characterization of $L$-dissipativity for the unknown LPV system~\eqref{eq:sys} based on input-scheduling-output data. Given the \emph{data-dictionary} $\dataset = \{u_k, p_k, y_k\}_{k=1}^N\in\Bfint{}{[1,N]}$ measured from the system \eqref{eq:sys}, we analyze $L$-dissipativity for length $L$ trajectories $(\bar{u}_{[1,L]}, \bar{p}_{[1,L]}, \bar{y}_{[1,L]})\in\Bfint{}{[1,L]}$ of this system.



As in \cite{romer2019one}, we define the matrix $V_\ell\in\mathbb{R}^{(\dnu+\dny)\ell\times (\dnu+\dny)L}$ for some $\ell\in\mathbb{N}$ as 
\begin{equation}
	V_\ell = \begin{bmatrix}
 		I_{\ell\dnu} & 0_{\ell\dnu \times \dnu(L-\ell)} & 0_{\ell\dnu \times \dny \ell} & 0_{\ell\dnu \times \dny(L-\ell)} \\
		0_{\ell\dny  \times \ell\dnu} & 0_{\ell\dny \times \dnu(L-\ell)} & I_{\ell\dny} & 0_{\ell\dny \times \dny(L-\ell)}
 	\end{bmatrix}
\end{equation}
such that $V_\ell\begin{bmatrix} \bar{\vect{u}}_{L} \\ \bar{\vect{y}}_{L} \end{bmatrix}=0$ if and only if $\bar{u}_1=\dots=\bar{u}_{\ell}=0$ and $\bar{y}_1=\dots=\bar{y}_{\ell}=0$.
%
%
Moreover, we define
%
\begin{subequations}\label{eq:Pi_H_F_p_def} 
\begin{align}
	\Pi_L  &= \begin{bmatrix}
		I_L\otimes Q & I_L\otimes S \\
		I_L\otimes S^\top & I_L\otimes R
	\end{bmatrix}, %\in \mathbb{R}^{L(\dnu+\dny) \times L(\dnu+\dny)}, 
	\\
	\Pi_H  &= \begin{bmatrix}
		\mc{H}_L(u_{[1,N]}) \\ \mc{H}_L(y_{[1,N]})
	\end{bmatrix}^{\!\top}\!\! \Pi_L \begin{bmatrix}
		\mc{H}_L(u_{[1,N]}) \\ \mc{H}_L(y_{[1,N]})
	\end{bmatrix}, %\in \mathbb{R}^{(N-L+1)\times(N-L+1)},  
	\\
	F(\bar{p}_{[1,L]})  &= \begin{bmatrix}
		F_1 \\ F_2(\bar{p}_{[1,L]}) 
	\end{bmatrix} \\\nonumber
	&= \begin{bmatrix} 
		V_\ell \begin{bmatrix} \mc{H}_L(u_{[1,N]}) \\ \mc{H}_L(y_{[1,N]}) \end{bmatrix} \\
		\mc{H}_L(u_{[1,N]}^{\mt{p}})-\bar{\mathcal{P}}^{\dnu}\mc{H}_L(u_{[1,N]}) \vphantom{\Big)}\\
		\mc{H}_L(y_{[1,N]}^{\mt{p}})-\bar{\mathcal{P}}^{\dny}\mc{H}_L(y_{[1,N]}) \vphantom{\Big)}
	\end{bmatrix},
\end{align}
\end{subequations}
%
where $F(\bar{p}_{[1,L]})\in\mathbb{R}^{(\dnu+\dny)(\ell+\dnp L)\times (N-L+1)}$. Note that $F_2(\bar{p}_{[1,L]})$ contains both the scheduling trajectory ${p}_{[1,N]}$ of the \emph{data-dictionary} $\dataset$, \emph{and} the scheduling trajectory $\bar{p}_{[1,L]}$ that is associated with any length $L$ system trajectory in $\Bfint{}{[1,L]}$, where $\bar{p}_{[1,L]}$ enters via $\mc{P}^\bullet$.
This leads us to the following result.
%
\begin{theorem}[$(L-\ell)$-dissipativity of LPV systems]\label{thm:LPV_diss}
The following statements hold:
%
\begin{enumerate}[label=(\roman*)]
\item \label{thm:LPV_diss:1} If~\eqref{eq:sys} is $L$-dissipative, then, for any $\dnx\leq\ell<L$ and any $\bar{p}_{[1,L]}\in\ms{P}_{[1,L]}$, it holds that
%
\begin{align}\label{eq:thm_LPV_diss}
g^\top \Pi_H g \geq 0\quad\forall g\in\mathbb{R}^{N-L+1}:\>F(\bar{p}_{[1,L]})g=0.
\end{align}
%
\item \label{thm:LPV_diss:2} Conversely, suppose the system of the form~\eqref{eq:sys}  generates a PE data set $\dataset$.
If there exists $\ell<L$ such that~\eqref{eq:thm_LPV_diss} holds for all $\bar{p}_{[1,L]}\in\ms{P}_{[1,L]}$, then~\eqref{eq:sys} is $(L-\ell)$-dissipative.
\end{enumerate}
\end{theorem}
\begin{proof}
We first prove item~\ref{thm:LPV_diss:1}. 
Take an arbitrary $g$ satisfying $F(\bar{p}_{[1,L]})g=0$.
Then, Corollary~\ref{cor:LPVDDR} implies that 
%
\begin{equation}\label{eq:thm1:traj}
	\begin{bmatrix} \bar{\vect{u}}_L \\ \bar{\vect{y}}_L \end{bmatrix} = \begin{bmatrix} \mc{H}_L(u_{[1,N]}) \\ \mc{H}_L(y_{[1,N]})\end{bmatrix} g 
\end{equation}
%
is a trajectory of~\eqref{eq:sys}. Moreover, due to $F_1g=0$, this trajectory satisfies $\bar{u}_1=\dots=\bar{u}_{\ell}=0$, $\bar{y}_1=\dots=\bar{y}_{\ell}=0$ and hence, since $\ell\geq \dnx$, it has zero initial conditions.
Furthermore, since~\eqref{eq:sys} is $L$-dissipative, the trajectory also necessarily satisfies
%
\begin{align}\label{eq:L_diss_stacked}
0 \leq \begin{bmatrix} \bar{\vect{u}}_L\\ \bar{\vect{y}}_L\end{bmatrix}^{\!\top} \!\! \Pi_L \begin{bmatrix} \bar{\vect{u}}_L\\ \bar{\vect{y}}_L \end{bmatrix}
=g^\top \Pi_H g,
\end{align}
%
i.e.,~\eqref{eq:thm_LPV_diss} holds.

We now prove item~\ref{thm:LPV_diss:2}. 
Let $(\bar{u}_{[\ell+1,L]},\bar{p}_{[\ell+1,L]},\bar{y}_{[\ell+1,L]})$ be an arbitrary trajectory in $\Bfint{}{[\ell+1,L]}$ with zero initial conditions.
We extend this trajectory by $\bar{u}_1=\dots=\bar{u}_{\ell}=0$ and $\bar{y}_1=\dots=\bar{y}_{\ell}=0$ such that $(\bar{u}_{[1,L]},\bar{p}_{[1,L]},\bar{y}_{[1,L]})$ for some $\bar{p}_{[1,\ell]}$ is also a trajectory in $\Bfint{}{[\ell+1,L]}$. %of~\eqref{eq:sys}.
Using Corollary~\ref{cor:LPVDDR}, there exists a vector $g$ such that both \eqref{eq:thm1:traj} and $F(\bar{p}_{[1,L]})=0$ hold.
According to~\eqref{eq:thm_LPV_diss}, we then have $g^\top\Pi_H g\geq0$, i.e.,
%
\begin{align*}
0 \leq g^\top\Pi_H g=\begin{bmatrix} \bar{\vect{u}}_L\\ \bar{\vect{y}}_L \end{bmatrix}^\top\Pi_L \begin{bmatrix} \bar{\vect{u}}_L\\ \bar{\vect{y}}_L \end{bmatrix}
=\sum_{k=\ell+1}^{L}\begin{bmatrix}\bar{u}_k\\\bar{y}_k\end{bmatrix}^\top
\Pi
\begin{bmatrix}\bar{u}_k\\\bar{y}_k\end{bmatrix}.
\end{align*}
%
Since $\{\bar{u}_k,\bar{y}_k\}_{k=\ell+1}^{L}$ was arbitrary, this implies that~\eqref{eq:sys} is $(L-\ell)$-dissipative.
\end{proof}



Theorem~\ref{thm:LPV_diss} shows that~\eqref{eq:thm_LPV_diss} is a necessary and sufficient condition for $L$-dissipativity of the LPV system \eqref{eq:sys}.
Intuitively, the dissipation inequality $g^\top \Pi_Hg\geq0$ needs to hold for all vectors $g$ corresponding to zero initial conditions and the affine constraints in the third and fourth block row of~\eqref{eq:thm_FL}, i.e., satisfying $F(\bar{p}_{[1,L]})g=0$.
The proof follows similar arguments as the corresponding result for LTI systems~\cite[Theorem 2]{romer2019one}.

Unfortunately, Condition~\eqref{eq:thm_LPV_diss} not only involves $\dataset$, but also {arbitrary admissible scheduling trajectories $\bar{p}_{[1,L]}\in\ms{P}_{[1,L]}$,} 
meaning that it is a semi-infinite feasibility problem, i.e., it is intractable in its current form.
In Section~\ref{sec:computational}, we show how a tractable reformulation can be derived by using Finsler's Lemma {and structural assumptions on the definition of $\ms{P}_{[1,L]}$}.

