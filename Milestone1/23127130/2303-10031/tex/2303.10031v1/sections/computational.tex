\subsection{Main concept}
In this section, \TR{using Theorem \ref{thm:LPV_diss},} we derive tractable conditions for the verification of dissipativity properties of LPV systems based on data.
To this end, we introduce the well-known Finsler's Lemma (see~\cite{boyd1994linear}),
\TR{which will prove to be an essential building block of our proposed approaches.}
%, which is an essential tool that will be used throughout the paper.
%
\begin{lemma}[{Finsler's Lemma \cite{boyd1994linear}}]\label{lem:Finsler}
Let $A\in\mathbb{R}^{n\times n}$, $B\in\mathbb{R}^{q\times n}$.
Then, the following statements are equivalent:
%
\begin{enumerate}[label={(\roman*)}]
\item $x^\top Ax\geq0$ for all $x$ satisfying $Bx=0$.

\item $(B^\perp)^\top AB^\perp\succeq0$

\item There exists a $\mu\in\mathbb{R}$ such that $A+\mu B^\top B\succeq0$.

\item There is an $X\in\mathbb{R}^{n\times q}$ such that $XB+B^\top X^\top - A \preceq 0$.
\end{enumerate}
\end{lemma}
%
According to Theorem~\ref{thm:LPV_diss}, analyzing dissipativity amounts to verifying Condition~\eqref{eq:thm_LPV_diss} for all \chris{admissible} scheduling signals $\bar{p}_{[1,L]}\in\ms{P}_{[1,L]}$. Based on ``(i) $\Leftrightarrow$ (ii)'' in Lemma~\ref{lem:Finsler}, we infer that~\eqref{eq:thm_LPV_diss} holds if and only if
%
\begin{align}\label{eq:dd_Ldiss_Finsler}
\big(F(\bar{p}_{[1,L]})^\perp\big)^\top \Pi_H F(\bar{p}_{[1,L]})^\perp\succeq0.
\end{align}
The same idea has been been employed in the LTI case in \cite{romer2019one}.
%
For the LTI case, this provides a simple and elegant condition to verify dissipativity, only requiring to check positive semi-definiteness of a given data-dependent matrix.
For LPV systems of the form \eqref{eq:sys}, \eqref{eq:dd_Ldiss_Finsler} allows us to check whether the dissipation inequality holds for a \emph{fixed} scheduling trajectory $\bar{p}_{[1,L]}$. However, we need to verify dissipativity as a \emph{system} property for the whole $\ms{P}_{[1,L]}$, which makes, together with the fact that
$F(\bar{p}_{[1,L]})$ depends nonlinearly on $\bar{p}_{[1,L]}$, the verification of~\eqref{eq:dd_Ldiss_Finsler} intractable.

Therefore, in the remainder of the section, we exploit items (iii) and (iv) of Lemma~\ref{lem:Finsler} in order to derive computational procedures for the verification of~\eqref{eq:thm_LPV_diss}, which are tailored to the assumed \chris{definition of $\ms{P}_{[1,L]}$}.
First, in Section~\ref{subsec:computational_polytopic}, we employ item (iv) to address \TR{the case when} $\mathbb{P}$ \TR{is a convex, polytopic set, by describing} $\ms{P}_{[1,L]}$ based on a convex hull argument.
Next, in Section~\ref{subsec:computational_S_procedure}, we use the S-procedure to handle admissible scheduling sets $\ms{P}_{[1,L]}$, defined by quadratic inequalities.

\subsection{Dissipativity verification via a convex hull argument}\label{subsec:computational_polytopic}
%
\subsubsection{Core concept} We now provide a computational procedure to verify~\eqref{eq:thm_LPV_diss} for any $\bar{p}_{[1,L]}\in\ms{P}_{[1,L]}$, where $\ms{P}_{[1,L]}:=\mb{P}^L$ and $\mathbb{P}$ is a
convex polytope \TR{with finite many vertices}.
%
\begin{assumption}[\chris{Polytopic description of $\mb{P}$}]\label{ass:polytopic}
The set $\mathbb{P}\subset\mb{R}^{\dnp}$ is a convex polytope, {generated by a finite \TR{number} of vertices:  %$\msf{p}_i$, $i=1,\dots,n_\mathrm{v}$, i.e., 
$\mathbb{P}=\mathrm{co}(\{\msf{p}_i\}_{i=1}^{n_\mathrm{v}})$, where $\mathrm{co}$ denotes the convex hull.} \hfill$\square$
\end{assumption}
%
As the scheduling variable varies in $\mb{P}$, i.e., \TR{$\bar{p}_k\in\mb{P},$ $\forall k \in\mathbb{Z}$}, every trajectory \TR{$\bar{p}_{[1,L]}$} is confined in the space \[\ms{P}_{[1,L]}=\mb{P}\times\dots\times\mb{P}=\mb{P}^L\]
which is generated with $n_\mathrm{v}^L$ vertices \TR{that} are all permutations of the original vertices of $\mb{P}$, i.e., $\ms{P}_{[1,L]}= \mathrm{co}( \mathrm{perm}_L(\{\msf{p}_i\}_{i=1}^{n_\mathrm{v}}) )$.
We now apply ``(i) $\Leftrightarrow$ (iv)'' in Lemma~\ref{lem:Finsler} to derive a tailored condition for verifying~\eqref{eq:thm_LPV_diss} via a convex hull argument, i.e., by means of Assumption~\ref{ass:polytopic}.
%
\begin{prop}[\chris{$L$-dissipativity via convex hull argument}]\label{prop:LPV_diss_polytopic}
Suppose Assumption~\ref{ass:polytopic} holds, and let $\{\bar{\msf{p}}_i\}_{i=1}^{n_\mathrm{v}^L}$ denote the vertices of $\mb{P}^L$.
Then,~\eqref{eq:thm_LPV_diss} holds if and only if there exist matrices $X_i\in\mathbb{R}^{(N-L+1)\times(N-L+1)}$, $i=1,\dots,n_\mathrm{v}^L$, satisfying
% 
\begin{align}\label{eq:prop_LPV_diss_polytopic}
	X_iF(\bar{\msf{p}}_i)+F(\bar{\msf{p}}_i)^\top X_i^\top -\Pi_H\preceq0.
\end{align}
%
for all $i=1,\dots,n_\mathrm{v}^L$.
\end{prop}
\begin{proof}
For any given scheduling trajectory $\bar{p}_{[1,L]}\in\ms{P}_{[1,L]}$, Lemma~\ref{lem:Finsler} implies that~\eqref{eq:thm_LPV_diss} holds if and only if there exists a matrix $X(\bar{p}_{[1,L]})\in\mathbb{R}^{(N-L+1)\times(N-L+1)}$ associated with $\bar{p}_{[1,L]}$ such that
%
\begin{align}\label{eq:dd_Ldiss_Finsler1}
	X(\bar{p}_{[1,L]})F(\bar{p}_{[1,L]})+F(\bar{p}_{[1,L]})^\top X(\bar{p}_{[1,L]})^\top\!\! -\Pi_H \preceq 0.
\end{align}
%
\TR{As $F(\bar{p}_{[1,L]})$ is linear in $\bar{p}_{[1,L]}$,} using Assumption~\ref{ass:polytopic} and multi-convexity, there exists a matrix-valued function $X:\mathbb{P}^{[1,L]} \to\mathbb{R}^{(N-L+1)\times(N-L+1)}$ satisfying~\eqref{eq:dd_Ldiss_Finsler1} for all $\bar{p}_{[1,L]}\in\ms{P}_{[1,L]}$ if and only if there exist a matrix $X_i\in\mathbb{R}^{(N-L+1)\times(N-L+1)}$ for every vertex $\bar{\msf{p}}_i$ satisfying~\eqref{eq:prop_LPV_diss_polytopic}. %, which shows the desired statement.
\end{proof}


Proposition~\ref{prop:LPV_diss_polytopic} provides a simple computational approach to verify condition~\eqref{eq:thm_LPV_diss} in case the scheduling variable is varying in a convex polytope (Assumption~\ref{ass:polytopic}).
Thus, we have reduced the \TR{condition of} data-driven dissipativity %characterization 
in Theorem~\ref{thm:LPV_diss} %equivalently 
to the existence of matrices $X_i$ such that~\eqref{eq:prop_LPV_diss_polytopic} holds, i.e., to an SDP subject to LMI constraints. 
 
\subsubsection{Reducing conservatism}\label{sss:conservatism}
The analysis in Proposition~\ref{prop:LPV_diss_polytopic} allows for maximal \emph{variation} of the scheduling variable, i.e., $p_k-p_{k-1}$ is only limited to remain inside $\mb{P}$. This can result in conservative conclusions, e.g., on the $\mc{L}_2$-gain of the LPV system, as the analysis also considers (possibly non-existent) fast variations of the scheduling. We can reduce this introduced conservatism by including \emph{rate bounds} on $p$, %i.e., bounds on the variation of $p$, 
by defining the {admissible scheduling set as $\ms{P}_{[1,L]}:=\mb{V}_\TR{L}$, where %
%
\begin{equation}
    \mb{V}_\TR{L}= \{ p_{[1,L]}\in\mb{P}^L \mid p_k-p_{k-1}\in\mb{D}, \ \forall k = 2, \dots, L \},
\end{equation}
%
where $\mb{P}$ satisfies Assumption~\ref{ass:polytopic}} and $\mb{D}$ is a convex polytope that defines the rate bound on $p$. {Note that $\mb{V}_\TR{L}$ is again a convex polytope.} By verifying \eqref{eq:prop_LPV_diss_polytopic} on the vertices of $\mb{V}_\TR{L}$, the rate bounds on $p$ are taken into account, which reduces the conservatism in the dissipativity analysis. 
%
\begin{remark}[Computational complexity vs. conservatism]\label{rem:manylmis}
	Verifying dissipativity via \eqref{eq:prop_LPV_diss_polytopic} \emph{without} incorporation of the rate bounds on $p$ requires solving an SDP with $n_\mathrm{v}^L$ LMI constraints. This number possibly further increases when including rate bounds, which increases the computational complexity exponentially for larger $L$. \TR{At the price of} conservatism, the number of decision variables can be reduced by choosing $X_i=X$. 
\end{remark}
{We will see in Section~\ref{sec:examples} that Proposition~\ref{prop:LPV_diss_polytopic} is readily applicable to small-scale LPV systems and it provides tight conditions for verifying $L$-dissipativity from data.}

\subsection{Dissipativity verification via the S-procedure}\label{subsec:computational_S_procedure}
In this section, we assume that $\bar{p}_{[1,L]}$ is varying in a space that is described by a quadratic inequality instead of a convex polytope, i.e., we assume that $\ms{P}_{[1,L]}$ is defined as follows:
%
\begin{assumption}[{Quadratic description of $\ms{P}_{[1,L]}$}]\label{ass:ellipsoidal}
The set $\ms{P}_{[1,L]}$ is described by
%
\begin{equation}\label{eq:ass_ellipsoidal}
	{\ms{P}_{[1,L]}}=\left\{\bar{p}_{[1,L]}\in\mb{P}^L \Bigm| \vphantom{ \begin{bmatrix} \hat{\mc{P}}^{\dnu+\dny} \\ I \end{bmatrix}^\top} %\right. \\
	%\left. 
	\begin{bmatrix} \bar{\mc{P}}^{\dnu,\dny} \\ I \end{bmatrix}^{\!\top}\!\! M_P \begin{bmatrix} \bar{\mc{P}}^{\dnu,\dny} \\ I \end{bmatrix}\succeq 0 \right\}, \vspace{-0mm}
\end{equation}
with $\bar{\mc{P}}^{\dnu,\dny}=\begin{bmatrix} \bar{\mathcal{P}}^{\dnu} & 0 \\ 0 & \bar{\mathcal{P}}^{\dny}\end{bmatrix}$, $\bar{\mathcal{P}}^{n_\bullet}=\bar{p}_{[1,L]}\bkron I_{n_\bullet}$.\hfill$\square$
\end{assumption}
%
For example, suppose the scheduling variable $\bar{p}_{[1,L]}$ always varies inside a spherical tube around a nominal trajectory $\breve{p}_{[1,L]}$, i.e., $\bar{p}_{[1,L]}$ satisfies  
\begin{align}\label{eq:p_norm_bound}
\lVert \bar{p}_k-\breve{p}_{k}\rVert_W\leq p_{\max}
\end{align}
%
for all $k=1,\dots,L$ with some $p_{\max}>0$.
{For $W=2$,} this can be equivalently described with \eqref{eq:ass_ellipsoidal}, where 
%
\begin{align}\label{eq:def_of_MP_ellipsiodal}
	M_P = \begin{bmatrix}
		-I & \breve{\mathcal{P}}^{\dnu,\dny} \\
		(\breve{\mathcal{P}}^{\dnu,\dny})^\top & p_{\max}^2 I -(\breve{\mathcal{P}}^{\dnu,\dny})^\top\breve{\mathcal{P}}^{\dnu,\dny}
	\end{bmatrix},
\end{align}
% 
with $\breve{\mc{P}}^{\dnu,\dny} =\begin{bmatrix}\breve{\mc{P}}^{\dnu}&0\\0&\breve{\mc{P}}^{\dny}\end{bmatrix}$, $\breve{\mc{P}}^{n_\bullet}=\breve{p}_{[1,L]}\bkron I_{n_\bullet}$. Note that with Assumption~\ref{ass:ellipsoidal}, we can describe types of $\ms{P}_{[1,L]}$ that have a spherical or \TR{hyper-rectangular} form.
Next, we employ the S-procedure to derive a tractable reformulation of~\eqref{eq:thm_LPV_diss} for scheduling sets $\ms{P}_{[1,L]}$ that satisfy Assumption~\ref{ass:ellipsoidal}.
To this end, let us write $F(\bar{p}_{[1,L]})$ in~\eqref{eq:Pi_H_F_p_def} as
%
\begin{equation*}
\hspace{-0.5mm}	F(\bar{p}_{[1,L]})\!  =\!\!  %F_3 + F_4 \bar{\mathcal{P}}^{\dnu,\dny} F_5 = \\ 
	\underbrace{\begin{bmatrix} F_1 \\ \mc{H}_L(u_{[1,N]}^{\mt{p}}) \\ \mc{H}_L(y_{[1,N]}^{\mt{p}}) \end{bmatrix}}_{F_3}\!  -\! 	\underbrace{\begin{bmatrix} 0 & 0 \\ I & 0 \\ 0 & I \end{bmatrix}}_{F_4}\!\! \begin{bmatrix} \bar{\mathcal{P}}^{\dnu}\!\!\! & 0 \\ 0 & \!\!\!\bar{\mathcal{P}}^{\dny}\end{bmatrix} \!\! \underbrace{\begin{bmatrix} \mc{H}_L(u_{[1,N]}) \\ \mc{H}_L(y_{[1,N]}) \end{bmatrix}}_{F_5}\!.
\end{equation*}
%
\begin{prop}[$L$-dissipativity via the S-procedure]\label{prop:LPV_diss_ellipsoidal}
Suppose Assumption~\ref{ass:ellipsoidal} holds.
Then,~\eqref{eq:thm_LPV_diss} holds if there exist a $\mu\in\mathbb{R}$ and a $\tau\geq0$ such that
%
\begin{align}\label{eq:prop_LPV_diss_ellipsoidal}
\begin{bmatrix}\mu F_4^\top F_4&\mu F_4^\top F_3\\
\mu F_3^\top F_4&\mu F_3^\top F_3+\Pi_H\end{bmatrix}-\tau \begin{bmatrix}I&0\\0&F_5\end{bmatrix}^{\!\top} \!\!
M_P
\begin{bmatrix}I&0\\0&F_5\end{bmatrix}\succeq0.
\end{align}
\end{prop}


\begin{proof}
Using the equivalence of (i) and (iii) in Lemma~\ref{lem:Finsler},~\eqref{eq:thm_LPV_diss} holds if there exists a $\mu\in\mathbb{R}$ satisfying
%
\begin{align}\label{eq:prop_LPV_diss_ellipsoidal_proof1}
	\Pi_H+\mu F^\top\!(\bar{p}_{[1,L]}) F(\bar{p}_{[1,L]})\succeq0
\end{align}
%
for any $\bar{p}_{[1,L]}\in\ms{P}_{[1,L]}$.%\Bfint{\mathbb{P}}{[1,L]}$.
This inequality %, %in turn, 
is equivalent to
%
\begin{align}\label{eq:prop_LPV_diss_ellipsoidal_proof2}
\begin{bmatrix}\bar{\mathcal{P}}^{\dnu,\dny}F_5\\I\end{bmatrix}^\top
\begin{bmatrix}\mu F_4^\top F_4&\mu F_4^\top F_3\\
\mu F_3^\top F_4&\mu F_3^\top F_3+\Pi_H\end{bmatrix}
\begin{bmatrix}\bar{\mathcal{P}}^{\dnu,\dny}F_5\\I\end{bmatrix}\succeq0.
\end{align}
%
Left- and right-multiplying the inequality in~\eqref{eq:ass_ellipsoidal} by $F_5^\top$ and $F_5$, respectively, we infer that any $\bar{p}_{[1,L]}\in\ms{P}_{[1,L]}$ %\Bfint{\mathbb{P}}{[1,L]}$ 
satisfies
%
\begin{align}\label{eq:prop_LPV_diss_ellipsoidal_proof3}
	\begin{bmatrix} \bar{\mathcal{P}}^{\dnu,\dny}F_5 \\ I \end{bmatrix}^\top
	\begin{bmatrix} I & 0 \\ 0 & F_5 \end{bmatrix}^\top
	M_P
	\begin{bmatrix} I & 0 \\ 0 & F_5 \end{bmatrix}
	\begin{bmatrix} \bar{\mathcal{P}}^{\dnu,\dny}F_5 \\ I \end{bmatrix}\succeq 0.
\end{align}
%
Multiplying~\eqref{eq:prop_LPV_diss_ellipsoidal} with $\begin{bmatrix}(\bar{\mathcal{P}}^{\dnu,\dny}F_5)^\top & I \end{bmatrix}$ from the left and with $\begin{bmatrix}(\bar{\mathcal{P}}^{\dnu,\dny}F_5)^\top & I \end{bmatrix}^\top$ from the right and using~\eqref{eq:prop_LPV_diss_ellipsoidal_proof3}, we obtain~\eqref{eq:prop_LPV_diss_ellipsoidal_proof2}, which concludes the proof.
\end{proof}

Proposition~\ref{prop:LPV_diss_ellipsoidal} provides a sufficient condition for~\eqref{eq:thm_LPV_diss} and hence, by Theorem~\ref{thm:LPV_diss}, for dissipativity of the underlying LPV system.
Verifying feasibility of~\eqref{eq:prop_LPV_diss_ellipsoidal} is an SDP with only two decision variables $\mu$ and $\tau$, and thus it can be implemented efficiently for moderate problem sizes and data lengths (see Section~\ref{sec:examples} for a numerical example).
The proof of Proposition~\ref{prop:LPV_diss_ellipsoidal} relies on a combination of Finsler's Lemma and the S-procedure, %each of which leads to one of the two 
\TR{giving}
multipliers $\mu$ and $\tau$ in \TR{\eqref{eq:prop_LPV_diss_ellipsoidal}.} %the proposition statement.

Note that, in contrast to Proposition~\ref{prop:LPV_diss_polytopic}, Proposition~\ref{prop:LPV_diss_ellipsoidal} only provides a \emph{sufficient} condition for~\eqref{eq:thm_LPV_diss}.
This is because the multiplier $\mu\in\mathbb{R}$, due to Finsler's Lemma (compare Lemma~\ref{lem:Finsler} (iii)), is chosen independently of the scheduling trajectory $\bar{p}_{[1,L]}$.
Therefore, if, e.g., a norm bound similar to~\eqref{eq:p_norm_bound} is available, then translating this bound into a convex polytope as in Assumption~\ref{ass:polytopic} and applying Proposition~\ref{prop:LPV_diss_polytopic} will generally lead to less conservative results than considering the quadratic description in Assumption~\ref{ass:ellipsoidal} and applying Proposition~\ref{prop:LPV_diss_ellipsoidal}.
On the other hand, as we will also demonstrate with a numerical example in Section~\ref{sec:examples}, Proposition~\ref{prop:LPV_diss_ellipsoidal} is from a computational perspective significantly more efficient than Proposition~\ref{prop:LPV_diss_polytopic}.

{Finally, we want to highlight that for all the introduced computational methods, we assume that the `true' admissible scheduling set $\ms{P}_{[1,L]}$ is equivalent with the assumed descriptions (Assumptions~\ref{ass:polytopic}~and~\ref{ass:ellipsoidal}). If the assumed descriptions are in fact \emph{over approximating} $\ms{P}_{[1,L]}$, which can happen when the LPV representation is in fact a \emph{surrogate} model for a nonlinear system, then this again introduces a source of conservatism in the analysis. This issue of conservatism and minimizing it is actively studied in %the theory of 
constructing LPV surrogate models of nonlinear systems \cite{SADEGHZADEH20204737}.}


\begin{remark}[\chris{Sampling-based $L$-dissipativity analysis}]\label{rem:PVmulti}
The described conservatism of Proposition~\ref{prop:LPV_diss_ellipsoidal} can be alleviated by considering \emph{parameter-dependent} multipliers for the application of Finsler's Lemma (Lemma~\ref{lem:Finsler}), similar to Proposition~\ref{prop:LPV_diss_polytopic}.
To be precise,~\eqref{eq:prop_LPV_diss_ellipsoidal_proof1} can be replaced by
%
\begin{align}\label{eq:prop_LPV_diss_ellipsoidal_pv}
	\Pi_H+\mu(\bar{p}_{[1,L]}) F(\bar{p}_{[1,L]})^\top F(\bar{p}_{[1,L]})\succeq0
\end{align}
%
for some $\mu:\ms{P}_{[1,L]}\to\mathbb{R}$.
In fact, by applying Finsler's Lemma point-wise for any parameter trajectory $\bar{p}_{[1,L]}\in\ms{P}_{[1,L]}$, the existence of a mapping $\mu(\cdot)$ such that~\eqref{eq:prop_LPV_diss_ellipsoidal_pv} holds is \emph{equivalent} to~\eqref{eq:thm_LPV_diss}.
In practice, inequality~\eqref{eq:prop_LPV_diss_ellipsoidal_pv} can be checked by drawing $N_\mathrm{s}$ (e.g., uniformly distributed) samples $\hat{p}_{[1,L]}^i$ from $\ms{P}_{[1,L]}$ and verifying the existence of $\mu_i\in\mathbb{R}$, $i=1,\dots,N_\mathrm{s}$ such that
%
\begin{align}
	\Pi_H+\mu_i F(\hat{p}_{[1,L]}^i)^\top F(\hat{p}_{[1,L]}^i)\succeq0
\end{align}
%
for all $i=1,\dots,N_\mathrm{s}$.
For any fixed number $N_\mathrm{s}$, this yields a necessary condition for~\eqref{eq:thm_LPV_diss} and, hence, for $L$-dissipativity.
The resulting dissipativity test is \emph{tight} in the limit $N_\mathrm{s}\to\infty$.
While choosing a dense grid over $\ms{P}_{[1,L]}$, i.e., a large value of $N_\mathrm{s}$, can be prohibitive due to the curse of dimensionality, good results can often be achieved in practice for reasonable values of $N_\mathrm{s}$, see the numerical example in Section~\ref{sec:examples}. 
\end{remark}