To properly analyze properties of a system, some form of \emph{representation} of the underlying behavior is required. However, one may lose information in a modelling process by estimating such a representation from data. In this section, we focus on a purely data-based representation of \eqref{eq:sys} that is used to analyze dissipativity of the system directly from data. %be able to have data-driven dissipativity analysis of the system. 
We can obtain such a representation by characterizing the behavior of \eqref{eq:sys}, restricted to a finite time interval $[1,\,L]$, i.e., $\Bfint{}{[1,\,L]}$.

In \cite{VerhoekTothHaesaertKoch2021}, a data-driven representation for general LPV systems has been derived based on behavioral LPV system theory \cite{Toth11_LPVBehav}. 
Using this fundamental result for LPV systems that have a representation in terms of \eqref{eq:sys}, a simplified data-driven representation has been given in \cite{VerhoekAbbasTothHaesaert2021}. This data-driven representation under shifted affine dependency is obtained by structuring the data in $\dataset$ in terms of \eqref{eq:sys}, such that for valid solution trajectories of the system, i.e., $(\bar{u}_{[1,L]},\bar{p}_{[1,L]},\bar{y}_{[1,L]}) \in \Bfint{}{[1,L]}$, there exists a $g\in\mathbb{R}^{N-L+1}$ that satisfies 
\begin{equation}\label{eq:thm_FL}
	\begin{bmatrix}
		\mc{H}_L\left(u_{[1,N]}\right) \\ 
		\mc{H}_L\left(y_{[1,N]}\right) \\ 
		\mc{H}_L\left(u_{[1,N]}^{\mt{p}}\right) - \bar{\mc{P}}^{\dnu}\mc{H}_L\left(u_{[1,N]}\right) \\ 
		\mc{H}_L\left(y_{[1,N]}^{\mt{p}}\right) - \bar{\mc{P}}^{\dny}\mc{H}_L\left(y_{[1,N]}\right)
	\end{bmatrix}g = \begin{bmatrix}
		\bar{\vect{u}}_{L} \\
		\bar{\vect{y}}_L \\
		0 \\
		0
	\end{bmatrix}.
\end{equation}
Here, for a sequence $\bar{u}_{[1,N]}: [1,N] \rightarrow \mathbb{R}^{n_\mathrm{u}}$,  its vectorization $\mr{vec}\left(\bar{u}_{[1,N]}\right)$ is defined as 
%
\(
	\bar{\vect{u}}_{\,[1,N]} = \begin{bmatrix} \bar{u}_1^\top & \cdots & \bar{u}_{N}^\top \end{bmatrix}^\top
\)
%
and we often write $\bar{\vect{u}}_{N}$ for $\bar{\vect{u}}_{\,[1,N]}$. Furthermore, $u_{[1,N]}^{\mt{p}}$ denotes the sequence $\{p_k\kron u_k\}_{k=1}^N$. Same notation is defined for $y$ respectively. Finally in \eqref{eq:thm_FL}, $\bar{\mc{P}}^{n}:=\bar{p}_{[1,L]}\bkron I_{n}$, where $\bkron$ is the block-diagonal Kronecker operator, i.e., for a sequence $\bar{p}_{[1,L]}$ we have $\bar{p}_{[1,L]}\bkron I_n:=\mathrm{diag}_{i=0}^{L}(\bar{p}_i\kron I_n)$. See \cite{VerhoekAbbasTothHaesaert2021} for a detailed derivation of \eqref{eq:thm_FL}.



To represent \emph{any} length $L$ trajectory of \eqref{eq:sys}, the data set $\dataset$ must be sufficiently rich, i.e., \emph{persistently exciting} (PE), up to a certain degree, which is dependent on both $L$ and the system order $\dnx$. 
For the persistency of excitation of the data set $\dataset$ with respect to the considered system class (system representable by a shifted-affine form), we formulate a derivative from the technical PE definition given in \cite{VerhoekTothHaesaertKoch2021, VerhoekTothHaesaert2023}. Using the terminology in \cite{MarkovskyDorfler2021}, we define the \emph{generalized PE} condition as:
%
\begin{definition}[Generalized \TR{persistency of excitation}, %condition, 
\cite{VerhoekTothHaesaert2023}]\label{def:PE}
	Given the data set $\dataset$ from a system represented as \eqref{eq:sys} with system order $\dnx$. The signal pair $(u_{[1,N]}, p_{[1,N]})$ in $\dataset$ is persistently exciting of degree $(L,\dnx)$ when for any $(\bar{u}_{[1,L]},\bar{p}_{[1,L]},\bar{y}_{[1,L]})\in\Bfint{}{[1,L]}$, there exists a $g\in\R^{N-L+1}$ such that \eqref{eq:thm_FL} holds.
\end{definition}
%
From \cite{VerhoekTothHaesaert2023}, we now introduce a result that is an analogous Fundamental Lemma for systems of the form in \eqref{eq:sys}. This result is centered around the generalized PE condition, and gives a condition to determine whether a given data set $\dataset$ satisfies the generalized PE condition, i.e., is sufficiently rich in terms of information.
%
\begin{theorem}[{Shifted-affine LPV Fundamental Lemma}, \cite{VerhoekTothHaesaert2023}]\label{thm:PE}
Given the PE data set $\dataset\in\Bfint{}{[1,N]}$ of degree $(L,\dnx)$. 
For $L\ge\LBf$ and any $\bar{p}_{[1,L]}\in\ms{P}_{[1,L]}$,
	\begin{equation} \label{eq:thm:beh}
		\proj_{\meu{N}_{\bar{p}}}(\meu{S}) = \Bfint{\bar{p}}{[1,L]}
	\end{equation}
	if and only if
	\begin{equation} \label{eq:thm:dim}
		\mr{dim}\Big\{\proj_{\meu{N}_{\bar{p}}}(\meu{S})\Big\} = \dnx + \dnu L,
	\end{equation}
	where
	\begingroup\allowdisplaybreaks
	\begin{subequations}\label{eq:nullrowdef}
	\begin{align}
		\meu{N}_{\bar{p}}:= & \mr{nullspace}\left\lbrace\begin{pmatrix}
		\mc{H}_L(u^{\mt{p}}_N) - \bar{\mc{P}}^{\dnu}\mc{H}_L(u_N) \\
		\mc{H}_L(y^{\mt{p}}_N) - \bar{\mc{P}}^{\dny}\mc{H}_L(y_N)
		\end{pmatrix} \right\rbrace, \\ 
		\meu{S} := & \mr{rowspace}\left\lbrace\begin{pmatrix}
		\mc{H}_L(u_N) \\
		\mc{H}_L(y_N)
		\end{pmatrix} \right\rbrace.
	\end{align}
	\end{subequations}
	\endgroup
\end{theorem}
%
\begin{proof}
See \cite{VerhoekTothHaesaert2023} for a proof.
\end{proof}
From Theorem~\ref{thm:PE} we can formulate the following two results:
\begin{corollary}[Generalized PE condition, \cite{VerhoekTothHaesaert2023}]
    A data set $\dataset \in\Bfint{}{[1,N]}$ satisfies the generalized PE condition of degree $(L,\dnx)$ if \eqref{eq:thm:dim} holds for all $p_{[1,L]}\in\ms{P}_{[1,L]}$.
\end{corollary}
%
\begin{theorem}[Data-driven LPV representation, \cite{VerhoekTothHaesaert2023}]\label{cor:LPVDDR}
Given a data set $\dataset=\{u_k,p_k,y_k\}_{k=1}^N \in\Bfint{}{[1,N]}$ from \eqref{eq:sys} with order $\dnx$, where $(u_{[1,N]},p_{[1,N]})$ is persistently exciting of order $(L,\dnx)$. Then, $(\bar{u}_{[1,L]},\bar{p}_{[1,L]},\bar{y}_{[1,L]})\in\Bfint{}{[1,L]}$ if and only if there exists a vector $g\in\mb{R}^{N-L+1}$ such that \eqref{eq:thm_FL} holds. 
\end{theorem}
%
\begin{proof}
\TR{Direct implication by Corollary \Ref{cor:LPVDDR} and Theorem \ref{thm:PE}.}
\end{proof}
%
These results allow to fully describe an LPV system, whose representation is in the form of \eqref{eq:sys}, using only a PE data set $\dataset$ by means of Theorem~\ref{cor:LPVDDR}. In the remainder of this paper, we will assume that we have the PE data set available and then use representation \eqref{eq:thm_FL} to analyze dissipativity of the considered LPV system in a fully data-driven setting.







