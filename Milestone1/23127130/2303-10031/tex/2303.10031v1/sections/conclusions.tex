This paper proposes a direct data-driven dissipativity analysis method for LPV systems
 which can be represented in an IO form with shifted affine dependence on the scheduling signal. Using a data-driven representation of the LPV system, which is constructed from a data set with persistently exciting inputs and scheduling,  we can analyze \TR{its} finite horizon dissipativity property %of an LPV system 
 for any quadratic performance specification \TR{by our proposed method}. This allows to give performance guarantees \TR{for} the considered system using only a single data set measured from the system. As we show in the paper, the analysis can be accomplished by solving an SDP subject to LMI constraints, which is built up using the data set and a definition of the scheduling set. By means of the presented examples, we demonstrate that our methods compare well with the classical model-based \TR{methods.} %under both the original and identified model of the system 
\TR{We also show} that the finite horizon dissipativity property converges to the infinite-horizon dissipativity property (employed in the model-based methods) already for relatively short horizons. 

For future work, we aim to analyze the convergence to the infinite-horizon dissipativity property, extend our results to the dissipativity formulation of Willems \cite{Willems1972} that includes a storage functions and investigate the handling of noisy data.
