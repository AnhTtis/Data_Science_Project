\IEEEPARstart{D}{irect} data-driven methods are attractive to analyze system behavior and generate stabilizing controllers from data without the need of identifying a mathematical description of the system. A cornerstone {of} direct data-driven analysis and control for discrete-time  \emph{linear time-invariant} (LTI) systems is the so-called \emph{Fundamental Lemma} by Willems et al. \cite{WillemsRapisardaMarkovskyMoor2005}. This result uses LTI behavioral system theory \cite{PoldermanWillems1997} to obtain a characterization of the system behavior based on a single sequence of \emph{input-output} (IO) data. Based on the Fundamental Lemma, numerous results have been developed for LTI systems, e.g., on data-driven simulation~\cite{MarkovskyRapisarda2008}, system analysis~\cite{romer2019one,koch2021provably,vanWaarde2022_matrixS}, and (predictive) control \cite{markovsky2007linear, dePersisTesi2020, coulson2019data,BerberichKohlerMullerAllgower2021DPC_guarantees}, many of which also guarantee robustness in the presence of noise.
Some extensions towards the \emph{nonlinear} system domain have been made as well, e.g., for Hammerstein and Wiener systems \cite{BerberichAllgower2020}, \emph{linear time-varying} (LTV) \cite{nortmann2020data} systems, flat nonlinear systems \cite{alsalti2021data}, nonlinear autoregressive exogenous (NARX) systems~\cite{mishra2021narx}, and bilinear systems~\cite{yuan2021data}. However, these results impose heavy restrictions on the {underlying} system as they leverage model transformations and linearizations. A particularly interesting extension of Willems' Fundamental Lemma is the one towards \emph{linear parameter-varying} (LPV) systems \cite{VerhoekAbbasTothHaesaert2021, VerhoekTothHaesaertKoch2021, VerhoekTothHaesaert2023, Verhoek2022_DDLPVstatefb, Verhoek2022_DDLPVstatefb_experiment}.

LPV systems are linear systems, where the model parameters, describing the linear signal relation, are dependent on a time-varying variable, referred to as the \emph{scheduling variable}. The latter variable is used to express nonlinearities, time variation, or exogenous effects. The main difference with respect to LTV systems is that the scheduling variable is \emph{not} known a priori; it is only assumed that it is measurable and allowed to vary in a given set. The LPV framework has been shown to be able to capture a relatively large subset of nonlinear systems in terms of LPV surrogate models \cite{Toth2010} and can therefore be considered as a promising extension towards data-driven analysis and control for nonlinear systems.

In this paper, we focus on direct data-driven dissipativity analysis of LPV systems. The dissipativity property of a system simultaneously gives guarantees on the stability and performance characteristics of the system. This makes dissipativity a \emph{fundamental building block} in analysis and control of dynamical systems \cite{SchererWeiland2021}, especially because it is an essential tool in performance-based controller synthesis. While theory on direct data-driven dissipativity verification for LTI systems is rather established \cite{romer2019one, koch2020verifying, koch2021provably, vanWaarde2022_matrixS}, including handling of noisy data and robust verification, the literature on dissipativity analysis for LPV systems (and nonlinear systems) is mainly focussed on \emph{model-based} approaches. To perform analysis in a data-based setting, it is often required to have an \emph{indirect} data-driven approach, i.e., first identifying the system, followed by a model-based analysis on the estimated model. There are some direct dissipativity analysis approaches for nonlinear systems available, based on a set-membership argument \cite{TangDaoutidis2021, martin2021approximation, martin2021data, martin2021dissipativity, MartinAllgower2022}. 
However, these methods are not based on the behavioral framework and often require restrictive assumptions on the measurements of the system, e.g., output measurements can only be handled by the construction of an extended state that must be controllable, or it is assumed that full state measurements are available from the system. 
Therefore, having a tool based on the behavioral LPV framework that allows for direct data-driven analysis of the dissipativity property of an LPV system using only scheduling and IO data, significantly contributes towards a data-driven framework with guarantees for LPV systems, and possibly a direct data-driven framework for nonlinear systems (see, e.g., \cite{Verhoek2022_DDLPVstatefb_experiment} for a promising result).

Our contributions in this paper are the following:
\begin{enumerate}[label={C\arabic*:}, align=left, ref={C\arabic*}, leftmargin=*]
	\item Characterization of finite-horizon dissipativity for LPV systems by means of a fully data-driven representation; \label{C:char}
	\item \new{Computational methods with different levels of conservatism and computational tractability} 
	for direct dissipativity analysis of a class of LPV systems using \emph{only} input-scheduling-output data; \label{C:comp}
	\item Extensive analysis of the methods in simulation studies. \label{C:simu}
\end{enumerate}

\smallskip

In this work, we consider LPV systems that can be represented by an LPV-IO representation with \emph{shifted-affine} scheduling dependency. This realization form is highly attractive in practice, cf.~\cite{kwiatkowski2006automated, sloth2011robust, HoffmannWerner2014, deLange2022lpv, Toth2011_SSrealizationTCST}, %
as it allows the application of powerful prediction error minimization approaches, but at the same time, due to a direct state-space realization with static-affine scheduling dependency, it also allows the use of well established convex analysis and control synthesis approaches on the model estimates. Constituting to Contribution~\ref{C:comp}, the presented direct data-driven dissipativity analysis techniques can be efficiently solved as a \emph{semi-definite program} (SDP), subject to a finite set of \emph{linear matrix inequality} (LMI) constraints, similar to  model-based analysis approaches, {but without requiring an analytic LPV model of the} system.


In the remainder of this work, we first present the problem setting in Section~\ref{sec:problemstatement}, followed by an introduction of the data-driven LPV representation in Section~\ref{sec:ddlpvrep}. Based on this, %
we {introduce a data-driven characterization of} dissipativity in
Section~\ref{sec:datadrivendissipativity}, corresponding {to} Contribution~\ref{C:char}. We derive computable analysis approaches
in Section~\ref{sec:computational}, providing Contribution~\ref{C:comp}. Finally, we {show applicability and analyze properties of the proposed methods  on simulation examples} in Section~\ref{sec:examples}, {corresponding to Contribution~\ref{C:simu},} and draw our conclusions on the presented results in Section \ref{sec:conclusions}. %


\subsection{Notation}
{The set of positive integers is denoted {as} $\mathbb{N}$, while $\mathbb{R}$ denotes the set of real numbers.}
The $p$-norm of a vector $x_k\in\mathbb{R}^{n_\mathrm{x}}$  is denoted by $\lVert x_k\rVert_p$, while  \new{$x_{k,j}$} stands for the $j$\tss{th} element of $x_k$. The identity matrix of size $n$ is given by $I_n$. For a matrix $A\in\mathbb{R}^{n\times m}$,  $A^\dagger$ stands for its Moore-Penrose inverse, while $A^\perp \in\mathbb{R}^{q\times n}$ corresponds to a  matrix of {row rank $q$}, where the rows form the basis of the left kernel of $A$ , i.e., $A^\perp A=0$. Furthermore, $A\kron B$ is the Kronecker product of two matrices $A$ and $B$. Given square matrices $A_1,\dots,A_n$, $\diag_{i=1}^n(A_i)$ gives a block-diagonal matrix with blocks $A_i$. For vector spaces $\meu{A}$ and $\meu{B}$,  $\proj_\meu{B}(\meu{A})$ denotes the projection of $\meu{A}$ onto $\meu{B}$. 
For a sequence $w_{[1,N]}: [1,N] \rightarrow \mathbb{R}^{n_\mathrm{w}}$, we denote
the Hankel matrix of depth $L$ associated with {it} %
as
\begin{align*}
\mc{H}_L(w_{[1,N]}) = \begin{bmatrix} 
    w_1 & w_2 & \cdots & w_{N-L+1} \\
    w_2 & w_3 & \cdots & w_{N-L+2} \\
    \vdots & \vdots & \ddots & \vdots \\
    w_{L} & w_{L+1} & \cdots & w_{N}
\end{bmatrix}.
\end{align*}















































