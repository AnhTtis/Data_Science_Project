To properly analyze properties of a system, some form of \emph{representation} of the underlying behavior is required. However, one may lose information in a modeling process by estimating such a representation from data. In this section, we focus on a purely data-based representation of \eqref{eq:sys} that is used to analyze dissipativity of the system directly from data. %
We can obtain such a representation by characterizing the behavior of \eqref{eq:sys}, restricted to a finite time interval $[1,\,L]$, i.e., $\Bfint{}{[1,\,L]}$.

In \cite{VerhoekTothHaesaertKoch2021}, a data-driven representation for general LPV systems has been derived based on behavioral LPV system theory \cite{Toth11_LPVBehav}. 
Using this fundamental result for LPV systems that have a representation in terms of \eqref{eq:sys}, a simplified data-driven representation has been given in \cite{VerhoekAbbasTothHaesaert2021}. This data-driven representation under shifted affine dependency is obtained by structuring the data in $\dataset$ in terms of \eqref{eq:sys}, such that for valid solution trajectories of the system, i.e., $(\bar{u}_{[1,L]},\bar{p}_{[1,L]},\bar{y}_{[1,L]}) \in \Bfint{}{[1,L]}$, there exists a $g\in\mathbb{R}^{N-L+1}$ that satisfies 
\begin{equation}\label{eq:thm_FL}
	\begin{bmatrix}
		\mc{H}_L\left(u_{[1,N]}\right) \\ 
		\mc{H}_L\left(y_{[1,N]}\right) \\ 
		\mc{H}_L\left(u_{[1,N]}^{\mt{p}}\right) - \bar{\mc{P}}^{\dnu}\mc{H}_L\left(u_{[1,N]}\right) \\ 
		\mc{H}_L\left(y_{[1,N]}^{\mt{p}}\right) - \bar{\mc{P}}^{\dny}\mc{H}_L\left(y_{[1,N]}\right)
	\end{bmatrix}g = \begin{bmatrix}
		\bar{\vect{u}}_{L} \\
		\bar{\vect{y}}_L \\
		0 \\
		0
	\end{bmatrix}.
\end{equation}
Here, for a sequence $\bar{u}_{[1,N]}: [1,N] \rightarrow \mathbb{R}^{n_\mathrm{u}}$,  its vectorization $\mr{vec}\left(\bar{u}_{[1,N]}\right)$ is defined as 
\(
	\bar{\vect{u}}_{\,[1,N]} = \begin{bmatrix} \bar{u}_1^\top & \cdots & \bar{u}_{N}^\top \end{bmatrix}^\top
\)
and we often write $\bar{\vect{u}}_{N}$ for $\bar{\vect{u}}_{\,[1,N]}$. Furthermore, $u_{[1,N]}^{\mt{p}}$ denotes the sequence $\{p_k\kron u_k\}_{k=1}^N$. The same notation is defined for $y$ respectively. Finally in \eqref{eq:thm_FL}, $\bar{\mc{P}}^{n}:=\bar{p}_{[1,L]}\bkron I_{n}$, where $\bkron$ is the block-diagonal Kronecker operator, i.e., for a sequence $\bar{p}_{[1,L]}$ we have $\bar{p}_{[1,L]}\bkron I_n:=\mathrm{diag}_{i=0}^{L}(\bar{p}_i\kron I_n)$. See \cite{VerhoekAbbasTothHaesaert2021} for a detailed derivation of \eqref{eq:thm_FL}.



To represent \emph{any} length $L$ trajectory of \eqref{eq:sys}, the data set $\dataset$ must be sufficiently rich. \new{The LPV Fundamental Lemma for LPV systems of the form~\eqref{eq:sys}, i.e., the Extended LPV Fundamental Lemma from~\cite{VerhoekTothHaesaert2023} provides a condition to characterize all possible solutions in $\Bfint{\bar{p}}{[1,L]}$ for a given $\bar{p}_{[1,L]}$.
\begin{prop}[Extended LPV Fundamental Lemma]\label{prop:FLeasy}
    Given a data set $\dataset$ from an LPV system represented by~\eqref{eq:sys}. For a $\bar{p}_{[1,L]}\in\ms{P}_{[1,L]}$, define the spaces
    \begingroup\allowdisplaybreaks
	\begin{subequations}\label{eq:nullrowdef}
	\begin{align}
		\meu{N}_{\bar{p}}:= & \mr{nullspace}\left\lbrace\begin{pmatrix}
		\mc{H}_L(u^{\mt{p}}_N) - \bar{\mc{P}}^{\dnu}\mc{H}_L(u_N) \\
		\mc{H}_L(y^{\mt{p}}_N) - \bar{\mc{P}}^{\dny}\mc{H}_L(y_N)
		\end{pmatrix} \right\rbrace, \\ 
		\meu{S} := & \mr{rowspace}\left\lbrace\begin{pmatrix}
		\mc{H}_L(u_N) \\
		\mc{H}_L(y_N)
		\end{pmatrix} \right\rbrace.
	\end{align}
	\end{subequations}
	\endgroup
	For ${L}\ge\dnr$, the data set $\dataset$ satisfies
    \begin{equation} \label{eq:thm:dim}
        \mr{dim}\big\{\proj_{\meu{N}_{p}}(\meu{S})\big\} = \dnx + \dnu L,
    \end{equation}
    for all $\bar{p}_{[1,L]}\in\ms{P}_{[1,L]}$, if and only if
    \begin{equation} \label{eq:thm:beh}
        \proj_{\meu{N}_{\bar{p}}}(\meu{S}) = \Bfint{\bar{p}}{[1,L]}
    \end{equation}
    for all $\bar{p}_{[1,L]}\in\ms{P}_{[1,L]}$. This is equivalent to the existence of a vector $g\in\mb{R}^{N-L+1}$ for any $(\bar{u}_{[1,L]},\bar{p}_{[1,L]},\bar{y}_{[1,L]})\in\Bfint{}{[1,L]}$ such that~\eqref{eq:thm_FL} holds.
\end{prop}
\begin{proof}
    See~\cite{VerhoekTothHaesaert2023}.
\end{proof}
We want to emphasize here that this result allows to fully describe an LPV system, whose representation is in the form of \eqref{eq:sys}, using \emph{only} a data set that satisfies condition~\eqref{eq:thm:dim}. This condition in-fact reminiscent of the data set coming from the LPV system being \emph{persistently exciting}\footnote{As condition~\eqref{eq:thm:dim} also contains the output of $\dataset$, similar to~\cite{MarkovskyDorfler2021}, {we can} refer to~\eqref{eq:thm:dim} as a \emph{generalized} LPV PE condition.} (PE) of a certain degree. As in the technical definition for PE in~\cite{VerhoekTothHaesaertKoch2021}, we see that the condition is, next to the input, also dependent on the scheduling. This introduces an input-design problem for the generation of a PE $\dataset$, which is {similar to the problem of PE experiment design for LPV system identification and it is beyond the} %
scope {of} this paper.
  {In the reminder of the paper,} we will refer to $\dataset$ satisfying~\eqref{eq:thm:dim} as ``$\dataset$ being PE of degree $(L,\dnx)$'' and we will consider that $\dataset$ satisfies this property to} use representation~\eqref{eq:thm_FL} to analyze dissipativity of the considered LPV system in a fully data-driven setting.







