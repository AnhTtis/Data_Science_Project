\section{Model-based $L$-dissipativity}\label{app:modelbasedLdissip}
In model-based dissipativity analysis, {definition and testing of} %
classical dissipativity, see \cite{Willems1972, HillMoylan1980}, is more straight-forward. 
{In this section, we give a model-based equivalent of Theorem~\ref{thm:LPV_diss} in terms of an $L$-dissipativity test}.

{Note that LPV systems represented by \eqref{eq:sys} have a direct minimal state-space realization in the form of}
\begin{align*}
    x_{k+1} & = A(p_k) x_k + B(p_k)u_k, \\
    y_k & = C(p_k) x_k + D(p_k)u_k.
\end{align*}
{as discussed in \cite{Toth2011_SSrealizationTCST}. This form is easily computable in the SISO case, while computation in the {MIMO} case requires the choice of algebraically independent state basis which can be accomplished via \cite{Toth15CDCb}.} 

It is well-known that any sequence $y_{[1,L]}$ can be written as
\begin{equation}
    \vect{y}_L = \ms{O}_L(p_{[1,L]}) x_1 + \ms{T}_L(p_{[1,L]}) \vect{u}_L,
\end{equation}
with
\begin{equation}
    \ms{O}_L(p_{[1,L]}) = \begin{bmatrix} C(p_1) \\ C(p_2)A(p_1) \\ \vdots \\ C(p_L)A(p_{L-1})\cdots A(p_1) \end{bmatrix}
\end{equation}
and
\begin{multline}
    \ms{T}_L(p_{[1,L]}) = \\
    \begin{bsmallmatrix} D(p_1) & 0 & \cdots & 0 \\ C(p_2)B(p_1) & D(p_2) & \sddots & \svdots \\ C(p_3)A(p_2)B(p_1) & C(p_3)B(p_2) & \sddots & \svdots  \\ \svdots & \sddots & \sddots & \svdots \\ C(p_L)A(p_{L-1})\cdots A(p_2)B(p_1) & \cdots & \cdots & D(p_L) \end{bsmallmatrix}.
\end{multline}
By the assumption of zero initial conditions in the definition for $L$-dissipativity, we have that
\begin{equation}
    \vect{y}_L = \ms{T}_L(p_{[1,L]}) \vect{u}_L.
\end{equation}
Now considering \eqref{eq:L_diss_stacked}, we can write
\begin{equation}
    0 \leq \begin{bmatrix} {\vect{u}}_L\\ {\vect{y}}_L\end{bmatrix}^{\!\top} \! \Pi_L \begin{bmatrix} {\vect{u}}_L\\ {\vect{y}}_L \end{bmatrix} = {\vect{u}}_L^\top \begin{bmatrix} I_L \\ \ms{T}_L(p_{[1,L]})\end{bmatrix}^{\!\top} \! \Pi_L \begin{bmatrix} I_L\\ \ms{T}_L(p_{[1,L]}) \end{bmatrix} \vect{u}_L.
\end{equation}
Hence, model-based verification of $L$-dissipativity  {corresponds to} %
\begin{equation}\label{eq:LMImodelbased}
    \begin{bmatrix} I_L \\ \ms{T}_L(p_{[1,L]})\end{bmatrix}^\top \Pi_L \begin{bmatrix} I_L\\ \ms{T}_L(p_{[1,L]}) \end{bmatrix} \possemidef0,
\end{equation}
for all scheduling trajectories in $\ms{P}_{[1,L]}$. Based on the form of $\ms{P}_{[1,L]}$, e.g., polytopic or quadratic, one can select an existing method to verify \eqref{eq:LMImodelbased} in a computationally attractive manner. 