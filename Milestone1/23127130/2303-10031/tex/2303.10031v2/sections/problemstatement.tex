As mentioned in Section~\ref{sec:introduction}, an attractive approach to address identification and potentially control design in the LPV framework is to represent systems with discrete-time \emph{shifted-affine} LPV-IO representation \new{\cite{Toth2011_SSrealizationTCST, HoffmannWerner2014}}.
Hence, we consider that the underlying system is described by %
\begin{subequations}\label{eq:sys}
\begin{align}\label{eq:lpv_shi_aff}
y_k+\sum_{i=1}^{\dna}a_i(p_{k-i})y_{k-i}=\sum_{i=0}^{\dnb}b_i(p_{k-i})u_{k-i},
\end{align}
with input $u_k\in\mathbb{R}^{\dnu}$, output $y_k\in\mathbb{R}^{\dny}$ and scheduling signal $p_k\in\mathbb{P}\subseteq\mathbb{R}^{\dnp}$, $\dna\geq1$, $\dnb\ge0$, and $k\in\N$ being the discrete time, where $\mb{P}$ is a convex and compact set named the \emph{scheduling space}. Let $n_\mathrm{r}=\max(\dna,\dnb)$. {The initial condition for \eqref{eq:lpv_shi_aff} is defined as $\{y_k\}_{k=1-\dna}^{0}$, $\{p_k\}_{k=1-n_\mathrm{r}}^{0}$ and $\{u_k\}_{k=1-\dnb}^{0}$, collected into the vector $w_0\in \mathbb{W}_0=\mathbb{R}^{\dna \dny}\times \mathbb{P}^{\dnr} \times \mathbb{R}^{\dnb \dnu}$.} {Due to the embedding principle, the scheduling trajectories are often restricted to a set $\ms{P}$, e.g., $\mb{P}^\N$, which describes the admissible {behavior of the signal $p_k$.} %
We will give more examples of definitions for $\ms{P}$ in Section~\ref{sec:computational}.} The functions $a_i$ and $b_i$  define the shifted-affine character of the representation, i.e., $a_i$ and $b_i$ are affine functions of the time-shifted values of the scheduling $p_k$:
\begin{align}\label{eq:staticaffinefunc}
a_i(p_{k-i})=\sum_{j=0}^{\dnp}a_{i,j}p_{k-i,j}, \quad b_i(p_{k-i})=\sum_{j=0}^{\dnp}b_{i,j}p_{k-i,j},
\end{align}
\end{subequations}
where $p_{k,0}=1$ for all $k$. It is assumed that $q^{n_\mathrm{r}}(1+\sum_{i=1}^{\dna}a_i(p_{k-i})q^{-i})$ and $q^{n_\mathrm{r}}\sum_{i=0}^{\dnb}b_i(p_{k-i})q^{-i}$  are coprime {and compose a full row rank matrix polynomial}, implying that the LPV-IO representation \eqref{eq:sys} is minimal and structurally controllable and observable \cite{Toth2010}. 


The LPV system represented by \eqref{eq:sys} is characterized in terms of all possible solutions of \eqref{eq:sys}:
\begin{multline}\label{eq:behav:sys}
\mf{B}:=\big\{(u,p,y)\in\left(\R^{\dnu}\times \mb{P}\times \R^{\dny}\right)^\N \mid p\in\ms{P} \text{ and} \\ \exists w_0 \in \mathbb{W}_0 \text{ s.t. \eqref{eq:sys} holds}\ \forall k \in \mathbb{N} \big\},
\end{multline}
which we call the behavior of \eqref{eq:sys}.
Similarly, we can define the projected IO behavior $\mf{B}_{\mr{IO}}$, as the set of admissible IO trajectories, i.e., $\mf{B}_{\mr{IO}}=\pi_{u,y}\mf{B}$, {where $\pi_{u,y}$ stands for  $\proj_{(\R^{\dnu}\times \R^{\dny})^\N} (\mf{B})$}. Also note that $\pi_{p}\mf{B}=\ms{P}$. \new{Moreover, we define $\dnx$ as the minimal required state-dimension among all possible LPV state-space realizations that can represent $\mf{B}$}, which we will refer to as the \emph{system order}. The \emph{lag} associated with $\mf{B}$ is the minimal $n_\mathrm{r}$ required to represent $\mf{B}$ in terms of~\eqref{eq:lpv_shi_aff}. \new{For {\emph{multiple-input multiple-output} (MIMO)} systems, $\dnx\geq\dnr$, while for \emph{single-input single output} (SISO) systems $\dnx=\dnr$.}
{Finally, we introduce $\mf{B}_p$, which denotes the set that contains all IO trajectories compatible with some scheduling signal $p\in\ms{P}$,}
\begin{equation}\label{eq:behav:p}
    \mf{B}_p=\{(u,y)\in\left(\R^{\dnu}\times \R^{\dny}\right)^\N \mid  (u,p,y)\in\mf{B} \}.
\end{equation}

Analyzing the \emph{behavior} allows us to study the system trajectories independent of the underlying representation. Therefore, the behavioral framework is a helpful tool in assessing stability and performance properties in the analysis and control of systems. As highlighted in Section~\ref{sec:introduction}, an {attractive approach} for simultaneously evaluating \emph{stability} and \emph{performance} is the concept of dissipativity, which is defined in \cite{HillMoylan1980} from a IO perspective in discrete-time as follows.
\begin{definition}[Dissipativity,  \cite{HillMoylan1980}]\label{def:dissipativity-hillmoylan}
    A system with input $u:\N\to\R^{\dnu}$, output $y:\N\to\R^{\dny}$ and behavior $\mf{B}$ is dissipative with respect to the supply rate $\Pi\in\mb{R}^{(\dnu+\dny)\times(\dnu+\dny)}$ if %
    \begin{equation}\label{eq:dissipationinequality}
    \sum_{k={1}}^{N} \begin{bmatrix} u_k \\ y_k \end{bmatrix}^{\!\top}\!\! \Pi \begin{bmatrix} u_k \\ y_k \end{bmatrix}\geq0 \quad \text{for any } N\in\mathbb{N},
    \end{equation}
    and $\forall (u,y)\in\pi_{u,y}\mf{B}$ with $w_0=0$ (zero initial condition).
\end{definition}
In this paper, we consider the case of QSR-dissipativity, i.e., the supply rate $\Pi$ is partitioned as
\begin{equation}\label{eq:supplyrate}
\Pi=\begin{bmatrix} Q & S \\ S^\top & R \end{bmatrix}, 
\end{equation}
with $Q=Q^\top \in\mb{R}^{\dnu\times\dnu}$, $S\in\mb{R}^{\dnu\times\dny}$, and $R=R^\top \in\mb{R}^{\dny\times\dny}$. This form for the supply rate is commonly used to analyze, e.g., passivity properties with $(Q,S,R)=(0,I,0)$ or the upper bound $\gamma$ on the $\mc{L}_2$-gain of the system with $(Q,S,R)=(\gamma^2 I,0,\unaryminus I)$ .

There are multiple {model-based} techniques to verify stability and performance via \eqref{eq:dissipationinequality} for LTI \cite{Willems1972, HillMoylan1980}, LPV~{\cite{Wu1995, KoelewijnToth2021}~and nonlinear \cite{Schaft2017_L2book, VerhoekKoelewijnToth2021_incremental}} systems. %
{However, when we have} only measured data available from %
{the system,} dissipativity analysis via Definition~\ref{def:dissipativity-hillmoylan} becomes a {difficult} problem, as we can only consider {a few} \emph{finite-time} trajectories. {For this reason, we introduce a finite-time notion of dissipativity, $L$-dissipativity, which allows for analysis with system trajectories of finite length.} Hence, given a signal $w\in\left(\R^{\dnw}\right)^\N$ and a compact set $\mb{T}\subseteq\N$, then $w_\mb{T}$ corresponds to the truncation of $w$ to the time interval $\mb{T}$. Similarly, the restricted behavioral set $\Bfint{}{\mb{T}}$  contains the truncation of all the trajectories in $\mf{B}$ to the time interval $\mb{T}$. In this paper, we consider the following definition of $L$-dissipativity, originally proposed in \cite{maupong2017lyapunov}.
\begin{definition}[$L$-dissipativity, \cite{maupong2017lyapunov}]\label{def:L_diss}
A system with input $u:\N\to\R^{\dnu}$, output $y:\N\to\R^{\dny}$ and behavior $\mf{B}$ is $L$-dissipative w.r.t. the supply rate \eqref{eq:supplyrate} when
\begin{align}\label{eq:def_L_diss}
\sum_{k=1}^{r}\begin{bmatrix}u_k\\y_k\end{bmatrix}^{\!\top} \!\!
\Pi\begin{bmatrix}u_k\\y_k\end{bmatrix}\geq0 \quad \text{for any } r\in[1,L]
\end{align}
and for all truncated system trajectories $(u_{[1,L]}, y_{[1,L]})\in\Bfint{\mr{IO}}{[1,L]}$ with $w_0=0$ (zero initial condition).
\end{definition}
Our main results on data-driven dissipativity analysis for LPV systems are built upon this definition. We are now ready to formulate the problem that we solve in this paper.
\subsubsection*{Problem statement} Consider a data-generating system that can be represented with \eqref{eq:sys}. Given a length $N$ data set $\dataset:= \{u_k,p_k,y_k\}_{k=1}^N$ (data-dictionary) sampled from the data-generating system, i.e., $(u_{[1,N]},p_{[1,N]},y_{[1,N]})\in\Bfint{}{[1,N]}$. For an $1\leq L<N$, how {can we} determine that \eqref{eq:sys} is $L$-dissipative w.r.t. the supply rate \eqref{eq:supplyrate} {using only $\dataset$}?


{If we can solve this problem, then} we can analyze %
performance properties of  {LPV} systems in a data-driven setting, \emph{without} the need of knowing their {dynamic} model \eqref{eq:sys}. 


