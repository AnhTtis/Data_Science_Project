\documentclass[%
reprint,
superscriptaddress,
%groupedaddress,
%unsortedaddress,
%runinaddress,
%frontmatterverbose, 
%preprint,
%preprintnumbers,
%nofootinbib,
%nobibnotes,
%bibnotes,
amsmath,amssymb,
 aps,
%pra,
%prb,
%rmp,
%prstab,
%prstper,
floatfix,
]{revtex4-1}

\usepackage{mathbbol}

\usepackage{graphicx}% Include figure files
\usepackage{dcolumn}% Align table columns on decimal point
\usepackage{bm}% bold math
\usepackage{color}
\usepackage{slashed}
%\usepackage{hyperref}% add hypertext capabilities
%\usepackage[mathlines]{lineno}% Enable numbering of text and display math
%\linenumbers\relax % Commence numbering lines

%\usepackage[showframe,%Uncomment any one of the following lines to test 
%%scale=0.7, marginratio={1:1, 2:3}, ignoreall,% default settings
%%text={7in,10in},centering,
%%margin=1.5in,
%%total={6.5in,8.75in}, top=1.2in, left=0.9in, includefoot,
%%height=10in,a5paper,hmargin={3cm,0.8in},
%]{geometry}
%\usepackage{hyperref}% add hypertext capabilities

\newcommand{\iu}{{i\mkern1mu}}
\def\ave#1#2{\langle #2 | #1 | #2 \rangle}
\def\Ave#1{\langle #1 \rangle}
\def\av#1#2{\frac{\ave{#1}{#2}}{\Ave{#2|#2}}}
\def\x{\Ave{H_1}}
\def\y{\Ave{H_2}}

\begin{document}

%\preprint{APS/123-QED}

\title{Zero Curvature Condition for Quantum Criticality}% Force line breaks with \\
%\thanks{A footnote to the article title}%


\author{Chaoming Song}
\email{c.song@miami.edu}
\affiliation{%
Department of Physics, University of Miami, Coral Gables, Florida 33146, USA}%

%\collaboration{MUSO Collaboration}%\noaffiliation
%\homepage{http://www.Second.institution.edu/~Charlie.Author}




%\collaboration{CLEO Collaboration}%\noaffiliation

%\date{}% It is always \today, today,
             %  but any date may be explicitly specified

\begin{abstract}
Quantum criticality typically lies outside the bounds of the conventional Landau paradigm. Despite its significance, there is currently no generic paradigm to replace the Landau theory for quantum phase transition, partly due to the rich variety of quantum orders. In this paper, we present a new paradigm of quantum criticality based on a novel geometric approach. Instead of focusing on microscopic orderings, our approach centers on the competition of commuting operators, which can be best investigated through the boundary geometry of their expectation values. We demonstrate that the quantum phase transition occurs precisely at the zero-curvature point on this boundary, which implies the competing operators are maximally commuting at the critical point.
\end{abstract}

%\keywords{Suggested keywords}%Use showkeys class option if keyword
                              %display desired
\maketitle

%\tableofcontents

%\section{\label{sec:level1}Introduction}

%{\it Introduction. --}

In recent years, there has been growing interest in the study of quantum criticality at zero temperature, revealing a wealth of fascinating phenomena absent in classical phase transitions \cite{sondhi1997continuous,sachdev1999quantum,sachdev2000quantum,zurek2005dynamics}. While classical phase transitions are well understood and described by the Landau-Ginzburg-Wilson (LGW) theory, quantum criticality poses a greater challenge due to quantum entanglement \cite{vidal2003entanglement}. In the LGW paradigm, the phase transition is typically characterized by local order parameters \cite{landau1937theory}, which do not capture the essential physics of many quantum systems. For instance,  quantum systems can undergo a topological phase transition without any change in local order parameters \cite{wen1989vacuum,wen1990topological,wen1993transitions,qi2008topological,kastner2008phase,wen2017colloquium,wen2019choreographed}. Another example is the deconfined quantum criticality 
\cite{senthil2004deconfined,senthil2004quantum,shyta2022frozen}, where the system competes over multiple orders but undergoes a continuous transition, contrary to the predictions of LGW of a first-order transition or phase coexistence.

Despite various models demonstrating the non-Landau nature of quantum criticality, there is currently no generic paradigm to replace the LGW theory for quantum criticality. This is partly due to the rich variety of microscopic orders discovered in quantum systems, which poses a challenge for identifying a universal framework that applies to all systems. However, a major distinction between quantum and classical systems lies in their non-commutativity. One may suspect quantum criticality arises from the non-commutative nature of physical observables, which gives rise to the unique feature of quantum phase transition beyond the scope of classical theory.

In this Letter, we present a novel approach to understanding quantum phase transitions. Instead of focusing on microscopic details of particular orders, we argue that quantum criticality arises from the non-commutativity of competing operators. We demonstrate that this competition can be investigated through the boundary geometry of their expectation values. The quantum critical point corresponds to the zero-curvature point of this boundary, which can be interpreted as maximally commuting and is closely linked to integrability.

%{\it Competing order parameter. ---} 
We start with a generic system of two competing operators, with the Hamiltonian
\begin{equation}\label{eq:H}
     H(\lambda_1, \lambda_2) = \lambda_1  H_1 + \lambda_ 2  H_2, 
\end{equation}
where Hermitian operators $ H_1$ and $ H_2$ are non-commutative, and  $\lambda_1$ and $\lambda_2$ are real parameters. Equation~(\ref{eq:H}) describes a competition between two non-commutative operators, potentially giving rise to the quantum phase transition. For any given wavefunction $\psi$, there are two natural order parameters $\Ave{H_1} \equiv  \av{ H_1}{\psi}$ and $\Ave{H_2} \equiv  \av{ H_2}{\psi}$, which introduces a mapping 
 \begin{equation}\label{eq:map}
    \psi \rightarrow   \left( \Ave{H_1} , \Ave{H_2} \right), 
 \end{equation}
from the system Hilbert space to a two-dimensional moduli space $\mathcal{M}$ of order parameters. For finite-dimensional Hilbert space, the moduli $\mathcal{M}$ is generally a semialgebraic set. Figure~\ref{fig:map} illustrates that the geometry of $\mathcal{M}$ is enclosed by a non-trivial boundary $\partial \mathcal{M}$. To understand this boundary better, we consider the energy functional
\begin{equation}\label{eq:E}
    E[\psi] = \lambda_1 \Ave{H_1} + \lambda_2 \Ave{H_2}.
\end{equation}
The variational principle $\delta E[\psi] =0$ implies that the stationary points of Eq.~(\ref{eq:E}) correspond to the eigenstates $\psi$ of the Hamiltonian~(\ref{eq:H}), satisfying
\begin{equation} \label{eq:var}
    \lambda_1 \delta \Ave{H_1} + \lambda_2 \delta \Ave{H_2} = 0,
\end{equation}
which corresponds to the singular set $\mathcal{M}_e$ of the mapping~(\ref{eq:map}), where the global minimum corresponds to the boundary $\partial \mathcal{M} \subset \mathcal{M}_e$. Each point on $\partial \mathcal{M}$ corresponds to expectation values of the ground state with fixed parameter $\boldsymbol\lambda \equiv (\lambda_1, \lambda_2)$ in Eq.~(\ref{eq:H}). Moreover, Eq~(\ref{eq:var}) implies that the vector $\boldsymbol\lambda$ is perpendicular to the boundary $\partial \mathcal{M}$ and more generally $\mathcal{M}_e$, being its normal vector, which determines uniquely the ground state expectations $\Ave{H_1}_{\boldsymbol\lambda}$ and $\Ave{H_2}_{\boldsymbol\lambda}$.

\begin{figure}
 \includegraphics[width=1\linewidth]{moduli.pdf}
 \caption{ Illustration of the moduli space (grey domain) for the expectation values $\Ave{H_1}$ and $\Ave{H_1}$. The solid curve represents the boundary $\partial \mathcal{M}$, corresponding to the ground. The dashed curve corresponds to the excited state and is not necessarily convex. }
 \label{fig:map}
\end{figure}

Our previous works \cite{song2023exact, song2023reduced} have shown that the eigenstate moduli $\mathcal{M}_{e}$ is a one-dimensional algebraic variety for finite-dimensional Hilbert space, determined by a single equation 
\begin{equation} \label{eq:ae}
f(\Ave{H_1}, \Ave{H_2}) = 0,
\end{equation}
where $f$ is a bivariate polynomial that is determined implicitly by Eq.~(\ref{eq:var}). The algebraic relation~(\ref{eq:ae}) encodes all information of the Hamiltonian family~(\ref{eq:H}). Below we will focus primarily on the boundary $\partial \mathcal{M}$, that is, the ground state expectations.   

The boundedness of the moduli $\mathcal{M}$ can be seen as a generalization of Heisenberg's uncertainty principle, whereby the non-trivial geometry of the boundary $\partial \mathcal{M}$ reflects a highly non-classical phenomenon.  One of the most notable features of the moduli $\mathcal{M}$ is their convexity, which we illustrate through a simple proof. Consider two arbitrary points $(x_1,y_1)$ and $(x_2,y_2) \in \mathcal{M}$, where $x_1  = \ave{H_1}{\psi_1}$ and $y_1 = \ave{H_2}{\psi_1}$ and $x_2  = \ave{H_1}{\psi_2}$ and $y_2 = \ave{H_2}{\psi_2}$, with both $\psi_1$ and $\psi_2$ normalized. We interpolate the wavefunctions $\psi(p)$ between $\psi_1$ and $\psi_2$, as 
\begin{equation}\label{eq:convex}
    \psi(p) = \sqrt{p} \psi_1 + \sqrt{1-p} e^{i \theta} \psi_2,
\end{equation}
with  $0\leq p \leq 1$, where the phase $\theta$ is chosen such that the expectation values $\left(\av{ H_1}{\psi}, \av{ H_2}{\psi}\right)$ interpolate linearly between points $(x_1, y_1)$  and $(x_2, y_2)$. We find that this requirement is fulfilled if we choose the phase $\theta$ to be of the form 
\begin{equation}
\theta = \arg \left(x_{12} \Delta y  - y_{12} \Delta x  - n_{12} \Delta n \right) + \frac{\pi}{2},
\end{equation}
where $\Delta x \equiv x_2 - x_1$, $\Delta y \equiv y_2 - x_1$ and $\Delta n\equiv y_2 x_1 - x_2 y_1$, and $x_{12} \equiv \langle \psi_1 |  H_1| \psi_2 \rangle$, $y_{12} \equiv \langle \psi_1 |  H_2| \psi_2 \rangle$ and $n_{12} \equiv \langle \psi_1 | \psi_2 \rangle$. Since $\psi(p=0) = \psi_1$ and $\psi(p=1) = \psi_2$, the intermediate value theorem guarantees that any point lying on the line segment between $(x_1,y_1)$ and $(x_2,y_2)$ must have a corresponding $p$ value. This completes our proof of convexity. 

An important consequence of the convexity of $\mathcal{M}$ is that the Gaussian curvature of its boundary $\partial \mathcal{M}$,
\begin{equation}
    \kappa \geq 0, 
\end{equation}
being non-negative wherever it is well-defined. This provides valuable insight into the geometry of the competing operators. It is important to note that this property holds only for the ground state, thus emphasizing the fundamental differences between the ground state, which governs the zero-temperature quantum phase transition, and the excited states that govern the finite-temperature conventional phase transition. 

%{\it Quantum critical point.---} 
The existence of a quantum critical point can be inferred from a singularity on the boundary $\partial \mathcal{M}$. Below we focus on the thermodynamic limit, after both $\Ave{H_1}$ and $\Ave{H_2}$ have been rescaled by the system size. In this limit, the functional form in Eq.~(\ref{eq:ae}) may not be analytic, indicating the presence of a singularity. However, the convexity of $\mathcal{M}$ imposes a constraint on the type of singularity that can occur. For instance, while a cusp may appear for excited states, it cannot exist for ground states.

\begin{figure}
 \includegraphics[width=1\linewidth]{curvature.pdf}
 \caption{ (a) Type I and (b) Type II quantum critical points (QC). The dashed curves represent the first excited states. 
 }. 
 \label{fig:curvature}
\end{figure}


At the critical point, there are two types of singularities on $\partial \mathcal{M}$, and we will first focus on type I. While the boundary $\partial \mathcal{M}$ is convex, the geometry enclosed by Eq.(\ref{eq:ae}) can have a concave region. Figure~\ref{fig:curvature}a illustrates such a phenomenon, which occurs when the ground states (solid curves) connect to the first excited states (dashed curves) at two different points $(\Ave{H_1}^*, \Ave{H_2}^*)$ and $(\Ave{H_1}^{**}, \Ave{H_2}^{**})$, resulting in a gap closing. The convexity property allows the true ground states to be lower than the one predicted by Eq.~(\ref{eq:ae}) by interpolating between these two points (shown by the red line). The concave region corresponds to the excited states, while the ground state boundary remains convex. The critical regions correspond to a single critical value in terms of the parameter $\boldsymbol\lambda_c = (\lambda_{1c}, \lambda_{2c})$, as the entire line segment corresponds to the same normal vector. This leads to non-zero changes in the order parameters $\Delta \Ave{H_1} = \Ave{H_1}^{**} - \Ave{H_1}^*$ and $\Delta \Ave{H_2}= \Ave{H_2}^{**} - \Ave{H_2}^*$, which causes a discontinuous phase transition. In general, the energy change $\Delta E = \lambda_{1c} \Delta \Ave{H_1} + \lambda_{2c} \Delta \Ave{H_2}$ is also discontinuous. However, there are cases where $\Delta E = 0$. This often occurs when the system has additional ground degeneracy at $\boldsymbol\lambda_c$, with $E^* = E^{**}$. Notably, when the boundary falls into the critical region, the curvature vanishes, as
\begin{equation}\label{eq:cur}
    \kappa_c = 0.
\end{equation}
The disappearance of curvature in the critical region is a key feature of the quantum phase transition. This indicates that the boundary $\partial \mathcal{M}$ is becoming tangential to the tangent line of the critical point, signifying that the system is undergoing a significant change in behavior.

Type II singularity has a similar fashion to type I, but the critical region now merges into a single point, as illustrated in Fig.\ref{fig:curvature}b. The zero-curvature condition(\ref{eq:cur}) also holds in this case, with an associated gap closing. However, unlike the type I singularity, in this case, $\Ave{H_1}$ and $\Ave{H_2}$ change continuously, and only their derivatives diverge at the critical points, resulting in a continuous phase transition.

To relate our theory to the Ehrenfest classification, we assume $\lambda_1 > 0$ without loss of generality. Since the parameters are projective, the system can be parameterized by the coupling constant $g = \lambda_2/\lambda_1$. Substituting Eq.~(\ref{eq:E}) into Eq.~(\ref{eq:var}) lead to $\langle H_1 \rangle = E(g) - g E'(g)$ and $\langle H_2 \rangle = E'(g)$, which are the Hellmann–Feynman theorems. Substituting the Gaussian curvature of a curve $\kappa \equiv \frac{\y''\x'-\x''\y'}{(\x'+\y')^{3/2}}$ , we obtain
\begin{equation}
\kappa = - \frac{1}{ (g+1)^{3/2} E''(g)},
\end{equation}
implying that the zero-curvature condition~(\ref{eq:cur}) implies the divergence of the second derivative of the energy. For the non-degenerate case, second-order perturbation theory suggests that $E''(g) = 2 \sum_{k>0} \frac{|\langle k |  H_2 | 0 \rangle |^2}{E_0 - E_k} \leq 0$, in line with the non-negativeness of the curvature. In particular, near the critical point the curvature $\kappa$ scales linearly with the energy gap $\Delta = E_1- E_0$, and vanishes when the gap closes. However, the convexity we proved is more general, even for cases where there is additional degeneracy. 

The type I and II transitions roughly correspond to first- and second-order phase transitions in the Ehrenfest sense. However, there is a fundamental difference between conventional and quantum phase transitions classified here. Unlike conventional phase transitions, quantum phase transitions do not necessarily occur due to broken symmetry. Instead, they result from the competition between non-commuting operators. Our geometric approach places additional constraints on quantum phase transitions. For example, both Type I and II transitions occur only at a fixed normal direction, implying a single critical point, $\boldsymbol\lambda_c$, instead of an intermediate coexistence region as predicted by LGW. While it is possible that multiple transitions exist corresponding to multiple zero-curvature points, a continuous interpolation between two normal directions would imply a non-vanishing curvature in this region.

%{\it Examples.---} 
Below we will apply our theory to several systems. One of the simplest examples of a quantum phase transition is perhaps the one-dimensional transverse field Ising model (TFIM) \cite{pfeuty1970one}, which is described by the Hamiltonian 
\begin{equation}
H_{TFIM} = -J \sum_i Z_i Z_{i+1} - h \sum_i X_i,
\end{equation}
where $X_i$ and $Z_i$ are Pauli matrices that act on the spin variables at site $i$, and $J$ and $h$ are the coupling constants representing the strength of the spin interaction and the transverse magnetic field, respectively. In this case, $ H_1 = - \frac{1}{L} \sum_i Z_i Z_{i+1}$ and $ H_2 = - \frac{1}{L}  \sum_iX_i$, where $L$ is the number of sites. 

The TFIM follows the conventional Landau paradigm of symmetry breaking, where the magnetization $m(g) \equiv - \Ave{H_2}$ serves as the order parameter, where $g \equiv h/J$. The model displays Kramers-Wannier duality, which interchanges $ H_1$ with $ H_2$ and $J$ with $h$, implying that $\Ave{H_1} = - m(1/g)$ represents the magnetization of the dual system. The exact expression of the magnetization 
\begin{equation}
m(g) \equiv \frac{1}{\pi}\int_0^\pi \frac{1+g \cos k}{\sqrt{1+g^2 + 2g\cos k}} dk
\end{equation}
in the thermodynamic limit \cite{pfeuty1970one}. Figure~\ref{fig:ising}a plots $\Ave{H_1}$ versus $\Ave{H_2}$ as the boundary of the moduli space $\mathcal{M}$. The system undergoes a quantum phase transition when $g_c = 1$ (red dot). The Gaussian curvature $\kappa$ can be computed directly as
\begin{equation}
\kappa = \frac{\pi g^2 (g+1)}{\left(g^2+1\right)^{3/2} \left(\left(g^2+1\right) K\left(\frac{4
g}{(g+1)^2}\right)-(g+1)^2 E\left(\frac{4 g}{(g+1)^2}\right)\right)},
\end{equation}
where $K(x)$ and $E(x)$ represent complete elliptic integrals of the first and second kinds, respectively. Figure~\ref{fig:ising}b plots the dependence of $\kappa$ on the coupling constant $g$, showing that it vanishes at the critical point as predicted by Eq.~(\ref{eq:cur}). Additionally, the curvature satisfies the self-dual relation $\kappa(g) = \kappa(1/g)$.

\begin{figure}
 \includegraphics[width=1\linewidth]{TFIsing.pdf}
 \caption{ (a) The moduli space and (b) curvature for both TFIM and TC.
 }
 \label{fig:ising}
\end{figure}

We next turn our attention to the toric code (TC), a model defined on a two-dimensional square lattice that exhibits unconventional topological order \cite{kitaev2003fault,feng2007topological}. The Hamiltonian describing the TC in a uniform magnetic field is given by 
\begin{equation}\label{eq:TC}
 H_{TC} =  -J \left ( \sum_v A_v + \sum_p B_p \right) - h \sum_e Z_e,
\end{equation}
where $A_v \equiv \prod_{e \in v} X_e $ and $B_p \equiv \prod_{e \in p} Z_e$ are stabilizer operators for the spins around each vertex $v$ and plaquette $p$, respectively. Here $X_e$ and $Z_e$ correspond to Pauli matrices acting on the edges of the lattice. The TC model~(\ref{eq:TC}) undergoes a quantum phase transition of the topological orders. In contrast to conventional Landau phase transitions, the topological phase transition in the toric code model cannot be characterized by any local order parameters. Remarkably, the transverse-field Ising model (TFIM) can be transformed into the TC model by mapping the magnetization $\sum_i X_i$ and the spin interaction operators $\sum Z_i Z_{i+1}$ to the operators $\sum_v A_v$ and $\sum_e Z_e$, respectively \cite{kogut1979introduction,zarei2019ising}. The operator $\sum_p B_p$ is related to the anyon excitation and is a c-number for the ground state. However, this mapping is non-local, transforming the local spin operator $X_i$ in the TFIM into a string operator in the TC, which reflects the non-Landau nature of the quantum criticality of the TC.

Despite their fundamental microscopic differences, the TFIM and the TC share an identical non-commutative nature arising from two competing operators $ H_1$ and $ H_2$ for the ground state. As a result, the TC model exhibits the same $\langle H_1\rangle - \langle H_2\rangle$ diagram and curvature function as the TFIM, as shown in Fig.~\ref{fig:ising}. This observation suggests that the quantum criticality may be largely independent of microscopic details and instead is determined by the degree of non-commutativity between competing operators and quantified by the geometry of the boundary $\partial \mathcal M$.

%{\it Integrability.---}
To better understand the zero curvature condition, we investigate the geometry of $\mathcal{M}$ when two operators $H_1$ and $H_2$ commute. However, to provide a more general analysis, we will consider a Hamiltonian of the form 
\begin{equation} \label{eq:Hn}
 H = \lambda_1  H_1 + \ldots + \lambda_n  H_n,
\end{equation}
where $H_i$ are $n$ independent Hermitian operators of a finite-dimensional Hilbert space $\mathcal{H}$ and $\lambda_i$ are real coefficients. This defines an $n$-dimensional real vector space of Hermitian operators. If all elements of the Hamiltonian commute, i.e., $[ H_i, H_j] = 0$ for all $i,j$, the system is completely integrable, with maximal $n = \dim \mathcal{H}$ equal to the dimension of the Hilbert space. Without loss of generality, we may let $H_n = I$, and $\lambda_n = -E$ such that the corresponding Schrödinger equation becomes $H\psi = 0$. As all the operators $H_i$ commute, they can be simultaneously diagonalized as $ H_i = \textrm{diag}(\Lambda_{i,1}, \ldots, \Lambda_{i,N})$, where $\Lambda_{i,j}$ represents the $j$-th eigenvalue of the operator $H_i$. Consequently, $\langle \psi | H_i | \psi \rangle = \sum_{j=1}^N \Lambda_{i,j} |\psi_j|^2$. Let $\tilde{\Lambda}$ be the inverse of the matrix $\Lambda$, then we have $\sum_{j=1}^N \tilde{\Lambda}{i,j} \langle \psi | H_j | \psi \rangle = |\psi_i|^2 \geq 0$. Substituting $\langle \psi | H_n | \psi \rangle = \langle \psi | \psi \rangle$, we obtain a set of $n$ linear inequalities
\begin{equation} \label{eq:simplex}
\sum_{j=1}^{n-1} \tilde{\Lambda}_{i,j} \Ave{H_j} + \tilde{\Lambda}_{i,n} \geq 0,
\end{equation}
where $\Ave{H_j} = \frac{\langle \psi | H_j | \psi \rangle}{\langle \psi  | \psi \rangle}$ denotes the expectation value of $ H_j$. This suggests that $\mathcal M$ is a $n-1$ dimensional simplex, as illustrated in Fig.~\ref{fig:simplex}. The boundary $\partial \mathcal M$ is given by the equality of Eq.~(\ref{eq:simplex}), implying that Eq.~(\ref{eq:ae}) factors into a product of linear combinations of $\Ave{H_i}$. 

\begin{figure}\vspace*{0.5cm}
 \includegraphics[width=0.8\linewidth]{simplex.pdf}
 \caption{ The moduli space  $\mathcal{M}$  (grey domain) for integrable model.} 
 \label{fig:simplex}
\end{figure}

The geometry of $\mathcal{M}$ for any $n < \dim\mathcal{H}$ can be obtained by projecting the simplex~(\ref{eq:simplex}) onto its $n$-dimensional subspace, resulting in a polytope. For instance, when $H_1$ and $H_2$ commute in Eq.~(\ref{eq:H}), the corresponding moduli space $\mathcal{H}$ is a polygon. Remarkably, the curvature $\kappa$ of the polygon is zero almost everywhere except for the vertices. However, for non-commuting operators, one would expect a curved boundary. Nonetheless, the zero-curvature condition implies that the system at the quantum critical point is close to an integrable system, where operators $H_1$ and $H_2$ are nearly commuting. 

%\section{Summary}
%{\it Conclusion.--}
In conclusion, we have presented a new paradigm for understanding quantum criticality distinct from conventional Landau phase transitions. Instead of focusing on microscopic orderings, we argue that the competition of non-commuting operators is the driving force behind the quantum phase transition, which can be best investigated through the boundary geometry of their expectation values. Based on this approach, we show both discontinuous and continuous quantum phase transitions are associated with a vanishing curvature on the moduli boundary, which implies integrability and maximal commuting near the critical point. Our approach presents several avenues for future research and challenges. For instance, it is straightforward to extend our theory from two operators to arbitrary sets of operators, where the zero-curvature condition links to the phase boundary of the multi-dimensional phase diagram. However, the eigenstate moduli have a richer geometry, providing an intriguing avenue for further investigation. Overall, our approach offers a novel perspective on quantum theory, particularly for many-body systems, and has the potential to yield significant implications for strongly correlated systems.

% \begin{acknowledgments}
% C.S. was supported partially by the National Science Foundation under Grants 2150830 and IBSS-1620294, the Institute of Education Sciences under Grant R324A180203, and the National Institutes of Health under Grant R01DC018542. N.F.J. was funded by AFOSR grants FA9550-20-1-0382 and FA9550-20-1-0383.

% \end{acknowledgments}


\bibliography{ref}% Produces the bibliography via BibTeX.



\end{document}
%
% ****** End of file apssamp.tex ******


leading to 
\begin{equation}\label{eq:expansion}
0 = g^t x + \frac{1}{2} x^t H x + O(x^3),
\end{equation}
where $x \equiv (T-T^*, V-V^*)$, $g \equiv \nabla f(T^*,V^*)$ and $H \equiv \nabla \nabla f(T^*,V^*)$ is the Hessian.

if $g$ is not vanishing, one can perform a linear transform $X \equiv g^t x$, and $Y \equiv g_{\perp}^t x$, leading to a parabolic $X = a Y^2$ where $a = g_{\perp}^t H  g_{\perp}$ is not vanishing, otherwise we have $X \sim Y^3$.

If $g = 0$, this case leads to singularities. If $\det H \neq 0$, the hessian should have one positive eigenvalue and one negative, otherwise, the equiation~(\ref{eq:expansion}) can not satisfy. then we have a linear transform $X$ and $Y$ such that $X^2 - Y^2 = 0$, leads to an intersection singularity. This is impossible for the ground states since ground states should be bounded. 

If $\det H = 0$, then we will have $X^2 = Y^3$, where $X$ is the eigenvector is the Hessian, and $Y$ is the corresponding orthogonal variable. 

It is also possible $H = 0$ which leads to higher order singularity. In all, the singularity on the ground state moduli should satisfy
\begin{equation}
    g = 0 \mathrm{\ \& \ } \det H = 0,
\end{equation}
In fact $g= 0$ is sufficient since $H$ cannot have a full rank as we discussed.

a subtle point is the normal vector $(t,U)$ can be still well-defined as the only eigenvector of $H$. In all, the expansion has a very interesting form at $U_c = U/t$, as 
\begin{equation}
f(T,V) \sim \left(t_c \delta T  + U_c \delta V \right)^2 
\end{equation}

We can plot $(g_T, g_V)$ which should approach the origin as $N$ increases.

We define $E = tT + u V$, and $E_* = uT-tV$, regular points
\begin{equation}
\delta E   \sim \delta E_* ^2,
\end{equation}
self-intersection
\begin{equation}
\delta  E^2 \sim \delta  E_* ^2,
\end{equation}
cusps
\begin{equation}
\delta  E^3 \sim \delta  E_* ^2   ,
\end{equation}


\begin{equation}
\delta  E^{2m-1}\sim \delta  E_* ^{2n}, 
\end{equation}

% \begin{figure}
%  \includegraphics[width=1\linewidth]{fig1.pdf}
%  \caption{ Mapping $\rho$ from the Hilbert space $\mathcal{H}$ (grey domain) to the moduli space  $\mathcal{M}$ (dark grey domain). The solid and dashed curves in $\mathcal{H}$ correspond to ground and excited states, where the former maps to the boundary $\partial \mathcal{M}$.
%  }. 
%  \label{fig:map}
% \end{figure}