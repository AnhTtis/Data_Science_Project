\section{INTRODUCTION}

\looseness=-1
Recent advances in hardware and software make autonomous robots more and more important in various environments, from manufacturing and warehouses to healthcare and living spaces.
Learning has been one of the most promising tools for operating robots in such unstructured environments, enabling them to acquire complex perception and reasoning capabilities.
However, the current status quo of designing robots does not account for the impact of learning: rather, many robots are still designed based on human experts' intuition or hand-crafted heuristics.
Therefore, such designs can lead to a sub-optimal performance by causing unexpected visual occlusions.
This is where the idea of guiding robot design to improve the robot learning capability comes in, inspired by the evolutionary process.

\looseness=-1
The evolution of physical attributes through natural selection has played a significant role in the emergence of advanced cognitive abilities among living beings~\cite{manning1998introduction}.
By embracing the idea of evolution, robots also have the potential to evolve their designs for better real-world performance. 
However, it is extremely challenging to encapsulate all the perception, control, and hardware design into a single holistic evaluation pipeline due to the complexity of the components. 
For instance, it will be extremely expensive to formulate a typical two-loop design optimization process, which searches robot morphologies in the outer loop and trains a policy for each given design in the inner loop.
As such, traditional design optimization techniques focus on enhancing certain attributes in isolation or exploring based on hand-designed heuristics.

\begin{figure}
  \centering
  \vspace{3mm}
  \includegraphics[width=0.5\textwidth]{figs/teaser.png}
  \vspace{-6mm}
  \caption{
  A side-by-side comparison of existing vs proposed hardware design optimization approaches.
  }
  \vspace{-5mm}
  \label{fig:teaser}
\end{figure}


\looseness=-1
This work particularly aims to discuss the design optimization for vision-based manipulation.
In the general context of manipulation, visual sensors provide a rich stream of information that allow robots to perform tasks such as grasping, object manipulation, and assembly.
The use of visual sensors in manipulation, however, inevitably poses challenges associated with complete or partial field-of-view occlusion, which can degrade performance due to limited perception. 
Whether an occlusion is harmless or fatal often depends on the task at hand and the stage of task execution.
Such scenarios motivate us to explore hardware optimization without neglecting the interplay between the robot's morphology, onboard perception abilities, and their interaction in different tasks.

\looseness=-1
We present a learning-oriented morphology optimization framework to improve the initial robot design crafted by a human expert.
Our key idea is to develop a \fullpolicynamenobold (\shortname) that is capable of controlling a range of morphologies using onboard visual observations, greatly reducing the costs of traditional two-loop design optimization. 
The controller is trained using a novel \onestagefullname (\onestageshortname) formulation, which unifies the typical two-stage approach of teacher and student training \cite{chen2020learning}.
In our experience, this novel formulation is essential for the student policy to learn behaviors characteristic of its own observation capability and not of a privileged agent.
Once the morphology-agnostic controller is learned, we optimize robot design parameters using Vizier~\cite{google_vizier} optimizer using the controller's performance as a surrogate measure (Fig. \ref{fig:teaser}).
We leverage simulation throughout this work to accelerate the process of looking for an optimal configuration without actually building a physical robot, which is both costly and time-consuming.

\looseness=-1
We evaluate the proposed technique for optimizing the morphology of a mobile manipulator.
We find that our framework can find an improved morphology that shows better performance on overall tasks and facilitates a more sample efficient learning.
Specifically, an optimized design leads to a $15$-$20$\% success rate improvement on various manipulation tasks and is $25$x more data efficient when trained from scratch.
With this work, we would like to highlight the untapped potential of learning-based robot design optimization and show how robot designs can be tailored for better performance and learning with onboard sensing.