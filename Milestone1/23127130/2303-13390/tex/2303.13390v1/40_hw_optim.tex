\section{HARDWARE OPTIMIZATION}\label{sec:hw_optim}

\begin{figure}
  \centering
  \includegraphics[width=0.48\textwidth]{figs/occlusions.png}
  \caption{\textbf{Visual Occlusion Examples} - a selected set of naturally occurring object and workspace occlusions due to the pose and configuration of the robot.}
  \label{fig:occlusions}
\end{figure}

The goal of the hardware optimization is to find a robot morphology that is best fitted for the tasks and workspaces considered.
Unlike typical two-loop morphology optimization frameworks, our key idea is to leverage the morphology-agnostic policy \policyR\ trained only once.
Note that, policy \policyR\ through learning is \textit{primed} to disregard occlusions from onboard observation to generate suitable actions.
Hence, if the \policyR~is successful, then the information is sufficient, meaning that the morphology does not cause any major occlusions that prevent the task success (Fig. \ref{fig:occlusions} shows common types of occlusions).
Such an approach provides a more accurate evaluation of observability and reachability in terms of ``learnability'', and it is substantially more feasible than per-design training.
\newpage
\subsection{Objective Function}
\looseness=-1
The objective function for the optimization is the success rate of the morphology-agnostic policy \policyR~on the task of reaching and/or manipulation.
The success rate of \policyR~serves as a \textit{surrogate} measure of the robot morphology quality.
Each morphology is evaluated $500$ times on a seeded set of reaching tasks, and $200$ times on each of the $3$ manipulation tasks.

\subsection{Optimization Algorithm}
\looseness=-1
We use a black-box optimization algorithm, Vizier~\cite{google_vizier}, to optimize robot design parameters.
Vizier is a sampling-based optimization algorithm that allows us to efficiently search for the optimal solution among a large number of possible morphologies.
We evaluate $50$ morphologies in parallel capping the total number of evaluations at $1000$. 

\subsection{Optimization Space}
\looseness=-1
As an initial design configuration, we use Everyday Robots' manipulator robot~\cite{jang2022bc, saycan2022arxiv, rt12022arxiv}, which features a seven degree-of-freedom arm and a two-finger gripper.
We use this design as a baseline design and refer to it as a \textit{human-expert} design in the upcoming sections.

\looseness=-1
The optimizer is configured to propose link length deltas to the initial robot configuration. We allow for modification of the following robot links for respective ranges (in meters):
\textit{torso}~($-0.3,0.3$),
\textit{shoulder}~($0.0,0.3$),
\textit{bicep}~($-0.05,0.4$),\linebreak
\textit{elbow}~($-0.05,0.3$),
\textit{forearm}~($-0.2,0.3$),
\textit{wrist}~($0.0,0.2$),\linebreak
\textit{gripper}~($-0.05,0.2$).
We show a few random robot morphology configurations on the left of Fig. \ref{fig:architecture}.

