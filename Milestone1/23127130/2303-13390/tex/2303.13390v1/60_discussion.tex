\section{CONCLUSION}
\looseness=-1
In this work, we explore the potential of optimizing robot morphology for improved learning and demonstrate that it can be achieved through a holistic task-oriented optimization process.
We propose a cost-effective method to improve the robot hardware by training a morphology-agnostic surrogate controller and demonstrate that a robot designed with learning considerations can excel at learning compared to a human-expert design.
We introduce a single-stage privileged learning framework that enables rapid acquisition of an onboard policy without artifacts on two-stage privileged learning transfer.
Through our experiments, we show that robot designs with enhanced learning capabilities can improve performance by $15\%$ on manipulation and by $20\%$ on complex manipulation tasks compared to a human-expert robot design.
Moreover, an optimized robot reaches a human-expert design performance level with $25$x less demonstration data. 
We hope this work contributes to the growing trend of learning-based robots and sheds light on opportunities in hardware designs that facilitate better learning.