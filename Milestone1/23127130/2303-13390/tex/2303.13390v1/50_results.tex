\section{EXPERIMENTS}
The results of our experimentation are presented in this section and structured to answer the following questions:

\begin{enumerate}
    \item[A.] Does a morphology-agnostic policy provide a good surrogate of a true performance?
    \item[B.] How does the \ourmorph~ design compare to a human-expert design?
    \item[C.] How does incorporating onboard sensing affects the morphology optimization process?
    \item[D.] What effects do additional robot tasks have on the optimized robot morphology?
\end{enumerate}


\subsection{Agnostic vs Targeted Policies}
First, we show how a surrogate policy performance compares to a true performance on a set of selected robot designs (see Fig.~ \ref{fig:optimized_vs_human}).
We report the performance of the morphology-agnostic controller \policyR\ (Section. \ref{sec:universal_pi}) and targeted controller~\targetedpolicy~(Section.~\ref{subsec:targeted}) on a \textit{human-expert} robot design~\mhuman\ and the best \ourmorph~solution candidate~\mours\ (described in Section~\ref{sec:hw_optim}).

We find that the reaching task performance of the \policyR\ and \targetedpolicy\ is nearly identical for \mours~(difference: $<1\%$).
On the other hand, the performance on \mhuman~design has a gap of $14.44\%$.
We hypothesize that \mours\ causes less occlusion and hence is easier to control using \policyR\ compared to \mhuman.
However, policy \targetedpolicy\; can learn to exploit the structure of occluding arm to infer the missing information, such as a rough pose of an end-effector, through targeted re-training.

\subsection{Comparison to a human-expert design}\label{sec:our_vs_baseline}
\looseness=-1
Next, we analyze the difference between \ourmorph~\mours\ and \textit{human-expert} \mhuman\ designs in terms of task performance and learning efficiency (Fig. \ref{fig:better_learner}).
We highlight the absolute performance improvement of \textit{targeted} policies when evaluated on \mours\ compared to \mhuman\ on the task of reaching.
Specifically, \mours\ achieves success rate of~$80.19$\% compared to~$61.24$\% with \mhuman.

\looseness=-1
The performance gain on \mours\ could be attributed to a reduced frequency of sensor occlusions.
We hypothesize that the reduced distortion of onboard information can also result in a higher quality of data during collection, which naturally leads to improved robot learning capabilities.
To measure the learning efficiency, we compare the success rates of the policies trained with a varying number of demonstrations.
Fig. \ref{fig:better_learner} shows that \mours\ is a better-suited robot for learning compared to~\mhuman, as it requires $25$x less data than \mhuman\ to reach the same performance when training the controller from scratch.

\subsection{Significance of onboard sensing}
\looseness=-1
Next, we seek to investigate the significance of onboard sensing during design optimization.
If the robot policy is given access to all of the information present in the environment, a robot may be able to reach its upper-bound performance, which is mostly constrained by kinematic and dynamic capabilities.
However, it might perform suboptimally when tested with onboard sensing because the morphology can limit its sensing capabilities via visual occlusions.

\looseness=-1
We use a privileged controller \policyP\ during the morphology optimization phase, which acts as an optimal motion planner with the ground-truth robot and task information.
Once the morphology is found, we evaluate its performance with the policy which relies on onboard information \policyR.
In Fig.~\ref{fig:best_priv_vs_best_real}, we compare the performance of privileged information morphology and onboard sensing morphology \mours.
We observe that if we only use the privileged policy during the morphology design, there is a significant $84\%$ drop in performance when the robot is evaluated with onboard sensing. In contrast, \mours\ performs well in both onboard and privileged settings.
This result demonstrates the significance of using onboard sensing during the robot design process.

\subsection{Multitask-based regularization}

\begin{figure}
  \centering
  \vspace{3mm}
  \includegraphics[width=0.49\textwidth]{figs/manipulation_detailed_short.png}
  \caption{\textbf{Various design performance comparison on the manipulation task}.
  Overall \ourmorph\ top solution candidates perform $\sim15\%$ better than \textit{human-expert} design.
  Different optimization candidates show varying levels of performance at different sub-tasks, allowing a human to make a final decision about the performance trade-offs.
  }
  \label{fig:manipulation_stats}
  \vspace{-5mm}
\end{figure}


\begin{figure}
  \centering
  \vspace{10mm}
  \includegraphics[width=0.48\textwidth]{figs/morph_change.png}
  \vspace{-8.5mm}
  \caption{\textbf{Task Complexity Effects on Robot Design} - 
  Optimized link lengths of the robot morphologies tend to converge to more plausible solutions when optimized on more complex tasks.
  }
  \label{fig:morph_regularization}
  \vspace{1mm}
\end{figure}

\looseness=-1
Finally, we investigate the direction of task-based regularization, through the exposure of the robot to a wider set of manipulation tasks.
There exists a large number of regularization approaches applied during the optimization to improve the practicality of the design such as energy, or material penalties~\cite{coello1998using, kouritem2022multi}.
However, we intend to avoid an explicit regularization that can directly impact the final morphology. Instead, we seek implicit regularization via learning on multiple tasks.

We repeat data collection, training, and optimization procedures with manipulation tasks and report the performance of multiple solution candidates in Fig. \ref{fig:manipulation_stats}.
In addition to seeing similar trends in performance improvements of \ourmorph~design \mours\ in manipulation similar to reaching task, we also observe overall improvements in the robot design solutions.
Fig.~\ref{fig:morph_regularization} compares two robot morphologies: one optimized for the reaching task, and one that is optimized for the manipulation task.
Not surprisingly, the type and complexity of the tasks being considered could significantly impact the optimal morphology design.
Hence, including multiple tasks during the optimization process can lead to a more regularized morphology, with shorter and more manageable links that are likely easier to manufacture.