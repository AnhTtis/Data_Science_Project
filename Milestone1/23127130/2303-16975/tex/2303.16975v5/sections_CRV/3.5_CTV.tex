
% \begin{table*}[t]
\small
\centering
\begin{tabular}{l|l}
% \code{ActionQuery} & 78.2 & 45.6 & 70.6 & 75.5\\
\hline
\textbf{Topics}  & \textbf{Tasks}  \\
\hline
Make beverage & Make Jello Shots, Make a Latte, Make Lemonade, Make Irish Coffee \\
Make food & Make Taco Salad, Grill Steak, Make Kimchi Fried Rice, Make Meringue, Make Kerala Fish Curry, Make Butter Pickles \\
Make dessert & Make French Toast, Make Banana Ice Cream, Make French Strawberry Cake, Make Pancakes \\
Fix car & Jack Up a Car, Add Oil to Your Car, Change a Tire \\
\hline
\end{tabular}
 \vspace{1mm}
\caption{\textbf{Task Topics in CTV}. Each task with a shared task topic contains shared action steps.}
\label{table:task_topics}
\end{table*}  


\section{CrossTask Verification (CTV) Dataset}
\label{section:cross_task_construct}

Drawing from the \etv dataset, we introduce CrossTask Verification (CTV) dataset, using videos from the CrossTask dataset~\cite{cross_task}, to evaluate task verification models on real-world videos. In CTV, we prioritize assessing real-world performance of task verification models over systematic study of their generalization capabilities, unlike \etv. Thus, CTV complements \etv dataset -- CTV and \etv together provide a solid test-bed for future research on task verification.


% Example tasks are shown in Table~\ref{table:task_topics}. Notably, some actions, like \textit{add sugar}, are common across tasks such as \textit{Make lemonade} and \textit{Make coffee}. The temporal annotation of each action is available. 
% In CrossTask Verification evaluation, given a task description and the video, the model needs to predict if the task is accomplished (entailed) in the video. 

% \begin{table*}[t]
\centering
\footnotesize{
\begin{tabular}{l|l|l}
    \thead{Query Type} & \thead{Signature} & \thead{Semantics} \\
    \hline
     \code{StateQuery} & \makecell[l]{(Object, State), Video $\mapsto \mathds{P}$} & \makecell[l]{Queries the state (\code{hot}, \code{cold}, \code{clean}, \code{ripe}) of object in a video\\and returns the probability of the object state being detected.\\ Example instructions: \emph{heat an apple, clean a spoon}.} \\
     \hline
     \code{RelationQuery} & \makecell[l]{(Object, Object/Receptacle, Relation), Video $\mapsto \mathds{P}$} & \makecell[l]{Queries the relation between two objects or an object and a\\ receptacle in a video and returns the probability of the relation \\
     being detected. Example instructions: \textit{put apple in basket}, \\ \textit{place spoon to the left of plate}.} \\
     \hline
     \code{ActionQuery} & \makecell[l]{(Subtask, \textsuperscript{$\ast$}Objects, \textsuperscript{$\ast$}Relation), Video $\mapsto \mathds{P}$} & \makecell[l]{Queries for a sub-task with one or more arguments ($\ast$) in a video \\and returns the probability of the sub-task being executed. \\ Example instructions: \emph{whisk mixture, pour lemonade into glass}.} \\
     % \hline
     % \code{AttributeQuery} & \makecell[l]{(Object, Attribute), Video $\mapsto$ Video} & \makecell[l]{Queries for object attributes like shape, color, and size in a video \\and return the video annotated with objects with same attribute. \\This could be further decomposed to \code{\{Color/Shape/Size\}Query}. \\Example instructions: \emph{grab a round ladle}, \emph{slice a green apple}}\\
     % \hline
     % \code{PropertyQuery} & (Property), Video $\mapsto$ Video & \makecell[l]{Queries for properties/affordances of objects in a video and \\ returns the video annotated with objects with same property. \\ Example instructions: \emph{screw wheel, lift the car}}\\ 
     % \hline
     % \code{CountQuery} & (Object, Count), Video $\mapsto$ & \makecell[l]{Queries for multi-item repetitive sub-tasks.\\ Example instructions: \emph{pick two apples, slice three tomatoes}\\ \red{RH: some confusion regarding sub-task repetition, how about} \\
     % \red{``heat 2 tomatoes"}}\\
     \hline
\end{tabular}
\vspace{2mm}
}
\caption{\textbf{NSG's query types for task verification in \etv and CTV.} The query types \code{StateQuery} and \code{RelationQuery} are used in \etv, whereas \code{ActionQuery} is used in CrossTask. Each query type $\tau$ is modeled using a neural network $f^{\theta_\tau}$ accepts unique arguments ($a$) and video frames ($v$) as input and generates an output probability $\mathds{P}=f^{\theta_{\tau}}(a, v)$ of the \textit{query} being true in the video $v$.}
\label{table:QTypes}
\end{table*} 
% Like \etv, CTV consists of paired task descriptions and videos for task verification. We construct two settings: (1) \textbf{Action sequence verification.} NL descriptions are obtained by leveraging action step annotations in CrossTask. The model's objective is to determine whether the action steps are carried out correctly and sequentially. (2) \textbf{Task~verification.} Task class labels are used as descriptions. Here, the model requires a broader understanding of the task as a whole. The dataset construction details are provided in Appendix~\ref{appendix:cross_task_construct}
%%%%%%%%%%%%%%%%%%%%%%%%%%%%%%%%%%%%%%%%%%%%%
\subsection{Dataset Generation}
\label{subsection:cross_task_construct_problem}
\noindent Like \etv, CTV consists of paired task descriptions and videos for task verification. CrossTask has 18 task classes, each with roughly 150 videos, from which we create $\approx$ 2.7K samples. We generate task descriptions by concatenating action step annotations in CrossTask. The model's objective is to determine whether the action steps (sub-tasks) and their sequence in the video align with the description. 
%(2) \textbf{Task~verification.} Task class labels are used as descriptions. Here, the model requires a broader understanding of the task as a whole. 
See Appendix~\ref{appendix:cross_task_construct} for dataset construction details.
% Additional dataset construction details are provided in Appendix~\ref{appendix:cross_task_construct}.
% \begin{figure}[t]
% \centering
%     \includegraphics[width=\linewidth]{plots/cross_task_verification.png}
%     \caption{CrossTask Verification}
%     \label{figure:cross_task_verification}
% % \vspace{-0.3cm}
% \end{figure}

\begin{figure}[t]
\centering
    \includegraphics[width=0.9\linewidth]{plots/cross_task_verification.pdf}
    \caption{\textbf{CrossTask Verification (CTV) dataset.}}
    \label{figure:cross_task_verification}
     % \vspace{-0.3cm}
\end{figure}

%%%%%%%%%%%%%%%%%%%%%%%%%%%%%%%%%%%
% \subsection{Dataset construction}
% We create the following splits: \noindent (1) \textbf{Action sequence verification} involves forming a task description for each video by concatenating a sequence of annotated \textit{action steps} from CrossTask.
% \begin{itemize}[leftmargin=*,noitemsep]
%     \item To ensure repetitive subtasks, as in \etv, we start by selecting the action steps within each task based on frequency. In the end, we select the 4 most frequent action steps with respect to each task. We then filter out video segments in CrossTask that don't contain the top 4 frequent action steps.
%     \item We also followed a similar process of EgoTV for generating negative descriptions. We include three ways of \textbf{negative description} including: (i) \textit{replacing action steps from other tasks}: where we substitute one of the action steps in the sequence to the action step in another task. (ii) \textit{replacing action steps from the same task}: instead of replacing step from another task, we reuse the unused action steps which were not the top 4 frequent steps in the same task. These action steps are closer in semantics and serve as hard negative. (iii) \textit{replace action step sequence order to impossible sequence}: we first list out all possible action step orders in CrossTask following \cite{mao2023action}. Then, we found an action sequence that doesn't exist in CrossTask, e.g., \textit{lower jack, upper jack, break on}. We called these sequence orders impossible since we can't lower the jack when the car is already on the ground. 
%     To make the task more challenging, we also generate \textbf{negative videos} by (i) \textit{shuffling the video order corresponding to each action step}: where the action steps weren't executed in the correct order. (ii) \textit{dropping a video segment corresponding to its action step}: where one action step is missing in the video. Therefore, the task wasn't executed successfully.
% \end{itemize}
% \noindent (2) \textbf{Task verification} is a more challenging setting where we directly use the \textbf{task class} of the video as the task description. Instead of generating negative samples as in \textit{action sequence verification}, the negative samples are from different task classes. We select a negative task class with shared action steps to ensure the model understands the procedure of a certain task instead of simply object-level recognition. We divide the task classes into 4 major topics as shown in Table \ref{table:task_topics}, where each topic contains shared action steps as described in Section \ref{appendix:cross_task_construct_problem}. The video follows the same process as \textit{Action sequence verification} where we select videos with the top 4 action steps.

%%%%%%%%%%%%%%%%%%%%%%%%%%%%%%%%%%%%%%%%%%%%

\subsection{Evaluation}
\noindent \textbf{Metrics.}
Following \etv, we use accuracy and F1 to measure the efficacy of the models on the CTV dataset. %We also evaluate the model's scalability on tasks with varying complexity and ordering.

\noindent \textbf{Generalization.}
%For \textit{action sequence verification}, 
We construct a test set using videos with seen action steps but in previously unseen compositions. To ensure novel compositions, we train on videos with up to 3 action steps and test on those with 4, as illustrated in Figure \ref{figure:cross_task_verification}. While this mirrors the Novel Task split in \etv, the CTV test set also contains unseen visual contexts (videos) -- a result of limited control during dataset creation. %Note that every sub-task in the test set is also present in the training set to ensure that all steps are observed (unseen compositions of seen sub-tasks). 
%For \textit{task verification}, we follow the same training and testing videos in the action sequence verification. The difference is that the task description was substituted from the action sequence to the task class name.
%For \textit{task verification}, similar to the previous setting, we also ensure the task description (task class) where seen and the sub-task composition in the video is unseen. 