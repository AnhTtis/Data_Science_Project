\begin{table*}[t]
\centering
\footnotesize{
\begin{tabular}{l|l|l}
    \thead{Query Type} & \thead{Signature} & \thead{Semantics} \\
    \hline
     \code{StateQuery} & \makecell[l]{(Object, State), Video $\mapsto \mathds{P}$} & \makecell[l]{Queries the state (\code{hot}, \code{cold}, \code{clean}, \code{ripe}) of object in a video\\and returns the probability of the object state being detected.\\ Example instructions: \emph{heat an apple, clean a spoon}.} \\
     \hline
     \code{RelationQuery} & \makecell[l]{(Object, Object/Receptacle, Relation), Video $\mapsto \mathds{P}$} & \makecell[l]{Queries the relation between two objects or an object and a\\ receptacle in a video and returns the probability of the relation \\
     being detected. Example instructions: \textit{put apple in basket}, \\ \textit{place spoon to the left of plate}.} \\
     \hline
     \code{ActionQuery} & \makecell[l]{(Subtask, \textsuperscript{$\ast$}Objects, \textsuperscript{$\ast$}Relation), Video $\mapsto \mathds{P}$} & \makecell[l]{Queries for a sub-task with one or more arguments ($\ast$) in a video \\and returns the probability of the sub-task being executed. \\ Example instructions: \emph{whisk mixture, pour lemonade into glass}.} \\
     % \hline
     % \code{AttributeQuery} & \makecell[l]{(Object, Attribute), Video $\mapsto$ Video} & \makecell[l]{Queries for object attributes like shape, color, and size in a video \\and return the video annotated with objects with same attribute. \\This could be further decomposed to \code{\{Color/Shape/Size\}Query}. \\Example instructions: \emph{grab a round ladle}, \emph{slice a green apple}}\\
     % \hline
     % \code{PropertyQuery} & (Property), Video $\mapsto$ Video & \makecell[l]{Queries for properties/affordances of objects in a video and \\ returns the video annotated with objects with same property. \\ Example instructions: \emph{screw wheel, lift the car}}\\ 
     % \hline
     % \code{CountQuery} & (Object, Count), Video $\mapsto$ & \makecell[l]{Queries for multi-item repetitive sub-tasks.\\ Example instructions: \emph{pick two apples, slice three tomatoes}\\ \red{RH: some confusion regarding sub-task repetition, how about} \\
     % \red{``heat 2 tomatoes"}}\\
     \hline
\end{tabular}
\vspace{2mm}
}
\caption{\textbf{NSG's query types for task verification in \etv and CTV.} The query types \code{StateQuery} and \code{RelationQuery} are used in \etv, whereas \code{ActionQuery} is used in CrossTask. Each query type $\tau$ is modeled using a neural network $f^{\theta_\tau}$ accepts unique arguments ($a$) and video frames ($v$) as input and generates an output probability $\mathds{P}=f^{\theta_{\tau}}(a, v)$ of the \textit{query} being true in the video $v$.}
\label{table:QTypes}
\end{table*} 
%%%%%%%%%%%%%%%%%%%%%%%%%%%%%%%%%%%%%%%%%%%%%%

\section{Neuro-Symbolic Grounding (NSG)}
\label{section:proposed_framework}

\etv requires visual grounding of task-relevant entities such as actions, state changes, etc. extracted from NL task descriptions for verifying tasks in videos. To enable grounding that generalizes to novel compositions of tasks and actions, we propose the Neuro-symbolic Grounding (NSG) approach. NSG consists of three modules:~a)~semantic parser, which converts task-relevant states from NL task descriptions into symbolic graphs,~b)~query encoders, which generate the probability of a node in the symbolic graph being grounded in a video segment, and~c)~video aligner, which uses the query encoders to align these symbolic graphs with videos. NSG thus uses intermediate symbolic representations between NL task descriptions and corresponding videos to achieve compositional generalization.

%%%%%%%%%%%%%%%%%%%%%%%%%%%%%%%%%%%%%%%%%%%%%%%%%
\subsection{Queries for Symbolic Operations}
\label{subsection:queries}

To encode tasks, NSG captures task-relevant visual and relational information in a structured manner via symbolic operators called \emph{queries}.
For instance, the task description \emph{heat an apple} can be symbolically captured by the query: \code{StateQuery(apple, hot)}.
Similarly, the task description \emph{place steak on grill} can be captured by \code{RelationQuery(steak, grill, on)}, which represents the relation (\code{on}) between objects \code{steak} and \code{grill}.
Queries are characterized by \textit{types} and \textit{arguments} and are stored in a text format. Table~\ref{table:QTypes} shows the various query types and their arguments. Different query types capture different aspects, e.g., attributes, relations, etc., thereby enabling a rich symbolic representation of everyday tasks.

\subsection{Semantic Parser for Task Descriptions}
\label{subsection:plan_parsing_from_instructions}

The symbolic operators, i.e., queries, allow the semantic parser to represent a task's partial-ordered steps using a symbolic graph. Specifically, the parser translates a NL task description into a graph $G(V, E)$, where a vertex $n_i \in V$ represents a query and an edge $e_{ij}: n_i \rightarrow n_j \in E$ is an ordering constraint indicating that $n_i$ must precede $n_j$~(Figure~\ref{figure:model-layout}a). We experiment with two different methods to parse language descriptions of tasks to graphs -- (i) finetuning language models and (ii) few-shot prompting of language models. For details, refer to Appendix~\ref{section:appendix_semantic_parsing}. We perform a topological sort with the graph $G$ and generate all the possible sequences of queries consistent with the sort. For example, the topological sorting of the graph in Figure~\ref{figure:model-layout}(a) yields two ordered sequences: $(n_0, n_1, n_2, n_3)$, $(n_0, n_2, n_1, n_3)$. Note that this does not include all physically possible ways to complete a task, but a super-set of all possible sequences of task-relevant queries, including some infeasible sequences\footnote{For instance, in Figure~\ref{figure:model-layout}a, $n_1$ and $n_2$ are at the same topological level, but the sub-task in query $n_1$ could invalidate pre-conditions for $n_2$. Hence, a physically plausible task requires $n_2$ followed by $n_1$ and not vice versa. Note that \etv does not have physically implausible tasks.}. However, this super-set is useful because a task can be verified as accomplished if any sequence in this set can be ascertained to occur in the video. 

Notably, all \etv tasks map to acyclic graphs through temporal disambiguation. While this can support tasks with repeated actions, such as: (Task) \textit{pick two apples}; (Graph) pick(apple) $\rightarrow$ pick(apple); tasks that require (recursively) repeating action sequences until a desired state is reached, might result in cyclic graphs.  Examples include unstacking an arbitrary number of dishes or searching for an ingredient. While currently absent in \etv, extending to such tasks would be a valuable future direction.

\subsection{Query Encoders for Grounding}
\label{subsection:query_encoders}

Query Encoders are neural network modules that evaluate whether a query is satisfied in an input video. Specifically, a query encoder $f^{\theta_{\tau}}$ for a query $n$ of type $\tau$ (e.g., \code{StateQuery}, \code{RelationQuery} etc.), accepts NL arguments ($a$) corresponding to objects and relations in $n$ and a video ($v$) to generate the probability $\mathds{P}=f^{\theta_{\tau}}(a, v)$ of the desired query being true in the video. Learnable parameters corresponding to different query type encoders in an NSG model are jointly represented as $\theta = \bigcup_{\tau}\theta_{\tau}$.

Both the text arguments $a$ of the query and the frames of the input video $v$ are encoded using a pre-trained CLIP encoder~\cite{clip}. The token-level and frame-level representations from CLIP are separately aggregated using two LSTMs~\cite{lstm} to obtain aggregated features for $a$ and $v$, respectively. These features are then fused and passed through the neural network $f^{\theta_{\tau}}$ to obtain the probability $\mathds{P}$ of the query being true in the video (see Figure~\ref{figure:model-layout}a).
%%%%%%%%%%%%%%%%%%%%%%%%%%%%%%%%%%%%%%%%%%%%%%%%%%%%%%%%

% \begin{figure*}[t]
% 	\centering
% 	\begin{subfigure}[t]{0.60\textwidth}
% 		\centering
% 		\includegraphics[width=\linewidth]{plots/model.pdf}
% 		\caption{\red{need caption; dotted lines to denote frozen model parameters.}}
%         \label{figure:model-layout}
% 	\end{subfigure}%
% 	~ 
% 	\begin{subfigure}[t]{0.36\textwidth}
% 		\centering
% 		\includegraphics[width=\linewidth]{plots/constraints.pdf}
% 		\caption{Query alignment constraints [3a]: eq~\ref{Z-a} At most one query per segment; [3b]: eq~\ref{Z-b} All queries must be aligned; [3c]: eq~\ref{Z-c} Ordering constraints between queries obtained from topological sorting. \red{explain the last figure;}}
%         \label{figure:constraints}
% 	\end{subfigure}%
% 	% \caption{}
% 	% \label{figure:model-layout}
% \end{figure*}

\begin{figure*}
\centering
    \includegraphics[width=\linewidth]{plots/model-layout.pdf}
    %\vspace{-10pt}
    \caption{\textbf{NSG model}~(a)~semantic parser converts NL descriptions into a graph $G$ of symbolic queries; query encoders $f^{\theta_{\tau}}$ detect queries in individual video segments $s_t$; and a video aligner aligns $G$ with video segments by computing alignment matrix $\mathrm Z$ via a constrained optimization problem (Eq.~\ref{Z-main}).~(b)~The constraints (Eqs.~\ref{Z-a}~\ref{Z-b}~\ref{Z-c}) and the recursive structure (Eq.~\ref{eq:dp}) enabling use of DP to solve for $\mathrm Z$. Here, the blue box denotes $F^{\ast}((n_j)_j^{N-1}, (s_t)_t^{S-1})$, the green boxes denote $ \log f^{\theta}(a_j, s_t) + F^{\ast}((n_j)_{j+1}^{N-1}, (s_t)_{t+1}^{S-1})$, and the red box denotes $F^{\ast}((n_j)_j^{N-1}, (s_t)_{t+1}^{S-1})$.}
    
    % (a) Semantic Parser parses the language instruction into a graph $G(V, E)$ where vertices represent queries and edges represent ordering constraints. Each query has a \emph{type} and a set of arguments ($a$) and is modeled using a Query Encoder $f^{\theta_{\tau}}$ corresponding to the type. The Query Encoder processes the arguments of the query $a_j$ and a video segment $s_t$ to find the probability of the arguments being true in the video segment $f^{\theta_{\tau}}(a_j, s_t)$. (b) Constraints for DP-based alignment (i) no two queries are aligned to the same segment (Eq.~\ref{Z-a}); (ii) all queries are accounted for in $S$ (Eq.~\ref{Z-b}); (iii) temporal ordering constraints between queries are respected (Eq.~\ref{Z-c}); (iv) visualization of Eq.~\ref{eq:dp}: Here, the blue box denotes $F^{\ast}((n_j)_j^{N-1}, (s_t)_t^{S-1})$, the green boxes denote $ \log f^{\theta}(a_j, s_t) + F^{\ast}((n_j)_{j+1}^{N-1}, (s_t)_{t+1}^{S-1})$, and the red box denotes $F^{\ast}((n_j)_j^{N-1}, (s_t)_{t+1}^{S-1})$. The DP computes the best alignment of the queries and segments in the blue box by computing the $\max$ over the optimal subproblems in the green and red boxes.}
    \label{figure:model-layout}
%\vspace{-10pt}
\end{figure*}
%%%%%%%%%%%%%%%%%%%%%%%%%%%%%%%%%%%%%%%%%%%%%%%%%
\subsection{Video Aligner for Task Verification}
\label{subsection:plan_verification}
% say we formulate as a query alignment problem
This module of NSG must align the graph representation $G$ of the task (generated by the semantic parser) with the video. To that end, it first segments the video, then jointly learns~a)~the query encoders, which detect the queries in the video segments and~b)~the alignment between video segments and the query sequences obtained from the topological sort on $G$. Such joint learning is required since the temporal locations of the queries in the video are unknown a priori requiring simultaneous detection and alignment. If the video is a positive match for the task encoded in $G$, at least one of the query sequences from $G$ must temporally align perfectly with the video segments for successful task verification. Conversely, for negative matches, no query sequence from $G$ would \emph{completely} align with the video segments. Going forward, we use $\langle\rangle$ and $()$ to denote ordered pairs and sequences, respectively.

\noindent \textbf{Video Segmentation:} The video is segmented into non-overlapping segments\footnote{Since pretrained, off-the-shelf video segmentation models are limited to predefined action classes~\cite{escorcia2016daps} or reliant on background frame change detection~\cite{yang2022temporal} and require downstream finetuning~\cite{gao2020accurate}, we leave their integration in NSG as future work.} with a moving window of arbitrary, but fixed size $k$\footnote{If required, the last segment is zero-padded to $k$ frames.} 

\noindent \textbf{Joint Optimization:} The objective of the optimization is to jointly learn the \emph{alignment} $\mathrm{Z}$ between queries and video segments along with the \emph{query encoders} $f^{\theta}$.
% that detect queries in the segments and output a task verification probability for \etv tasks.
% , using only the ground truth label $y$ as supervision.
Given:~a)~the temporal sequence of $S$ segments $(s_t)_{t=0}^{S-1}$ with each $s_t$ spanning $k$ image frames; and~b)~a sequence of $N$ queries $(n_j)_{j=0}^{N-1}$  from the topological sort on $G$, the alignment $\mathrm{Z}$ is defined as a matrix $\mathrm{Z} \in \{0, 1\}^{N \times S}$, where $Z_{jt} = 1$ implies that the $j^{th}$ query $n_j$ is aligned the video segment $s_t$. An example alignment with $N=2$ and $S=3$ is given by the matrix $\mathrm{Z} = \left[ \begin{array}{ccc} 1 & 0 & 0 \\ 0 & 0 & 1 \end{array} \right]$, where the rows are ordered queries $(n_0, n_1)$, the columns are temporal segments $(s_0, s_1, s_2)$, and $\langle n_0, s_0 \rangle$, $\langle n_1, s_2 \rangle$ are the aligned pairs. Assuming segmentation guarantees sufficient segments for query alignment: $S \geq N$. Using $\mathrm{Z}$ and $f^{\theta}$, the task verification probability $p^{\theta}$ can be defined as:
% \vspace{-0.3cm}
\begin{align}
     p^{\theta} = \sigma \bigg(\max_{\mathrm{Z} \in {\{0,1\}}^{N \times S}} \frac{1}{N}\sum_{j,t} \log f^{\theta}(a_j, s_t)Z_{jt}\bigg) \label{eq:p_theta} 
     %\vspace{-0.2cm}
\end{align}

Here $\sigma$ is the sigmoid function, $f^{\theta}(a_j, s_t)$ denotes the probability of querying segment $s_t$ using query $n_j$ with arguments $a_j$ (\S~\ref{subsection:query_encoders}), and $\max$ operator is over the best alignment $\mathrm{Z}$ between $N$ queries and $S$ segments. We use the ground-truth task verification label $y$ to compute $\mathrm{Z}$ and $f^{\theta}$ by minimizing the following loss:
\vspace{-0.2cm}
\begin{align}
     \min_{\theta} & \frac{1}{|\mathcal{D}|} \sum \mathcal{L}_{\text{BCE}}(p^{\theta}, y), \label{eq:bce}
    %\vspace{-0.2cm}
\end{align}

here $|\mathcal{D}|$ is the \etv dataset size and $\mathcal{L}_{\text{BCE}} (\cdot)$ is the binary cross entropy loss computed over $|\mathcal{D}|$ input,~output pairs.~Given the minimax nature of Eq.~\ref{eq:bce}, we use a 2-step iterative optimization process:~(i)~find the best alignment $\mathrm{Z}$ between queries and segments with fixed query encoder parameters $\theta$ (optimize Eq.~\ref{eq:p_theta} with fixed $f^{\theta}$);~(ii)~optimize $\theta$ using Eq.~\ref{eq:bce}, given $\mathrm{Z}$.

%%%%%%%%%%%%%%%%%%%%%%%%%%%%%%%%%%%%%%%%%%%%%%%%%%%%%%%%%

\noindent \textbf{Dynamic Programming (DP)-based Alignment:} Finding the best $\mathrm{Z}$ in Eq.~\ref{eq:p_theta} given $\theta$ requires iterating over combinations of $N$ queries and $S$ segments while respecting certain constraints. The constraints, visualized in Fig.~\ref{figure:model-layout}b, ensure that~a)~no two queries are aligned to the same segment\footnote{This ensures that the order of queries can be verified, which cannot be done when queries belong to the same segment.} (Eq.~\ref{Z-a}),~b)~all queries are accounted for in $S$ (Eq.~\ref{Z-b}), and~c)~the temporal orderings between queries in the query sequences are respected (Eq.~\ref{Z-c}). Specifically, if query $n_u$ precedes $n_v$ ($n_u \rightarrow n_v$), and query $n_v$ is paired with segment $s_{\bar{t}}$ (i.e. $Z_{v\bar{t}}=1$), then query $n_u$ cannot be paired with any segment that lies after $s_{\bar{t}}$ (i.e. $Z_{ut} \neq 1 \; \forall \; t \geq \bar{t}$). The resulting optimization problem for $\mathrm{Z}$, given $\theta$ is:
% \vspace{-0.2cm}
\begin{subequations}\label{Z-main}
\begin{align}
& \max_{\mathrm{Z} \in {\{0,1\}}^{N \times S}} \sum_{j,t} \log f^{\theta}(a_j, s_t)Z_{jt} \tag{\ref{Z-main}}, \quad \text{s.t.}\\
& \sum_{j=0}^{N-1} Z_{jt} \in \{0,1\}, \quad \forall \; 0 \leq t \leq S-1 \label{Z-a}\\
&\sum_{t=0}^{S-1} Z_{jt} = 1, \quad \forall \; 0 \leq j \leq N-1 \label{Z-b}\\
% & n_u \rightarrow n_v, \; Z_{v\bar{t}}=1 \Longrightarrow Z_{ut} \neq 1, \quad \forall \; t \geq \text{argmax}_{\bar{t}} Z_{v\bar{t}} \label{Z-c}
& n_u \rightarrow n_v, \; Z_{v\bar{t}}=1 \Longrightarrow Z_{ut} \neq 1, \quad \forall \; t \geq \bar{t} \label{Z-c}
\end{align}
\end{subequations}

 
Intuitively, the solution to Eq.~\ref{Z-main} gives us the best alignment score (note, the overlap with Eq.~\ref{eq:p_theta}). The iterations over $N$ queries and $S$ segments for solving Eq.~\ref{Z-main} are underpinned by an overlapping and optimal substructure. For instance, to optimally align queries $(n_j)_{j=0}^{N-1}$ and segments $(s_t)_{t=0}^{S-1}$, one could:~a)~pair $\langle n_0, s_0 \rangle$ and optimally align the remaining queries and segments $(n_j)_{j=1}^{N-1}, (s_t)_{t=1}^{S-1}$; or (2) skip $s_0$ and still optimally align \emph{all} queries, now with the remaining segments $(n_j)_{j=0}^{N-1}, (s_t)_{t=1}^{S-1}$ (see Fig.~\ref{figure:model-layout}b(iv)). This recursive substructure leads to a DP solution for Eq.~\ref{Z-main}. 

Let, $F^{\ast}((n_j)_{\bar{j}}^{N-1}, (s_t)_{\bar{t}}^{S-1})$ denote the best alignment score for queries $(n_j)_{\bar{j}}^{N-1}$ and segments $(s_t)_{\bar{t}}^{S-1}$ from Eq.~\ref{Z-main}. Based on the aforementioned reasoning, $F^{\ast}((n_j)_{\bar{j}}^{N-1}, (s_t)_{\bar{t}}^{S-1})$ can be recursively written as: 
% \vspace{-0.2cm}
\begin{multline}
    F^{\ast}((n_j)_{\bar{j}}^{N-1}, (s_t)_{\bar{t}}^{S-1}) = \text{max} \big( \log f^{\theta}(a_{\bar{j}}, s_{\bar{t}}) \\ + F^{\ast}((n_j)_{\bar{j}+1}^{N-1}, (s_t)_{\bar{t}+1}^{S-1}), F^{\ast}((n_j)_{\bar{j}}^{N-1}, (s_t)_{\bar{t}+1}^{S-1}) \big) \label{eq:dp}
\end{multline}

The base cases for the DP are: (i) $\mathrm{Z}=\mathds{I} \; \text{if} \; N=S$; (ii) $Z_{jt} = 1 \; \forall \; t \; \text{if} \; j=N-1$. It is worth noting that the DP subproblems, together with the base cases, satisfy the constraints in Eq.~\ref{Z-a}~\ref{Z-b}~\ref{Z-c}. Since the video may match any of the sequence in the super-set of query sequences (from the topological sort on $G$), we repeat this process of computing $F^{\ast}$ for each sequence and select the maximum value.

%%%%%%%%%%%%%%%%%%%%%%%%%%%%%%%%%%%%%%%%%%%%%%%%%%%%%%%%%5
\noindent \textbf{Optimizing Query Encoder Parameters $\theta$:} After obtaining the best alignment $\mathrm{Z}$ using DP, we substitute the corresponding value of $F^{\ast}((n_j)_{j=0}^{N-1}, (s_t)_{t=0}^{S-1})$ in Eq.~\ref{eq:p_theta} and subsequently Eq.~\ref{eq:bce}. In Eq.~\ref{eq:bce}, we use single mini-batch of training examples and take one gradient-update step of the Adam optimizer for the query encoder parameters $\theta$.

%%%%%%%%%%%%%%%%%%%%%%%%%%%%%%%%%%%%%%%%%%%%%%%%%%%%%%%%%%%%%%%%%%%%%%%%%%%%%%%%%


\begin{table*}[t]
% \small
\centering
\begin{adjustbox}{max width=.95\textwidth}
%\footnotesize{
\begin{tabular}{lccccccccc}
\hline
\multirow{2}{*}{\textbf{Model}}  & \textbf{Visual}  & \textbf{Text} & \textbf{MM} & \textbf{\begin{tabular}[c]{@{}c@{}} Novel  \end{tabular}} & \textbf{\begin{tabular}[c]{@{}c@{}}Novel \end{tabular}} & \textbf{\begin{tabular}[c]{@{}c@{}}Novel\end{tabular}} & \multirow{2}{*}{\textbf{Abstraction}}  &  \multirow{2}{*}{\textbf{Average}} \\ 
  & \textbf{feature} & \textbf{feature}  & \textbf{Fusion} & \textbf{Tasks} & \textbf{Steps} & \textbf{Scenes} & & \\
\hline
Text2text~\cite{roberta} &  & RoBERTa &  & 64.9 & 65.8 & 66.5 & 64.7 & 65.5 \\
\hline
\multicolumn{1}{l}{CLIP Hitchhiker \cite{clip_hitchiker}} & CLIP(I) & CLIP &  & 43.9  & 66.5  & 72.2  & 13.6 & 49.1\\ 
\multicolumn{1}{l}{CLIP4Clip mean~\cite{luo2022clip4clip}}& CLIP(I) & CLIP  &  & 49.3  & 70.9 &  74.9 & 16.1 & 52.8\\
\multicolumn{1}{l}{CLIP4Clip seqLSTM~\cite{luo2022clip4clip}} & CLIP(I) & CLIP  &  &  \underline{56.2} & 73.2  & 74.6  &  17.5 & 55.4\\ 
\multicolumn{1}{l}{CoCa \cite{coca}} & Tx(I) & Tx  &  Y &  51.5 & 71.6  &  71.9 & 43.5 & 59.6 \\ 
\multicolumn{1}{l}{VIOLIN-ResNet~\cite{violin_dataset}} & ResNet(I) & BERT  & Y & 47.4 & \textbf{80.4} & \textbf{85.6} & 42.5 & 64.0\\
\hline
\rowcolor{lightgray}
\multicolumn{1}{l}{MIL-NCE~\cite{miech2020end}} & S3D(V) & Word2vec &  & 30.5 & 69.6 & 73.5 & 24.3 & 49.5\\
\rowcolor{lightgray}
\multicolumn{1}{l}{VideoCLIP~\cite{videoclip}} & Tx(V) & Tx  & Y & 29.3  & 67.6  & 77.9  & 25.6  & 50.1 \\ 
\rowcolor{lightgray}
\multicolumn{1}{l}{VIOLIN-I3D~\cite{violin_dataset}} & I3D(V) & BERT & Y & 45.6 & \underline{79.7} & 83.9 & \underline{47.6} & 64.2 \\
\hline
\multicolumn{1}{l}{\textbf{NSG (ours)}}& CLIP(I) & CLIP & Y & \textbf{90.0} & 64.7 & \underline{84.9} & \textbf{80.4} & \textbf{80.0}\\ 
\multicolumn{1}{l}{Oracle Model} &CLIP(I) & CLIP   & Y &  95.0 & 96.7 & 97.6 & 97.2 & 96.6 \\ 
\hline
\end{tabular}%}
\end{adjustbox}
% \vspace{.2cm}
\caption{\textbf{Comparison of baselines with NSG on different data splits using F1-score.} MM fusion indicates multimodal fusion of vision and text features. %Average represent the average score among the four evaluation scenarios. 
Tx indicates the Transformer as feature extractor with image (I) and video (V) backbones. Video backbone models are highlighted in gray.  \underline{Underline} indicates second-best performance. %\red{RH: to add corresponding Accuracy results in the appendix} \rd{Add gray bg color to one of the blocks to differentiate image vs. video backbones? }
%\bc{We might replace the "Tx" to the actual name, e.g., CLIP, ViT.}
} %Con indicates contrastive loss, Gen indicates generation loss, and Class indicates classification loss.} 
%\vspace{-10pt}
\label{table:baseline_results}
\end{table*}

% \begin{table*}[t]
% % \small
% \centering
% \scriptsize{
% \begin{tabular}{llccccc}
% \hline
% \multicolumn{1}{l|}{\textbf{Model}} & \textbf{Split} & \textbf{\begin{tabular}[c]{@{}c@{}}\nt\end{tabular}} & \textbf{\begin{tabular}[c]{@{}c@{}}\nst\end{tabular}} & \textbf{\begin{tabular}[c]{@{}c@{}}\nc\end{tabular}} & \textbf{Abstraction} \\ 
% \hline
% \multicolumn{2}{l}{MIL-NCE~\cite{miech2020end}} & 39.7 & 70.6 & 74.5 & 26.7\\ 
% \multicolumn{2}{l}{CLIP Hitchhiker \cite{clip_hitchiker}} & 43.9  & 66.5  & 72.2  & 13.6 \\ 
% \multicolumn{2}{l}{CLIP4Clip mean~\cite{luo2022clip4clip}} & 49.3  & 70.9 &  74.9 & 16.1 \\
% \multicolumn{2}{l}{VideoCLIP~\cite{videoclip}} & 29.3  & 67.6  & 77.9  & 25.6   \\ 
% \hline
% \multicolumn{2}{l}{CLIP4Clip seqLSTM~\cite{luo2022clip4clip}} &  56.2 & 73.2  & 74.6  &  17.5 \\ 
% \multicolumn{2}{l}{CoCa \cite{coca}} &  51.5 & 71.6  &  71.9 & 43.5 \\ 
% \hline
% \multicolumn{2}{l}{VIOLIN-ResNet~\cite{violin_dataset}} & 47.4 & 80.4 & 85.6 & 42.5 \\
% \multicolumn{2}{l}{VIOLIN-I3D~\cite{violin_dataset}} & 45.6 & 79.7 & 83.9 & 47.6 \\
% \hline
% \multicolumn{2}{l}{\textbf{NSG (ours)}} & \textbf{90.0} & 64.7 & 84.9 & \textbf{80.4}\\ 
% \multicolumn{2}{l}{Oracle Model} & 95.0 & 96.7 & 97.6 & 97.2\\ 
% \hline
% Socratic~\cite{socratic} & & & & \\
% \hline
% \end{tabular}}
% \caption{Comparison of models (baselines and the proposed model) on different data splits. F1-score.}
% \label{table:baseline_results_new}
% \end{table*}

% \begin{table*}[t]
% \centering
% \scriptsize{
% \begin{tabular}{llccccc}
% \hline
% \multicolumn{1}{l|}{\textbf{Model}} & \textbf{Split} & \textbf{\begin{tabular}[c]{@{}c@{}}\nt\end{tabular}} & \textbf{\begin{tabular}[c]{@{}c@{}}\nst\end{tabular}} & \textbf{\begin{tabular}[c]{@{}c@{}}\nc\end{tabular}} & \textbf{Abstraction} \\ 
% \hline
% \multicolumn{2}{l}{Video Captioning} &   &   &   &   & \\ 
% \hline
% \multicolumn{2}{l}{VIOLIN-ResNet~\cite{violin_dataset}} & 47.4 & 80.4 & 85.6 & 42.5 \\
% %\multicolumn{2}{l}{VIOLIN-CLIP~\cite{clip}} & 38.7 & 85.5 & 90.9 & 83.9 & 44.8 \\ 
% \multicolumn{2}{l}{CLIP~\cite{clip}} & &&&&\\ 
% \multicolumn{2}{l}{CoCa \cite{coca}} &  51.5 & 71.6  &  71.9 & 45.2 \\ 
% \hline
% % \hline
% % \multicolumn{2}{l}{MViT-GloVe} & 64.8 & 88.2 & 92.7 & 88.1 & 65.6 \\ 
% % % \hline
% % \multicolumn{2}{l}{MViT-GloVe-attention} & 57.7 & \textbf{91.0} & \textbf{93.8} & 90.3 & 65.4\\ 
% % % \hline
% % \multicolumn{2}{l}{MViT-BERT} & 58.6 & 83.2 & 92.4 & 88.6 & 62.4 \\ 
% % % \hline
% % \multicolumn{2}{l}{MViT-BERT-attention} & 58.6 & 87.1 & 92.2 & 88.9 & 62.7\\ 
% % \hline
% \multicolumn{2}{l}{VIOLIN-I3D~\cite{violin_dataset}} & 45.6 & 79.7 & 83.9 & 47.6 \\
% %\multicolumn{2}{l}{VIOLIN-MIL-NCE~\cite{miech2020end}} & 48.8  & 76.2  &  84.6 & 83.6 & 46.2\\ 
% \multicolumn{2}{l}{MIL-NCE~\cite{miech2020end}} & &&&&\\ 
% \multicolumn{2}{l}{CLIP hitchHiker \cite{clip_hitchiker}} & 43.9  & 66.5  & 72.2  & 13.6 \\ 
% \multicolumn{2}{l}{CLIP4Clip mean~\cite{luo2022clip4clip}} & 49.3  & 70.9 &  74.9 & 16.1 \\ 
% \multicolumn{2}{l}{CLIP4Clip seqLSTM~\cite{luo2022clip4clip}} &  56.2 & 73.2  & 74.6  &  17.5 \\ 
% \multicolumn{2}{l}{VideoCLIP~\cite{videoclip}} &   &   &   &   &  \\ 
% % 
% % \multicolumn{2}{l}{CLIP4Clip tight \cite{luo2022clip4clip}} &   &   &   &   &  \\ 
% % \multicolumn{2}{l}{CLIP hitchHiker \cite{clip_hitchiker}} &   &   &   &   &  \\ 
% % \hline
% \multicolumn{2}{l}{Video Txformer model (Todo)} &   &   &   &   & \\ 
% \hline
% \multicolumn{2}{l}{\textbf{NSG (ours)}} & \textbf{90.0} & 64.7 & 84.9 & \textbf{80.4}\\ 
% % \multicolumn{2}{l}{\textbf{NSG Prompting (ours)}} &   &   &   &   & \\ 
% % \hline
% \multicolumn{2}{l}{Oracle Model} & 95.0 & 96.7 & 97.6 & 97.2\\ 
% \hline
% \end{tabular}}
% \caption{Comparison of models (baselines and the proposed model) on different data splits. F1-score.}
% \label{table:baseline_results}
% \end{table*}


% this is Acc table
% \begin{table*}[t]
% \small
% \begin{tabular}{llccccc}
% \hline
% \multicolumn{1}{l|}{\textbf{Model}} & \textbf{Split} & \textbf{\begin{tabular}[c]{@{}c@{}}sub-goal \\ composition\end{tabular}} & \textbf{\begin{tabular}[c]{@{}c@{}}verb-noun \\ composition\end{tabular}} & \textbf{\begin{tabular}[c]{@{}c@{}}context-verb-noun \\ composition\end{tabular}} & \textbf{\begin{tabular}[c]{@{}c@{}}context-goal \\ composition\end{tabular}} & \textbf{abstraction} \\ 
% \hline
% \multicolumn{2}{l}{Video Captioning} &   &   &   &   & \\ 
% \hline
% \multicolumn{2}{l}{VIOLIN-Image~\cite{violin_dataset}} & 55.3 & 81.0 & 86.4 & 85.3 & 58.0\\
% \multicolumn{2}{l}{CLIP~\cite{clip}} & 55.4 & 85.9 & 90.9 & 84.9 & 59.9\\ 
% \multicolumn{2}{l}{CoCa \cite{coca}} &   &   &   &   &  \\ 
% \hline
% % \hline
% % \multicolumn{2}{l}{MViT-GloVe} & 64.8 & 88.2 & 92.7 & 88.1 & 65.6 \\ 
% % % \hline
% % \multicolumn{2}{l}{MViT-GloVe-attention} & 57.7 & \textbf{91.0} & \textbf{93.8} & 90.3 & 65.4\\ 
% % % \hline
% % \multicolumn{2}{l}{MViT-BERT} & 58.6 & 83.2 & 92.4 & 88.6 & 62.4 \\ 
% % % \hline
% % \multicolumn{2}{l}{MViT-BERT-attention} & 58.6 & 87.1 & 92.2 & 88.9 & 62.7\\ 
% % \hline
% \multicolumn{2}{l}{VIOLIN-Video~\cite{violin_dataset}} & 56.7 & 80.8 & 87.6 & 84.0 & 62.4\\
% \multicolumn{2}{l}{MIL-NCE~\cite{miech2020end}} & 57.7  & 77.0  &  83.7 & 83.2 & 57.6\\ 
% \multicolumn{2}{l}{CLIP4Clip mean~\cite{luo2022clip4clip}} & 59.1  & 72.3  &   &   &  \\ 
% \multicolumn{2}{l}{CLIP4Clip seqLSTM~\cite{luo2022clip4clip}} &  61.3 & 76.6  & 81.7  & 76.0  &  53.0 \\ 
% \multicolumn{2}{l}{VideoCLIP ~\cite{}} &   &   &   &   &  \\ 
% % 
% % \multicolumn{2}{l}{CLIP4Clip tight \cite{luo2022clip4clip}} &   &   &   &   &  \\ 
% % \multicolumn{2}{l}{CLIP hitchHiker \cite{clip_hitchiker}} &   &   &   &   &  \\ 
% % \hline
% \multicolumn{2}{l}{Video Txformer model (Todo)} &   &   &   &   & \\ 
% \multicolumn{2}{l}{\textbf{NSG (ours)}} & \textbf{90.2} & 66.1 & 93.0 & 84.5 & \textbf{86.5}\\ 
% \multicolumn{2}{l}{\textbf{NSG Prompting (ours)}} &   &   &   &   & \\ 
% % \hline
% \multicolumn{2}{l}{Oracle Model} & 96.4 & 97.4 & 97.0 & 97.4 & 97.0\\ 
% \hline
% \end{tabular}
% \caption{Comparison of models (baselines and the proposed model) on different data splits. The 2 values separated by a "/" denote accuracy and F1-score, respectively.}
% \label{table:baseline_results}
% \end{table*}

%%%%%%%%%%%%%%%%%%%%%%%%%%%%%%%%%%%%%%%%%%%%%%%%%%%%%%%%%%%%%%%%%%%%%%%%%%%%%%%%%%%
%%%%%%%%%%%%%%%%%%%%%%%%%%%%%%%%%%%%%%%%%%%%%%%%%%%%%%%%%%%%%%%%%%%%%%%%%%%%%%%%%%%