\documentclass[10pt,twocolumn,letterpaper]{article}
\makeatletter
\@namedef{ver@everyshi.sty}{}
\makeatother
\usepackage{iccv}
\usepackage{times}
\usepackage{epsfig}
\usepackage{graphicx}

% \usepackage{tcolorbox}

\usepackage{amsmath}
\usepackage{dsfont}
\usepackage{amssymb}
%\usepackage{multicol}
% \usepackage{slashbox}
\usepackage{pifont} % http://ctan.org/pkg/pifont
\newcommand{\xmark}{\ding{55}}%
\newcommand{\cmark}{\ding{51}}%
\usepackage[table,dvipsnames]{xcolor}
\usepackage{tcolorbox}
\usepackage{adjustbox}

\usepackage{tabularx}
% \usepackage{subcaption}
\usepackage{makecell}
\usepackage{multirow}
\usepackage{enumitem}  % for leftmargin

%\usepackage{emoji}

%\usepackage{emoji}

\newcommand{\red}[1]{\textcolor{red}{#1}}
\newcommand{\rd}[1]{\textcolor{blue}{RD: #1}}
\newcommand{\bc}[1]{\textcolor{purple}{BC: #1}}
\newcommand{\nk}[1]{\textcolor{cyan}{NK: #1}}
\newcommand{\aks}[1]{\textcolor{orange}{Akshara: #1}}


\newcommand{\blueD}[1]{\textcolor{blue}{#1}}
\newcommand{\green}[1]{\textcolor{Green}{#1}}
\newcommand{\gray}[1]{\textcolor{darkgray}{#1}}
\newcommand{\yellow}[1]{\textcolor{yellow}{#1}}
\newcommand{\blue}[1]{\textcolor{cyan}{#1}}
\newcommand{\rdn}[1]{\todo[color=orange!20, size=\tiny]{RD: #1}}
\newcommand{\aksn}[1]{\todo[color=orange!20, size=\tiny]{Akshara: #1}}
\newcommand{\inlineimg}[1]{\raisebox{-0.2\baselineskip}{\includegraphics[height=0.95\baselineskip]{#1.png}}}

	
%\definecolor{Gray}{gray}{0.9}
\newcommand{\etv}{EgoTV\xspace}
\newcommand{\nsg}{NSG\xspace}
\newcommand{\nt}{Novel Tasks\xspace}
\newcommand{\nst}{Novel Steps\xspace}
\newcommand{\nc}{Novel Scenes\xspace}

% \definecolor{light-gray}{gray}{0.95}
\newcommand{\code}[1]{\textcolor{darkgray}{\texttt{#1}}}
\newcommand{\query}[1]{\textcolor{blue}{\texttt{#1}}}
\newcommand{\answer}[1]{\textcolor{Green}{\texttt{#1}}}
\newcommand{\T}{\mathbb{T}}

\DeclareMathOperator*{\argmax}{arg\,max}

% Include other packages here, before hyperref.

% If you comment hyperref and then uncomment it, you should delete
% egpaper.aux before re-running latex.  (Or just hit 'q' on the first latex
% run, let it finish, and you should be clear).
\usepackage[breaklinks=true,bookmarks=false]{hyperref}

\iccvfinalcopy % *** Uncomment this line for the final submission

\def\iccvPaperID{1945} % *** Enter the ICCV Paper ID here
\def\httilde{\mbox{\tt\raisebox{-.5ex}{\symbol{126}}}}

% Pages are numbered in submission mode, and unnumbered in camera-ready
\ificcvfinal\pagestyle{empty}\fi

\begin{document}

%%%%%%%%% TITLE
\title{\etv \inlineimg{figures/TV}: Egocentric Task Verification \\from Natural Language Task Descriptions}
% \title{\etv \emoji(tv): Egocentric Task Verification \\from Natural Language Task Descriptions}
% \author{First Author\\
% Institution1\\
% Institution1 address\\
% {\tt\small firstauthor@i1.org}
% % For a paper whose authors are all at the same institution,
% % omit the following lines up until the closing ``}''.
% % Additional authors and addresses can be added with ``\and'',
% % just like the second author.
% % To save space, use either the email address or home page, not both
% \and
% Second Author\\
% Institution2\\
% First line of institution2 address\\
% {\tt\small secondauthor@i2.org}
% }
% \author{%
%     Rishi Hazra$^1$  \quad
%     Brian Chen $^2$  \quad
%     Akshara Rai$^2$  \quad 
%     Nitin Kamra $^2$  \quad
%     Ruta Desai$^2$   \\
%     %\vspace{1mm} \\
%     \small{$^1$Örebro University, $^2$Meta}  \\
%     \small{
%     \texttt{rishi.hazra@oru.se}, \texttt{\{bc2754,nitinkamra,akshararai,rutadesai\}@meta.com}}   
% }

\author{Rishi Hazra\thanks{Work partly done while interning at Meta Reality Labs Research.}\\
Örebro University\\
\vspace{-0.3cm}
\and
Brian Chen\\
Meta Reality Labs Research\\
\vspace{-0.3cm}
\and
Akshara Rai\\
Meta AI Research\\
\vspace{-0.3cm}
\and
Nitin Kamra\\
Meta Reality Labs Research
\and
Ruta Desai\\
Meta Reality Labs Research\\ 
\vspace{-1cm}
\and
\vspace{-1cm}
{\tt\small rishi.hazra@oru.se}, {\tt\small{\{bc2754,nitinkamra,akshararai,rutadesai\}@meta.com}}\\
\vspace{0.7cm}
\and
% \vspace{-1cm}
{\tt\href{https://rishihazra.github.io/EgoTV}{\code{\textbf{EgoTV.github.io}}}}
}

\maketitle
\begin{figure*}[t]
\centering
    \includegraphics[width=\linewidth]{plots/egoTV.pdf}
    \caption{\textbf{\etv benchmark.} A positive example [Left] and a negative example [Right] from the train set along with illustrative examples from the test splits [Bottom] of \etv are shown. The test splits are focused on generalization to novel compositions of tasks, unseen sub-tasks or steps and scenes, and abstraction in NL task descriptions. The bounding boxes are solely for demonstration purposes and are not used during training/inference.}
    
    % are (i) Novel Tasks: unseen compositions of sub-tasks, (ii) Novel Steps: unseen sub-task and target object compositions, and (iii) Novel Scenes: seen tasks in unseen scenes. Additionally, we use the Abstraction split to measure the robustness of the models on abstract task descriptions. Note, that the bounding boxes are solely for demonstration purposes and are not used during training/inference. \aks{I think that the generalization row is a little unclear. Can you remove the train generalization examples, and make all examples a composition of the tasks in the first row? Also can you add a NL description of the generalization task? Finally do we need the underscore ``heat\_then\_clean(apple)", or can we just write ``heat then clean the apple"? In the method section we can describe that we convert the NL description to this template before passing to NSG.}}
    \label{figure:dataset}
%\vspace{-10pt}
\end{figure*}

% Remove page # from the first page of camera-ready.
\ificcvfinal\thispagestyle{empty}\fi

%%%%%%%%%%%%%%%%%%%%%%%%%%%%%%%%%5


%%%%%%%%% ABSTRACT

\begin{abstract}
The metaverse has gained significant attention from various industries due to its potential to create a fully immersive and interactive virtual world. However, the integration of deepfakes in the metaverse brings serious security implications, particularly with regard to impersonation. This paper examines the security implications of deepfakes in the metaverse, specifically in the context of gaming, online meetings, and virtual offices. The paper discusses how deepfakes can be used to impersonate in gaming scenarios, how online meetings in the metaverse open the door for impersonation, and how virtual offices in the metaverse lack physical authentication, making it easier for attackers to impersonate someone. The implications of these security concerns are discussed in relation to the confidentiality, integrity, and availability (CIA) triad. The paper further explores related issues such as the darkverse, and digital cloning, as well as regulatory and privacy concerns associated with addressing security threats in the virtual world.
\end{abstract}
\vspace{-10pt}
%%%%%%%%% BODY TEXT

\section{Introduction}
\label{section:introduction}
%% 1. why should someone care?

%The advent of advanced interactive computer vision systems~\cite{hololens} and recent progress in vision-language and multi-modal models~\cite{} opens doors for such next generation of assistive agents. 
% We envision that the future assistive agents would build up on these visual and language reasoning capabilities of today and empower users to achieve goals in their everyday lives. In particular, such agents would be able to reason about \emph{unseen} human goals... 
% We posit that such agents would require the ability to understand user goals described in natural language at high-level i.e., without complete details about as well as unseen user goals. 

%Recent progress in augmented reality systems~\cite{hololens, magicleap}, as well as vision-language and multi-modal models~\cite{}, opens doors for the next generation of assistive agents. 
Inspired by recent progress in visual systems~\cite{MagicLeap, ungureanu2020hololens}, we consider an assistive egocentric agent capable of reasoning about daily activities. When invoked via natural language commands, for e.g., while baking a cake, the agent understands the steps involved in baking, tracks progress through the various stages of the task, detects and proactively prevents mistakes by making suggestions. Such an agent would empower users to learn new skills and accomplish tasks efficiently.
% One could envision invoking such an agent merely through natural language descriptions of tasks similar to how present day assistants such as Alexa, Siri etc.~\cite{voice_assistants} are invoked. 
%We envision such agents to empower users in daily life by  invoking them naturally through 

%% 2. Why is it challenging? 
%While recent progress in vision-language and multi-modal models~\cite{} opens doors for such next generation of assistive agents, various challenges remain in making such agents a reality. 
%To make such agents a reality, 

Developing such an egocentric agent capable of tracking and verifying everyday tasks based on their natural language specification is challenging for multiple reasons. First, such an agent must reason about various ways of doing a \emph{multi-step} task specified in natural language. This entails decomposing the task into relevant actions, state changes, object interactions as well as any necessary causal and temporal relationships between these entities. Secondly, the agent must ground these entities in egocentric observations to track progress and detect mistakes. Lastly, to truly be useful, such an agent must support tracking and verification for a combination of tasks and, ideally, even unseen tasks. These three challenges -- causal and temporal reasoning about task structure from natural language, visual grounding of sub-tasks, and compositional generalization -- form the core goals of our work.

% %% 3. What are we doing? What is our approach?
% \aks{I think this is a matter of preference, but I personally don't like related work in intro. I would make this paragraph be about EgoTV and NSG. Starting with something like - "To this end, we propose...", ie, your next paragraph.}
% \nk{+1, we should move parts of this para to lit review and delete the rest.}
% Recent research on language modeling enables decomposing tasks into multiple steps from natural language descriptions~\cite{llm_zero_shot_planning,proscript}. However, such \emph{task decompositions} cannot directly be leveraged for task tracking in egocentric agents because of lack of grounding into the visual observations or context. In parallel, the computer vision community has advanced action recognition~\cite{}, object detection and tracking~\cite{}, hand object interaction and object state change detection~\cite{ego_4d,change_it,}, step classification in procedural tasks~\cite{}, and even vision language reasoning~\cite{nsvqa,nscl,star_situated_reasoning,clevrer}, which may help with the grounding challenge. However, majority of current research on identifying actions, objects, steps, or state changes does not account for the overall task structure. Likewise, predominant research on vision language understanding~\cite{} and multi-modal grounding~\cite{} does not consider the temporal and causal constraints that emerge in task tracking and verification. We therefore focus on the order-aware visual grounding problem in our work, with an eye towards compositional generalization to scale usability of these agents. In particular, we aim to achieve visual grounding of the actions and objects corresponding to each step or sub-task obtained from the task description decomposition in an order-aware manner.

%% 4. What are our results/contributions?
As our first contribution, we propose a benchmark -- \emph{\textbf{Ego}centric \textbf{T}ask \textbf{V}erification} (\etv \inlineimg{figures/TV}) -- and a corresponding dataset in the AI2-THOR~\cite{ai2thor} simulator. % \emoji{tv}
Given a natural language (NL) task description and a corresponding egocentric video of an agent, the goal of \etv is to verify whether the task was successfully completed in the video or not.
\etv contains multi-step tasks with \emph{ordering} constraints on the steps and \emph{abstracted} NL task descriptions with omitted low-level task details inspired by the needs of real-world assistants. We also provide splits of the dataset focused on different generalization aspects, e.g., unseen visual contexts, compositions of steps, and tasks (see Figure~\ref{figure:dataset}).
% Next, we create splits of the dataset focused on different aspects of generalization, ranging from generalization to unseen visual context to unseen compositions of steps and tasks. Figure~\ref{figure:dataset} shows an example task and overview of generalization splits from \etv. Succeeding at \etv tasks requires decomposing tasks into partially-ordered steps from the NL description and order-aware visual grounding of these steps into the video. 

Our second contribution is a novel approach for order-aware visual grounding~--~\emph{\textbf{N}euro-\textbf{S}ymbolic \textbf{G}rounding} (NSG), capable of compositional reasoning and generalizing to unseen tasks owing to its ability to leverage abstract NL descriptions and compositional structure of tasks (task decomposition, ordering).~In contrast, state-of-the-art vision-language models~\cite{coca,clip,videoclip,clip_hitchiker} struggle to ground NL descriptions in egocentric videos, and do not generalize to unseen tasks.~NSG outperforms these models by~$\mathbf{33.8}\%$~on compositional generalization and~$\mathbf{32.8}\%$~on abstractly described task verification. Finally, to evaluate \nsg on real-world data, we instantiate \etv on the CrossTask~\cite{cross_task} instructional video dataset. %Specifically, we synthetically create videos with mistakes in CrossTask. 
We find that it also outperforms state-of-the-art models at task verification on CrossTask. We hope that the \etv~benchmark and dataset will enable future research on egocentric agents capable of aiding in everyday tasks.

% We experiment with many for the \etv tasks. We find that while these models generalize well to unseen visual context, they struggle to perform grounding from abstracted task descriptions and to generalize to new compositions of tasks. To deal with these challenges, we take inspiration from recent research on and develop . ~\rd{unclear why neurosymbolic models would do well on abstraction.} 

% To summarize, our main contributions are:~1)~\etv: a benchmark and synthetic dataset to systematically study egocentric task verification.
% 2)~\nsg: a novel neuro-symbolic approach to enable the core reasoning capability for \etv -- order-aware visual grounding. We demonstrate \nsg's capability on our synthetic \etv dataset as well as a real-world dataset derived from CrossTask. We will release both of these datasets and our models for future research on egocentric task tracking and verification. 


% Assistive agents require the ability to track actions and state changes from an egocentric perspective for effective assistance in day-to-day tasks. For example, an agent helping a user prepare a recipe would need to both generate the steps of the recipe (\textit{plan generation}) and track the user's actions to ensure the plan is executed correctly (\textit{plan verification}). We formulate this as a Video Entailment task~\cite{violin_dataset,9710490} \rd{should we call our task video-based goal entailment?}, wherein, given an egocentric video of an agent (or human) performing a task (\textit{premise}) and a NL task description (\textit{hypothesis}), the objective is to learn a model to track whether the given task was successfully executed in the video. 
% An ideal model should also be able to seamlessly generalize to novel compositions (of actions and objects) unseen during training. \rd{add a line about what we mean by abstraction and why is it important.} To this end, we generate a novel Vision-Language dataset on the AI2-THOR simulator~\cite{ai2thor} to study compositional and abstraction-based generalization. Our dataset provides effective evaluation measures in a controlled setting, while closely reflecting the diversity of real-world events. We implement and train a variety of end-to-end models based on existing state-of-the-art approaches. We empirically demonstrate that neural models suffer from overfitting and cannot effectively generalize to novel compositions of actions, objects, and scenes. 
% To address this problem, we propose an end-to-end Neuro-Symbolic (NeSy) framework that performs plan generation and verification. At the heart of our approach is the hypothesis that symbolic reasoning models are good at generalization and capturing compositional substructure, while neural models are good at learning representations from sensory data~\cite{10.5555/3326943.3327039,nscl,clevrer}. \rd{summarize contributions in a bulleted list.} \rd{also add a line about the main result e.g., x\% improvement as compared to end-to-end models}. 

% \rd{we also evaluate NeSy with real-world data: add briefly about CrossTask experiments.}

% % \fbox{\begin{minipage}{\linewidth}
% % \textbf{Problem Statement}

% % Given:
% % (i) Premise: Egocentric video of an agent performing a task.
% % (ii) Hypothesis: NL description of the task.

% % Learn: A model to track whether the premise entails the hypothesis. The output of the model is True if the given task is executed successfully in the video.
% % \end{minipage}}

% \textbf{Contributions:} 
% \begin{itemize}
%     \item We generate a benchmark video-language dataset to study compositional and abstraction-based generalization.
%     \item We evaluate the performance of a variety of state-of-the-art models and show that these (baseline) models cannot effectively generalize to novel compositions of actions.
%     \item We propose a novel end-to-end NeSy approach that significantly outperforms the baselines on some compositional generalization splits while performing on par with them on the rest.
%     \item We also evaluate our NeSy approach with real-world data showing similar performance improvements.
% \end{itemize}

\section{Related Work}
\label{sec:related_work}
\subsection{Co-Speech Gesture Synthesis}
The early approaches for generating co-speech gestures often involve creating linguistic rules to translate speech input into a sequence of pre-collected gesture segments, which are typically referred to as rule-based methods \cite{cassell1994rulefullbody,cassell2001beat,kipp2004gesture,kopp2006bml}. \citet{wagner2014rulereview} provide a comprehensive review of these methods. Rule-based methods produce interpretable and controllable results, but creating gesture datasets and rules requires significant effort. To alleviate the manual effort of designing rules in rule-based methods, data-driven approaches have gradually become predominant in this field. \citet{nyatsanga2023data_driven_gesture_survey} offer a thorough survey of these methods. Early data-driven approaches aim to directly learn mapping rules from data through statistical models \cite{neff2008videogesture,levine2009prosodygesture,levine2010gesturecontroller} and combine them with predefined gesture units for gesture generation. Later, the powerful modeling capability of deep neural networks makes it possible to train complex end-to-end models using raw speech-gesture data directly. One option is deterministic models, such as MLP \cite{kucherenko2020gesticulator}, CNN \cite{habibie2021videogesture}, RNN \cite{yoon2019robot,yoon2020trimodalgesture,bhattacharya2021affectivegesture,liu2022hierarchicalgesture}, and Transformer \cite{bhattacharya2021text2gestures}. Another choice is generative models, including flow-based models \cite{alexanderson2020stylegesture,ye2022styleflowgesture}, VAEs \cite{li2021audio2gesture,ghorbani2022zeroeggs}, and VQ-VAE \cite{yi2022talkshow,yazdian2022gesture2vec,liu2022vqgesturevideo}. Due to the inherent many-to-many relationship between speech and gesture, end-to-end models can generate natural-looking gestures but face challenges in ensuring content matching between speech and generated gestures \cite{yoon2022genea}. To address this issue, some neural systems aim to explicitly model both rhythm and semantics from the perspective of model structure \cite{kucherenko2021speech2properties2gestures,ao2022rhythmicgesticulator,liu2022disco} or training supervision strategy \cite{liang2022seeg}. Furthermore, hybrid systems, such as the combination of deep features and motion graphs \cite{zhou2022gesturemaster}, have been proposed to harness the advantages of different approaches. Recently, diffusion models \cite{sohldickstein2015diffusion,song2020improvedscore,ho2020ddpm} have demonstrated impressive results in image synthesis \cite{ramesh2022dalle2} and human motion generation \cite{tevet2022humanmotiondiffusion, zhang2022motiondiffuse}. Inspired by these works, our system adapts the latent diffusion model \cite{rombach2022latentdiffusion} for the co-speech gesture generation task and achieves appealing results.

\subsection{Style Control for Human Motion}
A typical approach to style control for human motion involves specifying a motion clip as a reference and transferring the reference clip's style to the source motion. This task is also known as \emph{style transfer}. Early works in motion style transfer integrate traditional machine learning techniques with manually defined features to infer motion styles \cite{hsu2005motion_style_translation,ma2010motion_style_transfer,xia2015realtime_motion_style_transfer,yumer2016spectral_motion_style_transfer}. Recently, deep learning-based methods have significantly enhanced motion quality. \citet{holden2016deepmotion} first propose a learning framework enabling motion style control through optimization in the motion manifold space. \citet{du2019stylemotioncvae} improve transfer efficiency by training a conditional VAE. \citet{mason2018few-shot_motion_style_transfer} use few-shot learning to generate stylized locomotion. \citet{aberman2020adain} employ a temporally invariant adaptive instance normalization (AdaIN) layer for target style injection, eliminating the need for paired data during training. \citet{wen2021stylemotionflow} achieve unsupervised style transfer using a flow model. \citet{jang2022motionpuzzle} introduce a method capable of controlling styles for individual body parts.

Previous co-speech gesture synthesis systems with style control can be categorized based on whether or not they require style labels. For methods needing labeled data, early works can only learn an individual style for one generator \cite{levine2010gesturecontroller,neff2008videogesture,ginosar2019stylegesture}. \citet{ahuja2022lowresource} propose a strategy that efficiently adapts the source generator to another speaker style using low-resource data. Some works learn a speaker style embedding space with labeled speaker-motion data, enabling gesture style control by sampling from this space \cite{ahuja2020stylegesture,yoon2020trimodalgesture,bhattacharya2021affectivegesture}. \citet{alexanderson2020stylegesture} aimat controlling fine-grained styles, such as gesturing speed and spatial scope, using preprocessed control signal-motion data. Their later work \cite{alexanderson2022diffusiongesture} utilizes a diffusion model for audio-driven motion synthesis, achieving label-based style control by training the model on labeled data. For methods not requiring style labels, \citet{habibie2022motionmatching} propose a motion matching framework to achieve flexible style control. Other studies achieve arbitrary style control by imitating an example given as a video \cite{liu2022hierarchicalgesture} or a motion clip \cite{ghorbani2022zeroeggs,ye2022styleflowgesture,kuriyama2022tokenizedgestures}.  In this work, we utilize a CLIP-based encoder to extract a style embedding from an arbitrary text prompt and incorporate it into the generator via an AdaIN layer, guiding the synthesis of stylized gestures. Our system supports fine-grained multimodal style prompts as opposed to label-based style control. It employs a self-supervised learning scheme and eliminates the need for labeled data. Additionally, we use an autoregressive model rather than a parallel model, making it potentially suitable for real-time applications.
% \input{tables/Tab 1 (abridged)}

\section{\etv Benchmark and Dataset}
\label{section:dataset}
%We posit that the future egocentric assistants would empower users to learn new tasks and to perform known tasks better via task tracking, verification, and mistake detection. To make progress towards such agents, 
We present the \emph{\textbf{Ego}centric \textbf{T}ask \textbf{V}erification} (\etv) benchmark and dataset.
%
% \subsection{Criteria for \etv Benchmark}
To enable task tracking and verification for real-world egocentric agents, \etv contains:~1)~\emph{multi-step} tasks with \emph{ordering constraints} to capture the causal and temporal nature of everyday tasks,~2)~\emph{multimodality} -- language in addition to the egocentric video to allow language-based human-agent interaction. 
% ~3)~\emph{abstracted} task descriptions, since humans may invoke the agents with partial details about the task. \rd{should we remove abstraction from here since it is a split of the dataset rather than the main benchmark task?}
\etv also provides data for a systematic study of generalization in task verification (see Table~\ref{table:list_of_datasets_abridged}). To this end, we create the \etv dataset using a photo-realistic simulator AI2-THOR~\cite{ai2thor} -- as a rich testbed for future research on generalizable agents for task tracking and verification. We also provide a reduced study of \etv on real-world data using the CrossTask dataset~\cite{cross_task}.

% \etv tasks require grounding of both objects and actions. Furthermore, both causal and compositional reasoning of spatiotemporal (video) and multimodal (video, language) information is needed. These characteristics make \etv unique amongst existing vision-language and video benchmarks and datasets (see Table~\ref{table:list_of_datasets_abridge} for a detailed comparison). 
% \aks{I think that this paragraph should be in the related work, pointing the reader to the difference between prior benchmarks and EgoTV.}

\subsection{Definitions}
\noindent \textbf{Benchmark.} The objective is to determine if a task described in natural language has been correctly executed by the agent in a given egocentric video.
% It can be viewed as a video-based entailment problem where the task description is the hypothesis and the agent's video is the premise, and the premise must entail the hypothesis.% One can view the task description as a hypothesis and the agent's video as a premise. \etv benchmark can thus be considered a video-based entailment problem, where the premise must entail the hypothesis. 
% \aks{I think at this point we have mentioned this many times, and don't need to repeat here.}

\noindent \textbf{Tasks.} 
% In real-world tasks, ordering between steps might arise because of physics of the world e.g., need to pick up a knife before slicing, or semantics of the task e.g., need to slice vegetables before frying. At the same time, certain steps may be executed in any order. Consequently, 
Each task in \etv consists of multiple \emph{partially-ordered sub-tasks} or steps. A sub-task corresponds to a single object interaction via one of the six actions:~\emph{heat, clean, slice, cool, place, pick}, and is parameterized by a~\emph{target} object of interaction\footnote{Except the \emph{place} sub-task, which is additionally parameterized by a \textit{receptacle} object, we currently limit our \etv dataset to sub-tasks involving only a single target object.}. By using the ``actionable" properties of objects in AI2-THOR~\cite{ai2thor}, we ensure that the sub-tasks are parameterized with appropriate target objects in \etv, e.g., \emph{heat(book)} will never occur.

Real-world tasks consist of sub-tasks with ordering constraints, either due to physical restrictions (e.g., picking up a knife before slicing) or task semantics (e.g., slicing vegetables before frying).
We allow \etv tasks to be partially ordered, with some steps following strict ordering, e.g.~\emph{pick} sub-task happens before \emph{place} sub-task, while others are order-independent.
%While \etv sub-tasks respect physics-based ordering , we do not attempt to capture semantic ordering constraints in \etv for practical ease of synthetically creating tasks in AI2-THOR. Consequently, a task in \etv could contain a \emph{cool} sub-task followed by a \emph{heat} sub-task, which is less likely to occur in the real world \red{RH: I made sure to not use cool and heat together}. \nk{As in, did you remove them from the dataset or you just avoided using videos with both heat and cool together?} \red{RH: that's allowed in the simulator, I just avoided generating such cases in the dataset} 
The ordering constraints between sub-tasks are captured in the task description using specifiers such as \emph{and}, \emph{then}, and \emph{before/after}. For ease of description, we will refer to a task in the paper using $\left< \text{\emph{sub-task}} \right> \_ \left< \text{\emph{ordering-specifier}} \right> (object)$ notation, irrespective of the actual task description. An example task from \etv: ~\emph{heat\_then\_clean(apple)} is shown in Fig.~\ref{figure:dataset} with its NL description: ``apple is heated, then cleaned in a sinkbasin". The task consists of two ordered sub-tasks: heat $\rightarrow$ clean on \emph{target}: apple.
% \aks{Give actual NL description, then say how that is encoded in the dataset. There is no NL description in the paper so far, so it appears like the tasks are specified in this way: heat\_then\_clean\_then\_slice(apple)}
% \vspace{-5pt}
\subsection{Dataset} As shown in Fig.~\ref{figure:dataset}, \etv dataset consists of (task description, video) pairs with positive or negative task verification labels. By combining the six sub-tasks~\emph{heat, clean, slice, cool, put, pick} with different ordering constraints, we create 82 tasks for \etv (see Appendix~\ref{appendix:dataset_analysis_and_statistics} for an exhaustive list). Tasks are instantiated with 130 target objects (excluding visual variations in shape, texture, and color) and 24 receptacle objects, totaling 1038 task-object combinations. These are performed in 30 different kitchen scenes. We also provide comprehensive annotations for each video, including frame-by-frame breakdowns for sub-tasks, object bounding boxes, and object state information (e.g.,~\textit{hot, cold, etc.}) to facilitate future research.

\subsubsection{Generation}

%%% #TODO: move the action details to appendix
% Our dataset is generated using the AI2-THOR simulator~\cite{ai2thor}, which is built on Unity 3D engine. It consists of 3D indoor scenes, where AI agents can navigate and interact with objects using 7 high-level actions and 12 low-level actions\footnote{\textbf{high-level actions}: GotoLocation, PickupObject, PutObject, SliceObject, CleanObject, HeatObject, CoolObject; \textbf{low-level actions}: LookUp, LookDown, MoveAhead, RotateRight, RotateLeft, OpenObject, CloseObject, ToggleObjectOn, ToggleObjectOff, SliceObject, PickupObject, PutObject.}. 

%It not only serves as a meaningful proxy for real-world events, providing realistic physics simulations and mirroring the diversity and complexity of real-world events but also provides a fully-controlled setting to host complex reasoning tasks and effective diagnostics. 

% \rd{revisit/explain the goal for our entailment task before diving into the dataset. Also explain what is our criteria for the tasks in the dataset e.g., multisteps, ordering important etc. Also we use the word ``task" to refer to entailment problem as well as tasks in dataset, which is confusing right now. May be we can refer to the later as ``goals"?}

% \rd{I recommend re-organizing this section.\\
% 1. Start with the goal of the entailment task. \\ 
% 2. Desiderata of various ``goals" that need to be entailed.
% 2a. ingredients of a ``goal": goal type, parameterization i.e., objects, ordering type, and actions: verb-noun.\\
% 2b. Sample definition: video + hypothesis \\
% 3. Generation process with Ai2thor using planner \\
% 4. Dataset stats and the concept of splits. This will then be a nice segway to Sec. 3.1.\\
% Also don't want to create confusion between low-level actions vs. high-level actions. So recommend describing them only as an implementation detail for dataset generation. We could perhaps refer to high-level actions as ``steps".}

% \textbf{Sample Definition:} Our objective is to determine if a task described in natural language (hypothesis) has been correctly executed in a given egocentric video of an agent performing the task (premise) -- i.e. whether the premise entails the hypothesis. As shown in Figure~\ref{figure:dataset}, we use a dataset consisting of samples of (premise, hypothesis) pairs, where each premise is annotated with both a positive and negative hypothesis. Thus, for each video, we get two samples. Moreover, each sample is associated with a \textit{task type} (here, \textit{heat\_then\_clean\_then\_slice}) and a \textit{target object} (here, \textit{apple}) which enumerates the main actions in the task (heat, clean, slice) and their order of execution (heat $\prec$ clean $\prec$ slice).

\noindent \textbf{Task-video Generation.} We generate the videos in our dataset by leveraging the ALFRED setup~\cite{ALFRED20}. ALFRED allows us to specify the \etv tasks using Planning Domain Definition Language (PDDL) and then to generate plans for achieving these tasks using the Metric-FF planner~\cite{metric_ff}. We execute these plans using the AI2-THOR simulator and obtain their corresponding videos. Further details on encoding tasks using PDDL and planning are in Appendix~\ref{appendix:etv_taskvideogen}.

% To enhance the realism of the videos, a set of plans are generated by the planner rather than just the best plan. This allows us to leverage the partial-ordered nature of tasks in our dataset to generate multiple, valid task execution trajectories for a given task. For instance, 

% introduction of redundant actions and results in stochasticity in the plans.

% We generate data by mapping multiple tasks to the same goal definition. For example tasks \textit{heat\_then\_clean(apple)} and \textit{heat\_and\_clean(apple)} have the same goal definition (a \textit{hot, clean} apple), still, the actions must be executed in a specific order for the first task (\textit{heat(apple)} $\prec$ \textit{clean} apple) and in any order for the second task. The videos are generated by enacting plans generated by a planner based on the goal definition, and by modifying the preconditions of the actions to enforce ordering -- 

\noindent \textbf{Task-description Generation.} We convert the plans generated for each task into positive and negative task descriptions using templates. Appendix~\ref{appendix:task_templates} provides details on the process and example templates. 
% \nk{I suggest moving the dataset generation subsection to the appendix to save space. Retain only stats and evaluation subsections.}

% We define a set of templates for each task. The positive (hypothesis) templates take the form of a task (or goal) definition, while the negative (hypothesis) templates are created by either altering the sequence of high-level actions in the positive template or replacing some of them with alternative actions. 

\begin{figure}[t]
\centering
    \includegraphics[width=\linewidth]{plots/data_stats.pdf}
    \caption{\textbf{\etv dataset.} Sub-tasks and tasks, including their difficulty measures (Sec:~\ref{section:evaluation}) are shown per split. Novel Scenes have more tasks since all the train tasks are repeated in unseen scenes. Likewise, complexity and ordering are higher in Novel Tasks due to the addition of unseen sub-tasks.}
    % \caption{\etv dataset statistics. (a) Distribution of tasks [Inner Level] and sub-tasks [Outer Level] in each split. Among test splits, Novel Scenes have more tasks since all the train tasks are repeated in unseen scenes. (b) Distribution of task complexity and ordering in each split. Complexity and ordering is higher in Novel Tasks split.}
    \label{figure:stats}
%\vspace{-10pt}
\end{figure}



% The dataset consists of 82 tasks (see Table~\ref{tab:list_of_tasks} in Appendix for an exhaustive list of tasks) that are performed in 30 different kitchen scenes, featuring a diverse range of objects (over 130 objects, not including visual variations). 

%We provide comprehensive annotations for each video, including frame-by-frame breakdowns of high-level actions, low-level actions, objects bounding boxes, and relevant state information (\textit{hot, cold, etc.}). Note, that this information is not available to the models during training. Refer \S~\ref{appendix:dataset_analysis_and_statistics} in Appendix for a detailed analysis of the train and test splits.

% This allows us to create a wide number of tasks in our dataset. 

%  While different task types may have the same goal definition, they may require the execution of high-level actions in a specific order. 

% for dataset generation: shall we just refer to the ALFRED paper?

%=====================================================================

\subsubsection{Evaluation} 
\label{section:evaluation}

\noindent \textbf{Metrics.}
We use accuracy and F1 to measure the efficacy of models on \etv task verification benchmark. To capture the difficulty of tracking and verifying tasks, we introduce two measures:~(1)~\emph{Complexity}: measuring the number of sub-tasks in a task, which impacts the video length and requires higher action and object grounding, and~(2)~\emph{Ordering}: measuring the number of ordering constraints in a task and measures the difficulty of temporal reasoning required to track and verify tasks. We evaluate model scalability by testing on tasks with varying complexity and ordering.

% We evaluate model performance for \etv tasks with varying degrees of complexity and ordering so as to tease apart the scalability of various models with increase in difficulty of these tasks.

% The ordering axis provides a measure of the number of ordering constraints in the task ($ord = \#$ ordering constraints in the task). All models are evaluated along these axes. For example, $(com, ord)$, of some of the tasks are given as (i) heat\_simple: (1, 0), (ii) heat\_and\_clean: (2, 0), (iii) heat\_then\_clean: (2, 1), (iv) heat\_then\_clean\_then\_slice: (3, 2). \rd{previous sentence is confusing. Readers would not get that you are referring to co-ordinates on a x-y graph.} A list of tasks grouped along complexity and ordering axes can be found in Table~\ref{tab:list_of_tasks} in Appendix.

\noindent \textbf{Generalization.}
Our synthetic dataset facilitates a systematic exploration of generalization in task tracking and verification via four test splits that focus on generalization to novel steps, tasks, visual contexts/scenes, and abstract task descriptions.

% An ideal model must be able to seamlessly generalize to novel compositions of actions and objects unseen during training. To this end, we introduce 5 generalization splits (\textit{test set}), as shown in Figure~\ref{figure:dataset}:

\begin{itemize}[leftmargin=*,noitemsep]
    \item \textbf{\nt}: Unseen compositions of seen sub-tasks. For e.g., if train set is \{\textit{clean(apple)},~\textit{cool(apple)}\}, then this test split would contain tasks like: \{\textit{clean\_and\_cool(apple)},~\textit{clean\_then\_cool(apple)}, \textit{ cool\_then\_clean(apple)}\}.
    
    % \textit{heat\_then\_place(egg)}\}. \rd{is this correct, Rishi? \red{RH: the last one is incorrect. although that is possible, it would be easier to detect heat\_then\_place when you have already seen them together in some other context. Thus, in our case-- given a set of sub-tasks to execute a task, at least two of them should never have appeared in the same context regardless of the ordering between them, for the task to be in Novel Tasks split} Also, what about heat then place (apple) task? \red{RH: if train doesn't have heat(apple) or place(apple), then this would be novel steps, yes} Is that considered under the novel step split instead?}

    \item \textbf{\nst}: Unseen compositions of sub-task actions and target objects. For e.g., if the train set is \{\textit{clean(apple)},~\textit{cool(egg)},~\textit{clean\_and\_cool(tomato)}\}, then this test split would contain tasks like: \{\textit{clean(egg)},~\textit{cool(apple)},~\textit{clean\_and\_cool(apple)}\}.
    
    % ,~\textit{heat\_and\_place(apple)}\}. \red{RH: feels a bit confusing -- novel steps is a novel subTask\_tarObj composition, so heat\_and\_place(apple) would be novel steps if either heat(apple) or place(apple) or both has never appeared in the context of any train task. Here it sounds like novel task\_tarObj composition instead. We could make this more crisp?}

    % \item \textbf{context-verb-noun composition} split: Compositions of verb (high-level action) and noun (target object) in novel contexts (scenes). For instance, train set contains tasks \textit{clean(knife)} in kitchen scenes 1-20, whereas the test set contains tasks \textit{clean(knife)} in kitchen scenes 21-25.

    \item \textbf{\nc}: This test split contains the same tasks as in the train set. However, the tasks are executed in unseen kitchen scenes.

    \item \textbf{Abstraction}: Abstract task descriptions, which lack the low-level details of the task in the descriptions. For instance, for a \textit{heat\_and\_clean(apple)} task, the full task description in the train set could be ``apple is heated in a microwave and cleaned in sink basin", while the abstract task description in this split could be ``apple is heated and cleaned".
\end{itemize}

Note that all the test splits and the train set are disjoint from each other. \nst split tests an \etv model's ability to understand generalizable object affordances and tool usage. For instance, once a model learns the \emph{slice} action on an apple, this split tests if the model can apply it to an orange. On the other hand, the \nt split tests the generalization of a model's causal reasoning capabilities on unseen compositions and orderings of known sub-tasks.

\subsubsection{Statistics}
\etv dataset consists of 7,673 samples (train set: 5,363 and test set: 2,310). The split-wise division is \nt: $540$, \nst: $350$, \nc: $1082$, Abstraction: $338$. The total duration of the egocentric videos in the \etv dataset is 168 hours, with an average video length of 84 seconds. The task descriptions consist of 9 words on average, with a total vocabulary size of 72. On average, there are 4.6 sub-tasks per task in the \etv dataset, and each sub-task spans approximately 14 frames. Additionally, there are 2.4 ways to verify a task. Figure~\ref{figure:stats} shows a comparison of train and test splits (more analysis in Appendix~\ref{appendix:dataset_analysis_and_statistics}).

% %====================================================================
% \subsection{Dataset Analysis and Statistics}
% \label{subsection:dataset_analysis_and_statistics}

%====================================================================
% \begin{figure*}[t]
% \centering
%     \includegraphics[width=\linewidth]{plots/model.pdf}
%     \caption{\red{need caption; dotted lines to denote frozen model parameters. [just a placeholder]}.}
%     \label{figure:model-layout}
% % \vspace{-0.3cm}
% \end{figure*}
%=====================================================================
% % % \input{figures/complexity-ordering}

\begin{figure}[t]
\centering
    \includegraphics[width=\linewidth]{plots/com-ord-comparison.png}
    \caption{\textbf{NSG maintains consistent performance as task complexity and ordering difficulty increases.} F1-score of NSG vs. best-performing baseline for \etv tasks with varying complexity and ordering are shown.}
    \label{figure:complexity-ordering}
%\vspace{-10pt}
\end{figure}

\begin{figure}[t]
\centering
    \includegraphics[width=\linewidth]{plots/com-ord-comparison.png}
    \caption{\textbf{NSG maintains consistent performance as task complexity and ordering difficulty increases.} F1-score of NSG vs. best-performing baseline for \etv tasks with varying complexity and ordering are shown.}
    \label{figure:complexity-ordering}
%\vspace{-10pt}
\end{figure}


% \begin{figure*}[t]
% 	\centering
% 	\begin{subfigure}[t]{0.60\textwidth}
% 		\centering
% 		\includegraphics[width=\linewidth]{plots/model.pdf}
% 		\caption{\red{need caption; dotted lines to denote frozen model parameters.}}
%         \label{figure:model-layout}
% 	\end{subfigure}%
% 	~ 
% 	\begin{subfigure}[t]{0.36\textwidth}
% 		\centering
% 		\includegraphics[width=\linewidth]{plots/constraints.pdf}
% 		\caption{Query alignment constraints [3a]: eq~\ref{Z-a} At most one query per segment; [3b]: eq~\ref{Z-b} All queries must be aligned; [3c]: eq~\ref{Z-c} Ordering constraints between queries obtained from topological sorting. \red{explain the last figure;}}
%         \label{figure:constraints}
% 	\end{subfigure}%
% 	% \caption{}
% 	% \label{figure:model-layout}
% \end{figure*}

\begin{figure*}
\centering
    \includegraphics[width=\linewidth]{plots/model-layout.pdf}
    %\vspace{-10pt}
    \caption{\textbf{NSG model}~(a)~semantic parser converts NL descriptions into a graph $G$ of symbolic queries; query encoders $f^{\theta_{\tau}}$ detect queries in individual video segments $s_t$; and a video aligner aligns $G$ with video segments by computing alignment matrix $\mathrm Z$ via a constrained optimization problem (Eq.~\ref{Z-main}).~(b)~The constraints (Eqs.~\ref{Z-a}~\ref{Z-b}~\ref{Z-c}) and the recursive structure (Eq.~\ref{eq:dp}) enabling use of DP to solve for $\mathrm Z$. Here, the blue box denotes $F^{\ast}((n_j)_j^{N-1}, (s_t)_t^{S-1})$, the green boxes denote $ \log f^{\theta}(a_j, s_t) + F^{\ast}((n_j)_{j+1}^{N-1}, (s_t)_{t+1}^{S-1})$, and the red box denotes $F^{\ast}((n_j)_j^{N-1}, (s_t)_{t+1}^{S-1})$.}
    
    % (a) Semantic Parser parses the language instruction into a graph $G(V, E)$ where vertices represent queries and edges represent ordering constraints. Each query has a \emph{type} and a set of arguments ($a$) and is modeled using a Query Encoder $f^{\theta_{\tau}}$ corresponding to the type. The Query Encoder processes the arguments of the query $a_j$ and a video segment $s_t$ to find the probability of the arguments being true in the video segment $f^{\theta_{\tau}}(a_j, s_t)$. (b) Constraints for DP-based alignment (i) no two queries are aligned to the same segment (Eq.~\ref{Z-a}); (ii) all queries are accounted for in $S$ (Eq.~\ref{Z-b}); (iii) temporal ordering constraints between queries are respected (Eq.~\ref{Z-c}); (iv) visualization of Eq.~\ref{eq:dp}: Here, the blue box denotes $F^{\ast}((n_j)_j^{N-1}, (s_t)_t^{S-1})$, the green boxes denote $ \log f^{\theta}(a_j, s_t) + F^{\ast}((n_j)_{j+1}^{N-1}, (s_t)_{t+1}^{S-1})$, and the red box denotes $F^{\ast}((n_j)_j^{N-1}, (s_t)_{t+1}^{S-1})$. The DP computes the best alignment of the queries and segments in the blue box by computing the $\max$ over the optimal subproblems in the green and red boxes.}
    \label{figure:model-layout}
%\vspace{-10pt}
\end{figure*}

\section{Neuro-Symbolic Grounding (NSG)}
\label{section:proposed_framework}

\etv requires visual grounding of task-relevant entities such as actions, state changes, etc. extracted from NL task descriptions for verifying tasks in videos. To enable grounding that generalizes to novel compositions of tasks and actions, we propose the Neuro-symbolic Grounding (NSG) approach. NSG consists of three modules:~a)~semantic parser, which converts task-relevant states from NL task descriptions into symbolic graphs,~b)~query encoders, which generate the probability of a node in the symbolic graph being grounded in a video segment, and~c)~video aligner, which uses the query encoders to align these symbolic graphs with videos. NSG thus uses intermediate symbolic representations between NL task descriptions and corresponding videos to achieve compositional generalization.

% Given a NL task description and the corresponding video, the \etv benchmark requires a
% % a binary prediction of whether the task is accomplished in the video. To that end, a 
% model to a) understand the relevant objects, actions, and their orderings from the task description and b) ground these objects and actions in the video while respecting the ordering constraints, for verifying appropriate task execution in the video. 
% To facilitate extraction of task-relevant concepts from NL descriptions in a manner conducive to visual grounding and generalization to novel tasks, we propose Neuro-Symbolic Grounding (NSG) approach. 
% Furthermore, these reasoning capabilities must generalize to novel tasks with unseen compositions of actions and objects. 

% The semantic parser and the aligner are enabled by two ingredients: DSL to define symbolic operators, Neural encoder
% leverages Domain-specific language (DSL) to capture task-relevant information in a structured, symbolic manner. 
% Specifically, using DSL, NSG parses the task instructions into a partially-ordered symbolic graph?

% True to its name, NSG leverages a Domain-specific language (DSL) to define \emph{symbolic} operators that capture task-relevant information in a structured manner. The corresponding \emph{neural} counterpart in NSG are encoders, which evaluate the symbolic operators on continuous representations e.g., video embeddings. We first describe these ingredients before diving into the aforementioned modules of our NSG approach.
% oops sorry let me take care of this
% \vspace{-15pt}
\subsection{Queries for Symbolic Operations}
\label{subsection:queries}

To encode tasks, NSG captures task-relevant visual and relational information in a structured manner via symbolic operators called \emph{queries}.
For instance, the task description \emph{heat an apple} can be symbolically captured by the query: \code{StateQuery(apple, hot)}.
Similarly, the task description \emph{place steak on grill} can be captured by \code{RelationQuery(steak, grill, on)}, which represents the relation (\code{on}) between objects \code{steak} and \code{grill}.
Queries are characterized by \textit{types} and \textit{arguments} and are stored in a text format. Table~\ref{table:QTypes} in the appendix shows the various query types and their arguments. Different query types capture different aspects, e.g., attributes, relations, etc., thereby enabling a rich symbolic representation of everyday tasks.

\subsection{Semantic Parser for Task Descriptions}
\label{subsection:plan_parsing_from_instructions}

The symbolic operators, i.e., queries, allow the semantic parser to represent a task's partial-ordered steps using a symbolic graph. Specifically, the parser translates an NL task description into a graph $G(V, E)$, where a vertex $n_i \in V$ represents a query and an edge $e_{ij}: n_i \rightarrow n_j \in E$ is an ordering constraint indicating that $n_i$ must precede $n_j$~(Figure~\ref{figure:model-layout}a). We experiment with two different methods to parse language descriptions of tasks to graphs -- (i) finetuning language models and (ii) few-shot prompting of language models. Appendix~\ref{section:appendix_semantic_parsing} describes these approaches in detail. We perform a topological sort with the graph $G$ and generate all the possible sequences of queries consistent with the sort. For example, the topological sorting of the graph in Figure~\ref{figure:model-layout}(a) yields two ordered sequences: $(n_0, n_1, n_2, n_3)$, $(n_0, n_2, n_1, n_3)$. Note that this does not include all physically possible ways to complete a task, but a super-set of all possible sequences of task-relevant queries, including some infeasible sequences\footnote{For instance, in Figure~\ref{figure:model-layout}a, $n_1$ and $n_2$ are at the same topological level, but the sub-task in query $n_1$ could invalidate pre-conditions for $n_2$. Hence, a physically plausible task requires $n_2$ followed by $n_1$ and not vice versa. Note that \etv does not have physically implausible tasks.}. However, having this super-set is useful because a task can be verified as accomplished if any sequence in this set can be ascertained to occur in the video.

\subsection{Query Encoders for Grounding}
\label{subsection:query_encoders}

Query Encoders are neural network modules that evaluate whether a query is satisfied in an input video. Specifically, a query encoder $f^{\theta_{\tau}}$ for a query $n$ of type $\tau$ (e.g., \code{StateQuery}, \code{RelationQuery} etc.), accepts NL arguments ($a$) corresponding to objects and relations in $n$ and a video ($v$) to generate the probability $\mathds{P}=f^{\theta_{\tau}}(a, v)$ of the desired query being true in the video. Learnable parameters corresponding to different query type encoders in an NSG model are jointly represented as $\theta = \bigcup_{\tau}\theta_{\tau}$.

Both the text arguments $a$ of the query and the frames of the input video $v$ are encoded using a pre-trained CLIP encoder~\cite{clip}. The token-level and frame-level representations from CLIP are separately aggregated using two LSTMs~\cite{lstm} to obtain aggregated features for $a$ and $v$, respectively. These features are then fused and passed through the neural network $f^{\theta_{\tau}}$ to obtain the probability $\mathds{P}$ of the query being true in the video (see Figure~\ref{figure:model-layout}a).

%=====================================================================
% moved plan generation to the appendix (label{appendix_semantic_parsing})
%=====================================================================

\subsection{Video Aligner for Task Verification}
\label{subsection:plan_verification}
% say we formulate as a query alignment problem
This module of NSG must align the graph representation $G$ of the task (generated by the semantic parser) with the video. To that end, it first segments the video, then jointly learns~a)~the query encoders, which detect the queries in the video segments and~b)~the alignment between video segments and the query sequences obtained from the topological sort on $G$. Such joint learning is required since the temporal locations of the queries in the video are unknown a priori requiring simultaneous detection and alignment. If the video is a positive match for the task encoded in $G$, at least one of the query sequences from $G$ must temporally align perfectly with the video segments for successful task verification. Conversely, for negative matches, no query sequence from $G$ would \emph{completely} align with the video segments. Going forward, we use $\langle\rangle$ and $()$ to denote ordered pairs and sequences, respectively.

% We next explain the video segmentation and DP-based approach in detail.

% Such alignment between queries and segments is challenging since the temporal locations of the queries in the video are unknown apriori. Hence, queries must be localized in the video, even if in an implicit manner, for successful alignment. We next explain the video segmentation and DP-based alignment in detail.

% Given the super-set of query sequences from the sort, the NSG model must implicitly localize these queries in the video segments since the temporal locations of the queries in the video are not known for the task but are nevertheless essential for task verification. If the video is indeed a positive sample for the graph $G$ derived from the task specification, then at least one of the query sequences from $G$ must temporally align perfectly with the video segments for successful task verification. Vice versa, for negative samples, no query sequence from the graph $G$ would \emph{completely} align with the video segments.

\noindent \textbf{Video Segmentation:} The video is segmented into non-overlapping segments\footnote{Since pretrained, off-the-shelf video segmentation models are limited to predefined action classes~\cite{escorcia2016daps} or reliant on background frame change detection~\cite{yang2022temporal} and require downstream finetuning~\cite{gao2020accurate}, we leave their integration in NSG as future work.} with a moving window of arbitrary but fixed size $k$\footnote{If required, the last segment is zero-padded to $k$ frames.} 
% and obtain a temporal sequence of $S$ segments $\langle s_t | 0 \leq t \leq S-1 \rangle$.
% Once we have the graph representation $G$ for the task comprising of queries on its nodes, 


% \noindent \textbf{Query Encoder:} Next we must learn to identify if a query node $n$ is indeed being performed in a video segment $s$. We do this by training neural network modules. Specifically, a query $n$ with type $\T(n)=\tau$ (e.g., \code{StateQuery}, \code{RelationQuery} etc.) is executed by a neural network $f^{\theta_{\tau}}$. It accepts NL arguments ($\textit{args}$) e.g., objects and relations in the query $n$, and a video segment ($s$), to generate the probability $\mathds{P}=f^{\theta_{\tau}}(\textit{args}, s)$ of the desired query being true in the video segment. Both the text arguments of the query and the $k$-frames of the video $s$ are encoded using a pre-trained CLIP encoder~\cite{clip}. The resultant token-level and frame-level features are separately aggregated using two LSTMs~\cite{lstm} to obtain segment and sentence features, respectively, which are then fused and passed through the neural network $f^{\theta_{\tau}}$ to obtain the probability of the query being true in the video segment (see Figure~\ref{figure:model-layout}).

% Learnable parameters corresponding to different query types are jointly represented as $\theta = \bigcup_{\tau}\theta_{\tau}$. Going forward, for a query $n_j$ and segment $s_t$, the corresponding notation is simplified to $\mathds{P} = f^{\theta}(a_j, s_t)$, where $a_j$ is short for $args_j$, and the subscript $j$ uniquely determines the query encoder (i.e. \code{StateQuery, RelationQuery}). 

\noindent \textbf{Joint Optimization:} The objective of the optimization is to jointly learn the \emph{alignment} $\mathrm{Z}$ between queries and video segments along with the \emph{query encoders} $f^{\theta}$.
% that detect queries in the segments and output a task verification probability for \etv tasks.
% , using only the ground truth label $y$ as supervision.
Given:~a)~the temporal sequence of $S$ segments $(s_t)_{t=0}^{S-1}$ with each $s_t$ spanning $k$ image frames; and~b)~a sequence of $N$ queries $(n_j)_{j=0}^{N-1}$  from the topological sort on $G$, the alignment $\mathrm{Z}$ is defined as a matrix $\mathrm{Z} \in \{0, 1\}^{N \times S}$, where $Z_{jt} = 1$ implies that the $j^{th}$ query $n_j$ is aligned the video segment $s_t$. An example alignment with $N=2$ and $S=3$ is given by the matrix $\mathrm{Z} = \left[ \begin{array}{ccc} 1 & 0 & 0 \\ 0 & 0 & 1 \end{array} \right]$, where the rows are ordered queries $(n_0, n_1)$, the columns are temporal segments $(s_0, s_1, s_2)$, and $\langle n_0, s_0 \rangle$, $\langle n_1, s_2 \rangle$ are the aligned pairs. Assuming segmentation guarantees sufficient segments for query alignment: $S \geq N$. Using $\mathrm{Z}$ and $f^{\theta}$, the task verification probability $p^{\theta}$ can be defined as:
\begin{align}
     p^{\theta} = \sigma \bigg(\max_{\mathrm{Z} \in {\{0,1\}}^{N \times S}} \frac{1}{N}\sum_{j,t} \log f^{\theta}(a_j, s_t)Z_{jt}\bigg) \label{eq:p_theta} 
     %\vspace{-0.2cm}
\end{align}
Here $\sigma$ is the sigmoid function, $f^{\theta}(a_j, s_t)$ denotes the probability of querying segment $s_t$ using query $n_j$ with arguments $a_j$ (Sec.~\ref{subsection:query_encoders}), and $\max$ operator is over the best alignment $\mathrm{Z}$ between $N$ queries and $S$ segments. We use the ground-truth task verification label $y$ to compute $\mathrm{Z}$ and $f^{\theta}$ by minimizing the following loss:
\begin{align}
     \min_{\theta} & \frac{1}{|\mathcal{D}|} \sum \mathcal{L}_{\text{BCE}}(p^{\theta}, y), \label{eq:bce}
    %\vspace{-0.2cm}
\end{align}
here $|\mathcal{D}|$ is the \etv dataset size and $\mathcal{L}_{\text{BCE}} (\cdot)$ is the binary cross entropy loss computed over $|\mathcal{D}|$ input,~output pairs.~Given the minimax nature of Eq.~\ref{eq:bce}, we use a 2-step iterative optimization process:~(i)~find the best alignment $\mathrm{Z}$ between queries and segments with fixed query encoder parameters $\theta$ (optimize Eq.~\ref{eq:p_theta} with fixed $f^{\theta}$);~(ii)~optimize $\theta$ using Eq.~\ref{eq:bce}, given $\mathrm{Z}$.

% Let $S =\left[s_1,\dots,s_p\right]$ denote the temporal sequence of segments, with each $s_i$ spanning $k$ image frames and $N= \left[n_0, \dots, n_q\right]$ be the sequence of node queries from the topological sort on $G$. The alignment between $N$ and $S$ is encoded using an alignment matrix $\mathrm{Z} \in \{0, 1\}^{p \times q}$, where $z_{jt} = 1$ implies that the $j^{th}$ query $n_j$ is aligned with the video segment $s_t$. An example alignment with $p=2$ and $q=3$ is given by the matrix $\mathrm{Z} = \left[ \begin{array}{ccc} 1 & 0 & 0 \\ 0 & 0 & 1 \end{array} \right]$, where the rows are ordered queries $\left[n_0, n_1\right)$, the columns are temporal segments $\left[s_0, s_1, s_2\right]$ and $(n_0, s_0)$, $(n_1, s_2)$ are the aligned pairs. We assume that segmentation ensures $p \geq q$. \red{RH: add why}

% For simplicity, we explain the alignment of a single sequence of queries with video segments and generalize later to the super-set of all query sequences from the topological sort on $G$. 

% Given: (i) the temporal sequence of $S$ segments $\langle s_t | 0 \leq t \leq S-1 \rangle$ with each $s_t$ spanning $k$ image frames; and (ii) a sequence of $N$ queries $\{n_0, \dots, n_N\}$  from the topological sort on $G$, we define the alignment matrix $\mathrm{Z} \in \{0, 1\}^{N \times S}$, where $z_{jt} = 1$ implies that the $j^{th}$ query $n_j$ is aligned the video segment $s_t$. An example alignment with $N=2$ and $S=3$ is given by the matrix $\mathrm{Z} = \left[ \begin{array}{ccc} 1 & 0 & 0 \\ 0 & 0 & 1 \end{array} \right]$, where the rows are ordered queries $\{n_0, n_1\}$, the columns are temporal segments $\{s_0, s_1, s_2\}$ and $(n_0, s_0)$, $(n_1, s_2)$ are the aligned pairs. We assume that segmentation is done in a way that $S \geq N$. \red{RH: add why}

% The alignment is done by minimizing the binary cross-entropy loss with respect to the query model parameters $\theta$, given input $x_i$ (video, task description) and output $y_i$ (verification label): 
% \begin{align}
%      \min_{\theta} & \frac{1}{M} \sum_i \mathcal{L}_{\text{BCE}}(p^{\theta}(x_i), y_i) \label{eq:bce}\\
%      \text{where, }& p^{\theta}(x_i) = \sigma \bigg(\frac{1}{N}\sum_{j,t} \log n_j^{\theta}(s_t)Z_{jt}^{\ast}\bigg) \label{eq:p_theta}  
% \end{align}
% \begin{align}
%      \min_{\theta} & \frac{1}{M} \sum_i \mathcal{L}_{\text{BCE}}(p^{\theta}(x_i), y_i) \label{eq:bce}\\
%      \text{here, }& p^{\theta}(x_i) = \sigma \bigg(\max_{\mathrm{Z} \in {\{0,1\}}^{N \times S}} \frac{1}{N}\sum_{j,t} \log f^{\theta}(a_j, s_t)Z_{jt}\bigg) \label{eq:p_theta}  
% \end{align}

% \nk{Between eqs 2 and 3, we are introducing Z* for no reason. Can we simply get rid of the argmax in eq 3 and Z* in eq 2, and instead have a Z in eq 2 and a max inside the sigmoid instead? That should also clearly illustrate the minimax nature of the problem, which you can later exploit to justify the alternating optimization procedure over Z and $\theta$.} 
% Here, $p^{\theta}(x_i)$ is the probability of plan verification, obtained by adding the log probabilities of the query-segment pairs computed over the best alignment matrix $\mathrm{Z}$ (as denoted by the $\max$ operator). 

% Intuitively, for positive samples ($y_i=1$), this probability would be high, while for negative samples ($y_i=0$), it would be low (ideally $=0$) for all possible alignments, including the best one.

% and $\mathrm{Z}^{\ast}$ denotes the best localization of the queries in the video 
% \nk{What do we mean by best localization here? How is it defined? If I understand it correct, a video positively matched to a task description will still only align with exactly one of the query sequences generated from the task description graph $G$.}
% and is given as:
% \begin{equation}
%     \mathrm{Z}^{\ast} = \argmax_{\mathrm{Z} \in \{0,1\}^{N \times S}} \frac{1}{N} \sum_{j,t} \log n_j^{\theta}(s_t)Z_{jt}
% \label{eq:Z}
% \end{equation}

% \nk{Here j in $n_j$ is the query sequence index but it is conflicting with the query types in $n_i$ from the previous section while they don't have anything in common. We need to change the notation to correct this.}


% \nk{This intuitive picture seems slightly incorrect to me. $Z$ is only the best alignment if the label $y_i$ is 1 and the query sequence is indeed the one being executed in the video. If the label $y_i$ is 0, $p^\theta(x_i)$ would also have to be 0. For that wouldn't the elements of Z have to be maximally misaligned? But it seems like maximizing eq 3 wrt Z will force them to always align the video segments and the query sequence even when the label $y_i$ is 0. Am I misinterpreting something here?}.

% \nk{Instead of just declaring the alternating optimization routine, we should justify it properly based on the minimax nature of the problem}: 

% (i) find the best localization with fixed model parameters (optimize $\mathrm{Z}$ with $\theta$ fixed); (ii) optimize the model parameters given the best localization (optimize $\theta$ with $\mathrm{Z}$ fixed).

% \nk{The meaning of best localization is unclear to me. It seems like we are always assuming $y_i=1$ here. Wouldn't we need some kind of complete misalignment in case of $y_i=0$?}

%%%%%%%%%%%%%%%%%%%%%%%%%%%%%%%%%%%%%%%%%%%%%%%%%%%%%%%%%
% \nk{Do you mean integer programming? Because elements of Z must be binary, right?}

\noindent \textbf{Dynamic Programming (DP)-based Alignment:} Finding the best $\mathrm{Z}$ in Eq.~\ref{eq:p_theta} given $\theta$ requires iterating over combinations of $N$ queries and $S$ segments while respecting certain constraints. The constraints, visualized in Fig.~\ref{figure:model-layout}b, ensure that~a)~no two queries are aligned to the same segment\footnote{This ensures that the order of queries can be verified, which cannot be done when queries belong to the same segment.} (Eq.~\ref{Z-a}),~b)~all queries are accounted for in $S$ (Eq.~\ref{Z-b}), and~c)~the temporal orderings between queries in the query sequences are respected (Eq.~\ref{Z-c}). Specifically, if query $n_u$ precedes $n_v$ ($n_u \rightarrow n_v$), and query $n_v$ is paired with segment $s_{\bar{t}}$ (i.e. $Z_{v\bar{t}}=1$), then query $n_u$ cannot be paired with any segment that lies after $s_{\bar{t}}$ (i.e. $Z_{ut} \neq 1 \; \forall \; t \geq \bar{t}$). The resulting optimization problem for $\mathrm{Z}$, given $\theta$ is:

\begin{subequations}\label{Z-main}
\begin{align}
& \max_{\mathrm{Z} \in {\{0,1\}}^{N \times S}} \sum_{j,t} \log f^{\theta}(a_j, s_t)Z_{jt} \tag{\ref{Z-main}}, \quad \text{s.t.}\\
& \sum_{j=0}^{N-1} Z_{jt} \in \{0,1\}, \quad \forall \; 0 \leq t \leq S-1 \label{Z-a}\\
&\sum_{t=0}^{S-1} Z_{jt} = 1, \quad \forall \; 0 \leq j \leq N-1 \label{Z-b}\\
% & n_u \rightarrow n_v, \; Z_{v\bar{t}}=1 \Longrightarrow Z_{ut} \neq 1, \quad \forall \; t \geq \text{argmax}_{\bar{t}} Z_{v\bar{t}} \label{Z-c}
& n_u \rightarrow n_v, \; Z_{v\bar{t}}=1 \Longrightarrow Z_{ut} \neq 1, \quad \forall \; t \geq \bar{t} \label{Z-c}
\end{align}
\end{subequations}

% \begin{subequations}
% \begin{align*}
%     & \sum_{j=0}^{N-1} Z_{jk} \in \{0,1\} \; \forall \; 0 \leq k \leq S-1 \\
%     & \sum_{j,k} Z_{jk} = N \\
%     & n_u < n_v \Longrightarrow Z_{uk} \neq 1 \; \forall \; k \geq \text{argmax}_{\bar{k}} Z_{v\bar{k}}
% \end{align*}
% \end{subequations}

% \nk{I might be misinterpreting it but Eq 3b doesn't seem to be suggesting that all queries must be localized. It still permits a query to not be localized at all, provided that another query is localized into two video segments. Only the total number of localizations must be N. Is this correct?}
% As shown in Figure~\ref{figure:model-layout}(b), Eq.~\ref{Z-a} implies that at most one query can be paired with each segment; Eq.~\ref{Z-b} implies that all queries must be paired; Eq.~\ref{Z-c} handles the ordering of queries -- if $n_u$ precedes $n_v$ ($n_u \rightarrow n_v$), and query $n_v$ localizes in the segment $s_{\bar{t}}$, then query $n_u$ cannot localize in any segment ($Z_{ut} \neq 1$) that lies after $s_{\bar{t}}$ \nk{Eq 4c is currently not stating the fact that $n_v$ localizes in segment $s_{\bar{t}}$. We need to add $Z_{v\bar{t}}=1$}. 
Intuitively, the solution to Eq.~\ref{Z-main} gives us the best alignment score (note, the overlap with Eq.~\ref{eq:p_theta}). The iterations over $N$ queries and $S$ segments for solving Eq.~\ref{Z-main} are underpinned by an overlapping and optimal substructure. For instance, to optimally align queries $(n_j)_{j=0}^{N-1}$ and segments $(s_t)_{t=0}^{S-1}$, one could:~a)~pair $\langle n_0, s_0 \rangle$ and optimally align the remaining queries and segments $(n_j)_{j=1}^{N-1}, (s_t)_{t=1}^{S-1}$; or (2) skip $s_0$ and still optimally align \emph{all} queries, now with the remaining segments $(n_j)_{j=0}^{N-1}, (s_t)_{t=1}^{S-1}$ (see Fig.~\ref{figure:model-layout}b(iv)). This recursive substructure leads to a DP solution for Eq.~\ref{Z-main}. 

Let, $F^{\ast}((n_j)_j^{N-1}, (s_t)_t^{S-1})$ denote the best alignment score for queries $(n_j)_j^{N-1}$ and segments $(s_t)_t^{S-1}$ from Eq.~\ref{Z-main}. Based on the aforementioned reasoning, $F^{\ast}((n_j)_j^{N-1}, (s_t)_t^{S-1})$ can be recursively written as: 
\begin{multline}
    F^{\ast}((n_j)_j^{N-1}, (s_t)_t^{S-1}) = \text{max} \big( \log f^{\theta}(a_j, s_t) \\ + F^{\ast}((n_j)_{j+1}^{N-1}, (s_t)_{t+1}^{S-1}), F^{\ast}((n_j)_j^{N-1}, (s_t)_{t+1}^{S-1}) \big) \label{eq:dp}
\end{multline}
% Let us denote $F^{\ast}(n_{0:N-1}, s_{0:S-1})$ as the solution of Eq.~\ref{Z-main} which gives the best alignment score of queries $\{n_0, \dots, n_{N-1}\}$ and segments $\{s_0, \dots, s_{S-1}\}$. To solve for $F^{\ast}$, we consider two cases: (1) pair $(n_0, s_0)$ and optimally align the remaining queries and segments $F^{\ast}(n_{1:N-1}, s_{1:S-1})$; or (2) skip $s_0$ and optimally align \emph{all} queries, now with the remaining segments $F^{\ast}(n_{j:N-1}, s_{t+1:S-1})$. Given the optimal overlapping substructure, we can use dynamic programming (dp) to solve for $F^{\ast}$, given as:
% Here, we are computing the best alignment score for queries $\{n_j, \dots, n_{N-1}\}$ and segments $\{s_t, \dots, s_{S-1}\}$.
The base cases for the DP are: (i) $\mathrm{Z}=\mathds{I} \; \text{if} \; N=S$; (ii) $Z_{jt} = 1 \; \forall \; t \; \text{if} \; j=N-1$. It is worth noting that the DP subproblems, together with the base cases, satisfy the constraints in Eq.~\ref{Z-a}~\ref{Z-b}~\ref{Z-c}. Since the video may match any of the sequence in the super-set of query sequences (from the topological sort on $G$), we repeat this process of computing $F^{\ast}$ for each sequence and select the maximum value.

%%%%%%%%%%%%%%%%%%%%%%%%%%%%%%%%%%%%%%%%%%%%%%%%%%%%%%%%%5
\noindent \textbf{Optimizing Query Encoder Parameters $\theta$:} After obtaining the best alignment $\mathrm{Z}$ using DP, we substitute the corresponding value of $F^{\ast}((n_j)_{j=0}^{N-1}, (s_t)_{t=0}^{S-1})$ in Eq.~\ref{eq:p_theta} and subsequently Eq.~\ref{eq:bce}. In Eq.~\ref{eq:bce}, we use single mini-batch of training examples and take one gradient-update step of the Adam optimizer for the query encoder parameters $\theta$.

% Previous research has proposed a similar optimization solution that utilizes weak supervision to align a sequence of action steps with video frames~\cite{cross_task,change_it}, however, it assumes an ordered list of steps. In contrast, our approach not only generates the action steps but also incorporates partial ordering into the solution.

%%%%%%%%%%%%%%%%%%%%%%%%%%%%%%%%%%%%%%%%%%%%%%%%%%%%%%%%%%%%%%%%%%%%%%%%%%%%%%%%%


\begin{table*}[t]
% \small
\centering
\begin{adjustbox}{max width=.95\textwidth}
%\footnotesize{
\begin{tabular}{lccccccccc}
\hline
\multirow{2}{*}{\textbf{Model}}  & \textbf{Visual}  & \textbf{Text} & \textbf{MM} & \textbf{\begin{tabular}[c]{@{}c@{}} Novel  \end{tabular}} & \textbf{\begin{tabular}[c]{@{}c@{}}Novel \end{tabular}} & \textbf{\begin{tabular}[c]{@{}c@{}}Novel\end{tabular}} & \multirow{2}{*}{\textbf{Abstraction}}  &  \multirow{2}{*}{\textbf{Average}} \\ 
  & \textbf{feature} & \textbf{feature}  & \textbf{Fusion} & \textbf{Tasks} & \textbf{Steps} & \textbf{Scenes} & & \\
\hline
Text2text~\cite{roberta} &  & RoBERTa &  & 64.9 & 65.8 & 66.5 & 64.7 & 65.5 \\
\hline
\multicolumn{1}{l}{CLIP Hitchhiker \cite{clip_hitchiker}} & CLIP(I) & CLIP &  & 43.9  & 66.5  & 72.2  & 13.6 & 49.1\\ 
\multicolumn{1}{l}{CLIP4Clip mean~\cite{luo2022clip4clip}}& CLIP(I) & CLIP  &  & 49.3  & 70.9 &  74.9 & 16.1 & 52.8\\
\multicolumn{1}{l}{CLIP4Clip seqLSTM~\cite{luo2022clip4clip}} & CLIP(I) & CLIP  &  &  \underline{56.2} & 73.2  & 74.6  &  17.5 & 55.4\\ 
\multicolumn{1}{l}{CoCa \cite{coca}} & Tx(I) & Tx  &  Y &  51.5 & 71.6  &  71.9 & 43.5 & 59.6 \\ 
\multicolumn{1}{l}{VIOLIN-ResNet~\cite{violin_dataset}} & ResNet(I) & BERT  & Y & 47.4 & \textbf{80.4} & \textbf{85.6} & 42.5 & 64.0\\
\hline
\rowcolor{lightgray}
\multicolumn{1}{l}{MIL-NCE~\cite{miech2020end}} & S3D(V) & Word2vec &  & 30.5 & 69.6 & 73.5 & 24.3 & 49.5\\
\rowcolor{lightgray}
\multicolumn{1}{l}{VideoCLIP~\cite{videoclip}} & Tx(V) & Tx  & Y & 29.3  & 67.6  & 77.9  & 25.6  & 50.1 \\ 
\rowcolor{lightgray}
\multicolumn{1}{l}{VIOLIN-I3D~\cite{violin_dataset}} & I3D(V) & BERT & Y & 45.6 & \underline{79.7} & 83.9 & \underline{47.6} & 64.2 \\
\hline
\multicolumn{1}{l}{\textbf{NSG (ours)}}& CLIP(I) & CLIP & Y & \textbf{90.0} & 64.7 & \underline{84.9} & \textbf{80.4} & \textbf{80.0}\\ 
\multicolumn{1}{l}{Oracle Model} &CLIP(I) & CLIP   & Y &  95.0 & 96.7 & 97.6 & 97.2 & 96.6 \\ 
\hline
\end{tabular}%}
\end{adjustbox}
% \vspace{.2cm}
\caption{\textbf{Comparison of baselines with NSG on different data splits using F1-score.} MM fusion indicates multimodal fusion of vision and text features. %Average represent the average score among the four evaluation scenarios. 
Tx indicates the Transformer as feature extractor with image (I) and video (V) backbones. Video backbone models are highlighted in gray.  \underline{Underline} indicates second-best performance. %\red{RH: to add corresponding Accuracy results in the appendix} \rd{Add gray bg color to one of the blocks to differentiate image vs. video backbones? }
%\bc{We might replace the "Tx" to the actual name, e.g., CLIP, ViT.}
} %Con indicates contrastive loss, Gen indicates generation loss, and Class indicates classification loss.} 
%\vspace{-10pt}
\label{table:baseline_results}
\end{table*}

% \begin{table*}[t]
% % \small
% \centering
% \scriptsize{
% \begin{tabular}{llccccc}
% \hline
% \multicolumn{1}{l|}{\textbf{Model}} & \textbf{Split} & \textbf{\begin{tabular}[c]{@{}c@{}}\nt\end{tabular}} & \textbf{\begin{tabular}[c]{@{}c@{}}\nst\end{tabular}} & \textbf{\begin{tabular}[c]{@{}c@{}}\nc\end{tabular}} & \textbf{Abstraction} \\ 
% \hline
% \multicolumn{2}{l}{MIL-NCE~\cite{miech2020end}} & 39.7 & 70.6 & 74.5 & 26.7\\ 
% \multicolumn{2}{l}{CLIP Hitchhiker \cite{clip_hitchiker}} & 43.9  & 66.5  & 72.2  & 13.6 \\ 
% \multicolumn{2}{l}{CLIP4Clip mean~\cite{luo2022clip4clip}} & 49.3  & 70.9 &  74.9 & 16.1 \\
% \multicolumn{2}{l}{VideoCLIP~\cite{videoclip}} & 29.3  & 67.6  & 77.9  & 25.6   \\ 
% \hline
% \multicolumn{2}{l}{CLIP4Clip seqLSTM~\cite{luo2022clip4clip}} &  56.2 & 73.2  & 74.6  &  17.5 \\ 
% \multicolumn{2}{l}{CoCa \cite{coca}} &  51.5 & 71.6  &  71.9 & 43.5 \\ 
% \hline
% \multicolumn{2}{l}{VIOLIN-ResNet~\cite{violin_dataset}} & 47.4 & 80.4 & 85.6 & 42.5 \\
% \multicolumn{2}{l}{VIOLIN-I3D~\cite{violin_dataset}} & 45.6 & 79.7 & 83.9 & 47.6 \\
% \hline
% \multicolumn{2}{l}{\textbf{NSG (ours)}} & \textbf{90.0} & 64.7 & 84.9 & \textbf{80.4}\\ 
% \multicolumn{2}{l}{Oracle Model} & 95.0 & 96.7 & 97.6 & 97.2\\ 
% \hline
% Socratic~\cite{socratic} & & & & \\
% \hline
% \end{tabular}}
% \caption{Comparison of models (baselines and the proposed model) on different data splits. F1-score.}
% \label{table:baseline_results_new}
% \end{table*}

% \begin{table*}[t]
% \centering
% \scriptsize{
% \begin{tabular}{llccccc}
% \hline
% \multicolumn{1}{l|}{\textbf{Model}} & \textbf{Split} & \textbf{\begin{tabular}[c]{@{}c@{}}\nt\end{tabular}} & \textbf{\begin{tabular}[c]{@{}c@{}}\nst\end{tabular}} & \textbf{\begin{tabular}[c]{@{}c@{}}\nc\end{tabular}} & \textbf{Abstraction} \\ 
% \hline
% \multicolumn{2}{l}{Video Captioning} &   &   &   &   & \\ 
% \hline
% \multicolumn{2}{l}{VIOLIN-ResNet~\cite{violin_dataset}} & 47.4 & 80.4 & 85.6 & 42.5 \\
% %\multicolumn{2}{l}{VIOLIN-CLIP~\cite{clip}} & 38.7 & 85.5 & 90.9 & 83.9 & 44.8 \\ 
% \multicolumn{2}{l}{CLIP~\cite{clip}} & &&&&\\ 
% \multicolumn{2}{l}{CoCa \cite{coca}} &  51.5 & 71.6  &  71.9 & 45.2 \\ 
% \hline
% % \hline
% % \multicolumn{2}{l}{MViT-GloVe} & 64.8 & 88.2 & 92.7 & 88.1 & 65.6 \\ 
% % % \hline
% % \multicolumn{2}{l}{MViT-GloVe-attention} & 57.7 & \textbf{91.0} & \textbf{93.8} & 90.3 & 65.4\\ 
% % % \hline
% % \multicolumn{2}{l}{MViT-BERT} & 58.6 & 83.2 & 92.4 & 88.6 & 62.4 \\ 
% % % \hline
% % \multicolumn{2}{l}{MViT-BERT-attention} & 58.6 & 87.1 & 92.2 & 88.9 & 62.7\\ 
% % \hline
% \multicolumn{2}{l}{VIOLIN-I3D~\cite{violin_dataset}} & 45.6 & 79.7 & 83.9 & 47.6 \\
% %\multicolumn{2}{l}{VIOLIN-MIL-NCE~\cite{miech2020end}} & 48.8  & 76.2  &  84.6 & 83.6 & 46.2\\ 
% \multicolumn{2}{l}{MIL-NCE~\cite{miech2020end}} & &&&&\\ 
% \multicolumn{2}{l}{CLIP hitchHiker \cite{clip_hitchiker}} & 43.9  & 66.5  & 72.2  & 13.6 \\ 
% \multicolumn{2}{l}{CLIP4Clip mean~\cite{luo2022clip4clip}} & 49.3  & 70.9 &  74.9 & 16.1 \\ 
% \multicolumn{2}{l}{CLIP4Clip seqLSTM~\cite{luo2022clip4clip}} &  56.2 & 73.2  & 74.6  &  17.5 \\ 
% \multicolumn{2}{l}{VideoCLIP~\cite{videoclip}} &   &   &   &   &  \\ 
% % 
% % \multicolumn{2}{l}{CLIP4Clip tight \cite{luo2022clip4clip}} &   &   &   &   &  \\ 
% % \multicolumn{2}{l}{CLIP hitchHiker \cite{clip_hitchiker}} &   &   &   &   &  \\ 
% % \hline
% \multicolumn{2}{l}{Video Txformer model (Todo)} &   &   &   &   & \\ 
% \hline
% \multicolumn{2}{l}{\textbf{NSG (ours)}} & \textbf{90.0} & 64.7 & 84.9 & \textbf{80.4}\\ 
% % \multicolumn{2}{l}{\textbf{NSG Prompting (ours)}} &   &   &   &   & \\ 
% % \hline
% \multicolumn{2}{l}{Oracle Model} & 95.0 & 96.7 & 97.6 & 97.2\\ 
% \hline
% \end{tabular}}
% \caption{Comparison of models (baselines and the proposed model) on different data splits. F1-score.}
% \label{table:baseline_results}
% \end{table*}


% this is Acc table
% \begin{table*}[t]
% \small
% \begin{tabular}{llccccc}
% \hline
% \multicolumn{1}{l|}{\textbf{Model}} & \textbf{Split} & \textbf{\begin{tabular}[c]{@{}c@{}}sub-goal \\ composition\end{tabular}} & \textbf{\begin{tabular}[c]{@{}c@{}}verb-noun \\ composition\end{tabular}} & \textbf{\begin{tabular}[c]{@{}c@{}}context-verb-noun \\ composition\end{tabular}} & \textbf{\begin{tabular}[c]{@{}c@{}}context-goal \\ composition\end{tabular}} & \textbf{abstraction} \\ 
% \hline
% \multicolumn{2}{l}{Video Captioning} &   &   &   &   & \\ 
% \hline
% \multicolumn{2}{l}{VIOLIN-Image~\cite{violin_dataset}} & 55.3 & 81.0 & 86.4 & 85.3 & 58.0\\
% \multicolumn{2}{l}{CLIP~\cite{clip}} & 55.4 & 85.9 & 90.9 & 84.9 & 59.9\\ 
% \multicolumn{2}{l}{CoCa \cite{coca}} &   &   &   &   &  \\ 
% \hline
% % \hline
% % \multicolumn{2}{l}{MViT-GloVe} & 64.8 & 88.2 & 92.7 & 88.1 & 65.6 \\ 
% % % \hline
% % \multicolumn{2}{l}{MViT-GloVe-attention} & 57.7 & \textbf{91.0} & \textbf{93.8} & 90.3 & 65.4\\ 
% % % \hline
% % \multicolumn{2}{l}{MViT-BERT} & 58.6 & 83.2 & 92.4 & 88.6 & 62.4 \\ 
% % % \hline
% % \multicolumn{2}{l}{MViT-BERT-attention} & 58.6 & 87.1 & 92.2 & 88.9 & 62.7\\ 
% % \hline
% \multicolumn{2}{l}{VIOLIN-Video~\cite{violin_dataset}} & 56.7 & 80.8 & 87.6 & 84.0 & 62.4\\
% \multicolumn{2}{l}{MIL-NCE~\cite{miech2020end}} & 57.7  & 77.0  &  83.7 & 83.2 & 57.6\\ 
% \multicolumn{2}{l}{CLIP4Clip mean~\cite{luo2022clip4clip}} & 59.1  & 72.3  &   &   &  \\ 
% \multicolumn{2}{l}{CLIP4Clip seqLSTM~\cite{luo2022clip4clip}} &  61.3 & 76.6  & 81.7  & 76.0  &  53.0 \\ 
% \multicolumn{2}{l}{VideoCLIP ~\cite{}} &   &   &   &   &  \\ 
% % 
% % \multicolumn{2}{l}{CLIP4Clip tight \cite{luo2022clip4clip}} &   &   &   &   &  \\ 
% % \multicolumn{2}{l}{CLIP hitchHiker \cite{clip_hitchiker}} &   &   &   &   &  \\ 
% % \hline
% \multicolumn{2}{l}{Video Txformer model (Todo)} &   &   &   &   & \\ 
% \multicolumn{2}{l}{\textbf{NSG (ours)}} & \textbf{90.2} & 66.1 & 93.0 & 84.5 & \textbf{86.5}\\ 
% \multicolumn{2}{l}{\textbf{NSG Prompting (ours)}} &   &   &   &   & \\ 
% % \hline
% \multicolumn{2}{l}{Oracle Model} & 96.4 & 97.4 & 97.0 & 97.4 & 97.0\\ 
% \hline
% \end{tabular}
% \caption{Comparison of models (baselines and the proposed model) on different data splits. The 2 values separated by a "/" denote accuracy and F1-score, respectively.}
% \label{table:baseline_results}
% \end{table*}

%%%%%%%%%%%%%%%%%%%%%%%%%%%%%%%%%%%%%%%%%%%%%%%%%%%%%%%%%%%%%%%%%%%%%%%%%%%%%%%%%%%
% \begin{table*}[t]
% \small
% \begin{tabular}{llcccccc}
% \hline
% \multicolumn{1}{l|}{\textbf{Model}} & \textbf{Split} & \textbf{validation}  & \textbf{\begin{tabular}[c]{@{}c@{}}sub-goal \\ composition\end{tabular}} & \textbf{\begin{tabular}[c]{@{}c@{}}verb-noun \\ composition\end{tabular}} & \textbf{\begin{tabular}[c]{@{}c@{}}context-verb-noun \\ composition\end{tabular}} & \textbf{\begin{tabular}[c]{@{}c@{}}context-goal \\ composition\end{tabular}} & \textbf{abstraction} \\ 
% \hline
% \multicolumn{2}{l}{ResNet-GloVe} & 96.8 / 96.9 & 60.3 / 56.2 & 84.9 / 84.7 & 92.5 / 92.6 & 83.2 / 82.3 & 63.1 / 54.4\\ 
% \hline
% \multicolumn{2}{l}{ResNet-GloVe-attention} & 92.8 / 92.9 & 58.8 / 48.7 & 85.4 / 85.2 & 92.3 / 92.5 & \textbf{90.5 / 90.7} & 63.9 / 56.6\\ 
% \hline
% \multicolumn{2}{l}{ResNet-BERT} & 96.9 / 96.9 & 58.8 / 46.4 & 82.9 / 82.3 & 91.7 / 91.6 & 82.2 / 81.3 & 61.4 / 46.5\\ 
% \hline
% \multicolumn{2}{l}{ResNet-BERT-attention} & 86.6 / 87.1 & 55.3 / 47.4 & 81.0 / 80.4 & 86.4 / 86.9 & 85.3 / 85.6 & 58.0 / 42.5\\
% \hline
% \multicolumn{2}{l}{I3D-GloVe} & 99.3 / 99.3 & 60.5 / 55.1 & 82.6 / 82.4 & 88.8 / 89.1 & 83.2 / 83.1 & 63.1 / 55.3\\ 
% \hline
% \multicolumn{2}{l}{I3D-GloVe-attention} & 92.0 / 92.1 & 55.5 / 44.6 & 78.4 / 78.0 & 89.4 / 89.6 & 85.4 / 85.6 & 61.4 / 51.5\\ 
% \hline
% \multicolumn{2}{l}{I3D-BERT} & 98.9 / 98.9 & 56.7 / 44.5 & 80.8 / 80.3 & 89.6 / 89.8 & 84.7 / 84.4 & 62.4 / 47.1\\ 
% \hline
% \multicolumn{2}{l}{I3D-BERT-attention} & 90.2 / 90.4 & 56.7 / 45.6 & 80.8 / 79.7 & 87.6 / 87.9 & 84.0 / 83.9 & 62.4 / 47.6\\
% \hline
% \multicolumn{2}{l}{S3D-BERT} & 99.8 / 99.6 & 58.2 / 45.3 & 78.6 / 76.5 & 88.4 / 88.2 & 85.5 / 85.4 & 64.6 / 55.0\\ 
% \hline
% \multicolumn{2}{l}{MViT-GloVe} & \textbf{99.9 / 99.9} & 64.8 / 61.4 & 88.2 / 88.0 & 92.7 / 92.7 & 88.1 / 87.8 & 65.6 / 57.2 \\ 
% \hline
% \multicolumn{2}{l}{MViT-GloVe-attention} & 99.6 / 99.6 & 57.7 / 43.4 & \textbf{91.0 / 91.1} & \textbf{93.8 / 93.8} & 90.3 / 90.3 & 65.4 / 59.8\\ 
% \hline
% \multicolumn{2}{l}{MViT-BERT} & \textbf{99.9 / 99.9} & 58.6 / 47.7 & 83.2 / 82.2 & 92.4 / 92.4 & 88.6 / 88.6 & 62.4 / 49.6 \\ 
% \hline
% \multicolumn{2}{l}{MViT-BERT-attention} & 97.7 / 97.7 & 58.6 / 49.4 & 87.1 / 87.0 & 92.2 / 92.2 & 88.9 / 88.8 & 62.7 / 48.5\\ 
% \hline
% \multicolumn{2}{l}{CLIP-CLIP} & 99.0 / 98.0 & 55.4 / 38.7 & 85.9 / 85.5 & 90.9 / 90.9 & 84.9 / 83.9 & 59.9 / 44.8\\ 
% \hline
% \multicolumn{2}{l}{\textbf{NSG (Ours)}} & 97.1 / 97.5 & \textbf{90.2 / 90.0} & 66.1 / 52.7 & 93.0 / 92.9 & 84.5 / 84.9 & \textbf{86.5 / 87.3}\\ 
% \hline
% \multicolumn{2}{l}{Oracle Model} & 99.6 / 99.7 & 96.4 / 95.0 & 97.4 / 96.7 & 97.0 / 97.4 & 97.4 / 97.6 & 97.0 / 97.2\\ 
% \hline
% \end{tabular}
% \caption{Comparison of models (baselines and the proposed model) on different data splits. The 2 values separated by a "/" denote accuracy and F1-score, respectively.}
% \label{table:baseline_results}
% \end{table*}
%%%%%%%%%%%%%%%%%%%%%%%%%%%%%%%%%%%%%%%%%%%%%%%%%%%%%%%%%%%%%%%%%%%%%%%%%%%%%%%%%%%
%=====================================================================
% \subsection{Model}
% \label{subsection:model}
% \red{this subsection should be referenced in the previous subsections}
% \nk{This subsection should come before Plan Verification}

% \textbf{Vision Module:} The video is split into segments with $k$\footnote{$k$ can be arbitrary, but, must uphold the constraint from \S \ref{subsection:plan_verification} -- $S \geq N$} frames per segment using a sliding window. If required, the last segment is zero-padded to $k$ frames. Each sequence of $k$-frames is encoded using CLIP model to $\mathds{R}^{k \times d}$. For CLIP, $d=512$. We use an RNN-based aggregation to encode each segment to $s_j \in \mathds{R}^{d_v}$. 

% \textbf{Text Module:} The pretrained T5 from \S \ref{subsection:plan_generation} is used to map each hypothesis to a partial-order graph $G(V,E)$. Each node $n_i \in V$ is in the form of a query that is executed on the video segments $s_j$. During training, the weights of the T5 transformer are kept frozen.

% \textbf{Query Module:} Recall, that each query node is of the form $StateQuery(arg_1, arg2)$ or $RelationQuery(arg_1, arg_2, arg3)$. To query a segment $s_k$ with node $n_j$, we use the Query Module to calculate the probability $f_{\theta}(n_j, s_k)$, which is then used to solve Eq.~\ref{Z-main}. Each argument $arg_i$ is passed encoded using the same CLIP model to $n_j \in \mathds{R}^{d_n}$. Based on the query type ($StateQuery$ or $RelationQuery$), we pass the node and segment encodings through a $StateQuery$ or a $RelationQuery$ neural network to obtain $f_{\theta}(n_j, s_k)$.

%%%%%%%%%%%%%%%%%%%%%%%%%%%%%%%%%%%%%%%%%%%%%%%%%%%%%%%%%%%%%%%%%%%%%%%%%%%%%%%%%%%
%%%%%%%%%%%%%%%%%%%%%%%%%%%%%%%%%%%%%%%%%%%%%%%%%%%%%%%%%%%%%%%%%%%%%%%%%%%%%%%%%%%
\section{Experiments} 
\label{subsection:baseline_evaluation}
We compare various state-of-the-art (SOTA) VLMs with NSG on the \etv benchmark (see Appendix~\ref{appendix:nsg_training} for NSG's experimental training details).

\subsection{SOTA VLM Baselines}
We investigate 6 VLMs developed for video-language tasks requiring similar reasoning as \etv. Summarized in Table~\ref{table:baseline_results}, \textbf{CLIP4Clip}~\cite{luo2022clip4clip}, \textbf{CLIP Hitchhiker}~\cite{clip_hitchiker}, \textbf{CoCa}~\cite{coca} use image backbones followed by temporal aggregation, while \textbf{VideoCLIP}~\cite{videoclip}, \textbf{MIL-NCE}~\cite{miech2020end}, and \textbf{VIOLIN}~\cite{violin_dataset} use video backbones. With the exception of CoCa, which is trained with contrastive and captioning loss, all other models are trained using contrastive loss~\cite{miech2020end}. Lastly, VideoCLIP and VIOLIN use an explicit fusion of text-vision features. For each model, we freeze all pretrained feature extractors and finetune a fully-connected probe layer, along with the temporal aggregation layers where appropriate (CLIP4Clip-LSTM, VIOLIN), using \etv's train split. 

Finally, to establish upper bounds on \etv, we instantiate: (1) a \textbf{Text2text model}, which constructs video captions using ground-truth labels for objects and actions, encodes the captions and task descriptions using (pretrained) RoBERTa model~\cite{roberta} and measures alignment using the cosine similarity score (see Appendix~\ref{appendix:upper_bound_models}), and (2) an \textbf{Oracle model}, which is trained with full supervision on sub-tasks labels and locations in addition to task verification labels.

%%%%%%%%%%%%%%%%%%%%%%%%%%%%%%%%%%%%%%%%%%%%%%%%%%%%%%%%%%%%%%%%%%%%%%%
\subsection{Results}
 In Table~\ref{table:baseline_results}, we show the performance of NSG vs. SOTA VLMs per split of \etv.~(1)~\textbf{Novel Tasks}: NSG significantly outperforms other baselines due to its ability to decompose and detect sub-tasks while using DP alignment to handle temporal constraints among them. In contrast, other baselines rely on detecting the entire task under temporal constraints, which is more challenging. Further, image-based baselines outperform video-based baselines due to their ability to capture a greater degree of compositional detail through frame-level representations.~(2) \textbf{Novel Steps}: NSG's poor performance in this split could be attributed to its low precision in the \emph{slice} sub-task (which is dominant in this split), as shown in Figure~\ref{figure:complexity-confusion_mat} [Right]. We hypothesize that since NSG only uses the aligned segments while discarding the rest, learning to utilize context from neighboring segments to capture \emph{slice} (like picking up a knife) could be a promising future direction. (3) \textbf{Novel Scenes}: Here, NSG is comparable to the best baseline VIOLIN-ResNet. Since the tasks are identical to the train split, the success of a model is contingent on the vision encoder's ability to accurately detect the same sub-tasks in unseen scenes. Consequently, models with an additional temporal aggregation layer (VIOLIN) finetuned on \etv, tend to outperform image-based models that do not have temporal aggregation (CLIP Hitchhiker) and models with frozen video features (MIL-NCE, VideoCLIP). (4) \textbf{Abstraction}: NSG significantly outperforms the baselines, primarily due to its semantic parser, which captures the underlying structure of the description and encodes the relevant concepts, such as objects and sub-tasks, to generate an (abstract) symbolic output.

%%%%%%%%%%%%%%%%%%%%%%%%%%%%%%%%%%%%%%%%%%%%%%%%%%%%%%%%%

\subsection{Analysis of NSG}
\label{subsec:analysis}

\noindent \textbf{NSG learns to localize task-relevant entities without explicit supervision.} Figure~\ref{figure:complexity-confusion_mat} shows the confusion matrix of \code{StateQuery} \& \code{RelationQuery} outputs, which capture sub-tasks, with their ground truths. The high recall demonstrates NSG's ability to localize task-relevant entities, despite being trained using only task verification labels.

\noindent \textbf{Effect of query types on NSG.} While query types with multiple entity arguments might appear capable of modeling complex dependencies amongst entities and having more expressive power, encoding multiple entities jointly using a single encoder makes the grounding problem more challenging. Hence, in practice, we found that using a combination of \code{StateQuery} \& \code{RelationQuery} types as opposed to \code{ActionQuery} (which encodes multiple entities using a single encoder) enabled better grounding and led to better performance in terms of F1-score (Table~\ref{table:query_comparison}).

% % \input{figures/complexity-ordering}

\begin{figure}[t]
\centering
    \includegraphics[width=\linewidth]{plots/com-ord-comparison.png}
    \caption{\textbf{NSG maintains consistent performance as task complexity and ordering difficulty increases.} F1-score of NSG vs. best-performing baseline for \etv tasks with varying complexity and ordering are shown.}
    \label{figure:complexity-ordering}
%\vspace{-10pt}
\end{figure}

\begin{figure}[t]
\centering
    \includegraphics[width=\linewidth]{plots/complexity-confusion_mat.pdf}
    \caption{[Left] F1-score of NSG vs. best-performing baseline for \etv tasks with varying complexity averaged over all splits (Appendix~\ref{appendix:analysis} shows performance with varying ordering). [Right] Confusion Matrix for NSG Queries on validation split (SQuery: \code{StateQuery}, RQuery: \code{RelationQuery}). See Appendix~\ref{appendix:analysis} for results on all splits.}
    \label{figure:complexity-confusion_mat}
    % \vspace{-0.1cm}
\end{figure}

\noindent \textbf{NSG shows consistent performance with increasing task difficulty.}
In Figure~\ref{figure:complexity-confusion_mat}, NSG's performance is minimally affected by increase in task difficulty characterized by number of sub-tasks (complexity) and ordering constraints (\S~\ref{section:evaluation}) unlike the best-performing baseline (VIOLIN-ResNet). 

\noindent \textbf{NSG is robust to segmentation window size} The effect of $k$ on NSG is minimal (Appendix~\ref{appendix:analysis}).


\noindent \textbf{NSG also enables task verification on real-world data.} NSG outperforms all competitive baselines on CTV significantly with F1-score (NSG: $\mathbf{76.3}$, CoCa: 70.9, VideoCLIP: 49.7, VIOLIN 34.7), demonstrating its causal and compositional reasoning capabilities in real-world applications (see Appendix~\ref{appendix:NSG_crosstask} for details).

\begin{table}[t]
\small
\centering
\begin{tabular}{lcccc}
\hline
\multirow{2}{*}{NSG} & \begin{tabular}[c]{@{}c@{}} Novel  \end{tabular} & \begin{tabular}[c]{@{}c@{}}Novel \end{tabular} & \begin{tabular}[c]{@{}c@{}}Novel\end{tabular} & \multirow{2}{*}{Abstract.} \\ & Tasks & Steps & Scenes & \\
\hline
\code{Action} & 78.2 & 45.6 & 70.6 & 75.5\\
\code{State+Relation} & 90.0 & 64.7 & 84.9 & 80.4\\
\hline
\end{tabular}
\vspace{2pt}
\caption{\code{(State}~+~\code{Relation)Query}~vs.~\code{ActionQuery}}
\label{table:query_comparison}
% \vspace{-5pt}
\end{table}  

\noindent \textbf{Limitations of NSG.} (1) It does not consider multiple simultaneous actions like ``picking an apple while closing the refrigerator door", (2) The assumption of equal-length video segments may be unsuitable for sub-tasks with a highly variable duration. We defer exploration of these limitations to future work, (3) Since NSG aligns the video with the entire task graph, it requires the full task execution video. Without this, alignment is partial, rendering NSG ineffective for online task verification.

%%%%%%%%%%%%%%%%%%%%%%%%%%%%%%%%%%%%%%%%%%%%%%%%%%%%%%%%%%%%%%%%%%%%%%%%%%%%%%%%%%%t
%\section{Conclusion}
% Future work
% our finding 




%This study is limited by the number of participants and the lengths of the interactions. In addition, the displayed emotions were not the central focus of the conversations. Instead, they were used spontaneously whenever they matched the context leading to a varying experience for each of the participants.
%Still, it could prove a fantastic potential for using the display of emotion to deepen the connection between humans and robots. Nevertheless, it cannot be disregarded that humanizing robots also carries a significant risk. When the distinctions between humans and robots become blurred, it may soon become impossible for some people to tell the two apart. Therefore, the risks and benefits need to be carefully evaluated, and it might also be reasonable to establish internationally binding guidelines to differentiate robots and humans visually.
%For now, it has to be clearly stated, though, that any of the participants experienced no difficulty in telling that they were not, in fact, interacting with a human, indicating that the robot is not that human-like after all 
%Many participants reported a positive reaction to being smiled at. However, the display of other, potentially more negative emotions, like sadness, fear, or anger, has to be evaluated in further study.
%Future research could solidify our result with quantitative data, directly comparing the interaction with robots who do and do not show emotions.
%It can also be concluded that context awareness and the feeling of being understood by the robot were reported as a more significant benefit than the idea of the robot feeling or communicating joy. It encouraged the participants to talk more, and it could be further investigated whether this effect could be achieved through other visual or auditory cues that are not directly related to a human-typical expression of happiness.
%The study could show that the display of emotion by an Android was generally seen as positive and beneficial but also shed light on the fact that there many ethical questions that need to be investigated. 
%\section{Discussion}



% 

\section{Preliminaries of NeRF}\label{sec:priliminary}
The 3D scene can be represented with the NeRF model, and the neural network is utilized to map a point position $\boldsymbol{x} \in \mathbb{R}^3$  and a view direction $\boldsymbol{d} \in \mathbb{R}^3$ to the corresponding color $\boldsymbol{c} \in \mathbb{R}^3$ and volume density $\sigma$ \cite{mildenhall2020nerf}.  We apply the voxel grid to represent the scene considering the low computational cost of voxel-based NeRF \cite{sun2021direct}. The density voxel grid $\boldsymbol{V}_{\text{density}}$  and  feature voxel grid $\boldsymbol{V}_{\text{color}}$  with a shallow MLP  are adopted to represent the scene geometry and appearance, respectively. Given input queries $\boldsymbol{x} $ and $\boldsymbol{d} $,  the outputs are obtained with the interpolation
\begin{equation}
\begin{aligned}
 \sigma &= \text{inter}(\boldsymbol{x}, \boldsymbol{V}_{\text{density}})  \\
 \boldsymbol{c} &= \text{MLP}_{\theta}(\text{inter}(\boldsymbol{x}, \boldsymbol{V}_{\text{color}}), \boldsymbol{x} , \boldsymbol{d}) 
  \label{eq:important}
  \end{aligned}
\end{equation}
To render the image, the pixel color $\boldsymbol{C}(\boldsymbol{r})$ along the camera ray $\boldsymbol{r}(t) = \boldsymbol{r_0}+ t \boldsymbol{d}$ is approximated by the volume rendering
\begin{equation}
\boldsymbol{C}(\mathbf{r})=\int_{t_1}^{t_2} T(t) \sigma(\boldsymbol{r}(t)) \mathbf{c}(\boldsymbol{r}(t), \mathbf{d}) d t,
\end{equation}
where  $t_1$ and $t_2$ are near and far bounds for sampling points, $\boldsymbol{r_0}$ is the camera origin, and $T(t)$ is accumulated transmittance along the ray from $t_1$ to $t$ defined by
\begin{equation}
T(t)=\exp \left(-\int_{t_1}^t \sigma(\boldsymbol{r}(s)) d s\right).
\label{eq:transmittance}
\end{equation}
The NeRF model is trained by minimizing the  loss between the rendered pixel color $\boldsymbol{C}(\boldsymbol{r})$ and observed pixel color $\hat{\boldsymbol{C}}(\boldsymbol{r})$given by
\begin{equation}
\mathcal{L}_{\text {Color }}=\sum_{\boldsymbol{r} \in \mathcal{R}(\mathbf{P})}\|\hat{\boldsymbol{C}}(\boldsymbol{r})-\boldsymbol{C}(\boldsymbol{r})\|_2^2,
\end{equation}
where $\mathcal{R}(\mathbf{P})$ is the set of rendered rays in a batch.


\section{Implementation Detail of Detectors}
For FCOS3D, we utilize a ResNet-101-DCN\cite{he2016deep, dai2017deformable} as the backbone. The model is trained for 24 epochs using the SGD optimizer with an initial learning rate of 1e-4 and a momentum of 0.9. We set the weight decay to 1e-5, and the max norm of gradient clipping to 35. We also adopt a step decay learning rate scheduler with a 0.1$\times$ decrease at epoch 20 and 23, along with 1000 iterations of linear warm-up. For SMOKE, we employ a DLA-34\cite{yu2018deep} as the backbone. We use the Adam optimizer with an initial learning rate 1e-4, and the remaining settings are the same as FCOS3D. For both detectors, their backbones are initialized with ImageNet pre-trained weights. The batch size for training is set to 16.





\setlength{\tabcolsep}{8.5pt}
\begin{table}[t]
  \small
  \begin{center}
  \begin{tabular}{@{}ccccc@{}}
    \toprule
     3DAug  & DS & LET-AP &  LET-APH & LET-APL \\
    \midrule
      -     &  -           &  0.585 & 0.573 & 0.393 \\
     \checkmark  & - & 0.584 & 0.572   & 0.394 \\
    \checkmark & \checkmark &  \textbf{0.590} & \textbf{0.578} & \textbf{0.403} \\
    \bottomrule
  \end{tabular}
  \end{center}
  \caption{\textbf{Ablation Study} of Drive-3DAug for FCOS3D on Waymo validation set. 3DAug means we use DVGO \cite{sun2021direct} for data augmentation. DS means depth supervision.}
  \label{tab:ablation_depth}
\end{table}


\begin{figure}[t]
    \begin{center}
        \includegraphics[width=0.99\linewidth]{fig/depth_v1.pdf}
    \end{center}
    \caption{\textbf{Visualization of rendered depth map.}  The model with depth supervision depicts better performance.}
    \label{fig:depth}
\end{figure}



\section{Ablation Study on Depth Supervision.}
We qualitatively and quantitatively investigate the effect of depth supervision on background model training and 3D augmentation.
Table~\ref{tab:ablation_depth} shows that the 3D augmentation based on background model trained with depth supervision has better performance, with LET-AP (0.590 vs 0.585).
Figure~\ref{fig:depth} shows that NeRF can reconstruct the background with high quality given depth supervision, and the 3D background model quality can be decreased without depth information.
Thus, LET-AP, LET-APH and LET-APL on car have a slight decrease with 0.001 for 3D augmentation using background model without supervision.





\section{Visualization of Drive-3DAug}
% \tww{Aug car, person, cyclist, including RGB and depths map}
% We visualize the reconstructed depth map and new driving scene generated by Drive-3DAug in Figure~\ref{fig:depth}, indicating that 
We augment car in Figure~\ref{fig:more_vis}, indicating that
we can generate scene with high quality with Drive-3DAug.
Compared with car, pedestrian and cyclist are not rigid body and the size is small, not well applicable for NeRF modelling.
We model pedestrian and cyclist as rigid boby in the present study, which can cause the decay of object model performance.
As shown in Figure~\ref{fig:person}, although there exists flaw for augmented 
pedestrian and cyclist, we can still augment them to improve the detector performance.



\section{Cross-dataset Drive-3DAug}
We have reconstructed thousands of background and object models in Waymo and nuScenes dataset.
These models can serve as the general model assets, convenient for creating new driving scenes inside a specific dataset or cross different datasets.
As shown in Figure \ref{fig:cross}, we compose the object models from nuScenes and the background models from Waymo to create new driving scenes. This can further enlarge the diversity of the training data, and we can generate large amounts of data for the study of model generalization across different datasets.


\begin{figure*}[t]
    \begin{center}
\includegraphics[width=0.9\linewidth]{fig/more_aug.pdf}
    \end{center}
    \caption{\textbf{Visualization of the generated images by Drive-3DAug.} The yellow boxes indicate the newly added cars for the background.}
    \label{fig:more_vis}
\end{figure*}

\begin{figure*}[t]
    \begin{center}
        \includegraphics[width=0.9\linewidth]{fig/person.pdf}
    \end{center}
    \caption{\textbf{Visualization of the generated images by Drive-3DAug.} The yellow and red boxes indicate the augmented pedestrian and cyclist, respectively.}
    \label{fig:person}
\end{figure*}



\begin{figure*}[t]
    \begin{center}
        \includegraphics[width=0.9\linewidth]{fig/cross.pdf}
    \end{center}
    \caption{\textbf{Image Generation cross datasets.} We place the cars from nuScenes on the backgrounds of Waymo. The yellow boxes indicate the augmented cars.}
    \label{fig:cross}
\end{figure*}




% \par\vfill\par


% \clearpage

{\small
\bibliographystyle{ieee_fullname}
\bibliography{egbib}
}



\section{Preliminaries of NeRF}\label{sec:priliminary}
The 3D scene can be represented with the NeRF model, and the neural network is utilized to map a point position $\boldsymbol{x} \in \mathbb{R}^3$  and a view direction $\boldsymbol{d} \in \mathbb{R}^3$ to the corresponding color $\boldsymbol{c} \in \mathbb{R}^3$ and volume density $\sigma$ \cite{mildenhall2020nerf}.  We apply the voxel grid to represent the scene considering the low computational cost of voxel-based NeRF \cite{sun2021direct}. The density voxel grid $\boldsymbol{V}_{\text{density}}$  and  feature voxel grid $\boldsymbol{V}_{\text{color}}$  with a shallow MLP  are adopted to represent the scene geometry and appearance, respectively. Given input queries $\boldsymbol{x} $ and $\boldsymbol{d} $,  the outputs are obtained with the interpolation
\begin{equation}
\begin{aligned}
 \sigma &= \text{inter}(\boldsymbol{x}, \boldsymbol{V}_{\text{density}})  \\
 \boldsymbol{c} &= \text{MLP}_{\theta}(\text{inter}(\boldsymbol{x}, \boldsymbol{V}_{\text{color}}), \boldsymbol{x} , \boldsymbol{d}) 
  \label{eq:important}
  \end{aligned}
\end{equation}
To render the image, the pixel color $\boldsymbol{C}(\boldsymbol{r})$ along the camera ray $\boldsymbol{r}(t) = \boldsymbol{r_0}+ t \boldsymbol{d}$ is approximated by the volume rendering
\begin{equation}
\boldsymbol{C}(\mathbf{r})=\int_{t_1}^{t_2} T(t) \sigma(\boldsymbol{r}(t)) \mathbf{c}(\boldsymbol{r}(t), \mathbf{d}) d t,
\end{equation}
where  $t_1$ and $t_2$ are near and far bounds for sampling points, $\boldsymbol{r_0}$ is the camera origin, and $T(t)$ is accumulated transmittance along the ray from $t_1$ to $t$ defined by
\begin{equation}
T(t)=\exp \left(-\int_{t_1}^t \sigma(\boldsymbol{r}(s)) d s\right).
\label{eq:transmittance}
\end{equation}
The NeRF model is trained by minimizing the  loss between the rendered pixel color $\boldsymbol{C}(\boldsymbol{r})$ and observed pixel color $\hat{\boldsymbol{C}}(\boldsymbol{r})$given by
\begin{equation}
\mathcal{L}_{\text {Color }}=\sum_{\boldsymbol{r} \in \mathcal{R}(\mathbf{P})}\|\hat{\boldsymbol{C}}(\boldsymbol{r})-\boldsymbol{C}(\boldsymbol{r})\|_2^2,
\end{equation}
where $\mathcal{R}(\mathbf{P})$ is the set of rendered rays in a batch.


\section{Implementation Detail of Detectors}
For FCOS3D, we utilize a ResNet-101-DCN\cite{he2016deep, dai2017deformable} as the backbone. The model is trained for 24 epochs using the SGD optimizer with an initial learning rate of 1e-4 and a momentum of 0.9. We set the weight decay to 1e-5, and the max norm of gradient clipping to 35. We also adopt a step decay learning rate scheduler with a 0.1$\times$ decrease at epoch 20 and 23, along with 1000 iterations of linear warm-up. For SMOKE, we employ a DLA-34\cite{yu2018deep} as the backbone. We use the Adam optimizer with an initial learning rate 1e-4, and the remaining settings are the same as FCOS3D. For both detectors, their backbones are initialized with ImageNet pre-trained weights. The batch size for training is set to 16.





\setlength{\tabcolsep}{8.5pt}
\begin{table}[t]
  \small
  \begin{center}
  \begin{tabular}{@{}ccccc@{}}
    \toprule
     3DAug  & DS & LET-AP &  LET-APH & LET-APL \\
    \midrule
      -     &  -           &  0.585 & 0.573 & 0.393 \\
     \checkmark  & - & 0.584 & 0.572   & 0.394 \\
    \checkmark & \checkmark &  \textbf{0.590} & \textbf{0.578} & \textbf{0.403} \\
    \bottomrule
  \end{tabular}
  \end{center}
  \caption{\textbf{Ablation Study} of Drive-3DAug for FCOS3D on Waymo validation set. 3DAug means we use DVGO \cite{sun2021direct} for data augmentation. DS means depth supervision.}
  \label{tab:ablation_depth}
\end{table}


\begin{figure}[t]
    \begin{center}
        \includegraphics[width=0.99\linewidth]{fig/depth_v1.pdf}
    \end{center}
    \caption{\textbf{Visualization of rendered depth map.}  The model with depth supervision depicts better performance.}
    \label{fig:depth}
\end{figure}



\section{Ablation Study on Depth Supervision.}
We qualitatively and quantitatively investigate the effect of depth supervision on background model training and 3D augmentation.
Table~\ref{tab:ablation_depth} shows that the 3D augmentation based on background model trained with depth supervision has better performance, with LET-AP (0.590 vs 0.585).
Figure~\ref{fig:depth} shows that NeRF can reconstruct the background with high quality given depth supervision, and the 3D background model quality can be decreased without depth information.
Thus, LET-AP, LET-APH and LET-APL on car have a slight decrease with 0.001 for 3D augmentation using background model without supervision.





\section{Visualization of Drive-3DAug}
% \tww{Aug car, person, cyclist, including RGB and depths map}
% We visualize the reconstructed depth map and new driving scene generated by Drive-3DAug in Figure~\ref{fig:depth}, indicating that 
We augment car in Figure~\ref{fig:more_vis}, indicating that
we can generate scene with high quality with Drive-3DAug.
Compared with car, pedestrian and cyclist are not rigid body and the size is small, not well applicable for NeRF modelling.
We model pedestrian and cyclist as rigid boby in the present study, which can cause the decay of object model performance.
As shown in Figure~\ref{fig:person}, although there exists flaw for augmented 
pedestrian and cyclist, we can still augment them to improve the detector performance.



\section{Cross-dataset Drive-3DAug}
We have reconstructed thousands of background and object models in Waymo and nuScenes dataset.
These models can serve as the general model assets, convenient for creating new driving scenes inside a specific dataset or cross different datasets.
As shown in Figure \ref{fig:cross}, we compose the object models from nuScenes and the background models from Waymo to create new driving scenes. This can further enlarge the diversity of the training data, and we can generate large amounts of data for the study of model generalization across different datasets.


\begin{figure*}[t]
    \begin{center}
\includegraphics[width=0.9\linewidth]{fig/more_aug.pdf}
    \end{center}
    \caption{\textbf{Visualization of the generated images by Drive-3DAug.} The yellow boxes indicate the newly added cars for the background.}
    \label{fig:more_vis}
\end{figure*}

\begin{figure*}[t]
    \begin{center}
        \includegraphics[width=0.9\linewidth]{fig/person.pdf}
    \end{center}
    \caption{\textbf{Visualization of the generated images by Drive-3DAug.} The yellow and red boxes indicate the augmented pedestrian and cyclist, respectively.}
    \label{fig:person}
\end{figure*}



\begin{figure*}[t]
    \begin{center}
        \includegraphics[width=0.9\linewidth]{fig/cross.pdf}
    \end{center}
    \caption{\textbf{Image Generation cross datasets.} We place the cars from nuScenes on the backgrounds of Waymo. The yellow boxes indicate the augmented cars.}
    \label{fig:cross}
\end{figure*}



\end{document}

