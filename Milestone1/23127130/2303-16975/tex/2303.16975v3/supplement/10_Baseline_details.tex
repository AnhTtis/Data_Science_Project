
\section{Baseline descriptions and details}
\label{appendix:baselines_details}

% 
\begin{table*}[t]
% \small
\centering
\begin{tabular}{llccccc}
\hline
\multicolumn{1}{l|}{\textbf{Model}} & \textbf{Split} & \textbf{validation}  & \textbf{\begin{tabular}[c]{@{}c@{}}\nt\end{tabular}} & \textbf{\begin{tabular}[c]{@{}c@{}}\nst\end{tabular}} & \textbf{\begin{tabular}[c]{@{}c@{}}\nc\end{tabular}} & \textbf{Abstraction} \\ 
\hline
\multicolumn{2}{l}{ResNet-GloVe} & 96.8 / 96.9 & 60.3 / 56.2 & 84.9 / 84.7 & 83.2 / 82.3 & 63.1 / 54.4\\ 
\hline
\multicolumn{2}{l}{ResNet-GloVe-attention} & 92.8 / 92.9 & 58.8 / 48.7 & 85.4 / 85.2 & 90.5 / 90.7 & 63.9 / 56.6\\ 
\hline
\multicolumn{2}{l}{ResNet-BERT} & 96.9 / 96.9 & 58.8 / 46.4 & 82.9 / 82.3 & 82.2 / 81.3 & 61.4 / 46.5\\ 
\hline
\multicolumn{2}{l}{ResNet-BERT-attention} & 86.6 / 87.1 & 55.3 / 47.4 & 81.0 / 80.4 & 85.3 / 85.6 & 58.0 / 42.5\\
\hline
\multicolumn{2}{l}{I3D-GloVe} & 99.3 / 99.3 & 60.5 / 55.1 & 82.6 / 82.4 & 83.2 / 83.1 & 63.1 / 55.3\\ 
\hline
\multicolumn{2}{l}{I3D-GloVe-attention} & 92.0 / 92.1 & 55.5 / 44.6 & 78.4 / 78.0 & 85.4 / 85.6 & 61.4 / 51.5\\ 
\hline
\multicolumn{2}{l}{I3D-BERT} & 98.9 / 98.9 & 56.7 / 44.5 & 80.8 / 80.3 & 84.7 / 84.4 & 62.4 / 47.1\\ 
\hline
\multicolumn{2}{l}{I3D-BERT-attention} & 90.2 / 90.4 & 56.7 / 45.6 & 80.8 / 79.7 & 84.0 / 83.9 & 62.4 / 47.6\\
\hline
\multicolumn{2}{l}{S3D-BERT} & 99.8 / 99.6 & 58.2 / 45.3 & 78.6 / 76.5 & 85.5 / 85.4 & 64.6 / 55.0\\ 
\hline
\multicolumn{2}{l}{MViT-GloVe} & 99.9 / 99.9 & 64.8 / 61.4 & 88.2 / 88.0 & 88.1 / 87.8 & 65.6 / 57.2 \\ 
\hline
\multicolumn{2}{l}{MViT-GloVe-attention} & 99.6 / 99.6 & 57.7 / 43.4 & 91.0 / 91.1 & 90.3 / 90.3 & 65.4 / 59.8\\ 
\hline
\multicolumn{2}{l}{MViT-BERT} & 99.9 / 99.9 & 58.6 / 47.7 & 83.2 / 82.2 & 88.6 / 88.6 & 62.4 / 49.6 \\ 
\hline
\multicolumn{2}{l}{MViT-BERT-attention} & 97.7 / 97.7 & 58.6 / 49.4 & 87.1 / 87.0 & 88.9 / 88.8 & 62.7 / 48.5\\ 
% \hline
% \multicolumn{2}{l}{MIL-NCE} &  &  &  &  & \\ 
\hline
\multicolumn{2}{l}{CLIP-CLIP} & 99.0 / 98.0 & 55.4 / 38.7 & 85.9 / 85.5 & 84.9 / 83.9 & 59.9 / 44.8\\ 
\hline
\multicolumn{2}{l}{\textbf{NSG (Ours)}} & 97.1 / 97.5 & 90.2 / 90.0 & 73.6 / 64.7 & 84.5 / 84.9 & 80.5 / 80.4\\ 
\hline
\multicolumn{2}{l}{Oracle Model} & 99.6 / 99.7 & 96.4 / 95.0 & 97.4 / 96.7 & 97.4 / 97.6 & 97.0 / 97.2\\ 
\hline
\end{tabular}
\caption{Comparison of models (baselines and the proposed model) on different data splits. The 2 values separated by a "/" denote accuracy and F1-score, respectively.}
\label{table:appendix_baseline_results}
\end{table*}

\subsection{VLM Baselines}
\label{appendix:VLM_baselines}
Majority of VLMs can be characterized based on their approach of extracting and fusing features from vision and text modalities. VLMs for video tasks use either a video encoder or an image encoder with temporal aggregation using sequence model, e.g., transformers~\cite{tf_transformer} or LSTM~\cite{lstm} to obtain the video features. %VLMs using image encoders operate at frame-level and thus require temporal aggregation to embed the video. 
The vision and text encoders can be jointly trained using either~a)~contrastive loss %These image/video encoders can be trained jointly with text encoders using contrastive loss 
that aligns both modalities in a shared latent space e.g., CLIP~\cite{clip}, MIL-NCE~\cite{miech2020end},~b)~masked token prediction losses on the generated text~\cite{blip} aka captioning loss, or~c)~combination of captioning and contrastive losses~\cite{coca}. The vision-text features from encoders can either be fused (multimodal fusion) using attention-based mechanisms or by computing cross-modal similarity scores. 
% Alternatively, generative vision-text models use encoder-decoder architecture and are trained using masked token prediction losses on the generated text~\cite{blip}. Recent work has also proposed training VLMs using both of these losses~\cite{coca}. 
%can then be fused (multimodal fusion) together to obtain a joint representation for a downstream task using attention-based mechanisms. Instead of such explicit fusion, some VLMs also use cross-modal similarity scores for implicit fusion. 
% Some methods \cite{luo2022clip4clip,coca,violin_dataset} use a sequence model, e.g., transformers \cite{tf_transformer} or LSTM\cite{lstm}, to aggregate video features to adapt to longer videos.
We investigate 6 VLMs that span the space of these characteristics for \etv. 

%\begin{itemize}
    \noindent \textbf{CLIP4Clip~\cite{luo2022clip4clip}} uses CLIP-based \cite{clip} text and image encoders. %, followed by ViViT-like~\cite{} temporal transformer for video embedding. 
     Parameterized (e.g., LSTM-based) or non-parameterized (e.g., mean-pooling) aggregation of the resultant image features allows video representation using a single feature vector, without any explicit fusion.
    %image-text similarity computations are used for implicity fusing the modalities. 
    %The image features over time are then processed using a Transformer. They further propose parameterized e.g., LSTM-based and non-parameterized e.g., mean-pooling mechanisms for fusing \rd{unclear whether the similarity calculation is per frame or between video and text..}
    \noindent \textbf{CLIP Hitchhiker~\cite{clip_hitchiker}} uses a similar encoder structure as CLIP4Clip \cite{luo2022clip4clip} and performs weighted-mean pooling of frame embeddings using text-visual similarity scores.  %Instead of doing temporal aggregation of image representations to encode the video, it uses per-frame scoring mechanism based on image-text similarity. This is followed by a simple 
    % A weighted mean of the per-frame embeddings based on the image-text similarity scores is used to embed the video and text into a single feature vector, without any explicit fusion. %Given that the image-text similarity is implicitly used to compute this feature vector, explicit fusion of video and text input is not needed. 
    \noindent \textbf{CoCa~\cite{coca}} uses image-text (dual) encoder-decoder architecture trained using contrastive and captioning loss. The frame-level features are pooled via attentional pooling~\cite{attention_pooling} to model the temporal sequence of the video and then fused with the text features for downstream tasks.
    \noindent \textbf{MIL-NCE~\cite{miech2020end}} learns to encode video and text into a single vector using separate encoders (S3D\cite{s3d}, word2vec\cite{word2vec}) learned from scratch using the proposed multi-instance contrastive loss. 
    \noindent \textbf{VideoCLIP~\cite{videoclip}}~is built on top of MIL-NCE \cite{miech2020end}.~It adds additional transformer layers for video and text encoders, trained using contrastive loss. The resultant embeddings can be fused together for downstream tasks.
    \noindent \textbf{VIOLIN~\cite{violin_dataset}} uses pre-trained image/video (e.g., ResNet~\cite{resnet}, I3D~\cite{i3d}) and text encoders (e.g., GloVe~\cite{glove}, BERT~\cite{bert_model}) separately and fuses the resultant representations from each modality using bi-directional attention~\cite{bidaf}.~For each of the above VLMs, we freeze the pretrained feature extractors for each modality and finetune a fully-connected probe layer, along with the temporal aggregation layers where appropriate (CLIP4Clip-LSTM,~VIOLIN), using \etv's train split. 

    % As shown in Table~\ref{table:appendix_baseline_results}, we also experiment with different combinations of visual (ResNet~\cite{resnet}, I3D~\cite{i3d}, S3D~\cite{s3d}, MViT~\cite{mvit}, CLIP~\cite{clip}) and text encoders (GloVe~\cite{glove}, BERT~\cite{bert_model}, CLIP~\cite{clip}) for the VIOLIN baseline with and without bidirectional attentional flow~\cite{bidaf}.
    % evaluate its performance on our generalization splits.
%\end{itemize}



\subsection{Text2Text Baseline Model}
\label{appendix:upper_bound_models}

In this baseline, we generate video captions from ground-truth objects and sub-task labels and calculate a text similarity score (cosine similarity) between the video caption and the task description using a pretrained RoBERTa model~\cite{roberta} by using a manually set threshold. We start by splitting the video into segments and generate captions for each segment comprising the segment index (for temporal grounding), scene location (kitchen), (\emph{top-k}) objects in the scene, and the activity (sub-task). Example tasks are shown in Table~\ref{text2text_examples}. This baseline is analogous to Socratic Models~\cite{socratic} which generalizes to new applications in zero-shot by leveraging the multimodal capabilities from several pretrained models. For instance, the objects from each segment can be captured using open-vocabulary VLMs~\cite{glip,regionclip,coca}, while the activities for each segment can be detected using \cite{miech2020end,videoclip} by zero-shot classification with text-to-video feature similarity. Similarly, a final summarized video caption for the whole video can be generated using an LLM. %\red{RH: missing references}

We note that despite having ground-truth textual representations of objects and sub-tasks on a scene-by-scene basis, the Text2text baseline model fails to generalize. We attribute this to two reasons:~(1) The pretrained RoBERTa model has limited capacity to capture (out-of-domain) word-level sub-task orderings to determine entailment in \etv, and~(2) Text2Text lacks visual inputs and might suffer from lack of inferring relationships between objects like those captured by \code{RelationQuery} types. 

\begin{table*}[ht]
\centering
\begin{adjustbox}{width=\textwidth}
\begin{tabular}{|l|l|l|l|}
\hline
\textbf{Complex}\textbackslash \textbf{Order} & \textbf{0} & \textbf{1} & \textbf{2} \\ \hline
\textbf{1} & \begin{tabular}[c]{@{}l@{}}
clean\_simple, cool\_simple, heat\_simple\\
pick\_simple, place\_simple, slice\_simple\end{tabular} &  &  \\ \hline
\textbf{2} & \begin{tabular}[c]{@{}l@{}}
\blue{clean\_and\_cool}, clean\_and\_heat\\ clean\_and\_place, clean\_and\_slice\\
cool\_and\_place, heat\_and\_place\\
slice\_and\_cool, slice\_and\_heat\\ slice\_and\_place\end{tabular} & 
\begin{tabular}[c]{@{}l@{}}
\blue{clean\_then\_cool}, clean\_then\_heat\\ clean\_then\_place, clean\_then\_slice\\ \blue{cool\_then\_clean}, cool\_then\_place\\ cool\_then\_slice, heat\_then\_clean\\ 
heat\_then\_place, heat\_then\_slice\\
slice\_then\_clean, slice\_then\_cool\\ 
slice\_then\_heat, slice\_then\_place\end{tabular} &  \\ \hline
\textbf{3} & \begin{tabular}[c]{@{}l@{}}slice\_and\_clean\_and\_place,
\blue{cool\_and\_clean\_and\_place}\\ cool\_and\_slice\_and\_place, 
heat\_and\_clean\_and\_place\\ \blue{slice\_and\_heat\_and\_place},
slice\_and\_heat\_and\_clean\\ \blue{cool\_and\_slice\_and\_clean}\end{tabular} & 

\begin{tabular}[c]{@{}l@{}}\end{tabular} & \begin{tabular}[c]{@{}l@{}}\blue{clean\_then\_cool\_then\_place},
\blue{clean\_then\_cool\_then\_slice}\\ clean\_then\_heat\_then\_place,
clean\_then\_heat\_then\_slice\\ \blue{clean\_then\_slice\_then\_cool},
clean\_then\_slice\_then\_heat\\ \blue{cool\_then\_clean\_then\_place},
cool\_then\_clean\_then\_slice\\ \blue{cool\_then\_slice\_then\_clean},
heat\_then\_clean\_then\_place\\ heat\_then\_clean\_then\_slice,
heat\_then\_slice\_then\_clean\\ \blue{slice\_then\_clean\_then\_cool}, slice\_then\_clean\_then\_heat\\ slice\_then\_clean\_then\_place, \blue{slice\_then\_cool\_then\_clean}\\ slice\_then\_cool\_then\_place, slice\_then\_heat\_then\_clean\\ \blue{slice\_then\_heat\_then\_place},
\blue{clean\_and\_cool\_then\_place}\\ \blue{clean\_and\_cool\_then\_slice}, clean\_and\_heat\_then\_place\\ clean\_and\_heat\_then\_slice, \blue{clean\_and\_slice\_then\_cool}\\ clean\_and\_slice\_then\_heat,
\blue{clean\_then\_cool\_and\_slice}\\ clean\_then\_heat\_and\_slice,
\blue{cool\_and\_slice\_then\_clean}\\ \blue{cool\_then\_clean\_and\_slice}, heat\_and\_slice\_then\_clean\\ heat\_then\_clean\_and\_slice, slice\_and\_clean\_then\_place\\ slice\_and\_cool\_then\_place, \blue{slice\_and\_heat\_then\_place}\\ \blue{slice\_then\_clean\_and\_cool}, slice\_then\_clean\_and\_heat\\ \blue{clean\_then\_cool\_and\_place}, clean\_then\_heat\_and\_place\\ clean\_then\_slice\_and\_place,
slice\_then\_cool\_and\_place\\ \blue{slice\_then\_heat\_and\_place}, slice\_then\_clean\_and\_place\\ heat\_then\_clean\_and\_place, \blue{heat\_then\_slice\_and\_place}\\ \blue{cool\_then\_clean\_and\_place}, cool\_then\_slice\_and\_place\end{tabular} \\ \hline
\end{tabular}
\end{adjustbox}
\vspace{5pt}
\caption{\textbf{List of tasks in \etv dataset} arranged according to complexity (rows) and ordering (columns) in the tasks. A total of 82 tasks were considered for the dataset generation, split into train and novel composition sets. \blue{Blue} denotes novel composition tasks.}
\label{tab:list_of_tasks}
\end{table*}
\begin{figure*}
\centering
    \includegraphics[width=\linewidth]{plots/analysis-all-splits.pdf}
    \caption{\etv dataset analysis. Distribution of \emph{target} objects [row 1, left], \emph{kitchen-scenes} [row 1, right], and \emph{sub-tasks} [row 2] in each split. The Y-axis is in the log scale.}
    \label{figure:tasks-all-splits}
\end{figure*}
