
%% bare_conf_compsoc.tex
%% V1.4b
%% 2015/08/26
%% by Michael Shell
%% See:
%% http://www.michaelshell.org/
%% for current contact information.
%%
%% This is a skeleton file demonstrating the use of IEEEtran.cls
%% (requires IEEEtran.cls version 1.8b or later) with an IEEE Computer
%% Society conference paper.
%%
%% Support sites:
%% http://www.michaelshell.org/tex/ieeetran/
%% http://www.ctan.org/pkg/ieeetran
%% and
%% http://www.ieee.org/

%%*************************************************************************
%% Legal Notice:
%% This code is offered as-is without any warranty either expressed or
%% implied; without even the implied warranty of MERCHANTABILITY or
%% FITNESS FOR A PARTICULAR PURPOSE! 
%% User assumes all risk.
%% In no event shall the IEEE or any contributor to this code be liable for
%% any damages or losses, including, but not limited to, incidental,
%% consequential, or any other damages, resulting from the use or misuse
%% of any information contained here.
%%
%% All comments are the opinions of their respective authors and are not
%% necessarily endorsed by the IEEE.
%%
%% This work is distributed under the LaTeX Project Public License (LPPL)
%% ( http://www.latex-project.org/ ) version 1.3, and may be freely used,
%% distributed and modified. A copy of the LPPL, version 1.3, is included
%% in the base LaTeX documentation of all distributions of LaTeX released
%% 2003/12/01 or later.
%% Retain all contribution notices and credits.
%% ** Modified files should be clearly indicated as such, including  **
%% ** renaming them and changing author support contact information. **
%%*************************************************************************


% *** Authors should verify (and, if needed, correct) their LaTeX system  ***
% *** with the testflow diagnostic prior to trusting their LaTeX platform ***
% *** with production work. The IEEE's font choices and paper sizes can   ***
% *** trigger bugs that do not appear when using other class files.       ***                          ***
% The testflow support page is at:
% http://www.michaelshell.org/tex/testflow/


\documentclass[onecolumn]{IEEEtran}
% \documentclass[conference,compsoc]{IEEEtran}
% Some/most Computer Society conferences require the compsoc mode option,
% but others may want the standard conference format.
%
% If IEEEtran.cls has not been installed into the LaTeX system files,
% manually specify the path to it like:
% \documentclass[conference,compsoc]{../sty/IEEEtran}





% Some very useful LaTeX packages include:
% (uncomment the ones you want to load)


% *** MISC UTILITY PACKAGES ***
%
%\usepackage{ifpdf}
% Heiko Oberdiek's ifpdf.sty is very useful if you need conditional
% compilation based on whether the output is pdf or dvi.
% usage:
% \ifpdf
%   % pdf code
% \else
%   % dvi code
% \fi
% The latest version of ifpdf.sty can be obtained from:
% http://www.ctan.org/pkg/ifpdf
% Also, note that IEEEtran.cls V1.7 and later provides a builtin
% \ifCLASSINFOpdf conditional that works the same way.
% When switching from latex to pdflatex and vice-versa, the compiler may
% have to be run twice to clear warning/error messages.






% *** CITATION PACKAGES ***
%
\ifCLASSOPTIONcompsoc
  % IEEE Computer Society needs nocompress option
  % requires cite.sty v4.0 or later (November 2003)
  \usepackage[nocompress]{cite}
\else
  % normal IEEE
  \usepackage{cite}
\fi
% cite.sty was written by Donald Arseneau
% V1.6 and later of IEEEtran pre-defines the format of the cite.sty package
% \cite{} output to follow that of the IEEE. Loading the cite package will
% result in citation numbers being automatically sorted and properly
% "compressed/ranged". e.g., [1], [9], [2], [7], [5], [6] without using
% cite.sty will become [1], [2], [5]--[7], [9] using cite.sty. cite.sty's
% \cite will automatically add leading space, if needed. Use cite.sty's
% noadjust option (cite.sty V3.8 and later) if you want to turn this off
% such as if a citation ever needs to be enclosed in parenthesis.
% cite.sty is already installed on most LaTeX systems. Be sure and use
% version 5.0 (2009-03-20) and later if using hyperref.sty.
% The latest version can be obtained at:
% http://www.ctan.org/pkg/cite
% The documentation is contained in the cite.sty file itself.
%
% Note that some packages require special options to format as the Computer
% Society requires. In particular, Computer Society  papers do not use
% compressed citation ranges as is done in typical IEEE papers
% (e.g., [1]-[4]). Instead, they list every citation separately in order
% (e.g., [1], [2], [3], [4]). To get the latter we need to load the cite
% package with the nocompress option which is supported by cite.sty v4.0
% and later.





% *** GRAPHICS RELATED PACKAGES ***
%
\ifCLASSINFOpdf
   \usepackage[pdftex]{graphicx}
  % declare the path(s) where your graphic files are
  % \graphicspath{{../pdf/}{../jpeg/}}
  % and their extensions so you won't have to specify these with
  % every instance of \includegraphics
  % \DeclareGraphicsExtensions{.pdf,.jpeg,.png}
\else
  % or other class option (dvipsone, dvipdf, if not using dvips). graphicx
  % will default to the driver specified in the system graphics.cfg if no
  % driver is specified.
  % \usepackage[dvips]{graphicx}
  % declare the path(s) where your graphic files are
  % \graphicspath{{../eps/}}
  % and their extensions so you won't have to specify these with
  % every instance of \includegraphics
  % \DeclareGraphicsExtensions{.eps}
\fi
% graphicx was written by David Carlisle and Sebastian Rahtz. It is
% required if you want graphics, photos, etc. graphicx.sty is already
% installed on most LaTeX systems. The latest version and documentation
% can be obtained at: 
% http://www.ctan.org/pkg/graphicx
% Another good source of documentation is "Using Imported Graphics in
% LaTeX2e" by Keith Reckdahl which can be found at:
% http://www.ctan.org/pkg/epslatex
%
% latex, and pdflatex in dvi mode, support graphics in encapsulated
% postscript (.eps) format. pdflatex in pdf mode supports graphics
% in .pdf, .jpeg, .png and .mps (metapost) formats. Users should ensure
% that all non-photo figures use a vector format (.eps, .pdf, .mps) and
% not a bitmapped formats (.jpeg, .png). The IEEE frowns on bitmapped formats
% which can result in "jaggedy"/blurry rendering of lines and letters as
% well as large increases in file sizes.
%
% You can find documentation about the pdfTeX application at:
% http://www.tug.org/applications/pdftex





% *** MATH PACKAGES ***
%
\usepackage{amsmath}
% A popular package from the American Mathematical Society that provides
% many useful and powerful commands for dealing with mathematics.
%
% Note that the amsmath package sets \interdisplaylinepenalty to 10000
% thus preventing page breaks from occurring within multiline equations. Use:
%\interdisplaylinepenalty=2500
% after loading amsmath to restore such page breaks as IEEEtran.cls normally
% does. amsmath.sty is already installed on most LaTeX systems. The latest
% version and documentation can be obtained at:
% http://www.ctan.org/pkg/amsmath





% *** SPECIALIZED LIST PACKAGES ***
%
\usepackage{algorithm}
\usepackage{algorithmic}
% algorithmic.sty was written by Peter Williams and Rogerio Brito.
% This package provides an algorithmic environment fo describing algorithms.
% You can use the algorithmic environment in-text or within a figure
% environment to provide for a floating algorithm. Do NOT use the algorithm
% floating environment provided by algorithm.sty (by the same authors) or
% algorithm2e.sty (by Christophe Fiorio) as the IEEE does not use dedicated
% algorithm float types and packages that provide these will not provide
% correct IEEE style captions. The latest version and documentation of
% algorithmic.sty can be obtained at:
% http://www.ctan.org/pkg/algorithms
% Also of interest may be the (relatively newer and more customizable)
% algorithmicx.sty package by Szasz Janos:
% http://www.ctan.org/pkg/algorithmicx




% *** ALIGNMENT PACKAGES ***
%
%\usepackage{array}
% Frank Mittelbach's and David Carlisle's array.sty patches and improves
% the standard LaTeX2e array and tabular environments to provide better
% appearance and additional user controls. As the default LaTeX2e table
% generation code is lacking to the point of almost being broken with
% respect to the quality of the end results, all users are strongly
% advised to use an enhanced (at the very least that provided by array.sty)
% set of table tools. array.sty is already installed on most systems. The
% latest version and documentation can be obtained at:
% http://www.ctan.org/pkg/array


% IEEEtran contains the IEEEeqnarray family of commands that can be used to
% generate multiline equations as well as matrices, tables, etc., of high
% quality.




% *** SUBFIGURE PACKAGES ***
%\ifCLASSOPTIONcompsoc
%  \usepackage[caption=false,font=footnotesize,labelfont=sf,textfont=sf]{subfig}
%\else
%  \usepackage[caption=false,font=footnotesize]{subfig}
%\fi
% subfig.sty, written by Steven Douglas Cochran, is the modern replacement
% for subfigure.sty, the latter of which is no longer maintained and is
% incompatible with some LaTeX packages including fixltx2e. However,
% subfig.sty requires and automatically loads Axel Sommerfeldt's caption.sty
% which will override IEEEtran.cls' handling of captions and this will result
% in non-IEEE style figure/table captions. To prevent this problem, be sure
% and invoke subfig.sty's "caption=false" package option (available since
% subfig.sty version 1.3, 2005/06/28) as this is will preserve IEEEtran.cls
% handling of captions.
% Note that the Computer Society format requires a sans serif font rather
% than the serif font used in traditional IEEE formatting and thus the need
% to invoke different subfig.sty package options depending on whether
% compsoc mode has been enabled.
%
% The latest version and documentation of subfig.sty can be obtained at:
% http://www.ctan.org/pkg/subfig




% *** FLOAT PACKAGES ***
%
%\usepackage{fixltx2e}
% fixltx2e, the successor to the earlier fix2col.sty, was written by
% Frank Mittelbach and David Carlisle. This package corrects a few problems
% in the LaTeX2e kernel, the most notable of which is that in current
% LaTeX2e releases, the ordering of single and double column floats is not
% guaranteed to be preserved. Thus, an unpatched LaTeX2e can allow a
% single column figure to be placed prior to an earlier double column
% figure.
% Be aware that LaTeX2e kernels dated 2015 and later have fixltx2e.sty's
% corrections already built into the system in which case a warning will
% be issued if an attempt is made to load fixltx2e.sty as it is no longer
% needed.
% The latest version and documentation can be found at:
% http://www.ctan.org/pkg/fixltx2e


%\usepackage{stfloats}
% stfloats.sty was written by Sigitas Tolusis. This package gives LaTeX2e
% the ability to do double column floats at the bottom of the page as well
% as the top. (e.g., "\begin{figure*}[!b]" is not normally possible in
% LaTeX2e). It also provides a command:
%\fnbelowfloat
% to enable the placement of footnotes below bottom floats (the standard
% LaTeX2e kernel puts them above bottom floats). This is an invasive package
% which rewrites many portions of the LaTeX2e float routines. It may not work
% with other packages that modify the LaTeX2e float routines. The latest
% version and documentation can be obtained at:
% http://www.ctan.org/pkg/stfloats
% Do not use the stfloats baselinefloat ability as the IEEE does not allow
% \baselineskip to stretch. Authors submitting work to the IEEE should note
% that the IEEE rarely uses double column equations and that authors should try
% to avoid such use. Do not be tempted to use the cuted.sty or midfloat.sty
% packages (also by Sigitas Tolusis) as the IEEE does not format its papers in
% such ways.
% Do not attempt to use stfloats with fixltx2e as they are incompatible.
% Instead, use Morten Hogholm'a dblfloatfix which combines the features
% of both fixltx2e and stfloats:
%
% \usepackage{dblfloatfix}
% The latest version can be found at:
% http://www.ctan.org/pkg/dblfloatfix




% *** PDF, URL AND HYPERLINK PACKAGES ***
%
%\usepackage{url}
% url.sty was written by Donald Arseneau. It provides better support for
% handling and breaking URLs. url.sty is already installed on most LaTeX
% systems. The latest version and documentation can be obtained at:
% http://www.ctan.org/pkg/url
% Basically, \url{my_url_here}.
\usepackage{float}
\usepackage{bbm}
\usepackage{amssymb}
\usepackage{subfigure}
\usepackage{graphicx}
\usepackage{epstopdf}

% *** Do not adjust lengths that control margins, column widths, etc. ***
% *** Do not use packages that alter fonts (such as pslatex).         ***
% There should be no need to do such things with IEEEtran.cls V1.6 and later.
% (Unless specifically asked to do so by the journal or conference you plan
% to submit to, of course. )


% correct bad hyphenation here
\hyphenation{op-tical net-works semi-conduc-tor}


\begin{document}

%
% paper title
% Titles are generally capitalized except for words such as a, an, and, as,
% at, but, by, for, in, nor, of, on, or, the, to and up, which are usually
% not capitalized unless they are the first or last word of the title.
% Linebreaks \\ can be used within to get better formatting as desired.
% Do not put math or special symbols in the title.
\title{Joint Radar Communication based on HRIS-Aided Distributed MIMO System}


% author names and affiliations
% use a multiple column layout for up to three different
% affiliations
% \author{\IEEEauthorblockN{Michael Shell}
% \IEEEauthorblockA{School of Electrical and\\Computer Engineering\\
% Georgia Institute of Technology\\
% Atlanta, Georgia 30332--0250\\
% Email: http://www.michaelshell.org/contact.html}
% \and
% \IEEEauthorblockN{Homer Simpson}
% \IEEEauthorblockA{Twentieth Century Fox\\
% Springfield, USA\\
% Email: homer@thesimpsons.com}
% \and
% \IEEEauthorblockN{James Kirk\\ and Montgomery Scott}
% \IEEEauthorblockA{Starfleet Academy\\
% San Francisco, California 96678-2391\\
% Telephone: (800) 555--1212\\
% Fax: (888) 555--1212}}

% conference papers do not typically use \thanks and this command
% is locked out in conference mode. If really needed, such as for
% the acknowledgment of grants, issue a \IEEEoverridecommandlockouts
% after \documentclass

% for over three affiliations, or if they all won't fit within the width
% of the page (and note that there is less available width in this regard for
% compsoc conferences compared to traditional conferences), use this
% alternative format:
% 
%\author{\IEEEauthorblockN{Michael Shell\IEEEauthorrefmark{1},
%Homer Simpson\IEEEauthorrefmark{2},
%James Kirk\IEEEauthorrefmark{3}, 
%Montgomery Scott\IEEEauthorrefmark{3} and
%Eldon Tyrell\IEEEauthorrefmark{4}}
%\IEEEauthorblockA{\IEEEauthorrefmark{1}School of Electrical and Computer Engineering\\
%Georgia Institute of Technology,
%Atlanta, Georgia 30332--0250\\ Email: see http://www.michaelshell.org/contact.html}
%\IEEEauthorblockA{\IEEEauthorrefmark{2}Twentieth Century Fox, Springfield, USA\\
%Email: homer@thesimpsons.com}
%\IEEEauthorblockA{\IEEEauthorrefmark{3}Starfleet Academy, San Francisco, California 96678-2391\\
%Telephone: (800) 555--1212, Fax: (888) 555--1212}
%\IEEEauthorblockA{\IEEEauthorrefmark{4}Tyrell Inc., 123 Replicant Street, Los Angeles, California 90210--4321}}




% use for special paper notices
%\IEEEspecialpapernotice{(Invited Paper)}




% make the title area
% \maketitle

% As a general rule, do not put math, special symbols or citations
% in the abstract
% \begin{abstract}
% The abstract goes here.
% \end{abstract}

% no keywords




% For peer review papers, you can put extra information on the cover
% page as needed:
% \ifCLASSOPTIONpeerreview
% \begin{center} \bfseries EDICS Category: 3-BBND \end{center}
% \fi
%
% For peerreview papers, this IEEEtran command inserts a page break and
% creates the second title. It will be ignored for other modes.
\IEEEpeerreviewmaketitle

\section{Motivation}
Autonomous driving has developed rapidly in decades, which is motivated by the highly precise sensors on-board and breakthroughs in deep learning. The joint radar communication platform is the most significant sensor for autonomous driving systems to achieve targets detection and messages passing between different vehicles and control center. However, the on-board platform will increase the power supplement and hardware deployment in each car. We consider the outboard platform, which is deployed around the buildings or other roadside facilities, as shown in Fig.\ref{fig:sc}. The hybrid reconfigurable intelligence surface (HRIS) is one of the promising sensors to achieve both detection and communication in a passive way, which enhances the efficiency of the energy management in vehicles and spectrum sharing in radar communication platforms. Furthermore, it can be placed against the walls of buildings, which makes it flexible enough to be deployed in the city. Therefore, this outboard platform can monitor each vehicle passing by there and detect the target around the vehicle dynamically. It's obvious that the beamforming design of HRIS is important for us to get the desired performance in communication and high resolution target position in radar detection.
\begin{figure}[htbp]
    \centering
    \includegraphics[width=12cm]{IEEEtran/images/scene.pdf}
    \caption{Autonomous driving scenario}
    \label{fig:sc}
\end{figure}

However, there will be a trade-off between detecting and communicating in designing the HRIS because the user terminals use the reflected signal from the HRIS to communicate when the radar considers it as interference. So our aim is to design a HRIS-aided distributed MIMO system composed of some HRISs to realize the outboard joint radar communication platform.

\section{HRIS-Aided Distributed MIMO System}
% no \IEEEPARstart

We consider a generally distributed MIMO system which is composed of different hybrid reconfigurable intelligence surfaces (RISs) and multi-antennas base station as shown in Fig.\ref{fig:f1}.
\begin{figure}[htbp]
    \centering
    \includegraphics[width=12cm]{images/HRIS1.pdf}
    \caption{HRIS-aided distributed MIMO system}
    \label{fig:f1}
\end{figure}

In this work, we divide the HRIS-aided distributed MIMO system into four parts, base station (BS), hybrid RISs, user terminals and detecting zone. For the first, we use the multi-antennas base station to transmit joint beamforming, it can be realized by a weighted sum of communication symbols and radar waveforms. To be convenient, multi-antennas BS is a uniform linear array of $T$ elements, and we describe it as an active transmission source because of the power consumption. Therefore, the discrete time joint transmit beam signal of the BS in time index $n$ can be written as
\begin{equation}
\label{eq1}
    \begin{aligned}
    \centering
    x(t)=W_cc(t)+W_rs(t), t=0,...,D-1, 
    \end{aligned}
\end{equation}
where $c(t)\in \mathbbm{N}^{K\times1}$ represents $K$ communication symbol streams to pass information to $K$ user terminals, and the $T\times K$ matrix $W_c$ is used to control the communication beam. To be the same definition, the $s(t)\in \mathbbm{C}^{T\times1}$ is a set of individual waveforms of $T$ antennas, and the $T\times T$ matrix $W_r$ is the controllable matrix to generate radar waveforms. The whole length of the discrete time signal is $D$ steps.

The second part is HRIS, it has M blocks of subarray and N elements of each subarray. According to the properties of HRIS, it can reflect and receive the signal arrived at its surface by amplitude and phase codes. And, the reflected matrix from each subarray can be given as
\begin{equation}
\label{eq2}
    \begin{aligned}
    \centering
    \Psi_1(\beta,\psi)=diag([\beta_1e^{j\psi_1},...,\beta_Ne^{j\psi_N}]),
    \end{aligned}
\end{equation}
 where the $\beta$ and $\psi$ are the amplitude and phase codes, respectively. Because of the reflections and low energy consumption, the HRIS can be considered as the negative source.
 
 Limited by hardware design, each subarray is connected to $N_r$ radio frequency (RF) chains to finish received data acquisition. So, the received coefficient from $l$-th element to $r$-th RF chain of each subarray is given as 
 \begin{equation}
 \label{eq3}
     \begin{aligned}
     \centering
      [\phi_1]_{r,l}=(1-\beta_l)e^{j\gamma_{r,l}}|_{m=1},
     \end{aligned}
 \end{equation}
where the $1-\beta_l$ and $\gamma_{r,l}$ are the amplitude and additional phase codes, respectively. And $|_{m=1}$ means that the RF chain is attached in the first subarray.

Then, as shown in Fig\ref{fig:f2}, it demonstrates the absolute geometry of detecting zone and user terminals. In case of multi-users, there are $K$ user terminals near the detecting zone, their receive the communicated signal from the reflection of HRIS to communicate independently. And there is a fix range between the detecting zone center and the users, we divide the detecting zone into $P$ rows and $Q$ columns. Each block in detecting zone may scatter the signal from the BS and  HRISs, and the scattering signal will be received by RF chains on HRISs.

\begin{figure}[htbp]
    \centering
    \includegraphics[width=12cm]{IEEEtran/images/HRIS2.pdf}
    \caption{The absolute geometry of detecting zone and user terminals}
    \label{fig:f2}
\end{figure}


\section{The performance of joint radar communication}
Based on the HRIS-Aided distributed MIMO system, we aim to design the reflected beam and received beam to achieve the best joint radar communication. However, it's not realistic for us to design it separately because of the coupling amplitude coefficient $\beta$ and other properties. For the distributed MIMO radar, we want the lower peak side-lobe level (PSL). And for the MIMO communication, we want the higher signal to interference plus noise rate (SINR). To be convenient, we will ignore the received signal which is from BS to HRISs, and the direct communicated signal which is from BS to users.

\subsection{The evaluation metrics of MIMO communication }
We consider the downlink communication, and the the transmit signal is denoted by $x(t)$. The communicated signal from BS to HRISs and then reflected to users can be written as
\begin{equation}
\label{eq4}
    \begin{aligned}
    \centering
    u(t) = H\Psi(\beta,\psi)Gx(t) + v(t),
    \end{aligned}
\end{equation}
where $v(t)$ is the $(0,\sigma^2)$ additive Gaussian white noise (AWGN). And, $G$ and $H$ represent the channel from BS to HRISs and the channel from HRISs to users,respectively. The controllable matrix $\Psi(\beta,\psi) = diag([\Psi_1,...,\Psi_M]), \Psi\in \mathbbm{C}^{MN\times MN}$ is the whole reflected matrix. Based on the joint beamforming signal, we rewrite the Eq.\ref{eq4} into radar-to-users parts and inter-users parts. It can be expressed as
\begin{equation}
\label{eq5}
    \begin{aligned}
    \centering
    u(t)=H\Psi(\beta,\psi)GW_cc(t)+H\Psi(\beta,\psi)GW_rs(t)+v(t).
    \end{aligned}
\end{equation}

And we define these two equivalent channels as
\begin{equation}
\label{eq6}
    \begin{aligned}
    \centering
    F_c=H\Psi(\beta,\psi)GW_c,
    \end{aligned}
\end{equation}
\begin{equation}
\label{eq7}
    \begin{aligned}
    \centering
    F_r=H\Psi(\beta,\psi)GW_r,
    \end{aligned}
\end{equation}
respectively. Because users cannot typically collaborate with one another, the off-diagonal parts of $F_c$ cause inter-user interference, which should be reduced by precoding. In addition, the users don't have the prior knowledge about the radar waveform, the power of radar-to-users parts also need to be considered as noise. Finally, we can write the useful signal power of $k$-th user as
\begin{equation}
\label{eq8}
    \begin{aligned}
    \centering
    P_c=|[F_c]_{k,k}|^2,
    \end{aligned}
\end{equation}
and the interference power is composed by radar, inter-users and AWGN, which can be written as
\begin{equation}
\label{eq9}
    \begin{aligned}
    \centering
    P_N=\sum_{i\ne k}|[F_c]_{k,i}|^2+\sum_{i=1}^{T}|[F_r]_{k,i}|^2+\sigma^2.
    \end{aligned}
\end{equation}

Finally, the SINR of the users can be formulated as
\begin{equation}
\label{eq10}
    \begin{aligned}
    \centering
    SINR = \min \frac{|[F_c]_{k,k}|^2}{\sum_{i\ne k}|[F_c]_{k,i}|^2+\sum_{i=1}^{T}|[F_r]_{k,i}|^2+\sigma^2}.
    \end{aligned}
\end{equation}

To ensure the performance of communication, we think that the SINR is required to higher than the threshold $\Gamma_c$. Summarize, we choose the SINR as evaluation metrics of MIMO communication, and it may meet the requirement as follow
\begin{equation}
\label{eq11}
    \begin{aligned}
    \centering
    \min \frac{|[F_c]_{k,k}|^2}{\sum_{i\ne k}|[F_c]_{k,i}|^2+\sum_{i=1}^{T}|[F_r]_{k,i}|^2+\sigma^2}>\Gamma_c.
    \end{aligned}
\end{equation}



\subsection{The evaluation metrics of distributed radar receiving beam}
According to the bistatic radar system, we aim to design the receiving beam to get the best receiver performance. So, it's much important to calculate the scattering from different source. Here, we will discuss the positive source and the negative source emission, respectively.

Firstly, the BS is the positive source as aforementioned, so the signal from BS to the detecting zone is the positive source emission. The scattering signal can be expressed as
\begin{equation}
\label{eq12}
    \begin{aligned}
    \centering
    y(t;p,q)=a_t(p,q)x(t),
    \end{aligned}
\end{equation}
where $a_t(p,q)\in\mathbbm{C}^{1\times T}$ is the steering vector from BS to $(p,q)$ block in detecting zone. It's obviously that $r(t;p,q)$ is a function of $(p,q)$.

Secondly, the HRISs are the negative source, thus the signal reflected from HRISs to the detecting zone is the negative source emission. The negative source emitted signal is
\begin{equation}
\label{eq13}
    \begin{aligned}
    \centering
    r_e(t)=\Psi(\beta,\psi)Gx(t),
    \end{aligned}
\end{equation}
and the scattering signal from it can be written as
\begin{equation}
\label{eq14}
    \begin{aligned}
    \centering
    r_s(t;p,q)=a_h(p,q)r_e(t),
    \end{aligned}
\end{equation}
where $a_h(p,q)\in\mathbbm{C}^{1\times MN}$ is the steering vector from HRISs to $(p,q)$ block. Actually, it's also a function of $(p,q)$.

We only use the HRISs to receive the scattering signal, and it arrived at the HRISs is expressed as
\begin{equation}
\label{eq15}
    \begin{aligned}
    \centering
    r(t;m,n)=a_r(p,q)(r_s(t;p,q)+y(t;p,q)),
    \end{aligned}
\end{equation}
where $a_r(p,q)\in\mathbbm{C}^{MN\times M}$ is the steering matrix from $(p,q)$ block to $n$-th element in $m$-th subarray. However, based on the HRISs properties, there is a waveform conversion from element to RF chains. The received signal in each RF chain can be consequently formulated as
\begin{equation}
\label{eq16}
    \begin{aligned}
    \centering
    r(t;m,n_r)=\Phi(\beta,\gamma)a_r(p,q)(r_s(t;p,q)+y(t;p,q)).
    \end{aligned}
\end{equation}
where $\Phi(\beta,\gamma)=[\phi_1,...\phi_M]\in \mathbbm{C}^{N_r\times MN}$ is the received converted matrix.

Furthermore, we rewrite the Eq.\ref{eq16} as a function of $(p,q)$ because of the PSL requirements, as shown in Eq.\ref{eq17}
\begin{equation}
\label{eq17}
    \begin{aligned}
    \centering
    y_r(t;p,q)=\sum_{i=1}^{MN_r}\Phi(\beta,\gamma)a_r(p,q)[a_h(p,q)\Psi(\beta,\psi)G+a_t(p,q)]x(t)+v(t).
    \end{aligned}
\end{equation}

We assume that the center of the detecting zone is our interest point, and the boundary near the users is the interference. Let $\Omega$ as a set of detecting zone points, $\Omega_m$ and $\Omega_s$ are the main-lobe points and side-lobe points, respectively. So, the PSL used to evaluate the performance is expressed as (we ignore the noise)
\begin{equation}
\label{eq18}
    \begin{aligned}
    \centering
    PSL = \max \frac{E(|y_r(t;\Omega_s)|^2)}{E(|y_r(t;\Omega_m)|^2)}.
    \end{aligned}
\end{equation}

Finally, our optimized problem is to minimize the PSL of distributed MIMO radar system, and meet the requirement of communication at the same time. This optimization can be formulated as

\begin{equation}
\label{eq19}
    \begin{aligned}
    \centering
    \min_{W_r,W_c,\beta,\phi,\theta} \max \frac{E(|y_r(t;\Omega_s)|^2)}{E(|y_r(t;\Omega_m)|^2)},
    \end{aligned}
\end{equation}
\begin{equation}
\label{eq20}
    \begin{aligned}
    \centering
    subject \quad to \quad \frac{|[F_c]_{k,k}|^2}{\sum_{i\ne k}|[F_c]_{k,i}|^2+\sum_{i=1}^{T}|[F_r]_{k,i}|^2+\sigma^2}>\Gamma_c, k=1,...,K .
    \end{aligned}
\end{equation}

\newpage
\section{Approaches of designing transmitted beam and HRIS codes}
As aforementioned, to achieve the best performance in joint radar communication, we need to design the transmitted joint beam and the HRIS codes, which cannot be optimized at the same time. In this section, we assume that the BS is a uniform linear array (ULA) and that the number of HRIS is set to 1. Based on this geometry shown in Fig.\ref{fig:Geo_JCR}, we develop an artificially defined strategy in designing the transmitted joint beam, HRIS reflecting beam and HRIS received beam. 
\begin{figure}[htbp]
    \centering
    \includegraphics[width=12cm]{IEEEtran/images/Geometry_JCR.pdf}
    \caption{The geometry of the joint radar communication}
    \label{fig:Geo_JCR}
\end{figure}

\subsection{HRIS coding design}
To complete the HRIS coding scheme, we need to get the BS beam in advance. The first step is to define the threshold of communication, we use the signal-to-noise rate (SNR) to evaluate the performance of the communication, where the SNR of the $k$-th user is written as
\begin{equation}
\label{eq36}
    \begin{aligned}
    \centering
    \eta_c = \frac{h_k^Hh_k}{\sigma^2}
    \end{aligned}
\end{equation}
where $h_k^H$ is the row of $H_e$, and $H_e$ is equal to $H\Psi(\beta,\psi)G$. Assuming that the $H_e\in\mathbbm{C}^{K\times T}$ is the known matrix, and the reflected matrix $\Psi(\beta,\psi)$ in it is obtained before. We need to define the channel from BS to HRIS and the channel from HRIS to users, as shown in Fig.\ref{fig:Geo_JCR}. To be convenient, the BS is deployed in the YOZ plane and is perpendicular to the $z$-axis, the HRIS is deployed on the YOZ plane, the user terminals are on the XOY plane.

Firstly, we consider the channel $G$ definition. The $n$-th element of HRIS has the line of sight path at the angle $\theta_n$ relative to the center of BS, as shown in Fig.\ref{fig:theta_n}.
\begin{figure}[htbp]
	\centering  
	\vspace{-0.35cm} %设置与上面正文的距离
	\subfigtopskip=2pt %设置子图与上面正文或别的内容的距离
	\subfigbottomskip=2pt %设置第二行子图与第一行子图的距离,即下面的头与上面的脚的距离
	\subfigcapskip=-5pt %设置子图与子标题之间的距离
	\subfigure[G]{
		\label{fig:theta_n}
		\includegraphics[width=0.32\linewidth]{IEEEtran/images/Geo_JCR_BSH.pdf}}
	\quad %默认情况下两个子图之间空的较少,使用这个命令加大宽度
	\subfigure[$H_y$]{
		\label{fig:theta_k1}
		\includegraphics[width=0.32\linewidth]{IEEEtran/images/Geo_JCR_HU.pdf}}
	  %这里是空了一行,能够实现强制将四张图分成两行两列显示,而不是放不下图了再换行,使用\\也行。
	\subfigure[$H_y$]{
		\label{fig:theta_k2}
		\includegraphics[width=0.32\linewidth]{IEEEtran/images/Geo_JCR_HU2.pdf}}
	\quad
	\subfigure[$a_t$]{
		\label{fig:theta_t}
		\includegraphics[width=0.32\linewidth]{IEEEtran/images/Geo_JCR_BSD.pdf}}
	\caption{The geometry between each two parts. (a) shows the steering vector from BS to HRIS, which replaces the channel $G$. (b) and (c) are the geometry between the HRIS and users in the YOZ and XOZ planes. (d) illuminates the radar transmitted steering vector from BS to the detecting zone (DZ).}
	\label{level}
\end{figure}
Then the row of $G$ can be replaced by the transmitter antennas steering vector, as follows
\begin{equation}
\label{eq24}
    \begin{aligned}
    \centering
    a_b(n)=[1,e^{jkd\sin{\theta_n}},...,e^{j(T-1)kd\sin{\theta_n}}],
    \end{aligned}
\end{equation}
where $d$ is the inter-antenna spacing in ULA, $k$ is the wavenumber and $\sin{\theta_n}$ is
\begin{equation}
\label{eq25}
    \begin{aligned}
    \centering
    \sin{\theta_n}=\frac{y_n}{\sqrt{(y_n^2+(z_n-h)^2}},
    \end{aligned}
\end{equation}
$(y_n,z_n)$ is the absolute position of the $n$-th element, $h$ is the height of BS. For all elements in a fix HRIS, the channel $G$ is denoted by $A_b|_{m=1}=[a_b^T(1),...,a_b^T(N)]^T\in\mathbbm{C}^{N\times T}$. It's obviously that if we have M blocks of HRISs, the channel $G$ can be expressed as $G=[A_b^T|_{m=1},...,A_b^T|_{m=M}]^T\in\mathbbm{C}^{NM\times T}$. 

For the similar definition, the channel $H$ also can be substituted by the steering vector. However, the HRIS is a two-dimensional array, it will have two lines of sight path. According to the geometry of HRIS and user terminals depicted in Fig.\ref{fig:theta_k1} and Fig.\ref{fig:theta_k2}, the user is at the angle $\theta_{k1}$ and the $\theta_{k2}$ relative to the center of HRIS on the YOZ plane and XOZ plane, respectively. Thus, steering vectors of the central lines of the HRIS on the YOZ plane and XOZ plane are respectively expressed as 
\begin{equation}
\label{eq27}
    \begin{aligned}
    \centering
    a_{uy}(k)=[1,e^{jk\delta\sin{\theta_{k2}}},...,e^{j(\sqrt{N}-1)k\delta\sin{\theta_{k2}}}]^T,
    \end{aligned}
\end{equation}
\begin{equation}
\label{eq26}
    \begin{aligned}
    \centering
    a_{ux}(k)=[1,e^{jk\delta\sin{\theta_{k1}}},...,e^{j(\sqrt{N}-1)k\delta\sin{\theta_{k1}}}]^T,
    \end{aligned}
\end{equation}
where $\delta$ is the inter-element spacing in HRIS. According the geometric relationship between HRIS and users, the $\sin{\theta_{k1}}$ and $\sin{\theta_{k2}}$ are given as
\begin{equation}
\label{eq28}
    \begin{aligned}
    \centering
    \sin{\theta_{k1}}=\frac{h_r}{\sqrt{h_r^2+x_k^2}},
    \end{aligned}
\end{equation}
\begin{equation}
\label{eq29}
    \begin{aligned}
    \centering
    \sin{\theta_{k2}}=\frac{y_k-y_r}{\sqrt{h_r^2+(y_k-y_r)^2}},
    \end{aligned}
\end{equation}
respectively. And $(y_r,h_r)$ is the absolute position of HRIS center, $(x_k,y_k)$ is the position of the users. Then, we can use the $a_u(k)=vec(a_{ux}(k)a_{uy}^T(k))^T$ to represent a row of channel $H$, and we can use $A_u|_{m=1}=[a_u^T(1),...,a_u^T(K)]^T\in\mathbbm{C}^{K\times N}$ to represent all the channels. Accordingly, if there are $M$ blocks of HRISs, the channel $H$ will be expressed as $H=[A_u|_{m=1},...,A_u|_{m=M}]^T \in\mathbbm{C}^{K\times NM}$. In this case, we consider the BS is an omnidirectional antenna that broadcasts the signal, and just one user. Then, the channel $H$ and $G$ are declined into vector $h\in\mathbbm{C}^{1\times N}$ and $g\in\mathbbm{C}^{N\times 1}$, and we separate the reflected matrix $\Psi(\beta, \psi)$ into amplitude vector $\beta\in\mathbbm{R^{1\times N}}$ and phase shift vector $\psi$ that the phases are set to zero. Therefore, the received power at the user can be recast as:
\begin{equation}
\label{P_c}
    \begin{aligned}
    \centering
    P_c = \sum_{i=1}^N h_i\beta_ig_i\cdot(\sum_{i=1}^N h_i\beta_ig_i)^H,
    \end{aligned}
\end{equation}
then, we rewrite it in matrix and use $c_3$ to replace the $h^T\cdot g$, the $P_c$ is expressed as
\begin{equation}
\label{P_c_sim}
    \centering
    P_c = \beta C_3 \beta^H,
\end{equation}
where $C_3 = c_3 c_3^H$ is a Hermitian matrix. It's obvious that the SNR at the user is simply by $\beta C_3 \beta^H$ when we choose the noise variance as $\sigma^2=1$.

To be convenient, we also choose the SINR as the evaluation metric of the radar receiver, and we can calculate the SINR of the RF chain as follow:
\begin{equation}
\label{eq38}
    \begin{aligned}
    \centering
    \eta_{r}=\frac{A_r a_t a_t^H A_r^H}{A_rA_hA_h^HA_r^H+\sigma^2}.
    \end{aligned}
\end{equation}
where $A_h$ is equal to $a_h(p,q)\Psi(\beta,\psi)G$, $\Phi(\beta, \gamma)a_r(p,q)$ is replaced by $A_r$ ,and $a_h(p,q)$ is the same steering vector as $a_u(k)$. We can replace the user position $(x_k, y_k)$ in Eq.\ref{eq28} by the target position $(p,q)$, and use the Eq.\ref{eq26} and Eq.\ref{eq27} to obtain the steering vector $a_h(p,q)$ towards the target position $(p,q)$. In comparison to Eq.\ref{P_c_sim}, we can transfer the expression of $\eta_r$ into a combination of quadratic forms so that it can be more easier to solve. Firstly, we separate the received matrix $\Phi(\beta,\gamma)$ into amplitude vector $1- \beta$ and the phase shift vector $\gamma$ that the phases are set to zero. Thus, the received matrix $A_r$ is written as
\begin{equation}
\label{A-r}
    \centering
    A_r = (1-\beta)a_r.
\end{equation}

Secondly, we rewrite the $A_h$ in the same way as we did in Eq.\ref{P_c}, and it can be expressed as
\begin{equation}
\label{A_h}
    \centering
    A_hA_h^H = \sum_{i=1}^N a_{hi}\beta_ig_i\cdot(\sum_{i=1}^N a_{hi}\beta_ig_i)^H.
\end{equation}

Then, we use $c_2$ to replace the $a_h^T\cdot g$, so that the reflected matrix $A_h$ is written as
\begin{equation}
\label{A_h_c2}
    \centering
    A_hA_h^H = \beta C_2\beta^H,
\end{equation}
where $C_2=c_2c_2^H$ is a Hermitian matrix. Now, we can recast the SINR of the RF chain as follow:
\begin{equation}
\label{eta}
    \centering
    \eta_r = \frac{(1-\beta)a_ra_ta_t^Ha_r^H(1-\beta^H)}{\sigma^2+(1-\beta)a_r\beta C_2\beta^Ha_r^H(1-\beta^H)}.
\end{equation}

As seen in the above formula, the molecule is in quadratic form, and it is possible to convert it to standard form by substituting $C1$ for the $a_ra_ta_t^Ha_r^H$. Then, when we choose the noise variance as $\sigma^2=1$, the $\eta_r$ can be reduced to:
\begin{equation}
\label{etar}
    \centering
    \eta_r = \frac{(1-\beta)C_1(1-\beta^H)}{1+(1-\beta)a_r\beta C_2\beta^Ha_r^H(1-\beta^H)}.
\end{equation}

Finally, the optimization task in HRIS coding design is formulated as
\begin{equation}
\label{obj}
    \centering
    \min_\beta \quad \Gamma_r - \frac{(1-\beta)C_1(1-\beta^H)}{1+(1-\beta)a_r\beta C_2\beta^Ha_r^H(1-\beta^H)},
\end{equation}
\begin{equation}
\label{st}
    \begin{aligned}
    \centering
    subject \quad to \quad &\Gamma_c - \beta C_3 \beta^H<0\\
    & 0<\beta<1.
    \end{aligned}
\end{equation}

We add $\Gamma_r$ which is the ideal SINR of the RF chain to transfer the maximized optimized problem into a minimized problem.

\subsection{Transmitted beam design}
Since we have finished the HRIS coding design, the reflected matrix $\Psi(\beta,\psi)$ and the received matrix $\Phi(\beta,\gamma)$ are known. We rewrite the Eq.\ref{eq5}, as shown in Eq.\ref{eq21}.
\begin{equation}
\label{eq21}
    \begin{aligned}
    \centering
    u(t)=H_e W_c c(t)+H_e W_r s(t)+v(t).
    \end{aligned}
\end{equation}

We denote the covariance matrix of transmitted signal by $R=WW^H$, where $W=[W_c,W_r]$ is the joint controllable matrix of the ULA. And each column of the $W$ represented the codes of BS denotes by $w_i$, the covariance matrix which is the sum of different codes can be written as
\begin{equation}
\label{eq22}
    \begin{aligned}
    \centering
    R=\sum_{i=1}^{K+T}w_iw_i^H.
    \end{aligned}
\end{equation}

Then, by substituting the Eq.\ref{eq22} into the Eq.\ref{eq10}, we can rewrite the expression of the $k$-th user's SINR as
\begin{equation}
\label{eq23}
    \begin{aligned}
    \centering
    \mu_c = \frac{h_k^HR_kh_k}{h_k^HRh_k-h_k^HR_kh_k+\sigma^2}, k=1,...,K ,
    \end{aligned}
\end{equation}

To design the transmitted radar beam, we also choose the SINR as the evaluation metric of radar performance. The transmitted signal arrived at the target position is 
\begin{equation}
\label{eq30}
    \begin{aligned}
    \centering
    y(t)=&r_s(t;p,q)+y(t;p,q)\\
    =&(a_h(p,q)\Psi(\beta,\psi)G+a_t(p,q))x(t),
    \end{aligned}
\end{equation}
and the received signal at the $n_r$-th RF chain is 
\begin{equation}
    \begin{aligned}
    \centering
    r(t;n_r) = \Phi(\beta, \gamma)a_r(p,q)(a_h(p,q)\Psi(\beta,\psi)G+a_t(p,q))x(t) + v(t)
    \end{aligned}
\end{equation}

We consider the transmitted radar waveform that hits the detecting zone to be the valid signal, and the reflected signal from HRIS and the communicated waveform from the ULA that hit the detecting zone are the interference. Therefore, the valid signal power and interference are shown as
\begin{equation}
\label{eq31}
    \begin{aligned}
    \centering
    P_r = A_r a_t^H(p,q)W_rW_r^Ha_t(p,q)A_r^H,
    \end{aligned}
\end{equation}
\begin{equation}
\label{eq32}
    \begin{aligned}
    \centering
    P_n = A_r a_t^H(p,q)W_cW_c^Ha_t(p,q)A_r^H+A_r A_hWW^HA_h^HA_r^H+\sigma^2,
    \end{aligned}
\end{equation}

Finally, Combined with Eq.\ref{eq22}, the SINR of the radar for the point target at position $(p,q)$ is
\begin{equation}
\label{eq33}
    \begin{aligned}
    \centering
    \mu_r=&\frac{\sum_{i=1}^T A_r a_tR_{i+K}a_t^H A_r^H}{\sum_{i=1}^K A_r a_t^HR_ia_tA_r^H+A_r A_hRA_h^HA_r^H+\sigma^2}\\
    =&\frac{A_r a_t(R-\sum_{i=1}^KR_{i})a_t^H A_r^H}{A_ra_t\sum_{i=1}^KR_ia_t^H A_r^H+A_rA_hRA_h^HA_r^H+\sigma^2}.
    \end{aligned}
\end{equation}

Based on the geometric relationship between BS and the detecting zone as shown in Fig.\ref{fig:theta_t}, the steering vector $a_t$ can be calculated by
\begin{equation}
\label{eq34}
    \begin{aligned}
    \centering
    a_t=[e^{-j\frac{T-1}{2}d\sin{\theta_t}},...,1,...,e^{j\frac{T-1}{2}d\sin{\theta_t}}],
    \end{aligned}
\end{equation}
where $\sin{\theta_t}$ is given as
\begin{equation}
\label{eq35}
    \begin{aligned}
    \centering
    \sin{\theta_t}=\frac{q}{\sqrt{q^2+h^2}}.
    \end{aligned}
\end{equation}

Therefore, the optimization task in transmitted beam design is formulated as
\begin{equation}
\label{obj2}
    \centering
    \min_\beta \quad \Gamma_r - \frac{A_r a_t(R-\sum_{i=1}^KR_{i})a_t^H A_r^H}{A_ra_t\sum_{i=1}^KR_ia_t^H A_r^H+A_rA_hRA_h^HA_r^H+1},
\end{equation}
\begin{equation}
\label{st2}
    \begin{aligned}
    \centering
    subject \quad to \quad &\Gamma_c - \frac{h_k^HR_kh_k}{h_k^HRh_k-h_k^HR_kh_k+1}<0, k=1,...,K ,\\
    & rank(R_i)=1, i=1,...,K,...,K+T.
    \end{aligned}
\end{equation}


\section{Results}

\subsection{Results in HRIS coding design}
Here are three optimization algorithms we used for solving the non-convex HRIS coding design problem. The first one is a self-designed dichotomy algorithm. As shown in Algorithm \ref{al1}, its main idea is to split the whole vector into two parts and use grid search in each part to find the best fitness of our objective function. 
\begin{algorithm}
    \caption{Dichotomy based on bisection search}
    \renewcommand{\algorithmicrequire}{\textbf{Input:}}
    \renewcommand{\algorithmicensure}{\textbf{Output:}}
    \label{al1}
    \begin{algorithmic}[1]
        \REQUIRE $C_1$, $C_2$, $C_3$, $a_r$
        \STATE Initialization: randomize $\beta$, $\lambda$
        \STATE Set $obj = \Gamma_r - \frac{(1-\beta)C_1(1-\beta^H)}{1+(1-\beta)a_r\beta C_2\beta^Ha_r^H(1-\beta^H)} + \lambda\cdot 4^{\Gamma_c - \beta C_3 \beta^H}$
        \REPEAT
        \STATE Set frame as $iF = N/2$
        \STATE Consider that all the elements in $\beta(end-iF-1:end)$ are the same value.
        \STATE Find a minimum of $obj$ by searching for $\beta(end-iF-1:end)$
        \STATE Make $N = iF$
        \UNTIL $N = 1$
        \STATE Renew $N$
        \REPEAT
        \STATE Set frame as $iF = N/2$
        \STATE Consider that all the elements in $\beta(1:iF)$ are the same value.
        \STATE Find a minimum of $obj$ by searching for $\beta(1:iF)$
        \STATE Make $N = iF$
        \UNTIL $N = 1$
        \ENSURE $\beta$
    \end{algorithmic}
\end{algorithm}

By using the dichotomy, we can get the best solution $\beta$ fastest, though it's not the global optimum. And Fig.\ref{fig:dicho} gives us the radiation pattern of the HRIS reflected beam, and the main lobe is perpendicular to the YOZ plane. 

The second one is the genetic algorithm (GA), which is one of the most famous non-convex optimization algorithms. We use GA to accomplish the HRIS coding design and its parameters are given in Tab.\ref{tab:ga}. 
\begin{table}[H]
    \centering
    \caption{GA Parameter Configurations}
    \begin{tabular}{p{5cm}|p{5cm}}
    \hline \hline
        Parameters & Value \\
        \hline
        Population size & 100\\
        Parent number & 50\\
        Mutation rate & 0.0625\\
        Maximal generation & 3e6\\
    \hline \hline
    \end{tabular}
    \label{tab:ga}
\end{table}

\begin{figure}[htbp]
	\centering  
	\vspace{-0.35cm} %设置与上面正文的距离
	\subfigtopskip=2pt %设置子图与上面正文或别的内容的距离
	\subfigbottomskip=2pt %设置第二行子图与第一行子图的距离,即下面的头与上面的脚的距离
	\subfigcapskip=-5pt %设置子图与子标题之间的距离
	\subfigure[Self-designed dichotomy]{
		\label{fig:dicho}
		\includegraphics[width=0.32\linewidth]{images/AF_Dichotomy_rand.eps}}
	\subfigure[GA]{
		\label{fig:gaaf}
		\includegraphics[width=0.32\linewidth]{images/AF_GA_100_50_0.0625_3000000.eps}}
	\subfigure[Gradient descent]{
		\label{fig:py}
		\includegraphics[width=0.32\linewidth]{images/AF_Backward_rand.eps}}
	\caption{Optimized HRIS reflected beam. These three radiation patterns are optimized by a self-designed dichotomy algorithm, GA, and gradient descent based on Pytorch, respectively.}
	\label{rp}
\end{figure}

In Fig.\ref{ga}, the value of the objective function is descent along the generations, but the evolution of $\beta$ in each generation changes greatly. So, it cannot converge to a certain value.
\begin{figure}[htbp]
	\centering  
	\vspace{-0.35cm} %设置与上面正文的距离
	\subfigtopskip=2pt %设置子图与上面正文或别的内容的距离
	\subfigbottomskip=2pt %设置第二行子图与第一行子图的距离,即下面的头与上面的脚的距离
	\subfigcapskip=-5pt %设置子图与子标题之间的距离
	\subfigure[The value of objective function descent.]{
		\label{fig:gaobj}
		\includegraphics[width=0.47\linewidth]{images/Best_fitness_GA_100_50_0.0625_3000000.eps}}
	\subfigure[The evolution of $\beta$ in each generations.]{
		\label{fig:gabeta}
		\includegraphics[width=0.47\linewidth]{images/Best_solution_GA_100_50_0.0625_3000000.eps}}
	\caption{The optimization process of the GA in solving HRIS coding design.}
	\label{ga}
\end{figure}

The third one is the gradient descent. However, it's difficult for us to obtain the gradient of $obj$, the automatic gradient descent (AGD) in Pytorch is the best choice to get the gradient by computer. So, we use the AGD to solve this non-convex problem, and its optimized radiation pattern and the loss descent chart are shown in Fig.\ref{fig:py} and Fig.\ref{fig:loss}, respectively. 

Finally, it's important to evaluate which algorithm can achieve the best solution. We give the SINR comparison of different algorithms in both radar and communication in the Tab.\ref{tab:sinr}.
\begin{table}[H]
    \centering
    \caption{SINR comparison of different algorithms}
    \begin{tabular}{p{4.5cm}|p{4.5cm}|p{4.5cm}}
    \hline \hline
        Algorithm & $eta_r$ &$eta_c$ \\
        \hline
        Self-designed dichotomy & 7.7386 dB  &7.7808dB\\
        GA & 8.2687dB & 12.4921dB\\
        AGD & 8.2726dB& 6.6486dB\\
    \hline \hline
    \end{tabular}
    \label{tab:sinr}
\end{table}
\begin{figure}[H]
    \centering
    \includegraphics[width=12cm]{images/Obj_fun_Backward.eps}
    \caption{The loss descent in AGD}
    \label{fig:loss}
\end{figure}

As can be seen, the AGD gets the highest $eta_r$ and the $eta_c$ is higher than the threshold in communication. Turning to the radiation pattern in Fig.\ref{rp}, the main lobe is the same direction in each sub-figure, but the AGD pattern has the lowest side lode. Perhaps this is the critical requirement for optimizing radar performance in the HRIS coding design.

% An example of a floating figure using the graphicx package.
% Note that \label must occur AFTER (or within) \caption.
% For figures, \caption should occur after the \includegraphics.
% Note that IEEEtran v1.7 and later has special internal code that
% is designed to preserve the operation of \label within \caption
% even when the captionsoff option is in effect. However, because
% of issues like this, it may be the safest practice to put all your
% \label just after \caption rather than within \caption{}.
%
% Reminder: the "draftcls" or "draftclsnofoot", not "draft", class
% option should be used if it is desired that the figures are to be
% displayed while in draft mode.
%
%\begin{figure}[!t]
%\centering
%\includegraphics[width=2.5in]{myfigure}
% where an .eps filename suffix will be assumed under latex, 
% and a .pdf suffix will be assumed for pdflatex; or what has been declared
% via \DeclareGraphicsExtensions.
%\caption{Simulation results for the network.}
%\label{fig_sim}
%\end{figure}

% Note that the IEEE typically puts floats only at the top, even when this
% results in a large percentage of a column being occupied by floats.


% An example of a double column floating figure using two subfigures.
% (The subfig.sty package must be loaded for this to work.)
% The subfigure \label commands are set within each subfloat command,
% and the \label for the overall figure must come after \caption.
% \hfil is used as a separator to get equal spacing.
% Watch out that the combined width of all the subfigures on a 
% line do not exceed the text width or a line break will occur.
%
%\begin{figure*}[!t]
%\centering
%\subfloat[Case I]{\includegraphics[width=2.5in]{box}%
%\label{fig_first_case}}
%\hfil
%\subfloat[Case II]{\includegraphics[width=2.5in]{box}%
%\label{fig_second_case}}
%\caption{Simulation results for the network.}
%\label{fig_sim}
%\end{figure*}
%
% Note that often IEEE papers with subfigures do not employ subfigure
% captions (using the optional argument to \subfloat[]), but instead will
% reference/describe all of them (a), (b), etc., within the main caption.
% Be aware that for subfig.sty to generate the (a), (b), etc., subfigure
% labels, the optional argument to \subfloat must be present. If a
% subcaption is not desired, just leave its contents blank,
% e.g., \subfloat[].


% An example of a floating table. Note that, for IEEE style tables, the
% \caption command should come BEFORE the table and, given that table
% captions serve much like titles, are usually capitalized except for words
% such as a, an, and, as, at, but, by, for, in, nor, of, on, or, the, to
% and up, which are usually not capitalized unless they are the first or
% last word of the caption. Table text will default to \footnotesize as
% the IEEE normally uses this smaller font for tables.
% The \label must come after \caption as always.
%
%\begin{table}[!t]
%% increase table row spacing, adjust to taste
%\renewcommand{\arraystretch}{1.3}
% if using array.sty, it might be a good idea to tweak the value of
% \extrarowheight as needed to properly center the text within the cells
%\caption{An Example of a Table}
%\label{table_example}
%\centering
%% Some packages, such as MDW tools, offer better commands for making tables
%% than the plain LaTeX2e tabular which is used here.
%\begin{tabular}{|c||c|}
%\hline
%One & Two\\
%\hline
%Three & Four\\
%\hline
%\end{tabular}
%\end{table}


% Note that the IEEE does not put floats in the very first column
% - or typically anywhere on the first page for that matter. Also,
% in-text middle ("here") positioning is typically not used, but it
% is allowed and encouraged for Computer Society conferences (but
% not Computer Society journals). Most IEEE journals/conferences use
% top floats exclusively. 
% Note that, LaTeX2e, unlike IEEE journals/conferences, places
% footnotes above bottom floats. This can be corrected via the
% \fnbelowfloat command of the stfloats package.





% conference papers do not normally have an appendix



% % use section* for acknowledgment
% \ifCLASSOPTIONcompsoc
%   % The Computer Society usually uses the plural form
%   \section*{Acknowledgments}
% \else
%   % regular IEEE prefers the singular form
%   \section*{Acknowledgment}
% \fi


% The authors would like to thank...





% trigger a \newpage just before the given reference
% number - used to balance the columns on the last page
% adjust value as needed - may need to be readjusted if
% the document is modified later
%\IEEEtriggeratref{8}
% The "triggered" command can be changed if desired:
%\IEEEtriggercmd{\enlargethispage{-5in}}

% references section

% can use a bibliography generated by BibTeX as a .bbl file
% BibTeX documentation can be easily obtained at:
% http://mirror.ctan.org/biblio/bibtex/contrib/doc/
% The IEEEtran BibTeX style support page is at:
% http://www.michaelshell.org/tex/ieeetran/bibtex/
%\bibliographystyle{IEEEtran}
% argument is your BibTeX string definitions and bibliography database(s)
%\bibliography{IEEEabrv,../bib/paper}
%
% <OR> manually copy in the resultant .bbl file
% set second argument of \begin to the number of references
% (used to reserve space for the reference number labels box)
% \begin{thebibliography}{1}

% \bibitem{IEEEhowto:kopka}
% H.~Kopka and P.~W. Daly, \emph{A Guide to \LaTeX}, 3rd~ed.\hskip 1em plus
%   0.5em minus 0.4em\relax Harlow, England: Addison-Wesley, 1999.

% \end{thebibliography}




% that's all folks
\end{document}


