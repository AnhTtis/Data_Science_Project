\documentclass[runningheads]{llncs}
%
\usepackage[T1]{fontenc}
\usepackage{booktabs}

\usepackage{amsmath}
\usepackage{amssymb}
\usepackage{graphicx}
\usepackage{multicol}
\usepackage{multirow}
\usepackage{adjustbox}
\usepackage{hyperref}
\hypersetup{
    colorlinks=true,
    linkcolor=blue,
    filecolor=magenta,      
    urlcolor=cyan,
    pdftitle={Multi-Contrast MRI Super-Resolution via INR},
    pdfpagemode=FullScreen,
    }
\usepackage{orcidlink}
\usepackage{url}


\begin{document}
\title{Multi-contrast MRI Super-resolution via Implicit Neural Representations}

\titlerunning{Multi-contrast MRI Super-resolution via Implicit Neural Representations}


\author{}
\author{Julian McGinnis\thanks{equal contribution}\inst{1,2,3}\orcidlink{0009-0000-2224-7600} \and
Suprosanna Shit$^\star$\inst{1,4,5}\orcidlink{0000-0003-4435-7207} \and
Hongwei Bran Li \inst{5}\and 
Vasiliki Sideri-Lampretsa\inst{1} \and Robert Graf\inst{1,4}\orcidlink{0000-0001-6656-3680} \and Maik Dannecker\inst{1} \and Jiazhen Pan\inst{1}\orcidlink{0000-0002-6305-8117} \and Nil Stolt Ansó\inst{1} \and Mark Mühlau\inst{2,3} \and Jan S. Kirschke\inst{4} \and Daniel Rueckert\inst{1,6} \and Benedikt Wiestler\inst{4}
}
%
\authorrunning{McGinnis and Shit et al.}

\institute{
Institute for AI in Medicine, Technical University Munich, Germany
\and
TUM-Neuroimaging Center, Technical University of Munich, Germany 
\and
Department of Neurology, Technical University of Munich, Germany 
\and
Department of Neuroradiology, Technical University of Munich, Germany
\and
Department of Quantitative Biomedicine, University of Zurich, Switzerland
\and
BioMedIA, Imperial College London, United Kingdom
\\
\email{\{julian.mcginnis,suprosanna.shit\}@tum.de}}


\maketitle              % typeset the header of the contribution

\begin{abstract}

Clinical routine and retrospective cohorts commonly include multi-parametric Magnetic Resonance Imaging; however, they are mostly acquired in different anisotropic 2D views due to signal-to-noise-ratio and scan-time constraints.
Thus acquired views suffer from poor out-of-plane resolution and affect downstream volumetric image analysis that typically requires isotropic 3D scans.
Combining different views of multi-contrast scans into high-resolution isotropic 3D scans is challenging due to the lack of a large training cohort, which calls for a subject-specific framework.
This work proposes a novel solution to this problem leveraging Implicit Neural Representations (INR).
Our proposed INR jointly learns two different contrasts of complementary views in a continuous spatial function and benefits from exchanging anatomical information between them.
Trained within minutes on a single commodity GPU, our model provides realistic super-resolution across different pairs of contrasts in our experiments with three datasets.
Using Mutual Information (MI) as a metric, we find that our model converges to an optimum MI amongst sequences, achieving anatomically faithful reconstruction.\footnote{Code is available at: \url{https://github.com/jqmcginnis/multi_contrast_inr/}}


\keywords{Implicit Neural Representations \and Multi-Contrast Super-resolution \and Mutual Information.}
\end{abstract}

\section{Introduction}

\label{sec:intro}

% \textit{"Drawing and colour are not separate at all; in so far as you paint, you draw. The more the colour harmonizes, the more exact the drawing becomes."} - Paul Cezanne.

Art is a reflection of the figments of human imagination. 
While many are limited in their practical creative capabilities, by pushing the boundaries of digital media, new ways can be created for casual artists and experts alike to express their ideas. At the same time, current neural generative art takes away much of the control from humans. In this work, we attempt to take a step towards restoring some of that control, enabling neural networks to complement users and naturally extend their skills rather than taking hold over the generative process.

% \orr{TBD - make the opening colorful : 1. Add quote:  2. Elaborate: art is a rendering of figments of imaginations of humans. Most people are limited in their drawing capabilities, and by pushing the boundaries we allow new ways for casual artists and experts alike in expressing ideas. At the same time, neural generative art takes a lot of the control away. Here, we want to give back some of this control to humans, such that neural networks complement them and compensate their lack of skills, rather than replacing them.}

% The field of image synthesis has been significantly propelled by neural generative models, particularly by the latest text-to-image models that predominantly rely on large language-image models ~\cite{balaji2022eDiff-I, ramesh2022dalle, rombach2021highresolution, imagen2022saharia}. These models have revolutionized the field of computer vision as they can produce astonishing visual outcomes from text prompts only.

The field of image synthesis has been significantly propelled by neural generative models, particularly by the latest text-to-image models that predominantly rely on large language-image models ~\cite{balaji2022eDiff-I, ramesh2022dalle, rombach2021highresolution, imagen2022saharia}. These models have revolutionized the field of computer vision, as they can produce astonishing visual outcomes from text prompts alone.

The ability of text-to-image models has sparked a wave of editing methods that utilize these models. Many of these techniques rely on prompt editing ~\cite{ fu2022shapecrafter, hertz2022prompt, kawar2022imagic,lin2022magic3d,mokady2022null, poole2022dreamfusion}. Nevertheless, simplifying the interface to text alone means users lack the necessary level of granularity to produce their exact desired outcomes.
% which is} insufficient for effectively editing local content. 
% editing and manipulating visual content, as users lack the necessary level of control to achieve their desired outcomes
Sketch-guided editing, on the other hand, provides intuitive control that aligns with user's conventional drawing and painting skills. By incorporating user-guided sketches into text-to-image models, powerful editing systems can be created, offering a high degree of flexibility and fine-grained control for manipulating visual content~\cite{zhang2023controlnet, voynov2022sketch}.

Although sketch-guided and text-driven methods have proven successful in generating and manipulating 2D images \cite{meng2022sdedit, voynov2022sketch, cheng2023wacv}, it immediately raises the intriguing question of whether a similar approach could be developed to edit 3D shapes. 
Since direct text-to-3D models require an abundance of data to scale, state-of-the-art 3D generative models, such as DreamFusion~\cite{poole2022dreamfusion} and Magic3D~\cite{lin2022magic3d}, which build on the capabilities of text-to-image models, may be considered as an alternative.
% Due to the difficulty of scaling general direct text-to-3D models, incorporating conditions into a text-to-3D model is not straightforward. Thus, state of the art 3D generative models, such as DreamFusion~\cite{poole2022dreamfusion} \orrc{and Magic3D~\cite{lin2022magic3d}}, which build on the capabilities of text-to-image models, may be considered as an alternative.
However, maintaining control via conditioning with such models remains a challenging task, as these generative pipelines optimize a Neural Radiance Field (NeRF) \cite{mildenhall2020nerf} by amortizing gradients from a multitude of 2D views. In particular, providing consistent sketches across all possible views presents a hurdle for users. Instead, a plausible user interface should act with guidance from as few views as possible, e.g. up to two or three.


In this paper, we present \textbf{SKED}, a \textbf{SK}etch-guided 3D \textbf{ED}iting technique. Our method acts on reconstructed or generated NeRF models. We assume a text prompt and a minimum of two sketches as input and provide output edits over the neural field faithful to the input conditions.
Meeting all input requirements can be challenging as the text prompt may not match the sketch's semantics, and sketch views may lack coherence.
To undertake this complex task, we conceptually break it down into two subtasks that are easier to handle: one that depends on pure geometric reasoning and the other that exploits the rich semantic knowledge of the generative model. These two subtasks work together, with the former providing a coarse estimate of location and boundary, and the latter adding and refining geometric and texture details through fine-grained operations.


Our experiments highlight the effectiveness of our approach for editing various pretrained NeRF instances. We introduce assorted accessories, objects, and artifacts, which are generated and blended into the original neural field seamlessly.
Finally, we validate our method through quantitative evaluations and ablation studies to assert the contribution of individual components in our method. 
% By presenting examples in the paper, we illustrate that our method can generate realistic 3D artifacts with accurate texture and geometry using only a few basic sketches.



% Due to the absence of a direct text-to-3D model, incorporating conditions into a text-to-3D model is not straightforward. Thus, 3D generative models, such as DreamFusion~\cite{poole2022dreamfusion}, which build on the capabilities of text-to-image models, may be considered as an alternative.
% However, this is a challenging task since DreamFusion generates a NeRF by integrating many different 2D views. It is very hard to provide consistent sketches across all possible views. The challenge is to use sketches as a guide on only a few views (e.g., two or three) and generate 3D edit of the existing NeRF that is subject to being edited. 

% In this paper, we present \textbf{SKED}, a \textbf{SK}etch-guided 3D \textbf{ED}iting technique, that takes as input a text prompt and a few (two or more) sketches and edits a 3D given object represented as a NeRF in a geometrically plausible and controlled way. 
% We acknowledge the difficulty of this task, as there are no existing text-to-3D generative models available for manipulating the geometry of the existing object based on a text prompt. 
% To undertake this complex task, we conceptually break it down into two simpler subtasks that are easier to handle: one that depends on pure geometric reasoning and the other that exploits the rich semantic knowledge of generative model. These two subtasks work together, with the former providing a coarse estimate of location and the latter adding and refining geometric and texture details through fine-grained operations.

% Our experiments showcase the effectiveness of our approach in performing sketch-guided text-based edits on different base nerfs by introducing various accessories, objects, and artifacts. We also conduct ablation studies and experiments to evaluate the performance of individual components in our method. By presenting examples in the paper, we illustrate that our method can generate realistic 3D artifacts with accurate texture and geometry using only a few basic sketches.

%\dcc{Add here the traditional paragraph that tell about what we achieved and evaluated}
\subsubsection{Related Work.}
Single-image super-resolution (SISR) \cite{bhowmik2017training} aims at restoring a high-resolution (HR) image from a low-resolution (LR) input from a single sequence and targets applications such as low-field MR upsampling or optimization of MRI acquisition \cite{chen2018brain}.
Recent methods \cite{chen2018brain,georgescu2020convolutional} incorporate priors learned from a training set \cite{chen2018brain}, which is later combined with generative models \cite{chen2020mri}.
On the other hand, multi-image super-resolution (MISR) relies on the information from complementary views of the same sequence \cite{wu2021irem} and is especially relevant to capturing temporal redundancy in motion-corrupted low-resolution MRI \cite{gholipour2010robust,wesarg2010combining}.

Multi-contrast Super-resolution (MCSR) targets using inter-contrast priors \cite{rousseau2010non}. In conventional settings \cite{manjon2010mri}, an isotropic HR image of another contrast is used to guide the reconstruction of an anisotropic LR image.
Zeng et al. \cite{zeng2018simultaneous} use a two-stage architecture for both SISR and MCSR.
Utilizing a feature extraction network, Lyu et al. \cite{lyu2020multi} learn multi-contrast information in a joint feature space.
Later, multi-stage integration networks \cite{feng2021multi}, separatable attention \cite{feng2021exploring} and transformers \cite{li2022transformer} have been used to enhance joint feature space learning.
However, all current MCSR approaches are limited by their need for a large training dataset. Consequently, this constrains their usage to specific resolutions and further harbors the danger of hallucination of features (e.g., lesions, artifacts) present in the training set and does not generalize well to unseen data.

Originating from shape reconstruction \cite{park2019deepsdf} and multi-view scene representations \cite{mildenhall2020nerf}, Implicit Neural Representations (INR) have achieved state-of-the-art results by modeling a continuous function on a space from discrete measurements. Key reasons behind INR's success can be attributed to overcoming the low-frequency bias of Multi-Layer Perceptrons (MLP) \cite{tancik2020fourier,sitzmann2020implicit,saragadam2023wire}. 
Although MRI is a discrete measurement, the underlying anatomy is a continuous space. We find INR to be a good fit to model a continuous intensity function on the anatomical space. Once learned, it can be sampled at an arbitrary resolution to obtain the super-resolved MRI.
Following this spirit, INRs have recently been successfully employed in medical imaging applications ranging from k-space reconstruction \cite{huang2023neural} to SISR \cite{wu2021irem}.
Unlike \cite{wu2021irem,shen2022nerp}, which learn anatomical priors in single contrasts, and \cite{wu2022ASSR,amiranashvili2022learning}, which leverage INR with latent embeddings learned over a cohort, we focus on employing INR in subject-specific, multi-contrast settings.
\begin{figure*}[t!]
\includegraphics[width=1.0\linewidth, trim={0 0.3cm 0 0.1cm}, clip]{figures/architecture/architecture.pdf}
\vspace{-15pt}
\caption{
\textbf{Point2Vec pre-training.}
Our model divides the input point cloud into %
point patches using farthest point sampling (FPS) and $k$-NN aggregation.
We obtain patch embeddings by applying a mini-PointNet\,\colorsquare{m_pointnet} to each point patch (\emph{right}).
The teacher Transformer encoder\,\colorsquare{m_green} infers a contextualized %
representation for all patch embeddings which, after normalization and averaging over the last $K$ Transformer layers, serve as training targets.
The student's input is a masked view on the input data, \ie we randomly mask out a ratio of patch embeddings and only pass the remaining embeddings into the student Transformer encoder\,\colorsquare{m_blue}.
After applying a shallow decoder\,\colorsquare{m_red} on the outputs of the student, padded with learned mask embeddings\,\protect\maskembedding{}, we train the student and decoder to predict the latent teacher representation of the patch embeddings.
\vspace{-10pt}
}
\label{fig:model}
\end{figure*}
\section{Method}

The aim of this work is to unlock the full potential of data2vec-like\,\cite{baevski2022data2vec} pre-training on point clouds by addressing point cloud specific challenges.
To achieve this, we first summarize the technical concepts of data2vec (\refsec{method_d2v}) and show how to learn rich representations on point clouds using data2vec pre-training (\refsec{method_d2v_pcl}).
Finally, we propose \name{}, which accounts for the point cloud specific limitations of data2vec (\refsec{method_p2v}).

\subsection{Data2vec}\label{sec:method_d2v}
Data2vec\,\cite{baevski2022data2vec} is designed to pre-train Transformer-based models, which involve a feature encoder that maps the input data to a sequence of embeddings.
These embeddings are subsequently passed to a standard Transformer encoder to generate the final latent representations.
During pre-training, two versions of the Transformer encoder are kept: a \emph{student} and a \emph{teacher}.
The teacher is a momentum encoder, \ie its parameters $\Delta$ track the student's parameters $\theta$ by being updated after each training step according to an exponential moving average (EMA) rule\,\cite{caron2021dino, baevski2022data2vec, grill2020BYOL, he2020moco}: $\Delta \leftarrow \tau \Delta + (1-\tau)\theta$,
where $\tau \in [0,1]$ is the EMA decay rate.
The teacher provides the training targets, which the student predicts given a corrupted version of the same input.

In a first step, the teacher encodes the uncorrupted input sequence.
The training targets are then constructed by averaging the outputs of the last $K$ blocks of the teacher, which are normalized beforehand to prevent a single block from dominating the sum.
Due to the self-attention layers, these targets are \emph{contextualized}, \ie they incorporate global information from the whole input sequence.
This is an important difference to other masked-prediction methods such as BERT\,\cite{devlin2018bert} and MAE\,\cite{he2022mae}, where the targets only comprise local information, \eg a word or an image patch. %

The student is given a masked version of the same input, where some of the embeddings in the input sequence are substituted by a special learned \emph{mask embedding}. %
The student's task is to predict the targets corresponding to the masked parts of the input.
The model is trained by optimizing a Smooth L1 loss on the regressed targets. %







\subsection{Data2vec for Point Clouds}\label{sec:method_d2v_pcl}

To apply data2vec to point clouds, we utilize the same underlying model as Point-BERT\,\cite{yu2021pointbert} and Point-MAE\,\cite{pang2022pointmae}.
This model is well suited for data2vec pre-training: it extracts a sequence of patch embeddings from the input point cloud and feeds it to a standard Transformer encoder.
For downstream tasks, we append a task-specific head to the Transformer encoder (\refsec{experiments}).
Next, we describe the point cloud embedding and the Transformer in detail and conclude with a summary of data2vec for point clouds.


\parag{Point Cloud Embedding.}
First, we sample $n$ center points from the input point cloud using farthest point sampling (FPS)\,\cite{qi2017pointnetplusplus}.
Grouping the center points' $k$-nearest neighbors ($k$-NN) in the point cloud yields $n$ contiguous \emph{point patches}, \ie sub-clouds of $k$ elements.
Next, we normalize the point patches by subtracting the corresponding center point from the patch's points.
This untangles the positional and the structural information.
To account for the permutation-invariant property of point clouds, we employ a mini-PointNet\,\cite{qi2016pointnet} (\reffig{model}, \emph{right}) that maps each normalized point patch to a \emph{patch embedding}.

The mini-PointNet involves the following steps:
First, we map each point of a patch to a feature vector using a shared MLP.
Then, we concatenate max-pooled features to each feature vector.
The resulting feature vectors are then passed through a second shared MLP and a final max-pooling layer to obtain the patch embedding.

\paragraph{Transformer Encoder.}
The central component of the model is a standard Transformer encoder.
The patch embeddings form the input sequence to the Transformer encoder.
Since the point patches are normalized, the patch embeddings carry no positional information;
therefore, a two-layer MLP maps each center point to a position embedding, which is then added to the corresponding patch embedding.
Due to the special importance of positional information in point clouds, the position embeddings are added again before each subsequent Transformer block to ensure that the positional information is incorporated at every step of the encoding process.

\paragraph{\emakefirstuc{\datavec{}}.}

To establish a baseline, we apply the unmodified data2vec approach to the previously described underlying model of Point-BERT and Point-MAE.
Going forward, we will refer to this approach as \datavec{}.


\subsection{\emakefirstuc{\name{}}}\label{sec:method_p2v}
In \reffig{model}, we present the complete pipeline of our \name{} model.
Directly applying data2vec to point cloud data without modifications is not optimal, as the position embeddings are also added to the mask embeddings, revealing the overall shape of the point cloud to the student.
As positions are the only features for point clouds, this makes the masking far less effective, as noted by Pang \etal \cite{pang2022pointmae} in the context of masked autoencoders.

To solve this issue, we adopt an approach inspired by MAE\,\cite{he2022mae}, where we only feed the non-masked embeddings to the student\,\colorsquare{m_blue}.
A separate decoder\,\colorsquare{m_red}, implemented as a shallow Transformer encoder, takes the output of the student and the previously held-back masked embeddings\,\maskembedding{} as input and predicts the training targets.
In contrast to \datavec{}, this approach does not suffer from leaking positional information from the masked-out point patches to the student.
Moreover, utilizing an MAE-inspired setup provides additional benefits:
First, the student is more computationally efficient, as it only needs to process the non-masked embeddings.
Second, the model's inputs during fine-tuning are more similar to those during pre-training because the inputs during pre-training are no longer dominated by masked embeddings which are absent during fine-tuning.
This likely makes the learned representations more transferable to downstream tasks.



\begin{table}[t]

\centering
\resizebox{\linewidth}{!}{
\begin{tabular}{@{}l@{\hspace{2mm}}c@{\hspace{2mm}}*{3}{c@{\hspace{2mm}}}r@{}}
% \begin{tabular}{@{}l@{\hspace{5mm}}c@{\hspace{1.5mm}}c@{\hspace{5mm}}c@{\hspace{4mm}}c@{\hspace{4mm}}c@{\hspace{3mm}}c@{\hspace{1mm}}c@{\hspace{1mm}}c@{\hspace{1mm}}c@{\hspace{1mm}}}

\toprule

% \small{Method & \multicolumn{2}{c}{Neural Metrics} & \multicolumn{3}{c}{User study}} \\
% Method & \multicolumn{2}{c}{Neural Metrics} & \multicolumn{2}{c}{User study} \\
Method & \multicolumn{2}{c}{CLIP Metrics$\uparrow$}& \multicolumn{3}{c}{User Study$\downarrow$} \\
% &
% \multicolumn{6}{c}{Reconstructed HR PSNR$\uparrow$} \\
\cmidrule(l{1mm}r{1mm}){2-3} 
\cmidrule(l{1mm}r{1mm}){4-6} 
% \small{Inversion\& Editing & Clip-score & FVD & Editing  & Image-Quality & Temp-Consis } 
Inversion \& Editing & Tem-Con & Fram-Acc & Edit  & Image & Temp \\

% \cmidrule(l{1mm}r{1mm}){7-13}
\midrule
% NeuralAtlases~\cite{layeraltas} \& NullInv~\cite{null} \\
\small{Framewise Null \& p2p\cite{null,p2p}}  & 0.852 & \textbf{0.958} & 3.55 & 4.11 & 4.38 \\
\small{Framewise SDEit\cite{sdedit}} & 0.910 & 0.819 & 3.69 & 3.28 & 3.62 \\
\small{NLA, Null \& p2p\cite{layeraltas,null,p2p}} & 0.949 & 0.600 & 3.17 & 3.02 & 2.60\\
\small{Tune-A-Video \& DDIM\cite{tuneavideo,ddim}} & 0.958  & 0.750 & 2.78 & 2.80 &  2.70  \\
% \small{Framewise Null\cite{null}} \& p2p\cite{p2p}  & 0.852 & \textbf{0.958} & 3.55 & 4.11 & 4.38 \\
% \small{Framewise SDEit\cite{sdedit}} & 0.910 & 0.819 & 3.69 & 3.28 & 3.62 \\
% \small{NLA\cite{layeraltas}, Null\cite{null} \& p2p\cite{p2p}} & 0.949 & 0.600 & 3.17 & 3.02 & 2.60\\
% \small{Tune-A-Video\cite{tuneavideo} \& DDIM\cite{ddim}} & 0.958  & 0.750 & 2.78 & 2.80 &  2.70  \\
\midrule

Ours & \textbf{0.965} & 0.903 & \textbf{1.82} & \textbf{1.79} & \textbf{1.69} \\

\bottomrule
\end{tabular}
}
% \vspace{-1em}
\caption{\textbf{Quantitative evaluation against baselines.} In our user study, the results of our method are preferred over those from baselines. For CLIP-Score, we achieve the best temporal consistency and comparable framewise editing accuracy against an optimization-based image editing method~\cite{null}.}
\label{table:Quantitative_baseline}
% \vspace{-1em}
\end{table}


\section{Experiments}

\label{sec:exp}
% CAR: 
% Teaser 1. porsche 2. watercolor 
% 3. snow-covered 4. torun: under sky

% boat:
% 1. to run: iced water 2. 


% rabit
% pizza; pokemon latter; 

% sur
% 0226_surf_50_style_ukiyo_640_230227-013245
% cartoon, swarovski
% suppliment 24 frames
% suppliment 24 frames

\subsection{Implementation Details}
For zero-shot style and attribute editing, we directly use the trained stable diffusion v1.4~\cite{stable-diffusion} as the base model, we fuse the attentions in the interval of  $t\in[0.2\times T, T]$ of the DDIM step with total timestep $T=50$.
For shape editing, we utilize the pretrained model of the specific video~\cite{tuneavideo} at 100 iterations and fuse the attention at DDIM timestep $ t \in[0.5\times T, T]$, giving more freedom for new shape generation. Following previous works~\cite{gen1, text2live}, we use videos from DAVIS~\cite{davis} and other in-the-wild videos to evaluate our approach. The source prompt of the video is generated via the image caption model~\cite{BLIP}. Finally, we design the target prompt for each video by replacing or adding several words.
\vspace{-0.5em}
\subsection{Applications}
\noindent\textbf{Local attribute and global style editing.} Using pretrained text-to-image diffusion model~\cite{stable-diffusion}, our framework supports zero-shot local attribute and global style editing, as shown in Fig.~\ref{fig:exp_attribute_style_edit} and third row in Fig.\ref{fig:teaser}. 
In the first row, the texture and color of the feather are modified
by the target prompt \texttt{Swarovski crystal} and kept consistent across frames. In the second and third rows, our framework applies abstract style (\texttt{Ukiyo-e} and \texttt{Makoto Shinkai}). The image structure and temporal motion can be well preserved since we fuse both the spatial-temporal self-attention and cross-attention during the inversion and editing stage.

\noindent\textbf{Shape-aware editing.} Fig.~\ref{fig:exp_swan_shape_edit} and the second row in Fig.\ref{fig:teaser} present the result of difficult object shape editing, with a pretrained video model~\cite{tuneavideo}. This task is challenging because a naive full-resolution fusion of the spatial-temporal self-attention maps results in inaccurate shape results and wrong temporal motion, as shown in the ablation (Fig.\ref{fig:ablation_masked_attention}). Thanks to the proposed Attention Blending, we combine the motion of generated shape from the editing target and inverted attention from the input video. Results of \texttt{posche}, \texttt{duck} and \texttt{flamingo} show that we generate new content with poses and positions similar to input videos.
% \label{fig:exp_swan_shape_edit}

\noindent\textbf{Zero-shot image editing.} In addition, our framework can serve as a zero-shot image editing method such as local attribute editing (Fig.~\ref{fig:attention comparison}) and object shape editing (Fig.~\ref{fig: attention mixing}) by considering an image as a video with a single frame.
% local attribute editing
% object shape editing
% overall style editing
% zero-shot image enhancement (optinal)
% zero-shot object removal (optinal)
We provide more results in our supplementary material.
% zero-shot image enhancement (optinal)

% zero-shot object removal (optinal)
% \newpage
% \input{figs/main_result_shape_editing}
% page 5
% \newpage


\begin{table}[t]

\centering
\resizebox{\linewidth}{!}{
\begin{tabular}{@{}l@{\hspace{2mm}}c@{\hspace{2mm}}*{3}{c@{\hspace{2mm}}}r@{}}
% \begin{tabular}{@{}l@{\hspace{5mm}}c@{\hspace{1.5mm}}c@{\hspace{5mm}}c@{\hspace{4mm}}c@{\hspace{4mm}}c@{\hspace{3mm}}c@{\hspace{1mm}}c@{\hspace{1mm}}c@{\hspace{1mm}}c@{\hspace{1mm}}}

\toprule

% \small{Method & \multicolumn{2}{c}{Neural Metrics} & \multicolumn{3}{c}{User study}} \\
% Method & \multicolumn{2}{c}{Neural Metrics} & \multicolumn{2}{c}{User study} \\
Method & \multicolumn{2}{c}{CLIP Metrics$\uparrow$}& \multicolumn{3}{c}{User Study$\downarrow$} \\
% &
% \multicolumn{6}{c}{Reconstructed HR PSNR$\uparrow$} \\
\cmidrule(l{1mm}r{1mm}){2-3} 
\cmidrule(l{1mm}r{1mm}){4-6} 
% \small{Inversion\& Editing & Clip-score & FVD & Editing  & Image-Quality & Temp-Consis } 
Inversion \& Editing & Tem-Con & Fram-Acc & Edit  & Image & Temp \\

% \cmidrule(l{1mm}r{1mm}){7-13}
\midrule
% NeuralAtlases~\cite{layeraltas} \& NullInv~\cite{null} \\
\small{Framewise Null \& p2p\cite{null,p2p}}  & 0.852 & \textbf{0.958} & 3.55 & 4.11 & 4.38 \\
\small{Framewise SDEit\cite{sdedit}} & 0.910 & 0.819 & 3.69 & 3.28 & 3.62 \\
\small{NLA, Null \& p2p\cite{layeraltas,null,p2p}} & 0.949 & 0.600 & 3.17 & 3.02 & 2.60\\
\small{Tune-A-Video \& DDIM\cite{tuneavideo,ddim}} & 0.958  & 0.750 & 2.78 & 2.80 &  2.70  \\
% \small{Framewise Null\cite{null}} \& p2p\cite{p2p}  & 0.852 & \textbf{0.958} & 3.55 & 4.11 & 4.38 \\
% \small{Framewise SDEit\cite{sdedit}} & 0.910 & 0.819 & 3.69 & 3.28 & 3.62 \\
% \small{NLA\cite{layeraltas}, Null\cite{null} \& p2p\cite{p2p}} & 0.949 & 0.600 & 3.17 & 3.02 & 2.60\\
% \small{Tune-A-Video\cite{tuneavideo} \& DDIM\cite{ddim}} & 0.958  & 0.750 & 2.78 & 2.80 &  2.70  \\
\midrule

Ours & \textbf{0.965} & 0.903 & \textbf{1.82} & \textbf{1.79} & \textbf{1.69} \\

\bottomrule
\end{tabular}
}
% \vspace{-1em}
\caption{\textbf{Quantitative evaluation against baselines.} In our user study, the results of our method are preferred over those from baselines. For CLIP-Score, we achieve the best temporal consistency and comparable framewise editing accuracy against an optimization-based image editing method~\cite{null}.}
\label{table:Quantitative_baseline}
% \vspace{-1em}
\end{table}

\subsection{Baseline Comparisons}
Since there are no available zero-shot video editing methods based on diffusion models, we build the following four state-of-the-art baselines for comparison. (1)~Tune-A-Video~\cite{tuneavideo} overfits an inflated diffusion model on a single video to generate similar content. 
% During shape editing, we find our method can get better editing results using the earlier checkpoints.
% is the similar to use. 
(2) The Neural Layered Atlas~\cite{layeraltas}~(NLA) based method is combined with keyframe-editing via state-of-the-art image editing methods~\cite{null,p2p}.
(3) Frame-wise Null-text optimization~\cite{null} and then edit by prompt2prompt~\cite{p2p}.
(4) Frame-wise zero-shot editing using SDEdit~\cite{sdedit}.
For attention-based editing~(2,3,4), we use the same timesteps fusion parameters as ours.

We conduct the quantitative evaluation using the trained CLIP~\cite{CLIP1} model as previous methods~\cite{gen1,tuneavideo,pix2pix-zero}. Specially, we show the \textbf{`Tem-Con'}~\cite{gen1} to measure the temporal consistency in frames by computing the cosine similarity between all pairs of consecutive frames. \textbf{`Frame-Acc'}~\cite{pix2pix-zero,CLIP1,CLIPScore} is the frame-wise editing accuracy, which is the percentage of frames where the edited image has a higher CLIP similarity to the target prompt than the source prompt. 
In addition, three user studies metrics~(denoted as \textbf{`Edit'}, \textbf{`Image'}, and \textbf{`Temp'}) are conducted to measure the editing quality, overall frame-wise image fidelity, and temporal consistency of the video, respectively. We ask 20 subjects to rank different methods with 9 sets of comparisons in each study. 
From Tab.~\ref{table:Quantitative_baseline}, the proposed zero-shot method achieves the best temporal consistency against baselines and shows a comparable frame-wise editing accuracy as the pre-frame optimization method~\cite{null}. As for the user studies, the average ranking of our method earns user preferences the best in three aspects. 

\begin{figure}[t]
    \centering
    \includegraphics
    [width=0.47\textwidth]
    {figs/imgs/ablation/ablation_attention_inversion_0308_2314-cropped.pdf}
    \caption{ \textbf{Inversion attention compared with reconstruction attention using prompt `deserted shore $\xrightarrow{}$ \textcolor{red}{`glacier shore'}.} The attention maps obtained from the reconstruction stage fail to detect the boat's position, and can not provide suitable motion guidance for zero-shot video editing. 
    % \xiaodong{the purple has some semantic leak in the road, might be replaced to another example}
    } 
    % \vspace{-2em}
    \label{fig:ablation_inversion_attn}
\end{figure}

To provide a qualitative comparison, Fig.\ref{fig:baseline} provides the results of our method and other baselines at two different frames.
% The results of framewise editing methods , since these baselines ignore the temporal relationship.
% Framewise SDEdit~\cite{sdedit} is an efficient zero-shot method
The editing result of framewise SDEdit~\cite{sdedit} can not be localized and varies a lot among different frames. Frame-wise Null inversion achieves local editing at the cost of 500-iterations optimization for each frame but is still temporally inconsistent.
% Although it has the best quantitative `Frame-Acc', its temporal consistency is much worse than other methods in both CLP metrics and user study, because it optimizes each frame individually.
NLA-based~\cite{layeraltas} method 
% ranks second in a temporal-consistency user study and 
preserves the exact pixels in the atlas. However, it struggles to perform editing that involves new shapes or 3D structures. In addition, it takes hours to optimize the neural atlas for each input video.
% Recently, Tune-a-video~\cite{tuneavideo} proposes to tune an inflated stable diffusion model on a single video. 
While Tune-A-Video~\cite{tuneavideo} with DDIM~\cite{ddim} ranks second in editing quality and image fidelity of Tab.~\ref{table:Quantitative_baseline}, we observe that it has difficulty in reproducing the exact motion and spatial position as input video (right side of Fig.\ref{fig:baseline}). Besides, the background has annoying artifacts. Different from the above baselines, our method preserves the motion by fusion the attention during inversion and editing. Thus, our results outperform others by a large margin in our user study and frame consistency measured by CLIP.
% with comparable frame-wise editing accuracy as the image editing method~\cite{null}.


\subsection{Ablation Studies}
Although we have proved the effectiveness of the proposed strategies in Fig.~\ref{fig: attention mixing} and Fig.~\ref{fig:attention comparison} using toy image examples, here, we ablate these designs in the video.

\noindent\textbf{Attention during inversion.} In the right column of Fig.~\ref{fig:ablation_inversion_attn}, we use the attention map during reconstruction instead of inversion for zero-shot background editing. The visualized cross-attention map of the word \texttt{`boat'} in the first and last frame can not capture the correct position and structure of the boat,
% The motion of the object is also inconsistent between the input video and generated video,
which may be caused by the poor temporal modeling capacity of the image diffusion model and the accumulation of errors in DDIM inversion. In contrast, we propose using attention during inversion as the middle column, which provides stable guidance of semantic layout in the original video. We observe this huge difference in attention maps between inversion and reconstruction exists in most videos.
% during our experiments.
% \twocolumn[{
% \maketitle
% \begin{center}
%     \captionsetup{type=figure}
%     \vspace{-2em}
%     \includegraphics[draft,width=1.\textwidth, height=0.3\textwidth]{figs/images/teaser.pdf}
%     \vspace{-2em}
%         \captionof{figure}{Qualitative comparison of our methods with other baselines.}
% \end{center}
% }]
\begin{figure}[t]
    \centering
    \includegraphics
    [width=0.47\textwidth]
    {figs/imgs/ablation/ablation_attention_mask-cropped_0302_2351.pdf}
    \caption{\textbf{Ablation study of blended self-attention.} Without self-attention fusion, the generated video can not preserve the details of input videos (e.g., fence, trees, and car identity).
    % and the identity of the car changes in different timesteps. 
    If we replace full self-attention without a spatial mask, the structure of the original jeep misleads the generation of the Porsche car.}
    % \vspace{-2em}
    \label{fig:ablation_masked_attention}
\end{figure}



\noindent\textbf{Attention Blending Block} is studied in 
Fig.~\ref{fig:ablation_masked_attention}, where we remove all self-attention fusion or fuse all self-attention without a spatial mask. The third column shows that removing all self-attention maps brings a loss of fine details ( \eg, fences, poles, and trees in the background) and inconsistency of car identity over time. In contrast, if we fuse full-resolution self-attention as in the previous work~\cite{p2p}, the shape editing ability of the framework can be severely degraded so that the geometry of generated car resembles the input video, especially in the last few frames. Therefore, we propose to blend the self-attention maps with a mask obtained from cross-attention to preserve unedited details and ensure temporal consistency while editing the object shape.



% newpage

% newpage






\bibliographystyle{splncs04}
\bibliography{references}
\clearpage
\appendix
\title{Appendix}

\begin{table}[h!]
\centering
\caption{Summary of dataset statistics used in our experiments.}
\label{tab:dataset}
\begin{adjustbox}{width=\textwidth}
\begin{tabular}{l|c|c|c|c|c|c|c|c}
\bottomrule
\multirow{2}{*}{Dataset}  & \multicolumn{4}{c|}{Contrast-1} & \multicolumn{4}{c}{Contrast-2} \\
\cline{2-9}
 & Name & Inplane & Dim & Res. (mm) & Name & Inplane & Dim & Res. (mm) \\
 \midrule
BraTS 2019 & T1w & Axial  & (192,192,40)    & $1\times1\times4$      & T2w  & Coronal  & (192,48,160) & $1\times4\times1$   \\
MSSEG 2016 & T1w & Axial  & (160,224,40) & $1\times1\times4$   & Flair  & Sagittal  & (40,224,160) & $4\times1\times1$   \\
cMS & DIR & Axial  & (160,224,40)    & $1\times1\times4$     & Flair  & Sagittal  & (40,224,160) & $4\times1\times1$   \\
\bottomrule 
\end{tabular}
\end{adjustbox}
\end{table}

\begin{table}[h!]
\centering

\caption{To estimate the correct hyperparameters, we perform linear and grid searches on a hold-out set of subjects across all datasets. We list the sweeped hyperparameter ranges and the configurations for the final experiments.}


\label{tab:parameters}
\begin{adjustbox}{width=\textwidth}
\begin{tabular}{l|l|l}
\toprule
  Hyperparameter & Sweep Range & Final \\
 \midrule
  Fourier Features, Distribution Scale & [3.5, 5.0, step=0.1] & 4.0 \\
  Fourier Features, Scaling Factor & [0.5, 1.5, step=0.1] \& [1.0, 10, step=1.0] & 1.0 \\
  Dimension of Fourier Features & [256, 512, 1024]  & 512 \\
  Batch Size & [1000, 3000, 5000, 10000]  & 1000 \\
  Learning Rate & [1$e^{-4}$, 2$e^{-4}$, 4$e^{-4}$] & 4$e^{-4}$ \\
  Epochs & [30, 40, 50, 80, 100] & 50 \\
  Num of Layers & [4,5,6] & 5 \\
  Num of neurons & [256,512,1024,2048] & 1024 \\
\bottomrule
\end{tabular}
\end{adjustbox}
\end{table}

\begin{figure}[h!]
    \centering
    \includegraphics[width=0.48\textwidth,trim=5 0 52 17, clip]{figures/MSSEG_2016_Dataset.pdf}
    \includegraphics[width=0.48\textwidth,trim=5 0 52 17, clip]{figures/cMS_Dataset.pdf}
    \caption{Convergence of predicted $MI(\hat{I}_1, \hat{I}_2)$ shown in a \textbf{dashed line} to the ground truth state $MI(I_1, I_2)$ shown in \textbf{solid line} for five randomly selected subjects (shown in a \textbf{different color}) for two datasets. Note that initially, the MI between two predicted contrasts is high because of randomly initialized shared weights, and over the training period reaches a plateau close to the true equilibrium.} 
    \label{fig:qual}
\end{figure}

\begin{figure}[tbh!]
    \centering
    \includegraphics[clip, trim=7cm 4.25cm 3cm 4.25cm, scale=0.36]{figures/msseg.pdf}
    \includegraphics[clip, trim=7cm 4.25cm 3cm 4.25cm, scale=0.36]{figures/brats_figure.pdf}
    \caption{(Best viewed in fullscreen.) Qualitative comparisons of different models  for a typical subject of the MSSEG (upper part) and BraTS (lower part) dataset. Starting from limited out-of-plane information of the input LR scans, the split-head INR is capable of retrieving recoverable anatomical structures providing truthfulness to its prediction. 
    Exploiting the consistency and mutual anatomical information, the split-head INR can resolve ambiguities in joint reconstruction, as highlighted in yellow boxes, which is impossible if trained in a single contrast setting.
    }
    \label{fig:visual_supp}
    
\end{figure}


\end{document}
