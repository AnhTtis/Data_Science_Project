\title{Appendix}

\begin{table}[h!]
\centering
\caption{Summary of dataset statistics used in our experiments.}
\label{tab:dataset}
\begin{adjustbox}{width=\textwidth}
\begin{tabular}{l|c|c|c|c|c|c|c|c}
\bottomrule
\multirow{2}{*}{Dataset}  & \multicolumn{4}{c|}{Contrast-1} & \multicolumn{4}{c}{Contrast-2} \\
\cline{2-9}
 & Name & Inplane & Dim & Res. (mm) & Name & Inplane & Dim & Res. (mm) \\
 \midrule
BraTS 2019 & T1w & Axial  & (192,192,40)    & $1\times1\times4$      & T2w  & Coronal  & (192,48,160) & $1\times4\times1$   \\
MSSEG 2016 & T1w & Axial  & (160,224,40) & $1\times1\times4$   & Flair  & Sagittal  & (40,224,160) & $4\times1\times1$   \\
cMS & DIR & Axial  & (160,224,40)    & $1\times1\times4$     & Flair  & Sagittal  & (40,224,160) & $4\times1\times1$   \\
\bottomrule 
\end{tabular}
\end{adjustbox}
\end{table}

\begin{table}[h!]
\centering

\caption{To estimate the correct hyperparameters, we perform linear and grid searches on a hold-out set of subjects across all datasets. We list the sweeped hyperparameter ranges and the configurations for the final experiments.}


\label{tab:parameters}
\begin{adjustbox}{width=\textwidth}
\begin{tabular}{l|l|l}
\toprule
  Hyperparameter & Sweep Range & Final \\
 \midrule
  Fourier Features, Distribution Scale & [3.5, 5.0, step=0.1] & 4.0 \\
  Fourier Features, Scaling Factor & [0.5, 1.5, step=0.1] \& [1.0, 10, step=1.0] & 1.0 \\
  Dimension of Fourier Features & [256, 512, 1024]  & 512 \\
  Batch Size & [1000, 3000, 5000, 10000]  & 1000 \\
  Learning Rate & [1$e^{-4}$, 2$e^{-4}$, 4$e^{-4}$] & 4$e^{-4}$ \\
  Epochs & [30, 40, 50, 80, 100] & 50 \\
  Num of Layers & [4,5,6] & 5 \\
  Num of neurons & [256,512,1024,2048] & 1024 \\
\bottomrule
\end{tabular}
\end{adjustbox}
\end{table}

\begin{figure}[h!]
    \centering
    \includegraphics[width=0.48\textwidth,trim=5 0 52 17, clip]{figures/MSSEG_2016_Dataset.pdf}
    \includegraphics[width=0.48\textwidth,trim=5 0 52 17, clip]{figures/cMS_Dataset.pdf}
    \caption{Convergence of predicted $MI(\hat{I}_1, \hat{I}_2)$ shown in a \textbf{dashed line} to the ground truth state $MI(I_1, I_2)$ shown in \textbf{solid line} for five randomly selected subjects (shown in a \textbf{different color}) for two datasets. Note that initially, the MI between two predicted contrasts is high because of randomly initialized shared weights, and over the training period reaches a plateau close to the true equilibrium.} 
    \label{fig:qual}
\end{figure}

\begin{figure}[tbh!]
    \centering
    \includegraphics[clip, trim=7cm 4.25cm 3cm 4.25cm, scale=0.36]{figures/msseg.pdf}
    \includegraphics[clip, trim=7cm 4.25cm 3cm 4.25cm, scale=0.36]{figures/brats_figure.pdf}
    \caption{(Best viewed in fullscreen.) Qualitative comparisons of different models  for a typical subject of the MSSEG (upper part) and BraTS (lower part) dataset. Starting from limited out-of-plane information of the input LR scans, the split-head INR is capable of retrieving recoverable anatomical structures providing truthfulness to its prediction. 
    Exploiting the consistency and mutual anatomical information, the split-head INR can resolve ambiguities in joint reconstruction, as highlighted in yellow boxes, which is impossible if trained in a single contrast setting.
    }
    \label{fig:visual_supp}
    
\end{figure}
