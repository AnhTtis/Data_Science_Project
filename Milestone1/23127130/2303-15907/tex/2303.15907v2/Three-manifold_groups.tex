\documentclass[12pt,a4paper]{amsart}
\usepackage[utf8]{inputenc}

\usepackage{amssymb,amsmath,amsthm}
\usepackage[abbrev,alphabetic]{amsrefs}
\usepackage{tikz-cd}
\usetikzlibrary{arrows}
\usepackage{graphicx}
\usepackage[hidelinks]{hyperref}
\usepackage{mathtools}
\usepackage[titletoc,title]{appendix}
\usepackage{graphicx}
\usepackage{caption}
\usepackage{subcaption}
\usepackage{blindtext}
\usepackage[titletoc,title]{appendix}
\usepackage[english]{babel}
\usepackage{mathrsfs}
\usepackage{microtype}
\usepackage{xcolor}
\usepackage{stmaryrd}

\usepackage[capitalize]{cleveref}

\newtheorem{theorem}{Theorem}[section]
\newtheorem{lemma}[theorem]{Lemma}
\crefname{lemma}{Lemma}{Lemmata}
\newtheorem{corollary}[theorem]{Corollary}
\newtheorem{proposition}[theorem]{Proposition}
\theoremstyle{definition}
\newtheorem{definition}[theorem]{Definition}
\newtheorem{remark}[theorem]{Remark}
\newtheorem{example}[theorem]{Example}
\newtheorem{conjecture}[theorem]{Conjecture}
\newtheorem*{theorem*}{Theorem}
\newtheorem*{conjecture*}{Conjecture}


\newcommand{\isom}{\cong}

\newcommand{\N}{\mathbb{N}}
\newcommand{\Z}{\mathbb{Z}}
\newcommand{\Q}{\mathbb{Q}}
\newcommand{\R}{\mathbb{R}}
\newcommand{\C}{\mathbb{C}}

\DeclareMathOperator{\Ima}{Im}
\newcommand{\group}[2]{\langle #1 \mid #2 \rangle}
\newcommand{\normal}[1]{\langle\!\langle #1 \rangle\!\rangle}
\newcommand{\bignormal}[1]{\left\langle\left\langle #1 \right\rangle\right\rangle}
\newcommand{\st}{\mid}
\newcommand{\D}{\mathcal D}
\newcommand{\F}{\mathbb F}
\def\immerses{\looparrowright}
\def\injects{\hookrightarrow}

\DeclareSymbolFontAlphabet{\amsmathbb}{AMSb}

\DeclareMathOperator{\ab}{ab}
\DeclareMathOperator{\bs}{BS}
\DeclareMathOperator{\cd}{cd}
\DeclareMathOperator{\vcd}{vcd}
\DeclareMathOperator{\fp}{FP}
\DeclareMathOperator{\rk}{rk}
\DeclareMathOperator{\core}{Core}
\DeclareMathOperator{\comm}{Comm}
\DeclareMathOperator{\cay}{Cay}
\DeclareMathOperator{\coker}{coker}
\DeclareMathOperator{\cat}{CAT}


\newcounter{dawidcomments}
\newcommand{\dawid}[1]{\textbf{\color{red}(D\arabic{dawidcomments})} \marginpar{\tiny\raggedright\textbf{\color{red}(D\arabic{dawidcomments})Dawid:} #1}
\addtocounter{dawidcomments}{1}}

\newcounter{marcocomments}
\newcommand{\marco}[1]{\textbf{\color{blue}(M\arabic{marcocomments})} \marginpar{\tiny\raggedright\textbf{\color{blue}(M\arabic{marcocomments})Marco:} #1}
\addtocounter{marcocomments}{1}}


\title{The Atiyah conjecture for three-manifold groups}
\author{Dawid Kielak}
\author{Marco Linton}

\address[DK]{University of Oxford, Oxford, OX2 6GG, UK}

\email{kielak@maths.ox.ac.uk}

\address[ML]{University of Oxford, Oxford, OX2 6GG, UK}

\email{marco.linton@maths.ox.ac.uk}


\begin{document}

\begin{abstract}
	We show that finitely generated fundamental groups of three-manifolds satisfy the Strong Atiyah Conjecture over the complex numbers. This implies that when the group is additionally torsion-free, then its complex group ring satisfies the Kaplansky Zero-divisor Conjecture.
	
As an application, we give a very short proof of a significant generalisation of a recent result of Shalen dealing with the minimal index of freedom of three-manifold groups.
\end{abstract}	

\maketitle



\section{Introduction}

If $K$ is a field and $G$ is a torsion-free group, then the well-known Kaplansky Zero-divisor Conjecture predicts that the group ring $KG$ has no non-trivial zero-divisors. An affirmative answer has been established for many classes of groups: locally indicable groups \cite[Theorem 12]{higman_40}, elementary amenable groups \cite[Theorem 1.4]{kropholler_88}, free-by-elementary amenable groups \cite[Theorem 1.3]{linnell_93} and left orderable groups \cite[Proposition 6]{rolfsen_98}.

If a torsion-free group $G$ satisfies the Strong Atiyah Conjecture over $\C$, then its group ring $\C G$ embeds in a skew-field and thus cannot contain any non-trivial zero-divisors -- for the statement of the conjecture and related background see the book of L\"uck~\cite{luck_02}*{Section 10}. Linnell established the Strong Atiyah Conjecture over $\C$ for torsion-free groups within a large class $\mathcal{C}$ of groups \cite[Theorem 1.5]{linnell_93}. The class $\mathcal{C}$ is defined to be the smallest class of groups containing all free groups that is closed under directed unions and extensions by elementary amenable groups.

The Strong Atiyah Conjecture has been shown to hold for many three-manifold groups by  Friedl--L{\"{u}}ck \cite{friedl_19}. However, known results do not extend beyond fundamental groups of compact three-manifolds with empty or toroidal boundary. In this article, we show that all three-manifold groups lie in the class $\mathcal{C}$. In order to do this, we prove a strong algebraic result for lower-dimensional three-manifold groups, stated below. Recall that a group is free-by-cyclic if it admits a homomorphism to a cyclic group with a possibly infinitely generated free kernel.

\begin{theorem}
	\label{main thm}
	Fundamental groups of three-manifolds lie in Linnell's class $\mathcal C$. Moreover, if $G$ is such a group with $\cd_{\Q}(G) <3$ then $G$ is locally virtually free-by-cyclic, where $\cd_{\Q}$ denotes the rational cohomological dimension.
\end{theorem}

After the proof of \cref{main thm}, we present an example of a torsion-free three-manifold group that is locally virtually free-by-cyclic, but not virtually free-by-cyclic itself, demonstrating that \cref{main thm} is, in a sense, sharp.

A useful fact that seems to have escaped being mentioned in the literature is that Linnell's class $\mathcal{C}$ is closed under free products; we establish this fact in \cref{C_section}. Beyond that, we use standard three-manifold techniques (prime decomposition, the Sphere and Loop Theorems), virtual-fibring theorems, the Compact Core Theorem of Scott \cite{Scott1973} (a version of which was proved independently by Shalen, see \cite{Scott1973a}*{Footnote $\dagger$}), and the work of Friedl--L\"uck \cite{friedl_19}.

\begin{corollary}
	\label{main cor}
Let $G$ be the fundamental group of a three-manifold.
\begin{enumerate}
\item If $G$ is finitely generated or torsion-free, then $G$ satisfies the Strong Atiyah Conjecture over $\C$.
\item If $G$ is torsion-free, then $\C G$ has no non-trivial zero divisors and hence the Kaplansky Zero-divisor Conjecture holds for $\C G$.
\end{enumerate}
\end{corollary}

\cref{main cor} answers two questions of Aschenbrenner--Friedl--Wilton \cite{aschenbrenner_15}*{Question 7.2.6, (1) \& (2)}.

\smallskip

In a recent article \cite{Shalen2023}, Shalen studied the \emph{minimal index of freedom} $\mathrm{miof}(G)$ of finitely generated fundamental groups of three-manifolds. Given a finite generating set $S$ of $G$, the \emph{index of freedom} of $S$ is defined to be the cardinality of the maximal subset of $S$ that freely generates a free subgroup of $G$; $\mathrm{miof}(G)$ is then the minimal index of freedom among all finite generating sets of $G$.

Using representation varieties, Shalen proved the following.

  \begin{theorem}[\cite{Shalen2023}*{Theorem B}]
  	\label{Shalen}
  	Let $M$ be an orientable hyperbolic $3$-manifold, and let $G$ be a finitely generated subgroup of $\pi_1(M)$. The Euler characteristic $\chi(G)$ satisfies
  	\[
  	-\chi(G) < \mathrm{miof}(G).
  	\]
  \end{theorem}
Note that $\chi(G)$ is always defined for groups as above.

In \cref{shalen_general}, we show that the geometric hypothesis on $M$ can be dropped entirely, and the theorem is true for all finitely generated torsion-free three-manifold groups.

\subsection*{Acknowledgements}
This work has received funding from the European Research Council (ERC) under the European Union's Horizon 2020 research and innovation programme (Grant agreement No. 850930).

The authors would like to thank Adele Jackson for insightful conversations on Seifert-fibred spaces, and Peter Shalen for comments on an earlier version of this article.



\section{Linnell's class $\mathcal{C}$}
\label{C_section}


In this section we show that Linnell's class $\mathcal{C}$ is closed under free products, but is not closed under direct sums.


We first remind the reader of the definition of elementary amenable groups. Denote by $\mathcal{EA}_0$ the class of groups that are abelian or finite. For each ordinal $\alpha$, we define $\mathcal{EA}_\alpha$ to consist of extensions of groups in $\mathcal{EA}_{\beta}$ for $\beta < \alpha$, and all directed unions of groups, each of which lies in some $\mathcal{EA}_\beta$ with $\beta<\alpha$. The union of all the classes $\mathcal{EA}_\alpha$ is precisely the class of elementary amenable groups.

We may similarly stratify Linnell's class $\mathcal{C}$. Denote by $\mathcal{C}_0$ the class of free groups. For each ordinal $\alpha$, let $\mathcal{C}_{\alpha}$ consist of all elementary amenable extensions of groups in $\mathcal{C}_{\beta}$ for $\beta < \alpha$, and all directed unions of groups, each of which lies in some $\mathcal C_\beta$ with $\beta<\alpha$. It is clear that the class $\mathcal C$ is precisely the union of the classes $\mathcal C_\alpha$ taken over all ordinals $\alpha$.

It is easy to see that every  $\mathcal{EA}_\alpha$ and every $\mathcal C_\alpha$ are closed under taking subgroups.


\begin{proposition}
	\label{C_free_products}
	Linnell's class $\mathcal{C}$ is closed under arbitrary free products.
\end{proposition}

\begin{proof}	
	Let $\alpha$ be an ordinal. Consider a free product $*_{i\in I}G_i$ where for all $i$, $G_i\in \mathcal{C}_{\alpha}$. We claim that $*_{i\in I}G_i\in\mathcal{C}$. The proof is by transfinite induction. Since free products of free groups are free, the base case holds.
	
	Now consider an ordinal $\alpha>0$, and suppose that the claim is true for all ordinals $\beta$ with $\beta < \alpha$.
	Let $J \subseteq I$ be such that for every $i \not\in J$ we have
\[
1\to N_i\to G_i\to A_i\to 1
\]
with $N_i\in \mathcal{C}_{\beta_i}$ where $\beta_i < \alpha$, and with $A_i$ elementary amenable (and possibly trivial), and for every $i \in J$ we have $G_i = \bigcup_{j\in J_i}G_{i, j}$ with $G_{i, j}\in \mathcal{C}_{\beta_{i,j}}$ for some $\beta_{i,j}<\alpha$. For $\beta < \alpha$ let $G_{i,\beta}$ denote the union of the subgroups $G_{i,j}$ such that $\beta_{i,j}< \beta$; if no such subgroup exists, we take $G_{i,\beta}$ to be the trivial group. By definition, $G_{i,\beta} \in \mathcal C_\beta$.

 Consider the homomorphism
	\[
	*_{i\in I}G_i\to \bigoplus_{i\in I - J}A_i
	\]
	obtained from the homomorphisms above in the obvious way, and with groups $G_i$ for $i \in J$ lying in the kernel.
	Note that the image is a subgroup of a direct sum of elementary amenable groups, and hence an elementary amenable group itself.
	By the Kurosh Subgroup Theorem, the kernel $K$ is a free product of conjugates of the groups $N_i$, conjugates of the groups $G_i$ with $i \in J$, and a free group $F$.
	
	Take $\beta<\alpha$. Let $K_\beta$ denote the free product of $F$ with the conjugates of the groups $N_i$ that lie in $\mathcal C_\beta$, and with conjugates of groups $G_{i,\beta}$ for $i \in J$. By the inductive hypothesis, $K_\beta$ lies in $\mathcal C$. Note that $K = \bigcup_{\beta<\alpha} K_\beta$, and hence $K$ lies in $\mathcal C$. But then  
	$*_{i\in I}G_i$ lies in $\mathcal C$ as well, being an extension of $K$ by an elementary amenable group.
	
	We finish by observing that every free product of groups in $\mathcal C$ is a free product of groups lying in $\mathcal C_\alpha$ for some $\alpha$, since $\mathcal C$ is the union of the classes $\mathcal C_\alpha$, and ordinals are closed under taking unions.
\end{proof}


We remark here that Schick proved in \cite{schick_00} (see also \cite{schick_02}) that class $\mathcal{D}$, a class containing the torsion-free groups from $\mathcal{C}$, is closed under free products. However, the argument relies on class $\mathcal{D}$ being closed under direct sums, which class $\mathcal{C}$ is not by the following lemma.

\begin{lemma}
If $G = A\times B\in \mathcal{C}$, then $A, B\in \mathcal{C}$ and at least one of $A$ or $B$ is elementary amenable.
\end{lemma}

\begin{proof}
Class $\mathcal{C}$ is closed under taking subgroups, so certainly $A, B\in \mathcal{C}$. Suppose now that $A$ and $B$ are not elementary amenable. As elementary amenable groups are closed under directed unions and extensions, it follows that both $A$ and $B$ must contain a non-abelian free subgroup. In particular, $G$ contains a copy of $F_2\times F_2$. 

Let $\alpha$ be the first ordinal such that $\mathcal{C}_{\alpha}$ contains a group $K$ containing a copy of $F_2\times F_2$. If $K$ is a directed union of groups, each of which lies in some $C_{\beta}$ with $\beta<\alpha$, then since $F_2\times F_2$ is finitely generated, it appears as a subgroup of one of the groups in this directed union, contradicting minimality of $\alpha$. If $K$ is an extension of a group $N$ in $C_{\beta}$, for $\beta<\alpha$, by an elementary amenable group $Q$, then we claim that $N$ must also contain a copy of $F_2\times F_2$. Indeed let $L$ and $R$ be the kernels of the induced maps from the two $F_2$ factors to $Q$. As $Q$ is elementary amenable and non-trivial normal subgroups of non-abelian free groups are non-abelian free, we must have that $L$ and $R$ are non-abelian free groups. Since $L\times R\leqslant N$, the claim follows. But this also contradicts the minimality of $\alpha$. So no group in $\mathcal{C}$ can contain $F_2\times F_2$, a contradiction.
\end{proof}

Recall that a group has property \emph{$\mathrm{FAb}$} if its finite-index subgroups have finite abelianisations. Property $\mathrm{FAb}$ is a consquence of Property $T$. It is immediate that \emph{$\mathrm{FAb}$} passes to finite-index subgroups and overgroups. 

\begin{lemma}
	If $G$ is a finitely generated group with $\mathrm{FAb}$ lying in  $\mathcal C$ then every elementary amenable quotient of $G$ is finite. Therefore, $G$ is itself finite.
\end{lemma}   
\begin{proof}
	We will argue by contradiction.
	Take $\alpha$ to be the smallest ordinal such that there exists $Q \in \mathcal{EA}_\alpha$ that is infinite and fits into a short exact sequence
	\[1\to N \to G \to Q\to1 \]
	with $G$ a finitely generated group in $\mathcal C$ with property $\mathrm{FAb}$. 
	
	Since $Q$ is finitely generated and infinite, it cannot be abelian as $G$ has $\mathrm{FAb}$. Hence we see that $\alpha \neq 0$ and so $Q$ is an extension
	\[
	1\to Q_0 \to Q \to Q_1\to1
	\]
	with $Q_0$ and $Q_1$ lying in $Q \in \mathcal{EA}_\beta$ for some $\beta < \alpha$.
 	Thus $G$ maps onto $Q_1$ with kernel an extension of $N$ by an elementary amenable group. This forces the kernel to lie in $\mathcal C$, and hence $Q_1$ is finite by minimality of $\alpha$. But then a finite-index subgroup of $G$ fits into an exact sequence with kernel $N$ and quotient $Q_0$. By minimality of $\alpha$, the group $Q_0$ is finite as well. This proves that $Q$ was also finite, a contradiction.
	
	\smallskip
	We will now prove the last claim. Again, this will be done by contradiction. Suppose that $\gamma$ is the smallest ordinal such that $\mathcal C_\gamma$ contains an infinite, finitely generated group $G$ with $\mathrm{FAb}$. Since non-trivial free groups do not have $\mathrm{FAb}$, and since $G$ is finitely generated, the group $G$ must fit into a short exact sequence 
	\[
	1\to N \to G \to Q \to 1
	\] 
	with $N$ in $\mathcal C_\delta$ for some $\delta<\gamma$, and with $Q$ elementary amenable. But then $Q$ must be finite, and hence $N$ has property $\mathrm{FAb}$. This contradicts the minimality of $\gamma$.
\end{proof}

\section{Three-manifolds}


The first two statements in the following result appear as \cite[Lemma 14 \& Theorem 15]{baumslag_09}. The third is a straightforward application of the Kurosh Subgroup Theorem.

\begin{proposition}
	\label{closure}
	\begin{enumerate}
		\item A free product of any number of free-by-$\Z$ groups is free-by-$\Z$.
		\item A free product of finitely many virtually free-by-cyclic groups is virtually free-by-cyclic.
		\item A free product of finitely many virtually torsion-free groups is virtually torsion-free.
	\end{enumerate}	
\end{proposition}


The following is a summary of well-known facts about three-manifold groups.

\begin{proposition}
\label{facts}
Let $M$ be a connected three-manifold and let $G$ be its fundamental group. If $G$ is finitely generated, then there is a finitely generated free group $F$ and finitely many compact, connected, orientable, aspherical three-manifolds $M_1, \dots, M_n$, each with a (possibly trivial) incompressible boundary, such that a finite-index subgroup of $G$ is isomorphic to the free product $F*\left(*_{i=1}^n\pi_1(M_i)\right)$.
\end{proposition}

\begin{proof}
By replacing $M$ with its orientation double cover, we may assume that $M$ is orientable. The Compact Core Theorem allows us to assume that $M$ is compact. Since an orientable prime non-irreducible three-manifold is homeomorphic to $S^2\times S^1$, which has fundamental group $\Z$, we may combine the Prime Decomposition Theorem with the Loop Theorem (in the form of \cite[Lemma 1.4.2]{aschenbrenner_15}) to obtain a finitely generated free group $F$ and finitely many compact, connected, irreducible and orientable three-manifolds $M_1, \dots, M_k$ with (non-spherical and possibly empty) incompressible boundaries, such that $G\isom F*\left(*_{i=1}^k\pi_1(M_i)\right)$.

If $G$ is torsion-free, then each $M_i$ would also be aspherical by the Sphere Theorem. If $G$ is not torsion-free, then it suffices to show that $G$ is virtually torsion-free as then we can apply the above argument to an appropriate finite sheeted cover of $M$. Since each $M_i$ is either aspherical or covered by $S^3$, it follows that each $\pi_1(M_i)$ is either torsion-free or finite. Since a finite free product of virtually torsion-free groups is virtually torsion-free by \cref{closure}, it follows that $G$ is virtually torsion-free.
\end{proof}


\begin{theorem}
\label{manifold_free-by-cyclic}
Let $M$ be a connected three-manifold and let $G$ be its fundamental group. If $G$ is finitely generated and $\cd_{\Q}(G) <3$, then $G$ is virtually free-by-cyclic.
\end{theorem}

\begin{proof}
By \cref{closure,facts}, we may assume that $M$ is compact, orientable, aspherical and with incompressible boundary. We also may assume that $M$ is not contractible. Since $G$ is non-trivial, there is some embedded curve $\gamma\colon S^1\injects M$ which is not contained in an embedded three-ball and does not intersect the boundary. Since $\cd_{\Q}(G)<3$ and $M$ is aspherical, it follows that $M$ must have non-empty boundary. There is a tubular neighbourhood $T\subset M$ of $\gamma(S^1)$ whose boundary does not intersect the boundary of $M$. Denote by $N$ the compact three-manifold obtained from $M$ by removing $T$ and by $\nu\colon N\injects M$ the canonical inclusion. We claim that $\pi_1(M)$ is a subgroup of $\pi_1(M\cup_{\partial M}N)$ and that $M\cup_{\partial M}N$ remains irreducible.

We first show that $\pi_1(M)$ is a subgroup $\pi_1(M\cup_{\partial M}N)$. We have a canonical inclusion $M\injects M\cup_{\partial M}N$ and a map $M\cup_{\partial M}N\to M$ defined by the identity on $M$ and $\nu$ on $N$. As the composition of these two maps is the identity, we see that $\pi_1(M)$ is a subgroup of $\pi_1(M\cup_{\partial M}N)$.

We now show that $M\cup_{\partial M}N$ is irreducible. First note that as the image of $\gamma$ is not contained in an embedded ball, it follows that $N$ is also irreducible. Suppose that $M\cup_{\partial M}N$ is not irreducible. Then there is a sphere $S^2$, embedded in the interior of $M\cup_{\partial M}N$, that does not bound a ball. After an isotopy, we may assume that $S^2$ is transverse to $\partial M$. Thus, $S^2\cap \partial M$ is either empty or consists of embedded circles. If the intersection is non-empty, take some innermost circle $S^1\subset S^2\cap \partial M$. Then the two-disc $D$ it bounds in $S^2$ must embed in $M$ or $N$. As $M$ had incompressible boundary, it follows that we may isotope $D$ through $\partial M$ and reduce the number of components in $S^2\cap \partial M$. Continuing in this way, we see that we may isotope $S^2$ so that $S^2\cap \partial M$ is empty. If $S^2\cap \partial M$ is empty, then $S^2$ is contained in $M$ or $N$. As $M$ and $N$ are both irreducible, it follows that $M\cup_{\partial M}N$ is also irreducible.

As subgroups of virtually free-by-cyclic groups are virtually free-by-cyclic, it is enough to show that the fundamental group of  $M\cup_{\partial M}N$ is virtually free-by-cyclic. If $M\cup_{\partial M}N$ is a hyperbolic three-manifold, then it is virtually fibred by \cite[Theorem 17.14]{wise_21_quasiconvex} and \cite[Theorem 1.1]{Agol2008}. If $M\cup_{\partial M}N$ is a graph manifold, it is virtually fibred by \cite{wang_97}. In all other cases, it is virtually fibred by \cite[Corollary 1.3]{przytycki_18}. As $M\cup_{\partial M}N$ has non-empty boundary, it virtually fibres over a surface with boundary. Thus, $G$ is virtually free-by-cyclic.
\end{proof}


We need one final result due to Friedl--L\"uck and then we will be ready to prove our main theorem.


\begin{theorem}[Friedl--L\"uck {\cite[Theorem 3.2(3)]{friedl_19}}]
\label{aspherical_C}
Let $M$ be a closed, connected, orientable and aspherical three-manifold. The fundamental group $G$ of $M$ lies in Linnell's class $\mathcal C$.
\end{theorem}


\begin{proof}[Proof of \cref{main thm}]
	Let $M$ be a connected three-manifold with fundamental group $G$.
	
	Suppose first that $G$ is finitely generated. Since class $\mathcal{C}$ is closed under extensions by finite groups and under free products by \cref{C_free_products}, we may assume that $M$ is compact, orientable, aspherical and with incompressible boundary by \cref{facts}. If $M$ has empty boundary, we are done by \cref{aspherical_C}. Otherwise, we are done by \cref{manifold_free-by-cyclic}, since $M$ is homotopy equivalent to its spine, which is an aspherical two-complex, and hence $\cd_{\Q}(G) <3$.
	
	Finally, suppose that $\pi_1(M)$ is not finitely generated. Let $M_0\to M_1\to \ldots\to M$ be a sequence of covers such that $\pi_1(M_i)$ is finitely generated for all $i$ and such that $\bigcup_{i\geqslant 0}\pi_1(M_i) = \pi_1(M)$. Since we showed above that $\pi_1(M_i)$ is in $\mathcal{C}$ for all $i$, the group $\pi_1(M)$ is also in $\mathcal{C}$.
	
	If $\cd_{\Q}(G) <3$, then the same is true for every subgroup of $G$. Hence $G$ is locally free-by-cyclic by \cref{manifold_free-by-cyclic}.
\end{proof}

\begin{proof}[Proof of \cref{main cor}]
	Linnell \cite{linnell_93}*{Theorem 1.5} proved that all groups in $\mathcal C$ with a uniform bound on cardinalities of torsion subgroups satisfy the Strong Atiyah Conjecture over $\C$. It is very easy to see that torsion-free groups satisfying this conjecture do not have non-trivial zero-divisors, see \cite{luck_02}*{Lemma 10.15}. If $G$ is not torsion-free, but is finitely generated, \cref{facts} implies that $G$ is virtually torsion-free. Being virtually torsion-free gives a bound on the size of torsion subgroups. This finishes the proof.
\end{proof}

\subsection{An example}

We now present an example of a locally virtually free-by-cyclic three-manifold group that is not virtually free-by-cyclic, showing that \cref{main thm} is sharp. A straightforward example would be a three-manifold whose fundamental group has no bound on the order of its torsion elements. Instead, we construct an example that is Seifert fibred with base orbifold of infinite type and whose fundamental group is torsion-free.


We first require a couple of facts about Seifert fibred spaces. The reader is directed towards \cite[Section 10]{martelli_22} for the necessary background. 

Let $M$ be a compact orientable three-manifold and let $M\to S$ be a Seifert fibration. If $M$ fibres over the circle, then $M$ admits a foliation by compact surfaces. Note that if $M$ has non-trivial boundary, then we may glue solid tori along the boundary and extend this foliation. So now we may apply work of Eisenbud--Hirsch--Neumann \cite[Theorems 3.4 \& 6.1]{eisenbud_81} to conclude that if $S$ has genus at least two, then such a foliation must be homotopic to a foliation transverse to the fibres of the Seifert fibration and moreover exists only if $e(M) = 0$, where $e(M)$ denotes the Euler number of the Seifert fibration. Recall that the Euler number of a Seifert fibred space with boundary is only defined modulo the integers.

Now let $M_n$ denote the orientable Seifert fibred space with Euler number $e(M) = 1/n$  whose base orbifold is a genus two surface with two boundary components and a single cone point. The Euler characteristic of $M_n$ is zero, so any finite index subgroup of $\pi_1(M_n)$ that is free-by-cyclic, must be \{finitely generated free\}-by-cyclic by work of Feighn--Handel \cite{feighn_99}. In particular, by Stallings' Fibration Theorem \cite{stallings_62}, the corresponding cover of $M_n$ must fibre over the circle. Since any finite degree cover of $M_n$ induces a finite degree orbifold cover of the base orbifold, we see that the minimal degree of a cover of $M_n$ with Euler number zero is $n$. So by the previous paragraph, the minimal degree of a cover of $M_n$ that fibres over the circle is $n$.

Consider the three-manifold 
\[
M = M_2\cup_T M_3\cup_T\ldots
\]
where $T$ is one of the boundary tori of $M_i$. If $\pi_1(M)$ were virtually free-by-cyclic, then this would imply that every $M_n$ admits a finite cover of uniformly bounded index that fibres over the circle. But this cannot happen by the previous paragraph and so $\pi_1(M)$ is not virtually free-by-cyclic; it is locally virtually free-by-cyclic by \cref{main thm}.




\section{Shalen's theorem}
\label{section Shalen}

In this section we use the Atiyah conjecture, and show how the $L^2$-technology gives an alternative route to the following generalisation of \cref{Shalen}.
The key tool we use is the $L^2$-Freiheitssatz of Peterson--Thom~\cite{peterson_11}, which may be applied to any torsion-free group satisfying the Atiyah conjecture.

\begin{theorem}
\label{shalen_general}
	Let $G$ be a finitely generated torsion-free fundamental group of a three manifold. We have
	\[
	-\chi(G) < \mathrm{miof}(G).
	\]
\end{theorem}
\begin{proof}
	Since $G$ is finitely generated, it
	is the fundamental group of a compact three-manifold by the Compact Core Theorem. Prime decomposition allows us to see that the Euler characteristic $\chi(G)$ is well defined. By \cref{main cor}, the group $G$ satisfies the Strong Atiyah conjecture.
	
	By combining standard properties of $L^2$-Betti numbers (see \cite[Theorem 1.35]{luck_02}) with the work of Lott--L\"uck~\cite{LottLueck1995}*{Theorem 0.1}, we see that $b_1^{(2)}(G) = -\chi(G)$. For every finite generating set of $G$, the $L^2$-Freiheitssatz of Peterson--Thom~\cite{peterson_11}*{Corollary 4.7} allows us to find a subset freely generating a free group of rank $b_1^{(2)}(G) + 1$. So
	\[
	-\chi(G)  = b_1^{(2)}(G) < b_1^{(2)}(G) + 1 \leqslant  \mathrm{miof}(G). \qedhere
	\]
\end{proof}




\bibliographystyle{amsalpha}
\bibliography{bibliography}


\end{document}


