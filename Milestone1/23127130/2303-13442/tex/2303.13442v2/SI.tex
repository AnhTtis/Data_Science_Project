\section{Materials and characterization}
We obtained CB15 (Synthon Chemicals), TTAB and PF127 (Sigma-Aldrich) and used them as-is. To study the influence of mixed micelles on active droplet motility around room temperature, we varied the TTAB concentration between 9-15 wt.\%. 
We note that a low concentration of PF127 requires heating the system to high temperatures to deplete TTAB, causing evaporation issues, while at a higher concentration of PF127 one needs low temperatures to see the droplet motion.  This strongly limited the experimentally accessible parameter space for PF127, and we opted to keep the concentration fixed at 4 wt.\%  with $T_{\rm CMC}$ at $\approx \SI{21}{\celsius}$, see Fig.~\ref{SIfig:DLS}
\section{Microfluidic experiments}

CB15 droplets were mass produced in microfluidic flow focusing devices \cite{hokmabad2021_emergence}. Our Hele-Shaw experimental reservoir consisted of SU-8 photoresist spin coated on glass. We fabricated a cell with rectangular cross-section of area $(13\times8)\:\rm{mm^{2}}$ and height of \SI{50}{\um} via UV photolithography, filled it with a dilute droplet emulsion and sealed it with a glass cover slip. 
\par
We observed the dynamics of active droplets under bright-field microscope (Olympus IX-81) with a temperature control stage (Linkam PE100) which allowed for both heating and cooling protocols. The reservoir temperature was set at the desired initial temperature for 5 min. after which measurements were carried out. 

We performed temperature ramp experiments with a set heating/cooling rate of 1 K/min or 10 K/min. As shown in Fig. 2 (c), the system did not equilibrate instantaneously. We therefore directly measured the sample temperature using a thermistor taped to the top plate of the microfluidic reservoir. (Fig. 1) Video frames were recorded at 4 frames per second using a Canon digital camera (EOS 600d) with a digital resolution of 1920 x 1080 px. 



\section{Viscosity measurements}

\begin{figure*}
    \centering
    \includegraphics[width=.49\linewidth]{SI_figs/PF127_4pc_TTAB10pc_flow_curve.pdf}
    \includegraphics[width=.49\linewidth]{SI_figs/viscosity_vs_temp_flow_curve.pdf}
    \caption{(a) Shear stress versus shear rates at different temperatures for 4 wt.\% PF127 + 10 wt.\% TTAB. (b) Viscosity versus temperature for TTAB and PF127.}
    \label{SIfig:viscosity}
\end{figure*}
We measured the viscosity of our swimming media on an Anton-Paar MCR 502 rheometer using a cone-plate geometry with a gap width of \SI{0.1}{\mm}. Measurements were carried out at temperatures between \SI{15}{\celsius} and \SI{40}{\celsius} and shear rates between \SI{0.1}{\s^{-1}}  and \SI{100}{\s^{-1}} (see \figref{SIfig:viscosity}). We observe Newtonian rheology and a weakly temperature dependent viscosity. 

\section{Micelle size measurements}
\begin{figure*}
    \centering
    \includegraphics[width=.98\linewidth]{SI_figs/DLS.pdf}
    \caption{(a) Dynamic light scattering (DLS) measurements of PF127 and TTAB as a function of temperature. (b) Zoomed-in version of panel (a).}
    \label{SIfig:DLS}
\end{figure*}

We measured the average micellar hydrodynamic diameter for the co-surfactant mixtures used in our experiments by dynamic light scattering on  a Malvern Zetasizer Nano S. We placed a \SI{1}{\ml} sample in a polystyrene cuvette and recorded measurements between \SI{15}{\celsius} and  \SI{30}{\celsius}. At each temperature, the sample was allowed to equilibrate for \SI{120}{\s}, and all measurements were carried out in triplicate. We show the measured temperature dependent micelle sizes in Fig.~\ref{SIfig:DLS}.


\section{Differential scanning calorimetry}

\begin{figure}
    \centering
    \includegraphics[width=1\linewidth]{SI_figs/DSC1.pdf}
    \caption{DSC heating curves of PF127 and TTAB solutions. Measurements were performed with a heat rate of \SI{5}{\kelvin/\min}  and normalized to sample mass.}
    \label{SIfig:DSC}
\end{figure}
Micelle formation was investigated using differential scanning calorimetry (DSC). For the measurements, three sample types were prepared: (1) PF127 in 4 wt\% water, (2) PF127 4 wt\% + TTAB 1 wt\% in water and (3) PF127 4 wt\% + TTAB 10 wt\% in water. 
Solutions were directly transferred into DSC sample aluminium pans (volume \SI{100}{\ul}, Mettler-Toledo GmbH, Gießen, Germany). DSC pans were covered with aluminum lids. DSC measurements were performed on a DSC 823 instrument (Mettler-Toledo GmbH, Gießen, Germany). Heating-cooling cycles were recorded 
at a heating/cooling rate of \SI{5}{\kelvin/\min} between 0 and \SI{80}{\celsius}. The measurements were performed under a nitrogen atmosphere with a flow of \SI{30}{\ml/\min}. Heating curves were normalised to the sample mass. 

\section{Flow and chemical field measurements}
Measurements of external flow field were challenging since PF127 adsorbed on colloidal tracer particles \cite{kim2005_swelling-based} leading to tracer particle aggregation. Instead, $\rm{1\:\mu m}$ diameter tracer Silica particles (Cospheric SiO2MS-1.8) were added to the oil phase and the internal flow field was measured using particle image velocimetry (PIV). 


We captured videomicroscopy under a 40x objective using a 4MP camera (FLIR Grasshopper 3, GS3-U3-41C6M-C) at 40 fps at different set temperatures. 
\par
To study the sudden onset of motion from an inactive state as the system was cooled, images were recorded at a higher frame rate of 80 fps. Droplet speed over time was calculated from the recorded trajectories.
\par
To visualise the oil-filled micelle chemical trail behind the droplet, the oil phase was doped with Nile Red (Sigma-Aldrich) dye. We performed fluorescent microscopy on an Olympus IX81 microscope with a filter cube (excitation filter ET560/40x, beam splitter 585 LP and emissions filter ET630/75m, all Chroma Technology). Images were captured in 4x objective  using a 4 MP CMOS camera (FLIR Grasshopper 3, GS3-U3-41C6M-C) at 4 fps.

\section{Digital image processing and data analysis}
We extracted droplet coordinates from bright field microscopy using a sequence of background correction, binarization, blob
detection by contour analysis, and minimum enclosing
circle fits, and determined trajectories via a nearest-neighbour algorithm using in-house Python scripts building on numpy and opencv. For the strongly overexposed fluorescence data the droplet centroid was calculated via a distance transform algorithm on the fluorescence intensity. The polar intensity map in Fig. 3(b) was derived by taking the intensity in an annular region around the droplet at a distance of 1.2 droplet radii from the centroid~\cite{hokmabad2021_emergence,hokmabad2022_chemotactic,ramesh2022_interfacial}, as sketched in supporting figure~\ref{fig:SIkymo}.

\begin{figure}
    \includegraphics[width=\columnwidth]{SI_figs/SIkymo.png}
    \caption{Protocol to extract fluorescence kymographs from microvideo data, by the example of snapshot III from Fig. 3}
    \label{fig:SIkymo}
\end{figure}

Using the time-dependent droplet trajectory and temperature data, we estimated transition temperatures between dynamic states. The error bars for temperature in the regime map (\figref{fig:traj}f) represent the maximum variation within three different runs.

We performed PIV analyses  using the Matlab PIVlab module \cite{thielicke2014_pivlab} with a multi-pass interrogation window of 64 x 64 pixel and 32 x 32 pixel with 50\% overlap. The spatial resolution of the PIV output was \SI{4.3}{\um/px}.

\section{Numerics}
We simulate a diffusiophoretic particle of unit radius initially located at the center of a domain of size $L_x=10$, $L_y=100$, $L_z=2.2$ with $201 \times 2001 \times 45$ grids at subcritical \Pen. The particle is propelled by diffusiophoresis, a type of microswimmer similar to active droplet, as both move in response to the forces at the surfaces, which depend on the local chemical concentration field (see also~\cite{morozov2019_nonlinear}). The chemical reaction takes place at the particle surface and whenever there is a chemical concentration difference along the surface, a flow is generated within the interaction layer near the solid surface, with thickness $\lambda$ of nanometers, which propels the particle forward. 

We use the same non-dimensional governing equations and boundary conditions at the particle interface as those in previous studies~\cite{michelin2013_spontaneous,chen2021_instabilities} (here, solving for the full 3D problem as opposed to the axisymmetric analytical approach).  
The governing equations are given as
\begin{equation}\label{eq_num1}
\frac{\partial c}{\partial t}+\boldsymbol{u}\cdot\nabla c= \frac{1}{Pe}\nabla^2 c,
\end{equation}

\begin{equation}
\gdef\thesubequation{\theequation \mbox{\textit{a}},\textit{b}}
\frac{\partial \boldsymbol{u}}{\partial t}+(\boldsymbol{u}\cdot\nabla)\boldsymbol{u}=-\nabla p+\frac{Sc}{Pe}\nabla^2 \boldsymbol{u},\quad
\boldsymbol{\nabla} \cdot \boldsymbol{u}=0,
\end{equation}
where $c$ is the concentration, $u$ the velocity. $Pe$ is the P\'eclet number, which is the ratio of the advection to the diffusion and $Sc$ is the Schmidt number, which is the ratio between the momentum and diffusivities:
\begin{equation} \label{eq_num3}
\gdef\thesubequation{\theequation \mbox{\textit{a}},\textit{b}}
Pe=\frac{M\alpha L}{D^2}, \quad
Sc=\frac{\nu}{D}.
\end{equation}
where $M$ is the mobility, $M\sim \pm k_BT\lambda^2/(\rho\nu)$ with $k_B$ the Boltzmann constant and $T$ the temperature, $\rho$ the density, $\nu$ the viscosity, and $D$ is the diffusion coefficient. 

The boundary conditions are given as:
\begin{equation} \label{eq_num2}
\gdef\thesubequation{\theequation \mbox{\textit{a}},\textit{b}}
\partial_n c=-1 \quad
u_s=\nabla_s c,
\end{equation}
where  $\partial_n c$ represents the concentration gradient at the direction normal to particle surface, $u_s$ the slip velocity and $\nabla_s$ is the tangential gradient. The top and bottom boundaries (at $z$ direction) of the domain are set as solid walls, and all other domain boundaries (at $x$ and $y$ directions) are set as periodic.

We used a central second-order finite difference scheme to spatially discretize the governing equations, with homogenous staggered grids used in both the horizontal and vertical directions. The equations are integrated by a fractional-step method, with non-linear terms computed explicitly using a low-storage third-order Runge-Kutta scheme and the viscous terms computed implicitly by a Crank-Nicolson scheme~\cite{verzicco1996_finitedifference}.
For the particle boundary, we make use of the moving least squares (MLS) based immersed boundary (IB) method, where the particle interface is represented by a triangulated Lagrangian mesh~\cite{spandan2017_parallel}.
For the detailed numerical methods and validation, we refer to~\cite{chen2021_instabilities}.

Through numerical simulation, we found the particle starts to break the symmetry when $Pe\gtrapprox 6$. The velocity curves near the symmetry-breaking point are similar for different $Pe$, and the inset of figure \ref{fig:piv} (a) shows the velocity profile for $Pe=8$, which is similar to the velocity transition from passive dissolution to active propulsion.

\section{Additional data}
\begin{figure}
    \centering
    \includegraphics[width=1\linewidth]{SI_figs/heating10TTAB.png}
    \caption{Temperature dependent droplet dynamics for 10 wt. \% TTAB and no PF127. (a) Droplet trajectory and (b) droplet
 speed $v$, and measures of fluctuations in speed, $\langle\Delta v\rangle/\langle v\rangle$, and orientation, $\cos(\Delta\phi)$, over time. With increasing temperature, the droplet motion accelerates and destabilises. Scale bar \SI{250}{\um}.}
    \label{SIfig:ttab}
\end{figure}
In Fig.~\ref{SIfig:ttab}, we show a temperature-coded trajectory of a droplet studied under the same conditions as the experiments shown in the main manuscript, but in an aqueous solution of TTAB only at 10 wt\%. Here, the motion destabilises with increasing temperature. We illustrate this by temperature-coded plots of three quantities: the speed (top), which increases, but also fluctuates strongly for high temperatures. Further, two simple correlation estimates, both taken over a running time window of $\tau=\SI{2}{s}$: the standard deviation over average speed, $\langle \Delta v\rangle/\langle v \rangle$ as a measure of unsteadiness in speed, and the cosine of the angle between $\vec{v}(t)$ and $\vec{v}(t+\tau)$ via their inner product as a measure of rotational fluctuation. Both indicate strong decorrelation at elevated temperatures.

\section{Supplementary video captions}
\noindent\begin{minipage}{\columnwidth}
\noindent\includegraphics[width=\columnwidth]{SI_movies/MovieS1.png}
\textbf{Video S1.} Temperature dependent droplet dynamics at 4 wt.\% PF127 + 10 wt.\% TTAB, showing a reversible transition from meandering to straight swimming to arrest during a heating and subsequent cooling ramp with a set rate of 1 K/min. 
\end{minipage}

\noindent\begin{minipage}{\columnwidth}
\noindent\includegraphics[width=\columnwidth]{SI_movies/MovieS2.png}
\textbf{Video S2.}  Droplet trajectory at 4 wt.\% PF127 + 10 wt.\% TTAB, during multiple heating and subsequent cooling ramps at a set rate of 10 K/min.
\end{minipage}

\noindent\begin{minipage}{\columnwidth}
\noindent\includegraphics[width=\columnwidth]{SI_movies/MovieS3.png}
\textbf{Video S3.}  Temperature dependent droplet dynamics at 4 wt.\% PF127 + 13 wt.\% TTAB, showing a reversible transition from chaotic to straight swimming
to arrest during a heating and subsequent cooling ramp with a set rate of 5 K/min.\end{minipage}

\noindent\begin{minipage}{\columnwidth}
\noindent\includegraphics[width=\columnwidth]{SI_figs/speedtraced.png}
\textbf{Video S4.}  Fluorescent microscopy visualising droplet chemical trails upon heating and subsequent cooling ramp with a set rate of 5 K/min.
\end{minipage}

\noindent\begin{minipage}{\columnwidth}
\noindent\includegraphics[width=\columnwidth]{SI_movies/MovieS5.png}
\textbf{Video S5.} Internal flow visualisation with increasing temperature, starting at a mixed dipolar/quadrupolar mode (meandering), over a purely dipolar mode (straight) that recedes to the anterior (slowdown) and onset of motion during cooling. Clips during heating are rendered in real time (at 40 fps), during cooling sped up by 5x (8 fps to 40 fps).
\end{minipage}
\vspace{3mm}\noindent\begin{minipage}{\columnwidth}
\noindent\includegraphics[width=\columnwidth]{SI_movies/MovieS6.png}
\textbf{Video S6.} Onset of active motion during recooling.
\end{minipage}