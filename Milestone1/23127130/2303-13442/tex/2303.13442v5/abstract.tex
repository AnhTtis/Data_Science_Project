Self-propelling active matter relies on the conversion of energy from the undirected, nanoscopic scale to directed, macroscopic motion. One of the challenges in the design of synthetic active matter lies in the control of dynamic states, or motility gaits. Here, we present an experimental system of self-propelling droplets with thermally controllable and reversible dynamic states, from unsteady over meandering to persistent to arrested motion. These states are known to depend on the Péclet number of the molecular process powering the motion, which we can now tune by using a temperature sensitive mixture of surfactants as propulsion fuel. We quantify the droplet dynamics  by analysing flow and chemical fields for the individual states, comparing them to canonical models for autophoretic particles.  In the context of these models, we experimentally observe, in situ, the fundamental first broken symmetry that translates an isotropic, immotile base state to self-propelled motility.
