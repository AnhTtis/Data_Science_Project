% !TeX root=./main-revtex.tex
\section{Introduction}
Active matter is defined by the nonlinear conversion of free energy on the molecular scale into macroscopic dynamics~\cite{bowick2022_symmetry}: these nonlinear dynamics pose a challenge to the control of active agents, as small changes in system parameters can crucially tip the system over into a new dynamic equilibrium, for example from metastable inactivity to a state of directed motility.
\par
More so, like a sorcerer's apprentice~\cite{goethe1798_zauberlehrling} we find that stopping this motion is not a trivial task; after all, one is required to deplete an energy reservoir or reverse a dynamic instability~\cite{obyrne2022_time}. 
\par
This kind of control is important in the design and study of artificial and biological microswimmers, their theoretical modeling, experimental realization, and, ultimately, to provide design principles and dynamic control for technological application \cite{tu2017_motion,fusi2023_achieving}.
\par
Autophoretic particles \cite{bechinger2016_active,moran2019_microswimmers,zhang2021_chemicallypowered,zottl2023_modeling} and, particularly, their experimental counterpart, active droplets \cite{maass2016_swimming,babu2022_motile,birrer2022_we,dwivedi2022_selfpropelled,michelin2023_selfpropulsion,hanczyc2007_fatty,thutupalli2011_swarming,peddireddy2012_solubilization,izri2014_selfpropulsion,herminghaus2014_interfacial}, 
are popular active matter models driven by purely physicochemical mechanisms. 
Generally, their dynamics are characterized by a dimensionless P\'eclet number \Pen{}, quantifying the ratio of advective and diffusive transport of chemical fuel~\cite{michelin2013_spontaneous}. With increasing \Pen{}, autophoretic particles first transition  from \textit{passive} isotropic chemical conversion to \textit{active} self-propulsion, and further from persistent to unsteady motion via a sequence of broken symmetries and interfacial flow modes of increasing complexity~\cite{michelin2013_spontaneous,izri2014_selfpropulsion,suga2018_selfpropelled,izzet2020_tunable,meredith2020_predatorprey,hokmabad2021_emergence,suda2021_straighttocurvilinear}.
\par
Recent studies have investigated the control of speed and dynamic states of such microswimmers in response to externally applied stimuli such as temperature \cite{tu2017_selfpropelled,cholakova2021_rechargeable} or illumination \cite{florea2014_photochemopropulsion,kaneko2017_phototactic,xiao2018_moving,lancia2019_reorientation,alvarez2021_reconfigurable,ryabchun2022_runandhalt,benzion2022_cooperation,vankesteren2023_selfpropelling}.\par 
On heating, physical intuition might suggest that motion should accelerate and destabilize, either by the increase of translational and rotational diffusion with decreasing viscosity, or by increased activity from the molecular thermodynamics driving the motion. %FIG S5
However, the nonlinear dynamics of self-organized activity can drive counter-intuitive effects, as we have previously found for active droplets which destabilize with increasing viscosity due to an increase in \Pen{}~\cite{hokmabad2021_emergence}. 
\par
In this study, we explore another counter-intuitive response, this time to increasing temperature. We study active droplets using a temperature sensitive combination of co-surfactants in aqueous solution as a fuel medium. With  increasing ambient temperature we find a transition between distinct dynamic states from unsteady to oscillatory to steady, straight swimming to eventual arrest, which is reversible and cyclic with temperature. Notably, by this method we are able to drive \Pen{} across the critical activity threshold: we experimentally observe, live and in situ, the fundamental first symmetry breaking of the inactive isotropic base state into directed self-propulsion, analysing both chemical and hydrodynamic fields.
 This observation requires an undisturbed, unchanged ambient fuel medium, which we engineer by exploiting temperature dependent surfactant-polymer interactions~\cite{tam2006_insights} that were previously not considered in active droplets - opening new design possibilities to control the motion of micro-droplets as active agents in smart materials driven by purely physicochemical mechanisms~\cite{zhang2021_autonomous}.

%
\section{Experimental methods}
\par
\subsection{Self-propelling droplets in co-surfactant solutions}
\begin{figure*}
	\includegraphics[width=\linewidth]{figures/fig1.png}
	\caption{\textbf{Schematics of the experimental setup and droplet propulsion mechanism.} (a) With increasing/decreasing temperature, there is an increased/decreased affinity for TTAB to form mixed micelles with PF127. Simplified aggregation schematic \cite{nambam2012_effects}. (b) Droplet propulsion during solubilisation: an inhomogeneous distribution of empty TTAB micelles causes a self-sustaining Marangoni gradient at the oil-water interface. (c) Setup: The droplet (diameter $d=\SI{50\pm 5}{\um}$) swims in a quasi-2D ($\SI{13}{\mm}\times\SI{8}{\mm}\times\SI{50}{\um}$) cell on a temperature controlled microscope stage. All schematics are not to scale.} 
	\label{fig:setup}
\end{figure*}
\par
Our experimental system consists of oil droplets (CB15) immersed in an aqueous solution of the ionic surfactant tetradecyltrimethylammonium bromide (TTAB) at 9-1\SI{5}{\wtpc} (267--\SI{445}{\milli\molar}) and the triblock copolymer Pluronic F127  (PF127) at \SI{4}{\wtpc}, or \SI{3}{\milli\molar}. 
\par
CB15 droplets self-propel reliably in supramicellar solutions of pure TTAB above \SI{5}{\wtpc}~\cite{jin2017_chemotaxis,hokmabad2021_emergence}. Briefly put, the swimming is due to oil diffusing from the droplet into TTAB micelles~\cite{peddireddy2012_solubilization,herminghaus2014_interfacial,izzet2020_tunable}, which removes surfactant from the droplet posterior, while the anterior is replenished by the advection of fresh surfactant (Fig.~\ref{fig:setup}b). The key point here is that the empty micelles at the anterior are less thermodynamically stable than the oil-filled ones at the posterior~\cite{izzet2020_tunable,rosen2012_surfactants}: in consequence the critical micelle concentration is higher in front of a moving droplet. The resulting self-enhancing surface tension gradient drives the droplet forward until it is dissolved. Typically, a CB15 droplet of diameter \SI{50}{\um} will swim in \SI{5}{\wtpc} TTAB for 1-2 hours.   Based on this mechanism, one can define a \Pen\ of droplet motion that increases with viscosity, droplet radius  and surfactant concentration, i.e. chemical activity~\cite{izri2014_selfpropulsion,izzet2020_tunable,hokmabad2021_emergence,hokmabad2022_spontaneously,suda2021_straighttocurvilinear}.  In a surfacant solution where all micelles are oil-saturated, droplets would not propel, and they are are repelled by gradients in filled and attracted by gradients in empty micelles~\cite{jin2017_chemotaxis}. We may in this sense regard empty TTAB micelles as fresh, and oil-filled ones as spent fuel.
\par
PF127 is a nonionic triblock copolymer surfactant that in a pure aqueous solution forms micelles with a hydrophobic core of propylene oxide (PPO) and an outer shell of hydrated ethylene oxide (PEO) \cite{alexandridis1994_micellization,wanka1994_phase,bohorquez1999_study,stoeber2006_passive,jalaal2016_rheology,jalaal2018_gelcontrolled} above the critical micelle temperature,  $\text{CMT}\approx\SI{21}{\celsius}$ at $\SI{4}{\wtpc}$ PF127~\cite{bohorquez1999_study}. 
In the presence of ionic co-surfactants like TTAB, which bind strongly with PF127, mixed TTAB/PF127 aggregates form (\figref{fig:setup}a, Appendix \ref{SIsec:mixedmicelles}, Figs.\ \ref{SIfig:DLS}, \ref{SIfig:Zaverage} and \ref{SIfig:DSC}), with excess TTAB forming single-species micelles \cite{nambam2012_effects,hecht1994_interaction,li2001_binding,tam2006_insights}. 
It has been found \cite{li2001_binding} that the TTAB binding capacity of PF127 is amplified with increasing temperature, due to an increasing dehydration of the PPO blocks (hydrophobic effect~\cite{butt2003_physics,alexandridis1994_micellization}).


Under our experimental conditions, we expect a significant coverage of TTAB at the interface (Appendix \figref{SIfig:anchoring}).  Furthermore, CB15 droplets are completely immotile and hardly solubilize in pure PF127 solutions (Appendix \figref{SIfig:dissolution}), while in a mixed TTAB/PF127 medium self-propelling at speeds $\approx\SI{20}{\um/\second}$, comparable to experiments in pure TTAB solutions (cf. \cite{jin2017_chemotaxis,hokmabad2021_emergence} and Appendix \figref{SIfig:ttab}). We therefore regard TTAB as the primary surfactant mediating the solubilization and the interfacial gradients driving the droplet motion. This activity is controlled by PF127 binding and relasing TTAB micelles in the bulk medium (Fig.~\ref{fig:setup}a,b).
Thus, the P\'eclet number \Pen{} of droplet activity \textit{decreases} with \textit{increasing} temperature, as an increasing fraction of TTAB is bound in mixed micelles.

According to literature on the composition of PF127/TTAB aggregates \cite{hecht1994_interaction,li2001_binding,hecht1995_interaction}, the amount of bound TTAB in our swimming medium should exceed $\SI{1}{\wtpc}$ and further increase with temperature, which is on the order of the amounts required ($<\SI{5}{\wtpc}$) to suppress droplet motility~\cite{herminghaus2014_interfacial,hokmabad2021_emergence}, as observed in the experiments we show below (see also Appendix \figref{SIfig:swimming}). 
We further note that the swimming medium is Newtonian, with only weakly temperature dependent viscosity (Appendix \figref{SIfig:viscosity}).
%
\section{Results}
\subsection{Swimming dynamics controlled by temperature and fuel concentration}
\begin{figure*}
	\centering
	\includegraphics[width=1\linewidth]{figures/fig2.png}
	\caption{\textbf{Swimming dynamics controlled reversibly via temperature and fuel concentration} 
		(a) Droplet trajectory  transitioning from meandering to straight swimming to arrest during a heating and subsequent cooling ramp with a set rate of \SI{1}{\kelvin/\min} in a mixed surfactant swimming medium, color coded by droplet speed and recorded temperature. See also Movie S1. Color map and scale bars (\SI{250}{\um}) apply to all figures in the paper, and concentrations for PF127 and TTAB are always \SI{4}{\wtpc} and \SI{10}{\wtpc}, respectively, unless stated otherwise.
		(b) Map of the swimming dynamics depending on temperature and surfactant concentration. (c) Hysteresis between droplet stop and start transition temperatures. Error bars also apply to $T_\text{stop}$ in (b); experiments done in triplicate with 5--10 droplets each. (d) Example of unsteady motion at 1\SI{5}{\wtpc} TTAB and \SI{21}{\celsius} (see also Movie S2).\\
		(e) Droplet trajectory during multiple heating/cooling cycles set at \SI{10}{\kelvin/\min} and \SI{10}{\wtpc} TTAB. See also Movie S3.  
		(f) Corresponding droplet speed vs. time, with inset plot of the recorded temperature ramps. (g) Droplet speed vs. recorded temperature showing a hysteresis in the re-onset of motion during cooling: the arrest during cooling is extended, with a sudden recovery of the initial speed at $T \approx \SI{17}{\celsius}$.}
	\label{fig:traj}
\end{figure*}
%
We begin with an overview of the general swimming dynamics taken from wide-field video microscopy under changing the ambient temperature, and for a range of TTAB concentrations.
The setup contains a quasi-2D microfluidic cell on a temperature controlled stage (\figref{fig:setup}c). 
\figref{fig:traj}a (Movie S1) plots a trajectory colour-coded once by speed and once by temperature, recorded at a set heating/cooling rate of \SI{1}{\kelvin/\min}, using a swimming medium containing \SI{10}{\wtpc} TTAB.
\par
We start at $T \approx \SI{15}{\celsius}< \text{CMT}$. 
Below $T_\text{straight} = \SI{22}{\celsius}$, the droplet meanders, i.e. periodically reorients. Above, the motion is straight, gradually slows down and eventually stops at $T_\text{stop} \approx \SI{27}{\celsius}$. 
During a subsequent cooling ramp, the droplet remains immotile down to a significantly lower temperature $T_\text{start} \approx \SI{17}{\celsius}$, where it abruptly starts to meander again. 
\par
As well as by temperature, the swimming dynamics are also susceptible to the TTAB concentration. Previous studies on reference systems using a single surfactant species found a transition from straight to reorienting to unsteady swimming~\cite{izzet2020_tunable}. Here, in response to an increase in surfactant concentration, \Pen{} increases, and destabilizing higher order modes in the interfacial flow and chemical fields emerge. Similar mode changes have been found if \Pen{} is increased via the droplet size or the viscosity of the swimming medium~\cite{suda2021_straighttocurvilinear,hokmabad2021_emergence,hokmabad2022_spontaneously}. These effects are consistent with analytical models for isotropic autophoretic particles~\cite{michelin2013_spontaneous,michelin2023_selfpropulsion,morozov2019_nonlinear,morozov2020_adsorption,suda2021_straighttocurvilinear}, in the weakly nonlinear limit, and numerical studies~\cite{li2022_swimming,hu2022_spontaneous}, in the chaotic limit. 
\par
At low temperatures, the presence of a co-surfactant does not change these dynamics, as we find a transition from meandering to unsteady motion (Fig.~\ref{fig:traj}d, Movie S2) with increasing TTAB concentration.  We have summarized these swimming dynamics  in a map spanned by temperature and surfactant concentration in \figref{fig:traj}b.
With increasing temperature, we observe a universal transition via straight swimming to eventual arrest. We posit that the temperature dependent TTAB depletion lowers \Pen{} below the critical thresholds of higher order interfacial modes, down to $n=1$ for straight swimming and finally $n=0$, below the fundamental advection-diffusion instability. We do not provide a quantitative estimate of \Pen{} following~\cite{hokmabad2021_emergence}, as we cannot quantify the temperature dependence of the underlying physical chemistry parameters.
Fig.~\ref{fig:traj}b also shows a general trend of the transition temperatures to increase with TTAB concentration:  
for the droplet to arrest, more TTAB needs to be removed from the swimming medium. We found the stop/start hysteresis for all TTAB concentrations in use (Fig.~\ref{fig:traj}c).
\par
These state transitions are well reversible with temperature, up to some slowdown attributable to droplet shrinkage. The experiment shown in \figref{fig:traj}e-g and Movie S3 was recorded at a faster rate of \SI{10}{\kelvin/\min} to permit multiple heating and cooling cycles, with dynamics similar to the system cooled at slower rates. \figref{fig:traj}e and f show the droplet trajectory color coded by temperature and a corresponding plot of speed over time. The initial motion is recovered after each heating and cooling cycle, apart from a very gradual decrease in maximum speed which we may attribute to droplet shrinkage. We have further analyzed speed versus temperature in \figref{fig:traj}g, and found a  hysteresis cycle with a delayed re-onset of motion reproducible over multiple heating/cooling ramps. The sample in Movie S3 contained three larger droplets ($d\approx\SI{60}{\um}$): \Pen\ should increase with the droplet radius, and, correspondingly, these droplets switch from reorienting to straight motion at later times and therefore higher temperatures.
%
\subsection{Chemical and flow fields}
\begin{figure*}
	\centering
	\includegraphics[width=1\linewidth]{figures/fig3.png}
	\caption{\textbf{The hysteresis in the re-onset of droplet motion is caused by spent fuel aggregation.} (a) Droplet trajectory during heating and subsequent cooling ramp color coded by speed. (b) Kymograph showing the evolution of chemical concentration field around the droplet interface and the recorded temperature (colored symbols). (c) Snapshots of droplet chemical trails at different temperatures as marked by I-IV in (a) and (b).
		See also Movie S4. \SI{250}{\um} scale bars and colour maps as defined in Fig.~\ref{fig:traj}.}
	\label{fig:fluo}
\end{figure*}
We continue with a discussion of the chemical dynamics during the droplet arrest, to motivate the hysteresis in the re-onset of motion; and of the corresponding flow field to investigate the interfacial mode evolution.
\par
The self-generated field of spent fuel in the local enviroment affects the droplet motility, both via chemorepulsive gradients ~\cite{izzet2020_tunable,hokmabad2021_emergence} and via accumulation of filled micelles, which suppresses the interfacial activity~\cite{ramesh2023_interfacial}. We visualise this field by doping the droplet with the fluorescent dye Nile Red~\cite{hokmabad2022_chemotactic}, which co-moves with the oil phase into the filled micelles, and extract the fluorescence intensity $I$ from videomicroscopy data (Movie S4). In Fig.~\ref{fig:fluo}, we analyze these chemical dynamics for one droplet during a heating and cooling cycle, showing a speed-coded trajectory (a), a kymograph of $I$ around the droplet perimeter, $\theta$ vs. time and recorded temperature (b), and micrographs at the times marked I--IV (c). 
\par
During heating, the droplet transitions from unsteady to straight swimming to immotility (Fig.~\ref{fig:fluo}a). We note that this particular experiment featured some global drift causing translation even in the immotile state (see Movie S4).  In the kymograph  (Fig.~\ref{fig:fluo}b), at $T < \SI{26}{\celsius}$, the band corresponding to the chemical trail translates in the angular space due to the reorientation of the droplet (I). At $T \approx \SI{26}{\celsius}$, the droplet slows down and comes to a halt. As the system is cooled down to $T \approx \SI{21}{\celsius}$, the inactive droplet still solubilises isotropically, and oil-filled micelles accumulate around the perimeter $\theta$. Correspondingly, the band in the kymograph widens over the entire angular space (II). The accumulated filled micelles block empty micelles from reaching the interface \cite{ramesh2023_interfacial,morozov2020_adsorption}, such that in the presence of oil-filled micelles even more mixed micelles need to disintegrate to restart activity. Thus, the motility transition temperature is lowered, here to $T_\text{start} = \SI{16.8}{\celsius}$, where the droplet escapes the oil-filled micelle cloud (III) and swims away (IV). 
\par
\begin{figure*}
	\centering
	\includegraphics[width=1\linewidth]{figures/fig4.png}
	\caption{\textbf{Observation of the transition between passive dissolution and active propulsion.} 
		(a) Onset of motion during re-cooling, speed (normalized to steady state $V_\infty$) vs.\ time for multiple runs with $t_\text{onset}$ corresponding to $(\rm{d}V/\rm{d}t)_\text{max}$.  The initial overshoot increases with confinement $2R/H$. Inset: simulation for an isotropic autophoretic particle under comparable conditions ($H=2.2R$ confinement, $\Pen=8$), redimensionalised (Appendix~\ref{SIsec:numerics} and Movie S6).
		(b) Internal flow field with increasing temperature, starting at a mixed dipolar/quadrupolar mode (meandering), over a purely dipolar mode (straight) that recedes to the anterior (slowdown). 
		Vectors and color map inside the droplet indicate the velocity field $\vec{u}(x,y)$;  arrows around the perimeter mark the active regions on the droplet interface. Scale bar \SI{10}{\um}.}
	\label{fig:piv}
\end{figure*}
Before discussing the flow fields, we note two more consequences of oil saturation. First, the hysteresis in \figref{fig:traj}c is reduced by several degrees if the system is not heated to full droplet arrest, but it is never entirely suppressed (Appendix Fig.\figref{SIfig:incomplete}).
This can be understood as follows:  during the late stage of the heating ramp, the droplet is already dispersing oil into its local environment by recirculation, starting from the posterior  - an effect we have also found in self-throttling pumping droplets in \cite{ramesh2023_interfacial}. During heating, the droplets will come to a stop even before the interfacial activity has fully ceded (see the discussion of \figref{fig:piv}b below), and self-propulsion would always need to restart from inside an oil-rich region as shown in \figref{fig:fluo}c-II.

Second, the regime of straight swimming appears to be highly localized on cooling (Movies S1 and S6): the droplet switches after a few seconds to a meandering motion (Movie S1). We argue here that outside the strongly localized cloud of spent fuel (see Movie S4 and Fig.~\ref{fig:fluo}c), more empty TTAB micelles have been released, such that the droplet experiences a higher \Pen{} once it escapes its self-generated local trap. We note that during this escape there is a radial gradient from filled to empty micelles, which would also locally rectify the droplet motion.

To analyze the mode evolution causing arrest and sudden onset of motion during heating and cooling (Movie S5), we added tracer colloids to the oil phase, performed high resolution bright field videomicroscopy and analyzed the internal flow field $\vec{u}(x,y)$ by particle image velocimetry (PIV) at a series of equilibrated set temperatures. \figref{fig:piv}b shows the evolution of $\vec{u}$ with increasing temperature. At $T = \SI{16}{\celsius}$, we see a  mixed  dipolar and quadrupolar flow field (modes $n=1,2$) corresponding to the meandering trajectory in Fig.~\ref{fig:traj}a~\cite{suda2021_straighttocurvilinear,hokmabad2021_emergence}. At even higher temperature, $T = \SI{21}{\celsius}$, the droplet swims straight, \Pen{} decreases and the flow field is purely dipolar ($n=1$).  As the droplet begins to slow down, an inactive region spreads from the droplet posterior ($T = \SI{24}{\celsius}$). Finally ($T = \SI{27}{\celsius}$), just before the droplet stops ($n=0$), only a small region at the droplet anterior is active \cite{ramesh2023_interfacial}. As shown in Fig.~\ref{fig:fluo}II, the local environment isotropically saturates with spent fuel while the droplet is immotile.
\par
\subsection{Simulations}
The gradual increase of \Pen{} during cooling now allows us to directly observe the fundamental first transition from the immotile base state to self-propelled motion~\cite{michelin2013_spontaneous,michelin2023_selfpropulsion}. This can be motivated theoretically using hydrodynamic advection-diffusion models, canonically  derived in~\cite{michelin2013_spontaneous} as follows:

A spherical particle of radius $R$ is immersed in a fluid medium containing a chemical fuel at concentration $c$. At negligible Reynolds numbers, the flow is governed by the Stokes equations, $\mu\nabla^2\vec{u}=\nabla p$, $\nabla\cdot\vec{u}=0.$  The chemical field is coupled by an advection-diffusion equation, 

\begin{align}|\Pen|\left(\frac{\partial c}{\partial t} + \vec{u}\cdot\nabla c\right) &= \nabla^2c, &
	\Pen{}&\equiv \frac{\mathcal{AM}R}{D^2},			\end{align} and by the particle consuming fuel at its boundary, \mbox{$\partial_t c(R) =-\mathcal{A}$}.  The P\'eclet number is set by the activity $\mathcal{A}$, mobility $\mathcal{M}$ and diffusivity $D$ of the chemical species and the particle radius $R$. Using a decomposition into squirmer modes and a linear stability analysis around the isotropic base state (mode $n=0$), the authors of \cite{michelin2013_spontaneous} find a transition to the propulsive dipolar state ($n=1$) above a threshold value of $\Pen{}=4$.

The mixed surfactant approach allows us to observe the growth of the dipolar mode in situ and analyzing it in the context of the canonical model. We performed a simulation of the interfacial instability, following~\cite{michelin2013_spontaneous,picella2022_confined}, here solving for the full 3D problem and adapted to our cell geometry (see Appendix~\ref{SIsec:numerics}), and compared it to the experimental droplet speed $V$ at the onset of motion, with good agreement between re-dimensionalised numerical and experimental results (Fig.~\ref{fig:piv}a, Movie S6). For one, the timescales for the onset of motion, of $\mathcal{O}(\SI{1}{\s})$, and the evolution of the steady state at $V_\infty$, of $\mathcal{O}(\SI{5}{\s})$, are similar. Second, there is a characteristic overshoot in $V/V_\infty$ shortly after the onset of motion, common to various numerical approaches~\cite{schmitt2013_swimming,peng2023_weakly}, and already noted as `surprising' in \cite{michelin2013_spontaneous}. This initial push could be provided by the diffusive cloud of consumed or depleted fuel around the droplet (see eg. Fig.~\ref{fig:fluo}c-II), which causes a radial chemorepulsive gradient. 
Such gradients would be enhanced by confinement~\cite{anderson1989_colloid, picella2022_confined}: simulations have found the overshoot to increase with confinement~\cite{picella2022_confined}, and we find a similar tendency in our experimental data (Fig.\ \ref{fig:piv}(a)).

%
\section{Conclusions}
Tuning the dynamics of self-propelling droplets by temperature-driven surfactant interactions provides a promising framework to regulate micelle-mediated~\cite{babu2021_acceleration,wentworth2022_chemically} active droplet dynamics: we can now control self-propulsion from an unsteady or meandering state over quasi-ballistic propulsion to full arrest without needing to change the chemistry of the system. The gait control is encoded in the swimming medium and does not require complex micro-engineered swimmer design. Since the hydrophobic effect underlying the temperature dependent complex formation is entropy driven and similar aggregation effects are established for numerous surfactant-polymer combinations~\cite{nambam2012_effects,parmar2014_pluronic,li2001_binding,tam2006_insights,hecht1994_interaction}, this control method likely applies to active droplet models driven by micellar solubilisation in general, and could be tested in further studies, e.g. for SDS surfactant or DEP oil~\cite{izzet2020_tunable,thutupalli2018_flowinduced}. The transitions are almost fully reversible, excepting a slight reduction in peak speed that can be attributed to droplet shrinkage. 

Our hypothesis -- fuel binding by thermosensitive polymer cosurfactants --  does not account for the dynamics of adsorbed polymer at the interface, which might also be temperature dependent. However, we argue that these effects are, if present, secondary to the binding and release of TTAB in the swimming medium: generally, the desorption kinetics of large polymers are assumed to be exceedingly slow~\cite{butt2003_physics}.  
Thus, if these kinetics were the main drivers of thermoresponsive mode switching, it would not be consistent with our observations of cyclic reversibility and the fast response to changed external conditions, i.e. the instantaneous onset of motion in Fig.~\ref{fig:piv}a and particularly the fast, local adaptation to the fuel-rich medium outside the saturation area.

\par 
Our experiments fit into the framework of the  canonical theory for autophoretic particles, where the observed dynamic regimes correspond to interfacial modes becoming unstable with increasing or decreasing P\'eclet number. 
\par
While such higher order modes have been documented individually, the fundamental spontaneous transition from an isotropic zero order base state to a first order propulsion state is hard to observe experimentally, as the setup of the experiment usually provides sufficient disturbances to instantaneously set off droplet motion. By a non-invasive temperature driven crossing of the critical \Pen{} threshold, this is now  experimentally observable in both chemical and flow signatures.


%
% \subsection*{Methods}
 % All experimental and numerical methods are described  in detail in the Appendices.
 
% \subsection*{Author contributions}
% PR designed and performed experiments, analyzed data and wrote the paper, R\"a and SV performed experiments and analyzed data, YC designed and performed simulations, MJ designed experiments, CCM designed experiments, analyzed data and wrote the paper. All authors proofread the paper. 


% \subsection*{Data and code availability}
 % The data supporting the findings of this study and the numerical code underlying Fig. 4a are available at DOI:10.5281/zenodo.7818660.
 % \subsection*{Competing interests}
 % The authors declare no competing interests.

\subsection*{Acknowledgements}
We thank Dr.\ Babak Vajdi Hokmabad, Dr.\ Kevin Zhong and Dr.\ Stefan Karpitschka for invaluable advice and discussions, Dr.\ Stephan Weiss for providing the thermistor and Dr.\ Kristian Hantke for experimental support.
% %
