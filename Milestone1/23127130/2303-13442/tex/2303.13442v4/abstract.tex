Self-propelling active matter relies on the conversion of energy from the undirected, nanoscopic scale to directed, macroscopic motion. One of the challenges in the design of synthetic active matter lies in the control of dynamic states, or motility gaits. Here, we present an experimental system of self-propelling droplets with thermally controllable and reversible dynamic states, from unsteady over meandering to persistent to arrested motion. These states are known to depend on the Péclet number of the molecular process powering the motion, which we can now tune by using a temperature sensitive mixture of surfactants as propulsion fuel. We quantify the droplet dynamics  by analysing flow and chemical fields for the individual states, comparing them to canonical models for autophoretic particles.  In the context of these models, we experimentally observe, in situ, the fundamental first broken symmetry that translates an isotropic, immotile base state to self-propelled motility.
% We study the motion of active droplets in a aqueous mixture of supramicellar surfactant and a temperature sensitive co-surfactant. Depending on the ambient temperature, we observe a reversible transition from  unsteady to oscillatory to straight swimming to eventual arrest. We attribute this behaviour to the temperature dependent formation of mixed surfactant micelles, which bind and release the pure surfactant micelles fuelling the droplet propulsion. On subsequent cooling, we observe that the droplet arrest is extended with a sudden recovery at lower temperatures. Multiple temperature ramps and subsequent cooling cycles show that the hysteresis in droplet dynamics is recoverable. We quantify the droplet dynamics and hysteresis using flow field and chemical field measurements. Accumulation oil-filled micelles when the droplet is inactive leads to inhibition of interfacial activity and which prolongs the droplet arrest. 