% !TeX root=./main-revtex.tex
\section{Introduction}
Active matter is defined by the nonlinear conversion of free energy on the molecular scale into macroscopic dynamics - which means that small changes in molecular processes can crucially tip the overall system into a different area in phase space, corresponding to often counter-intuitive dynamical state changes.
\par
It is easy to move agents by external fields: with some more effort we can have them convert free energy on the molecular scale into self-propelled, autonomous motion. However, like a sorcerer's apprentice~\cite{goethe1798_hat} we find that stopping this motion is not a trivial task; after all, one is required to deplete an energy reservoir or reverse a dynamic instability. 
\par
This kind of control is important in the design and study of artificial and biological microswimmers, their theoretical modelling, experimental realisation, and, ultimately, to provide design principles and dynamic control for technological application.
Autophoretic particles \cite{bechinger2016_active,moran2019_microswimmers,zhang2021_chemically-powered,zottl2023_modeling} and, particularly, their experimental realisation by active droplets \cite{maass2016_swimming,babu2022_motile,birrer2022_we,michelin2023_selfpropulsion}
are popular active matter models driven by purely physicochemical mechanisms. 
Generally, their dynamics are characterized by a dimensionless Péclet number \Pen{}, quantifying the ratio of advective and diffusive transport of chemical fuel~\cite{michelin2013_spontaneous}. With increasing \Pen{}, autophoretic particles first transition  from \textit{passive} isotropic chemical conversion to \textit{active} self-propulsion, and further from persistent to unsteady motion via a sequence of broken symmetries and interfacial flow modes of increasing complexity~\cite{michelin2013_spontaneous,izri2014_self-propulsion,suga2018_self-propelled,izzet2020_tunable,meredith2020_predatorprey,hokmabad2021_emergence,suda2021_straight-to-curvilinear}.
\par
Recent studies have investigated the control of speed and dynamic states of such microswimmers in response to externally applied stimuli such as temperature \cite{tu2017_self-propelled,cholakova2021_rechargeable} or illumination \cite{florea2014_photo-chemopropulsion,kaneko2017_phototactic,xiao2018_moving,lancia2019_reorientation,alvarez2021_reconfigurable,ryabchun2022_run-and-halt,vankesteren2023_selfpropelling}. 
On heating, physical intuition suggests that motion should accelerate and destabilise, either by the increase of translational and rotational diffusion with decreasing viscosity, or by increased activity from the molecular thermodynamics driving the motion. 
However, the nonlinear dynamics of self-organised activity can drive counter-intuitive effects, as we have previously found for active droplets which  destabilise with increasing viscosity due to an associated increase in \Pen{}~\cite{hokmabad2021_emergence}. 
\par
In this study, we explore a counter-intuitive response to increasing temperature. We study active droplets using a temperature sensitive combination of co-surfactants as a fuel medium. We reversibly control dynamic states from unsteady to oscillatory to steady, straight swimming to eventual arrest with increasing ambient temperature. Notably, we are able to control \Pen{} to experimentally observe the fundamental first symmetry breaking of the inactive isotropic base state into directed self-propulsion in both chemical and hydrodynamic fields.
%
\section{Results}
\par
\subsection{Self-propelling droplets in co-surfactant solutions}
\begin{figure*}
    \includegraphics[width=\linewidth]{figures/fig1.png}
    \caption{\textbf{Schematics of the experimental setup and droplet propulsion mechanism.} (a) Setup: The droplet (diameter \SI{50}{\um}) swims in a quasi-2D ($\SI{13}{\mm}\times\SI{8}{\mm}\times\SI{50}{\um}$) cell on a temperature controlled microscope stage. (b) Temperature dependent, reversible formation of mixed surfactant micelles. (c) Droplet propulsion during solubilisation: an inhomogeneous distribution of empty TTAB micelles causes a self-sustaining Marangoni gradient at the oil-water interface.}
    \label{fig:setup}
\end{figure*}
\par
Our experimental system consists of oil droplets (CB15) immersed in an aqueous solution of the ionic surfactant tetradecyltrimethylammonium bromide at 9-15wt\% (TTAB) and the tri-block copolymer Pluronic F127  at 4wt\% (PF127) (\figref{fig:setup}c). 
\par
CB15 droplets have been shown in various studies to self-propel reliably in supramicellar solutions of pure TTAB~\cite{jin2017_chemotaxis,hokmabad2021_emergence}. Briefly put, the swimming is due to oil migrating from the droplet into TTAB micelles, which removes surfactant from the droplet posterior, while the anterior is replenished by the advection of fresh surfactant (Fig.~\ref{fig:setup}c). The resulting self-enhancing surface tension gradient drives the droplet forward until it is dissolved. Typically, a CB15 droplet of diameter \SI{50}{\um} will swim in 5wt\% TTAB for 1-2 hours.  Thus, empty TTAB micelles serve as fresh, and oil-filled ones as spent fuel.
\par
PF127 is a nonionic surfactant that forms micelles with a hydrophobic core of propylene oxide and an outer shell of hydrated ethylene oxide \cite{alexandridis1994_micellization,wanka1994_phase,bohorquez1999_study,stoeber2006_passive,jalaal2016_rheology,jalaal2018_gel-controlled} above the aggregation temperature $T_\text{CMC}$. 
In the presence of ionic co-surfactants like TTAB, which bind strongly with PF127 (\figref{fig:setup}b), mixed TTAB/PF127 micelles form, with excess TTAB forming single-species micelles \cite{hecht1994_interaction,li2001_binding,tam2006_insights} (Fig.~\ref{fig:setup}b). Importantly, the TTAB binding capacity of PF127 is amplified with increasing temperature and reduced by cooling. 
Our swimming medium is Newtonian, with mixed micelles forming above $T_\text{CMC}\approx\SI{21}{\celsius}$ and only weakly temperature dependent viscosity (see rheology, light scattering, and calorimetry data in SI).
\par
In the regimes considered here, TTAB is the primary cause of dissolution and the role of PF127 is simply to remove or release TTAB micelles from the bulk medium (Fig.~\ref{fig:setup}b,c).
The \Pen{} of droplet activity should therefore \textit{decrease} with \textit{increasing} temperature due to TTAB depletion.
%
\subsection{Swimming dynamics controlled by temperature and fuel concentration.}
\begin{figure*}
\centering
\includegraphics[width=1\linewidth]{figures/fig2.png}
\caption{\textbf{Swimming dynamics controlled reversibly via temperature and fuel concentration} 
(a) Droplet trajectory  transitioning from meandering to straight swimming to arrest during a heating and subsequent cooling ramp with a set rate of \SI{1}{\kelvin/\min} in a mixed surfactant swimming medium, colour coded by droplet speed and recorded temperature. See also Video S1. Colour map and scale bars (\SI{250}{\um}) apply to all figures in the paper, and concentrations for PF127 and TTAB are always 4wt\% and 10wt\%, respectively, unless stated otherwise.
(b) Map of the swimming dynamics depending on temperature and surfactant concentration. (c) Hysteresis between droplet stop and start transition temperatures. (d) Example of unsteady motion at 15 wt\% TTAB and \SI{21}{\celsius} (see also Video S3)\\
(e) Droplet trajectory during multiple heating/cooling cycles set at \SI{10}{\kelvin/\min} and 10 wt\% TTAB. See also Video S2. 
(f) Corresponding droplet speed vs. time, with inset plot of the recorded temperature ramps. (g) Droplet speed vs. recorded temperature showing a hysteresis in the re-onset of motion during cooling: the arrest during cooling is extended, with a sudden recovery of the initial speed at $T \approx \SI{17}{\celsius}$.}
\label{fig:traj}
\end{figure*}
%
We begin with an overview of the general swimming dynamics taken from wide-field video microscopy under changing the ambient temperature.
The setup comprises a quasi-2D microfluidic cell on a temperature controlled stage (\figref{fig:setup}a). 
\figref{fig:traj}a (Video S1) plots a trajectory colour-coded once by speed and once by temperature, recorded at a set heating/cooling rate of \SI{1}{\kelvin/\min}.
\par
We start at $T \approx \SI{15}{\celsius}< T_\text{CMC}$. 
Below $T_\text{slow} = \SI{22}{\celsius}$, the droplet meanders. Above, the motion is straight, gradually slows down and eventually stops at $T_\text{stop} \approx \SI{27}{\celsius}$. 
During a subsequent cooling ramp, the droplet remains immotile down to a significantly lower temperature $T_\text{start} \approx \SI{17}{\celsius}$, where it abruptly starts to meander again (Fig.~\ref{fig:traj}c). 
\par
As well as by temperature, the swimming dynamics are also susceptible to the TTAB concentration. Previous studies on single-surfactant systems found a transition from straight to reorienting to unsteady swimming~\cite{izzet2020_tunable,hokmabad2022_spontaneously} in response to an increase in fuel surfactant. This increase corresponds to increased \Pen{}, leading to the emergence of destabilising higher order modes in the interfacial flow and chemical fields, as predicted by canonical models for isotropic autophoretic particles~\cite{michelin2013_spontaneous,michelin2023_selfpropulsion,morozov2019_nonlinear}.
\par
At low temperatures, the presence of a co-surfactant does not change these dynamics, as we find a transition from meandering to unsteady motion (Fig.~\ref{fig:traj}d) with increasing TTAB concentration.  We have summarised these swimming dynamics  in a map spanned by temperature and surfactant concentration in \figref{fig:traj}b. 
With increasing temperature, we observe a universal transition via straight swimming to eventual arrest. We posit that the temperature dependent TTAB depletion lowers \Pen{} below the critical thresholds of higher order interfacial modes, down to $n=1$ for straight swimming and finally $n=0$, below the fundamental advection-diffusion instability. We do not provide a quantitative estimate of \Pen{} as in~\cite{hokmabad2021_emergence}, as we cannot quantify the temperature dependence of the underlying physical chemistry parameters. 
The transition temperatures increase with TTAB concentration:
for the droplet to arrest, more TTAB needs to be removed from the swimming medium. We found the stop/start hysteresis for all TTAB concentrations in use (Fig.~\ref{fig:traj}c).
\par
These dynamics are fully reversible with temperature. The experiment shown in \figref{fig:traj}e-g and Video S2 was recorded at a faster rate of \SI{10}{\kelvin/\min} to permit multiple heating and cooling cycles, with dynamics similar to the system cooled at slower rates. \figref{fig:traj}e and f show the droplet trajectory colour coded by temperature and a corresponding plot of speed over time. The initial motion is recovered after each heating and cooling cycle, apart from a very gradual decrease in maximum speed which we may attribute to droplet shrinkage. We have further analysed speed versus temperature in \figref{fig:traj}g), and found a  hysteresis cycle with a delayed re-onset of motion reproducible over multiple heating/cooling ramps. 
%
\subsection{Chemical and flow fields}
\begin{figure*}
\centering
\includegraphics[width=1\linewidth]{figures/fig3.png}
\caption{\textbf{The hysteresis in the re-onset of droplet motion is caused by spent fuel aggregation.} (a) Droplet trajectory during heating and subsequent cooling ramp colour coded by speed. (b) Kymograph showing the evolution of chemical concentration field around the droplet interface and the recorded temperature (coloured symbols). (c) Snapshots of droplet chemical trails at different temperatures as marked by I-IV in (a) and (b). 
See also Video S4. \SI{250}{\um} scale bars and colour maps as defined in Fig.~\ref{fig:traj}.}
\label{fig:fluo}
\end{figure*}
We continue with a discussion of the chemical dynamics during the droplet arrest, to motivate the hysteresis in the re-onset of motion; and of the corresponding flow field to investigate the interfacial mode evolution.
\par
The field of spent chemical fuel affects the interfacial Marangoni gradients driving the droplet~\cite{izzet2020_tunable,hokmabad2021_emergence,ramesh2022_interfacial}. We visualise it by doping the droplet with the fluorescent dye Nile Red, which co-moves with the oil phase into the filled micelles, and extracting the fluorescence intensity $I$ under videomicroscopy~\cite{hokmabad2022_chemotactic} (Video S4). In fig.~\ref{fig:fluo}, we analyze the chemical dynamics for one droplet during a heating and cooling cycle, by a speed-coded trajectory (a), a kymograph of $I$ around the droplet perimeter, $\theta$ vs. time and recorded temperature (b), and with micrographs at the times marked I--IV (c). 
\par
During heating, the droplet transitions from unsteady to straight swimming to immotility (Fig.~\ref{fig:fluo}a). We note that this particular experiment featured some global drift causing translation even in the immotile state (see Video S4 and SI).  In the kymograph, at $T < \SI{26}{\celsius}$, the band corresponding to the chemical trail translates in the angular space due to the reorientation of the droplet (I). At $T \approx \SI{26}{\celsius}$, the droplet slows down and comes to a halt. As the system is cooled down to $T \approx \SI{21}{\celsius}$, the inactive droplet still solubilises isotropically, and oil-filled micelles accumulate around the perimeter $\theta$. Correspondingly, the band in the kymograph widens over the entire angular space (II). The accumulated filled micelles block empty micelles from reaching the interface \cite{ramesh2022_interfacial,morozov2020_adsorption}, such that even more mixed micelles need to disintegrate to restart activity. Thus, the motility transition temperature is lowered, here to $T_\text{start} = \SI{16.8}{\celsius}$, where the droplet escapes the oil-filled micelle cloud (III) and swims away (IV). Notably, the regime of straight swimming is bypassed on cooling: outside the cloud of spent fuel, more TTAB has been released at the same temperature, such that the droplet experiences a higher \Pen{} once it escapes its self-generated trap. 
\par
\begin{figure*}
\centering
\includegraphics[width=1\linewidth]{figures/fig4.png}
\caption{\textbf{Observation of the fundamental transition between passive dissolution and active propulsion.} 
(a) Onset of motion during re-cooling, speed (normalised to final speed $V_\infty$) vs.\ time for multiple runs with $t_\text{onset}$ set to the point of maximum acceleration. Inset: simulation for an isotropic autophoretic particle under comparable conditions ($H=2.2R$ confinement, $\Pen=8$), following~\cite{michelin2013_spontaneous,picella2022_confined}. See also Video S6.
(b) Internal flow field with increasing temperature, starting at a mixed dipolar/quadrupolar mode (meandering), over a purely dipolar mode (straight) that recedes to the anterior (slowdown). 
Vectors and colour map inside the droplet indicate the velocity field $\vec{u}(x,y)$;  arrows around the perimeter mark the active regions on the droplet interface. Scale bar \SI{10}{\um}.}
\label{fig:piv}
\end{figure*}
\par
To analyse the mode evolution causing arrest and sudden onset of motion during heating and cooling (Video S5), we added tracer colloids to the oil phase, performed high resolution bright field videomicroscopy and analysed the internal flow field $\vec{u}(x,y)$ by particle image velocimetry (PIV). \figref{fig:piv}b shows the evolution of $\vec{u}$ with increasing temperature. At $T = \SI{16}{\celsius}$, we see a  mixed  dipolar and quadrupolar flow field (modes $n=1,2$) corresponding to the meandering trajectory in Fig.~\ref{fig:traj}a~\cite{suda2021_straight-to-curvilinear,hokmabad2021_emergence}. At even higher temperature, $T = \SI{21}{\celsius}$, the droplet swims straight, \Pen{} decreases and the flow field is purely dipolar ($n=1$).  As the droplet begins to slow down, an inactive region spreads from the droplet posterior ($T = \SI{24}{\celsius}$). Finally ($T = \SI{27}{\celsius}$), just before the droplet stops ($n=0$), only a small region at the droplet anterior is active \cite{ramesh2022_interfacial}. As shown in Fig.~\ref{fig:fluo}II, the local environment isotropically saturates with spent fuel while the droplet is immotile.
\par
The gradual increase of \Pen{} during cooling now allows us to directly observe the fundamental first transition from the immotile base state to self-propelled motion~\cite{michelin2013_spontaneous,michelin2023_selfpropulsion}. This can be realised theoretically using hydrodynamic models, originally proposed in~\cite{michelin2013_spontaneous} as follows:
A spherical particle of radius $R$ is immersed in a fluid medium containing a chemical fuel at concentration $c$. At negligible Reynolds numbers, the flow is governed by the Stokes equations, $\mu\nabla^2\vec{u}=\nabla p$, $\nabla\cdot\vec{u}=0.$  The chemical field is coupled by an advection-diffusion equation, 
\begin{align}|\Pen|\left(\frac{\partial c}{\partial t} + \vec{u}\cdot\nabla c\right) &= \nabla^2c, &
\Pen{}&\equiv \frac{\mathcal{AM}R}{D^2},			\end{align} and by the particle consuming fuel at its boundary, $\partial_t c(R) =-\mathcal{A}$.  The P\'eclet number is set by the activity $\mathcal{A}$, mobility $\mathcal{M}$ and diffusivity $D$ of the chemical species and the particle radius $R$. Using a decomposition into squirmer modes and a linear stability analysis around the isotropic base state (mode $n=0$), the authors find a transition to the propulsive dipolar state ($n=1$) above a threshold value of $\Pen{}=4$.

Our mixed surfactant method allows us to observe the growth of the dipolar mode in situ and a detailed comparison with the theory. We recorded the droplet speed $V$ at the onset of motion and performed a simulation of the interfacial instability, solving for the full 3D problem following~\cite{michelin2013_spontaneous,picella2022_confined} and adapted to our cell geometry (see SI), with good qualitative agreement between experimental and numerical data (Fig.~\ref{fig:piv}a, Video S6).
%
\section{Conclusions}
We show that controlling the dynamics of self-propelling droplets by micelle mediated interactions \cite{babu2021_acceleration,wentworth2022_chemically} provides a promising framework to regulate active droplet dynamics: we can now control self-propulsion from an unsteady or meandering state over quasi-ballistic propulsion to full arrest without needing to change the chemistry of the system. We note that the transitions are fully reversible. 
\par 
Our experiments fit into the framework of the  canonical theory for autophoretic particles, where the observed dynamic regimes correspond to interfacial modes becoming unstable with increasing or decreasing P\'eclet number. 
\par
While such higher order modes have been documented individually, the fundamental spontaneous from an isotropic zero order base state to a first order propulsion state is hard to observe experimentally, as the setup of the experiment usually provides sufficient disturbances to instantaneously set off droplet motion. By a temperature driven crossing of the critical \Pen{} threshold, this is now possible, and, as we have demonstrated by chemical and flow signatures, even quantifiable.  
%
\subsection*{Author contributions}
PR designed and performed experiments, analyzed data and wrote the paper, PR\"a and SV performed experiments and analyzed data, YC designed and performed simulations, MJ designed experiments, CCM designed experiments, analyzed data and wrote the paper. All authors proofread the paper. The authors report no conflict of interest.
%
\subsection*{Acknowledgements}
We thank Dr. Babak Vajdi Hokmabad  and Dr. Stefan Karpitschka for invaluable advice and discussions, Dr. Stephan Weiss for providing the thermistor and Dr. Kristian Hantke for experimental support.
%

