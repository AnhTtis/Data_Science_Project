\section{The Known Examples}
Before proving that $\TC(\al,1-\al)$, $Q_2(\frac{1}{3})$, and $Q_2(\frac{1}{3})^\times \oplus \GenG{\id}$ are the only examples (up to isomorphism) of $2$-generated axial algebras of Monster type with skew axet $X'(1+2)$, it would be useful to show that they satisfy the conditions we desire. 
\subsection{Three Dimensions}
Let $A=\TC(\al)$ with $\al\notin \{0,1, \frac{1}{2},-1\}$. In \cite{mcinroy2022axial} and \cite{franchi20212}, this example has already been stated. Let $x,y,z$ be the distinct axes of $\mcal{J}(\al)$-type and notice that $\{x,y,z\}$ form a basis for $A$. The element $\id=\frac{1}{\al+1}(x+y+z)$ is the identity element of $A$. Letting $w=\id-x$, one can see that $w$ is an axis of $\mcal{J}(1-\al)$-type and $\al\id=(-w+y+z)$. Looking at $\GenA{w,y}$, we have
\begin{eqnarray*}
wy &=& (\id-x)y = y-\frac{\al}{2}(x+y-z) = y-\frac{\al}{2}(-w+\id +y-z)\\
&=&y-\frac{\al}{2}(-w+y-z)-\frac{1}{2}(-w+y+z) = \frac{\al+1}{2}w +\frac{1-\al}{2}(y-z).
\end{eqnarray*}
Hence $z\in \GenA{w,y}$ and $x$ is too. Therefore $A=\GenA{w,y}$. As $A$ is now of $\mcal{M}(\al,1-\al)$-type with respect to the axes $w$ and $y$, let the Miyamoto involutions of $w$ and $y$ be $\tau_w$ and $\tau_y$ respectively. As $y$ is of $\mcal{J}(\al)$-type, $\tau_y$ is the identity. On the other hand, one can check that $x\in A_0(w)$ and $y-z\in A_{1-\al}(w)$. Further,
\[ y = \frac{\al}{2}x + \frac{\al+1}{2}w +\frac{1}{2}(y-z)\]
and so
\[ y^{\tau_w}=\frac{\al+1}{2}w+\frac{\al}{2}x -\frac{1}{2}(y-z) =z.\]
Therefore $A$ is skew with axet $X'(1+2)$. To avoid confusion about fusion laws, we denote the skew algebra by $\TC(\al,1-\al)$.

\begin{rem}
If one used the same method to $\TC(\frac{1}{2})$, then it would produce the algebra $S(1)$. By Remark 5.9 and Lemma 5.12 in \cite{mcinroy2021forbidden}, $S(1)$ has axet $X(6)$.
\end{rem}

\subsubsection{A Variation}
One may notice that for $\TC(-1)$, we have only $\jor{-1}$-axes and no identity. It would be absurd to try to make an axial algebra of Monster type out of it. We therefore use $\TC(2)$ to produce an axial algebra with fusion law $\mon{-1,2}$. Take $A=\TC(2)$ with $\{u,v,w\}$ being the three distinct axes of $\jor{2}$-type. There is an identity in $A$, $\id=\frac{1}{3}(u+v+w)$. Let $y:=\id-u$ and $z:=\id-v$ be $\jor{-1}$-axes. Let us look at $\GenA{w,y}$. We get
\[wy=(\id-u)w=w-uw=w-(u+w-v)=v-u\]
and
\[y(u-v)=(\id-u)(u-v)=-v+uv=-v+(u+v-w)=u-w.\]
Hence $v-w\in \GenA{w,y}$ as well as $v\in \GenA{w,y}$. Further, $u\in \GenA{w,y}$ and so $A=\GenA{w,y}$. These axes now satisfy the fusion law of $\mon{-1,2}$. We have that $\tu{y}$ is trivial as it is a $\jor{-1}$-axis while $\tu{w}$ is not. The reader can check:
\begin{itemize}
    \item $w \in A_1(w)$,
    \item $\id-w =-(y+z)\in A_0(w)$, and
    \item $y-z \in A_2(w)$.
\end{itemize}
We have
\[ y= \frac{1}{2}(y+z)+\frac{1}{2}(y-z),\]
and 
\[ y^{\tu{w}}=\frac{1}{2}(y+z)-\frac{1}{2}(y-z)=z.\]
Therefore $A$ is skew with axet $X'(1+2)$, and denote this algebra by $\TC(-1,2)$.
\begin{note}
We have $yz=(\id-u)(\id-v)=\id-u-v+(u+v)-w=\id-w=-y-z$ and so $\GenA{y,z}\cong \TC(-1)^\times$, the $2$-dimensional quotient of $\TC(-1)$. This will be important in the proof later.
\end{note}
\begin{rem}
Another way to construct $\TC(-1,2)$ would be by attaching a universal identity to $\TC(-1)^\times$ (similar to the construction of $Q_2(\frac{1}{3})^\times \oplus \GenG{\id}$).
\end{rem}

\subsection{Four Dimensions}
Let $\F$ have characteristic not equal to $3$ and let $Q_2(\frac{1}{3})$ be defined as in \cite[Section 5.3]{galt2021double} with $\eta=\frac{1}{3}$. The algebra has a basis of two single axes, $s_1$ and $s_2$, and two double axes, $d_1$ and $d_2$ with the multiplication defined in Table \ref{mult Q}.
\begin{table}[b]
\begin{center}
    \begin{tabular}{c|c|c|c|c}
         & $s_1$ & $s_2$ & $d_1$ &$d_2$   \\
         \hline
        $s_1$ & $s_1$ &$0$& $\frac{1}{3}s_1+\frac{1}{6}d_1-\frac{1}{6}d_2$& $\frac{1}{3}s_1-\frac{1}{6}d_1+\frac{1}{6}d_2$\\
        \hline
        $s_2$ & & $s_2$ & $\frac{1}{3}s_2+\frac{1}{6}d_1-\frac{1}{6}d_2$& $\frac{1}{3}s_2-\frac{1}{6}d_1+\frac{1}{6}d_2$\\
        \hline
        $d_1$ &  &  & $d_1$&$-\frac{1}{3}s_1-\frac{1}{3}s_2+\frac{1}{3}d_1+\frac{1}{3}d_2$ \\
        \hline
        $d_2$ &  &  &  & $d_2$
    \end{tabular}
       \caption{The multiplication table of $Q_2(\frac{1}{3})$}
       \label{mult Q}
       \end{center}
\end{table}
\subsubsection{In characteristic not equal to 5}
When the field has characteristic not equal to $5$, $A=Q_2(\frac{1}{3})$ is simple by Theorem 1.6 in \cite{galt2021double} and has identity element, $\id = \frac{3}{5}(s_1+s_2+d_1+d_2)$. Notice  $s_1$ is a $\jor{\frac{1}{3}}$-axis while $d_1$ is a $\mon{\frac{2}{3},\frac{1}{3}}$-axis. Let $t_1:=\id-d_1$ and $t_2:=\id -d_2$. Then $t_1$ and $t_2$ are $\mcal{M}(\frac{1}{3},\frac{2}{3})$-axes. Let us look at $\GenA{t_1,s_1}$. We have that 
\[ s_1t_1 = s_1(\id -d_1)=s_1-\frac{1}{3}s_1-\frac{1}{6}d_1+\frac{1}{6}d_2 = \frac{2}{3}s_1-\frac{1}{6}d_1+\frac{1}{6}d_2= \frac{2}{3}s_1+\frac{1}{6}t_1-\frac{1}{6}t_2.\]
Therefore $t_2\in \GenA{s_1,t_1}$. Notice that $-\frac{1}{3}\id = s_1+s_2-t_1-t_2$. We have
\begin{eqnarray*}t_1t_2 = (\id-d_1)(\id-d_2) &=& \id -d_1-d_2+d_1d_2\\
&=& \id -\frac{1}{3}s_1-\frac{1}{3}s_2-\frac{2}{3}d_1-\frac{2}{3}d_2\\
&=& \id -\frac{1}{3}s_1-\frac{1}{3}s_2+\frac{2}{3}(\id-d_1)+\frac{2}{3}(\id-d_2)-\frac{4}{3}\id\\
&=&-\frac{1}{3}\id -\frac{1}{3}s_1-\frac{1}{3}s_2+\frac{2}{3}t_1+\frac{2}{3}t_2\\
&=& s_1+s_2-t_1-t_2 -\frac{1}{3}s_1-\frac{1}{3}s_2+\frac{2}{3}t_1+\frac{2}{3}t_2\\
&=& \frac{2}{3}s_1+\frac{2}{3}s_2-\frac{1}{3}t_1-\frac{1}{3}t_2.
\end{eqnarray*}
Hence $s_2$ and $\id$ are in $\GenA{s_1,t_1}$. Therefore $d_1,d_2\in \GenA{s_1,t_1}$ and so $A=\GenA{s_1,t_1}$ and has fusion law $\mon{\frac{1}{3},\frac{2}{3}}$. As $s_1$ is a $\jor{\frac{1}{3}}$-axis, $\tu{{s_1}}$ is the identity map.  The reader can check that:
\begin{itemize}
    \item $t_1 = \frac{1}{5}(3s_1+3s_2-2d_1+3d_2)\in A_1(t_1)$,
    \item $d_1 \in A_0(t_1)$, 
    \item $s_1+s_2-d_2 \in A_\frac{1}{3}(t_1)$, and
    \item $s_1-s_2 \in  A_\frac{2}{3}(t_1)$.
\end{itemize}
Hence
\[s_1 = \frac{5}{12}t_1+\frac{1}{6}d_1+\frac{1}{4}(s_1+s_2-d_2)+\frac{1}{2}(s_1-s_2).\]
Therefore
\[ s_1^{\tu{t_1}} = \frac{5}{12}t_1+\frac{1}{6}d_1+\frac{1}{4}(s_1+s_2-d_2)-\frac{1}{2}(s_1-s_2)=s_2.\]
Hence $A$ is skew with axet $X'(1+2)$. To avoid confusion with fusion laws, we denote this skew algebra by $Q_2(\frac{1}{3},\frac{2}{3})$. 
\begin{note}
One may think we can make any $Q_2(\eta)$, $\eta\notin \{-\frac{1}{2},\frac{1}{3}\}$, a skew axial algebra of Monster type with this method. This is partly possible if $\GenA{t_1,s_1}=Q_2(\eta)$. However the algebra would be skew but not of Monster type. The fusion law would have five elements and be an extension of the Monster fusion law.
\end{note}
\subsubsection{In characteristic 5}
Suppose now that the field has characteristic $5$. Then $\frac{1}{3}=-\frac{1}{2}$ and $Q_2(\frac{1}{3})$ is not simple by Theorem 1.6 in \cite{galt2021double}. This algebra has an annihilating element rather than an identity and so it is impossible with the technique used so far. We have that $I=\GenG{s_1+s_2+d_1+d_2}$ is the radical of $Q_2(\frac{1}{3})$. The quotient is denoted by $Q_2(\frac{1}{3})^\times$ and it is spanned by axes $\{x,y,z\}$ with their multiplication stated in Table \ref{mult Qx} (without the final row and column).
\begin{table}[b]
\begin{center}
    \begin{tabular}{c|c|c|c|c}
         & $x$ & $y$ & $z$ & $\id$   \\
         \hline
        $x$ & $x$ &$0$& $3x+y+2z$ & $x$\\
        \hline
        $y$ & & $y$ & $x+3y+2z$ & $y$\\
        \hline
        $z$ &  &  & $z$ & $z$\\
        \hline 
        $\id$ & & & & $\id$
    \end{tabular}
       \caption{The multiplication table of $Q_2(\frac{1}{3})^\times \oplus \GenG{\id}$}
       \label{mult Qx}
       \end{center}
\end{table}
The reader can check that $Q_2(\frac{1}{3})^\times =\GenA{x,z}$ with $x$ a $\jor{\frac{1}{3}}$-axis and $z$ a $\mon{\frac{2}{3},\frac{1}{3}}$-axis and so cannot be skew. In fact, this algebra has an axet of type $X(4)$. 

Let $A=Q_2(\frac{1}{3})^\times \oplus \GenG{\id}$ to be a $4$-dimensional algebra, where we add a universal identity element, $\id$, and multiplication is stated in Table \ref{mult Qx}. Let $w:=\id -z$ and let us look at $\GenA{x,w}$. Notice that
\[ wx=x(\id-z)=x - (3x+y+2z)=-2x-y-2z.\]
Therefore $y+2z\in \GenA{x,w}$ and we have
\[ w(y+2z)=wy=(\id-z)y=y-(x+3y+2z)=-x-2y-2z.\]
Hence $y+z\in \GenA{x,w}$. Moreover, $y,z \in \GenA{x,w}$ and so is $\id$. Therefore $A=\GenA{w,x}$. Showing that $x$ and $w$ are axes is a straightforward task as $x$ and $z$ are axes of the subalgebra $Q_2(\frac{1}{3})^\times$. We have that $x$ is a $\jor{\frac{1}{3}}$-axis and $w$ is a $\mon{\frac{1}{3},\frac{2}{3}}$-axis. Therefore $\tu{x}$ is trivial and $\tu{w}$ is not. For $w$, we have the following eigenvectors:
\begin{itemize}
    \item $w\in A_1(w)$,
    \item $z \in A_0(w)$,
    \item $x+y+3z \in A_{\frac{1}{3}}(w)$, and
    \item $x-y \in A_{\frac{2}{3}}(w)$. 
\end{itemize}
We have
\[ x= z+\frac{1}{2}(x+y+3z)+\frac{1}{2}(x-y)\]
and so
\[ x^{\tu{w}}=z+\frac{1}{2}(x+y+3z)-\frac{1}{2}(x-y)=y.\]
Therefore $A$ is skew with axet $X'(1+2)$. Further, $A\not\cong Q_2(\frac{1}{3})$ as $A$ has an identity element while $Q_2(\frac{1}{3})$ has not, in characteristic $5$. 

\begin{note}
One may think we can apply this method to $Q_2(-\frac{1}{2})^\times$ in characteristic not equal to $5$ to produce new algebras. This can be done however we run in the same problem as $Q_2(\eta)$ where the fusion law is an extension of the Monster fusion law and one needs to check that it is $2$-generated.
\end{note}