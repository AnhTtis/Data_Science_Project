\section{Proof of Theorem \ref{Theorem}}

Let $(A,X)$ be an $\mon{\al,\bt}$-axial algebra satisfying the hypothesis of Theorem \ref{Theorem} with $X=\{a,b\}$. By Proposition \ref{span prop}, we may assume that $A$ is spanned by $\{a,b,c,\sg\}$ and multiplication is defined in Table \ref{mult table}. We assume $b$ is a $\jor{\al}$-axis with $c=b^{\tu{a}}$. We set 
$U:=\GenA{b,c}$ which is a  $\jor{\al}$-axial algebra. The proof is split into two parts: $P=0$ or $P\neq 0$. 

\subsection{Orthogonal axes}\label{P zero}
For this section, we will assume that $P=0$.
By Table \ref{mult table}, $bc=0$; that is $U\cong \TB$. Hence $d:=b+c$ is a $\jor{2\al}$-axis in $A$. Notice that $d$ is not primitive as $A_1(d)=\GenG{b,c}$. Let $F:=A_{\{0,1,\al\}}(a)$ be the fixed subalgebra of $\tu{a}$. We have that $a,d,\sg \in F$ and $b-c\notin F$ which implies $\text{dim}F\leq\text{dim}A - 1\leq 3$.

\begin{lem}
We have $\al=\frac{1}{3}$, $F\cong \TC(\frac{1}{3},\frac{2}{3})$ or $F\cong \TC(\frac{2}{3},\frac{1}{3})$, and $A$ has dimension $4$.
\proof  One can show $F=\GenA{a,d}$ is a primitive axial algebra. We know $a$ satisfies $\jor{\al}$ fusion law and $d$ satisfies the $\jor{2\al}$ fusion law. As $\al\neq 2\al$, then $F\cong \TB$, $F\cong \TC(\frac{1}{3},\frac{2}{3})$, or $F\cong \TC(\frac{2}{3},\frac{1}{3})$ by Proposition \ref{Rehren Thm}. Let us assume $F\cong \TB$. Then $ad=0$ thus $\sg =-\bt a -\frac{\bt}{2}(b+c)$. Therefore the $\al$-eigenvector of $\text{ad}_a$ by Lemma \ref{eigen a} is now
\[ \gm a +\frac{1}{2}\bt(b+c)+\sg = (\gm-\bt)a=-\lm_1 a.\]
To avoid contradiction, we must have $\lm_1=0$ and $a$ is a $\jor{\bt}$-axis in $A$. By Proposition \ref{Rehren Thm}, $A\cong \TC(\al,1-\al)$ and $\id$ exists. Since $bc=0$, we must have $c=\id -b$ but that would make $c$ a $\jor{\bt}$-axis in $A$, which would contradict our axet. Therefore $\al=\frac{1}{3}$. We have $F\cong \TC(\frac{1}{3},\frac{2}{3})$ or $F\cong \TC(\frac{2}{3},\frac{1}{3})$. Since $F$ is $3$-dimensional, $A$ has to be $4$-dimensional. \qed
\end{lem}

\begin{lem}
We have that $A$ has fusion law $\mon{\frac{1}{3}, \frac{2}{3}}$ and multiplication is shown in Table $\ref{mult 4dim}$.

\proof As $P=0$, we have that $\gm^f=0$, moreover $\lmf_1=\bt$, by Equation (\ref{proof3}). Substituting into Equation (\ref{proof2}), we have
\[\bt \dt = \frac{1}{2}\bt(\al-\bt)-\bt^2(\al-\bt)+\bt^2(\al-2\bt)=\bt\left(\frac{1}{2}(\al-\bt)-\bt^2\right).\]
With $\al=\frac{1}{3}$, we get $\lm_1=\frac{1}{4}(\bt+1)$. Further, with $P=0$, we get
\[ 0=-\frac{4}{3}\lm_1+\frac{2}{3}\bt+\frac{1}{9}=-\frac{1}{3}(\bt+1)+\frac{2}{3}\bt+\frac{1}{9}=\frac{1}{3}\bt-\frac{2}{9}.\]
Therefore $\bt=\frac{2}{3}$.

With $F\cong \TC(\frac{1}{3},\frac{2}{3})$ or $F\cong \TC(\frac{2}{3},\frac{1}{3})$, there are exactly three $\mcal{J}(\frac{1}{3})$-axes in $F$, which are $a$ and two other elements, $e$ and $f$. Without loss of generality, let $e=\id -d$ where $\id=\frac{3}{4}(a+e+f)=3(a-d+f)$. Therefore $\{a,d,f,b\}$ is a basis for $A$ and by the multiplication of $F$, we have
\begin{eqnarray*}
ad&=&a(\id-e)=a-ae=a-\frac{1}{6}(a+e-f)=\frac{5}{6}a+\frac{1}{6}f-\frac{1}{6}(\id-d)\\
&=&\frac{5}{6}a+\frac{1}{6}f+\frac{1}{6}d-\frac{1}{2}(a-d+f)= \frac{1}{3}a +\frac{2}{3}d-\frac{1}{3}f,
\end{eqnarray*}
\begin{eqnarray*}
fd&=&f(\id-e)=f-fe=f-\frac{1}{6}(f+e-a)=\frac{5}{6}f+\frac{1}{6}a-\frac{1}{6}(\id-d)\\
&=&\frac{5}{6}f+\frac{1}{6}a+\frac{1}{6}d-\frac{1}{2}(a-d+f)= \frac{1}{3}f +\frac{2}{3}d-\frac{1}{3}a,
\end{eqnarray*}
and
\begin{eqnarray*}
af&=&\frac{1}{6}(a+f-e)=\frac{1}{6}(a+f+d-\id)\\
&=&\frac{1}{6}(a+f+d-3(a-d+f))= -\frac{1}{3}a+\frac{2}{3}d-\frac{1}{3}f. 
\end{eqnarray*}
Further,
\[ bd = b(b+c)=b+bc=b.\]
We have $ad =a(b+c)= 2\sg +\frac{4}{3} a +\frac{2}{3} d$, by the multiplication table of $A$, and so
\[ \sg =-\frac{1}{2}a -\frac{1}{6}f.\]
Hence 
\[ ab = \sg +\frac{2}{3} a +\frac{2}{3} b = \frac{1}{6}a-\frac{1}{6}f+\frac{2}{3} b.\] 
As $\lmf_1=\bt$, notice that $\frac{1}{6}a-\frac{2}{9} b +\frac{1}{6}f \in A_0(b)$ by Lemma \ref{eigen b}. Thus
\[
0 = b\left(a-\frac{4}{3}b +f\right) = \frac{1}{6}a -\frac{1}{6}f+\frac{2}{3} b -\frac{4}{3} b +bf\\
\]
which is equivalent to
\[
bf =-\frac{1}{6}a +\frac{1}{6}f+\frac{2}{3}b.
\]
Using $c=d-b$, we get the multiplication in Table \ref{mult 4dim}.\qed
\end{lem}
\begin{table}[ht]
\begin{center}
    \begin{tabular}{c|c|c|c|c}
         & $b$ & $c$ & $a$ &$f$   \\
         \hline
        $b$ & $b$ &$0$& $\frac{2}{3}b+\frac{1}{6}a-\frac{1}{6}f$& $\frac{2}{3}b-\frac{1}{6}a+\frac{1}{6}f$\\
        \hline
        $c$ & & $c$ & $\frac{2}{3}c+\frac{1}{6}a-\frac{1}{6}f$& $\frac{2}{3}c-\frac{1}{6}a+\frac{1}{6}f$\\
        \hline
        $a$ &  &  & $a$&$\frac{2}{3}b+\frac{2}{3}c-\frac{1}{3}a-\frac{1}{3}f$ \\
        \hline
        $f$ &  &  &  & $f$
    \end{tabular}
       \caption{The multiplication table of $A$ when $P=0$}
       \label{mult 4dim}
       \end{center}
\end{table}
\begin{lem}
Let $\F$ have characteristic not equal to $5$. Then $A\cong Q_2(\frac{1}{3},\frac{2}{3})$.
\proof We define the map $\varphi: A \rightarrow Q_2(\frac{1}{3},\frac{2}{3})$ with $\varphi(a)=t_1$, $\varphi(b)=s_1$, $\varphi(c)=s_2$, and $\varphi(f)=t_2$. The reader can check that this is an isomorphism. \qed
\end{lem}
\begin{lem}
Let $\F$ have characteristic equal to $5$. Then $A\cong Q_2(\frac{1}{3})^\times \oplus \GenG{\id}$. 
\proof We have the map $\phi: A \rightarrow Q_2(\frac{1}{3})^\times\oplus \GenG{\id}$ with $\phi(a)=\id-z$, $\phi(b)=x$, $\phi(c)=y$, and $\phi(\id)=\id$. This is a straightforward check that $\phi$ is an isomorphism. \qed
\end{lem}

\subsection{Non-Orthogonal Case}\label{P not zero}
We will now assume $P\neq 0$. As $U$ is a axial algebra of $\jor{\al}$-type, $U$ is either $\TB$, $S(2)^\circ$, $\TC(-1)^\times$ or $3$-dimensional. We should note that $S(2)^\circ$ is $\text{Cl}^0(\F^2,b)$ in \cite{hall2015primitive}. We will look at each possibility separately to complete our proof. We recommend the reader to \cite[Section 3]{hall2015primitive} and  \cite[Section 5]{mcinroy2021forbidden} for more information on $2$-generated axial algebras of Jordan type.

\begin{lem}
We have $U\not\cong \TB$. 
\proof Let $U\cong \TB$. As $P\neq 0$ then $\sg = -\bt a$. Then $A$ is at most $3$-dimensional. Notice that $ab = \bt b$. Therefore $b\in A_\bt(a)$ and we have that $c=b^{\tu{a}}=-b$. This is a contradiction as $c^2=(-b)^2=b\neq c$. \qed
\end{lem}
\begin{lem}\label{S2}
We have $U\not\cong S(2)^\circ$.
\end{lem}
\proof Let $U\cong S(2)^\circ$ and $bc=\frac{1}{2}(b+c)$. Further $bc=P(a+\frac{1}{\bt}\sg)$. Since $P\neq 0$, we can express $\sg$ in terms of $a$, $b$ and $c$. Thus
\[ \sg = \frac{\bt}{2P}(b+c)-\bt a.\]
Note that
\[ab =\frac{\bt}{2P}(b+c)+\bt b\;\text{ and }\;ac = \frac{\bt}{2P}(b+c)+\bt c.\]
Hence $(b+c)$ is an eigenvector of $\text{ad}_a$ with eigenvalue $\mu=\frac{\bt}{P}+\bt$. By $a$ being primitive and $\bt\neq 0$, $\mu\notin \{ 1, \bt\}$. Hence $\mu \in \{0,\frac{1}{2}\}$.

Suppose $\mu=\frac{1}{2}$ and so $(b+c)\in A_\frac{1}{2}(a)$. By the Monster fusion law, $(b+c)^2 \in A_{\{0,1\}}(a)$. However
\[ (b+c)^2=b+c+2bc=2(b+c)\in A_\frac{1}{2}(a).\]
Thus $b+c=0$ which contradicts our choice of $U$. 

Therefore $\mu=0$ and $a$ is a $\jor{\bt}$-axis in $A$. As $b$ is a $\jor{\frac{1}{2}}$-axis and by Proposition \ref{Rehren Thm}, $\frac{1}{2}+\bt=1$. Thus $\bt=\frac{1}{2}=\al$ which contradicts the fusion law. \qed
\begin{lem}
If $U\cong \TC(-1)^\times$, then $A\cong \TC(-1,2)$.
\proof Suppose $U\cong \TC(-1)^\times$. We will use a similar method to Lemma \ref{S2}. We have $bc= -(b+c)$ and $bc=P(a+\frac{1}{\bt}\sg)$ thus
\[ \sg = -\bt a -\frac{\bt}{P}(b+c)\] 
and $A$ is at most 3-dimensional. Note that
\[ ab = -\frac{\bt}{P}(b+c)+\bt b\; \text{ and }\; ac = -\frac{\bt}{P}(b+c)+\bt c.
\]
Hence $(b+c)$ is an eigenvector of $\text{ad}_a$ with eigenvector $\mu=-\frac{2\bt}{P}+\bt$. Note that $\mu\notin \{1,\bt\}$ as it would contradict $a$ being primitive and $\bt\neq 0$ respectively. Thus $\mu \in \{0,-1\}$.

Suppose $\mu=-1$ moreover $(b+c)\in A_{-1}(a)$. By the Monster fusion law, $(b+c)^2\in A_{\{1,0\}}(a)$. However
\[ (b+c)^2=b+c+2bc=-(b+c)\in A_{-1}(a).\]
Thus $b+c=0$ which contradicts $U$.

Therefore $\mu=0$ and $a$ is $\jor{\bt}$-axis and by Proposition \ref{Rehren Thm}, we get $\bt=2$ and $A\cong \TC(-1,2)$. \qed
\end{lem}
If $U$ is $3$-dimensional, then $U$ is either $\TC(\al)$, $S(\dt)$ with $\dt\neq 2$ (denoted by $\text{Cl}^J(\F^2,b)$ in \cite{hall2015primitive}) or $\widehat{S}(2)^\circ$ (denoted by $\text{Cl}^{00}(\F^2,b)$ in \cite{hall2015primitive}).  

\begin{lem}
Let $U$ be $3$-dimensional, then $A=U$. Further, $U\cong \TC(\al)$ and $A\cong \TC(\al,1-\al)$ for $\al\neq -1$.
\proof Assume for a contradiction that $A\neq U$. Therefore $A$ is 4-dimensional with basis $\{a,b,c,\sg\}$. By Lemma \ref{eigen b}, we have
\begin{eqnarray*}
-\frac{P}{\bt}a +P b +c \in A_0(b)\text{ and } \bt a +\gm^f b +\sg \in A_\al(b).
\end{eqnarray*}
By the fusion law, the product of these two vectors will be in $A_\al(b)=\GenG{\bt a +\gm^fb +\sg}$ moreover the $c$ component is zero. We have
\begin{eqnarray*}
\left[-\frac{P}{\bt}a +P b +c \right]\left[\bt a +\gm^f b +\sg\right] &=& -Pa -\frac{P}{\bt}\gm^fab -\frac{P}{\bt}a\sg +P\bt ab\\
&+&P\gm^f b +Pb\sg +\bt ac +\gm^fbc +c\sg\\
&=&[...]a + [...]b+ [...]\sg\\
&+& \left[-\frac{1}{2}(\al-\bt)P+\bt^2+\dt^f\right]c.
\end{eqnarray*}
Thus
\[ \frac{1}{2}(\al-\bt)P =(\al-1)\gm^f.\]
However by Equation (\ref{proof3}), we get
\[\frac{1}{2}(\al-\bt)P =(\al-1)\gm^f  = \frac{1}{2}(1-\bt)P\]
which implies that $\al=1$. A contradiction and so $U=A$. 

The possible options for $U$ (and $A$) are $S(\dt)$ with $\dt\neq 2$, $\widehat{S}(2)^\circ$ and $\TC(\al)$. Let us look at each algebra separately.

Suppose $U\cong S(\dt)$. This is the same algebra in Section  3.5 of \cite{hall2015primitive}. It is shown that all non-trivial idempotents in $U$ have fusion law $\jor{\frac{1}{2}}$ and so $a$ must be a $\jor{\frac{1}{2}}$-axis in $A$. Thus $A_\bt(a)=\{0\}$ and the axet is not skew. 

Suppose $U\cong \widehat{S}(2)^\circ$. This is discussed in Section 5 of \cite{mcinroy2021forbidden}. We have that every non-trivial idempotent in $U$ is a $\jor{\frac{1}{2}}$-axis. As before, we get $A_\bt(a)=\{0\}$ and the axet cannot be skew. 

Therefore $U\cong \TC(\al)$. If $\al\neq-1$, then we get $A\cong \TC(\al,1-\al)$. If $\al= -1$, $A$ is not skew.  \qed
\end{lem}
As we have exhausted all the cases for $P=0$ and $P\neq 0$, this completes the proof of Theorem \ref{Theorem}.

