\section{Introduction}
Hall, Shpectorov and Rehren first defined axial algebras in \cite{hall2015primitive} and \cite{hall2015universal}. These are non-associative, commutative algebras which are strongly related to the $3$-transposition groups as well as some of the sporadic groups. They were inspired by Majorana algebras which were defined in \cite{ivanov2009monster} using work around the properties of the Griess algebra. In \cite{ivanov2010majorana}, the $2$-generated Majorana algebras were classified. The properties of Majorana algebras can be generalised even further to the other fusion laws, for example $\mon{\al,\bt}$, to produce what are called axial algebras.

To motivate why skew axial algebras are a topic of interest, we look at recent work of M\textsuperscript{c}Inroy and Shpectorov in \cite{mcinroy2021forbidden}. Their paper clarified a problem with shapes of an algebra. Shapes were defined from algebras however there are given shapes where no algebra exists, resulting in a circular argument. Their goal was to make the definition of shape independent of algebras to avoid this oxymoron. While tackling this challenge, they defined axets using group theoretic properties rather than through axial algebras. They noticed in a certain case that these axets are either regular or skew. As the known axial algebras are regular, they posed a question of if any axial algebra is skew? In this paper, we will answer such question.

As there are an infinite number of skew axets, we will be focusing on the smallest one, $X'(1+2)$. This axet does have axial algebra examples, which we will discuss later on. We will start by briefly introducing axial algebras and axets to the reader. This will not be a deep discussion and it will keep to the essentials. We will then give three examples of skew axial algebras. The first is $\TC(\al)$ when we assign a Monster fusion law to this algebra rather than the commonly assigned Jordan fusion law. For $\al=-1$, we change our approach to get a skew axial algebra with fusion law $\mon{-1,2}$. The second will be $Q_2(\frac{1}{3})$, an algebra which is related to Matsuo algebras. Again, this algebra is assigned a different fusion law $\mon{\frac{1}{3},\frac{2}{3}}$ rather than $\mon{\frac{2}{3},\frac{1}{3}}$. A third example, $Q_2(\frac{1}{3})^\times \oplus \GenG{\id}$ is also constructed in characteristic $5$. This is due to $Q_2(\frac{1}{3})$ being not simple in characteristic $5$ and it has a $3$-dimensional quotient, $Q_2(\frac{1}{3})^\times$. Adding a universal identity element, we get a new $2$-generated axial algebra which is also skew and with a Monster fusion law. This gives an answer to Problem 6.14 in \cite{mcinroy2022axial} however a complete classification is still in progress.

Convincing the reader that these skew algebras exist, the rest of the paper will be focused on classifying these algebras for skew axet $X'(1+2)$. This will be done by applying work from \cite{rehren2017generalised} and \cite{franchi20211} to produce a complete multiplication table. In the appendix, we will produce relations between constants, using GAP \cite{gap2022}, to make our lives easier. With this information, the proof is split into two parts:
\begin{itemize}
\item when two of the axes are orthogonal, and
\item when they are not orthogonal. 
\end{itemize}
When axes are orthogonal, this gives rise to double axes, which has been studied in multiple papers, for example \cite{joshi2020axial}. Looking at those parts separately, our work proves the following:
\begin{thm}\label{Theorem}
Let $\F$ be a field of characteristic not equal to $2$. Suppose $(A,X)$ is a $2$-generated primitive $\mon{\al,\bt}$-axial algebra over $\F$ and $\GenG{X}$ is isomorphic to $X'(1+2)$. Then either
\begin{enumerate}
\item[$1.$] $A \cong \TC(\al,1-\al)$ for $\al\neq \frac{1}{2}$ and $\al+\bt=1$, or
\item[$2.$] $(\al,\bt)=(\frac{1}{3},\frac{2}{3})$ and either
\begin{enumerate}
    \item[$i)$] $A\cong Q_2(\frac{1}{3},\frac{2}{3}$) if $\F$ has characteristic not equal to $5$, or
    \item[$ii)$] $A\cong Q_2(\frac{1}{3})^\times\oplus \GenG{\id}$ if $\F$ has characteristic equal to $5$.
\end{enumerate}
\end{enumerate}
\end{thm}
\begin{rem}
Adopting the notation in \cite{franchi20212}, we will denote the algebra $\TC(\al)$, $\al\neq -1$, with Monster fusion law $\mon{\al,1-\al}$ as $\TC(\al,1-\al)$ to avoid any confusion about which fusion law we have on this algebra. We denote $\TC(-1,2)$ to be $\TC(2)$ with $\mon{-1,2}$ fusion law.  Further, $Q_2(\frac{1}{3},\frac{2}{3})$ is $Q_2(\frac{1}{3})$ with a different fusion law to its definition in \cite{galt2021double}. The final algebra is defined in Table \ref{mult Qx}.
\end{rem}
\begin{note}
In all cases, $\al+\bt=1$. This is due to one of the axes in the axet being twisted by the identity element.
\end{note}
We will then prove the following corollary. 
\begin{cor}\label{Corollary}
Let $\F$ be a field of characteristic not equal to $2$. Suppose that $A=\GenA{p,q}$ is a primitive axial algebra over $\F$ such that $p$ is a $\mon{\al,\bt}$-axis and $q$ is a $\mcal{J}(\al)$-axis. Then $A$ is isomorphic to one of the following:
\begin{enumerate}
    \item[$1.$] an axial algebra of $\jor{\al}$-type, or
    \item[$2.$] an axial algebra listed in the Theorem $\ref{Theorem}$.
\end{enumerate}
\end{cor}

The final section will start the discussion on larger skew axets, $X'(k+2k)$, and how our result could help with such task. When $k$ is odd and choosing the right axes, there will be a subaxet of $X'(1+2)$ inside the larger skew axet. We will conclude by proposing a plan to tackle the odd $k$ case, which we hope to present in a secondary paper with work on even $k$.

The author would like to thank the anonymous referee for their useful comments and corrections.