\section{The Foundations}
This section will be working closely with \cite[Section 4]{franchi20211} and \cite[Section 3]{rehren2017generalised}, which will be cited accordingly. We will state some of the results in those papers however we will omit the proofs.

Fix $(A,X)$ to be an $\mon{\al,\bt}$-axial algebra with $X=\{a_0,a_1\}$. For $i\in\{0,1\}$, let $\tu{i}$ be the Miyamoto involution associated to $a_i$. Let $\GenG{X}$ be isomorphic to $X'(1+2)$ and without loss of generality, let $\tu{0}$  be a non-trivial involution and $\tu{1}$ be equal to the identity. Set $a_{2i}:=a_0^{\tu{0}^i}$ and $a_{2i+1}:=a_1^{\tu{0}^i}$ for $i\in \Z$. Hence, 
\begin{equation}\label{axes rel}
    a_{2i}=a_0,\;\; a_{4i+1}=a_1,\;\; a_{4i-1}=a_{-1}\; \text{for} \;i \in \Z.
\end{equation} 
For $i\in \Z$ and $m\in \N$, define 
\begin{equation}\label{s rel}
 s_{i,m}:=a_ia_{i+m}-\bt(a_i+a_{i+m}).   
\end{equation} Let $f$ be the semi-automorphism of $A$ such that $a_i^f=a_{1-i}$ for all $i\in \Z$. This map is well-defined by Corollary 3.8 in \cite{franchi20211}. This $f$ is called the \emph{flip} and we do not assume it is an automorphism of $A$. Notice that $f$ has order $2$, $s_{0,1}^f=s_{0,1}$ and $s_{0,2}^f=s_{1,2}$. Using Definition \ref{Proj}, we define \[\lm_1:=\lm_{a_0}(a_1), \; \; \lmf_1:=\lm_{a_1}(a_0), \; \; \lm_2:=\lm_{a_0}(a_2)=1, \; \;\text{and }
\lmf_2:=\lm_{a_1}(a_{-1}).\]
To ease notation, we define the following constants:
\[ \gm:=\bt-\lm_1,\; \ep:=(1-\al)\lm_1-\bt, \; \dt:=(1-\al)\lm_1+\bt(\al-\bt-1).\]
In a similar fashion,
\[ \gm^f:=\bt-\lmf_1, \; \ep^f:=(1-\al)\lmf_1-\bt, \; \dt^f:=(1-\al)\lmf_1+\bt(\al-\bt-1).\]
They have the following relations:
\[
(\al-1)\gm=\ep+\al\bt=\dt+\bt^2\;\text{ and } \;(\al-1)\gm^f=\ep^f+\al\bt=\dt^f+\bt^2.
\]

\begin{lem}\label{axes sigma}
We have 
\begin{eqnarray*}
a_0s_{0,1}&=&(\al-\bt)s_{0,1}+\dt a_0 +\frac{1}{2}\bt(\al-\bt)(a_1+a_{-1})\\
a_1 s_{0,1}&=& (\al-\bt)s_{0,1}+ \dt^f a_1+\bt(\al-\bt) a_0 \\
a_{-1} s_{0,1} &=& (\al-\bt)s_{0,1} + \dt^f a_{-1}+\bt(\al-\bt) a_0.
\end{eqnarray*}
\proof The first two equalities are part of Lemma 3.1 in \cite{rehren2017generalised}. 
By the second equality, 
\[(a_1s_{0,1})^{\tu{0}} = [(\al-\bt)s_{0,1}+ \dt^f a_{1}+\bt(\al-\bt) a_0 ]^{\tu{0}}\]
if and only if
\[ a_{-1}s_{0,1} = (\al-\bt)s_{0,1}+ \dt^f a_{-1}+\bt(\al-\bt) a_0 \]
due to $\tu{0}$ being an automorphism of $A$. \qed
\end{lem}
\begin{lem}\label{sigma2}
We have that $s_{1,2}$ is in the span of $\{a_0,a_1,a_{-1}, s_{0,1}\}$.
\proof Applying $f$ to the first equality of Lemma 4.7 in \cite{franchi20211}, we get:
\begin{eqnarray*}
(\al-2\bt)a_1s_{0,2}&=&\bt^2(\al-\bt)(a_3+a_{-1})+2\bt(\al-\bt)s_{1,2}\\
&+&\left[-2\al\bt\lmf_1+2\bt(1-\al)\lm_1\right.\\
&+&\left.\frac{\bt}{2}(4\al^2-2\al\bt+4\bt^2-\al-2\bt)\right](a_0+a_2)\\
&+&\frac{1}{(\al-\bt)}\left[(6\al^2-8\al\bt-2\al+4\bt)(\lmf_1)^2\right.\\
&+&\left. 2\al(\al-1)\lm_1\lmf_1+2\al(-2\al-2\bt+1)(\al-\bt)\lmf_1\right.\\
&-&\left.4\bt(\al-1)(\al-\bt)\lm_1-\al\bt(\al-\bt)\lmf_2\right.\\
&+&\left.2\bt(2\al^2+\bt^2-\al)(\al-\bt)\right]a_1\\
&+&\left[-4\al\lmf_1-4(\al-1)\lm_1\right.\\
&+&\left.(4\al^2-2\al\bt+4\bt^2-\al-2\bt)\right]s_{0,1}.
\end{eqnarray*}
Applying relations in $(\ref{axes rel})$ and the fact that $s_{0,2}=a_0a_2-\bt(a_0+a_2)=(1-2\bt)a_0$ we get
\begin{eqnarray*}
(\al-2\bt)(1-2\bt)a_1a_0&=&2\bt^2(\al-\bt)a_{-1}+2\bt(\al-\bt)s_{1,2}\\
&+&2\left[-2\al\bt\lmf_1+2\bt(1-\al)\lm_1\right.\\
&+&\left.\frac{\bt}{2}(4\al^2-2\al\bt+4\bt^2-\al-2\bt)\right]a_0\\
&+&\frac{1}{(\al-\bt)}\left[(6\al^2-8\al\bt-2\al+4\bt)(\lmf_1)^2\right.\\
&+&\left. 2\al(\al-1)\lm_1\lmf_1+2\al(-2\al-2\bt+1)(\al-\bt)\lmf_1\right.\\
&-&\left.4\bt(\al-1)(\al-\bt)\lm_1-\al\bt(\al-\bt)\lmf_2\right.\\
&+&\left.2\bt(2\al^2+\bt^2-\al)(\al-\bt)\right]a_1\\
&+&\left[-4\al\lmf_1-4(\al-1)\lm_1\right.\\
&+&\left.(4\al^2-2\al\bt+4\bt^2-\al-2\bt)\right]s_{0,1}.\\
\end{eqnarray*}
Equivalently,
\begin{eqnarray*}
-2\bt(\al-\bt)s_{1,2}&=&2\bt^2(\al-\bt)a_{-1}\\
&+&2\left[-2\al\bt\lmf_1+2\bt(1-\al)\lm_1\right.\\
&+&\left.\frac{\bt}{2}(4\al^2-2\al\bt+4\bt^2-\al-2\bt)\right]a_0\\
&+&\frac{1}{(\al-\bt)}\left[(6\al^2-8\al\bt-2\al+4\bt)(\lmf_1)^2\right.\\
&+&\left.2\al(\al-1)\lm_1\lmf_1+2\al(-2\al-2\bt+1)(\al-\bt)\lmf_1\right.\\
&-&\left.4\bt(\al-1)(\al-\bt)\lm_1-\al\bt(\al-\bt)\lmf_2\right.\\
&+&\left.2\bt(2\al^2+\bt^2-\al)(\al-\bt)\right]a_1\\
&+&\left[-4\al\lmf_1-4(\al-1)\lm_1\right.\\
&+&\left.(4\al^2-2\al\bt+4\bt^2-\al-2\bt)\right]s_{0,1}\\
&-&(\al-2\bt)(1-2\bt)\left[s_{0,1}+\bt(a_0+a_1)\right].
\end{eqnarray*}
Therefore
\[ s_{1,2}=Pa_0+Qa_1+Ra_{-1}+Ss_{0,1}\]
where 
\begin{eqnarray*}
P&:=& \frac{1}{(\al-\bt)}\left[2(\al-1)\lm_1+2\al\lmf_1+\al(1-2\al)\right],\\
Q&:=&-\frac{1}{2\bt(\al-\bt)^2}\left[(6\al^2-8\al\bt-2\al+4\bt)(\lmf_1)^2+2\al(\al-1)\lm_1\lmf_1\right.\\
&+&\left.2\al(-2\al-2\bt+1)(\al-\bt)\lmf_1-4\bt(\al-1)(\al-\bt)\lm_1\right.\\
&-&\left.\al\bt(\al-\bt)\lmf_2+2\bt(2\al^2+\bt^2-\al)(\al-\bt)\right.\\
&-&\left.\bt(\al-\bt)(\al-2\bt)(1-2\bt)\right],\\
R&:=&-\bt, \text{ and}\\
S&:=&\frac{P}{\bt}.
\end{eqnarray*}
\qed
\end{lem}
\begin{lem}\label{mult lemma}
With the above notation, $Q=R$ and $a_1a_{-1}=P(a_0+\frac{1}{\bt}\sg)$.
\end{lem}
\proof
Since $s_{1,2}$ is invariant under $\tu{0}$, notice that 
\[0=(s_{1,2}-s_{1,2}^{\tu{0}})=(Q-R)(a_1-a_{-1})\]
To avoid a contradiction, $Q=R$ and so
\[ s_{1,2}=P\left(a_0+\frac{1}{\bt}s_{0,1}\right)-\bt(a_1+a_{-1}).\]
Hence
\[a_1a_{-1}= s_{1,2}+\bt(a_1+a_{-1})=P\left(a_0+\frac{1}{\bt}s_{0,1}\right).\] \qed
\begin{prop}\label{span prop}
We have $A$ is linearly spanned by $\mcal{B}=\{a_{-1},a_0,a_1,s_{0,1}\}$ and so is at most $4$-dimensional.
\proof From above, multiplication of $\mcal{B}$ has been described and shown to be in $\Span{\mcal{B}}$ except for $s_{0,1}^2$. If $\al\neq 2\bt$,  $s_{0,1}^2 \in \Span{\mcal{B}}$ by Lemma 4.7 in \cite{franchi20211}. 
If $\al = 2\bt$, by Lemma 3.5 in \cite{franchi20212}, $s_{0,1}^2$ is computed. As $s_{0,3}=s_{0,1}$ by Equation $(\ref{s rel})$, $s_{0,1}^2\in \Span{\mcal{B}}$.  \qed
\end{prop}

We will now state  Theorem 4.1.1 in \cite{rehren2015axial} where $R$ is a field. This is a extremely useful result that will be used multiple times in our reasoning. 
\begin{prop}[Rehren]\label{Rehren Thm}
Let $\al,\bt\notin \{0,1\}$ be distinct values. Suppose that $V=\GenA{p,q}$ be an axial algebra such that $p$ is a $\mcal{J}(\al)$-axis and $q$ is a $\mcal{J}(\bt)$-axis. Then $V\cong \TB$ or $V\cong \TC(\al,1-\al)$.
\end{prop}

\begin{rem}
In Rehren's proof, he only shows algebra isomorphisms of $V\cong \TC(\al)$ or $V\cong \TC(2)$ if $\al=-1$. However, for $\al\neq-1$, it is clear that $V\cong \TC(\al,1-\al)$ due $(V,\{p,q\})$ satisfying a fusion law $\mon{\al,1-\al}$. Further, for $\al=-1$, it is easy to show that $V\cong \TC(-1,2)$. Again, this is clear since $(V,\{p,q\})$ satisfies a fusion law $\mon{-1,2}$.  
\end{rem}
