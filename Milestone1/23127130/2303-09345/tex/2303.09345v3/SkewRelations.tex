\section{Skew Relations}
To ease notation, set $a:=a_0$, $b:=a_1$, $c:=a_{-1}$, and $\sg:=s_{0,1}$. From Lemma \ref{axes sigma} and \ref{sigma2}, we have
\begin{eqnarray*}
a\sg&=&(\al-\bt)\sg+\dt a+\frac{1}{2}\bt(\al-\bt)(b+c),\\
b\sg&=& (\al-\bt)\sg+\bt(\al-\bt)a + \dt^f b,\\
c\sg&=& (\al-\bt)\sg+\bt(\al-\bt)a + \dt^f c,\\
 bc &=& P\left(a+\frac{1}{\bt}\sg\right).
\end{eqnarray*}
The last thing  to complete a multiplication table of $A$ is $\sg^2$. These expressions are in \cite{franchi20211} and \cite{franchi20212} for $\al\neq 2\bt$ and $\al=2\bt$ respectively. In this paper, we do not need the exact expression for $\sg^2$ and we simply write
\[ \sg^2 = \zt a + \te(b+c)+\kp \sg\]
with $\zt, \te, \kp \in \F$.
Notice that the $b$ and $c$ term are equal due to $\sg^2$ being invariant under $\tu{0}$ and Table \ref{mult table} gives us the complete multiplication of $A$.
\begin{table}[b]
\centering
\scalebox{0.9}{
    \begin{tabular}{c|c|c|c|c}
         & $a$ & $b$ & $c$ &$\sg$   \\
         \hline
        $a$ & $a$& $\bt a + \bt b + \sg$& $\bt a + \bt c+ \sg$& $\dt a+\frac{1}{2}\bt(\al-\bt)(b+c)+(\al-\bt)\sg$\\
        \hline
        $b$ & & $b$ & $P(a+\frac{1}{\bt}\sg)$ &$\bt(\al-\bt)a + \dt^f b+(\al-\bt)\sg$\\
        \hline
        $c$ &  &  & $c$& $\bt(\al-\bt)a + \dt^f c+(\al-\bt)\sg$\\
        \hline
        $\sg$ &  &  &  & $\zt a + \te(b+c)+\kp \sg$
    \end{tabular}}
\caption{The multiplication table of $A$}
\label{mult table}
\end{table}

The rest of this section will be producing results that will be used in the proof of Theorem \ref{Theorem}. First, we can calculate eigenspaces of $\text{ad}_a$ and $\text{ad}_b$.

\begin{lem}\label{eigen a}
The eigenspaces of $\text{ad}_a$ are the following:
\begin{enumerate}
    \item[$1$.] $\GenG{a}=A_1(a)$, 
    \item[$2$.] $\GenG{\ep a+\frac{1}{2}(\al-\bt)(b+c)-\sg}= A_0(a)$,
    \item[$3$.] $\GenG{\gm a +\frac{1}{2}\bt(b+c)+\sg}=A_\al(a)$, and
    \item[$4$.] $\GenG{b-c}= A_\bt(a)$.
\end{enumerate}
\proof See Lemma 4.4 in \cite{franchi20211}. \qed
\end{lem}
\begin{lem}\label{eigen b}
The eigenspaces of $\text{ad}_b$ are the following:
\begin{itemize}
    \item[$1$.] $\GenG{b}= A_1(b)$,
    \item[$2$.] $\GenG{-\frac{P}{\bt}a+Pb+c, (\al-\bt)a +\ep^fb-\sg}=A_0(b)$, and
    \item[$3$.] $\GenG{\bt a +\gm^fb+\sg}=A_\al(b)$.
\end{itemize}
\proof Using Table \ref{mult table}, the reader can check the generators are eigenvectors. As $A$ is at most $4$-dimensional, there are no more possible eigenvectors (up to linear combination). \qed 
\end{lem}
We can find certain relations from the conditions imposed by Lemma \ref{Seress}. To avoid focusing on how these relations are calculated, the reasoning is in the appendix for the reader to look at if they wish. 
\begin{lem}\label{skew rel}
We have that the following equations must hold:
\begin{equation}\label{proof1}
   \begin{split}
\lmf_2=-\frac{P}{\bt}\gm^f
\end{split} 
\end{equation} 
\begin{equation}\label{proof2}
    \begin{split}
\bt\dt =\frac{1}{2}\bt(\al-\bt)-\bt^2(\al-\bt)-(\al-2\bt)\dt^f.  
\end{split}
\end{equation} 
and
\begin{equation}\label{proof3}
\begin{split}
\frac{1}{2}(1-\bt)P=(\al-1)\gm^f.
\end{split}
\end{equation}

\end{lem}
\proof See the Appendix. \qed

