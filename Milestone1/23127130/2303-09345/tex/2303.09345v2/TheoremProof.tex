\section{Proof of the Theorem}
For this section, $U:=\GenA{b,c}$ is a $\jor{\al}$-axial algebra. The proof is split into three parts: $P=0$, $P\neq 0$ and $\al\neq \frac{1}{2}$, and $\al=\frac{1}{2}$. 
\subsection{Orthogonal axes}
This is the first case in Lemma \ref{skew rel} and one can check that $\al\neq \frac{1}{2}$. We have that $bc=0$ and so $U\cong 2B$. This does not tell us any underlying structure but we have that $d=b+c$ is a $\jor{2\al}$-axis in $A$. Note that this axis is not primitive in $A$ as $A_1(d)=\Span{b,c}$. Looking at the fixed subalgebra of $\tu{0}$, denoted by $F$, one can see that $a$, $d$, $\sg$ are in $F$. Hence $\text{dim}F\leq\text{dim}A - 1\leq 3$.

\begin{lem} We have $\al=\frac{1}{3}$, $F\cong 3C(\frac{1}{3},\frac{2}{3})$ and $A$ is $4$-dimensional.
\proof  One can show $F=\GenA{a,d}$ is a primitive axial algebra. We know $a$ satisfies $\jor{\al}$ fusion law and $d$ satisfies the $\jor{2\al}$ fusion law. As $\al\neq 2\al$, the $F\cong 2B$ or $F\cong 3C(\frac{1}{3},\frac{2}{3})$ by Rehren. Let us assume $F\cong 2B$. With $ad=0$, one gets $\sg =-\bt a -\frac{\bt}{2}(b+c)$. Therefore the $\al$-eigenvector is now
\[ \gm a +\frac{1}{2}\bt(b+c)+\sg = (\gm-\bt)a=-\lm_1 a.\]
To avoid contradiction, we let $\lm_1=0$ and $a$ is a $\jor{\bt}$-axis in $A$. Hence $A\cong 3C(\al,1-\al)$ and $\id$ exists. Since $bc=0$, we must have $c=\id -b$ but that would make $c$ a $\jor{\bt}$-axis in $A$, which would contradict our axet. Therefore $\al=\frac{1}{3}$. We have $F\cong 3C(\frac{1}{3},\frac{2}{3})$ or $F\cong 3C(\frac{2}{3},\frac{1}{3})$ but as they are isomorphic and we do not need the Miyamoto involution in this subalgebra, we can take either. Since $F$ is $3$-dimensional, $A$ has to be $4$-dimensional. \qed
\end{lem}

\begin{lem}
Let $P=0$. Then it is of $\mon{\frac{1}{3}, \frac{2}{3}}$-type and multiplication is shown in Table $\ref{mult 4dim}$.

\proof From the previous lemma, we have $F=\GenA{a,d}\cong 3C(\frac{1}{3}, \frac{2}{3})$. As $\al=\frac{1}{3}$, then substituting into $\al^2-2\al\bt+\frac{1}{2}\bt=0$, we have $\bt=\frac{2}{3}$. There are three $\mcal{J}(\frac{1}{3})$-axes in $F$, which are $a$ and two other elements, $e$ and $f$. Without loss of generality, let $e=\id -d$ where $\id=\frac{3}{4}(a+e+f)$. Therefore $\{a,d,f,b\}$ is a basis for $A$ and since $F\cong 3C(\frac{1}{3},\frac{2}{3})$, we have
\[
ad = \frac{1}{3}a +\frac{2}{3}d-\frac{1}{3}f, \; af =-\frac{1}{3}a +\frac{2}{3}d-\frac{1}{3}f, \; \text{and } df =-\frac{1}{3}a +\frac{2}{3}d+\frac{1}{3}f.
\]
Further,
\[ bd = b(b+c)=b+bc=b.\]
We have $ad =a(b+c)= 2\sg +\frac{4}{3} a +\frac{2}{3} d$, by the multiplication table of $A$, and so
\[ \sg =-\frac{1}{2}a -\frac{1}{6}f.\]
Hence 
\[ ab = \sg +\frac{2}{3} a +\frac{2}{3} b = \frac{1}{6}a-\frac{1}{6}f+\frac{2}{3} b.\] 
As $\lmf_1=\bt$, we have that $\frac{1}{6}a-\frac{2}{9} b +\frac{1}{6}f $ is a $0$-eigenvector of $b$ and so
\[
0 = b\left(a-\frac{4}{3}b +f\right) = \frac{1}{6}a -\frac{1}{6}f+\frac{2}{3} b -\frac{4}{3} b +bf\\
\]
which is equivalent to
\[
bf =-\frac{1}{6}a +\frac{1}{6}f+\frac{2}{3}b.
\]
Using $c=d-b$, we get the multiplication in Table \ref{mult 4dim}.\qed
\end{lem}
\begin{table}[h]
\begin{center}
    \begin{tabular}{c|c|c|c|c}
         & $b$ & $c$ & $a$ &$f$   \\
         \hline
        $b$ & $b$ &$0$& $\frac{2}{3}b+\frac{1}{6}a-\frac{1}{6}f$& $\frac{1}{3}b-\frac{1}{6}a+\frac{1}{6}f$\\
        \hline
        $c$ & & $c$ & $\frac{2}{3}c+\frac{1}{6}a-\frac{1}{6}f$& $\frac{2}{3}c-\frac{1}{6}a+\frac{1}{6}f$\\
        \hline
        $a$ &  &  & $a$&$\frac{2}{3}b+\frac{2}{3}c-\frac{1}{3}a-\frac{1}{3}f$ \\
        \hline
        $f$ &  &  &  & $f$
    \end{tabular}
       \caption{The multiplication table of $A$}
       \label{mult 4dim}
       \end{center}
\end{table}
\begin{prop}
Let $\F$ have characteristic not equal to $5$. Then $A\cong Q_2(\frac{1}{3},\frac{2}{3})$.
\proof We define the map $\varphi: A \rightarrow Q_2(\frac{1}{3},\frac{2}{3})$ with $\varphi(a)=t_1$, $\varphi(b)=s_1$, $\varphi(c)=s_2$, and $\varphi(f)=t_2$. The reader can check that this is an isomorphism. \qed
\end{prop}
\begin{prop}
Let $\F$ have characteristic equal to $5$. Then $A\cong Q_2(\frac{1}{3})^\times \oplus \GenG{\id}$. 
\proof We have the map $\phi: A \rightarrow Q_2(\frac{1}{3})\oplus \GenG{\id}$ with $\phi(a)=\id-z$, $\phi(b)=x$, $\phi(c)=y$, and $\phi(\id)=\id$. This is a straightforward check that $\phi$ is an isomorphism. \qed
\end{prop}





\subsection{Non-Orthogonal Case}
As $U$ is a axial algebra of $\jor{\al}$-type, $U$ is either $2B$, $3C(-1)^\times$ or $3C(\al)$.

\begin{lem}
We have $U\not\cong 2B$. 
\proof Let $U\cong 2B$. As $P\neq 0$ then $\sg = -\bt a$. Then $A$ is at most $3$-dimensional. Notice that
\[
ab = \bt b \text{ and } ac = \bt c
\]
Therefore $a$ is a $\jor{\bt}$-axis in $A$ and so by Rehren, $A\cong 3C(\al,1-\al)$. As $bc=0$ in $A$ and we must have $c=\id-b$. With the same reasoning as before, this contradicts our axet and so $U\not\cong 2B$. \qed
\end{lem}
\begin{lem}
If $U\cong 3C(-1)^\times$, then $A\cong 3C(-1,2)$.
\proof Suppose $U\cong 3C(-1)^\times$. We have $\al=-1$ and in both cases $P=2$. Further, $bc= -(b+c)$ and so we have that
\[ -(b+c)=bc = \frac{2}{\bt}(\bt a +\sg)\] 
Therefore $\sg = -\bt a -\frac{\bt}{2}(b+c)$ and $A$ is at most 3-dimensional. We have  
\[
ab = \frac{\bt}{2}b -\frac{\bt}{2} c\text{ and } ac = \frac{\bt}{2}c - \frac{\bt}{2} b
\]
Thus $a$ is $\jor{\bt}$-axis and by Rehren, we get $\bt=2$ and $A\cong 3C(-1,2)$. \qed
\end{lem}


\begin{lem}
If $U\cong 3C(\al)$, then $A\cong 3C(\al,1-\al)$ for $\al\neq -1$.
\proof Let $U\cong 3C(\al)$ and assume $A\neq U$. Therefore $A$ is 4-dimensional. We have that
\begin{eqnarray*}
-\frac{P}{\bt}a +P b +c \in A_0(b)\text{ and } \bt a +\gm^f b +\sg \in A_\al(b).
\end{eqnarray*}
By the fusion law properties of $b$ being an axis, we would like the product of these two vectors to be in $A_\al(b)$, that is, a multiple of $\bt a +\gm^f b + \sg$. Therefore,
\begin{eqnarray*}
\left[-\frac{P}{\bt}a +P b +c \right]\left[\bt a +\gm^f b +\sg\right] &=& -Pa -\frac{P}{\bt}\gm^fab -\frac{P}{\bt}a\sg +P\bt ab\\
&+&P\gm^f b +Pb\sg +\bt ac +\gm^fbc +c\sg\\
&=&[...]a + [...]b+ [...]\sg\\
&+& \left[-\frac{1}{2}(\al-\bt)P+\bt^2+\dt^f\right]c.
\end{eqnarray*}
Since we are assuming $\{a,b,c,\sg\}$ are linearly independent, we would like the $c$ component to be equal to zero, that is
\[ \frac{1}{2}(\al-\bt)P =(\al-1)\gm^f.\]
However by equation (\ref{proof3}) in the appendix, we get
\[\frac{1}{2}(\al-\bt)P =(\al-1)\gm^f  = \frac{1}{2}(1-\bt)P\]
which implies that $\al=1$. A contradiction and so $U=A$. Notice for $\al= -1$, $A\cong 3C(-1)$ is not skew.  \qed
\end{lem}
\subsection{Final Case}
We cannot use the structure theory as for $\al=\frac{1}{2}$ as we have infinite amount of possibilities for $U$. Luckily, we can use a different method by looking at an equation in \cite{franchi20211}. First notice that if $\al=\frac{1}{2}$, then it satisfies only the second case in Lemma \ref{skew rel}. Therefore if this algebra does exist, then we have
\[(\al,\bt,\lm_1,\lmf_1,\lmf_2,P)= \left(\frac{1}{2},\frac{1}{4}, \frac{1}{2},\frac{5}{8}, \frac{3}{4}, \frac{1}{2}\right).\]
\begin{lem}
    There does not exist any $\mon{\frac{1}{2},\frac{1}{4}}$-axial algebra with axet $X'(1+2)$.
    \proof Substituting $\al=2\bt$ into the first equation of Lemma 4.7 in \cite{franchi20211}, we have
    \[ 0= W a + Y(b+c)+ Z\sg  \]
    where 
    \begin{eqnarray*}
    Y&:=& 2\bt[-2\bt\lm_1+(1-2\bt)\lmf_1+\bt(4\bt-1)],\\
    W &:=& \frac{2}{\bt}\gm Y, \text{ and}\\
    Z&:=&\frac{2}{\bt}Y.
    \end{eqnarray*}
    Hence we have
    \[0=\frac{2}{\bt}Y\left(\gm a +\frac{1}{2}\bt (b+c)+\sg\right).\]
By substituting the values of $\bt$, $\lm_1$, $\lmf_1$ when $\al=\frac{1}{2}$, we get
\[ Y = \frac{1}{2}\left ( -\frac{1}{4}+\frac{5}{16} +0 \right) = \frac{1}{32}\neq 0.\]
Therefore $v=\gm a +\frac{1}{2}\bt (b+c)+\sg=0$. As $v$ spans $A_{\frac{1}{2}}(a)$, this eigenspace must be $0$-dimensional and so $a$ is a $\jor{\frac{1}{4}}$-axis in $A$. However by Rehren, $\frac{1}{2}+\frac{1}{4}=1$, which is an obvious contradiction in any characteristic.
    \qed
    
\end{lem}
