\section{Construction}
This paper will be working closely with section 4 in \cite{franchi20211} and section 3 in \cite{rehren2017generalised}, which will be cited accordingly. We will state some of the results in those papers however will omit the proofs.

Fix $(A,X)$ to be an $\mon{\al,\bt}$-axial algebra with $X=\{a_0,a_1\}$. For $i\in\{0,1\}$, let $\tu{i}$ to be the Miyamoto involution associated to $a_i$. Let $\GenG{X}$ to be isomorphic to $X'(1+2)$ and without loss of generality, let $\tu{0}$ to be an non-trivial involution and $\tu{1}$ to be equal to the identity. Set $a_{2i}:=a_0^{\tu{0}^i}$ and $a_{2i+1}:=a_1^{\tu{0}^i}$ for $i\in \Z$. Hence, 
\[ a_{2i}=a_0,\;\; a_{4i+1}=a_1,\;\; a_{4i-1}=a_{-1}.\]
For $i\in \Z$ and $m\in \N$, define $s_{i,m}=a_ia_{i+m}-\bt(a_i+a_{i+m})$. Let $f$ be the automorphism of the universal algebra such that $a_i^f=a_{1-i}$ for all $i\in \Z$. This $f$ is called the \emph{flip} and we do not assume it is an automorphism of $A$.  Further we define \[\lm_1:=\lm_{a_0}(a_1), \; \; \lmf_1:=\lm_{a_1}(a_0), \; \; \lm_2:=\lm_{a_0}(a_2)=1, \; \;\text{and }
\lmf_2:=\lm_{a_1}(a_{-1}).\]
To ease notation, we define the following constants:
\[ \gm:=\bt-\lm_1,\; \ep:=(1-\al)\lm_1-\bt, \; \dt:=(1-\al)\lm_1+\bt(\al-\bt-1).\]
In a similar fashion,
\[ \gm^f:=\bt-\lmf_1, \; \ep^f:=(1-\al)\lmf_1-\bt, \; \dt^f:=(1-\al)\lmf_1+\bt(\al-\bt-1).\]
They have the following relations:
\[
(\al-1)\gm=\ep+\al\bt=\dt+\bt^2\;\text{ and } \;(\al-1)\gm^f=\ep^f+\al\bt=\dt^f+\bt^2.
\]

\begin{lem}\label{axes sigma}
We have 
\begin{eqnarray*}
a_0s_{0,1}&=&(\al-\bt)s_{0,1}+\dt a_0 +\frac{1}{2}\bt(\al-\bt)(a_1+a_{-1})\\
a_1 s_{0,1}&=& (\al-\bt)s_{0,1}+\bt(\al-\bt) a_0 + \dt^f a_1\\
a_{-1} s_{0,1} &=& (\al-\bt)s_{0,1}+\bt(\al-\bt) a_0 + \dt^f a_{-1}
\end{eqnarray*}
\proof First two equalities are part of Lemma 3.1 in \cite{rehren2017generalised}. 
By the second equality, 
\[(a_1s_{0,1})^{\tu{0}} = [(\al-\bt)s_{0,1}+\bt(\al-\bt) a_0 + \dt^f a_{1}]^{\tu{0}}\]
if and only if
\[ a_{-1}s_{0,1} = (\al-\bt)s_{0,1}+\bt(\al-\bt) a_0 + \dt^f a_{-1}\]
due to $\tu{0}$ being an automorphism of $A$. \qed
\end{lem}
\begin{lem}\label{sigma2}
We have that $s_{1,2}$ is in the span of $\{a_0,a_1,a_{-1}, s_{0,1}\}$.
\proof Applying $f$ to the first equality of Lemma 4.7 in \cite{franchi20211}, we get:
\begin{eqnarray*}
(\al-2\bt)a_1s_{0,2}&=&\bt^2(\al-\bt)(a_3+a_{-1})\\
&+&\left[-2\al\bt\lmf_1+2\bt(1-\al)\lm_1\right.\\
&+&\left.\frac{\bt}{2}(4\al^2-2\al\bt+4\bt^2-\al-2\bt)\right](a_0+a_2)\\
&+&\frac{1}{(\al-\bt)}\left[(6\al^2-8\al\bt-2\al+4\bt)(\lmf_1)^2+2\al(\al-1)\lm_1\lmf_1\right.\\
&+&\left. 2\al(-2\al-2\bt+1)(\al-\bt)\lmf_1 - 4\bt(\al-1)(\al-\bt)\lm_1\right.\\
&-&\left.\al\bt(\al-\bt)\lmf_2+2\bt(2\al^2+\bt^2-\al)(\al-\bt)\right]a_1\\
&+&\left[-4\al\lmf_1-4(\al-1)\lm_1\right.\\
&+&\left.(4\al^2-2\al\bt+4\bt^2-\al-2\bt)\right]s_{0,1}\\
&+&2\bt(\al-\bt)s_{1,2}.
\end{eqnarray*}
Applying our relationships of $a_i$ and the fact that $s_{0,2}=a_0a_2-\bt(a_0+a_2)=(1-2\bt)a_0$ implies
\begin{eqnarray*}
(\al-2\bt)(1-2\bt)a_1a_0&=&2\bt^2(\al-\bt)a_{-1}\\
&+&2\left[-2\al\bt\lmf_1+2\bt(1-\al)\lm_1\right.\\
&+&\left.\frac{\bt}{2}(4\al^2-2\al\bt+4\bt^2-\al-2\bt)\right]a_0\\
&+&\frac{1}{(\al-\bt)}\left[(6\al^2-8\al\bt-2\al+4\bt)(\lmf_1)^2\right.\\
&+&\left. 2\al(\al-1)\lm_1\lmf_1+2\al(-2\al-2\bt+1)(\al-\bt)\lmf_1\right.\\
&-&\left.4\bt(\al-1)(\al-\bt)\lm_1-\al\bt(\al-\bt)\lmf_2\right.\\
&+&\left.2\bt(2\al^2+\bt^2-\al)(\al-\bt)\right]a_1\\
&+&\left[-4\al\lmf_1-4(\al-1)\lm_1\right.\\
&+&\left.(4\al^2-2\al\bt+4\bt^2-\al-2\bt)\right]s_{0,1}\\
&+&2\bt(\al-\bt)s_{1,2}.
\end{eqnarray*}
Equivalently,
\begin{eqnarray*}
-2\bt(\al-\bt)s_{1,2}&=&2\bt^2(\al-\bt)a_{-1}\\
&+&2\left[-2\al\bt\lmf_1+2\bt(1-\al)\lm_1\right.\\
&+&\left.\frac{\bt}{2}(4\al^2-2\al\bt+4\bt^2-\al-2\bt)\right]a_0\\
&+&\frac{1}{(\al-\bt)}\left[(6\al^2-8\al\bt-2\al+4\bt)(\lmf_1)^2\right.\\
&+&\left.2\al(\al-1)\lm_1\lmf_1+2\al(-2\al-2\bt+1)(\al-\bt)\lmf_1\right.\\
&-&\left.4\bt(\al-1)(\al-\bt)\lm_1-\al\bt(\al-\bt)\lmf_2\right.\\
&+&\left.2\bt(2\al^2+\bt^2-\al)(\al-\bt)\right]a_1\\
&+&\left[-4\al\lmf_1-4(\al-1)\lm_1\right.\\
&+&\left.(4\al^2-2\al\bt+4\bt^2-\al-2\bt)\right]s_{0,1}\\
&-&(\al-2\bt)(1-2\bt)\left[s_{0,1}+\bt(a_0+a_1)\right].
\end{eqnarray*}
Therefore
\[ s_{1,2}=Pa_0+Qa_1+Ra_{-1}+Ss_{0,1}\]
such that 
\begin{eqnarray*}
P&:=& \frac{1}{(\al-\bt)}\left[2(\al-1)\lm_1+2\al\lmf_1+\al(1-2\al)\right],\\
Q&:=&\frac{1}{(\al-\bt)}\left[(6\al^2-8\al\bt-2\al+4\bt)(\lmf_1)^2+2\al(\al-1)\lm_1\lmf_1\right.\\
&+&\left.2\al(-2\al-2\bt+1)(\al-\bt)\lmf_1-4\bt(\al-1)(\al-\bt)\lm_1\right.\\
&-&\left.\al\bt(\al-\bt)\lmf_2+2\bt(2\al^2+\bt^2-\al)(\al-\bt)-\bt(\al-2\bt)(1-2\bt)\right],\\
R&:=&-\bt, \text{ and}\\
S&:=&\frac{P}{\bt}.
\end{eqnarray*}
\qed
\end{lem}
\begin{rem}
Since $s_{1,2}$ is invariant under $\tu{0}$, notice that 
\[0=(s_{1,2}-s_{1,2}^{\tu{0}})=(Q-R)(a_1-a_{-1})\]
To avoid a contradiction, $Q=R$ and so
\[a_1a_{-1}= P\left(a_0+\frac{1}{\bt}s_{0,1}\right).\]
\end{rem}
\begin{prop}
$A$ is linearly spanned by the set $\mcal{B}=\{a_{-1},a_0,a_1,s_{0,1}\}$ and so is at most $4$-dimensional.
\proof From above, multiplication of $\mcal{B}$ has been described and show to be in $\Span{\mcal{B}}$ except for $s_{0,1}^2$. Further, we have $a_{-2}=a_2=a_0$, $s_{0,2}=(1-2\bt)a_0$ and $s_{1,2}\in \Span{\mcal{B}}$. 
If $\al=2\bt$, Lemma 4.7 in \cite{franchi20211} gives us $s_{0,1}^2 \in \Span{\mcal{B}}$ and so we are done. 
If $\al = 2\bt$, by Lemma 3.5 in \cite{franchi20212}, $s_{0,1}^2$ is computed. As $s_{0,3}=s_{0,1}$, $s_{0,1}^2\in \Span{\mcal{B}}$ and we are done.  \qed
\end{prop}
To ease notation, set $a:=a_0$, $b:=a_1$, $c:=a_{-1}$, and $\sg:=s_{0,1}$.
From Lemma 4.4 in \cite{franchi20211} and using constants defined earlier in this section, some eigenvectors for $a$ are
\begin{itemize}
    \item $a\in A_1(a)$, 
    \item $\ep a+\frac{1}{2}(\al-\bt)(b+c)-\sg \in A_0(a)$,
    \item$\gm a +\frac{1}{2}\bt(b+c)+\sg \in A_\al(a)$, and
    \item $b-c \in A_\bt(a)$.
\end{itemize}

We will now state  Theorem 4.1.1 in \cite{rehren2015axial} where $R$ is a field. This is an extremely useful result that will be used multiple times in our reasoning. 
\begin{prop}[Rehren]\label{Rehren Thm}
Let $\al,\bt\notin \{0,1\}$ be distinct values. Suppose that $V=\GenA{p,q}$ be a primitive axial algebra such that $p$ is a $\mcal{J}(\al)$-axis and $q$ is a $\mcal{J}(\bt)$-axis. Then either $V\cong 2B$ or $V\cong 3C(\al,1-\al)$. For the latter case, $\al$ is not to equal $-1$.
\end{prop}

\begin{rem}
    If we have the same conditions as above and $A=V$, then $A\not\cong 2B$. This is due to $2B$ having $\assoc$ fusion law which is trivially graded and so the Miyamoto involutions are equal to the identity. 
\end{rem}
