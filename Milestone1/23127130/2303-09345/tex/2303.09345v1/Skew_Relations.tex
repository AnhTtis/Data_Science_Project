\section{Skew Relations}
From Lemma \ref{axes sigma} and \ref{sigma2}, we have
\begin{eqnarray*}
a\sg&=&(\al-\bt)\sg+\dt a+\frac{1}{2}\bt(\al-\bt)(b+c),\\
b\sg&=& (\al-\bt)\sg+\bt(\al-\bt)a + \dt^f b,\\
c\sg&=& (\al-\bt)\sg+\bt(\al-\bt)a + \dt^f c,\\
 bc &=& P\left(a+\frac{1}{\bt}\sg\right).
\end{eqnarray*}
The last thing  to complete a multiplication table is $\sg^2$. These expressions are in \cite{franchi20211} and \cite{franchi20212} for $\al\neq 2\bt$ and $\al=2\bt$ respectively. This paper will not state those expressions but generalise it to
\[ \sg^2 = \zt a + \te(b+c)+\kp \sg.\]
Notice that the $b$ and $c$ term are equal due to $\sg^2$ is invariant under $\tu{0}$ and figure \ref{mult table} gives us the complete multiplication of $A$.
\begin{figure}[t]
\begin{center}
    \begin{tabular}{c|c|c|c|c}
         & $a$ & $b$ & $c$ &$\sg$   \\
         \hline
        $a$ & $a$& $\bt a + \bt b + \sg$& $\bt a + \bt c+ \sg$& $\dt a+\frac{1}{2}\bt(\al-\bt)(b+c)+(\al-\bt)\sg$\\
        \hline
        $b$ & & $b$ & $P(a+\frac{1}{\bt}\sg)$ &$\bt(\al-\bt)a + \dt^f b+(\al-\bt)\sg$\\
        \hline
        $c$ &  &  & $c$& $\bt(\al-\bt)a + \dt^f c+(\al-\bt)\sg$\\
        \hline
        $\sg$ &  &  &  & $\zt a + \te(b+c)+\kp \sg$
    \end{tabular}
\end{center}
\caption{The general multiplication table}
\label{mult table}
\end{figure}
One can calculate that some eigenvectors of $b$ are
\begin{itemize}
    \item $b \in A_1(b)$,
    \item $-\frac{P}{\bt}a+Pb+c \in A_0(b)$,
    \item $(\al-\bt)a +\ep^fb-\sg \in A_0(b)$, and
    \item $\bt a +\gm^fb+\sg \in A_\al(b)$.
\end{itemize}

We can find certain relations from looking at certain elements multiplied together. To avoid focusing on how these relations are calculated, the reasoning is in the appendix for the reader to look at if they wish. 
\begin{lem}\label{skew rel}
One of the following must hold:
  \begin{enumerate}
    \item[$1$.]$\al^2-2\al\bt+\frac{1}{2}\bt=0$ with $(\lm_1,\lmf_1,\lmf_2,P)=(\frac{\al(4\al^2-6\al+1)}{2(4\al-1)(\al-1)}, \bt, 0, 0)$
    \item[$2$.]$\al=2\bt$ with $(\lm_1,\lmf_1,\lmf_2,P)=(\frac{\bt(3-4\bt)}{2(1-2\bt)}, \frac{\bt+1}{2}, \frac{2\bt^2-3\bt+1}{2\bt}, 1-\al)$
    \item[$3$.]$\al+\bt=1$ with $(\lm_1,\lmf_1,\lmf_2,P)=(\frac{\bt+1}{2}, \frac{2-\bt}{2}, \frac{\al}{2}, \bt)$.
\end{enumerate}
\proof From the appendix, one can rearrange equation (\ref{proof2}) to get $\lm_1$ in terms of $\al$, $\bt$, and $\lmf_1$. Further, equation (\ref{proof1}) gives us $\lmf_2$ in terms of $\al$, $\bt$ and $\lmf_1$. Substituting these into equation (\ref{l2f}) and using GAP \cite{gap2022}, there are three possible cases:
\begin{enumerate}
\item $\lmf_1=\bt$,
\item $\al=2\bt$, or
\item $\al\bt+\frac{1}{2}\al-\bt-(2\al-1)\lmf_1=0$.
\end{enumerate}
For the first case,  $\lm_1$ can be expressed in terms of $\al$ and $\bt$ while $\lmf_2=0$ by equation (\ref{proof1}). Solving for equation (\ref{proof3}), one gets $\al^2-2\al\bt+\frac{1}{2}\bt=0$. In the second case, equation (\ref{proof3}) implies $\lmf_1=\frac{\bt+1}{2}$ and the rest follows. In the last case, note that if $\al=\frac{1}{2}$ then $\bt=\frac{1}{2}$ and so a contradiction. Therefore $\lmf_1$ can be expressed in terms of $\al$ and $\bt$. Substituting into equation (\ref{proof3}) gives us $\al+\bt=1$ and the rest follows. \qed
\end{lem}

