\section{Introduction}
Axial algebras are non-associative commutative algebras which were first defined in \cite{hall2015primitive} and \cite{hall2015universal}. Hall, Shpectorov and Rehren drew inspiration from previous work of Ivanov in \cite{ivanov2009monster} and his work with others in \cite{ivanov2010majorana} to produce this class of algebras. The most well-known example of these algebra is the Griess algebra, whose its automorphism group is the Monster group. The Griess algebra has been generalised to invoke certain properties of it and in turn has been related to Vertex Operator Algebras (VOAs) where the most famous one, nicknamed the Monster VOA, is used in Borcherds' proof of the Moonshine conjecture in \cite{borcherds1992monstrous}. The Griess algebra has a fusion law of $\mon{\frac{1}{4}, \frac{1}{32}}$ and this along with other properties have been generalised to algebras of fusion law $\mon{\al,\bt}$.

To motivate why skew axial algebras are a topic of interest, we will look at recent work of McInroy and Shpectorov in \cite{mcinroy2021forbidden}. This paper was created to clarify a problem with shapes of an algebra. Shapes were defined from algebras however there are given shapes where no algebra exists of that shape, resulting in a circular argument. Their goal was to make the definition of shape independent of algebras to avoid this oxymoron. While tackling this challenge, they defined axets using group theoretic properties rather than through axial algebras. Looking at closed axets, they were able to define shapes in a non-contradictory manner. As a result, they noticed and showed in a simple case, these axets are either regular or skew. As the common examples of axial algebras are regular, they posed a question of if any axial algebra is skew? In this paper, we will answer such question.

As there are an infinite number of skew axets, we will be focusing on the smallest one, $X'(1+2)$. This axet does have axial algebra examples, which we will discuss later on. We will start by briefly introducing axial algebras and axets to the reader. This will not be a deep discussion and will keep to the essentials as there are multiple sources for the reader to look at. We will then give two examples of skew axial algebras. The first is $3C(\al)$ when we assign a Monster fusion law to this algebra rather than the commonly assigned fusion law of $\jor{\al}$. The second will be $Q_2(\frac{1}{3})$, an algebra which is generated by a single and a double axis. This algebra has more in-depth discussion in \cite{galt2021double}. Again this algebra is assigned a different fusion law of $\mon{\frac{1}{3},\frac{2}{3}}$ rather than $\mon{\frac{2}{3},\frac{1}{3}}$. In fact, $Q_2(\frac{1}{3})$ gives an answer to Problem 6.14 in \cite{mcinroy2022axial} however a complete classification is a open problem.

Convincing the reader that these skew algebras exist, the rest of the paper will be focusing on classifying these algebras for skew axet $X'(1+2)$. This will be done by applying work from \cite{rehren2017generalised} and \cite{franchi20211} to produce a complete multiplication table. In the appendix, it is shown how these relations are calculated and GAP \cite{gap2022} is used to make our lives easier. With this information, the proof is split into three parts:
\begin{itemize}
\item when two of the axis are orthogonal,
\item when they are not orthogonal and $\al\neq \frac{1}{2}$, and
\item when $\al=\frac{1}{2}$. 
\end{itemize}
When axes are orthogonal, this gives rise to  double axes, an area that has been studied in \cite{joshi2020axial} and \cite{galt2021double}. Further when $\al=\frac{1}{2}$, we need to change our approach due to $\jor{\frac{1}{2}}$-axial algebras being an infinite family. This is from the classification of primitive axial algebras of Jordan type in \cite{hall2015primitive}.

Looking at those three parts, our work proves the following:
\begin{thm*}
Let $\F$ be a field of characteristic not equal to $2$ and $\al,\bt \notin \{0,1\}$. Suppose $(A,X)$ is a $2$-generated primitive $\mon{\al,\bt}$-axial algebra over $\F$ and $\GenG{X}$ is isomorphic to $X'(1+2)$. Then $A$ is either isomorphic to
\begin{enumerate}
\item[$1.$] $3C(\al,1-\al)$ where $\al+\bt=1$ and $\al\neq -1$, or
\item[$2.$] $Q_2(\frac{1}{3})$ where $(\al,\bt)=(\frac{1}{3},\frac{2}{3})$.
\end{enumerate}
\end{thm*}
\begin{rem}
Adopting the notation in \cite{franchi20212}, we will denote the algebra of $3C(\al)$ with Monster fusion law as $3C(\al,1-\al)$ to avoid any confusion in which fusion law we have on this algebra. Further, $Q_2(\frac{1}{3})$ has a different fusion law to its definition in \cite{galt2021double}.
\end{rem}
\begin{note}
In both cases, $\al+\bt=1$. This may be coincidence however this could be due to the construction of the algebra with one of the axes in the axet being changed by the identity element. 
\end{note}
We will then prove the following corollary. 
\begin{cor*}
Let $\F$ be a field of characteristic not equal to $2$ and $\al,\bt\notin \{0,1\}$. Suppose that $A=\GenA{p,q}$ is a primitive axial algebra over $\F$ such that $p$ is a $\mon{\al,\bt}$-axis and $q$ is a $\mcal{J}(\al)$-axis. Then $A$ is isomorphic to 
\begin{enumerate}
    \item[$1.$] an axial algebra of $\jor{\al}$-type,
    \item[$2.$] $3C(\al,1-\al)$, where $\al+\bt=1$ and $\al\neq -1$, or
    \item[$3.$] $Q_2(\frac{1}{3})$ with fusion law $\mon{\frac{1}{3},\frac{2}{3}}$.
\end{enumerate}
\end{cor*}

The final section will start the discussion on larger $k$ and how our result could help with such task. When $k$ is odd and choosing the right axes, there will be a subcase of $X'(1+2)$ inside the larger skew axet. We will conclude proposing a plan to tackle the odd $k$ case, which we hope to present in a secondary paper with work on even $k$.
\section*{Acknowledgements}
I would like to thank Professor Sergey Shpectorov for his guidance throughout my PhD studies so far and pushing me to complete this paper. I would also like to thank my family for their continuing support. 
