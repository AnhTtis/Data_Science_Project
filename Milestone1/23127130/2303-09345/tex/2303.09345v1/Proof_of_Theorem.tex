\section{Proof of the Theorem}
For this section, $U:=\GenA{b,c}$ is a primitive $\jor{\al}$-axial algebra. The proof is split into three parts: $P=0$, $P\neq 0$ and $\al\neq \frac{1}{2}$, and $\al=\frac{1}{2}$. 
\subsection{$P=0$}
This is the first case in Lemma \ref{skew rel} and one can check that $\al\neq \frac{1}{2}$. We have that $bc=0$ and so $U\cong 2B$. This does not tell us any underlying structure but we have that $d=b+c$ is a $\mon{2\al,\al}$-axis in $A$. Note that this axis is not primitive in $A$ as $A_1(d)=\Span{b,c}$. Looking at the fixed subalgebra of $\Miy{X}$, denoted by $F$, one can see that $a$, $d$, $\sg$ are in $F$. Hence $\text{dim}F\leq\text{dim}A - 1\leq 3$.

\begin{lem} We have $F\cong 3C(\frac{1}{3},\frac{2}{3})$.
\proof  One can show $F=\GenA{a,d}$ is a primitive axial algebra. We know $a$ satisfies $\mcal{F}\subseteq \jor{\al}$ fusion law and $d$ satisfies the $\jor{2\al}$ fusion law. By Rehren and Theorem 1.1 in \cite{hall2015primitive}, $F\cong 3C(\al,1-\al)$ or $F$ is of Jordan type.

Note that $\jor{\al}\neq\jor{2\al}$ for any $\al$ and so $F\not\cong 3C(\al)$. If $\al\not\in \mcal{F}$, then $a$ is a $\jor{\bt}$-axis in $A$ and so $A\cong 3C(\al,1-\al)$. As $bc=0$, we must have $c=\id -b$ and $c$ is $\jor{1-\al}$-axis. But no Miyamoto involution of $3C(\al,1-\al)$ switches $\jor{\al}$ and $\jor{1-\al}$ axes. Therefore a contradiction. 

We must have that $F\cong 3C(\al,1-\al)$ or $F\cong 3C(-1)^\times$. Suppose the latter and we have that $\lm_a(d)=\lm_a(b)+\lm_a(c)=2\lm_1$. Using the formula in Lemma \ref{skew rel}, $\lm_1=-\frac{11}{20}$ and so $\lm_a(d)=-\frac{22}{20}$.  However in Proposition 4.8 in \cite{hall2015primitive}, $\lm_a(d)=-\frac{1}{2}$ which is a contradiction. Therefore $F\cong 3C(\al,1-\al)$ and by Rehren, $2\al+\al=1$ and so we are done.\qed
\end{lem}
\begin{rem}
One may notice that for $F\cong 3C(\frac{1}{3},\frac{2}{3})$, $\lm_a(d)=\frac{5}{6}\neq \frac{1}{6}$. This is because $F$ is not $\jor{\frac{1}{3}}$-type and so the result in \cite{hall2015primitive} cannot be applied with any meaning. 
\end{rem}
As $F$ is $3$-dimensional, we must have that $A$ is $4$-dimensional. 

\begin{lem}
Suppose $A$ be a $4$-dimensional and $P=0$. Then it is of $\mon{\frac{1}{3}, \frac{2}{3}}$-type and multiplication can be explicitly calculated.

\proof From the working above, we have that $d=b+c$, and $F=\GenA{a,d}\cong 3C(\frac{1}{3}, \frac{2}{3})$. As we know $\al=\frac{1}{3}$ then substituting into $\al^2-2\al\bt+\frac{1}{2}\bt=0$, we have $\bt=\frac{2}{3}$. There are three idempotents in $F$ which are $\mcal{J}(\frac{1}{3})$-axes and they are $a$ and two other elements, $e$ and $f$. Without loss of generality, let $e=\id -d$ where $\id=\frac{3}{4}(a+e+f)$. Therefore $\{a,d,f,b\}$ is a basis for $A$ and since $F\cong 3C(\frac{1}{3},\frac{2}{3})$, we have
\[
ad = \frac{1}{3}a +\frac{2}{3}d-\frac{1}{3}f, \; af =-\frac{1}{3}a +\frac{2}{3}d-\frac{1}{3}f, \; \text{and } df =-\frac{1}{3}a +\frac{2}{3}d+\frac{1}{3}f.
\]
Further,
\[ bd = b(b+c)=b+bc=b.\]
However $ad =a(b+c)= 2\sg +\frac{4}{3} a +\frac{2}{3} d$ by the multiplication table. Rearranging for $\sg$, we get
\[ \sg =-\frac{1}{2}a -\frac{1}{6}f.\]
Hence 
\[ ab = \sg +\frac{2}{3} a +\frac{2}{3} b = \frac{1}{6}a-\frac{1}{6}f+\frac{2}{3} b.\] 
As $\lmf_1=\bt$, we have that $\frac{1}{6}a-\frac{2}{9} b +\frac{1}{6}f $ is a $0$-eigenvector of $b$ and so
\[
0 = b\left(a-\frac{4}{3}b +f\right) = \frac{1}{6}a -\frac{1}{6}f+\frac{2}{3} b -\frac{4}{3} b +bf\\
\]
which is equivalent to
\[
bf =-\frac{1}{6}a +\frac{1}{6}f+\frac{2}{3}b.
\]
\qed
\end{lem}
\begin{prop}
Let $A$ be the $4$-dimensional algebra defined as above. Then $A\cong Q_2(\frac{1}{3})$.
\proof We will show this by relabelling axes of $A$. Let $s_1=b$, $s_2=c$, $d_1=\id - a$, $d_2=\id -f$. These are all axes of $A$ since $a$ and $f$ are axes and so $\id-a$ and $\id-f$ are too. Further these four elements span the whole algebra. The reader can check that this produces the same multiplication table of $Q_2(\frac{1}{3})$ which is table \ref{mult Q}. Hence $A\cong Q_2(\frac{1}{3})$. \qed

\end{prop}


\subsection{$P\neq 0$ and $\al\neq \frac{1}{2}$}
As $U$ is a primitive axial algebra of $\jor{\al}$-type, $U$ is either $2B$, $3C(-1)^\times$ or $3C(\al)$.

\begin{lem}
We have $U\not\cong 2B$. 
\proof Let $U\cong 2B$. As $P\neq 0$ then $\sg = -\bt a$. Then $A$ is at most $3$-dimensional. Notice that
\[
ab = \bt b \text{ and } ac = \bt c
\]
Therefore $a$ is a $\jor{\bt}$-axis in $A$ and so by Rehren, $A\cong 3C(\al,1-\al)$. As $bc=0$ in $A$ and we must have $c=\id-b$. With the same reasoning as before, this produces a problem due to $\tu{a}$ switching a $\jor{\al}$-axis with a $\jor{1-\al}$-axis. This cannot happen and so $U\not\cong 2B$. \qed
\end{lem}
\begin{lem}
We have $U\not\cong 3C(-1)^\times$.
\proof Suppose $U\cong 3C(-1)^\times$. We have $\al=-1$ and in both cases $P=2$. Further, $bc= -(b+c)$ and so we have that
\[ -(b+c)=bc = \frac{2}{\bt}(\bt a +\sg)\] 
Therefore $\sg = -\bt a -\frac{\bt}{2}(b+c)$ and $A$ is at most 3-dimensional. We have  
\[
ab = \frac{\bt}{2}b -\frac{\bt}{2} c\text{ and } ac = \frac{\bt}{2}c - \frac{\bt}{2} b
\]
One can show that $a$ is $\jor{\bt}$-axis and by Rehren, we get $A\cong 2B$ due to $\al=-1$, which gives us a contradiction. \qed
\end{lem}


\begin{lem}
If $U\cong 3C(\al)$, then $A\cong 3C(\al,1-\al)$.
\proof Let $U\cong 3C(\al)$ and so if $A\neq U$, then $A$ is 4-dimensional. We have that
\begin{eqnarray*}
-\frac{P}{\bt}a +P b +c \in A_0(b)\text{ and } \bt a +\gm^f b +\sg \in A_\al(b).
\end{eqnarray*}
By the fusion law properties of $b$ being an axis, we would like the product of these two vectors to be in $A_\al(b)$, that is, a multiple of $\bt a +\gm^f b + \sg$. Therefore,
\begin{eqnarray*}
\left[-\frac{P}{\bt}a +P b +c \right]\left[\bt a +\gm^f b +\sg\right] &=& -Pa -\frac{P}{\bt}\gm^fab -\frac{P}{\bt}a\sg +P\bt ab\\
&+&P\gm^f b +Pb\sg +\bt ac +\gm^fbc +c\sg\\
&=&[...]a + [...]b+ [...]\sg\\
&+& \left[-\frac{1}{2}(\al-\bt)P+\bt^2+\dt^f\right]c.
\end{eqnarray*}
Since we are assuming $\{a,b,c,\sg\}$ are linearly independent, we would like the $c$ component to be equal to zero, that is
\[ \frac{1}{2}(\al-\bt)P =(\al-1)\gm^f.\]
However by equation (\ref{proof3}) in the appendix, we get
\[\frac{1}{2}(\al-\bt)P =(\al-1)\gm^f  = \frac{1}{2}(1-\bt)P\]
which implies that $\al=1$. A contradiction. Hence $U=A$\qed
\end{lem}
\subsection{$\al=\frac{1}{2}$}
For this case, we cannot use the structure theory as for $\al=\frac{1}{2}$ as we have infinite amount of possibilities for $U$. Luckily, we can use a different method by looking at an equation in \cite{franchi20211}. First notice that if $\al=\frac{1}{2}$, then it satisfies only the second case in Lemma \ref{skew rel}. Therefore if this algebra does exist, then we have
\[(\al,\bt,\lm_1,\lmf_1,\lmf_2,P)= \left(\frac{1}{2},\frac{1}{4}, \frac{1}{2},\frac{5}{8}, \frac{3}{4}, \frac{1}{2}\right).\]
\begin{lem}
    There does not exist any $\mon{\frac{1}{2},\frac{1}{4}}$-axial algebra with axet $X'(1+2)$.
    \proof Substituting $\al=2\bt$ into the first equation of Lemma 4.7 in \cite{franchi20211}, we have
    \[ 0= W a + Y(b+c)+ Z\sg  \]
    where 
    \begin{eqnarray*}
    Y&:=& 2\bt[-2\bt\lm_1+(1-2\bt)\lmf_1+\bt(4\bt-1)],\\
    X &:=& \frac{2}{\bt}\gm Y, \text{ and}\\
    Z&:=&\frac{2}{\bt}Y.
    \end{eqnarray*}
    Hence we have
    \[0=\frac{2}{\bt}Y\left(\gm a +\frac{1}{2}\bt (b+c)+\sg\right).\]
We either have $Y=0$ or $\gm a +\frac{1}{2}\bt (b+c)+\sg=0$ and by substituting the values of $\bt$, $\lm_1$, $\lmf_1$ when $\al=\frac{1}{2}$, we get
\[ Y = \frac{1}{2}\left ( -\frac{1}{4}+\frac{5}{16} +0 \right) = \frac{1}{32}\neq 0.\]
Therefore $v=\gm a +\frac{1}{2}\bt (b+c)+\sg=0$. As $v$ spans $A_{\frac{1}{2}}(a)$, this eigenspace must be $0$-dimensional and so $a$ is a primitive $\jor{\frac{1}{4}}$-axis in $A$. However by Rehren, $\frac{1}{2}+\frac{1}{4}=1$, which is an obvious contradiction.
    \qed
    
\end{lem}
