\usepackage{fullpage}
\usepackage[utf8]{inputenc}
\usepackage[czech,english]{babel}

%% Prefer Latin Modern fonts
\usepackage{lmodern}

%% Further useful packages (included in most LaTeX distributions)
\usepackage{amsmath}        % extensions for typesetting of math
\usepackage{amsfonts}       % math fonts
\usepackage{amssymb}
\usepackage{amsthm}         % theorems, definitions, etc.
\usepackage{bbding}         % various symbols (squares, asterisks, scissors, ...)
\usepackage{bm}             % boldface symbols (\bm)
\usepackage{graphicx}       % embedding of pictures
\usepackage{subcaption}
\usepackage{fancyvrb}       % improved verbatim environment
\usepackage[nottoc]{tocbibind} % makes sure that bibliography and the lists
			    % of figures/tables are included in the table
			    % of contents
\usepackage{dcolumn}        % improved alignment of table columns
\usepackage{booktabs}       % improved horizontal lines in tables
\usepackage{paralist}       % improved enumerate and itemize
\usepackage{xcolor}         % typesetting in color
\usepackage{tikz-cd}

\usepackage{enumitem}
\usepackage{placeins}

\allowdisplaybreaks
\overfullrule=1mm


\newcounter{cptTh}
\setcounter{cptTh}{0}
\newtheorem{theorem}[cptTh]{Theorem}
\newtheorem{proposition}[cptTh]{Proposition}
\newtheorem{lemma}[cptTh]{Lemma}
\newtheorem{property}[cptTh]{Property}
\newtheorem{corollary}[cptTh]{Corollary}
\newtheorem{conjecture}[cptTh]{Conjecture}
\newtheorem{example}[cptTh]{Example}
\newtheorem{problem}[cptTh]{Problem}
\newtheorem{dfn}[cptTh]{Definition}
\newtheorem{definition}[cptTh]{Definition}
\newtheorem{observation}[cptTh]{Observation}
\newtheorem{obs}[cptTh]{Observation}
\newtheorem{notation}[cptTh]{Notation}
\newtheorem{fact}[cptTh]{Fact}


\usepackage[linesnumbered,longend,noline,figure]{algorithm2e}
\SetKwProg{Fn}{Function}{}{end}
\DontPrintSemicolon

%%% An environment for typesetting of program code and input/output
%%% of programs. (Requires the fancyvrb package -- fancy verbatim.)

\DefineVerbatimEnvironment{code}{Verbatim}{fontsize=\small, frame=single}

%%% The field of all real and natural numbers
\newcommand{\R}{\mathbb{R}}
\newcommand{\N}{\mathbb{N}}

%%% Useful operators for statistics and probability
\DeclareMathOperator{\pr}{\textsf{P}}
\DeclareMathOperator{\E}{\textsf{E}\,}
\DeclareMathOperator{\var}{\textrm{var}}
\DeclareMathOperator{\sd}{\textrm{sd}}

%%% Transposition of a vector/matrix
\newcommand{\T}[1]{#1^\top}

%%% Various math goodies
\newcommand{\goto}{\rightarrow}
\newcommand{\gotop}{\stackrel{P}{\longrightarrow}}
\newcommand{\maon}[1]{o(n^{#1})}
\newcommand{\abs}[1]{\left|{#1}\right|}
\newcommand{\dint}{\int_0^\tau\!\!\int_0^\tau}
\newcommand{\isqr}[1]{\frac{1}{\sqrt{#1}}}

%%% Various table goodies
\newcommand{\pulrad}[1]{\raisebox{1.5ex}[0pt]{#1}}
\newcommand{\mc}[1]{\multicolumn{1}{c}{#1}}

\input{unicode.tex}

\def\Z{{\mathbb Z}}
\def\Q{{\mathbb Q}}
\def\R{{\mathbb R}}
\def\C{{\mathbb C}}
\def\T{{\mathbb T}}
\def\E{{\mathbb E}}
\def\set#1{\left\{#1\right\}}
\def\br#1{\left(#1\right)}
\def\abs#1{\left|#1\right|}
\def\rng#1{\left[#1\right]}
\def\norm#1{\left\|#1\right\|}
\def\Cay{{\rm Cay}}

\def\ceil#1{\left\lceil#1\right\rceil}

\newcommand{\eg}{e.g\., \ignorespaces}
\newcommand{\ie}{i.e\., \ignorespaces}

\def\NULL{\ensuremath{\text{\tt NULL}}}


%%% TODO
\newcounter{todo}
\DeclareRobustCommand{\TODO}[1]{%
\refstepcounter{todo}{\color{red}{\bf TODO} #1}%
\addcontentsline{tod}{section}{{\thetodo}~TODO #1}%
}%
\DeclareRobustCommand{\FIXME}[1]{%
\refstepcounter{todo}{\color{red}{\bf FIXME} #1}%
\addcontentsline{tod}{section}{{\thetodo}~FIXME #1}%
}%
\makeatletter
\newcommand\todoname{todo}
\newcommand\listtodoname{TODO seznam}
\newcommand\listoftodos{%
  \ifnum\thetodo=0\else%
  \section*{\listtodoname}\@starttoc{tod}%
  \addcontentsline{toc}{section}{\listtodoname}%
  \fi}%

\usepackage[listformat=simple]{caption}

\def\ext@table{lof}
\let\c@table\c@figure
\let\ftype@table\ftype@figure

\usepackage{calc}
\long\def\imagetop#1{\raisebox{12pt-\height}{#1}}

\def\doi#1{doi: \href{http://dx.doi.org/#1}{\nolinkurl{#1}}}

\def\Caption#1#2{\caption[#1]{#1 #2}}

\def\tabname{\kern -0.5pt T\kern-1.2pt a\kern -0.5pt b\.\kern-1pt}
\DeclareCaptionListFormat{fig}{\hbox to 0pt{\if\IsAlg TAlg\else Fig\fi\.\hss}\hphantom{\tabname} #2}
\DeclareCaptionListFormat{tab}{\hbox to 0pt{\tabname\hss}\hphantom{\tabname} #2}

\def\IsAlg{F}
\let\realalgorithm\algorithm
\def\algorithm{
  \def\IsAlg{T}%
  \def\figurename{Algorithm}%
  \realalgorithm}

\let\realtable\table
\def\table{\captionsetup{listformat=tab}\realtable}

\let\realfigure\figure
\def\figure{\captionsetup{listformat=fig}\realfigure}

\def\l@figure{\@dottedtocline{1}{0em}{4.2em}}
\let\l@table\l@figure

\makeatother

% fix footnotes
\let\realfootnoterule\footnoterule
\def\footnoterule{\vfill\realfootnoterule}

\usepackage{pgfplots}
\usepackage[colorlinks=true]{hyperref}
\def\.{\mbox{.}}

