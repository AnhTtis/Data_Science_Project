


\section{Tweet frequency and pattern analysis}
\label{sec:prolificacy_analysis}
Impactful Twitter profiles post a significant number of tweets over time. In this section, we shall investigate the number of total and  unique (tweets with the exact same content were removed) tweets, and the percentage of unique tweets posted by toxic 1\% and baseline profiles. We also consider the tweeting pattern as a measurable trait. It reveals the longitudinal nature of a profile's posting behavior. 


\subsection{Tweet Frequency}
\label{subsec:tweeting_frequency_analysis}
\subsubsection{Are toxic profiles prolific?}
In order to uncover the answer, we first note the total number of tweets of toxic 1\% and baseline profiles. 
Unique tweets (repetitions removed) were further counted to observe whether profiles repeatedly repost the same tweet. 
We present a CDF with the number of total and unique tweets in Fig.~\ref{fig:tweet_stats}. From this figure, we observe that 80\% of toxic profiles in our dataset post more than 1,000, as compared to 82\% in baseline profiles. It is interesting to observe that half of the toxic profiles at most tweeted 500 unique tweets and half of the baseline profiles at most posted 1500 unique tweets showing that toxic profiles almost tweet equal to baseline but post fewer original tweets. We note that base 40\% of our toxic and 20\% of baseline profiles post retweets less than 100 times.
Fig.~\ref{fig:tweet_stats_perc_uni} details the repetitive behavior of our profiles, it is evident that half of the toxic profiles in our dataset posted at most 55\% unique tweets (no repeats) and this is true for only a quarter of baseline profiles. Interestingly, a larger proportion of profiles in the toxic set occupy lower percentages of unique tweets, before crossing over with the baseline at 77\% unique tweets. At the higher percentages of unique tweets, the baseline increases gradually, whilst the bulk of the remaining toxic profiles have near 100\% unique tweets, with 25\% of toxic 1\% profiles posting more than 95\% unique tweets compared to only 15\% baseline profiles and 15\% of toxic 1\% profiles with no repetition vs. 7\%. Thus there is the occurrence of toxic profiles that repeatedly re-post the same toxic message, and profiles with individually crafted tweets containing toxicity. We report that the median toxicity of re-posted tweets is 0.47 for the toxic and 0.32 for the baseline.

%
\subsubsection{\bf Takeaway:}
Toxic profiles are comparable to general Twitter profiles in the total number of posted tweets but they retweet less than the baseline profiles. Notably, about 20\% of toxic profiles have a much higher proportion of unique tweets than the baseline. 



\begin{figure}[t]
\centering
% \vspace{-0.4cm}
\subfloat[Tweets count.]{
 \includegraphics[width=0.49\columnwidth]{figs/tweet_stats.pdf}\label{fig:tweet_stats}
}%\qquad
\subfloat[Percentage of unique tweets.]{
\includegraphics[width=0.48\columnwidth]{figs/tweet_stats_perc_uni_tweets.pdf}\label{fig:tweet_stats_perc_uni}
}
\vspace{-2mm}
\caption{\small Tweeting frequency of toxic and baseline profiles (\S\ref{subsec:tweeting_frequency_analysis}). In \protect\subref{fig:tweet_stats}, T, RT, and UT, respectively, represent the number of Tweets, retweets (RT), and unique tweets per profile.}
    \vspace{-4mm}
\end{figure}




\begin{figure*}[!t]
\centering
    \begin{subfigure}[t]{0.39\linewidth}
    \includegraphics[width=\linewidth]{figs/delta_sec_dual.pdf}
    \vspace{-8mm}
    \caption{{\small Seconds between consecutive tweets}}
    \label{fig:delta_combined_sec}
    \end{subfigure}
    \begin{subfigure}[t]{0.39\linewidth}
    \includegraphics[width=\linewidth]{figs/delta_days_dual.pdf}
    \vspace{-8mm}
    \caption{{\small Days between consecutive tweets}}
    \label{fig:delta_combined_days}
    \end{subfigure}
    \begin{subfigure}[t]{0.19\linewidth}
    \includegraphics[width=\linewidth]{figs/burstiness_scores.pdf}
    \vspace{-8mm}
    \caption{{\small Burstiness Score}}
    \label{fig:burstiness_scores}
    \end{subfigure}
    \vspace{-0.1cm}
    \caption{\small Tweeting pattern of toxic and baseline profiles in time between sequential tweets, and burstiness in time (cf. \S\ref{subsec:tweeting_pattern}).}
    \vspace{-0.5mm}
\end{figure*}