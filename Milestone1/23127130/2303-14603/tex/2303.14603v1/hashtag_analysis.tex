
\subsection{Hashtags} 
\label{subsec:hashtag_analysis}
Adding hashtags to tweets is a popular and easy way for users to convey a message to an interested audience, and to have a voice within intended communities.
We compare the tendency of sharing hashtags between the toxic and baseline profiles. 
\subsubsection{Do toxic profiles take the help of hashtags in their tweets?}
Hashtags place your message within the context of a topic or community.
First, with the total number of hashtags per profile (including repetitions), we discern from Fig.~\ref{fig:no_total_hts} that 50\% of the toxic profiles at least shared 300 hashtags in total but 50\% of baseline profiles shared many more: at least 1150 hashtags. 
Next, on the number of unique hashtags per profile (discounting repetitions), we see from Fig.~\ref{fig:no_unique_hts} that half of the toxic profiles at most shared about 15 unique hashtags, and half of the general profiles shared at most 100 unique hashtags. 
As such it is evident that the toxic users are using hashtags less than the baseline.
Next, Fig.~\ref{fig:no_unique_hts} shows us that the toxic profiles also share fewer unique hashtags than the baseline. Fig.~\ref{fig:p_unique_hts_combined} tells us that half of the toxic profiles at most shared 50\% unique hashtags as compared to 60\% of baseline. Jaccard similarity Fig.~\ref{fig:jaccard_similarity_hts}
of the hashtags shows us the strongest similarity amongst the hashtags of toxic profiles again, pointing to their narrow focus as observed through a small number of unique categories of SLDs in \S\ref{subsec:sld_similarity}.

\begin{figure}[th]
\centering
    \begin{subfigure}[t]{0.46\linewidth}
    \includegraphics[width=0.95\linewidth]{figs/no_total_hashtags.pdf}
    \vspace{-0.1mm}
    \caption{{\small \#Hashtags }}
    \label{fig:no_total_hts}
    \end{subfigure}
    \hfill
    \begin{subfigure}[t]{0.46\linewidth}
    \includegraphics[width=0.95\linewidth]{figs/no_unique_hashtags.pdf}
    \vspace{-0.1mm}
    \caption{{\small\#Unique hashtags}}
    \label{fig:no_unique_hts}
    \end{subfigure}
    \begin{subfigure}[t]{0.46\linewidth}
    \includegraphics[width=0.95\linewidth]{figs/p_unique_hashtags_combined_cdf.pdf}
    \vspace{-0.2mm}
    \caption{{\small \% Unique hashtags}}
    \label{fig:p_unique_hts_combined}
    \end{subfigure}
    \hfill
    \begin{subfigure}[t]{0.46\linewidth}
    \includegraphics[width=0.95\linewidth]{figs/jaccard_similarity_hashtags.pdf}
    \vspace{-0.2mm}
    \caption{{\small Jaccard Similarity (Htags)}}
    \label{fig:jaccard_similarity_hts}
    \end{subfigure}
    \vspace{-2mm}
    \caption{\small Hashtag analysis of toxic and baseline profiles (cf. \S\ref{subsec:hashtag_analysis}).}
    \vspace{-4mm}
\end{figure}

\begin{figure}[t]
\centering
    \begin{subfigure}[t]{0.49\linewidth}
    \includegraphics[width=0.99\linewidth]{figs/hashtags_cloud_top1p.png}
    \vspace{-4mm}
    \caption{{\small Toxic profiles}}
    \label{fig:hts_cloud_top1p}
    \end{subfigure}
    \hfill
    \begin{subfigure}[t]{0.49\linewidth}
    \includegraphics[width=0.99\linewidth]{figs/hashtags_cloud_baseline.png}
    \vspace{-4mm}
    \caption{{\small Baseline profiles}}
    \label{fig:hts_cloud_baseline}
    \end{subfigure}
    \vspace{-2mm}
    \caption{\small Hashtag word clouds of profile groups (cf. \S\ref{subsub:hashtag_cat}).}
    \vspace{-4mm}
\end{figure}
\subsubsection{Nature of hashtags shared by the top 1\% toxic and baseline profiles}
\label{subsub:hashtag_cat}

We now provide examples of highly occurring hashtags within the dataset. We present in 
Fig.~\ref{fig:hts_cloud_top1p} and ~\ref{fig:hts_cloud_baseline} the weighted word clouds of all hashtags collected from tweets of toxic and baseline profiles. The largest hashtag shared by toxic profiles is 'TreCru` which is an online video game known as Treasure Cruise. The remainder of the hashtags by toxic profiles are of a very explicit nature. On the other hand, the general Twitter profiles share hashtags about diverse topics including Covid, news, and politics such as \#BlackLivesMatter, \#Trump, \#Brexit, \#EndSARS which is a protest against police brutality in Nigeria. There are also everyday benign hashtags about music \#nowplaying, \#SoundCloud and shopping \#Giveaway, \#Job. 
On average toxic profiles share 59 unique hashtags per profile vs 275 by a baseline profile.
 
\subsubsection{Do toxic profiles share the same or similar hashtags in their tweets?}
\label{subsec:hash_similarity}

By leveraging the same Jaccard similarity metric defined in \S\ref{subsec:sld_similarity}, we inspect the overlapping hashtags used within and between the toxic and baseline profiles. While not visible in Fig.~\ref{fig:jaccard_similarity_hts}, it is noted that there is zero hashtag similarity in 90.2\% of toxic--toxic profiles, 85.5\% between toxic-baseline profiles, and 62.9\% of baseline--baseline profiles. What is visible in Fig.~\ref{fig:jaccard_similarity_hts} illustrates that a small proportion of toxic-toxic profiles have a much higher overlap of hashtags and thus an aligned area of discussion.
 
\subsubsection{\bf Takeaway:}
 Toxic profiles share fewer total and unique hashtags than baseline profiles. A larger majority of toxic profiles do not have overlapping hashtags with other toxic profiles (90.2\%), compared to the baseline (62.9\%). In the toxic profiles that do share hashtags with other toxic profiles, they are more aligned than the most overlapping baseline-baseline profiles.

