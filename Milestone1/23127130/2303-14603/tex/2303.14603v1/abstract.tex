\begin{abstract}
Toxicity is endemic to online social networks (OSNs) including Twitter. It follows a Pareto-like distribution where most of the toxicity is generated by a very small number of profiles and as such, analyzing and characterizing these ``toxic profiles'' is critical. Prior research has largely focused on sporadic, event-centric toxic content (i.e., tweets) to characterize toxicity on the platform. Instead, we approach the problem of characterizing toxic content from a profile-centric point of view.
%
We study 143K Twitter profiles and focus on the behavior of the top 1\% producers of toxic content on Twitter, based on toxicity scores of their tweets availed by Perspective API. With a total of 293M tweets, spanning 16 years of activity, the longitudinal data allows us to reconstruct the timelines of all profiles involved.
We use these timelines to gauge the behavior of the most toxic Twitter profiles compared to the rest of the Twitter population. 
%
We study the pattern of tweet posting from highly toxic accounts, based on the frequency and how prolific they are, the nature of hashtags and URLs, profile metadata, and Botometer scores. We find that the highly toxic profiles post coherent and well-articulated content, their tweets keep to a narrow theme with lower diversity in hashtags, URLs, and domains, they are thematically similar to each other, and have a high likelihood of bot-like behavior, likely to have progenitors with intentions to influence, based on high fake followers score.
%
Our work contributes insight into the top 1\% toxic profiles on Twitter and establishes the profile-centric approach to investigate toxicity on Twitter to be beneficial. The identification of the most toxic profiles can aid in the reporting and suspension of such profiles, making Twitter a better place for discussions.  Finally, we contribute to the research community with this large-scale and longitudinal dataset\footnote{https://github.com/hqayyum/twitter\_top\_toxic\_1percent}, annotated with six types of toxic scores.
\end{abstract}
