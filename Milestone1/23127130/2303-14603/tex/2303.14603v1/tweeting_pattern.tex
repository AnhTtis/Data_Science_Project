\subsection{Tweet Pattern}
\label{subsec:tweeting_pattern}
\subsubsection{Do toxic profiles follow any particular tweeting pattern?}
To capture a profile's tweeting manner, we consider the \textbf{\emph{Time delta}} i.e., the time between sequential or consecutive tweets (time noted from tweet timestamp) for each profile in our dataset. We consider histograms of time deltas in seconds and days in Fig.~\ref{fig:delta_combined_sec} and \ref{fig:delta_combined_days} respectively. 
For both toxic and baseline groups, the most populated period of time between tweets is within a few seconds. It is interesting to observe the occurrence of periodic behavior within the baseline profiles, indicating the presence of automation, despite these profiles not being overly toxic.
On the other hand, toxic profiles do not appear to post regularly at fixed intervals. There is a clear skew to the shorter time deltas in toxic profiles compared to the baseline.

\subsubsection{Is there consistency in the temporal tweeting pattern of the toxic profiles?}
We now explore if the time differences between the tweets that we have observed are distributed consistently within a profile's timeline, or if profiles `activate' briefly for bursts of activity and then go dormant.

To this end, we employ a normalized measure of burstiness to compare the tweeting behavior of toxic and baseline profiles. \textbf{{\emph{Burstiness}}}~\cite{kim2016measuring} is a quantification of inter-event time distribution from a given event sequence, that is, the distribution of time deltas between consecutive events. The 
\emph{Burstiness Score}
$
    B=\frac{\sigma-\mu}{\sigma+\mu}=\frac{r-1}{r+1},
    \label{eq:burstiness}
$
where $r=\sigma/\mu$ is the
coefficient of variation and $\sigma, \mu$ denote the standard deviation and mean of inter-event time distribution respectively. 
$B$ ranges continuously between $-1$ and $1$; regular time series (near constant inter-event times) would have scores closer to $B=-1$, $B=0$ is a random sequence, and $B=1$ is an extremely bursty time series (as $\sigma\rightarrow\inf$ for finite $\mu$). 
It is known that burstiness is dependent on the length of the time series, and since the number of events (tweets) differs among profiles, we adopt \textbf{\emph{Normalized Burstiness}}~\cite{kim2016measuring}, which removes this dependency, to facilitate direct comparison among profiles.
% \begin{equation}
$B(n,r) =\frac{\sqrt{n+1} r - \sqrt{n-1}} {(\sqrt{n+1}-2) r+\sqrt{n-1}}$,
    % \label{eq:normalised_burstiness}
% \end{equation} 
note that $B(n,r)$ can take values greater than one and less than -1. 
Fig.~\ref{fig:burstiness_scores} provides a probability density function or PDF of burstiness per profile. We observe that toxic profiles skew towards being more bursty with a curve peak at 0.6 as compared to 0.5 for the baseline Twitter profile. A two-sample t-test 
% calculated with \emph{`pingouin'} library in python gives 
yields a p-value of $4.26\times10^{-26}$, considerably less than $(\alpha=0.05)$ to indicate that the distribution of Burstiness is significantly different between the toxic and baseline profile groups. 

From this, we conclude that the toxic profiles are more irregular (and more bursty) in their tweeting behavior than the baseline profiles in general.
%
\subsubsection{\bf Takeaways:}
\begin{itemize}[leftmargin=*]
    % \item Toxic profiles may be dormant for a year with no tweets whatsoever, also true for some baseline profiles.
    
    \item Toxic profiles can be long-lived accounts with activity gaps of 8-9 years, and they are more likely to tweet in quick succession with minimal activity intervals.
    \item Toxic profiles do not appear to tweet at regular fixed intervals (a sign of automation), a behavior observed in the baseline profiles.
    % \item Toxic profiles can be long-lived accounts with activity gaps of 8-9 years.
    \item Generally parallels can be drawn between the temporal behavior of baseline and toxic profiles, however toxic profiles skew to favor shorter intervals between tweets and are more bursty.
\end{itemize}
%

