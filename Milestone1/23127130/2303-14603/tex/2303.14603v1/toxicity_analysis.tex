
\subsection{Tweet Toxicity}
\label{sec:toxicity_analysis}
\subsubsection{What type of misbehavior is common in toxic profiles?}  
The 6 scores from the Perspective API provide granular insight into the specific types of misbehavior exhibited by a profile. 
%
We plot the median scores of Toxicity, Severe Toxicity, Identity Attack, Inflammatory, and Insult per profile as a CDF in Fig.~\ref{fig:perspective_scores}. We observe that toxic profiles on all 6 dimensions of misbehavior exceed that of general Twitter profiles. We note that beyond ``Toxicity'', tweets high in ``Inflammatory'' and ``Insult'' are the next most prevalent within our toxic profiles. Interesting to note that Inflammatory is comparatively less prominent (after toxicity) among toxic profiles than baseline profiles. On the other hand, the lowest score for toxic profiles is ``Identity Attack'', This would be consistent with Twitter policy, which states that racist tweets are not tolerated~\cite{Twitter_conduct_policy}.
Figure~\ref{fig:gini_scores} is a CDF of the Gini index calculated on all 6 toxicity scores to gauge the consistency of misbehavior amongst a profile's tweets. We observe that toxic profiles in comparison to baseline profiles have lower Gini scores, implying the top 1\% toxic profiles exhibit misbehavior relatively consistently compared to the baseline.
%
\subsubsection{\bf Takeaway:}
The most common forms of toxicity among the Toxic 1\% profiles are ``Inflammatory'' and ``Insult'', with ``Identity Attack'' the least common. 
