\subsection{Top 1\% Toxic Twitter Profiles}
\label{sec:top1p-profiles}
We identify and study the upper echelon of toxic profiles as determined by the average {``Toxicity''} Perspective score of their tweets. We sort all the 143K profiles based on the median toxicity scores calculated on the toxicity scores of all tweets in their respective timelines (at this point we remove all profiles with less than 10 tweets and consider the rest 138K profiles). 
As a final step, motivated by the fact that toxicity follows the Pareto effect on Twitter~\cite{ribeiro2018characterizing}, we skim the top 1\% of profiles as a sample of the most toxic Twitter profiles, we refer to these profiles (1,380) as \textbf{\emph{`Top 1\% toxic profiles'}}. Note that 80\% of the 1\% contribute 1000 or more tweets.
We shall compare toxic profiles with the remainder of the population (136,620, or 99\%), referred to as \textbf{\emph{`baseline profiles'}} in text. 
To contextualize the 1\% on toxicity scores, the 1\% profiles have a median toxicity of 0.40, while those in the baseline have a median of 0.15. Further, almost all tweets from the 1\% have a Toxicity larger than 0.35, whereas it lies at <3\% for the baseline. 






