\begin{figure*}[t]
    \centering
    \begin{minipage}{.32\linewidth}
    \includegraphics[width=0.95\linewidth]{figs/profiles_data.pdf}
    \vspace{-2mm}
    \caption{\small Toxic and baseline profiles' features (e.g. \#friends, \#followers, etc.)
    %Profile numeric data  counts
    (cf. \S\ref{sec:profile_data_analysis})}
    \label{fig:profiles_data}
    \vspace{-1mm}
    \end{minipage}
    \hfill
\begin{minipage}{.33\linewidth}
    \centering
    \includegraphics[width=0.95\linewidth]{figs/profile_age.pdf}
    \vspace{-1mm}
    \caption{\small Creation dates of toxic and baseline profiles (cf. \S\ref{sec:profile_data_analysis}).}
    \label{fig:profile_age}
    \vspace{-1mm}
\end{minipage}
\hfill
\begin{minipage}{.33\linewidth}
    \centering
    \includegraphics[width=0.98\linewidth]{figs/botometer_scores.pdf}
    \vspace{-1mm}
    \caption{\small Botometer automation scores of toxic and baseline profiles (cf. \S\ref{sec:automation_analysis}).}
    \label{fig:botometer_scores}
    \vspace{-1mm}
    \end{minipage}
\end{figure*}


\section{Profile Level Analysis}
\label{sec:profile_analysis}
In this section, we observe profile-level characteristics that emerge from our toxic and baseline profiles. We shall start by analyzing details directly registered with Twitter (\S\ref{sec:profile_data_analysis}), followed by an analysis of automation as provided by the Botometer (\S\ref{sec:automation_analysis}).
%
\subsection{Account Metadata}
\label{sec:profile_data_analysis}
% \vspace{-3mm}
`Metadata is a ``data dictionary'' attached to every Twitter profile providing additional insight about a profile. It is a dictionary of 17 fields including name, location, account creation date, counts of friends, followers, statuses, and favorites. It also contains information if an account that is protected and/or verified.

\subsubsection{What does a toxic profile's Twitter account metadata say about them?}
We first inspect the proportion of profiles that are still present on Twitter. It is seen from Tab.~\ref{tab:twitter_profile} that all toxic profiles are still present on Twitter to this day, whereas 3.6\% of baseline profiles no longer exist. We unfortunately cannot further determine if these accounts were deleted or banned.
A majority of toxic profiles are verified; \emph{A profile with a blue badge to show that an account is Twitter verified}. Twitter allows automation~\cite{twitter_automation} and verifies the bot account~\cite{bot_verification}, it, however, does not allow misconduct. Also, 95\% of the Twitter profiles are protected; \emph{A profile that does not appear in third-party search engines, i.e, Google,~\cite{twitterAPI}}. No toxic profiles are banned in any specific country, whereas 0.002\% of baseline profiles are in multiple countries like Russia, Austria, and Belgium to name a few.
Additional numeric data is provided in the Twitter profile object, and the distribution of the values is represented in Fig.~\ref{fig:profiles_data}. 

\textbf{``Friends and Followers''} Friends of a Twitter profile are other profiles followed by said profile. While followers are other profiles that follow the said profile. We observe a toxic profile has on average 500 friends, and 9,500 followers, whereas the average baseline profiles have 800 friends and 10K followers in total. 

\textbf{``Listed''} gives the number of public lists that this profile is included within, these lists are used to collect similar accounts to strategize the timelines.
As few as a couple of toxic profiles exist as part of any list, whereas a median baseline profile on average is included within 50 lists. 

\textbf{``Favourites''} is the number of Tweets liked by a profile. We note that an average profile in both groups has liked an equally significant number of tweets (10K). 

\textbf{``Statuses''} are the number of tweets (including retweets) created by the profile in totality, beyond the 3200 restrictions imposed during the crawling of a profile's timeline. It is interesting to note that the number of tweets posted by a median toxic and baseline profile is approximately the same if all the historical tweets are taken into account.

\textbf{``Location''}
Another dimension is the reported location of the profile. This is a text string typically populated with a description of the profiles home town, state, and/or country. We leverage ``Geopy''~\cite{Geopy:Documentation}, a python library to resolve these strings into country names.
The countries with the highest occurrence are presented in Tab.~\ref{tab:twitter_profile_location}, only the top 3 countries are shown due to the quantity of found countries. 

We can observe that the majority of toxic and baseline profiles are based in the US, UK, and Canada, but there exists a long tail of other countries with which profiles are associated. Interestingly, the toxic profiles are more strongly concentrated in the US (61.36\%) compared to the proportion of baseline profiles in the US (29.42\%). This finding is likely to differ when a different language is considered. We also note that only 0.04\% of toxic and 14\% of baseline profiles enabled ``Geolocation'' in their profile.

\textbf{``Creation date''}
Finally, we inspect the creation date of the profiles in Fig.~\ref{fig:profile_age}. We observe that toxic profiles skew younger than the baseline profiles. A possible explanation is an increase in the creation of toxic profiles around 2014-2016, as observed by~\cite{kollanyi2016bots}. We acknowledge that the forced and voluntary deletion of toxic profiles may also bias these numbers. Interestingly, there was a notable decline in the growth of profiles after 2016, which coincides with a plateau of active users on Twitter~\cite{monthlyactiveusers}.


%
\begin{table}[t]
\resizebox{\columnwidth}{!}{
\begin{tabular}{l|cccc}
\toprule
&\bf Active profiles & \bf Protected & \bf Verified & \bf Withheld in countries \\ \midrule
\bf Toxic Profiles&  100\%& 92.74\% & 96.5\%& None \\ \hline
\bf Baseline Profiles&  96.4\% &95.74\% &82.6\% & 0.002\%\\
 \bottomrule
\end{tabular}
}
\caption{\small Twitter profiles data (cf. \S\ref{sec:profile_data_analysis}).}
\label{tab:twitter_profile}
\vspace{-4mm}
\end{table}


%
\begin{table}[t]
\resizebox{\columnwidth}{!}{
\begin{tabular}{l|l}
\toprule
& \bf Top 3 locations found in profiles  \\ \midrule
\bf Toxic Profiles& US(61.36\%), Canada(9.09\%), UK(9.09\%), 9 others(20.04\%)  \\ \hline
\bf Baseline Profiles& US(49.42\%), UK(22.61\%), Canada(7.0\%), 33 others(20.09\%)  \\
 \bottomrule
\end{tabular}
}
\caption{\small Twitter profiles location (cf. \S\ref{sec:profile_data_analysis}).}
\label{tab:twitter_profile_location}
% \vspace{-10mm}
\end{table}
%
\subsubsection{\bf Takeaways:}
\begin{itemize}[leftmargin=*]
    \item Toxic profiles, in general, have fewer friends, and followers and are not part of public lists in other accounts. Of the toxic profiles that have a location, 61\% of them are based in the US. 
    \item We observe an increase in the creation of toxic profiles in the US election years between 2014 and 2016, matching previous work. 
\end{itemize}
