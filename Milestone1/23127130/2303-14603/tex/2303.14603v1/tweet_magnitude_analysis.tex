\begin{figure*}[!ht]
    \centering
        \begin{subfigure}[t]{0.135\linewidth}
        \centering
            \includegraphics[width=\textwidth]{figs/flesch_reading_score.pdf}
            \vspace{-3mm}
            \caption{\small Flesch Kincaid\newline reading ease score}
            \label{fig:fkr_score}
    \end{subfigure}
    \begin{subfigure}[t]{0.135\linewidth}
            \centering
            \includegraphics[width=\textwidth]{figs/flesch_score.pdf}
            \vspace{-3mm}
            \caption{\small Flesch Kincaid\newline difficulty score}
            \label{fig:fk_score}
    \end{subfigure}
    \begin{subfigure}[t]{0.135\linewidth}
            \centering
            \includegraphics[width=\textwidth]{figs/Linsear_scores.pdf}
            \vspace{-3mm}
            \caption{\small Linsear write \newline score}
            \label{fig:lin_write_score}
    \end{subfigure}
    % 
    \begin{subfigure}[t]{0.135\linewidth}
            \centering
            \includegraphics[width=\textwidth]{figs/ARI_scores.pdf}
            \vspace{-3mm}
            \caption{\small ARI \newline score}
            \label{fig:auto_read_score}
    \end{subfigure}
    \begin{subfigure}[t]{0.135\linewidth}
            \centering
            \includegraphics[width=\textwidth]{figs/textual_diversity_score.pdf}
            \vspace{-3mm}
            \caption{\small Lexical \newline diversity score}
            \label{fig:lexical_diversity}
    \end{subfigure}
     \begin{subfigure}[t]{0.135\linewidth}
            \centering
            \includegraphics[width=\textwidth]{figs/no_words.pdf}
            \vspace{-3mm}
            \caption{\small \# Average \newline tweet words }
            \label{fig:tweet_words}
    \end{subfigure}
    \begin{subfigure}[t]{0.135\linewidth}
            \centering
            \includegraphics[width=\textwidth]{figs/no_char.pdf}
            \vspace{-3mm}
            \caption{\small \# Average \newline tweet char}
            \label{fig:tweet_char}
    \end{subfigure}
    \vspace{-2mm}
    \caption{\small Lexical analysis of toxic and baseline profiles' tweets (cf. \S\ref{subsec:lexical_analysis} for details) in our dataset.}
    \label{fig:lexical_analysis}
    \vspace{-2mm}
\end{figure*}


\begin{figure*}[!ht]
    \centering
    \begin{subfigure}[t]{0.4\linewidth}
            \includegraphics[width=\textwidth]{figs/median_perspective_scores.pdf}
            \vspace{-8mm}
            \caption{{\small Median Perspective scores}}
            \label{fig:perspective_scores}
    \end{subfigure}
    \qquad\qquad
    \begin{subfigure}[t]{0.4\linewidth}
            \includegraphics[width=\textwidth]{figs/gini_scores.pdf}
            \vspace{-8mm}
            \caption{{\small Gini index of Perspective scores}}
            \label{fig:gini_scores}
    \end{subfigure}
    \vspace{-2mm}
    \caption{\small Median (\protect\subref{fig:perspective_scores}) and Gini index (\protect\subref{fig:gini_scores}) of Perspective API scores (cf. \S\ref{sec:toxicity_analysis} for details).
    }
    \vspace{-4mm}
\end{figure*}

\section{Content analysis}
\label{sec:content_analysis}
The nature of a profile's tweets is determined by the actual text and, also by attached auxiliary content such as URLs and hashtags.
We perform a content analysis of timeline tweets with respect to each toxic or baseline profile. Specifically, we analyze the quality of the tweet's text (\S\ref{subsec:lexical_analysis}), toxicity level in the text (\S\ref{sec:toxicity_analysis}), and additional tweet attributes like URLs (\S\ref{subsec:url_analysis}), and hashtags (\S\ref{subsec:hashtag_analysis}). 
%









\subsection{Tweet lexicon} 
\label{subsec:lexical_analysis}
The way a tweet is constructed tells the degree of authority of the author and potential target audience. Thus, we now analyze the text within our profiles' tweets for length, grammatical correctness, and semantic correctness. 
%
\subsubsection{Do toxic profiles share verbose tweets?}
\label{subsectio:verbosity_analysis}
We question, how much of the character allowance in a tweet is utilized by our toxic profiles. 
To this end, we parse each tweet to extract the number of words and characters for both toxic and baseline profiles. The boxplots in Fig.~\ref{fig:tweet_words} and~\ref{fig:tweet_char} display the distribution of the average number of words and characters in tweets. 
%
We observe that toxic profiles post shorter tweets with fewer words than baseline Twitter profiles with an average of 11 words and 70 characters.
%
\subsubsection{Do toxic profiles share legible and easy-to-read tweets? }
With the tweet text in hand, we measure the Flesch Score~\cite{flesch1948new} (ease and difficulty), Linsear write scores, Automated Readability Index~\cite{senter1967automated}, and Lexical Diversity of toxic and baseline profiles. 

The \textbf{\emph{Flesch score}} indicates how difficult or easy it is to read the text~\cite{flesch}, and is computed as:
$
    206.835-1.015\times(\frac{total words}{total sentences})-84.6\times(\frac{total syllables}{total words})
$.
\textbf{\emph{Linsear write score}} measures the length of words in the number of syllables and divides this score ``r'' by the number of total sentences~\cite{mccannon2019readability}. If (r > 20, Lw = $\frac{r}{2}$) and if (r$\ge$20, Lw = $\frac{r}{2-1}$). 
\textbf{\emph{Automated Readability Index (ARI}} estimates the comprehensibility of a text corpus and is computed as (4.71$\times$average word length)+(0.5$\times$average sentence length)-21.43~\cite{ARI}. 
\textbf{\emph{Lexical diversity}}, defined as the ratio of a number of unique words to the total number of words, reveals noticeable repetitions of distinct words~\cite{doi:10.1080/15434303.2020.1844205}.
Higher values of the ARI, Flesch scores and lexical diversity of a given text indicate increased comprehensiveness, improved readability, and range and variety of vocabulary. 
A given text with a high Linsear write score generally includes words with more syllables and/or is written with richer language.

Fig.~\ref{fig:lexical_analysis} provides a summary of the 3 metrics.
In comparison to baseline profiles, toxic profiles share more legible and readable tweets. Toxic profiles use richer and more profound vocabulary, which as explained above is a predictable use of language and a sign of good writing style.
%
\subsubsection{\bf Takeaway:}
Top toxic 1\% profiles share shorter tweets written in more understandable language than baseline.
