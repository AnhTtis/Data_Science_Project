

\begin{figure}[!htb]
\centering
    \begin{subfigure}[t]{0.46\linewidth}
        \includegraphics[width=0.95\linewidth]{figs/no_total_urls.pdf}
        \vspace{-1mm}
        \caption{\#URLs}
        \label{fig:no_total_urls}
    \end{subfigure}
    \hfill
    \begin{subfigure}[t]{0.46\linewidth}
        \includegraphics[width=0.95\linewidth]{figs/no_unique_urls.pdf}
        \vspace{-1mm}
        \caption{\# Unique URLs}
        \label{fig:no_unique_urls}
    \end{subfigure}

    \begin{subfigure}[t]{0.46\linewidth}            \includegraphics[width=0.95\linewidth]{figs/p_unique_urls_combined_cdf.pdf}
        \vspace{-1mm}
        \caption{{\% Unique URLs }}
        \label{fig:p_unique_urls_combined_cdf}
    \end{subfigure}
    \hfill
    \begin{subfigure}[t]{0.46\linewidth}
        \includegraphics[width=0.95\linewidth]{figs/jaccard_similarity.pdf}
        \vspace{-1mm}
        \caption{{Jaccard Similarity (SLDs)}}
        \label{fig:jaccard_similarity}
    \end{subfigure}
    
    \vspace{-1mm}
    \caption{\small Profile level URL analysis (\S\ref{subsec:url_analysis}).}
    \vspace{-2mm}
\end{figure}



%
\subsection{URLs}
\label{subsec:url_analysis}
A shared URL is an indication that a profile seeks to point a reader to a resource external to the Twitter platform, either as a corroborative source of validation or for further reading about their tweet's subject matter. On the other hand, The repetition of a shared URL and posting URLs related to one subject (category) shows how much a profile emphasizes a topic.
From our 1,380 toxic profiles and 136,620 baseline profiles, we detect and extract a total of 57,725,668 URLs and 43,916,037 unique URLs in total. 
%
\subsubsection{How frequently do toxic profiles share URLs as part of their tweet text?}

To answer this question, we count total number of URLs and also note the number of URLs without repetition (we inspect the full length of the URLs, extracted from the tweet's metadata and we did not rely on the shortened version used in the tweet text). 
Fig.~\ref{fig:no_total_urls} illustrates a CDF on the total number of URLs per profile for both groups. On the low end of the figure, it is clear that baseline profiles engage more with sharing URLs than toxic profiles. We observe that 45\% of the toxic profiles shared 10 or fewer URLs in their tweets, compared to only 10\% of baseline profiles. 
On the other extreme, approximately 10\% of both profile groups are heavy URL hitters with more than 1,000 URLs in total, and 5\% of toxic profiles posted ~3,200 URLs compared to nearly no baseline profiles. 3,200 corresponds to the maximum number of tweets obtainable from a single profile. We note that 3.3\% baseline profiles shared ~6-8K URLs, these profiles shared multiple URLs per tweet in short form and on average shared 3 URLs per tweet --- these profiles were predominantly news services can also comment from \S\ref{subsec:tweeting_pattern} that there are profiles in baseline which are persistent with regular tweeting pattern, which might indicate these are bots.  

Fig.~\ref{fig:no_unique_urls} presents the unique number of URLs per profile (i.e.:~not counting repetitions). Half of the toxic profiles shared at most 0-10 unique URLs and half of the baseline profiles shared at most 0-95 unique URLs. It is interesting to observe that 22\% of toxic profiles used a singular unique URL and 13\% did not post any URLs at all.
Fig.~\ref{fig:p_unique_urls_combined_cdf} shows the proportion, per profile, of URLs that are repetitions among toxic and baseline profiles. We see that around 48\% of toxic profiles do not repeat URLs at all, compared to only 27\% of baseline profiles (right-hand side of the plot). 
In contrast, looking at profiles that repeat URLs the most, the top 33\% of toxic profiles (CDF values 0.0 to 0.33) have substantially more repetition than the corresponding group of baseline profiles. 
We note that toxic profiles in general share a lower percentage of unique URLs in their tweets. 
Fig.~\ref{fig:jaccard_similarity} shows us the Jaccard similarity of URLs amongst and between toxic and baseline groups. We observe that toxic profiles share URLs of the same nature.
A large proportion of toxic profiles (63.7\%) have no URL similarity with one another, in comparison to between baseline profiles (18.7\%), this could be the result of the toxic profiles operating independently, or with uniquely crafted/tracking URLs. The remaining 20\% of toxic profiles however do have a heightened shared URL similarity compared to the baseline, indicating the existence of coordination.
%

\begin{figure}[t]
\centering
    \begin{subfigure}[t]{0.49\linewidth}
    \includegraphics[width=0.99\linewidth]{figs/sld_cat_cloud_top1p.png}
    \vspace{-6mm}
    \caption{{\small Toxic profiles}}
    \label{fig:top1p_sld_cat_cloud}
    \end{subfigure}
    \hfill
    \begin{subfigure}[t]{0.49\linewidth}
    \includegraphics[width=0.99\linewidth]{figs/sld_cat_cloud_baseline.png}
    \vspace{-6mm}
    \caption{{\small Baseline profiles}}
    \label{fig:baseline_sld_cat_cloud}
    \end{subfigure}
    \vspace{-2mm}
    \caption{\small Second level domains (SLDs) categories shown as word clouds for toxic (\protect\subref{fig:top1p_sld_cat_cloud}) and baseline (\protect\subref{fig:baseline_sld_cat_cloud}) profiles (cf. \S\ref{subsub:url_cats} for details).}
    \vspace{-1mm}
\end{figure}




\subsubsection{What is the nature of categories in the URLs shared from tweets of toxic and baseline profiles?}
\label{subsub:url_cats}

For the URLs that have been shared, the domain can provide an indication of the type of content linked.
For example, {\tt{www.example.com}}'s second level domain (SLD) is {\tt{example.com}}.
Proceeding forwards, ``Domain'' and ``SLD'' are interchangeable. 
We classify the content type of the domain with the \textbf{\emph{FortiGuard}} classification service~\cite{Fortiguard}. FortiGuard uses link crawlers, customer logs, and machine learning to categorize websites~\cite{Triplet20}.
We successfully categorize 98.2\% unique domains for baseline and 95.4\% for toxic profiles with FortiGuard. The total number of found unique categories between the SLDs was 596 for toxic and 1,008 for the baseline. A toxic profile has on average 49 unique SLDs and a baseline profile has 487 unique SLDs in all of the tweets. 
We present in Figs.~\ref{fig:top1p_sld_cat_cloud} and ~\ref{fig:baseline_sld_cat_cloud} a weighted word cloud of the SLD categories, the size of text represents the percentage of SLDs in each group assigned the category label. 
We observe that toxic profiles are linked to domains categorized as Pornography (examples intentionally omitted), and Information Technology (e.g.  \texttt{youtube.com, SelfieSwipes.com, Kailani-Kai.com}). For the baseline profiles, the largest category is business (\texttt{huffingtonpost.com, manchester.ac.uk,gotthevo\newline te.org})
%
\subsubsection{Do toxic profiles share SLDs about the same subject/topic?}
\label{subsec:sld_similarity}
For this, we now do a direct comparison between the nature of SLDs of toxic and baseline profiles, amongst themselves and between each other. 
Specifically, in Fig.~\ref{fig:jaccard_similarity} we compute and show the CDF of pairwise Jaccard similarity calculated on the sets of SLDs present in each toxic and baseline profile.
The \textbf{\emph{Jaccard Index}} is computed between two sets $A$ and $B$ as $\frac{|A \bigcup B|}{|A \bigcap B|}$, and ranges between 0 (for no common elements between the two sets) to 1 (for a perfect match or overlap). 
We observe that the SLDs in toxic profiles have the greatest overlap. Also, 64\% of pairs of toxic 1\% profiles have no similarity compared to only 18\% of baseline pairs. 44\% of toxic~1\%-baseline pairs have disjoint sets of SLDs, indicating the presence of many SLDs present in toxic 1\% tweets that are absent from baseline tweets. Overall, there is little similarity, with 95\% of baseline-toxic 1\% and baseline-baseline pairs, and 88\% of toxic 1\%-toxic 1\% pairs with Jaccard similarity less than 0.1.

\subsubsection{\bf Takeaways:}
\begin{itemize}[leftmargin=*]
    \item Toxic Profiles use fewer URLs  and generally refer to unique URLs suggesting they refer less to external sources than the rest of the Twitter population.
    \item Of the domains linked by toxic profiles, we observed that the most popular domain categories are pornography, news, and information technology.
    \item Toxic 1\% profiles have a larger proportion of profiles (63.7\%) with no similarity between shared SLDs than baseline profiles (18.7\%). This indicates high uniqueness, either from the independent operation or customized tracking domains.

\end{itemize}
