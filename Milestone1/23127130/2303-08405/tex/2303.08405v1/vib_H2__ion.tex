\documentclass[aps,pra,reprint,groupedaddress,floatfix]{revtex4-2}
\usepackage{graphicx,bm,amsmath}
\usepackage{braket}



\newcommand{\rmd}{\mathrm{d}}
\newcommand{\rme}{\mathrm{e}}
\newcommand{\rmi}{\mathrm{i}}
\renewcommand{\vec}[1]{\boldsymbol #1}
\newcommand{\abs}[1]{\left|#1\right|}
\newcommand{\sumint}%
{\mathop{\hbox{$\displaystyle\sum\kern-13.2pt\int\kern1.5pt$}}}
\newcommand{\threej}[6]{ \left(\begin{array}{ccc}
   #1 & #2 & #3 \\
   #4 & #5 & #6 
\end{array}\right) }
\newcommand{\sixj}[6]{ \begin{Bmatrix}
   #1 & #2 & #3 \\
   #4 & #5 & #6 
\end{Bmatrix}}





\begin{document}



\title{Photoionization from the ground and excited vibrational states of H$_{2}^{+}$ }

\author{Adam Singor}
\email{adam.singor@postgrad.curtin.edu.au \\adamsingor@protonmail.com}
\author{Liam H. Scarlett}	
\author{Igor Bray}
\author{Dmitry V. Fursa}
\affiliation{Curtin Institute for Computation and Department of Physics and Astronomy, Curtin University, Perth, Western Australia 6102, Australia}


\date{\today}


\begin{abstract}
Photoionization cross sections for all bound vibrational levels of the 1s$\sigma_{\mathrm{g}}$ state of H$_{2}^{+}$ are presented.
The only approximation employed in our calculation of vibrationally--resolved photoionization cross sections is the Born--Oppenheimer approximation.
The origin of the near threshold oscillations in the vibrationally--resolved photoionization cross sections is described.
A benchmark set of photoionization cross sections are presented.
Fixed--nuclei photoionization cross sections are calculated using true continuum wave functions for H$_{2}^{+}$ at an internuclear separation of 2 a$_{0}$ and compared with previous calculations with excellent agreement found in many cases, but not all.




\end{abstract}


\maketitle




\section{Introduction}
\label{sec:intro}
A fully quantum--mechanical description of photon--atom and photon--molecule scattering has been understood since the mid 1920s with the development of the Kramers--Heisenberg--Waller (KHW) matrix element \cite{Kramers1925,Waller1929}. The KHW matrix element describes photon--molecule interactions to second order in perturbation theory. Photon scattering cross sections have proved essential in many applications, such as modeling opacity and radiative transport \cite{Sampson1959,Huebner2014,Colgan2016}, quantum illumination and radar \cite{Lloyd2008,Lanzagota2012}, and Raman spectroscopy \cite{Ferraro2003}, which in particular has important applications in hydrogen storage \cite{Panella2008,Reed2011,Weseluchabirczynska2012,Hadjiivanov2021}. 

 
\citet{Bates1953} calculated photoionization cross sections for the 1s$\sigma_{\mathrm{g}}$, 2s$\sigma_{\mathrm{g}}$, and 3s$\sigma_{\mathrm{g}}$ states of H$_{2}^{+}$.
\citet{Cohen1966} suggested that undulations that had been previously observed in photoionization cross sections of O$_{2}$ and N$_{2}$ could be explained by interference resulting from considering the two atoms as independent sources of photoelectrons. 
\citet{Cohen1966} also expected that such undulation would be present in photoionization cross sections for H$_{2}^{+}$.
It was also proposed~\cite{Cohen1966} that the integrated photoionization cross section for a diatomic molecule could be approximated by a corresponding hydrogen--like atomic cross section multiplied by a modulation factor,
\begin{equation}
\sigma^{\mathrm{ion}} = \sigma_{\mathrm{H}}^{\mathrm{ion}}(Z_{\mathrm{eff}})\left(1+\frac{\sin(k_{e}R)}{k_{e}R} \right), \label{eq:CF_model}
\end{equation}
where $k_{e}$ is the photoelectron momentum, $R$ is the internuclear separation, and $\sigma_{\mathrm{H}}^{\mathrm{ion}}(Z_{\mathrm{eff}})$ is the photoionization cross section for a hydrogen--like atom with an effective charge $Z_{\mathrm{eff}}$.


\citet{Bates1968} calculated the same photoionization cross sections for H$_{2}^{+}$ but over a much wider range of photon energies and found that the photoionization cross sections did not show undulations as had been previously suggested~\cite{Cohen1966}.
\citet{Brosolo1992}, and \citet{Brosolo1994} used a variational approach for calculating continuum wave functions which were then used to determine photoionization cross sections for H$_{2}^{+}$ and HeH$^{2+}$.
\citet{Colgan2003} used a time--dependent method to calculate single-- and multi--photon ionization cross sections for H$_{2}^{+}$.
\citet{Arkhipov2018} used a trajectory semiclassical method to calculate the photoionization cross section of H$_{2}^{+}$ for photon energies up to 100 a.u..
\citet{Fojon2006} performed a theoretical investigation of photoionization of H$_{2}^{+}$ by photons with an energy of a few hundred eV and concluded that using the linear combination of atomic orbitals approximation or modeling the ionized electron with a free wave function leads to an incorrect description of the interference effects.
\citet{Fernandez2007} conducted a theoretical study of photoionization of H$_{2}$ and H$_{2}^{+}$ and determined that the interference effects in the angular distributions depend strongly on the orientation of the molecule and the positions of the nuclei.
It was found that the first minima in the cross section for a given partial wave occurs when the electron momentum satisfies $k_{e}R\approx \ell\pi$.
The $k_{e}R\approx \ell \pi$ rule was explained for ejected electrons with a large kinetic energy \cite{Fernandez2009A}. 
Under such conditions the electron will only be affected by the potential when the electron is very close to the nuclei, which is similar to scattering by two $\delta$ functions or a one--dimensional infinite square well of width $R$.
\citet{Fernandez2009B} performed a theoretical investigation of one--photon single ionization of H$_{2}^{+}$, H$_{2}$, and Li$_{2}^{+}$ by circularly polarized photons. Their results showed that two--center interference effects that are observed in photoelectron angular distributions for linearly polarized light are also present in the case of circularly polarized light.


\citet{DellaPicca2008} determined partial cross sections for the photoionization of H$_{2}^{+}$ which contain Cooper minima. 
They found that Cooper minima can be attributed to zero absorption rather than confinement of the target electron.
\citet{DellaPicca2009} showed that the undulations in the total photoionization cross section are due to the presence of Cooper minima in the partial photoionization cross sections. 
It was also demonstrated that the undulations are periodic only for homonuclear targets.
\citet{DellaPicca2011} found that the position of Cooper minima present in the partial photoionization cross sections depends on the accuracy of the target states.
Additionally, the use of single--center target states leads to the appearance of multiple Cooper minima in the partial photoionization cross sections for both parallel and perpendicular orientations.
While the use of two--center target states results in the appearance of only one Cooper minima in each partial photoionization cross section and only for polarizations parallel to the internuclear axis.



Transitions between rotational and vibrational levels~\cite{Schulman1972,Schulman1975,Hsu1988}, dissociative processes~\cite{Dunn1968,Argyros1974,Saha1980,Zammit2017}, and fixed--nuclei photoionization of H$_{2}^{+}$ has been well--studied. 
However, photoionization from vibrationally excited states of H$_{2}^{+}$ has not been investigated. 
Atomic and molecular hydrogen are abundant in the interstellar medium with H$_{2}^{+}$ being formed by radiative association of protons and atomic hydrogen, and ionization of H$_{2}$~\cite{Dalgarno1985,Mukherjee2006}.
Photoionization of H$_{2}^{+}$ is of particular interest in astrophysics and plasma physics~\cite{Langhoff1985,Heays2017}, particularly the cooling rates of astrophysical plasma~\cite{Wiersma2009}, the composition of stellar and planetary atmospheres~\cite{Gruntman1996,Tseng2011}.
In all of these cases H$_{2}^{+}$ exists in a distribution of vibrational levels, and hence photoionization cross sections resolved in the initial vibrational level are required.

We present a comprehensive set of benchmark fixed--nuclei photoionization cross sections and calculate photoionization cross sections for the ground and excited vibrational levels of the 1s$\sigma_{\mathrm{g}}$ state of H$_{2}^{+}$. 
The origin of the oscillations present in the photoionization cross sections from vibrationally excited states near the ionization threshold is investigated.
In Sec.~\ref{sec:theory} we provide an overview of the H$_{2}^{+}$ target structure and the calculation of photoionization cross sections.
A comprehensive set of benchmark fixed--nuclei photoionization cross sections is presented and compared with previous results in Sec.~\ref{sec:FN-ion}.
In Sec.~\ref{sec:vib_ion}, photoionization cross sections for all bound vibrational levels of H$_{2}^{+}$ are presented.
Conclusions and future directions are formulated in Sec.~\ref{sec:conclusion}.




\section{Theory}
\label{sec:theory}
In this section the target structure of the H$_{2}^{+}$ molecule is described. A brief summary of the formalism for calculating photoionization cross sections is also given. 

\subsection{Target Structure}
We start by applying the Born--Oppenheimer approximation which assumes that the electronic and nuclear parts of the target wave function are separable. 
The non--relativistic electronic Schr\"odinger equation for the molecular hydrogen ion is separable quasi--radial and quasi--angular parts in prolate spheroidal coordinates and is given by
\begin{align}
\Bigg\{ \frac{\rmd}{\rmd\rho} \left[ \rho(\rho+R)\frac{\rmd}{\rmd\rho}\right] - \frac{m^{2}R^{2}}{4\rho(\rho+R)} + 2(2\rho+R) \label{eq:radial} \nonumber \\[-5pt]
\hspace{45pt} +2E\rho(\rho+R) - A_{\lambda}^{\abs{m}}(E,R) \Bigg\}\Xi_{\lambda}^{\abs{m}}(\rho;R) &= 0, \\
\Bigg\{ \frac{\rmd}{\rmd\eta} \left[ (1-\eta^{2})\frac{\rmd}{\rmd\eta}\right] - \frac{m^{2}}{1-\eta^{2}} +\tfrac{1}{2}ER^{2}(1-\eta^{2})  \label{eq:angular} \nonumber \\[-5pt]
\hspace{100pt}  + A_{\lambda}^{\abs{m}}(E,R) \Bigg\}\Upsilon_{\lambda}^{m}(\eta,\phi;R) &= 0.
\end{align}
Here the quasi--radial and quasi--angular coordinates are
\begin{equation}
\qquad \rho =\frac{r_{1}+r_{1}}{2}-\frac{R}{2}, \qquad\qquad \eta = \frac{r_{1}-r_{2}}{R},
\end{equation}
where $r_{1}$ and $r_{2}$ are the coordinates of the electron relative to the two nuclei, $R$ is the internuclear separation, $\phi$ is the usual azimuthal coordinate, $E$ is the energy eigenvalue, $A_{\lambda}^{\abs{m}}(E,R)$ is the separation constant, $m$ is the projection of the orbital angular momentum onto the internuclear axis, and $\lambda$ is a pseudo--angular momentum used to label states.
The pseudo--angular momentum $\lambda$ corresponds to the orbital angular momentum in the combined nuclei limit, i.e. $\lambda \to \ell$ as $R\to 0$.
The electronic states of H$_{2}^{+}$ can then be written as 
\begin{equation}
\psi_{\lambda}^{m}(\rho,\eta,\phi;R) = \Xi_{\lambda}^{\abs{m}}(\rho;R)\Upsilon_{\lambda}^{m}(\eta,\phi;R).
\end{equation}


Bound electronic states are obtained by diagonalizing the unseparated Hamiltonian in a Sturmian basis $\{ \varphi_{k\ell m }\}$, and are represented by
\begin{equation} \label{eq:bound_st_expansion}
\psi_{\lambda}^{m}(\rho,\eta,\phi;R) = \sum_{k=1}^{k_{\mathrm{max}}} \sum_{\ell=\abs{m}}^{\ell_{\mathrm{max}}} C_{k\ell m} \varphi_{k\ell m}(\rho,\eta,\phi),
\end{equation}
where 
\begin{equation}
\varphi_{k\ell m}(\rho,\eta,\phi) = f_{k}^{m}(\rho) Y_{\ell}^{m}(\eta,\phi), \label{eq:basis}
\end{equation}
\begin{align}
f_{k}^{m}(\rho) = \sqrt{2\alpha_{m}\frac{(k-1)!}{(k+m-1)!}}(2\alpha_{m}\rho)^{m/2} \rme^{-\alpha_{m}\rho}  \nonumber \\
\hspace{120pt}\times L_{k-1}^{m}(2\alpha_{m}\rho).
\end{align}
Here, $L_{k-1}^{m}$ are the associated Laguerre polynomials, $Y_{\ell}^{m}$ are spherical harmonics, $\alpha_{m}$ is an exponential fall--off parameter, and $k$ is an index.
These basis functions are orthogonal with respect to their coordinates: 
\begin{align}
\int_{0}^{\infty} \rmd\rho \, f_{k'}^{m}(\rho)f_{k}^{m}(\rho) = \delta_{k'k}, \\
\int_{0}^{2\pi}\!\rmd\phi \int_{-1}^{1}\!\rmd\eta\, Y_{\ell'}^{m'*}(\eta,\phi)Y_{\ell}^{m}(\eta,\phi) = \delta_{\ell'\ell}\delta_{m'm},
\end{align}
but not the volume element in prolate spheroidal coordinates. However, their overlap with the volume element is analytic, see Eq.~(\ref{eqA:ovrlap}).
The expansion coefficients, $C_{k\ell m}$ in Eq.~(\ref{eq:bound_st_expansion}), and the bound state energy levels, $E$, are obtained by solving the generalized eigenvalue problem
\begin{equation}
\vec{H} \vec{C} = E \vec{B} \vec{C}. 
\end{equation}
Here $\vec{H}$ is the Hamiltonian matrix with matrix elements given by Eq.~(\ref{eqA:Ham_me}), $\vec{B}$ is the overlap matrix with elements given by Eq.~(\ref{eqA:ovrlap}), and $\vec{C}$ contains the expansion coefficients as its columns.

True continuum target states are calculated using the approach given by \citet{Singor2022}.
The quasi--angular wave functions are obtained by expanding the spheroidal harmonics in Eq.~(\ref{eq:angular}) as a series of spherical harmonics of the same $m$. This leads to a pentadiagonal matrix which is then diagonalized to produce expansion coefficients and separation constants $A_{\lambda}^{\abs{m}}(E,R)$. The solution to the quasi--radial equation, Eq.~(\ref{eq:radial}), is started using a power series expansion and then propagated using an Adams--Moulton predictor--corrector algorithm. An asymptotic series is then used to normalize the wave function.\\

The non--relativistic Born-Oppenheimer Schr\"odinger equation for the vibrational wave functions is
\begin{equation}
\left[ -\frac{1}{2\mu}\frac{\rmd^{2}}{\rmd R^{2}} +\epsilon_{n}(R) - \varepsilon_{nv}\right] \nu_{nv}(R)=0,
\end{equation}
where $n$ specifies the electronic state, $\epsilon_{n}(R)$ is the potential energy curve for the electronic state $n$, $\mu$ is the reduced mass of H$_{2}^{+}$, and the centrifugal term is neglected in the current non--rotationally resolved calculations. 
Bound vibrational wave functions are found by expanding $\nu_{nv}(R)$ as
\begin{equation}
\nu_{nv}(R) = \sum_{k=1}^{k_{\mathrm{max}}} \tilde{C}_{knv} \,\phi_{k}(R),
\end{equation}
where 
\begin{equation}
\phi_{k}(R) = \frac{\sqrt{\tilde{\alpha}}}{k} \,2\tilde{\alpha}R \,\rme^{-\tilde{\alpha}R} \,L^{1}_{k-1}(2\tilde{\alpha}R).
\end{equation}
Here $\tilde{\alpha}$ is an exponential fall--off parameter, and $L^{1}_{k-1}$ are the associated Laguerre polynomials.





\subsection{Photon scattering}
The cross section for photoionization of H$_{2}^{+}$ in the initial state $\ket{n_{i}v_{i}}$ by an unpolarized photon with energy $\omega$ is
\begin{equation}
\sigma_{n_{i}v_{i}}^{\mathrm{ion}} = \sigma_{\mathrm{T}}\frac{\pi c^{3}\omega}{2} \sumint_{v_{t}}\sum_{\kappa=-1}^{1}\abs{\braket{n_{i}v_{i}|d_{\kappa}|(E_{i}+\omega) v_{t}}}^{2}.
\end{equation}
Here $\ket{n}$ denotes an electronic state, $\ket{v}$ denotes a vibrational level of a given electronic state, $c$ is the speed of light $\approx 137$ a.u., $\sigma_{\mathrm{T}} = 8\pi r_{0}/3 \approx 6.652 \times 10^{-29}$ m$^{2}$ is the Thomson cross section, and $\braket{n'v'|d_{\kappa}|Ev}$ is a dipole matrix element.
The $\kappa=0$ component of the dipole matrix element corresponds to photon polarization parallel to the internuclear axis and the $\kappa=\pm1$ components correspond to photon polarization perpendicular to the internuclear axis.
The final vibrational levels $\ket{v_{t}}$ can be summed over using closure.
This allows us to define the electronic part of the photoionization cross section
\begin{equation}
\sigma_{n_{i}}^{\mathrm{ion}}(R) = \sigma_{\mathrm{T}}\frac{\pi c^{3}\omega}{2} \sum_{\kappa=-1}^{1}\abs{\braket{n_{i}|d_{\kappa}|E_{i}+\omega }}^{2}. \label{eq:FN_ion}
\end{equation}
Explicit forms of the dipole matrix elements are given in appendix \ref{App:DipME}.
The vibrationally--resolved photoionization cross section as a function of incident photon energy is then
\begin{equation}
\sigma_{n_{i}v_{i}}^{\mathrm{ion}} = \braket{ v_{i}|\sigma_{n_{i}}^{\mathrm{ion}}(R)| v_{i}}, \label{eq:vib_ion}
\end{equation}
which requires evaluating Eq.~(\ref{eq:FN_ion}) over a range of internuclear separations.



\section{Results \& Discussion}
\label{sec:results}
In this section we present photoionization cross sections for the ground and excited vibrational levels of H$_{2}^{+}$.
Benchmark fixed--nuclei photoionization cross sections are presented and used to verify our results against previous calculations.
The model presented by \citet{Cohen1966} for the total cross section of H$_{2}^{+}$ is tested.
All cross sections are given in units of the Thomson cross section and presented as functions of the incident photon energy.
To obtain fixed-nuclei cross sections that represent an average over nuclear motion, calculations should be performed at the mean internuclear separation $R=2.06$ a$_{0}$. 
However, to compare with previous theoretical results we performed calculations at an internuclear separation of $R=2$ a$_{0}$.
Calculations are performed in both length and velocity gauges and are found to produce identical results for all cross sections and incident photon energies considered. For clarity, only the velocity gauge results are presented.





\subsection{Fixed--Nuclei Photoionization}
\label{sec:FN-ion}
The total photoionization cross section for the 1s$\sigma_{\mathrm{g}}$ state of H$_{2}^{+}$ is given by the solid line in Fig.~\ref{fig:ion_1sSg_total}.
This cross section is the sum of the photoionization cross sections for the molecule oriented with the internuclear axis parallel (dotted lines) and perpendicular (dot--dashed lines) to the photon polarization.
Going forward we will refer to these cross sections as the parallel and perpendicular photoionization cross sections.

\begin{figure}[!h]
	\centering
	\includegraphics[width=0.99\linewidth]{1sSg_total.eps}
	\caption{Total photoionization cross section for the ground electronic state of H$_{2}^{+}$. Solid lines are the total, parallel, and perpendicular cross sections.  We compare with the results of \citet{Fojon2006}, \citet{Richards1986}, \citet{Bates1968}, \citet{Brosolo1994}, \citet{Colgan2003}, and \citet{Tsednee2018}.} \label{fig:ion_1sSg_total}
\end{figure}

The perpendicular photoionization cross section is larger than the parallel photoionization cross section at all photon energies considered, the largest difference occurring near the ionization threshold.
The parallel and perpendicular photoionization cross sections presented here contain contributions from the lowest five partial waves, i.e. $\lambda = 1,3,5,7,9$.
For the energy range considered here higher partial waves are not required as convergence is achieved by $\lambda=9$.
Our cross sections are compared with those of \citet{Fojon2006}, \citet{Richards1986}, \citet{Bates1968}, \citet{Brosolo1994}, \citet{Colgan2003}, and \citet{Tsednee2018}. 
Excellent agreement is found in all cases expect for the results of \citet{Colgan2003}.


The p$\sigma_{\mathrm{u}}$, f$\sigma_{\mathrm{u}}$, h$\sigma_{\mathrm{u}}$ partial parallel and p$\pi_{\mathrm{u}}$, f$\pi_{\mathrm{u}}$, h$\pi_{\mathrm{u}}$ perpendicular photoionization cross sections are presented in Fig.~\ref{fig:ion_1sSg_para} and Fig.~\ref{fig:ion_1sSg_perp} respectively.
We compare our partial cross sections with the results of \citet{Fojon2006}, \citet{DellaPicca2011}, \citet{Richards1986}, and \citet{Bates1968}. 
Complete agreement with the results of \citet{Bates1968}, \citet{DellaPicca2011}, and \citet{Richards1986} is found. However, the results of Bates and \"Opik don't extend to photon energies high enough to reach the Cooper minimum in the f$\sigma_{\mathrm{u}}$ partial cross section and the energy grid of Richards and Larkins isn't dense enough at high energies to capture the structure of the Cooper minimum.
The partial cross sections of \citet{DellaPicca2011} using true wave functions have Cooper minima in the same locations as our cross sections.
Poor agreement is found with the partial cross sections of \citet{Fojon2006}, in particular the partial cross sections of \citet{Fojon2006} have Cooper minima in the p$\sigma_{\mathrm{u}}$ and p$\pi_{\mathrm{u}}$ partial cross sections that are not observed in ours.
The Cooper minimum in f$\sigma_{\mathrm{u}}$ partial cross section of \citet{Fojon2006} occurs at a much lower incident photon energy than ours.
These additional spurious Cooper minima observed in the partial photoionization cross sections of \citet{Fojon2006} are an artifact of an inaccurate description of the target structure.
We use a two--center basis expansion for bound states and true two--center continuum states while \citet{Fojon2006} uses a single--center B--spline basis expansion. 
The total photoionization cross section is less sensitive to the description of the target structure when convergence is achieved, despite the differences in our partial photoionization cross sections we find complete agreement with the total photoionization cross sections of \citet{Fojon2006}.


\begin{figure}[!h]
	\centering
	\includegraphics[width=0.99\linewidth]{1sSg_para_partial.eps}
	\caption{Partial photoionization cross sections to the $\sigma_{\mathrm{u}}$ continuum (parallel transition) from the ground electronic state of H$_{2}^{+}$. Solid lines are the present p$\sigma_{\mathrm{u}}$, f$\sigma_{\mathrm{u}}$, and h$\sigma_{\mathrm{u}}$ partial cross sections. We compare with the results of \citet{Fojon2006}, \citet{DellaPicca2011}, \citet{Richards1986}, and \citet{Bates1968}. The vertical dotted line are the expected locations of the Cooper minima according to the $k_{e}R\approx \ell \pi$ rule for $\ell=1$ and 3.} \label{fig:ion_1sSg_para}
\end{figure}

\begin{figure}[!h]
	\centering
	\includegraphics[width=0.99\linewidth]{1sSg_perp_partial.eps}
	\caption{Partial photoionization cross sections to the $\pi_{\mathrm{u}}$ continuum (perpendicular transition) from the ground electronic state of H$_{2}^{+}$. Solid lines are the p$\pi_{\mathrm{u}}$, f$\pi_{\mathrm{u}}$, and h$\pi_{\mathrm{u}}$ partial cross sections. We compare with the results of \citet{Fojon2006}, \citet{DellaPicca2011}, and \citet{Bates1968}.} \label{fig:ion_1sSg_perp}
\end{figure}


The vertical dotted lines in Fig.~\ref{fig:ion_1sSg_para} correspond to the expected locations of Cooper minima according to the $k_{e}R\approx \ell \pi$ rule.
We find that the p$\sigma_{\mathrm{u}}$ partial cross section does not exhibit a Cooper minimum and the Cooper minimum in the f$\sigma_{\mathrm{u}}$ partial cross section partial cross section occurs at an incident photon energy about 1 a.u. higher than predicted by the $k_{e}R\approx \ell \pi$ rule.



In Fig.~\ref{fig:ion_1sSg_analytic} we compare our total photoionization cross section with the model, Eq.~(\ref{eq:CF_model}), proposed by \citet{Cohen1966}. 
The ground state photoionization cross sections for He$^{+}$ was used as the atomic cross section in the Cohen--Fano model.
After shifting the model photoionization cross section to the correct threshold and rescaling it to match our photoionization cross section at the ionization threshold, we find that the Cohen--Fano model cross section has a similar overall shape but does not decay as quickly as our photoionization cross section.
This leads to significant differences in magnitude at higher incident photon energies.
The Cohen--Fano model makes three assumptions: (i) ionization is effectively a one--electron process, (ii) the initial state wave function is well described within the Linear Combination of Atomic Orbitals approximation, 
and (iii) the ionized electron is well described by a plane wave or single--center spherical wave.
In the case of H$_{2}^{+}$ only the first of these assumptions is valid, therefore it is not surprising that our photoionzation cross section differs from the Cohen--Fano model photoionization cross section.


\begin{figure}[!h]
	\centering
	\includegraphics[width=0.99\linewidth]{1sSg_analytic.eps}
	\caption{Photoionization cross section for the ground electronic state of H$_{2}^{+}$ with a comparison to the model proposed by \citet{Cohen1966}. The Cohen-Fano model cross section has been shifted and rescaled to match ours at threshold.} \label{fig:ion_1sSg_analytic}
\end{figure}


Total, parallel, and perpendicular photionization cross sections for the 2s$\sigma_{\mathrm{g}}$ state of H$_{2}^{+}$ are presented in Fig.~\ref{fig:ion_2sSg_total}.
Partial parallel and perpendicular photoionization cross sections for the 2s$\sigma_{\mathrm{g}}$ state of H$_{2}^{+}$ are shown in Fig.~\ref{fig:ion_2sSg_para} and Fig.~\ref{fig:ion_2sSg_perp} respectively.
Our results are compared with those of \citet{Bates1968} with perfect agreement being found in all cases.
The vertical dotted lines in Fig.~\ref{fig:ion_2sSg_para} indicate the expected locations of Cooper minima according to the $k_{e}R\approx \ell \pi$ rule. 
The Cooper minimum in our p$\sigma_{\mathrm{u}}$ partial photoionization cross section occurs at an incident photon energy $\sim 1$ a.u. lower than expected according to the $k_{e}R\approx \ell \pi$ rule. 
The Cooper minimum in the f$\sigma_{\mathrm{u}}$ partial photoionization cross section occurs at an incident photon energy $\sim 1$ a.u. higher than expected according to the $k_{e}R\approx \ell \pi$ rule.
The accuracy of the $k_{e}R\approx \ell \pi$ rule is expected to improve as $k_{e}$ and therefore $\ell$ increase, it is not expected that this rule will be particularly accurate in predicting the location of Cooper minima for the low values of $\ell$ considered here.


\begin{figure}[!h]
	\centering
	\includegraphics[width=0.99\linewidth]{2sSg_total.eps}
	\caption{Photoionization cross section for the 2s$\sigma_{\mathrm{g}}$ state of H$_{2}^{+}$. Solid lines are the total cross section, dashed lines are the parallel contributions, and the dot--dashed lines are the perpendicular contributions. We compare with the results of \citet{Bates1968}.} \label{fig:ion_2sSg_total}
\end{figure}

\begin{figure}[!h]
	\centering
	\includegraphics[width=0.99\linewidth]{2sSg_para_partial.eps}
	\caption{Partial photoionization cross sections for parallel transition from the 2s$\sigma_{\mathrm{g}}$ state of H$_{2}^{+}$. Solid lines are the total contribution from parallel tranistions, dotted lines are the p$\sigma_{\mathrm{u}}$ partial cross sections, dashed lines are the f$\sigma_{\mathrm{u}}$ partial cross sections, and the dot--dashed lines are the h$\sigma_{\mathrm{u}}$ partial cross sections. We compare with the results of \citet{Bates1968}. The vertical dotted line are the expected locations of the Cooper minima according to the $k_{e}R\approx \ell \pi$ rule.} \label{fig:ion_2sSg_para}
\end{figure}

\begin{figure}[!h]
	\centering
	\includegraphics[width=0.99\linewidth]{2sSg_perp_partial.eps}
	\caption{Partial photoionization cross sections for perpendicular transition from the 2s$\sigma_{\mathrm{g}}$ state of H$_{2}^{+}$. Solid lines are the total contribution from perpendicular tranistions, dotted lines are the p$\pi_{\mathrm{u}}$ partial cross sections, dashed lines are the f$\pi_{\mathrm{u}}$ partial cross sections, and the dot--dashed lines are the h$\pi_{\mathrm{u}}$ partial cross sections. We compare with the results of \citet{Bates1968}.} \label{fig:ion_2sSg_perp}
\end{figure}




\subsection{Vibrationally--Resolved Photoionization}
\label{sec:vib_ion}
In Sec.~\ref{sec:FN-ion} we presented a set of benchmark fixed--nuclei photoionization cross sections and found generally good agreement with previous results.
Having verified our fixed--nuclei cross sections we present a complete set of photoionization cross sections for the $v=0-18$ vibrational levels of the ground electronic state. 
Vibrationally--resolved photoionization cross sections can be approximated by a fixed--nuclei photoionization cross sections evaluated at the mean internuclear separation, $R_{m}=\braket{v_{i}|R|v_{i}}$, of a given vibrational level.



Such an approximation can be considered as a Taylor expansion of the vibrationally--resolved photoionization cross section about some internuclear separation $R_{0}$,
%
\begin{align}
\sigma_{n_{i}v_{i}}^{\mathrm{ion}} &\approx \sigma_{n_{i}}^{\mathrm{ion}}(R_{0}) + \frac{\rmd \sigma_{n_{i}}^{\mathrm{ion}} }{\rmd R} \Big|_{R_{0}} \braket{v_{i}|R-R_{0}|v_{i}} + \dots.
\end{align}
%
Choosing $R_{0}=R_{m}$ for a given vibrational level minimizes the error as the first order term vanishes and the approximation is then a second order approximation rather than first order.
The fixed--nuclei cross section evaluated at the mean internuclear separation of a given vibrational level is expected to give a good approximation to the corresponding vibrationally resolved cross section provided the electronic part of the photoionization cross section is non--zero in the region where it overlaps with the vibrational wave function.


In Fig.~\ref{fig:vib_vs_FN} our fixed--nuclei and $v=0$ total, parallel, and perpendicular photoionization cross sections are compared.
Excellent agreement is found between the $v=0$ and fixed--nuclei cross section at higher photon energies while poor agreement is found for incident photon energies near the ionization threshold. 
This disagreement is expected, as near the ionization threshold the assumption that ionization happens much faster than the vibrational period is no longer valid~\cite{Fojon2006}.

\begin{figure}[!h]
	\centering
	\includegraphics[width=0.99\linewidth]{vib_vs_FN.eps}
	\caption{Total, parallel, and perpendicular photoionization cross sections for the ground vibrational leve of the 1s$\sigma_{\mathrm{g}}$ state of H$_{2}^{+}$ with comparison to the corresponding fixed--nuclei results.} \label{fig:vib_vs_FN}
\end{figure}

\begin{figure}[!h]
	\centering
	\includegraphics[width=0.99\linewidth]{vib_vs_FN_excited.eps}
	\caption{The $v=0$, 6, 11, and 17 photoionization cross sections for the 1s$\sigma_{\mathrm{g}}$ state of H$_{2}^{+}$ compared with fixed--nuclei results calculated at the corresponding mean internuclear separations.} \label{fig:vib_vs_FN_excited}
\end{figure}


In Fig.~\ref{fig:vib_vs_FN_excited} the $v=0$, 6, 11, and 17 photoionization cross sections are compared with the corresponding fixed--nuclei results calculated at their respective mean internuclear separations.
Good agreement between the fixed--nuclei and vibrationally--resolved cross sections is found at high photon energies while agreement is poor near the respective ionization thresholds.
Similar agreement is found for all other excited vibrational levels provided the fixed--nuclei calculation is performed at the mean internuclear separation for the given vibrational level.
Clearly the fixed--nuclei approximation cannot reproduce the correct near threshold photoionization cross section for any vibrational level even when the calculations are performed at the mean internuclear separation.
An accurate account of vibrational motion is required to obtain vibrationally--resolved photoionization cross sections with the correct near threshold behavior.





Total, parallel, and perpendicular photoionization cross sections for all bound vibrational states of the 1s$\sigma_{\mathrm{g}}$ state are shown in Fig.~\ref{fig:vib_all}.
The total photoionization cross sections are typically dominated by perpendicular contributions.
For $v\geq 4$, the cross sections near their respective thresholds increase significantly in magnitude as the vibrational level $v$ increases.
The increase in the magnitude of the cross section as $v$ increasing is also observed in electron scattering~\cite{Scarlett2021}
However, for $v< 4$ the near threshold cross section decreases with increasing $v$.
This behaviour in the total cross section follows from the perpendicular photoionization cross sections.
Oscillations in the cross sections just above the ionization threshold can also be seen. 
We believe these oscillations are physical as the only approximation used in these calculations is the Born--Oppenheimer approximation. 
Similar behaviour is also observed in electron scattering cross sections \cite{Scarlett2021}.
The ionization thresholds also shifts to lower energies as the initial vibrational level increases.

\begin{figure}[!h]
	\centering
	\includegraphics[width=0.99\linewidth]{vib_all.eps}
	\caption{Total, parallel, and perpendicular photoionization cross sections for all bound vibrational levels of the 1s$\sigma_{\mathrm{g}}$ state of H$_{2}^{+}$.} \label{fig:vib_all}
\end{figure}

In Fig.~\ref{fig:vib_test} the origin of these oscillations is investigated.
The top panel of Fig.~\ref{fig:vib_test} shows the $v=1$ photoionization cross section in the region of the oscillations, while the bottom panel shows the components of the integrand in Eq.~(\ref{eq:vib_ion}) used to calculate the cross section in the top panel.
It can be seen that the overlap of the squared $v=1$ vibrational wave function with the electronic part of the photoionization cross section, Eq.~(\ref{eq:FN_ion}), increases as the incident photon energy increases until a node in the vibrational wave function is reached.
When a node is reached the overall overlap between the wave function squared and the cross section decreases. 
The overlap gained as the first non--zero point shifts to lower internuclear separations is small compared to overlap lost at larger $R$ due the decreasing nature of the cross section.
The overlap then begins to increase again as the first non-zero point of the cross section shifts away from the node to even lower internuclear separations.
Therefore the near threshold oscillations in the photoionization cross sections for vibrationally excited states is due to nodes in the excited vibrational wave functions.

The shift of the ionization threshold to lower energies as $v$ increases can also be explained by the by the overlap between the electronic part of the photoionization cross section and the norm squared of the vibrational wave function.
The vibrational wave function extends to larger internuclear separations as the vibrational level increases.
It can been seen in Fig.~\ref{fig:vib_test} at lower incident photon energies the first non--zero point of the electronic part of the photoionization cross section occurs at a larger internuclear separation.
Therefore, higher vibrational levels have wave functions that overlap with the electronic part of the photoionization cross section at lower incident photon energies, this results in the shift of the ionization threshold to lower energies.

 
\begin{figure}[!h]
	\centering
	\includegraphics[width=0.99\linewidth]{vib_test.eps}
	\caption{(Top) The near threshold $v=1$ photoionization cross section as a function of incident photon energy.
	(Bottom) Norm squared of the $v=1$ vibrational wave function and the electronic part of the photoionization cross section, Eq.~(\ref{eq:FN_ion}), as a function of internuclear separation $R$. The first non-zero point of the electronic part of the photoionisation cross section shifts to lower value of $R$ as the incident photon energy increases. The corresponding points in the top panel are indicated with dots of the same color.} \label{fig:vib_test}
\end{figure}







\section{Conclusion}
\label{sec:conclusion}
We have calculated cross sections for photoionization from all bound vibrational states of H$_{2}^{+}$ and presented a benchmark set of fixed-nuclei photoionization cross sections.
Total and partial photoionization cross sections are calculated using true electronic continuum wave functions and our benchmark fixed--nuclei results are found to be in excellent agreement with available theoretical results in many cases, but not all.
We find an accurate two--center description of the target is required to obtain accurate partial photoionization cross sections.
Specifically that true two--center continuum states are required to obtain Cooper minima in the fixed--nuclei cross sections at the correct incident photon energies, and that an inadequate description of the target leads to appearance of spurious Cooper minima.
It was found that the Cohen--Fano model photoionization cross section can describe the overall shape of the H$_{2}^{+}$ photoionization cross section, but does not reproduce the correct magnitude.
The near threshold cross section for photoionization from vibrationally excited levels of H$_{2}^{+}$ was found to contain oscillations.
These oscillations were shown to be due to nodes in the vibrational wave functions. 
The magnitude of the vibrationally--resolved photoionization cross sections increases as the initial vibrational level increases, while the ionization threshold shifts to lower energies.
The target structure formulated here and previously develop photon scattering techniques \cite{Singor2020,Singor2021} will be utilized to calculate Rayleigh and Raman scattering cross sections for H$_{2}^{+}$, and H$_{2}$ in future work.




\section*{Acknowledgements}
This work was supported by the Government of Western Australia Defense Science Centre Research Higher Degree Student Grant, and by the United States Air Force Office of Scientific Research, grant number FA2386-19-1-4044.
HPC resources were provided by the Pawsey Supercomputing Centre with funding from the Australian Government and the Government of Western Australia. 
A.S. acknowledges the contribution of an Australian Government Research Training Program Scholarship. 






\newpage

\onecolumngrid
\appendix

\setcounter{section}{0}
\section{Hamiltonian and Overlap Matrix Elements}
The overlap between the basis functions, Eq.~(\ref{eq:basis}), with the volume element
\begin{equation}
\rmd V = \left[\left(\rho+\tfrac{R}{2} \right)^{2} - \left(\tfrac{R}{2}\eta \right)^{2} \right]\rmd\rho\rmd\eta \rmd \phi,
\end{equation}
is given by~\cite{SavagePhD}
\begin{align}\label{eqA:ovrlap}
\braket{k'\ell'm'|k\ell m} &= \int \rmd V \,\varphi^{*}_{k'\ell'm'}(\rho,\eta,\phi)\varphi_{k\ell m}(\rho,\eta,\phi) \nonumber \\
&=  \delta_{\ell'\ell}\delta_{m'm}b_{k'k}^{m} - \delta_{k'k}\tfrac{R^{2}}{6}C_{\ell'020}^{\ell0}C_{\ell m 20}^{\ell'm'},
\end{align}
where 
%
\begin{equation}
b_{k'k}^{m} = \begin{cases}
\lambda_{m}^{-2}\sqrt{k(k+m)(k+1)(k+m+1)} & k'=k+2,  \\
-\lambda_{m}^{-2}\sqrt{k(k+m)} (4k+2m+\lambda_{m}R) & k'=k+1,  \\
\lambda_{m}^{-2}(6k^{2}-6k+6km -3m+m^{2}+2) +\tfrac{2k-1+m}{\lambda_{m}} + \tfrac{R^{2}}{6} & k'=k,  \\
-\lambda_{m}^{-2}\sqrt{(k-1)(k-1+m)}(4k-4+2m+\lambda_{m}R) & k'=k-1,  \\
\lambda_{m}^{-2}\sqrt{(k-1)(k-1+m)(k-2)(k-2+m)} & k'=k-2, \\
0 &$otherwise$,
\end{cases}
\end{equation}
and $C_{\ell m \ell' m'}^{LM}$ is a Clebsh--Gordon Coefficient.
%
The one--electron hamiltonian matrix elements in this basis are
\begin{align}\label{eqA:Ham_me}
\braket{k'\ell' m'|H|k\ell m} = \delta_{\ell'\ell}\delta_{m'm}\Bigg[ h_{k'k}^{\ell m} - \frac{m^{2}R}{8}\int_{0}^{\infty}\!\rmd \rho\, \frac{f_{k'}^{m}(\rho)f_{k}^{m}(\rho)}{\rho+R}\Bigg],
\end{align}
where
\begin{equation}
h_{k'k}^{\ell m} = \begin{cases}
-\tfrac{1}{8}\sqrt{k(k+m)(k+1)(k+1+m)} & k'=k+2, \\
\tfrac{1}{4}\sqrt{k(k+m)}\left(\tfrac{\lambda_{m}R}{2} - \tfrac{8}{\lambda_{m}} \right) & k'=k+1, \\
\tfrac{1}{4}\bigg[k^{2}-k+km-\tfrac{m}{2} +1+4R+2\ell(\ell+1) + (2k-1+m)\left(\tfrac{\lambda_{m}R}{2}+ \tfrac{8}{\lambda_{m}} \right) \bigg] & k'=k, \\
\tfrac{1}{4}\sqrt{(k-1)(k-1+m)}\left(\tfrac{\lambda_{m}R}{2} - \tfrac{8}{\lambda_{m}} \right) & k'=k-1, \\
-\tfrac{1}{8}\sqrt{(k-1)(k-1+m)(k-2)(k-2+m)} & k'=k-2, \\
0 &$otherwise$.
\end{cases}
\end{equation}



\section{Dipole Matrix Elements}
\label{App:DipME}
The dipole operator can be given in the length $(\mathsf{L})$ or velocity $(\mathsf{V})$ gauge,
\begin{equation}
\braket{n'\lambda'|d_{\kappa}|n\lambda} = \braket{n'\lambda'|\mathsf{L}_{\kappa}|n\lambda} = \frac{\braket{n'\lambda'|\mathsf{V}_{\kappa}|n\lambda}}{E-E'},
\end{equation}
where the length gauge dipole operator is
%
\begin{align}
\mathsf{L}_{\kappa} &= \mathcal{V}_{\kappa}(\rho) \sqrt{\tfrac{4\pi}{3}} \,Y_{1}^{\kappa}(\eta,\phi), \\
\mathcal{V}_{\kappa}(\rho) &= \begin{cases}
\rho +\tfrac{R}{2} & \kappa=0, \\
\sqrt{\rho(\rho+R)} & \kappa=\pm1,
\end{cases}
\end{align}
%
and the velocity gauge dipole operator is
%
\begin{align}
\mathsf{V}_{\kappa} &= \begin{cases}
\dfrac{1}{\rho(\rho+R) + \frac{R^{2}}{4}(1-\eta^{2})}\left[\rho(\rho+R)\eta\dfrac{\partial}{\partial\rho} + (1-\eta^{2})\left(\rho+\frac{R}{2}\right)\dfrac{\partial}{\partial\eta} \right] & \kappa=0, \\[10pt]
\dfrac{\mp\rme^{\pm\rmi\phi}}{\sqrt{2}}\dfrac{\sqrt{\rho(\rho+R)}}{\rho(\rho+R) + \frac{R^{2}}{4}(1-\eta^{2})}\left[\sqrt{1-\eta^{2}}\left(\rho+\frac{R}{2} \right)\dfrac{\partial}{\partial\rho} \pm \rmi \dfrac{R^{2}}{4} \dfrac{\sqrt{1-\eta^{2}}}{\rho(\rho+R)}\dfrac{\partial}{\partial\phi}  \right. \\
\hspace{140pt}-\left. \left(\eta\sqrt{1-\eta^{2}}\dfrac{\partial}{\partial\eta} \mp \dfrac{\rmi}{\sqrt{1-\eta^{2}}}\dfrac{\partial}{\partial\phi} \right)\right] & \kappa=\pm1.
\end{cases}
\end{align}
%
The components of the electronic length gauge dipole matrix elements are given by~\cite{Zammit2017}
\begin{align}
\braket{n'\lambda'|\mathsf{L}_{\kappa}|n\lambda} = \sum_{k'\ell'}\sum_{k\ell}\int_{0}^{\infty}\rmd\rho\, C_{k'\ell'm'} f_{k'}^{m'}(\rho) \mathcal{V}_{\kappa}(\rho) C_{k\ell m} f_{k}^{m}(\rho)J_{1,\kappa}^{(n',n)}(\rho),
\end{align}
where $C_{k\ell m}$ are the expansion coefficients in Eq.~(\ref{eq:bound_st_expansion}) and $J_{1,\kappa}^{n',n}(\rho)$ is the angular integral with the volume element,
%
\begin{align}
J_{L,M}^{(n',n)}(\rho) =& \int_{0}^{2\pi}\int_{-1}^{1}\rmd\eta\rmd\phi\, Y_{\ell'}^{m'*}(\eta,\phi) \sqrt{\tfrac{4\pi}{2L+1}}Y_{L}^{M}(\eta,\phi)Y_{\ell}^{m}(\eta,\phi) \left[ \rho(\rho+R)+\tfrac{R^{2}}{4}(1-\eta^{2}) \right].
\end{align}
%
The angular integral can be evaluated analytically
\begin{align}
J_{1,0}^{(n',n)}(\rho) =&  \left[\rho(\rho+R)+\tfrac{R^{2}}{10} \right]A_{m'}^{\ell'}{}_{0}^{1}{}_{m}^{\ell} - \tfrac{R^{2}}{10} A_{m'}^{\ell'}{}_{0}^{3}{}_{m}^{\ell}, \\
J_{1,\pm1}^{(n',n)}(\rho) =&  \left[\rho(\rho+R)+\tfrac{R^{2}}{5} \right]A_{m'}^{\ell'}{}_{\pm1}^{1}{}_{m}^{\ell} - \tfrac{R^{2}}{5\sqrt{6}} A_{m'}^{\ell'}{}_{\pm1}^{3}{}_{m}^{\ell},
\end{align}
where 
\begin{align}
A_{m_{1}}^{\ell_{1}}{}_{m_{2}}^{\ell_{2}}{}_{m_{3}}^{\ell_{3}} &= \int_{0}^{2\pi}\int_{-1}^{1} \!\rmd\eta\rmd\phi\, Y_{\ell_{1}}^{m_{1}*}(\eta,\phi) \sqrt{\tfrac{4\pi}{2\ell_{2}+1}} Y_{\ell_{2}}^{m_{2}}(\eta,\phi) Y_{\ell_{3}}^{m_{3}}(\eta,\phi) \nonumber \\
&= (-1)^{\ell_{2}}\,C_{\ell_{1}0\ell_{2}0}^{\ell_{3}0}\,C_{\ell_{3}m_{3}\ell_{2}m_{2}}^{\ell_{1}m_{1}}.
\end{align}
In the velocity gauge the components of the electronic dipole operator are
\begin{align}
\braket{n'\lambda'|\mathsf{V}_{0}|n\lambda} &= \sum_{k'\ell'}\sum_{k\ell}\int_{0}^{\infty}\rmd\rho\, C_{k'\ell'm'} f_{k'}^{m'}(\rho) \left[ \rho(\rho+R)\frac{\partial}{\partial\rho} + y_{\ell'\ell}\left(\rho+\tfrac{R}{2} \right)\right]A_{m'}^{\ell'}{}_{0}^{1}{}_{m}^{\ell} C_{k\ell m} f_{k}^{m}(\rho) , \\[7pt]
%
\braket{n'\lambda'|\mathsf{V}_{\pm1}|n\lambda} &= -\sum_{k'\ell'}\sum_{k\ell}\int_{0}^{\infty}\rmd\rho\, C_{k'\ell'm'} f_{k'}^{m'}(\rho) 
 \sqrt{\rho(\rho+R)}\left[ \left(\rho+\tfrac{R}{2}\right)\frac{\partial}{\partial\rho} + \frac{\abs{m}R^{2}}{4\rho(\rho+R)} +y_{\ell'\ell}\right]A_{m'}^{\ell'}{}_{\pm1}^{1}{}_{m}^{\ell} \nonumber \\
&\hspace{150pt} \times C_{k\ell m} f_{k}^{m}(\rho),
\intertext{where}
y_{\ell'\ell} &= \begin{cases}
-\ell &\ell'=\ell+1, \\
\ell+1 &\ell'=\ell-1.
\end{cases}
\end{align}




\newpage















\section*{References}
\bibliographystyle{apsrev}
\bibliography{references}

\end{document}

