\section{Conclusions and Future Work}\label{sec:conclusion}


Our work provides the first systematic attempt to experimentally evaluate algorithms devised for the semi-oblivious chase termination problem in the presence of (simple-)linear TGDs.
%
Our analysis revealed that for simple-linear TGDs, we can efficiently check whether the chase terminates even for very large databases and sets of TGDs.
%
Concerning linear TGDs, the overall runtime of the algorithm is quite reasonable, but there is still room for improvements. Interestingly, our analysis showed that the algorithm for linear TGDs consists of two separate components, the db-dependent and the db-independent components. This modular nature of the algorithm allows us to study and improve the two components separately. In particular, we have observed that the heavy component is the db-dependent one, and thus, we can focus our future efforts to improve the performance of that component. Although our analysis relied on an in-database and an in-memory implementation of the procedure for finding the shapes, we could adopt other techniques depending on the underlying application without affecting the db-independent component. An interesting direction is to materialize and incrementally keep updated the shapes in a database, which will improve the performance of the db-dependent component. 