\section{Preliminaries} \label{sec:prel}
%

We consider the disjoint countably infinite sets $\ins{C}$, $\ins{N}$, and $\ins{V}$ of {\em constants}, {\em (labeled) nulls}, and {\em variables}, respectively. We refer to constants, nulls and variables as {\em terms}. For an integer $n > 0$, we write $[n]$ for the set $\{1,\ldots,n\}$.


\medskip


\noindent 
\textbf{Relational Databases.} A {\em schema} $\ins{S}$ is a finite set of relation symbols (or predicates) with associated arity. We write $R/n$ to denote that $R$ has arity $n > 0$; we may also write $\arity{R}$ for the integer $n$.
%
A {\em (predicate) position} of $\ins{S}$ is a pair $(R,i)$, where $R/n \in \ins{S}$ and $i \in [n]$, that essentially identifies the $i$-th argument of $R$. We write $\pos{\ins{S}}$ for the set of positions of $\ins{S}$, that is, the set $\{(R,i) \mid R/n \in \ins{S} \text{ and } i \in [n]\}$.
%
An {\em atom} over $\ins{S}$ is an expression of the form $R(\bar t)$, where $R/n \in \ins{S}$ and $\bar t$ is an $n$-tuple of terms. A {\em fact} is an atom whose arguments consist only of constants.
%
For a variable $x$ in $\bar t = (t_1,\ldots,t_n)$, let $\posvar{R(\bar t)}{x} = \{(R,i) \mid t_i = x\}$. 
%
We write $\var{R(\bar t)}$ for the set of variables in $\bar t$. The notations $\posvar{\cdot}{x}$ and $\var{\cdot}$ extend to sets of atoms.
%
An {\em instance} over $\ins{S}$ is a (possibly infinite) set of atoms over $\ins{S}$ with constants and nulls. A {\em database} over $\ins{S}$ is a finite set of facts over $\ins{S}$. The {\em active domain} of an instance $I$, denoted $\adom{I}$, is the set of terms (constants and nulls) occurring in $I$. For a singleton instance $\{\alpha\}$, we simply write $\adom{\alpha}$ instead of $\adom{\{\alpha\}}$.
%For a finite instance $I$, we write $|I|$ for the number of atoms occurring in it, i.e., the cardinality of the set $I$.


\medskip


\noindent
\textbf{Substitutions and Homomorphisms.}
A {\em substitution} from a set of terms $T$ to a set of terms $T'$ is a function $h : T \ra T'$. Henceforth, we treat a substitution $h$ as the set of mappings $\{t \mapsto h(t) \mid t \in T\}$.
%
The restriction of $h$ to a subset $S$ of $T$, denoted $h_{|S}$, is the substitution $\{t \mapsto h(t) \mid t \in S\}$.
%
A {\em homomorphism} from a set of atoms $A$ to a set of atoms $B$ is a substitution $h$ from the set of terms in $A$ to the set of terms in $B$ such that $h$ is the identity on $\ins{C}$, and $R(t_1,\ldots,t_n) \in A$ implies $h(R(t_1,\ldots,t_n)) =  R(h(t_1),\ldots,h(t_n)) \in B$.
%An isomorphism between two tuples $\tuple{t} = (t_1,\ldots,t_n)$, $\tuple{u}=(u_1,\ldots,u_n)$ is a bijection $h : \{t_1,\ldots,t_n\} \rightarrow \{u_1,\ldots,u_n\}$; $\tuple{t}$ and $\tuple{u}$ are isomorphic, if an isomorphism between $\tuple{t}$ and $\tuple{u}$ exists. Two atoms $R(\bar t)$ and $R(\bar u)$ are isomorphic if $\bar t$ and $\bar u$ are isomorphic.


\medskip


\noindent
\textbf{Tuple-Generating Dependencies.} A {\em tuple-generating dependency} (TGD) $\sigma$ is a (constant-free) sentence
$
\forall \bar x \forall \bar y \left(\phi(\bar x,\bar y) \ra \exists \bar z\, \psi(\bar x,\bar z)\right),
$
where $\bar x, \bar y$ and $\bar z$ are tuples of variables of $\ins{V}$, and $\phi(\bar x,\bar y)$ and $\psi(\bar x,\bar z)$ are non-empty conjunctions of atoms that mention only variables from $\bar x \cup \bar y$ and $\bar x \cup \bar z$, respectively. Note that, by abuse of notation, we may treat a tuple of variables as a set of variables.
%
We write $\sigma$ as $\phi(\bar x,\bar y) \ra \exists \bar z\, \psi(\bar x,\bar z)$, and use comma instead of $\wedge$ for joining atoms. We refer to $\phi(\bar x,\bar y)$ and $\psi(\bar x,\bar z)$ as the {\em body} and {\em head} of $\sigma$, denoted $\body{\sigma}$ and $\head{\sigma}$, respectively.
%
The {\em frontier} of the TGD $\sigma$, denoted $\fr{\sigma}$, is the set of variables $\bar x$, i.e., the variables that appear both in the body and the head of $\sigma$. 
%
The {\em schema} of a set $\dep$ of TGDs, denoted $\sch{\dep}$, is the set of predicates occurring in $\dep$.
%and we write $\arity{\dep}$ for the maximum arity over all those predicates. 
%
%We assume, w.l.o.g., that no two TGDs of $\dep$ share a variable, and we let $||\dep|| = |\atoms{\dep}| \cdot |\sch{\dep}| \cdot \arity{\dep}$, where $\atoms{\dep}$ is the set of atoms occurring in the TGDs of $\dep$.
%
An instance $I$ satisfies a TGD $\sigma$ as the one above, written $I \models \sigma$, if whenever there exists a homomorphism $h$ from $\phi(\bar x, \bar y)$ to $I$, then there is $h' \supseteq h_{|\bar x}$ that is a homomorphism from $\psi(\bar x,\bar z)$ to $I$; we may treat a conjunction of atoms as a set of atoms. The instance $I$ satisfies a set $\dep$ of TGDs, written $I \models \dep$, if $I \models \sigma$ for each $\sigma \in \dep$.

\medskip

\noindent
\textbf{Linearity.} A TGD is called {\em linear} if it has only one body-atom, and the corresponding class that collects all the finite sets of linear TGDs is denoted $\class{L}$. We further call a linear TGD {\em simple} if no variable occurs more than once in its body-atom, and the corresponding class is denoted $\class{SL}$. It is clear that $\class{SL} \subsetneq \class{L}$.