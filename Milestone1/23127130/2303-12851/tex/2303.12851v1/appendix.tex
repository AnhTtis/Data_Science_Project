\section{Db-Independent Component} \label{sec:smaller-pred-profiles}


\begin{figure}[t]
	\centering
	\begin{subfigure}[b]{0.23\textwidth}
		\centering
		\includegraphics[width=\textwidth]{scatter-linear-total-d1.pdf}
		\caption{\ttotal}
		\label{fig:linear-total-d1}
	\end{subfigure}
	\hfill
	\begin{subfigure}[b]{0.23\textwidth}
		\centering
		\includegraphics[width=\textwidth]{scatter-linear-parse-graphcomponent-d1.pdf}
		\caption{\tparse vs. \tgraph+\tcomp}
		\label{fig:linear-parse-graphcomponent-d1}
	\end{subfigure}
	\hfill
	\begin{subfigure}[b]{0.23\textwidth}
		\centering
		\includegraphics[width=\textwidth]{scatter-linear-graph-component-d1.pdf}
		\caption{\tgraph vs. \tcomp}
		\label{fig:linear-graph-component-d1}
	\end{subfigure}
	\hfill
	\begin{subfigure}[b]{0.23\textwidth}
		\centering
		\includegraphics[width=\textwidth]{scatter-linear-component-d1.pdf}
		\caption{\tcomp}
		\label{fig:linear-graphcomponent-d1}
	\end{subfigure}
	\caption{Runtime of the db-independent component for the predicate profile [5,200].}
	\label{fig:linear-time-5-200}
\end{figure}


\begin{figure}[t]
	\centering
	\begin{subfigure}[b]{0.23\textwidth}
		\centering
		\includegraphics[width=\textwidth]{scatter-linear-total-d2.pdf}
		\caption{\ttotal}
		\label{fig:linear-total-d2}
	\end{subfigure}
	\hfill
	\begin{subfigure}[b]{0.23\textwidth}
		\centering
		\includegraphics[width=\textwidth]{scatter-linear-parse-graphcomponent-d2.pdf}
		\caption{\tparse vs. \tgraph+\tcomp}
		\label{fig:linear-parse-graphcomponent-d2}
	\end{subfigure}
	\hfill
	\begin{subfigure}[b]{0.23\textwidth}
		\centering
		\includegraphics[width=\textwidth]{scatter-linear-graph-component-d2.pdf}
		\caption{\tgraph vs. \tcomp}
		\label{fig:linear-graph-component-d2}
	\end{subfigure}
	\hfill
	\begin{subfigure}[b]{0.23\textwidth}
		\centering
		\includegraphics[width=\textwidth]{scatter-linear-component-d2.pdf}
		\caption{\tcomp}
		\label{fig:linear-graphcomponent-d2}
	\end{subfigure}
	\caption{Runtime of the db-independent component for the predicate profile [200,400].}
	\label{fig:linear-time-200-400}
\end{figure}


Recall that, for the sake of readability, in Section~\ref{sec:linear-ex} we omitted the plots that present the runtime of the db-independent component of the chase termination algorithm $\mathsf{IsChaseFinite[L]}$ for the smaller predicate profiles.
%
The plot for the predicate profile [5,200] is shown in Figure~\ref{fig:linear-time-5-200}, whereas for the profile [200,400] in Figure~\ref{fig:linear-time-200-400}.
%
We can observe, similarly to the predicate profile [400,600], that the time parameters \tparse and \tgraph increase linearly as long as we increase \nrule, whereas \tcomp increases very slowly. However, it is clear from the plots that, as long as we move to smaller predicate profiles and larger sets of TGDs, the above linear trends become less apparent. 
%
This phenomenon can be explained as follows. As we decrease the number of predicates that appear in the schema, it is likely that a large set of TGDs gives rise to a limited number of edges in the underlying dependency. This is because many TGDs simply lead to the same edges, which are of course considered once in the graph. This is confirmed by the following plot

%\smallskip

\centerline{\includegraphics[width=.23\textwidth]{nedges-nrules-linear.pdf}}

%\smallskip

\noindent which depicts the average number of edges in the dependency graphs of the sets of linear TGDs used in our experimental evaluation. For the smaller predicate profiles, increasing the number of TGDs leads to a smaller increase in the graph size compared with the larger predicate profiles. In particular, for the profile [5,200], the number of edges in the dependency graphs remains almost the same as we increase the number of TGDs.

