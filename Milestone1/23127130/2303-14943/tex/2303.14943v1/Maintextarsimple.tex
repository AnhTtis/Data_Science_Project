\documentclass[twocolumn,preprintnumbers,amsmath,amssymb]{revtex4}
\usepackage{amsmath}
\usepackage{rotating}
\usepackage{extarrows}
\usepackage{graphicx}% Include figure files
\usepackage{dcolumn}% Align table columns on decimal point
\usepackage{bm}% bold math
\usepackage[all]{xy}
\usepackage{indentfirst}
\usepackage{amsmath}
\usepackage{bbm}
\usepackage{multirow}
\usepackage{mathrsfs}
\usepackage{euscript}
\usepackage{amssymb}
\newcommand{\ketbra}[2]{|#1\rangle \langle #2|}
\newcommand{\ket}[1]{|#1\rangle}
\newcommand{\bra}[1]{\langle #1|}
\newcommand{\tr}{\textrm{tr}}
\newcommand{\ham}{\hat{H}}
\newcommand{\ha}{\hat{a}}
\newcommand{\hb}{\hat{b}}
\newcommand{\hc}{\hat{c}}
\newcommand{\hx}{\hat{x}}
\newcommand{\hy}{\hat{y}}
\newcommand{\hz}{\hat{z}}
\newcommand{\hw}{\hat{w}}
\newcommand{\hu}{\hat{u}}
\newcommand{\hv}{\hat{v}}
\newcommand{\hp}{\hat{p}}
\newcommand{\hS}{\hat{S}}
\newcommand{\va}{\vec{a}}
\newcommand{\vx}{\vec{x}}
\newcommand{\dt}{\diamondsuit}

\newcommand{\oI}{\overline{I}}
\newcommand{\hsA}{\hat{\textsf{A}}}
\newcommand{\hsB}{\hat{\textsf{B}}}
\newcommand{\hsC}{\hat{\textsf{C}}}
\newcommand{\hA}{\hat{A}}
\newcommand{\hB}{\hat{B}}
\newcommand{\hC}{\hat{C}}
\newcommand{\hX}{\hat{X}}
\newcommand{\hY}{\hat{Y}}
\newcommand{\sA}{\textsf{A}}
\newcommand{\sB}{\textsf{B}}
\newcommand{\sC}{\textsf{C}}
\newcommand{\bI}{\mathbbm{1}}

\newcommand{\cE}{\mathcal{E}}
\newcommand{\cA}{\mathcal{A}}
\newcommand{\cB}{\mathcal{B}}
\newcommand{\cC}{\mathcal{C}}
\newcommand{\cH}{\mathcal{H}}
\newcommand{\cW}{\mathcal{W}}
\newcommand{\cX}{\mathcal{X}}
\newcommand{\cD}{\mathcal{D}}
\newcommand{\cM}{\mathcal{M}}
\newcommand{\cN}{\mathcal{N}}

\newtheorem{theorem}{Theorem}
\newtheorem{definition}{Definition}
\newtheorem{proposition}{Proposition}
\newtheorem{lemma}{Lemma}
\newtheorem{corollary}{Corollary}
\newtheorem{example}{Example}
%%%%%%%%%%%%%%%%%%%%%%%%%%%%%%%%%%%%%%%%%%%%%%%%%%%%%%%%%%%%%%%%%%%%%%%%%%%%%
\begin{document}
%%%%%%%%%%%%%%%%%%%%%%%%%%%%%%%%%%%%%%%%%%%%%%%%%%%%%%%%%%%%%%%%%%%%%%%%%%%%%
%%%%%%%%%%%%%%%%%%%%%%%%%%%%%%%%%%%%%%%%%%%%%%%%%%%%%%%%%%%%%%%%%%%%%%%%%%%%%
\title{Genuine Multipartite Nonlocality for All Isolated Many-body Systems}
%%%%%%%%%%%%%%%%%%%%%%%%%%%%%%%%%%%%%%%%%%%%%%%%%%%%%%%%%%%%%%%%%%%%%%%%%%%%%
\author{Ming-Xing Luo$^{1,2}$, Shao-Ming Fei$^{3,4}$}

\affiliation{\small 1. School of Information Science and Technology,
\\
\small Southwest Jiaotong University, Chengdu 610031, China;
\\
\small 2. CAS Center for Excellence in Quantum Information and Quantum Physics, Hefei, 230026, China;
\\
3. School of Mathematical Sciences, Capital Normal University, Beijing 100048, China;
\\
4. Max-Planck-Institute for Mathematics in the Sciences, 04103 Leipzig, Germany.}

\begin{abstract}
Understanding the nonlocality of many-body systems offers valuable insights into the behaviors of these systems and may have practical applications in quantum simulation and quantum computing. Gisin's Theorem establishes the equivalence of three types of quantum correlations: Bell nonlocality, EPR-steering, and entanglement for isolated systems. No similar result exists with regard to genuine multipartite correlations. We answer this open problem by proposing a new network-inflation method. Our approach demonstrates that genuine multipartite nonlocality, genuine multipartite EPR-steering, and genuine multipartite entanglement are equivalent for any isolated many-body system. This is achieved through an extended Bell test on an inflated network consisting of multiple copies of the given sources. The device-independent method is also robust against noise.
\end{abstract}
%%%%%%%%%%%%%%%%%%%%%%%%%%%%%%%%%%%%%%%%%%%%%%%%%%%%%%%%%%%%%%%%%%%%%%%%%%%%%
%%%%%%%%%%%%%%%%%%%%%%%%%%%%%%%%%%%%%%%%%%%%%%%%%%%%%%%%%%%%%%%%%%%%%%%%%%%%%
%%%%%%%%%%%%%%%%%%%%%%%%%%%%%%%%%%%%%%%%%%%%%%%%%%%%%%%%%%%%%%%%%%%%%%%%%%%%%
\maketitle


Bell's Theorem displays that entangled quantum systems can exhibit correlations that violate certain restrictions which are valid for any local realistic theory \cite{Bell}.
This provides a novel way to address the issue raised by the Einstein-Podolsky-Rosen (EPR) argument \cite{EPR}, which suggests that a local operation by one observer can remotely manipulate the state of another observer, even if they are far apart. The so-called Bell nonlocality indicates more profound quantum effects than those explained by EPR's steering \cite{EPR,WJD07,HHH}. Another method involves detecting entanglement witness beyond all separable states in a device-dependent manner \cite{HHH}. However, this approach generally yields fewer quantum correlations than Bell nonlocality and EPR steering \cite{WJD07,EPR,Werner}. Interestingly, Gisin's proof using the Clauser-Horne-Shimony-Holt (CHSH) inequality \cite{CHSH} demonstrates that these three correlations are equivalent for isolated systems \cite{Gisin}.

There have been attempts to extend Gisin's Theorem to many-body systems. One such effort, proposed by Popescu and Rohrlich \cite{PR}, involves activating the bipartite pure state entanglement with the assistance of other observers' local operations and classical communication (LOCC). This can lead to a violation of the CHSH inequality \cite{CHSH} under postselection, as well as an extension to high-dimensional systems \cite{Fei,Cola}. Another method involves completing the CHSH test using two subsets of observers and local joint measurements for each group. The third approach is inspired by Hardy's method \cite{Hardy,Chou,Yu,Chen}. Unfortunately, these methods only justify the bipartite correlations and not address the genuine multipartite correlations \cite{Sy,Mermin,RMP}.

Genuine multipartite nonlocality (GMN) identifies correlations from multiple-particle systems by ruling out classical mixtures of bipartite separable correlations \cite{Sy,Mermin}. This is generally stronger than bipartite nonlocality, as even biseparable correlations can show bipartite nonlocality under specific bipartitions of the system due to local quantumness or post-quantumness such as Popescu-Rohrlich (PR)-box sources \cite{PR}. The diverse bipartite correlations have led to further extensions in the form of genuine multipartite entanglement (GME) or genuine multipartite EPR-steering (GMS) \cite{Gal,Ban}. However, identifying these correlations for general systems remains a challenge. Several methods have been proposed to demonstrate the GME \cite{GO,PH,BT,JNY,SV,JJ,MR,LC,AN,FM,IT,SH,ZY,GN,FF,SB} for specific and general states \cite{CD,ZD}, while verifying GMS or GMN relies on constructing special Bell-type inequalities such as the Mermin-Ardehali-Belinskii-Klyshko (MABK) inequalities \cite{Mermin}. Few results have been obtained on GMS \cite{HR,TG,YSQ} or GMN \cite{Uff,NJ,JC,JN,CY,VL}, but Scarani and Gisin \cite{SG} distinguished GMN from GME using generalized Greenberger-Horne-Zeilinger states, using generalized Greenberger-Horne-Zeilinger (GHZ) states \cite{GHZ}, which may not violate the MABK inequalities \cite{Mermin}. These results are further generalized for (odd) $n$-particle GHZ states by using Bell inequalities with dichotomic settings per observer \cite{Zuko}, as well as for general states \cite{CL,RD}.



\begin{figure}
\begin{center}
\includegraphics[width=0.45\textwidth]{Figure1.eps}
\end{center}
\caption{\small (Color online) Schematic causal network of Bell test on the inflated network. It takes use of a given source $S$ and one copy $\hS$ to construct a $5$-partite network. Each node in the directed acyclic graphs (DAGs) represents a variable denoting the type of measurements ($x$ or $\hx$), measurement outcomes ($a,c\hc$ or $\ha$) or sources ($S$ or $\hS$) \cite{Pear}. Each directed edge of two nodes encodes their causal dependence relation. The outcome $c\hc$ depends on two sources while each one of others depend one source.}
\label{fig00}
\end{figure}

In this work, we propose a new approach to extend Gisin's Theorem for isolated many-body systems by utilizing the biseparable no-signalling model. This model allows generating any biseparable correlation using Bell-type tests \cite{Sy, Gal, Ban}. The new Bell test will be conducted on an $n-2$-order inflated network \cite{ESF,WPG,TPLR,Pear} of a given $n$-partite source, as illustrated in Figure \ref{fig00}. While any bipartite correlations of two independent observers activated from the causal network in the biseparable no-signalling model exhibit a maximal violation of 2.7357. Generalized entanglement swapping \cite{ES, BGP} allows for activating the maximally entangled EPR states \cite{EPR} for independent particles, resulting in a group of Bell tests showing the maximum violation of the CHSH inequality \cite{CHSH}. This justifies the equivalence of GMN, GMS, and GME for any finite-dimensional isolated systems under postselection. We also demonstrate the robustness of the network method against noisy states.

\textbf{GMN of many-body systems}. The Bell test \cite{Bell} utilizes a source to distribute states to all space-like separated observers $\sA_1, \cdots, \sA_n$. Each observer's measurement outcome $a_i$ can depend on their local shares and the type of measurement denoted by $x_i$. A joint distribution of all outcomes conditional on measurement settings is genuinely multipartite nonlocal if it does not allow the following biseparable decomposition \cite{Sy}:
\begin{eqnarray}
P_{bs}(\va|\vx)
=\sum_{I,\oI}p_{I}P(\va_I|\vx_I) P(\va_{\oI}|\vx_{\oI}),
\label{eqn-2}
\end{eqnarray}
where $\vec{u}:=(u_1,\cdots, u_n)$ with $u\in \{a,x\}$, $\vec{u}_{X}:=(u_i,i\in X)$, $a_i, x_i\in \{0,\cdots, s\}$, $I$ and $\oI$ denote a bipartition of ${1, \cdots, n}$, and $\{p_{I}\}$ is a probability distribution over all bipartitions. $P(\va_I|\vx_I)$ is the joint distribution of the outcomes $\va_I$ conditional on the measurement settings $\vx_I$, and similar notation is used for $P(\va_{\oI}|\vx_{\oI})$. Both $\{P(\va_I|\vx_I)\}$ and $\{P(\va_{\oI}|\vx_{\oI})\}$ may exhibit local nonlocality \cite{Sy}, but they must satisfy the no-signalling (NS) principle \cite{PR, Gal, Ban}, which states that $p(a_i|x_i,x_j)=p(a_i|x_i)$ for any $i$ and $j\not=i$.

In quantum scenarios, suppose all observers share a state $\rho$ on Hilbert space $\cH_1\otimes \cdots \otimes \cH_n$. Genuine multipartite entanglement (GME) can be observed by ruling out any biseparable state given by \cite{Sy}:
\begin{eqnarray}
\rho_{bs}=\sum_{I,\oI}p_{I}\rho_{I,\oI}.
\label{eqn-2a}
\end{eqnarray}
Here, $\rho_{I,\oI}$ is a separable state on Hilbert space $\cH_I\otimes \cH_{\oI}$ with $\cH_I =\otimes_{i\in I}\cH_i$ and $\cH_{\oI} =\otimes_{j\in \oI}\cH_j$. All biseparable states form a convex set, making it possible to witness the GME of a given state by constructing a witness functional \cite{HHH}. However, this approach is both state and measurement dependent. Instead, if each observer $\sA_i$ performs a local measurement associated with the positive semi-definite operators $\{M^{a_i}_{x_i}\}$ satisfying $\sum_{a_i}M^{a_i}_{x_i}=\mathbbm{1}$ with $\mathbbm{1}$ being the identity operator for any $x_i$, according to Born's rule, there is a joint conditional distribution given by:
\begin{eqnarray}
P_q(\va|\vx)=\textrm{tr}[(\otimes_{i=1}^nM^{a_i}_{x_i})\rho].
\label{eqn-4}
\end{eqnarray}
The correlation may exhibit GMN by violating the decomposition (\ref{eqn-2}). A more tractable method is to construct a specific Bell-type inequality:
\begin{eqnarray}
\cB(P_{bs}(\va|\vx))\leq c,
\label{eqn-5}
\end{eqnarray}
which holds for any biseparable correlation (\ref{eqn-2}). The MABK inequalities \cite{Mermin} are the most well-known examples used in this context. Unfortunately, verifying either GME or GMN in general can be an NP-hard problem \cite{Gurvits}. Our goal is to show the equivalence of GME and GMN for any finite-dimensional isolated systems through a group of Bell tests.

\begin{figure}
\begin{center}
\includegraphics[width=0.45\textwidth]{Figure2.eps}
\end{center}
\caption{\small (Color online) (a) A bipartite Bell-type test in the NS model. Under some hypothesises of the measurements and states the statistics can be represented as a NS distribution $P(a,b|x,y)$, allowing for a causal network where the outcome $a$ (or $b$) is dependent on the measurement settings $x$ (or $y$) and the NS source. (b) A tripartite statistic test in the biseparable NS model. Sharing the biseparable source $\dt_{ABC}=p_1\dt_A\bot \dt_{BC}+p_2\dt_B\bot \dt_{AC}+p_3\dt_{AB}\bot \dt_{C}$ enables three observers to generate the tripartite correlation (\ref{eqn-2}) by classically mixing three tests with two independent sources per test.
}
\label{fig-1a}
\end{figure}

\textit{Biseparable NS model}. The local correlations $P_I$ and $P_{\oI}$ in Eq.(\ref{eqn-2}) can be NS distributions which will be generated by a Bell test with general NS sources such as the PR-box \cite{PR,BLM,Barrett,Pear}. Each observer can locally "measure" the shared states from the NS source to determine their outcome. All theories in the NS framework, similar to classical and quantum theories \cite{PR,BLM,Barrett,Pear}, make specific hypothesises such as the \textit{Convexity} of the associated spaces of the systems or measurement devices, \textit{Distinguishability} for local measurements, \textit{Commutativity} for local operations on distinct systems, \textit{Statistics} for representing the measurement outcomes, \textit{NS principle}, and \textit{Linearity} for both the systems and measurements. Under these hypothesises, a statistic experiment similar to the Bell test \cite{Bell} can be designed using an NS source. Suppose all observers $\sA_i$ ($i\in I$) who share an NS source perform local measurements $\{M^{a_i}_{x_i}\}$ on their shares. There is a function $F$ defining a distribution of all outcomes as:
\begin{eqnarray}
P(\va_I|\vx_I)=F(\dt, (M^{a_i}_{x_i},i\in I)),
\end{eqnarray}
where $\dt$ denotes the total state from NS source, as shown in Figure \ref{fig-1a}(a). This further enables a Bell-type experiment by $n$ observers sharing two independent NS sources $\dt_I$ and $\dt_{\oI}$ to generate the separable distribution $P(\va_I|\vx_I)P(\va_{\oI}|\vx_{\oI})$ from the NS principle. For the general biseparable distribution (\ref{eqn-2}), there is a specific Bell-type experiment by $n$ observers who share the following biseparable NS source as
\begin{eqnarray}
\dt_{bs}:=\sum_{I,\oI}p_{I}\dt_{I}\bot\dt_{\oI},
\label{BNS}
\end{eqnarray}
where $\dt_{I}\bot\dt_{\oI}$ means the independence of two NS sources $\dt_{I}$ and $\dt_{\oI}$. This leads to an equivalent biseparable NS model for representing any biseparable correlation. Fig.\ref{fig-1a}(b) shows one example with the tripartite biseparable NS source $\dt_{ABC}=p_1\dt_A\bot \dt_{BC}+p_2\dt_B\bot \dt_{AC}+p_3\dt_{AB}\bot \dt_{C}$.

\textit{Genuine tripartite correlations}. To illustrate the main idea, we use an inflated network \cite{ESF,WPG,TPLR} for tripartite systems, as shown in Figure \ref{fig00}. In a standard Bell experiment, there is a source $S$ that distributes one state to three observers $\sA$, $\sB$, and $\sC$. Here, the other copy $\hS$ is for $\hsA$, $\hsB$, and $\hsC$. Two observers, such as $\sC$ and $\hsC$, can perform local joint measurements to obtain the outcomes denoted by $c\hc$. This leads to a Bell-type experiment on a 5-partite network in the biseparable NS model. Informally, for a biseparable NS source, one-third of all activated bipartite correlations violate the CHSH inequality with a maximal bound of 2.5 \cite{CHSH}. While inspired by generalized entanglement swapping, one-half of all activated quantum correlations allow for the maximal violation of $2\sqrt{2}$. This provides a device-independent verification of genuine tripartite nonlocality. The idea is inspired by generalized entanglement swapping \cite{ES}.

\begin{theorem}
\label{Theorem1}
Any genuine tripartite entangled pure state is genuine tripartite nonlocal.

\end{theorem}

\textsf{Sketch of Proof.} The main technique involves conducting a Bell-type experiment on the inflated network shown in Fig.\ref{fig00}. Consider any tripartite biseparable NS source as shown in Fig.\ref{fig-1a}(b), for distributing states to three observers $\sA,\sB$ and $\sC$. An additional copy $\dt_{\hA\hB\hC}$ is for observers $\hsA, \hsB$, and $\hsC$. All observers complete the Bell-type experiment $\cW_{B;C}$ as shown in Fig.\ref{fig00}. The local measurements of $\sA$, $\hsA$, $\sB$, and $\hsB$ are denoted by $\{M^{a}_x\}$, $\{M^{\ha}_{\hx}\}$, $\{N_{b}\}$, $\{N_{\hb}\}$, respectively. $\{L_{c\hc}\}$ denotes the local joint measurements by $\sC$ and $\hsC$ on their shares. The proof is sufficient for dichotomic measurement settings and outputs, i.e., $a,b,c,x \in \{0,1\}$.

Under the proper hypotheses of experiments on the NS source \cite{PR,BLM,Barrett,Pear}, there is a joint distribution $P(a,\ha|x,\hx,b,\hb,c\hc)$ of the outcomes $a$ and $\ha$ conditional on the measurement settings $x$ and $\hx$, and the outcomes $(b,\hb,c\hc)$. The following proof includes five other Bell-type experiments, $\cW_{A;B}$, $\cW_{B;A}$, $\cW_{A;C}$, $\cW_{C;A}$, and $\cW_{C;B}$, with local measurement settings similar to $\cW_{B;C}$. The first result is the following lemma for all $96$ activated bipartite correlations.

\begin{lemma}
There are at least 64 local outcomes of $(u,\hu,v\hv)$ such that the activated bipartite distributions allow the following decompositions as
\begin{eqnarray}
&P(w,\hw|\tau,\hat{\tau};u,\hu,v\hv)=pP_{ns}(w,\hw|\tau,\hat{\tau};u,\hu,v\hv)
\nonumber
\\
&+(1-p)P_{s}(w,\hw|\tau,\hat{\tau};u,\hu,v\hv),
\label{eqn-7}
\end{eqnarray}
where $p$ satisfies $0\leq p\leq 1/4$, $P_{ns}$ may be any NS correlation, while $P_{s}$ is separable, $w\not\in \{u,v\}$, and $\tau$ and $\hat{\tau}$ denote the measurement settings of $w$ and $\hw$, respectively.

\label{lems1}
\end{lemma}


From the maximal violation of $4$ of the CHSH inequality \cite{CHSH} for any NS correlation \cite{PR}, Lemma \ref{lems1} implies that there are at least $64$ activated bipartite correlations $P_{u\hu{}v\hv}:=\{P(w,\hw|\tau,\hat{\tau};u\hu{}v\hv)\}$ that satisfy the following inequality:
\begin{eqnarray}
\textsf{CHSH}:=\sum_{i,j=0,1}(-1)^{i\cdot{}j}X_iY_j\leq 2.5,
\label{eqn-6}
\end{eqnarray}
where $X_iY_j=\sum_{x,y=0,1}(-1)^{x+y}P(a,b|x,y)$ is defined for any bipartite distribution $P(a,b|x,y)$. It follows that at most $32$ bipartite correlations violate the inequality (\ref{eqn-6}). This reveals a basic fact of any biseparable NS source:
\begin{eqnarray}
\textsl{cFact}:
\left\{
\begin{array}{ll}
|\Upsilon_{\rhd}| \leq 32;
\\
|\Upsilon_{\unlhd}|+|\Upsilon_{\rhd}|=96;
\end{array}
\right.
\label{eqn-6a}
\end{eqnarray}
where $\Upsilon_{\unlhd}$ and $\Upsilon_{\rhd}$ are defined by
$\Upsilon_{\unlhd}=\{(u,\hu,v\hv)|\textsf{CHSH}(P_{u\hu{}v\hv})\leq 2.5\}$ and $\Upsilon_{ \rhd}=\{(u,\hu,v\hv)|$ $\textsf{CHSH}(P_{u\hu{}v\hv})>2.5\}$.

In the quantum scenario, suppose an isolated system is genuinely entangled and in the state $\ket{\Phi}$ on Hilbert space $\cH_A\otimes \cH_B\otimes \cH_C$. Here we prove another fact for qubit states, while a similar proof holds for high-dimensional states. Consider a quantum realization of the inflated network in Fig.\ref{fig-1a} using two copies of $\ket{\Phi}_{ABC}$ and $\ket{\Phi}_{\hat{A}\hat{B}\hat{C}}$ on Hilbert space $\cH_{\hA}\otimes \cH_{\hB}\otimes \cH_{\hC}$. For the Bell test $\cW_{B;C}$, after proper local projections $\{M_b\}$, $\{M_{\hb}\}$, and $\{L_{c\hc}\}$ (joint measurements) on particles $B$, $\hB$, and $\{C,\hC\}$, respectively, the generalized entanglement swapping \cite{ES} allows the particles $A$ and $\hA$ to collapse into the EPR state \cite{EPR} for at least $8$ outcomes of $b,\hb$ and $c\hc$. Thus, a CHSH test \cite{CHSH} on the particles $A$ and $\hA$ may generate quantum correlations that maximally violate the inequality (\ref{eqn-6}). As other Bell tests have a similar result, there are at least $48$ bipartite quantum correlations violating the inequality (\ref{eqn-6}), that is,
\begin{eqnarray}
\textsl{qFact}:
\left\{
\begin{array}{ll}
|\Upsilon_{\rhd}|\geq 48;
\\
|\Upsilon_{\unlhd}|+|\Upsilon_{\rhd}|=96;
\end{array}
\right.
\label{eqn-6b}
\end{eqnarray}
where $\Upsilon_{\unlhd}$ and $\Upsilon_{\rhd}$ are defined corresponding to (\ref{eqn-6a}) in terms of the quantum correlations.

Both \textsl{cFact} and \textsl{qFact} imply that all quantum correlations from the Bell experiment on the inflated network, consisting of any genuinely entangled tripartite pure state, cannot be simulated using any biseparable NS source. This is because the activated quantum correlations in some Bell-type experiments will violate the inequality (\ref{eqn-6}) that holds for any bipartite correlations from a biseparable NS source. $\Box$

The LOCC in Theorem \ref{Theorem1} can be replaced by a post-selection strategy. Therefore, the only nontrivial requirement in the experiments is the independent and identically distributed (idd) source in the inflated network. This can be further improved to approximate the idd assumption in general.

\textit{Genuine multipartite correlations}. We now prove the result for any isolated many-particle systems. Take $n-1$ copies of a given source $S$ and arrange them into a generalized chain network, where the $i$-th source will send the states to the observers $\sA_{1}^{(i)}, \cdots, \sA_{n}^{(i)}$, $i=1, \cdots, n-1$. The following Bell-type test is defined on the $n-2$-order network inflation in the biseparable NS model. Each pair of two observers $\sA_{i+1}^{(i)}$ and $\sA_{i+1}^{(i+1)}$ can cooperate to perform local joint measurements on their shares, $i=1,\cdots, n-2$. The observers $\sA_{1}^{(1)}$ and $\sA_{n}^{(n-1)}$ will perform local measurements with two measurement settings while any other observer has one measurement setting. By using any biseparable NS source (\ref{BNS}) the key is to show the joint distribution of the outcomes of $\sA_{1}^{(1)}$ and $\sA_{n}^{(n-1)}$ conditional on others outcomes satisfy the following inequality:
\begin{eqnarray}
\textsf{CHSH}\leq 2+\frac{1}{e}\approx 2.7357,
\label{eqn-6b}
\end{eqnarray}
where $e$ denotes the natural constant. However, for an arbitrary genuine $n$-partite entangled pure state, there are quantum correlations that maximally violate the inequality (\ref{eqn-6b}).

\begin{theorem}
\label{Theorem2}
Any genuine multipartite entangled pure state is genuine multipartite nonlocal.

\end{theorem}

Compared to the six Bell-type tests for verifying tripartite systems in Theorem \ref{Theorem1}, the proof of Theorem \ref{Theorem2} provides a simpler method using only one Bell-type test on a specific inflated network. This simplified setup results in the inequality (\ref{eqn-6b}) with a larger upper bound than the inequality (\ref{eqn-6}) in terms of the CHSH method \cite{CHSH}. It allows for further improvement by using more Bell tests.

\textit{Example 1}. Consider the generalized GHZ state \cite{GHZ}:
\begin{eqnarray}
\ket{GHZ}=\cos\theta\ket{000}+\sin\theta\ket{111}
\label{ghz}
\end{eqnarray}
with $\theta\in (0,\frac{\pi}{2})$. By using the generalized entanglement swapping \cite{ES} on the $1$-order inflated network, any two independent particles, such as $A$ and $\hat{A}$, can be collapsed into the EPR state \cite{EPR} for one half of all the others' measurement outcomes. This results in $48$ bipartite correlations, each of which can violate the CHSH inequality \cite{CHSH} beyond the \textsl{cFact}. The issue of the "GHZ-gap" \cite{SG,Zuko} is then resolved. Another approach to tackle this problem is through utilizing the symmetry of the state and the Hardy inequality \cite{CY}.

\textit{Example 2}. Suppose a triangle network consists of three EPR states \cite{EPR}, where each observer has two particles from different EPR states. Such quantum networks may exhibit distinct correlations in the network compared to the singlet state under one measurement setting per party \cite{Renou2019}. Although the total state is GME, there is currently no method for verifying its GMN \cite{Luo2022,TPLR}. By following the proof of Theorem \ref{Theorem1}, a similar Bell-type test with four outcomes per party on the 1-order inflated network implies a fact of $|\Upsilon_{\rhd}|\geq 720$, which violates the fact of $|\Upsilon_{\rhd}|\leq 512$ in the biseparable NS model. This provides the first witness of the GMN for a triangle quantum network. Further extensions allow for their consisting of generalized bipartite entangled pure states.

\textbf{Robustness of GMN}. The present network inflation method for verifying the GMN of general states is robust against noise. In particular, for any genuinely entangled pure state $\ket{\Phi}$ on a $d^n$-dimensional Hilbert space $\otimes_{i=1}^n\cH_{i}$, the Werner state \cite{Werner} is defined by
\begin{eqnarray}
\rho_v=\frac{1-v}{d^n}\mathbbm{1}+v\ketbra{\Phi}{\Phi},
\label{eqn-10}
\end{eqnarray}
where $\mathbbm{1}$ is the identity operator on the Hilbert space $\otimes_{i=1}^n\cH_{i}$ and $v\in [0,1]$. It follows that the noisy state $\rho_v$ has the GMN for a large $v$. Take the generalized GHZ state (\ref{ghz}) as an example. The noise visibility in terms of the GME is given by $v>(8 \delta-1)/7$ and $\delta=\max\{\sin^2\alpha, \cos^2\alpha\}$ \cite{Deng,GS,Sy}. Instead, Theorem \ref{Theorem1} implies a generic noise visibility of $v\gtrapprox 0.9402$ in terms of the GMN, which is independent of $\theta$. For general states, it provides a noise visibility of $v>0.9674^{1/n}$ in terms of the GME from the proof of Theorem \ref{Theorem2} and the linear Bell-type inequality (\ref{eqn-6b}). This allows for relaxing the idd assumption of the inflated network in experiments.

\textbf{Discussion}. Verifying the GMN of multipartite entanglement depends on specific Bell-type tests. The present unified method makes use of a group of Bell experiments on the inflated network consisting of multiple copies of a given source. As the GMS \cite{HR} exhibits activating effects on Bell tests, it shows weaker correlations than the GMN but may be stronger than the GME. Hence, the present results demonstrate the equivalence of the three correlations of GME, GMS, and GMN for any isolated systems. Further extensions may be possible for general noisy states or network scenarios \cite{Luo2021,Alex,TPLR} or many-body systems \cite{AFOV,TAS}. Our results may stimulate new ways of verifying quantum nonlocality.


\section*{Acknowledgements}

We thank the helpful discussions of Qiongyi He, Jordi Tura, Luming Duan, and Alejandro Pozas-Kerstjens. This work was supported by the National Natural Science Foundation of China (Nos. 61772437, 12204386, 12075159 and 12171044), Sichuan Natural Science Foundation (No. 2023NSFSC0447), Beijing Natural Science Foundation (No.Z190005), Interdisciplinary Research of Southwest Jiaotong University China Interdisciplinary Research of Southwest Jiaotong University China (No.2682022KJ004), and the Academician Innovation Platform of Hainan Province.

\begin{thebibliography}{99}

\bibitem{Bell} J. S. Bell, On the Einstein-Podolsky-Rosen paradox, Phys. 1, 195 (1964).

\bibitem{EPR}A. Einstein, B. Podolsky, and N. Rosen, Can quantum-mechanical description of physical reality be considered complete? Phys. Rev. {47}, 777 (1935).

\bibitem{HHH} R. Horodecki, P. Horodecki, M. Horodecki, and K. Horodecki, Quantum entanglement, Rev. Mod. Phys. 81, 865 (2009).

\bibitem{WJD07}H. M. Wiseman, S. J. Jones, and  A. C. Doherty,  Steering, entanglement, nonlocality, and the Einstein-Podolsky-Rosen paradox, Phys. Rev. Lett. 98, 140402 (2007).

\bibitem{Werner}R. F. Werner, Quantum states with Einstein-Podolsky-Rosen correlations admitting a hidden-variable model, Phys. Rev. A 40, 4277 (1989).

\bibitem{CHSH} J. Clauser, M. Horne, A. Shimony, and R. Holt, Proposed experiment to test local hidden-variable theories, Phys. Rev. Lett. 23, 880 (1969).

\bibitem{Gisin} N. Gisin, Bell's inequality holds for all non-product states, Phys. Lett. A 154, 201 (1991); N. Gisin and A. Peres, Maximal violation of Bell's inequality for arbitrarily large spin, Phys. Lett. A 162, 15-17 (1992).


\bibitem{PR} S. Popescu and D. Rohrlich, Generic quantum nonlocality, Phys. Lett. A 166, 293 (1992).

\bibitem{Fei}M. Li and S.-M. Fei, Gisin's Theorem for arbitrary dimensional multipartite states, Phys. Rev. Lett. 104, 240502 (2010).

\bibitem{Cola}A. Coladangelo, K. Tong Goh, and V. Scarani, All pure bipartite entangled states can be self-tested, Nat. Commun. 8, 15485 (2017).

\bibitem{Hardy}L. Hardy, Quantum mechanics, local realistic theories, and Lorentz-invariant realistic theories, Phys. Rev. Lett. 68, 2981 (1992).

\bibitem{Chen}J.-L. Chen, C.F. Wu, L.C. Kwek, and C.H. Oh, Gisin's Theorem for three qubits, Phys. Rev. Lett. 93, 140407 (2004).

\bibitem{Chou}S.K. Choudhary, S. Ghosh, G. Kar, and R. Rahaman, Gisin's Theorem for Arbitrary Dimensional Multipartite States, Phys. Rev. A 81, 042107 (2010).

\bibitem{Yu} S. Yu, Q. Chen, C. Zhang, C. H. Lai, and C. H. Oh, All entangled pure states violate a single Bell's inequality, Phys. Rev. Lett. 109, 120402 (2012).

\bibitem{Sy}G. Svetlichny, Distinguishing three-body from two-body nonseparability by a Bell-type inequality, Phys. Rev. D 35, 3066 (1987).

\bibitem{Mermin} N. D. Mermin, Extreme quantum entanglement in a superposition of macroscopically distinct states, Phys. Rev. Lett. 65, 1838 (1990); M. Ardehali,Bell inequalities with a magnitude of violation that grows exponentially with the number of particles, Phys. Rev. A 46, 5375 (1992); A. V. Belinskii and D. N. Klyshko, Interference of light and Bell's theorem, Phys. Usp. 36, 653 (1993).

\bibitem{RMP}N. Brunner, D. Cavalcanti, S. Pironio, V. Scarani, and S. Wehner, Bell nonlocality, Rev. Mod. Phys. 86, 419 (2014).

\bibitem{Gal}R. Gallego, L. E. W\"{u}rflinger, A. Ac\'{i}n, and M. Navascu\'{e}s, Operational Framework for Nonlocality, Phys. Rev. Lett. 109, 070401 (2012).

\bibitem{Ban}J. Bancal, J. Barrett, N. Gisin, and S. Pironio, Definition of multipartite nonlocality, Phys. Rev. A 88, 014102 (2013).


\bibitem{GO}G. T\'{o}th and O. G\"{u}hne, Detecting genuine multipartite entanglement with two local measurements, Phys. Rev. Lett. 94, 060501 (2005).

\bibitem{PH}P. Krammer, H. Kampermann, D. Bru{\ss}, R. A. Bertlmann, L. C. Kwek, and C. Macchiavello, Multipartite entanglement detection via structure factors, Phys. Rev. Lett. 103, 100502 (2009).

\bibitem{BT} B. Jungnitsch, T. Moroder, and O. G\"{u}hne, Taming multiparticle entanglement, Phys. Rev. Lett. 106, 190502 (2011).

\bibitem{JNY} J.-D. Bancal, N. Gisin, Y.-C. Liang, and S. Pironio, Device-independent witnesses of genuine multipartite entanglement, Phys. Rev. Lett. 106, 250404 (2011).


\bibitem{SV}J. Sperling and W. Vogel, Multipartite entanglement witnesses, Phys. Rev. Lett. 111, 110503 (2013).

\bibitem{JJ}J. T. Barreiro, J.-D. Bancal, P. Schindler, D. Nigg, M. Hennrich, T. Monz, N. Gisin \& R. Blatt, Demonstration of genuine multipartite entanglement with device-independent witnesses, Nature Phys. 9, 559 (2013).

\bibitem{MR}M. Huber and R. Sengupta, Witnessing genuine multipartite entanglement with positive maps, Phys. Rev. Lett. 113, 100501 (2014).

\bibitem{LC}L. Knips, C. Schwemmer, N. Klein, M. Wieniak, and H. Weinfurter, Multipartite entanglement detection with minimal effort, Phys. Rev. Lett. 117, 210504 (2016).

\bibitem{AN}A. Ketterer, N. Wyderka, and O. G\"{u}hne,  Characterizing multipartite entanglement with moments of random correlations, Phys. Rev. Lett. 122, 120505 (2019).

\bibitem{FM}F. Shahandeh, M. Ringbauer, J. C. Loredo, and T. C. Ralph, Ultrafine entanglement witnessing, Phys. Rev. Lett. 118, 110502 (2017).

\bibitem{IT}I. Fr\'{e}rot and T. Roscilde, Optimal Entanglement Witnesses: A Scalable Data-Driven Approach, Phys. Rev. Lett. 127, 040401 (2021).

\bibitem{SH}S. Morelli, H. Yamasaki, M. Huber, and A. Tavakoli, Entanglement Detection with Imprecise Measurements, Phys. Rev. Lett. 128, 250501 (2022).

\bibitem{ZY}Z. Liu, Y. Tang, H. Dai, P. Liu, S. Chen, and X. Ma, Detecting Entanglement in Quantum Many-Body Systems via Permutation Moments, Phys. Rev. Lett. 129, 260501 (2022).

\bibitem{GN} G. Gour and N. R. Wallach, Classification of multipartite entanglement of all finite dimensionality, Phys. Rev. Lett. 111, 060502 (2013).

\bibitem{FF}F. Levi and F. Mintert, Hierarchies of multipartite entanglement, Phys. Rev. Lett. 110, 150402 (2013).

\bibitem{SB}I. Supic, J. Bowles, M.-O. Renou, A. Ac\'{i}n \& M. J. Hoban, Quantum networks self-test all entangled states, Nature Phys. (2023), https://doi.org/10.1038/s41567-023-01945-4.

\bibitem{CD}C. Branciard, D. Rosset, Y.-C. Liang, and N. Gisin, Measurement-device-independent entanglement witnesses for all entangled quantum states, Phys. Rev. Lett. 110, 060405 (2013).

\bibitem{ZD}M. Zwerger, W. D\;{u}r, J.-D. Bancal, and P. Sekatski, Device-independent detection of genuine multipartite entanglement for all pure states, Phys. Rev. Lett. 122, 060502 (2019).

\bibitem{HR}Q. Y. He and M. D. Reid, Genuine multipartite Einstein-Podolsky-Rosen steering, Phys. Rev. Lett. 111, 250403 (2013).

\bibitem{TG}R. Y. Teh, M. Gessner, M. D. Reid, and M. Fadel, Full multipartite steering inseparability, genuine multipartite steering, and monogamy for continuous-variable systems, Phys. Rev. A 105, 012202 (2022).

\bibitem{YSQ}Y. Xiang, S. Cheng, Q. Gong, Z. Ficek, and Q. He, Quantum steering: practical challenges and future directions, Phys. Rev. X Quantum 3, 030102(2022).

\bibitem{Uff}J. Uffink, Quadratic bell inequalities as tests for multipartite entanglement, Phys. Rev. Lett. 88, 230406 (2002).

\bibitem{NJ}N. Brunner, J. Sharam, and T. V\'{e}rtesi, Testing the structure of multipartite entanglement with Bell inequalities, Phys. Rev. Lett. 108, 110501 (2012).

\bibitem{JC}J.-D. Bancal, C. Branciard, N. Gisin, and S. Pironio, Quantifying multipartite nonlocality, Phys. Rev. Lett. 103, 090503 (2009).

\bibitem{JN}J.-D. Bancal, N. Brunner, N. Gisin, and Y.-C. Liang, Detecting genuine multipartite quantum nonlocality: a simple approach and generalization to arbitrary dimensions, Phys. Rev. Lett. 106, 020405 (2011).

\bibitem{CY}Q. Chen, S. Yu, C. Zhang, C. H. Lai, and C. H. Oh, Test of genuine multipartite nonlocality without inequalities, Phys. Rev. Lett. 112, 140404 (2014).

\bibitem{VL}V. Gebhart, L. Pezz\'{e}, and A. Smerzi, Genuine multipartite nonlocality with causal-diagram postselection, Phys. Rev. Lett. 127, 140401 (2021).

\bibitem{SG}V. Scarani and N. Gisin, Spectral decomposition of Bell's operators for qubits, J. Phys. A 34, 6043 (2001).

\bibitem{GHZ}D.M. Greenberger, M. A. Horne, and A. Zeilinger, in \textit{Bell's Theorem, Quantum Theory, and Conceptions of the Universe}, edited by M. Kafatos (Kluwer, Dordrecht, 1989), pp. 69-72.

\bibitem{Zuko}M. Zukowski, C. Brukner, W. Laskowski, and M. Wiesniak, Do all pure entangled states violate Bell's inequalities for correlation functions? Phys. Rev. Lett. 88, 210402 (2002).

\bibitem{CL} C. Schwemmer, L. Knips, M. C. Tran, A. de Rosier, W. Laskowski, T. Paterek, and H. Weinfurter, Genuine Multipartite Entanglement without Multipartite Correlations, Phys. Rev. Lett. 114, 180501 (2015).

\bibitem{RD}R. Augusiak, M. Demianowicz, J. Tura, and A. Ac\'{i}n, Entanglement and nonlocality are inequivalent for any number of parties, Phys. Rev. Lett. 115, 030404 (2015).

\bibitem{ESF}E. Wolfe, R. W. Spekkens, and T. Fritz, The inflation technique for causal inference with latent variables, J. Causal Inf. 7, 0020 (2019).

\bibitem{WPG}E. Wolfe, A. Pozas-Kerstjens, M. Grinberg, D. Rosset, A. Ac\'{i}n, and M. Navascu\'{e}es, Quantum inflation: a general approach to quantum causal compatibility, Phys. Rev. X 11, 021043 (2021).

\bibitem{TPLR}A. Tavakoli, A. Pozas-Kerstjens, M.-X. Luo, and M.-O. Renou, Bell nonlocality in networks, Rep. Prog. Phys. 85, 056001 (2022).

\bibitem{Pear}J. Pearl, \textit{Causality}, Cambridge University Press, Cambridge, England, 2009.

\bibitem{ES}M. Zukowski, A. Zeilinger, M. A. Horne, and A. K. Ekert, "Event-ready-detectors" Bell experiment via entanglement swapping, Phys. Rev. Lett. 71, 4287-4290 (1993).

\bibitem{BGP}C. Branciard, N. Gisin, and S. Pironio, Characterizing the nonlocal correlations created via entanglement swapping, Phys. Rev. Lett. 104, 170401 (2010).

\bibitem{Gurvits}L. Gurvits, \textit{Classical deterministic complexity of Edmonds' Problem and quantum entanglement}, in Proc. of the thirty-fifth ACM symposium on Theory of computing (ACM Press), pp.10-19, 2003.


\bibitem{BLM} J. Barrett, N. Linden, S. Massar, S. Pironio, S. Popescu and D. Roberts, Nonlocal correlations as an information-theoretic resource, Phys. Rev. A 71, 022101 (2005).

\bibitem{Barrett} J. Barrett, Information processing in generalized probabilistic theories, Phys. Rev. A 75, 032304 (2007).


\bibitem{Renou2019} M.-O. Renou, E. Baumer, S. Boreiri, N. Brunner, N. Gisin, and S. Beigi, Genuine quantum nonlocality in the triangle network, Phys. Rev. Lett. 123, 140401 (2019).

\bibitem{Luo2022}M.-X. Luo, Fully device-independent model on quantum networks, Phys.  Rev. Research 4, 013203 (2022).

\bibitem{Deng}D.-L. Deng and J.-L. Chen, Sufficient and necessary condition of separability for generalized Werner states, Ann. Phys. 324, 408 (2009).

\bibitem{GS} O. G\"{u}hne and M. Seevinck, Separability criteria for genuine multiparticle entanglement, New J. Phys. 12, 053002 (2010).

\bibitem{Luo2021}M. X. Luo, New genuinely multipartite entanglement, Adv. Quantum Technol. 4, 2000123(2021).

\bibitem{Alex} A. Pozas-Kerstjens, N. Gisin, and A. Tavakoli, Full network nonlocality, Phys. Rev. Lett. 128, 010403 (2022).

\bibitem{AFOV}L. Amico, R. Fazio, A. Osterloh, and V. Vedral, Entanglement in many-body systems, Rev. Mod. Phys. 80, 517 (2008).

\bibitem{TAS}J. Tura, R. Augusiak, A. B. Sainz, T. V\'{e}rtesi, M. Lewenstein, and A. Ac\'{i}n, Detecting non-locality in multipartite quantum systems with two-body correlation functions, Science 344, 1256 (2014).

\end{thebibliography}

\end{document}


