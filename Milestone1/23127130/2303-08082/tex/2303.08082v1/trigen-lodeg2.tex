\begin{tikzpicture}
\fill[gray!20] (1.6,0) -- ++ (60:0.8) coordinate (a123)
	-- ++(2.4,0) -- (4.8,0) -- cycle;
\fill (a123) circle[radius=0.1];
\draw[very thick] (a123) edge (1.6,0) edge +(2.4,0) edge +(120:1.6) edge +(60:2.4);

\draw[wall] (0,0) -- (4.8,0) -- (60:4.8) -- cycle;

\foreach \x in {0.8,1.6,2.4,3.2,4.0}
\draw (4.8,0)++(120:\x) edge (4.8-\x,0) -- (60:\x) -- (\x,0);

\tikzmath{\full=0.8/sqrt(3);\hp=0.45/sqrt(3);\hm=\full-\hp;}

\begin{scope}[edge]
\foreach \n in {4,5}
\draw[blue,->] (4.8,0) ++ (120:0.8*\n+0.45) \foreach \x in {1,...,\n}
{-- ++(-150:\hp) -- ++(-90:\full) -- ++(-150:\hm)}
-- ++(-150:\hp) -- ++(-90:\hp);

\foreach \n in {2,1} {
\draw[red,->] (4.8,0) ++ (120:{0.8*(3-\n)+0.45})
\foreach \x in {1,...,\n}
{-- ++(-150:\hm) -- ++(150:\full) -- ++(-150:\hp)}
-- ++(-150:\hm) -- ++(150:\hm)
\foreach \x in {1,2}
{-- ++(150:\hm) -- ++(90:\full) -- ++(150:\hp)};
}

\draw[red,->] (4.8,0) ++ (120:{0.8*3+0.45}) -- ++(-150:\hm) -- ++(150:\hm)
\foreach \x in {1,2}
{-- ++(150:\hm) -- ++(90:\full) -- ++(150:\hp)};
\end{scope}
\end{tikzpicture}