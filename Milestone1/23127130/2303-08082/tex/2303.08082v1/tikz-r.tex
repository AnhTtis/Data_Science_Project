%This is tikz code for some pics in stated_sln.
%This document is latexable!

%%%%%%%%%%%%%%%%%%%%%%%%%%%%%%%
% LINE DECORATION STYLES
%%%%%%%%%%%%%%%%%%%%%%%%%%%%%%%

\tikzset{knot diagram/every knot diagram/.style={background color=gray!20,clip width=6,end tolerance=5pt,clip radius=0.2cm}}

\tikzset{edge/.style={line width=0.8}}
\tikzset{wall/.style={very thick}}
\tikzset{det/.style={edge,decoration={markings,mark=at position .5 with
	{\node[draw,fill=gray!20,isosceles triangle,sharp corners,transform shape,inner sep=0.1cm] (det){};}},postaction={decorate}}}
\tikzset{-o-/.code 2 args={
\ifthenelse{\equal{#2}{}}{
}{
	\ifthenelse{\equal{#2}{>}\OR\equal{#2}{<}}{
		\pgfkeysalso{decoration={markings,mark=at position #1 with {\arrow[scale=0.8]{#2}}},postaction={decorate}}
	}{
		\pgfkeysalso{decoration={markings,mark=at position #1 with {\draw[black, fill={#2}] circle[radius=2pt];}},postaction={decorate}}
	}
}
}}

\newcommand{\picmargin}{\mathop{}\!}
\newcommand{\stsize}{\footnotesize}
%%%%%%%%%%%%%%%%%%%%%%%%%%%%%%%%%%%%%%%%%%%%%%

%Generate background with base size and adjusted by arrows
%Define variables: xw, yh, yd, yu, yt
%#1,2: width and height
%#3,4: left and right sides. -> = up, <- = down, w = wall, other = none
%#5: extra code
\newcommand{\TanglePic}[5]{
\picmargin
\begin{tikzpicture}[baseline=(ref.base)]
\tikzmath{\xw=#1; \yh=#2; \yd=0; \yu=0;}
\ifthenelse{\equal{#3}{<-}\OR\equal{#4}{<-}}{
	\tikzmath{\yd=-0.1;}
}{}
\ifthenelse{\equal{#3}{->}\OR\equal{#4}{->}}{
	\tikzmath{\yu=0.1;}
}{}
\tikzmath{\yt=\yh+\yu;}
\fill[gray!20] (0,\yd)rectangle(\xw,\yt);
\node(ref) at ({\xw/2},{\yh/2}) {\phantom{$-$}};
%draw walls
\begin{scope}[wall]
\ifthenelse{\equal{#3}{w}}{
	\draw (0,\yd) --(0,\yt);
}{\ifthenelse{\equal{#3}{}}{}{
	\draw[#3] (0,\yd) --(0,\yt);
}}
\ifthenelse{\equal{#4}{w}}{
	\draw (\xw,\yd) --(\xw,\yt);
}{\ifthenelse{\equal{#4}{}}{}{
	\draw[#4] (\xw,\yd) --(\xw,\yt);
}}
\end{scope}
#5
\end{tikzpicture}
\picmargin
}

%For two horizontal strands
%Define A,B on right, C,D on left. A,C top.
%#1,2: 1 is left, -> = up
%#3,4,5,6: states, top left, bottom left, top right, bottom right
%#7: extra code
\newcommand{\HorizontalTangle}[7]{
\TanglePic{0.9}{0.9}{#1}{#2}{
%draw states
\tikzmath{\ya=\yh/2+0.2; \yb=\yh/2-0.2;}
\path (0,\ya) coordinate (C) (0,\yb) coordinate (D);
\ifthenelse{\equal{#3}{}\AND\equal{#4}{}}{}{
	\draw[left,inner sep=2pt] (C)node{\stsize #3} (D)node{\stsize #4};
}
\path (\xw,\ya) coordinate (A) (\xw,\yb) coordinate (B);
\ifthenelse{\equal{#5}{}\AND\equal{#6}{}}{}{
	\draw[right,inner sep=2pt] (A)node{\stsize #5} (B)node{\stsize #6};
}
#7
}}

%Horizontal strands with a crossing
%#1,2: 1 is left, -> = up
%#3: p,n crossing, a arbitrary
%#4,5: marking, 4 is \
%#6,7,8,9: states, top left, bottom left, top right, bottom right
\newcommand{\crossSt}[9]{
\HorizontalTangle{#1}{#2}{#6}{#7}{#8}{#9}{
\ifthenelse{\equal{#3}{a}}{
	\draw[edge] (C) ..controls ({\xw/2},\ya) and ({\xw/2},\yb).. (B);
	\draw[edge] (D) ..controls ({\xw/2},\yb) and ({\xw/2},\ya).. (A);
}{
	\begin{knot}
	\strand[edge] (C) ..controls ({\xw/2},\ya) and ({\xw/2},\yb).. (B);
	\strand[edge] (D) ..controls ({\xw/2},\yb) and ({\xw/2},\ya).. (A);
	\ifthenelse{\(\equal{#3}{n}\AND\equal{#4}{#5}\)\OR\(\equal{#3}{p}\AND\NOT\equal{#4}{#5}\)}{
		\flipcrossings{1}
	}{}
	\end{knot}
}
\ifthenelse{\equal{#4}{>}\OR\equal{#4}{<}}{
	\tikzmath{\pos=0.9;}
}{
	\tikzmath{\pos=0.8;}
}
\path[edge,-o-={\pos}{#4}] (C) ..controls ({\xw/2},\ya) and ({\xw/2},\yb).. (B);
\path[edge,-o-={\pos}{#5}] (D) ..controls ({\xw/2},\yb) and ({\xw/2},\ya).. (A);
}}

%Horizontal strands with a crossing
%#1,2: 1 is left, -> = up
%#3: p,n crossing
%#4,5: marking, 4 is top
\newcommand{\cross}[5]{\crossSt{#1}{#2}{#3}{#4}{#5}{}{}{}{}}

%Horizontal strands with a crossing right wall
%#1 right
%#2 p,n crossing
%#3,4 marking, 3 is top
%#5,6 states
\newcommand{\crosswall}[6]{\crossSt{}{#1}{#2}{#3}{#4}{}{}{#5}{#6}}

%Two horizontal strands
%#1,2: 1 is left, -> = up
%#3,4: marking, 3 is top
%#5,6,7,8: states, top left, bottom left, top right, bottom right
\newcommand{\walltwowallSt}[8]{
\HorizontalTangle{#1}{#2}{#5}{#6}{#7}{#8}{
\draw[edge,-o-={0.5}{#3}] (C) -- (A);
\draw[edge,-o-={0.5}{#4}] (D) -- (B);
}}

%Two horizontal strands no states
%#1,2: 1 is left, -> = up
%#3,4: marking, 3 is top
\newcommand{\walltwowall}[4]{\walltwowallSt{#1}{#2}{#3}{#4}{}{}{}{}}

%Two horizontal strands on right wall
%#1 -> = up
%#2,3: marking, 2 is top
%#4,5: states, top right, bottom right
\newcommand{\twowall}[5]{\walltwowallSt{}{#1}{#2}{#3}{}{}{#4}{#5}}

%Two horizontal strands on left wall
%#1: -> = up
%#2,3: marking, 2 is top
%#4,5: states, top right, bottom right
\newcommand{\walltwo}[5]{\walltwowallSt{#1}{}{#2}{#3}{#4}{#5}{}{}}

%-----------------------------------------

%Horizontal with positive kink moving right
\newcommand{\kink}{
\TanglePic{0.9}{0.9}{}{}{
\begin{knot}
\strand[edge] (\xw,0.3) -- (0.6,0.3)
	..controls (0.2,0.3) and (0.2,0.7).. (0.45,0.7);
\strand[edge] (0,0.3) -- (0.2,0.3)
	..controls (0.7,0.3) and (0.7,0.7).. (0.45,0.7);
\end{knot}
\path[edge,-o-={0.5}{>}] (0.6,0.3) -- (\xw,0.3);
}}

%One horizontal strand
%#1: marking
\newcommand{\horizontaledge}[1]{
\TanglePic{0.9}{0.9}{}{}{
\draw[edge,-o-={0.8}{#1}] (0,{\yh/2}) --(\xw,{\yh/2});
}}

%One circle
%#1: marking < is CW
\newcommand{\circlediag}[2][]{
\TanglePic{0.9}{0.9}{}{}{
\draw[edge,-o-={0.1}{#2}] (0.45,0.45) circle (0.25);
}}

%-----------------------------------------

%Draw three states
%#1: left or right
%#2: x
%#3,4,5: y
%#6,7,8: labels
\newcommand{\ThreeStates}[8]{
\ifthenelse{\equal{#6}{}\AND\equal{#7}{}\AND\equal{#8}{}}{}{
	\draw (#2,#3)node[#1,inner sep=2pt]{{\stsize #6}};
	\draw (#2,#4)node[#1,inner sep=2pt]{{\stsize #7}};
	\draw (#2,#5)node[#1,inner sep=2pt]{{\stsize #8}};
}}

%Three strands and dots
%Define variables: xw yh y1-y4 ym yds s v
%optional: t or b, 2nd edge on top or bottom
%#2,3: width and height
%#4,5: 1 is left, -> = up
%#6: dots, 0 = none, 1 = right, 2 = left, 3 = both
%#7: extra code
\newcommand{\MultiStrand}[7][b]{
\TanglePic{#2}{#3}{#4}{#5}{
\tikzmath{\xw=#2; \y1=0.2; \y4=\yh-\y1; %\s=(\y4-\y1)/3;
\s=0.35; \y2=\y1+\s; \y3=\y4-\s; \v=(\y2+\y3)/2;}
\ifthenelse{\equal{#1}{b}}{
	\tikzmath{\ym=\y2; \yds=(\y2+\y4)/2;}
}{
	\tikzmath{\ym=\y3; \yds=(\y1+\y3)/2;}
}
\ifthenelse{\equal{#6}{1}\OR\equal{#6}{3}}{
	\draw ({\xw-0.15},\yds)node[rotate=90]{...};
}{}
\ifthenelse{\equal{#6}{2}\OR\equal{#6}{3}}{
	\draw (0.15,\yds)node[rotate=90]{...};
}{}
#7
}}

%A sink and a source
%#1: marking on the left half
\newcommand{\sinksourcethree}[1]{
\MultiStrand{1.4}{1.3}{}{}{3}{
\coordinate (V0) at (0.6,\v);
\coordinate (V1) at (0.8,\v);
\draw[edge,-o-={.50}{#1}] (0,\y1) to [out=0, in=-120] (V0);
\draw[edge,-o-={.57}{#1}] (0,\y2) to [out=0, in=-150] (V0);
\draw[edge,-o-={.50}{#1}] (0,\y4) to [out=0, in=120] (V0);
\draw[edge,-o-={.65}{#1}] (V1) to [out=-60, in=180] (\xw,\y1);
\draw[edge,-o-={.58}{#1}] (V1) to [out=-30, in=180] (\xw,\y2);
\draw[edge,-o-={.65}{#1}] (V1) to [out=60, in=180] (\xw,\y4);
}}

%Ellipse coupon with horizontal strands
%#1: label for coupon
%#2: markings
\newcommand{\coupon}[2]{
\MultiStrand{1.4}{1.3}{}{}{3}{
\node (V) at (\xw/2,\yh/2) [ellipse, minimum height=0.8cm,inner sep=0pt, draw]
	{\small{#1}};
\draw[edge,-o-={.35}{#2}] (0,\y1) ..controls (\xw*0.2,\y1).. (V.-120);
\draw[edge,-o-={.50}{#2}] (0,\y2) ..controls (\xw*0.2,\y2).. (V.-170);
\draw[edge,-o-={.35}{#2}] (0,\y4) ..controls (\xw*0.2,\y4).. (V.120);
\draw[edge,-o-={.70}{#2}] (V.-60) ..controls (\xw*0.8,\y1).. (\xw,\y1);
\draw[edge,-o-={.60}{#2}] (V.-10) ..controls (\xw*0.8,\y2).. (\xw,\y2);
\draw[edge,-o-={.70}{#2}] (V.60) ..controls (\xw*0.8,\y4).. (\xw,\y4);
}}

%Wide coupon with horizontal strands
%optional: t or b: 2nd edge on top or bottom
%#2: label for coupon
%#3,4,5: left labels
%#6,7,8: right labels
\newcommand{\widecoupon}[8]{
\MultiStrand[#1]{1.5}{1.4}{<-}{<-}{3}{
\ThreeStates{left}{0}{\y4}{\ym}{\y1}{#3}{#4}{#5}
\ThreeStates{right}{\xw}{\y4}{\ym}{\y1}{#6}{#7}{#8}
\node (V) at (\xw/2,\v) [circle,minimum height=1cm,
	inner sep=0pt,draw]{#2};
\begin{scope}[edge]
\draw (0,\y1) ..controls (\xw*0.2,\y1).. (V.-120);
\draw (0,\y4) ..controls (\xw*0.2,\y4).. (V.120);
\draw (V.-60) ..controls (\xw*0.8,\y1).. (\xw,\y1);
\draw (V.60) ..controls (\xw*0.8,\y4).. (\xw,\y4);
\ifthenelse{\equal{#1}{b}}{
	\draw (0,\ym) ..controls (\xw*0.1,\ym).. (V.-165);
	\draw (V.-15) ..controls (\xw*0.9,\ym).. (\xw,\ym);
}{
	\draw (0,\ym) ..controls (\xw*0.1,\ym).. (V.165);
	\draw (V.15) ..controls (\xw*0.9,\ym).. (\xw,\ym);
}
\end{scope}
}}

%A vertex near right wall. Three strands and vdots
%optional: t or b: 2nd edge on top or bottom
%#2: markings
\newcommand{\vertexnearwall}[2][b]{
\MultiStrand[#1]{1}{1.3}{}{w}{2}{
\coordinate (V) at (0.7,\v);
\draw[edge,-o-={.40}{#2}] (0,\y4) to [out=0, in=120] (V);
\draw[edge,-o-={.40}{#2}] (0,\y1) to [out=0, in=-120] (V);
\draw[edge,-o-={.45}{#2}] (0,\ym) ..controls (0.3,\ym).. (V);
}}

%Parallel strands to wall. Three strands and vdots
%optional: t(op) or b(ottom)
%#2 -> is up
%#3 markings
%#4,5,6 states
\newcommand{\nedgewall}[6][b]{
\MultiStrand[#1]{1}{1.3}{}{#2}{1}{
\ThreeStates{right}{\xw}{\y4}{\ym}{\y1}{#4}{#5}{#6}
\draw[edge, -o-={.5}{#3}] (0,\y4) -- (\xw,\y4);
\draw[edge, -o-={.5}{#3}] (0,\ym) -- (\xw,\ym);
\draw[edge, -o-={.5}{#3}] (0,\y1) -- (\xw,\y1);
}}

%Vertex with 3 strands to the right wall
%optional: t(op) or b(ottom)
%#2 -> is up
%#3 markings
%#4,5,6 states
\newcommand{\vertexwall}[6][b]{
\MultiStrand[#1]{1}{1.3}{}{#2}{1}{
\ThreeStates{right}{\xw}{\y4}{\ym}{\y1}{#4}{#5}{#6}
\coordinate (V) at (0.3,\v);
\draw[edge,-o-={.40}{#3}] (\xw,\y4) to [out=180, in=60] (V);
\draw[edge,-o-={.40}{#3}] (\xw,\y1) to [out=180, in=-60] (V);
\draw[edge,-o-={.45}{#3}] (\xw,\ym) ..controls (0.7,\ym).. (V);
}}


%-----------------------------------------

%For two horizontal strands
%Define A,B on right, C,D on left. A,C top.
%#1,2: 1 is left, -> = up
%#3,4,5,6: states, top left, bottom left, top right, bottom right
%#7: extra code

%Cap on wall
%optional w for wall
%#2 -> is up
%#3 left or right wall
%#4 marking for top end point
%#5,6 states
\newcommand{\capwall}[6][]{
\HorizontalTangle{}{#2}{}{}{#5}{#6}{
\draw[edge,-o-={0.8}{#4}] (\xw,\yb) ..controls (0.1,\yb) and (0.1,\ya).. (\xw,\ya);
}}

%Cap near wall is just cap on the opposite wall
%#1 marking
\newcommand{\capnearwall}[1]{
\HorizontalTangle{}{w}{}{}{}{}{
\draw[edge,-o-={0.8}{#1}] (0,\yb) ..controls (0.7,\yb) and (0.7,\ya).. (0,\ya);
}}

%-----------------------------------------

%Web resolution of crossing
%#1,2 marking
%#3 label the central (multiple) edge
\newcommand{\twoonetwowall}[3]{
\TanglePic{1.2}{0.9}{}{}{
%draw states
\tikzmath{\ya=\yh/2+0.2; \yb=\yh/2-0.2;}
\draw[edge] (0.3,{\yh/2}) coordinate (L) -- (0.9,{\yh/2}) coordinate (R);
\draw[edge,-o-={0.8}{#1}] (R) to[out=30, in=180] (\xw,\ya);
\draw[edge,-o-={0.8}{#2}] (R) to[out=-30, in=180] (\xw,\yb);
\draw ({\xw/2},{\yh/2})node[above]{\stsize #3};
\draw[edge,-o-={0.5}{#1}] (0,\ya) to[out=0, in=150] (L);
\draw[edge,-o-={0.5}{#2}] (0,\yb) to[out=0, in=-150] (L);
}}

%Vertex with 4 strands to the right wall
%#1 -> is up
%#2 marking
%#3,4 states
\newcommand{\vertexfour}[4][<-]{
\TanglePic{1}{1.2}{}{#1}{
\tikzmath{\s=0.35; \y2=(\yh-\s)/2; \y3=\y2+\s;}
\coordinate (V) at (0.4,{\yh/2});
\draw[edge] (V) edge +(150:\s) edge +(180:\s) edge +(-150:\s);
\draw[edge,-o-={.7}{#2}] (V) to [out=25, in=180]
	(\xw,\y3)node[right,inner sep=2pt]{{\stsize #3}};
\draw[edge,-o-={.7}{#2}] (V) to [out=-25, in=180]
	(\xw,\y2)node[right,inner sep=2pt]{{\stsize #4}};
}}

%Vertex with 4 strands to the right wall and a crossing
%#1 marking
%#2,3 states
\newcommand{\vertexfourcross}[3]{
\TanglePic{1}{1.2}{}{<-}{
\tikzmath{\s=0.35; \y2=(\yh-\s)/2; \y3=\y2+\s;}
\coordinate (V) at (0.4,{\yh/2});
\draw[edge] (V) edge +(150:\s) edge +(180:\s) edge +(-150:\s);
\begin{knot}
\strand[edge] (V) to [out=45, in=175, looseness=1.5] (\xw,\y2);
\strand[edge] (V) to [out=-45, in=-175, looseness=1.5] (\xw,\y3);
\end{knot}
\path[edge,-o-={.8}{#1}] (V) to [out=30, in=175, looseness=1.5]
	(\xw,\y2)node[right,inner sep=2pt]{{\stsize #3}};
\path[edge,-o-={.8}{#1}] (V) to [out=-30, in=-175, looseness=1.5]
	(\xw,\y3)node[right,inner sep=2pt]{{\stsize #2}};
}}

%Vertex with 4 strands to the left wall
%#1 -> is up
%#2 marking
%#3,4 states
\newcommand{\vertexfourleft}[4][<-]{
\TanglePic{1}{1.2}{#1}{}{
\tikzmath{\s=0.35; \y2=(\yh-\s)/2; \y3=\y2+\s;}
\coordinate (V) at (0.6,{\yh/2});
\draw[edge] (V) edge +(30:\s) edge +(0:\s) edge +(-30:\s);
\draw[edge,-o-={.7}{#2}] (V) to [out=155, in=0]
	(0,\y3)node[left,inner sep=2pt]{{\stsize #3}};
\draw[edge,-o-={.7}{#2}] (V) to [out=-155, in=0]
	(0,\y2)node[left,inner sep=2pt]{{\stsize #4}};
}}

%-----------------------------------------

%One bubble and one strand
%#1: t,b bubble position
%#2: bubble text
%#3,4: bubble state
%#5,6: states
\newcommand{\bubblestrand}[6]{
\TanglePic{1}{1}{<-}{<-}{
\ifthenelse{\equal{#1}{b}}{\tikzmath{\b=1;}}{\tikzmath{\b=-1;}}
\tikzmath{\yb=\yh/2-\b*0.2; \ys=\yh/2+\b*0.2;}
\node (B) at ({\xw/2},\yb) [circle,draw,inner sep=1pt] {\stsize #2};
\begin{scope}[every node/.style={inner sep=2pt}]
\begin{knot}
\strand[edge] (0,\ys)node[left]{\stsize #5}
	-- (\xw,\ys)node[right]{\stsize #6};
\strand[edge] (B) -- ({\xw/2},\yd);
\end{knot}
\draw[edge] (0,\yb)node[left]{\stsize #3} -- (B)
	-- (\xw,\yb)node[right]{\stsize #4};
\end{scope}
}}

%One bubble and one strand and splitting
%#1: bubble text
%#2,3: bubble state
%#4,5: states
\newcommand{\bubblesplit}[5]{
\TanglePic{2}{1}{<-}{<-}{
\tikzmath{\y1=\yh/2-0.2; \y2=\yh/2+0.2;}
\node (B) at ({\xw/2},\y2) [circle,draw,inner sep=1pt] {\stsize #1};
\draw[dashed] (0.6,\yd) -- (0.6,\yt) (1.4,\yd) -- (1.4,\yt);
\begin{knot}
\strand[edge,inner sep=2pt] (0,\y2)node[left]{\stsize #4}
	..controls (0.3,\y2) and (0.3,\y1).. (0.6,\y1)
	-- (1.4,\y1) ..controls (1.7,\y1) and (1.7,\y2)..
	(\xw,\y2)node[right]{\stsize #5};
\strand[edge,inner sep=2pt] (0,\y1)node[left]{\stsize #2}
	..controls (0.3,\y1) and (0.3,\y2).. (0.6,\y2)
	-- (B) -- (1.4,\y2) ..controls (1.7,\y2) and (1.7,\y1)..
	(\xw,\y1)node[right]{\stsize #3};
\strand[edge] (B) -- ({\xw/2},\yd);
\end{knot}
}}

%One bubble and one strand in monogon
\newcommand{\bubblemono}{
\TanglePic{2}{1.6}{}{<-}{
\tikzmath{\s=0.5; \y2=\yh/2; \y3=\y2+\s; \y1=\y2-\s;}
\fill[white] (0.5,\y2) circle[radius=0.2];
\node (B) at (1,\y2) [circle,draw,inner sep=1pt] {\stsize $\alpha$};
\begin{knot}
\strand[edge,inner sep=2pt] (\xw,\y3)node[right]{\stsize $j$}
	..controls +(-0.25,0) and +(0.25,0).. (1.5,\y2)
	..controls +(-0.25,0) and +(0.25,0).. (1,\y3)
	..controls +(-0.6,0) and +(0,0.25).. (0.2,\y2)
	..controls +(0,-0.25) and +(-0.6,0).. (1,\y1)
	..controls +(0.5,0).. (\xw,\y2)node[right]{\stsize $i$};
\strand[edge,inner sep=2pt] (0.7,\y2) -- (B)
	..controls (1.4,\y2) and (1.4,\y3).. (1.6,\y3)
	..controls +(0.2,0) and +(-0.2,0).. (\xw,\y1)node[right]{$s$};
\flipcrossings{3}
\end{knot}
\path[edge,-o-={0.1}{white}] (1,\y3)
	..controls +(-0.6,0) and +(0,0.25).. (0.2,\y2);
\draw[dashed] (1.6,\yd) -- (1.6,\yt);
}}

%One bubble and one strand in monogon split
\newcommand{\bubblemonosplit}{
\TanglePic{0.8}{1.6}{<-}{<-}{
\tikzmath{\s=0.5; \y2=\yh/2; \y3=\y2+\s; \y1=\y2-\s;}
\begin{knot}
\strand[edge,inner sep=2pt] (0,\y2)node[left]{\stsize $j'$}
	-- (\xw,\y3)node[right]{\stsize $j$};
\strand[edge,inner sep=2pt] (0,\y3)node[left]{\stsize $s'$}
	-- (\xw,\y1)node[right]{$s$};
\strand[edge,inner sep=2pt] (0,\y1)node[left]{\stsize $i'$}
	-- (\xw,\y2)node[right]{\stsize $i$};
\end{knot}
\path[edge,inner sep=2pt,-o-={0.7}{white}] (0,\y2) -- (\xw,\y3);
\path[edge,inner sep=2pt,-o-={0.3}{black}] (0,\y1) -- (\xw,\y2);
}}

%%%%%%%%%%%%%%%%%%%%%%%%%%%%%%%%%%%%%%%%%%%%%%

%\draw (current bounding box.south west) rectangle (current bounding box.north east);