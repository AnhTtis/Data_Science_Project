\begin{tikzpicture}[baseline=0cm]
\draw[wall] (0,0) -- (4.8,0) -- (60:4.8) -- cycle;

\foreach \x in {0.8,1.6,2.4,3.2,4.0}
\draw (4.8,0)++(120:\x) edge (4.8-\x,0) -- (60:\x) -- (\x,0);

\tikzmath{\full=0.8/sqrt(3);\hp=0.40/sqrt(3);\hm=\full-\hp;}
\tikzmath{\n=4;}
\tikzmath{\hw=0.2;}

\begin{scope}[edge,blue]
\path (4.8,0) ++ (120:0.8*\n+0.4) coordinate (S)
	node[anchor=south west] {$s=5$};
\draw[->]  (S) \foreach \x in {1,...,\n}
	{-- ++(-150:\hp) -- ++(-90:\full) -- ++(-150:\hm)}
	-- ++(-150:\hp) -- ++(-90:\hp);

\foreach \x in {1,...,\n} {
\path ({4.8-\x*0.8},0) ++ (60:{\x*0.8}) coordinate (B);
\fill (B) circle (0.1);
\fill[opacity=0.2] (B) ++ (-30:\hw) ++(60:\hw)
	-- ++(-120:{\x*0.8+2*\hw}) -- ++(150:{2*\hw})
	-- ++(60:{\x*0.8+2*\hw}) -- cycle;
}
\end{scope}
\end{tikzpicture}
\qquad
\begin{tikzpicture}[baseline=0cm]
\draw[wall] (0,0) -- (4.8,0) -- (60:4.8) -- cycle;

\foreach \x in {0.8,1.6,2.4,3.2,4.0}
\draw (4.8,0)++(120:\x) edge (4.8-\x,0) -- (60:\x) -- (\x,0);

\tikzmath{\full=0.8/sqrt(3);\hp=0.40/sqrt(3);\hm=\full-\hp;}
\tikzmath{\n=4;}
\tikzmath{\hw=0.2;}

\newcommand{\Cvst}{
(S) -- ++(-150:\hp)
	-- ++(-90:\full) -- ++(-30:\full) -- ++(-90:\full)
	-- ++(-150:\full) -- ++(-90:\full) -- ++(-150:\full)
	-- ++(-90:\full) -- ++(-30:\full) -- ++(-90:\hp)
}

\begin{scope}[edge,blue]
\path (4.8,0) ++ (120:0.8*\n+0.4) coordinate (S)
	node[anchor=south west] {$s=5$}
	({3.5*0.8},-0.1) node[anchor=north west]{$t=3$};
\draw[->] \Cvst;

\clip \Cvst |- (4.8,{-2*\hw}) |- cycle;
\foreach \x in {1,...,\n} {
\path ({4.8-\x*0.8},0) ++ (60:{\x*0.8}) coordinate (B);
\fill (B) circle (0.1);
\fill[opacity=0.2] (B) ++ (-30:\hw) ++(60:\hw)
	-- ++(-120:{\x*0.8+2*\hw}) -- ++(150:{2*\hw})
	-- ++(60:{\x*0.8+2*\hw}) -- cycle;
}
\end{scope}

\begin{scope}[red]
\clip \Cvst |- (0,{-2*\hw}) -- (60:4.8) -- cycle;
\foreach \x/\l in {3/0,4/3} {
\path ({4.8-\x*0.8},0) ++ (60:{\l*0.8}) coordinate (B);
\fill (B) circle (0.1);
\fill[opacity=0.2] (B) ++ (-30:\hw) ++(60:\hw)
	-- ++(-120:{\l*0.8+2*\hw}) -- ++(150:{2*\hw})
	-- ++(60:{\l*0.8+2*\hw}) -- cycle;
}
\end{scope}
\end{tikzpicture}
