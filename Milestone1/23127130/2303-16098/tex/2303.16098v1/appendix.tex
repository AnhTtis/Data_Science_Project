\section*{Appendix}
\label{firstpage}






\begin{table}[!h]
\centering
\begin{adjustbox}{width=\textwidth}
\begin{tabular}{
    >{\raggedleft\arraybackslash}p{.35\textwidth}p{.65\textwidth}}
\toprule[1pt]
  \multicolumn{1}{c}{\textbf{Tag}}     &
  \multicolumn{1}{l}{\textbf{Description}}  \\
\toprule[1pt]
\xml{title type="main"}     &   Name of the corpus  \\
\xml{title type="sub"}      &   Version of the corpus to which a file belongs   \\
\xml{authority}             &   Team responsible for the corpus         \\
\xml{sourceDesc}            &   Global description of the sources used in the corpus and their licenses \\
\xml{projectDesc}           &   Brief description of the corpus project \\
\xml{taxonomy}              &   Structure of the typologies (Source and Carolina) used to classify texts. They are subdivided in categories (\xml{category}) and their descriptions (\xml{catDesc}) \\
\xml{title}                 &   Id of the text within the corpus. It is a composition of 3 capital letters (according to the Broad Typology) + a serial number    \\
\xml{respStmt}              &   Person responsible for either the download, metadata, or extraction of the text, as indicated by the element \xml{resp} \\
\xml{measure unit="tokens"} &   Number of tokens in the text, indicated in the attribute "quantity" \\
\xml{authority}             &   Team responsabile for the distribution of the corpus' text  \\
\xml{date}                  &   Date in which the text was downloaded (type="Download") and date of processing and incorporation of such text into the corpus (type="Extraction")   \\
\xml{availability}          &   Status of availability of the corpus' text. The corpus' texts are always freely distributed \\
\xml{license}               &   The name of the license of the corpus' text. The attribute “target” contains the license url \\
\bottomrule[1pt]
\end{tabular}
\end{adjustbox}
\caption{Identification of metadata for \textit{File Description} category}
\label{tab:corpus_metadata_carolina}
\end{table}



\begin{table}[!h]
\centering
\begin{adjustbox}{width=\textwidth}
\begin{tabular}{
    >{\raggedleft\arraybackslash}p{.35\textwidth}p{.65\textwidth}}
\toprule[1pt]
  \multicolumn{1}{c}{\textbf{Tag}}     &
  \multicolumn{1}{l}{\textbf{Description}}  \\
\toprule[1pt]

\xml{name}                      & Original text title   \\
\xml{media}                     & Contains the attributes "mimeType", "url" and "source", all referring to the original file: "mimeType" refers to the file type, e.g. "text/csv", "application/pdf"; "source" contains the file name; and "url" refers to the file link    \\
\xml{author}                    & Author of the original file   \\
\xml{editor role="translator"}  & Translator of the original file   \\
\xml{sponsor}                   & Institution responsible for the original text \\
\xml{extent}                    & Original file size in bytes, tokens and pages. When the term “pages” does not apply to the text type, the quantity is “-1”    \\
\xml{publisher}                 & The publisher of the original text    \\
\xml{authority}                 & Authority responsible for the original text   \\
\xml{date}                      & Publishing date of the original text  \\
\xml{availability}              & Status of availability of the original text. Can be "free", when freely distributed, or "restricted", if a registration is needed to access the text  \\
\xml{license}                   & The name of the original text license. The attribute “target” contains the license url    \\
\xml{region}                    & Region of the linguistic variety  \\
\xml{seriesStmt}                & Contains the title of the whole work in the tag <title>, and the title of the work’s part in the tag <biblScope>, e.g. title of a story and title of a chapter    \\
\xml{sourceDesc}                & Description of the text origin, e.g. "Born digital"   \\
\xml{channel}                   & Attribute mode="w" for written texts, mode="s" for transcribed speech, or mode="m" for mixed  \\
\xml{constitution}              & Attribute type="single" for a complete text, type="frags" for a combination of incomplete texts, type="composite" for a combination of complete texts     \\
\xml{domain}                    & Social environment in which the textual types occur, e.g. academic, entertainment, pedagogical    \\
\xml{catRef}                    & Sources’ classification of their own texts. The attribute "scheme" indicates the Source Typology and "target" explicits the type (textual type + 3 capital letters of the domain + the first capital letter of the <channel> "mode")  \\
\xml{language}                  & Language variant of the text. The attibute "ident" contains "pt-BR", or "pt" if the variant is not specified  \\
\xml{catRef}                    & Initial grouping by similar web-domain content    \\
\xml{text}                      & Processed textual content \\


\bottomrule[1pt]
\end{tabular}
\end{adjustbox}
\caption{Identification of metadata for \textit{Source Description} category}
\label{tab:corpus_metadata_source}
\end{table}

\end{document}

