\documentclass[aps,prl,twocolumn,groupedaddress,notitlepage,%showpacs,
floatfix,superscriptaddress]{revtex4-2}

\pdfoutput=1 %\usepackage[english,turkish]{babel}
\usepackage{graphicx,graphics,epsfig,subfigure,times,bm,bbm,amssymb,amsmath,amsthm,mathrsfs,MnSymbol} \usepackage{gensymb} \usepackage{amsfonts} \usepackage{float} \usepackage[matrix,frame,arrow]{xypic} \usepackage[pdfstartview=FitH]{hyperref} %\usepackage{subfigure}
\usepackage{times}
\usepackage{float}
\usepackage{graphics}
\usepackage[T1]{fontenc}

% \usepackage[pdftex]{color} %\usepackage{hypernat}
\usepackage{braket}  %Dirac Notation in QM
\usepackage{enumerate} \usepackage[normalem]{ulem}
\usepackage[usenames,dvipsnames]{xcolor} \usepackage{multirow} \usepackage{mathtools}
\usepackage{bbm} \usepackage{titletoc} \DeclarePairedDelimiter{\ceil}{\lceil}{\rceil}

\definecolor{orange}{rgb}{1,0.5,0} %\newtheorem{thm}{Theorem}
%\newtheorem{critalpha}{Theorem}
%\newtheorem{mydef}{Definition}
%\newtheorem{col}{Corollary}

%\newtheorem{mytheorem}{Theorem}
%\newtheorem{mylemma}{Lemma}
%\newtheorem{mycorollary}{Corollary}
%\newtheorem{myproposition}{Proposition}

%%%%%%%YIGIT's commands

%\newcommand{\abs}[1]{\left\vert{#1}\right\vert}
%\newcommand{\RNum}[1]{\uppercase\expandafter{\romannumeral #1\relax}}
%\newcommand{\summm}[3]{\sum_{\substack{ #1 \\ #2 \\ #3 }}}
%\newcommand{\summ}[2]{\sum_{\substack{ #1 \\ #2}}}
%\newcommand{\po}[2]{\hat{\sigma}_{#1}^{#2}}
%\newcommand{\mbtiny}[1]{\mbox{\tiny\boldmath$#1$}}
%\newcommand{\mtiny}[1]{\mbox{\tiny$#1$}} %\newcommand{\bra}[1]{\left\langle #1
	%% \right|}
%%\newcommand{\ket}[1]{\left| #1 \right\rangle}
%\newcommand{\inner}[2]{\langle #1 | #2 \rangle} \newcommand{\scPi}[0]{\mbox{\Large$\pi$}}
%\newcommand{\scpi}[0]{\mbox{\normalsize$\pi$}} %%% end of YIGIT's commands
%
%\newcommand{\ignore}[1]{}
%\usepackage{geometry}\geometry{left=2.5cm,right=2.5cm,top=3cm,bottom=3cm}
%\newcommand{\red}{\color{red}} \newcommand{\green}{\color{green}}
%\newcommand{\blue}{\color{blue}}
%
%\ignore{ \documentclass[eprintnumbers,amsmath,amssymb,onecolumn,a4paper,caption ]{article}
	%	\usepackage{amsfonts} \usepackage{amssymb} \usepackage{mathrsfs} \usepackage{mathbbold}
	%	\usepackage{bbm} \usepackage{mathrsfs} %\usepackage{graphicx}% Include figure files
	%	\usepackage{dcolumn}% Align table columns on decimal point
	%	\usepackage{bm}% bold math
	%	\usepackage{times,epsfig,amssymb,amsmath} \usepackage{float} \usepackage{subfigure}
	%	%\usepackage{epsfig,graphics,subfigure,psfrag,amsmath,amssymb}
	%	\usepackage{geometry}\geometry{left=2.5cm,right=2.5cm,top=3cm,bottom=3cm}
	%	\newcommand{\red}{\color{red}} \newcommand{\green}{\color{green}} \usepackage{color}
	%	\newcommand{\blue}{\color{blue}}
	%
	%	\newcommand{\magenta}{\color{magenta}} \newcommand{\yellow}{\color{yellow}} }
%%
\hypersetup{ colorlinks=true,       % false: boxed links; true: colored links
	linkcolor=red,          % color of internal links
	citecolor=blue,        % color of links to bibliography
	filecolor=magenta,      % color of file links
	urlcolor=blue,           % color of external links
	runcolor=cyan }

%\hyphenpenalty=5000 \tolerance=1000

\begin{document}
	\title{Meson Instability of Quantum Many-body Scars in a 1D Lattice Gauge Theory}

	\author{Zi-Yong~Ge}
	\email{ziyong.ge@riken.jp}
	\affiliation{Theoretical Quantum Physics Laboratory, Cluster for Pioneering Research, RIKEN, Wako-shi, Saitama 351-0198, Japan}
	

	\author{Yu-Ran~Zhang}
	\email{yuran.zhang@riken.jp}
	\affiliation{Theoretical Quantum Physics Laboratory, Cluster for Pioneering Research, RIKEN, Wako-shi, Saitama 351-0198, Japan}
	\affiliation{Quantum Information Physics Theory Research Team, Center for Quantum Computing, RIKEN, Wako-shi, Saitama 351-0198, Japan}

	\author{Franco Nori}
	\email{fnori@riken.jp}
	\affiliation{Theoretical Quantum Physics Laboratory, Cluster for Pioneering Research, RIKEN, Wako-shi, Saitama 351-0198, Japan}
	\affiliation{Quantum Information Physics Theory Research Team, Center for Quantum Computing, RIKEN, Wako-shi, Saitama 351-0198, Japan}
	\affiliation{Department of Physics, University of Michigan, Ann Arbor, Michigan 48109-1040, USA}


	\begin{abstract}
	We investigate the stability of meson excitations (particle-antiparticle bound states) in quantum many-body scars 
	of a 1D $\mathbb{Z}_2$ lattice gauge theory coupled to spinless fermions.
	By introducing a string representation of the physical Hilbert space,
	we express a scar state $\ket {\Psi_{n,l}}$ as a superposition of all string bases with an identical string number $n$ and a total length $l$.
    The string correlation function of lattice fermions hosts an exponential decay as the distance increases for the small-$l$ scar state $\ket {\Psi_{n,l}}$,
    indicating the existence of stable mesons.
    However, for large $l$, the correlation function exhibits a power-law decay, signaling the emergence of a meson instability.
    Furthermore, we show that this mesonic-nonmesonic crossover can be detected by the quench dynamics, starting from two low-entangled initial states, respectively,
    which are experimentally feasible in quantum simulators.
    Our results expand the physics of quantum many-body scars in lattice gauge theories,
	and reveal that the nonmesonic state can also manifest ergodicity breaking.
	\end{abstract}



	\maketitle

	\textit{Introduction.}---%
	Due to the development of quantum simulations~\cite{doi:10.1080/00018730701223200,RevModPhys.80.885,Buluta108,Buluta_2011,doi:10.1080/00107514.2016.1151199, RevModPhys.86.153,Gross995,s41567-019-0733-z},  out-of-equilibrium quantum many-body physics has been attracting growing interests~\cite{RevModPhys.83.863}.
	The Eigenstate Thermalization Hypothesis (ETH) postulates that generic isolated nonintegrable quantum many-body systems exhibit ergodicity~\cite{PhysRevA.43.2046,PhysRevE.50.888,Srednicki_1999,Rigol2008Nature,DAlessio2016},
	and thus the unitary quantum evolution of the systems can result in an equilibrium state described by  statistical mechanics.
	Though ETH was thought to be  general, there are several counter-examples,
	e.g., quantum integrable systems~\cite{kinoshita2006quantum,PhysRevLett.98.050405} and many-body localizations~\cite{Basko2006,PhysRevB.82.174411,PhysRevLett.113.107204,PhysRevB.90.174202,PhysRevB.93.041424,Rahul2015,RevModPhys.91.021001}.
	These two examples are called strong ergodicity breaking, since most of the eigenstates violate the ETH.
	Recent experimental and theoretical works demonstrate that there exist a new type of ETH-violating eigenstates in some specific nonintegrable quantum many-body systems,  dubbed quantum many-body scar (QMBS) states~\cite{bernien2017probing,turner2018weak,PhysRevB.98.155134,PhysRevLett.122.173401,PhysRevB.98.235155,PhysRevB.98.235156,PhysRevLett.122.220603,PhysRevLett.122.040603,Shiraishi_2019,PhysRevLett.123.147201,PhysRevResearch.2.033044,PhysRevB.101.195131,PhysRevB.102.075132,PhysRevB.102.085140,PhysRevResearch.2.043305,PhysRevLett.124.180604,PhysRevLett.125.230602,PhysRevLett.126.120604,Moudgalya_2022}.
	Generally, the number of QMBS states is  exponentially smaller than the Hilbert space dimension,
	so QMBSs can be considered as  weak ergodicity breaking.
	One typical class of QMBS states are constructed by spectrum-generated algebras~\cite{PhysRevB.102.085140,PhysRevB.102.075132,Moudgalya_2022},
	of which the corresponding eigenenergies  are equally spaced, dubbed towers of QMBSs~\cite{Moudgalya_2022}.
	Thus, if the initial state is a superposition of these scar states, there will exist
	a perfect revival dynamics indicating the ETH violation.


	Empirically, kinetically constrained systems are thought more likely to host QMBSs.
	Thus, lattice gauge theories (LGTs)~\cite{PhysRevD.10.2445,RevModPhys.51.659,PhysRevB.65.024504,PhysRevD.17.2637,PhysRevD.19.3682,Fradkin2013,PhysRevA.73.022328,
		PhysRevLett.109.175302,di2019resolution,banuls2020simulating,PhysRevResearch.3.023079,PhysRevA.103.053703,PRXQuantum.3.010324}, as a typical instance of kinetically constrained systems,
	have attracted considerable interest to study QMBSs~\cite{PhysRevLett.122.130603, PhysRevB.99.195108, PhysRevX.10.021041,
		PhysRevB.101.024306,PhysRevLett.124.207602,PhysRevLett.126.220601,PhysRevB.106.L041101}.
	Meanwhile, particles can be pairwise confined into mesons in LGTs,
	which is found to be closely related to non-thermal states~\cite{PhysRevLett.122.130603, PhysRevB.99.195108, PhysRevX.10.021041,
				PhysRevB.101.024306,PhysRevLett.124.207602, ISI:000395814000014,birnkammer2022prethermalization}.
	 For instance, the scar dynamics in the \textit{PXP} model~\cite{PhysRevX.10.021041} can also be understood
	 in the picture of the string inversion of a $U(1)$ LGT,
	where a particle and an antiparticle are always bounded in a stable meson.
	Therefore, one natural question is whether stable mesons  are a necessary condition of ETH violation in LGTs.
	A recent work~\cite{PhysRevB.101.195131} reports a special type of QMBSs in a spin chain with conserved  domain wall (equivalent to a $\mathbb{Z}_2$ LGT),
    which are generated by nonlocal operators.
	Based on these QMBSs, we address the above questions, and further reveal more
	nontrivial physics when investigating QMBSs in LGTs.


	In this work, we study this type of QMBSs in a $\mathbb{Z}_2$ LGT, 
	and demonstrate that it can manifest both mesonic and nonmesonic features.
	First, we introduce a string representation to describe the physical Hilbert space in a specific gauge sector.
	In this representation, the exact wave function of the QMBS state $\ket {\Psi_{n,l}}$ is written as an equal superposition of
	all string bases with an identical string number $n$ and a total string length $l$.
	We identify the instability of mesons in $\ket {\Psi_{n,l}}$ by calculating the gauge invariant correlation functions of lattice fermions.
	Our results show that mesons are stable for small $l$, while these become unstable for large $l$.
     Moreover, we propose an approach to detect the instability of mesons for these QMBSs in quantum simulators,
     by observing the quench dynamics of the system initially at two different low-entangled states, respectively.


	\textit{Model.}---%
	Here we consider a 1D $\mathbb{Z}_2$ LGT minimally coupled to dynamical spinless fermions.
	The  Hamiltonian has the form $\hat H = \hat H_K + H_E + \hat H_\mu$, with
	\begin{align} \label{H} \nonumber
		&\hat  H_K =-J\sum_{j=1}^L\big(\hat f^\dagger_{j}\hat \tau^z_{j+\frac{1}{2}}\hat f_{j+1,} + \text{H.c.}\big),  \\
		&\hat H_E = -h\sum_{j=1}^L \hat \tau_{j+\frac{1}{2}}^x,\ \ \ \  \hat H_\mu = \mu\sum_{j=1}^L \hat n_{j},
	\end{align}
	where $\hat f^\dagger_{j}$ ($\hat f_{j}$) is the creation (annihilation) operator of the fermion living on site $j$,
	 $\hat n_{j} = \hat f^\dagger_{j}\hat f_{j}$ is the fermion number operator,
	the Pauli matrix $\hat \tau_{j+\frac{1}{2}}^\alpha$, describing gauge fields,
	acts on the link between sites $j$ and $j+1$ (labeled by $j+\frac{1}{2}$),
	and $L$ is the system size.
	The kinetic term $\hat H_K$ describes the fermions coupled minimally to the gauge field with an amplitude $J$,
	the second term $\hat H_E$ describes an electric field with a strength $h$, and
	the last term $\hat H_\mu$ denotes the chemical potential of the fermion.
    We consider periodic boundary conditions, i.e., $\hat f_{1}=\hat f_{L+1}$ and $\hat \tau_{1+\frac{1}{2}}^\alpha=\hat \tau_{L+1+\frac{1}{2}}^\alpha$.
	The Hamiltonian $\hat H$ is $\mathbb{Z}_2$ gauge invariant with a generator defined as
	$\hat G_j = \hat \tau_{j-\frac{1}{2}}^x (-1) ^{\hat n_j}\hat\tau_{j+\frac{1}{2}}^x$.
	We note that this $\mathbb{Z}_2$ LGT can be experimentally addressed~\cite{ Barbieroeaav7444,ISI:000494944200024,ISI:000494944200023,Ge_2021,PhysRevResearch.4.L022060,arXiv:2203.08905,arXiv:2206.08909,PhysRevB.107.125141}.



	Here, without loss of generality, we fix the system to the gauge sector $\hat G_j = 1$,
	where there are an even number of fermions for each physical basis.
	By introducing Majorana operators $\hat \gamma_j := \hat f^\dagger_j +\hat f_j$ and $\tilde\gamma_j := i(\hat f^\dagger_j -\hat f_j)$,
	the original Hamiltonian $\hat H$ can be mapped to a local spin chain~\cite{PhysRevLett.127.167203}
	\begin{align} \label{Hdual}\nonumber
		&\hat H_K = -\frac{J}{2}\sum_{j}(\hat Z_{j+\frac{1}{2}}-\hat X_{j-\frac{1}{2}}\hat Z_{j+\frac{1}{2}}\hat X_{j+\frac{3}{2}}),\\
		&\hat H_E = -h\sum_{j}\hat X_{j+\frac{1}{2}}, \ \ \ \ \hat H_\mu =\frac{\mu}{2}\sum_{j}(1-\hat X_{j-\frac{1}{2}}\hat X_{j+\frac{1}{2}}),
	\end{align}
	where $\hat X_{j+\frac{1}{2}}=\hat \tau^x_{j+\frac{1}{2}}$ and $\hat Z_{j+\frac{1}{2}}
	=-i\hat \gamma_j \hat \tau^z_{j+\frac{1}{2}}\tilde{\gamma}_{j+1}$
	are Pauli matrices.


 Figure~\ref{fig_1}(a) plots the half-chain von Neumann entropies, 
 $S=\text{Tr}\hat\rho_{L/2}\ln \hat\rho_{L/2}$, of the whole eigenstates in the half-filling case ($n=L/4$),	
 where $\hat\rho_{L/2}$ is the half-chain density matrix.
The entanglement entropies of states near the middle of the spectrum
approach the value for a random	state $S^{\text{ran}}=(L\ln 2-1)/2$~\cite{PhysRevLett.71.1291},
which demonstrates that most of the eigenstates obey ETH in this gauge sector.


As shown in  Ref.~\cite{PhysRevB.101.195131}, a pyramid-like structure of scar states exists for the dual Hamiltonian (\ref{Hdual}), see Figs.~\ref{fig_1}(a,b).
These scar states cannot be generated by local operators and are very distinct from the conventional towers of QMBSs.
However, they have not been fully investigated, and specifically, 
it is still unclear whether stable meson excitations dominate the scar dynamics, like most of conventional QMBSs in LGTs.
In addition, an experimental proposal to detect these QMBSs from the nontrivial quench dynamics
in quantum simulators is also a relevant issue.
Hereafter, we investigate the QMBSs of $\hat H$ from the viewpoint of LGTs,
and uncover whether these can be described by mesonic physics.
Note that while analytical discussions are based on the original Hamiltonian Eq.~(\ref{H}),
the numerical results are obtained by using the Hamiltonian (\ref{Hdual}).



     \begin{figure}[t] \includegraphics[width=0.47\textwidth]{Fig1.pdf}
	    \caption{Half-chain von Neumann  entropy $S$.
		(a) The distribution of $S$ for all eigenstates of Hamiltonian (\ref{H}) with $L=16$, $J=1$, $h=0.5$, and $\sum_{j=1}^L \hat n_j=8$.
		The color codes the density of states (warmer colors imply higher density).
	    The blue dashed line represents the entanglement entropy of the random state $S^{\text{ran}}=(L\ln 2-1)/2\approx5.05$.
        The data points in red circles correspond to the eigenstates in Eq.~(\ref{psi_nl}).
        (b) Pyramid-like structure of $\ket {\Psi_{n,l}}$ for $L=20$.
        Different colors represent different numbers of strings $n$.
       (c) The size scaling of the entanglement entropy for the QMBS state $\ket{\Psi_{L/4,L/2}}$. The dashed line shows linear fitting: $S\sim \ln L$.}
	\label{fig_1}
    \end{figure}


	\textit{String representation.}---%
	Before discussing the QMBSs, we introduce string bases in the $\hat G_j = 1$ sector,
	which are convenient for discussing meson excitations.
	Due to gauge invariance, the physical Hilbert space in the fixed gauge sector can be represented by open strings.
	The vacuum state of string excitations in the $\hat G_j = 1$ sector can be defined as
	\begin{align} \label{omega}
		\ket{\Omega} := \ket{F}\otimes\ket{\tau}=\ket{00...00}\otimes\ket{++...++},
	\end{align}
	where $\ket{F}=\ket{00...00}$ is the Fock state of matter fields representing no fermion occupation,
	and $\ket{\tau}=\ket{++...++}$ is the state of gauge fields with all links being polarized at $\hat \tau^x=1$.
	A state with one string excitation can be written as
		\begin{align} \label{Skl}
			\ket{\mathcal{S}_{k,\ell}}  :=\hat {\mathcal{S}}_{k,\ell}^\dagger\ket{\Omega}=\hat f_k^\dagger \big(\prod_{k\leq j<k+\ell} \hat \tau^m_{j+\frac{1}{2}}\big)\hat f_{k+\ell}^\dagger\ket{\Omega},
	\end{align}
    where $\hat \tau^m := \ket{-}\bra{+} $,
    $k$ denotes the string position, and $\ell$ the string length.
    We can also define the parity as $P_{\mathcal{S}_{k,\ell}} :=\exp{(-i\pi k)}$, which is determined by the string position.
    We note that, if the operator $\hat {\mathcal{S}}_{k,\ell}^\dagger$ is local [i.e., $\ell\sim O(1)$], 
    then the corresponding string excitation can be regarded as a meson.


   An arbitrary gauge invariant basis of the Hamiltonian (\ref{H}) can be written as
    \begin{align} \label{vstr}
    	\ket{\{\mathcal{S}_{k_j,\ell_j }\}_n^l }  =\hat {\mathcal{S}}_{k_1,\ell_1}^\dagger\hat {\mathcal{S}}_{k_2,\ell_2}^\dagger...
    	\hat {\mathcal{S}}_{k_{n-1},\ell_{n-1}}^\dagger\hat {\mathcal{S}}_{k_n,\ell_n}^\dagger\ket{\Omega},
    \end{align}
    where $k_j > k_i + \ell_i$, for $j>i$, $n$ is the number of strings, $l:= \sum_{j=1}^{n} \ell_j$ is the total string length,
    and the parity is $P_{\{\mathcal{S}_{k_j,\ell_j }\}_n^l }=\exp{(-i\pi \sum_{j=1}^n k_j)}$.
    Here, each string excitation contains two fermions, so the string number $n$ equals half of the  fermion number.
    Due to the $U(1)$ symmetry of $\hat H$, the quantum number $n$ is conserved.
    However, the total string length and the parity are not invariant under the action of $\hat H_K$.
    While $l$ determines the energy of the electric-field term $\hat H_E\ket{\{\mathcal{S}_{k_j,\ell_j }\}_n ^l} = h(2l-L)\ket{\{\mathcal{S}_{k_j,\ell_j }\}_n^l} $,
    $n$ determines the energy of the potential term $\hat H_\mu\ket{\{\mathcal{S}_{k_j,\ell_j }\}_n ^l} = 2\mu n\ket{\{\mathcal{S}_{k_j,\ell_j }\}_n^l} $.


    \textit{Exact quantum many-body scars.}---%
    We introduce the pyramid-like QMBS states in this $\mathbb{Z}_2$ LGT~\cite{PhysRevB.101.195131}, whose wave functions in the string representation are written as
    \begin{align} \label{psi_nl}
    	\ket {\Psi_{n,l}} = \mathcal{N}_{n,l}\sum_{\{k_j,\ell_j\}} P_{\{\mathcal{S}_{k_j,\ell_j }\}_n^l } \ket{\{\mathcal{S}_{k_j,\ell_j }\}_{n}^{l} },
    \end{align}
    where $\mathcal{N}_{n,l}$ is a normalization factor.
    That is, $\ket {\Psi_{n,l}} $ is an equal superposition of all string bases with both the same string number $n$ and total length $l$,
    and the phase is determined by the parity of each basis.
    Since $0<\ell_j <L$, the quantum numbers $n$ and $l$ satisfy $n\leq l \leq L-n$.
    Thus, there are $(L-2n+1)$ of eigenstates in the sector with fermion number $2n$.
    It can be proved that $\hat H_K \ket {\Psi_{n,l}} = 0$ [see more derivation details in Supplemental Material (SM)~\cite{SM}],
    and the eigenenergy of $\ket {\Psi_{n,l}} $ is  ${\varepsilon}_{n,l} = 2hl+2\mu n-hL $,
    which can be away from edges of the spectrum corresponding to a high-energy eigenstate, see Fig.~\ref{fig_1}(a).
    In addition, the scar states $\ket {\Psi_{n,l}}$ host the sub-volume-law entanglement entropy,
    i.e., $S\sim \ln L$, demonstrating the ETH violation, see Fig.~\ref{fig_1}(c).



The scar state  $\ket {\Psi_{n,l}}$ can also be expressed in terms of generating operators.
First, we consider a simple case $\ket {\Psi_{n,n}}$, which only contains $n$ length-$1$ string excitation.
We can construct a ladder operator~\cite{PhysRevB.101.024306}
    $ \hat S^\dagger := \sum_{j} P_{\mathcal{S}_{j,1}} \hat{\mathcal{S}}_{j,1}^\dagger$, % =\sum_{j} e^{-i\pi j}\hat f^\dagger_{j} \hat \tau^m_{j+\frac{1}{2}} \hat f^\dagger_{j+1},
    and the eigenstate $\ket {\Psi_{n,n}}$  can be obtained as
      \begin{align} \label{psinn}
    	\ket {\Psi_{n,n}} = \mathcal{A}_n (\hat S^\dagger) ^n\ket{\Omega},
    \end{align}
    where $\mathcal{A}_n$ is a normalization factor.
Then, we introduce another operator~\cite{PhysRevB.101.195131} $\hat L^\dagger_m =\sum_{j}  \big(\sum_{k\leq m}\prod_{\ell \leq m}\hat {\mathcal{P}}^-_{j+\frac{1}{2}-\ell} \big)\hat f_{j}\hat \tau^m_{j+\frac{1}{2}}\hat f^\dagger_{j+1}$,
   where $\hat {\mathcal{P}}^- := \ket{-}\bra{-}$.
   The action of $\hat L^\dagger_m$ is to enlarge the total string length by 1 without changing the parity.
    Using $\hat L_m^\dag$, we obtain the eigenstate $\ket {\Psi_{n,n+m}}$ as~\cite{PhysRevB.101.195131,SM}
       \begin{align} \label{psi2}
    	\ket {\Psi_{n,n+m}}  = \mathcal{D}_{n,m}\hat L_m^\dag \ket {\Psi_{n,n+m-1}}
    \end{align}
   where $\mathcal{D}_{n,m}$ is a normalization factor.
   Equations~(\ref{psinn}, \ref{psi2}) indicate that the scar state $\ket {\Psi_{n,n}}$, like the conventional tower of QMBSs, is generated by \textit{local} operators,
   while  $\ket {\Psi_{n,n+m}}$ is generated by \textit{nonlocal} operators.
   Thus, intuitively, as the total string length $l$ increases, the stability of mesons for the scar state $\ket {\Psi_{n,l}}$ is expected to be significantly broken.




    \begin{figure}[t] \includegraphics[width=0.45\textwidth]{Fig2.pdf}
    	\caption{String correlation function of lattice fermions defined in  Eq.~(\ref{Cf}) for (a) $l=L/4$ and (b) $l=L/2$.
    		The black dashed line is the linear fitting: $|C_{\text{f}}(r)|\sim 1/r^\alpha$, where $\alpha\approx0.35$.
    	The insert plots the half-chain correlation function $C_{\text{f}}(L/2)$ versus the system size $L$,
    	which tends to a power-law scaling.  }
    	\label{fig_2}
    \end{figure}

    \textit{Emergent instability of mesons.}---%
    Mesons, as a type of particle-antiparticle bound states, play an important role in the dynamics of LGTs.
    If the system is in a confined phase, then the low-energy excitation is described by mesons.
    In addition, in a high-energy regime, meson dynamics also closely relate to the ETH.
    Previous works shown that almost all of the nonthermal dynamics in LGTs originate from stable meson excitations~\cite{PhysRevLett.122.130603, PhysRevB.99.195108, PhysRevX.10.021041,PhysRevB.101.024306,PhysRevLett.124.207602, ISI:000395814000014,birnkammer2022prethermalization},
    including the revival dynamics in \textit{PXP} model~\cite{PhysRevX.10.021041}.
    Thus, one natural question is whether a stable meson excitation is a necessary condition to induce an ETH violation in LGTs.
    For the Hamiltonian (\ref{H}), the lattice fermion is confined in the ground state with an arbitrary finite $h$~\cite{PhysRevLett.127.167203},
    where the low-energy excitation is a meson. 
    However, for the high-energy dynamics, especially the scar dynamics, whether mesons are still stable is an open question.



    According to Eq.~(\ref{vstr}), we can find that lattice fermions are always pairwise into string excitations.
    For small $l$, e.g., $\ket{\Psi_{n,n}}$, two fermions are always bonded together on two nearest-neighbor sites, i.e., there only exist local string excitations (mesons).
    This suggests that these scar states should be described by stable mesons.
    However, as  $l$ increases, the distance between two fermions of a string excitation becomes large, and nonlocal string excitations can emerge.
    Hence, intuitively, free fermions are expected to exist in this case, leading to the instability of mesons.


    To further verify the above picture, we perform numerical simulations by calculating the gauge invariant correlation function versus the distance $r$~\cite{PhysRevLett.127.167203}
    \begin{align} \label{Cf}
    	C_{\text{f}}(r):=\bra{\Psi_{n,l}} \hat f_j^\dagger(\prod_{j\leq k < j+r} \hat \tau^z_{k+\frac{1}{2}}) \hat f_{j+r}\ket{\Psi_{n,l}},
    \end{align}
    which identifies elementary excitations.
    As shown in Fig.~\ref{fig_2},  the QMBS states $\ket{\Psi_{L/4,L/4}}$ and $\ket{\Psi_{L/4,L/2}}$ are studied with $L=32$.
    For $\ket{\Psi_{L/4,L/4}}$, we can find  that $C_{\text{f}}(r)$ nearly exhibits an exponential decay with increasing $r$, see Fig.~\ref{fig_2}(a).
    This result indicates that lattice fermions are not free and tend to form stable meson excitations for small-$l$ scar states.
    However, $C_{\text{f}}(r)$ tends to exhibit a power-law decay for the state $\ket{\Psi_{L/4,L/2}}$,
    demonstrating that the fermion is nearly free for large $l$, see Fig.~\ref{fig_2}(b).
    Thus, the instability of mesons in QMBSs $\ket{\Psi_{n,l}}$ arises, when increasing $l$.
	Figure \ref{fig_2} also reveals that the stable meson excitation is not a necessary condition to violate the ETH in LGTs.

    \begin{figure}[t] \includegraphics[width=0.48\textwidth]{Fig3.pdf}
    	    	\caption{Time evolution of the fidelity  $\mathcal{F}(t)$ after a quantum quench for the initial states (a) $\ket{\psi_1}$ and (b) $\ket{\psi_2}$ in Eq.~(\ref{psi0}).
    		    Here, we choose $J=1$, $h=0.5$, and $L=16$. }
    	    	\label{fig_3}
    \end{figure}

    \textit{Quench dynamics.}---%
    Another problem is whether the QMBS states $\ket{\Psi_{n,l}}$ can lead to nontrivial quench dynamics,
    which can be experimentally observed in quantum simulators.
   Here, we introduce two initial states
	 \begin{subequations}\label{psi0}
    \begin{align} %\nonumber
    	 &\ket{\psi_{1} }= \mathcal{B} \prod_{j} \big[1+(-1)^j\hat f^\dagger_{j} \hat \tau^m_{j+\frac{1}{2}} \hat f^\dagger_{j+1}\big]\ket{\Omega},\\
    	&\ket{\psi_{2} }= \frac{1}{2^{L/2}} \sum_{n,l}\sum _{\{k_j,\ell_j\}} P_{\{\mathcal{S}_{k_j,\ell_j }\}_n^l } \ket{\{\mathcal{S}_{k_j,\ell_j }\}_{n}^{l} },
    \end{align}
	\end{subequations}
   where $\mathcal{B}$ is the normalization factor.
   Here, $\ket{\psi_{1} }$ is a superposition of scar states $\ket{\Psi_{n,n}}$, i.e., $\ket{\psi_{1} }=\sum_n \alpha_n\ket{\Psi_{n,n}}$,
   and $\ket{\psi_{2} }$ is a superposition of all scar states $\ket{\Psi_{n,l}}$, i.e., $\ket{\psi_{2} }=\sum_{n,l} \beta_{n,l}\ket{\Psi_{n,l}}$.
  It is obvious that both initial states host low entanglement entropies. Specifically, with the dual transformation in Eq.~(\ref{Hdual}), $\ket{\psi_{1}}$ is related to the ground state of the \textit{PXP} model~\cite{PhysRevB.101.024306}, and $\ket{\psi_{2}} =\bigotimes_j \ket{V_{2j+\frac{1}{2},2j+\frac{3}{2}}}$, with $\ket{V_{2j+\frac{1}{2},2j+\frac{3}{2}}}=(\ket{++}-\ket{+-}+\ket{-+}+\ket{--})/2$.



   In Fig.~\ref{fig_3}, we present the fidelity $\mathcal{F}(t):=|\braket{\psi_{1,2}|e^{-i\hat H t}|\psi_{1,2}}|^2$, where the initial state evolves with $\hat H$.
   It shows that $\mathcal{F}(t)$ can exhibit perfect revival dynamics for both initial states in Eq.~(\ref{psi0}).
   For the initial state $\ket{\psi_{1} }$, the oscillation period is $T=\pi/(h+\mu)$, see Fig.~\ref{fig_3}(a).
   For the initial state $\ket{\psi_{2} }$, if $h/\mu = p/q$, with $p$ and $q$ being relatively prime,
   then the time for a perfect revival is $T=p\pi/h=q\pi/\mu$, see Fig.~\ref{fig_3}(b).
   The oscillation period is consistent with the eigenenergies of $\ket{\Psi_{n,l}}$.
   Moreover, the revival dynamics signals the ETH violation for QMBSs  $\ket{\Psi_{n,l}}$.


    We also probe the stability of mesons during the quench dynamics.
    Since $\ket{\psi_{1} }$ is a superposition of small-$l$ scar states, we expect a mesonic quench dynamics,
     i.e., mesons are always stable during the dynamics, like the scar dynamics in the \textit{PXP} model~\cite{PhysRevX.10.021041}.
    For the initial state $\ket{\psi_{2} }$, although it is a superposition of all scar states,
    the large-$l$ scar states should be dominant~\cite{SM}, e.g., $|\beta_{L/4,L/2}|\gg |\beta_{L/4,L/4}|$.
    Thus, it can lead to nonmesonic dynamics.
    We calculate the dynamics of the gauge invariant correlation function $C_{\text f}(r)$, defined in Eq.~(\ref{Cf}),
    to identify whether there emerges free fermions during the quench dynamics.
    For the initial state $\ket{\psi_{1} }$, we can find that lattice fermions are always short-range correlated,
    and $C_{\text f}(r)$ exhibits an exponential decay as $r$ increases during the quench dynamics, see Fig.~\ref{fig_4}(a,b).
    This indicates that mesons are very stable and cannot be decomposed into free fermions in this case.
    However, the situation becomes different for the initial state $\ket{\psi_{2} }$,
    where lattice fermions can be long-range correlated
    [$C_{\text f}(r)\sim \text{const} $ for $r\rightarrow \infty$] at the specific time, see Fig.~\ref{fig_4}(c,d).
    Mesons in this case are unstable and can be decomposed into free fermions.
    Therefore, the initial states in Eq.~(\ref{psi0}) can be used to detect the mesonic-nonmesonic crossover for QMBSs $\ket{\Psi_{n,l}}$
    during their quench dynamics.



     \begin{figure}[t] \includegraphics[width=0.48\textwidth]{Fig4.pdf}
    	\caption{Time evolution of the fermion correlation function $C_{\text f}(r)$ for two
			different initial states (a,b) $\ket{\psi_1}$ and (c,d) $\ket{\psi_2}$.
    	Here, we choose $J=1$, $h=0.5$, $\mu=0.4$, and $L=24$. }
    	\label{fig_4}
    \end{figure}

    \textit{Experimental proposal.}---%
   The initial states in Eq.~(\ref{psi0}) in dual systems  are conveniently to be prepared in quantum simulators.
    In addition, the dual Hamiltonian (\ref{Hdual}), with the three-body interaction terms, has also been realized by quantum gates, e.g, in superconducting circuits~\cite{arXiv:2203.08905,zhang2022digital}.
    Therefore, the mesonic-nonmesonic crossover for QMBSs in this $\mathbb{Z}_2$ LGT can be experimentally detected with digital quantum simulations.


     \textit{Summary.}---%
     We have investigated the instability of mesons in QMBSs of a $\mathbb{Z}_2$ LGT, minimally coupled to dynamical spinless fermions.
     By introducing the string representation, we express the wave function of each QMBS  as  an equal superposition of all string
     bases with an identical string number and total string length.
     We demonstrate that scar states with small total string length is described by stable mesons, like conventional nonthermal states in LGTs,
     while we find the meson instability in the scar states with large total string length.
     Furthermore, this instability of mesons in QMBSs can be observed from the quench dynamics with two experimentally accessible initial states.
     Our results show various physical results of QMBSs in LGTs and reveal that the nonmesonic states can also host ergodicity breaking in LGTs,
     which can also be experimentally verified in quantum simulators.
     An interesting issue is whether the above physics can be generalized to other gauge groups or high-dimensional LGTs~\cite{PhysRevB.38.2926,banuls2020simulating}.


    	\begin{acknowledgements}
    	\textit{Acknowledgements.}---%
    	This work is supported in part by:
    	Nippon Telegraph and Telephone Corporation (NTT) Research,
    	the Japan Science and Technology Agency (JST) [via
    	the Quantum Leap Flagship Program (Q-LEAP), and
    	the Moonshot R\&D Grant Number JPMJMS2061],
    	the Asian Office of Aerospace Research and Development (AOARD) (via Grant No. FA2386-20-1-4069), and
    	the Foundational Questions Institute Fund (FQXi) via Grant No. FQXi-IAF19-06.
    \end{acknowledgements}


%apsrev4-2.bst 2019-01-14 (MD) hand-edited version of apsrev4-1.bst
%Control: key (0)
%Control: author (72) initials jnrlst
%Control: editor formatted (1) identically to author
%Control: production of article title (-1) disabled
%Control: page (0) single
%Control: year (1) truncated
%Control: production of eprint (0) enabled
\begin{thebibliography}{77}%
	\makeatletter
	\providecommand \@ifxundefined [1]{%
		\@ifx{#1\undefined}
	}%
	\providecommand \@ifnum [1]{%
		\ifnum #1\expandafter \@firstoftwo
		\else \expandafter \@secondoftwo
		\fi
	}%
	\providecommand \@ifx [1]{%
		\ifx #1\expandafter \@firstoftwo
		\else \expandafter \@secondoftwo
		\fi
	}%
	\providecommand \natexlab [1]{#1}%
	\providecommand \enquote  [1]{``#1''}%
	\providecommand \bibnamefont  [1]{#1}%
	\providecommand \bibfnamefont [1]{#1}%
	\providecommand \citenamefont [1]{#1}%
	\providecommand \href@noop [0]{\@secondoftwo}%
	\providecommand \href [0]{\begingroup \@sanitize@url \@href}%
	\providecommand \@href[1]{\@@startlink{#1}\@@href}%
	\providecommand \@@href[1]{\endgroup#1\@@endlink}%
	\providecommand \@sanitize@url [0]{\catcode `\\12\catcode `\$12\catcode
		`\&12\catcode `\#12\catcode `\^12\catcode `\_12\catcode `\%12\relax}%
	\providecommand \@@startlink[1]{}%
	\providecommand \@@endlink[0]{}%
	\providecommand \url  [0]{\begingroup\@sanitize@url \@url }%
	\providecommand \@url [1]{\endgroup\@href {#1}{\urlprefix }}%
	\providecommand \urlprefix  [0]{URL }%
	\providecommand \Eprint [0]{\href }%
	\providecommand \doibase [0]{https://doi.org/}%
	\providecommand \selectlanguage [0]{\@gobble}%
	\providecommand \bibinfo  [0]{\@secondoftwo}%
	\providecommand \bibfield  [0]{\@secondoftwo}%
	\providecommand \translation [1]{[#1]}%
	\providecommand \BibitemOpen [0]{}%
	\providecommand \bibitemStop [0]{}%
	\providecommand \bibitemNoStop [0]{.\EOS\space}%
	\providecommand \EOS [0]{\spacefactor3000\relax}%
	\providecommand \BibitemShut  [1]{\csname bibitem#1\endcsname}%
	\let\auto@bib@innerbib\@empty
	%</preamble>
	\bibitem [{\citenamefont {Lewenstein}\ \emph {et~al.}(2007)\citenamefont
		{Lewenstein}, \citenamefont {Sanpera}, \citenamefont {Ahufinger},
		\citenamefont {Damski}, \citenamefont {Sen(De)},\ and\ \citenamefont
		{Sen}}]{doi:10.1080/00018730701223200}%
	\BibitemOpen
	\bibfield  {author} {\bibinfo {author} {\bibfnamefont {M.}~\bibnamefont
			{Lewenstein}}, \bibinfo {author} {\bibfnamefont {A.}~\bibnamefont {Sanpera}},
		\bibinfo {author} {\bibfnamefont {V.}~\bibnamefont {Ahufinger}}, \bibinfo
		{author} {\bibfnamefont {B.}~\bibnamefont {Damski}}, \bibinfo {author}
		{\bibfnamefont {A.}~\bibnamefont {Sen(De)}},\ and\ \bibinfo {author}
		{\bibfnamefont {U.}~\bibnamefont {Sen}},\ }\bibinfo {title} {Ultracold atomic
		gases in optical lattices: mimicking condensed matter physics and beyond},\
	\href {https://doi.org/10.1080/00018730701223200} {\bibfield  {journal}
		{\bibinfo  {journal} {Adv. Phys.}\ }\textbf {\bibinfo {volume} {56}},\
		\bibinfo {pages} {243} (\bibinfo {year} {2007})}\BibitemShut {NoStop}%
	\bibitem [{\citenamefont {Bloch}\ \emph {et~al.}(2008)\citenamefont {Bloch},
		\citenamefont {Dalibard},\ and\ \citenamefont {Zwerger}}]{RevModPhys.80.885}%
	\BibitemOpen
	\bibfield  {author} {\bibinfo {author} {\bibfnamefont {I.}~\bibnamefont
			{Bloch}}, \bibinfo {author} {\bibfnamefont {J.}~\bibnamefont {Dalibard}},\
		and\ \bibinfo {author} {\bibfnamefont {W.}~\bibnamefont {Zwerger}},\
	}\bibinfo {title} {Many-body physics with ultracold gases},\ \href
	{https://doi.org/10.1103/RevModPhys.80.885} {\bibfield  {journal} {\bibinfo
			{journal} {Rev. Mod. Phys.}\ }\textbf {\bibinfo {volume} {80}},\ \bibinfo
		{pages} {885} (\bibinfo {year} {2008})}\BibitemShut {NoStop}%
	\bibitem [{\citenamefont {Buluta}\ and\ \citenamefont
		{Nori}(2009)}]{Buluta108}%
	\BibitemOpen
	\bibfield  {author} {\bibinfo {author} {\bibfnamefont {I.}~\bibnamefont
			{Buluta}}\ and\ \bibinfo {author} {\bibfnamefont {F.}~\bibnamefont {Nori}},\
	}\bibinfo {title} {Quantum simulators},\ \href
	{https://doi.org/10.1126/science.1177838} {\bibfield  {journal} {\bibinfo
			{journal} {Science}\ }\textbf {\bibinfo {volume} {326}},\ \bibinfo {pages}
		{108} (\bibinfo {year} {2009})}\BibitemShut {NoStop}%
	\bibitem [{\citenamefont {Buluta}\ \emph {et~al.}(2011)\citenamefont {Buluta},
		\citenamefont {Ashhab},\ and\ \citenamefont {Nori}}]{Buluta_2011}%
	\BibitemOpen
	\bibfield  {author} {\bibinfo {author} {\bibfnamefont {I.}~\bibnamefont
			{Buluta}}, \bibinfo {author} {\bibfnamefont {S.}~\bibnamefont {Ashhab}},\
		and\ \bibinfo {author} {\bibfnamefont {F.}~\bibnamefont {Nori}},\ }\bibinfo
	{title} {Natural and artificial atoms for quantum computation},\ \href
	{https://doi.org/10.1088/0034-4885/74/10/104401} {\bibfield  {journal}
		{\bibinfo  {journal} {Rep. Prog. Phys.}\ }\textbf {\bibinfo {volume} {74}},\
		\bibinfo {pages} {104401} (\bibinfo {year} {2011})}\BibitemShut {NoStop}%
	\bibitem [{\citenamefont {Dalmonte}\ and\ \citenamefont
		{Montangero}(2016)}]{doi:10.1080/00107514.2016.1151199}%
	\BibitemOpen
	\bibfield  {author} {\bibinfo {author} {\bibfnamefont {M.}~\bibnamefont
			{Dalmonte}}\ and\ \bibinfo {author} {\bibfnamefont {S.}~\bibnamefont
			{Montangero}},\ }\bibinfo {title} {Lattice gauge theory simulations in the
		quantum information era},\ \href
	{https://doi.org/10.1080/00107514.2016.1151199} {\bibfield  {journal}
		{\bibinfo  {journal} {Contemp. Phys.}\ }\textbf {\bibinfo {volume} {57}},\
		\bibinfo {pages} {388} (\bibinfo {year} {2016})}\BibitemShut {NoStop}%
	\bibitem [{\citenamefont {Georgescu}\ \emph {et~al.}(2014)\citenamefont
		{Georgescu}, \citenamefont {Ashhab},\ and\ \citenamefont
		{Nori}}]{RevModPhys.86.153}%
	\BibitemOpen
	\bibfield  {author} {\bibinfo {author} {\bibfnamefont {I.~M.}\ \bibnamefont
			{Georgescu}}, \bibinfo {author} {\bibfnamefont {S.}~\bibnamefont {Ashhab}},\
		and\ \bibinfo {author} {\bibfnamefont {F.}~\bibnamefont {Nori}},\ }\bibinfo
	{title} {Quantum simulation},\ \href
	{https://doi.org/10.1103/RevModPhys.86.153} {\bibfield  {journal} {\bibinfo
			{journal} {Rev. Mod. Phys.}\ }\textbf {\bibinfo {volume} {86}},\ \bibinfo
		{pages} {153} (\bibinfo {year} {2014})}\BibitemShut {NoStop}%
	\bibitem [{\citenamefont {Gross}\ and\ \citenamefont {Bloch}(2017)}]{Gross995}%
	\BibitemOpen
	\bibfield  {author} {\bibinfo {author} {\bibfnamefont {C.}~\bibnamefont
			{Gross}}\ and\ \bibinfo {author} {\bibfnamefont {I.}~\bibnamefont {Bloch}},\
	}\bibinfo {title} {Quantum simulations with ultracold atoms in optical
		lattices},\ \href {https://doi.org/10.1126/science.aal3837} {\bibfield
		{journal} {\bibinfo  {journal} {Science}\ }\textbf {\bibinfo {volume}
			{357}},\ \bibinfo {pages} {995} (\bibinfo {year} {2017})}\BibitemShut
	{NoStop}%
	\bibitem [{\citenamefont {Browaeys}\ and\ \citenamefont
		{Lahaye}(2020)}]{s41567-019-0733-z}%
	\BibitemOpen
	\bibfield  {author} {\bibinfo {author} {\bibfnamefont {A.}~\bibnamefont
			{Browaeys}}\ and\ \bibinfo {author} {\bibfnamefont {T.}~\bibnamefont
			{Lahaye}},\ }\bibinfo {title} {Many-body physics with individually controlled
		Rydberg atoms},\ \href {https://doi.org/10.1038/s41567-019-0733-z} {\bibfield
		{journal} {\bibinfo  {journal} {Nat. Phys.}\ }\textbf {\bibinfo {volume}
			{16}},\ \bibinfo {pages} {132} (\bibinfo {year} {2020})}\BibitemShut
	{NoStop}%
	\bibitem [{\citenamefont {Polkovnikov}\ \emph {et~al.}(2011)\citenamefont
		{Polkovnikov}, \citenamefont {Sengupta}, \citenamefont {Silva},\ and\
		\citenamefont {Vengalattore}}]{RevModPhys.83.863}%
	\BibitemOpen
	\bibfield  {author} {\bibinfo {author} {\bibfnamefont {A.}~\bibnamefont
			{Polkovnikov}}, \bibinfo {author} {\bibfnamefont {K.}~\bibnamefont
			{Sengupta}}, \bibinfo {author} {\bibfnamefont {A.}~\bibnamefont {Silva}},\
		and\ \bibinfo {author} {\bibfnamefont {M.}~\bibnamefont {Vengalattore}},\
	}\bibinfo {title} {Colloquium: Nonequilibrium dynamics of closed interacting
		quantum systems},\ \href {https://doi.org/10.1103/RevModPhys.83.863}
	{\bibfield  {journal} {\bibinfo  {journal} {Rev. Mod. Phys.}\ }\textbf
		{\bibinfo {volume} {83}},\ \bibinfo {pages} {863} (\bibinfo {year}
		{2011})}\BibitemShut {NoStop}%
	\bibitem [{\citenamefont {Deutsch}(1991)}]{PhysRevA.43.2046}%
	\BibitemOpen
	\bibfield  {author} {\bibinfo {author} {\bibfnamefont {J.~M.}\ \bibnamefont
			{Deutsch}},\ }\bibinfo {title} {Quantum statistical mechanics in a closed
		system},\ \href {https://doi.org/10.1103/PhysRevA.43.2046} {\bibfield
		{journal} {\bibinfo  {journal} {Phys. Rev. A}\ }\textbf {\bibinfo {volume}
			{43}},\ \bibinfo {pages} {2046} (\bibinfo {year} {1991})}\BibitemShut
	{NoStop}%
	\bibitem [{\citenamefont {Srednicki}(1994)}]{PhysRevE.50.888}%
	\BibitemOpen
	\bibfield  {author} {\bibinfo {author} {\bibfnamefont {M.}~\bibnamefont
			{Srednicki}},\ }\bibinfo {title} {Chaos and quantum thermalization},\ \href
	{https://doi.org/10.1103/PhysRevE.50.888} {\bibfield  {journal} {\bibinfo
			{journal} {Phys. Rev. E}\ }\textbf {\bibinfo {volume} {50}},\ \bibinfo
		{pages} {888} (\bibinfo {year} {1994})}\BibitemShut {NoStop}%
	\bibitem [{\citenamefont {Srednicki}(1999)}]{Srednicki_1999}%
	\BibitemOpen
	\bibfield  {author} {\bibinfo {author} {\bibfnamefont {M.}~\bibnamefont
			{Srednicki}},\ }\bibinfo {title} {The approach to thermal equilibrium in
		quantized chaotic systems},\ \href
	{https://doi.org/10.1088/0305-4470/32/7/007} {\bibfield  {journal} {\bibinfo
			{journal} {Journal of Physics A: Mathematical and General}\ }\textbf
		{\bibinfo {volume} {32}},\ \bibinfo {pages} {1163} (\bibinfo {year}
		{1999})}\BibitemShut {NoStop}%
	\bibitem [{\citenamefont {{Rigol}}\ \emph {et~al.}(2008)\citenamefont
		{{Rigol}}, \citenamefont {{Dunjko}},\ and\ \citenamefont
		{{Olshanii}}}]{Rigol2008Nature}%
	\BibitemOpen
	\bibfield  {author} {\bibinfo {author} {\bibfnamefont {M.}~\bibnamefont
			{{Rigol}}}, \bibinfo {author} {\bibfnamefont {V.}~\bibnamefont {{Dunjko}}},\
		and\ \bibinfo {author} {\bibfnamefont {M.}~\bibnamefont {{Olshanii}}},\
	}\bibinfo {title} {{Thermalization and its mechanism for generic isolated
			quantum systems}},\ \href {https://doi.org/10.1038/nature06838} {\bibfield
		{journal} {\bibinfo  {journal} {\nat}\ }\textbf {\bibinfo {volume} {452}},\
		\bibinfo {pages} {854} (\bibinfo {year} {2008})}\BibitemShut {NoStop}%
	\bibitem [{\citenamefont {D'Alessio}\ \emph {et~al.}(2016)\citenamefont
		{D'Alessio}, \citenamefont {Kafri}, \citenamefont {Polkovnikov},\ and\
		\citenamefont {Rigol}}]{DAlessio2016}%
	\BibitemOpen
	\bibfield  {author} {\bibinfo {author} {\bibfnamefont {L.}~\bibnamefont
			{D'Alessio}}, \bibinfo {author} {\bibfnamefont {Y.}~\bibnamefont {Kafri}},
		\bibinfo {author} {\bibfnamefont {A.}~\bibnamefont {Polkovnikov}},\ and\
		\bibinfo {author} {\bibfnamefont {M.}~\bibnamefont {Rigol}},\ }\bibinfo
	{title} {From quantum chaos and eigenstate thermalization to statistical
		mechanics and thermodynamics},\ \href
	{https://doi.org/10.1080/00018732.2016.1198134} {\bibfield  {journal}
		{\bibinfo  {journal} {Advances in Physics}\ }\textbf {\bibinfo {volume}
			{65}},\ \bibinfo {pages} {239} (\bibinfo {year} {2016})}\BibitemShut
	{NoStop}%
	\bibitem [{\citenamefont {Kinoshita}\ \emph {et~al.}(2006)\citenamefont
		{Kinoshita}, \citenamefont {Wenger},\ and\ \citenamefont
		{Weiss}}]{kinoshita2006quantum}%
	\BibitemOpen
	\bibfield  {author} {\bibinfo {author} {\bibfnamefont {T.}~\bibnamefont
			{Kinoshita}}, \bibinfo {author} {\bibfnamefont {T.}~\bibnamefont {Wenger}},\
		and\ \bibinfo {author} {\bibfnamefont {D.~S.}\ \bibnamefont {Weiss}},\
	}\bibinfo {title} {A quantum Newton's cradle},\ \href
	{https://doi.org/10.1038/nature04693} {\bibfield  {journal} {\bibinfo
			{journal} {Nature}\ }\textbf {\bibinfo {volume} {440}},\ \bibinfo {pages}
		{900} (\bibinfo {year} {2006})}\BibitemShut {NoStop}%
	\bibitem [{\citenamefont {Rigol}\ \emph {et~al.}(2007)\citenamefont {Rigol},
		\citenamefont {Dunjko}, \citenamefont {Yurovsky},\ and\ \citenamefont
		{Olshanii}}]{PhysRevLett.98.050405}%
	\BibitemOpen
	\bibfield  {author} {\bibinfo {author} {\bibfnamefont {M.}~\bibnamefont
			{Rigol}}, \bibinfo {author} {\bibfnamefont {V.}~\bibnamefont {Dunjko}},
		\bibinfo {author} {\bibfnamefont {V.}~\bibnamefont {Yurovsky}},\ and\
		\bibinfo {author} {\bibfnamefont {M.}~\bibnamefont {Olshanii}},\ }\bibinfo
	{title} {Relaxation in a Completely Integrable Many-Body Quantum System: An
		Ab Initio Study of the Dynamics of the Highly Excited States of 1D Lattice
		Hard-Core Bosons},\ \href {https://doi.org/10.1103/PhysRevLett.98.050405}
	{\bibfield  {journal} {\bibinfo  {journal} {Phys. Rev. Lett.}\ }\textbf
		{\bibinfo {volume} {98}},\ \bibinfo {pages} {050405} (\bibinfo {year}
		{2007})}\BibitemShut {NoStop}%
	\bibitem [{\citenamefont {Basko}\ \emph {et~al.}(2006)\citenamefont {Basko},
		\citenamefont {Aleiner},\ and\ \citenamefont {Altshuler}}]{Basko2006}%
	\BibitemOpen
	\bibfield  {author} {\bibinfo {author} {\bibfnamefont {D.~M.}\ \bibnamefont
			{Basko}}, \bibinfo {author} {\bibfnamefont {I.~L.}\ \bibnamefont {Aleiner}},\
		and\ \bibinfo {author} {\bibfnamefont {B.~L.}\ \bibnamefont {Altshuler}},\
	}\bibinfo {title} {Metal-insulator transition in a weakly interacting
		many-electron system with localized single-particle states},\ \href
	{https://doi.org/10.1016/j.aop.2005.11.014} {\bibfield  {journal} {\bibinfo
			{journal} {Ann. Phys.}\ }\textbf {\bibinfo {volume} {321}} (\bibinfo {year}
		{2006})}\BibitemShut {NoStop}%
	\bibitem [{\citenamefont {Pal}\ and\ \citenamefont
		{Huse}(2010)}]{PhysRevB.82.174411}%
	\BibitemOpen
	\bibfield  {author} {\bibinfo {author} {\bibfnamefont {A.}~\bibnamefont
			{Pal}}\ and\ \bibinfo {author} {\bibfnamefont {D.~A.}\ \bibnamefont {Huse}},\
	}\bibinfo {title} {Many-body localization phase transition},\ \href
	{https://doi.org/10.1103/PhysRevB.82.174411} {\bibfield  {journal} {\bibinfo
			{journal} {Phys. Rev. B}\ }\textbf {\bibinfo {volume} {82}},\ \bibinfo
		{pages} {174411} (\bibinfo {year} {2010})}\BibitemShut {NoStop}%
	\bibitem [{\citenamefont {Kj\"all}\ \emph {et~al.}(2014)\citenamefont
		{Kj\"all}, \citenamefont {Bardarson},\ and\ \citenamefont
		{Pollmann}}]{PhysRevLett.113.107204}%
	\BibitemOpen
	\bibfield  {author} {\bibinfo {author} {\bibfnamefont {J.~A.}\ \bibnamefont
			{Kj\"all}}, \bibinfo {author} {\bibfnamefont {J.~H.}\ \bibnamefont
			{Bardarson}},\ and\ \bibinfo {author} {\bibfnamefont {F.}~\bibnamefont
			{Pollmann}},\ }\bibinfo {title} {Many-Body Localization in a Disordered
		Quantum Ising Chain},\ \href {https://doi.org/10.1103/PhysRevLett.113.107204}
	{\bibfield  {journal} {\bibinfo  {journal} {Phys. Rev. Lett.}\ }\textbf
		{\bibinfo {volume} {113}},\ \bibinfo {pages} {107204} (\bibinfo {year}
		{2014})}\BibitemShut {NoStop}%
	\bibitem [{\citenamefont {Huse}\ \emph {et~al.}(2014)\citenamefont {Huse},
		\citenamefont {Nandkishore},\ and\ \citenamefont
		{Oganesyan}}]{PhysRevB.90.174202}%
	\BibitemOpen
	\bibfield  {author} {\bibinfo {author} {\bibfnamefont {D.~A.}\ \bibnamefont
			{Huse}}, \bibinfo {author} {\bibfnamefont {R.}~\bibnamefont {Nandkishore}},\
		and\ \bibinfo {author} {\bibfnamefont {V.}~\bibnamefont {Oganesyan}},\
	}\bibinfo {title} {Phenomenology of fully many-body-localized systems},\
	\href {https://doi.org/10.1103/PhysRevB.90.174202} {\bibfield  {journal}
		{\bibinfo  {journal} {Phys. Rev. B}\ }\textbf {\bibinfo {volume} {90}},\
		\bibinfo {pages} {174202} (\bibinfo {year} {2014})}\BibitemShut {NoStop}%
	\bibitem [{\citenamefont {Serbyn}\ and\ \citenamefont
		{Moore}(2016)}]{PhysRevB.93.041424}%
	\BibitemOpen
	\bibfield  {author} {\bibinfo {author} {\bibfnamefont {M.}~\bibnamefont
			{Serbyn}}\ and\ \bibinfo {author} {\bibfnamefont {J.~E.}\ \bibnamefont
			{Moore}},\ }\bibinfo {title} {Spectral statistics across the many-body
		localization transition},\ \href {https://doi.org/10.1103/PhysRevB.93.041424}
	{\bibfield  {journal} {\bibinfo  {journal} {Phys. Rev. B}\ }\textbf {\bibinfo
			{volume} {93}},\ \bibinfo {pages} {041424} (\bibinfo {year}
		{2016})}\BibitemShut {NoStop}%
	\bibitem [{\citenamefont {Nandkishore}\ and\ \citenamefont
		{Huse}(2015)}]{Rahul2015}%
	\BibitemOpen
	\bibfield  {author} {\bibinfo {author} {\bibfnamefont {R.}~\bibnamefont
			{Nandkishore}}\ and\ \bibinfo {author} {\bibfnamefont {D.~A.}\ \bibnamefont
			{Huse}},\ }\bibinfo {title} {Many-Body Localization and Thermalization in
		Quantum Statistical Mechanics},\ \href
	{https://doi.org/10.1146/annurev-conmatphys-031214-014726} {\bibfield
		{journal} {\bibinfo  {journal} {Annual Review of Condensed Matter Physics}\
		}\textbf {\bibinfo {volume} {6}},\ \bibinfo {pages} {15} (\bibinfo {year}
		{2015})}\BibitemShut {NoStop}%
	\bibitem [{\citenamefont {Abanin}\ \emph {et~al.}(2019)\citenamefont {Abanin},
		\citenamefont {Altman}, \citenamefont {Bloch},\ and\ \citenamefont
		{Serbyn}}]{RevModPhys.91.021001}%
	\BibitemOpen
	\bibfield  {author} {\bibinfo {author} {\bibfnamefont {D.~A.}\ \bibnamefont
			{Abanin}}, \bibinfo {author} {\bibfnamefont {E.}~\bibnamefont {Altman}},
		\bibinfo {author} {\bibfnamefont {I.}~\bibnamefont {Bloch}},\ and\ \bibinfo
		{author} {\bibfnamefont {M.}~\bibnamefont {Serbyn}},\ }\bibinfo {title}
	{Colloquium: Many-body localization, thermalization, and entanglement},\
	\href {https://doi.org/10.1103/RevModPhys.91.021001} {\bibfield  {journal}
		{\bibinfo  {journal} {Rev. Mod. Phys.}\ }\textbf {\bibinfo {volume} {91}},\
		\bibinfo {pages} {021001} (\bibinfo {year} {2019})}\BibitemShut {NoStop}%
	\bibitem [{\citenamefont {Bernien}\ \emph {et~al.}(2017)\citenamefont
		{Bernien}, \citenamefont {Schwartz}, \citenamefont {Keesling}, \citenamefont
		{Levine}, \citenamefont {Omran}, \citenamefont {Pichler}, \citenamefont
		{Choi}, \citenamefont {Zibrov}, \citenamefont {Endres}, \citenamefont
		{Greiner} \emph {et~al.}}]{bernien2017probing}%
	\BibitemOpen
	\bibfield  {author} {\bibinfo {author} {\bibfnamefont {H.}~\bibnamefont
			{Bernien}}, \bibinfo {author} {\bibfnamefont {S.}~\bibnamefont {Schwartz}},
		\bibinfo {author} {\bibfnamefont {A.}~\bibnamefont {Keesling}}, \bibinfo
		{author} {\bibfnamefont {H.}~\bibnamefont {Levine}}, \bibinfo {author}
		{\bibfnamefont {A.}~\bibnamefont {Omran}}, \bibinfo {author} {\bibfnamefont
			{H.}~\bibnamefont {Pichler}}, \bibinfo {author} {\bibfnamefont
			{S.}~\bibnamefont {Choi}}, \bibinfo {author} {\bibfnamefont {A.~S.}\
			\bibnamefont {Zibrov}}, \bibinfo {author} {\bibfnamefont {M.}~\bibnamefont
			{Endres}}, \bibinfo {author} {\bibfnamefont {M.}~\bibnamefont {Greiner}},
		\emph {et~al.},\ }\bibinfo {title} {Probing many-body dynamics on a 51-atom
		quantum simulator},\ \href {https://doi.org/10.1038/nature24622} {\bibfield
		{journal} {\bibinfo  {journal} {Nature}\ }\textbf {\bibinfo {volume} {551}},\
		\bibinfo {pages} {579} (\bibinfo {year} {2017})}\BibitemShut {NoStop}%
	\bibitem [{\citenamefont {Turner}\ \emph
		{et~al.}(2018{\natexlab{a}})\citenamefont {Turner}, \citenamefont
		{Michailidis}, \citenamefont {Abanin}, \citenamefont {Serbyn},\ and\
		\citenamefont {Papi{\'c}}}]{turner2018weak}%
	\BibitemOpen
	\bibfield  {author} {\bibinfo {author} {\bibfnamefont {C.~J.}\ \bibnamefont
			{Turner}}, \bibinfo {author} {\bibfnamefont {A.~A.}\ \bibnamefont
			{Michailidis}}, \bibinfo {author} {\bibfnamefont {D.~A.}\ \bibnamefont
			{Abanin}}, \bibinfo {author} {\bibfnamefont {M.}~\bibnamefont {Serbyn}},\
		and\ \bibinfo {author} {\bibfnamefont {Z.}~\bibnamefont {Papi{\'c}}},\
	}\bibinfo {title} {Weak ergodicity breaking from quantum many-body scars},\
	\href {https://doi.org/https://doi.org/10.1038/s41567-018-0137-5} {\bibfield
		{journal} {\bibinfo  {journal} {Nature Physics}\ }\textbf {\bibinfo {volume}
			{14}},\ \bibinfo {pages} {745} (\bibinfo {year}
		{2018}{\natexlab{a}})}\BibitemShut {NoStop}%
	\bibitem [{\citenamefont {Turner}\ \emph
		{et~al.}(2018{\natexlab{b}})\citenamefont {Turner}, \citenamefont
		{Michailidis}, \citenamefont {Abanin}, \citenamefont {Serbyn},\ and\
		\citenamefont {Papi\ifmmode~\acute{c}\else \'{c}\fi{}}}]{PhysRevB.98.155134}%
	\BibitemOpen
	\bibfield  {author} {\bibinfo {author} {\bibfnamefont {C.~J.}\ \bibnamefont
			{Turner}}, \bibinfo {author} {\bibfnamefont {A.~A.}\ \bibnamefont
			{Michailidis}}, \bibinfo {author} {\bibfnamefont {D.~A.}\ \bibnamefont
			{Abanin}}, \bibinfo {author} {\bibfnamefont {M.}~\bibnamefont {Serbyn}},\
		and\ \bibinfo {author} {\bibfnamefont {Z.}~\bibnamefont
			{Papi\ifmmode~\acute{c}\else \'{c}\fi{}}},\ }\bibinfo {title} {Quantum
		scarred eigenstates in a Rydberg atom chain: Entanglement, breakdown of
		thermalization, and stability to perturbations},\ \href
	{https://doi.org/10.1103/PhysRevB.98.155134} {\bibfield  {journal} {\bibinfo
			{journal} {Phys. Rev. B}\ }\textbf {\bibinfo {volume} {98}},\ \bibinfo
		{pages} {155134} (\bibinfo {year} {2018}{\natexlab{b}})}\BibitemShut
	{NoStop}%
	\bibitem [{\citenamefont {Lin}\ and\ \citenamefont
		{Motrunich}(2019)}]{PhysRevLett.122.173401}%
	\BibitemOpen
	\bibfield  {author} {\bibinfo {author} {\bibfnamefont {C.-J.}\ \bibnamefont
			{Lin}}\ and\ \bibinfo {author} {\bibfnamefont {O.~I.}\ \bibnamefont
			{Motrunich}},\ }\bibinfo {title} {Exact Quantum Many-Body Scar States in the
		Rydberg-Blockaded Atom Chain},\ \href
	{https://doi.org/10.1103/PhysRevLett.122.173401} {\bibfield  {journal}
		{\bibinfo  {journal} {Phys. Rev. Lett.}\ }\textbf {\bibinfo {volume} {122}},\
		\bibinfo {pages} {173401} (\bibinfo {year} {2019})}\BibitemShut {NoStop}%
	\bibitem [{\citenamefont {Moudgalya}\ \emph
		{et~al.}(2018{\natexlab{a}})\citenamefont {Moudgalya}, \citenamefont
		{Rachel}, \citenamefont {Bernevig},\ and\ \citenamefont
		{Regnault}}]{PhysRevB.98.235155}%
	\BibitemOpen
	\bibfield  {author} {\bibinfo {author} {\bibfnamefont {S.}~\bibnamefont
			{Moudgalya}}, \bibinfo {author} {\bibfnamefont {S.}~\bibnamefont {Rachel}},
		\bibinfo {author} {\bibfnamefont {B.~A.}\ \bibnamefont {Bernevig}},\ and\
		\bibinfo {author} {\bibfnamefont {N.}~\bibnamefont {Regnault}},\ }\bibinfo
	{title} {Exact excited states of nonintegrable models},\ \href
	{https://doi.org/10.1103/PhysRevB.98.235155} {\bibfield  {journal} {\bibinfo
			{journal} {Phys. Rev. B}\ }\textbf {\bibinfo {volume} {98}},\ \bibinfo
		{pages} {235155} (\bibinfo {year} {2018}{\natexlab{a}})}\BibitemShut
	{NoStop}%
	\bibitem [{\citenamefont {Moudgalya}\ \emph
		{et~al.}(2018{\natexlab{b}})\citenamefont {Moudgalya}, \citenamefont
		{Regnault},\ and\ \citenamefont {Bernevig}}]{PhysRevB.98.235156}%
	\BibitemOpen
	\bibfield  {author} {\bibinfo {author} {\bibfnamefont {S.}~\bibnamefont
			{Moudgalya}}, \bibinfo {author} {\bibfnamefont {N.}~\bibnamefont
			{Regnault}},\ and\ \bibinfo {author} {\bibfnamefont {B.~A.}\ \bibnamefont
			{Bernevig}},\ }\bibinfo {title} {Entanglement of exact excited states of
		Affleck-Kennedy-Lieb-Tasaki models: Exact results, many-body scars, and
		violation of the strong eigenstate thermalization hypothesis},\ \href
	{https://doi.org/10.1103/PhysRevB.98.235156} {\bibfield  {journal} {\bibinfo
			{journal} {Phys. Rev. B}\ }\textbf {\bibinfo {volume} {98}},\ \bibinfo
		{pages} {235156} (\bibinfo {year} {2018}{\natexlab{b}})}\BibitemShut
	{NoStop}%
	\bibitem [{\citenamefont {Choi}\ \emph {et~al.}(2019)\citenamefont {Choi},
		\citenamefont {Turner}, \citenamefont {Pichler}, \citenamefont {Ho},
		\citenamefont {Michailidis}, \citenamefont {Papi\ifmmode~\acute{c}\else
			\'{c}\fi{}}, \citenamefont {Serbyn}, \citenamefont {Lukin},\ and\
		\citenamefont {Abanin}}]{PhysRevLett.122.220603}%
	\BibitemOpen
	\bibfield  {author} {\bibinfo {author} {\bibfnamefont {S.}~\bibnamefont
			{Choi}}, \bibinfo {author} {\bibfnamefont {C.~J.}\ \bibnamefont {Turner}},
		\bibinfo {author} {\bibfnamefont {H.}~\bibnamefont {Pichler}}, \bibinfo
		{author} {\bibfnamefont {W.~W.}\ \bibnamefont {Ho}}, \bibinfo {author}
		{\bibfnamefont {A.~A.}\ \bibnamefont {Michailidis}}, \bibinfo {author}
		{\bibfnamefont {Z.}~\bibnamefont {Papi\ifmmode~\acute{c}\else \'{c}\fi{}}},
		\bibinfo {author} {\bibfnamefont {M.}~\bibnamefont {Serbyn}}, \bibinfo
		{author} {\bibfnamefont {M.~D.}\ \bibnamefont {Lukin}},\ and\ \bibinfo
		{author} {\bibfnamefont {D.~A.}\ \bibnamefont {Abanin}},\ }\bibinfo {title}
	{Emergent SU(2) Dynamics and Perfect Quantum Many-Body Scars},\ \href
	{https://doi.org/10.1103/PhysRevLett.122.220603} {\bibfield  {journal}
		{\bibinfo  {journal} {Phys. Rev. Lett.}\ }\textbf {\bibinfo {volume} {122}},\
		\bibinfo {pages} {220603} (\bibinfo {year} {2019})}\BibitemShut {NoStop}%
	\bibitem [{\citenamefont {Ho}\ \emph {et~al.}(2019)\citenamefont {Ho},
		\citenamefont {Choi}, \citenamefont {Pichler},\ and\ \citenamefont
		{Lukin}}]{PhysRevLett.122.040603}%
	\BibitemOpen
	\bibfield  {author} {\bibinfo {author} {\bibfnamefont {W.~W.}\ \bibnamefont
			{Ho}}, \bibinfo {author} {\bibfnamefont {S.}~\bibnamefont {Choi}}, \bibinfo
		{author} {\bibfnamefont {H.}~\bibnamefont {Pichler}},\ and\ \bibinfo {author}
		{\bibfnamefont {M.~D.}\ \bibnamefont {Lukin}},\ }\bibinfo {title} {Periodic
		Orbits, Entanglement, and Quantum Many-Body Scars in Constrained Models:
		Matrix Product State Approach},\ \href
	{https://doi.org/10.1103/PhysRevLett.122.040603} {\bibfield  {journal}
		{\bibinfo  {journal} {Phys. Rev. Lett.}\ }\textbf {\bibinfo {volume} {122}},\
		\bibinfo {pages} {040603} (\bibinfo {year} {2019})}\BibitemShut {NoStop}%
	\bibitem [{\citenamefont {Shiraishi}(2019)}]{Shiraishi_2019}%
	\BibitemOpen
	\bibfield  {author} {\bibinfo {author} {\bibfnamefont {N.}~\bibnamefont
			{Shiraishi}},\ }\bibinfo {title} {Connection between quantum-many-body scars
		and the Affleck{\textendash}Kennedy{\textendash}Lieb{\textendash}Tasaki model
		from the viewpoint of embedded Hamiltonians},\ \href
	{https://doi.org/10.1088/1742-5468/ab342e} {\bibfield  {journal} {\bibinfo
			{journal} {Journal of Statistical Mechanics: Theory and Experiment}\ }\textbf
		{\bibinfo {volume} {2019}},\ \bibinfo {pages} {083103} (\bibinfo {year}
		{2019})}\BibitemShut {NoStop}%
	\bibitem [{\citenamefont {Schecter}\ and\ \citenamefont
		{Iadecola}(2019)}]{PhysRevLett.123.147201}%
	\BibitemOpen
	\bibfield  {author} {\bibinfo {author} {\bibfnamefont {M.}~\bibnamefont
			{Schecter}}\ and\ \bibinfo {author} {\bibfnamefont {T.}~\bibnamefont
			{Iadecola}},\ }\bibinfo {title} {Weak Ergodicity Breaking and Quantum
		Many-Body Scars in Spin-1 $XY$ Magnets},\ \href
	{https://doi.org/10.1103/PhysRevLett.123.147201} {\bibfield  {journal}
		{\bibinfo  {journal} {Phys. Rev. Lett.}\ }\textbf {\bibinfo {volume} {123}},\
		\bibinfo {pages} {147201} (\bibinfo {year} {2019})}\BibitemShut {NoStop}%
	\bibitem [{\citenamefont {Lin}\ \emph {et~al.}(2020)\citenamefont {Lin},
		\citenamefont {Chandran},\ and\ \citenamefont
		{Motrunich}}]{PhysRevResearch.2.033044}%
	\BibitemOpen
	\bibfield  {author} {\bibinfo {author} {\bibfnamefont {C.-J.}\ \bibnamefont
			{Lin}}, \bibinfo {author} {\bibfnamefont {A.}~\bibnamefont {Chandran}},\ and\
		\bibinfo {author} {\bibfnamefont {O.~I.}\ \bibnamefont {Motrunich}},\
	}\bibinfo {title} {Slow thermalization of exact quantum many-body scar states
		under perturbations},\ \href
	{https://doi.org/10.1103/PhysRevResearch.2.033044} {\bibfield  {journal}
		{\bibinfo  {journal} {Phys. Rev. Research}\ }\textbf {\bibinfo {volume}
			{2}},\ \bibinfo {pages} {033044} (\bibinfo {year} {2020})}\BibitemShut
	{NoStop}%
	\bibitem [{\citenamefont {Mark}\ \emph {et~al.}(2020)\citenamefont {Mark},
		\citenamefont {Lin},\ and\ \citenamefont {Motrunich}}]{PhysRevB.101.195131}%
	\BibitemOpen
	\bibfield  {author} {\bibinfo {author} {\bibfnamefont {D.~K.}\ \bibnamefont
			{Mark}}, \bibinfo {author} {\bibfnamefont {C.-J.}\ \bibnamefont {Lin}},\ and\
		\bibinfo {author} {\bibfnamefont {O.~I.}\ \bibnamefont {Motrunich}},\
	}\bibinfo {title} {Unified structure for exact towers of scar states in the
		Affleck-Kennedy-Lieb-Tasaki and other models},\ \href
	{https://doi.org/10.1103/PhysRevB.101.195131} {\bibfield  {journal} {\bibinfo
			{journal} {Phys. Rev. B}\ }\textbf {\bibinfo {volume} {101}},\ \bibinfo
		{pages} {195131} (\bibinfo {year} {2020})}\BibitemShut {NoStop}%
	\bibitem [{\citenamefont {Mark}\ and\ \citenamefont
		{Motrunich}(2020)}]{PhysRevB.102.075132}%
	\BibitemOpen
	\bibfield  {author} {\bibinfo {author} {\bibfnamefont {D.~K.}\ \bibnamefont
			{Mark}}\ and\ \bibinfo {author} {\bibfnamefont {O.~I.}\ \bibnamefont
			{Motrunich}},\ }\bibinfo {title} {$\ensuremath{\eta}$-pairing states as true
		scars in an extended Hubbard model},\ \href
	{https://doi.org/10.1103/PhysRevB.102.075132} {\bibfield  {journal} {\bibinfo
			{journal} {Phys. Rev. B}\ }\textbf {\bibinfo {volume} {102}},\ \bibinfo
		{pages} {075132} (\bibinfo {year} {2020})}\BibitemShut {NoStop}%
	\bibitem [{\citenamefont {Moudgalya}\ \emph {et~al.}(2020)\citenamefont
		{Moudgalya}, \citenamefont {Regnault},\ and\ \citenamefont
		{Bernevig}}]{PhysRevB.102.085140}%
	\BibitemOpen
	\bibfield  {author} {\bibinfo {author} {\bibfnamefont {S.}~\bibnamefont
			{Moudgalya}}, \bibinfo {author} {\bibfnamefont {N.}~\bibnamefont
			{Regnault}},\ and\ \bibinfo {author} {\bibfnamefont {B.~A.}\ \bibnamefont
			{Bernevig}},\ }\bibinfo {title} {$\ensuremath{\eta}$-pairing in Hubbard
		models: From spectrum generating algebras to quantum many-body scars},\ \href
	{https://doi.org/10.1103/PhysRevB.102.085140} {\bibfield  {journal} {\bibinfo
			{journal} {Phys. Rev. B}\ }\textbf {\bibinfo {volume} {102}},\ \bibinfo
		{pages} {085140} (\bibinfo {year} {2020})}\BibitemShut {NoStop}%
	\bibitem [{\citenamefont {O'Dea}\ \emph {et~al.}(2020)\citenamefont {O'Dea},
		\citenamefont {Burnell}, \citenamefont {Chandran},\ and\ \citenamefont
		{Khemani}}]{PhysRevResearch.2.043305}%
	\BibitemOpen
	\bibfield  {author} {\bibinfo {author} {\bibfnamefont {N.}~\bibnamefont
			{O'Dea}}, \bibinfo {author} {\bibfnamefont {F.}~\bibnamefont {Burnell}},
		\bibinfo {author} {\bibfnamefont {A.}~\bibnamefont {Chandran}},\ and\
		\bibinfo {author} {\bibfnamefont {V.}~\bibnamefont {Khemani}},\ }\bibinfo
	{title} {From tunnels to towers: Quantum scars from Lie algebras and
		$q$-deformed Lie algebras},\ \href
	{https://doi.org/10.1103/PhysRevResearch.2.043305} {\bibfield  {journal}
		{\bibinfo  {journal} {Phys. Rev. Research}\ }\textbf {\bibinfo {volume}
			{2}},\ \bibinfo {pages} {043305} (\bibinfo {year} {2020})}\BibitemShut
	{NoStop}%
	\bibitem [{\citenamefont {Shibata}\ \emph {et~al.}(2020)\citenamefont
		{Shibata}, \citenamefont {Yoshioka},\ and\ \citenamefont
		{Katsura}}]{PhysRevLett.124.180604}%
	\BibitemOpen
	\bibfield  {author} {\bibinfo {author} {\bibfnamefont {N.}~\bibnamefont
			{Shibata}}, \bibinfo {author} {\bibfnamefont {N.}~\bibnamefont {Yoshioka}},\
		and\ \bibinfo {author} {\bibfnamefont {H.}~\bibnamefont {Katsura}},\
	}\bibinfo {title} {Onsager's Scars in Disordered Spin Chains},\ \href
	{https://doi.org/10.1103/PhysRevLett.124.180604} {\bibfield  {journal}
		{\bibinfo  {journal} {Phys. Rev. Lett.}\ }\textbf {\bibinfo {volume} {124}},\
		\bibinfo {pages} {180604} (\bibinfo {year} {2020})}\BibitemShut {NoStop}%
	\bibitem [{\citenamefont {Pakrouski}\ \emph {et~al.}(2020)\citenamefont
		{Pakrouski}, \citenamefont {Pallegar}, \citenamefont {Popov},\ and\
		\citenamefont {Klebanov}}]{PhysRevLett.125.230602}%
	\BibitemOpen
	\bibfield  {author} {\bibinfo {author} {\bibfnamefont {K.}~\bibnamefont
			{Pakrouski}}, \bibinfo {author} {\bibfnamefont {P.~N.}\ \bibnamefont
			{Pallegar}}, \bibinfo {author} {\bibfnamefont {F.~K.}\ \bibnamefont
			{Popov}},\ and\ \bibinfo {author} {\bibfnamefont {I.~R.}\ \bibnamefont
			{Klebanov}},\ }\bibinfo {title} {Many-Body Scars as a Group Invariant Sector
		of Hilbert Space},\ \href {https://doi.org/10.1103/PhysRevLett.125.230602}
	{\bibfield  {journal} {\bibinfo  {journal} {Phys. Rev. Lett.}\ }\textbf
		{\bibinfo {volume} {125}},\ \bibinfo {pages} {230602} (\bibinfo {year}
		{2020})}\BibitemShut {NoStop}%
	\bibitem [{\citenamefont {Ren}\ \emph {et~al.}(2021)\citenamefont {Ren},
		\citenamefont {Liang},\ and\ \citenamefont {Fang}}]{PhysRevLett.126.120604}%
	\BibitemOpen
	\bibfield  {author} {\bibinfo {author} {\bibfnamefont {J.}~\bibnamefont
			{Ren}}, \bibinfo {author} {\bibfnamefont {C.}~\bibnamefont {Liang}},\ and\
		\bibinfo {author} {\bibfnamefont {C.}~\bibnamefont {Fang}},\ }\bibinfo
	{title} {Quasisymmetry Groups and Many-Body Scar Dynamics},\ \href
	{https://doi.org/10.1103/PhysRevLett.126.120604} {\bibfield  {journal}
		{\bibinfo  {journal} {Phys. Rev. Lett.}\ }\textbf {\bibinfo {volume} {126}},\
		\bibinfo {pages} {120604} (\bibinfo {year} {2021})}\BibitemShut {NoStop}%
	\bibitem [{\citenamefont {Moudgalya}\ \emph {et~al.}(2022)\citenamefont
		{Moudgalya}, \citenamefont {Bernevig},\ and\ \citenamefont
		{Regnault}}]{Moudgalya_2022}%
	\BibitemOpen
	\bibfield  {author} {\bibinfo {author} {\bibfnamefont {S.}~\bibnamefont
			{Moudgalya}}, \bibinfo {author} {\bibfnamefont {B.~A.}\ \bibnamefont
			{Bernevig}},\ and\ \bibinfo {author} {\bibfnamefont {N.}~\bibnamefont
			{Regnault}},\ }\bibinfo {title} {Quantum many-body scars and Hilbert space
		fragmentation: a review of exact results},\ \href
	{https://doi.org/10.1088/1361-6633/ac73a0} {\bibfield  {journal} {\bibinfo
			{journal} {Reports on Progress in Physics}\ }\textbf {\bibinfo {volume}
			{85}},\ \bibinfo {pages} {086501} (\bibinfo {year} {2022})}\BibitemShut
	{NoStop}%
	\bibitem [{\citenamefont {Wilson}(1974)}]{PhysRevD.10.2445}%
	\BibitemOpen
	\bibfield  {author} {\bibinfo {author} {\bibfnamefont {K.~G.}\ \bibnamefont
			{Wilson}},\ }\bibinfo {title} {Confinement of quarks},\ \href
	{https://doi.org/10.1103/PhysRevD.10.2445} {\bibfield  {journal} {\bibinfo
			{journal} {Phys. Rev. D}\ }\textbf {\bibinfo {volume} {10}},\ \bibinfo
		{pages} {2445} (\bibinfo {year} {1974})}\BibitemShut {NoStop}%
	\bibitem [{\citenamefont {Kogut}(1979)}]{RevModPhys.51.659}%
	\BibitemOpen
	\bibfield  {author} {\bibinfo {author} {\bibfnamefont {J.~B.}\ \bibnamefont
			{Kogut}},\ }\bibinfo {title} {An introduction to lattice gauge theory and
		spin systems},\ \href {https://doi.org/10.1103/RevModPhys.51.659} {\bibfield
		{journal} {\bibinfo  {journal} {Rev. Mod. Phys.}\ }\textbf {\bibinfo {volume}
			{51}},\ \bibinfo {pages} {659} (\bibinfo {year} {1979})}\BibitemShut
	{NoStop}%
	\bibitem [{\citenamefont {Moessner}\ \emph {et~al.}(2001)\citenamefont
		{Moessner}, \citenamefont {Sondhi},\ and\ \citenamefont
		{Fradkin}}]{PhysRevB.65.024504}%
	\BibitemOpen
	\bibfield  {author} {\bibinfo {author} {\bibfnamefont {R.}~\bibnamefont
			{Moessner}}, \bibinfo {author} {\bibfnamefont {S.~L.}\ \bibnamefont
			{Sondhi}},\ and\ \bibinfo {author} {\bibfnamefont {E.}~\bibnamefont
			{Fradkin}},\ }\bibinfo {title} {Short-ranged resonating valence bond physics,
		quantum dimer models, and Ising gauge theories},\ \href
	{https://doi.org/10.1103/PhysRevB.65.024504} {\bibfield  {journal} {\bibinfo
			{journal} {Phys. Rev. B}\ }\textbf {\bibinfo {volume} {65}},\ \bibinfo
		{pages} {024504} (\bibinfo {year} {2001})}\BibitemShut {NoStop}%
	\bibitem [{\citenamefont {Fradkin}\ and\ \citenamefont
		{Susskind}(1978)}]{PhysRevD.17.2637}%
	\BibitemOpen
	\bibfield  {author} {\bibinfo {author} {\bibfnamefont {E.}~\bibnamefont
			{Fradkin}}\ and\ \bibinfo {author} {\bibfnamefont {L.}~\bibnamefont
			{Susskind}},\ }\bibinfo {title} {Order and disorder in gauge systems and
		magnets},\ \href {https://doi.org/10.1103/PhysRevD.17.2637} {\bibfield
		{journal} {\bibinfo  {journal} {Phys. Rev. D}\ }\textbf {\bibinfo {volume}
			{17}},\ \bibinfo {pages} {2637} (\bibinfo {year} {1978})}\BibitemShut
	{NoStop}%
	\bibitem [{\citenamefont {Fradkin}\ and\ \citenamefont
		{Shenker}(1979)}]{PhysRevD.19.3682}%
	\BibitemOpen
	\bibfield  {author} {\bibinfo {author} {\bibfnamefont {E.}~\bibnamefont
			{Fradkin}}\ and\ \bibinfo {author} {\bibfnamefont {S.~H.}\ \bibnamefont
			{Shenker}},\ }\bibinfo {title} {Phase diagrams of lattice gauge theories with
		Higgs fields},\ \href {https://doi.org/10.1103/PhysRevD.19.3682} {\bibfield
		{journal} {\bibinfo  {journal} {Phys. Rev. D}\ }\textbf {\bibinfo {volume}
			{19}},\ \bibinfo {pages} {3682} (\bibinfo {year} {1979})}\BibitemShut
	{NoStop}%
	\bibitem [{\citenamefont {Fradkin}(2013)}]{Fradkin2013}%
	\BibitemOpen
	\bibfield  {author} {\bibinfo {author} {\bibfnamefont {E.}~\bibnamefont
			{Fradkin}},\ }\href@noop {} {\emph {\bibinfo {title} {Field theories of
				condensed matter physics}}}\ (\bibinfo  {publisher} {Cambridge University
		Press, Cambridge, England},\ \bibinfo {year} {2013})\BibitemShut {NoStop}%
	\bibitem [{\citenamefont {Byrnes}\ and\ \citenamefont
		{Yamamoto}(2006)}]{PhysRevA.73.022328}%
	\BibitemOpen
	\bibfield  {author} {\bibinfo {author} {\bibfnamefont {T.}~\bibnamefont
			{Byrnes}}\ and\ \bibinfo {author} {\bibfnamefont {Y.}~\bibnamefont
			{Yamamoto}},\ }\bibinfo {title} {Simulating lattice gauge theories on a
		quantum computer},\ \href {https://doi.org/10.1103/PhysRevA.73.022328}
	{\bibfield  {journal} {\bibinfo  {journal} {Phys. Rev. A}\ }\textbf {\bibinfo
			{volume} {73}},\ \bibinfo {pages} {022328} (\bibinfo {year}
		{2006})}\BibitemShut {NoStop}%
	\bibitem [{\citenamefont {Banerjee}\ \emph {et~al.}(2012)\citenamefont
		{Banerjee}, \citenamefont {Dalmonte}, \citenamefont {M\"uller}, \citenamefont
		{Rico}, \citenamefont {Stebler}, \citenamefont {Wiese},\ and\ \citenamefont
		{Zoller}}]{PhysRevLett.109.175302}%
	\BibitemOpen
	\bibfield  {author} {\bibinfo {author} {\bibfnamefont {D.}~\bibnamefont
			{Banerjee}}, \bibinfo {author} {\bibfnamefont {M.}~\bibnamefont {Dalmonte}},
		\bibinfo {author} {\bibfnamefont {M.}~\bibnamefont {M\"uller}}, \bibinfo
		{author} {\bibfnamefont {E.}~\bibnamefont {Rico}}, \bibinfo {author}
		{\bibfnamefont {P.}~\bibnamefont {Stebler}}, \bibinfo {author} {\bibfnamefont
			{U.-J.}\ \bibnamefont {Wiese}},\ and\ \bibinfo {author} {\bibfnamefont
			{P.}~\bibnamefont {Zoller}},\ }\bibinfo {title} {Atomic quantum simulation of
		dynamical gauge fields coupled to Fermionic matter: From string breaking to
		evolution after a quench},\ \href
	{https://doi.org/10.1103/PhysRevLett.109.175302} {\bibfield  {journal}
		{\bibinfo  {journal} {Phys. Rev. Lett.}\ }\textbf {\bibinfo {volume} {109}},\
		\bibinfo {pages} {175302} (\bibinfo {year} {2012})}\BibitemShut {NoStop}%
	\bibitem [{\citenamefont {Di~Stefano}\ \emph {et~al.}(2019)\citenamefont
		{Di~Stefano}, \citenamefont {Settineri}, \citenamefont {Macr{\`\i}},
		\citenamefont {Garziano}, \citenamefont {Stassi}, \citenamefont {Savasta},\
		and\ \citenamefont {Nori}}]{di2019resolution}%
	\BibitemOpen
	\bibfield  {author} {\bibinfo {author} {\bibfnamefont {O.}~\bibnamefont
			{Di~Stefano}}, \bibinfo {author} {\bibfnamefont {A.}~\bibnamefont
			{Settineri}}, \bibinfo {author} {\bibfnamefont {V.}~\bibnamefont
			{Macr{\`\i}}}, \bibinfo {author} {\bibfnamefont {L.}~\bibnamefont
			{Garziano}}, \bibinfo {author} {\bibfnamefont {R.}~\bibnamefont {Stassi}},
		\bibinfo {author} {\bibfnamefont {S.}~\bibnamefont {Savasta}},\ and\ \bibinfo
		{author} {\bibfnamefont {F.}~\bibnamefont {Nori}},\ }\bibinfo {title}
	{Resolution of gauge ambiguities in ultrastrong-coupling cavity quantum
		electrodynamics},\ \href {https://doi.org/10.1038/s41567-019-0534-4}
	{\bibfield  {journal} {\bibinfo  {journal} {Nature Physics}\ }\textbf
		{\bibinfo {volume} {15}},\ \bibinfo {pages} {803} (\bibinfo {year}
		{2019})}\BibitemShut {NoStop}%
	\bibitem [{\citenamefont {Banuls}\ \emph {et~al.}(2020)\citenamefont {Banuls},
		\citenamefont {Blatt}, \citenamefont {Catani}, \citenamefont {Celi},
		\citenamefont {Cirac}, \citenamefont {Dalmonte}, \citenamefont {Fallani},
		\citenamefont {Jansen}, \citenamefont {Lewenstein}, \citenamefont
		{Montangero} \emph {et~al.}}]{banuls2020simulating}%
	\BibitemOpen
	\bibfield  {author} {\bibinfo {author} {\bibfnamefont {M.~C.}\ \bibnamefont
			{Banuls}}, \bibinfo {author} {\bibfnamefont {R.}~\bibnamefont {Blatt}},
		\bibinfo {author} {\bibfnamefont {J.}~\bibnamefont {Catani}}, \bibinfo
		{author} {\bibfnamefont {A.}~\bibnamefont {Celi}}, \bibinfo {author}
		{\bibfnamefont {J.~I.}\ \bibnamefont {Cirac}}, \bibinfo {author}
		{\bibfnamefont {M.}~\bibnamefont {Dalmonte}}, \bibinfo {author}
		{\bibfnamefont {L.}~\bibnamefont {Fallani}}, \bibinfo {author} {\bibfnamefont
			{K.}~\bibnamefont {Jansen}}, \bibinfo {author} {\bibfnamefont
			{M.}~\bibnamefont {Lewenstein}}, \bibinfo {author} {\bibfnamefont
			{S.}~\bibnamefont {Montangero}}, \emph {et~al.},\ }\bibinfo {title}
	{Simulating lattice gauge theories within quantum technologies},\ \href
	{https://doi.org/10.1140/epjd/e2020-100571-8} {\bibfield  {journal} {\bibinfo
			{journal} {The European physical journal D}\ }\textbf {\bibinfo {volume}
			{74}},\ \bibinfo {pages} {1} (\bibinfo {year} {2020})}\BibitemShut {NoStop}%
	\bibitem [{\citenamefont {Settineri}\ \emph {et~al.}(2021)\citenamefont
		{Settineri}, \citenamefont {Di~Stefano}, \citenamefont {Zueco}, \citenamefont
		{Hughes}, \citenamefont {Savasta},\ and\ \citenamefont
		{Nori}}]{PhysRevResearch.3.023079}%
	\BibitemOpen
	\bibfield  {author} {\bibinfo {author} {\bibfnamefont {A.}~\bibnamefont
			{Settineri}}, \bibinfo {author} {\bibfnamefont {O.}~\bibnamefont
			{Di~Stefano}}, \bibinfo {author} {\bibfnamefont {D.}~\bibnamefont {Zueco}},
		\bibinfo {author} {\bibfnamefont {S.}~\bibnamefont {Hughes}}, \bibinfo
		{author} {\bibfnamefont {S.}~\bibnamefont {Savasta}},\ and\ \bibinfo {author}
		{\bibfnamefont {F.}~\bibnamefont {Nori}},\ }\bibinfo {title} {Gauge freedom,
		quantum measurements, and time-dependent interactions in cavity QED},\ \href
	{https://doi.org/10.1103/PhysRevResearch.3.023079} {\bibfield  {journal}
		{\bibinfo  {journal} {Phys. Rev. Res.}\ }\textbf {\bibinfo {volume} {3}},\
		\bibinfo {pages} {023079} (\bibinfo {year} {2021})}\BibitemShut {NoStop}%
	\bibitem [{\citenamefont {Savasta}\ \emph {et~al.}(2021)\citenamefont
		{Savasta}, \citenamefont {Di~Stefano}, \citenamefont {Settineri},
		\citenamefont {Zueco}, \citenamefont {Hughes},\ and\ \citenamefont
		{Nori}}]{PhysRevA.103.053703}%
	\BibitemOpen
	\bibfield  {author} {\bibinfo {author} {\bibfnamefont {S.}~\bibnamefont
			{Savasta}}, \bibinfo {author} {\bibfnamefont {O.}~\bibnamefont {Di~Stefano}},
		\bibinfo {author} {\bibfnamefont {A.}~\bibnamefont {Settineri}}, \bibinfo
		{author} {\bibfnamefont {D.}~\bibnamefont {Zueco}}, \bibinfo {author}
		{\bibfnamefont {S.}~\bibnamefont {Hughes}},\ and\ \bibinfo {author}
		{\bibfnamefont {F.}~\bibnamefont {Nori}},\ }\bibinfo {title} {Gauge principle
		and gauge invariance in two-level systems},\ \href
	{https://doi.org/10.1103/PhysRevA.103.053703} {\bibfield  {journal} {\bibinfo
			{journal} {Phys. Rev. A}\ }\textbf {\bibinfo {volume} {103}},\ \bibinfo
		{pages} {053703} (\bibinfo {year} {2021})}\BibitemShut {NoStop}%
	\bibitem [{\citenamefont {Rinaldi}\ \emph {et~al.}(2022)\citenamefont
		{Rinaldi}, \citenamefont {Han}, \citenamefont {Hassan}, \citenamefont {Feng},
		\citenamefont {Nori}, \citenamefont {McGuigan},\ and\ \citenamefont
		{Hanada}}]{PRXQuantum.3.010324}%
	\BibitemOpen
	\bibfield  {author} {\bibinfo {author} {\bibfnamefont {E.}~\bibnamefont
			{Rinaldi}}, \bibinfo {author} {\bibfnamefont {X.}~\bibnamefont {Han}},
		\bibinfo {author} {\bibfnamefont {M.}~\bibnamefont {Hassan}}, \bibinfo
		{author} {\bibfnamefont {Y.}~\bibnamefont {Feng}}, \bibinfo {author}
		{\bibfnamefont {F.}~\bibnamefont {Nori}}, \bibinfo {author} {\bibfnamefont
			{M.}~\bibnamefont {McGuigan}},\ and\ \bibinfo {author} {\bibfnamefont
			{M.}~\bibnamefont {Hanada}},\ }\bibinfo {title} {Matrix-Model Simulations
		Using Quantum Computing, Deep Learning, and Lattice Monte Carlo},\ \href
	{https://doi.org/10.1103/PRXQuantum.3.010324} {\bibfield  {journal} {\bibinfo
			{journal} {PRX Quantum}\ }\textbf {\bibinfo {volume} {3}},\ \bibinfo {pages}
		{010324} (\bibinfo {year} {2022})}\BibitemShut {NoStop}%
	\bibitem [{\citenamefont {James}\ \emph {et~al.}(2019)\citenamefont {James},
		\citenamefont {Konik},\ and\ \citenamefont
		{Robinson}}]{PhysRevLett.122.130603}%
	\BibitemOpen
	\bibfield  {author} {\bibinfo {author} {\bibfnamefont {A.~J.~A.}\
			\bibnamefont {James}}, \bibinfo {author} {\bibfnamefont {R.~M.}\ \bibnamefont
			{Konik}},\ and\ \bibinfo {author} {\bibfnamefont {N.~J.}\ \bibnamefont
			{Robinson}},\ }\bibinfo {title} {Nonthermal States Arising from Confinement
		in One and Two Dimensions},\ \href
	{https://doi.org/10.1103/PhysRevLett.122.130603} {\bibfield  {journal}
		{\bibinfo  {journal} {Phys. Rev. Lett.}\ }\textbf {\bibinfo {volume} {122}},\
		\bibinfo {pages} {130603} (\bibinfo {year} {2019})}\BibitemShut {NoStop}%
	\bibitem [{\citenamefont {Robinson}\ \emph {et~al.}(2019)\citenamefont
		{Robinson}, \citenamefont {James},\ and\ \citenamefont
		{Konik}}]{PhysRevB.99.195108}%
	\BibitemOpen
	\bibfield  {author} {\bibinfo {author} {\bibfnamefont {N.~J.}\ \bibnamefont
			{Robinson}}, \bibinfo {author} {\bibfnamefont {A.~J.~A.}\ \bibnamefont
			{James}},\ and\ \bibinfo {author} {\bibfnamefont {R.~M.}\ \bibnamefont
			{Konik}},\ }\bibinfo {title} {Signatures of rare states and thermalization in
		a theory with confinement},\ \href
	{https://doi.org/10.1103/PhysRevB.99.195108} {\bibfield  {journal} {\bibinfo
			{journal} {Phys. Rev. B}\ }\textbf {\bibinfo {volume} {99}},\ \bibinfo
		{pages} {195108} (\bibinfo {year} {2019})}\BibitemShut {NoStop}%
	\bibitem [{\citenamefont {Surace}\ \emph {et~al.}(2020)\citenamefont {Surace},
		\citenamefont {Mazza}, \citenamefont {Giudici}, \citenamefont {Lerose},
		\citenamefont {Gambassi},\ and\ \citenamefont
		{Dalmonte}}]{PhysRevX.10.021041}%
	\BibitemOpen
	\bibfield  {author} {\bibinfo {author} {\bibfnamefont {F.~M.}\ \bibnamefont
			{Surace}}, \bibinfo {author} {\bibfnamefont {P.~P.}\ \bibnamefont {Mazza}},
		\bibinfo {author} {\bibfnamefont {G.}~\bibnamefont {Giudici}}, \bibinfo
		{author} {\bibfnamefont {A.}~\bibnamefont {Lerose}}, \bibinfo {author}
		{\bibfnamefont {A.}~\bibnamefont {Gambassi}},\ and\ \bibinfo {author}
		{\bibfnamefont {M.}~\bibnamefont {Dalmonte}},\ }\bibinfo {title} {Lattice
		Gauge Theories and String Dynamics in Rydberg Atom Quantum Simulators},\
	\href {https://doi.org/10.1103/PhysRevX.10.021041} {\bibfield  {journal}
		{\bibinfo  {journal} {Phys. Rev. X}\ }\textbf {\bibinfo {volume} {10}},\
		\bibinfo {pages} {021041} (\bibinfo {year} {2020})}\BibitemShut {NoStop}%
	\bibitem [{\citenamefont {Iadecola}\ and\ \citenamefont
		{Schecter}(2020)}]{PhysRevB.101.024306}%
	\BibitemOpen
	\bibfield  {author} {\bibinfo {author} {\bibfnamefont {T.}~\bibnamefont
			{Iadecola}}\ and\ \bibinfo {author} {\bibfnamefont {M.}~\bibnamefont
			{Schecter}},\ }\bibinfo {title} {Quantum many-body scar states with emergent
		kinetic constraints and finite-entanglement revivals},\ \href
	{https://doi.org/10.1103/PhysRevB.101.024306} {\bibfield  {journal} {\bibinfo
			{journal} {Phys. Rev. B}\ }\textbf {\bibinfo {volume} {101}},\ \bibinfo
		{pages} {024306} (\bibinfo {year} {2020})}\BibitemShut {NoStop}%
	\bibitem [{\citenamefont {Yang}\ \emph {et~al.}(2020)\citenamefont {Yang},
		\citenamefont {Liu}, \citenamefont {Gorshkov},\ and\ \citenamefont
		{Iadecola}}]{PhysRevLett.124.207602}%
	\BibitemOpen
	\bibfield  {author} {\bibinfo {author} {\bibfnamefont {Z.-C.}\ \bibnamefont
			{Yang}}, \bibinfo {author} {\bibfnamefont {F.}~\bibnamefont {Liu}}, \bibinfo
		{author} {\bibfnamefont {A.~V.}\ \bibnamefont {Gorshkov}},\ and\ \bibinfo
		{author} {\bibfnamefont {T.}~\bibnamefont {Iadecola}},\ }\bibinfo {title}
	{Hilbert-Space Fragmentation from Strict Confinement},\ \href
	{https://doi.org/10.1103/PhysRevLett.124.207602} {\bibfield  {journal}
		{\bibinfo  {journal} {Phys. Rev. Lett.}\ }\textbf {\bibinfo {volume} {124}},\
		\bibinfo {pages} {207602} (\bibinfo {year} {2020})}\BibitemShut {NoStop}%
	\bibitem [{\citenamefont {Banerjee}\ and\ \citenamefont
		{Sen}(2021)}]{PhysRevLett.126.220601}%
	\BibitemOpen
	\bibfield  {author} {\bibinfo {author} {\bibfnamefont {D.}~\bibnamefont
			{Banerjee}}\ and\ \bibinfo {author} {\bibfnamefont {A.}~\bibnamefont {Sen}},\
	}\bibinfo {title} {Quantum Scars from Zero Modes in an Abelian Lattice Gauge
		Theory on Ladders},\ \href {https://doi.org/10.1103/PhysRevLett.126.220601}
	{\bibfield  {journal} {\bibinfo  {journal} {Phys. Rev. Lett.}\ }\textbf
		{\bibinfo {volume} {126}},\ \bibinfo {pages} {220601} (\bibinfo {year}
		{2021})}\BibitemShut {NoStop}%
	\bibitem [{\citenamefont {Aramthottil}\ \emph {et~al.}(2022)\citenamefont
		{Aramthottil}, \citenamefont {Bhattacharya}, \citenamefont
		{Gonz\'alez-Cuadra}, \citenamefont {Lewenstein}, \citenamefont {Barbiero},\
		and\ \citenamefont {Zakrzewski}}]{PhysRevB.106.L041101}%
	\BibitemOpen
	\bibfield  {author} {\bibinfo {author} {\bibfnamefont {A.~S.}\ \bibnamefont
			{Aramthottil}}, \bibinfo {author} {\bibfnamefont {U.}~\bibnamefont
			{Bhattacharya}}, \bibinfo {author} {\bibfnamefont {D.}~\bibnamefont
			{Gonz\'alez-Cuadra}}, \bibinfo {author} {\bibfnamefont {M.}~\bibnamefont
			{Lewenstein}}, \bibinfo {author} {\bibfnamefont {L.}~\bibnamefont
			{Barbiero}},\ and\ \bibinfo {author} {\bibfnamefont {J.}~\bibnamefont
			{Zakrzewski}},\ }\bibinfo {title} {Scar states in deconfined
		${\mathbb{Z}}_{2}$ lattice gauge theories},\ \href
	{https://doi.org/10.1103/PhysRevB.106.L041101} {\bibfield  {journal}
		{\bibinfo  {journal} {Phys. Rev. B}\ }\textbf {\bibinfo {volume} {106}},\
		\bibinfo {pages} {L041101} (\bibinfo {year} {2022})}\BibitemShut {NoStop}%
	\bibitem [{\citenamefont {Kormos}\ \emph {et~al.}(2017)\citenamefont {Kormos},
		\citenamefont {Collura}, \citenamefont {Takcs},\ and\ \citenamefont
		{Calabrese}}]{ISI:000395814000014}%
	\BibitemOpen
	\bibfield  {author} {\bibinfo {author} {\bibfnamefont {M.}~\bibnamefont
			{Kormos}}, \bibinfo {author} {\bibfnamefont {M.}~\bibnamefont {Collura}},
		\bibinfo {author} {\bibfnamefont {G.}~\bibnamefont {Takcs}},\ and\ \bibinfo
		{author} {\bibfnamefont {P.}~\bibnamefont {Calabrese}},\ }\bibinfo {title}
	{Real-time confinement following a quantum quench to a non-integrable
		model},\ \href {https://doi.org/10.1038/NPHYS3934} {\bibfield  {journal}
		{\bibinfo  {journal} {Nat. Phys.}\ }\textbf {\bibinfo {volume} {13}},\
		\bibinfo {pages} {246} (\bibinfo {year} {2017})}\BibitemShut {NoStop}%
	\bibitem [{\citenamefont {Birnkammer}\ \emph {et~al.}(2022)\citenamefont
		{Birnkammer}, \citenamefont {Bastianello},\ and\ \citenamefont
		{Knap}}]{birnkammer2022prethermalization}%
	\BibitemOpen
	\bibfield  {author} {\bibinfo {author} {\bibfnamefont {S.}~\bibnamefont
			{Birnkammer}}, \bibinfo {author} {\bibfnamefont {A.}~\bibnamefont
			{Bastianello}},\ and\ \bibinfo {author} {\bibfnamefont {M.}~\bibnamefont
			{Knap}},\ }\bibinfo {title} {Prethermalization in one-dimensional quantum
		many-body systems with confinement},\ \href
	{https://doi.org/10.1038/s41467-022-35301-6} {\bibfield  {journal} {\bibinfo
			{journal} {Nature Communications}\ }\textbf {\bibinfo {volume} {13}},\
		\bibinfo {pages} {7663} (\bibinfo {year} {2022})}\BibitemShut {NoStop}%
	\bibitem [{\citenamefont {Barbiero}\ \emph {et~al.}(2019)\citenamefont
		{Barbiero}, \citenamefont {Schweizer}, \citenamefont {Aidelsburger},
		\citenamefont {Demler}, \citenamefont {Goldman},\ and\ \citenamefont
		{Grusdt}}]{Barbieroeaav7444}%
	\BibitemOpen
	\bibfield  {author} {\bibinfo {author} {\bibfnamefont {L.}~\bibnamefont
			{Barbiero}}, \bibinfo {author} {\bibfnamefont {C.}~\bibnamefont {Schweizer}},
		\bibinfo {author} {\bibfnamefont {M.}~\bibnamefont {Aidelsburger}}, \bibinfo
		{author} {\bibfnamefont {E.}~\bibnamefont {Demler}}, \bibinfo {author}
		{\bibfnamefont {N.}~\bibnamefont {Goldman}},\ and\ \bibinfo {author}
		{\bibfnamefont {F.}~\bibnamefont {Grusdt}},\ }\bibinfo {title} {Coupling
		ultracold matter to dynamical gauge fields in optical lattices: From flux
		attachment to $Z_2$ lattice gauge theories},\ \href
	{https://doi.org/10.1126/sciadv.aav7444} {\bibfield  {journal} {\bibinfo
			{journal} {Sci. Adv.}\ }\textbf {\bibinfo {volume} {5}},\ \bibinfo {pages}
		{eaav7444} (\bibinfo {year} {2019})}\BibitemShut {NoStop}%
	\bibitem [{\citenamefont {Schweizer}\ \emph {et~al.}(2019)\citenamefont
		{Schweizer}, \citenamefont {Grusdt}, \citenamefont {Berngruber},
		\citenamefont {Barbiero}, \citenamefont {Demler}, \citenamefont {Goldman},
		\citenamefont {Bloch},\ and\ \citenamefont
		{Aidelsburger}}]{ISI:000494944200024}%
	\BibitemOpen
	\bibfield  {author} {\bibinfo {author} {\bibfnamefont {C.}~\bibnamefont
			{Schweizer}}, \bibinfo {author} {\bibfnamefont {F.}~\bibnamefont {Grusdt}},
		\bibinfo {author} {\bibfnamefont {M.}~\bibnamefont {Berngruber}}, \bibinfo
		{author} {\bibfnamefont {L.}~\bibnamefont {Barbiero}}, \bibinfo {author}
		{\bibfnamefont {E.}~\bibnamefont {Demler}}, \bibinfo {author} {\bibfnamefont
			{N.}~\bibnamefont {Goldman}}, \bibinfo {author} {\bibfnamefont
			{I.}~\bibnamefont {Bloch}},\ and\ \bibinfo {author} {\bibfnamefont
			{M.}~\bibnamefont {Aidelsburger}},\ }\bibinfo {title} {Floquet approach to
		Z(2) lattice gauge theories with ultracold atoms in optical lattices},\ \href
	{https://doi.org/10.1038/s41567-019-0649-7} {\bibfield  {journal} {\bibinfo
			{journal} {Nat. Phys.}\ }\textbf {\bibinfo {volume} {15}},\ \bibinfo {pages}
		{1168} (\bibinfo {year} {2019})}\BibitemShut {NoStop}%
	\bibitem [{\citenamefont {Goerg}\ \emph {et~al.}(2019)\citenamefont {Goerg},
		\citenamefont {Sandholzer}, \citenamefont {Minguzzi}, \citenamefont
		{Desbuquois}, \citenamefont {Messer},\ and\ \citenamefont
		{Esslinger}}]{ISI:000494944200023}%
	\BibitemOpen
	\bibfield  {author} {\bibinfo {author} {\bibfnamefont {F.}~\bibnamefont
			{Goerg}}, \bibinfo {author} {\bibfnamefont {K.}~\bibnamefont {Sandholzer}},
		\bibinfo {author} {\bibfnamefont {J.}~\bibnamefont {Minguzzi}}, \bibinfo
		{author} {\bibfnamefont {R.}~\bibnamefont {Desbuquois}}, \bibinfo {author}
		{\bibfnamefont {M.}~\bibnamefont {Messer}},\ and\ \bibinfo {author}
		{\bibfnamefont {T.}~\bibnamefont {Esslinger}},\ }\bibinfo {title}
	{Realization of density-dependent Peierls phases to engineer quantized gauge
		fields coupled to ultracold matter},\ \href
	{https://doi.org/10.1038/s41567-019-0615-4} {\bibfield  {journal} {\bibinfo
			{journal} {Nat. Phys.}\ }\textbf {\bibinfo {volume} {15}},\ \bibinfo {pages}
		{1161} (\bibinfo {year} {2019})}\BibitemShut {NoStop}%
	\bibitem [{\citenamefont {Ge}\ \emph {et~al.}(2021)\citenamefont {Ge},
		\citenamefont {Huang}, \citenamefont {Meng},\ and\ \citenamefont
		{Fan}}]{Ge_2021}%
	\BibitemOpen
	\bibfield  {author} {\bibinfo {author} {\bibfnamefont {Z.-Y.}\ \bibnamefont
			{Ge}}, \bibinfo {author} {\bibfnamefont {R.-Z.}\ \bibnamefont {Huang}},
		\bibinfo {author} {\bibfnamefont {Z.-Y.}\ \bibnamefont {Meng}},\ and\
		\bibinfo {author} {\bibfnamefont {H.}~\bibnamefont {Fan}},\ }\bibinfo {title}
	{Quantum simulation of lattice gauge theories on superconducting circuits:
		Quantum phase transition and quench dynamics},\ \href
	{https://doi.org/10.1088/1674-1056/ac380e} {\bibfield  {journal} {\bibinfo
			{journal} {Chin. Phys. B}\ }\textbf {\bibinfo {volume} {31}},\ \bibinfo
		{pages} {020304} (\bibinfo {year} {2021})}\BibitemShut {NoStop}%
	\bibitem [{\citenamefont {Wang}\ \emph {et~al.}(2022)\citenamefont {Wang},
		\citenamefont {Ge}, \citenamefont {Xiang}, \citenamefont {Song},
		\citenamefont {Huang}, \citenamefont {Song}, \citenamefont {Guo},
		\citenamefont {Su}, \citenamefont {Xu}, \citenamefont {Zheng},\ and\
		\citenamefont {Fan}}]{PhysRevResearch.4.L022060}%
	\BibitemOpen
	\bibfield  {author} {\bibinfo {author} {\bibfnamefont {Z.}~\bibnamefont
			{Wang}}, \bibinfo {author} {\bibfnamefont {Z.-Y.}\ \bibnamefont {Ge}},
		\bibinfo {author} {\bibfnamefont {Z.}~\bibnamefont {Xiang}}, \bibinfo
		{author} {\bibfnamefont {X.}~\bibnamefont {Song}}, \bibinfo {author}
		{\bibfnamefont {R.-Z.}\ \bibnamefont {Huang}}, \bibinfo {author}
		{\bibfnamefont {P.}~\bibnamefont {Song}}, \bibinfo {author} {\bibfnamefont
			{X.-Y.}\ \bibnamefont {Guo}}, \bibinfo {author} {\bibfnamefont
			{L.}~\bibnamefont {Su}}, \bibinfo {author} {\bibfnamefont {K.}~\bibnamefont
			{Xu}}, \bibinfo {author} {\bibfnamefont {D.}~\bibnamefont {Zheng}},\ and\
		\bibinfo {author} {\bibfnamefont {H.}~\bibnamefont {Fan}},\ }\bibinfo {title}
	{Observation of emergent ${\mathbb{Z}}_{2}$ gauge invariance in a
		superconducting circuit},\ \href
	{https://doi.org/10.1103/PhysRevResearch.4.L022060} {\bibfield  {journal}
		{\bibinfo  {journal} {Phys. Rev. Research}\ }\textbf {\bibinfo {volume}
			{4}},\ \bibinfo {pages} {L022060} (\bibinfo {year} {2022})}\BibitemShut
	{NoStop}%
	\bibitem [{\citenamefont {Mildenberger}\ \emph {et~al.}()\citenamefont
		{Mildenberger}, \citenamefont {Mruczkiewicz}, \citenamefont {Halimeh},
		\citenamefont {Jiang},\ and\ \citenamefont {Hauke}}]{arXiv:2203.08905}%
	\BibitemOpen
	\bibfield  {author} {\bibinfo {author} {\bibfnamefont {J.}~\bibnamefont
			{Mildenberger}}, \bibinfo {author} {\bibfnamefont {W.}~\bibnamefont
			{Mruczkiewicz}}, \bibinfo {author} {\bibfnamefont {J.~C.}\ \bibnamefont
			{Halimeh}}, \bibinfo {author} {\bibfnamefont {Z.}~\bibnamefont {Jiang}},\
		and\ \bibinfo {author} {\bibfnamefont {P.}~\bibnamefont {Hauke}},\ }\bibinfo
	{title} {Probing confinement in a $\mathbb{Z}_2$ lattice gauge theory on a
		quantum computer},\ \href {https://arxiv.org/abs/2203.08905} {\bibinfo
		{journal} {arXiv:2203.08905}\ }\BibitemShut {NoStop}%
	\bibitem [{\citenamefont {Irmejs}\ \emph {et~al.}()\citenamefont {Irmejs},
		\citenamefont {Banuls},\ and\ \citenamefont {Cirac}}]{arXiv:2206.08909}%
	\BibitemOpen
	\bibfield  {journal} {  }\bibfield  {author} {\bibinfo {author} {\bibfnamefont
			{R.}~\bibnamefont {Irmejs}}, \bibinfo {author} {\bibfnamefont {M.~C.}\
			\bibnamefont {Banuls}},\ and\ \bibinfo {author} {\bibfnamefont {J.~I.}\
			\bibnamefont {Cirac}},\ }\bibinfo {title} {Quantum Simulation of $Z_2$
		Lattice Gauge theory with minimal requirements},\ \href
	{https://arxiv.org/abs/2206.08909} {\bibinfo  {journal} {arXiv:2206.08909}\
	}\BibitemShut {NoStop}%
	\bibitem [{\citenamefont {Ge}\ and\ \citenamefont
		{Nori}(2023)}]{PhysRevB.107.125141}%
	\BibitemOpen
	\bibfield  {journal} {  }\bibfield  {author} {\bibinfo {author} {\bibfnamefont
			{Z.-Y.}\ \bibnamefont {Ge}}\ and\ \bibinfo {author} {\bibfnamefont
			{F.}~\bibnamefont {Nori}},\ }\bibinfo {title} {Confinement-induced
		enhancement of superconductivity in a spin-$\frac{1}{2}$ fermion chain
		coupled to a ${\mathbb{Z}}_{2}$ lattice gauge field},\ \href
	{https://doi.org/10.1103/PhysRevB.107.125141} {\bibfield  {journal} {\bibinfo
			{journal} {Phys. Rev. B}\ }\textbf {\bibinfo {volume} {107}},\ \bibinfo
		{pages} {125141} (\bibinfo {year} {2023})}\BibitemShut {NoStop}%
	\bibitem [{\citenamefont {Kebri\ifmmode\check{c}\else\v{c}\fi{}}\ \emph
		{et~al.}(2021)\citenamefont {Kebri\ifmmode\check{c}\else\v{c}\fi{}},
		\citenamefont {Barbiero}, \citenamefont {Reinmoser}, \citenamefont
		{Schollw\"ock},\ and\ \citenamefont {Grusdt}}]{PhysRevLett.127.167203}%
	\BibitemOpen
	\bibfield  {author} {\bibinfo {author} {\bibfnamefont {M.}~\bibnamefont
			{Kebri\ifmmode\check{c}\else\v{c}\fi{}}}, \bibinfo {author} {\bibfnamefont
			{L.}~\bibnamefont {Barbiero}}, \bibinfo {author} {\bibfnamefont
			{C.}~\bibnamefont {Reinmoser}}, \bibinfo {author} {\bibfnamefont
			{U.}~\bibnamefont {Schollw\"ock}},\ and\ \bibinfo {author} {\bibfnamefont
			{F.}~\bibnamefont {Grusdt}},\ }\bibinfo {title} {Confinement and Mott
		transitions of dynamical charges in one-dimensional lattice gauge theories},\
	\href {https://doi.org/10.1103/PhysRevLett.127.167203} {\bibfield  {journal}
		{\bibinfo  {journal} {Phys. Rev. Lett.}\ }\textbf {\bibinfo {volume} {127}},\
		\bibinfo {pages} {167203} (\bibinfo {year} {2021})}\BibitemShut {NoStop}%
	\bibitem [{\citenamefont {Page}(1993)}]{PhysRevLett.71.1291}%
	\BibitemOpen
	\bibfield  {author} {\bibinfo {author} {\bibfnamefont {D.~N.}\ \bibnamefont
			{Page}},\ }\bibinfo {title} {Average entropy of a subsystem},\ \href
	{https://doi.org/10.1103/PhysRevLett.71.1291} {\bibfield  {journal} {\bibinfo
			{journal} {Phys. Rev. Lett.}\ }\textbf {\bibinfo {volume} {71}},\ \bibinfo
		{pages} {1291} (\bibinfo {year} {1993})}\BibitemShut {NoStop}%
	\bibitem [{SM()}]{SM}%
	\BibitemOpen
	\bibinfo {title} {See Supplemental Material},\ \href@noop {} {\ }\BibitemShut
	{NoStop}%
	\bibitem [{\citenamefont {Zhang}\ \emph {et~al.}(2022)\citenamefont {Zhang},
		\citenamefont {Jiang}, \citenamefont {Deng}, \citenamefont {Wang},
		\citenamefont {Chen}, \citenamefont {Zhang}, \citenamefont {Ren},
		\citenamefont {Dong}, \citenamefont {Xu}, \citenamefont {Gao} \emph
		{et~al.}}]{zhang2022digital}%
	\BibitemOpen
	\bibfield  {author} {\bibinfo {author} {\bibfnamefont {X.}~\bibnamefont
			{Zhang}}, \bibinfo {author} {\bibfnamefont {W.}~\bibnamefont {Jiang}},
		\bibinfo {author} {\bibfnamefont {J.}~\bibnamefont {Deng}}, \bibinfo {author}
		{\bibfnamefont {K.}~\bibnamefont {Wang}}, \bibinfo {author} {\bibfnamefont
			{J.}~\bibnamefont {Chen}}, \bibinfo {author} {\bibfnamefont {P.}~\bibnamefont
			{Zhang}}, \bibinfo {author} {\bibfnamefont {W.}~\bibnamefont {Ren}}, \bibinfo
		{author} {\bibfnamefont {H.}~\bibnamefont {Dong}}, \bibinfo {author}
		{\bibfnamefont {S.}~\bibnamefont {Xu}}, \bibinfo {author} {\bibfnamefont
			{Y.}~\bibnamefont {Gao}}, \emph {et~al.},\ }\bibinfo {title} {Digital quantum
		simulation of Floquet symmetry-protected topological phases},\ \href
	{https://doi.org/10.1038/s41586-022-04854-3} {\bibfield  {journal} {\bibinfo
			{journal} {Nature}\ }\textbf {\bibinfo {volume} {607}},\ \bibinfo {pages}
		{468} (\bibinfo {year} {2022})}\BibitemShut {NoStop}%
	\bibitem [{\citenamefont {Dagotto}\ \emph {et~al.}(1988)\citenamefont
		{Dagotto}, \citenamefont {Fradkin},\ and\ \citenamefont
		{Moreo}}]{PhysRevB.38.2926}%
	\BibitemOpen
	\bibfield  {author} {\bibinfo {author} {\bibfnamefont {E.}~\bibnamefont
			{Dagotto}}, \bibinfo {author} {\bibfnamefont {E.}~\bibnamefont {Fradkin}},\
		and\ \bibinfo {author} {\bibfnamefont {A.}~\bibnamefont {Moreo}},\ }\bibinfo
	{title} {SU(2) gauge invariance and order parameters in strongly coupled
		electronic systems},\ \href {https://doi.org/10.1103/PhysRevB.38.2926}
	{\bibfield  {journal} {\bibinfo  {journal} {Phys. Rev. B}\ }\textbf {\bibinfo
			{volume} {38}},\ \bibinfo {pages} {2926} (\bibinfo {year}
		{1988})}\BibitemShut {NoStop}%
\end{thebibliography}%

	%%%%%%%%%%%%%%%%%%%%%%%%%%%%%%%%%
	
 \clearpage
 \widetext



	\begin{center}
	\section{Supplemental Material: \\
		\textit{Meson Instability of Quantum Many-body Scars in a 1D Lattice Gauge Theory} }
\end{center}

	\setcounter{equation}{0} \setcounter{figure}{0}
	\setcounter{table}{0} \setcounter{page}{1} \setcounter{secnumdepth}{3} \makeatletter
	\renewcommand{\theequation}{S\arabic{equation}}
	\renewcommand{\thefigure}{S\arabic{figure}}
	\renewcommand{\bibnumfmt}[1]{[S#1]}
	%	\renewcommand{\citenumfont}[1]{S#1}
	%\renewcommand\thesection{S\arabic{section}}
	%%%%%%%%%% Prefix a "S" to all equations, figures, tables and reset the counter %%%%%%%%%%
	
	\makeatletter
	\def\@hangfrom@section#1#2#3{\@hangfrom{#1#2#3}}
	\makeatother
	

	
	
	\subsection{Quantum Many-Body Scars}
	
	\subsubsection{Proof of $\hat H_K \ket {\Psi_{n,l}} = 0$}
	In the main text, we show that the wave function 
	\begin{align} 
		\ket {\Psi_{n,l}} = \mathcal{N}_{n,l}\sum P_{\{\mathcal{S}_{k_j,\ell_j }\}_n ^l} \ket{\{\mathcal{S}_{k_j,\ell_j }\}_n^l },
	\end{align}
	is an exact eigenstate of $\hat H$.
	Here we present details for proving this result.
	It is not difficult to find that $\hat H_E \ket {\Psi_{n,l}} =h(2l-L) \ket {\Psi_{n,l}}$,
	so we only need to prove $\hat H_K \ket {\Psi_{n,l}} = 0$.
	Since the action of $\hat H_K$ is increasing or reducing the total string length by one, while keeping $n$ invariant,  we have
	\begin{align} 
		\hat H_K \ket {\Psi_{n,l}} = \sum c_{\{\mathcal{S}_{k'_j,\ell'_j }\}_n ^{l-1} } \ket{\{\mathcal{S}_{k'_j,\ell'_j }\}_n^{l-1}  } +\sum c_{\{\mathcal{S}_{k'_j,\ell'_j }\}_n^{l+1}  } \ket{\{\mathcal{S}_{k'_j,\ell'_j }\}_n^{l+1}  }.
	\end{align}
	%Now we demonstrate that the factor $c_{\{\mathcal{S}_{k'_j,\ell'_j }\}_n^{l-1}} =c_{\{\mathcal{S}_{k'_j,\ell'_j }\}_n^{l+1}}=0$.
	%First, for simplicity, we consider some specific basis: $k'_j > k'_{j-1}+\ell'_{j-1}+1 $ for arbitrary $j$.
	Here, the factors have forms 
	\begin{align} \nonumber
		c_{\{\mathcal{S}_{k'_j,\ell'_j }\}_n^{l-1} }  =  \mathcal{N}_{n,l}\sum  [&(1-\delta_{ k'_{1}+\ell'_{1}+1,k'_2 })(P_{\mathcal{S}_{k'_1,\ell'_1+1},\mathcal{S}_{k'_2,\ell'_2}...}+P_{\mathcal{S}_{k'_1,\ell'_1},\mathcal{S}_{k'_2-1,\ell'_2+1}...})\\ \nonumber
		+&(1-\delta_{k'_{2}+\ell'_{2}+1,k'_3 })(P_{...,\mathcal{S}_{k'_2,\ell'_2+1},\mathcal{S}_{k'_3,\ell'_3}...}+P_{...,\mathcal{S}_{k'_2,\ell'_2},\mathcal{S}_{k'_3-1,\ell'_3+1}...})+...\\ \nonumber	c_{\{\mathcal{S}_{k'_j,\ell'_j }\}_n^{l+1} }  =  \mathcal{N}_{n,l}\sum  [&(1-\delta_{ \ell'_{1},1 })(P_{\mathcal{S}_{k'_1-1,\ell'_1-1},\mathcal{S}_{k'_2,\ell'_2}...}+P_{\mathcal{S}_{k'_1,\ell'_1-1},\mathcal{S}_{k'_2,\ell'_2}...})\\
		+&(1-\delta_{\ell'_{2},1})(P_{...,\mathcal{S}_{k'_2-1,\ell'_2-1},\mathcal{S}_{k'_3,\ell'_3}...}+P_{...,\mathcal{S}_{k'_2,\ell'_2-1},\mathcal{S}_{k'_3,\ell'_3}...})+....
	\end{align}
	Since the parity satisfies $P_{\mathcal{S}_{k'_1,\ell'_1},...,\mathcal{S}_{k'_j,\ell'_1},...,\mathcal{S}_{k'_n,\ell'_n}}=\exp{(i\pi\sum_j k'_j)}$,
	we have 
	\begin{align} \nonumber
		&P_{...\mathcal{S}_{k'_{j-1},\ell'_{j-1}},\mathcal{S}_{k'_j,\ell'_j+1},\mathcal{S}_{k'_{j+1},\ell'_{j+1}}...}
		=-P_{...\mathcal{S}_{k'_{j-1},\ell'_{j-1}},\mathcal{S}_{k'_j,\ell'_j},\mathcal{S}_{k'_{j+1}-1,\ell'_{j+1}+1}...}\\
		&P_{...\mathcal{S}_{k'_{j-1},\ell'_{j-1}},\mathcal{S}_{k'_j-1,\ell'_j-1},\mathcal{S}_{k'_{j+1},\ell'_{j+1}}...}
		=-P_{...\mathcal{S}_{k'_{j-1},\ell'_{j-1}},\mathcal{S}_{k'_j,\ell'_j-1},\mathcal{S}_{k'_{j+1},\ell'_{j+1}}...}.
	\end{align}
	Therefore, $c_{\{\mathcal{S}_{k'_j,\ell'_j }\}_n^{l-1}} =c_{\{\mathcal{S}_{k'_j,\ell'_j }\}_n^{l+1}}=0$, i.e., $\hat H_K \ket {\Psi_{n,l}} = 0$.
	
	
	\subsubsection{Proof of 	$\ket {\Psi_{n,n+m}}  = \mathcal{D}_{n,m}\hat L_m \ket {\Psi_{n,n+m-1}}$}
	Next we show the detail of proving Eq.~(8) in the main text, i.e,
	\begin{align}
		\ket {\Psi_{n,n+m}}  = \mathcal{D}_{n,m}\hat L_m \ket {\Psi_{n,n+m-1}} ,
	\end{align}
	where $\mathcal{D}_{n,m}$ is a normalization factor, and 
	\begin{align}
		\hat L^\dagger_m =\sum_{j}  \big(\sum_{k\leq m}\prod_{\ell \leq m}\hat {\mathcal{P}}^-_{j+\frac{1}{2}-\ell} \big)\hat f_{j}\hat \tau^m_{j+\frac{1}{2}}\hat f^\dagger_{j+1}.
	\end{align}
	It is not difficult to demonstrate
	\begin{align}
		\big(\sum_{k\leq m}\prod_{\ell' \leq m}\hat {\mathcal{P}}^-_{j+\frac{1}{2}-\ell'} \big)\hat f_{j}\hat \tau^m_{j+\frac{1}{2}}\hat f^\dagger_{j+1}\ket{\mathcal{S}_{k,\ell}} =
		\begin{cases}
			\delta_{j,k+\ell} \ell \ket{\mathcal{S}_{k,\ell+1}} \quad \ell \leq m  \\
			\delta_{j,k+\ell}  m\ket{\mathcal{S}_{k,\ell+1}} \quad \ell >m.
		\end{cases}	  
	\end{align}
	Thus, the action of $\hat L^\dagger_m $ is increasing the total string length of a basis without changing the parity and string number.
	Therefore,
	\begin{align}
		\hat L_m \ket {\Psi_{n,n+m-1}} = \sum \alpha_{\{\mathcal{S}_{k'_j,\ell'_j }\}_n ^{n+m} } \ket{\{\mathcal{S}_{k'_j,\ell'_j }\}_n^{n+m}  }.
	\end{align}
	For the wave function $\ket {\Psi_{n,n+m-1}} = \mathcal{N}_{n,n+m-1}\sum P_{\{\mathcal{S}_{k_j,\ell_j }\}_n^{n+m-1} } \ket{\{\mathcal{S}_{k_j,\ell_j }\}_n^{n+m-1}}$,
	the length of each string satisfies $\ell_j \leq m$.
	Hence, the factor has the form
	\begin{align}
		\alpha_{\{\mathcal{S}_{k'_j,\ell'_j }\}_n ^{n+m} } = \mathcal{N}_{n,n+m-1}P_{\{\mathcal{S}_{k'_j,\ell'_j }\}_n ^{n+m}}[(1-\delta_{\ell'_1,1})(\ell'_1-1)+(1-\delta_{\ell'_2,1})(\ell'_2-1)+...+ (1-\delta_{\ell_n,1})(\ell'_n-1)].
	\end{align}
	If $\ell'_j=1$, then $(1-\delta_{\ell_j,1})(\ell'_j-1)=(\ell'_j-1)=0$, and if $\ell'_j \neq1$, then $(1-\delta_{\ell_j,1})(\ell'_j-1)=(\ell'_j-1)$.
	Thus 
	\begin{align}
		\alpha_{\{\mathcal{S}_{k'_j,\ell'_j }\}_n ^{n+m} } = \mathcal{N}_{n,n+m-1}P_{\{\mathcal{S}_{k'_j,\ell'_j }\}_n ^{n+m}}\sum_{j=1}^{n}(\ell'_j-1)
		= \mathcal{N}_{n,n+m-1}P_{\{\mathcal{S}_{k'_j,\ell'_j }\}_n^{n+m}}(m-1) .
	\end{align}
	Therefore, we have
	\begin{align}
		\hat L_m \ket {\Psi_{n,n+m-1}} = (m-1)\mathcal{N}_{n,n+m-1}\sum P_{\{\mathcal{S}_{k'_j,\ell'_j }\}_n^{n+m}}\ket{\{\mathcal{S}_{k'_j,\ell'_j }\}_n^{n+m}  }
		=\frac{(m-1)\mathcal{N}_{n,n+m-1}}{\mathcal{N}_{n,n+m}}\ket {\Psi_{n,n+m}}.
	\end{align}
	That is, Eq.~(8) is proved, and the normalization factor satisfies
	\begin{align}
		\mathcal{D}_{n,m}
		=\frac{\mathcal{N}_{n,n+m}}{(m-1)\mathcal{N}_{n,n+m-1}}.
	\end{align}
	
	
	\begin{figure}[t] \includegraphics[width=0.6\textwidth]{Fig_Nnl.pdf}
		\caption{Distribution of $\mathcal{N}_{n,l}^{-2}$ for $L=32$ and $n=8$ (half filling). The orange dashed curve is a Gaussian fit.}
		\label{fig_s1}
	\end{figure}
	
	\subsection{Initial state}
	Here we discuss the initial state $\ket{\psi_{2} }$ in Eq.~(10b) of the main text, where it reads
	\begin{align} %\nonumber
		\ket{\psi_{2} }&= \frac{1}{2^{L/2}} \sum_{n,l}\sum _{\{k_j,\ell_j\}} P_{\{\mathcal{S}_{k_j,\ell_j }\}_n^l } \ket{\{\mathcal{S}_{k_j,\ell_j }\}_{n}^{l} }
		=\sum_{n,l} \beta_{n,l}\ket{\Psi_{n,l}}.
	\end{align}
	The amplitude $\beta_{n,l}$ satisfies $\beta_{n,l} = 1/\mathcal{N}_{n,l} 2^{L/2}$, 
	where $\mathcal{N}_{n,l} $ is the normalization factor defined in Eq.~(6) of the main text.
	In addition, $\mathcal{N}_{n,l}^{-2} $ is the number of string bases for the scar state $\ket{\Psi_{n,l}}$,
	and it can be obtained as 
	\begin{align} %\nonumber
		\mathcal{N}_{n,l}^{-2} =\binom{l-1}{n-1} \bigg[\binom{L-l-1}{n}+2\binom{L-l-1}{n-1}\bigg]+\binom{L-l-1}{n-1}\binom{l-1}{n},
	\end{align}
	where $\binom{\cdot}{\cdot}$ is the combinatorial number.
	In Fig.~\ref{fig_s1}, we show the result of $\mathcal{N}_{n,l}^{-2}$ versus $l$ for $L=32$ and $n=8$ (half filling).
	We can find that $\mathcal{N}_{n,l}^{-2}$  nearly satisfies a Gaussian distribution with the symmetric point at $l=L/2$.
	Therefore, for the initial state $\ket{\psi_{2} }$ the nonmesonic scar states dominate.


\end{document}
