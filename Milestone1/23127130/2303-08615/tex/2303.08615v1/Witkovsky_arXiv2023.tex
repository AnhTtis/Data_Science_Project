%% 
%% Copyright 2007-2020 Elsevier Ltd
%% 
%% This file is part of the 'Elsarticle Bundle'.
%% ---------------------------------------------
%% 
%% It may be distributed under the conditions of the LaTeX Project Public
%% License, either version 1.2 of this license or (at your option) any
%% later version.  The latest version of this license is in
%%    http://www.latex-project.org/lppl.txt
%% and version 1.2 or later is part of all distributions of LaTeX
%% version 1999/12/01 or later.
%% 
%% The list of all files belonging to the 'Elsarticle Bundle' is
%% given in the file `manifest.txt'.
%% 
%% Template article for Elsevier's document class `elsarticle'
%% with harvard style bibliographic references

%\documentclass[preprint,12pt,authoryear]{elsarticle}

%% Use the option review to obtain double line spacing
%% \documentclass[authoryear,preprint,review,12pt]{elsarticle}

%% Use the options 1p,twocolumn; 3p; 3p,twocolumn; 5p; or 5p,twocolumn
%% for a journal layout:
%% \documentclass[final,1p,times,authoryear]{elsarticle}
%% \documentclass[final,1p,times,twocolumn,authoryear]{elsarticle}
%% \documentclass[final,3p,times]{elsarticle}
\documentclass[final,12pt,3p,times]{elsarticle}
%% \documentclass[final,3p,times,twocolumn,authoryear]{elsarticle}
%% \documentclass[final,5p,times,authoryear]{elsarticle}
%% \documentclass[final,5p,times,twocolumn,authoryear]{elsarticle}

%% For including figures, graphicx.sty has been loaded in
%% elsarticle.cls. If you prefer to use the old commands
%% please give \usepackage{epsfig}

%% The amssymb package provides various useful mathematical symbols
%\usepackage{amssymb}
%% The amsthm package provides extended theorem environments
%\usepackage{amsthm}

%% The lineno packages adds line numbers. Start line numbering with
%% \begin{linenumbers}, end it with \end{linenumbers}. Or switch it on
%% for the whole article with \linenumbers.

\usepackage{lineno}

\usepackage{amsmath}
\usepackage{amstext}
\usepackage{amsgen}
\usepackage{amsfonts}
\usepackage{amssymb}

\usepackage{array}
\usepackage{graphicx}
\usepackage{hyperref}
\usepackage{url}

%\usepackage[numbered]{matlab-prettifier}
%\usepackage[framed,autolinebreaks,useliterate]{mcode}
\usepackage[framed,autolinebreaks]{mcode}

\newtheorem{theorem}{Theorem}
\newtheorem{corollary}{Corollary}

\journal{MEASUREMENT 2023, \url{https://www.measurement.sk/M2023/}}

\begin{document}

\begin{frontmatter}

%% Title, authors and addresses

%% use the tnoteref command within \title for footnotes;
%% use the tnotetext command for theassociated footnote;
%% use the fnref command within \author or \affiliation for footnotes;
%% use the fntext command for theassociated footnote;
%% use the corref command within \author for corresponding author footnotes;
%% use the cortext command for theassociated footnote;
%% use the ead command for the email address,
%% and the form \ead[url] for the home page:
%% \title{Title\tnoteref{label1}}
%% \tnotetext[label1]{}
%% \author{Name\corref{cor1}\fnref{label2}}
%% \ead{email address}
%% \ead[url]{home page}
%% \fntext[label2]{}
%% \cortext[cor1]{}
%% \affiliation{organization={},
%%            addressline={}, 
%%            city={},
%%            postcode={}, 
%%            state={},
%%            country={}}
%% \fntext[label3]{}

\title{Characteristic Function of the Tsallis $q$-Gaussian and Its Applications in Measurement and Metrology}

%% use optional labels to link authors explicitly to addresses:
%% \author[label1,label2]{}
%% \affiliation[label1]{organization={},
%%             addressline={},
%%             city={},
%%             postcode={},
%%             state={},
%%             country={}}
%%
%% \affiliation[label2]{organization={},
%%             addressline={},
%%             city={},
%%             postcode={},
%%             state={},
%%             country={}}

\author{Viktor Witkovsk\'y}

\ead{witkovsky@savba.sk}

%\affiliation{organization={Institute of Measurement Science of the Slovak Academy of Sciences},
            %addressline={Dubravska cesta 9}, 
            %city={Bratislava},
            %postcode={84104}, 
            %%state={},
            %country={Slovakia}}

%\affiliation{Institute of Measurement Science of the Slovak Academy of Sciences, Bratislava, Slovakia}

\address{Institute of Measurement Science of the Slovak Academy of Sciences, Bratislava, Slovakia}

\begin{abstract}
The Tsallis $q$-Gaussian distribution is a powerful generalization of the standard Gaussian distribution and is commonly used in various fields, including non-extensive statistical mechanics, financial markets, and image processing. It belongs to the $q$-distribution family, which is characterized by a non-additive entropy. Due to their versatility and practicality, $q$-Gaussians are a natural choice for modeling input quantities in measurement models. This paper presents the characteristic function of a linear combination of independent $q$-Gaussian random variables and proposes a numerical method for its inversion. The proposed technique enables the assessment of the probability distribution of output quantities in linear measurement models and the conduct of uncertainty analysis in metrology. 
\end{abstract}

%%Graphical abstract
%\begin{graphicalabstract}
%\includegraphics{grabs}
%\end{graphicalabstract}

%%Research highlights
%\begin{highlights}
%\item The q-Gaussian distribution and its characteristic function are discussed.
%\item Computing the distribution of a linear combination of q-Gaussian random variables and other independent random variables is provided.
%\item The application and statistical inference of the q-Gaussian distribution are provided.
%\end{highlights}

\begin{keyword}
%% keywords here, in the form: keyword \sep keyword
Tsallis $q$-Gaussian distribution \sep  characteristic function \sep  numerical inversion \sep linear measurement model \sep measurement uncertainty

%% PACS codes here, in the form: \PACS code \sep code

%% MSC codes here, in the form: \MSC code \sep code
%% or \MSC[2008] code \sep code (2000 is the default)
\MSC 60E05 \sep 60E10 \sep 62P35
\end{keyword}

\end{frontmatter}

%\linenumbers

\renewcommand{\baselinestretch}{1.3}


%% main text
\section{Introduction}\label{sec1}

In the field of measurement and uncertainty evaluation, a process consisting of three main stages is typically employed: formulation, propagation, and summarization. The first stage, formulation, is particularly crucial, as it involves a series of important steps that must be taken in order to accurately determine the measurement output quantity or measurand. These steps include defining the measurand, identifying the quantities that influence it, developing a model or measurement equation that relates the output to the inputs, and assigning probability density functions (PDFs) to the input quantities based on available knowledge. Typically, this knowledge is derived from direct measurements and expert knowledge. In addition to the common PDFs such as Gaussian (normal) and rectangular (uniform) distributions, other distributions based on reasonable principles and available information may also be used.

In some cases, it may be necessary to assign PDFs to quantities that have not been explicitly measured or for which only partial information is available. The principle of maximum entropy\footnote{
Entropy is a fundamental concept that has applications in many areas of science and engineering, including measurement, probability, statistics, and information theory. It provides a useful tool for characterizing uncertainty, randomness, and information content in a wide range of systems. The higher the entropy, the more uncertain or disordered the system is. In measurement and uncertainty evaluation, entropy is used to quantify the uncertainty in a measurement by characterizing the probability distribution of the measurement result.} is a valuable tool for this task, as it allows us to construct a PDF that accurately characterizes our incomplete knowledge of a quantity. This involves maximizing the traditional entropy, as defined, e.g., by Shannon, subject to constraints imposed by the available information. The principle of maximum entropy is particularly useful in situations where there are no indications available, and we must rely solely on the available information to estimate the PDF of a given quantity. To learn more about the principle of maximum entropy and its application in measurement and uncertainty evaluation, consult the \emph{Guide to the Expression of Uncertainty in Measurement} (GUM) \cite{GUM} and its supplements \cite{GUMS1, GUMS2}.


	
The Tsallis $q$-Gaussian distribution is a probability distribution introduced by Tsallis \cite{tsallis1988} as a generalization of the standard normal (Gaussian) distribution based on maximizing the  \emph{Tsallis entropy}  under appropriate constraints. It belongs to a larger family of probability distributions known as $q$-distributions, characterized by a non-additive entropy that generalizes the Boltzmann-Gibbs entropy used in statistical mechanics, for more details see, e.g., \cite{tsallis2009}. Non-additive entropy refers to a family of entropy measures that do not satisfy the additivity property of the traditional additive entropy measures. Additivity means that the entropy of a joint system can be obtained by adding the entropy of the individual systems. However, in some cases, additivity may not hold, particularly for complex systems with non-linear interactions between their components.

Non-additive entropy measures are used in various fields, such as physics, information theory, and economics, to quantify the degree of uncertainty or disorder in a system that involves non-linear interactions. Examples of non-additive entropy measures include Tsallis entropy and Renyi entropy, which are widely used in the study of complex systems and statistical mechanics. These measures have been found to be more suitable for modeling complex systems than the traditional additive entropy measures.

Compared to the standard Gaussian distribution the variance of the $q$-Gaussian distribution depends on both the variance parameter and the $q$-index parameter which leads to different behavior in the tails of the distribution. Depending on the value of the $q$-index parameter, the $q$-Gaussian distribution may have unique properties that make it useful for modeling specific data. When the value of the $q$-index parameter is less than one, the distribution has a finite support, meaning that it is bounded on both sides. This can be useful for modeling data that is known to have a specific range, such as time intervals or distances. On the other hand, when the value of the $q$-index parameter is greater than two, the variance of the $q$-Gaussian distribution does not exist, meaning that the distribution has infinite variance. This can be useful for modeling data that has extreme values, such as data with outliers or natural phenomena with rare but significant events. 

Combining of independent $q$-Gaussian random variables is a powerful tool for modeling the behavior of a measurand that involves diverse and complex input quantities. The $q$-Gaussians offer a flexible family of distributions that can capture a wide range of statistical behaviors, making it a suitable choice for such modeling tasks.
In this paper, we present a numerical method for inverting the characteristic function (CF) of a linear combination of independent $q$-Gaussian random variables, say $Y = \sum_{k=1}^n c_k X_k$. This technique can be used to assess the probability distribution of output quantities in linear measurement models using the Characteristic Function Approach (CFA), as described in \cite{witkovsky2017}. 
An alternative approach is through the propagation of distributions using Monte Carlo methods, as outlined in the GUM supplements \cite{GUMS1, GUMS2}.

\section{Characteristic function of $q$-Gaussian distribution}
 
The probability density function (PDF) of the Tsallis $q$-Gaussian distribution is given by:
\begin{equation}
f(x) = C_{q,\sigma}  \left[1 - (1-q) \frac{1}{2} \left(\frac{x-\mu}{\sigma}\right)^2 \right]_+^{\frac{1}{1-q}},
\end{equation}
where $C_{q,\sigma} $ is a normalization constant, $\mu$ (real) and $\sigma > 0$ are the location and scale parameters of the distribution, and $[z]_+ = \max(0,z)$. The shape parameter $q<3$ controls the degree of non-extensivity of the system, and the distribution reduces to the Gaussian distribution when $q = 1$.
 
The parametrization used in this paper is consistent with \href{https://reference.wolfram.com/language/ref/TsallisQGaussianDistribution.html}{Wolfram Mathematica}, where $\sigma > 0$ represents the scale parameter. It is important to note that this differs from the parametrization defined by Tsallis, for details see also \href{https://en.wikipedia.org/wiki/Q-Gaussian_distribution}{Wikipedia}, where $f(x) = C_{q,\beta}  \left[1 - (1-q) \beta \left(x-\mu\right)^2 \right]_+^{\frac{1}{1-q}}$,  which uses $\beta > 0$ as the rate parameter, such that $\beta = 1/(2\sigma^2)$ or $\sigma = \sqrt{1/(2\beta)}$. 

We use $X \sim \mathit{TQG}(\mu, \sigma, q)$ to denote a random variable with a Tsallis $q$-Gaussian distribution, where $\mu \in \mathbb{R}$ is the location parameter, $\sigma > 0$ is the scale parameter, and $q < 3$ is the shape parameter known as the Tsallis $q$-index. The distribution $\mathit{TQG}(\mu, \sigma, q)$ is parametrized such that it is a normal distribution with mean $\mu$ and variance $\sigma^2$ when $q=1$, i.e.~$X \sim \mathit{N}\left(\mu, \sigma^2\right)$. 

Moreover, for $q<1$, $X$ is a bounded random variable, and its distribution is proportional to a scaled and shifted symmetric beta distribution. The support of this distribution is limited to $\left\langle \mu - \sigma \sqrt{{2}/{(1-q)}}, \mu + \sigma \sqrt{{2}/{(1-q)}} \right\rangle$. For $1<q<3$, the distribution is formally proportional to a scaled and shifted Student $t$-distribution with $\nu = (3-q)/(q-1)$ degrees of freedom, which is often preferred over the Gaussian distribution due to its heavy tails, meaning that it has a higher probability of extreme events. It is worth noting that as $q$ approaches 1 from above, $\nu$ approaches infinity (i.e.~normal distribution). For $2 \leq q<3$, the $q$-Gaussian distribution is located in a region of extremely heavy tails where the moments of the distribution may not exist or may not be well-defined. If $q=2$, the $q$-Gaussian distribution is proportional to a scaled and shifted Cauchy distribution, otherwise it is a $t$-distribution with fractional degrees of freedom, $0 < \nu < 1$. Specifically, as $q$ approaches 3, $\nu\to 0$, and as $q$ approaches 2, $\nu\to 1$.

The main result presented in Theorem~1 was obtained through symbolic computation using Wolfram Mathematica and validated through comparison with the characteristic functions of the Student $t$ and symmetric beta distributions, see \cite{witkovsky2001} and \cite{witkovsky2023}.

	
\begin{theorem}[Characteristic function of the Tsallis $q$-Gaussian]
The characteristic function of the Tsallis $q$-Gaussian distribution with the parameters $\mu\in \mathbb{R}$, $\sigma >0$, and $q < 3$ is defined as
\begin{equation}
		\mathop{\mathrm{cf}_{\mathit{TQG}(\mu,\sigma,q)}}(t) = \exp(\mathrm{i} t \mu) \times  \mathop{\mathrm{cf}_{\mathit{TQG}(0,1,q)}}(\sigma t), 
\end{equation}
where $\mathrm{i} = \sqrt{-1}$ and $\mathop{\mathrm{cf}_{\mathit{TQG}(0,1,q)}}(t)$ is CF of the standard Tsallis $q$-Gaussian distribution,
\begin{itemize}
	\item 
for $q<1$ defined as
\begin{align}		
\mathop{\mathrm{cf}_{\mathit{TQG}(0,1,q)}}(t) &= \mathop{\mathit{\,_0F_1}}\left(\theta + \frac{1}{2}, -\frac{1}{4} (a t)^{2}\right) 
   =  2^{\theta - \frac{1}{2}} \Gamma\left(\theta + \frac{1}{2}\right)  (a t)^{-(\theta + \frac{1}{2})}  J_{\theta - \frac{1}{2}}\left(a t\right)\cr 
	&=   \mathop{\mathrm{cf}_{\mathit{BetaSymmetric}(\theta)}}(ct),
\end{align}
 where $a = \sqrt{\frac{2}{1-q}}$, $\theta = \frac{2-q}{1-q}$, $\mathop{\mathit{\,_0F_1}}\left(b,z\right)$ is a confluent hypergeometric function, $\Gamma\left(z\right)$ is gamma function, $J_{\nu}\left(z\right)$ is a Bessel function of the first kind, and $\mathop{\mathrm{cf}_{\mathit{BetaSymmetric}(\theta)}}(t)$ denotes CF of the symmetric beta distribution with the parameter $\theta$ and the support on the interval $\langle -1,1\rangle$, 
\item 
for $q=1$ defined as
\begin{align}
\mathop{\mathrm{cf}_{\mathit{TQG}(0,1,q)}}(t) &=  \exp\left(-\frac{t^2}{2}\right) =  \mathop{\mathrm{cf}_{\mathit{Normal}\left(0,1\right)}}(t). 
\end{align}
\item 
for $1< q<3$ defined as
\begin{align}
\mathop{\mathrm{cf}_{\mathit{TQG}(0,1,q)}}(t) &= \frac{\left( b \sqrt{\nu} |t| \right)^{\frac{\nu}{2}} K_{\frac{\nu}{2}}\left(b \sqrt{\nu} |t|\right) } { 2^{\frac{\nu}{2} - 1} \Gamma\left(\frac{\nu}{2}  \right)} =  \mathop{\mathrm{cf}_{\mathit{Student}(\nu)}}(bt),
\end{align}
where $b = \sqrt{\frac{2}{3-q}}$, $\nu = \frac{3-q}{q-1}$, $K_{\nu}\left(z\right)$ is a Bessel function of the second kind, and $\mathop{\mathrm{cf}_{\mathit{Student}(\nu)}}(t)$ denotes CF of the Student $t$-distribution with $\nu$ degrees of freedom with $\nu > 0$. 
\end{itemize}
\end{theorem}


	
\begin{corollary}[Characteristic function of a linear combination of the Tsallis $q$-Gaussians]
Let $\mathop{\mathrm{cf}_{Y}}(t)$ denote the characteristic function of a linear combination of independent random variables $Y = \sum_{k=1}^n c_k X_k$, where $c_k$ are real coefficients and $X_k$ are independent Tsallis $q$-Gaussians with characteristic functions $\mathop{\mathrm{cf}_{X_k}} = \mathop{\mathrm{cf}_{\mathit{TQG}(\mu_k,\sigma_k,q_k)}}(t)$ for $k=1,\ldots,n$. Then, the characteristic function of $Y$ can be expressed as follows:
\begin{equation}
\mathop{\mathrm{cf}_{Y}}(t) = \prod_{k=1}^n \mathop{\mathrm{cf}_{X_k}}\left(c_kt\right) = \prod_{k=1}^n \mathop{\mathrm{cf}_{\mathit{TQG}(\mu_k,\sigma_k,q_k)}}\left(c_kt\right).
\end{equation}
\end{corollary}
This formula allows us to compute the characteristic function of $Y$ by taking the product of the characteristic functions of each $X_k$, evaluated at $c_kt$.

\section{CharFunTool: The Characteristic Functions Toolbox}
The MATLAB algorithm \verb+cf_TsallisQGaussian+ is implemented in \texttt{CharFunTool} \cite{witkovsky2023}.  
%\url{https://github.com/witkovsky/CharFunTool/blob/master/CF_Repository/cf_TsallisQGaussian.m}, 
It evaluates CF of the measured quantity $Y$ expressed as a linear combination of independent random variables in the measurement equation $Y = f(X_1,\ldots,X_n) = \sum_{k=1}^n c_k X_k$. Here, $c_k$ are real coefficients, and $X_k$ are independent Tsallis $q$-Gaussians with arbitrary parameters for $k=1,\ldots,n$.




\section{Conclusions}
The characteristic function is a powerful tool for numerically deriving the cumulative distribution function (CDF), probability density function (PDF), and quantile function (QF). These functions are essential for specifying the measurement uncertainty and constructing coverage intervals when needed. To perform these calculations in \texttt{CharFunTool}, use the provided inversion algorithms such as \texttt{cf2DistGP} or \texttt{cf2DistGPA}. These algorithms employ the Gil-Pelaez inversion formulae and the adaptive Gauss-Kronrod quadrature rule for numerical integration of the oscillatory integrand function. Additionally, convergence acceleration techniques are employed to compute the limit of the alternating series.
 
For example, set the CF for specific parameters by 
\[
\verb+cf = @(t)cf_TsallisQGaussian(t,mu,sigma,q,coef)+ 
\]
and invert if to PDF/CDF/QF by using 
\[
\verb+result = cf2DistGP(cf)+. 
\]
For other examples see Figures~\ref{fig00}-\ref{fig03}, and for more details see also \cite{witkovsky2023}.


%% The Appendices part is started with the command \appendix;
%% appendix sections are then done as normal sections
%% \appendix

%% \section{}
%% \label{}

%% If you have bibdatabase file and want bibtex to generate the
%% bibitems, please use
%%
\bibliographystyle{elsarticle-harv} 
%\bibliography{Witkovsky2023}


%\begin{thebibliography}{00}
%\bibitem[ ()]{}
%\end{thebibliography}
\begin{thebibliography}{00}

%JCGM200:2012 (VIM). The International Vocabulary of Metrology – Basic and General Concepts And Associated Terms, 3rd Ed. ISO, BIPM, IEC, IFCC, ILAC,
  %IUPAC, IUPAP and OIML, (2012).
	
\bibitem{GUM}
JCGM 100:2008 (GUM). Evaluation of measurement data -- Guide to the expression of uncertainty in measurement (GUM 1995 with minor corrections), ISO, BIPM, IEC, IFCC, ILAC, IUPAC, IUPAP and OIML, (2008).

\bibitem{GUMS1}
JCGM 101:2008 (GUM S1). Evaluation of measurement data -- Supplement 1 to the Guide to the expression of uncertainty in measurement -- Propagation of distributions using a Monte Carlo method. ISO, BIPM, IEC, IFCC, ILAC, IUPAC, IUPAP and OIML, (2008).

\bibitem{GUMS2}
JCGM 102:2011 (GUM S2). Evaluation of measurement data -- Supplement 2 to the Guide to the expression of uncertainty in measurement -- Extension to any number of output quantities. ISO, BIPM, IEC, IFCC, ILAC, IUPAC, IUPAP and OIML, (2011).


%\bibitem{gell2004}
%Gell-Mann, M., Tsallis, C. (2004).
%\newblock {\em Nonextensive Entropy: Interdisciplinary Applications}.
%\newblock Oxford University Press.

\bibitem{tsallis1988}
Tsallis, C. (1988).
\newblock Possible generalization of {B}oltzmann-{G}ibbs statistics.
\newblock {\em Journal of Statistical Physics} 52, 479--487.

\bibitem{tsallis2009}
Tsallis, C. (2009).
\newblock {\em Introduction to Nonextensive Statistical Mechanics: Approaching
  a Complex World}, Vol.~1.
\newblock Springer Science \& Business Media.

\bibitem{witkovsky2001}
Witkovsk{\'y}, V. (2001).
\newblock On the exact computation of the density and of the quantiles of linear combinations of $t$ and $F$ random variables.
\newblock {\em Journal of Statistical Planning and Inference} 94(1), 1--13.

\bibitem{witkovsky2023}
Witkovsk{\'y}, V. (2023). 
\newblock {\em CharFunTool: The Characteristic Functions Toolbox}. \url{https://github.com/witkovsky/CharFunTool}.

\bibitem{witkovsky2017}
Witkovsk{\'y}, V., Wimmer, G., {\v{D}}uri{\v{s}}ov{\'a}, Z., {\v{D}}uri{\v{s}},
  S., Palen{\v{c}}{\'a}r, R. (2017).
\newblock Brief overview of methods for measurement uncertainty analysis: {GUM}
  uncertainty framework, {M}onte {C}arlo method, characteristic function
  approach.
\newblock In: {\em MEASUREMENT 2017, Proceedings of the 11th International
  Conference on Measurement}. Smolenice, Slovakia, May 29-31, pp. 35--38.

\end{thebibliography}


\renewcommand{\baselinestretch}{1}
\begin{figure}[ht]
\begin{lstlisting}
%% EXAMPLE: PDF/CDF/QF from the CF of a linear combination 
%  of q-Gaussian RVs with all q < 1

mu     = [0 0 0];
sigma  = [3 2 1];
q      = [-100 -10 0];
coef   = [0.8 0.15 0.05];
cf     = @(t) cf_TsallisQGaussian(t,mu,sigma,q,coef);
clear options
options.N    = 2^10;
options.xMin = sum(mu - sigma.*sqrt(2./(1-q)) .* coef);
options.xMax = sum(mu + sigma.*sqrt(2./(1-q)) .* coef);
x      = linspace(options.xMin,options.xMax)';
prob   = [0.01 0.05 0.1 0.5 0.9 0.950 .99];
result = cf2DistGP(cf,x,prob,options);
disp(result)
\end{lstlisting}
\includegraphics[width=0.5\textwidth]{./FigPDF0}
\includegraphics[width=0.5\textwidth]{./FigCDF0}
\renewcommand{\baselinestretch}{1.3}
	\caption{MATLAB script based on using the algorithms from \texttt{CharFunTool}. PDF, CDF, and QF computed from the characteristic function of a linear combination of $q$-Gaussian RVs, $Y = c_1 X_1 +c_2X_2 + c_3 X_3$, with all $q < 1$ ($q_1 = -100$, $q_2 = -10$ and $q_3 = 0$).
	\label{fig00}}
	\end{figure}
	
\renewcommand{\baselinestretch}{1}
\begin{figure}[ht]
\begin{lstlisting}
%% EXAMPLE: PDF/CDF/QF from the CF of a linear combination 
%  of q-Gaussian RVs

mu     = [0 1 2];
sigma  = [1 1 1];
q      = [-1 0.5 1.5];
coef   = [1 1 1]/3;
cf     = @(t) cf_TsallisQGaussian(t,mu,sigma,q,coef);
clear options
options.N = 2^10;
x      = linspace(-3,5)';
prob   = [0.01 0.05 0.1 0.5 0.9 0.950 .99];
result = cf2DistGP(cf,x,prob,options);
disp(result)
\end{lstlisting}
\includegraphics[width=0.5\textwidth]{./FigPDF1}
\includegraphics[width=0.5\textwidth]{./FigCDF1}
\renewcommand{\baselinestretch}{1.3}
	\caption{MATLAB script based on using the algorithms from \texttt{CharFunTool}. PDF, CDF, and QF computed from the characteristic function of a linear combination of $q$-Gaussian RVs.
	\label{fig01}}
	\end{figure}

	
\renewcommand{\baselinestretch}{1}
	\begin{figure}[ht]
\begin{lstlisting}
%% EXAMPLE: PDF/CDF/QF from the CF of a linear combination of q-Gaussian RVs 
%  using the Tsallis parametrization

mu     = [0 0 0 0 0];
beta   = [5 4 3 2 1];
sigma  = sqrt(1./(2*beta));
q      = [-5 -1 0 1 2];
coef   = [1 1 1 1 1]/5;
cf     = @(t) cf_TsallisQGaussian(t,mu,sigma,q,coef);
clear options
options.N = 2^14;
x      = linspace(-10,10,301);
prob   = [0.01 0.05 0.1 0.5 0.9 0.950 .99];
result = cf2DistGP(cf,x,prob,options);
disp(result)
\end{lstlisting}
\includegraphics[width=0.5\textwidth]{./FigPDF2}
\includegraphics[width=0.5\textwidth]{./FigCDF2}
	\renewcommand{\baselinestretch}{1.3}
	\caption{MATLAB script based on using the algorithms from \texttt{CharFunTool}. PDF, CDF, and QF computed from the CF of a linear combination of $q$-Gaussian RVs using the Tsallis parametrization.\label{fig02}}
	\end{figure}

\renewcommand{\baselinestretch}{1}
\begin{figure}[ht]
\begin{lstlisting}
%% EXAMPLE: CDF from the CF of a linear combination of q-Gaussian RVs with
%  extreme values of q (here max q = 2.9) computed with using cf2CDF_GPA 

mu     = [0 0 0];
sigma  = [1 0.5 0.1];
q      = [ 0 1 2.9];
coef   = [1 1 1]/3;
cf     = @(t) cf_TsallisQGaussian(t,mu,sigma,q,coef);
clear options
options.isAccelerated = true; 
	
% Plot the integrand functions used for accelerated computation of CDF
% options.isPlot = true;
 
x     = [1e+10 1e+20 1e+30 1e+40 1e+50 1e+60 1e+70 1e+80 1e+90]';
cdf   = cf2CDF_GPA(cf,x,options);
Table = table(x,cdf)

     x         cdf  
  --------  ---------
    1e+10    0.87974
    1e+20    0.96421
    1e+30    0.98935
    1e+40    0.99683
    1e+50    0.99906
    1e+60    0.99972
    1e+70    0.99992
    1e+80    0.99998
    1e+90    0.99999
		
\end{lstlisting}
\renewcommand{\baselinestretch}{1.3}
	\caption{MATLAB script based on using the algorithms from \texttt{CharFunTool}. CDF values computed from the CF of a linear combination of $q$-Gaussian RVs with extreme values of $q$ (here max $q$ = 2.9).\label{fig03}}
	\end{figure}
	

\end{document}
