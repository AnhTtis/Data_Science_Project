\section{Related Work}\label{sec:rel}

% This section summarizes related work on collusive fraud detection models and visual analytics approaches for fraud detection.

\subsection{Collusive Fraud Detection Models}

Anomaly detection models are widely applied to detect collusive fraud from graph data that records interpersonal events by identifying groups with unexpected behaviors~\cite{bindu2018discovering, molloy2016graph, li2012mining, joudaki2015using, akoglu2015graph}. Related methods can be divided into statistics-based models and ML methods.

Statistics-based models identify anomalies through the statistical information of nodes, edges, or sub-graphs. For instance, Akoglu et al.~\cite{akoglu2010oddball} extracted structural features, such as node degree or centrality, from the graph to find egonets. SpamCom~\cite{bindu2018discovering} identified spammer communities on Twitter by using structure and attribute features such as Twitter content similarity, user topology, and user profile. In healthcare scenarios, Chen et al.~\cite{chen2013novel} applied a spectrum analysis-based community detection method to detect patient referral fraud cases from a bipartite graph of physicians and specialists. Zhao et al.~\cite{zhao2019health} generated a dynamic heterogeneous information network containing patients, hospitals, and diseases. Then, they identified anomalies that fit predefined fraud patterns (e.g., the high-cost single treatment) over fixed or variable periods. Statistics-based methods can produce initial fraud candidates but may have erroneous results, requiring further validation by experts.

ML methods typically use GNN to detect fraud, as it is powerful for learning a deep representation of nodes. Previous studies are either conducted on homogeneous~\cite{wang2019fdgars, ding2019deep} or heterogeneous graphs~\cite{xu2021towards, wang2019semi, zhong2020financial}. Wang et al.~\cite{wang2019fdgars} constructed a network of reviewers in online app stores, where nodes (i.e., reviewers) are connected if they have reviewed the same app. The reviews and behavioral features of reviewers are extracted from review logs. Then, a graph convolutional network model is trained and used to detect more fraudsters based on the identified fraud. To detect collusion for fraudulent consumer loans from individuals with various roles (e.g., sellers and intermediaries), Xu et al.~\cite{xu2021towards} propose GRC, a novel GNN model, that learns representations of different types of individuals and detects loan fraud by using attention mechanisms and imposing conditional random fields. However, these ML methods are supervised or semi-supervised and thus require fraud-labeled data, which is lacking in our health insurance scenario.

Since the boundary between fraudulent and normal behavior in healthcare insurance could be unclear, automated models can hardly learn to judge correctly and achieve satisfying accuracy. Therefore, our approach integrates a graph-based detection model with a visual interface, which supports interactive data exploration, model optimization, and result validation.

\subsection{Visual Analytics Approaches for Fraud Detection}
For human-in-the-loop fraud detection, existing studies employ visual analytics to help users understand and implement detection tasks from the perspectives of individual portraits, dramatic changes, and interpersonal events.

Individual portraits include high-dimensional records, which can be described and compared by glyph representations~\cite{ko2014analyzing, cao2015targetvue, maccas2020vabank}.
TargetVue~\cite{cao2015targetvue}'s three circular glyphs depict Twitter users' communication activities, features, and social interactions. Juxtaposed glyphs allow users to compare the behaviors of different individuals and discover possible fraud, such as social bots.
To analyze and detect fraud patterns in banking transactions, Macas et al.~\cite{maccas2020vabank} offered different glyphs to characterize bank clients. Depending on the transaction amount, beneficiaries, and transaction time, the glyph has a circular or rectangular shape complemented with a series of symbols, which enhance the analyst's understanding of typical/atypical transaction profiles.

Dramatic changes are also an important point cut of fraudulent behaviors. Previous studies have designed multiple representation techniques to visualize temporal information, such as sequence visualization~\cite{DBLP:journals/ivs/MacasPM22, zhao2014fluxflow}, radial layouts~\cite{bertini2007spiralview, silva2021visualisation}, and calendar~\cite{lin2020taxthemis}, etc. FluxFlow~\cite{zhao2014fluxflow} demonstrates the impact of anomalous information (e.g., rumors) spreading through colored circles packed on a timeline. Bertini et al.~\cite{bertini2007spiralview} proposed SpiralView, which uses radar charts with spiral time axes to show how alerts change over time to detect suspicious periodic patterns. TaxThemis~\cite{lin2020taxthemis} uses calendar heatmaps to show evidence of transferring revenue through related taxpayers.

Interpersonal events can be summarized by graph visualization. For instance, 
financial fraud between buyers and sellers can be reflected by anomalous structural patterns composed of nodes and edges~\cite{didimo2011advanced}. Niu et al.~\cite{niu2018visual} used a node-link diagram to demonstrate the loan guarantee network, where each node belongs to a community defined by a random walk algorithm and is encoded with the corresponding color. In order to identify collective anomalies, Tao et al.~\cite{tao2018visual} proposed a high-order correlation graph to support analysis processes starting with an abnormal node. Corresponding nodes that contribute to the anomaly can be easily identified through the high-order correlation graph.

Our system incorporates graph and sequence visualization. To focus on collusive fraud in health insurance scenarios, our system provides richer contextual information, such as disease, drugs, and visit frequency.
